\section{Combined Docs LaTeX/front.tex}
\section{C Docs LaTeX/Docs LaTeX.tex}
\section{C Docs LaTeX/typst.app.tex}
\section{Docs LaTeX/typst.app/main.tex}
\title{typst.app/main}

\href{https://typst.app/blog/2024/typst-0.12/}{Learn
what\textquotesingle s new in Typst 0.12}

\href{/\#start}{\includesvg[width=0.94792in,height=0.36458in]{/assets/images/typst.svg}}

\begin{itemize}
\tightlist
\item
  \href{https://github.com/typst/}{}
\item
  \href{https://discord.gg/2uDybryKPe}{}
\item
  \href{https://mastodon.social/@typst}{}
\item
  \href{https://bsky.app/profile/typst.app}{}
\item
  \href{https://www.linkedin.com/company/typst/}{}
\item
  \href{/pricing/}{Pricing}
\item
  \href{/docs/}{Docs}
\item
  \href{/universe/}{Universe}
\item
  \href{https://forum.typst.app}{Forum}
\item
  \href{/signin/}{Sign in}
\item
  \phantomsection\label{header-btn}\href{/signup}{Sign up}
\end{itemize}

\phantomsection\label{start}
\section[Compose\\
\strut \\
faster]{\texorpdfstring{Compose\\
\protect\hypertarget{change}{}{ { papers } }\\
faster}{Compose   papers   faster}}\label{compose-papers-faster}

Focus on your text and let Typst take care of layout and formatting.

\href{/signup}{Sign up for free and try it now!}
\href{https://github.com/typst/typst}{View on GitHub}

\includegraphics[width=30.89583in,height=11.35417in]{/assets/images/mockup.png}

\phantomsection\label{social-proof}
A more productive workflow for science.

Eduardo

“ First time testing the new project creating wizard,
It\textquotesingle s great, I love it! �

\includegraphics[width=1.16667in,height=1.16667in]{/assets/images/submersion-avatar.png}

\href{https://www.reddit.com/r/LaTeX/comments/zyuyfc/comment/j2ddt10/?utm_source=share&utm_medium=web2x&context=3}{}

\_submersion\_

“ A serious contender for the title of LaTeX successor ✨ �

\includegraphics[width=1.16667in,height=1.16667in]{/assets/images/alistair-avatar.jpg}

Alistair

“ I started using the preview, and I love typst so much! �

\includegraphics[width=1.33333in,height=1.33333in]{/assets/images/bluetj-avatar.jpg}

BlueTJ

“ First impression: Damn, it feels smooth as heck. �

\includegraphics[width=1.16667in,height=1.16667in]{/assets/images/vallaaris-avatar.png}

\href{https://www.reddit.com/r/LaTeX/comments/zyuyfc/comment/j2d8for/?utm_source=share&utm_medium=web2x&context=3}{}

Vallaris

“ My first impression is: It\textquotesingle s really great! �

Louis

“ UI: Ultra clean ✨, Docs: Very comprehensive. �

\phantomsection\label{powerful-templates}
\subsection{Team Up With Templates}\label{team-up-with-templates}

Typst supercharges templates: They react to your content and format
everything instantly while you type. Select from a wide range of
\href{/universe}{community templates} or create your own!

\subsubsection{Changed Conferences?}\label{changed-conferences}

Switch the template in seconds, without touching your content.

\phantomsection\label{page-carousel}
\includegraphics[width=3.86458in,height=5.46875in]{/assets/images/ieee.png}

\includegraphics[width=3.86458in,height=5.46875in]{/assets/images/oxford.png}

\includegraphics[width=3.86458in,height=5.46875in]{/assets/images/mdpi.png}

\phantomsection\label{page-buttons}
{ IEEE }

{ Oxford }

{ MDPI }

\phantomsection\label{cloud-collaborative}
\includegraphics[width=8.9375in,height=8.84375in]{/assets/images/collab.png}

\subsection{Work With Your Peers}\label{work-with-your-peers}

Whether in the classroom, the faculty office, or at home: Typst runs in
your browser, so everyone on the team can just start writing.

\subsubsection{Level up your team.}\label{level-up-your-team.}

Store shared documents in team workspaces to bring everyone in your
working group on the same page.

\phantomsection\label{why-use-typst}
\subsection{Why should I use Typst instead of
...}\label{why-should-i-use-typst-instead-of-...}

\begin{itemize}
\tightlist
\item
  LaTeX
\item
  Word
\item
  Google Docs
\end{itemize}

\phantomsection\label{advantage}
\begin{itemize}
\tightlist
\item
  Typst...

  \begin{itemize}
  \tightlist
  \item
    previews your changes instantly.
  \item
    provides clear, understandable error messages.
  \item
    has a consistent styling system for configuring everything from
    fonts and margins to the look of headings and lists.
  \item
    uses familiar programming constructs instead of hard-to-understand
    macros.
  \end{itemize}
\item
  With Typst, you can...

  \begin{itemize}
  \tightlist
  \item
    easily collaborate in teams.
  \item
    use and create powerful templates that automatically format
    everything as you write.
  \item
    just change formatting later without going over the whole document
    by hand.
  \item
    enjoy higher-quality typographical output, including improved
    justification.
  \end{itemize}
\item
  With Typst, you...

  \begin{itemize}
  \tightlist
  \item
    can create professional-looking documents.
  \item
    have a lot more functionality, including better math support, figure
    handling, and a proper table of contents.
  \item
    profit from powerful templates that automatically format everything
    as you write.
  \item
    impress with higher-quality typographical output, including improved
    justification.
  \end{itemize}
\end{itemize}

\phantomsection\label{easy-to-learn}
\includegraphics[width=6.45833in,height=8.33333in]{/assets/images/rocket.png}

\subsection{\texorpdfstring{For Rocket Scientists.\\
And the rest of us,
too.}{For Rocket Scientists. And the rest of us, too.}}\label{for-rocket-scientists.-and-the-rest-of-us-too.}

Most preview users learned Typst in under 60 minutes. Nothing stands
between you and stunning papers.

\subsubsection{Get started quickly}\label{get-started-quickly}

with the \href{/docs/tutorial/}{4-step tutorial} . With human-friendly
error messages, Typst always helps you out if there is a problem.

\phantomsection\label{need-full-control}
\subsection{Need Full Control?}\label{need-full-control}

Anything templates can do, you can do yourself.

Leverage professional typesetting features to write and design any
document.

\phantomsection\label{feature-drawer}
\begin{itemize}
\tightlist
\item
  One language for markup and code
\item
  Powerful style recipes
\item
  Stack and Grid layouts
\item
  Sophisticated typography
\item
  Regex-powered text transformation
\item
  CSV and JSON capabilities
\end{itemize}

Learn more...

\phantomsection\label{ready-science}
\subsection{Ready for Science}\label{ready-for-science}

With Typst, you can set the figures, formulas, tables, bibliographies,
and plots you need to present your research.

\includegraphics[width=18.25in,height=9.05208in]{/assets/images/readyforscience.png}
\includegraphics[width=18.25in,height=9.05208in]{/assets/images/readyforscience.png}

\phantomsection\label{writing-formulas}
\subsection{Writing Formulas has Never Been
Easier}\label{writing-formulas-has-never-been-easier}

Say goodbye to fidgety formula editors and antiquated syntax.

\begin{itemize}
\tightlist
\item
  \texttt{\ }{\texttt{\ lim\ }}\texttt{\ }{\texttt{\ \_\ }}\texttt{\ }{\texttt{\ (\ }}\texttt{\ x\ }{\texttt{\ -\textgreater{}\ }}\texttt{\ }{\texttt{\ oo\ }}\texttt{\ }{\texttt{\ )\ }}\texttt{\ 1\ }{\texttt{\ /\ }}\texttt{\ x\ =\ 0}\strut \\
  \texttt{Â\ }
\item
  \texttt{\ }{\texttt{\ vec\ }}\texttt{\ }{\texttt{\ (\ }}\texttt{\ 1\ }{\texttt{\ ,\ }}\texttt{\ 2\ }{\texttt{\ )\ }}\texttt{\ }{\texttt{\ dot\ }}\texttt{\ }{\texttt{\ }{\texttt{\ vec\ }}\texttt{\ }{\texttt{\ (\ }}\texttt{\ 3\ }{\texttt{\ ,\ }}\texttt{\ 4\ }{\texttt{\ )\ }}\texttt{\ =\ 11\ }}\texttt{\ }
\item
  \texttt{\ \{\ x\ }{\texttt{\ in\ }}\texttt{\ }{\texttt{\ RR\ }}\texttt{\ \textbar{}\ x\ }{\texttt{\ "is\ natural"\ }}\texttt{\ }{\texttt{\ }{\texttt{\ and\ }}\texttt{\ x\ \textless{}\ 10\ \}\ }}\texttt{\ }
\end{itemize}

\begin{itemize}
\tightlist
\item
\item
\item
\end{itemize}

\phantomsection\label{math-buttons}
First example

Second example

Third example

\phantomsection\label{oss}
\subsection{Open Source}\label{open-source}

\includesvg[width=3.57292in,height=1.55208in]{/assets/images/typst-github.svg}

Typst is Open Source! You can follow along with the development on
\href{https://github.com/typst/typst}{GitHub} and discuss Typst on our
\href{https://discord.gg/2uDybryKPe}{Discord server.} With the
\href{https://github.com/typst/typst/releases}{command line compiler} ,
you can always compile your documents locally. No lock-in, ever.

\phantomsection\label{coming-soon}
\subsection{There\textquotesingle s more!}\label{theres-more}

There is still \href{/docs/roadmap}{so much more} we want to do with
Typst. Coming soon:

\begin{itemize}
\tightlist
\item
  HTML export
\item
  PDF/A and Tagged PDF
\end{itemize}

\phantomsection\label{cta}
\subsection{Try Typst now!}\label{try-typst-now}

Sign up to the web app or download the command line tool to start
writing locally.

\href{/signup}{Sign up for free and try it now!}
\href{https://github.com/typst/typst}{View on GitHub}

\phantomsection\label{mission}
We are two graduates from Berlin with a passion for software. Typst was
born out of our frustration with LaTeX. Not knowing what we were in for,
we decided to take matters into our own hands and started building.
\href{/about/}{Read our story.}

\begin{itemize}
\tightlist
\item
  \href{/\#start}{Home}
\item
  \href{/pricing/}{Pricing}
\item
  \href{/docs/}{Documentation}
\item
  \href{/universe/}{Universe}
\item
  \href{/about/}{About Us}
\item
  \href{/contact/}{Contact Us}
\item
  \href{/privacy/}{Privacy}
\item
  \href{https://typst.app/terms}{Terms and Conditions}
\item
  \href{/legal/}{Legal (Impressum)}
\end{itemize}

\begin{itemize}
\tightlist
\item
  \href{https://forum.typst.app}{Forum}
\item
  \href{/tools/}{Tools}
\item
  \href{/blog/}{Blog}
\item
  \href{https://github.com/typst/}{GitHub}
\item
  \href{https://discord.gg/2uDybryKPe}{Discord}
\item
  \href{https://mastodon.social/@typst}{Mastodon}
\item
  \href{https://bsky.app/profile/typst.app}{Bluesky}
\item
  \href{https://www.linkedin.com/company/typst/}{LinkedIn}
\item
  \href{https://instagram.com/typstapp/}{Instagram}
\end{itemize}

Made in Berlin

\phantomsection\label{modal}
\pandocbounded{\includesvg[keepaspectratio]{/assets/icons/16-arrow-right.svg}}

\phantomsection\label{md-insert}
Close

\pandocbounded{\includesvg[keepaspectratio]{/assets/icons/16-arrow-right.svg}}


\section{Docs LaTeX/typst.app/docs.tex}
\title{typst.app/docs}

\begin{itemize}
\tightlist
\item
  \href{/docs}{\includesvg[width=0.16667in,height=0.16667in]{/assets/icons/16-docs-dark.svg}}
\item
  \includesvg[width=0.16667in,height=0.16667in]{/assets/icons/16-arrow-right.svg}
\item
  \href{/docs/}{Overview}
\end{itemize}

\section{Overview}\label{overview}

Welcome to Typst\textquotesingle s documentation! Typst is a new
markup-based typesetting system for the sciences. It is designed to be
an alternative both to advanced tools like LaTeX and simpler tools like
Word and Google Docs. Our goal with Typst is to build a typesetting tool
that is highly capable \emph{and} a pleasure to use.

This documentation is split into two parts: A beginner-friendly tutorial
that introduces Typst through a practical use case and a comprehensive
reference that explains all of Typst\textquotesingle s concepts and
features.

We also invite you to join the community we\textquotesingle re building
around Typst. Typst is still a very young project, so your feedback is
more than valuable.

\href{/docs/tutorial}{\includesvg[width=0.33333in,height=0.33333in]{/assets/icons/32-tutorial-c.svg}
\textbf{Tutorial}}

Step-by-step guide to help you get started.

\href{/docs/reference}{\includesvg[width=0.33333in,height=0.33333in]{/assets/icons/32-reference-c.svg}
\textbf{Reference}}

Details about all syntax, concepts, types, and functions.








\section{Combined Docs LaTeX/docs.tex}
\section{C Docs LaTeX/docs/reference.tex}
\section{C Docs LaTeX/docs/reference/reference.tex}
\section{Docs LaTeX/typst.app/docs/reference/model.tex}
\title{typst.app/docs/reference/model}

\begin{itemize}
\tightlist
\item
  \href{/docs}{\includesvg[width=0.16667in,height=0.16667in]{/assets/icons/16-docs-dark.svg}}
\item
  \includesvg[width=0.16667in,height=0.16667in]{/assets/icons/16-arrow-right.svg}
\item
  \href{/docs/reference/}{Reference}
\item
  \includesvg[width=0.16667in,height=0.16667in]{/assets/icons/16-arrow-right.svg}
\item
  \href{/docs/reference/model/}{Model}
\end{itemize}

\section{Model}\label{summary}

Document structuring.

Here, you can find functions to structure your document and interact
with that structure. This includes section headings, figures,
bibliography management, cross-referencing and more.

\subsection{Definitions}\label{definitions}

\begin{itemize}
\tightlist
\item
  \href{/docs/reference/model/bibliography/}{\texttt{\ bibliography\ }}
  { A bibliography / reference listing. }
\item
  \href{/docs/reference/model/cite/}{\texttt{\ cite\ }} { Cite a work
  from the bibliography. }
\item
  \href{/docs/reference/model/document/}{\texttt{\ document\ }} { The
  root element of a document and its metadata. }
\item
  \href{/docs/reference/model/emph/}{\texttt{\ emph\ }} { Emphasizes
  content by toggling italics. }
\item
  \href{/docs/reference/model/enum/}{\texttt{\ enum\ }} { A numbered
  list. }
\item
  \href{/docs/reference/model/figure/}{\texttt{\ figure\ }} { A figure
  with an optional caption. }
\item
  \href{/docs/reference/model/footnote/}{\texttt{\ footnote\ }} { A
  footnote. }
\item
  \href{/docs/reference/model/heading/}{\texttt{\ heading\ }} { A
  section heading. }
\item
  \href{/docs/reference/model/link/}{\texttt{\ link\ }} { Links to a URL
  or a location in the document. }
\item
  \href{/docs/reference/model/list/}{\texttt{\ list\ }} { A bullet list.
  }
\item
  \href{/docs/reference/model/numbering/}{\texttt{\ numbering\ }} {
  Applies a numbering to a sequence of numbers. }
\item
  \href{/docs/reference/model/outline/}{\texttt{\ outline\ }} { A table
  of contents, figures, or other elements. }
\item
  \href{/docs/reference/model/par/}{\texttt{\ par\ }} { Arranges text,
  spacing and inline-level elements into a paragraph. }
\item
  \href{/docs/reference/model/parbreak/}{\texttt{\ parbreak\ }} { A
  paragraph break. }
\item
  \href{/docs/reference/model/quote/}{\texttt{\ quote\ }} { Displays a
  quote alongside an optional attribution. }
\item
  \href{/docs/reference/model/ref/}{\texttt{\ ref\ }} { A reference to a
  label or bibliography. }
\item
  \href{/docs/reference/model/strong/}{\texttt{\ strong\ }} { Strongly
  emphasizes content by increasing the font weight. }
\item
  \href{/docs/reference/model/table/}{\texttt{\ table\ }} { A table of
  items. }
\item
  \href{/docs/reference/model/terms/}{\texttt{\ terms\ }} { A list of
  terms and their descriptions. }
\end{itemize}

\href{/docs/reference/foundations/version/}{\pandocbounded{\includesvg[keepaspectratio]{/assets/icons/16-arrow-right.svg}}}

{ Version } { Previous page }

\href{/docs/reference/model/bibliography/}{\pandocbounded{\includesvg[keepaspectratio]{/assets/icons/16-arrow-right.svg}}}

{ Bibliography } { Next page }


\section{Docs LaTeX/typst.app/docs/reference/symbols.tex}
\title{typst.app/docs/reference/symbols}

\begin{itemize}
\tightlist
\item
  \href{/docs}{\includesvg[width=0.16667in,height=0.16667in]{/assets/icons/16-docs-dark.svg}}
\item
  \includesvg[width=0.16667in,height=0.16667in]{/assets/icons/16-arrow-right.svg}
\item
  \href{/docs/reference/}{Reference}
\item
  \includesvg[width=0.16667in,height=0.16667in]{/assets/icons/16-arrow-right.svg}
\item
  \href{/docs/reference/symbols/}{Symbols}
\end{itemize}

\section{Symbols}\label{summary}

These two modules give names to symbols and emoji to make them easy to
insert with a normal keyboard. Alternatively, you can also always
directly enter Unicode symbols into your text and formulas. In addition
to the symbols listed below, math mode defines \texttt{\ dif\ } and
\texttt{\ Dif\ } . These are not normal symbol values because they also
affect spacing and font style.

\subsection{Definitions}\label{definitions}

\begin{itemize}
\tightlist
\item
  \href{/docs/reference/symbols/sym/}{\texttt{\ sym\ }} { These two
  modules give names to symbols and emoji to make them easy to }
\item
  \href{/docs/reference/symbols/emoji/}{\texttt{\ emoji\ }} { These two
  modules give names to symbols and emoji to make them easy to }
\item
  \href{/docs/reference/symbols/symbol/}{\texttt{\ symbol\ }} { A
  Unicode symbol. }
\end{itemize}

\subsection{Shorthands}\label{shorthands}

Shorthands are concise sequences of characters that evoke specific
glyphs. Shorthands and other ways to produce symbols can be used
interchangeably. You can use different sets of shorthands in math and
markup mode. Some shorthands, like \texttt{\ \textasciitilde{}\ } for a
non-breaking space produce non-printing symbols, which are indicated
with gray placeholder text.

You can deactivate a shorthand\textquotesingle s interpretation by
escaping any of its characters. If you escape a single character in a
shorthand, the remaining unescaped characters may form a different
shorthand.

\subsubsection{Within Markup Mode}\label{within-markup-mode}

\begin{itemize}
\tightlist
\item
  \phantomsection\label{symbol-space.nobreak}{{ nbsp }
  \texttt{\ \textasciitilde{}\ }}
\item
  \phantomsection\label{symbol-dash.en}{{ â€`` } \texttt{\ -\/-\ }}
\item
  \phantomsection\label{symbol-dash.em}{{ â€'' } \texttt{\ -\/-\/-\ }}
\item
  \phantomsection\label{symbol-hyph.soft}{{ shy } \texttt{\ -?\ }}
\item
  \phantomsection\label{symbol-dots.h}{{ … } \texttt{\ .\ .\ .\ }}
\item
  \phantomsection\label{symbol-minus}{{ âˆ' } \texttt{\ -\ }}
\end{itemize}

{ }

\subsubsection{\texorpdfstring{{ }}{ }}\label{section}

Name: \texttt{\ }
\includesvg[width=0.16667in,height=0.16667in]{/assets/icons/16-copy.svg}

Escape: \texttt{\ \textbackslash{}u\ \{\ }{\texttt{\ }}\texttt{\ \}\ }
\includesvg[width=0.16667in,height=0.16667in]{/assets/icons/16-copy.svg}

Shorthand: \texttt{\ }
\includesvg[width=0.16667in,height=0.16667in]{/assets/icons/16-copy.svg}
{ }

Accent:
\includesvg[width=0.16667in,height=0.16667in]{/assets/icons/16-close.svg}

LaTeX: \texttt{\ }

{ }

\subsubsection{Within Math Mode}\label{within-math-mode}

\begin{itemize}
\tightlist
\item
  \phantomsection\label{symbol-bracket.l.double}{{ ⟦ }
  \texttt{\ {[}\textbar{}\ }}
\item
  \phantomsection\label{symbol-bracket.r.double}{{ ⟧ }
  \texttt{\ \textbar{}{]}\ }}
\item
  \phantomsection\label{symbol-bar.v.double}{{ â€-- }
  \texttt{\ \textbar{}\textbar{}\ }}
\item
  \phantomsection\label{symbol-ast.op}{{ âˆ--- } \texttt{\ *\ }}
\item
  \phantomsection\label{symbol-colon.eq}{{ â‰'' } \texttt{\ :=\ }}
\item
  \phantomsection\label{symbol-colon.double.eq}{{ â©´ }
  \texttt{\ ::=\ }}
\item
  \phantomsection\label{symbol-dots.h}{{ … } \texttt{\ .\ .\ .\ }}
\item
  \phantomsection\label{symbol-tilde.op}{{ ∼ }
  \texttt{\ \textasciitilde{}\ }}
\item
  \phantomsection\label{symbol-prime}{{ ′ }
  \texttt{\ \textquotesingle{}\ }}
\item
  \phantomsection\label{symbol-minus}{{ âˆ' } \texttt{\ -\ }}
\item
  \phantomsection\label{symbol-eq.colon}{{ ≕ } \texttt{\ =:\ }}
\item
  \phantomsection\label{symbol-eq.not}{{ ≠} \texttt{\ !=\ }}
\item
  \phantomsection\label{symbol-gt.double}{{ ≫ }
  \texttt{\ \textgreater{}\textgreater{}\ }}
\item
  \phantomsection\label{symbol-gt.eq}{{ ≥ }
  \texttt{\ \textgreater{}=\ }}
\item
  \phantomsection\label{symbol-gt.triple}{{ â‹™ }
  \texttt{\ \textgreater{}\textgreater{}\textgreater{}\ }}
\item
  \phantomsection\label{symbol-lt.double}{{ ≪ }
  \texttt{\ \textless{}\textless{}\ }}
\item
  \phantomsection\label{symbol-lt.eq}{{ ≤ } \texttt{\ \textless{}=\ }}
\item
  \phantomsection\label{symbol-lt.triple}{{ ⋘ }
  \texttt{\ \textless{}\textless{}\textless{}\ }}
\item
  \phantomsection\label{symbol-arrow.r}{{ â†' }
  \texttt{\ -\textgreater{}\ }}
\item
  \phantomsection\label{symbol-arrow.r.bar}{{ ↦ }
  \texttt{\ \textbar{}-\textgreater{}\ }}
\item
  \phantomsection\label{symbol-arrow.r.double}{{ â‡' }
  \texttt{\ =\textgreater{}\ }}
\item
  \phantomsection\label{symbol-arrow.r.double.bar}{{ ⤇ }
  \texttt{\ \textbar{}=\textgreater{}\ }}
\item
  \phantomsection\label{symbol-arrow.r.double.long}{{ ⟹ }
  \texttt{\ ==\textgreater{}\ }}
\item
  \phantomsection\label{symbol-arrow.r.long}{{ ⟶ }
  \texttt{\ -\/-\textgreater{}\ }}
\item
  \phantomsection\label{symbol-arrow.r.long.squiggly}{{ ⟿ }
  \texttt{\ \textasciitilde{}\textasciitilde{}\textgreater{}\ }}
\item
  \phantomsection\label{symbol-arrow.r.squiggly}{{ � }
  \texttt{\ \textasciitilde{}\textgreater{}\ }}
\item
  \phantomsection\label{symbol-arrow.r.tail}{{ ↣ }
  \texttt{\ \textgreater{}-\textgreater{}\ }}
\item
  \phantomsection\label{symbol-arrow.r.twohead}{{ ↠}
  \texttt{\ -\textgreater{}\textgreater{}\ }}
\item
  \phantomsection\label{symbol-arrow.l}{{ � }
  \texttt{\ \textless{}-\ }}
\item
  \phantomsection\label{symbol-arrow.l.double.long}{{ ⟸ }
  \texttt{\ \textless{}==\ }}
\item
  \phantomsection\label{symbol-arrow.l.long}{{ ⟵ }
  \texttt{\ \textless{}-\/-\ }}
\item
  \phantomsection\label{symbol-arrow.l.long.squiggly}{{ ⬳ }
  \texttt{\ \textless{}\textasciitilde{}\textasciitilde{}\ }}
\item
  \phantomsection\label{symbol-arrow.l.squiggly}{{ ⇜ }
  \texttt{\ \textless{}\textasciitilde{}\ }}
\item
  \phantomsection\label{symbol-arrow.l.tail}{{ ↢ }
  \texttt{\ \textless{}-\textless{}\ }}
\item
  \phantomsection\label{symbol-arrow.l.twohead}{{ ↞ }
  \texttt{\ \textless{}\textless{}-\ }}
\item
  \phantomsection\label{symbol-arrow.l.r}{{ â†'' }
  \texttt{\ \textless{}-\textgreater{}\ }}
\item
  \phantomsection\label{symbol-arrow.l.r.double}{{ â‡'' }
  \texttt{\ \textless{}=\textgreater{}\ }}
\item
  \phantomsection\label{symbol-arrow.l.r.double.long}{{ ⟺ }
  \texttt{\ \textless{}==\textgreater{}\ }}
\item
  \phantomsection\label{symbol-arrow.l.r.long}{{ ⟷ }
  \texttt{\ \textless{}-\/-\textgreater{}\ }}
\end{itemize}

{ }

\subsubsection{\texorpdfstring{{ }}{ }}\label{section-1}

Name: \texttt{\ }
\includesvg[width=0.16667in,height=0.16667in]{/assets/icons/16-copy.svg}

Escape: \texttt{\ \textbackslash{}u\ \{\ }{\texttt{\ }}\texttt{\ \}\ }
\includesvg[width=0.16667in,height=0.16667in]{/assets/icons/16-copy.svg}

Shorthand: \texttt{\ }
\includesvg[width=0.16667in,height=0.16667in]{/assets/icons/16-copy.svg}
{ }

Accent:
\includesvg[width=0.16667in,height=0.16667in]{/assets/icons/16-close.svg}

LaTeX: \texttt{\ }

{ }

\href{/docs/reference/math/vec/}{\pandocbounded{\includesvg[keepaspectratio]{/assets/icons/16-arrow-right.svg}}}

{ Vector } { Previous page }

\href{/docs/reference/symbols/sym/}{\pandocbounded{\includesvg[keepaspectratio]{/assets/icons/16-arrow-right.svg}}}

{ General } { Next page }


\section{Docs LaTeX/typst.app/docs/reference/introspection.tex}
\title{typst.app/docs/reference/introspection}

\begin{itemize}
\tightlist
\item
  \href{/docs}{\includesvg[width=0.16667in,height=0.16667in]{/assets/icons/16-docs-dark.svg}}
\item
  \includesvg[width=0.16667in,height=0.16667in]{/assets/icons/16-arrow-right.svg}
\item
  \href{/docs/reference/}{Reference}
\item
  \includesvg[width=0.16667in,height=0.16667in]{/assets/icons/16-arrow-right.svg}
\item
  \href{/docs/reference/introspection/}{Introspection}
\end{itemize}

\section{Introspection}\label{summary}

Interactions between document parts.

This category is home to Typst\textquotesingle s introspection
capabilities: With the \texttt{\ counter\ } function, you can access and
manipulate page, section, figure, and equation counters or create custom
ones. Meanwhile, the \texttt{\ query\ } function lets you search for
elements in the document to construct things like a list of figures or
headers which show the current chapter title.

Most of the functions are \emph{contextual.} It is recommended to read
the chapter on \href{/docs/reference/context/}{context} before
continuing here.

\subsection{Definitions}\label{definitions}

\begin{itemize}
\tightlist
\item
  \href{/docs/reference/introspection/counter/}{\texttt{\ counter\ }} {
  Counts through pages, elements, and more. }
\item
  \href{/docs/reference/introspection/here/}{\texttt{\ here\ }} {
  Provides the current location in the document. }
\item
  \href{/docs/reference/introspection/locate/}{\texttt{\ locate\ }} {
  Determines the location of an element in the document. }
\item
  \href{/docs/reference/introspection/location/}{\texttt{\ location\ }}
  { Identifies an element in the document. }
\item
  \href{/docs/reference/introspection/metadata/}{\texttt{\ metadata\ }}
  { Exposes a value to the query system without producing visible
  content. }
\item
  \href{/docs/reference/introspection/query/}{\texttt{\ query\ }} {
  Finds elements in the document. }
\item
  \href{/docs/reference/introspection/state/}{\texttt{\ state\ }} {
  Manages stateful parts of your document. }
\end{itemize}

\href{/docs/reference/visualize/stroke/}{\pandocbounded{\includesvg[keepaspectratio]{/assets/icons/16-arrow-right.svg}}}

{ Stroke } { Previous page }

\href{/docs/reference/introspection/counter/}{\pandocbounded{\includesvg[keepaspectratio]{/assets/icons/16-arrow-right.svg}}}

{ Counter } { Next page }


\section{Docs LaTeX/typst.app/docs/reference/syntax.tex}
\title{typst.app/docs/reference/syntax}

\begin{itemize}
\tightlist
\item
  \href{/docs}{\includesvg[width=0.16667in,height=0.16667in]{/assets/icons/16-docs-dark.svg}}
\item
  \includesvg[width=0.16667in,height=0.16667in]{/assets/icons/16-arrow-right.svg}
\item
  \href{/docs/reference/}{Reference}
\item
  \includesvg[width=0.16667in,height=0.16667in]{/assets/icons/16-arrow-right.svg}
\item
  \href{/docs/reference/syntax/}{Syntax}
\end{itemize}

\section{Syntax}\label{syntax}

Typst is a markup language. This means that you can use simple syntax to
accomplish common layout tasks. The lightweight markup syntax is
complemented by set and show rules, which let you style your document
easily and automatically. All this is backed by a tightly integrated
scripting language with built-in and user-defined functions.

\subsection{Modes}\label{modes}

Typst has three syntactical modes: Markup, math, and code. Markup mode
is the default in a Typst document, math mode lets you write
mathematical formulas, and code mode lets you use
Typst\textquotesingle s scripting features.

You can switch to a specific mode at any point by referring to the
following table:

\begin{longtable}[]{@{}lll@{}}
\toprule\noalign{}
New mode & Syntax & Example \\
\midrule\noalign{}
\endhead
\bottomrule\noalign{}
\endlastfoot
Code & Prefix the code with \texttt{\ \#\ } &
\texttt{\ Number:\ }{\texttt{\ \#\ }}\texttt{\ }{\texttt{\ (\ }}\texttt{\ }{\texttt{\ 1\ }}\texttt{\ }{\texttt{\ +\ }}\texttt{\ }{\texttt{\ 2\ }}\texttt{\ }{\texttt{\ )\ }}\texttt{\ } \\
Math & Surround equation with
\texttt{\ }{\texttt{\ \$\ }}\texttt{\ ..\ }{\texttt{\ \$\ }}\texttt{\ }
&
\texttt{\ }{\texttt{\ \$\ }}\texttt{\ }{\texttt{\ -\ }}\texttt{\ x\ }{\texttt{\ \$\ }}\texttt{\ is\ the\ opposite\ of\ }{\texttt{\ \$\ }}\texttt{\ x\ }{\texttt{\ \$\ }}\texttt{\ } \\
Markup & Surround markup with \texttt{\ {[}..{]}\ } &
\texttt{\ }{\texttt{\ let\ }}\texttt{\ name\ }{\texttt{\ =\ }}\texttt{\ }{\texttt{\ {[}\ }}\texttt{\ }{\texttt{\ *Typst!*\ }}\texttt{\ }{\texttt{\ {]}\ }}\texttt{\ } \\
\end{longtable}

Once you have entered code mode with \texttt{\ \#\ } , you
don\textquotesingle t need to use further hashes unless you switched
back to markup or math mode in between.

\subsection{Markup}\label{markup}

Typst provides built-in markup for the most common document elements.
Most of the syntax elements are just shortcuts for a corresponding
function. The table below lists all markup that is available and links
to the best place to learn more about their syntax and usage.

\begin{longtable}[]{@{}lll@{}}
\toprule\noalign{}
Name & Example & See \\
\midrule\noalign{}
\endhead
\bottomrule\noalign{}
\endlastfoot
Paragraph break & Blank line &
\href{/docs/reference/model/parbreak/}{\texttt{\ parbreak\ }} \\
Strong emphasis & \texttt{\ }{\texttt{\ *strong*\ }}\texttt{\ } &
\href{/docs/reference/model/strong/}{\texttt{\ strong\ }} \\
Emphasis & \texttt{\ }{\texttt{\ \_emphasis\_\ }}\texttt{\ } &
\href{/docs/reference/model/emph/}{\texttt{\ emph\ }} \\
Raw text &
\texttt{\ }{\texttt{\ \textasciigrave{}print(1)\textasciigrave{}\ }}\texttt{\ }
& \href{/docs/reference/text/raw/}{\texttt{\ raw\ }} \\
Link & \texttt{\ }{\texttt{\ https://typst.app/\ }}\texttt{\ } &
\href{/docs/reference/model/link/}{\texttt{\ link\ }} \\
Label &
\texttt{\ }{\texttt{\ \textless{}intro\textgreater{}\ }}\texttt{\ } &
\href{/docs/reference/foundations/label/}{\texttt{\ label\ }} \\
Reference & \texttt{\ }{\texttt{\ @intro\ }}\texttt{\ } &
\href{/docs/reference/model/ref/}{\texttt{\ ref\ }} \\
Heading & \texttt{\ }{\texttt{\ =\ Heading\ }}\texttt{\ } &
\href{/docs/reference/model/heading/}{\texttt{\ heading\ }} \\
Bullet list & \texttt{\ }{\texttt{\ -\ }}\texttt{\ item\ } &
\href{/docs/reference/model/list/}{\texttt{\ list\ }} \\
Numbered list & \texttt{\ }{\texttt{\ +\ }}\texttt{\ item\ } &
\href{/docs/reference/model/enum/}{\texttt{\ enum\ }} \\
Term list &
\texttt{\ }{\texttt{\ /\ }}\texttt{\ }{\texttt{\ Term\ }}\texttt{\ }{\texttt{\ :\ }}\texttt{\ description\ }
& \href{/docs/reference/model/terms/}{\texttt{\ terms\ }} \\
Math &
\texttt{\ }{\texttt{\ \$\ }}\texttt{\ x\ }{\texttt{\ \^{}\ }}\texttt{\ 2\ }{\texttt{\ \$\ }}\texttt{\ }
& \href{/docs/reference/math/}{Math} \\
Line break & \texttt{\ }{\texttt{\ \textbackslash{}\ }}\texttt{\ } &
\href{/docs/reference/text/linebreak/}{\texttt{\ linebreak\ }} \\
Smart quote &
\texttt{\ \textquotesingle{}single\textquotesingle{}\ or\ "double"\ } &
\href{/docs/reference/text/smartquote/}{\texttt{\ smartquote\ }} \\
Symbol shorthand &
\texttt{\ }{\texttt{\ \textasciitilde{}\ }}\texttt{\ } ,
\texttt{\ }{\texttt{\ -\/-\/-\ }}\texttt{\ } &
\href{/docs/reference/symbols/sym/}{Symbols} \\
Code expression &
\texttt{\ }{\texttt{\ \#\ }}\texttt{\ }{\texttt{\ rect\ }}\texttt{\ }{\texttt{\ (\ }}\texttt{\ width\ }{\texttt{\ :\ }}\texttt{\ }{\texttt{\ 1cm\ }}\texttt{\ }{\texttt{\ )\ }}\texttt{\ }
& \href{/docs/reference/scripting/\#expressions}{Scripting} \\
Character escape &
\texttt{\ Tweet\ at\ us\ }{\texttt{\ \textbackslash{}\#\ }}\texttt{\ ad\ }
& \hyperref[escapes]{Below} \\
Comment & \texttt{\ }{\texttt{\ /*\ block\ */\ }}\texttt{\ } ,
\texttt{\ }{\texttt{\ //\ line\ }}\texttt{\ } &
\hyperref[comments]{Below} \\
\end{longtable}

\subsection{Math mode}\label{math}

Math mode is a special markup mode that is used to typeset mathematical
formulas. It is entered by wrapping an equation in \texttt{\ \$\ }
characters. This works both in markup and code. The equation will be
typeset into its own block if it starts and ends with at least one space
(e.g.
\texttt{\ }{\texttt{\ \$\ }}\texttt{\ x\ }{\texttt{\ \^{}\ }}\texttt{\ 2\ }{\texttt{\ \$\ }}\texttt{\ }
). Inline math can be produced by omitting the whitespace (e.g.
\texttt{\ }{\texttt{\ \$\ }}\texttt{\ x\ }{\texttt{\ \^{}\ }}\texttt{\ 2\ }{\texttt{\ \$\ }}\texttt{\ }
). An overview over the syntax specific to math mode follows:

\begin{longtable}[]{@{}lll@{}}
\toprule\noalign{}
Name & Example & See \\
\midrule\noalign{}
\endhead
\bottomrule\noalign{}
\endlastfoot
Inline math &
\texttt{\ }{\texttt{\ \$\ }}\texttt{\ x\ }{\texttt{\ \^{}\ }}\texttt{\ 2\ }{\texttt{\ \$\ }}\texttt{\ }
& \href{/docs/reference/math/}{Math} \\
Block-level math &
\texttt{\ }{\texttt{\ \$\ }}\texttt{\ x\ }{\texttt{\ \^{}\ }}\texttt{\ 2\ }{\texttt{\ \$\ }}\texttt{\ }
& \href{/docs/reference/math/}{Math} \\
Bottom attachment &
\texttt{\ }{\texttt{\ \$\ }}\texttt{\ x\ }{\texttt{\ \_\ }}\texttt{\ 1\ }{\texttt{\ \$\ }}\texttt{\ }
& \href{/docs/reference/math/attach/}{\texttt{\ attach\ }} \\
Top attachment &
\texttt{\ }{\texttt{\ \$\ }}\texttt{\ x\ }{\texttt{\ \^{}\ }}\texttt{\ 2\ }{\texttt{\ \$\ }}\texttt{\ }
& \href{/docs/reference/math/attach/}{\texttt{\ attach\ }} \\
Fraction &
\texttt{\ }{\texttt{\ \$\ }}\texttt{\ 1\ +\ }{\texttt{\ (\ }}\texttt{\ a+b\ }{\texttt{\ )\ }}\texttt{\ }{\texttt{\ /\ }}\texttt{\ 5\ }{\texttt{\ \$\ }}\texttt{\ }
& \href{/docs/reference/math/frac/}{\texttt{\ frac\ }} \\
Line break &
\texttt{\ }{\texttt{\ \$\ }}\texttt{\ x\ }{\texttt{\ \textbackslash{}\ }}\texttt{\ y\ }{\texttt{\ \$\ }}\texttt{\ }
& \href{/docs/reference/text/linebreak/}{\texttt{\ linebreak\ }} \\
Alignment point &
\texttt{\ }{\texttt{\ \$\ }}\texttt{\ x\ }{\texttt{\ \&\ }}\texttt{\ =\ 2\ }{\texttt{\ \textbackslash{}\ }}\texttt{\ }{\texttt{\ \&\ }}\texttt{\ =\ 3\ }{\texttt{\ \$\ }}\texttt{\ }
& \href{/docs/reference/math/}{Math} \\
Variable access &
\texttt{\ }{\texttt{\ \$\ }}\texttt{\ }{\texttt{\ \#\ }}\texttt{\ }{\texttt{\ x\ }}\texttt{\ }{\texttt{\ \$\ }}\texttt{\ ,\ }{\texttt{\ \$\ }}\texttt{\ }{\texttt{\ pi\ }}\texttt{\ }{\texttt{\ \$\ }}\texttt{\ }
& \href{/docs/reference/math/}{Math} \\
Field access &
\texttt{\ }{\texttt{\ \$\ }}\texttt{\ }{\texttt{\ arrow\ }}\texttt{\ }{\texttt{\ .\ }}\texttt{\ }{\texttt{\ r\ }}\texttt{\ }{\texttt{\ .\ }}\texttt{\ }{\texttt{\ long\ }}\texttt{\ }{\texttt{\ \$\ }}\texttt{\ }
& \href{/docs/reference/scripting/\#fields}{Scripting} \\
Implied multiplication &
\texttt{\ }{\texttt{\ \$\ }}\texttt{\ x\ y\ }{\texttt{\ \$\ }}\texttt{\ }
& \href{/docs/reference/math/}{Math} \\
Symbol shorthand &
\texttt{\ }{\texttt{\ \$\ }}\texttt{\ }{\texttt{\ -\textgreater{}\ }}\texttt{\ }{\texttt{\ \$\ }}\texttt{\ }
,
\texttt{\ }{\texttt{\ \$\ }}\texttt{\ }{\texttt{\ !=\ }}\texttt{\ }{\texttt{\ \$\ }}\texttt{\ }
& \href{/docs/reference/symbols/sym/}{Symbols} \\
Text/string in math &
\texttt{\ }{\texttt{\ \$\ }}\texttt{\ a\ }{\texttt{\ "is\ natural"\ }}\texttt{\ }{\texttt{\ \$\ }}\texttt{\ }
& \href{/docs/reference/math/}{Math} \\
Math function call &
\texttt{\ }{\texttt{\ \$\ }}\texttt{\ }{\texttt{\ floor\ }}\texttt{\ }{\texttt{\ (\ }}\texttt{\ x\ }{\texttt{\ )\ }}\texttt{\ }{\texttt{\ \$\ }}\texttt{\ }
& \href{/docs/reference/math/}{Math} \\
Code expression &
\texttt{\ }{\texttt{\ \$\ }}\texttt{\ }{\texttt{\ \#\ }}\texttt{\ }{\texttt{\ rect\ }}\texttt{\ }{\texttt{\ (\ }}\texttt{\ width\ }{\texttt{\ :\ }}\texttt{\ }{\texttt{\ 1cm\ }}\texttt{\ }{\texttt{\ )\ }}\texttt{\ }{\texttt{\ \$\ }}\texttt{\ }
& \href{/docs/reference/scripting/\#expressions}{Scripting} \\
Character escape &
\texttt{\ }{\texttt{\ \$\ }}\texttt{\ x\ }{\texttt{\ \textbackslash{}\^{}\ }}\texttt{\ 2\ }{\texttt{\ \$\ }}\texttt{\ }
& \hyperref[escapes]{Below} \\
Comment &
\texttt{\ }{\texttt{\ \$\ }}\texttt{\ }{\texttt{\ /*\ comment\ */\ }}\texttt{\ }{\texttt{\ \$\ }}\texttt{\ }
& \hyperref[comments]{Below} \\
\end{longtable}

\subsection{Code mode}\label{code}

Within code blocks and expressions, new expressions can start without a
leading \texttt{\ \#\ } character. Many syntactic elements are specific
to expressions. Below is a table listing all syntax that is available in
code mode:

\begin{longtable}[]{@{}lll@{}}
\toprule\noalign{}
Name & Example & See \\
\midrule\noalign{}
\endhead
\bottomrule\noalign{}
\endlastfoot
None & \texttt{\ }{\texttt{\ none\ }}\texttt{\ } &
\href{/docs/reference/foundations/none/}{\texttt{\ none\ }} \\
Auto & \texttt{\ }{\texttt{\ auto\ }}\texttt{\ } &
\href{/docs/reference/foundations/auto/}{\texttt{\ auto\ }} \\
Boolean & \texttt{\ }{\texttt{\ false\ }}\texttt{\ } ,
\texttt{\ }{\texttt{\ true\ }}\texttt{\ } &
\href{/docs/reference/foundations/bool/}{\texttt{\ bool\ }} \\
Integer & \texttt{\ }{\texttt{\ 10\ }}\texttt{\ } ,
\texttt{\ }{\texttt{\ 0xff\ }}\texttt{\ } &
\href{/docs/reference/foundations/int/}{\texttt{\ int\ }} \\
Floating-point number & \texttt{\ }{\texttt{\ 3.14\ }}\texttt{\ } ,
\texttt{\ }{\texttt{\ 1e5\ }}\texttt{\ } &
\href{/docs/reference/foundations/float/}{\texttt{\ float\ }} \\
Length & \texttt{\ }{\texttt{\ 2pt\ }}\texttt{\ } ,
\texttt{\ }{\texttt{\ 3mm\ }}\texttt{\ } ,
\texttt{\ }{\texttt{\ 1em\ }}\texttt{\ } , .. &
\href{/docs/reference/layout/length/}{\texttt{\ length\ }} \\
Angle & \texttt{\ }{\texttt{\ 90deg\ }}\texttt{\ } ,
\texttt{\ }{\texttt{\ 1rad\ }}\texttt{\ } &
\href{/docs/reference/layout/angle/}{\texttt{\ angle\ }} \\
Fraction & \texttt{\ }{\texttt{\ 2fr\ }}\texttt{\ } &
\href{/docs/reference/layout/fraction/}{\texttt{\ fraction\ }} \\
Ratio & \texttt{\ }{\texttt{\ 50\%\ }}\texttt{\ } &
\href{/docs/reference/layout/ratio/}{\texttt{\ ratio\ }} \\
String & \texttt{\ }{\texttt{\ "hello"\ }}\texttt{\ } &
\href{/docs/reference/foundations/str/}{\texttt{\ str\ }} \\
Label &
\texttt{\ }{\texttt{\ \textless{}intro\textgreater{}\ }}\texttt{\ } &
\href{/docs/reference/foundations/label/}{\texttt{\ label\ }} \\
Math &
\texttt{\ }{\texttt{\ \$\ }}\texttt{\ x\ }{\texttt{\ \^{}\ }}\texttt{\ 2\ }{\texttt{\ \$\ }}\texttt{\ }
& \href{/docs/reference/math/}{Math} \\
Raw text &
\texttt{\ }{\texttt{\ \textasciigrave{}print(1)\textasciigrave{}\ }}\texttt{\ }
& \href{/docs/reference/text/raw/}{\texttt{\ raw\ }} \\
Variable access & \texttt{\ x\ } &
\href{/docs/reference/scripting/\#blocks}{Scripting} \\
Code block &
\texttt{\ }{\texttt{\ \{\ }}\texttt{\ }{\texttt{\ let\ }}\texttt{\ x\ }{\texttt{\ =\ }}\texttt{\ }{\texttt{\ 1\ }}\texttt{\ }{\texttt{\ ;\ }}\texttt{\ x\ }{\texttt{\ +\ }}\texttt{\ }{\texttt{\ 2\ }}\texttt{\ }{\texttt{\ \}\ }}\texttt{\ }
& \href{/docs/reference/scripting/\#blocks}{Scripting} \\
Content block &
\texttt{\ }{\texttt{\ {[}\ }}\texttt{\ }{\texttt{\ *Hello*\ }}\texttt{\ }{\texttt{\ {]}\ }}\texttt{\ }
& \href{/docs/reference/scripting/\#blocks}{Scripting} \\
Parenthesized expression &
\texttt{\ }{\texttt{\ (\ }}\texttt{\ }{\texttt{\ 1\ }}\texttt{\ }{\texttt{\ +\ }}\texttt{\ }{\texttt{\ 2\ }}\texttt{\ }{\texttt{\ )\ }}\texttt{\ }
& \href{/docs/reference/scripting/\#blocks}{Scripting} \\
Array &
\texttt{\ }{\texttt{\ (\ }}\texttt{\ }{\texttt{\ 1\ }}\texttt{\ }{\texttt{\ ,\ }}\texttt{\ }{\texttt{\ 2\ }}\texttt{\ }{\texttt{\ ,\ }}\texttt{\ }{\texttt{\ 3\ }}\texttt{\ }{\texttt{\ )\ }}\texttt{\ }
& \href{/docs/reference/foundations/array/}{Array} \\
Dictionary &
\texttt{\ }{\texttt{\ (\ }}\texttt{\ a\ }{\texttt{\ :\ }}\texttt{\ }{\texttt{\ "hi"\ }}\texttt{\ }{\texttt{\ ,\ }}\texttt{\ b\ }{\texttt{\ :\ }}\texttt{\ }{\texttt{\ 2\ }}\texttt{\ }{\texttt{\ )\ }}\texttt{\ }
& \href{/docs/reference/foundations/dictionary/}{Dictionary} \\
Unary operator & \texttt{\ }{\texttt{\ -\ }}\texttt{\ x\ } &
\href{/docs/reference/scripting/\#operators}{Scripting} \\
Binary operator & \texttt{\ x\ }{\texttt{\ +\ }}\texttt{\ y\ } &
\href{/docs/reference/scripting/\#operators}{Scripting} \\
Assignment &
\texttt{\ x\ }{\texttt{\ =\ }}\texttt{\ }{\texttt{\ 1\ }}\texttt{\ } &
\href{/docs/reference/scripting/\#operators}{Scripting} \\
Field access & \texttt{\ x\ }{\texttt{\ .\ }}\texttt{\ y\ } &
\href{/docs/reference/scripting/\#fields}{Scripting} \\
Method call &
\texttt{\ x\ }{\texttt{\ .\ }}\texttt{\ }{\texttt{\ flatten\ }}\texttt{\ }{\texttt{\ (\ }}\texttt{\ }{\texttt{\ )\ }}\texttt{\ }
& \href{/docs/reference/scripting/\#methods}{Scripting} \\
Function call &
\texttt{\ }{\texttt{\ min\ }}\texttt{\ }{\texttt{\ (\ }}\texttt{\ x\ }{\texttt{\ ,\ }}\texttt{\ y\ }{\texttt{\ )\ }}\texttt{\ }
& \href{/docs/reference/foundations/function/}{Function} \\
Argument spreading &
\texttt{\ }{\texttt{\ min\ }}\texttt{\ }{\texttt{\ (\ }}\texttt{\ }{\texttt{\ ..\ }}\texttt{\ nums\ }{\texttt{\ )\ }}\texttt{\ }
& \href{/docs/reference/foundations/arguments/}{Arguments} \\
Unnamed function &
\texttt{\ }{\texttt{\ (\ }}\texttt{\ x\ }{\texttt{\ ,\ }}\texttt{\ y\ }{\texttt{\ )\ }}\texttt{\ }{\texttt{\ =\textgreater{}\ }}\texttt{\ x\ }{\texttt{\ +\ }}\texttt{\ y\ }
& \href{/docs/reference/foundations/function/}{Function} \\
Let binding &
\texttt{\ }{\texttt{\ let\ }}\texttt{\ x\ }{\texttt{\ =\ }}\texttt{\ }{\texttt{\ 1\ }}\texttt{\ }
& \href{/docs/reference/scripting/\#bindings}{Scripting} \\
Named function &
\texttt{\ }{\texttt{\ let\ }}\texttt{\ }{\texttt{\ f\ }}\texttt{\ }{\texttt{\ (\ }}\texttt{\ x\ }{\texttt{\ )\ }}\texttt{\ }{\texttt{\ =\ }}\texttt{\ }{\texttt{\ 2\ }}\texttt{\ }{\texttt{\ *\ }}\texttt{\ x\ }
& \href{/docs/reference/foundations/function/}{Function} \\
Set rule &
\texttt{\ }{\texttt{\ set\ }}\texttt{\ }{\texttt{\ text\ }}\texttt{\ }{\texttt{\ (\ }}\texttt{\ }{\texttt{\ 14pt\ }}\texttt{\ }{\texttt{\ )\ }}\texttt{\ }
& \href{/docs/reference/styling/\#set-rules}{Styling} \\
Set-if rule &
\texttt{\ }{\texttt{\ set\ }}\texttt{\ }{\texttt{\ text\ }}\texttt{\ }{\texttt{\ (\ }}\texttt{\ }{\texttt{\ ..\ }}\texttt{\ }{\texttt{\ )\ }}\texttt{\ }{\texttt{\ if\ }}\texttt{\ ..\ }
& \href{/docs/reference/styling/\#set-rules}{Styling} \\
Show-set rule &
\texttt{\ }{\texttt{\ show\ }}\texttt{\ }{\texttt{\ heading\ }}\texttt{\ }{\texttt{\ :\ }}\texttt{\ }{\texttt{\ set\ }}\texttt{\ }{\texttt{\ block\ }}\texttt{\ }{\texttt{\ (\ }}\texttt{\ }{\texttt{\ ..\ }}\texttt{\ }{\texttt{\ )\ }}\texttt{\ }
& \href{/docs/reference/styling/\#show-rules}{Styling} \\
Show rule with function &
\texttt{\ }{\texttt{\ show\ }}\texttt{\ }{\texttt{\ raw\ }}\texttt{\ }{\texttt{\ :\ }}\texttt{\ it\ }{\texttt{\ =\textgreater{}\ }}\texttt{\ }{\texttt{\ \{\ }}\texttt{\ ..\ }{\texttt{\ \}\ }}\texttt{\ }
& \href{/docs/reference/styling/\#show-rules}{Styling} \\
Show-everything rule &
\texttt{\ }{\texttt{\ show\ }}\texttt{\ }{\texttt{\ :\ }}\texttt{\ }{\texttt{\ template\ }}\texttt{\ }
& \href{/docs/reference/styling/\#show-rules}{Styling} \\
Context expression &
\texttt{\ }{\texttt{\ context\ }}\texttt{\ text\ }{\texttt{\ .\ }}\texttt{\ lang\ }
& \href{/docs/reference/context/}{Context} \\
Conditional &
\texttt{\ }{\texttt{\ if\ }}\texttt{\ x\ }{\texttt{\ ==\ }}\texttt{\ }{\texttt{\ 1\ }}\texttt{\ }{\texttt{\ \{\ }}\texttt{\ ..\ }{\texttt{\ \}\ }}\texttt{\ }{\texttt{\ else\ }}\texttt{\ }{\texttt{\ \{\ }}\texttt{\ ..\ }{\texttt{\ \}\ }}\texttt{\ }
& \href{/docs/reference/scripting/\#conditionals}{Scripting} \\
For loop &
\texttt{\ }{\texttt{\ for\ }}\texttt{\ x\ }{\texttt{\ in\ }}\texttt{\ }{\texttt{\ (\ }}\texttt{\ }{\texttt{\ 1\ }}\texttt{\ }{\texttt{\ ,\ }}\texttt{\ }{\texttt{\ 2\ }}\texttt{\ }{\texttt{\ ,\ }}\texttt{\ }{\texttt{\ 3\ }}\texttt{\ }{\texttt{\ )\ }}\texttt{\ }{\texttt{\ \{\ }}\texttt{\ ..\ }{\texttt{\ \}\ }}\texttt{\ }
& \href{/docs/reference/scripting/\#loops}{Scripting} \\
While loop &
\texttt{\ }{\texttt{\ while\ }}\texttt{\ x\ }{\texttt{\ \textless{}\ }}\texttt{\ }{\texttt{\ 10\ }}\texttt{\ }{\texttt{\ \{\ }}\texttt{\ ..\ }{\texttt{\ \}\ }}\texttt{\ }
& \href{/docs/reference/scripting/\#loops}{Scripting} \\
Loop control flow &
\texttt{\ }{\texttt{\ break\ }}\texttt{\ ,\ }{\texttt{\ continue\ }}\texttt{\ }
& \href{/docs/reference/scripting/\#loops}{Scripting} \\
Return from function & \texttt{\ }{\texttt{\ return\ }}\texttt{\ x\ } &
\href{/docs/reference/foundations/function/}{Function} \\
Include module &
\texttt{\ }{\texttt{\ include\ }}\texttt{\ }{\texttt{\ "bar.typ"\ }}\texttt{\ }
& \href{/docs/reference/scripting/\#modules}{Scripting} \\
Import module &
\texttt{\ }{\texttt{\ import\ }}\texttt{\ }{\texttt{\ "bar.typ"\ }}\texttt{\ }
& \href{/docs/reference/scripting/\#modules}{Scripting} \\
Import items from module &
\texttt{\ }{\texttt{\ import\ }}\texttt{\ }{\texttt{\ "bar.typ"\ }}\texttt{\ }{\texttt{\ :\ }}\texttt{\ a\ }{\texttt{\ ,\ }}\texttt{\ b\ }{\texttt{\ ,\ }}\texttt{\ c\ }
& \href{/docs/reference/scripting/\#modules}{Scripting} \\
Comment & \texttt{\ }{\texttt{\ /*\ block\ */\ }}\texttt{\ } ,
\texttt{\ }{\texttt{\ //\ line\ }}\texttt{\ } &
\hyperref[comments]{Below} \\
\end{longtable}

\subsection{Comments}\label{comments}

Comments are ignored by Typst and will not be included in the output.
This is useful to exclude old versions or to add annotations. To comment
out a single line, start it with \texttt{\ //\ } :

\begin{verbatim}
// our data barely supports
// this claim

We show with $p < 0.05$
that the difference is
significant.
\end{verbatim}

\includegraphics[width=5in,height=\textheight,keepaspectratio]{/assets/docs/qmPJyf2DgB8m9bpdDccxUQAAAAAAAAAA.png}

Comments can also be wrapped between \texttt{\ /*\ } and \texttt{\ */\ }
. In this case, the comment can span over multiple lines:

\begin{verbatim}
Our study design is as follows:
/* Somebody write this up:
   - 1000 participants.
   - 2x2 data design. */
\end{verbatim}

\includegraphics[width=5in,height=\textheight,keepaspectratio]{/assets/docs/0bd3Pt_MGVIAagJ8npuMMAAAAAAAAAAA.png}

\subsection{Escape sequences}\label{escapes}

Escape sequences are used to insert special characters that are hard to
type or otherwise have special meaning in Typst. To escape a character,
precede it with a backslash. To insert any Unicode codepoint, you can
write a hexadecimal escape sequence:
\texttt{\ }{\texttt{\ \textbackslash{}u\{1f600\}\ }}\texttt{\ } . The
same kind of escape sequences also work in
\href{/docs/reference/foundations/str/}{strings} .

\begin{verbatim}
I got an ice cream for
\$1.50! \u{1f600}
\end{verbatim}

\includegraphics[width=5in,height=\textheight,keepaspectratio]{/assets/docs/2Hq1wVq0JUPd4EarGtBZUQAAAAAAAAAA.png}

\subsection{Paths}\label{paths}

Typst has various features that require a file path to reference
external resources such as images, Typst files, or data files. Paths are
represented as \href{/docs/reference/foundations/str/}{strings} . There
are two kinds of paths: Relative and absolute.

\begin{itemize}
\item
  A \textbf{relative path} searches from the location of the Typst file
  where the feature is invoked. It is the default:

\begin{verbatim}
#image("images/logo.png")
\end{verbatim}
\item
  An \textbf{absolute path} searches from the \emph{root} of the
  project. It starts with a leading \texttt{\ /\ } :

\begin{verbatim}
#image("/assets/logo.png")
\end{verbatim}
\end{itemize}

\subsubsection{Project root}\label{project-root}

By default, the project root is the parent directory of the main Typst
file. For security reasons, you cannot read any files outside of the
root directory.

If you want to set a specific folder as the root of your project, you
can use the CLI\textquotesingle s \texttt{\ -\/-root\ } flag. Make sure
that the main file is contained in the folder\textquotesingle s subtree!

\begin{verbatim}
typst compile --root .. file.typ
\end{verbatim}

In the web app, the project itself is the root directory. You can always
read all files within it, no matter which one is previewed (via the eye
toggle next to each Typst file in the file panel).

\subsubsection{Paths and packages}\label{paths-and-packages}

A package can only load files from its own directory. Within it,
absolute paths point to the package root, rather than the project root.
For this reason, it cannot directly load files from the project
directory. If a package needs resources from the project (such as a logo
image), you must pass the already loaded image, e.g. as a named
parameter
\texttt{\ logo:\ }{\texttt{\ image\ }}\texttt{\ }{\texttt{\ (\ }}\texttt{\ }{\texttt{\ "mylogo.svg"\ }}\texttt{\ }{\texttt{\ )\ }}\texttt{\ }
. Note that you can then still customize the image\textquotesingle s
appearance with a set rule within the package.

In the future, paths might become a
\href{https://github.com/typst/typst/issues/971}{distinct type from
strings} , so that they can retain knowledge of where they were
constructed. This way, resources could be loaded from a different root.

\href{/docs/reference/}{\pandocbounded{\includesvg[keepaspectratio]{/assets/icons/16-arrow-right.svg}}}

{ Reference } { Previous page }

\href{/docs/reference/styling/}{\pandocbounded{\includesvg[keepaspectratio]{/assets/icons/16-arrow-right.svg}}}

{ Styling } { Next page }


\section{Docs LaTeX/typst.app/docs/reference/scripting.tex}
\title{typst.app/docs/reference/scripting}

\begin{itemize}
\tightlist
\item
  \href{/docs}{\includesvg[width=0.16667in,height=0.16667in]{/assets/icons/16-docs-dark.svg}}
\item
  \includesvg[width=0.16667in,height=0.16667in]{/assets/icons/16-arrow-right.svg}
\item
  \href{/docs/reference/}{Reference}
\item
  \includesvg[width=0.16667in,height=0.16667in]{/assets/icons/16-arrow-right.svg}
\item
  \href{/docs/reference/scripting/}{Scripting}
\end{itemize}

\section{Scripting}\label{scripting}

Typst embeds a powerful scripting language. You can automate your
documents and create more sophisticated styles with code. Below is an
overview over the scripting concepts.

\subsection{Expressions}\label{expressions}

In Typst, markup and code are fused into one. All but the most common
elements are created with \emph{functions.} To make this as convenient
as possible, Typst provides compact syntax to embed a code expression
into markup: An expression is introduced with a hash ( \texttt{\ \#\ } )
and normal markup parsing resumes after the expression is finished. If a
character would continue the expression but should be interpreted as
text, the expression can forcibly be ended with a semicolon (
\texttt{\ ;\ } ).

\begin{verbatim}
#emph[Hello] \
#emoji.face \
#"hello".len()
\end{verbatim}

\includegraphics[width=5in,height=\textheight,keepaspectratio]{/assets/docs/Vvzr_VTofgbwk2d3ymxPMQAAAAAAAAAA.png}

The example above shows a few of the available expressions, including
\href{/docs/reference/foundations/function/}{function calls} ,
\href{/docs/reference/scripting/\#fields}{field accesses} , and
\href{/docs/reference/scripting/\#methods}{method calls} . More kinds of
expressions are discussed in the remainder of this chapter. A few kinds
of expressions are not compatible with the hash syntax (e.g. binary
operator expressions). To embed these into markup, you can use
parentheses, as in
\texttt{\ }{\texttt{\ \#\ }}\texttt{\ }{\texttt{\ (\ }}\texttt{\ }{\texttt{\ 1\ }}\texttt{\ }{\texttt{\ +\ }}\texttt{\ }{\texttt{\ 2\ }}\texttt{\ }{\texttt{\ )\ }}\texttt{\ }
.

\subsection{Blocks}\label{blocks}

To structure your code and embed markup into it, Typst provides two
kinds of \emph{blocks:}

\begin{itemize}
\item
  \textbf{Code block:}
  \texttt{\ }{\texttt{\ \{\ }}\texttt{\ }{\texttt{\ let\ }}\texttt{\ x\ }{\texttt{\ =\ }}\texttt{\ }{\texttt{\ 1\ }}\texttt{\ }{\texttt{\ ;\ }}\texttt{\ x\ }{\texttt{\ +\ }}\texttt{\ }{\texttt{\ 2\ }}\texttt{\ }{\texttt{\ \}\ }}\texttt{\ }\\
  When writing code, you\textquotesingle ll probably want to split up
  your computation into multiple statements, create some intermediate
  variables and so on. Code blocks let you write multiple expressions
  where one is expected. The individual expressions in a code block
  should be separated by line breaks or semicolons. The output values of
  the individual expressions in a code block are joined to determine the
  block\textquotesingle s value. Expressions without useful output, like
  \texttt{\ }{\texttt{\ let\ }}\texttt{\ } bindings yield
  \texttt{\ }{\texttt{\ none\ }}\texttt{\ } , which can be joined with
  any value without effect.
\item
  \textbf{Content block:}
  \texttt{\ }{\texttt{\ {[}\ }}\texttt{\ }{\texttt{\ *Hey*\ }}\texttt{\ there!\ }{\texttt{\ {]}\ }}\texttt{\ }\\
  With content blocks, you can handle markup/content as a programmatic
  value, store it in variables and pass it to
  \href{/docs/reference/foundations/function/}{functions} . Content
  blocks are delimited by square brackets and can contain arbitrary
  markup. A content block results in a value of type
  \href{/docs/reference/foundations/content/}{content} . An arbitrary
  number of content blocks can be passed as trailing arguments to
  functions. That is,
  \texttt{\ }{\texttt{\ list\ }}\texttt{\ }{\texttt{\ (\ }}\texttt{\ }{\texttt{\ {[}\ }}\texttt{\ A\ }{\texttt{\ {]}\ }}\texttt{\ }{\texttt{\ ,\ }}\texttt{\ }{\texttt{\ {[}\ }}\texttt{\ B\ }{\texttt{\ {]}\ }}\texttt{\ }{\texttt{\ )\ }}\texttt{\ }
  is equivalent to
  \texttt{\ }{\texttt{\ list\ }}\texttt{\ }{\texttt{\ {[}\ }}\texttt{\ A\ }{\texttt{\ {]}\ }}\texttt{\ }{\texttt{\ {[}\ }}\texttt{\ B\ }{\texttt{\ {]}\ }}\texttt{\ }
  .
\end{itemize}

Content and code blocks can be nested arbitrarily. In the example below,
\texttt{\ }{\texttt{\ {[}\ }}\texttt{\ hello\ }{\texttt{\ {]}\ }}\texttt{\ }
is joined with the output of
\texttt{\ a\ }{\texttt{\ +\ }}\texttt{\ }{\texttt{\ {[}\ }}\texttt{\ the\ }{\texttt{\ {]}\ }}\texttt{\ }{\texttt{\ +\ }}\texttt{\ b\ }
yielding
\texttt{\ }{\texttt{\ {[}\ }}\texttt{\ hello\ from\ the\ }{\texttt{\ *world*\ }}\texttt{\ }{\texttt{\ {]}\ }}\texttt{\ }
.

\begin{verbatim}
#{
  let a = [from]
  let b = [*world*]
  [hello ]
  a + [ the ] + b
}
\end{verbatim}

\includegraphics[width=5in,height=\textheight,keepaspectratio]{/assets/docs/9fGlmpI93XiZ44REV_aDoQAAAAAAAAAA.png}

\subsection{Bindings and Destructuring}\label{bindings}

As already demonstrated above, variables can be defined with
\texttt{\ }{\texttt{\ let\ }}\texttt{\ } bindings. The variable is
assigned the value of the expression that follows the \texttt{\ =\ }
sign. The assignment of a value is optional, if no value is assigned,
the variable will be initialized as
\texttt{\ }{\texttt{\ none\ }}\texttt{\ } . The
\texttt{\ }{\texttt{\ let\ }}\texttt{\ } keyword can also be used to
create a
\href{/docs/reference/foundations/function/\#defining-functions}{custom
named function} . Variables can be accessed for the rest of the
containing block (or the rest of the file if there is no containing
block).

\begin{verbatim}
#let name = "Typst"
This is #name's documentation.
It explains #name.

#let add(x, y) = x + y
Sum is #add(2, 3).
\end{verbatim}

\includegraphics[width=5in,height=\textheight,keepaspectratio]{/assets/docs/yrL9Iv6avU1LgnwbwwruLwAAAAAAAAAA.png}

Let bindings can also be used to destructure
\href{/docs/reference/foundations/array/}{arrays} and
\href{/docs/reference/foundations/dictionary/}{dictionaries} . In this
case, the left-hand side of the assignment should mirror an array or
dictionary. The \texttt{\ ..\ } operator can be used once in the pattern
to collect the remainder of the array\textquotesingle s or
dictionary\textquotesingle s items.

\begin{verbatim}
#let (x, y) = (1, 2)
The coordinates are #x, #y.

#let (a, .., b) = (1, 2, 3, 4)
The first element is #a.
The last element is #b.

#let books = (
  Shakespeare: "Hamlet",
  Homer: "The Odyssey",
  Austen: "Persuasion",
)

#let (Austen,) = books
Austen wrote #Austen.

#let (Homer: h) = books
Homer wrote #h.

#let (Homer, ..other) = books
#for (author, title) in other [
  #author wrote #title.
]
\end{verbatim}

\includegraphics[width=5in,height=\textheight,keepaspectratio]{/assets/docs/V0qKVNlCRuARFWpYu5iPEAAAAAAAAAAA.png}

You can use the underscore to discard elements in a destructuring
pattern:

\begin{verbatim}
#let (_, y, _) = (1, 2, 3)
The y coordinate is #y.
\end{verbatim}

\includegraphics[width=5in,height=\textheight,keepaspectratio]{/assets/docs/LTIPVXoxwTxHgnZJYm9hXgAAAAAAAAAA.png}

Destructuring also work in argument lists of functions ...

\begin{verbatim}
#let left = (2, 4, 5)
#let right = (3, 2, 6)
#left.zip(right).map(
  ((a,b)) => a + b
)
\end{verbatim}

\includegraphics[width=5in,height=\textheight,keepaspectratio]{/assets/docs/60UdChzzZGHopWziA6zwZwAAAAAAAAAA.png}

... and on the left-hand side of normal assignments. This can be useful
to swap variables among other things.

\begin{verbatim}
#{
  let a = 1
  let b = 2
  (a, b) = (b, a)
  [a = #a, b = #b]
}
\end{verbatim}

\includegraphics[width=5in,height=\textheight,keepaspectratio]{/assets/docs/LCvMaiUJqV2qC8Tphu5-bQAAAAAAAAAA.png}

\subsection{Conditionals}\label{conditionals}

With a conditional, you can display or compute different things
depending on whether some condition is fulfilled. Typst supports
\texttt{\ }{\texttt{\ if\ }}\texttt{\ } ,
\texttt{\ else\ }{\texttt{\ if\ }}\texttt{\ } and \texttt{\ else\ }
expression. When the condition evaluates to
\texttt{\ }{\texttt{\ true\ }}\texttt{\ } , the conditional yields the
value resulting from the if\textquotesingle s body, otherwise yields the
value resulting from the else\textquotesingle s body.

\begin{verbatim}
#if 1 < 2 [
  This is shown
] else [
  This is not.
]
\end{verbatim}

\includegraphics[width=5in,height=\textheight,keepaspectratio]{/assets/docs/nwPf6X84WrGm0BEnqRUsTQAAAAAAAAAA.png}

Each branch can have a code or content block as its body.

\begin{itemize}
\tightlist
\item
  \texttt{\ }{\texttt{\ if\ }}\texttt{\ condition\ }{\texttt{\ \{\ }}\texttt{\ ..\ }{\texttt{\ \}\ }}\texttt{\ }
\item
  \texttt{\ }{\texttt{\ if\ }}\texttt{\ condition\ }{\texttt{\ {[}\ }}\texttt{\ ..\ }{\texttt{\ {]}\ }}\texttt{\ }
\item
  \texttt{\ }{\texttt{\ if\ }}\texttt{\ condition\ }{\texttt{\ {[}\ }}\texttt{\ ..\ }{\texttt{\ {]}\ }}\texttt{\ }{\texttt{\ else\ }}\texttt{\ }{\texttt{\ \{\ }}\texttt{\ ..\ }{\texttt{\ \}\ }}\texttt{\ }
\item
  \texttt{\ }{\texttt{\ if\ }}\texttt{\ condition\ }{\texttt{\ {[}\ }}\texttt{\ ..\ }{\texttt{\ {]}\ }}\texttt{\ }{\texttt{\ else\ }}\texttt{\ }{\texttt{\ if\ }}\texttt{\ condition\ }{\texttt{\ \{\ }}\texttt{\ ..\ }{\texttt{\ \}\ }}\texttt{\ }{\texttt{\ else\ }}\texttt{\ }{\texttt{\ {[}\ }}\texttt{\ ..\ }{\texttt{\ {]}\ }}\texttt{\ }
\end{itemize}

\subsection{Loops}\label{loops}

With loops, you can repeat content or compute something iteratively.
Typst supports two types of loops:
\texttt{\ }{\texttt{\ for\ }}\texttt{\ } and
\texttt{\ }{\texttt{\ while\ }}\texttt{\ } loops. The former iterate
over a specified collection whereas the latter iterate as long as a
condition stays fulfilled. Just like blocks, loops \emph{join} the
results from each iteration into one value.

In the example below, the three sentences created by the for loop join
together into a single content value and the length-1 arrays in the
while loop join together into one larger array.

\begin{verbatim}
#for c in "ABC" [
  #c is a letter.
]

#let n = 2
#while n < 10 {
  n = (n * 2) - 1
  (n,)
}
\end{verbatim}

\includegraphics[width=5in,height=\textheight,keepaspectratio]{/assets/docs/74fsCbbxVkZaLeuPLKRiCwAAAAAAAAAA.png}

For loops can iterate over a variety of collections:

\begin{itemize}
\item
  \texttt{\ }{\texttt{\ for\ }}\texttt{\ value\ }{\texttt{\ in\ }}\texttt{\ array\ }{\texttt{\ \{\ }}\texttt{\ ..\ }{\texttt{\ \}\ }}\texttt{\ }\strut \\
  Iterates over the items in the
  \href{/docs/reference/foundations/array/}{array} . The destructuring
  syntax described in \href{/docs/reference/scripting/\#bindings}{Let
  binding} can also be used here.
\item
  \texttt{\ }{\texttt{\ for\ }}\texttt{\ pair\ }{\texttt{\ in\ }}\texttt{\ dict\ }{\texttt{\ \{\ }}\texttt{\ ..\ }{\texttt{\ \}\ }}\texttt{\ }\strut \\
  Iterates over the key-value pairs of the
  \href{/docs/reference/foundations/dictionary/}{dictionary} . The pairs
  can also be destructured by using
  \texttt{\ }{\texttt{\ for\ }}\texttt{\ }{\texttt{\ (\ }}\texttt{\ key\ }{\texttt{\ ,\ }}\texttt{\ value\ }{\texttt{\ )\ }}\texttt{\ }{\texttt{\ in\ }}\texttt{\ dict\ }{\texttt{\ \{\ }}\texttt{\ ..\ }{\texttt{\ \}\ }}\texttt{\ }
  . It is more efficient than
  \texttt{\ }{\texttt{\ for\ }}\texttt{\ pair\ }{\texttt{\ in\ }}\texttt{\ dict\ }{\texttt{\ .\ }}\texttt{\ }{\texttt{\ pairs\ }}\texttt{\ }{\texttt{\ (\ }}\texttt{\ }{\texttt{\ )\ }}\texttt{\ }{\texttt{\ \{\ }}\texttt{\ ..\ }{\texttt{\ \}\ }}\texttt{\ }
  because it doesn\textquotesingle t create a temporary array of all
  key-value pairs.
\item
  \texttt{\ }{\texttt{\ for\ }}\texttt{\ letter\ }{\texttt{\ in\ }}\texttt{\ }{\texttt{\ "abc"\ }}\texttt{\ }{\texttt{\ \{\ }}\texttt{\ ..\ }{\texttt{\ \}\ }}\texttt{\ }\strut \\
  Iterates over the characters of the
  \href{/docs/reference/foundations/str/}{string} . Technically, it
  iterates over the grapheme clusters of the string. Most of the time, a
  grapheme cluster is just a single codepoint. However, a grapheme
  cluster could contain multiple codepoints, like a flag emoji.
\item
  \texttt{\ }{\texttt{\ for\ }}\texttt{\ byte\ }{\texttt{\ in\ }}\texttt{\ }{\texttt{\ bytes\ }}\texttt{\ }{\texttt{\ (\ }}\texttt{\ }{\texttt{\ "😀"\ }}\texttt{\ }{\texttt{\ )\ }}\texttt{\ }{\texttt{\ \{\ }}\texttt{\ ..\ }{\texttt{\ \}\ }}\texttt{\ }\strut \\
  Iterates over the \href{/docs/reference/foundations/bytes/}{bytes} ,
  which can be converted from a
  \href{/docs/reference/foundations/str/}{string} or
  \href{/docs/reference/data-loading/read/}{read} from a file without
  encoding. Each byte value is an
  \href{/docs/reference/foundations/int/}{integer} between
  \texttt{\ }{\texttt{\ 0\ }}\texttt{\ } and
  \texttt{\ }{\texttt{\ 255\ }}\texttt{\ } .
\end{itemize}

To control the execution of the loop, Typst provides the
\texttt{\ }{\texttt{\ break\ }}\texttt{\ } and
\texttt{\ }{\texttt{\ continue\ }}\texttt{\ } statements. The former
performs an early exit from the loop while the latter skips ahead to the
next iteration of the loop.

\begin{verbatim}
#for letter in "abc nope" {
  if letter == " " {
    break
  }

  letter
}
\end{verbatim}

\includegraphics[width=5in,height=\textheight,keepaspectratio]{/assets/docs/i6FFy2h6Ocj5FGq9VflD-QAAAAAAAAAA.png}

The body of a loop can be a code or content block:

\begin{itemize}
\tightlist
\item
  \texttt{\ }{\texttt{\ for\ }}\texttt{\ ..\ in\ collection\ }{\texttt{\ \{\ }}\texttt{\ ..\ }{\texttt{\ \}\ }}\texttt{\ }
\item
  \texttt{\ }{\texttt{\ for\ }}\texttt{\ ..\ in\ collection\ }{\texttt{\ {[}\ }}\texttt{\ ..\ }{\texttt{\ {]}\ }}\texttt{\ }
\item
  \texttt{\ }{\texttt{\ while\ }}\texttt{\ condition\ }{\texttt{\ \{\ }}\texttt{\ ..\ }{\texttt{\ \}\ }}\texttt{\ }
\item
  \texttt{\ }{\texttt{\ while\ }}\texttt{\ condition\ }{\texttt{\ {[}\ }}\texttt{\ ..\ }{\texttt{\ {]}\ }}\texttt{\ }
\end{itemize}

\subsection{Fields}\label{fields}

You can use \emph{dot notation} to access fields on a value. For values
of type
\href{/docs/reference/foundations/content/}{\texttt{\ content\ }} , you
can also use the
\href{/docs/reference/foundations/content/\#definitions-fields}{\texttt{\ fields\ }}
function to list the fields.

The value in question can be either:

\begin{itemize}
\tightlist
\item
  a \href{/docs/reference/foundations/dictionary/}{dictionary} that has
  the specified key,
\item
  a \href{/docs/reference/symbols/symbol/}{symbol} that has the
  specified modifier,
\item
  a \href{/docs/reference/foundations/module/}{module} containing the
  specified definition,
\item
  \href{/docs/reference/foundations/content/}{content} consisting of an
  element that has the specified field. The available fields match the
  arguments of the
  \href{/docs/reference/foundations/function/\#element-functions}{element
  function} that were given when the element was constructed.
\end{itemize}

\begin{verbatim}
#let it = [= Heading]
#it.body \
#it.depth \
#it.fields()

#let dict = (greet: "Hello")
#dict.greet \
#emoji.face
\end{verbatim}

\includegraphics[width=5in,height=\textheight,keepaspectratio]{/assets/docs/5WDPdcADtV7mFakfulSQigAAAAAAAAAA.png}

\subsection{Methods}\label{methods}

A \emph{method call} is a convenient way to call a function that is
scoped to a value\textquotesingle s
\href{/docs/reference/foundations/type/}{type} . For example, we can
call the
\href{/docs/reference/foundations/str/\#definitions-len}{\texttt{\ str.len\ }}
function in the following two equivalent ways:

\begin{verbatim}
#str.len("abc") is the same as
#"abc".len()
\end{verbatim}

\includegraphics[width=5in,height=\textheight,keepaspectratio]{/assets/docs/j8kTSWLqLKb4876qGnfJQAAAAAAAAAAA.png}

The structure of a method call is
\texttt{\ value\ }{\texttt{\ .\ }}\texttt{\ }{\texttt{\ method\ }}\texttt{\ }{\texttt{\ (\ }}\texttt{\ }{\texttt{\ ..\ }}\texttt{\ args\ }{\texttt{\ )\ }}\texttt{\ }
and its equivalent full function call is
\texttt{\ }{\texttt{\ type\ }}\texttt{\ }{\texttt{\ (\ }}\texttt{\ value\ }{\texttt{\ )\ }}\texttt{\ }{\texttt{\ .\ }}\texttt{\ }{\texttt{\ method\ }}\texttt{\ }{\texttt{\ (\ }}\texttt{\ value\ }{\texttt{\ ,\ }}\texttt{\ }{\texttt{\ ..\ }}\texttt{\ args\ }{\texttt{\ )\ }}\texttt{\ }
. The documentation of each type lists it\textquotesingle s scoped
functions. You cannot currently define your own methods.

\begin{verbatim}
#let values = (1, 2, 3, 4)
#values.pop() \
#values.len() \

#("a, b, c"
    .split(", ")
    .join[ --- ])

#"abc".len() is the same as
#str.len("abc")
\end{verbatim}

\includegraphics[width=5in,height=\textheight,keepaspectratio]{/assets/docs/acgddNIdTiEri93pewLJSQAAAAAAAAAA.png}

There are a few special functions that modify the value they are called
on (e.g.
\href{/docs/reference/foundations/array/\#definitions-push}{\texttt{\ array.push\ }}
). These functions \emph{must} be called in method form. In some cases,
when the method is only called for its side effect, its return value
should be ignored (and not participate in joining). The canonical way to
discard a value is with a let binding:
\texttt{\ }{\texttt{\ let\ }}\texttt{\ \_\ }{\texttt{\ =\ }}\texttt{\ array\ }{\texttt{\ .\ }}\texttt{\ }{\texttt{\ remove\ }}\texttt{\ }{\texttt{\ (\ }}\texttt{\ }{\texttt{\ 1\ }}\texttt{\ }{\texttt{\ )\ }}\texttt{\ }
.

\subsection{Modules}\label{modules}

You can split up your Typst projects into multiple files called
\emph{modules.} A module can refer to the content and definitions of
another module in multiple ways:

\begin{itemize}
\item
  \textbf{Including:}
  \texttt{\ }{\texttt{\ include\ }}\texttt{\ }{\texttt{\ "bar.typ"\ }}\texttt{\ }\\
  Evaluates the file at the path \texttt{\ bar.typ\ } and returns the
  resulting \href{/docs/reference/foundations/content/}{content} .
\item
  \textbf{Import:}
  \texttt{\ }{\texttt{\ import\ }}\texttt{\ }{\texttt{\ "bar.typ"\ }}\texttt{\ }\\
  Evaluates the file at the path \texttt{\ bar.typ\ } and inserts the
  resulting \href{/docs/reference/foundations/module/}{module} into the
  current scope as \texttt{\ bar\ } (filename without extension). You
  can use the \texttt{\ as\ } keyword to rename the imported module:
  \texttt{\ }{\texttt{\ import\ }}\texttt{\ }{\texttt{\ "bar.typ"\ }}\texttt{\ }{\texttt{\ as\ }}\texttt{\ baz\ }
  . You can import nested items using dot notation:
  \texttt{\ }{\texttt{\ import\ }}\texttt{\ }{\texttt{\ "bar.typ"\ }}\texttt{\ }{\texttt{\ :\ }}\texttt{\ baz\ }{\texttt{\ .\ }}\texttt{\ a\ }
  .
\item
  \textbf{Import items:}
  \texttt{\ }{\texttt{\ import\ }}\texttt{\ }{\texttt{\ "bar.typ"\ }}\texttt{\ }{\texttt{\ :\ }}\texttt{\ a\ }{\texttt{\ ,\ }}\texttt{\ b\ }\\
  Evaluates the file at the path \texttt{\ bar.typ\ } , extracts the
  values of the variables \texttt{\ a\ } and \texttt{\ b\ } (that need
  to be defined in \texttt{\ bar.typ\ } , e.g. through
  \texttt{\ }{\texttt{\ let\ }}\texttt{\ } bindings) and defines them in
  the current file. Replacing \texttt{\ a,\ b\ } with \texttt{\ *\ }
  loads all variables defined in a module. You can use the
  \texttt{\ as\ } keyword to rename the individual items:
  \texttt{\ }{\texttt{\ import\ }}\texttt{\ }{\texttt{\ "bar.typ"\ }}\texttt{\ }{\texttt{\ :\ }}\texttt{\ a\ }{\texttt{\ as\ }}\texttt{\ one\ }{\texttt{\ ,\ }}\texttt{\ b\ }{\texttt{\ as\ }}\texttt{\ two\ }
\end{itemize}

Instead of a path, you can also use a
\href{/docs/reference/foundations/module/}{module value} , as shown in
the following example:

\begin{verbatim}
#import emoji: face
#face.grin
\end{verbatim}

\includegraphics[width=5in,height=\textheight,keepaspectratio]{/assets/docs/nDjZVIO8y_dHmoJCcIerKgAAAAAAAAAA.png}

\subsection{Packages}\label{packages}

To reuse building blocks across projects, you can also create and import
Typst \emph{packages.} A package import is specified as a triple of a
namespace, a name, and a version.

\begin{verbatim}
#import "@preview/example:0.1.0": add
#add(2, 7)
\end{verbatim}

\includegraphics[width=5in,height=\textheight,keepaspectratio]{/assets/docs/PZlFYONRfjgFeuwUfEQd5gAAAAAAAAAA.png}

The \texttt{\ preview\ } namespace contains packages shared by the
community. You can find all available community packages on
\href{https://typst.app/universe/}{Typst Universe} .

If you are using Typst locally, you can also create your own
system-local packages. For more details on this, see the
\href{https://github.com/typst/packages}{package repository} .

\subsection{Operators}\label{operators}

The following table lists all available unary and binary operators with
effect, arity (unary, binary) and precedence level (higher binds
stronger).

\begin{longtable}[]{@{}clcc@{}}
\toprule\noalign{}
Operator & Effect & Arity & Precedence \\
\midrule\noalign{}
\endhead
\bottomrule\noalign{}
\endlastfoot
\texttt{\ }{\texttt{\ -\ }}\texttt{\ } & Negation & Unary & 7 \\
\texttt{\ }{\texttt{\ +\ }}\texttt{\ } & No effect (exists for symmetry)
& Unary & 7 \\
\texttt{\ *\ } & Multiplication & Binary & 6 \\
\texttt{\ /\ } & Division & Binary & 6 \\
\texttt{\ }{\texttt{\ +\ }}\texttt{\ } & Addition & Binary & 5 \\
\texttt{\ }{\texttt{\ -\ }}\texttt{\ } & Subtraction & Binary & 5 \\
\texttt{\ ==\ } & Check equality & Binary & 4 \\
\texttt{\ !=\ } & Check inequality & Binary & 4 \\
\texttt{\ \textless{}\ } & Check less-than & Binary & 4 \\
\texttt{\ \textless{}=\ } & Check less-than or equal & Binary & 4 \\
\texttt{\ \textgreater{}\ } & Check greater-than & Binary & 4 \\
\texttt{\ \textgreater{}=\ } & Check greater-than or equal & Binary &
4 \\
\texttt{\ in\ } & Check if in collection & Binary & 4 \\
\texttt{\ }{\texttt{\ not\ }}\texttt{\ }{\texttt{\ in\ }}\texttt{\ } &
Check if not in collection & Binary & 4 \\
\texttt{\ }{\texttt{\ not\ }}\texttt{\ } & Logical "not" & Unary & 3 \\
\texttt{\ and\ } & Short-circuiting logical "and" & Binary & 3 \\
\texttt{\ or\ } & Short-circuiting logical "or & Binary & 2 \\
\texttt{\ =\ } & Assignment & Binary & 1 \\
\texttt{\ +=\ } & Add-Assignment & Binary & 1 \\
\texttt{\ -=\ } & Subtraction-Assignment & Binary & 1 \\
\texttt{\ *=\ } & Multiplication-Assignment & Binary & 1 \\
\texttt{\ /=\ } & Division-Assignment & Binary & 1 \\
\end{longtable}

\href{/docs/reference/styling/}{\pandocbounded{\includesvg[keepaspectratio]{/assets/icons/16-arrow-right.svg}}}

{ Styling } { Previous page }

\href{/docs/reference/context/}{\pandocbounded{\includesvg[keepaspectratio]{/assets/icons/16-arrow-right.svg}}}

{ Context } { Next page }


\section{Docs LaTeX/typst.app/docs/reference/text.tex}
\title{typst.app/docs/reference/text}

\begin{itemize}
\tightlist
\item
  \href{/docs}{\includesvg[width=0.16667in,height=0.16667in]{/assets/icons/16-docs-dark.svg}}
\item
  \includesvg[width=0.16667in,height=0.16667in]{/assets/icons/16-arrow-right.svg}
\item
  \href{/docs/reference/}{Reference}
\item
  \includesvg[width=0.16667in,height=0.16667in]{/assets/icons/16-arrow-right.svg}
\item
  \href{/docs/reference/text/}{Text}
\end{itemize}

\section{Text}\label{summary}

Text styling.

The \href{/docs/reference/text/text/}{text function} is of particular
interest.

\subsection{Definitions}\label{definitions}

\begin{itemize}
\tightlist
\item
  \href{/docs/reference/text/highlight/}{\texttt{\ highlight\ }} {
  Highlights text with a background color. }
\item
  \href{/docs/reference/text/linebreak/}{\texttt{\ linebreak\ }} {
  Inserts a line break. }
\item
  \href{/docs/reference/text/lorem/}{\texttt{\ lorem\ }} { Creates blind
  text. }
\item
  \href{/docs/reference/text/lower/}{\texttt{\ lower\ }} { Converts a
  string or content to lowercase. }
\item
  \href{/docs/reference/text/overline/}{\texttt{\ overline\ }} { Adds a
  line over text. }
\item
  \href{/docs/reference/text/raw/}{\texttt{\ raw\ }} { Raw text with
  optional syntax highlighting. }
\item
  \href{/docs/reference/text/smallcaps/}{\texttt{\ smallcaps\ }} {
  Displays text in small capitals. }
\item
  \href{/docs/reference/text/smartquote/}{\texttt{\ smartquote\ }} { A
  language-aware quote that reacts to its context. }
\item
  \href{/docs/reference/text/strike/}{\texttt{\ strike\ }} { Strikes
  through text. }
\item
  \href{/docs/reference/text/sub/}{\texttt{\ sub\ }} { Renders text in
  subscript. }
\item
  \href{/docs/reference/text/super/}{\texttt{\ super\ }} { Renders text
  in superscript. }
\item
  \href{/docs/reference/text/text/}{\texttt{\ text\ }} { Customizes the
  look and layout of text in a variety of ways. }
\item
  \href{/docs/reference/text/underline/}{\texttt{\ underline\ }} {
  Underlines text. }
\item
  \href{/docs/reference/text/upper/}{\texttt{\ upper\ }} { Converts a
  string or content to uppercase. }
\end{itemize}

\href{/docs/reference/model/terms/}{\pandocbounded{\includesvg[keepaspectratio]{/assets/icons/16-arrow-right.svg}}}

{ Term List } { Previous page }

\href{/docs/reference/text/highlight/}{\pandocbounded{\includesvg[keepaspectratio]{/assets/icons/16-arrow-right.svg}}}

{ Highlight } { Next page }


\section{Docs LaTeX/typst.app/docs/reference/styling.tex}
\title{typst.app/docs/reference/styling}

\begin{itemize}
\tightlist
\item
  \href{/docs}{\includesvg[width=0.16667in,height=0.16667in]{/assets/icons/16-docs-dark.svg}}
\item
  \includesvg[width=0.16667in,height=0.16667in]{/assets/icons/16-arrow-right.svg}
\item
  \href{/docs/reference/}{Reference}
\item
  \includesvg[width=0.16667in,height=0.16667in]{/assets/icons/16-arrow-right.svg}
\item
  \href{/docs/reference/styling/}{Styling}
\end{itemize}

\section{Styling}\label{styling}

Typst includes a flexible styling system that automatically applies
styling of your choice to your document. With \emph{set rules,} you can
configure basic properties of elements. This way, you create most common
styles. However, there might not be a built-in property for everything
you wish to do. For this reason, Typst further supports \emph{show
rules} that can completely redefine the appearance of elements.

\subsection{Set rules}\label{set-rules}

With set rules, you can customize the appearance of elements. They are
written as a \href{/docs/reference/foundations/function/}{function call}
to an
\href{/docs/reference/foundations/function/\#element-functions}{element
function} preceded by the \texttt{\ }{\texttt{\ set\ }}\texttt{\ }
keyword (or
\texttt{\ }{\texttt{\ \#\ }}\texttt{\ }{\texttt{\ set\ }}\texttt{\ } in
markup). Only optional parameters of that function can be provided to
the set rule. Refer to each function\textquotesingle s documentation to
see which parameters are optional. In the example below, we use two set
rules to change the
\href{/docs/reference/text/text/\#parameters-font}{font family} and
\href{/docs/reference/model/heading/\#parameters-numbering}{heading
numbering} .

\begin{verbatim}
#set heading(numbering: "I.")
#set text(
  font: "New Computer Modern"
)

= Introduction
With set rules, you can style
your document.
\end{verbatim}

\includegraphics[width=5in,height=\textheight,keepaspectratio]{/assets/docs/nW0VZeyhJmtpweEOjJR_fgAAAAAAAAAA.png}

A top level set rule stays in effect until the end of the file. When
nested inside of a block, it is only in effect until the end of that
block. With a block, you can thus restrict the effect of a rule to a
particular segment of your document. Below, we use a content block to
scope the list styling to one particular list.

\begin{verbatim}
This list is affected: #[
  #set list(marker: [--])
  - Dash
]

This one is not:
- Bullet
\end{verbatim}

\includegraphics[width=5in,height=\textheight,keepaspectratio]{/assets/docs/6ckQbXFff1zDBcdWezXxpgAAAAAAAAAA.png}

Sometimes, you\textquotesingle ll want to apply a set rule
conditionally. For this, you can use a \emph{set-if} rule.

\begin{verbatim}
#let task(body, critical: false) = {
  set text(red) if critical
  [- #body]
}

#task(critical: true)[Food today?]
#task(critical: false)[Work deadline]
\end{verbatim}

\includegraphics[width=5in,height=\textheight,keepaspectratio]{/assets/docs/_UlmqEOdrmM6d-OQ9TsAXwAAAAAAAAAA.png}

\subsection{Show rules}\label{show-rules}

With show rules, you can deeply customize the look of a type of element.
The most basic form of show rule is a \emph{show-set rule.} Such a rule
is written as the \texttt{\ }{\texttt{\ show\ }}\texttt{\ } keyword
followed by a \href{/docs/reference/foundations/selector/}{selector} , a
colon and then a set rule. The most basic form of selector is an
\href{/docs/reference/foundations/function/\#element-functions}{element
function} . This lets the set rule only apply to the selected element.
In the example below, headings become dark blue while all other text
stays black.

\begin{verbatim}
#show heading: set text(navy)

= This is navy-blue
But this stays black.
\end{verbatim}

\includegraphics[width=5in,height=\textheight,keepaspectratio]{/assets/docs/DS2Pe3XVhhMMVWT9eUfjSQAAAAAAAAAA.png}

With show-set rules you can mix and match properties from different
functions to achieve many different effects. But they still limit you to
what is predefined in Typst. For maximum flexibility, you can instead
write a show rule that defines how to format an element from scratch. To
write such a show rule, replace the set rule after the colon with an
arbitrary \href{/docs/reference/foundations/function/}{function} . This
function receives the element in question and can return arbitrary
content. The available \href{/docs/reference/scripting/\#fields}{fields}
on the element passed to the function again match the parameters of the
respective element function. Below, we define a show rule that formats
headings for a fantasy encyclopedia.

\begin{verbatim}
#set heading(numbering: "(I)")
#show heading: it => [
  #set align(center)
  #set text(font: "Inria Serif")
  \~ #emph(it.body)
     #counter(heading).display(
       it.numbering
     ) \~
]

= Dragon
With a base health of 15, the
dragon is the most powerful
creature.

= Manticore
While less powerful than the
dragon, the manticore gets
extra style points.
\end{verbatim}

\includegraphics[width=5in,height=\textheight,keepaspectratio]{/assets/docs/YrvkqpSwoILjuqAerzw9CAAAAAAAAAAA.png}

Like set rules, show rules are in effect until the end of the current
block or file.

Instead of a function, the right-hand side of a show rule can also take
a literal string or content block that should be directly substituted
for the element. And apart from a function, the left-hand side of a show
rule can also take a number of other \emph{selectors} that define what
to apply the transformation to:

\begin{itemize}
\item
  \textbf{Everything:}
  \texttt{\ }{\texttt{\ show\ }}\texttt{\ }{\texttt{\ :\ }}\texttt{\ rest\ }{\texttt{\ =\textgreater{}\ }}\texttt{\ ..\ }\\
  Transform everything after the show rule. This is useful to apply a
  more complex layout to your whole document without wrapping everything
  in a giant function call.
\item
  \textbf{Text:}
  \texttt{\ }{\texttt{\ show\ }}\texttt{\ }{\texttt{\ "Text"\ }}\texttt{\ }{\texttt{\ :\ }}\texttt{\ ..\ }\\
  Style, transform or replace text.
\item
  \textbf{Regex:}
  \texttt{\ }{\texttt{\ show\ }}\texttt{\ }{\texttt{\ regex\ }}\texttt{\ }{\texttt{\ (\ }}\texttt{\ }{\texttt{\ "\textbackslash{}w+"\ }}\texttt{\ }{\texttt{\ )\ }}\texttt{\ }{\texttt{\ :\ }}\texttt{\ ..\ }\\
  Select and transform text with a regular expression for even more
  flexibility. See the documentation of the
  \href{/docs/reference/foundations/regex/}{\texttt{\ regex\ } type} for
  details.
\item
  \textbf{Function with fields:}
  \texttt{\ }{\texttt{\ show\ }}\texttt{\ heading\ }{\texttt{\ .\ }}\texttt{\ }{\texttt{\ where\ }}\texttt{\ }{\texttt{\ (\ }}\texttt{\ level\ }{\texttt{\ :\ }}\texttt{\ }{\texttt{\ 1\ }}\texttt{\ }{\texttt{\ )\ }}\texttt{\ }{\texttt{\ :\ }}\texttt{\ ..\ }\\
  Transform only elements that have the specified fields. For example,
  you might want to only change the style of level-1 headings.
\item
  \textbf{Label:}
  \texttt{\ }{\texttt{\ show\ }}\texttt{\ }{\texttt{\ \textless{}intro\textgreater{}\ }}\texttt{\ }{\texttt{\ :\ }}\texttt{\ ..\ }\\
  Select and transform elements that have the specified label. See the
  documentation of the
  \href{/docs/reference/foundations/label/}{\texttt{\ label\ } type} for
  more details.
\end{itemize}

\begin{verbatim}
#show "Project": smallcaps
#show "badly": "great"

We started Project in 2019
and are still working on it.
Project is progressing badly.
\end{verbatim}

\includegraphics[width=5in,height=\textheight,keepaspectratio]{/assets/docs/NBzIViTspdnPhsbo3WGDLAAAAAAAAAAA.png}

\href{/docs/reference/syntax/}{\pandocbounded{\includesvg[keepaspectratio]{/assets/icons/16-arrow-right.svg}}}

{ Syntax } { Previous page }

\href{/docs/reference/scripting/}{\pandocbounded{\includesvg[keepaspectratio]{/assets/icons/16-arrow-right.svg}}}

{ Scripting } { Next page }


\section{Docs LaTeX/typst.app/docs/reference/foundations.tex}
\title{typst.app/docs/reference/foundations}

\begin{itemize}
\tightlist
\item
  \href{/docs}{\includesvg[width=0.16667in,height=0.16667in]{/assets/icons/16-docs-dark.svg}}
\item
  \includesvg[width=0.16667in,height=0.16667in]{/assets/icons/16-arrow-right.svg}
\item
  \href{/docs/reference/}{Reference}
\item
  \includesvg[width=0.16667in,height=0.16667in]{/assets/icons/16-arrow-right.svg}
\item
  \href{/docs/reference/foundations/}{Foundations}
\end{itemize}

\section{Foundations}\label{summary}

Foundational types and functions.

Here, you\textquotesingle ll find documentation for basic data types
like \href{/docs/reference/foundations/int/}{integers} and
\href{/docs/reference/foundations/str/}{strings} as well as details
about core computational functions.

\subsection{Definitions}\label{definitions}

\begin{itemize}
\tightlist
\item
  \href{/docs/reference/foundations/arguments/}{\texttt{\ arguments\ }}
  { Captured arguments to a function. }
\item
  \href{/docs/reference/foundations/array/}{\texttt{\ array\ }} { A
  sequence of values. }
\item
  \href{/docs/reference/foundations/assert/}{\texttt{\ assert\ }} {
  Ensures that a condition is fulfilled. }
\item
  \href{/docs/reference/foundations/auto/}{\texttt{\ auto\ }} { A value
  that indicates a smart default. }
\item
  \href{/docs/reference/foundations/bool/}{\texttt{\ bool\ }} { A type
  with two states. }
\item
  \href{/docs/reference/foundations/bytes/}{\texttt{\ bytes\ }} { A
  sequence of bytes. }
\item
  \href{/docs/reference/foundations/calc}{calc} { Module for
  calculations and processing of numeric values. }
\item
  \href{/docs/reference/foundations/content/}{\texttt{\ content\ }} { A
  piece of document content. }
\item
  \href{/docs/reference/foundations/datetime/}{\texttt{\ datetime\ }} {
  Represents a date, a time, or a combination of both. }
\item
  \href{/docs/reference/foundations/decimal/}{\texttt{\ decimal\ }} { A
  fixed-point decimal number type. }
\item
  \href{/docs/reference/foundations/dictionary/}{\texttt{\ dictionary\ }}
  { A map from string keys to values. }
\item
  \href{/docs/reference/foundations/duration/}{\texttt{\ duration\ }} {
  Represents a positive or negative span of time. }
\item
  \href{/docs/reference/foundations/eval/}{\texttt{\ eval\ }} {
  Evaluates a string as Typst code. }
\item
  \href{/docs/reference/foundations/float/}{\texttt{\ float\ }} { A
  floating-point number. }
\item
  \href{/docs/reference/foundations/function/}{\texttt{\ function\ }} {
  A mapping from argument values to a return value. }
\item
  \href{/docs/reference/foundations/int/}{\texttt{\ int\ }} { A whole
  number. }
\item
  \href{/docs/reference/foundations/label/}{\texttt{\ label\ }} { A
  label for an element. }
\item
  \href{/docs/reference/foundations/module/}{\texttt{\ module\ }} { An
  evaluated module, either built-in or resulting from a file. }
\item
  \href{/docs/reference/foundations/none/}{\texttt{\ none\ }} { A value
  that indicates the absence of any other value. }
\item
  \href{/docs/reference/foundations/panic/}{\texttt{\ panic\ }} { Fails
  with an error. }
\item
  \href{/docs/reference/foundations/plugin/}{\texttt{\ plugin\ }} { A
  WebAssembly plugin. }
\item
  \href{/docs/reference/foundations/regex/}{\texttt{\ regex\ }} { A
  regular expression. }
\item
  \href{/docs/reference/foundations/repr/}{\texttt{\ repr\ }} { Returns
  the string representation of a value. }
\item
  \href{/docs/reference/foundations/selector/}{\texttt{\ selector\ }} {
  A filter for selecting elements within the document. }
\item
  \href{/docs/reference/foundations/str/}{\texttt{\ str\ }} { A sequence
  of Unicode codepoints. }
\item
  \href{/docs/reference/foundations/style/}{\texttt{\ style\ }} {
  Provides access to active styles. }
\item
  \href{/docs/reference/foundations/sys}{sys} { Module for system
  interactions. }
\item
  \href{/docs/reference/foundations/type/}{\texttt{\ type\ }} {
  Describes a kind of value. }
\item
  \href{/docs/reference/foundations/version/}{\texttt{\ version\ }} { A
  version with an arbitrary number of components. }
\end{itemize}

\href{/docs/reference/context/}{\pandocbounded{\includesvg[keepaspectratio]{/assets/icons/16-arrow-right.svg}}}

{ Context } { Previous page }

\href{/docs/reference/foundations/arguments/}{\pandocbounded{\includesvg[keepaspectratio]{/assets/icons/16-arrow-right.svg}}}

{ Arguments } { Next page }


\section{Docs LaTeX/typst.app/docs/reference/layout.tex}
\title{typst.app/docs/reference/layout}

\begin{itemize}
\tightlist
\item
  \href{/docs}{\includesvg[width=0.16667in,height=0.16667in]{/assets/icons/16-docs-dark.svg}}
\item
  \includesvg[width=0.16667in,height=0.16667in]{/assets/icons/16-arrow-right.svg}
\item
  \href{/docs/reference/}{Reference}
\item
  \includesvg[width=0.16667in,height=0.16667in]{/assets/icons/16-arrow-right.svg}
\item
  \href{/docs/reference/layout/}{Layout}
\end{itemize}

\section{Layout}\label{summary}

Arranging elements on the page in different ways.

By combining layout functions, you can create complex and automatic
layouts.

\subsection{Definitions}\label{definitions}

\begin{itemize}
\tightlist
\item
  \href{/docs/reference/layout/align/}{\texttt{\ align\ }} { Aligns
  content horizontally and vertically. }
\item
  \href{/docs/reference/layout/alignment/}{\texttt{\ alignment\ }} {
  Where to {[}align{]} something along an axis. }
\item
  \href{/docs/reference/layout/angle/}{\texttt{\ angle\ }} { An angle
  describing a rotation. }
\item
  \href{/docs/reference/layout/block/}{\texttt{\ block\ }} { A
  block-level container. }
\item
  \href{/docs/reference/layout/box/}{\texttt{\ box\ }} { An inline-level
  container that sizes content. }
\item
  \href{/docs/reference/layout/colbreak/}{\texttt{\ colbreak\ }} {
  Forces a column break. }
\item
  \href{/docs/reference/layout/columns/}{\texttt{\ columns\ }} {
  Separates a region into multiple equally sized columns. }
\item
  \href{/docs/reference/layout/direction/}{\texttt{\ direction\ }} { The
  four directions into which content can be laid out. }
\item
  \href{/docs/reference/layout/fraction/}{\texttt{\ fraction\ }} {
  Defines how the remaining space in a layout is distributed. }
\item
  \href{/docs/reference/layout/grid/}{\texttt{\ grid\ }} { Arranges
  content in a grid. }
\item
  \href{/docs/reference/layout/h/}{\texttt{\ h\ }} { Inserts horizontal
  spacing into a paragraph. }
\item
  \href{/docs/reference/layout/hide/}{\texttt{\ hide\ }} { Hides content
  without affecting layout. }
\item
  \href{/docs/reference/layout/layout/}{\texttt{\ layout\ }} { Provides
  access to the current outer container\textquotesingle s (or
  page\textquotesingle s, if none) }
\item
  \href{/docs/reference/layout/length/}{\texttt{\ length\ }} { A size or
  distance, possibly expressed with contextual units. }
\item
  \href{/docs/reference/layout/measure/}{\texttt{\ measure\ }} {
  Measures the layouted size of content. }
\item
  \href{/docs/reference/layout/move/}{\texttt{\ move\ }} { Moves content
  without affecting layout. }
\item
  \href{/docs/reference/layout/pad/}{\texttt{\ pad\ }} { Adds spacing
  around content. }
\item
  \href{/docs/reference/layout/page/}{\texttt{\ page\ }} { Layouts its
  child onto one or multiple pages. }
\item
  \href{/docs/reference/layout/pagebreak/}{\texttt{\ pagebreak\ }} { A
  manual page break. }
\item
  \href{/docs/reference/layout/place/}{\texttt{\ place\ }} { Places
  content relatively to its parent container. }
\item
  \href{/docs/reference/layout/ratio/}{\texttt{\ ratio\ }} { A ratio of
  a whole. }
\item
  \href{/docs/reference/layout/relative/}{\texttt{\ relative\ }} { A
  length in relation to some known length. }
\item
  \href{/docs/reference/layout/repeat/}{\texttt{\ repeat\ }} { Repeats
  content to the available space. }
\item
  \href{/docs/reference/layout/rotate/}{\texttt{\ rotate\ }} { Rotates
  content without affecting layout. }
\item
  \href{/docs/reference/layout/scale/}{\texttt{\ scale\ }} { Scales
  content without affecting layout. }
\item
  \href{/docs/reference/layout/skew/}{\texttt{\ skew\ }} { Skews
  content. }
\item
  \href{/docs/reference/layout/stack/}{\texttt{\ stack\ }} { Arranges
  content and spacing horizontally or vertically. }
\item
  \href{/docs/reference/layout/v/}{\texttt{\ v\ }} { Inserts vertical
  spacing into a flow of blocks. }
\end{itemize}

\href{/docs/reference/symbols/symbol/}{\pandocbounded{\includesvg[keepaspectratio]{/assets/icons/16-arrow-right.svg}}}

{ Symbol } { Previous page }

\href{/docs/reference/layout/align/}{\pandocbounded{\includesvg[keepaspectratio]{/assets/icons/16-arrow-right.svg}}}

{ Align } { Next page }


\section{Docs LaTeX/typst.app/docs/reference/data-loading.tex}
\title{typst.app/docs/reference/data-loading}

\begin{itemize}
\tightlist
\item
  \href{/docs}{\includesvg[width=0.16667in,height=0.16667in]{/assets/icons/16-docs-dark.svg}}
\item
  \includesvg[width=0.16667in,height=0.16667in]{/assets/icons/16-arrow-right.svg}
\item
  \href{/docs/reference/}{Reference}
\item
  \includesvg[width=0.16667in,height=0.16667in]{/assets/icons/16-arrow-right.svg}
\item
  \href{/docs/reference/data-loading/}{Data Loading}
\end{itemize}

\section{Data Loading}\label{summary}

Data loading from external files.

These functions help you with loading and embedding data, for example
from the results of an experiment.

\subsection{Definitions}\label{definitions}

\begin{itemize}
\tightlist
\item
  \href{/docs/reference/data-loading/cbor/}{\texttt{\ cbor\ }} { Reads
  structured data from a CBOR file. }
\item
  \href{/docs/reference/data-loading/csv/}{\texttt{\ csv\ }} { Reads
  structured data from a CSV file. }
\item
  \href{/docs/reference/data-loading/json/}{\texttt{\ json\ }} { Reads
  structured data from a JSON file. }
\item
  \href{/docs/reference/data-loading/read/}{\texttt{\ read\ }} { Reads
  plain text or data from a file. }
\item
  \href{/docs/reference/data-loading/toml/}{\texttt{\ toml\ }} { Reads
  structured data from a TOML file. }
\item
  \href{/docs/reference/data-loading/xml/}{\texttt{\ xml\ }} { Reads
  structured data from an XML file. }
\item
  \href{/docs/reference/data-loading/yaml/}{\texttt{\ yaml\ }} { Reads
  structured data from a YAML file. }
\end{itemize}

\href{/docs/reference/introspection/state/}{\pandocbounded{\includesvg[keepaspectratio]{/assets/icons/16-arrow-right.svg}}}

{ State } { Previous page }

\href{/docs/reference/data-loading/cbor/}{\pandocbounded{\includesvg[keepaspectratio]{/assets/icons/16-arrow-right.svg}}}

{ CBOR } { Next page }


\section{Docs LaTeX/typst.app/docs/reference/context.tex}
\title{typst.app/docs/reference/context}

\begin{itemize}
\tightlist
\item
  \href{/docs}{\includesvg[width=0.16667in,height=0.16667in]{/assets/icons/16-docs-dark.svg}}
\item
  \includesvg[width=0.16667in,height=0.16667in]{/assets/icons/16-arrow-right.svg}
\item
  \href{/docs/reference/}{Reference}
\item
  \includesvg[width=0.16667in,height=0.16667in]{/assets/icons/16-arrow-right.svg}
\item
  \href{/docs/reference/context/}{Context}
\end{itemize}

\section{Context}\label{context}

Sometimes, we want to create content that reacts to its location in the
document. This could be a localized phrase that depends on the
configured text language or something as simple as a heading number
which prints the right value based on how many headings came before it.
However, Typst code isn\textquotesingle t directly aware of its location
in the document. Some code at the beginning of the source text could
yield content that ends up at the back of the document.

To produce content that is reactive to its surroundings, we must thus
specifically instruct Typst: We do this with the
\texttt{\ }{\texttt{\ context\ }}\texttt{\ } keyword, which precedes an
expression and ensures that it is computed with knowledge of its
environment. In return, the context expression itself ends up opaque. We
cannot directly access whatever results from it in our code, precisely
because it is contextual: There is no one correct result, there may be
multiple results in different places of the document. For this reason,
everything that depends on the contextual data must happen inside of the
context expression.

Aside from explicit context expressions, context is also established
implicitly in some places that are also aware of their location in the
document: \href{/docs/reference/styling/\#show-rules}{Show rules}
provide context \textsuperscript{\hyperref[1]{1}} and numberings in the
outline, for instance, also provide the proper context to resolve
counters.

\subsection{Style context}\label{style-context}

With set rules, we can adjust style properties for parts or the whole of
our document. We cannot access these without a known context, as they
may change throughout the course of the document. When context is
available, we can retrieve them simply by accessing them as fields on
the respective element function.

\begin{verbatim}
#set text(lang: "de")
#context text.lang
\end{verbatim}

\includegraphics[width=5in,height=\textheight,keepaspectratio]{/assets/docs/ETetUaSK2J1pHpdRRUWzagAAAAAAAAAA.png}

As explained above, a context expression is reactive to the different
environments it is placed into. In the example below, we create a single
context expression, store it in the \texttt{\ value\ } variable and use
it multiple times. Each use properly reacts to the current surroundings.

\begin{verbatim}
#let value = context text.lang
#value

#set text(lang: "de")
#value

#set text(lang: "fr")
#value
\end{verbatim}

\includegraphics[width=5in,height=\textheight,keepaspectratio]{/assets/docs/cUJma0l-7W2Pm0tXEKJmjAAAAAAAAAAA.png}

Crucially, upon creation, \texttt{\ value\ } becomes opaque
\href{/docs/reference/foundations/content/}{content} that we cannot peek
into. It can only be resolved when placed somewhere because only then
the context is known. The body of a context expression may be evaluated
zero, one, or multiple times, depending on how many different places it
is put into.

\subsection{Location context}\label{location-context}

We\textquotesingle ve already seen that context gives us access to set
rule values. But it can do more: It also lets us know \emph{where} in
the document we currently are, relative to other elements, and
absolutely on the pages. We can use this information to create very
flexible interactions between different document parts. This underpins
features like heading numbering, the table of contents, or page headers
dependent on section headings.

Some functions like
\href{/docs/reference/introspection/counter/\#definitions-get}{\texttt{\ counter.get\ }}
implicitly access the current location. In the example below, we want to
retrieve the value of the heading counter. Since it changes throughout
the document, we need to first enter a context expression. Then, we use
\texttt{\ get\ } to retrieve the counter\textquotesingle s current
value. This function accesses the current location from the context to
resolve the counter value. Counters have multiple levels and
\texttt{\ get\ } returns an array with the resolved numbers. Thus, we
get the following result:

\begin{verbatim}
#set heading(numbering: "1.")

= Introduction
#lorem(5)

#context counter(heading).get()

= Background
#lorem(5)

#context counter(heading).get()
\end{verbatim}

\includegraphics[width=5in,height=\textheight,keepaspectratio]{/assets/docs/bQONUXVpXWNuuUEOrLszpQAAAAAAAAAA.png}

For more flexibility, we can also use the
\href{/docs/reference/introspection/here/}{\texttt{\ here\ }} function
to directly extract the current
\href{/docs/reference/introspection/location/}{location} from the
context. The example below demonstrates this:

\begin{itemize}
\tightlist
\item
  We first have
  \texttt{\ }{\texttt{\ counter\ }}\texttt{\ }{\texttt{\ (\ }}\texttt{\ heading\ }{\texttt{\ )\ }}\texttt{\ }{\texttt{\ .\ }}\texttt{\ }{\texttt{\ get\ }}\texttt{\ }{\texttt{\ (\ }}\texttt{\ }{\texttt{\ )\ }}\texttt{\ }
  , which resolves to
  \texttt{\ }{\texttt{\ (\ }}\texttt{\ }{\texttt{\ 2\ }}\texttt{\ }{\texttt{\ ,\ }}\texttt{\ }{\texttt{\ )\ }}\texttt{\ }
  as before.
\item
  We then use the more powerful
  \href{/docs/reference/introspection/counter/\#definitions-at}{\texttt{\ counter.at\ }}
  with \href{/docs/reference/introspection/here/}{\texttt{\ here\ }} ,
  which in combination is equivalent to \texttt{\ get\ } , and thus get
  \texttt{\ }{\texttt{\ (\ }}\texttt{\ }{\texttt{\ 2\ }}\texttt{\ }{\texttt{\ ,\ }}\texttt{\ }{\texttt{\ )\ }}\texttt{\ }
  .
\item
  Finally, we use \texttt{\ at\ } with a
  \href{/docs/reference/foundations/label/}{label} to retrieve the value
  of the counter at a \emph{different} location in the document, in our
  case that of the introduction heading. This yields
  \texttt{\ }{\texttt{\ (\ }}\texttt{\ }{\texttt{\ 1\ }}\texttt{\ }{\texttt{\ ,\ }}\texttt{\ }{\texttt{\ )\ }}\texttt{\ }
  . Typst\textquotesingle s context system gives us time travel
  abilities and lets us retrieve the values of any counters and states
  at \emph{any} location in the document.
\end{itemize}

\begin{verbatim}
#set heading(numbering: "1.")

= Introduction <intro>
#lorem(5)

= Background <back>
#lorem(5)

#context [
  #counter(heading).get() \
  #counter(heading).at(here()) \
  #counter(heading).at(<intro>)
]
\end{verbatim}

\includegraphics[width=5in,height=\textheight,keepaspectratio]{/assets/docs/gip9ugheiaYydjAEj2_jbgAAAAAAAAAA.png}

As mentioned before, we can also use context to get the physical
position of elements on the pages. We do this with the
\href{/docs/reference/introspection/locate/}{\texttt{\ locate\ }}
function, which works similarly to \texttt{\ counter.at\ } : It takes a
location or other \href{/docs/reference/foundations/selector/}{selector}
that resolves to a unique element (could also be a label) and returns
the position on the pages for that element.

\begin{verbatim}
Background is at: \
#context locate(<back>).position()

= Introduction <intro>
#lorem(5)
#pagebreak()

= Background <back>
#lorem(5)
\end{verbatim}

\includegraphics[width=5in,height=\textheight,keepaspectratio]{/assets/docs/AV1GaGSFxqcGN8RTlxty3gAAAAAAAAAA.png}
\includegraphics[width=5in,height=\textheight,keepaspectratio]{/assets/docs/AV1GaGSFxqcGN8RTlxty3gAAAAAAAAAB.png}

There are other functions that make use of the location context, most
prominently
\href{/docs/reference/introspection/query/}{\texttt{\ query\ }} . Take a
look at the \href{/docs/reference/introspection/}{introspection}
category for more details on those.

\subsection{Nested contexts}\label{nested-contexts}

Context is also accessible from within function calls nested in context
blocks. In the example below, \texttt{\ foo\ } itself becomes a
contextual function, just like
\href{/docs/reference/layout/length/\#definitions-to-absolute}{\texttt{\ to-absolute\ }}
is.

\begin{verbatim}
#let foo() = 1em.to-absolute()
#context {
  foo() == text.size
}
\end{verbatim}

\includegraphics[width=5in,height=\textheight,keepaspectratio]{/assets/docs/tBYLufutRlRl2ZJ_PAm-owAAAAAAAAAA.png}

Context blocks can be nested. Contextual code will then always access
the innermost context. The example below demonstrates this: The first
\texttt{\ text.lang\ } will access the outer context
block\textquotesingle s styles and as such, it will \textbf{not} see the
effect of
\texttt{\ }{\texttt{\ set\ }}\texttt{\ }{\texttt{\ text\ }}\texttt{\ }{\texttt{\ (\ }}\texttt{\ lang\ }{\texttt{\ :\ }}\texttt{\ }{\texttt{\ "fr"\ }}\texttt{\ }{\texttt{\ )\ }}\texttt{\ }
. The nested context block around the second \texttt{\ text.lang\ } ,
however, starts after the set rule and will thus show its effect.

\begin{verbatim}
#set text(lang: "de")
#context [
  #set text(lang: "fr")
  #text.lang \
  #context text.lang
]
\end{verbatim}

\includegraphics[width=5in,height=\textheight,keepaspectratio]{/assets/docs/-8ZHuN0AkDNg1gXmAO7X2wAAAAAAAAAA.png}

You might wonder why Typst ignores the French set rule when computing
the first \texttt{\ text.lang\ } in the example above. The reason is
that, in the general case, Typst cannot know all the styles that will
apply as set rules can be applied to content after it has been
constructed. Below, \texttt{\ text.lang\ } is already computed when the
template function is applied. As such, it cannot possibly be aware of
the language change to French in the template.

\begin{verbatim}
#let template(body) = {
  set text(lang: "fr")
  upper(body)
}

#set text(lang: "de")
#context [
  #show: template
  #text.lang \
  #context text.lang
]
\end{verbatim}

\includegraphics[width=5in,height=\textheight,keepaspectratio]{/assets/docs/ptMaFdqycQGV8lm06g29-gAAAAAAAAAA.png}

The second \texttt{\ text.lang\ } , however, \emph{does} react to the
language change because evaluation of its surrounding context block is
deferred until the styles for it are known. This illustrates the
importance of picking the right insertion point for a context to get
access to precisely the right styles.

The same also holds true for the location context. Below, the first
\texttt{\ c\ }{\texttt{\ .\ }}\texttt{\ }{\texttt{\ display\ }}\texttt{\ }{\texttt{\ (\ }}\texttt{\ }{\texttt{\ )\ }}\texttt{\ }
call will access the outer context block and will thus not see the
effect of
\texttt{\ c\ }{\texttt{\ .\ }}\texttt{\ }{\texttt{\ update\ }}\texttt{\ }{\texttt{\ (\ }}\texttt{\ }{\texttt{\ 2\ }}\texttt{\ }{\texttt{\ )\ }}\texttt{\ }
while the second
\texttt{\ c\ }{\texttt{\ .\ }}\texttt{\ }{\texttt{\ display\ }}\texttt{\ }{\texttt{\ (\ }}\texttt{\ }{\texttt{\ )\ }}\texttt{\ }
accesses the inner context and will thus see it.

\begin{verbatim}
#let c = counter("mycounter")
#c.update(1)
#context [
  #c.update(2)
  #c.display() \
  #context c.display()
]
\end{verbatim}

\includegraphics[width=5in,height=\textheight,keepaspectratio]{/assets/docs/6mlAfSm7646viO4S8ua6gwAAAAAAAAAA.png}

\subsection{Compiler iterations}\label{compiler-iterations}

To resolve contextual interactions, the Typst compiler processes your
document multiple times. For instance, to resolve a \texttt{\ locate\ }
call, Typst first provides a placeholder position, layouts your document
and then recompiles with the known position from the finished layout.
The same approach is taken to resolve counters, states, and queries. In
certain cases, Typst may even need more than two iterations to resolve
everything. While that\textquotesingle s sometimes a necessity, it may
also be a sign of misuse of contextual functions (e.g. of
\href{/docs/reference/introspection/state/\#caution}{state} ). If Typst
cannot resolve everything within five attempts, it will stop and output
the warning "layout did not converge within 5 attempts."

A very careful reader might have noticed that not all of the functions
presented above actually make use of the current location. While
\texttt{\ }{\texttt{\ counter\ }}\texttt{\ }{\texttt{\ (\ }}\texttt{\ heading\ }{\texttt{\ )\ }}\texttt{\ }{\texttt{\ .\ }}\texttt{\ }{\texttt{\ get\ }}\texttt{\ }{\texttt{\ (\ }}\texttt{\ }{\texttt{\ )\ }}\texttt{\ }
definitely depends on it,
\texttt{\ }{\texttt{\ counter\ }}\texttt{\ }{\texttt{\ (\ }}\texttt{\ heading\ }{\texttt{\ )\ }}\texttt{\ }{\texttt{\ .\ }}\texttt{\ }{\texttt{\ at\ }}\texttt{\ }{\texttt{\ (\ }}\texttt{\ }{\texttt{\ \textless{}intro\textgreater{}\ }}\texttt{\ }{\texttt{\ )\ }}\texttt{\ }
, for instance, does not. However, it still requires context. While its
value is always the same \emph{within} one compilation iteration, it may
change over the course of multiple compiler iterations. If one could
call it directly at the top level of a module, the whole module and its
exports could change over the course of multiple compiler iterations,
which would not be desirable.

\phantomsection\label{1}
\textsuperscript{1}

Currently, all show rules provide styling context, but only show rules
on \href{/docs/reference/introspection/location/\#locatable}{locatable}
elements provide a location context.

\href{/docs/reference/scripting/}{\pandocbounded{\includesvg[keepaspectratio]{/assets/icons/16-arrow-right.svg}}}

{ Scripting } { Previous page }

\href{/docs/reference/foundations/}{\pandocbounded{\includesvg[keepaspectratio]{/assets/icons/16-arrow-right.svg}}}

{ Foundations } { Next page }


\section{Docs LaTeX/typst.app/docs/reference/math.tex}
\title{typst.app/docs/reference/math}

\begin{itemize}
\tightlist
\item
  \href{/docs}{\includesvg[width=0.16667in,height=0.16667in]{/assets/icons/16-docs-dark.svg}}
\item
  \includesvg[width=0.16667in,height=0.16667in]{/assets/icons/16-arrow-right.svg}
\item
  \href{/docs/reference/}{Reference}
\item
  \includesvg[width=0.16667in,height=0.16667in]{/assets/icons/16-arrow-right.svg}
\item
  \href{/docs/reference/math/}{Math}
\end{itemize}

\section{Math}\label{summary}

Typst has special \href{/docs/reference/syntax/\#math}{syntax} and
library functions to typeset mathematical formulas. Math formulas can be
displayed inline with text or as separate blocks. They will be typeset
into their own block if they start and end with at least one space (e.g.
\texttt{\ }{\texttt{\ \$\ }}\texttt{\ x\ }{\texttt{\ \^{}\ }}\texttt{\ 2\ }{\texttt{\ \$\ }}\texttt{\ }
).

\subsection{Variables}\label{variables}

In math, single letters are always displayed as is. Multiple letters,
however, are interpreted as variables and functions. To display multiple
letters verbatim, you can place them into quotes and to access single
letter variables, you can use the
\href{/docs/reference/scripting/\#expressions}{hash syntax} .

\begin{verbatim}
$ A = pi r^2 $
$ "area" = pi dot "radius"^2 $
$ cal(A) :=
    { x in RR | x "is natural" } $
#let x = 5
$ #x < 17 $
\end{verbatim}

\includegraphics[width=5in,height=\textheight,keepaspectratio]{/assets/docs/hSTnanxnhN2cMLti2SpIlwAAAAAAAAAA.png}

\subsection{Symbols}\label{symbols}

Math mode makes a wide selection of
\href{/docs/reference/symbols/sym/}{symbols} like \texttt{\ pi\ } ,
\texttt{\ dot\ } , or \texttt{\ RR\ } available. Many mathematical
symbols are available in different variants. You can select between
different variants by applying
\href{/docs/reference/symbols/symbol/}{modifiers} to the symbol. Typst
further recognizes a number of shorthand sequences like
\texttt{\ =\textgreater{}\ } that approximate a symbol. When such a
shorthand exists, the symbol\textquotesingle s documentation lists it.

\begin{verbatim}
$ x < y => x gt.eq.not y $
\end{verbatim}

\includegraphics[width=5in,height=\textheight,keepaspectratio]{/assets/docs/3QjDlBq8e4sckxD76_cbbgAAAAAAAAAA.png}

\subsection{Line Breaks}\label{line-breaks}

Formulas can also contain line breaks. Each line can contain one or
multiple \emph{alignment points} ( \texttt{\ \&\ } ) which are then
aligned.

\begin{verbatim}
$ sum_(k=0)^n k
    &= 1 + ... + n \
    &= (n(n+1)) / 2 $
\end{verbatim}

\includegraphics[width=5in,height=\textheight,keepaspectratio]{/assets/docs/4Y4RfouYZm3Jgju-7W3SZAAAAAAAAAAA.png}

\subsection{Function calls}\label{function-calls}

Math mode supports special function calls without the hash prefix. In
these "math calls", the argument list works a little differently than in
code:

\begin{itemize}
\tightlist
\item
  Within them, Typst is still in "math mode". Thus, you can write math
  directly into them, but need to use hash syntax to pass code
  expressions (except for strings, which are available in the math
  syntax).
\item
  They support positional and named arguments, but don\textquotesingle t
  support trailing content blocks and argument spreading.
\item
  They provide additional syntax for 2-dimensional argument lists. The
  semicolon ( \texttt{\ ;\ } ) merges preceding arguments separated by
  commas into an array argument.
\end{itemize}

\begin{verbatim}
$ frac(a^2, 2) $
$ vec(1, 2, delim: "[") $
$ mat(1, 2; 3, 4) $
$ lim_x =
    op("lim", limits: #true)_x $
\end{verbatim}

\includegraphics[width=5in,height=\textheight,keepaspectratio]{/assets/docs/gWTBh8i7ZWskmajIpEpUWQAAAAAAAAAA.png}

To write a verbatim comma or semicolon in a math call, escape it with a
backslash. The colon on the other hand is only recognized in a special
way if directly preceded by an identifier, so to display it verbatim in
those cases, you can just insert a space before it.

Functions calls preceded by a hash are normal code function calls and
not affected by these rules.

\subsection{Alignment}\label{alignment}

When equations include multiple \emph{alignment points} (
\texttt{\ \&\ } ), this creates blocks of alternatingly right- and
left-aligned columns. In the example below, the expression
\texttt{\ (3x\ +\ y)\ /\ 7\ } is right-aligned and \texttt{\ =\ 9\ } is
left-aligned. The word "given" is also left-aligned because
\texttt{\ \&\&\ } creates two alignment points in a row, alternating the
alignment twice. \texttt{\ \&\ \&\ } and \texttt{\ \&\&\ } behave
exactly the same way. Meanwhile, "multiply by 7" is right-aligned
because just one \texttt{\ \&\ } precedes it. Each alignment point
simply alternates between right-aligned/left-aligned.

\begin{verbatim}
$ (3x + y) / 7 &= 9 && "given" \
  3x + y &= 63 & "multiply by 7" \
  3x &= 63 - y && "subtract y" \
  x &= 21 - y/3 & "divide by 3" $
\end{verbatim}

\includegraphics[width=5in,height=\textheight,keepaspectratio]{/assets/docs/8SM9qVyRZ_Elks_C9882dAAAAAAAAAAA.png}

\subsection{Math fonts}\label{math-fonts}

You can set the math font by with a
\href{/docs/reference/styling/\#show-rules}{show-set rule} as
demonstrated below. Note that only special OpenType math fonts are
suitable for typesetting maths.

\begin{verbatim}
#show math.equation: set text(font: "Fira Math")
$ sum_(i in NN) 1 + i $
\end{verbatim}

\includegraphics[width=5in,height=\textheight,keepaspectratio]{/assets/docs/qG9Xcf2X5Ju0E76URIxfZgAAAAAAAAAA.png}

\subsection{Math module}\label{math-module}

All math functions are part of the \texttt{\ math\ }
\href{/docs/reference/scripting/\#modules}{module} , which is available
by default in equations. Outside of equations, they can be accessed with
the \texttt{\ math.\ } prefix.

\subsection{Definitions}\label{definitions}

\begin{itemize}
\tightlist
\item
  \href{/docs/reference/math/accent/}{\texttt{\ accent\ }} { Attaches an
  accent to a base. }
\item
  \href{/docs/reference/math/attach}{attach} { Subscript, superscripts,
  and limits. }
\item
  \href{/docs/reference/math/binom/}{\texttt{\ binom\ }} { A binomial
  expression. }
\item
  \href{/docs/reference/math/cancel/}{\texttt{\ cancel\ }} { Displays a
  diagonal line over a part of an equation. }
\item
  \href{/docs/reference/math/cases/}{\texttt{\ cases\ }} { A case
  distinction. }
\item
  \href{/docs/reference/math/class/}{\texttt{\ class\ }} { Forced use of
  a certain math class. }
\item
  \href{/docs/reference/math/equation/}{\texttt{\ equation\ }} { A
  mathematical equation. }
\item
  \href{/docs/reference/math/frac/}{\texttt{\ frac\ }} { A mathematical
  fraction. }
\item
  \href{/docs/reference/math/lr}{lr} { Delimiter matching. }
\item
  \href{/docs/reference/math/mat/}{\texttt{\ mat\ }} { A matrix. }
\item
  \href{/docs/reference/math/op/}{\texttt{\ op\ }} { A text operator in
  an equation. }
\item
  \href{/docs/reference/math/primes/}{\texttt{\ primes\ }} { Grouped
  primes. }
\item
  \href{/docs/reference/math/roots}{roots} { Square and non-square
  roots. }
\item
  \href{/docs/reference/math/sizes}{sizes} { Forced size styles for
  expressions within formulas. }
\item
  \href{/docs/reference/math/stretch/}{\texttt{\ stretch\ }} { Stretches
  a glyph. }
\item
  \href{/docs/reference/math/styles}{styles} { Alternate letterforms
  within formulas. }
\item
  \href{/docs/reference/math/underover}{underover} { Delimiters above or
  below parts of an equation. }
\item
  \href{/docs/reference/math/variants}{variants} { Alternate typefaces
  within formulas. }
\item
  \href{/docs/reference/math/vec/}{\texttt{\ vec\ }} { A column vector.
  }
\end{itemize}

\href{/docs/reference/text/upper/}{\pandocbounded{\includesvg[keepaspectratio]{/assets/icons/16-arrow-right.svg}}}

{ Uppercase } { Previous page }

\href{/docs/reference/math/accent/}{\pandocbounded{\includesvg[keepaspectratio]{/assets/icons/16-arrow-right.svg}}}

{ Accent } { Next page }


\section{Docs LaTeX/typst.app/docs/reference/visualize.tex}
\title{typst.app/docs/reference/visualize}

\begin{itemize}
\tightlist
\item
  \href{/docs}{\includesvg[width=0.16667in,height=0.16667in]{/assets/icons/16-docs-dark.svg}}
\item
  \includesvg[width=0.16667in,height=0.16667in]{/assets/icons/16-arrow-right.svg}
\item
  \href{/docs/reference/}{Reference}
\item
  \includesvg[width=0.16667in,height=0.16667in]{/assets/icons/16-arrow-right.svg}
\item
  \href{/docs/reference/visualize/}{Visualize}
\end{itemize}

\section{Visualize}\label{summary}

Drawing and data visualization.

If you want to create more advanced drawings or plots, also have a look
at the \href{https://github.com/johannes-wolf/cetz}{CetZ} package as
well as more specialized \href{https://typst.app/universe/}{packages}
for your use case.

\subsection{Definitions}\label{definitions}

\begin{itemize}
\tightlist
\item
  \href{/docs/reference/visualize/circle/}{\texttt{\ circle\ }} { A
  circle with optional content. }
\item
  \href{/docs/reference/visualize/color/}{\texttt{\ color\ }} { A color
  in a specific color space. }
\item
  \href{/docs/reference/visualize/ellipse/}{\texttt{\ ellipse\ }} { An
  ellipse with optional content. }
\item
  \href{/docs/reference/visualize/gradient/}{\texttt{\ gradient\ }} { A
  color gradient. }
\item
  \href{/docs/reference/visualize/image/}{\texttt{\ image\ }} { A raster
  or vector graphic. }
\item
  \href{/docs/reference/visualize/line/}{\texttt{\ line\ }} { A line
  from one point to another. }
\item
  \href{/docs/reference/visualize/path/}{\texttt{\ path\ }} { A path
  through a list of points, connected by Bezier curves. }
\item
  \href{/docs/reference/visualize/pattern/}{\texttt{\ pattern\ }} { A
  repeating pattern fill. }
\item
  \href{/docs/reference/visualize/polygon/}{\texttt{\ polygon\ }} { A
  closed polygon. }
\item
  \href{/docs/reference/visualize/rect/}{\texttt{\ rect\ }} { A
  rectangle with optional content. }
\item
  \href{/docs/reference/visualize/square/}{\texttt{\ square\ }} { A
  square with optional content. }
\item
  \href{/docs/reference/visualize/stroke/}{\texttt{\ stroke\ }} {
  Defines how to draw a line. }
\end{itemize}

\href{/docs/reference/layout/stack/}{\pandocbounded{\includesvg[keepaspectratio]{/assets/icons/16-arrow-right.svg}}}

{ Stack } { Previous page }

\href{/docs/reference/visualize/circle/}{\pandocbounded{\includesvg[keepaspectratio]{/assets/icons/16-arrow-right.svg}}}

{ Circle } { Next page }




\section{C Docs LaTeX/docs/reference/model.tex}
\section{Docs LaTeX/typst.app/docs/reference/model/par.tex}
\title{typst.app/docs/reference/model/par}

\begin{itemize}
\tightlist
\item
  \href{/docs}{\includesvg[width=0.16667in,height=0.16667in]{/assets/icons/16-docs-dark.svg}}
\item
  \includesvg[width=0.16667in,height=0.16667in]{/assets/icons/16-arrow-right.svg}
\item
  \href{/docs/reference/}{Reference}
\item
  \includesvg[width=0.16667in,height=0.16667in]{/assets/icons/16-arrow-right.svg}
\item
  \href{/docs/reference/model/}{Model}
\item
  \includesvg[width=0.16667in,height=0.16667in]{/assets/icons/16-arrow-right.svg}
\item
  \href{/docs/reference/model/par/}{Paragraph}
\end{itemize}

\section{\texorpdfstring{\texttt{\ par\ } {{ Element
}}}{ par   Element }}\label{summary}

\phantomsection\label{element-tooltip}
Element functions can be customized with \texttt{\ set\ } and
\texttt{\ show\ } rules.

Arranges text, spacing and inline-level elements into a paragraph.

Although this function is primarily used in set rules to affect
paragraph properties, it can also be used to explicitly render its
argument onto a paragraph of its own.

\subsection{Example}\label{example}

\begin{verbatim}
#set par(
  first-line-indent: 1em,
  spacing: 0.65em,
  justify: true,
)

We proceed by contradiction.
Suppose that there exists a set
of positive integers $a$, $b$, and
$c$ that satisfies the equation
$a^n + b^n = c^n$ for some
integer value of $n > 2$.

Without loss of generality,
let $a$ be the smallest of the
three integers. Then, we ...
\end{verbatim}

\includegraphics[width=5in,height=\textheight,keepaspectratio]{/assets/docs/yrIipb0QYzuDEgQNZF57rwAAAAAAAAAA.png}

\subsection{\texorpdfstring{{ Parameters
}}{ Parameters }}\label{parameters}

\phantomsection\label{parameters-tooltip}
Parameters are the inputs to a function. They are specified in
parentheses after the function name.

{ par } (

{ \hyperref[parameters-leading]{leading :}
\href{/docs/reference/layout/length/}{length} , } {
\hyperref[parameters-spacing]{spacing :}
\href{/docs/reference/layout/length/}{length} , } {
\hyperref[parameters-justify]{justify :}
\href{/docs/reference/foundations/bool/}{bool} , } {
\hyperref[parameters-linebreaks]{linebreaks :}
\href{/docs/reference/foundations/auto/}{auto}
\href{/docs/reference/foundations/str/}{str} , } {
\hyperref[parameters-first-line-indent]{first-line-indent :}
\href{/docs/reference/layout/length/}{length} , } {
\hyperref[parameters-hanging-indent]{hanging-indent :}
\href{/docs/reference/layout/length/}{length} , } {
\href{/docs/reference/foundations/content/}{content} , }

) -\textgreater{} \href{/docs/reference/foundations/content/}{content}

\subsubsection{\texorpdfstring{\texttt{\ leading\ }}{ leading }}\label{parameters-leading}

\href{/docs/reference/layout/length/}{length}

{{ Settable }}

\phantomsection\label{parameters-leading-settable-tooltip}
Settable parameters can be customized for all following uses of the
function with a \texttt{\ set\ } rule.

The spacing between lines.

Leading defines the spacing between the
\href{/docs/reference/text/text/\#parameters-bottom-edge}{bottom edge}
of one line and the
\href{/docs/reference/text/text/\#parameters-top-edge}{top edge} of the
following line. By default, these two properties are up to the font, but
they can also be configured manually with a text set rule.

By setting top edge, bottom edge, and leading, you can also configure a
consistent baseline-to-baseline distance. You could, for instance, set
the leading to \texttt{\ }{\texttt{\ 1em\ }}\texttt{\ } , the top-edge
to \texttt{\ }{\texttt{\ 0.8em\ }}\texttt{\ } , and the bottom-edge to
\texttt{\ }{\texttt{\ -\ }}\texttt{\ }{\texttt{\ 0.2em\ }}\texttt{\ } to
get a baseline gap of exactly \texttt{\ }{\texttt{\ 2em\ }}\texttt{\ } .
The exact distribution of the top- and bottom-edge values affects the
bounds of the first and last line.

Default: \texttt{\ }{\texttt{\ 0.65em\ }}\texttt{\ }

\subsubsection{\texorpdfstring{\texttt{\ spacing\ }}{ spacing }}\label{parameters-spacing}

\href{/docs/reference/layout/length/}{length}

{{ Settable }}

\phantomsection\label{parameters-spacing-settable-tooltip}
Settable parameters can be customized for all following uses of the
function with a \texttt{\ set\ } rule.

The spacing between paragraphs.

Just like leading, this defines the spacing between the bottom edge of a
paragraph\textquotesingle s last line and the top edge of the next
paragraph\textquotesingle s first line.

When a paragraph is adjacent to a
\href{/docs/reference/layout/block/}{\texttt{\ block\ }} that is not a
paragraph, that block\textquotesingle s
\href{/docs/reference/layout/block/\#parameters-above}{\texttt{\ above\ }}
or
\href{/docs/reference/layout/block/\#parameters-below}{\texttt{\ below\ }}
property takes precedence over the paragraph spacing. Headings, for
instance, reduce the spacing below them by default for a better look.

Default: \texttt{\ }{\texttt{\ 1.2em\ }}\texttt{\ }

\subsubsection{\texorpdfstring{\texttt{\ justify\ }}{ justify }}\label{parameters-justify}

\href{/docs/reference/foundations/bool/}{bool}

{{ Settable }}

\phantomsection\label{parameters-justify-settable-tooltip}
Settable parameters can be customized for all following uses of the
function with a \texttt{\ set\ } rule.

Whether to justify text in its line.

Hyphenation will be enabled for justified paragraphs if the
\href{/docs/reference/text/text/\#parameters-hyphenate}{text
function\textquotesingle s \texttt{\ hyphenate\ } property} is set to
\texttt{\ }{\texttt{\ auto\ }}\texttt{\ } and the current language is
known.

Note that the current
\href{/docs/reference/layout/align/\#parameters-alignment}{alignment}
still has an effect on the placement of the last line except if it ends
with a
\href{/docs/reference/text/linebreak/\#parameters-justify}{justified
line break} .

Default: \texttt{\ }{\texttt{\ false\ }}\texttt{\ }

\subsubsection{\texorpdfstring{\texttt{\ linebreaks\ }}{ linebreaks }}\label{parameters-linebreaks}

\href{/docs/reference/foundations/auto/}{auto} {or}
\href{/docs/reference/foundations/str/}{str}

{{ Settable }}

\phantomsection\label{parameters-linebreaks-settable-tooltip}
Settable parameters can be customized for all following uses of the
function with a \texttt{\ set\ } rule.

How to determine line breaks.

When this property is set to \texttt{\ }{\texttt{\ auto\ }}\texttt{\ } ,
its default value, optimized line breaks will be used for justified
paragraphs. Enabling optimized line breaks for ragged paragraphs may
also be worthwhile to improve the appearance of the text.

\begin{longtable}[]{@{}
  >{\raggedright\arraybackslash}p{(\linewidth - 2\tabcolsep) * \real{0.5000}}
  >{\raggedright\arraybackslash}p{(\linewidth - 2\tabcolsep) * \real{0.5000}}@{}}
\toprule\noalign{}
\begin{minipage}[b]{\linewidth}\raggedright
Variant
\end{minipage} & \begin{minipage}[b]{\linewidth}\raggedright
Details
\end{minipage} \\
\midrule\noalign{}
\endhead
\bottomrule\noalign{}
\endlastfoot
\texttt{\ "\ simple\ "\ } & Determine the line breaks in a simple
first-fit style. \\
\texttt{\ "\ optimized\ "\ } & Optimize the line breaks for the whole
paragraph.

Typst will try to produce more evenly filled lines of text by
considering the whole paragraph when calculating line breaks. \\
\end{longtable}

Default: \texttt{\ }{\texttt{\ auto\ }}\texttt{\ }

\includesvg[width=0.16667in,height=0.16667in]{/assets/icons/16-arrow-right.svg}
View example

\begin{verbatim}
#set page(width: 207pt)
#set par(linebreaks: "simple")
Some texts feature many longer
words. Those are often exceedingly
challenging to break in a visually
pleasing way.

#set par(linebreaks: "optimized")
Some texts feature many longer
words. Those are often exceedingly
challenging to break in a visually
pleasing way.
\end{verbatim}

\includegraphics[width=4.3125in,height=\textheight,keepaspectratio]{/assets/docs/r-fawkmmJ6Sniwi8--ib5gAAAAAAAAAA.png}

\subsubsection{\texorpdfstring{\texttt{\ first-line-indent\ }}{ first-line-indent }}\label{parameters-first-line-indent}

\href{/docs/reference/layout/length/}{length}

{{ Settable }}

\phantomsection\label{parameters-first-line-indent-settable-tooltip}
Settable parameters can be customized for all following uses of the
function with a \texttt{\ set\ } rule.

The indent the first line of a paragraph should have.

Only the first line of a consecutive paragraph will be indented (not the
first one in a block or on the page).

By typographic convention, paragraph breaks are indicated either by some
space between paragraphs or by indented first lines. Consider reducing
the \href{/docs/reference/layout/block/\#parameters-spacing}{paragraph
spacing} to the
\href{/docs/reference/model/par/\#parameters-leading}{\texttt{\ leading\ }}
when using this property (e.g. using
\texttt{\ }{\texttt{\ \#\ }}\texttt{\ }{\texttt{\ set\ }}\texttt{\ }{\texttt{\ par\ }}\texttt{\ }{\texttt{\ (\ }}\texttt{\ spacing\ }{\texttt{\ :\ }}\texttt{\ }{\texttt{\ 0.65em\ }}\texttt{\ }{\texttt{\ )\ }}\texttt{\ }
).

Default: \texttt{\ }{\texttt{\ 0pt\ }}\texttt{\ }

\subsubsection{\texorpdfstring{\texttt{\ hanging-indent\ }}{ hanging-indent }}\label{parameters-hanging-indent}

\href{/docs/reference/layout/length/}{length}

{{ Settable }}

\phantomsection\label{parameters-hanging-indent-settable-tooltip}
Settable parameters can be customized for all following uses of the
function with a \texttt{\ set\ } rule.

The indent all but the first line of a paragraph should have.

Default: \texttt{\ }{\texttt{\ 0pt\ }}\texttt{\ }

\subsubsection{\texorpdfstring{\texttt{\ body\ }}{ body }}\label{parameters-body}

\href{/docs/reference/foundations/content/}{content}

{Required} {{ Positional }}

\phantomsection\label{parameters-body-positional-tooltip}
Positional parameters are specified in order, without names.

The contents of the paragraph.

\subsection{\texorpdfstring{{ Definitions
}}{ Definitions }}\label{definitions}

\phantomsection\label{definitions-tooltip}
Functions and types and can have associated definitions. These are
accessed by specifying the function or type, followed by a period, and
then the definition\textquotesingle s name.

\subsubsection{\texorpdfstring{\texttt{\ line\ } {{ Element
}}}{ line   Element }}\label{definitions-line}

\phantomsection\label{definitions-line-element-tooltip}
Element functions can be customized with \texttt{\ set\ } and
\texttt{\ show\ } rules.

A paragraph line.

This element is exclusively used for line number configuration through
set rules and cannot be placed.

The
\href{/docs/reference/model/par/\#definitions-line-numbering}{\texttt{\ numbering\ }}
option is used to enable line numbers by specifying a numbering format.

par { . } { line } (

{ \hyperref[definitions-line-parameters-numbering]{numbering :}
\href{/docs/reference/foundations/none/}{none}
\href{/docs/reference/foundations/str/}{str}
\href{/docs/reference/foundations/function/}{function} , } {
\hyperref[definitions-line-parameters-number-align]{number-align :}
\href{/docs/reference/foundations/auto/}{auto}
\href{/docs/reference/layout/alignment/}{alignment} , } {
\hyperref[definitions-line-parameters-number-margin]{number-margin :}
\href{/docs/reference/layout/alignment/}{alignment} , } {
\hyperref[definitions-line-parameters-number-clearance]{number-clearance
:} \href{/docs/reference/foundations/auto/}{auto}
\href{/docs/reference/layout/length/}{length} , } {
\hyperref[definitions-line-parameters-numbering-scope]{numbering-scope
:} \href{/docs/reference/foundations/str/}{str} , }

) -\textgreater{} \href{/docs/reference/foundations/content/}{content}

\begin{verbatim}
#set par.line(numbering: "1")

Roses are red. \
Violets are blue. \
Typst is there for you.
\end{verbatim}

\includegraphics[width=5in,height=\textheight,keepaspectratio]{/assets/docs/b257YLHUEagbFWlPeD4gEwAAAAAAAAAA.png}

The \texttt{\ numbering\ } option takes either a predefined
\href{/docs/reference/model/numbering/}{numbering pattern} or a function
returning styled content. You can disable line numbers for text inside
certain elements by setting the numbering to
\texttt{\ }{\texttt{\ none\ }}\texttt{\ } using show-set rules.

\begin{verbatim}
// Styled red line numbers.
#set par.line(
  numbering: n => text(red)[#n]
)

// Disable numbers inside figures.
#show figure: set par.line(
  numbering: none
)

Roses are red. \
Violets are blue.

#figure(
  caption: [Without line numbers.]
)[
  Lorem ipsum \
  dolor sit amet
]

The text above is a sample \
originating from distant times.
\end{verbatim}

\includegraphics[width=5in,height=\textheight,keepaspectratio]{/assets/docs/WJNwFvR3ObvODT-MbWqflAAAAAAAAAAA.png}

This element exposes further options which may be used to control other
aspects of line numbering, such as its
\href{/docs/reference/model/par/\#definitions-line-number-align}{alignment}
or
\href{/docs/reference/model/par/\#definitions-line-number-margin}{margin}
. In addition, you can control whether the numbering is reset on each
page through the
\href{/docs/reference/model/par/\#definitions-line-numbering-scope}{\texttt{\ numbering-scope\ }}
option.

\paragraph{\texorpdfstring{\texttt{\ numbering\ }}{ numbering }}\label{definitions-line-numbering}

\href{/docs/reference/foundations/none/}{none} {or}
\href{/docs/reference/foundations/str/}{str} {or}
\href{/docs/reference/foundations/function/}{function}

{{ Settable }}

\phantomsection\label{definitions-line-numbering-settable-tooltip}
Settable parameters can be customized for all following uses of the
function with a \texttt{\ set\ } rule.

How to number each line. Accepts a
\href{/docs/reference/model/numbering/}{numbering pattern or function} .

Default: \texttt{\ }{\texttt{\ none\ }}\texttt{\ }

\includesvg[width=0.16667in,height=0.16667in]{/assets/icons/16-arrow-right.svg}
View example

\begin{verbatim}
#set par.line(numbering: "I")

Roses are red. \
Violets are blue. \
Typst is there for you.
\end{verbatim}

\includegraphics[width=5in,height=\textheight,keepaspectratio]{/assets/docs/O-oJqYc-OwEoxappxK4AZAAAAAAAAAAA.png}

\paragraph{\texorpdfstring{\texttt{\ number-align\ }}{ number-align }}\label{definitions-line-number-align}

\href{/docs/reference/foundations/auto/}{auto} {or}
\href{/docs/reference/layout/alignment/}{alignment}

{{ Settable }}

\phantomsection\label{definitions-line-number-align-settable-tooltip}
Settable parameters can be customized for all following uses of the
function with a \texttt{\ set\ } rule.

The alignment of line numbers associated with each line.

The default of \texttt{\ }{\texttt{\ auto\ }}\texttt{\ } indicates a
smart default where numbers grow horizontally away from the text,
considering the margin they\textquotesingle re in and the current text
direction.

Default: \texttt{\ }{\texttt{\ auto\ }}\texttt{\ }

\includesvg[width=0.16667in,height=0.16667in]{/assets/icons/16-arrow-right.svg}
View example

\begin{verbatim}
#set par.line(
  numbering: "I",
  number-align: left,
)

Hello world! \
Today is a beautiful day \
For exploring the world.
\end{verbatim}

\includegraphics[width=5in,height=\textheight,keepaspectratio]{/assets/docs/XfwBMgYjt2fGeRgFr_kj4AAAAAAAAAAA.png}

\paragraph{\texorpdfstring{\texttt{\ number-margin\ }}{ number-margin }}\label{definitions-line-number-margin}

\href{/docs/reference/layout/alignment/}{alignment}

{{ Settable }}

\phantomsection\label{definitions-line-number-margin-settable-tooltip}
Settable parameters can be customized for all following uses of the
function with a \texttt{\ set\ } rule.

The margin at which line numbers appear.

\emph{Note:} In a multi-column document, the line numbers for paragraphs
inside the last column will always appear on the \texttt{\ end\ } margin
(right margin for left-to-right text and left margin for right-to-left),
regardless of this configuration. That behavior cannot be changed at
this moment.

Default: \texttt{\ start\ }

\includesvg[width=0.16667in,height=0.16667in]{/assets/icons/16-arrow-right.svg}
View example

\begin{verbatim}
#set par.line(
  numbering: "1",
  number-margin: right,
)

= Report
- Brightness: Dark, yet darker
- Readings: Negative
\end{verbatim}

\includegraphics[width=5in,height=\textheight,keepaspectratio]{/assets/docs/vf0ZBrlygVUABySMskTaKQAAAAAAAAAA.png}

\paragraph{\texorpdfstring{\texttt{\ number-clearance\ }}{ number-clearance }}\label{definitions-line-number-clearance}

\href{/docs/reference/foundations/auto/}{auto} {or}
\href{/docs/reference/layout/length/}{length}

{{ Settable }}

\phantomsection\label{definitions-line-number-clearance-settable-tooltip}
Settable parameters can be customized for all following uses of the
function with a \texttt{\ set\ } rule.

The distance between line numbers and text.

The default value of \texttt{\ }{\texttt{\ auto\ }}\texttt{\ } results
in a clearance that is adaptive to the page width and yields reasonable
results in most cases.

Default: \texttt{\ }{\texttt{\ auto\ }}\texttt{\ }

\includesvg[width=0.16667in,height=0.16667in]{/assets/icons/16-arrow-right.svg}
View example

\begin{verbatim}
#set par.line(
  numbering: "1",
  number-clearance: 4pt,
)

Typesetting \
Styling \
Layout
\end{verbatim}

\includegraphics[width=5in,height=\textheight,keepaspectratio]{/assets/docs/MgiUB3LoxE0JROWoHJPslgAAAAAAAAAA.png}

\paragraph{\texorpdfstring{\texttt{\ numbering-scope\ }}{ numbering-scope }}\label{definitions-line-numbering-scope}

\href{/docs/reference/foundations/str/}{str}

{{ Settable }}

\phantomsection\label{definitions-line-numbering-scope-settable-tooltip}
Settable parameters can be customized for all following uses of the
function with a \texttt{\ set\ } rule.

Controls when to reset line numbering.

\emph{Note:} The line numbering scope must be uniform across each page
run (a page run is a sequence of pages without an explicit pagebreak in
between). For this reason, set rules for it should be defined before any
page content, typically at the very start of the document.

\begin{longtable}[]{@{}ll@{}}
\toprule\noalign{}
Variant & Details \\
\midrule\noalign{}
\endhead
\bottomrule\noalign{}
\endlastfoot
\texttt{\ "\ document\ "\ } & Indicates the line number counter spans
the whole document, that is, is never automatically reset. \\
\texttt{\ "\ page\ "\ } & Indicates the line number counter should be
reset at the start of every new page. \\
\end{longtable}

Default: \texttt{\ }{\texttt{\ "document"\ }}\texttt{\ }

\includesvg[width=0.16667in,height=0.16667in]{/assets/icons/16-arrow-right.svg}
View example

\begin{verbatim}
#set par.line(
  numbering: "1",
  numbering-scope: "page",
)

First line \
Second line
#pagebreak()
First line again \
Second line again
\end{verbatim}

\includegraphics[width=5in,height=\textheight,keepaspectratio]{/assets/docs/MmmIOu-UB2sC4GlOg3oj9AAAAAAAAAAA.png}
\includegraphics[width=5in,height=\textheight,keepaspectratio]{/assets/docs/MmmIOu-UB2sC4GlOg3oj9AAAAAAAAAAB.png}

\href{/docs/reference/model/outline/}{\pandocbounded{\includesvg[keepaspectratio]{/assets/icons/16-arrow-right.svg}}}

{ Outline } { Previous page }

\href{/docs/reference/model/parbreak/}{\pandocbounded{\includesvg[keepaspectratio]{/assets/icons/16-arrow-right.svg}}}

{ Paragraph Break } { Next page }


\section{Docs LaTeX/typst.app/docs/reference/model/list.tex}
\title{typst.app/docs/reference/model/list}

\begin{itemize}
\tightlist
\item
  \href{/docs}{\includesvg[width=0.16667in,height=0.16667in]{/assets/icons/16-docs-dark.svg}}
\item
  \includesvg[width=0.16667in,height=0.16667in]{/assets/icons/16-arrow-right.svg}
\item
  \href{/docs/reference/}{Reference}
\item
  \includesvg[width=0.16667in,height=0.16667in]{/assets/icons/16-arrow-right.svg}
\item
  \href{/docs/reference/model/}{Model}
\item
  \includesvg[width=0.16667in,height=0.16667in]{/assets/icons/16-arrow-right.svg}
\item
  \href{/docs/reference/model/list/}{Bullet List}
\end{itemize}

\section{\texorpdfstring{\texttt{\ list\ } {{ Element
}}}{ list   Element }}\label{summary}

\phantomsection\label{element-tooltip}
Element functions can be customized with \texttt{\ set\ } and
\texttt{\ show\ } rules.

A bullet list.

Displays a sequence of items vertically, with each item introduced by a
marker.

\subsection{Example}\label{example}

\begin{verbatim}
Normal list.
- Text
- Math
- Layout
- ...

Multiple lines.
- This list item spans multiple
  lines because it is indented.

Function call.
#list(
  [Foundations],
  [Calculate],
  [Construct],
  [Data Loading],
)
\end{verbatim}

\includegraphics[width=5in,height=\textheight,keepaspectratio]{/assets/docs/dGd96M9aTTHo-jKJ9y73kwAAAAAAAAAA.png}

\subsection{Syntax}\label{syntax}

This functions also has dedicated syntax: Start a line with a hyphen,
followed by a space to create a list item. A list item can contain
multiple paragraphs and other block-level content. All content that is
indented more than an item\textquotesingle s marker becomes part of that
item.

\subsection{\texorpdfstring{{ Parameters
}}{ Parameters }}\label{parameters}

\phantomsection\label{parameters-tooltip}
Parameters are the inputs to a function. They are specified in
parentheses after the function name.

{ list } (

{ \hyperref[parameters-tight]{tight :}
\href{/docs/reference/foundations/bool/}{bool} , } {
\hyperref[parameters-marker]{marker :}
\href{/docs/reference/foundations/content/}{content}
\href{/docs/reference/foundations/array/}{array}
\href{/docs/reference/foundations/function/}{function} , } {
\hyperref[parameters-indent]{indent :}
\href{/docs/reference/layout/length/}{length} , } {
\hyperref[parameters-body-indent]{body-indent :}
\href{/docs/reference/layout/length/}{length} , } {
\hyperref[parameters-spacing]{spacing :}
\href{/docs/reference/foundations/auto/}{auto}
\href{/docs/reference/layout/length/}{length} , } {
\hyperref[parameters-children]{..}
\href{/docs/reference/foundations/content/}{content} , }

) -\textgreater{} \href{/docs/reference/foundations/content/}{content}

\subsubsection{\texorpdfstring{\texttt{\ tight\ }}{ tight }}\label{parameters-tight}

\href{/docs/reference/foundations/bool/}{bool}

{{ Settable }}

\phantomsection\label{parameters-tight-settable-tooltip}
Settable parameters can be customized for all following uses of the
function with a \texttt{\ set\ } rule.

Defines the default
\href{/docs/reference/model/list/\#parameters-spacing}{spacing} of the
list. If it is \texttt{\ }{\texttt{\ false\ }}\texttt{\ } , the items
are spaced apart with
\href{/docs/reference/model/par/\#parameters-spacing}{paragraph spacing}
. If it is \texttt{\ }{\texttt{\ true\ }}\texttt{\ } , they use
\href{/docs/reference/model/par/\#parameters-leading}{paragraph leading}
instead. This makes the list more compact, which can look better if the
items are short.

In markup mode, the value of this parameter is determined based on
whether items are separated with a blank line. If items directly follow
each other, this is set to \texttt{\ }{\texttt{\ true\ }}\texttt{\ } ;
if items are separated by a blank line, this is set to
\texttt{\ }{\texttt{\ false\ }}\texttt{\ } . The markup-defined
tightness cannot be overridden with set rules.

Default: \texttt{\ }{\texttt{\ true\ }}\texttt{\ }

\includesvg[width=0.16667in,height=0.16667in]{/assets/icons/16-arrow-right.svg}
View example

\begin{verbatim}
- If a list has a lot of text, and
  maybe other inline content, it
  should not be tight anymore.

- To make a list wide, simply insert
  a blank line between the items.
\end{verbatim}

\includegraphics[width=5in,height=\textheight,keepaspectratio]{/assets/docs/4FUPGE5Zxz4-Z1S-m_IFCQAAAAAAAAAA.png}

\subsubsection{\texorpdfstring{\texttt{\ marker\ }}{ marker }}\label{parameters-marker}

\href{/docs/reference/foundations/content/}{content} {or}
\href{/docs/reference/foundations/array/}{array} {or}
\href{/docs/reference/foundations/function/}{function}

{{ Settable }}

\phantomsection\label{parameters-marker-settable-tooltip}
Settable parameters can be customized for all following uses of the
function with a \texttt{\ set\ } rule.

The marker which introduces each item.

Instead of plain content, you can also pass an array with multiple
markers that should be used for nested lists. If the list nesting depth
exceeds the number of markers, the markers are cycled. For total
control, you may pass a function that maps the list\textquotesingle s
nesting depth (starting from \texttt{\ }{\texttt{\ 0\ }}\texttt{\ } ) to
a desired marker.

Default:
\texttt{\ }{\texttt{\ (\ }}\texttt{\ }{\texttt{\ {[}\ }}\texttt{\ •\ }{\texttt{\ {]}\ }}\texttt{\ }{\texttt{\ ,\ }}\texttt{\ }{\texttt{\ {[}\ }}\texttt{\ ‣\ }{\texttt{\ {]}\ }}\texttt{\ }{\texttt{\ ,\ }}\texttt{\ }{\texttt{\ {[}\ }}\texttt{\ –\ }{\texttt{\ {]}\ }}\texttt{\ }{\texttt{\ )\ }}\texttt{\ }

\includesvg[width=0.16667in,height=0.16667in]{/assets/icons/16-arrow-right.svg}
View example

\begin{verbatim}
#set list(marker: [--])
- A more classic list
- With en-dashes

#set list(marker: ([•], [--]))
- Top-level
  - Nested
  - Items
- Items
\end{verbatim}

\includegraphics[width=5in,height=\textheight,keepaspectratio]{/assets/docs/rGFZOVIfGIEORB3iENBotQAAAAAAAAAA.png}

\subsubsection{\texorpdfstring{\texttt{\ indent\ }}{ indent }}\label{parameters-indent}

\href{/docs/reference/layout/length/}{length}

{{ Settable }}

\phantomsection\label{parameters-indent-settable-tooltip}
Settable parameters can be customized for all following uses of the
function with a \texttt{\ set\ } rule.

The indent of each item.

Default: \texttt{\ }{\texttt{\ 0pt\ }}\texttt{\ }

\subsubsection{\texorpdfstring{\texttt{\ body-indent\ }}{ body-indent }}\label{parameters-body-indent}

\href{/docs/reference/layout/length/}{length}

{{ Settable }}

\phantomsection\label{parameters-body-indent-settable-tooltip}
Settable parameters can be customized for all following uses of the
function with a \texttt{\ set\ } rule.

The spacing between the marker and the body of each item.

Default: \texttt{\ }{\texttt{\ 0.5em\ }}\texttt{\ }

\subsubsection{\texorpdfstring{\texttt{\ spacing\ }}{ spacing }}\label{parameters-spacing}

\href{/docs/reference/foundations/auto/}{auto} {or}
\href{/docs/reference/layout/length/}{length}

{{ Settable }}

\phantomsection\label{parameters-spacing-settable-tooltip}
Settable parameters can be customized for all following uses of the
function with a \texttt{\ set\ } rule.

The spacing between the items of the list.

If set to \texttt{\ }{\texttt{\ auto\ }}\texttt{\ } , uses paragraph
\href{/docs/reference/model/par/\#parameters-leading}{\texttt{\ leading\ }}
for tight lists and paragraph
\href{/docs/reference/model/par/\#parameters-spacing}{\texttt{\ spacing\ }}
for wide (non-tight) lists.

Default: \texttt{\ }{\texttt{\ auto\ }}\texttt{\ }

\subsubsection{\texorpdfstring{\texttt{\ children\ }}{ children }}\label{parameters-children}

\href{/docs/reference/foundations/content/}{content}

{Required} {{ Positional }}

\phantomsection\label{parameters-children-positional-tooltip}
Positional parameters are specified in order, without names.

{{ Variadic }}

\phantomsection\label{parameters-children-variadic-tooltip}
Variadic parameters can be specified multiple times.

The bullet list\textquotesingle s children.

When using the list syntax, adjacent items are automatically collected
into lists, even through constructs like for loops.

\includesvg[width=0.16667in,height=0.16667in]{/assets/icons/16-arrow-right.svg}
View example

\begin{verbatim}
#for letter in "ABC" [
  - Letter #letter
]
\end{verbatim}

\includegraphics[width=5in,height=\textheight,keepaspectratio]{/assets/docs/scttBXkLjYOvlJchbuo00wAAAAAAAAAA.png}

\subsection{\texorpdfstring{{ Definitions
}}{ Definitions }}\label{definitions}

\phantomsection\label{definitions-tooltip}
Functions and types and can have associated definitions. These are
accessed by specifying the function or type, followed by a period, and
then the definition\textquotesingle s name.

\subsubsection{\texorpdfstring{\texttt{\ item\ } {{ Element
}}}{ item   Element }}\label{definitions-item}

\phantomsection\label{definitions-item-element-tooltip}
Element functions can be customized with \texttt{\ set\ } and
\texttt{\ show\ } rules.

A bullet list item.

list { . } { item } (

{ \href{/docs/reference/foundations/content/}{content} }

) -\textgreater{} \href{/docs/reference/foundations/content/}{content}

\paragraph{\texorpdfstring{\texttt{\ body\ }}{ body }}\label{definitions-item-body}

\href{/docs/reference/foundations/content/}{content}

{Required} {{ Positional }}

\phantomsection\label{definitions-item-body-positional-tooltip}
Positional parameters are specified in order, without names.

The item\textquotesingle s body.

\href{/docs/reference/model/bibliography/}{\pandocbounded{\includesvg[keepaspectratio]{/assets/icons/16-arrow-right.svg}}}

{ Bibliography } { Previous page }

\href{/docs/reference/model/cite/}{\pandocbounded{\includesvg[keepaspectratio]{/assets/icons/16-arrow-right.svg}}}

{ Cite } { Next page }


\section{Docs LaTeX/typst.app/docs/reference/model/enum.tex}
\title{typst.app/docs/reference/model/enum}

\begin{itemize}
\tightlist
\item
  \href{/docs}{\includesvg[width=0.16667in,height=0.16667in]{/assets/icons/16-docs-dark.svg}}
\item
  \includesvg[width=0.16667in,height=0.16667in]{/assets/icons/16-arrow-right.svg}
\item
  \href{/docs/reference/}{Reference}
\item
  \includesvg[width=0.16667in,height=0.16667in]{/assets/icons/16-arrow-right.svg}
\item
  \href{/docs/reference/model/}{Model}
\item
  \includesvg[width=0.16667in,height=0.16667in]{/assets/icons/16-arrow-right.svg}
\item
  \href{/docs/reference/model/enum/}{Numbered List}
\end{itemize}

\section{\texorpdfstring{\texttt{\ enum\ } {{ Element
}}}{ enum   Element }}\label{summary}

\phantomsection\label{element-tooltip}
Element functions can be customized with \texttt{\ set\ } and
\texttt{\ show\ } rules.

A numbered list.

Displays a sequence of items vertically and numbers them consecutively.

\subsection{Example}\label{example}

\begin{verbatim}
Automatically numbered:
+ Preparations
+ Analysis
+ Conclusions

Manually numbered:
2. What is the first step?
5. I am confused.
+  Moving on ...

Multiple lines:
+ This enum item has multiple
  lines because the next line
  is indented.

Function call.
#enum[First][Second]
\end{verbatim}

\includegraphics[width=5in,height=\textheight,keepaspectratio]{/assets/docs/HrnJ1mRKvbXNf6U4DmZCaAAAAAAAAAAA.png}

You can easily switch all your enumerations to a different numbering
style with a set rule.

\begin{verbatim}
#set enum(numbering: "a)")

+ Starting off ...
+ Don't forget step two
\end{verbatim}

\includegraphics[width=5in,height=\textheight,keepaspectratio]{/assets/docs/hFb68a8DC-Rvf_eMOYtVMwAAAAAAAAAA.png}

You can also use
\href{/docs/reference/model/enum/\#definitions-item}{\texttt{\ enum.item\ }}
to programmatically customize the number of each item in the
enumeration:

\begin{verbatim}
#enum(
  enum.item(1)[First step],
  enum.item(5)[Fifth step],
  enum.item(10)[Tenth step]
)
\end{verbatim}

\includegraphics[width=5in,height=\textheight,keepaspectratio]{/assets/docs/uRQbjXrkv7FwltBxluVMMAAAAAAAAAAA.png}

\subsection{Syntax}\label{syntax}

This functions also has dedicated syntax:

\begin{itemize}
\tightlist
\item
  Starting a line with a plus sign creates an automatically numbered
  enumeration item.
\item
  Starting a line with a number followed by a dot creates an explicitly
  numbered enumeration item.
\end{itemize}

Enumeration items can contain multiple paragraphs and other block-level
content. All content that is indented more than an
item\textquotesingle s marker becomes part of that item.

\subsection{\texorpdfstring{{ Parameters
}}{ Parameters }}\label{parameters}

\phantomsection\label{parameters-tooltip}
Parameters are the inputs to a function. They are specified in
parentheses after the function name.

{ enum } (

{ \hyperref[parameters-tight]{tight :}
\href{/docs/reference/foundations/bool/}{bool} , } {
\hyperref[parameters-numbering]{numbering :}
\href{/docs/reference/foundations/str/}{str}
\href{/docs/reference/foundations/function/}{function} , } {
\hyperref[parameters-start]{start :}
\href{/docs/reference/foundations/int/}{int} , } {
\hyperref[parameters-full]{full :}
\href{/docs/reference/foundations/bool/}{bool} , } {
\hyperref[parameters-indent]{indent :}
\href{/docs/reference/layout/length/}{length} , } {
\hyperref[parameters-body-indent]{body-indent :}
\href{/docs/reference/layout/length/}{length} , } {
\hyperref[parameters-spacing]{spacing :}
\href{/docs/reference/foundations/auto/}{auto}
\href{/docs/reference/layout/length/}{length} , } {
\hyperref[parameters-number-align]{number-align :}
\href{/docs/reference/layout/alignment/}{alignment} , } {
\hyperref[parameters-children]{..}
\href{/docs/reference/foundations/content/}{content}
\href{/docs/reference/foundations/array/}{array} , }

) -\textgreater{} \href{/docs/reference/foundations/content/}{content}

\subsubsection{\texorpdfstring{\texttt{\ tight\ }}{ tight }}\label{parameters-tight}

\href{/docs/reference/foundations/bool/}{bool}

{{ Settable }}

\phantomsection\label{parameters-tight-settable-tooltip}
Settable parameters can be customized for all following uses of the
function with a \texttt{\ set\ } rule.

Defines the default
\href{/docs/reference/model/enum/\#parameters-spacing}{spacing} of the
enumeration. If it is \texttt{\ }{\texttt{\ false\ }}\texttt{\ } , the
items are spaced apart with
\href{/docs/reference/model/par/\#parameters-spacing}{paragraph spacing}
. If it is \texttt{\ }{\texttt{\ true\ }}\texttt{\ } , they use
\href{/docs/reference/model/par/\#parameters-leading}{paragraph leading}
instead. This makes the list more compact, which can look better if the
items are short.

In markup mode, the value of this parameter is determined based on
whether items are separated with a blank line. If items directly follow
each other, this is set to \texttt{\ }{\texttt{\ true\ }}\texttt{\ } ;
if items are separated by a blank line, this is set to
\texttt{\ }{\texttt{\ false\ }}\texttt{\ } . The markup-defined
tightness cannot be overridden with set rules.

Default: \texttt{\ }{\texttt{\ true\ }}\texttt{\ }

\includesvg[width=0.16667in,height=0.16667in]{/assets/icons/16-arrow-right.svg}
View example

\begin{verbatim}
+ If an enum has a lot of text, and
  maybe other inline content, it
  should not be tight anymore.

+ To make an enum wide, simply
  insert a blank line between the
  items.
\end{verbatim}

\includegraphics[width=5in,height=\textheight,keepaspectratio]{/assets/docs/CGCi1WYCPLux25Xc9ZWwDQAAAAAAAAAA.png}

\subsubsection{\texorpdfstring{\texttt{\ numbering\ }}{ numbering }}\label{parameters-numbering}

\href{/docs/reference/foundations/str/}{str} {or}
\href{/docs/reference/foundations/function/}{function}

{{ Settable }}

\phantomsection\label{parameters-numbering-settable-tooltip}
Settable parameters can be customized for all following uses of the
function with a \texttt{\ set\ } rule.

How to number the enumeration. Accepts a
\href{/docs/reference/model/numbering/}{numbering pattern or function} .

If the numbering pattern contains multiple counting symbols, they apply
to nested enums. If given a function, the function receives one argument
if \texttt{\ full\ } is \texttt{\ }{\texttt{\ false\ }}\texttt{\ } and
multiple arguments if \texttt{\ full\ } is
\texttt{\ }{\texttt{\ true\ }}\texttt{\ } .

Default: \texttt{\ }{\texttt{\ "1."\ }}\texttt{\ }

\includesvg[width=0.16667in,height=0.16667in]{/assets/icons/16-arrow-right.svg}
View example

\begin{verbatim}
#set enum(numbering: "1.a)")
+ Different
+ Numbering
  + Nested
  + Items
+ Style

#set enum(numbering: n => super[#n])
+ Superscript
+ Numbering!
\end{verbatim}

\includegraphics[width=5in,height=\textheight,keepaspectratio]{/assets/docs/b_5poTPc-SH9hcwOp4TcbAAAAAAAAAAA.png}

\subsubsection{\texorpdfstring{\texttt{\ start\ }}{ start }}\label{parameters-start}

\href{/docs/reference/foundations/int/}{int}

{{ Settable }}

\phantomsection\label{parameters-start-settable-tooltip}
Settable parameters can be customized for all following uses of the
function with a \texttt{\ set\ } rule.

Which number to start the enumeration with.

Default: \texttt{\ }{\texttt{\ 1\ }}\texttt{\ }

\includesvg[width=0.16667in,height=0.16667in]{/assets/icons/16-arrow-right.svg}
View example

\begin{verbatim}
#enum(
  start: 3,
  [Skipping],
  [Ahead],
)
\end{verbatim}

\includegraphics[width=5in,height=\textheight,keepaspectratio]{/assets/docs/NqaMIUfLtrq2fhf9xChjagAAAAAAAAAA.png}

\subsubsection{\texorpdfstring{\texttt{\ full\ }}{ full }}\label{parameters-full}

\href{/docs/reference/foundations/bool/}{bool}

{{ Settable }}

\phantomsection\label{parameters-full-settable-tooltip}
Settable parameters can be customized for all following uses of the
function with a \texttt{\ set\ } rule.

Whether to display the full numbering, including the numbers of all
parent enumerations.

Default: \texttt{\ }{\texttt{\ false\ }}\texttt{\ }

\includesvg[width=0.16667in,height=0.16667in]{/assets/icons/16-arrow-right.svg}
View example

\begin{verbatim}
#set enum(numbering: "1.a)", full: true)
+ Cook
  + Heat water
  + Add ingredients
+ Eat
\end{verbatim}

\includegraphics[width=5in,height=\textheight,keepaspectratio]{/assets/docs/ecL0fn92ARx_6xbLZYFkVAAAAAAAAAAA.png}

\subsubsection{\texorpdfstring{\texttt{\ indent\ }}{ indent }}\label{parameters-indent}

\href{/docs/reference/layout/length/}{length}

{{ Settable }}

\phantomsection\label{parameters-indent-settable-tooltip}
Settable parameters can be customized for all following uses of the
function with a \texttt{\ set\ } rule.

The indentation of each item.

Default: \texttt{\ }{\texttt{\ 0pt\ }}\texttt{\ }

\subsubsection{\texorpdfstring{\texttt{\ body-indent\ }}{ body-indent }}\label{parameters-body-indent}

\href{/docs/reference/layout/length/}{length}

{{ Settable }}

\phantomsection\label{parameters-body-indent-settable-tooltip}
Settable parameters can be customized for all following uses of the
function with a \texttt{\ set\ } rule.

The space between the numbering and the body of each item.

Default: \texttt{\ }{\texttt{\ 0.5em\ }}\texttt{\ }

\subsubsection{\texorpdfstring{\texttt{\ spacing\ }}{ spacing }}\label{parameters-spacing}

\href{/docs/reference/foundations/auto/}{auto} {or}
\href{/docs/reference/layout/length/}{length}

{{ Settable }}

\phantomsection\label{parameters-spacing-settable-tooltip}
Settable parameters can be customized for all following uses of the
function with a \texttt{\ set\ } rule.

The spacing between the items of the enumeration.

If set to \texttt{\ }{\texttt{\ auto\ }}\texttt{\ } , uses paragraph
\href{/docs/reference/model/par/\#parameters-leading}{\texttt{\ leading\ }}
for tight enumerations and paragraph
\href{/docs/reference/model/par/\#parameters-spacing}{\texttt{\ spacing\ }}
for wide (non-tight) enumerations.

Default: \texttt{\ }{\texttt{\ auto\ }}\texttt{\ }

\subsubsection{\texorpdfstring{\texttt{\ number-align\ }}{ number-align }}\label{parameters-number-align}

\href{/docs/reference/layout/alignment/}{alignment}

{{ Settable }}

\phantomsection\label{parameters-number-align-settable-tooltip}
Settable parameters can be customized for all following uses of the
function with a \texttt{\ set\ } rule.

The alignment that enum numbers should have.

By default, this is set to
\texttt{\ end\ }{\texttt{\ +\ }}\texttt{\ top\ } , which aligns enum
numbers towards end of the current text direction (in left-to-right
script, for example, this is the same as \texttt{\ right\ } ) and at the
top of the line. The choice of \texttt{\ end\ } for horizontal alignment
of enum numbers is usually preferred over \texttt{\ start\ } , as
numbers then grow away from the text instead of towards it, avoiding
certain visual issues. This option lets you override this behaviour,
however. (Also to note is that the
\href{/docs/reference/model/list/}{unordered list} uses a different
method for this, by giving the \texttt{\ marker\ } content an alignment
directly.).

Default: \texttt{\ end\ }{\texttt{\ +\ }}\texttt{\ top\ }

\includesvg[width=0.16667in,height=0.16667in]{/assets/icons/16-arrow-right.svg}
View example

\begin{verbatim}
#set enum(number-align: start + bottom)

Here are some powers of two:
1. One
2. Two
4. Four
8. Eight
16. Sixteen
32. Thirty two
\end{verbatim}

\includegraphics[width=5in,height=\textheight,keepaspectratio]{/assets/docs/s-zUl9r9z6yKdW4VnsLi_AAAAAAAAAAA.png}

\subsubsection{\texorpdfstring{\texttt{\ children\ }}{ children }}\label{parameters-children}

\href{/docs/reference/foundations/content/}{content} {or}
\href{/docs/reference/foundations/array/}{array}

{Required} {{ Positional }}

\phantomsection\label{parameters-children-positional-tooltip}
Positional parameters are specified in order, without names.

{{ Variadic }}

\phantomsection\label{parameters-children-variadic-tooltip}
Variadic parameters can be specified multiple times.

The numbered list\textquotesingle s items.

When using the enum syntax, adjacent items are automatically collected
into enumerations, even through constructs like for loops.

\includesvg[width=0.16667in,height=0.16667in]{/assets/icons/16-arrow-right.svg}
View example

\begin{verbatim}
#for phase in (
   "Launch",
   "Orbit",
   "Descent",
) [+ #phase]
\end{verbatim}

\includegraphics[width=5in,height=\textheight,keepaspectratio]{/assets/docs/9haSHPkr8gDAx-1cEtmf8QAAAAAAAAAA.png}

\subsection{\texorpdfstring{{ Definitions
}}{ Definitions }}\label{definitions}

\phantomsection\label{definitions-tooltip}
Functions and types and can have associated definitions. These are
accessed by specifying the function or type, followed by a period, and
then the definition\textquotesingle s name.

\subsubsection{\texorpdfstring{\texttt{\ item\ } {{ Element
}}}{ item   Element }}\label{definitions-item}

\phantomsection\label{definitions-item-element-tooltip}
Element functions can be customized with \texttt{\ set\ } and
\texttt{\ show\ } rules.

An enumeration item.

enum { . } { item } (

{ \hyperref[definitions-item-parameters-number]{}
\href{/docs/reference/foundations/none/}{none}
\href{/docs/reference/foundations/int/}{int} , } {
\href{/docs/reference/foundations/content/}{content} , }

) -\textgreater{} \href{/docs/reference/foundations/content/}{content}

\paragraph{\texorpdfstring{\texttt{\ number\ }}{ number }}\label{definitions-item-number}

\href{/docs/reference/foundations/none/}{none} {or}
\href{/docs/reference/foundations/int/}{int}

{{ Positional }}

\phantomsection\label{definitions-item-number-positional-tooltip}
Positional parameters are specified in order, without names.

{{ Settable }}

\phantomsection\label{definitions-item-number-settable-tooltip}
Settable parameters can be customized for all following uses of the
function with a \texttt{\ set\ } rule.

The item\textquotesingle s number.

Default: \texttt{\ }{\texttt{\ none\ }}\texttt{\ }

\paragraph{\texorpdfstring{\texttt{\ body\ }}{ body }}\label{definitions-item-body}

\href{/docs/reference/foundations/content/}{content}

{Required} {{ Positional }}

\phantomsection\label{definitions-item-body-positional-tooltip}
Positional parameters are specified in order, without names.

The item\textquotesingle s body.

\href{/docs/reference/model/link/}{\pandocbounded{\includesvg[keepaspectratio]{/assets/icons/16-arrow-right.svg}}}

{ Link } { Previous page }

\href{/docs/reference/model/numbering/}{\pandocbounded{\includesvg[keepaspectratio]{/assets/icons/16-arrow-right.svg}}}

{ Numbering } { Next page }


\section{Docs LaTeX/typst.app/docs/reference/model/heading.tex}
\title{typst.app/docs/reference/model/heading}

\begin{itemize}
\tightlist
\item
  \href{/docs}{\includesvg[width=0.16667in,height=0.16667in]{/assets/icons/16-docs-dark.svg}}
\item
  \includesvg[width=0.16667in,height=0.16667in]{/assets/icons/16-arrow-right.svg}
\item
  \href{/docs/reference/}{Reference}
\item
  \includesvg[width=0.16667in,height=0.16667in]{/assets/icons/16-arrow-right.svg}
\item
  \href{/docs/reference/model/}{Model}
\item
  \includesvg[width=0.16667in,height=0.16667in]{/assets/icons/16-arrow-right.svg}
\item
  \href{/docs/reference/model/heading/}{Heading}
\end{itemize}

\section{\texorpdfstring{\texttt{\ heading\ } {{ Element
}}}{ heading   Element }}\label{summary}

\phantomsection\label{element-tooltip}
Element functions can be customized with \texttt{\ set\ } and
\texttt{\ show\ } rules.

A section heading.

With headings, you can structure your document into sections. Each
heading has a \emph{level,} which starts at one and is unbounded
upwards. This level indicates the logical role of the following content
(section, subsection, etc.) A top-level heading indicates a top-level
section of the document (not the document\textquotesingle s title).

Typst can automatically number your headings for you. To enable
numbering, specify how you want your headings to be numbered with a
\href{/docs/reference/model/numbering/}{numbering pattern or function} .

Independently of the numbering, Typst can also automatically generate an
\href{/docs/reference/model/outline/}{outline} of all headings for you.
To exclude one or more headings from this outline, you can set the
\texttt{\ outlined\ } parameter to
\texttt{\ }{\texttt{\ false\ }}\texttt{\ } .

\subsection{Example}\label{example}

\begin{verbatim}
#set heading(numbering: "1.a)")

= Introduction
In recent years, ...

== Preliminaries
To start, ...
\end{verbatim}

\includegraphics[width=5in,height=\textheight,keepaspectratio]{/assets/docs/PajtbDMMN2eDYZCkAh9ZJwAAAAAAAAAA.png}

\subsection{Syntax}\label{syntax}

Headings have dedicated syntax: They can be created by starting a line
with one or multiple equals signs, followed by a space. The number of
equals signs determines the heading\textquotesingle s logical nesting
depth. The \texttt{\ offset\ } field can be set to configure the
starting depth.

\subsection{\texorpdfstring{{ Parameters
}}{ Parameters }}\label{parameters}

\phantomsection\label{parameters-tooltip}
Parameters are the inputs to a function. They are specified in
parentheses after the function name.

{ heading } (

{ \hyperref[parameters-level]{level :}
\href{/docs/reference/foundations/auto/}{auto}
\href{/docs/reference/foundations/int/}{int} , } {
\hyperref[parameters-depth]{depth :}
\href{/docs/reference/foundations/int/}{int} , } {
\hyperref[parameters-offset]{offset :}
\href{/docs/reference/foundations/int/}{int} , } {
\hyperref[parameters-numbering]{numbering :}
\href{/docs/reference/foundations/none/}{none}
\href{/docs/reference/foundations/str/}{str}
\href{/docs/reference/foundations/function/}{function} , } {
\hyperref[parameters-supplement]{supplement :}
\href{/docs/reference/foundations/none/}{none}
\href{/docs/reference/foundations/auto/}{auto}
\href{/docs/reference/foundations/content/}{content}
\href{/docs/reference/foundations/function/}{function} , } {
\hyperref[parameters-outlined]{outlined :}
\href{/docs/reference/foundations/bool/}{bool} , } {
\hyperref[parameters-bookmarked]{bookmarked :}
\href{/docs/reference/foundations/auto/}{auto}
\href{/docs/reference/foundations/bool/}{bool} , } {
\hyperref[parameters-hanging-indent]{hanging-indent :}
\href{/docs/reference/foundations/auto/}{auto}
\href{/docs/reference/layout/length/}{length} , } {
\href{/docs/reference/foundations/content/}{content} , }

) -\textgreater{} \href{/docs/reference/foundations/content/}{content}

\subsubsection{\texorpdfstring{\texttt{\ level\ }}{ level }}\label{parameters-level}

\href{/docs/reference/foundations/auto/}{auto} {or}
\href{/docs/reference/foundations/int/}{int}

{{ Settable }}

\phantomsection\label{parameters-level-settable-tooltip}
Settable parameters can be customized for all following uses of the
function with a \texttt{\ set\ } rule.

The absolute nesting depth of the heading, starting from one. If set to
\texttt{\ }{\texttt{\ auto\ }}\texttt{\ } , it is computed from
\texttt{\ offset\ }{\texttt{\ +\ }}\texttt{\ depth\ } .

This is primarily useful for usage in
\href{/docs/reference/styling/\#show-rules}{show rules} (either with
\href{/docs/reference/foundations/function/\#definitions-where}{\texttt{\ where\ }}
selectors or by accessing the level directly on a shown heading).

Default: \texttt{\ }{\texttt{\ auto\ }}\texttt{\ }

\includesvg[width=0.16667in,height=0.16667in]{/assets/icons/16-arrow-right.svg}
View example

\begin{verbatim}
#show heading.where(level: 2): set text(red)

= Level 1
== Level 2

#set heading(offset: 1)
= Also level 2
== Level 3
\end{verbatim}

\includegraphics[width=5in,height=\textheight,keepaspectratio]{/assets/docs/_pDm-P05bg_jGbl9uvGjlAAAAAAAAAAA.png}

\subsubsection{\texorpdfstring{\texttt{\ depth\ }}{ depth }}\label{parameters-depth}

\href{/docs/reference/foundations/int/}{int}

{{ Settable }}

\phantomsection\label{parameters-depth-settable-tooltip}
Settable parameters can be customized for all following uses of the
function with a \texttt{\ set\ } rule.

The relative nesting depth of the heading, starting from one. This is
combined with \texttt{\ offset\ } to compute the actual
\texttt{\ level\ } .

This is set by the heading syntax, such that
\texttt{\ }{\texttt{\ ==\ Heading\ }}\texttt{\ } creates a heading with
logical depth of 2, but actual level
\texttt{\ offset\ }{\texttt{\ +\ }}\texttt{\ }{\texttt{\ 2\ }}\texttt{\ }
. If you construct a heading manually, you should typically prefer this
over setting the absolute level.

Default: \texttt{\ }{\texttt{\ 1\ }}\texttt{\ }

\subsubsection{\texorpdfstring{\texttt{\ offset\ }}{ offset }}\label{parameters-offset}

\href{/docs/reference/foundations/int/}{int}

{{ Settable }}

\phantomsection\label{parameters-offset-settable-tooltip}
Settable parameters can be customized for all following uses of the
function with a \texttt{\ set\ } rule.

The starting offset of each heading\textquotesingle s \texttt{\ level\ }
, used to turn its relative \texttt{\ depth\ } into its absolute
\texttt{\ level\ } .

Default: \texttt{\ }{\texttt{\ 0\ }}\texttt{\ }

\includesvg[width=0.16667in,height=0.16667in]{/assets/icons/16-arrow-right.svg}
View example

\begin{verbatim}
= Level 1

#set heading(offset: 1, numbering: "1.1")
= Level 2

#heading(offset: 2, depth: 2)[
  I'm level 4
]
\end{verbatim}

\includegraphics[width=5in,height=\textheight,keepaspectratio]{/assets/docs/hKtWik5-HwMMejqOwDVKLAAAAAAAAAAA.png}

\subsubsection{\texorpdfstring{\texttt{\ numbering\ }}{ numbering }}\label{parameters-numbering}

\href{/docs/reference/foundations/none/}{none} {or}
\href{/docs/reference/foundations/str/}{str} {or}
\href{/docs/reference/foundations/function/}{function}

{{ Settable }}

\phantomsection\label{parameters-numbering-settable-tooltip}
Settable parameters can be customized for all following uses of the
function with a \texttt{\ set\ } rule.

How to number the heading. Accepts a
\href{/docs/reference/model/numbering/}{numbering pattern or function} .

Default: \texttt{\ }{\texttt{\ none\ }}\texttt{\ }

\includesvg[width=0.16667in,height=0.16667in]{/assets/icons/16-arrow-right.svg}
View example

\begin{verbatim}
#set heading(numbering: "1.a.")

= A section
== A subsection
=== A sub-subsection
\end{verbatim}

\includegraphics[width=5in,height=\textheight,keepaspectratio]{/assets/docs/dtIXlP8zFF4SfNqscPeLbAAAAAAAAAAA.png}

\subsubsection{\texorpdfstring{\texttt{\ supplement\ }}{ supplement }}\label{parameters-supplement}

\href{/docs/reference/foundations/none/}{none} {or}
\href{/docs/reference/foundations/auto/}{auto} {or}
\href{/docs/reference/foundations/content/}{content} {or}
\href{/docs/reference/foundations/function/}{function}

{{ Settable }}

\phantomsection\label{parameters-supplement-settable-tooltip}
Settable parameters can be customized for all following uses of the
function with a \texttt{\ set\ } rule.

A supplement for the heading.

For references to headings, this is added before the referenced number.

If a function is specified, it is passed the referenced heading and
should return content.

Default: \texttt{\ }{\texttt{\ auto\ }}\texttt{\ }

\includesvg[width=0.16667in,height=0.16667in]{/assets/icons/16-arrow-right.svg}
View example

\begin{verbatim}
#set heading(numbering: "1.", supplement: [Chapter])

= Introduction <intro>
In @intro, we see how to turn
Sections into Chapters. And
in @intro[Part], it is done
manually.
\end{verbatim}

\includegraphics[width=5in,height=\textheight,keepaspectratio]{/assets/docs/OZMUTnmWZCt9L0XUTDaRmQAAAAAAAAAA.png}

\subsubsection{\texorpdfstring{\texttt{\ outlined\ }}{ outlined }}\label{parameters-outlined}

\href{/docs/reference/foundations/bool/}{bool}

{{ Settable }}

\phantomsection\label{parameters-outlined-settable-tooltip}
Settable parameters can be customized for all following uses of the
function with a \texttt{\ set\ } rule.

Whether the heading should appear in the
\href{/docs/reference/model/outline/}{outline} .

Note that this property, if set to
\texttt{\ }{\texttt{\ true\ }}\texttt{\ } , ensures the heading is also
shown as a bookmark in the exported PDF\textquotesingle s outline (when
exporting to PDF). To change that behavior, use the
\texttt{\ bookmarked\ } property.

Default: \texttt{\ }{\texttt{\ true\ }}\texttt{\ }

\includesvg[width=0.16667in,height=0.16667in]{/assets/icons/16-arrow-right.svg}
View example

\begin{verbatim}
#outline()

#heading[Normal]
This is a normal heading.

#heading(outlined: false)[Hidden]
This heading does not appear
in the outline.
\end{verbatim}

\includegraphics[width=5in,height=\textheight,keepaspectratio]{/assets/docs/q3R6803Mv9D8hpPx5wD4TgAAAAAAAAAA.png}

\subsubsection{\texorpdfstring{\texttt{\ bookmarked\ }}{ bookmarked }}\label{parameters-bookmarked}

\href{/docs/reference/foundations/auto/}{auto} {or}
\href{/docs/reference/foundations/bool/}{bool}

{{ Settable }}

\phantomsection\label{parameters-bookmarked-settable-tooltip}
Settable parameters can be customized for all following uses of the
function with a \texttt{\ set\ } rule.

Whether the heading should appear as a bookmark in the exported
PDF\textquotesingle s outline. Doesn\textquotesingle t affect other
export formats, such as PNG.

The default value of \texttt{\ }{\texttt{\ auto\ }}\texttt{\ } indicates
that the heading will only appear in the exported PDF\textquotesingle s
outline if its \texttt{\ outlined\ } property is set to
\texttt{\ }{\texttt{\ true\ }}\texttt{\ } , that is, if it would also be
listed in Typst\textquotesingle s
\href{/docs/reference/model/outline/}{outline} . Setting this property
to either \texttt{\ }{\texttt{\ true\ }}\texttt{\ } (bookmark) or
\texttt{\ }{\texttt{\ false\ }}\texttt{\ } (don\textquotesingle t
bookmark) bypasses that behavior.

Default: \texttt{\ }{\texttt{\ auto\ }}\texttt{\ }

\includesvg[width=0.16667in,height=0.16667in]{/assets/icons/16-arrow-right.svg}
View example

\begin{verbatim}
#heading[Normal heading]
This heading will be shown in
the PDF's bookmark outline.

#heading(bookmarked: false)[Not bookmarked]
This heading won't be
bookmarked in the resulting
PDF.
\end{verbatim}

\includegraphics[width=5in,height=\textheight,keepaspectratio]{/assets/docs/_UvMUDZOtTdH4i83Hac2iwAAAAAAAAAA.png}

\subsubsection{\texorpdfstring{\texttt{\ hanging-indent\ }}{ hanging-indent }}\label{parameters-hanging-indent}

\href{/docs/reference/foundations/auto/}{auto} {or}
\href{/docs/reference/layout/length/}{length}

{{ Settable }}

\phantomsection\label{parameters-hanging-indent-settable-tooltip}
Settable parameters can be customized for all following uses of the
function with a \texttt{\ set\ } rule.

The indent all but the first line of a heading should have.

The default value of \texttt{\ }{\texttt{\ auto\ }}\texttt{\ } indicates
that the subsequent heading lines will be indented based on the width of
the numbering.

Default: \texttt{\ }{\texttt{\ auto\ }}\texttt{\ }

\includesvg[width=0.16667in,height=0.16667in]{/assets/icons/16-arrow-right.svg}
View example

\begin{verbatim}
#set heading(numbering: "1.")
#heading[A very, very, very, very, very, very long heading]
\end{verbatim}

\includegraphics[width=5in,height=\textheight,keepaspectratio]{/assets/docs/35Dg34kG-7rg1-RFp8FaIgAAAAAAAAAA.png}

\subsubsection{\texorpdfstring{\texttt{\ body\ }}{ body }}\label{parameters-body}

\href{/docs/reference/foundations/content/}{content}

{Required} {{ Positional }}

\phantomsection\label{parameters-body-positional-tooltip}
Positional parameters are specified in order, without names.

The heading\textquotesingle s title.

\href{/docs/reference/model/footnote/}{\pandocbounded{\includesvg[keepaspectratio]{/assets/icons/16-arrow-right.svg}}}

{ Footnote } { Previous page }

\href{/docs/reference/model/link/}{\pandocbounded{\includesvg[keepaspectratio]{/assets/icons/16-arrow-right.svg}}}

{ Link } { Next page }


\section{Docs LaTeX/typst.app/docs/reference/model/link.tex}
\title{typst.app/docs/reference/model/link}

\begin{itemize}
\tightlist
\item
  \href{/docs}{\includesvg[width=0.16667in,height=0.16667in]{/assets/icons/16-docs-dark.svg}}
\item
  \includesvg[width=0.16667in,height=0.16667in]{/assets/icons/16-arrow-right.svg}
\item
  \href{/docs/reference/}{Reference}
\item
  \includesvg[width=0.16667in,height=0.16667in]{/assets/icons/16-arrow-right.svg}
\item
  \href{/docs/reference/model/}{Model}
\item
  \includesvg[width=0.16667in,height=0.16667in]{/assets/icons/16-arrow-right.svg}
\item
  \href{/docs/reference/model/link/}{Link}
\end{itemize}

\section{\texorpdfstring{\texttt{\ link\ } {{ Element
}}}{ link   Element }}\label{summary}

\phantomsection\label{element-tooltip}
Element functions can be customized with \texttt{\ set\ } and
\texttt{\ show\ } rules.

Links to a URL or a location in the document.

By default, links are not styled any different from normal text.
However, you can easily apply a style of your choice with a show rule.

\subsection{Example}\label{example}

\begin{verbatim}
#show link: underline

https://example.com \

#link("https://example.com") \
#link("https://example.com")[
  See example.com
]
\end{verbatim}

\includegraphics[width=5in,height=\textheight,keepaspectratio]{/assets/docs/mBfQJYO4ObjIyuLi_FjKfgAAAAAAAAAA.png}

\subsection{Syntax}\label{syntax}

This function also has dedicated syntax: Text that starts with
\texttt{\ http://\ } or \texttt{\ https://\ } is automatically turned
into a link.

\subsection{\texorpdfstring{{ Parameters
}}{ Parameters }}\label{parameters}

\phantomsection\label{parameters-tooltip}
Parameters are the inputs to a function. They are specified in
parentheses after the function name.

{ link } (

{ \href{/docs/reference/foundations/str/}{str}
\href{/docs/reference/foundations/label/}{label}
\href{/docs/reference/introspection/location/}{location}
\href{/docs/reference/foundations/dictionary/}{dictionary} , } {
\href{/docs/reference/foundations/content/}{content} , }

) -\textgreater{} \href{/docs/reference/foundations/content/}{content}

\subsubsection{\texorpdfstring{\texttt{\ dest\ }}{ dest }}\label{parameters-dest}

\href{/docs/reference/foundations/str/}{str} {or}
\href{/docs/reference/foundations/label/}{label} {or}
\href{/docs/reference/introspection/location/}{location} {or}
\href{/docs/reference/foundations/dictionary/}{dictionary}

{Required} {{ Positional }}

\phantomsection\label{parameters-dest-positional-tooltip}
Positional parameters are specified in order, without names.

The destination the link points to.

\begin{itemize}
\item
  To link to web pages, \texttt{\ dest\ } should be a valid URL string.
  If the URL is in the \texttt{\ mailto:\ } or \texttt{\ tel:\ } scheme
  and the \texttt{\ body\ } parameter is omitted, the email address or
  phone number will be the link\textquotesingle s body, without the
  scheme.
\item
  To link to another part of the document, \texttt{\ dest\ } can take
  one of three forms:

  \begin{itemize}
  \item
    A \href{/docs/reference/foundations/label/}{label} attached to an
    element. If you also want automatic text for the link based on the
    element, consider using a
    \href{/docs/reference/model/ref/}{reference} instead.
  \item
    A
    \href{/docs/reference/introspection/location/}{\texttt{\ location\ }}
    (typically retrieved from
    \href{/docs/reference/introspection/here/}{\texttt{\ here\ }} ,
    \href{/docs/reference/introspection/locate/}{\texttt{\ locate\ }} or
    \href{/docs/reference/introspection/query/}{\texttt{\ query\ }} ).
  \item
    A dictionary with a \texttt{\ page\ } key of type
    \href{/docs/reference/foundations/int/}{integer} and \texttt{\ x\ }
    and \texttt{\ y\ } coordinates of type
    \href{/docs/reference/layout/length/}{length} . Pages are counted
    from one, and the coordinates are relative to the
    page\textquotesingle s top left corner.
  \end{itemize}
\end{itemize}

\includesvg[width=0.16667in,height=0.16667in]{/assets/icons/16-arrow-right.svg}
View example

\begin{verbatim}
= Introduction <intro>
#link("mailto:hello@typst.app") \
#link(<intro>)[Go to intro] \
#link((page: 1, x: 0pt, y: 0pt))[
  Go to top
]
\end{verbatim}

\includegraphics[width=5in,height=\textheight,keepaspectratio]{/assets/docs/r-LwcI2C1K4OtUWhtvg8QgAAAAAAAAAA.png}

\subsubsection{\texorpdfstring{\texttt{\ body\ }}{ body }}\label{parameters-body}

\href{/docs/reference/foundations/content/}{content}

{Required} {{ Positional }}

\phantomsection\label{parameters-body-positional-tooltip}
Positional parameters are specified in order, without names.

The content that should become a link.

If \texttt{\ dest\ } is an URL string, the parameter can be omitted. In
this case, the URL will be shown as the link.

\href{/docs/reference/model/heading/}{\pandocbounded{\includesvg[keepaspectratio]{/assets/icons/16-arrow-right.svg}}}

{ Heading } { Previous page }

\href{/docs/reference/model/enum/}{\pandocbounded{\includesvg[keepaspectratio]{/assets/icons/16-arrow-right.svg}}}

{ Numbered List } { Next page }


\section{Docs LaTeX/typst.app/docs/reference/model/terms.tex}
\title{typst.app/docs/reference/model/terms}

\begin{itemize}
\tightlist
\item
  \href{/docs}{\includesvg[width=0.16667in,height=0.16667in]{/assets/icons/16-docs-dark.svg}}
\item
  \includesvg[width=0.16667in,height=0.16667in]{/assets/icons/16-arrow-right.svg}
\item
  \href{/docs/reference/}{Reference}
\item
  \includesvg[width=0.16667in,height=0.16667in]{/assets/icons/16-arrow-right.svg}
\item
  \href{/docs/reference/model/}{Model}
\item
  \includesvg[width=0.16667in,height=0.16667in]{/assets/icons/16-arrow-right.svg}
\item
  \href{/docs/reference/model/terms/}{Term List}
\end{itemize}

\section{\texorpdfstring{\texttt{\ terms\ } {{ Element
}}}{ terms   Element }}\label{summary}

\phantomsection\label{element-tooltip}
Element functions can be customized with \texttt{\ set\ } and
\texttt{\ show\ } rules.

A list of terms and their descriptions.

Displays a sequence of terms and their descriptions vertically. When the
descriptions span over multiple lines, they use hanging indent to
communicate the visual hierarchy.

\subsection{Example}\label{example}

\begin{verbatim}
/ Ligature: A merged glyph.
/ Kerning: A spacing adjustment
  between two adjacent letters.
\end{verbatim}

\includegraphics[width=5in,height=\textheight,keepaspectratio]{/assets/docs/qjdQTTJFa_RYtcfu42IiawAAAAAAAAAA.png}

\subsection{Syntax}\label{syntax}

This function also has dedicated syntax: Starting a line with a slash,
followed by a term, a colon and a description creates a term list item.

\subsection{\texorpdfstring{{ Parameters
}}{ Parameters }}\label{parameters}

\phantomsection\label{parameters-tooltip}
Parameters are the inputs to a function. They are specified in
parentheses after the function name.

{ terms } (

{ \hyperref[parameters-tight]{tight :}
\href{/docs/reference/foundations/bool/}{bool} , } {
\hyperref[parameters-separator]{separator :}
\href{/docs/reference/foundations/content/}{content} , } {
\hyperref[parameters-indent]{indent :}
\href{/docs/reference/layout/length/}{length} , } {
\hyperref[parameters-hanging-indent]{hanging-indent :}
\href{/docs/reference/layout/length/}{length} , } {
\hyperref[parameters-spacing]{spacing :}
\href{/docs/reference/foundations/auto/}{auto}
\href{/docs/reference/layout/length/}{length} , } {
\hyperref[parameters-children]{..}
\href{/docs/reference/foundations/content/}{content}
\href{/docs/reference/foundations/array/}{array} , }

) -\textgreater{} \href{/docs/reference/foundations/content/}{content}

\subsubsection{\texorpdfstring{\texttt{\ tight\ }}{ tight }}\label{parameters-tight}

\href{/docs/reference/foundations/bool/}{bool}

{{ Settable }}

\phantomsection\label{parameters-tight-settable-tooltip}
Settable parameters can be customized for all following uses of the
function with a \texttt{\ set\ } rule.

Defines the default
\href{/docs/reference/model/terms/\#parameters-spacing}{spacing} of the
term list. If it is \texttt{\ }{\texttt{\ false\ }}\texttt{\ } , the
items are spaced apart with
\href{/docs/reference/model/par/\#parameters-spacing}{paragraph spacing}
. If it is \texttt{\ }{\texttt{\ true\ }}\texttt{\ } , they use
\href{/docs/reference/model/par/\#parameters-leading}{paragraph leading}
instead. This makes the list more compact, which can look better if the
items are short.

In markup mode, the value of this parameter is determined based on
whether items are separated with a blank line. If items directly follow
each other, this is set to \texttt{\ }{\texttt{\ true\ }}\texttt{\ } ;
if items are separated by a blank line, this is set to
\texttt{\ }{\texttt{\ false\ }}\texttt{\ } . The markup-defined
tightness cannot be overridden with set rules.

Default: \texttt{\ }{\texttt{\ true\ }}\texttt{\ }

\includesvg[width=0.16667in,height=0.16667in]{/assets/icons/16-arrow-right.svg}
View example

\begin{verbatim}
/ Fact: If a term list has a lot
  of text, and maybe other inline
  content, it should not be tight
  anymore.

/ Tip: To make it wide, simply
  insert a blank line between the
  items.
\end{verbatim}

\includegraphics[width=5in,height=\textheight,keepaspectratio]{/assets/docs/skkuR2BgltlCHUy9cPpX7gAAAAAAAAAA.png}

\subsubsection{\texorpdfstring{\texttt{\ separator\ }}{ separator }}\label{parameters-separator}

\href{/docs/reference/foundations/content/}{content}

{{ Settable }}

\phantomsection\label{parameters-separator-settable-tooltip}
Settable parameters can be customized for all following uses of the
function with a \texttt{\ set\ } rule.

The separator between the item and the description.

If you want to just separate them with a certain amount of space, use
\texttt{\ }{\texttt{\ h\ }}\texttt{\ }{\texttt{\ (\ }}\texttt{\ }{\texttt{\ 2cm\ }}\texttt{\ }{\texttt{\ ,\ }}\texttt{\ weak\ }{\texttt{\ :\ }}\texttt{\ }{\texttt{\ true\ }}\texttt{\ }{\texttt{\ )\ }}\texttt{\ }
as the separator and replace \texttt{\ }{\texttt{\ 2cm\ }}\texttt{\ }
with your desired amount of space.

Default:
\texttt{\ }{\texttt{\ h\ }}\texttt{\ }{\texttt{\ (\ }}\texttt{\ amount\ }{\texttt{\ :\ }}\texttt{\ }{\texttt{\ 0.6em\ }}\texttt{\ }{\texttt{\ ,\ }}\texttt{\ weak\ }{\texttt{\ :\ }}\texttt{\ }{\texttt{\ true\ }}\texttt{\ }{\texttt{\ )\ }}\texttt{\ }

\includesvg[width=0.16667in,height=0.16667in]{/assets/icons/16-arrow-right.svg}
View example

\begin{verbatim}
#set terms(separator: [: ])

/ Colon: A nice separator symbol.
\end{verbatim}

\includegraphics[width=5in,height=\textheight,keepaspectratio]{/assets/docs/xyyblMI8l_99lTt1_p5kWgAAAAAAAAAA.png}

\subsubsection{\texorpdfstring{\texttt{\ indent\ }}{ indent }}\label{parameters-indent}

\href{/docs/reference/layout/length/}{length}

{{ Settable }}

\phantomsection\label{parameters-indent-settable-tooltip}
Settable parameters can be customized for all following uses of the
function with a \texttt{\ set\ } rule.

The indentation of each item.

Default: \texttt{\ }{\texttt{\ 0pt\ }}\texttt{\ }

\subsubsection{\texorpdfstring{\texttt{\ hanging-indent\ }}{ hanging-indent }}\label{parameters-hanging-indent}

\href{/docs/reference/layout/length/}{length}

{{ Settable }}

\phantomsection\label{parameters-hanging-indent-settable-tooltip}
Settable parameters can be customized for all following uses of the
function with a \texttt{\ set\ } rule.

The hanging indent of the description.

This is in addition to the whole item\textquotesingle s
\texttt{\ indent\ } .

Default: \texttt{\ }{\texttt{\ 2em\ }}\texttt{\ }

\includesvg[width=0.16667in,height=0.16667in]{/assets/icons/16-arrow-right.svg}
View example

\begin{verbatim}
#set terms(hanging-indent: 0pt)
/ Term: This term list does not
  make use of hanging indents.
\end{verbatim}

\includegraphics[width=5in,height=\textheight,keepaspectratio]{/assets/docs/6yYrKErT2JtRwBRmpS8r5wAAAAAAAAAA.png}

\subsubsection{\texorpdfstring{\texttt{\ spacing\ }}{ spacing }}\label{parameters-spacing}

\href{/docs/reference/foundations/auto/}{auto} {or}
\href{/docs/reference/layout/length/}{length}

{{ Settable }}

\phantomsection\label{parameters-spacing-settable-tooltip}
Settable parameters can be customized for all following uses of the
function with a \texttt{\ set\ } rule.

The spacing between the items of the term list.

If set to \texttt{\ }{\texttt{\ auto\ }}\texttt{\ } , uses paragraph
\href{/docs/reference/model/par/\#parameters-leading}{\texttt{\ leading\ }}
for tight term lists and paragraph
\href{/docs/reference/model/par/\#parameters-spacing}{\texttt{\ spacing\ }}
for wide (non-tight) term lists.

Default: \texttt{\ }{\texttt{\ auto\ }}\texttt{\ }

\subsubsection{\texorpdfstring{\texttt{\ children\ }}{ children }}\label{parameters-children}

\href{/docs/reference/foundations/content/}{content} {or}
\href{/docs/reference/foundations/array/}{array}

{Required} {{ Positional }}

\phantomsection\label{parameters-children-positional-tooltip}
Positional parameters are specified in order, without names.

{{ Variadic }}

\phantomsection\label{parameters-children-variadic-tooltip}
Variadic parameters can be specified multiple times.

The term list\textquotesingle s children.

When using the term list syntax, adjacent items are automatically
collected into term lists, even through constructs like for loops.

\includesvg[width=0.16667in,height=0.16667in]{/assets/icons/16-arrow-right.svg}
View example

\begin{verbatim}
#for (year, product) in (
  "1978": "TeX",
  "1984": "LaTeX",
  "2019": "Typst",
) [/ #product: Born in #year.]
\end{verbatim}

\includegraphics[width=5in,height=\textheight,keepaspectratio]{/assets/docs/wkvQM6jeTkSTRoaT9Y0lSQAAAAAAAAAA.png}

\subsection{\texorpdfstring{{ Definitions
}}{ Definitions }}\label{definitions}

\phantomsection\label{definitions-tooltip}
Functions and types and can have associated definitions. These are
accessed by specifying the function or type, followed by a period, and
then the definition\textquotesingle s name.

\subsubsection{\texorpdfstring{\texttt{\ item\ } {{ Element
}}}{ item   Element }}\label{definitions-item}

\phantomsection\label{definitions-item-element-tooltip}
Element functions can be customized with \texttt{\ set\ } and
\texttt{\ show\ } rules.

A term list item.

terms { . } { item } (

{ \href{/docs/reference/foundations/content/}{content} , } {
\href{/docs/reference/foundations/content/}{content} , }

) -\textgreater{} \href{/docs/reference/foundations/content/}{content}

\paragraph{\texorpdfstring{\texttt{\ term\ }}{ term }}\label{definitions-item-term}

\href{/docs/reference/foundations/content/}{content}

{Required} {{ Positional }}

\phantomsection\label{definitions-item-term-positional-tooltip}
Positional parameters are specified in order, without names.

The term described by the list item.

\paragraph{\texorpdfstring{\texttt{\ description\ }}{ description }}\label{definitions-item-description}

\href{/docs/reference/foundations/content/}{content}

{Required} {{ Positional }}

\phantomsection\label{definitions-item-description-positional-tooltip}
Positional parameters are specified in order, without names.

The description of the term.

\href{/docs/reference/model/table/}{\pandocbounded{\includesvg[keepaspectratio]{/assets/icons/16-arrow-right.svg}}}

{ Table } { Previous page }

\href{/docs/reference/text/}{\pandocbounded{\includesvg[keepaspectratio]{/assets/icons/16-arrow-right.svg}}}

{ Text } { Next page }


\section{Docs LaTeX/typst.app/docs/reference/model/numbering.tex}
\title{typst.app/docs/reference/model/numbering}

\begin{itemize}
\tightlist
\item
  \href{/docs}{\includesvg[width=0.16667in,height=0.16667in]{/assets/icons/16-docs-dark.svg}}
\item
  \includesvg[width=0.16667in,height=0.16667in]{/assets/icons/16-arrow-right.svg}
\item
  \href{/docs/reference/}{Reference}
\item
  \includesvg[width=0.16667in,height=0.16667in]{/assets/icons/16-arrow-right.svg}
\item
  \href{/docs/reference/model/}{Model}
\item
  \includesvg[width=0.16667in,height=0.16667in]{/assets/icons/16-arrow-right.svg}
\item
  \href{/docs/reference/model/numbering/}{Numbering}
\end{itemize}

\section{\texorpdfstring{\texttt{\ numbering\ }}{ numbering }}\label{summary}

Applies a numbering to a sequence of numbers.

A numbering defines how a sequence of numbers should be displayed as
content. It is defined either through a pattern string or an arbitrary
function.

A numbering pattern consists of counting symbols, for which the actual
number is substituted, their prefixes, and one suffix. The prefixes and
the suffix are repeated as-is.

\subsection{Example}\label{example}

\begin{verbatim}
#numbering("1.1)", 1, 2, 3) \
#numbering("1.a.i", 1, 2) \
#numbering("I – 1", 12, 2) \
#numbering(
  (..nums) => nums
    .pos()
    .map(str)
    .join(".") + ")",
  1, 2, 3,
)
\end{verbatim}

\includegraphics[width=5in,height=\textheight,keepaspectratio]{/assets/docs/ViM4jxlRNjTCcZLHAqTQsQAAAAAAAAAA.png}

\subsection{Numbering patterns and numbering
functions}\label{numbering-patterns-and-numbering-functions}

There are multiple instances where you can provide a numbering pattern
or function in Typst. For example, when defining how to number
\href{/docs/reference/model/heading/}{headings} or
\href{/docs/reference/model/figure/}{figures} . Every time, the expected
format is the same as the one described below for the
\href{/docs/reference/model/numbering/\#parameters-numbering}{\texttt{\ numbering\ }}
parameter.

The following example illustrates that a numbering function is just a
regular \href{/docs/reference/foundations/function/}{function} that
accepts numbers and returns
\href{/docs/reference/foundations/content/}{\texttt{\ content\ }} .

\begin{verbatim}
#let unary(.., last) = "|" * last
#set heading(numbering: unary)
= First heading
= Second heading
= Third heading
\end{verbatim}

\includegraphics[width=5in,height=\textheight,keepaspectratio]{/assets/docs/y3Y2xT6PKYJ3nJF6y9bcPwAAAAAAAAAA.png}

\subsection{\texorpdfstring{{ Parameters
}}{ Parameters }}\label{parameters}

\phantomsection\label{parameters-tooltip}
Parameters are the inputs to a function. They are specified in
parentheses after the function name.

{ numbering } (

{ \href{/docs/reference/foundations/str/}{str}
\href{/docs/reference/foundations/function/}{function} , } {
\hyperref[parameters-numbers]{..}
\href{/docs/reference/foundations/int/}{int} , }

) -\textgreater{} { any }

\subsubsection{\texorpdfstring{\texttt{\ numbering\ }}{ numbering }}\label{parameters-numbering}

\href{/docs/reference/foundations/str/}{str} {or}
\href{/docs/reference/foundations/function/}{function}

{Required} {{ Positional }}

\phantomsection\label{parameters-numbering-positional-tooltip}
Positional parameters are specified in order, without names.

Defines how the numbering works.

\textbf{Counting symbols} are \texttt{\ 1\ } , \texttt{\ a\ } ,
\texttt{\ A\ } , \texttt{\ i\ } , \texttt{\ I\ } , \texttt{\ 一\ } ,
\texttt{\ 壹\ } , \texttt{\ �\ } , \texttt{\ �\ } ,
\texttt{\ ã‚¢\ } , \texttt{\ イ\ } , \texttt{\ ×?\ } , \texttt{\ ê°€\ }
, \texttt{\ ㄱ\ } , \texttt{\ *\ } , \texttt{\ â‘\ } , and
\texttt{\ ⓵\ } . They are replaced by the number in the sequence,
preserving the original case.

The \texttt{\ *\ } character means that symbols should be used to count,
in the order of \texttt{\ *\ } , \texttt{\ â€\ } , \texttt{\ ‡\ } ,
\texttt{\ §\ } , \texttt{\ ¶\ } , \texttt{\ ‖\ } . If there are more
than six items, the number is represented using repeated symbols.

\textbf{Suffixes} are all characters after the last counting symbol.
They are repeated as-is at the end of any rendered number.

\textbf{Prefixes} are all characters that are neither counting symbols
nor suffixes. They are repeated as-is at in front of their rendered
equivalent of their counting symbol.

This parameter can also be an arbitrary function that gets each number
as an individual argument. When given a function, the
\texttt{\ numbering\ } function just forwards the arguments to that
function. While this is not particularly useful in itself, it means that
you can just give arbitrary numberings to the \texttt{\ numbering\ }
function without caring whether they are defined as a pattern or
function.

\subsubsection{\texorpdfstring{\texttt{\ numbers\ }}{ numbers }}\label{parameters-numbers}

\href{/docs/reference/foundations/int/}{int}

{Required} {{ Positional }}

\phantomsection\label{parameters-numbers-positional-tooltip}
Positional parameters are specified in order, without names.

{{ Variadic }}

\phantomsection\label{parameters-numbers-variadic-tooltip}
Variadic parameters can be specified multiple times.

The numbers to apply the numbering to. Must be positive.

If \texttt{\ numbering\ } is a pattern and more numbers than counting
symbols are given, the last counting symbol with its prefix is repeated.

\href{/docs/reference/model/enum/}{\pandocbounded{\includesvg[keepaspectratio]{/assets/icons/16-arrow-right.svg}}}

{ Numbered List } { Previous page }

\href{/docs/reference/model/outline/}{\pandocbounded{\includesvg[keepaspectratio]{/assets/icons/16-arrow-right.svg}}}

{ Outline } { Next page }


\section{Docs LaTeX/typst.app/docs/reference/model/document.tex}
\title{typst.app/docs/reference/model/document}

\begin{itemize}
\tightlist
\item
  \href{/docs}{\includesvg[width=0.16667in,height=0.16667in]{/assets/icons/16-docs-dark.svg}}
\item
  \includesvg[width=0.16667in,height=0.16667in]{/assets/icons/16-arrow-right.svg}
\item
  \href{/docs/reference/}{Reference}
\item
  \includesvg[width=0.16667in,height=0.16667in]{/assets/icons/16-arrow-right.svg}
\item
  \href{/docs/reference/model/}{Model}
\item
  \includesvg[width=0.16667in,height=0.16667in]{/assets/icons/16-arrow-right.svg}
\item
  \href{/docs/reference/model/document/}{Document}
\end{itemize}

\section{\texorpdfstring{\texttt{\ document\ } {{ Element
}}}{ document   Element }}\label{summary}

\phantomsection\label{element-tooltip}
Element functions can be customized with \texttt{\ set\ } and
\texttt{\ show\ } rules.

The root element of a document and its metadata.

All documents are automatically wrapped in a \texttt{\ document\ }
element. You cannot create a document element yourself. This function is
only used with \href{/docs/reference/styling/\#set-rules}{set rules} to
specify document metadata. Such a set rule must not occur inside of any
layout container.

\begin{verbatim}
#set document(title: [Hello])

This has no visible output, but
embeds metadata into the PDF!
\end{verbatim}

\includegraphics[width=5in,height=\textheight,keepaspectratio]{/assets/docs/g-R4wkXOtFnr5QmDRHynVAAAAAAAAAAA.png}

Note that metadata set with this function is not rendered within the
document. Instead, it is embedded in the compiled PDF file.

\subsection{\texorpdfstring{{ Parameters
}}{ Parameters }}\label{parameters}

\phantomsection\label{parameters-tooltip}
Parameters are the inputs to a function. They are specified in
parentheses after the function name.

{ document } (

{ \hyperref[parameters-title]{title :}
\href{/docs/reference/foundations/none/}{none}
\href{/docs/reference/foundations/content/}{content} , } {
\hyperref[parameters-author]{author :}
\href{/docs/reference/foundations/str/}{str}
\href{/docs/reference/foundations/array/}{array} , } {
\hyperref[parameters-keywords]{keywords :}
\href{/docs/reference/foundations/str/}{str}
\href{/docs/reference/foundations/array/}{array} , } {
\hyperref[parameters-date]{date :}
\href{/docs/reference/foundations/none/}{none}
\href{/docs/reference/foundations/auto/}{auto}
\href{/docs/reference/foundations/datetime/}{datetime} , }

) -\textgreater{} \href{/docs/reference/foundations/content/}{content}

\subsubsection{\texorpdfstring{\texttt{\ title\ }}{ title }}\label{parameters-title}

\href{/docs/reference/foundations/none/}{none} {or}
\href{/docs/reference/foundations/content/}{content}

{{ Settable }}

\phantomsection\label{parameters-title-settable-tooltip}
Settable parameters can be customized for all following uses of the
function with a \texttt{\ set\ } rule.

The document\textquotesingle s title. This is often rendered as the
title of the PDF viewer window.

While this can be arbitrary content, PDF viewers only support plain text
titles, so the conversion might be lossy.

Default: \texttt{\ }{\texttt{\ none\ }}\texttt{\ }

\subsubsection{\texorpdfstring{\texttt{\ author\ }}{ author }}\label{parameters-author}

\href{/docs/reference/foundations/str/}{str} {or}
\href{/docs/reference/foundations/array/}{array}

{{ Settable }}

\phantomsection\label{parameters-author-settable-tooltip}
Settable parameters can be customized for all following uses of the
function with a \texttt{\ set\ } rule.

The document\textquotesingle s authors.

Default:
\texttt{\ }{\texttt{\ (\ }}\texttt{\ }{\texttt{\ )\ }}\texttt{\ }

\subsubsection{\texorpdfstring{\texttt{\ keywords\ }}{ keywords }}\label{parameters-keywords}

\href{/docs/reference/foundations/str/}{str} {or}
\href{/docs/reference/foundations/array/}{array}

{{ Settable }}

\phantomsection\label{parameters-keywords-settable-tooltip}
Settable parameters can be customized for all following uses of the
function with a \texttt{\ set\ } rule.

The document\textquotesingle s keywords.

Default:
\texttt{\ }{\texttt{\ (\ }}\texttt{\ }{\texttt{\ )\ }}\texttt{\ }

\subsubsection{\texorpdfstring{\texttt{\ date\ }}{ date }}\label{parameters-date}

\href{/docs/reference/foundations/none/}{none} {or}
\href{/docs/reference/foundations/auto/}{auto} {or}
\href{/docs/reference/foundations/datetime/}{datetime}

{{ Settable }}

\phantomsection\label{parameters-date-settable-tooltip}
Settable parameters can be customized for all following uses of the
function with a \texttt{\ set\ } rule.

The document\textquotesingle s creation date.

If this is \texttt{\ }{\texttt{\ auto\ }}\texttt{\ } (default), Typst
uses the current date and time. Setting it to
\texttt{\ }{\texttt{\ none\ }}\texttt{\ } prevents Typst from embedding
any creation date into the PDF metadata.

The year component must be at least zero in order to be embedded into a
PDF.

If you want to create byte-by-byte reproducible PDFs, set this to
something other than \texttt{\ }{\texttt{\ auto\ }}\texttt{\ } .

Default: \texttt{\ }{\texttt{\ auto\ }}\texttt{\ }

\href{/docs/reference/model/cite/}{\pandocbounded{\includesvg[keepaspectratio]{/assets/icons/16-arrow-right.svg}}}

{ Cite } { Previous page }

\href{/docs/reference/model/emph/}{\pandocbounded{\includesvg[keepaspectratio]{/assets/icons/16-arrow-right.svg}}}

{ Emphasis } { Next page }


\section{Docs LaTeX/typst.app/docs/reference/model/bibliography.tex}
\title{typst.app/docs/reference/model/bibliography}

\begin{itemize}
\tightlist
\item
  \href{/docs}{\includesvg[width=0.16667in,height=0.16667in]{/assets/icons/16-docs-dark.svg}}
\item
  \includesvg[width=0.16667in,height=0.16667in]{/assets/icons/16-arrow-right.svg}
\item
  \href{/docs/reference/}{Reference}
\item
  \includesvg[width=0.16667in,height=0.16667in]{/assets/icons/16-arrow-right.svg}
\item
  \href{/docs/reference/model/}{Model}
\item
  \includesvg[width=0.16667in,height=0.16667in]{/assets/icons/16-arrow-right.svg}
\item
  \href{/docs/reference/model/bibliography/}{Bibliography}
\end{itemize}

\section{\texorpdfstring{\texttt{\ bibliography\ } {{ Element
}}}{ bibliography   Element }}\label{summary}

\phantomsection\label{element-tooltip}
Element functions can be customized with \texttt{\ set\ } and
\texttt{\ show\ } rules.

A bibliography / reference listing.

You can create a new bibliography by calling this function with a path
to a bibliography file in either one of two formats:

\begin{itemize}
\tightlist
\item
  A Hayagriva \texttt{\ .yml\ } file. Hayagriva is a new bibliography
  file format designed for use with Typst. Visit its
  \href{https://github.com/typst/hayagriva/blob/main/docs/file-format.md}{documentation}
  for more details.
\item
  A BibLaTeX \texttt{\ .bib\ } file.
\end{itemize}

As soon as you add a bibliography somewhere in your document, you can
start citing things with reference syntax (
\texttt{\ }{\texttt{\ @key\ }}\texttt{\ } ) or explicit calls to the
\href{/docs/reference/model/cite/}{citation} function (
\texttt{\ }{\texttt{\ \#\ }}\texttt{\ }{\texttt{\ cite\ }}\texttt{\ }{\texttt{\ (\ }}\texttt{\ }{\texttt{\ \textless{}key\textgreater{}\ }}\texttt{\ }{\texttt{\ )\ }}\texttt{\ }
). The bibliography will only show entries for works that were
referenced in the document.

\subsection{Styles}\label{styles}

Typst offers a wide selection of built-in
\href{/docs/reference/model/bibliography/\#parameters-style}{citation
and bibliography styles} . Beyond those, you can add and use custom
\href{https://citationstyles.org/}{CSL} (Citation Style Language) files.
Wondering which style to use? Here are some good defaults based on what
discipline you\textquotesingle re working in:

\begin{longtable}[]{@{}ll@{}}
\toprule\noalign{}
Fields & Typical Styles \\
\midrule\noalign{}
\endhead
\bottomrule\noalign{}
\endlastfoot
Engineering, IT & \texttt{\ }{\texttt{\ "ieee"\ }}\texttt{\ } \\
Psychology, Life Sciences &
\texttt{\ }{\texttt{\ "apa"\ }}\texttt{\ } \\
Social sciences &
\texttt{\ }{\texttt{\ "chicago-author-date"\ }}\texttt{\ } \\
Humanities & \texttt{\ }{\texttt{\ "mla"\ }}\texttt{\ } ,
\texttt{\ }{\texttt{\ "chicago-notes"\ }}\texttt{\ } ,
\texttt{\ }{\texttt{\ "harvard-cite-them-right"\ }}\texttt{\ } \\
Economics &
\texttt{\ }{\texttt{\ "harvard-cite-them-right"\ }}\texttt{\ } \\
Physics &
\texttt{\ }{\texttt{\ "american-physics-society"\ }}\texttt{\ } \\
\end{longtable}

\subsection{Example}\label{example}

\begin{verbatim}
This was already noted by
pirates long ago. @arrgh

Multiple sources say ...
@arrgh @netwok.

#bibliography("works.bib")
\end{verbatim}

\includegraphics[width=5in,height=\textheight,keepaspectratio]{/assets/docs/IJ3xnmEzh6yEddeM44ev3wAAAAAAAAAA.png}

\subsection{\texorpdfstring{{ Parameters
}}{ Parameters }}\label{parameters}

\phantomsection\label{parameters-tooltip}
Parameters are the inputs to a function. They are specified in
parentheses after the function name.

{ bibliography } (

{ \href{/docs/reference/foundations/str/}{str}
\href{/docs/reference/foundations/array/}{array} , } {
\hyperref[parameters-title]{title :}
\href{/docs/reference/foundations/none/}{none}
\href{/docs/reference/foundations/auto/}{auto}
\href{/docs/reference/foundations/content/}{content} , } {
\hyperref[parameters-full]{full :}
\href{/docs/reference/foundations/bool/}{bool} , } {
\hyperref[parameters-style]{style :}
\href{/docs/reference/foundations/str/}{str} , }

) -\textgreater{} \href{/docs/reference/foundations/content/}{content}

\subsubsection{\texorpdfstring{\texttt{\ path\ }}{ path }}\label{parameters-path}

\href{/docs/reference/foundations/str/}{str} {or}
\href{/docs/reference/foundations/array/}{array}

{Required} {{ Positional }}

\phantomsection\label{parameters-path-positional-tooltip}
Positional parameters are specified in order, without names.

Path(s) to Hayagriva \texttt{\ .yml\ } and/or BibLaTeX \texttt{\ .bib\ }
files.

\subsubsection{\texorpdfstring{\texttt{\ title\ }}{ title }}\label{parameters-title}

\href{/docs/reference/foundations/none/}{none} {or}
\href{/docs/reference/foundations/auto/}{auto} {or}
\href{/docs/reference/foundations/content/}{content}

{{ Settable }}

\phantomsection\label{parameters-title-settable-tooltip}
Settable parameters can be customized for all following uses of the
function with a \texttt{\ set\ } rule.

The title of the bibliography.

\begin{itemize}
\tightlist
\item
  When set to \texttt{\ }{\texttt{\ auto\ }}\texttt{\ } , an appropriate
  title for the \href{/docs/reference/text/text/\#parameters-lang}{text
  language} will be used. This is the default.
\item
  When set to \texttt{\ }{\texttt{\ none\ }}\texttt{\ } , the
  bibliography will not have a title.
\item
  A custom title can be set by passing content.
\end{itemize}

The bibliography\textquotesingle s heading will not be numbered by
default, but you can force it to be with a show-set rule:
\texttt{\ }{\texttt{\ show\ }}\texttt{\ }{\texttt{\ bibliography\ }}\texttt{\ }{\texttt{\ :\ }}\texttt{\ }{\texttt{\ set\ }}\texttt{\ }{\texttt{\ heading\ }}\texttt{\ }{\texttt{\ (\ }}\texttt{\ numbering\ }{\texttt{\ :\ }}\texttt{\ }{\texttt{\ "1."\ }}\texttt{\ }{\texttt{\ )\ }}\texttt{\ }

Default: \texttt{\ }{\texttt{\ auto\ }}\texttt{\ }

\subsubsection{\texorpdfstring{\texttt{\ full\ }}{ full }}\label{parameters-full}

\href{/docs/reference/foundations/bool/}{bool}

{{ Settable }}

\phantomsection\label{parameters-full-settable-tooltip}
Settable parameters can be customized for all following uses of the
function with a \texttt{\ set\ } rule.

Whether to include all works from the given bibliography files, even
those that weren\textquotesingle t cited in the document.

To selectively add individual cited works without showing them, you can
also use the \texttt{\ cite\ } function with
\href{/docs/reference/model/cite/\#parameters-form}{\texttt{\ form\ }}
set to \texttt{\ }{\texttt{\ none\ }}\texttt{\ } .

Default: \texttt{\ }{\texttt{\ false\ }}\texttt{\ }

\subsubsection{\texorpdfstring{\texttt{\ style\ }}{ style }}\label{parameters-style}

\href{/docs/reference/foundations/str/}{str}

{{ Settable }}

\phantomsection\label{parameters-style-settable-tooltip}
Settable parameters can be customized for all following uses of the
function with a \texttt{\ set\ } rule.

The bibliography style.

Should be either one of the built-in styles (see below) or a path to a
\href{https://citationstyles.org/}{CSL file} . Some of the styles listed
below appear twice, once with their full name and once with a short
alias.

\includesvg[width=0.16667in,height=0.16667in]{/assets/icons/16-arrow-right.svg}
View options

\begin{longtable}[]{@{}ll@{}}
\toprule\noalign{}
Variant & Details \\
\midrule\noalign{}
\endhead
\bottomrule\noalign{}
\endlastfoot
\texttt{\ "\ alphanumeric\ "\ } & Alphanumeric \\
\texttt{\ "\ american-anthropological-association\ "\ } & American
Anthropological Association \\
\texttt{\ "\ american-chemical-society\ "\ } & American Chemical
Society \\
\texttt{\ "\ american-geophysical-union\ "\ } & American Geophysical
Union \\
\texttt{\ "\ american-institute-of-aeronautics-and-astronautics\ "\ } &
American Institute of Aeronautics and Astronautics \\
\texttt{\ "\ american-institute-of-physics\ "\ } & American Institute of
Physics 4th edition \\
\texttt{\ "\ american-medical-association\ "\ } & American Medical
Association 11th edition \\
\texttt{\ "\ american-meteorological-society\ "\ } & American
Meteorological Society \\
\texttt{\ "\ american-physics-society\ "\ } & American Physical
Society \\
\texttt{\ "\ american-physiological-society\ "\ } & American
Physiological Society \\
\texttt{\ "\ american-political-science-association\ "\ } & American
Political Science Association \\
\texttt{\ "\ american-psychological-association\ "\ } & American
Psychological Association 7th edition \\
\texttt{\ "\ american-society-for-microbiology\ "\ } & American Society
for Microbiology \\
\texttt{\ "\ american-society-of-civil-engineers\ "\ } & American
Society of Civil Engineers \\
\texttt{\ "\ american-society-of-mechanical-engineers\ "\ } & American
Society of Mechanical Engineers \\
\texttt{\ "\ american-sociological-association\ "\ } & American
Sociological Association 6th/7th edition \\
\texttt{\ "\ angewandte-chemie\ "\ } & Angewandte Chemie International
Edition \\
\texttt{\ "\ annual-reviews\ "\ } & Annual Reviews (sorted by order of
appearance) \\
\texttt{\ "\ annual-reviews-author-date\ "\ } & Annual Reviews
(author-date) \\
\texttt{\ "\ associacao-brasileira-de-normas-tecnicas\ "\ } &
Associação Brasileira de Normas Técnicas (Português - Brasil) \\
\texttt{\ "\ association-for-computing-machinery\ "\ } & Association for
Computing Machinery \\
\texttt{\ "\ biomed-central\ "\ } & BioMed Central \\
\texttt{\ "\ bristol-university-press\ "\ } & Bristol University
Press \\
\texttt{\ "\ british-medical-journal\ "\ } & BMJ \\
\texttt{\ "\ cell\ "\ } & Cell \\
\texttt{\ "\ chicago-author-date\ "\ } & Chicago Manual of Style 17th
edition (author-date) \\
\texttt{\ "\ chicago-fullnotes\ "\ } & Chicago Manual of Style 17th
edition (full note) \\
\texttt{\ "\ chicago-notes\ "\ } & Chicago Manual of Style 17th edition
(note) \\
\texttt{\ "\ copernicus\ "\ } & Copernicus Publications \\
\texttt{\ "\ council-of-science-editors\ "\ } & Council of Science
Editors, Citation-Sequence (numeric, brackets) \\
\texttt{\ "\ council-of-science-editors-author-date\ "\ } & Council of
Science Editors, Name-Year (author-date) \\
\texttt{\ "\ current-opinion\ "\ } & Current Opinion journals \\
\texttt{\ "\ deutsche-gesellschaft-für-psychologie\ "\ } & Deutsche
Gesellschaft für Psychologie 5. Auflage (Deutsch) \\
\texttt{\ "\ deutsche-sprache\ "\ } & Deutsche Sprache (Deutsch) \\
\texttt{\ "\ elsevier-harvard\ "\ } & Elsevier - Harvard (with
titles) \\
\texttt{\ "\ elsevier-vancouver\ "\ } & Elsevier - Vancouver \\
\texttt{\ "\ elsevier-with-titles\ "\ } & Elsevier (numeric, with
titles) \\
\texttt{\ "\ frontiers\ "\ } & Frontiers journals \\
\texttt{\ "\ future-medicine\ "\ } & Future Medicine journals \\
\texttt{\ "\ future-science\ "\ } & Future Science Group \\
\texttt{\ "\ gb-7714-2005-numeric\ "\ } & China National Standard GB/T
7714-2005 (numeric, 中æ--‡) \\
\texttt{\ "\ gb-7714-2015-author-date\ "\ } & China National Standard
GB/T 7714-2015 (author-date, 中æ--‡) \\
\texttt{\ "\ gb-7714-2015-note\ "\ } & China National Standard GB/T
7714-2015 (note, 中æ--‡) \\
\texttt{\ "\ gb-7714-2015-numeric\ "\ } & China National Standard GB/T
7714-2015 (numeric, 中æ--‡) \\
\texttt{\ "\ gost-r-705-2008-numeric\ "\ } & Russian GOST R 7.0.5-2008
(numeric) \\
\texttt{\ "\ harvard-cite-them-right\ "\ } & Cite Them Right 12th
edition - Harvard \\
\texttt{\ "\ institute-of-electrical-and-electronics-engineers\ "\ } &
IEEE \\
\texttt{\ "\ institute-of-physics-numeric\ "\ } & Institute of Physics
(numeric) \\
\texttt{\ "\ iso-690-author-date\ "\ } & ISO-690 (author-date,
English) \\
\texttt{\ "\ iso-690-numeric\ "\ } & ISO-690 (numeric, English) \\
\texttt{\ "\ karger\ "\ } & Karger journals \\
\texttt{\ "\ mary-ann-liebert-vancouver\ "\ } & Mary Ann Liebert -
Vancouver \\
\texttt{\ "\ modern-humanities-research-association\ "\ } & Modern
Humanities Research Association 4th edition (note with bibliography) \\
\texttt{\ "\ modern-language-association\ "\ } & Modern Language
Association 9th edition \\
\texttt{\ "\ modern-language-association-8\ "\ } & Modern Language
Association 8th edition \\
\texttt{\ "\ multidisciplinary-digital-publishing-institute\ "\ } &
Multidisciplinary Digital Publishing Institute \\
\texttt{\ "\ nature\ "\ } & Nature \\
\texttt{\ "\ pensoft\ "\ } & Pensoft Journals \\
\texttt{\ "\ public-library-of-science\ "\ } & Public Library of
Science \\
\texttt{\ "\ royal-society-of-chemistry\ "\ } & Royal Society of
Chemistry \\
\texttt{\ "\ sage-vancouver\ "\ } & SAGE - Vancouver \\
\texttt{\ "\ sist02\ "\ } & SIST02 (æ---¥æœ¬èªž) \\
\texttt{\ "\ spie\ "\ } & SPIE journals \\
\texttt{\ "\ springer-basic\ "\ } & Springer - Basic (numeric,
brackets) \\
\texttt{\ "\ springer-basic-author-date\ "\ } & Springer - Basic
(author-date) \\
\texttt{\ "\ springer-fachzeitschriften-medizin-psychologie\ "\ } &
Springer - Fachzeitschriften Medizin Psychologie (Deutsch) \\
\texttt{\ "\ springer-humanities-author-date\ "\ } & Springer -
Humanities (author-date) \\
\texttt{\ "\ springer-lecture-notes-in-computer-science\ "\ } & Springer
- Lecture Notes in Computer Science \\
\texttt{\ "\ springer-mathphys\ "\ } & Springer - MathPhys (numeric,
brackets) \\
\texttt{\ "\ springer-socpsych-author-date\ "\ } & Springer - SocPsych
(author-date) \\
\texttt{\ "\ springer-vancouver\ "\ } & Springer - Vancouver
(brackets) \\
\texttt{\ "\ taylor-and-francis-chicago-author-date\ "\ } & Taylor \&
Francis - Chicago Manual of Style (author-date) \\
\texttt{\ "\ taylor-and-francis-national-library-of-medicine\ "\ } &
Taylor \& Francis - National Library of Medicine \\
\texttt{\ "\ the-institution-of-engineering-and-technology\ "\ } & The
Institution of Engineering and Technology \\
\texttt{\ "\ the-lancet\ "\ } & The Lancet \\
\texttt{\ "\ thieme\ "\ } & Thieme-German (Deutsch) \\
\texttt{\ "\ trends\ "\ } & Trends journals \\
\texttt{\ "\ turabian-author-date\ "\ } & Turabian 9th edition
(author-date) \\
\texttt{\ "\ turabian-fullnote-8\ "\ } & Turabian 8th edition (full
note) \\
\texttt{\ "\ vancouver\ "\ } & Vancouver \\
\texttt{\ "\ vancouver-superscript\ "\ } & Vancouver (superscript) \\
\end{longtable}

Default: \texttt{\ }{\texttt{\ "ieee"\ }}\texttt{\ }

\href{/docs/reference/model/}{\pandocbounded{\includesvg[keepaspectratio]{/assets/icons/16-arrow-right.svg}}}

{ Model } { Previous page }

\href{/docs/reference/model/list/}{\pandocbounded{\includesvg[keepaspectratio]{/assets/icons/16-arrow-right.svg}}}

{ Bullet List } { Next page }


\section{Docs LaTeX/typst.app/docs/reference/model/ref.tex}
\title{typst.app/docs/reference/model/ref}

\begin{itemize}
\tightlist
\item
  \href{/docs}{\includesvg[width=0.16667in,height=0.16667in]{/assets/icons/16-docs-dark.svg}}
\item
  \includesvg[width=0.16667in,height=0.16667in]{/assets/icons/16-arrow-right.svg}
\item
  \href{/docs/reference/}{Reference}
\item
  \includesvg[width=0.16667in,height=0.16667in]{/assets/icons/16-arrow-right.svg}
\item
  \href{/docs/reference/model/}{Model}
\item
  \includesvg[width=0.16667in,height=0.16667in]{/assets/icons/16-arrow-right.svg}
\item
  \href{/docs/reference/model/ref/}{Reference}
\end{itemize}

\section{\texorpdfstring{\texttt{\ ref\ } {{ Element
}}}{ ref   Element }}\label{summary}

\phantomsection\label{element-tooltip}
Element functions can be customized with \texttt{\ set\ } and
\texttt{\ show\ } rules.

A reference to a label or bibliography.

Produces a textual reference to a label. For example, a reference to a
heading will yield an appropriate string such as "Section 1" for a
reference to the first heading. The references are also links to the
respective element. Reference syntax can also be used to
\href{/docs/reference/model/cite/}{cite} from a bibliography.

Referenceable elements include
\href{/docs/reference/model/heading/}{headings} ,
\href{/docs/reference/model/figure/}{figures} ,
\href{/docs/reference/math/equation/}{equations} , and
\href{/docs/reference/model/footnote/}{footnotes} . To create a custom
referenceable element like a theorem, you can create a figure of a
custom
\href{/docs/reference/model/figure/\#parameters-kind}{\texttt{\ kind\ }}
and write a show rule for it. In the future, there might be a more
direct way to define a custom referenceable element.

If you just want to link to a labelled element and not get an automatic
textual reference, consider using the
\href{/docs/reference/model/link/}{\texttt{\ link\ }} function instead.

\subsection{Example}\label{example}

\begin{verbatim}
#set heading(numbering: "1.")
#set math.equation(numbering: "(1)")

= Introduction <intro>
Recent developments in
typesetting software have
rekindled hope in previously
frustrated researchers. @distress
As shown in @results, we ...

= Results <results>
We discuss our approach in
comparison with others.

== Performance <perf>
@slow demonstrates what slow
software looks like.
$ T(n) = O(2^n) $ <slow>

#bibliography("works.bib")
\end{verbatim}

\includegraphics[width=5in,height=\textheight,keepaspectratio]{/assets/docs/bzf3klNJ674BqVarCEGU8wAAAAAAAAAA.png}

\subsection{Syntax}\label{syntax}

This function also has dedicated syntax: A reference to a label can be
created by typing an \texttt{\ @\ } followed by the name of the label
(e.g.
\texttt{\ }{\texttt{\ =\ Introduction\ }}\texttt{\ }{\texttt{\ \textless{}intro\textgreater{}\ }}\texttt{\ }
can be referenced by typing \texttt{\ }{\texttt{\ @intro\ }}\texttt{\ }
).

To customize the supplement, add content in square brackets after the
reference:
\texttt{\ }{\texttt{\ @intro\ }{\texttt{\ {[}\ }}\texttt{\ Chapter\ }{\texttt{\ {]}\ }}\texttt{\ }}\texttt{\ }
.

\subsection{Customization}\label{customization}

If you write a show rule for references, you can access the referenced
element through the \texttt{\ element\ } field of the reference. The
\texttt{\ element\ } may be \texttt{\ }{\texttt{\ none\ }}\texttt{\ }
even if it exists if Typst hasn\textquotesingle t discovered it yet, so
you always need to handle that case in your code.

\begin{verbatim}
#set heading(numbering: "1.")
#set math.equation(numbering: "(1)")

#show ref: it => {
  let eq = math.equation
  let el = it.element
  if el != none and el.func() == eq {
    // Override equation references.
    link(el.location(),numbering(
      el.numbering,
      ..counter(eq).at(el.location())
    ))
  } else {
    // Other references as usual.
    it
  }
}

= Beginnings <beginning>
In @beginning we prove @pythagoras.
$ a^2 + b^2 = c^2 $ <pythagoras>
\end{verbatim}

\includegraphics[width=5in,height=\textheight,keepaspectratio]{/assets/docs/_2kRnAjhpZZ-kvJsytflygAAAAAAAAAA.png}

\subsection{\texorpdfstring{{ Parameters
}}{ Parameters }}\label{parameters}

\phantomsection\label{parameters-tooltip}
Parameters are the inputs to a function. They are specified in
parentheses after the function name.

{ ref } (

{ \href{/docs/reference/foundations/label/}{label} , } {
\hyperref[parameters-supplement]{supplement :}
\href{/docs/reference/foundations/none/}{none}
\href{/docs/reference/foundations/auto/}{auto}
\href{/docs/reference/foundations/content/}{content}
\href{/docs/reference/foundations/function/}{function} , }

) -\textgreater{} \href{/docs/reference/foundations/content/}{content}

\subsubsection{\texorpdfstring{\texttt{\ target\ }}{ target }}\label{parameters-target}

\href{/docs/reference/foundations/label/}{label}

{Required} {{ Positional }}

\phantomsection\label{parameters-target-positional-tooltip}
Positional parameters are specified in order, without names.

The target label that should be referenced.

Can be a label that is defined in the document or an entry from the
\href{/docs/reference/model/bibliography/}{\texttt{\ bibliography\ }} .

\subsubsection{\texorpdfstring{\texttt{\ supplement\ }}{ supplement }}\label{parameters-supplement}

\href{/docs/reference/foundations/none/}{none} {or}
\href{/docs/reference/foundations/auto/}{auto} {or}
\href{/docs/reference/foundations/content/}{content} {or}
\href{/docs/reference/foundations/function/}{function}

{{ Settable }}

\phantomsection\label{parameters-supplement-settable-tooltip}
Settable parameters can be customized for all following uses of the
function with a \texttt{\ set\ } rule.

A supplement for the reference.

For references to headings or figures, this is added before the
referenced number. For citations, this can be used to add a page number.

If a function is specified, it is passed the referenced element and
should return content.

Default: \texttt{\ }{\texttt{\ auto\ }}\texttt{\ }

\includesvg[width=0.16667in,height=0.16667in]{/assets/icons/16-arrow-right.svg}
View example

\begin{verbatim}
#set heading(numbering: "1.")
#set ref(supplement: it => {
  if it.func() == heading {
    "Chapter"
  } else {
    "Thing"
  }
})

= Introduction <intro>
In @intro, we see how to turn
Sections into Chapters. And
in @intro[Part], it is done
manually.
\end{verbatim}

\includegraphics[width=5in,height=\textheight,keepaspectratio]{/assets/docs/fh477CUxS1KmPvq1dqsQ5QAAAAAAAAAA.png}

\href{/docs/reference/model/quote/}{\pandocbounded{\includesvg[keepaspectratio]{/assets/icons/16-arrow-right.svg}}}

{ Quote } { Previous page }

\href{/docs/reference/model/strong/}{\pandocbounded{\includesvg[keepaspectratio]{/assets/icons/16-arrow-right.svg}}}

{ Strong Emphasis } { Next page }


\section{Docs LaTeX/typst.app/docs/reference/model/table.tex}
\title{typst.app/docs/reference/model/table}

\begin{itemize}
\tightlist
\item
  \href{/docs}{\includesvg[width=0.16667in,height=0.16667in]{/assets/icons/16-docs-dark.svg}}
\item
  \includesvg[width=0.16667in,height=0.16667in]{/assets/icons/16-arrow-right.svg}
\item
  \href{/docs/reference/}{Reference}
\item
  \includesvg[width=0.16667in,height=0.16667in]{/assets/icons/16-arrow-right.svg}
\item
  \href{/docs/reference/model/}{Model}
\item
  \includesvg[width=0.16667in,height=0.16667in]{/assets/icons/16-arrow-right.svg}
\item
  \href{/docs/reference/model/table/}{Table}
\end{itemize}

\section{\texorpdfstring{\texttt{\ table\ } {{ Element
}}}{ table   Element }}\label{summary}

\phantomsection\label{element-tooltip}
Element functions can be customized with \texttt{\ set\ } and
\texttt{\ show\ } rules.

A table of items.

Tables are used to arrange content in cells. Cells can contain arbitrary
content, including multiple paragraphs and are specified in row-major
order. For a hands-on explanation of all the ways you can use and
customize tables in Typst, check out the
\href{/docs/guides/table-guide/}{table guide} .

Because tables are just grids with different defaults for some cell
properties (notably \texttt{\ stroke\ } and \texttt{\ inset\ } ), refer
to the \href{/docs/reference/layout/grid/}{grid documentation} for more
information on how to size the table tracks and specify the cell
appearance properties.

If you are unsure whether you should be using a table or a grid,
consider whether the content you are arranging semantically belongs
together as a set of related data points or similar or whether you are
just want to enhance your presentation by arranging unrelated content in
a grid. In the former case, a table is the right choice, while in the
latter case, a grid is more appropriate. Furthermore, Typst will
annotate its output in the future such that screenreaders will announce
content in \texttt{\ table\ } as tabular while a grid\textquotesingle s
content will be announced no different than multiple content blocks in
the document flow.

Note that, to override a particular cell\textquotesingle s properties or
apply show rules on table cells, you can use the
\href{/docs/reference/model/table/\#definitions-cell}{\texttt{\ table.cell\ }}
element. See its documentation for more information.

Although the \texttt{\ table\ } and the \texttt{\ grid\ } share most
properties, set and show rules on one of them do not affect the other.

To give a table a caption and make it
\href{/docs/reference/model/ref/}{referenceable} , put it into a
\href{/docs/reference/model/figure/}{figure} .

\subsection{Example}\label{example}

The example below demonstrates some of the most common table options.

\begin{verbatim}
#table(
  columns: (1fr, auto, auto),
  inset: 10pt,
  align: horizon,
  table.header(
    [], [*Volume*], [*Parameters*],
  ),
  image("cylinder.svg"),
  $ pi h (D^2 - d^2) / 4 $,
  [
    $h$: height \
    $D$: outer radius \
    $d$: inner radius
  ],
  image("tetrahedron.svg"),
  $ sqrt(2) / 12 a^3 $,
  [$a$: edge length]
)
\end{verbatim}

\includegraphics[width=5in,height=\textheight,keepaspectratio]{/assets/docs/KSzjBsOqtudzwvK6Zvp9uwAAAAAAAAAA.png}

Much like with grids, you can use
\href{/docs/reference/model/table/\#definitions-cell}{\texttt{\ table.cell\ }}
to customize the appearance and the position of each cell.

\begin{verbatim}
#set table(
  stroke: none,
  gutter: 0.2em,
  fill: (x, y) =>
    if x == 0 or y == 0 { gray },
  inset: (right: 1.5em),
)

#show table.cell: it => {
  if it.x == 0 or it.y == 0 {
    set text(white)
    strong(it)
  } else if it.body == [] {
    // Replace empty cells with 'N/A'
    pad(..it.inset)[_N/A_]
  } else {
    it
  }
}

#let a = table.cell(
  fill: green.lighten(60%),
)[A]
#let b = table.cell(
  fill: aqua.lighten(60%),
)[B]

#table(
  columns: 4,
  [], [Exam 1], [Exam 2], [Exam 3],

  [John], [], a, [],
  [Mary], [], a, a,
  [Robert], b, a, b,
)
\end{verbatim}

\includegraphics[width=5.66667in,height=\textheight,keepaspectratio]{/assets/docs/D_wYQ9Nqm8ZPq6ssgJwiZQAAAAAAAAAA.png}

\subsection{\texorpdfstring{{ Parameters
}}{ Parameters }}\label{parameters}

\phantomsection\label{parameters-tooltip}
Parameters are the inputs to a function. They are specified in
parentheses after the function name.

{ table } (

{ \hyperref[parameters-columns]{columns :}
\href{/docs/reference/foundations/auto/}{auto}
\href{/docs/reference/foundations/int/}{int}
\href{/docs/reference/layout/relative/}{relative}
\href{/docs/reference/layout/fraction/}{fraction}
\href{/docs/reference/foundations/array/}{array} , } {
\hyperref[parameters-rows]{rows :}
\href{/docs/reference/foundations/auto/}{auto}
\href{/docs/reference/foundations/int/}{int}
\href{/docs/reference/layout/relative/}{relative}
\href{/docs/reference/layout/fraction/}{fraction}
\href{/docs/reference/foundations/array/}{array} , } {
\hyperref[parameters-gutter]{gutter :}
\href{/docs/reference/foundations/auto/}{auto}
\href{/docs/reference/foundations/int/}{int}
\href{/docs/reference/layout/relative/}{relative}
\href{/docs/reference/layout/fraction/}{fraction}
\href{/docs/reference/foundations/array/}{array} , } {
\hyperref[parameters-column-gutter]{column-gutter :}
\href{/docs/reference/foundations/auto/}{auto}
\href{/docs/reference/foundations/int/}{int}
\href{/docs/reference/layout/relative/}{relative}
\href{/docs/reference/layout/fraction/}{fraction}
\href{/docs/reference/foundations/array/}{array} , } {
\hyperref[parameters-row-gutter]{row-gutter :}
\href{/docs/reference/foundations/auto/}{auto}
\href{/docs/reference/foundations/int/}{int}
\href{/docs/reference/layout/relative/}{relative}
\href{/docs/reference/layout/fraction/}{fraction}
\href{/docs/reference/foundations/array/}{array} , } {
\hyperref[parameters-fill]{fill :}
\href{/docs/reference/foundations/none/}{none}
\href{/docs/reference/visualize/color/}{color}
\href{/docs/reference/visualize/gradient/}{gradient}
\href{/docs/reference/foundations/array/}{array}
\href{/docs/reference/visualize/pattern/}{pattern}
\href{/docs/reference/foundations/function/}{function} , } {
\hyperref[parameters-align]{align :}
\href{/docs/reference/foundations/auto/}{auto}
\href{/docs/reference/foundations/array/}{array}
\href{/docs/reference/layout/alignment/}{alignment}
\href{/docs/reference/foundations/function/}{function} , } {
\hyperref[parameters-stroke]{stroke :}
\href{/docs/reference/foundations/none/}{none}
\href{/docs/reference/layout/length/}{length}
\href{/docs/reference/visualize/color/}{color}
\href{/docs/reference/visualize/gradient/}{gradient}
\href{/docs/reference/foundations/array/}{array}
\href{/docs/reference/visualize/stroke/}{stroke}
\href{/docs/reference/visualize/pattern/}{pattern}
\href{/docs/reference/foundations/dictionary/}{dictionary}
\href{/docs/reference/foundations/function/}{function} , } {
\hyperref[parameters-inset]{inset :}
\href{/docs/reference/layout/relative/}{relative}
\href{/docs/reference/foundations/array/}{array}
\href{/docs/reference/foundations/dictionary/}{dictionary}
\href{/docs/reference/foundations/function/}{function} , } {
\hyperref[parameters-children]{..}
\href{/docs/reference/foundations/content/}{content} , }

) -\textgreater{} \href{/docs/reference/foundations/content/}{content}

\subsubsection{\texorpdfstring{\texttt{\ columns\ }}{ columns }}\label{parameters-columns}

\href{/docs/reference/foundations/auto/}{auto} {or}
\href{/docs/reference/foundations/int/}{int} {or}
\href{/docs/reference/layout/relative/}{relative} {or}
\href{/docs/reference/layout/fraction/}{fraction} {or}
\href{/docs/reference/foundations/array/}{array}

{{ Settable }}

\phantomsection\label{parameters-columns-settable-tooltip}
Settable parameters can be customized for all following uses of the
function with a \texttt{\ set\ } rule.

The column sizes. See the \href{/docs/reference/layout/grid/}{grid
documentation} for more information on track sizing.

Default:
\texttt{\ }{\texttt{\ (\ }}\texttt{\ }{\texttt{\ )\ }}\texttt{\ }

\subsubsection{\texorpdfstring{\texttt{\ rows\ }}{ rows }}\label{parameters-rows}

\href{/docs/reference/foundations/auto/}{auto} {or}
\href{/docs/reference/foundations/int/}{int} {or}
\href{/docs/reference/layout/relative/}{relative} {or}
\href{/docs/reference/layout/fraction/}{fraction} {or}
\href{/docs/reference/foundations/array/}{array}

{{ Settable }}

\phantomsection\label{parameters-rows-settable-tooltip}
Settable parameters can be customized for all following uses of the
function with a \texttt{\ set\ } rule.

The row sizes. See the \href{/docs/reference/layout/grid/}{grid
documentation} for more information on track sizing.

Default:
\texttt{\ }{\texttt{\ (\ }}\texttt{\ }{\texttt{\ )\ }}\texttt{\ }

\subsubsection{\texorpdfstring{\texttt{\ gutter\ }}{ gutter }}\label{parameters-gutter}

\href{/docs/reference/foundations/auto/}{auto} {or}
\href{/docs/reference/foundations/int/}{int} {or}
\href{/docs/reference/layout/relative/}{relative} {or}
\href{/docs/reference/layout/fraction/}{fraction} {or}
\href{/docs/reference/foundations/array/}{array}

{{ Settable }}

\phantomsection\label{parameters-gutter-settable-tooltip}
Settable parameters can be customized for all following uses of the
function with a \texttt{\ set\ } rule.

The gaps between rows and columns. This is a shorthand for setting
\texttt{\ column-gutter\ } and \texttt{\ row-gutter\ } to the same
value. See the \href{/docs/reference/layout/grid/}{grid documentation}
for more information on gutters.

Default:
\texttt{\ }{\texttt{\ (\ }}\texttt{\ }{\texttt{\ )\ }}\texttt{\ }

\subsubsection{\texorpdfstring{\texttt{\ column-gutter\ }}{ column-gutter }}\label{parameters-column-gutter}

\href{/docs/reference/foundations/auto/}{auto} {or}
\href{/docs/reference/foundations/int/}{int} {or}
\href{/docs/reference/layout/relative/}{relative} {or}
\href{/docs/reference/layout/fraction/}{fraction} {or}
\href{/docs/reference/foundations/array/}{array}

{{ Settable }}

\phantomsection\label{parameters-column-gutter-settable-tooltip}
Settable parameters can be customized for all following uses of the
function with a \texttt{\ set\ } rule.

The gaps between columns. Takes precedence over \texttt{\ gutter\ } .
See the \href{/docs/reference/layout/grid/}{grid documentation} for more
information on gutters.

Default:
\texttt{\ }{\texttt{\ (\ }}\texttt{\ }{\texttt{\ )\ }}\texttt{\ }

\subsubsection{\texorpdfstring{\texttt{\ row-gutter\ }}{ row-gutter }}\label{parameters-row-gutter}

\href{/docs/reference/foundations/auto/}{auto} {or}
\href{/docs/reference/foundations/int/}{int} {or}
\href{/docs/reference/layout/relative/}{relative} {or}
\href{/docs/reference/layout/fraction/}{fraction} {or}
\href{/docs/reference/foundations/array/}{array}

{{ Settable }}

\phantomsection\label{parameters-row-gutter-settable-tooltip}
Settable parameters can be customized for all following uses of the
function with a \texttt{\ set\ } rule.

The gaps between rows. Takes precedence over \texttt{\ gutter\ } . See
the \href{/docs/reference/layout/grid/}{grid documentation} for more
information on gutters.

Default:
\texttt{\ }{\texttt{\ (\ }}\texttt{\ }{\texttt{\ )\ }}\texttt{\ }

\subsubsection{\texorpdfstring{\texttt{\ fill\ }}{ fill }}\label{parameters-fill}

\href{/docs/reference/foundations/none/}{none} {or}
\href{/docs/reference/visualize/color/}{color} {or}
\href{/docs/reference/visualize/gradient/}{gradient} {or}
\href{/docs/reference/foundations/array/}{array} {or}
\href{/docs/reference/visualize/pattern/}{pattern} {or}
\href{/docs/reference/foundations/function/}{function}

{{ Settable }}

\phantomsection\label{parameters-fill-settable-tooltip}
Settable parameters can be customized for all following uses of the
function with a \texttt{\ set\ } rule.

How to fill the cells.

This can be a color or a function that returns a color. The function
receives the cells\textquotesingle{} column and row indices, starting
from zero. This can be used to implement striped tables.

Default: \texttt{\ }{\texttt{\ none\ }}\texttt{\ }

\includesvg[width=0.16667in,height=0.16667in]{/assets/icons/16-arrow-right.svg}
View example

\begin{verbatim}
#table(
  fill: (x, _) =>
    if calc.odd(x) { luma(240) }
    else { white },
  align: (x, y) =>
    if y == 0 { center }
    else if x == 0 { left }
    else { right },
  columns: 4,
  [], [*Q1*], [*Q2*], [*Q3*],
  [Revenue:], [1000 €], [2000 €], [3000 €],
  [Expenses:], [500 €], [1000 €], [1500 €],
  [Profit:], [500 €], [1000 €], [1500 €],
)
\end{verbatim}

\includegraphics[width=5in,height=\textheight,keepaspectratio]{/assets/docs/HObhPJHvYkiYqHCjRK1JHwAAAAAAAAAA.png}

\subsubsection{\texorpdfstring{\texttt{\ align\ }}{ align }}\label{parameters-align}

\href{/docs/reference/foundations/auto/}{auto} {or}
\href{/docs/reference/foundations/array/}{array} {or}
\href{/docs/reference/layout/alignment/}{alignment} {or}
\href{/docs/reference/foundations/function/}{function}

{{ Settable }}

\phantomsection\label{parameters-align-settable-tooltip}
Settable parameters can be customized for all following uses of the
function with a \texttt{\ set\ } rule.

How to align the cells\textquotesingle{} content.

This can either be a single alignment, an array of alignments
(corresponding to each column) or a function that returns an alignment.
The function receives the cells\textquotesingle{} column and row
indices, starting from zero. If set to
\texttt{\ }{\texttt{\ auto\ }}\texttt{\ } , the outer alignment is used.

Default: \texttt{\ }{\texttt{\ auto\ }}\texttt{\ }

\includesvg[width=0.16667in,height=0.16667in]{/assets/icons/16-arrow-right.svg}
View example

\begin{verbatim}
#table(
  columns: 3,
  align: (left, center, right),
  [Hello], [Hello], [Hello],
  [A], [B], [C],
)
\end{verbatim}

\includegraphics[width=5in,height=\textheight,keepaspectratio]{/assets/docs/_fBgotCl-LtVjvGU4yJFLQAAAAAAAAAA.png}

\subsubsection{\texorpdfstring{\texttt{\ stroke\ }}{ stroke }}\label{parameters-stroke}

\href{/docs/reference/foundations/none/}{none} {or}
\href{/docs/reference/layout/length/}{length} {or}
\href{/docs/reference/visualize/color/}{color} {or}
\href{/docs/reference/visualize/gradient/}{gradient} {or}
\href{/docs/reference/foundations/array/}{array} {or}
\href{/docs/reference/visualize/stroke/}{stroke} {or}
\href{/docs/reference/visualize/pattern/}{pattern} {or}
\href{/docs/reference/foundations/dictionary/}{dictionary} {or}
\href{/docs/reference/foundations/function/}{function}

{{ Settable }}

\phantomsection\label{parameters-stroke-settable-tooltip}
Settable parameters can be customized for all following uses of the
function with a \texttt{\ set\ } rule.

How to \href{/docs/reference/visualize/stroke/}{stroke} the cells.

Strokes can be disabled by setting this to
\texttt{\ }{\texttt{\ none\ }}\texttt{\ } .

If it is necessary to place lines which can cross spacing between cells
produced by the \texttt{\ gutter\ } option, or to override the stroke
between multiple specific cells, consider specifying one or more of
\href{/docs/reference/model/table/\#definitions-hline}{\texttt{\ table.hline\ }}
and
\href{/docs/reference/model/table/\#definitions-vline}{\texttt{\ table.vline\ }}
alongside your table cells.

See the \href{/docs/reference/layout/grid/\#parameters-stroke}{grid
documentation} for more information on strokes.

Default:
\texttt{\ }{\texttt{\ 1pt\ }}\texttt{\ }{\texttt{\ +\ }}\texttt{\ black\ }

\subsubsection{\texorpdfstring{\texttt{\ inset\ }}{ inset }}\label{parameters-inset}

\href{/docs/reference/layout/relative/}{relative} {or}
\href{/docs/reference/foundations/array/}{array} {or}
\href{/docs/reference/foundations/dictionary/}{dictionary} {or}
\href{/docs/reference/foundations/function/}{function}

{{ Settable }}

\phantomsection\label{parameters-inset-settable-tooltip}
Settable parameters can be customized for all following uses of the
function with a \texttt{\ set\ } rule.

How much to pad the cells\textquotesingle{} content.

Default:
\texttt{\ }{\texttt{\ 0\%\ }}\texttt{\ }{\texttt{\ +\ }}\texttt{\ }{\texttt{\ 5pt\ }}\texttt{\ }

\includesvg[width=0.16667in,height=0.16667in]{/assets/icons/16-arrow-right.svg}
View example

\begin{verbatim}
#table(
  inset: 10pt,
  [Hello],
  [World],
)

#table(
  columns: 2,
  inset: (
    x: 20pt,
    y: 10pt,
  ),
  [Hello],
  [World],
)
\end{verbatim}

\includegraphics[width=5in,height=\textheight,keepaspectratio]{/assets/docs/f1kE1ENTTB02iZKKPoV_XwAAAAAAAAAA.png}

\subsubsection{\texorpdfstring{\texttt{\ children\ }}{ children }}\label{parameters-children}

\href{/docs/reference/foundations/content/}{content}

{Required} {{ Positional }}

\phantomsection\label{parameters-children-positional-tooltip}
Positional parameters are specified in order, without names.

{{ Variadic }}

\phantomsection\label{parameters-children-variadic-tooltip}
Variadic parameters can be specified multiple times.

The contents of the table cells, plus any extra table lines specified
with the
\href{/docs/reference/model/table/\#definitions-hline}{\texttt{\ table.hline\ }}
and
\href{/docs/reference/model/table/\#definitions-vline}{\texttt{\ table.vline\ }}
elements.

\subsection{\texorpdfstring{{ Definitions
}}{ Definitions }}\label{definitions}

\phantomsection\label{definitions-tooltip}
Functions and types and can have associated definitions. These are
accessed by specifying the function or type, followed by a period, and
then the definition\textquotesingle s name.

\subsubsection{\texorpdfstring{\texttt{\ cell\ } {{ Element
}}}{ cell   Element }}\label{definitions-cell}

\phantomsection\label{definitions-cell-element-tooltip}
Element functions can be customized with \texttt{\ set\ } and
\texttt{\ show\ } rules.

A cell in the table. Use this to position a cell manually or to apply
styling. To do the latter, you can either use the function to override
the properties for a particular cell, or use it in show rules to apply
certain styles to multiple cells at once.

Perhaps the most important use case of
\texttt{\ table\ }{\texttt{\ .\ }}\texttt{\ cell\ } is to make a cell
span multiple columns and/or rows with the \texttt{\ colspan\ } and
\texttt{\ rowspan\ } fields.

table { . } { cell } (

{ \href{/docs/reference/foundations/content/}{content} , } {
\hyperref[definitions-cell-parameters-x]{x :}
\href{/docs/reference/foundations/auto/}{auto}
\href{/docs/reference/foundations/int/}{int} , } {
\hyperref[definitions-cell-parameters-y]{y :}
\href{/docs/reference/foundations/auto/}{auto}
\href{/docs/reference/foundations/int/}{int} , } {
\hyperref[definitions-cell-parameters-colspan]{colspan :}
\href{/docs/reference/foundations/int/}{int} , } {
\hyperref[definitions-cell-parameters-rowspan]{rowspan :}
\href{/docs/reference/foundations/int/}{int} , } {
\hyperref[definitions-cell-parameters-fill]{fill :}
\href{/docs/reference/foundations/none/}{none}
\href{/docs/reference/foundations/auto/}{auto}
\href{/docs/reference/visualize/color/}{color}
\href{/docs/reference/visualize/gradient/}{gradient}
\href{/docs/reference/visualize/pattern/}{pattern} , } {
\hyperref[definitions-cell-parameters-align]{align :}
\href{/docs/reference/foundations/auto/}{auto}
\href{/docs/reference/layout/alignment/}{alignment} , } {
\hyperref[definitions-cell-parameters-inset]{inset :}
\href{/docs/reference/foundations/auto/}{auto}
\href{/docs/reference/layout/relative/}{relative}
\href{/docs/reference/foundations/dictionary/}{dictionary} , } {
\hyperref[definitions-cell-parameters-stroke]{stroke :}
\href{/docs/reference/foundations/none/}{none}
\href{/docs/reference/layout/length/}{length}
\href{/docs/reference/visualize/color/}{color}
\href{/docs/reference/visualize/gradient/}{gradient}
\href{/docs/reference/visualize/stroke/}{stroke}
\href{/docs/reference/visualize/pattern/}{pattern}
\href{/docs/reference/foundations/dictionary/}{dictionary} , } {
\hyperref[definitions-cell-parameters-breakable]{breakable :}
\href{/docs/reference/foundations/auto/}{auto}
\href{/docs/reference/foundations/bool/}{bool} , }

) -\textgreater{} \href{/docs/reference/foundations/content/}{content}

\begin{verbatim}
#show table.cell.where(y: 0): strong
#set table(
  stroke: (x, y) => if y == 0 {
    (bottom: 0.7pt + black)
  },
  align: (x, y) => (
    if x > 0 { center }
    else { left }
  )
)

#table(
  columns: 3,
  table.header(
    [Substance],
    [Subcritical °C],
    [Supercritical °C],
  ),
  [Hydrochloric Acid],
  [12.0], [92.1],
  [Sodium Myreth Sulfate],
  [16.6], [104],
  [Potassium Hydroxide],
  table.cell(colspan: 2)[24.7],
)
\end{verbatim}

\includegraphics[width=6.39583in,height=\textheight,keepaspectratio]{/assets/docs/2rQPm8gbRwbFqiITJlD6oAAAAAAAAAAA.png}

For example, you can override the fill, alignment or inset for a single
cell:

\begin{verbatim}
// You can also import those.
#import table: cell, header

#table(
  columns: 2,
  align: center,
  header(
    [*Trip progress*],
    [*Itinerary*],
  ),
  cell(
    align: right,
    fill: fuchsia.lighten(80%),
    [🚗],
  ),
  [Get in, folks!],
  [🚗], [Eat curbside hotdog],
  cell(align: left)[🌴🚗],
  cell(
    inset: 0.06em,
    text(1.62em)[🛖🌅🌊],
  ),
)
\end{verbatim}

\includegraphics[width=4.29167in,height=\textheight,keepaspectratio]{/assets/docs/VtayZlhMrUWzOmBAyEorDQAAAAAAAAAA.png}

You may also apply a show rule on \texttt{\ table.cell\ } to style all
cells at once. Combined with selectors, this allows you to apply styles
based on a cell\textquotesingle s position:

\begin{verbatim}
#show table.cell.where(x: 0): strong

#table(
  columns: 3,
  gutter: 3pt,
  [Name], [Age], [Strength],
  [Hannes], [36], [Grace],
  [Irma], [50], [Resourcefulness],
  [Vikram], [49], [Perseverance],
)
\end{verbatim}

\includegraphics[width=5in,height=\textheight,keepaspectratio]{/assets/docs/c2SP069qvMBzeFbrjVs8pwAAAAAAAAAA.png}

\paragraph{\texorpdfstring{\texttt{\ body\ }}{ body }}\label{definitions-cell-body}

\href{/docs/reference/foundations/content/}{content}

{Required} {{ Positional }}

\phantomsection\label{definitions-cell-body-positional-tooltip}
Positional parameters are specified in order, without names.

The cell\textquotesingle s body.

\paragraph{\texorpdfstring{\texttt{\ x\ }}{ x }}\label{definitions-cell-x}

\href{/docs/reference/foundations/auto/}{auto} {or}
\href{/docs/reference/foundations/int/}{int}

{{ Settable }}

\phantomsection\label{definitions-cell-x-settable-tooltip}
Settable parameters can be customized for all following uses of the
function with a \texttt{\ set\ } rule.

The cell\textquotesingle s column (zero-indexed). Functions identically
to the \texttt{\ x\ } field in
\href{/docs/reference/layout/grid/\#definitions-cell}{\texttt{\ grid.cell\ }}
.

Default: \texttt{\ }{\texttt{\ auto\ }}\texttt{\ }

\paragraph{\texorpdfstring{\texttt{\ y\ }}{ y }}\label{definitions-cell-y}

\href{/docs/reference/foundations/auto/}{auto} {or}
\href{/docs/reference/foundations/int/}{int}

{{ Settable }}

\phantomsection\label{definitions-cell-y-settable-tooltip}
Settable parameters can be customized for all following uses of the
function with a \texttt{\ set\ } rule.

The cell\textquotesingle s row (zero-indexed). Functions identically to
the \texttt{\ y\ } field in
\href{/docs/reference/layout/grid/\#definitions-cell}{\texttt{\ grid.cell\ }}
.

Default: \texttt{\ }{\texttt{\ auto\ }}\texttt{\ }

\paragraph{\texorpdfstring{\texttt{\ colspan\ }}{ colspan }}\label{definitions-cell-colspan}

\href{/docs/reference/foundations/int/}{int}

{{ Settable }}

\phantomsection\label{definitions-cell-colspan-settable-tooltip}
Settable parameters can be customized for all following uses of the
function with a \texttt{\ set\ } rule.

The amount of columns spanned by this cell.

Default: \texttt{\ }{\texttt{\ 1\ }}\texttt{\ }

\paragraph{\texorpdfstring{\texttt{\ rowspan\ }}{ rowspan }}\label{definitions-cell-rowspan}

\href{/docs/reference/foundations/int/}{int}

{{ Settable }}

\phantomsection\label{definitions-cell-rowspan-settable-tooltip}
Settable parameters can be customized for all following uses of the
function with a \texttt{\ set\ } rule.

The amount of rows spanned by this cell.

Default: \texttt{\ }{\texttt{\ 1\ }}\texttt{\ }

\paragraph{\texorpdfstring{\texttt{\ fill\ }}{ fill }}\label{definitions-cell-fill}

\href{/docs/reference/foundations/none/}{none} {or}
\href{/docs/reference/foundations/auto/}{auto} {or}
\href{/docs/reference/visualize/color/}{color} {or}
\href{/docs/reference/visualize/gradient/}{gradient} {or}
\href{/docs/reference/visualize/pattern/}{pattern}

{{ Settable }}

\phantomsection\label{definitions-cell-fill-settable-tooltip}
Settable parameters can be customized for all following uses of the
function with a \texttt{\ set\ } rule.

The cell\textquotesingle s
\href{/docs/reference/model/table/\#parameters-fill}{fill} override.

Default: \texttt{\ }{\texttt{\ auto\ }}\texttt{\ }

\paragraph{\texorpdfstring{\texttt{\ align\ }}{ align }}\label{definitions-cell-align}

\href{/docs/reference/foundations/auto/}{auto} {or}
\href{/docs/reference/layout/alignment/}{alignment}

{{ Settable }}

\phantomsection\label{definitions-cell-align-settable-tooltip}
Settable parameters can be customized for all following uses of the
function with a \texttt{\ set\ } rule.

The cell\textquotesingle s
\href{/docs/reference/model/table/\#parameters-align}{alignment}
override.

Default: \texttt{\ }{\texttt{\ auto\ }}\texttt{\ }

\paragraph{\texorpdfstring{\texttt{\ inset\ }}{ inset }}\label{definitions-cell-inset}

\href{/docs/reference/foundations/auto/}{auto} {or}
\href{/docs/reference/layout/relative/}{relative} {or}
\href{/docs/reference/foundations/dictionary/}{dictionary}

{{ Settable }}

\phantomsection\label{definitions-cell-inset-settable-tooltip}
Settable parameters can be customized for all following uses of the
function with a \texttt{\ set\ } rule.

The cell\textquotesingle s
\href{/docs/reference/model/table/\#parameters-inset}{inset} override.

Default: \texttt{\ }{\texttt{\ auto\ }}\texttt{\ }

\paragraph{\texorpdfstring{\texttt{\ stroke\ }}{ stroke }}\label{definitions-cell-stroke}

\href{/docs/reference/foundations/none/}{none} {or}
\href{/docs/reference/layout/length/}{length} {or}
\href{/docs/reference/visualize/color/}{color} {or}
\href{/docs/reference/visualize/gradient/}{gradient} {or}
\href{/docs/reference/visualize/stroke/}{stroke} {or}
\href{/docs/reference/visualize/pattern/}{pattern} {or}
\href{/docs/reference/foundations/dictionary/}{dictionary}

{{ Settable }}

\phantomsection\label{definitions-cell-stroke-settable-tooltip}
Settable parameters can be customized for all following uses of the
function with a \texttt{\ set\ } rule.

The cell\textquotesingle s
\href{/docs/reference/model/table/\#parameters-stroke}{stroke} override.

Default:
\texttt{\ }{\texttt{\ (\ }}\texttt{\ }{\texttt{\ :\ }}\texttt{\ }{\texttt{\ )\ }}\texttt{\ }

\paragraph{\texorpdfstring{\texttt{\ breakable\ }}{ breakable }}\label{definitions-cell-breakable}

\href{/docs/reference/foundations/auto/}{auto} {or}
\href{/docs/reference/foundations/bool/}{bool}

{{ Settable }}

\phantomsection\label{definitions-cell-breakable-settable-tooltip}
Settable parameters can be customized for all following uses of the
function with a \texttt{\ set\ } rule.

Whether rows spanned by this cell can be placed in different pages. When
equal to \texttt{\ }{\texttt{\ auto\ }}\texttt{\ } , a cell spanning
only fixed-size rows is unbreakable, while a cell spanning at least one
\texttt{\ }{\texttt{\ auto\ }}\texttt{\ } -sized row is breakable.

Default: \texttt{\ }{\texttt{\ auto\ }}\texttt{\ }

\subsubsection{\texorpdfstring{\texttt{\ hline\ } {{ Element
}}}{ hline   Element }}\label{definitions-hline}

\phantomsection\label{definitions-hline-element-tooltip}
Element functions can be customized with \texttt{\ set\ } and
\texttt{\ show\ } rules.

A horizontal line in the table.

Overrides any per-cell stroke, including stroke specified through the
table\textquotesingle s \texttt{\ stroke\ } field. Can cross spacing
between cells created through the table\textquotesingle s
\href{/docs/reference/model/table/\#parameters-column-gutter}{\texttt{\ column-gutter\ }}
option.

Use this function instead of the table\textquotesingle s
\texttt{\ stroke\ } field if you want to manually place a horizontal
line at a specific position in a single table. Consider using
\href{/docs/reference/model/table/\#parameters-stroke}{table\textquotesingle s
\texttt{\ stroke\ }} field or
\href{/docs/reference/model/table/\#definitions-cell-stroke}{\texttt{\ table.cell\ }
\textquotesingle s \texttt{\ stroke\ }} field instead if the line you
want to place is part of all your tables\textquotesingle{} designs.

table { . } { hline } (

{ \hyperref[definitions-hline-parameters-y]{y :}
\href{/docs/reference/foundations/auto/}{auto}
\href{/docs/reference/foundations/int/}{int} , } {
\hyperref[definitions-hline-parameters-start]{start :}
\href{/docs/reference/foundations/int/}{int} , } {
\hyperref[definitions-hline-parameters-end]{end :}
\href{/docs/reference/foundations/none/}{none}
\href{/docs/reference/foundations/int/}{int} , } {
\hyperref[definitions-hline-parameters-stroke]{stroke :}
\href{/docs/reference/foundations/none/}{none}
\href{/docs/reference/layout/length/}{length}
\href{/docs/reference/visualize/color/}{color}
\href{/docs/reference/visualize/gradient/}{gradient}
\href{/docs/reference/visualize/stroke/}{stroke}
\href{/docs/reference/visualize/pattern/}{pattern}
\href{/docs/reference/foundations/dictionary/}{dictionary} , } {
\hyperref[definitions-hline-parameters-position]{position :}
\href{/docs/reference/layout/alignment/}{alignment} , }

) -\textgreater{} \href{/docs/reference/foundations/content/}{content}

\begin{verbatim}
#set table.hline(stroke: .6pt)

#table(
  stroke: none,
  columns: (auto, 1fr),
  [09:00], [Badge pick up],
  [09:45], [Opening Keynote],
  [10:30], [Talk: Typst's Future],
  [11:15], [Session: Good PRs],
  table.hline(start: 1),
  [Noon], [_Lunch break_],
  table.hline(start: 1),
  [14:00], [Talk: Tracked Layout],
  [15:00], [Talk: Automations],
  [16:00], [Workshop: Tables],
  table.hline(),
  [19:00], [Day 1 Attendee Mixer],
)
\end{verbatim}

\includegraphics[width=5in,height=\textheight,keepaspectratio]{/assets/docs/Fl-W72wh8hlKb72YjlJ0PgAAAAAAAAAA.png}

\paragraph{\texorpdfstring{\texttt{\ y\ }}{ y }}\label{definitions-hline-y}

\href{/docs/reference/foundations/auto/}{auto} {or}
\href{/docs/reference/foundations/int/}{int}

{{ Settable }}

\phantomsection\label{definitions-hline-y-settable-tooltip}
Settable parameters can be customized for all following uses of the
function with a \texttt{\ set\ } rule.

The row above which the horizontal line is placed (zero-indexed).
Functions identically to the \texttt{\ y\ } field in
\href{/docs/reference/layout/grid/\#definitions-hline-y}{\texttt{\ grid.hline\ }}
.

Default: \texttt{\ }{\texttt{\ auto\ }}\texttt{\ }

\paragraph{\texorpdfstring{\texttt{\ start\ }}{ start }}\label{definitions-hline-start}

\href{/docs/reference/foundations/int/}{int}

{{ Settable }}

\phantomsection\label{definitions-hline-start-settable-tooltip}
Settable parameters can be customized for all following uses of the
function with a \texttt{\ set\ } rule.

The column at which the horizontal line starts (zero-indexed,
inclusive).

Default: \texttt{\ }{\texttt{\ 0\ }}\texttt{\ }

\paragraph{\texorpdfstring{\texttt{\ end\ }}{ end }}\label{definitions-hline-end}

\href{/docs/reference/foundations/none/}{none} {or}
\href{/docs/reference/foundations/int/}{int}

{{ Settable }}

\phantomsection\label{definitions-hline-end-settable-tooltip}
Settable parameters can be customized for all following uses of the
function with a \texttt{\ set\ } rule.

The column before which the horizontal line ends (zero-indexed,
exclusive).

Default: \texttt{\ }{\texttt{\ none\ }}\texttt{\ }

\paragraph{\texorpdfstring{\texttt{\ stroke\ }}{ stroke }}\label{definitions-hline-stroke}

\href{/docs/reference/foundations/none/}{none} {or}
\href{/docs/reference/layout/length/}{length} {or}
\href{/docs/reference/visualize/color/}{color} {or}
\href{/docs/reference/visualize/gradient/}{gradient} {or}
\href{/docs/reference/visualize/stroke/}{stroke} {or}
\href{/docs/reference/visualize/pattern/}{pattern} {or}
\href{/docs/reference/foundations/dictionary/}{dictionary}

{{ Settable }}

\phantomsection\label{definitions-hline-stroke-settable-tooltip}
Settable parameters can be customized for all following uses of the
function with a \texttt{\ set\ } rule.

The line\textquotesingle s stroke.

Specifying \texttt{\ }{\texttt{\ none\ }}\texttt{\ } removes any lines
previously placed across this line\textquotesingle s range, including
hlines or per-cell stroke below it.

Default:
\texttt{\ }{\texttt{\ 1pt\ }}\texttt{\ }{\texttt{\ +\ }}\texttt{\ black\ }

\paragraph{\texorpdfstring{\texttt{\ position\ }}{ position }}\label{definitions-hline-position}

\href{/docs/reference/layout/alignment/}{alignment}

{{ Settable }}

\phantomsection\label{definitions-hline-position-settable-tooltip}
Settable parameters can be customized for all following uses of the
function with a \texttt{\ set\ } rule.

The position at which the line is placed, given its row ( \texttt{\ y\ }
) - either \texttt{\ top\ } to draw above it or \texttt{\ bottom\ } to
draw below it.

This setting is only relevant when row gutter is enabled (and
shouldn\textquotesingle t be used otherwise - prefer just increasing the
\texttt{\ y\ } field by one instead), since then the position below a
row becomes different from the position above the next row due to the
spacing between both.

Default: \texttt{\ top\ }

\subsubsection{\texorpdfstring{\texttt{\ vline\ } {{ Element
}}}{ vline   Element }}\label{definitions-vline}

\phantomsection\label{definitions-vline-element-tooltip}
Element functions can be customized with \texttt{\ set\ } and
\texttt{\ show\ } rules.

A vertical line in the table. See the docs for
\href{/docs/reference/layout/grid/\#definitions-vline}{\texttt{\ grid.vline\ }}
for more information regarding how to use this element\textquotesingle s
fields.

Overrides any per-cell stroke, including stroke specified through the
table\textquotesingle s \texttt{\ stroke\ } field. Can cross spacing
between cells created through the table\textquotesingle s
\href{/docs/reference/model/table/\#parameters-row-gutter}{\texttt{\ row-gutter\ }}
option.

Similar to
\href{/docs/reference/model/table/\#definitions-hline}{\texttt{\ table.hline\ }}
, use this function if you want to manually place a vertical line at a
specific position in a single table and use the
\href{/docs/reference/model/table/\#parameters-stroke}{table\textquotesingle s
\texttt{\ stroke\ }} field or
\href{/docs/reference/model/table/\#definitions-cell-stroke}{\texttt{\ table.cell\ }
\textquotesingle s \texttt{\ stroke\ }} field instead if the line you
want to place is part of all your tables\textquotesingle{} designs.

table { . } { vline } (

{ \hyperref[definitions-vline-parameters-x]{x :}
\href{/docs/reference/foundations/auto/}{auto}
\href{/docs/reference/foundations/int/}{int} , } {
\hyperref[definitions-vline-parameters-start]{start :}
\href{/docs/reference/foundations/int/}{int} , } {
\hyperref[definitions-vline-parameters-end]{end :}
\href{/docs/reference/foundations/none/}{none}
\href{/docs/reference/foundations/int/}{int} , } {
\hyperref[definitions-vline-parameters-stroke]{stroke :}
\href{/docs/reference/foundations/none/}{none}
\href{/docs/reference/layout/length/}{length}
\href{/docs/reference/visualize/color/}{color}
\href{/docs/reference/visualize/gradient/}{gradient}
\href{/docs/reference/visualize/stroke/}{stroke}
\href{/docs/reference/visualize/pattern/}{pattern}
\href{/docs/reference/foundations/dictionary/}{dictionary} , } {
\hyperref[definitions-vline-parameters-position]{position :}
\href{/docs/reference/layout/alignment/}{alignment} , }

) -\textgreater{} \href{/docs/reference/foundations/content/}{content}

\paragraph{\texorpdfstring{\texttt{\ x\ }}{ x }}\label{definitions-vline-x}

\href{/docs/reference/foundations/auto/}{auto} {or}
\href{/docs/reference/foundations/int/}{int}

{{ Settable }}

\phantomsection\label{definitions-vline-x-settable-tooltip}
Settable parameters can be customized for all following uses of the
function with a \texttt{\ set\ } rule.

The column before which the horizontal line is placed (zero-indexed).
Functions identically to the \texttt{\ x\ } field in
\href{/docs/reference/layout/grid/\#definitions-vline}{\texttt{\ grid.vline\ }}
.

Default: \texttt{\ }{\texttt{\ auto\ }}\texttt{\ }

\paragraph{\texorpdfstring{\texttt{\ start\ }}{ start }}\label{definitions-vline-start}

\href{/docs/reference/foundations/int/}{int}

{{ Settable }}

\phantomsection\label{definitions-vline-start-settable-tooltip}
Settable parameters can be customized for all following uses of the
function with a \texttt{\ set\ } rule.

The row at which the vertical line starts (zero-indexed, inclusive).

Default: \texttt{\ }{\texttt{\ 0\ }}\texttt{\ }

\paragraph{\texorpdfstring{\texttt{\ end\ }}{ end }}\label{definitions-vline-end}

\href{/docs/reference/foundations/none/}{none} {or}
\href{/docs/reference/foundations/int/}{int}

{{ Settable }}

\phantomsection\label{definitions-vline-end-settable-tooltip}
Settable parameters can be customized for all following uses of the
function with a \texttt{\ set\ } rule.

The row on top of which the vertical line ends (zero-indexed,
exclusive).

Default: \texttt{\ }{\texttt{\ none\ }}\texttt{\ }

\paragraph{\texorpdfstring{\texttt{\ stroke\ }}{ stroke }}\label{definitions-vline-stroke}

\href{/docs/reference/foundations/none/}{none} {or}
\href{/docs/reference/layout/length/}{length} {or}
\href{/docs/reference/visualize/color/}{color} {or}
\href{/docs/reference/visualize/gradient/}{gradient} {or}
\href{/docs/reference/visualize/stroke/}{stroke} {or}
\href{/docs/reference/visualize/pattern/}{pattern} {or}
\href{/docs/reference/foundations/dictionary/}{dictionary}

{{ Settable }}

\phantomsection\label{definitions-vline-stroke-settable-tooltip}
Settable parameters can be customized for all following uses of the
function with a \texttt{\ set\ } rule.

The line\textquotesingle s stroke.

Specifying \texttt{\ }{\texttt{\ none\ }}\texttt{\ } removes any lines
previously placed across this line\textquotesingle s range, including
vlines or per-cell stroke below it.

Default:
\texttt{\ }{\texttt{\ 1pt\ }}\texttt{\ }{\texttt{\ +\ }}\texttt{\ black\ }

\paragraph{\texorpdfstring{\texttt{\ position\ }}{ position }}\label{definitions-vline-position}

\href{/docs/reference/layout/alignment/}{alignment}

{{ Settable }}

\phantomsection\label{definitions-vline-position-settable-tooltip}
Settable parameters can be customized for all following uses of the
function with a \texttt{\ set\ } rule.

The position at which the line is placed, given its column (
\texttt{\ x\ } ) - either \texttt{\ start\ } to draw before it or
\texttt{\ end\ } to draw after it.

The values \texttt{\ left\ } and \texttt{\ right\ } are also accepted,
but discouraged as they cause your table to be inconsistent between
left-to-right and right-to-left documents.

This setting is only relevant when column gutter is enabled (and
shouldn\textquotesingle t be used otherwise - prefer just increasing the
\texttt{\ x\ } field by one instead), since then the position after a
column becomes different from the position before the next column due to
the spacing between both.

Default: \texttt{\ start\ }

\subsubsection{\texorpdfstring{\texttt{\ header\ } {{ Element
}}}{ header   Element }}\label{definitions-header}

\phantomsection\label{definitions-header-element-tooltip}
Element functions can be customized with \texttt{\ set\ } and
\texttt{\ show\ } rules.

A repeatable table header.

You should wrap your tables\textquotesingle{} heading rows in this
function even if you do not plan to wrap your table across pages because
Typst will use this function to attach accessibility metadata to tables
in the future and ensure universal access to your document.

You can use the \texttt{\ repeat\ } parameter to control whether your
table\textquotesingle s header will be repeated across pages.

table { . } { header } (

{ \hyperref[definitions-header-parameters-repeat]{repeat :}
\href{/docs/reference/foundations/bool/}{bool} , } {
\hyperref[definitions-header-parameters-children]{..}
\href{/docs/reference/foundations/content/}{content} , }

) -\textgreater{} \href{/docs/reference/foundations/content/}{content}

\begin{verbatim}
#set page(height: 11.5em)
#set table(
  fill: (x, y) =>
    if x == 0 or y == 0 {
      gray.lighten(40%)
    },
  align: right,
)

#show table.cell.where(x: 0): strong
#show table.cell.where(y: 0): strong

#table(
  columns: 4,
  table.header(
    [], [Blue chip],
    [Fresh IPO], [Penny st'k],
  ),
  table.cell(
    rowspan: 6,
    align: horizon,
    rotate(-90deg, reflow: true)[
      *USD / day*
    ],
  ),
  [0.20], [104], [5],
  [3.17], [108], [4],
  [1.59], [84],  [1],
  [0.26], [98],  [15],
  [0.01], [195], [4],
  [7.34], [57],  [2],
)
\end{verbatim}

\includegraphics[width=5in,height=\textheight,keepaspectratio]{/assets/docs/IHpzp-b7mQ7ctAllSxEWfQAAAAAAAAAA.png}
\includegraphics[width=5in,height=\textheight,keepaspectratio]{/assets/docs/IHpzp-b7mQ7ctAllSxEWfQAAAAAAAAAB.png}

\paragraph{\texorpdfstring{\texttt{\ repeat\ }}{ repeat }}\label{definitions-header-repeat}

\href{/docs/reference/foundations/bool/}{bool}

{{ Settable }}

\phantomsection\label{definitions-header-repeat-settable-tooltip}
Settable parameters can be customized for all following uses of the
function with a \texttt{\ set\ } rule.

Whether this header should be repeated across pages.

Default: \texttt{\ }{\texttt{\ true\ }}\texttt{\ }

\paragraph{\texorpdfstring{\texttt{\ children\ }}{ children }}\label{definitions-header-children}

\href{/docs/reference/foundations/content/}{content}

{Required} {{ Positional }}

\phantomsection\label{definitions-header-children-positional-tooltip}
Positional parameters are specified in order, without names.

{{ Variadic }}

\phantomsection\label{definitions-header-children-variadic-tooltip}
Variadic parameters can be specified multiple times.

The cells and lines within the header.

\subsubsection{\texorpdfstring{\texttt{\ footer\ } {{ Element
}}}{ footer   Element }}\label{definitions-footer}

\phantomsection\label{definitions-footer-element-tooltip}
Element functions can be customized with \texttt{\ set\ } and
\texttt{\ show\ } rules.

A repeatable table footer.

Just like the
\href{/docs/reference/model/table/\#definitions-header}{\texttt{\ table.header\ }}
element, the footer can repeat itself on every page of the table. This
is useful for improving legibility by adding the column labels in both
the header and footer of a large table, totals, or other information
that should be visible on every page.

No other table cells may be placed after the footer.

table { . } { footer } (

{ \hyperref[definitions-footer-parameters-repeat]{repeat :}
\href{/docs/reference/foundations/bool/}{bool} , } {
\hyperref[definitions-footer-parameters-children]{..}
\href{/docs/reference/foundations/content/}{content} , }

) -\textgreater{} \href{/docs/reference/foundations/content/}{content}

\paragraph{\texorpdfstring{\texttt{\ repeat\ }}{ repeat }}\label{definitions-footer-repeat}

\href{/docs/reference/foundations/bool/}{bool}

{{ Settable }}

\phantomsection\label{definitions-footer-repeat-settable-tooltip}
Settable parameters can be customized for all following uses of the
function with a \texttt{\ set\ } rule.

Whether this footer should be repeated across pages.

Default: \texttt{\ }{\texttt{\ true\ }}\texttt{\ }

\paragraph{\texorpdfstring{\texttt{\ children\ }}{ children }}\label{definitions-footer-children}

\href{/docs/reference/foundations/content/}{content}

{Required} {{ Positional }}

\phantomsection\label{definitions-footer-children-positional-tooltip}
Positional parameters are specified in order, without names.

{{ Variadic }}

\phantomsection\label{definitions-footer-children-variadic-tooltip}
Variadic parameters can be specified multiple times.

The cells and lines within the footer.

\href{/docs/reference/model/strong/}{\pandocbounded{\includesvg[keepaspectratio]{/assets/icons/16-arrow-right.svg}}}

{ Strong Emphasis } { Previous page }

\href{/docs/reference/model/terms/}{\pandocbounded{\includesvg[keepaspectratio]{/assets/icons/16-arrow-right.svg}}}

{ Term List } { Next page }


\section{Docs LaTeX/typst.app/docs/reference/model/figure.tex}
\title{typst.app/docs/reference/model/figure}

\begin{itemize}
\tightlist
\item
  \href{/docs}{\includesvg[width=0.16667in,height=0.16667in]{/assets/icons/16-docs-dark.svg}}
\item
  \includesvg[width=0.16667in,height=0.16667in]{/assets/icons/16-arrow-right.svg}
\item
  \href{/docs/reference/}{Reference}
\item
  \includesvg[width=0.16667in,height=0.16667in]{/assets/icons/16-arrow-right.svg}
\item
  \href{/docs/reference/model/}{Model}
\item
  \includesvg[width=0.16667in,height=0.16667in]{/assets/icons/16-arrow-right.svg}
\item
  \href{/docs/reference/model/figure/}{Figure}
\end{itemize}

\section{\texorpdfstring{\texttt{\ figure\ } {{ Element
}}}{ figure   Element }}\label{summary}

\phantomsection\label{element-tooltip}
Element functions can be customized with \texttt{\ set\ } and
\texttt{\ show\ } rules.

A figure with an optional caption.

Automatically detects its kind to select the correct counting track. For
example, figures containing images will be numbered separately from
figures containing tables.

\subsection{Examples}\label{examples}

The example below shows a basic figure with an image:

\begin{verbatim}
@glacier shows a glacier. Glaciers
are complex systems.

#figure(
  image("glacier.jpg", width: 80%),
  caption: [A curious figure.],
) <glacier>
\end{verbatim}

\includegraphics[width=5in,height=\textheight,keepaspectratio]{/assets/docs/udw8J1zW8CDfoYB1XTzdLgAAAAAAAAAA.png}

You can also insert \href{/docs/reference/model/table/}{tables} into
figures to give them a caption. The figure will detect this and
automatically use a separate counter.

\begin{verbatim}
#figure(
  table(
    columns: 4,
    [t], [1], [2], [3],
    [y], [0.3s], [0.4s], [0.8s],
  ),
  caption: [Timing results],
)
\end{verbatim}

\includegraphics[width=5in,height=\textheight,keepaspectratio]{/assets/docs/_RaOJik9O5UoQO8Eq6OM9gAAAAAAAAAA.png}

This behaviour can be overridden by explicitly specifying the
figure\textquotesingle s \texttt{\ kind\ } . All figures of the same
kind share a common counter.

\subsection{Figure behaviour}\label{figure-behaviour}

By default, figures are placed within the flow of content. To make them
float to the top or bottom of the page, you can use the
\href{/docs/reference/model/figure/\#parameters-placement}{\texttt{\ placement\ }}
argument.

If your figure is too large and its contents are breakable across pages
(e.g. if it contains a large table), then you can make the figure itself
breakable across pages as well with this show rule:

\begin{verbatim}
#show figure: set block(breakable: true)
\end{verbatim}

See the
\href{/docs/reference/layout/block/\#parameters-breakable}{block}
documentation for more information about breakable and non-breakable
blocks.

\subsection{Caption customization}\label{caption-customization}

You can modify the appearance of the figure\textquotesingle s caption
with its associated
\href{/docs/reference/model/figure/\#definitions-caption}{\texttt{\ caption\ }}
function. In the example below, we emphasize all captions:

\begin{verbatim}
#show figure.caption: emph

#figure(
  rect[Hello],
  caption: [I am emphasized!],
)
\end{verbatim}

\includegraphics[width=5in,height=\textheight,keepaspectratio]{/assets/docs/_XYhSBTt1dmjYR9A4n_aCgAAAAAAAAAA.png}

By using a
\href{/docs/reference/foundations/function/\#definitions-where}{\texttt{\ where\ }}
selector, we can scope such rules to specific kinds of figures. For
example, to position the caption above tables, but keep it below for all
other kinds of figures, we could write the following show-set rule:

\begin{verbatim}
#show figure.where(
  kind: table
): set figure.caption(position: top)

#figure(
  table(columns: 2)[A][B][C][D],
  caption: [I'm up here],
)
\end{verbatim}

\includegraphics[width=5in,height=\textheight,keepaspectratio]{/assets/docs/5FXY-vQKID4Q1FYsR4Ix9AAAAAAAAAAA.png}

\subsection{\texorpdfstring{{ Parameters
}}{ Parameters }}\label{parameters}

\phantomsection\label{parameters-tooltip}
Parameters are the inputs to a function. They are specified in
parentheses after the function name.

{ figure } (

{ \href{/docs/reference/foundations/content/}{content} , } {
\hyperref[parameters-placement]{placement :}
\href{/docs/reference/foundations/none/}{none}
\href{/docs/reference/foundations/auto/}{auto}
\href{/docs/reference/layout/alignment/}{alignment} , } {
\hyperref[parameters-scope]{scope :}
\href{/docs/reference/foundations/str/}{str} , } {
\hyperref[parameters-caption]{caption :}
\href{/docs/reference/foundations/none/}{none}
\href{/docs/reference/foundations/content/}{content} , } {
\hyperref[parameters-kind]{kind :}
\href{/docs/reference/foundations/auto/}{auto}
\href{/docs/reference/foundations/str/}{str}
\href{/docs/reference/foundations/function/}{function} , } {
\hyperref[parameters-supplement]{supplement :}
\href{/docs/reference/foundations/none/}{none}
\href{/docs/reference/foundations/auto/}{auto}
\href{/docs/reference/foundations/content/}{content}
\href{/docs/reference/foundations/function/}{function} , } {
\hyperref[parameters-numbering]{numbering :}
\href{/docs/reference/foundations/none/}{none}
\href{/docs/reference/foundations/str/}{str}
\href{/docs/reference/foundations/function/}{function} , } {
\hyperref[parameters-gap]{gap :}
\href{/docs/reference/layout/length/}{length} , } {
\hyperref[parameters-outlined]{outlined :}
\href{/docs/reference/foundations/bool/}{bool} , }

) -\textgreater{} \href{/docs/reference/foundations/content/}{content}

\subsubsection{\texorpdfstring{\texttt{\ body\ }}{ body }}\label{parameters-body}

\href{/docs/reference/foundations/content/}{content}

{Required} {{ Positional }}

\phantomsection\label{parameters-body-positional-tooltip}
Positional parameters are specified in order, without names.

The content of the figure. Often, an
\href{/docs/reference/visualize/image/}{image} .

\subsubsection{\texorpdfstring{\texttt{\ placement\ }}{ placement }}\label{parameters-placement}

\href{/docs/reference/foundations/none/}{none} {or}
\href{/docs/reference/foundations/auto/}{auto} {or}
\href{/docs/reference/layout/alignment/}{alignment}

{{ Settable }}

\phantomsection\label{parameters-placement-settable-tooltip}
Settable parameters can be customized for all following uses of the
function with a \texttt{\ set\ } rule.

The figure\textquotesingle s placement on the page.

\begin{itemize}
\tightlist
\item
  \texttt{\ }{\texttt{\ none\ }}\texttt{\ } : The figure stays in-flow
  exactly where it was specified like other content.
\item
  \texttt{\ }{\texttt{\ auto\ }}\texttt{\ } : The figure picks
  \texttt{\ top\ } or \texttt{\ bottom\ } depending on which is closer.
\item
  \texttt{\ top\ } : The figure floats to the top of the page.
\item
  \texttt{\ bottom\ } : The figure floats to the bottom of the page.
\end{itemize}

The gap between the main flow content and the floating figure is
controlled by the
\href{/docs/reference/layout/place/\#parameters-clearance}{\texttt{\ clearance\ }}
argument on the \texttt{\ place\ } function.

Default: \texttt{\ }{\texttt{\ none\ }}\texttt{\ }

\includesvg[width=0.16667in,height=0.16667in]{/assets/icons/16-arrow-right.svg}
View example

\begin{verbatim}
#set page(height: 200pt)

= Introduction
#figure(
  placement: bottom,
  caption: [A glacier],
  image("glacier.jpg", width: 60%),
)
#lorem(60)
\end{verbatim}

\includegraphics[width=5in,height=\textheight,keepaspectratio]{/assets/docs/AvTTV4CvkxyZB8XrzNUT3wAAAAAAAAAA.png}
\includegraphics[width=5in,height=\textheight,keepaspectratio]{/assets/docs/AvTTV4CvkxyZB8XrzNUT3wAAAAAAAAAB.png}

\subsubsection{\texorpdfstring{\texttt{\ scope\ }}{ scope }}\label{parameters-scope}

\href{/docs/reference/foundations/str/}{str}

{{ Settable }}

\phantomsection\label{parameters-scope-settable-tooltip}
Settable parameters can be customized for all following uses of the
function with a \texttt{\ set\ } rule.

Relative to which containing scope the figure is placed.

Set this to \texttt{\ }{\texttt{\ "parent"\ }}\texttt{\ } to create a
full-width figure in a two-column document.

Has no effect if \texttt{\ placement\ } is
\texttt{\ }{\texttt{\ none\ }}\texttt{\ } .

\begin{longtable}[]{@{}ll@{}}
\toprule\noalign{}
Variant & Details \\
\midrule\noalign{}
\endhead
\bottomrule\noalign{}
\endlastfoot
\texttt{\ "\ column\ "\ } & Place into the current column. \\
\texttt{\ "\ parent\ "\ } & Place relative to the parent, letting the
content span over all columns. \\
\end{longtable}

Default: \texttt{\ }{\texttt{\ "column"\ }}\texttt{\ }

\includesvg[width=0.16667in,height=0.16667in]{/assets/icons/16-arrow-right.svg}
View example

\begin{verbatim}
#set page(height: 250pt, columns: 2)

= Introduction
#figure(
  placement: bottom,
  scope: "parent",
  caption: [A glacier],
  image("glacier.jpg", width: 60%),
)
#lorem(60)
\end{verbatim}

\includegraphics[width=5in,height=\textheight,keepaspectratio]{/assets/docs/_zX5K9NHfd2mYYCeJmag7wAAAAAAAAAA.png}
\includegraphics[width=5in,height=\textheight,keepaspectratio]{/assets/docs/_zX5K9NHfd2mYYCeJmag7wAAAAAAAAAB.png}

\subsubsection{\texorpdfstring{\texttt{\ caption\ }}{ caption }}\label{parameters-caption}

\href{/docs/reference/foundations/none/}{none} {or}
\href{/docs/reference/foundations/content/}{content}

{{ Settable }}

\phantomsection\label{parameters-caption-settable-tooltip}
Settable parameters can be customized for all following uses of the
function with a \texttt{\ set\ } rule.

The figure\textquotesingle s caption.

Default: \texttt{\ }{\texttt{\ none\ }}\texttt{\ }

\subsubsection{\texorpdfstring{\texttt{\ kind\ }}{ kind }}\label{parameters-kind}

\href{/docs/reference/foundations/auto/}{auto} {or}
\href{/docs/reference/foundations/str/}{str} {or}
\href{/docs/reference/foundations/function/}{function}

{{ Settable }}

\phantomsection\label{parameters-kind-settable-tooltip}
Settable parameters can be customized for all following uses of the
function with a \texttt{\ set\ } rule.

The kind of figure this is.

All figures of the same kind share a common counter.

If set to \texttt{\ }{\texttt{\ auto\ }}\texttt{\ } , the figure will
try to automatically determine its kind based on the type of its body.
Automatically detected kinds are
\href{/docs/reference/model/table/}{tables} and
\href{/docs/reference/text/raw/}{code} . In other cases, the inferred
kind is that of an \href{/docs/reference/visualize/image/}{image} .

Setting this to something other than
\texttt{\ }{\texttt{\ auto\ }}\texttt{\ } will override the automatic
detection. This can be useful if

\begin{itemize}
\tightlist
\item
  you wish to create a custom figure type that is not an
  \href{/docs/reference/visualize/image/}{image} , a
  \href{/docs/reference/model/table/}{table} or
  \href{/docs/reference/text/raw/}{code} ,
\item
  you want to force the figure to use a specific counter regardless of
  its content.
\end{itemize}

You can set the kind to be an element function or a string. If you set
it to an element function other than
\href{/docs/reference/model/table/}{\texttt{\ table\ }} ,
\href{/docs/reference/text/raw/}{\texttt{\ raw\ }} or
\href{/docs/reference/visualize/image/}{\texttt{\ image\ }} , you will
need to manually specify the figure\textquotesingle s supplement.

Default: \texttt{\ }{\texttt{\ auto\ }}\texttt{\ }

\includesvg[width=0.16667in,height=0.16667in]{/assets/icons/16-arrow-right.svg}
View example

\begin{verbatim}
#figure(
  circle(radius: 10pt),
  caption: [A curious atom.],
  kind: "atom",
  supplement: [Atom],
)
\end{verbatim}

\includegraphics[width=5in,height=\textheight,keepaspectratio]{/assets/docs/gnEhUtPlQLC9DmHftY4vzQAAAAAAAAAA.png}

\subsubsection{\texorpdfstring{\texttt{\ supplement\ }}{ supplement }}\label{parameters-supplement}

\href{/docs/reference/foundations/none/}{none} {or}
\href{/docs/reference/foundations/auto/}{auto} {or}
\href{/docs/reference/foundations/content/}{content} {or}
\href{/docs/reference/foundations/function/}{function}

{{ Settable }}

\phantomsection\label{parameters-supplement-settable-tooltip}
Settable parameters can be customized for all following uses of the
function with a \texttt{\ set\ } rule.

The figure\textquotesingle s supplement.

If set to \texttt{\ }{\texttt{\ auto\ }}\texttt{\ } , the figure will
try to automatically determine the correct supplement based on the
\texttt{\ kind\ } and the active
\href{/docs/reference/text/text/\#parameters-lang}{text language} . If
you are using a custom figure type, you will need to manually specify
the supplement.

If a function is specified, it is passed the first descendant of the
specified \texttt{\ kind\ } (typically, the figure\textquotesingle s
body) and should return content.

Default: \texttt{\ }{\texttt{\ auto\ }}\texttt{\ }

\includesvg[width=0.16667in,height=0.16667in]{/assets/icons/16-arrow-right.svg}
View example

\begin{verbatim}
#figure(
  [The contents of my figure!],
  caption: [My custom figure],
  supplement: [Bar],
  kind: "foo",
)
\end{verbatim}

\includegraphics[width=5in,height=\textheight,keepaspectratio]{/assets/docs/_ow3s-d4xSBN6VX-nVHVzQAAAAAAAAAA.png}

\subsubsection{\texorpdfstring{\texttt{\ numbering\ }}{ numbering }}\label{parameters-numbering}

\href{/docs/reference/foundations/none/}{none} {or}
\href{/docs/reference/foundations/str/}{str} {or}
\href{/docs/reference/foundations/function/}{function}

{{ Settable }}

\phantomsection\label{parameters-numbering-settable-tooltip}
Settable parameters can be customized for all following uses of the
function with a \texttt{\ set\ } rule.

How to number the figure. Accepts a
\href{/docs/reference/model/numbering/}{numbering pattern or function} .

Default: \texttt{\ }{\texttt{\ "1"\ }}\texttt{\ }

\subsubsection{\texorpdfstring{\texttt{\ gap\ }}{ gap }}\label{parameters-gap}

\href{/docs/reference/layout/length/}{length}

{{ Settable }}

\phantomsection\label{parameters-gap-settable-tooltip}
Settable parameters can be customized for all following uses of the
function with a \texttt{\ set\ } rule.

The vertical gap between the body and caption.

Default: \texttt{\ }{\texttt{\ 0.65em\ }}\texttt{\ }

\subsubsection{\texorpdfstring{\texttt{\ outlined\ }}{ outlined }}\label{parameters-outlined}

\href{/docs/reference/foundations/bool/}{bool}

{{ Settable }}

\phantomsection\label{parameters-outlined-settable-tooltip}
Settable parameters can be customized for all following uses of the
function with a \texttt{\ set\ } rule.

Whether the figure should appear in an
\href{/docs/reference/model/outline/}{\texttt{\ outline\ }} of figures.

Default: \texttt{\ }{\texttt{\ true\ }}\texttt{\ }

\subsection{\texorpdfstring{{ Definitions
}}{ Definitions }}\label{definitions}

\phantomsection\label{definitions-tooltip}
Functions and types and can have associated definitions. These are
accessed by specifying the function or type, followed by a period, and
then the definition\textquotesingle s name.

\subsubsection{\texorpdfstring{\texttt{\ caption\ } {{ Element
}}}{ caption   Element }}\label{definitions-caption}

\phantomsection\label{definitions-caption-element-tooltip}
Element functions can be customized with \texttt{\ set\ } and
\texttt{\ show\ } rules.

The caption of a figure. This element can be used in set and show rules
to customize the appearance of captions for all figures or figures of a
specific kind.

In addition to its \texttt{\ pos\ } and \texttt{\ body\ } , the
\texttt{\ caption\ } also provides the figure\textquotesingle s
\texttt{\ kind\ } , \texttt{\ supplement\ } , \texttt{\ counter\ } , and
\texttt{\ numbering\ } as fields. These parts can be used in
\href{/docs/reference/foundations/function/\#definitions-where}{\texttt{\ where\ }}
selectors and show rules to build a completely custom caption.

figure { . } { caption } (

{ \hyperref[definitions-caption-parameters-position]{position :}
\href{/docs/reference/layout/alignment/}{alignment} , } {
\hyperref[definitions-caption-parameters-separator]{separator :}
\href{/docs/reference/foundations/auto/}{auto}
\href{/docs/reference/foundations/content/}{content} , } {
\href{/docs/reference/foundations/content/}{content} , }

) -\textgreater{} \href{/docs/reference/foundations/content/}{content}

\begin{verbatim}
#show figure.caption: emph

#figure(
  rect[Hello],
  caption: [A rectangle],
)
\end{verbatim}

\includegraphics[width=5in,height=\textheight,keepaspectratio]{/assets/docs/_9Rae3k-14dcb00bWW4ciAAAAAAAAAAA.png}

\paragraph{\texorpdfstring{\texttt{\ position\ }}{ position }}\label{definitions-caption-position}

\href{/docs/reference/layout/alignment/}{alignment}

{{ Settable }}

\phantomsection\label{definitions-caption-position-settable-tooltip}
Settable parameters can be customized for all following uses of the
function with a \texttt{\ set\ } rule.

The caption\textquotesingle s position in the figure. Either
\texttt{\ top\ } or \texttt{\ bottom\ } .

Default: \texttt{\ bottom\ }

\includesvg[width=0.16667in,height=0.16667in]{/assets/icons/16-arrow-right.svg}
View example

\begin{verbatim}
#show figure.where(
  kind: table
): set figure.caption(position: top)

#figure(
  table(columns: 2)[A][B],
  caption: [I'm up here],
)

#figure(
  rect[Hi],
  caption: [I'm down here],
)

#figure(
  table(columns: 2)[A][B],
  caption: figure.caption(
    position: bottom,
    [I'm down here too!]
  )
)
\end{verbatim}

\includegraphics[width=5in,height=\textheight,keepaspectratio]{/assets/docs/IdFKmiavSqMTEqn8wUXuUgAAAAAAAAAA.png}

\paragraph{\texorpdfstring{\texttt{\ separator\ }}{ separator }}\label{definitions-caption-separator}

\href{/docs/reference/foundations/auto/}{auto} {or}
\href{/docs/reference/foundations/content/}{content}

{{ Settable }}

\phantomsection\label{definitions-caption-separator-settable-tooltip}
Settable parameters can be customized for all following uses of the
function with a \texttt{\ set\ } rule.

The separator which will appear between the number and body.

If set to \texttt{\ }{\texttt{\ auto\ }}\texttt{\ } , the separator will
be adapted to the current
\href{/docs/reference/text/text/\#parameters-lang}{language} and
\href{/docs/reference/text/text/\#parameters-region}{region} .

Default: \texttt{\ }{\texttt{\ auto\ }}\texttt{\ }

\includesvg[width=0.16667in,height=0.16667in]{/assets/icons/16-arrow-right.svg}
View example

\begin{verbatim}
#set figure.caption(separator: [ --- ])

#figure(
  rect[Hello],
  caption: [A rectangle],
)
\end{verbatim}

\includegraphics[width=5in,height=\textheight,keepaspectratio]{/assets/docs/F47AgUphmXiVn12oCb_ECAAAAAAAAAAA.png}

\paragraph{\texorpdfstring{\texttt{\ body\ }}{ body }}\label{definitions-caption-body}

\href{/docs/reference/foundations/content/}{content}

{Required} {{ Positional }}

\phantomsection\label{definitions-caption-body-positional-tooltip}
Positional parameters are specified in order, without names.

The caption\textquotesingle s body.

Can be used alongside \texttt{\ kind\ } , \texttt{\ supplement\ } ,
\texttt{\ counter\ } , \texttt{\ numbering\ } , and
\texttt{\ location\ } to completely customize the caption.

\includesvg[width=0.16667in,height=0.16667in]{/assets/icons/16-arrow-right.svg}
View example

\begin{verbatim}
#show figure.caption: it => [
  #underline(it.body) |
  #it.supplement
  #context it.counter.display(it.numbering)
]

#figure(
  rect[Hello],
  caption: [A rectangle],
)
\end{verbatim}

\includegraphics[width=5in,height=\textheight,keepaspectratio]{/assets/docs/JxID--FAnIhAECKLMVFiVwAAAAAAAAAA.png}

\href{/docs/reference/model/emph/}{\pandocbounded{\includesvg[keepaspectratio]{/assets/icons/16-arrow-right.svg}}}

{ Emphasis } { Previous page }

\href{/docs/reference/model/footnote/}{\pandocbounded{\includesvg[keepaspectratio]{/assets/icons/16-arrow-right.svg}}}

{ Footnote } { Next page }


\section{Docs LaTeX/typst.app/docs/reference/model/parbreak.tex}
\title{typst.app/docs/reference/model/parbreak}

\begin{itemize}
\tightlist
\item
  \href{/docs}{\includesvg[width=0.16667in,height=0.16667in]{/assets/icons/16-docs-dark.svg}}
\item
  \includesvg[width=0.16667in,height=0.16667in]{/assets/icons/16-arrow-right.svg}
\item
  \href{/docs/reference/}{Reference}
\item
  \includesvg[width=0.16667in,height=0.16667in]{/assets/icons/16-arrow-right.svg}
\item
  \href{/docs/reference/model/}{Model}
\item
  \includesvg[width=0.16667in,height=0.16667in]{/assets/icons/16-arrow-right.svg}
\item
  \href{/docs/reference/model/parbreak/}{Paragraph Break}
\end{itemize}

\section{\texorpdfstring{\texttt{\ parbreak\ } {{ Element
}}}{ parbreak   Element }}\label{summary}

\phantomsection\label{element-tooltip}
Element functions can be customized with \texttt{\ set\ } and
\texttt{\ show\ } rules.

A paragraph break.

This starts a new paragraph. Especially useful when used within code
like \href{/docs/reference/scripting/\#loops}{for loops} . Multiple
consecutive paragraph breaks collapse into a single one.

\subsection{Example}\label{example}

\begin{verbatim}
#for i in range(3) {
  [Blind text #i: ]
  lorem(5)
  parbreak()
}
\end{verbatim}

\includegraphics[width=5in,height=\textheight,keepaspectratio]{/assets/docs/Ugn0Cpe50EHdh4tXrmb4YAAAAAAAAAAA.png}

\subsection{Syntax}\label{syntax}

Instead of calling this function, you can insert a blank line into your
markup to create a paragraph break.

\subsection{\texorpdfstring{{ Parameters
}}{ Parameters }}\label{parameters}

\phantomsection\label{parameters-tooltip}
Parameters are the inputs to a function. They are specified in
parentheses after the function name.

{ parbreak } (

) -\textgreater{} \href{/docs/reference/foundations/content/}{content}

\href{/docs/reference/model/par/}{\pandocbounded{\includesvg[keepaspectratio]{/assets/icons/16-arrow-right.svg}}}

{ Paragraph } { Previous page }

\href{/docs/reference/model/quote/}{\pandocbounded{\includesvg[keepaspectratio]{/assets/icons/16-arrow-right.svg}}}

{ Quote } { Next page }


\section{Docs LaTeX/typst.app/docs/reference/model/outline.tex}
\title{typst.app/docs/reference/model/outline}

\begin{itemize}
\tightlist
\item
  \href{/docs}{\includesvg[width=0.16667in,height=0.16667in]{/assets/icons/16-docs-dark.svg}}
\item
  \includesvg[width=0.16667in,height=0.16667in]{/assets/icons/16-arrow-right.svg}
\item
  \href{/docs/reference/}{Reference}
\item
  \includesvg[width=0.16667in,height=0.16667in]{/assets/icons/16-arrow-right.svg}
\item
  \href{/docs/reference/model/}{Model}
\item
  \includesvg[width=0.16667in,height=0.16667in]{/assets/icons/16-arrow-right.svg}
\item
  \href{/docs/reference/model/outline/}{Outline}
\end{itemize}

\section{\texorpdfstring{\texttt{\ outline\ } {{ Element
}}}{ outline   Element }}\label{summary}

\phantomsection\label{element-tooltip}
Element functions can be customized with \texttt{\ set\ } and
\texttt{\ show\ } rules.

A table of contents, figures, or other elements.

This function generates a list of all occurrences of an element in the
document, up to a given depth. The element\textquotesingle s numbering
and page number will be displayed in the outline alongside its title or
caption. By default this generates a table of contents.

\subsection{Example}\label{example}

\begin{verbatim}
#outline()

= Introduction
#lorem(5)

= Prior work
#lorem(10)
\end{verbatim}

\includegraphics[width=5in,height=\textheight,keepaspectratio]{/assets/docs/pxzEoLgfS9GjzIb6I2LlEgAAAAAAAAAA.png}

\subsection{Alternative outlines}\label{alternative-outlines}

By setting the \texttt{\ target\ } parameter, the outline can be used to
generate a list of other kinds of elements than headings. In the example
below, we list all figures containing images by setting
\texttt{\ target\ } to
\texttt{\ figure\ }{\texttt{\ .\ }}\texttt{\ }{\texttt{\ where\ }}\texttt{\ }{\texttt{\ (\ }}\texttt{\ kind\ }{\texttt{\ :\ }}\texttt{\ image\ }{\texttt{\ )\ }}\texttt{\ }
. We could have also set it to just \texttt{\ figure\ } , but then the
list would also include figures containing tables or other material. For
more details on the \texttt{\ where\ } selector,
\href{/docs/reference/foundations/function/\#definitions-where}{see
here} .

\begin{verbatim}
#outline(
  title: [List of Figures],
  target: figure.where(kind: image),
)

#figure(
  image("tiger.jpg"),
  caption: [A nice figure!],
)
\end{verbatim}

\includegraphics[width=5in,height=\textheight,keepaspectratio]{/assets/docs/K0Fgir_M6IbOnlxFTpRoyAAAAAAAAAAA.png}

\subsection{Styling the outline}\label{styling-the-outline}

The outline element has several options for customization, such as its
\texttt{\ title\ } and \texttt{\ indent\ } parameters. If desired,
however, it is possible to have more control over the
outline\textquotesingle s look and style through the
\href{/docs/reference/model/outline/\#definitions-entry}{\texttt{\ outline.entry\ }}
element.

\subsection{\texorpdfstring{{ Parameters
}}{ Parameters }}\label{parameters}

\phantomsection\label{parameters-tooltip}
Parameters are the inputs to a function. They are specified in
parentheses after the function name.

{ outline } (

{ \hyperref[parameters-title]{title :}
\href{/docs/reference/foundations/none/}{none}
\href{/docs/reference/foundations/auto/}{auto}
\href{/docs/reference/foundations/content/}{content} , } {
\hyperref[parameters-target]{target :}
\href{/docs/reference/foundations/label/}{label}
\href{/docs/reference/foundations/selector/}{selector}
\href{/docs/reference/introspection/location/}{location}
\href{/docs/reference/foundations/function/}{function} , } {
\hyperref[parameters-depth]{depth :}
\href{/docs/reference/foundations/none/}{none}
\href{/docs/reference/foundations/int/}{int} , } {
\hyperref[parameters-indent]{indent :}
\href{/docs/reference/foundations/none/}{none}
\href{/docs/reference/foundations/auto/}{auto}
\href{/docs/reference/foundations/bool/}{bool}
\href{/docs/reference/layout/relative/}{relative}
\href{/docs/reference/foundations/function/}{function} , } {
\hyperref[parameters-fill]{fill :}
\href{/docs/reference/foundations/none/}{none}
\href{/docs/reference/foundations/content/}{content} , }

) -\textgreater{} \href{/docs/reference/foundations/content/}{content}

\subsubsection{\texorpdfstring{\texttt{\ title\ }}{ title }}\label{parameters-title}

\href{/docs/reference/foundations/none/}{none} {or}
\href{/docs/reference/foundations/auto/}{auto} {or}
\href{/docs/reference/foundations/content/}{content}

{{ Settable }}

\phantomsection\label{parameters-title-settable-tooltip}
Settable parameters can be customized for all following uses of the
function with a \texttt{\ set\ } rule.

The title of the outline.

\begin{itemize}
\tightlist
\item
  When set to \texttt{\ }{\texttt{\ auto\ }}\texttt{\ } , an appropriate
  title for the \href{/docs/reference/text/text/\#parameters-lang}{text
  language} will be used. This is the default.
\item
  When set to \texttt{\ }{\texttt{\ none\ }}\texttt{\ } , the outline
  will not have a title.
\item
  A custom title can be set by passing content.
\end{itemize}

The outline\textquotesingle s heading will not be numbered by default,
but you can force it to be with a show-set rule:
\texttt{\ }{\texttt{\ show\ }}\texttt{\ }{\texttt{\ outline\ }}\texttt{\ }{\texttt{\ :\ }}\texttt{\ }{\texttt{\ set\ }}\texttt{\ }{\texttt{\ heading\ }}\texttt{\ }{\texttt{\ (\ }}\texttt{\ numbering\ }{\texttt{\ :\ }}\texttt{\ }{\texttt{\ "1."\ }}\texttt{\ }{\texttt{\ )\ }}\texttt{\ }

Default: \texttt{\ }{\texttt{\ auto\ }}\texttt{\ }

\subsubsection{\texorpdfstring{\texttt{\ target\ }}{ target }}\label{parameters-target}

\href{/docs/reference/foundations/label/}{label} {or}
\href{/docs/reference/foundations/selector/}{selector} {or}
\href{/docs/reference/introspection/location/}{location} {or}
\href{/docs/reference/foundations/function/}{function}

{{ Settable }}

\phantomsection\label{parameters-target-settable-tooltip}
Settable parameters can be customized for all following uses of the
function with a \texttt{\ set\ } rule.

The type of element to include in the outline.

To list figures containing a specific kind of element, like a table, you
can write
\texttt{\ figure\ }{\texttt{\ .\ }}\texttt{\ }{\texttt{\ where\ }}\texttt{\ }{\texttt{\ (\ }}\texttt{\ kind\ }{\texttt{\ :\ }}\texttt{\ table\ }{\texttt{\ )\ }}\texttt{\ }
.

Default:
\texttt{\ heading\ }{\texttt{\ .\ }}\texttt{\ }{\texttt{\ where\ }}\texttt{\ }{\texttt{\ (\ }}\texttt{\ outlined\ }{\texttt{\ :\ }}\texttt{\ }{\texttt{\ true\ }}\texttt{\ }{\texttt{\ )\ }}\texttt{\ }

\includesvg[width=0.16667in,height=0.16667in]{/assets/icons/16-arrow-right.svg}
View example

\begin{verbatim}
#outline(
  title: [List of Tables],
  target: figure.where(kind: table),
)

#figure(
  table(
    columns: 4,
    [t], [1], [2], [3],
    [y], [0.3], [0.7], [0.5],
  ),
  caption: [Experiment results],
)
\end{verbatim}

\includegraphics[width=5in,height=\textheight,keepaspectratio]{/assets/docs/9oD_YO_3aaN85cAixeBP2gAAAAAAAAAA.png}

\subsubsection{\texorpdfstring{\texttt{\ depth\ }}{ depth }}\label{parameters-depth}

\href{/docs/reference/foundations/none/}{none} {or}
\href{/docs/reference/foundations/int/}{int}

{{ Settable }}

\phantomsection\label{parameters-depth-settable-tooltip}
Settable parameters can be customized for all following uses of the
function with a \texttt{\ set\ } rule.

The maximum level up to which elements are included in the outline. When
this argument is \texttt{\ }{\texttt{\ none\ }}\texttt{\ } , all
elements are included.

Default: \texttt{\ }{\texttt{\ none\ }}\texttt{\ }

\includesvg[width=0.16667in,height=0.16667in]{/assets/icons/16-arrow-right.svg}
View example

\begin{verbatim}
#set heading(numbering: "1.")
#outline(depth: 2)

= Yes
Top-level section.

== Still
Subsection.

=== Nope
Not included.
\end{verbatim}

\includegraphics[width=5in,height=\textheight,keepaspectratio]{/assets/docs/fYEfgTUmkbH0skbcMKeSFwAAAAAAAAAA.png}

\subsubsection{\texorpdfstring{\texttt{\ indent\ }}{ indent }}\label{parameters-indent}

\href{/docs/reference/foundations/none/}{none} {or}
\href{/docs/reference/foundations/auto/}{auto} {or}
\href{/docs/reference/foundations/bool/}{bool} {or}
\href{/docs/reference/layout/relative/}{relative} {or}
\href{/docs/reference/foundations/function/}{function}

{{ Settable }}

\phantomsection\label{parameters-indent-settable-tooltip}
Settable parameters can be customized for all following uses of the
function with a \texttt{\ set\ } rule.

How to indent the outline\textquotesingle s entries.

\begin{itemize}
\tightlist
\item
  \texttt{\ }{\texttt{\ none\ }}\texttt{\ } : No indent
\item
  \texttt{\ }{\texttt{\ auto\ }}\texttt{\ } : Indents the numbering of
  the nested entry with the title of its parent entry. This only has an
  effect if the entries are numbered (e.g., via
  \href{/docs/reference/model/heading/\#parameters-numbering}{heading
  numbering} ).
\item
  \href{/docs/reference/layout/relative/}{Relative length} : Indents the
  item by this length multiplied by its nesting level. Specifying
  \texttt{\ }{\texttt{\ 2em\ }}\texttt{\ } , for instance, would indent
  top-level headings (not nested) by
  \texttt{\ }{\texttt{\ 0em\ }}\texttt{\ } , second level headings by
  \texttt{\ }{\texttt{\ 2em\ }}\texttt{\ } (nested once), third-level
  headings by \texttt{\ }{\texttt{\ 4em\ }}\texttt{\ } (nested twice)
  and so on.
\item
  \href{/docs/reference/foundations/function/}{Function} : You can
  completely customize this setting with a function. That function
  receives the nesting level as a parameter (starting at 0 for top-level
  headings/elements) and can return a relative length or content making
  up the indent. For example,
  \texttt{\ n\ }{\texttt{\ =\textgreater{}\ }}\texttt{\ n\ }{\texttt{\ *\ }}\texttt{\ }{\texttt{\ 2em\ }}\texttt{\ }
  would be equivalent to just specifying
  \texttt{\ }{\texttt{\ 2em\ }}\texttt{\ } , while
  \texttt{\ n\ }{\texttt{\ =\textgreater{}\ }}\texttt{\ }{\texttt{\ {[}\ }}\texttt{\ →\ }{\texttt{\ {]}\ }}\texttt{\ }{\texttt{\ *\ }}\texttt{\ n\ }
  would indent with one arrow per nesting level.
\end{itemize}

\emph{Migration hints:} Specifying
\texttt{\ }{\texttt{\ true\ }}\texttt{\ } (equivalent to
\texttt{\ }{\texttt{\ auto\ }}\texttt{\ } ) or
\texttt{\ }{\texttt{\ false\ }}\texttt{\ } (equivalent to
\texttt{\ }{\texttt{\ none\ }}\texttt{\ } ) for this option is
deprecated and will be removed in a future release.

Default: \texttt{\ }{\texttt{\ none\ }}\texttt{\ }

\includesvg[width=0.16667in,height=0.16667in]{/assets/icons/16-arrow-right.svg}
View example

\begin{verbatim}
#set heading(numbering: "1.a.")

#outline(
  title: [Contents (Automatic)],
  indent: auto,
)

#outline(
  title: [Contents (Length)],
  indent: 2em,
)

#outline(
  title: [Contents (Function)],
  indent: n => [→ ] * n,
)

= About ACME Corp.
== History
=== Origins
#lorem(10)

== Products
#lorem(10)
\end{verbatim}

\includegraphics[width=5in,height=\textheight,keepaspectratio]{/assets/docs/VxzAmxCU1uGgVW2hebfhtwAAAAAAAAAA.png}

\subsubsection{\texorpdfstring{\texttt{\ fill\ }}{ fill }}\label{parameters-fill}

\href{/docs/reference/foundations/none/}{none} {or}
\href{/docs/reference/foundations/content/}{content}

{{ Settable }}

\phantomsection\label{parameters-fill-settable-tooltip}
Settable parameters can be customized for all following uses of the
function with a \texttt{\ set\ } rule.

Content to fill the space between the title and the page number. Can be
set to \texttt{\ }{\texttt{\ none\ }}\texttt{\ } to disable filling.

Default:
\texttt{\ }{\texttt{\ repeat\ }}\texttt{\ }{\texttt{\ (\ }}\texttt{\ body\ }{\texttt{\ :\ }}\texttt{\ }{\texttt{\ {[}\ }}\texttt{\ .\ }{\texttt{\ {]}\ }}\texttt{\ }{\texttt{\ )\ }}\texttt{\ }

\includesvg[width=0.16667in,height=0.16667in]{/assets/icons/16-arrow-right.svg}
View example

\begin{verbatim}
#outline(fill: line(length: 100%))

= A New Beginning
\end{verbatim}

\includegraphics[width=5in,height=\textheight,keepaspectratio]{/assets/docs/KQmhOQJ1ylUUEeut6OI0rQAAAAAAAAAA.png}

\subsection{\texorpdfstring{{ Definitions
}}{ Definitions }}\label{definitions}

\phantomsection\label{definitions-tooltip}
Functions and types and can have associated definitions. These are
accessed by specifying the function or type, followed by a period, and
then the definition\textquotesingle s name.

\subsubsection{\texorpdfstring{\texttt{\ entry\ } {{ Element
}}}{ entry   Element }}\label{definitions-entry}

\phantomsection\label{definitions-entry-element-tooltip}
Element functions can be customized with \texttt{\ set\ } and
\texttt{\ show\ } rules.

Represents each entry line in an outline, including the reference to the
outlined element, its page number, and the filler content between both.

This element is intended for use with show rules to control the
appearance of outlines. To customize an entry\textquotesingle s line,
you can build it from scratch by accessing the \texttt{\ level\ } ,
\texttt{\ element\ } , \texttt{\ body\ } , \texttt{\ fill\ } and
\texttt{\ page\ } fields on the entry.

outline { . } { entry } (

{ \href{/docs/reference/foundations/int/}{int} , } {
\href{/docs/reference/foundations/content/}{content} , } {
\href{/docs/reference/foundations/content/}{content} , } {
\href{/docs/reference/foundations/none/}{none}
\href{/docs/reference/foundations/content/}{content} , } {
\href{/docs/reference/foundations/content/}{content} , }

) -\textgreater{} \href{/docs/reference/foundations/content/}{content}

\begin{verbatim}
#set heading(numbering: "1.")

#show outline.entry.where(
  level: 1
): it => {
  v(12pt, weak: true)
  strong(it)
}

#outline(indent: auto)

= Introduction
= Background
== History
== State of the Art
= Analysis
== Setup
\end{verbatim}

\includegraphics[width=5in,height=\textheight,keepaspectratio]{/assets/docs/z5yX2QHZa1YP1epncxVx1wAAAAAAAAAA.png}

\paragraph{\texorpdfstring{\texttt{\ level\ }}{ level }}\label{definitions-entry-level}

\href{/docs/reference/foundations/int/}{int}

{Required} {{ Positional }}

\phantomsection\label{definitions-entry-level-positional-tooltip}
Positional parameters are specified in order, without names.

The nesting level of this outline entry. Starts at
\texttt{\ }{\texttt{\ 1\ }}\texttt{\ } for top-level entries.

\paragraph{\texorpdfstring{\texttt{\ element\ }}{ element }}\label{definitions-entry-element}

\href{/docs/reference/foundations/content/}{content}

{Required} {{ Positional }}

\phantomsection\label{definitions-entry-element-positional-tooltip}
Positional parameters are specified in order, without names.

The element this entry refers to. Its location will be available through
the
\href{/docs/reference/foundations/content/\#definitions-location}{\texttt{\ location\ }}
method on content and can be \href{/docs/reference/model/link/}{linked}
to.

\paragraph{\texorpdfstring{\texttt{\ body\ }}{ body }}\label{definitions-entry-body}

\href{/docs/reference/foundations/content/}{content}

{Required} {{ Positional }}

\phantomsection\label{definitions-entry-body-positional-tooltip}
Positional parameters are specified in order, without names.

The content which is displayed in place of the referred element at its
entry in the outline. For a heading, this would be its number followed
by the heading\textquotesingle s title, for example.

\paragraph{\texorpdfstring{\texttt{\ fill\ }}{ fill }}\label{definitions-entry-fill}

\href{/docs/reference/foundations/none/}{none} {or}
\href{/docs/reference/foundations/content/}{content}

{Required} {{ Positional }}

\phantomsection\label{definitions-entry-fill-positional-tooltip}
Positional parameters are specified in order, without names.

The content used to fill the space between the element\textquotesingle s
outline and its page number, as defined by the outline element this
entry is located in. When \texttt{\ }{\texttt{\ none\ }}\texttt{\ } ,
empty space is inserted in that gap instead.

Note that, when using show rules to override outline entries, it is
recommended to wrap the filling content in a
\href{/docs/reference/layout/box/}{\texttt{\ box\ }} with fractional
width. For example,
\texttt{\ }{\texttt{\ box\ }}\texttt{\ }{\texttt{\ (\ }}\texttt{\ width\ }{\texttt{\ :\ }}\texttt{\ }{\texttt{\ 1fr\ }}\texttt{\ }{\texttt{\ ,\ }}\texttt{\ }{\texttt{\ repeat\ }}\texttt{\ }{\texttt{\ {[}\ }}\texttt{\ -\ }{\texttt{\ {]}\ }}\texttt{\ }{\texttt{\ )\ }}\texttt{\ }
would show precisely as many \texttt{\ -\ } characters as necessary to
fill a particular gap.

\paragraph{\texorpdfstring{\texttt{\ page\ }}{ page }}\label{definitions-entry-page}

\href{/docs/reference/foundations/content/}{content}

{Required} {{ Positional }}

\phantomsection\label{definitions-entry-page-positional-tooltip}
Positional parameters are specified in order, without names.

The page number of the element this entry links to, formatted with the
numbering set for the referenced page.

\href{/docs/reference/model/numbering/}{\pandocbounded{\includesvg[keepaspectratio]{/assets/icons/16-arrow-right.svg}}}

{ Numbering } { Previous page }

\href{/docs/reference/model/par/}{\pandocbounded{\includesvg[keepaspectratio]{/assets/icons/16-arrow-right.svg}}}

{ Paragraph } { Next page }


\section{Docs LaTeX/typst.app/docs/reference/model/footnote.tex}
\title{typst.app/docs/reference/model/footnote}

\begin{itemize}
\tightlist
\item
  \href{/docs}{\includesvg[width=0.16667in,height=0.16667in]{/assets/icons/16-docs-dark.svg}}
\item
  \includesvg[width=0.16667in,height=0.16667in]{/assets/icons/16-arrow-right.svg}
\item
  \href{/docs/reference/}{Reference}
\item
  \includesvg[width=0.16667in,height=0.16667in]{/assets/icons/16-arrow-right.svg}
\item
  \href{/docs/reference/model/}{Model}
\item
  \includesvg[width=0.16667in,height=0.16667in]{/assets/icons/16-arrow-right.svg}
\item
  \href{/docs/reference/model/footnote/}{Footnote}
\end{itemize}

\section{\texorpdfstring{\texttt{\ footnote\ } {{ Element
}}}{ footnote   Element }}\label{summary}

\phantomsection\label{element-tooltip}
Element functions can be customized with \texttt{\ set\ } and
\texttt{\ show\ } rules.

A footnote.

Includes additional remarks and references on the same page with
footnotes. A footnote will insert a superscript number that links to the
note at the bottom of the page. Notes are numbered sequentially
throughout your document and can break across multiple pages.

To customize the appearance of the entry in the footnote listing, see
\href{/docs/reference/model/footnote/\#definitions-entry}{\texttt{\ footnote.entry\ }}
. The footnote itself is realized as a normal superscript, so you can
use a set rule on the
\href{/docs/reference/text/super/}{\texttt{\ super\ }} function to
customize it. You can also apply a show rule to customize only the
footnote marker (superscript number) in the running text.

\subsection{Example}\label{example}

\begin{verbatim}
Check the docs for more details.
#footnote[https://typst.app/docs]
\end{verbatim}

\includegraphics[width=5in,height=\textheight,keepaspectratio]{/assets/docs/Rux1n4IPwOkOpn1s57WxpAAAAAAAAAAA.png}

The footnote automatically attaches itself to the preceding word, even
if there is a space before it in the markup. To force space, you can use
the string
\texttt{\ }{\texttt{\ \#\ }}\texttt{\ }{\texttt{\ "\ "\ }}\texttt{\ } or
explicit \href{/docs/reference/layout/h/}{horizontal spacing} .

By giving a label to a footnote, you can have multiple references to it.

\begin{verbatim}
You can edit Typst documents online.
#footnote[https://typst.app/app] <fn>
Checkout Typst's website. @fn
And the online app. #footnote(<fn>)
\end{verbatim}

\includegraphics[width=5in,height=\textheight,keepaspectratio]{/assets/docs/xECSHtr0VhzFh5onpw48GQAAAAAAAAAA.png}

\emph{Note:} Set and show rules in the scope where \texttt{\ footnote\ }
is called may not apply to the footnote\textquotesingle s content. See
\href{https://github.com/typst/typst/issues/1467\#issuecomment-1588799440}{here}
for more information.

\subsection{\texorpdfstring{{ Parameters
}}{ Parameters }}\label{parameters}

\phantomsection\label{parameters-tooltip}
Parameters are the inputs to a function. They are specified in
parentheses after the function name.

{ footnote } (

{ \hyperref[parameters-numbering]{numbering :}
\href{/docs/reference/foundations/str/}{str}
\href{/docs/reference/foundations/function/}{function} , } {
\href{/docs/reference/foundations/label/}{label}
\href{/docs/reference/foundations/content/}{content} , }

) -\textgreater{} \href{/docs/reference/foundations/content/}{content}

\subsubsection{\texorpdfstring{\texttt{\ numbering\ }}{ numbering }}\label{parameters-numbering}

\href{/docs/reference/foundations/str/}{str} {or}
\href{/docs/reference/foundations/function/}{function}

{{ Settable }}

\phantomsection\label{parameters-numbering-settable-tooltip}
Settable parameters can be customized for all following uses of the
function with a \texttt{\ set\ } rule.

How to number footnotes.

By default, the footnote numbering continues throughout your document.
If you prefer per-page footnote numbering, you can reset the footnote
\href{/docs/reference/introspection/counter/}{counter} in the page
\href{/docs/reference/layout/page/\#parameters-header}{header} . In the
future, there might be a simpler way to achieve this.

Default: \texttt{\ }{\texttt{\ "1"\ }}\texttt{\ }

\includesvg[width=0.16667in,height=0.16667in]{/assets/icons/16-arrow-right.svg}
View example

\begin{verbatim}
#set footnote(numbering: "*")

Footnotes:
#footnote[Star],
#footnote[Dagger]
\end{verbatim}

\includegraphics[width=5in,height=\textheight,keepaspectratio]{/assets/docs/CVlSBedIWhhzGwE8LefQmwAAAAAAAAAA.png}

\subsubsection{\texorpdfstring{\texttt{\ body\ }}{ body }}\label{parameters-body}

\href{/docs/reference/foundations/label/}{label} {or}
\href{/docs/reference/foundations/content/}{content}

{Required} {{ Positional }}

\phantomsection\label{parameters-body-positional-tooltip}
Positional parameters are specified in order, without names.

The content to put into the footnote. Can also be the label of another
footnote this one should point to.

\subsection{\texorpdfstring{{ Definitions
}}{ Definitions }}\label{definitions}

\phantomsection\label{definitions-tooltip}
Functions and types and can have associated definitions. These are
accessed by specifying the function or type, followed by a period, and
then the definition\textquotesingle s name.

\subsubsection{\texorpdfstring{\texttt{\ entry\ } {{ Element
}}}{ entry   Element }}\label{definitions-entry}

\phantomsection\label{definitions-entry-element-tooltip}
Element functions can be customized with \texttt{\ set\ } and
\texttt{\ show\ } rules.

An entry in a footnote list.

This function is not intended to be called directly. Instead, it is used
in set and show rules to customize footnote listings.

footnote { . } { entry } (

{ \href{/docs/reference/foundations/content/}{content} , } {
\hyperref[definitions-entry-parameters-separator]{separator :}
\href{/docs/reference/foundations/content/}{content} , } {
\hyperref[definitions-entry-parameters-clearance]{clearance :}
\href{/docs/reference/layout/length/}{length} , } {
\hyperref[definitions-entry-parameters-gap]{gap :}
\href{/docs/reference/layout/length/}{length} , } {
\hyperref[definitions-entry-parameters-indent]{indent :}
\href{/docs/reference/layout/length/}{length} , }

) -\textgreater{} \href{/docs/reference/foundations/content/}{content}

\begin{verbatim}
#show footnote.entry: set text(red)

My footnote listing
#footnote[It's down here]
has red text!
\end{verbatim}

\includegraphics[width=5in,height=\textheight,keepaspectratio]{/assets/docs/OQcOLIwIWFG81ucXxeuiVwAAAAAAAAAA.png}

\emph{Note:} Footnote entry properties must be uniform across each page
run (a page run is a sequence of pages without an explicit pagebreak in
between). For this reason, set and show rules for footnote entries
should be defined before any page content, typically at the very start
of the document.

\paragraph{\texorpdfstring{\texttt{\ note\ }}{ note }}\label{definitions-entry-note}

\href{/docs/reference/foundations/content/}{content}

{Required} {{ Positional }}

\phantomsection\label{definitions-entry-note-positional-tooltip}
Positional parameters are specified in order, without names.

The footnote for this entry. It\textquotesingle s location can be used
to determine the footnote counter state.

\includesvg[width=0.16667in,height=0.16667in]{/assets/icons/16-arrow-right.svg}
View example

\begin{verbatim}
#show footnote.entry: it => {
  let loc = it.note.location()
  numbering(
    "1: ",
    ..counter(footnote).at(loc),
  )
  it.note.body
}

Customized #footnote[Hello]
listing #footnote[World! 🌏]
\end{verbatim}

\includegraphics[width=5in,height=\textheight,keepaspectratio]{/assets/docs/pITXewKM6sSB5ed44fUp7wAAAAAAAAAA.png}

\paragraph{\texorpdfstring{\texttt{\ separator\ }}{ separator }}\label{definitions-entry-separator}

\href{/docs/reference/foundations/content/}{content}

{{ Settable }}

\phantomsection\label{definitions-entry-separator-settable-tooltip}
Settable parameters can be customized for all following uses of the
function with a \texttt{\ set\ } rule.

The separator between the document body and the footnote listing.

Default:
\texttt{\ }{\texttt{\ line\ }}\texttt{\ }{\texttt{\ (\ }}\texttt{\ length\ }{\texttt{\ :\ }}\texttt{\ }{\texttt{\ 30\%\ }}\texttt{\ }{\texttt{\ +\ }}\texttt{\ }{\texttt{\ 0pt\ }}\texttt{\ }{\texttt{\ ,\ }}\texttt{\ stroke\ }{\texttt{\ :\ }}\texttt{\ }{\texttt{\ 0.5pt\ }}\texttt{\ }{\texttt{\ )\ }}\texttt{\ }

\includesvg[width=0.16667in,height=0.16667in]{/assets/icons/16-arrow-right.svg}
View example

\begin{verbatim}
#set footnote.entry(
  separator: repeat[.]
)

Testing a different separator.
#footnote[
  Unconventional, but maybe
  not that bad?
]
\end{verbatim}

\includegraphics[width=5in,height=\textheight,keepaspectratio]{/assets/docs/2BZbfiOf16u6fje-JM2KhwAAAAAAAAAA.png}

\paragraph{\texorpdfstring{\texttt{\ clearance\ }}{ clearance }}\label{definitions-entry-clearance}

\href{/docs/reference/layout/length/}{length}

{{ Settable }}

\phantomsection\label{definitions-entry-clearance-settable-tooltip}
Settable parameters can be customized for all following uses of the
function with a \texttt{\ set\ } rule.

The amount of clearance between the document body and the separator.

Default: \texttt{\ }{\texttt{\ 1em\ }}\texttt{\ }

\includesvg[width=0.16667in,height=0.16667in]{/assets/icons/16-arrow-right.svg}
View example

\begin{verbatim}
#set footnote.entry(clearance: 3em)

Footnotes also need ...
#footnote[
  ... some space to breathe.
]
\end{verbatim}

\includegraphics[width=5in,height=\textheight,keepaspectratio]{/assets/docs/jGI_-Yxsz0NqX0MjmZS_qQAAAAAAAAAA.png}

\paragraph{\texorpdfstring{\texttt{\ gap\ }}{ gap }}\label{definitions-entry-gap}

\href{/docs/reference/layout/length/}{length}

{{ Settable }}

\phantomsection\label{definitions-entry-gap-settable-tooltip}
Settable parameters can be customized for all following uses of the
function with a \texttt{\ set\ } rule.

The gap between footnote entries.

Default: \texttt{\ }{\texttt{\ 0.5em\ }}\texttt{\ }

\includesvg[width=0.16667in,height=0.16667in]{/assets/icons/16-arrow-right.svg}
View example

\begin{verbatim}
#set footnote.entry(gap: 0.8em)

Footnotes:
#footnote[Spaced],
#footnote[Apart]
\end{verbatim}

\includegraphics[width=5in,height=\textheight,keepaspectratio]{/assets/docs/3sggupXU7L_bO6KYRBDWHQAAAAAAAAAA.png}

\paragraph{\texorpdfstring{\texttt{\ indent\ }}{ indent }}\label{definitions-entry-indent}

\href{/docs/reference/layout/length/}{length}

{{ Settable }}

\phantomsection\label{definitions-entry-indent-settable-tooltip}
Settable parameters can be customized for all following uses of the
function with a \texttt{\ set\ } rule.

The indent of each footnote entry.

Default: \texttt{\ }{\texttt{\ 1em\ }}\texttt{\ }

\includesvg[width=0.16667in,height=0.16667in]{/assets/icons/16-arrow-right.svg}
View example

\begin{verbatim}
#set footnote.entry(indent: 0em)

Footnotes:
#footnote[No],
#footnote[Indent]
\end{verbatim}

\includegraphics[width=5in,height=\textheight,keepaspectratio]{/assets/docs/-zkE_ejQDpF6KTPTlZZ3gwAAAAAAAAAA.png}

\href{/docs/reference/model/figure/}{\pandocbounded{\includesvg[keepaspectratio]{/assets/icons/16-arrow-right.svg}}}

{ Figure } { Previous page }

\href{/docs/reference/model/heading/}{\pandocbounded{\includesvg[keepaspectratio]{/assets/icons/16-arrow-right.svg}}}

{ Heading } { Next page }


\section{Docs LaTeX/typst.app/docs/reference/model/strong.tex}
\title{typst.app/docs/reference/model/strong}

\begin{itemize}
\tightlist
\item
  \href{/docs}{\includesvg[width=0.16667in,height=0.16667in]{/assets/icons/16-docs-dark.svg}}
\item
  \includesvg[width=0.16667in,height=0.16667in]{/assets/icons/16-arrow-right.svg}
\item
  \href{/docs/reference/}{Reference}
\item
  \includesvg[width=0.16667in,height=0.16667in]{/assets/icons/16-arrow-right.svg}
\item
  \href{/docs/reference/model/}{Model}
\item
  \includesvg[width=0.16667in,height=0.16667in]{/assets/icons/16-arrow-right.svg}
\item
  \href{/docs/reference/model/strong/}{Strong Emphasis}
\end{itemize}

\section{\texorpdfstring{\texttt{\ strong\ } {{ Element
}}}{ strong   Element }}\label{summary}

\phantomsection\label{element-tooltip}
Element functions can be customized with \texttt{\ set\ } and
\texttt{\ show\ } rules.

Strongly emphasizes content by increasing the font weight.

Increases the current font weight by a given \texttt{\ delta\ } .

\subsection{Example}\label{example}

\begin{verbatim}
This is *strong.* \
This is #strong[too.] \

#show strong: set text(red)
And this is *evermore.*
\end{verbatim}

\includegraphics[width=5in,height=\textheight,keepaspectratio]{/assets/docs/8PFV4SUNXNbbYe9uHW1ITAAAAAAAAAAA.png}

\subsection{Syntax}\label{syntax}

This function also has dedicated syntax: To strongly emphasize content,
simply enclose it in stars/asterisks ( \texttt{\ *\ } ). Note that this
only works at word boundaries. To strongly emphasize part of a word, you
have to use the function.

\subsection{\texorpdfstring{{ Parameters
}}{ Parameters }}\label{parameters}

\phantomsection\label{parameters-tooltip}
Parameters are the inputs to a function. They are specified in
parentheses after the function name.

{ strong } (

{ \hyperref[parameters-delta]{delta :}
\href{/docs/reference/foundations/int/}{int} , } {
\href{/docs/reference/foundations/content/}{content} , }

) -\textgreater{} \href{/docs/reference/foundations/content/}{content}

\subsubsection{\texorpdfstring{\texttt{\ delta\ }}{ delta }}\label{parameters-delta}

\href{/docs/reference/foundations/int/}{int}

{{ Settable }}

\phantomsection\label{parameters-delta-settable-tooltip}
Settable parameters can be customized for all following uses of the
function with a \texttt{\ set\ } rule.

The delta to apply on the font weight.

Default: \texttt{\ }{\texttt{\ 300\ }}\texttt{\ }

\includesvg[width=0.16667in,height=0.16667in]{/assets/icons/16-arrow-right.svg}
View example

\begin{verbatim}
#set strong(delta: 0)
No *effect!*
\end{verbatim}

\includegraphics[width=5in,height=\textheight,keepaspectratio]{/assets/docs/SC7LmnRUxtrvxQL331fpfAAAAAAAAAAA.png}

\subsubsection{\texorpdfstring{\texttt{\ body\ }}{ body }}\label{parameters-body}

\href{/docs/reference/foundations/content/}{content}

{Required} {{ Positional }}

\phantomsection\label{parameters-body-positional-tooltip}
Positional parameters are specified in order, without names.

The content to strongly emphasize.

\href{/docs/reference/model/ref/}{\pandocbounded{\includesvg[keepaspectratio]{/assets/icons/16-arrow-right.svg}}}

{ Reference } { Previous page }

\href{/docs/reference/model/table/}{\pandocbounded{\includesvg[keepaspectratio]{/assets/icons/16-arrow-right.svg}}}

{ Table } { Next page }


\section{Docs LaTeX/typst.app/docs/reference/model/emph.tex}
\title{typst.app/docs/reference/model/emph}

\begin{itemize}
\tightlist
\item
  \href{/docs}{\includesvg[width=0.16667in,height=0.16667in]{/assets/icons/16-docs-dark.svg}}
\item
  \includesvg[width=0.16667in,height=0.16667in]{/assets/icons/16-arrow-right.svg}
\item
  \href{/docs/reference/}{Reference}
\item
  \includesvg[width=0.16667in,height=0.16667in]{/assets/icons/16-arrow-right.svg}
\item
  \href{/docs/reference/model/}{Model}
\item
  \includesvg[width=0.16667in,height=0.16667in]{/assets/icons/16-arrow-right.svg}
\item
  \href{/docs/reference/model/emph/}{Emphasis}
\end{itemize}

\section{\texorpdfstring{\texttt{\ emph\ } {{ Element
}}}{ emph   Element }}\label{summary}

\phantomsection\label{element-tooltip}
Element functions can be customized with \texttt{\ set\ } and
\texttt{\ show\ } rules.

Emphasizes content by toggling italics.

\begin{itemize}
\tightlist
\item
  If the current
  \href{/docs/reference/text/text/\#parameters-style}{text style} is
  \texttt{\ }{\texttt{\ "normal"\ }}\texttt{\ } , this turns it into
  \texttt{\ }{\texttt{\ "italic"\ }}\texttt{\ } .
\item
  If it is already \texttt{\ }{\texttt{\ "italic"\ }}\texttt{\ } or
  \texttt{\ }{\texttt{\ "oblique"\ }}\texttt{\ } , it turns it back to
  \texttt{\ }{\texttt{\ "normal"\ }}\texttt{\ } .
\end{itemize}

\subsection{Example}\label{example}

\begin{verbatim}
This is _emphasized._ \
This is #emph[too.]

#show emph: it => {
  text(blue, it.body)
}

This is _emphasized_ differently.
\end{verbatim}

\includegraphics[width=5in,height=\textheight,keepaspectratio]{/assets/docs/p3cGCgaJdrkrScOita7QfgAAAAAAAAAA.png}

\subsection{Syntax}\label{syntax}

This function also has dedicated syntax: To emphasize content, simply
enclose it in underscores ( \texttt{\ \_\ } ). Note that this only works
at word boundaries. To emphasize part of a word, you have to use the
function.

\subsection{\texorpdfstring{{ Parameters
}}{ Parameters }}\label{parameters}

\phantomsection\label{parameters-tooltip}
Parameters are the inputs to a function. They are specified in
parentheses after the function name.

{ emph } (

{ \href{/docs/reference/foundations/content/}{content} }

) -\textgreater{} \href{/docs/reference/foundations/content/}{content}

\subsubsection{\texorpdfstring{\texttt{\ body\ }}{ body }}\label{parameters-body}

\href{/docs/reference/foundations/content/}{content}

{Required} {{ Positional }}

\phantomsection\label{parameters-body-positional-tooltip}
Positional parameters are specified in order, without names.

The content to emphasize.

\href{/docs/reference/model/document/}{\pandocbounded{\includesvg[keepaspectratio]{/assets/icons/16-arrow-right.svg}}}

{ Document } { Previous page }

\href{/docs/reference/model/figure/}{\pandocbounded{\includesvg[keepaspectratio]{/assets/icons/16-arrow-right.svg}}}

{ Figure } { Next page }


\section{Docs LaTeX/typst.app/docs/reference/model/cite.tex}
\title{typst.app/docs/reference/model/cite}

\begin{itemize}
\tightlist
\item
  \href{/docs}{\includesvg[width=0.16667in,height=0.16667in]{/assets/icons/16-docs-dark.svg}}
\item
  \includesvg[width=0.16667in,height=0.16667in]{/assets/icons/16-arrow-right.svg}
\item
  \href{/docs/reference/}{Reference}
\item
  \includesvg[width=0.16667in,height=0.16667in]{/assets/icons/16-arrow-right.svg}
\item
  \href{/docs/reference/model/}{Model}
\item
  \includesvg[width=0.16667in,height=0.16667in]{/assets/icons/16-arrow-right.svg}
\item
  \href{/docs/reference/model/cite/}{Cite}
\end{itemize}

\section{\texorpdfstring{\texttt{\ cite\ } {{ Element
}}}{ cite   Element }}\label{summary}

\phantomsection\label{element-tooltip}
Element functions can be customized with \texttt{\ set\ } and
\texttt{\ show\ } rules.

Cite a work from the bibliography.

Before you starting citing, you need to add a
\href{/docs/reference/model/bibliography/}{bibliography} somewhere in
your document.

\subsection{Example}\label{example}

\begin{verbatim}
This was already noted by
pirates long ago. @arrgh

Multiple sources say ...
@arrgh @netwok.

You can also call `cite`
explicitly. #cite(<arrgh>)

#bibliography("works.bib")
\end{verbatim}

\includegraphics[width=5in,height=\textheight,keepaspectratio]{/assets/docs/VelsLOKdUATbBc5AK51_FgAAAAAAAAAA.png}

If your source name contains certain characters such as slashes, which
are not recognized by the \texttt{\ \textless{}\textgreater{}\ } syntax,
you can explicitly call \texttt{\ label\ } instead.

\begin{verbatim}
Computer Modern is an example of a modernist serif typeface.
#cite(label("DBLP:books/lib/Knuth86a")).
\end{verbatim}

\subsection{Syntax}\label{syntax}

This function indirectly has dedicated syntax.
\href{/docs/reference/model/ref/}{References} can be used to cite works
from the bibliography. The label then corresponds to the citation key.

\subsection{\texorpdfstring{{ Parameters
}}{ Parameters }}\label{parameters}

\phantomsection\label{parameters-tooltip}
Parameters are the inputs to a function. They are specified in
parentheses after the function name.

{ cite } (

{ \href{/docs/reference/foundations/label/}{label} , } {
\hyperref[parameters-supplement]{supplement :}
\href{/docs/reference/foundations/none/}{none}
\href{/docs/reference/foundations/content/}{content} , } {
\hyperref[parameters-form]{form :}
\href{/docs/reference/foundations/none/}{none}
\href{/docs/reference/foundations/str/}{str} , } {
\hyperref[parameters-style]{style :}
\href{/docs/reference/foundations/auto/}{auto}
\href{/docs/reference/foundations/str/}{str} , }

) -\textgreater{} \href{/docs/reference/foundations/content/}{content}

\subsubsection{\texorpdfstring{\texttt{\ key\ }}{ key }}\label{parameters-key}

\href{/docs/reference/foundations/label/}{label}

{Required} {{ Positional }}

\phantomsection\label{parameters-key-positional-tooltip}
Positional parameters are specified in order, without names.

The citation key that identifies the entry in the bibliography that
shall be cited, as a label.

\includesvg[width=0.16667in,height=0.16667in]{/assets/icons/16-arrow-right.svg}
View example

\begin{verbatim}
// All the same
@netwok \
#cite(<netwok>) \
#cite(label("netwok"))
\end{verbatim}

\includegraphics[width=5in,height=\textheight,keepaspectratio]{/assets/docs/fyv1W7ZKnlPyBVM6_1DvjgAAAAAAAAAA.png}

\subsubsection{\texorpdfstring{\texttt{\ supplement\ }}{ supplement }}\label{parameters-supplement}

\href{/docs/reference/foundations/none/}{none} {or}
\href{/docs/reference/foundations/content/}{content}

{{ Settable }}

\phantomsection\label{parameters-supplement-settable-tooltip}
Settable parameters can be customized for all following uses of the
function with a \texttt{\ set\ } rule.

A supplement for the citation such as page or chapter number.

In reference syntax, the supplement can be added in square brackets:

Default: \texttt{\ }{\texttt{\ none\ }}\texttt{\ }

\includesvg[width=0.16667in,height=0.16667in]{/assets/icons/16-arrow-right.svg}
View example

\begin{verbatim}
This has been proven. @distress[p.~7]

#bibliography("works.bib")
\end{verbatim}

\includegraphics[width=5in,height=\textheight,keepaspectratio]{/assets/docs/yJ9a0jIezaQUawq1k-YqqwAAAAAAAAAA.png}

\subsubsection{\texorpdfstring{\texttt{\ form\ }}{ form }}\label{parameters-form}

\href{/docs/reference/foundations/none/}{none} {or}
\href{/docs/reference/foundations/str/}{str}

{{ Settable }}

\phantomsection\label{parameters-form-settable-tooltip}
Settable parameters can be customized for all following uses of the
function with a \texttt{\ set\ } rule.

The kind of citation to produce. Different forms are useful in different
scenarios: A normal citation is useful as a source at the end of a
sentence, while a "prose" citation is more suitable for inclusion in the
flow of text.

If set to \texttt{\ }{\texttt{\ none\ }}\texttt{\ } , the cited work is
included in the bibliography, but nothing will be displayed.

\begin{longtable}[]{@{}ll@{}}
\toprule\noalign{}
Variant & Details \\
\midrule\noalign{}
\endhead
\bottomrule\noalign{}
\endlastfoot
\texttt{\ "\ normal\ "\ } & Display in the standard way for the active
style. \\
\texttt{\ "\ prose\ "\ } & Produces a citation that is suitable for
inclusion in a sentence. \\
\texttt{\ "\ full\ "\ } & Mimics a bibliography entry, with full
information about the cited work. \\
\texttt{\ "\ author\ "\ } & Shows only the cited work\textquotesingle s
author(s). \\
\texttt{\ "\ year\ "\ } & Shows only the cited work\textquotesingle s
year. \\
\end{longtable}

Default: \texttt{\ }{\texttt{\ "normal"\ }}\texttt{\ }

\includesvg[width=0.16667in,height=0.16667in]{/assets/icons/16-arrow-right.svg}
View example

\begin{verbatim}
#cite(<netwok>, form: "prose")
show the outsized effects of
pirate life on the human psyche.
\end{verbatim}

\includegraphics[width=5in,height=\textheight,keepaspectratio]{/assets/docs/xCamzQ_SHz1kKaOAByx_rAAAAAAAAAAA.png}

\subsubsection{\texorpdfstring{\texttt{\ style\ }}{ style }}\label{parameters-style}

\href{/docs/reference/foundations/auto/}{auto} {or}
\href{/docs/reference/foundations/str/}{str}

{{ Settable }}

\phantomsection\label{parameters-style-settable-tooltip}
Settable parameters can be customized for all following uses of the
function with a \texttt{\ set\ } rule.

The citation style.

Should be either \texttt{\ }{\texttt{\ auto\ }}\texttt{\ } , one of the
built-in styles (see below) or a path to a
\href{https://citationstyles.org/}{CSL file} . Some of the styles listed
below appear twice, once with their full name and once with a short
alias.

When set to \texttt{\ }{\texttt{\ auto\ }}\texttt{\ } , automatically
use the
\href{/docs/reference/model/bibliography/\#parameters-style}{bibliography\textquotesingle s
style} for the citations.

\includesvg[width=0.16667in,height=0.16667in]{/assets/icons/16-arrow-right.svg}
View options

\begin{longtable}[]{@{}ll@{}}
\toprule\noalign{}
Variant & Details \\
\midrule\noalign{}
\endhead
\bottomrule\noalign{}
\endlastfoot
\texttt{\ "\ alphanumeric\ "\ } & Alphanumeric \\
\texttt{\ "\ american-anthropological-association\ "\ } & American
Anthropological Association \\
\texttt{\ "\ american-chemical-society\ "\ } & American Chemical
Society \\
\texttt{\ "\ american-geophysical-union\ "\ } & American Geophysical
Union \\
\texttt{\ "\ american-institute-of-aeronautics-and-astronautics\ "\ } &
American Institute of Aeronautics and Astronautics \\
\texttt{\ "\ american-institute-of-physics\ "\ } & American Institute of
Physics 4th edition \\
\texttt{\ "\ american-medical-association\ "\ } & American Medical
Association 11th edition \\
\texttt{\ "\ american-meteorological-society\ "\ } & American
Meteorological Society \\
\texttt{\ "\ american-physics-society\ "\ } & American Physical
Society \\
\texttt{\ "\ american-physiological-society\ "\ } & American
Physiological Society \\
\texttt{\ "\ american-political-science-association\ "\ } & American
Political Science Association \\
\texttt{\ "\ american-psychological-association\ "\ } & American
Psychological Association 7th edition \\
\texttt{\ "\ american-society-for-microbiology\ "\ } & American Society
for Microbiology \\
\texttt{\ "\ american-society-of-civil-engineers\ "\ } & American
Society of Civil Engineers \\
\texttt{\ "\ american-society-of-mechanical-engineers\ "\ } & American
Society of Mechanical Engineers \\
\texttt{\ "\ american-sociological-association\ "\ } & American
Sociological Association 6th/7th edition \\
\texttt{\ "\ angewandte-chemie\ "\ } & Angewandte Chemie International
Edition \\
\texttt{\ "\ annual-reviews\ "\ } & Annual Reviews (sorted by order of
appearance) \\
\texttt{\ "\ annual-reviews-author-date\ "\ } & Annual Reviews
(author-date) \\
\texttt{\ "\ associacao-brasileira-de-normas-tecnicas\ "\ } &
Associação Brasileira de Normas Técnicas (Português - Brasil) \\
\texttt{\ "\ association-for-computing-machinery\ "\ } & Association for
Computing Machinery \\
\texttt{\ "\ biomed-central\ "\ } & BioMed Central \\
\texttt{\ "\ bristol-university-press\ "\ } & Bristol University
Press \\
\texttt{\ "\ british-medical-journal\ "\ } & BMJ \\
\texttt{\ "\ cell\ "\ } & Cell \\
\texttt{\ "\ chicago-author-date\ "\ } & Chicago Manual of Style 17th
edition (author-date) \\
\texttt{\ "\ chicago-fullnotes\ "\ } & Chicago Manual of Style 17th
edition (full note) \\
\texttt{\ "\ chicago-notes\ "\ } & Chicago Manual of Style 17th edition
(note) \\
\texttt{\ "\ copernicus\ "\ } & Copernicus Publications \\
\texttt{\ "\ council-of-science-editors\ "\ } & Council of Science
Editors, Citation-Sequence (numeric, brackets) \\
\texttt{\ "\ council-of-science-editors-author-date\ "\ } & Council of
Science Editors, Name-Year (author-date) \\
\texttt{\ "\ current-opinion\ "\ } & Current Opinion journals \\
\texttt{\ "\ deutsche-gesellschaft-für-psychologie\ "\ } & Deutsche
Gesellschaft für Psychologie 5. Auflage (Deutsch) \\
\texttt{\ "\ deutsche-sprache\ "\ } & Deutsche Sprache (Deutsch) \\
\texttt{\ "\ elsevier-harvard\ "\ } & Elsevier - Harvard (with
titles) \\
\texttt{\ "\ elsevier-vancouver\ "\ } & Elsevier - Vancouver \\
\texttt{\ "\ elsevier-with-titles\ "\ } & Elsevier (numeric, with
titles) \\
\texttt{\ "\ frontiers\ "\ } & Frontiers journals \\
\texttt{\ "\ future-medicine\ "\ } & Future Medicine journals \\
\texttt{\ "\ future-science\ "\ } & Future Science Group \\
\texttt{\ "\ gb-7714-2005-numeric\ "\ } & China National Standard GB/T
7714-2005 (numeric, 中æ--‡) \\
\texttt{\ "\ gb-7714-2015-author-date\ "\ } & China National Standard
GB/T 7714-2015 (author-date, 中æ--‡) \\
\texttt{\ "\ gb-7714-2015-note\ "\ } & China National Standard GB/T
7714-2015 (note, 中æ--‡) \\
\texttt{\ "\ gb-7714-2015-numeric\ "\ } & China National Standard GB/T
7714-2015 (numeric, 中æ--‡) \\
\texttt{\ "\ gost-r-705-2008-numeric\ "\ } & Russian GOST R 7.0.5-2008
(numeric) \\
\texttt{\ "\ harvard-cite-them-right\ "\ } & Cite Them Right 12th
edition - Harvard \\
\texttt{\ "\ institute-of-electrical-and-electronics-engineers\ "\ } &
IEEE \\
\texttt{\ "\ institute-of-physics-numeric\ "\ } & Institute of Physics
(numeric) \\
\texttt{\ "\ iso-690-author-date\ "\ } & ISO-690 (author-date,
English) \\
\texttt{\ "\ iso-690-numeric\ "\ } & ISO-690 (numeric, English) \\
\texttt{\ "\ karger\ "\ } & Karger journals \\
\texttt{\ "\ mary-ann-liebert-vancouver\ "\ } & Mary Ann Liebert -
Vancouver \\
\texttt{\ "\ modern-humanities-research-association\ "\ } & Modern
Humanities Research Association 4th edition (note with bibliography) \\
\texttt{\ "\ modern-language-association\ "\ } & Modern Language
Association 9th edition \\
\texttt{\ "\ modern-language-association-8\ "\ } & Modern Language
Association 8th edition \\
\texttt{\ "\ multidisciplinary-digital-publishing-institute\ "\ } &
Multidisciplinary Digital Publishing Institute \\
\texttt{\ "\ nature\ "\ } & Nature \\
\texttt{\ "\ pensoft\ "\ } & Pensoft Journals \\
\texttt{\ "\ public-library-of-science\ "\ } & Public Library of
Science \\
\texttt{\ "\ royal-society-of-chemistry\ "\ } & Royal Society of
Chemistry \\
\texttt{\ "\ sage-vancouver\ "\ } & SAGE - Vancouver \\
\texttt{\ "\ sist02\ "\ } & SIST02 (æ---¥æœ¬èªž) \\
\texttt{\ "\ spie\ "\ } & SPIE journals \\
\texttt{\ "\ springer-basic\ "\ } & Springer - Basic (numeric,
brackets) \\
\texttt{\ "\ springer-basic-author-date\ "\ } & Springer - Basic
(author-date) \\
\texttt{\ "\ springer-fachzeitschriften-medizin-psychologie\ "\ } &
Springer - Fachzeitschriften Medizin Psychologie (Deutsch) \\
\texttt{\ "\ springer-humanities-author-date\ "\ } & Springer -
Humanities (author-date) \\
\texttt{\ "\ springer-lecture-notes-in-computer-science\ "\ } & Springer
- Lecture Notes in Computer Science \\
\texttt{\ "\ springer-mathphys\ "\ } & Springer - MathPhys (numeric,
brackets) \\
\texttt{\ "\ springer-socpsych-author-date\ "\ } & Springer - SocPsych
(author-date) \\
\texttt{\ "\ springer-vancouver\ "\ } & Springer - Vancouver
(brackets) \\
\texttt{\ "\ taylor-and-francis-chicago-author-date\ "\ } & Taylor \&
Francis - Chicago Manual of Style (author-date) \\
\texttt{\ "\ taylor-and-francis-national-library-of-medicine\ "\ } &
Taylor \& Francis - National Library of Medicine \\
\texttt{\ "\ the-institution-of-engineering-and-technology\ "\ } & The
Institution of Engineering and Technology \\
\texttt{\ "\ the-lancet\ "\ } & The Lancet \\
\texttt{\ "\ thieme\ "\ } & Thieme-German (Deutsch) \\
\texttt{\ "\ trends\ "\ } & Trends journals \\
\texttt{\ "\ turabian-author-date\ "\ } & Turabian 9th edition
(author-date) \\
\texttt{\ "\ turabian-fullnote-8\ "\ } & Turabian 8th edition (full
note) \\
\texttt{\ "\ vancouver\ "\ } & Vancouver \\
\texttt{\ "\ vancouver-superscript\ "\ } & Vancouver (superscript) \\
\end{longtable}

Default: \texttt{\ }{\texttt{\ auto\ }}\texttt{\ }

\href{/docs/reference/model/list/}{\pandocbounded{\includesvg[keepaspectratio]{/assets/icons/16-arrow-right.svg}}}

{ Bullet List } { Previous page }

\href{/docs/reference/model/document/}{\pandocbounded{\includesvg[keepaspectratio]{/assets/icons/16-arrow-right.svg}}}

{ Document } { Next page }


\section{Docs LaTeX/typst.app/docs/reference/model/quote.tex}
\title{typst.app/docs/reference/model/quote}

\begin{itemize}
\tightlist
\item
  \href{/docs}{\includesvg[width=0.16667in,height=0.16667in]{/assets/icons/16-docs-dark.svg}}
\item
  \includesvg[width=0.16667in,height=0.16667in]{/assets/icons/16-arrow-right.svg}
\item
  \href{/docs/reference/}{Reference}
\item
  \includesvg[width=0.16667in,height=0.16667in]{/assets/icons/16-arrow-right.svg}
\item
  \href{/docs/reference/model/}{Model}
\item
  \includesvg[width=0.16667in,height=0.16667in]{/assets/icons/16-arrow-right.svg}
\item
  \href{/docs/reference/model/quote/}{Quote}
\end{itemize}

\section{\texorpdfstring{\texttt{\ quote\ } {{ Element
}}}{ quote   Element }}\label{summary}

\phantomsection\label{element-tooltip}
Element functions can be customized with \texttt{\ set\ } and
\texttt{\ show\ } rules.

Displays a quote alongside an optional attribution.

\subsection{Example}\label{example}

\begin{verbatim}
Plato is often misquoted as the author of #quote[I know that I know
nothing], however, this is a derivation form his original quote:

#set quote(block: true)

#quote(attribution: [Plato])[
  ... ἔοικα γοῦν τούτου γε σμικρῷ τινι αὐτῷ τούτῳ σοφώτερος εἶναι, ὅτι
  ἃ μὴ οἶδα οὐδὲ οἴομαι εἰδέναι.
]
#quote(attribution: [from the Henry Cary literal translation of 1897])[
  ... I seem, then, in just this little thing to be wiser than this man at
  any rate, that what I do not know I do not think I know either.
]
\end{verbatim}

\includegraphics[width=5in,height=\textheight,keepaspectratio]{/assets/docs/SJpe1zkhE_liZRMF5cAy4gAAAAAAAAAA.png}

By default block quotes are padded left and right by
\texttt{\ }{\texttt{\ 1em\ }}\texttt{\ } , alignment and padding can be
controlled with show rules:

\begin{verbatim}
#set quote(block: true)
#show quote: set align(center)
#show quote: set pad(x: 5em)

#quote[
  You cannot pass... I am a servant of the Secret Fire, wielder of the
  flame of Anor. You cannot pass. The dark fire will not avail you,
  flame of Udûn. Go back to the Shadow! You cannot pass.
]
\end{verbatim}

\includegraphics[width=5in,height=\textheight,keepaspectratio]{/assets/docs/QLNv4Pfp0zBKSvwxIfby-wAAAAAAAAAA.png}

\subsection{\texorpdfstring{{ Parameters
}}{ Parameters }}\label{parameters}

\phantomsection\label{parameters-tooltip}
Parameters are the inputs to a function. They are specified in
parentheses after the function name.

{ quote } (

{ \hyperref[parameters-block]{block :}
\href{/docs/reference/foundations/bool/}{bool} , } {
\hyperref[parameters-quotes]{quotes :}
\href{/docs/reference/foundations/auto/}{auto}
\href{/docs/reference/foundations/bool/}{bool} , } {
\hyperref[parameters-attribution]{attribution :}
\href{/docs/reference/foundations/none/}{none}
\href{/docs/reference/foundations/label/}{label}
\href{/docs/reference/foundations/content/}{content} , } {
\href{/docs/reference/foundations/content/}{content} , }

) -\textgreater{} \href{/docs/reference/foundations/content/}{content}

\subsubsection{\texorpdfstring{\texttt{\ block\ }}{ block }}\label{parameters-block}

\href{/docs/reference/foundations/bool/}{bool}

{{ Settable }}

\phantomsection\label{parameters-block-settable-tooltip}
Settable parameters can be customized for all following uses of the
function with a \texttt{\ set\ } rule.

Whether this is a block quote.

Default: \texttt{\ }{\texttt{\ false\ }}\texttt{\ }

\includesvg[width=0.16667in,height=0.16667in]{/assets/icons/16-arrow-right.svg}
View example

\begin{verbatim}
An inline citation would look like
this: #quote(
  attribution: [René Descartes]
)[
  cogito, ergo sum
], and a block equation like this:
#quote(
  block: true,
  attribution: [JFK]
)[
  Ich bin ein Berliner.
]
\end{verbatim}

\includegraphics[width=5in,height=\textheight,keepaspectratio]{/assets/docs/bYLjzIuUOzRO9HYX7xT11wAAAAAAAAAA.png}

\subsubsection{\texorpdfstring{\texttt{\ quotes\ }}{ quotes }}\label{parameters-quotes}

\href{/docs/reference/foundations/auto/}{auto} {or}
\href{/docs/reference/foundations/bool/}{bool}

{{ Settable }}

\phantomsection\label{parameters-quotes-settable-tooltip}
Settable parameters can be customized for all following uses of the
function with a \texttt{\ set\ } rule.

Whether double quotes should be added around this quote.

The double quotes used are inferred from the \texttt{\ quotes\ }
property on \href{/docs/reference/text/smartquote/}{smartquote} , which
is affected by the \texttt{\ lang\ } property on
\href{/docs/reference/text/text/}{text} .

\begin{itemize}
\tightlist
\item
  \texttt{\ }{\texttt{\ true\ }}\texttt{\ } : Wrap this quote in double
  quotes.
\item
  \texttt{\ }{\texttt{\ false\ }}\texttt{\ } : Do not wrap this quote in
  double quotes.
\item
  \texttt{\ }{\texttt{\ auto\ }}\texttt{\ } : Infer whether to wrap this
  quote in double quotes based on the \texttt{\ block\ } property. If
  \texttt{\ block\ } is \texttt{\ }{\texttt{\ false\ }}\texttt{\ } ,
  double quotes are automatically added.
\end{itemize}

Default: \texttt{\ }{\texttt{\ auto\ }}\texttt{\ }

\includesvg[width=0.16667in,height=0.16667in]{/assets/icons/16-arrow-right.svg}
View example

\begin{verbatim}
#set text(lang: "de")

Ein deutsch-sprechender Author
zitiert unter umständen JFK:
#quote[Ich bin ein Berliner.]

#set text(lang: "en")

And an english speaking one may
translate the quote:
#quote[I am a Berliner.]
\end{verbatim}

\includegraphics[width=5in,height=\textheight,keepaspectratio]{/assets/docs/3Qsm4wm5qgO3MH7h3rFICAAAAAAAAAAA.png}

\subsubsection{\texorpdfstring{\texttt{\ attribution\ }}{ attribution }}\label{parameters-attribution}

\href{/docs/reference/foundations/none/}{none} {or}
\href{/docs/reference/foundations/label/}{label} {or}
\href{/docs/reference/foundations/content/}{content}

{{ Settable }}

\phantomsection\label{parameters-attribution-settable-tooltip}
Settable parameters can be customized for all following uses of the
function with a \texttt{\ set\ } rule.

The attribution of this quote, usually the author or source. Can be a
label pointing to a bibliography entry or any content. By default only
displayed for block quotes, but can be changed using a
\texttt{\ }{\texttt{\ show\ }}\texttt{\ } rule.

Default: \texttt{\ }{\texttt{\ none\ }}\texttt{\ }

\includesvg[width=0.16667in,height=0.16667in]{/assets/icons/16-arrow-right.svg}
View example

\begin{verbatim}
#quote(attribution: [René Descartes])[
  cogito, ergo sum
]

#show quote.where(block: false): it => {
  ["] + h(0pt, weak: true) + it.body + h(0pt, weak: true) + ["]
  if it.attribution != none [ (#it.attribution)]
}

#quote(
  attribution: link("https://typst.app/home")[typst.com]
)[
  Compose papers faster
]

#set quote(block: true)

#quote(attribution: <tolkien54>)[
  You cannot pass... I am a servant
  of the Secret Fire, wielder of the
  flame of Anor. You cannot pass. The
  dark fire will not avail you, flame
  of Udûn. Go back to the Shadow! You
  cannot pass.
]

#bibliography("works.bib", style: "apa")
\end{verbatim}

\includegraphics[width=5in,height=\textheight,keepaspectratio]{/assets/docs/bB0B3x32glSn_oATlkF6mQAAAAAAAAAA.png}

\subsubsection{\texorpdfstring{\texttt{\ body\ }}{ body }}\label{parameters-body}

\href{/docs/reference/foundations/content/}{content}

{Required} {{ Positional }}

\phantomsection\label{parameters-body-positional-tooltip}
Positional parameters are specified in order, without names.

The quote.

\href{/docs/reference/model/parbreak/}{\pandocbounded{\includesvg[keepaspectratio]{/assets/icons/16-arrow-right.svg}}}

{ Paragraph Break } { Previous page }

\href{/docs/reference/model/ref/}{\pandocbounded{\includesvg[keepaspectratio]{/assets/icons/16-arrow-right.svg}}}

{ Reference } { Next page }




\section{C Docs LaTeX/docs/reference/symbols.tex}
\section{Docs LaTeX/typst.app/docs/reference/symbols/emoji.tex}
\title{typst.app/docs/reference/symbols/emoji}

\begin{itemize}
\tightlist
\item
  \href{/docs}{\includesvg[width=0.16667in,height=0.16667in]{/assets/icons/16-docs-dark.svg}}
\item
  \includesvg[width=0.16667in,height=0.16667in]{/assets/icons/16-arrow-right.svg}
\item
  \href{/docs/reference/}{Reference}
\item
  \includesvg[width=0.16667in,height=0.16667in]{/assets/icons/16-arrow-right.svg}
\item
  \href{/docs/reference/symbols/}{Symbols}
\item
  \includesvg[width=0.16667in,height=0.16667in]{/assets/icons/16-arrow-right.svg}
\item
  \href{/docs/reference/symbols/emoji/}{Emoji}
\end{itemize}

\section{emoji}\label{emoji}

Named emoji.

For example, \texttt{\ \#emoji.face\ } produces the 😀 emoji. If you
frequently use certain emojis, you can also import them from the
\texttt{\ emoji\ } module (
\texttt{\ }{\texttt{\ \#\ }}\texttt{\ }{\texttt{\ import\ }}\texttt{\ emoji\ }{\texttt{\ :\ }}\texttt{\ face\ }
) to use them without the \texttt{\ \#emoji.\ } prefix.

Click on a \href{/docs/reference/symbols/symbol/}{symbol} to copy it to
the clipboard.

\includesvg[width=0.16667in,height=0.16667in]{/assets/icons/16-search-gray.svg}

\begin{itemize}
\tightlist
\item
  \phantomsection\label{symbol-abacus}{{ 🧮 } \texttt{\ abacus\ }}
\item
  \phantomsection\label{symbol-abc}{{ ðŸ''¤ } \texttt{\ abc\ }}
\item
  \phantomsection\label{symbol-abcd}{{ ðŸ''¡ } \texttt{\ abcd\ }}
\item
  \phantomsection\label{symbol-ABCD}{{ ðŸ'' } \texttt{\ ABCD\ }}
\item
  \phantomsection\label{symbol-accordion}{{ ðŸª--- }
  \texttt{\ accordion\ }}
\item
  \phantomsection\label{symbol-aesculapius}{{ âš• }
  \texttt{\ aesculapius\ }}
\item
  \phantomsection\label{symbol-airplane}{{ ✈ } \texttt{\ airplane\ }}
\item
  \phantomsection\label{symbol-airplane.landing}{{ 🛬 }
  \texttt{\ airplane.\ landing\ }}
\item
  \phantomsection\label{symbol-airplane.small}{{ 🛩 }
  \texttt{\ airplane.\ small\ }}
\item
  \phantomsection\label{symbol-airplane.takeoff}{{ 🛫 }
  \texttt{\ airplane.\ takeoff\ }}
\item
  \phantomsection\label{symbol-alembic}{{ âš--- } \texttt{\ alembic\ }}
\item
  \phantomsection\label{symbol-alien}{{ ðŸ`½ } \texttt{\ alien\ }}
\item
  \phantomsection\label{symbol-alien.monster}{{ ðŸ`¾ }
  \texttt{\ alien.\ monster\ }}
\item
  \phantomsection\label{symbol-ambulance}{{ ðŸš` }
  \texttt{\ ambulance\ }}
\item
  \phantomsection\label{symbol-amphora}{{ � } \texttt{\ amphora\ }}
\item
  \phantomsection\label{symbol-anchor}{{ âš`` } \texttt{\ anchor\ }}
\item
  \phantomsection\label{symbol-anger}{{ ðŸ'¢ } \texttt{\ anger\ }}
\item
  \phantomsection\label{symbol-ant}{{ � } \texttt{\ ant\ }}
\item
  \phantomsection\label{symbol-apple.green}{{ ðŸ?? }
  \texttt{\ apple.\ green\ }}
\item
  \phantomsection\label{symbol-apple.red}{{ � }
  \texttt{\ apple.\ red\ }}
\item
  \phantomsection\label{symbol-arm.mech}{{ 🦾 }
  \texttt{\ arm.\ mech\ }}
\item
  \phantomsection\label{symbol-arm.muscle}{{ ðŸ'ª }
  \texttt{\ arm.\ muscle\ }}
\item
  \phantomsection\label{symbol-arm.selfie}{{ 🤳 }
  \texttt{\ arm.\ selfie\ }}
\item
  \phantomsection\label{symbol-arrow.r.filled}{{ âž¡ }
  \texttt{\ arrow.\ r.\ filled\ }}
\item
  \phantomsection\label{symbol-arrow.r.hook}{{ ↪ }
  \texttt{\ arrow.\ r.\ hook\ }}
\item
  \phantomsection\label{symbol-arrow.r.soon}{{ ðŸ''œ }
  \texttt{\ arrow.\ r.\ soon\ }}
\item
  \phantomsection\label{symbol-arrow.l.filled}{{ ⬠}
  \texttt{\ arrow.\ l.\ filled\ }}
\item
  \phantomsection\label{symbol-arrow.l.hook}{{ ↩ }
  \texttt{\ arrow.\ l.\ hook\ }}
\item
  \phantomsection\label{symbol-arrow.l.back}{{ ðŸ''™ }
  \texttt{\ arrow.\ l.\ back\ }}
\item
  \phantomsection\label{symbol-arrow.l.end}{{ ðŸ''š }
  \texttt{\ arrow.\ l.\ end\ }}
\item
  \phantomsection\label{symbol-arrow.t.filled}{{ ⬆ }
  \texttt{\ arrow.\ t.\ filled\ }}
\item
  \phantomsection\label{symbol-arrow.t.curve}{{ ⤴ }
  \texttt{\ arrow.\ t.\ curve\ }}
\item
  \phantomsection\label{symbol-arrow.t.top}{{ ðŸ''? }
  \texttt{\ arrow.\ t.\ top\ }}
\item
  \phantomsection\label{symbol-arrow.b.filled}{{ ⬇ }
  \texttt{\ arrow.\ b.\ filled\ }}
\item
  \phantomsection\label{symbol-arrow.b.curve}{{ ⤵ }
  \texttt{\ arrow.\ b.\ curve\ }}
\item
  \phantomsection\label{symbol-arrow.l.r}{{ â†'' }
  \texttt{\ arrow.\ l.\ r\ }}
\item
  \phantomsection\label{symbol-arrow.l.r.on}{{ ðŸ''› }
  \texttt{\ arrow.\ l.\ r.\ on\ }}
\item
  \phantomsection\label{symbol-arrow.t.b}{{ ↕ }
  \texttt{\ arrow.\ t.\ b\ }}
\item
  \phantomsection\label{symbol-arrow.bl}{{ ↙ }
  \texttt{\ arrow.\ bl\ }}
\item
  \phantomsection\label{symbol-arrow.br}{{ ↘ }
  \texttt{\ arrow.\ br\ }}
\item
  \phantomsection\label{symbol-arrow.tl}{{ â†-- }
  \texttt{\ arrow.\ tl\ }}
\item
  \phantomsection\label{symbol-arrow.tr}{{ â†--- }
  \texttt{\ arrow.\ tr\ }}
\item
  \phantomsection\label{symbol-arrows.cycle}{{ ðŸ''„ }
  \texttt{\ arrows.\ cycle\ }}
\item
  \phantomsection\label{symbol-ast}{{ * } \texttt{\ ast\ }}
\item
  \phantomsection\label{symbol-ast.box}{{ ✳ } \texttt{\ ast.\ box\ }}
\item
  \phantomsection\label{symbol-atm}{{ � } \texttt{\ atm\ }}
\item
  \phantomsection\label{symbol-atom}{{ âš› } \texttt{\ atom\ }}
\item
  \phantomsection\label{symbol-aubergine}{{ � }
  \texttt{\ aubergine\ }}
\item
  \phantomsection\label{symbol-avocado}{{ ðŸ¥` } \texttt{\ avocado\ }}
\item
  \phantomsection\label{symbol-axe}{{ ðŸª`` } \texttt{\ axe\ }}
\item
  \phantomsection\label{symbol-baby}{{ ðŸ`¶ } \texttt{\ baby\ }}
\item
  \phantomsection\label{symbol-baby.angel}{{ ðŸ`¼ }
  \texttt{\ baby.\ angel\ }}
\item
  \phantomsection\label{symbol-baby.box}{{ 🚼 }
  \texttt{\ baby.\ box\ }}
\item
  \phantomsection\label{symbol-babybottle}{{ � }
  \texttt{\ babybottle\ }}
\item
  \phantomsection\label{symbol-backpack}{{ ðŸŽ' } \texttt{\ backpack\ }}
\item
  \phantomsection\label{symbol-bacon}{{ ðŸ¥`` } \texttt{\ bacon\ }}
\item
  \phantomsection\label{symbol-badger}{{ 🦡 } \texttt{\ badger\ }}
\item
  \phantomsection\label{symbol-badminton}{{ � }
  \texttt{\ badminton\ }}
\item
  \phantomsection\label{symbol-bagel}{{ 🥯 } \texttt{\ bagel\ }}
\item
  \phantomsection\label{symbol-baggageclaim}{{ 🛄 }
  \texttt{\ baggageclaim\ }}
\item
  \phantomsection\label{symbol-baguette}{{ ðŸ¥-- }
  \texttt{\ baguette\ }}
\item
  \phantomsection\label{symbol-balloon}{{ 🎈 } \texttt{\ balloon\ }}
\item
  \phantomsection\label{symbol-ballot.check}{{ â˜` }
  \texttt{\ ballot.\ check\ }}
\item
  \phantomsection\label{symbol-ballotbox}{{ ðŸ---³ }
  \texttt{\ ballotbox\ }}
\item
  \phantomsection\label{symbol-banana}{{ � } \texttt{\ banana\ }}
\item
  \phantomsection\label{symbol-banjo}{{ 🪕 } \texttt{\ banjo\ }}
\item
  \phantomsection\label{symbol-bank}{{ � } \texttt{\ bank\ }}
\item
  \phantomsection\label{symbol-barberpole}{{ ðŸ'ˆ }
  \texttt{\ barberpole\ }}
\item
  \phantomsection\label{symbol-baseball}{{ âš¾ } \texttt{\ baseball\ }}
\item
  \phantomsection\label{symbol-basecap}{{ 🧢 } \texttt{\ basecap\ }}
\item
  \phantomsection\label{symbol-basket}{{ 🧺 } \texttt{\ basket\ }}
\item
  \phantomsection\label{symbol-basketball}{{ ⛹ }
  \texttt{\ basketball\ }}
\item
  \phantomsection\label{symbol-basketball.ball}{{ � }
  \texttt{\ basketball.\ ball\ }}
\item
  \phantomsection\label{symbol-bat}{{ 🦇 } \texttt{\ bat\ }}
\item
  \phantomsection\label{symbol-bathtub}{{ 🛀 } \texttt{\ bathtub\ }}
\item
  \phantomsection\label{symbol-bathtub.foam}{{ � }
  \texttt{\ bathtub.\ foam\ }}
\item
  \phantomsection\label{symbol-battery}{{ ðŸ''‹ } \texttt{\ battery\ }}
\item
  \phantomsection\label{symbol-battery.low}{{ 🪫 }
  \texttt{\ battery.\ low\ }}
\item
  \phantomsection\label{symbol-beach.palm}{{ ðŸ?? }
  \texttt{\ beach.\ palm\ }}
\item
  \phantomsection\label{symbol-beach.umbrella}{{ ðŸ?-- }
  \texttt{\ beach.\ umbrella\ }}
\item
  \phantomsection\label{symbol-beads}{{ ðŸ``¿ } \texttt{\ beads\ }}
\item
  \phantomsection\label{symbol-beans}{{ 🫘 } \texttt{\ beans\ }}
\item
  \phantomsection\label{symbol-bear}{{ � } \texttt{\ bear\ }}
\item
  \phantomsection\label{symbol-beaver}{{ 🦫 } \texttt{\ beaver\ }}
\item
  \phantomsection\label{symbol-bed}{{ � } \texttt{\ bed\ }}
\item
  \phantomsection\label{symbol-bed.person}{{ 🛌 }
  \texttt{\ bed.\ person\ }}
\item
  \phantomsection\label{symbol-bee}{{ ðŸ?? } \texttt{\ bee\ }}
\item
  \phantomsection\label{symbol-beer}{{ � } \texttt{\ beer\ }}
\item
  \phantomsection\label{symbol-beer.clink}{{ � }
  \texttt{\ beer.\ clink\ }}
\item
  \phantomsection\label{symbol-beet}{{ 🫜 } \texttt{\ beet\ }}
\item
  \phantomsection\label{symbol-beetle}{{ 🪲 } \texttt{\ beetle\ }}
\item
  \phantomsection\label{symbol-beetle.lady}{{ � }
  \texttt{\ beetle.\ lady\ }}
\item
  \phantomsection\label{symbol-bell}{{ ðŸ''\,'' } \texttt{\ bell\ }}
\item
  \phantomsection\label{symbol-bell.ding}{{ 🛎 }
  \texttt{\ bell.\ ding\ }}
\item
  \phantomsection\label{symbol-bell.not}{{ ðŸ''• }
  \texttt{\ bell.\ not\ }}
\item
  \phantomsection\label{symbol-bento}{{ � } \texttt{\ bento\ }}
\item
  \phantomsection\label{symbol-bicyclist}{{ 🚴 }
  \texttt{\ bicyclist\ }}
\item
  \phantomsection\label{symbol-bicyclist.mountain}{{ 🚵 }
  \texttt{\ bicyclist.\ mountain\ }}
\item
  \phantomsection\label{symbol-bike}{{ 🚲 } \texttt{\ bike\ }}
\item
  \phantomsection\label{symbol-bike.not}{{ 🚳 }
  \texttt{\ bike.\ not\ }}
\item
  \phantomsection\label{symbol-bikini}{{ ðŸ`™ } \texttt{\ bikini\ }}
\item
  \phantomsection\label{symbol-billiards}{{ 🎱 }
  \texttt{\ billiards\ }}
\item
  \phantomsection\label{symbol-bin}{{ ðŸ---` } \texttt{\ bin\ }}
\item
  \phantomsection\label{symbol-biohazard}{{ ☣ }
  \texttt{\ biohazard\ }}
\item
  \phantomsection\label{symbol-bird}{{ � } \texttt{\ bird\ }}
\item
  \phantomsection\label{symbol-bison}{{ 🦬 } \texttt{\ bison\ }}
\item
  \phantomsection\label{symbol-blood}{{ 🩸 } \texttt{\ blood\ }}
\item
  \phantomsection\label{symbol-blouse}{{ ðŸ`š } \texttt{\ blouse\ }}
\item
  \phantomsection\label{symbol-blowfish}{{ � } \texttt{\ blowfish\ }}
\item
  \phantomsection\label{symbol-blueberries}{{ � }
  \texttt{\ blueberries\ }}
\item
  \phantomsection\label{symbol-boar}{{ ðŸ?--- } \texttt{\ boar\ }}
\item
  \phantomsection\label{symbol-boat.sail}{{ ⛵ }
  \texttt{\ boat.\ sail\ }}
\item
  \phantomsection\label{symbol-boat.row}{{ 🚣 }
  \texttt{\ boat.\ row\ }}
\item
  \phantomsection\label{symbol-boat.motor}{{ 🛥 }
  \texttt{\ boat.\ motor\ }}
\item
  \phantomsection\label{symbol-boat.speed}{{ 🚤 }
  \texttt{\ boat.\ speed\ }}
\item
  \phantomsection\label{symbol-boat.canoe}{{ 🛶 }
  \texttt{\ boat.\ canoe\ }}
\item
  \phantomsection\label{symbol-bolt}{{ ðŸ''© } \texttt{\ bolt\ }}
\item
  \phantomsection\label{symbol-bomb}{{ ðŸ'£ } \texttt{\ bomb\ }}
\item
  \phantomsection\label{symbol-bone}{{ 🦴 } \texttt{\ bone\ }}
\item
  \phantomsection\label{symbol-book.red}{{ ðŸ``• }
  \texttt{\ book.\ red\ }}
\item
  \phantomsection\label{symbol-book.blue}{{ ðŸ``˜ }
  \texttt{\ book.\ blue\ }}
\item
  \phantomsection\label{symbol-book.green}{{ ðŸ``--- }
  \texttt{\ book.\ green\ }}
\item
  \phantomsection\label{symbol-book.orange}{{ ðŸ``™ }
  \texttt{\ book.\ orange\ }}
\item
  \phantomsection\label{symbol-book.spiral}{{ ðŸ``\,' }
  \texttt{\ book.\ spiral\ }}
\item
  \phantomsection\label{symbol-book.open}{{ ðŸ``-- }
  \texttt{\ book.\ open\ }}
\item
  \phantomsection\label{symbol-bookmark}{{ ðŸ''-- }
  \texttt{\ bookmark\ }}
\item
  \phantomsection\label{symbol-books}{{ ðŸ``š } \texttt{\ books\ }}
\item
  \phantomsection\label{symbol-boomerang}{{ 🪃 }
  \texttt{\ boomerang\ }}
\item
  \phantomsection\label{symbol-bordercontrol}{{ 🛂 }
  \texttt{\ bordercontrol\ }}
\item
  \phantomsection\label{symbol-bouquet}{{ ðŸ'? } \texttt{\ bouquet\ }}
\item
  \phantomsection\label{symbol-bow}{{ � } \texttt{\ bow\ }}
\item
  \phantomsection\label{symbol-bowl.spoon}{{ 🥣 }
  \texttt{\ bowl.\ spoon\ }}
\item
  \phantomsection\label{symbol-bowl.steam}{{ � }
  \texttt{\ bowl.\ steam\ }}
\item
  \phantomsection\label{symbol-bowling}{{ 🎳 } \texttt{\ bowling\ }}
\item
  \phantomsection\label{symbol-boxing}{{ 🥊 } \texttt{\ boxing\ }}
\item
  \phantomsection\label{symbol-boy}{{ ðŸ`¦ } \texttt{\ boy\ }}
\item
  \phantomsection\label{symbol-brain}{{ 🧠} \texttt{\ brain\ }}
\item
  \phantomsection\label{symbol-bread}{{ � } \texttt{\ bread\ }}
\item
  \phantomsection\label{symbol-brick}{{ 🧱 } \texttt{\ brick\ }}
\item
  \phantomsection\label{symbol-bride}{{ ðŸ`° } \texttt{\ bride\ }}
\item
  \phantomsection\label{symbol-bridge.fog}{{ � }
  \texttt{\ bridge.\ fog\ }}
\item
  \phantomsection\label{symbol-bridge.night}{{ 🌉 }
  \texttt{\ bridge.\ night\ }}
\item
  \phantomsection\label{symbol-briefcase}{{ ðŸ'¼ }
  \texttt{\ briefcase\ }}
\item
  \phantomsection\label{symbol-briefs}{{ 🩲 } \texttt{\ briefs\ }}
\item
  \phantomsection\label{symbol-brightness.high}{{ ðŸ''† }
  \texttt{\ brightness.\ high\ }}
\item
  \phantomsection\label{symbol-brightness.low}{{ ðŸ'' }
  \texttt{\ brightness.\ low\ }}
\item
  \phantomsection\label{symbol-broccoli}{{ 🥦 } \texttt{\ broccoli\ }}
\item
  \phantomsection\label{symbol-broom}{{ 🧹 } \texttt{\ broom\ }}
\item
  \phantomsection\label{symbol-brush}{{ ðŸ--Œ } \texttt{\ brush\ }}
\item
  \phantomsection\label{symbol-bubble.speech.r}{{ ðŸ'¬ }
  \texttt{\ bubble.\ speech.\ r\ }}
\item
  \phantomsection\label{symbol-bubble.speech.l}{{ ðŸ---¨ }
  \texttt{\ bubble.\ speech.\ l\ }}
\item
  \phantomsection\label{symbol-bubble.thought}{{ ðŸ'­ }
  \texttt{\ bubble.\ thought\ }}
\item
  \phantomsection\label{symbol-bubble.anger.r}{{ ðŸ---¯ }
  \texttt{\ bubble.\ anger.\ r\ }}
\item
  \phantomsection\label{symbol-bubbles}{{ 🫧 } \texttt{\ bubbles\ }}
\item
  \phantomsection\label{symbol-bubbletea}{{ 🧋 }
  \texttt{\ bubbletea\ }}
\item
  \phantomsection\label{symbol-bucket}{{ 🪣 } \texttt{\ bucket\ }}
\item
  \phantomsection\label{symbol-buffalo.water}{{ � }
  \texttt{\ buffalo.\ water\ }}
\item
  \phantomsection\label{symbol-bug}{{ � } \texttt{\ bug\ }}
\item
  \phantomsection\label{symbol-builder}{{ ðŸ`· } \texttt{\ builder\ }}
\item
  \phantomsection\label{symbol-burger}{{ ðŸ?'' } \texttt{\ burger\ }}
\item
  \phantomsection\label{symbol-burrito}{{ 🌯 } \texttt{\ burrito\ }}
\item
  \phantomsection\label{symbol-bus}{{ 🚌 } \texttt{\ bus\ }}
\item
  \phantomsection\label{symbol-bus.front}{{ � }
  \texttt{\ bus.\ front\ }}
\item
  \phantomsection\label{symbol-bus.small}{{ � }
  \texttt{\ bus.\ small\ }}
\item
  \phantomsection\label{symbol-bus.stop}{{ � }
  \texttt{\ bus.\ stop\ }}
\item
  \phantomsection\label{symbol-bus.trolley}{{ 🚎 }
  \texttt{\ bus.\ trolley\ }}
\item
  \phantomsection\label{symbol-butter}{{ 🧈 } \texttt{\ butter\ }}
\item
  \phantomsection\label{symbol-butterfly}{{ 🦋 }
  \texttt{\ butterfly\ }}
\item
  \phantomsection\label{symbol-button}{{ ðŸ''² } \texttt{\ button\ }}
\item
  \phantomsection\label{symbol-button.alt}{{ ðŸ''³ }
  \texttt{\ button.\ alt\ }}
\item
  \phantomsection\label{symbol-button.radio}{{ ðŸ''˜ }
  \texttt{\ button.\ radio\ }}
\item
  \phantomsection\label{symbol-cabinet.file}{{ ðŸ---„ }
  \texttt{\ cabinet.\ file\ }}
\item
  \phantomsection\label{symbol-cablecar}{{ 🚠} \texttt{\ cablecar\ }}
\item
  \phantomsection\label{symbol-cablecar.small}{{ 🚡 }
  \texttt{\ cablecar.\ small\ }}
\item
  \phantomsection\label{symbol-cactus}{{ 🌵 } \texttt{\ cactus\ }}
\item
  \phantomsection\label{symbol-cake}{{ 🎂 } \texttt{\ cake\ }}
\item
  \phantomsection\label{symbol-cake.fish}{{ � }
  \texttt{\ cake.\ fish\ }}
\item
  \phantomsection\label{symbol-cake.moon}{{ 🥮 }
  \texttt{\ cake.\ moon\ }}
\item
  \phantomsection\label{symbol-cake.slice}{{ � }
  \texttt{\ cake.\ slice\ }}
\item
  \phantomsection\label{symbol-calendar}{{ ðŸ`` } \texttt{\ calendar\ }}
\item
  \phantomsection\label{symbol-calendar.spiral}{{ ðŸ---`` }
  \texttt{\ calendar.\ spiral\ }}
\item
  \phantomsection\label{symbol-calendar.tearoff}{{ ðŸ``† }
  \texttt{\ calendar.\ tearoff\ }}
\item
  \phantomsection\label{symbol-camel}{{ � } \texttt{\ camel\ }}
\item
  \phantomsection\label{symbol-camel.dromedar}{{ � }
  \texttt{\ camel.\ dromedar\ }}
\item
  \phantomsection\label{symbol-camera}{{ ðŸ``· } \texttt{\ camera\ }}
\item
  \phantomsection\label{symbol-camera.flash}{{ ðŸ``¸ }
  \texttt{\ camera.\ flash\ }}
\item
  \phantomsection\label{symbol-camera.movie}{{ 🎥 }
  \texttt{\ camera.\ movie\ }}
\item
  \phantomsection\label{symbol-camera.movie.box}{{ 🎦 }
  \texttt{\ camera.\ movie.\ box\ }}
\item
  \phantomsection\label{symbol-camera.video}{{ ðŸ``¹ }
  \texttt{\ camera.\ video\ }}
\item
  \phantomsection\label{symbol-camping}{{ � } \texttt{\ camping\ }}
\item
  \phantomsection\label{symbol-can}{{ 🥫 } \texttt{\ can\ }}
\item
  \phantomsection\label{symbol-candle}{{ 🕯 } \texttt{\ candle\ }}
\item
  \phantomsection\label{symbol-candy}{{ � } \texttt{\ candy\ }}
\item
  \phantomsection\label{symbol-cane}{{ 🦯 } \texttt{\ cane\ }}
\item
  \phantomsection\label{symbol-car}{{ ðŸš--- } \texttt{\ car\ }}
\item
  \phantomsection\label{symbol-car.front}{{ 🚘 }
  \texttt{\ car.\ front\ }}
\item
  \phantomsection\label{symbol-car.pickup}{{ 🛻 }
  \texttt{\ car.\ pickup\ }}
\item
  \phantomsection\label{symbol-car.police}{{ ðŸš`` }
  \texttt{\ car.\ police\ }}
\item
  \phantomsection\label{symbol-car.police.front}{{ ðŸš'' }
  \texttt{\ car.\ police.\ front\ }}
\item
  \phantomsection\label{symbol-car.racing}{{ � }
  \texttt{\ car.\ racing\ }}
\item
  \phantomsection\label{symbol-car.rickshaw}{{ 🛺 }
  \texttt{\ car.\ rickshaw\ }}
\item
  \phantomsection\label{symbol-car.suv}{{ 🚙 } \texttt{\ car.\ suv\ }}
\item
  \phantomsection\label{symbol-card.credit}{{ ðŸ'³ }
  \texttt{\ card.\ credit\ }}
\item
  \phantomsection\label{symbol-card.id}{{ 🪪 } \texttt{\ card.\ id\ }}
\item
  \phantomsection\label{symbol-cardindex}{{ ðŸ``‡ }
  \texttt{\ cardindex\ }}
\item
  \phantomsection\label{symbol-carrot}{{ 🥕 } \texttt{\ carrot\ }}
\item
  \phantomsection\label{symbol-cart}{{ ðŸ›' } \texttt{\ cart\ }}
\item
  \phantomsection\label{symbol-cassette}{{ ðŸ``¼ }
  \texttt{\ cassette\ }}
\item
  \phantomsection\label{symbol-castle.eu}{{ � }
  \texttt{\ castle.\ eu\ }}
\item
  \phantomsection\label{symbol-castle.jp}{{ � }
  \texttt{\ castle.\ jp\ }}
\item
  \phantomsection\label{symbol-cat}{{ � } \texttt{\ cat\ }}
\item
  \phantomsection\label{symbol-cat.face}{{ � }
  \texttt{\ cat.\ face\ }}
\item
  \phantomsection\label{symbol-cat.face.angry}{{ 😾 }
  \texttt{\ cat.\ face.\ angry\ }}
\item
  \phantomsection\label{symbol-cat.face.cry}{{ 😿 }
  \texttt{\ cat.\ face.\ cry\ }}
\item
  \phantomsection\label{symbol-cat.face.heart}{{ 😻 }
  \texttt{\ cat.\ face.\ heart\ }}
\item
  \phantomsection\label{symbol-cat.face.joy}{{ 😹 }
  \texttt{\ cat.\ face.\ joy\ }}
\item
  \phantomsection\label{symbol-cat.face.kiss}{{ 😽 }
  \texttt{\ cat.\ face.\ kiss\ }}
\item
  \phantomsection\label{symbol-cat.face.laugh}{{ 😸 }
  \texttt{\ cat.\ face.\ laugh\ }}
\item
  \phantomsection\label{symbol-cat.face.shock}{{ 🙀 }
  \texttt{\ cat.\ face.\ shock\ }}
\item
  \phantomsection\label{symbol-cat.face.smile}{{ 😺 }
  \texttt{\ cat.\ face.\ smile\ }}
\item
  \phantomsection\label{symbol-cat.face.smirk}{{ 😼 }
  \texttt{\ cat.\ face.\ smirk\ }}
\item
  \phantomsection\label{symbol-chain}{{ ðŸ''--- } \texttt{\ chain\ }}
\item
  \phantomsection\label{symbol-chains}{{ â›`` } \texttt{\ chains\ }}
\item
  \phantomsection\label{symbol-chair}{{ ðŸª` } \texttt{\ chair\ }}
\item
  \phantomsection\label{symbol-champagne}{{ � }
  \texttt{\ champagne\ }}
\item
  \phantomsection\label{symbol-chart.bar}{{ ðŸ``Š }
  \texttt{\ chart.\ bar\ }}
\item
  \phantomsection\label{symbol-chart.up}{{ ðŸ``ˆ }
  \texttt{\ chart.\ up\ }}
\item
  \phantomsection\label{symbol-chart.down}{{ ðŸ``‰ }
  \texttt{\ chart.\ down\ }}
\item
  \phantomsection\label{symbol-chart.yen.up}{{ ðŸ'¹ }
  \texttt{\ chart.\ yen.\ up\ }}
\item
  \phantomsection\label{symbol-checkmark.heavy}{{ âœ'' }
  \texttt{\ checkmark.\ heavy\ }}
\item
  \phantomsection\label{symbol-checkmark.box}{{ ✠}
  \texttt{\ checkmark.\ box\ }}
\item
  \phantomsection\label{symbol-cheese}{{ 🧀 } \texttt{\ cheese\ }}
\item
  \phantomsection\label{symbol-cherries}{{ ðŸ?' } \texttt{\ cherries\ }}
\item
  \phantomsection\label{symbol-chess}{{ ♟ } \texttt{\ chess\ }}
\item
  \phantomsection\label{symbol-chestnut}{{ 🌰 } \texttt{\ chestnut\ }}
\item
  \phantomsection\label{symbol-chicken}{{ ðŸ?'' } \texttt{\ chicken\ }}
\item
  \phantomsection\label{symbol-chicken.baby}{{ � }
  \texttt{\ chicken.\ baby\ }}
\item
  \phantomsection\label{symbol-chicken.baby.egg}{{ � }
  \texttt{\ chicken.\ baby.\ egg\ }}
\item
  \phantomsection\label{symbol-chicken.baby.head}{{ � }
  \texttt{\ chicken.\ baby.\ head\ }}
\item
  \phantomsection\label{symbol-chicken.leg}{{ ðŸ?--- }
  \texttt{\ chicken.\ leg\ }}
\item
  \phantomsection\label{symbol-chicken.male}{{ ðŸ?{}`` }
  \texttt{\ chicken.\ male\ }}
\item
  \phantomsection\label{symbol-child}{{ ðŸ§' } \texttt{\ child\ }}
\item
  \phantomsection\label{symbol-chipmunk}{{ � } \texttt{\ chipmunk\ }}
\item
  \phantomsection\label{symbol-chocolate}{{ � }
  \texttt{\ chocolate\ }}
\item
  \phantomsection\label{symbol-chopsticks}{{ 🥢 }
  \texttt{\ chopsticks\ }}
\item
  \phantomsection\label{symbol-church}{{ ⛪ } \texttt{\ church\ }}
\item
  \phantomsection\label{symbol-church.love}{{ ðŸ'\,' }
  \texttt{\ church.\ love\ }}
\item
  \phantomsection\label{symbol-cigarette}{{ 🚬 }
  \texttt{\ cigarette\ }}
\item
  \phantomsection\label{symbol-cigarette.not}{{ 🚭 }
  \texttt{\ cigarette.\ not\ }}
\item
  \phantomsection\label{symbol-circle.black}{{ âš« }
  \texttt{\ circle.\ black\ }}
\item
  \phantomsection\label{symbol-circle.blue}{{ ðŸ''µ }
  \texttt{\ circle.\ blue\ }}
\item
  \phantomsection\label{symbol-circle.brown}{{ 🟤 }
  \texttt{\ circle.\ brown\ }}
\item
  \phantomsection\label{symbol-circle.green}{{ 🟢 }
  \texttt{\ circle.\ green\ }}
\item
  \phantomsection\label{symbol-circle.orange}{{ 🟠}
  \texttt{\ circle.\ orange\ }}
\item
  \phantomsection\label{symbol-circle.purple}{{ 🟣 }
  \texttt{\ circle.\ purple\ }}
\item
  \phantomsection\label{symbol-circle.white}{{ ⚪ }
  \texttt{\ circle.\ white\ }}
\item
  \phantomsection\label{symbol-circle.red}{{ ðŸ''´ }
  \texttt{\ circle.\ red\ }}
\item
  \phantomsection\label{symbol-circle.yellow}{{ 🟡 }
  \texttt{\ circle.\ yellow\ }}
\item
  \phantomsection\label{symbol-circle.stroked}{{ â­• }
  \texttt{\ circle.\ stroked\ }}
\item
  \phantomsection\label{symbol-circus}{{ 🎪 } \texttt{\ circus\ }}
\item
  \phantomsection\label{symbol-city}{{ � } \texttt{\ city\ }}
\item
  \phantomsection\label{symbol-city.dusk}{{ 🌆 }
  \texttt{\ city.\ dusk\ }}
\item
  \phantomsection\label{symbol-city.night}{{ 🌃 }
  \texttt{\ city.\ night\ }}
\item
  \phantomsection\label{symbol-city.sunset}{{ 🌇 }
  \texttt{\ city.\ sunset\ }}
\item
  \phantomsection\label{symbol-clamp}{{ ðŸ---œ } \texttt{\ clamp\ }}
\item
  \phantomsection\label{symbol-clapperboard}{{ 🎬 }
  \texttt{\ clapperboard\ }}
\item
  \phantomsection\label{symbol-climbing}{{ ðŸ§--- }
  \texttt{\ climbing\ }}
\item
  \phantomsection\label{symbol-clip}{{ ðŸ``Ž } \texttt{\ clip\ }}
\item
  \phantomsection\label{symbol-clipboard}{{ ðŸ``‹ }
  \texttt{\ clipboard\ }}
\item
  \phantomsection\label{symbol-clips}{{ ðŸ--‡ } \texttt{\ clips\ }}
\item
  \phantomsection\label{symbol-clock.one}{{ � }
  \texttt{\ clock.\ one\ }}
\item
  \phantomsection\label{symbol-clock.one.thirty}{{ 🕜 }
  \texttt{\ clock.\ one.\ thirty\ }}
\item
  \phantomsection\label{symbol-clock.two}{{ ðŸ•` }
  \texttt{\ clock.\ two\ }}
\item
  \phantomsection\label{symbol-clock.two.thirty}{{ � }
  \texttt{\ clock.\ two.\ thirty\ }}
\item
  \phantomsection\label{symbol-clock.three}{{ ðŸ•' }
  \texttt{\ clock.\ three\ }}
\item
  \phantomsection\label{symbol-clock.three.thirty}{{ 🕞 }
  \texttt{\ clock.\ three.\ thirty\ }}
\item
  \phantomsection\label{symbol-clock.four}{{ ðŸ•`` }
  \texttt{\ clock.\ four\ }}
\item
  \phantomsection\label{symbol-clock.four.thirty}{{ 🕟 }
  \texttt{\ clock.\ four.\ thirty\ }}
\item
  \phantomsection\label{symbol-clock.five}{{ ðŸ•'' }
  \texttt{\ clock.\ five\ }}
\item
  \phantomsection\label{symbol-clock.five.thirty}{{ 🕠}
  \texttt{\ clock.\ five.\ thirty\ }}
\item
  \phantomsection\label{symbol-clock.six}{{ 🕕 }
  \texttt{\ clock.\ six\ }}
\item
  \phantomsection\label{symbol-clock.six.thirty}{{ 🕡 }
  \texttt{\ clock.\ six.\ thirty\ }}
\item
  \phantomsection\label{symbol-clock.seven}{{ ðŸ•-- }
  \texttt{\ clock.\ seven\ }}
\item
  \phantomsection\label{symbol-clock.seven.thirty}{{ 🕢 }
  \texttt{\ clock.\ seven.\ thirty\ }}
\item
  \phantomsection\label{symbol-clock.eight}{{ ðŸ•--- }
  \texttt{\ clock.\ eight\ }}
\item
  \phantomsection\label{symbol-clock.eight.thirty}{{ 🕣 }
  \texttt{\ clock.\ eight.\ thirty\ }}
\item
  \phantomsection\label{symbol-clock.nine}{{ 🕘 }
  \texttt{\ clock.\ nine\ }}
\item
  \phantomsection\label{symbol-clock.nine.thirty}{{ 🕤 }
  \texttt{\ clock.\ nine.\ thirty\ }}
\item
  \phantomsection\label{symbol-clock.ten}{{ 🕙 }
  \texttt{\ clock.\ ten\ }}
\item
  \phantomsection\label{symbol-clock.ten.thirty}{{ 🕥 }
  \texttt{\ clock.\ ten.\ thirty\ }}
\item
  \phantomsection\label{symbol-clock.eleven}{{ 🕚 }
  \texttt{\ clock.\ eleven\ }}
\item
  \phantomsection\label{symbol-clock.eleven.thirty}{{ 🕦 }
  \texttt{\ clock.\ eleven.\ thirty\ }}
\item
  \phantomsection\label{symbol-clock.twelve}{{ 🕛 }
  \texttt{\ clock.\ twelve\ }}
\item
  \phantomsection\label{symbol-clock.twelve.thirty}{{ 🕧 }
  \texttt{\ clock.\ twelve.\ thirty\ }}
\item
  \phantomsection\label{symbol-clock.alarm}{{ â?° }
  \texttt{\ clock.\ alarm\ }}
\item
  \phantomsection\label{symbol-clock.old}{{ 🕰 }
  \texttt{\ clock.\ old\ }}
\item
  \phantomsection\label{symbol-clock.timer}{{ â?² }
  \texttt{\ clock.\ timer\ }}
\item
  \phantomsection\label{symbol-cloud}{{ � } \texttt{\ cloud\ }}
\item
  \phantomsection\label{symbol-cloud.dust}{{ ðŸ'¨ }
  \texttt{\ cloud.\ dust\ }}
\item
  \phantomsection\label{symbol-cloud.rain}{{ 🌧 }
  \texttt{\ cloud.\ rain\ }}
\item
  \phantomsection\label{symbol-cloud.snow}{{ 🌨 }
  \texttt{\ cloud.\ snow\ }}
\item
  \phantomsection\label{symbol-cloud.storm}{{ ⛈ }
  \texttt{\ cloud.\ storm\ }}
\item
  \phantomsection\label{symbol-cloud.sun}{{ â› }
  \texttt{\ cloud.\ sun\ }}
\item
  \phantomsection\label{symbol-cloud.sun.hidden}{{ 🌥 }
  \texttt{\ cloud.\ sun.\ hidden\ }}
\item
  \phantomsection\label{symbol-cloud.sun.rain}{{ 🌦 }
  \texttt{\ cloud.\ sun.\ rain\ }}
\item
  \phantomsection\label{symbol-cloud.thunder}{{ 🌩 }
  \texttt{\ cloud.\ thunder\ }}
\item
  \phantomsection\label{symbol-coat}{{ 🧥 } \texttt{\ coat\ }}
\item
  \phantomsection\label{symbol-coat.lab}{{ 🥼 }
  \texttt{\ coat.\ lab\ }}
\item
  \phantomsection\label{symbol-cockroach}{{ 🪳 }
  \texttt{\ cockroach\ }}
\item
  \phantomsection\label{symbol-cocktail.martini}{{ � }
  \texttt{\ cocktail.\ martini\ }}
\item
  \phantomsection\label{symbol-cocktail.tropical}{{ � }
  \texttt{\ cocktail.\ tropical\ }}
\item
  \phantomsection\label{symbol-coconut}{{ 🥥 } \texttt{\ coconut\ }}
\item
  \phantomsection\label{symbol-coffee}{{ ☕ } \texttt{\ coffee\ }}
\item
  \phantomsection\label{symbol-coffin}{{ âš° } \texttt{\ coffin\ }}
\item
  \phantomsection\label{symbol-coin}{{ 🪙 } \texttt{\ coin\ }}
\item
  \phantomsection\label{symbol-comet}{{ ☄ } \texttt{\ comet\ }}
\item
  \phantomsection\label{symbol-compass}{{ 🧭 } \texttt{\ compass\ }}
\item
  \phantomsection\label{symbol-computer}{{ ðŸ--¥ }
  \texttt{\ computer\ }}
\item
  \phantomsection\label{symbol-computermouse}{{ ðŸ--± }
  \texttt{\ computermouse\ }}
\item
  \phantomsection\label{symbol-confetti}{{ 🎊 } \texttt{\ confetti\ }}
\item
  \phantomsection\label{symbol-construction}{{ 🚧 }
  \texttt{\ construction\ }}
\item
  \phantomsection\label{symbol-controller}{{ 🎮 }
  \texttt{\ controller\ }}
\item
  \phantomsection\label{symbol-cookie}{{ � } \texttt{\ cookie\ }}
\item
  \phantomsection\label{symbol-cookie.fortune}{{ 🥠}
  \texttt{\ cookie.\ fortune\ }}
\item
  \phantomsection\label{symbol-cooking}{{ � } \texttt{\ cooking\ }}
\item
  \phantomsection\label{symbol-cool}{{ ðŸ†' } \texttt{\ cool\ }}
\item
  \phantomsection\label{symbol-copyright}{{ © } \texttt{\ copyright\ }}
\item
  \phantomsection\label{symbol-coral}{{ 🪸 } \texttt{\ coral\ }}
\item
  \phantomsection\label{symbol-corn}{{ 🌽 } \texttt{\ corn\ }}
\item
  \phantomsection\label{symbol-couch}{{ 🛋 } \texttt{\ couch\ }}
\item
  \phantomsection\label{symbol-couple}{{ ðŸ'\,` } \texttt{\ couple\ }}
\item
  \phantomsection\label{symbol-cow}{{ � } \texttt{\ cow\ }}
\item
  \phantomsection\label{symbol-cow.face}{{ � }
  \texttt{\ cow.\ face\ }}
\item
  \phantomsection\label{symbol-crab}{{ 🦀 } \texttt{\ crab\ }}
\item
  \phantomsection\label{symbol-crane}{{ ðŸ?--- } \texttt{\ crane\ }}
\item
  \phantomsection\label{symbol-crayon}{{ ðŸ--? } \texttt{\ crayon\ }}
\item
  \phantomsection\label{symbol-cricket}{{ ðŸ¦--- } \texttt{\ cricket\ }}
\item
  \phantomsection\label{symbol-cricketbat}{{ ðŸ?? }
  \texttt{\ cricketbat\ }}
\item
  \phantomsection\label{symbol-crocodile}{{ � }
  \texttt{\ crocodile\ }}
\item
  \phantomsection\label{symbol-croissant}{{ � }
  \texttt{\ croissant\ }}
\item
  \phantomsection\label{symbol-crossmark}{{ � }
  \texttt{\ crossmark\ }}
\item
  \phantomsection\label{symbol-crossmark.box}{{ â?Ž }
  \texttt{\ crossmark.\ box\ }}
\item
  \phantomsection\label{symbol-crown}{{ ðŸ`\,` } \texttt{\ crown\ }}
\item
  \phantomsection\label{symbol-crutch}{{ 🩼 } \texttt{\ crutch\ }}
\item
  \phantomsection\label{symbol-crystal}{{ ðŸ''® } \texttt{\ crystal\ }}
\item
  \phantomsection\label{symbol-cucumber}{{ ðŸ¥' } \texttt{\ cucumber\ }}
\item
  \phantomsection\label{symbol-cup.straw}{{ 🥤 }
  \texttt{\ cup.\ straw\ }}
\item
  \phantomsection\label{symbol-cupcake}{{ � } \texttt{\ cupcake\ }}
\item
  \phantomsection\label{symbol-curling}{{ 🥌 } \texttt{\ curling\ }}
\item
  \phantomsection\label{symbol-curry}{{ � } \texttt{\ curry\ }}
\item
  \phantomsection\label{symbol-custard}{{ � } \texttt{\ custard\ }}
\item
  \phantomsection\label{symbol-customs}{{ 🛃 } \texttt{\ customs\ }}
\item
  \phantomsection\label{symbol-cutlery}{{ � } \texttt{\ cutlery\ }}
\item
  \phantomsection\label{symbol-cyclone}{{ 🌀 } \texttt{\ cyclone\ }}
\item
  \phantomsection\label{symbol-dancing.man}{{ 🕺 }
  \texttt{\ dancing.\ man\ }}
\item
  \phantomsection\label{symbol-dancing.woman}{{ ðŸ'ƒ }
  \texttt{\ dancing.\ woman\ }}
\item
  \phantomsection\label{symbol-dancing.women.bunny}{{ ðŸ`¯ }
  \texttt{\ dancing.\ women.\ bunny\ }}
\item
  \phantomsection\label{symbol-darts}{{ 🎯 } \texttt{\ darts\ }}
\item
  \phantomsection\label{symbol-dash.wave.double}{{ 〰 }
  \texttt{\ dash.\ wave.\ double\ }}
\item
  \phantomsection\label{symbol-deer}{{ 🦌 } \texttt{\ deer\ }}
\item
  \phantomsection\label{symbol-desert}{{ � } \texttt{\ desert\ }}
\item
  \phantomsection\label{symbol-detective}{{ 🕵 }
  \texttt{\ detective\ }}
\item
  \phantomsection\label{symbol-diamond.blue}{{ ðŸ''· }
  \texttt{\ diamond.\ blue\ }}
\item
  \phantomsection\label{symbol-diamond.blue.small}{{ ðŸ''¹ }
  \texttt{\ diamond.\ blue.\ small\ }}
\item
  \phantomsection\label{symbol-diamond.orange}{{ ðŸ''¶ }
  \texttt{\ diamond.\ orange\ }}
\item
  \phantomsection\label{symbol-diamond.orange.small}{{ ðŸ''¸ }
  \texttt{\ diamond.\ orange.\ small\ }}
\item
  \phantomsection\label{symbol-diamond.dot}{{ ðŸ' }
  \texttt{\ diamond.\ dot\ }}
\item
  \phantomsection\label{symbol-die}{{ 🎲 } \texttt{\ die\ }}
\item
  \phantomsection\label{symbol-dino.pod}{{ 🦕 }
  \texttt{\ dino.\ pod\ }}
\item
  \phantomsection\label{symbol-dino.rex}{{ ðŸ¦-- }
  \texttt{\ dino.\ rex\ }}
\item
  \phantomsection\label{symbol-disc.cd}{{ ðŸ'¿ } \texttt{\ disc.\ cd\ }}
\item
  \phantomsection\label{symbol-disc.dvd}{{ ðŸ``€ }
  \texttt{\ disc.\ dvd\ }}
\item
  \phantomsection\label{symbol-disc.mini}{{ ðŸ'½ }
  \texttt{\ disc.\ mini\ }}
\item
  \phantomsection\label{symbol-discoball}{{ 🪩 }
  \texttt{\ discoball\ }}
\item
  \phantomsection\label{symbol-diving}{{ 🤿 } \texttt{\ diving\ }}
\item
  \phantomsection\label{symbol-dodo}{{ 🦤 } \texttt{\ dodo\ }}
\item
  \phantomsection\label{symbol-dog}{{ � } \texttt{\ dog\ }}
\item
  \phantomsection\label{symbol-dog.face}{{ � }
  \texttt{\ dog.\ face\ }}
\item
  \phantomsection\label{symbol-dog.guide}{{ 🦮 }
  \texttt{\ dog.\ guide\ }}
\item
  \phantomsection\label{symbol-dog.poodle}{{ � }
  \texttt{\ dog.\ poodle\ }}
\item
  \phantomsection\label{symbol-dollar}{{ ðŸ'² } \texttt{\ dollar\ }}
\item
  \phantomsection\label{symbol-dolphin}{{ � } \texttt{\ dolphin\ }}
\item
  \phantomsection\label{symbol-donut}{{ � } \texttt{\ donut\ }}
\item
  \phantomsection\label{symbol-door}{{ 🚪 } \texttt{\ door\ }}
\item
  \phantomsection\label{symbol-dove.peace}{{ 🕊 }
  \texttt{\ dove.\ peace\ }}
\item
  \phantomsection\label{symbol-dragon}{{ � } \texttt{\ dragon\ }}
\item
  \phantomsection\label{symbol-dragon.face}{{ � }
  \texttt{\ dragon.\ face\ }}
\item
  \phantomsection\label{symbol-dress}{{ ðŸ`--- } \texttt{\ dress\ }}
\item
  \phantomsection\label{symbol-dress.kimono}{{ ðŸ`˜ }
  \texttt{\ dress.\ kimono\ }}
\item
  \phantomsection\label{symbol-dress.sari}{{ 🥻 }
  \texttt{\ dress.\ sari\ }}
\item
  \phantomsection\label{symbol-drop}{{ ðŸ'§ } \texttt{\ drop\ }}
\item
  \phantomsection\label{symbol-drops}{{ ðŸ'¦ } \texttt{\ drops\ }}
\item
  \phantomsection\label{symbol-drum}{{ � } \texttt{\ drum\ }}
\item
  \phantomsection\label{symbol-drum.big}{{ 🪘 }
  \texttt{\ drum.\ big\ }}
\item
  \phantomsection\label{symbol-duck}{{ 🦆 } \texttt{\ duck\ }}
\item
  \phantomsection\label{symbol-dumpling}{{ 🥟 } \texttt{\ dumpling\ }}
\item
  \phantomsection\label{symbol-eagle}{{ 🦠} \texttt{\ eagle\ }}
\item
  \phantomsection\label{symbol-ear}{{ ðŸ`‚ } \texttt{\ ear\ }}
\item
  \phantomsection\label{symbol-ear.aid}{{ 🦻 } \texttt{\ ear.\ aid\ }}
\item
  \phantomsection\label{symbol-egg}{{ 🥚 } \texttt{\ egg\ }}
\item
  \phantomsection\label{symbol-eighteen.not}{{ ðŸ''ž }
  \texttt{\ eighteen.\ not\ }}
\item
  \phantomsection\label{symbol-elephant}{{ � } \texttt{\ elephant\ }}
\item
  \phantomsection\label{symbol-elevator}{{ ðŸ›--- }
  \texttt{\ elevator\ }}
\item
  \phantomsection\label{symbol-elf}{{ � } \texttt{\ elf\ }}
\item
  \phantomsection\label{symbol-email}{{ ðŸ``§ } \texttt{\ email\ }}
\item
  \phantomsection\label{symbol-excl}{{ â?--- } \texttt{\ excl\ }}
\item
  \phantomsection\label{symbol-excl.white}{{ â?• }
  \texttt{\ excl.\ white\ }}
\item
  \phantomsection\label{symbol-excl.double}{{ ‼ }
  \texttt{\ excl.\ double\ }}
\item
  \phantomsection\label{symbol-excl.quest}{{ â?‰ }
  \texttt{\ excl.\ quest\ }}
\item
  \phantomsection\label{symbol-explosion}{{ ðŸ'¥ }
  \texttt{\ explosion\ }}
\item
  \phantomsection\label{symbol-extinguisher}{{ 🧯 }
  \texttt{\ extinguisher\ }}
\item
  \phantomsection\label{symbol-eye}{{ ðŸ`? } \texttt{\ eye\ }}
\item
  \phantomsection\label{symbol-eyes}{{ ðŸ`€ } \texttt{\ eyes\ }}
\item
  \phantomsection\label{symbol-face.grin}{{ 😀 }
  \texttt{\ face.\ grin\ }}
\item
  \phantomsection\label{symbol-face.angry}{{ 😠}
  \texttt{\ face.\ angry\ }}
\item
  \phantomsection\label{symbol-face.angry.red}{{ 😡 }
  \texttt{\ face.\ angry.\ red\ }}
\item
  \phantomsection\label{symbol-face.anguish}{{ 😧 }
  \texttt{\ face.\ anguish\ }}
\item
  \phantomsection\label{symbol-face.astonish}{{ 😲 }
  \texttt{\ face.\ astonish\ }}
\item
  \phantomsection\label{symbol-face.bandage}{{ 🤕 }
  \texttt{\ face.\ bandage\ }}
\item
  \phantomsection\label{symbol-face.beam}{{ � }
  \texttt{\ face.\ beam\ }}
\item
  \phantomsection\label{symbol-face.blank}{{ 😶 }
  \texttt{\ face.\ blank\ }}
\item
  \phantomsection\label{symbol-face.clown}{{ 🤡 }
  \texttt{\ face.\ clown\ }}
\item
  \phantomsection\label{symbol-face.cold}{{ 🥶 }
  \texttt{\ face.\ cold\ }}
\item
  \phantomsection\label{symbol-face.concern}{{ 😦 }
  \texttt{\ face.\ concern\ }}
\item
  \phantomsection\label{symbol-face.cool}{{ 😎 }
  \texttt{\ face.\ cool\ }}
\item
  \phantomsection\label{symbol-face.cover}{{ 🤭 }
  \texttt{\ face.\ cover\ }}
\item
  \phantomsection\label{symbol-face.cowboy}{{ 🤠}
  \texttt{\ face.\ cowboy\ }}
\item
  \phantomsection\label{symbol-face.cry}{{ 😭 }
  \texttt{\ face.\ cry\ }}
\item
  \phantomsection\label{symbol-face.devil.smile}{{ 😈 }
  \texttt{\ face.\ devil.\ smile\ }}
\item
  \phantomsection\label{symbol-face.devil.frown}{{ ðŸ`¿ }
  \texttt{\ face.\ devil.\ frown\ }}
\item
  \phantomsection\label{symbol-face.diagonal}{{ 🫤 }
  \texttt{\ face.\ diagonal\ }}
\item
  \phantomsection\label{symbol-face.disguise}{{ 🥸 }
  \texttt{\ face.\ disguise\ }}
\item
  \phantomsection\label{symbol-face.distress}{{ 😫 }
  \texttt{\ face.\ distress\ }}
\item
  \phantomsection\label{symbol-face.dizzy}{{ 😵 }
  \texttt{\ face.\ dizzy\ }}
\item
  \phantomsection\label{symbol-face.dotted}{{ 🫥 }
  \texttt{\ face.\ dotted\ }}
\item
  \phantomsection\label{symbol-face.down}{{ 😞 }
  \texttt{\ face.\ down\ }}
\item
  \phantomsection\label{symbol-face.down.sweat}{{ ðŸ˜`` }
  \texttt{\ face.\ down.\ sweat\ }}
\item
  \phantomsection\label{symbol-face.drool}{{ 🤤 }
  \texttt{\ face.\ drool\ }}
\item
  \phantomsection\label{symbol-face.explode}{{ 🤯 }
  \texttt{\ face.\ explode\ }}
\item
  \phantomsection\label{symbol-face.eyeroll}{{ 🙄 }
  \texttt{\ face.\ eyeroll\ }}
\item
  \phantomsection\label{symbol-face.friendly}{{ ☺ }
  \texttt{\ face.\ friendly\ }}
\item
  \phantomsection\label{symbol-face.fear}{{ 😨 }
  \texttt{\ face.\ fear\ }}
\item
  \phantomsection\label{symbol-face.fear.sweat}{{ 😰 }
  \texttt{\ face.\ fear.\ sweat\ }}
\item
  \phantomsection\label{symbol-face.fever}{{ ðŸ¤' }
  \texttt{\ face.\ fever\ }}
\item
  \phantomsection\label{symbol-face.flush}{{ 😳 }
  \texttt{\ face.\ flush\ }}
\item
  \phantomsection\label{symbol-face.frown}{{ ☹ }
  \texttt{\ face.\ frown\ }}
\item
  \phantomsection\label{symbol-face.frown.slight}{{ � }
  \texttt{\ face.\ frown.\ slight\ }}
\item
  \phantomsection\label{symbol-face.frust}{{ 😣 }
  \texttt{\ face.\ frust\ }}
\item
  \phantomsection\label{symbol-face.goofy}{{ 🤪 }
  \texttt{\ face.\ goofy\ }}
\item
  \phantomsection\label{symbol-face.halo}{{ 😇 }
  \texttt{\ face.\ halo\ }}
\item
  \phantomsection\label{symbol-face.happy}{{ 😊 }
  \texttt{\ face.\ happy\ }}
\item
  \phantomsection\label{symbol-face.heart}{{ � }
  \texttt{\ face.\ heart\ }}
\item
  \phantomsection\label{symbol-face.hearts}{{ 🥰 }
  \texttt{\ face.\ hearts\ }}
\item
  \phantomsection\label{symbol-face.heat}{{ 🥵 }
  \texttt{\ face.\ heat\ }}
\item
  \phantomsection\label{symbol-face.hug}{{ ðŸ¤--- }
  \texttt{\ face.\ hug\ }}
\item
  \phantomsection\label{symbol-face.inv}{{ 🙃 }
  \texttt{\ face.\ inv\ }}
\item
  \phantomsection\label{symbol-face.joy}{{ 😂 }
  \texttt{\ face.\ joy\ }}
\item
  \phantomsection\label{symbol-face.kiss}{{ ðŸ˜--- }
  \texttt{\ face.\ kiss\ }}
\item
  \phantomsection\label{symbol-face.kiss.smile}{{ 😙 }
  \texttt{\ face.\ kiss.\ smile\ }}
\item
  \phantomsection\label{symbol-face.kiss.heart}{{ 😘 }
  \texttt{\ face.\ kiss.\ heart\ }}
\item
  \phantomsection\label{symbol-face.kiss.blush}{{ 😚 }
  \texttt{\ face.\ kiss.\ blush\ }}
\item
  \phantomsection\label{symbol-face.lick}{{ 😋 }
  \texttt{\ face.\ lick\ }}
\item
  \phantomsection\label{symbol-face.lie}{{ 🤥 }
  \texttt{\ face.\ lie\ }}
\item
  \phantomsection\label{symbol-face.mask}{{ 😷 }
  \texttt{\ face.\ mask\ }}
\item
  \phantomsection\label{symbol-face.meh}{{ ðŸ˜' }
  \texttt{\ face.\ meh\ }}
\item
  \phantomsection\label{symbol-face.melt}{{ 🫠}
  \texttt{\ face.\ melt\ }}
\item
  \phantomsection\label{symbol-face.money}{{ ðŸ¤` }
  \texttt{\ face.\ money\ }}
\item
  \phantomsection\label{symbol-face.monocle}{{ � }
  \texttt{\ face.\ monocle\ }}
\item
  \phantomsection\label{symbol-face.nausea}{{ 🤢 }
  \texttt{\ face.\ nausea\ }}
\item
  \phantomsection\label{symbol-face.nerd}{{ ðŸ¤`` }
  \texttt{\ face.\ nerd\ }}
\item
  \phantomsection\label{symbol-face.neutral}{{ � }
  \texttt{\ face.\ neutral\ }}
\item
  \phantomsection\label{symbol-face.open}{{ 😃 }
  \texttt{\ face.\ open\ }}
\item
  \phantomsection\label{symbol-face.party}{{ 🥳 }
  \texttt{\ face.\ party\ }}
\item
  \phantomsection\label{symbol-face.peek}{{ 🫣 }
  \texttt{\ face.\ peek\ }}
\item
  \phantomsection\label{symbol-face.plead}{{ 🥺 }
  \texttt{\ face.\ plead\ }}
\item
  \phantomsection\label{symbol-face.relief}{{ 😌 }
  \texttt{\ face.\ relief\ }}
\item
  \phantomsection\label{symbol-face.rofl}{{ 🤣 }
  \texttt{\ face.\ rofl\ }}
\item
  \phantomsection\label{symbol-face.sad}{{ ðŸ˜'' }
  \texttt{\ face.\ sad\ }}
\item
  \phantomsection\label{symbol-face.salute}{{ 🫡 }
  \texttt{\ face.\ salute\ }}
\item
  \phantomsection\label{symbol-face.shock}{{ 😱 }
  \texttt{\ face.\ shock\ }}
\item
  \phantomsection\label{symbol-face.shush}{{ 🤫 }
  \texttt{\ face.\ shush\ }}
\item
  \phantomsection\label{symbol-face.skeptic}{{ 🤨 }
  \texttt{\ face.\ skeptic\ }}
\item
  \phantomsection\label{symbol-face.sleep}{{ 😴 }
  \texttt{\ face.\ sleep\ }}
\item
  \phantomsection\label{symbol-face.sleepy}{{ 😪 }
  \texttt{\ face.\ sleepy\ }}
\item
  \phantomsection\label{symbol-face.smile}{{ 😄 }
  \texttt{\ face.\ smile\ }}
\item
  \phantomsection\label{symbol-face.smile.slight}{{ 🙂 }
  \texttt{\ face.\ smile.\ slight\ }}
\item
  \phantomsection\label{symbol-face.smile.sweat}{{ 😠}
  \texttt{\ face.\ smile.\ sweat\ }}
\item
  \phantomsection\label{symbol-face.smile.tear}{{ 🥲 }
  \texttt{\ face.\ smile.\ tear\ }}
\item
  \phantomsection\label{symbol-face.smirk}{{ � }
  \texttt{\ face.\ smirk\ }}
\item
  \phantomsection\label{symbol-face.sneeze}{{ 🤧 }
  \texttt{\ face.\ sneeze\ }}
\item
  \phantomsection\label{symbol-face.speak.not}{{ 🫢 }
  \texttt{\ face.\ speak.\ not\ }}
\item
  \phantomsection\label{symbol-face.squint}{{ 😆 }
  \texttt{\ face.\ squint\ }}
\item
  \phantomsection\label{symbol-face.stars}{{ 🤩 }
  \texttt{\ face.\ stars\ }}
\item
  \phantomsection\label{symbol-face.straight}{{ ðŸ˜` }
  \texttt{\ face.\ straight\ }}
\item
  \phantomsection\label{symbol-face.suffer}{{ ðŸ˜-- }
  \texttt{\ face.\ suffer\ }}
\item
  \phantomsection\label{symbol-face.surprise}{{ 😯 }
  \texttt{\ face.\ surprise\ }}
\item
  \phantomsection\label{symbol-face.symbols}{{ 🤬 }
  \texttt{\ face.\ symbols\ }}
\item
  \phantomsection\label{symbol-face.tear}{{ 😢 }
  \texttt{\ face.\ tear\ }}
\item
  \phantomsection\label{symbol-face.tear.relief}{{ 😥 }
  \texttt{\ face.\ tear.\ relief\ }}
\item
  \phantomsection\label{symbol-face.tear.withheld}{{ 🥹 }
  \texttt{\ face.\ tear.\ withheld\ }}
\item
  \phantomsection\label{symbol-face.teeth}{{ 😬 }
  \texttt{\ face.\ teeth\ }}
\item
  \phantomsection\label{symbol-face.think}{{ ðŸ¤'' }
  \texttt{\ face.\ think\ }}
\item
  \phantomsection\label{symbol-face.tired}{{ 🫩 }
  \texttt{\ face.\ tired\ }}
\item
  \phantomsection\label{symbol-face.tongue}{{ 😛 }
  \texttt{\ face.\ tongue\ }}
\item
  \phantomsection\label{symbol-face.tongue.squint}{{ � }
  \texttt{\ face.\ tongue.\ squint\ }}
\item
  \phantomsection\label{symbol-face.tongue.wink}{{ 😜 }
  \texttt{\ face.\ tongue.\ wink\ }}
\item
  \phantomsection\label{symbol-face.triumph}{{ 😤 }
  \texttt{\ face.\ triumph\ }}
\item
  \phantomsection\label{symbol-face.unhappy}{{ 😕 }
  \texttt{\ face.\ unhappy\ }}
\item
  \phantomsection\label{symbol-face.vomit}{{ 🤮 }
  \texttt{\ face.\ vomit\ }}
\item
  \phantomsection\label{symbol-face.weary}{{ 😩 }
  \texttt{\ face.\ weary\ }}
\item
  \phantomsection\label{symbol-face.wink}{{ 😉 }
  \texttt{\ face.\ wink\ }}
\item
  \phantomsection\label{symbol-face.woozy}{{ 🥴 }
  \texttt{\ face.\ woozy\ }}
\item
  \phantomsection\label{symbol-face.worry}{{ 😟 }
  \texttt{\ face.\ worry\ }}
\item
  \phantomsection\label{symbol-face.wow}{{ 😮 }
  \texttt{\ face.\ wow\ }}
\item
  \phantomsection\label{symbol-face.yawn}{{ 🥱 }
  \texttt{\ face.\ yawn\ }}
\item
  \phantomsection\label{symbol-face.zip}{{ � }
  \texttt{\ face.\ zip\ }}
\item
  \phantomsection\label{symbol-factory}{{ � } \texttt{\ factory\ }}
\item
  \phantomsection\label{symbol-fairy}{{ 🧚 } \texttt{\ fairy\ }}
\item
  \phantomsection\label{symbol-faith.christ}{{ � }
  \texttt{\ faith.\ christ\ }}
\item
  \phantomsection\label{symbol-faith.dharma}{{ ☸ }
  \texttt{\ faith.\ dharma\ }}
\item
  \phantomsection\label{symbol-faith.islam}{{ ☪ }
  \texttt{\ faith.\ islam\ }}
\item
  \phantomsection\label{symbol-faith.judaism}{{ ✡ }
  \texttt{\ faith.\ judaism\ }}
\item
  \phantomsection\label{symbol-faith.menorah}{{ 🕎 }
  \texttt{\ faith.\ menorah\ }}
\item
  \phantomsection\label{symbol-faith.om}{{ 🕉 }
  \texttt{\ faith.\ om\ }}
\item
  \phantomsection\label{symbol-faith.orthodox}{{ ☦ }
  \texttt{\ faith.\ orthodox\ }}
\item
  \phantomsection\label{symbol-faith.peace}{{ ☮ }
  \texttt{\ faith.\ peace\ }}
\item
  \phantomsection\label{symbol-faith.star.dot}{{ ðŸ''¯ }
  \texttt{\ faith.\ star.\ dot\ }}
\item
  \phantomsection\label{symbol-faith.worship}{{ � }
  \texttt{\ faith.\ worship\ }}
\item
  \phantomsection\label{symbol-faith.yinyang}{{ ☯ }
  \texttt{\ faith.\ yinyang\ }}
\item
  \phantomsection\label{symbol-falafel}{{ 🧆 } \texttt{\ falafel\ }}
\item
  \phantomsection\label{symbol-family}{{ ðŸ`ª } \texttt{\ family\ }}
\item
  \phantomsection\label{symbol-fax}{{ ðŸ`` } \texttt{\ fax\ }}
\item
  \phantomsection\label{symbol-feather}{{ 🪶 } \texttt{\ feather\ }}
\item
  \phantomsection\label{symbol-feeding.breast}{{ 🤱 }
  \texttt{\ feeding.\ breast\ }}
\item
  \phantomsection\label{symbol-fencing}{{ 🤺 } \texttt{\ fencing\ }}
\item
  \phantomsection\label{symbol-ferriswheel}{{ 🎡 }
  \texttt{\ ferriswheel\ }}
\item
  \phantomsection\label{symbol-filebox}{{ ðŸ---ƒ } \texttt{\ filebox\ }}
\item
  \phantomsection\label{symbol-filedividers}{{ ðŸ---‚ }
  \texttt{\ filedividers\ }}
\item
  \phantomsection\label{symbol-film}{{ 🎞 } \texttt{\ film\ }}
\item
  \phantomsection\label{symbol-finger.r}{{ ðŸ`‰ }
  \texttt{\ finger.\ r\ }}
\item
  \phantomsection\label{symbol-finger.l}{{ ðŸ`ˆ }
  \texttt{\ finger.\ l\ }}
\item
  \phantomsection\label{symbol-finger.t}{{ ðŸ`† }
  \texttt{\ finger.\ t\ }}
\item
  \phantomsection\label{symbol-finger.t.alt}{{ � }
  \texttt{\ finger.\ t.\ alt\ }}
\item
  \phantomsection\label{symbol-finger.b}{{ ðŸ`‡ }
  \texttt{\ finger.\ b\ }}
\item
  \phantomsection\label{symbol-finger.front}{{ 🫵 }
  \texttt{\ finger.\ front\ }}
\item
  \phantomsection\label{symbol-finger.m}{{ ðŸ--• }
  \texttt{\ finger.\ m\ }}
\item
  \phantomsection\label{symbol-fingerprint}{{ 🫆 }
  \texttt{\ fingerprint\ }}
\item
  \phantomsection\label{symbol-fingers.cross}{{ 🤞 }
  \texttt{\ fingers.\ cross\ }}
\item
  \phantomsection\label{symbol-fingers.pinch}{{ 🤌 }
  \texttt{\ fingers.\ pinch\ }}
\item
  \phantomsection\label{symbol-fingers.snap}{{ 🫰 }
  \texttt{\ fingers.\ snap\ }}
\item
  \phantomsection\label{symbol-fire}{{ ðŸ''¥ } \texttt{\ fire\ }}
\item
  \phantomsection\label{symbol-firecracker}{{ 🧨 }
  \texttt{\ firecracker\ }}
\item
  \phantomsection\label{symbol-fireengine}{{ ðŸš' }
  \texttt{\ fireengine\ }}
\item
  \phantomsection\label{symbol-fireworks}{{ 🎆 }
  \texttt{\ fireworks\ }}
\item
  \phantomsection\label{symbol-fish}{{ � } \texttt{\ fish\ }}
\item
  \phantomsection\label{symbol-fish.tropical}{{ ðŸ? }
  \texttt{\ fish.\ tropical\ }}
\item
  \phantomsection\label{symbol-fishing}{{ 🎣 } \texttt{\ fishing\ }}
\item
  \phantomsection\label{symbol-fist.front}{{ ðŸ`Š }
  \texttt{\ fist.\ front\ }}
\item
  \phantomsection\label{symbol-fist.r}{{ 🤜 } \texttt{\ fist.\ r\ }}
\item
  \phantomsection\label{symbol-fist.l}{{ 🤛 } \texttt{\ fist.\ l\ }}
\item
  \phantomsection\label{symbol-fist.raised}{{ ✊ }
  \texttt{\ fist.\ raised\ }}
\item
  \phantomsection\label{symbol-flag.black}{{ � }
  \texttt{\ flag.\ black\ }}
\item
  \phantomsection\label{symbol-flag.white}{{ � }
  \texttt{\ flag.\ white\ }}
\item
  \phantomsection\label{symbol-flag.goal}{{ ðŸ?? }
  \texttt{\ flag.\ goal\ }}
\item
  \phantomsection\label{symbol-flag.golf}{{ ⛳ }
  \texttt{\ flag.\ golf\ }}
\item
  \phantomsection\label{symbol-flag.red}{{ 🚩 }
  \texttt{\ flag.\ red\ }}
\item
  \phantomsection\label{symbol-flags.jp.crossed}{{ 🎌 }
  \texttt{\ flags.\ jp.\ crossed\ }}
\item
  \phantomsection\label{symbol-flamingo}{{ 🦩 } \texttt{\ flamingo\ }}
\item
  \phantomsection\label{symbol-flashlight}{{ ðŸ''¦ }
  \texttt{\ flashlight\ }}
\item
  \phantomsection\label{symbol-flatbread}{{ ðŸ«`` }
  \texttt{\ flatbread\ }}
\item
  \phantomsection\label{symbol-fleur}{{ ⚜ } \texttt{\ fleur\ }}
\item
  \phantomsection\label{symbol-floppy}{{ ðŸ'¾ } \texttt{\ floppy\ }}
\item
  \phantomsection\label{symbol-flower.hibiscus}{{ 🌺 }
  \texttt{\ flower.\ hibiscus\ }}
\item
  \phantomsection\label{symbol-flower.lotus}{{ 🪷 }
  \texttt{\ flower.\ lotus\ }}
\item
  \phantomsection\label{symbol-flower.pink}{{ 🌸 }
  \texttt{\ flower.\ pink\ }}
\item
  \phantomsection\label{symbol-flower.rose}{{ 🌹 }
  \texttt{\ flower.\ rose\ }}
\item
  \phantomsection\label{symbol-flower.sun}{{ 🌻 }
  \texttt{\ flower.\ sun\ }}
\item
  \phantomsection\label{symbol-flower.tulip}{{ 🌷 }
  \texttt{\ flower.\ tulip\ }}
\item
  \phantomsection\label{symbol-flower.white}{{ ðŸ'® }
  \texttt{\ flower.\ white\ }}
\item
  \phantomsection\label{symbol-flower.wilted}{{ 🥀 }
  \texttt{\ flower.\ wilted\ }}
\item
  \phantomsection\label{symbol-flower.yellow}{{ 🌼 }
  \texttt{\ flower.\ yellow\ }}
\item
  \phantomsection\label{symbol-fly}{{ 🪰 } \texttt{\ fly\ }}
\item
  \phantomsection\label{symbol-fog}{{ 🌫 } \texttt{\ fog\ }}
\item
  \phantomsection\label{symbol-folder}{{ ðŸ``? } \texttt{\ folder\ }}
\item
  \phantomsection\label{symbol-folder.open}{{ ðŸ``‚ }
  \texttt{\ folder.\ open\ }}
\item
  \phantomsection\label{symbol-fondue}{{ 🫕 } \texttt{\ fondue\ }}
\item
  \phantomsection\label{symbol-foot}{{ 🦶 } \texttt{\ foot\ }}
\item
  \phantomsection\label{symbol-football}{{ âš½ } \texttt{\ football\ }}
\item
  \phantomsection\label{symbol-football.am}{{ � }
  \texttt{\ football.\ am\ }}
\item
  \phantomsection\label{symbol-forex}{{ ðŸ'± } \texttt{\ forex\ }}
\item
  \phantomsection\label{symbol-fountain}{{ ⛲ } \texttt{\ fountain\ }}
\item
  \phantomsection\label{symbol-fox}{{ 🦊 } \texttt{\ fox\ }}
\item
  \phantomsection\label{symbol-free}{{ ðŸ†`` } \texttt{\ free\ }}
\item
  \phantomsection\label{symbol-fries}{{ � } \texttt{\ fries\ }}
\item
  \phantomsection\label{symbol-frisbee}{{ � } \texttt{\ frisbee\ }}
\item
  \phantomsection\label{symbol-frog.face}{{ � }
  \texttt{\ frog.\ face\ }}
\item
  \phantomsection\label{symbol-fuelpump}{{ ⛽ } \texttt{\ fuelpump\ }}
\item
  \phantomsection\label{symbol-garlic}{{ 🧄 } \texttt{\ garlic\ }}
\item
  \phantomsection\label{symbol-gear}{{ âš™ } \texttt{\ gear\ }}
\item
  \phantomsection\label{symbol-gem}{{ ðŸ'Ž } \texttt{\ gem\ }}
\item
  \phantomsection\label{symbol-genie}{{ 🧞 } \texttt{\ genie\ }}
\item
  \phantomsection\label{symbol-ghost}{{ ðŸ`» } \texttt{\ ghost\ }}
\item
  \phantomsection\label{symbol-giraffe}{{ ðŸ¦' } \texttt{\ giraffe\ }}
\item
  \phantomsection\label{symbol-girl}{{ ðŸ`§ } \texttt{\ girl\ }}
\item
  \phantomsection\label{symbol-glass.clink}{{ 🥂 }
  \texttt{\ glass.\ clink\ }}
\item
  \phantomsection\label{symbol-glass.milk}{{ 🥛 }
  \texttt{\ glass.\ milk\ }}
\item
  \phantomsection\label{symbol-glass.pour}{{ ðŸ«--- }
  \texttt{\ glass.\ pour\ }}
\item
  \phantomsection\label{symbol-glass.tumbler}{{ 🥃 }
  \texttt{\ glass.\ tumbler\ }}
\item
  \phantomsection\label{symbol-glasses}{{ ðŸ`\,`` }
  \texttt{\ glasses\ }}
\item
  \phantomsection\label{symbol-glasses.sun}{{ 🕶 }
  \texttt{\ glasses.\ sun\ }}
\item
  \phantomsection\label{symbol-globe.am}{{ 🌎 }
  \texttt{\ globe.\ am\ }}
\item
  \phantomsection\label{symbol-globe.as.au}{{ � }
  \texttt{\ globe.\ as.\ au\ }}
\item
  \phantomsection\label{symbol-globe.eu.af}{{ � }
  \texttt{\ globe.\ eu.\ af\ }}
\item
  \phantomsection\label{symbol-globe.meridian}{{ � }
  \texttt{\ globe.\ meridian\ }}
\item
  \phantomsection\label{symbol-gloves}{{ 🧤 } \texttt{\ gloves\ }}
\item
  \phantomsection\label{symbol-goal}{{ 🥠} \texttt{\ goal\ }}
\item
  \phantomsection\label{symbol-goat}{{ ðŸ?? } \texttt{\ goat\ }}
\item
  \phantomsection\label{symbol-goggles}{{ 🥽 } \texttt{\ goggles\ }}
\item
  \phantomsection\label{symbol-golfing}{{ � } \texttt{\ golfing\ }}
\item
  \phantomsection\label{symbol-gorilla}{{ � } \texttt{\ gorilla\ }}
\item
  \phantomsection\label{symbol-grapes}{{ � } \texttt{\ grapes\ }}
\item
  \phantomsection\label{symbol-guard.man}{{ ðŸ'‚ }
  \texttt{\ guard.\ man\ }}
\item
  \phantomsection\label{symbol-guitar}{{ 🎸 } \texttt{\ guitar\ }}
\item
  \phantomsection\label{symbol-gymnastics}{{ 🤸 }
  \texttt{\ gymnastics\ }}
\item
  \phantomsection\label{symbol-haircut}{{ ðŸ'‡ } \texttt{\ haircut\ }}
\item
  \phantomsection\label{symbol-hammer}{{ ðŸ''¨ } \texttt{\ hammer\ }}
\item
  \phantomsection\label{symbol-hammer.pick}{{ âš' }
  \texttt{\ hammer.\ pick\ }}
\item
  \phantomsection\label{symbol-hammer.wrench}{{ 🛠}
  \texttt{\ hammer.\ wrench\ }}
\item
  \phantomsection\label{symbol-hamsa}{{ 🪬 } \texttt{\ hamsa\ }}
\item
  \phantomsection\label{symbol-hamster.face}{{ � }
  \texttt{\ hamster.\ face\ }}
\item
  \phantomsection\label{symbol-hand.raised}{{ ✋ }
  \texttt{\ hand.\ raised\ }}
\item
  \phantomsection\label{symbol-hand.raised.alt}{{ 🤚 }
  \texttt{\ hand.\ raised.\ alt\ }}
\item
  \phantomsection\label{symbol-hand.r}{{ 🫱 } \texttt{\ hand.\ r\ }}
\item
  \phantomsection\label{symbol-hand.l}{{ 🫲 } \texttt{\ hand.\ l\ }}
\item
  \phantomsection\label{symbol-hand.t}{{ 🫴 } \texttt{\ hand.\ t\ }}
\item
  \phantomsection\label{symbol-hand.b}{{ 🫳 } \texttt{\ hand.\ b\ }}
\item
  \phantomsection\label{symbol-hand.ok}{{ ðŸ`Œ } \texttt{\ hand.\ ok\ }}
\item
  \phantomsection\label{symbol-hand.call}{{ 🤙 }
  \texttt{\ hand.\ call\ }}
\item
  \phantomsection\label{symbol-hand.love}{{ 🤟 }
  \texttt{\ hand.\ love\ }}
\item
  \phantomsection\label{symbol-hand.part}{{ ðŸ---- }
  \texttt{\ hand.\ part\ }}
\item
  \phantomsection\label{symbol-hand.peace}{{ ✌ }
  \texttt{\ hand.\ peace\ }}
\item
  \phantomsection\label{symbol-hand.pinch}{{ � }
  \texttt{\ hand.\ pinch\ }}
\item
  \phantomsection\label{symbol-hand.rock}{{ 🤘 }
  \texttt{\ hand.\ rock\ }}
\item
  \phantomsection\label{symbol-hand.splay}{{ ðŸ--? }
  \texttt{\ hand.\ splay\ }}
\item
  \phantomsection\label{symbol-hand.wave}{{ ðŸ`‹ }
  \texttt{\ hand.\ wave\ }}
\item
  \phantomsection\label{symbol-hand.write}{{ � }
  \texttt{\ hand.\ write\ }}
\item
  \phantomsection\label{symbol-handbag}{{ ðŸ`œ } \texttt{\ handbag\ }}
\item
  \phantomsection\label{symbol-handball}{{ 🤾 } \texttt{\ handball\ }}
\item
  \phantomsection\label{symbol-handholding.man.man}{{ ðŸ`¬ }
  \texttt{\ handholding.\ man.\ man\ }}
\item
  \phantomsection\label{symbol-handholding.woman.man}{{ ðŸ`« }
  \texttt{\ handholding.\ woman.\ man\ }}
\item
  \phantomsection\label{symbol-handholding.woman.woman}{{ ðŸ`­ }
  \texttt{\ handholding.\ woman.\ woman\ }}
\item
  \phantomsection\label{symbol-hands.folded}{{ � }
  \texttt{\ hands.\ folded\ }}
\item
  \phantomsection\label{symbol-hands.palms}{{ 🤲 }
  \texttt{\ hands.\ palms\ }}
\item
  \phantomsection\label{symbol-hands.clap}{{ ðŸ`? }
  \texttt{\ hands.\ clap\ }}
\item
  \phantomsection\label{symbol-hands.heart}{{ 🫶 }
  \texttt{\ hands.\ heart\ }}
\item
  \phantomsection\label{symbol-hands.open}{{ ðŸ`? }
  \texttt{\ hands.\ open\ }}
\item
  \phantomsection\label{symbol-hands.raised}{{ 🙌 }
  \texttt{\ hands.\ raised\ }}
\item
  \phantomsection\label{symbol-hands.shake}{{ � }
  \texttt{\ hands.\ shake\ }}
\item
  \phantomsection\label{symbol-harp}{{ 🪉 } \texttt{\ harp\ }}
\item
  \phantomsection\label{symbol-hash}{{ \# } \texttt{\ hash\ }}
\item
  \phantomsection\label{symbol-hat.ribbon}{{ ðŸ`\,' }
  \texttt{\ hat.\ ribbon\ }}
\item
  \phantomsection\label{symbol-hat.top}{{ 🎩 } \texttt{\ hat.\ top\ }}
\item
  \phantomsection\label{symbol-headphone}{{ 🎧 }
  \texttt{\ headphone\ }}
\item
  \phantomsection\label{symbol-heart}{{ â?¤ } \texttt{\ heart\ }}
\item
  \phantomsection\label{symbol-heart.arrow}{{ ðŸ'˜ }
  \texttt{\ heart.\ arrow\ }}
\item
  \phantomsection\label{symbol-heart.beat}{{ ðŸ'\,`` }
  \texttt{\ heart.\ beat\ }}
\item
  \phantomsection\label{symbol-heart.black}{{ ðŸ--¤ }
  \texttt{\ heart.\ black\ }}
\item
  \phantomsection\label{symbol-heart.blue}{{ ðŸ'™ }
  \texttt{\ heart.\ blue\ }}
\item
  \phantomsection\label{symbol-heart.box}{{ ðŸ'Ÿ }
  \texttt{\ heart.\ box\ }}
\item
  \phantomsection\label{symbol-heart.broken}{{ ðŸ'\,'' }
  \texttt{\ heart.\ broken\ }}
\item
  \phantomsection\label{symbol-heart.brown}{{ 🤎 }
  \texttt{\ heart.\ brown\ }}
\item
  \phantomsection\label{symbol-heart.double}{{ ðŸ'• }
  \texttt{\ heart.\ double\ }}
\item
  \phantomsection\label{symbol-heart.excl}{{ â?£ }
  \texttt{\ heart.\ excl\ }}
\item
  \phantomsection\label{symbol-heart.green}{{ ðŸ'š }
  \texttt{\ heart.\ green\ }}
\item
  \phantomsection\label{symbol-heart.grow}{{ ðŸ'--- }
  \texttt{\ heart.\ grow\ }}
\item
  \phantomsection\label{symbol-heart.orange}{{ 🧡 }
  \texttt{\ heart.\ orange\ }}
\item
  \phantomsection\label{symbol-heart.purple}{{ ðŸ'œ }
  \texttt{\ heart.\ purple\ }}
\item
  \phantomsection\label{symbol-heart.real}{{ 🫀 }
  \texttt{\ heart.\ real\ }}
\item
  \phantomsection\label{symbol-heart.revolve}{{ ðŸ'ž }
  \texttt{\ heart.\ revolve\ }}
\item
  \phantomsection\label{symbol-heart.ribbon}{{ ðŸ'? }
  \texttt{\ heart.\ ribbon\ }}
\item
  \phantomsection\label{symbol-heart.spark}{{ ðŸ'-- }
  \texttt{\ heart.\ spark\ }}
\item
  \phantomsection\label{symbol-heart.white}{{ � }
  \texttt{\ heart.\ white\ }}
\item
  \phantomsection\label{symbol-heart.yellow}{{ ðŸ'› }
  \texttt{\ heart.\ yellow\ }}
\item
  \phantomsection\label{symbol-hedgehog}{{ ðŸ¦'' }
  \texttt{\ hedgehog\ }}
\item
  \phantomsection\label{symbol-helicopter}{{ � }
  \texttt{\ helicopter\ }}
\item
  \phantomsection\label{symbol-helix}{{ 🧬 } \texttt{\ helix\ }}
\item
  \phantomsection\label{symbol-helmet.cross}{{ â›` }
  \texttt{\ helmet.\ cross\ }}
\item
  \phantomsection\label{symbol-helmet.military}{{ ðŸª-- }
  \texttt{\ helmet.\ military\ }}
\item
  \phantomsection\label{symbol-hippo}{{ 🦛 } \texttt{\ hippo\ }}
\item
  \phantomsection\label{symbol-hockey}{{ ðŸ?{}` } \texttt{\ hockey\ }}
\item
  \phantomsection\label{symbol-hole}{{ 🕳 } \texttt{\ hole\ }}
\item
  \phantomsection\label{symbol-honey}{{ � } \texttt{\ honey\ }}
\item
  \phantomsection\label{symbol-hongbao}{{ 🧧 } \texttt{\ hongbao\ }}
\item
  \phantomsection\label{symbol-hook}{{ � } \texttt{\ hook\ }}
\item
  \phantomsection\label{symbol-horn.postal}{{ ðŸ``¯ }
  \texttt{\ horn.\ postal\ }}
\item
  \phantomsection\label{symbol-horse}{{ � } \texttt{\ horse\ }}
\item
  \phantomsection\label{symbol-horse.carousel}{{ 🎠}
  \texttt{\ horse.\ carousel\ }}
\item
  \phantomsection\label{symbol-horse.face}{{ � }
  \texttt{\ horse.\ face\ }}
\item
  \phantomsection\label{symbol-horse.race}{{ � }
  \texttt{\ horse.\ race\ }}
\item
  \phantomsection\label{symbol-hospital}{{ � } \texttt{\ hospital\ }}
\item
  \phantomsection\label{symbol-hotdog}{{ 🌭 } \texttt{\ hotdog\ }}
\item
  \phantomsection\label{symbol-hotel}{{ � } \texttt{\ hotel\ }}
\item
  \phantomsection\label{symbol-hotel.love}{{ � }
  \texttt{\ hotel.\ love\ }}
\item
  \phantomsection\label{symbol-hotspring}{{ ♨ }
  \texttt{\ hotspring\ }}
\item
  \phantomsection\label{symbol-hourglass}{{ ⌛ }
  \texttt{\ hourglass\ }}
\item
  \phantomsection\label{symbol-hourglass.flow}{{ â?³ }
  \texttt{\ hourglass.\ flow\ }}
\item
  \phantomsection\label{symbol-house}{{ ðŸ? } \texttt{\ house\ }}
\item
  \phantomsection\label{symbol-house.derelict}{{ � }
  \texttt{\ house.\ derelict\ }}
\item
  \phantomsection\label{symbol-house.garden}{{ � }
  \texttt{\ house.\ garden\ }}
\item
  \phantomsection\label{symbol-house.multiple}{{ � }
  \texttt{\ house.\ multiple\ }}
\item
  \phantomsection\label{symbol-hundred}{{ ðŸ'¯ } \texttt{\ hundred\ }}
\item
  \phantomsection\label{symbol-hut}{{ ðŸ›-- } \texttt{\ hut\ }}
\item
  \phantomsection\label{symbol-ice}{{ 🧊 } \texttt{\ ice\ }}
\item
  \phantomsection\label{symbol-icecream}{{ � } \texttt{\ icecream\ }}
\item
  \phantomsection\label{symbol-icecream.shaved}{{ � }
  \texttt{\ icecream.\ shaved\ }}
\item
  \phantomsection\label{symbol-icecream.soft}{{ � }
  \texttt{\ icecream.\ soft\ }}
\item
  \phantomsection\label{symbol-icehockey}{{ ðŸ?' }
  \texttt{\ icehockey\ }}
\item
  \phantomsection\label{symbol-id}{{ ðŸ†'' } \texttt{\ id\ }}
\item
  \phantomsection\label{symbol-info}{{ ℹ } \texttt{\ info\ }}
\item
  \phantomsection\label{symbol-izakaya}{{ � } \texttt{\ izakaya\ }}
\item
  \phantomsection\label{symbol-jar}{{ 🫙 } \texttt{\ jar\ }}
\item
  \phantomsection\label{symbol-jeans}{{ ðŸ`-- } \texttt{\ jeans\ }}
\item
  \phantomsection\label{symbol-jigsaw}{{ 🧩 } \texttt{\ jigsaw\ }}
\item
  \phantomsection\label{symbol-joystick}{{ 🕹 } \texttt{\ joystick\ }}
\item
  \phantomsection\label{symbol-juggling}{{ 🤹 } \texttt{\ juggling\ }}
\item
  \phantomsection\label{symbol-juice}{{ 🧃 } \texttt{\ juice\ }}
\item
  \phantomsection\label{symbol-kaaba}{{ 🕋 } \texttt{\ kaaba\ }}
\item
  \phantomsection\label{symbol-kadomatsu}{{ � }
  \texttt{\ kadomatsu\ }}
\item
  \phantomsection\label{symbol-kangaroo}{{ 🦘 } \texttt{\ kangaroo\ }}
\item
  \phantomsection\label{symbol-gachi}{{ 🈷 } \texttt{\ gachi\ }}
\item
  \phantomsection\label{symbol-go}{{ 🈴 } \texttt{\ go\ }}
\item
  \phantomsection\label{symbol-hi}{{ ㊙ } \texttt{\ hi\ }}
\item
  \phantomsection\label{symbol-ka}{{ ðŸ‰` } \texttt{\ ka\ }}
\item
  \phantomsection\label{symbol-kachi}{{ 🈹 } \texttt{\ kachi\ }}
\item
  \phantomsection\label{symbol-kara}{{ 🈳 } \texttt{\ kara\ }}
\item
  \phantomsection\label{symbol-kon}{{ 🈲 } \texttt{\ kon\ }}
\item
  \phantomsection\label{symbol-man}{{ ðŸ`¨ } \texttt{\ man\ }}
\item
  \phantomsection\label{symbol-man.box}{{ 🚹 } \texttt{\ man.\ box\ }}
\item
  \phantomsection\label{symbol-man.crown}{{ 🤴 }
  \texttt{\ man.\ crown\ }}
\item
  \phantomsection\label{symbol-man.guapimao}{{ ðŸ`² }
  \texttt{\ man.\ guapimao\ }}
\item
  \phantomsection\label{symbol-man.levitate}{{ 🕴 }
  \texttt{\ man.\ levitate\ }}
\item
  \phantomsection\label{symbol-man.old}{{ ðŸ`´ } \texttt{\ man.\ old\ }}
\item
  \phantomsection\label{symbol-man.pregnant}{{ 🫃 }
  \texttt{\ man.\ pregnant\ }}
\item
  \phantomsection\label{symbol-man.turban}{{ ðŸ`³ }
  \texttt{\ man.\ turban\ }}
\item
  \phantomsection\label{symbol-man.tuxedo}{{ 🤵 }
  \texttt{\ man.\ tuxedo\ }}
\item
  \phantomsection\label{symbol-muryo}{{ 🈚 } \texttt{\ muryo\ }}
\item
  \phantomsection\label{symbol-shin}{{ 🈸 } \texttt{\ shin\ }}
\item
  \phantomsection\label{symbol-shuku}{{ ãŠ--- } \texttt{\ shuku\ }}
\item
  \phantomsection\label{symbol-toku}{{ � } \texttt{\ toku\ }}
\item
  \phantomsection\label{symbol-yo}{{ 🈺 } \texttt{\ yo\ }}
\item
  \phantomsection\label{symbol-yubi}{{ 🈯 } \texttt{\ yubi\ }}
\item
  \phantomsection\label{symbol-yuryo}{{ 🈶 } \texttt{\ yuryo\ }}
\item
  \phantomsection\label{symbol-koko}{{ � } \texttt{\ koko\ }}
\item
  \phantomsection\label{symbol-sa}{{ 🈂 } \texttt{\ sa\ }}
\item
  \phantomsection\label{symbol-kebab}{{ 🥙 } \texttt{\ kebab\ }}
\item
  \phantomsection\label{symbol-key}{{ ðŸ''\,` } \texttt{\ key\ }}
\item
  \phantomsection\label{symbol-key.old}{{ ðŸ---? }
  \texttt{\ key.\ old\ }}
\item
  \phantomsection\label{symbol-keyboard}{{ ⌨ } \texttt{\ keyboard\ }}
\item
  \phantomsection\label{symbol-kiss}{{ ðŸ'? } \texttt{\ kiss\ }}
\item
  \phantomsection\label{symbol-kissmark}{{ ðŸ'‹ } \texttt{\ kissmark\ }}
\item
  \phantomsection\label{symbol-kite}{{ � } \texttt{\ kite\ }}
\item
  \phantomsection\label{symbol-kiwi}{{ � } \texttt{\ kiwi\ }}
\item
  \phantomsection\label{symbol-knife}{{ ðŸ''ª } \texttt{\ knife\ }}
\item
  \phantomsection\label{symbol-knife.dagger}{{ ðŸ---¡ }
  \texttt{\ knife.\ dagger\ }}
\item
  \phantomsection\label{symbol-knot}{{ 🪢 } \texttt{\ knot\ }}
\item
  \phantomsection\label{symbol-koala}{{ � } \texttt{\ koala\ }}
\item
  \phantomsection\label{symbol-koinobori}{{ � }
  \texttt{\ koinobori\ }}
\item
  \phantomsection\label{symbol-label}{{ � } \texttt{\ label\ }}
\item
  \phantomsection\label{symbol-lacrosse}{{ � } \texttt{\ lacrosse\ }}
\item
  \phantomsection\label{symbol-ladder}{{ 🪜 } \texttt{\ ladder\ }}
\item
  \phantomsection\label{symbol-lamp.diya}{{ ðŸª'' }
  \texttt{\ lamp.\ diya\ }}
\item
  \phantomsection\label{symbol-laptop}{{ ðŸ'» } \texttt{\ laptop\ }}
\item
  \phantomsection\label{symbol-a}{{ ðŸ\ldots° } \texttt{\ a\ }}
\item
  \phantomsection\label{symbol-ab}{{ 🆎 } \texttt{\ ab\ }}
\item
  \phantomsection\label{symbol-b}{{ ðŸ\ldots± } \texttt{\ b\ }}
\item
  \phantomsection\label{symbol-cl}{{ ðŸ†` } \texttt{\ cl\ }}
\item
  \phantomsection\label{symbol-o}{{ ðŸ\ldots¾ } \texttt{\ o\ }}
\item
  \phantomsection\label{symbol-leaf.clover.three}{{ ☘ }
  \texttt{\ leaf.\ clover.\ three\ }}
\item
  \phantomsection\label{symbol-leaf.clover.four}{{ � }
  \texttt{\ leaf.\ clover.\ four\ }}
\item
  \phantomsection\label{symbol-leaf.fall}{{ � }
  \texttt{\ leaf.\ fall\ }}
\item
  \phantomsection\label{symbol-leaf.herb}{{ 🌿 }
  \texttt{\ leaf.\ herb\ }}
\item
  \phantomsection\label{symbol-leaf.maple}{{ ðŸ?? }
  \texttt{\ leaf.\ maple\ }}
\item
  \phantomsection\label{symbol-leaf.wind}{{ � }
  \texttt{\ leaf.\ wind\ }}
\item
  \phantomsection\label{symbol-leftluggage}{{ 🛠}
  \texttt{\ leftluggage\ }}
\item
  \phantomsection\label{symbol-leg}{{ 🦵 } \texttt{\ leg\ }}
\item
  \phantomsection\label{symbol-leg.mech}{{ 🦿 }
  \texttt{\ leg.\ mech\ }}
\item
  \phantomsection\label{symbol-lemon}{{ � } \texttt{\ lemon\ }}
\item
  \phantomsection\label{symbol-leopard}{{ � } \texttt{\ leopard\ }}
\item
  \phantomsection\label{symbol-letter.love}{{ ðŸ'Œ }
  \texttt{\ letter.\ love\ }}
\item
  \phantomsection\label{symbol-liberty}{{ ðŸ---½ } \texttt{\ liberty\ }}
\item
  \phantomsection\label{symbol-lightbulb}{{ ðŸ'¡ }
  \texttt{\ lightbulb\ }}
\item
  \phantomsection\label{symbol-lightning}{{ âš¡ }
  \texttt{\ lightning\ }}
\item
  \phantomsection\label{symbol-lion}{{ � } \texttt{\ lion\ }}
\item
  \phantomsection\label{symbol-lipstick}{{ ðŸ'„ } \texttt{\ lipstick\ }}
\item
  \phantomsection\label{symbol-litter}{{ 🚮 } \texttt{\ litter\ }}
\item
  \phantomsection\label{symbol-litter.not}{{ 🚯 }
  \texttt{\ litter.\ not\ }}
\item
  \phantomsection\label{symbol-lizard}{{ 🦎 } \texttt{\ lizard\ }}
\item
  \phantomsection\label{symbol-llama}{{ 🦙 } \texttt{\ llama\ }}
\item
  \phantomsection\label{symbol-lobster}{{ 🦞 } \texttt{\ lobster\ }}
\item
  \phantomsection\label{symbol-lock}{{ ðŸ''\,' } \texttt{\ lock\ }}
\item
  \phantomsection\label{symbol-lock.key}{{ ðŸ''? }
  \texttt{\ lock.\ key\ }}
\item
  \phantomsection\label{symbol-lock.open}{{ ðŸ''\,`` }
  \texttt{\ lock.\ open\ }}
\item
  \phantomsection\label{symbol-lock.pen}{{ ðŸ''? }
  \texttt{\ lock.\ pen\ }}
\item
  \phantomsection\label{symbol-lollipop}{{ � } \texttt{\ lollipop\ }}
\item
  \phantomsection\label{symbol-lotion}{{ 🧴 } \texttt{\ lotion\ }}
\item
  \phantomsection\label{symbol-luggage}{{ 🧳 } \texttt{\ luggage\ }}
\item
  \phantomsection\label{symbol-lungs}{{ � } \texttt{\ lungs\ }}
\item
  \phantomsection\label{symbol-mage}{{ 🧙 } \texttt{\ mage\ }}
\item
  \phantomsection\label{symbol-magnet}{{ 🧲 } \texttt{\ magnet\ }}
\item
  \phantomsection\label{symbol-magnify.r}{{ ðŸ''Ž }
  \texttt{\ magnify.\ r\ }}
\item
  \phantomsection\label{symbol-magnify.l}{{ ðŸ''? }
  \texttt{\ magnify.\ l\ }}
\item
  \phantomsection\label{symbol-mahjong.dragon.red}{{ 🀄 }
  \texttt{\ mahjong.\ dragon.\ red\ }}
\item
  \phantomsection\label{symbol-mail}{{ ✉ } \texttt{\ mail\ }}
\item
  \phantomsection\label{symbol-mail.arrow}{{ ðŸ``© }
  \texttt{\ mail.\ arrow\ }}
\item
  \phantomsection\label{symbol-mailbox.closed.empty}{{ ðŸ``ª }
  \texttt{\ mailbox.\ closed.\ empty\ }}
\item
  \phantomsection\label{symbol-mailbox.closed.full}{{ ðŸ``« }
  \texttt{\ mailbox.\ closed.\ full\ }}
\item
  \phantomsection\label{symbol-mailbox.open.empty}{{ ðŸ``­ }
  \texttt{\ mailbox.\ open.\ empty\ }}
\item
  \phantomsection\label{symbol-mailbox.open.full}{{ ðŸ``¬ }
  \texttt{\ mailbox.\ open.\ full\ }}
\item
  \phantomsection\label{symbol-mammoth}{{ 🦣 } \texttt{\ mammoth\ }}
\item
  \phantomsection\label{symbol-mango}{{ 🥭 } \texttt{\ mango\ }}
\item
  \phantomsection\label{symbol-map.world}{{ ðŸ---º }
  \texttt{\ map.\ world\ }}
\item
  \phantomsection\label{symbol-map.jp}{{ ðŸ---¾ } \texttt{\ map.\ jp\ }}
\item
  \phantomsection\label{symbol-martialarts}{{ 🥋 }
  \texttt{\ martialarts\ }}
\item
  \phantomsection\label{symbol-masks}{{ 🎭 } \texttt{\ masks\ }}
\item
  \phantomsection\label{symbol-mate}{{ 🧉 } \texttt{\ mate\ }}
\item
  \phantomsection\label{symbol-matryoshka}{{ 🪆 }
  \texttt{\ matryoshka\ }}
\item
  \phantomsection\label{symbol-meat}{{ 🥩 } \texttt{\ meat\ }}
\item
  \phantomsection\label{symbol-meat.bone}{{ ðŸ?-- }
  \texttt{\ meat.\ bone\ }}
\item
  \phantomsection\label{symbol-medal.first}{{ 🥇 }
  \texttt{\ medal.\ first\ }}
\item
  \phantomsection\label{symbol-medal.second}{{ 🥈 }
  \texttt{\ medal.\ second\ }}
\item
  \phantomsection\label{symbol-medal.third}{{ 🥉 }
  \texttt{\ medal.\ third\ }}
\item
  \phantomsection\label{symbol-medal.sports}{{ ðŸ? }
  \texttt{\ medal.\ sports\ }}
\item
  \phantomsection\label{symbol-medal.military}{{ ðŸŽ-- }
  \texttt{\ medal.\ military\ }}
\item
  \phantomsection\label{symbol-megaphone}{{ ðŸ``¢ }
  \texttt{\ megaphone\ }}
\item
  \phantomsection\label{symbol-megaphone.simple}{{ ðŸ``£ }
  \texttt{\ megaphone.\ simple\ }}
\item
  \phantomsection\label{symbol-melon}{{ � } \texttt{\ melon\ }}
\item
  \phantomsection\label{symbol-merperson}{{ 🧜 }
  \texttt{\ merperson\ }}
\item
  \phantomsection\label{symbol-metro}{{ â``‚ } \texttt{\ metro\ }}
\item
  \phantomsection\label{symbol-microbe}{{ 🦠} \texttt{\ microbe\ }}
\item
  \phantomsection\label{symbol-microphone}{{ 🎤 }
  \texttt{\ microphone\ }}
\item
  \phantomsection\label{symbol-microphone.studio}{{ 🎙 }
  \texttt{\ microphone.\ studio\ }}
\item
  \phantomsection\label{symbol-microscope}{{ ðŸ''¬ }
  \texttt{\ microscope\ }}
\item
  \phantomsection\label{symbol-milkyway}{{ 🌌 } \texttt{\ milkyway\ }}
\item
  \phantomsection\label{symbol-mirror}{{ 🪞 } \texttt{\ mirror\ }}
\item
  \phantomsection\label{symbol-mixer}{{ 🎛 } \texttt{\ mixer\ }}
\item
  \phantomsection\label{symbol-money.bag}{{ ðŸ'° }
  \texttt{\ money.\ bag\ }}
\item
  \phantomsection\label{symbol-money.dollar}{{ ðŸ'µ }
  \texttt{\ money.\ dollar\ }}
\item
  \phantomsection\label{symbol-money.euro}{{ ðŸ'¶ }
  \texttt{\ money.\ euro\ }}
\item
  \phantomsection\label{symbol-money.pound}{{ ðŸ'· }
  \texttt{\ money.\ pound\ }}
\item
  \phantomsection\label{symbol-money.yen}{{ ðŸ'´ }
  \texttt{\ money.\ yen\ }}
\item
  \phantomsection\label{symbol-money.wings}{{ ðŸ'¸ }
  \texttt{\ money.\ wings\ }}
\item
  \phantomsection\label{symbol-monkey}{{ ðŸ?' } \texttt{\ monkey\ }}
\item
  \phantomsection\label{symbol-monkey.face}{{ � }
  \texttt{\ monkey.\ face\ }}
\item
  \phantomsection\label{symbol-monkey.hear.not}{{ 🙉 }
  \texttt{\ monkey.\ hear.\ not\ }}
\item
  \phantomsection\label{symbol-monkey.see.not}{{ 🙈 }
  \texttt{\ monkey.\ see.\ not\ }}
\item
  \phantomsection\label{symbol-monkey.speak.not}{{ 🙊 }
  \texttt{\ monkey.\ speak.\ not\ }}
\item
  \phantomsection\label{symbol-moon.crescent}{{ 🌙 }
  \texttt{\ moon.\ crescent\ }}
\item
  \phantomsection\label{symbol-moon.full}{{ 🌕 }
  \texttt{\ moon.\ full\ }}
\item
  \phantomsection\label{symbol-moon.full.face}{{ � }
  \texttt{\ moon.\ full.\ face\ }}
\item
  \phantomsection\label{symbol-moon.new}{{ ðŸŒ` }
  \texttt{\ moon.\ new\ }}
\item
  \phantomsection\label{symbol-moon.new.face}{{ 🌚 }
  \texttt{\ moon.\ new.\ face\ }}
\item
  \phantomsection\label{symbol-moon.wane.one}{{ ðŸŒ-- }
  \texttt{\ moon.\ wane.\ one\ }}
\item
  \phantomsection\label{symbol-moon.wane.two}{{ ðŸŒ--- }
  \texttt{\ moon.\ wane.\ two\ }}
\item
  \phantomsection\label{symbol-moon.wane.three.face}{{ 🌜 }
  \texttt{\ moon.\ wane.\ three.\ face\ }}
\item
  \phantomsection\label{symbol-moon.wane.three}{{ 🌘 }
  \texttt{\ moon.\ wane.\ three\ }}
\item
  \phantomsection\label{symbol-moon.wax.one}{{ ðŸŒ' }
  \texttt{\ moon.\ wax.\ one\ }}
\item
  \phantomsection\label{symbol-moon.wax.two}{{ ðŸŒ`` }
  \texttt{\ moon.\ wax.\ two\ }}
\item
  \phantomsection\label{symbol-moon.wax.two.face}{{ 🌛 }
  \texttt{\ moon.\ wax.\ two.\ face\ }}
\item
  \phantomsection\label{symbol-moon.wax.three}{{ ðŸŒ'' }
  \texttt{\ moon.\ wax.\ three\ }}
\item
  \phantomsection\label{symbol-mortarboard}{{ ðŸŽ`` }
  \texttt{\ mortarboard\ }}
\item
  \phantomsection\label{symbol-mosque}{{ 🕌 } \texttt{\ mosque\ }}
\item
  \phantomsection\label{symbol-mosquito}{{ 🦟 } \texttt{\ mosquito\ }}
\item
  \phantomsection\label{symbol-motorcycle}{{ ðŸ?? }
  \texttt{\ motorcycle\ }}
\item
  \phantomsection\label{symbol-motorway}{{ 🛣 } \texttt{\ motorway\ }}
\item
  \phantomsection\label{symbol-mountain}{{ â›° } \texttt{\ mountain\ }}
\item
  \phantomsection\label{symbol-mountain.fuji}{{ ðŸ---» }
  \texttt{\ mountain.\ fuji\ }}
\item
  \phantomsection\label{symbol-mountain.snow}{{ ðŸ?'' }
  \texttt{\ mountain.\ snow\ }}
\item
  \phantomsection\label{symbol-mountain.sunrise}{{ 🌄 }
  \texttt{\ mountain.\ sunrise\ }}
\item
  \phantomsection\label{symbol-mouse}{{ ðŸ?? } \texttt{\ mouse\ }}
\item
  \phantomsection\label{symbol-mouse.face}{{ � }
  \texttt{\ mouse.\ face\ }}
\item
  \phantomsection\label{symbol-mousetrap}{{ 🪤 }
  \texttt{\ mousetrap\ }}
\item
  \phantomsection\label{symbol-mouth}{{ ðŸ`„ } \texttt{\ mouth\ }}
\item
  \phantomsection\label{symbol-mouth.bite}{{ 🫦 }
  \texttt{\ mouth.\ bite\ }}
\item
  \phantomsection\label{symbol-moyai}{{ ðŸ---¿ } \texttt{\ moyai\ }}
\item
  \phantomsection\label{symbol-museum}{{ � } \texttt{\ museum\ }}
\item
  \phantomsection\label{symbol-mushroom}{{ � } \texttt{\ mushroom\ }}
\item
  \phantomsection\label{symbol-musicalscore}{{ 🎼 }
  \texttt{\ musicalscore\ }}
\item
  \phantomsection\label{symbol-nails.polish}{{ ðŸ' }
  \texttt{\ nails.\ polish\ }}
\item
  \phantomsection\label{symbol-namebadge}{{ ðŸ``› }
  \texttt{\ namebadge\ }}
\item
  \phantomsection\label{symbol-nazar}{{ 🧿 } \texttt{\ nazar\ }}
\item
  \phantomsection\label{symbol-necktie}{{ ðŸ`\,'' }
  \texttt{\ necktie\ }}
\item
  \phantomsection\label{symbol-needle}{{ 🪡 } \texttt{\ needle\ }}
\item
  \phantomsection\label{symbol-nest.empty}{{ 🪹 }
  \texttt{\ nest.\ empty\ }}
\item
  \phantomsection\label{symbol-nest.eggs}{{ 🪺 }
  \texttt{\ nest.\ eggs\ }}
\item
  \phantomsection\label{symbol-new}{{ 🆕 } \texttt{\ new\ }}
\item
  \phantomsection\label{symbol-newspaper}{{ ðŸ``° }
  \texttt{\ newspaper\ }}
\item
  \phantomsection\label{symbol-newspaper.rolled}{{ ðŸ---ž }
  \texttt{\ newspaper.\ rolled\ }}
\item
  \phantomsection\label{symbol-ng}{{ ðŸ†-- } \texttt{\ ng\ }}
\item
  \phantomsection\label{symbol-ningyo}{{ 🎎 } \texttt{\ ningyo\ }}
\item
  \phantomsection\label{symbol-ninja}{{ 🥷 } \texttt{\ ninja\ }}
\item
  \phantomsection\label{symbol-noentry}{{ â›'' } \texttt{\ noentry\ }}
\item
  \phantomsection\label{symbol-nose}{{ ðŸ`ƒ } \texttt{\ nose\ }}
\item
  \phantomsection\label{symbol-notebook}{{ ðŸ``\,`` }
  \texttt{\ notebook\ }}
\item
  \phantomsection\label{symbol-notebook.deco}{{ ðŸ``\,'' }
  \texttt{\ notebook.\ deco\ }}
\item
  \phantomsection\label{symbol-notepad}{{ ðŸ---' } \texttt{\ notepad\ }}
\item
  \phantomsection\label{symbol-notes}{{ 🎵 } \texttt{\ notes\ }}
\item
  \phantomsection\label{symbol-notes.triple}{{ 🎶 }
  \texttt{\ notes.\ triple\ }}
\item
  \phantomsection\label{symbol-numbers}{{ ðŸ''¢ } \texttt{\ numbers\ }}
\item
  \phantomsection\label{symbol-octopus}{{ � } \texttt{\ octopus\ }}
\item
  \phantomsection\label{symbol-office}{{ � } \texttt{\ office\ }}
\item
  \phantomsection\label{symbol-oil}{{ 🛢 } \texttt{\ oil\ }}
\item
  \phantomsection\label{symbol-ok}{{ ðŸ†--- } \texttt{\ ok\ }}
\item
  \phantomsection\label{symbol-olive}{{ ðŸ«' } \texttt{\ olive\ }}
\item
  \phantomsection\label{symbol-oni}{{ ðŸ`¹ } \texttt{\ oni\ }}
\item
  \phantomsection\label{symbol-onion}{{ 🧠} \texttt{\ onion\ }}
\item
  \phantomsection\label{symbol-orangutan}{{ 🦧 }
  \texttt{\ orangutan\ }}
\item
  \phantomsection\label{symbol-otter}{{ 🦦 } \texttt{\ otter\ }}
\item
  \phantomsection\label{symbol-owl}{{ 🦉 } \texttt{\ owl\ }}
\item
  \phantomsection\label{symbol-ox}{{ � } \texttt{\ ox\ }}
\item
  \phantomsection\label{symbol-oyster}{{ 🦪 } \texttt{\ oyster\ }}
\item
  \phantomsection\label{symbol-package}{{ ðŸ``¦ } \texttt{\ package\ }}
\item
  \phantomsection\label{symbol-paella}{{ 🥘 } \texttt{\ paella\ }}
\item
  \phantomsection\label{symbol-page}{{ ðŸ``„ } \texttt{\ page\ }}
\item
  \phantomsection\label{symbol-page.curl}{{ ðŸ``ƒ }
  \texttt{\ page.\ curl\ }}
\item
  \phantomsection\label{symbol-page.pencil}{{ ðŸ``? }
  \texttt{\ page.\ pencil\ }}
\item
  \phantomsection\label{symbol-pager}{{ ðŸ``Ÿ } \texttt{\ pager\ }}
\item
  \phantomsection\label{symbol-pages.tabs}{{ ðŸ``\,` }
  \texttt{\ pages.\ tabs\ }}
\item
  \phantomsection\label{symbol-painting}{{ ðŸ--¼ }
  \texttt{\ painting\ }}
\item
  \phantomsection\label{symbol-palette}{{ 🎨 } \texttt{\ palette\ }}
\item
  \phantomsection\label{symbol-pancakes}{{ 🥞 } \texttt{\ pancakes\ }}
\item
  \phantomsection\label{symbol-panda}{{ � } \texttt{\ panda\ }}
\item
  \phantomsection\label{symbol-parachute}{{ 🪂 }
  \texttt{\ parachute\ }}
\item
  \phantomsection\label{symbol-park}{{ � } \texttt{\ park\ }}
\item
  \phantomsection\label{symbol-parking}{{ ðŸ\ldots¿ }
  \texttt{\ parking\ }}
\item
  \phantomsection\label{symbol-parrot}{{ 🦜 } \texttt{\ parrot\ }}
\item
  \phantomsection\label{symbol-partalteration}{{ 〽 }
  \texttt{\ partalteration\ }}
\item
  \phantomsection\label{symbol-party}{{ 🎉 } \texttt{\ party\ }}
\item
  \phantomsection\label{symbol-peach}{{ ðŸ?{}` } \texttt{\ peach\ }}
\item
  \phantomsection\label{symbol-peacock}{{ 🦚 } \texttt{\ peacock\ }}
\item
  \phantomsection\label{symbol-peanuts}{{ 🥜 } \texttt{\ peanuts\ }}
\item
  \phantomsection\label{symbol-pear}{{ ðŸ?? } \texttt{\ pear\ }}
\item
  \phantomsection\label{symbol-pedestrian}{{ 🚶 }
  \texttt{\ pedestrian\ }}
\item
  \phantomsection\label{symbol-pedestrian.not}{{ 🚷 }
  \texttt{\ pedestrian.\ not\ }}
\item
  \phantomsection\label{symbol-pen.ball}{{ ðŸ--Š }
  \texttt{\ pen.\ ball\ }}
\item
  \phantomsection\label{symbol-pen.fountain}{{ ðŸ--‹ }
  \texttt{\ pen.\ fountain\ }}
\item
  \phantomsection\label{symbol-pencil}{{ � } \texttt{\ pencil\ }}
\item
  \phantomsection\label{symbol-penguin}{{ � } \texttt{\ penguin\ }}
\item
  \phantomsection\label{symbol-pepper}{{ ðŸ«` } \texttt{\ pepper\ }}
\item
  \phantomsection\label{symbol-pepper.hot}{{ 🌶 }
  \texttt{\ pepper.\ hot\ }}
\item
  \phantomsection\label{symbol-person}{{ ðŸ§` } \texttt{\ person\ }}
\item
  \phantomsection\label{symbol-person.angry}{{ 🙎 }
  \texttt{\ person.\ angry\ }}
\item
  \phantomsection\label{symbol-person.beard}{{ ðŸ§'' }
  \texttt{\ person.\ beard\ }}
\item
  \phantomsection\label{symbol-person.blonde}{{ ðŸ`± }
  \texttt{\ person.\ blonde\ }}
\item
  \phantomsection\label{symbol-person.bow}{{ 🙇 }
  \texttt{\ person.\ bow\ }}
\item
  \phantomsection\label{symbol-person.crown}{{ 🫠}
  \texttt{\ person.\ crown\ }}
\item
  \phantomsection\label{symbol-person.deaf}{{ � }
  \texttt{\ person.\ deaf\ }}
\item
  \phantomsection\label{symbol-person.facepalm}{{ 🤦 }
  \texttt{\ person.\ facepalm\ }}
\item
  \phantomsection\label{symbol-person.frown}{{ � }
  \texttt{\ person.\ frown\ }}
\item
  \phantomsection\label{symbol-person.hijab}{{ 🧕 }
  \texttt{\ person.\ hijab\ }}
\item
  \phantomsection\label{symbol-person.kneel}{{ 🧎 }
  \texttt{\ person.\ kneel\ }}
\item
  \phantomsection\label{symbol-person.lotus}{{ 🧘 }
  \texttt{\ person.\ lotus\ }}
\item
  \phantomsection\label{symbol-person.massage}{{ ðŸ'† }
  \texttt{\ person.\ massage\ }}
\item
  \phantomsection\label{symbol-person.no}{{ 🙠}
  \texttt{\ person.\ no\ }}
\item
  \phantomsection\label{symbol-person.ok}{{ 🙆 }
  \texttt{\ person.\ ok\ }}
\item
  \phantomsection\label{symbol-person.old}{{ ðŸ§`` }
  \texttt{\ person.\ old\ }}
\item
  \phantomsection\label{symbol-person.pregnant}{{ 🫄 }
  \texttt{\ person.\ pregnant\ }}
\item
  \phantomsection\label{symbol-person.raise}{{ 🙋 }
  \texttt{\ person.\ raise\ }}
\item
  \phantomsection\label{symbol-person.sassy}{{ ðŸ'? }
  \texttt{\ person.\ sassy\ }}
\item
  \phantomsection\label{symbol-person.shrug}{{ 🤷 }
  \texttt{\ person.\ shrug\ }}
\item
  \phantomsection\label{symbol-person.stand}{{ � }
  \texttt{\ person.\ stand\ }}
\item
  \phantomsection\label{symbol-person.steam}{{ ðŸ§-- }
  \texttt{\ person.\ steam\ }}
\item
  \phantomsection\label{symbol-petri}{{ 🧫 } \texttt{\ petri\ }}
\item
  \phantomsection\label{symbol-phone}{{ ðŸ``± } \texttt{\ phone\ }}
\item
  \phantomsection\label{symbol-phone.arrow}{{ ðŸ``² }
  \texttt{\ phone.\ arrow\ }}
\item
  \phantomsection\label{symbol-phone.classic}{{ ☎ }
  \texttt{\ phone.\ classic\ }}
\item
  \phantomsection\label{symbol-phone.not}{{ ðŸ``µ }
  \texttt{\ phone.\ not\ }}
\item
  \phantomsection\label{symbol-phone.off}{{ ðŸ``´ }
  \texttt{\ phone.\ off\ }}
\item
  \phantomsection\label{symbol-phone.receiver}{{ ðŸ``ž }
  \texttt{\ phone.\ receiver\ }}
\item
  \phantomsection\label{symbol-phone.signal}{{ ðŸ``¶ }
  \texttt{\ phone.\ signal\ }}
\item
  \phantomsection\label{symbol-phone.vibrate}{{ ðŸ``³ }
  \texttt{\ phone.\ vibrate\ }}
\item
  \phantomsection\label{symbol-piano}{{ 🎹 } \texttt{\ piano\ }}
\item
  \phantomsection\label{symbol-pick}{{ â›? } \texttt{\ pick\ }}
\item
  \phantomsection\label{symbol-pie}{{ 🥧 } \texttt{\ pie\ }}
\item
  \phantomsection\label{symbol-pig}{{ ðŸ?-- } \texttt{\ pig\ }}
\item
  \phantomsection\label{symbol-pig.face}{{ � }
  \texttt{\ pig.\ face\ }}
\item
  \phantomsection\label{symbol-pig.nose}{{ � }
  \texttt{\ pig.\ nose\ }}
\item
  \phantomsection\label{symbol-pill}{{ ðŸ'Š } \texttt{\ pill\ }}
\item
  \phantomsection\label{symbol-pin}{{ ðŸ``Œ } \texttt{\ pin\ }}
\item
  \phantomsection\label{symbol-pin.round}{{ ðŸ``? }
  \texttt{\ pin.\ round\ }}
\item
  \phantomsection\label{symbol-pinata}{{ 🪠} \texttt{\ pinata\ }}
\item
  \phantomsection\label{symbol-pineapple}{{ ðŸ?? }
  \texttt{\ pineapple\ }}
\item
  \phantomsection\label{symbol-pingpong}{{ ðŸ?{}`` }
  \texttt{\ pingpong\ }}
\item
  \phantomsection\label{symbol-pistol}{{ ðŸ''« } \texttt{\ pistol\ }}
\item
  \phantomsection\label{symbol-pizza}{{ � } \texttt{\ pizza\ }}
\item
  \phantomsection\label{symbol-placard}{{ 🪧 } \texttt{\ placard\ }}
\item
  \phantomsection\label{symbol-planet}{{ � } \texttt{\ planet\ }}
\item
  \phantomsection\label{symbol-plant}{{ 🪴 } \texttt{\ plant\ }}
\item
  \phantomsection\label{symbol-plaster}{{ 🩹 } \texttt{\ plaster\ }}
\item
  \phantomsection\label{symbol-plate.cutlery}{{ � }
  \texttt{\ plate.\ cutlery\ }}
\item
  \phantomsection\label{symbol-playback.down}{{ â?¬ }
  \texttt{\ playback.\ down\ }}
\item
  \phantomsection\label{symbol-playback.eject}{{ â?? }
  \texttt{\ playback.\ eject\ }}
\item
  \phantomsection\label{symbol-playback.forward}{{ â?© }
  \texttt{\ playback.\ forward\ }}
\item
  \phantomsection\label{symbol-playback.pause}{{ â?¸ }
  \texttt{\ playback.\ pause\ }}
\item
  \phantomsection\label{symbol-playback.record}{{ â?º }
  \texttt{\ playback.\ record\ }}
\item
  \phantomsection\label{symbol-playback.repeat}{{ ðŸ''? }
  \texttt{\ playback.\ repeat\ }}
\item
  \phantomsection\label{symbol-playback.repeat.once}{{ ðŸ''‚ }
  \texttt{\ playback.\ repeat.\ once\ }}
\item
  \phantomsection\label{symbol-playback.repeat.v}{{ ðŸ''ƒ }
  \texttt{\ playback.\ repeat.\ v\ }}
\item
  \phantomsection\label{symbol-playback.restart}{{ â?® }
  \texttt{\ playback.\ restart\ }}
\item
  \phantomsection\label{symbol-playback.rewind}{{ â?ª }
  \texttt{\ playback.\ rewind\ }}
\item
  \phantomsection\label{symbol-playback.shuffle}{{ ðŸ''€ }
  \texttt{\ playback.\ shuffle\ }}
\item
  \phantomsection\label{symbol-playback.skip}{{ â?­ }
  \texttt{\ playback.\ skip\ }}
\item
  \phantomsection\label{symbol-playback.stop}{{ â?¹ }
  \texttt{\ playback.\ stop\ }}
\item
  \phantomsection\label{symbol-playback.toggle}{{ â?¯ }
  \texttt{\ playback.\ toggle\ }}
\item
  \phantomsection\label{symbol-playback.up}{{ â?« }
  \texttt{\ playback.\ up\ }}
\item
  \phantomsection\label{symbol-playingcard.flower}{{ 🎴 }
  \texttt{\ playingcard.\ flower\ }}
\item
  \phantomsection\label{symbol-playingcard.joker}{{ � }
  \texttt{\ playingcard.\ joker\ }}
\item
  \phantomsection\label{symbol-plunger}{{ 🪠} \texttt{\ plunger\ }}
\item
  \phantomsection\label{symbol-policeofficer}{{ ðŸ`® }
  \texttt{\ policeofficer\ }}
\item
  \phantomsection\label{symbol-poo}{{ ðŸ'© } \texttt{\ poo\ }}
\item
  \phantomsection\label{symbol-popcorn}{{ � } \texttt{\ popcorn\ }}
\item
  \phantomsection\label{symbol-post.eu}{{ � } \texttt{\ post.\ eu\ }}
\item
  \phantomsection\label{symbol-post.jp}{{ � } \texttt{\ post.\ jp\ }}
\item
  \phantomsection\label{symbol-postbox}{{ ðŸ``® } \texttt{\ postbox\ }}
\item
  \phantomsection\label{symbol-potato}{{ ðŸ¥'' } \texttt{\ potato\ }}
\item
  \phantomsection\label{symbol-potato.sweet}{{ ðŸ? }
  \texttt{\ potato.\ sweet\ }}
\item
  \phantomsection\label{symbol-pouch}{{ ðŸ`? } \texttt{\ pouch\ }}
\item
  \phantomsection\label{symbol-powerplug}{{ ðŸ''Œ }
  \texttt{\ powerplug\ }}
\item
  \phantomsection\label{symbol-present}{{ � } \texttt{\ present\ }}
\item
  \phantomsection\label{symbol-pretzel}{{ 🥨 } \texttt{\ pretzel\ }}
\item
  \phantomsection\label{symbol-printer}{{ ðŸ--¨ } \texttt{\ printer\ }}
\item
  \phantomsection\label{symbol-prints.foot}{{ ðŸ`£ }
  \texttt{\ prints.\ foot\ }}
\item
  \phantomsection\label{symbol-prints.paw}{{ � }
  \texttt{\ prints.\ paw\ }}
\item
  \phantomsection\label{symbol-prohibited}{{ 🚫 }
  \texttt{\ prohibited\ }}
\item
  \phantomsection\label{symbol-projector}{{ ðŸ``½ }
  \texttt{\ projector\ }}
\item
  \phantomsection\label{symbol-pumpkin.lantern}{{ 🎃 }
  \texttt{\ pumpkin.\ lantern\ }}
\item
  \phantomsection\label{symbol-purse}{{ ðŸ`› } \texttt{\ purse\ }}
\item
  \phantomsection\label{symbol-quest}{{ â?{}`` } \texttt{\ quest\ }}
\item
  \phantomsection\label{symbol-quest.white}{{ â?'' }
  \texttt{\ quest.\ white\ }}
\item
  \phantomsection\label{symbol-rabbit}{{ � } \texttt{\ rabbit\ }}
\item
  \phantomsection\label{symbol-rabbit.face}{{ � }
  \texttt{\ rabbit.\ face\ }}
\item
  \phantomsection\label{symbol-raccoon}{{ � } \texttt{\ raccoon\ }}
\item
  \phantomsection\label{symbol-radio}{{ ðŸ``» } \texttt{\ radio\ }}
\item
  \phantomsection\label{symbol-radioactive}{{ ☢ }
  \texttt{\ radioactive\ }}
\item
  \phantomsection\label{symbol-railway}{{ 🛤 } \texttt{\ railway\ }}
\item
  \phantomsection\label{symbol-rainbow}{{ 🌈 } \texttt{\ rainbow\ }}
\item
  \phantomsection\label{symbol-ram}{{ ðŸ?? } \texttt{\ ram\ }}
\item
  \phantomsection\label{symbol-rat}{{ � } \texttt{\ rat\ }}
\item
  \phantomsection\label{symbol-razor}{{ ðŸª' } \texttt{\ razor\ }}
\item
  \phantomsection\label{symbol-receipt}{{ 🧾 } \texttt{\ receipt\ }}
\item
  \phantomsection\label{symbol-recycling}{{ â™» }
  \texttt{\ recycling\ }}
\item
  \phantomsection\label{symbol-reg}{{ ® } \texttt{\ reg\ }}
\item
  \phantomsection\label{symbol-restroom}{{ 🚻 } \texttt{\ restroom\ }}
\item
  \phantomsection\label{symbol-rhino}{{ � } \texttt{\ rhino\ }}
\item
  \phantomsection\label{symbol-ribbon}{{ 🎀 } \texttt{\ ribbon\ }}
\item
  \phantomsection\label{symbol-ribbon.remind}{{ ðŸŽ--- }
  \texttt{\ ribbon.\ remind\ }}
\item
  \phantomsection\label{symbol-rice}{{ � } \texttt{\ rice\ }}
\item
  \phantomsection\label{symbol-rice.cracker}{{ � }
  \texttt{\ rice.\ cracker\ }}
\item
  \phantomsection\label{symbol-rice.ear}{{ 🌾 }
  \texttt{\ rice.\ ear\ }}
\item
  \phantomsection\label{symbol-rice.onigiri}{{ � }
  \texttt{\ rice.\ onigiri\ }}
\item
  \phantomsection\label{symbol-ring}{{ ðŸ'? } \texttt{\ ring\ }}
\item
  \phantomsection\label{symbol-ringbuoy}{{ 🛟 } \texttt{\ ringbuoy\ }}
\item
  \phantomsection\label{symbol-robot}{{ ðŸ¤-- } \texttt{\ robot\ }}
\item
  \phantomsection\label{symbol-rock}{{ 🪨 } \texttt{\ rock\ }}
\item
  \phantomsection\label{symbol-rocket}{{ 🚀 } \texttt{\ rocket\ }}
\item
  \phantomsection\label{symbol-rollercoaster}{{ 🎢 }
  \texttt{\ rollercoaster\ }}
\item
  \phantomsection\label{symbol-rosette}{{ � } \texttt{\ rosette\ }}
\item
  \phantomsection\label{symbol-rugby}{{ � } \texttt{\ rugby\ }}
\item
  \phantomsection\label{symbol-ruler}{{ ðŸ``? } \texttt{\ ruler\ }}
\item
  \phantomsection\label{symbol-ruler.triangle}{{ ðŸ``? }
  \texttt{\ ruler.\ triangle\ }}
\item
  \phantomsection\label{symbol-running}{{ � } \texttt{\ running\ }}
\item
  \phantomsection\label{symbol-safetypin}{{ 🧷 }
  \texttt{\ safetypin\ }}
\item
  \phantomsection\label{symbol-safetyvest}{{ 🦺 }
  \texttt{\ safetyvest\ }}
\item
  \phantomsection\label{symbol-sake}{{ � } \texttt{\ sake\ }}
\item
  \phantomsection\label{symbol-salad}{{ ðŸ¥--- } \texttt{\ salad\ }}
\item
  \phantomsection\label{symbol-salt}{{ 🧂 } \texttt{\ salt\ }}
\item
  \phantomsection\label{symbol-sandwich}{{ 🥪 } \texttt{\ sandwich\ }}
\item
  \phantomsection\label{symbol-santa.man}{{ 🎠}
  \texttt{\ santa.\ man\ }}
\item
  \phantomsection\label{symbol-santa.woman}{{ 🤶 }
  \texttt{\ santa.\ woman\ }}
\item
  \phantomsection\label{symbol-satdish}{{ ðŸ``¡ } \texttt{\ satdish\ }}
\item
  \phantomsection\label{symbol-satellite}{{ 🛰 }
  \texttt{\ satellite\ }}
\item
  \phantomsection\label{symbol-saw}{{ 🪚 } \texttt{\ saw\ }}
\item
  \phantomsection\label{symbol-saxophone}{{ 🎷 }
  \texttt{\ saxophone\ }}
\item
  \phantomsection\label{symbol-scales}{{ âš-- } \texttt{\ scales\ }}
\item
  \phantomsection\label{symbol-scarf}{{ 🧣 } \texttt{\ scarf\ }}
\item
  \phantomsection\label{symbol-school}{{ � } \texttt{\ school\ }}
\item
  \phantomsection\label{symbol-scissors}{{ ✂ } \texttt{\ scissors\ }}
\item
  \phantomsection\label{symbol-scooter}{{ 🛴 } \texttt{\ scooter\ }}
\item
  \phantomsection\label{symbol-scooter.motor}{{ 🛵 }
  \texttt{\ scooter.\ motor\ }}
\item
  \phantomsection\label{symbol-scorpion}{{ 🦂 } \texttt{\ scorpion\ }}
\item
  \phantomsection\label{symbol-screwdriver}{{ 🪛 }
  \texttt{\ screwdriver\ }}
\item
  \phantomsection\label{symbol-scroll}{{ ðŸ``œ } \texttt{\ scroll\ }}
\item
  \phantomsection\label{symbol-seal}{{ 🦭 } \texttt{\ seal\ }}
\item
  \phantomsection\label{symbol-seat}{{ ðŸ'º } \texttt{\ seat\ }}
\item
  \phantomsection\label{symbol-seedling}{{ 🌱 } \texttt{\ seedling\ }}
\item
  \phantomsection\label{symbol-shark}{{ 🦈 } \texttt{\ shark\ }}
\item
  \phantomsection\label{symbol-sheep}{{ ðŸ?{}` } \texttt{\ sheep\ }}
\item
  \phantomsection\label{symbol-shell.spiral}{{ � }
  \texttt{\ shell.\ spiral\ }}
\item
  \phantomsection\label{symbol-shield}{{ 🛡 } \texttt{\ shield\ }}
\item
  \phantomsection\label{symbol-ship}{{ 🚢 } \texttt{\ ship\ }}
\item
  \phantomsection\label{symbol-ship.cruise}{{ 🛳 }
  \texttt{\ ship.\ cruise\ }}
\item
  \phantomsection\label{symbol-ship.ferry}{{ â›´ }
  \texttt{\ ship.\ ferry\ }}
\item
  \phantomsection\label{symbol-shirt.sports}{{ 🎽 }
  \texttt{\ shirt.\ sports\ }}
\item
  \phantomsection\label{symbol-shirt.t}{{ ðŸ`• } \texttt{\ shirt.\ t\ }}
\item
  \phantomsection\label{symbol-shoe}{{ ðŸ`ž } \texttt{\ shoe\ }}
\item
  \phantomsection\label{symbol-shoe.ballet}{{ 🩰 }
  \texttt{\ shoe.\ ballet\ }}
\item
  \phantomsection\label{symbol-shoe.flat}{{ 🥿 }
  \texttt{\ shoe.\ flat\ }}
\item
  \phantomsection\label{symbol-shoe.heel}{{ ðŸ` }
  \texttt{\ shoe.\ heel\ }}
\item
  \phantomsection\label{symbol-shoe.hike}{{ 🥾 }
  \texttt{\ shoe.\ hike\ }}
\item
  \phantomsection\label{symbol-shoe.ice}{{ ⛸ }
  \texttt{\ shoe.\ ice\ }}
\item
  \phantomsection\label{symbol-shoe.roller}{{ 🛼 }
  \texttt{\ shoe.\ roller\ }}
\item
  \phantomsection\label{symbol-shoe.sandal.heel}{{ ðŸ`¡ }
  \texttt{\ shoe.\ sandal.\ heel\ }}
\item
  \phantomsection\label{symbol-shoe.ski}{{ 🎿 }
  \texttt{\ shoe.\ ski\ }}
\item
  \phantomsection\label{symbol-shoe.sneaker}{{ ðŸ`Ÿ }
  \texttt{\ shoe.\ sneaker\ }}
\item
  \phantomsection\label{symbol-shoe.tall}{{ ðŸ`¢ }
  \texttt{\ shoe.\ tall\ }}
\item
  \phantomsection\label{symbol-shoe.thong}{{ 🩴 }
  \texttt{\ shoe.\ thong\ }}
\item
  \phantomsection\label{symbol-shopping}{{ � } \texttt{\ shopping\ }}
\item
  \phantomsection\label{symbol-shorts}{{ 🩳 } \texttt{\ shorts\ }}
\item
  \phantomsection\label{symbol-shoshinsha}{{ ðŸ''° }
  \texttt{\ shoshinsha\ }}
\item
  \phantomsection\label{symbol-shovel}{{ � } \texttt{\ shovel\ }}
\item
  \phantomsection\label{symbol-shower}{{ 🚿 } \texttt{\ shower\ }}
\item
  \phantomsection\label{symbol-shrimp}{{ � } \texttt{\ shrimp\ }}
\item
  \phantomsection\label{symbol-shrimp.fried}{{ � }
  \texttt{\ shrimp.\ fried\ }}
\item
  \phantomsection\label{symbol-shrine}{{ ⛩ } \texttt{\ shrine\ }}
\item
  \phantomsection\label{symbol-sign.crossing}{{ 🚸 }
  \texttt{\ sign.\ crossing\ }}
\item
  \phantomsection\label{symbol-sign.stop}{{ ðŸ›` }
  \texttt{\ sign.\ stop\ }}
\item
  \phantomsection\label{symbol-silhouette}{{ ðŸ`¤ }
  \texttt{\ silhouette\ }}
\item
  \phantomsection\label{symbol-silhouette.double}{{ ðŸ`¥ }
  \texttt{\ silhouette.\ double\ }}
\item
  \phantomsection\label{symbol-silhouette.hug}{{ 🫂 }
  \texttt{\ silhouette.\ hug\ }}
\item
  \phantomsection\label{symbol-silhouette.speak}{{ ðŸ---£ }
  \texttt{\ silhouette.\ speak\ }}
\item
  \phantomsection\label{symbol-siren}{{ 🚨 } \texttt{\ siren\ }}
\item
  \phantomsection\label{symbol-skateboard}{{ 🛹 }
  \texttt{\ skateboard\ }}
\item
  \phantomsection\label{symbol-skewer.dango}{{ � }
  \texttt{\ skewer.\ dango\ }}
\item
  \phantomsection\label{symbol-skewer.oden}{{ � }
  \texttt{\ skewer.\ oden\ }}
\item
  \phantomsection\label{symbol-skiing}{{ â›· } \texttt{\ skiing\ }}
\item
  \phantomsection\label{symbol-skull}{{ ðŸ'€ } \texttt{\ skull\ }}
\item
  \phantomsection\label{symbol-skull.bones}{{ ☠}
  \texttt{\ skull.\ bones\ }}
\item
  \phantomsection\label{symbol-skunk}{{ 🦨 } \texttt{\ skunk\ }}
\item
  \phantomsection\label{symbol-sled}{{ 🛷 } \texttt{\ sled\ }}
\item
  \phantomsection\label{symbol-slide}{{ � } \texttt{\ slide\ }}
\item
  \phantomsection\label{symbol-slider}{{ 🎚 } \texttt{\ slider\ }}
\item
  \phantomsection\label{symbol-sloth}{{ 🦥 } \texttt{\ sloth\ }}
\item
  \phantomsection\label{symbol-slots}{{ 🎰 } \texttt{\ slots\ }}
\item
  \phantomsection\label{symbol-snail}{{ � } \texttt{\ snail\ }}
\item
  \phantomsection\label{symbol-snake}{{ ðŸ?? } \texttt{\ snake\ }}
\item
  \phantomsection\label{symbol-snowboarding}{{ � }
  \texttt{\ snowboarding\ }}
\item
  \phantomsection\label{symbol-snowflake}{{ â?„ }
  \texttt{\ snowflake\ }}
\item
  \phantomsection\label{symbol-snowman}{{ ⛄ } \texttt{\ snowman\ }}
\item
  \phantomsection\label{symbol-snowman.snow}{{ ☃ }
  \texttt{\ snowman.\ snow\ }}
\item
  \phantomsection\label{symbol-soap}{{ 🧼 } \texttt{\ soap\ }}
\item
  \phantomsection\label{symbol-socks}{{ 🧦 } \texttt{\ socks\ }}
\item
  \phantomsection\label{symbol-softball}{{ 🥎 } \texttt{\ softball\ }}
\item
  \phantomsection\label{symbol-sos}{{ 🆘 } \texttt{\ sos\ }}
\item
  \phantomsection\label{symbol-soup}{{ � } \texttt{\ soup\ }}
\item
  \phantomsection\label{symbol-spaghetti}{{ ðŸ?? }
  \texttt{\ spaghetti\ }}
\item
  \phantomsection\label{symbol-sparkle.box}{{ â?‡ }
  \texttt{\ sparkle.\ box\ }}
\item
  \phantomsection\label{symbol-sparkler}{{ 🎇 } \texttt{\ sparkler\ }}
\item
  \phantomsection\label{symbol-sparkles}{{ ✨ } \texttt{\ sparkles\ }}
\item
  \phantomsection\label{symbol-speaker}{{ ðŸ''ˆ } \texttt{\ speaker\ }}
\item
  \phantomsection\label{symbol-speaker.not}{{ ðŸ''‡ }
  \texttt{\ speaker.\ not\ }}
\item
  \phantomsection\label{symbol-speaker.wave}{{ ðŸ''‰ }
  \texttt{\ speaker.\ wave\ }}
\item
  \phantomsection\label{symbol-speaker.waves}{{ ðŸ''Š }
  \texttt{\ speaker.\ waves\ }}
\item
  \phantomsection\label{symbol-spider}{{ 🕷 } \texttt{\ spider\ }}
\item
  \phantomsection\label{symbol-spiderweb}{{ 🕸 }
  \texttt{\ spiderweb\ }}
\item
  \phantomsection\label{symbol-spinach}{{ 🥬 } \texttt{\ spinach\ }}
\item
  \phantomsection\label{symbol-splatter}{{ 🫟 } \texttt{\ splatter\ }}
\item
  \phantomsection\label{symbol-sponge}{{ 🧽 } \texttt{\ sponge\ }}
\item
  \phantomsection\label{symbol-spoon}{{ 🥄 } \texttt{\ spoon\ }}
\item
  \phantomsection\label{symbol-square.black}{{ ⬛ }
  \texttt{\ square.\ black\ }}
\item
  \phantomsection\label{symbol-square.black.tiny}{{ â--ª }
  \texttt{\ square.\ black.\ tiny\ }}
\item
  \phantomsection\label{symbol-square.black.small}{{ â---¾ }
  \texttt{\ square.\ black.\ small\ }}
\item
  \phantomsection\label{symbol-square.black.medium}{{ â---¼ }
  \texttt{\ square.\ black.\ medium\ }}
\item
  \phantomsection\label{symbol-square.white}{{ ⬜ }
  \texttt{\ square.\ white\ }}
\item
  \phantomsection\label{symbol-square.white.tiny}{{ â--« }
  \texttt{\ square.\ white.\ tiny\ }}
\item
  \phantomsection\label{symbol-square.white.small}{{ â---½ }
  \texttt{\ square.\ white.\ small\ }}
\item
  \phantomsection\label{symbol-square.white.medium}{{ â---» }
  \texttt{\ square.\ white.\ medium\ }}
\item
  \phantomsection\label{symbol-square.blue}{{ 🟦 }
  \texttt{\ square.\ blue\ }}
\item
  \phantomsection\label{symbol-square.brown}{{ 🟫 }
  \texttt{\ square.\ brown\ }}
\item
  \phantomsection\label{symbol-square.green}{{ 🟩 }
  \texttt{\ square.\ green\ }}
\item
  \phantomsection\label{symbol-square.orange}{{ 🟧 }
  \texttt{\ square.\ orange\ }}
\item
  \phantomsection\label{symbol-square.purple}{{ 🟪 }
  \texttt{\ square.\ purple\ }}
\item
  \phantomsection\label{symbol-square.red}{{ 🟥 }
  \texttt{\ square.\ red\ }}
\item
  \phantomsection\label{symbol-square.yellow}{{ 🟨 }
  \texttt{\ square.\ yellow\ }}
\item
  \phantomsection\label{symbol-squid}{{ ðŸ¦` } \texttt{\ squid\ }}
\item
  \phantomsection\label{symbol-stadium}{{ � } \texttt{\ stadium\ }}
\item
  \phantomsection\label{symbol-star}{{ â­? } \texttt{\ star\ }}
\item
  \phantomsection\label{symbol-star.arc}{{ ðŸ'« }
  \texttt{\ star.\ arc\ }}
\item
  \phantomsection\label{symbol-star.box}{{ ✴ }
  \texttt{\ star.\ box\ }}
\item
  \phantomsection\label{symbol-star.glow}{{ 🌟 }
  \texttt{\ star.\ glow\ }}
\item
  \phantomsection\label{symbol-star.shoot}{{ 🌠}
  \texttt{\ star.\ shoot\ }}
\item
  \phantomsection\label{symbol-stethoscope}{{ 🩺 }
  \texttt{\ stethoscope\ }}
\item
  \phantomsection\label{symbol-store.big}{{ � }
  \texttt{\ store.\ big\ }}
\item
  \phantomsection\label{symbol-store.small}{{ � }
  \texttt{\ store.\ small\ }}
\item
  \phantomsection\label{symbol-strawberry}{{ ðŸ?{}`` }
  \texttt{\ strawberry\ }}
\item
  \phantomsection\label{symbol-suit.club}{{ ♣ }
  \texttt{\ suit.\ club\ }}
\item
  \phantomsection\label{symbol-suit.diamond}{{ ♦ }
  \texttt{\ suit.\ diamond\ }}
\item
  \phantomsection\label{symbol-suit.heart}{{ ♥ }
  \texttt{\ suit.\ heart\ }}
\item
  \phantomsection\label{symbol-suit.spade}{{ â™ }
  \texttt{\ suit.\ spade\ }}
\item
  \phantomsection\label{symbol-sun}{{ ☀ } \texttt{\ sun\ }}
\item
  \phantomsection\label{symbol-sun.cloud}{{ 🌤 }
  \texttt{\ sun.\ cloud\ }}
\item
  \phantomsection\label{symbol-sun.face}{{ 🌞 }
  \texttt{\ sun.\ face\ }}
\item
  \phantomsection\label{symbol-sunrise}{{ 🌠} \texttt{\ sunrise\ }}
\item
  \phantomsection\label{symbol-superhero}{{ 🦸 }
  \texttt{\ superhero\ }}
\item
  \phantomsection\label{symbol-supervillain}{{ 🦹 }
  \texttt{\ supervillain\ }}
\item
  \phantomsection\label{symbol-surfing}{{ � } \texttt{\ surfing\ }}
\item
  \phantomsection\label{symbol-sushi}{{ � } \texttt{\ sushi\ }}
\item
  \phantomsection\label{symbol-swan}{{ 🦢 } \texttt{\ swan\ }}
\item
  \phantomsection\label{symbol-swimming}{{ � } \texttt{\ swimming\ }}
\item
  \phantomsection\label{symbol-swimsuit}{{ 🩱 } \texttt{\ swimsuit\ }}
\item
  \phantomsection\label{symbol-swords}{{ âš'' } \texttt{\ swords\ }}
\item
  \phantomsection\label{symbol-symbols}{{ ðŸ''£ } \texttt{\ symbols\ }}
\item
  \phantomsection\label{symbol-synagogue}{{ � }
  \texttt{\ synagogue\ }}
\item
  \phantomsection\label{symbol-syringe}{{ ðŸ'‰ } \texttt{\ syringe\ }}
\item
  \phantomsection\label{symbol-taco}{{ 🌮 } \texttt{\ taco\ }}
\item
  \phantomsection\label{symbol-takeout}{{ 🥡 } \texttt{\ takeout\ }}
\item
  \phantomsection\label{symbol-tamale}{{ ðŸ«'' } \texttt{\ tamale\ }}
\item
  \phantomsection\label{symbol-tanabata}{{ 🎋 } \texttt{\ tanabata\ }}
\item
  \phantomsection\label{symbol-tangerine}{{ � }
  \texttt{\ tangerine\ }}
\item
  \phantomsection\label{symbol-tap}{{ 🚰 } \texttt{\ tap\ }}
\item
  \phantomsection\label{symbol-tap.not}{{ 🚱 } \texttt{\ tap.\ not\ }}
\item
  \phantomsection\label{symbol-taxi}{{ 🚕 } \texttt{\ taxi\ }}
\item
  \phantomsection\label{symbol-taxi.front}{{ ðŸš-- }
  \texttt{\ taxi.\ front\ }}
\item
  \phantomsection\label{symbol-teacup}{{ � } \texttt{\ teacup\ }}
\item
  \phantomsection\label{symbol-teapot}{{ ðŸ«-- } \texttt{\ teapot\ }}
\item
  \phantomsection\label{symbol-teddy}{{ 🧸 } \texttt{\ teddy\ }}
\item
  \phantomsection\label{symbol-telescope}{{ ðŸ''­ }
  \texttt{\ telescope\ }}
\item
  \phantomsection\label{symbol-temple}{{ 🛕 } \texttt{\ temple\ }}
\item
  \phantomsection\label{symbol-ten}{{ ðŸ''Ÿ } \texttt{\ ten\ }}
\item
  \phantomsection\label{symbol-tengu}{{ ðŸ`º } \texttt{\ tengu\ }}
\item
  \phantomsection\label{symbol-tennis}{{ 🎾 } \texttt{\ tennis\ }}
\item
  \phantomsection\label{symbol-tent}{{ ⛺ } \texttt{\ tent\ }}
\item
  \phantomsection\label{symbol-testtube}{{ 🧪 } \texttt{\ testtube\ }}
\item
  \phantomsection\label{symbol-thermometer}{{ 🌡 }
  \texttt{\ thermometer\ }}
\item
  \phantomsection\label{symbol-thread}{{ 🧵 } \texttt{\ thread\ }}
\item
  \phantomsection\label{symbol-thumb.up}{{ ðŸ`? }
  \texttt{\ thumb.\ up\ }}
\item
  \phantomsection\label{symbol-thumb.down}{{ ðŸ`Ž }
  \texttt{\ thumb.\ down\ }}
\item
  \phantomsection\label{symbol-ticket.event}{{ 🎟 }
  \texttt{\ ticket.\ event\ }}
\item
  \phantomsection\label{symbol-ticket.travel}{{ 🎫 }
  \texttt{\ ticket.\ travel\ }}
\item
  \phantomsection\label{symbol-tiger}{{ ðŸ? } \texttt{\ tiger\ }}
\item
  \phantomsection\label{symbol-tiger.face}{{ � }
  \texttt{\ tiger.\ face\ }}
\item
  \phantomsection\label{symbol-tm}{{ â„¢ } \texttt{\ tm\ }}
\item
  \phantomsection\label{symbol-toilet}{{ 🚽 } \texttt{\ toilet\ }}
\item
  \phantomsection\label{symbol-toiletpaper}{{ 🧻 }
  \texttt{\ toiletpaper\ }}
\item
  \phantomsection\label{symbol-tomato}{{ ðŸ? } \texttt{\ tomato\ }}
\item
  \phantomsection\label{symbol-tombstone}{{ 🪦 }
  \texttt{\ tombstone\ }}
\item
  \phantomsection\label{symbol-tongue}{{ ðŸ` } \texttt{\ tongue\ }}
\item
  \phantomsection\label{symbol-toolbox}{{ 🧰 } \texttt{\ toolbox\ }}
\item
  \phantomsection\label{symbol-tooth}{{ 🦷 } \texttt{\ tooth\ }}
\item
  \phantomsection\label{symbol-toothbrush}{{ 🪥 }
  \texttt{\ toothbrush\ }}
\item
  \phantomsection\label{symbol-tornado}{{ 🌪 } \texttt{\ tornado\ }}
\item
  \phantomsection\label{symbol-tower.tokyo}{{ ðŸ---¼ }
  \texttt{\ tower.\ tokyo\ }}
\item
  \phantomsection\label{symbol-trackball}{{ ðŸ--² }
  \texttt{\ trackball\ }}
\item
  \phantomsection\label{symbol-tractor}{{ 🚜 } \texttt{\ tractor\ }}
\item
  \phantomsection\label{symbol-trafficlight.v}{{ 🚦 }
  \texttt{\ trafficlight.\ v\ }}
\item
  \phantomsection\label{symbol-trafficlight.h}{{ 🚥 }
  \texttt{\ trafficlight.\ h\ }}
\item
  \phantomsection\label{symbol-train}{{ 🚆 } \texttt{\ train\ }}
\item
  \phantomsection\label{symbol-train.car}{{ 🚃 }
  \texttt{\ train.\ car\ }}
\item
  \phantomsection\label{symbol-train.light}{{ 🚈 }
  \texttt{\ train.\ light\ }}
\item
  \phantomsection\label{symbol-train.metro}{{ 🚇 }
  \texttt{\ train.\ metro\ }}
\item
  \phantomsection\label{symbol-train.mono}{{ � }
  \texttt{\ train.\ mono\ }}
\item
  \phantomsection\label{symbol-train.mountain}{{ 🚞 }
  \texttt{\ train.\ mountain\ }}
\item
  \phantomsection\label{symbol-train.speed}{{ 🚄 }
  \texttt{\ train.\ speed\ }}
\item
  \phantomsection\label{symbol-train.speed.bullet}{{ 🚠}
  \texttt{\ train.\ speed.\ bullet\ }}
\item
  \phantomsection\label{symbol-train.steam}{{ 🚂 }
  \texttt{\ train.\ steam\ }}
\item
  \phantomsection\label{symbol-train.stop}{{ 🚉 }
  \texttt{\ train.\ stop\ }}
\item
  \phantomsection\label{symbol-train.suspend}{{ 🚟 }
  \texttt{\ train.\ suspend\ }}
\item
  \phantomsection\label{symbol-train.tram}{{ 🚊 }
  \texttt{\ train.\ tram\ }}
\item
  \phantomsection\label{symbol-train.tram.car}{{ 🚋 }
  \texttt{\ train.\ tram.\ car\ }}
\item
  \phantomsection\label{symbol-transgender}{{ ⚧ }
  \texttt{\ transgender\ }}
\item
  \phantomsection\label{symbol-tray.inbox}{{ ðŸ``¥ }
  \texttt{\ tray.\ inbox\ }}
\item
  \phantomsection\label{symbol-tray.mail}{{ ðŸ``¨ }
  \texttt{\ tray.\ mail\ }}
\item
  \phantomsection\label{symbol-tray.outbox}{{ ðŸ``¤ }
  \texttt{\ tray.\ outbox\ }}
\item
  \phantomsection\label{symbol-tree.deciduous}{{ 🌳 }
  \texttt{\ tree.\ deciduous\ }}
\item
  \phantomsection\label{symbol-tree.evergreen}{{ 🌲 }
  \texttt{\ tree.\ evergreen\ }}
\item
  \phantomsection\label{symbol-tree.leafless}{{ 🪾 }
  \texttt{\ tree.\ leafless\ }}
\item
  \phantomsection\label{symbol-tree.palm}{{ 🌴 }
  \texttt{\ tree.\ palm\ }}
\item
  \phantomsection\label{symbol-tree.xmas}{{ 🎄 }
  \texttt{\ tree.\ xmas\ }}
\item
  \phantomsection\label{symbol-triangle.r}{{ â--¶ }
  \texttt{\ triangle.\ r\ }}
\item
  \phantomsection\label{symbol-triangle.l}{{ â---€ }
  \texttt{\ triangle.\ l\ }}
\item
  \phantomsection\label{symbol-triangle.t}{{ ðŸ''¼ }
  \texttt{\ triangle.\ t\ }}
\item
  \phantomsection\label{symbol-triangle.b}{{ ðŸ''½ }
  \texttt{\ triangle.\ b\ }}
\item
  \phantomsection\label{symbol-triangle.t.red}{{ ðŸ''º }
  \texttt{\ triangle.\ t.\ red\ }}
\item
  \phantomsection\label{symbol-triangle.b.red}{{ ðŸ''» }
  \texttt{\ triangle.\ b.\ red\ }}
\item
  \phantomsection\label{symbol-trident}{{ ðŸ''± } \texttt{\ trident\ }}
\item
  \phantomsection\label{symbol-troll}{{ 🧌 } \texttt{\ troll\ }}
\item
  \phantomsection\label{symbol-trophy}{{ � } \texttt{\ trophy\ }}
\item
  \phantomsection\label{symbol-truck}{{ 🚚 } \texttt{\ truck\ }}
\item
  \phantomsection\label{symbol-truck.trailer}{{ 🚛 }
  \texttt{\ truck.\ trailer\ }}
\item
  \phantomsection\label{symbol-trumpet}{{ 🎺 } \texttt{\ trumpet\ }}
\item
  \phantomsection\label{symbol-tsukimi}{{ ðŸŽ` } \texttt{\ tsukimi\ }}
\item
  \phantomsection\label{symbol-turkey}{{ 🦃 } \texttt{\ turkey\ }}
\item
  \phantomsection\label{symbol-turtle}{{ � } \texttt{\ turtle\ }}
\item
  \phantomsection\label{symbol-tv}{{ ðŸ``º } \texttt{\ tv\ }}
\item
  \phantomsection\label{symbol-ufo}{{ 🛸 } \texttt{\ ufo\ }}
\item
  \phantomsection\label{symbol-umbrella.open}{{ ☂ }
  \texttt{\ umbrella.\ open\ }}
\item
  \phantomsection\label{symbol-umbrella.closed}{{ 🌂 }
  \texttt{\ umbrella.\ closed\ }}
\item
  \phantomsection\label{symbol-umbrella.rain}{{ â˜'' }
  \texttt{\ umbrella.\ rain\ }}
\item
  \phantomsection\label{symbol-umbrella.sun}{{ â›± }
  \texttt{\ umbrella.\ sun\ }}
\item
  \phantomsection\label{symbol-unicorn}{{ 🦄 } \texttt{\ unicorn\ }}
\item
  \phantomsection\label{symbol-unknown}{{ 🦳 } \texttt{\ unknown\ }}
\item
  \phantomsection\label{symbol-up}{{ 🆙 } \texttt{\ up\ }}
\item
  \phantomsection\label{symbol-urn}{{ âš± } \texttt{\ urn\ }}
\item
  \phantomsection\label{symbol-vampire}{{ 🧛 } \texttt{\ vampire\ }}
\item
  \phantomsection\label{symbol-violin}{{ 🎻 } \texttt{\ violin\ }}
\item
  \phantomsection\label{symbol-volcano}{{ 🌋 } \texttt{\ volcano\ }}
\item
  \phantomsection\label{symbol-volleyball}{{ ðŸ?? }
  \texttt{\ volleyball\ }}
\item
  \phantomsection\label{symbol-vs}{{ 🆚 } \texttt{\ vs\ }}
\item
  \phantomsection\label{symbol-waffle}{{ 🧇 } \texttt{\ waffle\ }}
\item
  \phantomsection\label{symbol-wand}{{ 🪄 } \texttt{\ wand\ }}
\item
  \phantomsection\label{symbol-warning}{{ âš } \texttt{\ warning\ }}
\item
  \phantomsection\label{symbol-watch}{{ ⌚ } \texttt{\ watch\ }}
\item
  \phantomsection\label{symbol-watch.stop}{{ â?± }
  \texttt{\ watch.\ stop\ }}
\item
  \phantomsection\label{symbol-watermelon}{{ � }
  \texttt{\ watermelon\ }}
\item
  \phantomsection\label{symbol-waterpolo}{{ 🤽 }
  \texttt{\ waterpolo\ }}
\item
  \phantomsection\label{symbol-wave}{{ 🌊 } \texttt{\ wave\ }}
\item
  \phantomsection\label{symbol-wc}{{ 🚾 } \texttt{\ wc\ }}
\item
  \phantomsection\label{symbol-weightlifting}{{ � }
  \texttt{\ weightlifting\ }}
\item
  \phantomsection\label{symbol-whale}{{ � } \texttt{\ whale\ }}
\item
  \phantomsection\label{symbol-whale.spout}{{ � }
  \texttt{\ whale.\ spout\ }}
\item
  \phantomsection\label{symbol-wheel}{{ 🛞 } \texttt{\ wheel\ }}
\item
  \phantomsection\label{symbol-wheelchair}{{ 🦽 }
  \texttt{\ wheelchair\ }}
\item
  \phantomsection\label{symbol-wheelchair.box}{{ ♿ }
  \texttt{\ wheelchair.\ box\ }}
\item
  \phantomsection\label{symbol-wheelchair.motor}{{ 🦼 }
  \texttt{\ wheelchair.\ motor\ }}
\item
  \phantomsection\label{symbol-wind}{{ 🌬 } \texttt{\ wind\ }}
\item
  \phantomsection\label{symbol-windchime}{{ � }
  \texttt{\ windchime\ }}
\item
  \phantomsection\label{symbol-window}{{ 🪟 } \texttt{\ window\ }}
\item
  \phantomsection\label{symbol-wine}{{ � } \texttt{\ wine\ }}
\item
  \phantomsection\label{symbol-wolf}{{ � } \texttt{\ wolf\ }}
\item
  \phantomsection\label{symbol-woman}{{ ðŸ`© } \texttt{\ woman\ }}
\item
  \phantomsection\label{symbol-woman.box}{{ 🚺 }
  \texttt{\ woman.\ box\ }}
\item
  \phantomsection\label{symbol-woman.crown}{{ ðŸ`¸ }
  \texttt{\ woman.\ crown\ }}
\item
  \phantomsection\label{symbol-woman.old}{{ ðŸ`µ }
  \texttt{\ woman.\ old\ }}
\item
  \phantomsection\label{symbol-woman.pregnant}{{ 🤰 }
  \texttt{\ woman.\ pregnant\ }}
\item
  \phantomsection\label{symbol-wood}{{ 🪵 } \texttt{\ wood\ }}
\item
  \phantomsection\label{symbol-worm}{{ 🪱 } \texttt{\ worm\ }}
\item
  \phantomsection\label{symbol-wrench}{{ ðŸ''§ } \texttt{\ wrench\ }}
\item
  \phantomsection\label{symbol-wrestling}{{ 🤼 }
  \texttt{\ wrestling\ }}
\item
  \phantomsection\label{symbol-xray}{{ 🩻 } \texttt{\ xray\ }}
\item
  \phantomsection\label{symbol-yarn}{{ 🧶 } \texttt{\ yarn\ }}
\item
  \phantomsection\label{symbol-yoyo}{{ 🪀 } \texttt{\ yoyo\ }}
\item
  \phantomsection\label{symbol-zebra}{{ ðŸ¦`` } \texttt{\ zebra\ }}
\item
  \phantomsection\label{symbol-zodiac.aquarius}{{ â™' }
  \texttt{\ zodiac.\ aquarius\ }}
\item
  \phantomsection\label{symbol-zodiac.aries}{{ ♈ }
  \texttt{\ zodiac.\ aries\ }}
\item
  \phantomsection\label{symbol-zodiac.cancer}{{ ♋ }
  \texttt{\ zodiac.\ cancer\ }}
\item
  \phantomsection\label{symbol-zodiac.capri}{{ â™` }
  \texttt{\ zodiac.\ capri\ }}
\item
  \phantomsection\label{symbol-zodiac.gemini}{{ ♊ }
  \texttt{\ zodiac.\ gemini\ }}
\item
  \phantomsection\label{symbol-zodiac.leo}{{ ♌ }
  \texttt{\ zodiac.\ leo\ }}
\item
  \phantomsection\label{symbol-zodiac.libra}{{ ♎ }
  \texttt{\ zodiac.\ libra\ }}
\item
  \phantomsection\label{symbol-zodiac.ophi}{{ ⛎ }
  \texttt{\ zodiac.\ ophi\ }}
\item
  \phantomsection\label{symbol-zodiac.pisces}{{ â™`` }
  \texttt{\ zodiac.\ pisces\ }}
\item
  \phantomsection\label{symbol-zodiac.sagit}{{ â™? }
  \texttt{\ zodiac.\ sagit\ }}
\item
  \phantomsection\label{symbol-zodiac.scorpio}{{ â™? }
  \texttt{\ zodiac.\ scorpio\ }}
\item
  \phantomsection\label{symbol-zodiac.taurus}{{ ♉ }
  \texttt{\ zodiac.\ taurus\ }}
\item
  \phantomsection\label{symbol-zodiac.virgo}{{ â™? }
  \texttt{\ zodiac.\ virgo\ }}
\item
  \phantomsection\label{symbol-zombie}{{ 🧟 } \texttt{\ zombie\ }}
\item
  \phantomsection\label{symbol-zzz}{{ ðŸ'¤ } \texttt{\ zzz\ }}
\end{itemize}

{ }

\subsubsection{\texorpdfstring{{ }}{ }}\label{section}

Name: \texttt{\ }
\includesvg[width=0.16667in,height=0.16667in]{/assets/icons/16-copy.svg}

Escape: \texttt{\ \textbackslash{}u\ \{\ }{\texttt{\ }}\texttt{\ \}\ }
\includesvg[width=0.16667in,height=0.16667in]{/assets/icons/16-copy.svg}

Shorthand: \texttt{\ }
\includesvg[width=0.16667in,height=0.16667in]{/assets/icons/16-copy.svg}
{ }

Accent:
\includesvg[width=0.16667in,height=0.16667in]{/assets/icons/16-close.svg}

LaTeX: \texttt{\ }

\paragraph{Variants}\label{variants}

{ }

\href{/docs/reference/symbols/sym/}{\pandocbounded{\includesvg[keepaspectratio]{/assets/icons/16-arrow-right.svg}}}

{ General } { Previous page }

\href{/docs/reference/symbols/symbol/}{\pandocbounded{\includesvg[keepaspectratio]{/assets/icons/16-arrow-right.svg}}}

{ Symbol } { Next page }


\section{Docs LaTeX/typst.app/docs/reference/symbols/symbol.tex}
\title{typst.app/docs/reference/symbols/symbol}

\begin{itemize}
\tightlist
\item
  \href{/docs}{\includesvg[width=0.16667in,height=0.16667in]{/assets/icons/16-docs-dark.svg}}
\item
  \includesvg[width=0.16667in,height=0.16667in]{/assets/icons/16-arrow-right.svg}
\item
  \href{/docs/reference/}{Reference}
\item
  \includesvg[width=0.16667in,height=0.16667in]{/assets/icons/16-arrow-right.svg}
\item
  \href{/docs/reference/symbols/}{Symbols}
\item
  \includesvg[width=0.16667in,height=0.16667in]{/assets/icons/16-arrow-right.svg}
\item
  \href{/docs/reference/symbols/symbol/}{Symbol}
\end{itemize}

\section{\texorpdfstring{{ symbol }}{ symbol }}\label{summary}

A Unicode symbol.

Typst defines common symbols so that they can easily be written with
standard keyboards. The symbols are defined in modules, from which they
can be accessed using \href{/docs/reference/scripting/\#fields}{field
access notation} :

\begin{itemize}
\tightlist
\item
  General symbols are defined in the
  \href{/docs/reference/symbols/sym/}{\texttt{\ sym\ } module}
\item
  Emoji are defined in the
  \href{/docs/reference/symbols/emoji/}{\texttt{\ emoji\ } module}
\end{itemize}

Moreover, you can define custom symbols with this type\textquotesingle s
constructor function.

\begin{verbatim}
#sym.arrow.r \
#sym.gt.eq.not \
$gt.eq.not$ \
#emoji.face.halo
\end{verbatim}

\includegraphics[width=5in,height=\textheight,keepaspectratio]{/assets/docs/VU7JCTNOvXZ0YxOKfFCHhgAAAAAAAAAA.png}

Many symbols have different variants, which can be selected by appending
the modifiers with dot notation. The order of the modifiers is not
relevant. Visit the documentation pages of the symbol modules and click
on a symbol to see its available variants.

\begin{verbatim}
$arrow.l$ \
$arrow.r$ \
$arrow.t.quad$
\end{verbatim}

\includegraphics[width=5in,height=\textheight,keepaspectratio]{/assets/docs/6bpO4zHphIuAdD1km_qbDAAAAAAAAAAA.png}

\subsection{\texorpdfstring{Constructor
{}}{Constructor }}\label{constructor}

\phantomsection\label{constructor-constructor-tooltip}
If a type has a constructor, you can call it like a function to create a
new value of the type.

Create a custom symbol with modifiers.

{ symbol } (

{ \hyperref[constructor-parameters-variants]{..}
\href{/docs/reference/foundations/str/}{str}
\href{/docs/reference/foundations/array/}{array} }

) -\textgreater{} \href{/docs/reference/symbols/symbol/}{symbol}

\begin{verbatim}
#let envelope = symbol(
  "🖂",
  ("stamped", "🖃"),
  ("stamped.pen", "🖆"),
  ("lightning", "🖄"),
  ("fly", "🖅"),
)

#envelope
#envelope.stamped
#envelope.stamped.pen
#envelope.lightning
#envelope.fly
\end{verbatim}

\includegraphics[width=5in,height=\textheight,keepaspectratio]{/assets/docs/KbY7ot9pSdzC8G6YXvE_VAAAAAAAAAAA.png}

\paragraph{\texorpdfstring{\texttt{\ variants\ }}{ variants }}\label{constructor-variants}

\href{/docs/reference/foundations/str/}{str} {or}
\href{/docs/reference/foundations/array/}{array}

{Required} {{ Positional }}

\phantomsection\label{constructor-variants-positional-tooltip}
Positional parameters are specified in order, without names.

{{ Variadic }}

\phantomsection\label{constructor-variants-variadic-tooltip}
Variadic parameters can be specified multiple times.

The variants of the symbol.

Can be a just a string consisting of a single character for the
modifierless variant or an array with two strings specifying the
modifiers and the symbol. Individual modifiers should be separated by
dots. When displaying a symbol, Typst selects the first from the
variants that have all attached modifiers and the minimum number of
other modifiers.

\href{/docs/reference/symbols/emoji/}{\pandocbounded{\includesvg[keepaspectratio]{/assets/icons/16-arrow-right.svg}}}

{ Emoji } { Previous page }

\href{/docs/reference/layout/}{\pandocbounded{\includesvg[keepaspectratio]{/assets/icons/16-arrow-right.svg}}}

{ Layout } { Next page }


\section{Docs LaTeX/typst.app/docs/reference/symbols/sym.tex}
\title{typst.app/docs/reference/symbols/sym}

\begin{itemize}
\tightlist
\item
  \href{/docs}{\includesvg[width=0.16667in,height=0.16667in]{/assets/icons/16-docs-dark.svg}}
\item
  \includesvg[width=0.16667in,height=0.16667in]{/assets/icons/16-arrow-right.svg}
\item
  \href{/docs/reference/}{Reference}
\item
  \includesvg[width=0.16667in,height=0.16667in]{/assets/icons/16-arrow-right.svg}
\item
  \href{/docs/reference/symbols/}{Symbols}
\item
  \includesvg[width=0.16667in,height=0.16667in]{/assets/icons/16-arrow-right.svg}
\item
  \href{/docs/reference/symbols/sym/}{General}
\end{itemize}

\section{sym}\label{sym}

Named general symbols.

For example, \texttt{\ \#sym.arrow\ } produces the â†' symbol. Within
\href{/docs/reference/math/}{formulas} , these symbols can be used
without the \texttt{\ \#sym.\ } prefix.

The \texttt{\ d\ } in an integral\textquotesingle s \texttt{\ dx\ } can
be written as
\texttt{\ }{\texttt{\ \$\ }}\texttt{\ }{\texttt{\ dif\ }}\texttt{\ x\ }{\texttt{\ \$\ }}\texttt{\ }
. Outside math formulas, \texttt{\ dif\ } can be accessed as
\texttt{\ math.dif\ } .

Click on a \href{/docs/reference/symbols/symbol/}{symbol} to copy it to
the clipboard.

\includesvg[width=0.16667in,height=0.16667in]{/assets/icons/16-search-gray.svg}

\begin{itemize}
\tightlist
\item
  \phantomsection\label{symbol-wj}{{ wjoin } \texttt{\ wj\ }}
\item
  \phantomsection\label{symbol-zwj}{{ zwj } \texttt{\ zwj\ }}
\item
  \phantomsection\label{symbol-zwnj}{{ zwnj } \texttt{\ zwnj\ }}
\item
  \phantomsection\label{symbol-zws}{{ zwsp } \texttt{\ zws\ }}
\item
  \phantomsection\label{symbol-lrm}{{ ‎ } \texttt{\ lrm\ }}
\item
  \phantomsection\label{symbol-rlm}{{ � } \texttt{\ rlm\ }}
\item
  \phantomsection\label{symbol-space}{{ â?£ } \texttt{\ space\ }}
\item
  \phantomsection\label{symbol-space.nobreak}{{ nbsp }
  \texttt{\ space.\ nobreak\ }}
\item
  \phantomsection\label{symbol-space.nobreak.narrow}{{   }
  \texttt{\ space.\ nobreak.\ narrow\ }}
\item
  \phantomsection\label{symbol-space.en}{{ ensp }
  \texttt{\ space.\ en\ }}
\item
  \phantomsection\label{symbol-space.quad}{{ emsp }
  \texttt{\ space.\ quad\ }}
\item
  \phantomsection\label{symbol-space.third}{{ â\ldots``emsp }
  \texttt{\ space.\ third\ }}
\item
  \phantomsection\label{symbol-space.quarter}{{ ¼emsp }
  \texttt{\ space.\ quarter\ }}
\item
  \phantomsection\label{symbol-space.sixth}{{ â\ldots™emsp }
  \texttt{\ space.\ sixth\ }}
\item
  \phantomsection\label{symbol-space.med}{{ mmsp }
  \texttt{\ space.\ med\ }}
\item
  \phantomsection\label{symbol-space.fig}{{ numsp }
  \texttt{\ space.\ fig\ }}
\item
  \phantomsection\label{symbol-space.punct}{{ puncsp }
  \texttt{\ space.\ punct\ }}
\item
  \phantomsection\label{symbol-space.thin}{{ thinsp }
  \texttt{\ space.\ thin\ }}
\item
  \phantomsection\label{symbol-space.hair}{{ hairsp }
  \texttt{\ space.\ hair\ }}
\item
  \phantomsection\label{symbol-paren.l}{{ ( } \texttt{\ paren.\ l\ }}
\item
  \phantomsection\label{symbol-paren.l.double}{{ ⦠}
  \texttt{\ paren.\ l.\ double\ }}
\item
  \phantomsection\label{symbol-paren.r}{{ ) } \texttt{\ paren.\ r\ }}
\item
  \phantomsection\label{symbol-paren.r.double}{{ ⦆ }
  \texttt{\ paren.\ r.\ double\ }}
\item
  \phantomsection\label{symbol-paren.t}{{ � } \texttt{\ paren.\ t\ }}
\item
  \phantomsection\label{symbol-paren.b}{{ â?? } \texttt{\ paren.\ b\ }}
\item
  \phantomsection\label{symbol-brace.l}{{ \{ } \texttt{\ brace.\ l\ }}
\item
  \phantomsection\label{symbol-brace.l.double}{{ ⦃ }
  \texttt{\ brace.\ l.\ double\ }}
\item
  \phantomsection\label{symbol-brace.r}{{ \} } \texttt{\ brace.\ r\ }}
\item
  \phantomsection\label{symbol-brace.r.double}{{ ⦄ }
  \texttt{\ brace.\ r.\ double\ }}
\item
  \phantomsection\label{symbol-brace.t}{{ â?ž } \texttt{\ brace.\ t\ }}
\item
  \phantomsection\label{symbol-brace.b}{{ â?Ÿ } \texttt{\ brace.\ b\ }}
\item
  \phantomsection\label{symbol-bracket.l}{{ {[} }
  \texttt{\ bracket.\ l\ }}
\item
  \phantomsection\label{symbol-bracket.l.double}{{ ⟦ }
  \texttt{\ bracket.\ l.\ double\ }}
\item
  \phantomsection\label{symbol-bracket.r}{{ {]} }
  \texttt{\ bracket.\ r\ }}
\item
  \phantomsection\label{symbol-bracket.r.double}{{ ⟧ }
  \texttt{\ bracket.\ r.\ double\ }}
\item
  \phantomsection\label{symbol-bracket.t}{{ ⎴ }
  \texttt{\ bracket.\ t\ }}
\item
  \phantomsection\label{symbol-bracket.b}{{ ⎵ }
  \texttt{\ bracket.\ b\ }}
\item
  \phantomsection\label{symbol-shell.l}{{ â?² } \texttt{\ shell.\ l\ }}
\item
  \phantomsection\label{symbol-shell.l.double}{{ ⟬ }
  \texttt{\ shell.\ l.\ double\ }}
\item
  \phantomsection\label{symbol-shell.r}{{ â?³ } \texttt{\ shell.\ r\ }}
\item
  \phantomsection\label{symbol-shell.r.double}{{ ⟭ }
  \texttt{\ shell.\ r.\ double\ }}
\item
  \phantomsection\label{symbol-shell.t}{{ â? } \texttt{\ shell.\ t\ }}
\item
  \phantomsection\label{symbol-shell.b}{{ â?¡ } \texttt{\ shell.\ b\ }}
\item
  \phantomsection\label{symbol-bar.v}{{ \textbar{} }
  \texttt{\ bar.\ v\ }}
\item
  \phantomsection\label{symbol-bar.v.double}{{ â€-- }
  \texttt{\ bar.\ v.\ double\ }}
\item
  \phantomsection\label{symbol-bar.v.triple}{{ ⦀ }
  \texttt{\ bar.\ v.\ triple\ }}
\item
  \phantomsection\label{symbol-bar.v.broken}{{ ¦ }
  \texttt{\ bar.\ v.\ broken\ }}
\item
  \phantomsection\label{symbol-bar.v.circle}{{ ⦶ }
  \texttt{\ bar.\ v.\ circle\ }}
\item
  \phantomsection\label{symbol-bar.h}{{ ― } \texttt{\ bar.\ h\ }}
\item
  \phantomsection\label{symbol-fence.l}{{ ⧘ } \texttt{\ fence.\ l\ }}
\item
  \phantomsection\label{symbol-fence.l.double}{{ ⧚ }
  \texttt{\ fence.\ l.\ double\ }}
\item
  \phantomsection\label{symbol-fence.r}{{ ⧙ } \texttt{\ fence.\ r\ }}
\item
  \phantomsection\label{symbol-fence.r.double}{{ ⧛ }
  \texttt{\ fence.\ r.\ double\ }}
\item
  \phantomsection\label{symbol-fence.dotted}{{ ⦙ }
  \texttt{\ fence.\ dotted\ }}
\item
  \phantomsection\label{symbol-angle}{{ ∠} \texttt{\ angle\ }}
\item
  \phantomsection\label{symbol-angle.l}{{ ⟨ } \texttt{\ angle.\ l\ }}
\item
  \phantomsection\label{symbol-angle.l.curly}{{ ⧼ }
  \texttt{\ angle.\ l.\ curly\ }}
\item
  \phantomsection\label{symbol-angle.l.dot}{{ â¦` }
  \texttt{\ angle.\ l.\ dot\ }}
\item
  \phantomsection\label{symbol-angle.l.double}{{ 《 }
  \texttt{\ angle.\ l.\ double\ }}
\item
  \phantomsection\label{symbol-angle.r}{{ ⟩ } \texttt{\ angle.\ r\ }}
\item
  \phantomsection\label{symbol-angle.r.curly}{{ ⧽ }
  \texttt{\ angle.\ r.\ curly\ }}
\item
  \phantomsection\label{symbol-angle.r.dot}{{ â¦' }
  \texttt{\ angle.\ r.\ dot\ }}
\item
  \phantomsection\label{symbol-angle.r.double}{{ 》 }
  \texttt{\ angle.\ r.\ double\ }}
\item
  \phantomsection\label{symbol-angle.acute}{{ ⦟ }
  \texttt{\ angle.\ acute\ }}
\item
  \phantomsection\label{symbol-angle.arc}{{ ∡ }
  \texttt{\ angle.\ arc\ }}
\item
  \phantomsection\label{symbol-angle.arc.rev}{{ ⦛ }
  \texttt{\ angle.\ arc.\ rev\ }}
\item
  \phantomsection\label{symbol-angle.oblique}{{ ⦦ }
  \texttt{\ angle.\ oblique\ }}
\item
  \phantomsection\label{symbol-angle.rev}{{ ⦣ }
  \texttt{\ angle.\ rev\ }}
\item
  \phantomsection\label{symbol-angle.right}{{ ∟ }
  \texttt{\ angle.\ right\ }}
\item
  \phantomsection\label{symbol-angle.right.rev}{{ ⯾ }
  \texttt{\ angle.\ right.\ rev\ }}
\item
  \phantomsection\label{symbol-angle.right.arc}{{ ⊾ }
  \texttt{\ angle.\ right.\ arc\ }}
\item
  \phantomsection\label{symbol-angle.right.dot}{{ � }
  \texttt{\ angle.\ right.\ dot\ }}
\item
  \phantomsection\label{symbol-angle.right.sq}{{ ⦜ }
  \texttt{\ angle.\ right.\ sq\ }}
\item
  \phantomsection\label{symbol-angle.s}{{ ⦞ } \texttt{\ angle.\ s\ }}
\item
  \phantomsection\label{symbol-angle.spatial}{{ ⟀ }
  \texttt{\ angle.\ spatial\ }}
\item
  \phantomsection\label{symbol-angle.spheric}{{ ∢ }
  \texttt{\ angle.\ spheric\ }}
\item
  \phantomsection\label{symbol-angle.spheric.rev}{{ ⦠}
  \texttt{\ angle.\ spheric.\ rev\ }}
\item
  \phantomsection\label{symbol-angle.spheric.top}{{ ⦡ }
  \texttt{\ angle.\ spheric.\ top\ }}
\item
  \phantomsection\label{symbol-ceil.l}{{ ⌈ } \texttt{\ ceil.\ l\ }}
\item
  \phantomsection\label{symbol-ceil.r}{{ ⌉ } \texttt{\ ceil.\ r\ }}
\item
  \phantomsection\label{symbol-floor.l}{{ ⌊ } \texttt{\ floor.\ l\ }}
\item
  \phantomsection\label{symbol-floor.r}{{ ⌋ } \texttt{\ floor.\ r\ }}
\item
  \phantomsection\label{symbol-amp}{{ \& } \texttt{\ amp\ }}
\item
  \phantomsection\label{symbol-amp.inv}{{ â\ldots‹ }
  \texttt{\ amp.\ inv\ }}
\item
  \phantomsection\label{symbol-ast.op}{{ âˆ--- } \texttt{\ ast.\ op\ }}
\item
  \phantomsection\label{symbol-ast.basic}{{ * }
  \texttt{\ ast.\ basic\ }}
\item
  \phantomsection\label{symbol-ast.low}{{ â?Ž } \texttt{\ ast.\ low\ }}
\item
  \phantomsection\label{symbol-ast.double}{{ â?{}` }
  \texttt{\ ast.\ double\ }}
\item
  \phantomsection\label{symbol-ast.triple}{{ â?‚ }
  \texttt{\ ast.\ triple\ }}
\item
  \phantomsection\label{symbol-ast.small}{{ ﹡ }
  \texttt{\ ast.\ small\ }}
\item
  \phantomsection\label{symbol-ast.circle}{{ ⊛ }
  \texttt{\ ast.\ circle\ }}
\item
  \phantomsection\label{symbol-ast.square}{{ ⧆ }
  \texttt{\ ast.\ square\ }}
\item
  \phantomsection\label{symbol-at}{{ @ } \texttt{\ at\ }}
\item
  \phantomsection\label{symbol-backslash}{{ \textbackslash{} }
  \texttt{\ backslash\ }}
\item
  \phantomsection\label{symbol-backslash.circle}{{ ⦸ }
  \texttt{\ backslash.\ circle\ }}
\item
  \phantomsection\label{symbol-backslash.not}{{ ⧷ }
  \texttt{\ backslash.\ not\ }}
\item
  \phantomsection\label{symbol-co}{{ â„ } \texttt{\ co\ }}
\item
  \phantomsection\label{symbol-colon}{{ : } \texttt{\ colon\ }}
\item
  \phantomsection\label{symbol-colon.double}{{ ∷ }
  \texttt{\ colon.\ double\ }}
\item
  \phantomsection\label{symbol-colon.eq}{{ â‰'' }
  \texttt{\ colon.\ eq\ }}
\item
  \phantomsection\label{symbol-colon.double.eq}{{ â©´ }
  \texttt{\ colon.\ double.\ eq\ }}
\item
  \phantomsection\label{symbol-comma}{{ , } \texttt{\ comma\ }}
\item
  \phantomsection\label{symbol-dagger}{{ †} \texttt{\ dagger\ }}
\item
  \phantomsection\label{symbol-dagger.double}{{ ‡ }
  \texttt{\ dagger.\ double\ }}
\item
  \phantomsection\label{symbol-dash.en}{{ â€`` } \texttt{\ dash.\ en\ }}
\item
  \phantomsection\label{symbol-dash.em}{{ â€'' } \texttt{\ dash.\ em\ }}
\item
  \phantomsection\label{symbol-dash.em.two}{{ ⸺ }
  \texttt{\ dash.\ em.\ two\ }}
\item
  \phantomsection\label{symbol-dash.em.three}{{ ⸻ }
  \texttt{\ dash.\ em.\ three\ }}
\item
  \phantomsection\label{symbol-dash.fig}{{ â€' }
  \texttt{\ dash.\ fig\ }}
\item
  \phantomsection\label{symbol-dash.wave}{{ 〜 }
  \texttt{\ dash.\ wave\ }}
\item
  \phantomsection\label{symbol-dash.colon}{{ ∹ }
  \texttt{\ dash.\ colon\ }}
\item
  \phantomsection\label{symbol-dash.circle}{{ � }
  \texttt{\ dash.\ circle\ }}
\item
  \phantomsection\label{symbol-dash.wave.double}{{ 〰 }
  \texttt{\ dash.\ wave.\ double\ }}
\item
  \phantomsection\label{symbol-dot.op}{{ â‹ } \texttt{\ dot.\ op\ }}
\item
  \phantomsection\label{symbol-dot.basic}{{ . }
  \texttt{\ dot.\ basic\ }}
\item
  \phantomsection\label{symbol-dot.c}{{ · } \texttt{\ dot.\ c\ }}
\item
  \phantomsection\label{symbol-dot.circle}{{ ⊙ }
  \texttt{\ dot.\ circle\ }}
\item
  \phantomsection\label{symbol-dot.circle.big}{{ ⨀ }
  \texttt{\ dot.\ circle.\ big\ }}
\item
  \phantomsection\label{symbol-dot.square}{{ ⊡ }
  \texttt{\ dot.\ square\ }}
\item
  \phantomsection\label{symbol-dot.double}{{ ¨ }
  \texttt{\ dot.\ double\ }}
\item
  \phantomsection\label{symbol-dot.triple}{{ ⃛ }
  \texttt{\ dot.\ triple\ }}
\item
  \phantomsection\label{symbol-dot.quad}{{ ⃜ }
  \texttt{\ dot.\ quad\ }}
\item
  \phantomsection\label{symbol-excl}{{ ! } \texttt{\ excl\ }}
\item
  \phantomsection\label{symbol-excl.double}{{ ‼ }
  \texttt{\ excl.\ double\ }}
\item
  \phantomsection\label{symbol-excl.inv}{{ ¡ } \texttt{\ excl.\ inv\ }}
\item
  \phantomsection\label{symbol-excl.quest}{{ â?‰ }
  \texttt{\ excl.\ quest\ }}
\item
  \phantomsection\label{symbol-quest}{{ ? } \texttt{\ quest\ }}
\item
  \phantomsection\label{symbol-quest.double}{{ â?‡ }
  \texttt{\ quest.\ double\ }}
\item
  \phantomsection\label{symbol-quest.excl}{{ â?ˆ }
  \texttt{\ quest.\ excl\ }}
\item
  \phantomsection\label{symbol-quest.inv}{{ ¿ }
  \texttt{\ quest.\ inv\ }}
\item
  \phantomsection\label{symbol-interrobang}{{ ‽ }
  \texttt{\ interrobang\ }}
\item
  \phantomsection\label{symbol-hash}{{ \# } \texttt{\ hash\ }}
\item
  \phantomsection\label{symbol-hyph}{{ � } \texttt{\ hyph\ }}
\item
  \phantomsection\label{symbol-hyph.minus}{{ - }
  \texttt{\ hyph.\ minus\ }}
\item
  \phantomsection\label{symbol-hyph.nobreak}{{ â€` }
  \texttt{\ hyph.\ nobreak\ }}
\item
  \phantomsection\label{symbol-hyph.point}{{ ‧ }
  \texttt{\ hyph.\ point\ }}
\item
  \phantomsection\label{symbol-hyph.soft}{{ shy }
  \texttt{\ hyph.\ soft\ }}
\item
  \phantomsection\label{symbol-percent}{{ \% } \texttt{\ percent\ }}
\item
  \phantomsection\label{symbol-permille}{{ ‰ } \texttt{\ permille\ }}
\item
  \phantomsection\label{symbol-pilcrow}{{ ¶ } \texttt{\ pilcrow\ }}
\item
  \phantomsection\label{symbol-pilcrow.rev}{{ â?‹ }
  \texttt{\ pilcrow.\ rev\ }}
\item
  \phantomsection\label{symbol-section}{{ § } \texttt{\ section\ }}
\item
  \phantomsection\label{symbol-semi}{{ ; } \texttt{\ semi\ }}
\item
  \phantomsection\label{symbol-semi.rev}{{ â?? }
  \texttt{\ semi.\ rev\ }}
\item
  \phantomsection\label{symbol-slash}{{ / } \texttt{\ slash\ }}
\item
  \phantomsection\label{symbol-slash.double}{{ ⫽ }
  \texttt{\ slash.\ double\ }}
\item
  \phantomsection\label{symbol-slash.triple}{{ â«» }
  \texttt{\ slash.\ triple\ }}
\item
  \phantomsection\label{symbol-slash.big}{{ ⧸ }
  \texttt{\ slash.\ big\ }}
\item
  \phantomsection\label{symbol-dots.h.c}{{ ⋯ }
  \texttt{\ dots.\ h.\ c\ }}
\item
  \phantomsection\label{symbol-dots.h}{{ … } \texttt{\ dots.\ h\ }}
\item
  \phantomsection\label{symbol-dots.v}{{ â‹® } \texttt{\ dots.\ v\ }}
\item
  \phantomsection\label{symbol-dots.down}{{ ⋱ }
  \texttt{\ dots.\ down\ }}
\item
  \phantomsection\label{symbol-dots.up}{{ â‹° } \texttt{\ dots.\ up\ }}
\item
  \phantomsection\label{symbol-tilde.op}{{ ∼ }
  \texttt{\ tilde.\ op\ }}
\item
  \phantomsection\label{symbol-tilde.basic}{{ \textasciitilde{} }
  \texttt{\ tilde.\ basic\ }}
\item
  \phantomsection\label{symbol-tilde.dot}{{ ⩪ }
  \texttt{\ tilde.\ dot\ }}
\item
  \phantomsection\label{symbol-tilde.eq}{{ ≃ }
  \texttt{\ tilde.\ eq\ }}
\item
  \phantomsection\label{symbol-tilde.eq.not}{{ ≄ }
  \texttt{\ tilde.\ eq.\ not\ }}
\item
  \phantomsection\label{symbol-tilde.eq.rev}{{ â‹? }
  \texttt{\ tilde.\ eq.\ rev\ }}
\item
  \phantomsection\label{symbol-tilde.equiv}{{ ≠}
  \texttt{\ tilde.\ equiv\ }}
\item
  \phantomsection\label{symbol-tilde.equiv.not}{{ ≇ }
  \texttt{\ tilde.\ equiv.\ not\ }}
\item
  \phantomsection\label{symbol-tilde.nequiv}{{ ≆ }
  \texttt{\ tilde.\ nequiv\ }}
\item
  \phantomsection\label{symbol-tilde.not}{{ � }
  \texttt{\ tilde.\ not\ }}
\item
  \phantomsection\label{symbol-tilde.rev}{{ ∽ }
  \texttt{\ tilde.\ rev\ }}
\item
  \phantomsection\label{symbol-tilde.rev.equiv}{{ ≌ }
  \texttt{\ tilde.\ rev.\ equiv\ }}
\item
  \phantomsection\label{symbol-tilde.triple}{{ ≋ }
  \texttt{\ tilde.\ triple\ }}
\item
  \phantomsection\label{symbol-acute}{{ ´ } \texttt{\ acute\ }}
\item
  \phantomsection\label{symbol-acute.double}{{ Ë? }
  \texttt{\ acute.\ double\ }}
\item
  \phantomsection\label{symbol-breve}{{ ˘ } \texttt{\ breve\ }}
\item
  \phantomsection\label{symbol-caret}{{ ‸ } \texttt{\ caret\ }}
\item
  \phantomsection\label{symbol-caron}{{ ˇ } \texttt{\ caron\ }}
\item
  \phantomsection\label{symbol-hat}{{ \^{} } \texttt{\ hat\ }}
\item
  \phantomsection\label{symbol-diaer}{{ ¨ } \texttt{\ diaer\ }}
\item
  \phantomsection\label{symbol-grave}{{ ` } \texttt{\ grave\ }}
\item
  \phantomsection\label{symbol-macron}{{ ¯ } \texttt{\ macron\ }}
\item
  \phantomsection\label{symbol-quote.double}{{ " }
  \texttt{\ quote.\ double\ }}
\item
  \phantomsection\label{symbol-quote.single}{{ \textquotesingle{} }
  \texttt{\ quote.\ single\ }}
\item
  \phantomsection\label{symbol-quote.l.double}{{ “ }
  \texttt{\ quote.\ l.\ double\ }}
\item
  \phantomsection\label{symbol-quote.l.single}{{ ‘ }
  \texttt{\ quote.\ l.\ single\ }}
\item
  \phantomsection\label{symbol-quote.r.double}{{ � }
  \texttt{\ quote.\ r.\ double\ }}
\item
  \phantomsection\label{symbol-quote.r.single}{{ ’ }
  \texttt{\ quote.\ r.\ single\ }}
\item
  \phantomsection\label{symbol-quote.angle.l.double}{{ « }
  \texttt{\ quote.\ angle.\ l.\ double\ }}
\item
  \phantomsection\label{symbol-quote.angle.l.single}{{ ‹ }
  \texttt{\ quote.\ angle.\ l.\ single\ }}
\item
  \phantomsection\label{symbol-quote.angle.r.double}{{ » }
  \texttt{\ quote.\ angle.\ r.\ double\ }}
\item
  \phantomsection\label{symbol-quote.angle.r.single}{{ › }
  \texttt{\ quote.\ angle.\ r.\ single\ }}
\item
  \phantomsection\label{symbol-quote.high.double}{{ ‟ }
  \texttt{\ quote.\ high.\ double\ }}
\item
  \phantomsection\label{symbol-quote.high.single}{{ ‛ }
  \texttt{\ quote.\ high.\ single\ }}
\item
  \phantomsection\label{symbol-quote.low.double}{{ „ }
  \texttt{\ quote.\ low.\ double\ }}
\item
  \phantomsection\label{symbol-quote.low.single}{{ ‚ }
  \texttt{\ quote.\ low.\ single\ }}
\item
  \phantomsection\label{symbol-prime}{{ ′ } \texttt{\ prime\ }}
\item
  \phantomsection\label{symbol-prime.rev}{{ ‵ }
  \texttt{\ prime.\ rev\ }}
\item
  \phantomsection\label{symbol-prime.double}{{ ″ }
  \texttt{\ prime.\ double\ }}
\item
  \phantomsection\label{symbol-prime.double.rev}{{ ‶ }
  \texttt{\ prime.\ double.\ rev\ }}
\item
  \phantomsection\label{symbol-prime.triple}{{ ‴ }
  \texttt{\ prime.\ triple\ }}
\item
  \phantomsection\label{symbol-prime.triple.rev}{{ ‷ }
  \texttt{\ prime.\ triple.\ rev\ }}
\item
  \phantomsection\label{symbol-prime.quad}{{ â?--- }
  \texttt{\ prime.\ quad\ }}
\item
  \phantomsection\label{symbol-plus}{{ + } \texttt{\ plus\ }}
\item
  \phantomsection\label{symbol-plus.circle}{{ ⊕ }
  \texttt{\ plus.\ circle\ }}
\item
  \phantomsection\label{symbol-plus.circle.arrow}{{ ⟴ }
  \texttt{\ plus.\ circle.\ arrow\ }}
\item
  \phantomsection\label{symbol-plus.circle.big}{{ � }
  \texttt{\ plus.\ circle.\ big\ }}
\item
  \phantomsection\label{symbol-plus.dot}{{ âˆ'' }
  \texttt{\ plus.\ dot\ }}
\item
  \phantomsection\label{symbol-plus.double}{{ ⧺ }
  \texttt{\ plus.\ double\ }}
\item
  \phantomsection\label{symbol-plus.minus}{{ ± }
  \texttt{\ plus.\ minus\ }}
\item
  \phantomsection\label{symbol-plus.small}{{ ï¹¢ }
  \texttt{\ plus.\ small\ }}
\item
  \phantomsection\label{symbol-plus.square}{{ ⊞ }
  \texttt{\ plus.\ square\ }}
\item
  \phantomsection\label{symbol-plus.triangle}{{ ⨹ }
  \texttt{\ plus.\ triangle\ }}
\item
  \phantomsection\label{symbol-plus.triple}{{ ⧻ }
  \texttt{\ plus.\ triple\ }}
\item
  \phantomsection\label{symbol-minus}{{ âˆ' } \texttt{\ minus\ }}
\item
  \phantomsection\label{symbol-minus.circle}{{ âŠ-- }
  \texttt{\ minus.\ circle\ }}
\item
  \phantomsection\label{symbol-minus.dot}{{ ∸ }
  \texttt{\ minus.\ dot\ }}
\item
  \phantomsection\label{symbol-minus.plus}{{ âˆ`` }
  \texttt{\ minus.\ plus\ }}
\item
  \phantomsection\label{symbol-minus.square}{{ ⊟ }
  \texttt{\ minus.\ square\ }}
\item
  \phantomsection\label{symbol-minus.tilde}{{ ≂ }
  \texttt{\ minus.\ tilde\ }}
\item
  \phantomsection\label{symbol-minus.triangle}{{ ⨺ }
  \texttt{\ minus.\ triangle\ }}
\item
  \phantomsection\label{symbol-div}{{ ÷ } \texttt{\ div\ }}
\item
  \phantomsection\label{symbol-div.circle}{{ ⨸ }
  \texttt{\ div.\ circle\ }}
\item
  \phantomsection\label{symbol-times}{{ Ã--- } \texttt{\ times\ }}
\item
  \phantomsection\label{symbol-times.big}{{ ⨉ }
  \texttt{\ times.\ big\ }}
\item
  \phantomsection\label{symbol-times.circle}{{ âŠ--- }
  \texttt{\ times.\ circle\ }}
\item
  \phantomsection\label{symbol-times.circle.big}{{ ⨂ }
  \texttt{\ times.\ circle.\ big\ }}
\item
  \phantomsection\label{symbol-times.div}{{ ⋇ }
  \texttt{\ times.\ div\ }}
\item
  \phantomsection\label{symbol-times.three.l}{{ â‹‹ }
  \texttt{\ times.\ three.\ l\ }}
\item
  \phantomsection\label{symbol-times.three.r}{{ ⋌ }
  \texttt{\ times.\ three.\ r\ }}
\item
  \phantomsection\label{symbol-times.l}{{ ⋉ } \texttt{\ times.\ l\ }}
\item
  \phantomsection\label{symbol-times.r}{{ â‹Š } \texttt{\ times.\ r\ }}
\item
  \phantomsection\label{symbol-times.square}{{ ⊠}
  \texttt{\ times.\ square\ }}
\item
  \phantomsection\label{symbol-times.triangle}{{ ⨻ }
  \texttt{\ times.\ triangle\ }}
\item
  \phantomsection\label{symbol-ratio}{{ ∶ } \texttt{\ ratio\ }}
\item
  \phantomsection\label{symbol-eq}{{ = } \texttt{\ eq\ }}
\item
  \phantomsection\label{symbol-eq.star}{{ ≛ } \texttt{\ eq.\ star\ }}
\item
  \phantomsection\label{symbol-eq.circle}{{ ⊜ }
  \texttt{\ eq.\ circle\ }}
\item
  \phantomsection\label{symbol-eq.colon}{{ ≕ }
  \texttt{\ eq.\ colon\ }}
\item
  \phantomsection\label{symbol-eq.def}{{ � } \texttt{\ eq.\ def\ }}
\item
  \phantomsection\label{symbol-eq.delta}{{ ≜ }
  \texttt{\ eq.\ delta\ }}
\item
  \phantomsection\label{symbol-eq.equi}{{ ≚ } \texttt{\ eq.\ equi\ }}
\item
  \phantomsection\label{symbol-eq.est}{{ ≙ } \texttt{\ eq.\ est\ }}
\item
  \phantomsection\label{symbol-eq.gt}{{ â‹? } \texttt{\ eq.\ gt\ }}
\item
  \phantomsection\label{symbol-eq.lt}{{ ⋜ } \texttt{\ eq.\ lt\ }}
\item
  \phantomsection\label{symbol-eq.m}{{ ≞ } \texttt{\ eq.\ m\ }}
\item
  \phantomsection\label{symbol-eq.not}{{ ≠} \texttt{\ eq.\ not\ }}
\item
  \phantomsection\label{symbol-eq.prec}{{ â‹ž } \texttt{\ eq.\ prec\ }}
\item
  \phantomsection\label{symbol-eq.quest}{{ ≟ }
  \texttt{\ eq.\ quest\ }}
\item
  \phantomsection\label{symbol-eq.small}{{ ﹦ }
  \texttt{\ eq.\ small\ }}
\item
  \phantomsection\label{symbol-eq.succ}{{ â‹Ÿ } \texttt{\ eq.\ succ\ }}
\item
  \phantomsection\label{symbol-eq.triple}{{ ≡ }
  \texttt{\ eq.\ triple\ }}
\item
  \phantomsection\label{symbol-eq.quad}{{ ≣ } \texttt{\ eq.\ quad\ }}
\item
  \phantomsection\label{symbol-gt}{{ \textgreater{} } \texttt{\ gt\ }}
\item
  \phantomsection\label{symbol-gt.circle}{{ � }
  \texttt{\ gt.\ circle\ }}
\item
  \phantomsection\label{symbol-gt.dot}{{ â‹--- } \texttt{\ gt.\ dot\ }}
\item
  \phantomsection\label{symbol-gt.approx}{{ ⪆ }
  \texttt{\ gt.\ approx\ }}
\item
  \phantomsection\label{symbol-gt.double}{{ ≫ }
  \texttt{\ gt.\ double\ }}
\item
  \phantomsection\label{symbol-gt.eq}{{ ≥ } \texttt{\ gt.\ eq\ }}
\item
  \phantomsection\label{symbol-gt.eq.slant}{{ ⩾ }
  \texttt{\ gt.\ eq.\ slant\ }}
\item
  \phantomsection\label{symbol-gt.eq.lt}{{ â‹› }
  \texttt{\ gt.\ eq.\ lt\ }}
\item
  \phantomsection\label{symbol-gt.eq.not}{{ ≱ }
  \texttt{\ gt.\ eq.\ not\ }}
\item
  \phantomsection\label{symbol-gt.equiv}{{ ≧ }
  \texttt{\ gt.\ equiv\ }}
\item
  \phantomsection\label{symbol-gt.lt}{{ ≷ } \texttt{\ gt.\ lt\ }}
\item
  \phantomsection\label{symbol-gt.lt.not}{{ ≹ }
  \texttt{\ gt.\ lt.\ not\ }}
\item
  \phantomsection\label{symbol-gt.neq}{{ ⪈ } \texttt{\ gt.\ neq\ }}
\item
  \phantomsection\label{symbol-gt.napprox}{{ ⪊ }
  \texttt{\ gt.\ napprox\ }}
\item
  \phantomsection\label{symbol-gt.nequiv}{{ ≩ }
  \texttt{\ gt.\ nequiv\ }}
\item
  \phantomsection\label{symbol-gt.not}{{ ≯ } \texttt{\ gt.\ not\ }}
\item
  \phantomsection\label{symbol-gt.ntilde}{{ ⋧ }
  \texttt{\ gt.\ ntilde\ }}
\item
  \phantomsection\label{symbol-gt.small}{{ ï¹¥ }
  \texttt{\ gt.\ small\ }}
\item
  \phantomsection\label{symbol-gt.tilde}{{ ≳ }
  \texttt{\ gt.\ tilde\ }}
\item
  \phantomsection\label{symbol-gt.tilde.not}{{ ≵ }
  \texttt{\ gt.\ tilde.\ not\ }}
\item
  \phantomsection\label{symbol-gt.tri}{{ ⊳ } \texttt{\ gt.\ tri\ }}
\item
  \phantomsection\label{symbol-gt.tri.eq}{{ ⊵ }
  \texttt{\ gt.\ tri.\ eq\ }}
\item
  \phantomsection\label{symbol-gt.tri.eq.not}{{ â‹­ }
  \texttt{\ gt.\ tri.\ eq.\ not\ }}
\item
  \phantomsection\label{symbol-gt.tri.not}{{ â‹« }
  \texttt{\ gt.\ tri.\ not\ }}
\item
  \phantomsection\label{symbol-gt.triple}{{ â‹™ }
  \texttt{\ gt.\ triple\ }}
\item
  \phantomsection\label{symbol-gt.triple.nested}{{ ⫸ }
  \texttt{\ gt.\ triple.\ nested\ }}
\item
  \phantomsection\label{symbol-lt}{{ \textless{} } \texttt{\ lt\ }}
\item
  \phantomsection\label{symbol-lt.circle}{{ ⧀ }
  \texttt{\ lt.\ circle\ }}
\item
  \phantomsection\label{symbol-lt.dot}{{ â‹-- } \texttt{\ lt.\ dot\ }}
\item
  \phantomsection\label{symbol-lt.approx}{{ ⪠}
  \texttt{\ lt.\ approx\ }}
\item
  \phantomsection\label{symbol-lt.double}{{ ≪ }
  \texttt{\ lt.\ double\ }}
\item
  \phantomsection\label{symbol-lt.eq}{{ ≤ } \texttt{\ lt.\ eq\ }}
\item
  \phantomsection\label{symbol-lt.eq.slant}{{ ⩽ }
  \texttt{\ lt.\ eq.\ slant\ }}
\item
  \phantomsection\label{symbol-lt.eq.gt}{{ â‹š }
  \texttt{\ lt.\ eq.\ gt\ }}
\item
  \phantomsection\label{symbol-lt.eq.not}{{ ≰ }
  \texttt{\ lt.\ eq.\ not\ }}
\item
  \phantomsection\label{symbol-lt.equiv}{{ ≦ }
  \texttt{\ lt.\ equiv\ }}
\item
  \phantomsection\label{symbol-lt.gt}{{ ≶ } \texttt{\ lt.\ gt\ }}
\item
  \phantomsection\label{symbol-lt.gt.not}{{ ≸ }
  \texttt{\ lt.\ gt.\ not\ }}
\item
  \phantomsection\label{symbol-lt.neq}{{ ⪇ } \texttt{\ lt.\ neq\ }}
\item
  \phantomsection\label{symbol-lt.napprox}{{ ⪉ }
  \texttt{\ lt.\ napprox\ }}
\item
  \phantomsection\label{symbol-lt.nequiv}{{ ≨ }
  \texttt{\ lt.\ nequiv\ }}
\item
  \phantomsection\label{symbol-lt.not}{{ ≮ } \texttt{\ lt.\ not\ }}
\item
  \phantomsection\label{symbol-lt.ntilde}{{ ⋦ }
  \texttt{\ lt.\ ntilde\ }}
\item
  \phantomsection\label{symbol-lt.small}{{ ﹤ }
  \texttt{\ lt.\ small\ }}
\item
  \phantomsection\label{symbol-lt.tilde}{{ ≲ }
  \texttt{\ lt.\ tilde\ }}
\item
  \phantomsection\label{symbol-lt.tilde.not}{{ ≴ }
  \texttt{\ lt.\ tilde.\ not\ }}
\item
  \phantomsection\label{symbol-lt.tri}{{ ⊲ } \texttt{\ lt.\ tri\ }}
\item
  \phantomsection\label{symbol-lt.tri.eq}{{ ⊴ }
  \texttt{\ lt.\ tri.\ eq\ }}
\item
  \phantomsection\label{symbol-lt.tri.eq.not}{{ ⋬ }
  \texttt{\ lt.\ tri.\ eq.\ not\ }}
\item
  \phantomsection\label{symbol-lt.tri.not}{{ ⋪ }
  \texttt{\ lt.\ tri.\ not\ }}
\item
  \phantomsection\label{symbol-lt.triple}{{ ⋘ }
  \texttt{\ lt.\ triple\ }}
\item
  \phantomsection\label{symbol-lt.triple.nested}{{ â«· }
  \texttt{\ lt.\ triple.\ nested\ }}
\item
  \phantomsection\label{symbol-approx}{{ ≈ } \texttt{\ approx\ }}
\item
  \phantomsection\label{symbol-approx.eq}{{ ≊ }
  \texttt{\ approx.\ eq\ }}
\item
  \phantomsection\label{symbol-approx.not}{{ ≉ }
  \texttt{\ approx.\ not\ }}
\item
  \phantomsection\label{symbol-prec}{{ ≺ } \texttt{\ prec\ }}
\item
  \phantomsection\label{symbol-prec.approx}{{ ⪷ }
  \texttt{\ prec.\ approx\ }}
\item
  \phantomsection\label{symbol-prec.curly.eq}{{ ≼ }
  \texttt{\ prec.\ curly.\ eq\ }}
\item
  \phantomsection\label{symbol-prec.curly.eq.not}{{ â‹ }
  \texttt{\ prec.\ curly.\ eq.\ not\ }}
\item
  \phantomsection\label{symbol-prec.double}{{ ⪻ }
  \texttt{\ prec.\ double\ }}
\item
  \phantomsection\label{symbol-prec.eq}{{ ⪯ } \texttt{\ prec.\ eq\ }}
\item
  \phantomsection\label{symbol-prec.equiv}{{ ⪳ }
  \texttt{\ prec.\ equiv\ }}
\item
  \phantomsection\label{symbol-prec.napprox}{{ ⪹ }
  \texttt{\ prec.\ napprox\ }}
\item
  \phantomsection\label{symbol-prec.neq}{{ ⪱ }
  \texttt{\ prec.\ neq\ }}
\item
  \phantomsection\label{symbol-prec.nequiv}{{ ⪵ }
  \texttt{\ prec.\ nequiv\ }}
\item
  \phantomsection\label{symbol-prec.not}{{ ⊀ }
  \texttt{\ prec.\ not\ }}
\item
  \phantomsection\label{symbol-prec.ntilde}{{ ⋨ }
  \texttt{\ prec.\ ntilde\ }}
\item
  \phantomsection\label{symbol-prec.tilde}{{ ≾ }
  \texttt{\ prec.\ tilde\ }}
\item
  \phantomsection\label{symbol-succ}{{ ≻ } \texttt{\ succ\ }}
\item
  \phantomsection\label{symbol-succ.approx}{{ ⪸ }
  \texttt{\ succ.\ approx\ }}
\item
  \phantomsection\label{symbol-succ.curly.eq}{{ ≽ }
  \texttt{\ succ.\ curly.\ eq\ }}
\item
  \phantomsection\label{symbol-succ.curly.eq.not}{{ â‹¡ }
  \texttt{\ succ.\ curly.\ eq.\ not\ }}
\item
  \phantomsection\label{symbol-succ.double}{{ ⪼ }
  \texttt{\ succ.\ double\ }}
\item
  \phantomsection\label{symbol-succ.eq}{{ ⪰ } \texttt{\ succ.\ eq\ }}
\item
  \phantomsection\label{symbol-succ.equiv}{{ ⪴ }
  \texttt{\ succ.\ equiv\ }}
\item
  \phantomsection\label{symbol-succ.napprox}{{ ⪺ }
  \texttt{\ succ.\ napprox\ }}
\item
  \phantomsection\label{symbol-succ.neq}{{ ⪲ }
  \texttt{\ succ.\ neq\ }}
\item
  \phantomsection\label{symbol-succ.nequiv}{{ ⪶ }
  \texttt{\ succ.\ nequiv\ }}
\item
  \phantomsection\label{symbol-succ.not}{{ � }
  \texttt{\ succ.\ not\ }}
\item
  \phantomsection\label{symbol-succ.ntilde}{{ â‹© }
  \texttt{\ succ.\ ntilde\ }}
\item
  \phantomsection\label{symbol-succ.tilde}{{ ≿ }
  \texttt{\ succ.\ tilde\ }}
\item
  \phantomsection\label{symbol-equiv}{{ ≡ } \texttt{\ equiv\ }}
\item
  \phantomsection\label{symbol-equiv.not}{{ ≢ }
  \texttt{\ equiv.\ not\ }}
\item
  \phantomsection\label{symbol-prop}{{ � } \texttt{\ prop\ }}
\item
  \phantomsection\label{symbol-original}{{ ⊶ } \texttt{\ original\ }}
\item
  \phantomsection\label{symbol-image}{{ ⊷ } \texttt{\ image\ }}
\item
  \phantomsection\label{symbol-emptyset}{{ ∠} \texttt{\ emptyset\ }}
\item
  \phantomsection\label{symbol-emptyset.arrow.r}{{ ⦳ }
  \texttt{\ emptyset.\ arrow.\ r\ }}
\item
  \phantomsection\label{symbol-emptyset.arrow.l}{{ ⦴ }
  \texttt{\ emptyset.\ arrow.\ l\ }}
\item
  \phantomsection\label{symbol-emptyset.bar}{{ ⦱ }
  \texttt{\ emptyset.\ bar\ }}
\item
  \phantomsection\label{symbol-emptyset.circle}{{ ⦲ }
  \texttt{\ emptyset.\ circle\ }}
\item
  \phantomsection\label{symbol-emptyset.rev}{{ ⦰ }
  \texttt{\ emptyset.\ rev\ }}
\item
  \phantomsection\label{symbol-nothing}{{ ∠} \texttt{\ nothing\ }}
\item
  \phantomsection\label{symbol-nothing.arrow.r}{{ ⦳ }
  \texttt{\ nothing.\ arrow.\ r\ }}
\item
  \phantomsection\label{symbol-nothing.arrow.l}{{ ⦴ }
  \texttt{\ nothing.\ arrow.\ l\ }}
\item
  \phantomsection\label{symbol-nothing.bar}{{ ⦱ }
  \texttt{\ nothing.\ bar\ }}
\item
  \phantomsection\label{symbol-nothing.circle}{{ ⦲ }
  \texttt{\ nothing.\ circle\ }}
\item
  \phantomsection\label{symbol-nothing.rev}{{ ⦰ }
  \texttt{\ nothing.\ rev\ }}
\item
  \phantomsection\label{symbol-without}{{ âˆ-- } \texttt{\ without\ }}
\item
  \phantomsection\label{symbol-complement}{{ � }
  \texttt{\ complement\ }}
\item
  \phantomsection\label{symbol-in}{{ ∈ } \texttt{\ in\ }}
\item
  \phantomsection\label{symbol-in.not}{{ ∉ } \texttt{\ in.\ not\ }}
\item
  \phantomsection\label{symbol-in.rev}{{ ∋ } \texttt{\ in.\ rev\ }}
\item
  \phantomsection\label{symbol-in.rev.not}{{ ∌ }
  \texttt{\ in.\ rev.\ not\ }}
\item
  \phantomsection\label{symbol-in.rev.small}{{ � }
  \texttt{\ in.\ rev.\ small\ }}
\item
  \phantomsection\label{symbol-in.small}{{ ∊ }
  \texttt{\ in.\ small\ }}
\item
  \phantomsection\label{symbol-subset}{{ ⊂ } \texttt{\ subset\ }}
\item
  \phantomsection\label{symbol-subset.dot}{{ ⪽ }
  \texttt{\ subset.\ dot\ }}
\item
  \phantomsection\label{symbol-subset.double}{{ â‹? }
  \texttt{\ subset.\ double\ }}
\item
  \phantomsection\label{symbol-subset.eq}{{ ⊆ }
  \texttt{\ subset.\ eq\ }}
\item
  \phantomsection\label{symbol-subset.eq.not}{{ ⊈ }
  \texttt{\ subset.\ eq.\ not\ }}
\item
  \phantomsection\label{symbol-subset.eq.sq}{{ âŠ` }
  \texttt{\ subset.\ eq.\ sq\ }}
\item
  \phantomsection\label{symbol-subset.eq.sq.not}{{ â‹¢ }
  \texttt{\ subset.\ eq.\ sq.\ not\ }}
\item
  \phantomsection\label{symbol-subset.neq}{{ ⊊ }
  \texttt{\ subset.\ neq\ }}
\item
  \phantomsection\label{symbol-subset.not}{{ ⊄ }
  \texttt{\ subset.\ not\ }}
\item
  \phantomsection\label{symbol-subset.sq}{{ � }
  \texttt{\ subset.\ sq\ }}
\item
  \phantomsection\label{symbol-subset.sq.neq}{{ ⋤ }
  \texttt{\ subset.\ sq.\ neq\ }}
\item
  \phantomsection\label{symbol-supset}{{ ⊃ } \texttt{\ supset\ }}
\item
  \phantomsection\label{symbol-supset.dot}{{ ⪾ }
  \texttt{\ supset.\ dot\ }}
\item
  \phantomsection\label{symbol-supset.double}{{ â‹` }
  \texttt{\ supset.\ double\ }}
\item
  \phantomsection\label{symbol-supset.eq}{{ ⊇ }
  \texttt{\ supset.\ eq\ }}
\item
  \phantomsection\label{symbol-supset.eq.not}{{ ⊉ }
  \texttt{\ supset.\ eq.\ not\ }}
\item
  \phantomsection\label{symbol-supset.eq.sq}{{ âŠ' }
  \texttt{\ supset.\ eq.\ sq\ }}
\item
  \phantomsection\label{symbol-supset.eq.sq.not}{{ â‹£ }
  \texttt{\ supset.\ eq.\ sq.\ not\ }}
\item
  \phantomsection\label{symbol-supset.neq}{{ ⊋ }
  \texttt{\ supset.\ neq\ }}
\item
  \phantomsection\label{symbol-supset.not}{{ ⊠}
  \texttt{\ supset.\ not\ }}
\item
  \phantomsection\label{symbol-supset.sq}{{ � }
  \texttt{\ supset.\ sq\ }}
\item
  \phantomsection\label{symbol-supset.sq.neq}{{ â‹¥ }
  \texttt{\ supset.\ sq.\ neq\ }}
\item
  \phantomsection\label{symbol-union}{{ ∪ } \texttt{\ union\ }}
\item
  \phantomsection\label{symbol-union.arrow}{{ ⊌ }
  \texttt{\ union.\ arrow\ }}
\item
  \phantomsection\label{symbol-union.big}{{ ⋃ }
  \texttt{\ union.\ big\ }}
\item
  \phantomsection\label{symbol-union.dot}{{ � }
  \texttt{\ union.\ dot\ }}
\item
  \phantomsection\label{symbol-union.dot.big}{{ ⨃ }
  \texttt{\ union.\ dot.\ big\ }}
\item
  \phantomsection\label{symbol-union.double}{{ â‹`` }
  \texttt{\ union.\ double\ }}
\item
  \phantomsection\label{symbol-union.minus}{{ â©? }
  \texttt{\ union.\ minus\ }}
\item
  \phantomsection\label{symbol-union.or}{{ â© } \texttt{\ union.\ or\ }}
\item
  \phantomsection\label{symbol-union.plus}{{ ⊎ }
  \texttt{\ union.\ plus\ }}
\item
  \phantomsection\label{symbol-union.plus.big}{{ ⨄ }
  \texttt{\ union.\ plus.\ big\ }}
\item
  \phantomsection\label{symbol-union.sq}{{ âŠ'' }
  \texttt{\ union.\ sq\ }}
\item
  \phantomsection\label{symbol-union.sq.big}{{ ⨆ }
  \texttt{\ union.\ sq.\ big\ }}
\item
  \phantomsection\label{symbol-union.sq.double}{{ â©? }
  \texttt{\ union.\ sq.\ double\ }}
\item
  \phantomsection\label{symbol-sect}{{ ∩ } \texttt{\ sect\ }}
\item
  \phantomsection\label{symbol-sect.and}{{ â©„ }
  \texttt{\ sect.\ and\ }}
\item
  \phantomsection\label{symbol-sect.big}{{ â‹‚ }
  \texttt{\ sect.\ big\ }}
\item
  \phantomsection\label{symbol-sect.dot}{{ â©€ }
  \texttt{\ sect.\ dot\ }}
\item
  \phantomsection\label{symbol-sect.double}{{ â‹' }
  \texttt{\ sect.\ double\ }}
\item
  \phantomsection\label{symbol-sect.sq}{{ âŠ`` } \texttt{\ sect.\ sq\ }}
\item
  \phantomsection\label{symbol-sect.sq.big}{{ ⨠}
  \texttt{\ sect.\ sq.\ big\ }}
\item
  \phantomsection\label{symbol-sect.sq.double}{{ â©Ž }
  \texttt{\ sect.\ sq.\ double\ }}
\item
  \phantomsection\label{symbol-infinity}{{ ∞ } \texttt{\ infinity\ }}
\item
  \phantomsection\label{symbol-infinity.bar}{{ ⧞ }
  \texttt{\ infinity.\ bar\ }}
\item
  \phantomsection\label{symbol-infinity.incomplete}{{ ⧜ }
  \texttt{\ infinity.\ incomplete\ }}
\item
  \phantomsection\label{symbol-infinity.tie}{{ � }
  \texttt{\ infinity.\ tie\ }}
\item
  \phantomsection\label{symbol-oo}{{ ∞ } \texttt{\ oo\ }}
\item
  \phantomsection\label{symbol-diff}{{ ∂ } \texttt{\ diff\ }}
\item
  \phantomsection\label{symbol-partial}{{ ∂ } \texttt{\ partial\ }}
\item
  \phantomsection\label{symbol-gradient}{{ ∇ } \texttt{\ gradient\ }}
\item
  \phantomsection\label{symbol-nabla}{{ ∇ } \texttt{\ nabla\ }}
\item
  \phantomsection\label{symbol-sum}{{ âˆ` } \texttt{\ sum\ }}
\item
  \phantomsection\label{symbol-sum.integral}{{ ⨋ }
  \texttt{\ sum.\ integral\ }}
\item
  \phantomsection\label{symbol-product}{{ � } \texttt{\ product\ }}
\item
  \phantomsection\label{symbol-product.co}{{ � }
  \texttt{\ product.\ co\ }}
\item
  \phantomsection\label{symbol-integral}{{ ∫ } \texttt{\ integral\ }}
\item
  \phantomsection\label{symbol-integral.arrow.hook}{{ â¨--- }
  \texttt{\ integral.\ arrow.\ hook\ }}
\item
  \phantomsection\label{symbol-integral.ccw}{{ â¨` }
  \texttt{\ integral.\ ccw\ }}
\item
  \phantomsection\label{symbol-integral.cont}{{ ∮ }
  \texttt{\ integral.\ cont\ }}
\item
  \phantomsection\label{symbol-integral.cont.ccw}{{ ∳ }
  \texttt{\ integral.\ cont.\ ccw\ }}
\item
  \phantomsection\label{symbol-integral.cont.cw}{{ ∲ }
  \texttt{\ integral.\ cont.\ cw\ }}
\item
  \phantomsection\label{symbol-integral.cw}{{ ∱ }
  \texttt{\ integral.\ cw\ }}
\item
  \phantomsection\label{symbol-integral.dash}{{ � }
  \texttt{\ integral.\ dash\ }}
\item
  \phantomsection\label{symbol-integral.dash.double}{{ ⨎ }
  \texttt{\ integral.\ dash.\ double\ }}
\item
  \phantomsection\label{symbol-integral.double}{{ ∬ }
  \texttt{\ integral.\ double\ }}
\item
  \phantomsection\label{symbol-integral.quad}{{ ⨌ }
  \texttt{\ integral.\ quad\ }}
\item
  \phantomsection\label{symbol-integral.sect}{{ ⨙ }
  \texttt{\ integral.\ sect\ }}
\item
  \phantomsection\label{symbol-integral.slash}{{ � }
  \texttt{\ integral.\ slash\ }}
\item
  \phantomsection\label{symbol-integral.square}{{ â¨-- }
  \texttt{\ integral.\ square\ }}
\item
  \phantomsection\label{symbol-integral.surf}{{ ∯ }
  \texttt{\ integral.\ surf\ }}
\item
  \phantomsection\label{symbol-integral.times}{{ ⨘ }
  \texttt{\ integral.\ times\ }}
\item
  \phantomsection\label{symbol-integral.triple}{{ ∭ }
  \texttt{\ integral.\ triple\ }}
\item
  \phantomsection\label{symbol-integral.union}{{ ⨚ }
  \texttt{\ integral.\ union\ }}
\item
  \phantomsection\label{symbol-integral.vol}{{ ∰ }
  \texttt{\ integral.\ vol\ }}
\item
  \phantomsection\label{symbol-laplace}{{ ∆ } \texttt{\ laplace\ }}
\item
  \phantomsection\label{symbol-forall}{{ ∀ } \texttt{\ forall\ }}
\item
  \phantomsection\label{symbol-exists}{{ ∃ } \texttt{\ exists\ }}
\item
  \phantomsection\label{symbol-exists.not}{{ ∄ }
  \texttt{\ exists.\ not\ }}
\item
  \phantomsection\label{symbol-top}{{ ⊤ } \texttt{\ top\ }}
\item
  \phantomsection\label{symbol-bot}{{ ⊥ } \texttt{\ bot\ }}
\item
  \phantomsection\label{symbol-not}{{ ¬ } \texttt{\ not\ }}
\item
  \phantomsection\label{symbol-and}{{ ∧ } \texttt{\ and\ }}
\item
  \phantomsection\label{symbol-and.big}{{ â‹€ } \texttt{\ and.\ big\ }}
\item
  \phantomsection\label{symbol-and.curly}{{ â‹? }
  \texttt{\ and.\ curly\ }}
\item
  \phantomsection\label{symbol-and.dot}{{ âŸ` } \texttt{\ and.\ dot\ }}
\item
  \phantomsection\label{symbol-and.double}{{ â©`` }
  \texttt{\ and.\ double\ }}
\item
  \phantomsection\label{symbol-or}{{ ∨ } \texttt{\ or\ }}
\item
  \phantomsection\label{symbol-or.big}{{ â‹? } \texttt{\ or.\ big\ }}
\item
  \phantomsection\label{symbol-or.curly}{{ â‹Ž }
  \texttt{\ or.\ curly\ }}
\item
  \phantomsection\label{symbol-or.dot}{{ ⟇ } \texttt{\ or.\ dot\ }}
\item
  \phantomsection\label{symbol-or.double}{{ â©'' }
  \texttt{\ or.\ double\ }}
\item
  \phantomsection\label{symbol-xor}{{ ⊕ } \texttt{\ xor\ }}
\item
  \phantomsection\label{symbol-xor.big}{{ � } \texttt{\ xor.\ big\ }}
\item
  \phantomsection\label{symbol-models}{{ ⊧ } \texttt{\ models\ }}
\item
  \phantomsection\label{symbol-forces}{{ ⊩ } \texttt{\ forces\ }}
\item
  \phantomsection\label{symbol-forces.not}{{ ⊮ }
  \texttt{\ forces.\ not\ }}
\item
  \phantomsection\label{symbol-therefore}{{ ∴ }
  \texttt{\ therefore\ }}
\item
  \phantomsection\label{symbol-because}{{ ∵ } \texttt{\ because\ }}
\item
  \phantomsection\label{symbol-qed}{{ ∎ } \texttt{\ qed\ }}
\item
  \phantomsection\label{symbol-compose}{{ ∘ } \texttt{\ compose\ }}
\item
  \phantomsection\label{symbol-convolve}{{ âˆ--- }
  \texttt{\ convolve\ }}
\item
  \phantomsection\label{symbol-multimap}{{ ⊸ } \texttt{\ multimap\ }}
\item
  \phantomsection\label{symbol-multimap.double}{{ ⧟ }
  \texttt{\ multimap.\ double\ }}
\item
  \phantomsection\label{symbol-tiny}{{ ⧾ } \texttt{\ tiny\ }}
\item
  \phantomsection\label{symbol-miny}{{ ⧿ } \texttt{\ miny\ }}
\item
  \phantomsection\label{symbol-divides}{{ ∣ } \texttt{\ divides\ }}
\item
  \phantomsection\label{symbol-divides.not}{{ ∤ }
  \texttt{\ divides.\ not\ }}
\item
  \phantomsection\label{symbol-wreath}{{ ≀ } \texttt{\ wreath\ }}
\item
  \phantomsection\label{symbol-parallel}{{ ∥ } \texttt{\ parallel\ }}
\item
  \phantomsection\label{symbol-parallel.struck}{{ ⫲ }
  \texttt{\ parallel.\ struck\ }}
\item
  \phantomsection\label{symbol-parallel.circle}{{ ⦷ }
  \texttt{\ parallel.\ circle\ }}
\item
  \phantomsection\label{symbol-parallel.eq}{{ â‹• }
  \texttt{\ parallel.\ eq\ }}
\item
  \phantomsection\label{symbol-parallel.equiv}{{ ⩨ }
  \texttt{\ parallel.\ equiv\ }}
\item
  \phantomsection\label{symbol-parallel.not}{{ ∦ }
  \texttt{\ parallel.\ not\ }}
\item
  \phantomsection\label{symbol-parallel.slanted.eq}{{ ⧣ }
  \texttt{\ parallel.\ slanted.\ eq\ }}
\item
  \phantomsection\label{symbol-parallel.slanted.eq.tilde}{{ ⧤ }
  \texttt{\ parallel.\ slanted.\ eq.\ tilde\ }}
\item
  \phantomsection\label{symbol-parallel.slanted.equiv}{{ ⧥ }
  \texttt{\ parallel.\ slanted.\ equiv\ }}
\item
  \phantomsection\label{symbol-parallel.tilde}{{ ⫳ }
  \texttt{\ parallel.\ tilde\ }}
\item
  \phantomsection\label{symbol-perp}{{ ⟂ } \texttt{\ perp\ }}
\item
  \phantomsection\label{symbol-perp.circle}{{ ⦹ }
  \texttt{\ perp.\ circle\ }}
\item
  \phantomsection\label{symbol-diameter}{{ ⌀ } \texttt{\ diameter\ }}
\item
  \phantomsection\label{symbol-join}{{ � } \texttt{\ join\ }}
\item
  \phantomsection\label{symbol-join.r}{{ âŸ-- } \texttt{\ join.\ r\ }}
\item
  \phantomsection\label{symbol-join.l}{{ ⟕ } \texttt{\ join.\ l\ }}
\item
  \phantomsection\label{symbol-join.l.r}{{ âŸ--- }
  \texttt{\ join.\ l.\ r\ }}
\item
  \phantomsection\label{symbol-degree}{{ ° } \texttt{\ degree\ }}
\item
  \phantomsection\label{symbol-degree.c}{{ ℃ }
  \texttt{\ degree.\ c\ }}
\item
  \phantomsection\label{symbol-degree.f}{{ ℉ }
  \texttt{\ degree.\ f\ }}
\item
  \phantomsection\label{symbol-smash}{{ ⨳ } \texttt{\ smash\ }}
\item
  \phantomsection\label{symbol-bitcoin}{{ â‚¿ } \texttt{\ bitcoin\ }}
\item
  \phantomsection\label{symbol-dollar}{{ \$ } \texttt{\ dollar\ }}
\item
  \phantomsection\label{symbol-euro}{{ € } \texttt{\ euro\ }}
\item
  \phantomsection\label{symbol-franc}{{ â‚£ } \texttt{\ franc\ }}
\item
  \phantomsection\label{symbol-lira}{{ ₺ } \texttt{\ lira\ }}
\item
  \phantomsection\label{symbol-peso}{{ ₱ } \texttt{\ peso\ }}
\item
  \phantomsection\label{symbol-pound}{{ £ } \texttt{\ pound\ }}
\item
  \phantomsection\label{symbol-ruble}{{ ₽ } \texttt{\ ruble\ }}
\item
  \phantomsection\label{symbol-rupee}{{ ₹ } \texttt{\ rupee\ }}
\item
  \phantomsection\label{symbol-won}{{ â‚© } \texttt{\ won\ }}
\item
  \phantomsection\label{symbol-yen}{{ ¥ } \texttt{\ yen\ }}
\item
  \phantomsection\label{symbol-ballot}{{ � } \texttt{\ ballot\ }}
\item
  \phantomsection\label{symbol-ballot.cross}{{ â˜' }
  \texttt{\ ballot.\ cross\ }}
\item
  \phantomsection\label{symbol-ballot.check}{{ â˜` }
  \texttt{\ ballot.\ check\ }}
\item
  \phantomsection\label{symbol-ballot.check.heavy}{{ ðŸ---¹ }
  \texttt{\ ballot.\ check.\ heavy\ }}
\item
  \phantomsection\label{symbol-checkmark}{{ âœ`` }
  \texttt{\ checkmark\ }}
\item
  \phantomsection\label{symbol-checkmark.light}{{ ðŸ---¸ }
  \texttt{\ checkmark.\ light\ }}
\item
  \phantomsection\label{symbol-checkmark.heavy}{{ âœ'' }
  \texttt{\ checkmark.\ heavy\ }}
\item
  \phantomsection\label{symbol-crossmark}{{ âœ--- }
  \texttt{\ crossmark\ }}
\item
  \phantomsection\label{symbol-crossmark.heavy}{{ ✘ }
  \texttt{\ crossmark.\ heavy\ }}
\item
  \phantomsection\label{symbol-floral}{{ â?¦ } \texttt{\ floral\ }}
\item
  \phantomsection\label{symbol-floral.l}{{ ☙ }
  \texttt{\ floral.\ l\ }}
\item
  \phantomsection\label{symbol-floral.r}{{ â?§ }
  \texttt{\ floral.\ r\ }}
\item
  \phantomsection\label{symbol-refmark}{{ ※ } \texttt{\ refmark\ }}
\item
  \phantomsection\label{symbol-copyright}{{ © } \texttt{\ copyright\ }}
\item
  \phantomsection\label{symbol-copyright.sound}{{ â„--- }
  \texttt{\ copyright.\ sound\ }}
\item
  \phantomsection\label{symbol-copyleft}{{ 🄯 } \texttt{\ copyleft\ }}
\item
  \phantomsection\label{symbol-trademark}{{ â„¢ }
  \texttt{\ trademark\ }}
\item
  \phantomsection\label{symbol-trademark.registered}{{ ® }
  \texttt{\ trademark.\ registered\ }}
\item
  \phantomsection\label{symbol-trademark.service}{{ â„ }
  \texttt{\ trademark.\ service\ }}
\item
  \phantomsection\label{symbol-maltese}{{ ✠} \texttt{\ maltese\ }}
\item
  \phantomsection\label{symbol-suit.club.filled}{{ ♣ }
  \texttt{\ suit.\ club.\ filled\ }}
\item
  \phantomsection\label{symbol-suit.club.stroked}{{ ♧ }
  \texttt{\ suit.\ club.\ stroked\ }}
\item
  \phantomsection\label{symbol-suit.diamond.filled}{{ ♦ }
  \texttt{\ suit.\ diamond.\ filled\ }}
\item
  \phantomsection\label{symbol-suit.diamond.stroked}{{ ♢ }
  \texttt{\ suit.\ diamond.\ stroked\ }}
\item
  \phantomsection\label{symbol-suit.heart.filled}{{ ♥ }
  \texttt{\ suit.\ heart.\ filled\ }}
\item
  \phantomsection\label{symbol-suit.heart.stroked}{{ ♡ }
  \texttt{\ suit.\ heart.\ stroked\ }}
\item
  \phantomsection\label{symbol-suit.spade.filled}{{ â™ }
  \texttt{\ suit.\ spade.\ filled\ }}
\item
  \phantomsection\label{symbol-suit.spade.stroked}{{ ♤ }
  \texttt{\ suit.\ spade.\ stroked\ }}
\item
  \phantomsection\label{symbol-note.up}{{ 🎜 } \texttt{\ note.\ up\ }}
\item
  \phantomsection\label{symbol-note.down}{{ � }
  \texttt{\ note.\ down\ }}
\item
  \phantomsection\label{symbol-note.whole}{{ ð?\ldots? }
  \texttt{\ note.\ whole\ }}
\item
  \phantomsection\label{symbol-note.half}{{ ð?\ldots ž }
  \texttt{\ note.\ half\ }}
\item
  \phantomsection\label{symbol-note.quarter}{{ ð?\ldots Ÿ }
  \texttt{\ note.\ quarter\ }}
\item
  \phantomsection\label{symbol-note.quarter.alt}{{ ♩ }
  \texttt{\ note.\ quarter.\ alt\ }}
\item
  \phantomsection\label{symbol-note.eighth}{{ ð? }
  \texttt{\ note.\ eighth\ }}
\item
  \phantomsection\label{symbol-note.eighth.alt}{{ ♪ }
  \texttt{\ note.\ eighth.\ alt\ }}
\item
  \phantomsection\label{symbol-note.eighth.beamed}{{ ♫ }
  \texttt{\ note.\ eighth.\ beamed\ }}
\item
  \phantomsection\label{symbol-note.sixteenth}{{ ð?\ldots¡ }
  \texttt{\ note.\ sixteenth\ }}
\item
  \phantomsection\label{symbol-note.sixteenth.beamed}{{ ♬ }
  \texttt{\ note.\ sixteenth.\ beamed\ }}
\item
  \phantomsection\label{symbol-note.grace}{{ � }
  \texttt{\ note.\ grace\ }}
\item
  \phantomsection\label{symbol-note.grace.slash}{{ ð?†'' }
  \texttt{\ note.\ grace.\ slash\ }}
\item
  \phantomsection\label{symbol-rest.whole}{{ ð?„» }
  \texttt{\ rest.\ whole\ }}
\item
  \phantomsection\label{symbol-rest.multiple}{{ � }
  \texttt{\ rest.\ multiple\ }}
\item
  \phantomsection\label{symbol-rest.multiple.measure}{{ ð?„© }
  \texttt{\ rest.\ multiple.\ measure\ }}
\item
  \phantomsection\label{symbol-rest.half}{{ ð?„¼ }
  \texttt{\ rest.\ half\ }}
\item
  \phantomsection\label{symbol-rest.quarter}{{ ð?„½ }
  \texttt{\ rest.\ quarter\ }}
\item
  \phantomsection\label{symbol-rest.eighth}{{ ð?„¾ }
  \texttt{\ rest.\ eighth\ }}
\item
  \phantomsection\label{symbol-rest.sixteenth}{{ ð?„¿ }
  \texttt{\ rest.\ sixteenth\ }}
\item
  \phantomsection\label{symbol-natural}{{ â™® } \texttt{\ natural\ }}
\item
  \phantomsection\label{symbol-natural.t}{{ ð?„® }
  \texttt{\ natural.\ t\ }}
\item
  \phantomsection\label{symbol-natural.b}{{ � }
  \texttt{\ natural.\ b\ }}
\item
  \phantomsection\label{symbol-flat}{{ â™­ } \texttt{\ flat\ }}
\item
  \phantomsection\label{symbol-flat.t}{{ � } \texttt{\ flat.\ t\ }}
\item
  \phantomsection\label{symbol-flat.b}{{ ð?„­ } \texttt{\ flat.\ b\ }}
\item
  \phantomsection\label{symbol-flat.double}{{ ð?„« }
  \texttt{\ flat.\ double\ }}
\item
  \phantomsection\label{symbol-flat.quarter}{{ ð?„³ }
  \texttt{\ flat.\ quarter\ }}
\item
  \phantomsection\label{symbol-sharp}{{ ♯ } \texttt{\ sharp\ }}
\item
  \phantomsection\label{symbol-sharp.t}{{ ð?„° } \texttt{\ sharp.\ t\ }}
\item
  \phantomsection\label{symbol-sharp.b}{{ ð?„± } \texttt{\ sharp.\ b\ }}
\item
  \phantomsection\label{symbol-sharp.double}{{ � }
  \texttt{\ sharp.\ double\ }}
\item
  \phantomsection\label{symbol-sharp.quarter}{{ ð?„² }
  \texttt{\ sharp.\ quarter\ }}
\item
  \phantomsection\label{symbol-bullet}{{ • } \texttt{\ bullet\ }}
\item
  \phantomsection\label{symbol-circle.stroked}{{ â---‹ }
  \texttt{\ circle.\ stroked\ }}
\item
  \phantomsection\label{symbol-circle.stroked.tiny}{{ ∘ }
  \texttt{\ circle.\ stroked.\ tiny\ }}
\item
  \phantomsection\label{symbol-circle.stroked.small}{{ ⚬ }
  \texttt{\ circle.\ stroked.\ small\ }}
\item
  \phantomsection\label{symbol-circle.stroked.big}{{ â---¯ }
  \texttt{\ circle.\ stroked.\ big\ }}
\item
  \phantomsection\label{symbol-circle.filled}{{ â---? }
  \texttt{\ circle.\ filled\ }}
\item
  \phantomsection\label{symbol-circle.filled.tiny}{{ � }
  \texttt{\ circle.\ filled.\ tiny\ }}
\item
  \phantomsection\label{symbol-circle.filled.small}{{ ∙ }
  \texttt{\ circle.\ filled.\ small\ }}
\item
  \phantomsection\label{symbol-circle.filled.big}{{ ⬤ }
  \texttt{\ circle.\ filled.\ big\ }}
\item
  \phantomsection\label{symbol-circle.dotted}{{ â---Œ }
  \texttt{\ circle.\ dotted\ }}
\item
  \phantomsection\label{symbol-circle.nested}{{ ⊚ }
  \texttt{\ circle.\ nested\ }}
\item
  \phantomsection\label{symbol-ellipse.stroked.h}{{ ⬭ }
  \texttt{\ ellipse.\ stroked.\ h\ }}
\item
  \phantomsection\label{symbol-ellipse.stroked.v}{{ ⬯ }
  \texttt{\ ellipse.\ stroked.\ v\ }}
\item
  \phantomsection\label{symbol-ellipse.filled.h}{{ ⬬ }
  \texttt{\ ellipse.\ filled.\ h\ }}
\item
  \phantomsection\label{symbol-ellipse.filled.v}{{ ⬮ }
  \texttt{\ ellipse.\ filled.\ v\ }}
\item
  \phantomsection\label{symbol-triangle.stroked.t}{{ â--³ }
  \texttt{\ triangle.\ stroked.\ t\ }}
\item
  \phantomsection\label{symbol-triangle.stroked.b}{{ â--½ }
  \texttt{\ triangle.\ stroked.\ b\ }}
\item
  \phantomsection\label{symbol-triangle.stroked.r}{{ â--· }
  \texttt{\ triangle.\ stroked.\ r\ }}
\item
  \phantomsection\label{symbol-triangle.stroked.l}{{ â---? }
  \texttt{\ triangle.\ stroked.\ l\ }}
\item
  \phantomsection\label{symbol-triangle.stroked.bl}{{ â---º }
  \texttt{\ triangle.\ stroked.\ bl\ }}
\item
  \phantomsection\label{symbol-triangle.stroked.br}{{ â---¿ }
  \texttt{\ triangle.\ stroked.\ br\ }}
\item
  \phantomsection\label{symbol-triangle.stroked.tl}{{ â---¸ }
  \texttt{\ triangle.\ stroked.\ tl\ }}
\item
  \phantomsection\label{symbol-triangle.stroked.tr}{{ â---¹ }
  \texttt{\ triangle.\ stroked.\ tr\ }}
\item
  \phantomsection\label{symbol-triangle.stroked.small.t}{{ â--µ }
  \texttt{\ triangle.\ stroked.\ small.\ t\ }}
\item
  \phantomsection\label{symbol-triangle.stroked.small.b}{{ â--¿ }
  \texttt{\ triangle.\ stroked.\ small.\ b\ }}
\item
  \phantomsection\label{symbol-triangle.stroked.small.r}{{ â--¹ }
  \texttt{\ triangle.\ stroked.\ small.\ r\ }}
\item
  \phantomsection\label{symbol-triangle.stroked.small.l}{{ â---ƒ }
  \texttt{\ triangle.\ stroked.\ small.\ l\ }}
\item
  \phantomsection\label{symbol-triangle.stroked.rounded}{{ 🛆 }
  \texttt{\ triangle.\ stroked.\ rounded\ }}
\item
  \phantomsection\label{symbol-triangle.stroked.nested}{{ � }
  \texttt{\ triangle.\ stroked.\ nested\ }}
\item
  \phantomsection\label{symbol-triangle.stroked.dot}{{ â---¬ }
  \texttt{\ triangle.\ stroked.\ dot\ }}
\item
  \phantomsection\label{symbol-triangle.filled.t}{{ â--² }
  \texttt{\ triangle.\ filled.\ t\ }}
\item
  \phantomsection\label{symbol-triangle.filled.b}{{ â--¼ }
  \texttt{\ triangle.\ filled.\ b\ }}
\item
  \phantomsection\label{symbol-triangle.filled.r}{{ â--¶ }
  \texttt{\ triangle.\ filled.\ r\ }}
\item
  \phantomsection\label{symbol-triangle.filled.l}{{ â---€ }
  \texttt{\ triangle.\ filled.\ l\ }}
\item
  \phantomsection\label{symbol-triangle.filled.bl}{{ â---£ }
  \texttt{\ triangle.\ filled.\ bl\ }}
\item
  \phantomsection\label{symbol-triangle.filled.br}{{ â---¢ }
  \texttt{\ triangle.\ filled.\ br\ }}
\item
  \phantomsection\label{symbol-triangle.filled.tl}{{ â---¤ }
  \texttt{\ triangle.\ filled.\ tl\ }}
\item
  \phantomsection\label{symbol-triangle.filled.tr}{{ â---¥ }
  \texttt{\ triangle.\ filled.\ tr\ }}
\item
  \phantomsection\label{symbol-triangle.filled.small.t}{{ â--´ }
  \texttt{\ triangle.\ filled.\ small.\ t\ }}
\item
  \phantomsection\label{symbol-triangle.filled.small.b}{{ â--¾ }
  \texttt{\ triangle.\ filled.\ small.\ b\ }}
\item
  \phantomsection\label{symbol-triangle.filled.small.r}{{ â--¸ }
  \texttt{\ triangle.\ filled.\ small.\ r\ }}
\item
  \phantomsection\label{symbol-triangle.filled.small.l}{{ â---‚ }
  \texttt{\ triangle.\ filled.\ small.\ l\ }}
\item
  \phantomsection\label{symbol-square.stroked}{{ â--¡ }
  \texttt{\ square.\ stroked\ }}
\item
  \phantomsection\label{symbol-square.stroked.tiny}{{ â--« }
  \texttt{\ square.\ stroked.\ tiny\ }}
\item
  \phantomsection\label{symbol-square.stroked.small}{{ â---½ }
  \texttt{\ square.\ stroked.\ small\ }}
\item
  \phantomsection\label{symbol-square.stroked.medium}{{ â---» }
  \texttt{\ square.\ stroked.\ medium\ }}
\item
  \phantomsection\label{symbol-square.stroked.big}{{ ⬜ }
  \texttt{\ square.\ stroked.\ big\ }}
\item
  \phantomsection\label{symbol-square.stroked.dotted}{{ ⬚ }
  \texttt{\ square.\ stroked.\ dotted\ }}
\item
  \phantomsection\label{symbol-square.stroked.rounded}{{ â--¢ }
  \texttt{\ square.\ stroked.\ rounded\ }}
\item
  \phantomsection\label{symbol-square.filled}{{ â-- }
  \texttt{\ square.\ filled\ }}
\item
  \phantomsection\label{symbol-square.filled.tiny}{{ â--ª }
  \texttt{\ square.\ filled.\ tiny\ }}
\item
  \phantomsection\label{symbol-square.filled.small}{{ â---¾ }
  \texttt{\ square.\ filled.\ small\ }}
\item
  \phantomsection\label{symbol-square.filled.medium}{{ â---¼ }
  \texttt{\ square.\ filled.\ medium\ }}
\item
  \phantomsection\label{symbol-square.filled.big}{{ ⬛ }
  \texttt{\ square.\ filled.\ big\ }}
\item
  \phantomsection\label{symbol-rect.stroked.h}{{ â--­ }
  \texttt{\ rect.\ stroked.\ h\ }}
\item
  \phantomsection\label{symbol-rect.stroked.v}{{ â--¯ }
  \texttt{\ rect.\ stroked.\ v\ }}
\item
  \phantomsection\label{symbol-rect.filled.h}{{ â--¬ }
  \texttt{\ rect.\ filled.\ h\ }}
\item
  \phantomsection\label{symbol-rect.filled.v}{{ â--® }
  \texttt{\ rect.\ filled.\ v\ }}
\item
  \phantomsection\label{symbol-penta.stroked}{{ ⬠}
  \texttt{\ penta.\ stroked\ }}
\item
  \phantomsection\label{symbol-penta.filled}{{ ⬟ }
  \texttt{\ penta.\ filled\ }}
\item
  \phantomsection\label{symbol-hexa.stroked}{{ ⬡ }
  \texttt{\ hexa.\ stroked\ }}
\item
  \phantomsection\label{symbol-hexa.filled}{{ ⬢ }
  \texttt{\ hexa.\ filled\ }}
\item
  \phantomsection\label{symbol-diamond.stroked}{{ â---‡ }
  \texttt{\ diamond.\ stroked\ }}
\item
  \phantomsection\label{symbol-diamond.stroked.small}{{ â‹„ }
  \texttt{\ diamond.\ stroked.\ small\ }}
\item
  \phantomsection\label{symbol-diamond.stroked.medium}{{ ⬦ }
  \texttt{\ diamond.\ stroked.\ medium\ }}
\item
  \phantomsection\label{symbol-diamond.stroked.dot}{{ � }
  \texttt{\ diamond.\ stroked.\ dot\ }}
\item
  \phantomsection\label{symbol-diamond.filled}{{ â---† }
  \texttt{\ diamond.\ filled\ }}
\item
  \phantomsection\label{symbol-diamond.filled.medium}{{ ⬥ }
  \texttt{\ diamond.\ filled.\ medium\ }}
\item
  \phantomsection\label{symbol-diamond.filled.small}{{ ⬩ }
  \texttt{\ diamond.\ filled.\ small\ }}
\item
  \phantomsection\label{symbol-lozenge.stroked}{{ â---Š }
  \texttt{\ lozenge.\ stroked\ }}
\item
  \phantomsection\label{symbol-lozenge.stroked.small}{{ ⬫ }
  \texttt{\ lozenge.\ stroked.\ small\ }}
\item
  \phantomsection\label{symbol-lozenge.stroked.medium}{{ ⬨ }
  \texttt{\ lozenge.\ stroked.\ medium\ }}
\item
  \phantomsection\label{symbol-lozenge.filled}{{ ⧫ }
  \texttt{\ lozenge.\ filled\ }}
\item
  \phantomsection\label{symbol-lozenge.filled.small}{{ ⬪ }
  \texttt{\ lozenge.\ filled.\ small\ }}
\item
  \phantomsection\label{symbol-lozenge.filled.medium}{{ ⬧ }
  \texttt{\ lozenge.\ filled.\ medium\ }}
\item
  \phantomsection\label{symbol-parallelogram.stroked}{{ â--± }
  \texttt{\ parallelogram.\ stroked\ }}
\item
  \phantomsection\label{symbol-parallelogram.filled}{{ â--° }
  \texttt{\ parallelogram.\ filled\ }}
\item
  \phantomsection\label{symbol-star.op}{{ ⋆ } \texttt{\ star.\ op\ }}
\item
  \phantomsection\label{symbol-star.stroked}{{ ☆ }
  \texttt{\ star.\ stroked\ }}
\item
  \phantomsection\label{symbol-star.filled}{{ ☠}
  \texttt{\ star.\ filled\ }}
\item
  \phantomsection\label{symbol-arrow.r}{{ â†' } \texttt{\ arrow.\ r\ }}
\item
  \phantomsection\label{symbol-arrow.r.long.bar}{{ ⟼ }
  \texttt{\ arrow.\ r.\ long.\ bar\ }}
\item
  \phantomsection\label{symbol-arrow.r.bar}{{ ↦ }
  \texttt{\ arrow.\ r.\ bar\ }}
\item
  \phantomsection\label{symbol-arrow.r.curve}{{ ⤷ }
  \texttt{\ arrow.\ r.\ curve\ }}
\item
  \phantomsection\label{symbol-arrow.r.turn}{{ ⮎ }
  \texttt{\ arrow.\ r.\ turn\ }}
\item
  \phantomsection\label{symbol-arrow.r.dashed}{{ ⇢ }
  \texttt{\ arrow.\ r.\ dashed\ }}
\item
  \phantomsection\label{symbol-arrow.r.dotted}{{ â¤` }
  \texttt{\ arrow.\ r.\ dotted\ }}
\item
  \phantomsection\label{symbol-arrow.r.double}{{ â‡' }
  \texttt{\ arrow.\ r.\ double\ }}
\item
  \phantomsection\label{symbol-arrow.r.double.bar}{{ ⤇ }
  \texttt{\ arrow.\ r.\ double.\ bar\ }}
\item
  \phantomsection\label{symbol-arrow.r.double.long}{{ ⟹ }
  \texttt{\ arrow.\ r.\ double.\ long\ }}
\item
  \phantomsection\label{symbol-arrow.r.double.long.bar}{{ ⟾ }
  \texttt{\ arrow.\ r.\ double.\ long.\ bar\ }}
\item
  \phantomsection\label{symbol-arrow.r.double.not}{{ � }
  \texttt{\ arrow.\ r.\ double.\ not\ }}
\item
  \phantomsection\label{symbol-arrow.r.filled}{{ âž¡ }
  \texttt{\ arrow.\ r.\ filled\ }}
\item
  \phantomsection\label{symbol-arrow.r.hook}{{ ↪ }
  \texttt{\ arrow.\ r.\ hook\ }}
\item
  \phantomsection\label{symbol-arrow.r.long}{{ ⟶ }
  \texttt{\ arrow.\ r.\ long\ }}
\item
  \phantomsection\label{symbol-arrow.r.long.squiggly}{{ ⟿ }
  \texttt{\ arrow.\ r.\ long.\ squiggly\ }}
\item
  \phantomsection\label{symbol-arrow.r.loop}{{ ↬ }
  \texttt{\ arrow.\ r.\ loop\ }}
\item
  \phantomsection\label{symbol-arrow.r.not}{{ ↛ }
  \texttt{\ arrow.\ r.\ not\ }}
\item
  \phantomsection\label{symbol-arrow.r.quad}{{ â­† }
  \texttt{\ arrow.\ r.\ quad\ }}
\item
  \phantomsection\label{symbol-arrow.r.squiggly}{{ � }
  \texttt{\ arrow.\ r.\ squiggly\ }}
\item
  \phantomsection\label{symbol-arrow.r.stop}{{ ⇥ }
  \texttt{\ arrow.\ r.\ stop\ }}
\item
  \phantomsection\label{symbol-arrow.r.stroked}{{ ⇨ }
  \texttt{\ arrow.\ r.\ stroked\ }}
\item
  \phantomsection\label{symbol-arrow.r.tail}{{ ↣ }
  \texttt{\ arrow.\ r.\ tail\ }}
\item
  \phantomsection\label{symbol-arrow.r.tilde}{{ ⥲ }
  \texttt{\ arrow.\ r.\ tilde\ }}
\item
  \phantomsection\label{symbol-arrow.r.triple}{{ ⇛ }
  \texttt{\ arrow.\ r.\ triple\ }}
\item
  \phantomsection\label{symbol-arrow.r.twohead.bar}{{ ⤠}
  \texttt{\ arrow.\ r.\ twohead.\ bar\ }}
\item
  \phantomsection\label{symbol-arrow.r.twohead}{{ ↠}
  \texttt{\ arrow.\ r.\ twohead\ }}
\item
  \phantomsection\label{symbol-arrow.r.wave}{{ � }
  \texttt{\ arrow.\ r.\ wave\ }}
\item
  \phantomsection\label{symbol-arrow.l}{{ � } \texttt{\ arrow.\ l\ }}
\item
  \phantomsection\label{symbol-arrow.l.bar}{{ ↤ }
  \texttt{\ arrow.\ l.\ bar\ }}
\item
  \phantomsection\label{symbol-arrow.l.curve}{{ ⤶ }
  \texttt{\ arrow.\ l.\ curve\ }}
\item
  \phantomsection\label{symbol-arrow.l.turn}{{ ⮌ }
  \texttt{\ arrow.\ l.\ turn\ }}
\item
  \phantomsection\label{symbol-arrow.l.dashed}{{ ⇠}
  \texttt{\ arrow.\ l.\ dashed\ }}
\item
  \phantomsection\label{symbol-arrow.l.dotted}{{ ⬸ }
  \texttt{\ arrow.\ l.\ dotted\ }}
\item
  \phantomsection\label{symbol-arrow.l.double}{{ � }
  \texttt{\ arrow.\ l.\ double\ }}
\item
  \phantomsection\label{symbol-arrow.l.double.bar}{{ ⤆ }
  \texttt{\ arrow.\ l.\ double.\ bar\ }}
\item
  \phantomsection\label{symbol-arrow.l.double.long}{{ ⟸ }
  \texttt{\ arrow.\ l.\ double.\ long\ }}
\item
  \phantomsection\label{symbol-arrow.l.double.long.bar}{{ ⟽ }
  \texttt{\ arrow.\ l.\ double.\ long.\ bar\ }}
\item
  \phantomsection\label{symbol-arrow.l.double.not}{{ � }
  \texttt{\ arrow.\ l.\ double.\ not\ }}
\item
  \phantomsection\label{symbol-arrow.l.filled}{{ ⬠}
  \texttt{\ arrow.\ l.\ filled\ }}
\item
  \phantomsection\label{symbol-arrow.l.hook}{{ ↩ }
  \texttt{\ arrow.\ l.\ hook\ }}
\item
  \phantomsection\label{symbol-arrow.l.long}{{ ⟵ }
  \texttt{\ arrow.\ l.\ long\ }}
\item
  \phantomsection\label{symbol-arrow.l.long.bar}{{ ⟻ }
  \texttt{\ arrow.\ l.\ long.\ bar\ }}
\item
  \phantomsection\label{symbol-arrow.l.long.squiggly}{{ ⬳ }
  \texttt{\ arrow.\ l.\ long.\ squiggly\ }}
\item
  \phantomsection\label{symbol-arrow.l.loop}{{ ↫ }
  \texttt{\ arrow.\ l.\ loop\ }}
\item
  \phantomsection\label{symbol-arrow.l.not}{{ ↚ }
  \texttt{\ arrow.\ l.\ not\ }}
\item
  \phantomsection\label{symbol-arrow.l.quad}{{ â­ }
  \texttt{\ arrow.\ l.\ quad\ }}
\item
  \phantomsection\label{symbol-arrow.l.squiggly}{{ ⇜ }
  \texttt{\ arrow.\ l.\ squiggly\ }}
\item
  \phantomsection\label{symbol-arrow.l.stop}{{ ⇤ }
  \texttt{\ arrow.\ l.\ stop\ }}
\item
  \phantomsection\label{symbol-arrow.l.stroked}{{ ⇦ }
  \texttt{\ arrow.\ l.\ stroked\ }}
\item
  \phantomsection\label{symbol-arrow.l.tail}{{ ↢ }
  \texttt{\ arrow.\ l.\ tail\ }}
\item
  \phantomsection\label{symbol-arrow.l.tilde}{{ â­‰ }
  \texttt{\ arrow.\ l.\ tilde\ }}
\item
  \phantomsection\label{symbol-arrow.l.triple}{{ ⇚ }
  \texttt{\ arrow.\ l.\ triple\ }}
\item
  \phantomsection\label{symbol-arrow.l.twohead.bar}{{ ⬶ }
  \texttt{\ arrow.\ l.\ twohead.\ bar\ }}
\item
  \phantomsection\label{symbol-arrow.l.twohead}{{ ↞ }
  \texttt{\ arrow.\ l.\ twohead\ }}
\item
  \phantomsection\label{symbol-arrow.l.wave}{{ ↜ }
  \texttt{\ arrow.\ l.\ wave\ }}
\item
  \phantomsection\label{symbol-arrow.t}{{ â†` } \texttt{\ arrow.\ t\ }}
\item
  \phantomsection\label{symbol-arrow.t.bar}{{ ↥ }
  \texttt{\ arrow.\ t.\ bar\ }}
\item
  \phantomsection\label{symbol-arrow.t.curve}{{ ⤴ }
  \texttt{\ arrow.\ t.\ curve\ }}
\item
  \phantomsection\label{symbol-arrow.t.turn}{{ â®? }
  \texttt{\ arrow.\ t.\ turn\ }}
\item
  \phantomsection\label{symbol-arrow.t.dashed}{{ ⇡ }
  \texttt{\ arrow.\ t.\ dashed\ }}
\item
  \phantomsection\label{symbol-arrow.t.double}{{ â‡` }
  \texttt{\ arrow.\ t.\ double\ }}
\item
  \phantomsection\label{symbol-arrow.t.filled}{{ ⬆ }
  \texttt{\ arrow.\ t.\ filled\ }}
\item
  \phantomsection\label{symbol-arrow.t.quad}{{ ⟰ }
  \texttt{\ arrow.\ t.\ quad\ }}
\item
  \phantomsection\label{symbol-arrow.t.stop}{{ â¤' }
  \texttt{\ arrow.\ t.\ stop\ }}
\item
  \phantomsection\label{symbol-arrow.t.stroked}{{ ⇧ }
  \texttt{\ arrow.\ t.\ stroked\ }}
\item
  \phantomsection\label{symbol-arrow.t.triple}{{ ⤊ }
  \texttt{\ arrow.\ t.\ triple\ }}
\item
  \phantomsection\label{symbol-arrow.t.twohead}{{ ↟ }
  \texttt{\ arrow.\ t.\ twohead\ }}
\item
  \phantomsection\label{symbol-arrow.b}{{ â†`` } \texttt{\ arrow.\ b\ }}
\item
  \phantomsection\label{symbol-arrow.b.bar}{{ ↧ }
  \texttt{\ arrow.\ b.\ bar\ }}
\item
  \phantomsection\label{symbol-arrow.b.curve}{{ ⤵ }
  \texttt{\ arrow.\ b.\ curve\ }}
\item
  \phantomsection\label{symbol-arrow.b.turn}{{ â®? }
  \texttt{\ arrow.\ b.\ turn\ }}
\item
  \phantomsection\label{symbol-arrow.b.dashed}{{ ⇣ }
  \texttt{\ arrow.\ b.\ dashed\ }}
\item
  \phantomsection\label{symbol-arrow.b.double}{{ â‡`` }
  \texttt{\ arrow.\ b.\ double\ }}
\item
  \phantomsection\label{symbol-arrow.b.filled}{{ ⬇ }
  \texttt{\ arrow.\ b.\ filled\ }}
\item
  \phantomsection\label{symbol-arrow.b.quad}{{ ⟱ }
  \texttt{\ arrow.\ b.\ quad\ }}
\item
  \phantomsection\label{symbol-arrow.b.stop}{{ â¤`` }
  \texttt{\ arrow.\ b.\ stop\ }}
\item
  \phantomsection\label{symbol-arrow.b.stroked}{{ ⇩ }
  \texttt{\ arrow.\ b.\ stroked\ }}
\item
  \phantomsection\label{symbol-arrow.b.triple}{{ ⤋ }
  \texttt{\ arrow.\ b.\ triple\ }}
\item
  \phantomsection\label{symbol-arrow.b.twohead}{{ ↡ }
  \texttt{\ arrow.\ b.\ twohead\ }}
\item
  \phantomsection\label{symbol-arrow.l.r}{{ â†'' }
  \texttt{\ arrow.\ l.\ r\ }}
\item
  \phantomsection\label{symbol-arrow.l.r.double}{{ â‡'' }
  \texttt{\ arrow.\ l.\ r.\ double\ }}
\item
  \phantomsection\label{symbol-arrow.l.r.double.long}{{ ⟺ }
  \texttt{\ arrow.\ l.\ r.\ double.\ long\ }}
\item
  \phantomsection\label{symbol-arrow.l.r.double.not}{{ ⇎ }
  \texttt{\ arrow.\ l.\ r.\ double.\ not\ }}
\item
  \phantomsection\label{symbol-arrow.l.r.filled}{{ ⬌ }
  \texttt{\ arrow.\ l.\ r.\ filled\ }}
\item
  \phantomsection\label{symbol-arrow.l.r.long}{{ ⟷ }
  \texttt{\ arrow.\ l.\ r.\ long\ }}
\item
  \phantomsection\label{symbol-arrow.l.r.not}{{ ↮ }
  \texttt{\ arrow.\ l.\ r.\ not\ }}
\item
  \phantomsection\label{symbol-arrow.l.r.stroked}{{ ⬄ }
  \texttt{\ arrow.\ l.\ r.\ stroked\ }}
\item
  \phantomsection\label{symbol-arrow.l.r.wave}{{ ↭ }
  \texttt{\ arrow.\ l.\ r.\ wave\ }}
\item
  \phantomsection\label{symbol-arrow.t.b}{{ ↕ }
  \texttt{\ arrow.\ t.\ b\ }}
\item
  \phantomsection\label{symbol-arrow.t.b.double}{{ ⇕ }
  \texttt{\ arrow.\ t.\ b.\ double\ }}
\item
  \phantomsection\label{symbol-arrow.t.b.filled}{{ � }
  \texttt{\ arrow.\ t.\ b.\ filled\ }}
\item
  \phantomsection\label{symbol-arrow.t.b.stroked}{{ ⇳ }
  \texttt{\ arrow.\ t.\ b.\ stroked\ }}
\item
  \phantomsection\label{symbol-arrow.tr}{{ â†--- }
  \texttt{\ arrow.\ tr\ }}
\item
  \phantomsection\label{symbol-arrow.tr.double}{{ â‡--- }
  \texttt{\ arrow.\ tr.\ double\ }}
\item
  \phantomsection\label{symbol-arrow.tr.filled}{{ ⬈ }
  \texttt{\ arrow.\ tr.\ filled\ }}
\item
  \phantomsection\label{symbol-arrow.tr.hook}{{ ⤤ }
  \texttt{\ arrow.\ tr.\ hook\ }}
\item
  \phantomsection\label{symbol-arrow.tr.stroked}{{ ⬀ }
  \texttt{\ arrow.\ tr.\ stroked\ }}
\item
  \phantomsection\label{symbol-arrow.br}{{ ↘ }
  \texttt{\ arrow.\ br\ }}
\item
  \phantomsection\label{symbol-arrow.br.double}{{ ⇘ }
  \texttt{\ arrow.\ br.\ double\ }}
\item
  \phantomsection\label{symbol-arrow.br.filled}{{ ⬊ }
  \texttt{\ arrow.\ br.\ filled\ }}
\item
  \phantomsection\label{symbol-arrow.br.hook}{{ ⤥ }
  \texttt{\ arrow.\ br.\ hook\ }}
\item
  \phantomsection\label{symbol-arrow.br.stroked}{{ ⬂ }
  \texttt{\ arrow.\ br.\ stroked\ }}
\item
  \phantomsection\label{symbol-arrow.tl}{{ â†-- }
  \texttt{\ arrow.\ tl\ }}
\item
  \phantomsection\label{symbol-arrow.tl.double}{{ â‡-- }
  \texttt{\ arrow.\ tl.\ double\ }}
\item
  \phantomsection\label{symbol-arrow.tl.filled}{{ ⬉ }
  \texttt{\ arrow.\ tl.\ filled\ }}
\item
  \phantomsection\label{symbol-arrow.tl.hook}{{ ⤣ }
  \texttt{\ arrow.\ tl.\ hook\ }}
\item
  \phantomsection\label{symbol-arrow.tl.stroked}{{ � }
  \texttt{\ arrow.\ tl.\ stroked\ }}
\item
  \phantomsection\label{symbol-arrow.bl}{{ ↙ }
  \texttt{\ arrow.\ bl\ }}
\item
  \phantomsection\label{symbol-arrow.bl.double}{{ ⇙ }
  \texttt{\ arrow.\ bl.\ double\ }}
\item
  \phantomsection\label{symbol-arrow.bl.filled}{{ ⬋ }
  \texttt{\ arrow.\ bl.\ filled\ }}
\item
  \phantomsection\label{symbol-arrow.bl.hook}{{ ⤦ }
  \texttt{\ arrow.\ bl.\ hook\ }}
\item
  \phantomsection\label{symbol-arrow.bl.stroked}{{ ⬃ }
  \texttt{\ arrow.\ bl.\ stroked\ }}
\item
  \phantomsection\label{symbol-arrow.tl.br}{{ ⤡ }
  \texttt{\ arrow.\ tl.\ br\ }}
\item
  \phantomsection\label{symbol-arrow.tr.bl}{{ ⤢ }
  \texttt{\ arrow.\ tr.\ bl\ }}
\item
  \phantomsection\label{symbol-arrow.ccw}{{ ↺ }
  \texttt{\ arrow.\ ccw\ }}
\item
  \phantomsection\label{symbol-arrow.ccw.half}{{ ↶ }
  \texttt{\ arrow.\ ccw.\ half\ }}
\item
  \phantomsection\label{symbol-arrow.cw}{{ ↻ }
  \texttt{\ arrow.\ cw\ }}
\item
  \phantomsection\label{symbol-arrow.cw.half}{{ ↷ }
  \texttt{\ arrow.\ cw.\ half\ }}
\item
  \phantomsection\label{symbol-arrow.zigzag}{{ ↯ }
  \texttt{\ arrow.\ zigzag\ }}
\item
  \phantomsection\label{symbol-arrows.rr}{{ ⇉ }
  \texttt{\ arrows.\ rr\ }}
\item
  \phantomsection\label{symbol-arrows.ll}{{ ⇇ }
  \texttt{\ arrows.\ ll\ }}
\item
  \phantomsection\label{symbol-arrows.tt}{{ ⇈ }
  \texttt{\ arrows.\ tt\ }}
\item
  \phantomsection\label{symbol-arrows.bb}{{ ⇊ }
  \texttt{\ arrows.\ bb\ }}
\item
  \phantomsection\label{symbol-arrows.lr}{{ ⇆ }
  \texttt{\ arrows.\ lr\ }}
\item
  \phantomsection\label{symbol-arrows.lr.stop}{{ ↹ }
  \texttt{\ arrows.\ lr.\ stop\ }}
\item
  \phantomsection\label{symbol-arrows.rl}{{ ⇄ }
  \texttt{\ arrows.\ rl\ }}
\item
  \phantomsection\label{symbol-arrows.tb}{{ ⇠}
  \texttt{\ arrows.\ tb\ }}
\item
  \phantomsection\label{symbol-arrows.bt}{{ ⇵ }
  \texttt{\ arrows.\ bt\ }}
\item
  \phantomsection\label{symbol-arrows.rrr}{{ ⇶ }
  \texttt{\ arrows.\ rrr\ }}
\item
  \phantomsection\label{symbol-arrows.lll}{{ ⬱ }
  \texttt{\ arrows.\ lll\ }}
\item
  \phantomsection\label{symbol-arrowhead.t}{{ ⌃ }
  \texttt{\ arrowhead.\ t\ }}
\item
  \phantomsection\label{symbol-arrowhead.b}{{ ⌄ }
  \texttt{\ arrowhead.\ b\ }}
\item
  \phantomsection\label{symbol-harpoon.rt}{{ ⇀ }
  \texttt{\ harpoon.\ rt\ }}
\item
  \phantomsection\label{symbol-harpoon.rt.bar}{{ ⥛ }
  \texttt{\ harpoon.\ rt.\ bar\ }}
\item
  \phantomsection\label{symbol-harpoon.rt.stop}{{ â¥`` }
  \texttt{\ harpoon.\ rt.\ stop\ }}
\item
  \phantomsection\label{symbol-harpoon.rb}{{ � }
  \texttt{\ harpoon.\ rb\ }}
\item
  \phantomsection\label{symbol-harpoon.rb.bar}{{ ⥟ }
  \texttt{\ harpoon.\ rb.\ bar\ }}
\item
  \phantomsection\label{symbol-harpoon.rb.stop}{{ â¥--- }
  \texttt{\ harpoon.\ rb.\ stop\ }}
\item
  \phantomsection\label{symbol-harpoon.lt}{{ ↼ }
  \texttt{\ harpoon.\ lt\ }}
\item
  \phantomsection\label{symbol-harpoon.lt.bar}{{ ⥚ }
  \texttt{\ harpoon.\ lt.\ bar\ }}
\item
  \phantomsection\label{symbol-harpoon.lt.stop}{{ â¥' }
  \texttt{\ harpoon.\ lt.\ stop\ }}
\item
  \phantomsection\label{symbol-harpoon.lb}{{ ↽ }
  \texttt{\ harpoon.\ lb\ }}
\item
  \phantomsection\label{symbol-harpoon.lb.bar}{{ ⥞ }
  \texttt{\ harpoon.\ lb.\ bar\ }}
\item
  \phantomsection\label{symbol-harpoon.lb.stop}{{ â¥-- }
  \texttt{\ harpoon.\ lb.\ stop\ }}
\item
  \phantomsection\label{symbol-harpoon.tl}{{ ↿ }
  \texttt{\ harpoon.\ tl\ }}
\item
  \phantomsection\label{symbol-harpoon.tl.bar}{{ ⥠}
  \texttt{\ harpoon.\ tl.\ bar\ }}
\item
  \phantomsection\label{symbol-harpoon.tl.stop}{{ ⥘ }
  \texttt{\ harpoon.\ tl.\ stop\ }}
\item
  \phantomsection\label{symbol-harpoon.tr}{{ ↾ }
  \texttt{\ harpoon.\ tr\ }}
\item
  \phantomsection\label{symbol-harpoon.tr.bar}{{ ⥜ }
  \texttt{\ harpoon.\ tr.\ bar\ }}
\item
  \phantomsection\label{symbol-harpoon.tr.stop}{{ â¥'' }
  \texttt{\ harpoon.\ tr.\ stop\ }}
\item
  \phantomsection\label{symbol-harpoon.bl}{{ ⇃ }
  \texttt{\ harpoon.\ bl\ }}
\item
  \phantomsection\label{symbol-harpoon.bl.bar}{{ ⥡ }
  \texttt{\ harpoon.\ bl.\ bar\ }}
\item
  \phantomsection\label{symbol-harpoon.bl.stop}{{ ⥙ }
  \texttt{\ harpoon.\ bl.\ stop\ }}
\item
  \phantomsection\label{symbol-harpoon.br}{{ ⇂ }
  \texttt{\ harpoon.\ br\ }}
\item
  \phantomsection\label{symbol-harpoon.br.bar}{{ � }
  \texttt{\ harpoon.\ br.\ bar\ }}
\item
  \phantomsection\label{symbol-harpoon.br.stop}{{ ⥕ }
  \texttt{\ harpoon.\ br.\ stop\ }}
\item
  \phantomsection\label{symbol-harpoon.lt.rt}{{ ⥎ }
  \texttt{\ harpoon.\ lt.\ rt\ }}
\item
  \phantomsection\label{symbol-harpoon.lb.rb}{{ � }
  \texttt{\ harpoon.\ lb.\ rb\ }}
\item
  \phantomsection\label{symbol-harpoon.lb.rt}{{ ⥋ }
  \texttt{\ harpoon.\ lb.\ rt\ }}
\item
  \phantomsection\label{symbol-harpoon.lt.rb}{{ ⥊ }
  \texttt{\ harpoon.\ lt.\ rb\ }}
\item
  \phantomsection\label{symbol-harpoon.tl.bl}{{ â¥` }
  \texttt{\ harpoon.\ tl.\ bl\ }}
\item
  \phantomsection\label{symbol-harpoon.tr.br}{{ � }
  \texttt{\ harpoon.\ tr.\ br\ }}
\item
  \phantomsection\label{symbol-harpoon.tl.br}{{ � }
  \texttt{\ harpoon.\ tl.\ br\ }}
\item
  \phantomsection\label{symbol-harpoon.tr.bl}{{ ⥌ }
  \texttt{\ harpoon.\ tr.\ bl\ }}
\item
  \phantomsection\label{symbol-harpoons.rtrb}{{ ⥤ }
  \texttt{\ harpoons.\ rtrb\ }}
\item
  \phantomsection\label{symbol-harpoons.blbr}{{ ⥥ }
  \texttt{\ harpoons.\ blbr\ }}
\item
  \phantomsection\label{symbol-harpoons.bltr}{{ ⥯ }
  \texttt{\ harpoons.\ bltr\ }}
\item
  \phantomsection\label{symbol-harpoons.lbrb}{{ ⥧ }
  \texttt{\ harpoons.\ lbrb\ }}
\item
  \phantomsection\label{symbol-harpoons.ltlb}{{ ⥢ }
  \texttt{\ harpoons.\ ltlb\ }}
\item
  \phantomsection\label{symbol-harpoons.ltrb}{{ ⇋ }
  \texttt{\ harpoons.\ ltrb\ }}
\item
  \phantomsection\label{symbol-harpoons.ltrt}{{ ⥦ }
  \texttt{\ harpoons.\ ltrt\ }}
\item
  \phantomsection\label{symbol-harpoons.rblb}{{ ⥩ }
  \texttt{\ harpoons.\ rblb\ }}
\item
  \phantomsection\label{symbol-harpoons.rtlb}{{ ⇌ }
  \texttt{\ harpoons.\ rtlb\ }}
\item
  \phantomsection\label{symbol-harpoons.rtlt}{{ ⥨ }
  \texttt{\ harpoons.\ rtlt\ }}
\item
  \phantomsection\label{symbol-harpoons.tlbr}{{ ⥮ }
  \texttt{\ harpoons.\ tlbr\ }}
\item
  \phantomsection\label{symbol-harpoons.tltr}{{ ⥣ }
  \texttt{\ harpoons.\ tltr\ }}
\item
  \phantomsection\label{symbol-tack.r}{{ ⊢ } \texttt{\ tack.\ r\ }}
\item
  \phantomsection\label{symbol-tack.r.not}{{ ⊬ }
  \texttt{\ tack.\ r.\ not\ }}
\item
  \phantomsection\label{symbol-tack.r.long}{{ � }
  \texttt{\ tack.\ r.\ long\ }}
\item
  \phantomsection\label{symbol-tack.r.short}{{ ⊦ }
  \texttt{\ tack.\ r.\ short\ }}
\item
  \phantomsection\label{symbol-tack.r.double}{{ ⊨ }
  \texttt{\ tack.\ r.\ double\ }}
\item
  \phantomsection\label{symbol-tack.r.double.not}{{ ⊭ }
  \texttt{\ tack.\ r.\ double.\ not\ }}
\item
  \phantomsection\label{symbol-tack.l}{{ ⊣ } \texttt{\ tack.\ l\ }}
\item
  \phantomsection\label{symbol-tack.l.long}{{ ⟞ }
  \texttt{\ tack.\ l.\ long\ }}
\item
  \phantomsection\label{symbol-tack.l.short}{{ â«ž }
  \texttt{\ tack.\ l.\ short\ }}
\item
  \phantomsection\label{symbol-tack.l.double}{{ ⫤ }
  \texttt{\ tack.\ l.\ double\ }}
\item
  \phantomsection\label{symbol-tack.t}{{ ⊥ } \texttt{\ tack.\ t\ }}
\item
  \phantomsection\label{symbol-tack.t.big}{{ ⟘ }
  \texttt{\ tack.\ t.\ big\ }}
\item
  \phantomsection\label{symbol-tack.t.double}{{ â«« }
  \texttt{\ tack.\ t.\ double\ }}
\item
  \phantomsection\label{symbol-tack.t.short}{{ â« }
  \texttt{\ tack.\ t.\ short\ }}
\item
  \phantomsection\label{symbol-tack.b}{{ ⊤ } \texttt{\ tack.\ b\ }}
\item
  \phantomsection\label{symbol-tack.b.big}{{ ⟙ }
  \texttt{\ tack.\ b.\ big\ }}
\item
  \phantomsection\label{symbol-tack.b.double}{{ ⫪ }
  \texttt{\ tack.\ b.\ double\ }}
\item
  \phantomsection\label{symbol-tack.b.short}{{ â«Ÿ }
  \texttt{\ tack.\ b.\ short\ }}
\item
  \phantomsection\label{symbol-tack.l.r}{{ ⟛ }
  \texttt{\ tack.\ l.\ r\ }}
\item
  \phantomsection\label{symbol-alpha}{{ α } \texttt{\ alpha\ }}
\item
  \phantomsection\label{symbol-beta}{{ β } \texttt{\ beta\ }}
\item
  \phantomsection\label{symbol-beta.alt}{{ Ï? } \texttt{\ beta.\ alt\ }}
\item
  \phantomsection\label{symbol-chi}{{ χ } \texttt{\ chi\ }}
\item
  \phantomsection\label{symbol-delta}{{ δ } \texttt{\ delta\ }}
\item
  \phantomsection\label{symbol-epsilon}{{ ε } \texttt{\ epsilon\ }}
\item
  \phantomsection\label{symbol-epsilon.alt}{{ ϵ }
  \texttt{\ epsilon.\ alt\ }}
\item
  \phantomsection\label{symbol-eta}{{ η } \texttt{\ eta\ }}
\item
  \phantomsection\label{symbol-gamma}{{ γ } \texttt{\ gamma\ }}
\item
  \phantomsection\label{symbol-iota}{{ ι } \texttt{\ iota\ }}
\item
  \phantomsection\label{symbol-kai}{{ Ï--- } \texttt{\ kai\ }}
\item
  \phantomsection\label{symbol-kappa}{{ κ } \texttt{\ kappa\ }}
\item
  \phantomsection\label{symbol-kappa.alt}{{ Ï° }
  \texttt{\ kappa.\ alt\ }}
\item
  \phantomsection\label{symbol-lambda}{{ λ } \texttt{\ lambda\ }}
\item
  \phantomsection\label{symbol-mu}{{ μ } \texttt{\ mu\ }}
\item
  \phantomsection\label{symbol-nu}{{ ν } \texttt{\ nu\ }}
\item
  \phantomsection\label{symbol-ohm}{{ Ω } \texttt{\ ohm\ }}
\item
  \phantomsection\label{symbol-ohm.inv}{{ ℧ } \texttt{\ ohm.\ inv\ }}
\item
  \phantomsection\label{symbol-omega}{{ ω } \texttt{\ omega\ }}
\item
  \phantomsection\label{symbol-omicron}{{ ο } \texttt{\ omicron\ }}
\item
  \phantomsection\label{symbol-phi}{{ φ } \texttt{\ phi\ }}
\item
  \phantomsection\label{symbol-phi.alt}{{ Ï• } \texttt{\ phi.\ alt\ }}
\item
  \phantomsection\label{symbol-pi}{{ π } \texttt{\ pi\ }}
\item
  \phantomsection\label{symbol-pi.alt}{{ Ï-- } \texttt{\ pi.\ alt\ }}
\item
  \phantomsection\label{symbol-psi}{{ ψ } \texttt{\ psi\ }}
\item
  \phantomsection\label{symbol-rho}{{ Ï? } \texttt{\ rho\ }}
\item
  \phantomsection\label{symbol-rho.alt}{{ ϱ } \texttt{\ rho.\ alt\ }}
\item
  \phantomsection\label{symbol-sigma}{{ σ } \texttt{\ sigma\ }}
\item
  \phantomsection\label{symbol-sigma.alt}{{ Ï‚ }
  \texttt{\ sigma.\ alt\ }}
\item
  \phantomsection\label{symbol-tau}{{ Ï„ } \texttt{\ tau\ }}
\item
  \phantomsection\label{symbol-theta}{{ θ } \texttt{\ theta\ }}
\item
  \phantomsection\label{symbol-theta.alt}{{ Ï` }
  \texttt{\ theta.\ alt\ }}
\item
  \phantomsection\label{symbol-upsilon}{{ Ï } \texttt{\ upsilon\ }}
\item
  \phantomsection\label{symbol-xi}{{ ξ } \texttt{\ xi\ }}
\item
  \phantomsection\label{symbol-zeta}{{ ζ } \texttt{\ zeta\ }}
\item
  \phantomsection\label{symbol-Alpha}{{ Î` } \texttt{\ Alpha\ }}
\item
  \phantomsection\label{symbol-Beta}{{ Î' } \texttt{\ Beta\ }}
\item
  \phantomsection\label{symbol-Chi}{{ Χ } \texttt{\ Chi\ }}
\item
  \phantomsection\label{symbol-Delta}{{ Î'' } \texttt{\ Delta\ }}
\item
  \phantomsection\label{symbol-Epsilon}{{ Ε } \texttt{\ Epsilon\ }}
\item
  \phantomsection\label{symbol-Eta}{{ Î--- } \texttt{\ Eta\ }}
\item
  \phantomsection\label{symbol-Gamma}{{ Î`` } \texttt{\ Gamma\ }}
\item
  \phantomsection\label{symbol-Iota}{{ Ι } \texttt{\ Iota\ }}
\item
  \phantomsection\label{symbol-Kai}{{ Ï? } \texttt{\ Kai\ }}
\item
  \phantomsection\label{symbol-Kappa}{{ Κ } \texttt{\ Kappa\ }}
\item
  \phantomsection\label{symbol-Lambda}{{ Λ } \texttt{\ Lambda\ }}
\item
  \phantomsection\label{symbol-Mu}{{ Μ } \texttt{\ Mu\ }}
\item
  \phantomsection\label{symbol-Nu}{{ Î? } \texttt{\ Nu\ }}
\item
  \phantomsection\label{symbol-Omega}{{ Ω } \texttt{\ Omega\ }}
\item
  \phantomsection\label{symbol-Omicron}{{ Ο } \texttt{\ Omicron\ }}
\item
  \phantomsection\label{symbol-Phi}{{ Φ } \texttt{\ Phi\ }}
\item
  \phantomsection\label{symbol-Pi}{{ Î } \texttt{\ Pi\ }}
\item
  \phantomsection\label{symbol-Psi}{{ Ψ } \texttt{\ Psi\ }}
\item
  \phantomsection\label{symbol-Rho}{{ Ρ } \texttt{\ Rho\ }}
\item
  \phantomsection\label{symbol-Sigma}{{ Σ } \texttt{\ Sigma\ }}
\item
  \phantomsection\label{symbol-Tau}{{ Τ } \texttt{\ Tau\ }}
\item
  \phantomsection\label{symbol-Theta}{{ Θ } \texttt{\ Theta\ }}
\item
  \phantomsection\label{symbol-Upsilon}{{ Υ } \texttt{\ Upsilon\ }}
\item
  \phantomsection\label{symbol-Xi}{{ Ξ } \texttt{\ Xi\ }}
\item
  \phantomsection\label{symbol-Zeta}{{ Î-- } \texttt{\ Zeta\ }}
\item
  \phantomsection\label{symbol-aleph}{{ ×? } \texttt{\ aleph\ }}
\item
  \phantomsection\label{symbol-alef}{{ ×? } \texttt{\ alef\ }}
\item
  \phantomsection\label{symbol-beth}{{ ×` } \texttt{\ beth\ }}
\item
  \phantomsection\label{symbol-bet}{{ ×` } \texttt{\ bet\ }}
\item
  \phantomsection\label{symbol-gimmel}{{ ×' } \texttt{\ gimmel\ }}
\item
  \phantomsection\label{symbol-gimel}{{ ×' } \texttt{\ gimel\ }}
\item
  \phantomsection\label{symbol-daleth}{{ ×`` } \texttt{\ daleth\ }}
\item
  \phantomsection\label{symbol-dalet}{{ ×`` } \texttt{\ dalet\ }}
\item
  \phantomsection\label{symbol-shin}{{ ש } \texttt{\ shin\ }}
\item
  \phantomsection\label{symbol-AA}{{ ð?''¸ } \texttt{\ AA\ }}
\item
  \phantomsection\label{symbol-BB}{{ ð?''¹ } \texttt{\ BB\ }}
\item
  \phantomsection\label{symbol-CC}{{ â„‚ } \texttt{\ CC\ }}
\item
  \phantomsection\label{symbol-DD}{{ ð?''» } \texttt{\ DD\ }}
\item
  \phantomsection\label{symbol-EE}{{ ð?''¼ } \texttt{\ EE\ }}
\item
  \phantomsection\label{symbol-FF}{{ ð?''½ } \texttt{\ FF\ }}
\item
  \phantomsection\label{symbol-GG}{{ ð?''¾ } \texttt{\ GG\ }}
\item
  \phantomsection\label{symbol-HH}{{ â„? } \texttt{\ HH\ }}
\item
  \phantomsection\label{symbol-II}{{ ð?•€ } \texttt{\ II\ }}
\item
  \phantomsection\label{symbol-JJ}{{ ð?•? } \texttt{\ JJ\ }}
\item
  \phantomsection\label{symbol-KK}{{ ð?•‚ } \texttt{\ KK\ }}
\item
  \phantomsection\label{symbol-LL}{{ � } \texttt{\ LL\ }}
\item
  \phantomsection\label{symbol-MM}{{ ð?•„ } \texttt{\ MM\ }}
\item
  \phantomsection\label{symbol-NN}{{ â„• } \texttt{\ NN\ }}
\item
  \phantomsection\label{symbol-OO}{{ � } \texttt{\ OO\ }}
\item
  \phantomsection\label{symbol-PP}{{ â„™ } \texttt{\ PP\ }}
\item
  \phantomsection\label{symbol-QQ}{{ â„š } \texttt{\ QQ\ }}
\item
  \phantomsection\label{symbol-RR}{{ â„? } \texttt{\ RR\ }}
\item
  \phantomsection\label{symbol-SS}{{ ð?•Š } \texttt{\ SS\ }}
\item
  \phantomsection\label{symbol-TT}{{ ð?•‹ } \texttt{\ TT\ }}
\item
  \phantomsection\label{symbol-UU}{{ � } \texttt{\ UU\ }}
\item
  \phantomsection\label{symbol-VV}{{ ð?•? } \texttt{\ VV\ }}
\item
  \phantomsection\label{symbol-WW}{{ ð?•Ž } \texttt{\ WW\ }}
\item
  \phantomsection\label{symbol-XX}{{ ð?•? } \texttt{\ XX\ }}
\item
  \phantomsection\label{symbol-YY}{{ ð?•? } \texttt{\ YY\ }}
\item
  \phantomsection\label{symbol-ZZ}{{ ℤ } \texttt{\ ZZ\ }}
\item
  \phantomsection\label{symbol-ell}{{ â„`` } \texttt{\ ell\ }}
\item
  \phantomsection\label{symbol-planck}{{ â„Ž } \texttt{\ planck\ }}
\item
  \phantomsection\label{symbol-planck.reduce}{{ â„? }
  \texttt{\ planck.\ reduce\ }}
\item
  \phantomsection\label{symbol-angstrom}{{ â„« } \texttt{\ angstrom\ }}
\item
  \phantomsection\label{symbol-kelvin}{{ K } \texttt{\ kelvin\ }}
\item
  \phantomsection\label{symbol-Re}{{ ℜ } \texttt{\ Re\ }}
\item
  \phantomsection\label{symbol-Im}{{ â„` } \texttt{\ Im\ }}
\item
  \phantomsection\label{symbol-dotless.i}{{ � }
  \texttt{\ dotless.\ i\ }}
\item
  \phantomsection\label{symbol-dotless.j}{{ � }
  \texttt{\ dotless.\ j\ }}
\end{itemize}

{ }

\subsubsection{\texorpdfstring{{ }}{ }}\label{section}

Name: \texttt{\ }
\includesvg[width=0.16667in,height=0.16667in]{/assets/icons/16-copy.svg}

Escape: \texttt{\ \textbackslash{}u\ \{\ }{\texttt{\ }}\texttt{\ \}\ }
\includesvg[width=0.16667in,height=0.16667in]{/assets/icons/16-copy.svg}

Shorthand: \texttt{\ }
\includesvg[width=0.16667in,height=0.16667in]{/assets/icons/16-copy.svg}
{ }

Accent:
\includesvg[width=0.16667in,height=0.16667in]{/assets/icons/16-close.svg}

LaTeX: \texttt{\ }

\paragraph{Variants}\label{variants}

{ }

\href{/docs/reference/symbols/}{\pandocbounded{\includesvg[keepaspectratio]{/assets/icons/16-arrow-right.svg}}}

{ Symbols } { Previous page }

\href{/docs/reference/symbols/emoji/}{\pandocbounded{\includesvg[keepaspectratio]{/assets/icons/16-arrow-right.svg}}}

{ Emoji } { Next page }




\section{C Docs LaTeX/docs/reference/introspection.tex}
\section{Docs LaTeX/typst.app/docs/reference/introspection/metadata.tex}
\title{typst.app/docs/reference/introspection/metadata}

\begin{itemize}
\tightlist
\item
  \href{/docs}{\includesvg[width=0.16667in,height=0.16667in]{/assets/icons/16-docs-dark.svg}}
\item
  \includesvg[width=0.16667in,height=0.16667in]{/assets/icons/16-arrow-right.svg}
\item
  \href{/docs/reference/}{Reference}
\item
  \includesvg[width=0.16667in,height=0.16667in]{/assets/icons/16-arrow-right.svg}
\item
  \href{/docs/reference/introspection/}{Introspection}
\item
  \includesvg[width=0.16667in,height=0.16667in]{/assets/icons/16-arrow-right.svg}
\item
  \href{/docs/reference/introspection/metadata/}{Metadata}
\end{itemize}

\section{\texorpdfstring{\texttt{\ metadata\ } {{ Element
}}}{ metadata   Element }}\label{summary}

\phantomsection\label{element-tooltip}
Element functions can be customized with \texttt{\ set\ } and
\texttt{\ show\ } rules.

Exposes a value to the query system without producing visible content.

This element can be retrieved with the
\href{/docs/reference/introspection/query/}{\texttt{\ query\ }} function
and from the command line with
\href{/docs/reference/introspection/query/\#command-line-queries}{\texttt{\ typst\ query\ }}
. Its purpose is to expose an arbitrary value to the introspection
system. To identify a metadata value among others, you can attach a
\href{/docs/reference/foundations/label/}{\texttt{\ label\ }} to it and
query for that label.

The \texttt{\ metadata\ } element is especially useful for command line
queries because it allows you to expose arbitrary values to the outside
world.

\begin{verbatim}
// Put metadata somewhere.
#metadata("This is a note") <note>

// And find it from anywhere else.
#context {
  query(<note>).first().value
}
\end{verbatim}

\includegraphics[width=5in,height=\textheight,keepaspectratio]{/assets/docs/sbF_Ac863-gI1m3qoL9avwAAAAAAAAAA.png}

\subsection{\texorpdfstring{{ Parameters
}}{ Parameters }}\label{parameters}

\phantomsection\label{parameters-tooltip}
Parameters are the inputs to a function. They are specified in
parentheses after the function name.

{ metadata } (

{ { any } }

) -\textgreater{} \href{/docs/reference/foundations/content/}{content}

\subsubsection{\texorpdfstring{\texttt{\ value\ }}{ value }}\label{parameters-value}

{ any }

{Required} {{ Positional }}

\phantomsection\label{parameters-value-positional-tooltip}
Positional parameters are specified in order, without names.

The value to embed into the document.

\href{/docs/reference/introspection/location/}{\pandocbounded{\includesvg[keepaspectratio]{/assets/icons/16-arrow-right.svg}}}

{ Location } { Previous page }

\href{/docs/reference/introspection/query/}{\pandocbounded{\includesvg[keepaspectratio]{/assets/icons/16-arrow-right.svg}}}

{ Query } { Next page }


\section{Docs LaTeX/typst.app/docs/reference/introspection/counter.tex}
\title{typst.app/docs/reference/introspection/counter}

\begin{itemize}
\tightlist
\item
  \href{/docs}{\includesvg[width=0.16667in,height=0.16667in]{/assets/icons/16-docs-dark.svg}}
\item
  \includesvg[width=0.16667in,height=0.16667in]{/assets/icons/16-arrow-right.svg}
\item
  \href{/docs/reference/}{Reference}
\item
  \includesvg[width=0.16667in,height=0.16667in]{/assets/icons/16-arrow-right.svg}
\item
  \href{/docs/reference/introspection/}{Introspection}
\item
  \includesvg[width=0.16667in,height=0.16667in]{/assets/icons/16-arrow-right.svg}
\item
  \href{/docs/reference/introspection/counter/}{Counter}
\end{itemize}

\section{\texorpdfstring{{ counter }}{ counter }}\label{summary}

Counts through pages, elements, and more.

With the counter function, you can access and modify counters for pages,
headings, figures, and more. Moreover, you can define custom counters
for other things you want to count.

Since counters change throughout the course of the document, their
current value is \emph{contextual.} It is recommended to read the
chapter on \href{/docs/reference/context/}{context} before continuing
here.

\subsection{Accessing a counter}\label{accessing}

To access the raw value of a counter, we can use the
\href{/docs/reference/introspection/counter/\#definitions-get}{\texttt{\ get\ }}
function. This function returns an
\href{/docs/reference/foundations/array/}{array} : Counters can have
multiple levels (in the case of headings for sections, subsections, and
so on), and each item in the array corresponds to one level.

\begin{verbatim}
#set heading(numbering: "1.")

= Introduction
Raw value of heading counter is
#context counter(heading).get()
\end{verbatim}

\includegraphics[width=5in,height=\textheight,keepaspectratio]{/assets/docs/jqVSznl_yGBcNN9ecF8OVAAAAAAAAAAA.png}

\subsection{Displaying a counter}\label{displaying}

Often, we want to display the value of a counter in a more
human-readable way. To do that, we can call the
\href{/docs/reference/introspection/counter/\#definitions-display}{\texttt{\ display\ }}
function on the counter. This function retrieves the current counter
value and formats it either with a provided or with an automatically
inferred \href{/docs/reference/model/numbering/}{numbering} .

\begin{verbatim}
#set heading(numbering: "1.")

= Introduction
Some text here.

= Background
The current value is: #context {
  counter(heading).display()
}

Or in roman numerals: #context {
  counter(heading).display("I")
}
\end{verbatim}

\includegraphics[width=5in,height=\textheight,keepaspectratio]{/assets/docs/7EUi61p1PXmzQyka_2NqiAAAAAAAAAAA.png}

\subsection{Modifying a counter}\label{modifying}

To modify a counter, you can use the \texttt{\ step\ } and
\texttt{\ update\ } methods:

\begin{itemize}
\item
  The \texttt{\ step\ } method increases the value of the counter by
  one. Because counters can have multiple levels , it optionally takes a
  \texttt{\ level\ } argument. If given, the counter steps at the given
  depth.
\item
  The \texttt{\ update\ } method allows you to arbitrarily modify the
  counter. In its basic form, you give it an integer (or an array for
  multiple levels). For more flexibility, you can instead also give it a
  function that receives the current value and returns a new value.
\end{itemize}

The heading counter is stepped before the heading is displayed, so
\texttt{\ Analysis\ } gets the number seven even though the counter is
at six after the second update.

\begin{verbatim}
#set heading(numbering: "1.")

= Introduction
#counter(heading).step()

= Background
#counter(heading).update(3)
#counter(heading).update(n => n * 2)

= Analysis
Let's skip 7.1.
#counter(heading).step(level: 2)

== Analysis
Still at #context {
  counter(heading).display()
}
\end{verbatim}

\includegraphics[width=5in,height=\textheight,keepaspectratio]{/assets/docs/EOYqv5YWVpmiQyBJoYpqQAAAAAAAAAAA.png}

\subsection{Page counter}\label{page-counter}

The page counter is special. It is automatically stepped at each
pagebreak. But like other counters, you can also step it manually. For
example, you could have Roman page numbers for your preface, then switch
to Arabic page numbers for your main content and reset the page counter
to one.

\begin{verbatim}
#set page(numbering: "(i)")

= Preface
The preface is numbered with
roman numerals.

#set page(numbering: "1 / 1")
#counter(page).update(1)

= Main text
Here, the counter is reset to one.
We also display both the current
page and total number of pages in
Arabic numbers.
\end{verbatim}

\includegraphics[width=5in,height=\textheight,keepaspectratio]{/assets/docs/PDCorO6nPZEoa3HjHUVgRwAAAAAAAAAA.png}
\includegraphics[width=5in,height=\textheight,keepaspectratio]{/assets/docs/PDCorO6nPZEoa3HjHUVgRwAAAAAAAAAB.png}

\subsection{Custom counters}\label{custom-counters}

To define your own counter, call the \texttt{\ counter\ } function with
a string as a key. This key identifies the counter globally.

\begin{verbatim}
#let mine = counter("mycounter")
#context mine.display() \
#mine.step()
#context mine.display() \
#mine.update(c => c * 3)
#context mine.display()
\end{verbatim}

\includegraphics[width=5in,height=\textheight,keepaspectratio]{/assets/docs/CxXLMyCvJp2FnmacPN3WUgAAAAAAAAAA.png}

\subsection{How to step}\label{how-to-step}

When you define and use a custom counter, in general, you should first
step the counter and then display it. This way, the stepping behaviour
of a counter can depend on the element it is stepped for. If you were
writing a counter for, let\textquotesingle s say, theorems, your
theorem\textquotesingle s definition would thus first include the
counter step and only then display the counter and the
theorem\textquotesingle s contents.

\begin{verbatim}
#let c = counter("theorem")
#let theorem(it) = block[
  #c.step()
  *Theorem #context c.display():*
  #it
]

#theorem[$1 = 1$]
#theorem[$2 < 3$]
\end{verbatim}

\includegraphics[width=5in,height=\textheight,keepaspectratio]{/assets/docs/af6Y7nOR_IldvYHIWDmkIQAAAAAAAAAA.png}

The rationale behind this is best explained on the example of the
heading counter: An update to the heading counter depends on the
heading\textquotesingle s level. By stepping directly before the
heading, we can correctly step from \texttt{\ 1\ } to \texttt{\ 1.1\ }
when encountering a level 2 heading. If we were to step after the
heading, we wouldn\textquotesingle t know what to step to.

Because counters should always be stepped before the elements they
count, they always start at zero. This way, they are at one for the
first display (which happens after the first step).

\subsection{Time travel}\label{time-travel}

Counters can travel through time! You can find out the final value of
the counter before it is reached and even determine what the value was
at any particular location in the document.

\begin{verbatim}
#let mine = counter("mycounter")

= Values
#context [
  Value here: #mine.get() \
  At intro: #mine.at(<intro>) \
  Final value: #mine.final()
]

#mine.update(n => n + 3)

= Introduction <intro>
#lorem(10)

#mine.step()
#mine.step()
\end{verbatim}

\includegraphics[width=5in,height=\textheight,keepaspectratio]{/assets/docs/wodRGpSsJgDZtfsMk_GNgwAAAAAAAAAA.png}

\subsection{Other kinds of state}\label{other-state}

The \texttt{\ counter\ } type is closely related to
\href{/docs/reference/introspection/state/}{state} type. Read its
documentation for more details on state management in Typst and why it
doesn\textquotesingle t just use normal variables for counters.

\subsection{\texorpdfstring{Constructor
{}}{Constructor }}\label{constructor}

\phantomsection\label{constructor-constructor-tooltip}
If a type has a constructor, you can call it like a function to create a
new value of the type.

Create a new counter identified by a key.

{ counter } (

{ \href{/docs/reference/foundations/str/}{str}
\href{/docs/reference/foundations/label/}{label}
\href{/docs/reference/foundations/selector/}{selector}
\href{/docs/reference/introspection/location/}{location}
\href{/docs/reference/foundations/function/}{function} }

) -\textgreater{} \href{/docs/reference/introspection/counter/}{counter}

\paragraph{\texorpdfstring{\texttt{\ key\ }}{ key }}\label{constructor-key}

\href{/docs/reference/foundations/str/}{str} {or}
\href{/docs/reference/foundations/label/}{label} {or}
\href{/docs/reference/foundations/selector/}{selector} {or}
\href{/docs/reference/introspection/location/}{location} {or}
\href{/docs/reference/foundations/function/}{function}

{Required} {{ Positional }}

\phantomsection\label{constructor-key-positional-tooltip}
Positional parameters are specified in order, without names.

The key that identifies this counter.

\begin{itemize}
\tightlist
\item
  If it is a string, creates a custom counter that is only affected by
  manual updates,
\item
  If it is the \href{/docs/reference/layout/page/}{\texttt{\ page\ }}
  function, counts through pages,
\item
  If it is a \href{/docs/reference/foundations/selector/}{selector} ,
  counts through elements that matches with the selector. For example,

  \begin{itemize}
  \tightlist
  \item
    provide an element function: counts elements of that type,
  \item
    provide a
    \href{/docs/reference/foundations/label/}{\texttt{\ }{\texttt{\ \textless{}label\textgreater{}\ }}\texttt{\ }}
    : counts elements with that label.
  \end{itemize}
\end{itemize}

\subsection{\texorpdfstring{{ Definitions
}}{ Definitions }}\label{definitions}

\phantomsection\label{definitions-tooltip}
Functions and types and can have associated definitions. These are
accessed by specifying the function or type, followed by a period, and
then the definition\textquotesingle s name.

\subsubsection{\texorpdfstring{\texttt{\ get\ } {{ Contextual
}}}{ get   Contextual }}\label{definitions-get}

\phantomsection\label{definitions-get-contextual-tooltip}
Contextual functions can only be used when the context is known

Retrieves the value of the counter at the current location. Always
returns an array of integers, even if the counter has just one number.

This is equivalent to
\texttt{\ counter\ }{\texttt{\ .\ }}\texttt{\ }{\texttt{\ at\ }}\texttt{\ }{\texttt{\ (\ }}\texttt{\ }{\texttt{\ here\ }}\texttt{\ }{\texttt{\ (\ }}\texttt{\ }{\texttt{\ )\ }}\texttt{\ }{\texttt{\ )\ }}\texttt{\ }
.

self { . } { get } (

) -\textgreater{} \href{/docs/reference/foundations/int/}{int}
\href{/docs/reference/foundations/array/}{array}

\subsubsection{\texorpdfstring{\texttt{\ display\ } {{ Contextual
}}}{ display   Contextual }}\label{definitions-display}

\phantomsection\label{definitions-display-contextual-tooltip}
Contextual functions can only be used when the context is known

Displays the current value of the counter with a numbering and returns
the formatted output.

\emph{Compatibility:} For compatibility with Typst 0.10 and lower, this
function also works without an established context. Then, it will create
opaque contextual content rather than directly returning the output of
the numbering. This behaviour will be removed in a future release.

self { . } { display } (

{ \href{/docs/reference/foundations/auto/}{auto}
\href{/docs/reference/foundations/str/}{str}
\href{/docs/reference/foundations/function/}{function} , } {
\hyperref[definitions-display-parameters-both]{both :}
\href{/docs/reference/foundations/bool/}{bool} , }

) -\textgreater{} { any }

\paragraph{\texorpdfstring{\texttt{\ numbering\ }}{ numbering }}\label{definitions-display-numbering}

\href{/docs/reference/foundations/auto/}{auto} {or}
\href{/docs/reference/foundations/str/}{str} {or}
\href{/docs/reference/foundations/function/}{function}

{{ Positional }}

\phantomsection\label{definitions-display-numbering-positional-tooltip}
Positional parameters are specified in order, without names.

A \href{/docs/reference/model/numbering/}{numbering pattern or a
function} , which specifies how to display the counter. If given a
function, that function receives each number of the counter as a
separate argument. If the amount of numbers varies, e.g. for the heading
argument, you can use an
\href{/docs/reference/foundations/arguments/}{argument sink} .

If this is omitted or set to \texttt{\ }{\texttt{\ auto\ }}\texttt{\ } ,
displays the counter with the numbering style for the counted element or
with the pattern \texttt{\ }{\texttt{\ "1.1"\ }}\texttt{\ } if no such
style exists.

Default: \texttt{\ }{\texttt{\ auto\ }}\texttt{\ }

\paragraph{\texorpdfstring{\texttt{\ both\ }}{ both }}\label{definitions-display-both}

\href{/docs/reference/foundations/bool/}{bool}

If enabled, displays the current and final top-level count together.
Both can be styled through a single numbering pattern. This is used by
the page numbering property to display the current and total number of
pages when a pattern like \texttt{\ }{\texttt{\ "1\ /\ 1"\ }}\texttt{\ }
is given.

Default: \texttt{\ }{\texttt{\ false\ }}\texttt{\ }

\subsubsection{\texorpdfstring{\texttt{\ at\ } {{ Contextual
}}}{ at   Contextual }}\label{definitions-at}

\phantomsection\label{definitions-at-contextual-tooltip}
Contextual functions can only be used when the context is known

Retrieves the value of the counter at the given location. Always returns
an array of integers, even if the counter has just one number.

The \texttt{\ selector\ } must match exactly one element in the
document. The most useful kinds of selectors for this are
\href{/docs/reference/foundations/label/}{labels} and
\href{/docs/reference/introspection/location/}{locations} .

\emph{Compatibility:} For compatibility with Typst 0.10 and lower, this
function also works without a known context if the \texttt{\ selector\ }
is a location. This behaviour will be removed in a future release.

self { . } { at } (

{ \href{/docs/reference/foundations/label/}{label}
\href{/docs/reference/foundations/selector/}{selector}
\href{/docs/reference/introspection/location/}{location}
\href{/docs/reference/foundations/function/}{function} }

) -\textgreater{} \href{/docs/reference/foundations/int/}{int}
\href{/docs/reference/foundations/array/}{array}

\paragraph{\texorpdfstring{\texttt{\ selector\ }}{ selector }}\label{definitions-at-selector}

\href{/docs/reference/foundations/label/}{label} {or}
\href{/docs/reference/foundations/selector/}{selector} {or}
\href{/docs/reference/introspection/location/}{location} {or}
\href{/docs/reference/foundations/function/}{function}

{Required} {{ Positional }}

\phantomsection\label{definitions-at-selector-positional-tooltip}
Positional parameters are specified in order, without names.

The place at which the counter\textquotesingle s value should be
retrieved.

\subsubsection{\texorpdfstring{\texttt{\ final\ } {{ Contextual
}}}{ final   Contextual }}\label{definitions-final}

\phantomsection\label{definitions-final-contextual-tooltip}
Contextual functions can only be used when the context is known

Retrieves the value of the counter at the end of the document. Always
returns an array of integers, even if the counter has just one number.

self { . } { final } (

{ \href{/docs/reference/foundations/none/}{none}
\href{/docs/reference/introspection/location/}{location} }

) -\textgreater{} \href{/docs/reference/foundations/int/}{int}
\href{/docs/reference/foundations/array/}{array}

\paragraph{\texorpdfstring{\texttt{\ location\ }}{ location }}\label{definitions-final-location}

\href{/docs/reference/foundations/none/}{none} {or}
\href{/docs/reference/introspection/location/}{location}

{{ Positional }}

\phantomsection\label{definitions-final-location-positional-tooltip}
Positional parameters are specified in order, without names.

\emph{Compatibility:} This argument is deprecated. It only exists for
compatibility with Typst 0.10 and lower and shouldn\textquotesingle t be
used anymore.

Default: \texttt{\ }{\texttt{\ none\ }}\texttt{\ }

\subsubsection{\texorpdfstring{\texttt{\ step\ }}{ step }}\label{definitions-step}

Increases the value of the counter by one.

The update will be in effect at the position where the returned content
is inserted into the document. If you don\textquotesingle t put the
output into the document, nothing happens! This would be the case, for
example, if you write
\texttt{\ }{\texttt{\ let\ }}\texttt{\ \_\ }{\texttt{\ =\ }}\texttt{\ }{\texttt{\ counter\ }}\texttt{\ }{\texttt{\ (\ }}\texttt{\ page\ }{\texttt{\ )\ }}\texttt{\ }{\texttt{\ .\ }}\texttt{\ }{\texttt{\ step\ }}\texttt{\ }{\texttt{\ (\ }}\texttt{\ }{\texttt{\ )\ }}\texttt{\ }
. Counter updates are always applied in layout order and in that case,
Typst wouldn\textquotesingle t know when to step the counter.

self { . } { step } (

{ \hyperref[definitions-step-parameters-level]{level :}
\href{/docs/reference/foundations/int/}{int} }

) -\textgreater{} \href{/docs/reference/foundations/content/}{content}

\paragraph{\texorpdfstring{\texttt{\ level\ }}{ level }}\label{definitions-step-level}

\href{/docs/reference/foundations/int/}{int}

The depth at which to step the counter. Defaults to
\texttt{\ }{\texttt{\ 1\ }}\texttt{\ } .

Default: \texttt{\ }{\texttt{\ 1\ }}\texttt{\ }

\subsubsection{\texorpdfstring{\texttt{\ update\ }}{ update }}\label{definitions-update}

Updates the value of the counter.

Just like with \texttt{\ step\ } , the update only occurs if you put the
resulting content into the document.

self { . } { update } (

{ \href{/docs/reference/foundations/int/}{int}
\href{/docs/reference/foundations/array/}{array}
\href{/docs/reference/foundations/function/}{function} }

) -\textgreater{} \href{/docs/reference/foundations/content/}{content}

\paragraph{\texorpdfstring{\texttt{\ update\ }}{ update }}\label{definitions-update-update}

\href{/docs/reference/foundations/int/}{int} {or}
\href{/docs/reference/foundations/array/}{array} {or}
\href{/docs/reference/foundations/function/}{function}

{Required} {{ Positional }}

\phantomsection\label{definitions-update-update-positional-tooltip}
Positional parameters are specified in order, without names.

If given an integer or array of integers, sets the counter to that
value. If given a function, that function receives the previous counter
value (with each number as a separate argument) and has to return the
new value (integer or array).

\href{/docs/reference/introspection/}{\pandocbounded{\includesvg[keepaspectratio]{/assets/icons/16-arrow-right.svg}}}

{ Introspection } { Previous page }

\href{/docs/reference/introspection/here/}{\pandocbounded{\includesvg[keepaspectratio]{/assets/icons/16-arrow-right.svg}}}

{ Here } { Next page }


\section{Docs LaTeX/typst.app/docs/reference/introspection/here.tex}
\title{typst.app/docs/reference/introspection/here}

\begin{itemize}
\tightlist
\item
  \href{/docs}{\includesvg[width=0.16667in,height=0.16667in]{/assets/icons/16-docs-dark.svg}}
\item
  \includesvg[width=0.16667in,height=0.16667in]{/assets/icons/16-arrow-right.svg}
\item
  \href{/docs/reference/}{Reference}
\item
  \includesvg[width=0.16667in,height=0.16667in]{/assets/icons/16-arrow-right.svg}
\item
  \href{/docs/reference/introspection/}{Introspection}
\item
  \includesvg[width=0.16667in,height=0.16667in]{/assets/icons/16-arrow-right.svg}
\item
  \href{/docs/reference/introspection/here/}{Here}
\end{itemize}

\section{\texorpdfstring{\texttt{\ here\ } {{ Contextual
}}}{ here   Contextual }}\label{summary}

\phantomsection\label{contextual-tooltip}
Contextual functions can only be used when the context is known

Provides the current location in the document.

You can think of \texttt{\ here\ } as a low-level building block that
directly extracts the current location from the active
\href{/docs/reference/context/}{context} . Some other functions use it
internally: For instance,
\texttt{\ counter\ }{\texttt{\ .\ }}\texttt{\ }{\texttt{\ get\ }}\texttt{\ }{\texttt{\ (\ }}\texttt{\ }{\texttt{\ )\ }}\texttt{\ }
is equivalent to
\texttt{\ counter\ }{\texttt{\ .\ }}\texttt{\ }{\texttt{\ at\ }}\texttt{\ }{\texttt{\ (\ }}\texttt{\ }{\texttt{\ here\ }}\texttt{\ }{\texttt{\ (\ }}\texttt{\ }{\texttt{\ )\ }}\texttt{\ }{\texttt{\ )\ }}\texttt{\ }
.

Within show rules on
\href{/docs/reference/introspection/location/\#locatable}{locatable}
elements,
\texttt{\ }{\texttt{\ here\ }}\texttt{\ }{\texttt{\ (\ }}\texttt{\ }{\texttt{\ )\ }}\texttt{\ }
will match the location of the shown element.

If you want to display the current page number, refer to the
documentation of the
\href{/docs/reference/introspection/counter/}{\texttt{\ counter\ }}
type. While \texttt{\ here\ } can be used to determine the physical page
number, typically you want the logical page number that may, for
instance, have been reset after a preface.

\subsection{Examples}\label{examples}

Determining the current position in the document in combination with the
\href{/docs/reference/introspection/location/\#definitions-position}{\texttt{\ position\ }}
method:

\begin{verbatim}
#context [
  I am located at
  #here().position()
]
\end{verbatim}

\includegraphics[width=5in,height=\textheight,keepaspectratio]{/assets/docs/5PrDc8FIHOrLs_qUjTj6iwAAAAAAAAAA.png}

Running a \href{/docs/reference/introspection/query/}{query} for
elements before the current position:

\begin{verbatim}
= Introduction
= Background

There are
#context query(
  selector(heading).before(here())
).len()
headings before me.

= Conclusion
\end{verbatim}

\includegraphics[width=5in,height=\textheight,keepaspectratio]{/assets/docs/5DWH6TcZBrEjuGuSwKqf8AAAAAAAAAAA.png}

Refer to the
\href{/docs/reference/foundations/selector/}{\texttt{\ selector\ }} type
for more details on before/after selectors.

\subsection{\texorpdfstring{{ Parameters
}}{ Parameters }}\label{parameters}

\phantomsection\label{parameters-tooltip}
Parameters are the inputs to a function. They are specified in
parentheses after the function name.

{ here } (

) -\textgreater{}
\href{/docs/reference/introspection/location/}{location}

\href{/docs/reference/introspection/counter/}{\pandocbounded{\includesvg[keepaspectratio]{/assets/icons/16-arrow-right.svg}}}

{ Counter } { Previous page }

\href{/docs/reference/introspection/locate/}{\pandocbounded{\includesvg[keepaspectratio]{/assets/icons/16-arrow-right.svg}}}

{ Locate } { Next page }


\section{Docs LaTeX/typst.app/docs/reference/introspection/locate.tex}
\title{typst.app/docs/reference/introspection/locate}

\begin{itemize}
\tightlist
\item
  \href{/docs}{\includesvg[width=0.16667in,height=0.16667in]{/assets/icons/16-docs-dark.svg}}
\item
  \includesvg[width=0.16667in,height=0.16667in]{/assets/icons/16-arrow-right.svg}
\item
  \href{/docs/reference/}{Reference}
\item
  \includesvg[width=0.16667in,height=0.16667in]{/assets/icons/16-arrow-right.svg}
\item
  \href{/docs/reference/introspection/}{Introspection}
\item
  \includesvg[width=0.16667in,height=0.16667in]{/assets/icons/16-arrow-right.svg}
\item
  \href{/docs/reference/introspection/locate/}{Locate}
\end{itemize}

\section{\texorpdfstring{\texttt{\ locate\ } {{ Contextual
}}}{ locate   Contextual }}\label{summary}

\phantomsection\label{contextual-tooltip}
Contextual functions can only be used when the context is known

Determines the location of an element in the document.

Takes a selector that must match exactly one element and returns that
element\textquotesingle s
\href{/docs/reference/introspection/location/}{\texttt{\ location\ }} .
This location can, in particular, be used to retrieve the physical
\href{/docs/reference/introspection/location/\#definitions-page}{\texttt{\ page\ }}
number and
\href{/docs/reference/introspection/location/\#definitions-position}{\texttt{\ position\ }}
(page, x, y) for that element.

\subsection{Examples}\label{examples}

Locating a specific element:

\begin{verbatim}
#context [
  Introduction is at: \
  #locate(<intro>).position()
]

= Introduction <intro>
\end{verbatim}

\includegraphics[width=5in,height=\textheight,keepaspectratio]{/assets/docs/fizxN7L7L7E8uWpTd8_mMgAAAAAAAAAA.png}

\subsection{Compatibility}\label{compatibility}

In Typst 0.10 and lower, the \texttt{\ locate\ } function took a closure
that made the current location in the document available (like
\href{/docs/reference/introspection/here/}{\texttt{\ here\ }} does now).
This usage pattern is deprecated. Compatibility with the old way will
remain for a while to give package authors time to upgrade. To that
effect, \texttt{\ locate\ } detects whether it received a selector or a
user-defined function and adjusts its semantics accordingly. This
behaviour will be removed in the future.

\subsection{\texorpdfstring{{ Parameters
}}{ Parameters }}\label{parameters}

\phantomsection\label{parameters-tooltip}
Parameters are the inputs to a function. They are specified in
parentheses after the function name.

{ locate } (

{ \href{/docs/reference/foundations/label/}{label}
\href{/docs/reference/foundations/selector/}{selector}
\href{/docs/reference/introspection/location/}{location}
\href{/docs/reference/foundations/function/}{function} }

) -\textgreater{} \href{/docs/reference/foundations/content/}{content}
\href{/docs/reference/introspection/location/}{location}

\subsubsection{\texorpdfstring{\texttt{\ selector\ }}{ selector }}\label{parameters-selector}

\href{/docs/reference/foundations/label/}{label} {or}
\href{/docs/reference/foundations/selector/}{selector} {or}
\href{/docs/reference/introspection/location/}{location} {or}
\href{/docs/reference/foundations/function/}{function}

{Required} {{ Positional }}

\phantomsection\label{parameters-selector-positional-tooltip}
Positional parameters are specified in order, without names.

A selector that should match exactly one element. This element will be
located.

Especially useful in combination with

\begin{itemize}
\tightlist
\item
  \href{/docs/reference/introspection/here/}{\texttt{\ here\ }} to
  locate the current context,
\item
  a
  \href{/docs/reference/introspection/location/}{\texttt{\ location\ }}
  retrieved from some queried element via the
  \href{/docs/reference/foundations/content/\#definitions-location}{\texttt{\ location()\ }}
  method on content.
\end{itemize}

\href{/docs/reference/introspection/here/}{\pandocbounded{\includesvg[keepaspectratio]{/assets/icons/16-arrow-right.svg}}}

{ Here } { Previous page }

\href{/docs/reference/introspection/location/}{\pandocbounded{\includesvg[keepaspectratio]{/assets/icons/16-arrow-right.svg}}}

{ Location } { Next page }


\section{Docs LaTeX/typst.app/docs/reference/introspection/query.tex}
\title{typst.app/docs/reference/introspection/query}

\begin{itemize}
\tightlist
\item
  \href{/docs}{\includesvg[width=0.16667in,height=0.16667in]{/assets/icons/16-docs-dark.svg}}
\item
  \includesvg[width=0.16667in,height=0.16667in]{/assets/icons/16-arrow-right.svg}
\item
  \href{/docs/reference/}{Reference}
\item
  \includesvg[width=0.16667in,height=0.16667in]{/assets/icons/16-arrow-right.svg}
\item
  \href{/docs/reference/introspection/}{Introspection}
\item
  \includesvg[width=0.16667in,height=0.16667in]{/assets/icons/16-arrow-right.svg}
\item
  \href{/docs/reference/introspection/query/}{Query}
\end{itemize}

\section{\texorpdfstring{\texttt{\ query\ } {{ Contextual
}}}{ query   Contextual }}\label{summary}

\phantomsection\label{contextual-tooltip}
Contextual functions can only be used when the context is known

Finds elements in the document.

The \texttt{\ query\ } functions lets you search your document for
elements of a particular type or with a particular label. To use it, you
first need to ensure that \href{/docs/reference/context/}{context} is
available.

\subsection{Finding elements}\label{finding-elements}

In the example below, we manually create a table of contents instead of
using the \href{/docs/reference/model/outline/}{\texttt{\ outline\ }}
function.

To do this, we first query for all headings in the document at level 1
and where \texttt{\ outlined\ } is true. Querying only for headings at
level 1 ensures that, for the purpose of this example, sub-headings are
not included in the table of contents. The \texttt{\ outlined\ } field
is used to exclude the "Table of Contents" heading itself.

Note that we open a \texttt{\ context\ } to be able to use the
\texttt{\ query\ } function.

\begin{verbatim}
#set page(numbering: "1")

#heading(outlined: false)[
  Table of Contents
]
#context {
  let chapters = query(
    heading.where(
      level: 1,
      outlined: true,
    )
  )
  for chapter in chapters {
    let loc = chapter.location()
    let nr = numbering(
      loc.page-numbering(),
      ..counter(page).at(loc),
    )
    [#chapter.body #h(1fr) #nr \ ]
  }
}

= Introduction
#lorem(10)
#pagebreak()

== Sub-Heading
#lorem(8)

= Discussion
#lorem(18)
\end{verbatim}

\includegraphics[width=5in,height=\textheight,keepaspectratio]{/assets/docs/jo-em7a3jFROfNLdVe33CwAAAAAAAAAA.png}
\includegraphics[width=5in,height=\textheight,keepaspectratio]{/assets/docs/jo-em7a3jFROfNLdVe33CwAAAAAAAAAB.png}

To get the page numbers, we first get the location of the elements
returned by \texttt{\ query\ } with
\href{/docs/reference/foundations/content/\#definitions-location}{\texttt{\ location\ }}
. We then also retrieve the
\href{/docs/reference/introspection/location/\#definitions-page-numbering}{page
numbering} and
\href{/docs/reference/introspection/counter/\#page-counter}{page
counter} at that location and apply the numbering to the counter.

\subsection{A word of caution}\label{caution}

To resolve all your queries, Typst evaluates and layouts parts of the
document multiple times. However, there is no guarantee that your
queries can actually be completely resolved. If you
aren\textquotesingle t careful a query can affect itselfâ€''leading to a
result that never stabilizes.

In the example below, we query for all headings in the document. We then
generate as many headings. In the beginning, there\textquotesingle s
just one heading, titled \texttt{\ Real\ } . Thus, \texttt{\ count\ } is
\texttt{\ 1\ } and one \texttt{\ Fake\ } heading is generated. Typst
sees that the query\textquotesingle s result has changed and processes
it again. This time, \texttt{\ count\ } is \texttt{\ 2\ } and two
\texttt{\ Fake\ } headings are generated. This goes on and on. As we can
see, the output has a finite amount of headings. This is because Typst
simply gives up after a few attempts.

In general, you should try not to write queries that affect themselves.
The same words of caution also apply to other introspection features
like \href{/docs/reference/introspection/counter/}{counters} and
\href{/docs/reference/introspection/state/}{state} .

\begin{verbatim}
= Real
#context {
  let elems = query(heading)
  let count = elems.len()
  count * [= Fake]
}
\end{verbatim}

\includegraphics[width=5in,height=\textheight,keepaspectratio]{/assets/docs/C2bjyzuukR06BSWIMgC89wAAAAAAAAAA.png}

\subsection{Command line queries}\label{command-line-queries}

You can also perform queries from the command line with the
\texttt{\ typst\ query\ } command. This command executes an arbitrary
query on the document and returns the resulting elements in serialized
form. Consider the following \texttt{\ example.typ\ } file which
contains some invisible
\href{/docs/reference/introspection/metadata/}{metadata} :

\begin{verbatim}
#metadata("This is a note") <note>
\end{verbatim}

You can execute a query on it as follows using Typst\textquotesingle s
CLI:

\begin{verbatim}
$ typst query example.typ "<note>"
[
  {
    "func": "metadata",
    "value": "This is a note",
    "label": "<note>"
  }
]
\end{verbatim}

Frequently, you\textquotesingle re interested in only one specific field
of the resulting elements. In the case of the \texttt{\ metadata\ }
element, the \texttt{\ value\ } field is the interesting one. You can
extract just this field with the \texttt{\ -\/-field\ } argument.

\begin{verbatim}
$ typst query example.typ "<note>" --field value
["This is a note"]
\end{verbatim}

If you are interested in just a single element, you can use the
\texttt{\ -\/-one\ } flag to extract just it.

\begin{verbatim}
$ typst query example.typ "<note>" --field value --one
"This is a note"
\end{verbatim}

\subsection{\texorpdfstring{{ Parameters
}}{ Parameters }}\label{parameters}

\phantomsection\label{parameters-tooltip}
Parameters are the inputs to a function. They are specified in
parentheses after the function name.

{ query } (

{ \href{/docs/reference/foundations/label/}{label}
\href{/docs/reference/foundations/selector/}{selector}
\href{/docs/reference/introspection/location/}{location}
\href{/docs/reference/foundations/function/}{function} , } {
\href{/docs/reference/foundations/none/}{none}
\href{/docs/reference/introspection/location/}{location} , }

) -\textgreater{} \href{/docs/reference/foundations/array/}{array}

\subsubsection{\texorpdfstring{\texttt{\ target\ }}{ target }}\label{parameters-target}

\href{/docs/reference/foundations/label/}{label} {or}
\href{/docs/reference/foundations/selector/}{selector} {or}
\href{/docs/reference/introspection/location/}{location} {or}
\href{/docs/reference/foundations/function/}{function}

{Required} {{ Positional }}

\phantomsection\label{parameters-target-positional-tooltip}
Positional parameters are specified in order, without names.

Can be

\begin{itemize}
\tightlist
\item
  an element function like a \texttt{\ heading\ } or \texttt{\ figure\ }
  ,
\item
  a \texttt{\ }{\texttt{\ \textless{}label\textgreater{}\ }}\texttt{\ }
  ,
\item
  a more complex selector like
  \texttt{\ heading\ }{\texttt{\ .\ }}\texttt{\ }{\texttt{\ where\ }}\texttt{\ }{\texttt{\ (\ }}\texttt{\ level\ }{\texttt{\ :\ }}\texttt{\ }{\texttt{\ 1\ }}\texttt{\ }{\texttt{\ )\ }}\texttt{\ }
  ,
\item
  or
  \texttt{\ }{\texttt{\ selector\ }}\texttt{\ }{\texttt{\ (\ }}\texttt{\ heading\ }{\texttt{\ )\ }}\texttt{\ }{\texttt{\ .\ }}\texttt{\ }{\texttt{\ before\ }}\texttt{\ }{\texttt{\ (\ }}\texttt{\ }{\texttt{\ here\ }}\texttt{\ }{\texttt{\ (\ }}\texttt{\ }{\texttt{\ )\ }}\texttt{\ }{\texttt{\ )\ }}\texttt{\ }
  .
\end{itemize}

Only
\href{/docs/reference/introspection/location/\#locatable}{locatable}
element functions are supported.

\subsubsection{\texorpdfstring{\texttt{\ location\ }}{ location }}\label{parameters-location}

\href{/docs/reference/foundations/none/}{none} {or}
\href{/docs/reference/introspection/location/}{location}

{{ Positional }}

\phantomsection\label{parameters-location-positional-tooltip}
Positional parameters are specified in order, without names.

\emph{Compatibility:} This argument is deprecated. It only exists for
compatibility with Typst 0.10 and lower and shouldn\textquotesingle t be
used anymore.

Default: \texttt{\ }{\texttt{\ none\ }}\texttt{\ }

\href{/docs/reference/introspection/metadata/}{\pandocbounded{\includesvg[keepaspectratio]{/assets/icons/16-arrow-right.svg}}}

{ Metadata } { Previous page }

\href{/docs/reference/introspection/state/}{\pandocbounded{\includesvg[keepaspectratio]{/assets/icons/16-arrow-right.svg}}}

{ State } { Next page }


\section{Docs LaTeX/typst.app/docs/reference/introspection/location.tex}
\title{typst.app/docs/reference/introspection/location}

\begin{itemize}
\tightlist
\item
  \href{/docs}{\includesvg[width=0.16667in,height=0.16667in]{/assets/icons/16-docs-dark.svg}}
\item
  \includesvg[width=0.16667in,height=0.16667in]{/assets/icons/16-arrow-right.svg}
\item
  \href{/docs/reference/}{Reference}
\item
  \includesvg[width=0.16667in,height=0.16667in]{/assets/icons/16-arrow-right.svg}
\item
  \href{/docs/reference/introspection/}{Introspection}
\item
  \includesvg[width=0.16667in,height=0.16667in]{/assets/icons/16-arrow-right.svg}
\item
  \href{/docs/reference/introspection/location/}{Location}
\end{itemize}

\section{\texorpdfstring{{ location }}{ location }}\label{summary}

Identifies an element in the document.

A location uniquely identifies an element in the document and lets you
access its absolute position on the pages. You can retrieve the current
location with the
\href{/docs/reference/introspection/here/}{\texttt{\ here\ }} function
and the location of a queried or shown element with the
\href{/docs/reference/foundations/content/\#definitions-location}{\texttt{\ location()\ }}
method on content.

\subsection{Locatable elements}\label{locatable}

Currently, only a subset of element functions is locatable. Aside from
headings and figures, this includes equations, references, quotes and
all elements with an explicit label. As a result, you \emph{can} query
for e.g. \href{/docs/reference/model/strong/}{\texttt{\ strong\ }}
elements, but you will find only those that have an explicit label
attached to them. This limitation will be resolved in the future.

\subsection{\texorpdfstring{{ Definitions
}}{ Definitions }}\label{definitions}

\phantomsection\label{definitions-tooltip}
Functions and types and can have associated definitions. These are
accessed by specifying the function or type, followed by a period, and
then the definition\textquotesingle s name.

\subsubsection{\texorpdfstring{\texttt{\ page\ }}{ page }}\label{definitions-page}

Returns the page number for this location.

Note that this does not return the value of the
\href{/docs/reference/introspection/counter/}{page counter} at this
location, but the true page number (starting from one).

If you want to know the value of the page counter, use
\texttt{\ }{\texttt{\ counter\ }}\texttt{\ }{\texttt{\ (\ }}\texttt{\ page\ }{\texttt{\ )\ }}\texttt{\ }{\texttt{\ .\ }}\texttt{\ }{\texttt{\ at\ }}\texttt{\ }{\texttt{\ (\ }}\texttt{\ loc\ }{\texttt{\ )\ }}\texttt{\ }
instead.

Can be used with
\href{/docs/reference/introspection/here/}{\texttt{\ here\ }} to
retrieve the physical page position of the current context:

self { . } { page } (

) -\textgreater{} \href{/docs/reference/foundations/int/}{int}

\begin{verbatim}
#context [
  I am located on
  page #here().page()
]
\end{verbatim}

\includegraphics[width=5in,height=\textheight,keepaspectratio]{/assets/docs/0ToVSLLUesTLkEw_YsnJkwAAAAAAAAAA.png}

\subsubsection{\texorpdfstring{\texttt{\ position\ }}{ position }}\label{definitions-position}

Returns a dictionary with the page number and the x, y position for this
location. The page number starts at one and the coordinates are measured
from the top-left of the page.

If you only need the page number, use \texttt{\ page()\ } instead as it
allows Typst to skip unnecessary work.

self { . } { position } (

) -\textgreater{}
\href{/docs/reference/foundations/dictionary/}{dictionary}

\subsubsection{\texorpdfstring{\texttt{\ page-numbering\ }}{ page-numbering }}\label{definitions-page-numbering}

Returns the page numbering pattern of the page at this location. This
can be used when displaying the page counter in order to obtain the
local numbering. This is useful if you are building custom indices or
outlines.

If the page numbering is set to
\texttt{\ }{\texttt{\ none\ }}\texttt{\ } at that location, this
function returns \texttt{\ }{\texttt{\ none\ }}\texttt{\ } .

self { . } { page-numbering } (

) -\textgreater{} \href{/docs/reference/foundations/none/}{none}
\href{/docs/reference/foundations/str/}{str}
\href{/docs/reference/foundations/function/}{function}

\href{/docs/reference/introspection/locate/}{\pandocbounded{\includesvg[keepaspectratio]{/assets/icons/16-arrow-right.svg}}}

{ Locate } { Previous page }

\href{/docs/reference/introspection/metadata/}{\pandocbounded{\includesvg[keepaspectratio]{/assets/icons/16-arrow-right.svg}}}

{ Metadata } { Next page }


\section{Docs LaTeX/typst.app/docs/reference/introspection/state.tex}
\title{typst.app/docs/reference/introspection/state}

\begin{itemize}
\tightlist
\item
  \href{/docs}{\includesvg[width=0.16667in,height=0.16667in]{/assets/icons/16-docs-dark.svg}}
\item
  \includesvg[width=0.16667in,height=0.16667in]{/assets/icons/16-arrow-right.svg}
\item
  \href{/docs/reference/}{Reference}
\item
  \includesvg[width=0.16667in,height=0.16667in]{/assets/icons/16-arrow-right.svg}
\item
  \href{/docs/reference/introspection/}{Introspection}
\item
  \includesvg[width=0.16667in,height=0.16667in]{/assets/icons/16-arrow-right.svg}
\item
  \href{/docs/reference/introspection/state/}{State}
\end{itemize}

\section{\texorpdfstring{{ state }}{ state }}\label{summary}

Manages stateful parts of your document.

Let\textquotesingle s say you have some computations in your document
and want to remember the result of your last computation to use it in
the next one. You might try something similar to the code below and
expect it to output 10, 13, 26, and 21. However this \textbf{does not
work} in Typst. If you test this code, you will see that Typst complains
with the following error message: \emph{Variables from outside the
function are read-only and cannot be modified.}

\begin{verbatim}
// This doesn't work!
#let x = 0
#let compute(expr) = {
  x = eval(
    expr.replace("x", str(x))
  )
  [New value is #x. ]
}

#compute("10") \
#compute("x + 3") \
#compute("x * 2") \
#compute("x - 5")
\end{verbatim}

\subsection{State and document markup}\label{state-and-markup}

Why does it do that? Because, in general, this kind of computation with
side effects is problematic in document markup and Typst is upfront
about that. For the results to make sense, the computation must proceed
in the same order in which the results will be laid out in the document.
In our simple example, that\textquotesingle s the case, but in general
it might not be.

Let\textquotesingle s look at a slightly different, but similar kind of
state: The heading numbering. We want to increase the heading counter at
each heading. Easy enough, right? Just add one. Well,
it\textquotesingle s not that simple. Consider the following example:

\begin{verbatim}
#set heading(numbering: "1.")
#let template(body) = [
  = Outline
  ...
  #body
]

#show: template

= Introduction
...
\end{verbatim}

\includegraphics[width=5in,height=\textheight,keepaspectratio]{/assets/docs/OC8Yphz4-mFQhH6Mm9lwwAAAAAAAAAAA.png}

Here, Typst first processes the body of the document after the show
rule, sees the \texttt{\ Introduction\ } heading, then passes the
resulting content to the \texttt{\ template\ } function and only then
sees the \texttt{\ Outline\ } . Just counting up would number the
\texttt{\ Introduction\ } with \texttt{\ 1\ } and the
\texttt{\ Outline\ } with \texttt{\ 2\ } .

\subsection{Managing state in Typst}\label{state-in-typst}

So what do we do instead? We use Typst\textquotesingle s state
management system. Calling the \texttt{\ state\ } function with an
identifying string key and an optional initial value gives you a state
value which exposes a few functions. The two most important ones are
\texttt{\ get\ } and \texttt{\ update\ } :

\begin{itemize}
\item
  The
  \href{/docs/reference/introspection/state/\#definitions-get}{\texttt{\ get\ }}
  function retrieves the current value of the state. Because the value
  can vary over the course of the document, it is a \emph{contextual}
  function that can only be used when
  \href{/docs/reference/context/}{context} is available.
\item
  The
  \href{/docs/reference/introspection/state/\#definitions-update}{\texttt{\ update\ }}
  function modifies the state. You can give it any value. If given a
  non-function value, it sets the state to that value. If given a
  function, that function receives the previous state and has to return
  the new state.
\end{itemize}

Our initial example would now look like this:

\begin{verbatim}
#let s = state("x", 0)
#let compute(expr) = [
  #s.update(x =>
    eval(expr.replace("x", str(x)))
  )
  New value is #context s.get().
]

#compute("10") \
#compute("x + 3") \
#compute("x * 2") \
#compute("x - 5")
\end{verbatim}

\includegraphics[width=5in,height=\textheight,keepaspectratio]{/assets/docs/TvB3cSxy6XwQVp0EXZ9-ewAAAAAAAAAA.png}

State managed by Typst is always updated in layout order, not in
evaluation order. The \texttt{\ update\ } method returns content and its
effect occurs at the position where the returned content is inserted
into the document.

As a result, we can now also store some of the computations in
variables, but they still show the correct results:

\begin{verbatim}
...

#let more = [
  #compute("x * 2") \
  #compute("x - 5")
]

#compute("10") \
#compute("x + 3") \
#more
\end{verbatim}

\includegraphics[width=5in,height=\textheight,keepaspectratio]{/assets/docs/leSHwxlkl8fBohZKt4lM4AAAAAAAAAAA.png}

This example is of course a bit silly, but in practice this is often
exactly what you want! A good example are heading counters, which is why
Typst\textquotesingle s
\href{/docs/reference/introspection/counter/}{counting system} is very
similar to its state system.

\subsection{Time Travel}\label{time-travel}

By using Typst\textquotesingle s state management system you also get
time travel capabilities! We can find out what the value of the state
will be at any position in the document from anywhere else. In
particular, the \texttt{\ at\ } method gives us the value of the state
at any particular location and the \texttt{\ final\ } methods gives us
the value of the state at the end of the document.

\begin{verbatim}
...

Value at `<here>` is
#context s.at(<here>)

#compute("10") \
#compute("x + 3") \
*Here.* <here> \
#compute("x * 2") \
#compute("x - 5")
\end{verbatim}

\includegraphics[width=5in,height=\textheight,keepaspectratio]{/assets/docs/FSbY2IZskPNKeQtPqbjroAAAAAAAAAAA.png}

\subsection{A word of caution}\label{caution}

To resolve the values of all states, Typst evaluates parts of your code
multiple times. However, there is no guarantee that your state
manipulation can actually be completely resolved.

For instance, if you generate state updates depending on the final value
of a state, the results might never converge. The example below
illustrates this. We initialize our state with \texttt{\ 1\ } and then
update it to its own final value plus 1. So it should be \texttt{\ 2\ }
, but then its final value is \texttt{\ 2\ } , so it should be
\texttt{\ 3\ } , and so on. This example displays a finite value because
Typst simply gives up after a few attempts.

\begin{verbatim}
// This is bad!
#let s = state("x", 1)
#context s.update(s.final() + 1)
#context s.get()
\end{verbatim}

\includegraphics[width=5in,height=\textheight,keepaspectratio]{/assets/docs/4ABrNAaHVbvzCF9JEmUebAAAAAAAAAAA.png}

In general, you should try not to generate state updates from within
context expressions. If possible, try to express your updates as
non-contextual values or functions that compute the new value from the
previous value. Sometimes, it cannot be helped, but in those cases it is
up to you to ensure that the result converges.

\subsection{\texorpdfstring{Constructor
{}}{Constructor }}\label{constructor}

\phantomsection\label{constructor-constructor-tooltip}
If a type has a constructor, you can call it like a function to create a
new value of the type.

Create a new state identified by a key.

{ state } (

{ \href{/docs/reference/foundations/str/}{str} , } { { any } , }

) -\textgreater{} \href{/docs/reference/introspection/state/}{state}

\paragraph{\texorpdfstring{\texttt{\ key\ }}{ key }}\label{constructor-key}

\href{/docs/reference/foundations/str/}{str}

{Required} {{ Positional }}

\phantomsection\label{constructor-key-positional-tooltip}
Positional parameters are specified in order, without names.

The key that identifies this state.

\paragraph{\texorpdfstring{\texttt{\ init\ }}{ init }}\label{constructor-init}

{ any }

{{ Positional }}

\phantomsection\label{constructor-init-positional-tooltip}
Positional parameters are specified in order, without names.

The initial value of the state.

Default: \texttt{\ }{\texttt{\ none\ }}\texttt{\ }

\subsection{\texorpdfstring{{ Definitions
}}{ Definitions }}\label{definitions}

\phantomsection\label{definitions-tooltip}
Functions and types and can have associated definitions. These are
accessed by specifying the function or type, followed by a period, and
then the definition\textquotesingle s name.

\subsubsection{\texorpdfstring{\texttt{\ get\ } {{ Contextual
}}}{ get   Contextual }}\label{definitions-get}

\phantomsection\label{definitions-get-contextual-tooltip}
Contextual functions can only be used when the context is known

Retrieves the value of the state at the current location.

This is equivalent to
\texttt{\ state\ }{\texttt{\ .\ }}\texttt{\ }{\texttt{\ at\ }}\texttt{\ }{\texttt{\ (\ }}\texttt{\ }{\texttt{\ here\ }}\texttt{\ }{\texttt{\ (\ }}\texttt{\ }{\texttt{\ )\ }}\texttt{\ }{\texttt{\ )\ }}\texttt{\ }
.

self { . } { get } (

) -\textgreater{} { any }

\subsubsection{\texorpdfstring{\texttt{\ at\ } {{ Contextual
}}}{ at   Contextual }}\label{definitions-at}

\phantomsection\label{definitions-at-contextual-tooltip}
Contextual functions can only be used when the context is known

Retrieves the value of the state at the given selector\textquotesingle s
unique match.

The \texttt{\ selector\ } must match exactly one element in the
document. The most useful kinds of selectors for this are
\href{/docs/reference/foundations/label/}{labels} and
\href{/docs/reference/introspection/location/}{locations} .

\emph{Compatibility:} For compatibility with Typst 0.10 and lower, this
function also works without a known context if the \texttt{\ selector\ }
is a location. This behaviour will be removed in a future release.

self { . } { at } (

{ \href{/docs/reference/foundations/label/}{label}
\href{/docs/reference/foundations/selector/}{selector}
\href{/docs/reference/introspection/location/}{location}
\href{/docs/reference/foundations/function/}{function} }

) -\textgreater{} { any }

\paragraph{\texorpdfstring{\texttt{\ selector\ }}{ selector }}\label{definitions-at-selector}

\href{/docs/reference/foundations/label/}{label} {or}
\href{/docs/reference/foundations/selector/}{selector} {or}
\href{/docs/reference/introspection/location/}{location} {or}
\href{/docs/reference/foundations/function/}{function}

{Required} {{ Positional }}

\phantomsection\label{definitions-at-selector-positional-tooltip}
Positional parameters are specified in order, without names.

The place at which the state\textquotesingle s value should be
retrieved.

\subsubsection{\texorpdfstring{\texttt{\ final\ } {{ Contextual
}}}{ final   Contextual }}\label{definitions-final}

\phantomsection\label{definitions-final-contextual-tooltip}
Contextual functions can only be used when the context is known

Retrieves the value of the state at the end of the document.

self { . } { final } (

{ \href{/docs/reference/foundations/none/}{none}
\href{/docs/reference/introspection/location/}{location} }

) -\textgreater{} { any }

\paragraph{\texorpdfstring{\texttt{\ location\ }}{ location }}\label{definitions-final-location}

\href{/docs/reference/foundations/none/}{none} {or}
\href{/docs/reference/introspection/location/}{location}

{{ Positional }}

\phantomsection\label{definitions-final-location-positional-tooltip}
Positional parameters are specified in order, without names.

\emph{Compatibility:} This argument is deprecated. It only exists for
compatibility with Typst 0.10 and lower and shouldn\textquotesingle t be
used anymore.

Default: \texttt{\ }{\texttt{\ none\ }}\texttt{\ }

\subsubsection{\texorpdfstring{\texttt{\ update\ }}{ update }}\label{definitions-update}

Update the value of the state.

The update will be in effect at the position where the returned content
is inserted into the document. If you don\textquotesingle t put the
output into the document, nothing happens! This would be the case, for
example, if you write
\texttt{\ }{\texttt{\ let\ }}\texttt{\ \_\ }{\texttt{\ =\ }}\texttt{\ }{\texttt{\ state\ }}\texttt{\ }{\texttt{\ (\ }}\texttt{\ }{\texttt{\ "key"\ }}\texttt{\ }{\texttt{\ )\ }}\texttt{\ }{\texttt{\ .\ }}\texttt{\ }{\texttt{\ update\ }}\texttt{\ }{\texttt{\ (\ }}\texttt{\ }{\texttt{\ 7\ }}\texttt{\ }{\texttt{\ )\ }}\texttt{\ }
. State updates are always applied in layout order and in that case,
Typst wouldn\textquotesingle t know when to update the state.

self { . } { update } (

{ { any } \href{/docs/reference/foundations/function/}{function} }

) -\textgreater{} \href{/docs/reference/foundations/content/}{content}

\paragraph{\texorpdfstring{\texttt{\ update\ }}{ update }}\label{definitions-update-update}

{ any } {or} \href{/docs/reference/foundations/function/}{function}

{Required} {{ Positional }}

\phantomsection\label{definitions-update-update-positional-tooltip}
Positional parameters are specified in order, without names.

If given a non function-value, sets the state to that value. If given a
function, that function receives the previous state and has to return
the new state.

\subsubsection{\texorpdfstring{\texttt{\ display\ }}{ display }}\label{definitions-display}

Displays the current value of the state.

\textbf{Deprecation planned:} Use
\href{/docs/reference/introspection/state/\#definitions-get}{\texttt{\ get\ }}
instead.

self { . } { display } (

{ \href{/docs/reference/foundations/none/}{none}
\href{/docs/reference/foundations/function/}{function} }

) -\textgreater{} \href{/docs/reference/foundations/content/}{content}

\paragraph{\texorpdfstring{\texttt{\ func\ }}{ func }}\label{definitions-display-func}

\href{/docs/reference/foundations/none/}{none} {or}
\href{/docs/reference/foundations/function/}{function}

{{ Positional }}

\phantomsection\label{definitions-display-func-positional-tooltip}
Positional parameters are specified in order, without names.

A function which receives the value of the state and can return
arbitrary content which is then displayed. If this is omitted, the value
is directly displayed.

Default: \texttt{\ }{\texttt{\ none\ }}\texttt{\ }

\href{/docs/reference/introspection/query/}{\pandocbounded{\includesvg[keepaspectratio]{/assets/icons/16-arrow-right.svg}}}

{ Query } { Previous page }

\href{/docs/reference/data-loading/}{\pandocbounded{\includesvg[keepaspectratio]{/assets/icons/16-arrow-right.svg}}}

{ Data Loading } { Next page }




\section{C Docs LaTeX/docs/reference/text.tex}
\section{Docs LaTeX/typst.app/docs/reference/text/linebreak.tex}
\title{typst.app/docs/reference/text/linebreak}

\begin{itemize}
\tightlist
\item
  \href{/docs}{\includesvg[width=0.16667in,height=0.16667in]{/assets/icons/16-docs-dark.svg}}
\item
  \includesvg[width=0.16667in,height=0.16667in]{/assets/icons/16-arrow-right.svg}
\item
  \href{/docs/reference/}{Reference}
\item
  \includesvg[width=0.16667in,height=0.16667in]{/assets/icons/16-arrow-right.svg}
\item
  \href{/docs/reference/text/}{Text}
\item
  \includesvg[width=0.16667in,height=0.16667in]{/assets/icons/16-arrow-right.svg}
\item
  \href{/docs/reference/text/linebreak/}{Line Break}
\end{itemize}

\section{\texorpdfstring{\texttt{\ linebreak\ } {{ Element
}}}{ linebreak   Element }}\label{summary}

\phantomsection\label{element-tooltip}
Element functions can be customized with \texttt{\ set\ } and
\texttt{\ show\ } rules.

Inserts a line break.

Advances the paragraph to the next line. A single trailing line break at
the end of a paragraph is ignored, but more than one creates additional
empty lines.

\subsection{Example}\label{example}

\begin{verbatim}
*Date:* 26.12.2022 \
*Topic:* Infrastructure Test \
*Severity:* High \
\end{verbatim}

\includegraphics[width=5in,height=\textheight,keepaspectratio]{/assets/docs/OEyyibskK4bIsTh7Qcp7OAAAAAAAAAAA.png}

\subsection{Syntax}\label{syntax}

This function also has dedicated syntax: To insert a line break, simply
write a backslash followed by whitespace. This always creates an
unjustified break.

\subsection{\texorpdfstring{{ Parameters
}}{ Parameters }}\label{parameters}

\phantomsection\label{parameters-tooltip}
Parameters are the inputs to a function. They are specified in
parentheses after the function name.

{ linebreak } (

{ \hyperref[parameters-justify]{justify :}
\href{/docs/reference/foundations/bool/}{bool} }

) -\textgreater{} \href{/docs/reference/foundations/content/}{content}

\subsubsection{\texorpdfstring{\texttt{\ justify\ }}{ justify }}\label{parameters-justify}

\href{/docs/reference/foundations/bool/}{bool}

{{ Settable }}

\phantomsection\label{parameters-justify-settable-tooltip}
Settable parameters can be customized for all following uses of the
function with a \texttt{\ set\ } rule.

Whether to justify the line before the break.

This is useful if you found a better line break opportunity in your
justified text than Typst did.

Default: \texttt{\ }{\texttt{\ false\ }}\texttt{\ }

\includesvg[width=0.16667in,height=0.16667in]{/assets/icons/16-arrow-right.svg}
View example

\begin{verbatim}
#set par(justify: true)
#let jb = linebreak(justify: true)

I have manually tuned the #jb
line breaks in this paragraph #jb
for an _interesting_ result. #jb
\end{verbatim}

\includegraphics[width=5in,height=\textheight,keepaspectratio]{/assets/docs/RlJnAEDPiPVRCZ7poOHTOwAAAAAAAAAA.png}

\href{/docs/reference/text/highlight/}{\pandocbounded{\includesvg[keepaspectratio]{/assets/icons/16-arrow-right.svg}}}

{ Highlight } { Previous page }

\href{/docs/reference/text/lorem/}{\pandocbounded{\includesvg[keepaspectratio]{/assets/icons/16-arrow-right.svg}}}

{ Lorem } { Next page }


\section{Docs LaTeX/typst.app/docs/reference/text/highlight.tex}
\title{typst.app/docs/reference/text/highlight}

\begin{itemize}
\tightlist
\item
  \href{/docs}{\includesvg[width=0.16667in,height=0.16667in]{/assets/icons/16-docs-dark.svg}}
\item
  \includesvg[width=0.16667in,height=0.16667in]{/assets/icons/16-arrow-right.svg}
\item
  \href{/docs/reference/}{Reference}
\item
  \includesvg[width=0.16667in,height=0.16667in]{/assets/icons/16-arrow-right.svg}
\item
  \href{/docs/reference/text/}{Text}
\item
  \includesvg[width=0.16667in,height=0.16667in]{/assets/icons/16-arrow-right.svg}
\item
  \href{/docs/reference/text/highlight/}{Highlight}
\end{itemize}

\section{\texorpdfstring{\texttt{\ highlight\ } {{ Element
}}}{ highlight   Element }}\label{summary}

\phantomsection\label{element-tooltip}
Element functions can be customized with \texttt{\ set\ } and
\texttt{\ show\ } rules.

Highlights text with a background color.

\subsection{Example}\label{example}

\begin{verbatim}
This is #highlight[important].
\end{verbatim}

\includegraphics[width=5in,height=\textheight,keepaspectratio]{/assets/docs/QtpA6ir9UWFHeXPRr2gD9AAAAAAAAAAA.png}

\subsection{\texorpdfstring{{ Parameters
}}{ Parameters }}\label{parameters}

\phantomsection\label{parameters-tooltip}
Parameters are the inputs to a function. They are specified in
parentheses after the function name.

{ highlight } (

{ \hyperref[parameters-fill]{fill :}
\href{/docs/reference/foundations/none/}{none}
\href{/docs/reference/visualize/color/}{color}
\href{/docs/reference/visualize/gradient/}{gradient}
\href{/docs/reference/visualize/pattern/}{pattern} , } {
\hyperref[parameters-stroke]{stroke :}
\href{/docs/reference/foundations/none/}{none}
\href{/docs/reference/layout/length/}{length}
\href{/docs/reference/visualize/color/}{color}
\href{/docs/reference/visualize/gradient/}{gradient}
\href{/docs/reference/visualize/stroke/}{stroke}
\href{/docs/reference/visualize/pattern/}{pattern}
\href{/docs/reference/foundations/dictionary/}{dictionary} , } {
\hyperref[parameters-top-edge]{top-edge :}
\href{/docs/reference/layout/length/}{length}
\href{/docs/reference/foundations/str/}{str} , } {
\hyperref[parameters-bottom-edge]{bottom-edge :}
\href{/docs/reference/layout/length/}{length}
\href{/docs/reference/foundations/str/}{str} , } {
\hyperref[parameters-extent]{extent :}
\href{/docs/reference/layout/length/}{length} , } {
\hyperref[parameters-radius]{radius :}
\href{/docs/reference/layout/relative/}{relative}
\href{/docs/reference/foundations/dictionary/}{dictionary} , } {
\href{/docs/reference/foundations/content/}{content} , }

) -\textgreater{} \href{/docs/reference/foundations/content/}{content}

\subsubsection{\texorpdfstring{\texttt{\ fill\ }}{ fill }}\label{parameters-fill}

\href{/docs/reference/foundations/none/}{none} {or}
\href{/docs/reference/visualize/color/}{color} {or}
\href{/docs/reference/visualize/gradient/}{gradient} {or}
\href{/docs/reference/visualize/pattern/}{pattern}

{{ Settable }}

\phantomsection\label{parameters-fill-settable-tooltip}
Settable parameters can be customized for all following uses of the
function with a \texttt{\ set\ } rule.

The color to highlight the text with.

Default:
\texttt{\ }{\texttt{\ rgb\ }}\texttt{\ }{\texttt{\ (\ }}\texttt{\ }{\texttt{\ "\#fffd11a1"\ }}\texttt{\ }{\texttt{\ )\ }}\texttt{\ }

\includesvg[width=0.16667in,height=0.16667in]{/assets/icons/16-arrow-right.svg}
View example

\begin{verbatim}
This is #highlight(
  fill: blue
)[highlighted with blue].
\end{verbatim}

\includegraphics[width=5in,height=\textheight,keepaspectratio]{/assets/docs/oW--DyYpfs3nP_lZOIl65gAAAAAAAAAA.png}

\subsubsection{\texorpdfstring{\texttt{\ stroke\ }}{ stroke }}\label{parameters-stroke}

\href{/docs/reference/foundations/none/}{none} {or}
\href{/docs/reference/layout/length/}{length} {or}
\href{/docs/reference/visualize/color/}{color} {or}
\href{/docs/reference/visualize/gradient/}{gradient} {or}
\href{/docs/reference/visualize/stroke/}{stroke} {or}
\href{/docs/reference/visualize/pattern/}{pattern} {or}
\href{/docs/reference/foundations/dictionary/}{dictionary}

{{ Settable }}

\phantomsection\label{parameters-stroke-settable-tooltip}
Settable parameters can be customized for all following uses of the
function with a \texttt{\ set\ } rule.

The highlight\textquotesingle s border color. See the
\href{/docs/reference/visualize/rect/\#parameters-stroke}{rectangle\textquotesingle s
documentation} for more details.

Default:
\texttt{\ }{\texttt{\ (\ }}\texttt{\ }{\texttt{\ :\ }}\texttt{\ }{\texttt{\ )\ }}\texttt{\ }

\includesvg[width=0.16667in,height=0.16667in]{/assets/icons/16-arrow-right.svg}
View example

\begin{verbatim}
This is a #highlight(
  stroke: fuchsia
)[stroked highlighting].
\end{verbatim}

\includegraphics[width=5in,height=\textheight,keepaspectratio]{/assets/docs/VdZlBJnkRgdzR_4L65zKzQAAAAAAAAAA.png}

\subsubsection{\texorpdfstring{\texttt{\ top-edge\ }}{ top-edge }}\label{parameters-top-edge}

\href{/docs/reference/layout/length/}{length} {or}
\href{/docs/reference/foundations/str/}{str}

{{ Settable }}

\phantomsection\label{parameters-top-edge-settable-tooltip}
Settable parameters can be customized for all following uses of the
function with a \texttt{\ set\ } rule.

The top end of the background rectangle.

\begin{longtable}[]{@{}ll@{}}
\toprule\noalign{}
Variant & Details \\
\midrule\noalign{}
\endhead
\bottomrule\noalign{}
\endlastfoot
\texttt{\ "\ ascender\ "\ } & The font\textquotesingle s ascender, which
typically exceeds the height of all glyphs. \\
\texttt{\ "\ cap-height\ "\ } & The approximate height of uppercase
letters. \\
\texttt{\ "\ x-height\ "\ } & The approximate height of non-ascending
lowercase letters. \\
\texttt{\ "\ baseline\ "\ } & The baseline on which the letters rest. \\
\texttt{\ "\ bounds\ "\ } & The top edge of the glyph\textquotesingle s
bounding box. \\
\end{longtable}

Default: \texttt{\ }{\texttt{\ "ascender"\ }}\texttt{\ }

\includesvg[width=0.16667in,height=0.16667in]{/assets/icons/16-arrow-right.svg}
View example

\begin{verbatim}
#set highlight(top-edge: "ascender")
#highlight[a] #highlight[aib]

#set highlight(top-edge: "x-height")
#highlight[a] #highlight[aib]
\end{verbatim}

\includegraphics[width=5in,height=\textheight,keepaspectratio]{/assets/docs/33w6KWiqvSrMq41iE5ob_QAAAAAAAAAA.png}

\subsubsection{\texorpdfstring{\texttt{\ bottom-edge\ }}{ bottom-edge }}\label{parameters-bottom-edge}

\href{/docs/reference/layout/length/}{length} {or}
\href{/docs/reference/foundations/str/}{str}

{{ Settable }}

\phantomsection\label{parameters-bottom-edge-settable-tooltip}
Settable parameters can be customized for all following uses of the
function with a \texttt{\ set\ } rule.

The bottom end of the background rectangle.

\begin{longtable}[]{@{}ll@{}}
\toprule\noalign{}
Variant & Details \\
\midrule\noalign{}
\endhead
\bottomrule\noalign{}
\endlastfoot
\texttt{\ "\ baseline\ "\ } & The baseline on which the letters rest. \\
\texttt{\ "\ descender\ "\ } & The font\textquotesingle s descender,
which typically exceeds the depth of all glyphs. \\
\texttt{\ "\ bounds\ "\ } & The bottom edge of the
glyph\textquotesingle s bounding box. \\
\end{longtable}

Default: \texttt{\ }{\texttt{\ "descender"\ }}\texttt{\ }

\includesvg[width=0.16667in,height=0.16667in]{/assets/icons/16-arrow-right.svg}
View example

\begin{verbatim}
#set highlight(bottom-edge: "descender")
#highlight[a] #highlight[ap]

#set highlight(bottom-edge: "baseline")
#highlight[a] #highlight[ap]
\end{verbatim}

\includegraphics[width=5in,height=\textheight,keepaspectratio]{/assets/docs/tZTem6RAQXJ8OFzIrL6AnAAAAAAAAAAA.png}

\subsubsection{\texorpdfstring{\texttt{\ extent\ }}{ extent }}\label{parameters-extent}

\href{/docs/reference/layout/length/}{length}

{{ Settable }}

\phantomsection\label{parameters-extent-settable-tooltip}
Settable parameters can be customized for all following uses of the
function with a \texttt{\ set\ } rule.

The amount by which to extend the background to the sides beyond (or
within if negative) the content.

Default: \texttt{\ }{\texttt{\ 0pt\ }}\texttt{\ }

\includesvg[width=0.16667in,height=0.16667in]{/assets/icons/16-arrow-right.svg}
View example

\begin{verbatim}
A long #highlight(extent: 4pt)[background].
\end{verbatim}

\includegraphics[width=5in,height=\textheight,keepaspectratio]{/assets/docs/L2wf2ozgvgMg2iI3FR9LdQAAAAAAAAAA.png}

\subsubsection{\texorpdfstring{\texttt{\ radius\ }}{ radius }}\label{parameters-radius}

\href{/docs/reference/layout/relative/}{relative} {or}
\href{/docs/reference/foundations/dictionary/}{dictionary}

{{ Settable }}

\phantomsection\label{parameters-radius-settable-tooltip}
Settable parameters can be customized for all following uses of the
function with a \texttt{\ set\ } rule.

How much to round the highlight\textquotesingle s corners. See the
\href{/docs/reference/visualize/rect/\#parameters-radius}{rectangle\textquotesingle s
documentation} for more details.

Default:
\texttt{\ }{\texttt{\ (\ }}\texttt{\ }{\texttt{\ :\ }}\texttt{\ }{\texttt{\ )\ }}\texttt{\ }

\includesvg[width=0.16667in,height=0.16667in]{/assets/icons/16-arrow-right.svg}
View example

\begin{verbatim}
Listen #highlight(
  radius: 5pt, extent: 2pt
)[carefully], it will be on the test.
\end{verbatim}

\includegraphics[width=5in,height=\textheight,keepaspectratio]{/assets/docs/MdD0cA7uGh7p2z32380_kAAAAAAAAAAA.png}

\subsubsection{\texorpdfstring{\texttt{\ body\ }}{ body }}\label{parameters-body}

\href{/docs/reference/foundations/content/}{content}

{Required} {{ Positional }}

\phantomsection\label{parameters-body-positional-tooltip}
Positional parameters are specified in order, without names.

The content that should be highlighted.

\href{/docs/reference/text/}{\pandocbounded{\includesvg[keepaspectratio]{/assets/icons/16-arrow-right.svg}}}

{ Text } { Previous page }

\href{/docs/reference/text/linebreak/}{\pandocbounded{\includesvg[keepaspectratio]{/assets/icons/16-arrow-right.svg}}}

{ Line Break } { Next page }


\section{Docs LaTeX/typst.app/docs/reference/text/smartquote.tex}
\title{typst.app/docs/reference/text/smartquote}

\begin{itemize}
\tightlist
\item
  \href{/docs}{\includesvg[width=0.16667in,height=0.16667in]{/assets/icons/16-docs-dark.svg}}
\item
  \includesvg[width=0.16667in,height=0.16667in]{/assets/icons/16-arrow-right.svg}
\item
  \href{/docs/reference/}{Reference}
\item
  \includesvg[width=0.16667in,height=0.16667in]{/assets/icons/16-arrow-right.svg}
\item
  \href{/docs/reference/text/}{Text}
\item
  \includesvg[width=0.16667in,height=0.16667in]{/assets/icons/16-arrow-right.svg}
\item
  \href{/docs/reference/text/smartquote/}{Smartquote}
\end{itemize}

\section{\texorpdfstring{\texttt{\ smartquote\ } {{ Element
}}}{ smartquote   Element }}\label{summary}

\phantomsection\label{element-tooltip}
Element functions can be customized with \texttt{\ set\ } and
\texttt{\ show\ } rules.

A language-aware quote that reacts to its context.

Automatically turns into an appropriate opening or closing quote based
on the active \href{/docs/reference/text/text/\#parameters-lang}{text
language} .

\subsection{Example}\label{example}

\begin{verbatim}
"This is in quotes."

#set text(lang: "de")
"Das ist in Anführungszeichen."

#set text(lang: "fr")
"C'est entre guillemets."
\end{verbatim}

\includegraphics[width=5in,height=\textheight,keepaspectratio]{/assets/docs/dhrUjSwC3cH8VIWvplWrzwAAAAAAAAAA.png}

\subsection{Syntax}\label{syntax}

This function also has dedicated syntax: The normal quote characters (
\texttt{\ \textquotesingle{}\ } and \texttt{\ "\ } ). Typst
automatically makes your quotes smart.

\subsection{\texorpdfstring{{ Parameters
}}{ Parameters }}\label{parameters}

\phantomsection\label{parameters-tooltip}
Parameters are the inputs to a function. They are specified in
parentheses after the function name.

{ smartquote } (

{ \hyperref[parameters-double]{double :}
\href{/docs/reference/foundations/bool/}{bool} , } {
\hyperref[parameters-enabled]{enabled :}
\href{/docs/reference/foundations/bool/}{bool} , } {
\hyperref[parameters-alternative]{alternative :}
\href{/docs/reference/foundations/bool/}{bool} , } {
\hyperref[parameters-quotes]{quotes :}
\href{/docs/reference/foundations/auto/}{auto}
\href{/docs/reference/foundations/str/}{str}
\href{/docs/reference/foundations/array/}{array}
\href{/docs/reference/foundations/dictionary/}{dictionary} , }

) -\textgreater{} \href{/docs/reference/foundations/content/}{content}

\subsubsection{\texorpdfstring{\texttt{\ double\ }}{ double }}\label{parameters-double}

\href{/docs/reference/foundations/bool/}{bool}

{{ Settable }}

\phantomsection\label{parameters-double-settable-tooltip}
Settable parameters can be customized for all following uses of the
function with a \texttt{\ set\ } rule.

Whether this should be a double quote.

Default: \texttt{\ }{\texttt{\ true\ }}\texttt{\ }

\subsubsection{\texorpdfstring{\texttt{\ enabled\ }}{ enabled }}\label{parameters-enabled}

\href{/docs/reference/foundations/bool/}{bool}

{{ Settable }}

\phantomsection\label{parameters-enabled-settable-tooltip}
Settable parameters can be customized for all following uses of the
function with a \texttt{\ set\ } rule.

Whether smart quotes are enabled.

To disable smartness for a single quote, you can also escape it with a
backslash.

Default: \texttt{\ }{\texttt{\ true\ }}\texttt{\ }

\includesvg[width=0.16667in,height=0.16667in]{/assets/icons/16-arrow-right.svg}
View example

\begin{verbatim}
#set smartquote(enabled: false)

These are "dumb" quotes.
\end{verbatim}

\includegraphics[width=5in,height=\textheight,keepaspectratio]{/assets/docs/ykeeFPAnOBNAwmLXSDxTqAAAAAAAAAAA.png}

\subsubsection{\texorpdfstring{\texttt{\ alternative\ }}{ alternative }}\label{parameters-alternative}

\href{/docs/reference/foundations/bool/}{bool}

{{ Settable }}

\phantomsection\label{parameters-alternative-settable-tooltip}
Settable parameters can be customized for all following uses of the
function with a \texttt{\ set\ } rule.

Whether to use alternative quotes.

Does nothing for languages that don\textquotesingle t have alternative
quotes, or if explicit quotes were set.

Default: \texttt{\ }{\texttt{\ false\ }}\texttt{\ }

\includesvg[width=0.16667in,height=0.16667in]{/assets/icons/16-arrow-right.svg}
View example

\begin{verbatim}
#set text(lang: "de")
#set smartquote(alternative: true)

"Das ist in anderen Anführungszeichen."
\end{verbatim}

\includegraphics[width=5in,height=\textheight,keepaspectratio]{/assets/docs/lyTPNxNjIzyFJbp-JlBMpgAAAAAAAAAA.png}

\subsubsection{\texorpdfstring{\texttt{\ quotes\ }}{ quotes }}\label{parameters-quotes}

\href{/docs/reference/foundations/auto/}{auto} {or}
\href{/docs/reference/foundations/str/}{str} {or}
\href{/docs/reference/foundations/array/}{array} {or}
\href{/docs/reference/foundations/dictionary/}{dictionary}

{{ Settable }}

\phantomsection\label{parameters-quotes-settable-tooltip}
Settable parameters can be customized for all following uses of the
function with a \texttt{\ set\ } rule.

The quotes to use.

\begin{itemize}
\tightlist
\item
  When set to \texttt{\ }{\texttt{\ auto\ }}\texttt{\ } , the
  appropriate single quotes for the
  \href{/docs/reference/text/text/\#parameters-lang}{text language} will
  be used. This is the default.
\item
  Custom quotes can be passed as a string, array, or dictionary of
  either

  \begin{itemize}
  \tightlist
  \item
    \href{/docs/reference/foundations/str/}{string} : a string
    consisting of two characters containing the opening and closing
    double quotes (characters here refer to Unicode grapheme clusters)
  \item
    \href{/docs/reference/foundations/array/}{array} : an array
    containing the opening and closing double quotes
  \item
    \href{/docs/reference/foundations/dictionary/}{dictionary} : an
    array containing the double and single quotes, each specified as
    either \texttt{\ }{\texttt{\ auto\ }}\texttt{\ } , string, or array
  \end{itemize}
\end{itemize}

Default: \texttt{\ }{\texttt{\ auto\ }}\texttt{\ }

\includesvg[width=0.16667in,height=0.16667in]{/assets/icons/16-arrow-right.svg}
View example

\begin{verbatim}
#set text(lang: "de")
'Das sind normale Anführungszeichen.'

#set smartquote(quotes: "()")
"Das sind eigene Anführungszeichen."

#set smartquote(quotes: (single: ("[[", "]]"),  double: auto))
'Das sind eigene Anführungszeichen.'
\end{verbatim}

\includegraphics[width=5in,height=\textheight,keepaspectratio]{/assets/docs/bSqE_vffbQfTFgF9cX2J6AAAAAAAAAAA.png}

\href{/docs/reference/text/smallcaps/}{\pandocbounded{\includesvg[keepaspectratio]{/assets/icons/16-arrow-right.svg}}}

{ Small Capitals } { Previous page }

\href{/docs/reference/text/strike/}{\pandocbounded{\includesvg[keepaspectratio]{/assets/icons/16-arrow-right.svg}}}

{ Strikethrough } { Next page }


\section{Docs LaTeX/typst.app/docs/reference/text/underline.tex}
\title{typst.app/docs/reference/text/underline}

\begin{itemize}
\tightlist
\item
  \href{/docs}{\includesvg[width=0.16667in,height=0.16667in]{/assets/icons/16-docs-dark.svg}}
\item
  \includesvg[width=0.16667in,height=0.16667in]{/assets/icons/16-arrow-right.svg}
\item
  \href{/docs/reference/}{Reference}
\item
  \includesvg[width=0.16667in,height=0.16667in]{/assets/icons/16-arrow-right.svg}
\item
  \href{/docs/reference/text/}{Text}
\item
  \includesvg[width=0.16667in,height=0.16667in]{/assets/icons/16-arrow-right.svg}
\item
  \href{/docs/reference/text/underline/}{Underline}
\end{itemize}

\section{\texorpdfstring{\texttt{\ underline\ } {{ Element
}}}{ underline   Element }}\label{summary}

\phantomsection\label{element-tooltip}
Element functions can be customized with \texttt{\ set\ } and
\texttt{\ show\ } rules.

Underlines text.

\subsection{Example}\label{example}

\begin{verbatim}
This is #underline[important].
\end{verbatim}

\includegraphics[width=5in,height=\textheight,keepaspectratio]{/assets/docs/xV-Fy8zwdVIfyHyOpdk_9AAAAAAAAAAA.png}

\subsection{\texorpdfstring{{ Parameters
}}{ Parameters }}\label{parameters}

\phantomsection\label{parameters-tooltip}
Parameters are the inputs to a function. They are specified in
parentheses after the function name.

{ underline } (

{ \hyperref[parameters-stroke]{stroke :}
\href{/docs/reference/foundations/auto/}{auto}
\href{/docs/reference/layout/length/}{length}
\href{/docs/reference/visualize/color/}{color}
\href{/docs/reference/visualize/gradient/}{gradient}
\href{/docs/reference/visualize/stroke/}{stroke}
\href{/docs/reference/visualize/pattern/}{pattern}
\href{/docs/reference/foundations/dictionary/}{dictionary} , } {
\hyperref[parameters-offset]{offset :}
\href{/docs/reference/foundations/auto/}{auto}
\href{/docs/reference/layout/length/}{length} , } {
\hyperref[parameters-extent]{extent :}
\href{/docs/reference/layout/length/}{length} , } {
\hyperref[parameters-evade]{evade :}
\href{/docs/reference/foundations/bool/}{bool} , } {
\hyperref[parameters-background]{background :}
\href{/docs/reference/foundations/bool/}{bool} , } {
\href{/docs/reference/foundations/content/}{content} , }

) -\textgreater{} \href{/docs/reference/foundations/content/}{content}

\subsubsection{\texorpdfstring{\texttt{\ stroke\ }}{ stroke }}\label{parameters-stroke}

\href{/docs/reference/foundations/auto/}{auto} {or}
\href{/docs/reference/layout/length/}{length} {or}
\href{/docs/reference/visualize/color/}{color} {or}
\href{/docs/reference/visualize/gradient/}{gradient} {or}
\href{/docs/reference/visualize/stroke/}{stroke} {or}
\href{/docs/reference/visualize/pattern/}{pattern} {or}
\href{/docs/reference/foundations/dictionary/}{dictionary}

{{ Settable }}

\phantomsection\label{parameters-stroke-settable-tooltip}
Settable parameters can be customized for all following uses of the
function with a \texttt{\ set\ } rule.

How to \href{/docs/reference/visualize/stroke/}{stroke} the line.

If set to \texttt{\ }{\texttt{\ auto\ }}\texttt{\ } , takes on the
text\textquotesingle s color and a thickness defined in the current
font.

Default: \texttt{\ }{\texttt{\ auto\ }}\texttt{\ }

\includesvg[width=0.16667in,height=0.16667in]{/assets/icons/16-arrow-right.svg}
View example

\begin{verbatim}
Take #underline(
  stroke: 1.5pt + red,
  offset: 2pt,
  [care],
)
\end{verbatim}

\includegraphics[width=5in,height=\textheight,keepaspectratio]{/assets/docs/tbLKc9iYaghdhC9NcJaJOQAAAAAAAAAA.png}

\subsubsection{\texorpdfstring{\texttt{\ offset\ }}{ offset }}\label{parameters-offset}

\href{/docs/reference/foundations/auto/}{auto} {or}
\href{/docs/reference/layout/length/}{length}

{{ Settable }}

\phantomsection\label{parameters-offset-settable-tooltip}
Settable parameters can be customized for all following uses of the
function with a \texttt{\ set\ } rule.

The position of the line relative to the baseline, read from the font
tables if \texttt{\ }{\texttt{\ auto\ }}\texttt{\ } .

Default: \texttt{\ }{\texttt{\ auto\ }}\texttt{\ }

\includesvg[width=0.16667in,height=0.16667in]{/assets/icons/16-arrow-right.svg}
View example

\begin{verbatim}
#underline(offset: 5pt)[
  The Tale Of A Faraway Line I
]
\end{verbatim}

\includegraphics[width=5in,height=\textheight,keepaspectratio]{/assets/docs/p2tUWXcYq-E_ZbDtwzCDrAAAAAAAAAAA.png}

\subsubsection{\texorpdfstring{\texttt{\ extent\ }}{ extent }}\label{parameters-extent}

\href{/docs/reference/layout/length/}{length}

{{ Settable }}

\phantomsection\label{parameters-extent-settable-tooltip}
Settable parameters can be customized for all following uses of the
function with a \texttt{\ set\ } rule.

The amount by which to extend the line beyond (or within if negative)
the content.

Default: \texttt{\ }{\texttt{\ 0pt\ }}\texttt{\ }

\includesvg[width=0.16667in,height=0.16667in]{/assets/icons/16-arrow-right.svg}
View example

\begin{verbatim}
#align(center,
  underline(extent: 2pt)[Chapter 1]
)
\end{verbatim}

\includegraphics[width=5in,height=\textheight,keepaspectratio]{/assets/docs/tbT2BOLPtcXW-alQPb8q6wAAAAAAAAAA.png}

\subsubsection{\texorpdfstring{\texttt{\ evade\ }}{ evade }}\label{parameters-evade}

\href{/docs/reference/foundations/bool/}{bool}

{{ Settable }}

\phantomsection\label{parameters-evade-settable-tooltip}
Settable parameters can be customized for all following uses of the
function with a \texttt{\ set\ } rule.

Whether the line skips sections in which it would collide with the
glyphs.

Default: \texttt{\ }{\texttt{\ true\ }}\texttt{\ }

\includesvg[width=0.16667in,height=0.16667in]{/assets/icons/16-arrow-right.svg}
View example

\begin{verbatim}
This #underline(evade: true)[is great].
This #underline(evade: false)[is less great].
\end{verbatim}

\includegraphics[width=5in,height=\textheight,keepaspectratio]{/assets/docs/PaJc2qUpoh1s97E6NZYz0QAAAAAAAAAA.png}

\subsubsection{\texorpdfstring{\texttt{\ background\ }}{ background }}\label{parameters-background}

\href{/docs/reference/foundations/bool/}{bool}

{{ Settable }}

\phantomsection\label{parameters-background-settable-tooltip}
Settable parameters can be customized for all following uses of the
function with a \texttt{\ set\ } rule.

Whether the line is placed behind the content it underlines.

Default: \texttt{\ }{\texttt{\ false\ }}\texttt{\ }

\includesvg[width=0.16667in,height=0.16667in]{/assets/icons/16-arrow-right.svg}
View example

\begin{verbatim}
#set underline(stroke: (thickness: 1em, paint: maroon, cap: "round"))
#underline(background: true)[This is stylized.] \
#underline(background: false)[This is partially hidden.]
\end{verbatim}

\includegraphics[width=5in,height=\textheight,keepaspectratio]{/assets/docs/W98M7AlnFoSVnlt9g5bIsAAAAAAAAAAA.png}

\subsubsection{\texorpdfstring{\texttt{\ body\ }}{ body }}\label{parameters-body}

\href{/docs/reference/foundations/content/}{content}

{Required} {{ Positional }}

\phantomsection\label{parameters-body-positional-tooltip}
Positional parameters are specified in order, without names.

The content to underline.

\href{/docs/reference/text/text/}{\pandocbounded{\includesvg[keepaspectratio]{/assets/icons/16-arrow-right.svg}}}

{ Text } { Previous page }

\href{/docs/reference/text/upper/}{\pandocbounded{\includesvg[keepaspectratio]{/assets/icons/16-arrow-right.svg}}}

{ Uppercase } { Next page }


\section{Docs LaTeX/typst.app/docs/reference/text/super.tex}
\title{typst.app/docs/reference/text/super}

\begin{itemize}
\tightlist
\item
  \href{/docs}{\includesvg[width=0.16667in,height=0.16667in]{/assets/icons/16-docs-dark.svg}}
\item
  \includesvg[width=0.16667in,height=0.16667in]{/assets/icons/16-arrow-right.svg}
\item
  \href{/docs/reference/}{Reference}
\item
  \includesvg[width=0.16667in,height=0.16667in]{/assets/icons/16-arrow-right.svg}
\item
  \href{/docs/reference/text/}{Text}
\item
  \includesvg[width=0.16667in,height=0.16667in]{/assets/icons/16-arrow-right.svg}
\item
  \href{/docs/reference/text/super/}{Superscript}
\end{itemize}

\section{\texorpdfstring{\texttt{\ super\ } {{ Element
}}}{ super   Element }}\label{summary}

\phantomsection\label{element-tooltip}
Element functions can be customized with \texttt{\ set\ } and
\texttt{\ show\ } rules.

Renders text in superscript.

The text is rendered smaller and its baseline is raised.

\subsection{Example}\label{example}

\begin{verbatim}
1#super[st] try!
\end{verbatim}

\includegraphics[width=5in,height=\textheight,keepaspectratio]{/assets/docs/052zwKrkvVHtZVzW65WFdQAAAAAAAAAA.png}

\subsection{\texorpdfstring{{ Parameters
}}{ Parameters }}\label{parameters}

\phantomsection\label{parameters-tooltip}
Parameters are the inputs to a function. They are specified in
parentheses after the function name.

{ super } (

{ \hyperref[parameters-typographic]{typographic :}
\href{/docs/reference/foundations/bool/}{bool} , } {
\hyperref[parameters-baseline]{baseline :}
\href{/docs/reference/layout/length/}{length} , } {
\hyperref[parameters-size]{size :}
\href{/docs/reference/layout/length/}{length} , } {
\href{/docs/reference/foundations/content/}{content} , }

) -\textgreater{} \href{/docs/reference/foundations/content/}{content}

\subsubsection{\texorpdfstring{\texttt{\ typographic\ }}{ typographic }}\label{parameters-typographic}

\href{/docs/reference/foundations/bool/}{bool}

{{ Settable }}

\phantomsection\label{parameters-typographic-settable-tooltip}
Settable parameters can be customized for all following uses of the
function with a \texttt{\ set\ } rule.

Whether to prefer the dedicated superscript characters of the font.

If this is enabled, Typst first tries to transform the text to
superscript codepoints. If that fails, it falls back to rendering raised
and shrunk normal letters.

Default: \texttt{\ }{\texttt{\ true\ }}\texttt{\ }

\includesvg[width=0.16667in,height=0.16667in]{/assets/icons/16-arrow-right.svg}
View example

\begin{verbatim}
N#super(typographic: true)[1]
N#super(typographic: false)[1]
\end{verbatim}

\includegraphics[width=5in,height=\textheight,keepaspectratio]{/assets/docs/1_zKQkbZObDWVLT4k-2LKQAAAAAAAAAA.png}

\subsubsection{\texorpdfstring{\texttt{\ baseline\ }}{ baseline }}\label{parameters-baseline}

\href{/docs/reference/layout/length/}{length}

{{ Settable }}

\phantomsection\label{parameters-baseline-settable-tooltip}
Settable parameters can be customized for all following uses of the
function with a \texttt{\ set\ } rule.

The baseline shift for synthetic superscripts. Does not apply if
\texttt{\ typographic\ } is true and the font has superscript codepoints
for the given \texttt{\ body\ } .

Default:
\texttt{\ }{\texttt{\ -\ }}\texttt{\ }{\texttt{\ 0.5em\ }}\texttt{\ }

\subsubsection{\texorpdfstring{\texttt{\ size\ }}{ size }}\label{parameters-size}

\href{/docs/reference/layout/length/}{length}

{{ Settable }}

\phantomsection\label{parameters-size-settable-tooltip}
Settable parameters can be customized for all following uses of the
function with a \texttt{\ set\ } rule.

The font size for synthetic superscripts. Does not apply if
\texttt{\ typographic\ } is true and the font has superscript codepoints
for the given \texttt{\ body\ } .

Default: \texttt{\ }{\texttt{\ 0.6em\ }}\texttt{\ }

\subsubsection{\texorpdfstring{\texttt{\ body\ }}{ body }}\label{parameters-body}

\href{/docs/reference/foundations/content/}{content}

{Required} {{ Positional }}

\phantomsection\label{parameters-body-positional-tooltip}
Positional parameters are specified in order, without names.

The text to display in superscript.

\href{/docs/reference/text/sub/}{\pandocbounded{\includesvg[keepaspectratio]{/assets/icons/16-arrow-right.svg}}}

{ Subscript } { Previous page }

\href{/docs/reference/text/text/}{\pandocbounded{\includesvg[keepaspectratio]{/assets/icons/16-arrow-right.svg}}}

{ Text } { Next page }


\section{Docs LaTeX/typst.app/docs/reference/text/raw.tex}
\title{typst.app/docs/reference/text/raw}

\begin{itemize}
\tightlist
\item
  \href{/docs}{\includesvg[width=0.16667in,height=0.16667in]{/assets/icons/16-docs-dark.svg}}
\item
  \includesvg[width=0.16667in,height=0.16667in]{/assets/icons/16-arrow-right.svg}
\item
  \href{/docs/reference/}{Reference}
\item
  \includesvg[width=0.16667in,height=0.16667in]{/assets/icons/16-arrow-right.svg}
\item
  \href{/docs/reference/text/}{Text}
\item
  \includesvg[width=0.16667in,height=0.16667in]{/assets/icons/16-arrow-right.svg}
\item
  \href{/docs/reference/text/raw/}{Raw Text / Code}
\end{itemize}

\section{\texorpdfstring{\texttt{\ raw\ } {{ Element
}}}{ raw   Element }}\label{summary}

\phantomsection\label{element-tooltip}
Element functions can be customized with \texttt{\ set\ } and
\texttt{\ show\ } rules.

Raw text with optional syntax highlighting.

Displays the text verbatim and in a monospace font. This is typically
used to embed computer code into your document.

\subsection{Example}\label{example}

\begin{verbatim}
Adding `rbx` to `rcx` gives
the desired result.

What is ```rust fn main()``` in Rust
would be ```c int main()``` in C.

```rust
fn main() {
    println!("Hello World!");
}
```

This has ``` `backticks` ``` in it
(but the spaces are trimmed). And
``` here``` the leading space is
also trimmed.
\end{verbatim}

\includegraphics[width=5in,height=\textheight,keepaspectratio]{/assets/docs/HG5qpETGRO7ndBI1Qrek9gAAAAAAAAAA.png}

You can also construct a
\href{/docs/reference/text/raw/}{\texttt{\ raw\ }} element
programmatically from a string (and provide the language tag via the
optional
\href{/docs/reference/text/raw/\#parameters-lang}{\texttt{\ lang\ }}
argument).

\begin{verbatim}
#raw("fn " + "main() {}", lang: "rust")
\end{verbatim}

\includegraphics[width=5in,height=\textheight,keepaspectratio]{/assets/docs/MNABiMKxTxPPaXzIwfuPPQAAAAAAAAAA.png}

\subsection{Syntax}\label{syntax}

This function also has dedicated syntax. You can enclose text in 1 or 3+
backticks ( \texttt{\ \textasciigrave{}\ } ) to make it raw. Two
backticks produce empty raw text. This works both in markup and code.

When you use three or more backticks, you can additionally specify a
language tag for syntax highlighting directly after the opening
backticks. Within raw blocks, everything (except for the language tag,
if applicable) is rendered as is, in particular, there are no escape
sequences.

The language tag is an identifier that directly follows the opening
backticks only if there are three or more backticks. If your text starts
with something that looks like an identifier, but no syntax highlighting
is needed, start the text with a single space (which will be trimmed) or
use the single backtick syntax. If your text should start or end with a
backtick, put a space before or after it (it will be trimmed).

\subsection{\texorpdfstring{{ Parameters
}}{ Parameters }}\label{parameters}

\phantomsection\label{parameters-tooltip}
Parameters are the inputs to a function. They are specified in
parentheses after the function name.

{ raw } (

{ \href{/docs/reference/foundations/str/}{str} , } {
\hyperref[parameters-block]{block :}
\href{/docs/reference/foundations/bool/}{bool} , } {
\hyperref[parameters-lang]{lang :}
\href{/docs/reference/foundations/none/}{none}
\href{/docs/reference/foundations/str/}{str} , } {
\hyperref[parameters-align]{align :}
\href{/docs/reference/layout/alignment/}{alignment} , } {
\hyperref[parameters-syntaxes]{syntaxes :}
\href{/docs/reference/foundations/str/}{str}
\href{/docs/reference/foundations/array/}{array} , } {
\hyperref[parameters-theme]{theme :}
\href{/docs/reference/foundations/none/}{none}
\href{/docs/reference/foundations/auto/}{auto}
\href{/docs/reference/foundations/str/}{str} , } {
\hyperref[parameters-tab-size]{tab-size :}
\href{/docs/reference/foundations/int/}{int} , }

) -\textgreater{} \href{/docs/reference/foundations/content/}{content}

\subsubsection{\texorpdfstring{\texttt{\ text\ }}{ text }}\label{parameters-text}

\href{/docs/reference/foundations/str/}{str}

{Required} {{ Positional }}

\phantomsection\label{parameters-text-positional-tooltip}
Positional parameters are specified in order, without names.

The raw text.

You can also use raw blocks creatively to create custom syntaxes for
your automations.

\includesvg[width=0.16667in,height=0.16667in]{/assets/icons/16-arrow-right.svg}
View example

\begin{verbatim}
// Parse numbers in raw blocks with the
// `mydsl` tag and sum them up.
#show raw.where(lang: "mydsl"): it => {
  let sum = 0
  for part in it.text.split("+") {
    sum += int(part.trim())
  }
  sum
}

```mydsl
1 + 2 + 3 + 4 + 5
```
\end{verbatim}

\includegraphics[width=5in,height=\textheight,keepaspectratio]{/assets/docs/6VperjQoP8Ey0LiUk5m0HQAAAAAAAAAA.png}

\subsubsection{\texorpdfstring{\texttt{\ block\ }}{ block }}\label{parameters-block}

\href{/docs/reference/foundations/bool/}{bool}

{{ Settable }}

\phantomsection\label{parameters-block-settable-tooltip}
Settable parameters can be customized for all following uses of the
function with a \texttt{\ set\ } rule.

Whether the raw text is displayed as a separate block.

In markup mode, using one-backtick notation makes this
\texttt{\ }{\texttt{\ false\ }}\texttt{\ } . Using three-backtick
notation makes it \texttt{\ }{\texttt{\ true\ }}\texttt{\ } if the
enclosed content contains at least one line break.

Default: \texttt{\ }{\texttt{\ false\ }}\texttt{\ }

\includesvg[width=0.16667in,height=0.16667in]{/assets/icons/16-arrow-right.svg}
View example

\begin{verbatim}
// Display inline code in a small box
// that retains the correct baseline.
#show raw.where(block: false): box.with(
  fill: luma(240),
  inset: (x: 3pt, y: 0pt),
  outset: (y: 3pt),
  radius: 2pt,
)

// Display block code in a larger block
// with more padding.
#show raw.where(block: true): block.with(
  fill: luma(240),
  inset: 10pt,
  radius: 4pt,
)

With `rg`, you can search through your files quickly.
This example searches the current directory recursively
for the text `Hello World`:

```bash
rg "Hello World"
```
\end{verbatim}

\includegraphics[width=5in,height=\textheight,keepaspectratio]{/assets/docs/PgXCmmr2Cn53ZnpWQOnMwwAAAAAAAAAA.png}

\subsubsection{\texorpdfstring{\texttt{\ lang\ }}{ lang }}\label{parameters-lang}

\href{/docs/reference/foundations/none/}{none} {or}
\href{/docs/reference/foundations/str/}{str}

{{ Settable }}

\phantomsection\label{parameters-lang-settable-tooltip}
Settable parameters can be customized for all following uses of the
function with a \texttt{\ set\ } rule.

The language to syntax-highlight in.

Apart from typical language tags known from Markdown, this supports the
\texttt{\ }{\texttt{\ "typ"\ }}\texttt{\ } ,
\texttt{\ }{\texttt{\ "typc"\ }}\texttt{\ } , and
\texttt{\ }{\texttt{\ "typm"\ }}\texttt{\ } tags for
\href{/docs/reference/syntax/\#markup}{Typst markup} ,
\href{/docs/reference/syntax/\#code}{Typst code} , and
\href{/docs/reference/syntax/\#math}{Typst math} , respectively.

Default: \texttt{\ }{\texttt{\ none\ }}\texttt{\ }

\includesvg[width=0.16667in,height=0.16667in]{/assets/icons/16-arrow-right.svg}
View example

\begin{verbatim}
```typ
This is *Typst!*
```

This is ```typ also *Typst*```, but inline!
\end{verbatim}

\includegraphics[width=5in,height=\textheight,keepaspectratio]{/assets/docs/bjU3PMlFs9msUi72QThHnAAAAAAAAAAA.png}

\subsubsection{\texorpdfstring{\texttt{\ align\ }}{ align }}\label{parameters-align}

\href{/docs/reference/layout/alignment/}{alignment}

{{ Settable }}

\phantomsection\label{parameters-align-settable-tooltip}
Settable parameters can be customized for all following uses of the
function with a \texttt{\ set\ } rule.

The horizontal alignment that each line in a raw block should have. This
option is ignored if this is not a raw block (if specified
\texttt{\ block:\ false\ } or single backticks were used in markup
mode).

By default, this is set to \texttt{\ start\ } , meaning that raw text is
aligned towards the start of the text direction inside the block by
default, regardless of the current context\textquotesingle s alignment
(allowing you to center the raw block itself without centering the text
inside it, for example).

Default: \texttt{\ start\ }

\includesvg[width=0.16667in,height=0.16667in]{/assets/icons/16-arrow-right.svg}
View example

\begin{verbatim}
#set raw(align: center)

```typc
let f(x) = x
code = "centered"
```
\end{verbatim}

\includegraphics[width=5in,height=\textheight,keepaspectratio]{/assets/docs/QoY61HWjc7MIUABTr8mvwwAAAAAAAAAA.png}

\subsubsection{\texorpdfstring{\texttt{\ syntaxes\ }}{ syntaxes }}\label{parameters-syntaxes}

\href{/docs/reference/foundations/str/}{str} {or}
\href{/docs/reference/foundations/array/}{array}

{{ Settable }}

\phantomsection\label{parameters-syntaxes-settable-tooltip}
Settable parameters can be customized for all following uses of the
function with a \texttt{\ set\ } rule.

One or multiple additional syntax definitions to load. The syntax
definitions should be in the
\href{https://www.sublimetext.com/docs/syntax.html}{\texttt{\ sublime-syntax\ }
file format} .

Default:
\texttt{\ }{\texttt{\ (\ }}\texttt{\ }{\texttt{\ )\ }}\texttt{\ }

\includesvg[width=0.16667in,height=0.16667in]{/assets/icons/16-arrow-right.svg}
View example

\begin{verbatim}
#set raw(syntaxes: "SExpressions.sublime-syntax")

```sexp
(defun factorial (x)
  (if (zerop x)
    ; with a comment
    1
    (* x (factorial (- x 1)))))
```
\end{verbatim}

\includegraphics[width=5in,height=\textheight,keepaspectratio]{/assets/docs/f1wEtKdjbuwy-LVNGIZ_igAAAAAAAAAA.png}

\subsubsection{\texorpdfstring{\texttt{\ theme\ }}{ theme }}\label{parameters-theme}

\href{/docs/reference/foundations/none/}{none} {or}
\href{/docs/reference/foundations/auto/}{auto} {or}
\href{/docs/reference/foundations/str/}{str}

{{ Settable }}

\phantomsection\label{parameters-theme-settable-tooltip}
Settable parameters can be customized for all following uses of the
function with a \texttt{\ set\ } rule.

The theme to use for syntax highlighting. Theme files should be in the
\href{https://www.sublimetext.com/docs/color_schemes_tmtheme.html}{\texttt{\ tmTheme\ }
file format} .

Applying a theme only affects the color of specifically highlighted
text. It does not consider the theme\textquotesingle s foreground and
background properties, so that you retain control over the color of raw
text. You can apply the foreground color yourself with the
\href{/docs/reference/text/text/}{\texttt{\ text\ }} function and the
background with a
\href{/docs/reference/layout/block/\#parameters-fill}{filled block} .
You could also use the
\href{/docs/reference/data-loading/xml/}{\texttt{\ xml\ }} function to
extract these properties from the theme.

Additionally, you can set the theme to
\texttt{\ }{\texttt{\ none\ }}\texttt{\ } to disable highlighting.

Default: \texttt{\ }{\texttt{\ auto\ }}\texttt{\ }

\includesvg[width=0.16667in,height=0.16667in]{/assets/icons/16-arrow-right.svg}
View example

\begin{verbatim}
#set raw(theme: "halcyon.tmTheme")
#show raw: it => block(
  fill: rgb("#1d2433"),
  inset: 8pt,
  radius: 5pt,
  text(fill: rgb("#a2aabc"), it)
)

```typ
= Chapter 1
#let hi = "Hello World"
```
\end{verbatim}

\includegraphics[width=5in,height=\textheight,keepaspectratio]{/assets/docs/_3ndU0y1KsOpDAMv999GwwAAAAAAAAAA.png}

\subsubsection{\texorpdfstring{\texttt{\ tab-size\ }}{ tab-size }}\label{parameters-tab-size}

\href{/docs/reference/foundations/int/}{int}

{{ Settable }}

\phantomsection\label{parameters-tab-size-settable-tooltip}
Settable parameters can be customized for all following uses of the
function with a \texttt{\ set\ } rule.

The size for a tab stop in spaces. A tab is replaced with enough spaces
to align with the next multiple of the size.

Default: \texttt{\ }{\texttt{\ 2\ }}\texttt{\ }

\includesvg[width=0.16667in,height=0.16667in]{/assets/icons/16-arrow-right.svg}
View example

\begin{verbatim}
#set raw(tab-size: 8)
```tsv
Year    Month   Day
2000    2   3
2001    2   1
2002    3   10
```
\end{verbatim}

\includegraphics[width=5in,height=\textheight,keepaspectratio]{/assets/docs/OAN98lLQ4wUhrTrjbVCTywAAAAAAAAAA.png}

\subsection{\texorpdfstring{{ Definitions
}}{ Definitions }}\label{definitions}

\phantomsection\label{definitions-tooltip}
Functions and types and can have associated definitions. These are
accessed by specifying the function or type, followed by a period, and
then the definition\textquotesingle s name.

\subsubsection{\texorpdfstring{\texttt{\ line\ } {{ Element
}}}{ line   Element }}\label{definitions-line}

\phantomsection\label{definitions-line-element-tooltip}
Element functions can be customized with \texttt{\ set\ } and
\texttt{\ show\ } rules.

A highlighted line of raw text.

This is a helper element that is synthesized by
\href{/docs/reference/text/raw/}{\texttt{\ raw\ }} elements.

It allows you to access various properties of the line, such as the line
number, the raw non-highlighted text, the highlighted text, and whether
it is the first or last line of the raw block.

raw { . } { line } (

{ \href{/docs/reference/foundations/int/}{int} , } {
\href{/docs/reference/foundations/int/}{int} , } {
\href{/docs/reference/foundations/str/}{str} , } {
\href{/docs/reference/foundations/content/}{content} , }

) -\textgreater{} \href{/docs/reference/foundations/content/}{content}

\paragraph{\texorpdfstring{\texttt{\ number\ }}{ number }}\label{definitions-line-number}

\href{/docs/reference/foundations/int/}{int}

{Required} {{ Positional }}

\phantomsection\label{definitions-line-number-positional-tooltip}
Positional parameters are specified in order, without names.

The line number of the raw line inside of the raw block, starts at 1.

\paragraph{\texorpdfstring{\texttt{\ count\ }}{ count }}\label{definitions-line-count}

\href{/docs/reference/foundations/int/}{int}

{Required} {{ Positional }}

\phantomsection\label{definitions-line-count-positional-tooltip}
Positional parameters are specified in order, without names.

The total number of lines in the raw block.

\paragraph{\texorpdfstring{\texttt{\ text\ }}{ text }}\label{definitions-line-text}

\href{/docs/reference/foundations/str/}{str}

{Required} {{ Positional }}

\phantomsection\label{definitions-line-text-positional-tooltip}
Positional parameters are specified in order, without names.

The line of raw text.

\paragraph{\texorpdfstring{\texttt{\ body\ }}{ body }}\label{definitions-line-body}

\href{/docs/reference/foundations/content/}{content}

{Required} {{ Positional }}

\phantomsection\label{definitions-line-body-positional-tooltip}
Positional parameters are specified in order, without names.

The highlighted raw text.

\href{/docs/reference/text/overline/}{\pandocbounded{\includesvg[keepaspectratio]{/assets/icons/16-arrow-right.svg}}}

{ Overline } { Previous page }

\href{/docs/reference/text/smallcaps/}{\pandocbounded{\includesvg[keepaspectratio]{/assets/icons/16-arrow-right.svg}}}

{ Small Capitals } { Next page }


\section{Docs LaTeX/typst.app/docs/reference/text/lower.tex}
\title{typst.app/docs/reference/text/lower}

\begin{itemize}
\tightlist
\item
  \href{/docs}{\includesvg[width=0.16667in,height=0.16667in]{/assets/icons/16-docs-dark.svg}}
\item
  \includesvg[width=0.16667in,height=0.16667in]{/assets/icons/16-arrow-right.svg}
\item
  \href{/docs/reference/}{Reference}
\item
  \includesvg[width=0.16667in,height=0.16667in]{/assets/icons/16-arrow-right.svg}
\item
  \href{/docs/reference/text/}{Text}
\item
  \includesvg[width=0.16667in,height=0.16667in]{/assets/icons/16-arrow-right.svg}
\item
  \href{/docs/reference/text/lower/}{Lowercase}
\end{itemize}

\section{\texorpdfstring{\texttt{\ lower\ }}{ lower }}\label{summary}

Converts a string or content to lowercase.

\subsection{Example}\label{example}

\begin{verbatim}
#lower("ABC") \
#lower[*My Text*] \
#lower[already low]
\end{verbatim}

\includegraphics[width=5in,height=\textheight,keepaspectratio]{/assets/docs/zbgdZcwg4Knc-ePylT0zpQAAAAAAAAAA.png}

\subsection{\texorpdfstring{{ Parameters
}}{ Parameters }}\label{parameters}

\phantomsection\label{parameters-tooltip}
Parameters are the inputs to a function. They are specified in
parentheses after the function name.

{ lower } (

{ \href{/docs/reference/foundations/str/}{str}
\href{/docs/reference/foundations/content/}{content} }

) -\textgreater{} \href{/docs/reference/foundations/str/}{str}
\href{/docs/reference/foundations/content/}{content}

\subsubsection{\texorpdfstring{\texttt{\ text\ }}{ text }}\label{parameters-text}

\href{/docs/reference/foundations/str/}{str} {or}
\href{/docs/reference/foundations/content/}{content}

{Required} {{ Positional }}

\phantomsection\label{parameters-text-positional-tooltip}
Positional parameters are specified in order, without names.

The text to convert to lowercase.

\href{/docs/reference/text/lorem/}{\pandocbounded{\includesvg[keepaspectratio]{/assets/icons/16-arrow-right.svg}}}

{ Lorem } { Previous page }

\href{/docs/reference/text/overline/}{\pandocbounded{\includesvg[keepaspectratio]{/assets/icons/16-arrow-right.svg}}}

{ Overline } { Next page }


\section{Docs LaTeX/typst.app/docs/reference/text/sub.tex}
\title{typst.app/docs/reference/text/sub}

\begin{itemize}
\tightlist
\item
  \href{/docs}{\includesvg[width=0.16667in,height=0.16667in]{/assets/icons/16-docs-dark.svg}}
\item
  \includesvg[width=0.16667in,height=0.16667in]{/assets/icons/16-arrow-right.svg}
\item
  \href{/docs/reference/}{Reference}
\item
  \includesvg[width=0.16667in,height=0.16667in]{/assets/icons/16-arrow-right.svg}
\item
  \href{/docs/reference/text/}{Text}
\item
  \includesvg[width=0.16667in,height=0.16667in]{/assets/icons/16-arrow-right.svg}
\item
  \href{/docs/reference/text/sub/}{Subscript}
\end{itemize}

\section{\texorpdfstring{\texttt{\ sub\ } {{ Element
}}}{ sub   Element }}\label{summary}

\phantomsection\label{element-tooltip}
Element functions can be customized with \texttt{\ set\ } and
\texttt{\ show\ } rules.

Renders text in subscript.

The text is rendered smaller and its baseline is lowered.

\subsection{Example}\label{example}

\begin{verbatim}
Revenue#sub[yearly]
\end{verbatim}

\includegraphics[width=5in,height=\textheight,keepaspectratio]{/assets/docs/q6m3B3bVOLKPuJFIogqIMwAAAAAAAAAA.png}

\subsection{\texorpdfstring{{ Parameters
}}{ Parameters }}\label{parameters}

\phantomsection\label{parameters-tooltip}
Parameters are the inputs to a function. They are specified in
parentheses after the function name.

{ sub } (

{ \hyperref[parameters-typographic]{typographic :}
\href{/docs/reference/foundations/bool/}{bool} , } {
\hyperref[parameters-baseline]{baseline :}
\href{/docs/reference/layout/length/}{length} , } {
\hyperref[parameters-size]{size :}
\href{/docs/reference/layout/length/}{length} , } {
\href{/docs/reference/foundations/content/}{content} , }

) -\textgreater{} \href{/docs/reference/foundations/content/}{content}

\subsubsection{\texorpdfstring{\texttt{\ typographic\ }}{ typographic }}\label{parameters-typographic}

\href{/docs/reference/foundations/bool/}{bool}

{{ Settable }}

\phantomsection\label{parameters-typographic-settable-tooltip}
Settable parameters can be customized for all following uses of the
function with a \texttt{\ set\ } rule.

Whether to prefer the dedicated subscript characters of the font.

If this is enabled, Typst first tries to transform the text to subscript
codepoints. If that fails, it falls back to rendering lowered and shrunk
normal letters.

Default: \texttt{\ }{\texttt{\ true\ }}\texttt{\ }

\includesvg[width=0.16667in,height=0.16667in]{/assets/icons/16-arrow-right.svg}
View example

\begin{verbatim}
N#sub(typographic: true)[1]
N#sub(typographic: false)[1]
\end{verbatim}

\includegraphics[width=5in,height=\textheight,keepaspectratio]{/assets/docs/eGuJ4coPHcIbozTvGKvULAAAAAAAAAAA.png}

\subsubsection{\texorpdfstring{\texttt{\ baseline\ }}{ baseline }}\label{parameters-baseline}

\href{/docs/reference/layout/length/}{length}

{{ Settable }}

\phantomsection\label{parameters-baseline-settable-tooltip}
Settable parameters can be customized for all following uses of the
function with a \texttt{\ set\ } rule.

The baseline shift for synthetic subscripts. Does not apply if
\texttt{\ typographic\ } is true and the font has subscript codepoints
for the given \texttt{\ body\ } .

Default: \texttt{\ }{\texttt{\ 0.2em\ }}\texttt{\ }

\subsubsection{\texorpdfstring{\texttt{\ size\ }}{ size }}\label{parameters-size}

\href{/docs/reference/layout/length/}{length}

{{ Settable }}

\phantomsection\label{parameters-size-settable-tooltip}
Settable parameters can be customized for all following uses of the
function with a \texttt{\ set\ } rule.

The font size for synthetic subscripts. Does not apply if
\texttt{\ typographic\ } is true and the font has subscript codepoints
for the given \texttt{\ body\ } .

Default: \texttt{\ }{\texttt{\ 0.6em\ }}\texttt{\ }

\subsubsection{\texorpdfstring{\texttt{\ body\ }}{ body }}\label{parameters-body}

\href{/docs/reference/foundations/content/}{content}

{Required} {{ Positional }}

\phantomsection\label{parameters-body-positional-tooltip}
Positional parameters are specified in order, without names.

The text to display in subscript.

\href{/docs/reference/text/strike/}{\pandocbounded{\includesvg[keepaspectratio]{/assets/icons/16-arrow-right.svg}}}

{ Strikethrough } { Previous page }

\href{/docs/reference/text/super/}{\pandocbounded{\includesvg[keepaspectratio]{/assets/icons/16-arrow-right.svg}}}

{ Superscript } { Next page }


\section{Docs LaTeX/typst.app/docs/reference/text/overline.tex}
\title{typst.app/docs/reference/text/overline}

\begin{itemize}
\tightlist
\item
  \href{/docs}{\includesvg[width=0.16667in,height=0.16667in]{/assets/icons/16-docs-dark.svg}}
\item
  \includesvg[width=0.16667in,height=0.16667in]{/assets/icons/16-arrow-right.svg}
\item
  \href{/docs/reference/}{Reference}
\item
  \includesvg[width=0.16667in,height=0.16667in]{/assets/icons/16-arrow-right.svg}
\item
  \href{/docs/reference/text/}{Text}
\item
  \includesvg[width=0.16667in,height=0.16667in]{/assets/icons/16-arrow-right.svg}
\item
  \href{/docs/reference/text/overline/}{Overline}
\end{itemize}

\section{\texorpdfstring{\texttt{\ overline\ } {{ Element
}}}{ overline   Element }}\label{summary}

\phantomsection\label{element-tooltip}
Element functions can be customized with \texttt{\ set\ } and
\texttt{\ show\ } rules.

Adds a line over text.

\subsection{Example}\label{example}

\begin{verbatim}
#overline[A line over text.]
\end{verbatim}

\includegraphics[width=5in,height=\textheight,keepaspectratio]{/assets/docs/BQmJqK4pMIkZOu3QEFxsZAAAAAAAAAAA.png}

\subsection{\texorpdfstring{{ Parameters
}}{ Parameters }}\label{parameters}

\phantomsection\label{parameters-tooltip}
Parameters are the inputs to a function. They are specified in
parentheses after the function name.

{ overline } (

{ \hyperref[parameters-stroke]{stroke :}
\href{/docs/reference/foundations/auto/}{auto}
\href{/docs/reference/layout/length/}{length}
\href{/docs/reference/visualize/color/}{color}
\href{/docs/reference/visualize/gradient/}{gradient}
\href{/docs/reference/visualize/stroke/}{stroke}
\href{/docs/reference/visualize/pattern/}{pattern}
\href{/docs/reference/foundations/dictionary/}{dictionary} , } {
\hyperref[parameters-offset]{offset :}
\href{/docs/reference/foundations/auto/}{auto}
\href{/docs/reference/layout/length/}{length} , } {
\hyperref[parameters-extent]{extent :}
\href{/docs/reference/layout/length/}{length} , } {
\hyperref[parameters-evade]{evade :}
\href{/docs/reference/foundations/bool/}{bool} , } {
\hyperref[parameters-background]{background :}
\href{/docs/reference/foundations/bool/}{bool} , } {
\href{/docs/reference/foundations/content/}{content} , }

) -\textgreater{} \href{/docs/reference/foundations/content/}{content}

\subsubsection{\texorpdfstring{\texttt{\ stroke\ }}{ stroke }}\label{parameters-stroke}

\href{/docs/reference/foundations/auto/}{auto} {or}
\href{/docs/reference/layout/length/}{length} {or}
\href{/docs/reference/visualize/color/}{color} {or}
\href{/docs/reference/visualize/gradient/}{gradient} {or}
\href{/docs/reference/visualize/stroke/}{stroke} {or}
\href{/docs/reference/visualize/pattern/}{pattern} {or}
\href{/docs/reference/foundations/dictionary/}{dictionary}

{{ Settable }}

\phantomsection\label{parameters-stroke-settable-tooltip}
Settable parameters can be customized for all following uses of the
function with a \texttt{\ set\ } rule.

How to \href{/docs/reference/visualize/stroke/}{stroke} the line.

If set to \texttt{\ }{\texttt{\ auto\ }}\texttt{\ } , takes on the
text\textquotesingle s color and a thickness defined in the current
font.

Default: \texttt{\ }{\texttt{\ auto\ }}\texttt{\ }

\includesvg[width=0.16667in,height=0.16667in]{/assets/icons/16-arrow-right.svg}
View example

\begin{verbatim}
#set text(fill: olive)
#overline(
  stroke: green.darken(20%),
  offset: -12pt,
  [The Forest Theme],
)
\end{verbatim}

\includegraphics[width=5in,height=\textheight,keepaspectratio]{/assets/docs/jXEAZxd9NFnCtgcbDVlzIQAAAAAAAAAA.png}

\subsubsection{\texorpdfstring{\texttt{\ offset\ }}{ offset }}\label{parameters-offset}

\href{/docs/reference/foundations/auto/}{auto} {or}
\href{/docs/reference/layout/length/}{length}

{{ Settable }}

\phantomsection\label{parameters-offset-settable-tooltip}
Settable parameters can be customized for all following uses of the
function with a \texttt{\ set\ } rule.

The position of the line relative to the baseline. Read from the font
tables if \texttt{\ }{\texttt{\ auto\ }}\texttt{\ } .

Default: \texttt{\ }{\texttt{\ auto\ }}\texttt{\ }

\includesvg[width=0.16667in,height=0.16667in]{/assets/icons/16-arrow-right.svg}
View example

\begin{verbatim}
#overline(offset: -1.2em)[
  The Tale Of A Faraway Line II
]
\end{verbatim}

\includegraphics[width=5in,height=\textheight,keepaspectratio]{/assets/docs/AUBIhMOFPefmpe2mV6TTrgAAAAAAAAAA.png}

\subsubsection{\texorpdfstring{\texttt{\ extent\ }}{ extent }}\label{parameters-extent}

\href{/docs/reference/layout/length/}{length}

{{ Settable }}

\phantomsection\label{parameters-extent-settable-tooltip}
Settable parameters can be customized for all following uses of the
function with a \texttt{\ set\ } rule.

The amount by which to extend the line beyond (or within if negative)
the content.

Default: \texttt{\ }{\texttt{\ 0pt\ }}\texttt{\ }

\includesvg[width=0.16667in,height=0.16667in]{/assets/icons/16-arrow-right.svg}
View example

\begin{verbatim}
#set overline(extent: 4pt)
#set underline(extent: 4pt)
#overline(underline[Typography Today])
\end{verbatim}

\includegraphics[width=5in,height=\textheight,keepaspectratio]{/assets/docs/11dFhng73-PPcouY1kGuxAAAAAAAAAAA.png}

\subsubsection{\texorpdfstring{\texttt{\ evade\ }}{ evade }}\label{parameters-evade}

\href{/docs/reference/foundations/bool/}{bool}

{{ Settable }}

\phantomsection\label{parameters-evade-settable-tooltip}
Settable parameters can be customized for all following uses of the
function with a \texttt{\ set\ } rule.

Whether the line skips sections in which it would collide with the
glyphs.

Default: \texttt{\ }{\texttt{\ true\ }}\texttt{\ }

\includesvg[width=0.16667in,height=0.16667in]{/assets/icons/16-arrow-right.svg}
View example

\begin{verbatim}
#overline(
  evade: false,
  offset: -7.5pt,
  stroke: 1pt,
  extent: 3pt,
  [Temple],
)
\end{verbatim}

\includegraphics[width=5in,height=\textheight,keepaspectratio]{/assets/docs/4typb8n1rt84GcGKwEvmQAAAAAAAAAAA.png}

\subsubsection{\texorpdfstring{\texttt{\ background\ }}{ background }}\label{parameters-background}

\href{/docs/reference/foundations/bool/}{bool}

{{ Settable }}

\phantomsection\label{parameters-background-settable-tooltip}
Settable parameters can be customized for all following uses of the
function with a \texttt{\ set\ } rule.

Whether the line is placed behind the content it overlines.

Default: \texttt{\ }{\texttt{\ false\ }}\texttt{\ }

\includesvg[width=0.16667in,height=0.16667in]{/assets/icons/16-arrow-right.svg}
View example

\begin{verbatim}
#set overline(stroke: (thickness: 1em, paint: maroon, cap: "round"))
#overline(background: true)[This is stylized.] \
#overline(background: false)[This is partially hidden.]
\end{verbatim}

\includegraphics[width=5in,height=\textheight,keepaspectratio]{/assets/docs/J1qF0GrkgS3hBoWTovrZ_AAAAAAAAAAA.png}

\subsubsection{\texorpdfstring{\texttt{\ body\ }}{ body }}\label{parameters-body}

\href{/docs/reference/foundations/content/}{content}

{Required} {{ Positional }}

\phantomsection\label{parameters-body-positional-tooltip}
Positional parameters are specified in order, without names.

The content to add a line over.

\href{/docs/reference/text/lower/}{\pandocbounded{\includesvg[keepaspectratio]{/assets/icons/16-arrow-right.svg}}}

{ Lowercase } { Previous page }

\href{/docs/reference/text/raw/}{\pandocbounded{\includesvg[keepaspectratio]{/assets/icons/16-arrow-right.svg}}}

{ Raw Text / Code } { Next page }


\section{Docs LaTeX/typst.app/docs/reference/text/upper.tex}
\title{typst.app/docs/reference/text/upper}

\begin{itemize}
\tightlist
\item
  \href{/docs}{\includesvg[width=0.16667in,height=0.16667in]{/assets/icons/16-docs-dark.svg}}
\item
  \includesvg[width=0.16667in,height=0.16667in]{/assets/icons/16-arrow-right.svg}
\item
  \href{/docs/reference/}{Reference}
\item
  \includesvg[width=0.16667in,height=0.16667in]{/assets/icons/16-arrow-right.svg}
\item
  \href{/docs/reference/text/}{Text}
\item
  \includesvg[width=0.16667in,height=0.16667in]{/assets/icons/16-arrow-right.svg}
\item
  \href{/docs/reference/text/upper/}{Uppercase}
\end{itemize}

\section{\texorpdfstring{\texttt{\ upper\ }}{ upper }}\label{summary}

Converts a string or content to uppercase.

\subsection{Example}\label{example}

\begin{verbatim}
#upper("abc") \
#upper[*my text*] \
#upper[ALREADY HIGH]
\end{verbatim}

\includegraphics[width=5in,height=\textheight,keepaspectratio]{/assets/docs/0rcLdDpP-7G0hFWoW3-J-wAAAAAAAAAA.png}

\subsection{\texorpdfstring{{ Parameters
}}{ Parameters }}\label{parameters}

\phantomsection\label{parameters-tooltip}
Parameters are the inputs to a function. They are specified in
parentheses after the function name.

{ upper } (

{ \href{/docs/reference/foundations/str/}{str}
\href{/docs/reference/foundations/content/}{content} }

) -\textgreater{} \href{/docs/reference/foundations/str/}{str}
\href{/docs/reference/foundations/content/}{content}

\subsubsection{\texorpdfstring{\texttt{\ text\ }}{ text }}\label{parameters-text}

\href{/docs/reference/foundations/str/}{str} {or}
\href{/docs/reference/foundations/content/}{content}

{Required} {{ Positional }}

\phantomsection\label{parameters-text-positional-tooltip}
Positional parameters are specified in order, without names.

The text to convert to uppercase.

\href{/docs/reference/text/underline/}{\pandocbounded{\includesvg[keepaspectratio]{/assets/icons/16-arrow-right.svg}}}

{ Underline } { Previous page }

\href{/docs/reference/math/}{\pandocbounded{\includesvg[keepaspectratio]{/assets/icons/16-arrow-right.svg}}}

{ Math } { Next page }


\section{Docs LaTeX/typst.app/docs/reference/text/strike.tex}
\title{typst.app/docs/reference/text/strike}

\begin{itemize}
\tightlist
\item
  \href{/docs}{\includesvg[width=0.16667in,height=0.16667in]{/assets/icons/16-docs-dark.svg}}
\item
  \includesvg[width=0.16667in,height=0.16667in]{/assets/icons/16-arrow-right.svg}
\item
  \href{/docs/reference/}{Reference}
\item
  \includesvg[width=0.16667in,height=0.16667in]{/assets/icons/16-arrow-right.svg}
\item
  \href{/docs/reference/text/}{Text}
\item
  \includesvg[width=0.16667in,height=0.16667in]{/assets/icons/16-arrow-right.svg}
\item
  \href{/docs/reference/text/strike/}{Strikethrough}
\end{itemize}

\section{\texorpdfstring{\texttt{\ strike\ } {{ Element
}}}{ strike   Element }}\label{summary}

\phantomsection\label{element-tooltip}
Element functions can be customized with \texttt{\ set\ } and
\texttt{\ show\ } rules.

Strikes through text.

\subsection{Example}\label{example}

\begin{verbatim}
This is #strike[not] relevant.
\end{verbatim}

\includegraphics[width=5in,height=\textheight,keepaspectratio]{/assets/docs/gYmGRzTLJUGSNzHzEZFB3gAAAAAAAAAA.png}

\subsection{\texorpdfstring{{ Parameters
}}{ Parameters }}\label{parameters}

\phantomsection\label{parameters-tooltip}
Parameters are the inputs to a function. They are specified in
parentheses after the function name.

{ strike } (

{ \hyperref[parameters-stroke]{stroke :}
\href{/docs/reference/foundations/auto/}{auto}
\href{/docs/reference/layout/length/}{length}
\href{/docs/reference/visualize/color/}{color}
\href{/docs/reference/visualize/gradient/}{gradient}
\href{/docs/reference/visualize/stroke/}{stroke}
\href{/docs/reference/visualize/pattern/}{pattern}
\href{/docs/reference/foundations/dictionary/}{dictionary} , } {
\hyperref[parameters-offset]{offset :}
\href{/docs/reference/foundations/auto/}{auto}
\href{/docs/reference/layout/length/}{length} , } {
\hyperref[parameters-extent]{extent :}
\href{/docs/reference/layout/length/}{length} , } {
\hyperref[parameters-background]{background :}
\href{/docs/reference/foundations/bool/}{bool} , } {
\href{/docs/reference/foundations/content/}{content} , }

) -\textgreater{} \href{/docs/reference/foundations/content/}{content}

\subsubsection{\texorpdfstring{\texttt{\ stroke\ }}{ stroke }}\label{parameters-stroke}

\href{/docs/reference/foundations/auto/}{auto} {or}
\href{/docs/reference/layout/length/}{length} {or}
\href{/docs/reference/visualize/color/}{color} {or}
\href{/docs/reference/visualize/gradient/}{gradient} {or}
\href{/docs/reference/visualize/stroke/}{stroke} {or}
\href{/docs/reference/visualize/pattern/}{pattern} {or}
\href{/docs/reference/foundations/dictionary/}{dictionary}

{{ Settable }}

\phantomsection\label{parameters-stroke-settable-tooltip}
Settable parameters can be customized for all following uses of the
function with a \texttt{\ set\ } rule.

How to \href{/docs/reference/visualize/stroke/}{stroke} the line.

If set to \texttt{\ }{\texttt{\ auto\ }}\texttt{\ } , takes on the
text\textquotesingle s color and a thickness defined in the current
font.

\emph{Note:} Please don\textquotesingle t use this for real redaction as
you can still copy paste the text.

Default: \texttt{\ }{\texttt{\ auto\ }}\texttt{\ }

\includesvg[width=0.16667in,height=0.16667in]{/assets/icons/16-arrow-right.svg}
View example

\begin{verbatim}
This is #strike(stroke: 1.5pt + red)[very stricken through]. \
This is #strike(stroke: 10pt)[redacted].
\end{verbatim}

\includegraphics[width=5in,height=\textheight,keepaspectratio]{/assets/docs/z5bibL2s5nJ9Rg5dVQco5QAAAAAAAAAA.png}

\subsubsection{\texorpdfstring{\texttt{\ offset\ }}{ offset }}\label{parameters-offset}

\href{/docs/reference/foundations/auto/}{auto} {or}
\href{/docs/reference/layout/length/}{length}

{{ Settable }}

\phantomsection\label{parameters-offset-settable-tooltip}
Settable parameters can be customized for all following uses of the
function with a \texttt{\ set\ } rule.

The position of the line relative to the baseline. Read from the font
tables if \texttt{\ }{\texttt{\ auto\ }}\texttt{\ } .

This is useful if you are unhappy with the offset your font provides.

Default: \texttt{\ }{\texttt{\ auto\ }}\texttt{\ }

\includesvg[width=0.16667in,height=0.16667in]{/assets/icons/16-arrow-right.svg}
View example

\begin{verbatim}
#set text(font: "Inria Serif")
This is #strike(offset: auto)[low-ish]. \
This is #strike(offset: -3.5pt)[on-top].
\end{verbatim}

\includegraphics[width=5in,height=\textheight,keepaspectratio]{/assets/docs/1OEdd7_f0OE1q_8jKEVHmQAAAAAAAAAA.png}

\subsubsection{\texorpdfstring{\texttt{\ extent\ }}{ extent }}\label{parameters-extent}

\href{/docs/reference/layout/length/}{length}

{{ Settable }}

\phantomsection\label{parameters-extent-settable-tooltip}
Settable parameters can be customized for all following uses of the
function with a \texttt{\ set\ } rule.

The amount by which to extend the line beyond (or within if negative)
the content.

Default: \texttt{\ }{\texttt{\ 0pt\ }}\texttt{\ }

\includesvg[width=0.16667in,height=0.16667in]{/assets/icons/16-arrow-right.svg}
View example

\begin{verbatim}
This #strike(extent: -2pt)[skips] parts of the word.
This #strike(extent: 2pt)[extends] beyond the word.
\end{verbatim}

\includegraphics[width=5in,height=\textheight,keepaspectratio]{/assets/docs/EqeD8OvCZeei8kbI8T5T0AAAAAAAAAAA.png}

\subsubsection{\texorpdfstring{\texttt{\ background\ }}{ background }}\label{parameters-background}

\href{/docs/reference/foundations/bool/}{bool}

{{ Settable }}

\phantomsection\label{parameters-background-settable-tooltip}
Settable parameters can be customized for all following uses of the
function with a \texttt{\ set\ } rule.

Whether the line is placed behind the content.

Default: \texttt{\ }{\texttt{\ false\ }}\texttt{\ }

\includesvg[width=0.16667in,height=0.16667in]{/assets/icons/16-arrow-right.svg}
View example

\begin{verbatim}
#set strike(stroke: red)
#strike(background: true)[This is behind.] \
#strike(background: false)[This is in front.]
\end{verbatim}

\includegraphics[width=5in,height=\textheight,keepaspectratio]{/assets/docs/5BzB-6LlvrhILN951-2KuQAAAAAAAAAA.png}

\subsubsection{\texorpdfstring{\texttt{\ body\ }}{ body }}\label{parameters-body}

\href{/docs/reference/foundations/content/}{content}

{Required} {{ Positional }}

\phantomsection\label{parameters-body-positional-tooltip}
Positional parameters are specified in order, without names.

The content to strike through.

\href{/docs/reference/text/smartquote/}{\pandocbounded{\includesvg[keepaspectratio]{/assets/icons/16-arrow-right.svg}}}

{ Smartquote } { Previous page }

\href{/docs/reference/text/sub/}{\pandocbounded{\includesvg[keepaspectratio]{/assets/icons/16-arrow-right.svg}}}

{ Subscript } { Next page }


\section{Docs LaTeX/typst.app/docs/reference/text/text.tex}
\title{typst.app/docs/reference/text/text}

\begin{itemize}
\tightlist
\item
  \href{/docs}{\includesvg[width=0.16667in,height=0.16667in]{/assets/icons/16-docs-dark.svg}}
\item
  \includesvg[width=0.16667in,height=0.16667in]{/assets/icons/16-arrow-right.svg}
\item
  \href{/docs/reference/}{Reference}
\item
  \includesvg[width=0.16667in,height=0.16667in]{/assets/icons/16-arrow-right.svg}
\item
  \href{/docs/reference/text/}{Text}
\item
  \includesvg[width=0.16667in,height=0.16667in]{/assets/icons/16-arrow-right.svg}
\item
  \href{/docs/reference/text/text/}{Text}
\end{itemize}

\section{\texorpdfstring{\texttt{\ text\ } {{ Element
}}}{ text   Element }}\label{summary}

\phantomsection\label{element-tooltip}
Element functions can be customized with \texttt{\ set\ } and
\texttt{\ show\ } rules.

Customizes the look and layout of text in a variety of ways.

This function is used frequently, both with set rules and directly.
While the set rule is often the simpler choice, calling the
\texttt{\ text\ } function directly can be useful when passing text as
an argument to another function.

\subsection{Example}\label{example}

\begin{verbatim}
#set text(18pt)
With a set rule.

#emph(text(blue)[
  With a function call.
])
\end{verbatim}

\includegraphics[width=5in,height=\textheight,keepaspectratio]{/assets/docs/TE1TKvqGw3ajR6jn3phXugAAAAAAAAAA.png}

\subsection{\texorpdfstring{{ Parameters
}}{ Parameters }}\label{parameters}

\phantomsection\label{parameters-tooltip}
Parameters are the inputs to a function. They are specified in
parentheses after the function name.

{ text } (

{ \hyperref[parameters-font]{font :}
\href{/docs/reference/foundations/str/}{str}
\href{/docs/reference/foundations/array/}{array} , } {
\hyperref[parameters-fallback]{fallback :}
\href{/docs/reference/foundations/bool/}{bool} , } {
\hyperref[parameters-style]{style :}
\href{/docs/reference/foundations/str/}{str} , } {
\hyperref[parameters-weight]{weight :}
\href{/docs/reference/foundations/int/}{int}
\href{/docs/reference/foundations/str/}{str} , } {
\hyperref[parameters-stretch]{stretch :}
\href{/docs/reference/layout/ratio/}{ratio} , } {
\hyperref[parameters-size]{size :}
\href{/docs/reference/layout/length/}{length} , } {
\hyperref[parameters-fill]{fill :}
\href{/docs/reference/visualize/color/}{color}
\href{/docs/reference/visualize/gradient/}{gradient}
\href{/docs/reference/visualize/pattern/}{pattern} , } {
\hyperref[parameters-stroke]{stroke :}
\href{/docs/reference/foundations/none/}{none}
\href{/docs/reference/layout/length/}{length}
\href{/docs/reference/visualize/color/}{color}
\href{/docs/reference/visualize/gradient/}{gradient}
\href{/docs/reference/visualize/stroke/}{stroke}
\href{/docs/reference/visualize/pattern/}{pattern}
\href{/docs/reference/foundations/dictionary/}{dictionary} , } {
\hyperref[parameters-tracking]{tracking :}
\href{/docs/reference/layout/length/}{length} , } {
\hyperref[parameters-spacing]{spacing :}
\href{/docs/reference/layout/relative/}{relative} , } {
\hyperref[parameters-cjk-latin-spacing]{cjk-latin-spacing :}
\href{/docs/reference/foundations/none/}{none}
\href{/docs/reference/foundations/auto/}{auto} , } {
\hyperref[parameters-baseline]{baseline :}
\href{/docs/reference/layout/length/}{length} , } {
\hyperref[parameters-overhang]{overhang :}
\href{/docs/reference/foundations/bool/}{bool} , } {
\hyperref[parameters-top-edge]{top-edge :}
\href{/docs/reference/layout/length/}{length}
\href{/docs/reference/foundations/str/}{str} , } {
\hyperref[parameters-bottom-edge]{bottom-edge :}
\href{/docs/reference/layout/length/}{length}
\href{/docs/reference/foundations/str/}{str} , } {
\hyperref[parameters-lang]{lang :}
\href{/docs/reference/foundations/str/}{str} , } {
\hyperref[parameters-region]{region :}
\href{/docs/reference/foundations/none/}{none}
\href{/docs/reference/foundations/str/}{str} , } {
\hyperref[parameters-script]{script :}
\href{/docs/reference/foundations/auto/}{auto}
\href{/docs/reference/foundations/str/}{str} , } {
\hyperref[parameters-dir]{dir :}
\href{/docs/reference/foundations/auto/}{auto}
\href{/docs/reference/layout/direction/}{direction} , } {
\hyperref[parameters-hyphenate]{hyphenate :}
\href{/docs/reference/foundations/auto/}{auto}
\href{/docs/reference/foundations/bool/}{bool} , } {
\hyperref[parameters-costs]{costs :}
\href{/docs/reference/foundations/dictionary/}{dictionary} , } {
\hyperref[parameters-kerning]{kerning :}
\href{/docs/reference/foundations/bool/}{bool} , } {
\hyperref[parameters-alternates]{alternates :}
\href{/docs/reference/foundations/bool/}{bool} , } {
\hyperref[parameters-stylistic-set]{stylistic-set :}
\href{/docs/reference/foundations/none/}{none}
\href{/docs/reference/foundations/int/}{int}
\href{/docs/reference/foundations/array/}{array} , } {
\hyperref[parameters-ligatures]{ligatures :}
\href{/docs/reference/foundations/bool/}{bool} , } {
\hyperref[parameters-discretionary-ligatures]{discretionary-ligatures :}
\href{/docs/reference/foundations/bool/}{bool} , } {
\hyperref[parameters-historical-ligatures]{historical-ligatures :}
\href{/docs/reference/foundations/bool/}{bool} , } {
\hyperref[parameters-number-type]{number-type :}
\href{/docs/reference/foundations/auto/}{auto}
\href{/docs/reference/foundations/str/}{str} , } {
\hyperref[parameters-number-width]{number-width :}
\href{/docs/reference/foundations/auto/}{auto}
\href{/docs/reference/foundations/str/}{str} , } {
\hyperref[parameters-slashed-zero]{slashed-zero :}
\href{/docs/reference/foundations/bool/}{bool} , } {
\hyperref[parameters-fractions]{fractions :}
\href{/docs/reference/foundations/bool/}{bool} , } {
\hyperref[parameters-features]{features :}
\href{/docs/reference/foundations/array/}{array}
\href{/docs/reference/foundations/dictionary/}{dictionary} , } {
\href{/docs/reference/foundations/content/}{content} , } {
\href{/docs/reference/foundations/str/}{str} , }

) -\textgreater{} \href{/docs/reference/foundations/content/}{content}

\subsubsection{\texorpdfstring{\texttt{\ font\ }}{ font }}\label{parameters-font}

\href{/docs/reference/foundations/str/}{str} {or}
\href{/docs/reference/foundations/array/}{array}

{{ Settable }}

\phantomsection\label{parameters-font-settable-tooltip}
Settable parameters can be customized for all following uses of the
function with a \texttt{\ set\ } rule.

A font family name or priority list of font family names.

When processing text, Typst tries all specified font families in order
until it finds a font that has the necessary glyphs. In the example
below, the font \texttt{\ Inria\ Serif\ } is preferred, but since it
does not contain Arabic glyphs, the arabic text uses
\texttt{\ Noto\ Sans\ Arabic\ } instead.

The collection of available fonts differs by platform:

\begin{itemize}
\item
  In the web app, you can see the list of available fonts by clicking on
  the "Ag" button. You can provide additional fonts by uploading
  \texttt{\ .ttf\ } or \texttt{\ .otf\ } files into your project. They
  will be discovered automatically. The priority is: project fonts
  \textgreater{} server fonts.
\item
  Locally, Typst uses your installed system fonts or embedded fonts in
  the CLI, which are \texttt{\ Libertinus\ Serif\ } ,
  \texttt{\ New\ Computer\ Modern\ } ,
  \texttt{\ New\ Computer\ Modern\ Math\ } , and
  \texttt{\ DejaVu\ Sans\ Mono\ } . In addition, you can use the
  \texttt{\ -\/-font-path\ } argument or \texttt{\ TYPST\_FONT\_PATHS\ }
  environment variable to add directories that should be scanned for
  fonts. The priority is: \texttt{\ -\/-font-paths\ } \textgreater{}
  system fonts \textgreater{} embedded fonts. Run
  \texttt{\ typst\ fonts\ } to see the fonts that Typst has discovered
  on your system. Note that you can pass the
  \texttt{\ -\/-ignore-system-fonts\ } parameter to the CLI to ensure
  Typst won\textquotesingle t search for system fonts.
\end{itemize}

Default: \texttt{\ }{\texttt{\ "libertinus\ serif"\ }}\texttt{\ }

\includesvg[width=0.16667in,height=0.16667in]{/assets/icons/16-arrow-right.svg}
View example

\begin{verbatim}
#set text(font: "PT Sans")
This is sans-serif.

#set text(font: (
  "Inria Serif",
  "Noto Sans Arabic",
))

This is Latin. \
هذا عربي.
\end{verbatim}

\includegraphics[width=5in,height=\textheight,keepaspectratio]{/assets/docs/yZSlTN4UXKYq5EjwCcVgvwAAAAAAAAAA.png}

\subsubsection{\texorpdfstring{\texttt{\ fallback\ }}{ fallback }}\label{parameters-fallback}

\href{/docs/reference/foundations/bool/}{bool}

{{ Settable }}

\phantomsection\label{parameters-fallback-settable-tooltip}
Settable parameters can be customized for all following uses of the
function with a \texttt{\ set\ } rule.

Whether to allow last resort font fallback when the primary font list
contains no match. This lets Typst search through all available fonts
for the most similar one that has the necessary glyphs.

\emph{Note:} Currently, there are no warnings when fallback is disabled
and no glyphs are found. Instead, your text shows up in the form of
"tofus": Small boxes that indicate the lack of an appropriate glyph. In
the future, you will be able to instruct Typst to issue warnings so you
know something is up.

Default: \texttt{\ }{\texttt{\ true\ }}\texttt{\ }

\includesvg[width=0.16667in,height=0.16667in]{/assets/icons/16-arrow-right.svg}
View example

\begin{verbatim}
#set text(font: "Inria Serif")
هذا عربي

#set text(fallback: false)
هذا عربي
\end{verbatim}

\includegraphics[width=5in,height=\textheight,keepaspectratio]{/assets/docs/sa8VqsYbdClSlqi08qJyhAAAAAAAAAAA.png}

\subsubsection{\texorpdfstring{\texttt{\ style\ }}{ style }}\label{parameters-style}

\href{/docs/reference/foundations/str/}{str}

{{ Settable }}

\phantomsection\label{parameters-style-settable-tooltip}
Settable parameters can be customized for all following uses of the
function with a \texttt{\ set\ } rule.

The desired font style.

When an italic style is requested and only an oblique one is available,
it is used. Similarly, the other way around, an italic style can stand
in for an oblique one. When neither an italic nor an oblique style is
available, Typst selects the normal style. Since most fonts are only
available either in an italic or oblique style, the difference between
italic and oblique style is rarely observable.

If you want to emphasize your text, you should do so using the
\href{/docs/reference/model/emph/}{emph} function instead. This makes it
easy to adapt the style later if you change your mind about how to
signify the emphasis.

\begin{longtable}[]{@{}ll@{}}
\toprule\noalign{}
Variant & Details \\
\midrule\noalign{}
\endhead
\bottomrule\noalign{}
\endlastfoot
\texttt{\ "\ normal\ "\ } & The default, typically upright style. \\
\texttt{\ "\ italic\ "\ } & A cursive style with custom letterform. \\
\texttt{\ "\ oblique\ "\ } & Just a slanted version of the normal
style. \\
\end{longtable}

Default: \texttt{\ }{\texttt{\ "normal"\ }}\texttt{\ }

\includesvg[width=0.16667in,height=0.16667in]{/assets/icons/16-arrow-right.svg}
View example

\begin{verbatim}
#text(font: "Libertinus Serif", style: "italic")[Italic]
#text(font: "DejaVu Sans", style: "oblique")[Oblique]
\end{verbatim}

\includegraphics[width=5in,height=\textheight,keepaspectratio]{/assets/docs/S5xaZcVoGLtnT_0XwbPSUQAAAAAAAAAA.png}

\subsubsection{\texorpdfstring{\texttt{\ weight\ }}{ weight }}\label{parameters-weight}

\href{/docs/reference/foundations/int/}{int} {or}
\href{/docs/reference/foundations/str/}{str}

{{ Settable }}

\phantomsection\label{parameters-weight-settable-tooltip}
Settable parameters can be customized for all following uses of the
function with a \texttt{\ set\ } rule.

The desired thickness of the font\textquotesingle s glyphs. Accepts an
integer between \texttt{\ }{\texttt{\ 100\ }}\texttt{\ } and
\texttt{\ }{\texttt{\ 900\ }}\texttt{\ } or one of the predefined weight
names. When the desired weight is not available, Typst selects the font
from the family that is closest in weight.

If you want to strongly emphasize your text, you should do so using the
\href{/docs/reference/model/strong/}{strong} function instead. This
makes it easy to adapt the style later if you change your mind about how
to signify the strong emphasis.

\begin{longtable}[]{@{}ll@{}}
\toprule\noalign{}
Variant & Details \\
\midrule\noalign{}
\endhead
\bottomrule\noalign{}
\endlastfoot
\texttt{\ "\ thin\ "\ } & Thin weight (100). \\
\texttt{\ "\ extralight\ "\ } & Extra light weight (200). \\
\texttt{\ "\ light\ "\ } & Light weight (300). \\
\texttt{\ "\ regular\ "\ } & Regular weight (400). \\
\texttt{\ "\ medium\ "\ } & Medium weight (500). \\
\texttt{\ "\ semibold\ "\ } & Semibold weight (600). \\
\texttt{\ "\ bold\ "\ } & Bold weight (700). \\
\texttt{\ "\ extrabold\ "\ } & Extrabold weight (800). \\
\texttt{\ "\ black\ "\ } & Black weight (900). \\
\end{longtable}

Default: \texttt{\ }{\texttt{\ "regular"\ }}\texttt{\ }

\includesvg[width=0.16667in,height=0.16667in]{/assets/icons/16-arrow-right.svg}
View example

\begin{verbatim}
#set text(font: "IBM Plex Sans")

#text(weight: "light")[Light] \
#text(weight: "regular")[Regular] \
#text(weight: "medium")[Medium] \
#text(weight: 500)[Medium] \
#text(weight: "bold")[Bold]
\end{verbatim}

\includegraphics[width=5in,height=\textheight,keepaspectratio]{/assets/docs/HLYJEJyYVhBAwk1NcGJZjQAAAAAAAAAA.png}

\subsubsection{\texorpdfstring{\texttt{\ stretch\ }}{ stretch }}\label{parameters-stretch}

\href{/docs/reference/layout/ratio/}{ratio}

{{ Settable }}

\phantomsection\label{parameters-stretch-settable-tooltip}
Settable parameters can be customized for all following uses of the
function with a \texttt{\ set\ } rule.

The desired width of the glyphs. Accepts a ratio between
\texttt{\ }{\texttt{\ 50\%\ }}\texttt{\ } and
\texttt{\ }{\texttt{\ 200\%\ }}\texttt{\ } . When the desired width is
not available, Typst selects the font from the family that is closest in
stretch. This will only stretch the text if a condensed or expanded
version of the font is available.

If you want to adjust the amount of space between characters instead of
stretching the glyphs itself, use the
\href{/docs/reference/text/text/\#parameters-tracking}{\texttt{\ tracking\ }}
property instead.

Default: \texttt{\ }{\texttt{\ 100\%\ }}\texttt{\ }

\includesvg[width=0.16667in,height=0.16667in]{/assets/icons/16-arrow-right.svg}
View example

\begin{verbatim}
#text(stretch: 75%)[Condensed] \
#text(stretch: 100%)[Normal]
\end{verbatim}

\includegraphics[width=5in,height=\textheight,keepaspectratio]{/assets/docs/QhcCPECtjtdl-HaT2kdIoQAAAAAAAAAA.png}

\subsubsection{\texorpdfstring{\texttt{\ size\ }}{ size }}\label{parameters-size}

\href{/docs/reference/layout/length/}{length}

{{ Settable }}

\phantomsection\label{parameters-size-settable-tooltip}
Settable parameters can be customized for all following uses of the
function with a \texttt{\ set\ } rule.

The size of the glyphs. This value forms the basis of the
\texttt{\ em\ } unit: \texttt{\ }{\texttt{\ 1em\ }}\texttt{\ } is
equivalent to the font size.

You can also give the font size itself in \texttt{\ em\ } units. Then,
it is relative to the previous font size.

Default: \texttt{\ }{\texttt{\ 11pt\ }}\texttt{\ }

\includesvg[width=0.16667in,height=0.16667in]{/assets/icons/16-arrow-right.svg}
View example

\begin{verbatim}
#set text(size: 20pt)
very #text(1.5em)[big] text
\end{verbatim}

\includegraphics[width=5in,height=\textheight,keepaspectratio]{/assets/docs/blheA65DgOU1lkslOoHidgAAAAAAAAAA.png}

\subsubsection{\texorpdfstring{\texttt{\ fill\ }}{ fill }}\label{parameters-fill}

\href{/docs/reference/visualize/color/}{color} {or}
\href{/docs/reference/visualize/gradient/}{gradient} {or}
\href{/docs/reference/visualize/pattern/}{pattern}

{{ Settable }}

\phantomsection\label{parameters-fill-settable-tooltip}
Settable parameters can be customized for all following uses of the
function with a \texttt{\ set\ } rule.

The glyph fill paint.

Default:
\texttt{\ }{\texttt{\ luma\ }}\texttt{\ }{\texttt{\ (\ }}\texttt{\ }{\texttt{\ 0\%\ }}\texttt{\ }{\texttt{\ )\ }}\texttt{\ }

\includesvg[width=0.16667in,height=0.16667in]{/assets/icons/16-arrow-right.svg}
View example

\begin{verbatim}
#set text(fill: red)
This text is red.
\end{verbatim}

\includegraphics[width=5in,height=\textheight,keepaspectratio]{/assets/docs/hjdTrz3B1HnAtRXCRTTtGAAAAAAAAAAA.png}

\subsubsection{\texorpdfstring{\texttt{\ stroke\ }}{ stroke }}\label{parameters-stroke}

\href{/docs/reference/foundations/none/}{none} {or}
\href{/docs/reference/layout/length/}{length} {or}
\href{/docs/reference/visualize/color/}{color} {or}
\href{/docs/reference/visualize/gradient/}{gradient} {or}
\href{/docs/reference/visualize/stroke/}{stroke} {or}
\href{/docs/reference/visualize/pattern/}{pattern} {or}
\href{/docs/reference/foundations/dictionary/}{dictionary}

{{ Settable }}

\phantomsection\label{parameters-stroke-settable-tooltip}
Settable parameters can be customized for all following uses of the
function with a \texttt{\ set\ } rule.

How to stroke the text.

Default: \texttt{\ }{\texttt{\ none\ }}\texttt{\ }

\includesvg[width=0.16667in,height=0.16667in]{/assets/icons/16-arrow-right.svg}
View example

\begin{verbatim}
#text(stroke: 0.5pt + red)[Stroked]
\end{verbatim}

\includegraphics[width=5in,height=\textheight,keepaspectratio]{/assets/docs/9XI8EQ1M6rOusSDRRIbaPQAAAAAAAAAA.png}

\subsubsection{\texorpdfstring{\texttt{\ tracking\ }}{ tracking }}\label{parameters-tracking}

\href{/docs/reference/layout/length/}{length}

{{ Settable }}

\phantomsection\label{parameters-tracking-settable-tooltip}
Settable parameters can be customized for all following uses of the
function with a \texttt{\ set\ } rule.

The amount of space that should be added between characters.

Default: \texttt{\ }{\texttt{\ 0pt\ }}\texttt{\ }

\includesvg[width=0.16667in,height=0.16667in]{/assets/icons/16-arrow-right.svg}
View example

\begin{verbatim}
#set text(tracking: 1.5pt)
Distant text.
\end{verbatim}

\includegraphics[width=5in,height=\textheight,keepaspectratio]{/assets/docs/_W5ZMMvgiXlv5B8vI6sbcQAAAAAAAAAA.png}

\subsubsection{\texorpdfstring{\texttt{\ spacing\ }}{ spacing }}\label{parameters-spacing}

\href{/docs/reference/layout/relative/}{relative}

{{ Settable }}

\phantomsection\label{parameters-spacing-settable-tooltip}
Settable parameters can be customized for all following uses of the
function with a \texttt{\ set\ } rule.

The amount of space between words.

Can be given as an absolute length, but also relative to the width of
the space character in the font.

If you want to adjust the amount of space between characters rather than
words, use the
\href{/docs/reference/text/text/\#parameters-tracking}{\texttt{\ tracking\ }}
property instead.

Default:
\texttt{\ }{\texttt{\ 100\%\ }}\texttt{\ }{\texttt{\ +\ }}\texttt{\ }{\texttt{\ 0pt\ }}\texttt{\ }

\includesvg[width=0.16667in,height=0.16667in]{/assets/icons/16-arrow-right.svg}
View example

\begin{verbatim}
#set text(spacing: 200%)
Text with distant words.
\end{verbatim}

\includegraphics[width=5in,height=\textheight,keepaspectratio]{/assets/docs/NLatl7xe_PftXpK8eI1WSgAAAAAAAAAA.png}

\subsubsection{\texorpdfstring{\texttt{\ cjk-latin-spacing\ }}{ cjk-latin-spacing }}\label{parameters-cjk-latin-spacing}

\href{/docs/reference/foundations/none/}{none} {or}
\href{/docs/reference/foundations/auto/}{auto}

{{ Settable }}

\phantomsection\label{parameters-cjk-latin-spacing-settable-tooltip}
Settable parameters can be customized for all following uses of the
function with a \texttt{\ set\ } rule.

Whether to automatically insert spacing between CJK and Latin
characters.

Default: \texttt{\ }{\texttt{\ auto\ }}\texttt{\ }

\includesvg[width=0.16667in,height=0.16667in]{/assets/icons/16-arrow-right.svg}
View example

\begin{verbatim}
#set text(cjk-latin-spacing: auto)
第4章介绍了基本的API。

#set text(cjk-latin-spacing: none)
第4章介绍了基本的API。
\end{verbatim}

\includegraphics[width=5in,height=\textheight,keepaspectratio]{/assets/docs/VxUeM1bvsLzygleocZmQUAAAAAAAAAAA.png}

\subsubsection{\texorpdfstring{\texttt{\ baseline\ }}{ baseline }}\label{parameters-baseline}

\href{/docs/reference/layout/length/}{length}

{{ Settable }}

\phantomsection\label{parameters-baseline-settable-tooltip}
Settable parameters can be customized for all following uses of the
function with a \texttt{\ set\ } rule.

An amount to shift the text baseline by.

Default: \texttt{\ }{\texttt{\ 0pt\ }}\texttt{\ }

\includesvg[width=0.16667in,height=0.16667in]{/assets/icons/16-arrow-right.svg}
View example

\begin{verbatim}
A #text(baseline: 3pt)[lowered]
word.
\end{verbatim}

\includegraphics[width=5in,height=\textheight,keepaspectratio]{/assets/docs/Kc1E9Ts9m1i30dvtf5ymQgAAAAAAAAAA.png}

\subsubsection{\texorpdfstring{\texttt{\ overhang\ }}{ overhang }}\label{parameters-overhang}

\href{/docs/reference/foundations/bool/}{bool}

{{ Settable }}

\phantomsection\label{parameters-overhang-settable-tooltip}
Settable parameters can be customized for all following uses of the
function with a \texttt{\ set\ } rule.

Whether certain glyphs can hang over into the margin in justified text.
This can make justification visually more pleasing.

Default: \texttt{\ }{\texttt{\ true\ }}\texttt{\ }

\includesvg[width=0.16667in,height=0.16667in]{/assets/icons/16-arrow-right.svg}
View example

\begin{verbatim}
#set par(justify: true)
This justified text has a hyphen in
the paragraph's first line. Hanging
the hyphen slightly into the margin
results in a clearer paragraph edge.

#set text(overhang: false)
This justified text has a hyphen in
the paragraph's first line. Hanging
the hyphen slightly into the margin
results in a clearer paragraph edge.
\end{verbatim}

\includegraphics[width=5in,height=\textheight,keepaspectratio]{/assets/docs/MnBRs6VvAtjUYVDK-btjfgAAAAAAAAAA.png}

\subsubsection{\texorpdfstring{\texttt{\ top-edge\ }}{ top-edge }}\label{parameters-top-edge}

\href{/docs/reference/layout/length/}{length} {or}
\href{/docs/reference/foundations/str/}{str}

{{ Settable }}

\phantomsection\label{parameters-top-edge-settable-tooltip}
Settable parameters can be customized for all following uses of the
function with a \texttt{\ set\ } rule.

The top end of the conceptual frame around the text used for layout and
positioning. This affects the size of containers that hold text.

\begin{longtable}[]{@{}ll@{}}
\toprule\noalign{}
Variant & Details \\
\midrule\noalign{}
\endhead
\bottomrule\noalign{}
\endlastfoot
\texttt{\ "\ ascender\ "\ } & The font\textquotesingle s ascender, which
typically exceeds the height of all glyphs. \\
\texttt{\ "\ cap-height\ "\ } & The approximate height of uppercase
letters. \\
\texttt{\ "\ x-height\ "\ } & The approximate height of non-ascending
lowercase letters. \\
\texttt{\ "\ baseline\ "\ } & The baseline on which the letters rest. \\
\texttt{\ "\ bounds\ "\ } & The top edge of the glyph\textquotesingle s
bounding box. \\
\end{longtable}

Default: \texttt{\ }{\texttt{\ "cap-height"\ }}\texttt{\ }

\includesvg[width=0.16667in,height=0.16667in]{/assets/icons/16-arrow-right.svg}
View example

\begin{verbatim}
#set rect(inset: 0pt)
#set text(size: 20pt)

#set text(top-edge: "ascender")
#rect(fill: aqua)[Typst]

#set text(top-edge: "cap-height")
#rect(fill: aqua)[Typst]
\end{verbatim}

\includegraphics[width=5in,height=\textheight,keepaspectratio]{/assets/docs/LDeMc2Iiqb_9L3aj1lNrpgAAAAAAAAAA.png}

\subsubsection{\texorpdfstring{\texttt{\ bottom-edge\ }}{ bottom-edge }}\label{parameters-bottom-edge}

\href{/docs/reference/layout/length/}{length} {or}
\href{/docs/reference/foundations/str/}{str}

{{ Settable }}

\phantomsection\label{parameters-bottom-edge-settable-tooltip}
Settable parameters can be customized for all following uses of the
function with a \texttt{\ set\ } rule.

The bottom end of the conceptual frame around the text used for layout
and positioning. This affects the size of containers that hold text.

\begin{longtable}[]{@{}ll@{}}
\toprule\noalign{}
Variant & Details \\
\midrule\noalign{}
\endhead
\bottomrule\noalign{}
\endlastfoot
\texttt{\ "\ baseline\ "\ } & The baseline on which the letters rest. \\
\texttt{\ "\ descender\ "\ } & The font\textquotesingle s descender,
which typically exceeds the depth of all glyphs. \\
\texttt{\ "\ bounds\ "\ } & The bottom edge of the
glyph\textquotesingle s bounding box. \\
\end{longtable}

Default: \texttt{\ }{\texttt{\ "baseline"\ }}\texttt{\ }

\includesvg[width=0.16667in,height=0.16667in]{/assets/icons/16-arrow-right.svg}
View example

\begin{verbatim}
#set rect(inset: 0pt)
#set text(size: 20pt)

#set text(bottom-edge: "baseline")
#rect(fill: aqua)[Typst]

#set text(bottom-edge: "descender")
#rect(fill: aqua)[Typst]
\end{verbatim}

\includegraphics[width=5in,height=\textheight,keepaspectratio]{/assets/docs/l4WLB64gFfplM-bDPX7pEQAAAAAAAAAA.png}

\subsubsection{\texorpdfstring{\texttt{\ lang\ }}{ lang }}\label{parameters-lang}

\href{/docs/reference/foundations/str/}{str}

{{ Settable }}

\phantomsection\label{parameters-lang-settable-tooltip}
Settable parameters can be customized for all following uses of the
function with a \texttt{\ set\ } rule.

An \href{https://en.wikipedia.org/wiki/ISO_639}{ISO 639-1/2/3 language
code.}

Setting the correct language affects various parts of Typst:

\begin{itemize}
\tightlist
\item
  The text processing pipeline can make more informed choices.
\item
  Hyphenation will use the correct patterns for the language.
\item
  \href{/docs/reference/text/smartquote/}{Smart quotes} turns into the
  correct quotes for the language.
\item
  And all other things which are language-aware.
\end{itemize}

Default: \texttt{\ }{\texttt{\ "en"\ }}\texttt{\ }

\includesvg[width=0.16667in,height=0.16667in]{/assets/icons/16-arrow-right.svg}
View example

\begin{verbatim}
#set text(lang: "de")
#outline()

= Einleitung
In diesem Dokument, ...
\end{verbatim}

\includegraphics[width=5in,height=\textheight,keepaspectratio]{/assets/docs/pV_uneCLTlX_ftfk4ZJI1QAAAAAAAAAA.png}

\subsubsection{\texorpdfstring{\texttt{\ region\ }}{ region }}\label{parameters-region}

\href{/docs/reference/foundations/none/}{none} {or}
\href{/docs/reference/foundations/str/}{str}

{{ Settable }}

\phantomsection\label{parameters-region-settable-tooltip}
Settable parameters can be customized for all following uses of the
function with a \texttt{\ set\ } rule.

An \href{https://en.wikipedia.org/wiki/ISO_3166-1_alpha-2}{ISO 3166-1
alpha-2 region code.}

This lets the text processing pipeline make more informed choices.

Default: \texttt{\ }{\texttt{\ none\ }}\texttt{\ }

\subsubsection{\texorpdfstring{\texttt{\ script\ }}{ script }}\label{parameters-script}

\href{/docs/reference/foundations/auto/}{auto} {or}
\href{/docs/reference/foundations/str/}{str}

{{ Settable }}

\phantomsection\label{parameters-script-settable-tooltip}
Settable parameters can be customized for all following uses of the
function with a \texttt{\ set\ } rule.

The OpenType writing script.

The combination of \texttt{\ lang\ } and \texttt{\ script\ } determine
how font features, such as glyph substitution, are implemented.
Frequently the value is a modified (all-lowercase) ISO 15924 script
identifier, and the \texttt{\ math\ } writing script is used for
features appropriate for mathematical symbols.

When set to \texttt{\ }{\texttt{\ auto\ }}\texttt{\ } , the default and
recommended setting, an appropriate script is chosen for each block of
characters sharing a common Unicode script property.

Default: \texttt{\ }{\texttt{\ auto\ }}\texttt{\ }

\includesvg[width=0.16667in,height=0.16667in]{/assets/icons/16-arrow-right.svg}
View example

\begin{verbatim}
#set text(
  font: "Libertinus Serif",
  size: 20pt,
)

#let scedilla = [Ş]
#scedilla // S with a cedilla

#set text(lang: "ro", script: "latn")
#scedilla // S with a subscript comma

#set text(lang: "ro", script: "grek")
#scedilla // S with a cedilla
\end{verbatim}

\includegraphics[width=5in,height=\textheight,keepaspectratio]{/assets/docs/IJovpbe1c5rRr9DM_KRhvgAAAAAAAAAA.png}

\subsubsection{\texorpdfstring{\texttt{\ dir\ }}{ dir }}\label{parameters-dir}

\href{/docs/reference/foundations/auto/}{auto} {or}
\href{/docs/reference/layout/direction/}{direction}

{{ Settable }}

\phantomsection\label{parameters-dir-settable-tooltip}
Settable parameters can be customized for all following uses of the
function with a \texttt{\ set\ } rule.

The dominant direction for text and inline objects. Possible values are:

\begin{itemize}
\tightlist
\item
  \texttt{\ }{\texttt{\ auto\ }}\texttt{\ } : Automatically infer the
  direction from the \texttt{\ lang\ } property.
\item
  \texttt{\ ltr\ } : Layout text from left to right.
\item
  \texttt{\ rtl\ } : Layout text from right to left.
\end{itemize}

When writing in right-to-left scripts like Arabic or Hebrew, you should
set the \href{/docs/reference/text/text/\#parameters-lang}{text
language} or direction. While individual runs of text are automatically
layouted in the correct direction, setting the dominant direction gives
the bidirectional reordering algorithm the necessary information to
correctly place punctuation and inline objects. Furthermore, setting the
direction affects the alignment values \texttt{\ start\ } and
\texttt{\ end\ } , which are equivalent to \texttt{\ left\ } and
\texttt{\ right\ } in \texttt{\ ltr\ } text and the other way around in
\texttt{\ rtl\ } text.

If you set this to \texttt{\ rtl\ } and experience bugs or in some way
bad looking output, please get in touch with us through the
\href{https://forum.typst.app/}{Forum} ,
\href{https://discord.gg/2uDybryKPe}{Discord server} , or our
\href{https://typst.app/contact}{contact form} .

Default: \texttt{\ }{\texttt{\ auto\ }}\texttt{\ }

\includesvg[width=0.16667in,height=0.16667in]{/assets/icons/16-arrow-right.svg}
View example

\begin{verbatim}
#set text(dir: rtl)
هذا عربي.
\end{verbatim}

\includegraphics[width=5in,height=\textheight,keepaspectratio]{/assets/docs/KrWAMeKAPNsts-l34CremAAAAAAAAAAA.png}

\subsubsection{\texorpdfstring{\texttt{\ hyphenate\ }}{ hyphenate }}\label{parameters-hyphenate}

\href{/docs/reference/foundations/auto/}{auto} {or}
\href{/docs/reference/foundations/bool/}{bool}

{{ Settable }}

\phantomsection\label{parameters-hyphenate-settable-tooltip}
Settable parameters can be customized for all following uses of the
function with a \texttt{\ set\ } rule.

Whether to hyphenate text to improve line breaking. When
\texttt{\ }{\texttt{\ auto\ }}\texttt{\ } , text will be hyphenated if
and only if justification is enabled.

Setting the \href{/docs/reference/text/text/\#parameters-lang}{text
language} ensures that the correct hyphenation patterns are used.

Default: \texttt{\ }{\texttt{\ auto\ }}\texttt{\ }

\includesvg[width=0.16667in,height=0.16667in]{/assets/icons/16-arrow-right.svg}
View example

\begin{verbatim}
#set page(width: 200pt)

#set par(justify: true)
This text illustrates how
enabling hyphenation can
improve justification.

#set text(hyphenate: false)
This text illustrates how
enabling hyphenation can
improve justification.
\end{verbatim}

\includegraphics[width=4.16667in,height=\textheight,keepaspectratio]{/assets/docs/4Pafis8Dv1GSWE8dIkAx2wAAAAAAAAAA.png}

\subsubsection{\texorpdfstring{\texttt{\ costs\ }}{ costs }}\label{parameters-costs}

\href{/docs/reference/foundations/dictionary/}{dictionary}

{{ Settable }}

\phantomsection\label{parameters-costs-settable-tooltip}
Settable parameters can be customized for all following uses of the
function with a \texttt{\ set\ } rule.

The "cost" of various choices when laying out text. A higher cost means
the layout engine will make the choice less often. Costs are specified
as a ratio of the default cost, so
\texttt{\ }{\texttt{\ 50\%\ }}\texttt{\ } will make text layout twice as
eager to make a given choice, while
\texttt{\ }{\texttt{\ 200\%\ }}\texttt{\ } will make it half as eager.

Currently, the following costs can be customized:

\begin{itemize}
\tightlist
\item
  \texttt{\ hyphenation\ } : splitting a word across multiple lines
\item
  \texttt{\ runt\ } : ending a paragraph with a line with a single word
\item
  \texttt{\ widow\ } : leaving a single line of paragraph on the next
  page
\item
  \texttt{\ orphan\ } : leaving single line of paragraph on the previous
  page
\end{itemize}

Hyphenation is generally avoided by placing the whole word on the next
line, so a higher hyphenation cost can result in awkward justification
spacing.

Runts are avoided by placing more or fewer words on previous lines, so a
higher runt cost can result in more awkward in justification spacing.

Text layout prevents widows and orphans by default because they are
generally discouraged by style guides. However, in some contexts they
are allowed because the prevention method, which moves a line to the
next page, can result in an uneven number of lines between pages. The
\texttt{\ widow\ } and \texttt{\ orphan\ } costs allow disabling these
modifications. (Currently, \texttt{\ }{\texttt{\ 0\%\ }}\texttt{\ }
allows widows/orphans; anything else, including the default of
\texttt{\ }{\texttt{\ 100\%\ }}\texttt{\ } , prevents them. More nuanced
cost specification for these modifications is planned for the future.)

Default:
\texttt{\ }{\texttt{\ (\ }}\texttt{\ hyphenation\ }{\texttt{\ :\ }}\texttt{\ }{\texttt{\ 100\%\ }}\texttt{\ }{\texttt{\ ,\ }}\texttt{\ runt\ }{\texttt{\ :\ }}\texttt{\ }{\texttt{\ 100\%\ }}\texttt{\ }{\texttt{\ ,\ }}\texttt{\ widow\ }{\texttt{\ :\ }}\texttt{\ }{\texttt{\ 100\%\ }}\texttt{\ }{\texttt{\ ,\ }}\texttt{\ orphan\ }{\texttt{\ :\ }}\texttt{\ }{\texttt{\ 100\%\ }}\texttt{\ }{\texttt{\ ,\ }}\texttt{\ }{\texttt{\ )\ }}\texttt{\ }

\includesvg[width=0.16667in,height=0.16667in]{/assets/icons/16-arrow-right.svg}
View example

\begin{verbatim}
#set text(hyphenate: true, size: 11.4pt)
#set par(justify: true)

#lorem(10)

// Set hyphenation to ten times the normal cost.
#set text(costs: (hyphenation: 1000%))

#lorem(10)
\end{verbatim}

\includegraphics[width=5in,height=\textheight,keepaspectratio]{/assets/docs/k9JLw8qIVUINakYrnv50nAAAAAAAAAAA.png}

\subsubsection{\texorpdfstring{\texttt{\ kerning\ }}{ kerning }}\label{parameters-kerning}

\href{/docs/reference/foundations/bool/}{bool}

{{ Settable }}

\phantomsection\label{parameters-kerning-settable-tooltip}
Settable parameters can be customized for all following uses of the
function with a \texttt{\ set\ } rule.

Whether to apply kerning.

When enabled, specific letter pairings move closer together or further
apart for a more visually pleasing result. The example below
demonstrates how decreasing the gap between the "T" and "o" results in a
more natural look. Setting this to
\texttt{\ }{\texttt{\ false\ }}\texttt{\ } disables kerning by turning
off the OpenType \texttt{\ kern\ } font feature.

Default: \texttt{\ }{\texttt{\ true\ }}\texttt{\ }

\includesvg[width=0.16667in,height=0.16667in]{/assets/icons/16-arrow-right.svg}
View example

\begin{verbatim}
#set text(size: 25pt)
Totally

#set text(kerning: false)
Totally
\end{verbatim}

\includegraphics[width=5in,height=\textheight,keepaspectratio]{/assets/docs/7Gj4TjnwP0QfSOeJi7dKdAAAAAAAAAAA.png}

\subsubsection{\texorpdfstring{\texttt{\ alternates\ }}{ alternates }}\label{parameters-alternates}

\href{/docs/reference/foundations/bool/}{bool}

{{ Settable }}

\phantomsection\label{parameters-alternates-settable-tooltip}
Settable parameters can be customized for all following uses of the
function with a \texttt{\ set\ } rule.

Whether to apply stylistic alternates.

Sometimes fonts contain alternative glyphs for the same codepoint.
Setting this to \texttt{\ }{\texttt{\ true\ }}\texttt{\ } switches to
these by enabling the OpenType \texttt{\ salt\ } font feature.

Default: \texttt{\ }{\texttt{\ false\ }}\texttt{\ }

\includesvg[width=0.16667in,height=0.16667in]{/assets/icons/16-arrow-right.svg}
View example

\begin{verbatim}
#set text(
  font: "IBM Plex Sans",
  size: 20pt,
)

0, a, g, ß

#set text(alternates: true)
0, a, g, ß
\end{verbatim}

\includegraphics[width=5in,height=\textheight,keepaspectratio]{/assets/docs/I0B88ggX_x3jq5W1mvWVIgAAAAAAAAAA.png}

\subsubsection{\texorpdfstring{\texttt{\ stylistic-set\ }}{ stylistic-set }}\label{parameters-stylistic-set}

\href{/docs/reference/foundations/none/}{none} {or}
\href{/docs/reference/foundations/int/}{int} {or}
\href{/docs/reference/foundations/array/}{array}

{{ Settable }}

\phantomsection\label{parameters-stylistic-set-settable-tooltip}
Settable parameters can be customized for all following uses of the
function with a \texttt{\ set\ } rule.

Which stylistic sets to apply. Font designers can categorize alternative
glyphs forms into stylistic sets. As this value is highly font-specific,
you need to consult your font to know which sets are available.

This can be set to an integer or an array of integers, all of which must
be between \texttt{\ }{\texttt{\ 1\ }}\texttt{\ } and
\texttt{\ }{\texttt{\ 20\ }}\texttt{\ } , enabling the corresponding
OpenType feature(s) from \texttt{\ ss01\ } to \texttt{\ ss20\ } .
Setting this to \texttt{\ }{\texttt{\ none\ }}\texttt{\ } will disable
all stylistic sets.

Default:
\texttt{\ }{\texttt{\ (\ }}\texttt{\ }{\texttt{\ )\ }}\texttt{\ }

\includesvg[width=0.16667in,height=0.16667in]{/assets/icons/16-arrow-right.svg}
View example

\begin{verbatim}
#set text(font: "IBM Plex Serif")
ß vs #text(stylistic-set: 5)[ß] \
10 years ago vs #text(stylistic-set: (1, 2, 3))[10 years ago]
\end{verbatim}

\includegraphics[width=5in,height=\textheight,keepaspectratio]{/assets/docs/W4bB6oEym3iwH_NdeQRsEAAAAAAAAAAA.png}

\subsubsection{\texorpdfstring{\texttt{\ ligatures\ }}{ ligatures }}\label{parameters-ligatures}

\href{/docs/reference/foundations/bool/}{bool}

{{ Settable }}

\phantomsection\label{parameters-ligatures-settable-tooltip}
Settable parameters can be customized for all following uses of the
function with a \texttt{\ set\ } rule.

Whether standard ligatures are active.

Certain letter combinations like "fi" are often displayed as a single
merged glyph called a \emph{ligature.} Setting this to
\texttt{\ }{\texttt{\ false\ }}\texttt{\ } disables these ligatures by
turning off the OpenType \texttt{\ liga\ } and \texttt{\ clig\ } font
features.

Default: \texttt{\ }{\texttt{\ true\ }}\texttt{\ }

\includesvg[width=0.16667in,height=0.16667in]{/assets/icons/16-arrow-right.svg}
View example

\begin{verbatim}
#set text(size: 20pt)
A fine ligature.

#set text(ligatures: false)
A fine ligature.
\end{verbatim}

\includegraphics[width=5in,height=\textheight,keepaspectratio]{/assets/docs/IQnLFKoKsoxRyhR3pSbfiwAAAAAAAAAA.png}

\subsubsection{\texorpdfstring{\texttt{\ discretionary-ligatures\ }}{ discretionary-ligatures }}\label{parameters-discretionary-ligatures}

\href{/docs/reference/foundations/bool/}{bool}

{{ Settable }}

\phantomsection\label{parameters-discretionary-ligatures-settable-tooltip}
Settable parameters can be customized for all following uses of the
function with a \texttt{\ set\ } rule.

Whether ligatures that should be used sparingly are active. Setting this
to \texttt{\ }{\texttt{\ true\ }}\texttt{\ } enables the OpenType
\texttt{\ dlig\ } font feature.

Default: \texttt{\ }{\texttt{\ false\ }}\texttt{\ }

\subsubsection{\texorpdfstring{\texttt{\ historical-ligatures\ }}{ historical-ligatures }}\label{parameters-historical-ligatures}

\href{/docs/reference/foundations/bool/}{bool}

{{ Settable }}

\phantomsection\label{parameters-historical-ligatures-settable-tooltip}
Settable parameters can be customized for all following uses of the
function with a \texttt{\ set\ } rule.

Whether historical ligatures are active. Setting this to
\texttt{\ }{\texttt{\ true\ }}\texttt{\ } enables the OpenType
\texttt{\ hlig\ } font feature.

Default: \texttt{\ }{\texttt{\ false\ }}\texttt{\ }

\subsubsection{\texorpdfstring{\texttt{\ number-type\ }}{ number-type }}\label{parameters-number-type}

\href{/docs/reference/foundations/auto/}{auto} {or}
\href{/docs/reference/foundations/str/}{str}

{{ Settable }}

\phantomsection\label{parameters-number-type-settable-tooltip}
Settable parameters can be customized for all following uses of the
function with a \texttt{\ set\ } rule.

Which kind of numbers / figures to select. When set to
\texttt{\ }{\texttt{\ auto\ }}\texttt{\ } , the default numbers for the
font are used.

\begin{longtable}[]{@{}ll@{}}
\toprule\noalign{}
Variant & Details \\
\midrule\noalign{}
\endhead
\bottomrule\noalign{}
\endlastfoot
\texttt{\ "\ lining\ "\ } & Numbers that fit well with capital text (the
OpenType \texttt{\ lnum\ } font feature). \\
\texttt{\ "\ old-style\ "\ } & Numbers that fit well into a flow of
upper- and lowercase text (the OpenType \texttt{\ onum\ } font
feature). \\
\end{longtable}

Default: \texttt{\ }{\texttt{\ auto\ }}\texttt{\ }

\includesvg[width=0.16667in,height=0.16667in]{/assets/icons/16-arrow-right.svg}
View example

\begin{verbatim}
#set text(font: "Noto Sans", 20pt)
#set text(number-type: "lining")
Number 9.

#set text(number-type: "old-style")
Number 9.
\end{verbatim}

\includegraphics[width=5in,height=\textheight,keepaspectratio]{/assets/docs/Jl5yPI_4pX3UbcVjqxI5_QAAAAAAAAAA.png}

\subsubsection{\texorpdfstring{\texttt{\ number-width\ }}{ number-width }}\label{parameters-number-width}

\href{/docs/reference/foundations/auto/}{auto} {or}
\href{/docs/reference/foundations/str/}{str}

{{ Settable }}

\phantomsection\label{parameters-number-width-settable-tooltip}
Settable parameters can be customized for all following uses of the
function with a \texttt{\ set\ } rule.

The width of numbers / figures. When set to
\texttt{\ }{\texttt{\ auto\ }}\texttt{\ } , the default numbers for the
font are used.

\begin{longtable}[]{@{}ll@{}}
\toprule\noalign{}
Variant & Details \\
\midrule\noalign{}
\endhead
\bottomrule\noalign{}
\endlastfoot
\texttt{\ "\ proportional\ "\ } & Numbers with glyph-specific widths
(the OpenType \texttt{\ pnum\ } font feature). \\
\texttt{\ "\ tabular\ "\ } & Numbers of equal width (the OpenType
\texttt{\ tnum\ } font feature). \\
\end{longtable}

Default: \texttt{\ }{\texttt{\ auto\ }}\texttt{\ }

\includesvg[width=0.16667in,height=0.16667in]{/assets/icons/16-arrow-right.svg}
View example

\begin{verbatim}
#set text(font: "Noto Sans", 20pt)
#set text(number-width: "proportional")
A 12 B 34. \
A 56 B 78.

#set text(number-width: "tabular")
A 12 B 34. \
A 56 B 78.
\end{verbatim}

\includegraphics[width=5in,height=\textheight,keepaspectratio]{/assets/docs/6iCMWj0AW9bSFKBJ48tdiwAAAAAAAAAA.png}

\subsubsection{\texorpdfstring{\texttt{\ slashed-zero\ }}{ slashed-zero }}\label{parameters-slashed-zero}

\href{/docs/reference/foundations/bool/}{bool}

{{ Settable }}

\phantomsection\label{parameters-slashed-zero-settable-tooltip}
Settable parameters can be customized for all following uses of the
function with a \texttt{\ set\ } rule.

Whether to have a slash through the zero glyph. Setting this to
\texttt{\ }{\texttt{\ true\ }}\texttt{\ } enables the OpenType
\texttt{\ zero\ } font feature.

Default: \texttt{\ }{\texttt{\ false\ }}\texttt{\ }

\includesvg[width=0.16667in,height=0.16667in]{/assets/icons/16-arrow-right.svg}
View example

\begin{verbatim}
0, #text(slashed-zero: true)[0]
\end{verbatim}

\includegraphics[width=5in,height=\textheight,keepaspectratio]{/assets/docs/NqkvE1KDtvrmSgKJnmfRWwAAAAAAAAAA.png}

\subsubsection{\texorpdfstring{\texttt{\ fractions\ }}{ fractions }}\label{parameters-fractions}

\href{/docs/reference/foundations/bool/}{bool}

{{ Settable }}

\phantomsection\label{parameters-fractions-settable-tooltip}
Settable parameters can be customized for all following uses of the
function with a \texttt{\ set\ } rule.

Whether to turn numbers into fractions. Setting this to
\texttt{\ }{\texttt{\ true\ }}\texttt{\ } enables the OpenType
\texttt{\ frac\ } font feature.

It is not advisable to enable this property globally as it will mess
with all appearances of numbers after a slash (e.g., in URLs). Instead,
enable it locally when you want a fraction.

Default: \texttt{\ }{\texttt{\ false\ }}\texttt{\ }

\includesvg[width=0.16667in,height=0.16667in]{/assets/icons/16-arrow-right.svg}
View example

\begin{verbatim}
1/2 \
#text(fractions: true)[1/2]
\end{verbatim}

\includegraphics[width=5in,height=\textheight,keepaspectratio]{/assets/docs/ReL3WGljBzDnfTHeymCXGQAAAAAAAAAA.png}

\subsubsection{\texorpdfstring{\texttt{\ features\ }}{ features }}\label{parameters-features}

\href{/docs/reference/foundations/array/}{array} {or}
\href{/docs/reference/foundations/dictionary/}{dictionary}

{{ Settable }}

\phantomsection\label{parameters-features-settable-tooltip}
Settable parameters can be customized for all following uses of the
function with a \texttt{\ set\ } rule.

Raw OpenType features to apply.

\begin{itemize}
\tightlist
\item
  If given an array of strings, sets the features identified by the
  strings to \texttt{\ }{\texttt{\ 1\ }}\texttt{\ } .
\item
  If given a dictionary mapping to numbers, sets the features identified
  by the keys to the values.
\end{itemize}

Default:
\texttt{\ }{\texttt{\ (\ }}\texttt{\ }{\texttt{\ :\ }}\texttt{\ }{\texttt{\ )\ }}\texttt{\ }

\includesvg[width=0.16667in,height=0.16667in]{/assets/icons/16-arrow-right.svg}
View example

\begin{verbatim}
// Enable the `frac` feature manually.
#set text(features: ("frac",))
1/2
\end{verbatim}

\includegraphics[width=5in,height=\textheight,keepaspectratio]{/assets/docs/YY_AfHqvOwZWtTBzfgDvMwAAAAAAAAAA.png}

\subsubsection{\texorpdfstring{\texttt{\ body\ }}{ body }}\label{parameters-body}

\href{/docs/reference/foundations/content/}{content}

{Required} {{ Positional }}

\phantomsection\label{parameters-body-positional-tooltip}
Positional parameters are specified in order, without names.

Content in which all text is styled according to the other arguments.

\subsubsection{\texorpdfstring{\texttt{\ text\ }}{ text }}\label{parameters-text}

\href{/docs/reference/foundations/str/}{str}

{Required} {{ Positional }}

\phantomsection\label{parameters-text-positional-tooltip}
Positional parameters are specified in order, without names.

The text.

\href{/docs/reference/text/super/}{\pandocbounded{\includesvg[keepaspectratio]{/assets/icons/16-arrow-right.svg}}}

{ Superscript } { Previous page }

\href{/docs/reference/text/underline/}{\pandocbounded{\includesvg[keepaspectratio]{/assets/icons/16-arrow-right.svg}}}

{ Underline } { Next page }


\section{Docs LaTeX/typst.app/docs/reference/text/smallcaps.tex}
\title{typst.app/docs/reference/text/smallcaps}

\begin{itemize}
\tightlist
\item
  \href{/docs}{\includesvg[width=0.16667in,height=0.16667in]{/assets/icons/16-docs-dark.svg}}
\item
  \includesvg[width=0.16667in,height=0.16667in]{/assets/icons/16-arrow-right.svg}
\item
  \href{/docs/reference/}{Reference}
\item
  \includesvg[width=0.16667in,height=0.16667in]{/assets/icons/16-arrow-right.svg}
\item
  \href{/docs/reference/text/}{Text}
\item
  \includesvg[width=0.16667in,height=0.16667in]{/assets/icons/16-arrow-right.svg}
\item
  \href{/docs/reference/text/smallcaps/}{Small Capitals}
\end{itemize}

\section{\texorpdfstring{\texttt{\ smallcaps\ } {{ Element
}}}{ smallcaps   Element }}\label{summary}

\phantomsection\label{element-tooltip}
Element functions can be customized with \texttt{\ set\ } and
\texttt{\ show\ } rules.

Displays text in small capitals.

\subsection{Example}\label{example}

\begin{verbatim}
Hello \
#smallcaps[Hello]
\end{verbatim}

\includegraphics[width=5in,height=\textheight,keepaspectratio]{/assets/docs/2GDSP4AltxmHWBvxVXZrwQAAAAAAAAAA.png}

\subsection{Smallcaps fonts}\label{smallcaps-fonts}

By default, this enables the OpenType \texttt{\ smcp\ } feature for the
font. Not all fonts support this feature. Sometimes smallcaps are part
of a dedicated font. This is, for example, the case for the \emph{Latin
Modern} family of fonts. In those cases, you can use a show-set rule to
customize the appearance of the text in smallcaps:

\begin{verbatim}
#show smallcaps: set text(font: "Latin Modern Roman Caps")
\end{verbatim}

In the future, this function will support synthesizing smallcaps from
normal letters, but this is not yet implemented.

\subsection{Smallcaps headings}\label{smallcaps-headings}

You can use a \href{/docs/reference/styling/\#show-rules}{show rule} to
apply smallcaps formatting to all your headings. In the example below,
we also center-align our headings and disable the standard bold font.

\begin{verbatim}
#set par(justify: true)
#set heading(numbering: "I.")

#show heading: smallcaps
#show heading: set align(center)
#show heading: set text(
  weight: "regular"
)

= Introduction
#lorem(40)
\end{verbatim}

\includegraphics[width=5in,height=\textheight,keepaspectratio]{/assets/docs/f0e4HVzW7NKFp4uqk6LvqgAAAAAAAAAA.png}

\subsection{\texorpdfstring{{ Parameters
}}{ Parameters }}\label{parameters}

\phantomsection\label{parameters-tooltip}
Parameters are the inputs to a function. They are specified in
parentheses after the function name.

{ smallcaps } (

{ \href{/docs/reference/foundations/content/}{content} }

) -\textgreater{} \href{/docs/reference/foundations/content/}{content}

\subsubsection{\texorpdfstring{\texttt{\ body\ }}{ body }}\label{parameters-body}

\href{/docs/reference/foundations/content/}{content}

{Required} {{ Positional }}

\phantomsection\label{parameters-body-positional-tooltip}
Positional parameters are specified in order, without names.

The content to display in small capitals.

\href{/docs/reference/text/raw/}{\pandocbounded{\includesvg[keepaspectratio]{/assets/icons/16-arrow-right.svg}}}

{ Raw Text / Code } { Previous page }

\href{/docs/reference/text/smartquote/}{\pandocbounded{\includesvg[keepaspectratio]{/assets/icons/16-arrow-right.svg}}}

{ Smartquote } { Next page }


\section{Docs LaTeX/typst.app/docs/reference/text/lorem.tex}
\title{typst.app/docs/reference/text/lorem}

\begin{itemize}
\tightlist
\item
  \href{/docs}{\includesvg[width=0.16667in,height=0.16667in]{/assets/icons/16-docs-dark.svg}}
\item
  \includesvg[width=0.16667in,height=0.16667in]{/assets/icons/16-arrow-right.svg}
\item
  \href{/docs/reference/}{Reference}
\item
  \includesvg[width=0.16667in,height=0.16667in]{/assets/icons/16-arrow-right.svg}
\item
  \href{/docs/reference/text/}{Text}
\item
  \includesvg[width=0.16667in,height=0.16667in]{/assets/icons/16-arrow-right.svg}
\item
  \href{/docs/reference/text/lorem/}{Lorem}
\end{itemize}

\section{\texorpdfstring{\texttt{\ lorem\ }}{ lorem }}\label{summary}

Creates blind text.

This function yields a Latin-like \emph{Lorem Ipsum} blind text with the
given number of words. The sequence of words generated by the function
is always the same but randomly chosen. As usual for blind texts, it
does not make any sense. Use it as a placeholder to try layouts.

\subsection{Example}\label{example}

\begin{verbatim}
= Blind Text
#lorem(30)

= More Blind Text
#lorem(15)
\end{verbatim}

\includegraphics[width=5in,height=\textheight,keepaspectratio]{/assets/docs/ivKswpaSkeLwjU6-8qNTjgAAAAAAAAAA.png}

\subsection{\texorpdfstring{{ Parameters
}}{ Parameters }}\label{parameters}

\phantomsection\label{parameters-tooltip}
Parameters are the inputs to a function. They are specified in
parentheses after the function name.

{ lorem } (

{ \href{/docs/reference/foundations/int/}{int} }

) -\textgreater{} \href{/docs/reference/foundations/str/}{str}

\subsubsection{\texorpdfstring{\texttt{\ words\ }}{ words }}\label{parameters-words}

\href{/docs/reference/foundations/int/}{int}

{Required} {{ Positional }}

\phantomsection\label{parameters-words-positional-tooltip}
Positional parameters are specified in order, without names.

The length of the blind text in words.

\href{/docs/reference/text/linebreak/}{\pandocbounded{\includesvg[keepaspectratio]{/assets/icons/16-arrow-right.svg}}}

{ Line Break } { Previous page }

\href{/docs/reference/text/lower/}{\pandocbounded{\includesvg[keepaspectratio]{/assets/icons/16-arrow-right.svg}}}

{ Lowercase } { Next page }




\section{C Docs LaTeX/docs/reference/foundations.tex}
\section{Docs LaTeX/typst.app/docs/reference/foundations/content.tex}
\title{typst.app/docs/reference/foundations/content}

\begin{itemize}
\tightlist
\item
  \href{/docs}{\includesvg[width=0.16667in,height=0.16667in]{/assets/icons/16-docs-dark.svg}}
\item
  \includesvg[width=0.16667in,height=0.16667in]{/assets/icons/16-arrow-right.svg}
\item
  \href{/docs/reference/}{Reference}
\item
  \includesvg[width=0.16667in,height=0.16667in]{/assets/icons/16-arrow-right.svg}
\item
  \href{/docs/reference/foundations/}{Foundations}
\item
  \includesvg[width=0.16667in,height=0.16667in]{/assets/icons/16-arrow-right.svg}
\item
  \href{/docs/reference/foundations/content/}{Content}
\end{itemize}

\section{\texorpdfstring{{ content }}{ content }}\label{summary}

A piece of document content.

This type is at the heart of Typst. All markup you write and most
\href{/docs/reference/foundations/function/}{functions} you call produce
content values. You can create a content value by enclosing markup in
square brackets. This is also how you pass content to functions.

\subsection{Example}\label{example}

\begin{verbatim}
Type of *Hello!* is
#type([*Hello!*])
\end{verbatim}

\includegraphics[width=5in,height=\textheight,keepaspectratio]{/assets/docs/X4qekl24YgH3SaXf1J0tagAAAAAAAAAA.png}

Content can be added with the \texttt{\ +\ } operator,
\href{/docs/reference/scripting/\#blocks}{joined together} and
multiplied with integers. Wherever content is expected, you can also
pass a \href{/docs/reference/foundations/str/}{string} or
\texttt{\ }{\texttt{\ none\ }}\texttt{\ } .

\subsection{Representation}\label{representation}

Content consists of elements with fields. When constructing an element
with its \emph{element function,} you provide these fields as arguments
and when you have a content value, you can access its fields with
\href{/docs/reference/scripting/\#field-access}{field access syntax} .

Some fields are required: These must be provided when constructing an
element and as a consequence, they are always available through field
access on content of that type. Required fields are marked as such in
the documentation.

Most fields are optional: Like required fields, they can be passed to
the element function to configure them for a single element. However,
these can also be configured with
\href{/docs/reference/styling/\#set-rules}{set rules} to apply them to
all elements within a scope. Optional fields are only available with
field access syntax when they were explicitly passed to the element
function, not when they result from a set rule.

Each element has a default appearance. However, you can also completely
customize its appearance with a
\href{/docs/reference/styling/\#show-rules}{show rule} . The show rule
is passed the element. It can access the element\textquotesingle s field
and produce arbitrary content from it.

In the web app, you can hover over a content variable to see exactly
which elements the content is composed of and what fields they have.
Alternatively, you can inspect the output of the
\href{/docs/reference/foundations/repr/}{\texttt{\ repr\ }} function.

\subsection{\texorpdfstring{{ Definitions
}}{ Definitions }}\label{definitions}

\phantomsection\label{definitions-tooltip}
Functions and types and can have associated definitions. These are
accessed by specifying the function or type, followed by a period, and
then the definition\textquotesingle s name.

\subsubsection{\texorpdfstring{\texttt{\ func\ }}{ func }}\label{definitions-func}

The content\textquotesingle s element function. This function can be
used to create the element contained in this content. It can be used in
set and show rules for the element. Can be compared with global
functions to check whether you have a specific kind of element.

self { . } { func } (

) -\textgreater{} \href{/docs/reference/foundations/function/}{function}

\subsubsection{\texorpdfstring{\texttt{\ has\ }}{ has }}\label{definitions-has}

Whether the content has the specified field.

self { . } { has } (

{ \href{/docs/reference/foundations/str/}{str} }

) -\textgreater{} \href{/docs/reference/foundations/bool/}{bool}

\paragraph{\texorpdfstring{\texttt{\ field\ }}{ field }}\label{definitions-has-field}

\href{/docs/reference/foundations/str/}{str}

{Required} {{ Positional }}

\phantomsection\label{definitions-has-field-positional-tooltip}
Positional parameters are specified in order, without names.

The field to look for.

\subsubsection{\texorpdfstring{\texttt{\ at\ }}{ at }}\label{definitions-at}

Access the specified field on the content. Returns the default value if
the field does not exist or fails with an error if no default value was
specified.

self { . } { at } (

{ \href{/docs/reference/foundations/str/}{str} , } {
\hyperref[definitions-at-parameters-default]{default :} { any } , }

) -\textgreater{} { any }

\paragraph{\texorpdfstring{\texttt{\ field\ }}{ field }}\label{definitions-at-field}

\href{/docs/reference/foundations/str/}{str}

{Required} {{ Positional }}

\phantomsection\label{definitions-at-field-positional-tooltip}
Positional parameters are specified in order, without names.

The field to access.

\paragraph{\texorpdfstring{\texttt{\ default\ }}{ default }}\label{definitions-at-default}

{ any }

A default value to return if the field does not exist.

\subsubsection{\texorpdfstring{\texttt{\ fields\ }}{ fields }}\label{definitions-fields}

Returns the fields of this content.

self { . } { fields } (

) -\textgreater{}
\href{/docs/reference/foundations/dictionary/}{dictionary}

\begin{verbatim}
#rect(
  width: 10cm,
  height: 10cm,
).fields()
\end{verbatim}

\includegraphics[width=5in,height=\textheight,keepaspectratio]{/assets/docs/zNlYUwJ_V8GS40gGav-GlwAAAAAAAAAA.png}

\subsubsection{\texorpdfstring{\texttt{\ location\ }}{ location }}\label{definitions-location}

The location of the content. This is only available on content returned
by \href{/docs/reference/introspection/query/}{query} or provided by a
\href{/docs/reference/styling/\#show-rules}{show rule} , for other
content it will be \texttt{\ }{\texttt{\ none\ }}\texttt{\ } . The
resulting location can be used with
\href{/docs/reference/introspection/counter/}{counters} ,
\href{/docs/reference/introspection/state/}{state} and
\href{/docs/reference/introspection/query/}{queries} .

self { . } { location } (

) -\textgreater{} \href{/docs/reference/foundations/none/}{none}
\href{/docs/reference/introspection/location/}{location}

\href{/docs/reference/foundations/calc/}{\pandocbounded{\includesvg[keepaspectratio]{/assets/icons/16-arrow-right.svg}}}

{ Calculation } { Previous page }

\href{/docs/reference/foundations/datetime/}{\pandocbounded{\includesvg[keepaspectratio]{/assets/icons/16-arrow-right.svg}}}

{ Datetime } { Next page }


\section{Docs LaTeX/typst.app/docs/reference/foundations/float.tex}
\title{typst.app/docs/reference/foundations/float}

\begin{itemize}
\tightlist
\item
  \href{/docs}{\includesvg[width=0.16667in,height=0.16667in]{/assets/icons/16-docs-dark.svg}}
\item
  \includesvg[width=0.16667in,height=0.16667in]{/assets/icons/16-arrow-right.svg}
\item
  \href{/docs/reference/}{Reference}
\item
  \includesvg[width=0.16667in,height=0.16667in]{/assets/icons/16-arrow-right.svg}
\item
  \href{/docs/reference/foundations/}{Foundations}
\item
  \includesvg[width=0.16667in,height=0.16667in]{/assets/icons/16-arrow-right.svg}
\item
  \href{/docs/reference/foundations/float/}{Float}
\end{itemize}

\section{\texorpdfstring{{ float }}{ float }}\label{summary}

A floating-point number.

A limited-precision representation of a real number. Typst uses 64 bits
to store floats. Wherever a float is expected, you can also pass an
\href{/docs/reference/foundations/int/}{integer} .

You can convert a value to a float with this type\textquotesingle s
constructor.

NaN and positive infinity are available as
\texttt{\ float\ }{\texttt{\ .\ }}\texttt{\ nan\ } and
\texttt{\ float\ }{\texttt{\ .\ }}\texttt{\ inf\ } respectively.

\subsection{Example}\label{example}

\begin{verbatim}
#3.14 \
#1e4 \
#(10 / 4)
\end{verbatim}

\includegraphics[width=5in,height=\textheight,keepaspectratio]{/assets/docs/Oh7PyPKhSHHcwVH4CSb0KwAAAAAAAAAA.png}

\subsection{\texorpdfstring{Constructor
{}}{Constructor }}\label{constructor}

\phantomsection\label{constructor-constructor-tooltip}
If a type has a constructor, you can call it like a function to create a
new value of the type.

Converts a value to a float.

\begin{itemize}
\tightlist
\item
  Booleans are converted to \texttt{\ 0.0\ } or \texttt{\ 1.0\ } .
\item
  Integers are converted to the closest 64-bit float. For integers with
  absolute value less than
  \texttt{\ calc\ }{\texttt{\ .\ }}\texttt{\ }{\texttt{\ pow\ }}\texttt{\ }{\texttt{\ (\ }}\texttt{\ }{\texttt{\ 2\ }}\texttt{\ }{\texttt{\ ,\ }}\texttt{\ }{\texttt{\ 53\ }}\texttt{\ }{\texttt{\ )\ }}\texttt{\ }
  , this conversion is exact.
\item
  Ratios are divided by 100\%.
\item
  Strings are parsed in base 10 to the closest 64-bit float. Exponential
  notation is supported.
\end{itemize}

{ float } (

{ \href{/docs/reference/foundations/bool/}{bool}
\href{/docs/reference/foundations/int/}{int}
\href{/docs/reference/foundations/float/}{float}
\href{/docs/reference/layout/ratio/}{ratio}
\href{/docs/reference/foundations/str/}{str}
\href{/docs/reference/foundations/decimal/}{decimal} }

) -\textgreater{} \href{/docs/reference/foundations/float/}{float}

\begin{verbatim}
#float(false) \
#float(true) \
#float(4) \
#float(40%) \
#float("2.7") \
#float("1e5")
\end{verbatim}

\includegraphics[width=5in,height=\textheight,keepaspectratio]{/assets/docs/PMa-HqZaL4--FN_1I0OHagAAAAAAAAAA.png}

\paragraph{\texorpdfstring{\texttt{\ value\ }}{ value }}\label{constructor-value}

\href{/docs/reference/foundations/bool/}{bool} {or}
\href{/docs/reference/foundations/int/}{int} {or}
\href{/docs/reference/foundations/float/}{float} {or}
\href{/docs/reference/layout/ratio/}{ratio} {or}
\href{/docs/reference/foundations/str/}{str} {or}
\href{/docs/reference/foundations/decimal/}{decimal}

{Required} {{ Positional }}

\phantomsection\label{constructor-value-positional-tooltip}
Positional parameters are specified in order, without names.

The value that should be converted to a float.

\subsection{\texorpdfstring{{ Definitions
}}{ Definitions }}\label{definitions}

\phantomsection\label{definitions-tooltip}
Functions and types and can have associated definitions. These are
accessed by specifying the function or type, followed by a period, and
then the definition\textquotesingle s name.

\subsubsection{\texorpdfstring{\texttt{\ is-nan\ }}{ is-nan }}\label{definitions-is-nan}

Checks if a float is not a number.

In IEEE 754, more than one bit pattern represents a NaN. This function
returns \texttt{\ true\ } if the float is any of those bit patterns.

self { . } { is-nan } (

) -\textgreater{} \href{/docs/reference/foundations/bool/}{bool}

\begin{verbatim}
#float.is-nan(0) \
#float.is-nan(1) \
#float.is-nan(float.nan)
\end{verbatim}

\includegraphics[width=5in,height=\textheight,keepaspectratio]{/assets/docs/9jd8hxPcunH7CdCSXWd1dwAAAAAAAAAA.png}

\subsubsection{\texorpdfstring{\texttt{\ is-infinite\ }}{ is-infinite }}\label{definitions-is-infinite}

Checks if a float is infinite.

Floats can represent positive infinity and negative infinity. This
function returns \texttt{\ }{\texttt{\ true\ }}\texttt{\ } if the float
is an infinity.

self { . } { is-infinite } (

) -\textgreater{} \href{/docs/reference/foundations/bool/}{bool}

\begin{verbatim}
#float.is-infinite(0) \
#float.is-infinite(1) \
#float.is-infinite(float.inf)
\end{verbatim}

\includegraphics[width=5in,height=\textheight,keepaspectratio]{/assets/docs/AIoKhvpoq-xeueiSPD9O7gAAAAAAAAAA.png}

\subsubsection{\texorpdfstring{\texttt{\ signum\ }}{ signum }}\label{definitions-signum}

Calculates the sign of a floating point number.

\begin{itemize}
\tightlist
\item
  If the number is positive (including
  \texttt{\ }{\texttt{\ +\ }}\texttt{\ }{\texttt{\ 0.0\ }}\texttt{\ } ),
  returns \texttt{\ }{\texttt{\ 1.0\ }}\texttt{\ } .
\item
  If the number is negative (including
  \texttt{\ }{\texttt{\ -\ }}\texttt{\ }{\texttt{\ 0.0\ }}\texttt{\ } ),
  returns
  \texttt{\ }{\texttt{\ -\ }}\texttt{\ }{\texttt{\ 1.0\ }}\texttt{\ } .
\item
  If the number is NaN, returns
  \texttt{\ float\ }{\texttt{\ .\ }}\texttt{\ nan\ } .
\end{itemize}

self { . } { signum } (

) -\textgreater{} \href{/docs/reference/foundations/float/}{float}

\begin{verbatim}
#(5.0).signum() \
#(-5.0).signum() \
#(0.0).signum() \
#float.nan.signum()
\end{verbatim}

\includegraphics[width=5in,height=\textheight,keepaspectratio]{/assets/docs/HHp-pldXJoLbqAEsg_2mmQAAAAAAAAAA.png}

\subsubsection{\texorpdfstring{\texttt{\ from-bytes\ }}{ from-bytes }}\label{definitions-from-bytes}

Converts bytes to a float.

float { . } { from-bytes } (

{ \href{/docs/reference/foundations/bytes/}{bytes} , } {
\hyperref[definitions-from-bytes-parameters-endian]{endian :}
\href{/docs/reference/foundations/str/}{str} , }

) -\textgreater{} \href{/docs/reference/foundations/float/}{float}

\begin{verbatim}
#float.from-bytes(bytes((0, 0, 0, 0, 0, 0, 240, 63))) \
#float.from-bytes(bytes((63, 240, 0, 0, 0, 0, 0, 0)), endian: "big")
\end{verbatim}

\includegraphics[width=5in,height=\textheight,keepaspectratio]{/assets/docs/TbCinqru71JKOm73kOJYdwAAAAAAAAAA.png}

\paragraph{\texorpdfstring{\texttt{\ bytes\ }}{ bytes }}\label{definitions-from-bytes-bytes}

\href{/docs/reference/foundations/bytes/}{bytes}

{Required} {{ Positional }}

\phantomsection\label{definitions-from-bytes-bytes-positional-tooltip}
Positional parameters are specified in order, without names.

The bytes that should be converted to a float.

Must be of length exactly 8 so that the result fits into a 64-bit float.

\paragraph{\texorpdfstring{\texttt{\ endian\ }}{ endian }}\label{definitions-from-bytes-endian}

\href{/docs/reference/foundations/str/}{str}

The endianness of the conversion.

\begin{longtable}[]{@{}ll@{}}
\toprule\noalign{}
Variant & Details \\
\midrule\noalign{}
\endhead
\bottomrule\noalign{}
\endlastfoot
\texttt{\ "\ big\ "\ } & Big-endian byte order: The highest-value byte
is at the beginning of the bytes. \\
\texttt{\ "\ little\ "\ } & Little-endian byte order: The lowest-value
byte is at the beginning of the bytes. \\
\end{longtable}

Default: \texttt{\ }{\texttt{\ "little"\ }}\texttt{\ }

\subsubsection{\texorpdfstring{\texttt{\ to-bytes\ }}{ to-bytes }}\label{definitions-to-bytes}

Converts a float to bytes.

self { . } { to-bytes } (

{ \hyperref[definitions-to-bytes-parameters-endian]{endian :}
\href{/docs/reference/foundations/str/}{str} }

) -\textgreater{} \href{/docs/reference/foundations/bytes/}{bytes}

\begin{verbatim}
#array(1.0.to-bytes(endian: "big")) \
#array(1.0.to-bytes())
\end{verbatim}

\includegraphics[width=5in,height=\textheight,keepaspectratio]{/assets/docs/oyz50tHIOoQRj_5WM6JIbAAAAAAAAAAA.png}

\paragraph{\texorpdfstring{\texttt{\ endian\ }}{ endian }}\label{definitions-to-bytes-endian}

\href{/docs/reference/foundations/str/}{str}

The endianness of the conversion.

\begin{longtable}[]{@{}ll@{}}
\toprule\noalign{}
Variant & Details \\
\midrule\noalign{}
\endhead
\bottomrule\noalign{}
\endlastfoot
\texttt{\ "\ big\ "\ } & Big-endian byte order: The highest-value byte
is at the beginning of the bytes. \\
\texttt{\ "\ little\ "\ } & Little-endian byte order: The lowest-value
byte is at the beginning of the bytes. \\
\end{longtable}

Default: \texttt{\ }{\texttt{\ "little"\ }}\texttt{\ }

\href{/docs/reference/foundations/eval/}{\pandocbounded{\includesvg[keepaspectratio]{/assets/icons/16-arrow-right.svg}}}

{ Evaluate } { Previous page }

\href{/docs/reference/foundations/function/}{\pandocbounded{\includesvg[keepaspectratio]{/assets/icons/16-arrow-right.svg}}}

{ Function } { Next page }


\section{Docs LaTeX/typst.app/docs/reference/foundations/label.tex}
\title{typst.app/docs/reference/foundations/label}

\begin{itemize}
\tightlist
\item
  \href{/docs}{\includesvg[width=0.16667in,height=0.16667in]{/assets/icons/16-docs-dark.svg}}
\item
  \includesvg[width=0.16667in,height=0.16667in]{/assets/icons/16-arrow-right.svg}
\item
  \href{/docs/reference/}{Reference}
\item
  \includesvg[width=0.16667in,height=0.16667in]{/assets/icons/16-arrow-right.svg}
\item
  \href{/docs/reference/foundations/}{Foundations}
\item
  \includesvg[width=0.16667in,height=0.16667in]{/assets/icons/16-arrow-right.svg}
\item
  \href{/docs/reference/foundations/label/}{Label}
\end{itemize}

\section{\texorpdfstring{{ label }}{ label }}\label{summary}

A label for an element.

Inserting a label into content attaches it to the closest preceding
element that is not a space. The preceding element must be in the same
scope as the label, which means that
\texttt{\ Hello\ }{\texttt{\ \#\ }}\texttt{\ }{\texttt{\ {[}\ }}\texttt{\ }{\texttt{\ \textless{}label\textgreater{}\ }}\texttt{\ }{\texttt{\ {]}\ }}\texttt{\ }
, for instance, wouldn\textquotesingle t work.

A labelled element can be \href{/docs/reference/model/ref/}{referenced}
, \href{/docs/reference/introspection/query/}{queried} for, and
\href{/docs/reference/styling/}{styled} through its label.

Once constructed, you can get the name of a label using
\href{/docs/reference/foundations/str/\#constructor}{\texttt{\ str\ }} .

\subsection{Example}\label{example}

\begin{verbatim}
#show <a>: set text(blue)
#show label("b"): set text(red)

= Heading <a>
*Strong* #label("b")
\end{verbatim}

\includegraphics[width=5in,height=\textheight,keepaspectratio]{/assets/docs/l3ZXI9iv-ZpcNuL82oagnwAAAAAAAAAA.png}

\subsection{Syntax}\label{syntax}

This function also has dedicated syntax: You can create a label by
enclosing its name in angle brackets. This works both in markup and
code. A label\textquotesingle s name can contain letters, numbers,
\texttt{\ \_\ } , \texttt{\ -\ } , \texttt{\ :\ } , and \texttt{\ .\ } .

Note that there is a syntactical difference when using the dedicated
syntax for this function. In the code below, the
\texttt{\ }{\texttt{\ \textless{}a\textgreater{}\ }}\texttt{\ }
terminates the heading and thus attaches to the heading itself, whereas
the
\texttt{\ }{\texttt{\ \#\ }}\texttt{\ }{\texttt{\ label\ }}\texttt{\ }{\texttt{\ (\ }}\texttt{\ }{\texttt{\ "b"\ }}\texttt{\ }{\texttt{\ )\ }}\texttt{\ }
is part of the heading and thus attaches to the
heading\textquotesingle s text.

\begin{verbatim}
// Equivalent to `#heading[Introduction] <a>`.
= Introduction <a>

// Equivalent to `#heading[Conclusion #label("b")]`.
= Conclusion #label("b")
\end{verbatim}

Currently, labels can only be attached to elements in markup mode, not
in code mode. This might change in the future.

\subsection{\texorpdfstring{Constructor
{}}{Constructor }}\label{constructor}

\phantomsection\label{constructor-constructor-tooltip}
If a type has a constructor, you can call it like a function to create a
new value of the type.

Creates a label from a string.

{ label } (

{ \href{/docs/reference/foundations/str/}{str} }

) -\textgreater{} \href{/docs/reference/foundations/label/}{label}

\paragraph{\texorpdfstring{\texttt{\ name\ }}{ name }}\label{constructor-name}

\href{/docs/reference/foundations/str/}{str}

{Required} {{ Positional }}

\phantomsection\label{constructor-name-positional-tooltip}
Positional parameters are specified in order, without names.

The name of the label.

\href{/docs/reference/foundations/int/}{\pandocbounded{\includesvg[keepaspectratio]{/assets/icons/16-arrow-right.svg}}}

{ Integer } { Previous page }

\href{/docs/reference/foundations/module/}{\pandocbounded{\includesvg[keepaspectratio]{/assets/icons/16-arrow-right.svg}}}

{ Module } { Next page }


\section{Docs LaTeX/typst.app/docs/reference/foundations/plugin.tex}
\title{typst.app/docs/reference/foundations/plugin}

\begin{itemize}
\tightlist
\item
  \href{/docs}{\includesvg[width=0.16667in,height=0.16667in]{/assets/icons/16-docs-dark.svg}}
\item
  \includesvg[width=0.16667in,height=0.16667in]{/assets/icons/16-arrow-right.svg}
\item
  \href{/docs/reference/}{Reference}
\item
  \includesvg[width=0.16667in,height=0.16667in]{/assets/icons/16-arrow-right.svg}
\item
  \href{/docs/reference/foundations/}{Foundations}
\item
  \includesvg[width=0.16667in,height=0.16667in]{/assets/icons/16-arrow-right.svg}
\item
  \href{/docs/reference/foundations/plugin/}{Plugin}
\end{itemize}

\section{\texorpdfstring{{ plugin }}{ plugin }}\label{summary}

A WebAssembly plugin.

Typst is capable of interfacing with plugins compiled to WebAssembly.
Plugin functions may accept multiple
\href{/docs/reference/foundations/bytes/}{byte buffers} as arguments and
return a single byte buffer. They should typically be wrapped in
idiomatic Typst functions that perform the necessary conversions between
native Typst types and bytes.

Plugins run in isolation from your system, which means that printing,
reading files, or anything like that will not be supported for security
reasons. To run as a plugin, a program needs to be compiled to a 32-bit
shared WebAssembly library. Many compilers will use the
\href{https://wasi.dev/}{WASI ABI} by default or as their only option
(e.g. emscripten), which allows printing, reading files, etc. This ABI
will not directly work with Typst. You will either need to compile to a
different target or
\href{https://github.com/astrale-sharp/wasm-minimal-protocol/blob/master/wasi-stub}{stub
all functions} .

\subsection{Plugins and Packages}\label{plugins-and-packages}

Plugins are distributed as packages. A package can make use of a plugin
simply by including a WebAssembly file and loading it. Because the
byte-based plugin interface is quite low-level, plugins are typically
exposed through wrapper functions, that also live in the same package.

\subsection{Purity}\label{purity}

Plugin functions must be pure: Given the same arguments, they must
always return the same value. The reason for this is that Typst
functions must be pure (which is quite fundamental to the language
design) and, since Typst function can call plugin functions, this
requirement is inherited. In particular, if a plugin function is called
twice with the same arguments, Typst might cache the results and call
your function only once.

\subsection{Example}\label{example}

\begin{verbatim}
#let myplugin = plugin("hello.wasm")
#let concat(a, b) = str(
  myplugin.concatenate(
    bytes(a),
    bytes(b),
  )
)

#concat("hello", "world")
\end{verbatim}

\includegraphics[width=5in,height=\textheight,keepaspectratio]{/assets/docs/Vj65tyYDxxD3OHZUaiQ94QAAAAAAAAAA.png}

\subsection{Protocol}\label{protocol}

To be used as a plugin, a WebAssembly module must conform to the
following protocol:

\subsubsection{Exports}\label{exports}

A plugin module can export functions to make them callable from Typst.
To conform to the protocol, an exported function should:

\begin{itemize}
\item
  Take \texttt{\ n\ } 32-bit integer arguments \texttt{\ a\_1\ } ,
  \texttt{\ a\_2\ } , ..., \texttt{\ a\_n\ } (interpreted as lengths, so
  \texttt{\ usize/size\_t\ } may be preferable), and return one 32-bit
  integer.
\item
  The function should first allocate a buffer \texttt{\ buf\ } of length
  \texttt{\ a\_1\ +\ a\_2\ +\ ...\ +\ a\_n\ } , and then call
  \texttt{\ wasm\_minimal\_protocol\_write\_args\_to\_buffer(buf.ptr)\ }
  .
\item
  The \texttt{\ a\_1\ } first bytes of the buffer now constitute the
  first argument, the \texttt{\ a\_2\ } next bytes the second argument,
  and so on.
\item
  The function can now do its job with the arguments and produce an
  output buffer. Before returning, it should call
  \texttt{\ wasm\_minimal\_protocol\_send\_result\_to\_host\ } to send
  its result back to the host.
\item
  To signal success, the function should return \texttt{\ 0\ } .
\item
  To signal an error, the function should return \texttt{\ 1\ } . The
  written buffer is then interpreted as an UTF-8 encoded error message.
\end{itemize}

\subsubsection{Imports}\label{imports}

Plugin modules need to import two functions that are provided by the
runtime. (Types and functions are described using WAT syntax.)

\begin{itemize}
\item
  \texttt{\ (import\ "typst\_env"\ "wasm\_minimal\_protocol\_write\_args\_to\_buffer"\ (func\ (param\ i32)))\ }

  Writes the arguments for the current function into a plugin-allocated
  buffer. When a plugin function is called, it
  \hyperref[exports]{receives the lengths} of its input buffers as
  arguments. It should then allocate a buffer whose capacity is at least
  the sum of these lengths. It should then call this function with a
  \texttt{\ ptr\ } to the buffer to fill it with the arguments, one
  after another.
\item
  \texttt{\ (import\ "typst\_env"\ "wasm\_minimal\_protocol\_send\_result\_to\_host"\ (func\ (param\ i32\ i32)))\ }

  Sends the output of the current function to the host (Typst). The
  first parameter shall be a pointer to a buffer ( \texttt{\ ptr\ } ),
  while the second is the length of that buffer ( \texttt{\ len\ } ).
  The memory pointed at by \texttt{\ ptr\ } can be freed immediately
  after this function returns. If the message should be interpreted as
  an error message, it should be encoded as UTF-8.
\end{itemize}

\subsection{Resources}\label{resources}

For more resources, check out the
\href{https://github.com/astrale-sharp/wasm-minimal-protocol}{wasm-minimal-protocol
repository} . It contains:

\begin{itemize}
\tightlist
\item
  A list of example plugin implementations and a test runner for these
  examples
\item
  Wrappers to help you write your plugin in Rust (Zig wrapper in
  development)
\item
  A stubber for WASI
\end{itemize}

\subsection{\texorpdfstring{Constructor
{}}{Constructor }}\label{constructor}

\phantomsection\label{constructor-constructor-tooltip}
If a type has a constructor, you can call it like a function to create a
new value of the type.

Creates a new plugin from a WebAssembly file.

{ plugin } (

{ \href{/docs/reference/foundations/str/}{str} }

) -\textgreater{} \href{/docs/reference/foundations/plugin/}{plugin}

\paragraph{\texorpdfstring{\texttt{\ path\ }}{ path }}\label{constructor-path}

\href{/docs/reference/foundations/str/}{str}

{Required} {{ Positional }}

\phantomsection\label{constructor-path-positional-tooltip}
Positional parameters are specified in order, without names.

Path to a WebAssembly file.

For more details, see the \href{/docs/reference/syntax/\#paths}{Paths
section} .

\href{/docs/reference/foundations/panic/}{\pandocbounded{\includesvg[keepaspectratio]{/assets/icons/16-arrow-right.svg}}}

{ Panic } { Previous page }

\href{/docs/reference/foundations/regex/}{\pandocbounded{\includesvg[keepaspectratio]{/assets/icons/16-arrow-right.svg}}}

{ Regex } { Next page }


\section{Docs LaTeX/typst.app/docs/reference/foundations/panic.tex}
\title{typst.app/docs/reference/foundations/panic}

\begin{itemize}
\tightlist
\item
  \href{/docs}{\includesvg[width=0.16667in,height=0.16667in]{/assets/icons/16-docs-dark.svg}}
\item
  \includesvg[width=0.16667in,height=0.16667in]{/assets/icons/16-arrow-right.svg}
\item
  \href{/docs/reference/}{Reference}
\item
  \includesvg[width=0.16667in,height=0.16667in]{/assets/icons/16-arrow-right.svg}
\item
  \href{/docs/reference/foundations/}{Foundations}
\item
  \includesvg[width=0.16667in,height=0.16667in]{/assets/icons/16-arrow-right.svg}
\item
  \href{/docs/reference/foundations/panic/}{Panic}
\end{itemize}

\section{\texorpdfstring{\texttt{\ panic\ }}{ panic }}\label{summary}

Fails with an error.

Arguments are displayed to the user (not rendered in the document) as
strings, converting with \texttt{\ repr\ } if necessary.

\subsection{Example}\label{example}

The code below produces the error
\texttt{\ panicked\ with:\ "this\ is\ wrong"\ } .

\begin{verbatim}
#panic("this is wrong")
\end{verbatim}

\subsection{\texorpdfstring{{ Parameters
}}{ Parameters }}\label{parameters}

\phantomsection\label{parameters-tooltip}
Parameters are the inputs to a function. They are specified in
parentheses after the function name.

{ panic } (

{ \hyperref[parameters-values]{..} { any } }

)

\subsubsection{\texorpdfstring{\texttt{\ values\ }}{ values }}\label{parameters-values}

{ any }

{Required} {{ Positional }}

\phantomsection\label{parameters-values-positional-tooltip}
Positional parameters are specified in order, without names.

{{ Variadic }}

\phantomsection\label{parameters-values-variadic-tooltip}
Variadic parameters can be specified multiple times.

The values to panic with and display to the user.

\href{/docs/reference/foundations/none/}{\pandocbounded{\includesvg[keepaspectratio]{/assets/icons/16-arrow-right.svg}}}

{ None } { Previous page }

\href{/docs/reference/foundations/plugin/}{\pandocbounded{\includesvg[keepaspectratio]{/assets/icons/16-arrow-right.svg}}}

{ Plugin } { Next page }


\section{Docs LaTeX/typst.app/docs/reference/foundations/array.tex}
\title{typst.app/docs/reference/foundations/array}

\begin{itemize}
\tightlist
\item
  \href{/docs}{\includesvg[width=0.16667in,height=0.16667in]{/assets/icons/16-docs-dark.svg}}
\item
  \includesvg[width=0.16667in,height=0.16667in]{/assets/icons/16-arrow-right.svg}
\item
  \href{/docs/reference/}{Reference}
\item
  \includesvg[width=0.16667in,height=0.16667in]{/assets/icons/16-arrow-right.svg}
\item
  \href{/docs/reference/foundations/}{Foundations}
\item
  \includesvg[width=0.16667in,height=0.16667in]{/assets/icons/16-arrow-right.svg}
\item
  \href{/docs/reference/foundations/array/}{Array}
\end{itemize}

\section{\texorpdfstring{{ array }}{ array }}\label{summary}

A sequence of values.

You can construct an array by enclosing a comma-separated sequence of
values in parentheses. The values do not have to be of the same type.

You can access and update array items with the \texttt{\ .at()\ }
method. Indices are zero-based and negative indices wrap around to the
end of the array. You can iterate over an array using a
\href{/docs/reference/scripting/\#loops}{for loop} . Arrays can be added
together with the \texttt{\ +\ } operator,
\href{/docs/reference/scripting/\#blocks}{joined together} and
multiplied with integers.

\textbf{Note:} An array of length one needs a trailing comma, as in
\texttt{\ }{\texttt{\ (\ }}\texttt{\ }{\texttt{\ 1\ }}\texttt{\ }{\texttt{\ ,\ }}\texttt{\ }{\texttt{\ )\ }}\texttt{\ }
. This is to disambiguate from a simple parenthesized expressions like
\texttt{\ }{\texttt{\ (\ }}\texttt{\ }{\texttt{\ 1\ }}\texttt{\ }{\texttt{\ +\ }}\texttt{\ }{\texttt{\ 2\ }}\texttt{\ }{\texttt{\ )\ }}\texttt{\ }{\texttt{\ *\ }}\texttt{\ }{\texttt{\ 3\ }}\texttt{\ }
. An empty array is written as
\texttt{\ }{\texttt{\ (\ }}\texttt{\ }{\texttt{\ )\ }}\texttt{\ } .

\subsection{Example}\label{example}

\begin{verbatim}
#let values = (1, 7, 4, -3, 2)

#values.at(0) \
#(values.at(0) = 3)
#values.at(-1) \
#values.find(calc.even) \
#values.filter(calc.odd) \
#values.map(calc.abs) \
#values.rev() \
#(1, (2, 3)).flatten() \
#(("A", "B", "C")
    .join(", ", last: " and "))
\end{verbatim}

\includegraphics[width=5in,height=\textheight,keepaspectratio]{/assets/docs/uC3P-2nGePaWZlTLapiUowAAAAAAAAAA.png}

\subsection{\texorpdfstring{Constructor
{}}{Constructor }}\label{constructor}

\phantomsection\label{constructor-constructor-tooltip}
If a type has a constructor, you can call it like a function to create a
new value of the type.

Converts a value to an array.

Note that this function is only intended for conversion of a
collection-like value to an array, not for creation of an array from
individual items. Use the array syntax \texttt{\ (1,\ 2,\ 3)\ } (or
\texttt{\ (1,)\ } for a single-element array) instead.

{ array } (

{ \href{/docs/reference/foundations/bytes/}{bytes}
\href{/docs/reference/foundations/array/}{array}
\href{/docs/reference/foundations/version/}{version} }

) -\textgreater{} \href{/docs/reference/foundations/array/}{array}

\begin{verbatim}
#let hi = "Hello 😃"
#array(bytes(hi))
\end{verbatim}

\includegraphics[width=5in,height=\textheight,keepaspectratio]{/assets/docs/X4h0etegVnRbtNlLnkRA5AAAAAAAAAAA.png}

\paragraph{\texorpdfstring{\texttt{\ value\ }}{ value }}\label{constructor-value}

\href{/docs/reference/foundations/bytes/}{bytes} {or}
\href{/docs/reference/foundations/array/}{array} {or}
\href{/docs/reference/foundations/version/}{version}

{Required} {{ Positional }}

\phantomsection\label{constructor-value-positional-tooltip}
Positional parameters are specified in order, without names.

The value that should be converted to an array.

\subsection{\texorpdfstring{{ Definitions
}}{ Definitions }}\label{definitions}

\phantomsection\label{definitions-tooltip}
Functions and types and can have associated definitions. These are
accessed by specifying the function or type, followed by a period, and
then the definition\textquotesingle s name.

\subsubsection{\texorpdfstring{\texttt{\ len\ }}{ len }}\label{definitions-len}

The number of values in the array.

self { . } { len } (

) -\textgreater{} \href{/docs/reference/foundations/int/}{int}

\subsubsection{\texorpdfstring{\texttt{\ first\ }}{ first }}\label{definitions-first}

Returns the first item in the array. May be used on the left-hand side
of an assignment. Fails with an error if the array is empty.

self { . } { first } (

) -\textgreater{} { any }

\subsubsection{\texorpdfstring{\texttt{\ last\ }}{ last }}\label{definitions-last}

Returns the last item in the array. May be used on the left-hand side of
an assignment. Fails with an error if the array is empty.

self { . } { last } (

) -\textgreater{} { any }

\subsubsection{\texorpdfstring{\texttt{\ at\ }}{ at }}\label{definitions-at}

Returns the item at the specified index in the array. May be used on the
left-hand side of an assignment. Returns the default value if the index
is out of bounds or fails with an error if no default value was
specified.

self { . } { at } (

{ \href{/docs/reference/foundations/int/}{int} , } {
\hyperref[definitions-at-parameters-default]{default :} { any } , }

) -\textgreater{} { any }

\paragraph{\texorpdfstring{\texttt{\ index\ }}{ index }}\label{definitions-at-index}

\href{/docs/reference/foundations/int/}{int}

{Required} {{ Positional }}

\phantomsection\label{definitions-at-index-positional-tooltip}
Positional parameters are specified in order, without names.

The index at which to retrieve the item. If negative, indexes from the
back.

\paragraph{\texorpdfstring{\texttt{\ default\ }}{ default }}\label{definitions-at-default}

{ any }

A default value to return if the index is out of bounds.

\subsubsection{\texorpdfstring{\texttt{\ push\ }}{ push }}\label{definitions-push}

Adds a value to the end of the array.

self { . } { push } (

{ { any } }

)

\paragraph{\texorpdfstring{\texttt{\ value\ }}{ value }}\label{definitions-push-value}

{ any }

{Required} {{ Positional }}

\phantomsection\label{definitions-push-value-positional-tooltip}
Positional parameters are specified in order, without names.

The value to insert at the end of the array.

\subsubsection{\texorpdfstring{\texttt{\ pop\ }}{ pop }}\label{definitions-pop}

Removes the last item from the array and returns it. Fails with an error
if the array is empty.

self { . } { pop } (

) -\textgreater{} { any }

\subsubsection{\texorpdfstring{\texttt{\ insert\ }}{ insert }}\label{definitions-insert}

Inserts a value into the array at the specified index, shifting all
subsequent elements to the right. Fails with an error if the index is
out of bounds.

To replace an element of an array, use
\href{/docs/reference/foundations/array/\#definitions-at}{\texttt{\ at\ }}
.

self { . } { insert } (

{ \href{/docs/reference/foundations/int/}{int} , } { { any } , }

)

\paragraph{\texorpdfstring{\texttt{\ index\ }}{ index }}\label{definitions-insert-index}

\href{/docs/reference/foundations/int/}{int}

{Required} {{ Positional }}

\phantomsection\label{definitions-insert-index-positional-tooltip}
Positional parameters are specified in order, without names.

The index at which to insert the item. If negative, indexes from the
back.

\paragraph{\texorpdfstring{\texttt{\ value\ }}{ value }}\label{definitions-insert-value}

{ any }

{Required} {{ Positional }}

\phantomsection\label{definitions-insert-value-positional-tooltip}
Positional parameters are specified in order, without names.

The value to insert into the array.

\subsubsection{\texorpdfstring{\texttt{\ remove\ }}{ remove }}\label{definitions-remove}

Removes the value at the specified index from the array and return it.

self { . } { remove } (

{ \href{/docs/reference/foundations/int/}{int} , } {
\hyperref[definitions-remove-parameters-default]{default :} { any } , }

) -\textgreater{} { any }

\paragraph{\texorpdfstring{\texttt{\ index\ }}{ index }}\label{definitions-remove-index}

\href{/docs/reference/foundations/int/}{int}

{Required} {{ Positional }}

\phantomsection\label{definitions-remove-index-positional-tooltip}
Positional parameters are specified in order, without names.

The index at which to remove the item. If negative, indexes from the
back.

\paragraph{\texorpdfstring{\texttt{\ default\ }}{ default }}\label{definitions-remove-default}

{ any }

A default value to return if the index is out of bounds.

\subsubsection{\texorpdfstring{\texttt{\ slice\ }}{ slice }}\label{definitions-slice}

Extracts a subslice of the array. Fails with an error if the start or
end index is out of bounds.

self { . } { slice } (

{ \href{/docs/reference/foundations/int/}{int} , } {
\href{/docs/reference/foundations/none/}{none}
\href{/docs/reference/foundations/int/}{int} , } {
\hyperref[definitions-slice-parameters-count]{count :}
\href{/docs/reference/foundations/int/}{int} , }

) -\textgreater{} \href{/docs/reference/foundations/array/}{array}

\paragraph{\texorpdfstring{\texttt{\ start\ }}{ start }}\label{definitions-slice-start}

\href{/docs/reference/foundations/int/}{int}

{Required} {{ Positional }}

\phantomsection\label{definitions-slice-start-positional-tooltip}
Positional parameters are specified in order, without names.

The start index (inclusive). If negative, indexes from the back.

\paragraph{\texorpdfstring{\texttt{\ end\ }}{ end }}\label{definitions-slice-end}

\href{/docs/reference/foundations/none/}{none} {or}
\href{/docs/reference/foundations/int/}{int}

{{ Positional }}

\phantomsection\label{definitions-slice-end-positional-tooltip}
Positional parameters are specified in order, without names.

The end index (exclusive). If omitted, the whole slice until the end of
the array is extracted. If negative, indexes from the back.

Default: \texttt{\ }{\texttt{\ none\ }}\texttt{\ }

\paragraph{\texorpdfstring{\texttt{\ count\ }}{ count }}\label{definitions-slice-count}

\href{/docs/reference/foundations/int/}{int}

The number of items to extract. This is equivalent to passing
\texttt{\ start\ +\ count\ } as the \texttt{\ end\ } position. Mutually
exclusive with \texttt{\ end\ } .

\subsubsection{\texorpdfstring{\texttt{\ contains\ }}{ contains }}\label{definitions-contains}

Whether the array contains the specified value.

This method also has dedicated syntax: You can write
\texttt{\ }{\texttt{\ 2\ }}\texttt{\ }{\texttt{\ in\ }}\texttt{\ }{\texttt{\ (\ }}\texttt{\ }{\texttt{\ 1\ }}\texttt{\ }{\texttt{\ ,\ }}\texttt{\ }{\texttt{\ 2\ }}\texttt{\ }{\texttt{\ ,\ }}\texttt{\ }{\texttt{\ 3\ }}\texttt{\ }{\texttt{\ )\ }}\texttt{\ }
instead of
\texttt{\ }{\texttt{\ (\ }}\texttt{\ }{\texttt{\ 1\ }}\texttt{\ }{\texttt{\ ,\ }}\texttt{\ }{\texttt{\ 2\ }}\texttt{\ }{\texttt{\ ,\ }}\texttt{\ }{\texttt{\ 3\ }}\texttt{\ }{\texttt{\ )\ }}\texttt{\ }{\texttt{\ .\ }}\texttt{\ }{\texttt{\ contains\ }}\texttt{\ }{\texttt{\ (\ }}\texttt{\ }{\texttt{\ 2\ }}\texttt{\ }{\texttt{\ )\ }}\texttt{\ }
.

self { . } { contains } (

{ { any } }

) -\textgreater{} \href{/docs/reference/foundations/bool/}{bool}

\paragraph{\texorpdfstring{\texttt{\ value\ }}{ value }}\label{definitions-contains-value}

{ any }

{Required} {{ Positional }}

\phantomsection\label{definitions-contains-value-positional-tooltip}
Positional parameters are specified in order, without names.

The value to search for.

\subsubsection{\texorpdfstring{\texttt{\ find\ }}{ find }}\label{definitions-find}

Searches for an item for which the given function returns
\texttt{\ }{\texttt{\ true\ }}\texttt{\ } and returns the first match or
\texttt{\ }{\texttt{\ none\ }}\texttt{\ } if there is no match.

self { . } { find } (

{ \href{/docs/reference/foundations/function/}{function} }

) -\textgreater{} { any } \href{/docs/reference/foundations/none/}{none}

\paragraph{\texorpdfstring{\texttt{\ searcher\ }}{ searcher }}\label{definitions-find-searcher}

\href{/docs/reference/foundations/function/}{function}

{Required} {{ Positional }}

\phantomsection\label{definitions-find-searcher-positional-tooltip}
Positional parameters are specified in order, without names.

The function to apply to each item. Must return a boolean.

\subsubsection{\texorpdfstring{\texttt{\ position\ }}{ position }}\label{definitions-position}

Searches for an item for which the given function returns
\texttt{\ }{\texttt{\ true\ }}\texttt{\ } and returns the index of the
first match or \texttt{\ }{\texttt{\ none\ }}\texttt{\ } if there is no
match.

self { . } { position } (

{ \href{/docs/reference/foundations/function/}{function} }

) -\textgreater{} \href{/docs/reference/foundations/none/}{none}
\href{/docs/reference/foundations/int/}{int}

\paragraph{\texorpdfstring{\texttt{\ searcher\ }}{ searcher }}\label{definitions-position-searcher}

\href{/docs/reference/foundations/function/}{function}

{Required} {{ Positional }}

\phantomsection\label{definitions-position-searcher-positional-tooltip}
Positional parameters are specified in order, without names.

The function to apply to each item. Must return a boolean.

\subsubsection{\texorpdfstring{\texttt{\ range\ }}{ range }}\label{definitions-range}

Create an array consisting of a sequence of numbers.

If you pass just one positional parameter, it is interpreted as the
\texttt{\ end\ } of the range. If you pass two, they describe the
\texttt{\ start\ } and \texttt{\ end\ } of the range.

This function is available both in the array function\textquotesingle s
scope and globally.

array { . } { range } (

{ \href{/docs/reference/foundations/int/}{int} , } {
\href{/docs/reference/foundations/int/}{int} , } {
\hyperref[definitions-range-parameters-step]{step :}
\href{/docs/reference/foundations/int/}{int} , }

) -\textgreater{} \href{/docs/reference/foundations/array/}{array}

\begin{verbatim}
#range(5) \
#range(2, 5) \
#range(20, step: 4) \
#range(21, step: 4) \
#range(5, 2, step: -1)
\end{verbatim}

\includegraphics[width=5in,height=\textheight,keepaspectratio]{/assets/docs/zrh5Y9Alyv5p1PUCuyz0bAAAAAAAAAAA.png}

\paragraph{\texorpdfstring{\texttt{\ start\ }}{ start }}\label{definitions-range-start}

\href{/docs/reference/foundations/int/}{int}

{{ Positional }}

\phantomsection\label{definitions-range-start-positional-tooltip}
Positional parameters are specified in order, without names.

The start of the range (inclusive).

Default: \texttt{\ }{\texttt{\ 0\ }}\texttt{\ }

\paragraph{\texorpdfstring{\texttt{\ end\ }}{ end }}\label{definitions-range-end}

\href{/docs/reference/foundations/int/}{int}

{Required} {{ Positional }}

\phantomsection\label{definitions-range-end-positional-tooltip}
Positional parameters are specified in order, without names.

The end of the range (exclusive).

\paragraph{\texorpdfstring{\texttt{\ step\ }}{ step }}\label{definitions-range-step}

\href{/docs/reference/foundations/int/}{int}

The distance between the generated numbers.

Default: \texttt{\ }{\texttt{\ 1\ }}\texttt{\ }

\subsubsection{\texorpdfstring{\texttt{\ filter\ }}{ filter }}\label{definitions-filter}

Produces a new array with only the items from the original one for which
the given function returns true.

self { . } { filter } (

{ \href{/docs/reference/foundations/function/}{function} }

) -\textgreater{} \href{/docs/reference/foundations/array/}{array}

\paragraph{\texorpdfstring{\texttt{\ test\ }}{ test }}\label{definitions-filter-test}

\href{/docs/reference/foundations/function/}{function}

{Required} {{ Positional }}

\phantomsection\label{definitions-filter-test-positional-tooltip}
Positional parameters are specified in order, without names.

The function to apply to each item. Must return a boolean.

\subsubsection{\texorpdfstring{\texttt{\ map\ }}{ map }}\label{definitions-map}

Produces a new array in which all items from the original one were
transformed with the given function.

self { . } { map } (

{ \href{/docs/reference/foundations/function/}{function} }

) -\textgreater{} \href{/docs/reference/foundations/array/}{array}

\paragraph{\texorpdfstring{\texttt{\ mapper\ }}{ mapper }}\label{definitions-map-mapper}

\href{/docs/reference/foundations/function/}{function}

{Required} {{ Positional }}

\phantomsection\label{definitions-map-mapper-positional-tooltip}
Positional parameters are specified in order, without names.

The function to apply to each item.

\subsubsection{\texorpdfstring{\texttt{\ enumerate\ }}{ enumerate }}\label{definitions-enumerate}

Returns a new array with the values alongside their indices.

The returned array consists of \texttt{\ (index,\ value)\ } pairs in the
form of length-2 arrays. These can be
\href{/docs/reference/scripting/\#bindings}{destructured} with a let
binding or for loop.

self { . } { enumerate } (

{ \hyperref[definitions-enumerate-parameters-start]{start :}
\href{/docs/reference/foundations/int/}{int} }

) -\textgreater{} \href{/docs/reference/foundations/array/}{array}

\paragraph{\texorpdfstring{\texttt{\ start\ }}{ start }}\label{definitions-enumerate-start}

\href{/docs/reference/foundations/int/}{int}

The index returned for the first pair of the returned list.

Default: \texttt{\ }{\texttt{\ 0\ }}\texttt{\ }

\subsubsection{\texorpdfstring{\texttt{\ zip\ }}{ zip }}\label{definitions-zip}

Zips the array with other arrays.

Returns an array of arrays, where the \texttt{\ i\ } th inner array
contains all the \texttt{\ i\ } th elements from each original array.

If the arrays to be zipped have different lengths, they are zipped up to
the last element of the shortest array and all remaining elements are
ignored.

This function is variadic, meaning that you can zip multiple arrays
together at once:
\texttt{\ }{\texttt{\ (\ }}\texttt{\ }{\texttt{\ 1\ }}\texttt{\ }{\texttt{\ ,\ }}\texttt{\ }{\texttt{\ 2\ }}\texttt{\ }{\texttt{\ )\ }}\texttt{\ }{\texttt{\ .\ }}\texttt{\ }{\texttt{\ zip\ }}\texttt{\ }{\texttt{\ (\ }}\texttt{\ }{\texttt{\ (\ }}\texttt{\ }{\texttt{\ "A"\ }}\texttt{\ }{\texttt{\ ,\ }}\texttt{\ }{\texttt{\ "B"\ }}\texttt{\ }{\texttt{\ )\ }}\texttt{\ }{\texttt{\ ,\ }}\texttt{\ }{\texttt{\ (\ }}\texttt{\ }{\texttt{\ 10\ }}\texttt{\ }{\texttt{\ ,\ }}\texttt{\ }{\texttt{\ 20\ }}\texttt{\ }{\texttt{\ )\ }}\texttt{\ }{\texttt{\ )\ }}\texttt{\ }
yields
\texttt{\ }{\texttt{\ (\ }}\texttt{\ }{\texttt{\ (\ }}\texttt{\ }{\texttt{\ 1\ }}\texttt{\ }{\texttt{\ ,\ }}\texttt{\ }{\texttt{\ "A"\ }}\texttt{\ }{\texttt{\ ,\ }}\texttt{\ }{\texttt{\ 10\ }}\texttt{\ }{\texttt{\ )\ }}\texttt{\ }{\texttt{\ ,\ }}\texttt{\ }{\texttt{\ (\ }}\texttt{\ }{\texttt{\ 2\ }}\texttt{\ }{\texttt{\ ,\ }}\texttt{\ }{\texttt{\ "B"\ }}\texttt{\ }{\texttt{\ ,\ }}\texttt{\ }{\texttt{\ 20\ }}\texttt{\ }{\texttt{\ )\ }}\texttt{\ }{\texttt{\ )\ }}\texttt{\ }
.

self { . } { zip } (

{ \hyperref[definitions-zip-parameters-exact]{exact :}
\href{/docs/reference/foundations/bool/}{bool} , } {
\hyperref[definitions-zip-parameters-others]{..}
\href{/docs/reference/foundations/array/}{array} , }

) -\textgreater{} \href{/docs/reference/foundations/array/}{array}

\paragraph{\texorpdfstring{\texttt{\ exact\ }}{ exact }}\label{definitions-zip-exact}

\href{/docs/reference/foundations/bool/}{bool}

Whether all arrays have to have the same length. For example,
\texttt{\ }{\texttt{\ (\ }}\texttt{\ }{\texttt{\ 1\ }}\texttt{\ }{\texttt{\ ,\ }}\texttt{\ }{\texttt{\ 2\ }}\texttt{\ }{\texttt{\ )\ }}\texttt{\ }{\texttt{\ .\ }}\texttt{\ }{\texttt{\ zip\ }}\texttt{\ }{\texttt{\ (\ }}\texttt{\ }{\texttt{\ (\ }}\texttt{\ }{\texttt{\ 1\ }}\texttt{\ }{\texttt{\ ,\ }}\texttt{\ }{\texttt{\ 2\ }}\texttt{\ }{\texttt{\ ,\ }}\texttt{\ }{\texttt{\ 3\ }}\texttt{\ }{\texttt{\ )\ }}\texttt{\ }{\texttt{\ ,\ }}\texttt{\ exact\ }{\texttt{\ :\ }}\texttt{\ }{\texttt{\ true\ }}\texttt{\ }{\texttt{\ )\ }}\texttt{\ }
produces an error.

Default: \texttt{\ }{\texttt{\ false\ }}\texttt{\ }

\paragraph{\texorpdfstring{\texttt{\ others\ }}{ others }}\label{definitions-zip-others}

\href{/docs/reference/foundations/array/}{array}

{Required} {{ Positional }}

\phantomsection\label{definitions-zip-others-positional-tooltip}
Positional parameters are specified in order, without names.

{{ Variadic }}

\phantomsection\label{definitions-zip-others-variadic-tooltip}
Variadic parameters can be specified multiple times.

The arrays to zip with.

\subsubsection{\texorpdfstring{\texttt{\ fold\ }}{ fold }}\label{definitions-fold}

Folds all items into a single value using an accumulator function.

self { . } { fold } (

{ { any } , } { \href{/docs/reference/foundations/function/}{function} ,
}

) -\textgreater{} { any }

\paragraph{\texorpdfstring{\texttt{\ init\ }}{ init }}\label{definitions-fold-init}

{ any }

{Required} {{ Positional }}

\phantomsection\label{definitions-fold-init-positional-tooltip}
Positional parameters are specified in order, without names.

The initial value to start with.

\paragraph{\texorpdfstring{\texttt{\ folder\ }}{ folder }}\label{definitions-fold-folder}

\href{/docs/reference/foundations/function/}{function}

{Required} {{ Positional }}

\phantomsection\label{definitions-fold-folder-positional-tooltip}
Positional parameters are specified in order, without names.

The folding function. Must have two parameters: One for the accumulated
value and one for an item.

\subsubsection{\texorpdfstring{\texttt{\ sum\ }}{ sum }}\label{definitions-sum}

Sums all items (works for all types that can be added).

self { . } { sum } (

{ \hyperref[definitions-sum-parameters-default]{default :} { any } }

) -\textgreater{} { any }

\paragraph{\texorpdfstring{\texttt{\ default\ }}{ default }}\label{definitions-sum-default}

{ any }

What to return if the array is empty. Must be set if the array can be
empty.

\subsubsection{\texorpdfstring{\texttt{\ product\ }}{ product }}\label{definitions-product}

Calculates the product all items (works for all types that can be
multiplied).

self { . } { product } (

{ \hyperref[definitions-product-parameters-default]{default :} { any } }

) -\textgreater{} { any }

\paragraph{\texorpdfstring{\texttt{\ default\ }}{ default }}\label{definitions-product-default}

{ any }

What to return if the array is empty. Must be set if the array can be
empty.

\subsubsection{\texorpdfstring{\texttt{\ any\ }}{ any }}\label{definitions-any}

Whether the given function returns
\texttt{\ }{\texttt{\ true\ }}\texttt{\ } for any item in the array.

self { . } { any } (

{ \href{/docs/reference/foundations/function/}{function} }

) -\textgreater{} \href{/docs/reference/foundations/bool/}{bool}

\paragraph{\texorpdfstring{\texttt{\ test\ }}{ test }}\label{definitions-any-test}

\href{/docs/reference/foundations/function/}{function}

{Required} {{ Positional }}

\phantomsection\label{definitions-any-test-positional-tooltip}
Positional parameters are specified in order, without names.

The function to apply to each item. Must return a boolean.

\subsubsection{\texorpdfstring{\texttt{\ all\ }}{ all }}\label{definitions-all}

Whether the given function returns
\texttt{\ }{\texttt{\ true\ }}\texttt{\ } for all items in the array.

self { . } { all } (

{ \href{/docs/reference/foundations/function/}{function} }

) -\textgreater{} \href{/docs/reference/foundations/bool/}{bool}

\paragraph{\texorpdfstring{\texttt{\ test\ }}{ test }}\label{definitions-all-test}

\href{/docs/reference/foundations/function/}{function}

{Required} {{ Positional }}

\phantomsection\label{definitions-all-test-positional-tooltip}
Positional parameters are specified in order, without names.

The function to apply to each item. Must return a boolean.

\subsubsection{\texorpdfstring{\texttt{\ flatten\ }}{ flatten }}\label{definitions-flatten}

Combine all nested arrays into a single flat one.

self { . } { flatten } (

) -\textgreater{} \href{/docs/reference/foundations/array/}{array}

\subsubsection{\texorpdfstring{\texttt{\ rev\ }}{ rev }}\label{definitions-rev}

Return a new array with the same items, but in reverse order.

self { . } { rev } (

) -\textgreater{} \href{/docs/reference/foundations/array/}{array}

\subsubsection{\texorpdfstring{\texttt{\ split\ }}{ split }}\label{definitions-split}

Split the array at occurrences of the specified value.

self { . } { split } (

{ { any } }

) -\textgreater{} \href{/docs/reference/foundations/array/}{array}

\paragraph{\texorpdfstring{\texttt{\ at\ }}{ at }}\label{definitions-split-at}

{ any }

{Required} {{ Positional }}

\phantomsection\label{definitions-split-at-positional-tooltip}
Positional parameters are specified in order, without names.

The value to split at.

\subsubsection{\texorpdfstring{\texttt{\ join\ }}{ join }}\label{definitions-join}

Combine all items in the array into one.

self { . } { join } (

{ { any } \href{/docs/reference/foundations/none/}{none} , } {
\hyperref[definitions-join-parameters-last]{last :} { any } , }

) -\textgreater{} { any }

\paragraph{\texorpdfstring{\texttt{\ separator\ }}{ separator }}\label{definitions-join-separator}

{ any } {or} \href{/docs/reference/foundations/none/}{none}

{{ Positional }}

\phantomsection\label{definitions-join-separator-positional-tooltip}
Positional parameters are specified in order, without names.

A value to insert between each item of the array.

Default: \texttt{\ }{\texttt{\ none\ }}\texttt{\ }

\paragraph{\texorpdfstring{\texttt{\ last\ }}{ last }}\label{definitions-join-last}

{ any }

An alternative separator between the last two items.

\subsubsection{\texorpdfstring{\texttt{\ intersperse\ }}{ intersperse }}\label{definitions-intersperse}

Returns an array with a copy of the separator value placed between
adjacent elements.

self { . } { intersperse } (

{ { any } }

) -\textgreater{} \href{/docs/reference/foundations/array/}{array}

\paragraph{\texorpdfstring{\texttt{\ separator\ }}{ separator }}\label{definitions-intersperse-separator}

{ any }

{Required} {{ Positional }}

\phantomsection\label{definitions-intersperse-separator-positional-tooltip}
Positional parameters are specified in order, without names.

The value that will be placed between each adjacent element.

\subsubsection{\texorpdfstring{\texttt{\ chunks\ }}{ chunks }}\label{definitions-chunks}

Splits an array into non-overlapping chunks, starting at the beginning,
ending with a single remainder chunk.

All chunks but the last have \texttt{\ chunk-size\ } elements. If
\texttt{\ exact\ } is set to \texttt{\ }{\texttt{\ true\ }}\texttt{\ } ,
the remainder is dropped if it contains less than
\texttt{\ chunk-size\ } elements.

self { . } { chunks } (

{ \href{/docs/reference/foundations/int/}{int} , } {
\hyperref[definitions-chunks-parameters-exact]{exact :}
\href{/docs/reference/foundations/bool/}{bool} , }

) -\textgreater{} \href{/docs/reference/foundations/array/}{array}

\begin{verbatim}
#let array = (1, 2, 3, 4, 5, 6, 7, 8)
#array.chunks(3)
#array.chunks(3, exact: true)
\end{verbatim}

\includegraphics[width=5in,height=\textheight,keepaspectratio]{/assets/docs/Nt1-jyrzTUv2d90xr98pvAAAAAAAAAAA.png}

\paragraph{\texorpdfstring{\texttt{\ chunk-size\ }}{ chunk-size }}\label{definitions-chunks-chunk-size}

\href{/docs/reference/foundations/int/}{int}

{Required} {{ Positional }}

\phantomsection\label{definitions-chunks-chunk-size-positional-tooltip}
Positional parameters are specified in order, without names.

How many elements each chunk may at most contain.

\paragraph{\texorpdfstring{\texttt{\ exact\ }}{ exact }}\label{definitions-chunks-exact}

\href{/docs/reference/foundations/bool/}{bool}

Whether to keep the remainder if its size is less than
\texttt{\ chunk-size\ } .

Default: \texttt{\ }{\texttt{\ false\ }}\texttt{\ }

\subsubsection{\texorpdfstring{\texttt{\ windows\ }}{ windows }}\label{definitions-windows}

Returns sliding windows of \texttt{\ window-size\ } elements over an
array.

If the array length is less than \texttt{\ window-size\ } , this will
return an empty array.

self { . } { windows } (

{ \href{/docs/reference/foundations/int/}{int} }

) -\textgreater{} \href{/docs/reference/foundations/array/}{array}

\begin{verbatim}
#let array = (1, 2, 3, 4, 5, 6, 7, 8)
#array.windows(5)
\end{verbatim}

\includegraphics[width=5in,height=\textheight,keepaspectratio]{/assets/docs/Gacy-jUdBfccX43fxzwqPgAAAAAAAAAA.png}

\paragraph{\texorpdfstring{\texttt{\ window-size\ }}{ window-size }}\label{definitions-windows-window-size}

\href{/docs/reference/foundations/int/}{int}

{Required} {{ Positional }}

\phantomsection\label{definitions-windows-window-size-positional-tooltip}
Positional parameters are specified in order, without names.

How many elements each window will contain.

\subsubsection{\texorpdfstring{\texttt{\ sorted\ }}{ sorted }}\label{definitions-sorted}

Return a sorted version of this array, optionally by a given key
function. The sorting algorithm used is stable.

Returns an error if two values could not be compared or if the key
function (if given) yields an error.

self { . } { sorted } (

{ \hyperref[definitions-sorted-parameters-key]{key :}
\href{/docs/reference/foundations/function/}{function} }

) -\textgreater{} \href{/docs/reference/foundations/array/}{array}

\paragraph{\texorpdfstring{\texttt{\ key\ }}{ key }}\label{definitions-sorted-key}

\href{/docs/reference/foundations/function/}{function}

If given, applies this function to the elements in the array to
determine the keys to sort by.

\subsubsection{\texorpdfstring{\texttt{\ dedup\ }}{ dedup }}\label{definitions-dedup}

Deduplicates all items in the array.

Returns a new array with all duplicate items removed. Only the first
element of each duplicate is kept.

self { . } { dedup } (

{ \hyperref[definitions-dedup-parameters-key]{key :}
\href{/docs/reference/foundations/function/}{function} }

) -\textgreater{} \href{/docs/reference/foundations/array/}{array}

\begin{verbatim}
#(1, 1, 2, 3, 1).dedup()
\end{verbatim}

\includegraphics[width=5in,height=\textheight,keepaspectratio]{/assets/docs/N8Cp27Nhseeu9VhP3b-g0gAAAAAAAAAA.png}

\paragraph{\texorpdfstring{\texttt{\ key\ }}{ key }}\label{definitions-dedup-key}

\href{/docs/reference/foundations/function/}{function}

If given, applies this function to the elements in the array to
determine the keys to deduplicate by.

\subsubsection{\texorpdfstring{\texttt{\ to-dict\ }}{ to-dict }}\label{definitions-to-dict}

Converts an array of pairs into a dictionary. The first value of each
pair is the key, the second the value.

If the same key occurs multiple times, the last value is selected.

self { . } { to-dict } (

) -\textgreater{}
\href{/docs/reference/foundations/dictionary/}{dictionary}

\begin{verbatim}
#(
  ("apples", 2),
  ("peaches", 3),
  ("apples", 5),
).to-dict()
\end{verbatim}

\includegraphics[width=5in,height=\textheight,keepaspectratio]{/assets/docs/LmuORFz3ft0CLd-WiUZHngAAAAAAAAAA.png}

\subsubsection{\texorpdfstring{\texttt{\ reduce\ }}{ reduce }}\label{definitions-reduce}

Reduces the elements to a single one, by repeatedly applying a reducing
operation.

If the array is empty, returns \texttt{\ }{\texttt{\ none\ }}\texttt{\ }
, otherwise, returns the result of the reduction.

The reducing function is a closure with two arguments: an "accumulator",
and an element.

For arrays with at least one element, this is the same as
\href{/docs/reference/foundations/array/\#definitions-fold}{\texttt{\ array.fold\ }}
with the first element of the array as the initial accumulator value,
folding every subsequent element into it.

self { . } { reduce } (

{ \href{/docs/reference/foundations/function/}{function} }

) -\textgreater{} { any }

\paragraph{\texorpdfstring{\texttt{\ reducer\ }}{ reducer }}\label{definitions-reduce-reducer}

\href{/docs/reference/foundations/function/}{function}

{Required} {{ Positional }}

\phantomsection\label{definitions-reduce-reducer-positional-tooltip}
Positional parameters are specified in order, without names.

The reducing function. Must have two parameters: One for the accumulated
value and one for an item.

\href{/docs/reference/foundations/arguments/}{\pandocbounded{\includesvg[keepaspectratio]{/assets/icons/16-arrow-right.svg}}}

{ Arguments } { Previous page }

\href{/docs/reference/foundations/assert/}{\pandocbounded{\includesvg[keepaspectratio]{/assets/icons/16-arrow-right.svg}}}

{ Assert } { Next page }


\section{Docs LaTeX/typst.app/docs/reference/foundations/calc.tex}
\title{typst.app/docs/reference/foundations/calc}

\begin{itemize}
\tightlist
\item
  \href{/docs}{\includesvg[width=0.16667in,height=0.16667in]{/assets/icons/16-docs-dark.svg}}
\item
  \includesvg[width=0.16667in,height=0.16667in]{/assets/icons/16-arrow-right.svg}
\item
  \href{/docs/reference/}{Reference}
\item
  \includesvg[width=0.16667in,height=0.16667in]{/assets/icons/16-arrow-right.svg}
\item
  \href{/docs/reference/foundations/}{Foundations}
\item
  \includesvg[width=0.16667in,height=0.16667in]{/assets/icons/16-arrow-right.svg}
\item
  \href{/docs/reference/foundations/calc}{Calculation}
\end{itemize}

\section{Calculation}\label{summary}

Module for calculations and processing of numeric values.

These definitions are part of the \texttt{\ calc\ } module and not
imported by default. In addition to the functions listed below, the
\texttt{\ calc\ } module also defines the constants \texttt{\ pi\ } ,
\texttt{\ tau\ } , \texttt{\ e\ } , and \texttt{\ inf\ } .

\subsection{Functions}\label{functions}

\subsubsection{\texorpdfstring{\texttt{\ abs\ }}{ abs }}\label{functions-abs}

Calculates the absolute value of a numeric value.

calc { . } { abs } (

{ \href{/docs/reference/foundations/int/}{int}
\href{/docs/reference/foundations/float/}{float}
\href{/docs/reference/layout/length/}{length}
\href{/docs/reference/layout/angle/}{angle}
\href{/docs/reference/layout/ratio/}{ratio}
\href{/docs/reference/layout/fraction/}{fraction}
\href{/docs/reference/foundations/decimal/}{decimal} }

) -\textgreater{} { any }

\begin{verbatim}
#calc.abs(-5) \
#calc.abs(5pt - 2cm) \
#calc.abs(2fr) \
#calc.abs(decimal("-342.440"))
\end{verbatim}

\includegraphics[width=5in,height=\textheight,keepaspectratio]{/assets/docs/1nPNk-RAyXUEHrAszyCnUgAAAAAAAAAA.png}

\paragraph{\texorpdfstring{\texttt{\ value\ }}{ value }}\label{functions-abs-value}

\href{/docs/reference/foundations/int/}{int} {or}
\href{/docs/reference/foundations/float/}{float} {or}
\href{/docs/reference/layout/length/}{length} {or}
\href{/docs/reference/layout/angle/}{angle} {or}
\href{/docs/reference/layout/ratio/}{ratio} {or}
\href{/docs/reference/layout/fraction/}{fraction} {or}
\href{/docs/reference/foundations/decimal/}{decimal}

{Required} {{ Positional }}

\phantomsection\label{functions-abs-value-positional-tooltip}
Positional parameters are specified in order, without names.

The value whose absolute value to calculate.

\subsubsection{\texorpdfstring{\texttt{\ pow\ }}{ pow }}\label{functions-pow}

Raises a value to some exponent.

calc { . } { pow } (

{ \href{/docs/reference/foundations/int/}{int}
\href{/docs/reference/foundations/float/}{float}
\href{/docs/reference/foundations/decimal/}{decimal} , } {
\href{/docs/reference/foundations/int/}{int}
\href{/docs/reference/foundations/float/}{float} , }

) -\textgreater{} \href{/docs/reference/foundations/int/}{int}
\href{/docs/reference/foundations/float/}{float}
\href{/docs/reference/foundations/decimal/}{decimal}

\begin{verbatim}
#calc.pow(2, 3) \
#calc.pow(decimal("2.5"), 2)
\end{verbatim}

\includegraphics[width=5in,height=\textheight,keepaspectratio]{/assets/docs/YQoOsFNxPEgW0b-n9B_VrAAAAAAAAAAA.png}

\paragraph{\texorpdfstring{\texttt{\ base\ }}{ base }}\label{functions-pow-base}

\href{/docs/reference/foundations/int/}{int} {or}
\href{/docs/reference/foundations/float/}{float} {or}
\href{/docs/reference/foundations/decimal/}{decimal}

{Required} {{ Positional }}

\phantomsection\label{functions-pow-base-positional-tooltip}
Positional parameters are specified in order, without names.

The base of the power.

If this is a
\href{/docs/reference/foundations/decimal/}{\texttt{\ decimal\ }} , the
exponent can only be an \href{/docs/reference/foundations/int/}{integer}
.

\paragraph{\texorpdfstring{\texttt{\ exponent\ }}{ exponent }}\label{functions-pow-exponent}

\href{/docs/reference/foundations/int/}{int} {or}
\href{/docs/reference/foundations/float/}{float}

{Required} {{ Positional }}

\phantomsection\label{functions-pow-exponent-positional-tooltip}
Positional parameters are specified in order, without names.

The exponent of the power.

\subsubsection{\texorpdfstring{\texttt{\ exp\ }}{ exp }}\label{functions-exp}

Raises a value to some exponent of e.

calc { . } { exp } (

{ \href{/docs/reference/foundations/int/}{int}
\href{/docs/reference/foundations/float/}{float} }

) -\textgreater{} \href{/docs/reference/foundations/float/}{float}

\begin{verbatim}
#calc.exp(1)
\end{verbatim}

\includegraphics[width=5in,height=\textheight,keepaspectratio]{/assets/docs/D3jiA5mgoQIx6MVn4Oy4zwAAAAAAAAAA.png}

\paragraph{\texorpdfstring{\texttt{\ exponent\ }}{ exponent }}\label{functions-exp-exponent}

\href{/docs/reference/foundations/int/}{int} {or}
\href{/docs/reference/foundations/float/}{float}

{Required} {{ Positional }}

\phantomsection\label{functions-exp-exponent-positional-tooltip}
Positional parameters are specified in order, without names.

The exponent of the power.

\subsubsection{\texorpdfstring{\texttt{\ sqrt\ }}{ sqrt }}\label{functions-sqrt}

Calculates the square root of a number.

calc { . } { sqrt } (

{ \href{/docs/reference/foundations/int/}{int}
\href{/docs/reference/foundations/float/}{float} }

) -\textgreater{} \href{/docs/reference/foundations/float/}{float}

\begin{verbatim}
#calc.sqrt(16) \
#calc.sqrt(2.5)
\end{verbatim}

\includegraphics[width=5in,height=\textheight,keepaspectratio]{/assets/docs/rSjz1bWkkKYqxmWjxezFTwAAAAAAAAAA.png}

\paragraph{\texorpdfstring{\texttt{\ value\ }}{ value }}\label{functions-sqrt-value}

\href{/docs/reference/foundations/int/}{int} {or}
\href{/docs/reference/foundations/float/}{float}

{Required} {{ Positional }}

\phantomsection\label{functions-sqrt-value-positional-tooltip}
Positional parameters are specified in order, without names.

The number whose square root to calculate. Must be non-negative.

\subsubsection{\texorpdfstring{\texttt{\ root\ }}{ root }}\label{functions-root}

Calculates the real nth root of a number.

If the number is negative, then n must be odd.

calc { . } { root } (

{ \href{/docs/reference/foundations/float/}{float} , } {
\href{/docs/reference/foundations/int/}{int} , }

) -\textgreater{} \href{/docs/reference/foundations/float/}{float}

\begin{verbatim}
#calc.root(16.0, 4) \
#calc.root(27.0, 3)
\end{verbatim}

\includegraphics[width=5in,height=\textheight,keepaspectratio]{/assets/docs/g3rxlqoTGgoCjLtiE7bcKAAAAAAAAAAA.png}

\paragraph{\texorpdfstring{\texttt{\ radicand\ }}{ radicand }}\label{functions-root-radicand}

\href{/docs/reference/foundations/float/}{float}

{Required} {{ Positional }}

\phantomsection\label{functions-root-radicand-positional-tooltip}
Positional parameters are specified in order, without names.

The expression to take the root of

\paragraph{\texorpdfstring{\texttt{\ index\ }}{ index }}\label{functions-root-index}

\href{/docs/reference/foundations/int/}{int}

{Required} {{ Positional }}

\phantomsection\label{functions-root-index-positional-tooltip}
Positional parameters are specified in order, without names.

Which root of the radicand to take

\subsubsection{\texorpdfstring{\texttt{\ sin\ }}{ sin }}\label{functions-sin}

Calculates the sine of an angle.

When called with an integer or a float, they will be interpreted as
radians.

calc { . } { sin } (

{ \href{/docs/reference/foundations/int/}{int}
\href{/docs/reference/foundations/float/}{float}
\href{/docs/reference/layout/angle/}{angle} }

) -\textgreater{} \href{/docs/reference/foundations/float/}{float}

\begin{verbatim}
#calc.sin(1.5) \
#calc.sin(90deg)
\end{verbatim}

\includegraphics[width=5in,height=\textheight,keepaspectratio]{/assets/docs/DRz-f64JvhHssm4WgSEL2QAAAAAAAAAA.png}

\paragraph{\texorpdfstring{\texttt{\ angle\ }}{ angle }}\label{functions-sin-angle}

\href{/docs/reference/foundations/int/}{int} {or}
\href{/docs/reference/foundations/float/}{float} {or}
\href{/docs/reference/layout/angle/}{angle}

{Required} {{ Positional }}

\phantomsection\label{functions-sin-angle-positional-tooltip}
Positional parameters are specified in order, without names.

The angle whose sine to calculate.

\subsubsection{\texorpdfstring{\texttt{\ cos\ }}{ cos }}\label{functions-cos}

Calculates the cosine of an angle.

When called with an integer or a float, they will be interpreted as
radians.

calc { . } { cos } (

{ \href{/docs/reference/foundations/int/}{int}
\href{/docs/reference/foundations/float/}{float}
\href{/docs/reference/layout/angle/}{angle} }

) -\textgreater{} \href{/docs/reference/foundations/float/}{float}

\begin{verbatim}
#calc.cos(1.5) \
#calc.cos(90deg)
\end{verbatim}

\includegraphics[width=5in,height=\textheight,keepaspectratio]{/assets/docs/RQec6gdJF5QvRwZkvI-50gAAAAAAAAAA.png}

\paragraph{\texorpdfstring{\texttt{\ angle\ }}{ angle }}\label{functions-cos-angle}

\href{/docs/reference/foundations/int/}{int} {or}
\href{/docs/reference/foundations/float/}{float} {or}
\href{/docs/reference/layout/angle/}{angle}

{Required} {{ Positional }}

\phantomsection\label{functions-cos-angle-positional-tooltip}
Positional parameters are specified in order, without names.

The angle whose cosine to calculate.

\subsubsection{\texorpdfstring{\texttt{\ tan\ }}{ tan }}\label{functions-tan}

Calculates the tangent of an angle.

When called with an integer or a float, they will be interpreted as
radians.

calc { . } { tan } (

{ \href{/docs/reference/foundations/int/}{int}
\href{/docs/reference/foundations/float/}{float}
\href{/docs/reference/layout/angle/}{angle} }

) -\textgreater{} \href{/docs/reference/foundations/float/}{float}

\begin{verbatim}
#calc.tan(1.5) \
#calc.tan(90deg)
\end{verbatim}

\includegraphics[width=5in,height=\textheight,keepaspectratio]{/assets/docs/Mu6UfN_4464KJhy78wvp_wAAAAAAAAAA.png}

\paragraph{\texorpdfstring{\texttt{\ angle\ }}{ angle }}\label{functions-tan-angle}

\href{/docs/reference/foundations/int/}{int} {or}
\href{/docs/reference/foundations/float/}{float} {or}
\href{/docs/reference/layout/angle/}{angle}

{Required} {{ Positional }}

\phantomsection\label{functions-tan-angle-positional-tooltip}
Positional parameters are specified in order, without names.

The angle whose tangent to calculate.

\subsubsection{\texorpdfstring{\texttt{\ asin\ }}{ asin }}\label{functions-asin}

Calculates the arcsine of a number.

calc { . } { asin } (

{ \href{/docs/reference/foundations/int/}{int}
\href{/docs/reference/foundations/float/}{float} }

) -\textgreater{} \href{/docs/reference/layout/angle/}{angle}

\begin{verbatim}
#calc.asin(0) \
#calc.asin(1)
\end{verbatim}

\includegraphics[width=5in,height=\textheight,keepaspectratio]{/assets/docs/R-fP2bsKqek6CrHRxsmlvQAAAAAAAAAA.png}

\paragraph{\texorpdfstring{\texttt{\ value\ }}{ value }}\label{functions-asin-value}

\href{/docs/reference/foundations/int/}{int} {or}
\href{/docs/reference/foundations/float/}{float}

{Required} {{ Positional }}

\phantomsection\label{functions-asin-value-positional-tooltip}
Positional parameters are specified in order, without names.

The number whose arcsine to calculate. Must be between -1 and 1.

\subsubsection{\texorpdfstring{\texttt{\ acos\ }}{ acos }}\label{functions-acos}

Calculates the arccosine of a number.

calc { . } { acos } (

{ \href{/docs/reference/foundations/int/}{int}
\href{/docs/reference/foundations/float/}{float} }

) -\textgreater{} \href{/docs/reference/layout/angle/}{angle}

\begin{verbatim}
#calc.acos(0) \
#calc.acos(1)
\end{verbatim}

\includegraphics[width=5in,height=\textheight,keepaspectratio]{/assets/docs/34tvtgPRx9Zb0oFQtdkEngAAAAAAAAAA.png}

\paragraph{\texorpdfstring{\texttt{\ value\ }}{ value }}\label{functions-acos-value}

\href{/docs/reference/foundations/int/}{int} {or}
\href{/docs/reference/foundations/float/}{float}

{Required} {{ Positional }}

\phantomsection\label{functions-acos-value-positional-tooltip}
Positional parameters are specified in order, without names.

The number whose arcsine to calculate. Must be between -1 and 1.

\subsubsection{\texorpdfstring{\texttt{\ atan\ }}{ atan }}\label{functions-atan}

Calculates the arctangent of a number.

calc { . } { atan } (

{ \href{/docs/reference/foundations/int/}{int}
\href{/docs/reference/foundations/float/}{float} }

) -\textgreater{} \href{/docs/reference/layout/angle/}{angle}

\begin{verbatim}
#calc.atan(0) \
#calc.atan(1)
\end{verbatim}

\includegraphics[width=5in,height=\textheight,keepaspectratio]{/assets/docs/Ks5iB4MwWXNAVXeSMihDJAAAAAAAAAAA.png}

\paragraph{\texorpdfstring{\texttt{\ value\ }}{ value }}\label{functions-atan-value}

\href{/docs/reference/foundations/int/}{int} {or}
\href{/docs/reference/foundations/float/}{float}

{Required} {{ Positional }}

\phantomsection\label{functions-atan-value-positional-tooltip}
Positional parameters are specified in order, without names.

The number whose arctangent to calculate.

\subsubsection{\texorpdfstring{\texttt{\ atan2\ }}{ atan2 }}\label{functions-atan2}

Calculates the four-quadrant arctangent of a coordinate.

The arguments are \texttt{\ (x,\ y)\ } , not \texttt{\ (y,\ x)\ } .

calc { . } { atan2 } (

{ \href{/docs/reference/foundations/int/}{int}
\href{/docs/reference/foundations/float/}{float} , } {
\href{/docs/reference/foundations/int/}{int}
\href{/docs/reference/foundations/float/}{float} , }

) -\textgreater{} \href{/docs/reference/layout/angle/}{angle}

\begin{verbatim}
#calc.atan2(1, 1) \
#calc.atan2(-2, -3)
\end{verbatim}

\includegraphics[width=5in,height=\textheight,keepaspectratio]{/assets/docs/R3PgftYITRsSLYBeQqKe3wAAAAAAAAAA.png}

\paragraph{\texorpdfstring{\texttt{\ x\ }}{ x }}\label{functions-atan2-x}

\href{/docs/reference/foundations/int/}{int} {or}
\href{/docs/reference/foundations/float/}{float}

{Required} {{ Positional }}

\phantomsection\label{functions-atan2-x-positional-tooltip}
Positional parameters are specified in order, without names.

The X coordinate.

\paragraph{\texorpdfstring{\texttt{\ y\ }}{ y }}\label{functions-atan2-y}

\href{/docs/reference/foundations/int/}{int} {or}
\href{/docs/reference/foundations/float/}{float}

{Required} {{ Positional }}

\phantomsection\label{functions-atan2-y-positional-tooltip}
Positional parameters are specified in order, without names.

The Y coordinate.

\subsubsection{\texorpdfstring{\texttt{\ sinh\ }}{ sinh }}\label{functions-sinh}

Calculates the hyperbolic sine of a hyperbolic angle.

calc { . } { sinh } (

{ \href{/docs/reference/foundations/float/}{float} }

) -\textgreater{} \href{/docs/reference/foundations/float/}{float}

\begin{verbatim}
#calc.sinh(0) \
#calc.sinh(1.5)
\end{verbatim}

\includegraphics[width=5in,height=\textheight,keepaspectratio]{/assets/docs/Si7LVr220y-yjr6frD5mYQAAAAAAAAAA.png}

\paragraph{\texorpdfstring{\texttt{\ value\ }}{ value }}\label{functions-sinh-value}

\href{/docs/reference/foundations/float/}{float}

{Required} {{ Positional }}

\phantomsection\label{functions-sinh-value-positional-tooltip}
Positional parameters are specified in order, without names.

The hyperbolic angle whose hyperbolic sine to calculate.

\subsubsection{\texorpdfstring{\texttt{\ cosh\ }}{ cosh }}\label{functions-cosh}

Calculates the hyperbolic cosine of a hyperbolic angle.

calc { . } { cosh } (

{ \href{/docs/reference/foundations/float/}{float} }

) -\textgreater{} \href{/docs/reference/foundations/float/}{float}

\begin{verbatim}
#calc.cosh(0) \
#calc.cosh(1.5)
\end{verbatim}

\includegraphics[width=5in,height=\textheight,keepaspectratio]{/assets/docs/Vut_ujHW8enJAdOI95v6bgAAAAAAAAAA.png}

\paragraph{\texorpdfstring{\texttt{\ value\ }}{ value }}\label{functions-cosh-value}

\href{/docs/reference/foundations/float/}{float}

{Required} {{ Positional }}

\phantomsection\label{functions-cosh-value-positional-tooltip}
Positional parameters are specified in order, without names.

The hyperbolic angle whose hyperbolic cosine to calculate.

\subsubsection{\texorpdfstring{\texttt{\ tanh\ }}{ tanh }}\label{functions-tanh}

Calculates the hyperbolic tangent of an hyperbolic angle.

calc { . } { tanh } (

{ \href{/docs/reference/foundations/float/}{float} }

) -\textgreater{} \href{/docs/reference/foundations/float/}{float}

\begin{verbatim}
#calc.tanh(0) \
#calc.tanh(1.5)
\end{verbatim}

\includegraphics[width=5in,height=\textheight,keepaspectratio]{/assets/docs/8omHKWMEXh9ltcsWpm4RDQAAAAAAAAAA.png}

\paragraph{\texorpdfstring{\texttt{\ value\ }}{ value }}\label{functions-tanh-value}

\href{/docs/reference/foundations/float/}{float}

{Required} {{ Positional }}

\phantomsection\label{functions-tanh-value-positional-tooltip}
Positional parameters are specified in order, without names.

The hyperbolic angle whose hyperbolic tangent to calculate.

\subsubsection{\texorpdfstring{\texttt{\ log\ }}{ log }}\label{functions-log}

Calculates the logarithm of a number.

If the base is not specified, the logarithm is calculated in base 10.

calc { . } { log } (

{ \href{/docs/reference/foundations/int/}{int}
\href{/docs/reference/foundations/float/}{float} , } {
\hyperref[functions-log-parameters-base]{base :}
\href{/docs/reference/foundations/float/}{float} , }

) -\textgreater{} \href{/docs/reference/foundations/float/}{float}

\begin{verbatim}
#calc.log(100)
\end{verbatim}

\includegraphics[width=5in,height=\textheight,keepaspectratio]{/assets/docs/4te-fP3EFYf9CFfXTNeLbgAAAAAAAAAA.png}

\paragraph{\texorpdfstring{\texttt{\ value\ }}{ value }}\label{functions-log-value}

\href{/docs/reference/foundations/int/}{int} {or}
\href{/docs/reference/foundations/float/}{float}

{Required} {{ Positional }}

\phantomsection\label{functions-log-value-positional-tooltip}
Positional parameters are specified in order, without names.

The number whose logarithm to calculate. Must be strictly positive.

\paragraph{\texorpdfstring{\texttt{\ base\ }}{ base }}\label{functions-log-base}

\href{/docs/reference/foundations/float/}{float}

The base of the logarithm. May not be zero.

Default: \texttt{\ }{\texttt{\ 10.0\ }}\texttt{\ }

\subsubsection{\texorpdfstring{\texttt{\ ln\ }}{ ln }}\label{functions-ln}

Calculates the natural logarithm of a number.

calc { . } { ln } (

{ \href{/docs/reference/foundations/int/}{int}
\href{/docs/reference/foundations/float/}{float} }

) -\textgreater{} \href{/docs/reference/foundations/float/}{float}

\begin{verbatim}
#calc.ln(calc.e)
\end{verbatim}

\includegraphics[width=5in,height=\textheight,keepaspectratio]{/assets/docs/ahMgc30uVaXMdJx4f9b76gAAAAAAAAAA.png}

\paragraph{\texorpdfstring{\texttt{\ value\ }}{ value }}\label{functions-ln-value}

\href{/docs/reference/foundations/int/}{int} {or}
\href{/docs/reference/foundations/float/}{float}

{Required} {{ Positional }}

\phantomsection\label{functions-ln-value-positional-tooltip}
Positional parameters are specified in order, without names.

The number whose logarithm to calculate. Must be strictly positive.

\subsubsection{\texorpdfstring{\texttt{\ fact\ }}{ fact }}\label{functions-fact}

Calculates the factorial of a number.

calc { . } { fact } (

{ \href{/docs/reference/foundations/int/}{int} }

) -\textgreater{} \href{/docs/reference/foundations/int/}{int}

\begin{verbatim}
#calc.fact(5)
\end{verbatim}

\includegraphics[width=5in,height=\textheight,keepaspectratio]{/assets/docs/Hx0vydXttNRUJbbdDSvGlwAAAAAAAAAA.png}

\paragraph{\texorpdfstring{\texttt{\ number\ }}{ number }}\label{functions-fact-number}

\href{/docs/reference/foundations/int/}{int}

{Required} {{ Positional }}

\phantomsection\label{functions-fact-number-positional-tooltip}
Positional parameters are specified in order, without names.

The number whose factorial to calculate. Must be non-negative.

\subsubsection{\texorpdfstring{\texttt{\ perm\ }}{ perm }}\label{functions-perm}

Calculates a permutation.

Returns the \texttt{\ k\ } -permutation of \texttt{\ n\ } , or the
number of ways to choose \texttt{\ k\ } items from a set of
\texttt{\ n\ } with regard to order.

calc { . } { perm } (

{ \href{/docs/reference/foundations/int/}{int} , } {
\href{/docs/reference/foundations/int/}{int} , }

) -\textgreater{} \href{/docs/reference/foundations/int/}{int}

\begin{verbatim}
$ "perm"(n, k) &= n!/((n - k)!) \
  "perm"(5, 3) &= #calc.perm(5, 3) $
\end{verbatim}

\includegraphics[width=5in,height=\textheight,keepaspectratio]{/assets/docs/7mAf4sPmhe6rKKzamBE-iAAAAAAAAAAA.png}

\paragraph{\texorpdfstring{\texttt{\ base\ }}{ base }}\label{functions-perm-base}

\href{/docs/reference/foundations/int/}{int}

{Required} {{ Positional }}

\phantomsection\label{functions-perm-base-positional-tooltip}
Positional parameters are specified in order, without names.

The base number. Must be non-negative.

\paragraph{\texorpdfstring{\texttt{\ numbers\ }}{ numbers }}\label{functions-perm-numbers}

\href{/docs/reference/foundations/int/}{int}

{Required} {{ Positional }}

\phantomsection\label{functions-perm-numbers-positional-tooltip}
Positional parameters are specified in order, without names.

The number of permutations. Must be non-negative.

\subsubsection{\texorpdfstring{\texttt{\ binom\ }}{ binom }}\label{functions-binom}

Calculates a binomial coefficient.

Returns the \texttt{\ k\ } -combination of \texttt{\ n\ } , or the
number of ways to choose \texttt{\ k\ } items from a set of
\texttt{\ n\ } without regard to order.

calc { . } { binom } (

{ \href{/docs/reference/foundations/int/}{int} , } {
\href{/docs/reference/foundations/int/}{int} , }

) -\textgreater{} \href{/docs/reference/foundations/int/}{int}

\begin{verbatim}
#calc.binom(10, 5)
\end{verbatim}

\includegraphics[width=5in,height=\textheight,keepaspectratio]{/assets/docs/3evQc1ME4eQqbzXhrmJ5lAAAAAAAAAAA.png}

\paragraph{\texorpdfstring{\texttt{\ n\ }}{ n }}\label{functions-binom-n}

\href{/docs/reference/foundations/int/}{int}

{Required} {{ Positional }}

\phantomsection\label{functions-binom-n-positional-tooltip}
Positional parameters are specified in order, without names.

The upper coefficient. Must be non-negative.

\paragraph{\texorpdfstring{\texttt{\ k\ }}{ k }}\label{functions-binom-k}

\href{/docs/reference/foundations/int/}{int}

{Required} {{ Positional }}

\phantomsection\label{functions-binom-k-positional-tooltip}
Positional parameters are specified in order, without names.

The lower coefficient. Must be non-negative.

\subsubsection{\texorpdfstring{\texttt{\ gcd\ }}{ gcd }}\label{functions-gcd}

Calculates the greatest common divisor of two integers.

calc { . } { gcd } (

{ \href{/docs/reference/foundations/int/}{int} , } {
\href{/docs/reference/foundations/int/}{int} , }

) -\textgreater{} \href{/docs/reference/foundations/int/}{int}

\begin{verbatim}
#calc.gcd(7, 42)
\end{verbatim}

\includegraphics[width=5in,height=\textheight,keepaspectratio]{/assets/docs/qOIwQyXpCnrSkONAOnIDgAAAAAAAAAAA.png}

\paragraph{\texorpdfstring{\texttt{\ a\ }}{ a }}\label{functions-gcd-a}

\href{/docs/reference/foundations/int/}{int}

{Required} {{ Positional }}

\phantomsection\label{functions-gcd-a-positional-tooltip}
Positional parameters are specified in order, without names.

The first integer.

\paragraph{\texorpdfstring{\texttt{\ b\ }}{ b }}\label{functions-gcd-b}

\href{/docs/reference/foundations/int/}{int}

{Required} {{ Positional }}

\phantomsection\label{functions-gcd-b-positional-tooltip}
Positional parameters are specified in order, without names.

The second integer.

\subsubsection{\texorpdfstring{\texttt{\ lcm\ }}{ lcm }}\label{functions-lcm}

Calculates the least common multiple of two integers.

calc { . } { lcm } (

{ \href{/docs/reference/foundations/int/}{int} , } {
\href{/docs/reference/foundations/int/}{int} , }

) -\textgreater{} \href{/docs/reference/foundations/int/}{int}

\begin{verbatim}
#calc.lcm(96, 13)
\end{verbatim}

\includegraphics[width=5in,height=\textheight,keepaspectratio]{/assets/docs/BsSZQG52_995RG9zgRumuAAAAAAAAAAA.png}

\paragraph{\texorpdfstring{\texttt{\ a\ }}{ a }}\label{functions-lcm-a}

\href{/docs/reference/foundations/int/}{int}

{Required} {{ Positional }}

\phantomsection\label{functions-lcm-a-positional-tooltip}
Positional parameters are specified in order, without names.

The first integer.

\paragraph{\texorpdfstring{\texttt{\ b\ }}{ b }}\label{functions-lcm-b}

\href{/docs/reference/foundations/int/}{int}

{Required} {{ Positional }}

\phantomsection\label{functions-lcm-b-positional-tooltip}
Positional parameters are specified in order, without names.

The second integer.

\subsubsection{\texorpdfstring{\texttt{\ floor\ }}{ floor }}\label{functions-floor}

Rounds a number down to the nearest integer.

If the number is already an integer, it is returned unchanged.

Note that this function will always return an
\href{/docs/reference/foundations/int/}{integer} , and will error if the
resulting \href{/docs/reference/foundations/float/}{\texttt{\ float\ }}
or \href{/docs/reference/foundations/decimal/}{\texttt{\ decimal\ }} is
larger than the maximum 64-bit signed integer or smaller than the
minimum for that type.

calc { . } { floor } (

{ \href{/docs/reference/foundations/int/}{int}
\href{/docs/reference/foundations/float/}{float}
\href{/docs/reference/foundations/decimal/}{decimal} }

) -\textgreater{} \href{/docs/reference/foundations/int/}{int}

\begin{verbatim}
#calc.floor(500.1)
#assert(calc.floor(3) == 3)
#assert(calc.floor(3.14) == 3)
#assert(calc.floor(decimal("-3.14")) == -4)
\end{verbatim}

\includegraphics[width=5in,height=\textheight,keepaspectratio]{/assets/docs/3pMWbIkij09wRgebD43VQgAAAAAAAAAA.png}

\paragraph{\texorpdfstring{\texttt{\ value\ }}{ value }}\label{functions-floor-value}

\href{/docs/reference/foundations/int/}{int} {or}
\href{/docs/reference/foundations/float/}{float} {or}
\href{/docs/reference/foundations/decimal/}{decimal}

{Required} {{ Positional }}

\phantomsection\label{functions-floor-value-positional-tooltip}
Positional parameters are specified in order, without names.

The number to round down.

\subsubsection{\texorpdfstring{\texttt{\ ceil\ }}{ ceil }}\label{functions-ceil}

Rounds a number up to the nearest integer.

If the number is already an integer, it is returned unchanged.

Note that this function will always return an
\href{/docs/reference/foundations/int/}{integer} , and will error if the
resulting \href{/docs/reference/foundations/float/}{\texttt{\ float\ }}
or \href{/docs/reference/foundations/decimal/}{\texttt{\ decimal\ }} is
larger than the maximum 64-bit signed integer or smaller than the
minimum for that type.

calc { . } { ceil } (

{ \href{/docs/reference/foundations/int/}{int}
\href{/docs/reference/foundations/float/}{float}
\href{/docs/reference/foundations/decimal/}{decimal} }

) -\textgreater{} \href{/docs/reference/foundations/int/}{int}

\begin{verbatim}
#calc.ceil(500.1)
#assert(calc.ceil(3) == 3)
#assert(calc.ceil(3.14) == 4)
#assert(calc.ceil(decimal("-3.14")) == -3)
\end{verbatim}

\includegraphics[width=5in,height=\textheight,keepaspectratio]{/assets/docs/XVF6AbxDnXwmraGN-Eh1MgAAAAAAAAAA.png}

\paragraph{\texorpdfstring{\texttt{\ value\ }}{ value }}\label{functions-ceil-value}

\href{/docs/reference/foundations/int/}{int} {or}
\href{/docs/reference/foundations/float/}{float} {or}
\href{/docs/reference/foundations/decimal/}{decimal}

{Required} {{ Positional }}

\phantomsection\label{functions-ceil-value-positional-tooltip}
Positional parameters are specified in order, without names.

The number to round up.

\subsubsection{\texorpdfstring{\texttt{\ trunc\ }}{ trunc }}\label{functions-trunc}

Returns the integer part of a number.

If the number is already an integer, it is returned unchanged.

Note that this function will always return an
\href{/docs/reference/foundations/int/}{integer} , and will error if the
resulting \href{/docs/reference/foundations/float/}{\texttt{\ float\ }}
or \href{/docs/reference/foundations/decimal/}{\texttt{\ decimal\ }} is
larger than the maximum 64-bit signed integer or smaller than the
minimum for that type.

calc { . } { trunc } (

{ \href{/docs/reference/foundations/int/}{int}
\href{/docs/reference/foundations/float/}{float}
\href{/docs/reference/foundations/decimal/}{decimal} }

) -\textgreater{} \href{/docs/reference/foundations/int/}{int}

\begin{verbatim}
#calc.trunc(15.9)
#assert(calc.trunc(3) == 3)
#assert(calc.trunc(-3.7) == -3)
#assert(calc.trunc(decimal("8493.12949582390")) == 8493)
\end{verbatim}

\includegraphics[width=5in,height=\textheight,keepaspectratio]{/assets/docs/0ASdokWmhACxp3cbdBzSiwAAAAAAAAAA.png}

\paragraph{\texorpdfstring{\texttt{\ value\ }}{ value }}\label{functions-trunc-value}

\href{/docs/reference/foundations/int/}{int} {or}
\href{/docs/reference/foundations/float/}{float} {or}
\href{/docs/reference/foundations/decimal/}{decimal}

{Required} {{ Positional }}

\phantomsection\label{functions-trunc-value-positional-tooltip}
Positional parameters are specified in order, without names.

The number to truncate.

\subsubsection{\texorpdfstring{\texttt{\ fract\ }}{ fract }}\label{functions-fract}

Returns the fractional part of a number.

If the number is an integer, returns \texttt{\ 0\ } .

calc { . } { fract } (

{ \href{/docs/reference/foundations/int/}{int}
\href{/docs/reference/foundations/float/}{float}
\href{/docs/reference/foundations/decimal/}{decimal} }

) -\textgreater{} \href{/docs/reference/foundations/int/}{int}
\href{/docs/reference/foundations/float/}{float}
\href{/docs/reference/foundations/decimal/}{decimal}

\begin{verbatim}
#calc.fract(-3.1)
#assert(calc.fract(3) == 0)
#assert(calc.fract(decimal("234.23949211")) == decimal("0.23949211"))
\end{verbatim}

\includegraphics[width=5in,height=\textheight,keepaspectratio]{/assets/docs/3TGIWh2MEGFIDAB8C1nEQAAAAAAAAAAA.png}

\paragraph{\texorpdfstring{\texttt{\ value\ }}{ value }}\label{functions-fract-value}

\href{/docs/reference/foundations/int/}{int} {or}
\href{/docs/reference/foundations/float/}{float} {or}
\href{/docs/reference/foundations/decimal/}{decimal}

{Required} {{ Positional }}

\phantomsection\label{functions-fract-value-positional-tooltip}
Positional parameters are specified in order, without names.

The number to truncate.

\subsubsection{\texorpdfstring{\texttt{\ round\ }}{ round }}\label{functions-round}

Rounds a number to the nearest integer away from zero.

Optionally, a number of decimal places can be specified.

If the number of digits is negative, its absolute value will indicate
the amount of significant integer digits to remove before the decimal
point.

Note that this function will return the same type as the operand. That
is, applying \texttt{\ round\ } to a
\href{/docs/reference/foundations/float/}{\texttt{\ float\ }} will
return a \texttt{\ float\ } , and to a
\href{/docs/reference/foundations/decimal/}{\texttt{\ decimal\ }} ,
another \texttt{\ decimal\ } . You may explicitly convert the output of
this function to an integer with
\href{/docs/reference/foundations/int/}{\texttt{\ int\ }} , but note
that such a conversion will error if the \texttt{\ float\ } or
\texttt{\ decimal\ } is larger than the maximum 64-bit signed integer or
smaller than the minimum integer.

In addition, this function can error if there is an attempt to round
beyond the maximum or minimum integer or \texttt{\ decimal\ } . If the
number is a \texttt{\ float\ } , such an attempt will cause
\texttt{\ float\ }{\texttt{\ .\ }}\texttt{\ inf\ } or
\texttt{\ }{\texttt{\ -\ }}\texttt{\ float\ }{\texttt{\ .\ }}\texttt{\ inf\ }
to be returned for maximum and minimum respectively.

calc { . } { round } (

{ \href{/docs/reference/foundations/int/}{int}
\href{/docs/reference/foundations/float/}{float}
\href{/docs/reference/foundations/decimal/}{decimal} , } {
\hyperref[functions-round-parameters-digits]{digits :}
\href{/docs/reference/foundations/int/}{int} , }

) -\textgreater{} \href{/docs/reference/foundations/int/}{int}
\href{/docs/reference/foundations/float/}{float}
\href{/docs/reference/foundations/decimal/}{decimal}

\begin{verbatim}
#calc.round(3.1415, digits: 2)
#assert(calc.round(3) == 3)
#assert(calc.round(3.14) == 3)
#assert(calc.round(3.5) == 4.0)
#assert(calc.round(3333.45, digits: -2) == 3300.0)
#assert(calc.round(-48953.45, digits: -3) == -49000.0)
#assert(calc.round(3333, digits: -2) == 3300)
#assert(calc.round(-48953, digits: -3) == -49000)
#assert(calc.round(decimal("-6.5")) == decimal("-7"))
#assert(calc.round(decimal("7.123456789"), digits: 6) == decimal("7.123457"))
#assert(calc.round(decimal("3333.45"), digits: -2) == decimal("3300"))
#assert(calc.round(decimal("-48953.45"), digits: -3) == decimal("-49000"))
\end{verbatim}

\includegraphics[width=5in,height=\textheight,keepaspectratio]{/assets/docs/S2fXMNcPylTq6uwl7ZRpoAAAAAAAAAAA.png}

\paragraph{\texorpdfstring{\texttt{\ value\ }}{ value }}\label{functions-round-value}

\href{/docs/reference/foundations/int/}{int} {or}
\href{/docs/reference/foundations/float/}{float} {or}
\href{/docs/reference/foundations/decimal/}{decimal}

{Required} {{ Positional }}

\phantomsection\label{functions-round-value-positional-tooltip}
Positional parameters are specified in order, without names.

The number to round.

\paragraph{\texorpdfstring{\texttt{\ digits\ }}{ digits }}\label{functions-round-digits}

\href{/docs/reference/foundations/int/}{int}

If positive, the number of decimal places.

If negative, the number of significant integer digits that should be
removed before the decimal point.

Default: \texttt{\ }{\texttt{\ 0\ }}\texttt{\ }

\subsubsection{\texorpdfstring{\texttt{\ clamp\ }}{ clamp }}\label{functions-clamp}

Clamps a number between a minimum and maximum value.

calc { . } { clamp } (

{ \href{/docs/reference/foundations/int/}{int}
\href{/docs/reference/foundations/float/}{float}
\href{/docs/reference/foundations/decimal/}{decimal} , } {
\href{/docs/reference/foundations/int/}{int}
\href{/docs/reference/foundations/float/}{float}
\href{/docs/reference/foundations/decimal/}{decimal} , } {
\href{/docs/reference/foundations/int/}{int}
\href{/docs/reference/foundations/float/}{float}
\href{/docs/reference/foundations/decimal/}{decimal} , }

) -\textgreater{} \href{/docs/reference/foundations/int/}{int}
\href{/docs/reference/foundations/float/}{float}
\href{/docs/reference/foundations/decimal/}{decimal}

\begin{verbatim}
#calc.clamp(5, 0, 4)
#assert(calc.clamp(5, 0, 10) == 5)
#assert(calc.clamp(5, 6, 10) == 6)
#assert(calc.clamp(decimal("5.45"), 2, decimal("45.9")) == decimal("5.45"))
#assert(calc.clamp(decimal("5.45"), decimal("6.75"), 12) == decimal("6.75"))
\end{verbatim}

\includegraphics[width=5in,height=\textheight,keepaspectratio]{/assets/docs/IT7doIU2fH1UJf0E_SPc6QAAAAAAAAAA.png}

\paragraph{\texorpdfstring{\texttt{\ value\ }}{ value }}\label{functions-clamp-value}

\href{/docs/reference/foundations/int/}{int} {or}
\href{/docs/reference/foundations/float/}{float} {or}
\href{/docs/reference/foundations/decimal/}{decimal}

{Required} {{ Positional }}

\phantomsection\label{functions-clamp-value-positional-tooltip}
Positional parameters are specified in order, without names.

The number to clamp.

\paragraph{\texorpdfstring{\texttt{\ min\ }}{ min }}\label{functions-clamp-min}

\href{/docs/reference/foundations/int/}{int} {or}
\href{/docs/reference/foundations/float/}{float} {or}
\href{/docs/reference/foundations/decimal/}{decimal}

{Required} {{ Positional }}

\phantomsection\label{functions-clamp-min-positional-tooltip}
Positional parameters are specified in order, without names.

The inclusive minimum value.

\paragraph{\texorpdfstring{\texttt{\ max\ }}{ max }}\label{functions-clamp-max}

\href{/docs/reference/foundations/int/}{int} {or}
\href{/docs/reference/foundations/float/}{float} {or}
\href{/docs/reference/foundations/decimal/}{decimal}

{Required} {{ Positional }}

\phantomsection\label{functions-clamp-max-positional-tooltip}
Positional parameters are specified in order, without names.

The inclusive maximum value.

\subsubsection{\texorpdfstring{\texttt{\ min\ }}{ min }}\label{functions-min}

Determines the minimum of a sequence of values.

calc { . } { min } (

{ \hyperref[functions-min-parameters-values]{..} { any } }

) -\textgreater{} { any }

\begin{verbatim}
#calc.min(1, -3, -5, 20, 3, 6) \
#calc.min("typst", "is", "cool")
\end{verbatim}

\includegraphics[width=5in,height=\textheight,keepaspectratio]{/assets/docs/afOSrjdOAc_1RzzU2hxUIgAAAAAAAAAA.png}

\paragraph{\texorpdfstring{\texttt{\ values\ }}{ values }}\label{functions-min-values}

{ any }

{Required} {{ Positional }}

\phantomsection\label{functions-min-values-positional-tooltip}
Positional parameters are specified in order, without names.

{{ Variadic }}

\phantomsection\label{functions-min-values-variadic-tooltip}
Variadic parameters can be specified multiple times.

The sequence of values from which to extract the minimum. Must not be
empty.

\subsubsection{\texorpdfstring{\texttt{\ max\ }}{ max }}\label{functions-max}

Determines the maximum of a sequence of values.

calc { . } { max } (

{ \hyperref[functions-max-parameters-values]{..} { any } }

) -\textgreater{} { any }

\begin{verbatim}
#calc.max(1, -3, -5, 20, 3, 6) \
#calc.max("typst", "is", "cool")
\end{verbatim}

\includegraphics[width=5in,height=\textheight,keepaspectratio]{/assets/docs/B8vbsVaOK7Ilt-aRhfDiFwAAAAAAAAAA.png}

\paragraph{\texorpdfstring{\texttt{\ values\ }}{ values }}\label{functions-max-values}

{ any }

{Required} {{ Positional }}

\phantomsection\label{functions-max-values-positional-tooltip}
Positional parameters are specified in order, without names.

{{ Variadic }}

\phantomsection\label{functions-max-values-variadic-tooltip}
Variadic parameters can be specified multiple times.

The sequence of values from which to extract the maximum. Must not be
empty.

\subsubsection{\texorpdfstring{\texttt{\ even\ }}{ even }}\label{functions-even}

Determines whether an integer is even.

calc { . } { even } (

{ \href{/docs/reference/foundations/int/}{int} }

) -\textgreater{} \href{/docs/reference/foundations/bool/}{bool}

\begin{verbatim}
#calc.even(4) \
#calc.even(5) \
#range(10).filter(calc.even)
\end{verbatim}

\includegraphics[width=5in,height=\textheight,keepaspectratio]{/assets/docs/YVF-q96_WeIoAbwweursAAAAAAAAAAAA.png}

\paragraph{\texorpdfstring{\texttt{\ value\ }}{ value }}\label{functions-even-value}

\href{/docs/reference/foundations/int/}{int}

{Required} {{ Positional }}

\phantomsection\label{functions-even-value-positional-tooltip}
Positional parameters are specified in order, without names.

The number to check for evenness.

\subsubsection{\texorpdfstring{\texttt{\ odd\ }}{ odd }}\label{functions-odd}

Determines whether an integer is odd.

calc { . } { odd } (

{ \href{/docs/reference/foundations/int/}{int} }

) -\textgreater{} \href{/docs/reference/foundations/bool/}{bool}

\begin{verbatim}
#calc.odd(4) \
#calc.odd(5) \
#range(10).filter(calc.odd)
\end{verbatim}

\includegraphics[width=5in,height=\textheight,keepaspectratio]{/assets/docs/54xiVFQnQ9FIdgInF0A_jAAAAAAAAAAA.png}

\paragraph{\texorpdfstring{\texttt{\ value\ }}{ value }}\label{functions-odd-value}

\href{/docs/reference/foundations/int/}{int}

{Required} {{ Positional }}

\phantomsection\label{functions-odd-value-positional-tooltip}
Positional parameters are specified in order, without names.

The number to check for oddness.

\subsubsection{\texorpdfstring{\texttt{\ rem\ }}{ rem }}\label{functions-rem}

Calculates the remainder of two numbers.

The value \texttt{\ calc.rem(x,\ y)\ } always has the same sign as
\texttt{\ x\ } , and is smaller in magnitude than \texttt{\ y\ } .

This can error if given a
\href{/docs/reference/foundations/decimal/}{\texttt{\ decimal\ }} input
and the dividend is too small in magnitude compared to the divisor.

calc { . } { rem } (

{ \href{/docs/reference/foundations/int/}{int}
\href{/docs/reference/foundations/float/}{float}
\href{/docs/reference/foundations/decimal/}{decimal} , } {
\href{/docs/reference/foundations/int/}{int}
\href{/docs/reference/foundations/float/}{float}
\href{/docs/reference/foundations/decimal/}{decimal} , }

) -\textgreater{} \href{/docs/reference/foundations/int/}{int}
\href{/docs/reference/foundations/float/}{float}
\href{/docs/reference/foundations/decimal/}{decimal}

\begin{verbatim}
#calc.rem(7, 3) \
#calc.rem(7, -3) \
#calc.rem(-7, 3) \
#calc.rem(-7, -3) \
#calc.rem(1.75, 0.5)
\end{verbatim}

\includegraphics[width=5in,height=\textheight,keepaspectratio]{/assets/docs/h9kAd8BZ_4qaZUm7WWIpgQAAAAAAAAAA.png}

\paragraph{\texorpdfstring{\texttt{\ dividend\ }}{ dividend }}\label{functions-rem-dividend}

\href{/docs/reference/foundations/int/}{int} {or}
\href{/docs/reference/foundations/float/}{float} {or}
\href{/docs/reference/foundations/decimal/}{decimal}

{Required} {{ Positional }}

\phantomsection\label{functions-rem-dividend-positional-tooltip}
Positional parameters are specified in order, without names.

The dividend of the remainder.

\paragraph{\texorpdfstring{\texttt{\ divisor\ }}{ divisor }}\label{functions-rem-divisor}

\href{/docs/reference/foundations/int/}{int} {or}
\href{/docs/reference/foundations/float/}{float} {or}
\href{/docs/reference/foundations/decimal/}{decimal}

{Required} {{ Positional }}

\phantomsection\label{functions-rem-divisor-positional-tooltip}
Positional parameters are specified in order, without names.

The divisor of the remainder.

\subsubsection{\texorpdfstring{\texttt{\ div-euclid\ }}{ div-euclid }}\label{functions-div-euclid}

Performs euclidean division of two numbers.

The result of this computation is that of a division rounded to the
integer \texttt{\ n\ } such that the dividend is greater than or equal
to \texttt{\ n\ } times the divisor.

calc { . } { div-euclid } (

{ \href{/docs/reference/foundations/int/}{int}
\href{/docs/reference/foundations/float/}{float}
\href{/docs/reference/foundations/decimal/}{decimal} , } {
\href{/docs/reference/foundations/int/}{int}
\href{/docs/reference/foundations/float/}{float}
\href{/docs/reference/foundations/decimal/}{decimal} , }

) -\textgreater{} \href{/docs/reference/foundations/int/}{int}
\href{/docs/reference/foundations/float/}{float}
\href{/docs/reference/foundations/decimal/}{decimal}

\begin{verbatim}
#calc.div-euclid(7, 3) \
#calc.div-euclid(7, -3) \
#calc.div-euclid(-7, 3) \
#calc.div-euclid(-7, -3) \
#calc.div-euclid(1.75, 0.5) \
#calc.div-euclid(decimal("1.75"), decimal("0.5"))
\end{verbatim}

\includegraphics[width=5in,height=\textheight,keepaspectratio]{/assets/docs/496IGVvoarlmERajiTFs_gAAAAAAAAAA.png}

\paragraph{\texorpdfstring{\texttt{\ dividend\ }}{ dividend }}\label{functions-div-euclid-dividend}

\href{/docs/reference/foundations/int/}{int} {or}
\href{/docs/reference/foundations/float/}{float} {or}
\href{/docs/reference/foundations/decimal/}{decimal}

{Required} {{ Positional }}

\phantomsection\label{functions-div-euclid-dividend-positional-tooltip}
Positional parameters are specified in order, without names.

The dividend of the division.

\paragraph{\texorpdfstring{\texttt{\ divisor\ }}{ divisor }}\label{functions-div-euclid-divisor}

\href{/docs/reference/foundations/int/}{int} {or}
\href{/docs/reference/foundations/float/}{float} {or}
\href{/docs/reference/foundations/decimal/}{decimal}

{Required} {{ Positional }}

\phantomsection\label{functions-div-euclid-divisor-positional-tooltip}
Positional parameters are specified in order, without names.

The divisor of the division.

\subsubsection{\texorpdfstring{\texttt{\ rem-euclid\ }}{ rem-euclid }}\label{functions-rem-euclid}

This calculates the least nonnegative remainder of a division.

Warning: Due to a floating point round-off error, the remainder may
equal the absolute value of the divisor if the dividend is much smaller
in magnitude than the divisor and the dividend is negative. This only
applies for floating point inputs.

In addition, this can error if given a
\href{/docs/reference/foundations/decimal/}{\texttt{\ decimal\ }} input
and the dividend is too small in magnitude compared to the divisor.

calc { . } { rem-euclid } (

{ \href{/docs/reference/foundations/int/}{int}
\href{/docs/reference/foundations/float/}{float}
\href{/docs/reference/foundations/decimal/}{decimal} , } {
\href{/docs/reference/foundations/int/}{int}
\href{/docs/reference/foundations/float/}{float}
\href{/docs/reference/foundations/decimal/}{decimal} , }

) -\textgreater{} \href{/docs/reference/foundations/int/}{int}
\href{/docs/reference/foundations/float/}{float}
\href{/docs/reference/foundations/decimal/}{decimal}

\begin{verbatim}
#calc.rem-euclid(7, 3) \
#calc.rem-euclid(7, -3) \
#calc.rem-euclid(-7, 3) \
#calc.rem-euclid(-7, -3) \
#calc.rem-euclid(1.75, 0.5) \
#calc.rem-euclid(decimal("1.75"), decimal("0.5"))
\end{verbatim}

\includegraphics[width=5in,height=\textheight,keepaspectratio]{/assets/docs/ysX2HLC-rfWACinwigYcWgAAAAAAAAAA.png}

\paragraph{\texorpdfstring{\texttt{\ dividend\ }}{ dividend }}\label{functions-rem-euclid-dividend}

\href{/docs/reference/foundations/int/}{int} {or}
\href{/docs/reference/foundations/float/}{float} {or}
\href{/docs/reference/foundations/decimal/}{decimal}

{Required} {{ Positional }}

\phantomsection\label{functions-rem-euclid-dividend-positional-tooltip}
Positional parameters are specified in order, without names.

The dividend of the remainder.

\paragraph{\texorpdfstring{\texttt{\ divisor\ }}{ divisor }}\label{functions-rem-euclid-divisor}

\href{/docs/reference/foundations/int/}{int} {or}
\href{/docs/reference/foundations/float/}{float} {or}
\href{/docs/reference/foundations/decimal/}{decimal}

{Required} {{ Positional }}

\phantomsection\label{functions-rem-euclid-divisor-positional-tooltip}
Positional parameters are specified in order, without names.

The divisor of the remainder.

\subsubsection{\texorpdfstring{\texttt{\ quo\ }}{ quo }}\label{functions-quo}

Calculates the quotient (floored division) of two numbers.

Note that this function will always return an
\href{/docs/reference/foundations/int/}{integer} , and will error if the
resulting \href{/docs/reference/foundations/float/}{\texttt{\ float\ }}
or \href{/docs/reference/foundations/decimal/}{\texttt{\ decimal\ }} is
larger than the maximum 64-bit signed integer or smaller than the
minimum for that type.

calc { . } { quo } (

{ \href{/docs/reference/foundations/int/}{int}
\href{/docs/reference/foundations/float/}{float}
\href{/docs/reference/foundations/decimal/}{decimal} , } {
\href{/docs/reference/foundations/int/}{int}
\href{/docs/reference/foundations/float/}{float}
\href{/docs/reference/foundations/decimal/}{decimal} , }

) -\textgreater{} \href{/docs/reference/foundations/int/}{int}

\begin{verbatim}
$ "quo"(a, b) &= floor(a/b) \
  "quo"(14, 5) &= #calc.quo(14, 5) \
  "quo"(3.46, 0.5) &= #calc.quo(3.46, 0.5) $
\end{verbatim}

\includegraphics[width=5in,height=\textheight,keepaspectratio]{/assets/docs/AEhIvOjgCcBZo0GMCLQ9tQAAAAAAAAAA.png}

\paragraph{\texorpdfstring{\texttt{\ dividend\ }}{ dividend }}\label{functions-quo-dividend}

\href{/docs/reference/foundations/int/}{int} {or}
\href{/docs/reference/foundations/float/}{float} {or}
\href{/docs/reference/foundations/decimal/}{decimal}

{Required} {{ Positional }}

\phantomsection\label{functions-quo-dividend-positional-tooltip}
Positional parameters are specified in order, without names.

The dividend of the quotient.

\paragraph{\texorpdfstring{\texttt{\ divisor\ }}{ divisor }}\label{functions-quo-divisor}

\href{/docs/reference/foundations/int/}{int} {or}
\href{/docs/reference/foundations/float/}{float} {or}
\href{/docs/reference/foundations/decimal/}{decimal}

{Required} {{ Positional }}

\phantomsection\label{functions-quo-divisor-positional-tooltip}
Positional parameters are specified in order, without names.

The divisor of the quotient.

\href{/docs/reference/foundations/bytes/}{\pandocbounded{\includesvg[keepaspectratio]{/assets/icons/16-arrow-right.svg}}}

{ Bytes } { Previous page }

\href{/docs/reference/foundations/content/}{\pandocbounded{\includesvg[keepaspectratio]{/assets/icons/16-arrow-right.svg}}}

{ Content } { Next page }


\section{Docs LaTeX/typst.app/docs/reference/foundations/function.tex}
\title{typst.app/docs/reference/foundations/function}

\begin{itemize}
\tightlist
\item
  \href{/docs}{\includesvg[width=0.16667in,height=0.16667in]{/assets/icons/16-docs-dark.svg}}
\item
  \includesvg[width=0.16667in,height=0.16667in]{/assets/icons/16-arrow-right.svg}
\item
  \href{/docs/reference/}{Reference}
\item
  \includesvg[width=0.16667in,height=0.16667in]{/assets/icons/16-arrow-right.svg}
\item
  \href{/docs/reference/foundations/}{Foundations}
\item
  \includesvg[width=0.16667in,height=0.16667in]{/assets/icons/16-arrow-right.svg}
\item
  \href{/docs/reference/foundations/function/}{Function}
\end{itemize}

\section{\texorpdfstring{{ function }}{ function }}\label{summary}

A mapping from argument values to a return value.

You can call a function by writing a comma-separated list of function
\emph{arguments} enclosed in parentheses directly after the function
name. Additionally, you can pass any number of trailing content blocks
arguments to a function \emph{after} the normal argument list. If the
normal argument list would become empty, it can be omitted. Typst
supports positional and named arguments. The former are identified by
position and type, while the latter are written as
\texttt{\ name:\ value\ } .

Within math mode, function calls have special behaviour. See the
\href{/docs/reference/math/}{math documentation} for more details.

\subsection{Example}\label{example}

\begin{verbatim}
// Call a function.
#list([A], [B])

// Named arguments and trailing
// content blocks.
#enum(start: 2)[A][B]

// Version without parentheses.
#list[A][B]
\end{verbatim}

\includegraphics[width=5in,height=\textheight,keepaspectratio]{/assets/docs/h8ulslRsTYE05Pu4qy5C6AAAAAAAAAAA.png}

Functions are a fundamental building block of Typst. Typst provides
functions for a variety of typesetting tasks. Moreover, the markup you
write is backed by functions and all styling happens through functions.
This reference lists all available functions and how you can use them.
Please also refer to the documentation about
\href{/docs/reference/styling/\#set-rules}{set} and
\href{/docs/reference/styling/\#show-rules}{show} rules to learn about
additional ways you can work with functions in Typst.

\subsection{Element functions}\label{element-functions}

Some functions are associated with \emph{elements} like
\href{/docs/reference/model/heading/}{headings} or
\href{/docs/reference/model/table/}{tables} . When called, these create
an element of their respective kind. In contrast to normal functions,
they can further be used in
\href{/docs/reference/styling/\#set-rules}{set rules} ,
\href{/docs/reference/styling/\#show-rules}{show rules} , and
\href{/docs/reference/foundations/selector/}{selectors} .

\subsection{Function scopes}\label{function-scopes}

Functions can hold related definitions in their own scope, similar to a
\href{/docs/reference/scripting/\#modules}{module} . Examples of this
are
\href{/docs/reference/foundations/assert/\#definitions-eq}{\texttt{\ assert.eq\ }}
or
\href{/docs/reference/model/list/\#definitions-item}{\texttt{\ list.item\ }}
. However, this feature is currently only available for built-in
functions.

\subsection{Defining functions}\label{defining-functions}

You can define your own function with a
\href{/docs/reference/scripting/\#bindings}{let binding} that has a
parameter list after the binding\textquotesingle s name. The parameter
list can contain mandatory positional parameters, named parameters with
default values and
\href{/docs/reference/foundations/arguments/}{argument sinks} .

The right-hand side of a function binding is the function body, which
can be a block or any other expression. It defines the
function\textquotesingle s return value and can depend on the
parameters. If the function body is a
\href{/docs/reference/scripting/\#blocks}{code block} , the return value
is the result of joining the values of each expression in the block.

Within a function body, the \texttt{\ return\ } keyword can be used to
exit early and optionally specify a return value. If no explicit return
value is given, the body evaluates to the result of joining all
expressions preceding the \texttt{\ return\ } .

Functions that don\textquotesingle t return any meaningful value return
\href{/docs/reference/foundations/none/}{\texttt{\ none\ }} instead. The
return type of such functions is not explicitly specified in the
documentation. (An example of this is
\href{/docs/reference/foundations/array/\#definitions-push}{\texttt{\ array.push\ }}
).

\begin{verbatim}
#let alert(body, fill: red) = {
  set text(white)
  set align(center)
  rect(
    fill: fill,
    inset: 8pt,
    radius: 4pt,
    [*Warning:\ #body*],
  )
}

#alert[
  Danger is imminent!
]

#alert(fill: blue)[
  KEEP OFF TRACKS
]
\end{verbatim}

\includegraphics[width=5in,height=\textheight,keepaspectratio]{/assets/docs/56wK4TQtzRt7_B3OQOxb7QAAAAAAAAAA.png}

\subsection{Importing functions}\label{importing-functions}

Functions can be imported from one file (
\href{/docs/reference/scripting/\#modules}{\texttt{\ module\ }} ) into
another using \texttt{\ }{\texttt{\ import\ }}\texttt{\ } . For example,
assume that we have defined the \texttt{\ alert\ } function from the
previous example in a file called \texttt{\ foo.typ\ } . We can import
it into another file by writing
\texttt{\ }{\texttt{\ import\ }}\texttt{\ }{\texttt{\ "foo.typ"\ }}\texttt{\ }{\texttt{\ :\ }}\texttt{\ alert\ }
.

\subsection{Unnamed functions}\label{unnamed}

You can also created an unnamed function without creating a binding by
specifying a parameter list followed by \texttt{\ =\textgreater{}\ } and
the function body. If your function has just one parameter, the
parentheses around the parameter list are optional. Unnamed functions
are mainly useful for show rules, but also for settable properties that
take functions like the page function\textquotesingle s
\href{/docs/reference/layout/page/\#parameters-footer}{\texttt{\ footer\ }}
property.

\begin{verbatim}
#show "once?": it => [#it #it]
once?
\end{verbatim}

\includegraphics[width=5in,height=\textheight,keepaspectratio]{/assets/docs/yXee-w_McX_Uho7Ghovc-QAAAAAAAAAA.png}

\subsection{Note on function purity}\label{note-on-function-purity}

In Typst, all functions are \emph{pure.} This means that for the same
arguments, they always return the same result. They cannot "remember"
things to produce another value when they are called a second time.

The only exception are built-in methods like
\href{/docs/reference/foundations/array/\#definitions-push}{\texttt{\ array.push(value)\ }}
. These can modify the values they are called on.

\subsection{\texorpdfstring{{ Definitions
}}{ Definitions }}\label{definitions}

\phantomsection\label{definitions-tooltip}
Functions and types and can have associated definitions. These are
accessed by specifying the function or type, followed by a period, and
then the definition\textquotesingle s name.

\subsubsection{\texorpdfstring{\texttt{\ with\ }}{ with }}\label{definitions-with}

Returns a new function that has the given arguments pre-applied.

self { . } { with } (

{ \hyperref[definitions-with-parameters-arguments]{..} { any } }

) -\textgreater{} \href{/docs/reference/foundations/function/}{function}

\paragraph{\texorpdfstring{\texttt{\ arguments\ }}{ arguments }}\label{definitions-with-arguments}

{ any }

{Required} {{ Positional }}

\phantomsection\label{definitions-with-arguments-positional-tooltip}
Positional parameters are specified in order, without names.

{{ Variadic }}

\phantomsection\label{definitions-with-arguments-variadic-tooltip}
Variadic parameters can be specified multiple times.

The arguments to apply to the function.

\subsubsection{\texorpdfstring{\texttt{\ where\ }}{ where }}\label{definitions-where}

Returns a selector that filters for elements belonging to this function
whose fields have the values of the given arguments.

self { . } { where } (

{ \hyperref[definitions-where-parameters-fields]{..} { any } }

) -\textgreater{} \href{/docs/reference/foundations/selector/}{selector}

\begin{verbatim}
#show heading.where(level: 2): set text(blue)
= Section
== Subsection
=== Sub-subsection
\end{verbatim}

\includegraphics[width=5in,height=\textheight,keepaspectratio]{/assets/docs/VOR4DpWIitR8ukDkDB2RigAAAAAAAAAA.png}

\paragraph{\texorpdfstring{\texttt{\ fields\ }}{ fields }}\label{definitions-where-fields}

{ any }

{Required} {{ Positional }}

\phantomsection\label{definitions-where-fields-positional-tooltip}
Positional parameters are specified in order, without names.

{{ Variadic }}

\phantomsection\label{definitions-where-fields-variadic-tooltip}
Variadic parameters can be specified multiple times.

The fields to filter for.

\href{/docs/reference/foundations/float/}{\pandocbounded{\includesvg[keepaspectratio]{/assets/icons/16-arrow-right.svg}}}

{ Float } { Previous page }

\href{/docs/reference/foundations/int/}{\pandocbounded{\includesvg[keepaspectratio]{/assets/icons/16-arrow-right.svg}}}

{ Integer } { Next page }


\section{Docs LaTeX/typst.app/docs/reference/foundations/regex.tex}
\title{typst.app/docs/reference/foundations/regex}

\begin{itemize}
\tightlist
\item
  \href{/docs}{\includesvg[width=0.16667in,height=0.16667in]{/assets/icons/16-docs-dark.svg}}
\item
  \includesvg[width=0.16667in,height=0.16667in]{/assets/icons/16-arrow-right.svg}
\item
  \href{/docs/reference/}{Reference}
\item
  \includesvg[width=0.16667in,height=0.16667in]{/assets/icons/16-arrow-right.svg}
\item
  \href{/docs/reference/foundations/}{Foundations}
\item
  \includesvg[width=0.16667in,height=0.16667in]{/assets/icons/16-arrow-right.svg}
\item
  \href{/docs/reference/foundations/regex/}{Regex}
\end{itemize}

\section{\texorpdfstring{{ regex }}{ regex }}\label{summary}

A regular expression.

Can be used as a \href{/docs/reference/styling/\#show-rules}{show rule
selector} and with \href{/docs/reference/foundations/str/}{string
methods} like \texttt{\ find\ } , \texttt{\ split\ } , and
\texttt{\ replace\ } .

\href{https://docs.rs/regex/latest/regex/\#syntax}{See here} for a
specification of the supported syntax.

\subsection{Example}\label{example}

\begin{verbatim}
// Works with string methods.
#"a,b;c".split(regex("[,;]"))

// Works with show rules.
#show regex("\d+"): set text(red)

The numbers 1 to 10.
\end{verbatim}

\includegraphics[width=5in,height=\textheight,keepaspectratio]{/assets/docs/UtfXJAklKdjyBZ3HmRwY-AAAAAAAAAAA.png}

\subsection{\texorpdfstring{Constructor
{}}{Constructor }}\label{constructor}

\phantomsection\label{constructor-constructor-tooltip}
If a type has a constructor, you can call it like a function to create a
new value of the type.

Create a regular expression from a string.

{ regex } (

{ \href{/docs/reference/foundations/str/}{str} }

) -\textgreater{} \href{/docs/reference/foundations/regex/}{regex}

\paragraph{\texorpdfstring{\texttt{\ regex\ }}{ regex }}\label{constructor-regex}

\href{/docs/reference/foundations/str/}{str}

{Required} {{ Positional }}

\phantomsection\label{constructor-regex-positional-tooltip}
Positional parameters are specified in order, without names.

The regular expression as a string.

Most regex escape sequences just work because they are not valid Typst
escape sequences. To produce regex escape sequences that are also valid
in Typst (e.g.
\texttt{\ }{\texttt{\ \textbackslash{}\textbackslash{}\ }}\texttt{\ } ),
you need to escape twice. Thus, to match a verbatim backslash, you would
need to write
\texttt{\ }{\texttt{\ regex\ }}\texttt{\ }{\texttt{\ (\ }}\texttt{\ }{\texttt{\ "\textbackslash{}\textbackslash{}\textbackslash{}\textbackslash{}"\ }}\texttt{\ }{\texttt{\ )\ }}\texttt{\ }
.

If you need many escape sequences, you can also create a raw element and
extract its text to use it for your regular expressions:

\includesvg[width=0.16667in,height=0.16667in]{/assets/icons/16-arrow-right.svg}
View example

\texttt{\ }{\texttt{\ regex\ }}\texttt{\ }{\texttt{\ (\ }}\texttt{\ }{\texttt{\ \textasciigrave{}\textbackslash{}d+\textbackslash{}.\textbackslash{}d+\textbackslash{}.\textbackslash{}d+\textasciigrave{}\ }}\texttt{\ }{\texttt{\ .\ }}\texttt{\ text\ }{\texttt{\ )\ }}\texttt{\ }
.

\href{/docs/reference/foundations/plugin/}{\pandocbounded{\includesvg[keepaspectratio]{/assets/icons/16-arrow-right.svg}}}

{ Plugin } { Previous page }

\href{/docs/reference/foundations/repr/}{\pandocbounded{\includesvg[keepaspectratio]{/assets/icons/16-arrow-right.svg}}}

{ Representation } { Next page }


\section{Docs LaTeX/typst.app/docs/reference/foundations/type.tex}
\title{typst.app/docs/reference/foundations/type}

\begin{itemize}
\tightlist
\item
  \href{/docs}{\includesvg[width=0.16667in,height=0.16667in]{/assets/icons/16-docs-dark.svg}}
\item
  \includesvg[width=0.16667in,height=0.16667in]{/assets/icons/16-arrow-right.svg}
\item
  \href{/docs/reference/}{Reference}
\item
  \includesvg[width=0.16667in,height=0.16667in]{/assets/icons/16-arrow-right.svg}
\item
  \href{/docs/reference/foundations/}{Foundations}
\item
  \includesvg[width=0.16667in,height=0.16667in]{/assets/icons/16-arrow-right.svg}
\item
  \href{/docs/reference/foundations/type/}{Type}
\end{itemize}

\section{\texorpdfstring{{ type }}{ type }}\label{summary}

Describes a kind of value.

To style your document, you need to work with values of different kinds:
Lengths specifying the size of your elements, colors for your text and
shapes, and more. Typst categorizes these into clearly defined
\emph{types} and tells you where it expects which type of value.

Apart from basic types for numeric values and
\href{/docs/reference/foundations/int/}{typical}
\href{/docs/reference/foundations/float/}{types}
\href{/docs/reference/foundations/str/}{known}
\href{/docs/reference/foundations/array/}{from}
\href{/docs/reference/foundations/dictionary/}{programming} languages,
Typst provides a special type for
\href{/docs/reference/foundations/content/}{\emph{content.}} A value of
this type can hold anything that you can enter into your document: Text,
elements like headings and shapes, and style information.

\subsection{Example}\label{example}

\begin{verbatim}
#let x = 10
#if type(x) == int [
  #x is an integer!
] else [
  #x is another value...
]

An image is of type
#type(image("glacier.jpg")).
\end{verbatim}

\includegraphics[width=5in,height=\textheight,keepaspectratio]{/assets/docs/dTjHaEMO5150e0-XVg1OzwAAAAAAAAAA.png}

The type of \texttt{\ 10\ } is \texttt{\ int\ } . Now, what is the type
of \texttt{\ int\ } or even \texttt{\ type\ } ?

\begin{verbatim}
#type(int) \
#type(type)
\end{verbatim}

\includegraphics[width=5in,height=\textheight,keepaspectratio]{/assets/docs/HqIgZy_wqBbnboRlZ-Iv4AAAAAAAAAAA.png}

\subsection{Compatibility}\label{compatibility}

In Typst 0.7 and lower, the \texttt{\ type\ } function returned a string
instead of a type. Compatibility with the old way will remain for a
while to give package authors time to upgrade, but it will be removed at
some point.

\begin{itemize}
\tightlist
\item
  Checks like
  \texttt{\ int\ }{\texttt{\ ==\ }}\texttt{\ }{\texttt{\ "integer"\ }}\texttt{\ }
  evaluate to \texttt{\ }{\texttt{\ true\ }}\texttt{\ }
\item
  Adding/joining a type and string will yield a string
\item
  The \texttt{\ in\ } operator on a type and a dictionary will evaluate
  to \texttt{\ }{\texttt{\ true\ }}\texttt{\ } if the dictionary has a
  string key matching the type\textquotesingle s name
\end{itemize}

\subsection{\texorpdfstring{Constructor
{}}{Constructor }}\label{constructor}

\phantomsection\label{constructor-constructor-tooltip}
If a type has a constructor, you can call it like a function to create a
new value of the type.

Determines a value\textquotesingle s type.

{ type } (

{ { any } }

) -\textgreater{} \href{/docs/reference/foundations/type/}{type}

\begin{verbatim}
#type(12) \
#type(14.7) \
#type("hello") \
#type(<glacier>) \
#type([Hi]) \
#type(x => x + 1) \
#type(type)
\end{verbatim}

\includegraphics[width=5in,height=\textheight,keepaspectratio]{/assets/docs/A7_wGHgPK0Jhrp3CDC6IegAAAAAAAAAA.png}

\paragraph{\texorpdfstring{\texttt{\ value\ }}{ value }}\label{constructor-value}

{ any }

{Required} {{ Positional }}

\phantomsection\label{constructor-value-positional-tooltip}
Positional parameters are specified in order, without names.

The value whose type\textquotesingle s to determine.

\href{/docs/reference/foundations/sys/}{\pandocbounded{\includesvg[keepaspectratio]{/assets/icons/16-arrow-right.svg}}}

{ System } { Previous page }

\href{/docs/reference/foundations/version/}{\pandocbounded{\includesvg[keepaspectratio]{/assets/icons/16-arrow-right.svg}}}

{ Version } { Next page }


\section{Docs LaTeX/typst.app/docs/reference/foundations/dictionary.tex}
\title{typst.app/docs/reference/foundations/dictionary}

\begin{itemize}
\tightlist
\item
  \href{/docs}{\includesvg[width=0.16667in,height=0.16667in]{/assets/icons/16-docs-dark.svg}}
\item
  \includesvg[width=0.16667in,height=0.16667in]{/assets/icons/16-arrow-right.svg}
\item
  \href{/docs/reference/}{Reference}
\item
  \includesvg[width=0.16667in,height=0.16667in]{/assets/icons/16-arrow-right.svg}
\item
  \href{/docs/reference/foundations/}{Foundations}
\item
  \includesvg[width=0.16667in,height=0.16667in]{/assets/icons/16-arrow-right.svg}
\item
  \href{/docs/reference/foundations/dictionary/}{Dictionary}
\end{itemize}

\section{\texorpdfstring{{ dictionary }}{ dictionary }}\label{summary}

A map from string keys to values.

You can construct a dictionary by enclosing comma-separated
\texttt{\ key:\ value\ } pairs in parentheses. The values do not have to
be of the same type. Since empty parentheses already yield an empty
array, you have to use the special \texttt{\ (:)\ } syntax to create an
empty dictionary.

A dictionary is conceptually similar to an array, but it is indexed by
strings instead of integers. You can access and create dictionary
entries with the \texttt{\ .at()\ } method. If you know the key
statically, you can alternatively use
\href{/docs/reference/scripting/\#fields}{field access notation} (
\texttt{\ .key\ } ) to access the value. Dictionaries can be added with
the \texttt{\ +\ } operator and
\href{/docs/reference/scripting/\#blocks}{joined together} . To check
whether a key is present in the dictionary, use the \texttt{\ in\ }
keyword.

You can iterate over the pairs in a dictionary using a
\href{/docs/reference/scripting/\#loops}{for loop} . This will iterate
in the order the pairs were inserted / declared.

\subsection{Example}\label{example}

\begin{verbatim}
#let dict = (
  name: "Typst",
  born: 2019,
)

#dict.name \
#(dict.launch = 20)
#dict.len() \
#dict.keys() \
#dict.values() \
#dict.at("born") \
#dict.insert("city", "Berlin ")
#("name" in dict)
\end{verbatim}

\includegraphics[width=5in,height=\textheight,keepaspectratio]{/assets/docs/1ByIQqDPZ4VVxPmFNoQXgwAAAAAAAAAA.png}

\subsection{\texorpdfstring{Constructor
{}}{Constructor }}\label{constructor}

\phantomsection\label{constructor-constructor-tooltip}
If a type has a constructor, you can call it like a function to create a
new value of the type.

Converts a value into a dictionary.

Note that this function is only intended for conversion of a
dictionary-like value to a dictionary, not for creation of a dictionary
from individual pairs. Use the dictionary syntax
\texttt{\ (key:\ value)\ } instead.

{ dictionary } (

{ \href{/docs/reference/foundations/module/}{module} }

) -\textgreater{}
\href{/docs/reference/foundations/dictionary/}{dictionary}

\begin{verbatim}
#dictionary(sys).at("version")
\end{verbatim}

\includegraphics[width=5in,height=\textheight,keepaspectratio]{/assets/docs/vrwNZ5Jfl6kz7gYnEOsM0AAAAAAAAAAA.png}

\paragraph{\texorpdfstring{\texttt{\ value\ }}{ value }}\label{constructor-value}

\href{/docs/reference/foundations/module/}{module}

{Required} {{ Positional }}

\phantomsection\label{constructor-value-positional-tooltip}
Positional parameters are specified in order, without names.

The value that should be converted to a dictionary.

\subsection{\texorpdfstring{{ Definitions
}}{ Definitions }}\label{definitions}

\phantomsection\label{definitions-tooltip}
Functions and types and can have associated definitions. These are
accessed by specifying the function or type, followed by a period, and
then the definition\textquotesingle s name.

\subsubsection{\texorpdfstring{\texttt{\ len\ }}{ len }}\label{definitions-len}

The number of pairs in the dictionary.

self { . } { len } (

) -\textgreater{} \href{/docs/reference/foundations/int/}{int}

\subsubsection{\texorpdfstring{\texttt{\ at\ }}{ at }}\label{definitions-at}

Returns the value associated with the specified key in the dictionary.
May be used on the left-hand side of an assignment if the key is already
present in the dictionary. Returns the default value if the key is not
part of the dictionary or fails with an error if no default value was
specified.

self { . } { at } (

{ \href{/docs/reference/foundations/str/}{str} , } {
\hyperref[definitions-at-parameters-default]{default :} { any } , }

) -\textgreater{} { any }

\paragraph{\texorpdfstring{\texttt{\ key\ }}{ key }}\label{definitions-at-key}

\href{/docs/reference/foundations/str/}{str}

{Required} {{ Positional }}

\phantomsection\label{definitions-at-key-positional-tooltip}
Positional parameters are specified in order, without names.

The key at which to retrieve the item.

\paragraph{\texorpdfstring{\texttt{\ default\ }}{ default }}\label{definitions-at-default}

{ any }

A default value to return if the key is not part of the dictionary.

\subsubsection{\texorpdfstring{\texttt{\ insert\ }}{ insert }}\label{definitions-insert}

Inserts a new pair into the dictionary. If the dictionary already
contains this key, the value is updated.

self { . } { insert } (

{ \href{/docs/reference/foundations/str/}{str} , } { { any } , }

)

\paragraph{\texorpdfstring{\texttt{\ key\ }}{ key }}\label{definitions-insert-key}

\href{/docs/reference/foundations/str/}{str}

{Required} {{ Positional }}

\phantomsection\label{definitions-insert-key-positional-tooltip}
Positional parameters are specified in order, without names.

The key of the pair that should be inserted.

\paragraph{\texorpdfstring{\texttt{\ value\ }}{ value }}\label{definitions-insert-value}

{ any }

{Required} {{ Positional }}

\phantomsection\label{definitions-insert-value-positional-tooltip}
Positional parameters are specified in order, without names.

The value of the pair that should be inserted.

\subsubsection{\texorpdfstring{\texttt{\ remove\ }}{ remove }}\label{definitions-remove}

Removes a pair from the dictionary by key and return the value.

self { . } { remove } (

{ \href{/docs/reference/foundations/str/}{str} , } {
\hyperref[definitions-remove-parameters-default]{default :} { any } , }

) -\textgreater{} { any }

\paragraph{\texorpdfstring{\texttt{\ key\ }}{ key }}\label{definitions-remove-key}

\href{/docs/reference/foundations/str/}{str}

{Required} {{ Positional }}

\phantomsection\label{definitions-remove-key-positional-tooltip}
Positional parameters are specified in order, without names.

The key of the pair to remove.

\paragraph{\texorpdfstring{\texttt{\ default\ }}{ default }}\label{definitions-remove-default}

{ any }

A default value to return if the key does not exist.

\subsubsection{\texorpdfstring{\texttt{\ keys\ }}{ keys }}\label{definitions-keys}

Returns the keys of the dictionary as an array in insertion order.

self { . } { keys } (

) -\textgreater{} \href{/docs/reference/foundations/array/}{array}

\subsubsection{\texorpdfstring{\texttt{\ values\ }}{ values }}\label{definitions-values}

Returns the values of the dictionary as an array in insertion order.

self { . } { values } (

) -\textgreater{} \href{/docs/reference/foundations/array/}{array}

\subsubsection{\texorpdfstring{\texttt{\ pairs\ }}{ pairs }}\label{definitions-pairs}

Returns the keys and values of the dictionary as an array of pairs. Each
pair is represented as an array of length two.

self { . } { pairs } (

) -\textgreater{} \href{/docs/reference/foundations/array/}{array}

\href{/docs/reference/foundations/decimal/}{\pandocbounded{\includesvg[keepaspectratio]{/assets/icons/16-arrow-right.svg}}}

{ Decimal } { Previous page }

\href{/docs/reference/foundations/duration/}{\pandocbounded{\includesvg[keepaspectratio]{/assets/icons/16-arrow-right.svg}}}

{ Duration } { Next page }


\section{Docs LaTeX/typst.app/docs/reference/foundations/str.tex}
\title{typst.app/docs/reference/foundations/str}

\begin{itemize}
\tightlist
\item
  \href{/docs}{\includesvg[width=0.16667in,height=0.16667in]{/assets/icons/16-docs-dark.svg}}
\item
  \includesvg[width=0.16667in,height=0.16667in]{/assets/icons/16-arrow-right.svg}
\item
  \href{/docs/reference/}{Reference}
\item
  \includesvg[width=0.16667in,height=0.16667in]{/assets/icons/16-arrow-right.svg}
\item
  \href{/docs/reference/foundations/}{Foundations}
\item
  \includesvg[width=0.16667in,height=0.16667in]{/assets/icons/16-arrow-right.svg}
\item
  \href{/docs/reference/foundations/str/}{String}
\end{itemize}

\section{\texorpdfstring{{ str }}{ str }}\label{summary}

A sequence of Unicode codepoints.

You can iterate over the grapheme clusters of the string using a
\href{/docs/reference/scripting/\#loops}{for loop} . Grapheme clusters
are basically characters but keep together things that belong together,
e.g. multiple codepoints that together form a flag emoji. Strings can be
added with the \texttt{\ +\ } operator,
\href{/docs/reference/scripting/\#blocks}{joined together} and
multiplied with integers.

Typst provides utility methods for string manipulation. Many of these
methods (e.g., \texttt{\ split\ } , \texttt{\ trim\ } and
\texttt{\ replace\ } ) operate on \emph{patterns:} A pattern can be
either a string or a \href{/docs/reference/foundations/regex/}{regular
expression} . This makes the methods quite versatile.

All lengths and indices are expressed in terms of UTF-8 bytes. Indices
are zero-based and negative indices wrap around to the end of the
string.

You can convert a value to a string with this type\textquotesingle s
constructor.

\subsection{Example}\label{example}

\begin{verbatim}
#"hello world!" \
#"\"hello\n  world\"!" \
#"1 2 3".split() \
#"1,2;3".split(regex("[,;]")) \
#(regex("\d+") in "ten euros") \
#(regex("\d+") in "10 euros")
\end{verbatim}

\includegraphics[width=5in,height=\textheight,keepaspectratio]{/assets/docs/gK89AnI9k7dy82m9R3F1jgAAAAAAAAAA.png}

\subsection{Escape sequences}\label{escapes}

Just like in markup, you can escape a few symbols in strings:

\begin{itemize}
\tightlist
\item
  \texttt{\ }{\texttt{\ \textbackslash{}\textbackslash{}\ }}\texttt{\ }
  for a backslash
\item
  \texttt{\ }{\texttt{\ \textbackslash{}"\ }}\texttt{\ } for a quote
\item
  \texttt{\ }{\texttt{\ \textbackslash{}n\ }}\texttt{\ } for a newline
\item
  \texttt{\ }{\texttt{\ \textbackslash{}r\ }}\texttt{\ } for a carriage
  return
\item
  \texttt{\ }{\texttt{\ \textbackslash{}t\ }}\texttt{\ } for a tab
\item
  \texttt{\ }{\texttt{\ \textbackslash{}u\{1f600\}\ }}\texttt{\ } for a
  hexadecimal Unicode escape sequence
\end{itemize}

\subsection{\texorpdfstring{Constructor
{}}{Constructor }}\label{constructor}

\phantomsection\label{constructor-constructor-tooltip}
If a type has a constructor, you can call it like a function to create a
new value of the type.

Converts a value to a string.

\begin{itemize}
\tightlist
\item
  Integers are formatted in base 10. This can be overridden with the
  optional \texttt{\ base\ } parameter.
\item
  Floats are formatted in base 10 and never in exponential notation.
\item
  From labels the name is extracted.
\item
  Bytes are decoded as UTF-8.
\end{itemize}

If you wish to convert from and to Unicode code points, see the
\href{/docs/reference/foundations/str/\#definitions-to-unicode}{\texttt{\ to-unicode\ }}
and
\href{/docs/reference/foundations/str/\#definitions-from-unicode}{\texttt{\ from-unicode\ }}
functions.

{ str } (

{ \href{/docs/reference/foundations/int/}{int}
\href{/docs/reference/foundations/float/}{float}
\href{/docs/reference/foundations/str/}{str}
\href{/docs/reference/foundations/bytes/}{bytes}
\href{/docs/reference/foundations/label/}{label}
\href{/docs/reference/foundations/decimal/}{decimal}
\href{/docs/reference/foundations/version/}{version}
\href{/docs/reference/foundations/type/}{type} , } {
\hyperref[constructor-parameters-base]{base :}
\href{/docs/reference/foundations/int/}{int} , }

) -\textgreater{} \href{/docs/reference/foundations/str/}{str}

\begin{verbatim}
#str(10) \
#str(4000, base: 16) \
#str(2.7) \
#str(1e8) \
#str(<intro>)
\end{verbatim}

\includegraphics[width=5in,height=\textheight,keepaspectratio]{/assets/docs/06jR9z-fP-M4eu8XB2MFnAAAAAAAAAAA.png}

\paragraph{\texorpdfstring{\texttt{\ value\ }}{ value }}\label{constructor-value}

\href{/docs/reference/foundations/int/}{int} {or}
\href{/docs/reference/foundations/float/}{float} {or}
\href{/docs/reference/foundations/str/}{str} {or}
\href{/docs/reference/foundations/bytes/}{bytes} {or}
\href{/docs/reference/foundations/label/}{label} {or}
\href{/docs/reference/foundations/decimal/}{decimal} {or}
\href{/docs/reference/foundations/version/}{version} {or}
\href{/docs/reference/foundations/type/}{type}

{Required} {{ Positional }}

\phantomsection\label{constructor-value-positional-tooltip}
Positional parameters are specified in order, without names.

The value that should be converted to a string.

\paragraph{\texorpdfstring{\texttt{\ base\ }}{ base }}\label{constructor-base}

\href{/docs/reference/foundations/int/}{int}

The base (radix) to display integers in, between 2 and 36.

Default: \texttt{\ }{\texttt{\ 10\ }}\texttt{\ }

\subsection{\texorpdfstring{{ Definitions
}}{ Definitions }}\label{definitions}

\phantomsection\label{definitions-tooltip}
Functions and types and can have associated definitions. These are
accessed by specifying the function or type, followed by a period, and
then the definition\textquotesingle s name.

\subsubsection{\texorpdfstring{\texttt{\ len\ }}{ len }}\label{definitions-len}

The length of the string in UTF-8 encoded bytes.

self { . } { len } (

) -\textgreater{} \href{/docs/reference/foundations/int/}{int}

\subsubsection{\texorpdfstring{\texttt{\ first\ }}{ first }}\label{definitions-first}

Extracts the first grapheme cluster of the string. Fails with an error
if the string is empty.

self { . } { first } (

) -\textgreater{} \href{/docs/reference/foundations/str/}{str}

\subsubsection{\texorpdfstring{\texttt{\ last\ }}{ last }}\label{definitions-last}

Extracts the last grapheme cluster of the string. Fails with an error if
the string is empty.

self { . } { last } (

) -\textgreater{} \href{/docs/reference/foundations/str/}{str}

\subsubsection{\texorpdfstring{\texttt{\ at\ }}{ at }}\label{definitions-at}

Extracts the first grapheme cluster after the specified index. Returns
the default value if the index is out of bounds or fails with an error
if no default value was specified.

self { . } { at } (

{ \href{/docs/reference/foundations/int/}{int} , } {
\hyperref[definitions-at-parameters-default]{default :} { any } , }

) -\textgreater{} { any }

\paragraph{\texorpdfstring{\texttt{\ index\ }}{ index }}\label{definitions-at-index}

\href{/docs/reference/foundations/int/}{int}

{Required} {{ Positional }}

\phantomsection\label{definitions-at-index-positional-tooltip}
Positional parameters are specified in order, without names.

The byte index. If negative, indexes from the back.

\paragraph{\texorpdfstring{\texttt{\ default\ }}{ default }}\label{definitions-at-default}

{ any }

A default value to return if the index is out of bounds.

\subsubsection{\texorpdfstring{\texttt{\ slice\ }}{ slice }}\label{definitions-slice}

Extracts a substring of the string. Fails with an error if the start or
end index is out of bounds.

self { . } { slice } (

{ \href{/docs/reference/foundations/int/}{int} , } {
\href{/docs/reference/foundations/none/}{none}
\href{/docs/reference/foundations/int/}{int} , } {
\hyperref[definitions-slice-parameters-count]{count :}
\href{/docs/reference/foundations/int/}{int} , }

) -\textgreater{} \href{/docs/reference/foundations/str/}{str}

\paragraph{\texorpdfstring{\texttt{\ start\ }}{ start }}\label{definitions-slice-start}

\href{/docs/reference/foundations/int/}{int}

{Required} {{ Positional }}

\phantomsection\label{definitions-slice-start-positional-tooltip}
Positional parameters are specified in order, without names.

The start byte index (inclusive). If negative, indexes from the back.

\paragraph{\texorpdfstring{\texttt{\ end\ }}{ end }}\label{definitions-slice-end}

\href{/docs/reference/foundations/none/}{none} {or}
\href{/docs/reference/foundations/int/}{int}

{{ Positional }}

\phantomsection\label{definitions-slice-end-positional-tooltip}
Positional parameters are specified in order, without names.

The end byte index (exclusive). If omitted, the whole slice until the
end of the string is extracted. If negative, indexes from the back.

Default: \texttt{\ }{\texttt{\ none\ }}\texttt{\ }

\paragraph{\texorpdfstring{\texttt{\ count\ }}{ count }}\label{definitions-slice-count}

\href{/docs/reference/foundations/int/}{int}

The number of bytes to extract. This is equivalent to passing
\texttt{\ start\ +\ count\ } as the \texttt{\ end\ } position. Mutually
exclusive with \texttt{\ end\ } .

\subsubsection{\texorpdfstring{\texttt{\ clusters\ }}{ clusters }}\label{definitions-clusters}

Returns the grapheme clusters of the string as an array of substrings.

self { . } { clusters } (

) -\textgreater{} \href{/docs/reference/foundations/array/}{array}

\subsubsection{\texorpdfstring{\texttt{\ codepoints\ }}{ codepoints }}\label{definitions-codepoints}

Returns the Unicode codepoints of the string as an array of substrings.

self { . } { codepoints } (

) -\textgreater{} \href{/docs/reference/foundations/array/}{array}

\subsubsection{\texorpdfstring{\texttt{\ to-unicode\ }}{ to-unicode }}\label{definitions-to-unicode}

Converts a character into its corresponding code point.

str { . } { to-unicode } (

{ \href{/docs/reference/foundations/str/}{str} }

) -\textgreater{} \href{/docs/reference/foundations/int/}{int}

\begin{verbatim}
#"a".to-unicode() \
#("a\u{0300}"
   .codepoints()
   .map(str.to-unicode))
\end{verbatim}

\includegraphics[width=5in,height=\textheight,keepaspectratio]{/assets/docs/q50tz6WAJPnwtBCYWbHrIwAAAAAAAAAA.png}

\paragraph{\texorpdfstring{\texttt{\ character\ }}{ character }}\label{definitions-to-unicode-character}

\href{/docs/reference/foundations/str/}{str}

{Required} {{ Positional }}

\phantomsection\label{definitions-to-unicode-character-positional-tooltip}
Positional parameters are specified in order, without names.

The character that should be converted.

\subsubsection{\texorpdfstring{\texttt{\ from-unicode\ }}{ from-unicode }}\label{definitions-from-unicode}

Converts a unicode code point into its corresponding string.

str { . } { from-unicode } (

{ \href{/docs/reference/foundations/int/}{int} }

) -\textgreater{} \href{/docs/reference/foundations/str/}{str}

\begin{verbatim}
#str.from-unicode(97)
\end{verbatim}

\includegraphics[width=5in,height=\textheight,keepaspectratio]{/assets/docs/vNzcsGO4Zd_u-P4qNnxrDQAAAAAAAAAA.png}

\paragraph{\texorpdfstring{\texttt{\ value\ }}{ value }}\label{definitions-from-unicode-value}

\href{/docs/reference/foundations/int/}{int}

{Required} {{ Positional }}

\phantomsection\label{definitions-from-unicode-value-positional-tooltip}
Positional parameters are specified in order, without names.

The code point that should be converted.

\subsubsection{\texorpdfstring{\texttt{\ contains\ }}{ contains }}\label{definitions-contains}

Whether the string contains the specified pattern.

This method also has dedicated syntax: You can write
\texttt{\ }{\texttt{\ "bc"\ }}\texttt{\ }{\texttt{\ in\ }}\texttt{\ }{\texttt{\ "abcd"\ }}\texttt{\ }
instead of
\texttt{\ }{\texttt{\ "abcd"\ }}\texttt{\ }{\texttt{\ .\ }}\texttt{\ }{\texttt{\ contains\ }}\texttt{\ }{\texttt{\ (\ }}\texttt{\ }{\texttt{\ "bc"\ }}\texttt{\ }{\texttt{\ )\ }}\texttt{\ }
.

self { . } { contains } (

{ \href{/docs/reference/foundations/str/}{str}
\href{/docs/reference/foundations/regex/}{regex} }

) -\textgreater{} \href{/docs/reference/foundations/bool/}{bool}

\paragraph{\texorpdfstring{\texttt{\ pattern\ }}{ pattern }}\label{definitions-contains-pattern}

\href{/docs/reference/foundations/str/}{str} {or}
\href{/docs/reference/foundations/regex/}{regex}

{Required} {{ Positional }}

\phantomsection\label{definitions-contains-pattern-positional-tooltip}
Positional parameters are specified in order, without names.

The pattern to search for.

\subsubsection{\texorpdfstring{\texttt{\ starts-with\ }}{ starts-with }}\label{definitions-starts-with}

Whether the string starts with the specified pattern.

self { . } { starts-with } (

{ \href{/docs/reference/foundations/str/}{str}
\href{/docs/reference/foundations/regex/}{regex} }

) -\textgreater{} \href{/docs/reference/foundations/bool/}{bool}

\paragraph{\texorpdfstring{\texttt{\ pattern\ }}{ pattern }}\label{definitions-starts-with-pattern}

\href{/docs/reference/foundations/str/}{str} {or}
\href{/docs/reference/foundations/regex/}{regex}

{Required} {{ Positional }}

\phantomsection\label{definitions-starts-with-pattern-positional-tooltip}
Positional parameters are specified in order, without names.

The pattern the string might start with.

\subsubsection{\texorpdfstring{\texttt{\ ends-with\ }}{ ends-with }}\label{definitions-ends-with}

Whether the string ends with the specified pattern.

self { . } { ends-with } (

{ \href{/docs/reference/foundations/str/}{str}
\href{/docs/reference/foundations/regex/}{regex} }

) -\textgreater{} \href{/docs/reference/foundations/bool/}{bool}

\paragraph{\texorpdfstring{\texttt{\ pattern\ }}{ pattern }}\label{definitions-ends-with-pattern}

\href{/docs/reference/foundations/str/}{str} {or}
\href{/docs/reference/foundations/regex/}{regex}

{Required} {{ Positional }}

\phantomsection\label{definitions-ends-with-pattern-positional-tooltip}
Positional parameters are specified in order, without names.

The pattern the string might end with.

\subsubsection{\texorpdfstring{\texttt{\ find\ }}{ find }}\label{definitions-find}

Searches for the specified pattern in the string and returns the first
match as a string or \texttt{\ }{\texttt{\ none\ }}\texttt{\ } if there
is no match.

self { . } { find } (

{ \href{/docs/reference/foundations/str/}{str}
\href{/docs/reference/foundations/regex/}{regex} }

) -\textgreater{} \href{/docs/reference/foundations/none/}{none}
\href{/docs/reference/foundations/str/}{str}

\paragraph{\texorpdfstring{\texttt{\ pattern\ }}{ pattern }}\label{definitions-find-pattern}

\href{/docs/reference/foundations/str/}{str} {or}
\href{/docs/reference/foundations/regex/}{regex}

{Required} {{ Positional }}

\phantomsection\label{definitions-find-pattern-positional-tooltip}
Positional parameters are specified in order, without names.

The pattern to search for.

\subsubsection{\texorpdfstring{\texttt{\ position\ }}{ position }}\label{definitions-position}

Searches for the specified pattern in the string and returns the index
of the first match as an integer or
\texttt{\ }{\texttt{\ none\ }}\texttt{\ } if there is no match.

self { . } { position } (

{ \href{/docs/reference/foundations/str/}{str}
\href{/docs/reference/foundations/regex/}{regex} }

) -\textgreater{} \href{/docs/reference/foundations/none/}{none}
\href{/docs/reference/foundations/int/}{int}

\paragraph{\texorpdfstring{\texttt{\ pattern\ }}{ pattern }}\label{definitions-position-pattern}

\href{/docs/reference/foundations/str/}{str} {or}
\href{/docs/reference/foundations/regex/}{regex}

{Required} {{ Positional }}

\phantomsection\label{definitions-position-pattern-positional-tooltip}
Positional parameters are specified in order, without names.

The pattern to search for.

\subsubsection{\texorpdfstring{\texttt{\ match\ }}{ match }}\label{definitions-match}

Searches for the specified pattern in the string and returns a
dictionary with details about the first match or
\texttt{\ }{\texttt{\ none\ }}\texttt{\ } if there is no match.

The returned dictionary has the following keys:

\begin{itemize}
\tightlist
\item
  \texttt{\ start\ } : The start offset of the match
\item
  \texttt{\ end\ } : The end offset of the match
\item
  \texttt{\ text\ } : The text that matched.
\item
  \texttt{\ captures\ } : An array containing a string for each matched
  capturing group. The first item of the array contains the first
  matched capturing, not the whole match! This is empty unless the
  \texttt{\ pattern\ } was a regex with capturing groups.
\end{itemize}

self { . } { match } (

{ \href{/docs/reference/foundations/str/}{str}
\href{/docs/reference/foundations/regex/}{regex} }

) -\textgreater{} \href{/docs/reference/foundations/none/}{none}
\href{/docs/reference/foundations/dictionary/}{dictionary}

\paragraph{\texorpdfstring{\texttt{\ pattern\ }}{ pattern }}\label{definitions-match-pattern}

\href{/docs/reference/foundations/str/}{str} {or}
\href{/docs/reference/foundations/regex/}{regex}

{Required} {{ Positional }}

\phantomsection\label{definitions-match-pattern-positional-tooltip}
Positional parameters are specified in order, without names.

The pattern to search for.

\subsubsection{\texorpdfstring{\texttt{\ matches\ }}{ matches }}\label{definitions-matches}

Searches for the specified pattern in the string and returns an array of
dictionaries with details about all matches. For details about the
returned dictionaries, see above.

self { . } { matches } (

{ \href{/docs/reference/foundations/str/}{str}
\href{/docs/reference/foundations/regex/}{regex} }

) -\textgreater{} \href{/docs/reference/foundations/array/}{array}

\paragraph{\texorpdfstring{\texttt{\ pattern\ }}{ pattern }}\label{definitions-matches-pattern}

\href{/docs/reference/foundations/str/}{str} {or}
\href{/docs/reference/foundations/regex/}{regex}

{Required} {{ Positional }}

\phantomsection\label{definitions-matches-pattern-positional-tooltip}
Positional parameters are specified in order, without names.

The pattern to search for.

\subsubsection{\texorpdfstring{\texttt{\ replace\ }}{ replace }}\label{definitions-replace}

Replace at most \texttt{\ count\ } occurrences of the given pattern with
a replacement string or function (beginning from the start). If no count
is given, all occurrences are replaced.

self { . } { replace } (

{ \href{/docs/reference/foundations/str/}{str}
\href{/docs/reference/foundations/regex/}{regex} , } {
\href{/docs/reference/foundations/str/}{str}
\href{/docs/reference/foundations/function/}{function} , } {
\hyperref[definitions-replace-parameters-count]{count :}
\href{/docs/reference/foundations/int/}{int} , }

) -\textgreater{} \href{/docs/reference/foundations/str/}{str}

\paragraph{\texorpdfstring{\texttt{\ pattern\ }}{ pattern }}\label{definitions-replace-pattern}

\href{/docs/reference/foundations/str/}{str} {or}
\href{/docs/reference/foundations/regex/}{regex}

{Required} {{ Positional }}

\phantomsection\label{definitions-replace-pattern-positional-tooltip}
Positional parameters are specified in order, without names.

The pattern to search for.

\paragraph{\texorpdfstring{\texttt{\ replacement\ }}{ replacement }}\label{definitions-replace-replacement}

\href{/docs/reference/foundations/str/}{str} {or}
\href{/docs/reference/foundations/function/}{function}

{Required} {{ Positional }}

\phantomsection\label{definitions-replace-replacement-positional-tooltip}
Positional parameters are specified in order, without names.

The string to replace the matches with or a function that gets a
dictionary for each match and can return individual replacement strings.

\paragraph{\texorpdfstring{\texttt{\ count\ }}{ count }}\label{definitions-replace-count}

\href{/docs/reference/foundations/int/}{int}

If given, only the first \texttt{\ count\ } matches of the pattern are
placed.

\subsubsection{\texorpdfstring{\texttt{\ trim\ }}{ trim }}\label{definitions-trim}

Removes matches of a pattern from one or both sides of the string, once
or repeatedly and returns the resulting string.

self { . } { trim } (

{ \href{/docs/reference/foundations/none/}{none}
\href{/docs/reference/foundations/str/}{str}
\href{/docs/reference/foundations/regex/}{regex} , } {
\hyperref[definitions-trim-parameters-at]{at :}
\href{/docs/reference/layout/alignment/}{alignment} , } {
\hyperref[definitions-trim-parameters-repeat]{repeat :}
\href{/docs/reference/foundations/bool/}{bool} , }

) -\textgreater{} \href{/docs/reference/foundations/str/}{str}

\paragraph{\texorpdfstring{\texttt{\ pattern\ }}{ pattern }}\label{definitions-trim-pattern}

\href{/docs/reference/foundations/none/}{none} {or}
\href{/docs/reference/foundations/str/}{str} {or}
\href{/docs/reference/foundations/regex/}{regex}

{{ Positional }}

\phantomsection\label{definitions-trim-pattern-positional-tooltip}
Positional parameters are specified in order, without names.

The pattern to search for. If \texttt{\ }{\texttt{\ none\ }}\texttt{\ }
, trims white spaces.

Default: \texttt{\ }{\texttt{\ none\ }}\texttt{\ }

\paragraph{\texorpdfstring{\texttt{\ at\ }}{ at }}\label{definitions-trim-at}

\href{/docs/reference/layout/alignment/}{alignment}

Can be \texttt{\ start\ } or \texttt{\ end\ } to only trim the start or
end of the string. If omitted, both sides are trimmed.

\paragraph{\texorpdfstring{\texttt{\ repeat\ }}{ repeat }}\label{definitions-trim-repeat}

\href{/docs/reference/foundations/bool/}{bool}

Whether to repeatedly removes matches of the pattern or just once.
Defaults to \texttt{\ }{\texttt{\ true\ }}\texttt{\ } .

Default: \texttt{\ }{\texttt{\ true\ }}\texttt{\ }

\subsubsection{\texorpdfstring{\texttt{\ split\ }}{ split }}\label{definitions-split}

Splits a string at matches of a specified pattern and returns an array
of the resulting parts.

self { . } { split } (

{ \href{/docs/reference/foundations/none/}{none}
\href{/docs/reference/foundations/str/}{str}
\href{/docs/reference/foundations/regex/}{regex} }

) -\textgreater{} \href{/docs/reference/foundations/array/}{array}

\paragraph{\texorpdfstring{\texttt{\ pattern\ }}{ pattern }}\label{definitions-split-pattern}

\href{/docs/reference/foundations/none/}{none} {or}
\href{/docs/reference/foundations/str/}{str} {or}
\href{/docs/reference/foundations/regex/}{regex}

{{ Positional }}

\phantomsection\label{definitions-split-pattern-positional-tooltip}
Positional parameters are specified in order, without names.

The pattern to split at. Defaults to whitespace.

Default: \texttt{\ }{\texttt{\ none\ }}\texttt{\ }

\subsubsection{\texorpdfstring{\texttt{\ rev\ }}{ rev }}\label{definitions-rev}

Reverse the string.

self { . } { rev } (

) -\textgreater{} \href{/docs/reference/foundations/str/}{str}

\href{/docs/reference/foundations/selector/}{\pandocbounded{\includesvg[keepaspectratio]{/assets/icons/16-arrow-right.svg}}}

{ Selector } { Previous page }

\href{/docs/reference/foundations/style/}{\pandocbounded{\includesvg[keepaspectratio]{/assets/icons/16-arrow-right.svg}}}

{ Style } { Next page }


\section{Docs LaTeX/typst.app/docs/reference/foundations/selector.tex}
\title{typst.app/docs/reference/foundations/selector}

\begin{itemize}
\tightlist
\item
  \href{/docs}{\includesvg[width=0.16667in,height=0.16667in]{/assets/icons/16-docs-dark.svg}}
\item
  \includesvg[width=0.16667in,height=0.16667in]{/assets/icons/16-arrow-right.svg}
\item
  \href{/docs/reference/}{Reference}
\item
  \includesvg[width=0.16667in,height=0.16667in]{/assets/icons/16-arrow-right.svg}
\item
  \href{/docs/reference/foundations/}{Foundations}
\item
  \includesvg[width=0.16667in,height=0.16667in]{/assets/icons/16-arrow-right.svg}
\item
  \href{/docs/reference/foundations/selector/}{Selector}
\end{itemize}

\section{\texorpdfstring{{ selector }}{ selector }}\label{summary}

A filter for selecting elements within the document.

You can construct a selector in the following ways:

\begin{itemize}
\tightlist
\item
  you can use an element
  \href{/docs/reference/foundations/function/}{function}
\item
  you can filter for an element function with
  \href{/docs/reference/foundations/function/\#definitions-where}{specific
  fields}
\item
  you can use a \href{/docs/reference/foundations/str/}{string} or
  \href{/docs/reference/foundations/regex/}{regular expression}
\item
  you can use a
  \href{/docs/reference/foundations/label/}{\texttt{\ }{\texttt{\ \textless{}label\textgreater{}\ }}\texttt{\ }}
\item
  you can use a
  \href{/docs/reference/introspection/location/}{\texttt{\ location\ }}
\item
  call the
  \href{/docs/reference/foundations/selector/}{\texttt{\ selector\ }}
  constructor to convert any of the above types into a selector value
  and use the methods below to refine it
\end{itemize}

Selectors are used to \href{/docs/reference/styling/\#show-rules}{apply
styling rules} to elements. You can also use selectors to
\href{/docs/reference/introspection/query/}{query} the document for
certain types of elements.

Furthermore, you can pass a selector to several of
Typst\textquotesingle s built-in functions to configure their behaviour.
One such example is the \href{/docs/reference/model/outline/}{outline}
where it can be used to change which elements are listed within the
outline.

Multiple selectors can be combined using the methods shown below.
However, not all kinds of selectors are supported in all places, at the
moment.

\subsection{Example}\label{example}

\begin{verbatim}
#context query(
  heading.where(level: 1)
    .or(heading.where(level: 2))
)

= This will be found
== So will this
=== But this will not.
\end{verbatim}

\includegraphics[width=5in,height=\textheight,keepaspectratio]{/assets/docs/SW-2iLP1LIGQ0ITsB7LGEQAAAAAAAAAA.png}

\subsection{\texorpdfstring{Constructor
{}}{Constructor }}\label{constructor}

\phantomsection\label{constructor-constructor-tooltip}
If a type has a constructor, you can call it like a function to create a
new value of the type.

Turns a value into a selector. The following values are accepted:

\begin{itemize}
\tightlist
\item
  An element function like a \texttt{\ heading\ } or \texttt{\ figure\ }
  .
\item
  A \texttt{\ }{\texttt{\ \textless{}label\textgreater{}\ }}\texttt{\ }
  .
\item
  A more complex selector like
  \texttt{\ heading\ }{\texttt{\ .\ }}\texttt{\ }{\texttt{\ where\ }}\texttt{\ }{\texttt{\ (\ }}\texttt{\ level\ }{\texttt{\ :\ }}\texttt{\ }{\texttt{\ 1\ }}\texttt{\ }{\texttt{\ )\ }}\texttt{\ }
  .
\end{itemize}

{ selector } (

{ \href{/docs/reference/foundations/str/}{str}
\href{/docs/reference/foundations/regex/}{regex}
\href{/docs/reference/foundations/label/}{label}
\href{/docs/reference/foundations/selector/}{selector}
\href{/docs/reference/introspection/location/}{location}
\href{/docs/reference/foundations/function/}{function} }

) -\textgreater{} \href{/docs/reference/foundations/selector/}{selector}

\paragraph{\texorpdfstring{\texttt{\ target\ }}{ target }}\label{constructor-target}

\href{/docs/reference/foundations/str/}{str} {or}
\href{/docs/reference/foundations/regex/}{regex} {or}
\href{/docs/reference/foundations/label/}{label} {or}
\href{/docs/reference/foundations/selector/}{selector} {or}
\href{/docs/reference/introspection/location/}{location} {or}
\href{/docs/reference/foundations/function/}{function}

{Required} {{ Positional }}

\phantomsection\label{constructor-target-positional-tooltip}
Positional parameters are specified in order, without names.

Can be an element function like a \texttt{\ heading\ } or
\texttt{\ figure\ } , a
\texttt{\ }{\texttt{\ \textless{}label\textgreater{}\ }}\texttt{\ } or a
more complex selector like
\texttt{\ heading\ }{\texttt{\ .\ }}\texttt{\ }{\texttt{\ where\ }}\texttt{\ }{\texttt{\ (\ }}\texttt{\ level\ }{\texttt{\ :\ }}\texttt{\ }{\texttt{\ 1\ }}\texttt{\ }{\texttt{\ )\ }}\texttt{\ }
.

\subsection{\texorpdfstring{{ Definitions
}}{ Definitions }}\label{definitions}

\phantomsection\label{definitions-tooltip}
Functions and types and can have associated definitions. These are
accessed by specifying the function or type, followed by a period, and
then the definition\textquotesingle s name.

\subsubsection{\texorpdfstring{\texttt{\ or\ }}{ or }}\label{definitions-or}

Selects all elements that match this or any of the other selectors.

self { . } { or } (

{ \hyperref[definitions-or-parameters-others]{..}
\href{/docs/reference/foundations/str/}{str}
\href{/docs/reference/foundations/regex/}{regex}
\href{/docs/reference/foundations/label/}{label}
\href{/docs/reference/foundations/selector/}{selector}
\href{/docs/reference/introspection/location/}{location}
\href{/docs/reference/foundations/function/}{function} }

) -\textgreater{} \href{/docs/reference/foundations/selector/}{selector}

\paragraph{\texorpdfstring{\texttt{\ others\ }}{ others }}\label{definitions-or-others}

\href{/docs/reference/foundations/str/}{str} {or}
\href{/docs/reference/foundations/regex/}{regex} {or}
\href{/docs/reference/foundations/label/}{label} {or}
\href{/docs/reference/foundations/selector/}{selector} {or}
\href{/docs/reference/introspection/location/}{location} {or}
\href{/docs/reference/foundations/function/}{function}

{Required} {{ Positional }}

\phantomsection\label{definitions-or-others-positional-tooltip}
Positional parameters are specified in order, without names.

{{ Variadic }}

\phantomsection\label{definitions-or-others-variadic-tooltip}
Variadic parameters can be specified multiple times.

The other selectors to match on.

\subsubsection{\texorpdfstring{\texttt{\ and\ }}{ and }}\label{definitions-and}

Selects all elements that match this and all of the other selectors.

self { . } { and } (

{ \hyperref[definitions-and-parameters-others]{..}
\href{/docs/reference/foundations/str/}{str}
\href{/docs/reference/foundations/regex/}{regex}
\href{/docs/reference/foundations/label/}{label}
\href{/docs/reference/foundations/selector/}{selector}
\href{/docs/reference/introspection/location/}{location}
\href{/docs/reference/foundations/function/}{function} }

) -\textgreater{} \href{/docs/reference/foundations/selector/}{selector}

\paragraph{\texorpdfstring{\texttt{\ others\ }}{ others }}\label{definitions-and-others}

\href{/docs/reference/foundations/str/}{str} {or}
\href{/docs/reference/foundations/regex/}{regex} {or}
\href{/docs/reference/foundations/label/}{label} {or}
\href{/docs/reference/foundations/selector/}{selector} {or}
\href{/docs/reference/introspection/location/}{location} {or}
\href{/docs/reference/foundations/function/}{function}

{Required} {{ Positional }}

\phantomsection\label{definitions-and-others-positional-tooltip}
Positional parameters are specified in order, without names.

{{ Variadic }}

\phantomsection\label{definitions-and-others-variadic-tooltip}
Variadic parameters can be specified multiple times.

The other selectors to match on.

\subsubsection{\texorpdfstring{\texttt{\ before\ }}{ before }}\label{definitions-before}

Returns a modified selector that will only match elements that occur
before the first match of \texttt{\ end\ } .

self { . } { before } (

{ \href{/docs/reference/foundations/label/}{label}
\href{/docs/reference/foundations/selector/}{selector}
\href{/docs/reference/introspection/location/}{location}
\href{/docs/reference/foundations/function/}{function} , } {
\hyperref[definitions-before-parameters-inclusive]{inclusive :}
\href{/docs/reference/foundations/bool/}{bool} , }

) -\textgreater{} \href{/docs/reference/foundations/selector/}{selector}

\paragraph{\texorpdfstring{\texttt{\ end\ }}{ end }}\label{definitions-before-end}

\href{/docs/reference/foundations/label/}{label} {or}
\href{/docs/reference/foundations/selector/}{selector} {or}
\href{/docs/reference/introspection/location/}{location} {or}
\href{/docs/reference/foundations/function/}{function}

{Required} {{ Positional }}

\phantomsection\label{definitions-before-end-positional-tooltip}
Positional parameters are specified in order, without names.

The original selection will end at the first match of \texttt{\ end\ } .

\paragraph{\texorpdfstring{\texttt{\ inclusive\ }}{ inclusive }}\label{definitions-before-inclusive}

\href{/docs/reference/foundations/bool/}{bool}

Whether \texttt{\ end\ } itself should match or not. This is only
relevant if both selectors match the same type of element. Defaults to
\texttt{\ }{\texttt{\ true\ }}\texttt{\ } .

Default: \texttt{\ }{\texttt{\ true\ }}\texttt{\ }

\subsubsection{\texorpdfstring{\texttt{\ after\ }}{ after }}\label{definitions-after}

Returns a modified selector that will only match elements that occur
after the first match of \texttt{\ start\ } .

self { . } { after } (

{ \href{/docs/reference/foundations/label/}{label}
\href{/docs/reference/foundations/selector/}{selector}
\href{/docs/reference/introspection/location/}{location}
\href{/docs/reference/foundations/function/}{function} , } {
\hyperref[definitions-after-parameters-inclusive]{inclusive :}
\href{/docs/reference/foundations/bool/}{bool} , }

) -\textgreater{} \href{/docs/reference/foundations/selector/}{selector}

\paragraph{\texorpdfstring{\texttt{\ start\ }}{ start }}\label{definitions-after-start}

\href{/docs/reference/foundations/label/}{label} {or}
\href{/docs/reference/foundations/selector/}{selector} {or}
\href{/docs/reference/introspection/location/}{location} {or}
\href{/docs/reference/foundations/function/}{function}

{Required} {{ Positional }}

\phantomsection\label{definitions-after-start-positional-tooltip}
Positional parameters are specified in order, without names.

The original selection will start at the first match of
\texttt{\ start\ } .

\paragraph{\texorpdfstring{\texttt{\ inclusive\ }}{ inclusive }}\label{definitions-after-inclusive}

\href{/docs/reference/foundations/bool/}{bool}

Whether \texttt{\ start\ } itself should match or not. This is only
relevant if both selectors match the same type of element. Defaults to
\texttt{\ }{\texttt{\ true\ }}\texttt{\ } .

Default: \texttt{\ }{\texttt{\ true\ }}\texttt{\ }

\href{/docs/reference/foundations/repr/}{\pandocbounded{\includesvg[keepaspectratio]{/assets/icons/16-arrow-right.svg}}}

{ Representation } { Previous page }

\href{/docs/reference/foundations/str/}{\pandocbounded{\includesvg[keepaspectratio]{/assets/icons/16-arrow-right.svg}}}

{ String } { Next page }


\section{Docs LaTeX/typst.app/docs/reference/foundations/eval.tex}
\title{typst.app/docs/reference/foundations/eval}

\begin{itemize}
\tightlist
\item
  \href{/docs}{\includesvg[width=0.16667in,height=0.16667in]{/assets/icons/16-docs-dark.svg}}
\item
  \includesvg[width=0.16667in,height=0.16667in]{/assets/icons/16-arrow-right.svg}
\item
  \href{/docs/reference/}{Reference}
\item
  \includesvg[width=0.16667in,height=0.16667in]{/assets/icons/16-arrow-right.svg}
\item
  \href{/docs/reference/foundations/}{Foundations}
\item
  \includesvg[width=0.16667in,height=0.16667in]{/assets/icons/16-arrow-right.svg}
\item
  \href{/docs/reference/foundations/eval/}{Evaluate}
\end{itemize}

\section{\texorpdfstring{\texttt{\ eval\ }}{ eval }}\label{summary}

Evaluates a string as Typst code.

This function should only be used as a last resort.

\subsection{Example}\label{example}

\begin{verbatim}
#eval("1 + 1") \
#eval("(1, 2, 3, 4)").len() \
#eval("*Markup!*", mode: "markup") \
\end{verbatim}

\includegraphics[width=5in,height=\textheight,keepaspectratio]{/assets/docs/KZfqDZ_7V1ElK4um94vvjwAAAAAAAAAA.png}

\subsection{\texorpdfstring{{ Parameters
}}{ Parameters }}\label{parameters}

\phantomsection\label{parameters-tooltip}
Parameters are the inputs to a function. They are specified in
parentheses after the function name.

{ eval } (

{ \href{/docs/reference/foundations/str/}{str} , } {
\hyperref[parameters-mode]{mode :}
\href{/docs/reference/foundations/str/}{str} , } {
\hyperref[parameters-scope]{scope :}
\href{/docs/reference/foundations/dictionary/}{dictionary} , }

) -\textgreater{} { any }

\subsubsection{\texorpdfstring{\texttt{\ source\ }}{ source }}\label{parameters-source}

\href{/docs/reference/foundations/str/}{str}

{Required} {{ Positional }}

\phantomsection\label{parameters-source-positional-tooltip}
Positional parameters are specified in order, without names.

A string of Typst code to evaluate.

\subsubsection{\texorpdfstring{\texttt{\ mode\ }}{ mode }}\label{parameters-mode}

\href{/docs/reference/foundations/str/}{str}

The \href{/docs/reference/syntax/\#modes}{syntactical mode} in which the
string is parsed.

\begin{longtable}[]{@{}ll@{}}
\toprule\noalign{}
Variant & Details \\
\midrule\noalign{}
\endhead
\bottomrule\noalign{}
\endlastfoot
\texttt{\ "\ code\ "\ } & Evaluate as code, as after a hash. \\
\texttt{\ "\ markup\ "\ } & Evaluate as markup, like in a Typst file. \\
\texttt{\ "\ math\ "\ } & Evaluate as math, as in an equation. \\
\end{longtable}

Default: \texttt{\ }{\texttt{\ "code"\ }}\texttt{\ }

\includesvg[width=0.16667in,height=0.16667in]{/assets/icons/16-arrow-right.svg}
View example

\begin{verbatim}
#eval("= Heading", mode: "markup")
#eval("1_2^3", mode: "math")
\end{verbatim}

\includegraphics[width=5in,height=\textheight,keepaspectratio]{/assets/docs/4OYmfbro6ZT1td5j4R5wyAAAAAAAAAAA.png}

\subsubsection{\texorpdfstring{\texttt{\ scope\ }}{ scope }}\label{parameters-scope}

\href{/docs/reference/foundations/dictionary/}{dictionary}

A scope of definitions that are made available.

Default:
\texttt{\ }{\texttt{\ (\ }}\texttt{\ }{\texttt{\ :\ }}\texttt{\ }{\texttt{\ )\ }}\texttt{\ }

\includesvg[width=0.16667in,height=0.16667in]{/assets/icons/16-arrow-right.svg}
View example

\begin{verbatim}
#eval("x + 1", scope: (x: 2)) \
#eval(
  "abc/xyz",
  mode: "math",
  scope: (
    abc: $a + b + c$,
    xyz: $x + y + z$,
  ),
)
\end{verbatim}

\includegraphics[width=5in,height=\textheight,keepaspectratio]{/assets/docs/0vD-OzSZwxX0Gqmm8_Sk9AAAAAAAAAAA.png}

\href{/docs/reference/foundations/duration/}{\pandocbounded{\includesvg[keepaspectratio]{/assets/icons/16-arrow-right.svg}}}

{ Duration } { Previous page }

\href{/docs/reference/foundations/float/}{\pandocbounded{\includesvg[keepaspectratio]{/assets/icons/16-arrow-right.svg}}}

{ Float } { Next page }


\section{Docs LaTeX/typst.app/docs/reference/foundations/bool.tex}
\title{typst.app/docs/reference/foundations/bool}

\begin{itemize}
\tightlist
\item
  \href{/docs}{\includesvg[width=0.16667in,height=0.16667in]{/assets/icons/16-docs-dark.svg}}
\item
  \includesvg[width=0.16667in,height=0.16667in]{/assets/icons/16-arrow-right.svg}
\item
  \href{/docs/reference/}{Reference}
\item
  \includesvg[width=0.16667in,height=0.16667in]{/assets/icons/16-arrow-right.svg}
\item
  \href{/docs/reference/foundations/}{Foundations}
\item
  \includesvg[width=0.16667in,height=0.16667in]{/assets/icons/16-arrow-right.svg}
\item
  \href{/docs/reference/foundations/bool/}{Boolean}
\end{itemize}

\section{\texorpdfstring{{ bool }}{ bool }}\label{summary}

A type with two states.

The boolean type has two values:
\texttt{\ }{\texttt{\ true\ }}\texttt{\ } and
\texttt{\ }{\texttt{\ false\ }}\texttt{\ } . It denotes whether
something is active or enabled.

\subsection{Example}\label{example}

\begin{verbatim}
#false \
#true \
#(1 < 2)
\end{verbatim}

\includegraphics[width=5in,height=\textheight,keepaspectratio]{/assets/docs/kY06WRyR--IwV2unWZl-NwAAAAAAAAAA.png}

\href{/docs/reference/foundations/auto/}{\pandocbounded{\includesvg[keepaspectratio]{/assets/icons/16-arrow-right.svg}}}

{ Auto } { Previous page }

\href{/docs/reference/foundations/bytes/}{\pandocbounded{\includesvg[keepaspectratio]{/assets/icons/16-arrow-right.svg}}}

{ Bytes } { Next page }


\section{Docs LaTeX/typst.app/docs/reference/foundations/repr.tex}
\title{typst.app/docs/reference/foundations/repr}

\begin{itemize}
\tightlist
\item
  \href{/docs}{\includesvg[width=0.16667in,height=0.16667in]{/assets/icons/16-docs-dark.svg}}
\item
  \includesvg[width=0.16667in,height=0.16667in]{/assets/icons/16-arrow-right.svg}
\item
  \href{/docs/reference/}{Reference}
\item
  \includesvg[width=0.16667in,height=0.16667in]{/assets/icons/16-arrow-right.svg}
\item
  \href{/docs/reference/foundations/}{Foundations}
\item
  \includesvg[width=0.16667in,height=0.16667in]{/assets/icons/16-arrow-right.svg}
\item
  \href{/docs/reference/foundations/repr/}{Representation}
\end{itemize}

\section{\texorpdfstring{\texttt{\ repr\ }}{ repr }}\label{summary}

Returns the string representation of a value.

When inserted into content, most values are displayed as this
representation in monospace with syntax-highlighting. The exceptions are
\texttt{\ }{\texttt{\ none\ }}\texttt{\ } , integers, floats, strings,
content, and functions.

\textbf{Note:} This function is for debugging purposes. Its output
should not be considered stable and may change at any time!

\subsection{Example}\label{example}

\begin{verbatim}
#none vs #repr(none) \
#"hello" vs #repr("hello") \
#(1, 2) vs #repr((1, 2)) \
#[*Hi*] vs #repr([*Hi*])
\end{verbatim}

\includegraphics[width=5in,height=\textheight,keepaspectratio]{/assets/docs/hOvQAQDTPr3WAVu4x8HkgwAAAAAAAAAA.png}

\subsection{\texorpdfstring{{ Parameters
}}{ Parameters }}\label{parameters}

\phantomsection\label{parameters-tooltip}
Parameters are the inputs to a function. They are specified in
parentheses after the function name.

{ repr } (

{ { any } }

) -\textgreater{} \href{/docs/reference/foundations/str/}{str}

\subsubsection{\texorpdfstring{\texttt{\ value\ }}{ value }}\label{parameters-value}

{ any }

{Required} {{ Positional }}

\phantomsection\label{parameters-value-positional-tooltip}
Positional parameters are specified in order, without names.

The value whose string representation to produce.

\href{/docs/reference/foundations/regex/}{\pandocbounded{\includesvg[keepaspectratio]{/assets/icons/16-arrow-right.svg}}}

{ Regex } { Previous page }

\href{/docs/reference/foundations/selector/}{\pandocbounded{\includesvg[keepaspectratio]{/assets/icons/16-arrow-right.svg}}}

{ Selector } { Next page }


\section{Docs LaTeX/typst.app/docs/reference/foundations/auto.tex}
\title{typst.app/docs/reference/foundations/auto}

\begin{itemize}
\tightlist
\item
  \href{/docs}{\includesvg[width=0.16667in,height=0.16667in]{/assets/icons/16-docs-dark.svg}}
\item
  \includesvg[width=0.16667in,height=0.16667in]{/assets/icons/16-arrow-right.svg}
\item
  \href{/docs/reference/}{Reference}
\item
  \includesvg[width=0.16667in,height=0.16667in]{/assets/icons/16-arrow-right.svg}
\item
  \href{/docs/reference/foundations/}{Foundations}
\item
  \includesvg[width=0.16667in,height=0.16667in]{/assets/icons/16-arrow-right.svg}
\item
  \href{/docs/reference/foundations/auto/}{Auto}
\end{itemize}

\section{\texorpdfstring{{ auto }}{ auto }}\label{summary}

A value that indicates a smart default.

The auto type has exactly one value:
\texttt{\ }{\texttt{\ auto\ }}\texttt{\ } .

Parameters that support the \texttt{\ }{\texttt{\ auto\ }}\texttt{\ }
value have some smart default or contextual behaviour. A good example is
the \href{/docs/reference/text/text/\#parameters-dir}{text direction}
parameter. Setting it to \texttt{\ }{\texttt{\ auto\ }}\texttt{\ } lets
Typst automatically determine the direction from the
\href{/docs/reference/text/text/\#parameters-lang}{text language} .

\href{/docs/reference/foundations/assert/}{\pandocbounded{\includesvg[keepaspectratio]{/assets/icons/16-arrow-right.svg}}}

{ Assert } { Previous page }

\href{/docs/reference/foundations/bool/}{\pandocbounded{\includesvg[keepaspectratio]{/assets/icons/16-arrow-right.svg}}}

{ Boolean } { Next page }


\section{Docs LaTeX/typst.app/docs/reference/foundations/datetime.tex}
\title{typst.app/docs/reference/foundations/datetime}

\begin{itemize}
\tightlist
\item
  \href{/docs}{\includesvg[width=0.16667in,height=0.16667in]{/assets/icons/16-docs-dark.svg}}
\item
  \includesvg[width=0.16667in,height=0.16667in]{/assets/icons/16-arrow-right.svg}
\item
  \href{/docs/reference/}{Reference}
\item
  \includesvg[width=0.16667in,height=0.16667in]{/assets/icons/16-arrow-right.svg}
\item
  \href{/docs/reference/foundations/}{Foundations}
\item
  \includesvg[width=0.16667in,height=0.16667in]{/assets/icons/16-arrow-right.svg}
\item
  \href{/docs/reference/foundations/datetime/}{Datetime}
\end{itemize}

\section{\texorpdfstring{{ datetime }}{ datetime }}\label{summary}

Represents a date, a time, or a combination of both.

Can be created by either specifying a custom datetime using this
type\textquotesingle s constructor function or getting the current date
with
\href{/docs/reference/foundations/datetime/\#definitions-today}{\texttt{\ datetime.today\ }}
.

\subsection{Example}\label{example}

\begin{verbatim}
#let date = datetime(
  year: 2020,
  month: 10,
  day: 4,
)

#date.display() \
#date.display(
  "y:[year repr:last_two]"
)

#let time = datetime(
  hour: 18,
  minute: 2,
  second: 23,
)

#time.display() \
#time.display(
  "h:[hour repr:12][period]"
)
\end{verbatim}

\includegraphics[width=5in,height=\textheight,keepaspectratio]{/assets/docs/aJRkqg11vpsxBq0NzqAo0gAAAAAAAAAA.png}

\subsection{Datetime and Duration}\label{datetime-and-duration}

You can get a \href{/docs/reference/foundations/duration/}{duration} by
subtracting two datetime:

\begin{verbatim}
#let first-of-march = datetime(day: 1, month: 3, year: 2024)
#let first-of-jan = datetime(day: 1, month: 1, year: 2024)
#let distance = first-of-march - first-of-jan
#distance.hours()
\end{verbatim}

\includegraphics[width=5in,height=\textheight,keepaspectratio]{/assets/docs/xJIPnvV5Iiw8osdkiAUb_AAAAAAAAAAA.png}

You can also add/subtract a datetime and a duration to retrieve a new,
offset datetime:

\begin{verbatim}
#let date = datetime(day: 1, month: 3, year: 2024)
#let two-days = duration(days: 2)
#let two-days-earlier = date - two-days
#let two-days-later = date + two-days

#date.display() \
#two-days-earlier.display() \
#two-days-later.display()
\end{verbatim}

\includegraphics[width=5in,height=\textheight,keepaspectratio]{/assets/docs/R-BPj6xQMFasAxM1n3h_iwAAAAAAAAAA.png}

\subsection{Format}\label{format}

You can specify a customized formatting using the
\href{/docs/reference/foundations/datetime/\#definitions-display}{\texttt{\ display\ }}
method. The format of a datetime is specified by providing
\emph{components} with a specified number of \emph{modifiers} . A
component represents a certain part of the datetime that you want to
display, and with the help of modifiers you can define how you want to
display that component. In order to display a component, you wrap the
name of the component in square brackets (e.g. \texttt{\ {[}year{]}\ }
will display the year). In order to add modifiers, you add a space after
the component name followed by the name of the modifier, a colon and the
value of the modifier (e.g. \texttt{\ {[}month\ repr:short{]}\ } will
display the short representation of the month).

The possible combination of components and their respective modifiers is
as follows:

\begin{itemize}
\tightlist
\item
  \texttt{\ year\ } : Displays the year of the datetime.

  \begin{itemize}
  \tightlist
  \item
    \texttt{\ padding\ } : Can be either \texttt{\ zero\ } ,
    \texttt{\ space\ } or \texttt{\ none\ } . Specifies how the year is
    padded.
  \item
    \texttt{\ repr\ } Can be either \texttt{\ full\ } in which case the
    full year is displayed or \texttt{\ last\_two\ } in which case only
    the last two digits are displayed.
  \item
    \texttt{\ sign\ } : Can be either \texttt{\ automatic\ } or
    \texttt{\ mandatory\ } . Specifies when the sign should be
    displayed.
  \end{itemize}
\item
  \texttt{\ month\ } : Displays the month of the datetime.

  \begin{itemize}
  \tightlist
  \item
    \texttt{\ padding\ } : Can be either \texttt{\ zero\ } ,
    \texttt{\ space\ } or \texttt{\ none\ } . Specifies how the month is
    padded.
  \item
    \texttt{\ repr\ } : Can be either \texttt{\ numerical\ } ,
    \texttt{\ long\ } or \texttt{\ short\ } . Specifies if the month
    should be displayed as a number or a word. Unfortunately, when
    choosing the word representation, it can currently only display the
    English version. In the future, it is planned to support
    localization.
  \end{itemize}
\item
  \texttt{\ day\ } : Displays the day of the datetime.

  \begin{itemize}
  \tightlist
  \item
    \texttt{\ padding\ } : Can be either \texttt{\ zero\ } ,
    \texttt{\ space\ } or \texttt{\ none\ } . Specifies how the day is
    padded.
  \end{itemize}
\item
  \texttt{\ week\_number\ } : Displays the week number of the datetime.

  \begin{itemize}
  \tightlist
  \item
    \texttt{\ padding\ } : Can be either \texttt{\ zero\ } ,
    \texttt{\ space\ } or \texttt{\ none\ } . Specifies how the week
    number is padded.
  \item
    \texttt{\ repr\ } : Can be either \texttt{\ ISO\ } ,
    \texttt{\ sunday\ } or \texttt{\ monday\ } . In the case of
    \texttt{\ ISO\ } , week numbers are between 1 and 53, while the
    other ones are between 0 and 53.
  \end{itemize}
\item
  \texttt{\ weekday\ } : Displays the weekday of the date.

  \begin{itemize}
  \tightlist
  \item
    \texttt{\ repr\ } Can be either \texttt{\ long\ } ,
    \texttt{\ short\ } , \texttt{\ sunday\ } or \texttt{\ monday\ } . In
    the case of \texttt{\ long\ } and \texttt{\ short\ } , the
    corresponding English name will be displayed (same as for the month,
    other languages are currently not supported). In the case of
    \texttt{\ sunday\ } and \texttt{\ monday\ } , the numerical value
    will be displayed (assuming Sunday and Monday as the first day of
    the week, respectively).
  \item
    \texttt{\ one\_indexed\ } : Can be either \texttt{\ true\ } or
    \texttt{\ false\ } . Defines whether the numerical representation of
    the week starts with 0 or 1.
  \end{itemize}
\item
  \texttt{\ hour\ } : Displays the hour of the date.

  \begin{itemize}
  \tightlist
  \item
    \texttt{\ padding\ } : Can be either \texttt{\ zero\ } ,
    \texttt{\ space\ } or \texttt{\ none\ } . Specifies how the hour is
    padded.
  \item
    \texttt{\ repr\ } : Can be either \texttt{\ 24\ } or \texttt{\ 12\ }
    . Changes whether the hour is displayed in the 24-hour or 12-hour
    format.
  \end{itemize}
\item
  \texttt{\ period\ } : The AM/PM part of the hour

  \begin{itemize}
  \tightlist
  \item
    \texttt{\ case\ } : Can be \texttt{\ lower\ } to display it in lower
    case and \texttt{\ upper\ } to display it in upper case.
  \end{itemize}
\item
  \texttt{\ minute\ } : Displays the minute of the date.

  \begin{itemize}
  \tightlist
  \item
    \texttt{\ padding\ } : Can be either \texttt{\ zero\ } ,
    \texttt{\ space\ } or \texttt{\ none\ } . Specifies how the minute
    is padded.
  \end{itemize}
\item
  \texttt{\ second\ } : Displays the second of the date.

  \begin{itemize}
  \tightlist
  \item
    \texttt{\ padding\ } : Can be either \texttt{\ zero\ } ,
    \texttt{\ space\ } or \texttt{\ none\ } . Specifies how the second
    is padded.
  \end{itemize}
\end{itemize}

Keep in mind that not always all components can be used. For example, if
you create a new datetime with
\texttt{\ }{\texttt{\ datetime\ }}\texttt{\ }{\texttt{\ (\ }}\texttt{\ year\ }{\texttt{\ :\ }}\texttt{\ }{\texttt{\ 2023\ }}\texttt{\ }{\texttt{\ ,\ }}\texttt{\ month\ }{\texttt{\ :\ }}\texttt{\ }{\texttt{\ 10\ }}\texttt{\ }{\texttt{\ ,\ }}\texttt{\ day\ }{\texttt{\ :\ }}\texttt{\ }{\texttt{\ 13\ }}\texttt{\ }{\texttt{\ )\ }}\texttt{\ }
, it will be stored as a plain date internally, meaning that you cannot
use components such as \texttt{\ hour\ } or \texttt{\ minute\ } , which
would only work on datetimes that have a specified time.

\subsection{\texorpdfstring{Constructor
{}}{Constructor }}\label{constructor}

\phantomsection\label{constructor-constructor-tooltip}
If a type has a constructor, you can call it like a function to create a
new value of the type.

Creates a new datetime.

You can specify the
\href{/docs/reference/foundations/datetime/}{datetime} using a year,
month, day, hour, minute, and second.

\emph{Note} : Depending on which components of the datetime you specify,
Typst will store it in one of the following three ways:

\begin{itemize}
\tightlist
\item
  If you specify year, month and day, Typst will store just a date.
\item
  If you specify hour, minute and second, Typst will store just a time.
\item
  If you specify all of year, month, day, hour, minute and second, Typst
  will store a full datetime.
\end{itemize}

Depending on how it is stored, the
\href{/docs/reference/foundations/datetime/\#definitions-display}{\texttt{\ display\ }}
method will choose a different formatting by default.

{ datetime } (

{ \hyperref[constructor-parameters-year]{year :}
\href{/docs/reference/foundations/int/}{int} , } {
\hyperref[constructor-parameters-month]{month :}
\href{/docs/reference/foundations/int/}{int} , } {
\hyperref[constructor-parameters-day]{day :}
\href{/docs/reference/foundations/int/}{int} , } {
\hyperref[constructor-parameters-hour]{hour :}
\href{/docs/reference/foundations/int/}{int} , } {
\hyperref[constructor-parameters-minute]{minute :}
\href{/docs/reference/foundations/int/}{int} , } {
\hyperref[constructor-parameters-second]{second :}
\href{/docs/reference/foundations/int/}{int} , }

) -\textgreater{} \href{/docs/reference/foundations/datetime/}{datetime}

\begin{verbatim}
#datetime(
  year: 2012,
  month: 8,
  day: 3,
).display()
\end{verbatim}

\includegraphics[width=5in,height=\textheight,keepaspectratio]{/assets/docs/6mpnNRypNysjXvXstSouiwAAAAAAAAAA.png}

\paragraph{\texorpdfstring{\texttt{\ year\ }}{ year }}\label{constructor-year}

\href{/docs/reference/foundations/int/}{int}

The year of the datetime.

\paragraph{\texorpdfstring{\texttt{\ month\ }}{ month }}\label{constructor-month}

\href{/docs/reference/foundations/int/}{int}

The month of the datetime.

\paragraph{\texorpdfstring{\texttt{\ day\ }}{ day }}\label{constructor-day}

\href{/docs/reference/foundations/int/}{int}

The day of the datetime.

\paragraph{\texorpdfstring{\texttt{\ hour\ }}{ hour }}\label{constructor-hour}

\href{/docs/reference/foundations/int/}{int}

The hour of the datetime.

\paragraph{\texorpdfstring{\texttt{\ minute\ }}{ minute }}\label{constructor-minute}

\href{/docs/reference/foundations/int/}{int}

The minute of the datetime.

\paragraph{\texorpdfstring{\texttt{\ second\ }}{ second }}\label{constructor-second}

\href{/docs/reference/foundations/int/}{int}

The second of the datetime.

\subsection{\texorpdfstring{{ Definitions
}}{ Definitions }}\label{definitions}

\phantomsection\label{definitions-tooltip}
Functions and types and can have associated definitions. These are
accessed by specifying the function or type, followed by a period, and
then the definition\textquotesingle s name.

\subsubsection{\texorpdfstring{\texttt{\ today\ }}{ today }}\label{definitions-today}

Returns the current date.

datetime { . } { today } (

{ \hyperref[definitions-today-parameters-offset]{offset :}
\href{/docs/reference/foundations/auto/}{auto}
\href{/docs/reference/foundations/int/}{int} }

) -\textgreater{} \href{/docs/reference/foundations/datetime/}{datetime}

\begin{verbatim}
Today's date is
#datetime.today().display().
\end{verbatim}

\includegraphics[width=5in,height=\textheight,keepaspectratio]{/assets/docs/SOSDKByfy_YbHbk7NejgOQAAAAAAAAAA.png}

\paragraph{\texorpdfstring{\texttt{\ offset\ }}{ offset }}\label{definitions-today-offset}

\href{/docs/reference/foundations/auto/}{auto} {or}
\href{/docs/reference/foundations/int/}{int}

An offset to apply to the current UTC date. If set to
\texttt{\ }{\texttt{\ auto\ }}\texttt{\ } , the offset will be the local
offset.

Default: \texttt{\ }{\texttt{\ auto\ }}\texttt{\ }

\subsubsection{\texorpdfstring{\texttt{\ display\ }}{ display }}\label{definitions-display}

Displays the datetime in a specified format.

Depending on whether you have defined just a date, a time or both, the
default format will be different. If you specified a date, it will be
\texttt{\ {[}year{]}-{[}month{]}-{[}day{]}\ } . If you specified a time,
it will be \texttt{\ {[}hour{]}:{[}minute{]}:{[}second{]}\ } . In the
case of a datetime, it will be
\texttt{\ {[}year{]}-{[}month{]}-{[}day{]}\ {[}hour{]}:{[}minute{]}:{[}second{]}\ }
.

See the \href{/docs/reference/foundations/datetime/\#format}{format
syntax} for more information.

self { . } { display } (

{ \href{/docs/reference/foundations/auto/}{auto}
\href{/docs/reference/foundations/str/}{str} }

) -\textgreater{} \href{/docs/reference/foundations/str/}{str}

\paragraph{\texorpdfstring{\texttt{\ pattern\ }}{ pattern }}\label{definitions-display-pattern}

\href{/docs/reference/foundations/auto/}{auto} {or}
\href{/docs/reference/foundations/str/}{str}

{{ Positional }}

\phantomsection\label{definitions-display-pattern-positional-tooltip}
Positional parameters are specified in order, without names.

The format used to display the datetime.

Default: \texttt{\ }{\texttt{\ auto\ }}\texttt{\ }

\subsubsection{\texorpdfstring{\texttt{\ year\ }}{ year }}\label{definitions-year}

The year if it was specified, or
\texttt{\ }{\texttt{\ none\ }}\texttt{\ } for times without a date.

self { . } { year } (

) -\textgreater{} \href{/docs/reference/foundations/none/}{none}
\href{/docs/reference/foundations/int/}{int}

\subsubsection{\texorpdfstring{\texttt{\ month\ }}{ month }}\label{definitions-month}

The month if it was specified, or
\texttt{\ }{\texttt{\ none\ }}\texttt{\ } for times without a date.

self { . } { month } (

) -\textgreater{} \href{/docs/reference/foundations/none/}{none}
\href{/docs/reference/foundations/int/}{int}

\subsubsection{\texorpdfstring{\texttt{\ weekday\ }}{ weekday }}\label{definitions-weekday}

The weekday (counting Monday as 1) or
\texttt{\ }{\texttt{\ none\ }}\texttt{\ } for times without a date.

self { . } { weekday } (

) -\textgreater{} \href{/docs/reference/foundations/none/}{none}
\href{/docs/reference/foundations/int/}{int}

\subsubsection{\texorpdfstring{\texttt{\ day\ }}{ day }}\label{definitions-day}

The day if it was specified, or
\texttt{\ }{\texttt{\ none\ }}\texttt{\ } for times without a date.

self { . } { day } (

) -\textgreater{} \href{/docs/reference/foundations/none/}{none}
\href{/docs/reference/foundations/int/}{int}

\subsubsection{\texorpdfstring{\texttt{\ hour\ }}{ hour }}\label{definitions-hour}

The hour if it was specified, or
\texttt{\ }{\texttt{\ none\ }}\texttt{\ } for dates without a time.

self { . } { hour } (

) -\textgreater{} \href{/docs/reference/foundations/none/}{none}
\href{/docs/reference/foundations/int/}{int}

\subsubsection{\texorpdfstring{\texttt{\ minute\ }}{ minute }}\label{definitions-minute}

The minute if it was specified, or
\texttt{\ }{\texttt{\ none\ }}\texttt{\ } for dates without a time.

self { . } { minute } (

) -\textgreater{} \href{/docs/reference/foundations/none/}{none}
\href{/docs/reference/foundations/int/}{int}

\subsubsection{\texorpdfstring{\texttt{\ second\ }}{ second }}\label{definitions-second}

The second if it was specified, or
\texttt{\ }{\texttt{\ none\ }}\texttt{\ } for dates without a time.

self { . } { second } (

) -\textgreater{} \href{/docs/reference/foundations/none/}{none}
\href{/docs/reference/foundations/int/}{int}

\subsubsection{\texorpdfstring{\texttt{\ ordinal\ }}{ ordinal }}\label{definitions-ordinal}

The ordinal (day of the year), or
\texttt{\ }{\texttt{\ none\ }}\texttt{\ } for times without a date.

self { . } { ordinal } (

) -\textgreater{} \href{/docs/reference/foundations/none/}{none}
\href{/docs/reference/foundations/int/}{int}

\href{/docs/reference/foundations/content/}{\pandocbounded{\includesvg[keepaspectratio]{/assets/icons/16-arrow-right.svg}}}

{ Content } { Previous page }

\href{/docs/reference/foundations/decimal/}{\pandocbounded{\includesvg[keepaspectratio]{/assets/icons/16-arrow-right.svg}}}

{ Decimal } { Next page }


\section{Docs LaTeX/typst.app/docs/reference/foundations/sys.tex}
\title{typst.app/docs/reference/foundations/sys}

\begin{itemize}
\tightlist
\item
  \href{/docs}{\includesvg[width=0.16667in,height=0.16667in]{/assets/icons/16-docs-dark.svg}}
\item
  \includesvg[width=0.16667in,height=0.16667in]{/assets/icons/16-arrow-right.svg}
\item
  \href{/docs/reference/}{Reference}
\item
  \includesvg[width=0.16667in,height=0.16667in]{/assets/icons/16-arrow-right.svg}
\item
  \href{/docs/reference/foundations/}{Foundations}
\item
  \includesvg[width=0.16667in,height=0.16667in]{/assets/icons/16-arrow-right.svg}
\item
  \href{/docs/reference/foundations/sys}{System}
\end{itemize}

\section{System}\label{summary}

Module for system interactions.

This module defines the following items:

\begin{itemize}
\item
  The \texttt{\ sys.version\ } constant (of type
  \href{/docs/reference/foundations/version/}{\texttt{\ version\ }} )
  that specifies the currently active Typst compiler version.
\item
  The \texttt{\ sys.inputs\ }
  \href{/docs/reference/foundations/dictionary/}{dictionary} , which
  makes external inputs available to the project. An input specified in
  the command line as \texttt{\ -\/-input\ key=value\ } becomes
  available under \texttt{\ sys.inputs.key\ } as
  \texttt{\ }{\texttt{\ "value"\ }}\texttt{\ } . To include spaces in
  the value, it may be enclosed with single or double quotes.

  The value is always of type
  \href{/docs/reference/foundations/str/}{string} . More complex data
  may be parsed manually using functions like
  \href{/docs/reference/data-loading/json/\#definitions-decode}{\texttt{\ json.decode\ }}
  .
\end{itemize}

\href{/docs/reference/foundations/style/}{\pandocbounded{\includesvg[keepaspectratio]{/assets/icons/16-arrow-right.svg}}}

{ Style } { Previous page }

\href{/docs/reference/foundations/type/}{\pandocbounded{\includesvg[keepaspectratio]{/assets/icons/16-arrow-right.svg}}}

{ Type } { Next page }


\section{Docs LaTeX/typst.app/docs/reference/foundations/style.tex}
\title{typst.app/docs/reference/foundations/style}

\begin{itemize}
\tightlist
\item
  \href{/docs}{\includesvg[width=0.16667in,height=0.16667in]{/assets/icons/16-docs-dark.svg}}
\item
  \includesvg[width=0.16667in,height=0.16667in]{/assets/icons/16-arrow-right.svg}
\item
  \href{/docs/reference/}{Reference}
\item
  \includesvg[width=0.16667in,height=0.16667in]{/assets/icons/16-arrow-right.svg}
\item
  \href{/docs/reference/foundations/}{Foundations}
\item
  \includesvg[width=0.16667in,height=0.16667in]{/assets/icons/16-arrow-right.svg}
\item
  \href{/docs/reference/foundations/style/}{Style}
\end{itemize}

\section{\texorpdfstring{\texttt{\ style\ }}{ style }}\label{summary}

Provides access to active styles.

\textbf{Deprecation planned.} Use
\href{/docs/reference/context/}{context} instead.

\begin{verbatim}
#let thing(body) = style(styles => {
  let size = measure(body, styles)
  [Width of "#body" is #size.width]
})

#thing[Hey] \
#thing[Welcome]
\end{verbatim}

\includegraphics[width=5in,height=\textheight,keepaspectratio]{/assets/docs/B9BBPtfgbYihKwTWFPTllQAAAAAAAAAA.png}

\subsection{\texorpdfstring{{ Parameters
}}{ Parameters }}\label{parameters}

\phantomsection\label{parameters-tooltip}
Parameters are the inputs to a function. They are specified in
parentheses after the function name.

{ style } (

{ \href{/docs/reference/foundations/function/}{function} }

) -\textgreater{} \href{/docs/reference/foundations/content/}{content}

\subsubsection{\texorpdfstring{\texttt{\ func\ }}{ func }}\label{parameters-func}

\href{/docs/reference/foundations/function/}{function}

{Required} {{ Positional }}

\phantomsection\label{parameters-func-positional-tooltip}
Positional parameters are specified in order, without names.

A function to call with the styles. Its return value is displayed in the
document.

This function is called once for each time the content returned by
\texttt{\ style\ } appears in the document. That makes it possible to
generate content that depends on the style context it appears in.

\href{/docs/reference/foundations/str/}{\pandocbounded{\includesvg[keepaspectratio]{/assets/icons/16-arrow-right.svg}}}

{ String } { Previous page }

\href{/docs/reference/foundations/sys/}{\pandocbounded{\includesvg[keepaspectratio]{/assets/icons/16-arrow-right.svg}}}

{ System } { Next page }


\section{Docs LaTeX/typst.app/docs/reference/foundations/decimal.tex}
\title{typst.app/docs/reference/foundations/decimal}

\begin{itemize}
\tightlist
\item
  \href{/docs}{\includesvg[width=0.16667in,height=0.16667in]{/assets/icons/16-docs-dark.svg}}
\item
  \includesvg[width=0.16667in,height=0.16667in]{/assets/icons/16-arrow-right.svg}
\item
  \href{/docs/reference/}{Reference}
\item
  \includesvg[width=0.16667in,height=0.16667in]{/assets/icons/16-arrow-right.svg}
\item
  \href{/docs/reference/foundations/}{Foundations}
\item
  \includesvg[width=0.16667in,height=0.16667in]{/assets/icons/16-arrow-right.svg}
\item
  \href{/docs/reference/foundations/decimal/}{Decimal}
\end{itemize}

\section{\texorpdfstring{{ decimal }}{ decimal }}\label{summary}

A fixed-point decimal number type.

This type should be used for precise arithmetic operations on numbers
represented in base 10. A typical use case is representing currency.

\subsection{Example}\label{example}

\begin{verbatim}
Decimal: #(decimal("0.1") + decimal("0.2")) \
Float: #(0.1 + 0.2)
\end{verbatim}

\includegraphics[width=5in,height=\textheight,keepaspectratio]{/assets/docs/W31Kvh6BvfIgTgIeq2uIEQAAAAAAAAAA.png}

\subsection{Construction and casts}\label{construction-and-casts}

To create a decimal number, use the
\texttt{\ }{\texttt{\ decimal\ }}\texttt{\ }{\texttt{\ (\ }}\texttt{\ string\ }{\texttt{\ )\ }}\texttt{\ }
constructor, such as in
\texttt{\ }{\texttt{\ decimal\ }}\texttt{\ }{\texttt{\ (\ }}\texttt{\ }{\texttt{\ "3.141592653"\ }}\texttt{\ }{\texttt{\ )\ }}\texttt{\ }
\textbf{(note the double quotes!)} . This constructor preserves all
given fractional digits, provided they are representable as per the
limits specified below (otherwise, an error is raised).

You can also convert any
\href{/docs/reference/foundations/int/}{integer} to a decimal with the
\texttt{\ }{\texttt{\ decimal\ }}\texttt{\ }{\texttt{\ (\ }}\texttt{\ int\ }{\texttt{\ )\ }}\texttt{\ }
constructor, e.g.
\texttt{\ }{\texttt{\ decimal\ }}\texttt{\ }{\texttt{\ (\ }}\texttt{\ }{\texttt{\ 59\ }}\texttt{\ }{\texttt{\ )\ }}\texttt{\ }
. However, note that constructing a decimal from a
\href{/docs/reference/foundations/float/}{floating-point number} , while
supported, \textbf{is an imprecise conversion and therefore
discouraged.} A warning will be raised if Typst detects that there was
an accidental \texttt{\ float\ } to \texttt{\ decimal\ } cast through
its constructor, e.g. if writing
\texttt{\ }{\texttt{\ decimal\ }}\texttt{\ }{\texttt{\ (\ }}\texttt{\ }{\texttt{\ 3.14\ }}\texttt{\ }{\texttt{\ )\ }}\texttt{\ }
(note the lack of double quotes, indicating this is an accidental
\texttt{\ float\ } cast and therefore imprecise). It is recommended to
use strings for constant decimal values instead (e.g.
\texttt{\ }{\texttt{\ decimal\ }}\texttt{\ }{\texttt{\ (\ }}\texttt{\ }{\texttt{\ "3.14"\ }}\texttt{\ }{\texttt{\ )\ }}\texttt{\ }
).

The precision of a \texttt{\ float\ } to \texttt{\ decimal\ } cast can
be slightly improved by rounding the result to 15 digits with
\href{/docs/reference/foundations/calc/\#functions-round}{\texttt{\ calc.round\ }}
, but there are still no precision guarantees for that kind of
conversion.

\subsection{Operations}\label{operations}

Basic arithmetic operations are supported on two decimals and on pairs
of decimals and integers.

Built-in operations between \texttt{\ float\ } and \texttt{\ decimal\ }
are not supported in order to guard against accidental loss of
precision. They will raise an error instead.

Certain \texttt{\ calc\ } functions, such as trigonometric functions and
power between two real numbers, are also only supported for
\texttt{\ float\ } (although raising \texttt{\ decimal\ } to integer
exponents is supported). You can opt into potentially imprecise
operations with the
\texttt{\ }{\texttt{\ float\ }}\texttt{\ }{\texttt{\ (\ }}\texttt{\ decimal\ }{\texttt{\ )\ }}\texttt{\ }
constructor, which casts the \texttt{\ decimal\ } number into a
\texttt{\ float\ } , allowing for operations without precision
guarantees.

\subsection{Displaying decimals}\label{displaying-decimals}

To display a decimal, simply insert the value into the document. To only
display a certain number of digits,
\href{/docs/reference/foundations/calc/\#functions-round}{round} the
decimal first. Localized formatting of decimals and other numbers is not
yet supported, but planned for the future.

You can convert decimals to strings using the
\href{/docs/reference/foundations/str/}{\texttt{\ str\ }} constructor.
This way, you can post-process the displayed representation, e.g. to
replace the period with a comma (as a stand-in for proper built-in
localization to languages that use the comma).

\subsection{Precision and limits}\label{precision-and-limits}

A \texttt{\ decimal\ } number has a limit of 28 to 29 significant
base-10 digits. This includes the sum of digits before and after the
decimal point. As such, numbers with more fractional digits have a
smaller range. The maximum and minimum \texttt{\ decimal\ } numbers have
a value of
\texttt{\ }{\texttt{\ 79228162514264337593543950335\ }}\texttt{\ } and
\texttt{\ }{\texttt{\ -\ }}\texttt{\ }{\texttt{\ 79228162514264337593543950335\ }}\texttt{\ }
respectively. In contrast with
\href{/docs/reference/foundations/float/}{\texttt{\ float\ }} , this
type does not support infinity or NaN, so overflowing or underflowing
operations will raise an error.

Typical operations between \texttt{\ decimal\ } numbers, such as
addition, multiplication, and
\href{/docs/reference/foundations/calc/\#functions-pow}{power} to an
integer, will be highly precise due to their fixed-point representation.
Note, however, that multiplication and division may not preserve all
digits in some edge cases: while they are considered precise, digits
past the limits specified above are rounded off and lost, so some loss
of precision beyond the maximum representable digits is possible. Note
that this behavior can be observed not only when dividing, but also when
multiplying by numbers between 0 and 1, as both operations can push a
number\textquotesingle s fractional digits beyond the limits described
above, leading to rounding. When those two operations do not surpass the
digit limits, they are fully precise.

\subsection{\texorpdfstring{Constructor
{}}{Constructor }}\label{constructor}

\phantomsection\label{constructor-constructor-tooltip}
If a type has a constructor, you can call it like a function to create a
new value of the type.

Converts a value to a \texttt{\ decimal\ } .

It is recommended to use a string to construct the decimal number, or an
\href{/docs/reference/foundations/int/}{integer} (if desired). The
string must contain a number in the format
\texttt{\ }{\texttt{\ "3.14159"\ }}\texttt{\ } (or
\texttt{\ }{\texttt{\ "-3.141519"\ }}\texttt{\ } for negative numbers).
The fractional digits are fully preserved; if that\textquotesingle s not
possible due to the limit of significant digits (around 28 to 29) having
been reached, an error is raised as the given decimal number
wouldn\textquotesingle t be representable.

While this constructor can be used with
\href{/docs/reference/foundations/float/}{floating-point numbers} to
cast them to \texttt{\ decimal\ } , doing so is \textbf{discouraged} as
\textbf{this cast is inherently imprecise.} It is easy to accidentally
perform this cast by writing
\texttt{\ }{\texttt{\ decimal\ }}\texttt{\ }{\texttt{\ (\ }}\texttt{\ }{\texttt{\ 1.234\ }}\texttt{\ }{\texttt{\ )\ }}\texttt{\ }
(note the lack of double quotes), which is why Typst will emit a warning
in that case. Please write
\texttt{\ }{\texttt{\ decimal\ }}\texttt{\ }{\texttt{\ (\ }}\texttt{\ }{\texttt{\ "1.234"\ }}\texttt{\ }{\texttt{\ )\ }}\texttt{\ }
instead for that particular case (initialization of a constant decimal).
Also note that floats that are NaN or infinite cannot be cast to
decimals and will raise an error.

{ decimal } (

{ \href{/docs/reference/foundations/int/}{int}
\href{/docs/reference/foundations/float/}{float}
\href{/docs/reference/foundations/str/}{str} }

) -\textgreater{} \href{/docs/reference/foundations/decimal/}{decimal}

\begin{verbatim}
#decimal("1.222222222222222")
\end{verbatim}

\includegraphics[width=5in,height=\textheight,keepaspectratio]{/assets/docs/RfqlB85Q5lIVeebJq7RlmgAAAAAAAAAA.png}

\paragraph{\texorpdfstring{\texttt{\ value\ }}{ value }}\label{constructor-value}

\href{/docs/reference/foundations/int/}{int} {or}
\href{/docs/reference/foundations/float/}{float} {or}
\href{/docs/reference/foundations/str/}{str}

{Required} {{ Positional }}

\phantomsection\label{constructor-value-positional-tooltip}
Positional parameters are specified in order, without names.

The value that should be converted to a decimal.

\href{/docs/reference/foundations/datetime/}{\pandocbounded{\includesvg[keepaspectratio]{/assets/icons/16-arrow-right.svg}}}

{ Datetime } { Previous page }

\href{/docs/reference/foundations/dictionary/}{\pandocbounded{\includesvg[keepaspectratio]{/assets/icons/16-arrow-right.svg}}}

{ Dictionary } { Next page }


\section{Docs LaTeX/typst.app/docs/reference/foundations/arguments.tex}
\title{typst.app/docs/reference/foundations/arguments}

\begin{itemize}
\tightlist
\item
  \href{/docs}{\includesvg[width=0.16667in,height=0.16667in]{/assets/icons/16-docs-dark.svg}}
\item
  \includesvg[width=0.16667in,height=0.16667in]{/assets/icons/16-arrow-right.svg}
\item
  \href{/docs/reference/}{Reference}
\item
  \includesvg[width=0.16667in,height=0.16667in]{/assets/icons/16-arrow-right.svg}
\item
  \href{/docs/reference/foundations/}{Foundations}
\item
  \includesvg[width=0.16667in,height=0.16667in]{/assets/icons/16-arrow-right.svg}
\item
  \href{/docs/reference/foundations/arguments/}{Arguments}
\end{itemize}

\section{\texorpdfstring{{ arguments }}{ arguments }}\label{summary}

Captured arguments to a function.

\subsection{Argument Sinks}\label{argument-sinks}

Like built-in functions, custom functions can also take a variable
number of arguments. You can specify an \emph{argument sink} which
collects all excess arguments as \texttt{\ ..sink\ } . The resulting
\texttt{\ sink\ } value is of the \texttt{\ arguments\ } type. It
exposes methods to access the positional and named arguments.

\begin{verbatim}
#let format(title, ..authors) = {
  let by = authors
    .pos()
    .join(", ", last: " and ")

  [*#title* \ _Written by #by;_]
}

#format("ArtosFlow", "Jane", "Joe")
\end{verbatim}

\includegraphics[width=5in,height=\textheight,keepaspectratio]{/assets/docs/DWzn69gGuCd1q_LVZvjEEgAAAAAAAAAA.png}

\subsection{Spreading}\label{spreading}

Inversely to an argument sink, you can \emph{spread} arguments, arrays
and dictionaries into a function call with the \texttt{\ ..spread\ }
operator:

\begin{verbatim}
#let array = (2, 3, 5)
#calc.min(..array)
#let dict = (fill: blue)
#text(..dict)[Hello]
\end{verbatim}

\includegraphics[width=5in,height=\textheight,keepaspectratio]{/assets/docs/kcmqtH9qxq6Bg8ZwwKnMCQAAAAAAAAAA.png}

\subsection{\texorpdfstring{Constructor
{}}{Constructor }}\label{constructor}

\phantomsection\label{constructor-constructor-tooltip}
If a type has a constructor, you can call it like a function to create a
new value of the type.

Construct spreadable arguments in place.

This function behaves like
\texttt{\ }{\texttt{\ let\ }}\texttt{\ }{\texttt{\ args\ }}\texttt{\ }{\texttt{\ (\ }}\texttt{\ }{\texttt{\ ..\ }}\texttt{\ sink\ }{\texttt{\ )\ }}\texttt{\ }{\texttt{\ =\ }}\texttt{\ sink\ }
.

{ arguments } (

{ \hyperref[constructor-parameters-arguments]{..} { any } }

) -\textgreater{}
\href{/docs/reference/foundations/arguments/}{arguments}

\begin{verbatim}
#let args = arguments(stroke: red, inset: 1em, [Body])
#box(..args)
\end{verbatim}

\includegraphics[width=5in,height=\textheight,keepaspectratio]{/assets/docs/JbzK099-rqq0pkW-oHCQsgAAAAAAAAAA.png}

\paragraph{\texorpdfstring{\texttt{\ arguments\ }}{ arguments }}\label{constructor-arguments}

{ any }

{Required} {{ Positional }}

\phantomsection\label{constructor-arguments-positional-tooltip}
Positional parameters are specified in order, without names.

{{ Variadic }}

\phantomsection\label{constructor-arguments-variadic-tooltip}
Variadic parameters can be specified multiple times.

The arguments to construct.

\subsection{\texorpdfstring{{ Definitions
}}{ Definitions }}\label{definitions}

\phantomsection\label{definitions-tooltip}
Functions and types and can have associated definitions. These are
accessed by specifying the function or type, followed by a period, and
then the definition\textquotesingle s name.

\subsubsection{\texorpdfstring{\texttt{\ at\ }}{ at }}\label{definitions-at}

Returns the positional argument at the specified index, or the named
argument with the specified name.

If the key is an \href{/docs/reference/foundations/int/}{integer} , this
is equivalent to first calling
\href{/docs/reference/foundations/arguments/\#definitions-pos}{\texttt{\ pos\ }}
and then
\href{/docs/reference/foundations/array/\#definitions-at}{\texttt{\ array.at\ }}
. If it is a \href{/docs/reference/foundations/str/}{string} , this is
equivalent to first calling
\href{/docs/reference/foundations/arguments/\#definitions-named}{\texttt{\ named\ }}
and then
\href{/docs/reference/foundations/dictionary/\#definitions-at}{\texttt{\ dictionary.at\ }}
.

self { . } { at } (

{ \href{/docs/reference/foundations/int/}{int}
\href{/docs/reference/foundations/str/}{str} , } {
\hyperref[definitions-at-parameters-default]{default :} { any } , }

) -\textgreater{} { any }

\paragraph{\texorpdfstring{\texttt{\ key\ }}{ key }}\label{definitions-at-key}

\href{/docs/reference/foundations/int/}{int} {or}
\href{/docs/reference/foundations/str/}{str}

{Required} {{ Positional }}

\phantomsection\label{definitions-at-key-positional-tooltip}
Positional parameters are specified in order, without names.

The index or name of the argument to get.

\paragraph{\texorpdfstring{\texttt{\ default\ }}{ default }}\label{definitions-at-default}

{ any }

A default value to return if the key is invalid.

\subsubsection{\texorpdfstring{\texttt{\ pos\ }}{ pos }}\label{definitions-pos}

Returns the captured positional arguments as an array.

self { . } { pos } (

) -\textgreater{} \href{/docs/reference/foundations/array/}{array}

\subsubsection{\texorpdfstring{\texttt{\ named\ }}{ named }}\label{definitions-named}

Returns the captured named arguments as a dictionary.

self { . } { named } (

) -\textgreater{}
\href{/docs/reference/foundations/dictionary/}{dictionary}

\href{/docs/reference/foundations/}{\pandocbounded{\includesvg[keepaspectratio]{/assets/icons/16-arrow-right.svg}}}

{ Foundations } { Previous page }

\href{/docs/reference/foundations/array/}{\pandocbounded{\includesvg[keepaspectratio]{/assets/icons/16-arrow-right.svg}}}

{ Array } { Next page }


\section{Docs LaTeX/typst.app/docs/reference/foundations/duration.tex}
\title{typst.app/docs/reference/foundations/duration}

\begin{itemize}
\tightlist
\item
  \href{/docs}{\includesvg[width=0.16667in,height=0.16667in]{/assets/icons/16-docs-dark.svg}}
\item
  \includesvg[width=0.16667in,height=0.16667in]{/assets/icons/16-arrow-right.svg}
\item
  \href{/docs/reference/}{Reference}
\item
  \includesvg[width=0.16667in,height=0.16667in]{/assets/icons/16-arrow-right.svg}
\item
  \href{/docs/reference/foundations/}{Foundations}
\item
  \includesvg[width=0.16667in,height=0.16667in]{/assets/icons/16-arrow-right.svg}
\item
  \href{/docs/reference/foundations/duration/}{Duration}
\end{itemize}

\section{\texorpdfstring{{ duration }}{ duration }}\label{summary}

Represents a positive or negative span of time.

\subsection{\texorpdfstring{Constructor
{}}{Constructor }}\label{constructor}

\phantomsection\label{constructor-constructor-tooltip}
If a type has a constructor, you can call it like a function to create a
new value of the type.

Creates a new duration.

You can specify the
\href{/docs/reference/foundations/duration/}{duration} using weeks,
days, hours, minutes and seconds. You can also get a duration by
subtracting two \href{/docs/reference/foundations/datetime/}{datetimes}
.

{ duration } (

{ \hyperref[constructor-parameters-seconds]{seconds :}
\href{/docs/reference/foundations/int/}{int} , } {
\hyperref[constructor-parameters-minutes]{minutes :}
\href{/docs/reference/foundations/int/}{int} , } {
\hyperref[constructor-parameters-hours]{hours :}
\href{/docs/reference/foundations/int/}{int} , } {
\hyperref[constructor-parameters-days]{days :}
\href{/docs/reference/foundations/int/}{int} , } {
\hyperref[constructor-parameters-weeks]{weeks :}
\href{/docs/reference/foundations/int/}{int} , }

) -\textgreater{} \href{/docs/reference/foundations/duration/}{duration}

\begin{verbatim}
#duration(
  days: 3,
  hours: 12,
).hours()
\end{verbatim}

\includegraphics[width=5in,height=\textheight,keepaspectratio]{/assets/docs/GmG9JKsQZEqcWXCc52iIiQAAAAAAAAAA.png}

\paragraph{\texorpdfstring{\texttt{\ seconds\ }}{ seconds }}\label{constructor-seconds}

\href{/docs/reference/foundations/int/}{int}

The number of seconds.

Default: \texttt{\ }{\texttt{\ 0\ }}\texttt{\ }

\paragraph{\texorpdfstring{\texttt{\ minutes\ }}{ minutes }}\label{constructor-minutes}

\href{/docs/reference/foundations/int/}{int}

The number of minutes.

Default: \texttt{\ }{\texttt{\ 0\ }}\texttt{\ }

\paragraph{\texorpdfstring{\texttt{\ hours\ }}{ hours }}\label{constructor-hours}

\href{/docs/reference/foundations/int/}{int}

The number of hours.

Default: \texttt{\ }{\texttt{\ 0\ }}\texttt{\ }

\paragraph{\texorpdfstring{\texttt{\ days\ }}{ days }}\label{constructor-days}

\href{/docs/reference/foundations/int/}{int}

The number of days.

Default: \texttt{\ }{\texttt{\ 0\ }}\texttt{\ }

\paragraph{\texorpdfstring{\texttt{\ weeks\ }}{ weeks }}\label{constructor-weeks}

\href{/docs/reference/foundations/int/}{int}

The number of weeks.

Default: \texttt{\ }{\texttt{\ 0\ }}\texttt{\ }

\subsection{\texorpdfstring{{ Definitions
}}{ Definitions }}\label{definitions}

\phantomsection\label{definitions-tooltip}
Functions and types and can have associated definitions. These are
accessed by specifying the function or type, followed by a period, and
then the definition\textquotesingle s name.

\subsubsection{\texorpdfstring{\texttt{\ seconds\ }}{ seconds }}\label{definitions-seconds}

The duration expressed in seconds.

This function returns the total duration represented in seconds as a
floating-point number rather than the second component of the duration.

self { . } { seconds } (

) -\textgreater{} \href{/docs/reference/foundations/float/}{float}

\subsubsection{\texorpdfstring{\texttt{\ minutes\ }}{ minutes }}\label{definitions-minutes}

The duration expressed in minutes.

This function returns the total duration represented in minutes as a
floating-point number rather than the second component of the duration.

self { . } { minutes } (

) -\textgreater{} \href{/docs/reference/foundations/float/}{float}

\subsubsection{\texorpdfstring{\texttt{\ hours\ }}{ hours }}\label{definitions-hours}

The duration expressed in hours.

This function returns the total duration represented in hours as a
floating-point number rather than the second component of the duration.

self { . } { hours } (

) -\textgreater{} \href{/docs/reference/foundations/float/}{float}

\subsubsection{\texorpdfstring{\texttt{\ days\ }}{ days }}\label{definitions-days}

The duration expressed in days.

This function returns the total duration represented in days as a
floating-point number rather than the second component of the duration.

self { . } { days } (

) -\textgreater{} \href{/docs/reference/foundations/float/}{float}

\subsubsection{\texorpdfstring{\texttt{\ weeks\ }}{ weeks }}\label{definitions-weeks}

The duration expressed in weeks.

This function returns the total duration represented in weeks as a
floating-point number rather than the second component of the duration.

self { . } { weeks } (

) -\textgreater{} \href{/docs/reference/foundations/float/}{float}

\href{/docs/reference/foundations/dictionary/}{\pandocbounded{\includesvg[keepaspectratio]{/assets/icons/16-arrow-right.svg}}}

{ Dictionary } { Previous page }

\href{/docs/reference/foundations/eval/}{\pandocbounded{\includesvg[keepaspectratio]{/assets/icons/16-arrow-right.svg}}}

{ Evaluate } { Next page }


\section{Docs LaTeX/typst.app/docs/reference/foundations/none.tex}
\title{typst.app/docs/reference/foundations/none}

\begin{itemize}
\tightlist
\item
  \href{/docs}{\includesvg[width=0.16667in,height=0.16667in]{/assets/icons/16-docs-dark.svg}}
\item
  \includesvg[width=0.16667in,height=0.16667in]{/assets/icons/16-arrow-right.svg}
\item
  \href{/docs/reference/}{Reference}
\item
  \includesvg[width=0.16667in,height=0.16667in]{/assets/icons/16-arrow-right.svg}
\item
  \href{/docs/reference/foundations/}{Foundations}
\item
  \includesvg[width=0.16667in,height=0.16667in]{/assets/icons/16-arrow-right.svg}
\item
  \href{/docs/reference/foundations/none/}{None}
\end{itemize}

\section{\texorpdfstring{{ none }}{ none }}\label{summary}

A value that indicates the absence of any other value.

The none type has exactly one value:
\texttt{\ }{\texttt{\ none\ }}\texttt{\ } .

When inserted into the document, it is not visible. This is also the
value that is produced by empty code blocks. It can be
\href{/docs/reference/scripting/\#blocks}{joined} with any value,
yielding the other value.

\subsection{Example}\label{example}

\begin{verbatim}
Not visible: #none
\end{verbatim}

\includegraphics[width=5in,height=\textheight,keepaspectratio]{/assets/docs/bWChCwjCUgpluIjZfBh2dgAAAAAAAAAA.png}

\href{/docs/reference/foundations/module/}{\pandocbounded{\includesvg[keepaspectratio]{/assets/icons/16-arrow-right.svg}}}

{ Module } { Previous page }

\href{/docs/reference/foundations/panic/}{\pandocbounded{\includesvg[keepaspectratio]{/assets/icons/16-arrow-right.svg}}}

{ Panic } { Next page }


\section{Docs LaTeX/typst.app/docs/reference/foundations/int.tex}
\title{typst.app/docs/reference/foundations/int}

\begin{itemize}
\tightlist
\item
  \href{/docs}{\includesvg[width=0.16667in,height=0.16667in]{/assets/icons/16-docs-dark.svg}}
\item
  \includesvg[width=0.16667in,height=0.16667in]{/assets/icons/16-arrow-right.svg}
\item
  \href{/docs/reference/}{Reference}
\item
  \includesvg[width=0.16667in,height=0.16667in]{/assets/icons/16-arrow-right.svg}
\item
  \href{/docs/reference/foundations/}{Foundations}
\item
  \includesvg[width=0.16667in,height=0.16667in]{/assets/icons/16-arrow-right.svg}
\item
  \href{/docs/reference/foundations/int/}{Integer}
\end{itemize}

\section{\texorpdfstring{{ int }}{ int }}\label{summary}

A whole number.

The number can be negative, zero, or positive. As Typst uses 64 bits to
store integers, integers cannot be smaller than
\texttt{\ }{\texttt{\ -\ }}\texttt{\ }{\texttt{\ 9223372036854775808\ }}\texttt{\ }
or larger than \texttt{\ }{\texttt{\ 9223372036854775807\ }}\texttt{\ }
.

The number can also be specified as hexadecimal, octal, or binary by
starting it with a zero followed by either \texttt{\ x\ } ,
\texttt{\ o\ } , or \texttt{\ b\ } .

You can convert a value to an integer with this type\textquotesingle s
constructor.

\subsection{Example}\label{example}

\begin{verbatim}
#(1 + 2) \
#(2 - 5) \
#(3 + 4 < 8)

#0xff \
#0o10 \
#0b1001
\end{verbatim}

\includegraphics[width=5in,height=\textheight,keepaspectratio]{/assets/docs/wfpxRJDZrNeGDA3RjEgFJgAAAAAAAAAA.png}

\subsection{\texorpdfstring{Constructor
{}}{Constructor }}\label{constructor}

\phantomsection\label{constructor-constructor-tooltip}
If a type has a constructor, you can call it like a function to create a
new value of the type.

Converts a value to an integer. Raises an error if there is an attempt
to produce an integer larger than the maximum 64-bit signed integer or
smaller than the minimum 64-bit signed integer.

\begin{itemize}
\tightlist
\item
  Booleans are converted to \texttt{\ 0\ } or \texttt{\ 1\ } .
\item
  Floats and decimals are truncated to the next 64-bit integer.
\item
  Strings are parsed in base 10.
\end{itemize}

{ int } (

{ \href{/docs/reference/foundations/bool/}{bool}
\href{/docs/reference/foundations/int/}{int}
\href{/docs/reference/foundations/float/}{float}
\href{/docs/reference/foundations/str/}{str}
\href{/docs/reference/foundations/decimal/}{decimal} }

) -\textgreater{} \href{/docs/reference/foundations/int/}{int}

\begin{verbatim}
#int(false) \
#int(true) \
#int(2.7) \
#int(decimal("3.8")) \
#(int("27") + int("4"))
\end{verbatim}

\includegraphics[width=5in,height=\textheight,keepaspectratio]{/assets/docs/4vDM_wHvGAGqziHd9y2LQQAAAAAAAAAA.png}

\paragraph{\texorpdfstring{\texttt{\ value\ }}{ value }}\label{constructor-value}

\href{/docs/reference/foundations/bool/}{bool} {or}
\href{/docs/reference/foundations/int/}{int} {or}
\href{/docs/reference/foundations/float/}{float} {or}
\href{/docs/reference/foundations/str/}{str} {or}
\href{/docs/reference/foundations/decimal/}{decimal}

{Required} {{ Positional }}

\phantomsection\label{constructor-value-positional-tooltip}
Positional parameters are specified in order, without names.

The value that should be converted to an integer.

\subsection{\texorpdfstring{{ Definitions
}}{ Definitions }}\label{definitions}

\phantomsection\label{definitions-tooltip}
Functions and types and can have associated definitions. These are
accessed by specifying the function or type, followed by a period, and
then the definition\textquotesingle s name.

\subsubsection{\texorpdfstring{\texttt{\ signum\ }}{ signum }}\label{definitions-signum}

Calculates the sign of an integer.

\begin{itemize}
\tightlist
\item
  If the number is positive, returns
  \texttt{\ }{\texttt{\ 1\ }}\texttt{\ } .
\item
  If the number is negative, returns
  \texttt{\ }{\texttt{\ -\ }}\texttt{\ }{\texttt{\ 1\ }}\texttt{\ } .
\item
  If the number is zero, returns \texttt{\ }{\texttt{\ 0\ }}\texttt{\ }
  .
\end{itemize}

self { . } { signum } (

) -\textgreater{} \href{/docs/reference/foundations/int/}{int}

\begin{verbatim}
#(5).signum() \
#(-5).signum() \
#(0).signum()
\end{verbatim}

\includegraphics[width=5in,height=\textheight,keepaspectratio]{/assets/docs/Vicm2VF6Z98sgjNZQYlaBgAAAAAAAAAA.png}

\subsubsection{\texorpdfstring{\texttt{\ bit-not\ }}{ bit-not }}\label{definitions-bit-not}

Calculates the bitwise NOT of an integer.

For the purposes of this function, the operand is treated as a signed
integer of 64 bits.

self { . } { bit-not } (

) -\textgreater{} \href{/docs/reference/foundations/int/}{int}

\begin{verbatim}
#4.bit-not() \
#(-1).bit-not()
\end{verbatim}

\includegraphics[width=5in,height=\textheight,keepaspectratio]{/assets/docs/3AYO-p6E-z3VLH4vNyWEKgAAAAAAAAAA.png}

\subsubsection{\texorpdfstring{\texttt{\ bit-and\ }}{ bit-and }}\label{definitions-bit-and}

Calculates the bitwise AND between two integers.

For the purposes of this function, the operands are treated as signed
integers of 64 bits.

self { . } { bit-and } (

{ \href{/docs/reference/foundations/int/}{int} }

) -\textgreater{} \href{/docs/reference/foundations/int/}{int}

\begin{verbatim}
#128.bit-and(192)
\end{verbatim}

\includegraphics[width=5in,height=\textheight,keepaspectratio]{/assets/docs/knwTrW-Xbj5sqbcdza7ewgAAAAAAAAAA.png}

\paragraph{\texorpdfstring{\texttt{\ rhs\ }}{ rhs }}\label{definitions-bit-and-rhs}

\href{/docs/reference/foundations/int/}{int}

{Required} {{ Positional }}

\phantomsection\label{definitions-bit-and-rhs-positional-tooltip}
Positional parameters are specified in order, without names.

The right-hand operand of the bitwise AND.

\subsubsection{\texorpdfstring{\texttt{\ bit-or\ }}{ bit-or }}\label{definitions-bit-or}

Calculates the bitwise OR between two integers.

For the purposes of this function, the operands are treated as signed
integers of 64 bits.

self { . } { bit-or } (

{ \href{/docs/reference/foundations/int/}{int} }

) -\textgreater{} \href{/docs/reference/foundations/int/}{int}

\begin{verbatim}
#64.bit-or(32)
\end{verbatim}

\includegraphics[width=5in,height=\textheight,keepaspectratio]{/assets/docs/zaVKMztfj-8VIfbLXeJFUAAAAAAAAAAA.png}

\paragraph{\texorpdfstring{\texttt{\ rhs\ }}{ rhs }}\label{definitions-bit-or-rhs}

\href{/docs/reference/foundations/int/}{int}

{Required} {{ Positional }}

\phantomsection\label{definitions-bit-or-rhs-positional-tooltip}
Positional parameters are specified in order, without names.

The right-hand operand of the bitwise OR.

\subsubsection{\texorpdfstring{\texttt{\ bit-xor\ }}{ bit-xor }}\label{definitions-bit-xor}

Calculates the bitwise XOR between two integers.

For the purposes of this function, the operands are treated as signed
integers of 64 bits.

self { . } { bit-xor } (

{ \href{/docs/reference/foundations/int/}{int} }

) -\textgreater{} \href{/docs/reference/foundations/int/}{int}

\begin{verbatim}
#64.bit-xor(96)
\end{verbatim}

\includegraphics[width=5in,height=\textheight,keepaspectratio]{/assets/docs/KUPqsOL5IXWcGSfAhFpL6wAAAAAAAAAA.png}

\paragraph{\texorpdfstring{\texttt{\ rhs\ }}{ rhs }}\label{definitions-bit-xor-rhs}

\href{/docs/reference/foundations/int/}{int}

{Required} {{ Positional }}

\phantomsection\label{definitions-bit-xor-rhs-positional-tooltip}
Positional parameters are specified in order, without names.

The right-hand operand of the bitwise XOR.

\subsubsection{\texorpdfstring{\texttt{\ bit-lshift\ }}{ bit-lshift }}\label{definitions-bit-lshift}

Shifts the operand\textquotesingle s bits to the left by the specified
amount.

For the purposes of this function, the operand is treated as a signed
integer of 64 bits. An error will occur if the result is too large to
fit in a 64-bit integer.

self { . } { bit-lshift } (

{ \href{/docs/reference/foundations/int/}{int} }

) -\textgreater{} \href{/docs/reference/foundations/int/}{int}

\begin{verbatim}
#33.bit-lshift(2) \
#(-1).bit-lshift(3)
\end{verbatim}

\includegraphics[width=5in,height=\textheight,keepaspectratio]{/assets/docs/kIVISyJsbGpK3k_fu59O2AAAAAAAAAAA.png}

\paragraph{\texorpdfstring{\texttt{\ shift\ }}{ shift }}\label{definitions-bit-lshift-shift}

\href{/docs/reference/foundations/int/}{int}

{Required} {{ Positional }}

\phantomsection\label{definitions-bit-lshift-shift-positional-tooltip}
Positional parameters are specified in order, without names.

The amount of bits to shift. Must not be negative.

\subsubsection{\texorpdfstring{\texttt{\ bit-rshift\ }}{ bit-rshift }}\label{definitions-bit-rshift}

Shifts the operand\textquotesingle s bits to the right by the specified
amount. Performs an arithmetic shift by default (extends the sign bit to
the left, such that negative numbers stay negative), but that can be
changed by the \texttt{\ logical\ } parameter.

For the purposes of this function, the operand is treated as a signed
integer of 64 bits.

self { . } { bit-rshift } (

{ \href{/docs/reference/foundations/int/}{int} , } {
\hyperref[definitions-bit-rshift-parameters-logical]{logical :}
\href{/docs/reference/foundations/bool/}{bool} , }

) -\textgreater{} \href{/docs/reference/foundations/int/}{int}

\begin{verbatim}
#64.bit-rshift(2) \
#(-8).bit-rshift(2) \
#(-8).bit-rshift(2, logical: true)
\end{verbatim}

\includegraphics[width=5in,height=\textheight,keepaspectratio]{/assets/docs/gebaB-CZOzDtnvfjDfjtTgAAAAAAAAAA.png}

\paragraph{\texorpdfstring{\texttt{\ shift\ }}{ shift }}\label{definitions-bit-rshift-shift}

\href{/docs/reference/foundations/int/}{int}

{Required} {{ Positional }}

\phantomsection\label{definitions-bit-rshift-shift-positional-tooltip}
Positional parameters are specified in order, without names.

The amount of bits to shift. Must not be negative.

Shifts larger than 63 are allowed and will cause the return value to
saturate. For non-negative numbers, the return value saturates at
\texttt{\ }{\texttt{\ 0\ }}\texttt{\ } , while, for negative numbers, it
saturates at
\texttt{\ }{\texttt{\ -\ }}\texttt{\ }{\texttt{\ 1\ }}\texttt{\ } if
\texttt{\ logical\ } is set to
\texttt{\ }{\texttt{\ false\ }}\texttt{\ } , or
\texttt{\ }{\texttt{\ 0\ }}\texttt{\ } if it is
\texttt{\ }{\texttt{\ true\ }}\texttt{\ } . This behavior is consistent
with just applying this operation multiple times. Therefore, the shift
will always succeed.

\paragraph{\texorpdfstring{\texttt{\ logical\ }}{ logical }}\label{definitions-bit-rshift-logical}

\href{/docs/reference/foundations/bool/}{bool}

Toggles whether a logical (unsigned) right shift should be performed
instead of arithmetic right shift. If this is
\texttt{\ }{\texttt{\ true\ }}\texttt{\ } , negative operands will not
preserve their sign bit, and bits which appear to the left after the
shift will be \texttt{\ }{\texttt{\ 0\ }}\texttt{\ } . This parameter
has no effect on non-negative operands.

Default: \texttt{\ }{\texttt{\ false\ }}\texttt{\ }

\subsubsection{\texorpdfstring{\texttt{\ from-bytes\ }}{ from-bytes }}\label{definitions-from-bytes}

Converts bytes to an integer.

int { . } { from-bytes } (

{ \href{/docs/reference/foundations/bytes/}{bytes} , } {
\hyperref[definitions-from-bytes-parameters-endian]{endian :}
\href{/docs/reference/foundations/str/}{str} , } {
\hyperref[definitions-from-bytes-parameters-signed]{signed :}
\href{/docs/reference/foundations/bool/}{bool} , }

) -\textgreater{} \href{/docs/reference/foundations/int/}{int}

\begin{verbatim}
#int.from-bytes(bytes((0, 0, 0, 0, 0, 0, 0, 1))) \
#int.from-bytes(bytes((1, 0, 0, 0, 0, 0, 0, 0)), endian: "big")
\end{verbatim}

\includegraphics[width=5in,height=\textheight,keepaspectratio]{/assets/docs/I0LPQ0WUii0fthcD20cosAAAAAAAAAAA.png}

\paragraph{\texorpdfstring{\texttt{\ bytes\ }}{ bytes }}\label{definitions-from-bytes-bytes}

\href{/docs/reference/foundations/bytes/}{bytes}

{Required} {{ Positional }}

\phantomsection\label{definitions-from-bytes-bytes-positional-tooltip}
Positional parameters are specified in order, without names.

The bytes that should be converted to an integer.

Must be of length at most 8 so that the result fits into a 64-bit signed
integer.

\paragraph{\texorpdfstring{\texttt{\ endian\ }}{ endian }}\label{definitions-from-bytes-endian}

\href{/docs/reference/foundations/str/}{str}

The endianness of the conversion.

\begin{longtable}[]{@{}ll@{}}
\toprule\noalign{}
Variant & Details \\
\midrule\noalign{}
\endhead
\bottomrule\noalign{}
\endlastfoot
\texttt{\ "\ big\ "\ } & Big-endian byte order: The highest-value byte
is at the beginning of the bytes. \\
\texttt{\ "\ little\ "\ } & Little-endian byte order: The lowest-value
byte is at the beginning of the bytes. \\
\end{longtable}

Default: \texttt{\ }{\texttt{\ "little"\ }}\texttt{\ }

\paragraph{\texorpdfstring{\texttt{\ signed\ }}{ signed }}\label{definitions-from-bytes-signed}

\href{/docs/reference/foundations/bool/}{bool}

Whether the bytes should be treated as a signed integer. If this is
\texttt{\ }{\texttt{\ true\ }}\texttt{\ } and the most significant bit
is set, the resulting number will negative.

Default: \texttt{\ }{\texttt{\ true\ }}\texttt{\ }

\subsubsection{\texorpdfstring{\texttt{\ to-bytes\ }}{ to-bytes }}\label{definitions-to-bytes}

Converts an integer to bytes.

self { . } { to-bytes } (

{ \hyperref[definitions-to-bytes-parameters-endian]{endian :}
\href{/docs/reference/foundations/str/}{str} , } {
\hyperref[definitions-to-bytes-parameters-size]{size :}
\href{/docs/reference/foundations/int/}{int} , }

) -\textgreater{} \href{/docs/reference/foundations/bytes/}{bytes}

\begin{verbatim}
#array(10000.to-bytes(endian: "big")) \
#array(10000.to-bytes(size: 4))
\end{verbatim}

\includegraphics[width=5in,height=\textheight,keepaspectratio]{/assets/docs/FF7gGW4eVOEhYjIZXy8BIgAAAAAAAAAA.png}

\paragraph{\texorpdfstring{\texttt{\ endian\ }}{ endian }}\label{definitions-to-bytes-endian}

\href{/docs/reference/foundations/str/}{str}

The endianness of the conversion.

\begin{longtable}[]{@{}ll@{}}
\toprule\noalign{}
Variant & Details \\
\midrule\noalign{}
\endhead
\bottomrule\noalign{}
\endlastfoot
\texttt{\ "\ big\ "\ } & Big-endian byte order: The highest-value byte
is at the beginning of the bytes. \\
\texttt{\ "\ little\ "\ } & Little-endian byte order: The lowest-value
byte is at the beginning of the bytes. \\
\end{longtable}

Default: \texttt{\ }{\texttt{\ "little"\ }}\texttt{\ }

\paragraph{\texorpdfstring{\texttt{\ size\ }}{ size }}\label{definitions-to-bytes-size}

\href{/docs/reference/foundations/int/}{int}

The size in bytes of the resulting bytes (must be at least zero). If the
integer is too large to fit in the specified size, the conversion will
truncate the remaining bytes based on the endianness. To keep the same
resulting value, if the endianness is big-endian, the truncation will
happen at the rightmost bytes. Otherwise, if the endianness is
little-endian, the truncation will happen at the leftmost bytes.

Be aware that if the integer is negative and the size is not enough to
make the number fit, when passing the resulting bytes to
\texttt{\ int.from-bytes\ } , the resulting number might be positive, as
the most significant bit might not be set to 1.

Default: \texttt{\ }{\texttt{\ 8\ }}\texttt{\ }

\href{/docs/reference/foundations/function/}{\pandocbounded{\includesvg[keepaspectratio]{/assets/icons/16-arrow-right.svg}}}

{ Function } { Previous page }

\href{/docs/reference/foundations/label/}{\pandocbounded{\includesvg[keepaspectratio]{/assets/icons/16-arrow-right.svg}}}

{ Label } { Next page }


\section{Docs LaTeX/typst.app/docs/reference/foundations/bytes.tex}
\title{typst.app/docs/reference/foundations/bytes}

\begin{itemize}
\tightlist
\item
  \href{/docs}{\includesvg[width=0.16667in,height=0.16667in]{/assets/icons/16-docs-dark.svg}}
\item
  \includesvg[width=0.16667in,height=0.16667in]{/assets/icons/16-arrow-right.svg}
\item
  \href{/docs/reference/}{Reference}
\item
  \includesvg[width=0.16667in,height=0.16667in]{/assets/icons/16-arrow-right.svg}
\item
  \href{/docs/reference/foundations/}{Foundations}
\item
  \includesvg[width=0.16667in,height=0.16667in]{/assets/icons/16-arrow-right.svg}
\item
  \href{/docs/reference/foundations/bytes/}{Bytes}
\end{itemize}

\section{\texorpdfstring{{ bytes }}{ bytes }}\label{summary}

A sequence of bytes.

This is conceptually similar to an array of
\href{/docs/reference/foundations/int/}{integers} between
\texttt{\ }{\texttt{\ 0\ }}\texttt{\ } and
\texttt{\ }{\texttt{\ 255\ }}\texttt{\ } , but represented much more
efficiently. You can iterate over it using a
\href{/docs/reference/scripting/\#loops}{for loop} .

You can convert

\begin{itemize}
\tightlist
\item
  a \href{/docs/reference/foundations/str/}{string} or an
  \href{/docs/reference/foundations/array/}{array} of integers to bytes
  with the \href{/docs/reference/foundations/bytes/}{\texttt{\ bytes\ }}
  constructor
\item
  bytes to a string with the
  \href{/docs/reference/foundations/str/}{\texttt{\ str\ }} constructor,
  with UTF-8 encoding
\item
  bytes to an array of integers with the
  \href{/docs/reference/foundations/array/}{\texttt{\ array\ }}
  constructor
\end{itemize}

When \href{/docs/reference/data-loading/read/}{reading} data from a
file, you can decide whether to load it as a string or as raw bytes.

\begin{verbatim}
#bytes((123, 160, 22, 0)) \
#bytes("Hello 😃")

#let data = read(
  "rhino.png",
  encoding: none,
)

// Magic bytes.
#array(data.slice(0, 4)) \
#str(data.slice(1, 4))
\end{verbatim}

\includegraphics[width=5in,height=\textheight,keepaspectratio]{/assets/docs/sJtYFgVyQkDZELEHje5ywwAAAAAAAAAA.png}

\subsection{\texorpdfstring{Constructor
{}}{Constructor }}\label{constructor}

\phantomsection\label{constructor-constructor-tooltip}
If a type has a constructor, you can call it like a function to create a
new value of the type.

Converts a value to bytes.

\begin{itemize}
\tightlist
\item
  Strings are encoded in UTF-8.
\item
  Arrays of integers between \texttt{\ }{\texttt{\ 0\ }}\texttt{\ } and
  \texttt{\ }{\texttt{\ 255\ }}\texttt{\ } are converted directly. The
  dedicated byte representation is much more efficient than the array
  representation and thus typically used for large byte buffers (e.g.
  image data).
\end{itemize}

{ bytes } (

{ \href{/docs/reference/foundations/str/}{str}
\href{/docs/reference/foundations/bytes/}{bytes}
\href{/docs/reference/foundations/array/}{array} }

) -\textgreater{} \href{/docs/reference/foundations/bytes/}{bytes}

\begin{verbatim}
#bytes("Hello 😃") \
#bytes((123, 160, 22, 0))
\end{verbatim}

\includegraphics[width=5in,height=\textheight,keepaspectratio]{/assets/docs/PlfVajGmfDLMY6p8X4S3BwAAAAAAAAAA.png}

\paragraph{\texorpdfstring{\texttt{\ value\ }}{ value }}\label{constructor-value}

\href{/docs/reference/foundations/str/}{str} {or}
\href{/docs/reference/foundations/bytes/}{bytes} {or}
\href{/docs/reference/foundations/array/}{array}

{Required} {{ Positional }}

\phantomsection\label{constructor-value-positional-tooltip}
Positional parameters are specified in order, without names.

The value that should be converted to bytes.

\subsection{\texorpdfstring{{ Definitions
}}{ Definitions }}\label{definitions}

\phantomsection\label{definitions-tooltip}
Functions and types and can have associated definitions. These are
accessed by specifying the function or type, followed by a period, and
then the definition\textquotesingle s name.

\subsubsection{\texorpdfstring{\texttt{\ len\ }}{ len }}\label{definitions-len}

The length in bytes.

self { . } { len } (

) -\textgreater{} \href{/docs/reference/foundations/int/}{int}

\subsubsection{\texorpdfstring{\texttt{\ at\ }}{ at }}\label{definitions-at}

Returns the byte at the specified index. Returns the default value if
the index is out of bounds or fails with an error if no default value
was specified.

self { . } { at } (

{ \href{/docs/reference/foundations/int/}{int} , } {
\hyperref[definitions-at-parameters-default]{default :} { any } , }

) -\textgreater{} { any }

\paragraph{\texorpdfstring{\texttt{\ index\ }}{ index }}\label{definitions-at-index}

\href{/docs/reference/foundations/int/}{int}

{Required} {{ Positional }}

\phantomsection\label{definitions-at-index-positional-tooltip}
Positional parameters are specified in order, without names.

The index at which to retrieve the byte.

\paragraph{\texorpdfstring{\texttt{\ default\ }}{ default }}\label{definitions-at-default}

{ any }

A default value to return if the index is out of bounds.

\subsubsection{\texorpdfstring{\texttt{\ slice\ }}{ slice }}\label{definitions-slice}

Extracts a subslice of the bytes. Fails with an error if the start or
end index is out of bounds.

self { . } { slice } (

{ \href{/docs/reference/foundations/int/}{int} , } {
\href{/docs/reference/foundations/none/}{none}
\href{/docs/reference/foundations/int/}{int} , } {
\hyperref[definitions-slice-parameters-count]{count :}
\href{/docs/reference/foundations/int/}{int} , }

) -\textgreater{} \href{/docs/reference/foundations/bytes/}{bytes}

\paragraph{\texorpdfstring{\texttt{\ start\ }}{ start }}\label{definitions-slice-start}

\href{/docs/reference/foundations/int/}{int}

{Required} {{ Positional }}

\phantomsection\label{definitions-slice-start-positional-tooltip}
Positional parameters are specified in order, without names.

The start index (inclusive).

\paragraph{\texorpdfstring{\texttt{\ end\ }}{ end }}\label{definitions-slice-end}

\href{/docs/reference/foundations/none/}{none} {or}
\href{/docs/reference/foundations/int/}{int}

{{ Positional }}

\phantomsection\label{definitions-slice-end-positional-tooltip}
Positional parameters are specified in order, without names.

The end index (exclusive). If omitted, the whole slice until the end is
extracted.

Default: \texttt{\ }{\texttt{\ none\ }}\texttt{\ }

\paragraph{\texorpdfstring{\texttt{\ count\ }}{ count }}\label{definitions-slice-count}

\href{/docs/reference/foundations/int/}{int}

The number of items to extract. This is equivalent to passing
\texttt{\ start\ +\ count\ } as the \texttt{\ end\ } position. Mutually
exclusive with \texttt{\ end\ } .

\href{/docs/reference/foundations/bool/}{\pandocbounded{\includesvg[keepaspectratio]{/assets/icons/16-arrow-right.svg}}}

{ Boolean } { Previous page }

\href{/docs/reference/foundations/calc/}{\pandocbounded{\includesvg[keepaspectratio]{/assets/icons/16-arrow-right.svg}}}

{ Calculation } { Next page }


\section{Docs LaTeX/typst.app/docs/reference/foundations/assert.tex}
\title{typst.app/docs/reference/foundations/assert}

\begin{itemize}
\tightlist
\item
  \href{/docs}{\includesvg[width=0.16667in,height=0.16667in]{/assets/icons/16-docs-dark.svg}}
\item
  \includesvg[width=0.16667in,height=0.16667in]{/assets/icons/16-arrow-right.svg}
\item
  \href{/docs/reference/}{Reference}
\item
  \includesvg[width=0.16667in,height=0.16667in]{/assets/icons/16-arrow-right.svg}
\item
  \href{/docs/reference/foundations/}{Foundations}
\item
  \includesvg[width=0.16667in,height=0.16667in]{/assets/icons/16-arrow-right.svg}
\item
  \href{/docs/reference/foundations/assert/}{Assert}
\end{itemize}

\section{\texorpdfstring{\texttt{\ assert\ }}{ assert }}\label{summary}

Ensures that a condition is fulfilled.

Fails with an error if the condition is not fulfilled. Does not produce
any output in the document.

If you wish to test equality between two values, see
\href{/docs/reference/foundations/assert/\#definitions-eq}{\texttt{\ assert.eq\ }}
and
\href{/docs/reference/foundations/assert/\#definitions-ne}{\texttt{\ assert.ne\ }}
.

\subsection{Example}\label{example}

\begin{verbatim}
#assert(1 < 2, message: "math broke")
\end{verbatim}

\subsection{\texorpdfstring{{ Parameters
}}{ Parameters }}\label{parameters}

\phantomsection\label{parameters-tooltip}
Parameters are the inputs to a function. They are specified in
parentheses after the function name.

{ assert } (

{ \href{/docs/reference/foundations/bool/}{bool} , } {
\hyperref[parameters-message]{message :}
\href{/docs/reference/foundations/str/}{str} , }

)

\subsubsection{\texorpdfstring{\texttt{\ condition\ }}{ condition }}\label{parameters-condition}

\href{/docs/reference/foundations/bool/}{bool}

{Required} {{ Positional }}

\phantomsection\label{parameters-condition-positional-tooltip}
Positional parameters are specified in order, without names.

The condition that must be true for the assertion to pass.

\subsubsection{\texorpdfstring{\texttt{\ message\ }}{ message }}\label{parameters-message}

\href{/docs/reference/foundations/str/}{str}

The error message when the assertion fails.

\subsection{\texorpdfstring{{ Definitions
}}{ Definitions }}\label{definitions}

\phantomsection\label{definitions-tooltip}
Functions and types and can have associated definitions. These are
accessed by specifying the function or type, followed by a period, and
then the definition\textquotesingle s name.

\subsubsection{\texorpdfstring{\texttt{\ eq\ }}{ eq }}\label{definitions-eq}

Ensures that two values are equal.

Fails with an error if the first value is not equal to the second. Does
not produce any output in the document.

assert { . } { eq } (

{ { any } , } { { any } , } {
\hyperref[definitions-eq-parameters-message]{message :}
\href{/docs/reference/foundations/str/}{str} , }

)

\includesvg[width=0.16667in,height=0.16667in]{/assets/icons/16-arrow-right.svg}
View example

\begin{verbatim}
#assert.eq(10, 10)
\end{verbatim}

\paragraph{\texorpdfstring{\texttt{\ left\ }}{ left }}\label{definitions-eq-left}

{ any }

{Required} {{ Positional }}

\phantomsection\label{definitions-eq-left-positional-tooltip}
Positional parameters are specified in order, without names.

The first value to compare.

\paragraph{\texorpdfstring{\texttt{\ right\ }}{ right }}\label{definitions-eq-right}

{ any }

{Required} {{ Positional }}

\phantomsection\label{definitions-eq-right-positional-tooltip}
Positional parameters are specified in order, without names.

The second value to compare.

\paragraph{\texorpdfstring{\texttt{\ message\ }}{ message }}\label{definitions-eq-message}

\href{/docs/reference/foundations/str/}{str}

An optional message to display on error instead of the representations
of the compared values.

\subsubsection{\texorpdfstring{\texttt{\ ne\ }}{ ne }}\label{definitions-ne}

Ensures that two values are not equal.

Fails with an error if the first value is equal to the second. Does not
produce any output in the document.

assert { . } { ne } (

{ { any } , } { { any } , } {
\hyperref[definitions-ne-parameters-message]{message :}
\href{/docs/reference/foundations/str/}{str} , }

)

\includesvg[width=0.16667in,height=0.16667in]{/assets/icons/16-arrow-right.svg}
View example

\begin{verbatim}
#assert.ne(3, 4)
\end{verbatim}

\paragraph{\texorpdfstring{\texttt{\ left\ }}{ left }}\label{definitions-ne-left}

{ any }

{Required} {{ Positional }}

\phantomsection\label{definitions-ne-left-positional-tooltip}
Positional parameters are specified in order, without names.

The first value to compare.

\paragraph{\texorpdfstring{\texttt{\ right\ }}{ right }}\label{definitions-ne-right}

{ any }

{Required} {{ Positional }}

\phantomsection\label{definitions-ne-right-positional-tooltip}
Positional parameters are specified in order, without names.

The second value to compare.

\paragraph{\texorpdfstring{\texttt{\ message\ }}{ message }}\label{definitions-ne-message}

\href{/docs/reference/foundations/str/}{str}

An optional message to display on error instead of the representations
of the compared values.

\href{/docs/reference/foundations/array/}{\pandocbounded{\includesvg[keepaspectratio]{/assets/icons/16-arrow-right.svg}}}

{ Array } { Previous page }

\href{/docs/reference/foundations/auto/}{\pandocbounded{\includesvg[keepaspectratio]{/assets/icons/16-arrow-right.svg}}}

{ Auto } { Next page }


\section{Docs LaTeX/typst.app/docs/reference/foundations/version.tex}
\title{typst.app/docs/reference/foundations/version}

\begin{itemize}
\tightlist
\item
  \href{/docs}{\includesvg[width=0.16667in,height=0.16667in]{/assets/icons/16-docs-dark.svg}}
\item
  \includesvg[width=0.16667in,height=0.16667in]{/assets/icons/16-arrow-right.svg}
\item
  \href{/docs/reference/}{Reference}
\item
  \includesvg[width=0.16667in,height=0.16667in]{/assets/icons/16-arrow-right.svg}
\item
  \href{/docs/reference/foundations/}{Foundations}
\item
  \includesvg[width=0.16667in,height=0.16667in]{/assets/icons/16-arrow-right.svg}
\item
  \href{/docs/reference/foundations/version/}{Version}
\end{itemize}

\section{\texorpdfstring{{ version }}{ version }}\label{summary}

A version with an arbitrary number of components.

The first three components have names that can be used as fields:
\texttt{\ major\ } , \texttt{\ minor\ } , \texttt{\ patch\ } . All
following components do not have names.

The list of components is semantically extended by an infinite list of
zeros. This means that, for example, \texttt{\ 0.8\ } is the same as
\texttt{\ 0.8.0\ } . As a special case, the empty version (that has no
components at all) is the same as \texttt{\ 0\ } , \texttt{\ 0.0\ } ,
\texttt{\ 0.0.0\ } , and so on.

The current version of the Typst compiler is available as
\texttt{\ sys.version\ } .

You can convert a version to an array of explicitly given components
using the \href{/docs/reference/foundations/array/}{\texttt{\ array\ }}
constructor.

\subsection{\texorpdfstring{Constructor
{}}{Constructor }}\label{constructor}

\phantomsection\label{constructor-constructor-tooltip}
If a type has a constructor, you can call it like a function to create a
new value of the type.

Creates a new version.

It can have any number of components (even zero).

{ version } (

{ \hyperref[constructor-parameters-components]{..}
\href{/docs/reference/foundations/int/}{int}
\href{/docs/reference/foundations/array/}{array} }

) -\textgreater{} \href{/docs/reference/foundations/version/}{version}

\begin{verbatim}
#version() \
#version(1) \
#version(1, 2, 3, 4) \
#version((1, 2, 3, 4)) \
#version((1, 2), 3)
\end{verbatim}

\includegraphics[width=5in,height=\textheight,keepaspectratio]{/assets/docs/Fx1_6ds8kbJ35Werk0qIqQAAAAAAAAAA.png}

\paragraph{\texorpdfstring{\texttt{\ components\ }}{ components }}\label{constructor-components}

\href{/docs/reference/foundations/int/}{int} {or}
\href{/docs/reference/foundations/array/}{array}

{Required} {{ Positional }}

\phantomsection\label{constructor-components-positional-tooltip}
Positional parameters are specified in order, without names.

{{ Variadic }}

\phantomsection\label{constructor-components-variadic-tooltip}
Variadic parameters can be specified multiple times.

The components of the version (array arguments are flattened)

\subsection{\texorpdfstring{{ Definitions
}}{ Definitions }}\label{definitions}

\phantomsection\label{definitions-tooltip}
Functions and types and can have associated definitions. These are
accessed by specifying the function or type, followed by a period, and
then the definition\textquotesingle s name.

\subsubsection{\texorpdfstring{\texttt{\ at\ }}{ at }}\label{definitions-at}

Retrieves a component of a version.

The returned integer is always non-negative. Returns \texttt{\ 0\ } if
the version isn\textquotesingle t specified to the necessary length.

self { . } { at } (

{ \href{/docs/reference/foundations/int/}{int} }

) -\textgreater{} \href{/docs/reference/foundations/int/}{int}

\paragraph{\texorpdfstring{\texttt{\ index\ }}{ index }}\label{definitions-at-index}

\href{/docs/reference/foundations/int/}{int}

{Required} {{ Positional }}

\phantomsection\label{definitions-at-index-positional-tooltip}
Positional parameters are specified in order, without names.

The index at which to retrieve the component. If negative, indexes from
the back of the explicitly given components.

\href{/docs/reference/foundations/type/}{\pandocbounded{\includesvg[keepaspectratio]{/assets/icons/16-arrow-right.svg}}}

{ Type } { Previous page }

\href{/docs/reference/model/}{\pandocbounded{\includesvg[keepaspectratio]{/assets/icons/16-arrow-right.svg}}}

{ Model } { Next page }


\section{Docs LaTeX/typst.app/docs/reference/foundations/module.tex}
\title{typst.app/docs/reference/foundations/module}

\begin{itemize}
\tightlist
\item
  \href{/docs}{\includesvg[width=0.16667in,height=0.16667in]{/assets/icons/16-docs-dark.svg}}
\item
  \includesvg[width=0.16667in,height=0.16667in]{/assets/icons/16-arrow-right.svg}
\item
  \href{/docs/reference/}{Reference}
\item
  \includesvg[width=0.16667in,height=0.16667in]{/assets/icons/16-arrow-right.svg}
\item
  \href{/docs/reference/foundations/}{Foundations}
\item
  \includesvg[width=0.16667in,height=0.16667in]{/assets/icons/16-arrow-right.svg}
\item
  \href{/docs/reference/foundations/module/}{Module}
\end{itemize}

\section{\texorpdfstring{{ module }}{ module }}\label{summary}

An evaluated module, either built-in or resulting from a file.

You can access definitions from the module using
\href{/docs/reference/scripting/\#fields}{field access notation} and
interact with it using the
\href{/docs/reference/scripting/\#modules}{import and include syntaxes}
. Alternatively, it is possible to convert a module to a dictionary, and
therefore access its contents dynamically, using the
\href{/docs/reference/foundations/dictionary/\#constructor}{dictionary
constructor} .

\subsection{Example}\label{example}

\begin{verbatim}
#import "utils.typ"
#utils.add(2, 5)

#import utils: sub
#sub(1, 4)
\end{verbatim}

\includegraphics[width=5in,height=\textheight,keepaspectratio]{/assets/docs/itOPaialNOb62A81RHFv_wAAAAAAAAAA.png}

\href{/docs/reference/foundations/label/}{\pandocbounded{\includesvg[keepaspectratio]{/assets/icons/16-arrow-right.svg}}}

{ Label } { Previous page }

\href{/docs/reference/foundations/none/}{\pandocbounded{\includesvg[keepaspectratio]{/assets/icons/16-arrow-right.svg}}}

{ None } { Next page }




\section{C Docs LaTeX/docs/reference/layout.tex}
\section{Docs LaTeX/typst.app/docs/reference/layout/repeat.tex}
\title{typst.app/docs/reference/layout/repeat}

\begin{itemize}
\tightlist
\item
  \href{/docs}{\includesvg[width=0.16667in,height=0.16667in]{/assets/icons/16-docs-dark.svg}}
\item
  \includesvg[width=0.16667in,height=0.16667in]{/assets/icons/16-arrow-right.svg}
\item
  \href{/docs/reference/}{Reference}
\item
  \includesvg[width=0.16667in,height=0.16667in]{/assets/icons/16-arrow-right.svg}
\item
  \href{/docs/reference/layout/}{Layout}
\item
  \includesvg[width=0.16667in,height=0.16667in]{/assets/icons/16-arrow-right.svg}
\item
  \href{/docs/reference/layout/repeat/}{Repeat}
\end{itemize}

\section{\texorpdfstring{\texttt{\ repeat\ } {{ Element
}}}{ repeat   Element }}\label{summary}

\phantomsection\label{element-tooltip}
Element functions can be customized with \texttt{\ set\ } and
\texttt{\ show\ } rules.

Repeats content to the available space.

This can be useful when implementing a custom index, reference, or
outline.

Space may be inserted between the instances of the body parameter, so be
sure to adjust the
\href{/docs/reference/layout/repeat/\#parameters-justify}{\texttt{\ justify\ }}
parameter accordingly.

Errors if there no bounds on the available space, as it would create
infinite content.

\subsection{Example}\label{example}

\begin{verbatim}
Sign on the dotted line:
#box(width: 1fr, repeat[.])

#set text(10pt)
#v(8pt, weak: true)
#align(right)[
  Berlin, the 22nd of December, 2022
]
\end{verbatim}

\includegraphics[width=5in,height=\textheight,keepaspectratio]{/assets/docs/LGILa4453RB6xoEobzmQcAAAAAAAAAAA.png}

\subsection{\texorpdfstring{{ Parameters
}}{ Parameters }}\label{parameters}

\phantomsection\label{parameters-tooltip}
Parameters are the inputs to a function. They are specified in
parentheses after the function name.

{ repeat } (

{ \href{/docs/reference/foundations/content/}{content} , } {
\hyperref[parameters-gap]{gap :}
\href{/docs/reference/layout/length/}{length} , } {
\hyperref[parameters-justify]{justify :}
\href{/docs/reference/foundations/bool/}{bool} , }

) -\textgreater{} \href{/docs/reference/foundations/content/}{content}

\subsubsection{\texorpdfstring{\texttt{\ body\ }}{ body }}\label{parameters-body}

\href{/docs/reference/foundations/content/}{content}

{Required} {{ Positional }}

\phantomsection\label{parameters-body-positional-tooltip}
Positional parameters are specified in order, without names.

The content to repeat.

\subsubsection{\texorpdfstring{\texttt{\ gap\ }}{ gap }}\label{parameters-gap}

\href{/docs/reference/layout/length/}{length}

{{ Settable }}

\phantomsection\label{parameters-gap-settable-tooltip}
Settable parameters can be customized for all following uses of the
function with a \texttt{\ set\ } rule.

The gap between each instance of the body.

Default: \texttt{\ }{\texttt{\ 0pt\ }}\texttt{\ }

\subsubsection{\texorpdfstring{\texttt{\ justify\ }}{ justify }}\label{parameters-justify}

\href{/docs/reference/foundations/bool/}{bool}

{{ Settable }}

\phantomsection\label{parameters-justify-settable-tooltip}
Settable parameters can be customized for all following uses of the
function with a \texttt{\ set\ } rule.

Whether to increase the gap between instances to completely fill the
available space.

Default: \texttt{\ }{\texttt{\ true\ }}\texttt{\ }

\href{/docs/reference/layout/relative/}{\pandocbounded{\includesvg[keepaspectratio]{/assets/icons/16-arrow-right.svg}}}

{ Relative Length } { Previous page }

\href{/docs/reference/layout/rotate/}{\pandocbounded{\includesvg[keepaspectratio]{/assets/icons/16-arrow-right.svg}}}

{ Rotate } { Next page }


\section{Docs LaTeX/typst.app/docs/reference/layout/pad.tex}
\title{typst.app/docs/reference/layout/pad}

\begin{itemize}
\tightlist
\item
  \href{/docs}{\includesvg[width=0.16667in,height=0.16667in]{/assets/icons/16-docs-dark.svg}}
\item
  \includesvg[width=0.16667in,height=0.16667in]{/assets/icons/16-arrow-right.svg}
\item
  \href{/docs/reference/}{Reference}
\item
  \includesvg[width=0.16667in,height=0.16667in]{/assets/icons/16-arrow-right.svg}
\item
  \href{/docs/reference/layout/}{Layout}
\item
  \includesvg[width=0.16667in,height=0.16667in]{/assets/icons/16-arrow-right.svg}
\item
  \href{/docs/reference/layout/pad/}{Padding}
\end{itemize}

\section{\texorpdfstring{\texttt{\ pad\ } {{ Element
}}}{ pad   Element }}\label{summary}

\phantomsection\label{element-tooltip}
Element functions can be customized with \texttt{\ set\ } and
\texttt{\ show\ } rules.

Adds spacing around content.

The spacing can be specified for each side individually, or for all
sides at once by specifying a positional argument.

\subsection{Example}\label{example}

\begin{verbatim}
#set align(center)

#pad(x: 16pt, image("typing.jpg"))
_Typing speeds can be
 measured in words per minute._
\end{verbatim}

\includegraphics[width=5in,height=\textheight,keepaspectratio]{/assets/docs/YnvzY3ls2HrcPgokDMxVqwAAAAAAAAAA.png}

\subsection{\texorpdfstring{{ Parameters
}}{ Parameters }}\label{parameters}

\phantomsection\label{parameters-tooltip}
Parameters are the inputs to a function. They are specified in
parentheses after the function name.

{ pad } (

{ \hyperref[parameters-left]{left :}
\href{/docs/reference/layout/relative/}{relative} , } {
\hyperref[parameters-top]{top :}
\href{/docs/reference/layout/relative/}{relative} , } {
\hyperref[parameters-right]{right :}
\href{/docs/reference/layout/relative/}{relative} , } {
\hyperref[parameters-bottom]{bottom :}
\href{/docs/reference/layout/relative/}{relative} , } {
\hyperref[parameters-x]{x :}
\href{/docs/reference/layout/relative/}{relative} , } {
\hyperref[parameters-y]{y :}
\href{/docs/reference/layout/relative/}{relative} , } {
\hyperref[parameters-rest]{rest :}
\href{/docs/reference/layout/relative/}{relative} , } {
\href{/docs/reference/foundations/content/}{content} , }

) -\textgreater{} \href{/docs/reference/foundations/content/}{content}

\subsubsection{\texorpdfstring{\texttt{\ left\ }}{ left }}\label{parameters-left}

\href{/docs/reference/layout/relative/}{relative}

{{ Settable }}

\phantomsection\label{parameters-left-settable-tooltip}
Settable parameters can be customized for all following uses of the
function with a \texttt{\ set\ } rule.

The padding at the left side.

Default:
\texttt{\ }{\texttt{\ 0\%\ }}\texttt{\ }{\texttt{\ +\ }}\texttt{\ }{\texttt{\ 0pt\ }}\texttt{\ }

\subsubsection{\texorpdfstring{\texttt{\ top\ }}{ top }}\label{parameters-top}

\href{/docs/reference/layout/relative/}{relative}

{{ Settable }}

\phantomsection\label{parameters-top-settable-tooltip}
Settable parameters can be customized for all following uses of the
function with a \texttt{\ set\ } rule.

The padding at the top side.

Default:
\texttt{\ }{\texttt{\ 0\%\ }}\texttt{\ }{\texttt{\ +\ }}\texttt{\ }{\texttt{\ 0pt\ }}\texttt{\ }

\subsubsection{\texorpdfstring{\texttt{\ right\ }}{ right }}\label{parameters-right}

\href{/docs/reference/layout/relative/}{relative}

{{ Settable }}

\phantomsection\label{parameters-right-settable-tooltip}
Settable parameters can be customized for all following uses of the
function with a \texttt{\ set\ } rule.

The padding at the right side.

Default:
\texttt{\ }{\texttt{\ 0\%\ }}\texttt{\ }{\texttt{\ +\ }}\texttt{\ }{\texttt{\ 0pt\ }}\texttt{\ }

\subsubsection{\texorpdfstring{\texttt{\ bottom\ }}{ bottom }}\label{parameters-bottom}

\href{/docs/reference/layout/relative/}{relative}

{{ Settable }}

\phantomsection\label{parameters-bottom-settable-tooltip}
Settable parameters can be customized for all following uses of the
function with a \texttt{\ set\ } rule.

The padding at the bottom side.

Default:
\texttt{\ }{\texttt{\ 0\%\ }}\texttt{\ }{\texttt{\ +\ }}\texttt{\ }{\texttt{\ 0pt\ }}\texttt{\ }

\subsubsection{\texorpdfstring{\texttt{\ x\ }}{ x }}\label{parameters-x}

\href{/docs/reference/layout/relative/}{relative}

{{ Settable }}

\phantomsection\label{parameters-x-settable-tooltip}
Settable parameters can be customized for all following uses of the
function with a \texttt{\ set\ } rule.

A shorthand to set \texttt{\ left\ } and \texttt{\ right\ } to the same
value.

Default:
\texttt{\ }{\texttt{\ 0\%\ }}\texttt{\ }{\texttt{\ +\ }}\texttt{\ }{\texttt{\ 0pt\ }}\texttt{\ }

\subsubsection{\texorpdfstring{\texttt{\ y\ }}{ y }}\label{parameters-y}

\href{/docs/reference/layout/relative/}{relative}

{{ Settable }}

\phantomsection\label{parameters-y-settable-tooltip}
Settable parameters can be customized for all following uses of the
function with a \texttt{\ set\ } rule.

A shorthand to set \texttt{\ top\ } and \texttt{\ bottom\ } to the same
value.

Default:
\texttt{\ }{\texttt{\ 0\%\ }}\texttt{\ }{\texttt{\ +\ }}\texttt{\ }{\texttt{\ 0pt\ }}\texttt{\ }

\subsubsection{\texorpdfstring{\texttt{\ rest\ }}{ rest }}\label{parameters-rest}

\href{/docs/reference/layout/relative/}{relative}

{{ Settable }}

\phantomsection\label{parameters-rest-settable-tooltip}
Settable parameters can be customized for all following uses of the
function with a \texttt{\ set\ } rule.

A shorthand to set all four sides to the same value.

Default:
\texttt{\ }{\texttt{\ 0\%\ }}\texttt{\ }{\texttt{\ +\ }}\texttt{\ }{\texttt{\ 0pt\ }}\texttt{\ }

\subsubsection{\texorpdfstring{\texttt{\ body\ }}{ body }}\label{parameters-body}

\href{/docs/reference/foundations/content/}{content}

{Required} {{ Positional }}

\phantomsection\label{parameters-body-positional-tooltip}
Positional parameters are specified in order, without names.

The content to pad at the sides.

\href{/docs/reference/layout/move/}{\pandocbounded{\includesvg[keepaspectratio]{/assets/icons/16-arrow-right.svg}}}

{ Move } { Previous page }

\href{/docs/reference/layout/page/}{\pandocbounded{\includesvg[keepaspectratio]{/assets/icons/16-arrow-right.svg}}}

{ Page } { Next page }


\section{Docs LaTeX/typst.app/docs/reference/layout/grid.tex}
\title{typst.app/docs/reference/layout/grid}

\begin{itemize}
\tightlist
\item
  \href{/docs}{\includesvg[width=0.16667in,height=0.16667in]{/assets/icons/16-docs-dark.svg}}
\item
  \includesvg[width=0.16667in,height=0.16667in]{/assets/icons/16-arrow-right.svg}
\item
  \href{/docs/reference/}{Reference}
\item
  \includesvg[width=0.16667in,height=0.16667in]{/assets/icons/16-arrow-right.svg}
\item
  \href{/docs/reference/layout/}{Layout}
\item
  \includesvg[width=0.16667in,height=0.16667in]{/assets/icons/16-arrow-right.svg}
\item
  \href{/docs/reference/layout/grid/}{Grid}
\end{itemize}

\section{\texorpdfstring{\texttt{\ grid\ } {{ Element
}}}{ grid   Element }}\label{summary}

\phantomsection\label{element-tooltip}
Element functions can be customized with \texttt{\ set\ } and
\texttt{\ show\ } rules.

Arranges content in a grid.

The grid element allows you to arrange content in a grid. You can define
the number of rows and columns, as well as the size of the gutters
between them. There are multiple sizing modes for columns and rows that
can be used to create complex layouts.

While the grid and table elements work very similarly, they are intended
for different use cases and carry different semantics. The grid element
is intended for presentational and layout purposes, while the
\href{/docs/reference/model/table/}{\texttt{\ table\ }} element is
intended for, in broad terms, presenting multiple related data points.
In the future, Typst will annotate its output such that screenreaders
will announce content in \texttt{\ table\ } as tabular while a
grid\textquotesingle s content will be announced no different than
multiple content blocks in the document flow. Set and show rules on one
of these elements do not affect the other.

A grid\textquotesingle s sizing is determined by the track sizes
specified in the arguments. Because each of the sizing parameters
accepts the same values, we will explain them just once, here. Each
sizing argument accepts an array of individual track sizes. A track size
is either:

\begin{itemize}
\item
  \texttt{\ }{\texttt{\ auto\ }}\texttt{\ } : The track will be sized to
  fit its contents. It will be at most as large as the remaining space.
  If there is more than one \texttt{\ }{\texttt{\ auto\ }}\texttt{\ }
  track width, and together they claim more than the available space,
  the \texttt{\ }{\texttt{\ auto\ }}\texttt{\ } tracks will fairly
  distribute the available space among themselves.
\item
  A fixed or relative length (e.g.
  \texttt{\ }{\texttt{\ 10pt\ }}\texttt{\ } or
  \texttt{\ }{\texttt{\ 20\%\ }}\texttt{\ }{\texttt{\ -\ }}\texttt{\ }{\texttt{\ 1cm\ }}\texttt{\ }
  ): The track will be exactly of this size.
\item
  A fractional length (e.g. \texttt{\ }{\texttt{\ 1fr\ }}\texttt{\ } ):
  Once all other tracks have been sized, the remaining space will be
  divided among the fractional tracks according to their fractions. For
  example, if there are two fractional tracks, each with a fraction of
  \texttt{\ }{\texttt{\ 1fr\ }}\texttt{\ } , they will each take up half
  of the remaining space.
\end{itemize}

To specify a single track, the array can be omitted in favor of a single
value. To specify multiple \texttt{\ }{\texttt{\ auto\ }}\texttt{\ }
tracks, enter the number of tracks instead of an array. For example,
\texttt{\ columns:\ } \texttt{\ }{\texttt{\ 3\ }}\texttt{\ } is
equivalent to \texttt{\ columns:\ }
\texttt{\ }{\texttt{\ (\ }}\texttt{\ }{\texttt{\ auto\ }}\texttt{\ }{\texttt{\ ,\ }}\texttt{\ }{\texttt{\ auto\ }}\texttt{\ }{\texttt{\ ,\ }}\texttt{\ }{\texttt{\ auto\ }}\texttt{\ }{\texttt{\ )\ }}\texttt{\ }
.

\subsection{Examples}\label{examples}

The example below demonstrates the different track sizing options. It
also shows how you can use
\href{/docs/reference/layout/grid/\#definitions-cell}{\texttt{\ grid.cell\ }}
to make an individual cell span two grid tracks.

\begin{verbatim}
// We use `rect` to emphasize the
// area of cells.
#set rect(
  inset: 8pt,
  fill: rgb("e4e5ea"),
  width: 100%,
)

#grid(
  columns: (60pt, 1fr, 2fr),
  rows: (auto, 60pt),
  gutter: 3pt,
  rect[Fixed width, auto height],
  rect[1/3 of the remains],
  rect[2/3 of the remains],
  rect(height: 100%)[Fixed height],
  grid.cell(
    colspan: 2,
    image("tiger.jpg", width: 100%),
  ),
)
\end{verbatim}

\includegraphics[width=5in,height=\textheight,keepaspectratio]{/assets/docs/nU6HFHUP8AJwyw_E8LwJrgAAAAAAAAAA.png}

You can also
\href{/docs/reference/foundations/arguments/\#spreading}{spread} an
array of strings or content into a grid to populate its cells.

\begin{verbatim}
#grid(
  columns: 5,
  gutter: 5pt,
  ..range(25).map(str)
)
\end{verbatim}

\includegraphics[width=5in,height=\textheight,keepaspectratio]{/assets/docs/qtEXI9WWslJNDT0wWvWAggAAAAAAAAAA.png}

\subsection{Styling the grid}\label{styling-the-grid}

The grid\textquotesingle s appearance can be customized through
different parameters. These are the most important ones:

\begin{itemize}
\tightlist
\item
  \href{/docs/reference/layout/grid/\#parameters-fill}{\texttt{\ fill\ }}
  to give all cells a background
\item
  \href{/docs/reference/layout/grid/\#parameters-align}{\texttt{\ align\ }}
  to change how cells are aligned
\item
  \href{/docs/reference/layout/grid/\#parameters-inset}{\texttt{\ inset\ }}
  to optionally add internal padding to each cell
\item
  \href{/docs/reference/layout/grid/\#parameters-stroke}{\texttt{\ stroke\ }}
  to optionally enable grid lines with a certain stroke
\end{itemize}

If you need to override one of the above options for a single cell, you
can use the
\href{/docs/reference/layout/grid/\#definitions-cell}{\texttt{\ grid.cell\ }}
element. Likewise, you can override individual grid lines with the
\href{/docs/reference/layout/grid/\#definitions-hline}{\texttt{\ grid.hline\ }}
and
\href{/docs/reference/layout/grid/\#definitions-vline}{\texttt{\ grid.vline\ }}
elements.

Alternatively, if you need the appearance options to depend on a
cell\textquotesingle s position (column and row), you may specify a
function to \texttt{\ fill\ } or \texttt{\ align\ } of the form
\texttt{\ (column,\ row)\ =\textgreater{}\ value\ } . You may also use a
show rule on
\href{/docs/reference/layout/grid/\#definitions-cell}{\texttt{\ grid.cell\ }}
- see that element\textquotesingle s examples or the examples below for
more information.

Locating most of your styling in set and show rules is recommended, as
it keeps the grid\textquotesingle s or table\textquotesingle s actual
usages clean and easy to read. It also allows you to easily change the
grid\textquotesingle s appearance in one place.

\subsubsection{Stroke styling
precedence}\label{stroke-styling-precedence}

There are three ways to set the stroke of a grid cell: through
\href{/docs/reference/layout/grid/\#definitions-cell-stroke}{\texttt{\ grid\ }{\texttt{\ .\ }}\texttt{\ cell\ }
\textquotesingle s \texttt{\ stroke\ } field} , by using
\href{/docs/reference/layout/grid/\#definitions-hline}{\texttt{\ grid\ }{\texttt{\ .\ }}\texttt{\ hline\ }}
and
\href{/docs/reference/layout/grid/\#definitions-vline}{\texttt{\ grid\ }{\texttt{\ .\ }}\texttt{\ vline\ }}
, or by setting the
\href{/docs/reference/layout/grid/\#parameters-stroke}{\texttt{\ grid\ }
\textquotesingle s \texttt{\ stroke\ } field} . When multiple of these
settings are present and conflict, the \texttt{\ hline\ } and
\texttt{\ vline\ } settings take the highest precedence, followed by the
\texttt{\ cell\ } settings, and finally the \texttt{\ grid\ } settings.

Furthermore, strokes of a repeated grid header or footer will take
precedence over regular cell strokes.

\subsection{\texorpdfstring{{ Parameters
}}{ Parameters }}\label{parameters}

\phantomsection\label{parameters-tooltip}
Parameters are the inputs to a function. They are specified in
parentheses after the function name.

{ grid } (

{ \hyperref[parameters-columns]{columns :}
\href{/docs/reference/foundations/auto/}{auto}
\href{/docs/reference/foundations/int/}{int}
\href{/docs/reference/layout/relative/}{relative}
\href{/docs/reference/layout/fraction/}{fraction}
\href{/docs/reference/foundations/array/}{array} , } {
\hyperref[parameters-rows]{rows :}
\href{/docs/reference/foundations/auto/}{auto}
\href{/docs/reference/foundations/int/}{int}
\href{/docs/reference/layout/relative/}{relative}
\href{/docs/reference/layout/fraction/}{fraction}
\href{/docs/reference/foundations/array/}{array} , } {
\hyperref[parameters-gutter]{gutter :}
\href{/docs/reference/foundations/auto/}{auto}
\href{/docs/reference/foundations/int/}{int}
\href{/docs/reference/layout/relative/}{relative}
\href{/docs/reference/layout/fraction/}{fraction}
\href{/docs/reference/foundations/array/}{array} , } {
\hyperref[parameters-column-gutter]{column-gutter :}
\href{/docs/reference/foundations/auto/}{auto}
\href{/docs/reference/foundations/int/}{int}
\href{/docs/reference/layout/relative/}{relative}
\href{/docs/reference/layout/fraction/}{fraction}
\href{/docs/reference/foundations/array/}{array} , } {
\hyperref[parameters-row-gutter]{row-gutter :}
\href{/docs/reference/foundations/auto/}{auto}
\href{/docs/reference/foundations/int/}{int}
\href{/docs/reference/layout/relative/}{relative}
\href{/docs/reference/layout/fraction/}{fraction}
\href{/docs/reference/foundations/array/}{array} , } {
\hyperref[parameters-fill]{fill :}
\href{/docs/reference/foundations/none/}{none}
\href{/docs/reference/visualize/color/}{color}
\href{/docs/reference/visualize/gradient/}{gradient}
\href{/docs/reference/foundations/array/}{array}
\href{/docs/reference/visualize/pattern/}{pattern}
\href{/docs/reference/foundations/function/}{function} , } {
\hyperref[parameters-align]{align :}
\href{/docs/reference/foundations/auto/}{auto}
\href{/docs/reference/foundations/array/}{array}
\href{/docs/reference/layout/alignment/}{alignment}
\href{/docs/reference/foundations/function/}{function} , } {
\hyperref[parameters-stroke]{stroke :}
\href{/docs/reference/foundations/none/}{none}
\href{/docs/reference/layout/length/}{length}
\href{/docs/reference/visualize/color/}{color}
\href{/docs/reference/visualize/gradient/}{gradient}
\href{/docs/reference/foundations/array/}{array}
\href{/docs/reference/visualize/stroke/}{stroke}
\href{/docs/reference/visualize/pattern/}{pattern}
\href{/docs/reference/foundations/dictionary/}{dictionary}
\href{/docs/reference/foundations/function/}{function} , } {
\hyperref[parameters-inset]{inset :}
\href{/docs/reference/layout/relative/}{relative}
\href{/docs/reference/foundations/array/}{array}
\href{/docs/reference/foundations/dictionary/}{dictionary}
\href{/docs/reference/foundations/function/}{function} , } {
\hyperref[parameters-children]{..}
\href{/docs/reference/foundations/content/}{content} , }

) -\textgreater{} \href{/docs/reference/foundations/content/}{content}

\subsubsection{\texorpdfstring{\texttt{\ columns\ }}{ columns }}\label{parameters-columns}

\href{/docs/reference/foundations/auto/}{auto} {or}
\href{/docs/reference/foundations/int/}{int} {or}
\href{/docs/reference/layout/relative/}{relative} {or}
\href{/docs/reference/layout/fraction/}{fraction} {or}
\href{/docs/reference/foundations/array/}{array}

{{ Settable }}

\phantomsection\label{parameters-columns-settable-tooltip}
Settable parameters can be customized for all following uses of the
function with a \texttt{\ set\ } rule.

The column sizes.

Either specify a track size array or provide an integer to create a grid
with that many \texttt{\ }{\texttt{\ auto\ }}\texttt{\ } -sized columns.
Note that opposed to rows and gutters, providing a single track size
will only ever create a single column.

Default:
\texttt{\ }{\texttt{\ (\ }}\texttt{\ }{\texttt{\ )\ }}\texttt{\ }

\subsubsection{\texorpdfstring{\texttt{\ rows\ }}{ rows }}\label{parameters-rows}

\href{/docs/reference/foundations/auto/}{auto} {or}
\href{/docs/reference/foundations/int/}{int} {or}
\href{/docs/reference/layout/relative/}{relative} {or}
\href{/docs/reference/layout/fraction/}{fraction} {or}
\href{/docs/reference/foundations/array/}{array}

{{ Settable }}

\phantomsection\label{parameters-rows-settable-tooltip}
Settable parameters can be customized for all following uses of the
function with a \texttt{\ set\ } rule.

The row sizes.

If there are more cells than fit the defined rows, the last row is
repeated until there are no more cells.

Default:
\texttt{\ }{\texttt{\ (\ }}\texttt{\ }{\texttt{\ )\ }}\texttt{\ }

\subsubsection{\texorpdfstring{\texttt{\ gutter\ }}{ gutter }}\label{parameters-gutter}

\href{/docs/reference/foundations/auto/}{auto} {or}
\href{/docs/reference/foundations/int/}{int} {or}
\href{/docs/reference/layout/relative/}{relative} {or}
\href{/docs/reference/layout/fraction/}{fraction} {or}
\href{/docs/reference/foundations/array/}{array}

{{ Settable }}

\phantomsection\label{parameters-gutter-settable-tooltip}
Settable parameters can be customized for all following uses of the
function with a \texttt{\ set\ } rule.

The gaps between rows and columns.

If there are more gutters than defined sizes, the last gutter is
repeated.

This is a shorthand to set \texttt{\ column-gutter\ } and
\texttt{\ row-gutter\ } to the same value.

Default:
\texttt{\ }{\texttt{\ (\ }}\texttt{\ }{\texttt{\ )\ }}\texttt{\ }

\subsubsection{\texorpdfstring{\texttt{\ column-gutter\ }}{ column-gutter }}\label{parameters-column-gutter}

\href{/docs/reference/foundations/auto/}{auto} {or}
\href{/docs/reference/foundations/int/}{int} {or}
\href{/docs/reference/layout/relative/}{relative} {or}
\href{/docs/reference/layout/fraction/}{fraction} {or}
\href{/docs/reference/foundations/array/}{array}

{{ Settable }}

\phantomsection\label{parameters-column-gutter-settable-tooltip}
Settable parameters can be customized for all following uses of the
function with a \texttt{\ set\ } rule.

The gaps between columns.

Default:
\texttt{\ }{\texttt{\ (\ }}\texttt{\ }{\texttt{\ )\ }}\texttt{\ }

\subsubsection{\texorpdfstring{\texttt{\ row-gutter\ }}{ row-gutter }}\label{parameters-row-gutter}

\href{/docs/reference/foundations/auto/}{auto} {or}
\href{/docs/reference/foundations/int/}{int} {or}
\href{/docs/reference/layout/relative/}{relative} {or}
\href{/docs/reference/layout/fraction/}{fraction} {or}
\href{/docs/reference/foundations/array/}{array}

{{ Settable }}

\phantomsection\label{parameters-row-gutter-settable-tooltip}
Settable parameters can be customized for all following uses of the
function with a \texttt{\ set\ } rule.

The gaps between rows.

Default:
\texttt{\ }{\texttt{\ (\ }}\texttt{\ }{\texttt{\ )\ }}\texttt{\ }

\subsubsection{\texorpdfstring{\texttt{\ fill\ }}{ fill }}\label{parameters-fill}

\href{/docs/reference/foundations/none/}{none} {or}
\href{/docs/reference/visualize/color/}{color} {or}
\href{/docs/reference/visualize/gradient/}{gradient} {or}
\href{/docs/reference/foundations/array/}{array} {or}
\href{/docs/reference/visualize/pattern/}{pattern} {or}
\href{/docs/reference/foundations/function/}{function}

{{ Settable }}

\phantomsection\label{parameters-fill-settable-tooltip}
Settable parameters can be customized for all following uses of the
function with a \texttt{\ set\ } rule.

How to fill the cells.

This can be a color or a function that returns a color. The function
receives the cells\textquotesingle{} column and row indices, starting
from zero. This can be used to implement striped grids.

Default: \texttt{\ }{\texttt{\ none\ }}\texttt{\ }

\includesvg[width=0.16667in,height=0.16667in]{/assets/icons/16-arrow-right.svg}
View example

\begin{verbatim}
#grid(
  fill: (x, y) =>
    if calc.even(x + y) { luma(230) }
    else { white },
  align: center + horizon,
  columns: 4,
  inset: 2pt,
  [X], [O], [X], [O],
  [O], [X], [O], [X],
  [X], [O], [X], [O],
  [O], [X], [O], [X],
)
\end{verbatim}

\includegraphics[width=5in,height=\textheight,keepaspectratio]{/assets/docs/YWpStHlSHlCZTmUmBJs9XQAAAAAAAAAA.png}

\subsubsection{\texorpdfstring{\texttt{\ align\ }}{ align }}\label{parameters-align}

\href{/docs/reference/foundations/auto/}{auto} {or}
\href{/docs/reference/foundations/array/}{array} {or}
\href{/docs/reference/layout/alignment/}{alignment} {or}
\href{/docs/reference/foundations/function/}{function}

{{ Settable }}

\phantomsection\label{parameters-align-settable-tooltip}
Settable parameters can be customized for all following uses of the
function with a \texttt{\ set\ } rule.

How to align the cells\textquotesingle{} content.

This can either be a single alignment, an array of alignments
(corresponding to each column) or a function that returns an alignment.
The function receives the cells\textquotesingle{} column and row
indices, starting from zero. If set to
\texttt{\ }{\texttt{\ auto\ }}\texttt{\ } , the outer alignment is used.

You can find an example for this argument at the
\href{/docs/reference/model/table/\#parameters-align}{\texttt{\ table.align\ }}
parameter.

Default: \texttt{\ }{\texttt{\ auto\ }}\texttt{\ }

\subsubsection{\texorpdfstring{\texttt{\ stroke\ }}{ stroke }}\label{parameters-stroke}

\href{/docs/reference/foundations/none/}{none} {or}
\href{/docs/reference/layout/length/}{length} {or}
\href{/docs/reference/visualize/color/}{color} {or}
\href{/docs/reference/visualize/gradient/}{gradient} {or}
\href{/docs/reference/foundations/array/}{array} {or}
\href{/docs/reference/visualize/stroke/}{stroke} {or}
\href{/docs/reference/visualize/pattern/}{pattern} {or}
\href{/docs/reference/foundations/dictionary/}{dictionary} {or}
\href{/docs/reference/foundations/function/}{function}

{{ Settable }}

\phantomsection\label{parameters-stroke-settable-tooltip}
Settable parameters can be customized for all following uses of the
function with a \texttt{\ set\ } rule.

How to \href{/docs/reference/visualize/stroke/}{stroke} the cells.

Grids have no strokes by default, which can be changed by setting this
option to the desired stroke.

If it is necessary to place lines which can cross spacing between cells
produced by the \texttt{\ gutter\ } option, or to override the stroke
between multiple specific cells, consider specifying one or more of
\href{/docs/reference/layout/grid/\#definitions-hline}{\texttt{\ grid.hline\ }}
and
\href{/docs/reference/layout/grid/\#definitions-vline}{\texttt{\ grid.vline\ }}
alongside your grid cells.

Default:
\texttt{\ }{\texttt{\ (\ }}\texttt{\ }{\texttt{\ :\ }}\texttt{\ }{\texttt{\ )\ }}\texttt{\ }

\includesvg[width=0.16667in,height=0.16667in]{/assets/icons/16-arrow-right.svg}
View example

\begin{verbatim}
#set page(height: 13em, width: 26em)

#let cv(..jobs) = grid(
    columns: 2,
    inset: 5pt,
    stroke: (x, y) => if x == 0 and y > 0 {
      (right: (
        paint: luma(180),
        thickness: 1.5pt,
        dash: "dotted"
      ))
    },
    grid.header(grid.cell(colspan: 2)[
      *Professional Experience*
      #box(width: 1fr, line(length: 100%, stroke: luma(180)))
    ]),
    ..{
      let last = none
      for job in jobs.pos() {
        (
          if job.year != last [*#job.year*],
          [
            *#job.company* - #job.role _(#job.timeframe)_ \
            #job.details
          ]
        )
        last = job.year
      }
    }
  )

  #cv(
    (
      year: 2012,
      company: [Pear Seed & Co.],
      role: [Lead Engineer],
      timeframe: [Jul - Dec],
      details: [
        - Raised engineers from 3x to 10x
        - Did a great job
      ],
    ),
    (
      year: 2012,
      company: [Mega Corp.],
      role: [VP of Sales],
      timeframe: [Mar - Jun],
      details: [- Closed tons of customers],
    ),
    (
      year: 2013,
      company: [Tiny Co.],
      role: [CEO],
      timeframe: [Jan - Dec],
      details: [- Delivered 4x more shareholder value],
    ),
    (
      year: 2014,
      company: [Glorbocorp Ltd],
      role: [CTO],
      timeframe: [Jan - Mar],
      details: [- Drove containerization forward],
    ),
  )
\end{verbatim}

\includegraphics[width=5.95833in,height=\textheight,keepaspectratio]{/assets/docs/5kfvlcbAPUFkWJtXr3FdMgAAAAAAAAAA.png}
\includegraphics[width=5.95833in,height=\textheight,keepaspectratio]{/assets/docs/5kfvlcbAPUFkWJtXr3FdMgAAAAAAAAAB.png}

\subsubsection{\texorpdfstring{\texttt{\ inset\ }}{ inset }}\label{parameters-inset}

\href{/docs/reference/layout/relative/}{relative} {or}
\href{/docs/reference/foundations/array/}{array} {or}
\href{/docs/reference/foundations/dictionary/}{dictionary} {or}
\href{/docs/reference/foundations/function/}{function}

{{ Settable }}

\phantomsection\label{parameters-inset-settable-tooltip}
Settable parameters can be customized for all following uses of the
function with a \texttt{\ set\ } rule.

How much to pad the cells\textquotesingle{} content.

You can find an example for this argument at the
\href{/docs/reference/model/table/\#parameters-inset}{\texttt{\ table.inset\ }}
parameter.

Default:
\texttt{\ }{\texttt{\ (\ }}\texttt{\ }{\texttt{\ :\ }}\texttt{\ }{\texttt{\ )\ }}\texttt{\ }

\subsubsection{\texorpdfstring{\texttt{\ children\ }}{ children }}\label{parameters-children}

\href{/docs/reference/foundations/content/}{content}

{Required} {{ Positional }}

\phantomsection\label{parameters-children-positional-tooltip}
Positional parameters are specified in order, without names.

{{ Variadic }}

\phantomsection\label{parameters-children-variadic-tooltip}
Variadic parameters can be specified multiple times.

The contents of the grid cells, plus any extra grid lines specified with
the
\href{/docs/reference/layout/grid/\#definitions-hline}{\texttt{\ grid.hline\ }}
and
\href{/docs/reference/layout/grid/\#definitions-vline}{\texttt{\ grid.vline\ }}
elements.

The cells are populated in row-major order.

\subsection{\texorpdfstring{{ Definitions
}}{ Definitions }}\label{definitions}

\phantomsection\label{definitions-tooltip}
Functions and types and can have associated definitions. These are
accessed by specifying the function or type, followed by a period, and
then the definition\textquotesingle s name.

\subsubsection{\texorpdfstring{\texttt{\ cell\ } {{ Element
}}}{ cell   Element }}\label{definitions-cell}

\phantomsection\label{definitions-cell-element-tooltip}
Element functions can be customized with \texttt{\ set\ } and
\texttt{\ show\ } rules.

A cell in the grid. You can use this function in the argument list of a
grid to override grid style properties for an individual cell or
manually positioning it within the grid. You can also use this function
in show rules to apply certain styles to multiple cells at once.

For example, you can override the position and stroke for a single cell:

grid { . } { cell } (

{ \href{/docs/reference/foundations/content/}{content} , } {
\hyperref[definitions-cell-parameters-x]{x :}
\href{/docs/reference/foundations/auto/}{auto}
\href{/docs/reference/foundations/int/}{int} , } {
\hyperref[definitions-cell-parameters-y]{y :}
\href{/docs/reference/foundations/auto/}{auto}
\href{/docs/reference/foundations/int/}{int} , } {
\hyperref[definitions-cell-parameters-colspan]{colspan :}
\href{/docs/reference/foundations/int/}{int} , } {
\hyperref[definitions-cell-parameters-rowspan]{rowspan :}
\href{/docs/reference/foundations/int/}{int} , } {
\hyperref[definitions-cell-parameters-fill]{fill :}
\href{/docs/reference/foundations/none/}{none}
\href{/docs/reference/foundations/auto/}{auto}
\href{/docs/reference/visualize/color/}{color}
\href{/docs/reference/visualize/gradient/}{gradient}
\href{/docs/reference/visualize/pattern/}{pattern} , } {
\hyperref[definitions-cell-parameters-align]{align :}
\href{/docs/reference/foundations/auto/}{auto}
\href{/docs/reference/layout/alignment/}{alignment} , } {
\hyperref[definitions-cell-parameters-inset]{inset :}
\href{/docs/reference/foundations/auto/}{auto}
\href{/docs/reference/layout/relative/}{relative}
\href{/docs/reference/foundations/dictionary/}{dictionary} , } {
\hyperref[definitions-cell-parameters-stroke]{stroke :}
\href{/docs/reference/foundations/none/}{none}
\href{/docs/reference/layout/length/}{length}
\href{/docs/reference/visualize/color/}{color}
\href{/docs/reference/visualize/gradient/}{gradient}
\href{/docs/reference/visualize/stroke/}{stroke}
\href{/docs/reference/visualize/pattern/}{pattern}
\href{/docs/reference/foundations/dictionary/}{dictionary} , } {
\hyperref[definitions-cell-parameters-breakable]{breakable :}
\href{/docs/reference/foundations/auto/}{auto}
\href{/docs/reference/foundations/bool/}{bool} , }

) -\textgreater{} \href{/docs/reference/foundations/content/}{content}

\begin{verbatim}
#set text(15pt, font: "Noto Sans Symbols 2")
#show regex("[♚-♟︎]"): set text(fill: rgb("21212A"))
#show regex("[♔-♙]"): set text(fill: rgb("111015"))

#grid(
  fill: (x, y) => rgb(
    if calc.odd(x + y) { "7F8396" }
    else { "EFF0F3" }
  ),
  columns: (1em,) * 8,
  rows: 1em,
  align: center + horizon,

  [♖], [♘], [♗], [♕], [♔], [♗], [♘], [♖],
  [♙], [♙], [♙], [♙], [],  [♙], [♙], [♙],
  grid.cell(
    x: 4, y: 3,
    stroke: blue.transparentize(60%)
  )[♙],

  ..(grid.cell(y: 6)[♟],) * 8,
  ..([♜], [♞], [♝], [♛], [♚], [♝], [♞], [♜])
    .map(grid.cell.with(y: 7)),
)
\end{verbatim}

\includegraphics[width=3.125in,height=\textheight,keepaspectratio]{/assets/docs/hagMogxzgYo1z-9CqYbmiQAAAAAAAAAA.png}

You may also apply a show rule on \texttt{\ grid.cell\ } to style all
cells at once, which allows you, for example, to apply styles based on a
cell\textquotesingle s position. Refer to the examples of the
\href{/docs/reference/model/table/\#definitions-cell}{\texttt{\ table.cell\ }}
element to learn more about this.

\paragraph{\texorpdfstring{\texttt{\ body\ }}{ body }}\label{definitions-cell-body}

\href{/docs/reference/foundations/content/}{content}

{Required} {{ Positional }}

\phantomsection\label{definitions-cell-body-positional-tooltip}
Positional parameters are specified in order, without names.

The cell\textquotesingle s body.

\paragraph{\texorpdfstring{\texttt{\ x\ }}{ x }}\label{definitions-cell-x}

\href{/docs/reference/foundations/auto/}{auto} {or}
\href{/docs/reference/foundations/int/}{int}

{{ Settable }}

\phantomsection\label{definitions-cell-x-settable-tooltip}
Settable parameters can be customized for all following uses of the
function with a \texttt{\ set\ } rule.

The cell\textquotesingle s column (zero-indexed). This field may be used
in show rules to style a cell depending on its column.

You may override this field to pick in which column the cell must be
placed. If no row ( \texttt{\ y\ } ) is chosen, the cell will be placed
in the first row (starting at row 0) with that column available (or a
new row if none). If both \texttt{\ x\ } and \texttt{\ y\ } are chosen,
however, the cell will be placed in that exact position. An error is
raised if that position is not available (thus, it is usually wise to
specify cells with a custom position before cells with automatic
positions).

Default: \texttt{\ }{\texttt{\ auto\ }}\texttt{\ }

\includesvg[width=0.16667in,height=0.16667in]{/assets/icons/16-arrow-right.svg}
View example

\begin{verbatim}
#let circ(c) = circle(
    fill: c, width: 5mm
)

#grid(
  columns: 4,
  rows: 7mm,
  stroke: .5pt + blue,
  align: center + horizon,
  inset: 1mm,

  grid.cell(x: 2, y: 2, circ(aqua)),
  circ(yellow),
  grid.cell(x: 3, circ(green)),
  circ(black),
)
\end{verbatim}

\includegraphics[width=5in,height=\textheight,keepaspectratio]{/assets/docs/1ClWJM7tWFhsIyNZJlD1owAAAAAAAAAA.png}

\paragraph{\texorpdfstring{\texttt{\ y\ }}{ y }}\label{definitions-cell-y}

\href{/docs/reference/foundations/auto/}{auto} {or}
\href{/docs/reference/foundations/int/}{int}

{{ Settable }}

\phantomsection\label{definitions-cell-y-settable-tooltip}
Settable parameters can be customized for all following uses of the
function with a \texttt{\ set\ } rule.

The cell\textquotesingle s row (zero-indexed). This field may be used in
show rules to style a cell depending on its row.

You may override this field to pick in which row the cell must be
placed. If no column ( \texttt{\ x\ } ) is chosen, the cell will be
placed in the first column (starting at column 0) available in the
chosen row. If all columns in the chosen row are already occupied, an
error is raised.

Default: \texttt{\ }{\texttt{\ auto\ }}\texttt{\ }

\includesvg[width=0.16667in,height=0.16667in]{/assets/icons/16-arrow-right.svg}
View example

\begin{verbatim}
#let tri(c) = polygon.regular(
  fill: c,
  size: 5mm,
  vertices: 3,
)

#grid(
  columns: 2,
  stroke: blue,
  inset: 1mm,

  tri(black),
  grid.cell(y: 1, tri(teal)),
  grid.cell(y: 1, tri(red)),
  grid.cell(y: 2, tri(orange))
)
\end{verbatim}

\includegraphics[width=5in,height=\textheight,keepaspectratio]{/assets/docs/KqESjHcjVY-CskMVImXGSAAAAAAAAAAA.png}

\paragraph{\texorpdfstring{\texttt{\ colspan\ }}{ colspan }}\label{definitions-cell-colspan}

\href{/docs/reference/foundations/int/}{int}

{{ Settable }}

\phantomsection\label{definitions-cell-colspan-settable-tooltip}
Settable parameters can be customized for all following uses of the
function with a \texttt{\ set\ } rule.

The amount of columns spanned by this cell.

Default: \texttt{\ }{\texttt{\ 1\ }}\texttt{\ }

\paragraph{\texorpdfstring{\texttt{\ rowspan\ }}{ rowspan }}\label{definitions-cell-rowspan}

\href{/docs/reference/foundations/int/}{int}

{{ Settable }}

\phantomsection\label{definitions-cell-rowspan-settable-tooltip}
Settable parameters can be customized for all following uses of the
function with a \texttt{\ set\ } rule.

The amount of rows spanned by this cell.

Default: \texttt{\ }{\texttt{\ 1\ }}\texttt{\ }

\paragraph{\texorpdfstring{\texttt{\ fill\ }}{ fill }}\label{definitions-cell-fill}

\href{/docs/reference/foundations/none/}{none} {or}
\href{/docs/reference/foundations/auto/}{auto} {or}
\href{/docs/reference/visualize/color/}{color} {or}
\href{/docs/reference/visualize/gradient/}{gradient} {or}
\href{/docs/reference/visualize/pattern/}{pattern}

{{ Settable }}

\phantomsection\label{definitions-cell-fill-settable-tooltip}
Settable parameters can be customized for all following uses of the
function with a \texttt{\ set\ } rule.

The cell\textquotesingle s
\href{/docs/reference/layout/grid/\#parameters-fill}{fill} override.

Default: \texttt{\ }{\texttt{\ auto\ }}\texttt{\ }

\paragraph{\texorpdfstring{\texttt{\ align\ }}{ align }}\label{definitions-cell-align}

\href{/docs/reference/foundations/auto/}{auto} {or}
\href{/docs/reference/layout/alignment/}{alignment}

{{ Settable }}

\phantomsection\label{definitions-cell-align-settable-tooltip}
Settable parameters can be customized for all following uses of the
function with a \texttt{\ set\ } rule.

The cell\textquotesingle s
\href{/docs/reference/layout/grid/\#parameters-align}{alignment}
override.

Default: \texttt{\ }{\texttt{\ auto\ }}\texttt{\ }

\paragraph{\texorpdfstring{\texttt{\ inset\ }}{ inset }}\label{definitions-cell-inset}

\href{/docs/reference/foundations/auto/}{auto} {or}
\href{/docs/reference/layout/relative/}{relative} {or}
\href{/docs/reference/foundations/dictionary/}{dictionary}

{{ Settable }}

\phantomsection\label{definitions-cell-inset-settable-tooltip}
Settable parameters can be customized for all following uses of the
function with a \texttt{\ set\ } rule.

The cell\textquotesingle s
\href{/docs/reference/layout/grid/\#parameters-inset}{inset} override.

Default: \texttt{\ }{\texttt{\ auto\ }}\texttt{\ }

\paragraph{\texorpdfstring{\texttt{\ stroke\ }}{ stroke }}\label{definitions-cell-stroke}

\href{/docs/reference/foundations/none/}{none} {or}
\href{/docs/reference/layout/length/}{length} {or}
\href{/docs/reference/visualize/color/}{color} {or}
\href{/docs/reference/visualize/gradient/}{gradient} {or}
\href{/docs/reference/visualize/stroke/}{stroke} {or}
\href{/docs/reference/visualize/pattern/}{pattern} {or}
\href{/docs/reference/foundations/dictionary/}{dictionary}

{{ Settable }}

\phantomsection\label{definitions-cell-stroke-settable-tooltip}
Settable parameters can be customized for all following uses of the
function with a \texttt{\ set\ } rule.

The cell\textquotesingle s
\href{/docs/reference/layout/grid/\#parameters-stroke}{stroke} override.

Default:
\texttt{\ }{\texttt{\ (\ }}\texttt{\ }{\texttt{\ :\ }}\texttt{\ }{\texttt{\ )\ }}\texttt{\ }

\paragraph{\texorpdfstring{\texttt{\ breakable\ }}{ breakable }}\label{definitions-cell-breakable}

\href{/docs/reference/foundations/auto/}{auto} {or}
\href{/docs/reference/foundations/bool/}{bool}

{{ Settable }}

\phantomsection\label{definitions-cell-breakable-settable-tooltip}
Settable parameters can be customized for all following uses of the
function with a \texttt{\ set\ } rule.

Whether rows spanned by this cell can be placed in different pages. When
equal to \texttt{\ }{\texttt{\ auto\ }}\texttt{\ } , a cell spanning
only fixed-size rows is unbreakable, while a cell spanning at least one
\texttt{\ }{\texttt{\ auto\ }}\texttt{\ } -sized row is breakable.

Default: \texttt{\ }{\texttt{\ auto\ }}\texttt{\ }

\subsubsection{\texorpdfstring{\texttt{\ hline\ } {{ Element
}}}{ hline   Element }}\label{definitions-hline}

\phantomsection\label{definitions-hline-element-tooltip}
Element functions can be customized with \texttt{\ set\ } and
\texttt{\ show\ } rules.

A horizontal line in the grid.

Overrides any per-cell stroke, including stroke specified through the
grid\textquotesingle s \texttt{\ stroke\ } field. Can cross spacing
between cells created through the grid\textquotesingle s
\texttt{\ column-gutter\ } option.

An example for this function can be found at the
\href{/docs/reference/model/table/\#definitions-hline}{\texttt{\ table.hline\ }}
element.

grid { . } { hline } (

{ \hyperref[definitions-hline-parameters-y]{y :}
\href{/docs/reference/foundations/auto/}{auto}
\href{/docs/reference/foundations/int/}{int} , } {
\hyperref[definitions-hline-parameters-start]{start :}
\href{/docs/reference/foundations/int/}{int} , } {
\hyperref[definitions-hline-parameters-end]{end :}
\href{/docs/reference/foundations/none/}{none}
\href{/docs/reference/foundations/int/}{int} , } {
\hyperref[definitions-hline-parameters-stroke]{stroke :}
\href{/docs/reference/foundations/none/}{none}
\href{/docs/reference/layout/length/}{length}
\href{/docs/reference/visualize/color/}{color}
\href{/docs/reference/visualize/gradient/}{gradient}
\href{/docs/reference/visualize/stroke/}{stroke}
\href{/docs/reference/visualize/pattern/}{pattern}
\href{/docs/reference/foundations/dictionary/}{dictionary} , } {
\hyperref[definitions-hline-parameters-position]{position :}
\href{/docs/reference/layout/alignment/}{alignment} , }

) -\textgreater{} \href{/docs/reference/foundations/content/}{content}

\paragraph{\texorpdfstring{\texttt{\ y\ }}{ y }}\label{definitions-hline-y}

\href{/docs/reference/foundations/auto/}{auto} {or}
\href{/docs/reference/foundations/int/}{int}

{{ Settable }}

\phantomsection\label{definitions-hline-y-settable-tooltip}
Settable parameters can be customized for all following uses of the
function with a \texttt{\ set\ } rule.

The row above which the horizontal line is placed (zero-indexed). If the
\texttt{\ position\ } field is set to \texttt{\ bottom\ } , the line is
placed below the row with the given index instead (see that
field\textquotesingle s docs for details).

Specifying \texttt{\ }{\texttt{\ auto\ }}\texttt{\ } causes the line to
be placed at the row below the last automatically positioned cell (that
is, cell without coordinate overrides) before the line among the
grid\textquotesingle s children. If there is no such cell before the
line, it is placed at the top of the grid (row 0). Note that specifying
for this option exactly the total amount of rows in the grid causes this
horizontal line to override the bottom border of the grid, while a value
of 0 overrides the top border.

Default: \texttt{\ }{\texttt{\ auto\ }}\texttt{\ }

\paragraph{\texorpdfstring{\texttt{\ start\ }}{ start }}\label{definitions-hline-start}

\href{/docs/reference/foundations/int/}{int}

{{ Settable }}

\phantomsection\label{definitions-hline-start-settable-tooltip}
Settable parameters can be customized for all following uses of the
function with a \texttt{\ set\ } rule.

The column at which the horizontal line starts (zero-indexed,
inclusive).

Default: \texttt{\ }{\texttt{\ 0\ }}\texttt{\ }

\paragraph{\texorpdfstring{\texttt{\ end\ }}{ end }}\label{definitions-hline-end}

\href{/docs/reference/foundations/none/}{none} {or}
\href{/docs/reference/foundations/int/}{int}

{{ Settable }}

\phantomsection\label{definitions-hline-end-settable-tooltip}
Settable parameters can be customized for all following uses of the
function with a \texttt{\ set\ } rule.

The column before which the horizontal line ends (zero-indexed,
exclusive). Therefore, the horizontal line will be drawn up to and
across column \texttt{\ end\ -\ 1\ } .

A value equal to \texttt{\ }{\texttt{\ none\ }}\texttt{\ } or to the
amount of columns causes it to extend all the way towards the end of the
grid.

Default: \texttt{\ }{\texttt{\ none\ }}\texttt{\ }

\paragraph{\texorpdfstring{\texttt{\ stroke\ }}{ stroke }}\label{definitions-hline-stroke}

\href{/docs/reference/foundations/none/}{none} {or}
\href{/docs/reference/layout/length/}{length} {or}
\href{/docs/reference/visualize/color/}{color} {or}
\href{/docs/reference/visualize/gradient/}{gradient} {or}
\href{/docs/reference/visualize/stroke/}{stroke} {or}
\href{/docs/reference/visualize/pattern/}{pattern} {or}
\href{/docs/reference/foundations/dictionary/}{dictionary}

{{ Settable }}

\phantomsection\label{definitions-hline-stroke-settable-tooltip}
Settable parameters can be customized for all following uses of the
function with a \texttt{\ set\ } rule.

The line\textquotesingle s stroke.

Specifying \texttt{\ }{\texttt{\ none\ }}\texttt{\ } removes any lines
previously placed across this line\textquotesingle s range, including
hlines or per-cell stroke below it.

Default:
\texttt{\ }{\texttt{\ 1pt\ }}\texttt{\ }{\texttt{\ +\ }}\texttt{\ black\ }

\paragraph{\texorpdfstring{\texttt{\ position\ }}{ position }}\label{definitions-hline-position}

\href{/docs/reference/layout/alignment/}{alignment}

{{ Settable }}

\phantomsection\label{definitions-hline-position-settable-tooltip}
Settable parameters can be customized for all following uses of the
function with a \texttt{\ set\ } rule.

The position at which the line is placed, given its row ( \texttt{\ y\ }
) - either \texttt{\ top\ } to draw above it or \texttt{\ bottom\ } to
draw below it.

This setting is only relevant when row gutter is enabled (and
shouldn\textquotesingle t be used otherwise - prefer just increasing the
\texttt{\ y\ } field by one instead), since then the position below a
row becomes different from the position above the next row due to the
spacing between both.

Default: \texttt{\ top\ }

\subsubsection{\texorpdfstring{\texttt{\ vline\ } {{ Element
}}}{ vline   Element }}\label{definitions-vline}

\phantomsection\label{definitions-vline-element-tooltip}
Element functions can be customized with \texttt{\ set\ } and
\texttt{\ show\ } rules.

A vertical line in the grid.

Overrides any per-cell stroke, including stroke specified through the
grid\textquotesingle s \texttt{\ stroke\ } field. Can cross spacing
between cells created through the grid\textquotesingle s
\texttt{\ row-gutter\ } option.

grid { . } { vline } (

{ \hyperref[definitions-vline-parameters-x]{x :}
\href{/docs/reference/foundations/auto/}{auto}
\href{/docs/reference/foundations/int/}{int} , } {
\hyperref[definitions-vline-parameters-start]{start :}
\href{/docs/reference/foundations/int/}{int} , } {
\hyperref[definitions-vline-parameters-end]{end :}
\href{/docs/reference/foundations/none/}{none}
\href{/docs/reference/foundations/int/}{int} , } {
\hyperref[definitions-vline-parameters-stroke]{stroke :}
\href{/docs/reference/foundations/none/}{none}
\href{/docs/reference/layout/length/}{length}
\href{/docs/reference/visualize/color/}{color}
\href{/docs/reference/visualize/gradient/}{gradient}
\href{/docs/reference/visualize/stroke/}{stroke}
\href{/docs/reference/visualize/pattern/}{pattern}
\href{/docs/reference/foundations/dictionary/}{dictionary} , } {
\hyperref[definitions-vline-parameters-position]{position :}
\href{/docs/reference/layout/alignment/}{alignment} , }

) -\textgreater{} \href{/docs/reference/foundations/content/}{content}

\paragraph{\texorpdfstring{\texttt{\ x\ }}{ x }}\label{definitions-vline-x}

\href{/docs/reference/foundations/auto/}{auto} {or}
\href{/docs/reference/foundations/int/}{int}

{{ Settable }}

\phantomsection\label{definitions-vline-x-settable-tooltip}
Settable parameters can be customized for all following uses of the
function with a \texttt{\ set\ } rule.

The column before which the horizontal line is placed (zero-indexed). If
the \texttt{\ position\ } field is set to \texttt{\ end\ } , the line is
placed after the column with the given index instead (see that
field\textquotesingle s docs for details).

Specifying \texttt{\ }{\texttt{\ auto\ }}\texttt{\ } causes the line to
be placed at the column after the last automatically positioned cell
(that is, cell without coordinate overrides) before the line among the
grid\textquotesingle s children. If there is no such cell before the
line, it is placed before the grid\textquotesingle s first column
(column 0). Note that specifying for this option exactly the total
amount of columns in the grid causes this vertical line to override the
end border of the grid (right in LTR, left in RTL), while a value of 0
overrides the start border (left in LTR, right in RTL).

Default: \texttt{\ }{\texttt{\ auto\ }}\texttt{\ }

\paragraph{\texorpdfstring{\texttt{\ start\ }}{ start }}\label{definitions-vline-start}

\href{/docs/reference/foundations/int/}{int}

{{ Settable }}

\phantomsection\label{definitions-vline-start-settable-tooltip}
Settable parameters can be customized for all following uses of the
function with a \texttt{\ set\ } rule.

The row at which the vertical line starts (zero-indexed, inclusive).

Default: \texttt{\ }{\texttt{\ 0\ }}\texttt{\ }

\paragraph{\texorpdfstring{\texttt{\ end\ }}{ end }}\label{definitions-vline-end}

\href{/docs/reference/foundations/none/}{none} {or}
\href{/docs/reference/foundations/int/}{int}

{{ Settable }}

\phantomsection\label{definitions-vline-end-settable-tooltip}
Settable parameters can be customized for all following uses of the
function with a \texttt{\ set\ } rule.

The row on top of which the vertical line ends (zero-indexed,
exclusive). Therefore, the vertical line will be drawn up to and across
row \texttt{\ end\ -\ 1\ } .

A value equal to \texttt{\ }{\texttt{\ none\ }}\texttt{\ } or to the
amount of rows causes it to extend all the way towards the bottom of the
grid.

Default: \texttt{\ }{\texttt{\ none\ }}\texttt{\ }

\paragraph{\texorpdfstring{\texttt{\ stroke\ }}{ stroke }}\label{definitions-vline-stroke}

\href{/docs/reference/foundations/none/}{none} {or}
\href{/docs/reference/layout/length/}{length} {or}
\href{/docs/reference/visualize/color/}{color} {or}
\href{/docs/reference/visualize/gradient/}{gradient} {or}
\href{/docs/reference/visualize/stroke/}{stroke} {or}
\href{/docs/reference/visualize/pattern/}{pattern} {or}
\href{/docs/reference/foundations/dictionary/}{dictionary}

{{ Settable }}

\phantomsection\label{definitions-vline-stroke-settable-tooltip}
Settable parameters can be customized for all following uses of the
function with a \texttt{\ set\ } rule.

The line\textquotesingle s stroke.

Specifying \texttt{\ }{\texttt{\ none\ }}\texttt{\ } removes any lines
previously placed across this line\textquotesingle s range, including
vlines or per-cell stroke below it.

Default:
\texttt{\ }{\texttt{\ 1pt\ }}\texttt{\ }{\texttt{\ +\ }}\texttt{\ black\ }

\paragraph{\texorpdfstring{\texttt{\ position\ }}{ position }}\label{definitions-vline-position}

\href{/docs/reference/layout/alignment/}{alignment}

{{ Settable }}

\phantomsection\label{definitions-vline-position-settable-tooltip}
Settable parameters can be customized for all following uses of the
function with a \texttt{\ set\ } rule.

The position at which the line is placed, given its column (
\texttt{\ x\ } ) - either \texttt{\ start\ } to draw before it or
\texttt{\ end\ } to draw after it.

The values \texttt{\ left\ } and \texttt{\ right\ } are also accepted,
but discouraged as they cause your grid to be inconsistent between
left-to-right and right-to-left documents.

This setting is only relevant when column gutter is enabled (and
shouldn\textquotesingle t be used otherwise - prefer just increasing the
\texttt{\ x\ } field by one instead), since then the position after a
column becomes different from the position before the next column due to
the spacing between both.

Default: \texttt{\ start\ }

\subsubsection{\texorpdfstring{\texttt{\ header\ } {{ Element
}}}{ header   Element }}\label{definitions-header}

\phantomsection\label{definitions-header-element-tooltip}
Element functions can be customized with \texttt{\ set\ } and
\texttt{\ show\ } rules.

A repeatable grid header.

If \texttt{\ repeat\ } is set to \texttt{\ true\ } , the header will be
repeated across pages. For an example, refer to the
\href{/docs/reference/model/table/\#definitions-header}{\texttt{\ table.header\ }}
element and the
\href{/docs/reference/layout/grid/\#parameters-stroke}{\texttt{\ grid.stroke\ }}
parameter.

grid { . } { header } (

{ \hyperref[definitions-header-parameters-repeat]{repeat :}
\href{/docs/reference/foundations/bool/}{bool} , } {
\hyperref[definitions-header-parameters-children]{..}
\href{/docs/reference/foundations/content/}{content} , }

) -\textgreater{} \href{/docs/reference/foundations/content/}{content}

\paragraph{\texorpdfstring{\texttt{\ repeat\ }}{ repeat }}\label{definitions-header-repeat}

\href{/docs/reference/foundations/bool/}{bool}

{{ Settable }}

\phantomsection\label{definitions-header-repeat-settable-tooltip}
Settable parameters can be customized for all following uses of the
function with a \texttt{\ set\ } rule.

Whether this header should be repeated across pages.

Default: \texttt{\ }{\texttt{\ true\ }}\texttt{\ }

\paragraph{\texorpdfstring{\texttt{\ children\ }}{ children }}\label{definitions-header-children}

\href{/docs/reference/foundations/content/}{content}

{Required} {{ Positional }}

\phantomsection\label{definitions-header-children-positional-tooltip}
Positional parameters are specified in order, without names.

{{ Variadic }}

\phantomsection\label{definitions-header-children-variadic-tooltip}
Variadic parameters can be specified multiple times.

The cells and lines within the header.

\subsubsection{\texorpdfstring{\texttt{\ footer\ } {{ Element
}}}{ footer   Element }}\label{definitions-footer}

\phantomsection\label{definitions-footer-element-tooltip}
Element functions can be customized with \texttt{\ set\ } and
\texttt{\ show\ } rules.

A repeatable grid footer.

Just like the
\href{/docs/reference/layout/grid/\#definitions-header}{\texttt{\ grid.header\ }}
element, the footer can repeat itself on every page of the table.

No other grid cells may be placed after the footer.

grid { . } { footer } (

{ \hyperref[definitions-footer-parameters-repeat]{repeat :}
\href{/docs/reference/foundations/bool/}{bool} , } {
\hyperref[definitions-footer-parameters-children]{..}
\href{/docs/reference/foundations/content/}{content} , }

) -\textgreater{} \href{/docs/reference/foundations/content/}{content}

\paragraph{\texorpdfstring{\texttt{\ repeat\ }}{ repeat }}\label{definitions-footer-repeat}

\href{/docs/reference/foundations/bool/}{bool}

{{ Settable }}

\phantomsection\label{definitions-footer-repeat-settable-tooltip}
Settable parameters can be customized for all following uses of the
function with a \texttt{\ set\ } rule.

Whether this footer should be repeated across pages.

Default: \texttt{\ }{\texttt{\ true\ }}\texttt{\ }

\paragraph{\texorpdfstring{\texttt{\ children\ }}{ children }}\label{definitions-footer-children}

\href{/docs/reference/foundations/content/}{content}

{Required} {{ Positional }}

\phantomsection\label{definitions-footer-children-positional-tooltip}
Positional parameters are specified in order, without names.

{{ Variadic }}

\phantomsection\label{definitions-footer-children-variadic-tooltip}
Variadic parameters can be specified multiple times.

The cells and lines within the footer.

\href{/docs/reference/layout/fraction/}{\pandocbounded{\includesvg[keepaspectratio]{/assets/icons/16-arrow-right.svg}}}

{ Fraction } { Previous page }

\href{/docs/reference/layout/hide/}{\pandocbounded{\includesvg[keepaspectratio]{/assets/icons/16-arrow-right.svg}}}

{ Hide } { Next page }


\section{Docs LaTeX/typst.app/docs/reference/layout/columns.tex}
\title{typst.app/docs/reference/layout/columns}

\begin{itemize}
\tightlist
\item
  \href{/docs}{\includesvg[width=0.16667in,height=0.16667in]{/assets/icons/16-docs-dark.svg}}
\item
  \includesvg[width=0.16667in,height=0.16667in]{/assets/icons/16-arrow-right.svg}
\item
  \href{/docs/reference/}{Reference}
\item
  \includesvg[width=0.16667in,height=0.16667in]{/assets/icons/16-arrow-right.svg}
\item
  \href{/docs/reference/layout/}{Layout}
\item
  \includesvg[width=0.16667in,height=0.16667in]{/assets/icons/16-arrow-right.svg}
\item
  \href{/docs/reference/layout/columns/}{Columns}
\end{itemize}

\section{\texorpdfstring{\texttt{\ columns\ } {{ Element
}}}{ columns   Element }}\label{summary}

\phantomsection\label{element-tooltip}
Element functions can be customized with \texttt{\ set\ } and
\texttt{\ show\ } rules.

Separates a region into multiple equally sized columns.

The \texttt{\ column\ } function lets you separate the interior of any
container into multiple columns. It will currently not balance the
height of the columns. Instead, the columns will take up the height of
their container or the remaining height on the page. Support for
balanced columns is planned for the future.

\subsection{Page-level columns}\label{page-level}

If you need to insert columns across your whole document, use the
\texttt{\ page\ } function\textquotesingle s
\href{/docs/reference/layout/page/\#parameters-columns}{\texttt{\ columns\ }
parameter} instead. This will create the columns directly at the
page-level rather than wrapping all of your content in a layout
container. As a result, things like
\href{/docs/reference/layout/pagebreak/}{pagebreaks} ,
\href{/docs/reference/model/footnote/}{footnotes} , and
\href{/docs/reference/model/par/\#definitions-line}{line numbers} will
continue to work as expected. For more information, also read the
\href{/docs/guides/page-setup-guide/\#columns}{relevant part of the page
setup guide} .

\subsection{Breaking out of columns}\label{breaking-out}

To temporarily break out of columns (e.g. for a paper\textquotesingle s
title), use parent-scoped floating placement:

\begin{verbatim}
#set page(columns: 2, height: 150pt)

#place(
  top + center,
  scope: "parent",
  float: true,
  text(1.4em, weight: "bold")[
    My document
  ],
)

#lorem(40)
\end{verbatim}

\includegraphics[width=5in,height=\textheight,keepaspectratio]{/assets/docs/qNRmHdtNgs8qpE-RUR-XyQAAAAAAAAAA.png}

\subsection{\texorpdfstring{{ Parameters
}}{ Parameters }}\label{parameters}

\phantomsection\label{parameters-tooltip}
Parameters are the inputs to a function. They are specified in
parentheses after the function name.

{ columns } (

{ \hyperref[parameters-count]{}
\href{/docs/reference/foundations/int/}{int} , } {
\hyperref[parameters-gutter]{gutter :}
\href{/docs/reference/layout/relative/}{relative} , } {
\href{/docs/reference/foundations/content/}{content} , }

) -\textgreater{} \href{/docs/reference/foundations/content/}{content}

\subsubsection{\texorpdfstring{\texttt{\ count\ }}{ count }}\label{parameters-count}

\href{/docs/reference/foundations/int/}{int}

{{ Positional }}

\phantomsection\label{parameters-count-positional-tooltip}
Positional parameters are specified in order, without names.

{{ Settable }}

\phantomsection\label{parameters-count-settable-tooltip}
Settable parameters can be customized for all following uses of the
function with a \texttt{\ set\ } rule.

The number of columns.

Default: \texttt{\ }{\texttt{\ 2\ }}\texttt{\ }

\subsubsection{\texorpdfstring{\texttt{\ gutter\ }}{ gutter }}\label{parameters-gutter}

\href{/docs/reference/layout/relative/}{relative}

{{ Settable }}

\phantomsection\label{parameters-gutter-settable-tooltip}
Settable parameters can be customized for all following uses of the
function with a \texttt{\ set\ } rule.

The size of the gutter space between each column.

Default:
\texttt{\ }{\texttt{\ 4\%\ }}\texttt{\ }{\texttt{\ +\ }}\texttt{\ }{\texttt{\ 0pt\ }}\texttt{\ }

\subsubsection{\texorpdfstring{\texttt{\ body\ }}{ body }}\label{parameters-body}

\href{/docs/reference/foundations/content/}{content}

{Required} {{ Positional }}

\phantomsection\label{parameters-body-positional-tooltip}
Positional parameters are specified in order, without names.

The content that should be layouted into the columns.

\href{/docs/reference/layout/colbreak/}{\pandocbounded{\includesvg[keepaspectratio]{/assets/icons/16-arrow-right.svg}}}

{ Column Break } { Previous page }

\href{/docs/reference/layout/direction/}{\pandocbounded{\includesvg[keepaspectratio]{/assets/icons/16-arrow-right.svg}}}

{ Direction } { Next page }


\section{Docs LaTeX/typst.app/docs/reference/layout/move.tex}
\title{typst.app/docs/reference/layout/move}

\begin{itemize}
\tightlist
\item
  \href{/docs}{\includesvg[width=0.16667in,height=0.16667in]{/assets/icons/16-docs-dark.svg}}
\item
  \includesvg[width=0.16667in,height=0.16667in]{/assets/icons/16-arrow-right.svg}
\item
  \href{/docs/reference/}{Reference}
\item
  \includesvg[width=0.16667in,height=0.16667in]{/assets/icons/16-arrow-right.svg}
\item
  \href{/docs/reference/layout/}{Layout}
\item
  \includesvg[width=0.16667in,height=0.16667in]{/assets/icons/16-arrow-right.svg}
\item
  \href{/docs/reference/layout/move/}{Move}
\end{itemize}

\section{\texorpdfstring{\texttt{\ move\ } {{ Element
}}}{ move   Element }}\label{summary}

\phantomsection\label{element-tooltip}
Element functions can be customized with \texttt{\ set\ } and
\texttt{\ show\ } rules.

Moves content without affecting layout.

The \texttt{\ move\ } function allows you to move content while the
layout still \textquotesingle sees\textquotesingle{} it at the original
positions. Containers will still be sized as if the content was not
moved.

\subsection{Example}\label{example}

\begin{verbatim}
#rect(inset: 0pt, move(
  dx: 6pt, dy: 6pt,
  rect(
    inset: 8pt,
    fill: white,
    stroke: black,
    [Abra cadabra]
  )
))
\end{verbatim}

\includegraphics[width=5in,height=\textheight,keepaspectratio]{/assets/docs/1MdBh-uXG6kGRG6DYdlcJAAAAAAAAAAA.png}

\subsection{\texorpdfstring{{ Parameters
}}{ Parameters }}\label{parameters}

\phantomsection\label{parameters-tooltip}
Parameters are the inputs to a function. They are specified in
parentheses after the function name.

{ move } (

{ \hyperref[parameters-dx]{dx :}
\href{/docs/reference/layout/relative/}{relative} , } {
\hyperref[parameters-dy]{dy :}
\href{/docs/reference/layout/relative/}{relative} , } {
\href{/docs/reference/foundations/content/}{content} , }

) -\textgreater{} \href{/docs/reference/foundations/content/}{content}

\subsubsection{\texorpdfstring{\texttt{\ dx\ }}{ dx }}\label{parameters-dx}

\href{/docs/reference/layout/relative/}{relative}

{{ Settable }}

\phantomsection\label{parameters-dx-settable-tooltip}
Settable parameters can be customized for all following uses of the
function with a \texttt{\ set\ } rule.

The horizontal displacement of the content.

Default:
\texttt{\ }{\texttt{\ 0\%\ }}\texttt{\ }{\texttt{\ +\ }}\texttt{\ }{\texttt{\ 0pt\ }}\texttt{\ }

\subsubsection{\texorpdfstring{\texttt{\ dy\ }}{ dy }}\label{parameters-dy}

\href{/docs/reference/layout/relative/}{relative}

{{ Settable }}

\phantomsection\label{parameters-dy-settable-tooltip}
Settable parameters can be customized for all following uses of the
function with a \texttt{\ set\ } rule.

The vertical displacement of the content.

Default:
\texttt{\ }{\texttt{\ 0\%\ }}\texttt{\ }{\texttt{\ +\ }}\texttt{\ }{\texttt{\ 0pt\ }}\texttt{\ }

\subsubsection{\texorpdfstring{\texttt{\ body\ }}{ body }}\label{parameters-body}

\href{/docs/reference/foundations/content/}{content}

{Required} {{ Positional }}

\phantomsection\label{parameters-body-positional-tooltip}
Positional parameters are specified in order, without names.

The content to move.

\href{/docs/reference/layout/measure/}{\pandocbounded{\includesvg[keepaspectratio]{/assets/icons/16-arrow-right.svg}}}

{ Measure } { Previous page }

\href{/docs/reference/layout/pad/}{\pandocbounded{\includesvg[keepaspectratio]{/assets/icons/16-arrow-right.svg}}}

{ Padding } { Next page }


\section{Docs LaTeX/typst.app/docs/reference/layout/skew.tex}
\title{typst.app/docs/reference/layout/skew}

\begin{itemize}
\tightlist
\item
  \href{/docs}{\includesvg[width=0.16667in,height=0.16667in]{/assets/icons/16-docs-dark.svg}}
\item
  \includesvg[width=0.16667in,height=0.16667in]{/assets/icons/16-arrow-right.svg}
\item
  \href{/docs/reference/}{Reference}
\item
  \includesvg[width=0.16667in,height=0.16667in]{/assets/icons/16-arrow-right.svg}
\item
  \href{/docs/reference/layout/}{Layout}
\item
  \includesvg[width=0.16667in,height=0.16667in]{/assets/icons/16-arrow-right.svg}
\item
  \href{/docs/reference/layout/skew/}{Skew}
\end{itemize}

\section{\texorpdfstring{\texttt{\ skew\ } {{ Element
}}}{ skew   Element }}\label{summary}

\phantomsection\label{element-tooltip}
Element functions can be customized with \texttt{\ set\ } and
\texttt{\ show\ } rules.

Skews content.

Skews an element in horizontal and/or vertical direction. The layout
will act as if the element was not skewed unless you specify
\texttt{\ reflow:\ }{\texttt{\ true\ }}\texttt{\ } .

\subsection{Example}\label{example}

\begin{verbatim}
#skew(ax: -12deg)[
  This is some fake italic text.
]
\end{verbatim}

\includegraphics[width=5in,height=\textheight,keepaspectratio]{/assets/docs/FUtSyVs-Ma5rvUP8B0w5fQAAAAAAAAAA.png}

\subsection{\texorpdfstring{{ Parameters
}}{ Parameters }}\label{parameters}

\phantomsection\label{parameters-tooltip}
Parameters are the inputs to a function. They are specified in
parentheses after the function name.

{ skew } (

{ \hyperref[parameters-ax]{ax :}
\href{/docs/reference/layout/angle/}{angle} , } {
\hyperref[parameters-ay]{ay :}
\href{/docs/reference/layout/angle/}{angle} , } {
\hyperref[parameters-origin]{origin :}
\href{/docs/reference/layout/alignment/}{alignment} , } {
\hyperref[parameters-reflow]{reflow :}
\href{/docs/reference/foundations/bool/}{bool} , } {
\href{/docs/reference/foundations/content/}{content} , }

) -\textgreater{} \href{/docs/reference/foundations/content/}{content}

\subsubsection{\texorpdfstring{\texttt{\ ax\ }}{ ax }}\label{parameters-ax}

\href{/docs/reference/layout/angle/}{angle}

{{ Settable }}

\phantomsection\label{parameters-ax-settable-tooltip}
Settable parameters can be customized for all following uses of the
function with a \texttt{\ set\ } rule.

The horizontal skewing angle.

Default: \texttt{\ }{\texttt{\ 0deg\ }}\texttt{\ }

\includesvg[width=0.16667in,height=0.16667in]{/assets/icons/16-arrow-right.svg}
View example

\begin{verbatim}
#skew(ax: 30deg)[Skewed]
\end{verbatim}

\includegraphics[width=5in,height=\textheight,keepaspectratio]{/assets/docs/H9k2hlR_HYwp5MND40z3rgAAAAAAAAAA.png}

\subsubsection{\texorpdfstring{\texttt{\ ay\ }}{ ay }}\label{parameters-ay}

\href{/docs/reference/layout/angle/}{angle}

{{ Settable }}

\phantomsection\label{parameters-ay-settable-tooltip}
Settable parameters can be customized for all following uses of the
function with a \texttt{\ set\ } rule.

The vertical skewing angle.

Default: \texttt{\ }{\texttt{\ 0deg\ }}\texttt{\ }

\includesvg[width=0.16667in,height=0.16667in]{/assets/icons/16-arrow-right.svg}
View example

\begin{verbatim}
#skew(ay: 30deg)[Skewed]
\end{verbatim}

\includegraphics[width=5in,height=\textheight,keepaspectratio]{/assets/docs/DIs5kgGdkepXxpgHWt0vxAAAAAAAAAAA.png}

\subsubsection{\texorpdfstring{\texttt{\ origin\ }}{ origin }}\label{parameters-origin}

\href{/docs/reference/layout/alignment/}{alignment}

{{ Settable }}

\phantomsection\label{parameters-origin-settable-tooltip}
Settable parameters can be customized for all following uses of the
function with a \texttt{\ set\ } rule.

The origin of the skew transformation.

The origin will stay fixed during the operation.

Default: \texttt{\ center\ }{\texttt{\ +\ }}\texttt{\ horizon\ }

\includesvg[width=0.16667in,height=0.16667in]{/assets/icons/16-arrow-right.svg}
View example

\begin{verbatim}
X #box(skew(ax: -30deg, origin: center + horizon)[X]) X \
X #box(skew(ax: -30deg, origin: bottom + left)[X]) X \
X #box(skew(ax: -30deg, origin: top + right)[X]) X
\end{verbatim}

\includegraphics[width=5in,height=\textheight,keepaspectratio]{/assets/docs/2Hq4GFYS1tSqCnluz3jbcQAAAAAAAAAA.png}

\subsubsection{\texorpdfstring{\texttt{\ reflow\ }}{ reflow }}\label{parameters-reflow}

\href{/docs/reference/foundations/bool/}{bool}

{{ Settable }}

\phantomsection\label{parameters-reflow-settable-tooltip}
Settable parameters can be customized for all following uses of the
function with a \texttt{\ set\ } rule.

Whether the skew transformation impacts the layout.

If set to \texttt{\ }{\texttt{\ false\ }}\texttt{\ } , the skewed
content will retain the bounding box of the original content. If set to
\texttt{\ }{\texttt{\ true\ }}\texttt{\ } , the bounding box will take
the transformation of the content into account and adjust the layout
accordingly.

Default: \texttt{\ }{\texttt{\ false\ }}\texttt{\ }

\includesvg[width=0.16667in,height=0.16667in]{/assets/icons/16-arrow-right.svg}
View example

\begin{verbatim}
Hello #skew(ay: 30deg, reflow: true, "World")!
\end{verbatim}

\includegraphics[width=5in,height=\textheight,keepaspectratio]{/assets/docs/-k-PUuRezD-q6j7vk-xQWAAAAAAAAAAA.png}

\subsubsection{\texorpdfstring{\texttt{\ body\ }}{ body }}\label{parameters-body}

\href{/docs/reference/foundations/content/}{content}

{Required} {{ Positional }}

\phantomsection\label{parameters-body-positional-tooltip}
Positional parameters are specified in order, without names.

The content to skew.

\href{/docs/reference/layout/scale/}{\pandocbounded{\includesvg[keepaspectratio]{/assets/icons/16-arrow-right.svg}}}

{ Scale } { Previous page }

\href{/docs/reference/layout/h/}{\pandocbounded{\includesvg[keepaspectratio]{/assets/icons/16-arrow-right.svg}}}

{ Spacing (H) } { Next page }


\section{Docs LaTeX/typst.app/docs/reference/layout/align.tex}
\title{typst.app/docs/reference/layout/align}

\begin{itemize}
\tightlist
\item
  \href{/docs}{\includesvg[width=0.16667in,height=0.16667in]{/assets/icons/16-docs-dark.svg}}
\item
  \includesvg[width=0.16667in,height=0.16667in]{/assets/icons/16-arrow-right.svg}
\item
  \href{/docs/reference/}{Reference}
\item
  \includesvg[width=0.16667in,height=0.16667in]{/assets/icons/16-arrow-right.svg}
\item
  \href{/docs/reference/layout/}{Layout}
\item
  \includesvg[width=0.16667in,height=0.16667in]{/assets/icons/16-arrow-right.svg}
\item
  \href{/docs/reference/layout/align/}{Align}
\end{itemize}

\section{\texorpdfstring{\texttt{\ align\ } {{ Element
}}}{ align   Element }}\label{summary}

\phantomsection\label{element-tooltip}
Element functions can be customized with \texttt{\ set\ } and
\texttt{\ show\ } rules.

Aligns content horizontally and vertically.

\subsection{Example}\label{example}

Let\textquotesingle s start with centering our content horizontally:

\begin{verbatim}
#set page(height: 120pt)
#set align(center)

Centered text, a sight to see \
In perfect balance, visually \
Not left nor right, it stands alone \
A work of art, a visual throne
\end{verbatim}

\includegraphics[width=5in,height=\textheight,keepaspectratio]{/assets/docs/kcNIG-bYA8T9BUDnjCUJGgAAAAAAAAAA.png}

To center something vertically, use \emph{horizon} alignment:

\begin{verbatim}
#set page(height: 120pt)
#set align(horizon)

Vertically centered, \
the stage had entered, \
a new paragraph.
\end{verbatim}

\includegraphics[width=5in,height=\textheight,keepaspectratio]{/assets/docs/y9OO-MSDQIHWsGPc_6pNnAAAAAAAAAAA.png}

\subsection{Combining alignments}\label{combining-alignments}

You can combine two alignments with the \texttt{\ +\ } operator.
Let\textquotesingle s also only apply this to one piece of content by
using the function form instead of a set rule:

\begin{verbatim}
#set page(height: 120pt)
Though left in the beginning ...

#align(right + bottom)[
  ... they were right in the end, \
  and with addition had gotten, \
  the paragraph to the bottom!
]
\end{verbatim}

\includegraphics[width=5in,height=\textheight,keepaspectratio]{/assets/docs/gXaqAMYC8Licj_UCK0JSFgAAAAAAAAAA.png}

\subsection{Nested alignment}\label{nested-alignment}

You can use varying alignments for layout containers and the elements
within them. This way, you can create intricate layouts:

\begin{verbatim}
#align(center, block[
  #set align(left)
  Though centered together \
  alone \
  we \
  are \
  left.
])
\end{verbatim}

\includegraphics[width=5in,height=\textheight,keepaspectratio]{/assets/docs/B6Y-WWFtiUjCHNJ9B8R8vQAAAAAAAAAA.png}

\subsection{Alignment within the same
line}\label{alignment-within-the-same-line}

The \texttt{\ align\ } function performs block-level alignment and thus
always interrupts the current paragraph. To have different alignment for
parts of the same line, you should use
\href{/docs/reference/layout/h/}{fractional spacing} instead:

\begin{verbatim}
Start #h(1fr) End
\end{verbatim}

\includegraphics[width=5in,height=\textheight,keepaspectratio]{/assets/docs/jlafwbE2ZuISwJPQNRzA3gAAAAAAAAAA.png}

\subsection{\texorpdfstring{{ Parameters
}}{ Parameters }}\label{parameters}

\phantomsection\label{parameters-tooltip}
Parameters are the inputs to a function. They are specified in
parentheses after the function name.

{ align } (

{ \hyperref[parameters-alignment]{}
\href{/docs/reference/layout/alignment/}{alignment} , } {
\href{/docs/reference/foundations/content/}{content} , }

) -\textgreater{} \href{/docs/reference/foundations/content/}{content}

\subsubsection{\texorpdfstring{\texttt{\ alignment\ }}{ alignment }}\label{parameters-alignment}

\href{/docs/reference/layout/alignment/}{alignment}

{{ Positional }}

\phantomsection\label{parameters-alignment-positional-tooltip}
Positional parameters are specified in order, without names.

{{ Settable }}

\phantomsection\label{parameters-alignment-settable-tooltip}
Settable parameters can be customized for all following uses of the
function with a \texttt{\ set\ } rule.

The \href{/docs/reference/layout/alignment/}{alignment} along both axes.

Default: \texttt{\ start\ }{\texttt{\ +\ }}\texttt{\ top\ }

\includesvg[width=0.16667in,height=0.16667in]{/assets/icons/16-arrow-right.svg}
View example

\begin{verbatim}
#set page(height: 6cm)
#set text(lang: "ar")

مثال
#align(
  end + horizon,
  rect(inset: 12pt)[ركن]
)
\end{verbatim}

\includegraphics[width=5in,height=\textheight,keepaspectratio]{/assets/docs/3176vm6IE_BNfZrVpc9_xAAAAAAAAAAA.png}

\subsubsection{\texorpdfstring{\texttt{\ body\ }}{ body }}\label{parameters-body}

\href{/docs/reference/foundations/content/}{content}

{Required} {{ Positional }}

\phantomsection\label{parameters-body-positional-tooltip}
Positional parameters are specified in order, without names.

The content to align.

\href{/docs/reference/layout/}{\pandocbounded{\includesvg[keepaspectratio]{/assets/icons/16-arrow-right.svg}}}

{ Layout } { Previous page }

\href{/docs/reference/layout/alignment/}{\pandocbounded{\includesvg[keepaspectratio]{/assets/icons/16-arrow-right.svg}}}

{ Alignment } { Next page }


\section{Docs LaTeX/typst.app/docs/reference/layout/rotate.tex}
\title{typst.app/docs/reference/layout/rotate}

\begin{itemize}
\tightlist
\item
  \href{/docs}{\includesvg[width=0.16667in,height=0.16667in]{/assets/icons/16-docs-dark.svg}}
\item
  \includesvg[width=0.16667in,height=0.16667in]{/assets/icons/16-arrow-right.svg}
\item
  \href{/docs/reference/}{Reference}
\item
  \includesvg[width=0.16667in,height=0.16667in]{/assets/icons/16-arrow-right.svg}
\item
  \href{/docs/reference/layout/}{Layout}
\item
  \includesvg[width=0.16667in,height=0.16667in]{/assets/icons/16-arrow-right.svg}
\item
  \href{/docs/reference/layout/rotate/}{Rotate}
\end{itemize}

\section{\texorpdfstring{\texttt{\ rotate\ } {{ Element
}}}{ rotate   Element }}\label{summary}

\phantomsection\label{element-tooltip}
Element functions can be customized with \texttt{\ set\ } and
\texttt{\ show\ } rules.

Rotates content without affecting layout.

Rotates an element by a given angle. The layout will act as if the
element was not rotated unless you specify
\texttt{\ reflow:\ }{\texttt{\ true\ }}\texttt{\ } .

\subsection{Example}\label{example}

\begin{verbatim}
#stack(
  dir: ltr,
  spacing: 1fr,
  ..range(16)
    .map(i => rotate(24deg * i)[X]),
)
\end{verbatim}

\includegraphics[width=5in,height=\textheight,keepaspectratio]{/assets/docs/KRNlJxFzPXxwMsKBe0vSFQAAAAAAAAAA.png}

\subsection{\texorpdfstring{{ Parameters
}}{ Parameters }}\label{parameters}

\phantomsection\label{parameters-tooltip}
Parameters are the inputs to a function. They are specified in
parentheses after the function name.

{ rotate } (

{ \hyperref[parameters-angle]{}
\href{/docs/reference/layout/angle/}{angle} , } {
\hyperref[parameters-origin]{origin :}
\href{/docs/reference/layout/alignment/}{alignment} , } {
\hyperref[parameters-reflow]{reflow :}
\href{/docs/reference/foundations/bool/}{bool} , } {
\href{/docs/reference/foundations/content/}{content} , }

) -\textgreater{} \href{/docs/reference/foundations/content/}{content}

\subsubsection{\texorpdfstring{\texttt{\ angle\ }}{ angle }}\label{parameters-angle}

\href{/docs/reference/layout/angle/}{angle}

{{ Positional }}

\phantomsection\label{parameters-angle-positional-tooltip}
Positional parameters are specified in order, without names.

{{ Settable }}

\phantomsection\label{parameters-angle-settable-tooltip}
Settable parameters can be customized for all following uses of the
function with a \texttt{\ set\ } rule.

The amount of rotation.

Default: \texttt{\ }{\texttt{\ 0deg\ }}\texttt{\ }

\includesvg[width=0.16667in,height=0.16667in]{/assets/icons/16-arrow-right.svg}
View example

\begin{verbatim}
#rotate(-1.571rad)[Space!]
\end{verbatim}

\includegraphics[width=5in,height=\textheight,keepaspectratio]{/assets/docs/_kx75fW11u8TY_Zj6luytwAAAAAAAAAA.png}

\subsubsection{\texorpdfstring{\texttt{\ origin\ }}{ origin }}\label{parameters-origin}

\href{/docs/reference/layout/alignment/}{alignment}

{{ Settable }}

\phantomsection\label{parameters-origin-settable-tooltip}
Settable parameters can be customized for all following uses of the
function with a \texttt{\ set\ } rule.

The origin of the rotation.

If, for instance, you wanted the bottom left corner of the rotated
element to stay aligned with the baseline, you would set it to
\texttt{\ bottom\ +\ left\ } instead.

Default: \texttt{\ center\ }{\texttt{\ +\ }}\texttt{\ horizon\ }

\includesvg[width=0.16667in,height=0.16667in]{/assets/icons/16-arrow-right.svg}
View example

\begin{verbatim}
#set text(spacing: 8pt)
#let square = square.with(width: 8pt)

#box(square())
#box(rotate(30deg, origin: center, square()))
#box(rotate(30deg, origin: top + left, square()))
#box(rotate(30deg, origin: bottom + right, square()))
\end{verbatim}

\includegraphics[width=5in,height=\textheight,keepaspectratio]{/assets/docs/ZzBCk0ymiIeT5xo4XXc-8QAAAAAAAAAA.png}

\subsubsection{\texorpdfstring{\texttt{\ reflow\ }}{ reflow }}\label{parameters-reflow}

\href{/docs/reference/foundations/bool/}{bool}

{{ Settable }}

\phantomsection\label{parameters-reflow-settable-tooltip}
Settable parameters can be customized for all following uses of the
function with a \texttt{\ set\ } rule.

Whether the rotation impacts the layout.

If set to \texttt{\ }{\texttt{\ false\ }}\texttt{\ } , the rotated
content will retain the bounding box of the original content. If set to
\texttt{\ }{\texttt{\ true\ }}\texttt{\ } , the bounding box will take
the rotation of the content into account and adjust the layout
accordingly.

Default: \texttt{\ }{\texttt{\ false\ }}\texttt{\ }

\includesvg[width=0.16667in,height=0.16667in]{/assets/icons/16-arrow-right.svg}
View example

\begin{verbatim}
Hello #rotate(90deg, reflow: true)[World]!
\end{verbatim}

\includegraphics[width=5in,height=\textheight,keepaspectratio]{/assets/docs/i8AMp2vxmKn3Nn0wwA1Z0wAAAAAAAAAA.png}

\subsubsection{\texorpdfstring{\texttt{\ body\ }}{ body }}\label{parameters-body}

\href{/docs/reference/foundations/content/}{content}

{Required} {{ Positional }}

\phantomsection\label{parameters-body-positional-tooltip}
Positional parameters are specified in order, without names.

The content to rotate.

\href{/docs/reference/layout/repeat/}{\pandocbounded{\includesvg[keepaspectratio]{/assets/icons/16-arrow-right.svg}}}

{ Repeat } { Previous page }

\href{/docs/reference/layout/scale/}{\pandocbounded{\includesvg[keepaspectratio]{/assets/icons/16-arrow-right.svg}}}

{ Scale } { Next page }


\section{Docs LaTeX/typst.app/docs/reference/layout/angle.tex}
\title{typst.app/docs/reference/layout/angle}

\begin{itemize}
\tightlist
\item
  \href{/docs}{\includesvg[width=0.16667in,height=0.16667in]{/assets/icons/16-docs-dark.svg}}
\item
  \includesvg[width=0.16667in,height=0.16667in]{/assets/icons/16-arrow-right.svg}
\item
  \href{/docs/reference/}{Reference}
\item
  \includesvg[width=0.16667in,height=0.16667in]{/assets/icons/16-arrow-right.svg}
\item
  \href{/docs/reference/layout/}{Layout}
\item
  \includesvg[width=0.16667in,height=0.16667in]{/assets/icons/16-arrow-right.svg}
\item
  \href{/docs/reference/layout/angle/}{Angle}
\end{itemize}

\section{\texorpdfstring{{ angle }}{ angle }}\label{summary}

An angle describing a rotation.

Typst supports the following angular units:

\begin{itemize}
\tightlist
\item
  Degrees: \texttt{\ }{\texttt{\ 180deg\ }}\texttt{\ }
\item
  Radians: \texttt{\ }{\texttt{\ 3.14rad\ }}\texttt{\ }
\end{itemize}

\subsection{Example}\label{example}

\begin{verbatim}
#rotate(10deg)[Hello there!]
\end{verbatim}

\includegraphics[width=5in,height=\textheight,keepaspectratio]{/assets/docs/bDyrcLTzr2eRmGWeZRN2_QAAAAAAAAAA.png}

\subsection{\texorpdfstring{{ Definitions
}}{ Definitions }}\label{definitions}

\phantomsection\label{definitions-tooltip}
Functions and types and can have associated definitions. These are
accessed by specifying the function or type, followed by a period, and
then the definition\textquotesingle s name.

\subsubsection{\texorpdfstring{\texttt{\ rad\ }}{ rad }}\label{definitions-rad}

Converts this angle to radians.

self { . } { rad } (

) -\textgreater{} \href{/docs/reference/foundations/float/}{float}

\subsubsection{\texorpdfstring{\texttt{\ deg\ }}{ deg }}\label{definitions-deg}

Converts this angle to degrees.

self { . } { deg } (

) -\textgreater{} \href{/docs/reference/foundations/float/}{float}

\href{/docs/reference/layout/alignment/}{\pandocbounded{\includesvg[keepaspectratio]{/assets/icons/16-arrow-right.svg}}}

{ Alignment } { Previous page }

\href{/docs/reference/layout/block/}{\pandocbounded{\includesvg[keepaspectratio]{/assets/icons/16-arrow-right.svg}}}

{ Block } { Next page }


\section{Docs LaTeX/typst.app/docs/reference/layout/colbreak.tex}
\title{typst.app/docs/reference/layout/colbreak}

\begin{itemize}
\tightlist
\item
  \href{/docs}{\includesvg[width=0.16667in,height=0.16667in]{/assets/icons/16-docs-dark.svg}}
\item
  \includesvg[width=0.16667in,height=0.16667in]{/assets/icons/16-arrow-right.svg}
\item
  \href{/docs/reference/}{Reference}
\item
  \includesvg[width=0.16667in,height=0.16667in]{/assets/icons/16-arrow-right.svg}
\item
  \href{/docs/reference/layout/}{Layout}
\item
  \includesvg[width=0.16667in,height=0.16667in]{/assets/icons/16-arrow-right.svg}
\item
  \href{/docs/reference/layout/colbreak/}{Column Break}
\end{itemize}

\section{\texorpdfstring{\texttt{\ colbreak\ } {{ Element
}}}{ colbreak   Element }}\label{summary}

\phantomsection\label{element-tooltip}
Element functions can be customized with \texttt{\ set\ } and
\texttt{\ show\ } rules.

Forces a column break.

The function will behave like a
\href{/docs/reference/layout/pagebreak/}{page break} when used in a
single column layout or the last column on a page. Otherwise, content
after the column break will be placed in the next column.

\subsection{Example}\label{example}

\begin{verbatim}
#set page(columns: 2)
Preliminary findings from our
ongoing research project have
revealed a hitherto unknown
phenomenon of extraordinary
significance.

#colbreak()
Through rigorous experimentation
and analysis, we have discovered
a hitherto uncharacterized process
that defies our current
understanding of the fundamental
laws of nature.
\end{verbatim}

\includegraphics[width=5in,height=\textheight,keepaspectratio]{/assets/docs/MXyldqpQM7MpLi9gC6sPGAAAAAAAAAAA.png}

\subsection{\texorpdfstring{{ Parameters
}}{ Parameters }}\label{parameters}

\phantomsection\label{parameters-tooltip}
Parameters are the inputs to a function. They are specified in
parentheses after the function name.

{ colbreak } (

{ \hyperref[parameters-weak]{weak :}
\href{/docs/reference/foundations/bool/}{bool} }

) -\textgreater{} \href{/docs/reference/foundations/content/}{content}

\subsubsection{\texorpdfstring{\texttt{\ weak\ }}{ weak }}\label{parameters-weak}

\href{/docs/reference/foundations/bool/}{bool}

{{ Settable }}

\phantomsection\label{parameters-weak-settable-tooltip}
Settable parameters can be customized for all following uses of the
function with a \texttt{\ set\ } rule.

If \texttt{\ }{\texttt{\ true\ }}\texttt{\ } , the column break is
skipped if the current column is already empty.

Default: \texttt{\ }{\texttt{\ false\ }}\texttt{\ }

\href{/docs/reference/layout/box/}{\pandocbounded{\includesvg[keepaspectratio]{/assets/icons/16-arrow-right.svg}}}

{ Box } { Previous page }

\href{/docs/reference/layout/columns/}{\pandocbounded{\includesvg[keepaspectratio]{/assets/icons/16-arrow-right.svg}}}

{ Columns } { Next page }


\section{Docs LaTeX/typst.app/docs/reference/layout/scale.tex}
\title{typst.app/docs/reference/layout/scale}

\begin{itemize}
\tightlist
\item
  \href{/docs}{\includesvg[width=0.16667in,height=0.16667in]{/assets/icons/16-docs-dark.svg}}
\item
  \includesvg[width=0.16667in,height=0.16667in]{/assets/icons/16-arrow-right.svg}
\item
  \href{/docs/reference/}{Reference}
\item
  \includesvg[width=0.16667in,height=0.16667in]{/assets/icons/16-arrow-right.svg}
\item
  \href{/docs/reference/layout/}{Layout}
\item
  \includesvg[width=0.16667in,height=0.16667in]{/assets/icons/16-arrow-right.svg}
\item
  \href{/docs/reference/layout/scale/}{Scale}
\end{itemize}

\section{\texorpdfstring{\texttt{\ scale\ } {{ Element
}}}{ scale   Element }}\label{summary}

\phantomsection\label{element-tooltip}
Element functions can be customized with \texttt{\ set\ } and
\texttt{\ show\ } rules.

Scales content without affecting layout.

Lets you mirror content by specifying a negative scale on a single axis.

\subsection{Example}\label{example}

\begin{verbatim}
#set align(center)
#scale(x: -100%)[This is mirrored.]
#scale(x: -100%, reflow: true)[This is mirrored.]
\end{verbatim}

\includegraphics[width=5in,height=\textheight,keepaspectratio]{/assets/docs/ShH8NomqhuEYrrdUbApjaAAAAAAAAAAA.png}

\subsection{\texorpdfstring{{ Parameters
}}{ Parameters }}\label{parameters}

\phantomsection\label{parameters-tooltip}
Parameters are the inputs to a function. They are specified in
parentheses after the function name.

{ scale } (

{ \hyperref[parameters-factor]{}
\href{/docs/reference/foundations/auto/}{auto}
\href{/docs/reference/layout/length/}{length}
\href{/docs/reference/layout/ratio/}{ratio} , } {
\hyperref[parameters-x]{x :}
\href{/docs/reference/foundations/auto/}{auto}
\href{/docs/reference/layout/length/}{length}
\href{/docs/reference/layout/ratio/}{ratio} , } {
\hyperref[parameters-y]{y :}
\href{/docs/reference/foundations/auto/}{auto}
\href{/docs/reference/layout/length/}{length}
\href{/docs/reference/layout/ratio/}{ratio} , } {
\hyperref[parameters-origin]{origin :}
\href{/docs/reference/layout/alignment/}{alignment} , } {
\hyperref[parameters-reflow]{reflow :}
\href{/docs/reference/foundations/bool/}{bool} , } {
\href{/docs/reference/foundations/content/}{content} , }

) -\textgreater{} \href{/docs/reference/foundations/content/}{content}

\subsubsection{\texorpdfstring{\texttt{\ factor\ }}{ factor }}\label{parameters-factor}

\href{/docs/reference/foundations/auto/}{auto} {or}
\href{/docs/reference/layout/length/}{length} {or}
\href{/docs/reference/layout/ratio/}{ratio}

{{ Positional }}

\phantomsection\label{parameters-factor-positional-tooltip}
Positional parameters are specified in order, without names.

{{ Settable }}

\phantomsection\label{parameters-factor-settable-tooltip}
Settable parameters can be customized for all following uses of the
function with a \texttt{\ set\ } rule.

The scaling factor for both axes, as a positional argument. This is just
an optional shorthand notation for setting \texttt{\ x\ } and
\texttt{\ y\ } to the same value.

Default: \texttt{\ }{\texttt{\ 100\%\ }}\texttt{\ }

\subsubsection{\texorpdfstring{\texttt{\ x\ }}{ x }}\label{parameters-x}

\href{/docs/reference/foundations/auto/}{auto} {or}
\href{/docs/reference/layout/length/}{length} {or}
\href{/docs/reference/layout/ratio/}{ratio}

{{ Settable }}

\phantomsection\label{parameters-x-settable-tooltip}
Settable parameters can be customized for all following uses of the
function with a \texttt{\ set\ } rule.

The horizontal scaling factor.

The body will be mirrored horizontally if the parameter is negative.

Default: \texttt{\ }{\texttt{\ 100\%\ }}\texttt{\ }

\subsubsection{\texorpdfstring{\texttt{\ y\ }}{ y }}\label{parameters-y}

\href{/docs/reference/foundations/auto/}{auto} {or}
\href{/docs/reference/layout/length/}{length} {or}
\href{/docs/reference/layout/ratio/}{ratio}

{{ Settable }}

\phantomsection\label{parameters-y-settable-tooltip}
Settable parameters can be customized for all following uses of the
function with a \texttt{\ set\ } rule.

The vertical scaling factor.

The body will be mirrored vertically if the parameter is negative.

Default: \texttt{\ }{\texttt{\ 100\%\ }}\texttt{\ }

\subsubsection{\texorpdfstring{\texttt{\ origin\ }}{ origin }}\label{parameters-origin}

\href{/docs/reference/layout/alignment/}{alignment}

{{ Settable }}

\phantomsection\label{parameters-origin-settable-tooltip}
Settable parameters can be customized for all following uses of the
function with a \texttt{\ set\ } rule.

The origin of the transformation.

Default: \texttt{\ center\ }{\texttt{\ +\ }}\texttt{\ horizon\ }

\includesvg[width=0.16667in,height=0.16667in]{/assets/icons/16-arrow-right.svg}
View example

\begin{verbatim}
A#box(scale(75%)[A])A \
B#box(scale(75%, origin: bottom + left)[B])B
\end{verbatim}

\includegraphics[width=5in,height=\textheight,keepaspectratio]{/assets/docs/dT49GhvKfj-Kj_N_KdtBqQAAAAAAAAAA.png}

\subsubsection{\texorpdfstring{\texttt{\ reflow\ }}{ reflow }}\label{parameters-reflow}

\href{/docs/reference/foundations/bool/}{bool}

{{ Settable }}

\phantomsection\label{parameters-reflow-settable-tooltip}
Settable parameters can be customized for all following uses of the
function with a \texttt{\ set\ } rule.

Whether the scaling impacts the layout.

If set to \texttt{\ }{\texttt{\ false\ }}\texttt{\ } , the scaled
content will be allowed to overlap other content. If set to
\texttt{\ }{\texttt{\ true\ }}\texttt{\ } , it will compute the new size
of the scaled content and adjust the layout accordingly.

Default: \texttt{\ }{\texttt{\ false\ }}\texttt{\ }

\includesvg[width=0.16667in,height=0.16667in]{/assets/icons/16-arrow-right.svg}
View example

\begin{verbatim}
Hello #scale(x: 20%, y: 40%, reflow: true)[World]!
\end{verbatim}

\includegraphics[width=5in,height=\textheight,keepaspectratio]{/assets/docs/8qEVgn4pU_8oLmlhe4cX2QAAAAAAAAAA.png}

\subsubsection{\texorpdfstring{\texttt{\ body\ }}{ body }}\label{parameters-body}

\href{/docs/reference/foundations/content/}{content}

{Required} {{ Positional }}

\phantomsection\label{parameters-body-positional-tooltip}
Positional parameters are specified in order, without names.

The content to scale.

\href{/docs/reference/layout/rotate/}{\pandocbounded{\includesvg[keepaspectratio]{/assets/icons/16-arrow-right.svg}}}

{ Rotate } { Previous page }

\href{/docs/reference/layout/skew/}{\pandocbounded{\includesvg[keepaspectratio]{/assets/icons/16-arrow-right.svg}}}

{ Skew } { Next page }


\section{Docs LaTeX/typst.app/docs/reference/layout/direction.tex}
\title{typst.app/docs/reference/layout/direction}

\begin{itemize}
\tightlist
\item
  \href{/docs}{\includesvg[width=0.16667in,height=0.16667in]{/assets/icons/16-docs-dark.svg}}
\item
  \includesvg[width=0.16667in,height=0.16667in]{/assets/icons/16-arrow-right.svg}
\item
  \href{/docs/reference/}{Reference}
\item
  \includesvg[width=0.16667in,height=0.16667in]{/assets/icons/16-arrow-right.svg}
\item
  \href{/docs/reference/layout/}{Layout}
\item
  \includesvg[width=0.16667in,height=0.16667in]{/assets/icons/16-arrow-right.svg}
\item
  \href{/docs/reference/layout/direction/}{Direction}
\end{itemize}

\section{\texorpdfstring{{ direction }}{ direction }}\label{summary}

The four directions into which content can be laid out.

Possible values are:

\begin{itemize}
\tightlist
\item
  \texttt{\ ltr\ } : Left to right.
\item
  \texttt{\ rtl\ } : Right to left.
\item
  \texttt{\ ttb\ } : Top to bottom.
\item
  \texttt{\ btt\ } : Bottom to top.
\end{itemize}

These values are available globally and also in the direction
type\textquotesingle s scope, so you can write either of the following
two:

\begin{verbatim}
#stack(dir: rtl)[A][B][C]
#stack(dir: direction.rtl)[A][B][C]
\end{verbatim}

\includegraphics[width=5in,height=\textheight,keepaspectratio]{/assets/docs/43rZPR36KLZcf8RLRLjX0wAAAAAAAAAA.png}

\subsection{\texorpdfstring{{ Definitions
}}{ Definitions }}\label{definitions}

\phantomsection\label{definitions-tooltip}
Functions and types and can have associated definitions. These are
accessed by specifying the function or type, followed by a period, and
then the definition\textquotesingle s name.

\subsubsection{\texorpdfstring{\texttt{\ axis\ }}{ axis }}\label{definitions-axis}

The axis this direction belongs to, either
\texttt{\ }{\texttt{\ "horizontal"\ }}\texttt{\ } or
\texttt{\ }{\texttt{\ "vertical"\ }}\texttt{\ } .

self { . } { axis } (

)

\begin{verbatim}
#ltr.axis() \
#ttb.axis()
\end{verbatim}

\includegraphics[width=5in,height=\textheight,keepaspectratio]{/assets/docs/JrNsSPuIGz5d-HyvpKlmRAAAAAAAAAAA.png}

\subsubsection{\texorpdfstring{\texttt{\ start\ }}{ start }}\label{definitions-start}

The start point of this direction, as an alignment.

self { . } { start } (

) -\textgreater{} \href{/docs/reference/layout/alignment/}{alignment}

\begin{verbatim}
#ltr.start() \
#rtl.start() \
#ttb.start() \
#btt.start()
\end{verbatim}

\includegraphics[width=5in,height=\textheight,keepaspectratio]{/assets/docs/N9RQCkuykNN4FsJgRg06GgAAAAAAAAAA.png}

\subsubsection{\texorpdfstring{\texttt{\ end\ }}{ end }}\label{definitions-end}

The end point of this direction, as an alignment.

self { . } { end } (

) -\textgreater{} \href{/docs/reference/layout/alignment/}{alignment}

\begin{verbatim}
#ltr.end() \
#rtl.end() \
#ttb.end() \
#btt.end()
\end{verbatim}

\includegraphics[width=5in,height=\textheight,keepaspectratio]{/assets/docs/NDjcpeKFmKqoCGermlx1dAAAAAAAAAAA.png}

\subsubsection{\texorpdfstring{\texttt{\ inv\ }}{ inv }}\label{definitions-inv}

The inverse direction.

self { . } { inv } (

) -\textgreater{} \href{/docs/reference/layout/direction/}{direction}

\begin{verbatim}
#ltr.inv() \
#rtl.inv() \
#ttb.inv() \
#btt.inv()
\end{verbatim}

\includegraphics[width=5in,height=\textheight,keepaspectratio]{/assets/docs/kBDvCk2AJ9dPd5ZUJjxcOgAAAAAAAAAA.png}

\href{/docs/reference/layout/columns/}{\pandocbounded{\includesvg[keepaspectratio]{/assets/icons/16-arrow-right.svg}}}

{ Columns } { Previous page }

\href{/docs/reference/layout/fraction/}{\pandocbounded{\includesvg[keepaspectratio]{/assets/icons/16-arrow-right.svg}}}

{ Fraction } { Next page }


\section{Docs LaTeX/typst.app/docs/reference/layout/h.tex}
\title{typst.app/docs/reference/layout/h}

\begin{itemize}
\tightlist
\item
  \href{/docs}{\includesvg[width=0.16667in,height=0.16667in]{/assets/icons/16-docs-dark.svg}}
\item
  \includesvg[width=0.16667in,height=0.16667in]{/assets/icons/16-arrow-right.svg}
\item
  \href{/docs/reference/}{Reference}
\item
  \includesvg[width=0.16667in,height=0.16667in]{/assets/icons/16-arrow-right.svg}
\item
  \href{/docs/reference/layout/}{Layout}
\item
  \includesvg[width=0.16667in,height=0.16667in]{/assets/icons/16-arrow-right.svg}
\item
  \href{/docs/reference/layout/h/}{Spacing (H)}
\end{itemize}

\section{\texorpdfstring{\texttt{\ h\ } {{ Element
}}}{ h   Element }}\label{summary}

\phantomsection\label{element-tooltip}
Element functions can be customized with \texttt{\ set\ } and
\texttt{\ show\ } rules.

Inserts horizontal spacing into a paragraph.

The spacing can be absolute, relative, or fractional. In the last case,
the remaining space on the line is distributed among all fractional
spacings according to their relative fractions.

\subsection{Example}\label{example}

\begin{verbatim}
First #h(1cm) Second \
First #h(30%) Second
\end{verbatim}

\includegraphics[width=5in,height=\textheight,keepaspectratio]{/assets/docs/8wL-xYLR6Y7MLlpoIuX_vAAAAAAAAAAA.png}

\subsection{Fractional spacing}\label{fractional-spacing}

With fractional spacing, you can align things within a line without
forcing a paragraph break (like
\href{/docs/reference/layout/align/}{\texttt{\ align\ }} would). Each
fractionally sized element gets space based on the ratio of its fraction
to the sum of all fractions.

\begin{verbatim}
First #h(1fr) Second \
First #h(1fr) Second #h(1fr) Third \
First #h(2fr) Second #h(1fr) Third
\end{verbatim}

\includegraphics[width=5in,height=\textheight,keepaspectratio]{/assets/docs/pBCqhY9Aheurjnzy2VgPBgAAAAAAAAAA.png}

\subsection{Mathematical Spacing}\label{math-spacing}

In \href{/docs/reference/math/}{mathematical formulas} , you can
additionally use these constants to add spacing between elements:
\texttt{\ thin\ } (1/6Â~em), \texttt{\ med\ } (2/9Â~em),
\texttt{\ thick\ } (5/18Â~em), \texttt{\ quad\ } (1Â~em),
\texttt{\ wide\ } (2Â~em).

\subsection{\texorpdfstring{{ Parameters
}}{ Parameters }}\label{parameters}

\phantomsection\label{parameters-tooltip}
Parameters are the inputs to a function. They are specified in
parentheses after the function name.

{ h } (

{ \href{/docs/reference/layout/relative/}{relative}
\href{/docs/reference/layout/fraction/}{fraction} , } {
\hyperref[parameters-weak]{weak :}
\href{/docs/reference/foundations/bool/}{bool} , }

) -\textgreater{} \href{/docs/reference/foundations/content/}{content}

\subsubsection{\texorpdfstring{\texttt{\ amount\ }}{ amount }}\label{parameters-amount}

\href{/docs/reference/layout/relative/}{relative} {or}
\href{/docs/reference/layout/fraction/}{fraction}

{Required} {{ Positional }}

\phantomsection\label{parameters-amount-positional-tooltip}
Positional parameters are specified in order, without names.

How much spacing to insert.

\subsubsection{\texorpdfstring{\texttt{\ weak\ }}{ weak }}\label{parameters-weak}

\href{/docs/reference/foundations/bool/}{bool}

{{ Settable }}

\phantomsection\label{parameters-weak-settable-tooltip}
Settable parameters can be customized for all following uses of the
function with a \texttt{\ set\ } rule.

If \texttt{\ }{\texttt{\ true\ }}\texttt{\ } , the spacing collapses at
the start or end of a paragraph. Moreover, from multiple adjacent weak
spacings all but the largest one collapse.

Weak spacing in markup also causes all adjacent markup spaces to be
removed, regardless of the amount of spacing inserted. To force a space
next to weak spacing, you can explicitly write
\texttt{\ }{\texttt{\ \#\ }}\texttt{\ }{\texttt{\ "\ "\ }}\texttt{\ }
(for a normal space) or
\texttt{\ }{\texttt{\ \textasciitilde{}\ }}\texttt{\ } (for a
non-breaking space). The latter can be useful to create a construct that
always attaches to the preceding word with one non-breaking space,
independently of whether a markup space existed in front or not.

Default: \texttt{\ }{\texttt{\ false\ }}\texttt{\ }

\includesvg[width=0.16667in,height=0.16667in]{/assets/icons/16-arrow-right.svg}
View example

\begin{verbatim}
#h(1cm, weak: true)
We identified a group of _weak_
specimens that fail to manifest
in most cases. However, when
#h(8pt, weak: true) supported
#h(8pt, weak: true) on both sides,
they do show up.

Further #h(0pt, weak: true) more,
even the smallest of them swallow
adjacent markup spaces.
\end{verbatim}

\includegraphics[width=5in,height=\textheight,keepaspectratio]{/assets/docs/c_7b_9WV6STCF2ERdGhpfQAAAAAAAAAA.png}

\href{/docs/reference/layout/skew/}{\pandocbounded{\includesvg[keepaspectratio]{/assets/icons/16-arrow-right.svg}}}

{ Skew } { Previous page }

\href{/docs/reference/layout/v/}{\pandocbounded{\includesvg[keepaspectratio]{/assets/icons/16-arrow-right.svg}}}

{ Spacing (V) } { Next page }


\section{Docs LaTeX/typst.app/docs/reference/layout/place.tex}
\title{typst.app/docs/reference/layout/place}

\begin{itemize}
\tightlist
\item
  \href{/docs}{\includesvg[width=0.16667in,height=0.16667in]{/assets/icons/16-docs-dark.svg}}
\item
  \includesvg[width=0.16667in,height=0.16667in]{/assets/icons/16-arrow-right.svg}
\item
  \href{/docs/reference/}{Reference}
\item
  \includesvg[width=0.16667in,height=0.16667in]{/assets/icons/16-arrow-right.svg}
\item
  \href{/docs/reference/layout/}{Layout}
\item
  \includesvg[width=0.16667in,height=0.16667in]{/assets/icons/16-arrow-right.svg}
\item
  \href{/docs/reference/layout/place/}{Place}
\end{itemize}

\section{\texorpdfstring{\texttt{\ place\ } {{ Element
}}}{ place   Element }}\label{summary}

\phantomsection\label{element-tooltip}
Element functions can be customized with \texttt{\ set\ } and
\texttt{\ show\ } rules.

Places content relatively to its parent container.

Placed content can be either overlaid (the default) or floating.
Overlaid content is aligned with the parent container according to the
given
\href{/docs/reference/layout/place/\#parameters-alignment}{\texttt{\ alignment\ }}
, and shown over any other content added so far in the container.
Floating content is placed at the top or bottom of the container,
displacing other content down or up respectively. In both cases, the
content position can be adjusted with
\href{/docs/reference/layout/place/\#parameters-dx}{\texttt{\ dx\ }} and
\href{/docs/reference/layout/place/\#parameters-dy}{\texttt{\ dy\ }}
offsets without affecting the layout.

The parent can be any container such as a
\href{/docs/reference/layout/block/}{\texttt{\ block\ }} ,
\href{/docs/reference/layout/box/}{\texttt{\ box\ }} ,
\href{/docs/reference/visualize/rect/}{\texttt{\ rect\ }} , etc. A top
level \texttt{\ place\ } call will place content directly in the text
area of the current page. This can be used for absolute positioning on
the page: with a \texttt{\ top\ +\ left\ }
\href{/docs/reference/layout/place/\#parameters-alignment}{\texttt{\ alignment\ }}
, the offsets \texttt{\ dx\ } and \texttt{\ dy\ } will set the position
of the element\textquotesingle s top left corner relatively to the top
left corner of the text area. For absolute positioning on the full page
including margins, you can use \texttt{\ place\ } in
\href{/docs/reference/layout/page/\#parameters-foreground}{\texttt{\ page.foreground\ }}
or
\href{/docs/reference/layout/page/\#parameters-background}{\texttt{\ page.background\ }}
.

\subsection{Examples}\label{examples}

\begin{verbatim}
#set page(height: 120pt)
Hello, world!

#rect(
  width: 100%,
  height: 2cm,
  place(horizon + right, square()),
)

#place(
  top + left,
  dx: -5pt,
  square(size: 5pt, fill: red),
)
\end{verbatim}

\includegraphics[width=5in,height=\textheight,keepaspectratio]{/assets/docs/b3Ue37sNl2HDpslyo5trfgAAAAAAAAAA.png}

\subsection{Effect on the position of other
elements}\label{effect-on-other-elements}

Overlaid elements don\textquotesingle t take space in the flow of
content, but a \texttt{\ place\ } call inserts an invisible block-level
element in the flow. This can affect the layout by breaking the current
paragraph. To avoid this, you can wrap the \texttt{\ place\ } call in a
\href{/docs/reference/layout/box/}{\texttt{\ box\ }} when the call is
made in the middle of a paragraph. The alignment and offsets will then
be relative to this zero-size box. To make sure it
doesn\textquotesingle t interfere with spacing, the box should be
attached to a word using a word joiner.

For example, the following defines a function for attaching an
annotation to the following word:

\begin{verbatim}
#let annotate(..args) = {
  box(place(..args))
  sym.wj
  h(0pt, weak: true)
}

A placed #annotate(square(), dy: 2pt)
square in my text.
\end{verbatim}

\includegraphics[width=5in,height=\textheight,keepaspectratio]{/assets/docs/QIJqPsAAp5jqe-EB4bZF1gAAAAAAAAAA.png}

The zero-width weak spacing serves to discard spaces between the
function call and the next word.

\subsection{\texorpdfstring{{ Parameters
}}{ Parameters }}\label{parameters}

\phantomsection\label{parameters-tooltip}
Parameters are the inputs to a function. They are specified in
parentheses after the function name.

{ place } (

{ \hyperref[parameters-alignment]{}
\href{/docs/reference/foundations/auto/}{auto}
\href{/docs/reference/layout/alignment/}{alignment} , } {
\hyperref[parameters-scope]{scope :}
\href{/docs/reference/foundations/str/}{str} , } {
\hyperref[parameters-float]{float :}
\href{/docs/reference/foundations/bool/}{bool} , } {
\hyperref[parameters-clearance]{clearance :}
\href{/docs/reference/layout/length/}{length} , } {
\hyperref[parameters-dx]{dx :}
\href{/docs/reference/layout/relative/}{relative} , } {
\hyperref[parameters-dy]{dy :}
\href{/docs/reference/layout/relative/}{relative} , } {
\href{/docs/reference/foundations/content/}{content} , }

) -\textgreater{} \href{/docs/reference/foundations/content/}{content}

\subsubsection{\texorpdfstring{\texttt{\ alignment\ }}{ alignment }}\label{parameters-alignment}

\href{/docs/reference/foundations/auto/}{auto} {or}
\href{/docs/reference/layout/alignment/}{alignment}

{{ Positional }}

\phantomsection\label{parameters-alignment-positional-tooltip}
Positional parameters are specified in order, without names.

{{ Settable }}

\phantomsection\label{parameters-alignment-settable-tooltip}
Settable parameters can be customized for all following uses of the
function with a \texttt{\ set\ } rule.

Relative to which position in the parent container to place the content.

\begin{itemize}
\tightlist
\item
  If \texttt{\ float\ } is \texttt{\ }{\texttt{\ false\ }}\texttt{\ } ,
  then this can be any alignment other than
  \texttt{\ }{\texttt{\ auto\ }}\texttt{\ } .
\item
  If \texttt{\ float\ } is \texttt{\ }{\texttt{\ true\ }}\texttt{\ } ,
  then this must be \texttt{\ }{\texttt{\ auto\ }}\texttt{\ } ,
  \texttt{\ top\ } , or \texttt{\ bottom\ } .
\end{itemize}

When \texttt{\ float\ } is \texttt{\ }{\texttt{\ false\ }}\texttt{\ }
and no vertical alignment is specified, the content is placed at the
current position on the vertical axis.

Default: \texttt{\ start\ }

\subsubsection{\texorpdfstring{\texttt{\ scope\ }}{ scope }}\label{parameters-scope}

\href{/docs/reference/foundations/str/}{str}

{{ Settable }}

\phantomsection\label{parameters-scope-settable-tooltip}
Settable parameters can be customized for all following uses of the
function with a \texttt{\ set\ } rule.

Relative to which containing scope something is placed.

The parent scope is primarily used with figures and, for this reason,
the figure function has a mirrored
\href{/docs/reference/model/figure/\#parameters-scope}{\texttt{\ scope\ }
parameter} . Nonetheless, it can also be more generally useful to break
out of the columns. A typical example would be to
\href{/docs/guides/page-setup-guide/\#columns}{create a single-column
title section} in a two-column document.

Note that parent-scoped placement is currently only supported if
\texttt{\ float\ } is \texttt{\ }{\texttt{\ true\ }}\texttt{\ } . This
may change in the future.

\begin{longtable}[]{@{}ll@{}}
\toprule\noalign{}
Variant & Details \\
\midrule\noalign{}
\endhead
\bottomrule\noalign{}
\endlastfoot
\texttt{\ "\ column\ "\ } & Place into the current column. \\
\texttt{\ "\ parent\ "\ } & Place relative to the parent, letting the
content span over all columns. \\
\end{longtable}

Default: \texttt{\ }{\texttt{\ "column"\ }}\texttt{\ }

\includesvg[width=0.16667in,height=0.16667in]{/assets/icons/16-arrow-right.svg}
View example

\begin{verbatim}
#set page(height: 150pt, columns: 2)
#place(
  top + center,
  scope: "parent",
  float: true,
  rect(width: 80%, fill: aqua),
)

#lorem(25)
\end{verbatim}

\includegraphics[width=5in,height=\textheight,keepaspectratio]{/assets/docs/9xhEXBaN2g3N9Vju7GUzFwAAAAAAAAAA.png}

\subsubsection{\texorpdfstring{\texttt{\ float\ }}{ float }}\label{parameters-float}

\href{/docs/reference/foundations/bool/}{bool}

{{ Settable }}

\phantomsection\label{parameters-float-settable-tooltip}
Settable parameters can be customized for all following uses of the
function with a \texttt{\ set\ } rule.

Whether the placed element has floating layout.

Floating elements are positioned at the top or bottom of the parent
container, displacing in-flow content. They are always placed in the
in-flow order relative to each other, as well as before any content
following a later
\href{/docs/reference/layout/place/\#definitions-flush}{\texttt{\ place.flush\ }}
element.

Default: \texttt{\ }{\texttt{\ false\ }}\texttt{\ }

\includesvg[width=0.16667in,height=0.16667in]{/assets/icons/16-arrow-right.svg}
View example

\begin{verbatim}
#set page(height: 150pt)
#let note(where, body) = place(
  center + where,
  float: true,
  clearance: 6pt,
  rect(body),
)

#lorem(10)
#note(bottom)[Bottom 1]
#note(bottom)[Bottom 2]
#lorem(40)
#note(top)[Top]
#lorem(10)
\end{verbatim}

\includegraphics[width=5in,height=\textheight,keepaspectratio]{/assets/docs/t5SJ49ulSlCH5SgTOH20JAAAAAAAAAAA.png}
\includegraphics[width=5in,height=\textheight,keepaspectratio]{/assets/docs/t5SJ49ulSlCH5SgTOH20JAAAAAAAAAAB.png}

\subsubsection{\texorpdfstring{\texttt{\ clearance\ }}{ clearance }}\label{parameters-clearance}

\href{/docs/reference/layout/length/}{length}

{{ Settable }}

\phantomsection\label{parameters-clearance-settable-tooltip}
Settable parameters can be customized for all following uses of the
function with a \texttt{\ set\ } rule.

The spacing between the placed element and other elements in a floating
layout.

Has no effect if \texttt{\ float\ } is
\texttt{\ }{\texttt{\ false\ }}\texttt{\ } .

Default: \texttt{\ }{\texttt{\ 1.5em\ }}\texttt{\ }

\subsubsection{\texorpdfstring{\texttt{\ dx\ }}{ dx }}\label{parameters-dx}

\href{/docs/reference/layout/relative/}{relative}

{{ Settable }}

\phantomsection\label{parameters-dx-settable-tooltip}
Settable parameters can be customized for all following uses of the
function with a \texttt{\ set\ } rule.

The horizontal displacement of the placed content.

Default:
\texttt{\ }{\texttt{\ 0\%\ }}\texttt{\ }{\texttt{\ +\ }}\texttt{\ }{\texttt{\ 0pt\ }}\texttt{\ }

\includesvg[width=0.16667in,height=0.16667in]{/assets/icons/16-arrow-right.svg}
View example

\begin{verbatim}
#set page(height: 100pt)
#for i in range(16) {
  let amount = i * 4pt
  place(center, dx: amount - 32pt, dy: amount)[A]
}
\end{verbatim}

\includegraphics[width=5in,height=\textheight,keepaspectratio]{/assets/docs/kAqGzNrSyPcytdYDwTZgaQAAAAAAAAAA.png}

This does not affect the layout of in-flow content. In other words, the
placed content is treated as if it were wrapped in a
\href{/docs/reference/layout/move/}{\texttt{\ move\ }} element.

\subsubsection{\texorpdfstring{\texttt{\ dy\ }}{ dy }}\label{parameters-dy}

\href{/docs/reference/layout/relative/}{relative}

{{ Settable }}

\phantomsection\label{parameters-dy-settable-tooltip}
Settable parameters can be customized for all following uses of the
function with a \texttt{\ set\ } rule.

The vertical displacement of the placed content.

This does not affect the layout of in-flow content. In other words, the
placed content is treated as if it were wrapped in a
\href{/docs/reference/layout/move/}{\texttt{\ move\ }} element.

Default:
\texttt{\ }{\texttt{\ 0\%\ }}\texttt{\ }{\texttt{\ +\ }}\texttt{\ }{\texttt{\ 0pt\ }}\texttt{\ }

\subsubsection{\texorpdfstring{\texttt{\ body\ }}{ body }}\label{parameters-body}

\href{/docs/reference/foundations/content/}{content}

{Required} {{ Positional }}

\phantomsection\label{parameters-body-positional-tooltip}
Positional parameters are specified in order, without names.

The content to place.

\subsection{\texorpdfstring{{ Definitions
}}{ Definitions }}\label{definitions}

\phantomsection\label{definitions-tooltip}
Functions and types and can have associated definitions. These are
accessed by specifying the function or type, followed by a period, and
then the definition\textquotesingle s name.

\subsubsection{\texorpdfstring{\texttt{\ flush\ } {{ Element
}}}{ flush   Element }}\label{definitions-flush}

\phantomsection\label{definitions-flush-element-tooltip}
Element functions can be customized with \texttt{\ set\ } and
\texttt{\ show\ } rules.

Asks the layout algorithm to place pending floating elements before
continuing with the content.

This is useful for preventing floating figures from spilling into the
next section.

place { . } { flush } (

) -\textgreater{} \href{/docs/reference/foundations/content/}{content}

\begin{verbatim}
#lorem(15)

#figure(
  rect(width: 100%, height: 50pt),
  placement: auto,
  caption: [A rectangle],
)

#place.flush()

This text appears after the figure.
\end{verbatim}

\includegraphics[width=3.125in,height=\textheight,keepaspectratio]{/assets/docs/8qp5vfUImMtnXndzjQCsNQAAAAAAAAAA.png}
\includegraphics[width=3.125in,height=\textheight,keepaspectratio]{/assets/docs/8qp5vfUImMtnXndzjQCsNQAAAAAAAAAB.png}

\href{/docs/reference/layout/pagebreak/}{\pandocbounded{\includesvg[keepaspectratio]{/assets/icons/16-arrow-right.svg}}}

{ Page Break } { Previous page }

\href{/docs/reference/layout/ratio/}{\pandocbounded{\includesvg[keepaspectratio]{/assets/icons/16-arrow-right.svg}}}

{ Ratio } { Next page }


\section{Docs LaTeX/typst.app/docs/reference/layout/alignment.tex}
\title{typst.app/docs/reference/layout/alignment}

\begin{itemize}
\tightlist
\item
  \href{/docs}{\includesvg[width=0.16667in,height=0.16667in]{/assets/icons/16-docs-dark.svg}}
\item
  \includesvg[width=0.16667in,height=0.16667in]{/assets/icons/16-arrow-right.svg}
\item
  \href{/docs/reference/}{Reference}
\item
  \includesvg[width=0.16667in,height=0.16667in]{/assets/icons/16-arrow-right.svg}
\item
  \href{/docs/reference/layout/}{Layout}
\item
  \includesvg[width=0.16667in,height=0.16667in]{/assets/icons/16-arrow-right.svg}
\item
  \href{/docs/reference/layout/alignment/}{Alignment}
\end{itemize}

\section{\texorpdfstring{{ alignment }}{ alignment }}\label{summary}

Where to \href{/docs/reference/layout/align/}{align} something along an
axis.

Possible values are:

\begin{itemize}
\tightlist
\item
  \texttt{\ start\ } : Aligns at the
  \href{/docs/reference/layout/direction/\#definitions-start}{start} of
  the \href{/docs/reference/text/text/\#parameters-dir}{text direction}
  .
\item
  \texttt{\ end\ } : Aligns at the
  \href{/docs/reference/layout/direction/\#definitions-end}{end} of the
  \href{/docs/reference/text/text/\#parameters-dir}{text direction} .
\item
  \texttt{\ left\ } : Align at the left.
\item
  \texttt{\ center\ } : Aligns in the middle, horizontally.
\item
  \texttt{\ right\ } : Aligns at the right.
\item
  \texttt{\ top\ } : Aligns at the top.
\item
  \texttt{\ horizon\ } : Aligns in the middle, vertically.
\item
  \texttt{\ bottom\ } : Align at the bottom.
\end{itemize}

These values are available globally and also in the alignment
type\textquotesingle s scope, so you can write either of the following
two:

\begin{verbatim}
#align(center)[Hi]
#align(alignment.center)[Hi]
\end{verbatim}

\includegraphics[width=5in,height=\textheight,keepaspectratio]{/assets/docs/ZprGjLBPSUJ5f2a4fil8IAAAAAAAAAAA.png}

\subsection{2D alignments}\label{2d-alignments}

To align along both axes at the same time, add the two alignments using
the \texttt{\ +\ } operator. For example, \texttt{\ top\ +\ right\ }
aligns the content to the top right corner.

\begin{verbatim}
#set page(height: 3cm)
#align(center + bottom)[Hi]
\end{verbatim}

\includegraphics[width=5in,height=\textheight,keepaspectratio]{/assets/docs/X3ZrV0nn1RgePWtIVMB4XgAAAAAAAAAA.png}

\subsection{Fields}\label{fields}

The \texttt{\ x\ } and \texttt{\ y\ } fields hold the
alignment\textquotesingle s horizontal and vertical components,
respectively (as yet another \texttt{\ alignment\ } ). They may be
\texttt{\ }{\texttt{\ none\ }}\texttt{\ } .

\begin{verbatim}
#(top + right).x \
#left.x \
#left.y (none)
\end{verbatim}

\includegraphics[width=5in,height=\textheight,keepaspectratio]{/assets/docs/ecr8JX7jRnRHSrOlOhwwRwAAAAAAAAAA.png}

\subsection{\texorpdfstring{{ Definitions
}}{ Definitions }}\label{definitions}

\phantomsection\label{definitions-tooltip}
Functions and types and can have associated definitions. These are
accessed by specifying the function or type, followed by a period, and
then the definition\textquotesingle s name.

\subsubsection{\texorpdfstring{\texttt{\ axis\ }}{ axis }}\label{definitions-axis}

The axis this alignment belongs to.

\begin{itemize}
\tightlist
\item
  \texttt{\ }{\texttt{\ "horizontal"\ }}\texttt{\ } for
  \texttt{\ start\ } , \texttt{\ left\ } , \texttt{\ center\ } ,
  \texttt{\ right\ } , and \texttt{\ end\ }
\item
  \texttt{\ }{\texttt{\ "vertical"\ }}\texttt{\ } for \texttt{\ top\ } ,
  \texttt{\ horizon\ } , and \texttt{\ bottom\ }
\item
  \texttt{\ }{\texttt{\ none\ }}\texttt{\ } for 2-dimensional alignments
\end{itemize}

self { . } { axis } (

)

\begin{verbatim}
#left.axis() \
#bottom.axis()
\end{verbatim}

\includegraphics[width=5in,height=\textheight,keepaspectratio]{/assets/docs/OHqui2ES_RRnmxyOZdFWIgAAAAAAAAAA.png}

\subsubsection{\texorpdfstring{\texttt{\ inv\ }}{ inv }}\label{definitions-inv}

The inverse alignment.

self { . } { inv } (

) -\textgreater{} \href{/docs/reference/layout/alignment/}{alignment}

\begin{verbatim}
#top.inv() \
#left.inv() \
#center.inv() \
#(left + bottom).inv()
\end{verbatim}

\includegraphics[width=5in,height=\textheight,keepaspectratio]{/assets/docs/tBAxSGcUdyogNGn2l8Pm_QAAAAAAAAAA.png}

\href{/docs/reference/layout/align/}{\pandocbounded{\includesvg[keepaspectratio]{/assets/icons/16-arrow-right.svg}}}

{ Align } { Previous page }

\href{/docs/reference/layout/angle/}{\pandocbounded{\includesvg[keepaspectratio]{/assets/icons/16-arrow-right.svg}}}

{ Angle } { Next page }


\section{Docs LaTeX/typst.app/docs/reference/layout/box.tex}
\title{typst.app/docs/reference/layout/box}

\begin{itemize}
\tightlist
\item
  \href{/docs}{\includesvg[width=0.16667in,height=0.16667in]{/assets/icons/16-docs-dark.svg}}
\item
  \includesvg[width=0.16667in,height=0.16667in]{/assets/icons/16-arrow-right.svg}
\item
  \href{/docs/reference/}{Reference}
\item
  \includesvg[width=0.16667in,height=0.16667in]{/assets/icons/16-arrow-right.svg}
\item
  \href{/docs/reference/layout/}{Layout}
\item
  \includesvg[width=0.16667in,height=0.16667in]{/assets/icons/16-arrow-right.svg}
\item
  \href{/docs/reference/layout/box/}{Box}
\end{itemize}

\section{\texorpdfstring{\texttt{\ box\ } {{ Element
}}}{ box   Element }}\label{summary}

\phantomsection\label{element-tooltip}
Element functions can be customized with \texttt{\ set\ } and
\texttt{\ show\ } rules.

An inline-level container that sizes content.

All elements except inline math, text, and boxes are block-level and
cannot occur inside of a paragraph. The box function can be used to
integrate such elements into a paragraph. Boxes take the size of their
contents by default but can also be sized explicitly.

\subsection{Example}\label{example}

\begin{verbatim}
Refer to the docs
#box(
  height: 9pt,
  image("docs.svg")
)
for more information.
\end{verbatim}

\includegraphics[width=5in,height=\textheight,keepaspectratio]{/assets/docs/eB9NAzu2xk-O1miffozwKQAAAAAAAAAA.png}

\subsection{\texorpdfstring{{ Parameters
}}{ Parameters }}\label{parameters}

\phantomsection\label{parameters-tooltip}
Parameters are the inputs to a function. They are specified in
parentheses after the function name.

{ box } (

{ \hyperref[parameters-width]{width :}
\href{/docs/reference/foundations/auto/}{auto}
\href{/docs/reference/layout/relative/}{relative}
\href{/docs/reference/layout/fraction/}{fraction} , } {
\hyperref[parameters-height]{height :}
\href{/docs/reference/foundations/auto/}{auto}
\href{/docs/reference/layout/relative/}{relative} , } {
\hyperref[parameters-baseline]{baseline :}
\href{/docs/reference/layout/relative/}{relative} , } {
\hyperref[parameters-fill]{fill :}
\href{/docs/reference/foundations/none/}{none}
\href{/docs/reference/visualize/color/}{color}
\href{/docs/reference/visualize/gradient/}{gradient}
\href{/docs/reference/visualize/pattern/}{pattern} , } {
\hyperref[parameters-stroke]{stroke :}
\href{/docs/reference/foundations/none/}{none}
\href{/docs/reference/layout/length/}{length}
\href{/docs/reference/visualize/color/}{color}
\href{/docs/reference/visualize/gradient/}{gradient}
\href{/docs/reference/visualize/stroke/}{stroke}
\href{/docs/reference/visualize/pattern/}{pattern}
\href{/docs/reference/foundations/dictionary/}{dictionary} , } {
\hyperref[parameters-radius]{radius :}
\href{/docs/reference/layout/relative/}{relative}
\href{/docs/reference/foundations/dictionary/}{dictionary} , } {
\hyperref[parameters-inset]{inset :}
\href{/docs/reference/layout/relative/}{relative}
\href{/docs/reference/foundations/dictionary/}{dictionary} , } {
\hyperref[parameters-outset]{outset :}
\href{/docs/reference/layout/relative/}{relative}
\href{/docs/reference/foundations/dictionary/}{dictionary} , } {
\hyperref[parameters-clip]{clip :}
\href{/docs/reference/foundations/bool/}{bool} , } {
\hyperref[parameters-body]{}
\href{/docs/reference/foundations/none/}{none}
\href{/docs/reference/foundations/content/}{content} , }

) -\textgreater{} \href{/docs/reference/foundations/content/}{content}

\subsubsection{\texorpdfstring{\texttt{\ width\ }}{ width }}\label{parameters-width}

\href{/docs/reference/foundations/auto/}{auto} {or}
\href{/docs/reference/layout/relative/}{relative} {or}
\href{/docs/reference/layout/fraction/}{fraction}

{{ Settable }}

\phantomsection\label{parameters-width-settable-tooltip}
Settable parameters can be customized for all following uses of the
function with a \texttt{\ set\ } rule.

The width of the box.

Boxes can have \href{/docs/reference/layout/fraction/}{fractional}
widths, as the example below demonstrates.

\emph{Note:} Currently, only boxes and only their widths might be
fractionally sized within paragraphs. Support for fractionally sized
images, shapes, and more might be added in the future.

Default: \texttt{\ }{\texttt{\ auto\ }}\texttt{\ }

\includesvg[width=0.16667in,height=0.16667in]{/assets/icons/16-arrow-right.svg}
View example

\begin{verbatim}
Line in #box(width: 1fr, line(length: 100%)) between.
\end{verbatim}

\includegraphics[width=5in,height=\textheight,keepaspectratio]{/assets/docs/dzJroqkPcQ8j1yD6nZSE0AAAAAAAAAAA.png}

\subsubsection{\texorpdfstring{\texttt{\ height\ }}{ height }}\label{parameters-height}

\href{/docs/reference/foundations/auto/}{auto} {or}
\href{/docs/reference/layout/relative/}{relative}

{{ Settable }}

\phantomsection\label{parameters-height-settable-tooltip}
Settable parameters can be customized for all following uses of the
function with a \texttt{\ set\ } rule.

The height of the box.

Default: \texttt{\ }{\texttt{\ auto\ }}\texttt{\ }

\subsubsection{\texorpdfstring{\texttt{\ baseline\ }}{ baseline }}\label{parameters-baseline}

\href{/docs/reference/layout/relative/}{relative}

{{ Settable }}

\phantomsection\label{parameters-baseline-settable-tooltip}
Settable parameters can be customized for all following uses of the
function with a \texttt{\ set\ } rule.

An amount to shift the box\textquotesingle s baseline by.

Default:
\texttt{\ }{\texttt{\ 0\%\ }}\texttt{\ }{\texttt{\ +\ }}\texttt{\ }{\texttt{\ 0pt\ }}\texttt{\ }

\includesvg[width=0.16667in,height=0.16667in]{/assets/icons/16-arrow-right.svg}
View example

\begin{verbatim}
Image: #box(baseline: 40%, image("tiger.jpg", width: 2cm)).
\end{verbatim}

\includegraphics[width=5in,height=\textheight,keepaspectratio]{/assets/docs/jNZmXcLZQWKojb5Yhz3uEQAAAAAAAAAA.png}

\subsubsection{\texorpdfstring{\texttt{\ fill\ }}{ fill }}\label{parameters-fill}

\href{/docs/reference/foundations/none/}{none} {or}
\href{/docs/reference/visualize/color/}{color} {or}
\href{/docs/reference/visualize/gradient/}{gradient} {or}
\href{/docs/reference/visualize/pattern/}{pattern}

{{ Settable }}

\phantomsection\label{parameters-fill-settable-tooltip}
Settable parameters can be customized for all following uses of the
function with a \texttt{\ set\ } rule.

The box\textquotesingle s background color. See the
\href{/docs/reference/visualize/rect/\#parameters-fill}{rectangle\textquotesingle s
documentation} for more details.

Default: \texttt{\ }{\texttt{\ none\ }}\texttt{\ }

\subsubsection{\texorpdfstring{\texttt{\ stroke\ }}{ stroke }}\label{parameters-stroke}

\href{/docs/reference/foundations/none/}{none} {or}
\href{/docs/reference/layout/length/}{length} {or}
\href{/docs/reference/visualize/color/}{color} {or}
\href{/docs/reference/visualize/gradient/}{gradient} {or}
\href{/docs/reference/visualize/stroke/}{stroke} {or}
\href{/docs/reference/visualize/pattern/}{pattern} {or}
\href{/docs/reference/foundations/dictionary/}{dictionary}

{{ Settable }}

\phantomsection\label{parameters-stroke-settable-tooltip}
Settable parameters can be customized for all following uses of the
function with a \texttt{\ set\ } rule.

The box\textquotesingle s border color. See the
\href{/docs/reference/visualize/rect/\#parameters-stroke}{rectangle\textquotesingle s
documentation} for more details.

Default:
\texttt{\ }{\texttt{\ (\ }}\texttt{\ }{\texttt{\ :\ }}\texttt{\ }{\texttt{\ )\ }}\texttt{\ }

\subsubsection{\texorpdfstring{\texttt{\ radius\ }}{ radius }}\label{parameters-radius}

\href{/docs/reference/layout/relative/}{relative} {or}
\href{/docs/reference/foundations/dictionary/}{dictionary}

{{ Settable }}

\phantomsection\label{parameters-radius-settable-tooltip}
Settable parameters can be customized for all following uses of the
function with a \texttt{\ set\ } rule.

How much to round the box\textquotesingle s corners. See the
\href{/docs/reference/visualize/rect/\#parameters-radius}{rectangle\textquotesingle s
documentation} for more details.

Default:
\texttt{\ }{\texttt{\ (\ }}\texttt{\ }{\texttt{\ :\ }}\texttt{\ }{\texttt{\ )\ }}\texttt{\ }

\subsubsection{\texorpdfstring{\texttt{\ inset\ }}{ inset }}\label{parameters-inset}

\href{/docs/reference/layout/relative/}{relative} {or}
\href{/docs/reference/foundations/dictionary/}{dictionary}

{{ Settable }}

\phantomsection\label{parameters-inset-settable-tooltip}
Settable parameters can be customized for all following uses of the
function with a \texttt{\ set\ } rule.

How much to pad the box\textquotesingle s content.

\emph{Note:} When the box contains text, its exact size depends on the
current \href{/docs/reference/text/text/\#parameters-top-edge}{text
edges} .

Default:
\texttt{\ }{\texttt{\ (\ }}\texttt{\ }{\texttt{\ :\ }}\texttt{\ }{\texttt{\ )\ }}\texttt{\ }

\includesvg[width=0.16667in,height=0.16667in]{/assets/icons/16-arrow-right.svg}
View example

\begin{verbatim}
#rect(inset: 0pt)[Tight]
\end{verbatim}

\includegraphics[width=5in,height=\textheight,keepaspectratio]{/assets/docs/GVDpvIL_te6KlSASD3i2EQAAAAAAAAAA.png}

\subsubsection{\texorpdfstring{\texttt{\ outset\ }}{ outset }}\label{parameters-outset}

\href{/docs/reference/layout/relative/}{relative} {or}
\href{/docs/reference/foundations/dictionary/}{dictionary}

{{ Settable }}

\phantomsection\label{parameters-outset-settable-tooltip}
Settable parameters can be customized for all following uses of the
function with a \texttt{\ set\ } rule.

How much to expand the box\textquotesingle s size without affecting the
layout.

This is useful to prevent padding from affecting line layout. For a
generalized version of the example below, see the documentation for the
\href{/docs/reference/text/raw/\#parameters-block}{raw
text\textquotesingle s block parameter} .

Default:
\texttt{\ }{\texttt{\ (\ }}\texttt{\ }{\texttt{\ :\ }}\texttt{\ }{\texttt{\ )\ }}\texttt{\ }

\includesvg[width=0.16667in,height=0.16667in]{/assets/icons/16-arrow-right.svg}
View example

\begin{verbatim}
An inline
#box(
  fill: luma(235),
  inset: (x: 3pt, y: 0pt),
  outset: (y: 3pt),
  radius: 2pt,
)[rectangle].
\end{verbatim}

\includegraphics[width=5in,height=\textheight,keepaspectratio]{/assets/docs/68KQkm_HskMy1aDAbQWYdwAAAAAAAAAA.png}

\subsubsection{\texorpdfstring{\texttt{\ clip\ }}{ clip }}\label{parameters-clip}

\href{/docs/reference/foundations/bool/}{bool}

{{ Settable }}

\phantomsection\label{parameters-clip-settable-tooltip}
Settable parameters can be customized for all following uses of the
function with a \texttt{\ set\ } rule.

Whether to clip the content inside the box.

Clipping is useful when the box\textquotesingle s content is larger than
the box itself, as any content that exceeds the box\textquotesingle s
bounds will be hidden.

Default: \texttt{\ }{\texttt{\ false\ }}\texttt{\ }

\includesvg[width=0.16667in,height=0.16667in]{/assets/icons/16-arrow-right.svg}
View example

\begin{verbatim}
#box(
  width: 50pt,
  height: 50pt,
  clip: true,
  image("tiger.jpg", width: 100pt, height: 100pt)
)
\end{verbatim}

\includegraphics[width=5in,height=\textheight,keepaspectratio]{/assets/docs/RAY1IirASCSdH0pM4209bwAAAAAAAAAA.png}

\subsubsection{\texorpdfstring{\texttt{\ body\ }}{ body }}\label{parameters-body}

\href{/docs/reference/foundations/none/}{none} {or}
\href{/docs/reference/foundations/content/}{content}

{{ Positional }}

\phantomsection\label{parameters-body-positional-tooltip}
Positional parameters are specified in order, without names.

{{ Settable }}

\phantomsection\label{parameters-body-settable-tooltip}
Settable parameters can be customized for all following uses of the
function with a \texttt{\ set\ } rule.

The contents of the box.

Default: \texttt{\ }{\texttt{\ none\ }}\texttt{\ }

\href{/docs/reference/layout/block/}{\pandocbounded{\includesvg[keepaspectratio]{/assets/icons/16-arrow-right.svg}}}

{ Block } { Previous page }

\href{/docs/reference/layout/colbreak/}{\pandocbounded{\includesvg[keepaspectratio]{/assets/icons/16-arrow-right.svg}}}

{ Column Break } { Next page }


\section{Docs LaTeX/typst.app/docs/reference/layout/fraction.tex}
\title{typst.app/docs/reference/layout/fraction}

\begin{itemize}
\tightlist
\item
  \href{/docs}{\includesvg[width=0.16667in,height=0.16667in]{/assets/icons/16-docs-dark.svg}}
\item
  \includesvg[width=0.16667in,height=0.16667in]{/assets/icons/16-arrow-right.svg}
\item
  \href{/docs/reference/}{Reference}
\item
  \includesvg[width=0.16667in,height=0.16667in]{/assets/icons/16-arrow-right.svg}
\item
  \href{/docs/reference/layout/}{Layout}
\item
  \includesvg[width=0.16667in,height=0.16667in]{/assets/icons/16-arrow-right.svg}
\item
  \href{/docs/reference/layout/fraction/}{Fraction}
\end{itemize}

\section{\texorpdfstring{{ fraction }}{ fraction }}\label{summary}

Defines how the remaining space in a layout is distributed.

Each fractionally sized element gets space based on the ratio of its
fraction to the sum of all fractions.

For more details, also see the \href{/docs/reference/layout/h/}{h} and
\href{/docs/reference/layout/v/}{v} functions and the
\href{/docs/reference/layout/grid/}{grid function} .

\subsection{Example}\label{example}

\begin{verbatim}
Left #h(1fr) Left-ish #h(2fr) Right
\end{verbatim}

\includegraphics[width=5in,height=\textheight,keepaspectratio]{/assets/docs/Mh5sjFkAJFlbM1vm_65COgAAAAAAAAAA.png}

\href{/docs/reference/layout/direction/}{\pandocbounded{\includesvg[keepaspectratio]{/assets/icons/16-arrow-right.svg}}}

{ Direction } { Previous page }

\href{/docs/reference/layout/grid/}{\pandocbounded{\includesvg[keepaspectratio]{/assets/icons/16-arrow-right.svg}}}

{ Grid } { Next page }


\section{Docs LaTeX/typst.app/docs/reference/layout/hide.tex}
\title{typst.app/docs/reference/layout/hide}

\begin{itemize}
\tightlist
\item
  \href{/docs}{\includesvg[width=0.16667in,height=0.16667in]{/assets/icons/16-docs-dark.svg}}
\item
  \includesvg[width=0.16667in,height=0.16667in]{/assets/icons/16-arrow-right.svg}
\item
  \href{/docs/reference/}{Reference}
\item
  \includesvg[width=0.16667in,height=0.16667in]{/assets/icons/16-arrow-right.svg}
\item
  \href{/docs/reference/layout/}{Layout}
\item
  \includesvg[width=0.16667in,height=0.16667in]{/assets/icons/16-arrow-right.svg}
\item
  \href{/docs/reference/layout/hide/}{Hide}
\end{itemize}

\section{\texorpdfstring{\texttt{\ hide\ } {{ Element
}}}{ hide   Element }}\label{summary}

\phantomsection\label{element-tooltip}
Element functions can be customized with \texttt{\ set\ } and
\texttt{\ show\ } rules.

Hides content without affecting layout.

The \texttt{\ hide\ } function allows you to hide content while the
layout still \textquotesingle sees\textquotesingle{} it. This is useful
to create whitespace that is exactly as large as some content. It may
also be useful to redact content because its arguments are not included
in the output.

\subsection{Example}\label{example}

\begin{verbatim}
Hello Jane \
#hide[Hello] Joe
\end{verbatim}

\includegraphics[width=5in,height=\textheight,keepaspectratio]{/assets/docs/w0ioP6Ne87hOMXgpgPJirgAAAAAAAAAA.png}

\subsection{\texorpdfstring{{ Parameters
}}{ Parameters }}\label{parameters}

\phantomsection\label{parameters-tooltip}
Parameters are the inputs to a function. They are specified in
parentheses after the function name.

{ hide } (

{ \href{/docs/reference/foundations/content/}{content} }

) -\textgreater{} \href{/docs/reference/foundations/content/}{content}

\subsubsection{\texorpdfstring{\texttt{\ body\ }}{ body }}\label{parameters-body}

\href{/docs/reference/foundations/content/}{content}

{Required} {{ Positional }}

\phantomsection\label{parameters-body-positional-tooltip}
Positional parameters are specified in order, without names.

The content to hide.

\href{/docs/reference/layout/grid/}{\pandocbounded{\includesvg[keepaspectratio]{/assets/icons/16-arrow-right.svg}}}

{ Grid } { Previous page }

\href{/docs/reference/layout/layout/}{\pandocbounded{\includesvg[keepaspectratio]{/assets/icons/16-arrow-right.svg}}}

{ Layout } { Next page }


\section{Docs LaTeX/typst.app/docs/reference/layout/relative.tex}
\title{typst.app/docs/reference/layout/relative}

\begin{itemize}
\tightlist
\item
  \href{/docs}{\includesvg[width=0.16667in,height=0.16667in]{/assets/icons/16-docs-dark.svg}}
\item
  \includesvg[width=0.16667in,height=0.16667in]{/assets/icons/16-arrow-right.svg}
\item
  \href{/docs/reference/}{Reference}
\item
  \includesvg[width=0.16667in,height=0.16667in]{/assets/icons/16-arrow-right.svg}
\item
  \href{/docs/reference/layout/}{Layout}
\item
  \includesvg[width=0.16667in,height=0.16667in]{/assets/icons/16-arrow-right.svg}
\item
  \href{/docs/reference/layout/relative/}{Relative Length}
\end{itemize}

\section{\texorpdfstring{{ relative }}{ relative }}\label{summary}

A length in relation to some known length.

This type is a combination of a
\href{/docs/reference/layout/length/}{length} with a
\href{/docs/reference/layout/ratio/}{ratio} . It results from addition
and subtraction of a length and a ratio. Wherever a relative length is
expected, you can also use a bare length or ratio.

\subsection{Example}\label{example}

\begin{verbatim}
#rect(width: 100% - 50pt)

#(100% - 50pt).length \
#(100% - 50pt).ratio
\end{verbatim}

\includegraphics[width=5in,height=\textheight,keepaspectratio]{/assets/docs/eMTS_wIJ-8rLzP6A-A6wPAAAAAAAAAAA.png}

A relative length has the following fields:

\begin{itemize}
\tightlist
\item
  \texttt{\ length\ } : Its length component.
\item
  \texttt{\ ratio\ } : Its ratio component.
\end{itemize}

\href{/docs/reference/layout/ratio/}{\pandocbounded{\includesvg[keepaspectratio]{/assets/icons/16-arrow-right.svg}}}

{ Ratio } { Previous page }

\href{/docs/reference/layout/repeat/}{\pandocbounded{\includesvg[keepaspectratio]{/assets/icons/16-arrow-right.svg}}}

{ Repeat } { Next page }


\section{Docs LaTeX/typst.app/docs/reference/layout/measure.tex}
\title{typst.app/docs/reference/layout/measure}

\begin{itemize}
\tightlist
\item
  \href{/docs}{\includesvg[width=0.16667in,height=0.16667in]{/assets/icons/16-docs-dark.svg}}
\item
  \includesvg[width=0.16667in,height=0.16667in]{/assets/icons/16-arrow-right.svg}
\item
  \href{/docs/reference/}{Reference}
\item
  \includesvg[width=0.16667in,height=0.16667in]{/assets/icons/16-arrow-right.svg}
\item
  \href{/docs/reference/layout/}{Layout}
\item
  \includesvg[width=0.16667in,height=0.16667in]{/assets/icons/16-arrow-right.svg}
\item
  \href{/docs/reference/layout/measure/}{Measure}
\end{itemize}

\section{\texorpdfstring{\texttt{\ measure\ } {{ Contextual
}}}{ measure   Contextual }}\label{summary}

\phantomsection\label{contextual-tooltip}
Contextual functions can only be used when the context is known

Measures the layouted size of content.

The \texttt{\ measure\ } function lets you determine the layouted size
of content. By default an infinite space is assumed, so the measured
dimensions may not necessarily match the final dimensions of the
content. If you want to measure in the current layout dimensions, you
can combine \texttt{\ measure\ } and
\href{/docs/reference/layout/layout/}{\texttt{\ layout\ }} .

\subsection{Example}\label{example}

The same content can have a different size depending on the
\href{/docs/reference/context/}{context} that it is placed into. In the
example below, the
\texttt{\ }{\texttt{\ \#\ }}\texttt{\ }{\texttt{\ content\ }}\texttt{\ }
is of course bigger when we increase the font size.

\begin{verbatim}
#let content = [Hello!]
#content
#set text(14pt)
#content
\end{verbatim}

\includegraphics[width=5in,height=\textheight,keepaspectratio]{/assets/docs/AhP31noWwrcSQXbwnmO-hwAAAAAAAAAA.png}

For this reason, you can only measure when context is available.

\begin{verbatim}
#let thing(body) = context {
  let size = measure(body)
  [Width of "#body" is #size.width]
}

#thing[Hey] \
#thing[Welcome]
\end{verbatim}

\includegraphics[width=5in,height=\textheight,keepaspectratio]{/assets/docs/-y6AuN3J3rl7Gz1x_VRjjwAAAAAAAAAA.png}

The measure function returns a dictionary with the entries
\texttt{\ width\ } and \texttt{\ height\ } , both of type
\href{/docs/reference/layout/length/}{\texttt{\ length\ }} .

\subsection{\texorpdfstring{{ Parameters
}}{ Parameters }}\label{parameters}

\phantomsection\label{parameters-tooltip}
Parameters are the inputs to a function. They are specified in
parentheses after the function name.

{ measure } (

{ \hyperref[parameters-width]{width :}
\href{/docs/reference/foundations/auto/}{auto}
\href{/docs/reference/layout/length/}{length} , } {
\hyperref[parameters-height]{height :}
\href{/docs/reference/foundations/auto/}{auto}
\href{/docs/reference/layout/length/}{length} , } {
\href{/docs/reference/foundations/content/}{content} , } {
\href{/docs/reference/foundations/none/}{none} { styles } , }

) -\textgreater{}
\href{/docs/reference/foundations/dictionary/}{dictionary}

\subsubsection{\texorpdfstring{\texttt{\ width\ }}{ width }}\label{parameters-width}

\href{/docs/reference/foundations/auto/}{auto} {or}
\href{/docs/reference/layout/length/}{length}

The width available to layout the content.

Setting this to \texttt{\ }{\texttt{\ auto\ }}\texttt{\ } indicates
infinite available width.

Note that using the \texttt{\ width\ } and \texttt{\ height\ }
parameters of this function is different from measuring a sized
\href{/docs/reference/layout/block/}{\texttt{\ block\ }} containing the
content. In the following example, the former will get the dimensions of
the inner content instead of the dimensions of the block.

Default: \texttt{\ }{\texttt{\ auto\ }}\texttt{\ }

\includesvg[width=0.16667in,height=0.16667in]{/assets/icons/16-arrow-right.svg}
View example

\begin{verbatim}
#context measure(lorem(100), width: 400pt)

#context measure(block(lorem(100), width: 400pt))
\end{verbatim}

\includegraphics[width=5in,height=\textheight,keepaspectratio]{/assets/docs/kGPOcZfxzWEfqWzKQCJaFgAAAAAAAAAA.png}

\subsubsection{\texorpdfstring{\texttt{\ height\ }}{ height }}\label{parameters-height}

\href{/docs/reference/foundations/auto/}{auto} {or}
\href{/docs/reference/layout/length/}{length}

The height available to layout the content.

Setting this to \texttt{\ }{\texttt{\ auto\ }}\texttt{\ } indicates
infinite available height.

Default: \texttt{\ }{\texttt{\ auto\ }}\texttt{\ }

\subsubsection{\texorpdfstring{\texttt{\ content\ }}{ content }}\label{parameters-content}

\href{/docs/reference/foundations/content/}{content}

{Required} {{ Positional }}

\phantomsection\label{parameters-content-positional-tooltip}
Positional parameters are specified in order, without names.

The content whose size to measure.

\subsubsection{\texorpdfstring{\texttt{\ styles\ }}{ styles }}\label{parameters-styles}

\href{/docs/reference/foundations/none/}{none} {or} { styles }

{{ Positional }}

\phantomsection\label{parameters-styles-positional-tooltip}
Positional parameters are specified in order, without names.

\emph{Compatibility:} This argument is deprecated. It only exists for
compatibility with Typst 0.10 and lower and shouldn\textquotesingle t be
used anymore.

Default: \texttt{\ }{\texttt{\ none\ }}\texttt{\ }

\href{/docs/reference/layout/length/}{\pandocbounded{\includesvg[keepaspectratio]{/assets/icons/16-arrow-right.svg}}}

{ Length } { Previous page }

\href{/docs/reference/layout/move/}{\pandocbounded{\includesvg[keepaspectratio]{/assets/icons/16-arrow-right.svg}}}

{ Move } { Next page }


\section{Docs LaTeX/typst.app/docs/reference/layout/page.tex}
\title{typst.app/docs/reference/layout/page}

\begin{itemize}
\tightlist
\item
  \href{/docs}{\includesvg[width=0.16667in,height=0.16667in]{/assets/icons/16-docs-dark.svg}}
\item
  \includesvg[width=0.16667in,height=0.16667in]{/assets/icons/16-arrow-right.svg}
\item
  \href{/docs/reference/}{Reference}
\item
  \includesvg[width=0.16667in,height=0.16667in]{/assets/icons/16-arrow-right.svg}
\item
  \href{/docs/reference/layout/}{Layout}
\item
  \includesvg[width=0.16667in,height=0.16667in]{/assets/icons/16-arrow-right.svg}
\item
  \href{/docs/reference/layout/page/}{Page}
\end{itemize}

\section{\texorpdfstring{\texttt{\ page\ } {{ Element
}}}{ page   Element }}\label{summary}

\phantomsection\label{element-tooltip}
Element functions can be customized with \texttt{\ set\ } and
\texttt{\ show\ } rules.

Layouts its child onto one or multiple pages.

Although this function is primarily used in set rules to affect page
properties, it can also be used to explicitly render its argument onto a
set of pages of its own.

Pages can be set to use \texttt{\ }{\texttt{\ auto\ }}\texttt{\ } as
their width or height. In this case, the pages will grow to fit their
content on the respective axis.

The \href{/docs/guides/page-setup-guide/}{Guide for Page Setup} explains
how to use this and related functions to set up a document with many
examples.

\subsection{Example}\label{example}

\begin{verbatim}
#set page("us-letter")

There you go, US friends!
\end{verbatim}

\includegraphics[width=12.75in,height=\textheight,keepaspectratio]{/assets/docs/Gsn3vxGfYJJE0DFa5w6toQAAAAAAAAAA.png}

\subsection{\texorpdfstring{{ Parameters
}}{ Parameters }}\label{parameters}

\phantomsection\label{parameters-tooltip}
Parameters are the inputs to a function. They are specified in
parentheses after the function name.

{ page } (

{ \hyperref[parameters-paper]{paper :}
\href{/docs/reference/foundations/str/}{str} , } {
\hyperref[parameters-width]{width :}
\href{/docs/reference/foundations/auto/}{auto}
\href{/docs/reference/layout/length/}{length} , } {
\hyperref[parameters-height]{height :}
\href{/docs/reference/foundations/auto/}{auto}
\href{/docs/reference/layout/length/}{length} , } {
\hyperref[parameters-flipped]{flipped :}
\href{/docs/reference/foundations/bool/}{bool} , } {
\hyperref[parameters-margin]{margin :}
\href{/docs/reference/foundations/auto/}{auto}
\href{/docs/reference/layout/relative/}{relative}
\href{/docs/reference/foundations/dictionary/}{dictionary} , } {
\hyperref[parameters-binding]{binding :}
\href{/docs/reference/foundations/auto/}{auto}
\href{/docs/reference/layout/alignment/}{alignment} , } {
\hyperref[parameters-columns]{columns :}
\href{/docs/reference/foundations/int/}{int} , } {
\hyperref[parameters-fill]{fill :}
\href{/docs/reference/foundations/none/}{none}
\href{/docs/reference/foundations/auto/}{auto}
\href{/docs/reference/visualize/color/}{color}
\href{/docs/reference/visualize/gradient/}{gradient}
\href{/docs/reference/visualize/pattern/}{pattern} , } {
\hyperref[parameters-numbering]{numbering :}
\href{/docs/reference/foundations/none/}{none}
\href{/docs/reference/foundations/str/}{str}
\href{/docs/reference/foundations/function/}{function} , } {
\hyperref[parameters-number-align]{number-align :}
\href{/docs/reference/layout/alignment/}{alignment} , } {
\hyperref[parameters-header]{header :}
\href{/docs/reference/foundations/none/}{none}
\href{/docs/reference/foundations/auto/}{auto}
\href{/docs/reference/foundations/content/}{content} , } {
\hyperref[parameters-header-ascent]{header-ascent :}
\href{/docs/reference/layout/relative/}{relative} , } {
\hyperref[parameters-footer]{footer :}
\href{/docs/reference/foundations/none/}{none}
\href{/docs/reference/foundations/auto/}{auto}
\href{/docs/reference/foundations/content/}{content} , } {
\hyperref[parameters-footer-descent]{footer-descent :}
\href{/docs/reference/layout/relative/}{relative} , } {
\hyperref[parameters-background]{background :}
\href{/docs/reference/foundations/none/}{none}
\href{/docs/reference/foundations/content/}{content} , } {
\hyperref[parameters-foreground]{foreground :}
\href{/docs/reference/foundations/none/}{none}
\href{/docs/reference/foundations/content/}{content} , } {
\href{/docs/reference/foundations/content/}{content} , }

) -\textgreater{} \href{/docs/reference/foundations/content/}{content}

\subsubsection{\texorpdfstring{\texttt{\ paper\ }}{ paper }}\label{parameters-paper}

\href{/docs/reference/foundations/str/}{str}

{{ Settable }}

\phantomsection\label{parameters-paper-settable-tooltip}
Settable parameters can be customized for all following uses of the
function with a \texttt{\ set\ } rule.

A standard paper size to set width and height.

This is just a shorthand for setting \texttt{\ width\ } and
\texttt{\ height\ } and, as such, cannot be retrieved in a context
expression.

\includesvg[width=0.16667in,height=0.16667in]{/assets/icons/16-arrow-right.svg}
View options

Default: \texttt{\ }{\texttt{\ "a4"\ }}\texttt{\ }

\includesvg[width=0.16667in,height=0.16667in]{/assets/icons/16-arrow-right.svg}
View paper sizes

\texttt{\ "\ a0\ "\ } , \texttt{\ "\ a1\ "\ } , \texttt{\ "\ a2\ "\ } ,
\texttt{\ "\ a3\ "\ } , \texttt{\ "\ a4\ "\ } , \texttt{\ "\ a5\ "\ } ,
\texttt{\ "\ a6\ "\ } , \texttt{\ "\ a7\ "\ } , \texttt{\ "\ a8\ "\ } ,
\texttt{\ "\ a9\ "\ } , \texttt{\ "\ a10\ "\ } , \texttt{\ "\ a11\ "\ }
, \texttt{\ "\ iso-b1\ "\ } , \texttt{\ "\ iso-b2\ "\ } ,
\texttt{\ "\ iso-b3\ "\ } , \texttt{\ "\ iso-b4\ "\ } ,
\texttt{\ "\ iso-b5\ "\ } , \texttt{\ "\ iso-b6\ "\ } ,
\texttt{\ "\ iso-b7\ "\ } , \texttt{\ "\ iso-b8\ "\ } ,
\texttt{\ "\ iso-c3\ "\ } , \texttt{\ "\ iso-c4\ "\ } ,
\texttt{\ "\ iso-c5\ "\ } , \texttt{\ "\ iso-c6\ "\ } ,
\texttt{\ "\ iso-c7\ "\ } , \texttt{\ "\ iso-c8\ "\ } ,
\texttt{\ "\ din-d3\ "\ } , \texttt{\ "\ din-d4\ "\ } ,
\texttt{\ "\ din-d5\ "\ } , \texttt{\ "\ din-d6\ "\ } ,
\texttt{\ "\ din-d7\ "\ } , \texttt{\ "\ din-d8\ "\ } ,
\texttt{\ "\ sis-g5\ "\ } , \texttt{\ "\ sis-e5\ "\ } ,
\texttt{\ "\ ansi-a\ "\ } , \texttt{\ "\ ansi-b\ "\ } ,
\texttt{\ "\ ansi-c\ "\ } , \texttt{\ "\ ansi-d\ "\ } ,
\texttt{\ "\ ansi-e\ "\ } , \texttt{\ "\ arch-a\ "\ } ,
\texttt{\ "\ arch-b\ "\ } , \texttt{\ "\ arch-c\ "\ } ,
\texttt{\ "\ arch-d\ "\ } , \texttt{\ "\ arch-e1\ "\ } ,
\texttt{\ "\ arch-e\ "\ } , \texttt{\ "\ jis-b0\ "\ } ,
\texttt{\ "\ jis-b1\ "\ } , \texttt{\ "\ jis-b2\ "\ } ,
\texttt{\ "\ jis-b3\ "\ } , \texttt{\ "\ jis-b4\ "\ } ,
\texttt{\ "\ jis-b5\ "\ } , \texttt{\ "\ jis-b6\ "\ } ,
\texttt{\ "\ jis-b7\ "\ } , \texttt{\ "\ jis-b8\ "\ } ,
\texttt{\ "\ jis-b9\ "\ } , \texttt{\ "\ jis-b10\ "\ } ,
\texttt{\ "\ jis-b11\ "\ } , \texttt{\ "\ sac-d0\ "\ } ,
\texttt{\ "\ sac-d1\ "\ } , \texttt{\ "\ sac-d2\ "\ } ,
\texttt{\ "\ sac-d3\ "\ } , \texttt{\ "\ sac-d4\ "\ } ,
\texttt{\ "\ sac-d5\ "\ } , \texttt{\ "\ sac-d6\ "\ } ,
\texttt{\ "\ iso-id-1\ "\ } , \texttt{\ "\ iso-id-2\ "\ } ,
\texttt{\ "\ iso-id-3\ "\ } , \texttt{\ "\ asia-f4\ "\ } ,
\texttt{\ "\ jp-shiroku-ban-4\ "\ } ,
\texttt{\ "\ jp-shiroku-ban-5\ "\ } ,
\texttt{\ "\ jp-shiroku-ban-6\ "\ } , \texttt{\ "\ jp-kiku-4\ "\ } ,
\texttt{\ "\ jp-kiku-5\ "\ } , \texttt{\ "\ jp-business-card\ "\ } ,
\texttt{\ "\ cn-business-card\ "\ } ,
\texttt{\ "\ eu-business-card\ "\ } , \texttt{\ "\ fr-tellière\ "\ } ,
\texttt{\ "\ fr-couronne-écriture\ "\ } ,
\texttt{\ "\ fr-couronne-édition\ "\ } , \texttt{\ "\ fr-raisin\ "\ } ,
\texttt{\ "\ fr-carré\ "\ } , \texttt{\ "\ fr-jésus\ "\ } ,
\texttt{\ "\ uk-brief\ "\ } , \texttt{\ "\ uk-draft\ "\ } ,
\texttt{\ "\ uk-foolscap\ "\ } , \texttt{\ "\ uk-quarto\ "\ } ,
\texttt{\ "\ uk-crown\ "\ } , \texttt{\ "\ uk-book-a\ "\ } ,
\texttt{\ "\ uk-book-b\ "\ } , \texttt{\ "\ us-letter\ "\ } ,
\texttt{\ "\ us-legal\ "\ } , \texttt{\ "\ us-tabloid\ "\ } ,
\texttt{\ "\ us-executive\ "\ } , \texttt{\ "\ us-foolscap-folio\ "\ } ,
\texttt{\ "\ us-statement\ "\ } , \texttt{\ "\ us-ledger\ "\ } ,
\texttt{\ "\ us-oficio\ "\ } , \texttt{\ "\ us-gov-letter\ "\ } ,
\texttt{\ "\ us-gov-legal\ "\ } , \texttt{\ "\ us-business-card\ "\ } ,
\texttt{\ "\ us-digest\ "\ } , \texttt{\ "\ us-trade\ "\ } ,
\texttt{\ "\ newspaper-compact\ "\ } ,
\texttt{\ "\ newspaper-berliner\ "\ } ,
\texttt{\ "\ newspaper-broadsheet\ "\ } ,
\texttt{\ "\ presentation-16-9\ "\ } ,
\texttt{\ "\ presentation-4-3\ "\ }

\subsubsection{\texorpdfstring{\texttt{\ width\ }}{ width }}\label{parameters-width}

\href{/docs/reference/foundations/auto/}{auto} {or}
\href{/docs/reference/layout/length/}{length}

{{ Settable }}

\phantomsection\label{parameters-width-settable-tooltip}
Settable parameters can be customized for all following uses of the
function with a \texttt{\ set\ } rule.

The width of the page.

Default: \texttt{\ }{\texttt{\ 595.28pt\ }}\texttt{\ }

\includesvg[width=0.16667in,height=0.16667in]{/assets/icons/16-arrow-right.svg}
View example

\begin{verbatim}
#set page(
  width: 3cm,
  margin: (x: 0cm),
)

#for i in range(3) {
  box(square(width: 1cm))
}
\end{verbatim}

\includegraphics[width=1.77083in,height=\textheight,keepaspectratio]{/assets/docs/xcDLR5uuky5aEnJroP3JfQAAAAAAAAAA.png}

\subsubsection{\texorpdfstring{\texttt{\ height\ }}{ height }}\label{parameters-height}

\href{/docs/reference/foundations/auto/}{auto} {or}
\href{/docs/reference/layout/length/}{length}

{{ Settable }}

\phantomsection\label{parameters-height-settable-tooltip}
Settable parameters can be customized for all following uses of the
function with a \texttt{\ set\ } rule.

The height of the page.

If this is set to \texttt{\ }{\texttt{\ auto\ }}\texttt{\ } , page
breaks can only be triggered manually by inserting a
\href{/docs/reference/layout/pagebreak/}{page break} . Most examples
throughout this documentation use
\texttt{\ }{\texttt{\ auto\ }}\texttt{\ } for the height of the page to
dynamically grow and shrink to fit their content.

Default: \texttt{\ }{\texttt{\ 841.89pt\ }}\texttt{\ }

\subsubsection{\texorpdfstring{\texttt{\ flipped\ }}{ flipped }}\label{parameters-flipped}

\href{/docs/reference/foundations/bool/}{bool}

{{ Settable }}

\phantomsection\label{parameters-flipped-settable-tooltip}
Settable parameters can be customized for all following uses of the
function with a \texttt{\ set\ } rule.

Whether the page is flipped into landscape orientation.

Default: \texttt{\ }{\texttt{\ false\ }}\texttt{\ }

\includesvg[width=0.16667in,height=0.16667in]{/assets/icons/16-arrow-right.svg}
View example

\begin{verbatim}
#set page(
  "us-business-card",
  flipped: true,
  fill: rgb("f2e5dd"),
)

#set align(bottom + end)
#text(14pt)[*Sam H. Richards*] \
_Procurement Manager_

#set text(10pt)
17 Main Street \
New York, NY 10001 \
+1 555 555 5555
\end{verbatim}

\includegraphics[width=3in,height=\textheight,keepaspectratio]{/assets/docs/NEPMLGLUMuCwXHZB6iu3CAAAAAAAAAAA.png}

\subsubsection{\texorpdfstring{\texttt{\ margin\ }}{ margin }}\label{parameters-margin}

\href{/docs/reference/foundations/auto/}{auto} {or}
\href{/docs/reference/layout/relative/}{relative} {or}
\href{/docs/reference/foundations/dictionary/}{dictionary}

{{ Settable }}

\phantomsection\label{parameters-margin-settable-tooltip}
Settable parameters can be customized for all following uses of the
function with a \texttt{\ set\ } rule.

The page\textquotesingle s margins.

\begin{itemize}
\tightlist
\item
  \texttt{\ }{\texttt{\ auto\ }}\texttt{\ } : The margins are set
  automatically to 2.5/21 times the smaller dimension of the page. This
  results in 2.5cm margins for an A4 page.
\item
  A single length: The same margin on all sides.
\item
  A dictionary: With a dictionary, the margins can be set individually.
  The dictionary can contain the following keys in order of precedence:

  \begin{itemize}
  \tightlist
  \item
    \texttt{\ top\ } : The top margin.
  \item
    \texttt{\ right\ } : The right margin.
  \item
    \texttt{\ bottom\ } : The bottom margin.
  \item
    \texttt{\ left\ } : The left margin.
  \item
    \texttt{\ inside\ } : The margin at the inner side of the page
    (where the
    \href{/docs/reference/layout/page/\#parameters-binding}{binding}
    is).
  \item
    \texttt{\ outside\ } : The margin at the outer side of the page
    (opposite to the
    \href{/docs/reference/layout/page/\#parameters-binding}{binding} ).
  \item
    \texttt{\ x\ } : The horizontal margins.
  \item
    \texttt{\ y\ } : The vertical margins.
  \item
    \texttt{\ rest\ } : The margins on all sides except those for which
    the dictionary explicitly sets a size.
  \end{itemize}
\end{itemize}

The values for \texttt{\ left\ } and \texttt{\ right\ } are mutually
exclusive with the values for \texttt{\ inside\ } and
\texttt{\ outside\ } .

Default: \texttt{\ }{\texttt{\ auto\ }}\texttt{\ }

\includesvg[width=0.16667in,height=0.16667in]{/assets/icons/16-arrow-right.svg}
View example

\begin{verbatim}
#set page(
 width: 3cm,
 height: 4cm,
 margin: (x: 8pt, y: 4pt),
)

#rect(
  width: 100%,
  height: 100%,
  fill: aqua,
)
\end{verbatim}

\includegraphics[width=1.77083in,height=\textheight,keepaspectratio]{/assets/docs/OMqyTIx7yDwyNT0DUFniFAAAAAAAAAAA.png}

\subsubsection{\texorpdfstring{\texttt{\ binding\ }}{ binding }}\label{parameters-binding}

\href{/docs/reference/foundations/auto/}{auto} {or}
\href{/docs/reference/layout/alignment/}{alignment}

{{ Settable }}

\phantomsection\label{parameters-binding-settable-tooltip}
Settable parameters can be customized for all following uses of the
function with a \texttt{\ set\ } rule.

On which side the pages will be bound.

\begin{itemize}
\tightlist
\item
  \texttt{\ }{\texttt{\ auto\ }}\texttt{\ } : Equivalent to
  \texttt{\ left\ } if the
  \href{/docs/reference/text/text/\#parameters-dir}{text direction} is
  left-to-right and \texttt{\ right\ } if it is right-to-left.
\item
  \texttt{\ left\ } : Bound on the left side.
\item
  \texttt{\ right\ } : Bound on the right side.
\end{itemize}

This affects the meaning of the \texttt{\ inside\ } and
\texttt{\ outside\ } options for margins.

Default: \texttt{\ }{\texttt{\ auto\ }}\texttt{\ }

\subsubsection{\texorpdfstring{\texttt{\ columns\ }}{ columns }}\label{parameters-columns}

\href{/docs/reference/foundations/int/}{int}

{{ Settable }}

\phantomsection\label{parameters-columns-settable-tooltip}
Settable parameters can be customized for all following uses of the
function with a \texttt{\ set\ } rule.

How many columns the page has.

If you need to insert columns into a page or other container, you can
also use the \href{/docs/reference/layout/columns/}{\texttt{\ columns\ }
function} .

Default: \texttt{\ }{\texttt{\ 1\ }}\texttt{\ }

\includesvg[width=0.16667in,height=0.16667in]{/assets/icons/16-arrow-right.svg}
View example

\begin{verbatim}
#set page(columns: 2, height: 4.8cm)
Climate change is one of the most
pressing issues of our time, with
the potential to devastate
communities, ecosystems, and
economies around the world. It's
clear that we need to take urgent
action to reduce our carbon
emissions and mitigate the impacts
of a rapidly changing climate.
\end{verbatim}

\includegraphics[width=5in,height=\textheight,keepaspectratio]{/assets/docs/Qem_NgF0Oyp_LY8JRVFBWQAAAAAAAAAA.png}

\subsubsection{\texorpdfstring{\texttt{\ fill\ }}{ fill }}\label{parameters-fill}

\href{/docs/reference/foundations/none/}{none} {or}
\href{/docs/reference/foundations/auto/}{auto} {or}
\href{/docs/reference/visualize/color/}{color} {or}
\href{/docs/reference/visualize/gradient/}{gradient} {or}
\href{/docs/reference/visualize/pattern/}{pattern}

{{ Settable }}

\phantomsection\label{parameters-fill-settable-tooltip}
Settable parameters can be customized for all following uses of the
function with a \texttt{\ set\ } rule.

The page\textquotesingle s background fill.

Setting this to something non-transparent instructs the printer to color
the complete page. If you are considering larger production runs, it may
be more environmentally friendly and cost-effective to source pre-dyed
pages and not set this property.

When set to \texttt{\ }{\texttt{\ none\ }}\texttt{\ } , the background
becomes transparent. Note that PDF pages will still appear with a
(usually white) background in viewers, but they are actually
transparent. (If you print them, no color is used for the background.)

The default of \texttt{\ }{\texttt{\ auto\ }}\texttt{\ } results in
\texttt{\ }{\texttt{\ none\ }}\texttt{\ } for PDF output, and
\texttt{\ white\ } for PNG and SVG.

Default: \texttt{\ }{\texttt{\ auto\ }}\texttt{\ }

\includesvg[width=0.16667in,height=0.16667in]{/assets/icons/16-arrow-right.svg}
View example

\begin{verbatim}
#set page(fill: rgb("444352"))
#set text(fill: rgb("fdfdfd"))
*Dark mode enabled.*
\end{verbatim}

\includegraphics[width=5in,height=\textheight,keepaspectratio]{/assets/docs/PLEs9jVtSM3FxsoATa6SAAAAAAAAAAAA.png}

\subsubsection{\texorpdfstring{\texttt{\ numbering\ }}{ numbering }}\label{parameters-numbering}

\href{/docs/reference/foundations/none/}{none} {or}
\href{/docs/reference/foundations/str/}{str} {or}
\href{/docs/reference/foundations/function/}{function}

{{ Settable }}

\phantomsection\label{parameters-numbering-settable-tooltip}
Settable parameters can be customized for all following uses of the
function with a \texttt{\ set\ } rule.

How to \href{/docs/reference/model/numbering/}{number} the pages.

If an explicit \texttt{\ footer\ } (or \texttt{\ header\ } for
top-aligned numbering) is given, the numbering is ignored.

Default: \texttt{\ }{\texttt{\ none\ }}\texttt{\ }

\includesvg[width=0.16667in,height=0.16667in]{/assets/icons/16-arrow-right.svg}
View example

\begin{verbatim}
#set page(
  height: 100pt,
  margin: (top: 16pt, bottom: 24pt),
  numbering: "1 / 1",
)

#lorem(48)
\end{verbatim}

\includegraphics[width=5in,height=\textheight,keepaspectratio]{/assets/docs/RY8f9OM2hb3s_Q3tS3fVgwAAAAAAAAAA.png}
\includegraphics[width=5in,height=\textheight,keepaspectratio]{/assets/docs/RY8f9OM2hb3s_Q3tS3fVgwAAAAAAAAAB.png}

\subsubsection{\texorpdfstring{\texttt{\ number-align\ }}{ number-align }}\label{parameters-number-align}

\href{/docs/reference/layout/alignment/}{alignment}

{{ Settable }}

\phantomsection\label{parameters-number-align-settable-tooltip}
Settable parameters can be customized for all following uses of the
function with a \texttt{\ set\ } rule.

The alignment of the page numbering.

If the vertical component is \texttt{\ top\ } , the numbering is placed
into the header and if it is \texttt{\ bottom\ } , it is placed in the
footer. Horizon alignment is forbidden. If an explicit matching
\texttt{\ header\ } or \texttt{\ footer\ } is given, the numbering is
ignored.

Default: \texttt{\ center\ }{\texttt{\ +\ }}\texttt{\ bottom\ }

\includesvg[width=0.16667in,height=0.16667in]{/assets/icons/16-arrow-right.svg}
View example

\begin{verbatim}
#set page(
  margin: (top: 16pt, bottom: 24pt),
  numbering: "1",
  number-align: right,
)

#lorem(30)
\end{verbatim}

\includegraphics[width=5in,height=\textheight,keepaspectratio]{/assets/docs/ErvjjUjlAuxqdtzputqWQAAAAAAAAAAA.png}

\subsubsection{\texorpdfstring{\texttt{\ header\ }}{ header }}\label{parameters-header}

\href{/docs/reference/foundations/none/}{none} {or}
\href{/docs/reference/foundations/auto/}{auto} {or}
\href{/docs/reference/foundations/content/}{content}

{{ Settable }}

\phantomsection\label{parameters-header-settable-tooltip}
Settable parameters can be customized for all following uses of the
function with a \texttt{\ set\ } rule.

The page\textquotesingle s header. Fills the top margin of each page.

\begin{itemize}
\tightlist
\item
  Content: Shows the content as the header.
\item
  \texttt{\ }{\texttt{\ auto\ }}\texttt{\ } : Shows the page number if a
  \texttt{\ numbering\ } is set and \texttt{\ number-align\ } is
  \texttt{\ top\ } .
\item
  \texttt{\ }{\texttt{\ none\ }}\texttt{\ } : Suppresses the header.
\end{itemize}

Default: \texttt{\ }{\texttt{\ auto\ }}\texttt{\ }

\includesvg[width=0.16667in,height=0.16667in]{/assets/icons/16-arrow-right.svg}
View example

\begin{verbatim}
#set par(justify: true)
#set page(
  margin: (top: 32pt, bottom: 20pt),
  header: [
    #set text(8pt)
    #smallcaps[Typst Academcy]
    #h(1fr) _Exercise Sheet 3_
  ],
)

#lorem(19)
\end{verbatim}

\includegraphics[width=5in,height=\textheight,keepaspectratio]{/assets/docs/nNqGFtf4s-uyEOhXjup1zgAAAAAAAAAA.png}

\subsubsection{\texorpdfstring{\texttt{\ header-ascent\ }}{ header-ascent }}\label{parameters-header-ascent}

\href{/docs/reference/layout/relative/}{relative}

{{ Settable }}

\phantomsection\label{parameters-header-ascent-settable-tooltip}
Settable parameters can be customized for all following uses of the
function with a \texttt{\ set\ } rule.

The amount the header is raised into the top margin.

Default:
\texttt{\ }{\texttt{\ 30\%\ }}\texttt{\ }{\texttt{\ +\ }}\texttt{\ }{\texttt{\ 0pt\ }}\texttt{\ }

\subsubsection{\texorpdfstring{\texttt{\ footer\ }}{ footer }}\label{parameters-footer}

\href{/docs/reference/foundations/none/}{none} {or}
\href{/docs/reference/foundations/auto/}{auto} {or}
\href{/docs/reference/foundations/content/}{content}

{{ Settable }}

\phantomsection\label{parameters-footer-settable-tooltip}
Settable parameters can be customized for all following uses of the
function with a \texttt{\ set\ } rule.

The page\textquotesingle s footer. Fills the bottom margin of each page.

\begin{itemize}
\tightlist
\item
  Content: Shows the content as the footer.
\item
  \texttt{\ }{\texttt{\ auto\ }}\texttt{\ } : Shows the page number if a
  \texttt{\ numbering\ } is set and \texttt{\ number-align\ } is
  \texttt{\ bottom\ } .
\item
  \texttt{\ }{\texttt{\ none\ }}\texttt{\ } : Suppresses the footer.
\end{itemize}

For just a page number, the \texttt{\ numbering\ } property typically
suffices. If you want to create a custom footer but still display the
page number, you can directly access the
\href{/docs/reference/introspection/counter/}{page counter} .

Default: \texttt{\ }{\texttt{\ auto\ }}\texttt{\ }

\includesvg[width=0.16667in,height=0.16667in]{/assets/icons/16-arrow-right.svg}
View example

\begin{verbatim}
#set par(justify: true)
#set page(
  height: 100pt,
  margin: 20pt,
  footer: context [
    #set align(right)
    #set text(8pt)
    #counter(page).display(
      "1 of I",
      both: true,
    )
  ]
)

#lorem(48)
\end{verbatim}

\includegraphics[width=5in,height=\textheight,keepaspectratio]{/assets/docs/GK4h2efX4DepjGGiUjzejQAAAAAAAAAA.png}
\includegraphics[width=5in,height=\textheight,keepaspectratio]{/assets/docs/GK4h2efX4DepjGGiUjzejQAAAAAAAAAB.png}

\subsubsection{\texorpdfstring{\texttt{\ footer-descent\ }}{ footer-descent }}\label{parameters-footer-descent}

\href{/docs/reference/layout/relative/}{relative}

{{ Settable }}

\phantomsection\label{parameters-footer-descent-settable-tooltip}
Settable parameters can be customized for all following uses of the
function with a \texttt{\ set\ } rule.

The amount the footer is lowered into the bottom margin.

Default:
\texttt{\ }{\texttt{\ 30\%\ }}\texttt{\ }{\texttt{\ +\ }}\texttt{\ }{\texttt{\ 0pt\ }}\texttt{\ }

\subsubsection{\texorpdfstring{\texttt{\ background\ }}{ background }}\label{parameters-background}

\href{/docs/reference/foundations/none/}{none} {or}
\href{/docs/reference/foundations/content/}{content}

{{ Settable }}

\phantomsection\label{parameters-background-settable-tooltip}
Settable parameters can be customized for all following uses of the
function with a \texttt{\ set\ } rule.

Content in the page\textquotesingle s background.

This content will be placed behind the page\textquotesingle s body. It
can be used to place a background image or a watermark.

Default: \texttt{\ }{\texttt{\ none\ }}\texttt{\ }

\includesvg[width=0.16667in,height=0.16667in]{/assets/icons/16-arrow-right.svg}
View example

\begin{verbatim}
#set page(background: rotate(24deg,
  text(18pt, fill: rgb("FFCBC4"))[
    *CONFIDENTIAL*
  ]
))

= Typst's secret plans
In the year 2023, we plan to take
over the world (of typesetting).
\end{verbatim}

\includegraphics[width=5in,height=\textheight,keepaspectratio]{/assets/docs/edMhg75ws-GIgq5IJNJbrQAAAAAAAAAA.png}

\subsubsection{\texorpdfstring{\texttt{\ foreground\ }}{ foreground }}\label{parameters-foreground}

\href{/docs/reference/foundations/none/}{none} {or}
\href{/docs/reference/foundations/content/}{content}

{{ Settable }}

\phantomsection\label{parameters-foreground-settable-tooltip}
Settable parameters can be customized for all following uses of the
function with a \texttt{\ set\ } rule.

Content in the page\textquotesingle s foreground.

This content will overlay the page\textquotesingle s body.

Default: \texttt{\ }{\texttt{\ none\ }}\texttt{\ }

\includesvg[width=0.16667in,height=0.16667in]{/assets/icons/16-arrow-right.svg}
View example

\begin{verbatim}
#set page(foreground: text(24pt)[🥸])

Reviewer 2 has marked our paper
"Weak Reject" because they did
not understand our approach...
\end{verbatim}

\includegraphics[width=5in,height=\textheight,keepaspectratio]{/assets/docs/UxB2Tju0zg4nh85hMFZNOwAAAAAAAAAA.png}

\subsubsection{\texorpdfstring{\texttt{\ body\ }}{ body }}\label{parameters-body}

\href{/docs/reference/foundations/content/}{content}

{Required} {{ Positional }}

\phantomsection\label{parameters-body-positional-tooltip}
Positional parameters are specified in order, without names.

The contents of the page(s).

Multiple pages will be created if the content does not fit on a single
page. A new page with the page properties prior to the function
invocation will be created after the body has been typeset.

\href{/docs/reference/layout/pad/}{\pandocbounded{\includesvg[keepaspectratio]{/assets/icons/16-arrow-right.svg}}}

{ Padding } { Previous page }

\href{/docs/reference/layout/pagebreak/}{\pandocbounded{\includesvg[keepaspectratio]{/assets/icons/16-arrow-right.svg}}}

{ Page Break } { Next page }


\section{Docs LaTeX/typst.app/docs/reference/layout/length.tex}
\title{typst.app/docs/reference/layout/length}

\begin{itemize}
\tightlist
\item
  \href{/docs}{\includesvg[width=0.16667in,height=0.16667in]{/assets/icons/16-docs-dark.svg}}
\item
  \includesvg[width=0.16667in,height=0.16667in]{/assets/icons/16-arrow-right.svg}
\item
  \href{/docs/reference/}{Reference}
\item
  \includesvg[width=0.16667in,height=0.16667in]{/assets/icons/16-arrow-right.svg}
\item
  \href{/docs/reference/layout/}{Layout}
\item
  \includesvg[width=0.16667in,height=0.16667in]{/assets/icons/16-arrow-right.svg}
\item
  \href{/docs/reference/layout/length/}{Length}
\end{itemize}

\section{\texorpdfstring{{ length }}{ length }}\label{summary}

A size or distance, possibly expressed with contextual units.

Typst supports the following length units:

\begin{itemize}
\tightlist
\item
  Points: \texttt{\ }{\texttt{\ 72pt\ }}\texttt{\ }
\item
  Millimeters: \texttt{\ }{\texttt{\ 254mm\ }}\texttt{\ }
\item
  Centimeters: \texttt{\ }{\texttt{\ 2.54cm\ }}\texttt{\ }
\item
  Inches: \texttt{\ }{\texttt{\ 1in\ }}\texttt{\ }
\item
  Relative to font size: \texttt{\ }{\texttt{\ 2.5em\ }}\texttt{\ }
\end{itemize}

You can multiply lengths with and divide them by integers and floats.

\subsection{Example}\label{example}

\begin{verbatim}
#rect(width: 20pt)
#rect(width: 2em)
#rect(width: 1in)

#(3em + 5pt).em \
#(20pt).em \
#(40em + 2pt).abs \
#(5em).abs
\end{verbatim}

\includegraphics[width=5in,height=\textheight,keepaspectratio]{/assets/docs/gpwKHS7y2wIB7BIxGEXoMwAAAAAAAAAA.png}

\subsection{Fields}\label{fields}

\begin{itemize}
\tightlist
\item
  \texttt{\ abs\ } : A length with just the absolute component of the
  current length (that is, excluding the \texttt{\ em\ } component).
\item
  \texttt{\ em\ } : The amount of \texttt{\ em\ } units in this length,
  as a \href{/docs/reference/foundations/float/}{float} .
\end{itemize}

\subsection{\texorpdfstring{{ Definitions
}}{ Definitions }}\label{definitions}

\phantomsection\label{definitions-tooltip}
Functions and types and can have associated definitions. These are
accessed by specifying the function or type, followed by a period, and
then the definition\textquotesingle s name.

\subsubsection{\texorpdfstring{\texttt{\ pt\ }}{ pt }}\label{definitions-pt}

Converts this length to points.

Fails with an error if this length has non-zero \texttt{\ em\ } units
(such as \texttt{\ 5em\ +\ 2pt\ } instead of just \texttt{\ 2pt\ } ).
Use the \texttt{\ abs\ } field (such as in
\texttt{\ (5em\ +\ 2pt).abs.pt()\ } ) to ignore the \texttt{\ em\ }
component of the length (thus converting only its absolute component).

self { . } { pt } (

) -\textgreater{} \href{/docs/reference/foundations/float/}{float}

\subsubsection{\texorpdfstring{\texttt{\ mm\ }}{ mm }}\label{definitions-mm}

Converts this length to millimeters.

Fails with an error if this length has non-zero \texttt{\ em\ } units.
See the
\href{/docs/reference/layout/length/\#definitions-pt}{\texttt{\ pt\ }}
method for more details.

self { . } { mm } (

) -\textgreater{} \href{/docs/reference/foundations/float/}{float}

\subsubsection{\texorpdfstring{\texttt{\ cm\ }}{ cm }}\label{definitions-cm}

Converts this length to centimeters.

Fails with an error if this length has non-zero \texttt{\ em\ } units.
See the
\href{/docs/reference/layout/length/\#definitions-pt}{\texttt{\ pt\ }}
method for more details.

self { . } { cm } (

) -\textgreater{} \href{/docs/reference/foundations/float/}{float}

\subsubsection{\texorpdfstring{\texttt{\ inches\ }}{ inches }}\label{definitions-inches}

Converts this length to inches.

Fails with an error if this length has non-zero \texttt{\ em\ } units.
See the
\href{/docs/reference/layout/length/\#definitions-pt}{\texttt{\ pt\ }}
method for more details.

self { . } { inches } (

) -\textgreater{} \href{/docs/reference/foundations/float/}{float}

\subsubsection{\texorpdfstring{\texttt{\ to-absolute\ }}{ to-absolute }}\label{definitions-to-absolute}

Resolve this length to an absolute length.

self { . } { to-absolute } (

) -\textgreater{} \href{/docs/reference/layout/length/}{length}

\begin{verbatim}
#set text(size: 12pt)
#context [
  #(6pt).to-absolute() \
  #(6pt + 10em).to-absolute() \
  #(10em).to-absolute()
]

#set text(size: 6pt)
#context [
  #(6pt).to-absolute() \
  #(6pt + 10em).to-absolute() \
  #(10em).to-absolute()
]
\end{verbatim}

\includegraphics[width=5in,height=\textheight,keepaspectratio]{/assets/docs/O8f4mxTZz-ziS7eclGAyvgAAAAAAAAAA.png}

\href{/docs/reference/layout/layout/}{\pandocbounded{\includesvg[keepaspectratio]{/assets/icons/16-arrow-right.svg}}}

{ Layout } { Previous page }

\href{/docs/reference/layout/measure/}{\pandocbounded{\includesvg[keepaspectratio]{/assets/icons/16-arrow-right.svg}}}

{ Measure } { Next page }


\section{Docs LaTeX/typst.app/docs/reference/layout/v.tex}
\title{typst.app/docs/reference/layout/v}

\begin{itemize}
\tightlist
\item
  \href{/docs}{\includesvg[width=0.16667in,height=0.16667in]{/assets/icons/16-docs-dark.svg}}
\item
  \includesvg[width=0.16667in,height=0.16667in]{/assets/icons/16-arrow-right.svg}
\item
  \href{/docs/reference/}{Reference}
\item
  \includesvg[width=0.16667in,height=0.16667in]{/assets/icons/16-arrow-right.svg}
\item
  \href{/docs/reference/layout/}{Layout}
\item
  \includesvg[width=0.16667in,height=0.16667in]{/assets/icons/16-arrow-right.svg}
\item
  \href{/docs/reference/layout/v/}{Spacing (V)}
\end{itemize}

\section{\texorpdfstring{\texttt{\ v\ } {{ Element
}}}{ v   Element }}\label{summary}

\phantomsection\label{element-tooltip}
Element functions can be customized with \texttt{\ set\ } and
\texttt{\ show\ } rules.

Inserts vertical spacing into a flow of blocks.

The spacing can be absolute, relative, or fractional. In the last case,
the remaining space on the page is distributed among all fractional
spacings according to their relative fractions.

\subsection{Example}\label{example}

\begin{verbatim}
#grid(
  rows: 3cm,
  columns: 6,
  gutter: 1fr,
  [A #parbreak() B],
  [A #v(0pt) B],
  [A #v(10pt) B],
  [A #v(0pt, weak: true) B],
  [A #v(40%, weak: true) B],
  [A #v(1fr) B],
)
\end{verbatim}

\includegraphics[width=5in,height=\textheight,keepaspectratio]{/assets/docs/DNC2m_0X9s5xLmHMABxCvgAAAAAAAAAA.png}

\subsection{\texorpdfstring{{ Parameters
}}{ Parameters }}\label{parameters}

\phantomsection\label{parameters-tooltip}
Parameters are the inputs to a function. They are specified in
parentheses after the function name.

{ v } (

{ \href{/docs/reference/layout/relative/}{relative}
\href{/docs/reference/layout/fraction/}{fraction} , } {
\hyperref[parameters-weak]{weak :}
\href{/docs/reference/foundations/bool/}{bool} , }

) -\textgreater{} \href{/docs/reference/foundations/content/}{content}

\subsubsection{\texorpdfstring{\texttt{\ amount\ }}{ amount }}\label{parameters-amount}

\href{/docs/reference/layout/relative/}{relative} {or}
\href{/docs/reference/layout/fraction/}{fraction}

{Required} {{ Positional }}

\phantomsection\label{parameters-amount-positional-tooltip}
Positional parameters are specified in order, without names.

How much spacing to insert.

\subsubsection{\texorpdfstring{\texttt{\ weak\ }}{ weak }}\label{parameters-weak}

\href{/docs/reference/foundations/bool/}{bool}

{{ Settable }}

\phantomsection\label{parameters-weak-settable-tooltip}
Settable parameters can be customized for all following uses of the
function with a \texttt{\ set\ } rule.

If \texttt{\ }{\texttt{\ true\ }}\texttt{\ } , the spacing collapses at
the start or end of a flow. Moreover, from multiple adjacent weak
spacings all but the largest one collapse. Weak spacings will always
collapse adjacent paragraph spacing, even if the paragraph spacing is
larger.

Default: \texttt{\ }{\texttt{\ false\ }}\texttt{\ }

\includesvg[width=0.16667in,height=0.16667in]{/assets/icons/16-arrow-right.svg}
View example

\begin{verbatim}
The following theorem is
foundational to the field:
#v(4pt, weak: true)
$ x^2 + y^2 = r^2 $
#v(4pt, weak: true)
The proof is simple:
\end{verbatim}

\includegraphics[width=5in,height=\textheight,keepaspectratio]{/assets/docs/7Xa6Zl_-zWfWaA6gosM_0QAAAAAAAAAA.png}

\href{/docs/reference/layout/h/}{\pandocbounded{\includesvg[keepaspectratio]{/assets/icons/16-arrow-right.svg}}}

{ Spacing (H) } { Previous page }

\href{/docs/reference/layout/stack/}{\pandocbounded{\includesvg[keepaspectratio]{/assets/icons/16-arrow-right.svg}}}

{ Stack } { Next page }


\section{Docs LaTeX/typst.app/docs/reference/layout/stack.tex}
\title{typst.app/docs/reference/layout/stack}

\begin{itemize}
\tightlist
\item
  \href{/docs}{\includesvg[width=0.16667in,height=0.16667in]{/assets/icons/16-docs-dark.svg}}
\item
  \includesvg[width=0.16667in,height=0.16667in]{/assets/icons/16-arrow-right.svg}
\item
  \href{/docs/reference/}{Reference}
\item
  \includesvg[width=0.16667in,height=0.16667in]{/assets/icons/16-arrow-right.svg}
\item
  \href{/docs/reference/layout/}{Layout}
\item
  \includesvg[width=0.16667in,height=0.16667in]{/assets/icons/16-arrow-right.svg}
\item
  \href{/docs/reference/layout/stack/}{Stack}
\end{itemize}

\section{\texorpdfstring{\texttt{\ stack\ } {{ Element
}}}{ stack   Element }}\label{summary}

\phantomsection\label{element-tooltip}
Element functions can be customized with \texttt{\ set\ } and
\texttt{\ show\ } rules.

Arranges content and spacing horizontally or vertically.

The stack places a list of items along an axis, with optional spacing
between each item.

\subsection{Example}\label{example}

\begin{verbatim}
#stack(
  dir: ttb,
  rect(width: 40pt),
  rect(width: 120pt),
  rect(width: 90pt),
)
\end{verbatim}

\includegraphics[width=5in,height=\textheight,keepaspectratio]{/assets/docs/rblc_gO4o5qSEPJtXD1qPgAAAAAAAAAA.png}

\subsection{\texorpdfstring{{ Parameters
}}{ Parameters }}\label{parameters}

\phantomsection\label{parameters-tooltip}
Parameters are the inputs to a function. They are specified in
parentheses after the function name.

{ stack } (

{ \hyperref[parameters-dir]{dir :}
\href{/docs/reference/layout/direction/}{direction} , } {
\hyperref[parameters-spacing]{spacing :}
\href{/docs/reference/foundations/none/}{none}
\href{/docs/reference/layout/relative/}{relative}
\href{/docs/reference/layout/fraction/}{fraction} , } {
\hyperref[parameters-children]{..}
\href{/docs/reference/layout/relative/}{relative}
\href{/docs/reference/layout/fraction/}{fraction}
\href{/docs/reference/foundations/content/}{content} , }

) -\textgreater{} \href{/docs/reference/foundations/content/}{content}

\subsubsection{\texorpdfstring{\texttt{\ dir\ }}{ dir }}\label{parameters-dir}

\href{/docs/reference/layout/direction/}{direction}

{{ Settable }}

\phantomsection\label{parameters-dir-settable-tooltip}
Settable parameters can be customized for all following uses of the
function with a \texttt{\ set\ } rule.

The direction along which the items are stacked. Possible values are:

\begin{itemize}
\tightlist
\item
  \texttt{\ ltr\ } : Left to right.
\item
  \texttt{\ rtl\ } : Right to left.
\item
  \texttt{\ ttb\ } : Top to bottom.
\item
  \texttt{\ btt\ } : Bottom to top.
\end{itemize}

You can use the \texttt{\ start\ } and \texttt{\ end\ } methods to
obtain the initial and final points (respectively) of a direction, as
\texttt{\ alignment\ } . You can also use the \texttt{\ axis\ } method
to determine whether a direction is
\texttt{\ }{\texttt{\ "horizontal"\ }}\texttt{\ } or
\texttt{\ }{\texttt{\ "vertical"\ }}\texttt{\ } . The \texttt{\ inv\ }
method returns a direction\textquotesingle s inverse direction.

For example,
\texttt{\ ttb\ }{\texttt{\ .\ }}\texttt{\ }{\texttt{\ start\ }}\texttt{\ }{\texttt{\ (\ }}\texttt{\ }{\texttt{\ )\ }}\texttt{\ }
is \texttt{\ top\ } ,
\texttt{\ ttb\ }{\texttt{\ .\ }}\texttt{\ }{\texttt{\ end\ }}\texttt{\ }{\texttt{\ (\ }}\texttt{\ }{\texttt{\ )\ }}\texttt{\ }
is \texttt{\ bottom\ } ,
\texttt{\ ttb\ }{\texttt{\ .\ }}\texttt{\ }{\texttt{\ axis\ }}\texttt{\ }{\texttt{\ (\ }}\texttt{\ }{\texttt{\ )\ }}\texttt{\ }
is \texttt{\ }{\texttt{\ "vertical"\ }}\texttt{\ } and
\texttt{\ ttb\ }{\texttt{\ .\ }}\texttt{\ }{\texttt{\ inv\ }}\texttt{\ }{\texttt{\ (\ }}\texttt{\ }{\texttt{\ )\ }}\texttt{\ }
is equal to \texttt{\ btt\ } .

Default: \texttt{\ ttb\ }

\subsubsection{\texorpdfstring{\texttt{\ spacing\ }}{ spacing }}\label{parameters-spacing}

\href{/docs/reference/foundations/none/}{none} {or}
\href{/docs/reference/layout/relative/}{relative} {or}
\href{/docs/reference/layout/fraction/}{fraction}

{{ Settable }}

\phantomsection\label{parameters-spacing-settable-tooltip}
Settable parameters can be customized for all following uses of the
function with a \texttt{\ set\ } rule.

Spacing to insert between items where no explicit spacing was provided.

Default: \texttt{\ }{\texttt{\ none\ }}\texttt{\ }

\subsubsection{\texorpdfstring{\texttt{\ children\ }}{ children }}\label{parameters-children}

\href{/docs/reference/layout/relative/}{relative} {or}
\href{/docs/reference/layout/fraction/}{fraction} {or}
\href{/docs/reference/foundations/content/}{content}

{Required} {{ Positional }}

\phantomsection\label{parameters-children-positional-tooltip}
Positional parameters are specified in order, without names.

{{ Variadic }}

\phantomsection\label{parameters-children-variadic-tooltip}
Variadic parameters can be specified multiple times.

The children to stack along the axis.

\href{/docs/reference/layout/v/}{\pandocbounded{\includesvg[keepaspectratio]{/assets/icons/16-arrow-right.svg}}}

{ Spacing (V) } { Previous page }

\href{/docs/reference/visualize/}{\pandocbounded{\includesvg[keepaspectratio]{/assets/icons/16-arrow-right.svg}}}

{ Visualize } { Next page }


\section{Docs LaTeX/typst.app/docs/reference/layout/pagebreak.tex}
\title{typst.app/docs/reference/layout/pagebreak}

\begin{itemize}
\tightlist
\item
  \href{/docs}{\includesvg[width=0.16667in,height=0.16667in]{/assets/icons/16-docs-dark.svg}}
\item
  \includesvg[width=0.16667in,height=0.16667in]{/assets/icons/16-arrow-right.svg}
\item
  \href{/docs/reference/}{Reference}
\item
  \includesvg[width=0.16667in,height=0.16667in]{/assets/icons/16-arrow-right.svg}
\item
  \href{/docs/reference/layout/}{Layout}
\item
  \includesvg[width=0.16667in,height=0.16667in]{/assets/icons/16-arrow-right.svg}
\item
  \href{/docs/reference/layout/pagebreak/}{Page Break}
\end{itemize}

\section{\texorpdfstring{\texttt{\ pagebreak\ } {{ Element
}}}{ pagebreak   Element }}\label{summary}

\phantomsection\label{element-tooltip}
Element functions can be customized with \texttt{\ set\ } and
\texttt{\ show\ } rules.

A manual page break.

Must not be used inside any containers.

\subsection{Example}\label{example}

\begin{verbatim}
The next page contains
more details on compound theory.
#pagebreak()

== Compound Theory
In 1984, the first ...
\end{verbatim}

\includegraphics[width=5in,height=\textheight,keepaspectratio]{/assets/docs/MJju6am_GVBgtJWStEY3AwAAAAAAAAAA.png}
\includegraphics[width=5in,height=\textheight,keepaspectratio]{/assets/docs/MJju6am_GVBgtJWStEY3AwAAAAAAAAAB.png}

\subsection{\texorpdfstring{{ Parameters
}}{ Parameters }}\label{parameters}

\phantomsection\label{parameters-tooltip}
Parameters are the inputs to a function. They are specified in
parentheses after the function name.

{ pagebreak } (

{ \hyperref[parameters-weak]{weak :}
\href{/docs/reference/foundations/bool/}{bool} , } {
\hyperref[parameters-to]{to :}
\href{/docs/reference/foundations/none/}{none}
\href{/docs/reference/foundations/str/}{str} , }

) -\textgreater{} \href{/docs/reference/foundations/content/}{content}

\subsubsection{\texorpdfstring{\texttt{\ weak\ }}{ weak }}\label{parameters-weak}

\href{/docs/reference/foundations/bool/}{bool}

{{ Settable }}

\phantomsection\label{parameters-weak-settable-tooltip}
Settable parameters can be customized for all following uses of the
function with a \texttt{\ set\ } rule.

If \texttt{\ }{\texttt{\ true\ }}\texttt{\ } , the page break is skipped
if the current page is already empty.

Default: \texttt{\ }{\texttt{\ false\ }}\texttt{\ }

\subsubsection{\texorpdfstring{\texttt{\ to\ }}{ to }}\label{parameters-to}

\href{/docs/reference/foundations/none/}{none} {or}
\href{/docs/reference/foundations/str/}{str}

{{ Settable }}

\phantomsection\label{parameters-to-settable-tooltip}
Settable parameters can be customized for all following uses of the
function with a \texttt{\ set\ } rule.

If given, ensures that the next page will be an even/odd page, with an
empty page in between if necessary.

\begin{longtable}[]{@{}ll@{}}
\toprule\noalign{}
Variant & Details \\
\midrule\noalign{}
\endhead
\bottomrule\noalign{}
\endlastfoot
\texttt{\ "\ even\ "\ } & Next page will be an even page. \\
\texttt{\ "\ odd\ "\ } & Next page will be an odd page. \\
\end{longtable}

Default: \texttt{\ }{\texttt{\ none\ }}\texttt{\ }

\includesvg[width=0.16667in,height=0.16667in]{/assets/icons/16-arrow-right.svg}
View example

\begin{verbatim}
#set page(height: 30pt)

First.
#pagebreak(to: "odd")
Third.
\end{verbatim}

\includegraphics[width=5in,height=\textheight,keepaspectratio]{/assets/docs/_4CDe0eaU4eyZtVUd1ArigAAAAAAAAAA.png}
\includegraphics[width=5in,height=\textheight,keepaspectratio]{/assets/docs/_4CDe0eaU4eyZtVUd1ArigAAAAAAAAAB.png}
\includegraphics[width=5in,height=\textheight,keepaspectratio]{/assets/docs/_4CDe0eaU4eyZtVUd1ArigAAAAAAAAAC.png}

\href{/docs/reference/layout/page/}{\pandocbounded{\includesvg[keepaspectratio]{/assets/icons/16-arrow-right.svg}}}

{ Page } { Previous page }

\href{/docs/reference/layout/place/}{\pandocbounded{\includesvg[keepaspectratio]{/assets/icons/16-arrow-right.svg}}}

{ Place } { Next page }


\section{Docs LaTeX/typst.app/docs/reference/layout/layout.tex}
\title{typst.app/docs/reference/layout/layout}

\begin{itemize}
\tightlist
\item
  \href{/docs}{\includesvg[width=0.16667in,height=0.16667in]{/assets/icons/16-docs-dark.svg}}
\item
  \includesvg[width=0.16667in,height=0.16667in]{/assets/icons/16-arrow-right.svg}
\item
  \href{/docs/reference/}{Reference}
\item
  \includesvg[width=0.16667in,height=0.16667in]{/assets/icons/16-arrow-right.svg}
\item
  \href{/docs/reference/layout/}{Layout}
\item
  \includesvg[width=0.16667in,height=0.16667in]{/assets/icons/16-arrow-right.svg}
\item
  \href{/docs/reference/layout/layout/}{Layout}
\end{itemize}

\section{\texorpdfstring{\texttt{\ layout\ }}{ layout }}\label{summary}

Provides access to the current outer container\textquotesingle s (or
page\textquotesingle s, if none) dimensions (width and height).

Accepts a function that receives a single parameter, which is a
dictionary with keys \texttt{\ width\ } and \texttt{\ height\ } , both
of type \href{/docs/reference/layout/length/}{\texttt{\ length\ }} . The
function is provided \href{/docs/reference/context/}{context} , meaning
you don\textquotesingle t need to use it in combination with the
\texttt{\ context\ } keyword. This is why
\href{/docs/reference/layout/measure/}{\texttt{\ measure\ }} can be
called in the example below.

\begin{verbatim}
#let text = lorem(30)
#layout(size => [
  #let (height,) = measure(
    block(width: size.width, text),
  )
  This text is #height high with
  the current page width: \
  #text
])
\end{verbatim}

\includegraphics[width=5in,height=\textheight,keepaspectratio]{/assets/docs/SI9ZxtAftdvELQJYlwu_CgAAAAAAAAAA.png}

Note that the \texttt{\ layout\ } function forces its contents into a
\href{/docs/reference/layout/block/}{block} -level container, so
placement relative to the page or pagebreaks are not possible within it.

If the \texttt{\ layout\ } call is placed inside a box with a width of
\texttt{\ }{\texttt{\ 800pt\ }}\texttt{\ } and a height of
\texttt{\ }{\texttt{\ 400pt\ }}\texttt{\ } , then the specified function
will be given the argument
\texttt{\ }{\texttt{\ (\ }}\texttt{\ width\ }{\texttt{\ :\ }}\texttt{\ }{\texttt{\ 800pt\ }}\texttt{\ }{\texttt{\ ,\ }}\texttt{\ height\ }{\texttt{\ :\ }}\texttt{\ }{\texttt{\ 400pt\ }}\texttt{\ }{\texttt{\ )\ }}\texttt{\ }
. If it is placed directly into the page, it receives the
page\textquotesingle s dimensions minus its margins. This is mostly
useful in combination with
\href{/docs/reference/layout/measure/}{measurement} .

You can also use this function to resolve
\href{/docs/reference/layout/ratio/}{\texttt{\ ratio\ }} to fixed
lengths. This might come in handy if you\textquotesingle re building
your own layout abstractions.

\begin{verbatim}
#layout(size => {
  let half = 50% * size.width
  [Half a page is #half wide.]
})
\end{verbatim}

\includegraphics[width=5in,height=\textheight,keepaspectratio]{/assets/docs/1AoOPrEARH2i9ZcdcamicAAAAAAAAAAA.png}

Note that the width or height provided by \texttt{\ layout\ } will be
infinite if the corresponding page dimension is set to
\texttt{\ }{\texttt{\ auto\ }}\texttt{\ } .

\subsection{\texorpdfstring{{ Parameters
}}{ Parameters }}\label{parameters}

\phantomsection\label{parameters-tooltip}
Parameters are the inputs to a function. They are specified in
parentheses after the function name.

{ layout } (

{ \href{/docs/reference/foundations/function/}{function} }

) -\textgreater{} \href{/docs/reference/foundations/content/}{content}

\subsubsection{\texorpdfstring{\texttt{\ func\ }}{ func }}\label{parameters-func}

\href{/docs/reference/foundations/function/}{function}

{Required} {{ Positional }}

\phantomsection\label{parameters-func-positional-tooltip}
Positional parameters are specified in order, without names.

A function to call with the outer container\textquotesingle s size. Its
return value is displayed in the document.

The container\textquotesingle s size is given as a
\href{/docs/reference/foundations/dictionary/}{dictionary} with the keys
\texttt{\ width\ } and \texttt{\ height\ } .

This function is called once for each time the content returned by
\texttt{\ layout\ } appears in the document. This makes it possible to
generate content that depends on the dimensions of its container.

\href{/docs/reference/layout/hide/}{\pandocbounded{\includesvg[keepaspectratio]{/assets/icons/16-arrow-right.svg}}}

{ Hide } { Previous page }

\href{/docs/reference/layout/length/}{\pandocbounded{\includesvg[keepaspectratio]{/assets/icons/16-arrow-right.svg}}}

{ Length } { Next page }


\section{Docs LaTeX/typst.app/docs/reference/layout/block.tex}
\title{typst.app/docs/reference/layout/block}

\begin{itemize}
\tightlist
\item
  \href{/docs}{\includesvg[width=0.16667in,height=0.16667in]{/assets/icons/16-docs-dark.svg}}
\item
  \includesvg[width=0.16667in,height=0.16667in]{/assets/icons/16-arrow-right.svg}
\item
  \href{/docs/reference/}{Reference}
\item
  \includesvg[width=0.16667in,height=0.16667in]{/assets/icons/16-arrow-right.svg}
\item
  \href{/docs/reference/layout/}{Layout}
\item
  \includesvg[width=0.16667in,height=0.16667in]{/assets/icons/16-arrow-right.svg}
\item
  \href{/docs/reference/layout/block/}{Block}
\end{itemize}

\section{\texorpdfstring{\texttt{\ block\ } {{ Element
}}}{ block   Element }}\label{summary}

\phantomsection\label{element-tooltip}
Element functions can be customized with \texttt{\ set\ } and
\texttt{\ show\ } rules.

A block-level container.

Such a container can be used to separate content, size it, and give it a
background or border.

\subsection{Examples}\label{examples}

With a block, you can give a background to content while still allowing
it to break across multiple pages.

\begin{verbatim}
#set page(height: 100pt)
#block(
  fill: luma(230),
  inset: 8pt,
  radius: 4pt,
  lorem(30),
)
\end{verbatim}

\includegraphics[width=5in,height=\textheight,keepaspectratio]{/assets/docs/ANNbdXVxvjEeHE66qUzAcwAAAAAAAAAA.png}
\includegraphics[width=5in,height=\textheight,keepaspectratio]{/assets/docs/ANNbdXVxvjEeHE66qUzAcwAAAAAAAAAB.png}

Blocks are also useful to force elements that would otherwise be inline
to become block-level, especially when writing show rules.

\begin{verbatim}
#show heading: it => it.body
= Blockless
More text.

#show heading: it => block(it.body)
= Blocky
More text.
\end{verbatim}

\includegraphics[width=5in,height=\textheight,keepaspectratio]{/assets/docs/oxrD9vHAqcb-9gLEkFF_PQAAAAAAAAAA.png}

\subsection{\texorpdfstring{{ Parameters
}}{ Parameters }}\label{parameters}

\phantomsection\label{parameters-tooltip}
Parameters are the inputs to a function. They are specified in
parentheses after the function name.

{ block } (

{ \hyperref[parameters-width]{width :}
\href{/docs/reference/foundations/auto/}{auto}
\href{/docs/reference/layout/relative/}{relative} , } {
\hyperref[parameters-height]{height :}
\href{/docs/reference/foundations/auto/}{auto}
\href{/docs/reference/layout/relative/}{relative}
\href{/docs/reference/layout/fraction/}{fraction} , } {
\hyperref[parameters-breakable]{breakable :}
\href{/docs/reference/foundations/bool/}{bool} , } {
\hyperref[parameters-fill]{fill :}
\href{/docs/reference/foundations/none/}{none}
\href{/docs/reference/visualize/color/}{color}
\href{/docs/reference/visualize/gradient/}{gradient}
\href{/docs/reference/visualize/pattern/}{pattern} , } {
\hyperref[parameters-stroke]{stroke :}
\href{/docs/reference/foundations/none/}{none}
\href{/docs/reference/layout/length/}{length}
\href{/docs/reference/visualize/color/}{color}
\href{/docs/reference/visualize/gradient/}{gradient}
\href{/docs/reference/visualize/stroke/}{stroke}
\href{/docs/reference/visualize/pattern/}{pattern}
\href{/docs/reference/foundations/dictionary/}{dictionary} , } {
\hyperref[parameters-radius]{radius :}
\href{/docs/reference/layout/relative/}{relative}
\href{/docs/reference/foundations/dictionary/}{dictionary} , } {
\hyperref[parameters-inset]{inset :}
\href{/docs/reference/layout/relative/}{relative}
\href{/docs/reference/foundations/dictionary/}{dictionary} , } {
\hyperref[parameters-outset]{outset :}
\href{/docs/reference/layout/relative/}{relative}
\href{/docs/reference/foundations/dictionary/}{dictionary} , } {
\hyperref[parameters-spacing]{spacing :}
\href{/docs/reference/layout/relative/}{relative}
\href{/docs/reference/layout/fraction/}{fraction} , } {
\hyperref[parameters-above]{above :}
\href{/docs/reference/foundations/auto/}{auto}
\href{/docs/reference/layout/relative/}{relative}
\href{/docs/reference/layout/fraction/}{fraction} , } {
\hyperref[parameters-below]{below :}
\href{/docs/reference/foundations/auto/}{auto}
\href{/docs/reference/layout/relative/}{relative}
\href{/docs/reference/layout/fraction/}{fraction} , } {
\hyperref[parameters-clip]{clip :}
\href{/docs/reference/foundations/bool/}{bool} , } {
\hyperref[parameters-sticky]{sticky :}
\href{/docs/reference/foundations/bool/}{bool} , } {
\hyperref[parameters-body]{}
\href{/docs/reference/foundations/none/}{none}
\href{/docs/reference/foundations/content/}{content} , }

) -\textgreater{} \href{/docs/reference/foundations/content/}{content}

\subsubsection{\texorpdfstring{\texttt{\ width\ }}{ width }}\label{parameters-width}

\href{/docs/reference/foundations/auto/}{auto} {or}
\href{/docs/reference/layout/relative/}{relative}

{{ Settable }}

\phantomsection\label{parameters-width-settable-tooltip}
Settable parameters can be customized for all following uses of the
function with a \texttt{\ set\ } rule.

The block\textquotesingle s width.

Default: \texttt{\ }{\texttt{\ auto\ }}\texttt{\ }

\includesvg[width=0.16667in,height=0.16667in]{/assets/icons/16-arrow-right.svg}
View example

\begin{verbatim}
#set align(center)
#block(
  width: 60%,
  inset: 8pt,
  fill: silver,
  lorem(10),
)
\end{verbatim}

\includegraphics[width=5in,height=\textheight,keepaspectratio]{/assets/docs/rmTSlZT-FzVZcPQGVLOIiwAAAAAAAAAA.png}

\subsubsection{\texorpdfstring{\texttt{\ height\ }}{ height }}\label{parameters-height}

\href{/docs/reference/foundations/auto/}{auto} {or}
\href{/docs/reference/layout/relative/}{relative} {or}
\href{/docs/reference/layout/fraction/}{fraction}

{{ Settable }}

\phantomsection\label{parameters-height-settable-tooltip}
Settable parameters can be customized for all following uses of the
function with a \texttt{\ set\ } rule.

The block\textquotesingle s height. When the height is larger than the
remaining space on a page and
\href{/docs/reference/layout/block/\#parameters-breakable}{\texttt{\ breakable\ }}
is \texttt{\ }{\texttt{\ true\ }}\texttt{\ } , the block will continue
on the next page with the remaining height.

Default: \texttt{\ }{\texttt{\ auto\ }}\texttt{\ }

\includesvg[width=0.16667in,height=0.16667in]{/assets/icons/16-arrow-right.svg}
View example

\begin{verbatim}
#set page(height: 80pt)
#set align(center)
#block(
  width: 80%,
  height: 150%,
  fill: aqua,
)
\end{verbatim}

\includegraphics[width=5in,height=\textheight,keepaspectratio]{/assets/docs/lezx_tGBIjN0y72kerj7yQAAAAAAAAAA.png}
\includegraphics[width=5in,height=\textheight,keepaspectratio]{/assets/docs/lezx_tGBIjN0y72kerj7yQAAAAAAAAAB.png}

\subsubsection{\texorpdfstring{\texttt{\ breakable\ }}{ breakable }}\label{parameters-breakable}

\href{/docs/reference/foundations/bool/}{bool}

{{ Settable }}

\phantomsection\label{parameters-breakable-settable-tooltip}
Settable parameters can be customized for all following uses of the
function with a \texttt{\ set\ } rule.

Whether the block can be broken and continue on the next page.

Default: \texttt{\ }{\texttt{\ true\ }}\texttt{\ }

\includesvg[width=0.16667in,height=0.16667in]{/assets/icons/16-arrow-right.svg}
View example

\begin{verbatim}
#set page(height: 80pt)
The following block will
jump to its own page.
#block(
  breakable: false,
  lorem(15),
)
\end{verbatim}

\includegraphics[width=5in,height=\textheight,keepaspectratio]{/assets/docs/I4HMzOAjAUbW-RK0a_YVHAAAAAAAAAAA.png}
\includegraphics[width=5in,height=\textheight,keepaspectratio]{/assets/docs/I4HMzOAjAUbW-RK0a_YVHAAAAAAAAAAB.png}

\subsubsection{\texorpdfstring{\texttt{\ fill\ }}{ fill }}\label{parameters-fill}

\href{/docs/reference/foundations/none/}{none} {or}
\href{/docs/reference/visualize/color/}{color} {or}
\href{/docs/reference/visualize/gradient/}{gradient} {or}
\href{/docs/reference/visualize/pattern/}{pattern}

{{ Settable }}

\phantomsection\label{parameters-fill-settable-tooltip}
Settable parameters can be customized for all following uses of the
function with a \texttt{\ set\ } rule.

The block\textquotesingle s background color. See the
\href{/docs/reference/visualize/rect/\#parameters-fill}{rectangle\textquotesingle s
documentation} for more details.

Default: \texttt{\ }{\texttt{\ none\ }}\texttt{\ }

\subsubsection{\texorpdfstring{\texttt{\ stroke\ }}{ stroke }}\label{parameters-stroke}

\href{/docs/reference/foundations/none/}{none} {or}
\href{/docs/reference/layout/length/}{length} {or}
\href{/docs/reference/visualize/color/}{color} {or}
\href{/docs/reference/visualize/gradient/}{gradient} {or}
\href{/docs/reference/visualize/stroke/}{stroke} {or}
\href{/docs/reference/visualize/pattern/}{pattern} {or}
\href{/docs/reference/foundations/dictionary/}{dictionary}

{{ Settable }}

\phantomsection\label{parameters-stroke-settable-tooltip}
Settable parameters can be customized for all following uses of the
function with a \texttt{\ set\ } rule.

The block\textquotesingle s border color. See the
\href{/docs/reference/visualize/rect/\#parameters-stroke}{rectangle\textquotesingle s
documentation} for more details.

Default:
\texttt{\ }{\texttt{\ (\ }}\texttt{\ }{\texttt{\ :\ }}\texttt{\ }{\texttt{\ )\ }}\texttt{\ }

\subsubsection{\texorpdfstring{\texttt{\ radius\ }}{ radius }}\label{parameters-radius}

\href{/docs/reference/layout/relative/}{relative} {or}
\href{/docs/reference/foundations/dictionary/}{dictionary}

{{ Settable }}

\phantomsection\label{parameters-radius-settable-tooltip}
Settable parameters can be customized for all following uses of the
function with a \texttt{\ set\ } rule.

How much to round the block\textquotesingle s corners. See the
\href{/docs/reference/visualize/rect/\#parameters-radius}{rectangle\textquotesingle s
documentation} for more details.

Default:
\texttt{\ }{\texttt{\ (\ }}\texttt{\ }{\texttt{\ :\ }}\texttt{\ }{\texttt{\ )\ }}\texttt{\ }

\subsubsection{\texorpdfstring{\texttt{\ inset\ }}{ inset }}\label{parameters-inset}

\href{/docs/reference/layout/relative/}{relative} {or}
\href{/docs/reference/foundations/dictionary/}{dictionary}

{{ Settable }}

\phantomsection\label{parameters-inset-settable-tooltip}
Settable parameters can be customized for all following uses of the
function with a \texttt{\ set\ } rule.

How much to pad the block\textquotesingle s content. See the
\href{/docs/reference/layout/box/\#parameters-inset}{box\textquotesingle s
documentation} for more details.

Default:
\texttt{\ }{\texttt{\ (\ }}\texttt{\ }{\texttt{\ :\ }}\texttt{\ }{\texttt{\ )\ }}\texttt{\ }

\subsubsection{\texorpdfstring{\texttt{\ outset\ }}{ outset }}\label{parameters-outset}

\href{/docs/reference/layout/relative/}{relative} {or}
\href{/docs/reference/foundations/dictionary/}{dictionary}

{{ Settable }}

\phantomsection\label{parameters-outset-settable-tooltip}
Settable parameters can be customized for all following uses of the
function with a \texttt{\ set\ } rule.

How much to expand the block\textquotesingle s size without affecting
the layout. See the
\href{/docs/reference/layout/box/\#parameters-outset}{box\textquotesingle s
documentation} for more details.

Default:
\texttt{\ }{\texttt{\ (\ }}\texttt{\ }{\texttt{\ :\ }}\texttt{\ }{\texttt{\ )\ }}\texttt{\ }

\subsubsection{\texorpdfstring{\texttt{\ spacing\ }}{ spacing }}\label{parameters-spacing}

\href{/docs/reference/layout/relative/}{relative} {or}
\href{/docs/reference/layout/fraction/}{fraction}

{{ Settable }}

\phantomsection\label{parameters-spacing-settable-tooltip}
Settable parameters can be customized for all following uses of the
function with a \texttt{\ set\ } rule.

The spacing around the block. When
\texttt{\ }{\texttt{\ auto\ }}\texttt{\ } , inherits the paragraph
\href{/docs/reference/model/par/\#parameters-spacing}{\texttt{\ spacing\ }}
.

For two adjacent blocks, the larger of the first block\textquotesingle s
\texttt{\ above\ } and the second block\textquotesingle s
\texttt{\ below\ } spacing wins. Moreover, block spacing takes
precedence over paragraph
\href{/docs/reference/model/par/\#parameters-spacing}{\texttt{\ spacing\ }}
.

Note that this is only a shorthand to set \texttt{\ above\ } and
\texttt{\ below\ } to the same value. Since the values for
\texttt{\ above\ } and \texttt{\ below\ } might differ, a
\href{/docs/reference/context/}{context} block only provides access to
\texttt{\ block\ }{\texttt{\ .\ }}\texttt{\ above\ } and
\texttt{\ block\ }{\texttt{\ .\ }}\texttt{\ below\ } , not to
\texttt{\ block\ }{\texttt{\ .\ }}\texttt{\ spacing\ } directly.

This property can be used in combination with a show rule to adjust the
spacing around arbitrary block-level elements.

Default: \texttt{\ }{\texttt{\ 1.2em\ }}\texttt{\ }

\includesvg[width=0.16667in,height=0.16667in]{/assets/icons/16-arrow-right.svg}
View example

\begin{verbatim}
#set align(center)
#show math.equation: set block(above: 8pt, below: 16pt)

This sum of $x$ and $y$:
$ x + y = z $
A second paragraph.
\end{verbatim}

\includegraphics[width=5in,height=\textheight,keepaspectratio]{/assets/docs/-Z0A6wte5TbEZ6mEwTPvngAAAAAAAAAA.png}

\subsubsection{\texorpdfstring{\texttt{\ above\ }}{ above }}\label{parameters-above}

\href{/docs/reference/foundations/auto/}{auto} {or}
\href{/docs/reference/layout/relative/}{relative} {or}
\href{/docs/reference/layout/fraction/}{fraction}

{{ Settable }}

\phantomsection\label{parameters-above-settable-tooltip}
Settable parameters can be customized for all following uses of the
function with a \texttt{\ set\ } rule.

The spacing between this block and its predecessor.

Default: \texttt{\ }{\texttt{\ auto\ }}\texttt{\ }

\subsubsection{\texorpdfstring{\texttt{\ below\ }}{ below }}\label{parameters-below}

\href{/docs/reference/foundations/auto/}{auto} {or}
\href{/docs/reference/layout/relative/}{relative} {or}
\href{/docs/reference/layout/fraction/}{fraction}

{{ Settable }}

\phantomsection\label{parameters-below-settable-tooltip}
Settable parameters can be customized for all following uses of the
function with a \texttt{\ set\ } rule.

The spacing between this block and its successor.

Default: \texttt{\ }{\texttt{\ auto\ }}\texttt{\ }

\subsubsection{\texorpdfstring{\texttt{\ clip\ }}{ clip }}\label{parameters-clip}

\href{/docs/reference/foundations/bool/}{bool}

{{ Settable }}

\phantomsection\label{parameters-clip-settable-tooltip}
Settable parameters can be customized for all following uses of the
function with a \texttt{\ set\ } rule.

Whether to clip the content inside the block.

Clipping is useful when the block\textquotesingle s content is larger
than the block itself, as any content that exceeds the
block\textquotesingle s bounds will be hidden.

Default: \texttt{\ }{\texttt{\ false\ }}\texttt{\ }

\includesvg[width=0.16667in,height=0.16667in]{/assets/icons/16-arrow-right.svg}
View example

\begin{verbatim}
#block(
  width: 50pt,
  height: 50pt,
  clip: true,
  image("tiger.jpg", width: 100pt, height: 100pt)
)
\end{verbatim}

\includegraphics[width=5in,height=\textheight,keepaspectratio]{/assets/docs/VV4XHW5eLH_lso6MwHK6pQAAAAAAAAAA.png}

\subsubsection{\texorpdfstring{\texttt{\ sticky\ }}{ sticky }}\label{parameters-sticky}

\href{/docs/reference/foundations/bool/}{bool}

{{ Settable }}

\phantomsection\label{parameters-sticky-settable-tooltip}
Settable parameters can be customized for all following uses of the
function with a \texttt{\ set\ } rule.

Whether this block must stick to the following one, with no break in
between.

This is, by default, set on heading blocks to prevent orphaned headings
at the bottom of the page.

Default: \texttt{\ }{\texttt{\ false\ }}\texttt{\ }

\includesvg[width=0.16667in,height=0.16667in]{/assets/icons/16-arrow-right.svg}
View example

\begin{verbatim}
// Disable stickiness of headings.
#show heading: set block(sticky: false)
#lorem(20)

= Chapter
#lorem(10)
\end{verbatim}

\includegraphics[width=5in,height=\textheight,keepaspectratio]{/assets/docs/9rTrIlbIWN6fRV2-gOoijQAAAAAAAAAA.png}
\includegraphics[width=5in,height=\textheight,keepaspectratio]{/assets/docs/9rTrIlbIWN6fRV2-gOoijQAAAAAAAAAB.png}

\subsubsection{\texorpdfstring{\texttt{\ body\ }}{ body }}\label{parameters-body}

\href{/docs/reference/foundations/none/}{none} {or}
\href{/docs/reference/foundations/content/}{content}

{{ Positional }}

\phantomsection\label{parameters-body-positional-tooltip}
Positional parameters are specified in order, without names.

{{ Settable }}

\phantomsection\label{parameters-body-settable-tooltip}
Settable parameters can be customized for all following uses of the
function with a \texttt{\ set\ } rule.

The contents of the block.

Default: \texttt{\ }{\texttt{\ none\ }}\texttt{\ }

\href{/docs/reference/layout/angle/}{\pandocbounded{\includesvg[keepaspectratio]{/assets/icons/16-arrow-right.svg}}}

{ Angle } { Previous page }

\href{/docs/reference/layout/box/}{\pandocbounded{\includesvg[keepaspectratio]{/assets/icons/16-arrow-right.svg}}}

{ Box } { Next page }


\section{Docs LaTeX/typst.app/docs/reference/layout/ratio.tex}
\title{typst.app/docs/reference/layout/ratio}

\begin{itemize}
\tightlist
\item
  \href{/docs}{\includesvg[width=0.16667in,height=0.16667in]{/assets/icons/16-docs-dark.svg}}
\item
  \includesvg[width=0.16667in,height=0.16667in]{/assets/icons/16-arrow-right.svg}
\item
  \href{/docs/reference/}{Reference}
\item
  \includesvg[width=0.16667in,height=0.16667in]{/assets/icons/16-arrow-right.svg}
\item
  \href{/docs/reference/layout/}{Layout}
\item
  \includesvg[width=0.16667in,height=0.16667in]{/assets/icons/16-arrow-right.svg}
\item
  \href{/docs/reference/layout/ratio/}{Ratio}
\end{itemize}

\section{\texorpdfstring{{ ratio }}{ ratio }}\label{summary}

A ratio of a whole.

Written as a number, followed by a percent sign.

\subsection{Example}\label{example}

\begin{verbatim}
#set align(center)
#scale(x: 150%)[
  Scaled apart.
]
\end{verbatim}

\includegraphics[width=5in,height=\textheight,keepaspectratio]{/assets/docs/xEgSJZQe3kQz-XQhwaSthwAAAAAAAAAA.png}

\href{/docs/reference/layout/place/}{\pandocbounded{\includesvg[keepaspectratio]{/assets/icons/16-arrow-right.svg}}}

{ Place } { Previous page }

\href{/docs/reference/layout/relative/}{\pandocbounded{\includesvg[keepaspectratio]{/assets/icons/16-arrow-right.svg}}}

{ Relative Length } { Next page }




\section{C Docs LaTeX/docs/reference/data-loading.tex}
\section{Docs LaTeX/typst.app/docs/reference/data-loading/xml.tex}
\title{typst.app/docs/reference/data-loading/xml}

\begin{itemize}
\tightlist
\item
  \href{/docs}{\includesvg[width=0.16667in,height=0.16667in]{/assets/icons/16-docs-dark.svg}}
\item
  \includesvg[width=0.16667in,height=0.16667in]{/assets/icons/16-arrow-right.svg}
\item
  \href{/docs/reference/}{Reference}
\item
  \includesvg[width=0.16667in,height=0.16667in]{/assets/icons/16-arrow-right.svg}
\item
  \href{/docs/reference/data-loading/}{Data Loading}
\item
  \includesvg[width=0.16667in,height=0.16667in]{/assets/icons/16-arrow-right.svg}
\item
  \href{/docs/reference/data-loading/xml/}{XML}
\end{itemize}

\section{\texorpdfstring{\texttt{\ xml\ }}{ xml }}\label{summary}

Reads structured data from an XML file.

The XML file is parsed into an array of dictionaries and strings. XML
nodes can be elements or strings. Elements are represented as
dictionaries with the following keys:

\begin{itemize}
\tightlist
\item
  \texttt{\ tag\ } : The name of the element as a string.
\item
  \texttt{\ attrs\ } : A dictionary of the element\textquotesingle s
  attributes as strings.
\item
  \texttt{\ children\ } : An array of the element\textquotesingle s
  child nodes.
\end{itemize}

The XML file in the example contains a root \texttt{\ news\ } tag with
multiple \texttt{\ article\ } tags. Each article has a
\texttt{\ title\ } , \texttt{\ author\ } , and \texttt{\ content\ } tag.
The \texttt{\ content\ } tag contains one or more paragraphs, which are
represented as \texttt{\ p\ } tags.

\subsection{Example}\label{example}

\begin{verbatim}
#let find-child(elem, tag) = {
  elem.children
    .find(e => "tag" in e and e.tag == tag)
}

#let article(elem) = {
  let title = find-child(elem, "title")
  let author = find-child(elem, "author")
  let pars = find-child(elem, "content")

  heading(title.children.first())
  text(10pt, weight: "medium")[
    Published by
    #author.children.first()
  ]

  for p in pars.children {
    if (type(p) == "dictionary") {
      parbreak()
      p.children.first()
    }
  }
}

#let data = xml("example.xml")
#for elem in data.first().children {
  if (type(elem) == "dictionary") {
    article(elem)
  }
}
\end{verbatim}

\includegraphics[width=5in,height=\textheight,keepaspectratio]{/assets/docs/ImsUm8fcO-Uh3s95k6HvEQAAAAAAAAAA.png}

\subsection{\texorpdfstring{{ Parameters
}}{ Parameters }}\label{parameters}

\phantomsection\label{parameters-tooltip}
Parameters are the inputs to a function. They are specified in
parentheses after the function name.

{ xml } (

{ \href{/docs/reference/foundations/str/}{str} }

) -\textgreater{} { any }

\subsubsection{\texorpdfstring{\texttt{\ path\ }}{ path }}\label{parameters-path}

\href{/docs/reference/foundations/str/}{str}

{Required} {{ Positional }}

\phantomsection\label{parameters-path-positional-tooltip}
Positional parameters are specified in order, without names.

Path to an XML file.

For more details, see the \href{/docs/reference/syntax/\#paths}{Paths
section} .

\subsection{\texorpdfstring{{ Definitions
}}{ Definitions }}\label{definitions}

\phantomsection\label{definitions-tooltip}
Functions and types and can have associated definitions. These are
accessed by specifying the function or type, followed by a period, and
then the definition\textquotesingle s name.

\subsubsection{\texorpdfstring{\texttt{\ decode\ }}{ decode }}\label{definitions-decode}

Reads structured data from an XML string/bytes.

xml { . } { decode } (

{ \href{/docs/reference/foundations/str/}{str}
\href{/docs/reference/foundations/bytes/}{bytes} }

) -\textgreater{} { any }

\paragraph{\texorpdfstring{\texttt{\ data\ }}{ data }}\label{definitions-decode-data}

\href{/docs/reference/foundations/str/}{str} {or}
\href{/docs/reference/foundations/bytes/}{bytes}

{Required} {{ Positional }}

\phantomsection\label{definitions-decode-data-positional-tooltip}
Positional parameters are specified in order, without names.

XML data.

\href{/docs/reference/data-loading/toml/}{\pandocbounded{\includesvg[keepaspectratio]{/assets/icons/16-arrow-right.svg}}}

{ TOML } { Previous page }

\href{/docs/reference/data-loading/yaml/}{\pandocbounded{\includesvg[keepaspectratio]{/assets/icons/16-arrow-right.svg}}}

{ YAML } { Next page }


\section{Docs LaTeX/typst.app/docs/reference/data-loading/json.tex}
\title{typst.app/docs/reference/data-loading/json}

\begin{itemize}
\tightlist
\item
  \href{/docs}{\includesvg[width=0.16667in,height=0.16667in]{/assets/icons/16-docs-dark.svg}}
\item
  \includesvg[width=0.16667in,height=0.16667in]{/assets/icons/16-arrow-right.svg}
\item
  \href{/docs/reference/}{Reference}
\item
  \includesvg[width=0.16667in,height=0.16667in]{/assets/icons/16-arrow-right.svg}
\item
  \href{/docs/reference/data-loading/}{Data Loading}
\item
  \includesvg[width=0.16667in,height=0.16667in]{/assets/icons/16-arrow-right.svg}
\item
  \href{/docs/reference/data-loading/json/}{JSON}
\end{itemize}

\section{\texorpdfstring{\texttt{\ json\ }}{ json }}\label{summary}

Reads structured data from a JSON file.

The file must contain a valid JSON value, such as object or array. JSON
objects will be converted into Typst dictionaries, and JSON arrays will
be converted into Typst arrays. Strings and booleans will be converted
into the Typst equivalents, \texttt{\ null\ } will be converted into
\texttt{\ }{\texttt{\ none\ }}\texttt{\ } , and numbers will be
converted to floats or integers depending on whether they are whole
numbers.

Be aware that integers larger than 2 \textsuperscript{63} -1 will be
converted to floating point numbers, which may result in an
approximative value.

The function returns a dictionary, an array or, depending on the JSON
file, another JSON data type.

The JSON files in the example contain objects with the keys
\texttt{\ temperature\ } , \texttt{\ unit\ } , and \texttt{\ weather\ }
.

\subsection{Example}\label{example}

\begin{verbatim}
#let forecast(day) = block[
  #box(square(
    width: 2cm,
    inset: 8pt,
    fill: if day.weather == "sunny" {
      yellow
    } else {
      aqua
    },
    align(
      bottom + right,
      strong(day.weather),
    ),
  ))
  #h(6pt)
  #set text(22pt, baseline: -8pt)
  #day.temperature °#day.unit
]

#forecast(json("monday.json"))
#forecast(json("tuesday.json"))
\end{verbatim}

\includegraphics[width=5in,height=\textheight,keepaspectratio]{/assets/docs/9TGGThvdnznDbVRRo5-HsgAAAAAAAAAA.png}

\subsection{\texorpdfstring{{ Parameters
}}{ Parameters }}\label{parameters}

\phantomsection\label{parameters-tooltip}
Parameters are the inputs to a function. They are specified in
parentheses after the function name.

{ json } (

{ \href{/docs/reference/foundations/str/}{str} }

) -\textgreater{} { any }

\subsubsection{\texorpdfstring{\texttt{\ path\ }}{ path }}\label{parameters-path}

\href{/docs/reference/foundations/str/}{str}

{Required} {{ Positional }}

\phantomsection\label{parameters-path-positional-tooltip}
Positional parameters are specified in order, without names.

Path to a JSON file.

For more details, see the \href{/docs/reference/syntax/\#paths}{Paths
section} .

\subsection{\texorpdfstring{{ Definitions
}}{ Definitions }}\label{definitions}

\phantomsection\label{definitions-tooltip}
Functions and types and can have associated definitions. These are
accessed by specifying the function or type, followed by a period, and
then the definition\textquotesingle s name.

\subsubsection{\texorpdfstring{\texttt{\ decode\ }}{ decode }}\label{definitions-decode}

Reads structured data from a JSON string/bytes.

json { . } { decode } (

{ \href{/docs/reference/foundations/str/}{str}
\href{/docs/reference/foundations/bytes/}{bytes} }

) -\textgreater{} { any }

\paragraph{\texorpdfstring{\texttt{\ data\ }}{ data }}\label{definitions-decode-data}

\href{/docs/reference/foundations/str/}{str} {or}
\href{/docs/reference/foundations/bytes/}{bytes}

{Required} {{ Positional }}

\phantomsection\label{definitions-decode-data-positional-tooltip}
Positional parameters are specified in order, without names.

JSON data.

\subsubsection{\texorpdfstring{\texttt{\ encode\ }}{ encode }}\label{definitions-encode}

Encodes structured data into a JSON string.

json { . } { encode } (

{ { any } , } { \hyperref[definitions-encode-parameters-pretty]{pretty
:} \href{/docs/reference/foundations/bool/}{bool} , }

) -\textgreater{} \href{/docs/reference/foundations/str/}{str}

\paragraph{\texorpdfstring{\texttt{\ value\ }}{ value }}\label{definitions-encode-value}

{ any }

{Required} {{ Positional }}

\phantomsection\label{definitions-encode-value-positional-tooltip}
Positional parameters are specified in order, without names.

Value to be encoded.

\paragraph{\texorpdfstring{\texttt{\ pretty\ }}{ pretty }}\label{definitions-encode-pretty}

\href{/docs/reference/foundations/bool/}{bool}

Whether to pretty print the JSON with newlines and indentation.

Default: \texttt{\ }{\texttt{\ true\ }}\texttt{\ }

\href{/docs/reference/data-loading/csv/}{\pandocbounded{\includesvg[keepaspectratio]{/assets/icons/16-arrow-right.svg}}}

{ CSV } { Previous page }

\href{/docs/reference/data-loading/read/}{\pandocbounded{\includesvg[keepaspectratio]{/assets/icons/16-arrow-right.svg}}}

{ Read } { Next page }


\section{Docs LaTeX/typst.app/docs/reference/data-loading/csv.tex}
\title{typst.app/docs/reference/data-loading/csv}

\begin{itemize}
\tightlist
\item
  \href{/docs}{\includesvg[width=0.16667in,height=0.16667in]{/assets/icons/16-docs-dark.svg}}
\item
  \includesvg[width=0.16667in,height=0.16667in]{/assets/icons/16-arrow-right.svg}
\item
  \href{/docs/reference/}{Reference}
\item
  \includesvg[width=0.16667in,height=0.16667in]{/assets/icons/16-arrow-right.svg}
\item
  \href{/docs/reference/data-loading/}{Data Loading}
\item
  \includesvg[width=0.16667in,height=0.16667in]{/assets/icons/16-arrow-right.svg}
\item
  \href{/docs/reference/data-loading/csv/}{CSV}
\end{itemize}

\section{\texorpdfstring{\texttt{\ csv\ }}{ csv }}\label{summary}

Reads structured data from a CSV file.

The CSV file will be read and parsed into a 2-dimensional array of
strings: Each row in the CSV file will be represented as an array of
strings, and all rows will be collected into a single array. Header rows
will not be stripped.

\subsection{Example}\label{example}

\begin{verbatim}
#let results = csv("example.csv")

#table(
  columns: 2,
  [*Condition*], [*Result*],
  ..results.flatten(),
)
\end{verbatim}

\includegraphics[width=5in,height=\textheight,keepaspectratio]{/assets/docs/wZK4j33X4RoMvhQZsQnpmQAAAAAAAAAA.png}

\subsection{\texorpdfstring{{ Parameters
}}{ Parameters }}\label{parameters}

\phantomsection\label{parameters-tooltip}
Parameters are the inputs to a function. They are specified in
parentheses after the function name.

{ csv } (

{ \href{/docs/reference/foundations/str/}{str} , } {
\hyperref[parameters-delimiter]{delimiter :}
\href{/docs/reference/foundations/str/}{str} , } {
\hyperref[parameters-row-type]{row-type :}
\href{/docs/reference/foundations/type/}{type} , }

) -\textgreater{} \href{/docs/reference/foundations/array/}{array}

\subsubsection{\texorpdfstring{\texttt{\ path\ }}{ path }}\label{parameters-path}

\href{/docs/reference/foundations/str/}{str}

{Required} {{ Positional }}

\phantomsection\label{parameters-path-positional-tooltip}
Positional parameters are specified in order, without names.

Path to a CSV file.

For more details, see the \href{/docs/reference/syntax/\#paths}{Paths
section} .

\subsubsection{\texorpdfstring{\texttt{\ delimiter\ }}{ delimiter }}\label{parameters-delimiter}

\href{/docs/reference/foundations/str/}{str}

The delimiter that separates columns in the CSV file. Must be a single
ASCII character.

Default: \texttt{\ }{\texttt{\ ","\ }}\texttt{\ }

\subsubsection{\texorpdfstring{\texttt{\ row-type\ }}{ row-type }}\label{parameters-row-type}

\href{/docs/reference/foundations/type/}{type}

How to represent the file\textquotesingle s rows.

\begin{itemize}
\tightlist
\item
  If set to \texttt{\ array\ } , each row is represented as a plain
  array of strings.
\item
  If set to \texttt{\ dictionary\ } , each row is represented as a
  dictionary mapping from header keys to strings. This option only makes
  sense when a header row is present in the CSV file.
\end{itemize}

Default: \texttt{\ array\ }

\subsection{\texorpdfstring{{ Definitions
}}{ Definitions }}\label{definitions}

\phantomsection\label{definitions-tooltip}
Functions and types and can have associated definitions. These are
accessed by specifying the function or type, followed by a period, and
then the definition\textquotesingle s name.

\subsubsection{\texorpdfstring{\texttt{\ decode\ }}{ decode }}\label{definitions-decode}

Reads structured data from a CSV string/bytes.

csv { . } { decode } (

{ \href{/docs/reference/foundations/str/}{str}
\href{/docs/reference/foundations/bytes/}{bytes} , } {
\hyperref[definitions-decode-parameters-delimiter]{delimiter :}
\href{/docs/reference/foundations/str/}{str} , } {
\hyperref[definitions-decode-parameters-row-type]{row-type :}
\href{/docs/reference/foundations/type/}{type} , }

) -\textgreater{} \href{/docs/reference/foundations/array/}{array}

\paragraph{\texorpdfstring{\texttt{\ data\ }}{ data }}\label{definitions-decode-data}

\href{/docs/reference/foundations/str/}{str} {or}
\href{/docs/reference/foundations/bytes/}{bytes}

{Required} {{ Positional }}

\phantomsection\label{definitions-decode-data-positional-tooltip}
Positional parameters are specified in order, without names.

CSV data.

\paragraph{\texorpdfstring{\texttt{\ delimiter\ }}{ delimiter }}\label{definitions-decode-delimiter}

\href{/docs/reference/foundations/str/}{str}

The delimiter that separates columns in the CSV file. Must be a single
ASCII character.

Default: \texttt{\ }{\texttt{\ ","\ }}\texttt{\ }

\paragraph{\texorpdfstring{\texttt{\ row-type\ }}{ row-type }}\label{definitions-decode-row-type}

\href{/docs/reference/foundations/type/}{type}

How to represent the file\textquotesingle s rows.

\begin{itemize}
\tightlist
\item
  If set to \texttt{\ array\ } , each row is represented as a plain
  array of strings.
\item
  If set to \texttt{\ dictionary\ } , each row is represented as a
  dictionary mapping from header keys to strings. This option only makes
  sense when a header row is present in the CSV file.
\end{itemize}

Default: \texttt{\ array\ }

\href{/docs/reference/data-loading/cbor/}{\pandocbounded{\includesvg[keepaspectratio]{/assets/icons/16-arrow-right.svg}}}

{ CBOR } { Previous page }

\href{/docs/reference/data-loading/json/}{\pandocbounded{\includesvg[keepaspectratio]{/assets/icons/16-arrow-right.svg}}}

{ JSON } { Next page }


\section{Docs LaTeX/typst.app/docs/reference/data-loading/toml.tex}
\title{typst.app/docs/reference/data-loading/toml}

\begin{itemize}
\tightlist
\item
  \href{/docs}{\includesvg[width=0.16667in,height=0.16667in]{/assets/icons/16-docs-dark.svg}}
\item
  \includesvg[width=0.16667in,height=0.16667in]{/assets/icons/16-arrow-right.svg}
\item
  \href{/docs/reference/}{Reference}
\item
  \includesvg[width=0.16667in,height=0.16667in]{/assets/icons/16-arrow-right.svg}
\item
  \href{/docs/reference/data-loading/}{Data Loading}
\item
  \includesvg[width=0.16667in,height=0.16667in]{/assets/icons/16-arrow-right.svg}
\item
  \href{/docs/reference/data-loading/toml/}{TOML}
\end{itemize}

\section{\texorpdfstring{\texttt{\ toml\ }}{ toml }}\label{summary}

Reads structured data from a TOML file.

The file must contain a valid TOML table. TOML tables will be converted
into Typst dictionaries, and TOML arrays will be converted into Typst
arrays. Strings, booleans and datetimes will be converted into the Typst
equivalents and numbers will be converted to floats or integers
depending on whether they are whole numbers.

The TOML file in the example consists of a table with the keys
\texttt{\ title\ } , \texttt{\ version\ } , and \texttt{\ authors\ } .

\subsection{Example}\label{example}

\begin{verbatim}
#let details = toml("details.toml")

Title: #details.title \
Version: #details.version \
Authors: #(details.authors
  .join(", ", last: " and "))
\end{verbatim}

\includegraphics[width=5in,height=\textheight,keepaspectratio]{/assets/docs/f26frHBWUfr7bIomQ1qwWAAAAAAAAAAA.png}

\subsection{\texorpdfstring{{ Parameters
}}{ Parameters }}\label{parameters}

\phantomsection\label{parameters-tooltip}
Parameters are the inputs to a function. They are specified in
parentheses after the function name.

{ toml } (

{ \href{/docs/reference/foundations/str/}{str} }

) -\textgreater{} { any }

\subsubsection{\texorpdfstring{\texttt{\ path\ }}{ path }}\label{parameters-path}

\href{/docs/reference/foundations/str/}{str}

{Required} {{ Positional }}

\phantomsection\label{parameters-path-positional-tooltip}
Positional parameters are specified in order, without names.

Path to a TOML file.

For more details, see the \href{/docs/reference/syntax/\#paths}{Paths
section} .

\subsection{\texorpdfstring{{ Definitions
}}{ Definitions }}\label{definitions}

\phantomsection\label{definitions-tooltip}
Functions and types and can have associated definitions. These are
accessed by specifying the function or type, followed by a period, and
then the definition\textquotesingle s name.

\subsubsection{\texorpdfstring{\texttt{\ decode\ }}{ decode }}\label{definitions-decode}

Reads structured data from a TOML string/bytes.

toml { . } { decode } (

{ \href{/docs/reference/foundations/str/}{str}
\href{/docs/reference/foundations/bytes/}{bytes} }

) -\textgreater{} { any }

\paragraph{\texorpdfstring{\texttt{\ data\ }}{ data }}\label{definitions-decode-data}

\href{/docs/reference/foundations/str/}{str} {or}
\href{/docs/reference/foundations/bytes/}{bytes}

{Required} {{ Positional }}

\phantomsection\label{definitions-decode-data-positional-tooltip}
Positional parameters are specified in order, without names.

TOML data.

\subsubsection{\texorpdfstring{\texttt{\ encode\ }}{ encode }}\label{definitions-encode}

Encodes structured data into a TOML string.

toml { . } { encode } (

{ { any } , } { \hyperref[definitions-encode-parameters-pretty]{pretty
:} \href{/docs/reference/foundations/bool/}{bool} , }

) -\textgreater{} \href{/docs/reference/foundations/str/}{str}

\paragraph{\texorpdfstring{\texttt{\ value\ }}{ value }}\label{definitions-encode-value}

{ any }

{Required} {{ Positional }}

\phantomsection\label{definitions-encode-value-positional-tooltip}
Positional parameters are specified in order, without names.

Value to be encoded.

\paragraph{\texorpdfstring{\texttt{\ pretty\ }}{ pretty }}\label{definitions-encode-pretty}

\href{/docs/reference/foundations/bool/}{bool}

Whether to pretty-print the resulting TOML.

Default: \texttt{\ }{\texttt{\ true\ }}\texttt{\ }

\href{/docs/reference/data-loading/read/}{\pandocbounded{\includesvg[keepaspectratio]{/assets/icons/16-arrow-right.svg}}}

{ Read } { Previous page }

\href{/docs/reference/data-loading/xml/}{\pandocbounded{\includesvg[keepaspectratio]{/assets/icons/16-arrow-right.svg}}}

{ XML } { Next page }


\section{Docs LaTeX/typst.app/docs/reference/data-loading/cbor.tex}
\title{typst.app/docs/reference/data-loading/cbor}

\begin{itemize}
\tightlist
\item
  \href{/docs}{\includesvg[width=0.16667in,height=0.16667in]{/assets/icons/16-docs-dark.svg}}
\item
  \includesvg[width=0.16667in,height=0.16667in]{/assets/icons/16-arrow-right.svg}
\item
  \href{/docs/reference/}{Reference}
\item
  \includesvg[width=0.16667in,height=0.16667in]{/assets/icons/16-arrow-right.svg}
\item
  \href{/docs/reference/data-loading/}{Data Loading}
\item
  \includesvg[width=0.16667in,height=0.16667in]{/assets/icons/16-arrow-right.svg}
\item
  \href{/docs/reference/data-loading/cbor/}{CBOR}
\end{itemize}

\section{\texorpdfstring{\texttt{\ cbor\ }}{ cbor }}\label{summary}

Reads structured data from a CBOR file.

The file must contain a valid CBOR serialization. Mappings will be
converted into Typst dictionaries, and sequences will be converted into
Typst arrays. Strings and booleans will be converted into the Typst
equivalents, null-values ( \texttt{\ null\ } ,
\texttt{\ \textasciitilde{}\ } or empty ``) will be converted into
\texttt{\ }{\texttt{\ none\ }}\texttt{\ } , and numbers will be
converted to floats or integers depending on whether they are whole
numbers.

Be aware that integers larger than 2 \textsuperscript{63} -1 will be
converted to floating point numbers, which may result in an
approximative value.

\subsection{\texorpdfstring{{ Parameters
}}{ Parameters }}\label{parameters}

\phantomsection\label{parameters-tooltip}
Parameters are the inputs to a function. They are specified in
parentheses after the function name.

{ cbor } (

{ \href{/docs/reference/foundations/str/}{str} }

) -\textgreater{} { any }

\subsubsection{\texorpdfstring{\texttt{\ path\ }}{ path }}\label{parameters-path}

\href{/docs/reference/foundations/str/}{str}

{Required} {{ Positional }}

\phantomsection\label{parameters-path-positional-tooltip}
Positional parameters are specified in order, without names.

Path to a CBOR file.

For more details, see the \href{/docs/reference/syntax/\#paths}{Paths
section} .

\subsection{\texorpdfstring{{ Definitions
}}{ Definitions }}\label{definitions}

\phantomsection\label{definitions-tooltip}
Functions and types and can have associated definitions. These are
accessed by specifying the function or type, followed by a period, and
then the definition\textquotesingle s name.

\subsubsection{\texorpdfstring{\texttt{\ decode\ }}{ decode }}\label{definitions-decode}

Reads structured data from CBOR bytes.

cbor { . } { decode } (

{ \href{/docs/reference/foundations/bytes/}{bytes} }

) -\textgreater{} { any }

\paragraph{\texorpdfstring{\texttt{\ data\ }}{ data }}\label{definitions-decode-data}

\href{/docs/reference/foundations/bytes/}{bytes}

{Required} {{ Positional }}

\phantomsection\label{definitions-decode-data-positional-tooltip}
Positional parameters are specified in order, without names.

cbor data.

\subsubsection{\texorpdfstring{\texttt{\ encode\ }}{ encode }}\label{definitions-encode}

Encode structured data into CBOR bytes.

cbor { . } { encode } (

{ { any } }

) -\textgreater{} \href{/docs/reference/foundations/bytes/}{bytes}

\paragraph{\texorpdfstring{\texttt{\ value\ }}{ value }}\label{definitions-encode-value}

{ any }

{Required} {{ Positional }}

\phantomsection\label{definitions-encode-value-positional-tooltip}
Positional parameters are specified in order, without names.

Value to be encoded.

\href{/docs/reference/data-loading/}{\pandocbounded{\includesvg[keepaspectratio]{/assets/icons/16-arrow-right.svg}}}

{ Data Loading } { Previous page }

\href{/docs/reference/data-loading/csv/}{\pandocbounded{\includesvg[keepaspectratio]{/assets/icons/16-arrow-right.svg}}}

{ CSV } { Next page }


\section{Docs LaTeX/typst.app/docs/reference/data-loading/yaml.tex}
\title{typst.app/docs/reference/data-loading/yaml}

\begin{itemize}
\tightlist
\item
  \href{/docs}{\includesvg[width=0.16667in,height=0.16667in]{/assets/icons/16-docs-dark.svg}}
\item
  \includesvg[width=0.16667in,height=0.16667in]{/assets/icons/16-arrow-right.svg}
\item
  \href{/docs/reference/}{Reference}
\item
  \includesvg[width=0.16667in,height=0.16667in]{/assets/icons/16-arrow-right.svg}
\item
  \href{/docs/reference/data-loading/}{Data Loading}
\item
  \includesvg[width=0.16667in,height=0.16667in]{/assets/icons/16-arrow-right.svg}
\item
  \href{/docs/reference/data-loading/yaml/}{YAML}
\end{itemize}

\section{\texorpdfstring{\texttt{\ yaml\ }}{ yaml }}\label{summary}

Reads structured data from a YAML file.

The file must contain a valid YAML object or array. YAML mappings will
be converted into Typst dictionaries, and YAML sequences will be
converted into Typst arrays. Strings and booleans will be converted into
the Typst equivalents, null-values ( \texttt{\ null\ } ,
\texttt{\ \textasciitilde{}\ } or empty ``) will be converted into
\texttt{\ }{\texttt{\ none\ }}\texttt{\ } , and numbers will be
converted to floats or integers depending on whether they are whole
numbers. Custom YAML tags are ignored, though the loaded value will
still be present.

Be aware that integers larger than 2 \textsuperscript{63} -1 will be
converted to floating point numbers, which may give an approximative
value.

The YAML files in the example contain objects with authors as keys, each
with a sequence of their own submapping with the keys "title" and
"published"

\subsection{Example}\label{example}

\begin{verbatim}
#let bookshelf(contents) = {
  for (author, works) in contents {
    author
    for work in works [
      - #work.title (#work.published)
    ]
  }
}

#bookshelf(
  yaml("scifi-authors.yaml")
)
\end{verbatim}

\includegraphics[width=5in,height=\textheight,keepaspectratio]{/assets/docs/zhzvOjbNeHnb4ZYJg032GwAAAAAAAAAA.png}

\subsection{\texorpdfstring{{ Parameters
}}{ Parameters }}\label{parameters}

\phantomsection\label{parameters-tooltip}
Parameters are the inputs to a function. They are specified in
parentheses after the function name.

{ yaml } (

{ \href{/docs/reference/foundations/str/}{str} }

) -\textgreater{} { any }

\subsubsection{\texorpdfstring{\texttt{\ path\ }}{ path }}\label{parameters-path}

\href{/docs/reference/foundations/str/}{str}

{Required} {{ Positional }}

\phantomsection\label{parameters-path-positional-tooltip}
Positional parameters are specified in order, without names.

Path to a YAML file.

For more details, see the \href{/docs/reference/syntax/\#paths}{Paths
section} .

\subsection{\texorpdfstring{{ Definitions
}}{ Definitions }}\label{definitions}

\phantomsection\label{definitions-tooltip}
Functions and types and can have associated definitions. These are
accessed by specifying the function or type, followed by a period, and
then the definition\textquotesingle s name.

\subsubsection{\texorpdfstring{\texttt{\ decode\ }}{ decode }}\label{definitions-decode}

Reads structured data from a YAML string/bytes.

yaml { . } { decode } (

{ \href{/docs/reference/foundations/str/}{str}
\href{/docs/reference/foundations/bytes/}{bytes} }

) -\textgreater{} { any }

\paragraph{\texorpdfstring{\texttt{\ data\ }}{ data }}\label{definitions-decode-data}

\href{/docs/reference/foundations/str/}{str} {or}
\href{/docs/reference/foundations/bytes/}{bytes}

{Required} {{ Positional }}

\phantomsection\label{definitions-decode-data-positional-tooltip}
Positional parameters are specified in order, without names.

YAML data.

\subsubsection{\texorpdfstring{\texttt{\ encode\ }}{ encode }}\label{definitions-encode}

Encode structured data into a YAML string.

yaml { . } { encode } (

{ { any } }

) -\textgreater{} \href{/docs/reference/foundations/str/}{str}

\paragraph{\texorpdfstring{\texttt{\ value\ }}{ value }}\label{definitions-encode-value}

{ any }

{Required} {{ Positional }}

\phantomsection\label{definitions-encode-value-positional-tooltip}
Positional parameters are specified in order, without names.

Value to be encoded.

\href{/docs/reference/data-loading/xml/}{\pandocbounded{\includesvg[keepaspectratio]{/assets/icons/16-arrow-right.svg}}}

{ XML } { Previous page }

\href{/docs/guides/}{\pandocbounded{\includesvg[keepaspectratio]{/assets/icons/16-arrow-right.svg}}}

{ Guides } { Next page }


\section{Docs LaTeX/typst.app/docs/reference/data-loading/read.tex}
\title{typst.app/docs/reference/data-loading/read}

\begin{itemize}
\tightlist
\item
  \href{/docs}{\includesvg[width=0.16667in,height=0.16667in]{/assets/icons/16-docs-dark.svg}}
\item
  \includesvg[width=0.16667in,height=0.16667in]{/assets/icons/16-arrow-right.svg}
\item
  \href{/docs/reference/}{Reference}
\item
  \includesvg[width=0.16667in,height=0.16667in]{/assets/icons/16-arrow-right.svg}
\item
  \href{/docs/reference/data-loading/}{Data Loading}
\item
  \includesvg[width=0.16667in,height=0.16667in]{/assets/icons/16-arrow-right.svg}
\item
  \href{/docs/reference/data-loading/read/}{Read}
\end{itemize}

\section{\texorpdfstring{\texttt{\ read\ }}{ read }}\label{summary}

Reads plain text or data from a file.

By default, the file will be read as UTF-8 and returned as a
\href{/docs/reference/foundations/str/}{string} .

If you specify \texttt{\ encoding:\ }{\texttt{\ none\ }}\texttt{\ } ,
this returns raw \href{/docs/reference/foundations/bytes/}{bytes}
instead.

\subsection{Example}\label{example}

\begin{verbatim}
An example for a HTML file: \
#let text = read("example.html")
#raw(text, lang: "html")

Raw bytes:
#read("tiger.jpg", encoding: none)
\end{verbatim}

\includegraphics[width=5in,height=\textheight,keepaspectratio]{/assets/docs/uS5DrZwzU2PIqO_vdJc7GQAAAAAAAAAA.png}

\subsection{\texorpdfstring{{ Parameters
}}{ Parameters }}\label{parameters}

\phantomsection\label{parameters-tooltip}
Parameters are the inputs to a function. They are specified in
parentheses after the function name.

{ read } (

{ \href{/docs/reference/foundations/str/}{str} , } {
\hyperref[parameters-encoding]{encoding :}
\href{/docs/reference/foundations/none/}{none}
\href{/docs/reference/foundations/str/}{str} , }

) -\textgreater{} \href{/docs/reference/foundations/str/}{str}
\href{/docs/reference/foundations/bytes/}{bytes}

\subsubsection{\texorpdfstring{\texttt{\ path\ }}{ path }}\label{parameters-path}

\href{/docs/reference/foundations/str/}{str}

{Required} {{ Positional }}

\phantomsection\label{parameters-path-positional-tooltip}
Positional parameters are specified in order, without names.

Path to a file.

For more details, see the \href{/docs/reference/syntax/\#paths}{Paths
section} .

\subsubsection{\texorpdfstring{\texttt{\ encoding\ }}{ encoding }}\label{parameters-encoding}

\href{/docs/reference/foundations/none/}{none} {or}
\href{/docs/reference/foundations/str/}{str}

The encoding to read the file with.

If set to \texttt{\ }{\texttt{\ none\ }}\texttt{\ } , this function
returns raw bytes.

\begin{longtable}[]{@{}ll@{}}
\toprule\noalign{}
Variant & Details \\
\midrule\noalign{}
\endhead
\bottomrule\noalign{}
\endlastfoot
\texttt{\ "\ utf8\ "\ } & The Unicode UTF-8 encoding. \\
\end{longtable}

Default: \texttt{\ }{\texttt{\ "utf8"\ }}\texttt{\ }

\href{/docs/reference/data-loading/json/}{\pandocbounded{\includesvg[keepaspectratio]{/assets/icons/16-arrow-right.svg}}}

{ JSON } { Previous page }

\href{/docs/reference/data-loading/toml/}{\pandocbounded{\includesvg[keepaspectratio]{/assets/icons/16-arrow-right.svg}}}

{ TOML } { Next page }




\section{C Docs LaTeX/docs/reference/math.tex}
\section{Docs LaTeX/typst.app/docs/reference/math/styles.tex}
\title{typst.app/docs/reference/math/styles}

\begin{itemize}
\tightlist
\item
  \href{/docs}{\includesvg[width=0.16667in,height=0.16667in]{/assets/icons/16-docs-dark.svg}}
\item
  \includesvg[width=0.16667in,height=0.16667in]{/assets/icons/16-arrow-right.svg}
\item
  \href{/docs/reference/}{Reference}
\item
  \includesvg[width=0.16667in,height=0.16667in]{/assets/icons/16-arrow-right.svg}
\item
  \href{/docs/reference/math/}{Math}
\item
  \includesvg[width=0.16667in,height=0.16667in]{/assets/icons/16-arrow-right.svg}
\item
  \href{/docs/reference/math/styles}{Styles}
\end{itemize}

\section{Styles}\label{summary}

Alternate letterforms within formulas.

These functions are distinct from the
\href{/docs/reference/text/text/}{\texttt{\ text\ }} function because
math fonts contain multiple variants of each letter.

\subsection{Functions}\label{functions}

\subsubsection{\texorpdfstring{\texttt{\ upright\ }}{ upright }}\label{functions-upright}

Upright (non-italic) font style in math.

math { . } { upright } (

{ \href{/docs/reference/foundations/content/}{content} }

) -\textgreater{} \href{/docs/reference/foundations/content/}{content}

\begin{verbatim}
$ upright(A) != A $
\end{verbatim}

\includegraphics[width=5in,height=\textheight,keepaspectratio]{/assets/docs/I3XzlEtlEFD5Cw96srS1ngAAAAAAAAAA.png}

\paragraph{\texorpdfstring{\texttt{\ body\ }}{ body }}\label{functions-upright-body}

\href{/docs/reference/foundations/content/}{content}

{Required} {{ Positional }}

\phantomsection\label{functions-upright-body-positional-tooltip}
Positional parameters are specified in order, without names.

The content to style.

\subsubsection{\texorpdfstring{\texttt{\ italic\ }}{ italic }}\label{functions-italic}

Italic font style in math.

For roman letters and greek lowercase letters, this is already the
default.

math { . } { italic } (

{ \href{/docs/reference/foundations/content/}{content} }

) -\textgreater{} \href{/docs/reference/foundations/content/}{content}

\paragraph{\texorpdfstring{\texttt{\ body\ }}{ body }}\label{functions-italic-body}

\href{/docs/reference/foundations/content/}{content}

{Required} {{ Positional }}

\phantomsection\label{functions-italic-body-positional-tooltip}
Positional parameters are specified in order, without names.

The content to style.

\subsubsection{\texorpdfstring{\texttt{\ bold\ }}{ bold }}\label{functions-bold}

Bold font style in math.

math { . } { bold } (

{ \href{/docs/reference/foundations/content/}{content} }

) -\textgreater{} \href{/docs/reference/foundations/content/}{content}

\begin{verbatim}
$ bold(A) := B^+ $
\end{verbatim}

\includegraphics[width=5in,height=\textheight,keepaspectratio]{/assets/docs/8-9k5ChF2PO13_x1ipPkAAAAAAAAAAAA.png}

\paragraph{\texorpdfstring{\texttt{\ body\ }}{ body }}\label{functions-bold-body}

\href{/docs/reference/foundations/content/}{content}

{Required} {{ Positional }}

\phantomsection\label{functions-bold-body-positional-tooltip}
Positional parameters are specified in order, without names.

The content to style.

\href{/docs/reference/math/stretch/}{\pandocbounded{\includesvg[keepaspectratio]{/assets/icons/16-arrow-right.svg}}}

{ Stretch } { Previous page }

\href{/docs/reference/math/op/}{\pandocbounded{\includesvg[keepaspectratio]{/assets/icons/16-arrow-right.svg}}}

{ Text Operator } { Next page }


\section{Docs LaTeX/typst.app/docs/reference/math/mat.tex}
\title{typst.app/docs/reference/math/mat}

\begin{itemize}
\tightlist
\item
  \href{/docs}{\includesvg[width=0.16667in,height=0.16667in]{/assets/icons/16-docs-dark.svg}}
\item
  \includesvg[width=0.16667in,height=0.16667in]{/assets/icons/16-arrow-right.svg}
\item
  \href{/docs/reference/}{Reference}
\item
  \includesvg[width=0.16667in,height=0.16667in]{/assets/icons/16-arrow-right.svg}
\item
  \href{/docs/reference/math/}{Math}
\item
  \includesvg[width=0.16667in,height=0.16667in]{/assets/icons/16-arrow-right.svg}
\item
  \href{/docs/reference/math/mat/}{Matrix}
\end{itemize}

\section{\texorpdfstring{\texttt{\ mat\ } {{ Element
}}}{ mat   Element }}\label{summary}

\phantomsection\label{element-tooltip}
Element functions can be customized with \texttt{\ set\ } and
\texttt{\ show\ } rules.

A matrix.

The elements of a row should be separated by commas, while the rows
themselves should be separated by semicolons. The semicolon syntax
merges preceding arguments separated by commas into an array. You can
also use this special syntax of math function calls to define custom
functions that take 2D data.

Content in cells can be aligned with the
\href{/docs/reference/math/mat/\#parameters-align}{\texttt{\ align\ }}
parameter, or content in cells that are in the same row can be aligned
with the \texttt{\ \&\ } symbol.

\subsection{Example}\label{example}

\begin{verbatim}
$ mat(
  1, 2, ..., 10;
  2, 2, ..., 10;
  dots.v, dots.v, dots.down, dots.v;
  10, 10, ..., 10;
) $
\end{verbatim}

\includegraphics[width=5in,height=\textheight,keepaspectratio]{/assets/docs/yiSilYGQ1wRBpIK3ON349AAAAAAAAAAA.png}

\subsection{\texorpdfstring{{ Parameters
}}{ Parameters }}\label{parameters}

\phantomsection\label{parameters-tooltip}
Parameters are the inputs to a function. They are specified in
parentheses after the function name.

math { . } { mat } (

{ \hyperref[parameters-delim]{delim :}
\href{/docs/reference/foundations/none/}{none}
\href{/docs/reference/foundations/str/}{str}
\href{/docs/reference/foundations/array/}{array}
\href{/docs/reference/symbols/symbol/}{symbol} , } {
\hyperref[parameters-align]{align :}
\href{/docs/reference/layout/alignment/}{alignment} , } {
\hyperref[parameters-augment]{augment :}
\href{/docs/reference/foundations/none/}{none}
\href{/docs/reference/foundations/int/}{int}
\href{/docs/reference/foundations/dictionary/}{dictionary} , } {
\hyperref[parameters-gap]{gap :}
\href{/docs/reference/layout/relative/}{relative} , } {
\hyperref[parameters-row-gap]{row-gap :}
\href{/docs/reference/layout/relative/}{relative} , } {
\hyperref[parameters-column-gap]{column-gap :}
\href{/docs/reference/layout/relative/}{relative} , } {
\hyperref[parameters-rows]{..}
\href{/docs/reference/foundations/array/}{array} , }

) -\textgreater{} \href{/docs/reference/foundations/content/}{content}

\subsubsection{\texorpdfstring{\texttt{\ delim\ }}{ delim }}\label{parameters-delim}

\href{/docs/reference/foundations/none/}{none} {or}
\href{/docs/reference/foundations/str/}{str} {or}
\href{/docs/reference/foundations/array/}{array} {or}
\href{/docs/reference/symbols/symbol/}{symbol}

{{ Settable }}

\phantomsection\label{parameters-delim-settable-tooltip}
Settable parameters can be customized for all following uses of the
function with a \texttt{\ set\ } rule.

The delimiter to use.

Can be a single character specifying the left delimiter, in which case
the right delimiter is inferred. Otherwise, can be an array containing a
left and a right delimiter.

Default:
\texttt{\ }{\texttt{\ (\ }}\texttt{\ }{\texttt{\ "("\ }}\texttt{\ }{\texttt{\ ,\ }}\texttt{\ }{\texttt{\ ")"\ }}\texttt{\ }{\texttt{\ )\ }}\texttt{\ }

\includesvg[width=0.16667in,height=0.16667in]{/assets/icons/16-arrow-right.svg}
View example

\begin{verbatim}
#set math.mat(delim: "[")
$ mat(1, 2; 3, 4) $
\end{verbatim}

\includegraphics[width=5in,height=\textheight,keepaspectratio]{/assets/docs/CpCAX34oIjWq-jvec_NKoQAAAAAAAAAA.png}

\subsubsection{\texorpdfstring{\texttt{\ align\ }}{ align }}\label{parameters-align}

\href{/docs/reference/layout/alignment/}{alignment}

{{ Settable }}

\phantomsection\label{parameters-align-settable-tooltip}
Settable parameters can be customized for all following uses of the
function with a \texttt{\ set\ } rule.

The horizontal alignment that each cell should have.

Default: \texttt{\ center\ }

\includesvg[width=0.16667in,height=0.16667in]{/assets/icons/16-arrow-right.svg}
View example

\begin{verbatim}
#set math.mat(align: right)
$ mat(-1, 1, 1; 1, -1, 1; 1, 1, -1) $
\end{verbatim}

\includegraphics[width=5in,height=\textheight,keepaspectratio]{/assets/docs/X3QXNtgXqEVUQfvJRQOPRwAAAAAAAAAA.png}

\subsubsection{\texorpdfstring{\texttt{\ augment\ }}{ augment }}\label{parameters-augment}

\href{/docs/reference/foundations/none/}{none} {or}
\href{/docs/reference/foundations/int/}{int} {or}
\href{/docs/reference/foundations/dictionary/}{dictionary}

{{ Settable }}

\phantomsection\label{parameters-augment-settable-tooltip}
Settable parameters can be customized for all following uses of the
function with a \texttt{\ set\ } rule.

Draws augmentation lines in a matrix.

\begin{itemize}
\tightlist
\item
  \texttt{\ }{\texttt{\ none\ }}\texttt{\ } : No lines are drawn.
\item
  A single number: A vertical augmentation line is drawn after the
  specified column number. Negative numbers start from the end.
\item
  A dictionary: With a dictionary, multiple augmentation lines can be
  drawn both horizontally and vertically. Additionally, the style of the
  lines can be set. The dictionary can contain the following keys:

  \begin{itemize}
  \tightlist
  \item
    \texttt{\ hline\ } : The offsets at which horizontal lines should be
    drawn. For example, an offset of \texttt{\ 2\ } would result in a
    horizontal line being drawn after the second row of the matrix.
    Accepts either an integer for a single line, or an array of integers
    for multiple lines. Like for a single number, negative numbers start
    from the end.
  \item
    \texttt{\ vline\ } : The offsets at which vertical lines should be
    drawn. For example, an offset of \texttt{\ 2\ } would result in a
    vertical line being drawn after the second column of the matrix.
    Accepts either an integer for a single line, or an array of integers
    for multiple lines. Like for a single number, negative numbers start
    from the end.
  \item
    \texttt{\ stroke\ } : How to
    \href{/docs/reference/visualize/stroke/}{stroke} the line. If set to
    \texttt{\ }{\texttt{\ auto\ }}\texttt{\ } , takes on a thickness of
    0.05em and square line caps.
  \end{itemize}
\end{itemize}

Default: \texttt{\ }{\texttt{\ none\ }}\texttt{\ }

\includesvg[width=0.16667in,height=0.16667in]{/assets/icons/16-arrow-right.svg}
View example

\begin{verbatim}
$ mat(1, 0, 1; 0, 1, 2; augment: #2) $
// Equivalent to:
$ mat(1, 0, 1; 0, 1, 2; augment: #(-1)) $
\end{verbatim}

\includegraphics[width=5in,height=\textheight,keepaspectratio]{/assets/docs/4iip0Z9ppDA0SxnJHJihkQAAAAAAAAAA.png}

\begin{verbatim}
$ mat(0, 0, 0; 1, 1, 1; augment: #(hline: 1, stroke: 2pt + green)) $
\end{verbatim}

\includegraphics[width=5in,height=\textheight,keepaspectratio]{/assets/docs/3PHAJpsviSZ-Rqtb3sBd4AAAAAAAAAAA.png}

\subsubsection{\texorpdfstring{\texttt{\ gap\ }}{ gap }}\label{parameters-gap}

\href{/docs/reference/layout/relative/}{relative}

{{ Settable }}

\phantomsection\label{parameters-gap-settable-tooltip}
Settable parameters can be customized for all following uses of the
function with a \texttt{\ set\ } rule.

The gap between rows and columns.

This is a shorthand to set \texttt{\ row-gap\ } and
\texttt{\ column-gap\ } to the same value.

Default:
\texttt{\ }{\texttt{\ 0\%\ }}\texttt{\ }{\texttt{\ +\ }}\texttt{\ }{\texttt{\ 0pt\ }}\texttt{\ }

\includesvg[width=0.16667in,height=0.16667in]{/assets/icons/16-arrow-right.svg}
View example

\begin{verbatim}
#set math.mat(gap: 1em)
$ mat(1, 2; 3, 4) $
\end{verbatim}

\includegraphics[width=5in,height=\textheight,keepaspectratio]{/assets/docs/kaypJSdE1P1lOWZ-cMMpyAAAAAAAAAAA.png}

\subsubsection{\texorpdfstring{\texttt{\ row-gap\ }}{ row-gap }}\label{parameters-row-gap}

\href{/docs/reference/layout/relative/}{relative}

{{ Settable }}

\phantomsection\label{parameters-row-gap-settable-tooltip}
Settable parameters can be customized for all following uses of the
function with a \texttt{\ set\ } rule.

The gap between rows.

Default:
\texttt{\ }{\texttt{\ 0\%\ }}\texttt{\ }{\texttt{\ +\ }}\texttt{\ }{\texttt{\ 0.2em\ }}\texttt{\ }

\includesvg[width=0.16667in,height=0.16667in]{/assets/icons/16-arrow-right.svg}
View example

\begin{verbatim}
#set math.mat(row-gap: 1em)
$ mat(1, 2; 3, 4) $
\end{verbatim}

\includegraphics[width=5in,height=\textheight,keepaspectratio]{/assets/docs/YNVJ8uCnPvrs8e0YkWIQFgAAAAAAAAAA.png}

\subsubsection{\texorpdfstring{\texttt{\ column-gap\ }}{ column-gap }}\label{parameters-column-gap}

\href{/docs/reference/layout/relative/}{relative}

{{ Settable }}

\phantomsection\label{parameters-column-gap-settable-tooltip}
Settable parameters can be customized for all following uses of the
function with a \texttt{\ set\ } rule.

The gap between columns.

Default:
\texttt{\ }{\texttt{\ 0\%\ }}\texttt{\ }{\texttt{\ +\ }}\texttt{\ }{\texttt{\ 0.5em\ }}\texttt{\ }

\includesvg[width=0.16667in,height=0.16667in]{/assets/icons/16-arrow-right.svg}
View example

\begin{verbatim}
#set math.mat(column-gap: 1em)
$ mat(1, 2; 3, 4) $
\end{verbatim}

\includegraphics[width=5in,height=\textheight,keepaspectratio]{/assets/docs/tKmrTRxYwVIL8x7N4tnRyQAAAAAAAAAA.png}

\subsubsection{\texorpdfstring{\texttt{\ rows\ }}{ rows }}\label{parameters-rows}

\href{/docs/reference/foundations/array/}{array}

{Required} {{ Positional }}

\phantomsection\label{parameters-rows-positional-tooltip}
Positional parameters are specified in order, without names.

{{ Variadic }}

\phantomsection\label{parameters-rows-variadic-tooltip}
Variadic parameters can be specified multiple times.

An array of arrays with the rows of the matrix.

\includesvg[width=0.16667in,height=0.16667in]{/assets/icons/16-arrow-right.svg}
View example

\begin{verbatim}
#let data = ((1, 2, 3), (4, 5, 6))
#let matrix = math.mat(..data)
$ v := matrix $
\end{verbatim}

\includegraphics[width=5in,height=\textheight,keepaspectratio]{/assets/docs/N-7caJ4FsPlOdlVrUrNk9gAAAAAAAAAA.png}

\href{/docs/reference/math/lr/}{\pandocbounded{\includesvg[keepaspectratio]{/assets/icons/16-arrow-right.svg}}}

{ Left/Right } { Previous page }

\href{/docs/reference/math/primes/}{\pandocbounded{\includesvg[keepaspectratio]{/assets/icons/16-arrow-right.svg}}}

{ Primes } { Next page }


\section{Docs LaTeX/typst.app/docs/reference/math/frac.tex}
\title{typst.app/docs/reference/math/frac}

\begin{itemize}
\tightlist
\item
  \href{/docs}{\includesvg[width=0.16667in,height=0.16667in]{/assets/icons/16-docs-dark.svg}}
\item
  \includesvg[width=0.16667in,height=0.16667in]{/assets/icons/16-arrow-right.svg}
\item
  \href{/docs/reference/}{Reference}
\item
  \includesvg[width=0.16667in,height=0.16667in]{/assets/icons/16-arrow-right.svg}
\item
  \href{/docs/reference/math/}{Math}
\item
  \includesvg[width=0.16667in,height=0.16667in]{/assets/icons/16-arrow-right.svg}
\item
  \href{/docs/reference/math/frac/}{Fraction}
\end{itemize}

\section{\texorpdfstring{\texttt{\ frac\ } {{ Element
}}}{ frac   Element }}\label{summary}

\phantomsection\label{element-tooltip}
Element functions can be customized with \texttt{\ set\ } and
\texttt{\ show\ } rules.

A mathematical fraction.

\subsection{Example}\label{example}

\begin{verbatim}
$ 1/2 < (x+1)/2 $
$ ((x+1)) / 2 = frac(a, b) $
\end{verbatim}

\includegraphics[width=5in,height=\textheight,keepaspectratio]{/assets/docs/9RFsr-VSObielPb4Nrr-zQAAAAAAAAAA.png}

\subsection{Syntax}\label{syntax}

This function also has dedicated syntax: Use a slash to turn
neighbouring expressions into a fraction. Multiple atoms can be grouped
into a single expression using round grouping parenthesis. Such
parentheses are removed from the output, but you can nest multiple to
force them.

\subsection{\texorpdfstring{{ Parameters
}}{ Parameters }}\label{parameters}

\phantomsection\label{parameters-tooltip}
Parameters are the inputs to a function. They are specified in
parentheses after the function name.

math { . } { frac } (

{ \href{/docs/reference/foundations/content/}{content} , } {
\href{/docs/reference/foundations/content/}{content} , }

) -\textgreater{} \href{/docs/reference/foundations/content/}{content}

\subsubsection{\texorpdfstring{\texttt{\ num\ }}{ num }}\label{parameters-num}

\href{/docs/reference/foundations/content/}{content}

{Required} {{ Positional }}

\phantomsection\label{parameters-num-positional-tooltip}
Positional parameters are specified in order, without names.

The fraction\textquotesingle s numerator.

\subsubsection{\texorpdfstring{\texttt{\ denom\ }}{ denom }}\label{parameters-denom}

\href{/docs/reference/foundations/content/}{content}

{Required} {{ Positional }}

\phantomsection\label{parameters-denom-positional-tooltip}
Positional parameters are specified in order, without names.

The fraction\textquotesingle s denominator.

\href{/docs/reference/math/equation/}{\pandocbounded{\includesvg[keepaspectratio]{/assets/icons/16-arrow-right.svg}}}

{ Equation } { Previous page }

\href{/docs/reference/math/lr/}{\pandocbounded{\includesvg[keepaspectratio]{/assets/icons/16-arrow-right.svg}}}

{ Left/Right } { Next page }


\section{Docs LaTeX/typst.app/docs/reference/math/cancel.tex}
\title{typst.app/docs/reference/math/cancel}

\begin{itemize}
\tightlist
\item
  \href{/docs}{\includesvg[width=0.16667in,height=0.16667in]{/assets/icons/16-docs-dark.svg}}
\item
  \includesvg[width=0.16667in,height=0.16667in]{/assets/icons/16-arrow-right.svg}
\item
  \href{/docs/reference/}{Reference}
\item
  \includesvg[width=0.16667in,height=0.16667in]{/assets/icons/16-arrow-right.svg}
\item
  \href{/docs/reference/math/}{Math}
\item
  \includesvg[width=0.16667in,height=0.16667in]{/assets/icons/16-arrow-right.svg}
\item
  \href{/docs/reference/math/cancel/}{Cancel}
\end{itemize}

\section{\texorpdfstring{\texttt{\ cancel\ } {{ Element
}}}{ cancel   Element }}\label{summary}

\phantomsection\label{element-tooltip}
Element functions can be customized with \texttt{\ set\ } and
\texttt{\ show\ } rules.

Displays a diagonal line over a part of an equation.

This is commonly used to show the elimination of a term.

\subsection{Example}\label{example}

\begin{verbatim}
Here, we can simplify:
$ (a dot b dot cancel(x)) /
    cancel(x) $
\end{verbatim}

\includegraphics[width=2.91667in,height=\textheight,keepaspectratio]{/assets/docs/fVEZvXjKTk2s3WO88t3K8AAAAAAAAAAA.png}

\subsection{\texorpdfstring{{ Parameters
}}{ Parameters }}\label{parameters}

\phantomsection\label{parameters-tooltip}
Parameters are the inputs to a function. They are specified in
parentheses after the function name.

math { . } { cancel } (

{ \href{/docs/reference/foundations/content/}{content} , } {
\hyperref[parameters-length]{length :}
\href{/docs/reference/layout/relative/}{relative} , } {
\hyperref[parameters-inverted]{inverted :}
\href{/docs/reference/foundations/bool/}{bool} , } {
\hyperref[parameters-cross]{cross :}
\href{/docs/reference/foundations/bool/}{bool} , } {
\hyperref[parameters-angle]{angle :}
\href{/docs/reference/foundations/auto/}{auto}
\href{/docs/reference/layout/angle/}{angle}
\href{/docs/reference/foundations/function/}{function} , } {
\hyperref[parameters-stroke]{stroke :}
\href{/docs/reference/layout/length/}{length}
\href{/docs/reference/visualize/color/}{color}
\href{/docs/reference/visualize/gradient/}{gradient}
\href{/docs/reference/visualize/stroke/}{stroke}
\href{/docs/reference/visualize/pattern/}{pattern}
\href{/docs/reference/foundations/dictionary/}{dictionary} , }

) -\textgreater{} \href{/docs/reference/foundations/content/}{content}

\subsubsection{\texorpdfstring{\texttt{\ body\ }}{ body }}\label{parameters-body}

\href{/docs/reference/foundations/content/}{content}

{Required} {{ Positional }}

\phantomsection\label{parameters-body-positional-tooltip}
Positional parameters are specified in order, without names.

The content over which the line should be placed.

\subsubsection{\texorpdfstring{\texttt{\ length\ }}{ length }}\label{parameters-length}

\href{/docs/reference/layout/relative/}{relative}

{{ Settable }}

\phantomsection\label{parameters-length-settable-tooltip}
Settable parameters can be customized for all following uses of the
function with a \texttt{\ set\ } rule.

The length of the line, relative to the length of the diagonal spanning
the whole element being "cancelled". A value of
\texttt{\ }{\texttt{\ 100\%\ }}\texttt{\ } would then have the line span
precisely the element\textquotesingle s diagonal.

Default:
\texttt{\ }{\texttt{\ 100\%\ }}\texttt{\ }{\texttt{\ +\ }}\texttt{\ }{\texttt{\ 3pt\ }}\texttt{\ }

\includesvg[width=0.16667in,height=0.16667in]{/assets/icons/16-arrow-right.svg}
View example

\begin{verbatim}
$ a + cancel(x, length: #200%)
    - cancel(x, length: #200%) $
\end{verbatim}

\includegraphics[width=2.91667in,height=\textheight,keepaspectratio]{/assets/docs/_RSKVrNDnF5_pAJyRMmcrAAAAAAAAAAA.png}

\subsubsection{\texorpdfstring{\texttt{\ inverted\ }}{ inverted }}\label{parameters-inverted}

\href{/docs/reference/foundations/bool/}{bool}

{{ Settable }}

\phantomsection\label{parameters-inverted-settable-tooltip}
Settable parameters can be customized for all following uses of the
function with a \texttt{\ set\ } rule.

Whether the cancel line should be inverted (flipped along the y-axis).
For the default angle setting, inverted means the cancel line points to
the top left instead of top right.

Default: \texttt{\ }{\texttt{\ false\ }}\texttt{\ }

\includesvg[width=0.16667in,height=0.16667in]{/assets/icons/16-arrow-right.svg}
View example

\begin{verbatim}
$ (a cancel((b + c), inverted: #true)) /
    cancel(b + c, inverted: #true) $
\end{verbatim}

\includegraphics[width=2.91667in,height=\textheight,keepaspectratio]{/assets/docs/GWluRapeZy8kHQiZ5c3XbQAAAAAAAAAA.png}

\subsubsection{\texorpdfstring{\texttt{\ cross\ }}{ cross }}\label{parameters-cross}

\href{/docs/reference/foundations/bool/}{bool}

{{ Settable }}

\phantomsection\label{parameters-cross-settable-tooltip}
Settable parameters can be customized for all following uses of the
function with a \texttt{\ set\ } rule.

Whether two opposing cancel lines should be drawn, forming a cross over
the element. Overrides \texttt{\ inverted\ } .

Default: \texttt{\ }{\texttt{\ false\ }}\texttt{\ }

\includesvg[width=0.16667in,height=0.16667in]{/assets/icons/16-arrow-right.svg}
View example

\begin{verbatim}
$ cancel(Pi, cross: #true) $
\end{verbatim}

\includegraphics[width=2.91667in,height=\textheight,keepaspectratio]{/assets/docs/biIi09LikcDnwaA0WaNwJQAAAAAAAAAA.png}

\subsubsection{\texorpdfstring{\texttt{\ angle\ }}{ angle }}\label{parameters-angle}

\href{/docs/reference/foundations/auto/}{auto} {or}
\href{/docs/reference/layout/angle/}{angle} {or}
\href{/docs/reference/foundations/function/}{function}

{{ Settable }}

\phantomsection\label{parameters-angle-settable-tooltip}
Settable parameters can be customized for all following uses of the
function with a \texttt{\ set\ } rule.

How much to rotate the cancel line.

\begin{itemize}
\tightlist
\item
  If given an angle, the line is rotated by that angle clockwise with
  respect to the y-axis.
\item
  If \texttt{\ }{\texttt{\ auto\ }}\texttt{\ } , the line assumes the
  default angle; that is, along the rising diagonal of the content box.
\item
  If given a function \texttt{\ angle\ =\textgreater{}\ angle\ } , the
  line is rotated, with respect to the y-axis, by the angle returned by
  that function. The function receives the default angle as its input.
\end{itemize}

Default: \texttt{\ }{\texttt{\ auto\ }}\texttt{\ }

\includesvg[width=0.16667in,height=0.16667in]{/assets/icons/16-arrow-right.svg}
View example

\begin{verbatim}
$ cancel(Pi)
  cancel(Pi, angle: #0deg)
  cancel(Pi, angle: #45deg)
  cancel(Pi, angle: #90deg)
  cancel(1/(1+x), angle: #(a => a + 45deg))
  cancel(1/(1+x), angle: #(a => a + 90deg)) $
\end{verbatim}

\includegraphics[width=2.91667in,height=\textheight,keepaspectratio]{/assets/docs/OCEmML9KQSY4Sru0zk3XGwAAAAAAAAAA.png}

\subsubsection{\texorpdfstring{\texttt{\ stroke\ }}{ stroke }}\label{parameters-stroke}

\href{/docs/reference/layout/length/}{length} {or}
\href{/docs/reference/visualize/color/}{color} {or}
\href{/docs/reference/visualize/gradient/}{gradient} {or}
\href{/docs/reference/visualize/stroke/}{stroke} {or}
\href{/docs/reference/visualize/pattern/}{pattern} {or}
\href{/docs/reference/foundations/dictionary/}{dictionary}

{{ Settable }}

\phantomsection\label{parameters-stroke-settable-tooltip}
Settable parameters can be customized for all following uses of the
function with a \texttt{\ set\ } rule.

How to \href{/docs/reference/visualize/stroke/}{stroke} the cancel line.

Default: \texttt{\ }{\texttt{\ 0.5pt\ }}\texttt{\ }

\includesvg[width=0.16667in,height=0.16667in]{/assets/icons/16-arrow-right.svg}
View example

\begin{verbatim}
$ cancel(
  sum x,
  stroke: #(
    paint: red,
    thickness: 1.5pt,
    dash: "dashed",
  ),
) $
\end{verbatim}

\includegraphics[width=2.91667in,height=\textheight,keepaspectratio]{/assets/docs/KCV7eimRh0Q3LxZudj8IDAAAAAAAAAAA.png}

\href{/docs/reference/math/binom/}{\pandocbounded{\includesvg[keepaspectratio]{/assets/icons/16-arrow-right.svg}}}

{ Binomial } { Previous page }

\href{/docs/reference/math/cases/}{\pandocbounded{\includesvg[keepaspectratio]{/assets/icons/16-arrow-right.svg}}}

{ Cases } { Next page }


\section{Docs LaTeX/typst.app/docs/reference/math/primes.tex}
\title{typst.app/docs/reference/math/primes}

\begin{itemize}
\tightlist
\item
  \href{/docs}{\includesvg[width=0.16667in,height=0.16667in]{/assets/icons/16-docs-dark.svg}}
\item
  \includesvg[width=0.16667in,height=0.16667in]{/assets/icons/16-arrow-right.svg}
\item
  \href{/docs/reference/}{Reference}
\item
  \includesvg[width=0.16667in,height=0.16667in]{/assets/icons/16-arrow-right.svg}
\item
  \href{/docs/reference/math/}{Math}
\item
  \includesvg[width=0.16667in,height=0.16667in]{/assets/icons/16-arrow-right.svg}
\item
  \href{/docs/reference/math/primes/}{Primes}
\end{itemize}

\section{\texorpdfstring{\texttt{\ primes\ } {{ Element
}}}{ primes   Element }}\label{summary}

\phantomsection\label{element-tooltip}
Element functions can be customized with \texttt{\ set\ } and
\texttt{\ show\ } rules.

Grouped primes.

\begin{verbatim}
$ a'''_b = a^'''_b $
\end{verbatim}

\includegraphics[width=5in,height=\textheight,keepaspectratio]{/assets/docs/uHgNvego3SyqChIc3iZ9sQAAAAAAAAAA.png}

\subsection{Syntax}\label{syntax}

This function has dedicated syntax: use apostrophes instead of primes.
They will automatically attach to the previous element, moving
superscripts to the next level.

\subsection{\texorpdfstring{{ Parameters
}}{ Parameters }}\label{parameters}

\phantomsection\label{parameters-tooltip}
Parameters are the inputs to a function. They are specified in
parentheses after the function name.

math { . } { primes } (

{ \href{/docs/reference/foundations/int/}{int} }

) -\textgreater{} \href{/docs/reference/foundations/content/}{content}

\subsubsection{\texorpdfstring{\texttt{\ count\ }}{ count }}\label{parameters-count}

\href{/docs/reference/foundations/int/}{int}

{Required} {{ Positional }}

\phantomsection\label{parameters-count-positional-tooltip}
Positional parameters are specified in order, without names.

The number of grouped primes.

\href{/docs/reference/math/mat/}{\pandocbounded{\includesvg[keepaspectratio]{/assets/icons/16-arrow-right.svg}}}

{ Matrix } { Previous page }

\href{/docs/reference/math/roots/}{\pandocbounded{\includesvg[keepaspectratio]{/assets/icons/16-arrow-right.svg}}}

{ Roots } { Next page }


\section{Docs LaTeX/typst.app/docs/reference/math/lr.tex}
\title{typst.app/docs/reference/math/lr}

\begin{itemize}
\tightlist
\item
  \href{/docs}{\includesvg[width=0.16667in,height=0.16667in]{/assets/icons/16-docs-dark.svg}}
\item
  \includesvg[width=0.16667in,height=0.16667in]{/assets/icons/16-arrow-right.svg}
\item
  \href{/docs/reference/}{Reference}
\item
  \includesvg[width=0.16667in,height=0.16667in]{/assets/icons/16-arrow-right.svg}
\item
  \href{/docs/reference/math/}{Math}
\item
  \includesvg[width=0.16667in,height=0.16667in]{/assets/icons/16-arrow-right.svg}
\item
  \href{/docs/reference/math/lr}{Left/Right}
\end{itemize}

\section{Left/Right}\label{summary}

Delimiter matching.

The \texttt{\ lr\ } function allows you to match two delimiters and
scale them with the content they contain. While this also happens
automatically for delimiters that match syntactically, \texttt{\ lr\ }
allows you to match two arbitrary delimiters and control their size
exactly. Apart from the \texttt{\ lr\ } function, Typst provides a few
more functions that create delimiter pairings for absolute, ceiled, and
floored values as well as norms.

\subsection{Example}\label{example}

\begin{verbatim}
$ [a, b/2] $
$ lr(]sum_(x=1)^n], size: #50%) x $
$ abs((x + y) / 2) $
\end{verbatim}

\includegraphics[width=5in,height=\textheight,keepaspectratio]{/assets/docs/ftGuzhHsliOe05r2qFQMwQAAAAAAAAAA.png}

\subsection{Functions}\label{functions}

\subsubsection{\texorpdfstring{\texttt{\ lr\ } {{ Element
}}}{ lr   Element }}\label{functions-lr}

\phantomsection\label{functions-lr-element-tooltip}
Element functions can be customized with \texttt{\ set\ } and
\texttt{\ show\ } rules.

Scales delimiters.

While matched delimiters scale by default, this can be used to scale
unmatched delimiters and to control the delimiter scaling more
precisely.

math { . } { lr } (

{ \hyperref[functions-lr-parameters-size]{size :}
\href{/docs/reference/foundations/auto/}{auto}
\href{/docs/reference/layout/relative/}{relative} , } {
\href{/docs/reference/foundations/content/}{content} , }

) -\textgreater{} \href{/docs/reference/foundations/content/}{content}

\paragraph{\texorpdfstring{\texttt{\ size\ }}{ size }}\label{functions-lr-size}

\href{/docs/reference/foundations/auto/}{auto} {or}
\href{/docs/reference/layout/relative/}{relative}

{{ Settable }}

\phantomsection\label{functions-lr-size-settable-tooltip}
Settable parameters can be customized for all following uses of the
function with a \texttt{\ set\ } rule.

The size of the brackets, relative to the height of the wrapped content.

Default: \texttt{\ }{\texttt{\ auto\ }}\texttt{\ }

\paragraph{\texorpdfstring{\texttt{\ body\ }}{ body }}\label{functions-lr-body}

\href{/docs/reference/foundations/content/}{content}

{Required} {{ Positional }}

\phantomsection\label{functions-lr-body-positional-tooltip}
Positional parameters are specified in order, without names.

The delimited content, including the delimiters.

\subsubsection{\texorpdfstring{\texttt{\ mid\ } {{ Element
}}}{ mid   Element }}\label{functions-mid}

\phantomsection\label{functions-mid-element-tooltip}
Element functions can be customized with \texttt{\ set\ } and
\texttt{\ show\ } rules.

Scales delimiters vertically to the nearest surrounding
\texttt{\ }{\texttt{\ lr\ }}\texttt{\ }{\texttt{\ (\ }}\texttt{\ }{\texttt{\ )\ }}\texttt{\ }
group.

math { . } { mid } (

{ \href{/docs/reference/foundations/content/}{content} }

) -\textgreater{} \href{/docs/reference/foundations/content/}{content}

\begin{verbatim}
$ { x mid(|) sum_(i=1)^n w_i|f_i (x)| < 1 } $
\end{verbatim}

\includegraphics[width=5in,height=\textheight,keepaspectratio]{/assets/docs/op-SkIh83R9BuQA_mC41YAAAAAAAAAAA.png}

\paragraph{\texorpdfstring{\texttt{\ body\ }}{ body }}\label{functions-mid-body}

\href{/docs/reference/foundations/content/}{content}

{Required} {{ Positional }}

\phantomsection\label{functions-mid-body-positional-tooltip}
Positional parameters are specified in order, without names.

The content to be scaled.

\subsubsection{\texorpdfstring{\texttt{\ abs\ }}{ abs }}\label{functions-abs}

Takes the absolute value of an expression.

math { . } { abs } (

{ \hyperref[functions-abs-parameters-size]{size :}
\href{/docs/reference/foundations/auto/}{auto}
\href{/docs/reference/layout/relative/}{relative} , } {
\href{/docs/reference/foundations/content/}{content} , }

) -\textgreater{} \href{/docs/reference/foundations/content/}{content}

\begin{verbatim}
$ abs(x/2) $
\end{verbatim}

\includegraphics[width=5in,height=\textheight,keepaspectratio]{/assets/docs/WJLuRK0YgTAAKX7q_RtueAAAAAAAAAAA.png}

\paragraph{\texorpdfstring{\texttt{\ size\ }}{ size }}\label{functions-abs-size}

\href{/docs/reference/foundations/auto/}{auto} {or}
\href{/docs/reference/layout/relative/}{relative}

The size of the brackets, relative to the height of the wrapped content.

\paragraph{\texorpdfstring{\texttt{\ body\ }}{ body }}\label{functions-abs-body}

\href{/docs/reference/foundations/content/}{content}

{Required} {{ Positional }}

\phantomsection\label{functions-abs-body-positional-tooltip}
Positional parameters are specified in order, without names.

The expression to take the absolute value of.

\subsubsection{\texorpdfstring{\texttt{\ norm\ }}{ norm }}\label{functions-norm}

Takes the norm of an expression.

math { . } { norm } (

{ \hyperref[functions-norm-parameters-size]{size :}
\href{/docs/reference/foundations/auto/}{auto}
\href{/docs/reference/layout/relative/}{relative} , } {
\href{/docs/reference/foundations/content/}{content} , }

) -\textgreater{} \href{/docs/reference/foundations/content/}{content}

\begin{verbatim}
$ norm(x/2) $
\end{verbatim}

\includegraphics[width=5in,height=\textheight,keepaspectratio]{/assets/docs/YC6RjZ5CBxOUd9-0Ud9TzQAAAAAAAAAA.png}

\paragraph{\texorpdfstring{\texttt{\ size\ }}{ size }}\label{functions-norm-size}

\href{/docs/reference/foundations/auto/}{auto} {or}
\href{/docs/reference/layout/relative/}{relative}

The size of the brackets, relative to the height of the wrapped content.

\paragraph{\texorpdfstring{\texttt{\ body\ }}{ body }}\label{functions-norm-body}

\href{/docs/reference/foundations/content/}{content}

{Required} {{ Positional }}

\phantomsection\label{functions-norm-body-positional-tooltip}
Positional parameters are specified in order, without names.

The expression to take the norm of.

\subsubsection{\texorpdfstring{\texttt{\ floor\ }}{ floor }}\label{functions-floor}

Floors an expression.

math { . } { floor } (

{ \hyperref[functions-floor-parameters-size]{size :}
\href{/docs/reference/foundations/auto/}{auto}
\href{/docs/reference/layout/relative/}{relative} , } {
\href{/docs/reference/foundations/content/}{content} , }

) -\textgreater{} \href{/docs/reference/foundations/content/}{content}

\begin{verbatim}
$ floor(x/2) $
\end{verbatim}

\includegraphics[width=5in,height=\textheight,keepaspectratio]{/assets/docs/PDEHlUdVGIVhIYs9pZubiAAAAAAAAAAA.png}

\paragraph{\texorpdfstring{\texttt{\ size\ }}{ size }}\label{functions-floor-size}

\href{/docs/reference/foundations/auto/}{auto} {or}
\href{/docs/reference/layout/relative/}{relative}

The size of the brackets, relative to the height of the wrapped content.

\paragraph{\texorpdfstring{\texttt{\ body\ }}{ body }}\label{functions-floor-body}

\href{/docs/reference/foundations/content/}{content}

{Required} {{ Positional }}

\phantomsection\label{functions-floor-body-positional-tooltip}
Positional parameters are specified in order, without names.

The expression to floor.

\subsubsection{\texorpdfstring{\texttt{\ ceil\ }}{ ceil }}\label{functions-ceil}

Ceils an expression.

math { . } { ceil } (

{ \hyperref[functions-ceil-parameters-size]{size :}
\href{/docs/reference/foundations/auto/}{auto}
\href{/docs/reference/layout/relative/}{relative} , } {
\href{/docs/reference/foundations/content/}{content} , }

) -\textgreater{} \href{/docs/reference/foundations/content/}{content}

\begin{verbatim}
$ ceil(x/2) $
\end{verbatim}

\includegraphics[width=5in,height=\textheight,keepaspectratio]{/assets/docs/8M0cDo0mVWiDmMeZvIBqOAAAAAAAAAAA.png}

\paragraph{\texorpdfstring{\texttt{\ size\ }}{ size }}\label{functions-ceil-size}

\href{/docs/reference/foundations/auto/}{auto} {or}
\href{/docs/reference/layout/relative/}{relative}

The size of the brackets, relative to the height of the wrapped content.

\paragraph{\texorpdfstring{\texttt{\ body\ }}{ body }}\label{functions-ceil-body}

\href{/docs/reference/foundations/content/}{content}

{Required} {{ Positional }}

\phantomsection\label{functions-ceil-body-positional-tooltip}
Positional parameters are specified in order, without names.

The expression to ceil.

\subsubsection{\texorpdfstring{\texttt{\ round\ }}{ round }}\label{functions-round}

Rounds an expression.

math { . } { round } (

{ \hyperref[functions-round-parameters-size]{size :}
\href{/docs/reference/foundations/auto/}{auto}
\href{/docs/reference/layout/relative/}{relative} , } {
\href{/docs/reference/foundations/content/}{content} , }

) -\textgreater{} \href{/docs/reference/foundations/content/}{content}

\begin{verbatim}
$ round(x/2) $
\end{verbatim}

\includegraphics[width=5in,height=\textheight,keepaspectratio]{/assets/docs/tF8zASmAKWpzYdWTOE8zPAAAAAAAAAAA.png}

\paragraph{\texorpdfstring{\texttt{\ size\ }}{ size }}\label{functions-round-size}

\href{/docs/reference/foundations/auto/}{auto} {or}
\href{/docs/reference/layout/relative/}{relative}

The size of the brackets, relative to the height of the wrapped content.

\paragraph{\texorpdfstring{\texttt{\ body\ }}{ body }}\label{functions-round-body}

\href{/docs/reference/foundations/content/}{content}

{Required} {{ Positional }}

\phantomsection\label{functions-round-body-positional-tooltip}
Positional parameters are specified in order, without names.

The expression to round.

\href{/docs/reference/math/frac/}{\pandocbounded{\includesvg[keepaspectratio]{/assets/icons/16-arrow-right.svg}}}

{ Fraction } { Previous page }

\href{/docs/reference/math/mat/}{\pandocbounded{\includesvg[keepaspectratio]{/assets/icons/16-arrow-right.svg}}}

{ Matrix } { Next page }


\section{Docs LaTeX/typst.app/docs/reference/math/class.tex}
\title{typst.app/docs/reference/math/class}

\begin{itemize}
\tightlist
\item
  \href{/docs}{\includesvg[width=0.16667in,height=0.16667in]{/assets/icons/16-docs-dark.svg}}
\item
  \includesvg[width=0.16667in,height=0.16667in]{/assets/icons/16-arrow-right.svg}
\item
  \href{/docs/reference/}{Reference}
\item
  \includesvg[width=0.16667in,height=0.16667in]{/assets/icons/16-arrow-right.svg}
\item
  \href{/docs/reference/math/}{Math}
\item
  \includesvg[width=0.16667in,height=0.16667in]{/assets/icons/16-arrow-right.svg}
\item
  \href{/docs/reference/math/class/}{Class}
\end{itemize}

\section{\texorpdfstring{\texttt{\ class\ } {{ Element
}}}{ class   Element }}\label{summary}

\phantomsection\label{element-tooltip}
Element functions can be customized with \texttt{\ set\ } and
\texttt{\ show\ } rules.

Forced use of a certain math class.

This is useful to treat certain symbols as if they were of a different
class, e.g. to make a symbol behave like a relation. The class of a
symbol defines the way it is laid out, including spacing around it, and
how its scripts are attached by default. Note that the latter can always
be overridden using
\href{/docs/reference/math/attach/\#functions-limits}{\texttt{\ limits\ }}
and
\href{/docs/reference/math/attach/\#functions-scripts}{\texttt{\ scripts\ }}
.

\subsection{Example}\label{example}

\begin{verbatim}
#let loves = math.class(
  "relation",
  sym.suit.heart,
)

$x loves y and y loves 5$
\end{verbatim}

\includegraphics[width=5in,height=\textheight,keepaspectratio]{/assets/docs/4-1urHqzMZfIf7fLTw_1MAAAAAAAAAAA.png}

\subsection{\texorpdfstring{{ Parameters
}}{ Parameters }}\label{parameters}

\phantomsection\label{parameters-tooltip}
Parameters are the inputs to a function. They are specified in
parentheses after the function name.

math { . } { class } (

{ \href{/docs/reference/foundations/str/}{str} , } {
\href{/docs/reference/foundations/content/}{content} , }

) -\textgreater{} \href{/docs/reference/foundations/content/}{content}

\subsubsection{\texorpdfstring{\texttt{\ class\ }}{ class }}\label{parameters-class}

\href{/docs/reference/foundations/str/}{str}

{Required} {{ Positional }}

\phantomsection\label{parameters-class-positional-tooltip}
Positional parameters are specified in order, without names.

The class to apply to the content.

\includesvg[width=0.16667in,height=0.16667in]{/assets/icons/16-arrow-right.svg}
View options

\begin{longtable}[]{@{}ll@{}}
\toprule\noalign{}
Variant & Details \\
\midrule\noalign{}
\endhead
\bottomrule\noalign{}
\endlastfoot
\texttt{\ "\ normal\ "\ } & The default class for non-special things. \\
\texttt{\ "\ punctuation\ "\ } & Punctuation, e.g. a comma. \\
\texttt{\ "\ opening\ "\ } & An opening delimiter, e.g. \texttt{\ (\ }
. \\
\texttt{\ "\ closing\ "\ } & A closing delimiter, e.g. \texttt{\ )\ }
. \\
\texttt{\ "\ fence\ "\ } & A delimiter that is the same on both sides,
e.g. \texttt{\ \textbar{}\ } . \\
\texttt{\ "\ large\ "\ } & A large operator like \texttt{\ sum\ } . \\
\texttt{\ "\ relation\ "\ } & A relation like \texttt{\ =\ } or
\texttt{\ prec\ } . \\
\texttt{\ "\ unary\ "\ } & A unary operator like \texttt{\ not\ } . \\
\texttt{\ "\ binary\ "\ } & A binary operator like \texttt{\ times\ }
. \\
\texttt{\ "\ vary\ "\ } & An operator that can be both unary or binary
like \texttt{\ +\ } . \\
\end{longtable}

\subsubsection{\texorpdfstring{\texttt{\ body\ }}{ body }}\label{parameters-body}

\href{/docs/reference/foundations/content/}{content}

{Required} {{ Positional }}

\phantomsection\label{parameters-body-positional-tooltip}
Positional parameters are specified in order, without names.

The content to which the class is applied.

\href{/docs/reference/math/cases/}{\pandocbounded{\includesvg[keepaspectratio]{/assets/icons/16-arrow-right.svg}}}

{ Cases } { Previous page }

\href{/docs/reference/math/equation/}{\pandocbounded{\includesvg[keepaspectratio]{/assets/icons/16-arrow-right.svg}}}

{ Equation } { Next page }


\section{Docs LaTeX/typst.app/docs/reference/math/binom.tex}
\title{typst.app/docs/reference/math/binom}

\begin{itemize}
\tightlist
\item
  \href{/docs}{\includesvg[width=0.16667in,height=0.16667in]{/assets/icons/16-docs-dark.svg}}
\item
  \includesvg[width=0.16667in,height=0.16667in]{/assets/icons/16-arrow-right.svg}
\item
  \href{/docs/reference/}{Reference}
\item
  \includesvg[width=0.16667in,height=0.16667in]{/assets/icons/16-arrow-right.svg}
\item
  \href{/docs/reference/math/}{Math}
\item
  \includesvg[width=0.16667in,height=0.16667in]{/assets/icons/16-arrow-right.svg}
\item
  \href{/docs/reference/math/binom/}{Binomial}
\end{itemize}

\section{\texorpdfstring{\texttt{\ binom\ } {{ Element
}}}{ binom   Element }}\label{summary}

\phantomsection\label{element-tooltip}
Element functions can be customized with \texttt{\ set\ } and
\texttt{\ show\ } rules.

A binomial expression.

\subsection{Example}\label{example}

\begin{verbatim}
$ binom(n, k) $
$ binom(n, k_1, k_2, k_3, ..., k_m) $
\end{verbatim}

\includegraphics[width=5in,height=\textheight,keepaspectratio]{/assets/docs/x7e1yoGny67cX0IzBxp69AAAAAAAAAAA.png}

\subsection{\texorpdfstring{{ Parameters
}}{ Parameters }}\label{parameters}

\phantomsection\label{parameters-tooltip}
Parameters are the inputs to a function. They are specified in
parentheses after the function name.

math { . } { binom } (

{ \href{/docs/reference/foundations/content/}{content} , } {
\hyperref[parameters-lower]{..}
\href{/docs/reference/foundations/content/}{content} , }

) -\textgreater{} \href{/docs/reference/foundations/content/}{content}

\subsubsection{\texorpdfstring{\texttt{\ upper\ }}{ upper }}\label{parameters-upper}

\href{/docs/reference/foundations/content/}{content}

{Required} {{ Positional }}

\phantomsection\label{parameters-upper-positional-tooltip}
Positional parameters are specified in order, without names.

The binomial\textquotesingle s upper index.

\subsubsection{\texorpdfstring{\texttt{\ lower\ }}{ lower }}\label{parameters-lower}

\href{/docs/reference/foundations/content/}{content}

{Required} {{ Positional }}

\phantomsection\label{parameters-lower-positional-tooltip}
Positional parameters are specified in order, without names.

{{ Variadic }}

\phantomsection\label{parameters-lower-variadic-tooltip}
Variadic parameters can be specified multiple times.

The binomial\textquotesingle s lower index.

\href{/docs/reference/math/attach/}{\pandocbounded{\includesvg[keepaspectratio]{/assets/icons/16-arrow-right.svg}}}

{ Attach } { Previous page }

\href{/docs/reference/math/cancel/}{\pandocbounded{\includesvg[keepaspectratio]{/assets/icons/16-arrow-right.svg}}}

{ Cancel } { Next page }


\section{Docs LaTeX/typst.app/docs/reference/math/cases.tex}
\title{typst.app/docs/reference/math/cases}

\begin{itemize}
\tightlist
\item
  \href{/docs}{\includesvg[width=0.16667in,height=0.16667in]{/assets/icons/16-docs-dark.svg}}
\item
  \includesvg[width=0.16667in,height=0.16667in]{/assets/icons/16-arrow-right.svg}
\item
  \href{/docs/reference/}{Reference}
\item
  \includesvg[width=0.16667in,height=0.16667in]{/assets/icons/16-arrow-right.svg}
\item
  \href{/docs/reference/math/}{Math}
\item
  \includesvg[width=0.16667in,height=0.16667in]{/assets/icons/16-arrow-right.svg}
\item
  \href{/docs/reference/math/cases/}{Cases}
\end{itemize}

\section{\texorpdfstring{\texttt{\ cases\ } {{ Element
}}}{ cases   Element }}\label{summary}

\phantomsection\label{element-tooltip}
Element functions can be customized with \texttt{\ set\ } and
\texttt{\ show\ } rules.

A case distinction.

Content across different branches can be aligned with the
\texttt{\ \&\ } symbol.

\subsection{Example}\label{example}

\begin{verbatim}
$ f(x, y) := cases(
  1 "if" (x dot y)/2 <= 0,
  2 "if" x "is even",
  3 "if" x in NN,
  4 "else",
) $
\end{verbatim}

\includegraphics[width=5in,height=\textheight,keepaspectratio]{/assets/docs/0X1AFPDieBd9jiawKpc0-AAAAAAAAAAA.png}

\subsection{\texorpdfstring{{ Parameters
}}{ Parameters }}\label{parameters}

\phantomsection\label{parameters-tooltip}
Parameters are the inputs to a function. They are specified in
parentheses after the function name.

math { . } { cases } (

{ \hyperref[parameters-delim]{delim :}
\href{/docs/reference/foundations/none/}{none}
\href{/docs/reference/foundations/str/}{str}
\href{/docs/reference/foundations/array/}{array}
\href{/docs/reference/symbols/symbol/}{symbol} , } {
\hyperref[parameters-reverse]{reverse :}
\href{/docs/reference/foundations/bool/}{bool} , } {
\hyperref[parameters-gap]{gap :}
\href{/docs/reference/layout/relative/}{relative} , } {
\hyperref[parameters-children]{..}
\href{/docs/reference/foundations/content/}{content} , }

) -\textgreater{} \href{/docs/reference/foundations/content/}{content}

\subsubsection{\texorpdfstring{\texttt{\ delim\ }}{ delim }}\label{parameters-delim}

\href{/docs/reference/foundations/none/}{none} {or}
\href{/docs/reference/foundations/str/}{str} {or}
\href{/docs/reference/foundations/array/}{array} {or}
\href{/docs/reference/symbols/symbol/}{symbol}

{{ Settable }}

\phantomsection\label{parameters-delim-settable-tooltip}
Settable parameters can be customized for all following uses of the
function with a \texttt{\ set\ } rule.

The delimiter to use.

Can be a single character specifying the left delimiter, in which case
the right delimiter is inferred. Otherwise, can be an array containing a
left and a right delimiter.

Default:
\texttt{\ }{\texttt{\ (\ }}\texttt{\ }{\texttt{\ "\{"\ }}\texttt{\ }{\texttt{\ ,\ }}\texttt{\ }{\texttt{\ "\}"\ }}\texttt{\ }{\texttt{\ )\ }}\texttt{\ }

\includesvg[width=0.16667in,height=0.16667in]{/assets/icons/16-arrow-right.svg}
View example

\begin{verbatim}
#set math.cases(delim: "[")
$ x = cases(1, 2) $
\end{verbatim}

\includegraphics[width=5in,height=\textheight,keepaspectratio]{/assets/docs/bErdOHWWOQLSKtsxtJeY5QAAAAAAAAAA.png}

\subsubsection{\texorpdfstring{\texttt{\ reverse\ }}{ reverse }}\label{parameters-reverse}

\href{/docs/reference/foundations/bool/}{bool}

{{ Settable }}

\phantomsection\label{parameters-reverse-settable-tooltip}
Settable parameters can be customized for all following uses of the
function with a \texttt{\ set\ } rule.

Whether the direction of cases should be reversed.

Default: \texttt{\ }{\texttt{\ false\ }}\texttt{\ }

\includesvg[width=0.16667in,height=0.16667in]{/assets/icons/16-arrow-right.svg}
View example

\begin{verbatim}
#set math.cases(reverse: true)
$ cases(1, 2) = x $
\end{verbatim}

\includegraphics[width=5in,height=\textheight,keepaspectratio]{/assets/docs/z6AQZKJsH9nM95e6Aw0hGgAAAAAAAAAA.png}

\subsubsection{\texorpdfstring{\texttt{\ gap\ }}{ gap }}\label{parameters-gap}

\href{/docs/reference/layout/relative/}{relative}

{{ Settable }}

\phantomsection\label{parameters-gap-settable-tooltip}
Settable parameters can be customized for all following uses of the
function with a \texttt{\ set\ } rule.

The gap between branches.

Default:
\texttt{\ }{\texttt{\ 0\%\ }}\texttt{\ }{\texttt{\ +\ }}\texttt{\ }{\texttt{\ 0.2em\ }}\texttt{\ }

\includesvg[width=0.16667in,height=0.16667in]{/assets/icons/16-arrow-right.svg}
View example

\begin{verbatim}
#set math.cases(gap: 1em)
$ x = cases(1, 2) $
\end{verbatim}

\includegraphics[width=5in,height=\textheight,keepaspectratio]{/assets/docs/-xscfzRH4Dw6Yi5TCvpkVwAAAAAAAAAA.png}

\subsubsection{\texorpdfstring{\texttt{\ children\ }}{ children }}\label{parameters-children}

\href{/docs/reference/foundations/content/}{content}

{Required} {{ Positional }}

\phantomsection\label{parameters-children-positional-tooltip}
Positional parameters are specified in order, without names.

{{ Variadic }}

\phantomsection\label{parameters-children-variadic-tooltip}
Variadic parameters can be specified multiple times.

The branches of the case distinction.

\href{/docs/reference/math/cancel/}{\pandocbounded{\includesvg[keepaspectratio]{/assets/icons/16-arrow-right.svg}}}

{ Cancel } { Previous page }

\href{/docs/reference/math/class/}{\pandocbounded{\includesvg[keepaspectratio]{/assets/icons/16-arrow-right.svg}}}

{ Class } { Next page }


\section{Docs LaTeX/typst.app/docs/reference/math/sizes.tex}
\title{typst.app/docs/reference/math/sizes}

\begin{itemize}
\tightlist
\item
  \href{/docs}{\includesvg[width=0.16667in,height=0.16667in]{/assets/icons/16-docs-dark.svg}}
\item
  \includesvg[width=0.16667in,height=0.16667in]{/assets/icons/16-arrow-right.svg}
\item
  \href{/docs/reference/}{Reference}
\item
  \includesvg[width=0.16667in,height=0.16667in]{/assets/icons/16-arrow-right.svg}
\item
  \href{/docs/reference/math/}{Math}
\item
  \includesvg[width=0.16667in,height=0.16667in]{/assets/icons/16-arrow-right.svg}
\item
  \href{/docs/reference/math/sizes}{Sizes}
\end{itemize}

\section{Sizes}\label{summary}

Forced size styles for expressions within formulas.

These functions allow manual configuration of the size of equation
elements to make them look as in a display/inline equation or as if used
in a root or sub/superscripts.

\subsection{Functions}\label{functions}

\subsubsection{\texorpdfstring{\texttt{\ display\ }}{ display }}\label{functions-display}

Forced display style in math.

This is the normal size for block equations.

math { . } { display } (

{ \href{/docs/reference/foundations/content/}{content} , } {
\hyperref[functions-display-parameters-cramped]{cramped :}
\href{/docs/reference/foundations/bool/}{bool} , }

) -\textgreater{} \href{/docs/reference/foundations/content/}{content}

\begin{verbatim}
$sum_i x_i/2 = display(sum_i x_i/2)$
\end{verbatim}

\includegraphics[width=5in,height=\textheight,keepaspectratio]{/assets/docs/Kw_xKFEpG79sGcim5bh7SgAAAAAAAAAA.png}

\paragraph{\texorpdfstring{\texttt{\ body\ }}{ body }}\label{functions-display-body}

\href{/docs/reference/foundations/content/}{content}

{Required} {{ Positional }}

\phantomsection\label{functions-display-body-positional-tooltip}
Positional parameters are specified in order, without names.

The content to size.

\paragraph{\texorpdfstring{\texttt{\ cramped\ }}{ cramped }}\label{functions-display-cramped}

\href{/docs/reference/foundations/bool/}{bool}

Whether to impose a height restriction for exponents, like regular sub-
and superscripts do.

Default: \texttt{\ }{\texttt{\ false\ }}\texttt{\ }

\subsubsection{\texorpdfstring{\texttt{\ inline\ }}{ inline }}\label{functions-inline}

Forced inline (text) style in math.

This is the normal size for inline equations.

math { . } { inline } (

{ \href{/docs/reference/foundations/content/}{content} , } {
\hyperref[functions-inline-parameters-cramped]{cramped :}
\href{/docs/reference/foundations/bool/}{bool} , }

) -\textgreater{} \href{/docs/reference/foundations/content/}{content}

\begin{verbatim}
$ sum_i x_i/2
    = inline(sum_i x_i/2) $
\end{verbatim}

\includegraphics[width=5in,height=\textheight,keepaspectratio]{/assets/docs/yhhyiAgPa8_SZLz7nNtNqAAAAAAAAAAA.png}

\paragraph{\texorpdfstring{\texttt{\ body\ }}{ body }}\label{functions-inline-body}

\href{/docs/reference/foundations/content/}{content}

{Required} {{ Positional }}

\phantomsection\label{functions-inline-body-positional-tooltip}
Positional parameters are specified in order, without names.

The content to size.

\paragraph{\texorpdfstring{\texttt{\ cramped\ }}{ cramped }}\label{functions-inline-cramped}

\href{/docs/reference/foundations/bool/}{bool}

Whether to impose a height restriction for exponents, like regular sub-
and superscripts do.

Default: \texttt{\ }{\texttt{\ false\ }}\texttt{\ }

\subsubsection{\texorpdfstring{\texttt{\ script\ }}{ script }}\label{functions-script}

Forced script style in math.

This is the smaller size used in powers or sub- or superscripts.

math { . } { script } (

{ \href{/docs/reference/foundations/content/}{content} , } {
\hyperref[functions-script-parameters-cramped]{cramped :}
\href{/docs/reference/foundations/bool/}{bool} , }

) -\textgreater{} \href{/docs/reference/foundations/content/}{content}

\begin{verbatim}
$sum_i x_i/2 = script(sum_i x_i/2)$
\end{verbatim}

\includegraphics[width=5in,height=\textheight,keepaspectratio]{/assets/docs/UAO0CCEy42RrRJk6xg_ljgAAAAAAAAAA.png}

\paragraph{\texorpdfstring{\texttt{\ body\ }}{ body }}\label{functions-script-body}

\href{/docs/reference/foundations/content/}{content}

{Required} {{ Positional }}

\phantomsection\label{functions-script-body-positional-tooltip}
Positional parameters are specified in order, without names.

The content to size.

\paragraph{\texorpdfstring{\texttt{\ cramped\ }}{ cramped }}\label{functions-script-cramped}

\href{/docs/reference/foundations/bool/}{bool}

Whether to impose a height restriction for exponents, like regular sub-
and superscripts do.

Default: \texttt{\ }{\texttt{\ true\ }}\texttt{\ }

\subsubsection{\texorpdfstring{\texttt{\ sscript\ }}{ sscript }}\label{functions-sscript}

Forced second script style in math.

This is the smallest size, used in second-level sub- and superscripts
(script of the script).

math { . } { sscript } (

{ \href{/docs/reference/foundations/content/}{content} , } {
\hyperref[functions-sscript-parameters-cramped]{cramped :}
\href{/docs/reference/foundations/bool/}{bool} , }

) -\textgreater{} \href{/docs/reference/foundations/content/}{content}

\begin{verbatim}
$sum_i x_i/2 = sscript(sum_i x_i/2)$
\end{verbatim}

\includegraphics[width=5in,height=\textheight,keepaspectratio]{/assets/docs/EpmDoJiJrfbN7kA0Km7ujwAAAAAAAAAA.png}

\paragraph{\texorpdfstring{\texttt{\ body\ }}{ body }}\label{functions-sscript-body}

\href{/docs/reference/foundations/content/}{content}

{Required} {{ Positional }}

\phantomsection\label{functions-sscript-body-positional-tooltip}
Positional parameters are specified in order, without names.

The content to size.

\paragraph{\texorpdfstring{\texttt{\ cramped\ }}{ cramped }}\label{functions-sscript-cramped}

\href{/docs/reference/foundations/bool/}{bool}

Whether to impose a height restriction for exponents, like regular sub-
and superscripts do.

Default: \texttt{\ }{\texttt{\ true\ }}\texttt{\ }

\href{/docs/reference/math/roots/}{\pandocbounded{\includesvg[keepaspectratio]{/assets/icons/16-arrow-right.svg}}}

{ Roots } { Previous page }

\href{/docs/reference/math/stretch/}{\pandocbounded{\includesvg[keepaspectratio]{/assets/icons/16-arrow-right.svg}}}

{ Stretch } { Next page }


\section{Docs LaTeX/typst.app/docs/reference/math/vec.tex}
\title{typst.app/docs/reference/math/vec}

\begin{itemize}
\tightlist
\item
  \href{/docs}{\includesvg[width=0.16667in,height=0.16667in]{/assets/icons/16-docs-dark.svg}}
\item
  \includesvg[width=0.16667in,height=0.16667in]{/assets/icons/16-arrow-right.svg}
\item
  \href{/docs/reference/}{Reference}
\item
  \includesvg[width=0.16667in,height=0.16667in]{/assets/icons/16-arrow-right.svg}
\item
  \href{/docs/reference/math/}{Math}
\item
  \includesvg[width=0.16667in,height=0.16667in]{/assets/icons/16-arrow-right.svg}
\item
  \href{/docs/reference/math/vec/}{Vector}
\end{itemize}

\section{\texorpdfstring{\texttt{\ vec\ } {{ Element
}}}{ vec   Element }}\label{summary}

\phantomsection\label{element-tooltip}
Element functions can be customized with \texttt{\ set\ } and
\texttt{\ show\ } rules.

A column vector.

Content in the vector\textquotesingle s elements can be aligned with the
\href{/docs/reference/math/vec/\#parameters-align}{\texttt{\ align\ }}
parameter, or the \texttt{\ \&\ } symbol.

\subsection{Example}\label{example}

\begin{verbatim}
$ vec(a, b, c) dot vec(1, 2, 3)
    = a + 2b + 3c $
\end{verbatim}

\includegraphics[width=5in,height=\textheight,keepaspectratio]{/assets/docs/LnRm06lLMggD8fCQZdA66QAAAAAAAAAA.png}

\subsection{\texorpdfstring{{ Parameters
}}{ Parameters }}\label{parameters}

\phantomsection\label{parameters-tooltip}
Parameters are the inputs to a function. They are specified in
parentheses after the function name.

math { . } { vec } (

{ \hyperref[parameters-delim]{delim :}
\href{/docs/reference/foundations/none/}{none}
\href{/docs/reference/foundations/str/}{str}
\href{/docs/reference/foundations/array/}{array}
\href{/docs/reference/symbols/symbol/}{symbol} , } {
\hyperref[parameters-align]{align :}
\href{/docs/reference/layout/alignment/}{alignment} , } {
\hyperref[parameters-gap]{gap :}
\href{/docs/reference/layout/relative/}{relative} , } {
\hyperref[parameters-children]{..}
\href{/docs/reference/foundations/content/}{content} , }

) -\textgreater{} \href{/docs/reference/foundations/content/}{content}

\subsubsection{\texorpdfstring{\texttt{\ delim\ }}{ delim }}\label{parameters-delim}

\href{/docs/reference/foundations/none/}{none} {or}
\href{/docs/reference/foundations/str/}{str} {or}
\href{/docs/reference/foundations/array/}{array} {or}
\href{/docs/reference/symbols/symbol/}{symbol}

{{ Settable }}

\phantomsection\label{parameters-delim-settable-tooltip}
Settable parameters can be customized for all following uses of the
function with a \texttt{\ set\ } rule.

The delimiter to use.

Can be a single character specifying the left delimiter, in which case
the right delimiter is inferred. Otherwise, can be an array containing a
left and a right delimiter.

Default:
\texttt{\ }{\texttt{\ (\ }}\texttt{\ }{\texttt{\ "("\ }}\texttt{\ }{\texttt{\ ,\ }}\texttt{\ }{\texttt{\ ")"\ }}\texttt{\ }{\texttt{\ )\ }}\texttt{\ }

\includesvg[width=0.16667in,height=0.16667in]{/assets/icons/16-arrow-right.svg}
View example

\begin{verbatim}
#set math.vec(delim: "[")
$ vec(1, 2) $
\end{verbatim}

\includegraphics[width=5in,height=\textheight,keepaspectratio]{/assets/docs/5LFZJ9d25bljXFp6kARHcgAAAAAAAAAA.png}

\subsubsection{\texorpdfstring{\texttt{\ align\ }}{ align }}\label{parameters-align}

\href{/docs/reference/layout/alignment/}{alignment}

{{ Settable }}

\phantomsection\label{parameters-align-settable-tooltip}
Settable parameters can be customized for all following uses of the
function with a \texttt{\ set\ } rule.

The horizontal alignment that each element should have.

Default: \texttt{\ center\ }

\includesvg[width=0.16667in,height=0.16667in]{/assets/icons/16-arrow-right.svg}
View example

\begin{verbatim}
#set math.vec(align: right)
$ vec(-1, 1, -1) $
\end{verbatim}

\includegraphics[width=5in,height=\textheight,keepaspectratio]{/assets/docs/ZtHlp9Y4zEtz53Ydf5unLAAAAAAAAAAA.png}

\subsubsection{\texorpdfstring{\texttt{\ gap\ }}{ gap }}\label{parameters-gap}

\href{/docs/reference/layout/relative/}{relative}

{{ Settable }}

\phantomsection\label{parameters-gap-settable-tooltip}
Settable parameters can be customized for all following uses of the
function with a \texttt{\ set\ } rule.

The gap between elements.

Default:
\texttt{\ }{\texttt{\ 0\%\ }}\texttt{\ }{\texttt{\ +\ }}\texttt{\ }{\texttt{\ 0.2em\ }}\texttt{\ }

\includesvg[width=0.16667in,height=0.16667in]{/assets/icons/16-arrow-right.svg}
View example

\begin{verbatim}
#set math.vec(gap: 1em)
$ vec(1, 2) $
\end{verbatim}

\includegraphics[width=5in,height=\textheight,keepaspectratio]{/assets/docs/uiK2bQUKjIzcO3IGp7RZPwAAAAAAAAAA.png}

\subsubsection{\texorpdfstring{\texttt{\ children\ }}{ children }}\label{parameters-children}

\href{/docs/reference/foundations/content/}{content}

{Required} {{ Positional }}

\phantomsection\label{parameters-children-positional-tooltip}
Positional parameters are specified in order, without names.

{{ Variadic }}

\phantomsection\label{parameters-children-variadic-tooltip}
Variadic parameters can be specified multiple times.

The elements of the vector.

\href{/docs/reference/math/variants/}{\pandocbounded{\includesvg[keepaspectratio]{/assets/icons/16-arrow-right.svg}}}

{ Variants } { Previous page }

\href{/docs/reference/symbols/}{\pandocbounded{\includesvg[keepaspectratio]{/assets/icons/16-arrow-right.svg}}}

{ Symbols } { Next page }


\section{Docs LaTeX/typst.app/docs/reference/math/roots.tex}
\title{typst.app/docs/reference/math/roots}

\begin{itemize}
\tightlist
\item
  \href{/docs}{\includesvg[width=0.16667in,height=0.16667in]{/assets/icons/16-docs-dark.svg}}
\item
  \includesvg[width=0.16667in,height=0.16667in]{/assets/icons/16-arrow-right.svg}
\item
  \href{/docs/reference/}{Reference}
\item
  \includesvg[width=0.16667in,height=0.16667in]{/assets/icons/16-arrow-right.svg}
\item
  \href{/docs/reference/math/}{Math}
\item
  \includesvg[width=0.16667in,height=0.16667in]{/assets/icons/16-arrow-right.svg}
\item
  \href{/docs/reference/math/roots}{Roots}
\end{itemize}

\section{Roots}\label{summary}

Square and non-square roots.

\subsection{Example}\label{example}

\begin{verbatim}
$ sqrt(3 - 2 sqrt(2)) = sqrt(2) - 1 $
$ root(3, x) $
\end{verbatim}

\includegraphics[width=5in,height=\textheight,keepaspectratio]{/assets/docs/YJMQ-3S5QEsCnosYijnvKwAAAAAAAAAA.png}

\subsection{Functions}\label{functions}

\subsubsection{\texorpdfstring{\texttt{\ root\ } {{ Element
}}}{ root   Element }}\label{functions-root}

\phantomsection\label{functions-root-element-tooltip}
Element functions can be customized with \texttt{\ set\ } and
\texttt{\ show\ } rules.

A general root.

math { . } { root } (

{ \hyperref[functions-root-parameters-index]{}
\href{/docs/reference/foundations/none/}{none}
\href{/docs/reference/foundations/content/}{content} , } {
\href{/docs/reference/foundations/content/}{content} , }

) -\textgreater{} \href{/docs/reference/foundations/content/}{content}

\begin{verbatim}
$ root(3, x) $
\end{verbatim}

\includegraphics[width=5in,height=\textheight,keepaspectratio]{/assets/docs/5dcBKGUow3rGrUB1Eg_gjwAAAAAAAAAA.png}

\paragraph{\texorpdfstring{\texttt{\ index\ }}{ index }}\label{functions-root-index}

\href{/docs/reference/foundations/none/}{none} {or}
\href{/docs/reference/foundations/content/}{content}

{{ Positional }}

\phantomsection\label{functions-root-index-positional-tooltip}
Positional parameters are specified in order, without names.

{{ Settable }}

\phantomsection\label{functions-root-index-settable-tooltip}
Settable parameters can be customized for all following uses of the
function with a \texttt{\ set\ } rule.

Which root of the radicand to take.

Default: \texttt{\ }{\texttt{\ none\ }}\texttt{\ }

\paragraph{\texorpdfstring{\texttt{\ radicand\ }}{ radicand }}\label{functions-root-radicand}

\href{/docs/reference/foundations/content/}{content}

{Required} {{ Positional }}

\phantomsection\label{functions-root-radicand-positional-tooltip}
Positional parameters are specified in order, without names.

The expression to take the root of.

\subsubsection{\texorpdfstring{\texttt{\ sqrt\ }}{ sqrt }}\label{functions-sqrt}

A square root.

math { . } { sqrt } (

{ \href{/docs/reference/foundations/content/}{content} }

) -\textgreater{} \href{/docs/reference/foundations/content/}{content}

\begin{verbatim}
$ sqrt(3 - 2 sqrt(2)) = sqrt(2) - 1 $
\end{verbatim}

\includegraphics[width=5in,height=\textheight,keepaspectratio]{/assets/docs/5thyKdLM1Lrfm53ILJWqaQAAAAAAAAAA.png}

\paragraph{\texorpdfstring{\texttt{\ radicand\ }}{ radicand }}\label{functions-sqrt-radicand}

\href{/docs/reference/foundations/content/}{content}

{Required} {{ Positional }}

\phantomsection\label{functions-sqrt-radicand-positional-tooltip}
Positional parameters are specified in order, without names.

The expression to take the square root of.

\href{/docs/reference/math/primes/}{\pandocbounded{\includesvg[keepaspectratio]{/assets/icons/16-arrow-right.svg}}}

{ Primes } { Previous page }

\href{/docs/reference/math/sizes/}{\pandocbounded{\includesvg[keepaspectratio]{/assets/icons/16-arrow-right.svg}}}

{ Sizes } { Next page }


\section{Docs LaTeX/typst.app/docs/reference/math/variants.tex}
\title{typst.app/docs/reference/math/variants}

\begin{itemize}
\tightlist
\item
  \href{/docs}{\includesvg[width=0.16667in,height=0.16667in]{/assets/icons/16-docs-dark.svg}}
\item
  \includesvg[width=0.16667in,height=0.16667in]{/assets/icons/16-arrow-right.svg}
\item
  \href{/docs/reference/}{Reference}
\item
  \includesvg[width=0.16667in,height=0.16667in]{/assets/icons/16-arrow-right.svg}
\item
  \href{/docs/reference/math/}{Math}
\item
  \includesvg[width=0.16667in,height=0.16667in]{/assets/icons/16-arrow-right.svg}
\item
  \href{/docs/reference/math/variants}{Variants}
\end{itemize}

\section{Variants}\label{summary}

Alternate typefaces within formulas.

These functions are distinct from the
\href{/docs/reference/text/text/}{\texttt{\ text\ }} function because
math fonts contain multiple variants of each letter.

\subsection{Functions}\label{functions}

\subsubsection{\texorpdfstring{\texttt{\ serif\ }}{ serif }}\label{functions-serif}

Serif (roman) font style in math.

This is already the default.

math { . } { serif } (

{ \href{/docs/reference/foundations/content/}{content} }

) -\textgreater{} \href{/docs/reference/foundations/content/}{content}

\paragraph{\texorpdfstring{\texttt{\ body\ }}{ body }}\label{functions-serif-body}

\href{/docs/reference/foundations/content/}{content}

{Required} {{ Positional }}

\phantomsection\label{functions-serif-body-positional-tooltip}
Positional parameters are specified in order, without names.

The content to style.

\subsubsection{\texorpdfstring{\texttt{\ sans\ }}{ sans }}\label{functions-sans}

Sans-serif font style in math.

math { . } { sans } (

{ \href{/docs/reference/foundations/content/}{content} }

) -\textgreater{} \href{/docs/reference/foundations/content/}{content}

\begin{verbatim}
$ sans(A B C) $
\end{verbatim}

\includegraphics[width=5in,height=\textheight,keepaspectratio]{/assets/docs/QH7JeXflCs-wCjP8nkBWrQAAAAAAAAAA.png}

\paragraph{\texorpdfstring{\texttt{\ body\ }}{ body }}\label{functions-sans-body}

\href{/docs/reference/foundations/content/}{content}

{Required} {{ Positional }}

\phantomsection\label{functions-sans-body-positional-tooltip}
Positional parameters are specified in order, without names.

The content to style.

\subsubsection{\texorpdfstring{\texttt{\ frak\ }}{ frak }}\label{functions-frak}

Fraktur font style in math.

math { . } { frak } (

{ \href{/docs/reference/foundations/content/}{content} }

) -\textgreater{} \href{/docs/reference/foundations/content/}{content}

\begin{verbatim}
$ frak(P) $
\end{verbatim}

\includegraphics[width=5in,height=\textheight,keepaspectratio]{/assets/docs/e8XkJAdgWXZDqbWs94GeeQAAAAAAAAAA.png}

\paragraph{\texorpdfstring{\texttt{\ body\ }}{ body }}\label{functions-frak-body}

\href{/docs/reference/foundations/content/}{content}

{Required} {{ Positional }}

\phantomsection\label{functions-frak-body-positional-tooltip}
Positional parameters are specified in order, without names.

The content to style.

\subsubsection{\texorpdfstring{\texttt{\ mono\ }}{ mono }}\label{functions-mono}

Monospace font style in math.

math { . } { mono } (

{ \href{/docs/reference/foundations/content/}{content} }

) -\textgreater{} \href{/docs/reference/foundations/content/}{content}

\begin{verbatim}
$ mono(x + y = z) $
\end{verbatim}

\includegraphics[width=5in,height=\textheight,keepaspectratio]{/assets/docs/VdkE7dQPvzJTe7BxDZHcnwAAAAAAAAAA.png}

\paragraph{\texorpdfstring{\texttt{\ body\ }}{ body }}\label{functions-mono-body}

\href{/docs/reference/foundations/content/}{content}

{Required} {{ Positional }}

\phantomsection\label{functions-mono-body-positional-tooltip}
Positional parameters are specified in order, without names.

The content to style.

\subsubsection{\texorpdfstring{\texttt{\ bb\ }}{ bb }}\label{functions-bb}

Blackboard bold (double-struck) font style in math.

For uppercase latin letters, blackboard bold is additionally available
through \href{/docs/reference/symbols/sym/}{symbols} of the form
\texttt{\ NN\ } and \texttt{\ RR\ } .

math { . } { bb } (

{ \href{/docs/reference/foundations/content/}{content} }

) -\textgreater{} \href{/docs/reference/foundations/content/}{content}

\begin{verbatim}
$ bb(b) $
$ bb(N) = NN $
$ f: NN -> RR $
\end{verbatim}

\includegraphics[width=5in,height=\textheight,keepaspectratio]{/assets/docs/7qs4sC1Ha0vO_Ei_dnjHuQAAAAAAAAAA.png}

\paragraph{\texorpdfstring{\texttt{\ body\ }}{ body }}\label{functions-bb-body}

\href{/docs/reference/foundations/content/}{content}

{Required} {{ Positional }}

\phantomsection\label{functions-bb-body-positional-tooltip}
Positional parameters are specified in order, without names.

The content to style.

\subsubsection{\texorpdfstring{\texttt{\ cal\ }}{ cal }}\label{functions-cal}

Calligraphic font style in math.

math { . } { cal } (

{ \href{/docs/reference/foundations/content/}{content} }

) -\textgreater{} \href{/docs/reference/foundations/content/}{content}

\begin{verbatim}
Let $cal(P)$ be the set of ...
\end{verbatim}

\includegraphics[width=5in,height=\textheight,keepaspectratio]{/assets/docs/kqxr3_NhGcBq3QZhkIfIjwAAAAAAAAAA.png}

This corresponds both to LaTeX\textquotesingle s
\texttt{\ \textbackslash{}mathcal\ } and
\texttt{\ \textbackslash{}mathscr\ } as both of these styles share the
same Unicode codepoints. Switching between the styles is thus only
possible if supported by the font via
\href{/docs/reference/text/text/\#parameters-features}{font features} .

For the default math font, the roundhand style is available through the
\texttt{\ ss01\ } feature. Therefore, you could define your own version
of \texttt{\ \textbackslash{}mathscr\ } like this:

\begin{verbatim}
#let scr(it) = text(
  features: ("ss01",),
  box($cal(it)$),
)

We establish $cal(P) != scr(P)$.
\end{verbatim}

\includegraphics[width=5in,height=\textheight,keepaspectratio]{/assets/docs/PLEOQqYY9qiWLwCVv8j_HAAAAAAAAAAA.png}

(The box is not conceptually necessary, but unfortunately currently
needed due to limitations in Typst\textquotesingle s text style handling
in math.)

\paragraph{\texorpdfstring{\texttt{\ body\ }}{ body }}\label{functions-cal-body}

\href{/docs/reference/foundations/content/}{content}

{Required} {{ Positional }}

\phantomsection\label{functions-cal-body-positional-tooltip}
Positional parameters are specified in order, without names.

The content to style.

\href{/docs/reference/math/underover/}{\pandocbounded{\includesvg[keepaspectratio]{/assets/icons/16-arrow-right.svg}}}

{ Under/Over } { Previous page }

\href{/docs/reference/math/vec/}{\pandocbounded{\includesvg[keepaspectratio]{/assets/icons/16-arrow-right.svg}}}

{ Vector } { Next page }


\section{Docs LaTeX/typst.app/docs/reference/math/op.tex}
\title{typst.app/docs/reference/math/op}

\begin{itemize}
\tightlist
\item
  \href{/docs}{\includesvg[width=0.16667in,height=0.16667in]{/assets/icons/16-docs-dark.svg}}
\item
  \includesvg[width=0.16667in,height=0.16667in]{/assets/icons/16-arrow-right.svg}
\item
  \href{/docs/reference/}{Reference}
\item
  \includesvg[width=0.16667in,height=0.16667in]{/assets/icons/16-arrow-right.svg}
\item
  \href{/docs/reference/math/}{Math}
\item
  \includesvg[width=0.16667in,height=0.16667in]{/assets/icons/16-arrow-right.svg}
\item
  \href{/docs/reference/math/op/}{Text Operator}
\end{itemize}

\section{\texorpdfstring{\texttt{\ op\ } {{ Element
}}}{ op   Element }}\label{summary}

\phantomsection\label{element-tooltip}
Element functions can be customized with \texttt{\ set\ } and
\texttt{\ show\ } rules.

A text operator in an equation.

\subsection{Example}\label{example}

\begin{verbatim}
$ tan x = (sin x)/(cos x) $
$ op("custom",
     limits: #true)_(n->oo) n $
\end{verbatim}

\includegraphics[width=5in,height=\textheight,keepaspectratio]{/assets/docs/n9yefElmfwTi92ejfLzhZwAAAAAAAAAA.png}

\subsection{Predefined Operators}\label{predefined}

Typst predefines the operators \texttt{\ arccos\ } , \texttt{\ arcsin\ }
, \texttt{\ arctan\ } , \texttt{\ arg\ } , \texttt{\ cos\ } ,
\texttt{\ cosh\ } , \texttt{\ cot\ } , \texttt{\ coth\ } ,
\texttt{\ csc\ } , \texttt{\ csch\ } , \texttt{\ ctg\ } ,
\texttt{\ deg\ } , \texttt{\ det\ } , \texttt{\ dim\ } ,
\texttt{\ exp\ } , \texttt{\ gcd\ } , \texttt{\ hom\ } , \texttt{\ id\ }
, \texttt{\ im\ } , \texttt{\ inf\ } , \texttt{\ ker\ } ,
\texttt{\ lg\ } , \texttt{\ lim\ } , \texttt{\ liminf\ } ,
\texttt{\ limsup\ } , \texttt{\ ln\ } , \texttt{\ log\ } ,
\texttt{\ max\ } , \texttt{\ min\ } , \texttt{\ mod\ } , \texttt{\ Pr\ }
, \texttt{\ sec\ } , \texttt{\ sech\ } , \texttt{\ sin\ } ,
\texttt{\ sinc\ } , \texttt{\ sinh\ } , \texttt{\ sup\ } ,
\texttt{\ tan\ } , \texttt{\ tanh\ } , \texttt{\ tg\ } and
\texttt{\ tr\ } .

\subsection{\texorpdfstring{{ Parameters
}}{ Parameters }}\label{parameters}

\phantomsection\label{parameters-tooltip}
Parameters are the inputs to a function. They are specified in
parentheses after the function name.

math { . } { op } (

{ \href{/docs/reference/foundations/content/}{content} , } {
\hyperref[parameters-limits]{limits :}
\href{/docs/reference/foundations/bool/}{bool} , }

) -\textgreater{} \href{/docs/reference/foundations/content/}{content}

\subsubsection{\texorpdfstring{\texttt{\ text\ }}{ text }}\label{parameters-text}

\href{/docs/reference/foundations/content/}{content}

{Required} {{ Positional }}

\phantomsection\label{parameters-text-positional-tooltip}
Positional parameters are specified in order, without names.

The operator\textquotesingle s text.

\subsubsection{\texorpdfstring{\texttt{\ limits\ }}{ limits }}\label{parameters-limits}

\href{/docs/reference/foundations/bool/}{bool}

{{ Settable }}

\phantomsection\label{parameters-limits-settable-tooltip}
Settable parameters can be customized for all following uses of the
function with a \texttt{\ set\ } rule.

Whether the operator should show attachments as limits in display mode.

Default: \texttt{\ }{\texttt{\ false\ }}\texttt{\ }

\href{/docs/reference/math/styles/}{\pandocbounded{\includesvg[keepaspectratio]{/assets/icons/16-arrow-right.svg}}}

{ Styles } { Previous page }

\href{/docs/reference/math/underover/}{\pandocbounded{\includesvg[keepaspectratio]{/assets/icons/16-arrow-right.svg}}}

{ Under/Over } { Next page }


\section{Docs LaTeX/typst.app/docs/reference/math/accent.tex}
\title{typst.app/docs/reference/math/accent}

\begin{itemize}
\tightlist
\item
  \href{/docs}{\includesvg[width=0.16667in,height=0.16667in]{/assets/icons/16-docs-dark.svg}}
\item
  \includesvg[width=0.16667in,height=0.16667in]{/assets/icons/16-arrow-right.svg}
\item
  \href{/docs/reference/}{Reference}
\item
  \includesvg[width=0.16667in,height=0.16667in]{/assets/icons/16-arrow-right.svg}
\item
  \href{/docs/reference/math/}{Math}
\item
  \includesvg[width=0.16667in,height=0.16667in]{/assets/icons/16-arrow-right.svg}
\item
  \href{/docs/reference/math/accent/}{Accent}
\end{itemize}

\section{\texorpdfstring{\texttt{\ accent\ } {{ Element
}}}{ accent   Element }}\label{summary}

\phantomsection\label{element-tooltip}
Element functions can be customized with \texttt{\ set\ } and
\texttt{\ show\ } rules.

Attaches an accent to a base.

\subsection{Example}\label{example}

\begin{verbatim}
$grave(a) = accent(a, `)$ \
$arrow(a) = accent(a, arrow)$ \
$tilde(a) = accent(a, \u{0303})$
\end{verbatim}

\includegraphics[width=5in,height=\textheight,keepaspectratio]{/assets/docs/wdLZED2cvtXKAU75vKtAKwAAAAAAAAAA.png}

\subsection{\texorpdfstring{{ Parameters
}}{ Parameters }}\label{parameters}

\phantomsection\label{parameters-tooltip}
Parameters are the inputs to a function. They are specified in
parentheses after the function name.

math { . } { accent } (

{ \href{/docs/reference/foundations/content/}{content} , } {
\href{/docs/reference/foundations/str/}{str}
\href{/docs/reference/foundations/content/}{content} , } {
\hyperref[parameters-size]{size :}
\href{/docs/reference/foundations/auto/}{auto}
\href{/docs/reference/layout/relative/}{relative} , }

) -\textgreater{} \href{/docs/reference/foundations/content/}{content}

\subsubsection{\texorpdfstring{\texttt{\ base\ }}{ base }}\label{parameters-base}

\href{/docs/reference/foundations/content/}{content}

{Required} {{ Positional }}

\phantomsection\label{parameters-base-positional-tooltip}
Positional parameters are specified in order, without names.

The base to which the accent is applied. May consist of multiple
letters.

\includesvg[width=0.16667in,height=0.16667in]{/assets/icons/16-arrow-right.svg}
View example

\begin{verbatim}
$arrow(A B C)$
\end{verbatim}

\includegraphics[width=5in,height=\textheight,keepaspectratio]{/assets/docs/aVpZuZcTglBCvF8kbjxN7AAAAAAAAAAA.png}

\subsubsection{\texorpdfstring{\texttt{\ accent\ }}{ accent }}\label{parameters-accent}

\href{/docs/reference/foundations/str/}{str} {or}
\href{/docs/reference/foundations/content/}{content}

{Required} {{ Positional }}

\phantomsection\label{parameters-accent-positional-tooltip}
Positional parameters are specified in order, without names.

The accent to apply to the base.

Supported accents include:

\begin{longtable}[]{@{}lll@{}}
\toprule\noalign{}
Accent & Name & Codepoint \\
\midrule\noalign{}
\endhead
\bottomrule\noalign{}
\endlastfoot
Grave & \texttt{\ grave\ } & \texttt{\ \textasciigrave{}\ } \\
Acute & \texttt{\ acute\ } & \texttt{\ ´\ } \\
Circumflex & \texttt{\ hat\ } & \texttt{\ \^{}\ } \\
Tilde & \texttt{\ tilde\ } & \texttt{\ \textasciitilde{}\ } \\
Macron & \texttt{\ macron\ } & \texttt{\ ¯\ } \\
Dash & \texttt{\ dash\ } & \texttt{\ ‾\ } \\
Breve & \texttt{\ breve\ } & \texttt{\ ˘\ } \\
Dot & \texttt{\ dot\ } & \texttt{\ .\ } \\
Double dot, Diaeresis & \texttt{\ dot.double\ } , \texttt{\ diaer\ } &
\texttt{\ ¨\ } \\
Triple dot & \texttt{\ dot.triple\ } & \texttt{\ ⃛\ } \\
Quadruple dot & \texttt{\ dot.quad\ } & \texttt{\ ⃜\ } \\
Circle & \texttt{\ circle\ } & \texttt{\ ∘\ } \\
Double acute & \texttt{\ acute.double\ } & \texttt{\ Ë?\ } \\
Caron & \texttt{\ caron\ } & \texttt{\ ˇ\ } \\
Right arrow & \texttt{\ arrow\ } , \texttt{\ -\textgreater{}\ } &
\texttt{\ →\ } \\
Left arrow & \texttt{\ arrow.l\ } , \texttt{\ \textless{}-\ } &
\texttt{\ �\ } \\
Left/Right arrow & \texttt{\ arrow.l.r\ } & \texttt{\ ↔\ } \\
Right harpoon & \texttt{\ harpoon\ } & \texttt{\ ⇀\ } \\
Left harpoon & \texttt{\ harpoon.lt\ } & \texttt{\ ↼\ } \\
\end{longtable}

\subsubsection{\texorpdfstring{\texttt{\ size\ }}{ size }}\label{parameters-size}

\href{/docs/reference/foundations/auto/}{auto} {or}
\href{/docs/reference/layout/relative/}{relative}

{{ Settable }}

\phantomsection\label{parameters-size-settable-tooltip}
Settable parameters can be customized for all following uses of the
function with a \texttt{\ set\ } rule.

The size of the accent, relative to the width of the base.

Default: \texttt{\ }{\texttt{\ auto\ }}\texttt{\ }

\href{/docs/reference/math/}{\pandocbounded{\includesvg[keepaspectratio]{/assets/icons/16-arrow-right.svg}}}

{ Math } { Previous page }

\href{/docs/reference/math/attach/}{\pandocbounded{\includesvg[keepaspectratio]{/assets/icons/16-arrow-right.svg}}}

{ Attach } { Next page }


\section{Docs LaTeX/typst.app/docs/reference/math/stretch.tex}
\title{typst.app/docs/reference/math/stretch}

\begin{itemize}
\tightlist
\item
  \href{/docs}{\includesvg[width=0.16667in,height=0.16667in]{/assets/icons/16-docs-dark.svg}}
\item
  \includesvg[width=0.16667in,height=0.16667in]{/assets/icons/16-arrow-right.svg}
\item
  \href{/docs/reference/}{Reference}
\item
  \includesvg[width=0.16667in,height=0.16667in]{/assets/icons/16-arrow-right.svg}
\item
  \href{/docs/reference/math/}{Math}
\item
  \includesvg[width=0.16667in,height=0.16667in]{/assets/icons/16-arrow-right.svg}
\item
  \href{/docs/reference/math/stretch/}{Stretch}
\end{itemize}

\section{\texorpdfstring{\texttt{\ stretch\ } {{ Element
}}}{ stretch   Element }}\label{summary}

\phantomsection\label{element-tooltip}
Element functions can be customized with \texttt{\ set\ } and
\texttt{\ show\ } rules.

Stretches a glyph.

This function can also be used to automatically stretch the base of an
attachment, so that it fits the top and bottom attachments.

Note that only some glyphs can be stretched, and which ones can depend
on the math font being used. However, most math fonts are the same in
this regard.

\begin{verbatim}
$ H stretch(=)^"define" U + p V $
$ f : X stretch(->>, size: #150%)_"surjective" Y $
$ x stretch(harpoons.ltrb, size: #3em) y
    stretch(\[, size: #150%) z $
\end{verbatim}

\includegraphics[width=5in,height=\textheight,keepaspectratio]{/assets/docs/s6743QhH3-etZ1y_QW-bLAAAAAAAAAAA.png}

\subsection{\texorpdfstring{{ Parameters
}}{ Parameters }}\label{parameters}

\phantomsection\label{parameters-tooltip}
Parameters are the inputs to a function. They are specified in
parentheses after the function name.

math { . } { stretch } (

{ \href{/docs/reference/foundations/content/}{content} , } {
\hyperref[parameters-size]{size :}
\href{/docs/reference/foundations/auto/}{auto}
\href{/docs/reference/layout/relative/}{relative} , }

) -\textgreater{} \href{/docs/reference/foundations/content/}{content}

\subsubsection{\texorpdfstring{\texttt{\ body\ }}{ body }}\label{parameters-body}

\href{/docs/reference/foundations/content/}{content}

{Required} {{ Positional }}

\phantomsection\label{parameters-body-positional-tooltip}
Positional parameters are specified in order, without names.

The glyph to stretch.

\subsubsection{\texorpdfstring{\texttt{\ size\ }}{ size }}\label{parameters-size}

\href{/docs/reference/foundations/auto/}{auto} {or}
\href{/docs/reference/layout/relative/}{relative}

{{ Settable }}

\phantomsection\label{parameters-size-settable-tooltip}
Settable parameters can be customized for all following uses of the
function with a \texttt{\ set\ } rule.

The size to stretch to, relative to the maximum size of the glyph and
its attachments.

Default: \texttt{\ }{\texttt{\ auto\ }}\texttt{\ }

\href{/docs/reference/math/sizes/}{\pandocbounded{\includesvg[keepaspectratio]{/assets/icons/16-arrow-right.svg}}}

{ Sizes } { Previous page }

\href{/docs/reference/math/styles/}{\pandocbounded{\includesvg[keepaspectratio]{/assets/icons/16-arrow-right.svg}}}

{ Styles } { Next page }


\section{Docs LaTeX/typst.app/docs/reference/math/attach.tex}
\title{typst.app/docs/reference/math/attach}

\begin{itemize}
\tightlist
\item
  \href{/docs}{\includesvg[width=0.16667in,height=0.16667in]{/assets/icons/16-docs-dark.svg}}
\item
  \includesvg[width=0.16667in,height=0.16667in]{/assets/icons/16-arrow-right.svg}
\item
  \href{/docs/reference/}{Reference}
\item
  \includesvg[width=0.16667in,height=0.16667in]{/assets/icons/16-arrow-right.svg}
\item
  \href{/docs/reference/math/}{Math}
\item
  \includesvg[width=0.16667in,height=0.16667in]{/assets/icons/16-arrow-right.svg}
\item
  \href{/docs/reference/math/attach}{Attach}
\end{itemize}

\section{Attach}\label{summary}

Subscript, superscripts, and limits.

Attachments can be displayed either as sub/superscripts, or limits.
Typst automatically decides which is more suitable depending on the
base, but you can also control this manually with the
\texttt{\ scripts\ } and \texttt{\ limits\ } functions.

If you want the base to stretch to fit long top and bottom attachments
(for example, an arrow with text above it), use the
\href{/docs/reference/math/stretch/}{\texttt{\ stretch\ }} function.

\subsection{Example}\label{example}

\begin{verbatim}
$ sum_(i=0)^n a_i = 2^(1+i) $
\end{verbatim}

\includegraphics[width=5in,height=\textheight,keepaspectratio]{/assets/docs/QRQ31w2n3rdGvD8KZ-ysUQAAAAAAAAAA.png}

\subsection{Syntax}\label{syntax}

This function also has dedicated syntax for attachments after the base:
Use the underscore ( \texttt{\ \_\ } ) to indicate a subscript i.e.
bottom attachment and the hat ( \texttt{\ \^{}\ } ) to indicate a
superscript i.e. top attachment.

\subsection{Functions}\label{functions}

\subsubsection{\texorpdfstring{\texttt{\ attach\ } {{ Element
}}}{ attach   Element }}\label{functions-attach}

\phantomsection\label{functions-attach-element-tooltip}
Element functions can be customized with \texttt{\ set\ } and
\texttt{\ show\ } rules.

A base with optional attachments.

math { . } { attach } (

{ \href{/docs/reference/foundations/content/}{content} , } {
\hyperref[functions-attach-parameters-t]{t :}
\href{/docs/reference/foundations/none/}{none}
\href{/docs/reference/foundations/content/}{content} , } {
\hyperref[functions-attach-parameters-b]{b :}
\href{/docs/reference/foundations/none/}{none}
\href{/docs/reference/foundations/content/}{content} , } {
\hyperref[functions-attach-parameters-tl]{tl :}
\href{/docs/reference/foundations/none/}{none}
\href{/docs/reference/foundations/content/}{content} , } {
\hyperref[functions-attach-parameters-bl]{bl :}
\href{/docs/reference/foundations/none/}{none}
\href{/docs/reference/foundations/content/}{content} , } {
\hyperref[functions-attach-parameters-tr]{tr :}
\href{/docs/reference/foundations/none/}{none}
\href{/docs/reference/foundations/content/}{content} , } {
\hyperref[functions-attach-parameters-br]{br :}
\href{/docs/reference/foundations/none/}{none}
\href{/docs/reference/foundations/content/}{content} , }

) -\textgreater{} \href{/docs/reference/foundations/content/}{content}

\begin{verbatim}
$ attach(
  Pi, t: alpha, b: beta,
  tl: 1, tr: 2+3, bl: 4+5, br: 6,
) $
\end{verbatim}

\includegraphics[width=5in,height=\textheight,keepaspectratio]{/assets/docs/hP1S-FGMSbXQwhVNN2LoxQAAAAAAAAAA.png}

\paragraph{\texorpdfstring{\texttt{\ base\ }}{ base }}\label{functions-attach-base}

\href{/docs/reference/foundations/content/}{content}

{Required} {{ Positional }}

\phantomsection\label{functions-attach-base-positional-tooltip}
Positional parameters are specified in order, without names.

The base to which things are attached.

\paragraph{\texorpdfstring{\texttt{\ t\ }}{ t }}\label{functions-attach-t}

\href{/docs/reference/foundations/none/}{none} {or}
\href{/docs/reference/foundations/content/}{content}

{{ Settable }}

\phantomsection\label{functions-attach-t-settable-tooltip}
Settable parameters can be customized for all following uses of the
function with a \texttt{\ set\ } rule.

The top attachment, smartly positioned at top-right or above the base.

You can wrap the base in
\texttt{\ }{\texttt{\ limits\ }}\texttt{\ }{\texttt{\ (\ }}\texttt{\ }{\texttt{\ )\ }}\texttt{\ }
or
\texttt{\ }{\texttt{\ scripts\ }}\texttt{\ }{\texttt{\ (\ }}\texttt{\ }{\texttt{\ )\ }}\texttt{\ }
to override the smart positioning.

Default: \texttt{\ }{\texttt{\ none\ }}\texttt{\ }

\paragraph{\texorpdfstring{\texttt{\ b\ }}{ b }}\label{functions-attach-b}

\href{/docs/reference/foundations/none/}{none} {or}
\href{/docs/reference/foundations/content/}{content}

{{ Settable }}

\phantomsection\label{functions-attach-b-settable-tooltip}
Settable parameters can be customized for all following uses of the
function with a \texttt{\ set\ } rule.

The bottom attachment, smartly positioned at the bottom-right or below
the base.

You can wrap the base in
\texttt{\ }{\texttt{\ limits\ }}\texttt{\ }{\texttt{\ (\ }}\texttt{\ }{\texttt{\ )\ }}\texttt{\ }
or
\texttt{\ }{\texttt{\ scripts\ }}\texttt{\ }{\texttt{\ (\ }}\texttt{\ }{\texttt{\ )\ }}\texttt{\ }
to override the smart positioning.

Default: \texttt{\ }{\texttt{\ none\ }}\texttt{\ }

\paragraph{\texorpdfstring{\texttt{\ tl\ }}{ tl }}\label{functions-attach-tl}

\href{/docs/reference/foundations/none/}{none} {or}
\href{/docs/reference/foundations/content/}{content}

{{ Settable }}

\phantomsection\label{functions-attach-tl-settable-tooltip}
Settable parameters can be customized for all following uses of the
function with a \texttt{\ set\ } rule.

The top-left attachment (before the base).

Default: \texttt{\ }{\texttt{\ none\ }}\texttt{\ }

\paragraph{\texorpdfstring{\texttt{\ bl\ }}{ bl }}\label{functions-attach-bl}

\href{/docs/reference/foundations/none/}{none} {or}
\href{/docs/reference/foundations/content/}{content}

{{ Settable }}

\phantomsection\label{functions-attach-bl-settable-tooltip}
Settable parameters can be customized for all following uses of the
function with a \texttt{\ set\ } rule.

The bottom-left attachment (before base).

Default: \texttt{\ }{\texttt{\ none\ }}\texttt{\ }

\paragraph{\texorpdfstring{\texttt{\ tr\ }}{ tr }}\label{functions-attach-tr}

\href{/docs/reference/foundations/none/}{none} {or}
\href{/docs/reference/foundations/content/}{content}

{{ Settable }}

\phantomsection\label{functions-attach-tr-settable-tooltip}
Settable parameters can be customized for all following uses of the
function with a \texttt{\ set\ } rule.

The top-right attachment (after the base).

Default: \texttt{\ }{\texttt{\ none\ }}\texttt{\ }

\paragraph{\texorpdfstring{\texttt{\ br\ }}{ br }}\label{functions-attach-br}

\href{/docs/reference/foundations/none/}{none} {or}
\href{/docs/reference/foundations/content/}{content}

{{ Settable }}

\phantomsection\label{functions-attach-br-settable-tooltip}
Settable parameters can be customized for all following uses of the
function with a \texttt{\ set\ } rule.

The bottom-right attachment (after the base).

Default: \texttt{\ }{\texttt{\ none\ }}\texttt{\ }

\subsubsection{\texorpdfstring{\texttt{\ scripts\ } {{ Element
}}}{ scripts   Element }}\label{functions-scripts}

\phantomsection\label{functions-scripts-element-tooltip}
Element functions can be customized with \texttt{\ set\ } and
\texttt{\ show\ } rules.

Forces a base to display attachments as scripts.

math { . } { scripts } (

{ \href{/docs/reference/foundations/content/}{content} }

) -\textgreater{} \href{/docs/reference/foundations/content/}{content}

\begin{verbatim}
$ scripts(sum)_1^2 != sum_1^2 $
\end{verbatim}

\includegraphics[width=5in,height=\textheight,keepaspectratio]{/assets/docs/yVmcJ82GwTKFuNMU4shSjAAAAAAAAAAA.png}

\paragraph{\texorpdfstring{\texttt{\ body\ }}{ body }}\label{functions-scripts-body}

\href{/docs/reference/foundations/content/}{content}

{Required} {{ Positional }}

\phantomsection\label{functions-scripts-body-positional-tooltip}
Positional parameters are specified in order, without names.

The base to attach the scripts to.

\subsubsection{\texorpdfstring{\texttt{\ limits\ } {{ Element
}}}{ limits   Element }}\label{functions-limits}

\phantomsection\label{functions-limits-element-tooltip}
Element functions can be customized with \texttt{\ set\ } and
\texttt{\ show\ } rules.

Forces a base to display attachments as limits.

math { . } { limits } (

{ \href{/docs/reference/foundations/content/}{content} , } {
\hyperref[functions-limits-parameters-inline]{inline :}
\href{/docs/reference/foundations/bool/}{bool} , }

) -\textgreater{} \href{/docs/reference/foundations/content/}{content}

\begin{verbatim}
$ limits(A)_1^2 != A_1^2 $
\end{verbatim}

\includegraphics[width=5in,height=\textheight,keepaspectratio]{/assets/docs/_7kc3fTt948a-U1_9wdyzgAAAAAAAAAA.png}

\paragraph{\texorpdfstring{\texttt{\ body\ }}{ body }}\label{functions-limits-body}

\href{/docs/reference/foundations/content/}{content}

{Required} {{ Positional }}

\phantomsection\label{functions-limits-body-positional-tooltip}
Positional parameters are specified in order, without names.

The base to attach the limits to.

\paragraph{\texorpdfstring{\texttt{\ inline\ }}{ inline }}\label{functions-limits-inline}

\href{/docs/reference/foundations/bool/}{bool}

{{ Settable }}

\phantomsection\label{functions-limits-inline-settable-tooltip}
Settable parameters can be customized for all following uses of the
function with a \texttt{\ set\ } rule.

Whether to also force limits in inline equations.

When applying limits globally (e.g., through a show rule), it is
typically a good idea to disable this.

Default: \texttt{\ }{\texttt{\ true\ }}\texttt{\ }

\href{/docs/reference/math/accent/}{\pandocbounded{\includesvg[keepaspectratio]{/assets/icons/16-arrow-right.svg}}}

{ Accent } { Previous page }

\href{/docs/reference/math/binom/}{\pandocbounded{\includesvg[keepaspectratio]{/assets/icons/16-arrow-right.svg}}}

{ Binomial } { Next page }


\section{Docs LaTeX/typst.app/docs/reference/math/underover.tex}
\title{typst.app/docs/reference/math/underover}

\begin{itemize}
\tightlist
\item
  \href{/docs}{\includesvg[width=0.16667in,height=0.16667in]{/assets/icons/16-docs-dark.svg}}
\item
  \includesvg[width=0.16667in,height=0.16667in]{/assets/icons/16-arrow-right.svg}
\item
  \href{/docs/reference/}{Reference}
\item
  \includesvg[width=0.16667in,height=0.16667in]{/assets/icons/16-arrow-right.svg}
\item
  \href{/docs/reference/math/}{Math}
\item
  \includesvg[width=0.16667in,height=0.16667in]{/assets/icons/16-arrow-right.svg}
\item
  \href{/docs/reference/math/underover}{Under/Over}
\end{itemize}

\section{Under/Over}\label{summary}

Delimiters above or below parts of an equation.

The braces and brackets further allow you to add an optional annotation
below or above themselves.

\subsection{Functions}\label{functions}

\subsubsection{\texorpdfstring{\texttt{\ underline\ } {{ Element
}}}{ underline   Element }}\label{functions-underline}

\phantomsection\label{functions-underline-element-tooltip}
Element functions can be customized with \texttt{\ set\ } and
\texttt{\ show\ } rules.

A horizontal line under content.

math { . } { underline } (

{ \href{/docs/reference/foundations/content/}{content} }

) -\textgreater{} \href{/docs/reference/foundations/content/}{content}

\begin{verbatim}
$ underline(1 + 2 + ... + 5) $
\end{verbatim}

\includegraphics[width=5in,height=\textheight,keepaspectratio]{/assets/docs/kPv2rkuOYqE5xrS9gynyqwAAAAAAAAAA.png}

\paragraph{\texorpdfstring{\texttt{\ body\ }}{ body }}\label{functions-underline-body}

\href{/docs/reference/foundations/content/}{content}

{Required} {{ Positional }}

\phantomsection\label{functions-underline-body-positional-tooltip}
Positional parameters are specified in order, without names.

The content above the line.

\subsubsection{\texorpdfstring{\texttt{\ overline\ } {{ Element
}}}{ overline   Element }}\label{functions-overline}

\phantomsection\label{functions-overline-element-tooltip}
Element functions can be customized with \texttt{\ set\ } and
\texttt{\ show\ } rules.

A horizontal line over content.

math { . } { overline } (

{ \href{/docs/reference/foundations/content/}{content} }

) -\textgreater{} \href{/docs/reference/foundations/content/}{content}

\begin{verbatim}
$ overline(1 + 2 + ... + 5) $
\end{verbatim}

\includegraphics[width=5in,height=\textheight,keepaspectratio]{/assets/docs/brbtze6pYcbdDHZXqYtX4QAAAAAAAAAA.png}

\paragraph{\texorpdfstring{\texttt{\ body\ }}{ body }}\label{functions-overline-body}

\href{/docs/reference/foundations/content/}{content}

{Required} {{ Positional }}

\phantomsection\label{functions-overline-body-positional-tooltip}
Positional parameters are specified in order, without names.

The content below the line.

\subsubsection{\texorpdfstring{\texttt{\ underbrace\ } {{ Element
}}}{ underbrace   Element }}\label{functions-underbrace}

\phantomsection\label{functions-underbrace-element-tooltip}
Element functions can be customized with \texttt{\ set\ } and
\texttt{\ show\ } rules.

A horizontal brace under content, with an optional annotation below.

math { . } { underbrace } (

{ \href{/docs/reference/foundations/content/}{content} , } {
\hyperref[functions-underbrace-parameters-annotation]{}
\href{/docs/reference/foundations/none/}{none}
\href{/docs/reference/foundations/content/}{content} , }

) -\textgreater{} \href{/docs/reference/foundations/content/}{content}

\begin{verbatim}
$ underbrace(1 + 2 + ... + 5, "numbers") $
\end{verbatim}

\includegraphics[width=5in,height=\textheight,keepaspectratio]{/assets/docs/CQPrguDXpL2KqqF50rooNAAAAAAAAAAA.png}

\paragraph{\texorpdfstring{\texttt{\ body\ }}{ body }}\label{functions-underbrace-body}

\href{/docs/reference/foundations/content/}{content}

{Required} {{ Positional }}

\phantomsection\label{functions-underbrace-body-positional-tooltip}
Positional parameters are specified in order, without names.

The content above the brace.

\paragraph{\texorpdfstring{\texttt{\ annotation\ }}{ annotation }}\label{functions-underbrace-annotation}

\href{/docs/reference/foundations/none/}{none} {or}
\href{/docs/reference/foundations/content/}{content}

{{ Positional }}

\phantomsection\label{functions-underbrace-annotation-positional-tooltip}
Positional parameters are specified in order, without names.

{{ Settable }}

\phantomsection\label{functions-underbrace-annotation-settable-tooltip}
Settable parameters can be customized for all following uses of the
function with a \texttt{\ set\ } rule.

The optional content below the brace.

Default: \texttt{\ }{\texttt{\ none\ }}\texttt{\ }

\subsubsection{\texorpdfstring{\texttt{\ overbrace\ } {{ Element
}}}{ overbrace   Element }}\label{functions-overbrace}

\phantomsection\label{functions-overbrace-element-tooltip}
Element functions can be customized with \texttt{\ set\ } and
\texttt{\ show\ } rules.

A horizontal brace over content, with an optional annotation above.

math { . } { overbrace } (

{ \href{/docs/reference/foundations/content/}{content} , } {
\hyperref[functions-overbrace-parameters-annotation]{}
\href{/docs/reference/foundations/none/}{none}
\href{/docs/reference/foundations/content/}{content} , }

) -\textgreater{} \href{/docs/reference/foundations/content/}{content}

\begin{verbatim}
$ overbrace(1 + 2 + ... + 5, "numbers") $
\end{verbatim}

\includegraphics[width=5in,height=\textheight,keepaspectratio]{/assets/docs/kkBGSVxyTk5_L1k_EG8I3gAAAAAAAAAA.png}

\paragraph{\texorpdfstring{\texttt{\ body\ }}{ body }}\label{functions-overbrace-body}

\href{/docs/reference/foundations/content/}{content}

{Required} {{ Positional }}

\phantomsection\label{functions-overbrace-body-positional-tooltip}
Positional parameters are specified in order, without names.

The content below the brace.

\paragraph{\texorpdfstring{\texttt{\ annotation\ }}{ annotation }}\label{functions-overbrace-annotation}

\href{/docs/reference/foundations/none/}{none} {or}
\href{/docs/reference/foundations/content/}{content}

{{ Positional }}

\phantomsection\label{functions-overbrace-annotation-positional-tooltip}
Positional parameters are specified in order, without names.

{{ Settable }}

\phantomsection\label{functions-overbrace-annotation-settable-tooltip}
Settable parameters can be customized for all following uses of the
function with a \texttt{\ set\ } rule.

The optional content above the brace.

Default: \texttt{\ }{\texttt{\ none\ }}\texttt{\ }

\subsubsection{\texorpdfstring{\texttt{\ underbracket\ } {{ Element
}}}{ underbracket   Element }}\label{functions-underbracket}

\phantomsection\label{functions-underbracket-element-tooltip}
Element functions can be customized with \texttt{\ set\ } and
\texttt{\ show\ } rules.

A horizontal bracket under content, with an optional annotation below.

math { . } { underbracket } (

{ \href{/docs/reference/foundations/content/}{content} , } {
\hyperref[functions-underbracket-parameters-annotation]{}
\href{/docs/reference/foundations/none/}{none}
\href{/docs/reference/foundations/content/}{content} , }

) -\textgreater{} \href{/docs/reference/foundations/content/}{content}

\begin{verbatim}
$ underbracket(1 + 2 + ... + 5, "numbers") $
\end{verbatim}

\includegraphics[width=5in,height=\textheight,keepaspectratio]{/assets/docs/gOJp15FKm4cOEOHbW4H-OwAAAAAAAAAA.png}

\paragraph{\texorpdfstring{\texttt{\ body\ }}{ body }}\label{functions-underbracket-body}

\href{/docs/reference/foundations/content/}{content}

{Required} {{ Positional }}

\phantomsection\label{functions-underbracket-body-positional-tooltip}
Positional parameters are specified in order, without names.

The content above the bracket.

\paragraph{\texorpdfstring{\texttt{\ annotation\ }}{ annotation }}\label{functions-underbracket-annotation}

\href{/docs/reference/foundations/none/}{none} {or}
\href{/docs/reference/foundations/content/}{content}

{{ Positional }}

\phantomsection\label{functions-underbracket-annotation-positional-tooltip}
Positional parameters are specified in order, without names.

{{ Settable }}

\phantomsection\label{functions-underbracket-annotation-settable-tooltip}
Settable parameters can be customized for all following uses of the
function with a \texttt{\ set\ } rule.

The optional content below the bracket.

Default: \texttt{\ }{\texttt{\ none\ }}\texttt{\ }

\subsubsection{\texorpdfstring{\texttt{\ overbracket\ } {{ Element
}}}{ overbracket   Element }}\label{functions-overbracket}

\phantomsection\label{functions-overbracket-element-tooltip}
Element functions can be customized with \texttt{\ set\ } and
\texttt{\ show\ } rules.

A horizontal bracket over content, with an optional annotation above.

math { . } { overbracket } (

{ \href{/docs/reference/foundations/content/}{content} , } {
\hyperref[functions-overbracket-parameters-annotation]{}
\href{/docs/reference/foundations/none/}{none}
\href{/docs/reference/foundations/content/}{content} , }

) -\textgreater{} \href{/docs/reference/foundations/content/}{content}

\begin{verbatim}
$ overbracket(1 + 2 + ... + 5, "numbers") $
\end{verbatim}

\includegraphics[width=5in,height=\textheight,keepaspectratio]{/assets/docs/1FDacJmC0p-s0HOdRor7WgAAAAAAAAAA.png}

\paragraph{\texorpdfstring{\texttt{\ body\ }}{ body }}\label{functions-overbracket-body}

\href{/docs/reference/foundations/content/}{content}

{Required} {{ Positional }}

\phantomsection\label{functions-overbracket-body-positional-tooltip}
Positional parameters are specified in order, without names.

The content below the bracket.

\paragraph{\texorpdfstring{\texttt{\ annotation\ }}{ annotation }}\label{functions-overbracket-annotation}

\href{/docs/reference/foundations/none/}{none} {or}
\href{/docs/reference/foundations/content/}{content}

{{ Positional }}

\phantomsection\label{functions-overbracket-annotation-positional-tooltip}
Positional parameters are specified in order, without names.

{{ Settable }}

\phantomsection\label{functions-overbracket-annotation-settable-tooltip}
Settable parameters can be customized for all following uses of the
function with a \texttt{\ set\ } rule.

The optional content above the bracket.

Default: \texttt{\ }{\texttt{\ none\ }}\texttt{\ }

\subsubsection{\texorpdfstring{\texttt{\ underparen\ } {{ Element
}}}{ underparen   Element }}\label{functions-underparen}

\phantomsection\label{functions-underparen-element-tooltip}
Element functions can be customized with \texttt{\ set\ } and
\texttt{\ show\ } rules.

A horizontal parenthesis under content, with an optional annotation
below.

math { . } { underparen } (

{ \href{/docs/reference/foundations/content/}{content} , } {
\hyperref[functions-underparen-parameters-annotation]{}
\href{/docs/reference/foundations/none/}{none}
\href{/docs/reference/foundations/content/}{content} , }

) -\textgreater{} \href{/docs/reference/foundations/content/}{content}

\begin{verbatim}
$ underparen(1 + 2 + ... + 5, "numbers") $
\end{verbatim}

\includegraphics[width=5in,height=\textheight,keepaspectratio]{/assets/docs/L9b8yvtULgB5qqliYqLlbAAAAAAAAAAA.png}

\paragraph{\texorpdfstring{\texttt{\ body\ }}{ body }}\label{functions-underparen-body}

\href{/docs/reference/foundations/content/}{content}

{Required} {{ Positional }}

\phantomsection\label{functions-underparen-body-positional-tooltip}
Positional parameters are specified in order, without names.

The content above the parenthesis.

\paragraph{\texorpdfstring{\texttt{\ annotation\ }}{ annotation }}\label{functions-underparen-annotation}

\href{/docs/reference/foundations/none/}{none} {or}
\href{/docs/reference/foundations/content/}{content}

{{ Positional }}

\phantomsection\label{functions-underparen-annotation-positional-tooltip}
Positional parameters are specified in order, without names.

{{ Settable }}

\phantomsection\label{functions-underparen-annotation-settable-tooltip}
Settable parameters can be customized for all following uses of the
function with a \texttt{\ set\ } rule.

The optional content below the parenthesis.

Default: \texttt{\ }{\texttt{\ none\ }}\texttt{\ }

\subsubsection{\texorpdfstring{\texttt{\ overparen\ } {{ Element
}}}{ overparen   Element }}\label{functions-overparen}

\phantomsection\label{functions-overparen-element-tooltip}
Element functions can be customized with \texttt{\ set\ } and
\texttt{\ show\ } rules.

A horizontal parenthesis over content, with an optional annotation
above.

math { . } { overparen } (

{ \href{/docs/reference/foundations/content/}{content} , } {
\hyperref[functions-overparen-parameters-annotation]{}
\href{/docs/reference/foundations/none/}{none}
\href{/docs/reference/foundations/content/}{content} , }

) -\textgreater{} \href{/docs/reference/foundations/content/}{content}

\begin{verbatim}
$ overparen(1 + 2 + ... + 5, "numbers") $
\end{verbatim}

\includegraphics[width=5in,height=\textheight,keepaspectratio]{/assets/docs/0O_PdeP9aD4IdbiAFPLHcwAAAAAAAAAA.png}

\paragraph{\texorpdfstring{\texttt{\ body\ }}{ body }}\label{functions-overparen-body}

\href{/docs/reference/foundations/content/}{content}

{Required} {{ Positional }}

\phantomsection\label{functions-overparen-body-positional-tooltip}
Positional parameters are specified in order, without names.

The content below the parenthesis.

\paragraph{\texorpdfstring{\texttt{\ annotation\ }}{ annotation }}\label{functions-overparen-annotation}

\href{/docs/reference/foundations/none/}{none} {or}
\href{/docs/reference/foundations/content/}{content}

{{ Positional }}

\phantomsection\label{functions-overparen-annotation-positional-tooltip}
Positional parameters are specified in order, without names.

{{ Settable }}

\phantomsection\label{functions-overparen-annotation-settable-tooltip}
Settable parameters can be customized for all following uses of the
function with a \texttt{\ set\ } rule.

The optional content above the parenthesis.

Default: \texttt{\ }{\texttt{\ none\ }}\texttt{\ }

\subsubsection{\texorpdfstring{\texttt{\ undershell\ } {{ Element
}}}{ undershell   Element }}\label{functions-undershell}

\phantomsection\label{functions-undershell-element-tooltip}
Element functions can be customized with \texttt{\ set\ } and
\texttt{\ show\ } rules.

A horizontal tortoise shell bracket under content, with an optional
annotation below.

math { . } { undershell } (

{ \href{/docs/reference/foundations/content/}{content} , } {
\hyperref[functions-undershell-parameters-annotation]{}
\href{/docs/reference/foundations/none/}{none}
\href{/docs/reference/foundations/content/}{content} , }

) -\textgreater{} \href{/docs/reference/foundations/content/}{content}

\begin{verbatim}
$ undershell(1 + 2 + ... + 5, "numbers") $
\end{verbatim}

\includegraphics[width=5in,height=\textheight,keepaspectratio]{/assets/docs/qJR4zaGYtEbCSgwj0kBzVgAAAAAAAAAA.png}

\paragraph{\texorpdfstring{\texttt{\ body\ }}{ body }}\label{functions-undershell-body}

\href{/docs/reference/foundations/content/}{content}

{Required} {{ Positional }}

\phantomsection\label{functions-undershell-body-positional-tooltip}
Positional parameters are specified in order, without names.

The content above the tortoise shell bracket.

\paragraph{\texorpdfstring{\texttt{\ annotation\ }}{ annotation }}\label{functions-undershell-annotation}

\href{/docs/reference/foundations/none/}{none} {or}
\href{/docs/reference/foundations/content/}{content}

{{ Positional }}

\phantomsection\label{functions-undershell-annotation-positional-tooltip}
Positional parameters are specified in order, without names.

{{ Settable }}

\phantomsection\label{functions-undershell-annotation-settable-tooltip}
Settable parameters can be customized for all following uses of the
function with a \texttt{\ set\ } rule.

The optional content below the tortoise shell bracket.

Default: \texttt{\ }{\texttt{\ none\ }}\texttt{\ }

\subsubsection{\texorpdfstring{\texttt{\ overshell\ } {{ Element
}}}{ overshell   Element }}\label{functions-overshell}

\phantomsection\label{functions-overshell-element-tooltip}
Element functions can be customized with \texttt{\ set\ } and
\texttt{\ show\ } rules.

A horizontal tortoise shell bracket over content, with an optional
annotation above.

math { . } { overshell } (

{ \href{/docs/reference/foundations/content/}{content} , } {
\hyperref[functions-overshell-parameters-annotation]{}
\href{/docs/reference/foundations/none/}{none}
\href{/docs/reference/foundations/content/}{content} , }

) -\textgreater{} \href{/docs/reference/foundations/content/}{content}

\begin{verbatim}
$ overshell(1 + 2 + ... + 5, "numbers") $
\end{verbatim}

\includegraphics[width=5in,height=\textheight,keepaspectratio]{/assets/docs/vPA0v0E_JXwsC1BpaClcEgAAAAAAAAAA.png}

\paragraph{\texorpdfstring{\texttt{\ body\ }}{ body }}\label{functions-overshell-body}

\href{/docs/reference/foundations/content/}{content}

{Required} {{ Positional }}

\phantomsection\label{functions-overshell-body-positional-tooltip}
Positional parameters are specified in order, without names.

The content below the tortoise shell bracket.

\paragraph{\texorpdfstring{\texttt{\ annotation\ }}{ annotation }}\label{functions-overshell-annotation}

\href{/docs/reference/foundations/none/}{none} {or}
\href{/docs/reference/foundations/content/}{content}

{{ Positional }}

\phantomsection\label{functions-overshell-annotation-positional-tooltip}
Positional parameters are specified in order, without names.

{{ Settable }}

\phantomsection\label{functions-overshell-annotation-settable-tooltip}
Settable parameters can be customized for all following uses of the
function with a \texttt{\ set\ } rule.

The optional content above the tortoise shell bracket.

Default: \texttt{\ }{\texttt{\ none\ }}\texttt{\ }

\href{/docs/reference/math/op/}{\pandocbounded{\includesvg[keepaspectratio]{/assets/icons/16-arrow-right.svg}}}

{ Text Operator } { Previous page }

\href{/docs/reference/math/variants/}{\pandocbounded{\includesvg[keepaspectratio]{/assets/icons/16-arrow-right.svg}}}

{ Variants } { Next page }


\section{Docs LaTeX/typst.app/docs/reference/math/equation.tex}
\title{typst.app/docs/reference/math/equation}

\begin{itemize}
\tightlist
\item
  \href{/docs}{\includesvg[width=0.16667in,height=0.16667in]{/assets/icons/16-docs-dark.svg}}
\item
  \includesvg[width=0.16667in,height=0.16667in]{/assets/icons/16-arrow-right.svg}
\item
  \href{/docs/reference/}{Reference}
\item
  \includesvg[width=0.16667in,height=0.16667in]{/assets/icons/16-arrow-right.svg}
\item
  \href{/docs/reference/math/}{Math}
\item
  \includesvg[width=0.16667in,height=0.16667in]{/assets/icons/16-arrow-right.svg}
\item
  \href{/docs/reference/math/equation/}{Equation}
\end{itemize}

\section{\texorpdfstring{\texttt{\ equation\ } {{ Element
}}}{ equation   Element }}\label{summary}

\phantomsection\label{element-tooltip}
Element functions can be customized with \texttt{\ set\ } and
\texttt{\ show\ } rules.

A mathematical equation.

Can be displayed inline with text or as a separate block.

\subsection{Example}\label{example}

\begin{verbatim}
#set text(font: "New Computer Modern")

Let $a$, $b$, and $c$ be the side
lengths of right-angled triangle.
Then, we know that:
$ a^2 + b^2 = c^2 $

Prove by induction:
$ sum_(k=1)^n k = (n(n+1)) / 2 $
\end{verbatim}

\includegraphics[width=5in,height=\textheight,keepaspectratio]{/assets/docs/JtxOgQArvspfmmStl8-3_gAAAAAAAAAA.png}

By default, block-level equations will not break across pages. This can
be changed through
\texttt{\ }{\texttt{\ show\ }}\texttt{\ math\ }{\texttt{\ .\ }}\texttt{\ }{\texttt{\ equation\ }}\texttt{\ }{\texttt{\ :\ }}\texttt{\ }{\texttt{\ set\ }}\texttt{\ }{\texttt{\ block\ }}\texttt{\ }{\texttt{\ (\ }}\texttt{\ breakable\ }{\texttt{\ :\ }}\texttt{\ }{\texttt{\ true\ }}\texttt{\ }{\texttt{\ )\ }}\texttt{\ }
.

\subsection{Syntax}\label{syntax}

This function also has dedicated syntax: Write mathematical markup
within dollar signs to create an equation. Starting and ending the
equation with at least one space lifts it into a separate block that is
centered horizontally. For more details about math syntax, see the
\href{/docs/reference/math/}{main math page} .

\subsection{\texorpdfstring{{ Parameters
}}{ Parameters }}\label{parameters}

\phantomsection\label{parameters-tooltip}
Parameters are the inputs to a function. They are specified in
parentheses after the function name.

math { . } { equation } (

{ \hyperref[parameters-block]{block :}
\href{/docs/reference/foundations/bool/}{bool} , } {
\hyperref[parameters-numbering]{numbering :}
\href{/docs/reference/foundations/none/}{none}
\href{/docs/reference/foundations/str/}{str}
\href{/docs/reference/foundations/function/}{function} , } {
\hyperref[parameters-number-align]{number-align :}
\href{/docs/reference/layout/alignment/}{alignment} , } {
\hyperref[parameters-supplement]{supplement :}
\href{/docs/reference/foundations/none/}{none}
\href{/docs/reference/foundations/auto/}{auto}
\href{/docs/reference/foundations/content/}{content}
\href{/docs/reference/foundations/function/}{function} , } {
\href{/docs/reference/foundations/content/}{content} , }

) -\textgreater{} \href{/docs/reference/foundations/content/}{content}

\subsubsection{\texorpdfstring{\texttt{\ block\ }}{ block }}\label{parameters-block}

\href{/docs/reference/foundations/bool/}{bool}

{{ Settable }}

\phantomsection\label{parameters-block-settable-tooltip}
Settable parameters can be customized for all following uses of the
function with a \texttt{\ set\ } rule.

Whether the equation is displayed as a separate block.

Default: \texttt{\ }{\texttt{\ false\ }}\texttt{\ }

\subsubsection{\texorpdfstring{\texttt{\ numbering\ }}{ numbering }}\label{parameters-numbering}

\href{/docs/reference/foundations/none/}{none} {or}
\href{/docs/reference/foundations/str/}{str} {or}
\href{/docs/reference/foundations/function/}{function}

{{ Settable }}

\phantomsection\label{parameters-numbering-settable-tooltip}
Settable parameters can be customized for all following uses of the
function with a \texttt{\ set\ } rule.

How to \href{/docs/reference/model/numbering/}{number} block-level
equations.

Default: \texttt{\ }{\texttt{\ none\ }}\texttt{\ }

\includesvg[width=0.16667in,height=0.16667in]{/assets/icons/16-arrow-right.svg}
View example

\begin{verbatim}
#set math.equation(numbering: "(1)")

We define:
$ phi.alt := (1 + sqrt(5)) / 2 $ <ratio>

With @ratio, we get:
$ F_n = floor(1 / sqrt(5) phi.alt^n) $
\end{verbatim}

\includegraphics[width=5in,height=\textheight,keepaspectratio]{/assets/docs/ICkRN4qFA2wn3VV_dGJcKAAAAAAAAAAA.png}

\subsubsection{\texorpdfstring{\texttt{\ number-align\ }}{ number-align }}\label{parameters-number-align}

\href{/docs/reference/layout/alignment/}{alignment}

{{ Settable }}

\phantomsection\label{parameters-number-align-settable-tooltip}
Settable parameters can be customized for all following uses of the
function with a \texttt{\ set\ } rule.

The alignment of the equation numbering.

By default, the alignment is
\texttt{\ end\ }{\texttt{\ +\ }}\texttt{\ horizon\ } . For the
horizontal component, you can use \texttt{\ right\ } , \texttt{\ left\ }
, or \texttt{\ start\ } and \texttt{\ end\ } of the text direction; for
the vertical component, you can use \texttt{\ top\ } ,
\texttt{\ horizon\ } , or \texttt{\ bottom\ } .

Default: \texttt{\ end\ }{\texttt{\ +\ }}\texttt{\ horizon\ }

\includesvg[width=0.16667in,height=0.16667in]{/assets/icons/16-arrow-right.svg}
View example

\begin{verbatim}
#set math.equation(numbering: "(1)", number-align: bottom)

We can calculate:
$ E &= sqrt(m_0^2 + p^2) \
    &approx 125 "GeV" $
\end{verbatim}

\includegraphics[width=5in,height=\textheight,keepaspectratio]{/assets/docs/EjQKswH-OBAc5Rwhl-7WNQAAAAAAAAAA.png}

\subsubsection{\texorpdfstring{\texttt{\ supplement\ }}{ supplement }}\label{parameters-supplement}

\href{/docs/reference/foundations/none/}{none} {or}
\href{/docs/reference/foundations/auto/}{auto} {or}
\href{/docs/reference/foundations/content/}{content} {or}
\href{/docs/reference/foundations/function/}{function}

{{ Settable }}

\phantomsection\label{parameters-supplement-settable-tooltip}
Settable parameters can be customized for all following uses of the
function with a \texttt{\ set\ } rule.

A supplement for the equation.

For references to equations, this is added before the referenced number.

If a function is specified, it is passed the referenced equation and
should return content.

Default: \texttt{\ }{\texttt{\ auto\ }}\texttt{\ }

\includesvg[width=0.16667in,height=0.16667in]{/assets/icons/16-arrow-right.svg}
View example

\begin{verbatim}
#set math.equation(numbering: "(1)", supplement: [Eq.])

We define:
$ phi.alt := (1 + sqrt(5)) / 2 $ <ratio>

With @ratio, we get:
$ F_n = floor(1 / sqrt(5) phi.alt^n) $
\end{verbatim}

\includegraphics[width=5in,height=\textheight,keepaspectratio]{/assets/docs/LsvSGn7Nchg2dddv3zDBtAAAAAAAAAAA.png}

\subsubsection{\texorpdfstring{\texttt{\ body\ }}{ body }}\label{parameters-body}

\href{/docs/reference/foundations/content/}{content}

{Required} {{ Positional }}

\phantomsection\label{parameters-body-positional-tooltip}
Positional parameters are specified in order, without names.

The contents of the equation.

\href{/docs/reference/math/class/}{\pandocbounded{\includesvg[keepaspectratio]{/assets/icons/16-arrow-right.svg}}}

{ Class } { Previous page }

\href{/docs/reference/math/frac/}{\pandocbounded{\includesvg[keepaspectratio]{/assets/icons/16-arrow-right.svg}}}

{ Fraction } { Next page }




\section{C Docs LaTeX/docs/reference/visualize.tex}
\section{Docs LaTeX/typst.app/docs/reference/visualize/circle.tex}
\title{typst.app/docs/reference/visualize/circle}

\begin{itemize}
\tightlist
\item
  \href{/docs}{\includesvg[width=0.16667in,height=0.16667in]{/assets/icons/16-docs-dark.svg}}
\item
  \includesvg[width=0.16667in,height=0.16667in]{/assets/icons/16-arrow-right.svg}
\item
  \href{/docs/reference/}{Reference}
\item
  \includesvg[width=0.16667in,height=0.16667in]{/assets/icons/16-arrow-right.svg}
\item
  \href{/docs/reference/visualize/}{Visualize}
\item
  \includesvg[width=0.16667in,height=0.16667in]{/assets/icons/16-arrow-right.svg}
\item
  \href{/docs/reference/visualize/circle/}{Circle}
\end{itemize}

\section{\texorpdfstring{\texttt{\ circle\ } {{ Element
}}}{ circle   Element }}\label{summary}

\phantomsection\label{element-tooltip}
Element functions can be customized with \texttt{\ set\ } and
\texttt{\ show\ } rules.

A circle with optional content.

\subsection{Example}\label{example}

\begin{verbatim}
// Without content.
#circle(radius: 25pt)

// With content.
#circle[
  #set align(center + horizon)
  Automatically \
  sized to fit.
]
\end{verbatim}

\includegraphics[width=5in,height=\textheight,keepaspectratio]{/assets/docs/H1niwFeoKUTVgzuqcmZ_VgAAAAAAAAAA.png}

\subsection{\texorpdfstring{{ Parameters
}}{ Parameters }}\label{parameters}

\phantomsection\label{parameters-tooltip}
Parameters are the inputs to a function. They are specified in
parentheses after the function name.

{ circle } (

{ \hyperref[parameters-radius]{radius :}
\href{/docs/reference/layout/length/}{length} , } {
\hyperref[parameters-width]{width :}
\href{/docs/reference/foundations/auto/}{auto}
\href{/docs/reference/layout/relative/}{relative} , } {
\hyperref[parameters-height]{height :}
\href{/docs/reference/foundations/auto/}{auto}
\href{/docs/reference/layout/relative/}{relative}
\href{/docs/reference/layout/fraction/}{fraction} , } {
\hyperref[parameters-fill]{fill :}
\href{/docs/reference/foundations/none/}{none}
\href{/docs/reference/visualize/color/}{color}
\href{/docs/reference/visualize/gradient/}{gradient}
\href{/docs/reference/visualize/pattern/}{pattern} , } {
\hyperref[parameters-stroke]{stroke :}
\href{/docs/reference/foundations/none/}{none}
\href{/docs/reference/foundations/auto/}{auto}
\href{/docs/reference/layout/length/}{length}
\href{/docs/reference/visualize/color/}{color}
\href{/docs/reference/visualize/gradient/}{gradient}
\href{/docs/reference/visualize/stroke/}{stroke}
\href{/docs/reference/visualize/pattern/}{pattern}
\href{/docs/reference/foundations/dictionary/}{dictionary} , } {
\hyperref[parameters-inset]{inset :}
\href{/docs/reference/layout/relative/}{relative}
\href{/docs/reference/foundations/dictionary/}{dictionary} , } {
\hyperref[parameters-outset]{outset :}
\href{/docs/reference/layout/relative/}{relative}
\href{/docs/reference/foundations/dictionary/}{dictionary} , } {
\hyperref[parameters-body]{}
\href{/docs/reference/foundations/none/}{none}
\href{/docs/reference/foundations/content/}{content} , }

) -\textgreater{} \href{/docs/reference/foundations/content/}{content}

\subsubsection{\texorpdfstring{\texttt{\ radius\ }}{ radius }}\label{parameters-radius}

\href{/docs/reference/layout/length/}{length}

{{ Settable }}

\phantomsection\label{parameters-radius-settable-tooltip}
Settable parameters can be customized for all following uses of the
function with a \texttt{\ set\ } rule.

The circle\textquotesingle s radius. This is mutually exclusive with
\texttt{\ width\ } and \texttt{\ height\ } .

Default: \texttt{\ }{\texttt{\ 0pt\ }}\texttt{\ }

\subsubsection{\texorpdfstring{\texttt{\ width\ }}{ width }}\label{parameters-width}

\href{/docs/reference/foundations/auto/}{auto} {or}
\href{/docs/reference/layout/relative/}{relative}

{{ Settable }}

\phantomsection\label{parameters-width-settable-tooltip}
Settable parameters can be customized for all following uses of the
function with a \texttt{\ set\ } rule.

The circle\textquotesingle s width. This is mutually exclusive with
\texttt{\ radius\ } and \texttt{\ height\ } .

In contrast to \texttt{\ radius\ } , this can be relative to the parent
container\textquotesingle s width.

Default: \texttt{\ }{\texttt{\ auto\ }}\texttt{\ }

\subsubsection{\texorpdfstring{\texttt{\ height\ }}{ height }}\label{parameters-height}

\href{/docs/reference/foundations/auto/}{auto} {or}
\href{/docs/reference/layout/relative/}{relative} {or}
\href{/docs/reference/layout/fraction/}{fraction}

{{ Settable }}

\phantomsection\label{parameters-height-settable-tooltip}
Settable parameters can be customized for all following uses of the
function with a \texttt{\ set\ } rule.

The circle\textquotesingle s height. This is mutually exclusive with
\texttt{\ radius\ } and \texttt{\ width\ } .

In contrast to \texttt{\ radius\ } , this can be relative to the parent
container\textquotesingle s height.

Default: \texttt{\ }{\texttt{\ auto\ }}\texttt{\ }

\subsubsection{\texorpdfstring{\texttt{\ fill\ }}{ fill }}\label{parameters-fill}

\href{/docs/reference/foundations/none/}{none} {or}
\href{/docs/reference/visualize/color/}{color} {or}
\href{/docs/reference/visualize/gradient/}{gradient} {or}
\href{/docs/reference/visualize/pattern/}{pattern}

{{ Settable }}

\phantomsection\label{parameters-fill-settable-tooltip}
Settable parameters can be customized for all following uses of the
function with a \texttt{\ set\ } rule.

How to fill the circle. See the
\href{/docs/reference/visualize/rect/\#parameters-fill}{rectangle\textquotesingle s
documentation} for more details.

Default: \texttt{\ }{\texttt{\ none\ }}\texttt{\ }

\subsubsection{\texorpdfstring{\texttt{\ stroke\ }}{ stroke }}\label{parameters-stroke}

\href{/docs/reference/foundations/none/}{none} {or}
\href{/docs/reference/foundations/auto/}{auto} {or}
\href{/docs/reference/layout/length/}{length} {or}
\href{/docs/reference/visualize/color/}{color} {or}
\href{/docs/reference/visualize/gradient/}{gradient} {or}
\href{/docs/reference/visualize/stroke/}{stroke} {or}
\href{/docs/reference/visualize/pattern/}{pattern} {or}
\href{/docs/reference/foundations/dictionary/}{dictionary}

{{ Settable }}

\phantomsection\label{parameters-stroke-settable-tooltip}
Settable parameters can be customized for all following uses of the
function with a \texttt{\ set\ } rule.

How to stroke the circle. See the
\href{/docs/reference/visualize/rect/\#parameters-stroke}{rectangle\textquotesingle s
documentation} for more details.

Default: \texttt{\ }{\texttt{\ auto\ }}\texttt{\ }

\subsubsection{\texorpdfstring{\texttt{\ inset\ }}{ inset }}\label{parameters-inset}

\href{/docs/reference/layout/relative/}{relative} {or}
\href{/docs/reference/foundations/dictionary/}{dictionary}

{{ Settable }}

\phantomsection\label{parameters-inset-settable-tooltip}
Settable parameters can be customized for all following uses of the
function with a \texttt{\ set\ } rule.

How much to pad the circle\textquotesingle s content. See the
\href{/docs/reference/layout/box/\#parameters-inset}{box\textquotesingle s
documentation} for more details.

Default:
\texttt{\ }{\texttt{\ 0\%\ }}\texttt{\ }{\texttt{\ +\ }}\texttt{\ }{\texttt{\ 5pt\ }}\texttt{\ }

\subsubsection{\texorpdfstring{\texttt{\ outset\ }}{ outset }}\label{parameters-outset}

\href{/docs/reference/layout/relative/}{relative} {or}
\href{/docs/reference/foundations/dictionary/}{dictionary}

{{ Settable }}

\phantomsection\label{parameters-outset-settable-tooltip}
Settable parameters can be customized for all following uses of the
function with a \texttt{\ set\ } rule.

How much to expand the circle\textquotesingle s size without affecting
the layout. See the
\href{/docs/reference/layout/box/\#parameters-outset}{box\textquotesingle s
documentation} for more details.

Default:
\texttt{\ }{\texttt{\ (\ }}\texttt{\ }{\texttt{\ :\ }}\texttt{\ }{\texttt{\ )\ }}\texttt{\ }

\subsubsection{\texorpdfstring{\texttt{\ body\ }}{ body }}\label{parameters-body}

\href{/docs/reference/foundations/none/}{none} {or}
\href{/docs/reference/foundations/content/}{content}

{{ Positional }}

\phantomsection\label{parameters-body-positional-tooltip}
Positional parameters are specified in order, without names.

{{ Settable }}

\phantomsection\label{parameters-body-settable-tooltip}
Settable parameters can be customized for all following uses of the
function with a \texttt{\ set\ } rule.

The content to place into the circle. The circle expands to fit this
content, keeping the 1-1 aspect ratio.

Default: \texttt{\ }{\texttt{\ none\ }}\texttt{\ }

\href{/docs/reference/visualize/}{\pandocbounded{\includesvg[keepaspectratio]{/assets/icons/16-arrow-right.svg}}}

{ Visualize } { Previous page }

\href{/docs/reference/visualize/color/}{\pandocbounded{\includesvg[keepaspectratio]{/assets/icons/16-arrow-right.svg}}}

{ Color } { Next page }


\section{Docs LaTeX/typst.app/docs/reference/visualize/polygon.tex}
\title{typst.app/docs/reference/visualize/polygon}

\begin{itemize}
\tightlist
\item
  \href{/docs}{\includesvg[width=0.16667in,height=0.16667in]{/assets/icons/16-docs-dark.svg}}
\item
  \includesvg[width=0.16667in,height=0.16667in]{/assets/icons/16-arrow-right.svg}
\item
  \href{/docs/reference/}{Reference}
\item
  \includesvg[width=0.16667in,height=0.16667in]{/assets/icons/16-arrow-right.svg}
\item
  \href{/docs/reference/visualize/}{Visualize}
\item
  \includesvg[width=0.16667in,height=0.16667in]{/assets/icons/16-arrow-right.svg}
\item
  \href{/docs/reference/visualize/polygon/}{Polygon}
\end{itemize}

\section{\texorpdfstring{\texttt{\ polygon\ } {{ Element
}}}{ polygon   Element }}\label{summary}

\phantomsection\label{element-tooltip}
Element functions can be customized with \texttt{\ set\ } and
\texttt{\ show\ } rules.

A closed polygon.

The polygon is defined by its corner points and is closed automatically.

\subsection{Example}\label{example}

\begin{verbatim}
#polygon(
  fill: blue.lighten(80%),
  stroke: blue,
  (20%, 0pt),
  (60%, 0pt),
  (80%, 2cm),
  (0%,  2cm),
)
\end{verbatim}

\includegraphics[width=5in,height=\textheight,keepaspectratio]{/assets/docs/TuzATomarVg-0NmUVu3QFAAAAAAAAAAA.png}

\subsection{\texorpdfstring{{ Parameters
}}{ Parameters }}\label{parameters}

\phantomsection\label{parameters-tooltip}
Parameters are the inputs to a function. They are specified in
parentheses after the function name.

{ polygon } (

{ \hyperref[parameters-fill]{fill :}
\href{/docs/reference/foundations/none/}{none}
\href{/docs/reference/visualize/color/}{color}
\href{/docs/reference/visualize/gradient/}{gradient}
\href{/docs/reference/visualize/pattern/}{pattern} , } {
\hyperref[parameters-fill-rule]{fill-rule :}
\href{/docs/reference/foundations/str/}{str} , } {
\hyperref[parameters-stroke]{stroke :}
\href{/docs/reference/foundations/none/}{none}
\href{/docs/reference/foundations/auto/}{auto}
\href{/docs/reference/layout/length/}{length}
\href{/docs/reference/visualize/color/}{color}
\href{/docs/reference/visualize/gradient/}{gradient}
\href{/docs/reference/visualize/stroke/}{stroke}
\href{/docs/reference/visualize/pattern/}{pattern}
\href{/docs/reference/foundations/dictionary/}{dictionary} , } {
\hyperref[parameters-vertices]{..}
\href{/docs/reference/foundations/array/}{array} , }

) -\textgreater{} \href{/docs/reference/foundations/content/}{content}

\subsubsection{\texorpdfstring{\texttt{\ fill\ }}{ fill }}\label{parameters-fill}

\href{/docs/reference/foundations/none/}{none} {or}
\href{/docs/reference/visualize/color/}{color} {or}
\href{/docs/reference/visualize/gradient/}{gradient} {or}
\href{/docs/reference/visualize/pattern/}{pattern}

{{ Settable }}

\phantomsection\label{parameters-fill-settable-tooltip}
Settable parameters can be customized for all following uses of the
function with a \texttt{\ set\ } rule.

How to fill the polygon.

When setting a fill, the default stroke disappears. To create a
rectangle with both fill and stroke, you have to configure both.

Default: \texttt{\ }{\texttt{\ none\ }}\texttt{\ }

\subsubsection{\texorpdfstring{\texttt{\ fill-rule\ }}{ fill-rule }}\label{parameters-fill-rule}

\href{/docs/reference/foundations/str/}{str}

{{ Settable }}

\phantomsection\label{parameters-fill-rule-settable-tooltip}
Settable parameters can be customized for all following uses of the
function with a \texttt{\ set\ } rule.

The drawing rule used to fill the polygon.

See the
\href{/docs/reference/visualize/path/\#parameters-fill-rule}{path
documentation} for an example.

\begin{longtable}[]{@{}ll@{}}
\toprule\noalign{}
Variant & Details \\
\midrule\noalign{}
\endhead
\bottomrule\noalign{}
\endlastfoot
\texttt{\ "\ non-zero\ "\ } & Specifies that "inside" is computed by a
non-zero sum of signed edge crossings. \\
\texttt{\ "\ even-odd\ "\ } & Specifies that "inside" is computed by an
odd number of edge crossings. \\
\end{longtable}

Default: \texttt{\ }{\texttt{\ "non-zero"\ }}\texttt{\ }

\subsubsection{\texorpdfstring{\texttt{\ stroke\ }}{ stroke }}\label{parameters-stroke}

\href{/docs/reference/foundations/none/}{none} {or}
\href{/docs/reference/foundations/auto/}{auto} {or}
\href{/docs/reference/layout/length/}{length} {or}
\href{/docs/reference/visualize/color/}{color} {or}
\href{/docs/reference/visualize/gradient/}{gradient} {or}
\href{/docs/reference/visualize/stroke/}{stroke} {or}
\href{/docs/reference/visualize/pattern/}{pattern} {or}
\href{/docs/reference/foundations/dictionary/}{dictionary}

{{ Settable }}

\phantomsection\label{parameters-stroke-settable-tooltip}
Settable parameters can be customized for all following uses of the
function with a \texttt{\ set\ } rule.

How to \href{/docs/reference/visualize/stroke/}{stroke} the polygon.
This can be:

Can be set to \texttt{\ }{\texttt{\ none\ }}\texttt{\ } to disable the
stroke or to \texttt{\ }{\texttt{\ auto\ }}\texttt{\ } for a stroke of
\texttt{\ }{\texttt{\ 1pt\ }}\texttt{\ } black if and if only if no fill
is given.

Default: \texttt{\ }{\texttt{\ auto\ }}\texttt{\ }

\subsubsection{\texorpdfstring{\texttt{\ vertices\ }}{ vertices }}\label{parameters-vertices}

\href{/docs/reference/foundations/array/}{array}

{Required} {{ Positional }}

\phantomsection\label{parameters-vertices-positional-tooltip}
Positional parameters are specified in order, without names.

{{ Variadic }}

\phantomsection\label{parameters-vertices-variadic-tooltip}
Variadic parameters can be specified multiple times.

The vertices of the polygon. Each point is specified as an array of two
\href{/docs/reference/layout/relative/}{relative lengths} .

\subsection{\texorpdfstring{{ Definitions
}}{ Definitions }}\label{definitions}

\phantomsection\label{definitions-tooltip}
Functions and types and can have associated definitions. These are
accessed by specifying the function or type, followed by a period, and
then the definition\textquotesingle s name.

\subsubsection{\texorpdfstring{\texttt{\ regular\ }}{ regular }}\label{definitions-regular}

A regular polygon, defined by its size and number of vertices.

polygon { . } { regular } (

{ \hyperref[definitions-regular-parameters-fill]{fill :}
\href{/docs/reference/foundations/none/}{none}
\href{/docs/reference/visualize/color/}{color}
\href{/docs/reference/visualize/gradient/}{gradient}
\href{/docs/reference/visualize/pattern/}{pattern} , } {
\hyperref[definitions-regular-parameters-stroke]{stroke :}
\href{/docs/reference/foundations/none/}{none}
\href{/docs/reference/foundations/auto/}{auto}
\href{/docs/reference/layout/length/}{length}
\href{/docs/reference/visualize/color/}{color}
\href{/docs/reference/visualize/gradient/}{gradient}
\href{/docs/reference/visualize/stroke/}{stroke}
\href{/docs/reference/visualize/pattern/}{pattern}
\href{/docs/reference/foundations/dictionary/}{dictionary} , } {
\hyperref[definitions-regular-parameters-size]{size :}
\href{/docs/reference/layout/length/}{length} , } {
\hyperref[definitions-regular-parameters-vertices]{vertices :}
\href{/docs/reference/foundations/int/}{int} , }

) -\textgreater{} \href{/docs/reference/foundations/content/}{content}

\begin{verbatim}
#polygon.regular(
  fill: blue.lighten(80%),
  stroke: blue,
  size: 30pt,
  vertices: 3,
)
\end{verbatim}

\includegraphics[width=5in,height=\textheight,keepaspectratio]{/assets/docs/nSKAw-cASGAIxDorv3UyHgAAAAAAAAAA.png}

\paragraph{\texorpdfstring{\texttt{\ fill\ }}{ fill }}\label{definitions-regular-fill}

\href{/docs/reference/foundations/none/}{none} {or}
\href{/docs/reference/visualize/color/}{color} {or}
\href{/docs/reference/visualize/gradient/}{gradient} {or}
\href{/docs/reference/visualize/pattern/}{pattern}

How to fill the polygon. See the general
\href{/docs/reference/visualize/polygon/\#parameters-fill}{polygon\textquotesingle s
documentation} for more details.

\paragraph{\texorpdfstring{\texttt{\ stroke\ }}{ stroke }}\label{definitions-regular-stroke}

\href{/docs/reference/foundations/none/}{none} {or}
\href{/docs/reference/foundations/auto/}{auto} {or}
\href{/docs/reference/layout/length/}{length} {or}
\href{/docs/reference/visualize/color/}{color} {or}
\href{/docs/reference/visualize/gradient/}{gradient} {or}
\href{/docs/reference/visualize/stroke/}{stroke} {or}
\href{/docs/reference/visualize/pattern/}{pattern} {or}
\href{/docs/reference/foundations/dictionary/}{dictionary}

How to stroke the polygon. See the general
\href{/docs/reference/visualize/polygon/\#parameters-stroke}{polygon\textquotesingle s
documentation} for more details.

\paragraph{\texorpdfstring{\texttt{\ size\ }}{ size }}\label{definitions-regular-size}

\href{/docs/reference/layout/length/}{length}

The diameter of the
\href{https://en.wikipedia.org/wiki/Circumcircle}{circumcircle} of the
regular polygon.

Default: \texttt{\ }{\texttt{\ 1em\ }}\texttt{\ }

\paragraph{\texorpdfstring{\texttt{\ vertices\ }}{ vertices }}\label{definitions-regular-vertices}

\href{/docs/reference/foundations/int/}{int}

The number of vertices in the polygon.

Default: \texttt{\ }{\texttt{\ 3\ }}\texttt{\ }

\href{/docs/reference/visualize/pattern/}{\pandocbounded{\includesvg[keepaspectratio]{/assets/icons/16-arrow-right.svg}}}

{ Pattern } { Previous page }

\href{/docs/reference/visualize/rect/}{\pandocbounded{\includesvg[keepaspectratio]{/assets/icons/16-arrow-right.svg}}}

{ Rectangle } { Next page }


\section{Docs LaTeX/typst.app/docs/reference/visualize/path.tex}
\title{typst.app/docs/reference/visualize/path}

\begin{itemize}
\tightlist
\item
  \href{/docs}{\includesvg[width=0.16667in,height=0.16667in]{/assets/icons/16-docs-dark.svg}}
\item
  \includesvg[width=0.16667in,height=0.16667in]{/assets/icons/16-arrow-right.svg}
\item
  \href{/docs/reference/}{Reference}
\item
  \includesvg[width=0.16667in,height=0.16667in]{/assets/icons/16-arrow-right.svg}
\item
  \href{/docs/reference/visualize/}{Visualize}
\item
  \includesvg[width=0.16667in,height=0.16667in]{/assets/icons/16-arrow-right.svg}
\item
  \href{/docs/reference/visualize/path/}{Path}
\end{itemize}

\section{\texorpdfstring{\texttt{\ path\ } {{ Element
}}}{ path   Element }}\label{summary}

\phantomsection\label{element-tooltip}
Element functions can be customized with \texttt{\ set\ } and
\texttt{\ show\ } rules.

A path through a list of points, connected by Bezier curves.

\subsection{Example}\label{example}

\begin{verbatim}
#path(
  fill: blue.lighten(80%),
  stroke: blue,
  closed: true,
  (0pt, 50pt),
  (100%, 50pt),
  ((50%, 0pt), (40pt, 0pt)),
)
\end{verbatim}

\includegraphics[width=5in,height=\textheight,keepaspectratio]{/assets/docs/fHH_90d6MEksjFQh_gCkDwAAAAAAAAAA.png}

\subsection{\texorpdfstring{{ Parameters
}}{ Parameters }}\label{parameters}

\phantomsection\label{parameters-tooltip}
Parameters are the inputs to a function. They are specified in
parentheses after the function name.

{ path } (

{ \hyperref[parameters-fill]{fill :}
\href{/docs/reference/foundations/none/}{none}
\href{/docs/reference/visualize/color/}{color}
\href{/docs/reference/visualize/gradient/}{gradient}
\href{/docs/reference/visualize/pattern/}{pattern} , } {
\hyperref[parameters-fill-rule]{fill-rule :}
\href{/docs/reference/foundations/str/}{str} , } {
\hyperref[parameters-stroke]{stroke :}
\href{/docs/reference/foundations/none/}{none}
\href{/docs/reference/foundations/auto/}{auto}
\href{/docs/reference/layout/length/}{length}
\href{/docs/reference/visualize/color/}{color}
\href{/docs/reference/visualize/gradient/}{gradient}
\href{/docs/reference/visualize/stroke/}{stroke}
\href{/docs/reference/visualize/pattern/}{pattern}
\href{/docs/reference/foundations/dictionary/}{dictionary} , } {
\hyperref[parameters-closed]{closed :}
\href{/docs/reference/foundations/bool/}{bool} , } {
\hyperref[parameters-vertices]{..}
\href{/docs/reference/foundations/array/}{array} , }

) -\textgreater{} \href{/docs/reference/foundations/content/}{content}

\subsubsection{\texorpdfstring{\texttt{\ fill\ }}{ fill }}\label{parameters-fill}

\href{/docs/reference/foundations/none/}{none} {or}
\href{/docs/reference/visualize/color/}{color} {or}
\href{/docs/reference/visualize/gradient/}{gradient} {or}
\href{/docs/reference/visualize/pattern/}{pattern}

{{ Settable }}

\phantomsection\label{parameters-fill-settable-tooltip}
Settable parameters can be customized for all following uses of the
function with a \texttt{\ set\ } rule.

How to fill the path.

When setting a fill, the default stroke disappears. To create a
rectangle with both fill and stroke, you have to configure both.

Default: \texttt{\ }{\texttt{\ none\ }}\texttt{\ }

\subsubsection{\texorpdfstring{\texttt{\ fill-rule\ }}{ fill-rule }}\label{parameters-fill-rule}

\href{/docs/reference/foundations/str/}{str}

{{ Settable }}

\phantomsection\label{parameters-fill-rule-settable-tooltip}
Settable parameters can be customized for all following uses of the
function with a \texttt{\ set\ } rule.

The drawing rule used to fill the path.

\begin{longtable}[]{@{}ll@{}}
\toprule\noalign{}
Variant & Details \\
\midrule\noalign{}
\endhead
\bottomrule\noalign{}
\endlastfoot
\texttt{\ "\ non-zero\ "\ } & Specifies that "inside" is computed by a
non-zero sum of signed edge crossings. \\
\texttt{\ "\ even-odd\ "\ } & Specifies that "inside" is computed by an
odd number of edge crossings. \\
\end{longtable}

Default: \texttt{\ }{\texttt{\ "non-zero"\ }}\texttt{\ }

\includesvg[width=0.16667in,height=0.16667in]{/assets/icons/16-arrow-right.svg}
View example

\begin{verbatim}
// We use `.with` to get a new
// function that has the common
// arguments pre-applied.
#let star = path.with(
  fill: red,
  closed: true,
  (25pt, 0pt),
  (10pt, 50pt),
  (50pt, 20pt),
  (0pt, 20pt),
  (40pt, 50pt),
)

#star(fill-rule: "non-zero")
#star(fill-rule: "even-odd")
\end{verbatim}

\includegraphics[width=5in,height=\textheight,keepaspectratio]{/assets/docs/MJEOUf62l7aK0PG-Hl3HKgAAAAAAAAAA.png}

\subsubsection{\texorpdfstring{\texttt{\ stroke\ }}{ stroke }}\label{parameters-stroke}

\href{/docs/reference/foundations/none/}{none} {or}
\href{/docs/reference/foundations/auto/}{auto} {or}
\href{/docs/reference/layout/length/}{length} {or}
\href{/docs/reference/visualize/color/}{color} {or}
\href{/docs/reference/visualize/gradient/}{gradient} {or}
\href{/docs/reference/visualize/stroke/}{stroke} {or}
\href{/docs/reference/visualize/pattern/}{pattern} {or}
\href{/docs/reference/foundations/dictionary/}{dictionary}

{{ Settable }}

\phantomsection\label{parameters-stroke-settable-tooltip}
Settable parameters can be customized for all following uses of the
function with a \texttt{\ set\ } rule.

How to \href{/docs/reference/visualize/stroke/}{stroke} the path. This
can be:

Can be set to \texttt{\ }{\texttt{\ none\ }}\texttt{\ } to disable the
stroke or to \texttt{\ }{\texttt{\ auto\ }}\texttt{\ } for a stroke of
\texttt{\ }{\texttt{\ 1pt\ }}\texttt{\ } black if and if only if no fill
is given.

Default: \texttt{\ }{\texttt{\ auto\ }}\texttt{\ }

\subsubsection{\texorpdfstring{\texttt{\ closed\ }}{ closed }}\label{parameters-closed}

\href{/docs/reference/foundations/bool/}{bool}

{{ Settable }}

\phantomsection\label{parameters-closed-settable-tooltip}
Settable parameters can be customized for all following uses of the
function with a \texttt{\ set\ } rule.

Whether to close this path with one last bezier curve. This curve will
takes into account the adjacent control points. If you want to close
with a straight line, simply add one last point that\textquotesingle s
the same as the start point.

Default: \texttt{\ }{\texttt{\ false\ }}\texttt{\ }

\subsubsection{\texorpdfstring{\texttt{\ vertices\ }}{ vertices }}\label{parameters-vertices}

\href{/docs/reference/foundations/array/}{array}

{Required} {{ Positional }}

\phantomsection\label{parameters-vertices-positional-tooltip}
Positional parameters are specified in order, without names.

{{ Variadic }}

\phantomsection\label{parameters-vertices-variadic-tooltip}
Variadic parameters can be specified multiple times.

The vertices of the path.

Each vertex can be defined in 3 ways:

\begin{itemize}
\tightlist
\item
  A regular point, as given to the
  \href{/docs/reference/visualize/line/}{\texttt{\ line\ }} or
  \href{/docs/reference/visualize/polygon/}{\texttt{\ polygon\ }}
  function.
\item
  An array of two points, the first being the vertex and the second
  being the control point. The control point is expressed relative to
  the vertex and is mirrored to get the second control point. The given
  control point is the one that affects the curve coming \emph{into}
  this vertex (even for the first point). The mirrored control point
  affects the curve going out of this vertex.
\item
  An array of three points, the first being the vertex and the next
  being the control points (control point for curves coming in and out,
  respectively).
\end{itemize}

\href{/docs/reference/visualize/line/}{\pandocbounded{\includesvg[keepaspectratio]{/assets/icons/16-arrow-right.svg}}}

{ Line } { Previous page }

\href{/docs/reference/visualize/pattern/}{\pandocbounded{\includesvg[keepaspectratio]{/assets/icons/16-arrow-right.svg}}}

{ Pattern } { Next page }


\section{Docs LaTeX/typst.app/docs/reference/visualize/pattern.tex}
\title{typst.app/docs/reference/visualize/pattern}

\begin{itemize}
\tightlist
\item
  \href{/docs}{\includesvg[width=0.16667in,height=0.16667in]{/assets/icons/16-docs-dark.svg}}
\item
  \includesvg[width=0.16667in,height=0.16667in]{/assets/icons/16-arrow-right.svg}
\item
  \href{/docs/reference/}{Reference}
\item
  \includesvg[width=0.16667in,height=0.16667in]{/assets/icons/16-arrow-right.svg}
\item
  \href{/docs/reference/visualize/}{Visualize}
\item
  \includesvg[width=0.16667in,height=0.16667in]{/assets/icons/16-arrow-right.svg}
\item
  \href{/docs/reference/visualize/pattern/}{Pattern}
\end{itemize}

\section{\texorpdfstring{{ pattern }}{ pattern }}\label{summary}

A repeating pattern fill.

Typst supports the most common pattern type of tiled patterns, where a
pattern is repeated in a grid-like fashion, covering the entire area of
an element that is filled or stroked. The pattern is defined by a tile
size and a body defining the content of each cell. You can also add
horizontal or vertical spacing between the cells of the pattern.

\subsection{Examples}\label{examples}

\begin{verbatim}
#let pat = pattern(size: (30pt, 30pt))[
  #place(line(start: (0%, 0%), end: (100%, 100%)))
  #place(line(start: (0%, 100%), end: (100%, 0%)))
]

#rect(fill: pat, width: 100%, height: 60pt, stroke: 1pt)
\end{verbatim}

\includegraphics[width=5in,height=\textheight,keepaspectratio]{/assets/docs/coeD6IerbqenB1CPjs7dfAAAAAAAAAAA.png}

Patterns are also supported on text, but only when setting the
\href{/docs/reference/visualize/pattern/\#parameters-relative}{relativeness}
to either \texttt{\ }{\texttt{\ auto\ }}\texttt{\ } (the default value)
or \texttt{\ }{\texttt{\ "parent"\ }}\texttt{\ } . To create
word-by-word or glyph-by-glyph patterns, you can wrap the words or
characters of your text in \href{/docs/reference/layout/box/}{boxes}
manually or through a \href{/docs/reference/styling/\#show-rules}{show
rule} .

\begin{verbatim}
#let pat = pattern(
  size: (30pt, 30pt),
  relative: "parent",
  square(
    size: 30pt,
    fill: gradient
      .conic(..color.map.rainbow),
  )
)

#set text(fill: pat)
#lorem(10)
\end{verbatim}

\includegraphics[width=5in,height=\textheight,keepaspectratio]{/assets/docs/Vk9hYVErruhpSxeZVudFjQAAAAAAAAAA.png}

You can also space the elements further or closer apart using the
\href{/docs/reference/visualize/pattern/\#parameters-spacing}{\texttt{\ spacing\ }}
feature of the pattern. If the spacing is lower than the size of the
pattern, the pattern will overlap. If it is higher, the pattern will
have gaps of the same color as the background of the pattern.

\begin{verbatim}
#let pat = pattern(
  size: (30pt, 30pt),
  spacing: (10pt, 10pt),
  relative: "parent",
  square(
    size: 30pt,
    fill: gradient
     .conic(..color.map.rainbow),
  ),
)

#rect(
  width: 100%,
  height: 60pt,
  fill: pat,
)
\end{verbatim}

\includegraphics[width=5in,height=\textheight,keepaspectratio]{/assets/docs/yPTj9FTOvqrbv-4eK83U7gAAAAAAAAAA.png}

\subsection{Relativeness}\label{relativeness}

The location of the starting point of the pattern is dependent on the
dimensions of a container. This container can either be the shape that
it is being painted on, or the closest surrounding container. This is
controlled by the \texttt{\ relative\ } argument of a pattern
constructor. By default, patterns are relative to the shape they are
being painted on, unless the pattern is applied on text, in which case
they are relative to the closest ancestor container.

Typst determines the ancestor container as follows:

\begin{itemize}
\tightlist
\item
  For shapes that are placed at the root/top level of the document, the
  closest ancestor is the page itself.
\item
  For other shapes, the ancestor is the innermost
  \href{/docs/reference/layout/block/}{\texttt{\ block\ }} or
  \href{/docs/reference/layout/box/}{\texttt{\ box\ }} that contains the
  shape. This includes the boxes and blocks that are implicitly created
  by show rules and elements. For example, a
  \href{/docs/reference/layout/rotate/}{\texttt{\ rotate\ }} will not
  affect the parent of a gradient, but a
  \href{/docs/reference/layout/grid/}{\texttt{\ grid\ }} will.
\end{itemize}

\subsection{\texorpdfstring{Constructor
{}}{Constructor }}\label{constructor}

\phantomsection\label{constructor-constructor-tooltip}
If a type has a constructor, you can call it like a function to create a
new value of the type.

Construct a new pattern.

{ pattern } (

{ \hyperref[constructor-parameters-size]{size :}
\href{/docs/reference/foundations/auto/}{auto}
\href{/docs/reference/foundations/array/}{array} , } {
\hyperref[constructor-parameters-spacing]{spacing :}
\href{/docs/reference/foundations/array/}{array} , } {
\hyperref[constructor-parameters-relative]{relative :}
\href{/docs/reference/foundations/auto/}{auto}
\href{/docs/reference/foundations/str/}{str} , } {
\href{/docs/reference/foundations/content/}{content} , }

) -\textgreater{} \href{/docs/reference/visualize/pattern/}{pattern}

\begin{verbatim}
#let pat = pattern(
  size: (20pt, 20pt),
  relative: "parent",
  place(
    dx: 5pt,
    dy: 5pt,
    rotate(45deg, square(
      size: 5pt,
      fill: black,
    )),
  ),
)

#rect(width: 100%, height: 60pt, fill: pat)
\end{verbatim}

\includegraphics[width=5in,height=\textheight,keepaspectratio]{/assets/docs/s7EOLk1zJeZ_4afTw83qRwAAAAAAAAAA.png}

\paragraph{\texorpdfstring{\texttt{\ size\ }}{ size }}\label{constructor-size}

\href{/docs/reference/foundations/auto/}{auto} {or}
\href{/docs/reference/foundations/array/}{array}

The bounding box of each cell of the pattern.

Default: \texttt{\ }{\texttt{\ auto\ }}\texttt{\ }

\paragraph{\texorpdfstring{\texttt{\ spacing\ }}{ spacing }}\label{constructor-spacing}

\href{/docs/reference/foundations/array/}{array}

The spacing between cells of the pattern.

Default:
\texttt{\ }{\texttt{\ (\ }}\texttt{\ }{\texttt{\ 0pt\ }}\texttt{\ }{\texttt{\ ,\ }}\texttt{\ }{\texttt{\ 0pt\ }}\texttt{\ }{\texttt{\ )\ }}\texttt{\ }

\paragraph{\texorpdfstring{\texttt{\ relative\ }}{ relative }}\label{constructor-relative}

\href{/docs/reference/foundations/auto/}{auto} {or}
\href{/docs/reference/foundations/str/}{str}

The \hyperref[relativeness]{relative placement} of the pattern.

For an element placed at the root/top level of the document, the parent
is the page itself. For other elements, the parent is the innermost
block, box, column, grid, or stack that contains the element.

\begin{longtable}[]{@{}ll@{}}
\toprule\noalign{}
Variant & Details \\
\midrule\noalign{}
\endhead
\bottomrule\noalign{}
\endlastfoot
\texttt{\ "\ self\ "\ } & The gradient is relative to itself (its own
bounding box). \\
\texttt{\ "\ parent\ "\ } & The gradient is relative to its parent (the
parent\textquotesingle s bounding box). \\
\end{longtable}

Default: \texttt{\ }{\texttt{\ auto\ }}\texttt{\ }

\paragraph{\texorpdfstring{\texttt{\ body\ }}{ body }}\label{constructor-body}

\href{/docs/reference/foundations/content/}{content}

{Required} {{ Positional }}

\phantomsection\label{constructor-body-positional-tooltip}
Positional parameters are specified in order, without names.

The content of each cell of the pattern.

\href{/docs/reference/visualize/path/}{\pandocbounded{\includesvg[keepaspectratio]{/assets/icons/16-arrow-right.svg}}}

{ Path } { Previous page }

\href{/docs/reference/visualize/polygon/}{\pandocbounded{\includesvg[keepaspectratio]{/assets/icons/16-arrow-right.svg}}}

{ Polygon } { Next page }


\section{Docs LaTeX/typst.app/docs/reference/visualize/rect.tex}
\title{typst.app/docs/reference/visualize/rect}

\begin{itemize}
\tightlist
\item
  \href{/docs}{\includesvg[width=0.16667in,height=0.16667in]{/assets/icons/16-docs-dark.svg}}
\item
  \includesvg[width=0.16667in,height=0.16667in]{/assets/icons/16-arrow-right.svg}
\item
  \href{/docs/reference/}{Reference}
\item
  \includesvg[width=0.16667in,height=0.16667in]{/assets/icons/16-arrow-right.svg}
\item
  \href{/docs/reference/visualize/}{Visualize}
\item
  \includesvg[width=0.16667in,height=0.16667in]{/assets/icons/16-arrow-right.svg}
\item
  \href{/docs/reference/visualize/rect/}{Rectangle}
\end{itemize}

\section{\texorpdfstring{\texttt{\ rect\ } {{ Element
}}}{ rect   Element }}\label{summary}

\phantomsection\label{element-tooltip}
Element functions can be customized with \texttt{\ set\ } and
\texttt{\ show\ } rules.

A rectangle with optional content.

\subsection{Example}\label{example}

\begin{verbatim}
// Without content.
#rect(width: 35%, height: 30pt)

// With content.
#rect[
  Automatically sized \
  to fit the content.
]
\end{verbatim}

\includegraphics[width=5in,height=\textheight,keepaspectratio]{/assets/docs/uMLkrKs8AmOe9L-qU4CYKgAAAAAAAAAA.png}

\subsection{\texorpdfstring{{ Parameters
}}{ Parameters }}\label{parameters}

\phantomsection\label{parameters-tooltip}
Parameters are the inputs to a function. They are specified in
parentheses after the function name.

{ rect } (

{ \hyperref[parameters-width]{width :}
\href{/docs/reference/foundations/auto/}{auto}
\href{/docs/reference/layout/relative/}{relative} , } {
\hyperref[parameters-height]{height :}
\href{/docs/reference/foundations/auto/}{auto}
\href{/docs/reference/layout/relative/}{relative}
\href{/docs/reference/layout/fraction/}{fraction} , } {
\hyperref[parameters-fill]{fill :}
\href{/docs/reference/foundations/none/}{none}
\href{/docs/reference/visualize/color/}{color}
\href{/docs/reference/visualize/gradient/}{gradient}
\href{/docs/reference/visualize/pattern/}{pattern} , } {
\hyperref[parameters-stroke]{stroke :}
\href{/docs/reference/foundations/none/}{none}
\href{/docs/reference/foundations/auto/}{auto}
\href{/docs/reference/layout/length/}{length}
\href{/docs/reference/visualize/color/}{color}
\href{/docs/reference/visualize/gradient/}{gradient}
\href{/docs/reference/visualize/stroke/}{stroke}
\href{/docs/reference/visualize/pattern/}{pattern}
\href{/docs/reference/foundations/dictionary/}{dictionary} , } {
\hyperref[parameters-radius]{radius :}
\href{/docs/reference/layout/relative/}{relative}
\href{/docs/reference/foundations/dictionary/}{dictionary} , } {
\hyperref[parameters-inset]{inset :}
\href{/docs/reference/layout/relative/}{relative}
\href{/docs/reference/foundations/dictionary/}{dictionary} , } {
\hyperref[parameters-outset]{outset :}
\href{/docs/reference/layout/relative/}{relative}
\href{/docs/reference/foundations/dictionary/}{dictionary} , } {
\hyperref[parameters-body]{}
\href{/docs/reference/foundations/none/}{none}
\href{/docs/reference/foundations/content/}{content} , }

) -\textgreater{} \href{/docs/reference/foundations/content/}{content}

\subsubsection{\texorpdfstring{\texttt{\ width\ }}{ width }}\label{parameters-width}

\href{/docs/reference/foundations/auto/}{auto} {or}
\href{/docs/reference/layout/relative/}{relative}

{{ Settable }}

\phantomsection\label{parameters-width-settable-tooltip}
Settable parameters can be customized for all following uses of the
function with a \texttt{\ set\ } rule.

The rectangle\textquotesingle s width, relative to its parent container.

Default: \texttt{\ }{\texttt{\ auto\ }}\texttt{\ }

\subsubsection{\texorpdfstring{\texttt{\ height\ }}{ height }}\label{parameters-height}

\href{/docs/reference/foundations/auto/}{auto} {or}
\href{/docs/reference/layout/relative/}{relative} {or}
\href{/docs/reference/layout/fraction/}{fraction}

{{ Settable }}

\phantomsection\label{parameters-height-settable-tooltip}
Settable parameters can be customized for all following uses of the
function with a \texttt{\ set\ } rule.

The rectangle\textquotesingle s height, relative to its parent
container.

Default: \texttt{\ }{\texttt{\ auto\ }}\texttt{\ }

\subsubsection{\texorpdfstring{\texttt{\ fill\ }}{ fill }}\label{parameters-fill}

\href{/docs/reference/foundations/none/}{none} {or}
\href{/docs/reference/visualize/color/}{color} {or}
\href{/docs/reference/visualize/gradient/}{gradient} {or}
\href{/docs/reference/visualize/pattern/}{pattern}

{{ Settable }}

\phantomsection\label{parameters-fill-settable-tooltip}
Settable parameters can be customized for all following uses of the
function with a \texttt{\ set\ } rule.

How to fill the rectangle.

When setting a fill, the default stroke disappears. To create a
rectangle with both fill and stroke, you have to configure both.

Default: \texttt{\ }{\texttt{\ none\ }}\texttt{\ }

\includesvg[width=0.16667in,height=0.16667in]{/assets/icons/16-arrow-right.svg}
View example

\begin{verbatim}
#rect(fill: blue)
\end{verbatim}

\includegraphics[width=5in,height=\textheight,keepaspectratio]{/assets/docs/Xp0gewyTPs1ard61igAjJAAAAAAAAAAA.png}

\subsubsection{\texorpdfstring{\texttt{\ stroke\ }}{ stroke }}\label{parameters-stroke}

\href{/docs/reference/foundations/none/}{none} {or}
\href{/docs/reference/foundations/auto/}{auto} {or}
\href{/docs/reference/layout/length/}{length} {or}
\href{/docs/reference/visualize/color/}{color} {or}
\href{/docs/reference/visualize/gradient/}{gradient} {or}
\href{/docs/reference/visualize/stroke/}{stroke} {or}
\href{/docs/reference/visualize/pattern/}{pattern} {or}
\href{/docs/reference/foundations/dictionary/}{dictionary}

{{ Settable }}

\phantomsection\label{parameters-stroke-settable-tooltip}
Settable parameters can be customized for all following uses of the
function with a \texttt{\ set\ } rule.

How to stroke the rectangle. This can be:

\begin{itemize}
\tightlist
\item
  \texttt{\ }{\texttt{\ none\ }}\texttt{\ } to disable stroking
\item
  \texttt{\ }{\texttt{\ auto\ }}\texttt{\ } for a stroke of
  \texttt{\ }{\texttt{\ 1pt\ }}\texttt{\ }{\texttt{\ +\ }}\texttt{\ black\ }
  if and if only if no fill is given.
\item
  Any kind of \href{/docs/reference/visualize/stroke/}{stroke}
\item
  A dictionary describing the stroke for each side individually. The
  dictionary can contain the following keys in order of precedence:

  \begin{itemize}
  \tightlist
  \item
    \texttt{\ top\ } : The top stroke.
  \item
    \texttt{\ right\ } : The right stroke.
  \item
    \texttt{\ bottom\ } : The bottom stroke.
  \item
    \texttt{\ left\ } : The left stroke.
  \item
    \texttt{\ x\ } : The horizontal stroke.
  \item
    \texttt{\ y\ } : The vertical stroke.
  \item
    \texttt{\ rest\ } : The stroke on all sides except those for which
    the dictionary explicitly sets a size.
  \end{itemize}
\end{itemize}

Default: \texttt{\ }{\texttt{\ auto\ }}\texttt{\ }

\includesvg[width=0.16667in,height=0.16667in]{/assets/icons/16-arrow-right.svg}
View example

\begin{verbatim}
#stack(
  dir: ltr,
  spacing: 1fr,
  rect(stroke: red),
  rect(stroke: 2pt),
  rect(stroke: 2pt + red),
)
\end{verbatim}

\includegraphics[width=5in,height=\textheight,keepaspectratio]{/assets/docs/RNPJxaHVa6js_P-8fJFExAAAAAAAAAAA.png}

\subsubsection{\texorpdfstring{\texttt{\ radius\ }}{ radius }}\label{parameters-radius}

\href{/docs/reference/layout/relative/}{relative} {or}
\href{/docs/reference/foundations/dictionary/}{dictionary}

{{ Settable }}

\phantomsection\label{parameters-radius-settable-tooltip}
Settable parameters can be customized for all following uses of the
function with a \texttt{\ set\ } rule.

How much to round the rectangle\textquotesingle s corners, relative to
the minimum of the width and height divided by two. This can be:

\begin{itemize}
\tightlist
\item
  A relative length for a uniform corner radius.
\item
  A dictionary: With a dictionary, the stroke for each side can be set
  individually. The dictionary can contain the following keys in order
  of precedence:

  \begin{itemize}
  \tightlist
  \item
    \texttt{\ top-left\ } : The top-left corner radius.
  \item
    \texttt{\ top-right\ } : The top-right corner radius.
  \item
    \texttt{\ bottom-right\ } : The bottom-right corner radius.
  \item
    \texttt{\ bottom-left\ } : The bottom-left corner radius.
  \item
    \texttt{\ left\ } : The top-left and bottom-left corner radii.
  \item
    \texttt{\ top\ } : The top-left and top-right corner radii.
  \item
    \texttt{\ right\ } : The top-right and bottom-right corner radii.
  \item
    \texttt{\ bottom\ } : The bottom-left and bottom-right corner radii.
  \item
    \texttt{\ rest\ } : The radii for all corners except those for which
    the dictionary explicitly sets a size.
  \end{itemize}
\end{itemize}

Default:
\texttt{\ }{\texttt{\ (\ }}\texttt{\ }{\texttt{\ :\ }}\texttt{\ }{\texttt{\ )\ }}\texttt{\ }

\includesvg[width=0.16667in,height=0.16667in]{/assets/icons/16-arrow-right.svg}
View example

\begin{verbatim}
#set rect(stroke: 4pt)
#rect(
  radius: (
    left: 5pt,
    top-right: 20pt,
    bottom-right: 10pt,
  ),
  stroke: (
    left: red,
    top: yellow,
    right: green,
    bottom: blue,
  ),
)
\end{verbatim}

\includegraphics[width=5in,height=\textheight,keepaspectratio]{/assets/docs/P93tDNSSrvmdfXv2L7MmYQAAAAAAAAAA.png}

\subsubsection{\texorpdfstring{\texttt{\ inset\ }}{ inset }}\label{parameters-inset}

\href{/docs/reference/layout/relative/}{relative} {or}
\href{/docs/reference/foundations/dictionary/}{dictionary}

{{ Settable }}

\phantomsection\label{parameters-inset-settable-tooltip}
Settable parameters can be customized for all following uses of the
function with a \texttt{\ set\ } rule.

How much to pad the rectangle\textquotesingle s content. See the
\href{/docs/reference/layout/box/\#parameters-outset}{box\textquotesingle s
documentation} for more details.

Default:
\texttt{\ }{\texttt{\ 0\%\ }}\texttt{\ }{\texttt{\ +\ }}\texttt{\ }{\texttt{\ 5pt\ }}\texttt{\ }

\subsubsection{\texorpdfstring{\texttt{\ outset\ }}{ outset }}\label{parameters-outset}

\href{/docs/reference/layout/relative/}{relative} {or}
\href{/docs/reference/foundations/dictionary/}{dictionary}

{{ Settable }}

\phantomsection\label{parameters-outset-settable-tooltip}
Settable parameters can be customized for all following uses of the
function with a \texttt{\ set\ } rule.

How much to expand the rectangle\textquotesingle s size without
affecting the layout. See the
\href{/docs/reference/layout/box/\#parameters-outset}{box\textquotesingle s
documentation} for more details.

Default:
\texttt{\ }{\texttt{\ (\ }}\texttt{\ }{\texttt{\ :\ }}\texttt{\ }{\texttt{\ )\ }}\texttt{\ }

\subsubsection{\texorpdfstring{\texttt{\ body\ }}{ body }}\label{parameters-body}

\href{/docs/reference/foundations/none/}{none} {or}
\href{/docs/reference/foundations/content/}{content}

{{ Positional }}

\phantomsection\label{parameters-body-positional-tooltip}
Positional parameters are specified in order, without names.

{{ Settable }}

\phantomsection\label{parameters-body-settable-tooltip}
Settable parameters can be customized for all following uses of the
function with a \texttt{\ set\ } rule.

The content to place into the rectangle.

When this is omitted, the rectangle takes on a default size of at most
\texttt{\ }{\texttt{\ 45pt\ }}\texttt{\ } by
\texttt{\ }{\texttt{\ 30pt\ }}\texttt{\ } .

Default: \texttt{\ }{\texttt{\ none\ }}\texttt{\ }

\href{/docs/reference/visualize/polygon/}{\pandocbounded{\includesvg[keepaspectratio]{/assets/icons/16-arrow-right.svg}}}

{ Polygon } { Previous page }

\href{/docs/reference/visualize/square/}{\pandocbounded{\includesvg[keepaspectratio]{/assets/icons/16-arrow-right.svg}}}

{ Square } { Next page }


\section{Docs LaTeX/typst.app/docs/reference/visualize/ellipse.tex}
\title{typst.app/docs/reference/visualize/ellipse}

\begin{itemize}
\tightlist
\item
  \href{/docs}{\includesvg[width=0.16667in,height=0.16667in]{/assets/icons/16-docs-dark.svg}}
\item
  \includesvg[width=0.16667in,height=0.16667in]{/assets/icons/16-arrow-right.svg}
\item
  \href{/docs/reference/}{Reference}
\item
  \includesvg[width=0.16667in,height=0.16667in]{/assets/icons/16-arrow-right.svg}
\item
  \href{/docs/reference/visualize/}{Visualize}
\item
  \includesvg[width=0.16667in,height=0.16667in]{/assets/icons/16-arrow-right.svg}
\item
  \href{/docs/reference/visualize/ellipse/}{Ellipse}
\end{itemize}

\section{\texorpdfstring{\texttt{\ ellipse\ } {{ Element
}}}{ ellipse   Element }}\label{summary}

\phantomsection\label{element-tooltip}
Element functions can be customized with \texttt{\ set\ } and
\texttt{\ show\ } rules.

An ellipse with optional content.

\subsection{Example}\label{example}

\begin{verbatim}
// Without content.
#ellipse(width: 35%, height: 30pt)

// With content.
#ellipse[
  #set align(center)
  Automatically sized \
  to fit the content.
]
\end{verbatim}

\includegraphics[width=5in,height=\textheight,keepaspectratio]{/assets/docs/u35LFJMn0LDLxUBqOdjmvgAAAAAAAAAA.png}

\subsection{\texorpdfstring{{ Parameters
}}{ Parameters }}\label{parameters}

\phantomsection\label{parameters-tooltip}
Parameters are the inputs to a function. They are specified in
parentheses after the function name.

{ ellipse } (

{ \hyperref[parameters-width]{width :}
\href{/docs/reference/foundations/auto/}{auto}
\href{/docs/reference/layout/relative/}{relative} , } {
\hyperref[parameters-height]{height :}
\href{/docs/reference/foundations/auto/}{auto}
\href{/docs/reference/layout/relative/}{relative}
\href{/docs/reference/layout/fraction/}{fraction} , } {
\hyperref[parameters-fill]{fill :}
\href{/docs/reference/foundations/none/}{none}
\href{/docs/reference/visualize/color/}{color}
\href{/docs/reference/visualize/gradient/}{gradient}
\href{/docs/reference/visualize/pattern/}{pattern} , } {
\hyperref[parameters-stroke]{stroke :}
\href{/docs/reference/foundations/none/}{none}
\href{/docs/reference/foundations/auto/}{auto}
\href{/docs/reference/layout/length/}{length}
\href{/docs/reference/visualize/color/}{color}
\href{/docs/reference/visualize/gradient/}{gradient}
\href{/docs/reference/visualize/stroke/}{stroke}
\href{/docs/reference/visualize/pattern/}{pattern}
\href{/docs/reference/foundations/dictionary/}{dictionary} , } {
\hyperref[parameters-inset]{inset :}
\href{/docs/reference/layout/relative/}{relative}
\href{/docs/reference/foundations/dictionary/}{dictionary} , } {
\hyperref[parameters-outset]{outset :}
\href{/docs/reference/layout/relative/}{relative}
\href{/docs/reference/foundations/dictionary/}{dictionary} , } {
\hyperref[parameters-body]{}
\href{/docs/reference/foundations/none/}{none}
\href{/docs/reference/foundations/content/}{content} , }

) -\textgreater{} \href{/docs/reference/foundations/content/}{content}

\subsubsection{\texorpdfstring{\texttt{\ width\ }}{ width }}\label{parameters-width}

\href{/docs/reference/foundations/auto/}{auto} {or}
\href{/docs/reference/layout/relative/}{relative}

{{ Settable }}

\phantomsection\label{parameters-width-settable-tooltip}
Settable parameters can be customized for all following uses of the
function with a \texttt{\ set\ } rule.

The ellipse\textquotesingle s width, relative to its parent container.

Default: \texttt{\ }{\texttt{\ auto\ }}\texttt{\ }

\subsubsection{\texorpdfstring{\texttt{\ height\ }}{ height }}\label{parameters-height}

\href{/docs/reference/foundations/auto/}{auto} {or}
\href{/docs/reference/layout/relative/}{relative} {or}
\href{/docs/reference/layout/fraction/}{fraction}

{{ Settable }}

\phantomsection\label{parameters-height-settable-tooltip}
Settable parameters can be customized for all following uses of the
function with a \texttt{\ set\ } rule.

The ellipse\textquotesingle s height, relative to its parent container.

Default: \texttt{\ }{\texttt{\ auto\ }}\texttt{\ }

\subsubsection{\texorpdfstring{\texttt{\ fill\ }}{ fill }}\label{parameters-fill}

\href{/docs/reference/foundations/none/}{none} {or}
\href{/docs/reference/visualize/color/}{color} {or}
\href{/docs/reference/visualize/gradient/}{gradient} {or}
\href{/docs/reference/visualize/pattern/}{pattern}

{{ Settable }}

\phantomsection\label{parameters-fill-settable-tooltip}
Settable parameters can be customized for all following uses of the
function with a \texttt{\ set\ } rule.

How to fill the ellipse. See the
\href{/docs/reference/visualize/rect/\#parameters-fill}{rectangle\textquotesingle s
documentation} for more details.

Default: \texttt{\ }{\texttt{\ none\ }}\texttt{\ }

\subsubsection{\texorpdfstring{\texttt{\ stroke\ }}{ stroke }}\label{parameters-stroke}

\href{/docs/reference/foundations/none/}{none} {or}
\href{/docs/reference/foundations/auto/}{auto} {or}
\href{/docs/reference/layout/length/}{length} {or}
\href{/docs/reference/visualize/color/}{color} {or}
\href{/docs/reference/visualize/gradient/}{gradient} {or}
\href{/docs/reference/visualize/stroke/}{stroke} {or}
\href{/docs/reference/visualize/pattern/}{pattern} {or}
\href{/docs/reference/foundations/dictionary/}{dictionary}

{{ Settable }}

\phantomsection\label{parameters-stroke-settable-tooltip}
Settable parameters can be customized for all following uses of the
function with a \texttt{\ set\ } rule.

How to stroke the ellipse. See the
\href{/docs/reference/visualize/rect/\#parameters-stroke}{rectangle\textquotesingle s
documentation} for more details.

Default: \texttt{\ }{\texttt{\ auto\ }}\texttt{\ }

\subsubsection{\texorpdfstring{\texttt{\ inset\ }}{ inset }}\label{parameters-inset}

\href{/docs/reference/layout/relative/}{relative} {or}
\href{/docs/reference/foundations/dictionary/}{dictionary}

{{ Settable }}

\phantomsection\label{parameters-inset-settable-tooltip}
Settable parameters can be customized for all following uses of the
function with a \texttt{\ set\ } rule.

How much to pad the ellipse\textquotesingle s content. See the
\href{/docs/reference/layout/box/\#parameters-inset}{box\textquotesingle s
documentation} for more details.

Default:
\texttt{\ }{\texttt{\ 0\%\ }}\texttt{\ }{\texttt{\ +\ }}\texttt{\ }{\texttt{\ 5pt\ }}\texttt{\ }

\subsubsection{\texorpdfstring{\texttt{\ outset\ }}{ outset }}\label{parameters-outset}

\href{/docs/reference/layout/relative/}{relative} {or}
\href{/docs/reference/foundations/dictionary/}{dictionary}

{{ Settable }}

\phantomsection\label{parameters-outset-settable-tooltip}
Settable parameters can be customized for all following uses of the
function with a \texttt{\ set\ } rule.

How much to expand the ellipse\textquotesingle s size without affecting
the layout. See the
\href{/docs/reference/layout/box/\#parameters-outset}{box\textquotesingle s
documentation} for more details.

Default:
\texttt{\ }{\texttt{\ (\ }}\texttt{\ }{\texttt{\ :\ }}\texttt{\ }{\texttt{\ )\ }}\texttt{\ }

\subsubsection{\texorpdfstring{\texttt{\ body\ }}{ body }}\label{parameters-body}

\href{/docs/reference/foundations/none/}{none} {or}
\href{/docs/reference/foundations/content/}{content}

{{ Positional }}

\phantomsection\label{parameters-body-positional-tooltip}
Positional parameters are specified in order, without names.

{{ Settable }}

\phantomsection\label{parameters-body-settable-tooltip}
Settable parameters can be customized for all following uses of the
function with a \texttt{\ set\ } rule.

The content to place into the ellipse.

When this is omitted, the ellipse takes on a default size of at most
\texttt{\ }{\texttt{\ 45pt\ }}\texttt{\ } by
\texttt{\ }{\texttt{\ 30pt\ }}\texttt{\ } .

Default: \texttt{\ }{\texttt{\ none\ }}\texttt{\ }

\href{/docs/reference/visualize/color/}{\pandocbounded{\includesvg[keepaspectratio]{/assets/icons/16-arrow-right.svg}}}

{ Color } { Previous page }

\href{/docs/reference/visualize/gradient/}{\pandocbounded{\includesvg[keepaspectratio]{/assets/icons/16-arrow-right.svg}}}

{ Gradient } { Next page }


\section{Docs LaTeX/typst.app/docs/reference/visualize/stroke.tex}
\title{typst.app/docs/reference/visualize/stroke}

\begin{itemize}
\tightlist
\item
  \href{/docs}{\includesvg[width=0.16667in,height=0.16667in]{/assets/icons/16-docs-dark.svg}}
\item
  \includesvg[width=0.16667in,height=0.16667in]{/assets/icons/16-arrow-right.svg}
\item
  \href{/docs/reference/}{Reference}
\item
  \includesvg[width=0.16667in,height=0.16667in]{/assets/icons/16-arrow-right.svg}
\item
  \href{/docs/reference/visualize/}{Visualize}
\item
  \includesvg[width=0.16667in,height=0.16667in]{/assets/icons/16-arrow-right.svg}
\item
  \href{/docs/reference/visualize/stroke/}{Stroke}
\end{itemize}

\section{\texorpdfstring{{ stroke }}{ stroke }}\label{summary}

Defines how to draw a line.

A stroke has a \emph{paint} (a solid color or gradient), a
\emph{thickness,} a line \emph{cap,} a line \emph{join,} a \emph{miter
limit,} and a \emph{dash} pattern. All of these values are optional and
have sensible defaults.

\subsection{Example}\label{example}

\begin{verbatim}
#set line(length: 100%)
#stack(
  spacing: 1em,
  line(stroke: 2pt + red),
  line(stroke: (paint: blue, thickness: 4pt, cap: "round")),
  line(stroke: (paint: blue, thickness: 1pt, dash: "dashed")),
  line(stroke: 2pt + gradient.linear(..color.map.rainbow)),
)
\end{verbatim}

\includegraphics[width=5in,height=\textheight,keepaspectratio]{/assets/docs/3NofubbwIllodsFawlNd8wAAAAAAAAAA.png}

\subsection{Simple strokes}\label{simple-strokes}

You can create a simple solid stroke from a color, a thickness, or a
combination of the two. Specifically, wherever a stroke is expected you
can pass any of the following values:

\begin{itemize}
\tightlist
\item
  A length specifying the stroke\textquotesingle s thickness. The color
  is inherited, defaulting to black.
\item
  A color to use for the stroke. The thickness is inherited, defaulting
  to \texttt{\ }{\texttt{\ 1pt\ }}\texttt{\ } .
\item
  A stroke combined from color and thickness using the \texttt{\ +\ }
  operator as in
  \texttt{\ }{\texttt{\ 2pt\ }}\texttt{\ }{\texttt{\ +\ }}\texttt{\ red\ }
  .
\end{itemize}

For full control, you can also provide a
\href{/docs/reference/foundations/dictionary/}{dictionary} or a
\texttt{\ stroke\ } object to any function that expects a stroke. The
dictionary\textquotesingle s keys may include any of the parameters for
the constructor function, shown below.

\subsection{Fields}\label{fields}

On a stroke object, you can access any of the fields listed in the
constructor function. For example,
\texttt{\ }{\texttt{\ (\ }}\texttt{\ }{\texttt{\ 2pt\ }}\texttt{\ }{\texttt{\ +\ }}\texttt{\ blue\ }{\texttt{\ )\ }}\texttt{\ }{\texttt{\ .\ }}\texttt{\ thickness\ }
is \texttt{\ }{\texttt{\ 2pt\ }}\texttt{\ } . Meanwhile,
\texttt{\ }{\texttt{\ stroke\ }}\texttt{\ }{\texttt{\ (\ }}\texttt{\ red\ }{\texttt{\ )\ }}\texttt{\ }{\texttt{\ .\ }}\texttt{\ cap\ }
is \texttt{\ }{\texttt{\ auto\ }}\texttt{\ } because
it\textquotesingle s unspecified. Fields set to
\texttt{\ }{\texttt{\ auto\ }}\texttt{\ } are inherited.

\subsection{\texorpdfstring{Constructor
{}}{Constructor }}\label{constructor}

\phantomsection\label{constructor-constructor-tooltip}
If a type has a constructor, you can call it like a function to create a
new value of the type.

Converts a value to a stroke or constructs a stroke with the given
parameters.

Note that in most cases you do not need to convert values to strokes in
order to use them, as they will be converted automatically. However,
this constructor can be useful to ensure a value has all the fields of a
stroke.

{ stroke } (

{ \href{/docs/reference/foundations/auto/}{auto}
\href{/docs/reference/visualize/color/}{color}
\href{/docs/reference/visualize/gradient/}{gradient}
\href{/docs/reference/visualize/pattern/}{pattern} , } {
\href{/docs/reference/foundations/auto/}{auto}
\href{/docs/reference/layout/length/}{length} , } {
\href{/docs/reference/foundations/auto/}{auto}
\href{/docs/reference/foundations/str/}{str} , } {
\href{/docs/reference/foundations/auto/}{auto}
\href{/docs/reference/foundations/str/}{str} , } {
\href{/docs/reference/foundations/none/}{none}
\href{/docs/reference/foundations/auto/}{auto}
\href{/docs/reference/foundations/str/}{str}
\href{/docs/reference/foundations/array/}{array}
\href{/docs/reference/foundations/dictionary/}{dictionary} , } {
\href{/docs/reference/foundations/auto/}{auto}
\href{/docs/reference/foundations/float/}{float} , }

) -\textgreater{} \href{/docs/reference/visualize/stroke/}{stroke}

\begin{verbatim}
#let my-func(x) = {
    x = stroke(x) // Convert to a stroke
    [Stroke has thickness #x.thickness.]
}
#my-func(3pt) \
#my-func(red) \
#my-func(stroke(cap: "round", thickness: 1pt))
\end{verbatim}

\includegraphics[width=5in,height=\textheight,keepaspectratio]{/assets/docs/oulcXDNcpunCSxVvCPXMJQAAAAAAAAAA.png}

\paragraph{\texorpdfstring{\texttt{\ paint\ }}{ paint }}\label{constructor-paint}

\href{/docs/reference/foundations/auto/}{auto} {or}
\href{/docs/reference/visualize/color/}{color} {or}
\href{/docs/reference/visualize/gradient/}{gradient} {or}
\href{/docs/reference/visualize/pattern/}{pattern}

{Required} {{ Positional }}

\phantomsection\label{constructor-paint-positional-tooltip}
Positional parameters are specified in order, without names.

The color or gradient to use for the stroke.

If set to \texttt{\ }{\texttt{\ auto\ }}\texttt{\ } , the value is
inherited, defaulting to \texttt{\ black\ } .

\paragraph{\texorpdfstring{\texttt{\ thickness\ }}{ thickness }}\label{constructor-thickness}

\href{/docs/reference/foundations/auto/}{auto} {or}
\href{/docs/reference/layout/length/}{length}

{Required} {{ Positional }}

\phantomsection\label{constructor-thickness-positional-tooltip}
Positional parameters are specified in order, without names.

The stroke\textquotesingle s thickness.

If set to \texttt{\ }{\texttt{\ auto\ }}\texttt{\ } , the value is
inherited, defaulting to \texttt{\ }{\texttt{\ 1pt\ }}\texttt{\ } .

\paragraph{\texorpdfstring{\texttt{\ cap\ }}{ cap }}\label{constructor-cap}

\href{/docs/reference/foundations/auto/}{auto} {or}
\href{/docs/reference/foundations/str/}{str}

{Required} {{ Positional }}

\phantomsection\label{constructor-cap-positional-tooltip}
Positional parameters are specified in order, without names.

How the ends of the stroke are rendered.

If set to \texttt{\ }{\texttt{\ auto\ }}\texttt{\ } , the value is
inherited, defaulting to \texttt{\ }{\texttt{\ "butt"\ }}\texttt{\ } .

\begin{longtable}[]{@{}ll@{}}
\toprule\noalign{}
Variant & Details \\
\midrule\noalign{}
\endhead
\bottomrule\noalign{}
\endlastfoot
\texttt{\ "\ butt\ "\ } & Square stroke cap with the edge at the
stroke\textquotesingle s end point. \\
\texttt{\ "\ round\ "\ } & Circular stroke cap centered at the
stroke\textquotesingle s end point. \\
\texttt{\ "\ square\ "\ } & Square stroke cap centered at the
stroke\textquotesingle s end point. \\
\end{longtable}

\paragraph{\texorpdfstring{\texttt{\ join\ }}{ join }}\label{constructor-join}

\href{/docs/reference/foundations/auto/}{auto} {or}
\href{/docs/reference/foundations/str/}{str}

{Required} {{ Positional }}

\phantomsection\label{constructor-join-positional-tooltip}
Positional parameters are specified in order, without names.

How sharp turns are rendered.

If set to \texttt{\ }{\texttt{\ auto\ }}\texttt{\ } , the value is
inherited, defaulting to \texttt{\ }{\texttt{\ "miter"\ }}\texttt{\ } .

\begin{longtable}[]{@{}ll@{}}
\toprule\noalign{}
Variant & Details \\
\midrule\noalign{}
\endhead
\bottomrule\noalign{}
\endlastfoot
\texttt{\ "\ miter\ "\ } & Segments are joined with sharp edges. Sharp
bends exceeding the miter limit are bevelled instead. \\
\texttt{\ "\ round\ "\ } & Segments are joined with circular corners. \\
\texttt{\ "\ bevel\ "\ } & Segments are joined with a bevel (a straight
edge connecting the butts of the joined segments). \\
\end{longtable}

\paragraph{\texorpdfstring{\texttt{\ dash\ }}{ dash }}\label{constructor-dash}

\href{/docs/reference/foundations/none/}{none} {or}
\href{/docs/reference/foundations/auto/}{auto} {or}
\href{/docs/reference/foundations/str/}{str} {or}
\href{/docs/reference/foundations/array/}{array} {or}
\href{/docs/reference/foundations/dictionary/}{dictionary}

{Required} {{ Positional }}

\phantomsection\label{constructor-dash-positional-tooltip}
Positional parameters are specified in order, without names.

The dash pattern to use. This can be:

\begin{itemize}
\tightlist
\item
  One of the predefined patterns:

  \begin{itemize}
  \tightlist
  \item
    \texttt{\ }{\texttt{\ "solid"\ }}\texttt{\ } or
    \texttt{\ }{\texttt{\ none\ }}\texttt{\ }
  \item
    \texttt{\ }{\texttt{\ "dotted"\ }}\texttt{\ }
  \item
    \texttt{\ }{\texttt{\ "densely-dotted"\ }}\texttt{\ }
  \item
    \texttt{\ }{\texttt{\ "loosely-dotted"\ }}\texttt{\ }
  \item
    \texttt{\ }{\texttt{\ "dashed"\ }}\texttt{\ }
  \item
    \texttt{\ }{\texttt{\ "densely-dashed"\ }}\texttt{\ }
  \item
    \texttt{\ }{\texttt{\ "loosely-dashed"\ }}\texttt{\ }
  \item
    \texttt{\ }{\texttt{\ "dash-dotted"\ }}\texttt{\ }
  \item
    \texttt{\ }{\texttt{\ "densely-dash-dotted"\ }}\texttt{\ }
  \item
    \texttt{\ }{\texttt{\ "loosely-dash-dotted"\ }}\texttt{\ }
  \end{itemize}
\item
  An \href{/docs/reference/foundations/array/}{array} with alternating
  lengths for dashes and gaps. You can also use the string
  \texttt{\ }{\texttt{\ "dot"\ }}\texttt{\ } for a length equal to the
  line thickness.
\item
  A \href{/docs/reference/foundations/dictionary/}{dictionary} with the
  keys \texttt{\ array\ } (same as the array above), and
  \texttt{\ phase\ } (of type
  \href{/docs/reference/layout/length/}{length} ), which defines where
  in the pattern to start drawing.
\end{itemize}

If set to \texttt{\ }{\texttt{\ auto\ }}\texttt{\ } , the value is
inherited, defaulting to \texttt{\ }{\texttt{\ none\ }}\texttt{\ } .

\includesvg[width=0.16667in,height=0.16667in]{/assets/icons/16-arrow-right.svg}
View options

\begin{longtable}[]{@{}ll@{}}
\toprule\noalign{}
Variant & Details \\
\midrule\noalign{}
\endhead
\bottomrule\noalign{}
\endlastfoot
\texttt{\ "\ solid\ "\ } & \\
\texttt{\ "\ dotted\ "\ } & \\
\texttt{\ "\ densely-dotted\ "\ } & \\
\texttt{\ "\ loosely-dotted\ "\ } & \\
\texttt{\ "\ dashed\ "\ } & \\
\texttt{\ "\ densely-dashed\ "\ } & \\
\texttt{\ "\ loosely-dashed\ "\ } & \\
\texttt{\ "\ dash-dotted\ "\ } & \\
\texttt{\ "\ densely-dash-dotted\ "\ } & \\
\texttt{\ "\ loosely-dash-dotted\ "\ } & \\
\end{longtable}

\includesvg[width=0.16667in,height=0.16667in]{/assets/icons/16-arrow-right.svg}
View example

\begin{verbatim}
#set line(length: 100%, stroke: 2pt)
#stack(
  spacing: 1em,
  line(stroke: (dash: "dashed")),
  line(stroke: (dash: (10pt, 5pt, "dot", 5pt))),
  line(stroke: (dash: (array: (10pt, 5pt, "dot", 5pt), phase: 10pt))),
)
\end{verbatim}

\includegraphics[width=5in,height=\textheight,keepaspectratio]{/assets/docs/P38gFluKZcw64WdZR85nHgAAAAAAAAAA.png}

\paragraph{\texorpdfstring{\texttt{\ miter-limit\ }}{ miter-limit }}\label{constructor-miter-limit}

\href{/docs/reference/foundations/auto/}{auto} {or}
\href{/docs/reference/foundations/float/}{float}

{Required} {{ Positional }}

\phantomsection\label{constructor-miter-limit-positional-tooltip}
Positional parameters are specified in order, without names.

Number at which protruding sharp bends are rendered with a bevel instead
or a miter join. The higher the number, the sharper an angle can be
before it is bevelled. Only applicable if \texttt{\ join\ } is
\texttt{\ }{\texttt{\ "miter"\ }}\texttt{\ } .

Specifically, the miter limit is the maximum ratio between the
corner\textquotesingle s protrusion length and the
stroke\textquotesingle s thickness.

If set to \texttt{\ }{\texttt{\ auto\ }}\texttt{\ } , the value is
inherited, defaulting to \texttt{\ }{\texttt{\ 4.0\ }}\texttt{\ } .

\includesvg[width=0.16667in,height=0.16667in]{/assets/icons/16-arrow-right.svg}
View example

\begin{verbatim}
#let points = ((15pt, 0pt), (0pt, 30pt), (30pt, 30pt), (10pt, 20pt))
#set path(stroke: 6pt + blue)
#stack(
    dir: ltr,
    spacing: 1cm,
    path(stroke: (miter-limit: 1), ..points),
    path(stroke: (miter-limit: 4), ..points),
    path(stroke: (miter-limit: 5), ..points),
)
\end{verbatim}

\includegraphics[width=5in,height=\textheight,keepaspectratio]{/assets/docs/3zeU1BuQq8_VfdTfAQbv5QAAAAAAAAAA.png}

\href{/docs/reference/visualize/square/}{\pandocbounded{\includesvg[keepaspectratio]{/assets/icons/16-arrow-right.svg}}}

{ Square } { Previous page }

\href{/docs/reference/introspection/}{\pandocbounded{\includesvg[keepaspectratio]{/assets/icons/16-arrow-right.svg}}}

{ Introspection } { Next page }


\section{Docs LaTeX/typst.app/docs/reference/visualize/gradient.tex}
\title{typst.app/docs/reference/visualize/gradient}

\begin{itemize}
\tightlist
\item
  \href{/docs}{\includesvg[width=0.16667in,height=0.16667in]{/assets/icons/16-docs-dark.svg}}
\item
  \includesvg[width=0.16667in,height=0.16667in]{/assets/icons/16-arrow-right.svg}
\item
  \href{/docs/reference/}{Reference}
\item
  \includesvg[width=0.16667in,height=0.16667in]{/assets/icons/16-arrow-right.svg}
\item
  \href{/docs/reference/visualize/}{Visualize}
\item
  \includesvg[width=0.16667in,height=0.16667in]{/assets/icons/16-arrow-right.svg}
\item
  \href{/docs/reference/visualize/gradient/}{Gradient}
\end{itemize}

\section{\texorpdfstring{{ gradient }}{ gradient }}\label{summary}

A color gradient.

Typst supports linear gradients through the
\href{/docs/reference/visualize/gradient/\#definitions-linear}{\texttt{\ gradient.linear\ }
function} , radial gradients through the
\href{/docs/reference/visualize/gradient/\#definitions-radial}{\texttt{\ gradient.radial\ }
function} , and conic gradients through the
\href{/docs/reference/visualize/gradient/\#definitions-conic}{\texttt{\ gradient.conic\ }
function} .

A gradient can be used for the following purposes:

\begin{itemize}
\tightlist
\item
  As a fill to paint the interior of a shape:
  \texttt{\ }{\texttt{\ rect\ }}\texttt{\ }{\texttt{\ (\ }}\texttt{\ fill\ }{\texttt{\ :\ }}\texttt{\ gradient\ }{\texttt{\ .\ }}\texttt{\ }{\texttt{\ linear\ }}\texttt{\ }{\texttt{\ (\ }}\texttt{\ }{\texttt{\ ..\ }}\texttt{\ }{\texttt{\ )\ }}\texttt{\ }{\texttt{\ )\ }}\texttt{\ }
\item
  As a stroke to paint the outline of a shape:
  \texttt{\ }{\texttt{\ rect\ }}\texttt{\ }{\texttt{\ (\ }}\texttt{\ stroke\ }{\texttt{\ :\ }}\texttt{\ }{\texttt{\ 1pt\ }}\texttt{\ }{\texttt{\ +\ }}\texttt{\ gradient\ }{\texttt{\ .\ }}\texttt{\ }{\texttt{\ linear\ }}\texttt{\ }{\texttt{\ (\ }}\texttt{\ }{\texttt{\ ..\ }}\texttt{\ }{\texttt{\ )\ }}\texttt{\ }{\texttt{\ )\ }}\texttt{\ }
\item
  As the fill of text:
  \texttt{\ }{\texttt{\ set\ }}\texttt{\ }{\texttt{\ text\ }}\texttt{\ }{\texttt{\ (\ }}\texttt{\ fill\ }{\texttt{\ :\ }}\texttt{\ gradient\ }{\texttt{\ .\ }}\texttt{\ }{\texttt{\ linear\ }}\texttt{\ }{\texttt{\ (\ }}\texttt{\ }{\texttt{\ ..\ }}\texttt{\ }{\texttt{\ )\ }}\texttt{\ }{\texttt{\ )\ }}\texttt{\ }
\item
  As a color map you can
  \href{/docs/reference/visualize/gradient/\#definitions-sample}{sample}
  from:
  \texttt{\ gradient\ }{\texttt{\ .\ }}\texttt{\ }{\texttt{\ linear\ }}\texttt{\ }{\texttt{\ (\ }}\texttt{\ }{\texttt{\ ..\ }}\texttt{\ }{\texttt{\ )\ }}\texttt{\ }{\texttt{\ .\ }}\texttt{\ }{\texttt{\ sample\ }}\texttt{\ }{\texttt{\ (\ }}\texttt{\ }{\texttt{\ 50\%\ }}\texttt{\ }{\texttt{\ )\ }}\texttt{\ }
\end{itemize}

\subsection{Examples}\label{examples}

\begin{verbatim}
#stack(
  dir: ltr,
  spacing: 1fr,
  square(fill: gradient.linear(..color.map.rainbow)),
  square(fill: gradient.radial(..color.map.rainbow)),
  square(fill: gradient.conic(..color.map.rainbow)),
)
\end{verbatim}

\includegraphics[width=5in,height=\textheight,keepaspectratio]{/assets/docs/_ynuy5GKkV7ADtX87C9EiAAAAAAAAAAA.png}

Gradients are also supported on text, but only when setting the
\href{/docs/reference/visualize/gradient/\#definitions-relative}{relativeness}
to either \texttt{\ }{\texttt{\ auto\ }}\texttt{\ } (the default value)
or \texttt{\ }{\texttt{\ "parent"\ }}\texttt{\ } . To create
word-by-word or glyph-by-glyph gradients, you can wrap the words or
characters of your text in \href{/docs/reference/layout/box/}{boxes}
manually or through a \href{/docs/reference/styling/\#show-rules}{show
rule} .

\begin{verbatim}
#set text(fill: gradient.linear(red, blue))
#let rainbow(content) = {
  set text(fill: gradient.linear(..color.map.rainbow))
  box(content)
}

This is a gradient on text, but with a #rainbow[twist]!
\end{verbatim}

\includegraphics[width=4.85417in,height=\textheight,keepaspectratio]{/assets/docs/ch0LALUCwuQoVDnxrE_UZwAAAAAAAAAA.png}

\subsection{Stops}\label{stops}

A gradient is composed of a series of stops. Each of these stops has a
color and an offset. The offset is a
\href{/docs/reference/layout/ratio/}{ratio} between
\texttt{\ }{\texttt{\ 0\%\ }}\texttt{\ } and
\texttt{\ }{\texttt{\ 100\%\ }}\texttt{\ } or an angle between
\texttt{\ }{\texttt{\ 0deg\ }}\texttt{\ } and
\texttt{\ }{\texttt{\ 360deg\ }}\texttt{\ } . The offset is a relative
position that determines how far along the gradient the stop is located.
The stop\textquotesingle s color is the color of the gradient at that
position. You can choose to omit the offsets when defining a gradient.
In this case, Typst will space all stops evenly.

\subsection{Relativeness}\label{relativeness}

The location of the \texttt{\ }{\texttt{\ 0\%\ }}\texttt{\ } and
\texttt{\ }{\texttt{\ 100\%\ }}\texttt{\ } stops depends on the
dimensions of a container. This container can either be the shape that
it is being painted on, or the closest surrounding container. This is
controlled by the \texttt{\ relative\ } argument of a gradient
constructor. By default, gradients are relative to the shape they are
being painted on, unless the gradient is applied on text, in which case
they are relative to the closest ancestor container.

Typst determines the ancestor container as follows:

\begin{itemize}
\tightlist
\item
  For shapes that are placed at the root/top level of the document, the
  closest ancestor is the page itself.
\item
  For other shapes, the ancestor is the innermost
  \href{/docs/reference/layout/block/}{\texttt{\ block\ }} or
  \href{/docs/reference/layout/box/}{\texttt{\ box\ }} that contains the
  shape. This includes the boxes and blocks that are implicitly created
  by show rules and elements. For example, a
  \href{/docs/reference/layout/rotate/}{\texttt{\ rotate\ }} will not
  affect the parent of a gradient, but a
  \href{/docs/reference/layout/grid/}{\texttt{\ grid\ }} will.
\end{itemize}

\subsection{Color spaces and
interpolation}\label{color-spaces-and-interpolation}

Gradients can be interpolated in any color space. By default, gradients
are interpolated in the
\href{/docs/reference/visualize/color/\#definitions-oklab}{Oklab} color
space, which is a
\href{https://programmingdesignsystems.com/color/perceptually-uniform-color-spaces/index.html}{perceptually
uniform} color space. This means that the gradient will be perceived as
having a smooth progression of colors. This is particularly useful for
data visualization.

However, you can choose to interpolate the gradient in any supported
color space you want, but beware that some color spaces are not suitable
for perceptually interpolating between colors. Consult the table below
when choosing an interpolation space.

\begin{longtable}[]{@{}ll@{}}
\toprule\noalign{}
Color space & Perceptually uniform? \\
\midrule\noalign{}
\endhead
\bottomrule\noalign{}
\endlastfoot
\href{/docs/reference/visualize/color/\#definitions-oklab}{Oklab} &
\emph{Yes} \\
\href{/docs/reference/visualize/color/\#definitions-oklch}{Oklch} &
\emph{Yes} \\
\href{/docs/reference/visualize/color/\#definitions-rgb}{sRGB} &
\emph{No} \\
\href{/docs/reference/visualize/color/\#definitions-linear-rgb}{linear-RGB}
& \emph{Yes} \\
\href{/docs/reference/visualize/color/\#definitions-cmyk}{CMYK} &
\emph{No} \\
\href{/docs/reference/visualize/color/\#definitions-luma}{Grayscale} &
\emph{Yes} \\
\href{/docs/reference/visualize/color/\#definitions-hsl}{HSL} &
\emph{No} \\
\href{/docs/reference/visualize/color/\#definitions-hsv}{HSV} &
\emph{No} \\
\end{longtable}

\includegraphics[width=5in,height=\textheight,keepaspectratio]{/assets/docs/hDyhl3_sixunf7X8Ctx-hAAAAAAAAAAA.png}

\subsection{Direction}\label{direction}

Some gradients are sensitive to direction. For example, a linear
gradient has an angle that determines its direction. Typst uses a
clockwise angle, with 0° being from left to right, 90° from top to
bottom, 180° from right to left, and 270° from bottom to top.

\begin{verbatim}
#stack(
  dir: ltr,
  spacing: 1fr,
  square(fill: gradient.linear(red, blue, angle: 0deg)),
  square(fill: gradient.linear(red, blue, angle: 90deg)),
  square(fill: gradient.linear(red, blue, angle: 180deg)),
  square(fill: gradient.linear(red, blue, angle: 270deg)),
)
\end{verbatim}

\includegraphics[width=5in,height=\textheight,keepaspectratio]{/assets/docs/cXgxeaTP2ci7NL16a3rB_gAAAAAAAAAA.png}

\subsection{Presets}\label{presets}

Typst predefines color maps that you can use with your gradients. See
the
\href{/docs/reference/visualize/color/\#predefined-color-maps}{\texttt{\ color\ }}
documentation for more details.

\subsection{Note on file sizes}\label{note-on-file-sizes}

Gradients can be quite large, especially if they have many stops. This
is because gradients are stored as a list of colors and offsets, which
can take up a lot of space. If you are concerned about file sizes, you
should consider the following:

\begin{itemize}
\tightlist
\item
  SVG gradients are currently inefficiently encoded. This will be
  improved in the future.
\item
  PDF gradients in the
  \href{/docs/reference/visualize/color/\#definitions-oklab}{\texttt{\ color.oklab\ }}
  ,
  \href{/docs/reference/visualize/color/\#definitions-hsv}{\texttt{\ color.hsv\ }}
  ,
  \href{/docs/reference/visualize/color/\#definitions-hsl}{\texttt{\ color.hsl\ }}
  , and
  \href{/docs/reference/visualize/color/\#definitions-oklch}{\texttt{\ color.oklch\ }}
  color spaces are stored as a list of
  \href{/docs/reference/visualize/color/\#definitions-rgb}{\texttt{\ color.rgb\ }}
  colors with extra stops in between. This avoids needing to encode
  these color spaces in your PDF file, but it does add extra stops to
  your gradient, which can increase the file size.
\end{itemize}

\subsection{\texorpdfstring{{ Definitions
}}{ Definitions }}\label{definitions}

\phantomsection\label{definitions-tooltip}
Functions and types and can have associated definitions. These are
accessed by specifying the function or type, followed by a period, and
then the definition\textquotesingle s name.

\subsubsection{\texorpdfstring{\texttt{\ linear\ }}{ linear }}\label{definitions-linear}

Creates a new linear gradient, in which colors transition along a
straight line.

gradient { . } { linear } (

{ \hyperref[definitions-linear-parameters-stops]{..}
\href{/docs/reference/visualize/color/}{color}
\href{/docs/reference/foundations/array/}{array} , } {
\hyperref[definitions-linear-parameters-space]{space :} { any } , } {
\hyperref[definitions-linear-parameters-relative]{relative :}
\href{/docs/reference/foundations/auto/}{auto}
\href{/docs/reference/foundations/str/}{str} , } {
\href{/docs/reference/layout/direction/}{direction} , } {
\href{/docs/reference/layout/angle/}{angle} , }

) -\textgreater{} \href{/docs/reference/visualize/gradient/}{gradient}

\begin{verbatim}
#rect(
  width: 100%,
  height: 20pt,
  fill: gradient.linear(
    ..color.map.viridis,
  ),
)
\end{verbatim}

\includegraphics[width=5in,height=\textheight,keepaspectratio]{/assets/docs/3vCVaADmPcUqYOLma4-wcgAAAAAAAAAA.png}

\paragraph{\texorpdfstring{\texttt{\ stops\ }}{ stops }}\label{definitions-linear-stops}

\href{/docs/reference/visualize/color/}{color} {or}
\href{/docs/reference/foundations/array/}{array}

{Required} {{ Positional }}

\phantomsection\label{definitions-linear-stops-positional-tooltip}
Positional parameters are specified in order, without names.

{{ Variadic }}

\phantomsection\label{definitions-linear-stops-variadic-tooltip}
Variadic parameters can be specified multiple times.

The color \hyperref[stops]{stops} of the gradient.

\paragraph{\texorpdfstring{\texttt{\ space\ }}{ space }}\label{definitions-linear-space}

{ any }

The color space in which to interpolate the gradient.

Defaults to a perceptually uniform color space called
\href{/docs/reference/visualize/color/\#definitions-oklab}{Oklab} .

Default: \texttt{\ oklab\ }

\paragraph{\texorpdfstring{\texttt{\ relative\ }}{ relative }}\label{definitions-linear-relative}

\href{/docs/reference/foundations/auto/}{auto} {or}
\href{/docs/reference/foundations/str/}{str}

The \hyperref[relativeness]{relative placement} of the gradient.

For an element placed at the root/top level of the document, the parent
is the page itself. For other elements, the parent is the innermost
block, box, column, grid, or stack that contains the element.

\begin{longtable}[]{@{}ll@{}}
\toprule\noalign{}
Variant & Details \\
\midrule\noalign{}
\endhead
\bottomrule\noalign{}
\endlastfoot
\texttt{\ "\ self\ "\ } & The gradient is relative to itself (its own
bounding box). \\
\texttt{\ "\ parent\ "\ } & The gradient is relative to its parent (the
parent\textquotesingle s bounding box). \\
\end{longtable}

Default: \texttt{\ }{\texttt{\ auto\ }}\texttt{\ }

\paragraph{\texorpdfstring{\texttt{\ dir\ }}{ dir }}\label{definitions-linear-dir}

\href{/docs/reference/layout/direction/}{direction}

{{ Positional }}

\phantomsection\label{definitions-linear-dir-positional-tooltip}
Positional parameters are specified in order, without names.

The direction of the gradient.

Default: \texttt{\ ltr\ }

\paragraph{\texorpdfstring{\texttt{\ angle\ }}{ angle }}\label{definitions-linear-angle}

\href{/docs/reference/layout/angle/}{angle}

{Required} {{ Positional }}

\phantomsection\label{definitions-linear-angle-positional-tooltip}
Positional parameters are specified in order, without names.

The angle of the gradient.

\subsubsection{\texorpdfstring{\texttt{\ radial\ }}{ radial }}\label{definitions-radial}

Creates a new radial gradient, in which colors radiate away from an
origin.

The gradient is defined by two circles: the focal circle and the end
circle. The focal circle is a circle with center
\texttt{\ focal-center\ } and radius \texttt{\ focal-radius\ } , that
defines the points at which the gradient starts and has the color of the
first stop. The end circle is a circle with center \texttt{\ center\ }
and radius \texttt{\ radius\ } , that defines the points at which the
gradient ends and has the color of the last stop. The gradient is then
interpolated between these two circles.

Using these four values, also called the focal point for the starting
circle and the center and radius for the end circle, we can define a
gradient with more interesting properties than a basic radial gradient.

gradient { . } { radial } (

{ \hyperref[definitions-radial-parameters-stops]{..}
\href{/docs/reference/visualize/color/}{color}
\href{/docs/reference/foundations/array/}{array} , } {
\hyperref[definitions-radial-parameters-space]{space :} { any } , } {
\hyperref[definitions-radial-parameters-relative]{relative :}
\href{/docs/reference/foundations/auto/}{auto}
\href{/docs/reference/foundations/str/}{str} , } {
\hyperref[definitions-radial-parameters-center]{center :}
\href{/docs/reference/foundations/array/}{array} , } {
\hyperref[definitions-radial-parameters-radius]{radius :}
\href{/docs/reference/layout/ratio/}{ratio} , } {
\hyperref[definitions-radial-parameters-focal-center]{focal-center :}
\href{/docs/reference/foundations/auto/}{auto}
\href{/docs/reference/foundations/array/}{array} , } {
\hyperref[definitions-radial-parameters-focal-radius]{focal-radius :}
\href{/docs/reference/layout/ratio/}{ratio} , }

) -\textgreater{} \href{/docs/reference/visualize/gradient/}{gradient}

\begin{verbatim}
#stack(
  dir: ltr,
  spacing: 1fr,
  circle(fill: gradient.radial(
    ..color.map.viridis,
  )),
  circle(fill: gradient.radial(
    ..color.map.viridis,
    focal-center: (10%, 40%),
    focal-radius: 5%,
  )),
)
\end{verbatim}

\includegraphics[width=5in,height=\textheight,keepaspectratio]{/assets/docs/IfkE7bIcLhrH24l0nk0sIQAAAAAAAAAA.png}

\paragraph{\texorpdfstring{\texttt{\ stops\ }}{ stops }}\label{definitions-radial-stops}

\href{/docs/reference/visualize/color/}{color} {or}
\href{/docs/reference/foundations/array/}{array}

{Required} {{ Positional }}

\phantomsection\label{definitions-radial-stops-positional-tooltip}
Positional parameters are specified in order, without names.

{{ Variadic }}

\phantomsection\label{definitions-radial-stops-variadic-tooltip}
Variadic parameters can be specified multiple times.

The color \hyperref[stops]{stops} of the gradient.

\paragraph{\texorpdfstring{\texttt{\ space\ }}{ space }}\label{definitions-radial-space}

{ any }

The color space in which to interpolate the gradient.

Defaults to a perceptually uniform color space called
\href{/docs/reference/visualize/color/\#definitions-oklab}{Oklab} .

Default: \texttt{\ oklab\ }

\paragraph{\texorpdfstring{\texttt{\ relative\ }}{ relative }}\label{definitions-radial-relative}

\href{/docs/reference/foundations/auto/}{auto} {or}
\href{/docs/reference/foundations/str/}{str}

The \hyperref[relativeness]{relative placement} of the gradient.

For an element placed at the root/top level of the document, the parent
is the page itself. For other elements, the parent is the innermost
block, box, column, grid, or stack that contains the element.

\begin{longtable}[]{@{}ll@{}}
\toprule\noalign{}
Variant & Details \\
\midrule\noalign{}
\endhead
\bottomrule\noalign{}
\endlastfoot
\texttt{\ "\ self\ "\ } & The gradient is relative to itself (its own
bounding box). \\
\texttt{\ "\ parent\ "\ } & The gradient is relative to its parent (the
parent\textquotesingle s bounding box). \\
\end{longtable}

Default: \texttt{\ }{\texttt{\ auto\ }}\texttt{\ }

\paragraph{\texorpdfstring{\texttt{\ center\ }}{ center }}\label{definitions-radial-center}

\href{/docs/reference/foundations/array/}{array}

The center of the end circle of the gradient.

A value of
\texttt{\ }{\texttt{\ (\ }}\texttt{\ }{\texttt{\ 50\%\ }}\texttt{\ }{\texttt{\ ,\ }}\texttt{\ }{\texttt{\ 50\%\ }}\texttt{\ }{\texttt{\ )\ }}\texttt{\ }
means that the end circle is centered inside of its container.

Default:
\texttt{\ }{\texttt{\ (\ }}\texttt{\ }{\texttt{\ 50\%\ }}\texttt{\ }{\texttt{\ ,\ }}\texttt{\ }{\texttt{\ 50\%\ }}\texttt{\ }{\texttt{\ )\ }}\texttt{\ }

\paragraph{\texorpdfstring{\texttt{\ radius\ }}{ radius }}\label{definitions-radial-radius}

\href{/docs/reference/layout/ratio/}{ratio}

The radius of the end circle of the gradient.

By default, it is set to \texttt{\ }{\texttt{\ 50\%\ }}\texttt{\ } . The
ending radius must be bigger than the focal radius.

Default: \texttt{\ }{\texttt{\ 50\%\ }}\texttt{\ }

\paragraph{\texorpdfstring{\texttt{\ focal-center\ }}{ focal-center }}\label{definitions-radial-focal-center}

\href{/docs/reference/foundations/auto/}{auto} {or}
\href{/docs/reference/foundations/array/}{array}

The center of the focal circle of the gradient.

The focal center must be inside of the end circle.

A value of
\texttt{\ }{\texttt{\ (\ }}\texttt{\ }{\texttt{\ 50\%\ }}\texttt{\ }{\texttt{\ ,\ }}\texttt{\ }{\texttt{\ 50\%\ }}\texttt{\ }{\texttt{\ )\ }}\texttt{\ }
means that the focal circle is centered inside of its container.

By default it is set to the same as the center of the last circle.

Default: \texttt{\ }{\texttt{\ auto\ }}\texttt{\ }

\paragraph{\texorpdfstring{\texttt{\ focal-radius\ }}{ focal-radius }}\label{definitions-radial-focal-radius}

\href{/docs/reference/layout/ratio/}{ratio}

The radius of the focal circle of the gradient.

The focal center must be inside of the end circle.

By default, it is set to \texttt{\ }{\texttt{\ 0\%\ }}\texttt{\ } . The
focal radius must be smaller than the ending radius`.

Default: \texttt{\ }{\texttt{\ 0\%\ }}\texttt{\ }

\subsubsection{\texorpdfstring{\texttt{\ conic\ }}{ conic }}\label{definitions-conic}

Creates a new conic gradient, in which colors change radially around a
center point.

You can control the center point of the gradient by using the
\texttt{\ center\ } argument. By default, the center point is the center
of the shape.

gradient { . } { conic } (

{ \hyperref[definitions-conic-parameters-stops]{..}
\href{/docs/reference/visualize/color/}{color}
\href{/docs/reference/foundations/array/}{array} , } {
\hyperref[definitions-conic-parameters-angle]{angle :}
\href{/docs/reference/layout/angle/}{angle} , } {
\hyperref[definitions-conic-parameters-space]{space :} { any } , } {
\hyperref[definitions-conic-parameters-relative]{relative :}
\href{/docs/reference/foundations/auto/}{auto}
\href{/docs/reference/foundations/str/}{str} , } {
\hyperref[definitions-conic-parameters-center]{center :}
\href{/docs/reference/foundations/array/}{array} , }

) -\textgreater{} \href{/docs/reference/visualize/gradient/}{gradient}

\begin{verbatim}
#stack(
  dir: ltr,
  spacing: 1fr,
  circle(fill: gradient.conic(
    ..color.map.viridis,
  )),
  circle(fill: gradient.conic(
    ..color.map.viridis,
    center: (20%, 30%),
  )),
)
\end{verbatim}

\includegraphics[width=5in,height=\textheight,keepaspectratio]{/assets/docs/Mqmcewscuekk2Rsln7oKygAAAAAAAAAA.png}

\paragraph{\texorpdfstring{\texttt{\ stops\ }}{ stops }}\label{definitions-conic-stops}

\href{/docs/reference/visualize/color/}{color} {or}
\href{/docs/reference/foundations/array/}{array}

{Required} {{ Positional }}

\phantomsection\label{definitions-conic-stops-positional-tooltip}
Positional parameters are specified in order, without names.

{{ Variadic }}

\phantomsection\label{definitions-conic-stops-variadic-tooltip}
Variadic parameters can be specified multiple times.

The color \hyperref[stops]{stops} of the gradient.

\paragraph{\texorpdfstring{\texttt{\ angle\ }}{ angle }}\label{definitions-conic-angle}

\href{/docs/reference/layout/angle/}{angle}

The angle of the gradient.

Default: \texttt{\ }{\texttt{\ 0deg\ }}\texttt{\ }

\paragraph{\texorpdfstring{\texttt{\ space\ }}{ space }}\label{definitions-conic-space}

{ any }

The color space in which to interpolate the gradient.

Defaults to a perceptually uniform color space called
\href{/docs/reference/visualize/color/\#definitions-oklab}{Oklab} .

Default: \texttt{\ oklab\ }

\paragraph{\texorpdfstring{\texttt{\ relative\ }}{ relative }}\label{definitions-conic-relative}

\href{/docs/reference/foundations/auto/}{auto} {or}
\href{/docs/reference/foundations/str/}{str}

The \hyperref[relativeness]{relative placement} of the gradient.

For an element placed at the root/top level of the document, the parent
is the page itself. For other elements, the parent is the innermost
block, box, column, grid, or stack that contains the element.

\begin{longtable}[]{@{}ll@{}}
\toprule\noalign{}
Variant & Details \\
\midrule\noalign{}
\endhead
\bottomrule\noalign{}
\endlastfoot
\texttt{\ "\ self\ "\ } & The gradient is relative to itself (its own
bounding box). \\
\texttt{\ "\ parent\ "\ } & The gradient is relative to its parent (the
parent\textquotesingle s bounding box). \\
\end{longtable}

Default: \texttt{\ }{\texttt{\ auto\ }}\texttt{\ }

\paragraph{\texorpdfstring{\texttt{\ center\ }}{ center }}\label{definitions-conic-center}

\href{/docs/reference/foundations/array/}{array}

The center of the last circle of the gradient.

A value of
\texttt{\ }{\texttt{\ (\ }}\texttt{\ }{\texttt{\ 50\%\ }}\texttt{\ }{\texttt{\ ,\ }}\texttt{\ }{\texttt{\ 50\%\ }}\texttt{\ }{\texttt{\ )\ }}\texttt{\ }
means that the end circle is centered inside of its container.

Default:
\texttt{\ }{\texttt{\ (\ }}\texttt{\ }{\texttt{\ 50\%\ }}\texttt{\ }{\texttt{\ ,\ }}\texttt{\ }{\texttt{\ 50\%\ }}\texttt{\ }{\texttt{\ )\ }}\texttt{\ }

\subsubsection{\texorpdfstring{\texttt{\ sharp\ }}{ sharp }}\label{definitions-sharp}

Creates a sharp version of this gradient.

Sharp gradients have discrete jumps between colors, instead of a smooth
transition. They are particularly useful for creating color lists for a
preset gradient.

self { . } { sharp } (

{ \href{/docs/reference/foundations/int/}{int} , } {
\hyperref[definitions-sharp-parameters-smoothness]{smoothness :}
\href{/docs/reference/layout/ratio/}{ratio} , }

) -\textgreater{} \href{/docs/reference/visualize/gradient/}{gradient}

\begin{verbatim}
#set rect(width: 100%, height: 20pt)
#let grad = gradient.linear(..color.map.rainbow)
#rect(fill: grad)
#rect(fill: grad.sharp(5))
#rect(fill: grad.sharp(5, smoothness: 20%))
\end{verbatim}

\includegraphics[width=5in,height=\textheight,keepaspectratio]{/assets/docs/k1IrJvVHW9DTjXfwHdfh_QAAAAAAAAAA.png}

\paragraph{\texorpdfstring{\texttt{\ steps\ }}{ steps }}\label{definitions-sharp-steps}

\href{/docs/reference/foundations/int/}{int}

{Required} {{ Positional }}

\phantomsection\label{definitions-sharp-steps-positional-tooltip}
Positional parameters are specified in order, without names.

The number of stops in the gradient.

\paragraph{\texorpdfstring{\texttt{\ smoothness\ }}{ smoothness }}\label{definitions-sharp-smoothness}

\href{/docs/reference/layout/ratio/}{ratio}

How much to smooth the gradient.

Default: \texttt{\ }{\texttt{\ 0\%\ }}\texttt{\ }

\subsubsection{\texorpdfstring{\texttt{\ repeat\ }}{ repeat }}\label{definitions-repeat}

Repeats this gradient a given number of times, optionally mirroring it
at each repetition.

self { . } { repeat } (

{ \href{/docs/reference/foundations/int/}{int} , } {
\hyperref[definitions-repeat-parameters-mirror]{mirror :}
\href{/docs/reference/foundations/bool/}{bool} , }

) -\textgreater{} \href{/docs/reference/visualize/gradient/}{gradient}

\begin{verbatim}
#circle(
  radius: 40pt,
  fill: gradient
    .radial(aqua, white)
    .repeat(4),
)
\end{verbatim}

\includegraphics[width=5in,height=\textheight,keepaspectratio]{/assets/docs/ydbGAMwgwvGMCJpfMs1wAAAAAAAAAAAA.png}

\paragraph{\texorpdfstring{\texttt{\ repetitions\ }}{ repetitions }}\label{definitions-repeat-repetitions}

\href{/docs/reference/foundations/int/}{int}

{Required} {{ Positional }}

\phantomsection\label{definitions-repeat-repetitions-positional-tooltip}
Positional parameters are specified in order, without names.

The number of times to repeat the gradient.

\paragraph{\texorpdfstring{\texttt{\ mirror\ }}{ mirror }}\label{definitions-repeat-mirror}

\href{/docs/reference/foundations/bool/}{bool}

Whether to mirror the gradient at each repetition.

Default: \texttt{\ }{\texttt{\ false\ }}\texttt{\ }

\subsubsection{\texorpdfstring{\texttt{\ kind\ }}{ kind }}\label{definitions-kind}

Returns the kind of this gradient.

self { . } { kind } (

) -\textgreater{} \href{/docs/reference/foundations/function/}{function}

\subsubsection{\texorpdfstring{\texttt{\ stops\ }}{ stops }}\label{definitions-stops}

Returns the stops of this gradient.

self { . } { stops } (

) -\textgreater{} \href{/docs/reference/foundations/array/}{array}

\subsubsection{\texorpdfstring{\texttt{\ space\ }}{ space }}\label{definitions-space}

Returns the mixing space of this gradient.

self { . } { space } (

) -\textgreater{} { any }

\subsubsection{\texorpdfstring{\texttt{\ relative\ }}{ relative }}\label{definitions-relative}

Returns the relative placement of this gradient.

self { . } { relative } (

) -\textgreater{} \href{/docs/reference/foundations/auto/}{auto}

\subsubsection{\texorpdfstring{\texttt{\ angle\ }}{ angle }}\label{definitions-angle}

Returns the angle of this gradient.

self { . } { angle } (

) -\textgreater{} \href{/docs/reference/foundations/none/}{none}
\href{/docs/reference/layout/angle/}{angle}

\subsubsection{\texorpdfstring{\texttt{\ sample\ }}{ sample }}\label{definitions-sample}

Sample the gradient at a given position.

The position is either a position along the gradient (a
\href{/docs/reference/layout/ratio/}{ratio} between
\texttt{\ }{\texttt{\ 0\%\ }}\texttt{\ } and
\texttt{\ }{\texttt{\ 100\%\ }}\texttt{\ } ) or an
\href{/docs/reference/layout/angle/}{angle} . Any value outside of this
range will be clamped.

self { . } { sample } (

{ \href{/docs/reference/layout/angle/}{angle}
\href{/docs/reference/layout/ratio/}{ratio} }

) -\textgreater{} \href{/docs/reference/visualize/color/}{color}

\paragraph{\texorpdfstring{\texttt{\ t\ }}{ t }}\label{definitions-sample-t}

\href{/docs/reference/layout/angle/}{angle} {or}
\href{/docs/reference/layout/ratio/}{ratio}

{Required} {{ Positional }}

\phantomsection\label{definitions-sample-t-positional-tooltip}
Positional parameters are specified in order, without names.

The position at which to sample the gradient.

\subsubsection{\texorpdfstring{\texttt{\ samples\ }}{ samples }}\label{definitions-samples}

Samples the gradient at multiple positions at once and returns the
results as an array.

self { . } { samples } (

{ \hyperref[definitions-samples-parameters-ts]{..}
\href{/docs/reference/layout/angle/}{angle}
\href{/docs/reference/layout/ratio/}{ratio} }

) -\textgreater{} \href{/docs/reference/foundations/array/}{array}

\paragraph{\texorpdfstring{\texttt{\ ts\ }}{ ts }}\label{definitions-samples-ts}

\href{/docs/reference/layout/angle/}{angle} {or}
\href{/docs/reference/layout/ratio/}{ratio}

{Required} {{ Positional }}

\phantomsection\label{definitions-samples-ts-positional-tooltip}
Positional parameters are specified in order, without names.

{{ Variadic }}

\phantomsection\label{definitions-samples-ts-variadic-tooltip}
Variadic parameters can be specified multiple times.

The positions at which to sample the gradient.

\href{/docs/reference/visualize/ellipse/}{\pandocbounded{\includesvg[keepaspectratio]{/assets/icons/16-arrow-right.svg}}}

{ Ellipse } { Previous page }

\href{/docs/reference/visualize/image/}{\pandocbounded{\includesvg[keepaspectratio]{/assets/icons/16-arrow-right.svg}}}

{ Image } { Next page }


\section{Docs LaTeX/typst.app/docs/reference/visualize/square.tex}
\title{typst.app/docs/reference/visualize/square}

\begin{itemize}
\tightlist
\item
  \href{/docs}{\includesvg[width=0.16667in,height=0.16667in]{/assets/icons/16-docs-dark.svg}}
\item
  \includesvg[width=0.16667in,height=0.16667in]{/assets/icons/16-arrow-right.svg}
\item
  \href{/docs/reference/}{Reference}
\item
  \includesvg[width=0.16667in,height=0.16667in]{/assets/icons/16-arrow-right.svg}
\item
  \href{/docs/reference/visualize/}{Visualize}
\item
  \includesvg[width=0.16667in,height=0.16667in]{/assets/icons/16-arrow-right.svg}
\item
  \href{/docs/reference/visualize/square/}{Square}
\end{itemize}

\section{\texorpdfstring{\texttt{\ square\ } {{ Element
}}}{ square   Element }}\label{summary}

\phantomsection\label{element-tooltip}
Element functions can be customized with \texttt{\ set\ } and
\texttt{\ show\ } rules.

A square with optional content.

\subsection{Example}\label{example}

\begin{verbatim}
// Without content.
#square(size: 40pt)

// With content.
#square[
  Automatically \
  sized to fit.
]
\end{verbatim}

\includegraphics[width=5in,height=\textheight,keepaspectratio]{/assets/docs/DjWoCmaGrn_miIIjOqjv7gAAAAAAAAAA.png}

\subsection{\texorpdfstring{{ Parameters
}}{ Parameters }}\label{parameters}

\phantomsection\label{parameters-tooltip}
Parameters are the inputs to a function. They are specified in
parentheses after the function name.

{ square } (

{ \hyperref[parameters-size]{size :}
\href{/docs/reference/foundations/auto/}{auto}
\href{/docs/reference/layout/length/}{length} , } {
\hyperref[parameters-width]{width :}
\href{/docs/reference/foundations/auto/}{auto}
\href{/docs/reference/layout/relative/}{relative} , } {
\hyperref[parameters-height]{height :}
\href{/docs/reference/foundations/auto/}{auto}
\href{/docs/reference/layout/relative/}{relative}
\href{/docs/reference/layout/fraction/}{fraction} , } {
\hyperref[parameters-fill]{fill :}
\href{/docs/reference/foundations/none/}{none}
\href{/docs/reference/visualize/color/}{color}
\href{/docs/reference/visualize/gradient/}{gradient}
\href{/docs/reference/visualize/pattern/}{pattern} , } {
\hyperref[parameters-stroke]{stroke :}
\href{/docs/reference/foundations/none/}{none}
\href{/docs/reference/foundations/auto/}{auto}
\href{/docs/reference/layout/length/}{length}
\href{/docs/reference/visualize/color/}{color}
\href{/docs/reference/visualize/gradient/}{gradient}
\href{/docs/reference/visualize/stroke/}{stroke}
\href{/docs/reference/visualize/pattern/}{pattern}
\href{/docs/reference/foundations/dictionary/}{dictionary} , } {
\hyperref[parameters-radius]{radius :}
\href{/docs/reference/layout/relative/}{relative}
\href{/docs/reference/foundations/dictionary/}{dictionary} , } {
\hyperref[parameters-inset]{inset :}
\href{/docs/reference/layout/relative/}{relative}
\href{/docs/reference/foundations/dictionary/}{dictionary} , } {
\hyperref[parameters-outset]{outset :}
\href{/docs/reference/layout/relative/}{relative}
\href{/docs/reference/foundations/dictionary/}{dictionary} , } {
\hyperref[parameters-body]{}
\href{/docs/reference/foundations/none/}{none}
\href{/docs/reference/foundations/content/}{content} , }

) -\textgreater{} \href{/docs/reference/foundations/content/}{content}

\subsubsection{\texorpdfstring{\texttt{\ size\ }}{ size }}\label{parameters-size}

\href{/docs/reference/foundations/auto/}{auto} {or}
\href{/docs/reference/layout/length/}{length}

{{ Settable }}

\phantomsection\label{parameters-size-settable-tooltip}
Settable parameters can be customized for all following uses of the
function with a \texttt{\ set\ } rule.

The square\textquotesingle s side length. This is mutually exclusive
with \texttt{\ width\ } and \texttt{\ height\ } .

Default: \texttt{\ }{\texttt{\ auto\ }}\texttt{\ }

\subsubsection{\texorpdfstring{\texttt{\ width\ }}{ width }}\label{parameters-width}

\href{/docs/reference/foundations/auto/}{auto} {or}
\href{/docs/reference/layout/relative/}{relative}

{{ Settable }}

\phantomsection\label{parameters-width-settable-tooltip}
Settable parameters can be customized for all following uses of the
function with a \texttt{\ set\ } rule.

The square\textquotesingle s width. This is mutually exclusive with
\texttt{\ size\ } and \texttt{\ height\ } .

In contrast to \texttt{\ size\ } , this can be relative to the parent
container\textquotesingle s width.

Default: \texttt{\ }{\texttt{\ auto\ }}\texttt{\ }

\subsubsection{\texorpdfstring{\texttt{\ height\ }}{ height }}\label{parameters-height}

\href{/docs/reference/foundations/auto/}{auto} {or}
\href{/docs/reference/layout/relative/}{relative} {or}
\href{/docs/reference/layout/fraction/}{fraction}

{{ Settable }}

\phantomsection\label{parameters-height-settable-tooltip}
Settable parameters can be customized for all following uses of the
function with a \texttt{\ set\ } rule.

The square\textquotesingle s height. This is mutually exclusive with
\texttt{\ size\ } and \texttt{\ width\ } .

In contrast to \texttt{\ size\ } , this can be relative to the parent
container\textquotesingle s height.

Default: \texttt{\ }{\texttt{\ auto\ }}\texttt{\ }

\subsubsection{\texorpdfstring{\texttt{\ fill\ }}{ fill }}\label{parameters-fill}

\href{/docs/reference/foundations/none/}{none} {or}
\href{/docs/reference/visualize/color/}{color} {or}
\href{/docs/reference/visualize/gradient/}{gradient} {or}
\href{/docs/reference/visualize/pattern/}{pattern}

{{ Settable }}

\phantomsection\label{parameters-fill-settable-tooltip}
Settable parameters can be customized for all following uses of the
function with a \texttt{\ set\ } rule.

How to fill the square. See the
\href{/docs/reference/visualize/rect/\#parameters-fill}{rectangle\textquotesingle s
documentation} for more details.

Default: \texttt{\ }{\texttt{\ none\ }}\texttt{\ }

\subsubsection{\texorpdfstring{\texttt{\ stroke\ }}{ stroke }}\label{parameters-stroke}

\href{/docs/reference/foundations/none/}{none} {or}
\href{/docs/reference/foundations/auto/}{auto} {or}
\href{/docs/reference/layout/length/}{length} {or}
\href{/docs/reference/visualize/color/}{color} {or}
\href{/docs/reference/visualize/gradient/}{gradient} {or}
\href{/docs/reference/visualize/stroke/}{stroke} {or}
\href{/docs/reference/visualize/pattern/}{pattern} {or}
\href{/docs/reference/foundations/dictionary/}{dictionary}

{{ Settable }}

\phantomsection\label{parameters-stroke-settable-tooltip}
Settable parameters can be customized for all following uses of the
function with a \texttt{\ set\ } rule.

How to stroke the square. See the
\href{/docs/reference/visualize/rect/\#parameters-stroke}{rectangle\textquotesingle s
documentation} for more details.

Default: \texttt{\ }{\texttt{\ auto\ }}\texttt{\ }

\subsubsection{\texorpdfstring{\texttt{\ radius\ }}{ radius }}\label{parameters-radius}

\href{/docs/reference/layout/relative/}{relative} {or}
\href{/docs/reference/foundations/dictionary/}{dictionary}

{{ Settable }}

\phantomsection\label{parameters-radius-settable-tooltip}
Settable parameters can be customized for all following uses of the
function with a \texttt{\ set\ } rule.

How much to round the square\textquotesingle s corners. See the
\href{/docs/reference/visualize/rect/\#parameters-radius}{rectangle\textquotesingle s
documentation} for more details.

Default:
\texttt{\ }{\texttt{\ (\ }}\texttt{\ }{\texttt{\ :\ }}\texttt{\ }{\texttt{\ )\ }}\texttt{\ }

\subsubsection{\texorpdfstring{\texttt{\ inset\ }}{ inset }}\label{parameters-inset}

\href{/docs/reference/layout/relative/}{relative} {or}
\href{/docs/reference/foundations/dictionary/}{dictionary}

{{ Settable }}

\phantomsection\label{parameters-inset-settable-tooltip}
Settable parameters can be customized for all following uses of the
function with a \texttt{\ set\ } rule.

How much to pad the square\textquotesingle s content. See the
\href{/docs/reference/layout/box/\#parameters-inset}{box\textquotesingle s
documentation} for more details.

Default:
\texttt{\ }{\texttt{\ 0\%\ }}\texttt{\ }{\texttt{\ +\ }}\texttt{\ }{\texttt{\ 5pt\ }}\texttt{\ }

\subsubsection{\texorpdfstring{\texttt{\ outset\ }}{ outset }}\label{parameters-outset}

\href{/docs/reference/layout/relative/}{relative} {or}
\href{/docs/reference/foundations/dictionary/}{dictionary}

{{ Settable }}

\phantomsection\label{parameters-outset-settable-tooltip}
Settable parameters can be customized for all following uses of the
function with a \texttt{\ set\ } rule.

How much to expand the square\textquotesingle s size without affecting
the layout. See the
\href{/docs/reference/layout/box/\#parameters-outset}{box\textquotesingle s
documentation} for more details.

Default:
\texttt{\ }{\texttt{\ (\ }}\texttt{\ }{\texttt{\ :\ }}\texttt{\ }{\texttt{\ )\ }}\texttt{\ }

\subsubsection{\texorpdfstring{\texttt{\ body\ }}{ body }}\label{parameters-body}

\href{/docs/reference/foundations/none/}{none} {or}
\href{/docs/reference/foundations/content/}{content}

{{ Positional }}

\phantomsection\label{parameters-body-positional-tooltip}
Positional parameters are specified in order, without names.

{{ Settable }}

\phantomsection\label{parameters-body-settable-tooltip}
Settable parameters can be customized for all following uses of the
function with a \texttt{\ set\ } rule.

The content to place into the square. The square expands to fit this
content, keeping the 1-1 aspect ratio.

When this is omitted, the square takes on a default size of at most
\texttt{\ }{\texttt{\ 30pt\ }}\texttt{\ } .

Default: \texttt{\ }{\texttt{\ none\ }}\texttt{\ }

\href{/docs/reference/visualize/rect/}{\pandocbounded{\includesvg[keepaspectratio]{/assets/icons/16-arrow-right.svg}}}

{ Rectangle } { Previous page }

\href{/docs/reference/visualize/stroke/}{\pandocbounded{\includesvg[keepaspectratio]{/assets/icons/16-arrow-right.svg}}}

{ Stroke } { Next page }


\section{Docs LaTeX/typst.app/docs/reference/visualize/image.tex}
\title{typst.app/docs/reference/visualize/image}

\begin{itemize}
\tightlist
\item
  \href{/docs}{\includesvg[width=0.16667in,height=0.16667in]{/assets/icons/16-docs-dark.svg}}
\item
  \includesvg[width=0.16667in,height=0.16667in]{/assets/icons/16-arrow-right.svg}
\item
  \href{/docs/reference/}{Reference}
\item
  \includesvg[width=0.16667in,height=0.16667in]{/assets/icons/16-arrow-right.svg}
\item
  \href{/docs/reference/visualize/}{Visualize}
\item
  \includesvg[width=0.16667in,height=0.16667in]{/assets/icons/16-arrow-right.svg}
\item
  \href{/docs/reference/visualize/image/}{Image}
\end{itemize}

\section{\texorpdfstring{\texttt{\ image\ } {{ Element
}}}{ image   Element }}\label{summary}

\phantomsection\label{element-tooltip}
Element functions can be customized with \texttt{\ set\ } and
\texttt{\ show\ } rules.

A raster or vector graphic.

You can wrap the image in a
\href{/docs/reference/model/figure/}{\texttt{\ figure\ }} to give it a
number and caption.

Like most elements, images are \emph{block-level} by default and thus do
not integrate themselves into adjacent paragraphs. To force an image to
become inline, put it into a
\href{/docs/reference/layout/box/}{\texttt{\ box\ }} .

\subsection{Example}\label{example}

\begin{verbatim}
#figure(
  image("molecular.jpg", width: 80%),
  caption: [
    A step in the molecular testing
    pipeline of our lab.
  ],
)
\end{verbatim}

\includegraphics[width=5in,height=\textheight,keepaspectratio]{/assets/docs/znWnPh4HT5GrpkEcbnfOxAAAAAAAAAAA.png}

\subsection{\texorpdfstring{{ Parameters
}}{ Parameters }}\label{parameters}

\phantomsection\label{parameters-tooltip}
Parameters are the inputs to a function. They are specified in
parentheses after the function name.

{ image } (

{ \href{/docs/reference/foundations/str/}{str} , } {
\hyperref[parameters-format]{format :}
\href{/docs/reference/foundations/auto/}{auto}
\href{/docs/reference/foundations/str/}{str} , } {
\hyperref[parameters-width]{width :}
\href{/docs/reference/foundations/auto/}{auto}
\href{/docs/reference/layout/relative/}{relative} , } {
\hyperref[parameters-height]{height :}
\href{/docs/reference/foundations/auto/}{auto}
\href{/docs/reference/layout/relative/}{relative}
\href{/docs/reference/layout/fraction/}{fraction} , } {
\hyperref[parameters-alt]{alt :}
\href{/docs/reference/foundations/none/}{none}
\href{/docs/reference/foundations/str/}{str} , } {
\hyperref[parameters-fit]{fit :}
\href{/docs/reference/foundations/str/}{str} , }

) -\textgreater{} \href{/docs/reference/foundations/content/}{content}

\subsubsection{\texorpdfstring{\texttt{\ path\ }}{ path }}\label{parameters-path}

\href{/docs/reference/foundations/str/}{str}

{Required} {{ Positional }}

\phantomsection\label{parameters-path-positional-tooltip}
Positional parameters are specified in order, without names.

Path to an image file

For more details, see the \href{/docs/reference/syntax/\#paths}{Paths
section} .

\subsubsection{\texorpdfstring{\texttt{\ format\ }}{ format }}\label{parameters-format}

\href{/docs/reference/foundations/auto/}{auto} {or}
\href{/docs/reference/foundations/str/}{str}

{{ Settable }}

\phantomsection\label{parameters-format-settable-tooltip}
Settable parameters can be customized for all following uses of the
function with a \texttt{\ set\ } rule.

The image\textquotesingle s format. Detected automatically by default.

Supported formats are PNG, JPEG, GIF, and SVG. Using a PDF as an image
is \href{https://github.com/typst/typst/issues/145}{not currently
supported} .

\begin{longtable}[]{@{}ll@{}}
\toprule\noalign{}
Variant & Details \\
\midrule\noalign{}
\endhead
\bottomrule\noalign{}
\endlastfoot
\texttt{\ "\ png\ "\ } & Raster format for illustrations and transparent
graphics. \\
\texttt{\ "\ jpg\ "\ } & Lossy raster format suitable for photos. \\
\texttt{\ "\ gif\ "\ } & Raster format that is typically used for short
animated clips. \\
\texttt{\ "\ svg\ "\ } & The vector graphics format of the web. \\
\end{longtable}

Default: \texttt{\ }{\texttt{\ auto\ }}\texttt{\ }

\subsubsection{\texorpdfstring{\texttt{\ width\ }}{ width }}\label{parameters-width}

\href{/docs/reference/foundations/auto/}{auto} {or}
\href{/docs/reference/layout/relative/}{relative}

{{ Settable }}

\phantomsection\label{parameters-width-settable-tooltip}
Settable parameters can be customized for all following uses of the
function with a \texttt{\ set\ } rule.

The width of the image.

Default: \texttt{\ }{\texttt{\ auto\ }}\texttt{\ }

\subsubsection{\texorpdfstring{\texttt{\ height\ }}{ height }}\label{parameters-height}

\href{/docs/reference/foundations/auto/}{auto} {or}
\href{/docs/reference/layout/relative/}{relative} {or}
\href{/docs/reference/layout/fraction/}{fraction}

{{ Settable }}

\phantomsection\label{parameters-height-settable-tooltip}
Settable parameters can be customized for all following uses of the
function with a \texttt{\ set\ } rule.

The height of the image.

Default: \texttt{\ }{\texttt{\ auto\ }}\texttt{\ }

\subsubsection{\texorpdfstring{\texttt{\ alt\ }}{ alt }}\label{parameters-alt}

\href{/docs/reference/foundations/none/}{none} {or}
\href{/docs/reference/foundations/str/}{str}

{{ Settable }}

\phantomsection\label{parameters-alt-settable-tooltip}
Settable parameters can be customized for all following uses of the
function with a \texttt{\ set\ } rule.

A text describing the image.

Default: \texttt{\ }{\texttt{\ none\ }}\texttt{\ }

\subsubsection{\texorpdfstring{\texttt{\ fit\ }}{ fit }}\label{parameters-fit}

\href{/docs/reference/foundations/str/}{str}

{{ Settable }}

\phantomsection\label{parameters-fit-settable-tooltip}
Settable parameters can be customized for all following uses of the
function with a \texttt{\ set\ } rule.

How the image should adjust itself to a given area (the area is defined
by the \texttt{\ width\ } and \texttt{\ height\ } fields). Note that
\texttt{\ fit\ } doesn\textquotesingle t visually change anything if the
area\textquotesingle s aspect ratio is the same as the
image\textquotesingle s one.

\begin{longtable}[]{@{}ll@{}}
\toprule\noalign{}
Variant & Details \\
\midrule\noalign{}
\endhead
\bottomrule\noalign{}
\endlastfoot
\texttt{\ "\ cover\ "\ } & The image should completely cover the area
(preserves aspect ratio by cropping the image only horizontally or
vertically). This is the default. \\
\texttt{\ "\ contain\ "\ } & The image should be fully contained in the
area (preserves aspect ratio; doesn\textquotesingle t crop the image;
one dimension can be narrower than specified). \\
\texttt{\ "\ stretch\ "\ } & The image should be stretched so that it
exactly fills the area, even if this means that the image will be
distorted (doesn\textquotesingle t preserve aspect ratio and
doesn\textquotesingle t crop the image). \\
\end{longtable}

Default: \texttt{\ }{\texttt{\ "cover"\ }}\texttt{\ }

\includesvg[width=0.16667in,height=0.16667in]{/assets/icons/16-arrow-right.svg}
View example

\begin{verbatim}
#set page(width: 300pt, height: 50pt, margin: 10pt)
#image("tiger.jpg", width: 100%, fit: "cover")
#image("tiger.jpg", width: 100%, fit: "contain")
#image("tiger.jpg", width: 100%, fit: "stretch")
\end{verbatim}

\includegraphics[width=6.25in,height=\textheight,keepaspectratio]{/assets/docs/oZRwamqZZ0p_tV8oioYxxgAAAAAAAAAA.png}
\includegraphics[width=6.25in,height=\textheight,keepaspectratio]{/assets/docs/oZRwamqZZ0p_tV8oioYxxgAAAAAAAAAB.png}
\includegraphics[width=6.25in,height=\textheight,keepaspectratio]{/assets/docs/oZRwamqZZ0p_tV8oioYxxgAAAAAAAAAC.png}

\subsection{\texorpdfstring{{ Definitions
}}{ Definitions }}\label{definitions}

\phantomsection\label{definitions-tooltip}
Functions and types and can have associated definitions. These are
accessed by specifying the function or type, followed by a period, and
then the definition\textquotesingle s name.

\subsubsection{\texorpdfstring{\texttt{\ decode\ }}{ decode }}\label{definitions-decode}

Decode a raster or vector graphic from bytes or a string.

image { . } { decode } (

{ \href{/docs/reference/foundations/str/}{str}
\href{/docs/reference/foundations/bytes/}{bytes} , } {
\hyperref[definitions-decode-parameters-format]{format :}
\href{/docs/reference/foundations/auto/}{auto}
\href{/docs/reference/foundations/str/}{str} , } {
\hyperref[definitions-decode-parameters-width]{width :}
\href{/docs/reference/foundations/auto/}{auto}
\href{/docs/reference/layout/relative/}{relative} , } {
\hyperref[definitions-decode-parameters-height]{height :}
\href{/docs/reference/foundations/auto/}{auto}
\href{/docs/reference/layout/relative/}{relative}
\href{/docs/reference/layout/fraction/}{fraction} , } {
\hyperref[definitions-decode-parameters-alt]{alt :}
\href{/docs/reference/foundations/none/}{none}
\href{/docs/reference/foundations/str/}{str} , } {
\hyperref[definitions-decode-parameters-fit]{fit :}
\href{/docs/reference/foundations/str/}{str} , }

) -\textgreater{} \href{/docs/reference/foundations/content/}{content}

\begin{verbatim}
#let original = read("diagram.svg")
#let changed = original.replace(
  "#2B80FF", // blue
  green.to-hex(),
)

#image.decode(original)
#image.decode(changed)
\end{verbatim}

\includegraphics[width=5in,height=\textheight,keepaspectratio]{/assets/docs/yVFFVjYQ7xibSWu-658yNwAAAAAAAAAA.png}

\paragraph{\texorpdfstring{\texttt{\ data\ }}{ data }}\label{definitions-decode-data}

\href{/docs/reference/foundations/str/}{str} {or}
\href{/docs/reference/foundations/bytes/}{bytes}

{Required} {{ Positional }}

\phantomsection\label{definitions-decode-data-positional-tooltip}
Positional parameters are specified in order, without names.

The data to decode as an image. Can be a string for SVGs.

\paragraph{\texorpdfstring{\texttt{\ format\ }}{ format }}\label{definitions-decode-format}

\href{/docs/reference/foundations/auto/}{auto} {or}
\href{/docs/reference/foundations/str/}{str}

The image\textquotesingle s format. Detected automatically by default.

\begin{longtable}[]{@{}ll@{}}
\toprule\noalign{}
Variant & Details \\
\midrule\noalign{}
\endhead
\bottomrule\noalign{}
\endlastfoot
\texttt{\ "\ png\ "\ } & Raster format for illustrations and transparent
graphics. \\
\texttt{\ "\ jpg\ "\ } & Lossy raster format suitable for photos. \\
\texttt{\ "\ gif\ "\ } & Raster format that is typically used for short
animated clips. \\
\texttt{\ "\ svg\ "\ } & The vector graphics format of the web. \\
\end{longtable}

\paragraph{\texorpdfstring{\texttt{\ width\ }}{ width }}\label{definitions-decode-width}

\href{/docs/reference/foundations/auto/}{auto} {or}
\href{/docs/reference/layout/relative/}{relative}

The width of the image.

\paragraph{\texorpdfstring{\texttt{\ height\ }}{ height }}\label{definitions-decode-height}

\href{/docs/reference/foundations/auto/}{auto} {or}
\href{/docs/reference/layout/relative/}{relative} {or}
\href{/docs/reference/layout/fraction/}{fraction}

The height of the image.

\paragraph{\texorpdfstring{\texttt{\ alt\ }}{ alt }}\label{definitions-decode-alt}

\href{/docs/reference/foundations/none/}{none} {or}
\href{/docs/reference/foundations/str/}{str}

A text describing the image.

\paragraph{\texorpdfstring{\texttt{\ fit\ }}{ fit }}\label{definitions-decode-fit}

\href{/docs/reference/foundations/str/}{str}

How the image should adjust itself to a given area.

\begin{longtable}[]{@{}ll@{}}
\toprule\noalign{}
Variant & Details \\
\midrule\noalign{}
\endhead
\bottomrule\noalign{}
\endlastfoot
\texttt{\ "\ cover\ "\ } & The image should completely cover the area
(preserves aspect ratio by cropping the image only horizontally or
vertically). This is the default. \\
\texttt{\ "\ contain\ "\ } & The image should be fully contained in the
area (preserves aspect ratio; doesn\textquotesingle t crop the image;
one dimension can be narrower than specified). \\
\texttt{\ "\ stretch\ "\ } & The image should be stretched so that it
exactly fills the area, even if this means that the image will be
distorted (doesn\textquotesingle t preserve aspect ratio and
doesn\textquotesingle t crop the image). \\
\end{longtable}

\href{/docs/reference/visualize/gradient/}{\pandocbounded{\includesvg[keepaspectratio]{/assets/icons/16-arrow-right.svg}}}

{ Gradient } { Previous page }

\href{/docs/reference/visualize/line/}{\pandocbounded{\includesvg[keepaspectratio]{/assets/icons/16-arrow-right.svg}}}

{ Line } { Next page }


\section{Docs LaTeX/typst.app/docs/reference/visualize/line.tex}
\title{typst.app/docs/reference/visualize/line}

\begin{itemize}
\tightlist
\item
  \href{/docs}{\includesvg[width=0.16667in,height=0.16667in]{/assets/icons/16-docs-dark.svg}}
\item
  \includesvg[width=0.16667in,height=0.16667in]{/assets/icons/16-arrow-right.svg}
\item
  \href{/docs/reference/}{Reference}
\item
  \includesvg[width=0.16667in,height=0.16667in]{/assets/icons/16-arrow-right.svg}
\item
  \href{/docs/reference/visualize/}{Visualize}
\item
  \includesvg[width=0.16667in,height=0.16667in]{/assets/icons/16-arrow-right.svg}
\item
  \href{/docs/reference/visualize/line/}{Line}
\end{itemize}

\section{\texorpdfstring{\texttt{\ line\ } {{ Element
}}}{ line   Element }}\label{summary}

\phantomsection\label{element-tooltip}
Element functions can be customized with \texttt{\ set\ } and
\texttt{\ show\ } rules.

A line from one point to another.

\subsection{Example}\label{example}

\begin{verbatim}
#set page(height: 100pt)

#line(length: 100%)
#line(end: (50%, 50%))
#line(
  length: 4cm,
  stroke: 2pt + maroon,
)
\end{verbatim}

\includegraphics[width=5in,height=\textheight,keepaspectratio]{/assets/docs/IBdLCKW0h9kNWs6W_8DKAwAAAAAAAAAA.png}

\subsection{\texorpdfstring{{ Parameters
}}{ Parameters }}\label{parameters}

\phantomsection\label{parameters-tooltip}
Parameters are the inputs to a function. They are specified in
parentheses after the function name.

{ line } (

{ \hyperref[parameters-start]{start :}
\href{/docs/reference/foundations/array/}{array} , } {
\hyperref[parameters-end]{end :}
\href{/docs/reference/foundations/none/}{none}
\href{/docs/reference/foundations/array/}{array} , } {
\hyperref[parameters-length]{length :}
\href{/docs/reference/layout/relative/}{relative} , } {
\hyperref[parameters-angle]{angle :}
\href{/docs/reference/layout/angle/}{angle} , } {
\hyperref[parameters-stroke]{stroke :}
\href{/docs/reference/layout/length/}{length}
\href{/docs/reference/visualize/color/}{color}
\href{/docs/reference/visualize/gradient/}{gradient}
\href{/docs/reference/visualize/stroke/}{stroke}
\href{/docs/reference/visualize/pattern/}{pattern}
\href{/docs/reference/foundations/dictionary/}{dictionary} , }

) -\textgreater{} \href{/docs/reference/foundations/content/}{content}

\subsubsection{\texorpdfstring{\texttt{\ start\ }}{ start }}\label{parameters-start}

\href{/docs/reference/foundations/array/}{array}

{{ Settable }}

\phantomsection\label{parameters-start-settable-tooltip}
Settable parameters can be customized for all following uses of the
function with a \texttt{\ set\ } rule.

The start point of the line.

Must be an array of exactly two relative lengths.

Default:
\texttt{\ }{\texttt{\ (\ }}\texttt{\ }{\texttt{\ 0\%\ }}\texttt{\ }{\texttt{\ +\ }}\texttt{\ }{\texttt{\ 0pt\ }}\texttt{\ }{\texttt{\ ,\ }}\texttt{\ }{\texttt{\ 0\%\ }}\texttt{\ }{\texttt{\ +\ }}\texttt{\ }{\texttt{\ 0pt\ }}\texttt{\ }{\texttt{\ )\ }}\texttt{\ }

\subsubsection{\texorpdfstring{\texttt{\ end\ }}{ end }}\label{parameters-end}

\href{/docs/reference/foundations/none/}{none} {or}
\href{/docs/reference/foundations/array/}{array}

{{ Settable }}

\phantomsection\label{parameters-end-settable-tooltip}
Settable parameters can be customized for all following uses of the
function with a \texttt{\ set\ } rule.

The offset from \texttt{\ start\ } where the line ends.

Default: \texttt{\ }{\texttt{\ none\ }}\texttt{\ }

\subsubsection{\texorpdfstring{\texttt{\ length\ }}{ length }}\label{parameters-length}

\href{/docs/reference/layout/relative/}{relative}

{{ Settable }}

\phantomsection\label{parameters-length-settable-tooltip}
Settable parameters can be customized for all following uses of the
function with a \texttt{\ set\ } rule.

The line\textquotesingle s length. This is only respected if
\texttt{\ end\ } is \texttt{\ }{\texttt{\ none\ }}\texttt{\ } .

Default:
\texttt{\ }{\texttt{\ 0\%\ }}\texttt{\ }{\texttt{\ +\ }}\texttt{\ }{\texttt{\ 30pt\ }}\texttt{\ }

\subsubsection{\texorpdfstring{\texttt{\ angle\ }}{ angle }}\label{parameters-angle}

\href{/docs/reference/layout/angle/}{angle}

{{ Settable }}

\phantomsection\label{parameters-angle-settable-tooltip}
Settable parameters can be customized for all following uses of the
function with a \texttt{\ set\ } rule.

The angle at which the line points away from the origin. This is only
respected if \texttt{\ end\ } is
\texttt{\ }{\texttt{\ none\ }}\texttt{\ } .

Default: \texttt{\ }{\texttt{\ 0deg\ }}\texttt{\ }

\subsubsection{\texorpdfstring{\texttt{\ stroke\ }}{ stroke }}\label{parameters-stroke}

\href{/docs/reference/layout/length/}{length} {or}
\href{/docs/reference/visualize/color/}{color} {or}
\href{/docs/reference/visualize/gradient/}{gradient} {or}
\href{/docs/reference/visualize/stroke/}{stroke} {or}
\href{/docs/reference/visualize/pattern/}{pattern} {or}
\href{/docs/reference/foundations/dictionary/}{dictionary}

{{ Settable }}

\phantomsection\label{parameters-stroke-settable-tooltip}
Settable parameters can be customized for all following uses of the
function with a \texttt{\ set\ } rule.

How to \href{/docs/reference/visualize/stroke/}{stroke} the line.

Default:
\texttt{\ }{\texttt{\ 1pt\ }}\texttt{\ }{\texttt{\ +\ }}\texttt{\ black\ }

\includesvg[width=0.16667in,height=0.16667in]{/assets/icons/16-arrow-right.svg}
View example

\begin{verbatim}
#set line(length: 100%)
#stack(
  spacing: 1em,
  line(stroke: 2pt + red),
  line(stroke: (paint: blue, thickness: 4pt, cap: "round")),
  line(stroke: (paint: blue, thickness: 1pt, dash: "dashed")),
  line(stroke: (paint: blue, thickness: 1pt, dash: ("dot", 2pt, 4pt, 2pt))),
)
\end{verbatim}

\includegraphics[width=5in,height=\textheight,keepaspectratio]{/assets/docs/Shwqpl9XrWkg6A1XzBok6AAAAAAAAAAA.png}

\href{/docs/reference/visualize/image/}{\pandocbounded{\includesvg[keepaspectratio]{/assets/icons/16-arrow-right.svg}}}

{ Image } { Previous page }

\href{/docs/reference/visualize/path/}{\pandocbounded{\includesvg[keepaspectratio]{/assets/icons/16-arrow-right.svg}}}

{ Path } { Next page }


\section{Docs LaTeX/typst.app/docs/reference/visualize/color.tex}
\title{typst.app/docs/reference/visualize/color}

\begin{itemize}
\tightlist
\item
  \href{/docs}{\includesvg[width=0.16667in,height=0.16667in]{/assets/icons/16-docs-dark.svg}}
\item
  \includesvg[width=0.16667in,height=0.16667in]{/assets/icons/16-arrow-right.svg}
\item
  \href{/docs/reference/}{Reference}
\item
  \includesvg[width=0.16667in,height=0.16667in]{/assets/icons/16-arrow-right.svg}
\item
  \href{/docs/reference/visualize/}{Visualize}
\item
  \includesvg[width=0.16667in,height=0.16667in]{/assets/icons/16-arrow-right.svg}
\item
  \href{/docs/reference/visualize/color/}{Color}
\end{itemize}

\section{\texorpdfstring{{ color }}{ color }}\label{summary}

A color in a specific color space.

Typst supports:

\begin{itemize}
\tightlist
\item
  sRGB through the
  \href{/docs/reference/visualize/color/\#definitions-rgb}{\texttt{\ rgb\ }
  function}
\item
  Device CMYK through
  \href{/docs/reference/visualize/color/\#definitions-cmyk}{\texttt{\ cmyk\ }
  function}
\item
  D65 Gray through the
  \href{/docs/reference/visualize/color/\#definitions-luma}{\texttt{\ luma\ }
  function}
\item
  Oklab through the
  \href{/docs/reference/visualize/color/\#definitions-oklab}{\texttt{\ oklab\ }
  function}
\item
  Oklch through the
  \href{/docs/reference/visualize/color/\#definitions-oklch}{\texttt{\ oklch\ }
  function}
\item
  Linear RGB through the
  \href{/docs/reference/visualize/color/\#definitions-linear-rgb}{\texttt{\ color.linear-rgb\ }
  function}
\item
  HSL through the
  \href{/docs/reference/visualize/color/\#definitions-hsl}{\texttt{\ color.hsl\ }
  function}
\item
  HSV through the
  \href{/docs/reference/visualize/color/\#definitions-hsv}{\texttt{\ color.hsv\ }
  function}
\end{itemize}

\subsection{Example}\label{example}

\begin{verbatim}
#rect(fill: aqua)
\end{verbatim}

\includegraphics[width=5in,height=\textheight,keepaspectratio]{/assets/docs/k-6wh2l9TTXmPhzZxpahjQAAAAAAAAAA.png}

\subsection{Predefined colors}\label{predefined-colors}

Typst defines the following built-in colors:

\begin{longtable}[]{@{}ll@{}}
\toprule\noalign{}
Color & Definition \\
\midrule\noalign{}
\endhead
\bottomrule\noalign{}
\endlastfoot
\texttt{\ black\ } &
\texttt{\ }{\texttt{\ luma\ }}\texttt{\ }{\texttt{\ (\ }}\texttt{\ }{\texttt{\ 0\ }}\texttt{\ }{\texttt{\ )\ }}\texttt{\ } \\
\texttt{\ gray\ } &
\texttt{\ }{\texttt{\ luma\ }}\texttt{\ }{\texttt{\ (\ }}\texttt{\ }{\texttt{\ 170\ }}\texttt{\ }{\texttt{\ )\ }}\texttt{\ } \\
\texttt{\ silver\ } &
\texttt{\ }{\texttt{\ luma\ }}\texttt{\ }{\texttt{\ (\ }}\texttt{\ }{\texttt{\ 221\ }}\texttt{\ }{\texttt{\ )\ }}\texttt{\ } \\
\texttt{\ white\ } &
\texttt{\ }{\texttt{\ luma\ }}\texttt{\ }{\texttt{\ (\ }}\texttt{\ }{\texttt{\ 255\ }}\texttt{\ }{\texttt{\ )\ }}\texttt{\ } \\
\texttt{\ navy\ } &
\texttt{\ }{\texttt{\ rgb\ }}\texttt{\ }{\texttt{\ (\ }}\texttt{\ }{\texttt{\ "\#001f3f"\ }}\texttt{\ }{\texttt{\ )\ }}\texttt{\ } \\
\texttt{\ blue\ } &
\texttt{\ }{\texttt{\ rgb\ }}\texttt{\ }{\texttt{\ (\ }}\texttt{\ }{\texttt{\ "\#0074d9"\ }}\texttt{\ }{\texttt{\ )\ }}\texttt{\ } \\
\texttt{\ aqua\ } &
\texttt{\ }{\texttt{\ rgb\ }}\texttt{\ }{\texttt{\ (\ }}\texttt{\ }{\texttt{\ "\#7fdbff"\ }}\texttt{\ }{\texttt{\ )\ }}\texttt{\ } \\
\texttt{\ teal\ } &
\texttt{\ }{\texttt{\ rgb\ }}\texttt{\ }{\texttt{\ (\ }}\texttt{\ }{\texttt{\ "\#39cccc"\ }}\texttt{\ }{\texttt{\ )\ }}\texttt{\ } \\
\texttt{\ eastern\ } &
\texttt{\ }{\texttt{\ rgb\ }}\texttt{\ }{\texttt{\ (\ }}\texttt{\ }{\texttt{\ "\#239dad"\ }}\texttt{\ }{\texttt{\ )\ }}\texttt{\ } \\
\texttt{\ purple\ } &
\texttt{\ }{\texttt{\ rgb\ }}\texttt{\ }{\texttt{\ (\ }}\texttt{\ }{\texttt{\ "\#b10dc9"\ }}\texttt{\ }{\texttt{\ )\ }}\texttt{\ } \\
\texttt{\ fuchsia\ } &
\texttt{\ }{\texttt{\ rgb\ }}\texttt{\ }{\texttt{\ (\ }}\texttt{\ }{\texttt{\ "\#f012be"\ }}\texttt{\ }{\texttt{\ )\ }}\texttt{\ } \\
\texttt{\ maroon\ } &
\texttt{\ }{\texttt{\ rgb\ }}\texttt{\ }{\texttt{\ (\ }}\texttt{\ }{\texttt{\ "\#85144b"\ }}\texttt{\ }{\texttt{\ )\ }}\texttt{\ } \\
\texttt{\ red\ } &
\texttt{\ }{\texttt{\ rgb\ }}\texttt{\ }{\texttt{\ (\ }}\texttt{\ }{\texttt{\ "\#ff4136"\ }}\texttt{\ }{\texttt{\ )\ }}\texttt{\ } \\
\texttt{\ orange\ } &
\texttt{\ }{\texttt{\ rgb\ }}\texttt{\ }{\texttt{\ (\ }}\texttt{\ }{\texttt{\ "\#ff851b"\ }}\texttt{\ }{\texttt{\ )\ }}\texttt{\ } \\
\texttt{\ yellow\ } &
\texttt{\ }{\texttt{\ rgb\ }}\texttt{\ }{\texttt{\ (\ }}\texttt{\ }{\texttt{\ "\#ffdc00"\ }}\texttt{\ }{\texttt{\ )\ }}\texttt{\ } \\
\texttt{\ olive\ } &
\texttt{\ }{\texttt{\ rgb\ }}\texttt{\ }{\texttt{\ (\ }}\texttt{\ }{\texttt{\ "\#3d9970"\ }}\texttt{\ }{\texttt{\ )\ }}\texttt{\ } \\
\texttt{\ green\ } &
\texttt{\ }{\texttt{\ rgb\ }}\texttt{\ }{\texttt{\ (\ }}\texttt{\ }{\texttt{\ "\#2ecc40"\ }}\texttt{\ }{\texttt{\ )\ }}\texttt{\ } \\
\texttt{\ lime\ } &
\texttt{\ }{\texttt{\ rgb\ }}\texttt{\ }{\texttt{\ (\ }}\texttt{\ }{\texttt{\ "\#01ff70"\ }}\texttt{\ }{\texttt{\ )\ }}\texttt{\ } \\
\end{longtable}

The predefined colors and the most important color constructors are
available globally and also in the color type\textquotesingle s scope,
so you can write either \texttt{\ color.red\ } or just \texttt{\ red\ }
.

\includegraphics[width=11.66667in,height=\textheight,keepaspectratio]{/assets/docs/IWvUAQq21Ue1zu9gwjch-gAAAAAAAAAA.png}

\subsection{Predefined color maps}\label{predefined-color-maps}

Typst also includes a number of preset color maps that can be used for
\href{/docs/reference/visualize/gradient/\#definitions-linear}{gradients}
. These are simply arrays of colors defined in the module
\texttt{\ color.map\ } .

\begin{verbatim}
#circle(fill: gradient.linear(..color.map.crest))
\end{verbatim}

\includegraphics[width=5in,height=\textheight,keepaspectratio]{/assets/docs/uG6iVgmQwH_6_-1N42yKHwAAAAAAAAAA.png}

\begin{longtable}[]{@{}ll@{}}
\toprule\noalign{}
Map & Details \\
\midrule\noalign{}
\endhead
\bottomrule\noalign{}
\endlastfoot
\texttt{\ turbo\ } & A perceptually uniform rainbow-like color map. Read
\href{https://ai.googleblog.com/2019/08/turbo-improved-rainbow-colormap-for.html}{this
blog post} for more details. \\
\texttt{\ cividis\ } & A blue to gray to yellow color map. See
\href{https://bids.github.io/colormap/}{this blog post} for more
details. \\
\texttt{\ rainbow\ } & Cycles through the full color spectrum. This
color map is best used by setting the interpolation color space to
\href{/docs/reference/visualize/color/\#definitions-hsl}{HSL} . The
rainbow gradient is \textbf{not suitable} for data visualization because
it is not perceptually uniform, so the differences between values become
unclear to your readers. It should only be used for decorative
purposes. \\
\texttt{\ spectral\ } & Red to yellow to blue color map. \\
\texttt{\ viridis\ } & A purple to teal to yellow color map. \\
\texttt{\ inferno\ } & A black to red to yellow color map. \\
\texttt{\ magma\ } & A black to purple to yellow color map. \\
\texttt{\ plasma\ } & A purple to pink to yellow color map. \\
\texttt{\ rocket\ } & A black to red to white color map. \\
\texttt{\ mako\ } & A black to teal to yellow color map. \\
\texttt{\ vlag\ } & A light blue to white to red color map. \\
\texttt{\ icefire\ } & A light teal to black to yellow color map. \\
\texttt{\ flare\ } & A orange to purple color map that is perceptually
uniform. \\
\texttt{\ crest\ } & A blue to white to red color map. \\
\end{longtable}

Some popular presets are not included because they are not available
under a free licence. Others, like
\href{https://jakevdp.github.io/blog/2014/10/16/how-bad-is-your-colormap/}{Jet}
, are not included because they are not color blind friendly. Feel free
to use or create a package with other presets that are useful to you!

\includegraphics[width=5.8125in,height=\textheight,keepaspectratio]{/assets/docs/S2ExoTDRK30Xf9wXJbWIZgAAAAAAAAAA.png}

\subsection{\texorpdfstring{{ Definitions
}}{ Definitions }}\label{definitions}

\phantomsection\label{definitions-tooltip}
Functions and types and can have associated definitions. These are
accessed by specifying the function or type, followed by a period, and
then the definition\textquotesingle s name.

\subsubsection{\texorpdfstring{\texttt{\ luma\ }}{ luma }}\label{definitions-luma}

Create a grayscale color.

A grayscale color is represented internally by a single
\texttt{\ lightness\ } component.

These components are also available using the
\href{/docs/reference/visualize/color/\#definitions-components}{\texttt{\ components\ }}
method.

color { . } { luma } (

{ \href{/docs/reference/foundations/int/}{int}
\href{/docs/reference/layout/ratio/}{ratio} , } {
\href{/docs/reference/layout/ratio/}{ratio} , } {
\href{/docs/reference/visualize/color/}{color} , }

) -\textgreater{} \href{/docs/reference/visualize/color/}{color}

\begin{verbatim}
#for x in range(250, step: 50) {
  box(square(fill: luma(x)))
}
\end{verbatim}

\includegraphics[width=5in,height=\textheight,keepaspectratio]{/assets/docs/bCTOWkOtpDPjuD2iPgTajQAAAAAAAAAA.png}

\paragraph{\texorpdfstring{\texttt{\ lightness\ }}{ lightness }}\label{definitions-luma-lightness}

\href{/docs/reference/foundations/int/}{int} {or}
\href{/docs/reference/layout/ratio/}{ratio}

{Required} {{ Positional }}

\phantomsection\label{definitions-luma-lightness-positional-tooltip}
Positional parameters are specified in order, without names.

The lightness component.

\paragraph{\texorpdfstring{\texttt{\ alpha\ }}{ alpha }}\label{definitions-luma-alpha}

\href{/docs/reference/layout/ratio/}{ratio}

{Required} {{ Positional }}

\phantomsection\label{definitions-luma-alpha-positional-tooltip}
Positional parameters are specified in order, without names.

The alpha component.

\paragraph{\texorpdfstring{\texttt{\ color\ }}{ color }}\label{definitions-luma-color}

\href{/docs/reference/visualize/color/}{color}

{Required} {{ Positional }}

\phantomsection\label{definitions-luma-color-positional-tooltip}
Positional parameters are specified in order, without names.

Alternatively: The color to convert to grayscale.

If this is given, the \texttt{\ lightness\ } should not be given.

\subsubsection{\texorpdfstring{\texttt{\ oklab\ }}{ oklab }}\label{definitions-oklab}

Create an \href{https://bottosson.github.io/posts/oklab/}{Oklab} color.

This color space is well suited for the following use cases:

\begin{itemize}
\tightlist
\item
  Color manipulation such as saturating while keeping perceived hue
\item
  Creating grayscale images with uniform perceived lightness
\item
  Creating smooth and uniform color transition and gradients
\end{itemize}

A linear Oklab color is represented internally by an array of four
components:

\begin{itemize}
\tightlist
\item
  lightness ( \href{/docs/reference/layout/ratio/}{\texttt{\ ratio\ }} )
\item
  a ( \href{/docs/reference/foundations/float/}{\texttt{\ float\ }} or
  \href{/docs/reference/layout/ratio/}{\texttt{\ ratio\ }} . Ratios are
  relative to \texttt{\ }{\texttt{\ 0.4\ }}\texttt{\ } ; meaning
  \texttt{\ }{\texttt{\ 50\%\ }}\texttt{\ } is equal to
  \texttt{\ }{\texttt{\ 0.2\ }}\texttt{\ } )
\item
  b ( \href{/docs/reference/foundations/float/}{\texttt{\ float\ }} or
  \href{/docs/reference/layout/ratio/}{\texttt{\ ratio\ }} . Ratios are
  relative to \texttt{\ }{\texttt{\ 0.4\ }}\texttt{\ } ; meaning
  \texttt{\ }{\texttt{\ 50\%\ }}\texttt{\ } is equal to
  \texttt{\ }{\texttt{\ 0.2\ }}\texttt{\ } )
\item
  alpha ( \href{/docs/reference/layout/ratio/}{\texttt{\ ratio\ }} )
\end{itemize}

These components are also available using the
\href{/docs/reference/visualize/color/\#definitions-components}{\texttt{\ components\ }}
method.

color { . } { oklab } (

{ \href{/docs/reference/layout/ratio/}{ratio} , } {
\href{/docs/reference/foundations/float/}{float}
\href{/docs/reference/layout/ratio/}{ratio} , } {
\href{/docs/reference/foundations/float/}{float}
\href{/docs/reference/layout/ratio/}{ratio} , } {
\href{/docs/reference/layout/ratio/}{ratio} , } {
\href{/docs/reference/visualize/color/}{color} , }

) -\textgreater{} \href{/docs/reference/visualize/color/}{color}

\begin{verbatim}
#square(
  fill: oklab(27%, 20%, -3%, 50%)
)
\end{verbatim}

\includegraphics[width=5in,height=\textheight,keepaspectratio]{/assets/docs/1dGzDbwdYzYb5NzJEzQzFAAAAAAAAAAA.png}

\paragraph{\texorpdfstring{\texttt{\ lightness\ }}{ lightness }}\label{definitions-oklab-lightness}

\href{/docs/reference/layout/ratio/}{ratio}

{Required} {{ Positional }}

\phantomsection\label{definitions-oklab-lightness-positional-tooltip}
Positional parameters are specified in order, without names.

The lightness component.

\paragraph{\texorpdfstring{\texttt{\ a\ }}{ a }}\label{definitions-oklab-a}

\href{/docs/reference/foundations/float/}{float} {or}
\href{/docs/reference/layout/ratio/}{ratio}

{Required} {{ Positional }}

\phantomsection\label{definitions-oklab-a-positional-tooltip}
Positional parameters are specified in order, without names.

The a ("green/red") component.

\paragraph{\texorpdfstring{\texttt{\ b\ }}{ b }}\label{definitions-oklab-b}

\href{/docs/reference/foundations/float/}{float} {or}
\href{/docs/reference/layout/ratio/}{ratio}

{Required} {{ Positional }}

\phantomsection\label{definitions-oklab-b-positional-tooltip}
Positional parameters are specified in order, without names.

The b ("blue/yellow") component.

\paragraph{\texorpdfstring{\texttt{\ alpha\ }}{ alpha }}\label{definitions-oklab-alpha}

\href{/docs/reference/layout/ratio/}{ratio}

{Required} {{ Positional }}

\phantomsection\label{definitions-oklab-alpha-positional-tooltip}
Positional parameters are specified in order, without names.

The alpha component.

\paragraph{\texorpdfstring{\texttt{\ color\ }}{ color }}\label{definitions-oklab-color}

\href{/docs/reference/visualize/color/}{color}

{Required} {{ Positional }}

\phantomsection\label{definitions-oklab-color-positional-tooltip}
Positional parameters are specified in order, without names.

Alternatively: The color to convert to Oklab.

If this is given, the individual components should not be given.

\subsubsection{\texorpdfstring{\texttt{\ oklch\ }}{ oklch }}\label{definitions-oklch}

Create an \href{https://bottosson.github.io/posts/oklab/}{Oklch} color.

This color space is well suited for the following use cases:

\begin{itemize}
\tightlist
\item
  Color manipulation involving lightness, chroma, and hue
\item
  Creating grayscale images with uniform perceived lightness
\item
  Creating smooth and uniform color transition and gradients
\end{itemize}

A linear Oklch color is represented internally by an array of four
components:

\begin{itemize}
\tightlist
\item
  lightness ( \href{/docs/reference/layout/ratio/}{\texttt{\ ratio\ }} )
\item
  chroma ( \href{/docs/reference/foundations/float/}{\texttt{\ float\ }}
  or \href{/docs/reference/layout/ratio/}{\texttt{\ ratio\ }} . Ratios
  are relative to \texttt{\ }{\texttt{\ 0.4\ }}\texttt{\ } ; meaning
  \texttt{\ }{\texttt{\ 50\%\ }}\texttt{\ } is equal to
  \texttt{\ }{\texttt{\ 0.2\ }}\texttt{\ } )
\item
  hue ( \href{/docs/reference/layout/angle/}{\texttt{\ angle\ }} )
\item
  alpha ( \href{/docs/reference/layout/ratio/}{\texttt{\ ratio\ }} )
\end{itemize}

These components are also available using the
\href{/docs/reference/visualize/color/\#definitions-components}{\texttt{\ components\ }}
method.

color { . } { oklch } (

{ \href{/docs/reference/layout/ratio/}{ratio} , } {
\href{/docs/reference/foundations/float/}{float}
\href{/docs/reference/layout/ratio/}{ratio} , } {
\href{/docs/reference/layout/angle/}{angle} , } {
\href{/docs/reference/layout/ratio/}{ratio} , } {
\href{/docs/reference/visualize/color/}{color} , }

) -\textgreater{} \href{/docs/reference/visualize/color/}{color}

\begin{verbatim}
#square(
  fill: oklch(40%, 0.2, 160deg, 50%)
)
\end{verbatim}

\includegraphics[width=5in,height=\textheight,keepaspectratio]{/assets/docs/gEJt1PBpGTajcUm46S-JNgAAAAAAAAAA.png}

\paragraph{\texorpdfstring{\texttt{\ lightness\ }}{ lightness }}\label{definitions-oklch-lightness}

\href{/docs/reference/layout/ratio/}{ratio}

{Required} {{ Positional }}

\phantomsection\label{definitions-oklch-lightness-positional-tooltip}
Positional parameters are specified in order, without names.

The lightness component.

\paragraph{\texorpdfstring{\texttt{\ chroma\ }}{ chroma }}\label{definitions-oklch-chroma}

\href{/docs/reference/foundations/float/}{float} {or}
\href{/docs/reference/layout/ratio/}{ratio}

{Required} {{ Positional }}

\phantomsection\label{definitions-oklch-chroma-positional-tooltip}
Positional parameters are specified in order, without names.

The chroma component.

\paragraph{\texorpdfstring{\texttt{\ hue\ }}{ hue }}\label{definitions-oklch-hue}

\href{/docs/reference/layout/angle/}{angle}

{Required} {{ Positional }}

\phantomsection\label{definitions-oklch-hue-positional-tooltip}
Positional parameters are specified in order, without names.

The hue component.

\paragraph{\texorpdfstring{\texttt{\ alpha\ }}{ alpha }}\label{definitions-oklch-alpha}

\href{/docs/reference/layout/ratio/}{ratio}

{Required} {{ Positional }}

\phantomsection\label{definitions-oklch-alpha-positional-tooltip}
Positional parameters are specified in order, without names.

The alpha component.

\paragraph{\texorpdfstring{\texttt{\ color\ }}{ color }}\label{definitions-oklch-color}

\href{/docs/reference/visualize/color/}{color}

{Required} {{ Positional }}

\phantomsection\label{definitions-oklch-color-positional-tooltip}
Positional parameters are specified in order, without names.

Alternatively: The color to convert to Oklch.

If this is given, the individual components should not be given.

\subsubsection{\texorpdfstring{\texttt{\ linear-rgb\ }}{ linear-rgb }}\label{definitions-linear-rgb}

Create an RGB(A) color with linear luma.

This color space is similar to sRGB, but with the distinction that the
color component are not gamma corrected. This makes it easier to perform
color operations such as blending and interpolation. Although, you
should prefer to use the
\href{/docs/reference/visualize/color/\#definitions-oklab}{\texttt{\ oklab\ }
function} for these.

A linear RGB(A) color is represented internally by an array of four
components:

\begin{itemize}
\tightlist
\item
  red ( \href{/docs/reference/layout/ratio/}{\texttt{\ ratio\ }} )
\item
  green ( \href{/docs/reference/layout/ratio/}{\texttt{\ ratio\ }} )
\item
  blue ( \href{/docs/reference/layout/ratio/}{\texttt{\ ratio\ }} )
\item
  alpha ( \href{/docs/reference/layout/ratio/}{\texttt{\ ratio\ }} )
\end{itemize}

These components are also available using the
\href{/docs/reference/visualize/color/\#definitions-components}{\texttt{\ components\ }}
method.

color { . } { linear-rgb } (

{ \href{/docs/reference/foundations/int/}{int}
\href{/docs/reference/layout/ratio/}{ratio} , } {
\href{/docs/reference/foundations/int/}{int}
\href{/docs/reference/layout/ratio/}{ratio} , } {
\href{/docs/reference/foundations/int/}{int}
\href{/docs/reference/layout/ratio/}{ratio} , } {
\href{/docs/reference/foundations/int/}{int}
\href{/docs/reference/layout/ratio/}{ratio} , } {
\href{/docs/reference/visualize/color/}{color} , }

) -\textgreater{} \href{/docs/reference/visualize/color/}{color}

\begin{verbatim}
#square(fill: color.linear-rgb(
  30%, 50%, 10%,
))
\end{verbatim}

\includegraphics[width=5in,height=\textheight,keepaspectratio]{/assets/docs/C39dYHKq1AmgEkOU8XX2kQAAAAAAAAAA.png}

\paragraph{\texorpdfstring{\texttt{\ red\ }}{ red }}\label{definitions-linear-rgb-red}

\href{/docs/reference/foundations/int/}{int} {or}
\href{/docs/reference/layout/ratio/}{ratio}

{Required} {{ Positional }}

\phantomsection\label{definitions-linear-rgb-red-positional-tooltip}
Positional parameters are specified in order, without names.

The red component.

\paragraph{\texorpdfstring{\texttt{\ green\ }}{ green }}\label{definitions-linear-rgb-green}

\href{/docs/reference/foundations/int/}{int} {or}
\href{/docs/reference/layout/ratio/}{ratio}

{Required} {{ Positional }}

\phantomsection\label{definitions-linear-rgb-green-positional-tooltip}
Positional parameters are specified in order, without names.

The green component.

\paragraph{\texorpdfstring{\texttt{\ blue\ }}{ blue }}\label{definitions-linear-rgb-blue}

\href{/docs/reference/foundations/int/}{int} {or}
\href{/docs/reference/layout/ratio/}{ratio}

{Required} {{ Positional }}

\phantomsection\label{definitions-linear-rgb-blue-positional-tooltip}
Positional parameters are specified in order, without names.

The blue component.

\paragraph{\texorpdfstring{\texttt{\ alpha\ }}{ alpha }}\label{definitions-linear-rgb-alpha}

\href{/docs/reference/foundations/int/}{int} {or}
\href{/docs/reference/layout/ratio/}{ratio}

{Required} {{ Positional }}

\phantomsection\label{definitions-linear-rgb-alpha-positional-tooltip}
Positional parameters are specified in order, without names.

The alpha component.

\paragraph{\texorpdfstring{\texttt{\ color\ }}{ color }}\label{definitions-linear-rgb-color}

\href{/docs/reference/visualize/color/}{color}

{Required} {{ Positional }}

\phantomsection\label{definitions-linear-rgb-color-positional-tooltip}
Positional parameters are specified in order, without names.

Alternatively: The color to convert to linear RGB(A).

If this is given, the individual components should not be given.

\subsubsection{\texorpdfstring{\texttt{\ rgb\ }}{ rgb }}\label{definitions-rgb}

Create an RGB(A) color.

The color is specified in the sRGB color space.

An RGB(A) color is represented internally by an array of four
components:

\begin{itemize}
\tightlist
\item
  red ( \href{/docs/reference/layout/ratio/}{\texttt{\ ratio\ }} )
\item
  green ( \href{/docs/reference/layout/ratio/}{\texttt{\ ratio\ }} )
\item
  blue ( \href{/docs/reference/layout/ratio/}{\texttt{\ ratio\ }} )
\item
  alpha ( \href{/docs/reference/layout/ratio/}{\texttt{\ ratio\ }} )
\end{itemize}

These components are also available using the
\href{/docs/reference/visualize/color/\#definitions-components}{\texttt{\ components\ }}
method.

color { . } { rgb } (

{ \href{/docs/reference/foundations/int/}{int}
\href{/docs/reference/layout/ratio/}{ratio} , } {
\href{/docs/reference/foundations/int/}{int}
\href{/docs/reference/layout/ratio/}{ratio} , } {
\href{/docs/reference/foundations/int/}{int}
\href{/docs/reference/layout/ratio/}{ratio} , } {
\href{/docs/reference/foundations/int/}{int}
\href{/docs/reference/layout/ratio/}{ratio} , } {
\href{/docs/reference/foundations/str/}{str} , } {
\href{/docs/reference/visualize/color/}{color} , }

) -\textgreater{} \href{/docs/reference/visualize/color/}{color}

\begin{verbatim}
#square(fill: rgb("#b1f2eb"))
#square(fill: rgb(87, 127, 230))
#square(fill: rgb(25%, 13%, 65%))
\end{verbatim}

\includegraphics[width=5in,height=\textheight,keepaspectratio]{/assets/docs/eWivZbkq7oFotM06OeK92AAAAAAAAAAA.png}

\paragraph{\texorpdfstring{\texttt{\ red\ }}{ red }}\label{definitions-rgb-red}

\href{/docs/reference/foundations/int/}{int} {or}
\href{/docs/reference/layout/ratio/}{ratio}

{Required} {{ Positional }}

\phantomsection\label{definitions-rgb-red-positional-tooltip}
Positional parameters are specified in order, without names.

The red component.

\paragraph{\texorpdfstring{\texttt{\ green\ }}{ green }}\label{definitions-rgb-green}

\href{/docs/reference/foundations/int/}{int} {or}
\href{/docs/reference/layout/ratio/}{ratio}

{Required} {{ Positional }}

\phantomsection\label{definitions-rgb-green-positional-tooltip}
Positional parameters are specified in order, without names.

The green component.

\paragraph{\texorpdfstring{\texttt{\ blue\ }}{ blue }}\label{definitions-rgb-blue}

\href{/docs/reference/foundations/int/}{int} {or}
\href{/docs/reference/layout/ratio/}{ratio}

{Required} {{ Positional }}

\phantomsection\label{definitions-rgb-blue-positional-tooltip}
Positional parameters are specified in order, without names.

The blue component.

\paragraph{\texorpdfstring{\texttt{\ alpha\ }}{ alpha }}\label{definitions-rgb-alpha}

\href{/docs/reference/foundations/int/}{int} {or}
\href{/docs/reference/layout/ratio/}{ratio}

{Required} {{ Positional }}

\phantomsection\label{definitions-rgb-alpha-positional-tooltip}
Positional parameters are specified in order, without names.

The alpha component.

\paragraph{\texorpdfstring{\texttt{\ hex\ }}{ hex }}\label{definitions-rgb-hex}

\href{/docs/reference/foundations/str/}{str}

{Required} {{ Positional }}

\phantomsection\label{definitions-rgb-hex-positional-tooltip}
Positional parameters are specified in order, without names.

Alternatively: The color in hexadecimal notation.

Accepts three, four, six or eight hexadecimal digits and optionally a
leading hash.

If this is given, the individual components should not be given.

\includesvg[width=0.16667in,height=0.16667in]{/assets/icons/16-arrow-right.svg}
View example

\begin{verbatim}
#text(16pt, rgb("#239dad"))[
  *Typst*
]
\end{verbatim}

\includegraphics[width=5in,height=\textheight,keepaspectratio]{/assets/docs/rKfIt6nqSzoBRXt7k7BMOwAAAAAAAAAA.png}

\paragraph{\texorpdfstring{\texttt{\ color\ }}{ color }}\label{definitions-rgb-color}

\href{/docs/reference/visualize/color/}{color}

{Required} {{ Positional }}

\phantomsection\label{definitions-rgb-color-positional-tooltip}
Positional parameters are specified in order, without names.

Alternatively: The color to convert to RGB(a).

If this is given, the individual components should not be given.

\subsubsection{\texorpdfstring{\texttt{\ cmyk\ }}{ cmyk }}\label{definitions-cmyk}

Create a CMYK color.

This is useful if you want to target a specific printer. The conversion
to RGB for display preview might differ from how your printer reproduces
the color.

A CMYK color is represented internally by an array of four components:

\begin{itemize}
\tightlist
\item
  cyan ( \href{/docs/reference/layout/ratio/}{\texttt{\ ratio\ }} )
\item
  magenta ( \href{/docs/reference/layout/ratio/}{\texttt{\ ratio\ }} )
\item
  yellow ( \href{/docs/reference/layout/ratio/}{\texttt{\ ratio\ }} )
\item
  key ( \href{/docs/reference/layout/ratio/}{\texttt{\ ratio\ }} )
\end{itemize}

These components are also available using the
\href{/docs/reference/visualize/color/\#definitions-components}{\texttt{\ components\ }}
method.

Note that CMYK colors are not currently supported when PDF/A output is
enabled.

color { . } { cmyk } (

{ \href{/docs/reference/layout/ratio/}{ratio} , } {
\href{/docs/reference/layout/ratio/}{ratio} , } {
\href{/docs/reference/layout/ratio/}{ratio} , } {
\href{/docs/reference/layout/ratio/}{ratio} , } {
\href{/docs/reference/visualize/color/}{color} , }

) -\textgreater{} \href{/docs/reference/visualize/color/}{color}

\begin{verbatim}
#square(
  fill: cmyk(27%, 0%, 3%, 5%)
)
\end{verbatim}

\includegraphics[width=5in,height=\textheight,keepaspectratio]{/assets/docs/1LHigtpFCZVjSNs83fP0eAAAAAAAAAAA.png}

\paragraph{\texorpdfstring{\texttt{\ cyan\ }}{ cyan }}\label{definitions-cmyk-cyan}

\href{/docs/reference/layout/ratio/}{ratio}

{Required} {{ Positional }}

\phantomsection\label{definitions-cmyk-cyan-positional-tooltip}
Positional parameters are specified in order, without names.

The cyan component.

\paragraph{\texorpdfstring{\texttt{\ magenta\ }}{ magenta }}\label{definitions-cmyk-magenta}

\href{/docs/reference/layout/ratio/}{ratio}

{Required} {{ Positional }}

\phantomsection\label{definitions-cmyk-magenta-positional-tooltip}
Positional parameters are specified in order, without names.

The magenta component.

\paragraph{\texorpdfstring{\texttt{\ yellow\ }}{ yellow }}\label{definitions-cmyk-yellow}

\href{/docs/reference/layout/ratio/}{ratio}

{Required} {{ Positional }}

\phantomsection\label{definitions-cmyk-yellow-positional-tooltip}
Positional parameters are specified in order, without names.

The yellow component.

\paragraph{\texorpdfstring{\texttt{\ key\ }}{ key }}\label{definitions-cmyk-key}

\href{/docs/reference/layout/ratio/}{ratio}

{Required} {{ Positional }}

\phantomsection\label{definitions-cmyk-key-positional-tooltip}
Positional parameters are specified in order, without names.

The key component.

\paragraph{\texorpdfstring{\texttt{\ color\ }}{ color }}\label{definitions-cmyk-color}

\href{/docs/reference/visualize/color/}{color}

{Required} {{ Positional }}

\phantomsection\label{definitions-cmyk-color-positional-tooltip}
Positional parameters are specified in order, without names.

Alternatively: The color to convert to CMYK.

If this is given, the individual components should not be given.

\subsubsection{\texorpdfstring{\texttt{\ hsl\ }}{ hsl }}\label{definitions-hsl}

Create an HSL color.

This color space is useful for specifying colors by hue, saturation and
lightness. It is also useful for color manipulation, such as saturating
while keeping perceived hue.

An HSL color is represented internally by an array of four components:

\begin{itemize}
\tightlist
\item
  hue ( \href{/docs/reference/layout/angle/}{\texttt{\ angle\ }} )
\item
  saturation ( \href{/docs/reference/layout/ratio/}{\texttt{\ ratio\ }}
  )
\item
  lightness ( \href{/docs/reference/layout/ratio/}{\texttt{\ ratio\ }} )
\item
  alpha ( \href{/docs/reference/layout/ratio/}{\texttt{\ ratio\ }} )
\end{itemize}

These components are also available using the
\href{/docs/reference/visualize/color/\#definitions-components}{\texttt{\ components\ }}
method.

color { . } { hsl } (

{ \href{/docs/reference/layout/angle/}{angle} , } {
\href{/docs/reference/foundations/int/}{int}
\href{/docs/reference/layout/ratio/}{ratio} , } {
\href{/docs/reference/foundations/int/}{int}
\href{/docs/reference/layout/ratio/}{ratio} , } {
\href{/docs/reference/foundations/int/}{int}
\href{/docs/reference/layout/ratio/}{ratio} , } {
\href{/docs/reference/visualize/color/}{color} , }

) -\textgreater{} \href{/docs/reference/visualize/color/}{color}

\begin{verbatim}
#square(
  fill: color.hsl(30deg, 50%, 60%)
)
\end{verbatim}

\includegraphics[width=5in,height=\textheight,keepaspectratio]{/assets/docs/MqR1NhT-m_ImBDX23hY7xgAAAAAAAAAA.png}

\paragraph{\texorpdfstring{\texttt{\ hue\ }}{ hue }}\label{definitions-hsl-hue}

\href{/docs/reference/layout/angle/}{angle}

{Required} {{ Positional }}

\phantomsection\label{definitions-hsl-hue-positional-tooltip}
Positional parameters are specified in order, without names.

The hue angle.

\paragraph{\texorpdfstring{\texttt{\ saturation\ }}{ saturation }}\label{definitions-hsl-saturation}

\href{/docs/reference/foundations/int/}{int} {or}
\href{/docs/reference/layout/ratio/}{ratio}

{Required} {{ Positional }}

\phantomsection\label{definitions-hsl-saturation-positional-tooltip}
Positional parameters are specified in order, without names.

The saturation component.

\paragraph{\texorpdfstring{\texttt{\ lightness\ }}{ lightness }}\label{definitions-hsl-lightness}

\href{/docs/reference/foundations/int/}{int} {or}
\href{/docs/reference/layout/ratio/}{ratio}

{Required} {{ Positional }}

\phantomsection\label{definitions-hsl-lightness-positional-tooltip}
Positional parameters are specified in order, without names.

The lightness component.

\paragraph{\texorpdfstring{\texttt{\ alpha\ }}{ alpha }}\label{definitions-hsl-alpha}

\href{/docs/reference/foundations/int/}{int} {or}
\href{/docs/reference/layout/ratio/}{ratio}

{Required} {{ Positional }}

\phantomsection\label{definitions-hsl-alpha-positional-tooltip}
Positional parameters are specified in order, without names.

The alpha component.

\paragraph{\texorpdfstring{\texttt{\ color\ }}{ color }}\label{definitions-hsl-color}

\href{/docs/reference/visualize/color/}{color}

{Required} {{ Positional }}

\phantomsection\label{definitions-hsl-color-positional-tooltip}
Positional parameters are specified in order, without names.

Alternatively: The color to convert to HSL.

If this is given, the individual components should not be given.

\subsubsection{\texorpdfstring{\texttt{\ hsv\ }}{ hsv }}\label{definitions-hsv}

Create an HSV color.

This color space is useful for specifying colors by hue, saturation and
value. It is also useful for color manipulation, such as saturating
while keeping perceived hue.

An HSV color is represented internally by an array of four components:

\begin{itemize}
\tightlist
\item
  hue ( \href{/docs/reference/layout/angle/}{\texttt{\ angle\ }} )
\item
  saturation ( \href{/docs/reference/layout/ratio/}{\texttt{\ ratio\ }}
  )
\item
  value ( \href{/docs/reference/layout/ratio/}{\texttt{\ ratio\ }} )
\item
  alpha ( \href{/docs/reference/layout/ratio/}{\texttt{\ ratio\ }} )
\end{itemize}

These components are also available using the
\href{/docs/reference/visualize/color/\#definitions-components}{\texttt{\ components\ }}
method.

color { . } { hsv } (

{ \href{/docs/reference/layout/angle/}{angle} , } {
\href{/docs/reference/foundations/int/}{int}
\href{/docs/reference/layout/ratio/}{ratio} , } {
\href{/docs/reference/foundations/int/}{int}
\href{/docs/reference/layout/ratio/}{ratio} , } {
\href{/docs/reference/foundations/int/}{int}
\href{/docs/reference/layout/ratio/}{ratio} , } {
\href{/docs/reference/visualize/color/}{color} , }

) -\textgreater{} \href{/docs/reference/visualize/color/}{color}

\begin{verbatim}
#square(
  fill: color.hsv(30deg, 50%, 60%)
)
\end{verbatim}

\includegraphics[width=5in,height=\textheight,keepaspectratio]{/assets/docs/dEOjXMxlVX8xgAuMFF-gkQAAAAAAAAAA.png}

\paragraph{\texorpdfstring{\texttt{\ hue\ }}{ hue }}\label{definitions-hsv-hue}

\href{/docs/reference/layout/angle/}{angle}

{Required} {{ Positional }}

\phantomsection\label{definitions-hsv-hue-positional-tooltip}
Positional parameters are specified in order, without names.

The hue angle.

\paragraph{\texorpdfstring{\texttt{\ saturation\ }}{ saturation }}\label{definitions-hsv-saturation}

\href{/docs/reference/foundations/int/}{int} {or}
\href{/docs/reference/layout/ratio/}{ratio}

{Required} {{ Positional }}

\phantomsection\label{definitions-hsv-saturation-positional-tooltip}
Positional parameters are specified in order, without names.

The saturation component.

\paragraph{\texorpdfstring{\texttt{\ value\ }}{ value }}\label{definitions-hsv-value}

\href{/docs/reference/foundations/int/}{int} {or}
\href{/docs/reference/layout/ratio/}{ratio}

{Required} {{ Positional }}

\phantomsection\label{definitions-hsv-value-positional-tooltip}
Positional parameters are specified in order, without names.

The value component.

\paragraph{\texorpdfstring{\texttt{\ alpha\ }}{ alpha }}\label{definitions-hsv-alpha}

\href{/docs/reference/foundations/int/}{int} {or}
\href{/docs/reference/layout/ratio/}{ratio}

{Required} {{ Positional }}

\phantomsection\label{definitions-hsv-alpha-positional-tooltip}
Positional parameters are specified in order, without names.

The alpha component.

\paragraph{\texorpdfstring{\texttt{\ color\ }}{ color }}\label{definitions-hsv-color}

\href{/docs/reference/visualize/color/}{color}

{Required} {{ Positional }}

\phantomsection\label{definitions-hsv-color-positional-tooltip}
Positional parameters are specified in order, without names.

Alternatively: The color to convert to HSL.

If this is given, the individual components should not be given.

\subsubsection{\texorpdfstring{\texttt{\ components\ }}{ components }}\label{definitions-components}

Extracts the components of this color.

The size and values of this array depends on the color space. You can
obtain the color space using
\href{/docs/reference/visualize/color/\#definitions-space}{\texttt{\ space\ }}
. Below is a table of the color spaces and their components:

\begin{longtable}[]{@{}lllll@{}}
\toprule\noalign{}
Color space & C1 & C2 & C3 & C4 \\
\midrule\noalign{}
\endhead
\bottomrule\noalign{}
\endlastfoot
\href{/docs/reference/visualize/color/\#definitions-luma}{\texttt{\ luma\ }}
& Lightness & & & \\
\href{/docs/reference/visualize/color/\#definitions-oklab}{\texttt{\ oklab\ }}
& Lightness & \texttt{\ a\ } & \texttt{\ b\ } & Alpha \\
\href{/docs/reference/visualize/color/\#definitions-oklch}{\texttt{\ oklch\ }}
& Lightness & Chroma & Hue & Alpha \\
\href{/docs/reference/visualize/color/\#definitions-linear-rgb}{\texttt{\ linear-rgb\ }}
& Red & Green & Blue & Alpha \\
\href{/docs/reference/visualize/color/\#definitions-rgb}{\texttt{\ rgb\ }}
& Red & Green & Blue & Alpha \\
\href{/docs/reference/visualize/color/\#definitions-cmyk}{\texttt{\ cmyk\ }}
& Cyan & Magenta & Yellow & Key \\
\href{/docs/reference/visualize/color/\#definitions-hsl}{\texttt{\ hsl\ }}
& Hue & Saturation & Lightness & Alpha \\
\href{/docs/reference/visualize/color/\#definitions-hsv}{\texttt{\ hsv\ }}
& Hue & Saturation & Value & Alpha \\
\end{longtable}

For the meaning and type of each individual value, see the documentation
of the corresponding color space. The alpha component is optional and
only included if the \texttt{\ alpha\ } argument is \texttt{\ true\ } .
The length of the returned array depends on the number of components and
whether the alpha component is included.

self { . } { components } (

{ \hyperref[definitions-components-parameters-alpha]{alpha :}
\href{/docs/reference/foundations/bool/}{bool} }

) -\textgreater{} \href{/docs/reference/foundations/array/}{array}

\begin{verbatim}
// note that the alpha component is included by default
#rgb(40%, 60%, 80%).components()
\end{verbatim}

\includegraphics[width=5in,height=\textheight,keepaspectratio]{/assets/docs/dzB_dzQf4SM_Ou0eAcFH9AAAAAAAAAAA.png}

\paragraph{\texorpdfstring{\texttt{\ alpha\ }}{ alpha }}\label{definitions-components-alpha}

\href{/docs/reference/foundations/bool/}{bool}

Whether to include the alpha component.

Default: \texttt{\ }{\texttt{\ true\ }}\texttt{\ }

\subsubsection{\texorpdfstring{\texttt{\ space\ }}{ space }}\label{definitions-space}

Returns the constructor function for this color\textquotesingle s space:

\begin{itemize}
\tightlist
\item
  \href{/docs/reference/visualize/color/\#definitions-luma}{\texttt{\ luma\ }}
\item
  \href{/docs/reference/visualize/color/\#definitions-oklab}{\texttt{\ oklab\ }}
\item
  \href{/docs/reference/visualize/color/\#definitions-oklch}{\texttt{\ oklch\ }}
\item
  \href{/docs/reference/visualize/color/\#definitions-linear-rgb}{\texttt{\ linear-rgb\ }}
\item
  \href{/docs/reference/visualize/color/\#definitions-rgb}{\texttt{\ rgb\ }}
\item
  \href{/docs/reference/visualize/color/\#definitions-cmyk}{\texttt{\ cmyk\ }}
\item
  \href{/docs/reference/visualize/color/\#definitions-hsl}{\texttt{\ hsl\ }}
\item
  \href{/docs/reference/visualize/color/\#definitions-hsv}{\texttt{\ hsv\ }}
\end{itemize}

self { . } { space } (

) -\textgreater{} { any }

\begin{verbatim}
#let color = cmyk(1%, 2%, 3%, 4%)
#(color.space() == cmyk)
\end{verbatim}

\includegraphics[width=5in,height=\textheight,keepaspectratio]{/assets/docs/tfic_6Fu9JDbk4Tz2rYgKAAAAAAAAAAA.png}

\subsubsection{\texorpdfstring{\texttt{\ to-hex\ }}{ to-hex }}\label{definitions-to-hex}

Returns the color\textquotesingle s RGB(A) hex representation (such as
\texttt{\ \#ffaa32\ } or \texttt{\ \#020304fe\ } ). The alpha component
(last two digits in \texttt{\ \#020304fe\ } ) is omitted if it is equal
to \texttt{\ ff\ } (255 / 100\%).

self { . } { to-hex } (

) -\textgreater{} \href{/docs/reference/foundations/str/}{str}

\subsubsection{\texorpdfstring{\texttt{\ lighten\ }}{ lighten }}\label{definitions-lighten}

Lightens a color by a given factor.

self { . } { lighten } (

{ \href{/docs/reference/layout/ratio/}{ratio} }

) -\textgreater{} \href{/docs/reference/visualize/color/}{color}

\paragraph{\texorpdfstring{\texttt{\ factor\ }}{ factor }}\label{definitions-lighten-factor}

\href{/docs/reference/layout/ratio/}{ratio}

{Required} {{ Positional }}

\phantomsection\label{definitions-lighten-factor-positional-tooltip}
Positional parameters are specified in order, without names.

The factor to lighten the color by.

\subsubsection{\texorpdfstring{\texttt{\ darken\ }}{ darken }}\label{definitions-darken}

Darkens a color by a given factor.

self { . } { darken } (

{ \href{/docs/reference/layout/ratio/}{ratio} }

) -\textgreater{} \href{/docs/reference/visualize/color/}{color}

\paragraph{\texorpdfstring{\texttt{\ factor\ }}{ factor }}\label{definitions-darken-factor}

\href{/docs/reference/layout/ratio/}{ratio}

{Required} {{ Positional }}

\phantomsection\label{definitions-darken-factor-positional-tooltip}
Positional parameters are specified in order, without names.

The factor to darken the color by.

\subsubsection{\texorpdfstring{\texttt{\ saturate\ }}{ saturate }}\label{definitions-saturate}

Increases the saturation of a color by a given factor.

self { . } { saturate } (

{ \href{/docs/reference/layout/ratio/}{ratio} }

) -\textgreater{} \href{/docs/reference/visualize/color/}{color}

\paragraph{\texorpdfstring{\texttt{\ factor\ }}{ factor }}\label{definitions-saturate-factor}

\href{/docs/reference/layout/ratio/}{ratio}

{Required} {{ Positional }}

\phantomsection\label{definitions-saturate-factor-positional-tooltip}
Positional parameters are specified in order, without names.

The factor to saturate the color by.

\subsubsection{\texorpdfstring{\texttt{\ desaturate\ }}{ desaturate }}\label{definitions-desaturate}

Decreases the saturation of a color by a given factor.

self { . } { desaturate } (

{ \href{/docs/reference/layout/ratio/}{ratio} }

) -\textgreater{} \href{/docs/reference/visualize/color/}{color}

\paragraph{\texorpdfstring{\texttt{\ factor\ }}{ factor }}\label{definitions-desaturate-factor}

\href{/docs/reference/layout/ratio/}{ratio}

{Required} {{ Positional }}

\phantomsection\label{definitions-desaturate-factor-positional-tooltip}
Positional parameters are specified in order, without names.

The factor to desaturate the color by.

\subsubsection{\texorpdfstring{\texttt{\ negate\ }}{ negate }}\label{definitions-negate}

Produces the complementary color using a provided color space. You can
think of it as the opposite side on a color wheel.

self { . } { negate } (

{ \hyperref[definitions-negate-parameters-space]{space :} { any } }

) -\textgreater{} \href{/docs/reference/visualize/color/}{color}

\begin{verbatim}
#square(fill: yellow)
#square(fill: yellow.negate())
#square(fill: yellow.negate(space: rgb))
\end{verbatim}

\includegraphics[width=5in,height=\textheight,keepaspectratio]{/assets/docs/oBWZW_i_eZ8A9K_46wXLaQAAAAAAAAAA.png}

\paragraph{\texorpdfstring{\texttt{\ space\ }}{ space }}\label{definitions-negate-space}

{ any }

The color space used for the transformation. By default, a perceptual
color space is used.

Default: \texttt{\ oklab\ }

\subsubsection{\texorpdfstring{\texttt{\ rotate\ }}{ rotate }}\label{definitions-rotate}

Rotates the hue of the color by a given angle.

self { . } { rotate } (

{ \href{/docs/reference/layout/angle/}{angle} , } {
\hyperref[definitions-rotate-parameters-space]{space :} { any } , }

) -\textgreater{} \href{/docs/reference/visualize/color/}{color}

\paragraph{\texorpdfstring{\texttt{\ angle\ }}{ angle }}\label{definitions-rotate-angle}

\href{/docs/reference/layout/angle/}{angle}

{Required} {{ Positional }}

\phantomsection\label{definitions-rotate-angle-positional-tooltip}
Positional parameters are specified in order, without names.

The angle to rotate the hue by.

\paragraph{\texorpdfstring{\texttt{\ space\ }}{ space }}\label{definitions-rotate-space}

{ any }

The color space used to rotate. By default, this happens in a perceptual
color space (
\href{/docs/reference/visualize/color/\#definitions-oklch}{\texttt{\ oklch\ }}
).

Default: \texttt{\ oklch\ }

\subsubsection{\texorpdfstring{\texttt{\ mix\ }}{ mix }}\label{definitions-mix}

Create a color by mixing two or more colors.

In color spaces with a hue component (hsl, hsv, oklch), only two colors
can be mixed at once. Mixing more than two colors in such a space will
result in an error!

color { . } { mix } (

{ \hyperref[definitions-mix-parameters-colors]{..}
\href{/docs/reference/visualize/color/}{color}
\href{/docs/reference/foundations/array/}{array} , } {
\hyperref[definitions-mix-parameters-space]{space :} { any } , }

) -\textgreater{} \href{/docs/reference/visualize/color/}{color}

\begin{verbatim}
#set block(height: 20pt, width: 100%)
#block(fill: red.mix(blue))
#block(fill: red.mix(blue, space: rgb))
#block(fill: color.mix(red, blue, white))
#block(fill: color.mix((red, 70%), (blue, 30%)))
\end{verbatim}

\includegraphics[width=5in,height=\textheight,keepaspectratio]{/assets/docs/0jAT6gZPo0X02CVXUm7YpAAAAAAAAAAA.png}

\paragraph{\texorpdfstring{\texttt{\ colors\ }}{ colors }}\label{definitions-mix-colors}

\href{/docs/reference/visualize/color/}{color} {or}
\href{/docs/reference/foundations/array/}{array}

{Required} {{ Positional }}

\phantomsection\label{definitions-mix-colors-positional-tooltip}
Positional parameters are specified in order, without names.

{{ Variadic }}

\phantomsection\label{definitions-mix-colors-variadic-tooltip}
Variadic parameters can be specified multiple times.

The colors, optionally with weights, specified as a pair (array of
length two) of color and weight (float or ratio).

The weights do not need to add to
\texttt{\ }{\texttt{\ 100\%\ }}\texttt{\ } , they are relative to the
sum of all weights.

\paragraph{\texorpdfstring{\texttt{\ space\ }}{ space }}\label{definitions-mix-space}

{ any }

The color space to mix in. By default, this happens in a perceptual
color space (
\href{/docs/reference/visualize/color/\#definitions-oklab}{\texttt{\ oklab\ }}
).

Default: \texttt{\ oklab\ }

\subsubsection{\texorpdfstring{\texttt{\ transparentize\ }}{ transparentize }}\label{definitions-transparentize}

Makes a color more transparent by a given factor.

This method is relative to the existing alpha value. If the scale is
positive, calculates \texttt{\ alpha\ -\ alpha\ *\ scale\ } . Negative
scales behave like \texttt{\ color.opacify(-scale)\ } .

self { . } { transparentize } (

{ \href{/docs/reference/layout/ratio/}{ratio} }

) -\textgreater{} \href{/docs/reference/visualize/color/}{color}

\begin{verbatim}
#block(fill: red)[opaque]
#block(fill: red.transparentize(50%))[half red]
#block(fill: red.transparentize(75%))[quarter red]
\end{verbatim}

\includegraphics[width=5in,height=\textheight,keepaspectratio]{/assets/docs/bnNXhQKfjc4AYVaZ1T3e3wAAAAAAAAAA.png}

\paragraph{\texorpdfstring{\texttt{\ scale\ }}{ scale }}\label{definitions-transparentize-scale}

\href{/docs/reference/layout/ratio/}{ratio}

{Required} {{ Positional }}

\phantomsection\label{definitions-transparentize-scale-positional-tooltip}
Positional parameters are specified in order, without names.

The factor to change the alpha value by.

\subsubsection{\texorpdfstring{\texttt{\ opacify\ }}{ opacify }}\label{definitions-opacify}

Makes a color more opaque by a given scale.

This method is relative to the existing alpha value. If the scale is
positive, calculates \texttt{\ alpha\ +\ scale\ -\ alpha\ *\ scale\ } .
Negative scales behave like \texttt{\ color.transparentize(-scale)\ } .

self { . } { opacify } (

{ \href{/docs/reference/layout/ratio/}{ratio} }

) -\textgreater{} \href{/docs/reference/visualize/color/}{color}

\begin{verbatim}
#let half-red = red.transparentize(50%)
#block(fill: half-red.opacify(100%))[opaque]
#block(fill: half-red.opacify(50%))[three quarters red]
#block(fill: half-red.opacify(-50%))[one quarter red]
\end{verbatim}

\includegraphics[width=5in,height=\textheight,keepaspectratio]{/assets/docs/1fq--2OrISH1g8_dvUBroAAAAAAAAAAA.png}

\paragraph{\texorpdfstring{\texttt{\ scale\ }}{ scale }}\label{definitions-opacify-scale}

\href{/docs/reference/layout/ratio/}{ratio}

{Required} {{ Positional }}

\phantomsection\label{definitions-opacify-scale-positional-tooltip}
Positional parameters are specified in order, without names.

The scale to change the alpha value by.

\href{/docs/reference/visualize/circle/}{\pandocbounded{\includesvg[keepaspectratio]{/assets/icons/16-arrow-right.svg}}}

{ Circle } { Previous page }

\href{/docs/reference/visualize/ellipse/}{\pandocbounded{\includesvg[keepaspectratio]{/assets/icons/16-arrow-right.svg}}}

{ Ellipse } { Next page }






\section{C Docs LaTeX/docs/docs.tex}
\section{C Docs LaTeX/docs/tutorial.tex}
\section{Docs LaTeX/typst.app/docs/tutorial/advanced-styling.tex}
\title{typst.app/docs/tutorial/advanced-styling}

\begin{itemize}
\tightlist
\item
  \href{/docs}{\includesvg[width=0.16667in,height=0.16667in]{/assets/icons/16-docs-dark.svg}}
\item
  \includesvg[width=0.16667in,height=0.16667in]{/assets/icons/16-arrow-right.svg}
\item
  \href{/docs/tutorial/}{Tutorial}
\item
  \includesvg[width=0.16667in,height=0.16667in]{/assets/icons/16-arrow-right.svg}
\item
  \href{/docs/tutorial/advanced-styling/}{Advanced Styling}
\end{itemize}

\section{Advanced Styling}\label{advanced-styling}

In the previous two chapters of this tutorial, you have learned how to
write a document in Typst and how to change its formatting. The report
you wrote throughout the last two chapters got a straight A and your
supervisor wants to base a conference paper on it! The report will of
course have to comply with the conference\textquotesingle s style guide.
Let\textquotesingle s see how we can achieve that.

Before we start, let\textquotesingle s create a team, invite your
supervisor and add them to the team. You can do this by going back to
the app dashboard with the back icon in the top left corner of the
editor. Then, choose the plus icon in the left toolbar and create a
team. Finally, click on the new team and go to its settings by clicking
\textquotesingle manage team\textquotesingle{} next to the team name.
Now you can invite your supervisor by email.

\pandocbounded{\includegraphics[keepaspectratio]{/assets/docs/3-advanced-team-settings.png}}

Next, move your project into the team: Open it, going to its settings by
choosing the gear icon in the left toolbar and selecting your new team
from the owners dropdown. Don\textquotesingle t forget to save your
changes!

Now, your supervisor can also edit the project and you can both see the
changes in real time. You can join our
\href{https://discord.gg/2uDybryKPe}{Discord server} to find other users
and try teams with them!

\subsection{The conference guidelines}\label{guidelines}

The layout guidelines are available on the conference website.
Let\textquotesingle s take a look at them:

\begin{itemize}
\tightlist
\item
  The font should be an 11pt serif font
\item
  The title should be in 17pt and bold
\item
  The paper contains a single-column abstract and two-column main text
\item
  The abstract should be centered
\item
  The main text should be justified
\item
  First level section headings should be 13pt, centered, and rendered in
  small capitals
\item
  Second level headings are run-ins, italicized and have the same size
  as the body text
\item
  Finally, the pages should be US letter sized, numbered in the center
  of the footer and the top right corner of each page should contain the
  title of the paper
\end{itemize}

We already know how to do many of these things, but for some of them,
we\textquotesingle ll need to learn some new tricks.

\subsection{Writing the right set rules}\label{set-rules}

Let\textquotesingle s start by writing some set rules for the document.

\begin{verbatim}
#set page(
  paper: "us-letter",
  header: align(right)[
    A fluid dynamic model for
    glacier flow
  ],
  numbering: "1",
)
#set par(justify: true)
#set text(
  font: "Libertinus Serif",
  size: 11pt,
)

#lorem(600)
\end{verbatim}

\includegraphics[width=12.75in,height=\textheight,keepaspectratio]{/assets/docs/p6Vtj1ockTIzscSwa5_kewAAAAAAAAAA.png}

You are already familiar with most of what is going on here. We set the
text size to \texttt{\ }{\texttt{\ 11pt\ }}\texttt{\ } and the font to
Libertinus Serif. We also enable paragraph justification and set the
page size to US letter.

The \texttt{\ header\ } argument is new: With it, we can provide content
to fill the top margin of every page. In the header, we specify our
paper\textquotesingle s title as requested by the conference style
guide. We use the \texttt{\ align\ } function to align the text to the
right.

Last but not least is the \texttt{\ numbering\ } argument. Here, we can
provide a \href{/docs/reference/model/numbering/}{numbering pattern}
that defines how to number the pages. By setting into to
\texttt{\ }{\texttt{\ "1"\ }}\texttt{\ } , Typst only displays the bare
page number. Setting it to \texttt{\ }{\texttt{\ "(1/1)"\ }}\texttt{\ }
would have displayed the current page and total number of pages
surrounded by parentheses. And we could even have provided a completely
custom function here to format things to our liking.

\subsection{Creating a title and abstract}\label{title-and-abstract}

Now, let\textquotesingle s add a title and an abstract.
We\textquotesingle ll start with the title. We center align it and
increase its font weight by enclosing it in
\texttt{\ }{\texttt{\ *stars*\ }}\texttt{\ } .

\begin{verbatim}
#align(center, text(17pt)[
  *A fluid dynamic model
  for glacier flow*
])
\end{verbatim}

\includegraphics[width=6.25in,height=\textheight,keepaspectratio]{/assets/docs/EYkMw9AAwHWkDqrGblkKBgAAAAAAAAAA.png}

This looks right. We used the \texttt{\ text\ } function to override the
previous text set rule locally, increasing the size to 17pt for the
function\textquotesingle s argument. Let\textquotesingle s also add the
author list: Since we are writing this paper together with our
supervisor, we\textquotesingle ll add our own and their name.

\begin{verbatim}
#grid(
  columns: (1fr, 1fr),
  align(center)[
    Therese Tungsten \
    Artos Institute \
    #link("mailto:tung@artos.edu")
  ],
  align(center)[
    Dr. John Doe \
    Artos Institute \
    #link("mailto:doe@artos.edu")
  ]
)
\end{verbatim}

\includegraphics[width=6.25in,height=\textheight,keepaspectratio]{/assets/docs/Iwl_3LT7ijX6dcpL71YOWAAAAAAAAAAA.png}

The two author blocks are laid out next to each other. We use the
\href{/docs/reference/layout/grid/}{\texttt{\ grid\ }} function to
create this layout. With a grid, we can control exactly how large each
column is and which content goes into which cell. The
\texttt{\ columns\ } argument takes an array of
\href{/docs/reference/layout/relative/}{relative lengths} or
\href{/docs/reference/layout/fraction/}{fractions} . In this case, we
passed it two equal fractional sizes, telling it to split the available
space into two equal columns. We then passed two content arguments to
the grid function. The first with our own details, and the second with
our supervisors\textquotesingle. We again use the \texttt{\ align\ }
function to center the content within the column. The grid takes an
arbitrary number of content arguments specifying the cells. Rows are
added automatically, but they can also be manually sized with the
\texttt{\ rows\ } argument.

Now, let\textquotesingle s add the abstract. Remember that the
conference wants the abstract to be set ragged and centered.

\begin{verbatim}
...

#align(center)[
  #set par(justify: false)
  *Abstract* \
  #lorem(80)
]
\end{verbatim}

\includegraphics[width=12.75in,height=\textheight,keepaspectratio]{/assets/docs/4IdrVTeq86rbgvB-RNog6gAAAAAAAAAA.png}

Well done! One notable thing is that we used a set rule within the
content argument of \texttt{\ align\ } to turn off justification for the
abstract. This does not affect the remainder of the document even though
it was specified after the first set rule because content blocks
\emph{scope} styling. Anything set within a content block will only
affect the content within that block.

Another tweak could be to save the paper title in a variable, so that we
do not have to type it twice, for header and title. We can do that with
the \texttt{\ }{\texttt{\ let\ }}\texttt{\ } keyword:

\begin{verbatim}
#let title = [
  A fluid dynamic model
  for glacier flow
]

...

#set page(
  header: align(
    right + horizon,
    title
  ),
  ...
)

#align(center, text(17pt)[
  *#title*
])

...
\end{verbatim}

\includegraphics[width=12.75in,height=\textheight,keepaspectratio]{/assets/docs/ZKsZ2Eei-RUgPTLeJrMEYgAAAAAAAAAA.png}

After we bound the content to the \texttt{\ title\ } variable, we can
use it in functions and also within markup (prefixed by \texttt{\ \#\ }
, like functions). This way, if we decide on another title, we can
easily change it in one place.

\subsection{Adding columns and headings}\label{columns-and-headings}

The paper above unfortunately looks like a wall of lead. To fix that,
let\textquotesingle s add some headings and switch our paper to a
two-column layout. Fortunately, that\textquotesingle s easy to do: We
just need to amend our \texttt{\ page\ } set rule with the
\texttt{\ columns\ } argument.

By adding \texttt{\ columns:\ }{\texttt{\ 2\ }}\texttt{\ } to the
argument list, we have wrapped the whole document in two columns.
However, that would also affect the title and authors overview. To keep
them spanning the whole page, we can wrap them in a function call to
\href{/docs/reference/layout/place/}{\texttt{\ place\ }} . Place expects
an alignment and the content it should place as positional arguments.
Using the named \texttt{\ scope\ } argument, we can decide if the items
should be placed relative to the current column or its parent (the
page). There is one more thing to configure: If no other arguments are
provided, \texttt{\ place\ } takes its content out of the flow of the
document and positions it over the other content without affecting the
layout of other content in its container:

\begin{verbatim}
#place(
  top + center,
  rect(fill: black),
)
#lorem(30)
\end{verbatim}

\includegraphics[width=5in,height=\textheight,keepaspectratio]{/assets/docs/30s7cU9X36lW286rJXE3RwAAAAAAAAAA.png}

If we hadn\textquotesingle t used \texttt{\ place\ } here, the square
would be in its own line, but here it overlaps the few lines of text
following it. Likewise, that text acts like as if there was no square.
To change this behavior, we can pass the argument
\texttt{\ float:\ }{\texttt{\ true\ }}\texttt{\ } to ensure that the
space taken up by the placed item at the top or bottom of the page is
not occupied by any other content.

\begin{verbatim}
#set page(
  paper: "us-letter",
  header: align(
    right + horizon,
    title
  ),
  numbering: "1",
  columns: 2,
)

#place(
  top + center,
  float: true,
  scope: "parent",
  clearance: 2em,
)[
  ...

  #par(justify: false)[
    *Abstract* \
    #lorem(80)
  ]
]

= Introduction
#lorem(300)

= Related Work
#lorem(200)
\end{verbatim}

\includegraphics[width=12.75in,height=\textheight,keepaspectratio]{/assets/docs/dAJVP8paZmMvnK23cMA_0AAAAAAAAAAA.png}

In this example, we also used the \texttt{\ clearance\ } argument of the
\texttt{\ place\ } function to provide the space between it and the body
instead of using the \href{/docs/reference/layout/v/}{\texttt{\ v\ }}
function. We can also remove the explicit
\texttt{\ }{\texttt{\ align\ }}\texttt{\ }{\texttt{\ (\ }}\texttt{\ center\ }{\texttt{\ ,\ }}\texttt{\ }{\texttt{\ ..\ }}\texttt{\ }{\texttt{\ )\ }}\texttt{\ }
calls around the various parts since they inherit the center alignment
from the placement.

Now there is only one thing left to do: Style our headings. We need to
make them centered and use small capitals. Because the
\texttt{\ heading\ } function does not offer a way to set any of that,
we need to write our own heading show rule.

\begin{verbatim}
#show heading: it => [
  #set align(center)
  #set text(13pt, weight: "regular")
  #block(smallcaps(it.body))
]

...
\end{verbatim}

\includegraphics[width=5.52083in,height=\textheight,keepaspectratio]{/assets/docs/ZJxJWdUySZNKlj1_Vn1NWgAAAAAAAAAA.png}

This looks great! We used a show rule that applies to all headings. We
give it a function that gets passed the heading as a parameter. That
parameter can be used as content but it also has some fields like
\texttt{\ title\ } , \texttt{\ numbers\ } , and \texttt{\ level\ } from
which we can compose a custom look. Here, we are center-aligning,
setting the font weight to
\texttt{\ }{\texttt{\ "regular"\ }}\texttt{\ } because headings are bold
by default, and use the
\href{/docs/reference/text/smallcaps/}{\texttt{\ smallcaps\ }} function
to render the heading\textquotesingle s title in small capitals.

The only remaining problem is that all headings look the same now. The
"Motivation" and "Problem Statement" subsections ought to be italic run
in headers, but right now, they look indistinguishable from the section
headings. We can fix that by using a \texttt{\ where\ } selector on our
set rule: This is a \href{/docs/reference/scripting/\#methods}{method}
we can call on headings (and other elements) that allows us to filter
them by their level. We can use it to differentiate between section and
subsection headings:

\begin{verbatim}
#show heading.where(
  level: 1
): it => block(width: 100%)[
  #set align(center)
  #set text(13pt, weight: "regular")
  #smallcaps(it.body)
]

#show heading.where(
  level: 2
): it => text(
  size: 11pt,
  weight: "regular",
  style: "italic",
  it.body + [.],
)
\end{verbatim}

\includegraphics[width=5.52083in,height=\textheight,keepaspectratio]{/assets/docs/eBNymJDskGFkYVAXkF9cuAAAAAAAAAAA.png}

This looks great! We wrote two show rules that each selectively apply to
the first and second level headings. We used a \texttt{\ where\ }
selector to filter the headings by their level. We then rendered the
subsection headings as run-ins. We also automatically add a period to
the end of the subsection headings.

Let\textquotesingle s review the conference\textquotesingle s style
guide:

\begin{itemize}
\tightlist
\item
  The font should be an 11pt serif font âœ``
\item
  The title should be in 17pt and bold âœ``
\item
  The paper contains a single-column abstract and two-column main text
  âœ``
\item
  The abstract should be centered âœ``
\item
  The main text should be justified âœ``
\item
  First level section headings should be centered, rendered in small
  caps and in 13pt âœ``
\item
  Second level headings are run-ins, italicized and have the same size
  as the body text âœ``
\item
  Finally, the pages should be US letter sized, numbered in the center
  and the top right corner of each page should contain the title of the
  paper âœ``
\end{itemize}

We are now in compliance with all of these styles and can submit the
paper to the conference! The finished paper looks like this:

\pandocbounded{\includegraphics[keepaspectratio]{/assets/docs/3-advanced-paper.png}}

\subsection{Review}\label{review}

You have now learned how to create headers and footers, how to use
functions and scopes to locally override styles, how to create more
complex layouts with the
\href{/docs/reference/layout/grid/}{\texttt{\ grid\ }} function and how
to write show rules for individual functions, and the whole document.
You also learned how to use the
\href{/docs/reference/styling/\#show-rules}{\texttt{\ where\ } selector}
to filter the headings by their level.

The paper was a great success! You\textquotesingle ve met a lot of
like-minded researchers at the conference and are planning a project
which you hope to publish at the same venue next year.
You\textquotesingle ll need to write a new paper using the same style
guide though, so maybe now you want to create a time-saving template for
you and your team?

In the next section, we will learn how to create templates that can be
reused in multiple documents. This is a more advanced topic, so feel
free to come back to it later if you don\textquotesingle t feel up to it
right now.

\href{/docs/tutorial/formatting/}{\pandocbounded{\includesvg[keepaspectratio]{/assets/icons/16-arrow-right.svg}}}

{ Formatting } { Previous page }

\href{/docs/tutorial/making-a-template/}{\pandocbounded{\includesvg[keepaspectratio]{/assets/icons/16-arrow-right.svg}}}

{ Making a Template } { Next page }


\section{Docs LaTeX/typst.app/docs/tutorial/formatting.tex}
\title{typst.app/docs/tutorial/formatting}

\begin{itemize}
\tightlist
\item
  \href{/docs}{\includesvg[width=0.16667in,height=0.16667in]{/assets/icons/16-docs-dark.svg}}
\item
  \includesvg[width=0.16667in,height=0.16667in]{/assets/icons/16-arrow-right.svg}
\item
  \href{/docs/tutorial/}{Tutorial}
\item
  \includesvg[width=0.16667in,height=0.16667in]{/assets/icons/16-arrow-right.svg}
\item
  \href{/docs/tutorial/formatting/}{Formatting}
\end{itemize}

\section{Formatting}\label{formatting}

So far, you have written a report with some text, a few equations and
images. However, it still looks very plain. Your teaching assistant does
not yet know that you are using a new typesetting system, and you want
your report to fit in with the other student\textquotesingle s
submissions. In this chapter, we will see how to format your report
using Typst\textquotesingle s styling system.

\subsection{Set rules}\label{set-rules}

As we have seen in the previous chapter, Typst has functions that
\emph{insert} content (e.g. the
\href{/docs/reference/visualize/image/}{\texttt{\ image\ }} function)
and others that \emph{manipulate} content that they received as
arguments (e.g. the
\href{/docs/reference/layout/align/}{\texttt{\ align\ }} function). The
first impulse you might have when you want, for example, to justify the
report, could be to look for a function that does that and wrap the
complete document in it.

\begin{verbatim}
#par(justify: true)[
  = Background
  In the case of glaciers, fluid
  dynamics principles can be used
  to understand how the movement
  and behaviour of the ice is
  influenced by factors such as
  temperature, pressure, and the
  presence of other fluids (such as
  water).
]
\end{verbatim}

\includegraphics[width=5in,height=\textheight,keepaspectratio]{/assets/docs/Dijg8l-irnssXE7n_oJpJQAAAAAAAAAA.png}

Wait, shouldn\textquotesingle t all arguments of a function be specified
within parentheses? Why is there a second set of square brackets with
content \emph{after} the parentheses? The answer is that, as passing
content to a function is such a common thing to do in Typst, there is
special syntax for it: Instead of putting the content inside of the
argument list, you can write it in square brackets directly after the
normal arguments, saving on punctuation.

As seen above, that works. The
\href{/docs/reference/model/par/}{\texttt{\ par\ }} function justifies
all paragraphs within it. However, wrapping the document in countless
functions and applying styles selectively and in-situ can quickly become
cumbersome.

Fortunately, Typst has a more elegant solution. With \emph{set rules,}
you can apply style properties to all occurrences of some kind of
content. You write a set rule by entering the
\texttt{\ }{\texttt{\ set\ }}\texttt{\ } keyword, followed by the name
of the function whose properties you want to set, and a list of
arguments in parentheses.

\begin{verbatim}
#set par(justify: true)

= Background
In the case of glaciers, fluid
dynamics principles can be used
to understand how the movement
and behaviour of the ice is
influenced by factors such as
temperature, pressure, and the
presence of other fluids (such as
water).
\end{verbatim}

\includegraphics[width=5in,height=\textheight,keepaspectratio]{/assets/docs/JHqbSYpLaF9kuNFQoo1lAgAAAAAAAAAA.png}

Want to know in more technical terms what is happening here?

Set rules can be conceptualized as setting default values for some of
the parameters of a function for all future uses of that function.

\subsection{The autocomplete panel}\label{autocomplete}

If you followed along and tried a few things in the app, you might have
noticed that always after you enter a \texttt{\ \#\ } character, a panel
pops up to show you the available functions, and, within an argument
list, the available parameters. That\textquotesingle s the autocomplete
panel. It can be very useful while you are writing your document: You
can apply its suggestions by hitting the Return key or navigate to the
desired completion with the arrow keys. The panel can be dismissed by
hitting the Escape key and opened again by typing \texttt{\ \#\ } or
hitting { Ctrl } + { Space } . Use the autocomplete panel to discover
the right arguments for functions. Most suggestions come with a small
description of what they do.

\pandocbounded{\includegraphics[keepaspectratio]{/assets/docs/2-formatting-autocomplete.png}}

\subsection{Set up the page}\label{page-setup}

Back to set rules: When writing a rule, you choose the function
depending on what type of element you want to style. Here is a list of
some functions that are commonly used in set rules:

\begin{itemize}
\tightlist
\item
  \href{/docs/reference/text/text/}{\texttt{\ text\ }} to set font
  family, size, color, and other properties of text
\item
  \href{/docs/reference/layout/page/}{\texttt{\ page\ }} to set the page
  size, margins, headers, enable columns, and footers
\item
  \href{/docs/reference/model/par/}{\texttt{\ par\ }} to justify
  paragraphs, set line spacing, and more
\item
  \href{/docs/reference/model/heading/}{\texttt{\ heading\ }} to set the
  appearance of headings and enable numbering
\item
  \href{/docs/reference/model/document/}{\texttt{\ document\ }} to set
  the metadata contained in the PDF output, such as title and author
\end{itemize}

Not all function parameters can be set. In general, only parameters that
tell a function \emph{how} to do something can be set, not those that
tell it \emph{what} to do it with. The function reference pages indicate
which parameters are settable.

Let\textquotesingle s add a few more styles to our document. We want
larger margins and a serif font. For the purposes of the example,
we\textquotesingle ll also set another page size.

\begin{verbatim}
#set page(
  paper: "a6",
  margin: (x: 1.8cm, y: 1.5cm),
)
#set text(
  font: "New Computer Modern",
  size: 10pt
)
#set par(
  justify: true,
  leading: 0.52em,
)

= Introduction
In this report, we will explore the
various factors that influence fluid
dynamics in glaciers and how they
contribute to the formation and
behaviour of these natural structures.

...

#align(center + bottom)[
  #image("glacier.jpg", width: 70%)

  *Glaciers form an important
  part of the earth's climate
  system.*
]
\end{verbatim}

\includegraphics[width=6.19792in,height=\textheight,keepaspectratio]{/assets/docs/vXvjGwfGgpk5eo7U4CVWMQAAAAAAAAAA.png}

There are a few things of note here.

First is the \href{/docs/reference/layout/page/}{\texttt{\ page\ }} set
rule. It receives two arguments: the page size and margins for the page.
The page size is a string. Typst accepts
\href{/docs/reference/layout/page/\#parameters-paper}{many standard page
sizes,} but you can also specify a custom page size. The margins are
specified as a
\href{/docs/reference/foundations/dictionary/}{dictionary.} Dictionaries
are a collection of key-value pairs. In this case, the keys are
\texttt{\ x\ } and \texttt{\ y\ } , and the values are the horizontal
and vertical margins, respectively. We could also have specified
separate margins for each side by passing a dictionary with the keys
\texttt{\ left\ } , \texttt{\ right\ } , \texttt{\ top\ } , and
\texttt{\ bottom\ } .

Next is the set \href{/docs/reference/text/text/}{\texttt{\ text\ }} set
rule. Here, we set the font size to
\texttt{\ }{\texttt{\ 10pt\ }}\texttt{\ } and font family to
\texttt{\ }{\texttt{\ "New\ Computer\ Modern"\ }}\texttt{\ } . The Typst
app comes with many fonts that you can try for your document. When you
are in the text function\textquotesingle s argument list, you can
discover the available fonts in the autocomplete panel.

We have also set the spacing between lines (a.k.a. leading): It is
specified as a \href{/docs/reference/layout/length/}{length} value, and
we used the \texttt{\ em\ } unit to specify the leading relative to the
size of the font: \texttt{\ }{\texttt{\ 1em\ }}\texttt{\ } is equivalent
to the current font size (which defaults to
\texttt{\ }{\texttt{\ 11pt\ }}\texttt{\ } ).

Finally, we have bottom aligned our image by adding a vertical alignment
to our center alignment. Vertical and horizontal alignments can be
combined with the \texttt{\ }{\texttt{\ +\ }}\texttt{\ } operator to
yield a 2D alignment.

\subsection{A hint of sophistication}\label{sophistication}

To structure our document more clearly, we now want to number our
headings. We can do this by setting the \texttt{\ numbering\ } parameter
of the \href{/docs/reference/model/heading/}{\texttt{\ heading\ }}
function.

\begin{verbatim}
#set heading(numbering: "1.")

= Introduction
#lorem(10)

== Background
#lorem(12)

== Methods
#lorem(15)
\end{verbatim}

\includegraphics[width=5in,height=\textheight,keepaspectratio]{/assets/docs/4WtF0u81AczurIYrwpRdcwAAAAAAAAAA.png}

We specified the string \texttt{\ }{\texttt{\ "1."\ }}\texttt{\ } as the
numbering parameter. This tells Typst to number the headings with arabic
numerals and to put a dot between the number of each level. We can also
use \href{/docs/reference/model/numbering/}{letters, roman numerals, and
symbols} for our headings:

\begin{verbatim}
#set heading(numbering: "1.a")

= Introduction
#lorem(10)

== Background
#lorem(12)

== Methods
#lorem(15)
\end{verbatim}

\includegraphics[width=5in,height=\textheight,keepaspectratio]{/assets/docs/Llv0DrZ6U-QKf1vju2wP4QAAAAAAAAAA.png}

This example also uses the
\href{/docs/reference/text/lorem/}{\texttt{\ lorem\ }} function to
generate some placeholder text. This function takes a number as an
argument and generates that many words of \emph{Lorem Ipsum} text.

Did you wonder why the headings and text set rules apply to all text and
headings, even if they are not produced with the respective functions?

Typst internally calls the \texttt{\ heading\ } function every time you
write \texttt{\ }{\texttt{\ =\ Conclusion\ }}\texttt{\ } . In fact, the
function call
\texttt{\ }{\texttt{\ \#\ }}\texttt{\ }{\texttt{\ heading\ }}\texttt{\ }{\texttt{\ {[}\ }}\texttt{\ Conclusion\ }{\texttt{\ {]}\ }}\texttt{\ }
is equivalent to the heading markup above. Other markup elements work
similarly, they are only \emph{syntax sugar} for the corresponding
function calls.

\subsection{Show rules}\label{show-rules}

You are already pretty happy with how this turned out. But one last
thing needs to be fixed: The report you are writing is intended for a
larger project and that project\textquotesingle s name should always be
accompanied by a logo, even in prose.

You consider your options. You could add an
\texttt{\ }{\texttt{\ \#\ }}\texttt{\ }{\texttt{\ image\ }}\texttt{\ }{\texttt{\ (\ }}\texttt{\ }{\texttt{\ "logo.svg"\ }}\texttt{\ }{\texttt{\ )\ }}\texttt{\ }
call before every instance of the logo using search and replace. That
sounds very tedious. Instead, you could maybe
\href{/docs/reference/foundations/function/\#defining-functions}{define
a custom function} that always yields the logo with its image. However,
there is an even easier way:

With show rules, you can redefine how Typst displays certain elements.
You specify which elements Typst should show differently and how they
should look. Show rules can be applied to instances of text, many
functions, and even the whole document.

\begin{verbatim}
#show "ArtosFlow": name => box[
  #box(image(
    "logo.svg",
    height: 0.7em,
  ))
  #name
]

This report is embedded in the
ArtosFlow project. ArtosFlow is a
project of the Artos Institute.
\end{verbatim}

\includegraphics[width=5in,height=\textheight,keepaspectratio]{/assets/docs/349_Itx4-rTeNxzJmNofvgAAAAAAAAAA.png}

There is a lot of new syntax in this example: We write the
\texttt{\ }{\texttt{\ show\ }}\texttt{\ } keyword, followed by a string
of text we want to show differently and a colon. Then, we write a
function that takes the content that shall be shown as an argument.
Here, we called that argument \texttt{\ name\ } . We can now use the
\texttt{\ name\ } variable in the function\textquotesingle s body to
print the ArtosFlow name. Our show rule adds the logo image in front of
the name and puts the result into a box to prevent linebreaks from
occurring between logo and name. The image is also put inside of a box,
so that it does not appear in its own paragraph.

The calls to the first box function and the image function did not
require a leading \texttt{\ \#\ } because they were not embedded
directly in markup. When Typst expects code instead of markup, the
leading \texttt{\ \#\ } is not needed to access functions, keywords, and
variables. This can be observed in parameter lists, function
definitions, and \href{/docs/reference/scripting/}{code blocks} .

\subsection{Review}\label{review}

You now know how to apply basic formatting to your Typst documents. You
learned how to set the font, justify your paragraphs, change the page
dimensions, and add numbering to your headings with set rules. You also
learned how to use a basic show rule to change how text appears
throughout your document.

You have handed in your report. Your supervisor was so happy with it
that they want to adapt it into a conference paper! In the next section,
we will learn how to format your document as a paper using more advanced
show rules and functions.

\href{/docs/tutorial/writing-in-typst/}{\pandocbounded{\includesvg[keepaspectratio]{/assets/icons/16-arrow-right.svg}}}

{ Writing in Typst } { Previous page }

\href{/docs/tutorial/advanced-styling/}{\pandocbounded{\includesvg[keepaspectratio]{/assets/icons/16-arrow-right.svg}}}

{ Advanced Styling } { Next page }


\section{Docs LaTeX/typst.app/docs/tutorial/writing-in-typst.tex}
\title{typst.app/docs/tutorial/writing-in-typst}

\begin{itemize}
\tightlist
\item
  \href{/docs}{\includesvg[width=0.16667in,height=0.16667in]{/assets/icons/16-docs-dark.svg}}
\item
  \includesvg[width=0.16667in,height=0.16667in]{/assets/icons/16-arrow-right.svg}
\item
  \href{/docs/tutorial/}{Tutorial}
\item
  \includesvg[width=0.16667in,height=0.16667in]{/assets/icons/16-arrow-right.svg}
\item
  \href{/docs/tutorial/writing-in-typst/}{Writing in Typst}
\end{itemize}

\section{Writing in Typst}\label{writing-in-typst}

Let\textquotesingle s get started! Suppose you got assigned to write a
technical report for university. It will contain prose, maths, headings,
and figures. To get started, you create a new project on the Typst app.
You\textquotesingle ll be taken to the editor where you see two panels:
A source panel where you compose your document and a preview panel where
you see the rendered document.

\pandocbounded{\includegraphics[keepaspectratio]{/assets/docs/1-writing-app.png}}

You already have a good angle for your report in mind. So
let\textquotesingle s start by writing the introduction. Enter some text
in the editor panel. You\textquotesingle ll notice that the text
immediately appears on the previewed page.

\begin{verbatim}
In this report, we will explore the
various factors that influence fluid
dynamics in glaciers and how they
contribute to the formation and
behaviour of these natural structures.
\end{verbatim}

\includegraphics[width=5in,height=\textheight,keepaspectratio]{/assets/docs/ePl1U-2a7w8qkmb3CLl_oAAAAAAAAAAA.png}

\emph{Throughout this tutorial, we\textquotesingle ll show code examples
like this one. Just like in the app, the first panel contains markup and
the second panel shows a preview. We shrunk the page to fit the examples
so you can see what\textquotesingle s going on.}

The next step is to add a heading and emphasize some text. Typst uses
simple markup for the most common formatting tasks. To add a heading,
enter the \texttt{\ =\ } character and to emphasize some text with
italics, enclose it in
\texttt{\ }{\texttt{\ \_underscores\_\ }}\texttt{\ } .

\begin{verbatim}
= Introduction
In this report, we will explore the
various factors that influence _fluid
dynamics_ in glaciers and how they
contribute to the formation and
behaviour of these natural structures.
\end{verbatim}

\includegraphics[width=5in,height=\textheight,keepaspectratio]{/assets/docs/p75v-z7QqVChplB2N8HZfwAAAAAAAAAA.png}

That was easy! To add a new paragraph, just add a blank line in between
two lines of text. If that paragraph needs a subheading, produce it by
typing \texttt{\ ==\ } instead of \texttt{\ =\ } . The number of
\texttt{\ =\ } characters determines the nesting level of the heading.

Now we want to list a few of the circumstances that influence glacier
dynamics. To do that, we use a numbered list. For each item of the list,
we type a \texttt{\ +\ } character at the beginning of the line. Typst
will automatically number the items.

\begin{verbatim}
+ The climate
+ The topography
+ The geology
\end{verbatim}

\includegraphics[width=5in,height=\textheight,keepaspectratio]{/assets/docs/U3IHQbhSNQ8ndkXIv_gPrgAAAAAAAAAA.png}

If we wanted to add a bulleted list, we would use the \texttt{\ -\ }
character instead of the \texttt{\ +\ } character. We can also nest
lists: For example, we can add a sub-list to the first item of the list
above by indenting it.

\begin{verbatim}
+ The climate
  - Temperature
  - Precipitation
+ The topography
+ The geology
\end{verbatim}

\includegraphics[width=5in,height=\textheight,keepaspectratio]{/assets/docs/xmS-BPiM_gDHkWk9_uhE_gAAAAAAAAAA.png}

\subsection{Adding a figure}\label{figure}

You think that your report would benefit from a figure.
Let\textquotesingle s add one. Typst supports images in the formats PNG,
JPEG, GIF, and SVG. To add an image file to your project, first open the
\emph{file panel} by clicking the box icon in the left sidebar. Here,
you can see a list of all files in your project. Currently, there is
only one: The main Typst file you are writing in. To upload another
file, click the button with the arrow in the top-right corner. This
opens the upload dialog, in which you can pick files to upload from your
computer. Select an image file for your report.

\pandocbounded{\includegraphics[keepaspectratio]{/assets/docs/1-writing-upload.png}}

We have seen before that specific symbols (called \emph{markup} ) have
specific meaning in Typst. We can use \texttt{\ =\ } , \texttt{\ -\ } ,
\texttt{\ +\ } , and \texttt{\ \_\ } to create headings, lists and
emphasized text, respectively. However, having a special symbol for
everything we want to insert into our document would soon become cryptic
and unwieldy. For this reason, Typst reserves markup symbols only for
the most common things. Everything else is inserted with
\emph{functions.} For our image to show up on the page, we use
Typst\textquotesingle s
\href{/docs/reference/visualize/image/}{\texttt{\ image\ }} function.

\begin{verbatim}
#image("glacier.jpg")
\end{verbatim}

\includegraphics[width=5in,height=\textheight,keepaspectratio]{/assets/docs/KwKlYCVb2uFZqZ8abt3-ggAAAAAAAAAA.png}

In general, a function produces some output for a set of
\emph{arguments} . When you \emph{call} a function within markup, you
provide the arguments and Typst inserts the result (the
function\textquotesingle s \emph{return value} ) into the document. In
our case, the \texttt{\ image\ } function takes one argument: The path
to the image file. To call a function in markup, we first need to type
the \texttt{\ \#\ } character, immediately followed by the name of the
function. Then, we enclose the arguments in parentheses. Typst
recognizes many different data types within argument lists. Our file
path is a short \href{/docs/reference/foundations/str/}{string of text}
, so we need to enclose it in double quotes.

The inserted image uses the whole width of the page. To change that,
pass the \texttt{\ width\ } argument to the \texttt{\ image\ } function.
This is a \emph{named} argument and therefore specified as a
\texttt{\ name:\ value\ } pair. If there are multiple arguments, they
are separated by commas, so we first need to put a comma behind the
path.

\begin{verbatim}
#image("glacier.jpg", width: 70%)
\end{verbatim}

\includegraphics[width=5in,height=\textheight,keepaspectratio]{/assets/docs/lpadKIOzcEsf_MGoSeZghAAAAAAAAAAA.png}

The \texttt{\ width\ } argument is a
\href{/docs/reference/layout/relative/}{relative length} . In our case,
we specified a percentage, determining that the image shall take up
\texttt{\ }{\texttt{\ 70\%\ }}\texttt{\ } of the page\textquotesingle s
width. We also could have specified an absolute value like
\texttt{\ }{\texttt{\ 1cm\ }}\texttt{\ } or
\texttt{\ }{\texttt{\ 0.7in\ }}\texttt{\ } .

Just like text, the image is now aligned at the left side of the page by
default. It\textquotesingle s also lacking a caption.
Let\textquotesingle s fix that by using the
\href{/docs/reference/model/figure/}{figure} function. This function
takes the figure\textquotesingle s contents as a positional argument and
an optional caption as a named argument.

Within the argument list of the \texttt{\ figure\ } function, Typst is
already in code mode. This means, you now have to remove the hash before
the image function call. The hash is only needed directly in markup (to
disambiguate text from function calls).

The caption consists of arbitrary markup. To give markup to a function,
we enclose it in square brackets. This construct is called a
\emph{content block.}

\begin{verbatim}
#figure(
  image("glacier.jpg", width: 70%),
  caption: [
    _Glaciers_ form an important part
    of the earth's climate system.
  ],
)
\end{verbatim}

\includegraphics[width=5in,height=\textheight,keepaspectratio]{/assets/docs/v5OnReUO8fD5Rfj2aJZVyQAAAAAAAAAA.png}

You continue to write your report and now want to reference the figure.
To do that, first attach a label to figure. A label uniquely identifies
an element in your document. Add one after the figure by enclosing some
name in angle brackets. You can then reference the figure in your text
by writing an \texttt{\ }{\texttt{\ @\ }}\texttt{\ } symbol followed by
that name. Headings and equations can also be labelled to make them
referenceable.

\begin{verbatim}
Glaciers as the one shown in
@glaciers will cease to exist if
we don't take action soon!

#figure(
  image("glacier.jpg", width: 70%),
  caption: [
    _Glaciers_ form an important part
    of the earth's climate system.
  ],
) <glaciers>
\end{verbatim}

\includegraphics[width=5in,height=\textheight,keepaspectratio]{/assets/docs/cwZ12iQ39B4L-_wQwhO2TAAAAAAAAAAA.png}

So far, we\textquotesingle ve passed content blocks (markup in square
brackets) and strings (text in double quotes) to our functions. Both
seem to contain text. What\textquotesingle s the difference?

A content block can contain text, but also any other kind of markup,
function calls, and more, whereas a string is really just a
\emph{sequence of characters} and nothing else.

For example, the image function expects a path to an image file. It
would not make sense to pass, e.g., a paragraph of text or another image
as the image\textquotesingle s path parameter. That\textquotesingle s
why only strings are allowed here. On the contrary, strings work
wherever content is expected because text is a valid kind of content.

\subsection{Adding a bibliography}\label{bibliography}

As you write up your report, you need to back up some of your claims.
You can add a bibliography to your document with the
\href{/docs/reference/model/bibliography/}{\texttt{\ bibliography\ }}
function. This function expects a path to a bibliography file.

Typst\textquotesingle s native bibliography format is
\href{https://github.com/typst/hayagriva/blob/main/docs/file-format.md}{Hayagriva}
, but for compatibility you can also use BibLaTeX files. As your
classmate has already done a literature survey and sent you a
\texttt{\ .bib\ } file, you\textquotesingle ll use that one. Upload the
file through the file panel to access it in Typst.

Once the document contains a bibliography, you can start citing from it.
Citations use the same syntax as references to a label. As soon as you
cite a source for the first time, it will appear in the bibliography
section of your document. Typst supports different citation and
bibliography styles. Consult the
\href{/docs/reference/model/bibliography/\#parameters-style}{reference}
for more details.

\begin{verbatim}
= Methods
We follow the glacier melting models
established in @glacier-melt.

#bibliography("works.bib")
\end{verbatim}

\includegraphics[width=5in,height=\textheight,keepaspectratio]{/assets/docs/QGPHT14ksdea0r_8vy01WAAAAAAAAAAA.png}

\subsection{Maths}\label{maths}

After fleshing out the methods section, you move on to the meat of the
document: Your equations. Typst has built-in mathematical typesetting
and uses its own math notation. Let\textquotesingle s start with a
simple equation. We wrap it in \texttt{\ \$\ } signs to let Typst know
it should expect a mathematical expression:

\begin{verbatim}
The equation $Q = rho A v + C$
defines the glacial flow rate.
\end{verbatim}

\includegraphics[width=5in,height=\textheight,keepaspectratio]{/assets/docs/_u5BjLoMFBZU2zg1OWULdgAAAAAAAAAA.png}

The equation is typeset inline, on the same line as the surrounding
text. If you want to have it on its own line instead, you should insert
a single space at its start and end:

\begin{verbatim}
The flow rate of a glacier is
defined by the following equation:

$ Q = rho A v + C $
\end{verbatim}

\includegraphics[width=5in,height=\textheight,keepaspectratio]{/assets/docs/GXI0mvGOqqSC165iRTK-QwAAAAAAAAAA.png}

We can see that Typst displayed the single letters \texttt{\ Q\ } ,
\texttt{\ A\ } , \texttt{\ v\ } , and \texttt{\ C\ } as-is, while it
translated \texttt{\ rho\ } into a Greek letter. Math mode will always
show single letters verbatim. Multiple letters, however, are interpreted
as symbols, variables, or function names. To imply a multiplication
between single letters, put spaces between them.

If you want to have a variable that consists of multiple letters, you
can enclose it in quotes:

\begin{verbatim}
The flow rate of a glacier is given
by the following equation:

$ Q = rho A v + "time offset" $
\end{verbatim}

\includegraphics[width=5in,height=\textheight,keepaspectratio]{/assets/docs/JSaojGBiKH-FLbIYqeWSgAAAAAAAAAAA.png}

You\textquotesingle ll also need a sum formula in your paper. We can use
the \texttt{\ sum\ } symbol and then specify the range of the summation
in sub- and superscripts:

\begin{verbatim}
Total displaced soil by glacial flow:

$ 7.32 beta +
  sum_(i=0)^nabla Q_i / 2 $
\end{verbatim}

\includegraphics[width=5in,height=\textheight,keepaspectratio]{/assets/docs/rTDyTGxJlXKPRHJub3ALRgAAAAAAAAAA.png}

To add a subscript to a symbol or variable, type a \texttt{\ \_\ }
character and then the subscript. Similarly, use the \texttt{\ \^{}\ }
character for a superscript. If your sub- or superscript consists of
multiple things, you must enclose them in round parentheses.

The above example also showed us how to insert fractions: Simply put a
\texttt{\ /\ } character between the numerator and the denominator and
Typst will automatically turn it into a fraction. Parentheses are
smartly resolved, so you can enter your expression as you would into a
calculator and Typst will replace parenthesized sub-expressions with the
appropriate notation.

\begin{verbatim}
Total displaced soil by glacial flow:

$ 7.32 beta +
  sum_(i=0)^nabla
    (Q_i (a_i - epsilon)) / 2 $
\end{verbatim}

\includegraphics[width=5in,height=\textheight,keepaspectratio]{/assets/docs/HgeB2Bx5Lh3a5NPfF6WEdwAAAAAAAAAA.png}

Not all math constructs have special syntax. Instead, we use functions,
just like the \texttt{\ image\ } function we have seen before. For
example, to insert a column vector, we can use the
\href{/docs/reference/math/vec/}{\texttt{\ vec\ }} function. Within math
mode, function calls don\textquotesingle t need to start with the
\texttt{\ \#\ } character.

\begin{verbatim}
$ v := vec(x_1, x_2, x_3) $
\end{verbatim}

\includegraphics[width=5in,height=\textheight,keepaspectratio]{/assets/docs/nj0pMnkuoX2t5FZ3_x6YKwAAAAAAAAAA.png}

Some functions are only available within math mode. For example, the
\href{/docs/reference/math/variants/\#functions-cal}{\texttt{\ cal\ }}
function is used to typeset calligraphic letters commonly used for sets.
The \href{/docs/reference/math/}{math section of the reference} provides
a complete list of all functions that math mode makes available.

One more thing: Many symbols, such as the arrow, have a lot of variants.
You can select among these variants by appending a dot and a modifier
name to a symbol\textquotesingle s name:

\begin{verbatim}
$ a arrow.squiggly b $
\end{verbatim}

\includegraphics[width=5in,height=\textheight,keepaspectratio]{/assets/docs/0GgQitNz41j-75F3FS6iAwAAAAAAAAAA.png}

This notation is also available in markup mode, but the symbol name must
be preceded with \texttt{\ \#sym.\ } there. See the
\href{/docs/reference/symbols/sym/}{symbols section} for a list of all
available symbols.

\subsection{Review}\label{review}

You have now seen how to write a basic document in Typst. You learned
how to emphasize text, write lists, insert images, align content, and
typeset mathematical expressions. You also learned about
Typst\textquotesingle s functions. There are many more kinds of content
that Typst lets you insert into your document, such as
\href{/docs/reference/model/table/}{tables} ,
\href{/docs/reference/visualize/}{shapes} , and
\href{/docs/reference/text/raw/}{code blocks} . You can peruse the
\href{/docs/reference/}{reference} to learn more about these and other
features.

For the moment, you have completed writing your report. You have already
saved a PDF by clicking on the download button in the top right corner.
However, you think the report could look a bit less plain. In the next
section, we\textquotesingle ll learn how to customize the look of our
document.

\href{/docs/tutorial/}{\pandocbounded{\includesvg[keepaspectratio]{/assets/icons/16-arrow-right.svg}}}

{ Tutorial } { Previous page }

\href{/docs/tutorial/formatting/}{\pandocbounded{\includesvg[keepaspectratio]{/assets/icons/16-arrow-right.svg}}}

{ Formatting } { Next page }


\section{Docs LaTeX/typst.app/docs/tutorial/making-a-template.tex}
\title{typst.app/docs/tutorial/making-a-template}

\begin{itemize}
\tightlist
\item
  \href{/docs}{\includesvg[width=0.16667in,height=0.16667in]{/assets/icons/16-docs-dark.svg}}
\item
  \includesvg[width=0.16667in,height=0.16667in]{/assets/icons/16-arrow-right.svg}
\item
  \href{/docs/tutorial/}{Tutorial}
\item
  \includesvg[width=0.16667in,height=0.16667in]{/assets/icons/16-arrow-right.svg}
\item
  \href{/docs/tutorial/making-a-template/}{Making a Template}
\end{itemize}

\section{Making a Template}\label{making-a-template}

In the previous three chapters of this tutorial, you have learned how to
write a document in Typst, apply basic styles, and customize its
appearance in-depth to comply with a publisher\textquotesingle s style
guide. Because the paper you wrote in the previous chapter was a
tremendous success, you have been asked to write a follow-up article for
the same conference. This time, you want to take the style you created
in the previous chapter and turn it into a reusable template. In this
chapter you will learn how to create a template that you and your team
can use with just one show rule. Let\textquotesingle s get started!

\subsection{A toy template}\label{toy-template}

In Typst, templates are functions in which you can wrap your whole
document. To learn how to do that, let\textquotesingle s first review
how to write your very own functions. They can do anything you want them
to, so why not go a bit crazy?

\begin{verbatim}
#let amazed(term) = box[✨ #term ✨]

You are #amazed[beautiful]!
\end{verbatim}

\includegraphics[width=5in,height=\textheight,keepaspectratio]{/assets/docs/hf-0MuyTNtENvqMuqT5IlgAAAAAAAAAA.png}

This function takes a single argument, \texttt{\ term\ } , and returns a
content block with the \texttt{\ term\ } surrounded by sparkles. We also
put the whole thing in a box so that the term we are amazed by cannot be
separated from its sparkles by a line break.

Many functions that come with Typst have optional named parameters. Our
functions can also have them. Let\textquotesingle s add a parameter to
our function that lets us choose the color of the text. We need to
provide a default color in case the parameter isn\textquotesingle t
given.

\begin{verbatim}
#let amazed(term, color: blue) = {
  text(color, box[✨ #term ✨])
}

You are #amazed[beautiful]!
I am #amazed(color: purple)[amazed]!
\end{verbatim}

\includegraphics[width=5in,height=\textheight,keepaspectratio]{/assets/docs/DeOx9bmyxPapZywkKVbTFwAAAAAAAAAA.png}

Templates now work by wrapping our whole document in a custom function
like \texttt{\ amazed\ } . But wrapping a whole document in a giant
function call would be cumbersome! Instead, we can use an "everything"
show rule to achieve the same with cleaner code. To write such a show
rule, put a colon directly behind the show keyword and then provide a
function. This function is given the rest of the document as a
parameter. The function can then do anything with this content. Since
the \texttt{\ amazed\ } function can be called with a single content
argument, we can just pass it by name to the show rule.
Let\textquotesingle s try it:

\begin{verbatim}
#show: amazed
I choose to focus on the good
in my life and let go of any
negative thoughts or beliefs.
In fact, I am amazing!
\end{verbatim}

\includegraphics[width=5in,height=\textheight,keepaspectratio]{/assets/docs/gIv_i_LbdQ0VPwrL8LD78QAAAAAAAAAA.png}

Our whole document will now be passed to the \texttt{\ amazed\ }
function, as if we wrapped it around it. Of course, this is not
especially useful with this particular function, but when combined with
set rules and named arguments, it can be very powerful.

\subsection{Embedding set and show rules}\label{set-and-show-rules}

To apply some set and show rules to our template, we can use
\texttt{\ set\ } and \texttt{\ show\ } within a content block in our
function and then insert the document into that content block.

\begin{verbatim}
#let template(doc) = [
  #set text(font: "Inria Serif")
  #show "something cool": [Typst]
  #doc
]

#show: template
I am learning something cool today.
It's going great so far!
\end{verbatim}

\includegraphics[width=5in,height=\textheight,keepaspectratio]{/assets/docs/A-HDnb3ZV5ZLdSR0m_DP1QAAAAAAAAAA.png}

Just like we already discovered in the previous chapter, set rules will
apply to everything within their content block. Since the everything
show rule passes our whole document to the \texttt{\ template\ }
function, the text set rule and string show rule in our template will
apply to the whole document. Let\textquotesingle s use this knowledge to
create a template that reproduces the body style of the paper we wrote
in the previous chapter.

\begin{verbatim}
#let conf(title, doc) = {
  set page(
    paper: "us-letter",
    header: align(
      right + horizon,
      title
    ),
    columns: 2,
    ...
  )
  set par(justify: true)
  set text(
    font: "Libertinus Serif",
    size: 11pt,
  )

  // Heading show rules.
  ...

  doc
}

#show: doc => conf(
  [Paper title],
  doc,
)

= Introduction
#lorem(90)

...
\end{verbatim}

\includegraphics[width=12.75in,height=\textheight,keepaspectratio]{/assets/docs/Zq1nZR6oWo-01oCtP5oKDAAAAAAAAAAA.png}

We copy-pasted most of that code from the previous chapter. The two
differences are this:

\begin{enumerate}
\item
  We wrapped everything in the function \texttt{\ conf\ } using an
  everything show rule. The function applies a few set and show rules
  and echoes the content it has been passed at the end.
\item
  Moreover, we used a curly-braced code block instead of a content
  block. This way, we don\textquotesingle t need to prefix all set rules
  and function calls with a \texttt{\ \#\ } . In exchange, we cannot
  write markup directly in the code block anymore.
\end{enumerate}

Also note where the title comes from: We previously had it inside of a
variable. Now, we are receiving it as the first parameter of the
template function. To do so, we passed a closure (that\textquotesingle s
a function without a name that is used right away) to the everything
show rule. We did that because the \texttt{\ conf\ } function expects
two positional arguments, the title and the body, but the show rule will
only pass the body. Therefore, we add a new function definition that
allows us to set a paper title and use the single parameter from the
show rule.

\subsection{Templates with named arguments}\label{named-arguments}

Our paper in the previous chapter had a title and an author list.
Let\textquotesingle s add these things to our template. In addition to
the title, we want our template to accept a list of authors with their
affiliations and the paper\textquotesingle s abstract. To keep things
readable, we\textquotesingle ll add those as named arguments. In the
end, we want it to work like this:

\begin{verbatim}
#show: doc => conf(
  title: [Towards Improved Modelling],
  authors: (
    (
      name: "Theresa Tungsten",
      affiliation: "Artos Institute",
      email: "tung@artos.edu",
    ),
    (
      name: "Eugene Deklan",
      affiliation: "Honduras State",
      email: "e.deklan@hstate.hn",
    ),
  ),
  abstract: lorem(80),
  doc,
)

...
\end{verbatim}

Let\textquotesingle s build this new template function. First, we add a
default value to the \texttt{\ title\ } argument. This way, we can call
the template without specifying a title. We also add the named
\texttt{\ authors\ } and \texttt{\ abstract\ } parameters with empty
defaults. Next, we copy the code that generates title, abstract and
authors from the previous chapter into the template, replacing the fixed
details with the parameters.

The new \texttt{\ authors\ } parameter expects an
\href{/docs/reference/foundations/array/}{array} of
\href{/docs/reference/foundations/dictionary/}{dictionaries} with the
keys \texttt{\ name\ } , \texttt{\ affiliation\ } and \texttt{\ email\ }
. Because we can have an arbitrary number of authors, we dynamically
determine if we need one, two or three columns for the author list.
First, we determine the number of authors using the
\href{/docs/reference/foundations/array/\#definitions-len}{\texttt{\ .len()\ }}
method on the \texttt{\ authors\ } array. Then, we set the number of
columns as the minimum of this count and three, so that we never create
more than three columns. If there are more than three authors, a new row
will be inserted instead. For this purpose, we have also added a
\texttt{\ row-gutter\ } parameter to the \texttt{\ grid\ } function.
Otherwise, the rows would be too close together. To extract the details
about the authors from the dictionary, we use the
\href{/docs/reference/scripting/\#fields}{field access syntax} .

We still have to provide an argument to the grid for each author: Here
is where the array\textquotesingle s
\href{/docs/reference/foundations/array/\#definitions-map}{\texttt{\ map\ }
method} comes in handy. It takes a function as an argument that gets
called with each item of the array. We pass it a function that formats
the details for each author and returns a new array containing content
values. We\textquotesingle ve now got one array of values that
we\textquotesingle d like to use as multiple arguments for the grid. We
can do that by using the
\href{/docs/reference/foundations/arguments/}{\texttt{\ spread\ }
operator} . It takes an array and applies each of its items as a
separate argument to the function.

The resulting template function looks like this:

\begin{verbatim}
#let conf(
  title: none,
  authors: (),
  abstract: [],
  doc,
) = {
  // Set and show rules from before.
  ...

  set align(center)
  text(17pt, title)

  let count = authors.len()
  let ncols = calc.min(count, 3)
  grid(
    columns: (1fr,) * ncols,
    row-gutter: 24pt,
    ..authors.map(author => [
      #author.name \
      #author.affiliation \
      #link("mailto:" + author.email)
    ]),
  )

  par(justify: false)[
    *Abstract* \
    #abstract
  ]

  set align(left)
  doc
}
\end{verbatim}

\subsection{A separate file}\label{separate-file}

Most of the time, a template is specified in a different file and then
imported into the document. This way, the main file you write in is kept
clutter free and your template is easily reused. Create a new text file
in the file panel by clicking the plus button and name it
\texttt{\ conf.typ\ } . Move the \texttt{\ conf\ } function definition
inside of that new file. Now you can access it from your main file by
adding an import before the show rule. Specify the path of the file
between the \texttt{\ }{\texttt{\ import\ }}\texttt{\ } keyword and a
colon, then name the function that you want to import.

Another thing that you can do to make applying templates just a bit more
elegant is to use the
\href{/docs/reference/foundations/function/\#definitions-with}{\texttt{\ .with\ }}
method on functions to pre-populate all the named arguments. This way,
you can avoid spelling out a closure and appending the content argument
at the bottom of your template list. Templates on
\href{https://typst.app/universe/}{Typst Universe} are designed to work
with this style of function call.

\begin{verbatim}
#import "conf.typ": conf
#show: conf.with(
  title: [
    Towards Improved Modelling
  ],
  authors: (
    (
      name: "Theresa Tungsten",
      affiliation: "Artos Institute",
      email: "tung@artos.edu",
    ),
    (
      name: "Eugene Deklan",
      affiliation: "Honduras State",
      email: "e.deklan@hstate.hn",
    ),
  ),
  abstract: lorem(80),
)

= Introduction
#lorem(90)

== Motivation
#lorem(140)

== Problem Statement
#lorem(50)

= Related Work
#lorem(200)
\end{verbatim}

\includegraphics[width=12.75in,height=\textheight,keepaspectratio]{/assets/docs/BxllQV4yc0ikxppO7QP73AAAAAAAAAAA.png}

We have now converted the conference paper into a reusable template for
that conference! Why not share it in the
\href{https://forum.typst.app/}{Forum} or on
\href{https://discord.gg/2uDybryKPe}{Typst\textquotesingle s Discord
server} so that others can use it too?

\subsection{Review}\label{review}

Congratulations, you have completed Typst\textquotesingle s Tutorial! In
this section, you have learned how to define your own functions and how
to create and apply templates that define reusable document styles.
You\textquotesingle ve made it far and learned a lot. You can now use
Typst to write your own documents and share them with others.

We are still a super young project and are looking for feedback. If you
have any questions, suggestions or you found a bug, please let us know
in the \href{https://forum.typst.app/}{Forum} , on our
\href{https://discord.gg/2uDybryKPe}{Discord server} , on
\href{https://github.com/typst/typst/}{GitHub} , or via the web
app\textquotesingle s feedback form (always available in the Help menu).

So what are you waiting for? \href{https://typst.app}{Sign up} and write
something!

\href{/docs/tutorial/advanced-styling/}{\pandocbounded{\includesvg[keepaspectratio]{/assets/icons/16-arrow-right.svg}}}

{ Advanced Styling } { Previous page }

\href{/docs/reference/}{\pandocbounded{\includesvg[keepaspectratio]{/assets/icons/16-arrow-right.svg}}}

{ Reference } { Next page }




\section{C Docs LaTeX/docs/changelog.tex}
\section{Docs LaTeX/typst.app/docs/changelog/0.12.0.tex}
\title{typst.app/docs/changelog/0.12.0}

\begin{itemize}
\tightlist
\item
  \href{/docs}{\includesvg[width=0.16667in,height=0.16667in]{/assets/icons/16-docs-dark.svg}}
\item
  \includesvg[width=0.16667in,height=0.16667in]{/assets/icons/16-arrow-right.svg}
\item
  \href{/docs/changelog/}{Changelog}
\item
  \includesvg[width=0.16667in,height=0.16667in]{/assets/icons/16-arrow-right.svg}
\item
  \href{/docs/changelog/0.12.0/}{0.12.0}
\end{itemize}

\section{Version 0.12.0 (October 18,
2024)}\label{version-0.12.0-october-18-2024}

\subsection{Highlights}\label{highlights}

\begin{itemize}
\tightlist
\item
  Added support for multi-column floating
  \href{/docs/reference/layout/place/\#parameters-scope}{placement} and
  \href{/docs/reference/model/figure/\#parameters-scope}{figures}
\item
  Added support for automatic
  \href{/docs/reference/model/par/\#definitions-line}{line numbering}
  (often used in academic papers)
\item
  Typst\textquotesingle s layout engine is now multithreaded. Typical
  speedups are 2-3x for larger documents. The multithreading operates on
  page break boundaries, so explicit page breaks are necessary for it to
  kick in.
\item
  Paragraph justification was optimized with a new two-pass algorithm.
  Speedups are larger for shorter paragraphs and go up to 6x.
\item
  Highly reduced PDF file sizes due to better font subsetting (thanks to
  \href{https://github.com/LaurenzV}{@LaurenzV} )
\item
  Emoji are now exported properly in PDF
\item
  Added initial support for PDF/A. For now, only the PDF/A-2b profile is
  supported, but more is planned for the future.
\item
  Added various options for configuring the CLI\textquotesingle s
  environment (fonts, package paths, etc.)
\item
  Text show rules now match across multiple text elements
\item
  Block-level equations can now optionally break over multiple pages
\item
  Fixed a bug where some fonts would not print correctly on professional
  printers
\item
  Fixed a long-standing bug which could cause headings to be orphaned at
  the bottom of the page
\end{itemize}

\subsection{Layout}\label{layout}

\begin{itemize}
\tightlist
\item
  Added support for multi-column floating placement and figures via
  \href{/docs/reference/layout/place/\#parameters-scope}{\texttt{\ place.scope\ }}
  and
  \href{/docs/reference/model/figure/\#parameters-scope}{\texttt{\ figure.scope\ }}
  . Two-column documents should now prefer
  \texttt{\ }{\texttt{\ set\ }}\texttt{\ }{\texttt{\ page\ }}\texttt{\ }{\texttt{\ (\ }}\texttt{\ columns\ }{\texttt{\ :\ }}\texttt{\ }{\texttt{\ 2\ }}\texttt{\ }{\texttt{\ )\ }}\texttt{\ }
  over
  \texttt{\ }{\texttt{\ show\ }}\texttt{\ }{\texttt{\ :\ }}\texttt{\ column\ }{\texttt{\ .\ }}\texttt{\ }{\texttt{\ with\ }}\texttt{\ }{\texttt{\ (\ }}\texttt{\ }{\texttt{\ 2\ }}\texttt{\ }{\texttt{\ )\ }}\texttt{\ }
  (see the \href{/docs/guides/page-setup-guide/\#columns}{page setup
  guide} ).
\item
  Added support for automatic
  \href{/docs/reference/model/par/\#definitions-line}{line numbering}
  (often used in academic papers)
\item
  Added
  \href{/docs/reference/model/par/\#parameters-spacing}{\texttt{\ par.spacing\ }}
  property for configuring paragraph spacing. This should now be used
  instead of
  \texttt{\ }{\texttt{\ show\ }}\texttt{\ }{\texttt{\ par\ }}\texttt{\ }{\texttt{\ :\ }}\texttt{\ }{\texttt{\ set\ }}\texttt{\ }{\texttt{\ block\ }}\texttt{\ }{\texttt{\ (\ }}\texttt{\ spacing\ }{\texttt{\ :\ }}\texttt{\ }{\texttt{\ ..\ }}\texttt{\ }{\texttt{\ )\ }}\texttt{\ }
  \textbf{(Breaking change)}
\item
  Block-level elements like lists, grids, and stacks now show themselves
  as blocks and are thus affected by all block properties (e.g.
  \texttt{\ stroke\ } ) rather than just \texttt{\ spacing\ }
  \textbf{(Breaking change)}
\item
  Added
  \href{/docs/reference/layout/block/\#parameters-sticky}{\texttt{\ block.sticky\ }}
  property which prevents a page break after a block
\item
  Added
  \href{/docs/reference/layout/place/\#definitions-flush}{\texttt{\ place.flush\ }}
  function which forces all floating figures to be placed before any
  further content
\item
  Added \href{/docs/reference/layout/skew/}{\texttt{\ skew\ }} function
\item
  Added \texttt{\ }{\texttt{\ auto\ }}\texttt{\ } option for
  \href{/docs/reference/layout/page/\#parameters-header}{\texttt{\ page.header\ }}
  and
  \href{/docs/reference/layout/page/\#parameters-footer}{\texttt{\ page.footer\ }}
  which results in an automatic header/footer based on the numbering
  (which was previously inaccessible after a change)
\item
  Added \texttt{\ gap\ } and \texttt{\ justify\ } parameters to
  \href{/docs/reference/layout/repeat/}{\texttt{\ repeat\ }} function
\item
  Added \texttt{\ width\ } and \texttt{\ height\ } parameters to the
  \href{/docs/reference/layout/measure/}{\texttt{\ measure\ }} function
  to define the space in which the content should be measured.
  Especially useful in combination with
  \href{/docs/reference/layout/layout/}{\texttt{\ layout\ }} .
\item
  The height of a \texttt{\ block\ } , \texttt{\ image\ } ,
  \texttt{\ rect\ } , \texttt{\ square\ } , \texttt{\ ellipse\ } , or
  \texttt{\ circle\ } can now be specified in
  \href{/docs/reference/layout/fraction/}{fractional units}
\item
  The \href{/docs/reference/layout/scale/}{\texttt{\ scale\ }} function
  now supports absolute lengths for \texttt{\ x\ } , \texttt{\ y\ } ,
  \texttt{\ factor\ } . This way an element of unknown size can be
  scaled to a fixed size.
\item
  The values of \texttt{\ block.above\ } and \texttt{\ block.below\ }
  can now be retrieved in context expressions.
\item
  Increased accuracy of conversions between absolute units (pt, mm, cm,
  in)
\item
  Fixed a bug which could cause headings to be orphaned at the bottom of
  the page
\item
  Fixed footnotes within breakable blocks appearing on the page where
  the breakable block ends instead of at the page where the footnote
  marker is
\item
  Fixed numbering of nested footnotes and footnotes in floats
\item
  Fixed empty pages appearing when a
  \href{/docs/reference/context/}{context} expression wraps whole pages
\item
  Fixed
  \texttt{\ }{\texttt{\ set\ }}\texttt{\ }{\texttt{\ block\ }}\texttt{\ }{\texttt{\ (\ }}\texttt{\ spacing\ }{\texttt{\ :\ }}\texttt{\ x\ }{\texttt{\ )\ }}\texttt{\ }
  behaving differently from
  \texttt{\ }{\texttt{\ set\ }}\texttt{\ }{\texttt{\ block\ }}\texttt{\ }{\texttt{\ (\ }}\texttt{\ above\ }{\texttt{\ :\ }}\texttt{\ x\ }{\texttt{\ ,\ }}\texttt{\ below\ }{\texttt{\ :\ }}\texttt{\ x\ }{\texttt{\ )\ }}\texttt{\ }
\item
  Fixed behavior of
  \href{/docs/reference/layout/rotate/}{\texttt{\ rotate\ }} and
  \href{/docs/reference/layout/scale/}{\texttt{\ scale\ }} with
  \texttt{\ reflow:\ }{\texttt{\ true\ }}\texttt{\ }
\item
  Fixed interaction of
  \texttt{\ }{\texttt{\ align\ }}\texttt{\ }{\texttt{\ (\ }}\texttt{\ horizon\ }{\texttt{\ )\ }}\texttt{\ }
  and
  \texttt{\ }{\texttt{\ v\ }}\texttt{\ }{\texttt{\ (\ }}\texttt{\ }{\texttt{\ 1fr\ }}\texttt{\ }{\texttt{\ )\ }}\texttt{\ }
\item
  Fixed various bugs where floating placement would yield overlapping
  results
\item
  Fixed a bug where widow/orphan prevention would unnecessarily move
  text into the next column
\item
  Fixed \href{/docs/reference/layout/h/\#parameters-weak}{weak spacing}
  not being trimmed at the start and end of lines in a paragraph (only
  at the start and end of paragraphs)
\item
  Fixed interaction of weak page break and
  \href{/docs/reference/layout/pagebreak/\#parameters-to}{\texttt{\ pagebreak.to\ }}
\item
  Fixed compilation output of a single weak page break
\item
  Fixed crash when \href{/docs/reference/layout/pad/}{padding} by
  \texttt{\ }{\texttt{\ 100\%\ }}\texttt{\ }
\end{itemize}

\subsection{Text}\label{text}

\begin{itemize}
\tightlist
\item
  Tuned hyphenation: It is less eager by default and hyphenations close
  to the edges of words are now discouraged more strongly \textbf{(May
  lead to larger layout reflows)}
\item
  New default font: Libertinus Serif. This is the maintained successor
  to the old default font Linux Libertine. \textbf{(May lead to smaller
  reflows)}
\item
  Setting the font to an unavailable family will now result in a warning
\item
  Implemented a new smart quote algorithm, fixing various bugs where
  smart quotes weren\textquotesingle t all that smart
\item
  Added
  \href{/docs/reference/text/text/\#parameters-costs}{\texttt{\ text.costs\ }}
  parameter for tweaking various parameters that affect the choices of
  the layout engine during text layout
\item
  Added \texttt{\ typm\ } highlighting mode for math in
  \href{/docs/reference/text/raw/\#parameters-lang}{raw blocks}
\item
  Added basic i18n for Galician, Catalan, Latin, Icelandic, Hebrew
\item
  Implemented hyphenation duplication for Czech, Croatian, Lower
  Sorbian, Polish, Portuguese, Slovak, and Spanish.
\item
  The \href{/docs/reference/text/smallcaps/}{\texttt{\ smallcaps\ }}
  function is now an element function and can thereby be used in
  show(-set) rules.
\item
  The
  \href{/docs/reference/text/raw/\#parameters-theme}{\texttt{\ raw.theme\ }}
  parameter can now be set to \texttt{\ }{\texttt{\ none\ }}\texttt{\ }
  to disable highlighting even in the presence of a language tag, and to
  \texttt{\ }{\texttt{\ auto\ }}\texttt{\ } to reset to the default
\item
  Multiple
  \href{/docs/reference/text/text/\#parameters-stylistic-set}{stylistic
  sets} can now be enabled at once
\item
  Fixed the Chinese translation for "Equation"
\item
  Fixed that hyphenation could occur outside of words
\item
  Fixed incorrect layout of bidirectional text in edge cases
\item
  Fixed layout of paragraphs with explicit trailing whitespace
\item
  Fixed bugs related to empty paragraphs created via \texttt{\ \#""\ }
\item
  Fixed accidental trailing spaces for line breaks immediately preceding
  an inline equation
\item
  Fixed
  \href{/docs/reference/text/text/\#parameters-historical-ligatures}{\texttt{\ text.historical-ligatures\ }}
  not working correctly
\item
  Fixed accidental repetition of Thai characters around line breaks in
  some circumstances
\item
  Fixed \href{/docs/reference/text/smartquote/}{smart quotes} for Swiss
  French
\item
  New font metadata exceptions for Archivo, Kaiti SC, and Kaiti TC
\item
  Updated bundled New Computer Modern fonts to version 6.0
\end{itemize}

\subsection{Math}\label{math}

\begin{itemize}
\tightlist
\item
  Block-level equations can now break over multiple pages if enabled via
  \texttt{\ }{\texttt{\ show\ }}\texttt{\ math\ }{\texttt{\ .\ }}\texttt{\ }{\texttt{\ equation\ }}\texttt{\ }{\texttt{\ :\ }}\texttt{\ }{\texttt{\ set\ }}\texttt{\ }{\texttt{\ block\ }}\texttt{\ }{\texttt{\ (\ }}\texttt{\ breakable\ }{\texttt{\ :\ }}\texttt{\ }{\texttt{\ true\ }}\texttt{\ }{\texttt{\ )\ }}\texttt{\ }
  .
\item
  Matrix and vector sizing is now more consistent across different cell
  contents
\item
  Added \href{/docs/reference/math/stretch/}{\texttt{\ stretch\ }}
  function for manually or automatically stretching characters like
  arrows or parentheses horizontally or vertically
\item
  Improved layout of attachments on parenthesized as well as under- or
  overlined expressions
\item
  Improved layout of nested attachments resulting from code like
  \texttt{\ }{\texttt{\ \#\ }}\texttt{\ }{\texttt{\ let\ }}\texttt{\ a0\ }{\texttt{\ =\ }}\texttt{\ }{\texttt{\ \$\ }}\texttt{\ a\ }{\texttt{\ \_\ }}\texttt{\ 0\ }{\texttt{\ \$\ }}\texttt{\ }{\texttt{\ ;\ }}\texttt{\ }{\texttt{\ \$\ }}\texttt{\ }{\texttt{\ a0\ }}\texttt{\ }{\texttt{\ \^{}\ }}\texttt{\ 1\ }{\texttt{\ \$\ }}\texttt{\ }
\item
  Improved layout of primes close to superscripts
\item
  Improved layout of fractions
\item
  Typst now makes use of math-specific height-dependent kerning
  information in some fonts for better attachment layout
\item
  The \texttt{\ floor\ } and \texttt{\ ceil\ } functions in math are now
  callable symbols, such that
  \texttt{\ }{\texttt{\ \$\ }}\texttt{\ }{\texttt{\ floor\ }}\texttt{\ }{\texttt{\ (\ }}\texttt{\ x\ }{\texttt{\ )\ }}\texttt{\ =\ }{\texttt{\ lr\ }}\texttt{\ }{\texttt{\ (\ }}\texttt{\ }{\texttt{\ floor\ }}\texttt{\ }{\texttt{\ .\ }}\texttt{\ }{\texttt{\ l\ }}\texttt{\ x\ }{\texttt{\ floor\ }}\texttt{\ }{\texttt{\ .\ }}\texttt{\ }{\texttt{\ r\ }}\texttt{\ }{\texttt{\ )\ }}\texttt{\ }{\texttt{\ \$\ }}\texttt{\ }
\item
  The
  \href{/docs/reference/math/mat/\#parameters-delim}{\texttt{\ mat.delim\ }}
  ,
  \href{/docs/reference/math/vec/\#parameters-delim}{\texttt{\ vec.delim\ }}
  , and
  \href{/docs/reference/math/cases/\#parameters-delim}{\texttt{\ cases.delim\ }}
  parameters now allow any character that is considered a delimiter or
  "fence" (e.g. \textbar) by Unicode. The
  \texttt{\ delim:\ }{\texttt{\ "\textbar{}\textbar{}"\ }}\texttt{\ }
  notation is \emph{not} supported anymore and should be replaced by
  \texttt{\ delim:\ bar\ }{\texttt{\ .\ }}\texttt{\ double\ }
  \textbf{(Minor breaking change)}
\item
  Added
  \href{/docs/reference/math/vec/\#parameters-align}{\texttt{\ vec.align\ }}
  and
  \href{/docs/reference/math/mat/\#parameters-align}{\texttt{\ mat.align\ }}
  parameters
\item
  Added
  \href{/docs/reference/math/underover/\#functions-underparen}{\texttt{\ underparen\ }}
  ,
  \href{/docs/reference/math/underover/\#functions-overparen}{\texttt{\ overparen\ }}
  ,
  \href{/docs/reference/math/underover/\#functions-undershell}{\texttt{\ undershell\ }}
  , and
  \href{/docs/reference/math/underover/\#functions-overshell}{\texttt{\ overshell\ }}
\item
  Added \texttt{\ \textasciitilde{}\ } shorthand for
  \texttt{\ tilde.op\ } in math mode \textbf{(Minor breaking change)}
\item
  Fixed baseline alignment of equation numbers
\item
  Fixed positioning of corner brackets (⌜, �, ⌞, ⌟)
\item
  Fixed baseline of large roots
\item
  Fixed multiple minor layout bugs with attachments
\item
  Fixed that alignment points could affect line height in math
\item
  Fixed that spaces could show up between text and invisible elements
  like
  \href{/docs/reference/introspection/metadata/}{\texttt{\ metadata\ }}
  in math
\item
  Fixed a crash with recursive show rules in math
\item
  Fixed
  \href{/docs/reference/math/lr/\#functions-lr-size}{\texttt{\ lr.size\ }}
  not affecting characters enclosed in
  \href{/docs/reference/math/lr/\#functions-mid}{\texttt{\ mid\ }} in
  some cases
\item
  Fixed resolving of em units in sub- and superscripts
\item
  Fixed bounding box of inline equations when a
  \href{/docs/reference/text/text/\#parameters-top-edge}{text edge} is
  set to \texttt{\ }{\texttt{\ "bounds"\ }}\texttt{\ }
\end{itemize}

\subsection{Introspection}\label{introspection}

\begin{itemize}
\tightlist
\item
  Implemented a new system by which Typst tracks where elements end up
  on the pages. This may lead to subtly different behavior in
  introspections. \textbf{(Breaking change)}
\item
  Fixed various bugs with wrong counter behavior in complex layout
  situations, through a new, more principled implementation
\item
  Counter updates can now be before the first, in between, and after the
  last page when isolated by weak page breaks. This allows, for
  instance, updating a counter before the first page header and
  background.
\item
  Fixed logical ordering of introspections within footnotes and figures
\item
  Fixed incorrect
  \href{/docs/reference/introspection/here/}{\texttt{\ here().position()\ }}
  when \href{/docs/reference/layout/place/}{\texttt{\ place\ }} was used
  in a context expression
\item
  Fixed resolved positions of elements (in particular, headings) whose
  show rule emits an invisible element (like a state update) before a
  page break
\item
  Fixed behavior of stepping a counter at a deeper level than its
  current state has
\item
  Fixed citation formatting not working in table headers and a few other
  places
\item
  Displaying the footnote counter will now respect the footnote
  numbering style
\end{itemize}

\subsection{Model}\label{model}

\begin{itemize}
\tightlist
\item
  Document set rules do not need to be at the very start of the document
  anymore. The only restriction is that they must not occur inside of
  layout containers.
\item
  The \texttt{\ spacing\ } property of
  \href{/docs/reference/model/list/\#parameters-spacing}{lists} ,
  \href{/docs/reference/model/enum/\#parameters-spacing}{enumerations} ,
  and \href{/docs/reference/model/terms/\#parameters-spacing}{term
  lists} is now also respected for tight lists
\item
  Tight lists now only attach (with tighter spacing) to preceding
  paragraphs, not arbitrary blocks
\item
  The \href{/docs/reference/model/quote/}{\texttt{\ quote\ }} element is
  now locatable (can be used in queries)
\item
  The bibliography heading now uses \texttt{\ depth\ } instead of
  \texttt{\ level\ } so that its level can still be configured via a
  show-set rule
\item
  Added support for more
  \href{/docs/reference/model/numbering/}{numbering} formats:
  Devanagari, Eastern Arabic, Bengali, and circled numbers
\item
  Added
  \href{/docs/reference/model/heading/\#parameters-hanging-indent}{\texttt{\ hanging-indent\ }}
  parameter to heading function to tweak the appearance of multi-line
  headings and improved default appearance of multi-line headings
\item
  Improved handling of bidirectional text in outline entry
\item
  Fixed document set rules being ignored in an otherwise empty document
\item
  Fixed document set rules not being usable in context expressions
\item
  Fixed bad interaction between
  \texttt{\ }{\texttt{\ set\ }}\texttt{\ document\ } and
  \texttt{\ }{\texttt{\ set\ }}\texttt{\ page\ }
\item
  Fixed
  \texttt{\ }{\texttt{\ show\ }}\texttt{\ }{\texttt{\ figure\ }}\texttt{\ }{\texttt{\ :\ }}\texttt{\ }{\texttt{\ set\ }}\texttt{\ }{\texttt{\ align\ }}\texttt{\ }{\texttt{\ (\ }}\texttt{\ }{\texttt{\ ..\ }}\texttt{\ }{\texttt{\ )\ }}\texttt{\ }
  . Since the default figure alignment is now a show-set rule, it is not
  revoked by
  \texttt{\ }{\texttt{\ show\ }}\texttt{\ }{\texttt{\ figure\ }}\texttt{\ }{\texttt{\ :\ }}\texttt{\ it\ }{\texttt{\ =\textgreater{}\ }}\texttt{\ it\ }{\texttt{\ .\ }}\texttt{\ body\ }
  anymore. \textbf{(Minor breaking change)}
\item
  Fixed numbering of footnote references
\item
  Fixed spacing after bibliography heading
\end{itemize}

\subsection{Bibliography}\label{bibliography}

\begin{itemize}
\tightlist
\item
  The Hayagriva YAML \texttt{\ publisher\ } field can now accept a
  dictionary with a \texttt{\ location\ } key. The top-level
  \texttt{\ location\ } key is now primarily intended for event and item
  locations.
\item
  Multiple page ranges with prefixes and suffixes are now allowed
\item
  Added \texttt{\ director\ } and catch-all editor types to BibLaTeX
  parsing
\item
  Added support for disambiguation to alphanumeric citation style
\item
  The year 0 will now render as 1BC
\item
  Fixes for sorting of bibliography entries
\item
  Fixed pluralization of page range labels
\item
  Fixed sorting of citations by their number
\item
  Fixed how citation number ranges collapse
\item
  Fixed when the short form of a title is used
\item
  Fixed parsing of unbalanced dollars in BibLaTeX \texttt{\ url\ } field
\item
  Updated built-in citation styles
\end{itemize}

\subsection{Visualization}\label{visualization}

\begin{itemize}
\tightlist
\item
  Added \texttt{\ fill-rule\ } parameter to
  \href{/docs/reference/visualize/path/\#parameters-fill-rule}{\texttt{\ path\ }}
  and
  \href{/docs/reference/visualize/polygon/\#parameters-fill-rule}{\texttt{\ polygon\ }}
  functions
\item
  Fixed color mixing and gradients for
  \href{/docs/reference/visualize/color/\#definitions-luma}{Luma colors}
\item
  Fixed conversion from Luma to CMYK colors
\item
  Fixed offset gradient strokes in PNG export
\item
  Fixed unintended cropping of some SVGs
\item
  SVGs with foreign objects now produce a warning as they will likely
  not render correctly in Typst
\end{itemize}

\subsection{Syntax}\label{syntax}

\begin{itemize}
\tightlist
\item
  Added support for nested imports like
  \texttt{\ }{\texttt{\ import\ }}\texttt{\ }{\texttt{\ "file.typ"\ }}\texttt{\ }{\texttt{\ :\ }}\texttt{\ module\ }{\texttt{\ .\ }}\texttt{\ item\ }
\item
  Added support for parenthesized imports like
  \texttt{\ }{\texttt{\ import\ }}\texttt{\ }{\texttt{\ "file.typ"\ }}\texttt{\ }{\texttt{\ :\ }}\texttt{\ }{\texttt{\ (\ }}\texttt{\ a\ }{\texttt{\ ,\ }}\texttt{\ b\ }{\texttt{\ ,\ }}\texttt{\ c\ }{\texttt{\ )\ }}\texttt{\ }
  . With those, the import list can break over multiple lines.
\item
  Fixed edge case in parsing of reference syntax
\item
  Fixed edge case in parsing of heading, list, enum, and term markers
  immediately followed by comments
\item
  Fixed rare crash in parsing of parenthesized expressions
\end{itemize}

\subsection{Scripting}\label{scripting}

\begin{itemize}
\tightlist
\item
  Added new fixed-point
  \href{/docs/reference/foundations/decimal/}{\texttt{\ decimal\ }}
  number type for highly precise arithmetic on numbers in base 10, as
  needed for finance
\item
  Added \texttt{\ std\ } module for accessing standard library
  definitions even when a variable with the same name shadows/overwrites
  it
\item
  Added
  \href{/docs/reference/foundations/array/\#definitions-to-dict}{\texttt{\ array.to-dict\ }}
  ,
  \href{/docs/reference/foundations/array/\#definitions-reduce}{\texttt{\ array.reduce\ }}
  ,
  \href{/docs/reference/foundations/array/\#definitions-windows}{\texttt{\ array.windows\ }}
  methods
\item
  Added \texttt{\ exact\ } argument to
  \href{/docs/reference/foundations/array/\#definitions-zip}{\texttt{\ array.zip\ }}
\item
  Added
  \href{/docs/reference/foundations/arguments/\#definitions-at}{\texttt{\ arguments.at\ }}
  method
\item
  Added
  \href{/docs/reference/foundations/int/\#definitions-from-bytes}{\texttt{\ int.from-bytes\ }}
  ,
  \href{/docs/reference/foundations/int/\#definitions-to-bytes}{\texttt{\ int.to-bytes\ }}
  ,
  \href{/docs/reference/foundations/float/\#definitions-from-bytes}{\texttt{\ float.from-bytes\ }}
  , and
  \href{/docs/reference/foundations/float/\#definitions-to-bytes}{\texttt{\ float.to-bytes\ }}
\item
  Added proper support for negative values of the \texttt{\ digits\ }
  parameter of
  \href{/docs/reference/foundations/calc/\#functions-round}{\texttt{\ calc.round\ }}
  (the behaviour existed before but was subtly broken)
\item
  Conversions from
  \href{/docs/reference/foundations/int/}{\texttt{\ int\ }} to
  \href{/docs/reference/foundations/float/}{\texttt{\ float\ }} will now
  error instead of saturating if the float is too large \textbf{(Minor
  breaking change)}
\item
  Added \texttt{\ float.nan\ } and \texttt{\ float.inf\ } , removed
  \texttt{\ calc.nan\ } \textbf{(Minor breaking change)}
\item
  Certain symbols are now generally callable like functions and not only
  specifically in math. Examples are accents or
  \href{/docs/reference/math/lr/\#functions-floor}{\texttt{\ floor\ }}
  and \href{/docs/reference/math/lr/\#functions-ceil}{\texttt{\ ceil\ }}
  .
\item
  Improved \href{/docs/reference/foundations/repr/}{\texttt{\ repr\ }}
  of relative values, sequences, infinities, NaN,
  \texttt{\ }{\texttt{\ type\ }}\texttt{\ }{\texttt{\ (\ }}\texttt{\ }{\texttt{\ none\ }}\texttt{\ }{\texttt{\ )\ }}\texttt{\ }
  and
  \texttt{\ }{\texttt{\ type\ }}\texttt{\ }{\texttt{\ (\ }}\texttt{\ }{\texttt{\ auto\ }}\texttt{\ }{\texttt{\ )\ }}\texttt{\ }
\item
  Fixed crash on whole packages (rather than just files) cyclically
  importing each other
\item
  Fixed return type of
  \href{/docs/reference/foundations/calc/\#functions-round}{\texttt{\ calc.round\ }}
  on integers when a non-zero value is provided for \texttt{\ digits\ }
\end{itemize}

\subsection{Styling}\label{styling}

\begin{itemize}
\tightlist
\item
  Text show rules now match across multiple text elements
\item
  The string \texttt{\ "\ } in a text show rule now matches smart quotes
\item
  Fixed a long-standing styling bug where the header and footer would
  incorrectly inherit styles from a lone element on the page (e.g. a
  heading)
\item
  Fixed \texttt{\ }{\texttt{\ set\ }}\texttt{\ page\ } not working
  directly after a counter/state update
\item
  Page fields configured via an explicit
  \texttt{\ }{\texttt{\ page\ }}\texttt{\ }{\texttt{\ (\ }}\texttt{\ }{\texttt{\ ..\ }}\texttt{\ }{\texttt{\ )\ }}\texttt{\ }{\texttt{\ {[}\ }}\texttt{\ ..\ }{\texttt{\ {]}\ }}\texttt{\ }
  call can now be properly retrieved in context expressions
\end{itemize}

\subsection{Export}\label{export}

\begin{itemize}
\tightlist
\item
  Highly reduced PDF file sizes due to better font subsetting
\item
  Emoji are now exported properly in PDF
\item
  Added initial support for PDF/A. For now, only the standard PDF/A-2b
  is supported, but more is planned for the future. Enabled via
  \texttt{\ -\/-pdf-standard\ a-2b\ } in the CLI and via the UI in File
  \textgreater{} Export as \textgreater{} PDF in the web app.
\item
  Setting
  \href{/docs/reference/layout/page/\#parameters-fill}{\texttt{\ page.fill\ }}
  to \texttt{\ }{\texttt{\ none\ }}\texttt{\ } will now lead to
  transparent pages instead of white ones in PNG and SVG. The new
  default of \texttt{\ }{\texttt{\ auto\ }}\texttt{\ } means transparent
  for PDF and white for PNG and SVG.
\item
  Improved text copy-paste from PDF in complex scenarios
\item
  Exported SVGs now contain the \texttt{\ data-typst-label\ } attribute
  on groups resulting from labelled
  \href{/docs/reference/layout/box/}{boxes} and
  \href{/docs/reference/layout/block/}{blocks}
\item
  Fixed a bug where some fonts would not print correctly on professional
  printers
\item
  Fixed a bug where transparency could leak from one PDF object to
  another
\item
  Fixed a bug with CMYK gradients in PDF
\item
  Fixed various bugs with export of Oklab gradients in PDF
\item
  Fixed crashes related to rendering of non-outline glyphs
\item
  Two small fixes for PDF standard conformance
\end{itemize}

\subsection{Performance}\label{performance}

\begin{itemize}
\tightlist
\item
  Typst\textquotesingle s layout engine is now multithreaded. Typical
  speedups are 2-3x for larger documents. The multithreading operates on
  page break boundaries, so explicit page breaks are necessary for it to
  kick in.
\item
  Paragraph justification was optimized with a new two-pass algorithm.
  Speedups are larger for shorter paragraphs and range from 1-6x.
\end{itemize}

\subsection{Command Line Interface}\label{command-line-interface}

\begin{itemize}
\tightlist
\item
  Added \texttt{\ -\/-pages\ } option to select specific page ranges to
  export
\item
  Added \texttt{\ -\/-package-path\ } and
  \texttt{\ -\/-package-cache-path\ } as well as
  \texttt{\ TYPST\_PACKAGE\_PATH\ } and
  \texttt{\ TYPST\_PACKAGE\_CACHE\_PATH\ } environment variables for
  configuring where packages are loaded from and cached in, respectively
\item
  Added \texttt{\ -\/-ignore-system-fonts\ } flag to disable system
  fonts fully for better reproducibility
\item
  Added \texttt{\ -\/-make-deps\ } argument for outputting the
  dependencies of the current compilation as a Makefile
\item
  Added \texttt{\ -\/-pretty\ } option to \texttt{\ typst\ query\ } ,
  with the default now being to minify (only applies to JSON format)
\item
  Added \texttt{\ -\/-backup-path\ } to \texttt{\ typst\ update\ } to
  configure where the previous version is backed up
\item
  Added useful links to help output
\item
  The CLI will now greet users who invoke just \texttt{\ typst\ } for
  the first time
\item
  The document can now be written to stdout by passing \texttt{\ -\ } as
  the output filename (for PDF or single-page image export)
\item
  Typst will now emit a proper error message instead of failing silently
  when the certificate specified by \texttt{\ -\/-cert\ } or
  \texttt{\ TYPST\_CERT\ } could not be loaded
\item
  The CLI now respects the \texttt{\ SOURCE\_DATE\_EPOCH\ } environment
  variable for better reproducibility
\item
  When exporting multiple images, you can now use \texttt{\ t\ } (total
  pages), \texttt{\ p\ } (current page), and \texttt{\ 0p\ }
  (zero-padded current page, same as current \texttt{\ n\ } ) in the
  output path
\item
  The input and output paths now allow non-UTF-8 values
\item
  Times are now formatted more consistently across the CLI
\item
  Fixed a bug related to the \texttt{\ -\/-open\ } flag
\item
  Fixed path completions for \texttt{\ typst\ } not working in zsh
\end{itemize}

\subsection{Tooling and Diagnostics}\label{tooling-and-diagnostics}

\begin{itemize}
\tightlist
\item
  The "compiler" field for specifying the minimum Typst version required
  by a package now supports imprecise bounds like 0.11 instead of 0.11.0
\item
  Added warning when a label is ignored by Typst because no preceding
  labellable element exists
\item
  Added hint when trying to apply labels in code mode
\item
  Added hint when trying to call a standard library function that has
  been shadowed/overwritten by a local definition
\item
  Added hint when trying to set both the language and the region in the
  \texttt{\ lang\ } parameter
\item
  Added hints when trying to compile non-Typst files (e.g. after having
  typed \texttt{\ typst\ c\ file.pdf\ } by accident)
\item
  Added hint when a string is used where a label is expected
\item
  Added hint when a stray end of a block comment ( \texttt{\ */\ } ) is
  encountered
\item
  Added hints when destructuring arrays with the wrong number of
  elements
\item
  Improved error message when trying to use a keyword as an identifier
  in a let binding
\item
  Improved error messages when accessing nonexistent fields
\item
  Improved error message when a package exists, but not the specified
  version
\item
  Improved hints for unknown variables
\item
  Improved hint when trying to convert a length with non-zero em
  component to an absolute unit
\item
  Fixed a crash that could be triggered by certain hover tooltips
\item
  Fixed an off-by-one error in to-source jumps when first-line-indent is
  enabled
\item
  Fixed suggestions for \texttt{\ .\ } after the end of an inline code
  expressions
\item
  Fixed autocompletions being duplicated in a specific case
\end{itemize}

\subsection{Symbols}\label{symbols}

\begin{itemize}
\tightlist
\item
  New: \texttt{\ parallelogram\ } , \texttt{\ original\ } ,
  \texttt{\ image\ } , \texttt{\ crossmark\ } , \texttt{\ rest\ } ,
  \texttt{\ natural\ } , \texttt{\ flat\ } , \texttt{\ sharp\ } ,
  \texttt{\ tiny\ } , \texttt{\ miny\ } , \texttt{\ copyleft\ } ,
  \texttt{\ trademark\ } , \texttt{\ emoji.beet\ } ,
  \texttt{\ emoji.fingerprint\ } , \texttt{\ emoji.harp\ } ,
  \texttt{\ emoji.shovel\ } , \texttt{\ emoji.splatter\ } ,
  \texttt{\ emoji.tree.leafless\ } ,
\item
  New variants: \texttt{\ club.stroked\ } , \texttt{\ diamond.stroked\ }
  , \texttt{\ heart.stroked\ } , \texttt{\ spade.stroked\ } ,
  \texttt{\ gt.neq\ } , \texttt{\ lt.neq\ } ,
  \texttt{\ checkmark.heavy\ } , \texttt{\ paren.double\ } ,
  \texttt{\ brace.double\ } , \texttt{\ shell.double\ } ,
  \texttt{\ arrow.turn\ } , \texttt{\ plus.double\ } ,
  \texttt{\ plus.triple\ } , \texttt{\ infinity.bar\ } ,
  \texttt{\ infinity.incomplete\ } , \texttt{\ infinity.tie\ } ,
  \texttt{\ multimap.double\ } , \texttt{\ ballot.check\ } ,
  \texttt{\ ballot.check.heavy\ } , \texttt{\ emptyset.bar\ } ,
  \texttt{\ emptyset.circle\ } , \texttt{\ emptyset.arrow.l\ } ,
  \texttt{\ emptyset.arrow.r\ } , \texttt{\ parallel.struck\ } ,
  \texttt{\ parallel.eq\ } , \texttt{\ parallel.equiv\ } ,
  \texttt{\ parallel.slanted\ } , \texttt{\ parallel.tilde\ } ,
  \texttt{\ angle.l.curly\ } , \texttt{\ angle.l.dot\ } ,
  \texttt{\ angle.r.curly\ } , \texttt{\ angle.r.dot\ } ,
  \texttt{\ angle.oblique\ } , \texttt{\ angle.s\ } ,
  \texttt{\ em.two\ } , \texttt{\ em.three\ }
\item
  Renamed: \texttt{\ turtle\ } to \texttt{\ shell\ } ,
  \texttt{\ notes\ } to \texttt{\ note\ } , \texttt{\ ballot.x\ } to
  \texttt{\ ballot.cross\ } , \texttt{\ succ.eq\ } to
  \texttt{\ succ.curly.eq\ } , \texttt{\ prec.eq\ } to
  \texttt{\ prec.curly.eq\ } , \texttt{\ servicemark\ } to
  \texttt{\ trademark.service\ } , \texttt{\ emoji.face.tired\ } to
  \texttt{\ emoji.face.distress\ } \textbf{(Breaking change)}
\item
  Changed codepoint: \texttt{\ prec.eq\ } , \texttt{\ prec.neq\ } ,
  \texttt{\ succ.eq\ } , \texttt{\ succ.neq\ } , \texttt{\ triangle\ }
  from â--· to â--³, \texttt{\ emoji.face.tired\ } \textbf{(Breaking
  change)}
\item
  Removed: \texttt{\ lt.curly\ } in favor of \texttt{\ prec\ } ,
  \texttt{\ gt.curly\ } in favor of \texttt{\ succ\ } \textbf{(Breaking
  change)}
\end{itemize}

\subsection{Deprecations}\label{deprecations}

\begin{itemize}
\tightlist
\item
  \href{/docs/reference/introspection/counter/\#definitions-display}{\texttt{\ counter.display\ }}
  without an established context
\item
  \href{/docs/reference/introspection/counter/\#definitions-final}{\texttt{\ counter.final\ }}
  with a location
\item
  \href{/docs/reference/introspection/state/\#definitions-final}{\texttt{\ state.final\ }}
  with a location
\item
  \href{/docs/reference/introspection/state/\#definitions-display}{\texttt{\ state.display\ }}
\item
  \href{/docs/reference/introspection/query/}{\texttt{\ query\ }} with a
  location as the second argument
\item
  \href{/docs/reference/introspection/locate/}{\texttt{\ locate\ }} with
  a callback function
\item
  \href{/docs/reference/layout/measure/}{\texttt{\ measure\ }} with
  styles
\item
  \href{/docs/reference/foundations/style/}{\texttt{\ style\ }}
\end{itemize}

\subsection{Development}\label{development}

\begin{itemize}
\tightlist
\item
  Added \texttt{\ typst-kit\ } crate which provides useful APIs for
  \texttt{\ World\ } implementors
\item
  Added go-to-definition API in \texttt{\ typst-ide\ }
\item
  Added package manifest parsing APIs to \texttt{\ typst-syntax\ }
\item
  As the compiler is now capable of multithreading, \texttt{\ World\ }
  implementations must satisfy \texttt{\ Send\ } and \texttt{\ Sync\ }
\item
  Changed signature of \texttt{\ World::main\ } to allow for the
  scenario where the main file could not be loaded
\item
  Removed \texttt{\ Tracer\ } in favor of
  \texttt{\ Warned\textless{}T\textgreater{}\ } and
  \texttt{\ typst::trace\ } function
\item
  The \texttt{\ xz2\ } dependency used by the self-updater is now
  statically linked
\item
  The Dockerfile now has an \texttt{\ ENTRYPOINT\ } directive
\end{itemize}

\subsection{Contributors}\label{contributors}

Thanks to everyone who contributed to this release!

\begin{itemize}
\tightlist
\item
  \href{https://github.com/Leedehai}{\includegraphics[width=0.66667in,height=0.66667in]{https://avatars.githubusercontent.com/u/18319900?s=64&v=4}}
\item
  \href{https://github.com/MDLC01}{\includegraphics[width=0.66667in,height=0.66667in]{https://avatars.githubusercontent.com/u/57839069?s=64&v=4}}
\item
  \href{https://github.com/Coekjan}{\includegraphics[width=0.66667in,height=0.66667in]{https://avatars.githubusercontent.com/u/69834864?s=64&v=4}}
\item
  \href{https://github.com/bluebear94}{\includegraphics[width=0.66667in,height=0.66667in]{https://avatars.githubusercontent.com/u/2975203?s=64&v=4}}
\item
  \href{https://github.com/mkorje}{\includegraphics[width=0.66667in,height=0.66667in]{https://avatars.githubusercontent.com/u/97375244?s=64&v=4}}
\item
  \href{https://github.com/EpicEricEE}{\includegraphics[width=0.66667in,height=0.66667in]{https://avatars.githubusercontent.com/u/7191192?s=64&v=4}}
\item
  \href{https://github.com/PgBiel}{\includegraphics[width=0.66667in,height=0.66667in]{https://avatars.githubusercontent.com/u/9021226?s=64&v=4}}
\item
  \href{https://github.com/frozolotl}{\includegraphics[width=0.66667in,height=0.66667in]{https://avatars.githubusercontent.com/u/44589151?s=64&v=4}}
\item
  \href{https://github.com/elegaanz}{\includegraphics[width=0.66667in,height=0.66667in]{https://avatars.githubusercontent.com/u/16254623?s=64&v=4}}
\item
  \href{https://github.com/Dherse}{\includegraphics[width=0.66667in,height=0.66667in]{https://avatars.githubusercontent.com/u/9665250?s=64&v=4}}
\item
  \href{https://github.com/knuesel}{\includegraphics[width=0.66667in,height=0.66667in]{https://avatars.githubusercontent.com/u/2412819?s=64&v=4}}
\item
  \href{https://github.com/Andrew15-5}{\includegraphics[width=0.66667in,height=0.66667in]{https://avatars.githubusercontent.com/u/37143421?s=64&v=4}}
\item
  \href{https://github.com/Enter-tainer}{\includegraphics[width=0.66667in,height=0.66667in]{https://avatars.githubusercontent.com/u/25521218?s=64&v=4}}
\item
  \href{https://github.com/LaurenzV}{\includegraphics[width=0.66667in,height=0.66667in]{https://avatars.githubusercontent.com/u/47084093?s=64&v=4}}
\item
  \href{https://github.com/Myriad-Dreamin}{\includegraphics[width=0.66667in,height=0.66667in]{https://avatars.githubusercontent.com/u/35292584?s=64&v=4}}
\item
  \href{https://github.com/rikhuijzer}{\includegraphics[width=0.66667in,height=0.66667in]{https://avatars.githubusercontent.com/u/20724914?s=64&v=4}}
\item
  \href{https://github.com/ssotoen}{\includegraphics[width=0.66667in,height=0.66667in]{https://avatars.githubusercontent.com/u/68116836?s=64&v=4}}
\item
  \href{https://github.com/tingerrr}{\includegraphics[width=0.66667in,height=0.66667in]{https://avatars.githubusercontent.com/u/137803093?s=64&v=4}}
\item
  \href{https://github.com/FlorentCLMichel}{\includegraphics[width=0.66667in,height=0.66667in]{https://avatars.githubusercontent.com/u/56166507?s=64&v=4}}
\item
  \href{https://github.com/T0mstone}{\includegraphics[width=0.66667in,height=0.66667in]{https://avatars.githubusercontent.com/u/39707032?s=64&v=4}}
\item
  \href{https://github.com/drupol}{\includegraphics[width=0.66667in,height=0.66667in]{https://avatars.githubusercontent.com/u/252042?s=64&v=4}}
\item
  \href{https://github.com/emilyyyylime}{\includegraphics[width=0.66667in,height=0.66667in]{https://avatars.githubusercontent.com/u/40892795?s=64&v=4}}
\item
  \href{https://github.com/A-Walrus}{\includegraphics[width=0.66667in,height=0.66667in]{https://avatars.githubusercontent.com/u/58790821?s=64&v=4}}
\item
  \href{https://github.com/LuizAugustoPapa}{\includegraphics[width=0.66667in,height=0.66667in]{https://avatars.githubusercontent.com/u/112978478?s=64&v=4}}
\item
  \href{https://github.com/PepinhoJp}{\includegraphics[width=0.66667in,height=0.66667in]{https://avatars.githubusercontent.com/u/24254834?s=64&v=4}}
\item
  \href{https://github.com/freundTech}{\includegraphics[width=0.66667in,height=0.66667in]{https://avatars.githubusercontent.com/u/9515067?s=64&v=4}}
\item
  \href{https://github.com/gabriel-araujjo}{\includegraphics[width=0.66667in,height=0.66667in]{https://avatars.githubusercontent.com/u/3980936?s=64&v=4}}
\item
  \href{https://github.com/istudyatuni}{\includegraphics[width=0.66667in,height=0.66667in]{https://avatars.githubusercontent.com/u/43654815?s=64&v=4}}
\item
  \href{https://github.com/jbirnick}{\includegraphics[width=0.66667in,height=0.66667in]{https://avatars.githubusercontent.com/u/6528009?s=64&v=4}}
\item
  \href{https://github.com/jsoref}{\includegraphics[width=0.66667in,height=0.66667in]{https://avatars.githubusercontent.com/u/2119212?s=64&v=4}}
\item
  \href{https://github.com/mattfbacon}{\includegraphics[width=0.66667in,height=0.66667in]{https://avatars.githubusercontent.com/u/58113890?s=64&v=4}}
\item
  \href{https://github.com/mtoohey31}{\includegraphics[width=0.66667in,height=0.66667in]{https://avatars.githubusercontent.com/u/36740602?s=64&v=4}}
\item
  \href{https://github.com/nz366}{\includegraphics[width=0.66667in,height=0.66667in]{https://avatars.githubusercontent.com/u/180265222?s=64&v=4}}
\item
  \href{https://github.com/omniwrench}{\includegraphics[width=0.66667in,height=0.66667in]{https://avatars.githubusercontent.com/u/44841357?s=64&v=4}}
\item
  \href{https://github.com/shinyfelix}{\includegraphics[width=0.66667in,height=0.66667in]{https://avatars.githubusercontent.com/u/101457412?s=64&v=4}}
\item
  \href{https://github.com/tulio240}{\includegraphics[width=0.66667in,height=0.66667in]{https://avatars.githubusercontent.com/u/113527485?s=64&v=4}}
\item
  \href{https://github.com/3w36zj6}{\includegraphics[width=0.66667in,height=0.66667in]{https://avatars.githubusercontent.com/u/52315048?s=64&v=4}}
\item
  \href{https://github.com/AnarchistHoneybun}{\includegraphics[width=0.66667in,height=0.66667in]{https://avatars.githubusercontent.com/u/74085528?s=64&v=4}}
\item
  \href{https://github.com/Bzero}{\includegraphics[width=0.66667in,height=0.66667in]{https://avatars.githubusercontent.com/u/64414289?s=64&v=4}}
\item
  \href{https://github.com/Heinenen}{\includegraphics[width=0.66667in,height=0.66667in]{https://avatars.githubusercontent.com/u/37484430?s=64&v=4}}
\item
  \href{https://github.com/HydroH}{\includegraphics[width=0.66667in,height=0.66667in]{https://avatars.githubusercontent.com/u/14823453?s=64&v=4}}
\item
  \href{https://github.com/JHenneberg}{\includegraphics[width=0.66667in,height=0.66667in]{https://avatars.githubusercontent.com/u/6606609?s=64&v=4}}
\item
  \href{https://github.com/Jacobgarm}{\includegraphics[width=0.66667in,height=0.66667in]{https://avatars.githubusercontent.com/u/37753339?s=64&v=4}}
\item
  \href{https://github.com/Jocs}{\includegraphics[width=0.66667in,height=0.66667in]{https://avatars.githubusercontent.com/u/9712830?s=64&v=4}}
\item
  \href{https://github.com/JonPichel}{\includegraphics[width=0.66667in,height=0.66667in]{https://avatars.githubusercontent.com/u/47296456?s=64&v=4}}
\item
  \href{https://github.com/JustForFun88}{\includegraphics[width=0.66667in,height=0.66667in]{https://avatars.githubusercontent.com/u/100504524?s=64&v=4}}
\item
  \href{https://github.com/LingkKang}{\includegraphics[width=0.66667in,height=0.66667in]{https://avatars.githubusercontent.com/u/104191582?s=64&v=4}}
\item
  \href{https://github.com/Lucy-73}{\includegraphics[width=0.66667in,height=0.66667in]{https://avatars.githubusercontent.com/u/115893318?s=64&v=4}}
\item
  \href{https://github.com/LuxxxLucy}{\includegraphics[width=0.66667in,height=0.66667in]{https://avatars.githubusercontent.com/u/19356905?s=64&v=4}}
\item
  \href{https://github.com/NiklasEi}{\includegraphics[width=0.66667in,height=0.66667in]{https://avatars.githubusercontent.com/u/12236672?s=64&v=4}}
\item
  \href{https://github.com/Orange149}{\includegraphics[width=0.66667in,height=0.66667in]{https://avatars.githubusercontent.com/u/89233794?s=64&v=4}}
\item
  \href{https://github.com/QuarticCat}{\includegraphics[width=0.66667in,height=0.66667in]{https://avatars.githubusercontent.com/u/70888415?s=64&v=4}}
\item
  \href{https://github.com/SillyFreak}{\includegraphics[width=0.66667in,height=0.66667in]{https://avatars.githubusercontent.com/u/1029192?s=64&v=4}}
\item
  \href{https://github.com/T1mVo}{\includegraphics[width=0.66667in,height=0.66667in]{https://avatars.githubusercontent.com/u/94936637?s=64&v=4}}
\item
  \href{https://github.com/Tom4sCruz}{\includegraphics[width=0.66667in,height=0.66667in]{https://avatars.githubusercontent.com/u/103905440?s=64&v=4}}
\item
  \href{https://github.com/UARTman}{\includegraphics[width=0.66667in,height=0.66667in]{https://avatars.githubusercontent.com/u/21099202?s=64&v=4}}
\item
  \href{https://github.com/YDX-2147483647}{\includegraphics[width=0.66667in,height=0.66667in]{https://avatars.githubusercontent.com/u/73375426?s=64&v=4}}
\item
  \href{https://github.com/aaron-jack-manning}{\includegraphics[width=0.66667in,height=0.66667in]{https://avatars.githubusercontent.com/u/86413837?s=64&v=4}}
\item
  \href{https://github.com/arbrauns}{\includegraphics[width=0.66667in,height=0.66667in]{https://avatars.githubusercontent.com/u/89400397?s=64&v=4}}
\item
  \href{https://github.com/astrale-sharp}{\includegraphics[width=0.66667in,height=0.66667in]{https://avatars.githubusercontent.com/u/53686698?s=64&v=4}}
\item
  \href{https://github.com/bk}{\includegraphics[width=0.66667in,height=0.66667in]{https://avatars.githubusercontent.com/u/25031?s=64&v=4}}
\item
  \href{https://github.com/chicoferreira}{\includegraphics[width=0.66667in,height=0.66667in]{https://avatars.githubusercontent.com/u/36338391?s=64&v=4}}
\item
  \href{https://github.com/ctmbl}{\includegraphics[width=0.66667in,height=0.66667in]{https://avatars.githubusercontent.com/u/79016298?s=64&v=4}}
\item
  \href{https://github.com/danielfleischer}{\includegraphics[width=0.66667in,height=0.66667in]{https://avatars.githubusercontent.com/u/22022514?s=64&v=4}}
\item
  \href{https://github.com/etiennecollin}{\includegraphics[width=0.66667in,height=0.66667in]{https://avatars.githubusercontent.com/u/99756528?s=64&v=4}}
\item
  \href{https://github.com/flauschpantoffel}{\includegraphics[width=0.66667in,height=0.66667in]{https://avatars.githubusercontent.com/u/79235640?s=64&v=4}}
\item
  \href{https://github.com/fynsta}{\includegraphics[width=0.66667in,height=0.66667in]{https://avatars.githubusercontent.com/u/63241108?s=64&v=4}}
\item
  \href{https://github.com/giannissc}{\includegraphics[width=0.66667in,height=0.66667in]{https://avatars.githubusercontent.com/u/20277283?s=64&v=4}}
\item
  \href{https://github.com/haenoe}{\includegraphics[width=0.66667in,height=0.66667in]{https://avatars.githubusercontent.com/u/57222371?s=64&v=4}}
\item
  \href{https://github.com/hardlydearly}{\includegraphics[width=0.66667in,height=0.66667in]{https://avatars.githubusercontent.com/u/167623323?s=64&v=4}}
\item
  \href{https://github.com/hettlage}{\includegraphics[width=0.66667in,height=0.66667in]{https://avatars.githubusercontent.com/u/11633365?s=64&v=4}}
\item
  \href{https://github.com/huajingyun01}{\includegraphics[width=0.66667in,height=0.66667in]{https://avatars.githubusercontent.com/u/74996522?s=64&v=4}}
\item
  \href{https://github.com/inferiorhumanorgans}{\includegraphics[width=0.66667in,height=0.66667in]{https://avatars.githubusercontent.com/u/871927?s=64&v=4}}
\item
  \href{https://github.com/jakobjpeters}{\includegraphics[width=0.66667in,height=0.66667in]{https://avatars.githubusercontent.com/u/59785931?s=64&v=4}}
\item
  \href{https://github.com/jiricekcz}{\includegraphics[width=0.66667in,height=0.66667in]{https://avatars.githubusercontent.com/u/36630605?s=64&v=4}}
\item
  \href{https://github.com/joserlopes}{\includegraphics[width=0.66667in,height=0.66667in]{https://avatars.githubusercontent.com/u/95137505?s=64&v=4}}
\item
  \href{https://github.com/kamack38}{\includegraphics[width=0.66667in,height=0.66667in]{https://avatars.githubusercontent.com/u/64226248?s=64&v=4}}
\item
  \href{https://github.com/kimushun1101}{\includegraphics[width=0.66667in,height=0.66667in]{https://avatars.githubusercontent.com/u/13430937?s=64&v=4}}
\item
  \href{https://github.com/kravchenkoloznia}{\includegraphics[width=0.66667in,height=0.66667in]{https://avatars.githubusercontent.com/u/32360199?s=64&v=4}}
\item
  \href{https://github.com/matze}{\includegraphics[width=0.66667in,height=0.66667in]{https://avatars.githubusercontent.com/u/115270?s=64&v=4}}
\item
  \href{https://github.com/niklasmohrin}{\includegraphics[width=0.66667in,height=0.66667in]{https://avatars.githubusercontent.com/u/47574893?s=64&v=4}}
\item
  \href{https://github.com/nishanthkarthik}{\includegraphics[width=0.66667in,height=0.66667in]{https://avatars.githubusercontent.com/u/7759435?s=64&v=4}}
\item
  \href{https://github.com/nixon-voxell}{\includegraphics[width=0.66667in,height=0.66667in]{https://avatars.githubusercontent.com/u/43715558?s=64&v=4}}
\item
  \href{https://github.com/saecki}{\includegraphics[width=0.66667in,height=0.66667in]{https://avatars.githubusercontent.com/u/43008152?s=64&v=4}}
\item
  \href{https://github.com/unclebean}{\includegraphics[width=0.66667in,height=0.66667in]{https://avatars.githubusercontent.com/u/8064750?s=64&v=4}}
\item
  \href{https://github.com/waywardmonkeys}{\includegraphics[width=0.66667in,height=0.66667in]{https://avatars.githubusercontent.com/u/178582?s=64&v=4}}
\item
  \href{https://github.com/wrzian}{\includegraphics[width=0.66667in,height=0.66667in]{https://avatars.githubusercontent.com/u/133046678?s=64&v=4}}
\item
  \href{https://github.com/zombiepigdragon}{\includegraphics[width=0.66667in,height=0.66667in]{https://avatars.githubusercontent.com/u/26581798?s=64&v=4}}
\end{itemize}

\href{/docs/changelog/}{\pandocbounded{\includesvg[keepaspectratio]{/assets/icons/16-arrow-right.svg}}}

{ Changelog } { Previous page }

\href{/docs/changelog/0.11.1/}{\pandocbounded{\includesvg[keepaspectratio]{/assets/icons/16-arrow-right.svg}}}

{ 0.11.1 } { Next page }


\section{Docs LaTeX/typst.app/docs/changelog/0.10.0.tex}
\title{typst.app/docs/changelog/0.10.0}

\begin{itemize}
\tightlist
\item
  \href{/docs}{\includesvg[width=0.16667in,height=0.16667in]{/assets/icons/16-docs-dark.svg}}
\item
  \includesvg[width=0.16667in,height=0.16667in]{/assets/icons/16-arrow-right.svg}
\item
  \href{/docs/changelog/}{Changelog}
\item
  \includesvg[width=0.16667in,height=0.16667in]{/assets/icons/16-arrow-right.svg}
\item
  \href{/docs/changelog/0.10.0/}{0.10.0}
\end{itemize}

\section{Version 0.10.0 (December 4,
2023)}\label{version-0.10.0-december-4-2023}

\subsection{Bibliography management}\label{bibliography-management}

\begin{itemize}
\tightlist
\item
  Added support for citation collapsing (e.g.
  \texttt{\ {[}1{]}-{[}3{]}\ } instead of
  \texttt{\ {[}1{]},\ {[}2{]},\ {[}3{]}\ } ) if requested by a CSL style
\item
  Fixed bug where an additional space would appear after a group of
  citations
\item
  Fixed link show rules for links in the bibliography
\item
  Fixed show-set rules on citations
\item
  Fixed bibliography-related crashes that happened on some systems
\item
  Corrected name of the GB/T 7714 family of styles from 7114 to 7714
\item
  Fixed missing title in some bibliography styles
\item
  Fixed printing of volumes in some styles
\item
  Fixed delimiter order for contributors in some styles (e.g. APA)
\item
  Fixed behavior of alphanumeric style
\item
  Fixed multiple bugs with GB/T 7714 style
\item
  Fixed escaping in Hayagriva values
\item
  Fixed crashes with empty dates in Hayagriva files
\item
  Fixed bug with spacing around math blocks
\item
  Fixed title case formatting after verbatim text and apostrophes
\item
  Page ranges in \texttt{\ .bib\ } files can now be arbitrary strings
\item
  Multi-line values in \texttt{\ .bib\ } files are now parsed correctly
\item
  Entry keys in \texttt{\ .bib\ } files now allow more characters
\item
  Fixed error message for empty dates in \texttt{\ .bib\ } files
\item
  Added support for years of lengths other than 4 without leading zeros
  in \texttt{\ .bib\ } files
\item
  More LaTeX commands (e.g. for quotes) are now respected in
  \texttt{\ .bib\ } files
\end{itemize}

\subsection{Visualization}\label{visualization}

\begin{itemize}
\tightlist
\item
  Added support for \href{/docs/reference/visualize/pattern/}{patterns}
  as fills and strokes
\item
  The \texttt{\ alpha\ } parameter of the
  \href{/docs/reference/visualize/color/\#definitions-components}{\texttt{\ components\ }}
  function on colors is now a named parameter \textbf{(Breaking change)}
\item
  Added support for the
  \href{/docs/reference/visualize/color/\#definitions-oklch}{Oklch}
  color space
\item
  Improved conversions between colors in different color spaces
\item
  Removed restrictions on
  \href{/docs/reference/visualize/color/\#definitions-oklab}{Oklab}
  chroma component
\item
  Fixed \href{/docs/reference/layout/block/\#parameters-clip}{clipping}
  on blocks and boxes without a stroke
\item
  Fixed bug with \href{/docs/reference/visualize/gradient/}{gradients}
  on math
\item
  Fixed bug with gradient rotation on text
\item
  Fixed bug with gradient colors in PDF
\item
  Fixed relative base of Oklab chroma ratios
\item
  Fixed Oklab color negation
\end{itemize}

\subsection{Text and Layout}\label{text-and-layout}

\begin{itemize}
\tightlist
\item
  CJK text can now be emphasized with the \texttt{\ *\ } and
  \texttt{\ \_\ } syntax even when there are no spaces
\item
  Added basic i18n for Greek and Estonian
\item
  Improved default
  \href{/docs/reference/model/figure/\#definitions-caption-separator}{figure
  caption separator} for Chinese, French, and Russian
\item
  Changed default
  \href{/docs/reference/model/figure/\#parameters-supplement}{figure
  supplement} for Russian to short form
\item
  Fixed
  \href{/docs/reference/text/text/\#parameters-cjk-latin-spacing}{CJK-Latin-spacing}
  before line breaks and in
  \href{/docs/reference/introspection/locate/}{\texttt{\ locate\ }}
  calls
\item
  Fixed line breaking at the end of links
\end{itemize}

\subsection{Math}\label{math}

\begin{itemize}
\tightlist
\item
  Added \href{/docs/reference/math/lr/\#functions-mid}{\texttt{\ mid\ }}
  function for scaling a delimiter up to the height of the surrounding
  \href{/docs/reference/math/lr/\#functions-lr}{\texttt{\ lr\ }} group
\item
  The \href{/docs/reference/math/op/}{\texttt{\ op\ }} function can now
  take any content, not just strings
\item
  Improved documentation for
  \href{/docs/reference/math/\#alignment}{math alignment}
\item
  Fixed swallowing of trailing comma when a symbol is used in a
  function-like way (e.g. \texttt{\ pi(a,b,)\ } )
\end{itemize}

\subsection{Scripting}\label{scripting}

\begin{itemize}
\tightlist
\item
  Any non-identifier dictionary key is now interpreted as an expression:
  For instance,
  \texttt{\ }{\texttt{\ (\ }}\texttt{\ }{\texttt{\ (\ }}\texttt{\ key\ }{\texttt{\ )\ }}\texttt{\ }{\texttt{\ :\ }}\texttt{\ value\ }{\texttt{\ )\ }}\texttt{\ }
  will create a dictionary with a dynamic key
\item
  The \href{/docs/reference/visualize/stroke/}{\texttt{\ stroke\ }} type
  now has a constructor that converts a value to a stroke or creates one
  from its parts
\item
  Added constructor for
  \href{/docs/reference/foundations/arguments/}{\texttt{\ arguments\ }}
  type
\item
  Added
  \href{/docs/reference/foundations/calc/\#functions-div-euclid}{\texttt{\ calc.div-euclid\ }}
  and
  \href{/docs/reference/foundations/calc/\#functions-rem-euclid}{\texttt{\ calc.rem-euclid\ }}
  functions
\item
  Fixed equality of
  \href{/docs/reference/foundations/arguments/}{\texttt{\ arguments\ }}
\item
  Fixed \href{/docs/reference/foundations/repr/}{\texttt{\ repr\ }} of
  \href{/docs/reference/visualize/color/\#definitions-cmyk}{\texttt{\ cmyk\ }}
  colors
\item
  Fixed crashes with provided elements like figure captions, outline
  entries, and footnote entries
\end{itemize}

\subsection{Tooling and Diagnostics}\label{tooling-and-diagnostics}

\begin{itemize}
\tightlist
\item
  Show rules that match on their own output now produce an appropriate
  error message instead of a crash (this is a first step, in the future
  they will just work)
\item
  Too highly or infinitely nested layouts now produce error messages
  instead of crashes
\item
  Added hints for invalid identifiers
\item
  Added hint when trying to use a manually constructed footnote or
  outline entry
\item
  Added missing details to autocompletions for types
\item
  Improved error message when passing a named argument where a
  positional one is expected
\item
  Jump from click now works on raw blocks
\end{itemize}

\subsection{Export}\label{export}

\begin{itemize}
\tightlist
\item
  PDF compilation output is now again fully byte-by-byte reproducible if
  the document\textquotesingle s
  \href{/docs/reference/model/document/\#parameters-date}{\texttt{\ date\ }}
  is set manually
\item
  Fixed color export in SVG
\item
  Fixed PDF metadata encoding of multiple
  \href{/docs/reference/model/document/\#parameters-author}{authors}
\end{itemize}

\subsection{Command line interface}\label{command-line-interface}

\begin{itemize}
\tightlist
\item
  Fixed a major bug where \texttt{\ typst\ watch\ } would confuse files
  and fail to pick up updates
\item
  Fetching of the release metadata in \texttt{\ typst\ update\ } now
  respects proxies
\item
  Fixed bug with \texttt{\ -\/-open\ } flag on Windows when the path
  contains a space
\item
  The \texttt{\ TYPST\_FONT\_PATHS\ } environment variable can now
  contain multiple paths (separated by \texttt{\ ;\ } on Windows and
  \texttt{\ :\ } elsewhere)
\item
  Updated embedded New Computer Modern fonts to version 4.7
\item
  The watching process doesn\textquotesingle t stop anymore when the
  main file contains invalid UTF-8
\end{itemize}

\subsection{Miscellaneous
Improvements}\label{miscellaneous-improvements}

\begin{itemize}
\tightlist
\item
  Parallelized image encoding in PDF export
\item
  Improved the internal representation of content for improved
  performance
\item
  Optimized introspection (query, counter, etc.) performance
\item
  The \href{/docs/reference/model/document/\#parameters-title}{document
  title} can now be arbitrary content instead of just a string
\item
  The
  \href{/docs/reference/model/enum/\#parameters-number-align}{\texttt{\ number-align\ }}
  parameter on numbered lists now also accepts vertical alignments
\item
  Fixed selectors on \href{/docs/reference/model/quote/}{quote} elements
\item
  Fixed parsing of
  \texttt{\ }{\texttt{\ \#\ }}\texttt{\ }{\texttt{\ return\ }}\texttt{\ }
  expression in markup
\item
  Fixed bug where inline equations were displayed in equation outlines
\item
  Fixed potential CRLF issue in
  \href{/docs/reference/text/raw/}{\texttt{\ raw\ }} blocks
\item
  Fixed a bug where Chinese numbering couldn\textquotesingle t exceed
  the number 255
\end{itemize}

\subsection{Development}\label{development}

\begin{itemize}
\tightlist
\item
  Merged \texttt{\ typst\ } and \texttt{\ typst-library\ } and extracted
  \texttt{\ typst-pdf\ } , \texttt{\ typst-svg\ } , and
  \texttt{\ typst-render\ } into separate crates
\item
  The Nix flake now includes the git revision when running
  \texttt{\ typst\ -\/-version\ }
\end{itemize}

\subsection{Contributors}\label{contributors}

Thanks to everyone who contributed to this release!

\begin{itemize}
\tightlist
\item
  \href{https://github.com/Dherse}{\includegraphics[width=0.66667in,height=0.66667in]{https://avatars.githubusercontent.com/u/9665250?s=64&v=4}}
\item
  \href{https://github.com/frozolotl}{\includegraphics[width=0.66667in,height=0.66667in]{https://avatars.githubusercontent.com/u/44589151?s=64&v=4}}
\item
  \href{https://github.com/Leedehai}{\includegraphics[width=0.66667in,height=0.66667in]{https://avatars.githubusercontent.com/u/18319900?s=64&v=4}}
\item
  \href{https://github.com/Andrew15-5}{\includegraphics[width=0.66667in,height=0.66667in]{https://avatars.githubusercontent.com/u/37143421?s=64&v=4}}
\item
  \href{https://github.com/MDLC01}{\includegraphics[width=0.66667in,height=0.66667in]{https://avatars.githubusercontent.com/u/57839069?s=64&v=4}}
\item
  \href{https://github.com/danieleades}{\includegraphics[width=0.66667in,height=0.66667in]{https://avatars.githubusercontent.com/u/33452915?s=64&v=4}}
\item
  \href{https://github.com/tingerrr}{\includegraphics[width=0.66667in,height=0.66667in]{https://avatars.githubusercontent.com/u/137803093?s=64&v=4}}
\item
  \href{https://github.com/Jollywatt}{\includegraphics[width=0.66667in,height=0.66667in]{https://avatars.githubusercontent.com/u/24970860?s=64&v=4}}
\item
  \href{https://github.com/cmoog}{\includegraphics[width=0.66667in,height=0.66667in]{https://avatars.githubusercontent.com/u/7585078?s=64&v=4}}
\item
  \href{https://github.com/peng1999}{\includegraphics[width=0.66667in,height=0.66667in]{https://avatars.githubusercontent.com/u/12483662?s=64&v=4}}
\item
  \href{https://github.com/Enter-tainer}{\includegraphics[width=0.66667in,height=0.66667in]{https://avatars.githubusercontent.com/u/25521218?s=64&v=4}}
\item
  \href{https://github.com/JakobSachs}{\includegraphics[width=0.66667in,height=0.66667in]{https://avatars.githubusercontent.com/u/28728963?s=64&v=4}}
\item
  \href{https://github.com/KronosTheLate}{\includegraphics[width=0.66667in,height=0.66667in]{https://avatars.githubusercontent.com/u/61620837?s=64&v=4}}
\item
  \href{https://github.com/MyrtleTurtle22}{\includegraphics[width=0.66667in,height=0.66667in]{https://avatars.githubusercontent.com/u/82775864?s=64&v=4}}
\item
  \href{https://github.com/T0mstone}{\includegraphics[width=0.66667in,height=0.66667in]{https://avatars.githubusercontent.com/u/39707032?s=64&v=4}}
\item
  \href{https://github.com/TheJosefOlsson}{\includegraphics[width=0.66667in,height=0.66667in]{https://avatars.githubusercontent.com/u/143743179?s=64&v=4}}
\item
  \href{https://github.com/antonWetzel}{\includegraphics[width=0.66667in,height=0.66667in]{https://avatars.githubusercontent.com/u/59712243?s=64&v=4}}
\item
  \href{https://github.com/denkspuren}{\includegraphics[width=0.66667in,height=0.66667in]{https://avatars.githubusercontent.com/u/4160411?s=64&v=4}}
\item
  \href{https://github.com/kokkonisd}{\includegraphics[width=0.66667in,height=0.66667in]{https://avatars.githubusercontent.com/u/18401822?s=64&v=4}}
\item
  \href{https://github.com/lihe07}{\includegraphics[width=0.66667in,height=0.66667in]{https://avatars.githubusercontent.com/u/53819558?s=64&v=4}}
\item
  \href{https://github.com/mattfbacon}{\includegraphics[width=0.66667in,height=0.66667in]{https://avatars.githubusercontent.com/u/58113890?s=64&v=4}}
\item
  \href{https://github.com/rezzubs}{\includegraphics[width=0.66667in,height=0.66667in]{https://avatars.githubusercontent.com/u/57254926?s=64&v=4}}
\item
  \href{https://github.com/samueltardieu}{\includegraphics[width=0.66667in,height=0.66667in]{https://avatars.githubusercontent.com/u/44656?s=64&v=4}}
\item
  \href{https://github.com/xalbd}{\includegraphics[width=0.66667in,height=0.66667in]{https://avatars.githubusercontent.com/u/119540449?s=64&v=4}}
\end{itemize}

\href{/docs/changelog/0.11.0/}{\pandocbounded{\includesvg[keepaspectratio]{/assets/icons/16-arrow-right.svg}}}

{ 0.11.0 } { Previous page }

\href{/docs/changelog/0.9.0/}{\pandocbounded{\includesvg[keepaspectratio]{/assets/icons/16-arrow-right.svg}}}

{ 0.9.0 } { Next page }


\section{Docs LaTeX/typst.app/docs/changelog/0.8.0.tex}
\title{typst.app/docs/changelog/0.8.0}

\begin{itemize}
\tightlist
\item
  \href{/docs}{\includesvg[width=0.16667in,height=0.16667in]{/assets/icons/16-docs-dark.svg}}
\item
  \includesvg[width=0.16667in,height=0.16667in]{/assets/icons/16-arrow-right.svg}
\item
  \href{/docs/changelog/}{Changelog}
\item
  \includesvg[width=0.16667in,height=0.16667in]{/assets/icons/16-arrow-right.svg}
\item
  \href{/docs/changelog/0.8.0/}{0.8.0}
\end{itemize}

\section{Version 0.8.0 (September 13,
2023)}\label{version-0.8.0-september-13-2023}

\subsection{Scripting}\label{scripting}

\begin{itemize}
\tightlist
\item
  Plugins (thanks to
  \href{https://github.com/astrale-sharp}{@astrale-sharp} and
  \href{https://github.com/arnaudgolfouse}{@arnaudgolfouse} )

  \begin{itemize}
  \tightlist
  \item
    Typst can now load
    \href{/docs/reference/foundations/plugin/}{plugins} that are
    compiled to WebAssembly
  \item
    Anything that can be compiled to WebAssembly can thus be loaded as a
    plugin
  \item
    These plugins are fully encapsulated (no access to file system or
    network)
  \item
    Plugins can be shipped as part of
    \href{/docs/reference/scripting/\#packages}{packages}
  \item
    Plugins work just the same in the web app
  \end{itemize}
\item
  Types are now first-class values \textbf{(Breaking change)}

  \begin{itemize}
  \tightlist
  \item
    A \href{/docs/reference/foundations/type/}{type} is now itself a
    value
  \item
    Some types can be called like functions (those that have a
    constructor), e.g.
    \href{/docs/reference/foundations/int/}{\texttt{\ int\ }} and
    \href{/docs/reference/foundations/str/}{\texttt{\ str\ }}
  \item
    Type checks are now of the form
    \texttt{\ }{\texttt{\ type\ }}\texttt{\ }{\texttt{\ (\ }}\texttt{\ }{\texttt{\ 10\ }}\texttt{\ }{\texttt{\ )\ }}\texttt{\ }{\texttt{\ ==\ }}\texttt{\ int\ }
    instead of the old
    \texttt{\ }{\texttt{\ type\ }}\texttt{\ }{\texttt{\ (\ }}\texttt{\ }{\texttt{\ 10\ }}\texttt{\ }{\texttt{\ )\ }}\texttt{\ }{\texttt{\ ==\ }}\texttt{\ }{\texttt{\ "integer"\ }}\texttt{\ }
    .
    \href{/docs/reference/foundations/type/\#compatibility}{Compatibility}
    with the old way will remain for a while to give package authors
    time to upgrade, but it will be removed at some point.
  \item
    Methods are now syntax sugar for calling a function scoped to a
    type, meaning that
    \texttt{\ }{\texttt{\ "hello"\ }}\texttt{\ }{\texttt{\ .\ }}\texttt{\ }{\texttt{\ len\ }}\texttt{\ }{\texttt{\ (\ }}\texttt{\ }{\texttt{\ )\ }}\texttt{\ }
    is equivalent to
    \texttt{\ str\ }{\texttt{\ .\ }}\texttt{\ }{\texttt{\ len\ }}\texttt{\ }{\texttt{\ (\ }}\texttt{\ }{\texttt{\ "hello"\ }}\texttt{\ }{\texttt{\ )\ }}\texttt{\ }
  \end{itemize}
\item
  Added support for
  \href{/docs/reference/scripting/\#modules}{\texttt{\ import\ }}
  renaming with \texttt{\ as\ }
\item
  Added a
  \href{/docs/reference/foundations/duration/}{\texttt{\ duration\ }}
  type
\item
  Added support for \href{/docs/reference/data-loading/cbor/}{CBOR}
  encoding and decoding
\item
  Added encoding and decoding functions from and to bytes for data
  formats:
  \href{/docs/reference/data-loading/json/\#definitions-decode}{\texttt{\ json.decode\ }}
  ,
  \href{/docs/reference/data-loading/json/\#definitions-encode}{\texttt{\ json.encode\ }}
  , and similar functions for other formats
\item
  Added
  \href{/docs/reference/foundations/array/\#definitions-intersperse}{\texttt{\ array.intersperse\ }}
  function
\item
  Added
  \href{/docs/reference/foundations/str/\#definitions-rev}{\texttt{\ str.rev\ }}
  function
\item
  Added \texttt{\ calc.tau\ } constant
\item
  Made \href{/docs/reference/foundations/bytes/}{bytes} joinable and
  addable
\item
  Made
  \href{/docs/reference/foundations/array/\#definitions-zip}{\texttt{\ array.zip\ }}
  function variadic
\item
  Fixed bug with
  \href{/docs/reference/foundations/eval/}{\texttt{\ eval\ }} when the
  \texttt{\ mode\ } was set to
  \texttt{\ }{\texttt{\ "math"\ }}\texttt{\ }
\item
  Fixed bug with
  \href{/docs/reference/foundations/str/\#definitions-ends-with}{\texttt{\ ends-with\ }}
  function on strings
\item
  Fixed bug with destructuring in combination with break, continue, and
  return
\item
  Fixed argument types of
  \href{/docs/reference/foundations/calc/\#functions-cosh}{hyperbolic
  functions} , they don\textquotesingle t allow angles anymore
  \textbf{(Breaking change)}
\item
  Renamed some color methods: \texttt{\ rgba\ } becomes
  \texttt{\ to-rgba\ } , \texttt{\ cmyk\ } becomes \texttt{\ to-cmyk\ }
  , and \texttt{\ luma\ } becomes \texttt{\ to-luma\ } \textbf{(Breaking
  change)}
\end{itemize}

\subsection{Export}\label{export}

\begin{itemize}
\tightlist
\item
  Added SVG export (thanks to
  \href{https://github.com/Enter-tainer}{@Enter-tainer} )
\item
  Fixed bugs with PDF font embedding
\item
  Added support for page labels that reflect the
  \href{/docs/reference/layout/page/\#parameters-numbering}{page
  numbering} style in the PDF
\end{itemize}

\subsection{Text and Layout}\label{text-and-layout}

\begin{itemize}
\tightlist
\item
  Added \href{/docs/reference/text/highlight/}{\texttt{\ highlight\ }}
  function for highlighting text with a background color
\item
  Added
  \href{/docs/reference/visualize/polygon/\#definitions-regular}{\texttt{\ polygon.regular\ }}
  function for drawing a regular polygon
\item
  Added support for tabs in
  \href{/docs/reference/text/raw/}{\texttt{\ raw\ }} elements alongside
  \href{/docs/reference/text/raw/\#parameters-tab-size}{\texttt{\ tab-width\ }}
  parameter
\item
  The layout engine now tries to prevent "runts" (final lines consisting
  of just a single word)
\item
  Added Finnish translations
\item
  Added hyphenation support for Polish
\item
  Improved handling of consecutive smart quotes of different kinds
\item
  Fixed vertical alignments for
  \href{/docs/reference/layout/page/\#parameters-number-align}{\texttt{\ number-align\ }}
  argument on page function \textbf{(Breaking change)}
\item
  Fixed weak pagebreaks after counter updates
\item
  Fixed missing text in SVG when the text font is set to "New Computer
  Modern"
\item
  Fixed translations for Chinese
\item
  Fixed crash for empty text in show rule
\item
  Fixed leading spaces when there\textquotesingle s a linebreak after a
  number and a comma
\item
  Fixed placement of floating elements in columns and other containers
\item
  Fixed sizing of block containing just a single box
\end{itemize}

\subsection{Math}\label{math}

\begin{itemize}
\tightlist
\item
  Added support for
  \href{/docs/reference/math/mat/\#parameters-augment}{augmented
  matrices}
\item
  Removed support for automatic matching of fences like
  \texttt{\ \textbar{}\ } and \texttt{\ \textbar{}\textbar{}\ } as there
  were too many false positives. You can use functions like
  \href{/docs/reference/math/lr/\#functions-abs}{\texttt{\ abs\ }} or
  \href{/docs/reference/math/lr/\#functions-norm}{\texttt{\ norm\ }} or
  an explicit
  \href{/docs/reference/math/lr/\#functions-lr}{\texttt{\ lr\ }} call
  instead. \textbf{(Breaking change)}
\item
  Fixed spacing after number with decimal point in math
\item
  Fixed bug with primes in subscript
\item
  Fixed weak spacing
\item
  Fixed crash when text within math contains a newline
\end{itemize}

\subsection{Tooling and Diagnostics}\label{tooling-and-diagnostics}

\begin{itemize}
\tightlist
\item
  Added hints when trying to call a function stored in a dictionary
  without extra parentheses
\item
  Fixed hint when referencing an equation without numbering
\item
  Added more details to some diagnostics (e.g. when SVG decoding fails)
\end{itemize}

\subsection{Command line interface}\label{command-line-interface}

\begin{itemize}
\tightlist
\item
  Added \texttt{\ typst\ update\ } command for self-updating the CLI
  (thanks to \href{https://github.com/jimvdl}{@jimvdl} )
\item
  Added download progress indicator for packages and updates
\item
  Added \texttt{\ -\/-format\ } argument to explicitly specify the
  output format
\item
  The CLI now respects proxy configuration through environment variables
  and has a new \texttt{\ -\/-cert\ } option for setting a custom CA
  certificate
\item
  Fixed crash when field wasn\textquotesingle t present and
  \texttt{\ -\/-one\ } is passed to \texttt{\ typst\ query\ }
\end{itemize}

\subsection{Miscellaneous
Improvements}\label{miscellaneous-improvements}

\begin{itemize}
\tightlist
\item
  Added \href{/docs/guides/page-setup-guide/}{page setup guide}
\item
  Added
  \href{/docs/reference/model/figure/\#definitions-caption}{\texttt{\ figure.caption\ }}
  function that can be used for simpler figure customization (
  \textbf{Breaking change} because \texttt{\ it.caption\ } now renders
  the full caption with supplement in figure show rules and manual
  outlines)
\item
  Moved \texttt{\ caption-pos\ } argument to \texttt{\ figure.caption\ }
  function and renamed it to \texttt{\ position\ } \textbf{(Breaking
  change)}
\item
  Added
  \href{/docs/reference/model/figure/\#definitions-caption-separator}{\texttt{\ separator\ }}
  argument to \texttt{\ figure.caption\ } function
\item
  Added support for combination of and/or and before/after
  \href{/docs/reference/foundations/selector/}{selectors}
\item
  Packages can now specify a
  \href{https://github.com/typst/packages\#package-format}{minimum
  compiler version} they require to work
\item
  Fixed parser bug where method calls could be moved onto their own line
  for
  \texttt{\ }{\texttt{\ \#\ }}\texttt{\ }{\texttt{\ let\ }}\texttt{\ }
  expressions in markup \textbf{(Breaking change)}
\item
  Fixed bugs in sentence and title case conversion for bibliographies
\item
  Fixed supplements for alphanumeric and author-title bibliography
  styles
\item
  Fixed off-by-one error in APA bibliography style
\end{itemize}

\subsection{Development}\label{development}

\begin{itemize}
\tightlist
\item
  Made \texttt{\ Span\ } and \texttt{\ FileId\ } more type-safe so that
  all error conditions must be handled by \texttt{\ World\ }
  implementors
\end{itemize}

\subsection{Contributors}\label{contributors}

Thanks to everyone who contributed to this release!

\begin{itemize}
\tightlist
\item
  \href{https://github.com/Beiri22}{\includegraphics[width=0.66667in,height=0.66667in]{https://avatars.githubusercontent.com/u/8210233?s=64&v=4}}
\item
  \href{https://github.com/bluebear94}{\includegraphics[width=0.66667in,height=0.66667in]{https://avatars.githubusercontent.com/u/2975203?s=64&v=4}}
\item
  \href{https://github.com/jimvdl}{\includegraphics[width=0.66667in,height=0.66667in]{https://avatars.githubusercontent.com/u/26407533?s=64&v=4}}
\item
  \href{https://github.com/LuxxxLucy}{\includegraphics[width=0.66667in,height=0.66667in]{https://avatars.githubusercontent.com/u/19356905?s=64&v=4}}
\item
  \href{https://github.com/mattfbacon}{\includegraphics[width=0.66667in,height=0.66667in]{https://avatars.githubusercontent.com/u/58113890?s=64&v=4}}
\item
  \href{https://github.com/sitandr}{\includegraphics[width=0.66667in,height=0.66667in]{https://avatars.githubusercontent.com/u/60141933?s=64&v=4}}
\item
  \href{https://github.com/xkevio}{\includegraphics[width=0.66667in,height=0.66667in]{https://avatars.githubusercontent.com/u/13004777?s=64&v=4}}
\item
  \href{https://github.com/Dherse}{\includegraphics[width=0.66667in,height=0.66667in]{https://avatars.githubusercontent.com/u/9665250?s=64&v=4}}
\item
  \href{https://github.com/Enter-tainer}{\includegraphics[width=0.66667in,height=0.66667in]{https://avatars.githubusercontent.com/u/25521218?s=64&v=4}}
\item
  \href{https://github.com/SimonRask}{\includegraphics[width=0.66667in,height=0.66667in]{https://avatars.githubusercontent.com/u/33556894?s=64&v=4}}
\item
  \href{https://github.com/Andrew15-5}{\includegraphics[width=0.66667in,height=0.66667in]{https://avatars.githubusercontent.com/u/37143421?s=64&v=4}}
\item
  \href{https://github.com/KillTheMule}{\includegraphics[width=0.66667in,height=0.66667in]{https://avatars.githubusercontent.com/u/4117685?s=64&v=4}}
\item
  \href{https://github.com/LaurenzV}{\includegraphics[width=0.66667in,height=0.66667in]{https://avatars.githubusercontent.com/u/47084093?s=64&v=4}}
\item
  \href{https://github.com/MDLC01}{\includegraphics[width=0.66667in,height=0.66667in]{https://avatars.githubusercontent.com/u/57839069?s=64&v=4}}
\item
  \href{https://github.com/NeillJohnston}{\includegraphics[width=0.66667in,height=0.66667in]{https://avatars.githubusercontent.com/u/16545367?s=64&v=4}}
\item
  \href{https://github.com/PgBiel}{\includegraphics[width=0.66667in,height=0.66667in]{https://avatars.githubusercontent.com/u/9021226?s=64&v=4}}
\item
  \href{https://github.com/SillyFreak}{\includegraphics[width=0.66667in,height=0.66667in]{https://avatars.githubusercontent.com/u/1029192?s=64&v=4}}
\item
  \href{https://github.com/abramchikd}{\includegraphics[width=0.66667in,height=0.66667in]{https://avatars.githubusercontent.com/u/32370126?s=64&v=4}}
\item
  \href{https://github.com/antonWetzel}{\includegraphics[width=0.66667in,height=0.66667in]{https://avatars.githubusercontent.com/u/59712243?s=64&v=4}}
\item
  \href{https://github.com/arj0019}{\includegraphics[width=0.66667in,height=0.66667in]{https://avatars.githubusercontent.com/u/92353079?s=64&v=4}}
\item
  \href{https://github.com/astrale-sharp}{\includegraphics[width=0.66667in,height=0.66667in]{https://avatars.githubusercontent.com/u/53686698?s=64&v=4}}
\item
  \href{https://github.com/damaxwell}{\includegraphics[width=0.66667in,height=0.66667in]{https://avatars.githubusercontent.com/u/918465?s=64&v=4}}
\item
  \href{https://github.com/dikkadev}{\includegraphics[width=0.66667in,height=0.66667in]{https://avatars.githubusercontent.com/u/64754924?s=64&v=4}}
\item
  \href{https://github.com/frozolotl}{\includegraphics[width=0.66667in,height=0.66667in]{https://avatars.githubusercontent.com/u/44589151?s=64&v=4}}
\item
  \href{https://github.com/kiviktnm}{\includegraphics[width=0.66667in,height=0.66667in]{https://avatars.githubusercontent.com/u/65563192?s=64&v=4}}
\item
  \href{https://github.com/klMse}{\includegraphics[width=0.66667in,height=0.66667in]{https://avatars.githubusercontent.com/u/61806749?s=64&v=4}}
\item
  \href{https://github.com/lolstork}{\includegraphics[width=0.66667in,height=0.66667in]{https://avatars.githubusercontent.com/u/137357423?s=64&v=4}}
\item
  \href{https://github.com/lukas-loering}{\includegraphics[width=0.66667in,height=0.66667in]{https://avatars.githubusercontent.com/u/52287649?s=64&v=4}}
\item
  \href{https://github.com/owiecc}{\includegraphics[width=0.66667in,height=0.66667in]{https://avatars.githubusercontent.com/u/6896639?s=64&v=4}}
\item
  \href{https://github.com/pavelzw}{\includegraphics[width=0.66667in,height=0.66667in]{https://avatars.githubusercontent.com/u/29506042?s=64&v=4}}
\item
  \href{https://github.com/raphCode}{\includegraphics[width=0.66667in,height=0.66667in]{https://avatars.githubusercontent.com/u/15750438?s=64&v=4}}
\item
  \href{https://github.com/sudormrfbin}{\includegraphics[width=0.66667in,height=0.66667in]{https://avatars.githubusercontent.com/u/23398472?s=64&v=4}}
\item
  \href{https://github.com/t-rapp}{\includegraphics[width=0.66667in,height=0.66667in]{https://avatars.githubusercontent.com/u/20061583?s=64&v=4}}
\item
  \href{https://github.com/zicklag}{\includegraphics[width=0.66667in,height=0.66667in]{https://avatars.githubusercontent.com/u/25393315?s=64&v=4}}
\item
  \href{https://github.com/zyoshoka}{\includegraphics[width=0.66667in,height=0.66667in]{https://avatars.githubusercontent.com/u/107108195?s=64&v=4}}
\end{itemize}

\href{/docs/changelog/0.9.0/}{\pandocbounded{\includesvg[keepaspectratio]{/assets/icons/16-arrow-right.svg}}}

{ 0.9.0 } { Previous page }

\href{/docs/changelog/0.7.0/}{\pandocbounded{\includesvg[keepaspectratio]{/assets/icons/16-arrow-right.svg}}}

{ 0.7.0 } { Next page }


\section{Docs LaTeX/typst.app/docs/changelog/0.11.0.tex}
\title{typst.app/docs/changelog/0.11.0}

\begin{itemize}
\tightlist
\item
  \href{/docs}{\includesvg[width=0.16667in,height=0.16667in]{/assets/icons/16-docs-dark.svg}}
\item
  \includesvg[width=0.16667in,height=0.16667in]{/assets/icons/16-arrow-right.svg}
\item
  \href{/docs/changelog/}{Changelog}
\item
  \includesvg[width=0.16667in,height=0.16667in]{/assets/icons/16-arrow-right.svg}
\item
  \href{/docs/changelog/0.11.0/}{0.11.0}
\end{itemize}

\section{Version 0.11.0 (March 15,
2024)}\label{version-0.11.0-march-15-2024}

\subsection{Tables}\label{tables}

\begin{itemize}
\tightlist
\item
  Tables are now \emph{much} more flexible, read the new
  \href{/docs/guides/table-guide/}{table guide} to get started
\item
  Added
  \href{/docs/reference/model/table/\#definitions-cell}{\texttt{\ table.cell\ }}
  element for per-cell configuration
\item
  Cells can now span multiple
  \href{/docs/reference/model/table/\#definitions-cell-colspan}{columns}
  or \href{/docs/reference/model/table/\#definitions-cell-rowspan}{rows}
\item
  The
  \href{/docs/reference/model/table/\#definitions-cell-stroke}{stroke}
  of individual cells can now be customized
\item
  The
  \href{/docs/reference/model/table/\#parameters-align}{\texttt{\ align\ }}
  and
  \href{/docs/reference/model/table/\#parameters-inset}{\texttt{\ inset\ }}
  arguments of the table function now also take
  \texttt{\ }{\texttt{\ (\ }}\texttt{\ x\ }{\texttt{\ ,\ }}\texttt{\ y\ }{\texttt{\ )\ }}\texttt{\ }{\texttt{\ =\textgreater{}\ }}\texttt{\ ..\ }
  functions
\item
  Added
  \href{/docs/reference/model/table/\#definitions-hline}{\texttt{\ table.hline\ }}
  and
  \href{/docs/reference/model/table/\#definitions-vline}{\texttt{\ table.vline\ }}
  for convenient line customization
\item
  Added
  \href{/docs/reference/model/table/\#definitions-header}{\texttt{\ table.header\ }}
  element for table headers that repeat on every page
\item
  Added
  \href{/docs/reference/model/table/\#definitions-footer}{\texttt{\ table.footer\ }}
  element for table footers that repeat on every page
\item
  All the new table functionality is also available for
  \href{/docs/reference/layout/grid/}{grids}
\item
  Fixed gutter-related bugs
\end{itemize}

\emph{Thanks to \href{https://github.com/PgBiel}{@PgBiel} for his work
on tables!}

\subsection{Templates}\label{templates}

\begin{itemize}
\tightlist
\item
  You can now use template packages to get started with new projects.
  Click \emph{Start from template} on the web app\textquotesingle s
  dashboard and choose your preferred template or run the
  \texttt{\ typst\ init\ \textless{}template\textgreater{}\ } command in
  the CLI. You can
  \href{https://typst.app/universe/search/?kind=templates}{browse the
  available templates here} .
\item
  Switching templates after the fact has become easier. You can just
  import a styling function from a different template package.
\item
  Package authors can now submit their own templates to the
  \href{https://github.com/typst/packages}{package repository} . Share a
  template for a paper, your institution, or an original work to help
  the community get a head start on their projects.
\item
  Templates and packages are now organized by category and discipline.
  Filter packages by either taxonomy in the \emph{Start from template}
  wizard. If you are a package author, take a look at the new
  documentation for
  \href{https://github.com/typst/packages/blob/main/CATEGORIES.md}{categories}
  and
  \href{https://github.com/typst/packages/blob/main/DISCIPLINES.md}{disciplines}
  .
\end{itemize}

\subsection{Context}\label{context}

\begin{itemize}
\tightlist
\item
  Added \emph{context expressions:} Read the chapter on
  \href{/docs/reference/context/}{context} to get started
\item
  With context, you can access settable properties, e.g.
  \texttt{\ }{\texttt{\ context\ }}\texttt{\ text\ }{\texttt{\ .\ }}\texttt{\ lang\ }
  to access the language set via
  \texttt{\ }{\texttt{\ set\ }}\texttt{\ }{\texttt{\ text\ }}\texttt{\ }{\texttt{\ (\ }}\texttt{\ lang\ }{\texttt{\ :\ }}\texttt{\ }{\texttt{\ ".."\ }}\texttt{\ }{\texttt{\ )\ }}\texttt{\ }
\item
  The following existing functions have been made contextual:
  \href{/docs/reference/introspection/query/}{\texttt{\ query\ }} ,
  \href{/docs/reference/introspection/locate/}{\texttt{\ locate\ }} ,
  \href{/docs/reference/layout/measure/}{\texttt{\ measure\ }} ,
  \href{/docs/reference/introspection/counter/\#definitions-display}{\texttt{\ counter.display\ }}
  ,
  \href{/docs/reference/introspection/counter/\#definitions-at}{\texttt{\ counter.at\ }}
  ,
  \href{/docs/reference/introspection/counter/\#definitions-final}{\texttt{\ counter.final\ }}
  ,
  \href{/docs/reference/introspection/state/\#definitions-at}{\texttt{\ state.at\ }}
  , and
  \href{/docs/reference/introspection/state/\#definitions-final}{\texttt{\ state.final\ }}
\item
  Added contextual methods
  \href{/docs/reference/introspection/counter/\#definitions-get}{\texttt{\ counter.get\ }}
  and
  \href{/docs/reference/introspection/state/\#definitions-get}{\texttt{\ state.get\ }}
  to retrieve the value of a counter or state in the current context
\item
  Added contextual function
  \href{/docs/reference/introspection/here/}{\texttt{\ here\ }} to
  retrieve the \href{/docs/reference/introspection/location/}{location}
  of the current context
\item
  The \href{/docs/reference/introspection/locate/}{\texttt{\ locate\ }}
  function now returns the location of a selector\textquotesingle s
  unique match. Its old behavior has been replaced by context
  expressions and only remains temporarily available for compatibility.
\item
  The
  \href{/docs/reference/introspection/counter/\#definitions-at}{\texttt{\ counter.at\ }}
  and
  \href{/docs/reference/introspection/state/\#definitions-at}{\texttt{\ state.at\ }}
  methods are now more flexible: They directly accept any kind of
  \href{/docs/reference/introspection/location/\#locatable}{locatable}
  selector with a unique match (e.g. a label) instead of just locations
\item
  When context is available,
  \href{/docs/reference/introspection/counter/\#definitions-display}{\texttt{\ counter.display\ }}
  now directly returns the result of applying the numbering instead of
  yielding opaque content. It should not be used anymore without
  context. (Deprecation planned)
\item
  The
  \href{/docs/reference/introspection/state/\#definitions-display}{\texttt{\ state.display\ }}
  function should not be used anymore, use
  \href{/docs/reference/introspection/state/\#definitions-get}{\texttt{\ state.get\ }}
  instead (Deprecation planned)
\item
  The \texttt{\ location\ } argument of
  \href{/docs/reference/introspection/query/}{\texttt{\ query\ }} ,
  \href{/docs/reference/introspection/counter/\#definitions-final}{\texttt{\ counter.final\ }}
  , and
  \href{/docs/reference/introspection/state/\#definitions-final}{\texttt{\ state.final\ }}
  should not be used anymore (Deprecation planned)
\item
  The
  \href{/docs/reference/layout/measure/\#parameters-styles}{\texttt{\ styles\ }}
  argument of the \texttt{\ measure\ } function should not be used
  anymore (Deprecation planned)
\item
  The \href{/docs/reference/foundations/style/}{\texttt{\ style\ }}
  function should not be used anymore, use context instead (Deprecation
  planned)
\item
  The correct context is now also provided in various other places where
  it is available, e.g. in show rules, layout callbacks, and numbering
  functions in the outline
\end{itemize}

\subsection{Styling}\label{styling}

\begin{itemize}
\tightlist
\item
  Fixed priority of multiple
  \href{/docs/reference/styling/\#show-rules}{show-set rules} : They now
  apply in the same order as normal set rules would
\item
  Show-set rules on the same element (e.g.
  \texttt{\ }{\texttt{\ show\ }}\texttt{\ heading\ }{\texttt{\ .\ }}\texttt{\ }{\texttt{\ where\ }}\texttt{\ }{\texttt{\ (\ }}\texttt{\ level\ }{\texttt{\ :\ }}\texttt{\ }{\texttt{\ 1\ }}\texttt{\ }{\texttt{\ )\ }}\texttt{\ }{\texttt{\ :\ }}\texttt{\ }{\texttt{\ set\ }}\texttt{\ }{\texttt{\ heading\ }}\texttt{\ }{\texttt{\ (\ }}\texttt{\ numbering\ }{\texttt{\ :\ }}\texttt{\ }{\texttt{\ "1."\ }}\texttt{\ }{\texttt{\ )\ }}\texttt{\ }
  ) now work properly
\item
  Setting properties on an element within a transformational show rule
  (e.g.
  \texttt{\ }{\texttt{\ show\ }}\texttt{\ }{\texttt{\ heading\ }}\texttt{\ }{\texttt{\ :\ }}\texttt{\ it\ }{\texttt{\ =\textgreater{}\ }}\texttt{\ }{\texttt{\ \{\ }}\texttt{\ }{\texttt{\ set\ }}\texttt{\ }{\texttt{\ heading\ }}\texttt{\ }{\texttt{\ (\ }}\texttt{\ }{\texttt{\ ..\ }}\texttt{\ }{\texttt{\ )\ }}\texttt{\ }{\texttt{\ ;\ }}\texttt{\ it\ }{\texttt{\ \}\ }}\texttt{\ }
  ) is \textbf{not} supported anymore (previously it also only worked
  sometimes); use show-set rules instead \textbf{(Breaking change)}
\item
  Text show rules that match their own output now work properly (e.g.
  \texttt{\ }{\texttt{\ show\ }}\texttt{\ }{\texttt{\ "cmd"\ }}\texttt{\ }{\texttt{\ :\ }}\texttt{\ }{\texttt{\ \textasciigrave{}cmd\textasciigrave{}\ }}\texttt{\ }
  )
\item
  The elements passed to show rules and returned by queries now contain
  all fields of their respective element functions rather than just
  specific ones
\item
  All settable properties can now be used in
  \href{/docs/reference/foundations/function/\#definitions-where}{where}
  selectors
\item
  \href{/docs/reference/foundations/selector/\#definitions-and}{And} and
  \href{/docs/reference/foundations/selector/\#definitions-or}{or}
  selectors can now be used with show rules
\item
  Errors within show rules and context expressions are now ignored in
  all but the last introspection iteration, in line with the behavior of
  the old
  \href{/docs/reference/introspection/locate/}{\texttt{\ locate\ }}
\item
  Fixed a bug where document set rules were allowed after content
\end{itemize}

\subsection{Layout}\label{layout}

\begin{itemize}
\tightlist
\item
  Added \texttt{\ reflow\ } argument to
  \href{/docs/reference/layout/rotate/}{\texttt{\ rotate\ }} and
  \href{/docs/reference/layout/scale/}{\texttt{\ scale\ }} which lets
  them affect the layout
\item
  Fixed a bug where
  \href{/docs/reference/layout/place/\#parameters-float}{floating
  placement} or
  \href{/docs/reference/model/figure/\#parameters-placement}{floating
  figures} could end up out of order
\item
  Fixed overlap of text and figure for full-page floating figures
\item
  Fixed various cases where the
  \href{/docs/reference/layout/hide/}{\texttt{\ hide\ }} function
  didn\textquotesingle t hide its contents properly
\item
  Fixed usage of \href{/docs/reference/layout/h/}{\texttt{\ h\ }} and
  \href{/docs/reference/layout/v/}{\texttt{\ v\ }} in
  \href{/docs/reference/layout/stack/}{stacks}
\item
  Invisible content like a counter update will no longer force a visible
  block for just itself
\item
  Fixed a bug with horizontal spacing followed by invisible content
  (like a counter update) directly at the start of a paragraph
\end{itemize}

\subsection{Text}\label{text}

\begin{itemize}
\tightlist
\item
  Added
  \href{/docs/reference/text/text/\#parameters-stroke}{\texttt{\ stroke\ }}
  property for text
\item
  Added basic i18n for Serbian and Catalan
\item
  Added support for contemporary Japanese
  \href{/docs/reference/model/numbering/}{numbering} method
\item
  Added patches for various wrong metadata in specific fonts
\item
  The \href{/docs/reference/text/text/\#parameters-dir}{text direction}
  can now be overridden within a paragraph
\item
  Fixed Danish \href{/docs/reference/text/smartquote/}{smart quotes}
\item
  Fixed font fallback next to a line break
\item
  Fixed width adjustment of JIS-style Japanese punctuation
\item
  Fixed Finnish translation of "Listing"
\item
  Fixed Z-ordering of multiple text decorations (underlines, etc.)
\item
  Fixed a bug due to which text
  \href{/docs/reference/text/text/\#parameters-features}{features} could
  not be overridden in consecutive set rules
\end{itemize}

\subsection{Model}\label{model}

\begin{itemize}
\tightlist
\item
  Added
  \href{/docs/reference/model/heading/\#parameters-depth}{\texttt{\ depth\ }}
  and
  \href{/docs/reference/model/heading/\#parameters-offset}{\texttt{\ offset\ }}
  arguments to heading to increase or decrease the heading level for a
  bunch of content; the heading syntax now sets \texttt{\ depth\ }
  rather than \texttt{\ level\ } \textbf{(Breaking change)}
\item
  List \href{/docs/reference/model/list/\#parameters-marker}{markers}
  now cycle by default
\item
  The \href{/docs/reference/model/quote/}{\texttt{\ quote\ }} function
  now more robustly selects the correct quotes based on language and
  nesting
\item
  Fixed indent bugs related to the default show rule of
  \href{/docs/reference/model/terms/}{terms}
\end{itemize}

\subsection{Math}\label{math}

\begin{itemize}
\tightlist
\item
  Inline equations now automatically linebreak at appropriate places
\item
  Added
  \href{/docs/reference/math/equation/\#parameters-number-align}{\texttt{\ number-align\ }}
  argument to equations
\item
  Added support for adjusting the
  \href{/docs/reference/math/accent/\#parameters-size}{\texttt{\ size\ }}
  of accents relative to their base
\item
  Improved positioning of accents
\item
  \href{/docs/reference/math/primes/}{Primes} are now always attached as
  \href{/docs/reference/math/attach/\#functions-scripts}{scripts} by
  default
\item
  Exposed \href{/docs/reference/math/primes/}{\texttt{\ math.primes\ }}
  element which backs the
  \texttt{\ }{\texttt{\ \$\ }}\texttt{\ f\ }{\texttt{\ \textquotesingle{}\ }}\texttt{\ }{\texttt{\ \$\ }}\texttt{\ }
  syntax in math
\item
  Math mode is not affected by
  \href{/docs/reference/model/strong/}{\texttt{\ strong\ }} and
  \href{/docs/reference/model/emph/}{\texttt{\ emph\ }} anymore
\item
  Fixed
  \href{/docs/reference/math/attach/\#functions-attach}{\texttt{\ attach\ }}
  under \href{/docs/reference/math/frac/}{fractions}
\item
  Fixed that \href{/docs/reference/math/class/}{\texttt{\ math.class\ }}
  did not affect smart limit placement
\item
  Fixed weak spacing in
  \href{/docs/reference/math/lr/\#functions-lr}{\texttt{\ lr\ }} groups
\item
  Fixed layout of large operators for Cambria Math font
\item
  Fixed math styling of Hebrew symbol codepoints
\end{itemize}

\subsection{Symbols}\label{symbols}

\begin{itemize}
\tightlist
\item
  Added \texttt{\ gradient\ } as an alias for \texttt{\ nabla\ }
\item
  Added \texttt{\ partial\ } as an alias for \texttt{\ diff\ } ,
  \texttt{\ diff\ } will be deprecated in the future
\item
  Added \texttt{\ colon.double\ } , \texttt{\ gt.approx\ } ,
  \texttt{\ gt.napprox\ } , \texttt{\ lt.approx\ } , and
  \texttt{\ lt.napprox\ }
\item
  Added \texttt{\ arrow.r.tilde\ } and \texttt{\ arrow.l.tilde\ }
\item
  Added \texttt{\ tilde.dot\ }
\item
  Added \texttt{\ forces\ } and \texttt{\ forces.not\ }
\item
  Added \texttt{\ space.nobreak.narrow\ }
\item
  Added \texttt{\ lrm\ } (Left-to-Right Mark) and \texttt{\ rlm\ }
  (Right-to-Left Mark)
\item
  Fixed \texttt{\ star.stroked\ } symbol (which previously had the wrong
  codepoint)
\end{itemize}

\subsection{Scripting}\label{scripting}

\begin{itemize}
\tightlist
\item
  Arrays can now be compared lexicographically
\item
  Added contextual method
  \href{/docs/reference/layout/length/\#definitions-to-absolute}{\texttt{\ to-absolute\ }}
  to lengths
\item
  Added
  \href{/docs/reference/foundations/calc/\#functions-root}{\texttt{\ calc.root\ }}
\item
  Added
  \href{/docs/reference/foundations/int/\#definitions-signum}{\texttt{\ int.signum\ }}
  and
  \href{/docs/reference/foundations/float/\#definitions-signum}{\texttt{\ float.signum\ }}
  methods
\item
  Added
  \href{/docs/reference/foundations/float/\#definitions-is-nan}{\texttt{\ float.is-nan\ }}
  and
  \href{/docs/reference/foundations/float/\#definitions-is-infinite}{\texttt{\ float.is-infinite\ }}
  methods
\item
  Added
  \href{/docs/reference/foundations/int/\#definitions-bit-not}{\texttt{\ int.bit-not\ }}
  ,
  \href{/docs/reference/foundations/int/\#definitions-bit-and}{\texttt{\ int.bit-and\ }}
  ,
  \href{/docs/reference/foundations/int/\#definitions-bit-or}{\texttt{\ int.bit-or\ }}
  ,
  \href{/docs/reference/foundations/int/\#definitions-bit-xor}{\texttt{\ int.bit-xor\ }}
  ,
  \href{/docs/reference/foundations/int/\#definitions-bit-lshift}{\texttt{\ int.bit-lshift\ }}
  , and
  \href{/docs/reference/foundations/int/\#definitions-bit-rshift}{\texttt{\ int.bit-rshift\ }}
  methods
\item
  Added
  \href{/docs/reference/foundations/array/\#definitions-chunks}{\texttt{\ array.chunks\ }}
  method
\item
  A module can now be converted to a dictionary with the
  \href{/docs/reference/foundations/dictionary/\#constructor}{dictionary
  constructor} to access its contents dynamically
\item
  Added
  \href{/docs/reference/data-loading/csv/\#parameters-row-type}{\texttt{\ row-type\ }}
  argument to \texttt{\ csv\ } function to configure how rows will be
  represented
\item
  \href{/docs/reference/data-loading/xml/}{XML parsing} now allows DTDs
  (document type definitions)
\item
  Improved formatting of negative numbers with
  \href{/docs/reference/foundations/str/}{\texttt{\ str\ }} and
  \href{/docs/reference/foundations/repr/}{\texttt{\ repr\ }}
\item
  For loops can now iterate over
  \href{/docs/reference/foundations/bytes/}{bytes}
\item
  Fixed a bug with pattern matching in for loops
\item
  Fixed a bug with labels not being part of
  \href{/docs/reference/foundations/content/\#definitions-fields}{\texttt{\ .\ }{\texttt{\ fields\ }}\texttt{\ }{\texttt{\ (\ }}\texttt{\ }{\texttt{\ )\ }}\texttt{\ }}
  dictionaries
\item
  Fixed a bug where unnamed argument sinks wouldn\textquotesingle t
  capture excess arguments
\item
  Fixed typo in \texttt{\ repr\ } output of strokes
\end{itemize}

\subsection{Syntax}\label{syntax}

\begin{itemize}
\tightlist
\item
  Added support for nested
  \href{/docs/reference/scripting/\#bindings}{destructuring patterns}
\item
  Special spaces (like thin or non-breaking spaces) are now parsed
  literally instead of being collapsed into normal spaces
  \textbf{(Breaking change)}
\item
  Korean text can now use emphasis syntax without adding spaces
  \textbf{(Breaking change)}
\item
  The token \href{/docs/reference/context/}{\texttt{\ context\ }} is now
  a keyword and cannot be used as an identifier anymore
  \textbf{(Breaking change)}
\item
  Nested line comments aren\textquotesingle t allowed anymore in block
  comments \textbf{(Breaking change)}
\item
  Fixed a bug where \texttt{\ x.)\ } would be treated as a field access
\item
  Text elements can now span across curly braces in markup
\item
  Fixed silently wrong parsing when function name is parenthesized
\item
  Fixed various bugs with parsing of destructuring patterns, arrays, and
  dictionaries
\end{itemize}

\subsection{Tooling \& Diagnostics}\label{tooling-diagnostics}

\begin{itemize}
\tightlist
\item
  Click-to-jump now works properly within
  \href{/docs/reference/text/raw/}{\texttt{\ raw\ }} text
\item
  Added suggestion for accessing a field if a method
  doesn\textquotesingle t exist
\item
  Improved hint for calling a function stored in a dictionary
\item
  Improved errors for mutable accessor functions on arrays and
  dictionaries
\item
  Fixed error message when calling constructor of type that
  doesn\textquotesingle t have one
\item
  Fixed confusing error message with nested dictionaries for strokes on
  different sides
\item
  Fixed autocompletion for multiple packages with the same name from
  different namespaces
\end{itemize}

\subsection{Visualization}\label{visualization}

\begin{itemize}
\tightlist
\item
  The \href{/docs/reference/visualize/image/}{\texttt{\ image\ }}
  function doesn\textquotesingle t upscale images beyond their natural
  size anymore
\item
  The \href{/docs/reference/visualize/image/}{\texttt{\ image\ }}
  function now respects rotation stored in EXIF metadata
\item
  Added support for SVG filters
\item
  Added alpha component to
  \href{/docs/reference/visualize/color/\#definitions-luma}{\texttt{\ luma\ }}
  colors
\item
  Added
  \href{/docs/reference/visualize/color/\#definitions-transparentize}{\texttt{\ color.transparentize\ }}
  and
  \href{/docs/reference/visualize/color/\#definitions-opacify}{\texttt{\ color.opacify\ }}
  methods
\item
  Improved
  \href{/docs/reference/visualize/color/\#definitions-negate}{\texttt{\ color.negate\ }}
  function
\item
  Added
  \href{/docs/reference/text/highlight/\#parameters-stroke}{\texttt{\ stroke\ }}
  and
  \href{/docs/reference/text/highlight/\#parameters-radius}{\texttt{\ radius\ }}
  arguments to \texttt{\ highlight\ } function
\item
  Changed default
  \href{/docs/reference/text/highlight/}{\texttt{\ highlight\ }} color
  to be transparent
\item
  CMYK to RGB conversion is now color-managed
\item
  Fixed crash with gradients in Oklch color space
\item
  Fixed color-mixing for hue-based spaces
\item
  Fixed bugs with color conversion
\item
  SVG sizes are not rounded anymore, preventing slightly wrong aspect
  ratios
\item
  Fixed a few other SVG-related bugs
\item
  \href{/docs/reference/visualize/color/\#definitions-components}{\texttt{\ color.components\ }}
  doesn\textquotesingle t round anything anymore
\end{itemize}

\subsection{Export}\label{export}

\begin{itemize}
\tightlist
\item
  PDFs now contain named destinations for headings derived from their
  labels
\item
  The internal PDF structure was changed to make it easier for external
  tools to extract or modify individual pages, avoiding a bug with Typst
  PDFs in Apple Preview
\item
  PDFs produced by Typst should now be byte-by-byte reproducible when
  \texttt{\ }{\texttt{\ set\ }}\texttt{\ }{\texttt{\ document\ }}\texttt{\ }{\texttt{\ (\ }}\texttt{\ date\ }{\texttt{\ :\ }}\texttt{\ }{\texttt{\ none\ }}\texttt{\ }{\texttt{\ )\ }}\texttt{\ }
  is set
\item
  Added missing flag to PDF annotation
\item
  Fixed multiple bugs with gradients in PDF export
\item
  Fixed a bug with patterns in PDF export
\item
  Fixed a bug with embedding of grayscale images in PDF export
\item
  Fixed a bug with To-Unicode mapping of CFF fonts in PDF export
\item
  Fixed a bug with the generation of the PDF outline
\item
  Fixed a sorting bug in PDF export leading to non-reproducible output
\item
  Fixed a bug with transparent text in PNG export
\item
  Exported SVG files now include units in their top-level
  \texttt{\ width\ } and \texttt{\ height\ }
\end{itemize}

\subsection{Command line interface}\label{command-line-interface}

\begin{itemize}
\tightlist
\item
  Added support for passing
  \href{/docs/reference/foundations/sys/}{inputs} via a CLI flag
\item
  When passing the filename \texttt{\ -\ } , Typst will now read input
  from stdin
\item
  Now uses the system-native TLS implementation for network fetching
  which should be generally more robust
\item
  Watch mode will now properly detect when a previously missing file is
  created
\item
  Added \texttt{\ -\/-color\ } flag to configure whether to print
  colored output
\item
  Fixed user agent with which packages are downloaded
\item
  Updated bundled fonts to the newest versions
\end{itemize}

\subsection{Development}\label{development}

\begin{itemize}
\tightlist
\item
  Added \texttt{\ -\/-vendor-openssl\ } to CLI to configure whether to
  link OpenSSL statically instead of dynamically (not applicable to
  Windows and Apple platforms)
\item
  Removed old tracing (and its verbosity) flag from the CLI
\item
  Added new \texttt{\ -\/-timings\ } flag which supersedes the old
  flamegraph profiling in the CLI
\item
  Added minimal CLI to \texttt{\ typst-docs\ } crate for extracting the
  language and standard library documentation as JSON
\item
  The \texttt{\ typst\_pdf::export\ } function\textquotesingle s
  \texttt{\ ident\ } argument switched from \texttt{\ Option\ } to
  \texttt{\ Smart\ } . It should only be set to
  \texttt{\ Smart::Custom\ } if you can provide a stable identifier
  (like the web app can). The CLI sets \texttt{\ Smart::Auto\ } .
\end{itemize}

\subsection{Contributors}\label{contributors}

Thanks to everyone who contributed to this release!

\begin{itemize}
\tightlist
\item
  \href{https://github.com/Leedehai}{\includegraphics[width=0.66667in,height=0.66667in]{https://avatars.githubusercontent.com/u/18319900?s=64&v=4}}
\item
  \href{https://github.com/PgBiel}{\includegraphics[width=0.66667in,height=0.66667in]{https://avatars.githubusercontent.com/u/9021226?s=64&v=4}}
\item
  \href{https://github.com/frozolotl}{\includegraphics[width=0.66667in,height=0.66667in]{https://avatars.githubusercontent.com/u/44589151?s=64&v=4}}
\item
  \href{https://github.com/Dherse}{\includegraphics[width=0.66667in,height=0.66667in]{https://avatars.githubusercontent.com/u/9665250?s=64&v=4}}
\item
  \href{https://github.com/MDLC01}{\includegraphics[width=0.66667in,height=0.66667in]{https://avatars.githubusercontent.com/u/57839069?s=64&v=4}}
\item
  \href{https://github.com/Andrew15-5}{\includegraphics[width=0.66667in,height=0.66667in]{https://avatars.githubusercontent.com/u/37143421?s=64&v=4}}
\item
  \href{https://github.com/Enter-tainer}{\includegraphics[width=0.66667in,height=0.66667in]{https://avatars.githubusercontent.com/u/25521218?s=64&v=4}}
\item
  \href{https://github.com/Myriad-Dreamin}{\includegraphics[width=0.66667in,height=0.66667in]{https://avatars.githubusercontent.com/u/35292584?s=64&v=4}}
\item
  \href{https://github.com/peng1999}{\includegraphics[width=0.66667in,height=0.66667in]{https://avatars.githubusercontent.com/u/12483662?s=64&v=4}}
\item
  \href{https://github.com/EpicEricEE}{\includegraphics[width=0.66667in,height=0.66667in]{https://avatars.githubusercontent.com/u/7191192?s=64&v=4}}
\item
  \href{https://github.com/jcbhmr}{\includegraphics[width=0.66667in,height=0.66667in]{https://avatars.githubusercontent.com/u/61068799?s=64&v=4}}
\item
  \href{https://github.com/Heinenen}{\includegraphics[width=0.66667in,height=0.66667in]{https://avatars.githubusercontent.com/u/37484430?s=64&v=4}}
\item
  \href{https://github.com/tingerrr}{\includegraphics[width=0.66667in,height=0.66667in]{https://avatars.githubusercontent.com/u/137803093?s=64&v=4}}
\item
  \href{https://github.com/Jollywatt}{\includegraphics[width=0.66667in,height=0.66667in]{https://avatars.githubusercontent.com/u/24970860?s=64&v=4}}
\item
  \href{https://github.com/LaurenzV}{\includegraphics[width=0.66667in,height=0.66667in]{https://avatars.githubusercontent.com/u/47084093?s=64&v=4}}
\item
  \href{https://github.com/YDX-2147483647}{\includegraphics[width=0.66667in,height=0.66667in]{https://avatars.githubusercontent.com/u/73375426?s=64&v=4}}
\item
  \href{https://github.com/k-84mo10}{\includegraphics[width=0.66667in,height=0.66667in]{https://avatars.githubusercontent.com/u/115140693?s=64&v=4}}
\item
  \href{https://github.com/s-cerevisiae}{\includegraphics[width=0.66667in,height=0.66667in]{https://avatars.githubusercontent.com/u/28475392?s=64&v=4}}
\item
  \href{https://github.com/01mf02}{\includegraphics[width=0.66667in,height=0.66667in]{https://avatars.githubusercontent.com/u/864342?s=64&v=4}}
\item
  \href{https://github.com/A-Walrus}{\includegraphics[width=0.66667in,height=0.66667in]{https://avatars.githubusercontent.com/u/58790821?s=64&v=4}}
\item
  \href{https://github.com/C0ffeeCode}{\includegraphics[width=0.66667in,height=0.66667in]{https://avatars.githubusercontent.com/u/27804482?s=64&v=4}}
\item
  \href{https://github.com/CosmicHorrorDev}{\includegraphics[width=0.66667in,height=0.66667in]{https://avatars.githubusercontent.com/u/30302768?s=64&v=4}}
\item
  \href{https://github.com/DavidMazarro}{\includegraphics[width=0.66667in,height=0.66667in]{https://avatars.githubusercontent.com/u/22799724?s=64&v=4}}
\item
  \href{https://github.com/Hofer-Julian}{\includegraphics[width=0.66667in,height=0.66667in]{https://avatars.githubusercontent.com/u/30049909?s=64&v=4}}
\item
  \href{https://github.com/Jocs}{\includegraphics[width=0.66667in,height=0.66667in]{https://avatars.githubusercontent.com/u/9712830?s=64&v=4}}
\item
  \href{https://github.com/Midbin}{\includegraphics[width=0.66667in,height=0.66667in]{https://avatars.githubusercontent.com/u/5499764?s=64&v=4}}
\item
  \href{https://github.com/Quaqqer}{\includegraphics[width=0.66667in,height=0.66667in]{https://avatars.githubusercontent.com/u/29512047?s=64&v=4}}
\item
  \href{https://github.com/artemist}{\includegraphics[width=0.66667in,height=0.66667in]{https://avatars.githubusercontent.com/u/1226638?s=64&v=4}}
\item
  \href{https://github.com/astrale-sharp}{\includegraphics[width=0.66667in,height=0.66667in]{https://avatars.githubusercontent.com/u/53686698?s=64&v=4}}
\item
  \href{https://github.com/cmoog}{\includegraphics[width=0.66667in,height=0.66667in]{https://avatars.githubusercontent.com/u/7585078?s=64&v=4}}
\item
  \href{https://github.com/drupol}{\includegraphics[width=0.66667in,height=0.66667in]{https://avatars.githubusercontent.com/u/252042?s=64&v=4}}
\item
  \href{https://github.com/dyc3}{\includegraphics[width=0.66667in,height=0.66667in]{https://avatars.githubusercontent.com/u/1808807?s=64&v=4}}
\item
  \href{https://github.com/eltociear}{\includegraphics[width=0.66667in,height=0.66667in]{https://avatars.githubusercontent.com/u/22633385?s=64&v=4}}
\item
  \href{https://github.com/emilyyyylime}{\includegraphics[width=0.66667in,height=0.66667in]{https://avatars.githubusercontent.com/u/40892795?s=64&v=4}}
\item
  \href{https://github.com/emmett-rayes}{\includegraphics[width=0.66667in,height=0.66667in]{https://avatars.githubusercontent.com/u/109030944?s=64&v=4}}
\item
  \href{https://github.com/espinielli}{\includegraphics[width=0.66667in,height=0.66667in]{https://avatars.githubusercontent.com/u/891692?s=64&v=4}}
\item
  \href{https://github.com/evfinkn}{\includegraphics[width=0.66667in,height=0.66667in]{https://avatars.githubusercontent.com/u/6924485?s=64&v=4}}
\item
  \href{https://github.com/f3rn0s}{\includegraphics[width=0.66667in,height=0.66667in]{https://avatars.githubusercontent.com/u/1351279?s=64&v=4}}
\item
  \href{https://github.com/freundTech}{\includegraphics[width=0.66667in,height=0.66667in]{https://avatars.githubusercontent.com/u/9515067?s=64&v=4}}
\item
  \href{https://github.com/geekvest}{\includegraphics[width=0.66667in,height=0.66667in]{https://avatars.githubusercontent.com/u/126322776?s=64&v=4}}
\item
  \href{https://github.com/h-arry-smith}{\includegraphics[width=0.66667in,height=0.66667in]{https://avatars.githubusercontent.com/u/69302535?s=64&v=4}}
\item
  \href{https://github.com/istudyatuni}{\includegraphics[width=0.66667in,height=0.66667in]{https://avatars.githubusercontent.com/u/43654815?s=64&v=4}}
\item
  \href{https://github.com/jaroeichler}{\includegraphics[width=0.66667in,height=0.66667in]{https://avatars.githubusercontent.com/u/88505041?s=64&v=4}}
\item
  \href{https://github.com/jbirnick}{\includegraphics[width=0.66667in,height=0.66667in]{https://avatars.githubusercontent.com/u/6528009?s=64&v=4}}
\item
  \href{https://github.com/kg583}{\includegraphics[width=0.66667in,height=0.66667in]{https://avatars.githubusercontent.com/u/41345727?s=64&v=4}}
\item
  \href{https://github.com/mattfbacon}{\includegraphics[width=0.66667in,height=0.66667in]{https://avatars.githubusercontent.com/u/58113890?s=64&v=4}}
\item
  \href{https://github.com/max397574}{\includegraphics[width=0.66667in,height=0.66667in]{https://avatars.githubusercontent.com/u/81827001?s=64&v=4}}
\item
  \href{https://github.com/mkpoli}{\includegraphics[width=0.66667in,height=0.66667in]{https://avatars.githubusercontent.com/u/3502597?s=64&v=4}}
\item
  \href{https://github.com/mkroening}{\includegraphics[width=0.66667in,height=0.66667in]{https://avatars.githubusercontent.com/u/28776973?s=64&v=4}}
\item
  \href{https://github.com/muzimuzhi}{\includegraphics[width=0.66667in,height=0.66667in]{https://avatars.githubusercontent.com/u/6376638?s=64&v=4}}
\item
  \href{https://github.com/nathaniel-brough}{\includegraphics[width=0.66667in,height=0.66667in]{https://avatars.githubusercontent.com/u/7277663?s=64&v=4}}
\item
  \href{https://github.com/natsukagami}{\includegraphics[width=0.66667in,height=0.66667in]{https://avatars.githubusercontent.com/u/9061737?s=64&v=4}}
\item
  \href{https://github.com/nvarner}{\includegraphics[width=0.66667in,height=0.66667in]{https://avatars.githubusercontent.com/u/17197562?s=64&v=4}}
\item
  \href{https://github.com/onerandomusername}{\includegraphics[width=0.66667in,height=0.66667in]{https://avatars.githubusercontent.com/u/71233171?s=64&v=4}}
\item
  \href{https://github.com/pineapplehunter}{\includegraphics[width=0.66667in,height=0.66667in]{https://avatars.githubusercontent.com/u/8869894?s=64&v=4}}
\item
  \href{https://github.com/t-rapp}{\includegraphics[width=0.66667in,height=0.66667in]{https://avatars.githubusercontent.com/u/20061583?s=64&v=4}}
\item
  \href{https://github.com/tertsdiepraam}{\includegraphics[width=0.66667in,height=0.66667in]{https://avatars.githubusercontent.com/u/11643477?s=64&v=4}}
\item
  \href{https://github.com/uetcis}{\includegraphics[width=0.66667in,height=0.66667in]{https://avatars.githubusercontent.com/u/25654046?s=64&v=4}}
\item
  \href{https://github.com/violetfauna}{\includegraphics[width=0.66667in,height=0.66667in]{https://avatars.githubusercontent.com/u/44212740?s=64&v=4}}
\item
  \href{https://github.com/voidiz}{\includegraphics[width=0.66667in,height=0.66667in]{https://avatars.githubusercontent.com/u/29259387?s=64&v=4}}
\item
  \href{https://github.com/xTeamStanly}{\includegraphics[width=0.66667in,height=0.66667in]{https://avatars.githubusercontent.com/u/68441372?s=64&v=4}}
\item
  \href{https://github.com/xalbd}{\includegraphics[width=0.66667in,height=0.66667in]{https://avatars.githubusercontent.com/u/119540449?s=64&v=4}}
\item
  \href{https://github.com/xkevio}{\includegraphics[width=0.66667in,height=0.66667in]{https://avatars.githubusercontent.com/u/13004777?s=64&v=4}}
\item
  \href{https://github.com/zica87}{\includegraphics[width=0.66667in,height=0.66667in]{https://avatars.githubusercontent.com/u/59327276?s=64&v=4}}
\end{itemize}

\href{/docs/changelog/0.11.1/}{\pandocbounded{\includesvg[keepaspectratio]{/assets/icons/16-arrow-right.svg}}}

{ 0.11.1 } { Previous page }

\href{/docs/changelog/0.10.0/}{\pandocbounded{\includesvg[keepaspectratio]{/assets/icons/16-arrow-right.svg}}}

{ 0.10.0 } { Next page }


\section{Docs LaTeX/typst.app/docs/changelog/0.11.1.tex}
\title{typst.app/docs/changelog/0.11.1}

\begin{itemize}
\tightlist
\item
  \href{/docs}{\includesvg[width=0.16667in,height=0.16667in]{/assets/icons/16-docs-dark.svg}}
\item
  \includesvg[width=0.16667in,height=0.16667in]{/assets/icons/16-arrow-right.svg}
\item
  \href{/docs/changelog/}{Changelog}
\item
  \includesvg[width=0.16667in,height=0.16667in]{/assets/icons/16-arrow-right.svg}
\item
  \href{/docs/changelog/0.11.1/}{0.11.1}
\end{itemize}

\section{Version 0.11.1 (May 17,
2024)}\label{version-0.11.1-may-17-2024}

\subsection{Security}\label{security}

\begin{itemize}
\tightlist
\item
  Fixed a vulnerability where image files at known paths could be
  embedded into the PDF even if they were outside of the project
  directory
\end{itemize}

\subsection{Bibliography}\label{bibliography}

\begin{itemize}
\tightlist
\item
  Fixed et-al handling in subsequent citations
\item
  Fixed suppression of title for citations and bibliography references
  with no author
\item
  Fixed handling of initials in citation styles without a delimiter
\item
  Fixed bug with citations in footnotes
\end{itemize}

\subsection{Text and Layout}\label{text-and-layout}

\begin{itemize}
\tightlist
\item
  Fixed interaction of
  \href{/docs/reference/model/par/\#parameters-first-line-indent}{\texttt{\ first-line-indent\ }}
  and \href{/docs/reference/model/outline/}{\texttt{\ outline\ }}
\item
  Fixed compression of CJK punctuation marks at line start and end
\item
  Fixed handling of \href{/docs/reference/visualize/rect/}{rectangles}
  with negative dimensions
\item
  Fixed layout of
  \href{/docs/reference/visualize/path/}{\texttt{\ path\ }} in
  explicitly sized container
\item
  Fixed broken \href{/docs/reference/text/raw/}{\texttt{\ raw\ }} text
  in right-to-left paragraphs
\item
  Fixed tab rendering in \texttt{\ raw\ } text with language
  \texttt{\ typ\ } or \texttt{\ typc\ }
\item
  Fixed highlighting of multi-line \texttt{\ raw\ } text enclosed by
  single backticks
\item
  Fixed indentation of overflowing lines in \texttt{\ raw\ } blocks
\item
  Fixed extra space when \texttt{\ raw\ } text ends with a backtick
\end{itemize}

\subsection{Math}\label{math}

\begin{itemize}
\tightlist
\item
  Fixed broken \href{/docs/reference/math/equation/}{equations} in
  right-to-left paragraphs
\item
  Fixed missing
  \href{/docs/reference/math/variants/\#functions-bb}{blackboard bold}
  letters
\item
  Fixed error on empty arguments in 2D math argument list
\item
  Fixed stretching via
  \href{/docs/reference/math/lr/\#functions-mid}{\texttt{\ mid\ }} for
  various characters
\item
  Fixed that alignment points in equations were affected by
  \texttt{\ }{\texttt{\ set\ }}\texttt{\ }{\texttt{\ align\ }}\texttt{\ }{\texttt{\ (\ }}\texttt{\ }{\texttt{\ ..\ }}\texttt{\ }{\texttt{\ )\ }}\texttt{\ }
\end{itemize}

\subsection{Export}\label{export}

\begin{itemize}
\tightlist
\item
  Fixed \href{/docs/reference/text/smartquote/}{smart quotes} in PDF
  outline
\item
  Fixed \href{/docs/reference/visualize/pattern/}{patterns} with spacing
  in PDF
\item
  Fixed wrong PDF page labels when
  \href{/docs/reference/layout/page/\#parameters-numbering}{page
  numbering} was disabled after being previously enabled
\end{itemize}

\subsection{Scripting}\label{scripting}

\begin{itemize}
\tightlist
\item
  Fixed overflow for large numbers in external data files (by converting
  to floats instead)
\item
  Fixed
  \href{/docs/reference/foundations/str/\#definitions-trim}{\texttt{\ str\ }{\texttt{\ .\ }}\texttt{\ }{\texttt{\ trim\ }}\texttt{\ }{\texttt{\ (\ }}\texttt{\ regex\ }{\texttt{\ ,\ }}\texttt{\ at\ }{\texttt{\ :\ }}\texttt{\ end\ }{\texttt{\ )\ }}\texttt{\ }}
  when the whole string is matched
\end{itemize}

\subsection{Miscellaneous}\label{miscellaneous}

\begin{itemize}
\tightlist
\item
  Fixed deformed strokes for specific shapes and thicknesses
\item
  Fixed newline handling in code mode: There can now be comments within
  chained method calls and between an \texttt{\ if\ } branch and the
  \texttt{\ else\ } keyword
\item
  Fixed inefficiency with incremental reparsing
\item
  Fixed autocompletions for relative file imports
\item
  Fixed crash in autocompletion handler
\item
  Fixed a bug where the path and entrypoint printed by
  \texttt{\ typst\ init\ } were not properly escaped
\item
  Fixed various documentation errors
\end{itemize}

\subsection{Contributors}\label{contributors}

Thanks to everyone who contributed to this release!

\begin{itemize}
\tightlist
\item
  \href{https://github.com/Leedehai}{\includegraphics[width=0.66667in,height=0.66667in]{https://avatars.githubusercontent.com/u/18319900?s=64&v=4}}
\item
  \href{https://github.com/elegaanz}{\includegraphics[width=0.66667in,height=0.66667in]{https://avatars.githubusercontent.com/u/16254623?s=64&v=4}}
\item
  \href{https://github.com/frozolotl}{\includegraphics[width=0.66667in,height=0.66667in]{https://avatars.githubusercontent.com/u/44589151?s=64&v=4}}
\item
  \href{https://github.com/A-Walrus}{\includegraphics[width=0.66667in,height=0.66667in]{https://avatars.githubusercontent.com/u/58790821?s=64&v=4}}
\item
  \href{https://github.com/MDLC01}{\includegraphics[width=0.66667in,height=0.66667in]{https://avatars.githubusercontent.com/u/57839069?s=64&v=4}}
\item
  \href{https://github.com/Myriad-Dreamin}{\includegraphics[width=0.66667in,height=0.66667in]{https://avatars.githubusercontent.com/u/35292584?s=64&v=4}}
\item
  \href{https://github.com/3w36zj6}{\includegraphics[width=0.66667in,height=0.66667in]{https://avatars.githubusercontent.com/u/52315048?s=64&v=4}}
\item
  \href{https://github.com/Enter-tainer}{\includegraphics[width=0.66667in,height=0.66667in]{https://avatars.githubusercontent.com/u/25521218?s=64&v=4}}
\item
  \href{https://github.com/EpicEricEE}{\includegraphics[width=0.66667in,height=0.66667in]{https://avatars.githubusercontent.com/u/7191192?s=64&v=4}}
\item
  \href{https://github.com/Jocs}{\includegraphics[width=0.66667in,height=0.66667in]{https://avatars.githubusercontent.com/u/9712830?s=64&v=4}}
\item
  \href{https://github.com/PgBiel}{\includegraphics[width=0.66667in,height=0.66667in]{https://avatars.githubusercontent.com/u/9021226?s=64&v=4}}
\item
  \href{https://github.com/QuarticCat}{\includegraphics[width=0.66667in,height=0.66667in]{https://avatars.githubusercontent.com/u/70888415?s=64&v=4}}
\item
  \href{https://github.com/Tom4sCruz}{\includegraphics[width=0.66667in,height=0.66667in]{https://avatars.githubusercontent.com/u/103905440?s=64&v=4}}
\item
  \href{https://github.com/bluebear94}{\includegraphics[width=0.66667in,height=0.66667in]{https://avatars.githubusercontent.com/u/2975203?s=64&v=4}}
\item
  \href{https://github.com/etiennecollin}{\includegraphics[width=0.66667in,height=0.66667in]{https://avatars.githubusercontent.com/u/99756528?s=64&v=4}}
\item
  \href{https://github.com/gabriel-araujjo}{\includegraphics[width=0.66667in,height=0.66667in]{https://avatars.githubusercontent.com/u/3980936?s=64&v=4}}
\item
  \href{https://github.com/joserlopes}{\includegraphics[width=0.66667in,height=0.66667in]{https://avatars.githubusercontent.com/u/95137505?s=64&v=4}}
\item
  \href{https://github.com/rikhuijzer}{\includegraphics[width=0.66667in,height=0.66667in]{https://avatars.githubusercontent.com/u/20724914?s=64&v=4}}
\item
  \href{https://github.com/wrzian}{\includegraphics[width=0.66667in,height=0.66667in]{https://avatars.githubusercontent.com/u/133046678?s=64&v=4}}
\end{itemize}

\href{/docs/changelog/0.12.0/}{\pandocbounded{\includesvg[keepaspectratio]{/assets/icons/16-arrow-right.svg}}}

{ 0.12.0 } { Previous page }

\href{/docs/changelog/0.11.0/}{\pandocbounded{\includesvg[keepaspectratio]{/assets/icons/16-arrow-right.svg}}}

{ 0.11.0 } { Next page }


\section{Docs LaTeX/typst.app/docs/changelog/earlier.tex}
\title{typst.app/docs/changelog/earlier}

\begin{itemize}
\tightlist
\item
  \href{/docs}{\includesvg[width=0.16667in,height=0.16667in]{/assets/icons/16-docs-dark.svg}}
\item
  \includesvg[width=0.16667in,height=0.16667in]{/assets/icons/16-arrow-right.svg}
\item
  \href{/docs/changelog/}{Changelog}
\item
  \includesvg[width=0.16667in,height=0.16667in]{/assets/icons/16-arrow-right.svg}
\item
  \href{/docs/changelog/earlier/}{Earlier}
\end{itemize}

\section{Changes in early, unversioned
Typst}\label{changes-in-early-unversioned-typst}

\subsection{March 28, 2023}\label{march-28-2023}

\begin{itemize}
\item
  \textbf{Breaking changes:}

  \begin{itemize}
  \tightlist
  \item
    Enumerations now require a space after their marker, that is,
    \texttt{\ 1.ok\ } must now be written as
    \texttt{\ }{\texttt{\ 1.\ }}\texttt{\ ok\ }
  \item
    Changed default style for \href{/docs/reference/model/terms/}{term
    lists} : Does not include a colon anymore and has a bit more indent
  \end{itemize}
\item
  Command line interface

  \begin{itemize}
  \tightlist
  \item
    Added \texttt{\ -\/-font-path\ } argument for CLI
  \item
    Embedded default fonts in CLI binary
  \item
    Fixed build of CLI if \texttt{\ git\ } is not installed
  \end{itemize}
\item
  Miscellaneous improvements

  \begin{itemize}
  \tightlist
  \item
    Added support for disabling \href{/docs/reference/math/mat/}{matrix}
    and \href{/docs/reference/math/vec/}{vector} delimiters. Generally
    with
    \texttt{\ }{\texttt{\ \#\ }}\texttt{\ }{\texttt{\ set\ }}\texttt{\ math\ }{\texttt{\ .\ }}\texttt{\ }{\texttt{\ mat\ }}\texttt{\ }{\texttt{\ (\ }}\texttt{\ delim\ }{\texttt{\ :\ }}\texttt{\ }{\texttt{\ none\ }}\texttt{\ }{\texttt{\ )\ }}\texttt{\ }
    or one-off with
    \texttt{\ }{\texttt{\ \$\ }}\texttt{\ }{\texttt{\ mat\ }}\texttt{\ }{\texttt{\ (\ }}\texttt{\ delim\ }{\texttt{\ :\ }}\texttt{\ }{\texttt{\ \#\ }}\texttt{\ }{\texttt{\ none\ }}\texttt{\ }{\texttt{\ ,\ }}\texttt{\ 1\ }{\texttt{\ ,\ }}\texttt{\ 2\ }{\texttt{\ ;\ }}\texttt{\ 3\ }{\texttt{\ ,\ }}\texttt{\ 4\ }{\texttt{\ )\ }}\texttt{\ }{\texttt{\ \$\ }}\texttt{\ }
    .
  \item
    Added
    \href{/docs/reference/model/terms/\#parameters-separator}{\texttt{\ separator\ }}
    argument to term lists
  \item
    Added
    \href{/docs/reference/math/lr/\#functions-round}{\texttt{\ round\ }}
    function for equations
  \item
    Numberings now allow zeros. To reset a counter, you can write
    \texttt{\ }{\texttt{\ \#\ }}\texttt{\ }{\texttt{\ counter\ }}\texttt{\ }{\texttt{\ (\ }}\texttt{\ }{\texttt{\ ..\ }}\texttt{\ }{\texttt{\ )\ }}\texttt{\ }{\texttt{\ .\ }}\texttt{\ }{\texttt{\ update\ }}\texttt{\ }{\texttt{\ (\ }}\texttt{\ }{\texttt{\ 0\ }}\texttt{\ }{\texttt{\ )\ }}\texttt{\ }
  \item
    Added documentation for
    \texttt{\ }{\texttt{\ page\ }}\texttt{\ }{\texttt{\ (\ }}\texttt{\ }{\texttt{\ )\ }}\texttt{\ }
    and
    \texttt{\ }{\texttt{\ position\ }}\texttt{\ }{\texttt{\ (\ }}\texttt{\ }{\texttt{\ )\ }}\texttt{\ }
    methods on
    \href{/docs/reference/introspection/location/}{\texttt{\ location\ }}
    type
  \item
    Added symbols for double, triple, and quadruple dot accent
  \item
    Added smart quotes for Norwegian Bokmål
  \item
    Added Nix flake
  \item
    Fixed bibliography ordering in IEEE style
  \item
    Fixed parsing of decimals in math:
    \texttt{\ }{\texttt{\ \$\ }}\texttt{\ 1.2\ }{\texttt{\ /\ }}\texttt{\ 3.4\ }{\texttt{\ \$\ }}\texttt{\ }
  \item
    Fixed parsing of unbalanced delimiters in fractions:
    \texttt{\ }{\texttt{\ \$\ }}\texttt{\ 1\ }{\texttt{\ /\ }}\texttt{\ (2\ (x)\ }{\texttt{\ \$\ }}\texttt{\ }
  \item
    Fixed unexpected parsing of numbers as enumerations, e.g. in
    \texttt{\ 1.2\ }
  \item
    Fixed combination of page fill and header
  \item
    Fixed compiler crash if
    \href{/docs/reference/layout/repeat/}{\texttt{\ repeat\ }} is used
    in page with automatic width
  \item
    Fixed \href{/docs/reference/math/mat/}{matrices} with explicit
    delimiter
  \item
    Fixed
    \href{/docs/reference/model/terms/\#parameters-indent}{\texttt{\ indent\ }}
    property of term lists
  \item
    Numerous documentation fixes
  \item
    Links in bibliographies are now affected by link styling
  \item
    Fixed hovering over comments in web app
  \end{itemize}
\end{itemize}

Thanks to everyone who contributed to this release!

\begin{itemize}
\tightlist
\item
  \href{https://github.com/figsoda}{\includegraphics[width=0.66667in,height=0.66667in]{https://avatars.githubusercontent.com/u/40620903?s=64&v=4}}
\item
  \href{https://github.com/loewenheim}{\includegraphics[width=0.66667in,height=0.66667in]{https://avatars.githubusercontent.com/u/7622248?s=64&v=4}}
\item
  \href{https://github.com/7sDream}{\includegraphics[width=0.66667in,height=0.66667in]{https://avatars.githubusercontent.com/u/7822577?s=64&v=4}}
\item
  \href{https://github.com/AlexanderEkdahl}{\includegraphics[width=0.66667in,height=0.66667in]{https://avatars.githubusercontent.com/u/1409734?s=64&v=4}}
\item
  \href{https://github.com/C284974}{\includegraphics[width=0.66667in,height=0.66667in]{https://avatars.githubusercontent.com/u/67071431?s=64&v=4}}
\item
  \href{https://github.com/Dragoncraft89}{\includegraphics[width=0.66667in,height=0.66667in]{https://avatars.githubusercontent.com/u/11162205?s=64&v=4}}
\item
  \href{https://github.com/Easyoakland}{\includegraphics[width=0.66667in,height=0.66667in]{https://avatars.githubusercontent.com/u/97992568?s=64&v=4}}
\item
  \href{https://github.com/KiruyaMomochi}{\includegraphics[width=0.66667in,height=0.66667in]{https://avatars.githubusercontent.com/u/65301509?s=64&v=4}}
\item
  \href{https://github.com/LaurenzV}{\includegraphics[width=0.66667in,height=0.66667in]{https://avatars.githubusercontent.com/u/47084093?s=64&v=4}}
\item
  \href{https://github.com/ModProg}{\includegraphics[width=0.66667in,height=0.66667in]{https://avatars.githubusercontent.com/u/11978847?s=64&v=4}}
\item
  \href{https://github.com/RafDevX}{\includegraphics[width=0.66667in,height=0.66667in]{https://avatars.githubusercontent.com/u/56204853?s=64&v=4}}
\item
  \href{https://github.com/Tom-stack3}{\includegraphics[width=0.66667in,height=0.66667in]{https://avatars.githubusercontent.com/u/76645845?s=64&v=4}}
\item
  \href{https://github.com/arctic-penguin}{\includegraphics[width=0.66667in,height=0.66667in]{https://avatars.githubusercontent.com/u/103587359?s=64&v=4}}
\item
  \href{https://github.com/barvirm}{\includegraphics[width=0.66667in,height=0.66667in]{https://avatars.githubusercontent.com/u/15948420?s=64&v=4}}
\item
  \href{https://github.com/dreamlax}{\includegraphics[width=0.66667in,height=0.66667in]{https://avatars.githubusercontent.com/u/4708805?s=64&v=4}}
\item
  \href{https://github.com/dvdvgt}{\includegraphics[width=0.66667in,height=0.66667in]{https://avatars.githubusercontent.com/u/40773635?s=64&v=4}}
\item
  \href{https://github.com/flxo}{\includegraphics[width=0.66667in,height=0.66667in]{https://avatars.githubusercontent.com/u/129436?s=64&v=4}}
\item
  \href{https://github.com/jakobrs}{\includegraphics[width=0.66667in,height=0.66667in]{https://avatars.githubusercontent.com/u/10761079?s=64&v=4}}
\item
  \href{https://github.com/jdm204}{\includegraphics[width=0.66667in,height=0.66667in]{https://avatars.githubusercontent.com/u/34479575?s=64&v=4}}
\item
  \href{https://github.com/kianmeng}{\includegraphics[width=0.66667in,height=0.66667in]{https://avatars.githubusercontent.com/u/134518?s=64&v=4}}
\item
  \href{https://github.com/liamsanft}{\includegraphics[width=0.66667in,height=0.66667in]{https://avatars.githubusercontent.com/u/38260698?s=64&v=4}}
\item
  \href{https://github.com/oskgo}{\includegraphics[width=0.66667in,height=0.66667in]{https://avatars.githubusercontent.com/u/92018610?s=64&v=4}}
\item
  \href{https://github.com/rpitasky}{\includegraphics[width=0.66667in,height=0.66667in]{https://avatars.githubusercontent.com/u/111201305?s=64&v=4}}
\item
  \href{https://github.com/thecaralice}{\includegraphics[width=0.66667in,height=0.66667in]{https://avatars.githubusercontent.com/u/43097806?s=64&v=4}}
\item
  \href{https://github.com/user202729}{\includegraphics[width=0.66667in,height=0.66667in]{https://avatars.githubusercontent.com/u/25191436?s=64&v=4}}
\item
  \href{https://github.com/vxpm}{\includegraphics[width=0.66667in,height=0.66667in]{https://avatars.githubusercontent.com/u/59714841?s=64&v=4}}
\end{itemize}

\subsection{March 21, 2023}\label{march-21-2023}

\begin{itemize}
\item
  Reference and bibliography management

  \begin{itemize}
  \tightlist
  \item
    \href{/docs/reference/model/bibliography/}{Bibliographies} and
    \href{/docs/reference/model/cite/}{citations} (currently supported
    styles are APA, Chicago Author Date, IEEE, and MLA)
  \item
    You can now \href{/docs/reference/model/ref/}{reference} sections,
    figures, formulas, and works from the bibliography with
    \texttt{\ }{\texttt{\ @label\ }}\texttt{\ }
  \item
    You can make an element referenceable with a label:

    \begin{itemize}
    \tightlist
    \item
      \texttt{\ }{\texttt{\ =\ Introduction\ }}\texttt{\ }{\texttt{\ \textless{}intro\textgreater{}\ }}\texttt{\ }
    \item
      \texttt{\ }{\texttt{\ \$\ }}\texttt{\ A\ =\ }{\texttt{\ pi\ }}\texttt{\ r\ }{\texttt{\ \^{}\ }}\texttt{\ 2\ }{\texttt{\ \$\ }}\texttt{\ }{\texttt{\ \textless{}area\textgreater{}\ }}\texttt{\ }
    \end{itemize}
  \end{itemize}
\item
  Introspection system for interactions between different parts of the
  document

  \begin{itemize}
  \tightlist
  \item
    \href{/docs/reference/introspection/counter/}{\texttt{\ counter\ }}
    function

    \begin{itemize}
    \tightlist
    \item
      Access and modify counters for pages, headings, figures, and
      equations
    \item
      Define and use your own custom counters
    \item
      Time travel: Find out what the counter value was or will be at
      some other point in the document (e.g. when you\textquotesingle re
      building a list of figures, you can determine the value of the
      figure counter at any given figure).
    \item
      Counters count in layout order and not in code order
    \end{itemize}
  \item
    \href{/docs/reference/introspection/state/}{\texttt{\ state\ }}
    function

    \begin{itemize}
    \tightlist
    \item
      Manage arbitrary state across your document
    \item
      Time travel: Find out the value of your state at any position in
      the document
    \item
      State is modified in layout order and not in code order
    \end{itemize}
  \item
    \href{/docs/reference/introspection/query/}{\texttt{\ query\ }}
    function

    \begin{itemize}
    \tightlist
    \item
      Find all occurrences of an element or a label, either in the whole
      document or before/after some location
    \item
      Link to elements, find out their position on the pages and access
      their fields
    \item
      Example use cases: Custom list of figures or page header with
      current chapter title
    \end{itemize}
  \item
    \href{/docs/reference/introspection/locate/}{\texttt{\ locate\ }}
    function

    \begin{itemize}
    \tightlist
    \item
      Determines the location of itself in the final layout
    \item
      Can be accessed to get the \texttt{\ page\ } and \texttt{\ x\ } ,
      \texttt{\ y\ } coordinates
    \item
      Can be used with counters and state to find out their values at
      that location
    \item
      Can be used with queries to find elements before or after its
      location
    \end{itemize}
  \end{itemize}
\item
  New \href{/docs/reference/layout/measure/}{\texttt{\ measure\ }}
  function

  \begin{itemize}
  \tightlist
  \item
    Measure the layouted size of elements
  \item
    To be used in combination with the new
    \href{/docs/reference/foundations/style/}{\texttt{\ style\ }}
    function that lets you generate different content based on the style
    context something is inserted into (because that affects the
    measured size of content)
  \end{itemize}
\item
  Exposed content representation

  \begin{itemize}
  \tightlist
  \item
    Content is not opaque anymore
  \item
    Content can be compared for equality
  \item
    The tree of content elements can be traversed with code
  \item
    Can be observed in hover tooltips or with
    \href{/docs/reference/foundations/repr/}{\texttt{\ repr\ }}
  \item
    New \href{/docs/reference/foundations/content/}{methods} on content:
    \texttt{\ func\ } , \texttt{\ has\ } , \texttt{\ at\ } , and
    \texttt{\ location\ }
  \item
    All optional fields on elements are now settable
  \item
    More uniform field names ( \texttt{\ heading.title\ } becomes
    \texttt{\ heading.body\ } , \texttt{\ list.items\ } becomes
    \texttt{\ list.children\ } , and a few more changes)
  \end{itemize}
\item
  Further improvements

  \begin{itemize}
  \tightlist
  \item
    Added \href{/docs/reference/model/figure/}{\texttt{\ figure\ }}
    function
  \item
    Added
    \href{/docs/reference/math/equation/\#parameters-numbering}{\texttt{\ numbering\ }}
    parameter on equation function
  \item
    Added
    \href{/docs/reference/layout/page/\#parameters-numbering}{\texttt{\ numbering\ }}
    and
    \href{/docs/reference/layout/page/\#parameters-number-align}{\texttt{\ number-align\ }}
    parameters on page function
  \item
    The page function\textquotesingle s
    \href{/docs/reference/layout/page/\#parameters-header}{\texttt{\ header\ }}
    and
    \href{/docs/reference/layout/page/\#parameters-footer}{\texttt{\ footer\ }}
    parameters do not take functions anymore. If you want to customize
    them based on the page number, use the new
    \href{/docs/reference/layout/page/\#parameters-numbering}{\texttt{\ numbering\ }}
    parameter or
    \href{/docs/reference/introspection/counter/}{\texttt{\ counter\ }}
    function instead.
  \item
    Added
    \href{/docs/reference/layout/page/\#parameters-footer-descent}{\texttt{\ footer-descent\ }}
    and
    \href{/docs/reference/layout/page/\#parameters-header-ascent}{\texttt{\ header-ascent\ }}
    parameters
  \item
    Better default alignment in header and footer
  \item
    Fixed Arabic vowel placement
  \item
    Fixed PDF font embedding issues
  \item
    Renamed \texttt{\ math.formula\ } to
    \href{/docs/reference/math/equation/}{\texttt{\ math.equation\ }}
  \item
    Font family must be a named argument now:
    \texttt{\ }{\texttt{\ \#\ }}\texttt{\ }{\texttt{\ set\ }}\texttt{\ }{\texttt{\ text\ }}\texttt{\ }{\texttt{\ (\ }}\texttt{\ font\ }{\texttt{\ :\ }}\texttt{\ }{\texttt{\ ".."\ }}\texttt{\ }{\texttt{\ )\ }}\texttt{\ }
  \item
    Added support for
    \href{/docs/reference/model/par/\#parameters-hanging-indent}{hanging
    indent}
  \item
    Renamed paragraph \texttt{\ indent\ } to
    \href{/docs/reference/model/par/\#parameters-first-line-indent}{\texttt{\ first-line-indent\ }}
  \item
    More accurate
    \href{/docs/reference/foundations/calc/\#functions-log}{logarithm}
    when base is \texttt{\ 2\ } or \texttt{\ 10\ }
  \item
    Improved some error messages
  \item
    Fixed layout of
    \href{/docs/reference/model/terms/}{\texttt{\ terms\ }} list
  \end{itemize}
\item
  Web app improvements

  \begin{itemize}
  \tightlist
  \item
    Added template gallery
  \item
    Added buttons to insert headings, equations, raw blocks, and
    references
  \item
    Jump to the source of something by clicking on it in the preview
    panel (works for text, equations, images, and more)
  \item
    You can now upload your own fonts and use them in your project
  \item
    Hover debugging and autocompletion now takes multiple files into
    account and works in show rules
  \item
    Hover tooltips now automatically collapse multiple consecutive equal
    values
  \item
    The preview now automatically scrolls to the right place when you
    type
  \item
    Links are now clickable in the preview area
  \item
    Toolbar, preview, and editor can now all be hidden
  \item
    Added autocompletion for raw block language tags
  \item
    Added autocompletion in SVG files
  \item
    New back button instead of four-dots button
  \item
    Lots of bug fixes
  \end{itemize}
\end{itemize}

\subsection{February 25, 2023}\label{february-25-2023}

\begin{itemize}
\tightlist
\item
  Font changes

  \begin{itemize}
  \tightlist
  \item
    New default font: Linux Libertine
  \item
    New default font for raw blocks: DejaVu Sans Mono
  \item
    New default font for math: Book weight of New Computer Modern Math
  \item
    Lots of new math fonts available
  \item
    Removed Latin Modern fonts in favor of New Computer Modern family
  \item
    Removed unnecessary smallcaps fonts which are already accessible
    through the corresponding main font and the
    \href{/docs/reference/text/smallcaps/}{\texttt{\ smallcaps\ }}
    function
  \end{itemize}
\item
  Improved default spacing for headings
\item
  Added \href{/docs/reference/foundations/panic/}{\texttt{\ panic\ }}
  function
\item
  Added
  \href{/docs/reference/foundations/str/\#definitions-clusters}{\texttt{\ clusters\ }}
  and
  \href{/docs/reference/foundations/str/\#definitions-codepoints}{\texttt{\ codepoints\ }}
  methods for strings
\item
  Support for multiple authors in
  \href{/docs/reference/model/document/\#parameters-author}{\texttt{\ set\ document\ }}
\item
  Fixed crash when string is accessed at a position that is not a char
  boundary
\item
  Fixed semicolon parsing in
  \texttt{\ }{\texttt{\ \#\ }}\texttt{\ }{\texttt{\ var\ }}\texttt{\ ;\ }
\item
  Fixed incremental parsing when inserting backslash at end of
  \texttt{\ }{\texttt{\ \#\ }}\texttt{\ }{\texttt{\ "abc"\ }}\texttt{\ }
\item
  Fixed names of a few font families (including Noto Sans Symbols and
  New Computer Modern families)
\item
  Fixed autocompletion for font families
\item
  Improved incremental compilation for user-defined functions
\end{itemize}

\subsection{February 15, 2023}\label{february-15-2023}

\begin{itemize}
\tightlist
\item
  \href{/docs/reference/layout/box/}{Box} and
  \href{/docs/reference/layout/block/}{block} have gained
  \texttt{\ fill\ } , \texttt{\ stroke\ } , \texttt{\ radius\ } , and
  \texttt{\ inset\ } properties
\item
  Blocks may now be explicitly sized, fixed-height blocks can still
  break across pages
\item
  Blocks can now be configured to be
  \href{/docs/reference/layout/block/\#parameters-breakable}{\texttt{\ breakable\ }}
  or not
\item
  \href{/docs/reference/model/enum/\#parameters-numbering}{Numbering
  style} can now be configured for nested enums
\item
  \href{/docs/reference/model/list/\#parameters-marker}{Markers} can now
  be configured for nested lists
\item
  The \href{/docs/reference/foundations/eval/}{\texttt{\ eval\ }}
  function now expects code instead of markup and returns an arbitrary
  value. Markup can still be evaluated by surrounding the string with
  brackets.
\item
  PDFs generated by Typst now contain XMP metadata
\item
  Link boxes are now disabled in PDF output
\item
  Tables don\textquotesingle t produce small empty cells before a
  pagebreak anymore
\item
  Fixed raw block highlighting bug
\end{itemize}

\subsection{February 12, 2023}\label{february-12-2023}

\begin{itemize}
\tightlist
\item
  Shapes, images, and transformations (move/rotate/scale/repeat) are now
  block-level. To integrate them into a paragraph, use a
  \href{/docs/reference/layout/box/}{\texttt{\ box\ }} as with other
  elements.
\item
  A colon is now required in an "everything" show rule: Write
  \texttt{\ }{\texttt{\ show\ }}\texttt{\ }{\texttt{\ :\ }}\texttt{\ it\ }{\texttt{\ =\textgreater{}\ }}\texttt{\ ..\ }
  instead of
  \texttt{\ }{\texttt{\ show\ }}\texttt{\ it\ }{\texttt{\ =\textgreater{}\ }}\texttt{\ ..\ }
  . This prevents intermediate states that ruin your whole document.
\item
  Non-math content like a shape or table in a math formula is now
  centered vertically
\item
  Support for widow and orphan prevention within containers
\item
  Support for \href{/docs/reference/text/text/\#parameters-dir}{RTL} in
  lists, grids, and tables
\item
  Support for explicit \texttt{\ }{\texttt{\ auto\ }}\texttt{\ } sizing
  for boxes and shapes
\item
  Support for fractional (i.e. \texttt{\ }{\texttt{\ 1fr\ }}\texttt{\ }
  ) widths for boxes
\item
  Fixed bug where columns jump to next page
\item
  Fixed bug where list items have no leading
\item
  Fixed relative sizing in lists, squares and grid auto columns
\item
  Fixed relative displacement in
  \href{/docs/reference/layout/place/}{\texttt{\ place\ }} function
\item
  Fixed that lines don\textquotesingle t have a size
\item
  Fixed bug where
  \texttt{\ }{\texttt{\ set\ }}\texttt{\ }{\texttt{\ document\ }}\texttt{\ }{\texttt{\ (\ }}\texttt{\ }{\texttt{\ ..\ }}\texttt{\ }{\texttt{\ )\ }}\texttt{\ }
  complains about being after content
\item
  Fixed parsing of
  \texttt{\ }{\texttt{\ not\ }}\texttt{\ }{\texttt{\ in\ }}\texttt{\ }
  operation
\item
  Fixed hover tooltips in math
\item
  Fixed bug where a heading show rule may not contain a pagebreak when
  an outline is present
\item
  Added
  \href{/docs/reference/layout/box/\#parameters-baseline}{\texttt{\ baseline\ }}
  property on \href{/docs/reference/layout/box/}{\texttt{\ box\ }}
\item
  Added \href{/docs/reference/math/op/}{\texttt{\ tg\ }} and
  \href{/docs/reference/math/op/}{\texttt{\ ctg\ }} operators in math
\item
  Added delimiter setting for
  \href{/docs/reference/math/cases/}{\texttt{\ cases\ }} function
\item
  Parentheses are now included when accepting a function autocompletion
\end{itemize}

\subsection{February 2, 2023}\label{february-2-2023}

\begin{itemize}
\tightlist
\item
  Merged text and math symbols, renamed a few symbols (including
  \texttt{\ infty\ } to \texttt{\ infinity\ } with the alias
  \texttt{\ oo\ } )
\item
  Fixed missing italic mappings
\item
  Math italics correction is now applied properly
\item
  Parentheses now scale in
  \texttt{\ }{\texttt{\ \$\ }}\texttt{\ }{\texttt{\ zeta\ }}\texttt{\ }{\texttt{\ (\ }}\texttt{\ x\ }{\texttt{\ /\ }}\texttt{\ 2\ }{\texttt{\ )\ }}\texttt{\ }{\texttt{\ \$\ }}\texttt{\ }
\item
  Fixed placement of large root index
\item
  Fixed spacing in
  \texttt{\ }{\texttt{\ \$\ }}\texttt{\ }{\texttt{\ abs\ }}\texttt{\ }{\texttt{\ (\ }}\texttt{\ }{\texttt{\ -\ }}\texttt{\ x\ }{\texttt{\ )\ }}\texttt{\ }{\texttt{\ \$\ }}\texttt{\ }
\item
  Fixed inconsistency between text and identifiers in math
\item
  Accents are now ignored when positioning superscripts
\item
  Fixed vertical alignment in matrices
\item
  Fixed \texttt{\ text\ } set rule in \texttt{\ raw\ } show rule
\item
  Heading and list markers now parse consistently
\item
  Allow arbitrary math directly in content
\end{itemize}

\subsection{January 30, 2023}\label{january-30-2023}

\href{https://typst.app/blog/2023/january-update}{Go to the announcement
blog post.}

\begin{itemize}
\tightlist
\item
  New expression syntax in markup/math

  \begin{itemize}
  \tightlist
  \item
    Blocks cannot be directly embedded in markup anymore
  \item
    Like other expressions, they now require a leading hash
  \item
    More expressions available with hash, including literals (
    \texttt{\ }{\texttt{\ \#\ }}\texttt{\ }{\texttt{\ "string"\ }}\texttt{\ }
    ) as well as field access and method call without space:
    \texttt{\ }{\texttt{\ \#\ }}\texttt{\ }{\texttt{\ emoji\ }}\texttt{\ }{\texttt{\ .\ }}\texttt{\ }{\texttt{\ face\ }}\texttt{\ }
  \end{itemize}
\item
  New import syntax

  \begin{itemize}
  \tightlist
  \item
    \texttt{\ }{\texttt{\ \#\ }}\texttt{\ }{\texttt{\ import\ }}\texttt{\ }{\texttt{\ "module.typ"\ }}\texttt{\ }
    creates binding named \texttt{\ module\ }
  \item
    \texttt{\ }{\texttt{\ \#\ }}\texttt{\ }{\texttt{\ import\ }}\texttt{\ }{\texttt{\ "module.typ"\ }}\texttt{\ }{\texttt{\ :\ }}\texttt{\ a\ }{\texttt{\ ,\ }}\texttt{\ b\ }
    or
    \texttt{\ }{\texttt{\ \#\ }}\texttt{\ }{\texttt{\ import\ }}\texttt{\ }{\texttt{\ "module.typ"\ }}\texttt{\ }{\texttt{\ :\ }}\texttt{\ }{\texttt{\ *\ }}\texttt{\ }
    to import items
  \item
    \texttt{\ }{\texttt{\ \#\ }}\texttt{\ }{\texttt{\ import\ }}\texttt{\ emoji\ }{\texttt{\ :\ }}\texttt{\ face\ }{\texttt{\ ,\ }}\texttt{\ turtle\ }
    to import from already bound module
  \end{itemize}
\item
  New symbol handling

  \begin{itemize}
  \tightlist
  \item
    Removed symbol notation
  \item
    Symbols are now in modules: \texttt{\ sym\ } , \texttt{\ emoji\ } ,
    and \texttt{\ math\ }
  \item
    Math module also reexports all of \texttt{\ sym\ }
  \item
    Modified through field access, still order-independent
  \item
    Unknown modifiers are not allowed anymore
  \item
    Support for custom symbol definitions with \texttt{\ symbol\ }
    function
  \item
    Symbols now listed in documentation
  \end{itemize}
\item
  New \texttt{\ math\ } module

  \begin{itemize}
  \tightlist
  \item
    Contains all math-related functions
  \item
    Variables and function calls directly in math (without hash) access
    this module instead of the global scope, but can also access local
    variables
  \item
    Can be explicitly used in code, e.g.
    \texttt{\ }{\texttt{\ \#\ }}\texttt{\ }{\texttt{\ set\ }}\texttt{\ math\ }{\texttt{\ .\ }}\texttt{\ }{\texttt{\ vec\ }}\texttt{\ }{\texttt{\ (\ }}\texttt{\ delim\ }{\texttt{\ :\ }}\texttt{\ }{\texttt{\ "{[}"\ }}\texttt{\ }{\texttt{\ )\ }}\texttt{\ }
  \end{itemize}
\item
  Delimiter matching in math

  \begin{itemize}
  \tightlist
  \item
    Any opening delimiters matches any closing one
  \item
    When matched, they automatically scale
  \item
    To prevent scaling, escape them
  \item
    To forcibly match two delimiters, use \texttt{\ lr\ } function
  \item
    Line breaks may occur between matched delimiters
  \item
    Delimiters may also be unbalanced
  \item
    You can also use the \texttt{\ lr\ } function to scale the brackets
    (or just one bracket) to a specific size manually
  \end{itemize}
\item
  Multi-line math with alignment

  \begin{itemize}
  \tightlist
  \item
    The \texttt{\ \textbackslash{}\ } character inserts a line break
  \item
    The \texttt{\ \&\ } character defines an alignment point
  \item
    Alignment points also work for underbraces, vectors, cases, and
    matrices
  \item
    Multiple alignment points are supported
  \end{itemize}
\item
  More capable math function calls

  \begin{itemize}
  \tightlist
  \item
    Function calls directly in math can now take code expressions with
    hash
  \item
    They can now also take named arguments
  \item
    Within math function calls, semicolons turn preceding arguments to
    arrays to support matrices:
    \texttt{\ }{\texttt{\ \$\ }}\texttt{\ }{\texttt{\ mat\ }}\texttt{\ }{\texttt{\ (\ }}\texttt{\ 1\ }{\texttt{\ ,\ }}\texttt{\ 2\ }{\texttt{\ ;\ }}\texttt{\ 3\ }{\texttt{\ ,\ }}\texttt{\ 4\ }{\texttt{\ )\ }}\texttt{\ }{\texttt{\ \$\ }}\texttt{\ }
  \end{itemize}
\item
  Arbitrary content in math

  \begin{itemize}
  \tightlist
  \item
    Text, images, and other arbitrary content can now be embedded in
    math
  \item
    Math now also supports font fallback to support e.g. CJK and emoji
  \end{itemize}
\item
  More math features

  \begin{itemize}
  \tightlist
  \item
    New text operators: \texttt{\ op\ } function, \texttt{\ lim\ } ,
    \texttt{\ max\ } , etc.
  \item
    New matrix function: \texttt{\ mat\ }
  \item
    New n-ary roots with \texttt{\ root\ } function:
    \texttt{\ }{\texttt{\ \$\ }}\texttt{\ }{\texttt{\ root\ }}\texttt{\ }{\texttt{\ (\ }}\texttt{\ 3\ }{\texttt{\ ,\ }}\texttt{\ x\ }{\texttt{\ )\ }}\texttt{\ }{\texttt{\ \$\ }}\texttt{\ }
  \item
    New under- and overbraces, -brackets, and -lines
  \item
    New \texttt{\ abs\ } and \texttt{\ norm\ } functions
  \item
    New shorthands: \texttt{\ {[}\textbar{}\ } ,
    \texttt{\ \textbar{}{]}\ } , and \texttt{\ \textbar{}\textbar{}\ }
  \item
    New \texttt{\ attach\ } function, overridable attachments with
    \texttt{\ script\ } and \texttt{\ limit\ }
  \item
    Manual spacing in math, with \texttt{\ h\ } , \texttt{\ thin\ } ,
    \texttt{\ med\ } , \texttt{\ thick\ } and \texttt{\ quad\ }
  \item
    Symbols and other content may now be used like a function, e.g.
    \texttt{\ }{\texttt{\ \$\ }}\texttt{\ }{\texttt{\ zeta\ }}\texttt{\ }{\texttt{\ (\ }}\texttt{\ x\ }{\texttt{\ )\ }}\texttt{\ }{\texttt{\ \$\ }}\texttt{\ }
  \item
    Added Fira Math font, removed Noto Sans Math font
  \item
    Support for alternative math fonts through
    \texttt{\ }{\texttt{\ \#\ }}\texttt{\ }{\texttt{\ show\ }}\texttt{\ math\ }{\texttt{\ .\ }}\texttt{\ }{\texttt{\ formula\ }}\texttt{\ }{\texttt{\ :\ }}\texttt{\ }{\texttt{\ set\ }}\texttt{\ }{\texttt{\ text\ }}\texttt{\ }{\texttt{\ (\ }}\texttt{\ }{\texttt{\ "Fira\ Math"\ }}\texttt{\ }{\texttt{\ )\ }}\texttt{\ }
  \end{itemize}
\item
  More library improvements

  \begin{itemize}
  \tightlist
  \item
    New \texttt{\ calc\ } module, \texttt{\ abs\ } , \texttt{\ min\ } ,
    \texttt{\ max\ } , \texttt{\ even\ } , \texttt{\ odd\ } and
    \texttt{\ mod\ } moved there
  \item
    New \texttt{\ message\ } argument on \texttt{\ assert\ } function
  \item
    The \texttt{\ pairs\ } method on dictionaries now returns an array
    of length-2 arrays instead of taking a closure
  \item
    The method call
    \texttt{\ dict\ }{\texttt{\ .\ }}\texttt{\ }{\texttt{\ at\ }}\texttt{\ }{\texttt{\ (\ }}\texttt{\ }{\texttt{\ "key"\ }}\texttt{\ }{\texttt{\ )\ }}\texttt{\ }
    now always fails if \texttt{\ "key"\ } doesn\textquotesingle t exist
    Previously, it was allowed in assignments. Alternatives are
    \texttt{\ dict\ }{\texttt{\ .\ }}\texttt{\ key\ }{\texttt{\ =\ }}\texttt{\ x\ }
    and
    \texttt{\ dict\ }{\texttt{\ .\ }}\texttt{\ }{\texttt{\ insert\ }}\texttt{\ }{\texttt{\ (\ }}\texttt{\ }{\texttt{\ "key"\ }}\texttt{\ }{\texttt{\ ,\ }}\texttt{\ x\ }{\texttt{\ )\ }}\texttt{\ }
    .
  \end{itemize}
\item
  Smarter editor functionality

  \begin{itemize}
  \tightlist
  \item
    Autocompletion for local variables
  \item
    Autocompletion for methods available on a value
  \item
    Autocompletion for symbols and modules
  \item
    Autocompletion for imports
  \item
    Hover over an identifier to see its value(s)
  \end{itemize}
\item
  Further editor improvements

  \begin{itemize}
  \tightlist
  \item
    New Font menu with previews
  \item
    Single projects may now be shared with share links
  \item
    New dashboard experience if projects are shared with you
  \item
    Keyboard Shortcuts are now listed in the menus and there are more of
    them
  \item
    New Offline indicator
  \item
    Tooltips for all buttons
  \item
    Improved account protection
  \item
    Moved Status indicator into the error list button
  \end{itemize}
\item
  Further fixes

  \begin{itemize}
  \tightlist
  \item
    Multiple bug fixes for incremental parser
  \item
    Fixed closure parameter capturing
  \item
    Fixed tons of math bugs
  \item
    Bugfixes for performance, file management, editing reliability
  \item
    Added redirection to the page originally navigated to after signin
  \end{itemize}
\end{itemize}

\href{/docs/changelog/0.1.0/}{\pandocbounded{\includesvg[keepaspectratio]{/assets/icons/16-arrow-right.svg}}}

{ 0.1.0 } { Previous page }

\href{/docs/roadmap/}{\pandocbounded{\includesvg[keepaspectratio]{/assets/icons/16-arrow-right.svg}}}

{ Roadmap } { Next page }


\section{Docs LaTeX/typst.app/docs/changelog/0.9.0.tex}
\title{typst.app/docs/changelog/0.9.0}

\begin{itemize}
\tightlist
\item
  \href{/docs}{\includesvg[width=0.16667in,height=0.16667in]{/assets/icons/16-docs-dark.svg}}
\item
  \includesvg[width=0.16667in,height=0.16667in]{/assets/icons/16-arrow-right.svg}
\item
  \href{/docs/changelog/}{Changelog}
\item
  \includesvg[width=0.16667in,height=0.16667in]{/assets/icons/16-arrow-right.svg}
\item
  \href{/docs/changelog/0.9.0/}{0.9.0}
\end{itemize}

\section{Version 0.9.0 (October 31,
2023)}\label{version-0.9.0-october-31-2023}

\subsection{Bibliography management}\label{bibliography-management}

\begin{itemize}
\tightlist
\item
  New bibliography engine based on
  \href{https://citationstyles.org/}{CSL} (Citation Style Language).
  Ships with about 100 commonly used citation styles and can load custom
  \texttt{\ .csl\ } files.
\item
  Added new
  \href{/docs/reference/model/cite/\#parameters-form}{\texttt{\ form\ }}
  argument to the \texttt{\ cite\ } function to produce different forms
  of citations (e.g. for producing a citation suitable for inclusion in
  prose)
\item
  The \href{/docs/reference/model/cite/}{\texttt{\ cite\ }} function now
  takes only a single label/key instead of allowing multiple. Adjacent
  citations are merged and formatted according to the citation
  style\textquotesingle s rules automatically. This works both with the
  reference syntax and explicit calls to the \texttt{\ cite\ } function.
  \textbf{(Breaking change)}
\item
  The \texttt{\ cite\ } function now takes a
  \href{/docs/reference/foundations/label/}{label} instead of a string
  \textbf{(Breaking change)}
\item
  Added
  \href{/docs/reference/model/bibliography/\#parameters-full}{\texttt{\ full\ }}
  argument to bibliography function to print the full bibliography even
  if not all works were cited
\item
  Bibliography entries can now contain Typst equations (wrapped in
  \texttt{\ }{\texttt{\ \$\ }}\texttt{\ ..\ }{\texttt{\ \$\ }}\texttt{\ }
  just like in markup), this works both for \texttt{\ .yml\ } and
  \texttt{\ .bib\ } bibliographies
\item
  The hayagriva YAML format was improved. See its
  \href{https://github.com/typst/hayagriva/blob/main/CHANGELOG.md}{changelog}
  for more details. \textbf{(Breaking change)}
\item
  A few bugs with \texttt{\ .bib\ } file parsing were fixed
\item
  Removed \texttt{\ brackets\ } argument of \texttt{\ cite\ } function
  in favor of \texttt{\ form\ }
\end{itemize}

\subsection{Visualization}\label{visualization}

\begin{itemize}
\tightlist
\item
  Gradients and colors (thanks to
  \href{https://github.com/Dherse}{@Dherse} )

  \begin{itemize}
  \tightlist
  \item
    Added support for
    \href{/docs/reference/visualize/gradient/}{gradients} on shapes and
    text
  \item
    Supports linear, radial, and conic gradients
  \item
    Added support for defining colors in more color spaces, including
    \href{/docs/reference/visualize/color/\#definitions-oklab}{Oklab} ,
    \href{/docs/reference/visualize/color/\#definitions-linear-rgb}{Linear
    RGB(A)} ,
    \href{/docs/reference/visualize/color/\#definitions-hsl}{HSL} , and
    \href{/docs/reference/visualize/color/\#definitions-hsv}{HSV}
  \item
    Added
    \href{/docs/reference/visualize/color/\#definitions-saturate}{\texttt{\ saturate\ }}
    ,
    \href{/docs/reference/visualize/color/\#definitions-desaturate}{\texttt{\ desaturate\ }}
    , and
    \href{/docs/reference/visualize/color/\#definitions-rotate}{\texttt{\ rotate\ }}
    functions on colors
  \item
    Added
    \href{/docs/reference/visualize/color/\#predefined-color-maps}{\texttt{\ color.map\ }}
    module with predefined color maps that can be used with gradients
  \item
    Rename \texttt{\ kind\ } function on colors to
    \href{/docs/reference/visualize/color/\#definitions-space}{\texttt{\ space\ }}
  \item
    Removed \texttt{\ to-rgba\ } , \texttt{\ to-cmyk\ } , and
    \texttt{\ to-luma\ } functions in favor of a new
    \href{/docs/reference/visualize/color/\#definitions-components}{\texttt{\ components\ }}
    function
  \end{itemize}
\item
  Improved rendering of
  \href{/docs/reference/visualize/rect/}{rectangles} with corner radius
  and varying stroke widths
\item
  Added support for properly clipping
  \href{/docs/reference/layout/box/\#parameters-clip}{boxes} and
  \href{/docs/reference/layout/block/\#parameters-clip}{blocks} with a
  border radius
\item
  Added \texttt{\ background\ } parameter to
  \href{/docs/reference/text/overline/}{\texttt{\ overline\ }} ,
  \href{/docs/reference/text/underline/}{\texttt{\ underline\ }} , and
  \href{/docs/reference/text/strike/}{\texttt{\ strike\ }} functions
\item
  Fixed inaccurate color embedding in PDFs
\item
  Fixed ICC profile handling for images embedded in PDFs
\end{itemize}

\subsection{Text and Layout}\label{text-and-layout}

\begin{itemize}
\tightlist
\item
  Added support for automatically adding proper
  \href{/docs/reference/text/text/\#parameters-cjk-latin-spacing}{spacing}
  between CJK and Latin text (enabled by default)
\item
  Added support for automatic adjustment of more CJK punctuation
\item
  Added \href{/docs/reference/model/quote/}{\texttt{\ quote\ }} element
  for inserting inline and block quotes with optional attributions
\item
  Added
  \href{/docs/reference/text/raw/\#definitions-line}{\texttt{\ raw.line\ }}
  element for customizing the display of individual lines of raw text,
  e.g. to add line numbers while keeping proper syntax highlighting
\item
  Added support for per-side
  \href{/docs/reference/model/table/\#parameters-inset}{inset}
  customization to table function
\item
  Added Hungarian and Romanian translations
\item
  Added support for Czech hyphenation
\item
  Added support for setting custom
  \href{/docs/reference/text/smartquote/}{smart quotes}
\item
  The default
  \href{/docs/reference/model/figure/\#definitions-caption-separator}{figure
  separator} now reacts to the currently set language and region
\item
  Improved line breaking of links / URLs (especially helpful for
  bibliographies with many URLs)
\item
  Improved handling of consecutive hyphens in justification algorithm
\item
  Fixed interaction of justification and hanging indent
\item
  Fixed a bug with line breaking of short lines without spaces when
  justification is enabled
\item
  Fixed font fallback for hyphen generated by hyphenation
\item
  Fixed handling of word joiner and other no-break characters during
  hyphenation
\item
  Fixed crash when hyphenating after an empty line
\item
  Fixed line breaking of composite emoji like ���🌈
\item
  Fixed missing text in some SVGs
\item
  Fixed font fallback in SVGs
\item
  Fixed behavior of
  \href{/docs/reference/layout/pagebreak/\#parameters-to}{\texttt{\ to\ }}
  argument on \texttt{\ pagebreak\ } function
\item
  Fixed
  \texttt{\ }{\texttt{\ set\ }}\texttt{\ }{\texttt{\ align\ }}\texttt{\ }{\texttt{\ (\ }}\texttt{\ }{\texttt{\ ..\ }}\texttt{\ }{\texttt{\ )\ }}\texttt{\ }
  for equations
\item
  Fixed spacing around \href{/docs/reference/layout/place/}{placed}
  elements
\item
  Fixed coalescing of
  \href{/docs/reference/layout/block/\#parameters-above}{\texttt{\ above\ }}
  and
  \href{/docs/reference/layout/block/\#parameters-below}{\texttt{\ below\ }}
  spacing if given in em units and the font sizes differ
\item
  Fixed handling of \texttt{\ extent\ } parameter of
  \href{/docs/reference/text/underline/}{\texttt{\ underline\ }} ,
  \href{/docs/reference/text/overline/}{\texttt{\ overline\ }} , and
  \href{/docs/reference/text/strike/}{\texttt{\ strike\ }} functions
\item
  Fixed crash for
  \href{/docs/reference/layout/place/\#parameters-float}{floating placed
  elements} with no specified vertical alignment
\item
  Partially fixed a bug with citations in footnotes
\end{itemize}

\subsection{Math}\label{math}

\begin{itemize}
\tightlist
\item
  Added \texttt{\ gap\ } argument for
  \href{/docs/reference/math/vec/\#parameters-gap}{\texttt{\ vec\ }} ,
  \href{/docs/reference/math/mat/\#parameters-gap}{\texttt{\ mat\ }} ,
  and
  \href{/docs/reference/math/cases/\#parameters-gap}{\texttt{\ cases\ }}
  function
\item
  Added \texttt{\ size\ } argument for
  \href{/docs/reference/math/lr/\#functions-abs}{\texttt{\ abs\ }} ,
  \href{/docs/reference/math/lr/\#functions-norm}{\texttt{\ norm\ }} ,
  \href{/docs/reference/math/lr/\#functions-floor}{\texttt{\ floor\ }} ,
  \href{/docs/reference/math/lr/\#functions-ceil}{\texttt{\ ceil\ }} ,
  and
  \href{/docs/reference/math/lr/\#functions-round}{\texttt{\ round\ }}
  functions
\item
  Added
  \href{/docs/reference/math/cases/\#parameters-reverse}{\texttt{\ reverse\ }}
  parameter to cases function
\item
  Added support for multinomial coefficients to
  \href{/docs/reference/math/binom/}{\texttt{\ binom\ }} function
\item
  Removed \texttt{\ rotation\ } argument on
  \href{/docs/reference/math/cancel/}{\texttt{\ cancel\ }} function in
  favor of a new and more flexible \texttt{\ angle\ } argument
  \textbf{(Breaking change)}
\item
  Added \texttt{\ wide\ } constant, which inserts twice the spacing of
  \texttt{\ quad\ }
\item
  Added \texttt{\ csch\ } and \texttt{\ sech\ }
  \href{/docs/reference/math/op/}{operators}
\item
  \texttt{\ ↼\ } , \texttt{\ ⇀\ } , \texttt{\ ↔\ } , and
  \texttt{\ ⟷\ } can now be used as
  \href{/docs/reference/math/accent/}{accents}
\item
  Added \texttt{\ integral.dash\ } , \texttt{\ integral.dash.double\ } ,
  and \texttt{\ integral.slash\ }
  \href{/docs/reference/symbols/sym/}{symbols}
\item
  Added support for specifying negative indices for
  \href{/docs/reference/math/mat/\#parameters-augment}{augmentation}
  lines to position the line from the back
\item
  Fixed default color of matrix
  \href{/docs/reference/math/mat/\#parameters-augment}{augmentation}
  lines
\item
  Fixed attachment of primes to inline expressions
\item
  Math content now respects the text
  \href{/docs/reference/text/text/\#parameters-baseline}{baseline}
  setting
\end{itemize}

\subsection{Performance}\label{performance}

\begin{itemize}
\tightlist
\item
  Fixed a bug related to show rules in templates which would effectively
  disable incremental compilation in affected documents
\item
  Micro-optimized code in several hot paths, which brings substantial
  performance gains, in particular in incremental compilations
\item
  Improved incremental parsing, which affects the whole incremental
  compilation pipeline
\item
  Added support for incremental parsing in the CLI
\item
  Added support for incremental SVG encoding during PDF export, which
  greatly improves export performance for documents with many SVG
\end{itemize}

\subsection{Tooling and Diagnostics}\label{tooling-and-diagnostics}

\begin{itemize}
\tightlist
\item
  Improved autocompletion for variables that are in-scope
\item
  Added autocompletion for package imports
\item
  Added autocompletion for
  \href{/docs/reference/foundations/label/}{labels}
\item
  Added tooltip that shows which variables a function captures (when
  hovering over the equals sign or arrow of the function)
\item
  Diagnostics are now deduplicated
\item
  Improved diagnostics when trying to apply unary \texttt{\ +\ } or
  \texttt{\ -\ } to types that only support binary \texttt{\ +\ } and
  \texttt{\ -\ }
\item
  Error messages now state which label or citation key
  isn\textquotesingle t present in the document or its bibliography
\item
  Fixed a bug where function argument parsing errors were shadowed by
  function execution errors (e.g. when trying to call
  \href{/docs/reference/foundations/array/\#definitions-sorted}{\texttt{\ array.sorted\ }}
  and passing the key function as a positional argument instead of a
  named one).
\end{itemize}

\subsection{Export}\label{export}

\begin{itemize}
\tightlist
\item
  Added support for configuring the document\textquotesingle s creation
  \href{/docs/reference/model/document/\#parameters-date}{\texttt{\ date\ }}
  . If the \texttt{\ date\ } is set to
  \texttt{\ }{\texttt{\ auto\ }}\texttt{\ } (the default), the
  PDF\textquotesingle s creation date will be set to the current date
  and time.
\item
  Added support for configuring document
  \href{/docs/reference/model/document/\#parameters-keywords}{\texttt{\ keywords\ }}
\item
  Generated PDFs now contain PDF document IDs
\item
  The PDF creator tool metadata now includes the Typst version
\end{itemize}

\subsection{Web app}\label{web-app}

\begin{itemize}
\tightlist
\item
  Added version picker to pin a project to an older compiler version
  (with support for Typst 0.6.0+)
\item
  Fixed desyncs between editor and compiler and improved overall
  stability
\item
  The app now continues to highlight the document when typing while the
  document is being compiled
\end{itemize}

\subsection{Command line interface}\label{command-line-interface}

\begin{itemize}
\tightlist
\item
  Added support for discovering fonts through fontconfig
\item
  Now clears the screen instead of resetting the terminal
\item
  Now automatically picks correct file extension for selected output
  format
\item
  Now only regenerates images for changed pages when using
  \texttt{\ typst\ watch\ } with PNG or SVG export
\end{itemize}

\subsection{Miscellaneous
Improvements}\label{miscellaneous-improvements}

\begin{itemize}
\tightlist
\item
  Added
  \href{/docs/reference/foundations/version/}{\texttt{\ version\ }} type
  and \texttt{\ sys.version\ } constant specifying the current compiler
  version. Can be used to gracefully support multiple versions.
\item
  The U+2212 MINUS SIGN is now used when displaying a numeric value, in
  the \href{/docs/reference/foundations/repr/}{\texttt{\ repr\ }} of any
  numeric value and to replace a normal hyphen in text mode when before
  a digit. This improves, in particular, how negative integer values are
  displayed in math mode.
\item
  Added support for specifying a default value instead of failing for
  \texttt{\ remove\ } function in
  \href{/docs/reference/foundations/array/\#definitions-remove}{array}
  and
  \href{/docs/reference/foundations/dictionary/\#definitions-remove}{dictionary}
\item
  Simplified page setup guide examples
\item
  Switched the documentation from using the word "hashtag" to the word
  "hash" where appropriate
\item
  Added support for
  \href{/docs/reference/foundations/array/\#definitions-zip}{\texttt{\ array.zip\ }}
  without any further arguments
\item
  Fixed crash when a plugin tried to read out of bounds memory
\item
  Fixed crashes when handling infinite
  \href{/docs/reference/layout/length/}{lengths}
\item
  Fixed introspection (mostly bibliography) bugs due to weak page break
  close to the end of the document
\end{itemize}

\subsection{Development}\label{development}

\begin{itemize}
\tightlist
\item
  Extracted \texttt{\ typst::ide\ } into separate
  \texttt{\ typst\_ide\ } crate
\item
  Removed a few remaining \texttt{\ \textquotesingle{}static\ } bounds
  on \texttt{\ \&dyn\ World\ }
\item
  Removed unnecessary dependency, which reduces the binary size
\item
  Fixed compilation of \texttt{\ typst\ } by itself (without
  \texttt{\ typst-library\ } )
\item
  Fixed warnings with Nix flake when using \texttt{\ lib.getExe\ }
\end{itemize}

\subsection{Contributors}\label{contributors}

Thanks to everyone who contributed to this release!

\begin{itemize}
\tightlist
\item
  \href{https://github.com/Dherse}{\includegraphics[width=0.66667in,height=0.66667in]{https://avatars.githubusercontent.com/u/9665250?s=64&v=4}}
\item
  \href{https://github.com/EpicEricEE}{\includegraphics[width=0.66667in,height=0.66667in]{https://avatars.githubusercontent.com/u/7191192?s=64&v=4}}
\item
  \href{https://github.com/tingerrr}{\includegraphics[width=0.66667in,height=0.66667in]{https://avatars.githubusercontent.com/u/137803093?s=64&v=4}}
\item
  \href{https://github.com/LuxxxLucy}{\includegraphics[width=0.66667in,height=0.66667in]{https://avatars.githubusercontent.com/u/19356905?s=64&v=4}}
\item
  \href{https://github.com/bluebear94}{\includegraphics[width=0.66667in,height=0.66667in]{https://avatars.githubusercontent.com/u/2975203?s=64&v=4}}
\item
  \href{https://github.com/MDLC01}{\includegraphics[width=0.66667in,height=0.66667in]{https://avatars.githubusercontent.com/u/57839069?s=64&v=4}}
\item
  \href{https://github.com/Jollywatt}{\includegraphics[width=0.66667in,height=0.66667in]{https://avatars.githubusercontent.com/u/24970860?s=64&v=4}}
\item
  \href{https://github.com/SekoiaTree}{\includegraphics[width=0.66667in,height=0.66667in]{https://avatars.githubusercontent.com/u/51149447?s=64&v=4}}
\item
  \href{https://github.com/DVDTSB}{\includegraphics[width=0.66667in,height=0.66667in]{https://avatars.githubusercontent.com/u/66365801?s=64&v=4}}
\item
  \href{https://github.com/HydroH}{\includegraphics[width=0.66667in,height=0.66667in]{https://avatars.githubusercontent.com/u/14823453?s=64&v=4}}
\item
  \href{https://github.com/KillTheMule}{\includegraphics[width=0.66667in,height=0.66667in]{https://avatars.githubusercontent.com/u/4117685?s=64&v=4}}
\item
  \href{https://github.com/LaurenzV}{\includegraphics[width=0.66667in,height=0.66667in]{https://avatars.githubusercontent.com/u/47084093?s=64&v=4}}
\item
  \href{https://github.com/frozolotl}{\includegraphics[width=0.66667in,height=0.66667in]{https://avatars.githubusercontent.com/u/44589151?s=64&v=4}}
\item
  \href{https://github.com/peng1999}{\includegraphics[width=0.66667in,height=0.66667in]{https://avatars.githubusercontent.com/u/12483662?s=64&v=4}}
\item
  \href{https://github.com/0scvr}{\includegraphics[width=0.66667in,height=0.66667in]{https://avatars.githubusercontent.com/u/71343264?s=64&v=4}}
\item
  \href{https://github.com/7sDream}{\includegraphics[width=0.66667in,height=0.66667in]{https://avatars.githubusercontent.com/u/7822577?s=64&v=4}}
\item
  \href{https://github.com/8LWXpg}{\includegraphics[width=0.66667in,height=0.66667in]{https://avatars.githubusercontent.com/u/105704427?s=64&v=4}}
\item
  \href{https://github.com/Enter-tainer}{\includegraphics[width=0.66667in,height=0.66667in]{https://avatars.githubusercontent.com/u/25521218?s=64&v=4}}
\item
  \href{https://github.com/FlyinPancake}{\includegraphics[width=0.66667in,height=0.66667in]{https://avatars.githubusercontent.com/u/36113055?s=64&v=4}}
\item
  \href{https://github.com/Myriad-Dreamin}{\includegraphics[width=0.66667in,height=0.66667in]{https://avatars.githubusercontent.com/u/35292584?s=64&v=4}}
\item
  \href{https://github.com/T0mstone}{\includegraphics[width=0.66667in,height=0.66667in]{https://avatars.githubusercontent.com/u/39707032?s=64&v=4}}
\item
  \href{https://github.com/TheJosefOlsson}{\includegraphics[width=0.66667in,height=0.66667in]{https://avatars.githubusercontent.com/u/143743179?s=64&v=4}}
\item
  \href{https://github.com/WannesMalfait}{\includegraphics[width=0.66667in,height=0.66667in]{https://avatars.githubusercontent.com/u/46323945?s=64&v=4}}
\item
  \href{https://github.com/WeetHet}{\includegraphics[width=0.66667in,height=0.66667in]{https://avatars.githubusercontent.com/u/43210583?s=64&v=4}}
\item
  \href{https://github.com/Weissnix4711}{\includegraphics[width=0.66667in,height=0.66667in]{https://avatars.githubusercontent.com/u/42943391?s=64&v=4}}
\item
  \href{https://github.com/Zheoni}{\includegraphics[width=0.66667in,height=0.66667in]{https://avatars.githubusercontent.com/u/38019254?s=64&v=4}}
\item
  \href{https://github.com/antonWetzel}{\includegraphics[width=0.66667in,height=0.66667in]{https://avatars.githubusercontent.com/u/59712243?s=64&v=4}}
\item
  \href{https://github.com/arnaudgolfouse}{\includegraphics[width=0.66667in,height=0.66667in]{https://avatars.githubusercontent.com/u/53786772?s=64&v=4}}
\item
  \href{https://github.com/drupol}{\includegraphics[width=0.66667in,height=0.66667in]{https://avatars.githubusercontent.com/u/252042?s=64&v=4}}
\item
  \href{https://github.com/dxniel-19}{\includegraphics[width=0.66667in,height=0.66667in]{https://avatars.githubusercontent.com/u/33750461?s=64&v=4}}
\item
  \href{https://github.com/extua}{\includegraphics[width=0.66667in,height=0.66667in]{https://avatars.githubusercontent.com/u/66693681?s=64&v=4}}
\item
  \href{https://github.com/fritzrehde}{\includegraphics[width=0.66667in,height=0.66667in]{https://avatars.githubusercontent.com/u/80471265?s=64&v=4}}
\item
  \href{https://github.com/johannes-wolf}{\includegraphics[width=0.66667in,height=0.66667in]{https://avatars.githubusercontent.com/u/519002?s=64&v=4}}
\item
  \href{https://github.com/qrnch-jan}{\includegraphics[width=0.66667in,height=0.66667in]{https://avatars.githubusercontent.com/u/60492138?s=64&v=4}}
\item
  \href{https://github.com/toddlerer}{\includegraphics[width=0.66667in,height=0.66667in]{https://avatars.githubusercontent.com/u/74579078?s=64&v=4}}
\end{itemize}

\href{/docs/changelog/0.10.0/}{\pandocbounded{\includesvg[keepaspectratio]{/assets/icons/16-arrow-right.svg}}}

{ 0.10.0 } { Previous page }

\href{/docs/changelog/0.8.0/}{\pandocbounded{\includesvg[keepaspectratio]{/assets/icons/16-arrow-right.svg}}}

{ 0.8.0 } { Next page }


\section{Docs LaTeX/typst.app/docs/changelog/0.7.0.tex}
\title{typst.app/docs/changelog/0.7.0}

\begin{itemize}
\tightlist
\item
  \href{/docs}{\includesvg[width=0.16667in,height=0.16667in]{/assets/icons/16-docs-dark.svg}}
\item
  \includesvg[width=0.16667in,height=0.16667in]{/assets/icons/16-arrow-right.svg}
\item
  \href{/docs/changelog/}{Changelog}
\item
  \includesvg[width=0.16667in,height=0.16667in]{/assets/icons/16-arrow-right.svg}
\item
  \href{/docs/changelog/0.7.0/}{0.7.0}
\end{itemize}

\section{Version 0.7.0 (August 7,
2023)}\label{version-0.7.0-august-7-2023}

\subsection{Text and Layout}\label{text-and-layout}

\begin{itemize}
\tightlist
\item
  Added support for floating figures through the
  \href{/docs/reference/model/figure/\#parameters-placement}{\texttt{\ placement\ }}
  argument on the figure function
\item
  Added support for arbitrary floating content through the
  \href{/docs/reference/layout/place/\#parameters-float}{\texttt{\ float\ }}
  argument on the place function
\item
  Added support for loading \texttt{\ .sublime-syntax\ } files as
  highlighting
  \href{/docs/reference/text/raw/\#parameters-syntaxes}{syntaxes} for
  raw blocks
\item
  Added support for loading \texttt{\ .tmTheme\ } files as highlighting
  \href{/docs/reference/text/raw/\#parameters-theme}{themes} for raw
  blocks
\item
  Added \emph{bounds} option to
  \href{/docs/reference/text/text/\#parameters-top-edge}{\texttt{\ top-edge\ }}
  and
  \href{/docs/reference/text/text/\#parameters-bottom-edge}{\texttt{\ bottom-edge\ }}
  arguments of text function for tight bounding boxes
\item
  Removed nonsensical top- and bottom-edge options, e.g. \emph{ascender}
  for the bottom edge \textbf{(Breaking change)}
\item
  Added
  \href{/docs/reference/text/text/\#parameters-script}{\texttt{\ script\ }}
  argument to text function
\item
  Added
  \href{/docs/reference/text/smartquote/\#parameters-alternative}{\texttt{\ alternative\ }}
  argument to smart quote function
\item
  Added basic i18n for Japanese
\item
  Added hyphenation support for \texttt{\ nb\ } and \texttt{\ nn\ }
  language codes in addition to \texttt{\ no\ }
\item
  Fixed positioning of \href{/docs/reference/layout/place/}{placed
  elements} in containers
\item
  Fixed overflowing containers due to optimized line breaks
\end{itemize}

\subsection{Export}\label{export}

\begin{itemize}
\tightlist
\item
  Greatly improved export of SVG images to PDF. Many thanks to
  \href{https://github.com/LaurenzV}{@LaurenzV} for their work on this.
\item
  Added support for the alpha channel of RGBA colors in PDF export
\item
  Fixed a bug with PPI (pixels per inch) for PNG export
\end{itemize}

\subsection{Math}\label{math}

\begin{itemize}
\tightlist
\item
  Improved layout of primes (e.g. in
  \texttt{\ }{\texttt{\ \$\ }}\texttt{\ a\ }{\texttt{\ \textquotesingle{}\ }}\texttt{\ }{\texttt{\ \_\ }}\texttt{\ 1\ }{\texttt{\ \$\ }}\texttt{\ }
  )
\item
  Improved display of multi-primes (e.g. in
  \texttt{\ }{\texttt{\ \$\ }}\texttt{\ a\ }{\texttt{\ \textquotesingle{}\ }}\texttt{\ }{\texttt{\ \textquotesingle{}\ }}\texttt{\ }{\texttt{\ \$\ }}\texttt{\ }
  )
\item
  Improved layout of
  \href{/docs/reference/math/roots/\#functions-root}{roots}
\item
  Changed relations to show attachments as
  \href{/docs/reference/math/attach/\#functions-limits}{limits} by
  default (e.g. in
  \texttt{\ }{\texttt{\ \$\ }}\texttt{\ a\ }{\texttt{\ -\textgreater{}\ }}\texttt{\ }{\texttt{\ \^{}\ }}\texttt{\ x\ b\ }{\texttt{\ \$\ }}\texttt{\ }
  )
\item
  Large operators and delimiters are now always vertically centered
\item
  \href{/docs/reference/layout/box/}{Boxes} in equations now sit on the
  baseline instead of being vertically centered by default. Notably,
  this does not affect \href{/docs/reference/layout/block/}{blocks}
  because they are not inline elements.
\item
  Added support for
  \href{/docs/reference/layout/h/\#parameters-weak}{weak spacing}
\item
  Added support for OpenType character variants
\item
  Added support for customizing the
  \href{/docs/reference/math/class/}{math class} of content
\item
  Fixed spacing around \texttt{\ .\ } , \texttt{\ \textbackslash{}/\ } ,
  and \texttt{\ ...\ }
\item
  Fixed spacing between closing delimiters and large operators
\item
  Fixed a bug with math font weight selection
\item
  Symbols and Operators \textbf{(Breaking changes)}

  \begin{itemize}
  \tightlist
  \item
    Added \texttt{\ id\ } , \texttt{\ im\ } , and \texttt{\ tr\ } text
    \href{/docs/reference/math/op/}{operators}
  \item
    Renamed \texttt{\ ident\ } to \texttt{\ equiv\ } with alias
    \texttt{\ eq.triple\ } and removed \texttt{\ ident.strict\ } in
    favor of \texttt{\ eq.quad\ }
  \item
    Renamed \texttt{\ ast.sq\ } to \texttt{\ ast.square\ } and
    \texttt{\ integral.sq\ } to \texttt{\ integral.square\ }
  \item
    Renamed \texttt{\ .eqq\ } modifier to \texttt{\ .equiv\ } (and
    \texttt{\ .neqq\ } to \texttt{\ .nequiv\ } ) for \texttt{\ tilde\ }
    , \texttt{\ gt\ } , \texttt{\ lt\ } , \texttt{\ prec\ } , and
    \texttt{\ succ\ }
  \item
    Added \texttt{\ emptyset\ } as alias for \texttt{\ nothing\ }
  \item
    Added \texttt{\ lt.curly\ } and \texttt{\ gt.curly\ } as aliases for
    \texttt{\ prec\ } and \texttt{\ succ\ }
  \item
    Added \texttt{\ aleph\ } , \texttt{\ beth\ } , and
    \texttt{\ gimmel\ } as alias for \texttt{\ alef\ } ,
    \texttt{\ bet\ } , and \texttt{\ gimel\ }
  \end{itemize}
\end{itemize}

\subsection{Scripting}\label{scripting}

\begin{itemize}
\tightlist
\item
  Fields

  \begin{itemize}
  \tightlist
  \item
    Added \texttt{\ abs\ } and \texttt{\ em\ } field to
    \href{/docs/reference/layout/length/}{lengths}
  \item
    Added \texttt{\ ratio\ } and \texttt{\ length\ } field to
    \href{/docs/reference/layout/relative/}{relative lengths}
  \item
    Added \texttt{\ x\ } and \texttt{\ y\ } field to
    \href{/docs/reference/layout/align/\#parameters-alignment}{2d
    alignments}
  \item
    Added \texttt{\ paint\ } , \texttt{\ thickness\ } , \texttt{\ cap\ }
    , \texttt{\ join\ } , \texttt{\ dash\ } , and
    \texttt{\ miter-limit\ } field to
    \href{/docs/reference/visualize/stroke/}{strokes}
  \end{itemize}
\item
  Accessor and utility methods

  \begin{itemize}
  \tightlist
  \item
    Added
    \href{/docs/reference/foundations/array/\#definitions-dedup}{\texttt{\ dedup\ }}
    method to arrays
  \item
    Added \texttt{\ pt\ } , \texttt{\ mm\ } , \texttt{\ cm\ } , and
    \texttt{\ inches\ } method to
    \href{/docs/reference/layout/length/}{lengths}
  \item
    Added \texttt{\ deg\ } and \texttt{\ rad\ } method to
    \href{/docs/reference/layout/angle/}{angles}
  \item
    Added \texttt{\ kind\ } , \texttt{\ hex\ } , \texttt{\ rgba\ } ,
    \texttt{\ cmyk\ } , and \texttt{\ luma\ } method to
    \href{/docs/reference/visualize/color/}{colors}
  \item
    Added \texttt{\ axis\ } , \texttt{\ start\ } , \texttt{\ end\ } ,
    and \texttt{\ inv\ } method to
    \href{/docs/reference/layout/stack/\#parameters-dir}{directions}
  \item
    Added \texttt{\ axis\ } and \texttt{\ inv\ } method to
    \href{/docs/reference/layout/align/\#parameters-alignment}{alignments}
  \item
    Added \texttt{\ inv\ } method to
    \href{/docs/reference/layout/align/\#parameters-alignment}{2d
    alignments}
  \item
    Added \texttt{\ start\ } argument to
    \href{/docs/reference/foundations/array/\#definitions-enumerate}{\texttt{\ enumerate\ }}
    method on arrays
  \end{itemize}
\item
  Added
  \href{/docs/reference/visualize/color/\#definitions-mix}{\texttt{\ color.mix\ }}
  function
\item
  Added \texttt{\ mode\ } and \texttt{\ scope\ } arguments to
  \href{/docs/reference/foundations/eval/}{\texttt{\ eval\ }} function
\item
  Added \href{/docs/reference/foundations/bytes/}{\texttt{\ bytes\ }}
  type for holding large byte buffers

  \begin{itemize}
  \tightlist
  \item
    Added
    \href{/docs/reference/data-loading/read/\#parameters-encoding}{\texttt{\ encoding\ }}
    argument to read function to read a file as bytes instead of a
    string
  \item
    Added
    \href{/docs/reference/visualize/image/\#definitions-decode}{\texttt{\ image.decode\ }}
    function for decoding an image directly from a string or bytes
  \item
    Added \href{/docs/reference/foundations/bytes/}{\texttt{\ bytes\ }}
    function for converting a string or an array of integers to bytes
  \item
    Added \href{/docs/reference/foundations/array/}{\texttt{\ array\ }}
    function for converting bytes to an array of integers
  \item
    Added support for converting bytes to a string with the
    \href{/docs/reference/foundations/str/}{\texttt{\ str\ }} function
  \end{itemize}
\end{itemize}

\subsection{Tooling and Diagnostics}\label{tooling-and-diagnostics}

\begin{itemize}
\tightlist
\item
  Added support for compiler warnings
\item
  Added warning when compilation does not converge within five attempts
  due to intense use of introspection features
\item
  Added warnings for empty emphasis ( \texttt{\ \_\_\ } and
  \texttt{\ **\ } )
\item
  Improved error message for invalid field assignments
\item
  Improved error message after single \texttt{\ \#\ }
\item
  Improved error message when a keyword is used where an identifier is
  expected
\item
  Fixed parameter autocompletion for functions that are in modules
\item
  Import autocompletion now only shows the latest package version until
  a colon is typed
\item
  Fixed autocompletion for dictionary key containing a space
\item
  Fixed autocompletion for \texttt{\ for\ } loops
\end{itemize}

\subsection{Command line interface}\label{command-line-interface}

\begin{itemize}
\tightlist
\item
  Added \texttt{\ typst\ query\ } subcommand to execute a
  \href{/docs/reference/introspection/query/\#command-line-queries}{query}
  on the command line
\item
  The \texttt{\ -\/-root\ } and \texttt{\ -\/-font-paths\ } arguments
  cannot appear in front of the command anymore \textbf{(Breaking
  change)}
\item
  Local and cached packages are now stored in directories of the form
  \texttt{\ \{namespace\}/\{name\}/\{version\}\ } instead of
  \texttt{\ \{namespace\}/\{name\}-\{version\}\ } \textbf{(Breaking
  change)}
\item
  Now prioritizes explicitly given fonts (via
  \texttt{\ -\/-font-paths\ } ) over system and embedded fonts when both
  exist
\item
  Fixed \texttt{\ typst\ watch\ } not working with some text editors
\item
  Fixed displayed compilation time (now includes export)
\end{itemize}

\subsection{Miscellaneous
Improvements}\label{miscellaneous-improvements}

\begin{itemize}
\tightlist
\item
  Added
  \href{/docs/reference/model/heading/\#parameters-bookmarked}{\texttt{\ bookmarked\ }}
  argument to heading to control whether a heading becomes part of the
  PDF outline
\item
  Added
  \href{/docs/reference/model/figure/\#definitions-caption-position}{\texttt{\ caption-pos\ }}
  argument to control the position of a figure\textquotesingle s caption
\item
  Added
  \href{/docs/reference/introspection/metadata/}{\texttt{\ metadata\ }}
  function for exposing an arbitrary value to the introspection system
\item
  Fixed that a
  \href{/docs/reference/introspection/state/}{\texttt{\ state\ }} was
  identified by the pair \texttt{\ (key,\ init)\ } instead of just its
  \texttt{\ key\ }
\item
  Improved indent logic of
  \href{/docs/reference/model/enum/}{enumerations} . Instead of
  requiring at least as much indent as the end of the marker, they now
  require only one more space indent than the start of the marker. As a
  result, even long markers like \texttt{\ 12.\ } work with just 2
  spaces of indent.
\item
  Fixed bug with indent logic of
  \href{/docs/reference/text/raw/}{\texttt{\ raw\ }} blocks
\item
  Fixed a parsing bug with dictionaries
\end{itemize}

\subsection{Development}\label{development}

\begin{itemize}
\tightlist
\item
  Extracted parser and syntax tree into \texttt{\ typst-syntax\ } crate
\item
  The \texttt{\ World::today\ } implementation of Typst dependents may
  need fixing if they have the same
  \href{https://github.com/typst/typst/issues/1842}{bug} that the CLI
  world had
\end{itemize}

\subsection{Contributors}\label{contributors}

Thanks to everyone who contributed to this release!

\begin{itemize}
\tightlist
\item
  \href{https://github.com/damaxwell}{\includegraphics[width=0.66667in,height=0.66667in]{https://avatars.githubusercontent.com/u/918465?s=64&v=4}}
\item
  \href{https://github.com/bluebear94}{\includegraphics[width=0.66667in,height=0.66667in]{https://avatars.githubusercontent.com/u/2975203?s=64&v=4}}
\item
  \href{https://github.com/PgBiel}{\includegraphics[width=0.66667in,height=0.66667in]{https://avatars.githubusercontent.com/u/9021226?s=64&v=4}}
\item
  \href{https://github.com/Beiri22}{\includegraphics[width=0.66667in,height=0.66667in]{https://avatars.githubusercontent.com/u/8210233?s=64&v=4}}
\item
  \href{https://github.com/Dherse}{\includegraphics[width=0.66667in,height=0.66667in]{https://avatars.githubusercontent.com/u/9665250?s=64&v=4}}
\item
  \href{https://github.com/LaurenzV}{\includegraphics[width=0.66667in,height=0.66667in]{https://avatars.githubusercontent.com/u/47084093?s=64&v=4}}
\item
  \href{https://github.com/Mafii}{\includegraphics[width=0.66667in,height=0.66667in]{https://avatars.githubusercontent.com/u/10061519?s=64&v=4}}
\item
  \href{https://github.com/adriandelgado}{\includegraphics[width=0.66667in,height=0.66667in]{https://avatars.githubusercontent.com/u/11708972?s=64&v=4}}
\item
  \href{https://github.com/lolstork}{\includegraphics[width=0.66667in,height=0.66667in]{https://avatars.githubusercontent.com/u/137357423?s=64&v=4}}
\item
  \href{https://github.com/AlistairKeiller}{\includegraphics[width=0.66667in,height=0.66667in]{https://avatars.githubusercontent.com/u/43255248?s=64&v=4}}
\item
  \href{https://github.com/DVDTSB}{\includegraphics[width=0.66667in,height=0.66667in]{https://avatars.githubusercontent.com/u/66365801?s=64&v=4}}
\item
  \href{https://github.com/Enter-tainer}{\includegraphics[width=0.66667in,height=0.66667in]{https://avatars.githubusercontent.com/u/25521218?s=64&v=4}}
\item
  \href{https://github.com/EpicEricEE}{\includegraphics[width=0.66667in,height=0.66667in]{https://avatars.githubusercontent.com/u/7191192?s=64&v=4}}
\item
  \href{https://github.com/Liamolucko}{\includegraphics[width=0.66667in,height=0.66667in]{https://avatars.githubusercontent.com/u/43807659?s=64&v=4}}
\item
  \href{https://github.com/LingMan}{\includegraphics[width=0.66667in,height=0.66667in]{https://avatars.githubusercontent.com/u/18645382?s=64&v=4}}
\item
  \href{https://github.com/MDLC01}{\includegraphics[width=0.66667in,height=0.66667in]{https://avatars.githubusercontent.com/u/57839069?s=64&v=4}}
\item
  \href{https://github.com/Myriad-Dreamin}{\includegraphics[width=0.66667in,height=0.66667in]{https://avatars.githubusercontent.com/u/35292584?s=64&v=4}}
\item
  \href{https://github.com/StrangeGirlMurph}{\includegraphics[width=0.66667in,height=0.66667in]{https://avatars.githubusercontent.com/u/62220780?s=64&v=4}}
\item
  \href{https://github.com/T0mstone}{\includegraphics[width=0.66667in,height=0.66667in]{https://avatars.githubusercontent.com/u/39707032?s=64&v=4}}
\item
  \href{https://github.com/TheLukeGuy}{\includegraphics[width=0.66667in,height=0.66667in]{https://avatars.githubusercontent.com/u/64972126?s=64&v=4}}
\item
  \href{https://github.com/antonWetzel}{\includegraphics[width=0.66667in,height=0.66667in]{https://avatars.githubusercontent.com/u/59712243?s=64&v=4}}
\item
  \href{https://github.com/epbuennig}{\includegraphics[width=0.66667in,height=0.66667in]{https://avatars.githubusercontent.com/u/103939966?s=64&v=4}}
\item
  \href{https://github.com/kg583}{\includegraphics[width=0.66667in,height=0.66667in]{https://avatars.githubusercontent.com/u/41345727?s=64&v=4}}
\item
  \href{https://github.com/lynn}{\includegraphics[width=0.66667in,height=0.66667in]{https://avatars.githubusercontent.com/u/16232127?s=64&v=4}}
\item
  \href{https://github.com/mattfbacon}{\includegraphics[width=0.66667in,height=0.66667in]{https://avatars.githubusercontent.com/u/58113890?s=64&v=4}}
\item
  \href{https://github.com/mavaddat}{\includegraphics[width=0.66667in,height=0.66667in]{https://avatars.githubusercontent.com/u/5055400?s=64&v=4}}
\item
  \href{https://github.com/neunenak}{\includegraphics[width=0.66667in,height=0.66667in]{https://avatars.githubusercontent.com/u/311545?s=64&v=4}}
\item
  \href{https://github.com/pineapplehunter}{\includegraphics[width=0.66667in,height=0.66667in]{https://avatars.githubusercontent.com/u/8869894?s=64&v=4}}
\item
  \href{https://github.com/raphCode}{\includegraphics[width=0.66667in,height=0.66667in]{https://avatars.githubusercontent.com/u/15750438?s=64&v=4}}
\item
  \href{https://github.com/sitandr}{\includegraphics[width=0.66667in,height=0.66667in]{https://avatars.githubusercontent.com/u/60141933?s=64&v=4}}
\item
  \href{https://github.com/sudormrfbin}{\includegraphics[width=0.66667in,height=0.66667in]{https://avatars.githubusercontent.com/u/23398472?s=64&v=4}}
\item
  \href{https://github.com/tretre91}{\includegraphics[width=0.66667in,height=0.66667in]{https://avatars.githubusercontent.com/u/63788850?s=64&v=4}}
\end{itemize}

\href{/docs/changelog/0.8.0/}{\pandocbounded{\includesvg[keepaspectratio]{/assets/icons/16-arrow-right.svg}}}

{ 0.8.0 } { Previous page }

\href{/docs/changelog/0.6.0/}{\pandocbounded{\includesvg[keepaspectratio]{/assets/icons/16-arrow-right.svg}}}

{ 0.6.0 } { Next page }


\section{Docs LaTeX/typst.app/docs/changelog/0.5.0.tex}
\title{typst.app/docs/changelog/0.5.0}

\begin{itemize}
\tightlist
\item
  \href{/docs}{\includesvg[width=0.16667in,height=0.16667in]{/assets/icons/16-docs-dark.svg}}
\item
  \includesvg[width=0.16667in,height=0.16667in]{/assets/icons/16-arrow-right.svg}
\item
  \href{/docs/changelog/}{Changelog}
\item
  \includesvg[width=0.16667in,height=0.16667in]{/assets/icons/16-arrow-right.svg}
\item
  \href{/docs/changelog/0.5.0/}{0.5.0}
\end{itemize}

\section{Version 0.5.0 (June 9, 2023)}\label{version-0.5.0-june-9-2023}

\subsection{Text and Layout}\label{text-and-layout}

\begin{itemize}
\tightlist
\item
  Added \href{/docs/reference/text/raw/}{\texttt{\ raw\ }} syntax
  highlighting for many more languages
\item
  Added support for Korean
  \href{/docs/reference/model/numbering/}{numbering}
\item
  Added basic i18n for a few more languages (NL, SV, DA)
\item
  Improved line breaking for East Asian languages
\item
  Expanded functionality of outline
  \href{/docs/reference/model/outline/\#parameters-indent}{\texttt{\ indent\ }}
  property
\item
  Fixed footnotes in columns
\item
  Fixed page breaking bugs with
  \href{/docs/reference/model/footnote/}{footnotes}
\item
  Fixed bug with handling of footnotes in lists, tables, and figures
\item
  Fixed a bug with CJK punctuation adjustment
\item
  Fixed a crash with rounded rectangles
\item
  Fixed alignment of
  \href{/docs/reference/visualize/line/}{\texttt{\ line\ }} elements
\end{itemize}

\subsection{Math}\label{math}

\begin{itemize}
\tightlist
\item
  \textbf{Breaking change:} The syntax rules for mathematical
  \href{/docs/reference/math/attach/\#functions-attach}{attachments}
  were improved:
  \texttt{\ }{\texttt{\ \$\ }}\texttt{\ f\ }{\texttt{\ \^{}\ }}\texttt{\ }{\texttt{\ abs\ }}\texttt{\ }{\texttt{\ (\ }}\texttt{\ 3\ }{\texttt{\ )\ }}\texttt{\ }{\texttt{\ \$\ }}\texttt{\ }
  now parses as
  \texttt{\ }{\texttt{\ \$\ }}\texttt{\ f\ }{\texttt{\ \^{}\ }}\texttt{\ }{\texttt{\ (\ }}\texttt{\ }{\texttt{\ abs\ }}\texttt{\ }{\texttt{\ (\ }}\texttt{\ 3\ }{\texttt{\ )\ }}\texttt{\ }{\texttt{\ )\ }}\texttt{\ }{\texttt{\ \$\ }}\texttt{\ }
  instead of
  \texttt{\ }{\texttt{\ \$\ }}\texttt{\ (f\ }{\texttt{\ \^{}\ }}\texttt{\ }{\texttt{\ abs\ }}\texttt{\ )(3)\ }{\texttt{\ \$\ }}\texttt{\ }
  . To disambiguate, add a space:
  \texttt{\ }{\texttt{\ \$\ }}\texttt{\ f\ }{\texttt{\ \^{}\ }}\texttt{\ }{\texttt{\ zeta\ }}\texttt{\ (3)\ }{\texttt{\ \$\ }}\texttt{\ }
  .
\item
  Added \href{/docs/reference/math/sizes/}{forced size} commands for
  math (e.g.,
  \href{/docs/reference/math/sizes/\#functions-display}{\texttt{\ display\ }}
  )
\item
  Added
  \href{/docs/reference/math/equation/\#parameters-supplement}{\texttt{\ supplement\ }}
  parameter to
  \href{/docs/reference/math/equation/}{\texttt{\ equation\ }} , used by
  \href{/docs/reference/model/ref/}{references}
\item
  New \href{/docs/reference/symbols/sym/}{symbols} : \texttt{\ bullet\ }
  , \texttt{\ xor\ } , \texttt{\ slash.big\ } , \texttt{\ sigma.alt\ } ,
  \texttt{\ tack.r.not\ } , \texttt{\ tack.r.short\ } ,
  \texttt{\ tack.r.double.not\ }
\item
  Fixed a bug with symbols in matrices
\item
  Fixed a crash in the
  \href{/docs/reference/math/attach/\#functions-attach}{\texttt{\ attach\ }}
  function
\end{itemize}

\subsection{Scripting}\label{scripting}

\begin{itemize}
\tightlist
\item
  Added new
  \href{/docs/reference/foundations/datetime/}{\texttt{\ datetime\ }}
  type and
  \href{/docs/reference/foundations/datetime/\#definitions-today}{\texttt{\ datetime.today\ }}
  to retrieve the current date
\item
  Added
  \href{/docs/reference/foundations/str/\#definitions-from-unicode}{\texttt{\ str.from-unicode\ }}
  and
  \href{/docs/reference/foundations/str/\#definitions-to-unicode}{\texttt{\ str.to-unicode\ }}
  functions
\item
  Added
  \href{/docs/reference/foundations/content/\#definitions-fields}{\texttt{\ fields\ }}
  method on content
\item
  Added \texttt{\ base\ } parameter to
  \href{/docs/reference/foundations/str/}{\texttt{\ str\ }} function
\item
  Added
  \href{/docs/reference/foundations/calc/\#functions-exp}{\texttt{\ calc.exp\ }}
  and
  \href{/docs/reference/foundations/calc/\#functions-ln}{\texttt{\ calc.ln\ }}
\item
  Improved accuracy of
  \href{/docs/reference/foundations/calc/\#functions-pow}{\texttt{\ calc.pow\ }}
  and
  \href{/docs/reference/foundations/calc/\#functions-log}{\texttt{\ calc.log\ }}
  for specific bases
\item
  Fixed
  \href{/docs/reference/foundations/dictionary/\#definitions-remove}{removal}
  order for dictionary
\item
  Fixed \texttt{\ .at(default:\ ..)\ } for
  \href{/docs/reference/foundations/str/\#definitions-at}{strings} and
  \href{/docs/reference/foundations/content/\#definitions-at}{content}
\item
  Fixed field access on styled elements
\item
  Removed deprecated \texttt{\ calc.mod\ } function
\end{itemize}

\subsection{Command line interface}\label{command-line-interface}

\begin{itemize}
\tightlist
\item
  Added PNG export via
  \texttt{\ typst\ compile\ source.typ\ output-\{n\}.png\ } . The output
  path must contain \texttt{\ \{n\}\ } if the document has multiple
  pages.
\item
  Added \texttt{\ -\/-diagnostic-format=short\ } for Unix-style short
  diagnostics
\item
  Doesn\textquotesingle t emit color codes anymore if stderr
  isn\textquotesingle t a TTY
\item
  Now sets the correct exit when invoked with a nonexistent file
\item
  Now ignores UTF-8 BOM in Typst files
\end{itemize}

\subsection{Miscellaneous
Improvements}\label{miscellaneous-improvements}

\begin{itemize}
\tightlist
\item
  Improved errors for mismatched delimiters
\item
  Improved error message for failed length comparisons
\item
  Fixed a bug with images not showing up in Apple Preview
\item
  Fixed multiple bugs with the PDF outline
\item
  Fixed citations and other searchable elements in
  \href{/docs/reference/layout/hide/}{\texttt{\ hide\ }}
\item
  Fixed bugs with
  \href{/docs/reference/model/ref/\#parameters-supplement}{reference
  supplements}
\item
  Fixed Nix flake
\end{itemize}

\subsection{Contributors}\label{contributors}

Thanks to everyone who contributed to this release!

\begin{itemize}
\tightlist
\item
  \href{https://github.com/EpicEricEE}{\includegraphics[width=0.66667in,height=0.66667in]{https://avatars.githubusercontent.com/u/7191192?s=64&v=4}}
\item
  \href{https://github.com/LaurenzV}{\includegraphics[width=0.66667in,height=0.66667in]{https://avatars.githubusercontent.com/u/47084093?s=64&v=4}}
\item
  \href{https://github.com/PgBiel}{\includegraphics[width=0.66667in,height=0.66667in]{https://avatars.githubusercontent.com/u/9021226?s=64&v=4}}
\item
  \href{https://github.com/SUPERCILEX}{\includegraphics[width=0.66667in,height=0.66667in]{https://avatars.githubusercontent.com/u/9490724?s=64&v=4}}
\item
  \href{https://github.com/figsoda}{\includegraphics[width=0.66667in,height=0.66667in]{https://avatars.githubusercontent.com/u/40620903?s=64&v=4}}
\item
  \href{https://github.com/lynn}{\includegraphics[width=0.66667in,height=0.66667in]{https://avatars.githubusercontent.com/u/16232127?s=64&v=4}}
\item
  \href{https://github.com/peng1999}{\includegraphics[width=0.66667in,height=0.66667in]{https://avatars.githubusercontent.com/u/12483662?s=64&v=4}}
\item
  \href{https://github.com/sandalbanditten}{\includegraphics[width=0.66667in,height=0.66667in]{https://avatars.githubusercontent.com/u/45170745?s=64&v=4}}
\item
  \href{https://github.com/BasixKOR}{\includegraphics[width=0.66667in,height=0.66667in]{https://avatars.githubusercontent.com/u/7413880?s=64&v=4}}
\item
  \href{https://github.com/JettChenT}{\includegraphics[width=0.66667in,height=0.66667in]{https://avatars.githubusercontent.com/u/45888395?s=64&v=4}}
\item
  \href{https://github.com/Leedehai}{\includegraphics[width=0.66667in,height=0.66667in]{https://avatars.githubusercontent.com/u/18319900?s=64&v=4}}
\item
  \href{https://github.com/MDLC01}{\includegraphics[width=0.66667in,height=0.66667in]{https://avatars.githubusercontent.com/u/57839069?s=64&v=4}}
\item
  \href{https://github.com/StrangeGirlMurph}{\includegraphics[width=0.66667in,height=0.66667in]{https://avatars.githubusercontent.com/u/62220780?s=64&v=4}}
\item
  \href{https://github.com/alixmacdonald10}{\includegraphics[width=0.66667in,height=0.66667in]{https://avatars.githubusercontent.com/u/52908801?s=64&v=4}}
\item
  \href{https://github.com/coughmedicine}{\includegraphics[width=0.66667in,height=0.66667in]{https://avatars.githubusercontent.com/u/72562074?s=64&v=4}}
\item
  \href{https://github.com/epbuennig}{\includegraphics[width=0.66667in,height=0.66667in]{https://avatars.githubusercontent.com/u/103939966?s=64&v=4}}
\item
  \href{https://github.com/erikwastaken}{\includegraphics[width=0.66667in,height=0.66667in]{https://avatars.githubusercontent.com/u/46598911?s=64&v=4}}
\item
  \href{https://github.com/glocq}{\includegraphics[width=0.66667in,height=0.66667in]{https://avatars.githubusercontent.com/u/16959183?s=64&v=4}}
\item
  \href{https://github.com/howjmay}{\includegraphics[width=0.66667in,height=0.66667in]{https://avatars.githubusercontent.com/u/13924801?s=64&v=4}}
\item
  \href{https://github.com/jassler}{\includegraphics[width=0.66667in,height=0.66667in]{https://avatars.githubusercontent.com/u/24298790?s=64&v=4}}
\item
  \href{https://github.com/lino-levan}{\includegraphics[width=0.66667in,height=0.66667in]{https://avatars.githubusercontent.com/u/11367844?s=64&v=4}}
\item
  \href{https://github.com/lucifer1004}{\includegraphics[width=0.66667in,height=0.66667in]{https://avatars.githubusercontent.com/u/13583761?s=64&v=4}}
\item
  \href{https://github.com/matthiasGmayer}{\includegraphics[width=0.66667in,height=0.66667in]{https://avatars.githubusercontent.com/u/28257008?s=64&v=4}}
\item
  \href{https://github.com/naim94a}{\includegraphics[width=0.66667in,height=0.66667in]{https://avatars.githubusercontent.com/u/227396?s=64&v=4}}
\item
  \href{https://github.com/ollema}{\includegraphics[width=0.66667in,height=0.66667in]{https://avatars.githubusercontent.com/u/31876997?s=64&v=4}}
\item
  \href{https://github.com/sbatial}{\includegraphics[width=0.66667in,height=0.66667in]{https://avatars.githubusercontent.com/u/46973479?s=64&v=4}}
\item
  \href{https://github.com/sitandr}{\includegraphics[width=0.66667in,height=0.66667in]{https://avatars.githubusercontent.com/u/60141933?s=64&v=4}}
\item
  \href{https://github.com/xiongmao86}{\includegraphics[width=0.66667in,height=0.66667in]{https://avatars.githubusercontent.com/u/1770218?s=64&v=4}}
\end{itemize}

\href{/docs/changelog/0.6.0/}{\pandocbounded{\includesvg[keepaspectratio]{/assets/icons/16-arrow-right.svg}}}

{ 0.6.0 } { Previous page }

\href{/docs/changelog/0.4.0/}{\pandocbounded{\includesvg[keepaspectratio]{/assets/icons/16-arrow-right.svg}}}

{ 0.4.0 } { Next page }


\section{Docs LaTeX/typst.app/docs/changelog/0.1.0.tex}
\title{typst.app/docs/changelog/0.1.0}

\begin{itemize}
\tightlist
\item
  \href{/docs}{\includesvg[width=0.16667in,height=0.16667in]{/assets/icons/16-docs-dark.svg}}
\item
  \includesvg[width=0.16667in,height=0.16667in]{/assets/icons/16-arrow-right.svg}
\item
  \href{/docs/changelog/}{Changelog}
\item
  \includesvg[width=0.16667in,height=0.16667in]{/assets/icons/16-arrow-right.svg}
\item
  \href{/docs/changelog/0.1.0/}{0.1.0}
\end{itemize}

\section{Version 0.1.0 (April 04,
2023)}\label{version-0.1.0-april-04-2023}

\subsection{Breaking changes}\label{breaking-changes}

\begin{itemize}
\tightlist
\item
  When using the CLI, you now have to use subcommands:

  \begin{itemize}
  \tightlist
  \item
    \texttt{\ typst\ compile\ file.typ\ } or
    \texttt{\ typst\ c\ file.typ\ } to create a PDF
  \item
    \texttt{\ typst\ watch\ file.typ\ } or
    \texttt{\ typst\ w\ file.typ\ } to compile and watch
  \item
    \texttt{\ typst\ fonts\ } to list all fonts
  \end{itemize}
\item
  Manual counters now start at zero. Read the "How to step" section
  \href{/docs/reference/introspection/counter/}{here} for more details
\item
  The
  \href{/docs/reference/model/bibliography/\#parameters-style}{bibliography
  styles} \texttt{\ }{\texttt{\ "author-date"\ }}\texttt{\ } and
  \texttt{\ }{\texttt{\ "author-title"\ }}\texttt{\ } were renamed to
  \texttt{\ }{\texttt{\ "chicago-author-date"\ }}\texttt{\ } and
  \texttt{\ }{\texttt{\ "chicago-author-title"\ }}\texttt{\ }
\end{itemize}

\subsection{Figure improvements}\label{figure-improvements}

\begin{itemize}
\tightlist
\item
  Figures now automatically detect their content and adapt their
  behavior. Figures containing tables, for instance, are automatically
  prefixed with "Table X" and have a separate counter
\item
  The figure\textquotesingle s supplement (e.g. "Figure" or "Table") can
  now be customized
\item
  In addition, figures can now be completely customized because the show
  rule gives access to the automatically resolved kind, supplement, and
  counter
\end{itemize}

\subsection{Bibliography improvements}\label{bibliography-improvements}

\begin{itemize}
\tightlist
\item
  The
  \href{/docs/reference/model/bibliography/}{\texttt{\ bibliography\ }}
  now also accepts multiple bibliography paths (as an array)
\item
  Parsing of BibLaTeX files is now more permissive (accepts non-numeric
  edition, pages, volumes, dates, and Jabref-style comments; fixed
  abbreviation parsing)
\item
  Labels and references can now include \texttt{\ :\ } and
  \texttt{\ .\ } except at the end
\item
  Fixed APA bibliography ordering
\end{itemize}

\subsection{Drawing additions}\label{drawing-additions}

\begin{itemize}
\tightlist
\item
  Added \href{/docs/reference/visualize/polygon/}{\texttt{\ polygon\ }}
  function for drawing polygons
\item
  Added support for clipping in
  \href{/docs/reference/layout/box/\#parameters-clip}{boxes} and
  \href{/docs/reference/layout/block/\#parameters-clip}{blocks}
\end{itemize}

\subsection{Command line interface}\label{command-line-interface}

\begin{itemize}
\tightlist
\item
  Now returns with non-zero status code if there is an error
\item
  Now watches the root directory instead of the current one
\item
  Now puts the PDF file next to input file by default
\item
  Now accepts more kinds of input files (e.g. \texttt{\ /dev/stdin\ } )
\item
  Added \texttt{\ -\/-open\ } flag to directly open the PDF
\end{itemize}

\subsection{Miscellaneous
improvements}\label{miscellaneous-improvements}

\begin{itemize}
\tightlist
\item
  Added \href{/docs/reference/data-loading/yaml/}{\texttt{\ yaml\ }}
  function to load data from YAML files
\item
  Added basic i18n for a few more languages (IT, RU, ZH, FR, PT)
\item
  Added numbering support for Hebrew
\item
  Added support for \href{/docs/reference/foundations/int/}{integers}
  with base 2, 8, and 16
\item
  Added symbols for double bracket and laplace operator
\item
  The \href{/docs/reference/model/link/}{\texttt{\ link\ }} function now
  accepts \href{/docs/reference/foundations/label/}{labels}
\item
  The link syntax now allows more characters
\item
  Improved justification of Japanese and Chinese text
\item
  Calculation functions behave more consistently w.r.t to non-real
  results
\item
  Replaced deprecated angle brackets
\item
  Reduced maximum function call depth from 256 to 64
\item
  Fixed
  \href{/docs/reference/model/par/\#parameters-first-line-indent}{\texttt{\ first-line-indent\ }}
  being not applied when a paragraph starts with styled text
\item
  Fixed extraneous spacing in unary operators in equations
\item
  Fixed block spacing, e.g. in
  \texttt{\ }{\texttt{\ block\ }}\texttt{\ }{\texttt{\ (\ }}\texttt{\ above\ }{\texttt{\ :\ }}\texttt{\ }{\texttt{\ 1cm\ }}\texttt{\ }{\texttt{\ ,\ }}\texttt{\ below\ }{\texttt{\ :\ }}\texttt{\ }{\texttt{\ 1cm\ }}\texttt{\ }{\texttt{\ ,\ }}\texttt{\ }{\texttt{\ ..\ }}\texttt{\ }{\texttt{\ )\ }}\texttt{\ }
\item
  Fixed styling of text operators in math
\item
  Fixed invalid parsing of language tag in raw block with a single
  backtick
\item
  Fixed bugs with displaying counters and state
\item
  Fixed crash related to page counter
\item
  Fixed crash when
  \href{/docs/reference/symbols/symbol/}{\texttt{\ symbol\ }} function
  was called without arguments
\item
  Fixed crash in bibliography generation
\item
  Fixed access to label of certain content elements
\item
  Fixed line number in error message for CSV parsing
\item
  Fixed invalid autocompletion after certain markup elements
\end{itemize}

\subsection{Contributors}\label{contributors}

Thanks to everyone who contributed to this release!

\begin{itemize}
\tightlist
\item
  \href{https://github.com/Dherse}{\includegraphics[width=0.66667in,height=0.66667in]{https://avatars.githubusercontent.com/u/9665250?s=64&v=4}}
\item
  \href{https://github.com/barvirm}{\includegraphics[width=0.66667in,height=0.66667in]{https://avatars.githubusercontent.com/u/15948420?s=64&v=4}}
\item
  \href{https://github.com/HarmoGlace}{\includegraphics[width=0.66667in,height=0.66667in]{https://avatars.githubusercontent.com/u/23212967?s=64&v=4}}
\item
  \href{https://github.com/figsoda}{\includegraphics[width=0.66667in,height=0.66667in]{https://avatars.githubusercontent.com/u/40620903?s=64&v=4}}
\item
  \href{https://github.com/frozolotl}{\includegraphics[width=0.66667in,height=0.66667in]{https://avatars.githubusercontent.com/u/44589151?s=64&v=4}}
\item
  \href{https://github.com/johannes-wolf}{\includegraphics[width=0.66667in,height=0.66667in]{https://avatars.githubusercontent.com/u/519002?s=64&v=4}}
\item
  \href{https://github.com/naim94a}{\includegraphics[width=0.66667in,height=0.66667in]{https://avatars.githubusercontent.com/u/227396?s=64&v=4}}
\item
  \href{https://github.com/CGMossa}{\includegraphics[width=0.66667in,height=0.66667in]{https://avatars.githubusercontent.com/u/1063624?s=64&v=4}}
\item
  \href{https://github.com/FilipAndersson245}{\includegraphics[width=0.66667in,height=0.66667in]{https://avatars.githubusercontent.com/u/17986183?s=64&v=4}}
\item
  \href{https://github.com/GabrielDertoni}{\includegraphics[width=0.66667in,height=0.66667in]{https://avatars.githubusercontent.com/u/13651052?s=64&v=4}}
\item
  \href{https://github.com/P-Andersson}{\includegraphics[width=0.66667in,height=0.66667in]{https://avatars.githubusercontent.com/u/17784951?s=64&v=4}}
\item
  \href{https://github.com/SekoiaTree}{\includegraphics[width=0.66667in,height=0.66667in]{https://avatars.githubusercontent.com/u/51149447?s=64&v=4}}
\item
  \href{https://github.com/asayers}{\includegraphics[width=0.66667in,height=0.66667in]{https://avatars.githubusercontent.com/u/877109?s=64&v=4}}
\item
  \href{https://github.com/birktj}{\includegraphics[width=0.66667in,height=0.66667in]{https://avatars.githubusercontent.com/u/2932651?s=64&v=4}}
\item
  \href{https://github.com/frungl}{\includegraphics[width=0.66667in,height=0.66667in]{https://avatars.githubusercontent.com/u/82895777?s=64&v=4}}
\item
  \href{https://github.com/jinhao-huang}{\includegraphics[width=0.66667in,height=0.66667in]{https://avatars.githubusercontent.com/u/80573215?s=64&v=4}}
\item
  \href{https://github.com/loewenheim}{\includegraphics[width=0.66667in,height=0.66667in]{https://avatars.githubusercontent.com/u/7622248?s=64&v=4}}
\item
  \href{https://github.com/lvignoli}{\includegraphics[width=0.66667in,height=0.66667in]{https://avatars.githubusercontent.com/u/97944962?s=64&v=4}}
\item
  \href{https://github.com/lxndio}{\includegraphics[width=0.66667in,height=0.66667in]{https://avatars.githubusercontent.com/u/1679267?s=64&v=4}}
\item
  \href{https://github.com/marcoradocchia}{\includegraphics[width=0.66667in,height=0.66667in]{https://avatars.githubusercontent.com/u/74802223?s=64&v=4}}
\item
  \href{https://github.com/mateusfccp}{\includegraphics[width=0.66667in,height=0.66667in]{https://avatars.githubusercontent.com/u/4605213?s=64&v=4}}
\item
  \href{https://github.com/ollelogdahl}{\includegraphics[width=0.66667in,height=0.66667in]{https://avatars.githubusercontent.com/u/37961475?s=64&v=4}}
\item
  \href{https://github.com/samlich}{\includegraphics[width=0.66667in,height=0.66667in]{https://avatars.githubusercontent.com/u/1349989?s=64&v=4}}
\item
  \href{https://github.com/user202729}{\includegraphics[width=0.66667in,height=0.66667in]{https://avatars.githubusercontent.com/u/25191436?s=64&v=4}}
\item
  \href{https://github.com/utilForever}{\includegraphics[width=0.66667in,height=0.66667in]{https://avatars.githubusercontent.com/u/5622661?s=64&v=4}}
\item
  \href{https://github.com/wrenger}{\includegraphics[width=0.66667in,height=0.66667in]{https://avatars.githubusercontent.com/u/20145354?s=64&v=4}}
\item
  \href{https://github.com/yichenchong}{\includegraphics[width=0.66667in,height=0.66667in]{https://avatars.githubusercontent.com/u/40590448?s=64&v=4}}
\item
  \href{https://github.com/zrr1999}{\includegraphics[width=0.66667in,height=0.66667in]{https://avatars.githubusercontent.com/u/46243324?s=64&v=4}}
\end{itemize}

\href{/docs/changelog/0.2.0/}{\pandocbounded{\includesvg[keepaspectratio]{/assets/icons/16-arrow-right.svg}}}

{ 0.2.0 } { Previous page }

\href{/docs/changelog/earlier/}{\pandocbounded{\includesvg[keepaspectratio]{/assets/icons/16-arrow-right.svg}}}

{ Earlier } { Next page }


\section{Docs LaTeX/typst.app/docs/changelog/0.3.0.tex}
\title{typst.app/docs/changelog/0.3.0}

\begin{itemize}
\tightlist
\item
  \href{/docs}{\includesvg[width=0.16667in,height=0.16667in]{/assets/icons/16-docs-dark.svg}}
\item
  \includesvg[width=0.16667in,height=0.16667in]{/assets/icons/16-arrow-right.svg}
\item
  \href{/docs/changelog/}{Changelog}
\item
  \includesvg[width=0.16667in,height=0.16667in]{/assets/icons/16-arrow-right.svg}
\item
  \href{/docs/changelog/0.3.0/}{0.3.0}
\end{itemize}

\section{Version 0.3.0 (April 26,
2023)}\label{version-0.3.0-april-26-2023}

\subsection{Breaking changes}\label{breaking-changes}

\begin{itemize}
\tightlist
\item
  Renamed a few symbols: What was previous \texttt{\ dot.op\ } is now
  just \texttt{\ dot\ } and the basic dot is \texttt{\ dot.basic\ } .
  The same applies to \texttt{\ ast\ } and \texttt{\ tilde\ } .
\item
  Renamed \texttt{\ mod\ } to
  \href{/docs/reference/foundations/calc/\#functions-rem}{\texttt{\ rem\ }}
  to more accurately reflect the behavior. It will remain available as
  \texttt{\ mod\ } until the next update as a grace period.
\item
  A lone underscore is not a valid identifier anymore, it can now only
  be used in patterns
\item
  Removed \texttt{\ before\ } and \texttt{\ after\ } arguments from
  \href{/docs/reference/introspection/query/}{\texttt{\ query\ }} . This
  is now handled through flexible
  \href{/docs/reference/foundations/selector/}{selectors} combinator
  methods
\item
  Added support for
  \href{/docs/reference/math/attach/\#functions-attach}{attachments}
  (sub-, superscripts) that precede the base symbol. The
  \texttt{\ top\ } and \texttt{\ bottom\ } arguments have been renamed
  to \texttt{\ t\ } and \texttt{\ b\ } .
\end{itemize}

\subsection{New features}\label{new-features}

\begin{itemize}
\tightlist
\item
  Added support for more complex
  \href{/docs/reference/visualize/stroke/}{strokes} (configurable caps,
  joins, and dash patterns)
\item
  Added \href{/docs/reference/math/cancel/}{\texttt{\ cancel\ }}
  function for equations
\item
  Added support for
  \href{/docs/reference/scripting/\#bindings}{destructuring} in argument
  lists and assignments
\item
  Added
  \href{/docs/reference/visualize/image/\#parameters-alt}{\texttt{\ alt\ }}
  text argument to image function
\item
  Added \href{/docs/reference/data-loading/toml/}{\texttt{\ toml\ }}
  function for loading data from a TOML file
\item
  Added
  \href{/docs/reference/foundations/array/\#definitions-zip}{\texttt{\ zip\ }}
  ,
  \href{/docs/reference/foundations/array/\#definitions-sum}{\texttt{\ sum\ }}
  , and
  \href{/docs/reference/foundations/array/\#definitions-product}{\texttt{\ product\ }}
  methods for arrays
\item
  Added \texttt{\ fact\ } , \texttt{\ perm\ } , \texttt{\ binom\ } ,
  \texttt{\ gcd\ } , \texttt{\ lcm\ } , \texttt{\ atan2\ } ,
  \texttt{\ quo\ } , \texttt{\ trunc\ } , and \texttt{\ fract\ }
  \href{/docs/reference/foundations/calc/}{calculation} functions
\end{itemize}

\subsection{Improvements}\label{improvements}

\begin{itemize}
\tightlist
\item
  Text in SVGs now displays properly
\item
  Typst now generates a PDF heading outline
\item
  \href{/docs/reference/model/ref/}{References} now provides the
  referenced element as a field in show rules
\item
  Refined linebreak algorithm for better Chinese justification
\item
  Locations are now a valid kind of selector
\item
  Added a few symbols for algebra
\item
  Added Spanish smart quote support
\item
  Added
  \href{/docs/reference/foundations/selector/}{\texttt{\ selector\ }}
  function to turn a selector-like value into a selector on which
  combinator methods can be called
\item
  Improved some error messages
\item
  The outline and bibliography headings can now be styled with show-set
  rules
\item
  Operations on numbers now produce an error instead of overflowing
\end{itemize}

\subsection{Bug fixes}\label{bug-fixes}

\begin{itemize}
\tightlist
\item
  Fixed wrong linebreak before punctuation that follows inline
  equations, citations, and other elements
\item
  Fixed a bug with
  \href{/docs/reference/foundations/arguments/}{argument sinks}
\item
  Fixed strokes with thickness zero
\item
  Fixed hiding and show rules in math
\item
  Fixed alignment in matrices
\item
  Fixed some alignment bugs in equations
\item
  Fixed grid cell alignment
\item
  Fixed alignment of list marker and enum markers in presence of global
  alignment settings
\item
  Fixed \href{/docs/reference/visualize/path/}{path} closing
\item
  Fixed compiler crash with figure references
\item
  A single trailing line breaks is now ignored in math, just like in
  text
\end{itemize}

\subsection{Command line interface}\label{command-line-interface}

\begin{itemize}
\tightlist
\item
  Font path and compilation root can now be set with the environment
  variables \texttt{\ TYPST\_FONT\_PATHS\ } and \texttt{\ TYPST\_ROOT\ }
\item
  The output of \texttt{\ typst\ fonts\ } now includes the embedded
  fonts
\end{itemize}

\subsection{Development}\label{development}

\begin{itemize}
\tightlist
\item
  Added instrumentation for debugging and optimization
\item
  Added \texttt{\ -\/-update\ } flag and \texttt{\ UPDATE\_EXPECT\ }
  environment variable to update reference images for tests
\item
  You can now run a specific subtest with \texttt{\ -\/-subtest\ }
\item
  Tests now run on multiple threads
\end{itemize}

\subsection{Contributors}\label{contributors}

Thanks to everyone who contributed to this release!

\begin{itemize}
\tightlist
\item
  \href{https://github.com/SUPERCILEX}{\includegraphics[width=0.66667in,height=0.66667in]{https://avatars.githubusercontent.com/u/9490724?s=64&v=4}}
\item
  \href{https://github.com/HarmoGlace}{\includegraphics[width=0.66667in,height=0.66667in]{https://avatars.githubusercontent.com/u/23212967?s=64&v=4}}
\item
  \href{https://github.com/Marmare314}{\includegraphics[width=0.66667in,height=0.66667in]{https://avatars.githubusercontent.com/u/49279081?s=64&v=4}}
\item
  \href{https://github.com/PgBiel}{\includegraphics[width=0.66667in,height=0.66667in]{https://avatars.githubusercontent.com/u/9021226?s=64&v=4}}
\item
  \href{https://github.com/barvirm}{\includegraphics[width=0.66667in,height=0.66667in]{https://avatars.githubusercontent.com/u/15948420?s=64&v=4}}
\item
  \href{https://github.com/Dherse}{\includegraphics[width=0.66667in,height=0.66667in]{https://avatars.githubusercontent.com/u/9665250?s=64&v=4}}
\item
  \href{https://github.com/Leedehai}{\includegraphics[width=0.66667in,height=0.66667in]{https://avatars.githubusercontent.com/u/18319900?s=64&v=4}}
\item
  \href{https://github.com/joeldierkes}{\includegraphics[width=0.66667in,height=0.66667in]{https://avatars.githubusercontent.com/u/8125678?s=64&v=4}}
\item
  \href{https://github.com/peng1999}{\includegraphics[width=0.66667in,height=0.66667in]{https://avatars.githubusercontent.com/u/12483662?s=64&v=4}}
\item
  \href{https://github.com/AlistairKeiller}{\includegraphics[width=0.66667in,height=0.66667in]{https://avatars.githubusercontent.com/u/43255248?s=64&v=4}}
\item
  \href{https://github.com/LaurenzV}{\includegraphics[width=0.66667in,height=0.66667in]{https://avatars.githubusercontent.com/u/47084093?s=64&v=4}}
\item
  \href{https://github.com/Luis-Licea}{\includegraphics[width=0.66667in,height=0.66667in]{https://avatars.githubusercontent.com/u/48164786?s=64&v=4}}
\item
  \href{https://github.com/SekoiaTree}{\includegraphics[width=0.66667in,height=0.66667in]{https://avatars.githubusercontent.com/u/51149447?s=64&v=4}}
\item
  \href{https://github.com/astrale-sharp}{\includegraphics[width=0.66667in,height=0.66667in]{https://avatars.githubusercontent.com/u/53686698?s=64&v=4}}
\item
  \href{https://github.com/birktj}{\includegraphics[width=0.66667in,height=0.66667in]{https://avatars.githubusercontent.com/u/2932651?s=64&v=4}}
\item
  \href{https://github.com/dccsillag}{\includegraphics[width=0.66667in,height=0.66667in]{https://avatars.githubusercontent.com/u/15617291?s=64&v=4}}
\item
  \href{https://github.com/goggle}{\includegraphics[width=0.66667in,height=0.66667in]{https://avatars.githubusercontent.com/u/1856425?s=64&v=4}}
\item
  \href{https://github.com/johannes-wolf}{\includegraphics[width=0.66667in,height=0.66667in]{https://avatars.githubusercontent.com/u/519002?s=64&v=4}}
\item
  \href{https://github.com/mattfbacon}{\includegraphics[width=0.66667in,height=0.66667in]{https://avatars.githubusercontent.com/u/58113890?s=64&v=4}}
\item
  \href{https://github.com/michidk}{\includegraphics[width=0.66667in,height=0.66667in]{https://avatars.githubusercontent.com/u/3979930?s=64&v=4}}
\item
  \href{https://github.com/neunenak}{\includegraphics[width=0.66667in,height=0.66667in]{https://avatars.githubusercontent.com/u/311545?s=64&v=4}}
\item
  \href{https://github.com/pan93412}{\includegraphics[width=0.66667in,height=0.66667in]{https://avatars.githubusercontent.com/u/28441561?s=64&v=4}}
\item
  \href{https://github.com/rpitasky}{\includegraphics[width=0.66667in,height=0.66667in]{https://avatars.githubusercontent.com/u/111201305?s=64&v=4}}
\item
  \href{https://github.com/thinety}{\includegraphics[width=0.66667in,height=0.66667in]{https://avatars.githubusercontent.com/u/51510921?s=64&v=4}}
\item
  \href{https://github.com/tranzystorekk}{\includegraphics[width=0.66667in,height=0.66667in]{https://avatars.githubusercontent.com/u/5671049?s=64&v=4}}
\item
  \href{https://github.com/werifu}{\includegraphics[width=0.66667in,height=0.66667in]{https://avatars.githubusercontent.com/u/53432474?s=64&v=4}}
\end{itemize}

\href{/docs/changelog/0.4.0/}{\pandocbounded{\includesvg[keepaspectratio]{/assets/icons/16-arrow-right.svg}}}

{ 0.4.0 } { Previous page }

\href{/docs/changelog/0.2.0/}{\pandocbounded{\includesvg[keepaspectratio]{/assets/icons/16-arrow-right.svg}}}

{ 0.2.0 } { Next page }


\section{Docs LaTeX/typst.app/docs/changelog/0.4.0.tex}
\title{typst.app/docs/changelog/0.4.0}

\begin{itemize}
\tightlist
\item
  \href{/docs}{\includesvg[width=0.16667in,height=0.16667in]{/assets/icons/16-docs-dark.svg}}
\item
  \includesvg[width=0.16667in,height=0.16667in]{/assets/icons/16-arrow-right.svg}
\item
  \href{/docs/changelog/}{Changelog}
\item
  \includesvg[width=0.16667in,height=0.16667in]{/assets/icons/16-arrow-right.svg}
\item
  \href{/docs/changelog/0.4.0/}{0.4.0}
\end{itemize}

\section{Version 0.4.0 (May 20, 2023)}\label{version-0.4.0-may-20-2023}

\subsection{Footnotes}\label{footnotes}

\begin{itemize}
\tightlist
\item
  Implemented support for footnotes
\item
  The \href{/docs/reference/model/footnote/}{\texttt{\ footnote\ }}
  function inserts a footnote
\item
  The
  \href{/docs/reference/model/footnote/\#definitions-entry}{\texttt{\ footnote.entry\ }}
  function can be used to customize the footnote listing
\item
  The \texttt{\ }{\texttt{\ "chicago-notes"\ }}\texttt{\ }
  \href{/docs/reference/model/cite/\#parameters-style}{citation style}
  is now available
\end{itemize}

\subsection{Documentation}\label{documentation}

\begin{itemize}
\tightlist
\item
  Added a \href{/docs/guides/guide-for-latex-users/}{Guide for LaTeX
  users}
\item
  Now shows default values for optional arguments
\item
  Added richer outlines in "On this Page"
\item
  Added initial support for search keywords: "Table of Contents" will
  now find the \href{/docs/reference/model/outline/}{outline} function.
  Suggestions for more keywords are welcome!
\item
  Fixed issue with search result ranking
\item
  Fixed many more small issues
\end{itemize}

\subsection{Math}\label{math}

\begin{itemize}
\tightlist
\item
  \textbf{Breaking change} : Alignment points ( \texttt{\ \&\ } ) in
  equations now alternate between left and right alignment
\item
  Added support for writing roots with Unicode: For example,
  \texttt{\ }{\texttt{\ \$\ }}\texttt{\ }{\texttt{\ root\ }}\texttt{\ }{\texttt{\ (\ }}\texttt{\ x+y\ }{\texttt{\ )\ }}\texttt{\ }{\texttt{\ \$\ }}\texttt{\ }
  can now also be written as
  \texttt{\ }{\texttt{\ \$\ }}\texttt{\ }{\texttt{\ √\ }}\texttt{\ }{\texttt{\ (\ }}\texttt{\ x+y\ }{\texttt{\ )\ }}\texttt{\ }{\texttt{\ \$\ }}\texttt{\ }
\item
  Fixed uneven vertical
  \href{/docs/reference/math/attach/\#functions-attach}{\texttt{\ attachment\ }}
  alignment
\item
  Fixed spacing on decorated elements (e.g., spacing around a
  \href{/docs/reference/math/cancel/}{canceled} operator)
\item
  Fixed styling for stretchable symbols
\item
  Added \texttt{\ tack.r.double\ } , \texttt{\ tack.l.double\ } ,
  \texttt{\ dotless.i\ } and \texttt{\ dotless.j\ }
  \href{/docs/reference/symbols/sym/}{symbols}
\item
  Fixed show rules on symbols (e.g.
  \texttt{\ }{\texttt{\ show\ }}\texttt{\ sym\ }{\texttt{\ .\ }}\texttt{\ }{\texttt{\ tack\ }}\texttt{\ }{\texttt{\ :\ }}\texttt{\ }{\texttt{\ set\ }}\texttt{\ }{\texttt{\ text\ }}\texttt{\ }{\texttt{\ (\ }}\texttt{\ blue\ }{\texttt{\ )\ }}\texttt{\ }
  )
\item
  Fixed missing rename from \texttt{\ ast.op\ } to \texttt{\ ast\ } that
  should have been in the previous release
\end{itemize}

\subsection{Scripting}\label{scripting}

\begin{itemize}
\tightlist
\item
  Added function scopes: A function can now hold related definitions in
  its own scope, similar to a module. The new
  \href{/docs/reference/foundations/assert/\#definitions-eq}{\texttt{\ assert.eq\ }}
  function, for instance, is part of the
  \href{/docs/reference/foundations/assert/}{\texttt{\ assert\ }}
  function\textquotesingle s scope. Note that function scopes are
  currently only available for built-in functions.
\item
  Added
  \href{/docs/reference/foundations/assert/\#definitions-eq}{\texttt{\ assert.eq\ }}
  and
  \href{/docs/reference/foundations/assert/\#definitions-ne}{\texttt{\ assert.ne\ }}
  functions for simpler equality and inequality assertions with more
  helpful error messages
\item
  Exposed \href{/docs/reference/model/list/\#definitions-item}{list} ,
  \href{/docs/reference/model/enum/\#definitions-item}{enum} , and
  \href{/docs/reference/model/terms/\#definitions-item}{term list} items
  in their respective functions\textquotesingle{} scope
\item
  The \texttt{\ at\ } methods on
  \href{/docs/reference/foundations/str/\#definitions-at}{strings} ,
  \href{/docs/reference/foundations/array/\#definitions-at}{arrays} ,
  \href{/docs/reference/foundations/dictionary/\#definitions-at}{dictionaries}
  , and
  \href{/docs/reference/foundations/content/\#definitions-at}{content}
  now support specifying a default value
\item
  Added support for passing a function to
  \href{/docs/reference/foundations/str/\#definitions-replace}{\texttt{\ replace\ }}
  that is called with each match.
\item
  Fixed
  \href{/docs/reference/foundations/str/\#definitions-replace}{replacement}
  strings: They are now inserted completely verbatim instead of
  supporting the previous (unintended) magic dollar syntax for capture
  groups
\item
  Fixed bug with trailing placeholders in destructuring patterns
\item
  Fixed bug with underscore in parameter destructuring
\item
  Fixed crash with nested patterns and when hovering over an invalid
  pattern
\item
  Better error messages when casting to an
  \href{/docs/reference/foundations/int/}{integer} or
  \href{/docs/reference/foundations/float/}{float} fails
\end{itemize}

\subsection{Text and Layout}\label{text-and-layout}

\begin{itemize}
\tightlist
\item
  Implemented sophisticated CJK punctuation adjustment
\item
  Disabled
  \href{/docs/reference/text/text/\#parameters-overhang}{overhang} for
  CJK punctuation
\item
  Added basic translations for Traditional Chinese
\item
  Fixed \href{/docs/reference/text/raw/\#parameters-align}{alignment} of
  text inside raw blocks (centering a raw block, e.g. through a figure,
  will now keep the text itself left-aligned)
\item
  Added support for passing a array instead of a function to configure
  table cell
  \href{/docs/reference/model/table/\#parameters-align}{alignment} and
  \href{/docs/reference/model/table/\#parameters-fill}{fill} per column
\item
  Fixed automatic figure
  \href{/docs/reference/model/figure/\#parameters-kind}{\texttt{\ kind\ }}
  detection
\item
  Made alignment of
  \href{/docs/reference/model/enum/\#parameters-number-align}{enum
  numbers} configurable, defaulting to \texttt{\ end\ }
\item
  Figures can now be made breakable with a show-set rule for blocks in
  figure
\item
  Initial fix for smart quotes in RTL languages
\end{itemize}

\subsection{Export}\label{export}

\begin{itemize}
\tightlist
\item
  Fixed ligatures in PDF export: They are now copyable and searchable
\item
  Exported PDFs now embed ICC profiles for images that have them
\item
  Fixed export of strokes with zero thickness
\end{itemize}

\subsection{Web app}\label{web-app}

\begin{itemize}
\tightlist
\item
  Projects can now contain folders
\item
  Added upload by drag-and-drop into the file panel
\item
  Files from the file panel can now be dragged into the editor to insert
  them into a Typst file
\item
  You can now copy-paste images and other files from your computer
  directly into the editor
\item
  Added a button to resend confirmation email
\item
  Added an option to invert preview colors in dark mode
\item
  Added tips to the loading screen and the Help menu. Feel free to
  propose more!
\item
  Added syntax highlighting for YAML files
\item
  Allowed middle mouse button click on many buttons to navigate into a
  new tab
\item
  Allowed more project names
\item
  Fixed overridden Vim mode keybindings
\item
  Fixed many bugs regarding file upload and more
\end{itemize}

\subsection{Miscellaneous
Improvements}\label{miscellaneous-improvements}

\begin{itemize}
\tightlist
\item
  Improved performance of counters, state, and queries
\item
  Improved incremental parsing for more efficient recompilations
\item
  Added support for \texttt{\ .yaml\ } extension in addition to
  \texttt{\ .yml\ } for bibliographies
\item
  The CLI now emits escape codes only if the output is a TTY
\item
  For users of the \texttt{\ typst\ } crate: The \texttt{\ Document\ }
  is now \texttt{\ Sync\ } again and the \texttt{\ World\ }
  doesn\textquotesingle t have to be
  \texttt{\ \textquotesingle{}static\ } anymore
\end{itemize}

\subsection{Contributors}\label{contributors}

Thanks to everyone who contributed to this release!

\begin{itemize}
\tightlist
\item
  \href{https://github.com/Leedehai}{\includegraphics[width=0.66667in,height=0.66667in]{https://avatars.githubusercontent.com/u/18319900?s=64&v=4}}
\item
  \href{https://github.com/PgBiel}{\includegraphics[width=0.66667in,height=0.66667in]{https://avatars.githubusercontent.com/u/9021226?s=64&v=4}}
\item
  \href{https://github.com/Marmare314}{\includegraphics[width=0.66667in,height=0.66667in]{https://avatars.githubusercontent.com/u/49279081?s=64&v=4}}
\item
  \href{https://github.com/SUPERCILEX}{\includegraphics[width=0.66667in,height=0.66667in]{https://avatars.githubusercontent.com/u/9490724?s=64&v=4}}
\item
  \href{https://github.com/peng1999}{\includegraphics[width=0.66667in,height=0.66667in]{https://avatars.githubusercontent.com/u/12483662?s=64&v=4}}
\item
  \href{https://github.com/sitandr}{\includegraphics[width=0.66667in,height=0.66667in]{https://avatars.githubusercontent.com/u/60141933?s=64&v=4}}
\item
  \href{https://github.com/stevenskevin}{\includegraphics[width=0.66667in,height=0.66667in]{https://avatars.githubusercontent.com/u/48657161?s=64&v=4}}
\item
  \href{https://github.com/AlistairKeiller}{\includegraphics[width=0.66667in,height=0.66667in]{https://avatars.githubusercontent.com/u/43255248?s=64&v=4}}
\item
  \href{https://github.com/HarmoGlace}{\includegraphics[width=0.66667in,height=0.66667in]{https://avatars.githubusercontent.com/u/23212967?s=64&v=4}}
\item
  \href{https://github.com/LaurenzV}{\includegraphics[width=0.66667in,height=0.66667in]{https://avatars.githubusercontent.com/u/47084093?s=64&v=4}}
\item
  \href{https://github.com/MultisampledNight}{\includegraphics[width=0.66667in,height=0.66667in]{https://avatars.githubusercontent.com/u/80128916?s=64&v=4}}
\item
  \href{https://github.com/albertothedev}{\includegraphics[width=0.66667in,height=0.66667in]{https://avatars.githubusercontent.com/u/46131317?s=64&v=4}}
\item
  \href{https://github.com/dvdvgt}{\includegraphics[width=0.66667in,height=0.66667in]{https://avatars.githubusercontent.com/u/40773635?s=64&v=4}}
\item
  \href{https://github.com/emme1444}{\includegraphics[width=0.66667in,height=0.66667in]{https://avatars.githubusercontent.com/u/23585909?s=64&v=4}}
\item
  \href{https://github.com/goggle}{\includegraphics[width=0.66667in,height=0.66667in]{https://avatars.githubusercontent.com/u/1856425?s=64&v=4}}
\item
  \href{https://github.com/jannisko}{\includegraphics[width=0.66667in,height=0.66667in]{https://avatars.githubusercontent.com/u/40455076?s=64&v=4}}
\item
  \href{https://github.com/jassler}{\includegraphics[width=0.66667in,height=0.66667in]{https://avatars.githubusercontent.com/u/24298790?s=64&v=4}}
\item
  \href{https://github.com/johannes-wolf}{\includegraphics[width=0.66667in,height=0.66667in]{https://avatars.githubusercontent.com/u/519002?s=64&v=4}}
\item
  \href{https://github.com/liferooter}{\includegraphics[width=0.66667in,height=0.66667in]{https://avatars.githubusercontent.com/u/54807745?s=64&v=4}}
\item
  \href{https://github.com/maciemesser}{\includegraphics[width=0.66667in,height=0.66667in]{https://avatars.githubusercontent.com/u/34891249?s=64&v=4}}
\item
  \href{https://github.com/mattfbacon}{\includegraphics[width=0.66667in,height=0.66667in]{https://avatars.githubusercontent.com/u/58113890?s=64&v=4}}
\item
  \href{https://github.com/maxwell-thum}{\includegraphics[width=0.66667in,height=0.66667in]{https://avatars.githubusercontent.com/u/119913396?s=64&v=4}}
\item
  \href{https://github.com/maybechris}{\includegraphics[width=0.66667in,height=0.66667in]{https://avatars.githubusercontent.com/u/103339277?s=64&v=4}}
\item
  \href{https://github.com/michidk}{\includegraphics[width=0.66667in,height=0.66667in]{https://avatars.githubusercontent.com/u/3979930?s=64&v=4}}
\item
  \href{https://github.com/naim94a}{\includegraphics[width=0.66667in,height=0.66667in]{https://avatars.githubusercontent.com/u/227396?s=64&v=4}}
\item
  \href{https://github.com/owiecc}{\includegraphics[width=0.66667in,height=0.66667in]{https://avatars.githubusercontent.com/u/6896639?s=64&v=4}}
\item
  \href{https://github.com/pan93412}{\includegraphics[width=0.66667in,height=0.66667in]{https://avatars.githubusercontent.com/u/28441561?s=64&v=4}}
\item
  \href{https://github.com/szdytom}{\includegraphics[width=0.66667in,height=0.66667in]{https://avatars.githubusercontent.com/u/33175397?s=64&v=4}}
\end{itemize}

\href{/docs/changelog/0.5.0/}{\pandocbounded{\includesvg[keepaspectratio]{/assets/icons/16-arrow-right.svg}}}

{ 0.5.0 } { Previous page }

\href{/docs/changelog/0.3.0/}{\pandocbounded{\includesvg[keepaspectratio]{/assets/icons/16-arrow-right.svg}}}

{ 0.3.0 } { Next page }


\section{Docs LaTeX/typst.app/docs/changelog/0.6.0.tex}
\title{typst.app/docs/changelog/0.6.0}

\begin{itemize}
\tightlist
\item
  \href{/docs}{\includesvg[width=0.16667in,height=0.16667in]{/assets/icons/16-docs-dark.svg}}
\item
  \includesvg[width=0.16667in,height=0.16667in]{/assets/icons/16-arrow-right.svg}
\item
  \href{/docs/changelog/}{Changelog}
\item
  \includesvg[width=0.16667in,height=0.16667in]{/assets/icons/16-arrow-right.svg}
\item
  \href{/docs/changelog/0.6.0/}{0.6.0}
\end{itemize}

\section{Version 0.6.0 (June 30,
2023)}\label{version-0.6.0-june-30-2023}

\subsection{Package Management}\label{package-management}

\begin{itemize}
\tightlist
\item
  Typst now has built-in
  \href{/docs/reference/scripting/\#packages}{package management}
\item
  You can import \href{https://typst.app/universe/}{published} community
  packages or create and use
  \href{https://github.com/typst/packages\#local-packages}{system-local}
  ones
\item
  Published packages are also supported in the web app
\end{itemize}

\subsection{Math}\label{math}

\begin{itemize}
\tightlist
\item
  Added support for optical size variants of glyphs in math mode
\item
  Added argument to enable
  \href{/docs/reference/math/attach/\#functions-limits}{\texttt{\ limits\ }}
  conditionally depending on whether the equation is set in
  \href{/docs/reference/math/sizes/\#functions-display}{\texttt{\ display\ }}
  or
  \href{/docs/reference/math/sizes/\#functions-inline}{\texttt{\ inline\ }}
  style
\item
  Added \texttt{\ gt.eq.slant\ } and \texttt{\ lt.eq.slant\ } symbols
\item
  Increased precedence of factorials in math mode (
  \texttt{\ }{\texttt{\ \$\ }}\texttt{\ 1\ }{\texttt{\ /\ }}\texttt{\ n!\ }{\texttt{\ \$\ }}\texttt{\ }
  works correctly now)
\item
  Improved
  \href{/docs/reference/math/underover/\#functions-underline}{underlines}
  and
  \href{/docs/reference/math/underover/\#functions-overline}{overlines}
  in math mode
\item
  Fixed usage of
  \href{/docs/reference/math/attach/\#functions-limits}{\texttt{\ limits\ }}
  function in show rules
\item
  Fixed bugs with line breaks in equations
\end{itemize}

\subsection{Text and Layout}\label{text-and-layout}

\begin{itemize}
\tightlist
\item
  Added support for alternating page
  \href{/docs/reference/layout/page/\#parameters-margin}{margins} with
  the \texttt{\ inside\ } and \texttt{\ outside\ } keys
\item
  Added support for specifying the page
  \href{/docs/reference/layout/page/\#parameters-binding}{\texttt{\ binding\ }}
\item
  Added
  \href{/docs/reference/layout/pagebreak/\#parameters-to}{\texttt{\ to\ }}
  argument to pagebreak function to skip to the next even or odd page
\item
  Added basic i18n for a few more languages (TR, SQ, TL)
\item
  Fixed bug with missing table row at page break
\item
  Fixed bug with \href{/docs/reference/text/underline/}{underlines}
\item
  Fixed bug superfluous table lines
\item
  Fixed smart quotes after line breaks
\item
  Fixed a crash related to text layout
\end{itemize}

\subsection{Command line interface}\label{command-line-interface}

\begin{itemize}
\tightlist
\item
  \textbf{Breaking change:} Added requirement for \texttt{\ -\/-root\ }
  / \texttt{\ TYPST\_ROOT\ } directory to contain the input file because
  it designates the \emph{project} root. Existing setups that use
  \texttt{\ TYPST\_ROOT\ } to emulate package management should switch
  to \href{https://github.com/typst/packages\#local-packages}{local
  packages}
\item
  \textbf{Breaking change:} Now denies file access outside of the
  project root
\item
  Added support for local packages and on-demand package download
\item
  Now watches all relevant files, within the root and all packages
\item
  Now displays compilation time
\end{itemize}

\subsection{Miscellaneous
Improvements}\label{miscellaneous-improvements}

\begin{itemize}
\tightlist
\item
  Added
  \href{/docs/reference/model/outline/\#definitions-entry}{\texttt{\ outline.entry\ }}
  to customize outline entries with show rules
\item
  Added some hints for error messages
\item
  Added some missing syntaxes for
  \href{/docs/reference/text/raw/}{\texttt{\ raw\ }} highlighting
\item
  Improved rendering of rotated images in PNG export and web app
\item
  Made \href{/docs/reference/model/footnote/}{footnotes} reusable and
  referenceable
\item
  Fixed bug with citations and bibliographies in
  \href{/docs/reference/introspection/locate/}{\texttt{\ locate\ }}
\item
  Fixed inconsistent tense in documentation
\end{itemize}

\subsection{Development}\label{development}

\begin{itemize}
\tightlist
\item
  Added
  \href{https://github.com/typst/typst/blob/main/CONTRIBUTING.md}{contribution
  guide}
\item
  Reworked \texttt{\ World\ } interface to accommodate for package
  management and make it a bit simpler to implement \emph{(Breaking
  change for implementors)}
\end{itemize}

\subsection{Contributors}\label{contributors}

Thanks to everyone who contributed to this release!

\begin{itemize}
\tightlist
\item
  \href{https://github.com/bluebear94}{\includegraphics[width=0.66667in,height=0.66667in]{https://avatars.githubusercontent.com/u/2975203?s=64&v=4}}
\item
  \href{https://github.com/figsoda}{\includegraphics[width=0.66667in,height=0.66667in]{https://avatars.githubusercontent.com/u/40620903?s=64&v=4}}
\item
  \href{https://github.com/sitandr}{\includegraphics[width=0.66667in,height=0.66667in]{https://avatars.githubusercontent.com/u/60141933?s=64&v=4}}
\item
  \href{https://github.com/MDLC01}{\includegraphics[width=0.66667in,height=0.66667in]{https://avatars.githubusercontent.com/u/57839069?s=64&v=4}}
\item
  \href{https://github.com/Mafii}{\includegraphics[width=0.66667in,height=0.66667in]{https://avatars.githubusercontent.com/u/10061519?s=64&v=4}}
\item
  \href{https://github.com/azerupi}{\includegraphics[width=0.66667in,height=0.66667in]{https://avatars.githubusercontent.com/u/7647338?s=64&v=4}}
\item
  \href{https://github.com/damaxwell}{\includegraphics[width=0.66667in,height=0.66667in]{https://avatars.githubusercontent.com/u/918465?s=64&v=4}}
\item
  \href{https://github.com/AndyBarcia}{\includegraphics[width=0.66667in,height=0.66667in]{https://avatars.githubusercontent.com/u/40731413?s=64&v=4}}
\item
  \href{https://github.com/DVDTSB}{\includegraphics[width=0.66667in,height=0.66667in]{https://avatars.githubusercontent.com/u/66365801?s=64&v=4}}
\item
  \href{https://github.com/Jollywatt}{\includegraphics[width=0.66667in,height=0.66667in]{https://avatars.githubusercontent.com/u/24970860?s=64&v=4}}
\item
  \href{https://github.com/Luis-Licea}{\includegraphics[width=0.66667in,height=0.66667in]{https://avatars.githubusercontent.com/u/48164786?s=64&v=4}}
\item
  \href{https://github.com/Myriad-Dreamin}{\includegraphics[width=0.66667in,height=0.66667in]{https://avatars.githubusercontent.com/u/35292584?s=64&v=4}}
\item
  \href{https://github.com/PgBiel}{\includegraphics[width=0.66667in,height=0.66667in]{https://avatars.githubusercontent.com/u/9021226?s=64&v=4}}
\item
  \href{https://github.com/SUPERCILEX}{\includegraphics[width=0.66667in,height=0.66667in]{https://avatars.githubusercontent.com/u/9490724?s=64&v=4}}
\item
  \href{https://github.com/TomBinford}{\includegraphics[width=0.66667in,height=0.66667in]{https://avatars.githubusercontent.com/u/28466971?s=64&v=4}}
\item
  \href{https://github.com/abdulmelikbekmez}{\includegraphics[width=0.66667in,height=0.66667in]{https://avatars.githubusercontent.com/u/61517310?s=64&v=4}}
\item
  \href{https://github.com/alerque}{\includegraphics[width=0.66667in,height=0.66667in]{https://avatars.githubusercontent.com/u/173595?s=64&v=4}}
\item
  \href{https://github.com/chicoferreira}{\includegraphics[width=0.66667in,height=0.66667in]{https://avatars.githubusercontent.com/u/36338391?s=64&v=4}}
\item
  \href{https://github.com/jskherman}{\includegraphics[width=0.66667in,height=0.66667in]{https://avatars.githubusercontent.com/u/68434444?s=64&v=4}}
\item
  \href{https://github.com/lucifer1004}{\includegraphics[width=0.66667in,height=0.66667in]{https://avatars.githubusercontent.com/u/13583761?s=64&v=4}}
\item
  \href{https://github.com/raphCode}{\includegraphics[width=0.66667in,height=0.66667in]{https://avatars.githubusercontent.com/u/15750438?s=64&v=4}}
\item
  \href{https://github.com/thehydrogen}{\includegraphics[width=0.66667in,height=0.66667in]{https://avatars.githubusercontent.com/u/34059898?s=64&v=4}}
\item
  \href{https://github.com/zach-capalbo}{\includegraphics[width=0.66667in,height=0.66667in]{https://avatars.githubusercontent.com/u/1325621?s=64&v=4}}
\end{itemize}

\href{/docs/changelog/0.7.0/}{\pandocbounded{\includesvg[keepaspectratio]{/assets/icons/16-arrow-right.svg}}}

{ 0.7.0 } { Previous page }

\href{/docs/changelog/0.5.0/}{\pandocbounded{\includesvg[keepaspectratio]{/assets/icons/16-arrow-right.svg}}}

{ 0.5.0 } { Next page }


\section{Docs LaTeX/typst.app/docs/changelog/0.2.0.tex}
\title{typst.app/docs/changelog/0.2.0}

\begin{itemize}
\tightlist
\item
  \href{/docs}{\includesvg[width=0.16667in,height=0.16667in]{/assets/icons/16-docs-dark.svg}}
\item
  \includesvg[width=0.16667in,height=0.16667in]{/assets/icons/16-arrow-right.svg}
\item
  \href{/docs/changelog/}{Changelog}
\item
  \includesvg[width=0.16667in,height=0.16667in]{/assets/icons/16-arrow-right.svg}
\item
  \href{/docs/changelog/0.2.0/}{0.2.0}
\end{itemize}

\section{Version 0.2.0 (April 11,
2023)}\label{version-0.2.0-april-11-2023}

\subsection{Breaking changes}\label{breaking-changes}

\begin{itemize}
\tightlist
\item
  Removed support for iterating over index and value in
  \href{/docs/reference/scripting/\#loops}{for loops} . This is now
  handled via unpacking and enumerating. Same goes for the
  \href{/docs/reference/foundations/array/\#definitions-map}{\texttt{\ map\ }}
  method.
\item
  \href{/docs/reference/foundations/dictionary/}{Dictionaries} now
  iterate in insertion order instead of alphabetical order.
\end{itemize}

\subsection{New features}\label{new-features}

\begin{itemize}
\tightlist
\item
  Added \href{/docs/reference/scripting/\#bindings}{unpacking syntax}
  for let bindings, which allows things like
  \texttt{\ }{\texttt{\ let\ }}\texttt{\ }{\texttt{\ (\ }}\texttt{\ 1\ }{\texttt{\ ,\ }}\texttt{\ 2\ }{\texttt{\ )\ }}\texttt{\ }{\texttt{\ =\ }}\texttt{\ array\ }
\item
  Added
  \href{/docs/reference/foundations/array/\#definitions-enumerate}{\texttt{\ enumerate\ }}
  method
\item
  Added \href{/docs/reference/visualize/path/}{\texttt{\ path\ }}
  function for drawing Bézier paths
\item
  Added \href{/docs/reference/layout/layout/}{\texttt{\ layout\ }}
  function to access the size of the surrounding page or container
\item
  Added \texttt{\ key\ } parameter to
  \href{/docs/reference/foundations/array/\#definitions-sorted}{\texttt{\ sorted\ }}
  method
\end{itemize}

\subsection{Command line interface}\label{command-line-interface}

\begin{itemize}
\tightlist
\item
  Fixed \texttt{\ -\/-open\ } flag blocking the program
\item
  New Computer Modern font is now embedded into the binary
\item
  Shell completions and man pages can now be generated by setting the
  \texttt{\ GEN\_ARTIFACTS\ } environment variable to a target directory
  and then building Typst
\end{itemize}

\subsection{Miscellaneous
improvements}\label{miscellaneous-improvements}

\begin{itemize}
\tightlist
\item
  Fixed page numbering in outline
\item
  Added basic i18n for a few more languages (AR, NB, CS, NN, PL, SL, ES,
  UA, VI)
\item
  Added a few numbering patterns (Ihora, Chinese)
\item
  Added \texttt{\ sinc\ } \href{/docs/reference/math/op/}{operator}
\item
  Fixed bug where math could not be hidden with
  \href{/docs/reference/layout/hide/}{\texttt{\ hide\ }}
\item
  Fixed sizing issues with box, block, and shapes
\item
  Fixed some translations
\item
  Fixed inversion of "R" in
  \href{/docs/reference/math/variants/\#functions-cal}{\texttt{\ cal\ }}
  and
  \href{/docs/reference/math/variants/\#functions-frak}{\texttt{\ frak\ }}
  styles
\item
  Fixed some styling issues in math
\item
  Fixed supplements of references to headings
\item
  Fixed syntax highlighting of identifiers in certain scenarios
\item
  \href{/docs/reference/layout/ratio/}{Ratios} can now be multiplied
  with more types and be converted to
  \href{/docs/reference/foundations/float/}{floats} with the
  \href{/docs/reference/foundations/float/}{\texttt{\ float\ }} function
\end{itemize}

\subsection{Contributors}\label{contributors}

Thanks to everyone who contributed to this release!

\begin{itemize}
\tightlist
\item
  \href{https://github.com/Marmare314}{\includegraphics[width=0.66667in,height=0.66667in]{https://avatars.githubusercontent.com/u/49279081?s=64&v=4}}
\item
  \href{https://github.com/dccsillag}{\includegraphics[width=0.66667in,height=0.66667in]{https://avatars.githubusercontent.com/u/15617291?s=64&v=4}}
\item
  \href{https://github.com/Dherse}{\includegraphics[width=0.66667in,height=0.66667in]{https://avatars.githubusercontent.com/u/9665250?s=64&v=4}}
\item
  \href{https://github.com/EpicEricEE}{\includegraphics[width=0.66667in,height=0.66667in]{https://avatars.githubusercontent.com/u/7191192?s=64&v=4}}
\item
  \href{https://github.com/Leedehai}{\includegraphics[width=0.66667in,height=0.66667in]{https://avatars.githubusercontent.com/u/18319900?s=64&v=4}}
\item
  \href{https://github.com/G1gg1L3s}{\includegraphics[width=0.66667in,height=0.66667in]{https://avatars.githubusercontent.com/u/43041209?s=64&v=4}}
\item
  \href{https://github.com/PgBiel}{\includegraphics[width=0.66667in,height=0.66667in]{https://avatars.githubusercontent.com/u/9021226?s=64&v=4}}
\item
  \href{https://github.com/RLangendam}{\includegraphics[width=0.66667in,height=0.66667in]{https://avatars.githubusercontent.com/u/1749390?s=64&v=4}}
\item
  \href{https://github.com/Raphael-CV}{\includegraphics[width=0.66667in,height=0.66667in]{https://avatars.githubusercontent.com/u/95867256?s=64&v=4}}
\item
  \href{https://github.com/SekoiaTree}{\includegraphics[width=0.66667in,height=0.66667in]{https://avatars.githubusercontent.com/u/51149447?s=64&v=4}}
\item
  \href{https://github.com/SteamedFish}{\includegraphics[width=0.66667in,height=0.66667in]{https://avatars.githubusercontent.com/u/69167?s=64&v=4}}
\item
  \href{https://github.com/asayers}{\includegraphics[width=0.66667in,height=0.66667in]{https://avatars.githubusercontent.com/u/877109?s=64&v=4}}
\item
  \href{https://github.com/asibahi}{\includegraphics[width=0.66667in,height=0.66667in]{https://avatars.githubusercontent.com/u/17417266?s=64&v=4}}
\item
  \href{https://github.com/astrale-sharp}{\includegraphics[width=0.66667in,height=0.66667in]{https://avatars.githubusercontent.com/u/53686698?s=64&v=4}}
\item
  \href{https://github.com/classabbyamp}{\includegraphics[width=0.66667in,height=0.66667in]{https://avatars.githubusercontent.com/u/5366828?s=64&v=4}}
\item
  \href{https://github.com/cmoog}{\includegraphics[width=0.66667in,height=0.66667in]{https://avatars.githubusercontent.com/u/7585078?s=64&v=4}}
\item
  \href{https://github.com/felipeacsi}{\includegraphics[width=0.66667in,height=0.66667in]{https://avatars.githubusercontent.com/u/1522083?s=64&v=4}}
\item
  \href{https://github.com/figsoda}{\includegraphics[width=0.66667in,height=0.66667in]{https://avatars.githubusercontent.com/u/40620903?s=64&v=4}}
\item
  \href{https://github.com/ichxorya}{\includegraphics[width=0.66667in,height=0.66667in]{https://avatars.githubusercontent.com/u/26222301?s=64&v=4}}
\item
  \href{https://github.com/jakobrs}{\includegraphics[width=0.66667in,height=0.66667in]{https://avatars.githubusercontent.com/u/10761079?s=64&v=4}}
\item
  \href{https://github.com/michidk}{\includegraphics[width=0.66667in,height=0.66667in]{https://avatars.githubusercontent.com/u/3979930?s=64&v=4}}
\item
  \href{https://github.com/radimsuckr}{\includegraphics[width=0.66667in,height=0.66667in]{https://avatars.githubusercontent.com/u/2447438?s=64&v=4}}
\item
  \href{https://github.com/rqy2002}{\includegraphics[width=0.66667in,height=0.66667in]{https://avatars.githubusercontent.com/u/22741844?s=64&v=4}}
\item
  \href{https://github.com/s-zymon}{\includegraphics[width=0.66667in,height=0.66667in]{https://avatars.githubusercontent.com/u/12126978?s=64&v=4}}
\item
  \href{https://github.com/tbethe}{\includegraphics[width=0.66667in,height=0.66667in]{https://avatars.githubusercontent.com/u/58276357?s=64&v=4}}
\item
  \href{https://github.com/teenjuna}{\includegraphics[width=0.66667in,height=0.66667in]{https://avatars.githubusercontent.com/u/53595243?s=64&v=4}}
\item
  \href{https://github.com/viddrobnic}{\includegraphics[width=0.66667in,height=0.66667in]{https://avatars.githubusercontent.com/u/5161200?s=64&v=4}}
\end{itemize}

\href{/docs/changelog/0.3.0/}{\pandocbounded{\includesvg[keepaspectratio]{/assets/icons/16-arrow-right.svg}}}

{ 0.3.0 } { Previous page }

\href{/docs/changelog/0.1.0/}{\pandocbounded{\includesvg[keepaspectratio]{/assets/icons/16-arrow-right.svg}}}

{ 0.1.0 } { Next page }




\section{C Docs LaTeX/docs/guides.tex}
\section{Docs LaTeX/typst.app/docs/guides/table-guide.tex}
\title{typst.app/docs/guides/table-guide}

\begin{itemize}
\tightlist
\item
  \href{/docs}{\includesvg[width=0.16667in,height=0.16667in]{/assets/icons/16-docs-dark.svg}}
\item
  \includesvg[width=0.16667in,height=0.16667in]{/assets/icons/16-arrow-right.svg}
\item
  \href{/docs/guides/}{Guides}
\item
  \includesvg[width=0.16667in,height=0.16667in]{/assets/icons/16-arrow-right.svg}
\item
  \href{/docs/guides/table-guide/}{Table guide}
\end{itemize}

\section{Table guide}\label{table-guide}

Tables are a great way to present data to your readers in an easily
readable, compact, and organized manner. They are not only used for
numerical values, but also survey responses, task planning, schedules,
and more. Because of this wide set of possible applications, there is no
single best way to lay out a table. Instead, think about the data you
want to highlight, your document\textquotesingle s overarching design,
and ultimately how your table can best serve your readers.

Typst can help you with your tables by automating styling, importing
data from other applications, and more! This guide takes you through a
few of the most common questions you may have when adding a table to
your document with Typst. Feel free to skip to the section most relevant
to you â€`` we designed this guide to be read out of order.

If you want to look up a detail of how tables work, you should also
\href{/docs/reference/model/table/}{check out their reference page} .
And if you are looking for a table of contents rather than a normal
table, the reference page of the
\href{/docs/reference/model/outline/}{\texttt{\ outline\ } function} is
the right place to learn more.

\subsection{How to create a basic table?}\label{basic-tables}

In order to create a table in Typst, use the
\href{/docs/reference/model/table/}{\texttt{\ table\ } function} . For a
basic table, you need to tell the table function two things:

\begin{itemize}
\tightlist
\item
  The number of columns
\item
  The content for each of the table cells
\end{itemize}

So, let\textquotesingle s say you want to create a table with two
columns describing the ingredients for a cookie recipe:

\begin{verbatim}
#table(
  columns: 2,
  [*Amount*], [*Ingredient*],
  [360g], [Baking flour],
  [250g], [Butter (room temp.)],
  [150g], [Brown sugar],
  [100g], [Cane sugar],
  [100g], [70% cocoa chocolate],
  [100g], [35-40% cocoa chocolate],
  [2], [Eggs],
  [Pinch], [Salt],
  [Drizzle], [Vanilla extract],
)
\end{verbatim}

\includegraphics[width=5in,height=\textheight,keepaspectratio]{/assets/docs/DrLInmCn8ozR2FQ7a7txfwAAAAAAAAAA.png}

This example shows how to call, configure, and populate a table. Both
the column count and cell contents are passed to the table as arguments.
The \href{/docs/reference/foundations/function/}{argument list} is
surrounded by round parentheses. In it, we first pass the column count
as a named argument. Then, we pass multiple
\href{/docs/reference/foundations/content/}{content blocks} as
positional arguments. Each content block contains the contents for a
single cell.

To make the example more legible, we have placed two content block
arguments on each line, mimicking how they would appear in the table.
You could also write each cell on its own line. Typst does not care on
which line you place the arguments. Instead, Typst will place the
content cells from left to right (or right to left, if that is the
writing direction of your language) and then from top to bottom. It will
automatically add enough rows to your table so that it fits all of your
content.

It is best to wrap the header row of your table in the
\href{/docs/reference/model/table/\#definitions-header}{\texttt{\ table.header\ }
function} . This clarifies your intent and will also allow future
versions of Typst to make the output more accessible to users with a
screen reader:

\begin{verbatim}
#table(
  columns: 2,
  table.header[*Amount*][*Ingredient*],
  [360g], [Baking flour],
 // ... the remaining cells
)
\end{verbatim}

\includegraphics[width=5in,height=\textheight,keepaspectratio]{/assets/docs/aiVJ7GWTsso2QP8oAOCqrgAAAAAAAAAA.png}

You could also write a show rule that automatically
\href{/docs/reference/model/strong/}{strongly emphasizes} the contents
of the first cells for all tables. This quickly becomes useful if your
document contains multiple tables!

\begin{verbatim}
#show table.cell.where(y: 0): strong

#table(
  columns: 2,
  table.header[Amount][Ingredient],
  [360g], [Baking flour],
 // ... the remaining cells
)
\end{verbatim}

\includegraphics[width=5in,height=\textheight,keepaspectratio]{/assets/docs/U7CxDwHdU4boWRgp53c3XgAAAAAAAAAA.png}

We are using a show rule with a selector for cell coordinates here
instead of applying our styles directly to \texttt{\ table.header\ } .
This is due to a current limitation of Typst that will be fixed in a
future release.

Congratulations, you have created your first table! Now you can proceed
to \hyperref[column-sizes]{change column sizes} ,
\hyperref[strokes]{adjust the strokes} , \hyperref[fills]{add striped
rows} , and more!

\subsection{How to change the column sizes?}\label{column-sizes}

If you create a table and specify the number of columns, Typst will make
each column large enough to fit its largest cell. Often, you want
something different, for example, to make a table span the whole width
of the page. You can provide a list, specifying how wide you want each
column to be, through the \texttt{\ columns\ } argument. There are a few
different ways to specify column widths:

\begin{itemize}
\tightlist
\item
  First, there is \texttt{\ }{\texttt{\ auto\ }}\texttt{\ } . This is
  the default behavior and tells Typst to grow the column to fit its
  contents. If there is not enough space, Typst will try its best to
  distribute the space among the
  \texttt{\ }{\texttt{\ auto\ }}\texttt{\ } -sized columns.
\item
  \href{/docs/reference/layout/length/}{Lengths} like
  \texttt{\ }{\texttt{\ 6cm\ }}\texttt{\ } ,
  \texttt{\ }{\texttt{\ 0.7in\ }}\texttt{\ } , or
  \texttt{\ }{\texttt{\ 120pt\ }}\texttt{\ } . As usual, you can also
  use the font-dependent \texttt{\ em\ } unit. This is a multiple of
  your current font size. It\textquotesingle s useful if you want to
  size your table so that it always fits about the same amount of text,
  independent of font size.
\item
  A \href{/docs/reference/layout/ratio/}{ratio in percent} such as
  \texttt{\ }{\texttt{\ 40\%\ }}\texttt{\ } . This will make the column
  take up 40\% of the total horizontal space available to the table, so
  either the inner width of the page or the table\textquotesingle s
  container. You can also mix ratios and lengths into
  \href{/docs/reference/layout/relative/}{relative lengths} . Be mindful
  that even if you specify a list of column widths that sum up to 100\%,
  your table could still become larger than its container. This is
  because there can be
  \href{/docs/reference/model/table/\#parameters-gutter}{gutter} between
  columns that is not included in the column widths. If you want to make
  a table fill the page, the next option is often very useful.
\item
  A \href{/docs/reference/layout/fraction/}{fractional part of the free
  space} using the \texttt{\ fr\ } unit, such as \texttt{\ 1fr\ } . This
  unit allows you to distribute the available space to columns. It works
  as follows: First, Typst sums up the lengths of all columns that do
  not use \texttt{\ fr\ } s. Then, it determines how much horizontal
  space is left. This horizontal space then gets distributed to all
  columns denominated in \texttt{\ fr\ } s. During this process, a
  \texttt{\ 2fr\ } column will become twice as wide as a
  \texttt{\ 1fr\ } column. This is where the name comes from: The width
  of the column is its fraction of the total fractionally sized columns.
\end{itemize}

Let\textquotesingle s put this to use with a table that contains the
dates, numbers, and descriptions of some routine checks. The first two
columns are \texttt{\ auto\ } -sized and the last column is
\texttt{\ 1fr\ } wide as to fill the whole page.

\begin{verbatim}
#table(
  columns: (auto, auto, 1fr),
  table.header[Date][°No][Description],
  [24/01/03], [813], [Filtered participant pool],
  [24/01/03], [477], [Transitioned to sec. regimen],
  [24/01/11], [051], [Cycled treatment substrate],
)
\end{verbatim}

\includegraphics[width=5in,height=\textheight,keepaspectratio]{/assets/docs/2U_qe89XvypLADel8g81vgAAAAAAAAAA.png}

Here, we have passed our list of column lengths as an
\href{/docs/reference/foundations/array/}{array} , enclosed in round
parentheses, with its elements separated by commas. The first two
columns are automatically sized, so that they take on the size of their
content and the third column is sized as
\texttt{\ }{\texttt{\ 1fr\ }}\texttt{\ } so that it fills up the
remainder of the space on the page. If you wanted to instead change the
second column to be a bit more spacious, you could replace its entry in
the \texttt{\ columns\ } array with a value like
\texttt{\ }{\texttt{\ 6em\ }}\texttt{\ } .

\subsection{How to caption and reference my
table?}\label{captions-and-references}

A table is just as valuable as the information your readers draw from
it. You can enhance the effectiveness of both your prose and your table
by making a clear connection between the two with a cross-reference.
Typst can help you with automatic
\href{/docs/reference/model/ref/}{references} and the
\href{/docs/reference/model/figure/}{\texttt{\ figure\ } function} .

Just like with images, wrapping a table in the \texttt{\ figure\ }
function allows you to add a caption and a label, so you can reference
the figure elsewhere. Wrapping your table in a figure also lets you use
the figure\textquotesingle s \texttt{\ placement\ } parameter to float
it to the top or bottom of a page.

Let\textquotesingle s take a look at a captioned table and how to
reference it in prose:

\begin{verbatim}
#show table.cell.where(y: 0): set text(weight: "bold")

#figure(
  table(
    columns: 4,
    stroke: none,

    table.header[Test Item][Specification][Test Result][Compliance],
    [Voltage], [220V ± 5%], [218V], [Pass],
    [Current], [5A ± 0.5A], [4.2A], [Fail],
  ),
  caption: [Probe results for design A],
) <probe-a>

The results from @probe-a show that the design is not yet optimal.
We will show how its performance can be improved in this section.
\end{verbatim}

\includegraphics[width=8.27083in,height=\textheight,keepaspectratio]{/assets/docs/q8w9c3xaiqkQD9Ab1PAe6AAAAAAAAAAA.png}

The example shows how to wrap a table in a figure, set a caption and a
label, and how to reference that label. We start by using the
\texttt{\ figure\ } function. It expects the contents of the figure as a
positional argument. We just put the table function call in its argument
list, omitting the \texttt{\ \#\ } character because it is only needed
when calling a function in markup mode. We also add the caption as a
named argument (above or below) the table.

After the figure call, we put a label in angle brackets (
\texttt{\ }{\texttt{\ \textless{}probe-a\textgreater{}\ }}\texttt{\ } ).
This tells Typst to remember this element and make it referenceable
under this name throughout your document. We can then reference it in
prose by using the at sign and the label name
\texttt{\ }{\texttt{\ @probe-a\ }}\texttt{\ } . Typst will print a
nicely formatted reference and automatically update the label if the
table\textquotesingle s number changes.

\subsection{How to get a striped table?}\label{fills}

Many tables use striped rows or columns instead of strokes to
differentiate between rows and columns. This effect is often called
\emph{zebra stripes.} Tables with zebra stripes are popular in Business
and commercial Data Analytics applications, while academic applications
tend to use strokes instead.

To add zebra stripes to a table, we use the \texttt{\ table\ }
function\textquotesingle s \texttt{\ fill\ } argument. It can take three
kinds of arguments:

\begin{itemize}
\tightlist
\item
  A single color (this can also be a gradient or a pattern) to fill all
  cells with. Because we want some cells to have another color, this is
  not useful if we want to build zebra tables.
\item
  An array with colors which Typst cycles through for each column. We
  can use an array with two elements to get striped columns.
\item
  A function that takes the horizontal coordinate \texttt{\ x\ } and the
  vertical coordinate \texttt{\ y\ } of a cell and returns its fill. We
  can use this to create horizontal stripes or
  \href{/docs/reference/layout/grid/\#definitions-cell}{checkerboard
  patterns} .
\end{itemize}

Let\textquotesingle s start with an example of a horizontally striped
table:

\begin{verbatim}
#set text(font: "IBM Plex Sans")

// Medium bold table header.
#show table.cell.where(y: 0): set text(weight: "medium")

// Bold titles.
#show table.cell.where(x: 1): set text(weight: "bold")

// See the strokes section for details on this!
#let frame(stroke) = (x, y) => (
  left: if x > 0 { 0pt } else { stroke },
  right: stroke,
  top: if y < 2 { stroke } else { 0pt },
  bottom: stroke,
)

#set table(
  fill: (rgb("EAF2F5"), none),
  stroke: frame(rgb("21222C")),
)

#table(
  columns: (0.4fr, 1fr, 1fr, 1fr),

  table.header[Month][Title][Author][Genre],
  [January], [The Great Gatsby], [F. Scott Fitzgerald], [Classic],
  [February], [To Kill a Mockingbird], [Harper Lee], [Drama],
  [March], [1984], [George Orwell], [Dystopian],
  [April], [The Catcher in the Rye], [J.D. Salinger], [Coming-of-Age],
)
\end{verbatim}

\includegraphics[width=9.44792in,height=\textheight,keepaspectratio]{/assets/docs/UJoXkTjV0r6grf2p2oGJMAAAAAAAAAAA.png}

This example shows a book club reading list. The line
\texttt{\ fill:\ }{\texttt{\ (\ }}\texttt{\ }{\texttt{\ rgb\ }}\texttt{\ }{\texttt{\ (\ }}\texttt{\ }{\texttt{\ "EAF2F5"\ }}\texttt{\ }{\texttt{\ )\ }}\texttt{\ }{\texttt{\ ,\ }}\texttt{\ }{\texttt{\ none\ }}\texttt{\ }{\texttt{\ )\ }}\texttt{\ }
in \texttt{\ table\ } \textquotesingle s set rule is all that is needed
to add striped columns. It tells Typst to alternate between coloring
columns with a light blue (in the
\href{/docs/reference/visualize/color/\#definitions-rgb}{\texttt{\ rgb\ }}
function call) and nothing ( \texttt{\ }{\texttt{\ none\ }}\texttt{\ }
). Note that we extracted all of our styling from the \texttt{\ table\ }
function call itself into set and show rules, so that we can
automatically reuse it for multiple tables.

Because setting the stripes itself is easy we also added some other
styles to make it look nice. The other code in the example provides a
dark blue \hyperref[stroke-functions]{stroke} around the table and below
the first line and emboldens the first row and the column with the book
title. See the \hyperref[strokes]{strokes} section for details on how we
achieved this stroke configuration.

Let\textquotesingle s next take a look at how we can change only the set
rule to achieve horizontal stripes instead:

\begin{verbatim}
#set table(
  fill: (_, y) => if calc.odd(y) { rgb("EAF2F5") },
  stroke: frame(rgb("21222C")),
)
\end{verbatim}

\includegraphics[width=9.44792in,height=\textheight,keepaspectratio]{/assets/docs/25lCWRlQDgNwC6P_ss5rsgAAAAAAAAAA.png}

We just need to replace the set rule from the previous example with this
one and get horizontal stripes instead. Here, we are passing a function
to \texttt{\ fill\ } . It discards the horizontal coordinate with an
underscore and then checks if the vertical coordinate \texttt{\ y\ } of
the cell is odd. If so, the cell gets a light blue fill, otherwise, no
fill is returned.

Of course, you can make this function arbitrarily complex. For example,
if you want to stripe the rows with a light and darker shade of blue,
you could do something like this:

\begin{verbatim}
#set table(
  fill: (_, y) => (none, rgb("EAF2F5"), rgb("DDEAEF")).at(calc.rem(y, 3)),
  stroke: frame(rgb("21222C")),
)
\end{verbatim}

\includegraphics[width=9.44792in,height=\textheight,keepaspectratio]{/assets/docs/cCvrBfZ8ZZjy8abtCE_cmgAAAAAAAAAA.png}

This example shows an alternative approach to write our fill function.
The function uses an array with three colors and then cycles between its
values for each row by indexing the array with the remainder of
\texttt{\ y\ } divided by 3.

Finally, here is a bonus example that uses the \emph{stroke} to achieve
striped rows:

\begin{verbatim}
#set table(
  stroke: (x, y) => (
    y: 1pt,
    left: if x > 0 { 0pt } else if calc.even(y) { 1pt },
    right: if calc.even(y) { 1pt },
  ),
)
\end{verbatim}

\includegraphics[width=9.44792in,height=\textheight,keepaspectratio]{/assets/docs/GD_bV9_znidgeZ0v5zIXUAAAAAAAAAAA.png}

\subsubsection{Manually overriding a cell\textquotesingle s fill
color}\label{fill-override}

Sometimes, the fill of a cell needs not to vary based on its position in
the table, but rather based on its contents. We can use the
\href{/docs/reference/model/table/\#definitions-cell}{\texttt{\ table.cell\ }
element} in the \texttt{\ table\ } \textquotesingle s parameter list to
wrap a cell\textquotesingle s content and override its fill.

For example, here is a list of all German presidents, with the cell
borders colored in the color of their party.

\begin{verbatim}
#set text(font: "Roboto")

#let cdu(name) = ([CDU], table.cell(fill: black, text(fill: white, name)))
#let spd(name) = ([SPD], table.cell(fill: red, text(fill: white, name)))
#let fdp(name) = ([FDP], table.cell(fill: yellow, name))

#table(
  columns: (auto, auto, 1fr),
  stroke: (x: none),

  table.header[Tenure][Party][President],
  [1949-1959], ..fdp[Theodor Heuss],
  [1959-1969], ..cdu[Heinrich Lübke],
  [1969-1974], ..spd[Gustav Heinemann],
  [1974-1979], ..fdp[Walter Scheel],
  [1979-1984], ..cdu[Karl Carstens],
  [1984-1994], ..cdu[Richard von Weizsäcker],
  [1994-1999], ..cdu[Roman Herzog],
  [1999-2004], ..spd[Johannes Rau],
  [2004-2010], ..cdu[Horst Köhler],
  [2010-2012], ..cdu[Christian Wulff],
  [2012-2017], [n/a], [Joachim Gauck],
  [2017-],     ..spd[Frank-Walter-Steinmeier],
)
\end{verbatim}

\includegraphics[width=5.90625in,height=\textheight,keepaspectratio]{/assets/docs/Kc5oSTV9kIbzwSd275OEwQAAAAAAAAAA.png}

In this example, we make use of variables because there only have been a
total of three parties whose members have become president (and one
unaffiliated president). Their colors will repeat multiple times, so we
store a function that produces an array with their
party\textquotesingle s name and a table cell with that
party\textquotesingle s color and the president\textquotesingle s name (
\texttt{\ cdu\ } , \texttt{\ spd\ } , and \texttt{\ fdp\ } ). We then
use these functions in the \texttt{\ table\ } argument list instead of
directly adding the name. We use the
\href{/docs/reference/foundations/arguments/\#spreading}{spread
operator} \texttt{\ ..\ } to turn the items of the arrays into single
cells. We could also write something like
\texttt{\ }{\texttt{\ {[}\ }}\texttt{\ FDP\ }{\texttt{\ {]}\ }}\texttt{\ ,\ table\ }{\texttt{\ .\ }}\texttt{\ }{\texttt{\ cell\ }}\texttt{\ }{\texttt{\ (\ }}\texttt{\ fill\ }{\texttt{\ :\ }}\texttt{\ yellow\ }{\texttt{\ )\ }}\texttt{\ }{\texttt{\ {[}\ }}\texttt{\ Theodor\ Heuss\ }{\texttt{\ {]}\ }}\texttt{\ }
for each cell directly in the \texttt{\ table\ } \textquotesingle s
argument list, but that becomes unreadable, especially for the parties
whose colors are dark so that they require white text. We also delete
vertical strokes and set the font to Roboto.

The party column and the cell color in this example communicate
redundant information on purpose: Communicating important data using
color only is a bad accessibility practice. It disadvantages users with
vision impairment and is in violation of universal access standards,
such as the
\href{https://www.w3.org/WAI/WCAG21/Understanding/use-of-color.html}{WCAG
2.1 Success Criterion 1.4.1} . To improve this table, we added a column
printing the party name. Alternatively, you could have made sure to
choose a color-blindness friendly palette and mark up your cells with an
additional label that screen readers can read out loud. The latter
feature is not currently supported by Typst, but will be added in a
future release. You can check how colors look for color-blind readers
with
\href{https://chromewebstore.google.com/detail/colorblindly/floniaahmccleoclneebhhmnjgdfijgg}{this
Chrome extension} ,
\href{https://helpx.adobe.com/photoshop/using/proofing-colors.html}{Photoshop}
, or
\href{https://docs.gimp.org/2.10/en/gimp-display-filter-dialog.html}{GIMP}
.

\subsection{How to adjust the lines in a table?}\label{strokes}

By default, Typst adds strokes between each row and column of a table.
You can adjust these strokes in a variety of ways. Which one is the most
practical, depends on the modification you want to make and your intent:

\begin{itemize}
\tightlist
\item
  Do you want to style all tables in your document, irrespective of
  their size and content? Use the \texttt{\ table\ }
  function\textquotesingle s
  \href{/docs/reference/model/table/\#parameters-stroke}{stroke}
  argument in a set rule.
\item
  Do you want to customize all lines in a single table? Use the
  \texttt{\ table\ } function\textquotesingle s
  \href{/docs/reference/model/table/\#parameters-stroke}{stroke}
  argument when calling the table function.
\item
  Do you want to change, add, or remove the stroke around a single cell?
  Use the \texttt{\ table.cell\ } element in the argument list of your
  table call.
\item
  Do you want to change, add, or remove a single horizontal or vertical
  stroke in a single table? Use the
  \href{/docs/reference/model/table/\#definitions-hline}{\texttt{\ table.hline\ }}
  and
  \href{/docs/reference/model/table/\#definitions-vline}{\texttt{\ table.vline\ }}
  elements in the argument list of your table call.
\end{itemize}

We will go over all of these options with examples next! First, we will
tackle the \texttt{\ table\ } function\textquotesingle s
\href{/docs/reference/model/table/\#parameters-stroke}{stroke} argument.
Here, you can adjust both how the table\textquotesingle s lines get
drawn and configure which lines are drawn at all.

Let\textquotesingle s start by modifying the color and thickness of the
stroke:

\begin{verbatim}
#table(
  columns: 4,
  stroke: 0.5pt + rgb("666675"),
  [*Monday*], [11.5], [13.0], [4.0],
  [*Tuesday*], [8.0], [14.5], [5.0],
  [*Wednesday*], [9.0], [18.5], [13.0],
)
\end{verbatim}

\includegraphics[width=5in,height=\textheight,keepaspectratio]{/assets/docs/y0uGf9aX-aUGWqCKMwHEJwAAAAAAAAAA.png}

This makes the table lines a bit less wide and uses a bluish gray. You
can see that we added a width in point to a color to achieve our
customized stroke. This addition yields a value of the
\href{/docs/reference/visualize/stroke/}{stroke type} . Alternatively,
you can use the dictionary representation for strokes which allows you
to access advanced features such as dashed lines.

The previous example showed how to use the stroke argument in the table
function\textquotesingle s invocation. Alternatively, you can specify
the stroke argument in the \texttt{\ table\ } \textquotesingle s set
rule. This will have exactly the same effect on all subsequent
\texttt{\ table\ } calls as if the stroke argument was specified in the
argument list. This is useful if you are writing a template or want to
style your whole document.

\begin{verbatim}
// Renders the exact same as the last example
#set table(stroke: 0.5pt + rgb("666675"))

#table(
  columns: 4,
  [*Monday*], [11.5], [13.0], [4.0],
  [*Tuesday*], [8.0], [14.5], [5.0],
  [*Wednesday*], [9.0], [18.5], [13.0],
)
\end{verbatim}

For small tables, you sometimes want to suppress all strokes because
they add too much visual noise. To do this, just set the stroke argument
to \texttt{\ }{\texttt{\ none\ }}\texttt{\ } :

\begin{verbatim}
#table(
  columns: 4,
  stroke: none,
  [*Monday*], [11.5], [13.0], [4.0],
  [*Tuesday*], [8.0], [14.5], [5.0],
  [*Wednesday*], [9.0], [18.5], [13.0],
)
\end{verbatim}

\includegraphics[width=5in,height=\textheight,keepaspectratio]{/assets/docs/vQmODupZkk6MbI3nByjkqwAAAAAAAAAA.png}

If you want more fine-grained control of where lines get placed in your
table, you can also pass a dictionary with the keys \texttt{\ top\ } ,
\texttt{\ left\ } , \texttt{\ right\ } , \texttt{\ bottom\ }
(controlling the respective cell sides), \texttt{\ x\ } , \texttt{\ y\ }
(controlling vertical and horizontal strokes), and \texttt{\ rest\ }
(covers all strokes not styled by other dictionary entries). All keys
are optional; omitted keys will be treated as if their value was the
default value. For example, to get a table with only horizontal lines,
you can do this:

\begin{verbatim}
#table(
  columns: 2,
  stroke: (x: none),
  align: horizon,
  [☒], [Close cabin door],
  [☐], [Start engines],
  [☐], [Radio tower],
  [☐], [Push back],
)
\end{verbatim}

\includegraphics[width=5in,height=\textheight,keepaspectratio]{/assets/docs/eb2IfSiEgTxcBIC79R6b-QAAAAAAAAAA.png}

This turns off all vertical strokes and leaves the horizontal strokes in
place. To achieve the reverse effect (only horizontal strokes), set the
stroke argument to
\texttt{\ }{\texttt{\ (\ }}\texttt{\ y\ }{\texttt{\ :\ }}\texttt{\ }{\texttt{\ none\ }}\texttt{\ }{\texttt{\ )\ }}\texttt{\ }
instead.

\hyperref[stroke-functions]{Further down in the guide} , we cover how to
use a function in the stroke argument to customize all strokes
individually. This is how you achieve more complex stroking patterns.

\subsubsection{Adding individual lines in the
table}\label{individual-lines}

If you want to add a single horizontal or vertical line in your table,
for example to separate a group of rows, you can use the
\href{/docs/reference/model/table/\#definitions-hline}{\texttt{\ table.hline\ }}
and
\href{/docs/reference/model/table/\#definitions-vline}{\texttt{\ table.vline\ }}
elements for horizontal and vertical lines, respectively. Add them to
the argument list of the \texttt{\ table\ } function just like you would
add individual cells and a header.

Let\textquotesingle s take a look at the following example from the
reference:

\begin{verbatim}
#set table.hline(stroke: 0.6pt)

#table(
  stroke: none,
  columns: (auto, 1fr),
  // Morning schedule abridged.
  [14:00], [Talk: Tracked Layout],
  [15:00], [Talk: Automations],
  [16:00], [Workshop: Tables],
  table.hline(),
  [19:00], [Day 1 Attendee Mixer],
)
\end{verbatim}

\includegraphics[width=5in,height=\textheight,keepaspectratio]{/assets/docs/3OGDUdhafmvnsbwhpEZnqAAAAAAAAAAA.png}

In this example, you can see that we have placed a call to
\texttt{\ table.hline\ } between the cells, producing a horizontal line
at that spot. We also used a set rule on the element to reduce its
stroke width to make it fit better with the weight of the font.

By default, Typst places horizontal and vertical lines after the current
row or column, depending on their position in the argument list. You can
also manually move them to a different position by adding the
\texttt{\ y\ } (for \texttt{\ hline\ } ) or \texttt{\ x\ } (for
\texttt{\ vline\ } ) argument. For example, the code below would produce
the same result:

\begin{verbatim}
#set table.hline(stroke: 0.6pt)

#table(
  stroke: none,
  columns: (auto, 1fr),
  // Morning schedule abridged.
  table.hline(y: 3),
  [14:00], [Talk: Tracked Layout],
  [15:00], [Talk: Automations],
  [16:00], [Workshop: Tables],
  [19:00], [Day 1 Attendee Mixer],
)
\end{verbatim}

Let\textquotesingle s imagine you are working with a template that shows
none of the table strokes except for one between the first and second
row. Now, since you have one table that also has labels in the first
column, you want to add an extra vertical line to it. However, you do
not want this vertical line to cross into the top row. You can achieve
this with the \texttt{\ start\ } argument:

\begin{verbatim}
// Base template already configured tables, but we need some
// extra configuration for this table.
#{
  set table(align: (x, _) => if x == 0 { left } else { right })
  show table.cell.where(x: 0): smallcaps
  table(
    columns: (auto, 1fr, 1fr, 1fr),
    table.vline(x: 1, start: 1),
    table.header[Trainset][Top Speed][Length][Weight],
    [TGV Réseau], [320 km/h], [200m], [383t],
    [ICE 403], [330 km/h], [201m], [409t],
    [Shinkansen N700], [300 km/h], [405m], [700t],
  )
}
\end{verbatim}

\includegraphics[width=7.08333in,height=\textheight,keepaspectratio]{/assets/docs/LINgvIDoMEPxmydMRnWXAgAAAAAAAAAA.png}

In this example, we have added \texttt{\ table.vline\ } at the start of
our positional argument list. But because the line is not supposed to go
to the left of the first column, we specified the \texttt{\ x\ }
argument as \texttt{\ }{\texttt{\ 1\ }}\texttt{\ } . We also set the
\texttt{\ start\ } argument to \texttt{\ }{\texttt{\ 1\ }}\texttt{\ } so
that the line does only start after the first row.

The example also contains two more things: We use the align argument
with a function to right-align the data in all but the first column and
use a show rule to make the first column of table cells appear in small
capitals. Because these styles are specific to this one table, we put
everything into a \href{/docs/reference/scripting/\#blocks}{code block}
, so that the styling does not affect any further tables.

\subsubsection{Overriding the strokes of a single
cell}\label{stroke-override}

Imagine you want to change the stroke around a single cell. Maybe your
cell is very important and needs highlighting! For this scenario, there
is the
\href{/docs/reference/model/table/\#definitions-cell}{\texttt{\ table.cell\ }
function} . Instead of adding your content directly in the argument list
of the table, you wrap it in a \texttt{\ table.cell\ } call. Now, you
can use \texttt{\ table.cell\ } \textquotesingle s argument list to
override the table properties, such as the stroke, for this cell only.

Here\textquotesingle s an example with a matrix of two of the Big Five
personality factors, with one intersection highlighted.

\begin{verbatim}
#table(
  columns: 3,
  stroke: (x: none),

  [], [*High Neuroticism*], [*Low Neuroticism*],

  [*High Agreeableness*],
  table.cell(stroke: orange + 2pt)[
    _Sensitive_ \ Prone to emotional distress but very empathetic.
  ],
  [_Compassionate_ \ Caring and stable, often seen as a supportive figure.],

  [*Low Agreeableness*],
  [_Contentious_ \ Competitive and easily agitated.],
  [_Detached_ \ Independent and calm, may appear aloof.],
)
\end{verbatim}

\includegraphics[width=9.44792in,height=\textheight,keepaspectratio]{/assets/docs/zT-rPhodrO99vfmjEknOKwAAAAAAAAAA.png}

Above, you can see that we used the \texttt{\ table.cell\ } element in
the table\textquotesingle s argument list and passed the cell content to
it. We have used its \texttt{\ stroke\ } argument to set a wider orange
stroke. Despite the fact that we disabled vertical strokes on the table,
the orange stroke appeared on all sides of the modified cell, showing
that the table\textquotesingle s stroke configuration is overwritten.

\subsubsection{Complex document-wide stroke
customization}\label{stroke-functions}

This section explains how to customize all lines at once in one or
multiple tables. This allows you to draw only the first horizontal line
or omit the outer lines, without knowing how many cells the table has.
This is achieved by providing a function to the table\textquotesingle s
\texttt{\ stroke\ } parameter. The function should return a stroke given
the zero-indexed x and y position of the current cell. You should only
need these functions if you are a template author, do not use a
template, or need to heavily customize your tables. Otherwise, your
template should set appropriate default table strokes.

For example, this is a set rule that draws all horizontal lines except
for the very first and last line.

\begin{verbatim}
#show table.cell.where(x: 0): set text(style: "italic")
#show table.cell.where(y: 0): set text(style: "normal", weight: "bold")
#set table(stroke: (_, y) => if y > 0 { (top: 0.8pt) })

#table(
  columns: 3,
  align: center + horizon,
  table.header[Technique][Advantage][Drawback],
  [Diegetic], [Immersive], [May be contrived],
  [Extradiegetic], [Breaks immersion], [Obtrusive],
  [Omitted], [Fosters engagement], [May fracture audience],
)
\end{verbatim}

\includegraphics[width=5in,height=\textheight,keepaspectratio]{/assets/docs/SKiPog79AfwUoglArT-17wAAAAAAAAAA.png}

In the set rule, we pass a function that receives two arguments,
assigning the vertical coordinate to \texttt{\ y\ } and discarding the
horizontal coordinate. It then returns a stroke dictionary with a
\texttt{\ }{\texttt{\ 0.8pt\ }}\texttt{\ } top stroke for all but the
first line. The cells in the first line instead implicitly receive
\texttt{\ }{\texttt{\ none\ }}\texttt{\ } as the return value. You can
easily modify this function to just draw the inner vertical lines
instead as
\texttt{\ }{\texttt{\ (\ }}\texttt{\ x\ }{\texttt{\ ,\ }}\texttt{\ \_\ }{\texttt{\ )\ }}\texttt{\ }{\texttt{\ =\textgreater{}\ }}\texttt{\ }{\texttt{\ if\ }}\texttt{\ x\ }{\texttt{\ \textgreater{}\ }}\texttt{\ }{\texttt{\ 0\ }}\texttt{\ }{\texttt{\ \{\ }}\texttt{\ }{\texttt{\ (\ }}\texttt{\ left\ }{\texttt{\ :\ }}\texttt{\ }{\texttt{\ 0.8pt\ }}\texttt{\ }{\texttt{\ )\ }}\texttt{\ }{\texttt{\ \}\ }}\texttt{\ }
.

Let\textquotesingle s try a few more stroking functions. The next
function will only draw a line below the first row:

\begin{verbatim}
#set table(stroke: (_, y) => if y == 0 { (bottom: 1pt) })

// Table as seen above
\end{verbatim}

\includegraphics[width=5in,height=\textheight,keepaspectratio]{/assets/docs/nyvsIK-tDvHuRwbnwhj4kQAAAAAAAAAA.png}

If you understood the first example, it becomes obvious what happens
here. We check if we are in the first row. If so, we return a bottom
stroke. Otherwise, we\textquotesingle ll return
\texttt{\ }{\texttt{\ none\ }}\texttt{\ } implicitly.

The next example shows how to draw all but the outer lines:

\begin{verbatim}
#set table(stroke: (x, y) => (
  left: if x > 0 { 0.8pt },
  top: if y > 0 { 0.8pt },
))

// Table as seen above
\end{verbatim}

\includegraphics[width=5in,height=\textheight,keepaspectratio]{/assets/docs/QRh4-eyPqD4Yz9VResk9dwAAAAAAAAAA.png}

This example uses both the \texttt{\ x\ } and \texttt{\ y\ }
coordinates. It omits the left stroke in the first column and the top
stroke in the first row. The right and bottom lines are not drawn.

Finally, here is a table that draws all lines except for the vertical
lines in the first row and horizontal lines in the table body. It looks
a bit like a calendar.

\begin{verbatim}
#set table(stroke: (x, y) => (
  left: if x == 0 or y > 0 { 1pt } else { 0pt },
  right: 1pt,
  top: if y <= 1 { 1pt } else { 0pt },
  bottom: 1pt,
))

// Table as seen above
\end{verbatim}

\includegraphics[width=5in,height=\textheight,keepaspectratio]{/assets/docs/qw8tKe_4tpVbtFAmFf0POQAAAAAAAAAA.png}

This example is a bit more complex. We start by drawing all the strokes
on the right of the cells. But this means that we have drawn strokes in
the top row, too, and we don\textquotesingle t need those! We use the
fact that \texttt{\ left\ } will override \texttt{\ right\ } and only
draw the left line if we are not in the first row or if we are in the
first column. In all other cases, we explicitly remove the left line.
Finally, we draw the horizontal lines by first setting the bottom line
and then for the first two rows with the \texttt{\ top\ } key,
suppressing all other top lines. The last line appears because there is
no \texttt{\ top\ } line that could suppress it.

\subsubsection{How to achieve a double line?}\label{double-stroke}

Typst does not yet have a native way to draw double strokes, but there
are multiple ways to emulate them, for example with
\href{/docs/reference/visualize/pattern/}{patterns} . We will show a
different workaround in this section: Table gutters.

Tables can space their cells apart using the \texttt{\ gutter\ }
argument. When a gutter is applied, a stroke is drawn on each of the now
separated cells. We can selectively add gutter between the rows or
columns for which we want to draw a double line. The
\texttt{\ row-gutter\ } and \texttt{\ column-gutter\ } arguments allow
us to do this. They accept arrays of gutter values.
Let\textquotesingle s take a look at an example:

\begin{verbatim}
#table(
  columns: 3,
  stroke: (x: none),
  row-gutter: (2.2pt, auto),
  table.header[Date][Exercise Type][Calories Burned],
  [2023-03-15], [Swimming], [400],
  [2023-03-17], [Weightlifting], [250],
  [2023-03-18], [Yoga], [200],
)
\end{verbatim}

\includegraphics[width=5in,height=\textheight,keepaspectratio]{/assets/docs/ketpWMF_8TQEKQlrbhiNRQAAAAAAAAAA.png}

We can see that we used an array for \texttt{\ row-gutter\ } that
specifies a \texttt{\ }{\texttt{\ 2.2pt\ }}\texttt{\ } gap between the
first and second row. It then continues with \texttt{\ auto\ } (which is
the default, in this case \texttt{\ }{\texttt{\ 0pt\ }}\texttt{\ }
gutter) which will be the gutter between all other rows, since it is the
last entry in the array.

\subsection{How to align the contents of the cells in my
table?}\label{alignment}

You can use multiple mechanisms to align the content in your table. You
can either use the \texttt{\ table\ } function\textquotesingle s
\texttt{\ align\ } argument to set the alignment for your whole table
(or use it in a set rule to set the alignment for tables throughout your
document) or the
\href{/docs/reference/layout/align/}{\texttt{\ align\ }} function (or
\texttt{\ table.cell\ } \textquotesingle s \texttt{\ align\ } argument)
to override the alignment of a single cell.

When using the \texttt{\ table\ } function\textquotesingle s align
argument, you can choose between three methods to specify an
\href{/docs/reference/layout/alignment/}{alignment} :

\begin{itemize}
\tightlist
\item
  Just specify a single alignment like \texttt{\ right\ } (aligns in the
  top-right corner) or \texttt{\ center\ +\ horizon\ } (centers all cell
  content). This changes the alignment of all cells.
\item
  Provide an array. Typst will cycle through this array for each column.
\item
  Provide a function that is passed the horizontal \texttt{\ x\ } and
  vertical \texttt{\ y\ } coordinate of a cell and returns an alignment.
\end{itemize}

For example, this travel itinerary right-aligns the day column and
left-aligns everything else by providing an array in the
\texttt{\ align\ } argument:

\begin{verbatim}
#set text(font: "IBM Plex Sans")
#show table.cell.where(y: 0): set text(weight: "bold")

#table(
  columns: 4,
  align: (right, left, left, left),
  fill: (_, y) => if calc.odd(y) { green.lighten(90%) },
  stroke: none,

  table.header[Day][Location][Hotel or Apartment][Activities],
  [1], [Paris, France], [Hotel de L'Europe], [Arrival, Evening River Cruise],
  [2], [Paris, France], [Hotel de L'Europe], [Louvre Museum, Eiffel Tower],
  [3], [Lyon, France], [Lyon City Hotel], [City Tour, Local Cuisine Tasting],
  [4], [Geneva, Switzerland], [Lakeview Inn], [Lake Geneva, Red Cross Museum],
  [5], [Zermatt, Switzerland], [Alpine Lodge], [Visit Matterhorn, Skiing],
)
\end{verbatim}

\includegraphics[width=7.08333in,height=\textheight,keepaspectratio]{/assets/docs/JDfAUmIJzQHzE6NL8LKAnwAAAAAAAAAA.png}

However, this example does not yet look perfect â€'' the header cells
should be bottom-aligned. Let\textquotesingle s use a function instead
to do so:

\begin{verbatim}
#set text(font: "IBM Plex Sans")
#show table.cell.where(y: 0): set text(weight: "bold")

#table(
  columns: 4,
  align: (x, y) =>
    if x == 0 { right } else { left } +
    if y == 0 { bottom } else { top },
  fill: (_, y) => if calc.odd(y) { green.lighten(90%) },
  stroke: none,

  table.header[Day][Location][Hotel or Apartment][Activities],
  [1], [Paris, France], [Hotel de L'Europe], [Arrival, Evening River Cruise],
  [2], [Paris, France], [Hotel de L'Europe], [Louvre Museum, Eiffel Tower],
 // ... remaining days omitted
)
\end{verbatim}

\includegraphics[width=7.08333in,height=\textheight,keepaspectratio]{/assets/docs/ENpsdZXkKtSK7dBT6e9dAgAAAAAAAAAA.png}

In the function, we calculate a horizontal and vertical alignment based
on whether we are in the first column (
\texttt{\ x\ }{\texttt{\ ==\ }}\texttt{\ }{\texttt{\ 0\ }}\texttt{\ } )
or the first row (
\texttt{\ y\ }{\texttt{\ ==\ }}\texttt{\ }{\texttt{\ 0\ }}\texttt{\ } ).
We then make use of the fact that we can add horizontal and vertical
alignments with \texttt{\ +\ } to receive a single, two-dimensional
alignment.

You can find an example of using \texttt{\ table.cell\ } to change a
single cell\textquotesingle s alignment on
\href{/docs/reference/model/table/\#definitions-cell}{its reference
page} .

\subsection{How to merge cells?}\label{merge-cells}

When a table contains logical groupings or the same data in multiple
adjacent cells, merging multiple cells into a single, larger cell can be
advantageous. Another use case for cell groups are table headers with
multiple rows: That way, you can group for example a sales data table by
quarter in the first row and by months in the second row.

A merged cell spans multiple rows and/or columns. You can achieve it
with the
\href{/docs/reference/model/table/\#definitions-cell}{\texttt{\ table.cell\ }}
function\textquotesingle s \texttt{\ rowspan\ } and \texttt{\ colspan\ }
arguments: Just specify how many rows or columns you want your cell to
span.

The example below contains an attendance calendar for an office with
in-person and remote days for each team member. To make the table more
glanceable, we merge adjacent cells with the same value:

\begin{verbatim}
#let ofi = [Office]
#let rem = [_Remote_]
#let lea = [*On leave*]

#show table.cell.where(y: 0): set text(
  fill: white,
  weight: "bold",
)

#table(
  columns: 6 * (1fr,),
  align: (x, y) => if x == 0 or y == 0 { left } else { center },
  stroke: (x, y) => (
    // Separate black cells with white strokes.
    left: if y == 0 and x > 0 { white } else { black },
    rest: black,
  ),
  fill: (_, y) => if y == 0 { black },

  table.header(
    [Team member],
    [Monday],
    [Tuesday],
    [Wednesday],
    [Thursday],
    [Friday]
  ),
  [Evelyn Archer],
    table.cell(colspan: 2, ofi),
    table.cell(colspan: 2, rem),
    ofi,
  [Lila Montgomery],
    table.cell(colspan: 5, lea),
  [Nolan Pearce],
    rem,
    table.cell(colspan: 2, ofi),
    rem,
    ofi,
)
\end{verbatim}

\includegraphics[width=12.98958in,height=\textheight,keepaspectratio]{/assets/docs/yyaKf-LLQbyy3QccVJXb2wAAAAAAAAAA.png}

In the example, we first define variables with "Office", "Remote", and
"On leave" so we don\textquotesingle t have to write these labels out
every time. We can then use these variables in the table body either
directly or in a \texttt{\ table.cell\ } call if the team member spends
multiple consecutive days in office, remote, or on leave.

The example also contains a black header (created with
\texttt{\ table\ } \textquotesingle s \texttt{\ fill\ } argument) with
white strokes ( \texttt{\ table\ } \textquotesingle s
\texttt{\ stroke\ } argument) and white text (set by the
\texttt{\ table.cell\ } set rule). Finally, we align all the content of
all table cells in the body in the center. If you want to know more
about the functions passed to \texttt{\ align\ } , \texttt{\ stroke\ } ,
and \texttt{\ fill\ } , you can check out the sections on
\href{/docs/reference/layout/alignment/}{alignment} ,
\hyperref[stroke-functions]{strokes} , and \hyperref[fills]{striped
tables} .

This table would be a great candidate for fully automated generation
from an external data source! Check out the
\hyperref[importing-data]{section about importing data} to learn more
about that.

\subsection{How to rotate a table?}\label{rotate-table}

When tables have many columns, a portrait paper orientation can quickly
get cramped. Hence, you\textquotesingle ll sometimes want to switch your
tables to landscape orientation. There are two ways to accomplish this
in Typst:

\begin{itemize}
\tightlist
\item
  If you want to rotate only the table but not the other content of the
  page and the page itself, use the
  \href{/docs/reference/layout/rotate/}{\texttt{\ rotate\ } function}
  with the \texttt{\ reflow\ } argument set to
  \texttt{\ }{\texttt{\ true\ }}\texttt{\ } .
\item
  If you want to rotate the whole page the table is on, you can use the
  \href{/docs/reference/layout/page/}{\texttt{\ page\ } function} with
  its \texttt{\ flipped\ } argument set to
  \texttt{\ }{\texttt{\ true\ }}\texttt{\ } . The header, footer, and
  page number will now also appear on the long edge of the page. This
  has the advantage that the table will appear right side up when read
  on a computer, but it also means that a page in your document has
  different dimensions than all the others, which can be jarring to your
  readers.
\end{itemize}

Below, we will demonstrate both techniques with a student grade book
table.

First, we will rotate the table on the page. The example also places
some text on the right of the table.

\begin{verbatim}
#set page("a5", columns: 2, numbering: "— 1 —")
#show table.cell.where(y: 0): set text(weight: "bold")

#rotate(
  -90deg,
  reflow: true,

  table(
    columns: (1fr,) + 5 * (auto,),
    inset: (x: 0.6em,),
    stroke: (_, y) => (
      x: 1pt,
      top: if y <= 1 { 1pt } else { 0pt },
      bottom: 1pt,
    ),
    align: (left, right, right, right, right, left),

    table.header(
      [Student Name],
      [Assignment 1], [Assignment 2],
      [Mid-term], [Final Exam],
      [Total Grade],
    ),
    [Jane Smith], [78%], [82%], [75%], [80%], [B],
    [Alex Johnson], [90%], [95%], [94%], [96%], [A+],
    [John Doe], [85%], [90%], [88%], [92%], [A],
    [Maria Garcia], [88%], [84%], [89%], [85%], [B+],
    [Zhang Wei], [93%], [89%], [90%], [91%], [A-],
    [Marina Musterfrau], [96%], [91%], [74%], [69%], [B-],
  ),
)

#lorem(80)
\end{verbatim}

\includegraphics[width=8.73958in,height=\textheight,keepaspectratio]{/assets/docs/t_mhLmSe89ZV_R--e5hFagAAAAAAAAAA.png}

What we have here is a two-column document on ISO A5 paper with page
numbers on the bottom. The table has six columns and contains a few
customizations to \hyperref[strokes]{stroke} , alignment and spacing.
But the most important part is that the table is wrapped in a call to
the \texttt{\ rotate\ } function with the \texttt{\ reflow\ } argument
being \texttt{\ }{\texttt{\ true\ }}\texttt{\ } . This will make the
table rotate 90 degrees counterclockwise. The reflow argument is needed
so that the table\textquotesingle s rotation affects the layout. If it
was omitted, Typst would lay out the page as if the table was not
rotated ( \texttt{\ }{\texttt{\ true\ }}\texttt{\ } might become the
default in the future).

The example also shows how to produce many columns of the same size: To
the initial \texttt{\ }{\texttt{\ 1fr\ }}\texttt{\ } column, we add an
array with five \texttt{\ }{\texttt{\ auto\ }}\texttt{\ } items that we
create by multiplying an array with one
\texttt{\ }{\texttt{\ auto\ }}\texttt{\ } item by five. Note that arrays
with just one item need a trailing comma to distinguish them from merely
parenthesized expressions.

The second example shows how to rotate the whole page, so that the table
stays upright:

\begin{verbatim}
#set page("a5", numbering: "— 1 —")
#show table.cell.where(y: 0): set text(weight: "bold")

#page(flipped: true)[
  #table(
    columns: (1fr,) + 5 * (auto,),
    inset: (x: 0.6em,),
    stroke: (_, y) => (
      x: 1pt,
      top: if y <= 1 { 1pt } else { 0pt },
      bottom: 1pt,
    ),
    align: (left, right, right, right, right, left),

    table.header(
      [Student Name],
      [Assignment 1], [Assignment 2],
      [Mid-term], [Final Exam],
      [Total Grade],
    ),
    [Jane Smith], [78%], [82%], [75%], [80%], [B],
    [Alex Johnson], [90%], [95%], [94%], [96%], [A+],
    [John Doe], [85%], [90%], [88%], [92%], [A],
    [Maria Garcia], [88%], [84%], [89%], [85%], [B+],
    [Zhang Wei], [93%], [89%], [90%], [91%], [A-],
    [Marina Musterfrau], [96%], [91%], [74%], [69%], [B-],
  )

  #pad(x: 15%, top: 1.5em)[
    = Winter 2023/24 results
    #lorem(80)
  ]
]
\end{verbatim}

\includegraphics[width=12.40625in,height=\textheight,keepaspectratio]{/assets/docs/9Goo6xTF0vprclhyFSb4zwAAAAAAAAAA.png}

Here, we take the same table and the other content we want to set with
it and put it into a call to the
\href{/docs/reference/layout/page/}{\texttt{\ page\ }} function while
supplying \texttt{\ }{\texttt{\ true\ }}\texttt{\ } to the
\texttt{\ flipped\ } argument. This will instruct Typst to create new
pages with width and height swapped and place the contents of the
function call onto a new page. Notice how the page number is also on the
long edge of the paper now. At the bottom of the page, we use the
\href{/docs/reference/layout/pad/}{\texttt{\ pad\ }} function to
constrain the width of the paragraph to achieve a nice and legible line
length.

\subsection{How to break a table across
pages?}\label{table-across-pages}

It is best to contain a table on a single page. However, some tables
just have many rows, so breaking them across pages becomes unavoidable.
Fortunately, Typst supports breaking tables across pages out of the box.
If you are using the
\href{/docs/reference/model/table/\#definitions-header}{\texttt{\ table.header\ }}
and
\href{/docs/reference/model/table/\#definitions-footer}{\texttt{\ table.footer\ }}
functions, their contents will be repeated on each page as the first and
last rows, respectively. If you want to disable this behavior, you can
set \texttt{\ repeat\ } to \texttt{\ }{\texttt{\ false\ }}\texttt{\ } on
either of them.

If you have placed your table inside of a
\href{/docs/reference/model/figure/}{figure} , it becomes unable to
break across pages by default. However, you can change this behavior.
Let\textquotesingle s take a look:

\begin{verbatim}
#set page(width: 9cm, height: 6cm)
#show table.cell.where(y: 0): set text(weight: "bold")
#show figure: set block(breakable: true)

#figure(
  caption: [Training regimen for Marathon],
  table(
    columns: 3,
    fill: (_, y) => if y == 0 { gray.lighten(75%) },

    table.header[Week][Distance (km)][Time (hh:mm:ss)],
    [1], [5],  [00:30:00],
    [2], [7],  [00:45:00],
    [3], [10], [01:00:00],
    [4], [12], [01:10:00],
    [5], [15], [01:25:00],
    [6], [18], [01:40:00],
    [7], [20], [01:50:00],
    [8], [22], [02:00:00],
    [...], [...], [...],
    table.footer[_Goal_][_42.195_][_02:45:00_],
  )
)
\end{verbatim}

\includegraphics[width=5.3125in,height=\textheight,keepaspectratio]{/assets/docs/LN81XNtn32FKL9Eh3VVjowAAAAAAAAAA.png}
\includegraphics[width=5.3125in,height=\textheight,keepaspectratio]{/assets/docs/LN81XNtn32FKL9Eh3VVjowAAAAAAAAAB.png}

A figure automatically produces a
\href{/docs/reference/layout/block/}{block} which cannot break by
default. However, we can reconfigure the block of the figure using a
show rule to make it \texttt{\ breakable\ } . Now, the figure spans
multiple pages with the headers and footers repeating.

\subsection{How to import data into a table?}\label{importing-data}

Often, you need to put data that you obtained elsewhere into a table.
Sometimes, this is from Microsoft Excel or Google Sheets, sometimes it
is from a dataset on the web or from your experiment. Fortunately, Typst
can load many \href{/docs/reference/data-loading/}{common file formats}
, so you can use scripting to include their data in a table.

The most common file format for tabular data is CSV. You can obtain a
CSV file from Excel by choosing "Save as" in the \emph{File} menu and
choosing the file format "CSV UTF-8 (Comma-delimited) (.csv)". Save the
file and, if you are using the web app, upload it to your project.

In our case, we will be building a table about Moore\textquotesingle s
Law. For this purpose, we are using a statistic with
\href{https://ourworldindata.org/grapher/transistors-per-microprocessor}{how
many transistors the average microprocessor consists of per year from
Our World in Data} . Let\textquotesingle s start by pressing the
"Download" button to get a CSV file with the raw data.

Be sure to move the file to your project or somewhere Typst can see it,
if you are using the CLI. Once you did that, we can open the file to see
how it is structured:

\begin{verbatim}
Entity,Code,Year,Transistors per microprocessor
World,OWID_WRL,1971,2308.2417
World,OWID_WRL,1972,3554.5222
World,OWID_WRL,1974,6097.5625
\end{verbatim}

The file starts with a header and contains four columns: Entity (which
is to whom the metric applies), Code, the year, and the number of
transistors per microprocessor. Only the last two columns change between
each row, so we can disregard "Entity" and "Code".

First, let\textquotesingle s start by loading this file with the
\href{/docs/reference/data-loading/csv/}{\texttt{\ csv\ }} function. It
accepts the file name of the file we want to load as a string argument:

\begin{verbatim}
#let moore = csv("moore.csv")
\end{verbatim}

We have loaded our file (assuming we named it \texttt{\ moore.csv\ } )
and \href{/docs/reference/scripting/\#bindings}{bound it} to the new
variable \texttt{\ moore\ } . This will not produce any output, so
there\textquotesingle s nothing to see yet. If we want to examine what
Typst loaded, we can either hover the name of the variable in the web
app or print some items from the array:

\begin{verbatim}
#let moore = csv("moore.csv")

#moore.slice(0, 3)
\end{verbatim}

\includegraphics[width=5in,height=\textheight,keepaspectratio]{/assets/docs/4Wrq2xnRY1_FUBMwiwPizQAAAAAAAAAA.png}

With the arguments
\texttt{\ }{\texttt{\ (\ }}\texttt{\ }{\texttt{\ 0\ }}\texttt{\ }{\texttt{\ ,\ }}\texttt{\ }{\texttt{\ 3\ }}\texttt{\ }{\texttt{\ )\ }}\texttt{\ }
, the
\href{/docs/reference/foundations/array/\#definitions-slice}{\texttt{\ slice\ }}
method returns the first three items in the array (with the indices 0,
1, and 2). We can see that each row is its own array with one item per
cell.

Now, let\textquotesingle s write a loop that will transform this data
into an array of cells that we can use with the table function.

\begin{verbatim}
#let moore = csv("moore.csv")

#table(
  columns: 2,
  ..for (.., year, count) in moore {
    (year, count)
  }
)
\end{verbatim}

\includegraphics[width=5in,height=\textheight,keepaspectratio]{/assets/docs/1JHLF5egGPQD3C2V0wZONgAAAAAAAAAA.png}

The example above uses a for loop that iterates over the rows in our CSV
file and returns an array for each iteration. We use the for
loop\textquotesingle s
\href{/docs/reference/scripting/\#bindings}{destructuring} capability to
discard all but the last two items of each row. We then create a new
array with just these two. Because Typst will concatenate the array
results of all the loop iterations, we get a one-dimensional array in
which the year column and the number of transistors alternate. We can
then insert the array as cells. For this we use the
\href{/docs/reference/foundations/arguments/\#spreading}{spread
operator} ( \texttt{\ ..\ } ). By prefixing an array, or, in our case an
expression that yields an array, with two dots, we tell Typst that the
array\textquotesingle s items should be used as positional arguments.

Alternatively, we can also use the
\href{/docs/reference/foundations/array/\#definitions-map}{\texttt{\ map\ }}
,
\href{/docs/reference/foundations/array/\#definitions-slice}{\texttt{\ slice\ }}
, and
\href{/docs/reference/foundations/array/\#definitions-flatten}{\texttt{\ flatten\ }}
array methods to write this in a more functional style:

\begin{verbatim}
#let moore = csv("moore.csv")

#table(
   columns: moore.first().len(),
   ..moore.map(m => m.slice(2)).flatten(),
)
\end{verbatim}

This example renders the same as the previous one, but first uses the
\texttt{\ map\ } function to change each row of the data. We pass a
function to map that gets run on each row of the CSV and returns a new
value to replace that row with. We use it to discard the first two
columns with \texttt{\ slice\ } . Then, we spread the data into the
\texttt{\ table\ } function. However, we need to pass a one-dimensional
array and \texttt{\ moore\ } \textquotesingle s value is two-dimensional
(that means that each of its row values contains an array with the cell
data). That\textquotesingle s why we call \texttt{\ flatten\ } which
converts it to a one-dimensional array. We also extract the number of
columns from the data itself.

Now that we have nice code for our table, we should try to also make the
table itself nice! The transistor counts go from millions in 1995 to
trillions in 2021 and changes are difficult to see with so many digits.
We could try to present our data logarithmically to make it more
digestible:

\begin{verbatim}
#let moore = csv("moore.csv")
#let moore-log = moore.slice(1).map(m => {
  let (.., year, count) = m
  let log = calc.log(float(count))
  let rounded = str(calc.round(log, digits: 2))
  (year, rounded)
})

#show table.cell.where(x: 0): strong

#table(
   columns: moore-log.first().len(),
   align: right,
   fill: (_, y) => if calc.odd(y) { rgb("D7D9E0") },
   stroke: none,

   table.header[Year][Transistor count ($log_10$)],
   table.hline(stroke: rgb("4D4C5B")),
   ..moore-log.flatten(),
)
\end{verbatim}

\includegraphics[width=5in,height=\textheight,keepaspectratio]{/assets/docs/ZOXJwHX6lPP_GLnz4eAxbQAAAAAAAAAA.png}

In this example, we first drop the header row from the data since we are
adding our own. Then, we discard all but the last two columns as above.
We do this by \href{/docs/reference/scripting/\#bindings}{destructuring}
the array \texttt{\ m\ } , discarding all but the two last items. We
then convert the string in \texttt{\ count\ } to a floating point
number, calculate its logarithm and store it in the variable
\texttt{\ log\ } . Finally, we round it to two digits, convert it to a
string, and store it in the variable \texttt{\ rounded\ } . Then, we
return an array with \texttt{\ year\ } and \texttt{\ rounded\ } that
replaces the original row. In our table, we have added our custom header
that tells the reader that we\textquotesingle ve applied a logarithm to
the values. Then, we spread the flattened data as above.

We also styled the table with \hyperref[fills]{stripes} , a
\hyperref[individual-lines]{horizontal line} below the first row,
\hyperref[alignment]{aligned} everything to the right, and emboldened
the first column. Click on the links to go to the relevant guide
sections and see how it\textquotesingle s done!

\subsection{What if I need the table function for something that
isn\textquotesingle t a table?}\label{table-and-grid}

Tabular layouts of content can be useful not only for matrices of
closely related data, like shown in the examples throughout this guide,
but also for presentational purposes. Typst differentiates between grids
that are for layout and presentational purposes only and tables, in
which the arrangement of the cells itself conveys information.

To make this difference clear to other software and allow templates to
heavily style tables, Typst has two functions for grid and table layout:

\begin{itemize}
\tightlist
\item
  The \href{/docs/reference/model/table/}{\texttt{\ table\ }} function
  explained throughout this guide which is intended for tabular data.
\item
  The \href{/docs/reference/layout/grid/}{\texttt{\ grid\ }} function
  which is intended for presentational purposes and page layout.
\end{itemize}

Both elements work the same way and have the same arguments. You can
apply everything you have learned about tables in this guide to grids.
There are only three differences:

\begin{itemize}
\tightlist
\item
  You\textquotesingle ll need to use the
  \href{/docs/reference/layout/grid/\#definitions-cell}{\texttt{\ grid.cell\ }}
  ,
  \href{/docs/reference/layout/grid/\#definitions-vline}{\texttt{\ grid.vline\ }}
  , and
  \href{/docs/reference/layout/grid/\#definitions-hline}{\texttt{\ grid.hline\ }}
  elements instead of
  \href{/docs/reference/model/table/\#definitions-cell}{\texttt{\ table.cell\ }}
  ,
  \href{/docs/reference/model/table/\#definitions-vline}{\texttt{\ table.vline\ }}
  , and
  \href{/docs/reference/model/table/\#definitions-hline}{\texttt{\ table.hline\ }}
  .
\item
  The grid has different defaults: It draws no strokes by default and
  has no spacing ( \texttt{\ inset\ } ) inside of its cells.
\item
  Elements like \texttt{\ figure\ } do not react to grids since they are
  supposed to have no semantical bearing on the document structure.
\end{itemize}

\href{/docs/guides/page-setup-guide/}{\pandocbounded{\includesvg[keepaspectratio]{/assets/icons/16-arrow-right.svg}}}

{ Page setup guide } { Previous page }

\href{/docs/changelog/}{\pandocbounded{\includesvg[keepaspectratio]{/assets/icons/16-arrow-right.svg}}}

{ Changelog } { Next page }


\section{Docs LaTeX/typst.app/docs/guides/page-setup-guide.tex}
\title{typst.app/docs/guides/page-setup-guide}

\begin{itemize}
\tightlist
\item
  \href{/docs}{\includesvg[width=0.16667in,height=0.16667in]{/assets/icons/16-docs-dark.svg}}
\item
  \includesvg[width=0.16667in,height=0.16667in]{/assets/icons/16-arrow-right.svg}
\item
  \href{/docs/guides/}{Guides}
\item
  \includesvg[width=0.16667in,height=0.16667in]{/assets/icons/16-arrow-right.svg}
\item
  \href{/docs/guides/page-setup-guide/}{Page setup guide}
\end{itemize}

\section{Page setup guide}\label{page-setup-guide}

Your page setup is a big part of the first impression your document
gives. Line lengths, margins, and columns influence
\href{https://practicaltypography.com/page-margins.html}{appearance} and
\href{https://designregression.com/article/line-length-revisited-following-the-research}{legibility}
while the right headers and footers will help your reader easily
navigate your document. This guide will help you to customize pages,
margins, headers, footers, and page numbers so that they are the right
fit for your content and you can get started with writing.

In Typst, each page has a width, a height, and margins on all four
sides. The top and bottom margins may contain a header and footer. The
set rule of the \href{/docs/reference/layout/page/}{\texttt{\ page\ }}
element is where you control all of the page setup. If you make changes
with this set rule, Typst will ensure that there is a new and conforming
empty page afterward, so it may insert a page break. Therefore, it is
best to specify your
\href{/docs/reference/layout/page/}{\texttt{\ page\ }} set rule at the
start of your document or in your template.

\begin{verbatim}
#set rect(
  width: 100%,
  height: 100%,
  inset: 4pt,
)

#set page(
  paper: "iso-b7",
  header: rect(fill: aqua)[Header],
  footer: rect(fill: aqua)[Footer],
  number-align: center,
)

#rect(fill: aqua.lighten(40%))
\end{verbatim}

\includegraphics[width=5.19792in,height=\textheight,keepaspectratio]{/assets/docs/f-lyLMO5vdXuAMoj28s3AgAAAAAAAAAA.png}

This example visualizes the dimensions for page content, headers, and
footers. The page content is the page size (ISO B7) minus each
side\textquotesingle s default margin. In the top and the bottom margin,
there are stroked rectangles visualizing the header and footer. They do
not touch the main content, instead, they are offset by 30\% of the
respective margin. You can control this offset by specifying the
\href{/docs/reference/layout/page/\#parameters-header-ascent}{\texttt{\ header-ascent\ }}
and
\href{/docs/reference/layout/page/\#parameters-footer-descent}{\texttt{\ footer-descent\ }}
arguments.

Below, the guide will go more into detail on how to accomplish common
page setup requirements with examples.

\subsection{Customize page size and margins}\label{customize-margins}

Typst\textquotesingle s default page size is A4 paper. Depending on your
region and your use case, you will want to change this. You can do this
by using the \href{/docs/reference/layout/page/}{\texttt{\ page\ }} set
rule and passing it a string argument to use a common page size. Options
include the complete ISO 216 series (e.g. \texttt{\ "iso-a4"\ } ,
\texttt{\ "iso-c2"\ } ), customary US formats like
\texttt{\ "us-legal"\ } or \texttt{\ "us-letter"\ } , and more. Check
out the reference for the
\href{/docs/reference/layout/page/\#parameters-paper}{page\textquotesingle s
paper argument} to learn about all available options.

\begin{verbatim}
#set page("us-letter")

This page likes freedom.
\end{verbatim}

\includegraphics[width=12.75in,height=\textheight,keepaspectratio]{/assets/docs/Qkb9fYb5SMfDaJ63aYycTwAAAAAAAAAA.png}

If you need to customize your page size to some dimensions, you can
specify the named arguments
\href{/docs/reference/layout/page/\#parameters-width}{\texttt{\ width\ }}
and
\href{/docs/reference/layout/page/\#parameters-height}{\texttt{\ height\ }}
instead.

\begin{verbatim}
#set page(width: 12cm, height: 12cm)

This page is a square.
\end{verbatim}

\includegraphics[width=7.08333in,height=\textheight,keepaspectratio]{/assets/docs/RltrP5jjKi_qENhy9B-HyQAAAAAAAAAA.png}

\subsubsection{Change the page\textquotesingle s
margins}\label{change-margins}

Margins are a vital ingredient for good typography:
\href{http://webtypography.net/2.1.2}{Typographers consider lines that
fit between 45 and 75 characters best length for legibility} and your
margins and \hyperref[columns]{columns} help define line widths. By
default, Typst will create margins proportional to the page size of your
document. To set custom margins, you will use the
\href{/docs/reference/layout/page/\#parameters-margin}{\texttt{\ margin\ }}
argument in the \href{/docs/reference/layout/page/}{\texttt{\ page\ }}
set rule.

The \texttt{\ margin\ } argument will accept a length if you want to set
all margins to the same width. However, you often want to set different
margins on each side. To do this, you can pass a dictionary:

\begin{verbatim}
#set page(margin: (
  top: 3cm,
  bottom: 2cm,
  x: 1.5cm,
))

#lorem(100)
\end{verbatim}

\includegraphics[width=5in,height=\textheight,keepaspectratio]{/assets/docs/yJb2DmVYA18DCHWDRA50QQAAAAAAAAAA.png}

The page margin dictionary can have keys for each side (
\texttt{\ top\ } , \texttt{\ bottom\ } , \texttt{\ left\ } ,
\texttt{\ right\ } ), but you can also control left and right together
by setting the \texttt{\ x\ } key of the margin dictionary, like in the
example. Likewise, the top and bottom margins can be adjusted together
by setting the \texttt{\ y\ } key.

If you do not specify margins for all sides in the margin dictionary,
the old margins will remain in effect for the unset sides. To prevent
this and set all remaining margins to a common size, you can use the
\texttt{\ rest\ } key. For example,
\texttt{\ }{\texttt{\ \#\ }}\texttt{\ }{\texttt{\ set\ }}\texttt{\ }{\texttt{\ page\ }}\texttt{\ }{\texttt{\ (\ }}\texttt{\ margin\ }{\texttt{\ :\ }}\texttt{\ }{\texttt{\ (\ }}\texttt{\ left\ }{\texttt{\ :\ }}\texttt{\ }{\texttt{\ 1.5in\ }}\texttt{\ }{\texttt{\ ,\ }}\texttt{\ rest\ }{\texttt{\ :\ }}\texttt{\ }{\texttt{\ 1in\ }}\texttt{\ }{\texttt{\ )\ }}\texttt{\ }{\texttt{\ )\ }}\texttt{\ }
will set the left margin to 1.5 inches and the remaining margins to one
inch.

\subsubsection{Different margins on alternating
pages}\label{alternating-margins}

Sometimes, you\textquotesingle ll need to alternate horizontal margins
for even and odd pages, for example, to have more room towards the spine
of a book than on the outsides of its pages. Typst keeps track of
whether a page is to the left or right of the binding. You can use this
information and set the \texttt{\ inside\ } or \texttt{\ outside\ } keys
of the margin dictionary. The \texttt{\ inside\ } margin points towards
the spine, and the \texttt{\ outside\ } margin points towards the edge
of the bound book.

\begin{verbatim}
#set page(margin: (inside: 2.5cm, outside: 2cm, y: 1.75cm))
\end{verbatim}

Typst will assume that documents written in Left-to-Right scripts are
bound on the left while books written in Right-to-Left scripts are bound
on the right. However, you will need to change this in some cases: If
your first page is output by a different app, the binding is reversed
from Typst\textquotesingle s perspective. Also, some books, like
English-language Mangas are customarily bound on the right, despite
English using Left-to-Right script. To change the binding side and
explicitly set where the \texttt{\ inside\ } and \texttt{\ outside\ }
are, set the
\href{/docs/reference/layout/page/\#parameters-binding}{\texttt{\ binding\ }}
argument in the \href{/docs/reference/layout/page/}{\texttt{\ page\ }}
set rule.

\begin{verbatim}
// Produce a book bound on the right,
// even though it is set in Spanish.
#set text(lang: "es")
#set page(binding: right)
\end{verbatim}

If \texttt{\ binding\ } is \texttt{\ left\ } , \texttt{\ inside\ }
margins will be on the left on odd pages, and vice versa.

\subsection{Add headers and footers}\label{headers-and-footers}

Headers and footers are inserted in the top and bottom margins of every
page. You can add custom headers and footers or just insert a page
number.

In case you need more than just a page number, the best way to insert a
header and a footer are the
\href{/docs/reference/layout/page/\#parameters-header}{\texttt{\ header\ }}
and
\href{/docs/reference/layout/page/\#parameters-footer}{\texttt{\ footer\ }}
arguments of the \href{/docs/reference/layout/page/}{\texttt{\ page\ }}
set rule. You can pass any content as their values:

\begin{verbatim}
#set page(header: [
  _Lisa Strassner's Thesis_
  #h(1fr)
  National Academy of Sciences
])

#lorem(150)
\end{verbatim}

\includegraphics[width=8.73958in,height=\textheight,keepaspectratio]{/assets/docs/8IFKrlz4CTSpYgWgKq02tAAAAAAAAAAA.png}

Headers are bottom-aligned by default so that they do not collide with
the top edge of the page. You can change this by wrapping your header in
the \href{/docs/reference/layout/align/}{\texttt{\ align\ }} function.

\subsubsection{Different header and footer on specific
pages}\label{specific-pages}

You\textquotesingle ll need different headers and footers on some pages.
For example, you may not want a header and footer on the title page. The
example below shows how to conditionally remove the header on the first
page:

\begin{verbatim}
#set page(header: context {
  if counter(page).get().first() > 1 [
    _Lisa Strassner's Thesis_
    #h(1fr)
    National Academy of Sciences
  ]
})

#lorem(150)
\end{verbatim}

This example may look intimidating, but let\textquotesingle s break it
down: By using the \texttt{\ }{\texttt{\ context\ }}\texttt{\ } keyword,
we are telling Typst that the header depends on where we are in the
document. We then ask Typst if the page
\href{/docs/reference/introspection/counter/}{counter} is larger than
one at our (context-dependent) current position. The page counter starts
at one, so we are skipping the header on a single page. Counters may
have multiple levels. This feature is used for items like headings, but
the page counter will always have a single level, so we can just look at
the first one.

You can, of course, add an \texttt{\ else\ } to this example to add a
different header to the first page instead.

\subsubsection{Adapt headers and footers on pages with specific
elements}\label{specific-elements}

The technique described in the previous section can be adapted to
perform more advanced tasks using Typst\textquotesingle s labels. For
example, pages with big tables could omit their headers to help keep
clutter down. We will mark our tables with a
\texttt{\ \textless{}big-table\textgreater{}\ }
\href{/docs/reference/foundations/label/}{label} and use the
\href{/docs/reference/introspection/query/}{query system} to find out if
such a label exists on the current page:

\begin{verbatim}
#set page(header: context {
  let page-counter =
  let matches = query(<big-table>)
  let current = counter(page).get()
  let has-table = matches.any(m =>
    counter(page).at(m.location()) == current
  )

  if not has-table [
    _Lisa Strassner's Thesis_
    #h(1fr)
    National Academy of Sciences
  ]
}))

#lorem(100)
#pagebreak()

#table(
  columns: 2 * (1fr,),
  [A], [B],
  [C], [D],
) <big-table>
\end{verbatim}

Here, we query for all instances of the
\texttt{\ \textless{}big-table\textgreater{}\ } label. We then check
that none of the tables are on the page at our current position. If so,
we print the header. This example also uses variables to be more
concise. Just as above, you could add an \texttt{\ else\ } to add
another header instead of deleting it.

\subsection{Add and customize page numbers}\label{page-numbers}

Page numbers help readers keep track of and reference your document more
easily. The simplest way to insert page numbers is the
\href{/docs/reference/layout/page/\#parameters-numbering}{\texttt{\ numbering\ }}
argument of the \href{/docs/reference/layout/page/}{\texttt{\ page\ }}
set rule. You can pass a
\href{/docs/reference/model/numbering/\#parameters-numbering}{\emph{numbering
pattern}} string that shows how you want your pages to be numbered.

\begin{verbatim}
#set page(numbering: "1")

This is a numbered page.
\end{verbatim}

\includegraphics[width=7.38542in,height=\textheight,keepaspectratio]{/assets/docs/tItTkMk79gxARzJosrxVsgAAAAAAAAAA.png}

Above, you can check out the simplest conceivable example. It adds a
single Arabic page number at the center of the footer. You can specify
other characters than \texttt{\ "1"\ } to get other numerals. For
example, \texttt{\ "i"\ } will yield lowercase Roman numerals. Any
character that is not interpreted as a number will be output as-is. For
example, put dashes around your page number by typing this:

\begin{verbatim}
#set page(numbering: "— 1 —")

This is a — numbered — page.
\end{verbatim}

\includegraphics[width=7.38542in,height=\textheight,keepaspectratio]{/assets/docs/1_R2fSOS46hX8DevWO3SHwAAAAAAAAAA.png}

You can add the total number of pages by entering a second number
character in the string.

\begin{verbatim}
#set page(numbering: "1 of 1")

This is one of many numbered pages.
\end{verbatim}

\includegraphics[width=7.38542in,height=\textheight,keepaspectratio]{/assets/docs/y7IZgFjQvtyaq__YsS98MAAAAAAAAAAA.png}

Go to the
\href{/docs/reference/model/numbering/\#parameters-numbering}{\texttt{\ numbering\ }
function reference} to learn more about the arguments you can pass here.

In case you need to right- or left-align the page number, use the
\href{/docs/reference/layout/page/\#parameters-number-align}{\texttt{\ number-align\ }}
argument of the \href{/docs/reference/layout/page/}{\texttt{\ page\ }}
set rule. Alternating alignment between even and odd pages is not
currently supported using this property. To do this,
you\textquotesingle ll need to specify a custom footer with your
footnote and query the page counter as described in the section on
conditionally omitting headers and footers.

\subsubsection{Custom footer with page
numbers}\label{custom-footer-with-page-numbers}

Sometimes, you need to add other content than a page number to your
footer. However, once a footer is specified, the
\href{/docs/reference/layout/page/\#parameters-numbering}{\texttt{\ numbering\ }}
argument of the \href{/docs/reference/layout/page/}{\texttt{\ page\ }}
set rule is ignored. This section shows you how to add a custom footer
with page numbers and more.

\begin{verbatim}
#set page(footer: context [
  *American Society of Proceedings*
  #h(1fr)
  #counter(page).display(
    "1/1",
    both: true,
  )
])

This page has a custom footer.
\end{verbatim}

\includegraphics[width=7.38542in,height=\textheight,keepaspectratio]{/assets/docs/pa4FfkSAmZ8SbHMJhFhITAAAAAAAAAAA.png}

First, we add some strongly emphasized text on the left and add free
space to fill the line. Then, we call \texttt{\ counter(page)\ } to
retrieve the page counter and use its \texttt{\ display\ } function to
show its current value. We also set \texttt{\ both\ } to
\texttt{\ }{\texttt{\ true\ }}\texttt{\ } so that our numbering pattern
applies to the current \emph{and} final page number.

We can also get more creative with the page number. For example,
let\textquotesingle s insert a circle for each page.

\begin{verbatim}
#set page(footer: context [
  *Fun Typography Club*
  #h(1fr)
  #let (num,) = counter(page).get()
  #let circles = num * (
    box(circle(
      radius: 2pt,
      fill: navy,
    )),
  )
  #box(
    inset: (bottom: 1pt),
    circles.join(h(1pt))
  )
])

This page has a custom footer.
\end{verbatim}

\includegraphics[width=7.38542in,height=\textheight,keepaspectratio]{/assets/docs/Px4MgFDGZh5TfTBwUSS_KAAAAAAAAAAA.png}

In this example, we use the number of pages to create an array of
\href{/docs/reference/visualize/circle/}{circles} . The circles are
wrapped in a \href{/docs/reference/layout/box/}{box} so they can all
appear on the same line because they are blocks and would otherwise
create paragraph breaks. The length of this
\href{/docs/reference/foundations/array/}{array} depends on the current
page number.

We then insert the circles at the right side of the footer, with 1pt of
space between them. The join method of an array will attempt to
\href{/docs/reference/scripting/\#blocks}{\emph{join}} the different
values of an array into a single value, interspersed with its argument.
In our case, we get a single content value with circles and spaces
between them that we can use with the align function. Finally, we use
another box to ensure that the text and the circles can share a line and
use the
\href{/docs/reference/layout/box/\#parameters-inset}{\texttt{\ inset\ }
argument} to raise the circles a bit so they line up nicely with the
text.

\subsubsection{Reset the page number and skip pages}\label{skip-pages}

Do you, at some point in your document, need to reset the page number?
Maybe you want to start with the first page only after the title page.
Or maybe you need to skip a few page numbers because you will insert
pages into the final printed product.

The right way to modify the page number is to manipulate the page
\href{/docs/reference/introspection/counter/}{counter} . The simplest
manipulation is to set the counter back to 1.

\begin{verbatim}
#counter(page).update(1)
\end{verbatim}

This line will reset the page counter back to one. It should be placed
at the start of a page because it will otherwise create a page break.
You can also update the counter given its previous value by passing a
function:

\begin{verbatim}
#counter(page).update(n => n + 5)
\end{verbatim}

In this example, we skip five pages. \texttt{\ n\ } is the current value
of the page counter and \texttt{\ n\ +\ 5\ } is the return value of our
function.

In case you need to retrieve the actual page number instead of the value
of the page counter, you can use the
\href{/docs/reference/introspection/location/\#definitions-page}{\texttt{\ page\ }}
method on the return value of the
\href{/docs/reference/introspection/here/}{\texttt{\ here\ }} function:

\begin{verbatim}
#counter(page).update(n => n + 5)

// This returns one even though the
// page counter was incremented by 5.
#context here().page()
\end{verbatim}

\includegraphics[width=5in,height=\textheight,keepaspectratio]{/assets/docs/09ytRFFbm_ZOLxjka15n_QAAAAAAAAAA.png}

You can also obtain the page numbering pattern from the location
returned by \texttt{\ here\ } with the
\href{/docs/reference/introspection/location/\#definitions-page-numbering}{\texttt{\ page-numbering\ }}
method.

\subsection{Add columns}\label{columns}

Add columns to your document to fit more on a page while maintaining
legible line lengths. Columns are vertical blocks of text which are
separated by some whitespace. This space is called the gutter.

To lay out your content in columns, just specify the desired number of
columns in a
\href{/docs/reference/layout/page/\#parameters-columns}{\texttt{\ page\ }}
set rule. To adjust the amount of space between the columns, add a set
rule on the \href{/docs/reference/layout/columns/}{\texttt{\ columns\ }
function} , specifying the \texttt{\ gutter\ } parameter.

\begin{verbatim}
#set page(columns: 2)
#set columns(gutter: 12pt)

#lorem(30)
\end{verbatim}

\includegraphics[width=5in,height=\textheight,keepaspectratio]{/assets/docs/FgQ5mR5BdbOwdaPvmQss-wAAAAAAAAAA.png}

Very commonly, scientific papers have a single-column title and
abstract, while the main body is set in two-columns. To achieve this
effect, Typst\textquotesingle s
\href{/docs/reference/layout/place/}{\texttt{\ place\ } function} can
temporarily escape the two-column layout by specifying
\texttt{\ float:\ }{\texttt{\ true\ }}\texttt{\ } and
\texttt{\ scope:\ }{\texttt{\ "parent"\ }}\texttt{\ } :

\begin{verbatim}
#set page(columns: 2)
#set par(justify: true)

#place(
  top + center,
  float: true,
  scope: "parent",
  text(1.4em, weight: "bold")[
    Impacts of Odobenidae
  ],
)

== About seals in the wild
#lorem(80)
\end{verbatim}

\includegraphics[width=5in,height=\textheight,keepaspectratio]{/assets/docs/rwgbgpUV52923GFV_9bCEQAAAAAAAAAA.png}

\emph{Floating placement} refers to elements being pushed to the top or
bottom of the column or page, with the remaining content flowing in
between. It is also frequently used for
\href{/docs/reference/model/figure/\#parameters-placement}{figures} .

\subsubsection{Use columns anywhere in your
document}\label{columns-anywhere}

To create columns within a nested layout, e.g. within a rectangle, you
can use the \href{/docs/reference/layout/columns/}{\texttt{\ columns\ }
function} directly. However, it really should only be used within nested
layouts. At the page-level, the page set rule is preferrable because it
has better interactions with things like page-level floats, footnotes,
and line numbers.

\begin{verbatim}
#rect(
  width: 6cm,
  height: 3.5cm,
  columns(2, gutter: 12pt)[
    In the dimly lit gas station,
    a solitary taxi stood silently,
    its yellow paint fading with
    time. Its windows were dark,
    its engine idle, and its tires
    rested on the cold concrete.
  ]
)
\end{verbatim}

\includegraphics[width=5in,height=\textheight,keepaspectratio]{/assets/docs/4ald2xJRDpaFMxVXu7aiUQAAAAAAAAAA.png}

\subsubsection{Balanced columns}\label{balanced-columns}

If the columns on the last page of a document differ greatly in length,
they may create a lopsided and unappealing layout.
That\textquotesingle s why typographers will often equalize the length
of columns on the last page. This effect is called balancing columns.
Typst cannot yet balance columns automatically. However, you can balance
columns manually by placing
\href{/docs/reference/layout/colbreak/}{\texttt{\ }{\texttt{\ \#\ }}\texttt{\ }{\texttt{\ colbreak\ }}\texttt{\ }{\texttt{\ (\ }}\texttt{\ }{\texttt{\ )\ }}\texttt{\ }}
at an appropriate spot in your markup, creating the desired column break
manually.

\subsection{One-off modifications}\label{one-off-modifications}

You do not need to override your page settings if you need to insert a
single page with a different setup. For example, you may want to insert
a page that\textquotesingle s flipped to landscape to insert a big table
or change the margin and columns for your title page. In this case, you
can call \href{/docs/reference/layout/page/}{\texttt{\ page\ }} as a
function with your content as an argument and the overrides as the other
arguments. This will insert enough new pages with your overridden
settings to place your content on them. Typst will revert to the page
settings from the set rule after the call.

\begin{verbatim}
#page(flipped: true)[
  = Multiplication table

  #table(
    columns: 5 * (1fr,),
    ..for x in range(1, 10) {
      for y in range(1, 6) {
        (str(x*y),)
      }
    }
  )
]
\end{verbatim}

\includegraphics[width=8.73958in,height=\textheight,keepaspectratio]{/assets/docs/U5FByA07ZCcGdizR_XbtEwAAAAAAAAAA.png}

\href{/docs/guides/guide-for-latex-users/}{\pandocbounded{\includesvg[keepaspectratio]{/assets/icons/16-arrow-right.svg}}}

{ Guide for LaTeX users } { Previous page }

\href{/docs/guides/table-guide/}{\pandocbounded{\includesvg[keepaspectratio]{/assets/icons/16-arrow-right.svg}}}

{ Table guide } { Next page }


\section{Docs LaTeX/typst.app/docs/guides/guide-for-latex-users.tex}
\title{typst.app/docs/guides/guide-for-latex-users}

\begin{itemize}
\tightlist
\item
  \href{/docs}{\includesvg[width=0.16667in,height=0.16667in]{/assets/icons/16-docs-dark.svg}}
\item
  \includesvg[width=0.16667in,height=0.16667in]{/assets/icons/16-arrow-right.svg}
\item
  \href{/docs/guides/}{Guides}
\item
  \includesvg[width=0.16667in,height=0.16667in]{/assets/icons/16-arrow-right.svg}
\item
  \href{/docs/guides/guide-for-latex-users/}{Guide for LaTeX users}
\end{itemize}

\section{Guide for LaTeX users}\label{guide-for-latex-users}

This page is a good starting point if you have used LaTeX before and
want to try out Typst. We will explore the main differences between
these two systems from a user perspective. Although Typst is not built
upon LaTeX and has a different syntax, you will learn how to use your
LaTeX skills to get a head start.

Just like LaTeX, Typst is a markup-based typesetting system: You compose
your document in a text file and mark it up with commands and other
syntax. Then, you use a compiler to typeset the source file into a PDF.
However, Typst also differs from LaTeX in several aspects: For one,
Typst uses more dedicated syntax (like you may know from Markdown) for
common tasks. Typst\textquotesingle s commands are also more principled:
They all work the same, so unlike in LaTeX, you just need to understand
a few general concepts instead of learning different conventions for
each package. Moreover Typst compiles faster than LaTeX: Compilation
usually takes milliseconds, not seconds, so the web app and the compiler
can both provide instant previews.

In the following, we will cover some of the most common questions a user
switching from LaTeX will have when composing a document in Typst. If
you prefer a step-by-step introduction to Typst, check out our
\href{/docs/tutorial/}{tutorial} .

\subsection{Installation}\label{installation}

You have two ways to use Typst: In \href{https://typst.app/signup/}{our
web app} or by \href{https://github.com/typst/typst/releases}{installing
the compiler} on your computer. When you use the web app, we provide a
batteries-included collaborative editor and run Typst in your browser,
no installation required.

If you choose to use Typst on your computer instead, you can download
the compiler as a single, small binary which any user can run, no root
privileges required. Unlike LaTeX, packages are downloaded when you
first use them and then cached locally, keeping your Typst installation
lean. You can use your own editor and decide where to store your files
with the local compiler.

\subsection{How do I create a new, empty
document?}\label{getting-started}

That\textquotesingle s easy. You just create a new, empty text file (the
file extension is \texttt{\ .typ\ } ). No boilerplate is needed to get
started. Simply start by writing your text. It will be set on an empty
A4-sized page. If you are using the web app, click "+ Empty document" to
create a new project with a file and enter the editor.
\href{/docs/reference/model/parbreak/}{Paragraph breaks} work just as
they do in LaTeX, just use a blank line.

\begin{verbatim}
Hey there!

Here are two paragraphs. The
output is shown to the right.
\end{verbatim}

\includegraphics[width=5in,height=\textheight,keepaspectratio]{/assets/docs/1Xaxf-Fz-W7agcXqKmLyjQAAAAAAAAAA.png}

If you want to start from an preexisting LaTeX document instead, you can
use \href{https://pandoc.org}{Pandoc} to convert your source code to
Typst markup. This conversion is also built into our web app, so you can
upload your \texttt{\ .tex\ } file to start your project in Typst.

\subsection{How do I create section headings, emphasis,
...?}\label{elements}

LaTeX uses the command \texttt{\ \textbackslash{}section\ } to create a
section heading. Nested headings are indicated with
\texttt{\ \textbackslash{}subsection\ } ,
\texttt{\ \textbackslash{}subsubsection\ } , etc. Depending on your
document class, there is also \texttt{\ \textbackslash{}part\ } or
\texttt{\ \textbackslash{}chapter\ } .

In Typst, \href{/docs/reference/model/heading/}{headings} are less
verbose: You prefix the line with the heading on it with an equals sign
and a space to get a first-order heading:
\texttt{\ }{\texttt{\ =\ Introduction\ }}\texttt{\ } . If you need a
second-order heading, you use two equals signs:
\texttt{\ }{\texttt{\ ==\ In\ this\ paper\ }}\texttt{\ } . You can nest
headings as deeply as you\textquotesingle d like by adding more equals
signs.

Emphasis (usually rendered as italic text) is expressed by enclosing
text in \texttt{\ }{\texttt{\ \_underscores\_\ }}\texttt{\ } and strong
emphasis (usually rendered in boldface) by using
\texttt{\ }{\texttt{\ *stars*\ }}\texttt{\ } instead.

Here is a list of common markup commands used in LaTeX and their Typst
equivalents. You can also check out the
\href{/docs/reference/syntax/}{full syntax cheat sheet} .

\begin{longtable}[]{@{}llll@{}}
\toprule\noalign{}
Element & LaTeX & Typst & See \\
\midrule\noalign{}
\endhead
\bottomrule\noalign{}
\endlastfoot
Strong emphasis & \texttt{\ \textbackslash{}textbf\{strong\}\ } &
\texttt{\ }{\texttt{\ *strong*\ }}\texttt{\ } &
\href{/docs/reference/model/strong/}{\texttt{\ strong\ }} \\
Emphasis & \texttt{\ \textbackslash{}emph\{emphasis\}\ } &
\texttt{\ }{\texttt{\ \_emphasis\_\ }}\texttt{\ } &
\href{/docs/reference/model/emph/}{\texttt{\ emph\ }} \\
Monospace / code & \texttt{\ \textbackslash{}texttt\{print(1)\}\ } &
\texttt{\ }{\texttt{\ \textasciigrave{}print(1)\textasciigrave{}\ }}\texttt{\ }
& \href{/docs/reference/text/raw/}{\texttt{\ raw\ }} \\
Link & \texttt{\ \textbackslash{}url\{https://typst.app\}\ } &
\texttt{\ }{\texttt{\ https://typst.app/\ }}\texttt{\ } &
\href{/docs/reference/model/link/}{\texttt{\ link\ }} \\
Label & \texttt{\ \textbackslash{}label\{intro\}\ } &
\texttt{\ }{\texttt{\ \textless{}intro\textgreater{}\ }}\texttt{\ } &
\href{/docs/reference/foundations/label/}{\texttt{\ label\ }} \\
Reference & \texttt{\ \textbackslash{}ref\{intro\}\ } &
\texttt{\ }{\texttt{\ @intro\ }}\texttt{\ } &
\href{/docs/reference/model/ref/}{\texttt{\ ref\ }} \\
Citation & \texttt{\ \textbackslash{}cite\{humphrey97\}\ } &
\texttt{\ }{\texttt{\ @humphrey97\ }}\texttt{\ } &
\href{/docs/reference/model/cite/}{\texttt{\ cite\ }} \\
Bullet list & \texttt{\ itemize\ } environment &
\texttt{\ }{\texttt{\ -\ }}\texttt{\ List\ } &
\href{/docs/reference/model/list/}{\texttt{\ list\ }} \\
Numbered list & \texttt{\ enumerate\ } environment &
\texttt{\ }{\texttt{\ +\ }}\texttt{\ List\ } &
\href{/docs/reference/model/enum/}{\texttt{\ enum\ }} \\
Term list & \texttt{\ description\ } environment &
\texttt{\ }{\texttt{\ /\ }}\texttt{\ }{\texttt{\ Term\ }}\texttt{\ }{\texttt{\ :\ }}\texttt{\ List\ }
& \href{/docs/reference/model/terms/}{\texttt{\ terms\ }} \\
Figure & \texttt{\ figure\ } environment & \texttt{\ figure\ } function
& \href{/docs/reference/model/figure/}{\texttt{\ figure\ }} \\
Table & \texttt{\ table\ } environment & \texttt{\ table\ } function &
\href{/docs/reference/model/table/}{\texttt{\ table\ }} \\
Equation & \texttt{\ \$x\$\ } , \texttt{\ align\ } /
\texttt{\ equation\ } environments &
\texttt{\ }{\texttt{\ \$\ }}\texttt{\ x\ }{\texttt{\ \$\ }}\texttt{\ } ,
\texttt{\ }{\texttt{\ \$\ }}\texttt{\ x\ =\ y\ }{\texttt{\ \$\ }}\texttt{\ }
& \href{/docs/reference/math/equation/}{\texttt{\ equation\ }} \\
\end{longtable}

\href{/docs/reference/model/list/}{Lists} do not rely on environments in
Typst. Instead, they have lightweight syntax like headings. To create an
unordered list ( \texttt{\ itemize\ } ), prefix each line of an item
with a hyphen:

\begin{verbatim}
To write this list in Typst...

```latex
\begin{itemize}
  \item Fast
  \item Flexible
  \item Intuitive
\end{itemize}
```

...just type this:

- Fast
- Flexible
- Intuitive
\end{verbatim}

\includegraphics[width=5in,height=\textheight,keepaspectratio]{/assets/docs/EtxJBpmzn-Q3caUem89mOQAAAAAAAAAA.png}

Nesting lists works just by using proper indentation. Adding a blank
line in between items results in a more
\href{/docs/reference/model/list/\#parameters-tight}{widely} spaced
list.

To get a \href{/docs/reference/model/enum/}{numbered list} (
\texttt{\ enumerate\ } ) instead, use a \texttt{\ +\ } instead of the
hyphen. For a \href{/docs/reference/model/terms/}{term list} (
\texttt{\ description\ } ), write
\texttt{\ }{\texttt{\ /\ }}\texttt{\ }{\texttt{\ Term\ }}\texttt{\ }{\texttt{\ :\ }}\texttt{\ Description\ }
instead.

\subsection{How do I use a command?}\label{commands}

LaTeX heavily relies on commands (prefixed by backslashes). It uses
these \emph{macros} to affect the typesetting process and to insert and
manipulate content. Some commands accept arguments, which are most
frequently enclosed in curly braces:
\texttt{\ \textbackslash{}cite\{rasmus\}\ } .

Typst differentiates between
\href{/docs/reference/scripting/\#blocks}{markup mode and code mode} .
The default is markup mode, where you compose text and apply syntactic
constructs such as
\texttt{\ }{\texttt{\ *stars\ for\ bold\ text*\ }}\texttt{\ } . Code
mode, on the other hand, parallels programming languages like Python,
providing the option to input and execute segments of code.

Within Typst\textquotesingle s markup, you can switch to code mode for a
single command (or rather, \emph{expression} ) using a hash (
\texttt{\ \#\ } ). This is how you call functions to, for example, split
your project into different
\href{/docs/reference/scripting/\#modules}{files} or render text based
on some \href{/docs/reference/scripting/\#conditionals}{condition} .
Within code mode, it is possible to include normal markup
\href{/docs/reference/foundations/content/}{\emph{content}} by using
square brackets. Within code mode, this content is treated just as any
other normal value for a variable.

\begin{verbatim}
First, a rectangle:
#rect()

Let me show how to do
#underline([_underlined_ text])

We can also do some maths:
#calc.max(3, 2 * 4)

And finally a little loop:
#for x in range(3) [
  Hi #x.
]
\end{verbatim}

\includegraphics[width=5in,height=\textheight,keepaspectratio]{/assets/docs/YRh-AUkIq4C1mI8DVTggswAAAAAAAAAA.png}

A function call always involves the name of the function (
\href{/docs/reference/visualize/rect/}{\texttt{\ rect\ }} ,
\href{/docs/reference/text/underline/}{\texttt{\ underline\ }} ,
\href{/docs/reference/foundations/calc/\#functions-max}{\texttt{\ calc.max\ }}
,
\href{/docs/reference/foundations/array/\#definitions-range}{\texttt{\ range\ }}
) followed by parentheses (as opposed to LaTeX where the square brackets
and curly braces are optional if the macro requires no arguments). The
expected list of arguments passed within those parentheses depends on
the concrete function and is specified in the
\href{/docs/reference/}{reference} .

\subsubsection{Arguments}\label{arguments}

A function can have multiple arguments. Some arguments are positional,
i.e., you just provide the value: The function
\texttt{\ }{\texttt{\ \#\ }}\texttt{\ }{\texttt{\ lower\ }}\texttt{\ }{\texttt{\ (\ }}\texttt{\ }{\texttt{\ "SCREAM"\ }}\texttt{\ }{\texttt{\ )\ }}\texttt{\ }
returns its argument in all-lowercase. Many functions use named
arguments instead of positional arguments to increase legibility. For
example, the dimensions and stroke of a rectangle are defined with named
arguments:

\begin{verbatim}
#rect(
  width: 2cm,
  height: 1cm,
  stroke: red,
)
\end{verbatim}

\includegraphics[width=5in,height=\textheight,keepaspectratio]{/assets/docs/qhelTU2eEzhkL0zp__ciIAAAAAAAAAAA.png}

You specify a named argument by first entering its name (above,
it\textquotesingle s \texttt{\ width\ } , \texttt{\ height\ } , and
\texttt{\ stroke\ } ), then a colon, followed by the value (
\texttt{\ 2cm\ } , \texttt{\ 1cm\ } , \texttt{\ red\ } ). You can find
the available named arguments in the \href{/docs/reference/}{reference
page} for each function or in the autocomplete panel when typing. Named
arguments are similar to how some LaTeX environments are configured, for
example, you would type
\texttt{\ \textbackslash{}begin\{enumerate\}{[}label=\{\textbackslash{}alph*)\}{]}\ }
to start a list with the labels \texttt{\ a)\ } , \texttt{\ b)\ } , and
so on.

Often, you want to provide some
\href{/docs/reference/foundations/content/}{content} to a function. For
example, the LaTeX command
\texttt{\ \textbackslash{}underline\{Alternative\ A\}\ } would translate
to
\texttt{\ }{\texttt{\ \#\ }}\texttt{\ }{\texttt{\ underline\ }}\texttt{\ }{\texttt{\ (\ }}\texttt{\ }{\texttt{\ {[}\ }}\texttt{\ Alternative\ A\ }{\texttt{\ {]}\ }}\texttt{\ }{\texttt{\ )\ }}\texttt{\ }
in Typst. The square brackets indicate that a value is
\href{/docs/reference/foundations/content/}{content} . Within these
brackets, you can use normal markup. However, that\textquotesingle s a
lot of parentheses for a pretty simple construct. This is why you can
also move trailing content arguments after the parentheses (and omit the
parentheses if they would end up empty).

\begin{verbatim}
Typst is an #underline[alternative]
to LaTeX.

#rect(fill: aqua)[Get started here!]
\end{verbatim}

\includegraphics[width=5in,height=\textheight,keepaspectratio]{/assets/docs/mXv-36m3C2iLosBLIxMjHwAAAAAAAAAA.png}

\subsubsection{Data types}\label{data-types}

You likely already noticed that the arguments have distinctive data
types. Typst supports many \href{/docs/reference/foundations/type/}{data
types} . Below, there is a table with some of the most important ones
and how to write them. In order to specify values of any of these types,
you have to be in code mode!

\begin{longtable}[]{@{}ll@{}}
\toprule\noalign{}
Data type & Example \\
\midrule\noalign{}
\endhead
\bottomrule\noalign{}
\endlastfoot
\href{/docs/reference/foundations/content/}{Content} &
\texttt{\ }{\texttt{\ {[}\ }}\texttt{\ }{\texttt{\ *fast*\ }}\texttt{\ typesetting\ }{\texttt{\ {]}\ }}\texttt{\ } \\
\href{/docs/reference/foundations/str/}{String} &
\texttt{\ }{\texttt{\ "Pietro\ S.\ Author"\ }}\texttt{\ } \\
\href{/docs/reference/foundations/int/}{Integer} &
\texttt{\ }{\texttt{\ 23\ }}\texttt{\ } \\
\href{/docs/reference/foundations/float/}{Floating point number} &
\texttt{\ }{\texttt{\ 1.459\ }}\texttt{\ } \\
\href{/docs/reference/layout/length/}{Absolute length} &
\texttt{\ }{\texttt{\ 12pt\ }}\texttt{\ } ,
\texttt{\ }{\texttt{\ 5in\ }}\texttt{\ } ,
\texttt{\ }{\texttt{\ 0.3cm\ }}\texttt{\ } , ... \\
\href{/docs/reference/layout/ratio/}{Relative length} &
\texttt{\ }{\texttt{\ 65\%\ }}\texttt{\ } \\
\end{longtable}

The difference between content and string is that content can contain
markup, including function calls, while a string really is just a plain
sequence of characters.

Typst provides \href{/docs/reference/scripting/\#conditionals}{control
flow constructs} and
\href{/docs/reference/scripting/\#operators}{operators} such as
\texttt{\ +\ } for adding things or \texttt{\ ==\ } for checking
equality between two variables.

You can also store values, including functions, in your own
\href{/docs/reference/scripting/\#bindings}{variables} . This can be
useful to perform computations on them, create reusable automations, or
reference a value multiple times. The variable binding is accomplished
with the let keyword, which works similar to
\texttt{\ \textbackslash{}newcommand\ } :

\begin{verbatim}
// Store the integer `5`.
#let five = 5

// Define a function that
// increments a value.
#let inc(i) = i + 1

// Reference the variables.
I have #five fingers.

If I had one more, I'd have
#inc(five) fingers. Whoa!
\end{verbatim}

\includegraphics[width=5in,height=\textheight,keepaspectratio]{/assets/docs/B033fxNtvCwYya9MX4R6xwAAAAAAAAAA.png}

\subsubsection{Commands to affect the remaining document}\label{rules}

In LaTeX, some commands like
\texttt{\ \textbackslash{}textbf\{bold\ text\}\ } receive an argument in
curly braces and only affect that argument. Other commands such as
\texttt{\ \textbackslash{}bfseries\ bold\ text\ } act as switches (LaTeX
calls this a declaration), altering the appearance of all subsequent
content within the document or current scope.

In Typst, the same function can be used both to affect the appearance
for the remainder of the document, a block (or scope), or just its
arguments. For example,
\texttt{\ }{\texttt{\ \#\ }}\texttt{\ }{\texttt{\ text\ }}\texttt{\ }{\texttt{\ (\ }}\texttt{\ weight\ }{\texttt{\ :\ }}\texttt{\ }{\texttt{\ "bold"\ }}\texttt{\ }{\texttt{\ )\ }}\texttt{\ }{\texttt{\ {[}\ }}\texttt{\ bold\ text\ }{\texttt{\ {]}\ }}\texttt{\ }
will only embolden its argument, while
\texttt{\ }{\texttt{\ \#\ }}\texttt{\ }{\texttt{\ set\ }}\texttt{\ }{\texttt{\ text\ }}\texttt{\ }{\texttt{\ (\ }}\texttt{\ weight\ }{\texttt{\ :\ }}\texttt{\ }{\texttt{\ "bold"\ }}\texttt{\ }{\texttt{\ )\ }}\texttt{\ }
will embolden any text until the end of the current block, or, if there
is none, document. The effects of a function are immediately obvious
based on whether it is used in a call or a
\href{/docs/reference/styling/\#set-rules}{set rule.}

\begin{verbatim}
I am starting out with small text.

#set text(14pt)

This is a bit #text(18pt)[larger,]
don't you think?
\end{verbatim}

\includegraphics[width=5in,height=\textheight,keepaspectratio]{/assets/docs/aX-wYquEk7ekHdAWcGZ3-wAAAAAAAAAA.png}

Set rules may appear anywhere in the document. They can be thought of as
default argument values of their respective function:

\begin{verbatim}
#set enum(numbering: "I.")

Good results can only be obtained by
+ following best practices
+ being aware of current results
  of other researchers
+ checking the data for biases
\end{verbatim}

\includegraphics[width=5in,height=\textheight,keepaspectratio]{/assets/docs/FTzkApPthDlHdofpWCjnfwAAAAAAAAAA.png}

The \texttt{\ +\ } is syntactic sugar (think of it as an abbreviation)
for a call to the \href{/docs/reference/model/enum/}{\texttt{\ enum\ }}
function, to which we apply a set rule above.
\href{/docs/reference/syntax/}{Most syntax is linked to a function in
this way.} If you need to style an element beyond what its arguments
enable, you can completely redefine its appearance with a
\href{/docs/reference/styling/\#show-rules}{show rule} (somewhat
comparable to \texttt{\ \textbackslash{}renewcommand\ } ).

You can achieve the effects of LaTeX commands like
\texttt{\ \textbackslash{}textbf\ } ,
\texttt{\ \textbackslash{}textsf\ } ,
\texttt{\ \textbackslash{}rmfamily\ } ,
\texttt{\ \textbackslash{}mdseries\ } , and
\texttt{\ \textbackslash{}itshape\ } with the
\href{/docs/reference/text/text/\#parameters-font}{\texttt{\ font\ }} ,
\href{/docs/reference/text/text/\#parameters-style}{\texttt{\ style\ }}
, and
\href{/docs/reference/text/text/\#parameters-weight}{\texttt{\ weight\ }}
arguments of the \texttt{\ text\ } function. The text function can be
used in a set rule (declaration style) or with a content argument. To
replace \texttt{\ \textbackslash{}textsc\ } , you can use the
\href{/docs/reference/text/smallcaps/}{\texttt{\ smallcaps\ }} function,
which renders its content argument as smallcaps. Should you want to use
it declaration style (like \texttt{\ \textbackslash{}scshape\ } ), you
can use an \href{/docs/reference/styling/\#show-rules}{\emph{everything}
show rule} that applies the function to the rest of the scope:

\begin{verbatim}
#show: smallcaps

Boisterous Accusations
\end{verbatim}

\includegraphics[width=5in,height=\textheight,keepaspectratio]{/assets/docs/uKURZLdKdC2JssMYxVk2sQAAAAAAAAAA.png}

\subsection{How do I load a document class?}\label{templates}

In LaTeX, you start your main \texttt{\ .tex\ } file with the
\texttt{\ \textbackslash{}documentclass\{article\}\ } command to define
how your document is supposed to look. In that command, you may have
replaced \texttt{\ article\ } with another value such as
\texttt{\ report\ } and \texttt{\ amsart\ } to select a different look.

When using Typst, you style your documents with
\href{/docs/reference/foundations/function/}{functions} . Typically, you
use a template that provides a function that styles your whole document.
First, you import the function from a template file. Then, you apply it
to your whole document. This is accomplished with a
\href{/docs/reference/styling/\#show-rules}{show rule} that wraps the
following document in a given function. The following example
illustrates how it works:

\begin{verbatim}
#import "conf.typ": conf
#show: conf.with(
  title: [
    Towards Improved Modelling
  ],
  authors: (
    (
      name: "Theresa Tungsten",
      affiliation: "Artos Institute",
      email: "tung@artos.edu",
    ),
    (
      name: "Eugene Deklan",
      affiliation: "Honduras State",
      email: "e.deklan@hstate.hn",
    ),
  ),
  abstract: lorem(80),
)

Let's get started writing this
article by putting insightful
paragraphs right here!
\end{verbatim}

\includegraphics[width=12.75in,height=\textheight,keepaspectratio]{/assets/docs/k47Rrbrcbzmkzl-uxGd1cAAAAAAAAAAA.png}

The
\href{/docs/reference/scripting/\#modules}{\texttt{\ }{\texttt{\ import\ }}\texttt{\ }}
statement makes \href{/docs/reference/foundations/function/}{functions}
(and other definitions) from another file available. In this example, it
imports the \texttt{\ conf\ } function from the \texttt{\ conf.typ\ }
file. This function formats a document as a conference article. We use a
show rule to apply it to the document and also configure some metadata
of the article. After applying the show rule, we can start writing our
article right away!

You can also use templates from Typst Universe (which is
Typst\textquotesingle s equivalent of CTAN) using an import statement
like this:
\texttt{\ }{\texttt{\ \#\ }}\texttt{\ }{\texttt{\ import\ }}\texttt{\ }{\texttt{\ "@preview/elsearticle:0.2.1"\ }}\texttt{\ }{\texttt{\ :\ }}\texttt{\ elsearticle\ }
. Check the documentation of an individual template to learn the name of
its template function. Templates and packages from Typst Universe are
automatically downloaded when you first use them.

In the web app, you can choose to create a project from a template on
Typst Universe or even create your own using the template wizard.
Locally, you can use the \texttt{\ typst\ init\ } CLI to create a new
project from a template. Check out
\href{https://typst.app/universe/search/?kind=templates}{the list of
templates} published on Typst Universe. You can also take a look at the
\href{https://github.com/qjcg/awesome-typst}{\texttt{\ awesome-typst\ }
repository} to find community templates that aren\textquotesingle t
available through Universe.

You can also \href{/docs/tutorial/making-a-template/}{create your own,
custom templates.} They are shorter and more readable than the
corresponding LaTeX \texttt{\ .sty\ } files by orders of magnitude, so
give it a try!

Functions are Typst\textquotesingle s "commands" and can transform their
arguments to an output value, including document \emph{content.}
Functions are "pure", which means that they cannot have any effects
beyond creating an output value / output content. This is in stark
contrast to LaTeX macros that can have arbitrary effects on your
document.

To let a function style your whole document, the show rule processes
everything that comes after it and calls the function specified after
the colon with the result as an argument. The \texttt{\ .with\ } part is
a \emph{method} that takes the \texttt{\ conf\ } function and
pre-configures some if its arguments before passing it on to the show
rule.

\subsection{How do I load packages?}\label{packages}

Typst is "batteries included," so the equivalent of many popular LaTeX
packages is built right-in. Below, we compiled a table with frequently
loaded packages and their corresponding Typst functions.

\begin{longtable}[]{@{}ll@{}}
\toprule\noalign{}
LaTeX Package & Typst Alternative \\
\midrule\noalign{}
\endhead
\bottomrule\noalign{}
\endlastfoot
graphicx, svg &
\href{/docs/reference/visualize/image/}{\texttt{\ image\ }} function \\
tabularx & \href{/docs/reference/model/table/}{\texttt{\ table\ }} ,
\href{/docs/reference/layout/grid/}{\texttt{\ grid\ }} functions \\
fontenc, inputenc, unicode-math & Just start writing! \\
babel, polyglossia &
\href{/docs/reference/text/text/\#parameters-lang}{\texttt{\ text\ }}
function:
\texttt{\ }{\texttt{\ \#\ }}\texttt{\ }{\texttt{\ set\ }}\texttt{\ }{\texttt{\ text\ }}\texttt{\ }{\texttt{\ (\ }}\texttt{\ lang\ }{\texttt{\ :\ }}\texttt{\ }{\texttt{\ "zh"\ }}\texttt{\ }{\texttt{\ )\ }}\texttt{\ } \\
amsmath & \href{/docs/reference/math/}{Math mode} \\
amsfonts, amssymb & \href{/docs/reference/symbols/}{\texttt{\ sym\ }}
module and \href{/docs/reference/syntax/\#math}{syntax} \\
geometry, fancyhdr &
\href{/docs/reference/layout/page/}{\texttt{\ page\ }} function \\
xcolor &
\href{/docs/reference/text/text/\#parameters-fill}{\texttt{\ text\ }}
function:
\texttt{\ }{\texttt{\ \#\ }}\texttt{\ }{\texttt{\ set\ }}\texttt{\ }{\texttt{\ text\ }}\texttt{\ }{\texttt{\ (\ }}\texttt{\ fill\ }{\texttt{\ :\ }}\texttt{\ }{\texttt{\ rgb\ }}\texttt{\ }{\texttt{\ (\ }}\texttt{\ }{\texttt{\ "\#0178A4"\ }}\texttt{\ }{\texttt{\ )\ }}\texttt{\ }{\texttt{\ )\ }}\texttt{\ } \\
hyperref & \href{/docs/reference/model/link/}{\texttt{\ link\ }}
function \\
bibtex, biblatex, natbib &
\href{/docs/reference/model/cite/}{\texttt{\ cite\ }} ,
\href{/docs/reference/model/bibliography/}{\texttt{\ bibliography\ }}
functions \\
lstlisting, minted & \href{/docs/reference/text/raw/}{\texttt{\ raw\ }}
function and syntax \\
parskip &
\href{/docs/reference/layout/block/\#parameters-spacing}{\texttt{\ block\ }}
and
\href{/docs/reference/model/par/\#parameters-first-line-indent}{\texttt{\ par\ }}
functions \\
csquotes & Set the
\href{/docs/reference/text/text/\#parameters-lang}{\texttt{\ text\ }}
language and type \texttt{\ "\ } or \texttt{\ \textquotesingle{}\ } \\
caption & \href{/docs/reference/model/figure/}{\texttt{\ figure\ }}
function \\
enumitem & \href{/docs/reference/model/list/}{\texttt{\ list\ }} ,
\href{/docs/reference/model/enum/}{\texttt{\ enum\ }} ,
\href{/docs/reference/model/terms/}{\texttt{\ terms\ }} functions \\
\end{longtable}

Although \emph{many} things are built-in, not everything can be.
That\textquotesingle s why Typst has its own
\href{https://typst.app/universe/}{package ecosystem} where the
community share its creations and automations. Let\textquotesingle s
take, for instance, the \emph{cetz} package: This package allows you to
create complex drawings and plots. To use cetz in your document, you can
just write:

\begin{verbatim}
#import "@preview/cetz:0.2.1"
\end{verbatim}

(The \texttt{\ @preview\ } is a \emph{namespace} that is used while the
package manager is still in its early and experimental state. It will be
replaced in the future.)

Aside from the official package hub, you might also want to check out
the \href{https://github.com/qjcg/awesome-typst}{awesome-typst
repository} , which compiles a curated list of resources created for
Typst.

If you need to load functions and variables from another file within
your project, for example to use a template, you can use the same
\href{/docs/reference/scripting/\#modules}{\texttt{\ import\ }}
statement with a file name rather than a package specification. To
instead include the textual content of another file, you can use an
\href{/docs/reference/scripting/\#modules}{\texttt{\ include\ }}
statement. It will retrieve the content of the specified file and put it
in your document.

\subsection{How do I input maths?}\label{maths}

To enter math mode in Typst, just enclose your equation in dollar signs.
You can enter display mode by adding spaces or newlines between the
equation\textquotesingle s contents and its enclosing dollar signs.

\begin{verbatim}
The sum of the numbers from
$1$ to $n$ is:

$ sum_(k=1)^n k = (n(n+1))/2 $
\end{verbatim}

\includegraphics[width=5in,height=\textheight,keepaspectratio]{/assets/docs/_cQQzHlyveS6_BoZXnzzQQAAAAAAAAAA.png}

\href{/docs/reference/math/}{Math mode} works differently than regular
markup or code mode. Numbers and single characters are displayed
verbatim, while multiple consecutive (non-number) characters will be
interpreted as Typst variables.

Typst pre-defines a lot of useful variables in math mode. All Greek (
\texttt{\ alpha\ } , \texttt{\ beta\ } , ...) and some Hebrew letters (
\texttt{\ alef\ } , \texttt{\ bet\ } , ...) are available through their
name. Some symbols are additionally available through shorthands, such
as \texttt{\ \textless{}=\ } , \texttt{\ \textgreater{}=\ } , and
\texttt{\ -\textgreater{}\ } .

Refer to the \href{/docs/reference/symbols/}{symbol pages} for a full
list of the symbols. If a symbol is missing, you can also access it
through a \href{/docs/reference/syntax/\#escapes}{Unicode escape
sequence} .

Alternate and related forms of symbols can often be selected by
\href{/docs/reference/symbols/symbol/}{appending a modifier} after a
period. For example, \texttt{\ arrow.l.squiggly\ } inserts a squiggly
left-pointing arrow. If you want to insert multiletter text in your
expression instead, enclose it in double quotes:

\begin{verbatim}
$ delta "if" x <= 5 $
\end{verbatim}

\includegraphics[width=5in,height=\textheight,keepaspectratio]{/assets/docs/jk1sG5lCs-ZQ-s2nXNTbTwAAAAAAAAAA.png}

In Typst, delimiters will scale automatically for their expressions,
just as if \texttt{\ \textbackslash{}left\ } and
\texttt{\ \textbackslash{}right\ } commands were implicitly inserted in
LaTeX. You can customize delimiter behaviour using the
\href{/docs/reference/math/lr/\#functions-lr}{\texttt{\ lr\ } function}
. To prevent a pair of delimiters from scaling, you can escape them with
backslashes.

Typst will automatically set terms around a slash \texttt{\ /\ } as a
fraction while honoring operator precedence. All round parentheses not
made redundant by the fraction will appear in the output.

\begin{verbatim}
$ f(x) = (x + 1) / x $
\end{verbatim}

\includegraphics[width=5in,height=\textheight,keepaspectratio]{/assets/docs/hWpFm1Wb3In32sS_SEIHvgAAAAAAAAAA.png}

\href{/docs/reference/math/attach/\#functions-attach}{Sub- and
superscripts} work similarly in Typst and LaTeX.
\texttt{\ }{\texttt{\ \$\ }}\texttt{\ x\ }{\texttt{\ \^{}\ }}\texttt{\ 2\ }{\texttt{\ \$\ }}\texttt{\ }
will produce a superscript,
\texttt{\ }{\texttt{\ \$\ }}\texttt{\ x\ }{\texttt{\ \_\ }}\texttt{\ 2\ }{\texttt{\ \$\ }}\texttt{\ }
yields a subscript. If you want to include more than one value in a sub-
or superscript, enclose their contents in parentheses:
\texttt{\ }{\texttt{\ \$\ }}\texttt{\ x\ }{\texttt{\ \_\ }}\texttt{\ }{\texttt{\ (\ }}\texttt{\ a\ }{\texttt{\ -\textgreater{}\ }}\texttt{\ }{\texttt{\ epsilon\ }}\texttt{\ }{\texttt{\ )\ }}\texttt{\ }{\texttt{\ \$\ }}\texttt{\ }
.

Since variables in math mode do not need to be prepended with a
\texttt{\ \#\ } or a \texttt{\ /\ } , you can also call functions
without these special characters:

\begin{verbatim}
$ f(x, y) := cases(
  1 "if" (x dot y)/2 <= 0,
  2 "if" x "is even",
  3 "if" x in NN,
  4 "else",
) $
\end{verbatim}

\includegraphics[width=5in,height=\textheight,keepaspectratio]{/assets/docs/0X1AFPDieBd9jiawKpc0-AAAAAAAAAAA.png}

The above example uses the
\href{/docs/reference/math/cases/}{\texttt{\ cases\ } function} to
describe f. Within the cases function, arguments are delimited using
commas and the arguments are also interpreted as math. If you need to
interpret arguments as Typst values instead, prefix them with a
\texttt{\ \#\ } :

\begin{verbatim}
$ (a + b)^2
  = a^2
  + text(fill: #maroon, 2 a b)
  + b^2 $
\end{verbatim}

\includegraphics[width=5in,height=\textheight,keepaspectratio]{/assets/docs/Wmx0wcFeGyknnvRFv8bI5QAAAAAAAAAA.png}

You can use all Typst functions within math mode and insert any content.
If you want them to work normally, with code mode in the argument list,
you can prefix their call with a \texttt{\ \#\ } . Nobody can stop you
from using rectangles or emoji as your variables anymore:

\begin{verbatim}
$ sum^10_(🥸=1)
  #rect(width: 4mm, height: 2mm)/🥸
  = 🧠 maltese $
\end{verbatim}

\includegraphics[width=5in,height=\textheight,keepaspectratio]{/assets/docs/aSVPcEv5ICbnBtcXfhpmfgAAAAAAAAAA.png}

If you\textquotesingle d like to enter your mathematical symbols
directly as Unicode, that is possible, too!

Math calls can have two-dimensional argument lists using \texttt{\ ;\ }
as a delimiter. The most common use for this is the
\href{/docs/reference/math/mat/}{\texttt{\ mat\ } function} that creates
matrices:

\begin{verbatim}
$ mat(
  1, 2, ..., 10;
  2, 2, ..., 10;
  dots.v, dots.v, dots.down, dots.v;
  10, 10, ..., 10;
) $
\end{verbatim}

\includegraphics[width=5in,height=\textheight,keepaspectratio]{/assets/docs/yiSilYGQ1wRBpIK3ON349AAAAAAAAAAA.png}

\subsection{How do I get the "LaTeX look?"}\label{latex-look}

Papers set in LaTeX have an unmistakeable look. This is mostly due to
their font, Computer Modern, justification, narrow line spacing, and
wide margins.

The example below

\begin{itemize}
\tightlist
\item
  sets wide
  \href{/docs/reference/layout/page/\#parameters-margin}{margins}
\item
  enables
  \href{/docs/reference/model/par/\#parameters-justify}{justification} ,
  \href{/docs/reference/model/par/\#parameters-leading}{tighter lines}
  and
  \href{/docs/reference/model/par/\#parameters-first-line-indent}{first-line-indent}
\item
  \href{/docs/reference/text/text/\#parameters-font}{sets the font} to
  "New Computer Modern", an OpenType derivative of Computer Modern for
  both text and \href{/docs/reference/text/raw/}{code blocks}
\item
  disables paragraph
  \href{/docs/reference/layout/block/\#parameters-spacing}{spacing}
\item
  increases
  \href{/docs/reference/layout/block/\#parameters-spacing}{spacing}
  around \href{/docs/reference/model/heading/}{headings}
\end{itemize}

\begin{verbatim}
#set page(margin: 1.75in)
#set par(leading: 0.55em, spacing: 0.55em, first-line-indent: 1.8em, justify: true)
#set text(font: "New Computer Modern")
#show raw: set text(font: "New Computer Modern Mono")
#show heading: set block(above: 1.4em, below: 1em)
\end{verbatim}

This should be a good starting point! If you want to go further, why not
create a reusable template?

\subsection{Bibliographies}\label{bibliographies}

Typst includes a fully-featured bibliography system that is compatible
with BibTeX files. You can continue to use your \texttt{\ .bib\ }
literature libraries by loading them with the
\href{/docs/reference/model/bibliography/}{\texttt{\ bibliography\ }}
function. Another possibility is to use
\href{https://github.com/typst/hayagriva/blob/main/docs/file-format.md}{Typst\textquotesingle s
YAML-based native format} .

Typst uses the Citation Style Language to define and process citation
and bibliography styles. You can compare CSL files to
BibLaTeX\textquotesingle s \texttt{\ .bbx\ } files. The compiler already
includes
\href{/docs/reference/model/bibliography/\#parameters-style}{over 80
citation styles} , but you can use any CSL-compliant style from the
\href{https://github.com/citation-style-language/styles}{CSL repository}
or write your own.

You can cite an entry in your bibliography or reference a label in your
document with the same syntax: \texttt{\ }{\texttt{\ @key\ }}\texttt{\ }
(this would reference an entry called \texttt{\ key\ } ). Alternatively,
you can use the \href{/docs/reference/model/cite/}{\texttt{\ cite\ }}
function.

Alternative forms for your citation, such as year only and citations for
natural use in prose (cf. \texttt{\ \textbackslash{}citet\ } and
\texttt{\ \textbackslash{}textcite\ } ) are available with
\href{/docs/reference/model/cite/\#parameters-form}{\texttt{\ }{\texttt{\ \#\ }}\texttt{\ }{\texttt{\ cite\ }}\texttt{\ }{\texttt{\ (\ }}\texttt{\ }{\texttt{\ \textless{}key\textgreater{}\ }}\texttt{\ }{\texttt{\ ,\ }}\texttt{\ form\ }{\texttt{\ :\ }}\texttt{\ }{\texttt{\ "prose"\ }}\texttt{\ }{\texttt{\ )\ }}\texttt{\ }}
.

You can find more information on the documentation page of the
\href{/docs/reference/model/bibliography/}{\texttt{\ bibliography\ }}
function.

\subsection{What limitations does Typst currently have compared to
LaTeX?}\label{limitations}

Although Typst can be a LaTeX replacement for many today, there are
still features that Typst does not (yet) support. Here is a list of them
which, where applicable, contains possible workarounds.

\begin{itemize}
\item
  \textbf{Well-established plotting ecosystem.} LaTeX users often create
  elaborate charts along with their documents in PGF/TikZ. The Typst
  ecosystem does not yet offer the same breadth of available options,
  but the ecosystem around the
  \href{https://github.com/cetz-package/cetz}{\texttt{\ cetz\ }} package
  is catching up quickly.
\item
  \textbf{Change page margins without a pagebreak.} In LaTeX, margins
  can always be adjusted, even without a pagebreak. To change margins in
  Typst, you use the
  \href{/docs/reference/layout/page/}{\texttt{\ page\ } function} which
  will force a page break. If you just want a few paragraphs to stretch
  into the margins, then reverting to the old margins, you can use the
  \href{/docs/reference/layout/pad/}{\texttt{\ pad\ } function} with
  negative padding.
\item
  \textbf{Include PDFs as images.} In LaTeX, it has become customary to
  insert vector graphics as PDF or EPS files. Typst supports neither
  format as an image format, but you can easily convert both into SVG
  files with \href{https://cloudconvert.com/pdf-to-svg}{online tools} or
  \href{https://inkscape.org/}{Inkscape} . The web app will
  automatically convert PDF files to SVG files upon uploading them.
\end{itemize}

\href{/docs/guides/}{\pandocbounded{\includesvg[keepaspectratio]{/assets/icons/16-arrow-right.svg}}}

{ Guides } { Previous page }

\href{/docs/guides/page-setup-guide/}{\pandocbounded{\includesvg[keepaspectratio]{/assets/icons/16-arrow-right.svg}}}

{ Page setup guide } { Next page }




\section{C Docs LaTeX/docs/docs.tex}
\section{Docs LaTeX/typst.app/docs/reference.tex}
\title{typst.app/docs/reference}

\begin{itemize}
\tightlist
\item
  \href{/docs}{\includesvg[width=0.16667in,height=0.16667in]{/assets/icons/16-docs-dark.svg}}
\item
  \includesvg[width=0.16667in,height=0.16667in]{/assets/icons/16-arrow-right.svg}
\item
  \href{/docs/reference/}{Reference}
\end{itemize}

\section{Reference}\label{reference}

This reference documentation is a comprehensive guide to all of
Typst\textquotesingle s syntax, concepts, types, and functions. If you
are completely new to Typst, we recommend starting with the
\href{/docs/tutorial/}{tutorial} and then coming back to the reference
to learn more about Typst\textquotesingle s features as you need them.

\subsection{Language}\label{language}

The reference starts with a language part that gives an overview over
\href{/docs/reference/syntax/}{Typst\textquotesingle s syntax} and
contains information about concepts involved in
\href{/docs/reference/styling/}{styling documents,} using
\href{/docs/reference/scripting/}{Typst\textquotesingle s scripting
capabilities.}

\subsection{Functions}\label{functions}

The second part includes chapters on all functions used to insert,
style, transform, and layout content in Typst documents. Each function
is documented with a description of its purpose, a list of its
parameters, and examples of how to use it.

The final part of the reference explains all functions that are used
within Typst\textquotesingle s code mode to manipulate and transform
data. Just as in the previous part, each function is documented with a
description of its purpose, a list of its parameters, and examples of
how to use it.

\href{/docs/tutorial/making-a-template/}{\pandocbounded{\includesvg[keepaspectratio]{/assets/icons/16-arrow-right.svg}}}

{ Making a Template } { Previous page }

\href{/docs/reference/syntax/}{\pandocbounded{\includesvg[keepaspectratio]{/assets/icons/16-arrow-right.svg}}}

{ Syntax } { Next page }


\section{Docs LaTeX/typst.app/docs/tutorial.tex}
\title{typst.app/docs/tutorial}

\begin{itemize}
\tightlist
\item
  \href{/docs}{\includesvg[width=0.16667in,height=0.16667in]{/assets/icons/16-docs-dark.svg}}
\item
  \includesvg[width=0.16667in,height=0.16667in]{/assets/icons/16-arrow-right.svg}
\item
  \href{/docs/tutorial/}{Tutorial}
\end{itemize}

\section{Tutorial}\label{tutorial}

Welcome to Typst\textquotesingle s tutorial! In this tutorial, you will
learn how to write and format documents in Typst. We will start with
everyday tasks and gradually introduce more advanced features. This
tutorial does not assume prior knowledge of Typst, other markup
languages, or programming. We do assume that you know how to edit a text
file.

The best way to start is to sign up to the Typst app for free and follow
along with the steps below. The app gives you instant preview, syntax
highlighting and helpful autocompletions. Alternatively, you can follow
along in your local text editor with the
\href{https://github.com/typst/typst}{open-source CLI} .

\subsection{When to use Typst}\label{when-typst}

Before we get started, let\textquotesingle s check what Typst is and
when to use it. Typst is a markup language for typesetting documents. It
is designed to be easy to learn, fast, and versatile. Typst takes text
files with markup in them and outputs PDFs.

Typst is a good choice for writing any long form text such as essays,
articles, scientific papers, books, reports, and homework assignments.
Moreover, Typst is a great fit for any documents containing mathematical
notation, such as papers in the math, physics, and engineering fields.
Finally, due to its strong styling and automation features, it is an
excellent choice for any set of documents that share a common style,
such as a book series.

\subsection{What you will learn}\label{learnings}

This tutorial has four chapters. Each chapter builds on the previous
one. Here is what you will learn in each of them:

\begin{enumerate}
\tightlist
\item
  \href{/docs/tutorial/writing-in-typst/}{Writing in Typst:} Learn how
  to write text and insert images, equations, and other elements.
\item
  \href{/docs/tutorial/formatting/}{Formatting:} Learn how to adjust the
  formatting of your document, including font size, heading styles, and
  more.
\item
  \href{/docs/tutorial/advanced-styling/}{Advanced Styling:} Create a
  complex page layout for a scientific paper with typographic features
  such as an author list and run-in headings.
\item
  \href{/docs/tutorial/making-a-template/}{Making a Template:} Build a
  reusable template from the paper you created in the previous chapter.
\end{enumerate}

We hope you\textquotesingle ll enjoy Typst!

\href{/docs/}{\pandocbounded{\includesvg[keepaspectratio]{/assets/icons/16-arrow-right.svg}}}

{ Overview } { Previous page }

\href{/docs/tutorial/writing-in-typst/}{\pandocbounded{\includesvg[keepaspectratio]{/assets/icons/16-arrow-right.svg}}}

{ Writing in Typst } { Next page }


\section{Docs LaTeX/typst.app/docs/community.tex}
\title{typst.app/docs/community}

\begin{itemize}
\tightlist
\item
  \href{/docs}{\includesvg[width=0.16667in,height=0.16667in]{/assets/icons/16-docs-dark.svg}}
\item
  \includesvg[width=0.16667in,height=0.16667in]{/assets/icons/16-arrow-right.svg}
\item
  \href{/docs/community/}{Community}
\end{itemize}

\section{Community}\label{community}

Hey and welcome to the Community page! We\textquotesingle re so glad
you\textquotesingle re here. Typst is developed by an early-stage
startup and it is still early days, but it would be pointless without
people like you who are interested in it.

We would love to not only hear from you but to also provide spaces where
you can discuss any topic around Typst, typesetting, writing, the
sciences, and typography with other likeminded people.

\textbf{Our \href{https://forum.typst.app/}{Forum} is the best place to
get answers for questions on Typst and to show off your creations.} If
you would like to chat with the community and shape the future
development of Typst, we would like to also invite you to our
\href{https://discord.gg/2uDybryKPe}{Discord server} . We coordinate our
Open-Source work there, but you can also iterate on Typst projects and
discuss off-topic things with the community members. Both the Forum and
the Discord server are open for everyone. Of course, you are also very
welcome to connect with us on social media (
\href{https://mastodon.social/@typst}{Mastodon} ,
\href{https://bsky.app/profile/typst.app}{Bluesky} ,
\href{https://instagram.com/typstapp/}{Instagram} ,
\href{https://linkedin.com/company/typst}{LinkedIn} , and
\href{https://github.com/typst}{GitHub} ).

\subsection{What to share?}\label{what-to-share}

For our community, we want to foster versatility and inclusivity. You
are welcome to post about any topic that you think would interest other
community members, but if you need a little inspiration, here are a few
ideas:

\begin{itemize}
\tightlist
\item
  Share and discuss your thoughts and ideas for new features or
  improvements you\textquotesingle d like to see in Typst
\item
  Showcase documents you\textquotesingle ve created with Typst, or share
  any unique or creative ways you\textquotesingle ve used the platform
\item
  Share importable files or templates that you use to style your
  documents
\item
  Alert us of bugs you encounter while using Typst
\end{itemize}

\subsection{Following the development}\label{following-the-development}

Typst is still under very active development and breaking changes can
occur at any point. The compiler is developed in the open on
\href{https://github.com/typst/typst}{GitHub} .

We will update the members of our Discord server and our social media
followers when new features become available. We\textquotesingle ll also
update you on the development progress of large features.

\subsection{How to support Typst}\label{support-typst}

If you want to support Typst, there are multiple ways to do that! You
can \href{https://github.com/typst/typst}{contribute to the code} or
\href{https://github.com/search?q=repo\%3Atypst\%2Ftypst+impl+LocalName+for&type=code}{translate
the strings in Typst} to your native language if it\textquotesingle s
not supported yet. You can also help us by
\href{https://typst.app/pricing}{subscribing to the paid tier of our web
app} or \href{https://github.com/sponsors/typst}{sponsoring our Open
Source efforts!} Multiple recurring sponsorship tiers are available and
all of them come with a set of goodies. No matter how you contribute,
thank you for your support!

\subsection{Community Rules}\label{rules}

We want to make our community a safe and inclusive space for everyone.
Therefore, we will not tolerate any sexual harassment, sexism, political
attacks, derogatory language or personal insults, racism, doxing, and
other inappropriate behaviour. We pledge to remove members that are in
violation of these rules. \href{https://typst.app/contact/}{Contact us}
if you think another community member acted inappropriately towards you.
All complaints will be reviewed and investigated promptly and fairly.

In addition, our \href{https://typst.app/privacy/}{privacy policy}
applies on all community spaces operated by us, such as the Discord
server. Please also note that the terms of service and privacy policies
of the respective services apply.

\subsection{See you soon!}\label{see-you}

Thanks again for learning more about Typst. We would be delighted to
meet you on our \href{https://discord.gg/2uDybryKPe}{Discord server} !

\href{/docs/roadmap/}{\pandocbounded{\includesvg[keepaspectratio]{/assets/icons/16-arrow-right.svg}}}

{ Roadmap } { Previous page }


\section{Docs LaTeX/typst.app/docs/roadmap.tex}
\title{typst.app/docs/roadmap}

\begin{itemize}
\tightlist
\item
  \href{/docs}{\includesvg[width=0.16667in,height=0.16667in]{/assets/icons/16-docs-dark.svg}}
\item
  \includesvg[width=0.16667in,height=0.16667in]{/assets/icons/16-arrow-right.svg}
\item
  \href{/docs/roadmap/}{Roadmap}
\end{itemize}

\section{Roadmap}\label{roadmap}

This page lists planned features for the Typst language, compiler,
library and web app. Since priorities and development realities change,
this roadmap is not set in stone. Features that are listed here will not
necessarily be implemented and features that will be implemented might
be missing here. Moreover, this roadmap only lists larger, more
fundamental features and bugs.

Are you missing something on the roadmap? Typst relies on your feedback
as a user to plan for and prioritize new features. Get started by filing
a new issue on \href{https://github.com/typst/typst/issues}{GitHub} or
discuss your feature request with the
\href{https://typst.app/docs/community}{community} .

\subsection{Language and Compiler}\label{language-and-compiler}

\begin{itemize}
\item
  \textbf{Styling}

  \begin{itemize}
  \tightlist
  \item
    Support for revoking style rules
  \item
    Ancestry selectors (e.g., within)
  \item
    \st{Fix show rule recursion crashes}
  \item
    \st{Fix show-set issues}
  \end{itemize}
\item
  \textbf{Scripting}

  \begin{itemize}
  \tightlist
  \item
    Function for debug logging
  \item
    Fix issues with paths being strings
  \item
    Custom types (that work with set and show rules)
  \item
    Type hints
  \item
    Function hoisting (if feasible)
  \item
    \st{Data loading functions}
  \item
    \st{Support for compiler warnings}
  \item
    \st{Types as first-class values}
  \item
    \st{More fields and methods on primitives}
  \item
    \st{WebAssembly plugins}
  \item
    \st{Get values of set rules}
  \item
    \st{Replace locate, etc. with unified context system}
  \item
    \st{Allow expressions as dictionary keys}
  \item
    \st{Package management}
  \end{itemize}
\item
  \textbf{Model}

  \begin{itemize}
  \tightlist
  \item
    Fix issues with numbering patterns
  \item
    Support a path or bytes in places that currently only support paths,
    superseding \texttt{\ .decode\ } scoped functions
  \item
    Better support for custom referenceable things
  \item
    Richer built-in outline customization
  \item
    Enum continuation
  \item
    \st{Bibliography and citation customization via CSL (Citation Style
    Language)}
  \item
    \st{Relative counters, e.g. for figure numbering per section}
  \end{itemize}
\item
  \textbf{Text}

  \begin{itemize}
  \tightlist
  \item
    Font fallback warnings
  \item
    Bold, italic, and smallcaps synthesis
  \item
    Variable fonts support
  \item
    Ruby and Warichu
  \item
    Kashida justification
  \item
    \st{Support for basic CJK text layout rules}
  \item
    \st{Fix SVG font fallback}
  \item
    \st{Themes for raw text and more/custom syntaxes}
  \end{itemize}
\item
  \textbf{Math}

  \begin{itemize}
  \tightlist
  \item
    Fix syntactic quirks
  \item
    Fix single letter strings
  \item
    Fix font handling
  \item
    Fix attachment parsing priorities
  \item
    Provide more primitives
  \item
    Improve equation numbering
  \item
    Big fractions
  \end{itemize}
\item
  \textbf{Layout}

  \begin{itemize}
  \tightlist
  \item
    Fix footnote issues
  \item
    Fix issues with list (in particular baselines \& alignment)
  \item
    Support for "sticky" blocks that stay with the next one
  \item
    Improve widow \& orphan prevention
  \item
    Expand floating layout (e.g. over two columns)
  \item
    Support for side-floats and other "collision" layouts
  \item
    Better support for more canvas-like layouts
  \item
    Unified layout primitives across normal content and math
  \item
    Page adjustment from within flow
  \item
    Chained layout regions
  \item
    Grid-based typesetting
  \item
    Balanced columns
  \item
    Drop caps
  \item
    End notes, maybe margin notes
  \item
    \st{Footnotes}
  \item
    \st{Basic floating layout}
  \item
    \st{Row span and column span in table}
  \item
    \st{Per-cell table stroke customization}
  \end{itemize}
\item
  \textbf{Visualize}

  \begin{itemize}
  \tightlist
  \item
    Arrows
  \item
    Better path drawing, possibly path operations
  \item
    Color management
  \item
    \st{More configurable strokes}
  \item
    \st{Gradients}
  \item
    \st{Patterns}
  \end{itemize}
\item
  \textbf{Introspection}

  \begin{itemize}
  \tightlist
  \item
    Support for freezing content, so that e.g. numbers in it remain the
    same if it appears multiple times
  \end{itemize}
\item
  \textbf{Export}

  \begin{itemize}
  \tightlist
  \item
    PDF/A support
  \item
    HTML export
  \item
    Tagged PDF for Accessibility
  \item
    PDF/X support
  \item
    EPUB export
  \item
    \st{PNG export}
  \item
    \st{SVG export}
  \item
    \st{Support for transparency in PDF}
  \item
    \st{Fix issues with SVGs in PDF}
  \item
    \st{Fix emoji export in PDF} (not yet released)
  \item
    \st{Selectable text in SVGs in PDF} (not yet released)
  \item
    \st{Better font subsetting for smaller PDFs} (not yet released)
  \end{itemize}
\item
  \textbf{CLI}

  \begin{itemize}
  \tightlist
  \item
    Support for downloading fonts on-demand automatically
  \item
    \st{\mbox{\texttt{\ typst\ query\ }} for querying document elements}
  \item
    \st{\mbox{\texttt{\ typst\ init\ }} for creating a project from a
    template}
  \item
    \st{\mbox{\texttt{\ typst\ update\ }} for self-updating the CLI}
  \end{itemize}
\item
  \textbf{Tooling}

  \begin{itemize}
  \tightlist
  \item
    Documentation generator and doc comments
  \item
    Autoformatter
  \item
    Linter
  \end{itemize}
\item
  \textbf{Performance}

  \begin{itemize}
  \tightlist
  \item
    Reduce memory usage
  \item
    \st{Optimize runtime of optimal paragraph layout} (not yet released)
  \item
    \st{Parallelize layout engine} (not yet released)
  \end{itemize}
\item
  \textbf{Development}

  \begin{itemize}
  \tightlist
  \item
    Benchmarking
  \item
    Better contributor documentation
  \end{itemize}
\end{itemize}

\subsection{Web App}\label{web-app}

\begin{itemize}
\item
  \textbf{Editing}

  \begin{itemize}
  \tightlist
  \item
    Smarter \& more action buttons
  \item
    Inline documentation
  \item
    Preview autocomplete entry
  \item
    Color Picker
  \item
    Symbol picker
  \item
    Basic, built-in image editor (cropping, etc.)
  \item
    GUI inspector for editing function calls
  \item
    Cursor in preview
  \item
    \st{Hover tooltips for debugging}
  \item
    \st{Scroll to cursor position in preview}
  \item
    \st{Folders in projects}
  \item
    \st{Outline panel}
  \item
    \st{More export options}
  \item
    \st{Preview in a separate window}
  \item
    \st{Sync literature with Zotero and Mendely}
  \item
    \st{Paste modal}
  \item
    \st{Improve panel}
  \end{itemize}
\item
  \textbf{Writing}

  \begin{itemize}
  \tightlist
  \item
    Word count
  \item
    Structure view
  \item
    Text completion by LLM
  \item
    \st{Spell check}
  \end{itemize}
\item
  \textbf{Collaboration}

  \begin{itemize}
  \tightlist
  \item
    Change tracking
  \item
    Version history
  \item
    \st{Chat-like comments}
  \item
    \st{Git integration}
  \end{itemize}
\item
  \textbf{Project management}

  \begin{itemize}
  \tightlist
  \item
    Drag-and-drop for projects
  \item
    Template generation by LLM
  \item
    \st{LaTeX, Word, Markdown import}
  \item
    \st{Thumbnails for projects}
  \end{itemize}
\item
  \textbf{Settings}

  \begin{itemize}
  \tightlist
  \item
    Keyboard shortcuts configuration
  \item
    Better project settings
  \item
    Avatar Cropping
  \item
    \st{System Theme setting}
  \end{itemize}
\item
  \textbf{Other}

  \begin{itemize}
  \tightlist
  \item
    Offline PWA
  \item
    Mobile improvements
  \item
    Two-Factor Authentication
  \item
    Advanced search in projects
  \item
    Private packages in teams
  \item
    \st{LDAP Single sign-on}
  \item
    \st{Compiler version picker}
  \item
    \st{Presentation mode}
  \item
    \st{Support for On-Premises deployment}
  \item
    \st{Typst Universe}
  \end{itemize}
\end{itemize}

\href{/docs/changelog/earlier/}{\pandocbounded{\includesvg[keepaspectratio]{/assets/icons/16-arrow-right.svg}}}

{ Earlier } { Previous page }

\href{/docs/community/}{\pandocbounded{\includesvg[keepaspectratio]{/assets/icons/16-arrow-right.svg}}}

{ Community } { Next page }


\section{Docs LaTeX/typst.app/docs/changelog.tex}
\title{typst.app/docs/changelog}

\begin{itemize}
\tightlist
\item
  \href{/docs}{\includesvg[width=0.16667in,height=0.16667in]{/assets/icons/16-docs-dark.svg}}
\item
  \includesvg[width=0.16667in,height=0.16667in]{/assets/icons/16-arrow-right.svg}
\item
  \href{/docs/changelog/}{Changelog}
\end{itemize}

\section{Changelog}\label{changelog}

Learn what has changed in the latest Typst releases and move your
documents forward. This section documents all changes to Typst since its
initial public release.

\subsection{Versions}\label{versions}

\begin{itemize}
\tightlist
\item
  \href{/docs/changelog/0.12.0/}{Typst 0.12.0}
\item
  \href{/docs/changelog/0.11.1/}{Typst 0.11.1}
\item
  \href{/docs/changelog/0.11.0/}{Typst 0.11.0}
\item
  \href{/docs/changelog/0.10.0/}{Typst 0.10.0}
\item
  \href{/docs/changelog/0.9.0/}{Typst 0.9.0}
\item
  \href{/docs/changelog/0.8.0/}{Typst 0.8.0}
\item
  \href{/docs/changelog/0.7.0/}{Typst 0.7.0}
\item
  \href{/docs/changelog/0.6.0/}{Typst 0.6.0}
\item
  \href{/docs/changelog/0.5.0/}{Typst 0.5.0}
\item
  \href{/docs/changelog/0.4.0/}{Typst 0.4.0}
\item
  \href{/docs/changelog/0.3.0/}{Typst 0.3.0}
\item
  \href{/docs/changelog/0.2.0/}{Typst 0.2.0}
\item
  \href{/docs/changelog/0.1.0/}{Typst 0.1.0}
\item
  \href{/docs/changelog/earlier/}{Earlier}
\end{itemize}

\href{/docs/guides/table-guide/}{\pandocbounded{\includesvg[keepaspectratio]{/assets/icons/16-arrow-right.svg}}}

{ Table guide } { Previous page }

\href{/docs/changelog/0.12.0/}{\pandocbounded{\includesvg[keepaspectratio]{/assets/icons/16-arrow-right.svg}}}

{ 0.12.0 } { Next page }


\section{Docs LaTeX/typst.app/docs/guides.tex}
\title{typst.app/docs/guides}

\begin{itemize}
\tightlist
\item
  \href{/docs}{\includesvg[width=0.16667in,height=0.16667in]{/assets/icons/16-docs-dark.svg}}
\item
  \includesvg[width=0.16667in,height=0.16667in]{/assets/icons/16-arrow-right.svg}
\item
  \href{/docs/guides/}{Guides}
\end{itemize}

\section{Guides}\label{guides}

Welcome to the Guides section! Here, you\textquotesingle ll find helpful
material for specific user groups or use cases. Currently, two guides
are available: An introduction to Typst for LaTeX users, and a detailed
look at page setup. Feel free to propose other topics for guides!

\subsection{List of Guides}\label{list-of-guides}

\begin{itemize}
\tightlist
\item
  \href{/docs/guides/guide-for-latex-users/}{Guide for LaTeX users}
\item
  \href{/docs/guides/page-setup-guide/}{Page setup guide}
\item
  \href{/docs/guides/table-guide/}{Table guide}
\end{itemize}

\href{/docs/reference/data-loading/yaml/}{\pandocbounded{\includesvg[keepaspectratio]{/assets/icons/16-arrow-right.svg}}}

{ YAML } { Previous page }

\href{/docs/guides/guide-for-latex-users/}{\pandocbounded{\includesvg[keepaspectratio]{/assets/icons/16-arrow-right.svg}}}

{ Guide for LaTeX users } { Next page }








