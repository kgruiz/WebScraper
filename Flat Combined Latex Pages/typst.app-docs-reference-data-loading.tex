\title{typst.app/docs/reference/data-loading/xml}

\begin{itemize}
\tightlist
\item
  \href{/docs}{\includesvg[width=0.16667in,height=0.16667in]{/assets/icons/16-docs-dark.svg}}
\item
  \includesvg[width=0.16667in,height=0.16667in]{/assets/icons/16-arrow-right.svg}
\item
  \href{/docs/reference/}{Reference}
\item
  \includesvg[width=0.16667in,height=0.16667in]{/assets/icons/16-arrow-right.svg}
\item
  \href{/docs/reference/data-loading/}{Data Loading}
\item
  \includesvg[width=0.16667in,height=0.16667in]{/assets/icons/16-arrow-right.svg}
\item
  \href{/docs/reference/data-loading/xml/}{XML}
\end{itemize}

\section{\texorpdfstring{\texttt{\ xml\ }}{ xml }}\label{summary}

Reads structured data from an XML file.

The XML file is parsed into an array of dictionaries and strings. XML
nodes can be elements or strings. Elements are represented as
dictionaries with the following keys:

\begin{itemize}
\tightlist
\item
  \texttt{\ tag\ } : The name of the element as a string.
\item
  \texttt{\ attrs\ } : A dictionary of the element\textquotesingle s
  attributes as strings.
\item
  \texttt{\ children\ } : An array of the element\textquotesingle s
  child nodes.
\end{itemize}

The XML file in the example contains a root \texttt{\ news\ } tag with
multiple \texttt{\ article\ } tags. Each article has a
\texttt{\ title\ } , \texttt{\ author\ } , and \texttt{\ content\ } tag.
The \texttt{\ content\ } tag contains one or more paragraphs, which are
represented as \texttt{\ p\ } tags.

\subsection{Example}\label{example}

\begin{verbatim}
#let find-child(elem, tag) = {
  elem.children
    .find(e => "tag" in e and e.tag == tag)
}

#let article(elem) = {
  let title = find-child(elem, "title")
  let author = find-child(elem, "author")
  let pars = find-child(elem, "content")

  heading(title.children.first())
  text(10pt, weight: "medium")[
    Published by
    #author.children.first()
  ]

  for p in pars.children {
    if (type(p) == "dictionary") {
      parbreak()
      p.children.first()
    }
  }
}

#let data = xml("example.xml")
#for elem in data.first().children {
  if (type(elem) == "dictionary") {
    article(elem)
  }
}
\end{verbatim}

\includegraphics[width=5in,height=\textheight,keepaspectratio]{/assets/docs/ImsUm8fcO-Uh3s95k6HvEQAAAAAAAAAA.png}

\subsection{\texorpdfstring{{ Parameters
}}{ Parameters }}\label{parameters}

\phantomsection\label{parameters-tooltip}
Parameters are the inputs to a function. They are specified in
parentheses after the function name.

{ xml } (

{ \href{/docs/reference/foundations/str/}{str} }

) -\textgreater{} { any }

\subsubsection{\texorpdfstring{\texttt{\ path\ }}{ path }}\label{parameters-path}

\href{/docs/reference/foundations/str/}{str}

{Required} {{ Positional }}

\phantomsection\label{parameters-path-positional-tooltip}
Positional parameters are specified in order, without names.

Path to an XML file.

For more details, see the \href{/docs/reference/syntax/\#paths}{Paths
section} .

\subsection{\texorpdfstring{{ Definitions
}}{ Definitions }}\label{definitions}

\phantomsection\label{definitions-tooltip}
Functions and types and can have associated definitions. These are
accessed by specifying the function or type, followed by a period, and
then the definition\textquotesingle s name.

\subsubsection{\texorpdfstring{\texttt{\ decode\ }}{ decode }}\label{definitions-decode}

Reads structured data from an XML string/bytes.

xml { . } { decode } (

{ \href{/docs/reference/foundations/str/}{str}
\href{/docs/reference/foundations/bytes/}{bytes} }

) -\textgreater{} { any }

\paragraph{\texorpdfstring{\texttt{\ data\ }}{ data }}\label{definitions-decode-data}

\href{/docs/reference/foundations/str/}{str} {or}
\href{/docs/reference/foundations/bytes/}{bytes}

{Required} {{ Positional }}

\phantomsection\label{definitions-decode-data-positional-tooltip}
Positional parameters are specified in order, without names.

XML data.

\href{/docs/reference/data-loading/toml/}{\pandocbounded{\includesvg[keepaspectratio]{/assets/icons/16-arrow-right.svg}}}

{ TOML } { Previous page }

\href{/docs/reference/data-loading/yaml/}{\pandocbounded{\includesvg[keepaspectratio]{/assets/icons/16-arrow-right.svg}}}

{ YAML } { Next page }


\title{typst.app/docs/reference/data-loading/json}

\begin{itemize}
\tightlist
\item
  \href{/docs}{\includesvg[width=0.16667in,height=0.16667in]{/assets/icons/16-docs-dark.svg}}
\item
  \includesvg[width=0.16667in,height=0.16667in]{/assets/icons/16-arrow-right.svg}
\item
  \href{/docs/reference/}{Reference}
\item
  \includesvg[width=0.16667in,height=0.16667in]{/assets/icons/16-arrow-right.svg}
\item
  \href{/docs/reference/data-loading/}{Data Loading}
\item
  \includesvg[width=0.16667in,height=0.16667in]{/assets/icons/16-arrow-right.svg}
\item
  \href{/docs/reference/data-loading/json/}{JSON}
\end{itemize}

\section{\texorpdfstring{\texttt{\ json\ }}{ json }}\label{summary}

Reads structured data from a JSON file.

The file must contain a valid JSON value, such as object or array. JSON
objects will be converted into Typst dictionaries, and JSON arrays will
be converted into Typst arrays. Strings and booleans will be converted
into the Typst equivalents, \texttt{\ null\ } will be converted into
\texttt{\ }{\texttt{\ none\ }}\texttt{\ } , and numbers will be
converted to floats or integers depending on whether they are whole
numbers.

Be aware that integers larger than 2 \textsuperscript{63} -1 will be
converted to floating point numbers, which may result in an
approximative value.

The function returns a dictionary, an array or, depending on the JSON
file, another JSON data type.

The JSON files in the example contain objects with the keys
\texttt{\ temperature\ } , \texttt{\ unit\ } , and \texttt{\ weather\ }
.

\subsection{Example}\label{example}

\begin{verbatim}
#let forecast(day) = block[
  #box(square(
    width: 2cm,
    inset: 8pt,
    fill: if day.weather == "sunny" {
      yellow
    } else {
      aqua
    },
    align(
      bottom + right,
      strong(day.weather),
    ),
  ))
  #h(6pt)
  #set text(22pt, baseline: -8pt)
  #day.temperature °#day.unit
]

#forecast(json("monday.json"))
#forecast(json("tuesday.json"))
\end{verbatim}

\includegraphics[width=5in,height=\textheight,keepaspectratio]{/assets/docs/9TGGThvdnznDbVRRo5-HsgAAAAAAAAAA.png}

\subsection{\texorpdfstring{{ Parameters
}}{ Parameters }}\label{parameters}

\phantomsection\label{parameters-tooltip}
Parameters are the inputs to a function. They are specified in
parentheses after the function name.

{ json } (

{ \href{/docs/reference/foundations/str/}{str} }

) -\textgreater{} { any }

\subsubsection{\texorpdfstring{\texttt{\ path\ }}{ path }}\label{parameters-path}

\href{/docs/reference/foundations/str/}{str}

{Required} {{ Positional }}

\phantomsection\label{parameters-path-positional-tooltip}
Positional parameters are specified in order, without names.

Path to a JSON file.

For more details, see the \href{/docs/reference/syntax/\#paths}{Paths
section} .

\subsection{\texorpdfstring{{ Definitions
}}{ Definitions }}\label{definitions}

\phantomsection\label{definitions-tooltip}
Functions and types and can have associated definitions. These are
accessed by specifying the function or type, followed by a period, and
then the definition\textquotesingle s name.

\subsubsection{\texorpdfstring{\texttt{\ decode\ }}{ decode }}\label{definitions-decode}

Reads structured data from a JSON string/bytes.

json { . } { decode } (

{ \href{/docs/reference/foundations/str/}{str}
\href{/docs/reference/foundations/bytes/}{bytes} }

) -\textgreater{} { any }

\paragraph{\texorpdfstring{\texttt{\ data\ }}{ data }}\label{definitions-decode-data}

\href{/docs/reference/foundations/str/}{str} {or}
\href{/docs/reference/foundations/bytes/}{bytes}

{Required} {{ Positional }}

\phantomsection\label{definitions-decode-data-positional-tooltip}
Positional parameters are specified in order, without names.

JSON data.

\subsubsection{\texorpdfstring{\texttt{\ encode\ }}{ encode }}\label{definitions-encode}

Encodes structured data into a JSON string.

json { . } { encode } (

{ { any } , } { \hyperref[definitions-encode-parameters-pretty]{pretty
:} \href{/docs/reference/foundations/bool/}{bool} , }

) -\textgreater{} \href{/docs/reference/foundations/str/}{str}

\paragraph{\texorpdfstring{\texttt{\ value\ }}{ value }}\label{definitions-encode-value}

{ any }

{Required} {{ Positional }}

\phantomsection\label{definitions-encode-value-positional-tooltip}
Positional parameters are specified in order, without names.

Value to be encoded.

\paragraph{\texorpdfstring{\texttt{\ pretty\ }}{ pretty }}\label{definitions-encode-pretty}

\href{/docs/reference/foundations/bool/}{bool}

Whether to pretty print the JSON with newlines and indentation.

Default: \texttt{\ }{\texttt{\ true\ }}\texttt{\ }

\href{/docs/reference/data-loading/csv/}{\pandocbounded{\includesvg[keepaspectratio]{/assets/icons/16-arrow-right.svg}}}

{ CSV } { Previous page }

\href{/docs/reference/data-loading/read/}{\pandocbounded{\includesvg[keepaspectratio]{/assets/icons/16-arrow-right.svg}}}

{ Read } { Next page }


\title{typst.app/docs/reference/data-loading/csv}

\begin{itemize}
\tightlist
\item
  \href{/docs}{\includesvg[width=0.16667in,height=0.16667in]{/assets/icons/16-docs-dark.svg}}
\item
  \includesvg[width=0.16667in,height=0.16667in]{/assets/icons/16-arrow-right.svg}
\item
  \href{/docs/reference/}{Reference}
\item
  \includesvg[width=0.16667in,height=0.16667in]{/assets/icons/16-arrow-right.svg}
\item
  \href{/docs/reference/data-loading/}{Data Loading}
\item
  \includesvg[width=0.16667in,height=0.16667in]{/assets/icons/16-arrow-right.svg}
\item
  \href{/docs/reference/data-loading/csv/}{CSV}
\end{itemize}

\section{\texorpdfstring{\texttt{\ csv\ }}{ csv }}\label{summary}

Reads structured data from a CSV file.

The CSV file will be read and parsed into a 2-dimensional array of
strings: Each row in the CSV file will be represented as an array of
strings, and all rows will be collected into a single array. Header rows
will not be stripped.

\subsection{Example}\label{example}

\begin{verbatim}
#let results = csv("example.csv")

#table(
  columns: 2,
  [*Condition*], [*Result*],
  ..results.flatten(),
)
\end{verbatim}

\includegraphics[width=5in,height=\textheight,keepaspectratio]{/assets/docs/wZK4j33X4RoMvhQZsQnpmQAAAAAAAAAA.png}

\subsection{\texorpdfstring{{ Parameters
}}{ Parameters }}\label{parameters}

\phantomsection\label{parameters-tooltip}
Parameters are the inputs to a function. They are specified in
parentheses after the function name.

{ csv } (

{ \href{/docs/reference/foundations/str/}{str} , } {
\hyperref[parameters-delimiter]{delimiter :}
\href{/docs/reference/foundations/str/}{str} , } {
\hyperref[parameters-row-type]{row-type :}
\href{/docs/reference/foundations/type/}{type} , }

) -\textgreater{} \href{/docs/reference/foundations/array/}{array}

\subsubsection{\texorpdfstring{\texttt{\ path\ }}{ path }}\label{parameters-path}

\href{/docs/reference/foundations/str/}{str}

{Required} {{ Positional }}

\phantomsection\label{parameters-path-positional-tooltip}
Positional parameters are specified in order, without names.

Path to a CSV file.

For more details, see the \href{/docs/reference/syntax/\#paths}{Paths
section} .

\subsubsection{\texorpdfstring{\texttt{\ delimiter\ }}{ delimiter }}\label{parameters-delimiter}

\href{/docs/reference/foundations/str/}{str}

The delimiter that separates columns in the CSV file. Must be a single
ASCII character.

Default: \texttt{\ }{\texttt{\ ","\ }}\texttt{\ }

\subsubsection{\texorpdfstring{\texttt{\ row-type\ }}{ row-type }}\label{parameters-row-type}

\href{/docs/reference/foundations/type/}{type}

How to represent the file\textquotesingle s rows.

\begin{itemize}
\tightlist
\item
  If set to \texttt{\ array\ } , each row is represented as a plain
  array of strings.
\item
  If set to \texttt{\ dictionary\ } , each row is represented as a
  dictionary mapping from header keys to strings. This option only makes
  sense when a header row is present in the CSV file.
\end{itemize}

Default: \texttt{\ array\ }

\subsection{\texorpdfstring{{ Definitions
}}{ Definitions }}\label{definitions}

\phantomsection\label{definitions-tooltip}
Functions and types and can have associated definitions. These are
accessed by specifying the function or type, followed by a period, and
then the definition\textquotesingle s name.

\subsubsection{\texorpdfstring{\texttt{\ decode\ }}{ decode }}\label{definitions-decode}

Reads structured data from a CSV string/bytes.

csv { . } { decode } (

{ \href{/docs/reference/foundations/str/}{str}
\href{/docs/reference/foundations/bytes/}{bytes} , } {
\hyperref[definitions-decode-parameters-delimiter]{delimiter :}
\href{/docs/reference/foundations/str/}{str} , } {
\hyperref[definitions-decode-parameters-row-type]{row-type :}
\href{/docs/reference/foundations/type/}{type} , }

) -\textgreater{} \href{/docs/reference/foundations/array/}{array}

\paragraph{\texorpdfstring{\texttt{\ data\ }}{ data }}\label{definitions-decode-data}

\href{/docs/reference/foundations/str/}{str} {or}
\href{/docs/reference/foundations/bytes/}{bytes}

{Required} {{ Positional }}

\phantomsection\label{definitions-decode-data-positional-tooltip}
Positional parameters are specified in order, without names.

CSV data.

\paragraph{\texorpdfstring{\texttt{\ delimiter\ }}{ delimiter }}\label{definitions-decode-delimiter}

\href{/docs/reference/foundations/str/}{str}

The delimiter that separates columns in the CSV file. Must be a single
ASCII character.

Default: \texttt{\ }{\texttt{\ ","\ }}\texttt{\ }

\paragraph{\texorpdfstring{\texttt{\ row-type\ }}{ row-type }}\label{definitions-decode-row-type}

\href{/docs/reference/foundations/type/}{type}

How to represent the file\textquotesingle s rows.

\begin{itemize}
\tightlist
\item
  If set to \texttt{\ array\ } , each row is represented as a plain
  array of strings.
\item
  If set to \texttt{\ dictionary\ } , each row is represented as a
  dictionary mapping from header keys to strings. This option only makes
  sense when a header row is present in the CSV file.
\end{itemize}

Default: \texttt{\ array\ }

\href{/docs/reference/data-loading/cbor/}{\pandocbounded{\includesvg[keepaspectratio]{/assets/icons/16-arrow-right.svg}}}

{ CBOR } { Previous page }

\href{/docs/reference/data-loading/json/}{\pandocbounded{\includesvg[keepaspectratio]{/assets/icons/16-arrow-right.svg}}}

{ JSON } { Next page }


\title{typst.app/docs/reference/data-loading/toml}

\begin{itemize}
\tightlist
\item
  \href{/docs}{\includesvg[width=0.16667in,height=0.16667in]{/assets/icons/16-docs-dark.svg}}
\item
  \includesvg[width=0.16667in,height=0.16667in]{/assets/icons/16-arrow-right.svg}
\item
  \href{/docs/reference/}{Reference}
\item
  \includesvg[width=0.16667in,height=0.16667in]{/assets/icons/16-arrow-right.svg}
\item
  \href{/docs/reference/data-loading/}{Data Loading}
\item
  \includesvg[width=0.16667in,height=0.16667in]{/assets/icons/16-arrow-right.svg}
\item
  \href{/docs/reference/data-loading/toml/}{TOML}
\end{itemize}

\section{\texorpdfstring{\texttt{\ toml\ }}{ toml }}\label{summary}

Reads structured data from a TOML file.

The file must contain a valid TOML table. TOML tables will be converted
into Typst dictionaries, and TOML arrays will be converted into Typst
arrays. Strings, booleans and datetimes will be converted into the Typst
equivalents and numbers will be converted to floats or integers
depending on whether they are whole numbers.

The TOML file in the example consists of a table with the keys
\texttt{\ title\ } , \texttt{\ version\ } , and \texttt{\ authors\ } .

\subsection{Example}\label{example}

\begin{verbatim}
#let details = toml("details.toml")

Title: #details.title \
Version: #details.version \
Authors: #(details.authors
  .join(", ", last: " and "))
\end{verbatim}

\includegraphics[width=5in,height=\textheight,keepaspectratio]{/assets/docs/f26frHBWUfr7bIomQ1qwWAAAAAAAAAAA.png}

\subsection{\texorpdfstring{{ Parameters
}}{ Parameters }}\label{parameters}

\phantomsection\label{parameters-tooltip}
Parameters are the inputs to a function. They are specified in
parentheses after the function name.

{ toml } (

{ \href{/docs/reference/foundations/str/}{str} }

) -\textgreater{} { any }

\subsubsection{\texorpdfstring{\texttt{\ path\ }}{ path }}\label{parameters-path}

\href{/docs/reference/foundations/str/}{str}

{Required} {{ Positional }}

\phantomsection\label{parameters-path-positional-tooltip}
Positional parameters are specified in order, without names.

Path to a TOML file.

For more details, see the \href{/docs/reference/syntax/\#paths}{Paths
section} .

\subsection{\texorpdfstring{{ Definitions
}}{ Definitions }}\label{definitions}

\phantomsection\label{definitions-tooltip}
Functions and types and can have associated definitions. These are
accessed by specifying the function or type, followed by a period, and
then the definition\textquotesingle s name.

\subsubsection{\texorpdfstring{\texttt{\ decode\ }}{ decode }}\label{definitions-decode}

Reads structured data from a TOML string/bytes.

toml { . } { decode } (

{ \href{/docs/reference/foundations/str/}{str}
\href{/docs/reference/foundations/bytes/}{bytes} }

) -\textgreater{} { any }

\paragraph{\texorpdfstring{\texttt{\ data\ }}{ data }}\label{definitions-decode-data}

\href{/docs/reference/foundations/str/}{str} {or}
\href{/docs/reference/foundations/bytes/}{bytes}

{Required} {{ Positional }}

\phantomsection\label{definitions-decode-data-positional-tooltip}
Positional parameters are specified in order, without names.

TOML data.

\subsubsection{\texorpdfstring{\texttt{\ encode\ }}{ encode }}\label{definitions-encode}

Encodes structured data into a TOML string.

toml { . } { encode } (

{ { any } , } { \hyperref[definitions-encode-parameters-pretty]{pretty
:} \href{/docs/reference/foundations/bool/}{bool} , }

) -\textgreater{} \href{/docs/reference/foundations/str/}{str}

\paragraph{\texorpdfstring{\texttt{\ value\ }}{ value }}\label{definitions-encode-value}

{ any }

{Required} {{ Positional }}

\phantomsection\label{definitions-encode-value-positional-tooltip}
Positional parameters are specified in order, without names.

Value to be encoded.

\paragraph{\texorpdfstring{\texttt{\ pretty\ }}{ pretty }}\label{definitions-encode-pretty}

\href{/docs/reference/foundations/bool/}{bool}

Whether to pretty-print the resulting TOML.

Default: \texttt{\ }{\texttt{\ true\ }}\texttt{\ }

\href{/docs/reference/data-loading/read/}{\pandocbounded{\includesvg[keepaspectratio]{/assets/icons/16-arrow-right.svg}}}

{ Read } { Previous page }

\href{/docs/reference/data-loading/xml/}{\pandocbounded{\includesvg[keepaspectratio]{/assets/icons/16-arrow-right.svg}}}

{ XML } { Next page }


\title{typst.app/docs/reference/data-loading/cbor}

\begin{itemize}
\tightlist
\item
  \href{/docs}{\includesvg[width=0.16667in,height=0.16667in]{/assets/icons/16-docs-dark.svg}}
\item
  \includesvg[width=0.16667in,height=0.16667in]{/assets/icons/16-arrow-right.svg}
\item
  \href{/docs/reference/}{Reference}
\item
  \includesvg[width=0.16667in,height=0.16667in]{/assets/icons/16-arrow-right.svg}
\item
  \href{/docs/reference/data-loading/}{Data Loading}
\item
  \includesvg[width=0.16667in,height=0.16667in]{/assets/icons/16-arrow-right.svg}
\item
  \href{/docs/reference/data-loading/cbor/}{CBOR}
\end{itemize}

\section{\texorpdfstring{\texttt{\ cbor\ }}{ cbor }}\label{summary}

Reads structured data from a CBOR file.

The file must contain a valid CBOR serialization. Mappings will be
converted into Typst dictionaries, and sequences will be converted into
Typst arrays. Strings and booleans will be converted into the Typst
equivalents, null-values ( \texttt{\ null\ } ,
\texttt{\ \textasciitilde{}\ } or empty ``) will be converted into
\texttt{\ }{\texttt{\ none\ }}\texttt{\ } , and numbers will be
converted to floats or integers depending on whether they are whole
numbers.

Be aware that integers larger than 2 \textsuperscript{63} -1 will be
converted to floating point numbers, which may result in an
approximative value.

\subsection{\texorpdfstring{{ Parameters
}}{ Parameters }}\label{parameters}

\phantomsection\label{parameters-tooltip}
Parameters are the inputs to a function. They are specified in
parentheses after the function name.

{ cbor } (

{ \href{/docs/reference/foundations/str/}{str} }

) -\textgreater{} { any }

\subsubsection{\texorpdfstring{\texttt{\ path\ }}{ path }}\label{parameters-path}

\href{/docs/reference/foundations/str/}{str}

{Required} {{ Positional }}

\phantomsection\label{parameters-path-positional-tooltip}
Positional parameters are specified in order, without names.

Path to a CBOR file.

For more details, see the \href{/docs/reference/syntax/\#paths}{Paths
section} .

\subsection{\texorpdfstring{{ Definitions
}}{ Definitions }}\label{definitions}

\phantomsection\label{definitions-tooltip}
Functions and types and can have associated definitions. These are
accessed by specifying the function or type, followed by a period, and
then the definition\textquotesingle s name.

\subsubsection{\texorpdfstring{\texttt{\ decode\ }}{ decode }}\label{definitions-decode}

Reads structured data from CBOR bytes.

cbor { . } { decode } (

{ \href{/docs/reference/foundations/bytes/}{bytes} }

) -\textgreater{} { any }

\paragraph{\texorpdfstring{\texttt{\ data\ }}{ data }}\label{definitions-decode-data}

\href{/docs/reference/foundations/bytes/}{bytes}

{Required} {{ Positional }}

\phantomsection\label{definitions-decode-data-positional-tooltip}
Positional parameters are specified in order, without names.

cbor data.

\subsubsection{\texorpdfstring{\texttt{\ encode\ }}{ encode }}\label{definitions-encode}

Encode structured data into CBOR bytes.

cbor { . } { encode } (

{ { any } }

) -\textgreater{} \href{/docs/reference/foundations/bytes/}{bytes}

\paragraph{\texorpdfstring{\texttt{\ value\ }}{ value }}\label{definitions-encode-value}

{ any }

{Required} {{ Positional }}

\phantomsection\label{definitions-encode-value-positional-tooltip}
Positional parameters are specified in order, without names.

Value to be encoded.

\href{/docs/reference/data-loading/}{\pandocbounded{\includesvg[keepaspectratio]{/assets/icons/16-arrow-right.svg}}}

{ Data Loading } { Previous page }

\href{/docs/reference/data-loading/csv/}{\pandocbounded{\includesvg[keepaspectratio]{/assets/icons/16-arrow-right.svg}}}

{ CSV } { Next page }


\title{typst.app/docs/reference/data-loading/yaml}

\begin{itemize}
\tightlist
\item
  \href{/docs}{\includesvg[width=0.16667in,height=0.16667in]{/assets/icons/16-docs-dark.svg}}
\item
  \includesvg[width=0.16667in,height=0.16667in]{/assets/icons/16-arrow-right.svg}
\item
  \href{/docs/reference/}{Reference}
\item
  \includesvg[width=0.16667in,height=0.16667in]{/assets/icons/16-arrow-right.svg}
\item
  \href{/docs/reference/data-loading/}{Data Loading}
\item
  \includesvg[width=0.16667in,height=0.16667in]{/assets/icons/16-arrow-right.svg}
\item
  \href{/docs/reference/data-loading/yaml/}{YAML}
\end{itemize}

\section{\texorpdfstring{\texttt{\ yaml\ }}{ yaml }}\label{summary}

Reads structured data from a YAML file.

The file must contain a valid YAML object or array. YAML mappings will
be converted into Typst dictionaries, and YAML sequences will be
converted into Typst arrays. Strings and booleans will be converted into
the Typst equivalents, null-values ( \texttt{\ null\ } ,
\texttt{\ \textasciitilde{}\ } or empty ``) will be converted into
\texttt{\ }{\texttt{\ none\ }}\texttt{\ } , and numbers will be
converted to floats or integers depending on whether they are whole
numbers. Custom YAML tags are ignored, though the loaded value will
still be present.

Be aware that integers larger than 2 \textsuperscript{63} -1 will be
converted to floating point numbers, which may give an approximative
value.

The YAML files in the example contain objects with authors as keys, each
with a sequence of their own submapping with the keys "title" and
"published"

\subsection{Example}\label{example}

\begin{verbatim}
#let bookshelf(contents) = {
  for (author, works) in contents {
    author
    for work in works [
      - #work.title (#work.published)
    ]
  }
}

#bookshelf(
  yaml("scifi-authors.yaml")
)
\end{verbatim}

\includegraphics[width=5in,height=\textheight,keepaspectratio]{/assets/docs/zhzvOjbNeHnb4ZYJg032GwAAAAAAAAAA.png}

\subsection{\texorpdfstring{{ Parameters
}}{ Parameters }}\label{parameters}

\phantomsection\label{parameters-tooltip}
Parameters are the inputs to a function. They are specified in
parentheses after the function name.

{ yaml } (

{ \href{/docs/reference/foundations/str/}{str} }

) -\textgreater{} { any }

\subsubsection{\texorpdfstring{\texttt{\ path\ }}{ path }}\label{parameters-path}

\href{/docs/reference/foundations/str/}{str}

{Required} {{ Positional }}

\phantomsection\label{parameters-path-positional-tooltip}
Positional parameters are specified in order, without names.

Path to a YAML file.

For more details, see the \href{/docs/reference/syntax/\#paths}{Paths
section} .

\subsection{\texorpdfstring{{ Definitions
}}{ Definitions }}\label{definitions}

\phantomsection\label{definitions-tooltip}
Functions and types and can have associated definitions. These are
accessed by specifying the function or type, followed by a period, and
then the definition\textquotesingle s name.

\subsubsection{\texorpdfstring{\texttt{\ decode\ }}{ decode }}\label{definitions-decode}

Reads structured data from a YAML string/bytes.

yaml { . } { decode } (

{ \href{/docs/reference/foundations/str/}{str}
\href{/docs/reference/foundations/bytes/}{bytes} }

) -\textgreater{} { any }

\paragraph{\texorpdfstring{\texttt{\ data\ }}{ data }}\label{definitions-decode-data}

\href{/docs/reference/foundations/str/}{str} {or}
\href{/docs/reference/foundations/bytes/}{bytes}

{Required} {{ Positional }}

\phantomsection\label{definitions-decode-data-positional-tooltip}
Positional parameters are specified in order, without names.

YAML data.

\subsubsection{\texorpdfstring{\texttt{\ encode\ }}{ encode }}\label{definitions-encode}

Encode structured data into a YAML string.

yaml { . } { encode } (

{ { any } }

) -\textgreater{} \href{/docs/reference/foundations/str/}{str}

\paragraph{\texorpdfstring{\texttt{\ value\ }}{ value }}\label{definitions-encode-value}

{ any }

{Required} {{ Positional }}

\phantomsection\label{definitions-encode-value-positional-tooltip}
Positional parameters are specified in order, without names.

Value to be encoded.

\href{/docs/reference/data-loading/xml/}{\pandocbounded{\includesvg[keepaspectratio]{/assets/icons/16-arrow-right.svg}}}

{ XML } { Previous page }

\href{/docs/guides/}{\pandocbounded{\includesvg[keepaspectratio]{/assets/icons/16-arrow-right.svg}}}

{ Guides } { Next page }


\title{typst.app/docs/reference/data-loading/read}

\begin{itemize}
\tightlist
\item
  \href{/docs}{\includesvg[width=0.16667in,height=0.16667in]{/assets/icons/16-docs-dark.svg}}
\item
  \includesvg[width=0.16667in,height=0.16667in]{/assets/icons/16-arrow-right.svg}
\item
  \href{/docs/reference/}{Reference}
\item
  \includesvg[width=0.16667in,height=0.16667in]{/assets/icons/16-arrow-right.svg}
\item
  \href{/docs/reference/data-loading/}{Data Loading}
\item
  \includesvg[width=0.16667in,height=0.16667in]{/assets/icons/16-arrow-right.svg}
\item
  \href{/docs/reference/data-loading/read/}{Read}
\end{itemize}

\section{\texorpdfstring{\texttt{\ read\ }}{ read }}\label{summary}

Reads plain text or data from a file.

By default, the file will be read as UTF-8 and returned as a
\href{/docs/reference/foundations/str/}{string} .

If you specify \texttt{\ encoding:\ }{\texttt{\ none\ }}\texttt{\ } ,
this returns raw \href{/docs/reference/foundations/bytes/}{bytes}
instead.

\subsection{Example}\label{example}

\begin{verbatim}
An example for a HTML file: \
#let text = read("example.html")
#raw(text, lang: "html")

Raw bytes:
#read("tiger.jpg", encoding: none)
\end{verbatim}

\includegraphics[width=5in,height=\textheight,keepaspectratio]{/assets/docs/uS5DrZwzU2PIqO_vdJc7GQAAAAAAAAAA.png}

\subsection{\texorpdfstring{{ Parameters
}}{ Parameters }}\label{parameters}

\phantomsection\label{parameters-tooltip}
Parameters are the inputs to a function. They are specified in
parentheses after the function name.

{ read } (

{ \href{/docs/reference/foundations/str/}{str} , } {
\hyperref[parameters-encoding]{encoding :}
\href{/docs/reference/foundations/none/}{none}
\href{/docs/reference/foundations/str/}{str} , }

) -\textgreater{} \href{/docs/reference/foundations/str/}{str}
\href{/docs/reference/foundations/bytes/}{bytes}

\subsubsection{\texorpdfstring{\texttt{\ path\ }}{ path }}\label{parameters-path}

\href{/docs/reference/foundations/str/}{str}

{Required} {{ Positional }}

\phantomsection\label{parameters-path-positional-tooltip}
Positional parameters are specified in order, without names.

Path to a file.

For more details, see the \href{/docs/reference/syntax/\#paths}{Paths
section} .

\subsubsection{\texorpdfstring{\texttt{\ encoding\ }}{ encoding }}\label{parameters-encoding}

\href{/docs/reference/foundations/none/}{none} {or}
\href{/docs/reference/foundations/str/}{str}

The encoding to read the file with.

If set to \texttt{\ }{\texttt{\ none\ }}\texttt{\ } , this function
returns raw bytes.

\begin{longtable}[]{@{}ll@{}}
\toprule\noalign{}
Variant & Details \\
\midrule\noalign{}
\endhead
\bottomrule\noalign{}
\endlastfoot
\texttt{\ "\ utf8\ "\ } & The Unicode UTF-8 encoding. \\
\end{longtable}

Default: \texttt{\ }{\texttt{\ "utf8"\ }}\texttt{\ }

\href{/docs/reference/data-loading/json/}{\pandocbounded{\includesvg[keepaspectratio]{/assets/icons/16-arrow-right.svg}}}

{ JSON } { Previous page }

\href{/docs/reference/data-loading/toml/}{\pandocbounded{\includesvg[keepaspectratio]{/assets/icons/16-arrow-right.svg}}}

{ TOML } { Next page }


