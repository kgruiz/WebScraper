\title{typst.app/universe/package/modern-cug-thesis}

\phantomsection\label{banner}
\phantomsection\label{template-thumbnail}
\pandocbounded{\includegraphics[keepaspectratio]{https://packages.typst.org/preview/thumbnails/modern-cug-thesis-0.1.0-small.webp}}

\section{modern-cug-thesis}\label{modern-cug-thesis}

{ 0.1.0 }

中国地质大学(武汉)学ä½?论æ--‡æ¨¡æ?¿ã€‚China University of
Geosciences Thesis based on Typst.

\href{/app?template=modern-cug-thesis&version=0.1.0}{Create project in
app}

\phantomsection\label{readme}
\textbf{cug-thesis-thesis}
适ç''¨äºŽä¸­å›½åœ°è´¨å¤§å­¦ï¼ˆæ­¦æ±‰ï¼‰å­¦ä½?论æ--‡æ¨¡æ?¿ï¼Œå\ldots·æœ‰ä¾¿æ?·ã€?简å?•ã€?实æ---¶æ¸²æŸ``等特性。欢迎å?„ä½?å?Œå­¦ã€?æ~¡å?‹ä»¬å‰?æ?¥
\href{https://github.com/Rsweater/cug-thesis-typst/issues}{Issues}
交æµ?å­¦ä¹~\textasciitilde{}

\pandocbounded{\includegraphics[keepaspectratio]{https://cdn.jsdelivr.net/gh/Rsweater/images/img/preview.gif}}

\subsection{为什么考è™` Typst
实现学ä½?论æ--‡æ¨¡æ?¿ï¼Ÿ}\label{uxe4uxbauxe4uxe4uxb9ux2c6uxe8ux192uxe8-typst-uxe5ux17euxe7ux17euxe5uxe4uxbduxe8uxbauxe6uxe6uxe6uxefuxbcuxff}

\begin{enumerate}
\tightlist
\item
  é¦--è¦?是为了学ä¹~。看到 Typst
  惊人的�长速度,确实有点�激动。Typst 似乎继承了
  Markdown�Tex�Wiki �自的优点于自身。
\item
  本人写æ--‡æ¡£ç›¸å¯¹æ?¥è¯´è¾ƒä¸ºç²---心,使ç''¨ Word
  模æ?¿ä¼šå¿?ä¸?ä½?çš„å??å¤?去检查æ~¼å¼?是å?¦ç¬¦å?ˆè¦?求。å?ˆå?¬è¯´
  Latex
  写毕业论æ--‡å?¯èƒ½å?Žé?¢ç¼--è¯`一次需è¦?å‡~å??ç§'\textasciitilde\textasciitilde{}
  虽然这个自己å?ªæ˜¯å?¬è¯´ï¼Œä½†æ˜¯ LaTex 在线ç¼--è¾`çš„æ--¹å¼?
  Overleaf
  达到一定的ç¼--è¯`æ---¶é---´æ''¶è´¹è¿™ä¸ªæ˜¯çœŸçš„,就æˆ`å°?论æ--‡éƒ½å‹‰å¼ºå¤Ÿç''¨ã€‚自己使ç''¨å¼€æº?çš„
  Overleaf
  �建的平�功能上总是缺点什么,奈何自己��懂\textasciitilde{}
  自己�建的本地的 Tex
  环境éš?解决了ç¼--è¯`æ---¶é---´ä»˜è´¹é---®é¢˜ã€‚但是涉å?Šåˆ°çš„å®?åŒ\ldots ã€?环境,å‰?段æ---¶é---´æ‰``å¼€çª?然ä¸?能ç''¨äº†ï¼Œæ?£é¼``å?Šå¤©ä¸?知是何原å›~\textasciitilde{}
  直至é‡?æ--°è£\ldots 了2024å¹´çš„ LaTex
  环境æ‰?é‡?æ--°è¿?行自己的学ä½?论æ--‡ã€‚
\item
  惊å--œçš„å?{}`现 Typst ç¼--è¯`速度真的é?žå¸¸å¿«\textasciitilde{}
  ç»?过一段æ---¶é---´çš„了解,å?{}`现基本满足制作学ä½?论æ--‡çš„需求,于是乎\textasciitilde 就有了这个cug-thesis-typst。
\end{enumerate}

\subsection{�考规范}\label{uxe5uxe8ux192uxe8uxe8ux153ux192}

\begin{itemize}
\tightlist
\item
  本ç§`ç''Ÿè®ºæ--‡æ¨¡æ?¿å?‚考:
  \href{https://bksy.cug.edu.cn/info/1489/1851.htm}{学士学ä½?论æ--‡å†™ä½œè§„范(2018版)-中国地质大学本ç§`ç''Ÿé™¢}
\item
  ç~''究ç''Ÿè®ºæ--‡æ¨¡æ?¿å?‚考:
  \href{https://xgxy.cug.edu.cn/info/1073/3509.htm}{ç~''究ç''Ÿå­¦ä½?论æ--‡å†™ä½œè§„范(2015版)-中国地质大学地ç?†ä¸Žä¿¡æ?¯å·¥ç¨‹å­¦é™¢}
  (对ç~''究ç''Ÿé™¢ç›¸å\ldots³é€šçŸ¥é™„件进行了整ç?†ï¼‰
\end{itemize}

\subsection{模æ?¿è®¤å?¯åº¦é---®é¢˜}\label{uxe6uxe6uxe8uxe5uxe5uxbauxe9uxe9}

\textbf{值å¾---æ??é†'的是}
毕竟是æ°`é---´å®žçŽ°æ¨¡æ?¿ï¼Œæœ‰ä¸?被学院认å?¯çš„å?¯èƒ½æ€§\textasciitilde{}

ç›®å‰?已知æƒ\ldots 况,计ç®---机学院ã€?地信学院对于学ä½?论æ--‡è¦?求ä¸?是太苛刻。去年计ç®---机学院师å\ldots„使ç''¨äº†
Github 的 Latex 模�
\href{https://github.com/Timozer/CUGThesis}{Timozer/CUGThesis:
中国地质大学(武汉)ç~''究ç''Ÿå­¦ä½?论æ--‡ TeX 模æ?¿}
完æˆ?å­¦ä½?论æ--‡ã€‚

而ä¸''å``ˆ\textasciitilde{}
å'±ä»¬çš„ç~''究ç''Ÿå­¦ä½?论æ--‡å†™ä½œè§„范(2015版)似乎è¦?求似乎ä¸?是特别苛刻。请自行æ--Ÿé\ldots Œ\textasciitilde{}

\begin{quote}
�声\textasciitilde{}
ç~''究ç''Ÿå­¦ä½?论æ--‡å†™ä½œè§„范(2015版)似乎还有一处å‰?å?ŽçŸ›ç›¾çš„è¦?求,æ--¯\textasciitilde{}
\end{quote}

\subsection{使ç''¨æ--¹æ³•}\label{uxe4uxbduxe7uxe6uxb9uxe6uxb3}

\subsubsection{Typst
在线ç¼--è¾`}\label{typst-uxe5ux153uxe7uxbauxe7uxbcuxe8uxbe}

本模æ?¿å·²ä¸Šä¼ \href{https://typst.app/universe}{Typst Universe}
,您å?¯ä»¥ä½¿ç''¨ Typst 的官æ--¹ Web App
进行ç¼--è¾`。å?ªéœ€è¦?在 \href{https://typst.app/}{Typst Web App}
中的 \texttt{\ Start\ from\ template\ } 里选择
\texttt{\ modern-cug-thesis\ } ,��从模�创建项目。

æˆ--è€\ldots ,直接点击注册
\href{https://typst.app/app?template=modern-cug-thesis&version=0.1.0}{typst.app.universe.cug-thesis}
,并开始ç¼--写ä½~的论æ--‡\textasciitilde{}

\subsubsection{如果ä½~ç»?常使ç''¨ VS
Code,也æ¯''较推è??使ç''¨è¿™ä¸ª\textasciitilde{}}\label{uxe5uxe6ux17eux153uxe4uxbd-uxe7uxe5uxe4uxbduxe7-vs-codeuxefuxbcux153uxe4uxb9uxffuxe6uxe8uxbeux192uxe6ux17euxe8uxe4uxbduxe7uxe8uxe4uxaa}

\textbf{使ç''¨æ­¥éª¤} :安è£\ldots{} typst (å`½ä»¤è¡Œå·¥å\ldots·) â†'
VS Code
æ?'件(实æ---¶é¢„览ã€?智能æ??é†'),éš?å?Žå°±å?¯ä»¥å‡†å¤‡å¼€å§‹é¡¹ç›®äº†(æ‰``开项目æ--‡ä»¶ã€?æ'°å†™è®ºæ--‡å†\ldots 容)

\begin{enumerate}
\item
  \textbf{安è£\ldots{} typst :}

  \begin{itemize}
  \tightlist
  \item
    \textbf{macOS:} \texttt{\ brew\ install\ typst\ }
  \item
    \textbf{Windows:}
    \texttt{\ winget\ install\ -\/-id\ Typst.Typst\ -l\ "D:\textbackslash{}bw\_ch\textbackslash{}toolkits\textbackslash{}typst"\ }
  \end{itemize}
\item
  \textbf{安è£\ldots æ?'件} :在 VS Code 中安è£
  \href{https://marketplace.visualstudio.com/items?itemName=myriad-dreamin.tinymist}{Tinymist
  Typst}
\item
  \textbf{准备项目æ--‡ä»¶} :

  \begin{itemize}
  \tightlist
  \item
    \textbf{æ--¹æ³•ä¸€ï¼šClone Repo} : 使ç''¨å`½ä»¤
    \texttt{\ git\ clone\ https://github.com/Rsweater/cug-thesis-typst.git\ }
    将整个项目å\ldots‹éš†åˆ°æœ¬åœ°ï¼Œå¯»æ‰¾
    \texttt{\ template/thesis.typ\ } 。
  \item
    \textbf{æ--¹æ³•äºŒï¼šä½¿ç''¨ Typst Packages} :按下
    \texttt{\ Ctrl\ +\ Shift\ +\ P\ } æ‰``å¼€å`½ä»¤ç•Œé?¢ï¼Œè¾``å\ldots¥
    \texttt{\ Typst:\ Show\ available\ Typst\ templates\ (gallery)\ for\ picking\ up\ a\ template\ }
    æ‰``å¼€ Tinymist æ??供的 Template Gallery,然å?Žä»Žé‡Œé?¢æ‰¾åˆ°
    \texttt{\ cug-thesis\ } ,点击 \texttt{\ �\ }
    按é'®è¿›è¡Œæ''¶è---?,以å?Šç‚¹å‡» \texttt{\ +\ }
    å?·ï¼Œå°±å?¯ä»¥åˆ›å»ºå¯¹åº''的论æ--‡æ¨¡æ?¿äº†ï¼Œä¼šå‡ºçŽ°
    \texttt{\ ref.bib\ } 以� \texttt{\ thesis.typ\ } 。
  \end{itemize}
\item
  æ‰``开开始ç¼--写论æ--‡å†\ldots 容\textasciitilde{}
\end{enumerate}

\subsection{Q\&A}\label{qa}

\subsubsection{使ç''¨è¿™ä¸ªæ¨¡æ?¿éœ€è¦?了解些什么?}\label{uxe4uxbduxe7uxe8uxe4uxaauxe6uxe6uxe9ux153uxe8uxe4uxbauxe8uxe4uxbauxe4uxe4uxb9ux2c6uxefuxbcuxff}

需�掌�一些 Markdown Like
æ~‡è®°ç''¨æ?¥ç¼--写æ--‡æ¡£ï¼Œäº†è§£æ--‡ç«~大致ç»``æž„å?³å?¯ã€?è§?
\texttt{\ template\textbackslash{}thesis.typ\ } 中介ç»?ã€`。

\textbf{å?‚考资æ--™ï¼š}

\begin{itemize}
\tightlist
\item
  官ç½`Tutorial:
  \href{https://typst.app/docs/tutorial/writing-in-typst/}{Writing in
  Typst â€`` Typst Documentation} ã€?
  \href{https://typst-doc-cn.github.io/docs/tutorial/writing-in-typst/}{Tutorial中æ--‡ç¿»è¯`}
\item
  Typst 语法官ç½`æ--‡æ¡£ï¼š
  \href{https://typst.app/docs/reference/syntax/}{Syntax â€`` Typst
  Documentation} �
  \href{https://typst-doc-cn.github.io/docs/reference/syntax/}{语法中æ--‡ç¿»è¯`}
\item
  中æ--‡ç¤¾åŒºå°?è``?书:
  \href{https://typst-doc-cn.github.io/tutorial/basic/writing-markup.html}{The
  Raindrop-Blue Book (Typst中æ--‡æ•™ç¨‹)}
\end{itemize}

\subsubsection{æˆ`ä¸?会代ç~?ã€?ä¸?会 LaTeX
å?¯ä»¥ä½¿ç''¨å?---?从接触到使ç''¨éœ€è¦?多ä¹\ldots ?}\label{uxe6ux2c6uxe4uxe4uxbcux161uxe4uxe7-uxe3uxe4uxe4uxbcux161-latex-uxe5uxe4uxe4uxbduxe7uxe5uxefuxbcuxffuxe4ux17euxe6ux17euxe8uxe5ux2c6uxe4uxbduxe7uxe9ux153uxe8uxe5ux161uxe4uxb9uxefuxbcuxff}

å?¯ä»¥çš„。å›~为æ--‡æ¡£æ~·å¼?该模æ?¿å·²ç»?æ??供,Typst
有æ~‡è®°æ¨¡å¼?(语法ç³--),使ç''¨èµ·æ?¥å°±ç±»ä¼¼äºŽ
Markdown,完å\ldots¨ä¸?需è¦?较多的代ç~?功底。

如果有 Markdown 基础,基本上�以直接上手\textasciitilde{}
如果没有,跳回第一个é---®é¢˜ï¼ŒæŸ¥çœ‹ç›¸å\ldots³è¯´æ˜Žã€‚

\subsubsection{Typst 是个啥玩�?相较于 LaTeX
有啥优势?}\label{typst-uxe6uxe4uxaauxe5uxe7ux17euxe6uxefuxbcuxffuxe7uxe8uxbeux192uxe4uxbaux17e-latex-uxe6ux153uxe5uxe4uxbcuxe5ux161uxefuxbcuxff}

\textbf{æ??供两篇} 写的很ç''¨å¿ƒçš„ \textbf{æ--‡ç«~:}

\begin{itemize}
\tightlist
\item
  \href{https://mp.weixin.qq.com/s/58IYHA3pROuh4iDHB4o1Vw}{探索
  Typst,一ç§?类似于 LaTeX çš„æ--°æŽ'版系统} (è¯`æ--‡ï¼‰ã€?
  \href{https://blog.jreyesr.com/posts/typst/}{原æ--‡}
\item
  \href{https://zhuanlan.zhihu.com/p/669097092}{Typst
  中æ--‡ç''¨æˆ·ä½¿ç''¨ä½``验 - OrangeX4大佬}
\end{itemize}

\subsection{致谢}\label{uxe8uxe8}

\begin{itemize}
\tightlist
\item
  æ„Ÿè°¢
  \href{https://github.com/nju-lug/modern-nju-thesis}{modern-nju-thesis}
  � \href{https://github.com/sysu/better-thesis}{better-thesis} �
  \href{https://github.com/hitszosa/universal-hit-thesis}{HIT-Thesis-Typst}
  为本模æ?¿æ??供了项目实现æ€?路。
\item
  æ„Ÿè°¢ \href{https://github.com/Timozer/CUGThesis}{Timozer/CUGThesis:
  中国地质大学(武汉)ç~''究ç''Ÿå­¦ä½?论æ--‡ TeX 模æ?¿}
  æ??供了页é?¢å¸ƒå±€ä¾?æ?®ã€‚
\item
  æ„Ÿè°¢ \href{https://jq.qq.com/?_wv=1027&k=m58va1kd}{Typst
  é?žå®˜æ--¹ä¸­æ--‡äº¤æµ?群} 中大佬的热心帮助。
\end{itemize}

\subsection{License}\label{license}

This project is licensed under the MIT License.

\href{/app?template=modern-cug-thesis&version=0.1.0}{Create project in
app}

\subsubsection{How to use}\label{how-to-use}

Click the button above to create a new project using this template in
the Typst app.

You can also use the Typst CLI to start a new project on your computer
using this command:

\begin{verbatim}
typst init @preview/modern-cug-thesis:0.1.0
\end{verbatim}

\includesvg[width=0.16667in,height=0.16667in]{/assets/icons/16-copy.svg}

\subsubsection{About}\label{about}

\begin{description}
\tightlist
\item[Author :]
Brevin
\item[License:]
MIT
\item[Current version:]
0.1.0
\item[Last updated:]
November 28, 2024
\item[First released:]
November 28, 2024
\item[Archive size:]
226 kB
\href{https://packages.typst.org/preview/modern-cug-thesis-0.1.0.tar.gz}{\pandocbounded{\includesvg[keepaspectratio]{/assets/icons/16-download.svg}}}
\item[Repository:]
\href{https://github.com/Rsweater/cug-thesis-typst}{GitHub}
\item[Categor y :]
\begin{itemize}
\tightlist
\item[]
\item
  \pandocbounded{\includesvg[keepaspectratio]{/assets/icons/16-mortarboard.svg}}
  \href{https://typst.app/universe/search/?category=thesis}{Thesis}
\end{itemize}
\end{description}

\subsubsection{Where to report issues?}\label{where-to-report-issues}

This template is a project of Brevin . Report issues on
\href{https://github.com/Rsweater/cug-thesis-typst}{their repository} .
You can also try to ask for help with this template on the
\href{https://forum.typst.app}{Forum} .

Please report this template to the Typst team using the
\href{https://typst.app/contact}{contact form} if you believe it is a
safety hazard or infringes upon your rights.

\phantomsection\label{versions}
\subsubsection{Version history}\label{version-history}

\begin{longtable}[]{@{}ll@{}}
\toprule\noalign{}
Version & Release Date \\
\midrule\noalign{}
\endhead
\bottomrule\noalign{}
\endlastfoot
0.1.0 & November 28, 2024 \\
\end{longtable}

Typst GmbH did not create this template and cannot guarantee correct
functionality of this template or compatibility with any version of the
Typst compiler or app.


\title{typst.app/universe/package/modern-resume}

\phantomsection\label{banner}
\phantomsection\label{template-thumbnail}
\pandocbounded{\includegraphics[keepaspectratio]{https://packages.typst.org/preview/thumbnails/modern-resume-0.1.0-small.webp}}

\section{modern-resume}\label{modern-resume}

{ 0.1.0 }

A modern resume/CV template.

\href{/app?template=modern-resume&version=0.1.0}{Create project in app}

\phantomsection\label{readme}
\href{https://github.com/peterpf/modern-typst-resume/stargazers}{\pandocbounded{\includesvg[keepaspectratio]{https://img.shields.io/badge/Say\%20Thanks-\%F0\%9F\%91\%8D-1EAEDB.svg}}}

A customizable resume/CV template focusing on clean and concise
presentation, with a touch of color.

\subsection{Requirements}\label{requirements}

To compile this project you need the following:

\begin{itemize}
\tightlist
\item
  Typst
\item
  Roboto font family
\end{itemize}

\subsection{Compiling}\label{compiling}

Build the document once with

\begin{Shaded}
\begin{Highlighting}[]
\ExtensionTok{typst}\NormalTok{ compile main.typ}
\end{Highlighting}
\end{Shaded}

Build the document whenever you save changes by running

\begin{Shaded}
\begin{Highlighting}[]
\ExtensionTok{typst}\NormalTok{ watch main.typ}
\end{Highlighting}
\end{Shaded}

\subsection{Usage}\label{usage}

The following code provides a minimum working example:

\begin{Shaded}
\begin{Highlighting}[]
\NormalTok{\#import "@preview/modern{-}resume": *}

\NormalTok{\#show: modern{-}resume.with(}
\NormalTok{  author: "John Doe",           // Optional parameter}
\NormalTok{  job{-}title: "Data Scientist",  // Optional parameter}
\NormalTok{  bio: lorem(5),                // Optional parameter}
\NormalTok{  avatar: image("avatar.png"),  // Optional parameter}
\NormalTok{  contact{-}options: (            // All entries are optional}
\NormalTok{    email: link("mailto:john.doe@gmail.com")[john.doe\textbackslash{}@gmail.com],}
\NormalTok{    mobile: "+43 1234 5678",}
\NormalTok{    location: "Austria",}
\NormalTok{    linkedin: link("https://www.linkedin.com/in/jdoe")[linkedin/jdoe],}
\NormalTok{    github: link("https://github.com/jdoe")[github.com/jdoe],}
\NormalTok{    website: link("https://jdoe.dev")[jdoe.dev],}
\NormalTok{  ),}
\NormalTok{)}

\NormalTok{== Education}

\NormalTok{\#experience{-}edu(}
\NormalTok{  title: "Master\textquotesingle{}s degree",}
\NormalTok{  subtitle: "University of Sciences",}
\NormalTok{  task{-}description: [}
\NormalTok{    {-} Short summary of the most important courses}
\NormalTok{    {-} Explanation of master thesis topic}
\NormalTok{  ],}
\NormalTok{  date{-}from: "10/2021",}
\NormalTok{  date{-}to: "07/2023",}
\NormalTok{)}

\NormalTok{// More content goes here}
\end{Highlighting}
\end{Shaded}

See
\href{https://github.com/typst/packages/raw/main/packages/preview/modern-resume/0.1.0/main.typ}{main.typ}
for a full example that showcases all available elements.

\subsection{Output examples}\label{output-examples}

Example outputs for different color palettes:

\begin{longtable}[]{@{}cc@{}}
\toprule\noalign{}
Default colors & Pink colors \\
\midrule\noalign{}
\endhead
\bottomrule\noalign{}
\endlastfoot
\pandocbounded{\includegraphics[keepaspectratio]{https://github.com/typst/packages/raw/main/packages/preview/modern-resume/0.1.0/docs/images/demo-navy-dark.png}}
&
\pandocbounded{\includegraphics[keepaspectratio]{https://github.com/typst/packages/raw/main/packages/preview/modern-resume/0.1.0/docs/images/demo-pink.png}} \\
\end{longtable}

\subsection{Customization}\label{customization}

Note: customization is currently only supported when cloning the
template locally. Allowing customization via a “Typst
universe�-installed template is a feature that is actively worked on.

The template allows you to make it yours by defining a custom color
palette. Customize the color theme by changing the values of the
\texttt{\ color\ } dictionary in
\href{https://github.com/typst/packages/raw/main/packages/preview/modern-resume/0.1.0/lib.typ}{lib.typ}
. For example:

\begin{itemize}
\item
  The default color palette:

\begin{Shaded}
\begin{Highlighting}[]
\NormalTok{\#let colors = (}
\NormalTok{  primary: rgb("\#313C4E"),}
\NormalTok{  secondary: rgb("\#222A33"),}
\NormalTok{  accent{-}color: rgb("\#449399"),}
\NormalTok{  text{-}primary: black,}
\NormalTok{  text{-}secondary: rgb("\#7C7C7C"),}
\NormalTok{  text{-}tertiary: white,}
\NormalTok{)}
\end{Highlighting}
\end{Shaded}
\item
  A pink color palette:

\begin{Shaded}
\begin{Highlighting}[]
\NormalTok{\#let colors = (}
\NormalTok{  primary: rgb("\#e755e0"),}
\NormalTok{  secondary: rgb("\#ad00c2"),}
\NormalTok{  accent{-}color: rgb("\#00d032"),}
\NormalTok{  text{-}primary: black,}
\NormalTok{  text{-}secondary: rgb("\#7C7C7C"),}
\NormalTok{  text{-}tertiary: white,}
\NormalTok{)}
\end{Highlighting}
\end{Shaded}
\end{itemize}

\subsection{Elements}\label{elements}

This section introduces the visual elements that are part of this
template.

\subsubsection{Pills}\label{pills}

Import this element from the template module with \texttt{\ pill\ } .

\pandocbounded{\includegraphics[keepaspectratio]{https://github.com/typst/packages/raw/main/packages/preview/modern-resume/0.1.0/docs/images/pills.png}}

\begin{Shaded}
\begin{Highlighting}[]
\NormalTok{\#pill("German (native)")}
\NormalTok{\#pill("English (C1)")}
\end{Highlighting}
\end{Shaded}

\pandocbounded{\includegraphics[keepaspectratio]{https://github.com/typst/packages/raw/main/packages/preview/modern-resume/0.1.0/docs/images/pills-filled.png}}

\begin{Shaded}
\begin{Highlighting}[]
\NormalTok{\#pill("Teamwork", fill: true)}
\NormalTok{\#pill("Critical thinking", fill: true)}
\end{Highlighting}
\end{Shaded}

\subsubsection{Educational/work
experience}\label{educationalwork-experience}

Import the elements from the template module with
\texttt{\ experience-edu\ } and \texttt{\ experience-work\ }
respectively.

\pandocbounded{\includegraphics[keepaspectratio]{https://github.com/typst/packages/raw/main/packages/preview/modern-resume/0.1.0/docs/images/educational-experience.png}}

\begin{Shaded}
\begin{Highlighting}[]
\NormalTok{\#experience{-}edu(}
\NormalTok{  title: "Master\textquotesingle{}s degree",}
\NormalTok{  subtitle: "University of Sciences",}
\NormalTok{  task{-}description: [}
\NormalTok{    {-} Short summary of the most important courses}
\NormalTok{    {-} Explanation of master thesis topic}
\NormalTok{  ],}
\NormalTok{  date{-}from: "10/2021",}
\NormalTok{  date{-}to: "07/2023",}
\NormalTok{)}
\end{Highlighting}
\end{Shaded}

\pandocbounded{\includegraphics[keepaspectratio]{https://github.com/typst/packages/raw/main/packages/preview/modern-resume/0.1.0/docs/images/work-experience.png}}

\begin{Shaded}
\begin{Highlighting}[]
\NormalTok{\#experience{-}work(}
\NormalTok{  title: "Full Stack Software Engineer",}
\NormalTok{  subtitle: [\#link("https://www.google.com")[Some IT Company]],}
\NormalTok{  facility{-}description: "Company operating in sector XY",}
\NormalTok{  task{-}description: [}
\NormalTok{    {-} Short summary of your responsibilities}
\NormalTok{  ],}
\NormalTok{  date{-}from: "09/2018",}
\NormalTok{  date{-}to: "07/2021",}
\NormalTok{)}
\end{Highlighting}
\end{Shaded}

\subsubsection{Project}\label{project}

Import this element from the template module with \texttt{\ project\ } .

\pandocbounded{\includegraphics[keepaspectratio]{https://github.com/typst/packages/raw/main/packages/preview/modern-resume/0.1.0/docs/images/project.png}}

\begin{Shaded}
\begin{Highlighting}[]
\NormalTok{\#project(}
\NormalTok{  title: "Project 2",}
\NormalTok{  subtitle: "Data Visualization, Data Engineering",}
\NormalTok{  description: [}
\NormalTok{    {-} \#lorem(20)}
\NormalTok{  ],}
\NormalTok{  date{-}from: "08/2022",}
\NormalTok{  date{-}to: "09/2022",}
\NormalTok{)}
\end{Highlighting}
\end{Shaded}

\subsection{Contributing}\label{contributing}

I’m grateful for any improvements and suggestions.

\subsection{Acknowledgements}\label{acknowledgements}

This project would not be what it is without:

\begin{itemize}
\tightlist
\item
  \href{https://github.com/FortAwesome/Font-Awesome/}{Font Awesome Free}
  \textbar{} providing the icons
\end{itemize}

\href{/app?template=modern-resume&version=0.1.0}{Create project in app}

\subsubsection{How to use}\label{how-to-use}

Click the button above to create a new project using this template in
the Typst app.

You can also use the Typst CLI to start a new project on your computer
using this command:

\begin{verbatim}
typst init @preview/modern-resume:0.1.0
\end{verbatim}

\includesvg[width=0.16667in,height=0.16667in]{/assets/icons/16-copy.svg}

\subsubsection{About}\label{about}

\begin{description}
\tightlist
\item[Author :]
\href{https://github.com/peterpf}{Peter Egger}
\item[License:]
Unlicense
\item[Current version:]
0.1.0
\item[Last updated:]
June 13, 2024
\item[First released:]
June 13, 2024
\item[Minimum Typst version:]
0.10.0
\item[Archive size:]
1.26 MB
\href{https://packages.typst.org/preview/modern-resume-0.1.0.tar.gz}{\pandocbounded{\includesvg[keepaspectratio]{/assets/icons/16-download.svg}}}
\item[Repository:]
\href{https://github.com/peterpf/modern-typst-resume}{GitHub}
\item[Categor y :]
\begin{itemize}
\tightlist
\item[]
\item
  \pandocbounded{\includesvg[keepaspectratio]{/assets/icons/16-user.svg}}
  \href{https://typst.app/universe/search/?category=cv}{CV}
\end{itemize}
\end{description}

\subsubsection{Where to report issues?}\label{where-to-report-issues}

This template is a project of Peter Egger . Report issues on
\href{https://github.com/peterpf/modern-typst-resume}{their repository}
. You can also try to ask for help with this template on the
\href{https://forum.typst.app}{Forum} .

Please report this template to the Typst team using the
\href{https://typst.app/contact}{contact form} if you believe it is a
safety hazard or infringes upon your rights.

\phantomsection\label{versions}
\subsubsection{Version history}\label{version-history}

\begin{longtable}[]{@{}ll@{}}
\toprule\noalign{}
Version & Release Date \\
\midrule\noalign{}
\endhead
\bottomrule\noalign{}
\endlastfoot
0.1.0 & June 13, 2024 \\
\end{longtable}

Typst GmbH did not create this template and cannot guarantee correct
functionality of this template or compatibility with any version of the
Typst compiler or app.


\title{typst.app/universe/package/minimal-presentation}

\phantomsection\label{banner}
\phantomsection\label{template-thumbnail}
\pandocbounded{\includegraphics[keepaspectratio]{https://packages.typst.org/preview/thumbnails/minimal-presentation-0.3.0-small.webp}}

\section{minimal-presentation}\label{minimal-presentation}

{ 0.3.0 }

A modern minimalistic presentation template ready to use

\href{/app?template=minimal-presentation&version=0.3.0}{Create project
in app}

\phantomsection\label{readme}
A modern minimalistic presentation template ready to use.

\subsection{Usage}\label{usage}

You can use this template in the Typst web app by clicking “Start from
template� on the dashboard and searching for
\texttt{\ minimal-presentation\ } .

Alternatively, you can use the CLI to kick this project off using the
command

\begin{verbatim}
typst init @preview/minimal-presentation
\end{verbatim}

Typst will create a new directory with all the files needed to get you
started.

\subsection{Configuration}\label{configuration}

This template exports the \texttt{\ project\ } function with the
following named arguments:

\begin{itemize}
\tightlist
\item
  \texttt{\ title\ } : The book’s title as content.
\item
  \texttt{\ sub-title\ } : The book’s subtitle as content.
\item
  \texttt{\ author\ } : Content or an array of content to specify the
  author.
\item
  \texttt{\ aspect-ratio\ } : Defaults to \texttt{\ 16-9\ } . Can be
  also \texttt{\ 4-3\ } .
\end{itemize}

The function also accepts a single, positional argument for the body of
the book.

The template will initialize your package with a sample call to the
\texttt{\ project\ } function in a show rule. If you, however, want to
change an existing project to use this template, you can add a show rule
like this at the top of your file:

\begin{Shaded}
\begin{Highlighting}[]
\NormalTok{\#import "@preview/minimal{-}presentation:0.1.0": *}

\NormalTok{\#set text(font: "Lato")}
\NormalTok{\#show math.equation: set text(font: "Lato Math")}
\NormalTok{\#show raw: set text(font: "Fira Code")}

\NormalTok{\#show: project.with(}
\NormalTok{  title: "Minimalist presentation template",}
\NormalTok{  sub{-}title: "This is where your presentation begins",}
\NormalTok{  author: "Flavio Barisi",}
\NormalTok{  date: "10/08/2023",}
\NormalTok{  index{-}title: "Contents",}
\NormalTok{  logo: image("./logo.svg"),}
\NormalTok{  logo{-}light: image("./logo\_light.svg"),}
\NormalTok{  cover: image("./image\_3.jpg")}
\NormalTok{)}

\NormalTok{= This is a section}

\NormalTok{== This is a slide title}

\NormalTok{\#lorem(10)}

\NormalTok{{-} \#lorem(10)}
\NormalTok{  {-} \#lorem(10)}
\NormalTok{  {-} \#lorem(10)}
\NormalTok{  {-} \#lorem(10)}

\NormalTok{== One column image}

\NormalTok{\#figure(}
\NormalTok{  image("image\_1.jpg", height: 10.5cm),}
\NormalTok{  caption: [An image],}
\NormalTok{) \textless{}image\_label\textgreater{}}

\NormalTok{== Two columns image}

\NormalTok{\#columns{-}content()[}
\NormalTok{  \#figure(}
\NormalTok{    image("image\_1.jpg", width: 100\%),}
\NormalTok{    caption: [An image],}
\NormalTok{  ) \textless{}image\_label\_1\textgreater{}}
\NormalTok{][}
\NormalTok{  \#figure(}
\NormalTok{    image("image\_1.jpg", width: 100\%),}
\NormalTok{    caption: [An image],}
\NormalTok{  ) \textless{}image\_label\_2\textgreater{}}
\NormalTok{]}

\NormalTok{== Two columns}

\NormalTok{\#columns{-}content()[}
\NormalTok{  {-} \#lorem(10)}
\NormalTok{  {-} \#lorem(10)}
\NormalTok{  {-} \#lorem(10)}
\NormalTok{][}
\NormalTok{  \#figure(}
\NormalTok{    image("image\_3.jpg", width: 100\%),}
\NormalTok{    caption: [An image],}
\NormalTok{  ) \textless{}image\_label\_3\textgreater{}}
\NormalTok{]}

\NormalTok{= This is a section}

\NormalTok{== This is a slide title}

\NormalTok{\#lorem(10)}

\NormalTok{= This is a section}

\NormalTok{== This is a slide title}

\NormalTok{\#lorem(10)}

\NormalTok{= This is a section}

\NormalTok{== This is a slide title}

\NormalTok{\#lorem(10)}

\NormalTok{= This is a very v v v v v v v v v v v v v v v v v v v v  long section}

\NormalTok{== This is a very v v v v v v v v v v v v v v v v v v v v  long slide title}

\NormalTok{= sub{-}title test}

\NormalTok{== Slide title}

\NormalTok{\#lorem(50)}

\NormalTok{=== Slide sub{-}title 1}

\NormalTok{\#lorem(50)}

\NormalTok{=== Slide sub{-}title 2}

\NormalTok{\#lorem(50)}

\end{Highlighting}
\end{Shaded}

\subsection{Fonts}\label{fonts}

You can use the font selected by the author of this plugin, by download
theme at the following link:

\url{https://github.com/flavio20002/typst-presentation-minimal-template/tree/main/fonts}

You can then import thme in your system, import them in the typst web
app or just put them in a folder and launch the compilation with the
following argoument:

\begin{verbatim}
typst watch main.typ --root . --font-path fonts
\end{verbatim}

\href{/app?template=minimal-presentation&version=0.3.0}{Create project
in app}

\subsubsection{How to use}\label{how-to-use}

Click the button above to create a new project using this template in
the Typst app.

You can also use the Typst CLI to start a new project on your computer
using this command:

\begin{verbatim}
typst init @preview/minimal-presentation:0.3.0
\end{verbatim}

\includesvg[width=0.16667in,height=0.16667in]{/assets/icons/16-copy.svg}

\subsubsection{About}\label{about}

\begin{description}
\tightlist
\item[Author :]
Flavio Barisi
\item[License:]
MIT-0
\item[Current version:]
0.3.0
\item[Last updated:]
November 18, 2024
\item[First released:]
September 2, 2024
\item[Minimum Typst version:]
0.12.0
\item[Archive size:]
755 kB
\href{https://packages.typst.org/preview/minimal-presentation-0.3.0.tar.gz}{\pandocbounded{\includesvg[keepaspectratio]{/assets/icons/16-download.svg}}}
\item[Repository:]
\href{https://github.com/flavio20002/typst-presentation-minimal-template}{GitHub}
\item[Categor y :]
\begin{itemize}
\tightlist
\item[]
\item
  \pandocbounded{\includesvg[keepaspectratio]{/assets/icons/16-presentation.svg}}
  \href{https://typst.app/universe/search/?category=presentation}{Presentation}
\end{itemize}
\end{description}

\subsubsection{Where to report issues?}\label{where-to-report-issues}

This template is a project of Flavio Barisi . Report issues on
\href{https://github.com/flavio20002/typst-presentation-minimal-template}{their
repository} . You can also try to ask for help with this template on the
\href{https://forum.typst.app}{Forum} .

Please report this template to the Typst team using the
\href{https://typst.app/contact}{contact form} if you believe it is a
safety hazard or infringes upon your rights.

\phantomsection\label{versions}
\subsubsection{Version history}\label{version-history}

\begin{longtable}[]{@{}ll@{}}
\toprule\noalign{}
Version & Release Date \\
\midrule\noalign{}
\endhead
\bottomrule\noalign{}
\endlastfoot
0.3.0 & November 18, 2024 \\
\href{https://typst.app/universe/package/minimal-presentation/0.2.0/}{0.2.0}
& October 23, 2024 \\
\href{https://typst.app/universe/package/minimal-presentation/0.1.0/}{0.1.0}
& September 2, 2024 \\
\end{longtable}

Typst GmbH did not create this template and cannot guarantee correct
functionality of this template or compatibility with any version of the
Typst compiler or app.


\title{typst.app/universe/package/quetta}

\phantomsection\label{banner}
\section{quetta}\label{quetta}

{ 0.2.0 }

Write Tengwar easily with Typst.

\phantomsection\label{readme}
A simple module to write
\href{https://en.wikipedia.org/wiki/Tengwar}{tengwar} in
\href{https://typst.app/}{Typst} .

\subsection{Requirements}\label{requirements}

\begin{itemize}
\tightlist
\item
  \href{https://github.com/typst/typst}{Typst} version 0.11.0 or 0.11.1
\item
  The
  \href{https://www.fontspace.com/tengwar-annatar-font-f2244}{Tengwar
  Annatar} fonts version 1.20
\end{itemize}

To use this module with the \href{https://typst.app/}{Typst web app} ,
you need to upload the font files to your project.

\subsection{Usage}\label{usage}

The main functionality of this module is provided by functions taking
content and converting all text in Tenwar:

\begin{itemize}
\tightlist
\item
  \texttt{\ quenya\ } converts text using the mode of Quenya,
\item
  \texttt{\ gondor\ } converts text using the Sindarin mode of Gondor.
\end{itemize}

The original text is used as a phonetic transcription. (This module does
not translate English into Quenya or Sindarin.) See the
\href{https://github.com/FlorentCLMichel/quetta/blob/main/manual.pdf}{manual}
for more information.

The following line may be used to convert the whole document below to
Tengwar in Quenya mode (other \texttt{\ show\ } rules might interfere
with it):

\begin{verbatim}
#show: quetta.quenya
\end{verbatim}

\textbf{Example:}

\begin{verbatim}
#import "@preview/quetta:0.2.0"

// Use the function `quenya` to write a small amount of text in Tengwar (Quenya mode)
#text(size: 16pt, 
      fill: gradient.linear(blue, green)
     )[#box(quetta.quenya[_tengwar_])]

#v(1em)

// A `show` rule may be more convenient for larger contents; beware that it may interfere with other ones, though
#show: quetta.quenya

Namárië!

#h(1em) _Namárië!_

#h(2em) *Namárië!*
\end{verbatim}

\subsection{Roadmap}\label{roadmap}

\begin{itemize}
\tightlist
\item
  Number conversion: done
\item
  Support for the Quenya mode: done
\item
  Support for the mode of Gondor: done
\item
  Support for the mode of Beleriand: backlog
\item
  Support for the Black Speech: backlog
\end{itemize}

\subsection{Changelog}\label{changelog}

\subsubsection{v0.2.0}\label{v0.2.0}

\begin{itemize}
\tightlist
\item
  Add support for Sindarinâ€''Mode of Gondor
\item
  \textbf{Breaking change:} The symbol used to prevent combination was
  changed from \texttt{\ :\ } to \texttt{\ \textbar{}\ } .
\item
  Small changes to the kerning between several tengwar and to tehtar
  positions.
\end{itemize}

\subsubsection{v0.1.0}\label{v0.1.0}

Initial release with Quenya support.

\subsection{How can I contribute?}\label{how-can-i-contribute}

I (the original author) am definitely not en expert in either Typst nor
Tengwar. I could thus use some help in all areas. I would especially
welcome contributions or suggestions on the following:

\begin{itemize}
\tightlist
\item
  Identify and resolve inefficiencies in the Typst code.
\item
  Identify cases where the result differs from the expected one. (In
  particular, there are probably rules for writing in Tengwar that I
  either am not aware of or have not properly understood. Any advice on
  that is warmly welcome!)
\item
  References on Tengar, Quenya, and Sindarin.
\item
  Support for other Tengwar fonts.
\end{itemize}

\subsubsection{How to add}\label{how-to-add}

Copy this into your project and use the import as \texttt{\ quetta\ }

\begin{verbatim}
#import "@preview/quetta:0.2.0"
\end{verbatim}

\includesvg[width=0.16667in,height=0.16667in]{/assets/icons/16-copy.svg}

Check the docs for
\href{https://typst.app/docs/reference/scripting/\#packages}{more
information on how to import packages} .

\subsubsection{About}\label{about}

\begin{description}
\tightlist
\item[Author :]
\href{https://github.com/FlorentCLMichel}{Florent Michel}
\item[License:]
MIT
\item[Current version:]
0.2.0
\item[Last updated:]
September 24, 2024
\item[First released:]
July 31, 2024
\item[Minimum Typst version:]
0.11.0
\item[Archive size:]
8.96 kB
\href{https://packages.typst.org/preview/quetta-0.2.0.tar.gz}{\pandocbounded{\includesvg[keepaspectratio]{/assets/icons/16-download.svg}}}
\item[Repository:]
\href{https://github.com/FlorentCLMichel/quetta}{GitHub}
\item[Discipline s :]
\begin{itemize}
\tightlist
\item[]
\item
  \href{https://typst.app/universe/search/?discipline=linguistics}{Linguistics}
\item
  \href{https://typst.app/universe/search/?discipline=literature}{Literature}
\end{itemize}
\item[Categor ies :]
\begin{itemize}
\tightlist
\item[]
\item
  \pandocbounded{\includesvg[keepaspectratio]{/assets/icons/16-text.svg}}
  \href{https://typst.app/universe/search/?category=text}{Text}
\item
  \pandocbounded{\includesvg[keepaspectratio]{/assets/icons/16-world.svg}}
  \href{https://typst.app/universe/search/?category=languages}{Languages}
\item
  \pandocbounded{\includesvg[keepaspectratio]{/assets/icons/16-smile.svg}}
  \href{https://typst.app/universe/search/?category=fun}{Fun}
\end{itemize}
\end{description}

\subsubsection{Where to report issues?}\label{where-to-report-issues}

This package is a project of Florent Michel . Report issues on
\href{https://github.com/FlorentCLMichel/quetta}{their repository} . You
can also try to ask for help with this package on the
\href{https://forum.typst.app}{Forum} .

Please report this package to the Typst team using the
\href{https://typst.app/contact}{contact form} if you believe it is a
safety hazard or infringes upon your rights.

\phantomsection\label{versions}
\subsubsection{Version history}\label{version-history}

\begin{longtable}[]{@{}ll@{}}
\toprule\noalign{}
Version & Release Date \\
\midrule\noalign{}
\endhead
\bottomrule\noalign{}
\endlastfoot
0.2.0 & September 24, 2024 \\
\href{https://typst.app/universe/package/quetta/0.1.0/}{0.1.0} & July
31, 2024 \\
\end{longtable}

Typst GmbH did not create this package and cannot guarantee correct
functionality of this package or compatibility with any version of the
Typst compiler or app.


\title{typst.app/universe/package/october}

\phantomsection\label{banner}
\phantomsection\label{template-thumbnail}
\pandocbounded{\includegraphics[keepaspectratio]{https://packages.typst.org/preview/thumbnails/october-1.0.0-small.webp}}

\section{october}\label{october}

{ 1.0.0 }

Simple printable year calendar

\href{/app?template=october&version=1.0.0}{Create project in app}

\phantomsection\label{readme}
This template generates a monthly calendar, designed to be printed in
landscape.

The calendar function accepts one parameter for the year, which should
be formatted as an integer. Otherwise, the current year can be passed in
with \texttt{\ datetime.today().year()\ } .

\begin{Shaded}
\begin{Highlighting}[]
\NormalTok{ \#show: calendar.with(}
\NormalTok{  year: datetime.today().year()}
\NormalTok{)}
\end{Highlighting}
\end{Shaded}

There isn’t much space for writing in each day box, it’s more suited
to blocking out days with a highlighter. For example, to mark out free
days in a variable schedule of work shifts.

\href{/app?template=october&version=1.0.0}{Create project in app}

\subsubsection{How to use}\label{how-to-use}

Click the button above to create a new project using this template in
the Typst app.

You can also use the Typst CLI to start a new project on your computer
using this command:

\begin{verbatim}
typst init @preview/october:1.0.0
\end{verbatim}

\includesvg[width=0.16667in,height=0.16667in]{/assets/icons/16-copy.svg}

\subsubsection{About}\label{about}

\begin{description}
\tightlist
\item[Author :]
\href{mailto:pierre.marshall@gmail.com}{Pierre Marshall}
\item[License:]
MIT-0
\item[Current version:]
1.0.0
\item[Last updated:]
October 18, 2024
\item[First released:]
October 18, 2024
\item[Minimum Typst version:]
0.11.1
\item[Archive size:]
1.94 kB
\href{https://packages.typst.org/preview/october-1.0.0.tar.gz}{\pandocbounded{\includesvg[keepaspectratio]{/assets/icons/16-download.svg}}}
\item[Repository:]
\href{https://github.com/extua/october}{GitHub}
\item[Categor y :]
\begin{itemize}
\tightlist
\item[]
\item
  \pandocbounded{\includesvg[keepaspectratio]{/assets/icons/16-envelope.svg}}
  \href{https://typst.app/universe/search/?category=office}{Office}
\end{itemize}
\end{description}

\subsubsection{Where to report issues?}\label{where-to-report-issues}

This template is a project of Pierre Marshall . Report issues on
\href{https://github.com/extua/october}{their repository} . You can also
try to ask for help with this template on the
\href{https://forum.typst.app}{Forum} .

Please report this template to the Typst team using the
\href{https://typst.app/contact}{contact form} if you believe it is a
safety hazard or infringes upon your rights.

\phantomsection\label{versions}
\subsubsection{Version history}\label{version-history}

\begin{longtable}[]{@{}ll@{}}
\toprule\noalign{}
Version & Release Date \\
\midrule\noalign{}
\endhead
\bottomrule\noalign{}
\endlastfoot
1.0.0 & October 18, 2024 \\
\end{longtable}

Typst GmbH did not create this template and cannot guarantee correct
functionality of this template or compatibility with any version of the
Typst compiler or app.


\title{typst.app/universe/package/upb-corporate-design-slides}

\phantomsection\label{banner}
\phantomsection\label{template-thumbnail}
\pandocbounded{\includegraphics[keepaspectratio]{https://packages.typst.org/preview/thumbnails/upb-corporate-design-slides-0.1.1-small.webp}}

\section{upb-corporate-design-slides}\label{upb-corporate-design-slides}

{ 0.1.1 }

Presentation template for Paderborn University (UPB)

\href{/app?template=upb-corporate-design-slides&version=0.1.1}{Create
project in app}

\phantomsection\label{readme}
This template can be used to create presentations in
\href{https://typst.app/docs/}{Typst} with the corporate design of
\href{https://www.uni-paderborn.de/}{Paderborn University} using the
\href{https://touying-typ.github.io/}{Touying} slide engine.

\subsection{Usage}\label{usage}

Create a new typst project based on this template locally.

\begin{Shaded}
\begin{Highlighting}[]
\ExtensionTok{typst}\NormalTok{ init @preview/upb{-}corporate{-}design{-}slides}
\BuiltInTok{cd}\NormalTok{ upb{-}corporate{-}design{-}slides}
\end{Highlighting}
\end{Shaded}

Or create a project on the typst web app based on this template.

\subsubsection{Font setup}\label{font-setup}

The font \texttt{\ Karla\ } needs to be installed on your system. The
\href{https://www.uni-paderborn.de/universitaet/presse-kommunikation-marketing/brandportal}{UPB
brand portal} recommends to download it from google fonts.

\begin{itemize}
\tightlist
\item
  If you use arch linux, you can also get it from the
  \href{https://aur.archlinux.org/packages/ttf-karla}{AUR} .
\item
  If you use NixOS, karla is available in the \texttt{\ 24.11\ } and
  \texttt{\ unstable\ } channels.
\item
  If you use the typst web app, you will need to upload the font.
\end{itemize}

\href{/app?template=upb-corporate-design-slides&version=0.1.1}{Create
project in app}

\subsubsection{How to use}\label{how-to-use}

Click the button above to create a new project using this template in
the Typst app.

You can also use the Typst CLI to start a new project on your computer
using this command:

\begin{verbatim}
typst init @preview/upb-corporate-design-slides:0.1.1
\end{verbatim}

\includesvg[width=0.16667in,height=0.16667in]{/assets/icons/16-copy.svg}

\subsubsection{About}\label{about}

\begin{description}
\tightlist
\item[Author s :]
\href{https://kuchenmampfer.de/}{Tammes Burghard} \& Marvin Feiter
\item[License:]
MIT
\item[Current version:]
0.1.1
\item[Last updated:]
November 28, 2024
\item[First released:]
November 4, 2024
\item[Archive size:]
32.7 kB
\href{https://packages.typst.org/preview/upb-corporate-design-slides-0.1.1.tar.gz}{\pandocbounded{\includesvg[keepaspectratio]{/assets/icons/16-download.svg}}}
\item[Repository:]
\href{https://codeberg.org/Kuchenmampfer/upb-corporate-design-slides}{Codeberg}
\item[Categor y :]
\begin{itemize}
\tightlist
\item[]
\item
  \pandocbounded{\includesvg[keepaspectratio]{/assets/icons/16-presentation.svg}}
  \href{https://typst.app/universe/search/?category=presentation}{Presentation}
\end{itemize}
\end{description}

\subsubsection{Where to report issues?}\label{where-to-report-issues}

This template is a project of Tammes Burghard and Marvin Feiter . Report
issues on
\href{https://codeberg.org/Kuchenmampfer/upb-corporate-design-slides}{their
repository} . You can also try to ask for help with this template on the
\href{https://forum.typst.app}{Forum} .

Please report this template to the Typst team using the
\href{https://typst.app/contact}{contact form} if you believe it is a
safety hazard or infringes upon your rights.

\phantomsection\label{versions}
\subsubsection{Version history}\label{version-history}

\begin{longtable}[]{@{}ll@{}}
\toprule\noalign{}
Version & Release Date \\
\midrule\noalign{}
\endhead
\bottomrule\noalign{}
\endlastfoot
0.1.1 & November 28, 2024 \\
\href{https://typst.app/universe/package/upb-corporate-design-slides/0.1.0/}{0.1.0}
& November 4, 2024 \\
\end{longtable}

Typst GmbH did not create this template and cannot guarantee correct
functionality of this template or compatibility with any version of the
Typst compiler or app.


\title{typst.app/universe/package/weave}

\phantomsection\label{banner}
\section{weave}\label{weave}

{ 0.2.0 }

A helper library for chaining lambda abstractions

\phantomsection\label{readme}
A helper library for chaining lambda abstractions, imitating the
\texttt{\ \textbar{}\textgreater{}\ } or \texttt{\ .\ } operator in some
functional languages.

The function \texttt{\ compose\ } is the \texttt{\ pipe\ } function in
the mathematical order. Functions suffixed with underscore have their
arguments flipped.

\subsection{Changelog}\label{changelog}

\begin{itemize}
\tightlist
\item
  0.2.0 Redesigned interface to work with typst’s \texttt{\ with\ }
  keyword.
\item
  0.1.0 Initial release
\end{itemize}

\subsection{Basic usage}\label{basic-usage}

It can help improve readability with nested applications to a content
value, or make the diff cleaner.

\begin{Shaded}
\begin{Highlighting}[]
\NormalTok{\#compose\_((}
\NormalTok{  text.with(blue),}
\NormalTok{  emph,}
\NormalTok{  strong,}
\NormalTok{  underline,}
\NormalTok{  strike,}
\NormalTok{))[This is a very long content with a lot of words]}
\NormalTok{// Is equivalent to}
\NormalTok{\#text(}
\NormalTok{  blue,}
\NormalTok{  emph(}
\NormalTok{    strong(}
\NormalTok{      underline(}
\NormalTok{        strike[This is a very long content with a lot of words]}
\NormalTok{      )}
\NormalTok{    )}
\NormalTok{  )}
\NormalTok{)}
\end{Highlighting}
\end{Shaded}

You can use it for show rules just like the example above.

\begin{Shaded}
\begin{Highlighting}[]
\NormalTok{\#show link: compose\_.with((}
\NormalTok{  text.with(fill: blue),}
\NormalTok{  emph,}
\NormalTok{  underline,}
\NormalTok{))}
\NormalTok{// These two are equivalent}
\NormalTok{\#show link: text.with(fill: blue)}
\NormalTok{\#show link: emph}
\NormalTok{\#show link: underline}
\end{Highlighting}
\end{Shaded}

This can also be useful when you need to destructure lists, as it allows
creating binds that are scoped by each lambda expression.

\begin{Shaded}
\begin{Highlighting}[]
\NormalTok{\#let two\_and\_one = pipe(}
\NormalTok{  (1, 2),}
\NormalTok{  (}
\NormalTok{    ((a, b)) =\textgreater{} (a, b, {-}1), // becomes a list of length three}
\NormalTok{    ((a, b, \_)) =\textgreater{} (b, a), // discard the third element and swap}
\NormalTok{  ),}
\NormalTok{)}
\end{Highlighting}
\end{Shaded}

\subsubsection{How to add}\label{how-to-add}

Copy this into your project and use the import as \texttt{\ weave\ }

\begin{verbatim}
#import "@preview/weave:0.2.0"
\end{verbatim}

\includesvg[width=0.16667in,height=0.16667in]{/assets/icons/16-copy.svg}

Check the docs for
\href{https://typst.app/docs/reference/scripting/\#packages}{more
information on how to import packages} .

\subsubsection{About}\label{about}

\begin{description}
\tightlist
\item[Author :]
\href{https://github.com/leana8959}{Léana 江}
\item[License:]
MIT
\item[Current version:]
0.2.0
\item[Last updated:]
October 21, 2024
\item[First released:]
October 21, 2024
\item[Archive size:]
1.92 kB
\href{https://packages.typst.org/preview/weave-0.2.0.tar.gz}{\pandocbounded{\includesvg[keepaspectratio]{/assets/icons/16-download.svg}}}
\item[Categor y :]
\begin{itemize}
\tightlist
\item[]
\item
  \pandocbounded{\includesvg[keepaspectratio]{/assets/icons/16-code.svg}}
  \href{https://typst.app/universe/search/?category=scripting}{Scripting}
\end{itemize}
\end{description}

\subsubsection{Where to report issues?}\label{where-to-report-issues}

This package is a project of Léana 江 . You can also try to ask for
help with this package on the \href{https://forum.typst.app}{Forum} .

Please report this package to the Typst team using the
\href{https://typst.app/contact}{contact form} if you believe it is a
safety hazard or infringes upon your rights.

\phantomsection\label{versions}
\subsubsection{Version history}\label{version-history}

\begin{longtable}[]{@{}ll@{}}
\toprule\noalign{}
Version & Release Date \\
\midrule\noalign{}
\endhead
\bottomrule\noalign{}
\endlastfoot
0.2.0 & October 21, 2024 \\
\href{https://typst.app/universe/package/weave/0.1.0/}{0.1.0} & October
21, 2024 \\
\end{longtable}

Typst GmbH did not create this package and cannot guarantee correct
functionality of this package or compatibility with any version of the
Typst compiler or app.


\title{typst.app/universe/package/letter-pro}

\phantomsection\label{banner}
\phantomsection\label{template-thumbnail}
\pandocbounded{\includegraphics[keepaspectratio]{https://packages.typst.org/preview/thumbnails/letter-pro-3.0.0-small.webp}}

\section{letter-pro}\label{letter-pro}

{ 3.0.0 }

DIN 5008 letter template for Typst.

\href{/app?template=letter-pro&version=3.0.0}{Create project in app}

\phantomsection\label{readme}
A template for creating business letters following the DIN 5008
standard.

\subsection{Overview}\label{overview}

typst-letter-pro provides a convenient and professional way to generate
business letters with a standardized layout. The template follows the
guidelines specified in the DIN 5008 standard, ensuring that your
letters adhere to the commonly accepted business communication
practices.

The goal of typst-letter-pro is to simplify the process of creating
business letters while maintaining a clean and professional appearance.
It offers predefined sections for the sender and recipient information,
subject, date, header, footer and more.

\subsection{\texorpdfstring{\href{https://raw.githubusercontent.com/wiki/Sematre/typst-letter-pro/documentation-v3.0.0.pdf}{Documentation}}{Documentation}}\label{documentation}

\subsection{Example}\label{example}

Text source:
\href{https://web.archive.org/web/20230927152049/https://www.deutschepost.de/de/b/briefvorlagen/beschwerden.html\#Einspruch}{Musterbrief
Widerspruch gegen Einkommensteuerbescheid}

\subsubsection{\texorpdfstring{Preview (
\href{https://raw.githubusercontent.com/wiki/Sematre/typst-letter-pro/simple_letter.pdf}{PDF
version} )}{Preview ( PDF version )}}\label{preview-pdf-version}

\pandocbounded{\includegraphics[keepaspectratio]{https://github.com/typst/packages/raw/main/packages/preview/letter-pro/3.0.0/template/thumbnail.png}}

\subsubsection{Code}\label{code}

\begin{Shaded}
\begin{Highlighting}[]
\NormalTok{\#import "@preview/letter{-}pro:3.0.0": letter{-}simple}

\NormalTok{\#set text(lang: "de")}

\NormalTok{\#show: letter{-}simple.with(}
\NormalTok{  sender: (}
\NormalTok{    name: "Anja Ahlsen",}
\NormalTok{    address: "Deutschherrenufer 28, 60528 Frankfurt",}
\NormalTok{    extra: [}
\NormalTok{      Telefon: \#link("tel:+4915228817386")[+49 152 28817386]\textbackslash{}}
\NormalTok{      E{-}Mail: \#link("mailto:aahlsen@example.com")[aahlsen\textbackslash{}@example.com]\textbackslash{}}
\NormalTok{    ],}
\NormalTok{  ),}
  
\NormalTok{  annotations: [Einschreiben {-} Rückschein],}
\NormalTok{  recipient: [}
\NormalTok{    Finanzamt Frankfurt\textbackslash{}}
\NormalTok{    Einkommenssteuerstelle\textbackslash{}}
\NormalTok{    Gutleutstraße 5\textbackslash{}}
\NormalTok{    60329 Frankfurt}
\NormalTok{  ],}
  
\NormalTok{  reference{-}signs: (}
\NormalTok{    ([Steuernummer], [333/24692/5775]),}
\NormalTok{  ),}
  
\NormalTok{  date: "12. November 2014",}
\NormalTok{  subject: "Einspruch gegen den ESt{-}Bescheid",}
\NormalTok{)}

\NormalTok{Sehr geehrte Damen und Herren,}

\NormalTok{die von mir bei den Werbekosten geltend gemachte Abschreibung für den im}
\NormalTok{vergangenen Jahr angeschafften Fotokopierer wurde von Ihnen nicht berücksichtigt.}
\NormalTok{Der Fotokopierer steht in meinem Büro und wird von mir ausschließlich zu beruflichen}
\NormalTok{Zwecken verwendet.}

\NormalTok{Ich lege deshalb Einspruch gegen den oben genannten Einkommensteuerbescheid ein}
\NormalTok{und bitte Sie, die Abschreibung anzuerkennen.}

\NormalTok{Anbei erhalten Sie eine Kopie der Rechnung des Gerätes.}

\NormalTok{Mit freundlichen Grüßen}
\NormalTok{\#v(1cm)}
\NormalTok{Anja Ahlsen}

\NormalTok{\#v(1fr)}
\NormalTok{*Anlagen:*}
\NormalTok{{-} Rechnung}
\end{Highlighting}
\end{Shaded}

\subsection{Usage}\label{usage}

\subsubsection{Preview repository}\label{preview-repository}

Import the package in your document:

\begin{Shaded}
\begin{Highlighting}[]
\NormalTok{\#import "@preview/letter{-}pro:3.0.0": letter{-}simple}
\end{Highlighting}
\end{Shaded}

\subsubsection{Local namespace}\label{local-namespace}

Download the repository to the local package namespace using Git:

\begin{Shaded}
\begin{Highlighting}[]
\ExtensionTok{$}\NormalTok{ git clone }\AttributeTok{{-}c}\NormalTok{ advice.detachedHead=false https://github.com/Sematre/typst{-}letter{-}pro.git }\AttributeTok{{-}{-}depth}\NormalTok{ 1 }\AttributeTok{{-}{-}branch}\NormalTok{ v3.0.0 \textasciitilde{}/.local/share/typst/packages/local/letter{-}pro/3.0.0}
\end{Highlighting}
\end{Shaded}

Then import the package in your document:

\begin{Shaded}
\begin{Highlighting}[]
\NormalTok{\#import "@local/letter{-}pro:3.0.0": letter{-}simple}
\end{Highlighting}
\end{Shaded}

\subsubsection{Manual}\label{manual}

Download the \texttt{\ letter-pro-v3.0.0.typ\ } file from the
\href{https://github.com/Sematre/typst-letter-pro/releases}{releases
page} and place it next to your document file, e.g., using \emph{wget} :

\begin{Shaded}
\begin{Highlighting}[]
\ExtensionTok{$}\NormalTok{ wget https://github.com/Sematre/typst{-}letter{-}pro/releases/download/v3.0.0/letter{-}pro{-}v3.0.0.typ}
\end{Highlighting}
\end{Shaded}

Then import the package in your document:

\begin{Shaded}
\begin{Highlighting}[]
\NormalTok{\#import "letter{-}pro{-}v3.0.0.typ": letter{-}simple}
\end{Highlighting}
\end{Shaded}

\subsection{Contributing}\label{contributing}

Contributions to typst-letter-pro are welcome! If you encounter any
issues or have suggestions for improvements, please open an issue on
GitHub or submit a pull request.

Before making any significant changes, please discuss your ideas with
the project maintainers to ensure they align with the project’s goals
and direction.

\subsection{Acknowledgments}\label{acknowledgments}

This project is inspired by the following projects and resources:

\begin{itemize}
\tightlist
\item
  \href{https://de.wikipedia.org/wiki/DIN_5008}{Wikipedia / DIN 5008}
\item
  \href{https://web.archive.org/web/20240223035339/https://www.deutschepost.de/de/b/briefvorlagen/normbrief-din-5008-vorlage.html}{Deutsche
  Post / DIN 5008 Vorlage}
\item
  \href{https://www.deutschepost.de/dam/dpag/images/P_p/printmailing/downloads/dp-automationsfaehige-briefsendungen-2024.pdf}{Deutsche
  Post / Automationsfähige Briefsendungen}
\item
  \href{https://www.edv-lehrgang.de/din-5008/}{EDV Lehrgang / DIN-5008}
\item
  \href{https://github.com/ludwig-austermann/typst-din-5008-letter}{Ludwig
  Austermann / typst-din-5008-letter}
\item
  \href{https://github.com/pascal-huber/typst-letter-template}{Pascal
  Huber / typst-letter-template}
\end{itemize}

\subsection{License}\label{license}

Distributed under the \textbf{MIT License} . See \texttt{\ LICENSE\ }
for more information.

\href{/app?template=letter-pro&version=3.0.0}{Create project in app}

\subsubsection{How to use}\label{how-to-use}

Click the button above to create a new project using this template in
the Typst app.

You can also use the Typst CLI to start a new project on your computer
using this command:

\begin{verbatim}
typst init @preview/letter-pro:3.0.0
\end{verbatim}

\includesvg[width=0.16667in,height=0.16667in]{/assets/icons/16-copy.svg}

\subsubsection{About}\label{about}

\begin{description}
\tightlist
\item[Author :]
Sematre
\item[License:]
MIT
\item[Current version:]
3.0.0
\item[Last updated:]
October 28, 2024
\item[First released:]
April 2, 2024
\item[Archive size:]
7.08 kB
\href{https://packages.typst.org/preview/letter-pro-3.0.0.tar.gz}{\pandocbounded{\includesvg[keepaspectratio]{/assets/icons/16-download.svg}}}
\item[Repository:]
\href{https://github.com/Sematre/typst-letter-pro}{GitHub}
\item[Categor y :]
\begin{itemize}
\tightlist
\item[]
\item
  \pandocbounded{\includesvg[keepaspectratio]{/assets/icons/16-envelope.svg}}
  \href{https://typst.app/universe/search/?category=office}{Office}
\end{itemize}
\end{description}

\subsubsection{Where to report issues?}\label{where-to-report-issues}

This template is a project of Sematre . Report issues on
\href{https://github.com/Sematre/typst-letter-pro}{their repository} .
You can also try to ask for help with this template on the
\href{https://forum.typst.app}{Forum} .

Please report this template to the Typst team using the
\href{https://typst.app/contact}{contact form} if you believe it is a
safety hazard or infringes upon your rights.

\phantomsection\label{versions}
\subsubsection{Version history}\label{version-history}

\begin{longtable}[]{@{}ll@{}}
\toprule\noalign{}
Version & Release Date \\
\midrule\noalign{}
\endhead
\bottomrule\noalign{}
\endlastfoot
3.0.0 & October 28, 2024 \\
\href{https://typst.app/universe/package/letter-pro/2.1.0/}{2.1.0} &
April 2, 2024 \\
\end{longtable}

Typst GmbH did not create this template and cannot guarantee correct
functionality of this template or compatibility with any version of the
Typst compiler or app.


\title{typst.app/universe/package/ichigo}

\phantomsection\label{banner}
\phantomsection\label{template-thumbnail}
\pandocbounded{\includegraphics[keepaspectratio]{https://packages.typst.org/preview/thumbnails/ichigo-0.2.0-small.webp}}

\section{ichigo}\label{ichigo}

{ 0.2.0 }

A customizable Typst template for homework

\href{/app?template=ichigo&version=0.2.0}{Create project in app}

\phantomsection\label{readme}
Homework Template - 作业模�

\subsection{Usage -
使ç''¨æ--¹æ³•}\label{usage---uxe4uxbduxe7uxe6uxb9uxe6uxb3}

\begin{Shaded}
\begin{Highlighting}[]
\NormalTok{\#import "@preview/ichigo:0.2.0": config, prob}

\NormalTok{\#show: config.with(}
\NormalTok{  course{-}name: "Typst 使用小练习",}
\NormalTok{  serial{-}str: "第 1 次作业",}
\NormalTok{  author{-}info: [}
\NormalTok{    sjfhsjfh from PKU{-}Typst}
\NormalTok{  ],}
\NormalTok{  author{-}names: "sjfhsjfh",}
\NormalTok{)}

\NormalTok{\#prob[}
\NormalTok{  Calculate the 25th number in the Fibonacci sequence using Typst}
\NormalTok{][}
\NormalTok{  \#let f(n) = \{}
\NormalTok{    if n \textless{}= 2 \{}
\NormalTok{      return 1}
\NormalTok{    \}}
\NormalTok{    return f(n {-} 1) + f(n {-} 2)}
\NormalTok{  \}}
\NormalTok{  \#f(25)}
\NormalTok{]}
\end{Highlighting}
\end{Shaded}

\subsection{Documentation - æ--‡æ¡£}\label{documentation---uxe6uxe6}

\href{https://github.com/PKU-Typst/ichigo/releases/download/v0.2.0/documentation.pdf}{Click
to download - 点击下载}

\subsection{TODO - å¾\ldots 办}\label{todo---uxe5uxbeuxe5ux161ux17e}

\begin{itemize}
\tightlist
\item
  {[} {]} Theme list \& documentation
\end{itemize}

\href{/app?template=ichigo&version=0.2.0}{Create project in app}

\subsubsection{How to use}\label{how-to-use}

Click the button above to create a new project using this template in
the Typst app.

You can also use the Typst CLI to start a new project on your computer
using this command:

\begin{verbatim}
typst init @preview/ichigo:0.2.0
\end{verbatim}

\includesvg[width=0.16667in,height=0.16667in]{/assets/icons/16-copy.svg}

\subsubsection{About}\label{about}

\begin{description}
\tightlist
\item[Author :]
\href{https://github.com/PKU-Typst}{PKU-Typst}
\item[License:]
MIT
\item[Current version:]
0.2.0
\item[Last updated:]
November 18, 2024
\item[First released:]
October 3, 2024
\item[Archive size:]
17.1 kB
\href{https://packages.typst.org/preview/ichigo-0.2.0.tar.gz}{\pandocbounded{\includesvg[keepaspectratio]{/assets/icons/16-download.svg}}}
\item[Repository:]
\href{https://github.com/PKU-Typst/ichigo}{GitHub}
\item[Categor y :]
\begin{itemize}
\tightlist
\item[]
\item
  \pandocbounded{\includesvg[keepaspectratio]{/assets/icons/16-speak.svg}}
  \href{https://typst.app/universe/search/?category=report}{Report}
\end{itemize}
\end{description}

\subsubsection{Where to report issues?}\label{where-to-report-issues}

This template is a project of PKU-Typst . Report issues on
\href{https://github.com/PKU-Typst/ichigo}{their repository} . You can
also try to ask for help with this template on the
\href{https://forum.typst.app}{Forum} .

Please report this template to the Typst team using the
\href{https://typst.app/contact}{contact form} if you believe it is a
safety hazard or infringes upon your rights.

\phantomsection\label{versions}
\subsubsection{Version history}\label{version-history}

\begin{longtable}[]{@{}ll@{}}
\toprule\noalign{}
Version & Release Date \\
\midrule\noalign{}
\endhead
\bottomrule\noalign{}
\endlastfoot
0.2.0 & November 18, 2024 \\
\href{https://typst.app/universe/package/ichigo/0.1.0/}{0.1.0} & October
3, 2024 \\
\end{longtable}

Typst GmbH did not create this template and cannot guarantee correct
functionality of this template or compatibility with any version of the
Typst compiler or app.


\title{typst.app/universe/package/socialhub-fa}

\phantomsection\label{banner}
\section{socialhub-fa}\label{socialhub-fa}

{ 1.0.0 }

A Typst library for Social Media references with icons based on Font
Awesome.

\phantomsection\label{readme}
The \texttt{\ socialhub-fa\ } package is designed to help you create
your curriculum vitae (CV). It allows you to easily reference your
social media profiles with the typical icon of the service plus a link
to your profile.

\subsection{Features}\label{features}

\begin{itemize}
\tightlist
\item
  Support for popular social media, developer and career platforms
\item
  Uniform design for all entries
\item
  Based on the Internet’s icon library
  \href{https://fontawesome.com/}{Font Awesome}
\item
  Easy to use
\item
  Allows the customization of the look (extra args are passed to
  \href{https://typst.app/docs/reference/text/text/}{\texttt{\ text\ }}
  )
\end{itemize}

\subsection{Fonts Installation}\label{fonts-installation}

\subsubsection{Linux}\label{linux}

\begin{enumerate}
\tightlist
\item
  \href{https://fontawesome.com/download}{Download Font Awesome for
  Desktop}
\item
  Unzip the file
\item
  Switch into the \texttt{\ otfs\ } folder within the unzipped folder
\item
  Run \texttt{\ mkdir\ -p\ /usr/share/fonts/truetype/\ }
\item
  Run
  \texttt{\ install\ -m644\ \textquotesingle{}Font\ Awesome\ 6\ Brands-Regular-400.otf\textquotesingle{}\ /usr/share/fonts/truetype/\ }
\item
  Unfortunately not all brands are included in the brands font file, so
  you must also run
  \texttt{\ install\ -m644\ \textquotesingle{}Font\ Awesome\ 6\ Free-Regular-400.otf\textquotesingle{}\ /usr/share/fonts/truetype/\ }
\end{enumerate}

\subsection{Usage}\label{usage}

\subsubsection{Using Typst’s package
manager}\label{using-typstuxe2s-package-manager}

You can install the library using the
\href{https://github.com/typst/packages}{typst packages} :

\begin{Shaded}
\begin{Highlighting}[]
\NormalTok{\#import "@preview/socialhub{-}fa:1.0.0": *}
\end{Highlighting}
\end{Shaded}

\subsubsection{Install manually}\label{install-manually}

Put the \texttt{\ socialhub-fa.typ\ } file in your project directory and
import it:

\begin{Shaded}
\begin{Highlighting}[]
\NormalTok{\#import "socialhub{-}fa.typ": *}
\end{Highlighting}
\end{Shaded}

\subsubsection{Minimal Example}\label{minimal-example}

\begin{Shaded}
\begin{Highlighting}[]
\NormalTok{// \#import "@preview/socialhub{-}fa:1.0.0": github{-}info, gitlab{-}info}
\NormalTok{\#import "socialhub{-}fa.typ": github{-}info, gitlab{-}info}

\NormalTok{This project was created by \#github{-}info("Bi0T1N"). You can also find me on \#gitlab{-}info("GitLab", rgb("\#811052"), url: "https://gitlab.com/Bi0T1N").}
\end{Highlighting}
\end{Shaded}

\subsubsection{Examples}\label{examples}

See the
\href{https://github.com/typst/packages/raw/main/packages/preview/socialhub-fa/1.0.0/examples/examples.typ}{\texttt{\ examples.typ\ }}
file for a complete example. The
\href{https://github.com/typst/packages/raw/main/packages/preview/socialhub-fa/1.0.0/examples/}{generated
PDF files} are also available for preview.

\subsection{Troubleshooting}\label{troubleshooting}

\subsubsection{Icons are not displayed
correctly}\label{icons-are-not-displayed-correctly}

Make sure that you have installed the required Font Awesome
ligature-based font files.

\subsection{Contribution}\label{contribution}

Feel free to open an issue or a pull request if you find any problems or
have any suggestions.

\subsection{License}\label{license}

This library is licensed under the MIT license. Feel free to use it in
your project.

\subsubsection{How to add}\label{how-to-add}

Copy this into your project and use the import as
\texttt{\ socialhub-fa\ }

\begin{verbatim}
#import "@preview/socialhub-fa:1.0.0"
\end{verbatim}

\includesvg[width=0.16667in,height=0.16667in]{/assets/icons/16-copy.svg}

Check the docs for
\href{https://typst.app/docs/reference/scripting/\#packages}{more
information on how to import packages} .

\subsubsection{About}\label{about}

\begin{description}
\tightlist
\item[Author :]
Nico Neumann (Bi0T1N)
\item[License:]
MIT
\item[Current version:]
1.0.0
\item[Last updated:]
December 3, 2023
\item[First released:]
December 3, 2023
\item[Archive size:]
3.08 kB
\href{https://packages.typst.org/preview/socialhub-fa-1.0.0.tar.gz}{\pandocbounded{\includesvg[keepaspectratio]{/assets/icons/16-download.svg}}}
\item[Repository:]
\href{https://github.com/Bi0T1N/typst-socialhub-fa}{GitHub}
\end{description}

\subsubsection{Where to report issues?}\label{where-to-report-issues}

This package is a project of Nico Neumann (Bi0T1N) . Report issues on
\href{https://github.com/Bi0T1N/typst-socialhub-fa}{their repository} .
You can also try to ask for help with this package on the
\href{https://forum.typst.app}{Forum} .

Please report this package to the Typst team using the
\href{https://typst.app/contact}{contact form} if you believe it is a
safety hazard or infringes upon your rights.

\phantomsection\label{versions}
\subsubsection{Version history}\label{version-history}

\begin{longtable}[]{@{}ll@{}}
\toprule\noalign{}
Version & Release Date \\
\midrule\noalign{}
\endhead
\bottomrule\noalign{}
\endlastfoot
1.0.0 & December 3, 2023 \\
\end{longtable}

Typst GmbH did not create this package and cannot guarantee correct
functionality of this package or compatibility with any version of the
Typst compiler or app.


\title{typst.app/universe/package/tbl}

\phantomsection\label{banner}
\section{tbl}\label{tbl}

{ 0.0.4 }

Complex tables, written concisely

\phantomsection\label{readme}
This is a library for \href{https://typst.app/}{Typst} built upon Pg
Biel’s fabulous
\href{https://github.com/PgBiel/typst-tablex}{\texttt{\ tablex\ }}
library.

It allows the creation of complex tables in Typst using a compact syntax
based on the \texttt{\ tbl\ } preprocessor for the traditional UNIX
TROFF typesetting system. There are also some novel features that are
not currently offered by Typst itself or \texttt{\ tablex\ } , namely:

\begin{itemize}
\tightlist
\item
  Decimal point alignment (using the \texttt{\ decimalpoint\ } region
  option and \texttt{\ N\ } -classified columns)
\item
  Columns of equal width (using the \texttt{\ e\ } column modifier)
\item
  Columns with a guaranteed minimum width (using the \texttt{\ w\ }
  column modifier)
\item
  Cells that are ignored when calculating column widths (using the
  \texttt{\ z\ } column modifier)
\item
  Equation tables (using the \texttt{\ mode:\ "math"\ } region option)
\end{itemize}

Many other features exist to condense common configurations to a concise
syntax.

For example:

\begin{verbatim}
#import "@preview/tbl:0.0.4"
#show: tbl.template.with(box: true, tab: "|")

```tbl
      R | L
      R   N.
software|version
_
     AFL|2.39b
    Mutt|1.8.0
    Ruby|1.8.7.374
TeX Live|2015
```
\end{verbatim}

\pandocbounded{\includegraphics[keepaspectratio]{https://raw.githubusercontent.com/maxcrees/tbl.typ/v0.0.4/test/00/02_software.png}}

Many other examples and copious documentation are provided in the
\href{https://maxre.es/tbl.typ/v0.0.4.pdf}{\texttt{\ README.pdf\ }}
file.

\href{https://github.com/maxcrees/tbl.typ}{The source repository} also
includes a test suite based on those examples, which can be ran using
the GNU \texttt{\ make\ } command. See \texttt{\ make\ help\ } for
details.

\subsubsection{How to add}\label{how-to-add}

Copy this into your project and use the import as \texttt{\ tbl\ }

\begin{verbatim}
#import "@preview/tbl:0.0.4"
\end{verbatim}

\includesvg[width=0.16667in,height=0.16667in]{/assets/icons/16-copy.svg}

Check the docs for
\href{https://typst.app/docs/reference/scripting/\#packages}{more
information on how to import packages} .

\subsubsection{About}\label{about}

\begin{description}
\tightlist
\item[Author :]
Max Rees
\item[License:]
MPL-2.0
\item[Current version:]
0.0.4
\item[Last updated:]
August 19, 2023
\item[First released:]
July 29, 2023
\item[Archive size:]
14.3 kB
\href{https://packages.typst.org/preview/tbl-0.0.4.tar.gz}{\pandocbounded{\includesvg[keepaspectratio]{/assets/icons/16-download.svg}}}
\item[Repository:]
\href{https://github.com/maxcrees/tbl.typ}{GitHub}
\end{description}

\subsubsection{Where to report issues?}\label{where-to-report-issues}

This package is a project of Max Rees . Report issues on
\href{https://github.com/maxcrees/tbl.typ}{their repository} . You can
also try to ask for help with this package on the
\href{https://forum.typst.app}{Forum} .

Please report this package to the Typst team using the
\href{https://typst.app/contact}{contact form} if you believe it is a
safety hazard or infringes upon your rights.

\phantomsection\label{versions}
\subsubsection{Version history}\label{version-history}

\begin{longtable}[]{@{}ll@{}}
\toprule\noalign{}
Version & Release Date \\
\midrule\noalign{}
\endhead
\bottomrule\noalign{}
\endlastfoot
0.0.4 & August 19, 2023 \\
\href{https://typst.app/universe/package/tbl/0.0.3/}{0.0.3} & July 29,
2023 \\
\end{longtable}

Typst GmbH did not create this package and cannot guarantee correct
functionality of this package or compatibility with any version of the
Typst compiler or app.


\title{typst.app/universe/package/guided-resume-starter-cgc}

\phantomsection\label{banner}
\phantomsection\label{template-thumbnail}
\pandocbounded{\includegraphics[keepaspectratio]{https://packages.typst.org/preview/thumbnails/guided-resume-starter-cgc-2.0.0-small.webp}}

\section{guided-resume-starter-cgc}\label{guided-resume-starter-cgc}

{ 2.0.0 }

A guided starter resume for people looking to focus on highlighting
their experience -\/- without having to worry about the hassle of
formatting.

\href{/app?template=guided-resume-starter-cgc&version=2.0.0}{Create
project in app}

\phantomsection\label{readme}
This template is a starter resume for people looking to focus on the
content of their resume, without having to worry about the hassle of
formatting.

\subsection{Get Started!}\label{get-started}

\subsubsection{Quickstart: Typst
Universe}\label{quickstart-typst-universe}

\begin{enumerate}
\tightlist
\item
  If you haven’t already, \href{https://typst.app/}{create a (free!)
  Typst account} .
\item
  Once you have an account, go to the template on
  \href{https://typst.app/universe/package/resume-starter-cgc}{Typst
  Universe}
\item
  Click on “Create Project in App�, give your project a title, and
  press “Create�.
\item
  Start editing! This copy is your own personal copy to edit however you
  want!
\end{enumerate}

There are two files included in this project:

\begin{itemize}
\tightlist
\item
  \texttt{\ starter.typ\ } contains the full template, along with a
  written guide to help you put your best (single-paged) foot forward!
\item
  \texttt{\ resume.typ\ } is the same template, but without the full
  guide included.
\item
  \texttt{\ templates/resume.template.typ\ } contains the formatting and
  style for the underlying pieces.
\end{itemize}

\textbf{I would highly recommend reading \texttt{\ starter.typ\ } or
skimming through the
\href{https://blog.chaoticgood.computer/writing/notes/typst-resume-template}{online
guide} to understand best practices when using the template.}

\subsubsection{Alternative: Typst CLI}\label{alternative-typst-cli}

If you’d prefer to simply download \& modify the template, you can use
the \href{https://github.com/typst/typst}{Typst CLI} to download it
instead:

\begin{Shaded}
\begin{Highlighting}[]
\ExtensionTok{typst}\NormalTok{ init @preview/resume{-}starter{-}cgc}
\end{Highlighting}
\end{Shaded}

\subsection{Layout}\label{layout}

\subsubsection{Header}\label{header}

The resume can be created with a header with the following attributes:

\begin{itemize}
\tightlist
\item
  \texttt{\ author\ } : Your name
\item
  \texttt{\ location\ } : The city, state/province, and country you
  reside in.
\item
  \texttt{\ contacts\ } : A list of contact information and additional
  information
\end{itemize}

\paragraph{Header Example}\label{header-example}

\begin{Shaded}
\begin{Highlighting}[]
\NormalTok{\#show: resume.with(}
\NormalTok{  author: "Dr. Emmit \textbackslash{}"Doc\textbackslash{}" Brown",}
\NormalTok{  location: "Hill Valley, CA",}
\NormalTok{  contacts: (}
\NormalTok{    [\#link("mailto:your\_email@yourmail.com")[Email]],}
\NormalTok{    [\#link("https://your{-}cool{-}site.com")[Website]],}
\NormalTok{    [\#link("https://github.com/your{-}linkedin")[GitHub]],}
\NormalTok{    [\#link("https://linkedin.com/in/your{-}linkedin")[LinkedIn]],}
\NormalTok{  )}
\NormalTok{)}
\end{Highlighting}
\end{Shaded}

\subsubsection{Education}\label{education}

The Education ( \texttt{\ \#edu\ } ) section can be used to highlight
for formal education and certifications.

\begin{itemize}
\tightlist
\item
  \texttt{\ institution\ } : Name of the institution where you study, or
  have graduated from.
\item
  \texttt{\ date\ } : Your graduation date, or expected graduation date.
\item
  \texttt{\ degrees\ } : The degrees you received at the institution

  \begin{itemize}
  \tightlist
  \item
    Each entry is two sections: the \textbf{title} of the degree, and
    the \textbf{subject} that you studied.
  \end{itemize}
\item
  \texttt{\ gpa\ } (optional): Your GPA, or other additional
  information.
\end{itemize}

\paragraph{Education Example}\label{education-example}

\begin{Shaded}
\begin{Highlighting}[]
\NormalTok{\#edu(}
\NormalTok{  institution: "University of Colombia",}
\NormalTok{  date: "Aug 1948",}
\NormalTok{  gpa: "3.9 of 4.0, Summa Cum Laude",}
\NormalTok{  degrees: (}
\NormalTok{    ("Bachelor\textquotesingle{}s of Science", "Nuclear Engineering"),}
\NormalTok{    ("Minors", "Automobile Design, Arabic"),}
\NormalTok{    ("Focus", "Childcare, Education")}
\NormalTok{  ),}
\NormalTok{)}
\end{Highlighting}
\end{Shaded}

\subsubsection{Skills}\label{skills}

An additional Skills ( \texttt{\ \#skills\ } ) section to list skills
relevant to the job you’re applying for.

The input is a list of \texttt{\ Label:\ Skills{[}{]}\ } , in order to
easily toggle comments on skills that you may want to leave in but not
render for a particular application.

\paragraph{Skills Example}\label{skills-example}

\begin{Shaded}
\begin{Highlighting}[]
\NormalTok{\#skills((}
\NormalTok{  ("Expertise", (}
\NormalTok{    [Theoretical Physics],}
\NormalTok{    [Time Travel],}
\NormalTok{    [Nuclear Material Management],}
\NormalTok{    [Student Mentoring],}
\NormalTok{  )),}
\NormalTok{  ("Software", (}
\NormalTok{    [AutoDesk CAD],}
\NormalTok{    [Delorean OS],}
\NormalTok{    [Windows 1],}
\NormalTok{  )),}
\NormalTok{  ("Languages", (}
\NormalTok{    [C++],}
\NormalTok{    [C Language],}
\NormalTok{    [MatLab],}
\NormalTok{    [Punch Cards],}
\NormalTok{  )),}
\NormalTok{))}
\end{Highlighting}
\end{Shaded}

\subsubsection{Experience}\label{experience}

The bulk of your resume, the Experience ( \texttt{\ \#exp\ } ) sections
provide a compact \& concise formatting for bulleted details of your
previous and current work experience.

This section is meant to be flexible, and can also be used to talk about
projects and other experiences that may fall outside of the traditional
definition of “work.�

\begin{itemize}
\tightlist
\item
  \texttt{\ role\ } : The title of your position/role in this
  experience.
\item
  \texttt{\ project\ } : The company you worked at, or the name of the
  project you worked on.
\item
  \texttt{\ date\ } : The start and end dates of this experience.
\item
  \texttt{\ details\ } : A description of the work you did in this
  position

  \begin{itemize}
  \tightlist
  \item
    It is \textbf{highly encouraged} to use bullet points in this
    section.
  \end{itemize}
\item
  \texttt{\ location\ } (optional): The location of the experience
\item
  \texttt{\ summary\ } (optional): A brief summary of the company’s
  mission or project goal.
\end{itemize}

\paragraph{Experience Example}\label{experience-example}

\begin{Shaded}
\begin{Highlighting}[]
\NormalTok{\#exp(}
\NormalTok{  role: "Theoretical Physics Consultant",}
\NormalTok{  project: "Doc Brown\textquotesingle{}s Garage",}
\NormalTok{  date: "June 1953 {-} Oct 2015",}
\NormalTok{  location: "Hill Valley, CA",}
\NormalTok{  summary: "Specializing in development of time travel devices and student tutoring",}
\NormalTok{  details: [}
\NormalTok{    {-} Lead development of time travel devices, resulting in the ability to travel back and forth through time}
\NormalTok{    {-} Managed and executed a budget of \textbackslash{}$14 million dollars gained from an unexplained family fortune}
\NormalTok{    {-} Oversaw QA testing for time travel devices, minimizing risk of maternal time{-}travel related incidents}
\NormalTok{  ]}
\NormalTok{)}
\end{Highlighting}
\end{Shaded}

\subsection{Questions \& Suggestions}\label{questions-suggestions}

Have any questions, comments, or suggestions about the template? Please
feel free to reach out at
\href{mailto:mentoring@chaoticgood.computer}{\texttt{\ mentoring@chaoticgood.computer\ }}
!

\href{/app?template=guided-resume-starter-cgc&version=2.0.0}{Create
project in app}

\subsubsection{How to use}\label{how-to-use}

Click the button above to create a new project using this template in
the Typst app.

You can also use the Typst CLI to start a new project on your computer
using this command:

\begin{verbatim}
typst init @preview/guided-resume-starter-cgc:2.0.0
\end{verbatim}

\includesvg[width=0.16667in,height=0.16667in]{/assets/icons/16-copy.svg}

\subsubsection{About}\label{about}

\begin{description}
\tightlist
\item[Author s :]
\href{https://chaoticgood.computer}{Spencer Elkington} \&
\href{mailto:spencer@chaoticgood.computer}{Spencer Elkington}
\item[License:]
Unlicense
\item[Current version:]
2.0.0
\item[Last updated:]
May 6, 2024
\item[First released:]
May 6, 2024
\item[Archive size:]
22.6 kB
\href{https://packages.typst.org/preview/guided-resume-starter-cgc-2.0.0.tar.gz}{\pandocbounded{\includesvg[keepaspectratio]{/assets/icons/16-download.svg}}}
\item[Repository:]
\href{https://github.com/chaoticgoodcomputing/typst-resume-starter}{GitHub}
\item[Categor y :]
\begin{itemize}
\tightlist
\item[]
\item
  \pandocbounded{\includesvg[keepaspectratio]{/assets/icons/16-user.svg}}
  \href{https://typst.app/universe/search/?category=cv}{CV}
\end{itemize}
\end{description}

\subsubsection{Where to report issues?}\label{where-to-report-issues}

This template is a project of Spencer Elkington and Spencer Elkington .
Report issues on
\href{https://github.com/chaoticgoodcomputing/typst-resume-starter}{their
repository} . You can also try to ask for help with this template on the
\href{https://forum.typst.app}{Forum} .

Please report this template to the Typst team using the
\href{https://typst.app/contact}{contact form} if you believe it is a
safety hazard or infringes upon your rights.

\phantomsection\label{versions}
\subsubsection{Version history}\label{version-history}

\begin{longtable}[]{@{}ll@{}}
\toprule\noalign{}
Version & Release Date \\
\midrule\noalign{}
\endhead
\bottomrule\noalign{}
\endlastfoot
2.0.0 & May 6, 2024 \\
\end{longtable}

Typst GmbH did not create this template and cannot guarantee correct
functionality of this template or compatibility with any version of the
Typst compiler or app.


\title{typst.app/universe/package/clean-math-thesis}

\phantomsection\label{banner}
\phantomsection\label{template-thumbnail}
\pandocbounded{\includegraphics[keepaspectratio]{https://packages.typst.org/preview/thumbnails/clean-math-thesis-0.2.0-small.webp}}

\section{clean-math-thesis}\label{clean-math-thesis}

{ 0.2.0 }

A simple and good looking template for mathematical theses

\href{/app?template=clean-math-thesis&version=0.2.0}{Create project in
app}

\phantomsection\label{readme}
\href{https://github.com/sebaseb98/clean-math-thesis/actions/workflows/build.yml}{\pandocbounded{\includesvg[keepaspectratio]{https://github.com/sebaseb98/clean-math-thesis/actions/workflows/build.yml/badge.svg}}}
\href{https://github.com/sebaseb98/clean-math-thesis}{\pandocbounded{\includegraphics[keepaspectratio]{https://img.shields.io/badge/GitHub-repo-blue}}}
\href{https://opensource.org/licenses/MIT}{\pandocbounded{\includesvg[keepaspectratio]{https://img.shields.io/badge/License-MIT-success.svg}}}

\href{https://typst.app/home/}{Typst} thesis template for mathematical
theses built for simple, efficient use and a clean look. Of course, it
can also be used for other subjects, but the following math-specific
features are already contained in the template:

\begin{itemize}
\tightlist
\item
  theorems, lemmas, corollaries, proofs etc. prepared using
  \href{https://typst.app/universe/package/great-theorems}{great-theorems}
\item
  equation settings (using either
  \href{https://typst.app/universe/package/equate}{equate} for numbering
  of subequations or
  \href{https://typst.app/universe/package/i-figured/}{i-figured} for
  equation numbering which includes the chapter number)
\item
  pseudocode package
  \href{https://typst.app/universe/package/lovelace}{lovelace} included.
\end{itemize}

Additionally, it has headers built with
\href{https://typst.app/universe/package/hydra}{hydra} .

\subsection{Set-Up}\label{set-up}

The template is already filled with dummy data, to give users an
\href{https://github.com/sebaseb98/clean-math-thesis/blob/main/template/main.pdf}{impression
how it looks like} . The thesis is obtained by compiling
\texttt{\ main.typ\ } .

\begin{itemize}
\tightlist
\item
  after
  \href{https://github.com/typst/typst?tab=readme-ov-file\#installation}{installing
  Typst} you can conveniently use the following to create a new folder
  containing this project.
\end{itemize}

\begin{Shaded}
\begin{Highlighting}[]
\ExtensionTok{typst}\NormalTok{ init @preview/clean{-}math{-}thesis:0.2.0}
\end{Highlighting}
\end{Shaded}

\begin{itemize}
\tightlist
\item
  edit the data in \texttt{\ main.typ\ } â†'
  \texttt{\ \#show\ template.with({[}your\ data{]})\ }
\end{itemize}

\subsubsection{Parameters of the
Template}\label{parameters-of-the-template}

\ul{personal/subject related information}

\begin{itemize}
\tightlist
\item
  \texttt{\ author\ } : Name of the author of the thesis.
\item
  \texttt{\ title\ } : Title of the thesis.
\item
  \texttt{\ supervisor1\ } : Name of the first supervisor.
\item
  \texttt{\ supervisor2\ } : Name of the second supervisor.
\item
  \texttt{\ degree\ } : Degree for which the thesis is submitted.
\item
  \texttt{\ program\ } : Program under which the thesis is submitted.
\item
  \texttt{\ university\ } : Name of the university.
\item
  \texttt{\ institute\ } : Name of the institute.
\item
  \texttt{\ deadline\ } : Submission deadline of the thesis.
\end{itemize}

\ul{file paths for logos etc.}

\begin{itemize}
\tightlist
\item
  \texttt{\ uni-logo\ } : Image, e.g.
  \texttt{\ image("images/logo\_placeholder.svg",\ width:\ 50\%)\ }
\item
  \texttt{\ institute-logo\ } : Image.
\end{itemize}

\ul{formatting settings}

\begin{itemize}
\tightlist
\item
  \texttt{\ citation-style\ } : Citation style to be used in the thesis.
\item
  \texttt{\ body-font\ } : Font to be used for the body text.
\item
  \texttt{\ cover-font\ } : Font to be used for the cover text.
\end{itemize}

\ul{content that needs to be placed differently then normal chapters}

\begin{itemize}
\tightlist
\item
  \texttt{\ abstract\ } : Content for the abstract section.
\end{itemize}

\ul{equation settings}

\begin{itemize}
\tightlist
\item
  \texttt{\ equate-settings\ } : either none -\textgreater{} use
  i-figured; or tuple with the settings for the equations (see
  \href{https://typst.app/universe/package/equate}{docs} ), e.g.
  (breakable: true, sub-numbering: true, number-mode: “label�) The
  switching between these is currently not optimal: i-figured needs a
  prefix ( \texttt{\ eq:\ } ) so if we label an equation like
  \texttt{\ \textless{}equation\textgreater{}\ } the corresponding
  reference is \texttt{\ @eq:equation\ } and for equate we don’t have
  this prefix, i.e. the reference would be \texttt{\ @equation\ } in
  this example. This is something to be improved in future releases.
\item
  \texttt{\ equation-numbering-pattern\ } : specify the
  \href{https://typst.app/docs/reference/model/numbering/\#parameters-numbering}{numbering}
  of the equations. The second counting symbol (e.g. the \texttt{\ a\ }
  in \texttt{\ "(1.a)"\ } ) is either used for subequation numbering or
  for the numbering of equations in the chapters. \ul{colors}
\item
  \texttt{\ cover-color\ } : Color used for the cover.
\item
  \texttt{\ heading-color\ } : Color used for headings.
\item
  \texttt{\ link-color\ } : Color used for links and references.
\end{itemize}

\subsubsection{Other Customizations}\label{other-customizations}

\begin{itemize}
\tightlist
\item
  \texttt{\ declaration.typ\ } should be modified
\item
  when adding chapters, remember to include them into the
  \texttt{\ main.typ\ } .
\item
  (optional) change colors and appearance of the theorem environment in
  the \texttt{\ customization/\ } -folder.
\end{itemize}

\subsubsection{Use of the template in existing
projects}\label{use-of-the-template-in-existing-projects}

If you want to change an existing typst project structure to use this
template, just type the following lines

\begin{Shaded}
\begin{Highlighting}[]
\NormalTok{\#import "@preview/clean{-}math{-}thesis:0.1.0": template}

\NormalTok{\#show: template.with(}
\NormalTok{  // your user specific data, parameters explained above}
\NormalTok{)}

\NormalTok{\#include "my\_content.typ"  // and eventually more files}
\end{Highlighting}
\end{Shaded}

\subsection{Disclaimer}\label{disclaimer}

This template was created after Sebastian finished his master’s
thesis. We do not guarantee that it will be accepted by any university,
please clarify in advance if it fulfills all requirements. If not, this
template might still be a good starting point.

\subsection{Acknowledgements}\label{acknowledgements}

As inspiration on how to structure this template, we used the
\href{https://typst.app/universe/package/modern-unito-thesis}{modern-unito-thesis}
template. The design is inspired by the
\href{https://github.com/FAU-AMMN/fau-book}{fau-book} template.

\subsection{Feedback \& Improvements}\label{feedback-improvements}

If you encounter problems, please open issues. In case you found useful
extensions or improved anything We are also very happy to accept pull
requests.

\href{/app?template=clean-math-thesis&version=0.2.0}{Create project in
app}

\subsubsection{How to use}\label{how-to-use}

Click the button above to create a new project using this template in
the Typst app.

You can also use the Typst CLI to start a new project on your computer
using this command:

\begin{verbatim}
typst init @preview/clean-math-thesis:0.2.0
\end{verbatim}

\includesvg[width=0.16667in,height=0.16667in]{/assets/icons/16-copy.svg}

\subsubsection{About}\label{about}

\begin{description}
\tightlist
\item[Author s :]
\href{https://github.com/sebaseb98}{Sebastian Eberle} \&
\href{https://github.com/JoshuaLampert}{Joshua Lampert}
\item[License:]
MIT
\item[Current version:]
0.2.0
\item[Last updated:]
November 26, 2024
\item[First released:]
November 12, 2024
\item[Minimum Typst version:]
0.12.0
\item[Archive size:]
9.60 kB
\href{https://packages.typst.org/preview/clean-math-thesis-0.2.0.tar.gz}{\pandocbounded{\includesvg[keepaspectratio]{/assets/icons/16-download.svg}}}
\item[Repository:]
\href{https://github.com/sebaseb98/clean-math-thesis}{GitHub}
\item[Discipline :]
\begin{itemize}
\tightlist
\item[]
\item
  \href{https://typst.app/universe/search/?discipline=mathematics}{Mathematics}
\end{itemize}
\item[Categor y :]
\begin{itemize}
\tightlist
\item[]
\item
  \pandocbounded{\includesvg[keepaspectratio]{/assets/icons/16-mortarboard.svg}}
  \href{https://typst.app/universe/search/?category=thesis}{Thesis}
\end{itemize}
\end{description}

\subsubsection{Where to report issues?}\label{where-to-report-issues}

This template is a project of Sebastian Eberle and Joshua Lampert .
Report issues on
\href{https://github.com/sebaseb98/clean-math-thesis}{their repository}
. You can also try to ask for help with this template on the
\href{https://forum.typst.app}{Forum} .

Please report this template to the Typst team using the
\href{https://typst.app/contact}{contact form} if you believe it is a
safety hazard or infringes upon your rights.

\phantomsection\label{versions}
\subsubsection{Version history}\label{version-history}

\begin{longtable}[]{@{}ll@{}}
\toprule\noalign{}
Version & Release Date \\
\midrule\noalign{}
\endhead
\bottomrule\noalign{}
\endlastfoot
0.2.0 & November 26, 2024 \\
\href{https://typst.app/universe/package/clean-math-thesis/0.1.0/}{0.1.0}
& November 12, 2024 \\
\end{longtable}

Typst GmbH did not create this template and cannot guarantee correct
functionality of this template or compatibility with any version of the
Typst compiler or app.


\title{typst.app/universe/package/rose-pine}

\phantomsection\label{banner}
\section{rose-pine}\label{rose-pine}

{ 0.2.0 }

Soho vibes for Typst in a form of easily applicable theme.

\phantomsection\label{readme}
\includegraphics[width=0.83333in,height=\textheight,keepaspectratio]{https://raw.githubusercontent.com/rose-pine/rose-pine-theme/main/assets/icon.png}

\subsection{Rosé Pine for Typst}\label{rosuxe3-pine-for-typst}

All natural pine, faux fur and a bit of soho vibes for the classy
minimalist

\href{https://github.com/rose-pine/rose-pine-theme}{\pandocbounded{\includegraphics[keepaspectratio]{https://img.shields.io/badge/community-ros\%C3\%A9\%20pine-26233a?labelColor=191724&logo=data:image/svg+xml;base64,PHN2ZyB3aWR0aD0iMjUwIiBoZWlnaHQ9IjIzNyIgdmlld0JveD0iMCAwIDI1MCAyMzciIGZpbGw9Im5vbmUiIHhtbG5zPSJodHRwOi8vd3d3LnczLm9yZy8yMDAwL3N2ZyI+CjxwYXRoIGQ9Ik0xNjEuMjI3IDE2MS4yNTFDMTMyLjE1NCAxNjkuMDQxIDExNC45MDEgMTk4LjkyNCAxMjIuNjkxIDIyNy45OTdDMTIzLjkyNSAyMzIuNjAzIDEyOC42NTkgMjM1LjMzNiAxMzMuMjY0IDIzNC4xMDJMMTg1LjkwNyAyMTkuOTk2QzIxOS41ODUgMjEwLjk3MiAyMzkuNTcgMTc2LjM1NCAyMzAuNTQ2IDE0Mi42NzdMMTYxLjIyNyAxNjEuMjUxWiIgZmlsbD0iIzI0NjI3QiIvPgo8cGF0aCBkPSJNODguMTgzNiAxNTkuOTg4QzExNy4yNTcgMTY3Ljc3OCAxMzQuNTEgMTk3LjY2MiAxMjYuNzIgMjI2LjczNUMxMjUuNDg2IDIzMS4zNCAxMjAuNzUyIDIzNC4wNzMgMTE2LjE0NyAyMzIuODM5TDYzLjUwNDEgMjE4LjczM0MyOS44MjY0IDIwOS43MSA5Ljg0MDk0IDE3NS4wOTIgMTguODY0OSAxNDEuNDE0TDg4LjE4MzYgMTU5Ljk4OFoiIGZpbGw9IiMyNDYyN0IiLz4KPHBhdGggZD0iTTE4Ni44NjcgMTcyLjk4QzE1Mi4wMDIgMTcyLjk4IDEyMy43MzcgMjAxLjI0NSAxMjMuNzM3IDIzNi4xMTFIMTg2Ljg3QzIyMS43MzYgMjM2LjExMSAyNTAgMjA3Ljg0NiAyNTAgMTcyLjk4TDE4Ni44NjcgMTcyLjk4WiIgZmlsbD0iIzMxNzQ4RiIvPgo8cGF0aCBkPSJNNjMuMTMyNyAxNzIuOThDOTcuOTk4NCAxNzIuOTggMTI2LjI2MyAyMDEuMjQ1IDEyNi4yNjMgMjM2LjExMUg2My4xM0MyOC4yNjQyIDIzNi4xMTEgLTEuNTI0MDNlLTA2IDIwNy44NDYgMCAxNzIuOThMNjMuMTMyNyAxNzIuOThaIiBmaWxsPSIjMzE3NDhGIi8+CjxwYXRoIGQ9Ik0xNzEuNzE3IDc1LjEyNjNDMTcxLjcxNyAxMDEuMjc2IDE1MC41MTggMTIyLjQ3NSAxMjQuMzY5IDEyMi40NzVDOTguMjE4OCAxMjIuNDc1IDc3LjAyMDIgMTAxLjI3NiA3Ny4wMjAyIDc1LjEyNjNDNzcuMDIwMiA0OC45NzY0IDk4LjIxODggMjcuNzc3OCAxMjQuMzY5IDI3Ljc3NzhDMTUwLjUxOCAyNy43Nzc4IDE3MS43MTcgNDguOTc2NCAxNzEuNzE3IDc1LjEyNjNaIiBmaWxsPSIjRUJCQ0JBIi8+CjxwYXRoIGQ9Ik0xNDQuMjE3IDg2LjIzNzlDMTYxLjY0OSA1Ni4wNDMyIDE1MS4zMDMgMTcuNDMyOSAxMjEuMTA4IDBMMTA2LjA2IDI2LjA2NDRDODguNjI3IDU2LjI1OSA5OC45NzM2IDk0Ljg2OTQgMTI5LjE2OCAxMTIuMzAyTDE0NC4yMTcgODYuMjM3OVoiIGZpbGw9IiNFQkJDQkEiLz4KPHBhdGggZD0iTTEyNS4yOTkgNjAuOTc4OUMxMTYuMjc1IDI3LjMwMTIgODEuNjU3NSA3LjMxNTY3IDQ3Ljk3OTcgMTYuMzM5Nkw2NC4zMTk3IDc3LjMyMTFDNzMuMzQzNiAxMTAuOTk5IDEwNy45NjEgMTMwLjk4NCAxNDEuNjM5IDEyMS45NkwxMjUuMjk5IDYwLjk3ODlaIiBmaWxsPSIjRUJCQ0JBIi8+CjxwYXRoIGQ9Ik0xMjQuOTI2IDYwLjk3ODlDMTMzLjk1IDI3LjMwMTIgMTY4LjU2NyA3LjMxNTY3IDIwMi4yNDUgMTYuMzM5NkwxODUuOTA1IDc3LjMyMTFDMTc2Ljg4MSAxMTAuOTk5IDE0Mi4yNjMgMTMwLjk4NCAxMDguNTg2IDEyMS45NkwxMjQuOTI2IDYwLjk3ODlaIiBmaWxsPSIjRUJCQ0JBIi8+Cjwvc3ZnPgo=&style=for-the-badge}}}

\subsection{Usage}\label{usage}

\begin{enumerate}
\tightlist
\item
  Open a Typst document
\item
  Import this library:
\end{enumerate}

\begin{Shaded}
\begin{Highlighting}[]
\NormalTok{\#import "@preview/rose{-}pine:0.2.0": apply}
\end{Highlighting}
\end{Shaded}

\begin{enumerate}
\setcounter{enumi}{2}
\tightlist
\item
  Apply the theme to the document using \texttt{\ \#show\ } :
\end{enumerate}

\begin{Shaded}
\begin{Highlighting}[]
\NormalTok{\#show: apply()}
\end{Highlighting}
\end{Shaded}

\begin{enumerate}
\setcounter{enumi}{3}
\tightlist
\item
  Use other colors anywhere you want:
\end{enumerate}

\begin{Shaded}
\begin{Highlighting}[]
\NormalTok{\#import "@preview/rose{-}pine:0.2.0": rose{-}pine}

\NormalTok{\#text(fill: rose{-}pine.love)[Some red text]}
\end{Highlighting}
\end{Shaded}

You can also use other variants by importing them/passing their name to
\texttt{\ apply()\ } . E.g.

\begin{Shaded}
\begin{Highlighting}[]
\NormalTok{\#import "@preview/rose{-}pine:0.2.0": apply, rose{-}pine{-}dawn}
\NormalTok{\#show: apply(variant: "rose{-}pine{-}dawn")}
\end{Highlighting}
\end{Shaded}

\subsection{Gallery}\label{gallery}

\pandocbounded{\includegraphics[keepaspectratio]{https://github.com/oplik0/rose-pine-typst/assets/25460763/0f3f4c3a-a923-4587-bd80-92e60165dc7e}}

\pandocbounded{\includegraphics[keepaspectratio]{https://github.com/oplik0/rose-pine-typst/assets/25460763/9b842b68-4f57-4536-955f-7f510b77a579}}

\pandocbounded{\includegraphics[keepaspectratio]{https://github.com/oplik0/rose-pine-typst/assets/25460763/5b2d211d-d645-4b72-9b58-b73bccfae1ae}}

\subsection{Thanks to}\label{thanks-to}

\begin{itemize}
\tightlist
\item
  \href{https://github.com/oplik0}{oplik0}
\end{itemize}

\subsection{Contributing}\label{contributing}

\begin{quote}
Prefer using \href{https://github.com/rose-pine/build}{@rose-pine/build}
when possible
\end{quote}

Modify \texttt{\ src/template.typ\ } using Rosé Pine variables, then
build variants:

\begin{Shaded}
\begin{Highlighting}[]
\ExtensionTok{npx}\NormalTok{ @rose{-}pine/build@0.8.2 }\AttributeTok{{-}t}\NormalTok{ src/template.typ }\AttributeTok{{-}o}\NormalTok{ src/themes }\AttributeTok{{-}p}\NormalTok{ $ }\AttributeTok{{-}f}\NormalTok{ hex}
\end{Highlighting}
\end{Shaded}

\emph{Generated by
\href{https://github.com/rose-pine/build}{@rose-pine/build@0.8.2}}

To rebuild the syntax highlighting themes you similarly need to run

\begin{Shaded}
\begin{Highlighting}[]
\ExtensionTok{npx}\NormalTok{ @rose{-}pine/build@0.8.2 }\AttributeTok{{-}t}\NormalTok{ src/rose{-}pine{-}template.tmTheme }\AttributeTok{{-}o}\NormalTok{ src/themes }\AttributeTok{{-}p}\NormalTok{ $ }\AttributeTok{{-}f}\NormalTok{ hex }\AttributeTok{{-}{-}help}
\end{Highlighting}
\end{Shaded}

\subsubsection{How to add}\label{how-to-add}

Copy this into your project and use the import as \texttt{\ rose-pine\ }

\begin{verbatim}
#import "@preview/rose-pine:0.2.0"
\end{verbatim}

\includesvg[width=0.16667in,height=0.16667in]{/assets/icons/16-copy.svg}

Check the docs for
\href{https://typst.app/docs/reference/scripting/\#packages}{more
information on how to import packages} .

\subsubsection{About}\label{about}

\begin{description}
\tightlist
\item[Author s :]
oplik0 \& Rosé Pine
\item[License:]
MIT
\item[Current version:]
0.2.0
\item[Last updated:]
March 12, 2024
\item[First released:]
September 19, 2023
\item[Archive size:]
6.85 kB
\href{https://packages.typst.org/preview/rose-pine-0.2.0.tar.gz}{\pandocbounded{\includesvg[keepaspectratio]{/assets/icons/16-download.svg}}}
\item[Repository:]
\href{https://github.com/rose-pine/typst}{GitHub}
\end{description}

\subsubsection{Where to report issues?}\label{where-to-report-issues}

This package is a project of oplik0 and Rosé Pine . Report issues on
\href{https://github.com/rose-pine/typst}{their repository} . You can
also try to ask for help with this package on the
\href{https://forum.typst.app}{Forum} .

Please report this package to the Typst team using the
\href{https://typst.app/contact}{contact form} if you believe it is a
safety hazard or infringes upon your rights.

\phantomsection\label{versions}
\subsubsection{Version history}\label{version-history}

\begin{longtable}[]{@{}ll@{}}
\toprule\noalign{}
Version & Release Date \\
\midrule\noalign{}
\endhead
\bottomrule\noalign{}
\endlastfoot
0.2.0 & March 12, 2024 \\
\href{https://typst.app/universe/package/rose-pine/0.1.0/}{0.1.0} &
September 19, 2023 \\
\end{longtable}

Typst GmbH did not create this package and cannot guarantee correct
functionality of this package or compatibility with any version of the
Typst compiler or app.


\title{typst.app/universe/package/marginalia}

\phantomsection\label{banner}
\section{marginalia}\label{marginalia}

{ 0.1.1 }

Configurable margin-notes and matching wide blocks.

\phantomsection\label{readme}
\subsection{Setup}\label{setup}

Put something akin to the following at the start of your
\texttt{\ .typ\ } file:

\begin{Shaded}
\begin{Highlighting}[]
\NormalTok{\#import "@preview/marginalia:0.1.1" as marginalia: note, wideblock}
\NormalTok{\#let config = (}
\NormalTok{  // inner: ( far: 5mm, width: 15mm, sep: 5mm ),}
\NormalTok{  // outer: ( far: 5mm, width: 15mm, sep: 5mm ),}
\NormalTok{  // top: 2.5cm,}
\NormalTok{  // bottom: 2.5cm,}
\NormalTok{  // book: false,}
\NormalTok{  // clearance: 8pt,}
\NormalTok{  // flush{-}numbers: false,}
\NormalTok{  // numbering: /* numbering{-}function */,}
\NormalTok{)}
\NormalTok{\#marginalia.configure(..config)}
\NormalTok{\#set page(}
\NormalTok{  // setup margins:}
\NormalTok{  ..marginalia.page{-}setup(..config),}
\NormalTok{  /* other page setup */}
\NormalTok{)}
\end{Highlighting}
\end{Shaded}

If \texttt{\ book\ } is \texttt{\ false\ } , \texttt{\ inner\ } and
\texttt{\ outer\ } correspond to the left and right margins
respectively. If book is true, the margins swap sides on even and odd
pages. Notes are placed in the outside margin by default.

Where you can then customize \texttt{\ config\ } to your preferences.
Shown here (as comments) are the default values taken if the
corresponding keys are unset.
\href{https://github.com/nleanba/typst-marginalia/blob/v0.1.1/Marginalia.pdf}{Please
refer to the PDF documentation for more details on the configuration and
the provided commands.}

\subsection{Margin-Notes}\label{margin-notes}

Provided via the \texttt{\ \#note{[}...{]}\ } command.

\begin{itemize}
\tightlist
\item
  \texttt{\ \#note(reverse:\ true){[}...{]}\ } to put it on the inside
  margin.
\item
  \texttt{\ \#note(numbered:\ false){[}...{]}\ } to remove the marker.
\end{itemize}

Note: it is recommended to reset the counter for the markers regularly,
e.g. by putting \texttt{\ marginalia.notecounter.update(0)\ } into the
code for your header.

\subsection{Wide Blocks}\label{wide-blocks}

Provided via the \texttt{\ \#wideblock{[}...{]}\ } command.

\begin{itemize}
\tightlist
\item
  \texttt{\ \#wideblock(reverse:\ true){[}...{]}\ } to extend into the
  inside margin instead.
\item
  \texttt{\ \#wideblock(double:\ true){[}...{]}\ } to extend into both
  margins.
\end{itemize}

Note: \texttt{\ reverse\ } and \texttt{\ double\ } are mutually
exclusive.

Note: Wideblocks do not handle pagebreaks in \texttt{\ book:\ true\ }
documents well.

\subsection{Figures}\label{figures}

You can use figures as normal, also within wideblocks. To get captions
on the side, use

\begin{Shaded}
\begin{Highlighting}[]
\NormalTok{\#set figure(gap: 0pt)}
\NormalTok{\#set figure.caption(position: top)}
\NormalTok{\#show figure.caption.where(position: top): note.with(numbered: false, dy: 1em)}
\end{Highlighting}
\end{Shaded}

For small figures, the package also provides a \texttt{\ notefigure\ }
command which places the figure in the margin.

\begin{Shaded}
\begin{Highlighting}[]
\NormalTok{\#marginalia.notefigure(}
\NormalTok{  rect(width: 100\%),}
\NormalTok{  label: \textless{}aaa\textgreater{},}
\NormalTok{  caption: [A notefigure.],}
\NormalTok{)}
\end{Highlighting}
\end{Shaded}

\begin{center}\rule{0.5\linewidth}{0.5pt}\end{center}

\href{https://github.com/nleanba/typst-marginalia/blob/v0.1.1/Marginalia.pdf}{\pandocbounded{\includesvg[keepaspectratio]{https://raw.githubusercontent.com/nleanba/typst-marginalia/refs/tags/v0.1.1/preview.svg}}}

\subsubsection{How to add}\label{how-to-add}

Copy this into your project and use the import as
\texttt{\ marginalia\ }

\begin{verbatim}
#import "@preview/marginalia:0.1.1"
\end{verbatim}

\includesvg[width=0.16667in,height=0.16667in]{/assets/icons/16-copy.svg}

Check the docs for
\href{https://typst.app/docs/reference/scripting/\#packages}{more
information on how to import packages} .

\subsubsection{About}\label{about}

\begin{description}
\tightlist
\item[Author :]
\href{https://github.com/nleanba}{nleanba}
\item[License:]
Unlicense
\item[Current version:]
0.1.1
\item[Last updated:]
November 25, 2024
\item[First released:]
November 19, 2024
\item[Minimum Typst version:]
0.12.0
\item[Archive size:]
6.17 kB
\href{https://packages.typst.org/preview/marginalia-0.1.1.tar.gz}{\pandocbounded{\includesvg[keepaspectratio]{/assets/icons/16-download.svg}}}
\item[Repository:]
\href{https://github.com/nleanba/typst-marginalia}{GitHub}
\item[Categor ies :]
\begin{itemize}
\tightlist
\item[]
\item
  \pandocbounded{\includesvg[keepaspectratio]{/assets/icons/16-layout.svg}}
  \href{https://typst.app/universe/search/?category=layout}{Layout}
\item
  \pandocbounded{\includesvg[keepaspectratio]{/assets/icons/16-hammer.svg}}
  \href{https://typst.app/universe/search/?category=utility}{Utility}
\end{itemize}
\end{description}

\subsubsection{Where to report issues?}\label{where-to-report-issues}

This package is a project of nleanba . Report issues on
\href{https://github.com/nleanba/typst-marginalia}{their repository} .
You can also try to ask for help with this package on the
\href{https://forum.typst.app}{Forum} .

Please report this package to the Typst team using the
\href{https://typst.app/contact}{contact form} if you believe it is a
safety hazard or infringes upon your rights.

\phantomsection\label{versions}
\subsubsection{Version history}\label{version-history}

\begin{longtable}[]{@{}ll@{}}
\toprule\noalign{}
Version & Release Date \\
\midrule\noalign{}
\endhead
\bottomrule\noalign{}
\endlastfoot
0.1.1 & November 25, 2024 \\
\href{https://typst.app/universe/package/marginalia/0.1.0/}{0.1.0} &
November 19, 2024 \\
\end{longtable}

Typst GmbH did not create this package and cannot guarantee correct
functionality of this package or compatibility with any version of the
Typst compiler or app.


\title{typst.app/universe/package/bubble}

\phantomsection\label{banner}
\phantomsection\label{template-thumbnail}
\pandocbounded{\includegraphics[keepaspectratio]{https://packages.typst.org/preview/thumbnails/bubble-0.2.2-small.webp}}

\section{bubble}\label{bubble}

{ 0.2.2 }

Simple and colorful template for Typst

\href{/app?template=bubble&version=0.2.2}{Create project in app}

\phantomsection\label{readme}
Simple and colorful template for \href{https://typst.app/}{Typst} . This
template uses a main color (default is \texttt{\ \#E94845\ } ) applied
to list items, links, inline blocks, selected words and headings. Every
page is numbered and has the title of the document and the name of the
author at the top.

You can see an example PDF
\href{https://github.com/hzkonor/bubble-template/blob/main/main.pdf}{here}
.

\subsection{Usage}\label{usage}

You can use this template in the Typst web app by clicking “Start from
template� on the dashboard and searching for \texttt{\ bubble\ } .

Alternatively, you can use the CLI to kick this project off using the
command

\begin{Shaded}
\begin{Highlighting}[]
\ExtensionTok{typst}\NormalTok{ init @preview/bubble}
\end{Highlighting}
\end{Shaded}

Typst will create a new directory with all the files needed to get you
started.

\subsection{Configuration}\label{configuration}

This template exports the \texttt{\ bubble\ } function with the
following named arguments:

\begin{itemize}
\tightlist
\item
  \texttt{\ title\ } : Title of the document
\item
  \texttt{\ subtitle\ } : Subtitle of the document
\item
  \texttt{\ author\ } : Name of the author(s)
\item
  \texttt{\ affiliation\ } : It is supposed to be the name of your
  university for example
\item
  \texttt{\ year\ } : The year you’re in
\item
  \texttt{\ class\ } : For which class this document is
\item
  \texttt{\ other\ } : Array of other information \emph{default is none}
\item
  \texttt{\ date\ } : Date of the document, current date if none is set
  \emph{default is current date}
\item
  \texttt{\ logo\ } : Path of the logo displayed at the top right of the
  title page, must be set like an image :
  \texttt{\ image("path-to-img")\ } \emph{default is none}
\item
  \texttt{\ main-color\ } : Main color used in the document
  \emph{default is \texttt{\ \#E94645\ }}
\item
  \texttt{\ alpha\ } : Percentage of transparency for the bubbles on the
  title page \emph{default is 60\%}
\item
  \texttt{\ color-words\ } : An array of strings that you want to be
  colored automatically in the main-color (be careful to put a trailing
  comma if you have only one string in the array as noted
  \href{https://typst.app/docs/reference/foundations/array/}{here} )
  \emph{default is an empty array}
\end{itemize}

This template also exports these functions :

\begin{itemize}
\tightlist
\item
  \texttt{\ blockquote\ } : Function that highlights quotes with a grey
  bar at the left
\item
  \texttt{\ primary-color\ } : to use your main color
\item
  \texttt{\ secondary-color\ } : to use your secondary color (which is
  your main color with the alpha transparency set)
\end{itemize}

If you want to change an existing project to use this template, you can
add a show rule like this at the top of your file:

\begin{Shaded}
\begin{Highlighting}[]
\NormalTok{\#import "@preview/bubble:0.2.2": *}

\NormalTok{\#show: bubble.with(}
\NormalTok{  title: "Bubble template",}
\NormalTok{  subtitle: "Simple and colorful template",}
\NormalTok{  author: "hzkonor",}
\NormalTok{  affiliation: "University",}
\NormalTok{  date: datetime.today().display(),}
\NormalTok{  year: "Year",}
\NormalTok{  class: "Class",}
\NormalTok{  other: ("Made with Typst", "https://typst.com"),}
\NormalTok{  logo: image("logo.png"),}
\NormalTok{  color{-}words: ("important",)}
\NormalTok{) }

\NormalTok{// Your content goes here}
\end{Highlighting}
\end{Shaded}

\href{/app?template=bubble&version=0.2.2}{Create project in app}

\subsubsection{How to use}\label{how-to-use}

Click the button above to create a new project using this template in
the Typst app.

You can also use the Typst CLI to start a new project on your computer
using this command:

\begin{verbatim}
typst init @preview/bubble:0.2.2
\end{verbatim}

\includesvg[width=0.16667in,height=0.16667in]{/assets/icons/16-copy.svg}

\subsubsection{About}\label{about}

\begin{description}
\tightlist
\item[Author :]
\href{https://github.com/hzkonor}{Conor}
\item[License:]
MIT-0
\item[Current version:]
0.2.2
\item[Last updated:]
October 21, 2024
\item[First released:]
April 16, 2024
\item[Archive size:]
36.2 kB
\href{https://packages.typst.org/preview/bubble-0.2.2.tar.gz}{\pandocbounded{\includesvg[keepaspectratio]{/assets/icons/16-download.svg}}}
\item[Repository:]
\href{https://github.com/hzkonor/bubble-template}{GitHub}
\item[Categor ies :]
\begin{itemize}
\tightlist
\item[]
\item
  \pandocbounded{\includesvg[keepaspectratio]{/assets/icons/16-atom.svg}}
  \href{https://typst.app/universe/search/?category=paper}{Paper}
\item
  \pandocbounded{\includesvg[keepaspectratio]{/assets/icons/16-speak.svg}}
  \href{https://typst.app/universe/search/?category=report}{Report}
\end{itemize}
\end{description}

\subsubsection{Where to report issues?}\label{where-to-report-issues}

This template is a project of Conor . Report issues on
\href{https://github.com/hzkonor/bubble-template}{their repository} .
You can also try to ask for help with this template on the
\href{https://forum.typst.app}{Forum} .

Please report this template to the Typst team using the
\href{https://typst.app/contact}{contact form} if you believe it is a
safety hazard or infringes upon your rights.

\phantomsection\label{versions}
\subsubsection{Version history}\label{version-history}

\begin{longtable}[]{@{}ll@{}}
\toprule\noalign{}
Version & Release Date \\
\midrule\noalign{}
\endhead
\bottomrule\noalign{}
\endlastfoot
0.2.2 & October 21, 2024 \\
\href{https://typst.app/universe/package/bubble/0.2.1/}{0.2.1} & August
2, 2024 \\
\href{https://typst.app/universe/package/bubble/0.2.0/}{0.2.0} & July
23, 2024 \\
\href{https://typst.app/universe/package/bubble/0.1.0/}{0.1.0} & April
16, 2024 \\
\end{longtable}

Typst GmbH did not create this template and cannot guarantee correct
functionality of this template or compatibility with any version of the
Typst compiler or app.


\title{typst.app/universe/package/chromo}

\phantomsection\label{banner}
\section{chromo}\label{chromo}

{ 0.1.0 }

Generate printer tests (likely CMYK) in typst.

\phantomsection\label{readme}
Generate printer tests directly in Typst. For now, only generates with
CMYK colors (as it is by far the most used).

I personally place one of these test on all my exam papers to ensure the
printer’s quality over time.

\subsection{Documentation}\label{documentation}

To import any of the functions needed, you may want to use the following
line:

\begin{Shaded}
\begin{Highlighting}[]
\NormalTok{\#import "@preview/chromo:0.1.0": square{-}printer{-}test, gradient{-}printer{-}test, circular{-}printer{-}test, crosshair{-}printer{-}test}
\end{Highlighting}
\end{Shaded}

\subsubsection{Square test}\label{square-test}

\begin{Shaded}
\begin{Highlighting}[]
\NormalTok{\#square{-}printer{-}test()}
\end{Highlighting}
\end{Shaded}

\subsubsection{Gradient test}\label{gradient-test}

\begin{Shaded}
\begin{Highlighting}[]
\NormalTok{\#gradient{-}printer{-}test()}
\end{Highlighting}
\end{Shaded}

\subsubsection{Circular test}\label{circular-test}

\begin{Shaded}
\begin{Highlighting}[]
\NormalTok{\#circular{-}printer{-}test()}
\end{Highlighting}
\end{Shaded}

\subsubsection{Crosshair test}\label{crosshair-test}

\begin{Shaded}
\begin{Highlighting}[]
\NormalTok{\#crosshair{-}printer{-}test()}
\end{Highlighting}
\end{Shaded}

\subsection{Contributors}\label{contributors}

\begin{itemize}
\tightlist
\item
  \href{https://github.com/julien-cpsn}{Julien-cpsn}
\end{itemize}

\subsubsection{How to add}\label{how-to-add}

Copy this into your project and use the import as \texttt{\ chromo\ }

\begin{verbatim}
#import "@preview/chromo:0.1.0"
\end{verbatim}

\includesvg[width=0.16667in,height=0.16667in]{/assets/icons/16-copy.svg}

Check the docs for
\href{https://typst.app/docs/reference/scripting/\#packages}{more
information on how to import packages} .

\subsubsection{About}\label{about}

\begin{description}
\tightlist
\item[Author :]
Julien Caposiena
\item[License:]
MIT
\item[Current version:]
0.1.0
\item[Last updated:]
January 29, 2024
\item[First released:]
January 29, 2024
\item[Archive size:]
2.00 kB
\href{https://packages.typst.org/preview/chromo-0.1.0.tar.gz}{\pandocbounded{\includesvg[keepaspectratio]{/assets/icons/16-download.svg}}}
\item[Repository:]
\href{https://github.com/julien-cpsn/typst-chromo}{GitHub}
\end{description}

\subsubsection{Where to report issues?}\label{where-to-report-issues}

This package is a project of Julien Caposiena . Report issues on
\href{https://github.com/julien-cpsn/typst-chromo}{their repository} .
You can also try to ask for help with this package on the
\href{https://forum.typst.app}{Forum} .

Please report this package to the Typst team using the
\href{https://typst.app/contact}{contact form} if you believe it is a
safety hazard or infringes upon your rights.

\phantomsection\label{versions}
\subsubsection{Version history}\label{version-history}

\begin{longtable}[]{@{}ll@{}}
\toprule\noalign{}
Version & Release Date \\
\midrule\noalign{}
\endhead
\bottomrule\noalign{}
\endlastfoot
0.1.0 & January 29, 2024 \\
\end{longtable}

Typst GmbH did not create this package and cannot guarantee correct
functionality of this package or compatibility with any version of the
Typst compiler or app.


\title{typst.app/universe/package/wavy}

\phantomsection\label{banner}
\section{wavy}\label{wavy}

{ 0.1.1 }

Draw digital timing diagram in Typst using Wavedrom.

\phantomsection\label{readme}
Draw digital timing diagram in Typst using
\href{https://wavedrom.com/}{Wavedrom} .

\pandocbounded{\includesvg[keepaspectratio]{https://github.com/typst/packages/raw/main/packages/preview/wavy/0.1.1/wavy.svg}}

\begin{Shaded}
\begin{Highlighting}[]
\NormalTok{\#import "@preview/wavy:0.1.1"}

\NormalTok{\#set page(height: auto, width: auto, fill: black, margin: 2em)}
\NormalTok{\#set text(fill: white)}

\NormalTok{\#show raw.where(lang: "wavy"): it =\textgreater{} wavy.render(it.text)}

\NormalTok{= Wavy}

\NormalTok{Typst, now with waves.}

\NormalTok{\textasciigrave{}\textasciigrave{}\textasciigrave{}wavy}
\NormalTok{\{}
\NormalTok{  signal:}
\NormalTok{  [}
\NormalTok{    \{name:\textquotesingle{}clk\textquotesingle{},wave:\textquotesingle{}p......\textquotesingle{}\},}
\NormalTok{    \{name:\textquotesingle{}bus\textquotesingle{},wave:\textquotesingle{}x.34.5x\textquotesingle{},data:\textquotesingle{}head body tail\textquotesingle{}\},}
\NormalTok{    \{name:\textquotesingle{}wire\textquotesingle{},wave:\textquotesingle{}0.1..0.\textquotesingle{}\}}
\NormalTok{  ]}
\NormalTok{\}}
\NormalTok{\textasciigrave{}\textasciigrave{}\textasciigrave{}}

\NormalTok{\textasciigrave{}\textasciigrave{}\textasciigrave{}js}
\NormalTok{\{}
\NormalTok{  signal:}
\NormalTok{  [}
\NormalTok{    \{name:\textquotesingle{}clk\textquotesingle{},wave:\textquotesingle{}p......\textquotesingle{}\},}
\NormalTok{    \{name:\textquotesingle{}bus\textquotesingle{},wave:\textquotesingle{}x.34.5x\textquotesingle{},data:\textquotesingle{}head body tail\textquotesingle{}\},}
\NormalTok{    \{name:\textquotesingle{}wire\textquotesingle{},wave:\textquotesingle{}0.1..0.\textquotesingle{}\}}
\NormalTok{  ]}
\NormalTok{\}}
\NormalTok{\textasciigrave{}\textasciigrave{}\textasciigrave{}}
\end{Highlighting}
\end{Shaded}

\subsection{Documentation}\label{documentation}

\subsubsection{\texorpdfstring{\texttt{\ render\ }}{ render }}\label{render}

Render a wavedrom json5 string to an image

\paragraph{Arguments}\label{arguments}

\begin{itemize}
\tightlist
\item
  \texttt{\ src\ } : \texttt{\ str\ } - wavedrom json5 string
\item
  All other arguments are passed to \texttt{\ image.decode\ } so you can
  customize the image size
\end{itemize}

\paragraph{Returns}\label{returns}

The image, of type \texttt{\ content\ }

\subsubsection{How to add}\label{how-to-add}

Copy this into your project and use the import as \texttt{\ wavy\ }

\begin{verbatim}
#import "@preview/wavy:0.1.1"
\end{verbatim}

\includesvg[width=0.16667in,height=0.16667in]{/assets/icons/16-copy.svg}

Check the docs for
\href{https://typst.app/docs/reference/scripting/\#packages}{more
information on how to import packages} .

\subsubsection{About}\label{about}

\begin{description}
\tightlist
\item[Author :]
Wenzhuo Liu
\item[License:]
MIT
\item[Current version:]
0.1.1
\item[Last updated:]
December 3, 2023
\item[First released:]
November 8, 2023
\item[Archive size:]
42.9 kB
\href{https://packages.typst.org/preview/wavy-0.1.1.tar.gz}{\pandocbounded{\includesvg[keepaspectratio]{/assets/icons/16-download.svg}}}
\item[Repository:]
\href{https://github.com/Enter-tainer/wavy}{GitHub}
\end{description}

\subsubsection{Where to report issues?}\label{where-to-report-issues}

This package is a project of Wenzhuo Liu . Report issues on
\href{https://github.com/Enter-tainer/wavy}{their repository} . You can
also try to ask for help with this package on the
\href{https://forum.typst.app}{Forum} .

Please report this package to the Typst team using the
\href{https://typst.app/contact}{contact form} if you believe it is a
safety hazard or infringes upon your rights.

\phantomsection\label{versions}
\subsubsection{Version history}\label{version-history}

\begin{longtable}[]{@{}ll@{}}
\toprule\noalign{}
Version & Release Date \\
\midrule\noalign{}
\endhead
\bottomrule\noalign{}
\endlastfoot
0.1.1 & December 3, 2023 \\
\href{https://typst.app/universe/package/wavy/0.1.0/}{0.1.0} & November
8, 2023 \\
\end{longtable}

Typst GmbH did not create this package and cannot guarantee correct
functionality of this package or compatibility with any version of the
Typst compiler or app.


\title{typst.app/universe/package/numbly}

\phantomsection\label{banner}
\section{numbly}\label{numbly}

{ 0.1.0 }

A package that helps you to specify different numbering formats for
different levels of headings.

\phantomsection\label{readme}
A package that helps you to specify different numbering formats for
different levels of headings.

Suppose you want to specify the following numbering format for your
document:

\begin{itemize}
\tightlist
\item
  Appendix A. Guide

  \begin{itemize}
  \tightlist
  \item
    A.1. Installation

    \begin{itemize}
    \tightlist
    \item
      Step 1. Download
    \item
      Step 2. Install
    \end{itemize}
  \item
    A.2. Usage
  \end{itemize}
\end{itemize}

You might use \texttt{\ if\ } to achieve this:

\begin{Shaded}
\begin{Highlighting}[]
\NormalTok{\#set heading(numbering: (..nums) =\textgreater{} \{}
\NormalTok{  nums = nums.pos()}
\NormalTok{  if nums.len() == 1 \{}
\NormalTok{    return "Appendix " + numbering("A.", ..nums)}
\NormalTok{  \} else if nums.len() == 2 \{}
\NormalTok{    return numbering("A.1.", ..nums)}
\NormalTok{  \} else \{}
\NormalTok{    return "Step " + numbering("1.", nums.last())}
\NormalTok{  \}}
\NormalTok{\})}

\NormalTok{= Guide}
\NormalTok{== Installation}
\NormalTok{=== Download}
\NormalTok{=== Install}
\NormalTok{== Usage}
\end{Highlighting}
\end{Shaded}

But with \texttt{\ numbly\ } , you can do this more easily:

\begin{Shaded}
\begin{Highlighting}[]
\NormalTok{\#import "@preview/numbly:0.1.0": numbly}
\NormalTok{\#set heading(numbering: numbly(}
\NormalTok{  "Appendix \{1:A\}.", // use \{level:format\} to specify the format}
\NormalTok{  "\{1:A\}.\{2\}.", // if format is not specified, arabic numbers will be used}
\NormalTok{  "Step \{3\}.", // here, we only want the 3rd level}
\NormalTok{))}
\end{Highlighting}
\end{Shaded}

\subsubsection{How to add}\label{how-to-add}

Copy this into your project and use the import as \texttt{\ numbly\ }

\begin{verbatim}
#import "@preview/numbly:0.1.0"
\end{verbatim}

\includesvg[width=0.16667in,height=0.16667in]{/assets/icons/16-copy.svg}

Check the docs for
\href{https://typst.app/docs/reference/scripting/\#packages}{more
information on how to import packages} .

\subsubsection{About}\label{about}

\begin{description}
\tightlist
\item[Author :]
\href{https://github.com/flaribbit}{flaribbit}
\item[License:]
MIT
\item[Current version:]
0.1.0
\item[Last updated:]
July 1, 2024
\item[First released:]
July 1, 2024
\item[Archive size:]
1.75 kB
\href{https://packages.typst.org/preview/numbly-0.1.0.tar.gz}{\pandocbounded{\includesvg[keepaspectratio]{/assets/icons/16-download.svg}}}
\item[Repository:]
\href{https://github.com/flaribbit/numbly}{GitHub}
\item[Categor y :]
\begin{itemize}
\tightlist
\item[]
\item
  \pandocbounded{\includesvg[keepaspectratio]{/assets/icons/16-hammer.svg}}
  \href{https://typst.app/universe/search/?category=utility}{Utility}
\end{itemize}
\end{description}

\subsubsection{Where to report issues?}\label{where-to-report-issues}

This package is a project of flaribbit . Report issues on
\href{https://github.com/flaribbit/numbly}{their repository} . You can
also try to ask for help with this package on the
\href{https://forum.typst.app}{Forum} .

Please report this package to the Typst team using the
\href{https://typst.app/contact}{contact form} if you believe it is a
safety hazard or infringes upon your rights.

\phantomsection\label{versions}
\subsubsection{Version history}\label{version-history}

\begin{longtable}[]{@{}ll@{}}
\toprule\noalign{}
Version & Release Date \\
\midrule\noalign{}
\endhead
\bottomrule\noalign{}
\endlastfoot
0.1.0 & July 1, 2024 \\
\end{longtable}

Typst GmbH did not create this package and cannot guarantee correct
functionality of this package or compatibility with any version of the
Typst compiler or app.


\title{typst.app/universe/package/rubby}

\phantomsection\label{banner}
\section{rubby}\label{rubby}

{ 0.10.1 }

Add ruby (furigana) next to base text.

\phantomsection\label{readme}
\subsection{Usage}\label{usage}

\begin{Shaded}
\begin{Highlighting}[]
\NormalTok{\#import "@preview/rubby:0.10.1": get{-}ruby}

\NormalTok{\#let ruby = get{-}ruby(}
\NormalTok{  size: 0.5em,         // Ruby font size}
\NormalTok{  dy: 0pt,             // Vertical offset of the ruby}
\NormalTok{  pos: top,            // Ruby position (top or bottom)}
\NormalTok{  alignment: "center", // Ruby alignment ("center", "start", "between", "around")}
\NormalTok{  delimiter: "|",      // The delimiter between words}
\NormalTok{  auto{-}spacing: true,  // Automatically add necessary space around words}
\NormalTok{)}

\NormalTok{// Ruby goes first, base text {-} second.}
\NormalTok{\#ruby[ふりがな][振り仮名]}

\NormalTok{Treat each kanji as a separate word:}
\NormalTok{\#ruby[とう|きょう|こう|ぎょう|だい|がく][東|京|工|業|大|学]}
\end{Highlighting}
\end{Shaded}

If you don’t want automatically wrap text with delimiter:

\begin{Shaded}
\begin{Highlighting}[]
\NormalTok{\#let ruby = get{-}ruby(auto{-}spacing: false)}
\end{Highlighting}
\end{Shaded}

See also \url{https://github.com/rinmyo/ruby-typ/blob/main/manual.pdf}
and \texttt{\ example.typ\ } .

\subsection{Notes}\label{notes}

Original project is at \url{https://github.com/rinmyo/ruby-typ} which
itself is based on
\href{https://zenn.dev/saito_atsushi/articles/ff9490458570e1}{the post}
of 齊è---¤æ•¦å¿--- (Saito Atsushi). This project is a modified version
of
\href{https://github.com/rinmyo/ruby-typ/commit/23ca86180757cf70f2b9f5851abb5ea5a3b4c6a1}{this
commit} .

\texttt{\ auto-spacing\ } adds missing delimiter around the
\texttt{\ content\ } / \texttt{\ string\ } which then adds space around
base text if ruby is wider than the base text.

Problems appear only if ruby is wider than its base text and
\texttt{\ auto-spacing\ } is not set to \texttt{\ true\ } (default is
\texttt{\ true\ } ).

You can always use a one-letter function (variable) name to shorten the
function call length (if you have to use it a lot), e.g.,
\texttt{\ \#let\ r\ =\ get-ruby()\ } (or \texttt{\ f\ } â€'' short for
furigana). But be careful as there are functions with names
\texttt{\ v\ } and \texttt{\ h\ } and there could be a new built-in
function with a name \texttt{\ r\ } or \texttt{\ f\ } which may break
your document (Typst right now is in beta, so breaking changes are
possible).

Although you can open issues or send PRs, I won’t be able to always
reply quickly (sometimes I’m very busy).

\subsection{Development}\label{development}

This repository should exist as a \texttt{\ @local\ } package with the
version from the \texttt{\ typst.toml\ } .

Here is a short description of the development process:

\begin{enumerate}
\tightlist
\item
  run \texttt{\ git\ checkout\ dev\ \&\&\ git\ pull\ } ;
\item
  make changes;
\item
  test changes, if not done or something isn’t working then go to step
  1;
\item
  when finished, run
  \texttt{\ just\ change-version\ \textless{}new\ semantic\ version\textgreater{}\ }
  ;
\item
  document changes in the \texttt{\ CHANGELOG.md\ } ;
\item
  commit all changes (only locally);
\item
  create a \texttt{\ @local\ } Typst package with the new version and
  test it;
\item
  if everything is working then run \texttt{\ git\ push\ } ;
\item
  realize that you’ve missed something and fix it (then push changes
  again);
\item
  run \texttt{\ git\ checkout\ master\ \&\&\ git\ merge\ dev\ } to sync
  \texttt{\ master\ } to \texttt{\ dev\ } ;
\item
  run \texttt{\ just\ create-release\ } .
\end{enumerate}

\subsection{Publishing a Typst
package}\label{publishing-a-typst-package}

\begin{enumerate}
\tightlist
\item
  To make a new package version for merging into
  \texttt{\ typst/packages\ } repository run
  \texttt{\ just\ mark-PR-version\ } ;
\item
  copy newly created directory (with a version name) and place it in the
  appropriate place in your fork of the \texttt{\ typst/packages\ }
  repository;
\item
  run
  \texttt{\ git\ fetch\ upstream\ \&\&\ git\ merge\ upstream\ main\ } to
  sync fork with \texttt{\ typst/packages\ } ;
\item
  go to a new branch with
  \texttt{\ git\ checkout\ -b\ \textless{}package-version\textgreater{}\ }
  ;
\item
  commit newly added directory with commit message:
  \texttt{\ package:version\ } ;
\item
  run \texttt{\ gh\ pr\ create\ } and follow further CLI instructions.
\end{enumerate}

\subsection{Changelog}\label{changelog}

You can view the change log in the \texttt{\ CHANGELOG.md\ } file in the
root of the project.

\subsection{License}\label{license}

This Typst package is licensed under AGPL v3.0. You can view the license
in the LICENSE file in the root of the project or at
\url{https://www.gnu.org/licenses/agpl-3.0.txt} . There is also a NOTICE
file for 3rd party copyright notices.

Copyright © 2023 Andrew Voynov

\subsubsection{How to add}\label{how-to-add}

Copy this into your project and use the import as \texttt{\ rubby\ }

\begin{verbatim}
#import "@preview/rubby:0.10.1"
\end{verbatim}

\includesvg[width=0.16667in,height=0.16667in]{/assets/icons/16-copy.svg}

Check the docs for
\href{https://typst.app/docs/reference/scripting/\#packages}{more
information on how to import packages} .

\subsubsection{About}\label{about}

\begin{description}
\tightlist
\item[Author :]
Andrew Voynov
\item[License:]
AGPL-3.0-only
\item[Current version:]
0.10.1
\item[Last updated:]
December 3, 2023
\item[First released:]
July 3, 2023
\item[Minimum Typst version:]
0.8.0
\item[Archive size:]
16.0 kB
\href{https://packages.typst.org/preview/rubby-0.10.1.tar.gz}{\pandocbounded{\includesvg[keepaspectratio]{/assets/icons/16-download.svg}}}
\item[Repository:]
\href{https://github.com/Andrew15-5/rubby}{GitHub}
\end{description}

\subsubsection{Where to report issues?}\label{where-to-report-issues}

This package is a project of Andrew Voynov . Report issues on
\href{https://github.com/Andrew15-5/rubby}{their repository} . You can
also try to ask for help with this package on the
\href{https://forum.typst.app}{Forum} .

Please report this package to the Typst team using the
\href{https://typst.app/contact}{contact form} if you believe it is a
safety hazard or infringes upon your rights.

\phantomsection\label{versions}
\subsubsection{Version history}\label{version-history}

\begin{longtable}[]{@{}ll@{}}
\toprule\noalign{}
Version & Release Date \\
\midrule\noalign{}
\endhead
\bottomrule\noalign{}
\endlastfoot
\href{https://typst.app/universe/package/rubby/0.9.2/}{0.9.2} &
September 15, 2023 \\
\href{https://typst.app/universe/package/rubby/0.8.0/}{0.8.0} & July 3,
2023 \\
0.10.1 & December 3, 2023 \\
\href{https://typst.app/universe/package/rubby/0.10.0/}{0.10.0} &
November 25, 2023 \\
\end{longtable}

Typst GmbH did not create this package and cannot guarantee correct
functionality of this package or compatibility with any version of the
Typst compiler or app.


\title{typst.app/universe/package/pigmentpedia}

\phantomsection\label{banner}
\section{pigmentpedia}\label{pigmentpedia}

{ 0.1.0 }

An extended color library for Typst.

\phantomsection\label{readme}
An extended color library for Typst. Most of these colors are commonly
used on the web.

\subsection{Usage}\label{usage}

Add the package with the following code (remember to add the asterisk
\texttt{\ :\ *\ } at the end) and pick a color.

\begin{Shaded}
\begin{Highlighting}[]
\NormalTok{\#include "@preview/pigmentpedia:0.1.0": *}

\NormalTok{\#set text(firebrick)}

\NormalTok{This text has a firebrick color.}
\end{Highlighting}
\end{Shaded}

There are over 100 different colors currently available and more will be
added.

\subsubsection{How to add}\label{how-to-add}

Copy this into your project and use the import as
\texttt{\ pigmentpedia\ }

\begin{verbatim}
#import "@preview/pigmentpedia:0.1.0"
\end{verbatim}

\includesvg[width=0.16667in,height=0.16667in]{/assets/icons/16-copy.svg}

Check the docs for
\href{https://typst.app/docs/reference/scripting/\#packages}{more
information on how to import packages} .

\subsubsection{About}\label{about}

\begin{description}
\tightlist
\item[Author :]
\href{https://github.com/neuralpain}{neuralpain}
\item[License:]
MIT
\item[Current version:]
0.1.0
\item[Last updated:]
November 29, 2024
\item[First released:]
November 29, 2024
\item[Archive size:]
2.75 kB
\href{https://packages.typst.org/preview/pigmentpedia-0.1.0.tar.gz}{\pandocbounded{\includesvg[keepaspectratio]{/assets/icons/16-download.svg}}}
\item[Repository:]
\href{https://github.com/neuralpain/pigmentpedia}{GitHub}
\item[Categor ies :]
\begin{itemize}
\tightlist
\item[]
\item
  \pandocbounded{\includesvg[keepaspectratio]{/assets/icons/16-hammer.svg}}
  \href{https://typst.app/universe/search/?category=utility}{Utility}
\item
  \pandocbounded{\includesvg[keepaspectratio]{/assets/icons/16-text.svg}}
  \href{https://typst.app/universe/search/?category=text}{Text}
\end{itemize}
\end{description}

\subsubsection{Where to report issues?}\label{where-to-report-issues}

This package is a project of neuralpain . Report issues on
\href{https://github.com/neuralpain/pigmentpedia}{their repository} .
You can also try to ask for help with this package on the
\href{https://forum.typst.app}{Forum} .

Please report this package to the Typst team using the
\href{https://typst.app/contact}{contact form} if you believe it is a
safety hazard or infringes upon your rights.

\phantomsection\label{versions}
\subsubsection{Version history}\label{version-history}

\begin{longtable}[]{@{}ll@{}}
\toprule\noalign{}
Version & Release Date \\
\midrule\noalign{}
\endhead
\bottomrule\noalign{}
\endlastfoot
0.1.0 & November 29, 2024 \\
\end{longtable}

Typst GmbH did not create this package and cannot guarantee correct
functionality of this package or compatibility with any version of the
Typst compiler or app.


\title{typst.app/universe/package/iconic-salmon-fa}

\phantomsection\label{banner}
\section{iconic-salmon-fa}\label{iconic-salmon-fa}

{ 1.0.0 }

A Typst library for Social Media references with icons based on Font
Awesome.

\phantomsection\label{readme}
The \texttt{\ iconic-salmon-fa\ } package is designed to help you create
your curriculum vitae (CV). It allows you to easily reference your
social media profiles with the typical icon of the service plus a link
to your profile.\\
The package name is a combination of the acronym \emph{SociAL Media
icONs} and the word \emph{iconic} because all these icons have an iconic
design (and iconic also contains the word \emph{icon} ).

\subsection{Features}\label{features}

\begin{itemize}
\tightlist
\item
  Support for popular social media, developer and career platforms
\item
  Uniform design for all entries
\item
  Based on the Internet’s icon library
  \href{https://fontawesome.com/}{Font Awesome}
\item
  Easy to use
\item
  Allows the customization of the look (extra args are passed to
  \href{https://typst.app/docs/reference/text/text/}{\texttt{\ text\ }}
  )
\end{itemize}

\subsection{Fonts Installation}\label{fonts-installation}

\subsubsection{Linux}\label{linux}

\begin{enumerate}
\tightlist
\item
  \href{https://fontawesome.com/download}{Download Font Awesome for
  Desktop}
\item
  Unzip the file
\item
  Switch into the \texttt{\ otfs\ } folder within the unzipped folder
\item
  Run \texttt{\ mkdir\ -p\ /usr/share/fonts/truetype/\ }
\item
  Run
  \texttt{\ install\ -m644\ \textquotesingle{}Font\ Awesome\ 6\ Brands-Regular-400.otf\textquotesingle{}\ /usr/share/fonts/truetype/\ }
\item
  Unfortunately not all brands are included in the brands font file, so
  you must also run
  \texttt{\ install\ -m644\ \textquotesingle{}Font\ Awesome\ 6\ Free-Regular-400.otf\textquotesingle{}\ /usr/share/fonts/truetype/\ }
\end{enumerate}

\subsection{Usage}\label{usage}

\subsubsection{Using Typst’s package
manager}\label{using-typstuxe2s-package-manager}

You can install the library using the
\href{https://github.com/typst/packages}{typst packages} :

\begin{Shaded}
\begin{Highlighting}[]
\NormalTok{\#import "@preview/iconic{-}salmon{-}fa:1.0.0": *}
\end{Highlighting}
\end{Shaded}

\subsubsection{Install manually}\label{install-manually}

Put the \texttt{\ iconic-salmon-fa.typ\ } file in your project directory
and import it:

\begin{Shaded}
\begin{Highlighting}[]
\NormalTok{\#import "iconic{-}salmon{-}fa.typ": *}
\end{Highlighting}
\end{Shaded}

\subsubsection{Minimal Example}\label{minimal-example}

\begin{Shaded}
\begin{Highlighting}[]
\NormalTok{// \#import "@preview/iconic{-}salmon{-}fa:1.0.0": github{-}info, gitlab{-}info}
\NormalTok{\#import "iconic{-}salmon{-}fa.typ": github{-}info, gitlab{-}info}

\NormalTok{This project was created by \#github{-}info("Bi0T1N"). You can also find me on \#gitlab{-}info("GitLab", rgb("\#811052"), url: "https://gitlab.com/Bi0T1N").}
\end{Highlighting}
\end{Shaded}

\subsubsection{Examples}\label{examples}

See the
\href{https://github.com/typst/packages/raw/main/packages/preview/iconic-salmon-fa/1.0.0/examples/examples.typ}{\texttt{\ examples.typ\ }}
file for a complete example. The
\href{https://github.com/typst/packages/raw/main/packages/preview/iconic-salmon-fa/1.0.0/examples/}{generated
PDF files} are also available for preview.

\subsection{Troubleshooting}\label{troubleshooting}

\subsubsection{Icons are not displayed
correctly}\label{icons-are-not-displayed-correctly}

Make sure that you have installed the required Font Awesome
ligature-based font files.

\subsection{Contribution}\label{contribution}

Feel free to open an issue or a pull request if you find any problems or
have any suggestions.

\subsection{License}\label{license}

This library is licensed under the MIT license. Feel free to use it in
your project.

\subsection{Trademark Disclaimer}\label{trademark-disclaimer}

Product names, logos, brands and other trademarks referred to in this
project are the property of their respective trademark holders.\\
These trademark holders are not affiliated with this Typst library, nor
are the authors officially endorsed by them, nor do the authors claim
ownership of these trademarks.

\subsubsection{How to add}\label{how-to-add}

Copy this into your project and use the import as
\texttt{\ iconic-salmon-fa\ }

\begin{verbatim}
#import "@preview/iconic-salmon-fa:1.0.0"
\end{verbatim}

\includesvg[width=0.16667in,height=0.16667in]{/assets/icons/16-copy.svg}

Check the docs for
\href{https://typst.app/docs/reference/scripting/\#packages}{more
information on how to import packages} .

\subsubsection{About}\label{about}

\begin{description}
\tightlist
\item[Author :]
Nico Neumann (Bi0T1N)
\item[License:]
MIT
\item[Current version:]
1.0.0
\item[Last updated:]
May 16, 2024
\item[First released:]
May 16, 2024
\item[Archive size:]
3.32 kB
\href{https://packages.typst.org/preview/iconic-salmon-fa-1.0.0.tar.gz}{\pandocbounded{\includesvg[keepaspectratio]{/assets/icons/16-download.svg}}}
\item[Repository:]
\href{https://github.com/Bi0T1N/typst-iconic-salmon-fa}{GitHub}
\item[Categor y :]
\begin{itemize}
\tightlist
\item[]
\item
  \pandocbounded{\includesvg[keepaspectratio]{/assets/icons/16-package.svg}}
  \href{https://typst.app/universe/search/?category=components}{Components}
\end{itemize}
\end{description}

\subsubsection{Where to report issues?}\label{where-to-report-issues}

This package is a project of Nico Neumann (Bi0T1N) . Report issues on
\href{https://github.com/Bi0T1N/typst-iconic-salmon-fa}{their
repository} . You can also try to ask for help with this package on the
\href{https://forum.typst.app}{Forum} .

Please report this package to the Typst team using the
\href{https://typst.app/contact}{contact form} if you believe it is a
safety hazard or infringes upon your rights.

\phantomsection\label{versions}
\subsubsection{Version history}\label{version-history}

\begin{longtable}[]{@{}ll@{}}
\toprule\noalign{}
Version & Release Date \\
\midrule\noalign{}
\endhead
\bottomrule\noalign{}
\endlastfoot
1.0.0 & May 16, 2024 \\
\end{longtable}

Typst GmbH did not create this package and cannot guarantee correct
functionality of this package or compatibility with any version of the
Typst compiler or app.


\title{typst.app/universe/package/bytefield}

\phantomsection\label{banner}
\section{bytefield}\label{bytefield}

{ 0.0.6 }

A package to create network protocol headers, memory map, register
definitions and more.

\phantomsection\label{readme}
A simple way to create network protocol headers, memory maps, register
definitions and more in typst.

âš~ï¸? Warning. As this package is still in an early stage, things might
break with the next version.

ℹ� If you find a bug or a feature which is missing, please open an
issue and/or send an PR.

\subsection{Example}\label{example}

\pandocbounded{\includegraphics[keepaspectratio]{https://github.com/typst/packages/raw/main/packages/preview/bytefield/0.0.6/docs/bytefield_example.png}}

\begin{Shaded}
\begin{Highlighting}[]
\NormalTok{\#import "@preview/bytefield:0.0.6": *}

\NormalTok{\#bytefield(}
\NormalTok{// Config the header}
\NormalTok{bitheader(}
\NormalTok{"bytes",}
\NormalTok{// adds every multiple of 8 to the header.}
\NormalTok{0, [start], // number with label}
\NormalTok{5,}
\NormalTok{// number without label}
\NormalTok{12, [\#text(14pt, fill: red, "test")],}
\NormalTok{23, [end\_test],}
\NormalTok{24, [start\_break],}
\NormalTok{36, [Fix], // will not be shown}
\NormalTok{angle: {-}50deg, // angle (default: {-}60deg)}
\NormalTok{text{-}size: 8pt, // length (default: global header\_font\_size or 9pt)}
\NormalTok{),}
\NormalTok{// Add data fields (bit, bits, byte, bytes) and notes}
\NormalTok{// A note always aligns on the same row as the start of the next data field.}
\NormalTok{note(left)[\#text(16pt, fill: blue, font: "Consolas", "Testing")],}
\NormalTok{bytes(3,fill: red.lighten(30\%))[Test],}
\NormalTok{note(right)[\#set text(9pt); \#sym.arrow.l This field \textbackslash{} breaks into 2 rows.],}
\NormalTok{bytes(2)[Break],}
\NormalTok{note(left)[\#set text(9pt); and continues \textbackslash{} here \#sym.arrow],}
\NormalTok{bits(24,fill: green.lighten(30\%))[Fill],}
\NormalTok{group(right,3)[spanning 3 rows],}
\NormalTok{bytes(12)[\#set text(20pt); *Multi* Row],}
\NormalTok{note(left, bracket: true)[Flags],}
\NormalTok{bits(4)[\#text(8pt)[reserved]],}
\NormalTok{flag[\#text(8pt)[SYN]],}
\NormalTok{flag(fill: orange.lighten(60\%))[\#text(8pt)[ACK]],}
\NormalTok{flag[\#text(8pt)[BOB]],}
\NormalTok{bits(25, fill: purple.lighten(60\%))[Padding],}
\NormalTok{// padding(fill: purple.lighten(40\%))[Padding],}
\NormalTok{bytes(2)[Next],}
\NormalTok{bytes(8, fill: yellow.lighten(60\%))[Multi break],}
\NormalTok{note(right)[\#emoji.checkmark Finish],}
\NormalTok{bytes(2)[\_End\_],}
\NormalTok{)}
\end{Highlighting}
\end{Shaded}

\subsection{Usage}\label{usage}

To use this library through the Typst package manager import bytefield
with \texttt{\ \#import\ "@preview/bytefield:0.0.6":\ *\ } at the top of
your file.

The package contains some of the most common network protocol headers
which are available under: \texttt{\ common.ipv4\ } ,
\texttt{\ common.ipv6\ } , \texttt{\ common.icmp\ } ,
\texttt{\ common.icmpv6\ } , \texttt{\ common.dns\ } ,
\texttt{\ common.tcp\ } , \texttt{\ common.udp\ } .

\subsection{Features}\label{features}

Here is a unsorted list of features which is possible right now.

\begin{itemize}
\tightlist
\item
  Adding fields with \texttt{\ bit\ } , \texttt{\ bits\ } ,
  \texttt{\ byte\ } or \texttt{\ bytes\ } function.

  \begin{itemize}
  \tightlist
  \item
    Fields can be colored
  \item
    Multirow and breaking fields are supported.
  \end{itemize}
\item
  Adding notes to the left or right with \texttt{\ note\ } or
  \texttt{\ group\ } function.
\item
  Config the header with the \texttt{\ bitheader\ } function. !Only one
  header per bytefield is processed currently.

  \begin{itemize}
  \tightlist
  \item
    Show numbers
  \item
    Show numbers and labels
  \item
    Show only labels
  \end{itemize}
\item
  Change the bit order in the header with \texttt{\ msb:left\ } or
  \texttt{\ msb:right\ } (default)
\end{itemize}

See
\href{https://github.com/typst/packages/raw/main/packages/preview/bytefield/0.0.6/example.typ}{example.typ}
for more information.

See
\href{https://github.com/typst/packages/raw/main/packages/preview/bytefield/0.0.6/CHANGELOG.md}{CHANGELOG.md}

\subsubsection{How to add}\label{how-to-add}

Copy this into your project and use the import as \texttt{\ bytefield\ }

\begin{verbatim}
#import "@preview/bytefield:0.0.6"
\end{verbatim}

\includesvg[width=0.16667in,height=0.16667in]{/assets/icons/16-copy.svg}

Check the docs for
\href{https://typst.app/docs/reference/scripting/\#packages}{more
information on how to import packages} .

\subsubsection{About}\label{about}

\begin{description}
\tightlist
\item[Author :]
\href{https://github.com/jomaway}{Jomaway}
\item[License:]
MIT
\item[Current version:]
0.0.6
\item[Last updated:]
May 24, 2024
\item[First released:]
September 3, 2023
\item[Minimum Typst version:]
0.10.0
\item[Archive size:]
12.0 kB
\href{https://packages.typst.org/preview/bytefield-0.0.6.tar.gz}{\pandocbounded{\includesvg[keepaspectratio]{/assets/icons/16-download.svg}}}
\item[Repository:]
\href{https://github.com/jomaway/typst-bytefield}{GitHub}
\end{description}

\subsubsection{Where to report issues?}\label{where-to-report-issues}

This package is a project of Jomaway . Report issues on
\href{https://github.com/jomaway/typst-bytefield}{their repository} .
You can also try to ask for help with this package on the
\href{https://forum.typst.app}{Forum} .

Please report this package to the Typst team using the
\href{https://typst.app/contact}{contact form} if you believe it is a
safety hazard or infringes upon your rights.

\phantomsection\label{versions}
\subsubsection{Version history}\label{version-history}

\begin{longtable}[]{@{}ll@{}}
\toprule\noalign{}
Version & Release Date \\
\midrule\noalign{}
\endhead
\bottomrule\noalign{}
\endlastfoot
0.0.6 & May 24, 2024 \\
\href{https://typst.app/universe/package/bytefield/0.0.5/}{0.0.5} &
March 11, 2024 \\
\href{https://typst.app/universe/package/bytefield/0.0.4/}{0.0.4} &
February 21, 2024 \\
\href{https://typst.app/universe/package/bytefield/0.0.3/}{0.0.3} &
November 20, 2023 \\
\href{https://typst.app/universe/package/bytefield/0.0.2/}{0.0.2} &
October 27, 2023 \\
\href{https://typst.app/universe/package/bytefield/0.0.1/}{0.0.1} &
September 3, 2023 \\
\end{longtable}

Typst GmbH did not create this package and cannot guarantee correct
functionality of this package or compatibility with any version of the
Typst compiler or app.


\title{typst.app/universe/package/cartao}

\phantomsection\label{banner}
\section{cartao}\label{cartao}

{ 0.1.0 }

Dead simple flashcards with Typst.

\phantomsection\label{readme}
Dead simple flashcards with Typst.

\subsection{Example usage:}\label{example-usage}

\begin{Shaded}
\begin{Highlighting}[]
\NormalTok{\#import "@preview/cartao:0.1.0": card, letter8up, a48up}

\NormalTok{\#set page(}
\NormalTok{  paper: "a4",}
\NormalTok{  // paper: "us{-}letter",}
\NormalTok{  // paper: "presentation{-}16{-}9",}
\NormalTok{  margin: (x: 0cm, y: 0cm),}
\NormalTok{)}

\NormalTok{// build the cards}
\NormalTok{\#a48up}
\NormalTok{// \#letter8up}
\NormalTok{// \#present}

\NormalTok{// define your cards}
\NormalTok{\#card(}
\NormalTok{  [Header],}
\NormalTok{  [Footer],}
\NormalTok{  [Question?],}
\NormalTok{  [answer]}
\NormalTok{)}

\NormalTok{\#card(}
\NormalTok{  [portuguese],}
\NormalTok{  [Hint: Its the title of this package!],}
\NormalTok{  [card],}
\NormalTok{  [cartão]}
\NormalTok{)}

\NormalTok{\#card(}
\NormalTok{  [french],}
\NormalTok{  [Hint: close to the portuguese],}
\NormalTok{  [card],}
\NormalTok{  [carte]}
\NormalTok{)}
\end{Highlighting}
\end{Shaded}

\subsection{Documentation}\label{documentation}

\subsubsection{\texorpdfstring{\texttt{\ card\ }}{ card }}\label{card}

Defines a card by updating the below \texttt{\ counter\ } and
\texttt{\ state\ } (s), and dropping a label.

\begin{Shaded}
\begin{Highlighting}[]
\NormalTok{\#let card(header, footer, question, answer) = [}
\NormalTok{  \#cardnumber.step()}
\NormalTok{  \#cardheader.update(header)}
\NormalTok{  \#cardfooter.update(footer)}
\NormalTok{  \#cardquestion.update(question)}
\NormalTok{  \#cardanswer.update(answer)}
\NormalTok{  \textless{}card\textgreater{}}
\NormalTok{]}
\end{Highlighting}
\end{Shaded}

\paragraph{Arguments}\label{arguments}

\begin{itemize}
\tightlist
\item
  \texttt{\ header\ }
\item
  \texttt{\ footer\ }
\item
  \texttt{\ question\ }
\item
  \texttt{\ answer\ }
\end{itemize}

\subsubsection{card builders}\label{card-builders}

\textbf{How they work}

\begin{enumerate}
\tightlist
\item
  Find all locations of the \texttt{\ \textless{}card\textgreater{}\ }
  label
\item
  Get the values of the \texttt{\ cardnumber\ } counter, and
  \texttt{\ cardheader\ } , \texttt{\ cardfooter\ } ,
  \texttt{\ cardquestion\ } , \texttt{\ cardanswer\ } states at each
  \texttt{\ \textless{}card\textgreater{}\ } .
\item
  Populates an array of questions and an array of answers using these
  values

  \begin{itemize}
  \tightlist
  \item
    The \texttt{\ \#a48up\ } and \texttt{\ \#letter8up\ } functions
    describe the layout of each card for each item in these arrays, and
    also rearrange the answers so that the layout makes sense when
    printed double sided.
  \end{itemize}
\item
  Loop over the arrays and dump each item’s \texttt{\ content\ } onto
  the page.

  \begin{itemize}
  \tightlist
  \item
    in the case of \texttt{\ \#a48up\ } and \texttt{\ letter8up\ } ,
    each item is dumped into a 2-column table.
  \end{itemize}
\end{enumerate}

\texttt{\ cartao\ } comes builtin with the following card building
functions. Take a look at the source for how they work, and use them as
a guide to help you build your own flashcards with different
sizes/formats.

\subsubsection{\texorpdfstring{\texttt{\ a48up\ }}{ a48up }}\label{a48up}

Produces a 2x8 portrait card layout on a4 paper.

Designed to be printed double-sided on the perforated 8-up a4 card paper
you can find on
\href{https://www.amazon.ca/s?k=a4+perforated+card&crid=37RT2L4H5XSD0&sprefix=a4+perforated+ca\%2Caps\%2C648&ref=nb_sb_noss}{Amazon}

Usage

\begin{Shaded}
\begin{Highlighting}[]
\NormalTok{\#a48up}
\end{Highlighting}
\end{Shaded}

\subsubsection{\texorpdfstring{\texttt{\ letter8up\ }}{ letter8up }}\label{letter8up}

Produces a 2x8 portrait card layout on us-letter paper.

Usage

\begin{Shaded}
\begin{Highlighting}[]
\NormalTok{\#letter8up}
\end{Highlighting}
\end{Shaded}

\subsubsection{\texorpdfstring{\texttt{\ present\ }}{ present }}\label{present}

A 16:9 presentation of the flashcards with questions and answers on
different slides

Usage

\begin{Shaded}
\begin{Highlighting}[]
\NormalTok{\#present}
\end{Highlighting}
\end{Shaded}

\subsubsection{How to add}\label{how-to-add}

Copy this into your project and use the import as \texttt{\ cartao\ }

\begin{verbatim}
#import "@preview/cartao:0.1.0"
\end{verbatim}

\includesvg[width=0.16667in,height=0.16667in]{/assets/icons/16-copy.svg}

Check the docs for
\href{https://typst.app/docs/reference/scripting/\#packages}{more
information on how to import packages} .

\subsubsection{About}\label{about}

\begin{description}
\tightlist
\item[Author :]
Gavin Vales
\item[License:]
MIT
\item[Current version:]
0.1.0
\item[Last updated:]
November 21, 2023
\item[First released:]
November 21, 2023
\item[Archive size:]
3.40 kB
\href{https://packages.typst.org/preview/cartao-0.1.0.tar.gz}{\pandocbounded{\includesvg[keepaspectratio]{/assets/icons/16-download.svg}}}
\end{description}

\subsubsection{Where to report issues?}\label{where-to-report-issues}

This package is a project of Gavin Vales . You can also try to ask for
help with this package on the \href{https://forum.typst.app}{Forum} .

Please report this package to the Typst team using the
\href{https://typst.app/contact}{contact form} if you believe it is a
safety hazard or infringes upon your rights.

\phantomsection\label{versions}
\subsubsection{Version history}\label{version-history}

\begin{longtable}[]{@{}ll@{}}
\toprule\noalign{}
Version & Release Date \\
\midrule\noalign{}
\endhead
\bottomrule\noalign{}
\endlastfoot
0.1.0 & November 21, 2023 \\
\end{longtable}

Typst GmbH did not create this package and cannot guarantee correct
functionality of this package or compatibility with any version of the
Typst compiler or app.


