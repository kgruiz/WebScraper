\title{typst.app/universe/package/wrap-indent}

\phantomsection\label{banner}
\section{wrap-indent}\label{wrap-indent}

{ 0.1.0 }

Wrap content in custom functions with just indentation

\phantomsection\label{readme}
\texttt{\ wrap-indent\ } is a package for wrapping content in custom
functions with just indentation. This lets you avoid using trailing
square brackets to wrap content, instead you just indent it!

This system works by re-purposing Typst’s existing
\href{https://typst.app/docs/reference/model/terms/}{term-list} syntax
via a custom show rule on \texttt{\ terms.item\ } . We give it our
custom function within
\href{https://typst.app/docs/reference/introspection/state/}{state} via
a new \texttt{\ wrap-in()\ } function.

\subsection{Here’s a minimal
example!}\label{hereuxe2s-a-minimal-example}

\pandocbounded{\includegraphics[keepaspectratio]{https://github.com/typst/packages/raw/main/packages/preview/wrap-indent/0.1.0/example_page1.png}}

\begin{Shaded}
\begin{Highlighting}[]
\NormalTok{\#set page(height: auto, width: 3.5in, margin: 0.25in)}

\NormalTok{\#import "@preview/wrap{-}indent:0.1.0": wrap{-}in, allow{-}wrapping}

\NormalTok{\#show terms.item: allow{-}wrapping}

\NormalTok{/ First {-}{-}:}
\NormalTok{  A normal term list}

\NormalTok{  with multiple paragraphs}

\NormalTok{But this text is separated}


\NormalTok{\#line(length: 100\%)}


\NormalTok{\#let custom{-}block(content) = rect(content,}
\NormalTok{  fill: orange.lighten(90\%),}
\NormalTok{  stroke: 1.5pt + gradient.linear(..color.map.flare)}
\NormalTok{)}

\NormalTok{/ \#wrap{-}in(custom{-}block):}
\NormalTok{  A *custom block* using the \textasciigrave{}wrap{-}in\textasciigrave{} function}

\NormalTok{  with indented text \textbackslash{}}
\NormalTok{  over multiple lines}

\NormalTok{And this text is \_still\_ separated!}
\end{Highlighting}
\end{Shaded}

And in its own code block, here’s the required initialization:

\begin{Shaded}
\begin{Highlighting}[]
\NormalTok{\#import "@preview/wrap{-}indent:0.1.0": wrap{-}in, allow{-}wrapping}

\NormalTok{\#show terms.item: allow{-}wrapping}
\end{Highlighting}
\end{Shaded}

\subsection{And here’s a more complicated
example!}\label{and-hereuxe2s-a-more-complicated-example}

\pandocbounded{\includegraphics[keepaspectratio]{https://github.com/typst/packages/raw/main/packages/preview/wrap-indent/0.1.0/example_page2.png}}

\begin{Shaded}
\begin{Highlighting}[]
\NormalTok{\#set page(height: auto, width: 4.1in, margin: 0.25in)}

\NormalTok{\#show heading: set text(size: 0.75em)}
\NormalTok{\#show heading: set block(below: 1em)}
\NormalTok{\#set heading(numbering: "1) ")}

\NormalTok{= Normal function call:}

\NormalTok{// A function for wrapping some text:}
\NormalTok{\#let custom{-}quote(body) = rect(}
\NormalTok{  body,}
\NormalTok{  width: 100\%,}
\NormalTok{  fill: luma(95\%),}
\NormalTok{  stroke: (left: 2pt + luma(30\%))}
\NormalTok{)}

\NormalTok{\#custom{-}quote[}
\NormalTok{  Some text in a \_custom quote\_ spread over multiple lines}
\NormalTok{  so it actually looks like it was typed in a document.}
\NormalTok{]}
\NormalTok{This text is outside the quote box}


\NormalTok{= Wrappped function call!}

\NormalTok{/ \#wrap{-}in(custom{-}quote):}
\NormalTok{  Some text in a \_custom quote\_ spread over multiple lines}
\NormalTok{  so it actually looks like it was typed in a document.}

\NormalTok{This text is \_still\_ outside the quote box!}


\NormalTok{= Arbitrary functions should \_just work\#emoji.tm;\_}

\NormalTok{/ \#wrap{-}in(x =\textgreater{} ellipse(align(center, x),}
\NormalTok{    stroke: 3pt + gradient.conic(..color.map.rainbow)}
\NormalTok{  )):}
\NormalTok{  Some text in a \_rainbow ellipse\_ spread}
\NormalTok{  over multiple lines so it actually looks}
\NormalTok{  like it was typed in a document.}


\NormalTok{= One{-}liners look great!}

\NormalTok{/ \#wrap{-}in(underline): Here\textquotesingle{}s one line underlined}


\NormalTok{= Let\textquotesingle{}s do some math:}

\NormalTok{\#let named{-}thm(name) = (content) =\textgreater{} \{}
\NormalTok{  pad(left: 2em, par(hanging{-}indent: {-}2em)[}
\NormalTok{    *Theorem* (\#name) \#emph(content)}
\NormalTok{  ])}
\NormalTok{\}}

\NormalTok{/ \#wrap{-}in(named{-}thm("Operational Soundness")):}
\NormalTok{  If $med tack e : tau$ and $e$ reduces to $e\textquotesingle{}$}
\NormalTok{  by zero or more steps and $"Irred"(e\textquotesingle{})$,}
\NormalTok{  then $e\textquotesingle{} in "Val"$ and $med tack e\textquotesingle{} : tau$.}


\NormalTok{= In{-}line styling doesn\textquotesingle{}t create blocks:}

\NormalTok{/ \#wrap{-}in(highlight):}
\NormalTok{  This text is highlighted.}
\NormalTok{This text isn\textquotesingle{}t.}

\NormalTok{Notice how there was *no* paragraph break between the}
\NormalTok{two sentences? This is a useful result that makes}
\NormalTok{\textasciigrave{}wrap{-}indent\textasciigrave{} really flexible!}

\NormalTok{(if you want separate blocks, use \textasciigrave{}block\textasciigrave{} in your function)}


\NormalTok{= Does it work with nesting?}

\NormalTok{/ \#wrap{-}in(custom{-}quote):}
\NormalTok{  Testing...}
\NormalTok{  / \#wrap{-}in(align.with(center)):}
\NormalTok{    / \#wrap{-}in(rect):}
\NormalTok{      / \#wrap{-}in(emph):}
\NormalTok{        Signs point to yes!}


\NormalTok{= Final thoughts}

\NormalTok{/ Note {-}{-}:}
\NormalTok{  Regular term lists still work!}

\NormalTok{/ Disclaimer {-}{-}:}
\NormalTok{  You may run into issues with other term list}
\NormalTok{  show rules conflicting with this rule. \textbackslash{}}
\NormalTok{  (although set rules should be unaffected)}

\NormalTok{  If you run into issues, \_let me know!\_ I\textquotesingle{}d love to hear}
\NormalTok{  about it to make this package as robust as possible.}


\NormalTok{= And}

\NormalTok{\#let big{-}statement(content) = \{}
\NormalTok{  align(center, text(}
\NormalTok{    underline(stroke: 1.5pt, content),}
\NormalTok{    size: 32pt,}
\NormalTok{    weight: "bold",}
\NormalTok{    style: "italic",}
\NormalTok{    fill: eastern,}
\NormalTok{  ))}
\NormalTok{\}}

\NormalTok{/ \#wrap{-}in(big{-}statement):}
\NormalTok{  That\textquotesingle{}s a wrap!}
\end{Highlighting}
\end{Shaded}

\subsection{References}\label{references}

You can find my original writeup here for more context:\\
\url{https://typst.app/project/r5ogFas7lj7E48iHw_M4yh}

And also see the GitHub issue that prompted me to make this:\\
\url{https://github.com/typst/typst/issues/1921}

\subsubsection{How to add}\label{how-to-add}

Copy this into your project and use the import as
\texttt{\ wrap-indent\ }

\begin{verbatim}
#import "@preview/wrap-indent:0.1.0"
\end{verbatim}

\includesvg[width=0.16667in,height=0.16667in]{/assets/icons/16-copy.svg}

Check the docs for
\href{https://typst.app/docs/reference/scripting/\#packages}{more
information on how to import packages} .

\subsubsection{About}\label{about}

\begin{description}
\tightlist
\item[Author :]
Ian Wrzesinski (LectronPusher)
\item[License:]
MIT
\item[Current version:]
0.1.0
\item[Last updated:]
May 3, 2024
\item[First released:]
May 3, 2024
\item[Archive size:]
3.69 kB
\href{https://packages.typst.org/preview/wrap-indent-0.1.0.tar.gz}{\pandocbounded{\includesvg[keepaspectratio]{/assets/icons/16-download.svg}}}
\item[Categor ies :]
\begin{itemize}
\tightlist
\item[]
\item
  \pandocbounded{\includesvg[keepaspectratio]{/assets/icons/16-code.svg}}
  \href{https://typst.app/universe/search/?category=scripting}{Scripting}
\item
  \pandocbounded{\includesvg[keepaspectratio]{/assets/icons/16-hammer.svg}}
  \href{https://typst.app/universe/search/?category=utility}{Utility}
\end{itemize}
\end{description}

\subsubsection{Where to report issues?}\label{where-to-report-issues}

This package is a project of Ian Wrzesinski (LectronPusher) . You can
also try to ask for help with this package on the
\href{https://forum.typst.app}{Forum} .

Please report this package to the Typst team using the
\href{https://typst.app/contact}{contact form} if you believe it is a
safety hazard or infringes upon your rights.

\phantomsection\label{versions}
\subsubsection{Version history}\label{version-history}

\begin{longtable}[]{@{}ll@{}}
\toprule\noalign{}
Version & Release Date \\
\midrule\noalign{}
\endhead
\bottomrule\noalign{}
\endlastfoot
0.1.0 & May 3, 2024 \\
\end{longtable}

Typst GmbH did not create this package and cannot guarantee correct
functionality of this package or compatibility with any version of the
Typst compiler or app.


\title{typst.app/universe/package/anatomy}

\phantomsection\label{banner}
\section{anatomy}\label{anatomy}

{ 0.1.1 }

Anatomy of a Font. Visualise metrics.

\phantomsection\label{readme}
\emph{Anatomy of a Font} . Visualise metrics.

Import the \texttt{\ anatomy\ } package:

\begin{Shaded}
\begin{Highlighting}[]
\NormalTok{\#import "@preview/anatomy:0.1.1": metrics}
\end{Highlighting}
\end{Shaded}

\subsection{Display Metrics}\label{display-metrics}

\texttt{\ metrics(72pt,\ "EB\ Garamond",\ display:\ "Typewriter")\ }
will be rendered as follows:

\pandocbounded{\includesvg[keepaspectratio]{https://raw.githubusercontent.com/E8D08F/packages/main/packages/preview/anatomy/0.1.1/img/export-1.svg}}

Additionally, a closure using \texttt{\ metrics\ } dictionary as
parameter can be used to layout additional elements below:

\begin{Shaded}
\begin{Highlighting}[]
\NormalTok{\#metrics(54pt, "一點明體",}
\NormalTok{  display: "電傳打字機",}
\NormalTok{  use: metrics =\textgreater{} table(}
\NormalTok{    columns: 2,}
\NormalTok{    ..metrics.pairs().flatten().map(x =\textgreater{} [ \#x ])}
\NormalTok{  )}
\NormalTok{)}
\end{Highlighting}
\end{Shaded}

It will generate:

\pandocbounded{\includesvg[keepaspectratio]{https://raw.githubusercontent.com/E8D08F/packages/main/packages/preview/anatomy/0.1.1/img/export-2.svg}}

\subsection{Remark on Hybrid
Typesetting}\label{remark-on-hybrid-typesetting}

To typeset CJK text, adopting font’s ascender/descender as
\texttt{\ top-edge\ } / \texttt{\ bottom-edge\ } makes more sense in
some cases. As for most CJK fonts, the difference between ascender and
descender heights will be exact 1em.

Tested with
\texttt{\ metrics(54pt,\ "Hiragino\ Mincho\ ProN",\ "テレタイプ端末")\ }
:

\pandocbounded{\includesvg[keepaspectratio]{https://raw.githubusercontent.com/E8D08F/packages/main/packages/preview/anatomy/0.1.1/img/export-3.svg}}

Since Typst will use metrics of the font which has the largest advance
height amongst the list provided in
\texttt{\ set\ text(font:\ (\ ...\ ))\ } to set up top and bottom edges
of a line, leading might not work as expected in hybrid typesetting.
This issue can be solved by passing the document to
\texttt{\ metrics(use:\ metrics\ =\textgreater{}\ \{\ ...\ \})\ } like
this:

\begin{Shaded}
\begin{Highlighting}[]
\NormalTok{\#show: doc =\textgreater{} metrics(font.size, font.main,}
\NormalTok{  // Retrieve the metrics of the CJK font}
\NormalTok{  use: metrics =\textgreater{} \{}
\NormalTok{    set text(}
\NormalTok{      font.size,}
\NormalTok{      font: ( font.latin, font.main ),}
\NormalTok{      features: ( "pkna", ),}
\NormalTok{      // Use CJK font’s ascender/descender as top/bottom edges}
\NormalTok{      top{-}edge: metrics.ascender,}
\NormalTok{      bottom{-}edge: metrics.descender,}
\NormalTok{      // ...}
\NormalTok{    )}

\NormalTok{    doc}
\NormalTok{  \}}
\NormalTok{)}
\end{Highlighting}
\end{Shaded}

\subsubsection{How to add}\label{how-to-add}

Copy this into your project and use the import as \texttt{\ anatomy\ }

\begin{verbatim}
#import "@preview/anatomy:0.1.1"
\end{verbatim}

\includesvg[width=0.16667in,height=0.16667in]{/assets/icons/16-copy.svg}

Check the docs for
\href{https://typst.app/docs/reference/scripting/\#packages}{more
information on how to import packages} .

\subsubsection{About}\label{about}

\begin{description}
\tightlist
\item[Author :]
Toto
\item[License:]
MIT
\item[Current version:]
0.1.1
\item[Last updated:]
February 19, 2024
\item[First released:]
February 17, 2024
\item[Archive size:]
2.59 kB
\href{https://packages.typst.org/preview/anatomy-0.1.1.tar.gz}{\pandocbounded{\includesvg[keepaspectratio]{/assets/icons/16-download.svg}}}
\end{description}

\subsubsection{Where to report issues?}\label{where-to-report-issues}

This package is a project of Toto . You can also try to ask for help
with this package on the \href{https://forum.typst.app}{Forum} .

Please report this package to the Typst team using the
\href{https://typst.app/contact}{contact form} if you believe it is a
safety hazard or infringes upon your rights.

\phantomsection\label{versions}
\subsubsection{Version history}\label{version-history}

\begin{longtable}[]{@{}ll@{}}
\toprule\noalign{}
Version & Release Date \\
\midrule\noalign{}
\endhead
\bottomrule\noalign{}
\endlastfoot
0.1.1 & February 19, 2024 \\
\href{https://typst.app/universe/package/anatomy/0.1.0/}{0.1.0} &
February 17, 2024 \\
\end{longtable}

Typst GmbH did not create this package and cannot guarantee correct
functionality of this package or compatibility with any version of the
Typst compiler or app.


\title{typst.app/universe/package/mino}

\phantomsection\label{banner}
\section{mino}\label{mino}

{ 0.1.2 }

Render tetris fumen in typst.

\phantomsection\label{readme}
Render tetris \href{https://harddrop.com/fumen/}{fumen} in typst!

\pandocbounded{\includesvg[keepaspectratio]{https://github.com/typst/packages/raw/main/packages/preview/mino/0.1.2/mino.svg}}

\begin{Shaded}
\begin{Highlighting}[]
\NormalTok{\#import "typst{-}package/lib.typ": decode{-}fumen, render{-}field}
\NormalTok{// Uncomment the following line to use the mino from the official package registry}
\NormalTok{// \#import "@preview/mino:0.1.1": decode{-}fumen, render{-}field}
\NormalTok{\#set page(margin: 1.5cm)}

\NormalTok{\#align(center)[}
\NormalTok{  \#text(size: 25pt)[}
\NormalTok{    DT Cannon}
\NormalTok{  ]}
\NormalTok{]}

\NormalTok{\#let fumen = decode{-}fumen("v115@vhPJHYaAkeEfEXoC+BlvlzByEEfE03k2AlP5ABwfAA?A+rQAAqsBsqBvtBTpBVhQeAlvlzByEEfE03k2AlP5ABwvDf?E33ZBBlfbOBV5AAAOfQeAlvlzByEEfE03+2BlP5ABwvDfEV?5k2AlPJVBjzAAA6WQAAzdBpeB0XBXUBFlQnAlvlzByEEfE0?3+2BlP5ABwvDfEXhWEBUYPNBkkuRA1GCLBUupAAdqQnAlvl?zByEEfE038UBlP5ABwvDfEXhWEBUYPNBkkuRA1GCLBU+rAA?AAPAA")}

\NormalTok{\#for i in range(fumen.len()) \{}
\NormalTok{  let field = fumen.at(i).at("field")}
\NormalTok{  [}
\NormalTok{    \#box[}
\NormalTok{      \#render{-}field(field, rows: 8, cell{-}size: 13pt) }
\NormalTok{      (\#(i+1))}
\NormalTok{      \#fumen.at(i).comment}
\NormalTok{    ]}
\NormalTok{    \#h(0.5pt)}
\NormalTok{  ]}
\NormalTok{\}}
\end{Highlighting}
\end{Shaded}

\subsection{Documentation}\label{documentation}

\subsubsection{\texorpdfstring{\texttt{\ decode-fumen\ }}{ decode-fumen }}\label{decode-fumen}

Decode a fumen string into a series of pages.

\paragraph{Arguments}\label{arguments}

\begin{itemize}
\tightlist
\item
  \texttt{\ data\ } : \texttt{\ str\ } - The fumen string to decode
\end{itemize}

\paragraph{Returns}\label{returns}

The pages, of type
\texttt{\ Array\textless{}\{\ field:\ Array\textless{}string,\ 20\textgreater{},\ comment:\ string\ \}\textgreater{}\ }
.

\begin{verbatim}
(
  (
    field: (
      "....",
      "....",
      ...
    ),
    comment: "..."
  ),
  ...
)
\end{verbatim}

\subsubsection{\texorpdfstring{\texttt{\ render-field\ }}{ render-field }}\label{render-field}

\paragraph{Arguments}\label{arguments-1}

\begin{itemize}
\tightlist
\item
  \texttt{\ field\ } : \texttt{\ array\ } of \texttt{\ str\ } - The
  field to render
\item
  \texttt{\ rows\ } : \texttt{\ number\ } - The number of rows to
  render, default to \texttt{\ 20\ }
\item
  \texttt{\ cell-size\ } : \texttt{\ length\ } - The size of each cell,
  default to \texttt{\ 10pt\ }
\item
  \texttt{\ bg-color\ } : \texttt{\ color\ } - The background color,
  default to \texttt{\ \#f3f3ed\ }
\item
  \texttt{\ stroke\ } : The stroke for the field, default to
  \texttt{\ none\ }
\item
  \texttt{\ radius\ } : The border radius for the field, default to
  \texttt{\ 0.25\ *\ cell-size\ }
\item
  \texttt{\ shadow\ } : Whether to show shadow for cells, default to
  \texttt{\ true\ }
\item
  \texttt{\ highlight\ } : Whether to highlight cells, default to
  \texttt{\ true\ }
\item
  \texttt{\ color-data\ } : The color data for the field, default to
  \texttt{\ default-color-data\ } :
\item
  \texttt{\ overdraw\ } : (default, 5) Draw each cell multiple times to
  avoid thin lines between cells. See
  \url{https://github.com/linebender/vello/issues/49}
\end{itemize}

\begin{Shaded}
\begin{Highlighting}[]
\NormalTok{\#let default{-}color = (}
\NormalTok{  "Z": rgb("\#ef624d"),}
\NormalTok{  "S": rgb("\#66c65c"),}
\NormalTok{  "L": rgb("\#ef9535"),}
\NormalTok{  "J": rgb("\#1983bf"),}
\NormalTok{  "T": rgb("\#9c27b0"),}
\NormalTok{  "O": rgb("\#f7d33e"),}
\NormalTok{  "I": rgb("\#41afde"),}
\NormalTok{  "X": rgb("\#686868")}
\NormalTok{)}
\end{Highlighting}
\end{Shaded}

\begin{itemize}
\tightlist
\item
  \texttt{\ highlight-color-data\ } : The highlight color data for the
  field, default to \texttt{\ default-highlight-color\ } :
\end{itemize}

\begin{Shaded}
\begin{Highlighting}[]
\NormalTok{\#let default{-}highlight{-}color = (}
\NormalTok{  "Z": rgb("\#ff9484"),}
\NormalTok{  "S": rgb("\#88ee86"),}
\NormalTok{  "L": rgb("\#ffbf60"),}
\NormalTok{  "J": rgb("\#1ba6f9"),}
\NormalTok{  "T": rgb("\#e56add"),}
\NormalTok{  "O": rgb("\#fff952"),}
\NormalTok{  "I": rgb("\#43d3ff"),}
\NormalTok{  "X": rgb("\#949494")}
\NormalTok{)}
\end{Highlighting}
\end{Shaded}

\begin{itemize}
\tightlist
\item
  \texttt{\ shadow-color\ } : The shadow color for the field, default to
  \texttt{\ \#6f6f6f17\ }
\end{itemize}

\subsubsection{Credit}\label{credit}

The styles and color scheme are inspired by four.lol

\subsubsection{How to add}\label{how-to-add}

Copy this into your project and use the import as \texttt{\ mino\ }

\begin{verbatim}
#import "@preview/mino:0.1.2"
\end{verbatim}

\includesvg[width=0.16667in,height=0.16667in]{/assets/icons/16-copy.svg}

Check the docs for
\href{https://typst.app/docs/reference/scripting/\#packages}{more
information on how to import packages} .

\subsubsection{About}\label{about}

\begin{description}
\tightlist
\item[Author :]
Wenzhuo Liu
\item[License:]
MIT
\item[Current version:]
0.1.2
\item[Last updated:]
May 27, 2024
\item[First released:]
January 8, 2024
\item[Archive size:]
11.3 kB
\href{https://packages.typst.org/preview/mino-0.1.2.tar.gz}{\pandocbounded{\includesvg[keepaspectratio]{/assets/icons/16-download.svg}}}
\item[Repository:]
\href{https://github.com/Enter-tainer/mino}{GitHub}
\end{description}

\subsubsection{Where to report issues?}\label{where-to-report-issues}

This package is a project of Wenzhuo Liu . Report issues on
\href{https://github.com/Enter-tainer/mino}{their repository} . You can
also try to ask for help with this package on the
\href{https://forum.typst.app}{Forum} .

Please report this package to the Typst team using the
\href{https://typst.app/contact}{contact form} if you believe it is a
safety hazard or infringes upon your rights.

\phantomsection\label{versions}
\subsubsection{Version history}\label{version-history}

\begin{longtable}[]{@{}ll@{}}
\toprule\noalign{}
Version & Release Date \\
\midrule\noalign{}
\endhead
\bottomrule\noalign{}
\endlastfoot
0.1.2 & May 27, 2024 \\
\href{https://typst.app/universe/package/mino/0.1.1/}{0.1.1} & January
15, 2024 \\
\href{https://typst.app/universe/package/mino/0.1.0/}{0.1.0} & January
8, 2024 \\
\end{longtable}

Typst GmbH did not create this package and cannot guarantee correct
functionality of this package or compatibility with any version of the
Typst compiler or app.


\title{typst.app/universe/package/circuiteria}

\phantomsection\label{banner}
\section{circuiteria}\label{circuiteria}

{ 0.1.0 }

Drawing block circuits with Typst made easy, using CeTZ

\phantomsection\label{readme}
Circuiteria is a \href{https://typst.app/}{Typst} package for drawing
block circuit diagrams using the
\href{https://typst.app/universe/package/cetz}{CeTZ} package.

\pandocbounded{\includegraphics[keepaspectratio]{https://github.com/typst/packages/raw/main/packages/preview/circuiteria/0.1.0/gallery/platypus.png}}

\subsection{Examples}\label{examples}

\begin{longtable}[]{@{}ll@{}}
\toprule\noalign{}
\endhead
\bottomrule\noalign{}
\endlastfoot
\multicolumn{2}{@{}l@{}}{%
\href{https://github.com/typst/packages/raw/main/packages/preview/circuiteria/0.1.0/gallery/test.typ}{\includegraphics[width=5.20833in,height=\textheight,keepaspectratio]{https://github.com/typst/packages/raw/main/packages/preview/circuiteria/0.1.0/gallery/test.png}}} \\
\multicolumn{2}{@{}l@{}}{%
A bit of eveything} \\
\multicolumn{2}{@{}l@{}}{%
\href{https://github.com/typst/packages/raw/main/packages/preview/circuiteria/0.1.0/gallery/test5.typ}{\includegraphics[width=5.20833in,height=\textheight,keepaspectratio]{https://github.com/typst/packages/raw/main/packages/preview/circuiteria/0.1.0/gallery/test5.png}}} \\
\multicolumn{2}{@{}l@{}}{%
Wires everywhere} \\
\href{https://github.com/typst/packages/raw/main/packages/preview/circuiteria/0.1.0/gallery/test4.typ}{\includegraphics[width=2.60417in,height=\textheight,keepaspectratio]{https://github.com/typst/packages/raw/main/packages/preview/circuiteria/0.1.0/gallery/test4.png}}
&
\href{https://github.com/typst/packages/raw/main/packages/preview/circuiteria/0.1.0/gallery/test6.typ}{\includegraphics[width=2.60417in,height=\textheight,keepaspectratio]{https://github.com/typst/packages/raw/main/packages/preview/circuiteria/0.1.0/gallery/test6.png}} \\
Groups & Rotated \\
\end{longtable}

\begin{quote}
\textbf{Note}\\
These circuit layouts were copied from a digital design course given by
prof. S. Zahno and recreated using this package
\end{quote}

\emph{Click on the example image to jump to the code.}

\subsection{Usage}\label{usage}

For more information, see the
\href{https://github.com/typst/packages/raw/main/packages/preview/circuiteria/0.1.0/manual.pdf}{manual}

To use this package, simply import
\href{https://typst.app/universe/package/circuiteria}{circuiteria} and
call the \texttt{\ circuit\ } function:

\begin{Shaded}
\begin{Highlighting}[]
\NormalTok{\#import "@preview/circuiteria:0.1.0"}
\NormalTok{\#circuiteria.circuit(\{}
\NormalTok{  import circuiteria: *}
\NormalTok{  ...}
\NormalTok{\})}
\end{Highlighting}
\end{Shaded}

\subsubsection{How to add}\label{how-to-add}

Copy this into your project and use the import as
\texttt{\ circuiteria\ }

\begin{verbatim}
#import "@preview/circuiteria:0.1.0"
\end{verbatim}

\includesvg[width=0.16667in,height=0.16667in]{/assets/icons/16-copy.svg}

Check the docs for
\href{https://typst.app/docs/reference/scripting/\#packages}{more
information on how to import packages} .

\subsubsection{About}\label{about}

\begin{description}
\tightlist
\item[Author :]
\href{https://git.kb28.ch/HEL}{Louis Heredero}
\item[License:]
Apache-2.0
\item[Current version:]
0.1.0
\item[Last updated:]
October 3, 2024
\item[First released:]
October 3, 2024
\item[Minimum Typst version:]
0.11.0
\item[Archive size:]
193 kB
\href{https://packages.typst.org/preview/circuiteria-0.1.0.tar.gz}{\pandocbounded{\includesvg[keepaspectratio]{/assets/icons/16-download.svg}}}
\item[Repository:]
\href{https://git.kb28.ch/HEL/circuiteria}{git.kb28.ch}
\item[Categor y :]
\begin{itemize}
\tightlist
\item[]
\item
  \pandocbounded{\includesvg[keepaspectratio]{/assets/icons/16-chart.svg}}
  \href{https://typst.app/universe/search/?category=visualization}{Visualization}
\end{itemize}
\end{description}

\subsubsection{Where to report issues?}\label{where-to-report-issues}

This package is a project of Louis Heredero . Report issues on
\href{https://git.kb28.ch/HEL/circuiteria}{their repository} . You can
also try to ask for help with this package on the
\href{https://forum.typst.app}{Forum} .

Please report this package to the Typst team using the
\href{https://typst.app/contact}{contact form} if you believe it is a
safety hazard or infringes upon your rights.

\phantomsection\label{versions}
\subsubsection{Version history}\label{version-history}

\begin{longtable}[]{@{}ll@{}}
\toprule\noalign{}
Version & Release Date \\
\midrule\noalign{}
\endhead
\bottomrule\noalign{}
\endlastfoot
0.1.0 & October 3, 2024 \\
\end{longtable}

Typst GmbH did not create this package and cannot guarantee correct
functionality of this package or compatibility with any version of the
Typst compiler or app.


\title{typst.app/universe/package/ucpc-solutions}

\phantomsection\label{banner}
\phantomsection\label{template-thumbnail}
\pandocbounded{\includegraphics[keepaspectratio]{https://packages.typst.org/preview/thumbnails/ucpc-solutions-0.1.0-small.webp}}

\section{ucpc-solutions}\label{ucpc-solutions}

{ 0.1.0 }

The port of UCPC solutions theme.

\href{/app?template=ucpc-solutions&version=0.1.0}{Create project in app}

\phantomsection\label{readme}
\href{https://github.com/ShapeLayer/ucpc-solutions__typst}{ucpc-solutions}
is the template for solutions editorial of algorithm contests, used
widely in the \href{https://acmicpc.net/}{“Baekjoon Online Judge�}
users community in Korea.

The original version of ucpc-solution is written in LaTeX(
\href{https://github.com/ucpcc/2020-solutions-theme}{ucpcc/2020-solutions-theme}
), and this is the port of LaTeX ver. This contains content-generating
utils for making solutions editorial and
\href{https://solved.ac/}{“solved.ac�} difficulty expression
presets, a rating system for Baekjoon Online Judge’s problems.

\subsection{Getting Started}\label{getting-started}

\begin{Shaded}
\begin{Highlighting}[]
\NormalTok{\#import "@preview/ucpc{-}solutions:0.1.0" as ucpc}

\NormalTok{\#show: ucpc.ucpc.with(}
\NormalTok{  title: "Contest Name",}
\NormalTok{  authors: ("Contest Authors", ),}
\NormalTok{)}
\end{Highlighting}
\end{Shaded}

\subsubsection{Requirements}\label{requirements}

\textbf{Fonts}

\begin{itemize}
\tightlist
\item
  \href{https://fonts.google.com/specimen/Inter}{Inter}
\item
  (optional) \href{https://fonts.google.com/specimen/Gothic+A1}{Gothic
  A1}
\item
  (optional)
  \href{https://github.com/orioncactus/pretendard/blob/main/packages/pretendard/docs/en/README.md}{Pretendard}
\end{itemize}

\subsection{Examples}\label{examples}

See
\href{https://github.com/typst/packages/raw/main/packages/preview/ucpc-solutions/0.1.0/examples/}{\texttt{\ /examples\ }}
.

You can also see other usecase using the original LaTeX theme. See the
\href{https://github.com/ucpcc/2020-solutions-theme\#\%ED\%85\%8C\%EB\%A7\%88-\%EC\%82\%AC\%EC\%9A\%A9-\%EC\%98\%88}{(KR)
“Theme Usage Examples(í\ldots Œë§ˆ 사용 예)â€? section} in the
origin repository’s README.

\subsection{For Contributing}\label{for-contributing}

Requirements: \href{https://github.com/casey/just}{just} ,
\href{https://github.com/tingerrr/typst-test}{typst-test}

\textbf{Recompile Refs for Testing}

\begin{Shaded}
\begin{Highlighting}[]
\ExtensionTok{just}\NormalTok{ update{-}test}
\end{Highlighting}
\end{Shaded}

\textbf{Run Test}

\begin{Shaded}
\begin{Highlighting}[]
\ExtensionTok{just}\NormalTok{ test}
\end{Highlighting}
\end{Shaded}

\begin{center}\rule{0.5\linewidth}{0.5pt}\end{center}

\begin{itemize}
\item
  Special Thanks: \href{https://github.com/kiwiyou}{@kiwiyou} - about
  technical issue
\item
  Since this ported version has been re-implemented only for appearance,
  this repository does not include the source code of any distribution
  or variant of ucpc-solutions.
\end{itemize}

\href{/app?template=ucpc-solutions&version=0.1.0}{Create project in app}

\subsubsection{How to use}\label{how-to-use}

Click the button above to create a new project using this template in
the Typst app.

You can also use the Typst CLI to start a new project on your computer
using this command:

\begin{verbatim}
typst init @preview/ucpc-solutions:0.1.0
\end{verbatim}

\includesvg[width=0.16667in,height=0.16667in]{/assets/icons/16-copy.svg}

\subsubsection{About}\label{about}

\begin{description}
\tightlist
\item[Author s :]
\href{https://github.com/ShapeLayer}{Jonghyeon Park} \&
\href{https://github.com/ucpcc}{The Union of Collegiate Programming
Contest Clubs}
\item[License:]
MIT
\item[Current version:]
0.1.0
\item[Last updated:]
August 14, 2024
\item[First released:]
August 14, 2024
\item[Minimum Typst version:]
0.1.0
\item[Archive size:]
22.2 kB
\href{https://packages.typst.org/preview/ucpc-solutions-0.1.0.tar.gz}{\pandocbounded{\includesvg[keepaspectratio]{/assets/icons/16-download.svg}}}
\item[Repository:]
\href{https://github.com/ShapeLayer/ucpc-solutions__typst}{GitHub}
\item[Categor y :]
\begin{itemize}
\tightlist
\item[]
\item
  \pandocbounded{\includesvg[keepaspectratio]{/assets/icons/16-presentation.svg}}
  \href{https://typst.app/universe/search/?category=presentation}{Presentation}
\end{itemize}
\end{description}

\subsubsection{Where to report issues?}\label{where-to-report-issues}

This template is a project of Jonghyeon Park and The Union of Collegiate
Programming Contest Clubs . Report issues on
\href{https://github.com/ShapeLayer/ucpc-solutions__typst}{their
repository} . You can also try to ask for help with this template on the
\href{https://forum.typst.app}{Forum} .

Please report this template to the Typst team using the
\href{https://typst.app/contact}{contact form} if you believe it is a
safety hazard or infringes upon your rights.

\phantomsection\label{versions}
\subsubsection{Version history}\label{version-history}

\begin{longtable}[]{@{}ll@{}}
\toprule\noalign{}
Version & Release Date \\
\midrule\noalign{}
\endhead
\bottomrule\noalign{}
\endlastfoot
0.1.0 & August 14, 2024 \\
\end{longtable}

Typst GmbH did not create this template and cannot guarantee correct
functionality of this template or compatibility with any version of the
Typst compiler or app.


\title{typst.app/universe/package/dining-table}

\phantomsection\label{banner}
\section{dining-table}\label{dining-table}

{ 0.1.0 }

Column-wise table definitions for big data

\phantomsection\label{readme}
Version 0.1.0

Implements a layer on top of table to allow the user to define a table
by column rather than by row, to automatically handle headers and
footers, to implement table footnotes, to handle nested column quirks
for you, to handle rendering nested data structures.

Basically, if you are tabulating data where each row is an observation,
and some features (columns) are to be grouped (a common case for
scientific data) then this package might be worth checking out. Another
use case is where you have multiple tables with identical layouts, and
you wish to keep them all consistent with one another.

\pandocbounded{\includegraphics[keepaspectratio]{https://github.com/typst/packages/raw/main/packages/preview/dining-table/0.1.0/examples/ledger.png}}

\subsection{Usage}\label{usage}

See the manual for in-depth usage, but for a quick reference, here is
the ledger example (which is fully featured)

\begin{Shaded}
\begin{Highlighting}[]
\NormalTok{\#import "@preview/dining{-}table:0.1.0"}

\NormalTok{\#let data = (}
\NormalTok{  (}
\NormalTok{    date: datetime.today(),}
\NormalTok{    particulars: lorem(05),}
\NormalTok{    ledger: [JRS123] + dining{-}table.note.make[Hello World],}
\NormalTok{    amount: (unit: $100$, decimal: $00$),}
\NormalTok{    total: (unit: $99$, decimal: $00$),}
\NormalTok{  ),}
\NormalTok{)*7 }

\NormalTok{\#import "@preview/typpuccino:0.1.0"}
\NormalTok{\#let bg{-}fill{-}1 = typpuccino.latte.base}
\NormalTok{\#let bg{-}fill{-}2 = typpuccino.latte.mantle}

\NormalTok{\#let example = (}
\NormalTok{  (}
\NormalTok{    key: "date",}
\NormalTok{    header: align(left)[Date],}
\NormalTok{    display: (it)=\textgreater{}it.display(auto),}
\NormalTok{    fill: bg{-}fill{-}1,}
\NormalTok{    align: start,}
\NormalTok{    gutter: 0.5em,}
\NormalTok{  ),}
\NormalTok{  (}
\NormalTok{    key: "particulars",}
\NormalTok{    header: text(tracking: 5pt)[Particulars],}
\NormalTok{    width: 1fr,}
\NormalTok{    gutter: 0.5em,}
\NormalTok{  ),}
\NormalTok{  (}
\NormalTok{    key: "ledger",}
\NormalTok{    header: [Ledger],}
\NormalTok{    fill: bg{-}fill{-}2,}
\NormalTok{    width: 2cm,}
\NormalTok{    gutter: 0.5em,}
\NormalTok{  ),}
\NormalTok{  (}
\NormalTok{    header: align(center)[Amount],}
\NormalTok{    fill: bg{-}fill{-}1,}
\NormalTok{    gutter: 0.5em,}
\NormalTok{    hline: arguments(stroke: dining{-}table.lightrule),}
\NormalTok{    children: (}
\NormalTok{      (}
\NormalTok{        key: "amount.unit", }
\NormalTok{        header: align(left)[£], }
\NormalTok{        width: 5em, }
\NormalTok{        align: right,}
\NormalTok{        vline: arguments(stroke: dining{-}table.lightrule),}
\NormalTok{        gutter: 0em,}
\NormalTok{      ),}
\NormalTok{      (}
\NormalTok{        key: "amount.decimal",}
\NormalTok{        header: align(right, text(number{-}type: "old{-}style")[.00]), }
\NormalTok{        align: left}
\NormalTok{      ),}
\NormalTok{    )}
\NormalTok{  ),}
\NormalTok{  (}
\NormalTok{    header: align(center)[Total],}
\NormalTok{    gutter: 0.5em,}
\NormalTok{    hline: arguments(stroke: dining{-}table.lightrule),}
\NormalTok{    children: (}
\NormalTok{      (}
\NormalTok{        key: "total.unit", }
\NormalTok{        header: align(left)[£], }
\NormalTok{        width: 5em, }
\NormalTok{        align: right,}
\NormalTok{        vline: arguments(stroke: dining{-}table.lightrule),}
\NormalTok{        gutter: 0em,}
\NormalTok{      ),}
\NormalTok{      (}
\NormalTok{        key: "total.decimal",}
\NormalTok{        header: align(right, text(number{-}type: "old{-}style")[.00]), }
\NormalTok{        align: left}
\NormalTok{      ),}
\NormalTok{    )}
\NormalTok{  ),}
\NormalTok{)}

\NormalTok{\#set text(size: 11pt)}
\NormalTok{\#set page(height: auto, margin: 1em)}
\NormalTok{\#dining{-}table.make(columns: example, }
\NormalTok{  data: data, }
\NormalTok{  notes: dining{-}table.note.display{-}list}
\NormalTok{)}
\end{Highlighting}
\end{Shaded}

\subsubsection{How to add}\label{how-to-add}

Copy this into your project and use the import as
\texttt{\ dining-table\ }

\begin{verbatim}
#import "@preview/dining-table:0.1.0"
\end{verbatim}

\includesvg[width=0.16667in,height=0.16667in]{/assets/icons/16-copy.svg}

Check the docs for
\href{https://typst.app/docs/reference/scripting/\#packages}{more
information on how to import packages} .

\subsubsection{About}\label{about}

\begin{description}
\tightlist
\item[Author :]
James R. Swift
\item[License:]
Unlicense
\item[Current version:]
0.1.0
\item[Last updated:]
July 10, 2024
\item[First released:]
July 10, 2024
\item[Archive size:]
598 kB
\href{https://packages.typst.org/preview/dining-table-0.1.0.tar.gz}{\pandocbounded{\includesvg[keepaspectratio]{/assets/icons/16-download.svg}}}
\item[Repository:]
\href{https://github.com/JamesxX/dining-table}{GitHub}
\item[Discipline s :]
\begin{itemize}
\tightlist
\item[]
\item
  \href{https://typst.app/universe/search/?discipline=agriculture}{Agriculture}
\item
  \href{https://typst.app/universe/search/?discipline=biology}{Biology}
\item
  \href{https://typst.app/universe/search/?discipline=chemistry}{Chemistry}
\item
  \href{https://typst.app/universe/search/?discipline=communication}{Communication}
\item
  \href{https://typst.app/universe/search/?discipline=computer-science}{Computer
  Science}
\item
  \href{https://typst.app/universe/search/?discipline=economics}{Economics}
\item
  \href{https://typst.app/universe/search/?discipline=physics}{Physics}
\end{itemize}
\item[Categor ies :]
\begin{itemize}
\tightlist
\item[]
\item
  \pandocbounded{\includesvg[keepaspectratio]{/assets/icons/16-package.svg}}
  \href{https://typst.app/universe/search/?category=components}{Components}
\item
  \pandocbounded{\includesvg[keepaspectratio]{/assets/icons/16-chart.svg}}
  \href{https://typst.app/universe/search/?category=visualization}{Visualization}
\item
  \pandocbounded{\includesvg[keepaspectratio]{/assets/icons/16-list-unordered.svg}}
  \href{https://typst.app/universe/search/?category=model}{Model}
\end{itemize}
\end{description}

\subsubsection{Where to report issues?}\label{where-to-report-issues}

This package is a project of James R. Swift . Report issues on
\href{https://github.com/JamesxX/dining-table}{their repository} . You
can also try to ask for help with this package on the
\href{https://forum.typst.app}{Forum} .

Please report this package to the Typst team using the
\href{https://typst.app/contact}{contact form} if you believe it is a
safety hazard or infringes upon your rights.

\phantomsection\label{versions}
\subsubsection{Version history}\label{version-history}

\begin{longtable}[]{@{}ll@{}}
\toprule\noalign{}
Version & Release Date \\
\midrule\noalign{}
\endhead
\bottomrule\noalign{}
\endlastfoot
0.1.0 & July 10, 2024 \\
\end{longtable}

Typst GmbH did not create this package and cannot guarantee correct
functionality of this package or compatibility with any version of the
Typst compiler or app.


\title{typst.app/universe/package/hydra}

\phantomsection\label{banner}
\section{hydra}\label{hydra}

{ 0.5.1 }

Query and display headings in your documents and templates.

{ } Featured Package

\phantomsection\label{readme}
Hydra is a Typst package allowing you to easily display the heading like
elements anywhere in your document. It’s primary focus is to provide
the reader with a reminder of where they currently are in your document
only when it is needed.

\subsection{Example}\label{example}

\begin{Shaded}
\begin{Highlighting}[]
\NormalTok{\#import "@preview/hydra:0.5.1": hydra}

\NormalTok{\#set page(paper: "a7", margin: (y: 4em), numbering: "1", header: context \{}
\NormalTok{  if calc.odd(here().page()) \{}
\NormalTok{    align(right, emph(hydra(1)))}
\NormalTok{  \} else \{}
\NormalTok{    align(left, emph(hydra(2)))}
\NormalTok{  \}}
\NormalTok{  line(length: 100\%)}
\NormalTok{\})}
\NormalTok{\#set heading(numbering: "1.1")}
\NormalTok{\#show heading.where(level: 1): it =\textgreater{} pagebreak(weak: true) + it}

\NormalTok{= Introduction}
\NormalTok{\#lorem(50)}

\NormalTok{= Content}
\NormalTok{== First Section}
\NormalTok{\#lorem(50)}
\NormalTok{== Second Section}
\NormalTok{\#lorem(100)}
\end{Highlighting}
\end{Shaded}

\pandocbounded{\includegraphics[keepaspectratio]{https://github.com/typst/packages/raw/main/packages/preview/hydra/0.5.1/examples/example.png}}

\subsection{Documentation}\label{documentation}

For a more in-depth description of hydra’s functionality and the
reference read its
\href{https://github.com/typst/packages/raw/main/packages/preview/hydra/0.5.1/doc/manual.pdf}{manual}
.

\subsection{Contribution}\label{contribution}

For contributing, please take a look
\href{https://github.com/typst/packages/raw/main/packages/preview/hydra/0.5.1/CONTRIBUTING.md}{CONTRIBUTING}
.

\subsection{Etymology}\label{etymology}

The package name hydra /ˈhaɪdrə/ is a word play headings and headers,
inspired by the monster in greek and roman mythology resembling a
serpent with many heads.

\subsubsection{How to add}\label{how-to-add}

Copy this into your project and use the import as \texttt{\ hydra\ }

\begin{verbatim}
#import "@preview/hydra:0.5.1"
\end{verbatim}

\includesvg[width=0.16667in,height=0.16667in]{/assets/icons/16-copy.svg}

Check the docs for
\href{https://typst.app/docs/reference/scripting/\#packages}{more
information on how to import packages} .

\subsubsection{About}\label{about}

\begin{description}
\tightlist
\item[Author :]
\href{mailto:me@tinger.dev}{tinger}
\item[License:]
MIT
\item[Current version:]
0.5.1
\item[Last updated:]
July 25, 2024
\item[First released:]
November 19, 2023
\item[Minimum Typst version:]
0.11.0
\item[Archive size:]
238 kB
\href{https://packages.typst.org/preview/hydra-0.5.1.tar.gz}{\pandocbounded{\includesvg[keepaspectratio]{/assets/icons/16-download.svg}}}
\item[Repository:]
\href{https://github.com/tingerrr/hydra}{GitHub}
\item[Categor ies :]
\begin{itemize}
\tightlist
\item[]
\item
  \pandocbounded{\includesvg[keepaspectratio]{/assets/icons/16-package.svg}}
  \href{https://typst.app/universe/search/?category=components}{Components}
\item
  \pandocbounded{\includesvg[keepaspectratio]{/assets/icons/16-code.svg}}
  \href{https://typst.app/universe/search/?category=scripting}{Scripting}
\end{itemize}
\end{description}

\subsubsection{Where to report issues?}\label{where-to-report-issues}

This package is a project of tinger . Report issues on
\href{https://github.com/tingerrr/hydra}{their repository} . You can
also try to ask for help with this package on the
\href{https://forum.typst.app}{Forum} .

Please report this package to the Typst team using the
\href{https://typst.app/contact}{contact form} if you believe it is a
safety hazard or infringes upon your rights.

\phantomsection\label{versions}
\subsubsection{Version history}\label{version-history}

\begin{longtable}[]{@{}ll@{}}
\toprule\noalign{}
Version & Release Date \\
\midrule\noalign{}
\endhead
\bottomrule\noalign{}
\endlastfoot
0.5.1 & July 25, 2024 \\
\href{https://typst.app/universe/package/hydra/0.5.0/}{0.5.0} & July 3,
2024 \\
\href{https://typst.app/universe/package/hydra/0.4.0/}{0.4.0} & March
21, 2024 \\
\href{https://typst.app/universe/package/hydra/0.3.0/}{0.3.0} & January
8, 2024 \\
\href{https://typst.app/universe/package/hydra/0.2.0/}{0.2.0} & November
25, 2023 \\
\href{https://typst.app/universe/package/hydra/0.1.0/}{0.1.0} & November
19, 2023 \\
\end{longtable}

Typst GmbH did not create this package and cannot guarantee correct
functionality of this package or compatibility with any version of the
Typst compiler or app.


\title{typst.app/universe/package/crossregex}

\phantomsection\label{banner}
\section{crossregex}\label{crossregex}

{ 0.2.0 }

A crossword-like regex game written in Typst.

\phantomsection\label{readme}
A crossword-like game written in Typst. You should fill in letters to
satisfy regular expression constraints. Currently, \emph{squares} and
\emph{regular hexagons} are supported.

\begin{quote}
{[}!note{]} This is not a puzzle solver, but a puzzle layout builder.
\end{quote}

It takes inspiration from a web image, which derives our standard
example.

\pandocbounded{\includesvg[keepaspectratio]{https://github.com/typst/packages/raw/main/packages/preview/crossregex/0.2.0/examples/standard.svg}}

\pandocbounded{\includesvg[keepaspectratio]{https://github.com/typst/packages/raw/main/packages/preview/crossregex/0.2.0/examples/sudoku-main.svg}}

More examples and source code:
\url{https://github.com/QuadnucYard/crossregex-typ}

\subsection{Basic Usage}\label{basic-usage}

We use \texttt{\ crossregex-square\ } and \texttt{\ crossregex-hex\ } to
build square and hex layouts respectively. They have the same argument
formats. A \texttt{\ crossregex\ } dispatcher function can be used for
dynamic grid kind, which is compatible with version 0.1.0.

\begin{Shaded}
\begin{Highlighting}[]
\NormalTok{\#import "@preview/crossregex:0.2.0": crossregex}
\NormalTok{// or import and use \textasciigrave{}crossregex{-}hex\textasciigrave{}}

\NormalTok{\#crossregex(}
\NormalTok{  3,}
\NormalTok{  constraints: (}
\NormalTok{    \textasciigrave{}A.*\textasciigrave{}, \textasciigrave{}B.*\textasciigrave{}, \textasciigrave{}C.*\textasciigrave{}, \textasciigrave{}D.*\textasciigrave{}, \textasciigrave{}E.*\textasciigrave{},}
\NormalTok{    \textasciigrave{}F.*\textasciigrave{}, \textasciigrave{}G.*\textasciigrave{}, \textasciigrave{}H.*\textasciigrave{}, \textasciigrave{}I.*\textasciigrave{}, \textasciigrave{}J.*\textasciigrave{},}
\NormalTok{    \textasciigrave{}K.*\textasciigrave{}, \textasciigrave{}L.*\textasciigrave{}, \textasciigrave{}M.*\textasciigrave{}, \textasciigrave{}N.*\textasciigrave{}, \textasciigrave{}O.*\textasciigrave{},}
\NormalTok{  ),}
\NormalTok{  answer: (}
\NormalTok{    "ABC",}
\NormalTok{    "DEFG",}
\NormalTok{    "HIJKL",}
\NormalTok{    "MNOP",}
\NormalTok{    "QRS",}
\NormalTok{  ),}
\NormalTok{)}
\end{Highlighting}
\end{Shaded}

\begin{Shaded}
\begin{Highlighting}[]
\NormalTok{\#import "@preview/crossregex:0.2.0": crossregex{-}square}

\NormalTok{\#crossregex{-}square(}
\NormalTok{  9,}
\NormalTok{  alphabet: regex("[0{-}9]"),}
\NormalTok{  constraints: (}
\NormalTok{    \textasciigrave{}.*\textasciigrave{},}
\NormalTok{    \textasciigrave{}.*\textasciigrave{},}
\NormalTok{    \textasciigrave{}.*\textasciigrave{},}
\NormalTok{    \textasciigrave{}.*\textasciigrave{},}
\NormalTok{    \textasciigrave{}.*[12]\{2\}8\textasciigrave{},}
\NormalTok{    \textasciigrave{}[1{-}9]9.*\textasciigrave{},}
\NormalTok{    \textasciigrave{}.*\textasciigrave{},}
\NormalTok{    \textasciigrave{}.*\textasciigrave{},}
\NormalTok{    \textasciigrave{}.*\textasciigrave{},}
\NormalTok{    \textasciigrave{}[1{-}9]7[29]\{2\}8.6.*\textasciigrave{},}
\NormalTok{    \textasciigrave{}.*2[\^{}3]\{2\}1.\textasciigrave{},}
\NormalTok{    \textasciigrave{}.9.315[\^{}6]+\textasciigrave{},}
\NormalTok{    \textasciigrave{}.+4[15]\{2\}79.\textasciigrave{},}
\NormalTok{    \textasciigrave{}[75]\{2\}18.63[1{-}9]+\textasciigrave{},}
\NormalTok{    \textasciigrave{}8.*[\^{}2][\^{}3][\^{}1]+56[\^{}6]\textasciigrave{},}
\NormalTok{    \textasciigrave{}[\^{}5{-}6][0{-}9][56]\{2\}.*9\textasciigrave{},}
\NormalTok{    \textasciigrave{}.*\textasciigrave{},}
\NormalTok{    \textasciigrave{}[98]\{2\}5.*[27]\{2\}6\textasciigrave{},}
\NormalTok{  ),}
\NormalTok{  answer: (}
\NormalTok{    "934872651",}
\NormalTok{    "812456937",}
\NormalTok{    "576913482",}
\NormalTok{    "125784369",}
\NormalTok{    "467395128",}
\NormalTok{    "398261574",}
\NormalTok{    "241537896",}
\NormalTok{    "783649215",}
\NormalTok{    "659128743",}
\NormalTok{  ),}
\NormalTok{  cell: rect(width: 1.4em, height: 1.4em, radius: 0.1em, stroke: 1pt + orange, fill: orange.transparentize(80\%)),}
\NormalTok{  cell{-}config: (size: 1.6em, text{-}style: (size: 1.4em)),}
\NormalTok{)}
\end{Highlighting}
\end{Shaded}

\subsection{Document}\label{document}

Details are shown in the doc comments above the \texttt{\ crossregex\ }
function in \texttt{\ lib.typ\ } . You can choose to turn off some
views.

Feel free to open issues if any problems.

\subsection{Changelog}\label{changelog}

\subsubsection{0.2.0}\label{section}

\begin{itemize}
\tightlist
\item
  Feature: Supports square shapes.
\item
  Feature: Supports customization the appearance of everything, even the
  cells.
\item
  Feature: Supports custom alphabets.
\item
  Fix: An mistake related to import in the README example.
\end{itemize}

\subsubsection{0.1.0}\label{section-1}

First release with basic hex features.

\subsubsection{How to add}\label{how-to-add}

Copy this into your project and use the import as
\texttt{\ crossregex\ }

\begin{verbatim}
#import "@preview/crossregex:0.2.0"
\end{verbatim}

\includesvg[width=0.16667in,height=0.16667in]{/assets/icons/16-copy.svg}

Check the docs for
\href{https://typst.app/docs/reference/scripting/\#packages}{more
information on how to import packages} .

\subsubsection{About}\label{about}

\begin{description}
\tightlist
\item[Author :]
QuadnucYard
\item[License:]
MIT
\item[Current version:]
0.2.0
\item[Last updated:]
September 22, 2024
\item[First released:]
September 3, 2024
\item[Minimum Typst version:]
0.11.0
\item[Archive size:]
197 kB
\href{https://packages.typst.org/preview/crossregex-0.2.0.tar.gz}{\pandocbounded{\includesvg[keepaspectratio]{/assets/icons/16-download.svg}}}
\item[Repository:]
\href{https://github.com/QuadnucYard/crossregex-typ}{GitHub}
\item[Categor y :]
\begin{itemize}
\tightlist
\item[]
\item
  \pandocbounded{\includesvg[keepaspectratio]{/assets/icons/16-smile.svg}}
  \href{https://typst.app/universe/search/?category=fun}{Fun}
\end{itemize}
\end{description}

\subsubsection{Where to report issues?}\label{where-to-report-issues}

This package is a project of QuadnucYard . Report issues on
\href{https://github.com/QuadnucYard/crossregex-typ}{their repository} .
You can also try to ask for help with this package on the
\href{https://forum.typst.app}{Forum} .

Please report this package to the Typst team using the
\href{https://typst.app/contact}{contact form} if you believe it is a
safety hazard or infringes upon your rights.

\phantomsection\label{versions}
\subsubsection{Version history}\label{version-history}

\begin{longtable}[]{@{}ll@{}}
\toprule\noalign{}
Version & Release Date \\
\midrule\noalign{}
\endhead
\bottomrule\noalign{}
\endlastfoot
0.2.0 & September 22, 2024 \\
\href{https://typst.app/universe/package/crossregex/0.1.0/}{0.1.0} &
September 3, 2024 \\
\end{longtable}

Typst GmbH did not create this package and cannot guarantee correct
functionality of this package or compatibility with any version of the
Typst compiler or app.


\title{typst.app/universe/package/resume-ng}

\phantomsection\label{banner}
\phantomsection\label{template-thumbnail}
\pandocbounded{\includegraphics[keepaspectratio]{https://packages.typst.org/preview/thumbnails/resume-ng-1.0.0-small.webp}}

\section{resume-ng}\label{resume-ng}

{ 1.0.0 }

A Typst resume designed for optimal information density and aesthetic
appeal.

\href{/app?template=resume-ng&version=1.0.0}{Create project in app}

\phantomsection\label{readme}
A typst resume designed for optimal information density and aesthetic
appeal.

A LaTeX version

\texttt{\ main.typ\ } will be a good start.

A minimal exmaple would be:

\begin{Shaded}
\begin{Highlighting}[]
\NormalTok{\#show: project.with(}
\NormalTok{  title: "Resume{-}ng",}
\NormalTok{  author: (name: "FengKaiyu"),}
\NormalTok{  contacts: }
\NormalTok{    (}
\NormalTok{      "+86 188{-}888{-}8888",}
\NormalTok{       link("https://github.com", "github.com/fky2015"),  }
\NormalTok{      // More items...}
\NormalTok{    )}
\NormalTok{)}

\NormalTok{\#resume{-}section("Educations")}
\NormalTok{\#resume{-}education(}
\NormalTok{  university: "BIT",}
\NormalTok{  degree: "Your degree",}
\NormalTok{  school: "Your Major and school",}
\NormalTok{  start: "2021{-}09",}
\NormalTok{  end: "2024{-}06"}
\NormalTok{)[}
\NormalTok{*GPA: 3.62/4.0*. My main research interest }
\NormalTok{is in \#strong("Byzantine Consensus Algorithm"), }
\NormalTok{and I have some research and engineering experience in the field of distributed systems.}
\NormalTok{]}

\NormalTok{\#resume{-}section[Work Experience]}
\NormalTok{\#resume{-}work(}
\NormalTok{  company: "A company",}
\NormalTok{  duty: "Your duty",}
\NormalTok{  start: "2020.10",}
\NormalTok{  end: "2021.03",}
\NormalTok{)[}
\NormalTok{  {-} *Independently responsible for the design, development, testing and deployment of XXX business backend.* Implemented station letter template rendering service through FaaS, Kafka and other platforms. Provided SDK code to upstream, added or upgraded various offline and online logic.}
\NormalTok{  {-} *Participate in XXX\textquotesingle{}s requirement analysis, system technical solution design; complete requirement development, grey scale testing, go{-}live and monitoring.*}
\NormalTok{]}

\NormalTok{\#resume{-}section[Projects]}

\NormalTok{\#resume{-}project(}
\NormalTok{  title: "Project name",}
\NormalTok{  duty: "Your duty",}
\NormalTok{  start: "2021.11",}
\NormalTok{  end: "2022.07",}
\NormalTok{)[}
\NormalTok{  {-} Implemented a memory pool manager based on an extensible hash table and LRU{-}K, and developed a concurrent B+ tree supporting optimistic locking for read and write operations.}
\NormalTok{  {-} Utilized the volcano model to implement executors for queries, updates, joins, and aggregations, and performed query rewriting and pushing down optimizations.}
\NormalTok{  {-} Implemented concurrency control using 2PL (two{-}phase locking), supporting deadlock handling, multiple isolation levels, table locks, and row locks.}
\NormalTok{]}
\end{Highlighting}
\end{Shaded}

\href{/app?template=resume-ng&version=1.0.0}{Create project in app}

\subsubsection{How to use}\label{how-to-use}

Click the button above to create a new project using this template in
the Typst app.

You can also use the Typst CLI to start a new project on your computer
using this command:

\begin{verbatim}
typst init @preview/resume-ng:1.0.0
\end{verbatim}

\includesvg[width=0.16667in,height=0.16667in]{/assets/icons/16-copy.svg}

\subsubsection{About}\label{about}

\begin{description}
\tightlist
\item[Author :]
\href{https://github.com/fky2015}{FengKaiyu}
\item[License:]
MIT
\item[Current version:]
1.0.0
\item[Last updated:]
October 8, 2024
\item[First released:]
October 8, 2024
\item[Archive size:]
5.27 kB
\href{https://packages.typst.org/preview/resume-ng-1.0.0.tar.gz}{\pandocbounded{\includesvg[keepaspectratio]{/assets/icons/16-download.svg}}}
\item[Repository:]
\href{https://github.com/fky2015/resume-ng-typst}{GitHub}
\item[Categor y :]
\begin{itemize}
\tightlist
\item[]
\item
  \pandocbounded{\includesvg[keepaspectratio]{/assets/icons/16-user.svg}}
  \href{https://typst.app/universe/search/?category=cv}{CV}
\end{itemize}
\end{description}

\subsubsection{Where to report issues?}\label{where-to-report-issues}

This template is a project of FengKaiyu . Report issues on
\href{https://github.com/fky2015/resume-ng-typst}{their repository} .
You can also try to ask for help with this template on the
\href{https://forum.typst.app}{Forum} .

Please report this template to the Typst team using the
\href{https://typst.app/contact}{contact form} if you believe it is a
safety hazard or infringes upon your rights.

\phantomsection\label{versions}
\subsubsection{Version history}\label{version-history}

\begin{longtable}[]{@{}ll@{}}
\toprule\noalign{}
Version & Release Date \\
\midrule\noalign{}
\endhead
\bottomrule\noalign{}
\endlastfoot
1.0.0 & October 8, 2024 \\
\end{longtable}

Typst GmbH did not create this template and cannot guarantee correct
functionality of this template or compatibility with any version of the
Typst compiler or app.


\title{typst.app/universe/package/dvdtyp}

\phantomsection\label{banner}
\phantomsection\label{template-thumbnail}
\pandocbounded{\includegraphics[keepaspectratio]{https://packages.typst.org/preview/thumbnails/dvdtyp-1.0.0-small.webp}}

\section{dvdtyp}\label{dvdtyp}

{ 1.0.0 }

a colorful template for writting handouts or notes

\href{/app?template=dvdtyp&version=1.0.0}{Create project in app}

\phantomsection\label{readme}
A colorful template for writting handouts or notes

\pandocbounded{\includegraphics[keepaspectratio]{https://github.com/typst/packages/raw/main/packages/preview/dvdtyp/1.0.0/thumbnail.png}}

\href{/app?template=dvdtyp&version=1.0.0}{Create project in app}

\subsubsection{How to use}\label{how-to-use}

Click the button above to create a new project using this template in
the Typst app.

You can also use the Typst CLI to start a new project on your computer
using this command:

\begin{verbatim}
typst init @preview/dvdtyp:1.0.0
\end{verbatim}

\includesvg[width=0.16667in,height=0.16667in]{/assets/icons/16-copy.svg}

\subsubsection{About}\label{about}

\begin{description}
\tightlist
\item[Author :]
DVDTSB
\item[License:]
MIT-0
\item[Current version:]
1.0.0
\item[Last updated:]
July 10, 2024
\item[First released:]
July 10, 2024
\item[Archive size:]
3.21 kB
\href{https://packages.typst.org/preview/dvdtyp-1.0.0.tar.gz}{\pandocbounded{\includesvg[keepaspectratio]{/assets/icons/16-download.svg}}}
\item[Repository:]
\href{https://github.com/DVDTSB/dvdtyp}{GitHub}
\item[Categor ies :]
\begin{itemize}
\tightlist
\item[]
\item
  \pandocbounded{\includesvg[keepaspectratio]{/assets/icons/16-smile.svg}}
  \href{https://typst.app/universe/search/?category=fun}{Fun}
\item
  \pandocbounded{\includesvg[keepaspectratio]{/assets/icons/16-layout.svg}}
  \href{https://typst.app/universe/search/?category=layout}{Layout}
\item
  \pandocbounded{\includesvg[keepaspectratio]{/assets/icons/16-text.svg}}
  \href{https://typst.app/universe/search/?category=text}{Text}
\end{itemize}
\end{description}

\subsubsection{Where to report issues?}\label{where-to-report-issues}

This template is a project of DVDTSB . Report issues on
\href{https://github.com/DVDTSB/dvdtyp}{their repository} . You can also
try to ask for help with this template on the
\href{https://forum.typst.app}{Forum} .

Please report this template to the Typst team using the
\href{https://typst.app/contact}{contact form} if you believe it is a
safety hazard or infringes upon your rights.

\phantomsection\label{versions}
\subsubsection{Version history}\label{version-history}

\begin{longtable}[]{@{}ll@{}}
\toprule\noalign{}
Version & Release Date \\
\midrule\noalign{}
\endhead
\bottomrule\noalign{}
\endlastfoot
1.0.0 & July 10, 2024 \\
\end{longtable}

Typst GmbH did not create this template and cannot guarantee correct
functionality of this template or compatibility with any version of the
Typst compiler or app.


\title{typst.app/universe/package/icicle}

\phantomsection\label{banner}
\phantomsection\label{template-thumbnail}
\pandocbounded{\includegraphics[keepaspectratio]{https://packages.typst.org/preview/thumbnails/icicle-0.1.0-small.webp}}

\section{icicle}\label{icicle}

{ 0.1.0 }

Help the Typst Guys reach the helicopter pad and save Christmas!

\href{/app?template=icicle&version=0.1.0}{Create project in app}

\phantomsection\label{readme}
Help the Typst Guys reach the helicopter pad and save Christmas!
Navigate them with the WASD keys and solve puzzles with snowballs to
make way for the Typst Guys.

This small Christmas-themed game is playable in the Typst editor and
best enjoyed with the web app or \texttt{\ typst\ watch\ } . It was
first released for the 24 Days to Christmas campaign in winter of 2023.

\subsection{Usage}\label{usage}

You can use this template in the Typst web app by clicking “Start from
template� on the dashboard and searching for \texttt{\ icicle\ } .

Alternatively, you can use the CLI to kick this project off using the
command

\begin{verbatim}
typst init @preview/icicle
\end{verbatim}

Typst will create a new directory with all the files needed to get you
started.

\subsection{Configuration}\label{configuration}

This template exports the \texttt{\ game\ } function, which accepts a
positional argument for the game input.

The template will initialize your package with a sample call to the
\texttt{\ game\ } function in a show rule. If you want to change an
existing project to use this template, you can add a show rule like this
at the top of your file:

\begin{Shaded}
\begin{Highlighting}[]
\NormalTok{\#import "@preview/icicle:0.1.0": game}
\NormalTok{\#show: game}

\NormalTok{// Move with WASD.}
\end{Highlighting}
\end{Shaded}

You can also add your own levels by adding an array of level definition
strings in the \texttt{\ game\ } function’s named \texttt{\ levels\ }
argument. Each level file must conform to the following format:

\begin{itemize}
\tightlist
\item
  First, a line with two comma separated integers indicating the
  player’s starting position.
\item
  Then, a matrix with the characters f (floor), x (wall), w (water), or
  g (goal).
\item
  Finally, a matrix with the characters b (snowball) or \_ (nothing).
\end{itemize}

The three arguments must be separated by double newlines. Additionally,
each row in the matrices space-separates its values. Newlines terminate
the rows. Comments can be added with a double slash. Find an example for
a valid level string below:

\begin{verbatim}
// The starting position
0, 0

// The back layer
f f f w f f f
f f f w f f f
f f x w f f f
f f f w f f f
f f f w f x x
x x x g x x x

// The front layer.
_ _ b _ _ _ _
_ _ b _ _ _ _
_ _ _ _ b _ _
_ _ _ _ b _ _
_ _ _ _ _ _ _
_ _ _ _ _ _ _
\end{verbatim}

It’s best to put levels into separate files and load them with the
\texttt{\ read\ } function.

\href{/app?template=icicle&version=0.1.0}{Create project in app}

\subsubsection{How to use}\label{how-to-use}

Click the button above to create a new project using this template in
the Typst app.

You can also use the Typst CLI to start a new project on your computer
using this command:

\begin{verbatim}
typst init @preview/icicle:0.1.0
\end{verbatim}

\includesvg[width=0.16667in,height=0.16667in]{/assets/icons/16-copy.svg}

\subsubsection{About}\label{about}

\begin{description}
\tightlist
\item[Author :]
\href{https://typst.app}{Typst GmbH}
\item[License:]
MIT-0
\item[Current version:]
0.1.0
\item[Last updated:]
March 6, 2024
\item[First released:]
March 6, 2024
\item[Minimum Typst version:]
0.8.0
\item[Archive size:]
143 kB
\href{https://packages.typst.org/preview/icicle-0.1.0.tar.gz}{\pandocbounded{\includesvg[keepaspectratio]{/assets/icons/16-download.svg}}}
\item[Repository:]
\href{https://github.com/typst/templates}{GitHub}
\item[Categor y :]
\begin{itemize}
\tightlist
\item[]
\item
  \pandocbounded{\includesvg[keepaspectratio]{/assets/icons/16-smile.svg}}
  \href{https://typst.app/universe/search/?category=fun}{Fun}
\end{itemize}
\end{description}

\subsubsection{Where to report issues?}\label{where-to-report-issues}

This template is a project of Typst GmbH . Report issues on
\href{https://github.com/typst/templates}{their repository} . You can
also try to ask for help with this template on the
\href{https://forum.typst.app}{Forum} .

\phantomsection\label{versions}
\subsubsection{Version history}\label{version-history}

\begin{longtable}[]{@{}ll@{}}
\toprule\noalign{}
Version & Release Date \\
\midrule\noalign{}
\endhead
\bottomrule\noalign{}
\endlastfoot
0.1.0 & March 6, 2024 \\
\end{longtable}


\title{typst.app/universe/package/georges-yetyp}

\phantomsection\label{banner}
\phantomsection\label{template-thumbnail}
\pandocbounded{\includegraphics[keepaspectratio]{https://packages.typst.org/preview/thumbnails/georges-yetyp-0.2.0-small.webp}}

\section{georges-yetyp}\label{georges-yetyp}

{ 0.2.0 }

Unofficial template for Polytech Grenoble internship reports

\href{/app?template=georges-yetyp&version=0.2.0}{Create project in app}

\phantomsection\label{readme}
\href{https://github.com/typst/packages/raw/main/packages/preview/georges-yetyp/0.2.0/README.fr.md}{French
version}

Typst template for Polytech (Grenoble) internship reports.

\href{https://github.com/typst/packages/raw/main/packages/preview/georges-yetyp/0.2.0/thumbnail.png}{\pandocbounded{\includegraphics[keepaspectratio]{https://github.com/typst/packages/raw/main/packages/preview/georges-yetyp/0.2.0/thumbnail.png}}}

\subsection{Usage}\label{usage}

Either use this template
\href{https://typst.app/?template=georges-yetyp&version=0.1.0}{in the
Typst web app} , or use the command line to initialize a new project
based on this template:

\begin{Shaded}
\begin{Highlighting}[]
\ExtensionTok{typst}\NormalTok{ init @preview/georges{-}yetyp}
\end{Highlighting}
\end{Shaded}

Then, replace \texttt{\ logo.png\ } with the logo of the company you
worked for, fill in all the details in the \texttt{\ rapport\ }
parameters, and start writing below.

\subsection{Other schools}\label{other-schools}

Adding support for other schools of the Polytech network would be fairly
easy if you want to re-use this template. All that is needed is a copy
of their logo (with the authorization to use it). Submissions are
welcome.

\href{/app?template=georges-yetyp&version=0.2.0}{Create project in app}

\subsubsection{How to use}\label{how-to-use}

Click the button above to create a new project using this template in
the Typst app.

You can also use the Typst CLI to start a new project on your computer
using this command:

\begin{verbatim}
typst init @preview/georges-yetyp:0.2.0
\end{verbatim}

\includesvg[width=0.16667in,height=0.16667in]{/assets/icons/16-copy.svg}

\subsubsection{About}\label{about}

\begin{description}
\tightlist
\item[Author :]
\href{https://ana.gelez.xyz/}{Ana Gelez}
\item[License:]
GPL-3.0
\item[Current version:]
0.2.0
\item[Last updated:]
September 26, 2024
\item[First released:]
July 5, 2024
\item[Minimum Typst version:]
0.11.0
\item[Archive size:]
116 kB
\href{https://packages.typst.org/preview/georges-yetyp-0.2.0.tar.gz}{\pandocbounded{\includesvg[keepaspectratio]{/assets/icons/16-download.svg}}}
\item[Repository:]
\href{https://github.com/elegaanz/georges-yetyp}{GitHub}
\item[Discipline s :]
\begin{itemize}
\tightlist
\item[]
\item
  \href{https://typst.app/universe/search/?discipline=engineering}{Engineering}
\item
  \href{https://typst.app/universe/search/?discipline=computer-science}{Computer
  Science}
\item
  \href{https://typst.app/universe/search/?discipline=geology}{Geology}
\item
  \href{https://typst.app/universe/search/?discipline=medicine}{Medicine}
\end{itemize}
\item[Categor y :]
\begin{itemize}
\tightlist
\item[]
\item
  \pandocbounded{\includesvg[keepaspectratio]{/assets/icons/16-speak.svg}}
  \href{https://typst.app/universe/search/?category=report}{Report}
\end{itemize}
\end{description}

\subsubsection{Where to report issues?}\label{where-to-report-issues}

This template is a project of Ana Gelez . Report issues on
\href{https://github.com/elegaanz/georges-yetyp}{their repository} . You
can also try to ask for help with this template on the
\href{https://forum.typst.app}{Forum} .

Please report this template to the Typst team using the
\href{https://typst.app/contact}{contact form} if you believe it is a
safety hazard or infringes upon your rights.

\phantomsection\label{versions}
\subsubsection{Version history}\label{version-history}

\begin{longtable}[]{@{}ll@{}}
\toprule\noalign{}
Version & Release Date \\
\midrule\noalign{}
\endhead
\bottomrule\noalign{}
\endlastfoot
0.2.0 & September 26, 2024 \\
\href{https://typst.app/universe/package/georges-yetyp/0.1.0/}{0.1.0} &
July 5, 2024 \\
\end{longtable}

Typst GmbH did not create this template and cannot guarantee correct
functionality of this template or compatibility with any version of the
Typst compiler or app.


\title{typst.app/universe/package/knowledge-key}

\phantomsection\label{banner}
\phantomsection\label{template-thumbnail}
\pandocbounded{\includegraphics[keepaspectratio]{https://packages.typst.org/preview/thumbnails/knowledge-key-1.0.1-small.webp}}

\section{knowledge-key}\label{knowledge-key}

{ 1.0.1 }

A compact cheat-sheet

\href{/app?template=knowledge-key&version=1.0.1}{Create project in app}

\phantomsection\label{readme}
This is a typst template for a compact cheat-sheet.

\subsection{Usage}\label{usage}

You can use this template in the Typst web app by clicking “Start from
template� on the dashboard and searching for
\texttt{\ knowledge-key\ } . Alternatively, you can use the CLI to kick
this project off using the command

\begin{verbatim}
typst init @preview/knowledge-key
\end{verbatim}

Typst will create a new directory with all the files needed to get you
started.

\subsection{Configuration}\label{configuration}

This template exports the \texttt{\ ieee\ } function with the following
named arguments:

\begin{itemize}
\tightlist
\item
  \texttt{\ title\ } : The title of the cheat-sheet
\item
  \texttt{\ authors\ } : A string of authors
\end{itemize}

The function also accepts a single, positional argument for the body of
the paper.

The template will initialize your package with a sample call to the
\texttt{\ knowledge-key\ } function in a show rule. If you want to
change an existing project to use this template, you can add a show rule
like this at the top of your file:

\begin{Shaded}
\begin{Highlighting}[]
\NormalTok{\#import "@preview/knowledge{-}key:1.0.1": *}

\NormalTok{\#show: knowledge{-}key.with(}
\NormalTok{  title: [Title],}
\NormalTok{  authors: "Author1, Author2"}
\NormalTok{)}

\NormalTok{// Your content goes below.}
\end{Highlighting}
\end{Shaded}

\href{/app?template=knowledge-key&version=1.0.1}{Create project in app}

\subsubsection{How to use}\label{how-to-use}

Click the button above to create a new project using this template in
the Typst app.

You can also use the Typst CLI to start a new project on your computer
using this command:

\begin{verbatim}
typst init @preview/knowledge-key:1.0.1
\end{verbatim}

\includesvg[width=0.16667in,height=0.16667in]{/assets/icons/16-copy.svg}

\subsubsection{About}\label{about}

\begin{description}
\tightlist
\item[Author :]
Nick Goetti
\item[License:]
MIT-0
\item[Current version:]
1.0.1
\item[Last updated:]
October 25, 2024
\item[First released:]
October 1, 2024
\item[Minimum Typst version:]
0.11.0
\item[Archive size:]
166 kB
\href{https://packages.typst.org/preview/knowledge-key-1.0.1.tar.gz}{\pandocbounded{\includesvg[keepaspectratio]{/assets/icons/16-download.svg}}}
\item[Repository:]
\href{https://github.com/ngoetti/knowledge-key}{GitHub}
\item[Discipline s :]
\begin{itemize}
\tightlist
\item[]
\item
  \href{https://typst.app/universe/search/?discipline=computer-science}{Computer
  Science}
\item
  \href{https://typst.app/universe/search/?discipline=engineering}{Engineering}
\end{itemize}
\item[Categor y :]
\begin{itemize}
\tightlist
\item[]
\item
  \pandocbounded{\includesvg[keepaspectratio]{/assets/icons/16-map.svg}}
  \href{https://typst.app/universe/search/?category=flyer}{Flyer}
\end{itemize}
\end{description}

\subsubsection{Where to report issues?}\label{where-to-report-issues}

This template is a project of Nick Goetti . Report issues on
\href{https://github.com/ngoetti/knowledge-key}{their repository} . You
can also try to ask for help with this template on the
\href{https://forum.typst.app}{Forum} .

Please report this template to the Typst team using the
\href{https://typst.app/contact}{contact form} if you believe it is a
safety hazard or infringes upon your rights.

\phantomsection\label{versions}
\subsubsection{Version history}\label{version-history}

\begin{longtable}[]{@{}ll@{}}
\toprule\noalign{}
Version & Release Date \\
\midrule\noalign{}
\endhead
\bottomrule\noalign{}
\endlastfoot
1.0.1 & October 25, 2024 \\
\href{https://typst.app/universe/package/knowledge-key/1.0.0/}{1.0.0} &
October 1, 2024 \\
\end{longtable}

Typst GmbH did not create this template and cannot guarantee correct
functionality of this template or compatibility with any version of the
Typst compiler or app.


\title{typst.app/universe/package/cram-snap}

\phantomsection\label{banner}
\phantomsection\label{template-thumbnail}
\pandocbounded{\includegraphics[keepaspectratio]{https://packages.typst.org/preview/thumbnails/cram-snap-0.2.1-small.webp}}

\section{cram-snap}\label{cram-snap}

{ 0.2.1 }

Compact and legible cheat sheets

{ } Featured Template

\href{/app?template=cram-snap&version=0.2.1}{Create project in app}

\phantomsection\label{readme}
Simple cheatsheet template for \href{https://typst.app/}{Typst} that
allows you to snap a quick picture of essential information and cram it
into a useful cheatsheet format.

\subsection{Usage}\label{usage}

You can use this template in the Typst web app by clicking “Start from
template� on the dashboard and searching for \texttt{\ cram-snap\ } .

Alternatively, you can use the CLI to kick this project off using the
command

\begin{verbatim}
typst init @preview/cram-snap
\end{verbatim}

Typst will create a new directory with all the files needed to get you
started.

\subsection{Configuration}\label{configuration}

This template exports the \texttt{\ cram-snap\ } function with the
following named arguments:

\begin{itemize}
\tightlist
\item
  \texttt{\ title\ } : Title of the document
\item
  \texttt{\ subtitle\ } : Subtitle of the document
\item
  \texttt{\ icon\ } : Image that appears next to the title
\item
  \texttt{\ column-number\ } : Number of columns
\end{itemize}

The \texttt{\ theader\ } function is a wrapper around the
\texttt{\ table.header\ } function that creates a header and takes
\texttt{\ colspan\ } as argument to span the header across multiple
table columns (by default it spans across two)

If you want to change an existing project to use this template, you can
add a show rule like this at the top of your file:

\begin{Shaded}
\begin{Highlighting}[]
\NormalTok{\#import "@preview/cram{-}snap:0.2.1": *}

\NormalTok{\#set page(paper: "a4", flipped: true, margin: 1cm)}
\NormalTok{\#set text(font: "Arial", size: 11pt)}

\NormalTok{\#show: cram{-}snap.with(}
\NormalTok{  title: [Cheatsheet],}
\NormalTok{  subtitle: [Cheatsheet for an amazing program],}
\NormalTok{  icon: image("icon.png"),}
\NormalTok{  column{-}number: 3,}
\NormalTok{)}

\NormalTok{// Use it if you want different table columns (the default are: (2fr, 3fr))}
\NormalTok{\#set table(columns: (2fr, 3fr, 3fr))}

\NormalTok{\#table(}
\NormalTok{  theader(colspan: 3)[Great heading that is really looooong],}
\NormalTok{  [Closing the program], [Type \textasciigrave{}:q\textasciigrave{}], [You can also type \textasciigrave{}QQ\textasciigrave{}]}
\NormalTok{)}
\end{Highlighting}
\end{Shaded}

\href{/app?template=cram-snap&version=0.2.1}{Create project in app}

\subsubsection{How to use}\label{how-to-use}

Click the button above to create a new project using this template in
the Typst app.

You can also use the Typst CLI to start a new project on your computer
using this command:

\begin{verbatim}
typst init @preview/cram-snap:0.2.1
\end{verbatim}

\includesvg[width=0.16667in,height=0.16667in]{/assets/icons/16-copy.svg}

\subsubsection{About}\label{about}

\begin{description}
\tightlist
\item[Author :]
kamack38
\item[License:]
MIT
\item[Current version:]
0.2.1
\item[Last updated:]
October 25, 2024
\item[First released:]
May 13, 2024
\item[Archive size:]
3.79 kB
\href{https://packages.typst.org/preview/cram-snap-0.2.1.tar.gz}{\pandocbounded{\includesvg[keepaspectratio]{/assets/icons/16-download.svg}}}
\item[Repository:]
\href{https://github.com/kamack38/cram-snap}{GitHub}
\item[Categor y :]
\begin{itemize}
\tightlist
\item[]
\item
  \pandocbounded{\includesvg[keepaspectratio]{/assets/icons/16-map.svg}}
  \href{https://typst.app/universe/search/?category=flyer}{Flyer}
\end{itemize}
\end{description}

\subsubsection{Where to report issues?}\label{where-to-report-issues}

This template is a project of kamack38 . Report issues on
\href{https://github.com/kamack38/cram-snap}{their repository} . You can
also try to ask for help with this template on the
\href{https://forum.typst.app}{Forum} .

Please report this template to the Typst team using the
\href{https://typst.app/contact}{contact form} if you believe it is a
safety hazard or infringes upon your rights.

\phantomsection\label{versions}
\subsubsection{Version history}\label{version-history}

\begin{longtable}[]{@{}ll@{}}
\toprule\noalign{}
Version & Release Date \\
\midrule\noalign{}
\endhead
\bottomrule\noalign{}
\endlastfoot
0.2.1 & October 25, 2024 \\
\href{https://typst.app/universe/package/cram-snap/0.2.0/}{0.2.0} &
October 15, 2024 \\
\href{https://typst.app/universe/package/cram-snap/0.1.0/}{0.1.0} & May
13, 2024 \\
\end{longtable}

Typst GmbH did not create this template and cannot guarantee correct
functionality of this template or compatibility with any version of the
Typst compiler or app.


\title{typst.app/universe/package/cvssc}

\phantomsection\label{banner}
\section{cvssc}\label{cvssc}

{ 0.1.0 }

Common Vulnerability Scoring System Calculator

\phantomsection\label{readme}
\phantomsection\label{readme-top}{}

\href{https://github.com/DrakeAxelrod/cvssc/graphs/contributors}{\pandocbounded{\includegraphics[keepaspectratio]{https://img.shields.io/github/contributors/DrakeAxelrod/cvssc.svg?style=for-the-badge}}}
\href{https://github.com/DrakeAxelrod/cvssc/network/members}{\pandocbounded{\includegraphics[keepaspectratio]{https://img.shields.io/github/forks/DrakeAxelrod/cvssc.svg?style=for-the-badge}}}
\href{https://github.com/DrakeAxelrod/cvssc/stargazers}{\pandocbounded{\includegraphics[keepaspectratio]{https://img.shields.io/github/stars/DrakeAxelrod/cvssc.svg?style=for-the-badge}}}
\href{https://github.com/DrakeAxelrod/cvssc/issues}{\pandocbounded{\includegraphics[keepaspectratio]{https://img.shields.io/github/issues/DrakeAxelrod/cvssc.svg?style=for-the-badge}}}
\href{https://github.com/DrakeAxelrod/cvssc/blob/master/LICENSE.txt}{\pandocbounded{\includegraphics[keepaspectratio]{https://img.shields.io/github/license/DrakeAxelrod/cvssc.svg?style=for-the-badge}}}

\hfill\break

\subsubsection{cvssc}\label{cvssc-1}

\paragraph{Common Vulnerability Scoring System
Calculator}\label{common-vulnerability-scoring-system-calculator}

The CVSS Typst Library is a \href{https://github.com/typst/}{Typst}
package designed to facilitate the calculation of Common Vulnerability
Scoring System (CVSS) scores for vulnerabilities across multiple
versions, including CVSS 2.0, 3.0, 3.1, and 4.0. This library provides
developers, security analysts, and researchers with a reliable and
efficient toolset for assessing the severity of security vulnerabilities
based on the CVSS standards.\\
\href{https://github.com/DrakeAxelrod/cvssc/tree/main/cvssc/0.1.0/src/docs.pdf}{\textbf{Explore
the docs »}}\\
\strut \\
\href{https://github.com/DrakeAxelrod/cvssc/issues}{Report Bug} ·
\href{https://github.com/DrakeAxelrod/cvssc/issues}{Request Feature}

Table of Contents

\begin{enumerate}
\tightlist
\item
  \href{https://github.com/typst/packages/raw/main/packages/preview/cvssc/0.1.0/\#about-the-project}{About
  The Project}

  \begin{itemize}
  \tightlist
  \item
    \href{https://github.com/typst/packages/raw/main/packages/preview/cvssc/0.1.0/\#built-with}{Built
    With}
  \end{itemize}
\item
  \href{https://github.com/typst/packages/raw/main/packages/preview/cvssc/0.1.0/\#getting-started}{Getting
  Started}

  \begin{itemize}
  \tightlist
  \item
    \href{https://github.com/typst/packages/raw/main/packages/preview/cvssc/0.1.0/\#prerequisites}{Prerequisites}
  \item
    \href{https://github.com/typst/packages/raw/main/packages/preview/cvssc/0.1.0/\#installation}{Installation}
  \end{itemize}
\item
  \href{https://github.com/typst/packages/raw/main/packages/preview/cvssc/0.1.0/\#usage}{Usage}
\item
  \href{https://github.com/typst/packages/raw/main/packages/preview/cvssc/0.1.0/\#roadmap}{Roadmap}
\item
  \href{https://github.com/typst/packages/raw/main/packages/preview/cvssc/0.1.0/\#contributing}{Contributing}
\item
  \href{https://github.com/typst/packages/raw/main/packages/preview/cvssc/0.1.0/\#license}{License}
\item
  \href{https://github.com/typst/packages/raw/main/packages/preview/cvssc/0.1.0/\#contact}{Contact}
\item
  \href{https://github.com/typst/packages/raw/main/packages/preview/cvssc/0.1.0/\#acknowledgments}{Acknowledgments}
\end{enumerate}

(
\href{https://github.com/typst/packages/raw/main/packages/preview/cvssc/0.1.0/\#readme-top}{back
to top} )

\subsubsection{Built With}\label{built-with}

\begin{itemize}
\tightlist
\item
  \href{https://typst.app/}{\pandocbounded{\includegraphics[keepaspectratio]{https://img.shields.io/badge/Typst-239dad?style=for-the-badge&logo=typst&logoColor=white}}}
\item
  \href{https://www.rust-lang.org/}{\pandocbounded{\includegraphics[keepaspectratio]{https://img.shields.io/badge/Rust-b7410e?style=for-the-badge&logo=rust&logoColor=white}}}
\item
  \href{https://webassembly.org/}{\pandocbounded{\includegraphics[keepaspectratio]{https://img.shields.io/badge/WebAssembly-654FF0?style=for-the-badge&logo=webassembly&logoColor=white}}}
\end{itemize}

(
\href{https://github.com/typst/packages/raw/main/packages/preview/cvssc/0.1.0/\#readme-top}{back
to top} )

\subsection{Getting Started}\label{getting-started}

Ensure you have the Typst CLI installed.

\begin{enumerate}
\tightlist
\item
  Import the package
\end{enumerate}

\begin{Shaded}
\begin{Highlighting}[]
\NormalTok{\#import "@preview/cvssc:0.1.0";}
\end{Highlighting}
\end{Shaded}

\begin{enumerate}
\setcounter{enumi}{1}
\tightlist
\item
  Use the various library functions to calculate CVSS scores and
  severities.
\end{enumerate}

\begin{Shaded}
\begin{Highlighting}[]
\NormalTok{\#import "@preview/cvssc:0.1.0";}

\NormalTok{\#cvssc.v2("CVSS:2.0/AV:L/AC:H/Au:M/C:P/I:C/A:C")}

\NormalTok{\#cvssc.v3("CVSS:3.0/AV:N/AC:L/PR:N/UI:N/S:U/C:H/I:H/A:H")}

\NormalTok{\#cvssc.v3("CVSS:3.1/AV:N/AC:L/PR:N/UI:N/S:U/C:L/I:L/A:H")}

\NormalTok{\#cvssc.v4("CVSS:4.0/AV:A/AC:H/AT:P/PR:L/UI:P/VC:H/VI:H/VA:L/SC:L/SI:L/SA:L")}
\end{Highlighting}
\end{Shaded}

\subsubsection{Prerequisites}\label{prerequisites}

\begin{itemize}
\tightlist
\item
  typst (see \href{https://typst.app/}{Typst} )
\end{itemize}

\subsection{Usage}\label{usage}

\emph{Please refer to the
\href{https://github.com/typst/packages/raw/main/packages/preview/cvssc/0.1.0/cvssc/0.1.0/src/docs.pdf}{Docs}}

(
\href{https://github.com/typst/packages/raw/main/packages/preview/cvssc/0.1.0/\#readme-top}{back
to top} )

\subsection{Roadmap}\label{roadmap}

See the \href{https://github.com/DrakeAxelrod/cvssc/issues}{open issues}
for a full list of proposed features (and known issues).

(
\href{https://github.com/typst/packages/raw/main/packages/preview/cvssc/0.1.0/\#readme-top}{back
to top} )

\subsection{Contributing}\label{contributing}

Contributions are what make the open source community such an amazing
place to learn, inspire, and create. Any contributions you make are
\textbf{greatly appreciated} .

If you have a suggestion that would make this better, please fork the
repo and create a pull request. You can also simply open an issue with
the tag “enhancement�. Don’t forget to give the project a star!
Thanks again!

\begin{enumerate}
\tightlist
\item
  Fork the Project
\item
  Create your Feature Branch (
  \texttt{\ git\ checkout\ -b\ feature/AmazingFeature\ } )
\item
  Commit your Changes (
  \texttt{\ git\ commit\ -m\ \textquotesingle{}Add\ some\ AmazingFeature\textquotesingle{}\ }
  )
\item
  Push to the Branch (
  \texttt{\ git\ push\ origin\ feature/AmazingFeature\ } )
\item
  Open a Pull Request
\end{enumerate}

(
\href{https://github.com/typst/packages/raw/main/packages/preview/cvssc/0.1.0/\#readme-top}{back
to top} )

\subsection{License}\label{license}

Distributed under the MIT License. See \texttt{\ LICENSE.txt\ } for more
information.

(
\href{https://github.com/typst/packages/raw/main/packages/preview/cvssc/0.1.0/\#readme-top}{back
to top} )

\subsection{Contact}\label{contact}

Drake Axelrod -
\href{https://github.com/typst/packages/raw/main/packages/preview/cvssc/0.1.0/\%5Bhttps://github/\%5D(https://github.com/DrakeAxelrod/)}{Github
Profile} -
\href{mailto:drakeaxelrod@gmail.com}{\nolinkurl{drakeaxelrod@gmail.com}}

Project Link: \url{https://github.com/DrakeAxelrod/cvssc}

\subsection{Contributors}\label{contributors}

\begin{longtable}[]{@{}
  >{\centering\arraybackslash}p{(\linewidth - 0\tabcolsep) * \real{1.0000}}@{}}
\toprule\noalign{}
\endhead
\bottomrule\noalign{}
\endlastfoot
\begin{minipage}[t]{\linewidth}\centering
\href{https://github.com/DrakeAxelrod}{\pandocbounded{\includegraphics[keepaspectratio]{https://avatars.githubusercontent.com/u/51012876?v=4?s=64}}\\
\textsubscript{\textbf{Drake Axelrod}}}\\
\strut
\end{minipage} \\
\end{longtable}

(
\href{https://github.com/typst/packages/raw/main/packages/preview/cvssc/0.1.0/\#readme-top}{back
to top} )

\subsection{Acknowledgments}\label{acknowledgments}

\begin{itemize}
\tightlist
\item
  \href{https://typst.app/}{Typst}
\item
  \href{https://www.first.org/}{First.org}
\item
  \href{https://docs.rs/nvd-cvss}{Rust Library - nvd-cvss}
\end{itemize}

(
\href{https://github.com/typst/packages/raw/main/packages/preview/cvssc/0.1.0/\#readme-top}{back
to top} )

\subsubsection{How to add}\label{how-to-add}

Copy this into your project and use the import as \texttt{\ cvssc\ }

\begin{verbatim}
#import "@preview/cvssc:0.1.0"
\end{verbatim}

\includesvg[width=0.16667in,height=0.16667in]{/assets/icons/16-copy.svg}

Check the docs for
\href{https://typst.app/docs/reference/scripting/\#packages}{more
information on how to import packages} .

\subsubsection{About}\label{about}

\begin{description}
\tightlist
\item[Author :]
\href{mailto:drakeaxelrod@gmail.com}{Drake Axelrod}
\item[License:]
MIT
\item[Current version:]
0.1.0
\item[Last updated:]
October 28, 2024
\item[First released:]
October 28, 2024
\item[Minimum Typst version:]
0.12.0
\item[Archive size:]
193 kB
\href{https://packages.typst.org/preview/cvssc-0.1.0.tar.gz}{\pandocbounded{\includesvg[keepaspectratio]{/assets/icons/16-download.svg}}}
\end{description}

\subsubsection{Where to report issues?}\label{where-to-report-issues}

This package is a project of Drake Axelrod . You can also try to ask for
help with this package on the \href{https://forum.typst.app}{Forum} .

Please report this package to the Typst team using the
\href{https://typst.app/contact}{contact form} if you believe it is a
safety hazard or infringes upon your rights.

\phantomsection\label{versions}
\subsubsection{Version history}\label{version-history}

\begin{longtable}[]{@{}ll@{}}
\toprule\noalign{}
Version & Release Date \\
\midrule\noalign{}
\endhead
\bottomrule\noalign{}
\endlastfoot
0.1.0 & October 28, 2024 \\
\end{longtable}

Typst GmbH did not create this package and cannot guarantee correct
functionality of this package or compatibility with any version of the
Typst compiler or app.


\title{typst.app/universe/package/slashion}

\phantomsection\label{banner}
\section{slashion}\label{slashion}

{ 0.1.1 }

Fractions with slash.

\phantomsection\label{readme}
You might not like the inline fraction displayed in a vertical layout.
Just use \textbf{Slashion} to convert it to a slash fraction.

\begin{Shaded}
\begin{Highlighting}[]
\NormalTok{\#import "@preview/slashion:0.1.1": slash{-}frac}
\NormalTok{\#show math.equation.where(block: false): slash{-}frac}
\end{Highlighting}
\end{Shaded}

You may also use it solely

\begin{Shaded}
\begin{Highlighting}[]
\NormalTok{\#import "@preview/slashion:0.1.1": slash{-}frac as sfrac}
\NormalTok{$sfrac(1/2)$, $sfrac(3, 4)$ or even $sfrac((5 + 6) / 7 + 8)$ are acceptable.}
\end{Highlighting}
\end{Shaded}

\subsection{Notice}\label{notice}

\begin{enumerate}
\tightlist
\item
  This function converts only the outermoest fraction.
\item
  This function has an option to turn off the auto parenthesizing
  feature: \texttt{\ slash-frac.with(parens:\ false)\ }
\end{enumerate}

\subsubsection{How to add}\label{how-to-add}

Copy this into your project and use the import as \texttt{\ slashion\ }

\begin{verbatim}
#import "@preview/slashion:0.1.1"
\end{verbatim}

\includesvg[width=0.16667in,height=0.16667in]{/assets/icons/16-copy.svg}

Check the docs for
\href{https://typst.app/docs/reference/scripting/\#packages}{more
information on how to import packages} .

\subsubsection{About}\label{about}

\begin{description}
\tightlist
\item[Author :]
sjfhsjfh
\item[License:]
MIT
\item[Current version:]
0.1.1
\item[Last updated:]
November 13, 2024
\item[First released:]
November 12, 2024
\item[Archive size:]
2.23 kB
\href{https://packages.typst.org/preview/slashion-0.1.1.tar.gz}{\pandocbounded{\includesvg[keepaspectratio]{/assets/icons/16-download.svg}}}
\item[Repository:]
\href{https://github.com/sjfhsjfh/slashion}{GitHub}
\item[Categor y :]
\begin{itemize}
\tightlist
\item[]
\item
  \pandocbounded{\includesvg[keepaspectratio]{/assets/icons/16-layout.svg}}
  \href{https://typst.app/universe/search/?category=layout}{Layout}
\end{itemize}
\end{description}

\subsubsection{Where to report issues?}\label{where-to-report-issues}

This package is a project of sjfhsjfh . Report issues on
\href{https://github.com/sjfhsjfh/slashion}{their repository} . You can
also try to ask for help with this package on the
\href{https://forum.typst.app}{Forum} .

Please report this package to the Typst team using the
\href{https://typst.app/contact}{contact form} if you believe it is a
safety hazard or infringes upon your rights.

\phantomsection\label{versions}
\subsubsection{Version history}\label{version-history}

\begin{longtable}[]{@{}ll@{}}
\toprule\noalign{}
Version & Release Date \\
\midrule\noalign{}
\endhead
\bottomrule\noalign{}
\endlastfoot
0.1.1 & November 13, 2024 \\
\href{https://typst.app/universe/package/slashion/0.1.0/}{0.1.0} &
November 12, 2024 \\
\end{longtable}

Typst GmbH did not create this package and cannot guarantee correct
functionality of this package or compatibility with any version of the
Typst compiler or app.


\title{typst.app/universe/package/derive-it}

\phantomsection\label{banner}
\section{derive-it}\label{derive-it}

{ 0.1.1 }

Simple functions for creating fitch-style natural deduction proofs and
derivations.

\phantomsection\label{readme}
A Typst package to create Fitch-style natural deductions.

\pandocbounded{\includegraphics[keepaspectratio]{https://github.com/typst/packages/raw/main/packages/preview/derive-it/0.1.1/examples/example.png}}

\subsection{Usage}\label{usage}

This package provides two functions:

\texttt{\ ded-nat\ } is a function that expects 2 parameters:

\begin{itemize}
\tightlist
\item
  \texttt{\ stcolor\ } : the stroke color of the indentation guides. The
  default is \texttt{\ black\ } .
\item
  \texttt{\ arr\ } : an array with the shape, it can be provided in two
  shapes.

  \begin{itemize}
  \tightlist
  \item
    4 items: (dependency: text content, indentation: integer starting
    from 0, formula: text content, rule: text content).
  \item
    3 items: the same as above, but without the dependency.
  \end{itemize}
\end{itemize}

\texttt{\ ded-nat-boxed\ } is a function that expects 4 parameters, and
returns the deduction in a \texttt{\ box\ } :

\begin{itemize}
\tightlist
\item
  \texttt{\ stcolor\ } : the stroke color of the indentation guides. The
  default is \texttt{\ black\ } .
\item
  \texttt{\ premises-and-conclusion\ } : bool, whether to automatically
  insert or not the premises and conclusion of the derivation above the
  lines. The default is \texttt{\ true\ } .
\item
  \texttt{\ premise-rule-text\ } : text content, used for finding the
  premises’ formulas when \texttt{\ premises-and-conclusion\ } is set
  to \texttt{\ true\ } . The default is \texttt{\ "PR"\ } .
\item
  \texttt{\ arr\ } : an array with the shape, it can be provided in two
  shapes.

  \begin{itemize}
  \tightlist
  \item
    4 items: (dependency: text content, indentation: integer starting
    from 0, formula: text content, rule: text content).
  \item
    3 items: the same as above, but without the dependency.
  \end{itemize}
\end{itemize}

\subsubsection{Example}\label{example}

\begin{Shaded}
\begin{Highlighting}[]
\NormalTok{\#import "@preview/derive{-}it:0.1.1": *}

\NormalTok{\#ded{-}nat(stcolor: black, arr:(}
\NormalTok{    ("1", 0, $forall x (P x) and forall x (Q x)$, "PR"),}
\NormalTok{    ("2", 0, $forall x (P x {-}\textgreater{} R x)$, "PR"),}
  
\NormalTok{    ("1", 0, $forall x (P x)$, "S 1"),}
\NormalTok{    ("1", 0, $P a$, "IU 3"),}
\NormalTok{    ("2", 0, $P a {-}\textgreater{} R a$, "IU 2"),}
\NormalTok{    ("1,2", 0, $R a$, "MP 4, 5"),}
  
\NormalTok{    ("1,2", 0, $forall x (R x)$, "GU 6"),}
\NormalTok{))}

\NormalTok{\#ded{-}nat{-}boxed(stcolor: black, premises{-}and{-}conclusion: false, arr: (}
\NormalTok{  ("1", 0, $forall x (S x b) and not forall y (P y {-}\textgreater{} Q b y)$, "PR"),}
\NormalTok{  ("2", 0, $forall x forall y (Q x y {-}\textgreater{} not Q y x)$, "PR"),}
\NormalTok{    ("3", 1, $not forall x (not P x) {-}\textgreater{} forall y (S y b {-}\textgreater{} Q b y)$, "Sup. RAA"),}
\NormalTok{    ("1", 1, $not forall y (P y {-}\textgreater{} Q b y)$, "S 1"),}
\NormalTok{    ("1", 1, $exists y not (P y {-}\textgreater{} Q b y)$, "EMC 4"),}
\NormalTok{      ("6", 2, $not (P a {-}\textgreater{} Q b a)$, "Sup. IE 5"),}
\NormalTok{        ("7", 3, $not (P a and not Q b a)$, "Sup. RAA"),}
\NormalTok{        ("7", 3, $not P a or not not Q  b a$, "DM 7"),}
\NormalTok{          ("9", 4, $not P a$, "Sup. PC"),}
\NormalTok{          ("9", 4, $not P a or Q b a$, "Disy. 9"),}
\NormalTok{        ("", 3, $not P a {-}\textgreater{} (not P a or Q b a)$, "PC 9{-}10"),}
\NormalTok{          ("12", 4, $not  not Q b a$, "Sup. PC"),}
\NormalTok{          ("12", 4, $Q b a$, "DN 12"),}
\NormalTok{          ("12", 4, $not P a or Q b a$, "Disy. 13"),}
\NormalTok{        ("", 3, $not not Q b a {-}\textgreater{} (not P a or Q b a)$, "PC 12{-}14"),}
\NormalTok{        ("7", 3, $not P a or Q b a$, "Dil. 8,11,15"),}
\NormalTok{        ("7", 3, $P a {-}\textgreater{} Q b a$, "IM 16"),}
\NormalTok{        ("6,7", 3, $(P a {-}\textgreater{} Q b a) and not (P a {-}\textgreater{} Q b a)$, "Conj. 6, 17"),}
\NormalTok{      ("6", 2, $P a and not Q b a$, "RAA 7{-}18"),}
\NormalTok{      ("6", 2, $P a$, "S 19"),}
\NormalTok{      ("6", 2, $exists x (P x)$, "GE 20"),}
\NormalTok{      ("6", 2, $not forall x (not P x)$, "EMC 21"),}
\NormalTok{      ("3,6", 2, $forall y (S y b {-}\textgreater{} Q b y)$, "MP 3, 22"),}
\NormalTok{      ("3,6", 2, $S a b {-}\textgreater{} Q b a$, "IU 23"),}
\NormalTok{      ("1", 2, $forall x (S x b)$, "S 1"),}
\NormalTok{      ("1", 2, $S a b$, "IU 25"),}
\NormalTok{      ("1,3,6", 2, $Q b a$, "MP 24, 25"),}
\NormalTok{      ("6", 2, $not Q b a$, "S 19"),}
\NormalTok{      ("1,3,6", 2, $Q b a or not exists y not (P y {-}\textgreater{} Q b y)$, "Disy. 27"),}
\NormalTok{      ("1,3,6", 2, $not exists y not (P y {-}\textgreater{} Q b y)$, "MTP 28, 29"),}
\NormalTok{    ("1,3", 1, $not exists y not (P y {-}\textgreater{} Q b y)$, "IE 5, 6, 30"),}
\NormalTok{    ("1,3", 1, $not exists y not (P y {-}\textgreater{} Q b y) and exists y not (P y {-}\textgreater{} Q b y)$, "Conj. 5, 31"),}

\NormalTok{  ("1", 0, $not (not forall x (not P x) {-}\textgreater{} forall y ( S y b {-}\textgreater{} Q b y))$, "RAA 3{-}32"),}
\NormalTok{))}
\end{Highlighting}
\end{Shaded}

In order to compile locally \texttt{\ examples/example.typ\ } the
command is:

\begin{Shaded}
\begin{Highlighting}[]
\ExtensionTok{typst}\NormalTok{ compile examples/example.typ }\AttributeTok{{-}root}\NormalTok{ .}
\end{Highlighting}
\end{Shaded}

\subsubsection{How to add}\label{how-to-add}

Copy this into your project and use the import as \texttt{\ derive-it\ }

\begin{verbatim}
#import "@preview/derive-it:0.1.1"
\end{verbatim}

\includesvg[width=0.16667in,height=0.16667in]{/assets/icons/16-copy.svg}

Check the docs for
\href{https://typst.app/docs/reference/scripting/\#packages}{more
information on how to import packages} .

\subsubsection{About}\label{about}

\begin{description}
\tightlist
\item[Author :]
\href{https://github.com/0rphee}{0rphee}
\item[License:]
MIT
\item[Current version:]
0.1.1
\item[Last updated:]
November 14, 2024
\item[First released:]
November 12, 2024
\item[Archive size:]
3.31 kB
\href{https://packages.typst.org/preview/derive-it-0.1.1.tar.gz}{\pandocbounded{\includesvg[keepaspectratio]{/assets/icons/16-download.svg}}}
\item[Repository:]
\href{https://github.com/0rphee/derive-it}{GitHub}
\item[Discipline s :]
\begin{itemize}
\tightlist
\item[]
\item
  \href{https://typst.app/universe/search/?discipline=mathematics}{Mathematics}
\item
  \href{https://typst.app/universe/search/?discipline=philosophy}{Philosophy}
\end{itemize}
\item[Categor ies :]
\begin{itemize}
\tightlist
\item[]
\item
  \pandocbounded{\includesvg[keepaspectratio]{/assets/icons/16-layout.svg}}
  \href{https://typst.app/universe/search/?category=layout}{Layout}
\item
  \pandocbounded{\includesvg[keepaspectratio]{/assets/icons/16-chart.svg}}
  \href{https://typst.app/universe/search/?category=visualization}{Visualization}
\end{itemize}
\end{description}

\subsubsection{Where to report issues?}\label{where-to-report-issues}

This package is a project of 0rphee . Report issues on
\href{https://github.com/0rphee/derive-it}{their repository} . You can
also try to ask for help with this package on the
\href{https://forum.typst.app}{Forum} .

Please report this package to the Typst team using the
\href{https://typst.app/contact}{contact form} if you believe it is a
safety hazard or infringes upon your rights.

\phantomsection\label{versions}
\subsubsection{Version history}\label{version-history}

\begin{longtable}[]{@{}ll@{}}
\toprule\noalign{}
Version & Release Date \\
\midrule\noalign{}
\endhead
\bottomrule\noalign{}
\endlastfoot
0.1.1 & November 14, 2024 \\
\href{https://typst.app/universe/package/derive-it/0.1.0/}{0.1.0} &
November 12, 2024 \\
\end{longtable}

Typst GmbH did not create this package and cannot guarantee correct
functionality of this package or compatibility with any version of the
Typst compiler or app.


\title{typst.app/universe/package/mannot}

\phantomsection\label{banner}
\section{mannot}\label{mannot}

{ 0.1.0 }

A package for highlighting and annotating in math blocks.

\phantomsection\label{readme}
A package for highlighting and annotating in math blocks in
\href{https://typst.app/}{Typst} .

A full documentation is
\href{https://github.com/typst/packages/raw/main/packages/preview/mannot/0.1.0/docs/doc.pdf}{here}
.

\subsection{Example}\label{example}

\begin{Shaded}
\begin{Highlighting}[]
\NormalTok{$}
\NormalTok{mark(1, tag: \#\textless{}num\textgreater{}) / mark(x + 1, tag: \#\textless{}den\textgreater{}, color: \#blue)}
\NormalTok{+ mark(2, tag: \#\textless{}quo\textgreater{}, color: \#red)}

\NormalTok{\#annot(\textless{}num\textgreater{}, pos: top)[Numerator]}
\NormalTok{\#annot(\textless{}den\textgreater{})[Denominator]}
\NormalTok{\#annot(\textless{}quo\textgreater{}, pos: right, yshift: 1em)[Quotient]}
\NormalTok{$}
\end{Highlighting}
\end{Shaded}

\pandocbounded{\includesvg[keepaspectratio]{https://github.com/typst/packages/raw/main/packages/preview/mannot/0.1.0/examples/showcase.svg}}

\subsection{Usage}\label{usage}

Import and initialize the package \texttt{\ mannot\ } on the top of your
document.

\begin{Shaded}
\begin{Highlighting}[]
\NormalTok{\#import "@preview/mannot:0.1.0": *}
\NormalTok{\#show: mannot{-}init}
\end{Highlighting}
\end{Shaded}

To highlight a part of a math block, use the \texttt{\ mark\ } function:

\begin{Shaded}
\begin{Highlighting}[]
\NormalTok{$}
\NormalTok{mark(x)}
\NormalTok{$}
\end{Highlighting}
\end{Shaded}

\pandocbounded{\includesvg[keepaspectratio]{https://github.com/typst/packages/raw/main/packages/preview/mannot/0.1.0/examples/usage1.svg}}

You can also specify a color for the highlighted part:

\begin{Shaded}
\begin{Highlighting}[]
\NormalTok{$ // Need \# before color names.}
\NormalTok{mark(3, color: \#red) mark(x, color: \#blue)}
\NormalTok{+ mark(integral x dif x, color: \#green)}
\NormalTok{$}
\end{Highlighting}
\end{Shaded}

\pandocbounded{\includesvg[keepaspectratio]{https://github.com/typst/packages/raw/main/packages/preview/mannot/0.1.0/examples/usage2.svg}}

To add an annotation to a highlighted part, use the \texttt{\ annot\ }
function. You need to specify the tag of the marked content:

\begin{Shaded}
\begin{Highlighting}[]
\NormalTok{$}
\NormalTok{mark(x, tag: \#\textless{}x\textgreater{})  // Need \# before tags.}
\NormalTok{\#annot(\textless{}x\textgreater{})[Annotation]}
\NormalTok{$}
\end{Highlighting}
\end{Shaded}

\pandocbounded{\includesvg[keepaspectratio]{https://github.com/typst/packages/raw/main/packages/preview/mannot/0.1.0/examples/usage3.svg}}

You can customize the position of the annotation and the vertical
distance from the marked content:

\begin{Shaded}
\begin{Highlighting}[]
\NormalTok{$}
\NormalTok{mark(integral x dif x, tag: \#\textless{}i\textgreater{}, color: \#green)}
\NormalTok{+ mark(3, tag: \#\textless{}3\textgreater{}, color: \#red) mark(x, tag: \#\textless{}x\textgreater{}, color: \#blue)}

\NormalTok{\#annot(\textless{}i\textgreater{}, pos: left)[Set pos to left.]}
\NormalTok{\#annot(\textless{}i\textgreater{}, pos: top + left)[Top left.]}
\NormalTok{\#annot(\textless{}3\textgreater{}, pos: top, yshift: 1.2em)[Use yshift.]}
\NormalTok{\#annot(\textless{}x\textgreater{}, pos: right, yshift: 1.2em)[Auto arrow.]}
\NormalTok{$}
\end{Highlighting}
\end{Shaded}

\pandocbounded{\includesvg[keepaspectratio]{https://github.com/typst/packages/raw/main/packages/preview/mannot/0.1.0/examples/usage4.svg}}

For convenience, you can define custom mark functions:

\begin{Shaded}
\begin{Highlighting}[]
\NormalTok{\#let rmark = mark.with(color: red)}
\NormalTok{\#let gmark = mark.with(color: green)}
\NormalTok{\#let bmark = mark.with(color: blue)}

\NormalTok{$}
\NormalTok{integral\_rmark(0, tag: \#\textless{}i0\textgreater{})\^{}bmark(1, tag: \#\textless{}i1\textgreater{})}
\NormalTok{mark(x\^{}2 + 1, tag: \#\textless{}i2\textgreater{}) dif gmark(x, tag: \#\textless{}i3\textgreater{})}

\NormalTok{\#annot(\textless{}i0\textgreater{})[Begin]}
\NormalTok{\#annot(\textless{}i1\textgreater{}, pos: top)[End]}
\NormalTok{\#annot(\textless{}i2\textgreater{}, pos: top + right)[Integrand]}
\NormalTok{\#annot(\textless{}i3\textgreater{}, pos: right, yshift: .6em)[Variable]}
\NormalTok{$}
\end{Highlighting}
\end{Shaded}

\pandocbounded{\includesvg[keepaspectratio]{https://github.com/typst/packages/raw/main/packages/preview/mannot/0.1.0/examples/usage5.svg}}

\subsubsection{How to add}\label{how-to-add}

Copy this into your project and use the import as \texttt{\ mannot\ }

\begin{verbatim}
#import "@preview/mannot:0.1.0"
\end{verbatim}

\includesvg[width=0.16667in,height=0.16667in]{/assets/icons/16-copy.svg}

Check the docs for
\href{https://typst.app/docs/reference/scripting/\#packages}{more
information on how to import packages} .

\subsubsection{About}\label{about}

\begin{description}
\tightlist
\item[Author :]
ryuryu-ymj
\item[License:]
MIT
\item[Current version:]
0.1.0
\item[Last updated:]
October 21, 2024
\item[First released:]
October 21, 2024
\item[Minimum Typst version:]
0.12.0
\item[Archive size:]
6.84 kB
\href{https://packages.typst.org/preview/mannot-0.1.0.tar.gz}{\pandocbounded{\includesvg[keepaspectratio]{/assets/icons/16-download.svg}}}
\item[Repository:]
\href{https://github.com/ryuryu-ymj/mannot}{GitHub}
\item[Categor ies :]
\begin{itemize}
\tightlist
\item[]
\item
  \pandocbounded{\includesvg[keepaspectratio]{/assets/icons/16-chart.svg}}
  \href{https://typst.app/universe/search/?category=visualization}{Visualization}
\item
  \pandocbounded{\includesvg[keepaspectratio]{/assets/icons/16-layout.svg}}
  \href{https://typst.app/universe/search/?category=layout}{Layout}
\end{itemize}
\end{description}

\subsubsection{Where to report issues?}\label{where-to-report-issues}

This package is a project of ryuryu-ymj . Report issues on
\href{https://github.com/ryuryu-ymj/mannot}{their repository} . You can
also try to ask for help with this package on the
\href{https://forum.typst.app}{Forum} .

Please report this package to the Typst team using the
\href{https://typst.app/contact}{contact form} if you believe it is a
safety hazard or infringes upon your rights.

\phantomsection\label{versions}
\subsubsection{Version history}\label{version-history}

\begin{longtable}[]{@{}ll@{}}
\toprule\noalign{}
Version & Release Date \\
\midrule\noalign{}
\endhead
\bottomrule\noalign{}
\endlastfoot
0.1.0 & October 21, 2024 \\
\end{longtable}

Typst GmbH did not create this package and cannot guarantee correct
functionality of this package or compatibility with any version of the
Typst compiler or app.


\title{typst.app/universe/package/tud-corporate-design-slides}

\phantomsection\label{banner}
\phantomsection\label{template-thumbnail}
\pandocbounded{\includegraphics[keepaspectratio]{https://packages.typst.org/preview/thumbnails/tud-corporate-design-slides-0.1.0-small.webp}}

\section{tud-corporate-design-slides}\label{tud-corporate-design-slides}

{ 0.1.0 }

Presentation template for TU Dresden (Technische Universität Dresden).

\href{/app?template=tud-corporate-design-slides&version=0.1.0}{Create
project in app}

\phantomsection\label{readme}
This template can be used to create presentations in
\href{https://github.com/typst/typst}{Typst} with the corporate design
of \href{https://www.tu-dresden.de/}{TU Dresden} .

\subsection{Usage}\label{usage}

Create a new typst project based on this template locally.

\begin{Shaded}
\begin{Highlighting}[]
\ExtensionTok{typst}\NormalTok{ init @preview/tud{-}corporate{-}design{-}slides}
\BuiltInTok{cd}\NormalTok{ tud{-}corporate{-}design{-}slides}
\end{Highlighting}
\end{Shaded}

Or create a project on the typst web app based on this template.

\subsubsection{Font setup}\label{font-setup}

The fonts \texttt{\ Open\ Sans\ } needs to be installed on your system:

You can download the fonts from the
\href{https://tu-dresden.de/intern/services-und-hilfe/ressourcen/dateien/kommunizieren_und_publizieren/corporate-design/cd-elemente/schrift-tud-open-sans}{TU
Dresden website} .

Once you download the fonts, make sure to install and activate them on
your system.

\subsubsection{Compile (and watch) your typst
file}\label{compile-and-watch-your-typst-file}

\begin{Shaded}
\begin{Highlighting}[]
\ExtensionTok{typst}\NormalTok{ w main.typ}
\end{Highlighting}
\end{Shaded}

This will watch your file and recompile it to a pdf when the file is
saved. For writing, you can use
\href{https://code.visualstudio.com/}{Vscode} with these extensions:
\href{https://marketplace.visualstudio.com/items?itemName=nvarner.typst-lsp}{Typst
LSP} and
\href{https://marketplace.visualstudio.com/items?itemName=mgt19937.typst-preview}{Typst
Preview} . Or use the \href{https://typst.app/}{typst web app} (here you
need to upload the fonts).

\subsection{Todos}\label{todos}

\begin{itemize}
\tightlist
\item
  {[} {]} Add more slide layouts (e.g. 2-column layout)
\item
  {[} {]} Port to \href{https://github.com/touying-typ/touying}{touying}
\end{itemize}

\href{/app?template=tud-corporate-design-slides&version=0.1.0}{Create
project in app}

\subsubsection{How to use}\label{how-to-use}

Click the button above to create a new project using this template in
the Typst app.

You can also use the Typst CLI to start a new project on your computer
using this command:

\begin{verbatim}
typst init @preview/tud-corporate-design-slides:0.1.0
\end{verbatim}

\includesvg[width=0.16667in,height=0.16667in]{/assets/icons/16-copy.svg}

\subsubsection{About}\label{about}

\begin{description}
\tightlist
\item[Author :]
\href{https://github.com/jakoblistabarth}{Jakob Listabarth}
\item[License:]
MIT
\item[Current version:]
0.1.0
\item[Last updated:]
October 10, 2024
\item[First released:]
October 10, 2024
\item[Archive size:]
9.54 kB
\href{https://packages.typst.org/preview/tud-corporate-design-slides-0.1.0.tar.gz}{\pandocbounded{\includesvg[keepaspectratio]{/assets/icons/16-download.svg}}}
\item[Repository:]
\href{https://github.com/jakoblistabarth/tud-corporate-design-slides-typst}{GitHub}
\item[Categor y :]
\begin{itemize}
\tightlist
\item[]
\item
  \pandocbounded{\includesvg[keepaspectratio]{/assets/icons/16-presentation.svg}}
  \href{https://typst.app/universe/search/?category=presentation}{Presentation}
\end{itemize}
\end{description}

\subsubsection{Where to report issues?}\label{where-to-report-issues}

This template is a project of Jakob Listabarth . Report issues on
\href{https://github.com/jakoblistabarth/tud-corporate-design-slides-typst}{their
repository} . You can also try to ask for help with this template on the
\href{https://forum.typst.app}{Forum} .

Please report this template to the Typst team using the
\href{https://typst.app/contact}{contact form} if you believe it is a
safety hazard or infringes upon your rights.

\phantomsection\label{versions}
\subsubsection{Version history}\label{version-history}

\begin{longtable}[]{@{}ll@{}}
\toprule\noalign{}
Version & Release Date \\
\midrule\noalign{}
\endhead
\bottomrule\noalign{}
\endlastfoot
0.1.0 & October 10, 2024 \\
\end{longtable}

Typst GmbH did not create this template and cannot guarantee correct
functionality of this template or compatibility with any version of the
Typst compiler or app.


\title{typst.app/universe/package/m-jaxon}

\phantomsection\label{banner}
\section{m-jaxon}\label{m-jaxon}

{ 0.1.1 }

Render LaTeX equation in typst using MathJax.

\phantomsection\label{readme}
Render LaTeX equation in typst using MathJax.

\textbf{Note:} This package is made for fun and to demonstrate the
capability of typst plugins. And it is \textbf{slow} . To actually
convert LaTeX equations to typst ones, you should use \textbf{pandoc} or
\textbf{texmath} .

\pandocbounded{\includesvg[keepaspectratio]{https://github.com/typst/packages/raw/main/packages/preview/m-jaxon/0.1.1/mj.svg}}

\begin{Shaded}
\begin{Highlighting}[]
\NormalTok{\#import "./typst{-}package/lib.typ" as m{-}jaxon}
\NormalTok{// Uncomment the following line to use the m{-}jaxon from the official package registry}
\NormalTok{// \#import "@preview/m{-}jaxon:0.1.1"}

\NormalTok{= M{-}Jaxon}

\NormalTok{Typst, now with *MathJax*.}

\NormalTok{The equation of mass{-}energy equivalence is often written as $E=m c\^{}2$ in modern physics.}

\NormalTok{But we can also write it using M{-}Jaxon as: \#m{-}jaxon.render("E = mc\^{}2", inline: true)}
\end{Highlighting}
\end{Shaded}

\subsection{Limitations}\label{limitations}

\begin{itemize}
\tightlist
\item
  The baseline of the inline equation still looks a bit off.
\end{itemize}

\subsection{Documentation}\label{documentation}

\subsubsection{\texorpdfstring{\texttt{\ render\ }}{ render }}\label{render}

Render a LaTeX equation string to an svg image. Depending on the
\texttt{\ inline\ } argument, the image will be rendered as an inline
image or a block image.

\paragraph{Arguments}\label{arguments}

\begin{itemize}
\tightlist
\item
  \texttt{\ src\ } : \texttt{\ str\ } or \texttt{\ raw\ } block - The
  LaTeX equation string
\item
  \texttt{\ inline\ } : \texttt{\ bool\ } - Whether to render the image
  as an inline image or a block image
\end{itemize}

\paragraph{Returns}\label{returns}

The image, of type \texttt{\ content\ }

\subsubsection{How to add}\label{how-to-add}

Copy this into your project and use the import as \texttt{\ m-jaxon\ }

\begin{verbatim}
#import "@preview/m-jaxon:0.1.1"
\end{verbatim}

\includesvg[width=0.16667in,height=0.16667in]{/assets/icons/16-copy.svg}

Check the docs for
\href{https://typst.app/docs/reference/scripting/\#packages}{more
information on how to import packages} .

\subsubsection{About}\label{about}

\begin{description}
\tightlist
\item[Author :]
Wenzhuo Liu
\item[License:]
MIT
\item[Current version:]
0.1.1
\item[Last updated:]
January 17, 2024
\item[First released:]
December 14, 2023
\item[Archive size:]
633 kB
\href{https://packages.typst.org/preview/m-jaxon-0.1.1.tar.gz}{\pandocbounded{\includesvg[keepaspectratio]{/assets/icons/16-download.svg}}}
\item[Repository:]
\href{https://github.com/Enter-tainer/m-jaxon}{GitHub}
\end{description}

\subsubsection{Where to report issues?}\label{where-to-report-issues}

This package is a project of Wenzhuo Liu . Report issues on
\href{https://github.com/Enter-tainer/m-jaxon}{their repository} . You
can also try to ask for help with this package on the
\href{https://forum.typst.app}{Forum} .

Please report this package to the Typst team using the
\href{https://typst.app/contact}{contact form} if you believe it is a
safety hazard or infringes upon your rights.

\phantomsection\label{versions}
\subsubsection{Version history}\label{version-history}

\begin{longtable}[]{@{}ll@{}}
\toprule\noalign{}
Version & Release Date \\
\midrule\noalign{}
\endhead
\bottomrule\noalign{}
\endlastfoot
0.1.1 & January 17, 2024 \\
\href{https://typst.app/universe/package/m-jaxon/0.1.0/}{0.1.0} &
December 14, 2023 \\
\end{longtable}

Typst GmbH did not create this package and cannot guarantee correct
functionality of this package or compatibility with any version of the
Typst compiler or app.


\title{typst.app/universe/package/umbra}

\phantomsection\label{banner}
\section{umbra}\label{umbra}

{ 0.1.0 }

Basic shadows for Typst

{ } Featured Package

\phantomsection\label{readme}
Umbra is a library for drawing basic gradient shadows in
\href{https://typst.app/}{typst} . It currently provides only one
function for drawing a shadow along one edge of a path.

\subsection{Examples}\label{examples}

\subsubsection{Basic Shadow}\label{basic-shadow}

\href{https://github.com/typst/packages/raw/main/packages/preview/umbra/0.1.0/gallery/basic.typ}{\includegraphics[width=\linewidth,height=2.60417in,keepaspectratio]{https://github.com/typst/packages/raw/main/packages/preview/umbra/0.1.0/gallery/basic.png}}

\subsubsection{Neumorphism}\label{neumorphism}

\href{https://github.com/typst/packages/raw/main/packages/preview/umbra/0.1.0/gallery/neumorphism.typ}{\includegraphics[width=\linewidth,height=2.60417in,keepaspectratio]{https://github.com/typst/packages/raw/main/packages/preview/umbra/0.1.0/gallery/neumorphism.png}}

\subsubsection{Torn Paper}\label{torn-paper}

\href{https://github.com/typst/packages/raw/main/packages/preview/umbra/0.1.0/gallery/torn-paper.typ}{\includegraphics[width=\linewidth,height=2.60417in,keepaspectratio]{https://github.com/typst/packages/raw/main/packages/preview/umbra/0.1.0/gallery/torn-paper.png}}

\emph{Click on the example image to jump to the code.}

\subsection{Usage}\label{usage}

The following code creates a very basic square shadow:

\begin{verbatim}
#import "@preview/umbra:0.1.0": shadow-path

#shadow-path((10%, 10%), (10%, 90%), (90%, 90%), (90%, 10%), closed: true)
\end{verbatim}

The function syntax is similar to the normal path syntax. The following
arguments were added:

\begin{itemize}
\tightlist
\item
  \texttt{\ shadow-radius\ } (default \texttt{\ 0.5cm\ } ): The shadow
  size in the direction normal to the edge
\item
  \texttt{\ shadow-stops\ } (default \texttt{\ (gray,\ white)\ } ): The
  colours to be used in the shadow, passed directly to
  \texttt{\ gradient\ }
\item
  \texttt{\ correction\ } (default \texttt{\ 5deg\ } ): A small
  correction factor to be added to round shadows at corners. Otherwise,
  there will be a small gap between the two shadows
\end{itemize}

\subsubsection{Vertex Order}\label{vertex-order}

The order of the vertices defines the direction of the shadow. If the
shadow is the wrong way around, just reverse the vertices.

\subsubsection{Transparency}\label{transparency}

This package is designed in such a way that it should support
transparency in the gradients (i.e. corners define shadows using a path
which approximates the arc, instead of an entire circle). However, typst
doesn’t currently support transparency in gradients. (
\href{https://github.com/typst/typst/issues/2546}{issue} ).

In addition, the aforementioned correction factor would likely cause
issues with transparent gradients.

\subsubsection{How to add}\label{how-to-add}

Copy this into your project and use the import as \texttt{\ umbra\ }

\begin{verbatim}
#import "@preview/umbra:0.1.0"
\end{verbatim}

\includesvg[width=0.16667in,height=0.16667in]{/assets/icons/16-copy.svg}

Check the docs for
\href{https://typst.app/docs/reference/scripting/\#packages}{more
information on how to import packages} .

\subsubsection{About}\label{about}

\begin{description}
\tightlist
\item[Author :]
\href{https://github.com/davystrong}{David Armstrong}
\item[License:]
MIT
\item[Current version:]
0.1.0
\item[Last updated:]
August 30, 2024
\item[First released:]
August 30, 2024
\item[Minimum Typst version:]
0.10.0
\item[Archive size:]
3.50 kB
\href{https://packages.typst.org/preview/umbra-0.1.0.tar.gz}{\pandocbounded{\includesvg[keepaspectratio]{/assets/icons/16-download.svg}}}
\item[Repository:]
\href{https://github.com/davystrong/umbra}{GitHub}
\item[Categor y :]
\begin{itemize}
\tightlist
\item[]
\item
  \pandocbounded{\includesvg[keepaspectratio]{/assets/icons/16-chart.svg}}
  \href{https://typst.app/universe/search/?category=visualization}{Visualization}
\end{itemize}
\end{description}

\subsubsection{Where to report issues?}\label{where-to-report-issues}

This package is a project of David Armstrong . Report issues on
\href{https://github.com/davystrong/umbra}{their repository} . You can
also try to ask for help with this package on the
\href{https://forum.typst.app}{Forum} .

Please report this package to the Typst team using the
\href{https://typst.app/contact}{contact form} if you believe it is a
safety hazard or infringes upon your rights.

\phantomsection\label{versions}
\subsubsection{Version history}\label{version-history}

\begin{longtable}[]{@{}ll@{}}
\toprule\noalign{}
Version & Release Date \\
\midrule\noalign{}
\endhead
\bottomrule\noalign{}
\endlastfoot
0.1.0 & August 30, 2024 \\
\end{longtable}

Typst GmbH did not create this package and cannot guarantee correct
functionality of this package or compatibility with any version of the
Typst compiler or app.


\title{typst.app/universe/package/neoplot}

\phantomsection\label{banner}
\section{neoplot}\label{neoplot}

{ 0.0.2 }

Gnuplot in Typst

\phantomsection\label{readme}
A Typst package to use \href{http://www.gnuplot.info/}{gnuplot} in
Typst.

\begin{Shaded}
\begin{Highlighting}[]
\NormalTok{\#import "@preview/neoplot:0.0.2" as gp}
\end{Highlighting}
\end{Shaded}

Execute gnuplot commands:

\begin{Shaded}
\begin{Highlighting}[]
\NormalTok{\#gp.exec(}
\NormalTok{    kind: "command",}
\NormalTok{    \textasciigrave{}\textasciigrave{}\textasciigrave{}gnuplot}
\NormalTok{    reset;}
\NormalTok{    set samples 1000;}
\NormalTok{    plot sin(x),}
\NormalTok{         cos(x)}
\NormalTok{    \textasciigrave{}\textasciigrave{}\textasciigrave{}}
\NormalTok{)}
\end{Highlighting}
\end{Shaded}

Execute a gnuplot script:

\begin{Shaded}
\begin{Highlighting}[]
\NormalTok{\#gp.exec(}
\NormalTok{    \textasciigrave{}\textasciigrave{}\textasciigrave{}gnuplot}
\NormalTok{    reset}
\NormalTok{    \# Can add comments since it is a script}
\NormalTok{    set samples 1000}
\NormalTok{    \# Use a backslash to extend commands}
\NormalTok{    plot sin(x), \textbackslash{}}
\NormalTok{         cos(x)}
\NormalTok{    \textasciigrave{}\textasciigrave{}\textasciigrave{}}
\NormalTok{)}
\end{Highlighting}
\end{Shaded}

To read a data file:

\begin{verbatim}
# datafile.dat
# x  y
  0  0
  2  4
  4  0
\end{verbatim}

\begin{Shaded}
\begin{Highlighting}[]
\NormalTok{\#gp.exec(}
\NormalTok{    \textasciigrave{}\textasciigrave{}\textasciigrave{}gnuplot}
\NormalTok{    $data \textless{}\textless{}EOD}
\NormalTok{    0  0}
\NormalTok{    2  4}
\NormalTok{    4  0}
\NormalTok{    EOD}
\NormalTok{    plot $data with linespoints}
\NormalTok{    \textasciigrave{}\textasciigrave{}\textasciigrave{}}
\NormalTok{)}
\end{Highlighting}
\end{Shaded}

or

\begin{Shaded}
\begin{Highlighting}[]
\NormalTok{\#gp.exec(}
\NormalTok{    // Use a datablock since Typst doesn\textquotesingle{}t support WASI}
\NormalTok{    "$data \textless{}\textless{}EOD\textbackslash{}n" +}
\NormalTok{    // Load "datafile.dat" using Typst}
\NormalTok{    read("datafile.dat") +}
\NormalTok{    "EOD\textbackslash{}n" +}
\NormalTok{    "plot $data with linespoints"}
\NormalTok{)}
\end{Highlighting}
\end{Shaded}

To print \texttt{\ \$data\ } :

\begin{Shaded}
\begin{Highlighting}[]
\NormalTok{\#gp.exec("print $data")}
\end{Highlighting}
\end{Shaded}

\subsubsection{How to add}\label{how-to-add}

Copy this into your project and use the import as \texttt{\ neoplot\ }

\begin{verbatim}
#import "@preview/neoplot:0.0.2"
\end{verbatim}

\includesvg[width=0.16667in,height=0.16667in]{/assets/icons/16-copy.svg}

Check the docs for
\href{https://typst.app/docs/reference/scripting/\#packages}{more
information on how to import packages} .

\subsubsection{About}\label{about}

\begin{description}
\tightlist
\item[Author :]
\href{https://github.com/KNnut}{KNnut}
\item[License:]
BSD-3-Clause
\item[Current version:]
0.0.2
\item[Last updated:]
July 16, 2024
\item[First released:]
June 17, 2024
\item[Minimum Typst version:]
0.11.1
\item[Archive size:]
512 kB
\href{https://packages.typst.org/preview/neoplot-0.0.2.tar.gz}{\pandocbounded{\includesvg[keepaspectratio]{/assets/icons/16-download.svg}}}
\item[Repository:]
\href{https://github.com/KNnut/neoplot}{GitHub}
\item[Discipline :]
\begin{itemize}
\tightlist
\item[]
\item
  \href{https://typst.app/universe/search/?discipline=mathematics}{Mathematics}
\end{itemize}
\item[Categor ies :]
\begin{itemize}
\tightlist
\item[]
\item
  \pandocbounded{\includesvg[keepaspectratio]{/assets/icons/16-chart.svg}}
  \href{https://typst.app/universe/search/?category=visualization}{Visualization}
\item
  \pandocbounded{\includesvg[keepaspectratio]{/assets/icons/16-integration.svg}}
  \href{https://typst.app/universe/search/?category=integration}{Integration}
\end{itemize}
\end{description}

\subsubsection{Where to report issues?}\label{where-to-report-issues}

This package is a project of KNnut . Report issues on
\href{https://github.com/KNnut/neoplot}{their repository} . You can also
try to ask for help with this package on the
\href{https://forum.typst.app}{Forum} .

Please report this package to the Typst team using the
\href{https://typst.app/contact}{contact form} if you believe it is a
safety hazard or infringes upon your rights.

\phantomsection\label{versions}
\subsubsection{Version history}\label{version-history}

\begin{longtable}[]{@{}ll@{}}
\toprule\noalign{}
Version & Release Date \\
\midrule\noalign{}
\endhead
\bottomrule\noalign{}
\endlastfoot
0.0.2 & July 16, 2024 \\
\href{https://typst.app/universe/package/neoplot/0.0.1/}{0.0.1} & June
17, 2024 \\
\end{longtable}

Typst GmbH did not create this package and cannot guarantee correct
functionality of this package or compatibility with any version of the
Typst compiler or app.


\title{typst.app/universe/package/km}

\phantomsection\label{banner}
\section{km}\label{km}

{ 0.1.0 }

Draw simple Karnaugh maps

\phantomsection\label{readme}
Draw simple Karnaugh maps. Use with:

\begin{Shaded}
\begin{Highlighting}[]
\NormalTok{\#import "@preview/km:0.1.0": karnaugh}

\NormalTok{\#karnaugh(("C", "AB"),}
\NormalTok{  implicants: (}
\NormalTok{    (0, 1, 1, 2),}
\NormalTok{    (1, 2, 2, 1),}
\NormalTok{  ),}
\NormalTok{  (}
\NormalTok{    (0, 1, 0, 0),}
\NormalTok{    (0, 1, 1, 1),}
\NormalTok{  )}
\NormalTok{)}
\end{Highlighting}
\end{Shaded}

Samples are available in
\href{https://github.com/typst/packages/blob/main/packages/preview/km/0.1.0/sample.pdf}{\texttt{\ sample.pdf\ }}
.

\subsubsection{How to add}\label{how-to-add}

Copy this into your project and use the import as \texttt{\ km\ }

\begin{verbatim}
#import "@preview/km:0.1.0"
\end{verbatim}

\includesvg[width=0.16667in,height=0.16667in]{/assets/icons/16-copy.svg}

Check the docs for
\href{https://typst.app/docs/reference/scripting/\#packages}{more
information on how to import packages} .

\subsubsection{About}\label{about}

\begin{description}
\tightlist
\item[Author :]
\href{mailto:the@unpopular.me}{Toto}
\item[License:]
MIT
\item[Current version:]
0.1.0
\item[Last updated:]
June 18, 2024
\item[First released:]
June 18, 2024
\item[Minimum Typst version:]
0.11.0
\item[Archive size:]
2.56 kB
\href{https://packages.typst.org/preview/km-0.1.0.tar.gz}{\pandocbounded{\includesvg[keepaspectratio]{/assets/icons/16-download.svg}}}
\item[Repository:]
\href{https://git.sr.ht/~toto/karnaugh}{git.sr.ht}
\item[Discipline :]
\begin{itemize}
\tightlist
\item[]
\item
  \href{https://typst.app/universe/search/?discipline=mathematics}{Mathematics}
\end{itemize}
\item[Categor y :]
\begin{itemize}
\tightlist
\item[]
\item
  \pandocbounded{\includesvg[keepaspectratio]{/assets/icons/16-hammer.svg}}
  \href{https://typst.app/universe/search/?category=utility}{Utility}
\end{itemize}
\end{description}

\subsubsection{Where to report issues?}\label{where-to-report-issues}

This package is a project of Toto . Report issues on
\href{https://git.sr.ht/~toto/karnaugh}{their repository} . You can also
try to ask for help with this package on the
\href{https://forum.typst.app}{Forum} .

Please report this package to the Typst team using the
\href{https://typst.app/contact}{contact form} if you believe it is a
safety hazard or infringes upon your rights.

\phantomsection\label{versions}
\subsubsection{Version history}\label{version-history}

\begin{longtable}[]{@{}ll@{}}
\toprule\noalign{}
Version & Release Date \\
\midrule\noalign{}
\endhead
\bottomrule\noalign{}
\endlastfoot
0.1.0 & June 18, 2024 \\
\end{longtable}

Typst GmbH did not create this package and cannot guarantee correct
functionality of this package or compatibility with any version of the
Typst compiler or app.


\title{typst.app/universe/package/bone-resume}

\phantomsection\label{banner}
\phantomsection\label{template-thumbnail}
\pandocbounded{\includegraphics[keepaspectratio]{https://packages.typst.org/preview/thumbnails/bone-resume-0.3.0-small.webp}}

\section{bone-resume}\label{bone-resume}

{ 0.3.0 }

A colorful resume template for chinese.

\href{/app?template=bone-resume&version=0.3.0}{Create project in app}

\phantomsection\label{readme}
This is a Typst template.

\subsection{Usage}\label{usage}

You can use this template in the Typst web app by clicking “Start from
template� on the dashboard and searching for \texttt{\ bone-resumee\ }
.

Alternatively, you can use the CLI to kick this project off using the
command

\begin{verbatim}
typst init @preview/bone-resume
\end{verbatim}

Typst will create a new directory with all the files needed to get you
started.

\subsubsection{Fonts}\label{fonts}

\begin{itemize}
\tightlist
\item
  \href{https://github.com/adobe-fonts/source-han-sans}{Source Han Sans}
\item
  \href{https://github.com/lxgw/LxgwWenkaiGB}{LXGW WenKai GB}
\item
  \href{https://www.nerdfonts.com/}{Hack Nerd Font}
\end{itemize}

\subsection{Configuration}\label{configuration}

This template exports the \texttt{\ resume-init\ } function with the
following named arguments:

\begin{itemize}
\tightlist
\item
  \texttt{\ authors\ } : Your name.
\item
  \texttt{\ title(optional)\ } : The resume’s title as content.
\item
  \texttt{\ footer(optional)\ } : The resume’s footer as content.
\end{itemize}

The function also accepts a single, positional argument for the body of
the paper.

The template will initialize your package with a sample call to the
\texttt{\ bone-resume\ } function in a show rule. If you want to change
an existing project to use this template, you can add a show rule like
this at the top of your file:

\begin{Shaded}
\begin{Highlighting}[]
\NormalTok{\#import "@preview/bone{-}resume:0.1.0": bone{-}resume}

\NormalTok{\#show: bone{-}resume.with(}
\NormalTok{  author: "六个骨头"}
\NormalTok{)}

\NormalTok{= 个人介绍}
\NormalTok{A Student.}

\NormalTok{// Your content goes below.}
\end{Highlighting}
\end{Shaded}

\href{/app?template=bone-resume&version=0.3.0}{Create project in app}

\subsubsection{How to use}\label{how-to-use}

Click the button above to create a new project using this template in
the Typst app.

You can also use the Typst CLI to start a new project on your computer
using this command:

\begin{verbatim}
typst init @preview/bone-resume:0.3.0
\end{verbatim}

\includesvg[width=0.16667in,height=0.16667in]{/assets/icons/16-copy.svg}

\subsubsection{About}\label{about}

\begin{description}
\tightlist
\item[Author :]
zrr1999
\item[License:]
Apache-2.0
\item[Current version:]
0.3.0
\item[Last updated:]
September 2, 2024
\item[First released:]
June 3, 2024
\item[Archive size:]
7.42 kB
\href{https://packages.typst.org/preview/bone-resume-0.3.0.tar.gz}{\pandocbounded{\includesvg[keepaspectratio]{/assets/icons/16-download.svg}}}
\item[Categor y :]
\begin{itemize}
\tightlist
\item[]
\item
  \pandocbounded{\includesvg[keepaspectratio]{/assets/icons/16-user.svg}}
  \href{https://typst.app/universe/search/?category=cv}{CV}
\end{itemize}
\end{description}

\subsubsection{Where to report issues?}\label{where-to-report-issues}

This template is a project of zrr1999 . You can also try to ask for help
with this template on the \href{https://forum.typst.app}{Forum} .

Please report this template to the Typst team using the
\href{https://typst.app/contact}{contact form} if you believe it is a
safety hazard or infringes upon your rights.

\phantomsection\label{versions}
\subsubsection{Version history}\label{version-history}

\begin{longtable}[]{@{}ll@{}}
\toprule\noalign{}
Version & Release Date \\
\midrule\noalign{}
\endhead
\bottomrule\noalign{}
\endlastfoot
0.3.0 & September 2, 2024 \\
\href{https://typst.app/universe/package/bone-resume/0.2.0/}{0.2.0} &
July 4, 2024 \\
\href{https://typst.app/universe/package/bone-resume/0.1.0/}{0.1.0} &
June 3, 2024 \\
\end{longtable}

Typst GmbH did not create this template and cannot guarantee correct
functionality of this template or compatibility with any version of the
Typst compiler or app.


