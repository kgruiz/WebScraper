\title{typst.app/universe/package/silver-dev-cv}

\phantomsection\label{banner}
\phantomsection\label{template-thumbnail}
\pandocbounded{\includegraphics[keepaspectratio]{https://packages.typst.org/preview/thumbnails/silver-dev-cv-1.0.1-small.webp}}

\section{silver-dev-cv}\label{silver-dev-cv}

{ 1.0.1 }

A CV template by an engineer-recruiter, used by https://silver.dev

\href{/app?template=silver-dev-cv&version=1.0.1}{Create project in app}

\phantomsection\label{readme}
This Typst CV template is a streamlined version of the the Latex
template \href{https://github.com/jxpeng98/Typst-CV-Resume}{Modernpro} .

\subsection{How to start}\label{how-to-start}

\subsubsection{Use Typst CLI}\label{use-typst-cli}

If you use Typst CLI, you can use the following command to create a new
project:

\begin{Shaded}
\begin{Highlighting}[]
\ExtensionTok{typst}\NormalTok{ init silver{-}dev{-}cv}
\end{Highlighting}
\end{Shaded}

It will create a folder named \texttt{\ silver-dev-cv\ } with the
following structure:

\begin{Shaded}
\begin{Highlighting}[]
\NormalTok{silver{-}dev{-}cv}
\NormalTok{└── cv.typ}
\end{Highlighting}
\end{Shaded}

\subsubsection{Typst website}\label{typst-website}

If you want to use the template via \href{https://typst.app/}{Typst} ,
You can \texttt{\ start\ from\ template\ } and search for
\texttt{\ silver-dev-cv\ } .

\subsection{How to use the template}\label{how-to-use-the-template}

\subsubsection{The arguments}\label{the-arguments}

The template has the following arguments:

\begin{longtable}[]{@{}lll@{}}
\toprule\noalign{}
Argument & Description & Default \\
\midrule\noalign{}
\endhead
\bottomrule\noalign{}
\endlastfoot
\texttt{\ font-type\ } & The font type. You can choose any supported
font in your system. & \texttt{\ Times\ New\ Roman\ } \\
\texttt{\ continue-header\ } & Whether to continue the header on the
follwing pages. & \texttt{\ false\ } \\
\texttt{\ name\ } & Your name. & \texttt{\ ""\ } \\
\texttt{\ address\ } & Your address. & \texttt{\ ""\ } \\
\texttt{\ lastupdated\ } & Whether to show the last updated date. &
\texttt{\ true\ } \\
\texttt{\ pagecount\ } & Whether to show the page count. &
\texttt{\ true\ } \\
\texttt{\ date\ } & The date of the CV. & \texttt{\ today\ } \\
\texttt{\ contacts\ } & contact details, e.g phone number, email, etc. &
\texttt{\ (text:\ "",\ link:\ "")\ } \\
\end{longtable}

\subsubsection{Starting the CV}\label{starting-the-cv}

\begin{Shaded}
\begin{Highlighting}[]
\NormalTok{\#import "@preview/silver{-}dev{-}cv:1.0.0": *}

\NormalTok{\#show: cv.with(}
\NormalTok{  font{-}type: "PT Serif",}
\NormalTok{  continue{-}header: "false",}
\NormalTok{  name: "",}
\NormalTok{  address: "",}
\NormalTok{  lastupdated: "true",}
\NormalTok{  pagecount: "true",}
\NormalTok{  date: "2024{-}07{-}03",}
\NormalTok{  contacts: (}
\NormalTok{    (text: "08856", link: ""),}
\NormalTok{    (text: "example.com", link: "https://www.example.com"),}
\NormalTok{    (text: "github.com", link: "https://www.github.com"),}
\NormalTok{    (text: "123@example.com", link: "mailto:123@example.com"),}
\NormalTok{  )}
\NormalTok{)}
\end{Highlighting}
\end{Shaded}

\subsubsection{Content}\label{content}

Once you set up the arguments, you can start to add details to your CV /
Resume.

I preset the following functions for you to create different parts:

\begin{longtable}[]{@{}ll@{}}
\toprule\noalign{}
Function & Description \\
\midrule\noalign{}
\endhead
\bottomrule\noalign{}
\endlastfoot
\texttt{\ \#section("Section\ Name")\ } & Start a new section \\
\texttt{\ \#sectionsep\ } & End the section \\
\texttt{\ \#oneline-title-item(title:\ "",\ content:\ "")\ } & Add a
one-line item ( \textbf{Title:} content) \\
\texttt{\ \#oneline-two(entry1:\ "",\ entry2:\ "")\ } & Add a one-line
item with two entries, aligned left and right \\
\texttt{\ \#descript("descriptions")\ } & Add a description for
self-introduction \\
\texttt{\ \#award(award:\ "",\ date:\ "",\ institution:\ "")\ } & Add an
award ( \textbf{award} , \emph{institution} \emph{date} ) \\
\texttt{\ \#education(institution:\ "",\ major:\ "",\ date:\ "",\ institution:\ "",\ core-modules:\ "")\ }
& Add an education experience \\
\texttt{\ \#job(position:\ "",\ institution:\ "",\ location:\ "",\ date:\ "",\ description:\ {[}{]})\ }
& Add a job experience (description is optional) \\
\texttt{\ \#twoline-item(entry1:\ "",\ entry2:\ "",\ entry3:\ "",\ entry4:\ "")\ }
& Two line items, similar to education and job experiences \\
\end{longtable}

\subsection{License}\label{license}

The template is released under the MIT License. For more information,
please refer to the
\href{https://github.com/jxpeng98/Typst-CV-Resume/blob/main/LICENSE}{LICENSE}
file.

\href{/app?template=silver-dev-cv&version=1.0.1}{Create project in app}

\subsubsection{How to use}\label{how-to-use}

Click the button above to create a new project using this template in
the Typst app.

You can also use the Typst CLI to start a new project on your computer
using this command:

\begin{verbatim}
typst init @preview/silver-dev-cv:1.0.1
\end{verbatim}

\includesvg[width=0.16667in,height=0.16667in]{/assets/icons/16-copy.svg}

\subsubsection{About}\label{about}

\begin{description}
\tightlist
\item[Author s :]
Gabriel Benmergui \& Santiago Barraza
\item[License:]
MIT
\item[Current version:]
1.0.1
\item[Last updated:]
November 26, 2024
\item[First released:]
October 31, 2024
\item[Archive size:]
4.13 kB
\href{https://packages.typst.org/preview/silver-dev-cv-1.0.1.tar.gz}{\pandocbounded{\includesvg[keepaspectratio]{/assets/icons/16-download.svg}}}
\item[Categor y :]
\begin{itemize}
\tightlist
\item[]
\item
  \pandocbounded{\includesvg[keepaspectratio]{/assets/icons/16-user.svg}}
  \href{https://typst.app/universe/search/?category=cv}{CV}
\end{itemize}
\end{description}

\subsubsection{Where to report issues?}\label{where-to-report-issues}

This template is a project of Gabriel Benmergui and Santiago Barraza .
You can also try to ask for help with this template on the
\href{https://forum.typst.app}{Forum} .

Please report this template to the Typst team using the
\href{https://typst.app/contact}{contact form} if you believe it is a
safety hazard or infringes upon your rights.

\phantomsection\label{versions}
\subsubsection{Version history}\label{version-history}

\begin{longtable}[]{@{}ll@{}}
\toprule\noalign{}
Version & Release Date \\
\midrule\noalign{}
\endhead
\bottomrule\noalign{}
\endlastfoot
1.0.1 & November 26, 2024 \\
\href{https://typst.app/universe/package/silver-dev-cv/1.0.0/}{1.0.0} &
October 31, 2024 \\
\end{longtable}

Typst GmbH did not create this template and cannot guarantee correct
functionality of this template or compatibility with any version of the
Typst compiler or app.


\title{typst.app/universe/package/polytonoi}

\phantomsection\label{banner}
\section{polytonoi}\label{polytonoi}

{ 0.1.0 }

Renders Roman letters into polytonic Greek.

\phantomsection\label{readme}
A typst package for rendering text into polytonic Greek using a
hopefully-intuitive transliteration scheme.

\subsection{Usage}\label{usage}

First, be sure you include the package at the top of your typst file:

\begin{Shaded}
\begin{Highlighting}[]
\NormalTok{@import "preview/polytonoi@0.1.0: *}
\end{Highlighting}
\end{Shaded}

The package currently exposes one function,
\texttt{\ \#ptgk(\textless{}string\textgreater{})\ } , which will
convert \texttt{\ \textless{}string\textgreater{}\ } into polytonic
Greek text in the same location where the function appears in the typst
document.

For example: \texttt{\ \#ptgk("polu/s")\ } would render: πολÏ?Ï‚

\textbf{NOTE:} Quotation marks within the function call (see above
example) are \textbf{mandatory} , and the code will not work without
them.

Where possible, Greek letters have been linked with their closest Roman
equivalent (e.g. a -\/-\textgreater{} α, b -\/-\textgreater{} β).
Where not possible, I’ve tried to stick to common convention, such as
what is used by the Perseus Project for their transliteration. A couple
letters (ξ and ψ) are made up of two letters ( \texttt{\ ks\ } and
\texttt{\ ps\ } respectively), which the \texttt{\ \#ptgk()\ } function
handles as special cases. See below for the full transliteration scheme.

\paragraph{Additional Usage Notes}\label{additional-usage-notes}

\begin{enumerate}
\tightlist
\item
  Any character that isn’t specifically accounted for (including white
  space, most punctuation, numbers, etc.) is rendered as-is.
\item
  Smooth breathing marks are automatically added to a vowel that begins
  a word, unless that first vowel is followed by another. In this case,
  you’ll need to manually add it to the second vowel.
\end{enumerate}

\subsubsection{Text Formatting}\label{text-formatting}

As of now, the text is processed as a string, which means that any
formatting markup (such as \texttt{\ \_\ } or \texttt{\ *\ } ) is
\textbf{not} included in how the text is rendered, and is instead passed
through and will display literally. To apply any kind of formatting to
the Greek text, the markup or commands must be put outside the text
passed to the function. Compare the following two examples to see how
this works:

\texttt{\ \#ptgk("\_Arxh\textbackslash{}\_")\ } would display as
\_ἈÏ?χὴ\_

whereas

\texttt{\ \_\#ptgk("Arxh\textbackslash{}")\_\ } would display as
\emph{ἈÏ?χὴ}

\subsection{Transliteration Scheme}\label{transliteration-scheme}

\begin{longtable}[]{@{}lll@{}}
\toprule\noalign{}
Roman letter & Greek result & Notes \\
\midrule\noalign{}
\endhead
\bottomrule\noalign{}
\endlastfoot
a & α & \\
b & β & \\
g & γ & \\
d & δ & \\
e & ε & \\
z & ζ & \\
h & η & \\
q & θ & \\
i & ι & \\
k & κ & \\
l & λ & \\
m & μ & \\
n & ν & \\
ks & ξ & \\
o & ο & \\
p & π & \\
r & Ï? & \\
s & σ/ς & Renders as final sigma automatically \\
t & Ï„ & \\
u & Ï & \\
v & φ & \\
x & χ & \\
ps & ψ & \\
w & ω & \\
\end{longtable}

Upper-case letters are handled the same way, just with an upper-case
letter as input. The upper-case versions of the two “combined�
letters (Ξ and Ψ) can be entered either as “KS�/“PS� or
“Ks�/“Ps�.

\subsubsection{Accents \& Breathing
Marks}\label{accents-breathing-marks}

As mentioned above, this package will automatically put a smooth
breathing mark on a vowel that begins a word, unless that vowel is
followed immediately by a second vowel. In that instance, you’ll have
to manually put the smooth breathing mark in its correct place. (This is
to avoid having to code for edge cases, such as where a word starts with
three vowels in a row.) By the same token, rough breathing must always
be entered manually.

\begin{longtable}[]{@{}llll@{}}
\toprule\noalign{}
Input & Greek & Notes & Example \\
\midrule\noalign{}
\endhead
\bottomrule\noalign{}
\endlastfoot
( & rough breathing & Put before the vowel & \texttt{\ (a\ }
-\/-\textgreater{} á¼? \\
) & smooth breathing & Put before the vowel & \texttt{\ )a\ }
-\/-\textgreater{} á¼€ \\
\textbackslash{} & Grave / varia & Put after the vowel &
\texttt{\ a\textbackslash{}\ } -\/-\textgreater{} á½° \\
/ & Acute / oxia / tonos & Put after the vowel & \texttt{\ a/\ }
-\/-\textgreater{} ά \\
= & Tilde / perispomeni & Put after the vowel & \texttt{\ a=\ }
-\/-\textgreater{} ᾶ \\
\textbar{} & Iota subscript & Put after the vowel &
\texttt{\ a\textbar{}\ } -\/-\textgreater{} á¾³ \\
: & Diaresis & Put after the vowel & \texttt{\ i:\ } -\/-\textgreater{}
ÏŠ \\
\end{longtable}

Multiple diacriticals can be added to a vowel, e.g.
\texttt{\ (h\textbar{}\ } -\/-\textgreater{} á¾`

\subsubsection{Punctuation}\label{punctuation}

Most Roman punctuation characters are left unchanged. The exceptions are
\texttt{\ ;\ } (semicolon) and \texttt{\ ?\ } (question mark), which are
rendered as a high dot (·) and the Greek question mark (;)
respectively.

\subsection{Feedback}\label{feedback}

Feedback is welcome, and please don’t hesitate to open an issue if
something doesn’t work. I’ve tried to account for edge cases, but I
certainly can’t guarantee that I’ve found everything.

\subsubsection{How to add}\label{how-to-add}

Copy this into your project and use the import as \texttt{\ polytonoi\ }

\begin{verbatim}
#import "@preview/polytonoi:0.1.0"
\end{verbatim}

\includesvg[width=0.16667in,height=0.16667in]{/assets/icons/16-copy.svg}

Check the docs for
\href{https://typst.app/docs/reference/scripting/\#packages}{more
information on how to import packages} .

\subsubsection{About}\label{about}

\begin{description}
\tightlist
\item[Author :]
Dei Layborer
\item[License:]
GPL-3.0-only
\item[Current version:]
0.1.0
\item[Last updated:]
December 28, 2023
\item[First released:]
December 28, 2023
\item[Archive size:]
15.6 kB
\href{https://packages.typst.org/preview/polytonoi-0.1.0.tar.gz}{\pandocbounded{\includesvg[keepaspectratio]{/assets/icons/16-download.svg}}}
\item[Repository:]
\href{https://github.com/dei-layborer/polytonoi}{GitHub}
\end{description}

\subsubsection{Where to report issues?}\label{where-to-report-issues}

This package is a project of Dei Layborer . Report issues on
\href{https://github.com/dei-layborer/polytonoi}{their repository} . You
can also try to ask for help with this package on the
\href{https://forum.typst.app}{Forum} .

Please report this package to the Typst team using the
\href{https://typst.app/contact}{contact form} if you believe it is a
safety hazard or infringes upon your rights.

\phantomsection\label{versions}
\subsubsection{Version history}\label{version-history}

\begin{longtable}[]{@{}ll@{}}
\toprule\noalign{}
Version & Release Date \\
\midrule\noalign{}
\endhead
\bottomrule\noalign{}
\endlastfoot
0.1.0 & December 28, 2023 \\
\end{longtable}

Typst GmbH did not create this package and cannot guarantee correct
functionality of this package or compatibility with any version of the
Typst compiler or app.


\title{typst.app/universe/package/chordx}

\phantomsection\label{banner}
\section{chordx}\label{chordx}

{ 0.5.0 }

A package to write song lyrics with chord diagrams in Typst.

\phantomsection\label{readme}
A package to write song lyrics with chord diagrams in Typst.

\textbf{Table of Contents}

\begin{itemize}
\tightlist
\item
  \href{https://github.com/typst/packages/raw/main/packages/preview/chordx/0.5.0/\#introduction}{Introduction}
\item
  \href{https://github.com/typst/packages/raw/main/packages/preview/chordx/0.5.0/\#usage}{Usage}

  \begin{itemize}
  \tightlist
  \item
    \href{https://github.com/typst/packages/raw/main/packages/preview/chordx/0.5.0/\#typst-packages}{Typst
    Packages}
  \item
    \href{https://github.com/typst/packages/raw/main/packages/preview/chordx/0.5.0/\#local-packages}{Local
    Packages}
  \end{itemize}
\item
  \href{https://github.com/typst/packages/raw/main/packages/preview/chordx/0.5.0/\#documentation}{Documentation}
\item
  \href{https://github.com/typst/packages/raw/main/packages/preview/chordx/0.5.0/\#examples}{Examples}

  \begin{itemize}
  \tightlist
  \item
    \href{https://github.com/typst/packages/raw/main/packages/preview/chordx/0.5.0/\#chart-chords}{Chart
    Chords}
  \item
    \href{https://github.com/typst/packages/raw/main/packages/preview/chordx/0.5.0/\#piano-chords}{Piano
    Chords}
  \item
    \href{https://github.com/typst/packages/raw/main/packages/preview/chordx/0.5.0/\#single-chords}{Single
    Chords}
  \end{itemize}
\item
  \href{https://github.com/typst/packages/raw/main/packages/preview/chordx/0.5.0/\#changelog}{Changelog}
\item
  \href{https://github.com/typst/packages/raw/main/packages/preview/chordx/0.5.0/\#license}{License}
\end{itemize}

\subsection{Introduction}\label{introduction}

With \texttt{\ chordx\ } you can easily generate song lyrics with chords
for writing songbooks.

\texttt{\ chordx\ } generates chord charts for stringed instruments
(e.g. guitar, ukulele, etc.), piano chords (with diferent piano layouts)
and single chords that are chords without charts used to write the
chords over a word to write songbooks.

\subsection{Usage}\label{usage}

\texttt{\ chordx\ } exports 3 functions to generate diferents types fo
charts:

\begin{itemize}
\tightlist
\item
  \texttt{\ chart-chord\ } : used to generate chart chords for stringed
  instruments.
\item
  \texttt{\ piano-chord\ } : used to generate piano chords.
\item
  \texttt{\ single-chord\ } : used to show the chord name over a word.
\end{itemize}

\subsubsection{Typst Packages}\label{typst-packages}

Typst added an experimental package repository and you can import
\texttt{\ chordx\ } as follows:

\begin{Shaded}
\begin{Highlighting}[]
\NormalTok{\#import "@preview/chordx:0.5.0": *}
\end{Highlighting}
\end{Shaded}

\subsubsection{Local Packages}\label{local-packages}

If the package hasn’t been released yet, or if you just want to use it
from this repository, you can use \emph{\emph{local-packages}} .

You can read the documentation about typst
\href{https://github.com/typst/packages\#local-packages}{local-packages}
and learn about the path folders used in differents operating systems
(Linux / MacOS / Windows).

In Linux you can do:

\begin{Shaded}
\begin{Highlighting}[]
\FunctionTok{git}\NormalTok{ clone https://github.com/ljgago/typst{-}chords \textasciitilde{}/.local/share/typst/packages/local/chordx/0.5.0}
\end{Highlighting}
\end{Shaded}

And import the package in your file:

\begin{Shaded}
\begin{Highlighting}[]
\NormalTok{\#import "@local/chordx:0.5.0": *}
\end{Highlighting}
\end{Shaded}

\subsection{Documentation}\label{documentation}

Here
\href{https://github.com/ljgago/typst-chords/blob/v0.5.0/docs/chordx-docs.pdf}{chordx-docs}
you have the reference documentation that describes the functions and
parameters used in this package. ( \emph{Generated with
\href{https://github.com/Mc-Zen/tidy}{tidy}} )

\subsection{Examples:}\label{examples}

\subsubsection{Chart Chords}\label{chart-chords}

\begin{Shaded}
\begin{Highlighting}[]
\NormalTok{\#import "@preview/chordx:0.5.0": chart{-}chord}

\NormalTok{\#let chart{-}chord{-}sharp = chart{-}chord.with(size: 18pt)}
\NormalTok{\#let chart{-}chord{-}round = chart{-}chord.with(size: 1.5em, design: "round")}

\NormalTok{// Design "sharp"}
\NormalTok{\#chart{-}chord{-}sharp(tabs: "x32o1o", fingers: "n32n1n")[C]}
\NormalTok{\#chart{-}chord{-}sharp(tabs: "ooo3", fingers: "ooo3")[C]}

\NormalTok{// Desigh "round" with position "bottom"}
\NormalTok{\#chart{-}chord{-}round(tabs: "xn332n", fingers: "o13421", fret: 3, capos: "115", position: "bottom")[Cm]}
\NormalTok{\#chart{-}chord{-}round(tabs: "onnn", fingers: "n111", capos: "313", position: "bottom")[Cm]}

\NormalTok{// Design "round" with background color in chord name}
\NormalTok{\#chart{-}chord{-}round(tabs: "xn332n", fingers: "o13421", fret: 3, capos: "115", background: silver)[Cm]}
\NormalTok{\#chart{-}chord{-}round(tabs: "onnn", fingers: "n111", capos: "313", background: silver)[Cm]}
\end{Highlighting}
\end{Shaded}

\subsubsection{\texorpdfstring{\href{https://github.com/ljgago/typst-chords/blob/v0.5.0/examples/chart-chords.typ}{\protect\pandocbounded{\includesvg[keepaspectratio]{https://raw.githubusercontent.com/ljgago/typst-chords/v0.5.0/examples/chart-chords.svg}}}}{Chart Chord}}\label{chart-chord}

\subsubsection{Piano Chords}\label{piano-chords}

\begin{Shaded}
\begin{Highlighting}[]
\NormalTok{\#import "@preview/chordx:0.5.0": piano{-}chord}

\NormalTok{\#let piano{-}chord{-}sharp = piano{-}chord.with(layout: "F", size: 18pt)}
\NormalTok{\#let piano{-}chord{-}round = piano{-}chord.with(layout: "F", size: 1.5em, design: "round")}

\NormalTok{\#piano{-}chord{-}sharp(keys: "B1, D2\#, F2\#", fill{-}key: blue)[B]}
\NormalTok{\#piano{-}chord{-}round(keys: "B1, D2\#, F2\#", fill{-}key: yellow, position: "bottom")[B]}
\NormalTok{\#piano{-}chord{-}round(keys: "B1, D2\#, F2\#", fill{-}key: red)[B]}
\end{Highlighting}
\end{Shaded}

\subsubsection{\texorpdfstring{\href{https://github.com/ljgago/typst-chords/blob/v0.5.0/examples/piano-chords.typ}{\protect\pandocbounded{\includesvg[keepaspectratio]{https://raw.githubusercontent.com/ljgago/typst-chords/v0.5.0/examples/piano-chords.svg}}}}{Piano Chord}}\label{piano-chord}

\subsubsection{Single Chords}\label{single-chords}

\begin{Shaded}
\begin{Highlighting}[]
\NormalTok{\#import "@preview/chordx:0.5.0": single{-}chord}

\NormalTok{\#let chord = single{-}chord.with(}
\NormalTok{  font: "PT Sans",}
\NormalTok{  size: 12pt,}
\NormalTok{  weight: "semibold",}
\NormalTok{  background: silver}
\NormalTok{)}

\NormalTok{\#chord[Jingle][G][2] bells, jingle bells, jingle \#chord[all][C][2] the \#chord[way!][G][2] \textbackslash{}}
\NormalTok{\#chord[Oh][C][] what fun it \#chord[is][G][] to ride \textbackslash{}}
\NormalTok{In a \#chord[one{-}horse][A7][2] open \#chord[sleigh,][D7][3] hey!}
\end{Highlighting}
\end{Shaded}

\subsection{\texorpdfstring{\href{https://github.com/ljgago/typst-chords/blob/v0.5.0/examples/single-chords.typ}{\protect\pandocbounded{\includesvg[keepaspectratio]{https://raw.githubusercontent.com/ljgago/typst-chords/v0.5.0/examples/single-chords.svg}}}}{Single Chord}}\label{single-chord}

\subsection{Changelog}\label{changelog}

You can read the latest changes in
\href{https://github.com/typst/packages/raw/main/packages/preview/chordx/0.5.0/CHANGELOG.md}{CHANGELOG.md}

\subsection{License}\label{license}

\href{https://github.com/typst/packages/raw/main/packages/preview/chordx/0.5.0/LICENSE}{MIT
License}

\subsubsection{How to add}\label{how-to-add}

Copy this into your project and use the import as \texttt{\ chordx\ }

\begin{verbatim}
#import "@preview/chordx:0.5.0"
\end{verbatim}

\includesvg[width=0.16667in,height=0.16667in]{/assets/icons/16-copy.svg}

Check the docs for
\href{https://typst.app/docs/reference/scripting/\#packages}{more
information on how to import packages} .

\subsubsection{About}\label{about}

\begin{description}
\tightlist
\item[Author :]
\href{https://github.com/ljgago}{Leonardo Gago}
\item[License:]
MIT
\item[Current version:]
0.5.0
\item[Last updated:]
November 4, 2024
\item[First released:]
July 17, 2023
\item[Minimum Typst version:]
0.12.0
\item[Archive size:]
10.3 kB
\href{https://packages.typst.org/preview/chordx-0.5.0.tar.gz}{\pandocbounded{\includesvg[keepaspectratio]{/assets/icons/16-download.svg}}}
\item[Repository:]
\href{https://github.com/ljgago/typst-chords}{GitHub}
\end{description}

\subsubsection{Where to report issues?}\label{where-to-report-issues}

This package is a project of Leonardo Gago . Report issues on
\href{https://github.com/ljgago/typst-chords}{their repository} . You
can also try to ask for help with this package on the
\href{https://forum.typst.app}{Forum} .

Please report this package to the Typst team using the
\href{https://typst.app/contact}{contact form} if you believe it is a
safety hazard or infringes upon your rights.

\phantomsection\label{versions}
\subsubsection{Version history}\label{version-history}

\begin{longtable}[]{@{}ll@{}}
\toprule\noalign{}
Version & Release Date \\
\midrule\noalign{}
\endhead
\bottomrule\noalign{}
\endlastfoot
0.5.0 & November 4, 2024 \\
\href{https://typst.app/universe/package/chordx/0.4.0/}{0.4.0} & July
10, 2024 \\
\href{https://typst.app/universe/package/chordx/0.3.0/}{0.3.0} & March
3, 2024 \\
\href{https://typst.app/universe/package/chordx/0.2.0/}{0.2.0} &
September 3, 2023 \\
\href{https://typst.app/universe/package/chordx/0.1.0/}{0.1.0} & July
17, 2023 \\
\end{longtable}

Typst GmbH did not create this package and cannot guarantee correct
functionality of this package or compatibility with any version of the
Typst compiler or app.


\title{typst.app/universe/package/classy-german-invoice}

\phantomsection\label{banner}
\phantomsection\label{template-thumbnail}
\pandocbounded{\includegraphics[keepaspectratio]{https://packages.typst.org/preview/thumbnails/classy-german-invoice-0.3.0-small.webp}}

\section{classy-german-invoice}\label{classy-german-invoice}

{ 0.3.0 }

Minimalistic invoice for Germany-based freelancers

{ } Featured Template

\href{/app?template=classy-german-invoice&version=0.3.0}{Create project
in app}

\phantomsection\label{readme}
A template for writing invoices, inspired by the
\href{https://github.com/mrzool/invoice-boilerplate/}{beautiful LaTeX
template by @mrzool.}

\begin{Shaded}
\begin{Highlighting}[]
\NormalTok{\#import "@preview/classy{-}german{-}invoice:0.3.0": invoice}

\NormalTok{\#show: invoice(}
\NormalTok{  // Invoice number}
\NormalTok{  "2023{-}001",}
\NormalTok{  // Invoice date}
\NormalTok{  datetime(year: 2024, month: 09, day: 03),}
\NormalTok{  // Items}
\NormalTok{  (}
\NormalTok{    (}
\NormalTok{      description: "The first service provided. The first service provided. The first service provided",}
\NormalTok{      price: 200,}
\NormalTok{    ),}
\NormalTok{    (}
\NormalTok{      description: "The second service provided",}
\NormalTok{      price: 150.2}
\NormalTok{    ),}
\NormalTok{  ),}
\NormalTok{  // Author}
\NormalTok{  (}
\NormalTok{    name: "Kerstin Humm",}
\NormalTok{    street: "Straße der Privatsphäre und Stille 1",}
\NormalTok{    zip: "54321",}
\NormalTok{    city: "Potsdam",}
\NormalTok{    tax\_nr: "12345/67890",}
\NormalTok{    // optional signature, can be omitted}
\NormalTok{    signature: image("example\_signature.png", width: 5em)}
\NormalTok{  ),}
\NormalTok{  // Recipient}
\NormalTok{  (}
\NormalTok{    name: "Erika Mustermann",}
\NormalTok{    street: "Musterallee",}
\NormalTok{    zip: "12345",}
\NormalTok{    city: "Musterstadt",}
\NormalTok{  ),}
\NormalTok{  // Bank account}
\NormalTok{  (}
\NormalTok{    name: "Todd Name",}
\NormalTok{    bank: "Deutsche Postbank AG",}
\NormalTok{    iban: "DE89370400440532013000",}
\NormalTok{    bic: "PBNKDEFF",}
\NormalTok{    // There is currently only one gendered term in this template.}
\NormalTok{    // You can overwrite it, or omit it and just choose the default.}
\NormalTok{    gender: (account\_holder: "Kontoinhaberin")}
\NormalTok{  ),}
\NormalTok{  // Umsatzsteuersatz (VAT)}
\NormalTok{  vat: 0.19,}
\NormalTok{  kleinunternehmer: true,}
\NormalTok{)}
\end{Highlighting}
\end{Shaded}

\pandocbounded{\includegraphics[keepaspectratio]{https://github.com/typst/packages/raw/main/packages/preview/classy-german-invoice/0.3.0/thumbnail.png}}

\subsection{Scope}\label{scope}

This template should work well for freelancers and small companies in
the german market, that don’t have an existing system in place for
order tracking. Or to put it the other way round; This template is for
people that mostly have to fulfill outside requirements with their
invoices and don’t so much benefit from extensive tracking themselfes.

\subsection{Features}\label{features}

\begin{itemize}
\tightlist
\item
  {[}X{]} multiple invoice items
\item
  {[}X{]} configurable VAT
\item
  {[}X{]} configurable § 19 UStG (Kleinunternehmerregelung) note
\item
  {[}X{]} configurable signature from PNG file
\item
  {[}X{]} employs both lining and old-style number types, depending on
  the application
\item
  {[}X{]} \href{https://en.wikipedia.org/wiki/EPC_QR_code}{EPC QR Code}
  for easier banking transactions
\item
  {[} {]} recipient address is guaranteed to fit in a windowed envolope
  (DIN 5008)
\end{itemize}

\subsection{Disclaimer}\label{disclaimer}

This template doesn’t constitute legal advice. Please check for
yourself wether it fulfills your legal requirements!

\href{/app?template=classy-german-invoice&version=0.3.0}{Create project
in app}

\subsubsection{How to use}\label{how-to-use}

Click the button above to create a new project using this template in
the Typst app.

You can also use the Typst CLI to start a new project on your computer
using this command:

\begin{verbatim}
typst init @preview/classy-german-invoice:0.3.0
\end{verbatim}

\includesvg[width=0.16667in,height=0.16667in]{/assets/icons/16-copy.svg}

\subsubsection{About}\label{about}

\begin{description}
\tightlist
\item[Author :]
\href{https://github.com/erictapen}{Kerstin Humm}
\item[License:]
MIT-0
\item[Current version:]
0.3.0
\item[Last updated:]
September 18, 2024
\item[First released:]
September 11, 2024
\item[Minimum Typst version:]
0.10.0
\item[Archive size:]
23.9 kB
\href{https://packages.typst.org/preview/classy-german-invoice-0.3.0.tar.gz}{\pandocbounded{\includesvg[keepaspectratio]{/assets/icons/16-download.svg}}}
\item[Repository:]
\href{https://github.com/erictapen/typst-invoice}{GitHub}
\item[Categor ies :]
\begin{itemize}
\tightlist
\item[]
\item
  \pandocbounded{\includesvg[keepaspectratio]{/assets/icons/16-layout.svg}}
  \href{https://typst.app/universe/search/?category=layout}{Layout}
\item
  \pandocbounded{\includesvg[keepaspectratio]{/assets/icons/16-envelope.svg}}
  \href{https://typst.app/universe/search/?category=office}{Office}
\end{itemize}
\end{description}

\subsubsection{Where to report issues?}\label{where-to-report-issues}

This template is a project of Kerstin Humm . Report issues on
\href{https://github.com/erictapen/typst-invoice}{their repository} .
You can also try to ask for help with this template on the
\href{https://forum.typst.app}{Forum} .

Please report this template to the Typst team using the
\href{https://typst.app/contact}{contact form} if you believe it is a
safety hazard or infringes upon your rights.

\phantomsection\label{versions}
\subsubsection{Version history}\label{version-history}

\begin{longtable}[]{@{}ll@{}}
\toprule\noalign{}
Version & Release Date \\
\midrule\noalign{}
\endhead
\bottomrule\noalign{}
\endlastfoot
0.3.0 & September 18, 2024 \\
\href{https://typst.app/universe/package/classy-german-invoice/0.2.0/}{0.2.0}
& September 11, 2024 \\
\end{longtable}

Typst GmbH did not create this template and cannot guarantee correct
functionality of this template or compatibility with any version of the
Typst compiler or app.


\title{typst.app/universe/package/springer-spaniel}

\phantomsection\label{banner}
\phantomsection\label{template-thumbnail}
\pandocbounded{\includegraphics[keepaspectratio]{https://packages.typst.org/preview/thumbnails/springer-spaniel-0.1.0-small.webp}}

\section{springer-spaniel}\label{springer-spaniel}

{ 0.1.0 }

A loose recreation of the Springer Contributed Chapter template on
Overleaf

{ } Featured Template

\href{/app?template=springer-spaniel&version=0.1.0}{Create project in
app}

\phantomsection\label{readme}
Version 0.1.0

This is an loose recreation of the \emph{Springer Contributed Chapter}
LaTeX template on Overleaf. It aims to provide template-level support
for commonly used packages so you don’t have to choose between style
and features.

\subsection{Media}\label{media}

\includegraphics[width=0.32\linewidth,height=\textheight,keepaspectratio]{https://github.com/typst/packages/raw/main/packages/preview/springer-spaniel/0.1.0/thumbnails/1.png}
\includegraphics[width=0.32\linewidth,height=\textheight,keepaspectratio]{https://github.com/typst/packages/raw/main/packages/preview/springer-spaniel/0.1.0/thumbnails/2.png}
\includegraphics[width=0.32\linewidth,height=\textheight,keepaspectratio]{https://github.com/typst/packages/raw/main/packages/preview/springer-spaniel/0.1.0/thumbnails/3.png}

\subsection{Getting Started}\label{getting-started}

These instructions will get you a copy of the project up and running on
the typst web app. Perhaps a short code example on importing the package
and a very simple teaser usage.

\begin{Shaded}
\begin{Highlighting}[]
\NormalTok{\#import "@preview/springer{-}spaniel:0.1.0"}
\NormalTok{\#import springer{-}spaniel.ctheorems: * // provides "proof", "theorem", "lemma"}

\NormalTok{\#show: springer{-}spaniel.template(}
\NormalTok{  title: [Contribution Title],}
\NormalTok{  authors: (}
\NormalTok{    (}
\NormalTok{      name: "Name of First Author",}
\NormalTok{      institute: "Name",}
\NormalTok{      address: "Address of Institute",}
\NormalTok{      email: "name@email.address"}
\NormalTok{    ),}
\NormalTok{    // ... and so on}
\NormalTok{  ),}
\NormalTok{  abstract: lorem(75),}

\NormalTok{  // debug: true, // Highlights structural elements and links}
\NormalTok{  // frame: 1pt, // A border around the page for white on white display}
\NormalTok{  // printer{-}test: true, // Suitably placed CMYK printer tests}
\NormalTok{)}

\NormalTok{= Section Heading}
\NormalTok{== Subsection Heading}
\NormalTok{=== Subsubsection Heading}
\NormalTok{==== Paragraph Heading}
\NormalTok{===== Subparagraph Heading}
\end{Highlighting}
\end{Shaded}

\subsubsection{Local Installation}\label{local-installation}

To install this project locally, follow the steps below;

\begin{itemize}
\tightlist
\item
  Install Just
\item
  Clone repository
\item
  In a bash compatible shell, \texttt{\ just\ install-preview\ }
\end{itemize}

\href{/app?template=springer-spaniel&version=0.1.0}{Create project in
app}

\subsubsection{How to use}\label{how-to-use}

Click the button above to create a new project using this template in
the Typst app.

You can also use the Typst CLI to start a new project on your computer
using this command:

\begin{verbatim}
typst init @preview/springer-spaniel:0.1.0
\end{verbatim}

\includesvg[width=0.16667in,height=0.16667in]{/assets/icons/16-copy.svg}

\subsubsection{About}\label{about}

\begin{description}
\tightlist
\item[Author :]
James R. Swift
\item[License:]
Unlicense
\item[Current version:]
0.1.0
\item[Last updated:]
July 16, 2024
\item[First released:]
July 16, 2024
\item[Archive size:]
437 kB
\href{https://packages.typst.org/preview/springer-spaniel-0.1.0.tar.gz}{\pandocbounded{\includesvg[keepaspectratio]{/assets/icons/16-download.svg}}}
\item[Repository:]
\href{https://github.com/JamesxX/springer-spaniel}{GitHub}
\item[Discipline s :]
\begin{itemize}
\tightlist
\item[]
\item
  \href{https://typst.app/universe/search/?discipline=chemistry}{Chemistry}
\item
  \href{https://typst.app/universe/search/?discipline=physics}{Physics}
\item
  \href{https://typst.app/universe/search/?discipline=mathematics}{Mathematics}
\end{itemize}
\item[Categor ies :]
\begin{itemize}
\tightlist
\item[]
\item
  \pandocbounded{\includesvg[keepaspectratio]{/assets/icons/16-docs.svg}}
  \href{https://typst.app/universe/search/?category=book}{Book}
\item
  \pandocbounded{\includesvg[keepaspectratio]{/assets/icons/16-speak.svg}}
  \href{https://typst.app/universe/search/?category=report}{Report}
\end{itemize}
\end{description}

\subsubsection{Where to report issues?}\label{where-to-report-issues}

This template is a project of James R. Swift . Report issues on
\href{https://github.com/JamesxX/springer-spaniel}{their repository} .
You can also try to ask for help with this template on the
\href{https://forum.typst.app}{Forum} .

Please report this template to the Typst team using the
\href{https://typst.app/contact}{contact form} if you believe it is a
safety hazard or infringes upon your rights.

\phantomsection\label{versions}
\subsubsection{Version history}\label{version-history}

\begin{longtable}[]{@{}ll@{}}
\toprule\noalign{}
Version & Release Date \\
\midrule\noalign{}
\endhead
\bottomrule\noalign{}
\endlastfoot
0.1.0 & July 16, 2024 \\
\end{longtable}

Typst GmbH did not create this template and cannot guarantee correct
functionality of this template or compatibility with any version of the
Typst compiler or app.


\title{typst.app/universe/package/modern-sustech-thesis}

\phantomsection\label{banner}
\phantomsection\label{template-thumbnail}
\pandocbounded{\includegraphics[keepaspectratio]{https://packages.typst.org/preview/thumbnails/modern-sustech-thesis-0.1.1-small.webp}}

\section{modern-sustech-thesis}\label{modern-sustech-thesis}

{ 0.1.1 }

å?---æ--¹ç§`技大学本ç§`毕业设计(论æ--‡ï¼‰æ¨¡æ?¿. SUSTech
Bachelor Thesis Template.

\href{/app?template=modern-sustech-thesis&version=0.1.1}{Create project
in app}

\phantomsection\label{readme}
功能需求ã€?å?ˆä½œå¼€å?{}`请移步模æ?¿å¯¹åº''çš„ github ä»``åº``:
\href{https://github.com/Duolei-Wang/modern-sustech-thesis}{modern-sustech-thesis}
.

\subsection{typst.app ç½`页版使ç''¨è¯´æ˜Ž (Use
online)}\label{typst.app-uxe7uxbduxe9uxb5uxe7ux2c6uxe4uxbduxe7uxe8uxe6ux17e-use-online}

使ç''¨æ­¥éª¤ï¼š

\begin{itemize}
\item
  æ‰``å¼€ typst.app 从模æ?¿æ--°å»ºé¡¹ç›®ï¼ˆstart from template)
\item
  论æ--‡æ‰€éœ€å­---ä½``需è¦?手动上ä¼~到ä½~的项目æ--‡ä»¶åˆ---表.

  点击左侧 Explore
  Files,上ä¼~å­---ä½``æ--‡ä»¶ï¼Œä¸Šä¼~å?Žçš„å­---ä½``æ--‡ä»¶å­˜å‚¨ä½?置没有特殊è¦?求,typst
  拥有优秀的å†\ldots æ~¸ï¼Œå?¯ä»¥å®Œæˆ?自动æ?œç´¢.

  ç''±äºŽæ~¼å¼?渲æŸ``引æ``Žçš„æ~¸å¿ƒéœ€è¦?指定å­---ä½``çš„å??称,æˆ`在模æ?¿æµ‹è¯•é˜¶æ®µä½¿ç''¨äº†è‹¥å¹²æ~‡å‡†å­---ä½``,这些å­---ä½``å?¯ä»¥åœ¨æˆ`çš„
  github ä»``åº``
  \href{https://github.com/Duolei-Wang/modern-sustech-thesis}{modern-sustech-thesis}
  /template/fonts 里找到.

  æ­¤å¤--,å?¯ä»¥æ‰‹åŠ¨æ›´æ''¹å­---ä½``é\ldots?置,在正æ--‡å‰?使ç''¨
  ‘\#set’
  å`½ä»¤å?³å?¯ï¼Œç''±äºŽæ~‡é¢˜ã€?æ­£æ--‡å­---ä½``ä¸?å?Œï¼Œæ­¤å¤„大致语法如下:
\end{itemize}

\begin{Shaded}
\begin{Highlighting}[]
\NormalTok{// headings}
\NormalTok{  show heading.where(level: 1): it =\textgreater{}\{}
\NormalTok{    set text(}
\NormalTok{      font: fonts.HeiTi,}
\NormalTok{      size: fonts.No3,}
\NormalTok{      weight: "regular",}
\NormalTok{    )}
\NormalTok{    align(center)[}
\NormalTok{      // \#it}
\NormalTok{      \#strong(it)}
\NormalTok{    ]}
\NormalTok{    text()[\#v(0.5em)]}
\NormalTok{  \}}

\NormalTok{  show heading.where(level: 2): it =\textgreater{}\{}
\NormalTok{    set text(}
\NormalTok{      font: fonts.HeiTi,}
\NormalTok{      size: fonts.No4,}
\NormalTok{      weight: "regular"}
\NormalTok{      )}
\NormalTok{    it}
\NormalTok{    text()[\#v(0.5em)]}
\NormalTok{  \}}

\NormalTok{  show heading.where(level: 3): it =\textgreater{}\{}
\NormalTok{    set text(}
\NormalTok{      font: fonts.HeiTi,}
\NormalTok{      size: fonts.No4{-}Small,}
\NormalTok{      weight: "regular"}
\NormalTok{      )}
\NormalTok{    it}
\NormalTok{    text()[\#v(0.5em)] }
\NormalTok{  \}}

\NormalTok{  // paragraph}
\NormalTok{  set block(spacing: 1.5em)}
\NormalTok{  set par(}
\NormalTok{    justify: true,}
\NormalTok{    first{-}line{-}indent: 2em,}
\NormalTok{    leading: 1.5em)}
\end{Highlighting}
\end{Shaded}

headings
设定了å?„个登记æ~‡é¢˜çš„æ~¼å¼?,å\ldots¶ä¸­ä¸€çº§æ~‡é¢˜éœ€è¦?å±\ldots 中对é½?.
‘font: fonts.HeiTi’
å?³ä¸ºå­---ä½``çš„å\ldots³é''®å?‚数,å?‚数的值是å­---ä½``çš„å??称(å­---符串).
typst 将会在ç¼--è¯`器å†\ldots æ~¸ã€?项目目录中æ?œç´¢. typst
å†\ldots æ~¸è‡ªå¸¦äº† Source Sans(é»`ä½``)å'Œ Source
Serif(宋ä½``)系åˆ---,但是中æ--‡è®ºæ--‡æ‰€éœ€çš„仿宋ã€?楷ä½``ä»?需自己上ä¼~.

按ç\ldots§æ¯•ä¸šè®¾è®¡è¦?求,以 markdown
æ~¼å¼?书写ä½~的毕业论æ--‡ï¼Œå?ªéœ€è¦?:

\begin{itemize}
\item
  在 configs/info 里填å\ldots¥ä¸ªäººä¿¡æ?¯.
  如有æ~‡é¢˜ç¼--è¯`é''™è¯¯ï¼ˆæ¯''如æˆ`默认了有三行æ~‡é¢˜ï¼‰ï¼Œå?¯ä»¥è‡ªè¡ŒæŒ‰ç\ldots§ç¼--è¯`器æ??示把相å\ldots³ä»£ç~?注释æˆ--è€\ldots ä¿®æ''¹.
  大ä½``语法å'Œå†\ldots 容与基本的ç¼--程语言æ---~差别.
\item
  在 content.typ 里以 typst 特定的 markdown
  语法书写ä½~的论æ--‡å†\ldots 容. 有å\ldots³ typst 中 markdown
  的语法å?˜æ›´ï¼Œä¸ªäººè®¤ä¸ºçš„主è¦?å?˜åŒ--ç½---åˆ---如下:

  \begin{itemize}
  \tightlist
  \item
    æ~‡é¢˜æ~?使ç''¨ ‘=’ 而é?ž ‘\#’,‘\#’ 在 typst
    里是å®?å`½ä»¤çš„开头.
  \item
    æ•°å­¦å\ldots¬å¼?ä¸?需è¦?å??æ--œæ?~,数学符å?·å?¯ä»¥æŸ¥é˜\ldots :
    \url{https://typst.app/docs/reference/symbols/sym/} .
    值å¾---注æ„?的是,typst
    中语法ä¸?通过å?~åŠ~çš„æ--¹å¼?实现,如 “ä¸?ç­‰å?·â€? 在
    LaTex 中是 ‘\textbackslash not\{=\}’. 而在 typst 中,使ç''¨
    ‘eq.not’ çš„æ--¹å¼?æ?¥è°ƒç''¨ ‘eq’(等å?·ï¼‰çš„
    ‘not’(ä¸?等)å?˜ä½``实现.
  \item
    引ç''¨æ~‡ç­¾é‡‡ç''¨ ‘@label’ æ?¥å®žçŽ°ï¼Œè‡ªå®šä¹‰æ~‡ç­¾é€šè¿‡
    ‘’ �实现. 对于 BibTex
    æ~¼å¼?的引ç''¨ï¼ˆrefer.bib),与 LaTex
    �路相�,第一个缩略�将会被认定为 label.
  \end{itemize}
\item
  自定义æ~¼å¼?çš„æ€?è·¯.
  如有é¢?å¤--的需è¦?自定义æ~¼å¼?的需求,å?¯ä»¥è‡ªè¡Œå­¦ä¹~
  ‘\#set’, ‘\#show’
  å`½ä»¤ï¼Œè¿™å?¯èƒ½éœ€è¦?一定的ç¼--程语言知识,å?Žç»­æˆ`会更æ--°éƒ¨åˆ†ç®€ç•¥æ•™ç¨‹åœ¨æˆ`çš„
  github ä»``åº``里: \url{https://github.com/Duolei-Wang/lang-typst}
  .
\item
  本模æ?¿çš„ç»``æž„

  \begin{enumerate}
  \tightlist
  \item
    å†\ldots 容主ä½``. æ--‡ç«~主ä½``å†\ldots 容书写在 content.typ
    æ--‡ä»¶ä¸­ï¼Œé™„录部分书写在 appendix.typ æ--‡ä»¶ä¸­.
  \item
    å†\ldots 容顺åº?. æ--‡ç«~å†\ldots 容顺åº?ç''± main.typ
    决定,通过 typst 中 ‘\#include’
    指令实现了页é?¢çš„æ?'å\ldots¥.
  \item
    å†\ldots 容æ~¼å¼?. å†\ldots 容æ~¼å¼?ç''± /sections/*.typ
    控制,body.typ
    控制了æ--‡ç«~主ä½``çš„æ~¼å¼?,å\ldots¶ä½™ä¸Žå??称一致. cover
    为��,commitment 为承诺书,outline 为目录,abstract
    为æ`˜è¦?.
  \end{enumerate}
\end{itemize}

版本�:0.1.1

\begin{itemize}
\tightlist
\item
  Fixed the fatal bug.
  修正了å?‚æ•°ä¼~é€'失败é€~æˆ?çš„å°?é?¢ç­‰é¡µé?¢æ---~法正常更æ''¹ä¿¡æ?¯.
\end{itemize}

TODO:

\begin{itemize}
\tightlist
\item
  {[} {]} 引ç''¨æ~¼å¼? check.
\end{itemize}

å?---æ--¹ç§`技大学本ç§`毕业设计(论æ--‡ï¼‰æ¨¡æ?¿ï¼Œè®ºæ--‡æ~¼å¼?å?‚ç\ldots§
\href{https://tao.sustech.edu.cn/studentService/graduation_project.html}{å?---æ--¹ç§`技大学本ç§`ç''Ÿæ¯•ä¸šè®¾è®¡ï¼ˆè®ºæ--‡ï¼‰æ'°å†™è§„范}
.
如有ç--?æ¼?敬请è°\ldots 解,本模æ?¿ä¸ºæœ¬äººæ¯•ä¸šä¹‹å‰?自ç''¨ï¼Œå¦‚有使ç''¨ï¼Œç¨³å®šæ€§è¯·è‡ªè¡Œè´Ÿè´£.

\begin{itemize}
\item
  本模�主��考了 \href{https://github.com/iydon}{iydon}
  ä»``åº``çš„çš„ \$\textbackslash LaTeX\$ 模æ?¿
  \href{https://github.com/iydon/sustechthesis}{sustechthesis}
  ï¼›ç»``构组织å?‚ç\ldots§äº†
  \href{https://github.com/shuosc}{shuosc} ä»``åº``çš„
  \href{https://github.com/shuosc/SHU-Bachelor-Thesis-Typst}{SHU-Bachelor-Thesis-Typst}
  模æ?¿ï¼›å›¾ç‰‡ç´~æ??使ç''¨äº†
  \href{https://github.com/GuTaoZi}{GuTaoZi}
  çš„å?Œå†\ldots 容ä»``åº``里的模æ?¿.
\item
  æ„Ÿè°¢
  \href{https://github.com/shuosc/SHU-Bachelor-Thesis-Typst}{SHU-Bachelor-Thesis}
  çš„ç»``构组织让æˆ`å­¦ä¹~到了很多,给æˆ`的页é?¢ç»„织æ??供了ç?µæ„Ÿï¼Œ
\item
  在查找图片ç´~æ??çš„æ---¶å€™ï¼Œä½¿ç''¨äº† GuTaoZi ä»``åº``
  \href{https://github.com/GuTaoZi/SUSTech-thesis-typst}{SUSTech-thesis-typst}
  里的svg ç´~æ??,特此感谢.
\end{itemize}

本模æ?¿ã€?ä»``åº``处于个人安利 typst
的需è¦?â€''â€''在线模æ?¿éœ€ä¸Šä¼~至 typst/packages
官æ--¹ä»``åº``æ‰?能被æ?œç´¢åˆ°ï¼Œå¦‚有开å?{}`å'ŒæŽ¥ç®¡ç­‰éœ€æ±‚请务å¿\ldots è?''ç³»æˆ`:

QQ: 782564506

mail:
\href{mailto:wangdl2020@mail.sustech.edu.cn}{\nolinkurl{wangdl2020@mail.sustech.edu.cn}}

\href{/app?template=modern-sustech-thesis&version=0.1.1}{Create project
in app}

\subsubsection{How to use}\label{how-to-use}

Click the button above to create a new project using this template in
the Typst app.

You can also use the Typst CLI to start a new project on your computer
using this command:

\begin{verbatim}
typst init @preview/modern-sustech-thesis:0.1.1
\end{verbatim}

\includesvg[width=0.16667in,height=0.16667in]{/assets/icons/16-copy.svg}

\subsubsection{About}\label{about}

\begin{description}
\tightlist
\item[Author :]
MuTsingQAQ
\item[License:]
MIT
\item[Current version:]
0.1.1
\item[Last updated:]
April 29, 2024
\item[First released:]
April 17, 2024
\item[Archive size:]
50.6 kB
\href{https://packages.typst.org/preview/modern-sustech-thesis-0.1.1.tar.gz}{\pandocbounded{\includesvg[keepaspectratio]{/assets/icons/16-download.svg}}}
\item[Repository:]
\href{https://github.com/Duolei-Wang/sustech-thesis-typst}{GitHub}
\item[Categor y :]
\begin{itemize}
\tightlist
\item[]
\item
  \pandocbounded{\includesvg[keepaspectratio]{/assets/icons/16-mortarboard.svg}}
  \href{https://typst.app/universe/search/?category=thesis}{Thesis}
\end{itemize}
\end{description}

\subsubsection{Where to report issues?}\label{where-to-report-issues}

This template is a project of MuTsingQAQ . Report issues on
\href{https://github.com/Duolei-Wang/sustech-thesis-typst}{their
repository} . You can also try to ask for help with this template on the
\href{https://forum.typst.app}{Forum} .

Please report this template to the Typst team using the
\href{https://typst.app/contact}{contact form} if you believe it is a
safety hazard or infringes upon your rights.

\phantomsection\label{versions}
\subsubsection{Version history}\label{version-history}

\begin{longtable}[]{@{}ll@{}}
\toprule\noalign{}
Version & Release Date \\
\midrule\noalign{}
\endhead
\bottomrule\noalign{}
\endlastfoot
0.1.1 & April 29, 2024 \\
\href{https://typst.app/universe/package/modern-sustech-thesis/0.1.0/}{0.1.0}
& April 17, 2024 \\
\end{longtable}

Typst GmbH did not create this template and cannot guarantee correct
functionality of this template or compatibility with any version of the
Typst compiler or app.


\title{typst.app/universe/package/idwtet}

\phantomsection\label{banner}
\section{idwtet}\label{idwtet}

{ 0.3.0 }

Package for uniform, correct and simplified typst code demonstration.

\phantomsection\label{readme}
The name \texttt{\ idwtet\ } stands for “I Don’t Wanna Type
Everything Twice�. It provides a \texttt{\ typst-ex\ } and a
\texttt{\ typst-ex-code\ } codeblock, which \emph{shows \textbf{and}
executes} typst code.

It is meant for code demonstration, e.g. when publishing a package, and
provides some niceties:

\begin{itemize}
\tightlist
\item
  the code should always be correct in the examples: As the example code
  is used for the code block, but also for evaluation, there is no need
  to write it twice
\item
  easy configuration with simple, uniform and good look
\end{itemize}

However, there are some limitations:

\begin{itemize}
\tightlist
\item
  Every code block has its own local scope and the default behaviour is
  that variables are not reachable from outside. A similar restriction
  lies on import statements. That is why, there is the
  \texttt{\ eval-scope\ } argument, which captures variables and
  simulates global variables. Additionally, a \texttt{\ typst\ }
  codeblock is provided for a consistent look.
\item
  Locality can be displayed to the users by automatically wrapping code
  in \texttt{\ typst-ex-code\ } , but \texttt{\ typst-ex\ } does not
  provide such functionality. It might thus be difficult for users to
  understand code examples this way.
\item
  The page width has to be defined in absolute terms. It is quite nice,
  for a showcase, to take the least possible space, but tracking the
  widths of all boxes and then setting the page width accordingly is not
  (yet) possible.
\end{itemize}

\subsection{Usage}\label{usage}

Only one function is defined,
\texttt{\ init(body,\ bcolor:\ luma(210),\ inset:\ 5pt,\ border:\ 2pt,\ radius:\ 2pt,\ content-font:\ "linux\ libertine",\ code-font-size:\ 9pt,\ content-font-size:\ 11pt,\ code-return-box:\ true,\ wrap-code:\ false,\ eval-scope:\ (:),\ escape-bracket:\ "\%")\ }
, which is supposed to be used with a show rule.

Then raw codeblocks (with \texttt{\ block=true\ } ) of the languages
\texttt{\ typst\ } , \texttt{\ typst-ex\ } , \texttt{\ typst-code\ } and
\texttt{\ typst-ex-code\ } are modified. The main feature of this
package are \texttt{\ typst-ex\ } and \texttt{\ typst-ex-code\ } . The
\texttt{\ typst\ } and \texttt{\ typst-code\ } blocks do not evaluate
anything, but their design fits that of the others.

The parameters of \texttt{\ init\ } are:

\begin{itemize}
\tightlist
\item
  \texttt{\ body\ } : for usage with show rule, hence the whole
  document.
\item
  \texttt{\ bcolor\ } : the background- (and border-) color of the
  blocks
\item
  \texttt{\ inset\ } : inset param of code and content blocks, should be
  ≥ 2pt
\item
  \texttt{\ border\ } : border thickness
\item
  \texttt{\ radius\ } : block radius
\item
  \texttt{\ content-font\ } : The font used in the previewed content /
  result.
\item
  \texttt{\ code-font-size\ } : The fontsize used in the code blocks.
\item
  \texttt{\ content-font-size\ } : The fontsize used in the preview
  content / result.
\item
  \texttt{\ code-return-box\ } : If to show the code return type on
  \texttt{\ typst-ex-code\ } blocks.
\item
  \texttt{\ wrap-code\ } : If to wrap the code in \texttt{\ \#\{\ } and
  \texttt{\ \}\ } , to symbolize local scope.
\item
  \texttt{\ eval-scope\ } : A dictionary with the keys as the variable
  names and the values as another dictionary with keys
  \texttt{\ value\ } and \texttt{\ code\ } , both of these are optional.
  The former has the defined value, the latter the code recreate the
  variable, for usage in the code blocks.
\item
  \texttt{\ escape-bracket\ } : The text to wrap a variable with, to
  access the \texttt{\ code\ } part of a \texttt{\ eval-scope\ }
  variable.
\end{itemize}

\subsection{Hiding code and
replacements}\label{hiding-code-and-replacements}

There are currently two methods to change the code:

\begin{itemize}
\tightlist
\item
  use the \texttt{\ eval-scope\ } argument from the \texttt{\ init\ }
  function. There is a \texttt{\ code\ } field in the dictionaries,
  which enables the usage of the key escaped in
  \texttt{\ escape-bracket\ } to be replaced in the codeblock code half
  and to be removed in the codeblock result half, as the value is given
  via scope. Take a look at the example below, where
  \texttt{\ \%ouset\%\ } is used this way.
\item
  use the \texttt{\ ENDHIDDEN\ } feature. When escaped in
  \texttt{\ escape-bracket\ } , everything above the statement is
  removed from the codeblock code half BUT everything (without the
  \texttt{\ ENDHIDDEN\ } statement) is evaluated. Take a look at the
  example in the examples folder.
\end{itemize}

\subsection{Example}\label{example}

\begin{Shaded}
\begin{Highlighting}[]
\NormalTok{\#set page(margin: 0.5cm, width: 14cm, height: auto)}
\NormalTok{\#import "@preview/idwtet:0.1.0"}
\NormalTok{\#show: idwtet.init.with(eval{-}scope: (}
\NormalTok{  ouset: (}
\NormalTok{    value: \{import "@preview/ouset:0.1.1": ouset; ouset\},}
\NormalTok{    code: "\#import \textbackslash{}"@preview/ouset:0.1.1\textbackslash{}": ouset"}
\NormalTok{  )}
\NormalTok{))}

\NormalTok{== ouset package \#text(gray)[(v0.1.1)]}
\NormalTok{\textasciigrave{}\textasciigrave{}\textasciigrave{}typst{-}ex}
\NormalTok{\%ouset\%}
\NormalTok{$}
\NormalTok{"Expression 1" ouset(\&, \textless{}==\textgreater{}, "Theorem 1") "Expression 2"\textbackslash{}}
\NormalTok{               ouset(\&, ==\textgreater{},, "Theorem 7") "Expression 3"}
\NormalTok{$}
\NormalTok{\textasciigrave{}\textasciigrave{}\textasciigrave{}}
\NormalTok{Or something like}
\NormalTok{\textasciigrave{}\textasciigrave{}\textasciigrave{}typst{-}ex{-}code}
\NormalTok{let a = range(10)}
\NormalTok{a}
\NormalTok{\textasciigrave{}\textasciigrave{}\textasciigrave{}}
\end{Highlighting}
\end{Shaded}

Further examples are given in the repo example folder.

\subsubsection{How to add}\label{how-to-add}

Copy this into your project and use the import as \texttt{\ idwtet\ }

\begin{verbatim}
#import "@preview/idwtet:0.3.0"
\end{verbatim}

\includesvg[width=0.16667in,height=0.16667in]{/assets/icons/16-copy.svg}

Check the docs for
\href{https://typst.app/docs/reference/scripting/\#packages}{more
information on how to import packages} .

\subsubsection{About}\label{about}

\begin{description}
\tightlist
\item[Author :]
Ludwig Austermann
\item[License:]
MIT
\item[Current version:]
0.3.0
\item[Last updated:]
September 25, 2023
\item[First released:]
August 19, 2023
\item[Minimum Typst version:]
0.8.0
\item[Archive size:]
3.84 kB
\href{https://packages.typst.org/preview/idwtet-0.3.0.tar.gz}{\pandocbounded{\includesvg[keepaspectratio]{/assets/icons/16-download.svg}}}
\item[Repository:]
\href{https://github.com/ludwig-austermann/typst-idwtet}{GitHub}
\end{description}

\subsubsection{Where to report issues?}\label{where-to-report-issues}

This package is a project of Ludwig Austermann . Report issues on
\href{https://github.com/ludwig-austermann/typst-idwtet}{their
repository} . You can also try to ask for help with this package on the
\href{https://forum.typst.app}{Forum} .

Please report this package to the Typst team using the
\href{https://typst.app/contact}{contact form} if you believe it is a
safety hazard or infringes upon your rights.

\phantomsection\label{versions}
\subsubsection{Version history}\label{version-history}

\begin{longtable}[]{@{}ll@{}}
\toprule\noalign{}
Version & Release Date \\
\midrule\noalign{}
\endhead
\bottomrule\noalign{}
\endlastfoot
0.3.0 & September 25, 2023 \\
\href{https://typst.app/universe/package/idwtet/0.2.0/}{0.2.0} & August
19, 2023 \\
\end{longtable}

Typst GmbH did not create this package and cannot guarantee correct
functionality of this package or compatibility with any version of the
Typst compiler or app.


\title{typst.app/universe/package/boxr}

\phantomsection\label{banner}
\section{boxr}\label{boxr}

{ 0.1.0 }

A modular, and easy to use, package for creating cardboard cutouts in
Typst.

\phantomsection\label{readme}
Boxr is a modular, and easy to use, package for creating cardboard
cutouts in Typst.

\subsection{Usage}\label{usage}

Create a boxr structure in your project with the following code:

\begin{verbatim}
#import "@preview/boxr:0.1.0": *

#render-structure(
  "box",
  width: 100pt,
  height: 100pt,
  depth: 100pt,
  tab-size: 20pt,
  [
    Hello from boxr!
  ]
)
\end{verbatim}

The \texttt{\ render-structure\ } function is the main function for
boxr. It either takes a path to one of the default structures provided
by boxr (e.g.: \texttt{\ "box"\ } ) or an unpacked json file with your
own custom structure (e.g.: \texttt{\ json(my-structure.json)\ } ).
These describe the structure of the cutout.\\
The other named arguments depend on the structure you are rendering. All
unnamed arguments are passed to the structure as content and will be
rendered on each box face (not triangles or tabs).

\subsection{Creating your own
structures}\label{creating-your-own-structures}

Structures are defined in \texttt{\ .json\ } files. An example structure
that just shows a box with a tab on one face is shown below:

\begin{verbatim}
{
  "variables": ["height", "width", "tab-size"],
  "width": "width",
  "height": "height + tab-size",
  "offset-x": "",
  "offset-y": "tab-size",
  "root": {
    "type": "box",
    "id": 0,
    "width": "width",
    "height": "height",
    "children": {
      "top": "tab(tab-size, tab-size)"
    }
  }
}
\end{verbatim}

The \texttt{\ variables\ } key is a list of variable names that can be
passed to the structure. These will be required to be passed to the
\texttt{\ render-structure\ } function.\\
The \texttt{\ width\ } and \texttt{\ height\ } keys are evaluated to
calculate the width and height of the structure.\\
The \texttt{\ offset-x\ } and \texttt{\ offset-y\ } keys are evaluated
to place the structure in the middle of its bounds. It is relative to
the root node. In this case for example, the top tab adds a
\texttt{\ tab-size\ } on top of the box as opposed to the bottom, where
there is no tab. So this \texttt{\ tab-size\ } is added to the
\texttt{\ offset-y\ } .\\
\texttt{\ root\ } denotes the first node in the structure.\\
A node can be of the following types:

\begin{itemize}
\tightlist
\item
  \texttt{\ box\ } :

  \begin{itemize}
  \tightlist
  \item
    The root node has a \texttt{\ width\ } and a \texttt{\ height\ } .
    All following nodes have a \texttt{\ size\ } . Child nodes use
    \texttt{\ size\ } and the parent node’s \texttt{\ width\ } and
    \texttt{\ height\ } to calculate their own width and height.
  \item
    Can have \texttt{\ children\ } nodes.
  \item
    Can have an \texttt{\ id\ } key that is used to place content on the
    face of the box. The id-th unnamed argument is placed on the face.
    Multiple faces can have the same id.
  \item
    Can have a \texttt{\ no-fold\ } key. If this exists, no fold stroke
    will be drawn between this box and its parent.
  \end{itemize}
\item
  \texttt{\ triangle-\textless{}left\textbar{}right\textgreater{}\ } :

  \begin{itemize}
  \tightlist
  \item
    Has a \texttt{\ width\ } and \texttt{\ height\ } .
  \item
    \texttt{\ left\ } and \texttt{\ right\ } denote the direction the
    other right angled line is facing relative to the base.
  \item
    Can have \texttt{\ children\ } nodes.
  \item
    Can have a \texttt{\ no-fold\ } key. If this exists, no fold stroke
    will be drawn between this triangle and its parent.
  \end{itemize}
\item
  \texttt{\ tab\ } :

  \begin{itemize}
  \tightlist
  \item
    Is not a json object, but a string that denotes a tab. The tab is
    placed on the parent node.
  \item
    Has a tab-size and a cutin-size inside the \texttt{\ ()\ } separted
    by a \texttt{\ ,\ } .
  \end{itemize}
\item
  \texttt{\ none\ } :

  \begin{itemize}
  \tightlist
  \item
    Is not a json object, but a string that denotes no node. This is
    useful for deleting a cut-stroke between two nodes.
  \end{itemize}
\end{itemize}

Every string value in the json file (
\texttt{\ width:\ "\_\_",\ height:\ "\_\_",\ ...\ offset-x/y:\ "\_\_"\ }
and the values between the \texttt{\ \textbar{}\ } for tabs) is
evaluated as regular typst code. This means that you can use all named
variables passed to the structure. All inputs are converted to points
and the result of the evaluation will be converted back to a length.

\subsubsection{How to add}\label{how-to-add}

Copy this into your project and use the import as \texttt{\ boxr\ }

\begin{verbatim}
#import "@preview/boxr:0.1.0"
\end{verbatim}

\includesvg[width=0.16667in,height=0.16667in]{/assets/icons/16-copy.svg}

Check the docs for
\href{https://typst.app/docs/reference/scripting/\#packages}{more
information on how to import packages} .

\subsubsection{About}\label{about}

\begin{description}
\tightlist
\item[Author :]
\href{https://github.com/Lypsilonx}{Lypsilonx}
\item[License:]
MIT
\item[Current version:]
0.1.0
\item[Last updated:]
May 23, 2024
\item[First released:]
May 23, 2024
\item[Archive size:]
6.23 kB
\href{https://packages.typst.org/preview/boxr-0.1.0.tar.gz}{\pandocbounded{\includesvg[keepaspectratio]{/assets/icons/16-download.svg}}}
\item[Repository:]
\href{https://github.com/Lypsilonx/boxr}{GitHub}
\item[Discipline :]
\begin{itemize}
\tightlist
\item[]
\item
  \href{https://typst.app/universe/search/?discipline=design}{Design}
\end{itemize}
\item[Categor ies :]
\begin{itemize}
\tightlist
\item[]
\item
  \pandocbounded{\includesvg[keepaspectratio]{/assets/icons/16-chart.svg}}
  \href{https://typst.app/universe/search/?category=visualization}{Visualization}
\item
  \pandocbounded{\includesvg[keepaspectratio]{/assets/icons/16-layout.svg}}
  \href{https://typst.app/universe/search/?category=layout}{Layout}
\end{itemize}
\end{description}

\subsubsection{Where to report issues?}\label{where-to-report-issues}

This package is a project of Lypsilonx . Report issues on
\href{https://github.com/Lypsilonx/boxr}{their repository} . You can
also try to ask for help with this package on the
\href{https://forum.typst.app}{Forum} .

Please report this package to the Typst team using the
\href{https://typst.app/contact}{contact form} if you believe it is a
safety hazard or infringes upon your rights.

\phantomsection\label{versions}
\subsubsection{Version history}\label{version-history}

\begin{longtable}[]{@{}ll@{}}
\toprule\noalign{}
Version & Release Date \\
\midrule\noalign{}
\endhead
\bottomrule\noalign{}
\endlastfoot
0.1.0 & May 23, 2024 \\
\end{longtable}

Typst GmbH did not create this package and cannot guarantee correct
functionality of this package or compatibility with any version of the
Typst compiler or app.


\title{typst.app/universe/package/abiding-ifacconf}

\phantomsection\label{banner}
\phantomsection\label{template-thumbnail}
\pandocbounded{\includegraphics[keepaspectratio]{https://packages.typst.org/preview/thumbnails/abiding-ifacconf-0.1.0-small.webp}}

\section{abiding-ifacconf}\label{abiding-ifacconf}

{ 0.1.0 }

An IFAC-style paper template to publish at conferences for International
Federation of Automatic Control

\href{/app?template=abiding-ifacconf&version=0.1.0}{Create project in
app}

\phantomsection\label{readme}
\subsection{(unofficial) IFAC Conference Template for
Typst}\label{unofficial-ifac-conference-template-for-typst}

IFAC stands for \href{https://ifac-control.org/}{International
Federation of Automatic Control} . This repository is meant to be a port
of the existing author tools for conference papers (e.g., for LaTeX, see
\href{https://www.ifac-control.org/conferences/author-guide/copy_of_ifacconf_latex.zip/view}{ifacconf\_latex.zip}
) for Typst.

\subsection{Usage}\label{usage}

Running the following command will create a new directory with all the
files that are needed:

\begin{verbatim}
typst init @preview/abiding-ifacconf
\end{verbatim}

\subsection{Configuration}\label{configuration}

This template exports the \texttt{\ ifacconf\ } function with the
following named arguments:

\begin{itemize}
\tightlist
\item
  \texttt{\ authors\ } : (default: ()) array of authors. For each author
  you can specify a name, email (optional), and affiliation. The
  affiliation must be an integer corresponding to an entry in the
  1-indexed affiliations list (or 0 for no affiliation).
\item
  \texttt{\ affiliations\ } : (default: ()) array of affiliations. For
  each affiliation you can specify a department, organization, and
  address. Everything is optional (i.e., an affiliation can be an empty
  array).
\item
  \texttt{\ abstract\ } : (default: none) the paper’s abstract. Can be
  omitted if you don’t have one.
\item
  \texttt{\ keywords\ } : (default: ()) array of keywords to display
  after the abstract
\item
  \texttt{\ sponsor\ } : (default: none) acknowledgment of sponsor or
  financial support (appears as a footnote on the first page)
\end{itemize}

\subsection{Minimal Working Example}\label{minimal-working-example}

\begin{Shaded}
\begin{Highlighting}[]
\NormalTok{\#import "@preview:abiding{-}ifacconf:0.1.0": *}
\NormalTok{\#show: ifacconf{-}rules}
\NormalTok{\#show: ifacconf.with(}
\NormalTok{  title: "Minimal Working Example",}
\NormalTok{  authors: (}
\NormalTok{    (}
\NormalTok{      name: "First A. Author",}
\NormalTok{      email: "author@boulder.nist.gov",}
\NormalTok{      affiliation: 1,}
\NormalTok{    ),}
\NormalTok{  ),}
\NormalTok{  affiliations: (}
\NormalTok{    (}
\NormalTok{      department: "Engineering",}
\NormalTok{      organization: "National Institute of Standards and Technology",}
\NormalTok{      address: "Boulder, CO 80305 USA",}
\NormalTok{    ),}
\NormalTok{  ),}
\NormalTok{  abstract: [}
\NormalTok{    Abstract should be 50{-}100 words.}
\NormalTok{  ],}
\NormalTok{  keywords: ("keyword1", "keyword2"),}
\NormalTok{  sponsor: [}
\NormalTok{    Sponsor information.}
\NormalTok{  ],}
\NormalTok{)}

\NormalTok{= Introduction}

\NormalTok{A minimum working example (with bibliography) @Abl56.}

\NormalTok{\#lorem(80)}

\NormalTok{\#lorem(80)}

\NormalTok{\#bibliography("refs.bib")}
\end{Highlighting}
\end{Shaded}

\subsection{Full(er) Example}\label{fuller-example}

See
\href{https://github.com/avonmoll/ifacconf-typst/blob/main/template/main.typ}{\texttt{\ main.typ\ }}
.

\subsection{Dependencies}\label{dependencies}

\begin{itemize}
\tightlist
\item
  typst 0.11.0
\item
  ctheorems 1.1.0 (a Typst package for handling theorem-like
  environments)
\end{itemize}

\subsection{Notes, features, etc.}\label{notes-features-etc.}

\begin{itemize}
\tightlist
\item
  the call to \texttt{\ \#show:\ ifacconf-rules\ } is necessary for some
  show rules defined in \texttt{\ template.typ\ } to get activated
\item
  \texttt{\ ifac-conference.csl\ } is a lightly modified version of
  \texttt{\ apa.csl\ } and is included in order to change the citation
  format from, e.g., \texttt{\ (Able\ 1956)\ } to
  \texttt{\ Able\ (1956)\ } in order to match
  \texttt{\ ifacconf\_latex\ }
\item
  Tables have formatting rules that get activated inside calls to
  \texttt{\ figure\ } with \texttt{\ kind:\ "table"\ } ; a convenience
  function \texttt{\ tablefig\ } is provided which sets this
  automatically
\item
  all theorem-like environments that were available in
  \texttt{\ ifacconf\_latex\ } are defined in \texttt{\ template.typ\ }
  ; simply call, for example,
  \texttt{\ \#theorem{[}Content...{]}\ ...\ \#proof{[}Proof...{]}\ }
\item
  the LaTeX version does not include a QED symbol at the end of proofs,
  however one is included here (this is easy to change)
\item
  Typst did not seem to like BibTeX citation keys containing colons
  (which was how they came from \texttt{\ ifacconf\_latex\ } )
\item
  alignment for linebreaks in long equations is somewhat manual (e.g.,
  for equation (2) in \texttt{\ ifacconf.typ\ } ) but probably there is
  a better way to handle this now or in the future
\item
  the files \texttt{\ refs.bib\ } (essentially) and
  \texttt{\ bifurcation.jpg\ } come from \texttt{\ ifacconf\_latex\ }
\item
  the file \texttt{\ ifacconf.typ\ } is modeled directly after
  \texttt{\ ifacconf.tex\ } by Juan a. de la Puente
\item
  the \texttt{\ citep\ } function renders citations like
  \texttt{\ (Keohane,\ 1958)\ } instead of the default style of
  \texttt{\ Keohane\ (1958)\ }
\end{itemize}

\subsection{License}\label{license}

This template is licensed according to the MIT No Attribution license
(see \texttt{\ LICENSE.MD\ } ).

The files in the \texttt{\ CSL\ } folder are licensed according to
\texttt{\ CSL/LICENSE.md\ } (CC BY/SA 4.0) because it is a lightly
modified version of \texttt{\ apa.csl\ } by Brenton M. Wiernik which is
also licensed by a CC BY/SA license.

\href{/app?template=abiding-ifacconf&version=0.1.0}{Create project in
app}

\subsubsection{How to use}\label{how-to-use}

Click the button above to create a new project using this template in
the Typst app.

You can also use the Typst CLI to start a new project on your computer
using this command:

\begin{verbatim}
typst init @preview/abiding-ifacconf:0.1.0
\end{verbatim}

\includesvg[width=0.16667in,height=0.16667in]{/assets/icons/16-copy.svg}

\subsubsection{About}\label{about}

\begin{description}
\tightlist
\item[Author :]
\href{https://avonmoll.github.io}{Alexander Von Moll}
\item[License:]
MIT-0
\item[Current version:]
0.1.0
\item[Last updated:]
March 21, 2024
\item[First released:]
March 21, 2024
\item[Minimum Typst version:]
0.11.0
\item[Archive size:]
27.8 kB
\href{https://packages.typst.org/preview/abiding-ifacconf-0.1.0.tar.gz}{\pandocbounded{\includesvg[keepaspectratio]{/assets/icons/16-download.svg}}}
\item[Repository:]
\href{https://github.com/avonmoll/ifacconf-typst}{GitHub}
\item[Discipline s :]
\begin{itemize}
\tightlist
\item[]
\item
  \href{https://typst.app/universe/search/?discipline=computer-science}{Computer
  Science}
\item
  \href{https://typst.app/universe/search/?discipline=engineering}{Engineering}
\end{itemize}
\item[Categor y :]
\begin{itemize}
\tightlist
\item[]
\item
  \pandocbounded{\includesvg[keepaspectratio]{/assets/icons/16-atom.svg}}
  \href{https://typst.app/universe/search/?category=paper}{Paper}
\end{itemize}
\end{description}

\subsubsection{Where to report issues?}\label{where-to-report-issues}

This template is a project of Alexander Von Moll . Report issues on
\href{https://github.com/avonmoll/ifacconf-typst}{their repository} .
You can also try to ask for help with this template on the
\href{https://forum.typst.app}{Forum} .

Please report this template to the Typst team using the
\href{https://typst.app/contact}{contact form} if you believe it is a
safety hazard or infringes upon your rights.

\phantomsection\label{versions}
\subsubsection{Version history}\label{version-history}

\begin{longtable}[]{@{}ll@{}}
\toprule\noalign{}
Version & Release Date \\
\midrule\noalign{}
\endhead
\bottomrule\noalign{}
\endlastfoot
0.1.0 & March 21, 2024 \\
\end{longtable}

Typst GmbH did not create this template and cannot guarantee correct
functionality of this template or compatibility with any version of the
Typst compiler or app.


\title{typst.app/universe/package/optimal-ovgu-thesis}

\phantomsection\label{banner}
\phantomsection\label{template-thumbnail}
\pandocbounded{\includegraphics[keepaspectratio]{https://packages.typst.org/preview/thumbnails/optimal-ovgu-thesis-0.1.1-small.webp}}

\section{optimal-ovgu-thesis}\label{optimal-ovgu-thesis}

{ 0.1.1 }

A thesis template for Otto von Guericke University Magdeburg

\href{/app?template=optimal-ovgu-thesis&version=0.1.1}{Create project in
app}

\phantomsection\label{readme}
This template was created for a master thesis at the faculty of computer
science (FIN), but should work as well for other faculties.

\subsection{File structure}\label{file-structure}

\begin{verbatim}
.
├── assets                          // Images, CSV-Files, etc. 
│   └── figure                      // Image files
│       └── optimal-ovgu-thesis    
├── chapter                         // Content
│   ├── 01-Einleitung.typ
│   ├── ...
│   └── 99-Appendix.typ
├── expose.typ                      // Exposé template
├── metadata.typ                    // Metadata and template config
├── thesis.bib                      // Bibliography (e.g. generated by Zotero + Better BibTex)
└── thesis.typ                      // Thesis template
\end{verbatim}

\subsection{Logos on the title page}\label{logos-on-the-title-page}

The header- and organisation-logo can be set in the
\texttt{\ metadata.typ\ } file (see example below). There are two
example logo files in \texttt{\ assets/figure/optimal-ovgu-thesis\ } .
Please refer to
\href{https://www.cd.ovgu.de/Fakult\%C3\%A4ten.html}{cd.ovgu.de} for
more information regarding the OvGU corporate design and for the signet
and logo of your faculty.

Header logos are set in \texttt{\ metadata.typ\ } :

\begin{Shaded}
\begin{Highlighting}[]
\NormalTok{// Example 1: Use UCC logo as organisation{-}logo and the FIN faculty header as header{-}logo}
\NormalTok{\#let organisation{-}logo = image("assets/figure/optimal{-}ovgu{-}thesis/ucc.svg", width: 2cm)}
\NormalTok{\#let header{-}logo = image("assets/figure/optimal{-}ovgu{-}thesis/fin{-}de.svg", width: 100\%)}

\NormalTok{// Example 2: Do not use logos at all}
\NormalTok{\#let organisation{-}logo = none}
\NormalTok{\#let header{-}logo = none}
\end{Highlighting}
\end{Shaded}

\subsection{Fonts}\label{fonts}

This template requires these two fonts to be installed on your system:

\begin{itemize}
\tightlist
\item
  New Computer Modern
\item
  New Computer Modern Sans
\end{itemize}

\subsubsection{NixOS}\label{nixos}

In your \texttt{\ configuration.nix\ } :

\begin{Shaded}
\begin{Highlighting}[]
\NormalTok{  fonts.packages = }\KeywordTok{with}\NormalTok{ pkgs}\OperatorTok{;} \OperatorTok{[}
\NormalTok{    liberation\_ttf }\CommentTok{\# here are your other fonts (liberation is just an example)}
  \OperatorTok{]} \OperatorTok{++}\NormalTok{ texlive.newcomputermodern.pkgs; }\CommentTok{\# ← New Computer Modern font}
\end{Highlighting}
\end{Shaded}

\subsection{Development}\label{development}

In case you want to contribute, check out the repo into a
\href{https://github.com/typst/packages?tab=readme-ov-file\#local-packages}{typst
package directory}

\subsubsection{Example for Linux:}\label{example-for-linux}

Local package path:
\texttt{\ \textasciitilde{}/.local/share/typst/packages/local/optimal-ovgu-thesis/0.1.1\ }

\begin{Shaded}
\begin{Highlighting}[]
\FunctionTok{mkdir} \AttributeTok{{-}p}\NormalTok{ \textasciitilde{}/.local/share/typst/packages/local/optimal{-}ovgu{-}thesis}
\BuiltInTok{cd}\NormalTok{ \textasciitilde{}/.local/share/typst/packages/local/optimal{-}ovgu{-}thesis}
\FunctionTok{git}\NormalTok{ clone git@github.com:v411e/optimal{-}ovgu{-}thesis.git}
\FunctionTok{mv}\NormalTok{ optimal{-}ovgu{-}thesis 0.1.1}
\end{Highlighting}
\end{Shaded}

This will make the package available locally, so you can use
\texttt{\ typst\ init\ "@local/optimal-ovgu-thesis:0.1.1"\ } to create a
test-project from the template.

\href{/app?template=optimal-ovgu-thesis&version=0.1.1}{Create project in
app}

\subsubsection{How to use}\label{how-to-use}

Click the button above to create a new project using this template in
the Typst app.

You can also use the Typst CLI to start a new project on your computer
using this command:

\begin{verbatim}
typst init @preview/optimal-ovgu-thesis:0.1.1
\end{verbatim}

\includesvg[width=0.16667in,height=0.16667in]{/assets/icons/16-copy.svg}

\subsubsection{About}\label{about}

\begin{description}
\tightlist
\item[Author :]
\href{https://github.com/v411e}{Valentin Rieß}
\item[License:]
MIT
\item[Current version:]
0.1.1
\item[Last updated:]
November 25, 2024
\item[First released:]
May 17, 2024
\item[Archive size:]
36.9 kB
\href{https://packages.typst.org/preview/optimal-ovgu-thesis-0.1.1.tar.gz}{\pandocbounded{\includesvg[keepaspectratio]{/assets/icons/16-download.svg}}}
\item[Repository:]
\href{https://github.com/v411e/optimal-ovgu-thesis}{GitHub}
\item[Categor y :]
\begin{itemize}
\tightlist
\item[]
\item
  \pandocbounded{\includesvg[keepaspectratio]{/assets/icons/16-mortarboard.svg}}
  \href{https://typst.app/universe/search/?category=thesis}{Thesis}
\end{itemize}
\end{description}

\subsubsection{Where to report issues?}\label{where-to-report-issues}

This template is a project of Valentin Rieß . Report issues on
\href{https://github.com/v411e/optimal-ovgu-thesis}{their repository} .
You can also try to ask for help with this template on the
\href{https://forum.typst.app}{Forum} .

Please report this template to the Typst team using the
\href{https://typst.app/contact}{contact form} if you believe it is a
safety hazard or infringes upon your rights.

\phantomsection\label{versions}
\subsubsection{Version history}\label{version-history}

\begin{longtable}[]{@{}ll@{}}
\toprule\noalign{}
Version & Release Date \\
\midrule\noalign{}
\endhead
\bottomrule\noalign{}
\endlastfoot
0.1.1 & November 25, 2024 \\
\href{https://typst.app/universe/package/optimal-ovgu-thesis/0.1.0/}{0.1.0}
& May 17, 2024 \\
\end{longtable}

Typst GmbH did not create this template and cannot guarantee correct
functionality of this template or compatibility with any version of the
Typst compiler or app.


\title{typst.app/universe/package/modern-hsh-thesis}

\phantomsection\label{banner}
\phantomsection\label{template-thumbnail}
\pandocbounded{\includegraphics[keepaspectratio]{https://packages.typst.org/preview/thumbnails/modern-hsh-thesis-1.0.0-small.webp}}

\section{modern-hsh-thesis}\label{modern-hsh-thesis}

{ 1.0.0 }

Template for writing a bachelors or masters thesis at the Hochschule
Hannover, Faculty 4.

\href{/app?template=modern-hsh-thesis&version=1.0.0}{Create project in
app}

\phantomsection\label{readme}
Version 1.0.0

A template for writing a bachelors or masters thesis at the Hochschule
Hannover, Faculty 4.

\subsection{Getting Started}\label{getting-started}

\subsubsection{WebApp}\label{webapp}

Choose the template in the typst web app and follow the instructions
there.

\subsubsection{Terminal}\label{terminal}

\begin{Shaded}
\begin{Highlighting}[]
\ExtensionTok{typst}\NormalTok{ init @preview/modern{-}hsh{-}thesis:1.0.0}
\end{Highlighting}
\end{Shaded}

\subsubsection{Import}\label{import}

\begin{Shaded}
\begin{Highlighting}[]
\NormalTok{\#import "@preview/modern{-}hsh{-}thesis:1.0.0": *}

\NormalTok{\#show: project.with(}
\NormalTok{  title: "Beispiel{-}Titel",}
\NormalTok{  subtitle: "Bachelorarbeit im Studiengang Mediendesigninformatik",}
\NormalTok{  author: "Vorname Nachname",}
\NormalTok{  author\_email: "vorname@nachname.tld",}
\NormalTok{  matrikelnummer: 1234567,}
\NormalTok{  prof: [}
\NormalTok{    Prof. Dr. Vorname Nachname\textbackslash{}}
\NormalTok{    Abteilung Informatik, Fakultät IV\textbackslash{}}
\NormalTok{    Hochschule Hannover\textbackslash{}    }
\NormalTok{    \#link("mailto:vorname.nachname@hs{-}hannover.de")}
    
\NormalTok{  ],}
\NormalTok{  second\_prof: [}
\NormalTok{    Prof. Dr. Vorname Nachname\textbackslash{}}
\NormalTok{    Abteilung Informatik, Fakultät IV\textbackslash{}}
\NormalTok{    Hochschule Hannover\textbackslash{}    }
\NormalTok{    \#link("mailto:vorname.nachname@hs{-}hannover.de")}
\NormalTok{  ],}
\NormalTok{  date: "01. August 2024",}
\NormalTok{  glossaryColumns: 1,}
\NormalTok{  bibliography: bibliography(("sources.bib", "sources.yaml"), style: "institute{-}of{-}electrical{-}and{-}electronics{-}engineers", title: "Literaturverzeichnis")}
\NormalTok{)}
\end{Highlighting}
\end{Shaded}

\subsubsection{Additional functions}\label{additional-functions}

\texttt{\ customFunctions.typ\ } contains additional functions that can
be used in the template.

\texttt{\ \#smallLine\ } : A small line that can be used to separate
sections.

\texttt{\ \#task\ } : A card that can be used to create a list of tracks
(see example in 1-einleitung.typ).

\texttt{\ \#track\ } or \texttt{\ \#\#narrowTrack\ } : A track that can
be displayed inside a task (see example in 1-einleitung.typ).

\texttt{\ \#useCase\ } : Display a Use Case (see example in
1-einleitung.typ).

\texttt{\ \#attributedQuote\ } : Display a quote with an attribution.

\texttt{\ \#diagramFigure\ } , \texttt{\ \#codeFigure\ } ,
\texttt{\ \#imageFigure\ } , \texttt{\ \#treeFigure\ } : Wrap an
image/code/diagram/tree-list in a figure with a caption.

\texttt{\ \#imageFigureNoPad\ } : Display a figure without padding.

\texttt{\ \#getCurrentHeadingHydra\ } , \texttt{\ \#getCurrentHeading\ }
: Get the heading of the current page.

\subsubsection{Development Environment}\label{development-environment}

\begin{enumerate}
\tightlist
\item
  Install Typst \url{https://github.com/typst-community/typst-install}
\item
  Clone the repository
\item
  CD into the repository
\item
  Run
  \texttt{\ git\ pull\ \&\&\ just\ install\ \&\&\ just\ install-preview\ }
  to install/update the template
\item
  Run
  \texttt{\ typst\ init\ @local/modern-hsh-thesis:1.0.0\ \&\&\ typst\ compile\ modern-hsh-thesis/main.typ\ }
  to compile the template
\end{enumerate}

\subsection{Additional Documentation}\label{additional-documentation}

Take a look at this complete Bachelor’s thesis example using the
\texttt{\ modern-hsh-thesis\ } template:
\href{https://github.com/MrToWy/Bachelorarbeit}{Bachelor’s Thesis
Example}

\href{/app?template=modern-hsh-thesis&version=1.0.0}{Create project in
app}

\subsubsection{How to use}\label{how-to-use}

Click the button above to create a new project using this template in
the Typst app.

You can also use the Typst CLI to start a new project on your computer
using this command:

\begin{verbatim}
typst init @preview/modern-hsh-thesis:1.0.0
\end{verbatim}

\includesvg[width=0.16667in,height=0.16667in]{/assets/icons/16-copy.svg}

\subsubsection{About}\label{about}

\begin{description}
\tightlist
\item[Author :]
\href{https://github.com/MrToWy}{Tobias Wylega}
\item[License:]
MIT
\item[Current version:]
1.0.0
\item[Last updated:]
September 8, 2024
\item[First released:]
September 8, 2024
\item[Minimum Typst version:]
0.11.1
\item[Archive size:]
31.6 kB
\href{https://packages.typst.org/preview/modern-hsh-thesis-1.0.0.tar.gz}{\pandocbounded{\includesvg[keepaspectratio]{/assets/icons/16-download.svg}}}
\item[Repository:]
\href{https://github.com/MrToWy/hsh-thesis}{GitHub}
\item[Categor y :]
\begin{itemize}
\tightlist
\item[]
\item
  \pandocbounded{\includesvg[keepaspectratio]{/assets/icons/16-mortarboard.svg}}
  \href{https://typst.app/universe/search/?category=thesis}{Thesis}
\end{itemize}
\end{description}

\subsubsection{Where to report issues?}\label{where-to-report-issues}

This template is a project of Tobias Wylega . Report issues on
\href{https://github.com/MrToWy/hsh-thesis}{their repository} . You can
also try to ask for help with this template on the
\href{https://forum.typst.app}{Forum} .

Please report this template to the Typst team using the
\href{https://typst.app/contact}{contact form} if you believe it is a
safety hazard or infringes upon your rights.

\phantomsection\label{versions}
\subsubsection{Version history}\label{version-history}

\begin{longtable}[]{@{}ll@{}}
\toprule\noalign{}
Version & Release Date \\
\midrule\noalign{}
\endhead
\bottomrule\noalign{}
\endlastfoot
1.0.0 & September 8, 2024 \\
\end{longtable}

Typst GmbH did not create this template and cannot guarantee correct
functionality of this template or compatibility with any version of the
Typst compiler or app.


\title{typst.app/universe/package/truthfy}

\phantomsection\label{banner}
\section{truthfy}\label{truthfy}

{ 0.5.0 }

Make empty or automatically filled truth table

\phantomsection\label{readme}
Make an empty or filled truth table in Typst

\begin{Shaded}
\begin{Highlighting}[]
\ExtensionTok{truth{-}table{-}empty}\ErrorTok{(}\ExtensionTok{info:}\NormalTok{ array}\PreprocessorTok{[}\SpecialStringTok{math\_block}\PreprocessorTok{]}\NormalTok{, data: array}\PreprocessorTok{[}\SpecialStringTok{str}\PreprocessorTok{]}\KeywordTok{)}\BuiltInTok{:}\NormalTok{ table}
    \CommentTok{\# Create an empty (or filled with "data") truth table. }

\ExtensionTok{truth{-}table}\ErrorTok{(}\ExtensionTok{..info:}\NormalTok{ array}\PreprocessorTok{[}\SpecialStringTok{math\_block}\PreprocessorTok{]}\KeywordTok{)}\BuiltInTok{:}\NormalTok{ table}
    \CommentTok{\# Create a filled truth table.}

\ExtensionTok{karnaugh{-}empty}\ErrorTok{(}\ExtensionTok{info:}\NormalTok{ array}\PreprocessorTok{[}\SpecialStringTok{math\_block}\PreprocessorTok{]}\NormalTok{, data: array}\PreprocessorTok{[}\SpecialStringTok{str}\PreprocessorTok{]}\KeywordTok{)}\BuiltInTok{:}\NormalTok{ table}
    \CommentTok{\# Create an empty karnaugh table.}

\ExtensionTok{NAND:}\NormalTok{ Equivalent to sym.arrow.t}
\ExtensionTok{NOR:}\NormalTok{ Equivalent to sym.arrow.b}
\end{Highlighting}
\end{Shaded}

\subsection{\texorpdfstring{\texttt{\ sc\ }}{ sc }}\label{sc}

Theses functions have a new named argument, called \texttt{\ sc\ } for
symbol-convention.

You can add you own function to customise the render of the 0 and the 1.
See examples.

Syntax:

\begin{Shaded}
\begin{Highlighting}[]
\NormalTok{\#let sc(symb) = \{}
\NormalTok{    if (symb) \{}
\NormalTok{        "an one"}
\NormalTok{    \} else \{}
\NormalTok{        "a zero"}
\NormalTok{    \}}
\NormalTok{\}}
\end{Highlighting}
\end{Shaded}

\subsection{\texorpdfstring{\texttt{\ reverse\ }}{ reverse }}\label{reverse}

Reverse your table, see issue \#3

\subsection{Simple}\label{simple}

\begin{Shaded}
\begin{Highlighting}[]
\NormalTok{\#import "@preview/truthfy:0.4.0": truth{-}table, truth{-}table{-}empty}

\NormalTok{\#truth{-}table($A and B$, $B or A$, $A =\textgreater{} B$, $(A =\textgreater{} B) \textless{}=\textgreater{} A$, $ A xor B$)}

\NormalTok{\#truth{-}table($p =\textgreater{} q$, $not p =\textgreater{} (q =\textgreater{} p)$, $p or q$, $not p or q$)}
\end{Highlighting}
\end{Shaded}

\pandocbounded{\includegraphics[keepaspectratio]{https://github.com/Thumuss/truthfy/assets/42680097/7edb921d-659e-4348-a12a-07bcc3822012}}

\begin{Shaded}
\begin{Highlighting}[]
\NormalTok{\#import "@preview/truthfy:0.4.0": truth{-}table, truth{-}table{-}empty}

\NormalTok{\#truth{-}table(sc: (a) =\textgreater{} \{if (a) \{"a"\} else \{"b"\}\}, $a and b$)}

\NormalTok{\#truth{-}table{-}empty(sc: (a) =\textgreater{} \{if (a) \{"x"\} else \{"$"\}\}, ($a and b$,), (false, [], true))}
\end{Highlighting}
\end{Shaded}

\pandocbounded{\includegraphics[keepaspectratio]{https://github.com/Thumuss/truthfy/assets/42680097/1ccf6077-5cfb-4643-b621-1dc9529b8176}}

If you have any idea to add in this package, add a new issue
\href{https://github.com/Thumuss/truthfy/issues}{here} !

\texttt{\ 0.1.0\ } : Create the package.\\
\texttt{\ 0.2.0\ } :

\begin{itemize}
\tightlist
\item
  You can now use \texttt{\ t\ } , \texttt{\ r\ } , \texttt{\ u\ } ,
  \texttt{\ e\ } , \texttt{\ f\ } , \texttt{\ a\ } , \texttt{\ l\ } ,
  \texttt{\ s\ } without any problems!
\item
  You can now add subscript to a letter
\item
  Only \texttt{\ generate-table\ } and \texttt{\ generate-empty\ } are
  now exported
\item
  Better example with more cases
\item
  Implemented the \texttt{\ a\ ?\ b\ :\ c\ } operator\\
\end{itemize}

\texttt{\ 0.3.0\ } :

\begin{itemize}
\tightlist
\item
  Changing the name of \texttt{\ generate-table\ } and
  \texttt{\ generate-empty\ } to \texttt{\ truth-table\ } and
  \texttt{\ truth-table-empty\ }
\item
  Adding support of \texttt{\ NAND\ } and \texttt{\ NOR\ } operators.
\item
  Adding support of custom \texttt{\ sc\ } function.
\item
  Better example and \href{http://readme.md/}{README.md}
\end{itemize}

\texttt{\ 0.4.0\ } :

\begin{itemize}
\tightlist
\item
  Add \texttt{\ karnaugh-empty\ }
\item
  Images re-added (see \#2)
\item
  Add \texttt{\ reverse\ } option (see \#3)
\end{itemize}

\subsubsection{How to add}\label{how-to-add}

Copy this into your project and use the import as \texttt{\ truthfy\ }

\begin{verbatim}
#import "@preview/truthfy:0.5.0"
\end{verbatim}

\includesvg[width=0.16667in,height=0.16667in]{/assets/icons/16-copy.svg}

Check the docs for
\href{https://typst.app/docs/reference/scripting/\#packages}{more
information on how to import packages} .

\subsubsection{About}\label{about}

\begin{description}
\tightlist
\item[Author :]
\href{https://github.com/Thumuss}{Quemin Thomas}
\item[License:]
MIT
\item[Current version:]
0.5.0
\item[Last updated:]
September 14, 2024
\item[First released:]
October 9, 2023
\item[Archive size:]
4.54 kB
\href{https://packages.typst.org/preview/truthfy-0.5.0.tar.gz}{\pandocbounded{\includesvg[keepaspectratio]{/assets/icons/16-download.svg}}}
\item[Repository:]
\href{https://github.com/Thumuss/truthfy}{GitHub}
\end{description}

\subsubsection{Where to report issues?}\label{where-to-report-issues}

This package is a project of Quemin Thomas . Report issues on
\href{https://github.com/Thumuss/truthfy}{their repository} . You can
also try to ask for help with this package on the
\href{https://forum.typst.app}{Forum} .

Please report this package to the Typst team using the
\href{https://typst.app/contact}{contact form} if you believe it is a
safety hazard or infringes upon your rights.

\phantomsection\label{versions}
\subsubsection{Version history}\label{version-history}

\begin{longtable}[]{@{}ll@{}}
\toprule\noalign{}
Version & Release Date \\
\midrule\noalign{}
\endhead
\bottomrule\noalign{}
\endlastfoot
0.5.0 & September 14, 2024 \\
\href{https://typst.app/universe/package/truthfy/0.4.0/}{0.4.0} & June
10, 2024 \\
\href{https://typst.app/universe/package/truthfy/0.3.0/}{0.3.0} &
February 6, 2024 \\
\href{https://typst.app/universe/package/truthfy/0.2.0/}{0.2.0} &
October 16, 2023 \\
\href{https://typst.app/universe/package/truthfy/0.1.0/}{0.1.0} &
October 9, 2023 \\
\end{longtable}

Typst GmbH did not create this package and cannot guarantee correct
functionality of this package or compatibility with any version of the
Typst compiler or app.


\title{typst.app/universe/package/canonical-nthu-thesis}

\phantomsection\label{banner}
\phantomsection\label{template-thumbnail}
\pandocbounded{\includegraphics[keepaspectratio]{https://packages.typst.org/preview/thumbnails/canonical-nthu-thesis-0.2.0-small.webp}}

\section{canonical-nthu-thesis}\label{canonical-nthu-thesis}

{ 0.2.0 }

A template for master theses and doctoral dissertations for NTHU
(National Tsing Hua University).

\href{/app?template=canonical-nthu-thesis&version=0.2.0}{Create project
in app}

\phantomsection\label{readme}
A \href{https://typst.app/docs/}{Typst} template for master theses and
doctoral dissertations for NTHU (National Tsing Hua University).

國立æ¸\ldots è?¯å¤§å­¸ç¢©å£«ï¼ˆå?šå£«ï¼‰è«--æ--‡
\href{https://typst.app/docs/}{Typst} 模�。

\begin{itemize}
\tightlist
\item
  \href{https://typst.app/universe/package/canonical-nthu-thesis}{Typst
  Universe Package}
\item
  \href{https://codeberg.org/kotatsuyaki/canonical-nthu-thesis}{Codeberg
  Repo}
\end{itemize}

\pandocbounded{\includegraphics[keepaspectratio]{https://github.com/typst/packages/raw/main/packages/preview/canonical-nthu-thesis/0.2.0/covers.png}}

\subsection{Usage}\label{usage}

\subsubsection{Installing the Chinese
fonts}\label{installing-the-chinese-fonts}

This template uses the official fonts from the Ministry of Education of
Taiwan (Edukai/TW-MOE-Std-Kai), which are required to be downloaded and
installed manually from
\href{https://language.moe.gov.tw/001/Upload/Files/site_content/M0001/edukai-5.0.zip}{language.moe.gov.tw}
. The Typst web app has the fonts installed by default, so there is no
need to install the fonts on the web app.

此模æ?¿ä¸­æ--‡éƒ¨åˆ†ä½¿ç''¨æ•™è‚²éƒ¨æ¨™æº--楷書å­---é«''(Edukai/TW-MOE-Std-Kai),在本地編譯æ--‡ä»¶å‰?需è¦?自
\href{https://language.moe.gov.tw/001/Upload/Files/site_content/M0001/edukai-5.0.zip}{language.moe.gov.tw}
下載並手動安�。Typst web
appå·²é~?è£?該å­---é«'',æ•\ldots 無需é¡?å¤--安è£?。

\subsubsection{Editing}\label{editing}

All the content of the thesis are in the \texttt{\ thesis.typ\ } file.
In the beginning of \texttt{\ thesis.typ\ } , there is a call to the
\texttt{\ setup-thesis(info,\ style)\ } function that configures the
metadata (the titles and the author etc.) and the styling of the thesis
document. Replace the values with your own.

所有è«--æ--‡å\ldots§å®¹çš†ä½?æ--¼ \texttt{\ thesis.typ\ }
æª''案å\ldots§ã€‚該æª''案å‰?段的部分å`¼å?«äº†
\texttt{\ setup-thesis(info,\ style)\ }
函å¼?,設置è«--æ--‡çš„雜é~\ldots 資訊(標題å?Šä½œè€\ldots 等)å?Šå¤--觀é?¸é~\ldots ,請置æ?›ç‚ºè‡ªå·±çš„資訊。

\subsubsection{Local usage}\label{local-usage}

\begin{Shaded}
\begin{Highlighting}[]
\ExtensionTok{$}\NormalTok{ typst init @preview/canonical{-}nthu{-}thesis:0.2.0 my{-}thesis}
\ExtensionTok{$}\NormalTok{ cd my{-}thesis}
\ExtensionTok{$}\NormalTok{ typst watch thesis.typ}
\end{Highlighting}
\end{Shaded}

\subsection{Development}\label{development}

Development and issue tracking happens on the
\href{https://codeberg.org/kotatsuyaki/canonical-nthu-thesis}{repository
on Codeberg} .

\subsection{License}\label{license}

This project is licensed under the MIT License.

\href{/app?template=canonical-nthu-thesis&version=0.2.0}{Create project
in app}

\subsubsection{How to use}\label{how-to-use}

Click the button above to create a new project using this template in
the Typst app.

You can also use the Typst CLI to start a new project on your computer
using this command:

\begin{verbatim}
typst init @preview/canonical-nthu-thesis:0.2.0
\end{verbatim}

\includesvg[width=0.16667in,height=0.16667in]{/assets/icons/16-copy.svg}

\subsubsection{About}\label{about}

\begin{description}
\tightlist
\item[Author :]
kotatsuyaki
\item[License:]
MIT
\item[Current version:]
0.2.0
\item[Last updated:]
August 1, 2024
\item[First released:]
June 17, 2024
\item[Archive size:]
48.4 kB
\href{https://packages.typst.org/preview/canonical-nthu-thesis-0.2.0.tar.gz}{\pandocbounded{\includesvg[keepaspectratio]{/assets/icons/16-download.svg}}}
\item[Repository:]
\href{https://codeberg.org/kotatsuyaki/canonical-nthu-thesis}{Codeberg}
\item[Categor y :]
\begin{itemize}
\tightlist
\item[]
\item
  \pandocbounded{\includesvg[keepaspectratio]{/assets/icons/16-mortarboard.svg}}
  \href{https://typst.app/universe/search/?category=thesis}{Thesis}
\end{itemize}
\end{description}

\subsubsection{Where to report issues?}\label{where-to-report-issues}

This template is a project of kotatsuyaki . Report issues on
\href{https://codeberg.org/kotatsuyaki/canonical-nthu-thesis}{their
repository} . You can also try to ask for help with this template on the
\href{https://forum.typst.app}{Forum} .

Please report this template to the Typst team using the
\href{https://typst.app/contact}{contact form} if you believe it is a
safety hazard or infringes upon your rights.

\phantomsection\label{versions}
\subsubsection{Version history}\label{version-history}

\begin{longtable}[]{@{}ll@{}}
\toprule\noalign{}
Version & Release Date \\
\midrule\noalign{}
\endhead
\bottomrule\noalign{}
\endlastfoot
0.2.0 & August 1, 2024 \\
\href{https://typst.app/universe/package/canonical-nthu-thesis/0.1.0/}{0.1.0}
& June 17, 2024 \\
\end{longtable}

Typst GmbH did not create this template and cannot guarantee correct
functionality of this template or compatibility with any version of the
Typst compiler or app.


\title{typst.app/universe/package/rubber-article}

\phantomsection\label{banner}
\phantomsection\label{template-thumbnail}
\pandocbounded{\includegraphics[keepaspectratio]{https://packages.typst.org/preview/thumbnails/rubber-article-0.1.0-small.webp}}

\section{rubber-article}\label{rubber-article}

{ 0.1.0 }

A simple template recreating the look of the classic LaTeX article.

\href{/app?template=rubber-article&version=0.1.0}{Create project in app}

\phantomsection\label{readme}
Version 0.1.0

This template is a intended as a starting point for creating documents,
which should have the classic LaTeX Article look.

\subsection{Getting Started}\label{getting-started}

These instructions will get you a copy of the project up and running on
the typst web app. Perhaps a short code example on importing the package
and a very simple teaser usage.

\begin{Shaded}
\begin{Highlighting}[]
\NormalTok{\#import "@preview/rubber{-}article:0.1.0": *}

\NormalTok{\#show: article.with()}

\NormalTok{\#maketitle(}
\NormalTok{  title: "The Title of the Paper",}
\NormalTok{  authors: (}
\NormalTok{    "Authors Name",}
\NormalTok{  ),}
\NormalTok{  date: "September 1970",}
\NormalTok{)}
\end{Highlighting}
\end{Shaded}

\subsection{Further Functionality}\label{further-functionality}

The template provides a few more functions to customize the document.

\begin{Shaded}
\begin{Highlighting}[]
\NormalTok{\#show article.with(}
\NormalTok{  lang:"de",}
\NormalTok{  eq{-}numbering:none,}
\NormalTok{  text{-}size:10pt,}
\NormalTok{  page{-}numbering: "1",}
\NormalTok{  page{-}numbering{-}align: center,}
\NormalTok{  heading{-}numbering: "1.1  ",}
\NormalTok{)}
\end{Highlighting}
\end{Shaded}

\href{/app?template=rubber-article&version=0.1.0}{Create project in app}

\subsubsection{How to use}\label{how-to-use}

Click the button above to create a new project using this template in
the Typst app.

You can also use the Typst CLI to start a new project on your computer
using this command:

\begin{verbatim}
typst init @preview/rubber-article:0.1.0
\end{verbatim}

\includesvg[width=0.16667in,height=0.16667in]{/assets/icons/16-copy.svg}

\subsubsection{About}\label{about}

\begin{description}
\tightlist
\item[Author :]
Niko Pikall
\item[License:]
Unlicense
\item[Current version:]
0.1.0
\item[Last updated:]
September 8, 2024
\item[First released:]
September 8, 2024
\item[Archive size:]
3.00 kB
\href{https://packages.typst.org/preview/rubber-article-0.1.0.tar.gz}{\pandocbounded{\includesvg[keepaspectratio]{/assets/icons/16-download.svg}}}
\item[Repository:]
\href{https://github.com/npikall/rubber-article.git}{GitHub}
\item[Discipline s :]
\begin{itemize}
\tightlist
\item[]
\item
  \href{https://typst.app/universe/search/?discipline=biology}{Biology}
\item
  \href{https://typst.app/universe/search/?discipline=chemistry}{Chemistry}
\item
  \href{https://typst.app/universe/search/?discipline=engineering}{Engineering}
\item
  \href{https://typst.app/universe/search/?discipline=geography}{Geography}
\item
  \href{https://typst.app/universe/search/?discipline=mathematics}{Mathematics}
\item
  \href{https://typst.app/universe/search/?discipline=physics}{Physics}
\end{itemize}
\item[Categor ies :]
\begin{itemize}
\tightlist
\item[]
\item
  \pandocbounded{\includesvg[keepaspectratio]{/assets/icons/16-atom.svg}}
  \href{https://typst.app/universe/search/?category=paper}{Paper}
\item
  \pandocbounded{\includesvg[keepaspectratio]{/assets/icons/16-speak.svg}}
  \href{https://typst.app/universe/search/?category=report}{Report}
\end{itemize}
\end{description}

\subsubsection{Where to report issues?}\label{where-to-report-issues}

This template is a project of Niko Pikall . Report issues on
\href{https://github.com/npikall/rubber-article.git}{their repository} .
You can also try to ask for help with this template on the
\href{https://forum.typst.app}{Forum} .

Please report this template to the Typst team using the
\href{https://typst.app/contact}{contact form} if you believe it is a
safety hazard or infringes upon your rights.

\phantomsection\label{versions}
\subsubsection{Version history}\label{version-history}

\begin{longtable}[]{@{}ll@{}}
\toprule\noalign{}
Version & Release Date \\
\midrule\noalign{}
\endhead
\bottomrule\noalign{}
\endlastfoot
0.1.0 & September 8, 2024 \\
\end{longtable}

Typst GmbH did not create this template and cannot guarantee correct
functionality of this template or compatibility with any version of the
Typst compiler or app.


\title{typst.app/universe/package/silky-report-insa}

\phantomsection\label{banner}
\phantomsection\label{template-thumbnail}
\pandocbounded{\includegraphics[keepaspectratio]{https://packages.typst.org/preview/thumbnails/silky-report-insa-0.4.0-small.webp}}

\section{silky-report-insa}\label{silky-report-insa}

{ 0.4.0 }

A template made for reports and other documents of INSA, a French
engineering school.

\href{/app?template=silky-report-insa&version=0.4.0}{Create project in
app}

\phantomsection\label{readme}
Typst Template for full documents and reports for the french engineering
school INSA.

\subsection{Table of contents}\label{table-of-contents}

\begin{enumerate}
\tightlist
\item
  \href{https://github.com/typst/packages/raw/main/packages/preview/silky-report-insa/0.4.0/\#examples}{Examples
  \& Usage}

  \begin{enumerate}
  \tightlist
  \item
    \href{https://github.com/typst/packages/raw/main/packages/preview/silky-report-insa/0.4.0/\#\%F0\%9F\%A7\%AA-tp-report}{🧪
    TP report}
  \item
    \href{https://github.com/typst/packages/raw/main/packages/preview/silky-report-insa/0.4.0/\#\%F0\%9F\%93\%9A-internship-report}{ðŸ``š
    Internship report}
  \item
    \href{https://github.com/typst/packages/raw/main/packages/preview/silky-report-insa/0.4.0/\#\%F0\%9F\%97\%92\%EF\%B8\%8F-blank-templates}{ðŸ---'ï¸?
    Blank templates}
  \end{enumerate}
\item
  \href{https://github.com/typst/packages/raw/main/packages/preview/silky-report-insa/0.4.0/\#fonts}{Fonts
  information}
\item
  \href{https://github.com/typst/packages/raw/main/packages/preview/silky-report-insa/0.4.0/\#notes}{Notes}
\item
  \href{https://github.com/typst/packages/raw/main/packages/preview/silky-report-insa/0.4.0/\#license}{License}
\item
  \href{https://github.com/typst/packages/raw/main/packages/preview/silky-report-insa/0.4.0/\#changelog}{Changelog}
\end{enumerate}

\subsection{Examples \& Usage}\label{examples-usage}

\subsubsection{🧪 TP report}\label{uxf0uxffuxaa-tp-report}

\pandocbounded{\includegraphics[keepaspectratio]{https://github.com/typst/packages/raw/main/packages/preview/silky-report-insa/0.4.0/thumbnail-insa-report.png}}

This is the default report for the \texttt{\ silky-report-insa\ }
package. It uses the \texttt{\ insa-report\ } show rule.\\
It is primarily used for reports of Practical Works (Travaux Pratiques).

\paragraph{Example}\label{example}

\begin{Shaded}
\begin{Highlighting}[]
\NormalTok{\#import "@preview/silky{-}report{-}insa:0.4.0": *}
\NormalTok{\#show: doc =\textgreater{} insa{-}report(}
\NormalTok{  id: 3,}
\NormalTok{  pre{-}title: "STPI 2",}
\NormalTok{  title: "Interférences et diffraction",}
\NormalTok{  authors: [}
\NormalTok{    *LE JEUNE Youenn*}

\NormalTok{    *MAUVY Eva*}
    
\NormalTok{    Groupe D}

\NormalTok{    Binôme 5}
\NormalTok{  ],}
\NormalTok{  date: "11/04/2023",}
\NormalTok{  insa: "rennes",}
\NormalTok{  doc)}

\NormalTok{= Introduction}
\NormalTok{Le but de ce TP est d’interpréter les figures de diffraction observées avec différents objets diffractants}
\NormalTok{et d’en déduire les dimensions de ces objets.}

\NormalTok{= Partie théorique {-} Phénomène d\textquotesingle{}interférence}
\NormalTok{== Diffraction par une fente double}
\NormalTok{Lors du passage de la lumière par une fente double de largeur $a$ et de distance $b$ entre les centres}
\NormalTok{des fentes...}
\end{Highlighting}
\end{Shaded}

\paragraph{Parameters}\label{parameters}

\begin{longtable}[]{@{}llll@{}}
\toprule\noalign{}
Parameter & Description & Type & Example \\
\midrule\noalign{}
\endhead
\bottomrule\noalign{}
\endlastfoot
\textbf{id} & TP number & int & \texttt{\ 1\ } \\
\textbf{pre-title} & Text written before the title & str &
\texttt{\ "STPI\ 2"\ } \\
\textbf{title} & Title of the TP & str &
\texttt{\ "Interférences\ et\ diffraction"\ } \\
\textbf{authors} & Authors & content &
\texttt{\ {[}\textbackslash{}*LE\ JEUNE\ Youenn\textbackslash{}*{]}\ } \\
\textbf{date} & Date of the TP & datetime/str &
\texttt{\ "11/04/2023"\ } \\
\textbf{insa} & INSA name ( \texttt{\ rennes\ } , \texttt{\ hdf\ } …)
& str & \texttt{\ "rennes"\ } \\
\textbf{lang} & Language & str & \texttt{\ "fr"\ } \\
\end{longtable}

\subsubsection{ðŸ``š Internship
report}\label{uxf0uxffux161-internship-report}

\pandocbounded{\includegraphics[keepaspectratio]{https://github.com/typst/packages/raw/main/packages/preview/silky-report-insa/0.4.0/thumbnail-insa-stage.png}}

If you want to make an internship report, you will need to use another
show rule: \texttt{\ insa-stage\ } .

\paragraph{Example}\label{example-1}

\begin{Shaded}
\begin{Highlighting}[]
\NormalTok{\#import "@preview/silky{-}report{-}insa:0.4.0": *}
\NormalTok{\#show: doc =\textgreater{} insa{-}stage(}
\NormalTok{  "Youenn LE JEUNE",}
\NormalTok{  "INFO",}
\NormalTok{  "2023{-}2024",}
\NormalTok{  "Real{-}time virtual interaction with deformable structure",}
\NormalTok{  "Sapienza University of Rome",}
\NormalTok{  image("logo{-}example.png"),}
\NormalTok{  "Marilena VENDITELLI",}
\NormalTok{  "Bertrand COUASNON",}
\NormalTok{  [}
\NormalTok{    Résumé du stage en français.}
\NormalTok{  ],}
\NormalTok{  [}
\NormalTok{    Summary of the internship in english.}
\NormalTok{  ],}
\NormalTok{  insa: "rennes",}
\NormalTok{  lang: "fr",}
\NormalTok{  doc}
\NormalTok{)}

\NormalTok{= Introduction}
\NormalTok{Présentation de l\textquotesingle{}entreprise, tout ça tout ça.}

\NormalTok{\#pagebreak()}
\NormalTok{= Travail réalisé}
\NormalTok{== Première partie}
\NormalTok{Blabla}

\NormalTok{== Seconde partie}
\NormalTok{Bleble}

\NormalTok{\#pagebreak()}
\NormalTok{= Conclusion}
\NormalTok{Conclusion random}

\NormalTok{\#pagebreak()}
\NormalTok{= Annexes}
\end{Highlighting}
\end{Shaded}

This template can also be used for a report that is written in english:
in this case, add the \texttt{\ lang:\ "en"\ } parameter to the function
call in the show rule.

\paragraph{Parameters}\label{parameters-1}

\begin{longtable}[]{@{}lllll@{}}
\toprule\noalign{}
\textbf{Parameter} & Required & Type & Description & Example \\
\midrule\noalign{}
\endhead
\bottomrule\noalign{}
\endlastfoot
\textbf{name} & yes & str & Name of the student &
\texttt{\ "Youenn\ LE\ JEUNE"\ } \\
\textbf{department} & yes & str & Department of the student &
\texttt{\ "INFO"\ } \\
\textbf{year} & yes & str & School year during the internship &
\texttt{\ "2023-2024"\ } \\
\textbf{title} & yes & str & Title of the internship &
\texttt{\ "Real-time\ virtual\ interaction\ with\ deformable\ structure"\ } \\
\textbf{company} & yes & str & Company &
\texttt{\ Sapienza\ University\ of\ Rome\ } \\
\textbf{company-logo} & yes & content & Logo of the company &
\texttt{\ image("logo-example.png")\ } \\
\textbf{company-tutor} & yes & str & Tutor in the company &
\texttt{\ "Marilena\ VENDITELLI"\ } \\
\textbf{insa-tutor} & yes & str & Tutor at INSA &
\texttt{\ "Bertrand\ COUASNON"\ } \\
\textbf{insa-tutor-suffix} & no & str & Suffix at the end of
“encadrant� in french & \texttt{\ "e"\ } \\
\textbf{summary-french} & yes & content & Summary in French &
\texttt{\ {[}\ Résumé\ du\ stage\ en\ français.\ {]}\ } \\
\textbf{summary-english} & yes & content & Summary in English &
\texttt{\ {[}\ Summary\ of\ the\ internship\ in\ english.\ {]}\ } \\
\textbf{student-suffix} & no & str & Suffix at the end of
“ingénieur� in french & \texttt{\ "e"\ } \\
\textbf{thanks-page} & no & content & Special thanks page. &
\texttt{\ {[}\ Thanks\ to\ my\ *supervisor*,\ blah\ blah\ blah.\ {]}\ } \\
\textbf{omit-outline} & no & bool & Whether to skip the outline page or
not & \texttt{\ false\ } \\
\textbf{insa} & no & str & INSA name ( \texttt{\ rennes\ } ,
\texttt{\ hdf\ } …) & \texttt{\ "rennes"\ } \\
\textbf{lang} & no & str & Language of the template. Some strings are
translated. & \texttt{\ "fr"\ } \\
\end{longtable}

\subsubsection{ðŸ---'ï¸? Blank
templates}\label{uxf0uxffuxef-blank-templates}

\pandocbounded{\includegraphics[keepaspectratio]{https://github.com/typst/packages/raw/main/packages/preview/silky-report-insa/0.4.0/thumbnail-insa-document.png}}

If you do not want the preformatted output with “TP x�, the title
and date in the header, etc. you can simply use the
\texttt{\ insa-document\ } show rule and customize all by yourself.

\paragraph{Blank template types}\label{blank-template-types}

The graphic charter provides 3 different document types, that are
translated in this Typst template under those names:

\begin{itemize}
\tightlist
\item
  \textbf{\texttt{\ light\ }} , which does not have many color and can
  be printed easily. Has 3 spots to write on the cover:
  \texttt{\ cover-top-left\ } , \texttt{\ cover-middle-left\ } and
  \texttt{\ cover-bottom-right\ } .
\item
  \textbf{\texttt{\ colored\ }} , which is beautiful but consumes a lot
  of ink to print. Only has 1 spot to write on the cover:
  \texttt{\ cover-top-left\ } .
\item
  \textbf{\texttt{\ pfe\ }} , which is primarily used for internship
  reports. Has 4 spots to write on both the front and back covers:
  \texttt{\ cover-top-left\ } , \texttt{\ cover-middle-left\ } ,
  \texttt{\ cover-bottom-right\ } and \texttt{\ back-cover\ } .
\end{itemize}

The document type must be the first argument of the
\texttt{\ insa-document\ } function.

\paragraph{Example}\label{example-2}

\begin{Shaded}
\begin{Highlighting}[]
\NormalTok{\#import "@preview/silky{-}report{-}insa:0.4.0": *}
\NormalTok{\#show: doc =\textgreater{} insa{-}document(}
\NormalTok{  "light",}
\NormalTok{  cover{-}top{-}left: [*Document important*],}
\NormalTok{  cover{-}middle{-}left: [}
\NormalTok{    NOM Prénom}

\NormalTok{    Département INFO}
\NormalTok{  ],}
\NormalTok{  cover{-}bottom{-}right: "uwu",}
\NormalTok{  page{-}header: "En{-}tête au pif",}
\NormalTok{  doc}
\NormalTok{)}
\end{Highlighting}
\end{Shaded}

\paragraph{Parameters}\label{parameters-2}

\begin{longtable}[]{@{}lll@{}}
\toprule\noalign{}
\textbf{Parameter} & Type & Description \\
\midrule\noalign{}
\endhead
\bottomrule\noalign{}
\endlastfoot
\textbf{cover-type} & str & ( \textbf{REQUIRED} ) Type of cover.
Available are: light, colored, pfe. \\
\textbf{cover-top-left} & content & \\
\textbf{cover-middle-left} & content & \\
\textbf{cover-bottom-right} & content & \\
\textbf{back-cover} & content & What to display on the back cover. \\
\textbf{page-header} & content & Header of the pages (except the front
and back). If \texttt{\ none\ } , will display the INSA logo. If not
empty, will display the passed content with an underline. \\
\textbf{page-footer} & content & Footer of the pages (except the front
and back). The page counter will be displayed at the right of the
footer, except if the page number is 0. \\
\textbf{include-back-cover} & bool & whether to add the back cover or
not. \\
\textbf{insa} & str & INSA name ( \texttt{\ rennes\ } , \texttt{\ hdf\ }
…) \\
\textbf{lang} & str & Language of the template. Some strings are
translated. \\
\textbf{metadata-title} & content & Title of the document that will be
embedded in the PDF metadata. \\
\textbf{metadata-authors} & str list & Authors that will be embedded in
the PDF metadata. \\
\textbf{metadata-date} & datetime & Date that will be set as the
document creation date. If not specified, will be set to now. \\
\end{longtable}

\subsection{Fonts}\label{fonts}

The graphic charter recommends the fonts \textbf{League Spartan} for
headings and \textbf{Source Serif} for regular text. To have the best
look, you should install those fonts.

\begin{quote}
You can download the fonts from
\href{https://github.com/SkytAsul/INSA-Typst-Template/tree/main/fonts}{here}
.
\end{quote}

To behave correctly on computers lacking those specific fonts, this
template will automatically fallback to similar ones:

\begin{itemize}
\tightlist
\item
  \textbf{League Spartan} -\textgreater{} \textbf{Arial} (approved by
  INSA’s graphic charter, by default in Windows) -\textgreater{}
  \textbf{Liberation Sans} (by default in most Linux)
\item
  \textbf{Source Serif} -\textgreater{} \textbf{Source Serif 4}
  (downloadable for free) -\textgreater{} \textbf{Georgia} (approved by
  the graphic charter) -\textgreater{} \textbf{Linux Libertine} (default
  Typst font)
\end{itemize}

\subsubsection{Note on variable fonts}\label{note-on-variable-fonts}

If you want to install those fonts on your computer, Typst might not
recognize them if you install their \emph{Variable} versions. You should
install the static versions ( \textbf{League Spartan Bold} and most
versions of \textbf{Source Serif} ).

Keep an eye on \href{https://github.com/typst/typst/issues/185}{the
issue in Typst bug tracker} to see when variable fonts will be used!

\subsection{Notes}\label{notes}

This template is being developed by Youenn LE JEUNE from the INSA de
Rennes in \href{https://github.com/SkytAsul/INSA-Typst-Template}{this
repository} .

For now it includes assets from the graphic charters of those INSAs:

\begin{itemize}
\tightlist
\item
  Rennes ( \texttt{\ rennes\ } )
\item
  Hauts de France ( \texttt{\ hdf\ } )
\item
  Centre Val de Loire ( \texttt{\ cvl\ } ) Users from other INSAs can
  open a pull request on the repository with the assets for their INSA.
\end{itemize}

If you have any other feature request, open an issue on the repository.

\subsection{License}\label{license}

The typst template is licensed under the
\href{https://github.com/SkytAsul/INSA-Typst-Template/blob/main/LICENSE}{MIT
license} . This does \emph{not} apply to the image assets. Those image
files are property of Groupe INSA.

\subsection{Changelog}\label{changelog}

\subsubsection{0.4.0}\label{section}

\begin{itemize}
\tightlist
\item
  Added INSA CVL assets
\item
  Added \texttt{\ insa-tutor-suffix\ } option to \texttt{\ insa-stage\ }
\end{itemize}

\subsubsection{0.3.1}\label{section-1}

\begin{itemize}
\tightlist
\item
  Added \texttt{\ insa\ } option to all templates
\item
  Added INSA HdF assets
\item
  Added \texttt{\ student-suffix\ } option to \texttt{\ insa-stage\ }
\item
  Made outline not shown in outline
\end{itemize}

\subsubsection{0.3.0}\label{section-2}

\begin{itemize}
\tightlist
\item
  Added \texttt{\ omit-outline\ } option to \texttt{\ insa-stage\ }
\item
  Added \texttt{\ thanks-page\ } parameter to \texttt{\ insa-stage\ }
\item
  Added metadata-related options to \texttt{\ insa-document\ }
\item
  Made some PDF metadata automatically exported for
  \texttt{\ insa-stage\ } and \texttt{\ insa-report\ }
\item
  Made page number not displayed if equals to 0
\item
  Adjusted positions of elements in back covers
\item
  Fixed some translations
\item
  Updated README to have changelog, visual examples of all documents and
  parameters table
\end{itemize}

\href{/app?template=silky-report-insa&version=0.4.0}{Create project in
app}

\subsubsection{How to use}\label{how-to-use}

Click the button above to create a new project using this template in
the Typst app.

You can also use the Typst CLI to start a new project on your computer
using this command:

\begin{verbatim}
typst init @preview/silky-report-insa:0.4.0
\end{verbatim}

\includesvg[width=0.16667in,height=0.16667in]{/assets/icons/16-copy.svg}

\subsubsection{About}\label{about}

\begin{description}
\tightlist
\item[Author :]
SkytAsul
\item[License:]
MIT
\item[Current version:]
0.4.0
\item[Last updated:]
November 21, 2024
\item[First released:]
March 19, 2024
\item[Archive size:]
4.48 MB
\href{https://packages.typst.org/preview/silky-report-insa-0.4.0.tar.gz}{\pandocbounded{\includesvg[keepaspectratio]{/assets/icons/16-download.svg}}}
\item[Repository:]
\href{https://github.com/SkytAsul/INSA-Typst-Template}{GitHub}
\item[Discipline s :]
\begin{itemize}
\tightlist
\item[]
\item
  \href{https://typst.app/universe/search/?discipline=engineering}{Engineering}
\item
  \href{https://typst.app/universe/search/?discipline=computer-science}{Computer
  Science}
\item
  \href{https://typst.app/universe/search/?discipline=mathematics}{Mathematics}
\item
  \href{https://typst.app/universe/search/?discipline=physics}{Physics}
\item
  \href{https://typst.app/universe/search/?discipline=education}{Education}
\end{itemize}
\item[Categor y :]
\begin{itemize}
\tightlist
\item[]
\item
  \pandocbounded{\includesvg[keepaspectratio]{/assets/icons/16-speak.svg}}
  \href{https://typst.app/universe/search/?category=report}{Report}
\end{itemize}
\end{description}

\subsubsection{Where to report issues?}\label{where-to-report-issues}

This template is a project of SkytAsul . Report issues on
\href{https://github.com/SkytAsul/INSA-Typst-Template}{their repository}
. You can also try to ask for help with this template on the
\href{https://forum.typst.app}{Forum} .

Please report this template to the Typst team using the
\href{https://typst.app/contact}{contact form} if you believe it is a
safety hazard or infringes upon your rights.

\phantomsection\label{versions}
\subsubsection{Version history}\label{version-history}

\begin{longtable}[]{@{}ll@{}}
\toprule\noalign{}
Version & Release Date \\
\midrule\noalign{}
\endhead
\bottomrule\noalign{}
\endlastfoot
0.4.0 & November 21, 2024 \\
\href{https://typst.app/universe/package/silky-report-insa/0.3.1/}{0.3.1}
& September 24, 2024 \\
\href{https://typst.app/universe/package/silky-report-insa/0.3.0/}{0.3.0}
& August 7, 2024 \\
\href{https://typst.app/universe/package/silky-report-insa/0.2.1/}{0.2.1}
& July 24, 2024 \\
\href{https://typst.app/universe/package/silky-report-insa/0.2.0/}{0.2.0}
& June 10, 2024 \\
\href{https://typst.app/universe/package/silky-report-insa/0.1.0/}{0.1.0}
& March 19, 2024 \\
\end{longtable}

Typst GmbH did not create this template and cannot guarantee correct
functionality of this template or compatibility with any version of the
Typst compiler or app.


\title{typst.app/universe/package/curryst}

\phantomsection\label{banner}
\section{curryst}\label{curryst}

{ 0.3.0 }

Typeset trees of inference rules.

{ } Featured Package

\phantomsection\label{readme}
A Typst package for typesetting proof trees.

\subsection{Import}\label{import}

You can import the latest version of this package with:

\begin{Shaded}
\begin{Highlighting}[]
\NormalTok{\#import "@preview/curryst:0.3.0": rule, proof{-}tree}
\end{Highlighting}
\end{Shaded}

\subsection{Basic usage}\label{basic-usage}

To display a proof tree, you first need to create a tree, using the
\texttt{\ rule\ } function. Its first argument is the conclusion, and
the other positional arguments are the premises. It also accepts a
\texttt{\ name\ } for the rule name, displayed on the right of the bar,
as well as a \texttt{\ label\ } , displayed on the left of the bar.

\begin{Shaded}
\begin{Highlighting}[]
\NormalTok{\#let tree = rule(}
\NormalTok{  label: [Label],}
\NormalTok{  name: [Rule name],}
\NormalTok{  [Conclusion],}
\NormalTok{  [Premise 1],}
\NormalTok{  [Premise 2],}
\NormalTok{  [Premise 3]}
\NormalTok{)}
\end{Highlighting}
\end{Shaded}

Then, you can display the tree with the \texttt{\ proof-tree\ }
function:

\begin{Shaded}
\begin{Highlighting}[]
\NormalTok{\#proof{-}tree(tree)}
\end{Highlighting}
\end{Shaded}

In this case, we get the following result:

\pandocbounded{\includesvg[keepaspectratio]{https://github.com/typst/packages/raw/main/packages/preview/curryst/0.3.0/examples/usage.svg}}

Proof trees can be part of mathematical formulas:

\begin{Shaded}
\begin{Highlighting}[]
\NormalTok{Consider the following tree:}
\NormalTok{$}
\NormalTok{  Pi quad = quad \#proof{-}tree(}
\NormalTok{    rule(}
\NormalTok{      $phi$,}
\NormalTok{      $Pi\_1$,}
\NormalTok{      $Pi\_2$,}
\NormalTok{    )}
\NormalTok{  )}
\NormalTok{$}
\NormalTok{$Pi$ constitutes a derivation of $phi$.s}
\end{Highlighting}
\end{Shaded}

\pandocbounded{\includesvg[keepaspectratio]{https://github.com/typst/packages/raw/main/packages/preview/curryst/0.3.0/examples/math-formula.svg}}

You can specify a rule as the premises of a rule in order to create a
tree:

\begin{Shaded}
\begin{Highlighting}[]
\NormalTok{\#proof{-}tree(}
\NormalTok{  rule(}
\NormalTok{    name: $R$,}
\NormalTok{    $C\_1 or C\_2 or C\_3$,}
\NormalTok{    rule(}
\NormalTok{      name: $A$,}
\NormalTok{      $C\_1 or C\_2 or L$,}
\NormalTok{      rule(}
\NormalTok{        $C\_1 or L$,}
\NormalTok{        $Pi\_1$,}
\NormalTok{      ),}
\NormalTok{    ),}
\NormalTok{    rule(}
\NormalTok{      $C\_2 or overline(L)$,}
\NormalTok{      $Pi\_2$,}
\NormalTok{    ),}
\NormalTok{  )}
\NormalTok{)}
\end{Highlighting}
\end{Shaded}

\pandocbounded{\includesvg[keepaspectratio]{https://github.com/typst/packages/raw/main/packages/preview/curryst/0.3.0/examples/rule-as-premise.svg}}

As an example, here is a natural deduction proof tree generated with
Curryst:

\pandocbounded{\includesvg[keepaspectratio]{https://github.com/typst/packages/raw/main/packages/preview/curryst/0.3.0/examples/natural-deduction.svg}}

Show code

\begin{Shaded}
\begin{Highlighting}[]
\NormalTok{\#let ax = rule.with(name: [ax])}
\NormalTok{\#let and{-}el = rule.with(name: $and\_e\^{}ell$)}
\NormalTok{\#let and{-}er = rule.with(name: $and\_e\^{}r$)}
\NormalTok{\#let impl{-}i = rule.with(name: $scripts({-}\textgreater{})\_i$)}
\NormalTok{\#let impl{-}e = rule.with(name: $scripts({-}\textgreater{})\_e$)}
\NormalTok{\#let not{-}i = rule.with(name: $not\_i$)}
\NormalTok{\#let not{-}e = rule.with(name: $not\_e$)}

\NormalTok{\#proof{-}tree(}
\NormalTok{  impl{-}i(}
\NormalTok{    $tack (p {-}\textgreater{} q) {-}\textgreater{} not (p and not q)$,}
\NormalTok{    not{-}i(}
\NormalTok{      $p {-}\textgreater{} q tack  not (p and not q)$,}
\NormalTok{      not{-}e(}
\NormalTok{        $ underbrace(p {-}\textgreater{} q\textbackslash{}, p and not q, Gamma) tack bot $,}
\NormalTok{        impl{-}e(}
\NormalTok{          $Gamma tack q$,}
\NormalTok{          ax($Gamma tack p {-}\textgreater{} q$),}
\NormalTok{          and{-}el(}
\NormalTok{            $Gamma tack p$,}
\NormalTok{            ax($Gamma tack p and not q$),}
\NormalTok{          ),}
\NormalTok{        ),}
\NormalTok{        and{-}er(}
\NormalTok{          $Gamma tack not q$,}
\NormalTok{          ax($Gamma tack p and not q$),}
\NormalTok{        ),}
\NormalTok{      ),}
\NormalTok{    ),}
\NormalTok{  )}
\NormalTok{)}
\end{Highlighting}
\end{Shaded}

\subsection{Advanced usage}\label{advanced-usage}

The \texttt{\ proof-tree\ } function accepts multiple named arguments
that let you customize the tree:

\begin{description}
\tightlist
\item[\texttt{\ prem-min-spacing\ }]
The minimum amount of space between two premises.
\item[\texttt{\ title-inset\ }]
The amount width with which to extend the horizontal bar beyond the
content. Also determines how far from the bar labels and names are
displayed.
\item[\texttt{\ stroke\ }]
The stroke to use for the horizontal bars.
\item[\texttt{\ horizontal-spacing\ }]
The space between the bottom of the bar and the conclusion, and between
the top of the bar and the premises.
\item[\texttt{\ min-bar-height\ }]
The minimum height of the box containing the horizontal bar.
\end{description}

For more information, please refer to the documentation in
\href{https://github.com/typst/packages/raw/main/packages/preview/curryst/0.3.0/curryst.typ}{\texttt{\ curryst.typ\ }}
.

\subsubsection{How to add}\label{how-to-add}

Copy this into your project and use the import as \texttt{\ curryst\ }

\begin{verbatim}
#import "@preview/curryst:0.3.0"
\end{verbatim}

\includesvg[width=0.16667in,height=0.16667in]{/assets/icons/16-copy.svg}

Check the docs for
\href{https://typst.app/docs/reference/scripting/\#packages}{more
information on how to import packages} .

\subsubsection{About}\label{about}

\begin{description}
\tightlist
\item[Author s :]
\href{https://github.com/remih23}{Rémi Hutin} ,
\href{https://github.com/pauladam94}{Paul Adam} , \&
\href{https://github.com/MDLC01}{Malo}
\item[License:]
MIT
\item[Current version:]
0.3.0
\item[Last updated:]
April 16, 2024
\item[First released:]
December 7, 2023
\item[Minimum Typst version:]
0.11.0
\item[Archive size:]
4.71 kB
\href{https://packages.typst.org/preview/curryst-0.3.0.tar.gz}{\pandocbounded{\includesvg[keepaspectratio]{/assets/icons/16-download.svg}}}
\item[Repository:]
\href{https://github.com/pauladam94/curryst}{GitHub}
\item[Discipline s :]
\begin{itemize}
\tightlist
\item[]
\item
  \href{https://typst.app/universe/search/?discipline=computer-science}{Computer
  Science}
\item
  \href{https://typst.app/universe/search/?discipline=mathematics}{Mathematics}
\end{itemize}
\item[Categor ies :]
\begin{itemize}
\tightlist
\item[]
\item
  \pandocbounded{\includesvg[keepaspectratio]{/assets/icons/16-package.svg}}
  \href{https://typst.app/universe/search/?category=components}{Components}
\item
  \pandocbounded{\includesvg[keepaspectratio]{/assets/icons/16-chart.svg}}
  \href{https://typst.app/universe/search/?category=visualization}{Visualization}
\item
  \pandocbounded{\includesvg[keepaspectratio]{/assets/icons/16-integration.svg}}
  \href{https://typst.app/universe/search/?category=integration}{Integration}
\end{itemize}
\end{description}

\subsubsection{Where to report issues?}\label{where-to-report-issues}

This package is a project of Rémi Hutin, Paul Adam, and Malo . Report
issues on \href{https://github.com/pauladam94/curryst}{their repository}
. You can also try to ask for help with this package on the
\href{https://forum.typst.app}{Forum} .

Please report this package to the Typst team using the
\href{https://typst.app/contact}{contact form} if you believe it is a
safety hazard or infringes upon your rights.

\phantomsection\label{versions}
\subsubsection{Version history}\label{version-history}

\begin{longtable}[]{@{}ll@{}}
\toprule\noalign{}
Version & Release Date \\
\midrule\noalign{}
\endhead
\bottomrule\noalign{}
\endlastfoot
0.3.0 & April 16, 2024 \\
\href{https://typst.app/universe/package/curryst/0.2.0/}{0.2.0} & March
19, 2024 \\
\href{https://typst.app/universe/package/curryst/0.1.1/}{0.1.1} &
January 31, 2024 \\
\href{https://typst.app/universe/package/curryst/0.1.0/}{0.1.0} &
December 7, 2023 \\
\end{longtable}

Typst GmbH did not create this package and cannot guarantee correct
functionality of this package or compatibility with any version of the
Typst compiler or app.


\title{typst.app/universe/package/touying}

\phantomsection\label{banner}
\section{touying}\label{touying}

{ 0.5.3 }

A powerful package for creating presentation slides in Typst.

{ } Featured Package

\phantomsection\label{readme}
\href{https://github.com/touying-typ/touying}{Touying} (投影 in
chinese, /tóuyÇ?ng/, meaning projection) is a user-friendly, powerful
and efficient package for creating presentation slides in Typst. Partial
code is inherited from
\href{https://github.com/andreasKroepelin/polylux}{Polylux} . Therefore,
some concepts and APIs remain consistent with Polylux.

Touying provides automatically injected global configurations, which is
convenient for configuring themes. Besides, Touying does not rely on
\texttt{\ counter\ } and \texttt{\ context\ } to implement
\texttt{\ \#pause\ } , resulting in better performance.

If you like it, consider
\href{https://github.com/touying-typ/touying}{giving a star on GitHub} .
Touying is a community-driven project, feel free to suggest any ideas
and contribute.

\href{https://touying-typ.github.io/}{\pandocbounded{\includegraphics[keepaspectratio]{https://img.shields.io/badge/docs-book-green}}}
\href{https://github.com/touying-typ/touying/wiki}{\pandocbounded{\includegraphics[keepaspectratio]{https://img.shields.io/badge/docs-gallery-orange}}}
\pandocbounded{\includegraphics[keepaspectratio]{https://img.shields.io/github/license/touying-typ/touying}}
\pandocbounded{\includegraphics[keepaspectratio]{https://img.shields.io/github/v/release/touying-typ/touying}}
\pandocbounded{\includegraphics[keepaspectratio]{https://img.shields.io/github/stars/touying-typ/touying}}
\pandocbounded{\includegraphics[keepaspectratio]{https://img.shields.io/badge/themes-6-aqua}}

\subsection{Document}\label{document}

Read \href{https://touying-typ.github.io/}{the document} to learn all
about Touying.

We will maintain \textbf{English} and \textbf{Chinese} versions of the
documentation for Touying, and for each major version, we will maintain
a documentation copy. This allows you to easily refer to old versions of
the Touying documentation and migrate to new versions.

\textbf{Note that the documentation may be outdated, and you can also
use Tinymist to view Touying’s annotated documentation by hovering
over the code.}

\subsection{Gallery}\label{gallery}

Touying offers \href{https://github.com/touying-typ/touying/wiki}{a
gallery page} via wiki, where you can browse elegant slides created by
Touying users. You’re also encouraged to contribute your own beautiful
slides here!

\subsection{Special Features}\label{special-features}

\begin{enumerate}
\tightlist
\item
  Split slides by headings
  \href{https://touying-typ.github.io/docs/sections}{document}
\end{enumerate}

\begin{Shaded}
\begin{Highlighting}[]
\NormalTok{= Section}

\NormalTok{== Subsection}

\NormalTok{=== First Slide}

\NormalTok{Hello, Touying!}

\NormalTok{=== Second Slide}

\NormalTok{Hello, Typst!}
\end{Highlighting}
\end{Shaded}

\begin{enumerate}
\setcounter{enumi}{1}
\tightlist
\item
  \texttt{\ \#pause\ } and \texttt{\ \#meanwhile\ } animations
  \href{https://touying-typ.github.io/docs/dynamic/simple}{document}
\end{enumerate}

\begin{Shaded}
\begin{Highlighting}[]
\NormalTok{\#slide[}
\NormalTok{  First}

\NormalTok{  \#pause}

\NormalTok{  Second}

\NormalTok{  \#meanwhile}

\NormalTok{  Third}

\NormalTok{  \#pause}

\NormalTok{  Fourth}
\NormalTok{]}
\end{Highlighting}
\end{Shaded}

\pandocbounded{\includegraphics[keepaspectratio]{https://github.com/touying-typ/touying/assets/34951714/24ca19a3-b27c-4d31-ab75-09c37911e6ac}}

\begin{enumerate}
\setcounter{enumi}{2}
\tightlist
\item
  Math Equation Animation
  \href{https://touying-typ.github.io/docs/dynamic/equation}{document}
\end{enumerate}

\pandocbounded{\includegraphics[keepaspectratio]{https://github.com/touying-typ/touying/assets/34951714/8640fe0a-95e4-46ac-b570-c8c79f993de4}}

\begin{enumerate}
\setcounter{enumi}{3}
\tightlist
\item
  \texttt{\ touying-reducer\ } Cetz and Fletcher Animations
  \href{https://touying-typ.github.io/docs/dynamic/other}{document}
\end{enumerate}

\pandocbounded{\includegraphics[keepaspectratio]{https://github.com/touying-typ/touying/assets/34951714/9ba71f54-2a5d-4144-996c-4a42833cc5cc}}

\begin{enumerate}
\setcounter{enumi}{4}
\tightlist
\item
  Correct outline and bookmark (no duplicate and correct page number)
\end{enumerate}

\pandocbounded{\includegraphics[keepaspectratio]{https://github.com/touying-typ/touying/assets/34951714/7b62fcaf-6342-4dba-901b-818c16682529}}

\begin{enumerate}
\setcounter{enumi}{5}
\tightlist
\item
  Dewdrop Theme Navigation Bar
  \href{https://touying-typ.github.io/docs/themes/dewdrop}{document}
\end{enumerate}

\pandocbounded{\includegraphics[keepaspectratio]{https://github.com/touying-typ/touying/assets/34951714/0426516d-aa3c-4b7a-b7b6-2d5d276fb971}}

\begin{enumerate}
\setcounter{enumi}{6}
\tightlist
\item
  Semi-transparent cover mode
  \href{https://touying-typ.github.io/docs/dynamic/cover}{document}
\end{enumerate}

\pandocbounded{\includegraphics[keepaspectratio]{https://github.com/touying-typ/touying/assets/34951714/22a9ea66-c8b5-431e-a52c-2c8ca3f18e49}}

\begin{enumerate}
\setcounter{enumi}{7}
\tightlist
\item
  Speaker notes for dual-screen
  \href{https://touying-typ.github.io/docs/external/pympress}{document}
\end{enumerate}

\pandocbounded{\includegraphics[keepaspectratio]{https://github.com/touying-typ/touying/assets/34951714/afbe17cb-46d4-4507-90e8-959c53de95d5}}

\begin{enumerate}
\setcounter{enumi}{8}
\tightlist
\item
  Export slides to PPTX and HTML formats and show presentation online.
  \href{https://github.com/touying-typ/touying-exporter}{touying-exporter}
  \href{https://github.com/touying-typ/touying-template}{touying-template}
  \href{https://touying-typ.github.io/touying-template/}{online}
\end{enumerate}

\pandocbounded{\includegraphics[keepaspectratio]{https://github.com/touying-typ/touying-exporter/assets/34951714/207ddffc-87c8-4976-9bf4-4c6c5e2573ea}}

\subsection{Quick start}\label{quick-start}

Before you begin, make sure you have installed the Typst environment. If
not, you can use the \href{https://typst.app/}{Web App} or the
\href{https://marketplace.visualstudio.com/items?itemName=myriad-dreamin.tinymist}{Tinymist
LSP} extensions for VS Code.

To use Touying, you only need to include the following code in your
document:

\begin{Shaded}
\begin{Highlighting}[]
\NormalTok{\#import "@preview/touying:0.5.3": *}
\NormalTok{\#import themes.simple: *}

\NormalTok{\#show: simple{-}theme.with(aspect{-}ratio: "16{-}9")}

\NormalTok{= Title}

\NormalTok{== First Slide}

\NormalTok{Hello, Touying!}

\NormalTok{\#pause}

\NormalTok{Hello, Typst!}
\end{Highlighting}
\end{Shaded}

\pandocbounded{\includegraphics[keepaspectratio]{https://github.com/touying-typ/touying/assets/34951714/f5bdbf8f-7bf9-45fd-9923-0fa5d66450b2}}

It’s simple. Congratulations on creating your first Touying slide!
🎉

\textbf{Tip:} You can use Typst syntax like
\texttt{\ \#import\ "config.typ":\ *\ } or
\texttt{\ \#include\ "content.typ"\ } to implement Touying’s
multi-file architecture.

\subsection{More Complex Examples}\label{more-complex-examples}

In fact, Touying provides various styles for writing slides. For
example, the above example uses first-level and second-level titles to
create new slides. However, you can also use the
\texttt{\ \#slide{[}..{]}\ } format to access more powerful features
provided by Touying.

\begin{Shaded}
\begin{Highlighting}[]
\NormalTok{\#import "@preview/touying:0.5.3": *}
\NormalTok{\#import themes.university: *}
\NormalTok{\#import "@preview/cetz:0.2.2"}
\NormalTok{\#import "@preview/fletcher:0.5.1" as fletcher: node, edge}
\NormalTok{\#import "@preview/ctheorems:1.1.2": *}
\NormalTok{\#import "@preview/numbly:0.1.0": numbly}

\NormalTok{// cetz and fletcher bindings for touying}
\NormalTok{\#let cetz{-}canvas = touying{-}reducer.with(reduce: cetz.canvas, cover: cetz.draw.hide.with(bounds: true))}
\NormalTok{\#let fletcher{-}diagram = touying{-}reducer.with(reduce: fletcher.diagram, cover: fletcher.hide)}

\NormalTok{// Theorems configuration by ctheorems}
\NormalTok{\#show: thmrules.with(qed{-}symbol: $square$)}
\NormalTok{\#let theorem = thmbox("theorem", "Theorem", fill: rgb("\#eeffee"))}
\NormalTok{\#let corollary = thmplain(}
\NormalTok{  "corollary",}
\NormalTok{  "Corollary",}
\NormalTok{  base: "theorem",}
\NormalTok{  titlefmt: strong}
\NormalTok{)}
\NormalTok{\#let definition = thmbox("definition", "Definition", inset: (x: 1.2em, top: 1em))}
\NormalTok{\#let example = thmplain("example", "Example").with(numbering: none)}
\NormalTok{\#let proof = thmproof("proof", "Proof")}

\NormalTok{\#show: university{-}theme.with(}
\NormalTok{  aspect{-}ratio: "16{-}9",}
\NormalTok{  // config{-}common(handout: true),}
\NormalTok{  config{-}info(}
\NormalTok{    title: [Title],}
\NormalTok{    subtitle: [Subtitle],}
\NormalTok{    author: [Authors],}
\NormalTok{    date: datetime.today(),}
\NormalTok{    institution: [Institution],}
\NormalTok{    logo: emoji.school,}
\NormalTok{  ),}
\NormalTok{)}

\NormalTok{\#set heading(numbering: numbly("\{1\}.", default: "1.1"))}

\NormalTok{\#title{-}slide()}

\NormalTok{== Outline \textless{}touying:hidden\textgreater{}}

\NormalTok{\#components.adaptive{-}columns(outline(title: none, indent: 1em))}

\NormalTok{= Animation}

\NormalTok{== Simple Animation}

\NormalTok{We can use \textasciigrave{}\#pause\textasciigrave{} to \#pause display something later.}

\NormalTok{\#pause}

\NormalTok{Just like this.}

\NormalTok{\#meanwhile}

\NormalTok{Meanwhile, \#pause we can also use \textasciigrave{}\#meanwhile\textasciigrave{} to \#pause display other content synchronously.}

\NormalTok{\#speaker{-}note[}
\NormalTok{  + This is a speaker note.}
\NormalTok{  + You won\textquotesingle{}t see it unless you use \textasciigrave{}config{-}common(show{-}notes{-}on{-}second{-}screen: right)\textasciigrave{}}
\NormalTok{]}


\NormalTok{== Complex Animation}

\NormalTok{At subslide \#touying{-}fn{-}wrapper((self: none) =\textgreater{} str(self.subslide)), we can}

\NormalTok{use \#uncover("2{-}")[\textasciigrave{}\#uncover\textasciigrave{} function] for reserving space,}

\NormalTok{use \#only("2{-}")[\textasciigrave{}\#only\textasciigrave{} function] for not reserving space,}

\NormalTok{\#alternatives[call \textasciigrave{}\#only\textasciigrave{} multiple times \textbackslash{}u\{2717\}][use \textasciigrave{}\#alternatives\textasciigrave{} function \#sym.checkmark] for choosing one of the alternatives.}


\NormalTok{== Callback Style Animation}

\NormalTok{\#slide(repeat: 3, self =\textgreater{} [}
\NormalTok{  \#let (uncover, only, alternatives) = utils.methods(self)}

\NormalTok{  At subslide \#self.subslide, we can}

\NormalTok{  use \#uncover("2{-}")[\textasciigrave{}\#uncover\textasciigrave{} function] for reserving space,}

\NormalTok{  use \#only("2{-}")[\textasciigrave{}\#only\textasciigrave{} function] for not reserving space,}

\NormalTok{  \#alternatives[call \textasciigrave{}\#only\textasciigrave{} multiple times \textbackslash{}u\{2717\}][use \textasciigrave{}\#alternatives\textasciigrave{} function \#sym.checkmark] for choosing one of the alternatives.}
\NormalTok{])}


\NormalTok{== Math Equation Animation}

\NormalTok{Equation with \textasciigrave{}pause\textasciigrave{}:}

\NormalTok{$}
\NormalTok{  f(x) \&= pause x\^{}2 + 2x + 1  \textbackslash{}}
\NormalTok{       \&= pause (x + 1)\^{}2  \textbackslash{}}
\NormalTok{$}

\NormalTok{\#meanwhile}

\NormalTok{Here, \#pause we have the expression of $f(x)$.}

\NormalTok{\#pause}

\NormalTok{By factorizing, we can obtain this result.}


\NormalTok{== CeTZ Animation}

\NormalTok{CeTZ Animation in Touying:}

\NormalTok{\#cetz{-}canvas(\{}
\NormalTok{  import cetz.draw: *}
  
\NormalTok{  rect((0,0), (5,5))}

\NormalTok{  (pause,)}

\NormalTok{  rect((0,0), (1,1))}
\NormalTok{  rect((1,1), (2,2))}
\NormalTok{  rect((2,2), (3,3))}

\NormalTok{  (pause,)}

\NormalTok{  line((0,0), (2.5, 2.5), name: "line")}
\NormalTok{\})}


\NormalTok{== Fletcher Animation}

\NormalTok{Fletcher Animation in Touying:}

\NormalTok{\#fletcher{-}diagram(}
\NormalTok{  node{-}stroke: .1em,}
\NormalTok{  node{-}fill: gradient.radial(blue.lighten(80\%), blue, center: (30\%, 20\%), radius: 80\%),}
\NormalTok{  spacing: 4em,}
\NormalTok{  edge(({-}1,0), "r", "{-}|\textgreater{}", \textasciigrave{}open(path)\textasciigrave{}, label{-}pos: 0, label{-}side: center),}
\NormalTok{  node((0,0), \textasciigrave{}reading\textasciigrave{}, radius: 2em),}
\NormalTok{  edge((0,0), (0,0), \textasciigrave{}read()\textasciigrave{}, "{-}{-}|\textgreater{}", bend: 130deg),}
\NormalTok{  pause,}
\NormalTok{  edge(\textasciigrave{}read()\textasciigrave{}, "{-}|\textgreater{}"),}
\NormalTok{  node((1,0), \textasciigrave{}eof\textasciigrave{}, radius: 2em),}
\NormalTok{  pause,}
\NormalTok{  edge(\textasciigrave{}close()\textasciigrave{}, "{-}|\textgreater{}"),}
\NormalTok{  node((2,0), \textasciigrave{}closed\textasciigrave{}, radius: 2em, extrude: ({-}2.5, 0)),}
\NormalTok{  edge((0,0), (2,0), \textasciigrave{}close()\textasciigrave{}, "{-}|\textgreater{}", bend: {-}40deg),}
\NormalTok{)}


\NormalTok{= Theorems}

\NormalTok{== Prime numbers}

\NormalTok{\#definition[}
\NormalTok{  A natural number is called a \#highlight[\_prime number\_] if it is greater}
\NormalTok{  than 1 and cannot be written as the product of two smaller natural numbers.}
\NormalTok{]}
\NormalTok{\#example[}
\NormalTok{  The numbers $2$, $3$, and $17$ are prime.}
\NormalTok{  @cor\_largest\_prime shows that this list is not exhaustive!}
\NormalTok{]}

\NormalTok{\#theorem("Euclid")[}
\NormalTok{  There are infinitely many primes.}
\NormalTok{]}
\NormalTok{\#proof[}
\NormalTok{  Suppose to the contrary that $p\_1, p\_2, dots, p\_n$ is a finite enumeration}
\NormalTok{  of all primes. Set $P = p\_1 p\_2 dots p\_n$. Since $P + 1$ is not in our list,}
\NormalTok{  it cannot be prime. Thus, some prime factor $p\_j$ divides $P + 1$.  Since}
\NormalTok{  $p\_j$ also divides $P$, it must divide the difference $(P + 1) {-} P = 1$, a}
\NormalTok{  contradiction.}
\NormalTok{]}

\NormalTok{\#corollary[}
\NormalTok{  There is no largest prime number.}
\NormalTok{] \textless{}cor\_largest\_prime\textgreater{}}
\NormalTok{\#corollary[}
\NormalTok{  There are infinitely many composite numbers.}
\NormalTok{]}

\NormalTok{\#theorem[}
\NormalTok{  There are arbitrarily long stretches of composite numbers.}
\NormalTok{]}

\NormalTok{\#proof[}
\NormalTok{  For any $n \textgreater{} 2$, consider $}
\NormalTok{    n! + 2, quad n! + 3, quad ..., quad n! + n \#qedhere}
\NormalTok{  $}
\NormalTok{]}


\NormalTok{= Others}

\NormalTok{== Side{-}by{-}side}

\NormalTok{\#slide(composer: (1fr, 1fr))[}
\NormalTok{  First column.}
\NormalTok{][}
\NormalTok{  Second column.}
\NormalTok{]}


\NormalTok{== Multiple Pages}

\NormalTok{\#lorem(200)}


\NormalTok{\#show: appendix}

\NormalTok{= Appendix}

\NormalTok{== Appendix}

\NormalTok{Please pay attention to the current slide number.}
\end{Highlighting}
\end{Shaded}

\pandocbounded{\includegraphics[keepaspectratio]{https://github.com/user-attachments/assets/3488f256-a0b3-43d0-a266-009d9d0a7bd3}}

\subsection{Acknowledgements}\label{acknowledgements}

Thanks to…

\begin{itemize}
\tightlist
\item
  \href{https://github.com/andreasKroepelin}{@andreasKroepelin} for the
  \texttt{\ polylux\ } package
\item
  \href{https://github.com/Enivex}{@Enivex} for the
  \texttt{\ metropolis\ } theme
\item
  \href{https://github.com/drupol}{@drupol} for the
  \texttt{\ university\ } theme
\item
  \href{https://github.com/pride7}{@pride7} for the \texttt{\ aqua\ }
  theme
\item
  \href{https://github.com/Coekjan}{@Coekjan} and
  \href{https://github.com/QuadnucYard}{@QuadnucYard} for the
  \texttt{\ stargazer\ } theme
\item
  \href{https://github.com/ntjess}{@ntjess} for contributing to
  \texttt{\ fit-to-height\ } , \texttt{\ fit-to-width\ } and
  \texttt{\ cover-with-rect\ }
\end{itemize}

\subsection{Poster}\label{poster}

\pandocbounded{\includegraphics[keepaspectratio]{https://github.com/user-attachments/assets/e1ddb672-8e8f-472d-b364-b8caed1da16b}}

\href{https://github.com/touying-typ/touying-poster}{View Code}

\subsection{Star History}\label{star-history}

\href{https://star-history.com/\#touying-typ/touying&Date}{\pandocbounded{\includegraphics[keepaspectratio]{https://api.star-history.com/svg?repos=touying-typ/touying&type=Date}}}

\subsubsection{How to add}\label{how-to-add}

Copy this into your project and use the import as \texttt{\ touying\ }

\begin{verbatim}
#import "@preview/touying:0.5.3"
\end{verbatim}

\includesvg[width=0.16667in,height=0.16667in]{/assets/icons/16-copy.svg}

Check the docs for
\href{https://typst.app/docs/reference/scripting/\#packages}{more
information on how to import packages} .

\subsubsection{About}\label{about}

\begin{description}
\tightlist
\item[Author s :]
OrangeX4 , Andreas Kröpelin , ntjess , Enivex , Pol Dellaiera , pride7
, \& Coekjan
\item[License:]
MIT
\item[Current version:]
0.5.3
\item[Last updated:]
October 15, 2024
\item[First released:]
January 11, 2024
\item[Minimum Typst version:]
0.11.0
\item[Archive size:]
302 kB
\href{https://packages.typst.org/preview/touying-0.5.3.tar.gz}{\pandocbounded{\includesvg[keepaspectratio]{/assets/icons/16-download.svg}}}
\item[Repository:]
\href{https://github.com/touying-typ/touying}{GitHub}
\item[Categor y :]
\begin{itemize}
\tightlist
\item[]
\item
  \pandocbounded{\includesvg[keepaspectratio]{/assets/icons/16-presentation.svg}}
  \href{https://typst.app/universe/search/?category=presentation}{Presentation}
\end{itemize}
\end{description}

\subsubsection{Where to report issues?}\label{where-to-report-issues}

This package is a project of OrangeX4, Andreas Kröpelin, ntjess,
Enivex, Pol Dellaiera, pride7, and Coekjan . Report issues on
\href{https://github.com/touying-typ/touying}{their repository} . You
can also try to ask for help with this package on the
\href{https://forum.typst.app}{Forum} .

Please report this package to the Typst team using the
\href{https://typst.app/contact}{contact form} if you believe it is a
safety hazard or infringes upon your rights.

\phantomsection\label{versions}
\subsubsection{Version history}\label{version-history}

\begin{longtable}[]{@{}ll@{}}
\toprule\noalign{}
Version & Release Date \\
\midrule\noalign{}
\endhead
\bottomrule\noalign{}
\endlastfoot
0.5.3 & October 15, 2024 \\
\href{https://typst.app/universe/package/touying/0.5.2/}{0.5.2} &
September 3, 2024 \\
\href{https://typst.app/universe/package/touying/0.5.1/}{0.5.1} &
September 3, 2024 \\
\href{https://typst.app/universe/package/touying/0.5.0/}{0.5.0} &
September 2, 2024 \\
\href{https://typst.app/universe/package/touying/0.4.2/}{0.4.2} & May
27, 2024 \\
\href{https://typst.app/universe/package/touying/0.4.1/}{0.4.1} & May
13, 2024 \\
\href{https://typst.app/universe/package/touying/0.4.0/}{0.4.0} & April
6, 2024 \\
\href{https://typst.app/universe/package/touying/0.3.3/}{0.3.3} & March
26, 2024 \\
\href{https://typst.app/universe/package/touying/0.3.2/}{0.3.2} & March
15, 2024 \\
\href{https://typst.app/universe/package/touying/0.3.1/}{0.3.1} & March
7, 2024 \\
\href{https://typst.app/universe/package/touying/0.3.0/}{0.3.0} & March
6, 2024 \\
\href{https://typst.app/universe/package/touying/0.2.1/}{0.2.1} &
February 17, 2024 \\
\href{https://typst.app/universe/package/touying/0.2.0/}{0.2.0} &
January 20, 2024 \\
\href{https://typst.app/universe/package/touying/0.1.0/}{0.1.0} &
January 11, 2024 \\
\end{longtable}

Typst GmbH did not create this package and cannot guarantee correct
functionality of this package or compatibility with any version of the
Typst compiler or app.


