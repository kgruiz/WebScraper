\title{typst.app/universe/package/modern-report-umfds}

\phantomsection\label{banner}
\phantomsection\label{template-thumbnail}
\pandocbounded{\includegraphics[keepaspectratio]{https://packages.typst.org/preview/thumbnails/modern-report-umfds-0.1.1-small.webp}}

\section{modern-report-umfds}\label{modern-report-umfds}

{ 0.1.1 }

A template for writing reports for the Faculty of Sciences of the
University of Montpellier

\href{/app?template=modern-report-umfds&version=0.1.1}{Create project in
app}

\phantomsection\label{readme}
A template for writing reports for the Faculty of Sciences of the
University of Montpellier.

Basic usage:

\begin{Shaded}
\begin{Highlighting}[]
\NormalTok{\#import "@preview/modern{-}report{-}umfds:0.1.1": umfds}

\NormalTok{\#show: umfds.with(}
\NormalTok{  title: [Your report title],}
\NormalTok{  authors: (}
\NormalTok{    "Author 1",}
\NormalTok{    "Author 2",}
\NormalTok{    "Author 3",}
\NormalTok{    "Author 4"}
\NormalTok{  ),}
\NormalTok{  date: datetime.today().display("[day] [month repr:long] [year]"), // or any string}
\NormalTok{  img: image("cover.png"), // optional}
\NormalTok{  abstract: [}
\NormalTok{    Your abstract, optional}
\NormalTok{  ],}
\NormalTok{  bibliography: bibliography("refs.bib", full: true), // optional}
\NormalTok{  lang: "en", // or "fr"}
\NormalTok{)}

\NormalTok{// Your report content}
\end{Highlighting}
\end{Shaded}

\href{/app?template=modern-report-umfds&version=0.1.1}{Create project in
app}

\subsubsection{How to use}\label{how-to-use}

Click the button above to create a new project using this template in
the Typst app.

You can also use the Typst CLI to start a new project on your computer
using this command:

\begin{verbatim}
typst init @preview/modern-report-umfds:0.1.1
\end{verbatim}

\includesvg[width=0.16667in,height=0.16667in]{/assets/icons/16-copy.svg}

\subsubsection{About}\label{about}

\begin{description}
\tightlist
\item[Author s :]
Pablo Laviron \& Sébastien Vial
\item[License:]
MIT-0
\item[Current version:]
0.1.1
\item[Last updated:]
October 14, 2024
\item[First released:]
October 7, 2024
\item[Archive size:]
80.0 kB
\href{https://packages.typst.org/preview/modern-report-umfds-0.1.1.tar.gz}{\pandocbounded{\includesvg[keepaspectratio]{/assets/icons/16-download.svg}}}
\item[Repository:]
\href{https://github.com/UM-nerds/modern-report-umfds}{GitHub}
\item[Categor y :]
\begin{itemize}
\tightlist
\item[]
\item
  \pandocbounded{\includesvg[keepaspectratio]{/assets/icons/16-speak.svg}}
  \href{https://typst.app/universe/search/?category=report}{Report}
\end{itemize}
\end{description}

\subsubsection{Where to report issues?}\label{where-to-report-issues}

This template is a project of Pablo Laviron and Sébastien Vial . Report
issues on \href{https://github.com/UM-nerds/modern-report-umfds}{their
repository} . You can also try to ask for help with this template on the
\href{https://forum.typst.app}{Forum} .

Please report this template to the Typst team using the
\href{https://typst.app/contact}{contact form} if you believe it is a
safety hazard or infringes upon your rights.

\phantomsection\label{versions}
\subsubsection{Version history}\label{version-history}

\begin{longtable}[]{@{}ll@{}}
\toprule\noalign{}
Version & Release Date \\
\midrule\noalign{}
\endhead
\bottomrule\noalign{}
\endlastfoot
0.1.1 & October 14, 2024 \\
\href{https://typst.app/universe/package/modern-report-umfds/0.1.0/}{0.1.0}
& October 7, 2024 \\
\end{longtable}

Typst GmbH did not create this template and cannot guarantee correct
functionality of this template or compatibility with any version of the
Typst compiler or app.


\title{typst.app/universe/package/anti-matter}

\phantomsection\label{banner}
\section{anti-matter}\label{anti-matter}

{ 0.1.1 }

Simple page numbering of front and back matter.

\phantomsection\label{readme}
This typst packages allows you to simply mark the end and start of your
front matter and back matter to change style and value of your page
number without manually setting and keeping track of inner and outer
page counters.

\subsection{Example}\label{example}

\begin{Shaded}
\begin{Highlighting}[]
\NormalTok{\#import "@preview/anti{-}matter:0.1.1": anti{-}matter, fence, set{-}numbering}

\NormalTok{\#set page("a4", height: auto)}
\NormalTok{\#show heading.where(level: 1): it =\textgreater{} pagebreak(weak: true) + it}

\NormalTok{\#show: anti{-}matter}

\NormalTok{\#set{-}numbering(none)}
\NormalTok{\#align(center)[My Title Page]}
\NormalTok{\#pagebreak()}
\NormalTok{\#set{-}numbering("I")}

\NormalTok{\#include "front{-}matter.typ"}
\NormalTok{\#fence()}

\NormalTok{\#include "chapters.typ"}
\NormalTok{\#fence()}

\NormalTok{\#include "back{-}matter.typ"}
\end{Highlighting}
\end{Shaded}

\pandocbounded{\includegraphics[keepaspectratio]{https://github.com/typst/packages/raw/main/packages/preview/anti-matter/0.1.1/example/example.png}}

\subsection{Features}\label{features}

\begin{itemize}
\tightlist
\item
  Marking the start and end of front/back matter.
\item
  Specifying the numbering styles for each part fo the document
\end{itemize}

\subsection{FAQ}\label{faq}

\begin{enumerate}
\tightlist
\item
  Why are the pages not correctly counted?

  \begin{itemize}
  \tightlist
  \item
    If you are setting your own page header, you must use
    \texttt{\ step\ } , see section II in the
    \href{https://github.com/typst/packages/raw/main/packages/preview/anti-matter/0.1.1/docs/manual.pdf}{manual}
    .
  \end{itemize}
\item
  Why is my outline not displaying the correct numbering?

  \begin{itemize}
  \tightlist
  \item
    If you configure your own \texttt{\ outline.entry\ } , you must use
    \texttt{\ page-number\ } , see section II in the
    \href{https://github.com/typst/packages/raw/main/packages/preview/anti-matter/0.1.1/docs/manual.pdf}{manual}
    .
  \end{itemize}
\item
  Why does my front/inner/back matter numbering start on the wrong page?

  \begin{itemize}
  \tightlist
  \item
    The fences must be on the last page of their respective part, if you
    have a \texttt{\ pagebreak\ } forcing them on the next page it will
    also incorrectly label that page.
  \item
    Otherwise please open an issue with a minimal reproducible example.
  \end{itemize}
\end{enumerate}

\subsection{Etymology}\label{etymology}

The package name \texttt{\ anti-matter\ } was choosen as a word play on
front/back matter.

\subsection{Glossary}\label{glossary}

\begin{itemize}
\tightlist
\item
  \href{https://en.wikipedia.org/wiki/Book_design\#Front_matter}{front
  matter} - The first part of a thesis or book (intro, outline, etc.)
\item
  \href{https://en.wikipedia.org/wiki/Book_design\#Back_matter_(end_matter)}{back
  matter} - The last part of a thesis or book (bibliography, listings,
  acknowledgements, etc.)
\end{itemize}

\subsubsection{How to add}\label{how-to-add}

Copy this into your project and use the import as
\texttt{\ anti-matter\ }

\begin{verbatim}
#import "@preview/anti-matter:0.1.1"
\end{verbatim}

\includesvg[width=0.16667in,height=0.16667in]{/assets/icons/16-copy.svg}

Check the docs for
\href{https://typst.app/docs/reference/scripting/\#packages}{more
information on how to import packages} .

\subsubsection{About}\label{about}

\begin{description}
\tightlist
\item[Author :]
\href{mailto:me@tinger.dev}{tinger}
\item[License:]
MIT
\item[Current version:]
0.1.1
\item[Last updated:]
December 3, 2023
\item[First released:]
September 29, 2023
\item[Minimum Typst version:]
0.8.0
\item[Archive size:]
3.70 kB
\href{https://packages.typst.org/preview/anti-matter-0.1.1.tar.gz}{\pandocbounded{\includesvg[keepaspectratio]{/assets/icons/16-download.svg}}}
\item[Repository:]
\href{https://github.com/tingerrr/anti-matter}{GitHub}
\end{description}

\subsubsection{Where to report issues?}\label{where-to-report-issues}

This package is a project of tinger . Report issues on
\href{https://github.com/tingerrr/anti-matter}{their repository} . You
can also try to ask for help with this package on the
\href{https://forum.typst.app}{Forum} .

Please report this package to the Typst team using the
\href{https://typst.app/contact}{contact form} if you believe it is a
safety hazard or infringes upon your rights.

\phantomsection\label{versions}
\subsubsection{Version history}\label{version-history}

\begin{longtable}[]{@{}ll@{}}
\toprule\noalign{}
Version & Release Date \\
\midrule\noalign{}
\endhead
\bottomrule\noalign{}
\endlastfoot
0.1.1 & December 3, 2023 \\
\href{https://typst.app/universe/package/anti-matter/0.1.0/}{0.1.0} &
November 29, 2023 \\
\href{https://typst.app/universe/package/anti-matter/0.0.2/}{0.0.2} &
October 2, 2023 \\
\href{https://typst.app/universe/package/anti-matter/0.0.1/}{0.0.1} &
September 29, 2023 \\
\end{longtable}

Typst GmbH did not create this package and cannot guarantee correct
functionality of this package or compatibility with any version of the
Typst compiler or app.


\title{typst.app/universe/package/pillar}

\phantomsection\label{banner}
\section{pillar}\label{pillar}

{ 0.2.0 }

Faster column specifications for tables.

\phantomsection\label{readme}
\emph{Shorthand notations for table column specifications in
\href{https://typst.app/}{Typst} .}

\href{https://typst.app/universe/package/pillar}{\pandocbounded{\includegraphics[keepaspectratio]{https://img.shields.io/badge/dynamic/toml?url=https\%3A\%2F\%2Fraw.githubusercontent.com\%2FMc-Zen\%2Fpillar\%2Fmain\%2Ftypst.toml&query=\%24.package.version&prefix=v&logo=typst&label=package&color=239DAD}}}
\href{https://github.com/Mc-Zen/pillar/actions/workflows/run_tests.yml}{\pandocbounded{\includesvg[keepaspectratio]{https://github.com/Mc-Zen/pillar/actions/workflows/run_tests.yml/badge.svg}}}
\href{https://github.com/Mc-Zen/pillar/blob/main/LICENSE}{\pandocbounded{\includegraphics[keepaspectratio]{https://img.shields.io/badge/license-MIT-blue}}}

\begin{itemize}
\tightlist
\item
  \href{https://github.com/typst/packages/raw/main/packages/preview/pillar/0.2.0/\#introduction}{Introduction}
\item
  \href{https://github.com/typst/packages/raw/main/packages/preview/pillar/0.2.0/\#column-specification}{Column
  specification}
\item
  \href{https://github.com/typst/packages/raw/main/packages/preview/pillar/0.2.0/\#number-alignment}{Number
  alignment}
\item
  \href{https://github.com/typst/packages/raw/main/packages/preview/pillar/0.2.0/\#pillarcols}{\texttt{\ pillar.cols()\ }}
\item
  \href{https://github.com/typst/packages/raw/main/packages/preview/pillar/0.2.0/\#pillartable}{\texttt{\ pillar.table()\ }}
\item
  \href{https://github.com/typst/packages/raw/main/packages/preview/pillar/0.2.0/\#vline-customization}{\texttt{\ vline\ }
  customization}
\end{itemize}

\subsection{Introduction}\label{introduction}

With \textbf{pillar} , you can significantly simplify the column setup
of tables by unifying the specification of the number, alignment, and
separation of columns. This package is in particular designed for
scientific tables, which typically have simple styling:

\pandocbounded{\includegraphics[keepaspectratio]{https://github.com/user-attachments/assets/c0c60651-c682-4968-9ac9-0fa1e8d85dad}}

In order to prepare this table with just the built-in methods, some code
like the following would be required.

\begin{Shaded}
\begin{Highlighting}[]
\NormalTok{\#table(}
\NormalTok{  columns: 5,}
\NormalTok{  align: (center,) * 4 + (right,),}
\NormalTok{  stroke: none,}


\NormalTok{  [Piano Key], table.vline(), [MIDI Number], [Note Name], [Pitch Name], table.vline(), [$f$ in Hz],}
\NormalTok{  ..}
\NormalTok{)}
\end{Highlighting}
\end{Shaded}

Using \textbf{pillar} , the same can be achieved with

\begin{Shaded}
\begin{Highlighting}[]
\NormalTok{\#table(}
\NormalTok{    ..pillar.cols("c|ccc|r"),}

\NormalTok{    [Piano Key], [MIDI Number], [Note Name], [Pitch Name], [$f$ in Hz], ..}
\NormalTok{)}
\end{Highlighting}
\end{Shaded}

or alternatively

\begin{Shaded}
\begin{Highlighting}[]
\NormalTok{\#pillar.table(}
\NormalTok{    cols: "c|ccc|r",}

\NormalTok{    [Piano Key], [MIDI Number], [Note Name], [Pitch Name], [$f$ in Hz], ..}
\NormalTok{)}
\end{Highlighting}
\end{Shaded}

\textbf{Pillar} is designed for interoperability. It uses the powerful
standard tables and provides generated arguments for \texttt{\ table\ }
’s \texttt{\ columns\ } , \texttt{\ align\ } , \texttt{\ stroke\ } ,
and for the specified vertical lines. This means that all features of
the built-in tables (and also \texttt{\ show\ } and \texttt{\ set\ }
rules) can be applied as usual.

\subsection{Column specification}\label{column-specification}

This package works with \emph{column specification strings} . Each
column is described by its alignment which can be \texttt{\ l\ } (left),
\texttt{\ c\ } (center), \texttt{\ r\ } (right), or \texttt{\ a\ }
(auto).

The width of a column can optionally be specified by appending a
(relative) length, or fraction in square brackets to the alignment
specifier, e.g., \texttt{\ c{[}2cm{]}\ } or \texttt{\ r{[}1fr{]}\ } .

Vertical lines can be added between columns with a
\texttt{\ \textbar{}\ } character. Double lines can be produced with
\texttt{\ \textbar{}\textbar{}\ } (see
\href{https://github.com/typst/packages/raw/main/packages/preview/pillar/0.2.0/\#vline-customization}{\texttt{\ vline\ }
customization} ). The stroke of the vertical line can be changed by
appending anything that is usually allowed as a stroke argument in
square brackets, e.g., \texttt{\ \textbar{}{[}2pt{]}\ } ,
\texttt{\ \textbar{}{[}red{]}\ } or
\texttt{\ \textbar{}{[}(dash:\ \textbackslash{}"dashed\textbackslash{}"){]}\ }
.

A column specification string may contain any number of spaces (e.g., to
improve readability) â€'' all spaces will be ignored.

\emph{If you find yourself writing highly complex column specifications,
consider not using this package and resort to the parameters that the
built-in tables provide. This package is intended for quick and
relatively simple column specifications.}

\subsection{Number alignment}\label{number-alignment}

Choosing capital letters \texttt{\ L\ } , \texttt{\ C\ } ,
\texttt{\ R\ } , or \texttt{\ A\ } instead of lower-case letters
activates number alignment at the decimal separator for a specific
column (similar to the column type “S� of the LaTeX package
\href{https://github.com/josephwright/siunitx}{siunitx} ). This feature
is provided via the Typst package \textbf{Zero} .
\href{https://github.com/Mc-Zen/zero}{Here} you can read up on the
configuration of number formatting.

\begin{Shaded}
\begin{Highlighting}[]
\NormalTok{\#let percm = $"cm"\^{}({-}1)$}

\NormalTok{\#pillar.table(}
\NormalTok{  cols: "l|CCCC",}
\NormalTok{  [], [$Δ ν\_0$ in \#percm], [$B\textquotesingle{}\_ν$ in \#percm], [$B\textquotesingle{}\textquotesingle{}\_ν$ in \#percm], [$D\textquotesingle{}\_ν$ in \#percm],}
\NormalTok{  table.hline(),}
\NormalTok{  [Measurement], [14525.278],   [1.41],    [1.47],    [1.5e{-}5],}
\NormalTok{  [Uncertainty], [2],           [0.3],     [0.3],     [4e{-}6],}
\NormalTok{  [Ref. [2]],    [14525,74856], [1.37316], [1.43777], [5.401e{-}6]}
\NormalTok{)}
\end{Highlighting}
\end{Shaded}

\pandocbounded{\includegraphics[keepaspectratio]{https://github.com/user-attachments/assets/066cd34e-7043-48c7-b067-e3256e942f14}}

Non-number entries (e.g., in the header) are automatically recognized in
some cases and will not be aligned. In ambiguous cases, adding a leading
or trailing space tells Zero not to apply alignment to this cell, e.g.,
\texttt{\ {[}Voltage\ {]}\ } instead of \texttt{\ {[}Voltage{]}\ } .

\subsection{\texorpdfstring{\texttt{\ pillar.cols()\ }}{ pillar.cols() }}\label{pillar.cols}

This function produces an argument list that may contain arguments for
\texttt{\ columns\ } , \texttt{\ align\ } , \texttt{\ stroke\ } , and
\texttt{\ column-gutter\ } as well as instances of
\texttt{\ table.vline()\ } . These arguments are intended to be expanded
with the \texttt{\ ..\ } syntax into the argument list of
\texttt{\ table\ } as shown in the examples.

\subsection{\texorpdfstring{\texttt{\ pillar.table()\ }}{ pillar.table() }}\label{pillar.table}

This is a thin wrapper that behaves just like the built-in
\texttt{\ table\ } , except that it extracts the first positional
argument if it is a string, and runs it through
\texttt{\ pillar.cols()\ } .

\subsection{\texorpdfstring{\texttt{\ vline\ }
customization}{ vline  customization}}\label{vline-customization}

In order to customize the default line setting, just use set rules on
\texttt{\ table.vline\ } , e.g.,

\begin{Shaded}
\begin{Highlighting}[]
\NormalTok{\#set table.vline(stroke: .7pt)}

\NormalTok{\#table(..pillar.cols("c|cc"), ..)}
\end{Highlighting}
\end{Shaded}

Double lines are currently experimental and are realized through column
gutters. They could also be realized through patterns, but this can
produce artifacts with some renderers. As it currently is, double lines
are not supported before the first and after the last column. On the
other hand, with the current method, double lines are styled with set
rules on \texttt{\ table.vline\ } which is nice and not achievable in
the same way with patterns.

\subsection{Examples}\label{examples}

\subsubsection{Double lines}\label{double-lines}

The following example uses a double line for visually separating
repeated table columns.

\begin{Shaded}
\begin{Highlighting}[]
\NormalTok{\#pillar.table(}
\NormalTok{  cols: "ccc ||[.7pt] ccc",}
  
\NormalTok{  ..([\textbackslash{}\#], [$α$ in °], [$β$ in °]) * 2,}
\NormalTok{  table.hline(),}
\NormalTok{  [1], [34.3], [11.1],  [6], [34.0], [12.9],}
\NormalTok{  [2], [34.2], [11.2],  [7], [34.3], [12.8],}
\NormalTok{  [3], [34.6], [11.4],  [8], [33.9], [11.9],}
\NormalTok{  [4], [34.7], [10.3],  [9], [34.4], [11.8],}
\NormalTok{  [5], [34.3], [11.1], [10], [34.4], [11.8],}
\NormalTok{)}
\end{Highlighting}
\end{Shaded}

\pandocbounded{\includegraphics[keepaspectratio]{https://github.com/user-attachments/assets/e05e7bad-61b6-44f9-af34-5e558f338cdc}}

\subsubsection{Further customization}\label{further-customization}

This example shows the codes of the first ten printable ASCII
characters, demonstrating stroke and column width customization.

\begin{Shaded}
\begin{Highlighting}[]
\NormalTok{\#pillar.table(}
\NormalTok{  cols: "ccc|ccc|[1.8pt + blue] l[5cm]",}
  
\NormalTok{  [Dec],[Hex],[Bin],[Symbol], [HTML code], [HTML name], [Description],}
\NormalTok{  table.hline(),}
\NormalTok{  [32], [20], [00100000], [\&\#32;], [],         [SP], [Space],}
\NormalTok{  [33], [21], [00100001], [\&\#33;], [\&excl;],   [!],  [Exclamation mark],}
\NormalTok{  [34], [22], [00100010], [\&\#34;], [\&quot;],   ["],  [Double quotes],}
\NormalTok{  [35], [23], [00100011], [\&\#35;], [\&num;],    [\textbackslash{}\#], [Number sign],}
\NormalTok{  [36], [24], [00100100], [\&\#36;], [\&dollar;], [\textbackslash{}$], [Dollar sign],}
\NormalTok{  [37], [25], [00100101], [\&\#37;], [\&percnt;], [\%],  [Percent sign],}
\NormalTok{  [38], [26], [00100110], [\&\#38;], [\&amp;],    [\&],  [Ampersand],}
\NormalTok{  [39], [27], [00100111], [\&\#39;], [\&apos;],   [\textquotesingle{}],  [Single quote],}
\NormalTok{  [40], [28], [00101000], [\&\#40;], [\&lparen;], [(],  [Opening parenthesis],}
\NormalTok{  [41], [29], [00101001], [\&\#41;], [\&rparen;], [)],  [Closing parenthesis],}
\NormalTok{)}
\end{Highlighting}
\end{Shaded}

\pandocbounded{\includegraphics[keepaspectratio]{https://github.com/user-attachments/assets/9fae998e-033d-4d7e-9344-fe3778bbd9e6}}

\subsection{Tests}\label{tests}

This package uses
\href{https://github.com/tingerrr/typst-test/}{typst-test} for running
\href{https://github.com/typst/packages/raw/main/packages/preview/pillar/0.2.0/tests/}{tests}
.

\subsubsection{How to add}\label{how-to-add}

Copy this into your project and use the import as \texttt{\ pillar\ }

\begin{verbatim}
#import "@preview/pillar:0.2.0"
\end{verbatim}

\includesvg[width=0.16667in,height=0.16667in]{/assets/icons/16-copy.svg}

Check the docs for
\href{https://typst.app/docs/reference/scripting/\#packages}{more
information on how to import packages} .

\subsubsection{About}\label{about}

\begin{description}
\tightlist
\item[Author :]
\href{https://github.com/Mc-Zen}{Mc-Zen}
\item[License:]
MIT
\item[Current version:]
0.2.0
\item[Last updated:]
August 22, 2024
\item[First released:]
May 27, 2024
\item[Minimum Typst version:]
0.11.0
\item[Archive size:]
5.52 kB
\href{https://packages.typst.org/preview/pillar-0.2.0.tar.gz}{\pandocbounded{\includesvg[keepaspectratio]{/assets/icons/16-download.svg}}}
\item[Categor ies :]
\begin{itemize}
\tightlist
\item[]
\item
  \pandocbounded{\includesvg[keepaspectratio]{/assets/icons/16-chart.svg}}
  \href{https://typst.app/universe/search/?category=visualization}{Visualization}
\item
  \pandocbounded{\includesvg[keepaspectratio]{/assets/icons/16-layout.svg}}
  \href{https://typst.app/universe/search/?category=layout}{Layout}
\end{itemize}
\end{description}

\subsubsection{Where to report issues?}\label{where-to-report-issues}

This package is a project of Mc-Zen . You can also try to ask for help
with this package on the \href{https://forum.typst.app}{Forum} .

Please report this package to the Typst team using the
\href{https://typst.app/contact}{contact form} if you believe it is a
safety hazard or infringes upon your rights.

\phantomsection\label{versions}
\subsubsection{Version history}\label{version-history}

\begin{longtable}[]{@{}ll@{}}
\toprule\noalign{}
Version & Release Date \\
\midrule\noalign{}
\endhead
\bottomrule\noalign{}
\endlastfoot
0.2.0 & August 22, 2024 \\
\href{https://typst.app/universe/package/pillar/0.1.0/}{0.1.0} & May 27,
2024 \\
\end{longtable}

Typst GmbH did not create this package and cannot guarantee correct
functionality of this package or compatibility with any version of the
Typst compiler or app.


\title{typst.app/universe/package/blindex}

\phantomsection\label{banner}
\section{blindex}\label{blindex}

{ 0.1.0 }

Index-making of Biblical literature citations in Typst

\phantomsection\label{readme}
Blindex ( \texttt{\ blindex:0.1.0\ } ) is a Typst package specifically
designed for the generation of indices of Biblical literature citations
in documents. Target audience includes theologians and authors of
documents that frequently cite biblical literature.

\subsection{Index Sorting Options}\label{index-sorting-options}

The generated indices are gathered and sorted by Biblical literature
books, which can be ordered according to various Biblical literature
book ordering conventions, including:

\begin{itemize}
\tightlist
\item
  \texttt{\ "LXX"\ } â€`` The Septuagint;
\item
  \texttt{\ "Greek-Bible"\ } â€`` Septuagint + New Testament (King
  James);
\item
  \texttt{\ "Hebrew-Tanakh"\ } â€`` The Hebrew (Torah + Neviim +
  Ketuvim);
\item
  \texttt{\ "Hebrew-Bible"\ } â€`` The Hebrew Tanakh + New Testament
  (King James);
\item
  \texttt{\ "Protestant-Bible"\ } â€`` The Protestant Old + New
  Testaments;
\item
  \texttt{\ "Catholic-Bible"\ } â€`` The Catholic Old + New Testaments;
\item
  \texttt{\ "Orthodox-Bible"\ } â€`` The Orthodox Old + New Testaments;
\item
  \texttt{\ "Oecumenic-Bible"\ } â€`` The Jewish Tanakh + Old Testament
  Deuterocanonical + New Testament;
\item
  \texttt{\ "code"\ } â€`` All registered Biblical literature books: All
  Protestant + All Apocripha.
\end{itemize}

\subsection{Biblical Literature
Abbrevations}\label{biblical-literature-abbrevations}

It is common practice among theologians to refer to biblical literature
books by their abbreviations. Practice shows that abbreviation
conventions are language- and tradition- dependent. Therefore,
\texttt{\ blindex\ } usage reflects this fact, while offering a way to
input arbitrary language-tradition abbreviations, in the
\texttt{\ lang.typ\ } source file.

\subsubsection{Language and Traditions
(Variants)}\label{language-and-traditions-variants}

The \texttt{\ blindex\ } implementation generalizes the concept of
\textbf{tradition} (in the context of biblical literature book
abbreviation bookkeeping) as language \textbf{variants} , since the
software can have things such as a “default� of “n-char�
variants.

As of the current release, supported languages include the following few
ones:

\begin{longtable}[]{@{}llll@{}}
\toprule\noalign{}
Language & Variant & Description & Name \\
\midrule\noalign{}
\endhead
\bottomrule\noalign{}
\endlastfoot
English & 3-char & A 3-char abbreviations & \texttt{\ en-3\ } \\
English & Logos & Used in \texttt{\ logos.com\ } &
\texttt{\ en-logos\ } \\
Portuguese (BR) & Protestant & Protestant for Brazil &
\texttt{\ br-pro\ } \\
Portuguese (BR) & Catholic & Catholic for Brazil &
\texttt{\ br-cat\ } \\
\end{longtable}

Additional language-variations can be added to the \texttt{\ lang.typ\ }
source file by the author and/or by pull requests to the
\texttt{\ dev\ } branch of the (UNFORKED!) development repository
\texttt{\ https://github.com/cnaak/blindex.typ\ } .

\subsection{Low-Level Indexing
Command}\label{low-level-indexing-command}

The \texttt{\ blindex\ } library has a low-level, index entry marking
function \texttt{\ \#blindex(abrv,\ lang,\ entry)\ } , whose arguments
are (abbreviation, language, entry), as in:

\begin{Shaded}
\begin{Highlighting}[]
\NormalTok{"the citation..." \#blindex("1Thess", "en", [1.1{-}{-}3]) citation\textquotesingle{}s typesetting...}
\end{Highlighting}
\end{Shaded}

Following the usual index making strategy in Typst, this user
\texttt{\ \#blindex\ } command only adds the index-marking
\texttt{\ \#metadata\ } in the document, without producing any visible
typeset output.

Biblical literature index listings can be generated (typeset) in
arbitrary amounts and locations throughout the document, just by calling
the user \texttt{\ \#mkIndex\ } command:

\begin{Shaded}
\begin{Highlighting}[]
\NormalTok{\#mkIndex()}
\end{Highlighting}
\end{Shaded}

Optional arguments control style and sorting convention parameters, as
exemplified below.

\subsection{Higher-Level Quoting-Indexing
Commands}\label{higher-level-quoting-indexing-commands}

The library also offers higher-level functions to assemble the entire
(i) citation typesetting, (ii) index entry, (iii) citation typesetting,
and (iv) bibliography entrying (with some typesetting (styling)
options), of the passage. Such commands are \texttt{\ \#iQuot(...)\ }
and \texttt{\ \#bQuot(...)\ } , respectively for \textbf{inline} and
\textbf{block} quoting of Biblical literature, with automatic indexing
and bibliography citation. Mandatory arguments are identical for either
command:

\begin{Shaded}
\begin{Highlighting}[]
\NormalTok{paragraph text...}
\NormalTok{\#iQuot(body, abrv, lang, pssg, version, cited)}
\NormalTok{more text...}

\NormalTok{// Displayed block quote of Biblical literature:}
\NormalTok{\#bQuot(body, abrv, lang, pssg, version, cited)}
\end{Highlighting}
\end{Shaded}

In which:

\begin{itemize}
\tightlist
\item
  \texttt{\ body\ } ( \texttt{\ content\ } ) is the quoted biblical
  literature text;
\item
  \texttt{\ abrv\ } ( \texttt{\ string\ } ) is the book abbreviation
  according to the
\item
  \texttt{\ lang\ } ( \texttt{\ string\ } ) language-variant (see
  above);
\item
  \texttt{\ pssg\ } ( \texttt{\ content\ } ) is the quoted text passage
  â€'' usually chapter and verses â€'' as they will appear in the text
  and in the biblical literature index;
\item
  \texttt{\ version\ } ( \texttt{\ string\ } ) is a translation
  identifier, such as \texttt{\ "LXX"\ } , or \texttt{\ "KJV"\ } ; and
\item
  \texttt{\ cited\ } ( \texttt{\ label\ } ) is the corresponding
  bibliography entry label, which can be constructed through:
\end{itemize}

\texttt{\ label("bib-key")\ } , where \texttt{\ bib-key\ } is the
bibliographic entry key, in the bibliography database â€'' whether
\texttt{\ bibTeX\ } or \texttt{\ Hayagriva\ } .

\subsection{Higher-Level Example}\label{higher-level-example}

\begin{Shaded}
\begin{Highlighting}[]
\NormalTok{\#set page(paper: "a7", fill: rgb("\#eec"))}
\NormalTok{\#import "@preview/blindex:0.1.0": *}

\NormalTok{The Septuagint (LXX) starts with \#iQuot([ΕΝ ἀρχῇ ἐποίησεν ὁ Θεὸς τὸν οὐρανὸν καὶ τὴν γῆν.],}
\NormalTok{"Gen", "en", [1.1], "LXX", label("2012{-}LXX{-}SBB")).}

\NormalTok{\#pagebreak()}

\NormalTok{Moreover, the book of Odes begins with: \#iQuot([ᾠδὴ Μωυσέως ἐν τῇ ἐξόδῳ], "Ode", "en", [1.0],}
\NormalTok{"LXX", label("2012{-}LXX{-}SBB")).}

\NormalTok{\#pagebreak()}

\NormalTok{= Biblical Citations}
\NormalTok{Books are sorted following the LXX ordering.}

\NormalTok{\#mkIndex(cols: 1, sorting{-}tradition: "LXX")}

\NormalTok{\#pagebreak()}

\NormalTok{\#bibliography("test{-}01{-}readme.yml", title: "References", style: "ieee")}
\end{Highlighting}
\end{Shaded}

The listing of the bibliography file, \texttt{\ test-01-readme.yml\ } ,
as shown in the example, is:

\begin{Shaded}
\begin{Highlighting}[]
\FunctionTok{2012{-}LXX{-}SBB}\KeywordTok{:}
\AttributeTok{  }\FunctionTok{type}\KeywordTok{:}\AttributeTok{ book}
\AttributeTok{  }\FunctionTok{title}\KeywordTok{:}
\AttributeTok{    }\FunctionTok{value}\KeywordTok{:}\AttributeTok{ }\StringTok{"Septuaginta: Edição Acadêmica Capa dura – Edição de luxo"}
\AttributeTok{    }\FunctionTok{sentence{-}case}\KeywordTok{:}\AttributeTok{ }\StringTok{"Septuaginta: edição acadêmica capa dura – edição de luxo"}
\AttributeTok{    }\FunctionTok{short}\KeywordTok{:}\AttributeTok{ Septuaginta}
\AttributeTok{  }\FunctionTok{publisher}\KeywordTok{:}\AttributeTok{ Sociedade Bíblica do Brasil, SBB}
\AttributeTok{  }\FunctionTok{editor}\KeywordTok{:}\AttributeTok{ Rahlfs, Alfred}
\AttributeTok{  }\FunctionTok{affiliated}\KeywordTok{:}
\AttributeTok{    }\KeywordTok{{-}}\AttributeTok{ }\FunctionTok{role}\KeywordTok{:}\AttributeTok{ collaborator}
\AttributeTok{      }\FunctionTok{names}\KeywordTok{:}\AttributeTok{ }\KeywordTok{[}\AttributeTok{ }\StringTok{"Hanhart, Robert"}\KeywordTok{,}\AttributeTok{ }\KeywordTok{]}
\AttributeTok{  }\FunctionTok{pages}\KeywordTok{:}\AttributeTok{ }\DecValTok{2240}
\AttributeTok{  }\FunctionTok{date}\KeywordTok{:}\AttributeTok{ 2012{-}01{-}11}
\AttributeTok{  }\FunctionTok{edition}\KeywordTok{:}\AttributeTok{ }\DecValTok{1}
\AttributeTok{  }\FunctionTok{ISBN}\KeywordTok{:}\AttributeTok{ 978{-}3438052278}
\AttributeTok{  }\FunctionTok{language}\KeywordTok{:}\AttributeTok{ el}
\end{Highlighting}
\end{Shaded}

This example results in a 4-page document like this one:

\pandocbounded{\includegraphics[keepaspectratio]{https://raw.githubusercontent.com/cnaak/blindex.typ/55d275e4fdab1f47c13e1fe01cbb2b397de5e0fb/thumbnail.png}}

\subsection{Citing}\label{citing}

This package can be cited with the following bibliography database
entry:

\begin{Shaded}
\begin{Highlighting}[]
\FunctionTok{blindex{-}package}\KeywordTok{:}
\AttributeTok{  }\FunctionTok{type}\KeywordTok{:}\AttributeTok{ Web}
\AttributeTok{  }\FunctionTok{author}\KeywordTok{:}\AttributeTok{ Naaktgeboren, C.}
\AttributeTok{  }\FunctionTok{title}\KeywordTok{:}
\AttributeTok{    }\FunctionTok{value}\KeywordTok{:}\AttributeTok{ }\StringTok{"Blindex: Index{-}making of Biblical literature citations in Typst"}
\AttributeTok{    }\FunctionTok{short}\KeywordTok{:}\AttributeTok{ }\StringTok{"Blindex: Index{-}making in Typst"}
\AttributeTok{  }\FunctionTok{url}\KeywordTok{:}\AttributeTok{ https://github.com/cnaak/blindex.typ}
\AttributeTok{  }\FunctionTok{version}\KeywordTok{:}\AttributeTok{ }\FloatTok{0.1.0}
\AttributeTok{  }\FunctionTok{date}\KeywordTok{:}\AttributeTok{ 2024{-}08}
\end{Highlighting}
\end{Shaded}

\subsubsection{How to add}\label{how-to-add}

Copy this into your project and use the import as \texttt{\ blindex\ }

\begin{verbatim}
#import "@preview/blindex:0.1.0"
\end{verbatim}

\includesvg[width=0.16667in,height=0.16667in]{/assets/icons/16-copy.svg}

Check the docs for
\href{https://typst.app/docs/reference/scripting/\#packages}{more
information on how to import packages} .

\subsubsection{About}\label{about}

\begin{description}
\tightlist
\item[Author :]
Naaktgeboren, C.
\item[License:]
MIT
\item[Current version:]
0.1.0
\item[Last updated:]
August 14, 2024
\item[First released:]
August 14, 2024
\item[Minimum Typst version:]
0.11.1
\item[Archive size:]
11.1 kB
\href{https://packages.typst.org/preview/blindex-0.1.0.tar.gz}{\pandocbounded{\includesvg[keepaspectratio]{/assets/icons/16-download.svg}}}
\item[Discipline :]
\begin{itemize}
\tightlist
\item[]
\item
  \href{https://typst.app/universe/search/?discipline=theology}{Theology}
\end{itemize}
\item[Categor ies :]
\begin{itemize}
\tightlist
\item[]
\item
  \pandocbounded{\includesvg[keepaspectratio]{/assets/icons/16-list-unordered.svg}}
  \href{https://typst.app/universe/search/?category=model}{Model}
\item
  \pandocbounded{\includesvg[keepaspectratio]{/assets/icons/16-code.svg}}
  \href{https://typst.app/universe/search/?category=scripting}{Scripting}
\end{itemize}
\end{description}

\subsubsection{Where to report issues?}\label{where-to-report-issues}

This package is a project of Naaktgeboren, C. . You can also try to ask
for help with this package on the \href{https://forum.typst.app}{Forum}
.

Please report this package to the Typst team using the
\href{https://typst.app/contact}{contact form} if you believe it is a
safety hazard or infringes upon your rights.

\phantomsection\label{versions}
\subsubsection{Version history}\label{version-history}

\begin{longtable}[]{@{}ll@{}}
\toprule\noalign{}
Version & Release Date \\
\midrule\noalign{}
\endhead
\bottomrule\noalign{}
\endlastfoot
0.1.0 & August 14, 2024 \\
\end{longtable}

Typst GmbH did not create this package and cannot guarantee correct
functionality of this package or compatibility with any version of the
Typst compiler or app.


\title{typst.app/universe/package/sourcerer}

\phantomsection\label{banner}
\section{sourcerer}\label{sourcerer}

{ 0.2.1 }

Customizable and flexible source-code blocks

\phantomsection\label{readme}
Sourcerer is a Typst package for displaying stylized source code blocks,
with some extra features. Main features include:

\begin{itemize}
\tightlist
\item
  Rendering source code with numbering
\item
  Rendering only a range of lines from the source code, keeping the
  original highlighting of the code (For example, block comments are
  still rendered well, even if cut)
\item
  Adding in-code line labels which are easily referenceable (via
  \texttt{\ reference\ } )
\item
  Considerable customization options for the display of the code block
\item
  Consistent and pretty cutoff between pages
\item
  Displaying the language used for a code block in a readable manner,
  in-code-block
\end{itemize}

First, import the package via:

\begin{Shaded}
\begin{Highlighting}[]
\NormalTok{\#import "@preview/sourcerer:0.2.1": code}
\end{Highlighting}
\end{Shaded}

Then, display custom code blocks via the \texttt{\ code\ } function,
like so:

\begin{Shaded}
\begin{Highlighting}[]
\NormalTok{\#code(}
\NormalTok{  lang: "Typst",}
\NormalTok{  \textasciigrave{}\textasciigrave{}\textasciigrave{}typ}
\NormalTok{  Woah, that\textquotesingle{}s pretty \#smallcaps(cool)!}
\NormalTok{  That\textquotesingle{}s neat too.}
\NormalTok{  \textasciigrave{}\textasciigrave{}\textasciigrave{}}
\NormalTok{)}
\end{Highlighting}
\end{Shaded}

This results in:

\includegraphics[width=7.8125in,height=\textheight,keepaspectratio]{https://github.com/typst/packages/raw/main/packages/preview/sourcerer/0.2.1/assets/sourcerer.png}

To view all of the options of the \texttt{\ code\ } function, consult
the
\href{https://github.com/typst/packages/raw/main/packages/preview/sourcerer/0.2.1/DOCS.md}{documentation}
.

\subsubsection{How to add}\label{how-to-add}

Copy this into your project and use the import as \texttt{\ sourcerer\ }

\begin{verbatim}
#import "@preview/sourcerer:0.2.1"
\end{verbatim}

\includesvg[width=0.16667in,height=0.16667in]{/assets/icons/16-copy.svg}

Check the docs for
\href{https://typst.app/docs/reference/scripting/\#packages}{more
information on how to import packages} .

\subsubsection{About}\label{about}

\begin{description}
\tightlist
\item[Author :]
\href{mailto:miestrode@proton.me}{Yoav Grimland}
\item[License:]
MIT
\item[Current version:]
0.2.1
\item[Last updated:]
November 10, 2023
\item[First released:]
November 6, 2023
\item[Minimum Typst version:]
0.9.0
\item[Archive size:]
3.98 kB
\href{https://packages.typst.org/preview/sourcerer-0.2.1.tar.gz}{\pandocbounded{\includesvg[keepaspectratio]{/assets/icons/16-download.svg}}}
\item[Repository:]
\href{https://github.com/miestrode/sourcerer}{GitHub}
\end{description}

\subsubsection{Where to report issues?}\label{where-to-report-issues}

This package is a project of Yoav Grimland . Report issues on
\href{https://github.com/miestrode/sourcerer}{their repository} . You
can also try to ask for help with this package on the
\href{https://forum.typst.app}{Forum} .

Please report this package to the Typst team using the
\href{https://typst.app/contact}{contact form} if you believe it is a
safety hazard or infringes upon your rights.

\phantomsection\label{versions}
\subsubsection{Version history}\label{version-history}

\begin{longtable}[]{@{}ll@{}}
\toprule\noalign{}
Version & Release Date \\
\midrule\noalign{}
\endhead
\bottomrule\noalign{}
\endlastfoot
0.2.1 & November 10, 2023 \\
\href{https://typst.app/universe/package/sourcerer/0.2.0/}{0.2.0} &
November 7, 2023 \\
\href{https://typst.app/universe/package/sourcerer/0.1.0/}{0.1.0} &
November 6, 2023 \\
\end{longtable}

Typst GmbH did not create this package and cannot guarantee correct
functionality of this package or compatibility with any version of the
Typst compiler or app.


\title{typst.app/universe/package/fuzzy-cnoi-statement}

\phantomsection\label{banner}
\phantomsection\label{template-thumbnail}
\pandocbounded{\includegraphics[keepaspectratio]{https://packages.typst.org/preview/thumbnails/fuzzy-cnoi-statement-0.1.2-small.webp}}

\section{fuzzy-cnoi-statement}\label{fuzzy-cnoi-statement}

{ 0.1.2 }

A template for CNOI(Olympiad in Informatics in China)-style statements
for competitive programming

\href{/app?template=fuzzy-cnoi-statement&version=0.1.2}{Create project
in app}

\phantomsection\label{readme}
Fuzzy CNOI Statement is a template for CNOI(Olympiad in Informatics in
China)-style statements for competitive programming.

Fuzzy CNOI Statement 是一个 CNOI 题é?¢æŽ'版风æ~¼çš„ Typst
模�。

It is mainly designed to mimic the appearance of official CNOI-style
statements, which are usually generated by
\href{https://gitee.com/mulab/oi_tools}{TUACK} .

å\ldots¶ä¸»è¦?模仿国å†\ldots{} NOI
ç³»åˆ---æ¯''赛官æ--¹é¢˜é?¢çš„å¤--观。这些题é?¢ä¸€èˆ¬ç''±
\href{https://gitee.com/mulab/oi_tools}{TUACK} ç''Ÿæˆ?。

This template is not affiliated with the China Computer Federation (CCF)
or the NOI Committee. When using this template, it is recommended to
indicate the unofficial nature of the contest to avoid
misunderstandings.

该模æ?¿ä¸Žä¸­å›½è®¡ç®---机学会(CCF)ã€?NOI
å§''å`˜ä¼šå®˜æ--¹æ---~å\ldots³ã€‚在使ç''¨è¯¥æ¨¡æ?¿æ---¶ï¼Œå»ºè®®æ~‡æ˜Žæ¯''赛的é?žå®˜æ--¹æ€§è´¨ï¼Œä»¥å\ldots?é€~æˆ?误解。

\subsection{Usage}\label{usage}

Here are the fonts that this template will use, you can change the font
by passing parameters:\\
以下是该模æ?¿ä¼šç''¨åˆ°çš„å­---ä½``,ä½~å?¯ä»¥é€šè¿‡ä¼~å\ldots¥å?‚æ•°çš„æ--¹å¼?æ›´æ?¢å­---ä½``:

\begin{itemize}
\tightlist
\item
  Consolas
\item
  New Computer Modern
\item
  æ--¹æ­£ä¹¦å®‹ï¼ˆFZShuSong-Z01S)
\item
  æ--¹æ­£é»`ä½``(FZHei-B01S)
\item
  æ--¹æ­£ä»¿å®‹ï¼ˆFZFangSong-Z02S)
\item
  æ--¹æ­£æ¥·ä½``(FZKai-Z03S)
\end{itemize}

\begin{Shaded}
\begin{Highlighting}[]
\NormalTok{// Define your contest information and problem list}
\NormalTok{// 定义比赛信息和题目列表}

\NormalTok{\#let (init, title, problem{-}table, next{-}problem, filename, current{-}filename, current{-}sample{-}filename, data{-}constraints{-}table{-}args) = document{-}class(}
\NormalTok{  contest{-}info,}
\NormalTok{  prob{-}list,}
\NormalTok{)}

\NormalTok{\#show: init}

\NormalTok{\#title()}

\NormalTok{\#problem{-}table()}

\NormalTok{*注意事项(请仔细阅读)*}
\NormalTok{+ ...}

\NormalTok{\#next{-}problem()}
\NormalTok{== 题目描述}
\NormalTok{...}
\end{Highlighting}
\end{Shaded}

Refer to \texttt{\ main.typ\ } for a complete example.

\texttt{\ main.typ\ } æ??供了一个完整的示例。

\href{/app?template=fuzzy-cnoi-statement&version=0.1.2}{Create project
in app}

\subsubsection{How to use}\label{how-to-use}

Click the button above to create a new project using this template in
the Typst app.

You can also use the Typst CLI to start a new project on your computer
using this command:

\begin{verbatim}
typst init @preview/fuzzy-cnoi-statement:0.1.2
\end{verbatim}

\includesvg[width=0.16667in,height=0.16667in]{/assets/icons/16-copy.svg}

\subsubsection{About}\label{about}

\begin{description}
\tightlist
\item[Author :]
Wallbreaker5th
\item[License:]
MIT-0
\item[Current version:]
0.1.2
\item[Last updated:]
November 12, 2024
\item[First released:]
March 19, 2024
\item[Archive size:]
19.1 kB
\href{https://packages.typst.org/preview/fuzzy-cnoi-statement-0.1.2.tar.gz}{\pandocbounded{\includesvg[keepaspectratio]{/assets/icons/16-download.svg}}}
\item[Repository:]
\href{https://github.com/Wallbreaker5th/fuzzy-cnoi-statement}{GitHub}
\item[Discipline s :]
\begin{itemize}
\tightlist
\item[]
\item
  \href{https://typst.app/universe/search/?discipline=computer-science}{Computer
  Science}
\item
  \href{https://typst.app/universe/search/?discipline=education}{Education}
\end{itemize}
\item[Categor y :]
\begin{itemize}
\tightlist
\item[]
\item
  \pandocbounded{\includesvg[keepaspectratio]{/assets/icons/16-envelope.svg}}
  \href{https://typst.app/universe/search/?category=office}{Office}
\end{itemize}
\end{description}

\subsubsection{Where to report issues?}\label{where-to-report-issues}

This template is a project of Wallbreaker5th . Report issues on
\href{https://github.com/Wallbreaker5th/fuzzy-cnoi-statement}{their
repository} . You can also try to ask for help with this template on the
\href{https://forum.typst.app}{Forum} .

Please report this template to the Typst team using the
\href{https://typst.app/contact}{contact form} if you believe it is a
safety hazard or infringes upon your rights.

\phantomsection\label{versions}
\subsubsection{Version history}\label{version-history}

\begin{longtable}[]{@{}ll@{}}
\toprule\noalign{}
Version & Release Date \\
\midrule\noalign{}
\endhead
\bottomrule\noalign{}
\endlastfoot
0.1.2 & November 12, 2024 \\
\href{https://typst.app/universe/package/fuzzy-cnoi-statement/0.1.1/}{0.1.1}
& March 19, 2024 \\
\href{https://typst.app/universe/package/fuzzy-cnoi-statement/0.1.0/}{0.1.0}
& March 19, 2024 \\
\end{longtable}

Typst GmbH did not create this template and cannot guarantee correct
functionality of this template or compatibility with any version of the
Typst compiler or app.


\title{typst.app/universe/package/songb}

\phantomsection\label{banner}
\section{songb}\label{songb}

{ 0.1.0 }

A songbook package, to display chords above the lyrics and show a
number-based index (similar to patacrep)

\phantomsection\label{readme}
Attempt at creating a songbook package to replace
\href{https://github.com/patacrep/patacrep}{patacrep} (which is based on
LaTeX + \href{https://songs.sourceforge.net/}{Songs} ).

\subsection{Quickstart}\label{quickstart}

First, create a \texttt{\ main.typ\ } file, like the following:

\begin{Shaded}
\begin{Highlighting}[]
\NormalTok{\#set page(paper: "a6",margin: (inside: 14mm, outside: 6mm, y: 10mm))}

\NormalTok{\#import "@preview/songb:0.1.0": autobreak, index{-}by{-}letter}

\NormalTok{// helper function, to include you own songs (feel free to customize)}
\NormalTok{\#let song(path) = \{}
\NormalTok{    // WARNING: autobreak is currently broken (does not converge)}
\NormalTok{    // see https://github.com/typst/typst/discussions/4530}
\NormalTok{    autobreak(include path)}
\NormalTok{    v({-}1.19em)}
\NormalTok{\}}

\NormalTok{// indexes (put them wherever you want, or comment them out)}
\NormalTok{= Song Index}
\NormalTok{\#index{-}by{-}letter(\textless{}song\textgreater{})}

\NormalTok{= Singer Index}
\NormalTok{\#index{-}by{-}letter(\textless{}singer\textgreater{})}

\NormalTok{\#pagebreak()}

\NormalTok{// include all you songs, in the right order}
\NormalTok{\#song("./songs/first\_song.typ")}

\NormalTok{\#song("./songs/other\_song.typ")}

\NormalTok{// ...}
\end{Highlighting}
\end{Shaded}

Then, create your song files, like \texttt{\ songs/first\_song.typ\ } :

\begin{Shaded}
\begin{Highlighting}[]
\NormalTok{\#import "@preview/songb:0.1.0": song, chorus, verse, chord}

\NormalTok{\#show: doc =\textgreater{} song(}
\NormalTok{  title: "First Song",}
\NormalTok{  singer: "Sing",}
\NormalTok{  doc,}
\NormalTok{)}

\NormalTok{\#chorus[}
\NormalTok{  \#chord[Am]First line,\#chord[G][ ]of the chorus\textbackslash{}}
\NormalTok{  \#chord[Am]Second line,\#chord[G][ ]of the chorus.}
\NormalTok{]}


\NormalTok{\#verse[}
\NormalTok{  \#chord[Em]First verse\textbackslash{}}
\NormalTok{  With multiple\textbackslash{}}
\NormalTok{  \#chord[C]Lines}
\NormalTok{]}

\NormalTok{If there is \#chord[D][a] bridge\textbackslash{}}
\NormalTok{you can write it directly}
\end{Highlighting}
\end{Shaded}

\subsection{Writing a song}\label{writing-a-song}

\subsubsection{song}\label{song}

\begin{Shaded}
\begin{Highlighting}[]
\NormalTok{\#let song(}
\NormalTok{  title: none,}
\NormalTok{  title{-}index: none,}
\NormalTok{  singer: none,}
\NormalTok{  singer{-}index: none,}
\NormalTok{  references: (),}
\NormalTok{  line{-}color: rgb(0xd0, 0xd0, 0xd0),}
\NormalTok{  header{-}display: (number, title, singer) =\textgreater{} (...),}
\NormalTok{  doc}
\NormalTok{)}
\end{Highlighting}
\end{Shaded}

\subsubsection{chord}\label{chord}

\begin{Shaded}
\begin{Highlighting}[]
\NormalTok{// first argument: chord name}
\NormalTok{// optional second argument: text below the chord (useful for whitespace for instance)}
\NormalTok{\#let chord(..content)}
\end{Highlighting}
\end{Shaded}

\subsubsection{verse}\label{verse}

\begin{Shaded}
\begin{Highlighting}[]
\NormalTok{\#let verse(body)}
\end{Highlighting}
\end{Shaded}

\subsubsection{chorus}\label{chorus}

\begin{Shaded}
\begin{Highlighting}[]
\NormalTok{\#let chorus(body)}
\end{Highlighting}
\end{Shaded}

\subsection{Organizing songs}\label{organizing-songs}

\subsubsection{autobreak}\label{autobreak}

\begin{quote}
{[}!WARNING{]} Currently broken (lack of convergence for bigger
documents) See \url{https://github.com/typst/typst/discussions/4530}
\end{quote}

This function aims at putting the content on a single page (or on facing
pages), by introducing pagebreaks when needed.

\begin{Shaded}
\begin{Highlighting}[]
\NormalTok{\#let autobreak(content)}
\end{Highlighting}
\end{Shaded}

\subsubsection{index-by-letter}\label{index-by-letter}

\begin{Shaded}
\begin{Highlighting}[]
\NormalTok{\#let index{-}by{-}letter(label, letter{-}highlight: (letter) =\textgreater{} (...))}
\end{Highlighting}
\end{Shaded}

label: \texttt{\ \textless{}song\textgreater{}\ } or
\texttt{\ \textless{}singer\textgreater{}\ } are provided by the
\texttt{\ song\ } function.

\subsubsection{How to add}\label{how-to-add}

Copy this into your project and use the import as \texttt{\ songb\ }

\begin{verbatim}
#import "@preview/songb:0.1.0"
\end{verbatim}

\includesvg[width=0.16667in,height=0.16667in]{/assets/icons/16-copy.svg}

Check the docs for
\href{https://typst.app/docs/reference/scripting/\#packages}{more
information on how to import packages} .

\subsubsection{About}\label{about}

\begin{description}
\tightlist
\item[Author :]
\href{mailto:git@olivier.pfad.fr}{Oliverpool}
\item[License:]
EUPL-1.2+
\item[Current version:]
0.1.0
\item[Last updated:]
July 25, 2024
\item[First released:]
July 25, 2024
\item[Archive size:]
12.7 kB
\href{https://packages.typst.org/preview/songb-0.1.0.tar.gz}{\pandocbounded{\includesvg[keepaspectratio]{/assets/icons/16-download.svg}}}
\item[Repository:]
\href{https://codeberg.org/pfad.fr/typst-songbook}{Codeberg}
\item[Discipline :]
\begin{itemize}
\tightlist
\item[]
\item
  \href{https://typst.app/universe/search/?discipline=music}{Music}
\end{itemize}
\end{description}

\subsubsection{Where to report issues?}\label{where-to-report-issues}

This package is a project of Oliverpool . Report issues on
\href{https://codeberg.org/pfad.fr/typst-songbook}{their repository} .
You can also try to ask for help with this package on the
\href{https://forum.typst.app}{Forum} .

Please report this package to the Typst team using the
\href{https://typst.app/contact}{contact form} if you believe it is a
safety hazard or infringes upon your rights.

\phantomsection\label{versions}
\subsubsection{Version history}\label{version-history}

\begin{longtable}[]{@{}ll@{}}
\toprule\noalign{}
Version & Release Date \\
\midrule\noalign{}
\endhead
\bottomrule\noalign{}
\endlastfoot
0.1.0 & July 25, 2024 \\
\end{longtable}

Typst GmbH did not create this package and cannot guarantee correct
functionality of this package or compatibility with any version of the
Typst compiler or app.


\title{typst.app/universe/package/enunciado-facil-fcfm}

\phantomsection\label{banner}
\phantomsection\label{template-thumbnail}
\pandocbounded{\includegraphics[keepaspectratio]{https://packages.typst.org/preview/thumbnails/enunciado-facil-fcfm-0.1.0-small.webp}}

\section{enunciado-facil-fcfm}\label{enunciado-facil-fcfm}

{ 0.1.0 }

Documentos de ejercicios (controles, auxiliares, tareas, pautas) para la
FCFM, UChile

\href{/app?template=enunciado-facil-fcfm&version=0.1.0}{Create project
in app}

\phantomsection\label{readme}
Template de Typst para documentos de la FCFM (auxiliares, controles,
pautas)

\subsection{Ejemplo de uso}\label{ejemplo-de-uso}

\subsubsection{\texorpdfstring{En
\href{https://typst.app/}{typst.app}}{En typst.app}}\label{en-typst.app}

Si utilizas la aplicación web oficial, puedes presionar “Start from
template� y buscar “enunciado-facil-fcfm� para crear un proyecto
ya inicializado con el template.

\subsubsection{En CLI}\label{en-cli}

Si usas Typst de manera local, puedes ejecutar:

\begin{Shaded}
\begin{Highlighting}[]
\ExtensionTok{typst}\NormalTok{ init @preview/enunciado{-}facil{-}fcfm:0.1.0}
\end{Highlighting}
\end{Shaded}

lo cual inicializará un proyecto usando el template en el directorio
actual.

\subsubsection{Manualmente}\label{manualmente}

Basta crear un archivo con el siguiente contenido para usar el template:

\begin{Shaded}
\begin{Highlighting}[]
\NormalTok{\#import "@preview/enunciado{-}facil{-}fcfm:0.1.0" as template}

\NormalTok{\#show: template.conf.with(}
\NormalTok{  titulo: "Auxiliar 1",}
\NormalTok{  subtitulo: "Typst",}
\NormalTok{  titulo{-}extra: (}
\NormalTok{    [*Profesora*: Ada Lovelace],}
\NormalTok{    [*Auxiliares*: Grace Hopper y Alan Turing],}
\NormalTok{  ),}
\NormalTok{  departamento: template.departamentos.dcc,}
\NormalTok{  curso: "CC4034 {-} Composición de documentos",}
\NormalTok{)}

\NormalTok{...el resto del documento comienza acá}
\end{Highlighting}
\end{Shaded}

Puedes ver un ejemplo más completo en
\href{https://github.com/typst/packages/raw/main/packages/preview/enunciado-facil-fcfm/0.1.0/template/main.typ}{main.typ}
. Para aprender la sintáxis de Typst existe la
\href{https://typst.app/docs}{documentación oficial} . Si vienes desde
LaTeX, te recomiendo la
\href{https://typst.app/docs/guides/guide-for-latex-users/}{guía para
usuarios de LaTeX} .

\subsection{Configuración}\label{configuraciuxe3uxb3n}

La función \texttt{\ conf\ } importada desde el template recibe los
siguientes parámetros:

\begin{longtable}[]{@{}ll@{}}
\toprule\noalign{}
Parámetro & Descripción \\
\midrule\noalign{}
\endhead
\bottomrule\noalign{}
\endlastfoot
\texttt{\ titulo\ } & Título del documento \\
\texttt{\ subtitulo\ } & Subtítulo del documento \\
\texttt{\ titulo-extra\ } & Arreglo con bloques de contenido adicionales
a agregar después del título. Útil para mostrar los nombres del equipo
docente. \\
\texttt{\ departamento\ } & Diccionario que contiene el nombre (
\texttt{\ string\ } ) y el logo del departamento ( \texttt{\ content\ }
). El template viene con uno ya creado para cada departamento bajo
\texttt{\ template.departamentos\ } . Valor por defecto:
\texttt{\ template.departamentos.dcc\ } \\
\texttt{\ curso\ } & Código y/o nombre del curso. \\
\texttt{\ page-conf\ } & Diccionario con parámetros adicionales
(tamaño de página, márgenes, etc) para pasarle a la función
\href{https://typst.app/docs/reference/layout/page/}{page} . \\
\end{longtable}

\subsection{FAQ}\label{faq}

\subsubsection{Cómo cambiar el logo del
departamento}\label{cuxe3uxb3mo-cambiar-el-logo-del-departamento}

El parámetro \texttt{\ departamento\ } solamente es un diccionario de
Typst con las llaves \texttt{\ nombre\ } y \texttt{\ logo\ } . Puedes
crear un diccionario con un logo personalizado y pasárselo al template:

\begin{Shaded}
\begin{Highlighting}[]
\NormalTok{\#import "@preview/enunciado{-}facil{-}fcfm:0.1.0" as template}

\NormalTok{\#let mi{-}departamento = (}
\NormalTok{  nombre: "Mi súper departamento personalizado",}
\NormalTok{  logo: image("mi{-}super{-}logo.png"),}
\NormalTok{)}

\NormalTok{\#show: template.conf.with(}
\NormalTok{  titulo: "Documento con logo personalizado",}
\NormalTok{  departamento: mi{-}departamento,}
\NormalTok{  curso: "CC4034 {-} Composición de documentos",}
\NormalTok{)}
\end{Highlighting}
\end{Shaded}

\subsubsection{Cómo cambiar márgenes, tamaño de página,
etcétera}\label{cuxe3uxb3mo-cambiar-muxe3rgenes-tamauxe3o-de-puxe3gina-etcuxe3tera}

Para cambiar la configuración de la página hay que interceptar la
\href{https://typst.app/docs/reference/styling/\#set-rules}{set rule}
que se hace sobre \texttt{\ page\ } . Para ello, el template expone el
parámetro \texttt{\ page-conf\ } que permit sobreescribir la
configuración de página del template. Por ejemplo, para cambiar el
tamaño del papel a A4:

\begin{Shaded}
\begin{Highlighting}[]
\NormalTok{\#import "@preview/enunciado{-}facil{-}fcfm:0.1.0" as template}

\NormalTok{\#show: template.conf.with(}
\NormalTok{  titulo: "Documento con tamaño A4",}
\NormalTok{  departamento: template.departamentos.dcc,}
\NormalTok{  curso: "CC4034 {-} Composición de documentos",}
\NormalTok{  page{-}conf: (paper: "a4")}
\NormalTok{)}
\end{Highlighting}
\end{Shaded}

\subsubsection{Cómo cambiar la fuente, headings,
etc}\label{cuxe3uxb3mo-cambiar-la-fuente-headings-etc}

Usando \href{https://typst.app/docs/reference/styling/}{show y set
rules} puedes personalizar mucho más el template. Por ejemplo, para
cambiar la fuente:

\begin{Shaded}
\begin{Highlighting}[]
\NormalTok{\#import "@preview/enunciado{-}facil{-}fcfm:0.1.0" as template}

\NormalTok{// En este caso hay que cambiar la fuente}
\NormalTok{// antes de que se configure el template}
\NormalTok{// para que se aplique en el título y encabezado}
\NormalTok{\#set text(font: "New Computer Modern")}

\NormalTok{\#show: template.conf.with(}
\NormalTok{  titulo: "Documento con la fuente de LaTeX",}
\NormalTok{  departamento: template.departamentos.dcc,}
\NormalTok{  curso: "CC4034 {-} Composición de documentos",}
\NormalTok{)}
\end{Highlighting}
\end{Shaded}

\href{/app?template=enunciado-facil-fcfm&version=0.1.0}{Create project
in app}

\subsubsection{How to use}\label{how-to-use}

Click the button above to create a new project using this template in
the Typst app.

You can also use the Typst CLI to start a new project on your computer
using this command:

\begin{verbatim}
typst init @preview/enunciado-facil-fcfm:0.1.0
\end{verbatim}

\includesvg[width=0.16667in,height=0.16667in]{/assets/icons/16-copy.svg}

\subsubsection{About}\label{about}

\begin{description}
\tightlist
\item[Author :]
\href{https://github.com/bkorecic}{Blaz Korecic}
\item[License:]
MIT
\item[Current version:]
0.1.0
\item[Last updated:]
October 9, 2024
\item[First released:]
October 9, 2024
\item[Archive size:]
264 kB
\href{https://packages.typst.org/preview/enunciado-facil-fcfm-0.1.0.tar.gz}{\pandocbounded{\includesvg[keepaspectratio]{/assets/icons/16-download.svg}}}
\item[Repository:]
\href{https://github.com/bkorecic/enunciado-facil-fcfm}{GitHub}
\item[Categor y :]
\begin{itemize}
\tightlist
\item[]
\item
  \pandocbounded{\includesvg[keepaspectratio]{/assets/icons/16-speak.svg}}
  \href{https://typst.app/universe/search/?category=report}{Report}
\end{itemize}
\end{description}

\subsubsection{Where to report issues?}\label{where-to-report-issues}

This template is a project of Blaz Korecic . Report issues on
\href{https://github.com/bkorecic/enunciado-facil-fcfm}{their
repository} . You can also try to ask for help with this template on the
\href{https://forum.typst.app}{Forum} .

Please report this template to the Typst team using the
\href{https://typst.app/contact}{contact form} if you believe it is a
safety hazard or infringes upon your rights.

\phantomsection\label{versions}
\subsubsection{Version history}\label{version-history}

\begin{longtable}[]{@{}ll@{}}
\toprule\noalign{}
Version & Release Date \\
\midrule\noalign{}
\endhead
\bottomrule\noalign{}
\endlastfoot
0.1.0 & October 9, 2024 \\
\end{longtable}

Typst GmbH did not create this template and cannot guarantee correct
functionality of this template or compatibility with any version of the
Typst compiler or app.


\title{typst.app/universe/package/clatter}

\phantomsection\label{banner}
\section{clatter}\label{clatter}

{ 0.1.0 }

Just the PDF417 generator from rxing.

\phantomsection\label{readme}
clatter is a simple Typst package for generating PDF417 barcodes,
utilizing the \href{https://github.com/rxing-core/rxing}{rxing} library.

\subsection{Features}\label{features}

\begin{itemize}
\tightlist
\item
  \textbf{Easy to Use} : The package provides a single, intuitive
  function to generate barcodes.
\item
  \textbf{Flexible Sizing} : Control the size of the barcode with
  optional width and height parameters.
\item
  \textbf{Customizable Orientation} : Barcodes can be rendered
  horizontally or vertically, with automatic adjustment based on size.
\end{itemize}

\subsection{Usage}\label{usage}

The primary function provided by this package is \texttt{\ pdf417\ } .

\subsubsection{Parameters}\label{parameters}

\begin{itemize}
\tightlist
\item
  \texttt{\ text\ } (required): The text to encode in the barcode.
\item
  \texttt{\ width\ } (optional): The desired width of the barcode.
\item
  \texttt{\ height\ } (optional): The desired height of the barcode.
\item
  \texttt{\ direction\ } (optional): Sets the orientation of the
  barcode, either \texttt{\ "horizontal"\ } or \texttt{\ "vertical"\ } .
  If not specified, the orientation is automatically determined based on
  the provided dimensions.
\end{itemize}

\subsubsection{Sizing Behavior}\label{sizing-behavior}

\begin{itemize}
\tightlist
\item
  By default, the barcode is rendered horizontally at a reasonable size.
\item
  If both \texttt{\ width\ } and \texttt{\ height\ } are provided, the
  barcode will fit within the specified dimensions (i.e.
  \texttt{\ fit:\ "contain"\ } ).
\item
  If the \texttt{\ height\ } is greater than the \texttt{\ width\ } ,
  the barcode will automatically switch to vertical orientation unless
  \texttt{\ direction\ } is manually set.
\end{itemize}

\subsubsection{Example Usage}\label{example-usage}

\begin{Shaded}
\begin{Highlighting}[]
\NormalTok{\#import "@preview/clatter:0.1.0": pdf417}

\NormalTok{// Generate a sized horizontal PDF417 barcode }
\NormalTok{// Note: The specified size may not be exact, as the barcode will fit within the box, maintaining its aspect ratio.}
\NormalTok{\#pdf417("sized{-}barcode", width: 50mm, height: 20mm)}

\NormalTok{// Generate a vertical barcode}
\NormalTok{\#pdf417("vertical{-}barcode", direction: "vertical")}

\NormalTok{// Generate a barcode and position it on the page}
\NormalTok{\#place(top + right, pdf417("absolutely{-}positioned{-}barcode", width: 50mm), dx: {-}5mm, dy: 5mm)}
\end{Highlighting}
\end{Shaded}

\begin{center}\rule{0.5\linewidth}{0.5pt}\end{center}

{Of course, such a lengthy README can’t be written without the help of
ChatGPT.}

\subsubsection{How to add}\label{how-to-add}

Copy this into your project and use the import as \texttt{\ clatter\ }

\begin{verbatim}
#import "@preview/clatter:0.1.0"
\end{verbatim}

\includesvg[width=0.16667in,height=0.16667in]{/assets/icons/16-copy.svg}

Check the docs for
\href{https://typst.app/docs/reference/scripting/\#packages}{more
information on how to import packages} .

\subsubsection{About}\label{about}

\begin{description}
\tightlist
\item[Author :]
\href{mailto:whygowe@gmail.com}{Hung-I Wang}
\item[License:]
MIT
\item[Current version:]
0.1.0
\item[Last updated:]
August 14, 2024
\item[First released:]
August 14, 2024
\item[Archive size:]
411 kB
\href{https://packages.typst.org/preview/clatter-0.1.0.tar.gz}{\pandocbounded{\includesvg[keepaspectratio]{/assets/icons/16-download.svg}}}
\item[Repository:]
\href{https://github.com/Gowee/typst-clatter}{GitHub}
\end{description}

\subsubsection{Where to report issues?}\label{where-to-report-issues}

This package is a project of Hung-I Wang . Report issues on
\href{https://github.com/Gowee/typst-clatter}{their repository} . You
can also try to ask for help with this package on the
\href{https://forum.typst.app}{Forum} .

Please report this package to the Typst team using the
\href{https://typst.app/contact}{contact form} if you believe it is a
safety hazard or infringes upon your rights.

\phantomsection\label{versions}
\subsubsection{Version history}\label{version-history}

\begin{longtable}[]{@{}ll@{}}
\toprule\noalign{}
Version & Release Date \\
\midrule\noalign{}
\endhead
\bottomrule\noalign{}
\endlastfoot
0.1.0 & August 14, 2024 \\
\end{longtable}

Typst GmbH did not create this package and cannot guarantee correct
functionality of this package or compatibility with any version of the
Typst compiler or app.


\title{typst.app/universe/package/touying-unistra-pristine}

\phantomsection\label{banner}
\phantomsection\label{template-thumbnail}
\pandocbounded{\includegraphics[keepaspectratio]{https://packages.typst.org/preview/thumbnails/touying-unistra-pristine-1.2.0-small.webp}}

\section{touying-unistra-pristine}\label{touying-unistra-pristine}

{ 1.2.0 }

Touying theme adhering to the core principles of the style guide of the
University of Strasbourg, France

\href{/app?template=touying-unistra-pristine&version=1.2.0}{Create
project in app}

\phantomsection\label{readme}
\begin{quote}
{[}!WARNING{]} This theme is \textbf{NOT} affiliated with the University
of Strasbourg. The logo and the fonts are the property of the University
of Strasbourg.
\end{quote}

\textbf{touying-unistra-pristine} is a
\href{https://github.com/touying-typ/touying}{Touying} theme for
creating presentation slides in
\href{https://github.com/typst/typst}{Typst} , adhering to the core
principles of the \href{https://langagevisuel.unistra.fr/}{style guide
of the University of Strasbourg, France} (French). It is an
\textbf{unofficial} theme and it is \textbf{NOT} affiliated with the
University of Strasbourg.

This theme was partly created using components from
\href{https://github.com/typst-tud/tud-slides}{tud-slides} and
\href{https://github.com/piepert/grape-suite}{grape-suite} .

\begin{itemize}
\tightlist
\item
  \textbf{Focus Slides} , with predefined themes and custom colors
  support.
\item
  \textbf{Hero Slides} .
\item
  \textbf{Gallery Slides} .
\item
  \textbf{Admonitions} (with localization and plural support).
\item
  \textbf{Universally Toggleable Header/Footer} (see
  \href{https://github.com/typst/packages/raw/main/packages/preview/touying-unistra-pristine/1.2.0/\#Configuration}{Configuration}
  ).
\item
  Subset of predefined colors taken from the
  \href{https://langagevisuel.unistra.fr/index.php?id=396}{style guide
  of the University of Strasbourg} (see
  \href{https://github.com/typst/packages/raw/main/packages/preview/touying-unistra-pristine/1.2.0/colors.typ}{colors.typ}
  ).
\end{itemize}

See
\href{https://github.com/typst/packages/raw/main/packages/preview/touying-unistra-pristine/1.2.0/example/example.pdf}{example/example.pdf}
for an example PDF output, and
\href{https://github.com/typst/packages/raw/main/packages/preview/touying-unistra-pristine/1.2.0/example/example.typ}{example/example.typ}
for the corresponding Typst file.

These steps assume that you already have
\href{https://typst.app/}{Typst} installed and/or running.

\subsection{Import from Typst
Universe}\label{import-from-typst-universe}

\begin{Shaded}
\begin{Highlighting}[]
\NormalTok{\#import "@preview/touying:0.5.3": *}
\NormalTok{\#import "@preview/touying{-}unistra{-}pristine:1.2.0": *}

\NormalTok{\#show: unistra{-}theme.with(}
\NormalTok{  aspect{-}ratio: "16{-}9",}
\NormalTok{  config{-}info(}
\NormalTok{    title: [Title],}
\NormalTok{    subtitle: [\_Subtitle\_],}
\NormalTok{    author: [Author],}
\NormalTok{    date: datetime.today().display("[month repr:long] [day], [year repr:full]"),}
\NormalTok{  ),}
\NormalTok{)}

\NormalTok{\#title{-}slide[]}

\NormalTok{= Example Section Title}

\NormalTok{== Example Slide}

\NormalTok{A slide with *important information*.}

\NormalTok{\#lorem(50)}
\end{Highlighting}
\end{Shaded}

\subsection{Local installation}\label{local-installation}

\subsubsection{1. Clone the project}\label{clone-the-project}

\texttt{\ git\ clone\ https://github.com/spidersouris/touying-unistra-pristine\ }

\subsubsection{2. Import Touying and
touying-unistra-pristine}\label{import-touying-and-touying-unistra-pristine}

See
\href{https://github.com/typst/packages/raw/main/packages/preview/touying-unistra-pristine/1.2.0/example/example.typ}{example/example.typ}
for a complete example with configuration.

\begin{Shaded}
\begin{Highlighting}[]
\NormalTok{\#import "@preview/touying:0.5.3": *}
\NormalTok{\#import "src/unistra.typ": *}
\NormalTok{\#import "src/colors.typ": *}
\NormalTok{\#import "src/admonition.typ": *}

\NormalTok{\#show: unistra{-}theme.with(}
\NormalTok{  aspect{-}ratio: "16{-}9",}
\NormalTok{  config{-}info(}
\NormalTok{    title: [Title],}
\NormalTok{    subtitle: [\_Subtitle\_],}
\NormalTok{    author: [Author],}
\NormalTok{    date: datetime.today().display("[month repr:long] [day], [year repr:full]"),}
\NormalTok{  ),}
\NormalTok{)}

\NormalTok{\#title{-}slide[]}

\NormalTok{= Example Section Title}

\NormalTok{== Example Slide}

\NormalTok{A slide with *important information*.}

\NormalTok{\#lorem(50)}
\end{Highlighting}
\end{Shaded}

\begin{quote}
{[}!NOTE{]} The default font used by touying-unistra-pristine is
“Unistra A�, a font that can only be downloaded by students and
staff from the University of Strasbourg. If the font is not installed on
your computer, Segoe UI or Roboto will be used as a fallback, in that
specific order. You can change that behavior in the
\href{https://github.com/typst/packages/raw/main/packages/preview/touying-unistra-pristine/1.2.0/\#Configuration}{settings}
.
\end{quote}

The theme can be configured to your liking by adding the
\texttt{\ config-store()\ } object when initializing
\texttt{\ unistra-theme\ } . An example with the \texttt{\ quotes\ }
setting can be found in
\href{https://github.com/typst/packages/raw/main/packages/preview/touying-unistra-pristine/1.2.0/example/example.typ}{example/example.typ}
.

A complete list of settings can be found in the
\texttt{\ config-store\ } object in
\href{https://github.com/typst/packages/raw/main/packages/preview/touying-unistra-pristine/1.2.0/src/unistra.typ}{src/unistra.typ}
.

\href{/app?template=touying-unistra-pristine&version=1.2.0}{Create
project in app}

\subsubsection{How to use}\label{how-to-use}

Click the button above to create a new project using this template in
the Typst app.

You can also use the Typst CLI to start a new project on your computer
using this command:

\begin{verbatim}
typst init @preview/touying-unistra-pristine:1.2.0
\end{verbatim}

\includesvg[width=0.16667in,height=0.16667in]{/assets/icons/16-copy.svg}

\subsubsection{About}\label{about}

\begin{description}
\tightlist
\item[Author :]
\href{https://edoyen.com/}{Enzo Doyen}
\item[License:]
MIT
\item[Current version:]
1.2.0
\item[Last updated:]
November 22, 2024
\item[First released:]
September 11, 2024
\item[Minimum Typst version:]
0.12.0
\item[Archive size:]
19.7 kB
\href{https://packages.typst.org/preview/touying-unistra-pristine-1.2.0.tar.gz}{\pandocbounded{\includesvg[keepaspectratio]{/assets/icons/16-download.svg}}}
\item[Repository:]
\href{https://github.com/spidersouris/touying-unistra-pristine}{GitHub}
\item[Categor y :]
\begin{itemize}
\tightlist
\item[]
\item
  \pandocbounded{\includesvg[keepaspectratio]{/assets/icons/16-presentation.svg}}
  \href{https://typst.app/universe/search/?category=presentation}{Presentation}
\end{itemize}
\end{description}

\subsubsection{Where to report issues?}\label{where-to-report-issues}

This template is a project of Enzo Doyen . Report issues on
\href{https://github.com/spidersouris/touying-unistra-pristine}{their
repository} . You can also try to ask for help with this template on the
\href{https://forum.typst.app}{Forum} .

Please report this template to the Typst team using the
\href{https://typst.app/contact}{contact form} if you believe it is a
safety hazard or infringes upon your rights.

\phantomsection\label{versions}
\subsubsection{Version history}\label{version-history}

\begin{longtable}[]{@{}ll@{}}
\toprule\noalign{}
Version & Release Date \\
\midrule\noalign{}
\endhead
\bottomrule\noalign{}
\endlastfoot
1.2.0 & November 22, 2024 \\
\href{https://typst.app/universe/package/touying-unistra-pristine/1.1.0/}{1.1.0}
& October 17, 2024 \\
\href{https://typst.app/universe/package/touying-unistra-pristine/1.0.0/}{1.0.0}
& September 11, 2024 \\
\end{longtable}

Typst GmbH did not create this template and cannot guarantee correct
functionality of this template or compatibility with any version of the
Typst compiler or app.


\title{typst.app/universe/package/numblex}

\phantomsection\label{banner}
\section{numblex}\label{numblex}

{ 0.2.0 }

Numbering helper.

\phantomsection\label{readme}
How to number the heading like this?

\begin{itemize}
\tightlist
\item
  Appendix A. XXXX

  \begin{itemize}
  \tightlist
  \item
    A.1. YYY
  \item
    A.2. ZZZ
  \end{itemize}
\end{itemize}

Or this?

\begin{itemize}
\tightlist
\item
  一��题

  \begin{itemize}
  \tightlist
  \item
    1. 论点

    \begin{itemize}
    \tightlist
    \item
      (1) ��
    \end{itemize}
  \end{itemize}
\end{itemize}

You might use a function:

\begin{Shaded}
\begin{Highlighting}[]
\NormalTok{\#set heading(numbering: (..nums) =\textgreater{} \{}
\NormalTok{  if nums.pos().len() == 1 \{}
\NormalTok{    return "Appendix " + numbering("A.", ..nums)}
\NormalTok{  \}}
\NormalTok{  return numbering("A.1.", ..nums)}
\NormalTok{\}}
\end{Highlighting}
\end{Shaded}

Or set up a couple of \texttt{\ set\ } rules:

\begin{Shaded}
\begin{Highlighting}[]
\NormalTok{\#set heading(numbering: "A.1.")}
\NormalTok{\#show heading.where(level: 1): set heading(numbering: (n) =\textgreater{} "Appendix " + numbering("A.", n))}
\NormalTok{// No, you can\textquotesingle{}t use "Appendix A." since Typst would treat the first "A" as a numbering}
\end{Highlighting}
\end{Shaded}

Or simply use Numblex:

\begin{Shaded}
\begin{Highlighting}[]
\NormalTok{\#import "@preview/numblex:0.2.0": numblex}

\NormalTok{\#set heading(numbering: numblex("\{Appendix [A].:d==1;[A].\}\{[1].\}"))}
\end{Highlighting}
\end{Shaded}

\subsection{Usage}\label{usage}

\begin{Shaded}
\begin{Highlighting}[]
\NormalTok{\#import "@preview/numblex:0.2.0": numblex}

\NormalTok{\#set heading(numbering: numblex("\{Section [A].:d==1;[A].\}\{[1].\}\{[1])\}"))}
\end{Highlighting}
\end{Shaded}

You can read the
\href{https://github.com/ParaN3xus/numblex/blob/main/manual.pdf}{Manual}
for more information.

\subsubsection{How to add}\label{how-to-add}

Copy this into your project and use the import as \texttt{\ numblex\ }

\begin{verbatim}
#import "@preview/numblex:0.2.0"
\end{verbatim}

\includesvg[width=0.16667in,height=0.16667in]{/assets/icons/16-copy.svg}

Check the docs for
\href{https://typst.app/docs/reference/scripting/\#packages}{more
information on how to import packages} .

\subsubsection{About}\label{about}

\begin{description}
\tightlist
\item[Author s :]
\href{https://github.com/ParaN3xus}{ParaN3xus} \&
\href{https://github.com/sjfhsjfh}{sjfhsjfh}
\item[License:]
MIT
\item[Current version:]
0.2.0
\item[Last updated:]
June 24, 2024
\item[First released:]
May 3, 2024
\item[Archive size:]
217 kB
\href{https://packages.typst.org/preview/numblex-0.2.0.tar.gz}{\pandocbounded{\includesvg[keepaspectratio]{/assets/icons/16-download.svg}}}
\item[Repository:]
\href{https://github.com/ParaN3xus/numblex}{GitHub}
\item[Categor y :]
\begin{itemize}
\tightlist
\item[]
\item
  \pandocbounded{\includesvg[keepaspectratio]{/assets/icons/16-hammer.svg}}
  \href{https://typst.app/universe/search/?category=utility}{Utility}
\end{itemize}
\end{description}

\subsubsection{Where to report issues?}\label{where-to-report-issues}

This package is a project of ParaN3xus and sjfhsjfh . Report issues on
\href{https://github.com/ParaN3xus/numblex}{their repository} . You can
also try to ask for help with this package on the
\href{https://forum.typst.app}{Forum} .

Please report this package to the Typst team using the
\href{https://typst.app/contact}{contact form} if you believe it is a
safety hazard or infringes upon your rights.

\phantomsection\label{versions}
\subsubsection{Version history}\label{version-history}

\begin{longtable}[]{@{}ll@{}}
\toprule\noalign{}
Version & Release Date \\
\midrule\noalign{}
\endhead
\bottomrule\noalign{}
\endlastfoot
0.2.0 & June 24, 2024 \\
\href{https://typst.app/universe/package/numblex/0.1.1/}{0.1.1} & May 6,
2024 \\
\href{https://typst.app/universe/package/numblex/0.1.0/}{0.1.0} & May 3,
2024 \\
\end{longtable}

Typst GmbH did not create this package and cannot guarantee correct
functionality of this package or compatibility with any version of the
Typst compiler or app.


\title{typst.app/universe/package/basalt-backlinks}

\phantomsection\label{banner}
\section{basalt-backlinks}\label{basalt-backlinks}

{ 0.1.0 }

Generate and get backlinks.

\phantomsection\label{readme}
A Typst package for generating and getting backlinks.

\begin{Shaded}
\begin{Highlighting}[]
\NormalTok{\#import "@preview/basalt{-}backlinks:0.1.0" as backlinks}
\NormalTok{\#show link: backlinks.generate}
\end{Highlighting}
\end{Shaded}

\begin{Shaded}
\begin{Highlighting}[]
\NormalTok{Here\textquotesingle{}s some content I want to link to. \textless{}linktome\textgreater{}}

\NormalTok{\#pagebreak()}

\NormalTok{\#link(\textless{}linktome\textgreater{})[I\textquotesingle{}m linking to the content.]}

\NormalTok{\#pagebreak()}

\NormalTok{\#link(\textless{}linktome\textgreater{})[I\textquotesingle{}m also linking to the content!]}

\NormalTok{\#pagebreak()}

\NormalTok{\#context \{}
\NormalTok{  let backs = backlinks.get(\textless{}linktome\textgreater{})}
\NormalTok{  for (i, back) in backs.enumerate() [}
\NormalTok{    \#link(back.location())[}
\NormalTok{      Backlink for \textbackslash{}\textless{}linktome\textgreater{} (\textbackslash{}\#\#i)}
\NormalTok{    ]}

\NormalTok{  ]}
\NormalTok{\}}
\end{Highlighting}
\end{Shaded}

\subsubsection{How to add}\label{how-to-add}

Copy this into your project and use the import as
\texttt{\ basalt-backlinks\ }

\begin{verbatim}
#import "@preview/basalt-backlinks:0.1.0"
\end{verbatim}

\includesvg[width=0.16667in,height=0.16667in]{/assets/icons/16-copy.svg}

Check the docs for
\href{https://typst.app/docs/reference/scripting/\#packages}{more
information on how to import packages} .

\subsubsection{About}\label{about}

\begin{description}
\tightlist
\item[Author :]
Gabriel Talbert Bunt
\item[License:]
MIT
\item[Current version:]
0.1.0
\item[Last updated:]
October 7, 2024
\item[First released:]
October 7, 2024
\item[Archive size:]
1.59 kB
\href{https://packages.typst.org/preview/basalt-backlinks-0.1.0.tar.gz}{\pandocbounded{\includesvg[keepaspectratio]{/assets/icons/16-download.svg}}}
\item[Repository:]
\href{https://github.com/GabrielDTB/basalt-backlinks}{GitHub}
\end{description}

\subsubsection{Where to report issues?}\label{where-to-report-issues}

This package is a project of Gabriel Talbert Bunt . Report issues on
\href{https://github.com/GabrielDTB/basalt-backlinks}{their repository}
. You can also try to ask for help with this package on the
\href{https://forum.typst.app}{Forum} .

Please report this package to the Typst team using the
\href{https://typst.app/contact}{contact form} if you believe it is a
safety hazard or infringes upon your rights.

\phantomsection\label{versions}
\subsubsection{Version history}\label{version-history}

\begin{longtable}[]{@{}ll@{}}
\toprule\noalign{}
Version & Release Date \\
\midrule\noalign{}
\endhead
\bottomrule\noalign{}
\endlastfoot
0.1.0 & October 7, 2024 \\
\end{longtable}

Typst GmbH did not create this package and cannot guarantee correct
functionality of this package or compatibility with any version of the
Typst compiler or app.


\title{typst.app/universe/package/unofficial-fhict-document-template}

\phantomsection\label{banner}
\phantomsection\label{template-thumbnail}
\pandocbounded{\includegraphics[keepaspectratio]{https://packages.typst.org/preview/thumbnails/unofficial-fhict-document-template-1.1.1-small.webp}}

\section{unofficial-fhict-document-template}\label{unofficial-fhict-document-template}

{ 1.1.1 }

This is a document template for creating professional-looking documents
with Typst, tailored for FHICT (Fontys Hogeschool ICT).

\href{/app?template=unofficial-fhict-document-template&version=1.1.1}{Create
project in app}

\phantomsection\label{readme}
\pandocbounded{\includegraphics[keepaspectratio]{https://img.shields.io/github/stars/TomVer99/FHICT-typst-template?style=flat-square}}
\pandocbounded{\includegraphics[keepaspectratio]{https://img.shields.io/github/v/release/TomVer99/FHICT-typst-template?style=flat-square}}

\pandocbounded{\includegraphics[keepaspectratio]{https://img.shields.io/maintenance/Yes/2024?style=flat-square}}
\pandocbounded{\includegraphics[keepaspectratio]{https://img.shields.io/github/issues-raw/TomVer99/FHICT-typst-template?label=Issues&style=flat-square}}
\pandocbounded{\includegraphics[keepaspectratio]{https://img.shields.io/github/commits-since/TomVer99/FHICT-typst-template/latest?style=flat-square}}

This is a document template for creating professional-looking documents
with Typst, tailored for FHICT (Fontys Hogeschool ICT).

\subsection{Introduction}\label{introduction}

Creating well-structured and visually appealing documents is crucial in
academic and professional settings. This template is designed to help
FHICT students and faculty produce professional looking documents.

\includegraphics[width=0.49\linewidth,height=\textheight,keepaspectratio]{https://github.com/typst/packages/raw/main/packages/preview/unofficial-fhict-document-template/1.1.1/thumbnail.png}
\includegraphics[width=0.49\linewidth,height=\textheight,keepaspectratio]{https://github.com/typst/packages/raw/main/packages/preview/unofficial-fhict-document-template/1.1.1/showcase-r.png}

\subsection{Features}\label{features}

\begin{itemize}
\tightlist
\item
  Consistent formatting for titles, headings, subheadings, paragraphs
  and other elements.
\item
  Clean and professional document layout.
\item
  FHICT Style.
\item
  Configurable document options.
\item
  Helper functions.
\item
  Multiple languages support (nl, en, de, fr, es).
\end{itemize}

\subsection{Requirements}\label{requirements}

\begin{itemize}
\tightlist
\item
  Roboto font installed on your system.
\item
  Typst builder installed on your system (Explained in
  \texttt{\ Getting\ Started\ } ).
\end{itemize}

\subsection{Getting Started}\label{getting-started}

To get started with this Typst document template, follow these steps:

\begin{enumerate}
\tightlist
\item
  \textbf{Check for the roboto font} : Check if you have the roboto font
  installed on your system. If you don’t, you can download it from
  \href{https://fonts.google.com/specimen/Roboto}{Google Fonts} .
\item
  \textbf{Install Typst} : I recommend to use VSCode with
  \href{https://marketplace.visualstudio.com/items?itemName=myriad-dreamin.tinymist}{Tinymist
  Typst Extension} . You will also need a PDF viewer in VSCode if you
  want to view the document live.
\item
  \textbf{Import the template} : Import the template into your own typst
  document.
  \texttt{\ \#import\ "@preview/unofficial-fhict-document-template:1.1.1":\ *\ }
\item
  \textbf{Set the available options} : Set the available options in the
  template file to your liking.
\item
  \textbf{Start writing} : Start writing your document.
\end{enumerate}

\subsection{Helpful Links / Resources}\label{helpful-links-resources}

\begin{itemize}
\tightlist
\item
  The manual contains a list of all available options and helper
  functions. It can be found
  \href{https://github.com/TomVer99/FHICT-typst-template/blob/main/documentation/manual.pdf}{here}
  or attached to the latest release.
\item
  The \href{https://typst.app/docs/}{Typst Documentation} is a great
  resource for learning how to use Typst.
\item
  The bibliography file is written in
  \href{http://www.bibtex.org/Format/}{BibTeX} . You can use
  \href{https://truben.no/latex/bibtex/}{BibTeX Editor} to easily create
  and edit your bibliography.
\item
  You can use sub files to split your document into multiple files. This
  is especially useful for large documents.
\end{itemize}

\subsection{Contributing}\label{contributing}

I welcome contributions to improve and expand this document template. If
you have ideas, suggestions, or encounter issues, please consider
contributing by creating a pull request or issue.

\subsubsection{Adding a new language}\label{adding-a-new-language}

Currently, the template supports the following languages:
\texttt{\ Dutch\ } \texttt{\ (nl)\ } , \texttt{\ English\ }
\texttt{\ (en)\ } , \texttt{\ German\ } \texttt{\ (de)\ } ,
\texttt{\ French\ } \texttt{\ (fr)\ } , and \texttt{\ Spanish\ }
\texttt{\ (es)\ } . If you want to add a new language, you can do so by
following these steps:

\begin{enumerate}
\tightlist
\item
  Add the language to the \texttt{\ language.yml\ } file in the
  \texttt{\ assets\ } folder. Copy the \texttt{\ en\ } section and
  replace the values with the new language.
\item
  Add a flag \texttt{\ XX-flag.svg\ } to the \texttt{\ assets\ } folder.
\item
  Update the README with the new language.
\item
  Create a pull request with the changes.
\end{enumerate}

\subsection{Disclaimer}\label{disclaimer}

This template / repository is not endorsed by, directly affiliated with,
maintained, authorized or sponsored by Fontys Hogeschool ICT. It is
provided as-is, without any warranty or guarantee of any kind. Use at
your own risk.

The author was/is a student at Fontys Hogeschool ICT and created this
template for personal use. It is shared publicly in the hope that it
will be useful to others.

\href{/app?template=unofficial-fhict-document-template&version=1.1.1}{Create
project in app}

\subsubsection{How to use}\label{how-to-use}

Click the button above to create a new project using this template in
the Typst app.

You can also use the Typst CLI to start a new project on your computer
using this command:

\begin{verbatim}
typst init @preview/unofficial-fhict-document-template:1.1.1
\end{verbatim}

\includesvg[width=0.16667in,height=0.16667in]{/assets/icons/16-copy.svg}

\subsubsection{About}\label{about}

\begin{description}
\tightlist
\item[Author :]
TomVer99
\item[License:]
MIT
\item[Current version:]
1.1.1
\item[Last updated:]
November 12, 2024
\item[First released:]
June 3, 2024
\item[Minimum Typst version:]
0.12.0
\item[Archive size:]
227 kB
\href{https://packages.typst.org/preview/unofficial-fhict-document-template-1.1.1.tar.gz}{\pandocbounded{\includesvg[keepaspectratio]{/assets/icons/16-download.svg}}}
\item[Repository:]
\href{https://github.com/TomVer99/FHICT-typst-template}{GitHub}
\item[Categor ies :]
\begin{itemize}
\tightlist
\item[]
\item
  \pandocbounded{\includesvg[keepaspectratio]{/assets/icons/16-speak.svg}}
  \href{https://typst.app/universe/search/?category=report}{Report}
\item
  \pandocbounded{\includesvg[keepaspectratio]{/assets/icons/16-layout.svg}}
  \href{https://typst.app/universe/search/?category=layout}{Layout}
\item
  \pandocbounded{\includesvg[keepaspectratio]{/assets/icons/16-mortarboard.svg}}
  \href{https://typst.app/universe/search/?category=thesis}{Thesis}
\end{itemize}
\end{description}

\subsubsection{Where to report issues?}\label{where-to-report-issues}

This template is a project of TomVer99 . Report issues on
\href{https://github.com/TomVer99/FHICT-typst-template}{their
repository} . You can also try to ask for help with this template on the
\href{https://forum.typst.app}{Forum} .

Please report this template to the Typst team using the
\href{https://typst.app/contact}{contact form} if you believe it is a
safety hazard or infringes upon your rights.

\phantomsection\label{versions}
\subsubsection{Version history}\label{version-history}

\begin{longtable}[]{@{}ll@{}}
\toprule\noalign{}
Version & Release Date \\
\midrule\noalign{}
\endhead
\bottomrule\noalign{}
\endlastfoot
1.1.1 & November 12, 2024 \\
\href{https://typst.app/universe/package/unofficial-fhict-document-template/1.1.0/}{1.1.0}
& November 6, 2024 \\
\href{https://typst.app/universe/package/unofficial-fhict-document-template/1.0.2/}{1.0.2}
& September 17, 2024 \\
\href{https://typst.app/universe/package/unofficial-fhict-document-template/1.0.1/}{1.0.1}
& September 11, 2024 \\
\href{https://typst.app/universe/package/unofficial-fhict-document-template/1.0.0/}{1.0.0}
& August 19, 2024 \\
\href{https://typst.app/universe/package/unofficial-fhict-document-template/0.11.0/}{0.11.0}
& July 22, 2024 \\
\href{https://typst.app/universe/package/unofficial-fhict-document-template/0.10.1/}{0.10.1}
& June 12, 2024 \\
\href{https://typst.app/universe/package/unofficial-fhict-document-template/0.10.0/}{0.10.0}
& June 3, 2024 \\
\end{longtable}

Typst GmbH did not create this template and cannot guarantee correct
functionality of this template or compatibility with any version of the
Typst compiler or app.


\title{typst.app/universe/package/appreciated-letter}

\phantomsection\label{banner}
\phantomsection\label{template-thumbnail}
\pandocbounded{\includegraphics[keepaspectratio]{https://packages.typst.org/preview/thumbnails/appreciated-letter-0.1.0-small.webp}}

\section{appreciated-letter}\label{appreciated-letter}

{ 0.1.0 }

Correspond with business associates and your friends via mail

\href{/app?template=appreciated-letter&version=0.1.0}{Create project in
app}

\phantomsection\label{readme}
A basic letter with sender and recipient address. The letter is ready
for a DIN DL windowed envelope.

\subsection{Usage}\label{usage}

You can use this template in the Typst web app by clicking “Start from
template� on the dashboard and searching for
\texttt{\ appreciated-letter\ } .

Alternatively, you can use the CLI to kick this project off using the
command

\begin{verbatim}
typst init @preview/appreciated-letter
\end{verbatim}

Typst will create a new directory with all the files needed to get you
started.

\subsection{Configuration}\label{configuration}

This template exports the \texttt{\ letter\ } function with the
following named arguments:

\begin{itemize}
\tightlist
\item
  \texttt{\ sender\ } : The letter’s sender as content. This is
  displayed at the top of the page.
\item
  \texttt{\ recipient\ } : The address of the letter’s recipient as
  content. This is displayed near the top of the page.
\item
  \texttt{\ date\ } : The date, and possibly place, the letter was
  written at as content. Flushed to the right after the address.
\item
  \texttt{\ subject\ } : The subject line for the letter as content.
\item
  \texttt{\ name\ } : The name the letter closes with as content.
\end{itemize}

The function also accepts a single, positional argument for the body of
the letter.

The template will initialize your package with a sample call to the
\texttt{\ letter\ } function in a show rule. If you, however, want to
change an existing project to use this template, you can add a show rule
like this at the top of your file:

\begin{Shaded}
\begin{Highlighting}[]
\NormalTok{\#import "@preview/appreciated{-}letter:0.1.0": letter}

\NormalTok{\#show: letter.with(}
\NormalTok{  sender: [}
\NormalTok{    Jane Smith, Universal Exports, 1 Heavy Plaza, Morristown, NJ 07964}
\NormalTok{  ],}
\NormalTok{  recipient: [}
\NormalTok{    Mr. John Doe \textbackslash{}}
\NormalTok{    Acme Corp. \textbackslash{}}
\NormalTok{    123 Glennwood Ave \textbackslash{}}
\NormalTok{    Quarto Creek, VA 22438}
\NormalTok{  ],}
\NormalTok{  date: [Morristown, June 9th, 2023],}
\NormalTok{  subject: [Revision of our Producrement Contract],}
\NormalTok{  name: [Jane Smith \textbackslash{} Regional Director],}
\NormalTok{)}

\NormalTok{Dear Joe,}

\NormalTok{\#lorem(99)}

\NormalTok{Best,}
\end{Highlighting}
\end{Shaded}

\href{/app?template=appreciated-letter&version=0.1.0}{Create project in
app}

\subsubsection{How to use}\label{how-to-use}

Click the button above to create a new project using this template in
the Typst app.

You can also use the Typst CLI to start a new project on your computer
using this command:

\begin{verbatim}
typst init @preview/appreciated-letter:0.1.0
\end{verbatim}

\includesvg[width=0.16667in,height=0.16667in]{/assets/icons/16-copy.svg}

\subsubsection{About}\label{about}

\begin{description}
\tightlist
\item[Author :]
\href{https://typst.app}{Typst GmbH}
\item[License:]
MIT-0
\item[Current version:]
0.1.0
\item[Last updated:]
March 6, 2024
\item[First released:]
March 6, 2024
\item[Minimum Typst version:]
0.6.0
\item[Archive size:]
2.33 kB
\href{https://packages.typst.org/preview/appreciated-letter-0.1.0.tar.gz}{\pandocbounded{\includesvg[keepaspectratio]{/assets/icons/16-download.svg}}}
\item[Repository:]
\href{https://github.com/typst/templates}{GitHub}
\item[Categor y :]
\begin{itemize}
\tightlist
\item[]
\item
  \pandocbounded{\includesvg[keepaspectratio]{/assets/icons/16-envelope.svg}}
  \href{https://typst.app/universe/search/?category=office}{Office}
\end{itemize}
\end{description}

\subsubsection{Where to report issues?}\label{where-to-report-issues}

This template is a project of Typst GmbH . Report issues on
\href{https://github.com/typst/templates}{their repository} . You can
also try to ask for help with this template on the
\href{https://forum.typst.app}{Forum} .

\phantomsection\label{versions}
\subsubsection{Version history}\label{version-history}

\begin{longtable}[]{@{}ll@{}}
\toprule\noalign{}
Version & Release Date \\
\midrule\noalign{}
\endhead
\bottomrule\noalign{}
\endlastfoot
0.1.0 & March 6, 2024 \\
\end{longtable}


\title{typst.app/universe/package/commute}

\phantomsection\label{banner}
\section{commute}\label{commute}

{ 0.2.0 }

A proof of concept library for commutative diagrams.

{ } Featured Package

\phantomsection\label{readme}
Proof-of-concept commutative diagrams library for
\href{https://typst.app/home}{typst}

See {[}EricWay1024/tikzcd-editor{]}{[}
\url{https://github.com/EricWay1024/tikzcd-editor} {]} for a web-based
visual diagram editor for this library!

\begin{verbatim}
#import "@preview/commute:0.2.0": node, arr, commutative-diagram

#align(center)[#commutative-diagram(
  node((0, 0), $X$),
  node((0, 1), $Y$),
  node((1, 0), $X \/ "ker"(f)$, "quot"),
  arr($X$, $Y$, $f$),
  arr("quot", (0, 1), $tilde(f)$, label-pos: right, "dashed", "inj"),
  arr($X$, "quot", $pi$),
)]
\end{verbatim}

\pandocbounded{\includegraphics[keepaspectratio]{https://github.com/typst/packages/assets/20535498/71eb8d47-b6f9-43fa-a1fd-7ff58b8d0025}}

For more usage examples look at \texttt{\ example.typ\ }

The library provides 3 functions: \texttt{\ node\ } , \texttt{\ arr\ } ,
and \texttt{\ commutative-diagram\ } . You can clone this repo and
import \texttt{\ lib.typ\ } :

\begin{verbatim}
#import "path/to/commute/lib.typ": node, arr, commutative-diagram
\end{verbatim}

Or directly use the builtin package manager:

\begin{verbatim}
#import "@preview/commute:0.2.0": node, arr, commutative-diagram
\end{verbatim}

\subsection{\texorpdfstring{\texttt{\ commutative-diagram\ }}{ commutative-diagram }}\label{commutative-diagram}

\begin{verbatim}
commutative-diagram(
  node-padding: (70pt, 70pt),
  arr-clearance: 0.7em,
  padding: 1.5em,
  debug: false,
  ..entities
)
\end{verbatim}

\texttt{\ commutative-diagram\ } returns a rectangular region containing
the nodes and arrows. All the unnamed arguments passed to
\texttt{\ commutative-diagram\ } are treated as nodes or arrows of the
diagram. These can be constructed using the \texttt{\ node\ } and
\texttt{\ arr\ } functions explained below. The other arguments are as
follows:

\begin{itemize}
\tightlist
\item
  \texttt{\ node-padding\ } : \texttt{\ (length,\ length)\ } . The space
  to leave between adjacent nodes. It’s a tuple, \texttt{\ (h,\ v)\ }
  , containing the horizontal and vertical spacing respectively.
\item
  \texttt{\ arr-clearance\ } : \texttt{\ length\ } . The default space
  between arrows’ base/tip and the diagram’s nodes.
\item
  \texttt{\ padding\ } : \texttt{\ length\ } . The padding around the
  whole diagram
\item
  \texttt{\ debug\ } : \texttt{\ bool\ } . Whether or not to display
  debug information.
\end{itemize}

\subsection{\texorpdfstring{\texttt{\ node\ }}{ node }}\label{node}

\begin{verbatim}
node(
  pos,
  label,
  id: label,
)
\end{verbatim}

Creates a new diagram node. Has the following positional arguments:

\begin{itemize}
\tightlist
\item
  \texttt{\ pos\ } : \texttt{\ (integer,\ integer)\ } . The position of
  the node in \texttt{\ (row,\ column)\ } format. Must be integers, but
  can be negative, the only thing that counts is the difference between
  the coordinares of the variuos nodes in the diagram.
\item
  \texttt{\ label\ } : \texttt{\ content\ } . The node’s label.
\item
  \texttt{\ id\ } : \texttt{\ any\ } . The node’s id, defaults to its
  label if not specified.
\end{itemize}

\subsection{\texorpdfstring{\texttt{\ arr\ }}{ arr }}\label{arr}

\begin{verbatim}
arr(
  start,
  end,
  label,
  start-space: none,
  end-space: none,
  label-pos: left,
  curve: 0deg,
  stroke: 0.45pt,
  ..options
)
\end{verbatim}

Creates an arrow. Has the following arguments:

\begin{itemize}
\tightlist
\item
  \texttt{\ start\ } : \texttt{\ (integer,\ integer)\ } or
  \texttt{\ any\ } . The position of the node from which the arrow
  starts, in \texttt{\ (row,\ column)\ } format, or its id.
\item
  \texttt{\ end\ } : \texttt{\ (integer,\ integer)\ } or
  \texttt{\ any\ } . The position of the node where the arrow ends, in
  \texttt{\ (row,\ column)\ } format, or its id.
\item
  \texttt{\ label\ } : \texttt{\ content\ } . The label to put on the
  arrow.
\item
  \texttt{\ start-space\ } : \texttt{\ length\ } . The space between the
  start node and the beginning of the arrow. You can pass
  \texttt{\ none\ } to leave a sensible default, customizable using the
  \texttt{\ arr-clearance\ } parameter of the
  \texttt{\ commutative-diagram\ } function.
\item
  \texttt{\ end-space\ } : \texttt{\ length\ } . Similar to the above.
\item
  \texttt{\ label-pos\ } : \texttt{\ length\ } or \texttt{\ left\ } or
  \texttt{\ right\ } . Where to position the arrow’s label relative to
  the arrow. A positive length means that, when looking towards the tip
  of the arrow, the label is on the left. \texttt{\ left\ } and
  \texttt{\ right\ } measure the label to automatically get a reasonable
  length. If set to \texttt{\ 0\ } ( \texttt{\ 0\ } the number, which is
  different from \texttt{\ 0pt\ } or \texttt{\ 0em\ } ) then the label
  is placed on top of the arrow, with a white background to help with
  legibility.
\item
  \texttt{\ curve\ } : \texttt{\ angle\ } . The difference in
  orientation between the start and the end of the arrow. If positive,
  the arrow curves to the right, when looking towards the tip.
\item
  \texttt{\ stroke\ } : \texttt{\ stroke\ } . The thickness and color of
  the arrows. The default should probably be fine.
\item
  \texttt{\ options\ } : \texttt{\ string\ } s. After the mandatory
  positional arguments \texttt{\ start\ } , \texttt{\ end\ } and
  \texttt{\ label\ } , any remaining unnamed argument is treated as an
  extra option. Recognized options are:

  \begin{itemize}
  \tightlist
  \item
    \texttt{\ "inj"\ } , gives the arrow a hook at the start, used for
    injective functions
  \item
    \texttt{\ "surj"\ } , gives the arrow a double tip, used for
    surjective functions
  \item
    \texttt{\ "bij"\ } , gives the arrow a tip also at the start, used
    for bijective functions
  \item
    \texttt{\ "def"\ } , gives the arrow a bar at the start, used for
    function definitions
  \item
    \texttt{\ "nat"\ } , gives the arrow a double stem, used for natural
    transformations
  \item
    All the options supported by the \texttt{\ dash\ } parameter of
    Typst’s \texttt{\ stroke\ } type, such as \texttt{\ "dashed"\ } ,
    \texttt{\ "densely-dotted"\ } , etc. These change the appearance of
    the arrow’s stem
  \end{itemize}
\end{itemize}

\subsubsection{How to add}\label{how-to-add}

Copy this into your project and use the import as \texttt{\ commute\ }

\begin{verbatim}
#import "@preview/commute:0.2.0"
\end{verbatim}

\includesvg[width=0.16667in,height=0.16667in]{/assets/icons/16-copy.svg}

Check the docs for
\href{https://typst.app/docs/reference/scripting/\#packages}{more
information on how to import packages} .

\subsubsection{About}\label{about}

\begin{description}
\tightlist
\item[Author :]
\href{https://gitlab.com/giacomogallina}{giacomogallina}
\item[License:]
MIT
\item[Current version:]
0.2.0
\item[Last updated:]
November 1, 2023
\item[First released:]
July 21, 2023
\item[Archive size:]
6.15 kB
\href{https://packages.typst.org/preview/commute-0.2.0.tar.gz}{\pandocbounded{\includesvg[keepaspectratio]{/assets/icons/16-download.svg}}}
\item[Repository:]
\href{https://gitlab.com/giacomogallina/commute}{GitLab}
\end{description}

\subsubsection{Where to report issues?}\label{where-to-report-issues}

This package is a project of giacomogallina . Report issues on
\href{https://gitlab.com/giacomogallina/commute}{their repository} . You
can also try to ask for help with this package on the
\href{https://forum.typst.app}{Forum} .

Please report this package to the Typst team using the
\href{https://typst.app/contact}{contact form} if you believe it is a
safety hazard or infringes upon your rights.

\phantomsection\label{versions}
\subsubsection{Version history}\label{version-history}

\begin{longtable}[]{@{}ll@{}}
\toprule\noalign{}
Version & Release Date \\
\midrule\noalign{}
\endhead
\bottomrule\noalign{}
\endlastfoot
0.2.0 & November 1, 2023 \\
\href{https://typst.app/universe/package/commute/0.1.0/}{0.1.0} & July
21, 2023 \\
\end{longtable}

Typst GmbH did not create this package and cannot guarantee correct
functionality of this package or compatibility with any version of the
Typst compiler or app.


\title{typst.app/universe/package/nth}

\phantomsection\label{banner}
\section{nth}\label{nth}

{ 1.0.1 }

Add english ordinals to numbers, eg. 1st, 2nd, 3rd, 4th.

\phantomsection\label{readme}
Provides functions \texttt{\ \#nth()\ } and \texttt{\ \#nths()\ } which
take a number and return it suffixed by an english ordinal.

This package is named after the nth
\href{https://ctan.org/pkg/nth}{LaTeX macro} by Donald Arseneau.

\subsection{Usage}\label{usage}

Include this line in your document to import the package.

\begin{Shaded}
\begin{Highlighting}[]
\NormalTok{\#import "@preview/nth:1.0.1": *}
\end{Highlighting}
\end{Shaded}

Then, you can use \texttt{\ \#nth()\ } to markup ordinal numbers in your
document.

For example, \texttt{\ \#nth(1)\ } shows 1st,\\
\texttt{\ \#nth(2)\ } shows 2nd,\\
\texttt{\ \#nth(3)\ } shows 3rd,\\
\texttt{\ \#nth(4)\ } shows 4th,\\
and \texttt{\ \#nth(11)\ } shows 11th.

If you want the ordinal to be in superscript, use \texttt{\ \#nths\ }
with an ‘s’ at the end.

For example, \texttt{\ \#nths(1)\ } shows 1 \textsuperscript{st} .

\subsubsection{How to add}\label{how-to-add}

Copy this into your project and use the import as \texttt{\ nth\ }

\begin{verbatim}
#import "@preview/nth:1.0.1"
\end{verbatim}

\includesvg[width=0.16667in,height=0.16667in]{/assets/icons/16-copy.svg}

Check the docs for
\href{https://typst.app/docs/reference/scripting/\#packages}{more
information on how to import packages} .

\subsubsection{About}\label{about}

\begin{description}
\tightlist
\item[Author s :]
\href{mailto:pierre.marshall@gmail.com}{Pierre Marshall} ,
\href{https://github.com/fnoaman}{fnoaman} , \&
\href{https://github.com/jeffa5}{Andrew Jeffery}
\item[License:]
MIT-0
\item[Current version:]
1.0.1
\item[Last updated:]
June 21, 2024
\item[First released:]
September 22, 2023
\item[Minimum Typst version:]
0.8.0
\item[Archive size:]
2.38 kB
\href{https://packages.typst.org/preview/nth-1.0.1.tar.gz}{\pandocbounded{\includesvg[keepaspectratio]{/assets/icons/16-download.svg}}}
\item[Repository:]
\href{https://github.com/extua/nth}{GitHub}
\item[Categor y :]
\begin{itemize}
\tightlist
\item[]
\item
  \pandocbounded{\includesvg[keepaspectratio]{/assets/icons/16-text.svg}}
  \href{https://typst.app/universe/search/?category=text}{Text}
\end{itemize}
\end{description}

\subsubsection{Where to report issues?}\label{where-to-report-issues}

This package is a project of Pierre Marshall, fnoaman, and Andrew
Jeffery . Report issues on \href{https://github.com/extua/nth}{their
repository} . You can also try to ask for help with this package on the
\href{https://forum.typst.app}{Forum} .

Please report this package to the Typst team using the
\href{https://typst.app/contact}{contact form} if you believe it is a
safety hazard or infringes upon your rights.

\phantomsection\label{versions}
\subsubsection{Version history}\label{version-history}

\begin{longtable}[]{@{}ll@{}}
\toprule\noalign{}
Version & Release Date \\
\midrule\noalign{}
\endhead
\bottomrule\noalign{}
\endlastfoot
1.0.1 & June 21, 2024 \\
\href{https://typst.app/universe/package/nth/1.0.0/}{1.0.0} & December
23, 2023 \\
\href{https://typst.app/universe/package/nth/0.2.0/}{0.2.0} & October 2,
2023 \\
\href{https://typst.app/universe/package/nth/0.1.0/}{0.1.0} & September
22, 2023 \\
\end{longtable}

Typst GmbH did not create this package and cannot guarantee correct
functionality of this package or compatibility with any version of the
Typst compiler or app.


\title{typst.app/universe/package/fauve-cdb}

\phantomsection\label{banner}
\phantomsection\label{template-thumbnail}
\pandocbounded{\includegraphics[keepaspectratio]{https://packages.typst.org/preview/thumbnails/fauve-cdb-0.1.0-small.webp}}

\section{fauve-cdb}\label{fauve-cdb}

{ 0.1.0 }

The unofficial implementation of the Collège Doctoral de Bretagne
thesis manuscript template.

\href{/app?template=fauve-cdb&version=0.1.0}{Create project in app}

\phantomsection\label{readme}
Typst template for doctoral dissertations of the French
\href{https://www.doctorat-bretagne.fr/}{Collège doctoral de Bretagne
(CdB)} . The original LaTeX template can be found
\href{https://gitlab.com/ed-matisse/latex-template}{here} .

You can use this template in the Typst web app by clicking “Start from
template� on the dashboard and searching for \texttt{\ fauve-cdb\ } .

Alternatively, you can use the CLI to kick this project off using the
command

\begin{verbatim}
typst init @preview/fauve-cdb
\end{verbatim}

Typst will create a new directory with all the files needed to get you
started.

The original LaTeX template allows selecting different themes
corresponding to different schools of the CdB. For now, we only
implemented the \href{https://ed-matisse.doctorat-bretagne.fr/}{MATISSE}
theme.

\begin{quote}
Fauve is an artistic movement of which French painter
\href{https://en.wikipedia.org/wiki/Henri_Matisse}{Henri Matisse} was a
leader.
\end{quote}

\href{/app?template=fauve-cdb&version=0.1.0}{Create project in app}

\subsubsection{How to use}\label{how-to-use}

Click the button above to create a new project using this template in
the Typst app.

You can also use the Typst CLI to start a new project on your computer
using this command:

\begin{verbatim}
typst init @preview/fauve-cdb:0.1.0
\end{verbatim}

\includesvg[width=0.16667in,height=0.16667in]{/assets/icons/16-copy.svg}

\subsubsection{About}\label{about}

\begin{description}
\tightlist
\item[Author s :]
\href{mailto:timothe.albouy@gmail.com}{Timothé Albouy} \&
\href{https://grodino.github.io}{Augustin Godinot}
\item[License:]
MIT-0
\item[Current version:]
0.1.0
\item[Last updated:]
September 25, 2024
\item[First released:]
September 25, 2024
\item[Archive size:]
122 kB
\href{https://packages.typst.org/preview/fauve-cdb-0.1.0.tar.gz}{\pandocbounded{\includesvg[keepaspectratio]{/assets/icons/16-download.svg}}}
\item[Discipline s :]
\begin{itemize}
\tightlist
\item[]
\item
  \href{https://typst.app/universe/search/?discipline=computer-science}{Computer
  Science}
\item
  \href{https://typst.app/universe/search/?discipline=mathematics}{Mathematics}
\end{itemize}
\item[Categor y :]
\begin{itemize}
\tightlist
\item[]
\item
  \pandocbounded{\includesvg[keepaspectratio]{/assets/icons/16-mortarboard.svg}}
  \href{https://typst.app/universe/search/?category=thesis}{Thesis}
\end{itemize}
\end{description}

\subsubsection{Where to report issues?}\label{where-to-report-issues}

This template is a project of Timothé Albouy and Augustin Godinot . You
can also try to ask for help with this template on the
\href{https://forum.typst.app}{Forum} .

Please report this template to the Typst team using the
\href{https://typst.app/contact}{contact form} if you believe it is a
safety hazard or infringes upon your rights.

\phantomsection\label{versions}
\subsubsection{Version history}\label{version-history}

\begin{longtable}[]{@{}ll@{}}
\toprule\noalign{}
Version & Release Date \\
\midrule\noalign{}
\endhead
\bottomrule\noalign{}
\endlastfoot
0.1.0 & September 25, 2024 \\
\end{longtable}

Typst GmbH did not create this template and cannot guarantee correct
functionality of this template or compatibility with any version of the
Typst compiler or app.


\title{typst.app/universe/package/wrap-it}

\phantomsection\label{banner}
\section{wrap-it}\label{wrap-it}

{ 0.1.1 }

Wrap text around figures and content

{ } Featured Package

\phantomsection\label{readme}
Until \ul{\ul{\url{https://github.com/typst/typst/issues/553}}} is
resolved, \texttt{\ typst\ } doesn’t natively support wrapping text
around figures or other content. However, you can use
\texttt{\ wrap-it\ } to mimic much of this functionality:

\begin{itemize}
\item
  Wrapping images left or right of their text
\item
  Specifying margins
\item
  And more
\end{itemize}

Detailed descriptions of each parameter are available in the
\ul{\ul{\href{https://github.com/ntjess/wrap-it/blob/main/docs/manual.pdf}{wrap-it
documentation}}} .

The easiest method is to import \texttt{\ wrap-it:\ wrap-content\ } from
the \texttt{\ @preview\ } package:

\begin{Shaded}
\begin{Highlighting}[]
\NormalTok{\#import "@preview/wrap{-}it:0.1.0": wrap{-}content}
\end{Highlighting}
\end{Shaded}

\subsection{Vanilla}\label{vanilla}

\begin{Shaded}
\begin{Highlighting}[]
\NormalTok{\#let fig = figure(}
\NormalTok{rect(fill: teal, radius: 0.5em, width: 8em),}
\NormalTok{caption: [A figure],}
\NormalTok{)}
\NormalTok{\#let body = lorem(30)}
\NormalTok{\#wrap{-}content(fig, body)}
\end{Highlighting}
\end{Shaded}

\pandocbounded{\includegraphics[keepaspectratio]{https://www.github.com/ntjess/wrap-it/raw/v0.1.1/assets/example-1.png}}

\subsection{Changing alignment and
margin}\label{changing-alignment-and-margin}

\begin{Shaded}
\begin{Highlighting}[]
\NormalTok{\#wrap{-}content(}
\NormalTok{fig,}
\NormalTok{body,}
\NormalTok{align: bottom + right,}
\NormalTok{column{-}gutter: 2em}
\NormalTok{)}
\end{Highlighting}
\end{Shaded}

\pandocbounded{\includegraphics[keepaspectratio]{https://www.github.com/ntjess/wrap-it/raw/v0.1.1/assets/example-2.png}}

\subsection{Uniform margin around the
image}\label{uniform-margin-around-the-image}

The easiest way to get a uniform, highly-customizable margin is through
boxing your image:

\begin{Shaded}
\begin{Highlighting}[]
\NormalTok{\#let boxed = box(fig, inset: 0.25em)}
\NormalTok{\#wrap{-}content(boxed)[}
\NormalTok{\#lorem(30)}
\NormalTok{]}
\end{Highlighting}
\end{Shaded}

\pandocbounded{\includegraphics[keepaspectratio]{https://www.github.com/ntjess/wrap-it/raw/v0.1.1/assets/example-3.png}}

\subsection{Wrapping two images in the same
paragraph}\label{wrapping-two-images-in-the-same-paragraph}

Note that for longer captions (as is the case in the bottom figure
below), providing an explicit \texttt{\ columns\ } parameter is
necessary to inform caption text of where to wrap.

\begin{Shaded}
\begin{Highlighting}[]
\NormalTok{\#let fig2 = figure(}
\NormalTok{rect(fill: lime, radius: 0.5em),}
\NormalTok{caption: [\#lorem(10)],}
\NormalTok{)}
\NormalTok{\#wrap{-}top{-}bottom(}
\NormalTok{bottom{-}kwargs: (columns: (1fr, 2fr)),}
\NormalTok{box(fig, inset: 0.25em),}
\NormalTok{fig2,}
\NormalTok{lorem(50),}
\NormalTok{)}
\end{Highlighting}
\end{Shaded}

\pandocbounded{\includegraphics[keepaspectratio]{https://www.github.com/ntjess/wrap-it/raw/v0.1.1/assets/example-4.png}}

\subsection{Adding a label to a wrapped
figure}\label{adding-a-label-to-a-wrapped-figure}

Typst can only append labels to figures in content mode. So, when
wrapping text around a figure that needs a label, you must first place
your figure in a content block with its label, then wrap it:

\begin{Shaded}
\begin{Highlighting}[]
\NormalTok{\#show ref: it =\textgreater{} underline(text(blue, it))}
\NormalTok{\#let fig = [}
\NormalTok{  \#figure(}
\NormalTok{    rect(fill: red, radius: 0.5em, width: 8em),}
\NormalTok{    caption:[Labeled]}
\NormalTok{  )\textless{}fig:lbl\textgreater{}}
\NormalTok{]}
\NormalTok{\#wrap{-}content(fig, [Fortunately, @fig:lbl\textquotesingle{}s label can be referenced within the wrapped text. \#lorem(15)])}
\end{Highlighting}
\end{Shaded}

\pandocbounded{\includegraphics[keepaspectratio]{https://www.github.com/ntjess/wrap-it/raw/v0.1.1/assets/example-5.png}}

\subsubsection{How to add}\label{how-to-add}

Copy this into your project and use the import as \texttt{\ wrap-it\ }

\begin{verbatim}
#import "@preview/wrap-it:0.1.1"
\end{verbatim}

\includesvg[width=0.16667in,height=0.16667in]{/assets/icons/16-copy.svg}

Check the docs for
\href{https://typst.app/docs/reference/scripting/\#packages}{more
information on how to import packages} .

\subsubsection{About}\label{about}

\begin{description}
\tightlist
\item[Author :]
Nathan Jessurun
\item[License:]
Unlicense
\item[Current version:]
0.1.1
\item[Last updated:]
November 28, 2024
\item[First released:]
January 26, 2024
\item[Archive size:]
5.30 kB
\href{https://packages.typst.org/preview/wrap-it-0.1.1.tar.gz}{\pandocbounded{\includesvg[keepaspectratio]{/assets/icons/16-download.svg}}}
\item[Repository:]
\href{https://github.com/ntjess/wrap-it}{GitHub}
\end{description}

\subsubsection{Where to report issues?}\label{where-to-report-issues}

This package is a project of Nathan Jessurun . Report issues on
\href{https://github.com/ntjess/wrap-it}{their repository} . You can
also try to ask for help with this package on the
\href{https://forum.typst.app}{Forum} .

Please report this package to the Typst team using the
\href{https://typst.app/contact}{contact form} if you believe it is a
safety hazard or infringes upon your rights.

\phantomsection\label{versions}
\subsubsection{Version history}\label{version-history}

\begin{longtable}[]{@{}ll@{}}
\toprule\noalign{}
Version & Release Date \\
\midrule\noalign{}
\endhead
\bottomrule\noalign{}
\endlastfoot
0.1.1 & November 28, 2024 \\
\href{https://typst.app/universe/package/wrap-it/0.1.0/}{0.1.0} &
January 26, 2024 \\
\end{longtable}

Typst GmbH did not create this package and cannot guarantee correct
functionality of this package or compatibility with any version of the
Typst compiler or app.


\title{typst.app/universe/package/clean-math-presentation}

\phantomsection\label{banner}
\phantomsection\label{template-thumbnail}
\pandocbounded{\includegraphics[keepaspectratio]{https://packages.typst.org/preview/thumbnails/clean-math-presentation-0.1.0-small.webp}}

\section{clean-math-presentation}\label{clean-math-presentation}

{ 0.1.0 }

A simple and good looking template for mathematical presentations

\href{/app?template=clean-math-presentation&version=0.1.0}{Create
project in app}

\phantomsection\label{readme}
\href{https://github.com/JoshuaLampert/clean-math-presentation/actions/workflows/build.yml}{\pandocbounded{\includesvg[keepaspectratio]{https://github.com/JoshuaLampert/clean-math-presentation/actions/workflows/build.yml/badge.svg}}}
\href{https://github.com/JoshuaLampert/clean-math-presentation}{\pandocbounded{\includegraphics[keepaspectratio]{https://img.shields.io/badge/GitHub-repo-blue}}}
\href{https://opensource.org/licenses/MIT}{\pandocbounded{\includesvg[keepaspectratio]{https://img.shields.io/badge/License-MIT-success.svg}}}

\href{https://typst.app/home/}{Typst} template for presentations built
for simple, efficient use and a clean look using
\href{https://touying-typ.github.io/}{touying} . The template provides a
custom title page, a footer, a header, and built-in support for theorem
blocks and proofs.

\subsection{Usage}\label{usage}

The template is already filled with dummy data, to give users an
impression how it looks like. The paper is obtained by compiling
\texttt{\ main.typ\ } .

\begin{itemize}
\tightlist
\item
  after
  \href{https://github.com/typst/typst?tab=readme-ov-file\#installation}{installing
  Typst} you can conveniently use the following to create a new folder
  containing this project.
\end{itemize}

\begin{Shaded}
\begin{Highlighting}[]
\ExtensionTok{typst}\NormalTok{ init @preview/clean{-}math{-}presentation:0.1.0}
\end{Highlighting}
\end{Shaded}

\begin{itemize}
\tightlist
\item
  edit the data in \texttt{\ main.typ\ } â†'
  \texttt{\ \#show\ template.with({[}your\ data{]})\ }
\end{itemize}

\subsubsection{Parameters of the
Template}\label{parameters-of-the-template}

\begin{itemize}
\tightlist
\item
  \texttt{\ title\ } : Title of the presentation.
\item
  \texttt{\ subtitle\ } : Subtitle of the presentation, optional.
\item
  \texttt{\ short-title\ } : Short version of the presentation to be
  shown in the footer, optional. If not short title is provided, the
  \texttt{\ title\ } will be shown in the footer.
\item
  \texttt{\ date\ } : Date of the presentation.
\item
  \texttt{\ authors\ } : List of names of the authors of the paper. Each
  entry of the list is a dictionary with the following keys:

  \begin{itemize}
  \tightlist
  \item
    \texttt{\ name\ } : Name of the author.
  \item
    \texttt{\ affiliation-id\ } : The ID of the affiliation in
    \texttt{\ affiliations\ } , see below.
  \end{itemize}
\item
  \texttt{\ affiliations\ } : List of affiliations of the authors. Each
  entry of the list is a dictionary with the following keys:

  \begin{itemize}
  \tightlist
  \item
    \texttt{\ id\ } : ID of the affiliation, which is used to link the
    authors to the affiliation, see above.
  \item
    \texttt{\ name\ } : Name of the affiliation.
  \end{itemize}
\item
  \texttt{\ author\ } : The name of the presenting author, which will be
  displayed in the footer of each slide. If the \texttt{\ author\ }
  matches one of the \texttt{\ authors\ } , this name will be underlined
  in the title slide.
\end{itemize}

Other arguments like \texttt{\ align\ } , \texttt{\ progess-bar\ } and
more are available and similar to other templates in touying, especially
the \href{https://touying-typ.github.io/docs/themes/stargazer}{stargazer
theme} . The colorscheme can be adjusted by passing
\texttt{\ config-colors\ } to the \texttt{\ template\ } , e.g.

\begin{Shaded}
\begin{Highlighting}[]
\NormalTok{config{-}colors(}
\NormalTok{  primary: rgb("\#6068d6"),}
\NormalTok{  secondary: rgb("\#2f1971"),}
\NormalTok{)}
\end{Highlighting}
\end{Shaded}

The title page can be created with \texttt{\ \#title-slide\ } . It
accepts optionally a \texttt{\ background\ } , which can be an image or
\texttt{\ none\ } (default) and up to two logos \texttt{\ logo1\ } and
\texttt{\ logo2\ } ( \texttt{\ none\ } by default).

The theme provides different types of slides like
\texttt{\ \#outline-slide\ } , \texttt{\ \#focus-slide\ } ,
\texttt{\ \#ending-slide\ } , and the usual \texttt{\ \#slide\ } .
Additionally,it already provides support for theorems and alike by the
functions \texttt{\ \#theorem\ } , \texttt{\ \#lemma\ } ,
\texttt{\ \#corollary\ } , \texttt{\ \#definition\ } ,
\texttt{\ \#example\ } , and \texttt{\ \#proof\ } .

\subsection{Acknowledgements}\label{acknowledgements}

Some parts of this template are based on the
\href{https://github.com/touying-typ/touying/blob/main/themes/stargazer.typ}{stargazer}
theme from touying.

\subsection{Feedback \& Improvements}\label{feedback-improvements}

If you encounter problems, please open issues. In case you found useful
extensions or improved anything We are also very happy to accept pull
requests.

\href{/app?template=clean-math-presentation&version=0.1.0}{Create
project in app}

\subsubsection{How to use}\label{how-to-use}

Click the button above to create a new project using this template in
the Typst app.

You can also use the Typst CLI to start a new project on your computer
using this command:

\begin{verbatim}
typst init @preview/clean-math-presentation:0.1.0
\end{verbatim}

\includesvg[width=0.16667in,height=0.16667in]{/assets/icons/16-copy.svg}

\subsubsection{About}\label{about}

\begin{description}
\tightlist
\item[Author :]
\href{https://github.com/JoshuaLampert}{Joshua Lampert}
\item[License:]
MIT
\item[Current version:]
0.1.0
\item[Last updated:]
November 21, 2024
\item[First released:]
November 21, 2024
\item[Minimum Typst version:]
0.12.0
\item[Archive size:]
10.3 kB
\href{https://packages.typst.org/preview/clean-math-presentation-0.1.0.tar.gz}{\pandocbounded{\includesvg[keepaspectratio]{/assets/icons/16-download.svg}}}
\item[Repository:]
\href{https://github.com/JoshuaLampert/clean-math-presentation}{GitHub}
\item[Discipline s :]
\begin{itemize}
\tightlist
\item[]
\item
  \href{https://typst.app/universe/search/?discipline=mathematics}{Mathematics}
\item
  \href{https://typst.app/universe/search/?discipline=engineering}{Engineering}
\item
  \href{https://typst.app/universe/search/?discipline=computer-science}{Computer
  Science}
\end{itemize}
\item[Categor y :]
\begin{itemize}
\tightlist
\item[]
\item
  \pandocbounded{\includesvg[keepaspectratio]{/assets/icons/16-presentation.svg}}
  \href{https://typst.app/universe/search/?category=presentation}{Presentation}
\end{itemize}
\end{description}

\subsubsection{Where to report issues?}\label{where-to-report-issues}

This template is a project of Joshua Lampert . Report issues on
\href{https://github.com/JoshuaLampert/clean-math-presentation}{their
repository} . You can also try to ask for help with this template on the
\href{https://forum.typst.app}{Forum} .

Please report this template to the Typst team using the
\href{https://typst.app/contact}{contact form} if you believe it is a
safety hazard or infringes upon your rights.

\phantomsection\label{versions}
\subsubsection{Version history}\label{version-history}

\begin{longtable}[]{@{}ll@{}}
\toprule\noalign{}
Version & Release Date \\
\midrule\noalign{}
\endhead
\bottomrule\noalign{}
\endlastfoot
0.1.0 & November 21, 2024 \\
\end{longtable}

Typst GmbH did not create this template and cannot guarantee correct
functionality of this template or compatibility with any version of the
Typst compiler or app.


\title{typst.app/universe/package/indic-numerals}

\phantomsection\label{banner}
\section{indic-numerals}\label{indic-numerals}

{ 0.1.0 }

convert arabic numerals to indic numerals and vice versa

\phantomsection\label{readme}
\href{https://github.com/cecoeco/indic-numerals/blob/main/LICENSE.md}{\pandocbounded{\includesvg[keepaspectratio]{https://img.shields.io/badge/License-MIT-blue.svg}}}

\subsection{indic-numerals}\label{indic-numerals-1}

\emph{convert arabic numerals to indic numerals and vice versa}

\begin{Shaded}
\begin{Highlighting}[]
\NormalTok{\#import "@preview/indic{-}numerals:0.1.0": arabic{-}to{-}indic, indic{-}to{-}arabic}

\NormalTok{\#indic{-}to{-}arabic("௦௧௨௩௪௫௬௭௮௯", "tamil") // Output: 0123456789}

\NormalTok{\#arabic{-}to{-}indic("0123456789", "tamil") // Output: ௦௧௨௩௪௫௬௭௮௯}
\end{Highlighting}
\end{Shaded}

\subsubsection{How to add}\label{how-to-add}

Copy this into your project and use the import as
\texttt{\ indic-numerals\ }

\begin{verbatim}
#import "@preview/indic-numerals:0.1.0"
\end{verbatim}

\includesvg[width=0.16667in,height=0.16667in]{/assets/icons/16-copy.svg}

Check the docs for
\href{https://typst.app/docs/reference/scripting/\#packages}{more
information on how to import packages} .

\subsubsection{About}\label{about}

\begin{description}
\tightlist
\item[Author :]
Ceco Elijah Maples
\item[License:]
MIT
\item[Current version:]
0.1.0
\item[Last updated:]
November 4, 2024
\item[First released:]
November 4, 2024
\item[Archive size:]
1.85 kB
\href{https://packages.typst.org/preview/indic-numerals-0.1.0.tar.gz}{\pandocbounded{\includesvg[keepaspectratio]{/assets/icons/16-download.svg}}}
\item[Repository:]
\href{https://github.com/cecoeco/indic-numerals}{GitHub}
\end{description}

\subsubsection{Where to report issues?}\label{where-to-report-issues}

This package is a project of Ceco Elijah Maples . Report issues on
\href{https://github.com/cecoeco/indic-numerals}{their repository} . You
can also try to ask for help with this package on the
\href{https://forum.typst.app}{Forum} .

Please report this package to the Typst team using the
\href{https://typst.app/contact}{contact form} if you believe it is a
safety hazard or infringes upon your rights.

\phantomsection\label{versions}
\subsubsection{Version history}\label{version-history}

\begin{longtable}[]{@{}ll@{}}
\toprule\noalign{}
Version & Release Date \\
\midrule\noalign{}
\endhead
\bottomrule\noalign{}
\endlastfoot
0.1.0 & November 4, 2024 \\
\end{longtable}

Typst GmbH did not create this package and cannot guarantee correct
functionality of this package or compatibility with any version of the
Typst compiler or app.


\title{typst.app/universe/package/silky-letter-insa}

\phantomsection\label{banner}
\phantomsection\label{template-thumbnail}
\pandocbounded{\includegraphics[keepaspectratio]{https://packages.typst.org/preview/thumbnails/silky-letter-insa-0.2.2-small.webp}}

\section{silky-letter-insa}\label{silky-letter-insa}

{ 0.2.2 }

A template made for letters and short documents of INSA, a French
engineering school.

\href{/app?template=silky-letter-insa&version=0.2.2}{Create project in
app}

\phantomsection\label{readme}
Typst Template for short documents and letters for the french
engineering school INSA.

\subsection{Example}\label{example}

By default, the template initializes with the \texttt{\ insa-letter\ }
show rule, with parameters that you must fill in by yourself.

Here is an example of filled template:

\begin{Shaded}
\begin{Highlighting}[]
\NormalTok{\#import "@preview/silky{-}letter{-}insa:0.2.2": *}
\NormalTok{\#show: doc =\textgreater{} insa{-}letter(}
\NormalTok{  author: "Youenn LE JEUNE, Kelian NINET",}
\NormalTok{  insa: "rennes"}
\NormalTok{  doc)}

\NormalTok{\#v(15pt)}
\NormalTok{\#align(center, text(size: 22pt, weight: "bold", smallcaps("Probabilités {-} Annale 2022 (V1)")))}
\NormalTok{\#v(5pt)}

\NormalTok{\#set heading(numbering: "1.")}
\NormalTok{\#show heading.where(level: 2): it =\textgreater{} [}
\NormalTok{  \#counter(heading).display()}
\NormalTok{  \#text(weight: "medium", style: "italic", size: 13pt, it.body)}

\NormalTok{]}

\NormalTok{= Intervalle de confiance}
\NormalTok{== Calculer sur l’échantillon une estimation de la moyenne.}
\NormalTok{$ overline(x\_n) = 1/n sum\_(i=1)\^{}n x\_i = 1885 $}

\NormalTok{== Calculer sur l’échantillon une estimation de la variance.}
\NormalTok{$}
\NormalTok{"Variance biaisée :" s\^{}2 \&= 1/n sum\_(i=1)\^{}n (x\_i {-} overline(x\_n))\^{}2 = 218\^{}2\textbackslash{}}
\NormalTok{"Variance corrigée :" s\textquotesingle{}\^{}2 \&=  n/(n{-}1) s\^{}2 = 231\^{}2}
\NormalTok{$}

\NormalTok{Le bon estimateur est le second.}

\NormalTok{== Écrire le code R permettant d’évaluer les deux bornes de l’intervalle de confiance du temps d’exécution avec une confiance de 92\%.}
\NormalTok{Nous sommes dans le cas d\textquotesingle{}une recherche de moyenne avec variance inconnue, l\textquotesingle{}intervalle sera donc}
\NormalTok{$ [overline(X) + t\_(n{-}1)(alpha/2) S\textquotesingle{}/sqrt(n), quad overline(X) + t\_(n{-}1)(1 {-} alpha/2) S\textquotesingle{}/sqrt(n)] $}
\NormalTok{En R, avec l\textquotesingle{}échantillon nommé \textasciigrave{}data\textasciigrave{}, ça donne}
\NormalTok{\textasciigrave{}\textasciigrave{}\textasciigrave{}R}
\NormalTok{data = c(1653, 2059, 2281, 1813, 2180, 1721, 1857, 1677, 1728)}
\NormalTok{moyenne = mean(data)}
\NormalTok{s\_prime = sqrt(var(data)) \# car la variance de R est déjà corrigée}
\NormalTok{n = 9}
\NormalTok{alpha = 0.08}

\NormalTok{IC\_min = moyenne + qt(alpha / 2, df = n {-} 1) * s\_prime / sqrt(n)}
\NormalTok{IC\_max = moyenne + qt(1 {-} alpha / 2, df = n {-} 1) * s\_prime / sqrt(n)}
\NormalTok{\textasciigrave{}\textasciigrave{}\textasciigrave{}}

\NormalTok{Ici on a $[1730, 2040]$.}
\end{Highlighting}
\end{Shaded}

\subsection{Fonts}\label{fonts}

The graphic charter recommends the fonts \textbf{League Spartan} for
headings and \textbf{Source Serif} for regular text. To have the best
look, you should install those fonts.

To behave correctly on computers without those specific fonts installed,
this template will automatically fallback to other similar fonts:

\begin{itemize}
\tightlist
\item
  \textbf{League Spartan} -\textgreater{} \textbf{Arial} (approved by
  INSA’s graphic charter, by default in Windows) -\textgreater{}
  \textbf{Liberation Sans} (by default in most Linux)
\item
  \textbf{Source Serif} -\textgreater{} \textbf{Source Serif 4}
  (downloadable for free) -\textgreater{} \textbf{Georgia} (approved by
  the graphic charter) -\textgreater{} \textbf{Linux Libertine} (default
  Typst font)
\end{itemize}

\subsubsection{Note on variable fonts}\label{note-on-variable-fonts}

If you want to install those fonts on your computer, Typst might not
recognize them if you install their \emph{Variable} versions. You should
install the static versions ( \textbf{League Spartan Bold} and most
versions of \textbf{Source Serif} ).

Keep an eye on \href{https://github.com/typst/typst/issues/185}{the
issue in Typst bug tracker} to see when variable fonts will be used!

\subsection{Notes}\label{notes}

This template is being developed by Youenn LE JEUNE from the INSA de
Rennes in \href{https://github.com/SkytAsul/INSA-Typst-Template}{this
repository} .

For now it includes assets from the graphic charters of those INSAs:

\begin{itemize}
\tightlist
\item
  Rennes ( \texttt{\ rennes\ } )
\item
  Hauts de France ( \texttt{\ hdf\ } )
\item
  Centre Val de Loire ( \texttt{\ cvl\ } ) Users from other INSAs can
  open a pull request on the repository with the assets for their INSA.
\end{itemize}

If you have any other feature request, open an issue on the repository
as well.

\subsection{License}\label{license}

The typst template is licensed under the
\href{https://github.com/SkytAsul/INSA-Typst-Template/blob/main/LICENSE}{MIT
license} . This does \emph{not} apply to the image assets. Those image
files are property of Groupe INSA.

\subsection{Changelog}\label{changelog}

\subsubsection{0.2.2}\label{section}

\begin{itemize}
\tightlist
\item
  Added INSA CVL assets
\end{itemize}

\subsubsection{0.2.1}\label{section-1}

\begin{itemize}
\tightlist
\item
  Added \texttt{\ insa\ } option
\item
  Added INSA HdF assets
\end{itemize}

\href{/app?template=silky-letter-insa&version=0.2.2}{Create project in
app}

\subsubsection{How to use}\label{how-to-use}

Click the button above to create a new project using this template in
the Typst app.

You can also use the Typst CLI to start a new project on your computer
using this command:

\begin{verbatim}
typst init @preview/silky-letter-insa:0.2.2
\end{verbatim}

\includesvg[width=0.16667in,height=0.16667in]{/assets/icons/16-copy.svg}

\subsubsection{About}\label{about}

\begin{description}
\tightlist
\item[Author :]
SkytAsul
\item[License:]
MIT
\item[Current version:]
0.2.2
\item[Last updated:]
November 21, 2024
\item[First released:]
March 23, 2024
\item[Archive size:]
269 kB
\href{https://packages.typst.org/preview/silky-letter-insa-0.2.2.tar.gz}{\pandocbounded{\includesvg[keepaspectratio]{/assets/icons/16-download.svg}}}
\item[Repository:]
\href{https://github.com/SkytAsul/INSA-Typst-Template}{GitHub}
\item[Discipline s :]
\begin{itemize}
\tightlist
\item[]
\item
  \href{https://typst.app/universe/search/?discipline=engineering}{Engineering}
\item
  \href{https://typst.app/universe/search/?discipline=computer-science}{Computer
  Science}
\item
  \href{https://typst.app/universe/search/?discipline=mathematics}{Mathematics}
\item
  \href{https://typst.app/universe/search/?discipline=physics}{Physics}
\end{itemize}
\item[Categor y :]
\begin{itemize}
\tightlist
\item[]
\item
  \pandocbounded{\includesvg[keepaspectratio]{/assets/icons/16-envelope.svg}}
  \href{https://typst.app/universe/search/?category=office}{Office}
\end{itemize}
\end{description}

\subsubsection{Where to report issues?}\label{where-to-report-issues}

This template is a project of SkytAsul . Report issues on
\href{https://github.com/SkytAsul/INSA-Typst-Template}{their repository}
. You can also try to ask for help with this template on the
\href{https://forum.typst.app}{Forum} .

Please report this template to the Typst team using the
\href{https://typst.app/contact}{contact form} if you believe it is a
safety hazard or infringes upon your rights.

\phantomsection\label{versions}
\subsubsection{Version history}\label{version-history}

\begin{longtable}[]{@{}ll@{}}
\toprule\noalign{}
Version & Release Date \\
\midrule\noalign{}
\endhead
\bottomrule\noalign{}
\endlastfoot
0.2.2 & November 21, 2024 \\
\href{https://typst.app/universe/package/silky-letter-insa/0.2.1/}{0.2.1}
& September 24, 2024 \\
\href{https://typst.app/universe/package/silky-letter-insa/0.2.0/}{0.2.0}
& June 10, 2024 \\
\href{https://typst.app/universe/package/silky-letter-insa/0.1.0/}{0.1.0}
& March 23, 2024 \\
\end{longtable}

Typst GmbH did not create this template and cannot guarantee correct
functionality of this template or compatibility with any version of the
Typst compiler or app.


\title{typst.app/universe/package/crudo}

\phantomsection\label{banner}
\section{crudo}\label{crudo}

{ 0.1.1 }

Take slices from raw blocks

\phantomsection\label{readme}
Crudo allows conveniently working with \texttt{\ raw\ } blocks in terms
of individual lines. It allows you to e.g.

\begin{itemize}
\tightlist
\item
  filter lines by content
\item
  filter lines by range (slicing)
\item
  transform lines
\item
  join multiple raw blocks
\end{itemize}

While transforming the content, the original
\href{https://typst.app/docs/reference/text/raw/\#parameters}{parameters}
specified on the given raw block will be preserved.

\subsection{Getting Started}\label{getting-started}

The full version of this example can be found in
\href{https://github.com/typst/packages/raw/main/packages/preview/crudo/0.1.1/gallery/thumbnail.typ}{gallery/thumbnail.typ}
.

\begin{Shaded}
\begin{Highlighting}[]
\NormalTok{From}

\NormalTok{\#let preamble = \textasciigrave{}\textasciigrave{}\textasciigrave{}typ}
\NormalTok{\#import "@preview/crudo:0.1.0"}

\NormalTok{\textasciigrave{}\textasciigrave{}\textasciigrave{}}
\NormalTok{\#preamble}

\NormalTok{and}

\NormalTok{\#let example = \textasciigrave{}\textasciigrave{}\textasciigrave{}\textasciigrave{}typ}
\NormalTok{\#crudo.r2l(\textasciigrave{}\textasciigrave{}\textasciigrave{}c}
\NormalTok{int main() \{}
\NormalTok{  return 0;}
\NormalTok{\}}
\NormalTok{\textasciigrave{}\textasciigrave{}\textasciigrave{})}
\NormalTok{\textasciigrave{}\textasciigrave{}\textasciigrave{}\textasciigrave{}}
\NormalTok{\#example}

\NormalTok{we get}

\NormalTok{\#let full{-}example = crudo.join(preamble, example)}
\NormalTok{\#full{-}example}

\NormalTok{If you execute that, you get}

\NormalTok{\#eval(full{-}example.text, mode: "markup")}
\end{Highlighting}
\end{Shaded}

\pandocbounded{\includegraphics[keepaspectratio]{https://github.com/typst/packages/raw/main/packages/preview/crudo/0.1.1/thumbnail.png}}

\subsection{Usage}\label{usage}

See the
\href{https://github.com/typst/packages/raw/main/packages/preview/crudo/0.1.1/docs/manual.pdf}{manual}
for details.

\subsubsection{How to add}\label{how-to-add}

Copy this into your project and use the import as \texttt{\ crudo\ }

\begin{verbatim}
#import "@preview/crudo:0.1.1"
\end{verbatim}

\includesvg[width=0.16667in,height=0.16667in]{/assets/icons/16-copy.svg}

Check the docs for
\href{https://typst.app/docs/reference/scripting/\#packages}{more
information on how to import packages} .

\subsubsection{About}\label{about}

\begin{description}
\tightlist
\item[Author :]
\href{https://github.com/SillyFreak/}{Clemens Koza}
\item[License:]
MIT
\item[Current version:]
0.1.1
\item[Last updated:]
September 30, 2024
\item[First released:]
July 15, 2024
\item[Minimum Typst version:]
0.9.0
\item[Archive size:]
4.11 kB
\href{https://packages.typst.org/preview/crudo-0.1.1.tar.gz}{\pandocbounded{\includesvg[keepaspectratio]{/assets/icons/16-download.svg}}}
\item[Repository:]
\href{https://github.com/SillyFreak/typst-crudo}{GitHub}
\item[Categor ies :]
\begin{itemize}
\tightlist
\item[]
\item
  \pandocbounded{\includesvg[keepaspectratio]{/assets/icons/16-text.svg}}
  \href{https://typst.app/universe/search/?category=text}{Text}
\item
  \pandocbounded{\includesvg[keepaspectratio]{/assets/icons/16-code.svg}}
  \href{https://typst.app/universe/search/?category=scripting}{Scripting}
\item
  \pandocbounded{\includesvg[keepaspectratio]{/assets/icons/16-hammer.svg}}
  \href{https://typst.app/universe/search/?category=utility}{Utility}
\end{itemize}
\end{description}

\subsubsection{Where to report issues?}\label{where-to-report-issues}

This package is a project of Clemens Koza . Report issues on
\href{https://github.com/SillyFreak/typst-crudo}{their repository} . You
can also try to ask for help with this package on the
\href{https://forum.typst.app}{Forum} .

Please report this package to the Typst team using the
\href{https://typst.app/contact}{contact form} if you believe it is a
safety hazard or infringes upon your rights.

\phantomsection\label{versions}
\subsubsection{Version history}\label{version-history}

\begin{longtable}[]{@{}ll@{}}
\toprule\noalign{}
Version & Release Date \\
\midrule\noalign{}
\endhead
\bottomrule\noalign{}
\endlastfoot
0.1.1 & September 30, 2024 \\
\href{https://typst.app/universe/package/crudo/0.1.0/}{0.1.0} & July 15,
2024 \\
\end{longtable}

Typst GmbH did not create this package and cannot guarantee correct
functionality of this package or compatibility with any version of the
Typst compiler or app.


\title{typst.app/universe/package/conchord}

\phantomsection\label{banner}
\section{conchord}\label{conchord}

{ 0.2.0 }

Easily write lyrics with chords, generate chord diagrams and tabs.

{ } Featured Package

\phantomsection\label{readme}
\begin{quote}
Notice: I’m preparing the update, so the documentation there is
referring to the new version.
\end{quote}

\texttt{\ conchord\ } (concise chord) is a
\href{https://github.com/typst/typst}{Typst} package to write lyrics
with chords and generate colorful fretboard diagram (aka chord diagram).
From \texttt{\ 0.1.1\ } there is also experimental tabs support (though
quite simple and unstable yet). It is inspired by
\href{https://github.com/ljgago/typst-chords}{chordx} package and my
previous tiny project for generating chord diagrams SVG-s.

\texttt{\ conchord\ } makes it easy to add new chords, both for diagrams
and lyrics. Unlike \href{https://github.com/ljgago/typst-chords}{chordx}
, you don’t need to think about layout and pass lots of arrays for
drawing barres. Just pass a string with held frets and it will work:

\begin{Shaded}
\begin{Highlighting}[]
\NormalTok{\#import "@preview/conchord:0.2.0": new{-}chordgen, overchord}

\NormalTok{\#let chord = new{-}chordgen()}

\NormalTok{\#box(chord("x32010", name: "C"))}
\NormalTok{\#box(chord("x33222", name: "F\#m/C\#"))}
\NormalTok{\#box(chord("x,9,7,8,9,9"))}
\end{Highlighting}
\end{Shaded}

\pandocbounded{\includegraphics[keepaspectratio]{https://github.com/typst/packages/raw/main/packages/preview/conchord/0.2.0/examples/simple.png}}

\begin{quote}
\texttt{\ x\ } means closed string, \texttt{\ 0\ } is open, other number
is a fret. In case of frets larger than \texttt{\ 9\ } frets should be
separated with commas, otherwise you can list them without any
separators.
\end{quote}

\begin{quote}
Chord diagram works like a usual block, so to put them into one line you
need to wrap them into boxes. In real code it is recommended to create a
wrapper function to customize box margins etc (see larger example
below).
\end{quote}

It is easy to customize the colors and styles of chords with
\texttt{\ colors\ } argument and \texttt{\ show\ } rules for text. You
can also put \texttt{\ !\ } and \texttt{\ *\ } marks in the end of the
string to force diagram generation. \texttt{\ !\ } forces barre,
\texttt{\ *\ } removes it:

\begin{Shaded}
\begin{Highlighting}[]
\NormalTok{\#let custom{-}chord = new{-}chordgen(string{-}number: 3,}
\NormalTok{    colors: (shadow{-}barre: orange,}
\NormalTok{        grid: gray.darken(30\%),}
\NormalTok{        hold: red,}
\NormalTok{        barre: purple)}
\NormalTok{)}

\NormalTok{\#set text(fill: purple)}
\NormalTok{\#box(custom{-}chord("320", name: "C"))}
\NormalTok{\#box(custom{-}chord("2,4,4,*", name: "Bm"))}
\NormalTok{\#box(custom{-}chord("2,2,2, *"))}
\NormalTok{\#box(custom{-}chord("x,3,2, !"))}
\end{Highlighting}
\end{Shaded}

\pandocbounded{\includegraphics[keepaspectratio]{https://github.com/typst/packages/raw/main/packages/preview/conchord/0.2.0/examples/crazy.png}}

\begin{quote}
NOTE: be careful when using \textbf{!} , if barre cannot be used, it
will result into nonsense.
\end{quote}

For lyrics, you don’t need to add chord to word and specify the number
of char in words (unlike
\href{https://github.com/ljgago/typst-chords}{chordx} ). Simply add
\texttt{\ \#overchord\ } to the place you want a chord. Compose with
native Typst stylistic things for non-plain look (you don’t need to
dig into \href{https://github.com/ljgago/typst-chords}{chordx} ’s
custom arguments):

\begin{Shaded}
\begin{Highlighting}[]
\NormalTok{\#let och(it) = overchord(strong(it))}

\NormalTok{=== \#raw("[Verse 1]")}

\NormalTok{\#och[Em] Another head }
\NormalTok{\#och[C] hangs lowly \textbackslash{}}
\NormalTok{\#och[G] Child is slowly}
\NormalTok{\#och[D] taken}

\NormalTok{...}
\end{Highlighting}
\end{Shaded}

\begin{quote}
Complete example of lyrics with chords (see
\href{https://github.com/typst/packages/raw/main/packages/preview/conchord/0.2.0/examples/zombie.typ}{full
source} ):
\end{quote}

\pandocbounded{\includegraphics[keepaspectratio]{https://github.com/typst/packages/raw/main/packages/preview/conchord/0.2.0/examples/zombie.png}}

I was quite amazed with general idea of
\href{https://github.com/ljgago/typst-chords}{chordx} , but a bit
frustated with implementation, so I decided to quickly rewrite my old js
code to Typst. I use \texttt{\ cetz\ } there, so code is quite clean.

\begin{quote}
Note: This package doesn’t use any piece of
\href{https://github.com/ljgago/typst-chords}{chordx} , only the general
idea is taken.
\end{quote}

Brief comparison may be seen there, some concepts explained below:

\pandocbounded{\includegraphics[keepaspectratio]{https://github.com/typst/packages/raw/main/packages/preview/conchord/0.2.0/examples/compare.png}}

\subsection{Think about frets, not
layout}\label{think-about-frets-not-layout}

Write frets for chord as you hold it, like a string like “123456�
(see examples above). You don’t need to think about layouting and
subtracting frets, \texttt{\ conchord\ } does it for you.

\begin{quote}
NOTE: I can’t guarantee that will be the best chord layout. Moreover,
the logic is quite simple: e.g., barre can’t be multiple and can’t
be put anywhere except first bar in the image. However, surprisingly, it
works well in almost all of the common cases, so the exceptions are
really rare.
\end{quote}

If you need to create something too \emph{custom/complex} \st{(but not
\emph{concise} )} , maybe it is worth to try
\href{https://github.com/ljgago/typst-chords}{chordx} . You can also try
using core function \texttt{\ render-chord\ } for more manual~control,
but it is still limited by one barre starting from one (but that barre
may be shifted). If you think that feature should be supported, you can
create issue there.

\subsection{Shadow barre}\label{shadow-barre}

Some chord generators put barre only where it \emph{ought to} be (any
less will not hold some strings). Others put it where it can be
(sometimes maximal size, sometimes some other logic). I use simple barre
where it \textbf{ought to} be, and add \emph{shadow barre} where it
\textbf{could} maximally be. You can easily disable it by either setting
\texttt{\ use-shadow-barre:\ false\ } on \texttt{\ new-chordgen\ } (only
necessary part of barre rendered) or by setting color of
\texttt{\ shadow-barre\ } the same as \texttt{\ barre\ } (maximal
possible barre).

\subsection{Name auto-scaling}\label{name-auto-scaling}

Chord name font size is \emph{reduced} for \emph{large} chord names, so
the name fits well into chord diagram (see example above). That makes it
much more pretty to stack several chords together. To achieve
chordx-like behavior, you can always use
\texttt{\ \#figure(chord("…"),\ caption:\ …)\ } .

\subsection{Easier chords for lyrics}\label{easier-chords-for-lyrics}

Just add chord labels above lyrics in arbitrary place, don’t think
about what letter exactly it should be located. By default it aligns the
chord label to the left, so it produces pretty results out-of-box. You
can pass other alignments to \texttt{\ alignment\ } argument, or use the
chords straight inside words.

The command is \emph{much} simpler than chordx (of course, it is a
trade-off):

\begin{Shaded}
\begin{Highlighting}[]
\NormalTok{\#let overchord(body, align: start, height: 1em, width: {-}0.25em) = box(place(align, body), height: 1em + height, width: width)}
\end{Highlighting}
\end{Shaded}

Feel free to use it for your purposes outside of the package.

It takes on default \texttt{\ -0.25em\ } width to remove one adjacent
space, so

\begin{itemize}
\tightlist
\item
  To make it work on monospace/other special fonts, you will need to
  adjust \texttt{\ width\ } argument. The problem is that I can’t
  \texttt{\ measure\ } space, but maybe that will be eventually fixed.
\item
  To add chord inside word, you have to add \emph{one} space, like
  \texttt{\ wo\ \#chord{[}Am{]}rd\ } .
\end{itemize}

\subsection{Colors}\label{colors}

Customize the colors of chord elements. \texttt{\ new-chordgen\ }
accepts the \texttt{\ colors\ } dictionary with following possible
fields:

\begin{itemize}
\tightlist
\item
  \texttt{\ grid\ } : color of grid, default is
  \texttt{\ gray.darken(20\%)\ }
\item
  \texttt{\ open\ } : color of circles for open strings, default is
  \texttt{\ black\ }
\item
  \texttt{\ muted\ } : color of crosses for muted strings, default is
  \texttt{\ black\ }
\item
  \texttt{\ hold\ } : color of held positions, default is
  \texttt{\ \#5d6eaf\ }
\item
  \texttt{\ barre\ } : color of main barre part, default is
  \texttt{\ \#5d6eaf\ }
\item
  \texttt{\ shadow-barre\ } : color of “unnecessary� barre part,
  default is \texttt{\ \#5d6eaf.lighten(30\%)\ }
\end{itemize}

\subsubsection{Customizing text}\label{customizing-text}

\textbf{Important} : \emph{frets} are rendered using \texttt{\ raw\ }
elements. So if you want to customize their font or color, please use
\texttt{\ \#show\ raw:\ set\ text(fill:\ ...)\ } or similar things.

The chord’s name, on the other hand, uses default font, so to set it,
just use \texttt{\ \#set\ text(font:\ ...)\ } in the corresponding
scope.

\subsection{Assertions}\label{assertions}

Currently \href{https://github.com/ljgago/typst-chords}{chordx} has
almost no checks inside for correctness of passed chords.
\texttt{\ conchord\ } , on the other side, checks for

\begin{itemize}
\tightlist
\item
  Number of passed\&parsed frets equal to set string-number
\item
  Only numbers and \texttt{\ x\ } passed as frets
\item
  All frets fitting in the diagram
\end{itemize}

\begin{quote}
Everything there is highly experimental and unstable
\end{quote}

\pandocbounded{\includegraphics[keepaspectratio]{https://github.com/typst/packages/raw/main/packages/preview/conchord/0.2.0/examples/tabs.png}}

\begin{Shaded}
\begin{Highlighting}[]
\NormalTok{\#let chord = new{-}chordgen(scale{-}length: 0.6pt)}

\NormalTok{\#let ending(n) = \{}
\NormalTok{    rect(stroke: (left: black, top: black), inset: 0.2em, n + h(3em))}
\NormalTok{    v(0.5em)}
\NormalTok{\}}
\NormalTok{*This thing doesn\textquotesingle{}t follow musical notation rules, it is used just for demonstration purposes*:}

\NormalTok{\#tabs.new(\textasciigrave{}\textasciigrave{}\textasciigrave{}}
\NormalTok{2/4 2/4{-}3 2/4{-}2 2/4{-}3 |}
\NormalTok{2/4{-}2 2/4{-}3 2/4 2/4 2/4 |}
\NormalTok{2/4{-}2 p 0/2{-}3 3/2{-}2}
\NormalTok{|:}

\NormalTok{0/1+0/6 0/1 0/1{-}3 2/1 | 3/1+3/5{-}2 3/1 3/1{-}3 5/1 | 2/1+0/4{-}2 2/1 0/1{-}3 3/2{-}3 | \textbackslash{} \textbackslash{}}
\NormalTok{3/2{-}2 \textasciigrave{}5/2{-}3}
\NormalTok{p{-}2}
\NormalTok{\#\#}
\NormalTok{  chord("022000", name: "Em")}
\NormalTok{\#\#south}
\NormalTok{0/2{-}3 3/2 | | \#\# [...] \#\# p{-}4. | | 7/1{-}3 0/1{-}2 p{-}3 0/1 3/1 }

\NormalTok{\#\#}
\NormalTok{    ending[1.]}
\NormalTok{\#\#west}

\NormalTok{|}
\NormalTok{2/1{-}3}
\NormalTok{2/1}
\NormalTok{3/1 0/1 2/1{-}2 p{-}3 0/2{-}3 3/2{-}3}
\NormalTok{\#\#}
\NormalTok{  ending[2.]}
\NormalTok{\#\#west}
\NormalTok{|}
\NormalTok{2/1{-}2 2/1 0/1{-}3 3/2 :| 0/6{-}2 | \^{}0/6{-}2 || \textbackslash{}}
\NormalTok{1/1 2/1 2/2 2/2 2/3 2/3 4/4 4/4 4/4 4/4 4/4 4/4 2/3 2/3 2/3 2/3  2/3 2/3 2/3 2/3 2/3 2/3 2/3 2/3 2/3}
\NormalTok{\#\#}
\NormalTok{[notice there is no manual break]}
\NormalTok{\#\#east}
\NormalTok{| 2/3 2/3 8/3 7/3 6/3 5/3 4/3 2/3  5/3 8/3 9/3  7/3 2/3 | 2/3 2/2 2/3 2/4 |}
\NormalTok{10/1{-}3 10/1{-}3 10/1{-}3 10/1{-}4 10/1{-}4 10/1{-}4 10/1{-}4 10/1{-}5. 10/1{-}5. 10/1{-}5 10/1{-}5 10/1{-}2 \textbackslash{}}
\NormalTok{1/3bfullr+2/5{-}2 1/2b1/2{-}1 2/3v{-}1}
\NormalTok{\textasciigrave{}\textasciigrave{}\textasciigrave{}, eval{-}scope: (chord: chord, ending: ending)}
\NormalTok{ )}


\NormalTok{Not a lot customization is available yet, but something is already possible:}

\NormalTok{\#show raw: set text(red.darken(30\%), font: "Comic Sans MS")}

\NormalTok{\#tabs.new("0/1+2/5{-}1 \^{}0/1+\textasciigrave{}3/5{-}2.. 2/3 |: 2/3{-}1 2/3 2/3 | 3/3 ||",}
\NormalTok{  scale{-}length: 0.2cm,}
\NormalTok{  one{-}beat{-}length: 12,}
\NormalTok{  s{-}num: 5,}
\NormalTok{  colors: (}
\NormalTok{    lines: gradient.linear(yellow, blue),}
\NormalTok{    bars: green,}
\NormalTok{    connects: red}
\NormalTok{  ),}
\NormalTok{  enable{-}scale: false}
\NormalTok{)}
\end{Highlighting}
\end{Shaded}

As you can see from example, you can use raw strings or code blocks to
write tabs, there is no real difference.

The general idea is very simple: to write a number on some line, write
\texttt{\ \textless{}fret\ number\textgreater{}/\textless{}string\textgreater{}\ }
.

\textbf{Spaces are important!} All notes and special symbols work well
only if properly separated.

\subsubsection{Duration}\label{duration}

By default they will be quarter notes. To change that, you have to
specify the duration:
\texttt{\ \textless{}fret\textgreater{}/\textless{}string\textgreater{}-\textless{}duration\textgreater{}\ }
, where duration is \$log\_2\$ from note duration. So a whole note will
be \texttt{\ -0\ } , a half: \texttt{\ -1\ } and so on. You can also use
as many dots as you want to multiply duration by 1.5, e.g.
\texttt{\ -2.\ }

Once you change the duration, all the following notes will use it, so
you have to specify duration every time it is changed (basically,
always, but it really depends on composition). Of course, you can just
ignore all that duration staff.

\subsubsection{Bars and repetitions}\label{bars-and-repetitions}

To add simple bar, just add \texttt{\ \textbar{}\ } . To add double bar
line, use \texttt{\ \textbar{}\ \textbar{}\ } . To add end
movement/composition, add \texttt{\ \textbar{}\textbar{}\ } . To add
repetitions, use \texttt{\ \textbar{}:\ } and \texttt{\ :\textbar{}\ }
respectively.

\subsubsection{Linebreaks}\label{linebreaks}

Notes and bars that don’t fit in line will be automatically moved to
next. However, sometimes it isn’t ideal and may be a bit bugged, so it
is recommended to do that manually, using \texttt{\ \textbackslash{}\ }
.

The line is autoscaled if it is possible and not too ugly. You can
change the maximum and minimum scaling size with \texttt{\ scale-max\ }
and \texttt{\ scale-min\ } . It is also possible to completely disable
scaling with \texttt{\ enable-scale:\ false\ } .

\subsubsection{Ties and slides}\label{ties-and-slides}

You can \emph{tie} notes or \emph{slide} between them. To use ties, you
have to add \texttt{\ \^{}\ } in front of \emph{second} tied note, like
\texttt{\ 1/1\ \^{}3/1\ } . To use slides you have to do the same, but
with `.

\emph{Current limitation:} tying and sliding works only on the same
string and may work really bad if tied/slided through line break.

\subsection{Bends and vibratos}\label{bends-and-vibratos}

Add \texttt{\ b\ } after note, but before the duration (e.g.
\texttt{\ 2/3b-2\ } ) to add a bend. After \texttt{\ b\ } you can write
custom text to be written on top (for example, \texttt{\ b1/2\ } ). Add
\texttt{\ r\ } to the end to add a release.

Adding vibratos works the same way, via adding \texttt{\ v\ } to the
note. The length of vibrato will be the same as the length of the note.

Unfortunately, they are all supported things for now. But wait, there is
still one cool thing left!

\subsubsection{Custom content}\label{custom-content}

Add any typst code you want between \texttt{\ \#\#\ …\ \#\#\ } . It
will be rendered with \texttt{\ cetz\ } on top of the line where you
wrote it. That means you can write \emph{lyrics, chords, add complex
things like endings} , even \textbf{draw the elements that are still
missing} (well, it is still worth to create issue there, I will try to
do something).

That code is evaluated with \texttt{\ eval\ } , so you will need to pass
dictionary to \texttt{\ eval-scope\ } with all things you want to use.

You can set align of these elements by writing cetz anchors after the
second (e.g., \texttt{\ west\ } , \texttt{\ south\ } and their
combinations, like \texttt{\ west-south\ } ).

Additionally, if you enjoy drawing missing things, you can also use
\texttt{\ preamble\ } and \texttt{\ extra\ } arguments in
\texttt{\ tabs.new\ } where you can put any \texttt{\ cetz\ } inner
things (tabs uses canvas, and that allow you drawing on it) before or
after the tabs are drawn.

\subsubsection{Plans}\label{plans}

\begin{enumerate}
\tightlist
\item
  Add \emph{(optional)} “rhythm section� under tabs
\item
  Add more signs\&lines
\item
  Add more built-in things to attach above tabs
\end{enumerate}

It is far from what I want to do, so maybe there will be much more! I
will be very glad to receive \emph{any feedback} , both issues and PR-s
are very welcome (though I can’t promise I will be able to work on it
immediately)!

\subsubsection{How to add}\label{how-to-add}

Copy this into your project and use the import as \texttt{\ conchord\ }

\begin{verbatim}
#import "@preview/conchord:0.2.0"
\end{verbatim}

\includesvg[width=0.16667in,height=0.16667in]{/assets/icons/16-copy.svg}

Check the docs for
\href{https://typst.app/docs/reference/scripting/\#packages}{more
information on how to import packages} .

\subsubsection{About}\label{about}

\begin{description}
\tightlist
\item[Author :]
sitandr
\item[License:]
MIT
\item[Current version:]
0.2.0
\item[Last updated:]
February 6, 2024
\item[First released:]
July 24, 2023
\item[Minimum Typst version:]
0.8.0
\item[Archive size:]
12.8 kB
\href{https://packages.typst.org/preview/conchord-0.2.0.tar.gz}{\pandocbounded{\includesvg[keepaspectratio]{/assets/icons/16-download.svg}}}
\item[Repository:]
\href{https://github.com/sitandr/conchord}{GitHub}
\end{description}

\subsubsection{Where to report issues?}\label{where-to-report-issues}

This package is a project of sitandr . Report issues on
\href{https://github.com/sitandr/conchord}{their repository} . You can
also try to ask for help with this package on the
\href{https://forum.typst.app}{Forum} .

Please report this package to the Typst team using the
\href{https://typst.app/contact}{contact form} if you believe it is a
safety hazard or infringes upon your rights.

\phantomsection\label{versions}
\subsubsection{Version history}\label{version-history}

\begin{longtable}[]{@{}ll@{}}
\toprule\noalign{}
Version & Release Date \\
\midrule\noalign{}
\endhead
\bottomrule\noalign{}
\endlastfoot
0.2.0 & February 6, 2024 \\
\href{https://typst.app/universe/package/conchord/0.1.1/}{0.1.1} &
September 19, 2023 \\
\href{https://typst.app/universe/package/conchord/0.1.0/}{0.1.0} & July
24, 2023 \\
\end{longtable}

Typst GmbH did not create this package and cannot guarantee correct
functionality of this package or compatibility with any version of the
Typst compiler or app.


\title{typst.app/universe/package/ibanator}

\phantomsection\label{banner}
\section{ibanator}\label{ibanator}

{ 0.1.0 }

A package for validating and formatting International Bank Account
Numbers (IBANs) according to ISO 13616-1.

\phantomsection\label{readme}
\begin{quote}
Validate and format IBAN numbers according to ISO 13616-1.
\end{quote}

\subsection{Usage}\label{usage}

\begin{Shaded}
\begin{Highlighting}[]
\NormalTok{\#import "@preview/ibanator:0.1.0": iban}

\NormalTok{\#iban("DE89370400440532013000")}
\end{Highlighting}
\end{Shaded}

\includegraphics[width=3.64583in,height=\textheight,keepaspectratio]{https://github.com/typst/packages/raw/main/packages/preview/ibanator/0.1.0/docs/example.png}

To disable validation, set the \texttt{\ validate\ } flag to false:

\begin{Shaded}
\begin{Highlighting}[]
\NormalTok{\#iban("DE89370400440532013000", validate: false)}
\end{Highlighting}
\end{Shaded}

\subsubsection{How to add}\label{how-to-add}

Copy this into your project and use the import as \texttt{\ ibanator\ }

\begin{verbatim}
#import "@preview/ibanator:0.1.0"
\end{verbatim}

\includesvg[width=0.16667in,height=0.16667in]{/assets/icons/16-copy.svg}

Check the docs for
\href{https://typst.app/docs/reference/scripting/\#packages}{more
information on how to import packages} .

\subsubsection{About}\label{about}

\begin{description}
\tightlist
\item[Author :]
@mainrs
\item[License:]
EUPL-1.2
\item[Current version:]
0.1.0
\item[Last updated:]
April 8, 2024
\item[First released:]
April 8, 2024
\item[Archive size:]
19.0 kB
\href{https://packages.typst.org/preview/ibanator-0.1.0.tar.gz}{\pandocbounded{\includesvg[keepaspectratio]{/assets/icons/16-download.svg}}}
\item[Repository:]
\href{https://github.com/mainrs/typst-iban-formatter.git}{GitHub}
\item[Categor y :]
\begin{itemize}
\tightlist
\item[]
\item
  \pandocbounded{\includesvg[keepaspectratio]{/assets/icons/16-text.svg}}
  \href{https://typst.app/universe/search/?category=text}{Text}
\end{itemize}
\end{description}

\subsubsection{Where to report issues?}\label{where-to-report-issues}

This package is a project of @mainrs . Report issues on
\href{https://github.com/mainrs/typst-iban-formatter.git}{their
repository} . You can also try to ask for help with this package on the
\href{https://forum.typst.app}{Forum} .

Please report this package to the Typst team using the
\href{https://typst.app/contact}{contact form} if you believe it is a
safety hazard or infringes upon your rights.

\phantomsection\label{versions}
\subsubsection{Version history}\label{version-history}

\begin{longtable}[]{@{}ll@{}}
\toprule\noalign{}
Version & Release Date \\
\midrule\noalign{}
\endhead
\bottomrule\noalign{}
\endlastfoot
0.1.0 & April 8, 2024 \\
\end{longtable}

Typst GmbH did not create this package and cannot guarantee correct
functionality of this package or compatibility with any version of the
Typst compiler or app.


