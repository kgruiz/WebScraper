\title{typst.app/universe/package/minerva-report-fcfm}

\phantomsection\label{banner}
\phantomsection\label{template-thumbnail}
\pandocbounded{\includegraphics[keepaspectratio]{https://packages.typst.org/preview/thumbnails/minerva-report-fcfm-0.2.1-small.webp}}

\section{minerva-report-fcfm}\label{minerva-report-fcfm}

{ 0.2.1 }

Template de artículos, informes y tareas para la Facultad de Ciencias
Físicas y Matemáticas (FCFM).

\href{/app?template=minerva-report-fcfm&version=0.2.1}{Create project in
app}

\phantomsection\label{readme}
Template para hacer tareas, informes y trabajos, para estudiantes y
académicos de la Facultad de Ciencias Físicas y Matemáticas de la
Universidad de Chile que han usado templates similares para LaTeX.

\subsection{Guía Rápida}\label{guuxe3a-ruxe3pida}

\subsubsection{\texorpdfstring{\href{https://typst.app/}{Webapp}}{Webapp}}\label{webapp}

Si utilizas la webapp de Typst puedes presionar “Start from
template� y buscar “minerva-report-fcfm� para crear un nuevo
proyecto con este template.

\subsubsection{Typst CLI}\label{typst-cli}

Teniendo el CLI con la versión 0.11.0 o mayor, puedes realizar:

\begin{Shaded}
\begin{Highlighting}[]
\ExtensionTok{typst}\NormalTok{ init @preview/minerva{-}report{-}fcfm:0.2.1}
\end{Highlighting}
\end{Shaded}

Esto va a descargar el template en la cache de typst y luego va a
iniciar el proyecto en la carpeta actual.

\subsection{Configuración}\label{configuraciuxe3uxb3n}

La mayoría de la configuración se realiza a través del archivo
\texttt{\ meta.typ\ } , allí podrás elegir un título, indicar los
autores, el equipo docente, entre otras configuraciones.

El campo \texttt{\ autores\ } solo puede ser \texttt{\ string\ } o un
\texttt{\ array\ } de strings.

La configuración \texttt{\ departamento\ } puede ser personalizada a
cualquier organización pasandole un diccionario de esta forma:

\begin{Shaded}
\begin{Highlighting}[]
\NormalTok{\#let departamento = (}
\NormalTok{  nombre: (}
\NormalTok{    "Universidad Técnica Federico Santa María",}
\NormalTok{  )}
\NormalTok{)}
\end{Highlighting}
\end{Shaded}

Las demás configuraciones pueden ser un \texttt{\ content\ }
arbitrario, o un \texttt{\ string\ } .

\subsubsection{Configuración
Avanzada}\label{configuraciuxe3uxb3n-avanzada}

Algunos aspectos más avanzados pueden ser configurados a través de la
show rule que inicializa el documento
\texttt{\ \#show:\ minerva.report.with(\ ...\ )\ } , los parámetros
opcionales que recibe la función \texttt{\ report\ } son los
siguientes:

\begin{longtable}[]{@{}lll@{}}
\toprule\noalign{}
nombre & tipo & descrición \\
\midrule\noalign{}
\endhead
\bottomrule\noalign{}
\endlastfoot
portada & (meta) =\textgreater{} content & Una función que recibe el
diccionario \texttt{\ meta.typ\ } y retorna una página. \\
header & (meta) =\textgreater{} content & Header a aplicarse a cada
página. \\
footer & (meta) =\textgreater{} content & Footer a aplicarse a cada
página. \\
showrules & bool & El template aplica ciertas show-rules para que sea
más fácil de utilizar. Si quires más personalización, es probable
que necesites desactivarlas y luego solo utilizar las que necesites. \\
\end{longtable}

\paragraph{Show Rules}\label{show-rules}

El template incluye show rules que pueden ser incluidas opcionalmente.
Todas estas show rules pueden ser activadas agregando:

\begin{Shaded}
\begin{Highlighting}[]
\NormalTok{\#show: minerva.\textless{}nombre{-}función\textgreater{}}
\end{Highlighting}
\end{Shaded}

Justo después de la línea
\texttt{\ \#show\ minerva.report.with(\ ...\ )\ } reemplazando
\texttt{\ \textless{}nombre-función\textgreater{}\ } por el nombre de
la show rule a aplicar.

\subparagraph{primer-heading-en-nueva-pag (activada por
defecto)}\label{primer-heading-en-nueva-pag-activada-por-defecto}

Esta show rule hace que el primer heading que tenga
\texttt{\ outlined:\ true\ } se muestre en una nueva página (con
\texttt{\ weak:\ true\ } ). Notar que al ser \texttt{\ weak:\ true\ } si
la página ya de por si estaba vacía, no se crea otra página adicional,
pero para que la página realmente se considere vacía no debe contener
absolutamente nada, incluso tener elementos invisibles va a causar que
se agregue una página extra.

\subparagraph{operadores-es (activada por
defecto)}\label{operadores-es-activada-por-defecto}

Cambia los operadores matemáticos que define Typst por defecto a sus
contrapartes en español, esto es, cambia \texttt{\ lim\ } por
\texttt{\ lím\ } , \texttt{\ inf\ } por \texttt{\ ínf\ } y así con
todos.

\subparagraph{formato-numeros-es}\label{formato-numeros-es}

Cambia los números dentro de las ecuaciones para que usen coma decimal
en vez de punto decimal, como es convención en el español. Esta show
rule no viene activa por defecto.

\href{/app?template=minerva-report-fcfm&version=0.2.1}{Create project in
app}

\subsubsection{How to use}\label{how-to-use}

Click the button above to create a new project using this template in
the Typst app.

You can also use the Typst CLI to start a new project on your computer
using this command:

\begin{verbatim}
typst init @preview/minerva-report-fcfm:0.2.1
\end{verbatim}

\includesvg[width=0.16667in,height=0.16667in]{/assets/icons/16-copy.svg}

\subsubsection{About}\label{about}

\begin{description}
\tightlist
\item[Author :]
\href{https://github.com/Dav1com}{David Ibáñez}
\item[License:]
MIT-0
\item[Current version:]
0.2.1
\item[Last updated:]
October 14, 2024
\item[First released:]
April 15, 2024
\item[Minimum Typst version:]
0.11.0
\item[Archive size:]
246 kB
\href{https://packages.typst.org/preview/minerva-report-fcfm-0.2.1.tar.gz}{\pandocbounded{\includesvg[keepaspectratio]{/assets/icons/16-download.svg}}}
\item[Repository:]
\href{https://github.com/Dav1com/minerva-report-fcfm}{GitHub}
\item[Categor y :]
\begin{itemize}
\tightlist
\item[]
\item
  \pandocbounded{\includesvg[keepaspectratio]{/assets/icons/16-speak.svg}}
  \href{https://typst.app/universe/search/?category=report}{Report}
\end{itemize}
\end{description}

\subsubsection{Where to report issues?}\label{where-to-report-issues}

This template is a project of David Ibáñez . Report issues on
\href{https://github.com/Dav1com/minerva-report-fcfm}{their repository}
. You can also try to ask for help with this template on the
\href{https://forum.typst.app}{Forum} .

Please report this template to the Typst team using the
\href{https://typst.app/contact}{contact form} if you believe it is a
safety hazard or infringes upon your rights.

\phantomsection\label{versions}
\subsubsection{Version history}\label{version-history}

\begin{longtable}[]{@{}ll@{}}
\toprule\noalign{}
Version & Release Date \\
\midrule\noalign{}
\endhead
\bottomrule\noalign{}
\endlastfoot
0.2.1 & October 14, 2024 \\
\href{https://typst.app/universe/package/minerva-report-fcfm/0.2.0/}{0.2.0}
& April 29, 2024 \\
\href{https://typst.app/universe/package/minerva-report-fcfm/0.1.0/}{0.1.0}
& April 15, 2024 \\
\end{longtable}

Typst GmbH did not create this template and cannot guarantee correct
functionality of this template or compatibility with any version of the
Typst compiler or app.


\title{typst.app/universe/package/silky-slides-insa}

\phantomsection\label{banner}
\phantomsection\label{template-thumbnail}
\pandocbounded{\includegraphics[keepaspectratio]{https://packages.typst.org/preview/thumbnails/silky-slides-insa-0.1.1-small.webp}}

\section{silky-slides-insa}\label{silky-slides-insa}

{ 0.1.1 }

A template made for presentations of INSA, a French engineering school.

\href{/app?template=silky-slides-insa&version=0.1.1}{Create project in
app}

\phantomsection\label{readme}
\pandocbounded{\includegraphics[keepaspectratio]{https://github.com/typst/packages/raw/main/packages/preview/silky-slides-insa/0.1.1/thumbnail-full.png}}

Typst Template for presentation for the french engineering school INSA.

\subsection{Table of contents}\label{table-of-contents}

\begin{enumerate}
\tightlist
\item
  \href{https://github.com/typst/packages/raw/main/packages/preview/silky-slides-insa/0.1.1/\#examples}{Example}
\item
  \href{https://github.com/typst/packages/raw/main/packages/preview/silky-slides-insa/0.1.1/\#usage}{Usage}
\item
  \href{https://github.com/typst/packages/raw/main/packages/preview/silky-slides-insa/0.1.1/\#fonts}{Fonts
  information}
\item
  \href{https://github.com/typst/packages/raw/main/packages/preview/silky-slides-insa/0.1.1/\#notes}{Notes}
\item
  \href{https://github.com/typst/packages/raw/main/packages/preview/silky-slides-insa/0.1.1/\#license}{License}
\item
  \href{https://github.com/typst/packages/raw/main/packages/preview/silky-slides-insa/0.1.1/\#changelog}{Changelog}
\end{enumerate}

\subsection{Example}\label{example}

\begin{Shaded}
\begin{Highlighting}[]
\NormalTok{\#import "@preview/silky{-}slides{-}insa:0.1.1": *}
\NormalTok{\#show: insa{-}slides.with(}
\NormalTok{  title: "Titre du diaporama",}
\NormalTok{  title{-}visual: none,}
\NormalTok{  subtitle: "Sous{-}titre (noms et prénoms ?)",}
\NormalTok{  insa: "rennes"}
\NormalTok{)}

\NormalTok{= Titre de section}

\NormalTok{== Titre d\textquotesingle{}une slide}

\NormalTok{{-} Liste}
\NormalTok{  {-} dans}
\NormalTok{    {-} une liste}

\NormalTok{On peut aussi faire un \#text(fill: insa{-}colors.secondary)[texte] avec les \#text(fill: insa{-}colors.primary)[couleurs de l\textquotesingle{}INSA] !}

\NormalTok{== Une autre slide}

\NormalTok{Du texte}

\NormalTok{\#pause}

\NormalTok{Et un autre texte qui apparaît plus tard !}

\NormalTok{\#section{-}slide[Une autre section][Avec une petite description]}

\NormalTok{Coucou}
\end{Highlighting}
\end{Shaded}

\subsection{Usage}\label{usage}

\subsubsection{Slide show rule}\label{slide-show-rule}

You call it with \texttt{\ \#show:\ insa-slides.with(..parameters)\ } .

\begin{longtable}[]{@{}llll@{}}
\toprule\noalign{}
Parameter & Description & Type & Example \\
\midrule\noalign{}
\endhead
\bottomrule\noalign{}
\endlastfoot
\textbf{title} & Title of the presentation & content &
\texttt{\ {[}Titre\ de\ la\ prez{]}\ } \\
\textbf{title-visual} & Content shown at the right of the title slide &
content & none \\
\textbf{subtitle} & Subtitle of the presentation & content &
\texttt{\ {[}Sous-titre{]}\ } \\
\textbf{insa} & INSA name ( \texttt{\ rennes\ } , \texttt{\ hdf\ } …)
& str & \texttt{\ "rennes"\ } \\
\end{longtable}

If you assign a content to \texttt{\ title-visual\ } , the title slide
will automatically switch layout to the “visual� one from the
graphic charter. If you do not assign a visual content, the title slide
will only contain the title and subtitle and will choose the simple
layout.

\subsubsection{Section slide}\label{section-slide}

A section slide is automatically created when you put a level-1 header
in your markup. For example:

\begin{Shaded}
\begin{Highlighting}[]
\NormalTok{= Slide section}
\NormalTok{Blablabla}
\end{Highlighting}
\end{Shaded}

Will create a section slide with the title “Slide section� and will
be followed by a content slide containing “Blablabla�.

If you want to put a subtitle in your section slide, you must
explicitely use the \texttt{\ section-slide\ } function like so:

\begin{Shaded}
\begin{Highlighting}[]
\NormalTok{\#section{-}slide[Titre de section][Description de section]}
\end{Highlighting}
\end{Shaded}

\subsection{Fonts}\label{fonts}

The graphic charter recommends the fonts \textbf{League Spartan} for
headings and \textbf{Source Serif} for regular text. To have the best
look, you should install those fonts.

\begin{quote}
You can download the fonts from
\href{https://github.com/SkytAsul/INSA-Typst-Template/tree/main/fonts}{here}
.
\end{quote}

To behave correctly on computers lacking those specific fonts, this
template will automatically fallback to similar ones:

\begin{itemize}
\tightlist
\item
  \textbf{League Spartan} -\textgreater{} \textbf{Arial} (approved by
  INSA’s graphic charter, by default in Windows) -\textgreater{}
  \textbf{Liberation Sans} (by default in most Linux)
\item
  \textbf{Source Serif} -\textgreater{} \textbf{Source Serif 4}
  (downloadable for free) -\textgreater{} \textbf{Georgia} (approved by
  the graphic charter) -\textgreater{} \textbf{Linux Libertine} (default
  Typst font)
\end{itemize}

\subsubsection{Note on variable fonts}\label{note-on-variable-fonts}

If you want to install those fonts on your computer, Typst might not
recognize them if you install their \emph{Variable} versions. You should
install the static versions ( \textbf{League Spartan Bold} and most
versions of \textbf{Source Serif} ).

Keep an eye on \href{https://github.com/typst/typst/issues/185}{the
issue in Typst bug tracker} to see when variable fonts will be used!

\subsection{Notes}\label{notes}

This template is being developed by Youenn LE JEUNE from the INSA de
Rennes in \href{https://github.com/SkytAsul/INSA-Typst-Template}{this
repository} .

For now it includes assets from the graphic charters of those INSAs:

\begin{itemize}
\tightlist
\item
  Rennes ( \texttt{\ rennes\ } )
\item
  Hauts de France ( \texttt{\ hdf\ } )
\item
  Centre Val de Loire ( \texttt{\ cvl\ } ) Users from other INSAs can
  open a pull request on the repository with the assets for their INSA.
\end{itemize}

If you have any other feature request, open an issue on the repository.

\subsection{License}\label{license}

The typst template is licensed under the
\href{https://github.com/SkytAsul/INSA-Typst-Template/blob/main/LICENSE}{MIT
license} . This does \emph{not} apply to the image assets. Those image
files are property of Groupe INSA.

\subsection{Changelog}\label{changelog}

\subsubsection{0.1.1}\label{section}

\begin{itemize}
\tightlist
\item
  Added INSA CVL assets
\end{itemize}

\subsubsection{0.1.0}\label{section-1}

\begin{itemize}
\tightlist
\item
  Created the template
\end{itemize}

\href{/app?template=silky-slides-insa&version=0.1.1}{Create project in
app}

\subsubsection{How to use}\label{how-to-use}

Click the button above to create a new project using this template in
the Typst app.

You can also use the Typst CLI to start a new project on your computer
using this command:

\begin{verbatim}
typst init @preview/silky-slides-insa:0.1.1
\end{verbatim}

\includesvg[width=0.16667in,height=0.16667in]{/assets/icons/16-copy.svg}

\subsubsection{About}\label{about}

\begin{description}
\tightlist
\item[Author :]
SkytAsul
\item[License:]
MIT
\item[Current version:]
0.1.1
\item[Last updated:]
November 21, 2024
\item[First released:]
October 16, 2024
\item[Archive size:]
227 kB
\href{https://packages.typst.org/preview/silky-slides-insa-0.1.1.tar.gz}{\pandocbounded{\includesvg[keepaspectratio]{/assets/icons/16-download.svg}}}
\item[Repository:]
\href{https://github.com/SkytAsul/INSA-Typst-Template}{GitHub}
\item[Discipline s :]
\begin{itemize}
\tightlist
\item[]
\item
  \href{https://typst.app/universe/search/?discipline=engineering}{Engineering}
\item
  \href{https://typst.app/universe/search/?discipline=computer-science}{Computer
  Science}
\item
  \href{https://typst.app/universe/search/?discipline=mathematics}{Mathematics}
\item
  \href{https://typst.app/universe/search/?discipline=physics}{Physics}
\item
  \href{https://typst.app/universe/search/?discipline=education}{Education}
\end{itemize}
\item[Categor y :]
\begin{itemize}
\tightlist
\item[]
\item
  \pandocbounded{\includesvg[keepaspectratio]{/assets/icons/16-presentation.svg}}
  \href{https://typst.app/universe/search/?category=presentation}{Presentation}
\end{itemize}
\end{description}

\subsubsection{Where to report issues?}\label{where-to-report-issues}

This template is a project of SkytAsul . Report issues on
\href{https://github.com/SkytAsul/INSA-Typst-Template}{their repository}
. You can also try to ask for help with this template on the
\href{https://forum.typst.app}{Forum} .

Please report this template to the Typst team using the
\href{https://typst.app/contact}{contact form} if you believe it is a
safety hazard or infringes upon your rights.

\phantomsection\label{versions}
\subsubsection{Version history}\label{version-history}

\begin{longtable}[]{@{}ll@{}}
\toprule\noalign{}
Version & Release Date \\
\midrule\noalign{}
\endhead
\bottomrule\noalign{}
\endlastfoot
0.1.1 & November 21, 2024 \\
\href{https://typst.app/universe/package/silky-slides-insa/0.1.0/}{0.1.0}
& October 16, 2024 \\
\end{longtable}

Typst GmbH did not create this template and cannot guarantee correct
functionality of this template or compatibility with any version of the
Typst compiler or app.


\title{typst.app/universe/package/academic-conf-pre}

\phantomsection\label{banner}
\phantomsection\label{template-thumbnail}
\pandocbounded{\includegraphics[keepaspectratio]{https://packages.typst.org/preview/thumbnails/academic-conf-pre-0.1.0-small.webp}}

\section{academic-conf-pre}\label{academic-conf-pre}

{ 0.1.0 }

Slide Theme for Acadmic Presentations in Australia

\href{/app?template=academic-conf-pre&version=0.1.0}{Create project in
app}

\phantomsection\label{readme}
\subsection{1. Introduction}\label{introduction}

This is a template for \textbf{academic conference presentations} . It
is designed for the use in Typst which is simplier and more
user-friendly than LaTeX.

\subsection{2. How to use this template}\label{how-to-use-this-template}

themes documents are under \textbf{/themes/}

examples documents are under \textbf{/examples/}

if you don’t want to redesign the template, just follow the typ files
under the examples.

\subsection{3. How it looks}\label{how-it-looks}

The colors are entirely controllable, and I have provided three
relatively comfortable color schemes with \textbf{green} , \textbf{blue}
, and \textbf{red} as the base tones, which can be easily adjusted.

It is notable, there is a logo displayed in the center of the
presentation, and its \textbf{appearance} , \textbf{transparency} , and
\textbf{position} can be fully adjusted. Here, I use the University of
Sydney’s logo as an example.

You could see the PDFs under \textbf{examples/xxx.pdf}

\href{/app?template=academic-conf-pre&version=0.1.0}{Create project in
app}

\subsubsection{How to use}\label{how-to-use}

Click the button above to create a new project using this template in
the Typst app.

You can also use the Typst CLI to start a new project on your computer
using this command:

\begin{verbatim}
typst init @preview/academic-conf-pre:0.1.0
\end{verbatim}

\includesvg[width=0.16667in,height=0.16667in]{/assets/icons/16-copy.svg}

\subsubsection{About}\label{about}

\begin{description}
\tightlist
\item[Author :]
\href{mailto:isjun.liu@gmail.com}{JL-ghcoder}
\item[License:]
MIT
\item[Current version:]
0.1.0
\item[Last updated:]
November 5, 2024
\item[First released:]
November 5, 2024
\item[Archive size:]
1.29 MB
\href{https://packages.typst.org/preview/academic-conf-pre-0.1.0.tar.gz}{\pandocbounded{\includesvg[keepaspectratio]{/assets/icons/16-download.svg}}}
\item[Repository:]
\href{https://github.com/JL-ghcoder/Typst-Pre-Template}{GitHub}
\item[Categor y :]
\begin{itemize}
\tightlist
\item[]
\item
  \pandocbounded{\includesvg[keepaspectratio]{/assets/icons/16-presentation.svg}}
  \href{https://typst.app/universe/search/?category=presentation}{Presentation}
\end{itemize}
\end{description}

\subsubsection{Where to report issues?}\label{where-to-report-issues}

This template is a project of JL-ghcoder . Report issues on
\href{https://github.com/JL-ghcoder/Typst-Pre-Template}{their
repository} . You can also try to ask for help with this template on the
\href{https://forum.typst.app}{Forum} .

Please report this template to the Typst team using the
\href{https://typst.app/contact}{contact form} if you believe it is a
safety hazard or infringes upon your rights.

\phantomsection\label{versions}
\subsubsection{Version history}\label{version-history}

\begin{longtable}[]{@{}ll@{}}
\toprule\noalign{}
Version & Release Date \\
\midrule\noalign{}
\endhead
\bottomrule\noalign{}
\endlastfoot
0.1.0 & November 5, 2024 \\
\end{longtable}

Typst GmbH did not create this template and cannot guarantee correct
functionality of this template or compatibility with any version of the
Typst compiler or app.


\title{typst.app/universe/package/untypsignia}

\phantomsection\label{banner}
\section{untypsignia}\label{untypsignia}

{ 0.1.1 }

Unofficial typesetter\textquotesingle s insignia emulations

\phantomsection\label{readme}
The \texttt{\ untypsignia\ } is a 3rd-party, unofficial, unendorsed
Typst package that exposes commands for rendering, as
\texttt{\ content\ } texts, some typesetters names in a stylized
fashion, emulating their respective \emph{insignia} , i.e., marks by
which they are known.

\subsection{Name}\label{name}

The package name is a blend of:

\begin{itemize}
\tightlist
\item
  “un�, from “unofficial�,
\item
  “typ�, from “Typst�, and
\item
  “signia�, from “insignia�, which means marks by which anything
  is known.
\end{itemize}

\subsection{Description}\label{description}

The typical use case of \texttt{\ untypsignia\ } in Typst is to emulate
a given typesetting system’s mark, if available, when referring to
them, in sentences like: “This document is typeset in \texttt{\ XYZ\ }
�, as traditionally done in \texttt{\ TeX\ } systems and derivatives
thereof.

Currently available insignia emulations include:

\begin{itemize}
\tightlist
\item
  \texttt{\ TeX\ } ,
\item
  \texttt{\ LaTeX\ } , and
\item
  \texttt{\ Typst\ } (see below)
\end{itemize}

Despite there’s no such a thing as a Typst “official� typography,
according to this post on
\href{https://discord.com/channels/1054443721975922748/1054443722592497796/1107039477714665522}{Discord}
, it can be typeset with “whatever font� the surrounding text is
being typeset. Moreover, Typst
\href{https://typst.app/legal/brand/}{branding page} requires
capitalization of the initial “T� whenever the name is used in
prose. Therefore, the “Typst� support in this package is a mere,
still unofficial, implementation of the capitalization of “Typst� in
the currently used font.

\subsection{Font Requirements}\label{font-requirements}

For the \texttt{\ TeX\ } system and it’s derivatives, the
\texttt{\ "New\ Computer\ Modern"\ } font is required.

\subsection{Usage}\label{usage}

The package exposes the following few, parameterless, functions:

\begin{itemize}
\tightlist
\item
  \texttt{\ \#texmark()\ } ,
\item
  \texttt{\ \#latexmark()\ } , and
\item
  \texttt{\ \#typstmark()\ } .
\end{itemize}

Except for the \texttt{\ \#typstmark()\ } , each such command outputs
their respective namesake signus emulation, in the document’s current
\texttt{\ text\ } settings, with the exception of font â€'' meaning text
size, color, etc… will apply to the signus emulation.

Aditionally, the signus emulation is produced, as \texttt{\ contexts\ }
text inside a \texttt{\ box\ } â€'' hence not images â€'' so as to avoid
hyphenation to take place. This also applies to the
\texttt{\ \#typstmark()\ } function, for lack of specific guidance, and
also because “Typst� is a short word.

\subsection{Example}\label{example}

\begin{Shaded}
\begin{Highlighting}[]
\NormalTok{\#set page(width: auto, height: auto, margin: 12pt, fill: rgb("19181f"))}
\NormalTok{\#set par(leading: 1.5em)}
\NormalTok{\#set text(font: "Rouge Script", fill: rgb("80f4b6"))}

\NormalTok{\#import "@preview/untypsignia:0.1.1": *}

\NormalTok{\#let say() = [I prefer \#typstmark() over \#texmark() or \#latexmark().]}

\NormalTok{\#for sz in (20, 16, 14, 12, 10, 8) \{}
\NormalTok{  set text(size: sz * 1pt)}
\NormalTok{  say()}
\NormalTok{  linebreak()}
\NormalTok{\}}
\end{Highlighting}
\end{Shaded}

This example results in a 1-page document like this one:

\pandocbounded{\includegraphics[keepaspectratio]{https://raw.githubusercontent.com/cnaak/untypsignia.typ/86b221379931edcbfa91b50159a4ff930ecbec47/thumbnail.png}}

\subsection{Citing}\label{citing}

This package can be cited with the following bibliography database
entry:

\begin{Shaded}
\begin{Highlighting}[]
\FunctionTok{untypsignia{-}package}\KeywordTok{:}
\AttributeTok{  }\FunctionTok{type}\KeywordTok{:}\AttributeTok{ Web}
\AttributeTok{  }\FunctionTok{author}\KeywordTok{:}\AttributeTok{ Naaktgeboren, C.}
\AttributeTok{  }\FunctionTok{title}\KeywordTok{:}
\AttributeTok{    }\FunctionTok{value}\KeywordTok{:}\AttributeTok{ }\StringTok{"untypsignia: unofficial typesetter\textquotesingle{}s insignia emulations"}
\AttributeTok{  }\FunctionTok{url}\KeywordTok{:}\AttributeTok{ https://github.com/cnaak/untypsignia.typ}
\AttributeTok{  }\FunctionTok{version}\KeywordTok{:}\AttributeTok{ }\FloatTok{0.1.1}
\AttributeTok{  }\FunctionTok{date}\KeywordTok{:}\AttributeTok{ 2024{-}08}
\end{Highlighting}
\end{Shaded}

\subsubsection{How to add}\label{how-to-add}

Copy this into your project and use the import as
\texttt{\ untypsignia\ }

\begin{verbatim}
#import "@preview/untypsignia:0.1.1"
\end{verbatim}

\includesvg[width=0.16667in,height=0.16667in]{/assets/icons/16-copy.svg}

Check the docs for
\href{https://typst.app/docs/reference/scripting/\#packages}{more
information on how to import packages} .

\subsubsection{About}\label{about}

\begin{description}
\tightlist
\item[Author :]
Naaktgeboren, C.
\item[License:]
MIT
\item[Current version:]
0.1.1
\item[Last updated:]
August 21, 2024
\item[First released:]
August 14, 2024
\item[Minimum Typst version:]
0.11.1
\item[Archive size:]
2.13 kB
\href{https://packages.typst.org/preview/untypsignia-0.1.1.tar.gz}{\pandocbounded{\includesvg[keepaspectratio]{/assets/icons/16-download.svg}}}
\item[Discipline :]
\begin{itemize}
\tightlist
\item[]
\item
  \href{https://typst.app/universe/search/?discipline=computer-science}{Computer
  Science}
\end{itemize}
\item[Categor ies :]
\begin{itemize}
\tightlist
\item[]
\item
  \pandocbounded{\includesvg[keepaspectratio]{/assets/icons/16-chart.svg}}
  \href{https://typst.app/universe/search/?category=visualization}{Visualization}
\item
  \pandocbounded{\includesvg[keepaspectratio]{/assets/icons/16-hammer.svg}}
  \href{https://typst.app/universe/search/?category=utility}{Utility}
\item
  \pandocbounded{\includesvg[keepaspectratio]{/assets/icons/16-smile.svg}}
  \href{https://typst.app/universe/search/?category=fun}{Fun}
\end{itemize}
\end{description}

\subsubsection{Where to report issues?}\label{where-to-report-issues}

This package is a project of Naaktgeboren, C. . You can also try to ask
for help with this package on the \href{https://forum.typst.app}{Forum}
.

Please report this package to the Typst team using the
\href{https://typst.app/contact}{contact form} if you believe it is a
safety hazard or infringes upon your rights.

\phantomsection\label{versions}
\subsubsection{Version history}\label{version-history}

\begin{longtable}[]{@{}ll@{}}
\toprule\noalign{}
Version & Release Date \\
\midrule\noalign{}
\endhead
\bottomrule\noalign{}
\endlastfoot
0.1.1 & August 21, 2024 \\
\href{https://typst.app/universe/package/untypsignia/0.1.0/}{0.1.0} &
August 14, 2024 \\
\end{longtable}

Typst GmbH did not create this package and cannot guarantee correct
functionality of this package or compatibility with any version of the
Typst compiler or app.


\title{typst.app/universe/package/roremu}

\phantomsection\label{banner}
\section{roremu}\label{roremu}

{ 0.1.0 }

æ---¥æœ¬èªžã?®ãƒ€ãƒŸãƒ¼ãƒ†ã‚­ã‚¹ãƒˆç''Ÿæˆ?(Lorem Ipsum)

\phantomsection\label{readme}
æ---¥æœ¬èªžã?®ãƒ€ãƒŸãƒ¼ãƒ†ã‚­ã‚¹ãƒˆï¼ˆLipsum)ç''Ÿæˆ?ツール。

Blind text (Lorem ipsum) generator for Japanese.

\subsection{ç''¨æ³• / Usage}\label{uxe7uxe6uxb3-usage}

\begin{Shaded}
\begin{Highlighting}[]
\NormalTok{\#import "@preview/roremu:0.1.0": roremu}

\NormalTok{\#roremu(8) \# 吾輩は猫である。}

\NormalTok{\#roremu(8, offset: 8) \#名前はまだ無い。}

\NormalTok{\#roremu(17, custom{-}text: "私はその人を常に先生と呼んでいた。")}
\end{Highlighting}
\end{Shaded}

\subsection{テキスト / Text
Source}\label{uxe3ux192uxe3uxe3uxb9uxe3ux192ux2c6-text-source}

�目漱石『
\href{https://ja.wikipedia.org/wiki/\%E5\%90\%BE\%E8\%BC\%A9\%E3\%81\%AF\%E7\%8C\%AB\%E3\%81\%A7\%E3\%81\%82\%E3\%82\%8B}{�輩�猫��る}
�(
\href{https://www.aozora.gr.jp/cards/000148/card789.html}{é?'空æ--‡åº«ç‰ˆ}
より抜粋ã€?ルãƒ``抜ã??)

\subsection{å??称ç''±æ?¥ / Why
“roremu�?}\label{uxe5uxe7uxe7uxe6-why-uxe2ux153roremuuxe2}

lorem「ロレãƒ~ã€?ã?®ãƒ­ãƒ¼ãƒžå­---表記。

“roremuâ€? is the romanization of ロレãƒ~ (lorem).

\subsection{ライセンス /
License}\label{uxe3ux192uxe3uxe3uxe3ux192uxb3uxe3uxb9-license}

Unlicense

\subsubsection{How to add}\label{how-to-add}

Copy this into your project and use the import as \texttt{\ roremu\ }

\begin{verbatim}
#import "@preview/roremu:0.1.0"
\end{verbatim}

\includesvg[width=0.16667in,height=0.16667in]{/assets/icons/16-copy.svg}

Check the docs for
\href{https://typst.app/docs/reference/scripting/\#packages}{more
information on how to import packages} .

\subsubsection{About}\label{about}

\begin{description}
\tightlist
\item[Author :]
mkpoli
\item[License:]
Unlicense
\item[Current version:]
0.1.0
\item[Last updated:]
January 23, 2024
\item[First released:]
January 23, 2024
\item[Archive size:]
7.83 kB
\href{https://packages.typst.org/preview/roremu-0.1.0.tar.gz}{\pandocbounded{\includesvg[keepaspectratio]{/assets/icons/16-download.svg}}}
\item[Repository:]
\href{https://github.com/mkpoli/roremu}{GitHub}
\end{description}

\subsubsection{Where to report issues?}\label{where-to-report-issues}

This package is a project of mkpoli . Report issues on
\href{https://github.com/mkpoli/roremu}{their repository} . You can also
try to ask for help with this package on the
\href{https://forum.typst.app}{Forum} .

Please report this package to the Typst team using the
\href{https://typst.app/contact}{contact form} if you believe it is a
safety hazard or infringes upon your rights.

\phantomsection\label{versions}
\subsubsection{Version history}\label{version-history}

\begin{longtable}[]{@{}ll@{}}
\toprule\noalign{}
Version & Release Date \\
\midrule\noalign{}
\endhead
\bottomrule\noalign{}
\endlastfoot
0.1.0 & January 23, 2024 \\
\end{longtable}

Typst GmbH did not create this package and cannot guarantee correct
functionality of this package or compatibility with any version of the
Typst compiler or app.


\title{typst.app/universe/package/yagenda}

\phantomsection\label{banner}
\phantomsection\label{template-thumbnail}
\pandocbounded{\includegraphics[keepaspectratio]{https://packages.typst.org/preview/thumbnails/yagenda-0.1.0-small.webp}}

\section{yagenda}\label{yagenda}

{ 0.1.0 }

A tabular template for meeting agendas with agenda items defined in
Yaml.

\href{/app?template=yagenda&version=0.1.0}{Create project in app}

\phantomsection\label{readme}
A Typst template for meeting agendas using Yaml for agenda items. To get
started:

\begin{Shaded}
\begin{Highlighting}[]
\NormalTok{typst init @preview/yagenda:0.1.0}
\end{Highlighting}
\end{Shaded}

And edit the \texttt{\ main.typ\ } example. The data are drawn from
\texttt{\ agenda.yaml\ } .

\pandocbounded{\includegraphics[keepaspectratio]{https://github.com/typst/packages/raw/main/packages/preview/yagenda/0.1.0/thumbnail.png}}

\subsection{Contributing}\label{contributing}

PRs are welcome! And if you encounter any bugs or have any
requests/ideas, feel free to open an issue.

\subsection{Acknowledgements}\label{acknowledgements}

The Typst grid layout was designed by
\href{https://discord.com/channels/1054443721975922748/1219401775908655115}{PgSuper
on Discord} .

\href{/app?template=yagenda&version=0.1.0}{Create project in app}

\subsubsection{How to use}\label{how-to-use}

Click the button above to create a new project using this template in
the Typst app.

You can also use the Typst CLI to start a new project on your computer
using this command:

\begin{verbatim}
typst init @preview/yagenda:0.1.0
\end{verbatim}

\includesvg[width=0.16667in,height=0.16667in]{/assets/icons/16-copy.svg}

\subsubsection{About}\label{about}

\begin{description}
\tightlist
\item[Author :]
\href{https://github.com/baptiste}{baptiste}
\item[License:]
MPL-2.0
\item[Current version:]
0.1.0
\item[Last updated:]
April 8, 2024
\item[First released:]
April 8, 2024
\item[Archive size:]
9.17 kB
\href{https://packages.typst.org/preview/yagenda-0.1.0.tar.gz}{\pandocbounded{\includesvg[keepaspectratio]{/assets/icons/16-download.svg}}}
\item[Categor y :]
\begin{itemize}
\tightlist
\item[]
\item
  \pandocbounded{\includesvg[keepaspectratio]{/assets/icons/16-envelope.svg}}
  \href{https://typst.app/universe/search/?category=office}{Office}
\end{itemize}
\end{description}

\subsubsection{Where to report issues?}\label{where-to-report-issues}

This template is a project of baptiste . You can also try to ask for
help with this template on the \href{https://forum.typst.app}{Forum} .

Please report this template to the Typst team using the
\href{https://typst.app/contact}{contact form} if you believe it is a
safety hazard or infringes upon your rights.

\phantomsection\label{versions}
\subsubsection{Version history}\label{version-history}

\begin{longtable}[]{@{}ll@{}}
\toprule\noalign{}
Version & Release Date \\
\midrule\noalign{}
\endhead
\bottomrule\noalign{}
\endlastfoot
0.1.0 & April 8, 2024 \\
\end{longtable}

Typst GmbH did not create this template and cannot guarantee correct
functionality of this template or compatibility with any version of the
Typst compiler or app.


\title{typst.app/universe/package/bob-draw}

\phantomsection\label{banner}
\section{bob-draw}\label{bob-draw}

{ 0.1.0 }

svgbob for typst, powered by wasm

\phantomsection\label{readme}
svgbob for typst, powered by wasm

This package provides a typst plugin for rendering
\href{https://github.com/ivanceras/svgbob}{svgbob} diagrams.

\begin{Shaded}
\begin{Highlighting}[]
\NormalTok{\#import "@preview/bob{-}draw:0.1.0": *}
\NormalTok{\#render(\textasciigrave{}\textasciigrave{}\textasciigrave{}}
\NormalTok{         /\textbackslash{}\_/\textbackslash{}}
\NormalTok{bob {-}\textgreater{}  ( o.o )}
\NormalTok{         \textbackslash{} " /}
\NormalTok{  .{-}{-}{-}{-}{-}{-}/  /}
\NormalTok{ (        | |}
\NormalTok{  \textasciigrave{}====== o o}
\NormalTok{\textasciigrave{}\textasciigrave{}\textasciigrave{})}
\end{Highlighting}
\end{Shaded}

output:

\pandocbounded{\includesvg[keepaspectratio]{https://github.com/typst/packages/raw/main/packages/preview/bob-draw/0.1.0/examples/basic-example.svg}}

\subsection{Full example}\label{full-example}

\begin{Shaded}
\begin{Highlighting}[]
\NormalTok{\#import "@preview/bob{-}draw:0.1.0": *}
\NormalTok{\#show raw.where(lang: "bob"): it =\textgreater{} render(it)}

\NormalTok{\#let svg = bob2svg("\textless{}{-}{-}{-}\textgreater{}")}
\NormalTok{\#render("\textless{}{-}{-}{-}\textgreater{}")}
\NormalTok{\#render(}
\NormalTok{    \textasciigrave{}\textasciigrave{}\textasciigrave{}}
\NormalTok{      0       3  }
\NormalTok{       *{-}{-}{-}{-}{-}{-}{-}* }
\NormalTok{    1 /|    2 /| }
\NormalTok{     *{-}+{-}{-}{-}{-}{-}* | }
\NormalTok{     | |4    | |7}
\NormalTok{     | *{-}{-}{-}{-}{-}|{-}*}
\NormalTok{     |/      |/}
\NormalTok{     *{-}{-}{-}{-}{-}{-}{-}*}
\NormalTok{    5       6}
\NormalTok{    \textasciigrave{}\textasciigrave{}\textasciigrave{},}
\NormalTok{    width: 25\%,}
\NormalTok{)}

\NormalTok{\textasciigrave{}\textasciigrave{}\textasciigrave{}bob}
\NormalTok{"cats:"}
\NormalTok{ /\textbackslash{}\_/\textbackslash{}  /\textbackslash{}\_/\textbackslash{}  /\textbackslash{}\_/\textbackslash{}  /\textbackslash{}\_/\textbackslash{} }
\NormalTok{( o.o )( o.o )( o.o )( o.o )}
\NormalTok{\textasciigrave{}\textasciigrave{}\textasciigrave{}}
\end{Highlighting}
\end{Shaded}

\subsubsection{How to add}\label{how-to-add}

Copy this into your project and use the import as \texttt{\ bob-draw\ }

\begin{verbatim}
#import "@preview/bob-draw:0.1.0"
\end{verbatim}

\includesvg[width=0.16667in,height=0.16667in]{/assets/icons/16-copy.svg}

Check the docs for
\href{https://typst.app/docs/reference/scripting/\#packages}{more
information on how to import packages} .

\subsubsection{About}\label{about}

\begin{description}
\tightlist
\item[Author :]
Luca Ciucci
\item[License:]
MIT
\item[Current version:]
0.1.0
\item[Last updated:]
October 24, 2023
\item[First released:]
October 24, 2023
\item[Archive size:]
126 kB
\href{https://packages.typst.org/preview/bob-draw-0.1.0.tar.gz}{\pandocbounded{\includesvg[keepaspectratio]{/assets/icons/16-download.svg}}}
\item[Repository:]
\href{https://github.com/LucaCiucci/bob-typ}{GitHub}
\end{description}

\subsubsection{Where to report issues?}\label{where-to-report-issues}

This package is a project of Luca Ciucci . Report issues on
\href{https://github.com/LucaCiucci/bob-typ}{their repository} . You can
also try to ask for help with this package on the
\href{https://forum.typst.app}{Forum} .

Please report this package to the Typst team using the
\href{https://typst.app/contact}{contact form} if you believe it is a
safety hazard or infringes upon your rights.

\phantomsection\label{versions}
\subsubsection{Version history}\label{version-history}

\begin{longtable}[]{@{}ll@{}}
\toprule\noalign{}
Version & Release Date \\
\midrule\noalign{}
\endhead
\bottomrule\noalign{}
\endlastfoot
0.1.0 & October 24, 2023 \\
\end{longtable}

Typst GmbH did not create this package and cannot guarantee correct
functionality of this package or compatibility with any version of the
Typst compiler or app.


\title{typst.app/universe/package/rivet}

\phantomsection\label{banner}
\section{rivet}\label{rivet}

{ 0.1.0 }

Register / Instruction Visualizer \& Explainer Tool with Typst, using
CeTZ

\phantomsection\label{readme}
RIVET \emph{(Register / Instruction Visualizer \& Explainer Tool)} is a
\href{https://typst.app/}{Typst} package for visualizing binary
instructions or describing the contents of a register, using the
\href{https://typst.app/universe/package/cetz}{CeTZ} package.

It is based on the \href{https://git.kb28.ch/HEL/rivet/}{homonymous
Python script}

\subsection{Examples}\label{examples}

\begin{longtable}[]{@{}l@{}}
\toprule\noalign{}
\endhead
\bottomrule\noalign{}
\endlastfoot
\href{https://github.com/typst/packages/raw/main/packages/preview/rivet/0.1.0/gallery/example1.typ}{\includegraphics[width=10.41667in,height=\textheight,keepaspectratio]{https://github.com/typst/packages/raw/main/packages/preview/rivet/0.1.0/gallery/example1.png}} \\
A bit of eveything \\
\href{https://github.com/typst/packages/raw/main/packages/preview/rivet/0.1.0/gallery/example2.typ}{\includegraphics[width=10.41667in,height=\textheight,keepaspectratio]{https://github.com/typst/packages/raw/main/packages/preview/rivet/0.1.0/gallery/example2.png}} \\
RISC-V memory instructions (blueprint) \\
\end{longtable}

\emph{Click on the example image to jump to the code.}

\subsection{Usage}\label{usage}

For more information, see the
\href{https://github.com/typst/packages/raw/main/packages/preview/rivet/0.1.0/manual.pdf}{manual}

To use this package, simply import \texttt{\ schema\ } from
\href{https://typst.app/universe/package/rivet}{rivet} and call
\texttt{\ schema.load\ } to parse a schema description. Then use
\texttt{\ schema.render\ } to render it, et voilÃ~ !

\begin{Shaded}
\begin{Highlighting}[]
\NormalTok{\#import "@preview/rivet:0.1.0": schema}
\NormalTok{\#let doc = schema.load("path/to/schema.yaml")}
\NormalTok{\#schema.render(doc)}
\end{Highlighting}
\end{Shaded}

\subsubsection{How to add}\label{how-to-add}

Copy this into your project and use the import as \texttt{\ rivet\ }

\begin{verbatim}
#import "@preview/rivet:0.1.0"
\end{verbatim}

\includesvg[width=0.16667in,height=0.16667in]{/assets/icons/16-copy.svg}

Check the docs for
\href{https://typst.app/docs/reference/scripting/\#packages}{more
information on how to import packages} .

\subsubsection{About}\label{about}

\begin{description}
\tightlist
\item[Author :]
\href{https://git.kb28.ch/HEL}{Louis Heredero}
\item[License:]
Apache-2.0
\item[Current version:]
0.1.0
\item[Last updated:]
October 3, 2024
\item[First released:]
October 3, 2024
\item[Minimum Typst version:]
0.11.0
\item[Archive size:]
120 kB
\href{https://packages.typst.org/preview/rivet-0.1.0.tar.gz}{\pandocbounded{\includesvg[keepaspectratio]{/assets/icons/16-download.svg}}}
\item[Repository:]
\href{https://git.kb28.ch/HEL/rivet-typst}{git.kb28.ch}
\item[Categor y :]
\begin{itemize}
\tightlist
\item[]
\item
  \pandocbounded{\includesvg[keepaspectratio]{/assets/icons/16-chart.svg}}
  \href{https://typst.app/universe/search/?category=visualization}{Visualization}
\end{itemize}
\end{description}

\subsubsection{Where to report issues?}\label{where-to-report-issues}

This package is a project of Louis Heredero . Report issues on
\href{https://git.kb28.ch/HEL/rivet-typst}{their repository} . You can
also try to ask for help with this package on the
\href{https://forum.typst.app}{Forum} .

Please report this package to the Typst team using the
\href{https://typst.app/contact}{contact form} if you believe it is a
safety hazard or infringes upon your rights.

\phantomsection\label{versions}
\subsubsection{Version history}\label{version-history}

\begin{longtable}[]{@{}ll@{}}
\toprule\noalign{}
Version & Release Date \\
\midrule\noalign{}
\endhead
\bottomrule\noalign{}
\endlastfoot
0.1.0 & October 3, 2024 \\
\end{longtable}

Typst GmbH did not create this package and cannot guarantee correct
functionality of this package or compatibility with any version of the
Typst compiler or app.


\title{typst.app/universe/package/colorful-boxes}

\phantomsection\label{banner}
\section{colorful-boxes}\label{colorful-boxes}

{ 1.3.1 }

Predefined colorful boxes to spice up your document.

\phantomsection\label{readme}
Colorful boxes in \href{https://github.com/typst/typst}{Typst} .

Check out \href{https://typst.app/project/rp9q3upfc69bPUCbv0BjzX}{the
example project} to see all boxes in action

Current features include:

\begin{itemize}
\tightlist
\item
  a colorful box is in four different colors (black, red, blue, green)
\item
  a colorful box with a slanted headline
\item
  a box with a simple outline
\item
  a rotateable stickynote
\end{itemize}

\subsection{Colorbox}\label{colorbox}

\pandocbounded{\includegraphics[keepaspectratio]{https://github.com/typst/packages/raw/main/packages/preview/colorful-boxes/1.3.1/examples/colorbox.png}}

\subsubsection{Usage}\label{usage}

\begin{verbatim}
#colorbox(
  title: lorem(5),
  color: "blue",
  radius: 2pt,
  width: auto
)[
  #lorem(50)
]
\end{verbatim}

\subsection{Slanted Colorbox}\label{slanted-colorbox}

\pandocbounded{\includegraphics[keepaspectratio]{https://github.com/typst/packages/raw/main/packages/preview/colorful-boxes/1.3.1/examples/slanted-colorbox.png}}

\subsubsection{Usage}\label{usage-1}

\begin{verbatim}
#slanted-colorbox(
  title: lorem(5),
  color: "red",
  radius: 0pt,
  width: auto
)[
  #lorem(50)
]
\end{verbatim}

\subsection{Outlinebox}\label{outlinebox}

\pandocbounded{\includegraphics[keepaspectratio]{https://github.com/typst/packages/raw/main/packages/preview/colorful-boxes/1.3.1/examples/outline-colorbox.png}}

\subsubsection{Usage}\label{usage-2}

\begin{verbatim}
#outlinebox(
  title: lorem(5),
  width: auto,
  radius: 2pt,
  centering: false
)[
  #lorem(50)
]

#outlinebox(
  title: lorem(5),
  color: "green",
  width: auto,
  radius: 2pt,
  centering: true
)[
  #lorem(50)
]
\end{verbatim}

\subsection{Stickybox}\label{stickybox}

\pandocbounded{\includegraphics[keepaspectratio]{https://github.com/typst/packages/raw/main/packages/preview/colorful-boxes/1.3.1/examples/stickybox.png}}

\subsubsection{Usage}\label{usage-3}

\begin{verbatim}
#stickybox(
  rotation: 5deg,
  width: 5cm
)[
  #lorem(20)
]
\end{verbatim}

\subsubsection{How to add}\label{how-to-add}

Copy this into your project and use the import as
\texttt{\ colorful-boxes\ }

\begin{verbatim}
#import "@preview/colorful-boxes:1.3.1"
\end{verbatim}

\includesvg[width=0.16667in,height=0.16667in]{/assets/icons/16-copy.svg}

Check the docs for
\href{https://typst.app/docs/reference/scripting/\#packages}{more
information on how to import packages} .

\subsubsection{About}\label{about}

\begin{description}
\tightlist
\item[Author :]
\href{https://github.com/lkoehl}{Lukas Köhl}
\item[License:]
MIT
\item[Current version:]
1.3.1
\item[Last updated:]
March 16, 2024
\item[First released:]
August 6, 2023
\item[Archive size:]
3.09 kB
\href{https://packages.typst.org/preview/colorful-boxes-1.3.1.tar.gz}{\pandocbounded{\includesvg[keepaspectratio]{/assets/icons/16-download.svg}}}
\item[Repository:]
\href{https://github.com/lkoehl/typst-boxes}{GitHub}
\item[Categor y :]
\begin{itemize}
\tightlist
\item[]
\item
  \pandocbounded{\includesvg[keepaspectratio]{/assets/icons/16-package.svg}}
  \href{https://typst.app/universe/search/?category=components}{Components}
\end{itemize}
\end{description}

\subsubsection{Where to report issues?}\label{where-to-report-issues}

This package is a project of Lukas Köhl . Report issues on
\href{https://github.com/lkoehl/typst-boxes}{their repository} . You can
also try to ask for help with this package on the
\href{https://forum.typst.app}{Forum} .

Please report this package to the Typst team using the
\href{https://typst.app/contact}{contact form} if you believe it is a
safety hazard or infringes upon your rights.

\phantomsection\label{versions}
\subsubsection{Version history}\label{version-history}

\begin{longtable}[]{@{}ll@{}}
\toprule\noalign{}
Version & Release Date \\
\midrule\noalign{}
\endhead
\bottomrule\noalign{}
\endlastfoot
1.3.1 & March 16, 2024 \\
\href{https://typst.app/universe/package/colorful-boxes/1.2.0/}{1.2.0} &
September 13, 2023 \\
\href{https://typst.app/universe/package/colorful-boxes/1.1.0/}{1.1.0} &
August 19, 2023 \\
\href{https://typst.app/universe/package/colorful-boxes/1.0.0/}{1.0.0} &
August 6, 2023 \\
\end{longtable}

Typst GmbH did not create this package and cannot guarantee correct
functionality of this package or compatibility with any version of the
Typst compiler or app.


\title{typst.app/universe/package/casual-szu-report}

\phantomsection\label{banner}
\phantomsection\label{template-thumbnail}
\pandocbounded{\includegraphics[keepaspectratio]{https://packages.typst.org/preview/thumbnails/casual-szu-report-0.1.0-small.webp}}

\section{casual-szu-report}\label{casual-szu-report}

{ 0.1.0 }

A template for SZU course reports.

\href{/app?template=casual-szu-report&version=0.1.0}{Create project in
app}

\phantomsection\label{readme}
A Typst template for SZU course reports.

\subsection{Usage}\label{usage}

Example is at \texttt{\ template/main.typ\ } . TLDR:

\begin{Shaded}
\begin{Highlighting}[]
\NormalTok{\#import "lib.typ": template}

\NormalTok{\#show: template.with(}
\NormalTok{  course{-}title: [养鸡学习],}
\NormalTok{  experiment{-}title: [养鸡],}
\NormalTok{  faculty: [养鸡学院],}
\NormalTok{  major: [智能养鸡],}
\NormalTok{  instructor: [鸡老师],}
\NormalTok{  reporter: [鸡],}
\NormalTok{  student{-}id: [4144010590],}
\NormalTok{  class: [养鸡99班],}
\NormalTok{  experiment{-}date: datetime(year: 1983, month: 9, day: 27),}
\NormalTok{  features: (}
\NormalTok{    "Bibliography": "template/refs.bib",}
\NormalTok{  ),}
\NormalTok{)}
\NormalTok{// Start here}
\end{Highlighting}
\end{Shaded}

Features:

\begin{enumerate}
\tightlist
\item
  \texttt{\ Bibliography\ } : Bibliography file path.
\item
  \texttt{\ FontFamily\ } : Custom font family. Default:
  \texttt{\ ("Noto\ Serif",\ "Noto\ Serif\ CJK\ SC")\ }
\item
  \texttt{\ CitationStyle\ } : Citation Style supported by Typst.
  Default: \texttt{\ gb-7714-2015-numeric\ }
\end{enumerate}

\begin{itemize}
\tightlist
\item
  Only work if have \texttt{\ Bibliography\ } specified.
\end{itemize}

\begin{enumerate}
\setcounter{enumi}{3}
\tightlist
\item
  \texttt{\ CourseID\ } : Add a Course ID box on top-right of the cover.
  NOT YET IMPLEMENTED.
\end{enumerate}

\subsection{Method}\label{method}

The template will traverse body content, and split it into groups
according to Heading-1 layout. Each group content will be wrapped with
\texttt{\ table.cell\ } . So all content will be wrapped in container,
you can’t use \texttt{\ pagebreak()\ } in your body content.

\subsection{Warning}\label{warning}

This is not a serious work and may have some rough edges. And reports
from different faculties isn’t entirely uniform. Be careful when using
it.

\href{/app?template=casual-szu-report&version=0.1.0}{Create project in
app}

\subsubsection{How to use}\label{how-to-use}

Click the button above to create a new project using this template in
the Typst app.

You can also use the Typst CLI to start a new project on your computer
using this command:

\begin{verbatim}
typst init @preview/casual-szu-report:0.1.0
\end{verbatim}

\includesvg[width=0.16667in,height=0.16667in]{/assets/icons/16-copy.svg}

\subsubsection{About}\label{about}

\begin{description}
\tightlist
\item[Author :]
\href{mailto:jiang131072@gmail.com}{Keaton Jiang}
\item[License:]
MIT
\item[Current version:]
0.1.0
\item[Last updated:]
September 8, 2024
\item[First released:]
September 8, 2024
\item[Archive size:]
36.6 kB
\href{https://packages.typst.org/preview/casual-szu-report-0.1.0.tar.gz}{\pandocbounded{\includesvg[keepaspectratio]{/assets/icons/16-download.svg}}}
\item[Repository:]
\href{https://github.com/jiang131072/casual-szu-report}{GitHub}
\item[Categor y :]
\begin{itemize}
\tightlist
\item[]
\item
  \pandocbounded{\includesvg[keepaspectratio]{/assets/icons/16-speak.svg}}
  \href{https://typst.app/universe/search/?category=report}{Report}
\end{itemize}
\end{description}

\subsubsection{Where to report issues?}\label{where-to-report-issues}

This template is a project of Keaton Jiang . Report issues on
\href{https://github.com/jiang131072/casual-szu-report}{their
repository} . You can also try to ask for help with this template on the
\href{https://forum.typst.app}{Forum} .

Please report this template to the Typst team using the
\href{https://typst.app/contact}{contact form} if you believe it is a
safety hazard or infringes upon your rights.

\phantomsection\label{versions}
\subsubsection{Version history}\label{version-history}

\begin{longtable}[]{@{}ll@{}}
\toprule\noalign{}
Version & Release Date \\
\midrule\noalign{}
\endhead
\bottomrule\noalign{}
\endlastfoot
0.1.0 & September 8, 2024 \\
\end{longtable}

Typst GmbH did not create this template and cannot guarantee correct
functionality of this template or compatibility with any version of the
Typst compiler or app.


\title{typst.app/universe/package/oxifmt}

\phantomsection\label{banner}
\section{oxifmt}\label{oxifmt}

{ 0.2.1 }

Convenient Rust-like string formatting in Typst

\phantomsection\label{readme}
A Typst library that brings convenient string formatting and
interpolation through the \texttt{\ strfmt\ } function. Its syntax is
taken directly from Rust’s \texttt{\ format!\ } syntax, so feel free
to read its page for more information (
\url{https://doc.rust-lang.org/std/fmt/} ); however, this README should
have enough information and examples for all expected uses of the
library. Only a few things aren’t supported from the Rust syntax, such
as the \texttt{\ p\ } (pointer) format type, or the \texttt{\ .*\ }
precision specifier.

A few extras (beyond the Rust-like syntax) will be added over time,
though (feel free to drop suggestions at the repository:
\url{https://github.com/PgBiel/typst-oxifmt} ). The first “extra� so
far is the \texttt{\ fmt-decimal-separator:\ "string"\ } parameter,
which lets you customize the decimal separator for decimal numbers
(floats) inserted into strings. E.g.
\texttt{\ strfmt("Result:\ \{\}",\ 5.8,\ fmt-decimal-separator:\ ",")\ }
will return the string \texttt{\ "Result:\ 5,8"\ } (comma instead of
dot). See more below.

\textbf{Compatible with:} \href{https://github.com/typst/typst}{Typst}
v0.4.0+

\subsection{Table of Contents}\label{table-of-contents}

\begin{itemize}
\tightlist
\item
  \href{https://github.com/typst/packages/raw/main/packages/preview/oxifmt/0.2.1/\#usage}{Usage}

  \begin{itemize}
  \tightlist
  \item
    \href{https://github.com/typst/packages/raw/main/packages/preview/oxifmt/0.2.1/\#formatting-options}{Formatting
    options}
  \item
    \href{https://github.com/typst/packages/raw/main/packages/preview/oxifmt/0.2.1/\#examples}{Examples}
  \item
    \href{https://github.com/typst/packages/raw/main/packages/preview/oxifmt/0.2.1/\#grammar}{Grammar}
  \end{itemize}
\item
  \href{https://github.com/typst/packages/raw/main/packages/preview/oxifmt/0.2.1/\#issues-and-contributing}{Issues
  and Contributing}
\item
  \href{https://github.com/typst/packages/raw/main/packages/preview/oxifmt/0.2.1/\#testing}{Testing}
\item
  \href{https://github.com/typst/packages/raw/main/packages/preview/oxifmt/0.2.1/\#changelog}{Changelog}
\item
  \href{https://github.com/typst/packages/raw/main/packages/preview/oxifmt/0.2.1/\#license}{License}
\end{itemize}

\subsection{Usage}\label{usage}

You can use this library through Typst’s package manager (for Typst
v0.6.0+):

\begin{Shaded}
\begin{Highlighting}[]
\NormalTok{\#import "@preview/oxifmt:0.2.1": strfmt}
\end{Highlighting}
\end{Shaded}

For older Typst versions, download the \texttt{\ oxifmt.typ\ } file
either from Releases or directly from the repository. Then, move it to
your project’s folder, and write at the top of your Typst file(s):

\begin{Shaded}
\begin{Highlighting}[]
\NormalTok{\#import "oxifmt.typ": strfmt}
\end{Highlighting}
\end{Shaded}

Doing the above will give you access to the main function provided by
this library ( \texttt{\ strfmt\ } ), which accepts a format string,
followed by zero or more replacements to insert in that string
(according to \texttt{\ \{...\}\ } formats inserted in that string), an
optional \texttt{\ fmt-decimal-separator\ } parameter, and returns the
formatted string, as described below.

Its syntax is almost identical to Rust’s \texttt{\ format!\ } (as
specified here: \url{https://doc.rust-lang.org/std/fmt/} ). You can
escape formats by duplicating braces ( \texttt{\ \{\{\ } and
\texttt{\ \}\}\ } become \texttt{\ \{\ } and \texttt{\ \}\ } ). Here’s
an example (see more examples in the file
\texttt{\ tests/strfmt-tests.typ\ } ):

\begin{Shaded}
\begin{Highlighting}[]
\NormalTok{\#import "@preview/oxifmt:0.2.1": strfmt}

\NormalTok{\#let s = strfmt("I\textquotesingle{}m \{\}. I have \{num\} cars. I\textquotesingle{}m \{0\}. \{\} is \{\{cool\}\}.", "John", "Carl", num: 10)}
\NormalTok{\#assert.eq(s, "I\textquotesingle{}m John. I have 10 cars. I\textquotesingle{}m John. Carl is \{cool\}.")}
\end{Highlighting}
\end{Shaded}

Note that \texttt{\ \{\}\ } extracts positional arguments after the
string sequentially (the first \texttt{\ \{\}\ } extracts the first one,
the second \texttt{\ \{\}\ } extracts the second one, and so on), while
\texttt{\ \{0\}\ } , \texttt{\ \{1\}\ } , etc. will always extract the
first, the second etc. positional arguments after the string.
Additionally, \texttt{\ \{bananas\}\ } will extract the named argument
“bananas�.

\subsubsection{Formatting options}\label{formatting-options}

You can use \texttt{\ \{:spec\}\ } to customize your output. See the
Rust docs linked above for more info, but a summary is below.

(You may also want to check out the examples at
\href{https://github.com/typst/packages/raw/main/packages/preview/oxifmt/0.2.1/\#examples}{Examples}
.)

\begin{itemize}
\tightlist
\item
  Adding a \texttt{\ ?\ } at the end of \texttt{\ spec\ } (that is,
  writing e.g. \texttt{\ \{0:?\}\ } ) will call \texttt{\ repr()\ } to
  stringify your argument, instead of \texttt{\ str()\ } . Note that
  this only has an effect if your argument is a string, an integer, a
  float or a \texttt{\ label()\ } /
  \texttt{\ \textless{}label\textgreater{}\ } - for all other types
  (such as booleans or elements), \texttt{\ repr()\ } is always called
  (as \texttt{\ str()\ } is unsupported for those).

  \begin{itemize}
  \tightlist
  \item
    For strings, \texttt{\ ?\ } (and thus \texttt{\ repr()\ } ) has the
    effect of printing them with double quotes. For floats, this ensures
    a \texttt{\ .0\ } appears after it, even if it doesn’t have
    decimal digits. For integers, this doesn’t change anything.
    Finally, for labels, the \texttt{\ \textless{}label\textgreater{}\ }
    (with \texttt{\ ?\ } ) is printed as
    \texttt{\ \textless{}label\textgreater{}\ } instead of
    \texttt{\ label\ } .
  \item
    \textbf{TIP:} Prefer to always use \texttt{\ ?\ } when you’re
    inserting something that isn’t a string, number or label, in order
    to ensure consistent results even if the library eventually changes
    the non- \texttt{\ ?\ } representation.
  \end{itemize}
\item
  After the \texttt{\ :\ } , add e.g. \texttt{\ \_\textless{}8\ } to
  align the string to the left, padding it with as many \texttt{\ \_\ }
  s as necessary for it to be at least \texttt{\ 8\ } characters long
  (for example). Replace \texttt{\ \textless{}\ } by
  \texttt{\ \textgreater{}\ } for right alignment, or \texttt{\ \^{}\ }
  for center alignment. (If the \texttt{\ \_\ } is omitted, it defaults
  to ’ ’ (aligns with spaces).)

  \begin{itemize}
  \tightlist
  \item
    If you prefer to specify the minimum width (the \texttt{\ 8\ }
    there) as a separate argument to \texttt{\ strfmt\ } instead, you
    can specify \texttt{\ argument\$\ } in place of the width, which
    will extract it from the integer at \texttt{\ argument\ } . For
    example, \texttt{\ \_\^{}3\$\ } will center align the output with
    \texttt{\ \_\ } s, where the minimum width desired is specified by
    the fourth positional argument (index \texttt{\ 3\ } ), as an
    integer. This means that a call such as
    \texttt{\ strfmt("\{:\_\^{}3\$\}",\ 1,\ 2,\ 3,\ 4)\ } would produce
    \texttt{\ "\_\_1\_\_"\ } , as \texttt{\ 3\$\ } would evaluate to
    \texttt{\ 4\ } (the value at the fourth positional argument/index
    \texttt{\ 3\ } ). Similarly, \texttt{\ named\$\ } would take the
    width from the argument with name \texttt{\ named\ } , if it is an
    integer (otherwise, error).
  \end{itemize}
\item
  \textbf{For numbers:}

  \begin{itemize}
  \tightlist
  \item
    Specify \texttt{\ +\ } after the \texttt{\ :\ } to ensure zero or
    positive numbers are prefixed with \texttt{\ +\ } before them
    (instead of having no sign). \texttt{\ -\ } is also accepted but
    ignored (negative numbers always specify their sign anyways).
  \item
    Use something like \texttt{\ :09\ } to add zeroes to the left of the
    number until it has at least 9 digits / characters.

    \begin{itemize}
    \tightlist
    \item
      The \texttt{\ 9\ } here is also a width, so the same comment from
      before applies (you can add \texttt{\ \$\ } to take it from an
      argument to the \texttt{\ strfmt\ } function).
    \end{itemize}
  \item
    Use \texttt{\ :.5\ } to ensure your float is represented with 5
    decimal digits of precision (zeroes are added to the right if
    needed; otherwise, it is rounded, \textbf{not truncated} ).

    \begin{itemize}
    \tightlist
    \item
      Note that floating point inaccuracies can be sometimes observed
      here, which is an unfortunate current limitation.
    \item
      Similarly to \texttt{\ width\ } , the precision can also be
      specified via an argument with the \texttt{\ \$\ } syntax:
      \texttt{\ .5\$\ } will take the precision from the integer at
      argument number 5 (the sixth one), while \texttt{\ .test\$\ } will
      take it from the argument named \texttt{\ test\ } .
    \end{itemize}
  \item
    \textbf{Integers only:} Add \texttt{\ x\ } (lowercase hex) or
    \texttt{\ X\ } (uppercase) at the end of the \texttt{\ spec\ } to
    convert the number to hexadecimal. Also, \texttt{\ b\ } will convert
    it to binary, while \texttt{\ o\ } will convert to octal.

    \begin{itemize}
    \tightlist
    \item
      Specify a hashtag, e.g. \texttt{\ \#x\ } or \texttt{\ \#b\ } , to
      prepend the corresponding base prefix to the base-converted
      number, e.g. \texttt{\ 0xABC\ } instead of \texttt{\ ABC\ } .
    \end{itemize}
  \item
    Add \texttt{\ e\ } or \texttt{\ E\ } at the end of the
    \texttt{\ spec\ } to ensure the number is represented in scientific
    notation (with \texttt{\ e\ } or \texttt{\ E\ } as the exponent
    separator, respectively).
  \item
    For decimal numbers (floats), you can specify
    \texttt{\ fmt-decimal-separator:\ ","\ } to \texttt{\ strfmt\ } to
    have the decimal separator be a comma instead of a dot, for example.

    \begin{itemize}
    \tightlist
    \item
      To have this be the default, you can alias \texttt{\ strfmt\ } ,
      such as using
      \texttt{\ \#let\ strfmt\ =\ strfmt.with(fmt-decimal-separator:\ ",")\ }
      .
    \end{itemize}
  \item
    Number spec arguments (such as \texttt{\ .5\ } ) are ignored when
    the argument is not a number, but e.g. a string, even if it looks
    like a number (such as \texttt{\ "5"\ } ).
  \end{itemize}
\item
  Note that all spec arguments above \textbf{have to be specified in
  order} - if you mix up the order, it won’t work properly!

  \begin{itemize}
  \tightlist
  \item
    Check the grammar below for the proper order, but, in summary: fill
    (character) with align ( \texttt{\ \textless{}\ } ,
    \texttt{\ \textgreater{}\ } or \texttt{\ \^{}\ } ) -\textgreater{}
    sign ( \texttt{\ +\ } or \texttt{\ -\ } ) -\textgreater{}
    \texttt{\ \#\ } -\textgreater{} \texttt{\ 0\ } (for 0 left-padding
    of numbers) -\textgreater{} width (e.g. \texttt{\ 8\ } from
    \texttt{\ 08\ } or \texttt{\ 9\ } from \texttt{\ -\textless{}9\ } )
    -\textgreater{} \texttt{\ .precision\ } -\textgreater{} spec type (
    \texttt{\ ?\ } , \texttt{\ x\ } , \texttt{\ X\ } , \texttt{\ b\ } ,
    \texttt{\ o\ } , \texttt{\ e\ } , \texttt{\ E\ } )).
  \end{itemize}
\end{itemize}

Some examples:

\begin{Shaded}
\begin{Highlighting}[]
\NormalTok{\#import "@preview/oxifmt:0.2.1": strfmt}

\NormalTok{\#let s1 = strfmt("\{0:?\}, \{test:+012e\}, \{1:{-}\textless{}\#8x\}", "hi", {-}74, test: 569.4)}
\NormalTok{\#assert.eq(s1, "\textbackslash{}"hi\textbackslash{}", +00005.694e2, {-}0x4a{-}{-}{-}")}

\NormalTok{\#let s2 = strfmt("\{:\_\textgreater{}+11.5\}", 59.4)}
\NormalTok{\#assert.eq(s2, "\_\_+59.40000")}

\NormalTok{\#let s3 = strfmt("Dict: \{:!\textless{}10?\}", (a: 5))}
\NormalTok{\#assert.eq(s3, "Dict: (a: 5)!!!!")}
\end{Highlighting}
\end{Shaded}

\subsubsection{Examples}\label{examples}

\begin{itemize}
\tightlist
\item
  \textbf{Inserting labels, text and numbers into strings:}
\end{itemize}

\begin{Shaded}
\begin{Highlighting}[]
\NormalTok{\#import "@preview/oxifmt:0.2.1": strfmt}

\NormalTok{\#let s = strfmt("First: \{\}, Second: \{\}, Fourth: \{3\}, Banana: \{banana\} (brackets: \{\{escaped\}\})", 1, 2.1, 3, label("four"), banana: "Banana!!")}
\NormalTok{\#assert.eq(s, "First: 1, Second: 2.1, Fourth: four, Banana: Banana!! (brackets: \{escaped\})")}
\end{Highlighting}
\end{Shaded}

\begin{itemize}
\tightlist
\item
  \textbf{Forcing \texttt{\ repr()\ } with \texttt{\ \{:?\}\ }} (which
  adds quotes around strings, and other things - basically represents a
  Typst value):
\end{itemize}

\begin{Shaded}
\begin{Highlighting}[]
\NormalTok{\#import "@preview/oxifmt:0.2.1": strfmt}

\NormalTok{\#let s = strfmt("The value is: \{:?\} | Also the label is \{:?\}", "something", label("label"))}
\NormalTok{\#assert.eq(s, "The value is: \textbackslash{}"something\textbackslash{}" | Also the label is \textless{}label\textgreater{}")}
\end{Highlighting}
\end{Shaded}

\begin{itemize}
\tightlist
\item
  \textbf{Inserting other types than numbers and strings} (for now, they
  will always use \texttt{\ repr()\ } , even without
  \texttt{\ \{...:?\}\ } , although that is more explicit):
\end{itemize}

\begin{Shaded}
\begin{Highlighting}[]
\NormalTok{\#import "@preview/oxifmt:0.2.1": strfmt}

\NormalTok{\#let s = strfmt("Values: \{:?\}, \{1:?\}, \{stuff:?\}", (test: 500), ("a", 5.1), stuff: [a])}
\NormalTok{\#assert.eq(s, "Values: (test: 500), (\textbackslash{}"a\textbackslash{}", 5.1), [a]")}
\end{Highlighting}
\end{Shaded}

\begin{itemize}
\tightlist
\item
  \textbf{Padding to a certain width with characters:} Use
  \texttt{\ \{:x\textless{}8\}\ } , where \texttt{\ x\ } is the
  \textbf{character to pad with} (e.g. space or \texttt{\ \_\ } , but
  can be anything), \texttt{\ \textless{}\ } is the \textbf{alignment of
  the original text} relative to the padding (can be
  \texttt{\ \textless{}\ } for left aligned (padding goes to the right),
  \texttt{\ \textgreater{}\ } for right aligned (padded to its left) and
  \texttt{\ \^{}\ } for center aligned (padded at both left and right)),
  and \texttt{\ 8\ } is the \textbf{desired total width} (padding will
  add enough characters to reach this width; if the replacement string
  already has this width, no padding will be added):
\end{itemize}

\begin{Shaded}
\begin{Highlighting}[]
\NormalTok{\#import "@preview/oxifmt:0.2.1": strfmt}

\NormalTok{\#let s = strfmt("Left5 \{:{-}\textless{}5\}, Right6 \{:=\textgreater{}6\}, Center10 \{centered: \^{}10?\}, Left3 \{tleft:\_\textless{}3\}", "xx", 539, tleft: "okay", centered: [a])}
\NormalTok{\#assert.eq(s, "Left5 xx{-}{-}{-}, Right6 ===539, Center10     [a]    , Left3 okay")}
\NormalTok{// note how \textquotesingle{}okay\textquotesingle{} didn\textquotesingle{}t suffer any padding at all (it already had at least the desired total width).}
\end{Highlighting}
\end{Shaded}

\begin{itemize}
\tightlist
\item
  \textbf{Padding numbers with zeroes to the left:} It’s a similar
  functionality to the above, however you write \texttt{\ \{:08\}\ } for
  8 characters (for instance) - note that any characters in the
  number’s representation matter for width (including sign, dot and
  decimal part):
\end{itemize}

\begin{Shaded}
\begin{Highlighting}[]
\NormalTok{\#import "@preview/oxifmt:0.2.1": strfmt}

\NormalTok{\#let s = strfmt("Left{-}padded7 numbers: \{:07\} \{:07\} \{:07\} \{3:07\}", 123, {-}344, 44224059, 45.32)}
\NormalTok{\#assert.eq(s, "Left{-}padded7 numbers: 0000123 {-}000344 44224059 0045.32")}
\end{Highlighting}
\end{Shaded}

\begin{itemize}
\tightlist
\item
  \textbf{Defining padding-to width using parameters, not literals:} If
  you want the desired replacement width (the \texttt{\ 8\ } in
  \texttt{\ \{:08\}\ } or \texttt{\ \{:\ \^{}8\}\ } ) to be passed via
  parameter (instead of being hardcoded into the format string), you can
  specify \texttt{\ parameter\$\ } in place of the width, e.g.
  \texttt{\ \{:02\$\}\ } to take it from the third positional parameter,
  or \texttt{\ \{:a\textgreater{}banana\$\}\ } to take it from the
  parameter named \texttt{\ banana\ } - note that the chosen parameter
  \textbf{must be an integer} (desired total width):
\end{itemize}

\begin{Shaded}
\begin{Highlighting}[]
\NormalTok{\#import "@preview/oxifmt:0.2.1": strfmt}

\NormalTok{\#let s = strfmt("Padding depending on parameter: \{0:02$\} and \{0:a\textgreater{}banana$\}", 432, 0, 5, banana: 9)}
\NormalTok{\#assert.eq(s, "Padding depending on parameter: 00432 aaaaaa432")  // widths 5 and 9}
\end{Highlighting}
\end{Shaded}

\begin{itemize}
\tightlist
\item
  \textbf{Displaying \texttt{\ +\ } on positive numbers:} Just add a
  \texttt{\ +\ } at the “beginning�, i.e., before the
  \texttt{\ \#0\ } (if either is there), or after the custom fill and
  align (if it’s there and not \texttt{\ 0\ } - see
  \href{https://github.com/typst/packages/raw/main/packages/preview/oxifmt/0.2.1/\#grammar}{Grammar}
  for the exact positioning), like so:
\end{itemize}

\begin{Shaded}
\begin{Highlighting}[]
\NormalTok{\#import "@preview/oxifmt:0.2.1": strfmt}

\NormalTok{\#let s = strfmt("Some numbers: \{:+\} \{:+08\}; With fill and align: \{:\_\textless{}+8\}; Negative (no{-}op): \{neg:+\}", 123, 456, 4444, neg: {-}435)}
\NormalTok{\#assert.eq(s, "Some numbers: +123 +0000456; With fill and align: +4444\_\_\_; Negative (no{-}op): {-}435")}
\end{Highlighting}
\end{Shaded}

\begin{itemize}
\tightlist
\item
  \textbf{Converting numbers to bases 2, 8 and 16:} Use one of the
  following specifier types (i.e., characters which always go at the
  very end of the format): \texttt{\ b\ } (binary), \texttt{\ o\ }
  (octal), \texttt{\ x\ } (lowercase hexadecimal) or \texttt{\ X\ }
  (uppercase hexadecimal). You can also add a \texttt{\ \#\ } between
  \texttt{\ +\ } and \texttt{\ 0\ } (see the exact position at the
  \href{https://github.com/typst/packages/raw/main/packages/preview/oxifmt/0.2.1/\#grammar}{Grammar}
  ) to display a \textbf{base prefix} before the number (i.e.
  \texttt{\ 0b\ } for binary, \texttt{\ 0o\ } for octal and
  \texttt{\ 0x\ } for hexadecimal):
\end{itemize}

\begin{Shaded}
\begin{Highlighting}[]
\NormalTok{\#import "@preview/oxifmt:0.2.1": strfmt}

\NormalTok{\#let s = strfmt("Bases (10, 2, 8, 16(l), 16(U):) \{0\} \{0:b\} \{0:o\} \{0:x\} \{0:X\} | W/ prefixes and modifiers: \{0:\#b\} \{0:+\#09o\} \{0:\_\textgreater{}+\#9X\}", 124)}
\NormalTok{\#assert.eq(s, "Bases (10, 2, 8, 16(l), 16(U):) 124 1111100 174 7c 7C | W/ prefixes and modifiers: 0b1111100 +0o000174 \_\_\_\_+0x7C")}
\end{Highlighting}
\end{Shaded}

\begin{itemize}
\tightlist
\item
  \textbf{Picking float precision (right-extending with zeroes):} Add,
  at the end of the format (just before the spec type (such as
  \texttt{\ ?\ } ), if there’s any), either \texttt{\ .precision\ }
  (hardcoded, e.g. \texttt{\ .8\ } for 8 decimal digits) or
  \texttt{\ .parameter\$\ } (taking the precision value from the
  specified parameter, like with \texttt{\ width\ } ):
\end{itemize}

\begin{Shaded}
\begin{Highlighting}[]
\NormalTok{\#import "@preview/oxifmt:0.2.1": strfmt}

\NormalTok{\#let s = strfmt("\{0:.8\} \{0:.2$\} \{0:.potato$\}", 1.234, 0, 2, potato: 5)}
\NormalTok{\#assert.eq(s, "1.23400000 1.23 1.23400")}
\end{Highlighting}
\end{Shaded}

\begin{itemize}
\tightlist
\item
  \textbf{Scientific notation:} Use \texttt{\ e\ } (lowercase) or
  \texttt{\ E\ } (uppercase) as specifier types (can be combined with
  precision):
\end{itemize}

\begin{Shaded}
\begin{Highlighting}[]
\NormalTok{\#import "@preview/oxifmt:0.2.1": strfmt}

\NormalTok{\#let s = strfmt("\{0:e\} \{0:E\} \{0:+.9e\} | \{1:e\} | \{2:.4E\}", 124.2312, 50, {-}0.02)}
\NormalTok{\#assert.eq(s, "1.242312e2 1.242312E2 +1.242312000e2 | 5e1 | {-}2.0000E{-}2")}
\end{Highlighting}
\end{Shaded}

\begin{itemize}
\tightlist
\item
  \textbf{Customizing the decimal separator on floats:} Just specify
  \texttt{\ fmt-decimal-separator:\ ","\ } (comma as an example):
\end{itemize}

\begin{Shaded}
\begin{Highlighting}[]
\NormalTok{\#import "@preview/oxifmt:0.2.1": strfmt}

\NormalTok{\#let s = strfmt("\{0\} \{0:.6\} \{0:.5e\}", 1.432, fmt{-}decimal{-}separator: ",")}
\NormalTok{\#assert.eq(s, "1,432 1,432000 1,43200e0")}
\end{Highlighting}
\end{Shaded}

\subsubsection{Grammar}\label{grammar}

Here’s the grammar specification for valid format \texttt{\ spec\ } s
(in \texttt{\ \{name:spec\}\ } ), which is basically Rust’s format:

\begin{verbatim}
format_spec := [[fill]align][sign]['#']['0'][width]['.' precision]type
fill := character
align := '<' | '^' | '>'
sign := '+' | '-'
width := count
precision := count | '*'
type := '' | '?' | 'x?' | 'X?' | identifier
count := parameter | integer
parameter := argument '$'
\end{verbatim}

Note, however, that precision of type \texttt{\ .*\ } is not supported
yet and will raise an error.

\subsection{Issues and Contributing}\label{issues-and-contributing}

Please report any issues or send any contributions (through pull
requests) to the repository at
\url{https://github.com/PgBiel/typst-oxifmt}

\subsection{Testing}\label{testing}

If you wish to contribute, you may clone the repository and test this
package with the following commands (from the project root folder):

\begin{Shaded}
\begin{Highlighting}[]
\FunctionTok{git}\NormalTok{ clone https://github.com/PgBiel/typst{-}oxifmt}
\BuiltInTok{cd}\NormalTok{ typst{-}oxifmt/tests}
\ExtensionTok{typst}\NormalTok{ c strfmt{-}tests.typ }\AttributeTok{{-}{-}root}\NormalTok{ ..}
\end{Highlighting}
\end{Shaded}

The tests succeeded if you received no error messages from the last
command (please ensure you’re using a supported Typst version).

\subsection{Changelog}\label{changelog}

\subsubsection{v0.2.1}\label{v0.2.1}

\begin{itemize}
\tightlist
\item
  Fixed formatting of UTF-8 strings. Before, strings with multi-byte
  UTF-8 codepoints would cause formatting inconsistencies or even
  crashes. (
  \href{https://github.com/PgBiel/typst-oxifmt/issues/6}{Issue \#6} )
\item
  Fixed an inconsistency in negative number formatting. Now, it will
  always print a regular hyphen (e.g. ‘-2’), which is consistent
  with Rust’s behavior; before, it would occasionally print a minus
  sign instead (as observed in a comment to
  \href{https://github.com/PgBiel/typst-oxifmt/issues/4}{Issue \#4} ).
\item
  Added compatibility with Typst 0.8.0’s new type system.
\end{itemize}

\subsubsection{v0.2.0}\label{v0.2.0}

\begin{itemize}
\tightlist
\item
  The package’s name is now \texttt{\ oxifmt\ } !
\item
  \texttt{\ oxifmt:0.2.0\ } is now available through Typst’s Package
  Manager! You can now write
  \texttt{\ \#import\ "@preview/oxifmt:0.2.0":\ strfmt\ } to use the
  library.
\item
  Greatly improved the README, adding a section for common examples.
\item
  Fixed negative numbers being formatted with two minus signs.
\item
  Fixed custom precision of floats not working when they are exact
  integers.
\end{itemize}

\subsubsection{v0.1.0}\label{v0.1.0}

\begin{itemize}
\tightlist
\item
  Initial release, added \texttt{\ strfmt\ } .
\end{itemize}

\subsection{License}\label{license}

MIT-0 license (see the \texttt{\ LICENSE\ } file).

\subsubsection{How to add}\label{how-to-add}

Copy this into your project and use the import as \texttt{\ oxifmt\ }

\begin{verbatim}
#import "@preview/oxifmt:0.2.1"
\end{verbatim}

\includesvg[width=0.16667in,height=0.16667in]{/assets/icons/16-copy.svg}

Check the docs for
\href{https://typst.app/docs/reference/scripting/\#packages}{more
information on how to import packages} .

\subsubsection{About}\label{about}

\begin{description}
\tightlist
\item[Author :]
\href{https://github.com/PgBiel}{PgBiel}
\item[License:]
MIT-0
\item[Current version:]
0.2.1
\item[Last updated:]
May 6, 2024
\item[First released:]
August 2, 2023
\item[Archive size:]
12.0 kB
\href{https://packages.typst.org/preview/oxifmt-0.2.1.tar.gz}{\pandocbounded{\includesvg[keepaspectratio]{/assets/icons/16-download.svg}}}
\item[Repository:]
\href{https://github.com/PgBiel/typst-oxifmt}{GitHub}
\end{description}

\subsubsection{Where to report issues?}\label{where-to-report-issues}

This package is a project of PgBiel . Report issues on
\href{https://github.com/PgBiel/typst-oxifmt}{their repository} . You
can also try to ask for help with this package on the
\href{https://forum.typst.app}{Forum} .

Please report this package to the Typst team using the
\href{https://typst.app/contact}{contact form} if you believe it is a
safety hazard or infringes upon your rights.

\phantomsection\label{versions}
\subsubsection{Version history}\label{version-history}

\begin{longtable}[]{@{}ll@{}}
\toprule\noalign{}
Version & Release Date \\
\midrule\noalign{}
\endhead
\bottomrule\noalign{}
\endlastfoot
0.2.1 & May 6, 2024 \\
\href{https://typst.app/universe/package/oxifmt/0.2.0/}{0.2.0} & August
2, 2023 \\
\end{longtable}

Typst GmbH did not create this package and cannot guarantee correct
functionality of this package or compatibility with any version of the
Typst compiler or app.


\title{typst.app/universe/package/pointless-size}

\phantomsection\label{banner}
\section{pointless-size}\label{pointless-size}

{ 0.1.0 }

中æ--‡å­---å?·çš„å?·æ•°åˆ¶å?Šå­---ä½``度é‡?å?•ä½? Chinese size system
(hÃ~o-system) and type-related measurements units

\phantomsection\label{readme}
中æ--‡å­---å?·çš„å?·æ•°åˆ¶å?Šå­---ä½``度é‡?å?•ä½?。 Chinese size system
(hÃ~o-system) and type-related measurements units.

\begin{Shaded}
\begin{Highlighting}[]
\NormalTok{\#import "@preview/pointless{-}size:0.1.0": zh, zihao}

\NormalTok{\#set text(size: zh(5)) // 五号(10.5pt)}
\NormalTok{// or}
\NormalTok{\#set text(zh(5))}
\NormalTok{\#show: zihao(5)}

\NormalTok{// 小号用负数表示 use negative numbers for small sizes }
\NormalTok{\#zh({-}4) // 小四(12pt)}
\NormalTok{\#zh(1) // 一号(26pt)}
\NormalTok{\#zh({-}1) // 小一(24pt)}
\NormalTok{\#zh("{-}0") // 小初(36pt)}
\NormalTok{\#zh(0) // 初号(42pt)}
\end{Highlighting}
\end{Shaded}

å­---å?·æ²¡æœ‰ç»Ÿä¸€è§„定,本åŒ\ldots 默认与 CTeXã€?MS
Wordã€?WPSã€?Adobe 的中æ--‡è§„则一致。 Chinese size systems were
not standardized. By default, this package is consistent with Chinese
rules of CTeX, MS Word, WPS, Adobe.

如想覆ç›--定义:If you want to override:

\begin{Shaded}
\begin{Highlighting}[]
\NormalTok{\#import "@preview/pointless{-}size:0.1.0": zh as \_zh}

\NormalTok{\#let zh = \_zh.with(overrides: ((7, 5.25pt),))}

\NormalTok{\#assert.eq(\_zh(7), 5.5pt)}
\NormalTok{\#assert.eq(zh(7), 5.25pt)}
\end{Highlighting}
\end{Shaded}

\subsection{相å\ldots³é``¾æŽ¥ Relevant
links}\label{uxe7uxe5uxb3uxe9uxbeuxe6ux17e-relevant-links}

\begin{quote}
{[}!TIP{]}

\begin{itemize}
\tightlist
\item
  âœ\ldots{} = 一致 consistent
\item
  ðŸ`ª = 与æ??述的规则之一一致 consistent with one of the
  described rules
\item
  🚸 = ä¸?完å\ldots¨ä¸€è‡´ not fully consistent
\end{itemize}
\end{quote}

\begin{itemize}
\item
  🚸
  \href{https://www.w3.org/International/clreq/\#considerations_in_designing_type_area}{§2.3.5
  基本版å¼?设计的注æ„?事项 - 中æ--‡æŽ'版需求 \textbar{} W3C
  ç¼--è¾`è?‰ç¨¿} (中/英){[}2024-09-13{]}

  \begin{quote}
  “å?·â€?ç''±äºŽå½``å¹´é‡`属活å­---å?„地厂家的规范ä¸?一而ä¸?尽相å?Œâ€¦â€¦ä¸?作为规范性规定。
  \end{quote}

  §2.3.5 Considerations when Designing the Type Area - Requirements for
  Chinese Text Layout \textbar{} W3C Editor’s Draft (Chinese \&
  English)

  \begin{quote}
  These hÃ~o-systems were not standardized by the various foundries in
  the past. …It is not normative information.
  \end{quote}
\item
  âœ\ldots 表25 中æ--‡å­---å?· -
  \href{http://mirrors.ctan.org/language/chinese/ctex/ctex.pdf}{CTeX
  v2.5.0 (2022-07-14) å®?集手册 \textbar{} CTAN} (中æ--‡ï¼‰

  Table 25 Chinese text size - Documentation of the package CTeX v2.5.10
  (2022-07-14) (Chinese)

  \url{https://github.com/CTeX-org/ctex-kit/blob/0fb196c42c56287403fecca6eb6b137c00167f40/ctex/ctex.dtx\#L9974-L9993}
\item
  ðŸ`ª
  \href{https://ccjktype.fonts.adobe.com/2009/04/post_1.html}{å­---ä½``度é‡?å?•ä½?
  - CJK Type Blog \textbar{} Adobe} (中/英){[}2009-04-02{]} (
  \href{https://archive.today/QxXuk}{archive.today} )

  Type-related Measurements Units (Chinese \& English)
\item
  âœ
  \href{https://www.wps.cn/learning/question/detail/id/2940}{如何转æ?¢å­---å?·ã€?ç£\ldots ã€?px?-
  技巧é---®ç­'' \textbar{} WPSå­¦å~‚} (中æ--‡ï¼‰{[}2020-05-07{]}

  How to convert between hÃ~o, point, and pixel? - Tech Q\&A \textbar{}
  WPS learning (Chinese)
\item
  \href{https://www.thetype.com/typechat/ep-135/}{\#135
  显明解行å?·å?·ç?? - å­---è°ˆå­---ç•\ldots{} \textbar{} The Type}
  (中æ--‡ï¼Œå¸¦æ--‡å­---说明的æ'­å®¢ï¼‰{[}2020-09-09{]} (
  \href{https://archive.today/qaG8D}{archive.today} )

  (Chinese, podcast with show notes)
\item
  \href{https://github.com/CTeX-org/ctex-kit/issues/543}{\#543 ctexsize:
  é‡?设å?„级å­---å?·å¤§å°? - CTeX-org/ctex-kit \textbar{} GitHub}
  (中æ--‡ï¼‰{[}2020-10-13{]}

  \#543 ctextsize: Redesign the font size system (Chinese)
\item
  âœ
  \href{https://std.samr.gov.cn/gb/search/gbDetailed?id=BBE32B661B7E8FC8E05397BE0A0AB906}{GB
  40070â€``2021 å„¿ç«¥é?'å°`å¹´å­¦ä¹~ç''¨å``?è¿`视防控å?«ç''Ÿè¦?求 -
  国家æ~‡å‡† \textbar{}
  å\ldots¨å›½æ~‡å‡†ä¿¡æ?¯å\ldots¬å\ldots±æœ?务平å?°}
  (中æ--‡ï¼‰{[}2021-02-20{]}

  å\ldots¶ä¸­ç''¨åˆ°äº†å?·æ•°åˆ¶ï¼Œä¾‹å¦‚4.3.1“å°?学一ã€?二年级ç''¨å­---åº''ä¸?å°?于16P(3å?·ï¼‰å­---â€?。总ç»``下æ?¥æ˜¯ä¸‰å?·
  16ptã€?å››å?· 14ptã€?å°?å›› 12ptã€?äº''å?· 10.5ptã€?å°?äº'' 9pt。

  GB 40070â€``2021 Hygienic requirements of study products for myopia
  prevention and control in children and adolescents - National
  standards \textbar{} National public service platform for standards
  information (Chinese)

  The standard uses hÃ~o-system, e.g. 4.3.1 “texts for grade 1/2 of
  primary school should not be less than 16P (size 3)�. To summarize,
  size 3 = 16pt, size 4 = 14pt, size small 4 = 12pt, size 5 = 10.5pt,
  size small 5 = 9pt.
\item
  ðŸ`ª
  \href{https://zh.wikipedia.org/wiki/\%E5\%AD\%97\%E5\%8F\%B7_(\%E5\%8D\%B0\%E5\%88\%B7)}{å­---å?·ï¼ˆå?°åˆ·ï¼‰\textbar{}
  维基百ç§`} (中æ--‡ï¼‰

  HÃ~o (typography) \textbar{} Wikipedia (Chinese)
\end{itemize}

\subsubsection{How to add}\label{how-to-add}

Copy this into your project and use the import as
\texttt{\ pointless-size\ }

\begin{verbatim}
#import "@preview/pointless-size:0.1.0"
\end{verbatim}

\includesvg[width=0.16667in,height=0.16667in]{/assets/icons/16-copy.svg}

Check the docs for
\href{https://typst.app/docs/reference/scripting/\#packages}{more
information on how to import packages} .

\subsubsection{About}\label{about}

\begin{description}
\tightlist
\item[Author :]
Y.D.X.
\item[License:]
MIT
\item[Current version:]
0.1.0
\item[Last updated:]
September 14, 2024
\item[First released:]
September 14, 2024
\item[Archive size:]
3.72 kB
\href{https://packages.typst.org/preview/pointless-size-0.1.0.tar.gz}{\pandocbounded{\includesvg[keepaspectratio]{/assets/icons/16-download.svg}}}
\item[Repository:]
\href{https://github.com/YDX-2147483647/typst-pointless-size}{GitHub}
\item[Categor ies :]
\begin{itemize}
\tightlist
\item[]
\item
  \pandocbounded{\includesvg[keepaspectratio]{/assets/icons/16-text.svg}}
  \href{https://typst.app/universe/search/?category=text}{Text}
\item
  \pandocbounded{\includesvg[keepaspectratio]{/assets/icons/16-world.svg}}
  \href{https://typst.app/universe/search/?category=languages}{Languages}
\end{itemize}
\end{description}

\subsubsection{Where to report issues?}\label{where-to-report-issues}

This package is a project of Y.D.X. . Report issues on
\href{https://github.com/YDX-2147483647/typst-pointless-size}{their
repository} . You can also try to ask for help with this package on the
\href{https://forum.typst.app}{Forum} .

Please report this package to the Typst team using the
\href{https://typst.app/contact}{contact form} if you believe it is a
safety hazard or infringes upon your rights.

\phantomsection\label{versions}
\subsubsection{Version history}\label{version-history}

\begin{longtable}[]{@{}ll@{}}
\toprule\noalign{}
Version & Release Date \\
\midrule\noalign{}
\endhead
\bottomrule\noalign{}
\endlastfoot
0.1.0 & September 14, 2024 \\
\end{longtable}

Typst GmbH did not create this package and cannot guarantee correct
functionality of this package or compatibility with any version of the
Typst compiler or app.


\title{typst.app/universe/package/red-agora}

\phantomsection\label{banner}
\phantomsection\label{template-thumbnail}
\pandocbounded{\includegraphics[keepaspectratio]{https://packages.typst.org/preview/thumbnails/red-agora-0.1.1-small.webp}}

\section{red-agora}\label{red-agora}

{ 0.1.1 }

Quickly scaffold a report for your projects and internships at ENSIAS
and elsewhere

{ } Featured Template

\href{/app?template=red-agora&version=0.1.1}{Create project in app}

\phantomsection\label{readme}
A Typst template to quickly make reports for projects at ENSIAS. This
template was created based on our reports that we also made for our
projects.

\subsection{What does it provide?}\label{what-does-it-provide}

For now, it provides a first page style that matches the common reports
style used and encouraged at ENSIAS.

It also provides a style for first level headings to act as chapters.

More improvements will come soon.

\subsection{Usage}\label{usage}

\begin{Shaded}
\begin{Highlighting}[]
\NormalTok{\#import "@preview/red{-}agora:0.1.1": project}

\NormalTok{\#show: project.with(}
\NormalTok{  title: "Injecting a backdoor in the xz library and taking over NASA and SpaceX spaceship tracking servers (for education purposes only)",}
\NormalTok{  subtitle: "Second year internship report",}
\NormalTok{  authors: (}
\NormalTok{    "Amine Hadnane",}
\NormalTok{    "Mehdi Essalehi"}
\NormalTok{  ),}
\NormalTok{  school{-}logo: image("images/ENSIAS.svg"), // Replace with [] to remove the school logo}
\NormalTok{  company{-}logo: image("images/company.svg"),}
\NormalTok{  mentors: (}
\NormalTok{    "Pr. John Smith (Internal)",}
\NormalTok{    "Jane Doe (External)"}
\NormalTok{  ),}
\NormalTok{  jury: (}
\NormalTok{    "Pr. John Smith",}
\NormalTok{    "Pr. Jane Doe"}
\NormalTok{  ),}
\NormalTok{  branch: "Software Engineering",}
\NormalTok{  academic{-}year: "2077{-}2078",}
\NormalTok{  french: false // Use french instead of english}
\NormalTok{  footer{-}text: "ENSIAS" // Text used in left side of the footer}
\NormalTok{)}

\NormalTok{// Put then your content here}
\end{Highlighting}
\end{Shaded}

\subsection{Changelog}\label{changelog}

\textbf{0.1.0 - Initial release}

\begin{itemize}
\tightlist
\item
  First page style
\item
  Level 1 headings chapter style
\end{itemize}

\textbf{0.1.1}

\begin{itemize}
\tightlist
\item
  Fixed major issue where custom school \& company logos would throw an
  error
\item
  Added option to customize footer left side text (thus fixing the issue
  of it being hardcoded)
\end{itemize}

\href{/app?template=red-agora&version=0.1.1}{Create project in app}

\subsubsection{How to use}\label{how-to-use}

Click the button above to create a new project using this template in
the Typst app.

You can also use the Typst CLI to start a new project on your computer
using this command:

\begin{verbatim}
typst init @preview/red-agora:0.1.1
\end{verbatim}

\includesvg[width=0.16667in,height=0.16667in]{/assets/icons/16-copy.svg}

\subsubsection{About}\label{about}

\begin{description}
\tightlist
\item[Author s :]
\href{https://github.com/essmehdi}{Mehdi Essalehi} \&
\href{https://github.com/amin-hdn}{Amine Hadnane}
\item[License:]
MIT
\item[Current version:]
0.1.1
\item[Last updated:]
April 29, 2024
\item[First released:]
April 19, 2024
\item[Archive size:]
4.37 kB
\href{https://packages.typst.org/preview/red-agora-0.1.1.tar.gz}{\pandocbounded{\includesvg[keepaspectratio]{/assets/icons/16-download.svg}}}
\item[Repository:]
\href{https://github.com/essmehdi/ensias-report-template}{GitHub}
\item[Categor y :]
\begin{itemize}
\tightlist
\item[]
\item
  \pandocbounded{\includesvg[keepaspectratio]{/assets/icons/16-speak.svg}}
  \href{https://typst.app/universe/search/?category=report}{Report}
\end{itemize}
\end{description}

\subsubsection{Where to report issues?}\label{where-to-report-issues}

This template is a project of Mehdi Essalehi and Amine Hadnane . Report
issues on
\href{https://github.com/essmehdi/ensias-report-template}{their
repository} . You can also try to ask for help with this template on the
\href{https://forum.typst.app}{Forum} .

Please report this template to the Typst team using the
\href{https://typst.app/contact}{contact form} if you believe it is a
safety hazard or infringes upon your rights.

\phantomsection\label{versions}
\subsubsection{Version history}\label{version-history}

\begin{longtable}[]{@{}ll@{}}
\toprule\noalign{}
Version & Release Date \\
\midrule\noalign{}
\endhead
\bottomrule\noalign{}
\endlastfoot
0.1.1 & April 29, 2024 \\
\href{https://typst.app/universe/package/red-agora/0.1.0/}{0.1.0} &
April 19, 2024 \\
\end{longtable}

Typst GmbH did not create this template and cannot guarantee correct
functionality of this template or compatibility with any version of the
Typst compiler or app.


\title{typst.app/universe/package/one-liner}

\phantomsection\label{banner}
\section{one-liner}\label{one-liner}

{ 0.1.0 }

Automatically adjust the text size to make it fit on one line filling
the available space.

\phantomsection\label{readme}
One-liner is a package containing a helper function to fit text to the
available width, without wrapping, by adjusting the text size based upon
the context. This is useful in templates where you don’t know the
length of text that is supposed to fit in specific locations in your
template.

\subsection{Example}\label{example}

In the current version(0.1.0) one-liner contains 1 function:
fit-to-width that can used as follows:

\begin{Shaded}
\begin{Highlighting}[]
\NormalTok{\#import "@preview/one{-}liner:0.1.0": fit{-}to{-}width }

\NormalTok{\#block(}
\NormalTok{  height: 3cm,}
\NormalTok{  width: 10cm,}
\NormalTok{  fill: luma(230),}
\NormalTok{  inset: 8pt,}
\NormalTok{  radius: 4pt,}
\NormalTok{  align(horizon + center,fit{-}to{-}width(lorem(2))),}
\NormalTok{)}
\end{Highlighting}
\end{Shaded}

Here we have a block of specific dimensions. Using fit-to-width will
change the font-size to the content passed to fit-to-width will fit the
full width without wrapping the content.

\pandocbounded{\includegraphics[keepaspectratio]{https://github.com/typst/packages/raw/main/packages/preview/one-liner/0.1.0/img/example1.png}}

\subsection{fit-to-width function}\label{fit-to-width-function}

Besides content the function has two parameters:

\emph{max-text-size} of type length. It has a default of 64pt. When
fit-to-width is limited by the max-text-size you will see that not the
entire width of space is used.

\emph{min-text-size} of type length. It has a default of 4pt. When
fit-to-width is limited by the min-text-size you will see that the text
will wrap, because the min-text-size is bigger than the size that would
be required to prevent wrapping.

\subsection{Disclaimer}\label{disclaimer}

This package was only tested in a few of my own documents and only to
fit text. Not tested with other content yet.

\subsubsection{How to add}\label{how-to-add}

Copy this into your project and use the import as \texttt{\ one-liner\ }

\begin{verbatim}
#import "@preview/one-liner:0.1.0"
\end{verbatim}

\includesvg[width=0.16667in,height=0.16667in]{/assets/icons/16-copy.svg}

Check the docs for
\href{https://typst.app/docs/reference/scripting/\#packages}{more
information on how to import packages} .

\subsubsection{About}\label{about}

\begin{description}
\tightlist
\item[Author :]
\href{https://github.com/mtolk}{Marco}
\item[License:]
MIT
\item[Current version:]
0.1.0
\item[Last updated:]
November 12, 2024
\item[First released:]
November 12, 2024
\item[Minimum Typst version:]
0.12.0
\item[Archive size:]
2.00 kB
\href{https://packages.typst.org/preview/one-liner-0.1.0.tar.gz}{\pandocbounded{\includesvg[keepaspectratio]{/assets/icons/16-download.svg}}}
\item[Repository:]
\href{https://github.com/mtolk/one-liner}{GitHub}
\item[Categor ies :]
\begin{itemize}
\tightlist
\item[]
\item
  \pandocbounded{\includesvg[keepaspectratio]{/assets/icons/16-layout.svg}}
  \href{https://typst.app/universe/search/?category=layout}{Layout}
\item
  \pandocbounded{\includesvg[keepaspectratio]{/assets/icons/16-text.svg}}
  \href{https://typst.app/universe/search/?category=text}{Text}
\item
  \pandocbounded{\includesvg[keepaspectratio]{/assets/icons/16-hammer.svg}}
  \href{https://typst.app/universe/search/?category=utility}{Utility}
\end{itemize}
\end{description}

\subsubsection{Where to report issues?}\label{where-to-report-issues}

This package is a project of Marco . Report issues on
\href{https://github.com/mtolk/one-liner}{their repository} . You can
also try to ask for help with this package on the
\href{https://forum.typst.app}{Forum} .

Please report this package to the Typst team using the
\href{https://typst.app/contact}{contact form} if you believe it is a
safety hazard or infringes upon your rights.

\phantomsection\label{versions}
\subsubsection{Version history}\label{version-history}

\begin{longtable}[]{@{}ll@{}}
\toprule\noalign{}
Version & Release Date \\
\midrule\noalign{}
\endhead
\bottomrule\noalign{}
\endlastfoot
0.1.0 & November 12, 2024 \\
\end{longtable}

Typst GmbH did not create this package and cannot guarantee correct
functionality of this package or compatibility with any version of the
Typst compiler or app.


\title{typst.app/universe/package/efilrst}

\phantomsection\label{banner}
\section{efilrst}\label{efilrst}

{ 0.3.0 }

A simple referenceable list library for typst.

\phantomsection\label{readme}
A simple referenceable list library for Typst. If you ever wanted to
reference elements in a list by a key, this library is for you. The name
comes from “reflist� but sorted alphabetically because we are not
allowed to use descriptive names for packages in Typst
🤷��♂�.

\subsection{Example}\label{example}

\begin{Shaded}
\begin{Highlighting}[]
\NormalTok{\#import "@preview/efilrst:0.1.0" as efilrst}
\NormalTok{\#show ref: efilrst.show{-}rule}

\NormalTok{\#let constraint = efilrst.reflist.with(}
\NormalTok{  name: "Constraint", }
\NormalTok{  list{-}style: "C1.1.1)", }
\NormalTok{  ref{-}style: "C1.1.1")}

\NormalTok{\#constraint(}
\NormalTok{  counter{-}name: "continuable",}
\NormalTok{  [My cool constraint A],\textless{}c:a\textgreater{},}
\NormalTok{  [My also cool constraint B],\textless{}c:b\textgreater{},}
\NormalTok{  [My non{-}referenceable constraint C],}
\NormalTok{)}

\NormalTok{See how my @c:a is better than @c:b but not as cool as @c:e.}

\NormalTok{\#constraint(}
\NormalTok{  counter{-}name: "continuable",}
\NormalTok{  [We continue the list with D],\textless{}c:d\textgreater{},}
\NormalTok{  [And then add constraint E],\textless{}c:e\textgreater{},}
\NormalTok{)}

\NormalTok{\#constraint(}
\NormalTok{  [This is a new list!],\textless{}c:f\textgreater{},}
\NormalTok{  (}
\NormalTok{    [And it has a sublist!],\textless{}c:g\textgreater{},}
\NormalTok{    [With a constraint H],\textless{}c:h\textgreater{},}
\NormalTok{  )}
\NormalTok{)}

\NormalTok{\#constraint(}
\NormalTok{  [This is another list!],\textless{}c:i\textgreater{},}
\NormalTok{)}
\end{Highlighting}
\end{Shaded}

This generates the following output:

\pandocbounded{\includegraphics[keepaspectratio]{https://github.com/typst/packages/raw/main/packages/preview/efilrst/0.3.0/img/image.png}}

\subsection{License}\label{license}

This project is licensed under the MIT License - see the
\href{https://github.com/typst/packages/raw/main/packages/preview/efilrst/0.3.0/LICENSE}{LICENSE}
file for details.

\subsection{TODO}\label{todo}

\begin{itemize}
\tightlist
\item
  {[}x{]} Add continuation of lists through the \texttt{\ counter\ }
  function
\item
  {[}x{]} Add support for nested lists
\end{itemize}

\subsection{Changelog}\label{changelog}

\subsubsection{0.1.0}\label{section}

\begin{itemize}
\tightlist
\item
  Initial release
\end{itemize}

\subsubsection{0.2.0}\label{section-1}

\begin{itemize}
\tightlist
\item
  Add continuation of lists through the \texttt{\ counter\ } function
\end{itemize}

\subsubsection{0.3.0}\label{section-2}

\begin{itemize}
\tightlist
\item
  Add support for nested lists
\end{itemize}

\subsubsection{How to add}\label{how-to-add}

Copy this into your project and use the import as \texttt{\ efilrst\ }

\begin{verbatim}
#import "@preview/efilrst:0.3.0"
\end{verbatim}

\includesvg[width=0.16667in,height=0.16667in]{/assets/icons/16-copy.svg}

Check the docs for
\href{https://typst.app/docs/reference/scripting/\#packages}{more
information on how to import packages} .

\subsubsection{About}\label{about}

\begin{description}
\tightlist
\item[Author :]
\href{https://github.com/jmigual}{Joan Marcè i Igual}
\item[License:]
MIT
\item[Current version:]
0.3.0
\item[Last updated:]
November 25, 2024
\item[First released:]
August 27, 2024
\item[Minimum Typst version:]
0.12.0
\item[Archive size:]
2.77 kB
\href{https://packages.typst.org/preview/efilrst-0.3.0.tar.gz}{\pandocbounded{\includesvg[keepaspectratio]{/assets/icons/16-download.svg}}}
\item[Repository:]
\href{https://github.com/jmigual/typst-efilrst}{GitHub}
\end{description}

\subsubsection{Where to report issues?}\label{where-to-report-issues}

This package is a project of Joan Marcè i Igual . Report issues on
\href{https://github.com/jmigual/typst-efilrst}{their repository} . You
can also try to ask for help with this package on the
\href{https://forum.typst.app}{Forum} .

Please report this package to the Typst team using the
\href{https://typst.app/contact}{contact form} if you believe it is a
safety hazard or infringes upon your rights.

\phantomsection\label{versions}
\subsubsection{Version history}\label{version-history}

\begin{longtable}[]{@{}ll@{}}
\toprule\noalign{}
Version & Release Date \\
\midrule\noalign{}
\endhead
\bottomrule\noalign{}
\endlastfoot
0.3.0 & November 25, 2024 \\
\href{https://typst.app/universe/package/efilrst/0.2.0/}{0.2.0} &
November 12, 2024 \\
\href{https://typst.app/universe/package/efilrst/0.1.0/}{0.1.0} & August
27, 2024 \\
\end{longtable}

Typst GmbH did not create this package and cannot guarantee correct
functionality of this package or compatibility with any version of the
Typst compiler or app.


\title{typst.app/universe/package/easytable}

\phantomsection\label{banner}
\section{easytable}\label{easytable}

{ 0.1.0 }

Simple Table Package

\phantomsection\label{readme}
A Typst library for writing simple tables.

\subsection{Usage}\label{usage}

\begin{Shaded}
\begin{Highlighting}[]
\NormalTok{\#import "@preview/easytable:0.1.0": easytable, elem}
\NormalTok{\#import elem: *}
\end{Highlighting}
\end{Shaded}

\subsection{Manual}\label{manual}

\begin{itemize}
\tightlist
\item
  You can create a table by specifying data or layout elements as
  arguments to the \texttt{\ easytable\ } function.
\item
  The following elements are provided in the \texttt{\ elem\ } module.

  \begin{itemize}
  \tightlist
  \item
    \texttt{\ elem.tr\ } : a data row
  \item
    \texttt{\ elem.th\ } : a header row
  \item
    \texttt{\ elem.hline\ } : a horizontal line
  \item
    \texttt{\ elem.vline\ } : a vertical line
  \item
    \texttt{\ elem.cwidth\ } : a column-width specifier
  \item
    \texttt{\ elem.cstyle\ } : a column-style (font, alignment, etc.)
    specifier
  \end{itemize}
\end{itemize}

See
\href{https://github.com/typst/packages/raw/main/packages/preview/easytable/0.1.0/manual.pdf}{manual}
in detail.

\subsection{Examples}\label{examples}

\subsubsection{A Simple Table}\label{a-simple-table}

\begin{Shaded}
\begin{Highlighting}[]
\NormalTok{\#easytable(\{}
\NormalTok{  th[Header 1 ][Header 2][Header 3  ]}
\NormalTok{  tr[How      ][I       ][want      ]}
\NormalTok{  tr[a        ][drink,  ][alcoholic ]}
\NormalTok{  tr[of       ][course, ][after     ]}
\NormalTok{  tr[the      ][heavy   ][lectures  ]}
\NormalTok{  tr[involving][quantum ][mechanics.]}
\NormalTok{\})}
\end{Highlighting}
\end{Shaded}

\pandocbounded{\includegraphics[keepaspectratio]{https://github.com/monaqa/typst-easytable/assets/48883418/690b466b-56d9-4660-8ca5-25cc25e379f9}}

\subsubsection{Setting Column Alignment and
Width}\label{setting-column-alignment-and-width}

\begin{Shaded}
\begin{Highlighting}[]
\NormalTok{\#easytable(\{}
\NormalTok{  cwidth(100pt, 1fr, 20\%)}
\NormalTok{  cstyle(left, center, right)}
\NormalTok{  th[Header 1 ][Header 2][Header 3  ]}
\NormalTok{  tr[How      ][I       ][want      ]}
\NormalTok{  tr[a        ][drink,  ][alcoholic ]}
\NormalTok{  tr[of       ][course, ][after     ]}
\NormalTok{  tr[the      ][heavy   ][lectures  ]}
\NormalTok{  tr[involving][quantum ][mechanics.]}
\NormalTok{\})}
\end{Highlighting}
\end{Shaded}

\pandocbounded{\includegraphics[keepaspectratio]{https://github.com/monaqa/typst-easytable/assets/48883418/8ff574b4-bf1f-46ca-8a2d-584ab701a989}}

\subsubsection{Customizing Styles}\label{customizing-styles}

\begin{Shaded}
\begin{Highlighting}[]
\NormalTok{\#easytable(\{}
\NormalTok{  let tr = tr.with(trans: pad.with(x: 3pt))}

\NormalTok{  th[Header 1][Header 2][Header 3]}
\NormalTok{  tr[How][I][want]}
\NormalTok{  tr[a][drink,][alcoholic]}
\NormalTok{  tr[of][course,][after]}
\NormalTok{  tr[the][heavy][lectures]}
\NormalTok{  tr[involving][quantum][mechanics.]}
\NormalTok{\})}
\end{Highlighting}
\end{Shaded}

\pandocbounded{\includegraphics[keepaspectratio]{https://github.com/monaqa/typst-easytable/assets/48883418/8a1ed0d0-4a9e-4a28-a0ff-b8f7a09cb8a8}}

\begin{Shaded}
\begin{Highlighting}[]
\NormalTok{\#easytable(\{}
\NormalTok{  let th = th.with(trans: emph)}
\NormalTok{  let tr = tr.with(}
\NormalTok{    cell\_style: (x: none, y: none)}
\NormalTok{      =\textgreater{} (fill: if calc.even(y) \{}
\NormalTok{        luma(95\%)}
\NormalTok{      \} else \{}
\NormalTok{        none}
\NormalTok{      \})}
\NormalTok{  )}

\NormalTok{  th[Header 1][Header 2][Header 3]}
\NormalTok{  tr[How][I][want]}
\NormalTok{  tr[a][drink,][alcoholic]}
\NormalTok{  tr[of][course,][after]}
\NormalTok{  tr[the][heavy][lectures]}
\NormalTok{  tr[involving][quantum][mechanics.]}
\NormalTok{\})}
\end{Highlighting}
\end{Shaded}

\pandocbounded{\includegraphics[keepaspectratio]{https://github.com/monaqa/typst-easytable/assets/48883418/5f8bf796-b2bd-41c4-a79e-fd97c2824ecd}}

\begin{Shaded}
\begin{Highlighting}[]
\NormalTok{\#easytable(\{}
\NormalTok{  th[Header 1][Header 2][Header 3]}
\NormalTok{  tr[How][I][want]}
\NormalTok{  hline(stroke: red)}
\NormalTok{  tr[a][drink,][alcoholic]}
\NormalTok{  tr[of][course,][after]}
\NormalTok{  tr[the][heavy][lectures]}
\NormalTok{  tr[involving][quantum][mechanics.]}

\NormalTok{  // Specifying the insertion point directly}
\NormalTok{  hline(stroke: 2pt + green, y: 4)}
\NormalTok{  vline(}
\NormalTok{    stroke: (paint: blue, thickness: 1pt, dash: "dashed"),}
\NormalTok{    x: 2,}
\NormalTok{    start: 1,}
\NormalTok{    end: 5,}
\NormalTok{  )}
\NormalTok{\})}
\end{Highlighting}
\end{Shaded}

\pandocbounded{\includegraphics[keepaspectratio]{https://github.com/monaqa/typst-easytable/assets/48883418/cf400dad-a7fc-4f3a-991d-9611adab41c6}}

\subsubsection{How to add}\label{how-to-add}

Copy this into your project and use the import as \texttt{\ easytable\ }

\begin{verbatim}
#import "@preview/easytable:0.1.0"
\end{verbatim}

\includesvg[width=0.16667in,height=0.16667in]{/assets/icons/16-copy.svg}

Check the docs for
\href{https://typst.app/docs/reference/scripting/\#packages}{more
information on how to import packages} .

\subsubsection{About}\label{about}

\begin{description}
\tightlist
\item[Author :]
monaqa
\item[License:]
MIT
\item[Current version:]
0.1.0
\item[Last updated:]
February 24, 2024
\item[First released:]
February 24, 2024
\item[Archive size:]
3.36 kB
\href{https://packages.typst.org/preview/easytable-0.1.0.tar.gz}{\pandocbounded{\includesvg[keepaspectratio]{/assets/icons/16-download.svg}}}
\item[Repository:]
\href{https://github.com/monaqa/typst-easytable}{GitHub}
\end{description}

\subsubsection{Where to report issues?}\label{where-to-report-issues}

This package is a project of monaqa . Report issues on
\href{https://github.com/monaqa/typst-easytable}{their repository} . You
can also try to ask for help with this package on the
\href{https://forum.typst.app}{Forum} .

Please report this package to the Typst team using the
\href{https://typst.app/contact}{contact form} if you believe it is a
safety hazard or infringes upon your rights.

\phantomsection\label{versions}
\subsubsection{Version history}\label{version-history}

\begin{longtable}[]{@{}ll@{}}
\toprule\noalign{}
Version & Release Date \\
\midrule\noalign{}
\endhead
\bottomrule\noalign{}
\endlastfoot
0.1.0 & February 24, 2024 \\
\end{longtable}

Typst GmbH did not create this package and cannot guarantee correct
functionality of this package or compatibility with any version of the
Typst compiler or app.


\title{typst.app/universe/package/vantage-cv}

\phantomsection\label{banner}
\phantomsection\label{template-thumbnail}
\pandocbounded{\includegraphics[keepaspectratio]{https://packages.typst.org/preview/thumbnails/vantage-cv-1.0.0-small.webp}}

\section{vantage-cv}\label{vantage-cv}

{ 1.0.0 }

An ATS friendly simple Typst CV template

\href{/app?template=vantage-cv&version=1.0.0}{Create project in app}

\phantomsection\label{readme}
An ATS friendly simple Typst CV template, inspired by
\href{https://github.com/GeorgeHoneywood/alta-typst}{alta-typst by
George Honeywood} .

\subsection{Features}\label{features}

\begin{itemize}
\tightlist
\item
  \textbf{Two-column layout} : Experience on the left and other
  important details on the right, organized for easy scanning.
\item
  \textbf{Customizable icons} : Add and replace icons to suit your
  personal style.
\item
  \textbf{Responsive design} : Adjusts well for various print formats.
\end{itemize}

\subsection{Usage}\label{usage}

\subsubsection{Running Locally with Typst
CLI}\label{running-locally-with-typst-cli}

\begin{enumerate}
\item
  \textbf{Install Typst CLI} : Follow the installation instructions on
  the \href{https://github.com/typst/typst\#installation}{Typst CLI
  GitHub page} to set up Typst on your machine.
\item
  \textbf{Clone the repository} :

\begin{Shaded}
\begin{Highlighting}[]
\FunctionTok{git}\NormalTok{ clone https://github.com/sardorml/vantage{-}typst.git}
\BuiltInTok{cd}\NormalTok{ vantage{-}typst}
\end{Highlighting}
\end{Shaded}
\item
  \textbf{Run Typst} :

  Use the following command to render your CV:

\begin{Shaded}
\begin{Highlighting}[]
\ExtensionTok{typst}\NormalTok{ compile example.typ}
\end{Highlighting}
\end{Shaded}

  This will generate a PDF output in the same directory.
\item
  \textbf{Edit your CV} :

  Open the \texttt{\ example.typ\ } file in your preferred text editor
  to customize the layout.
\end{enumerate}

\subsubsection{Configuration}\label{configuration}

You can easily customize your personal data by editing the
\texttt{\ configuration.yaml\ } file. This file allows you to set your
name, contact information, work experience, education, and skills.
Here’s how to do it:

\begin{enumerate}
\tightlist
\item
  Open the \texttt{\ configuration.yaml\ } file in your text editor.
\item
  Update the fields with your personal information.
\item
  Save the file, and your CV will automatically reflect these changes
  when you compile it.
\end{enumerate}

\subsection{Icons}\label{icons}

You can enhance your CV with additional icons by following these steps:

\begin{enumerate}
\item
  \textbf{Upload Icons} : Place your \texttt{\ .svg\ } files in the
  \texttt{\ icons/\ } folder.
\item
  \textbf{Reference Icons} : Modify the \texttt{\ links\ } array in the
  Typst file to include your new icons by referencing their filenames as
  the \texttt{\ name\ } values.

  Example:

\begin{Shaded}
\begin{Highlighting}[]
\NormalTok{links: [}
\NormalTok{  \{ name: "your{-}icon{-}name", url: "https://example.com" \},}
\NormalTok{]}
\end{Highlighting}
\end{Shaded}
\end{enumerate}

For existing icons, the current selection is sourced from
\href{https://lucide.dev/icons/}{Lucide Icons} .

\subsection{License}\label{license}

This project is licensed under the
\href{https://github.com/typst/packages/raw/main/packages/preview/vantage-cv/1.0.0/LICENSE}{MIT
License} .

Icons are from Lucide Icons and are subject to
\href{https://lucide.dev/license}{their terms} .

\subsection{Acknowledgments}\label{acknowledgments}

\begin{itemize}
\tightlist
\item
  Inspired by the work of
  \href{https://github.com/GeorgeHoneywood/alta-typst}{George Honeywood}
  .
\item
  Thanks to \href{https://lucide.dev/icons/}{Lucide Icons} for providing
  the icon library.
\end{itemize}

For any questions or contributions, feel free to open an issue or submit
a pull request!

\href{/app?template=vantage-cv&version=1.0.0}{Create project in app}

\subsubsection{How to use}\label{how-to-use}

Click the button above to create a new project using this template in
the Typst app.

You can also use the Typst CLI to start a new project on your computer
using this command:

\begin{verbatim}
typst init @preview/vantage-cv:1.0.0
\end{verbatim}

\includesvg[width=0.16667in,height=0.16667in]{/assets/icons/16-copy.svg}

\subsubsection{About}\label{about}

\begin{description}
\tightlist
\item[Author :]
\href{https://github.com/sardorml}{Sardor}
\item[License:]
MIT
\item[Current version:]
1.0.0
\item[Last updated:]
October 9, 2024
\item[First released:]
October 9, 2024
\item[Archive size:]
6.59 kB
\href{https://packages.typst.org/preview/vantage-cv-1.0.0.tar.gz}{\pandocbounded{\includesvg[keepaspectratio]{/assets/icons/16-download.svg}}}
\item[Repository:]
\href{https://github.com/sardorml/vantage-typst}{GitHub}
\item[Categor ies :]
\begin{itemize}
\tightlist
\item[]
\item
  \pandocbounded{\includesvg[keepaspectratio]{/assets/icons/16-user.svg}}
  \href{https://typst.app/universe/search/?category=cv}{CV}
\item
  \pandocbounded{\includesvg[keepaspectratio]{/assets/icons/16-layout.svg}}
  \href{https://typst.app/universe/search/?category=layout}{Layout}
\end{itemize}
\end{description}

\subsubsection{Where to report issues?}\label{where-to-report-issues}

This template is a project of Sardor . Report issues on
\href{https://github.com/sardorml/vantage-typst}{their repository} . You
can also try to ask for help with this template on the
\href{https://forum.typst.app}{Forum} .

Please report this template to the Typst team using the
\href{https://typst.app/contact}{contact form} if you believe it is a
safety hazard or infringes upon your rights.

\phantomsection\label{versions}
\subsubsection{Version history}\label{version-history}

\begin{longtable}[]{@{}ll@{}}
\toprule\noalign{}
Version & Release Date \\
\midrule\noalign{}
\endhead
\bottomrule\noalign{}
\endlastfoot
1.0.0 & October 9, 2024 \\
\end{longtable}

Typst GmbH did not create this template and cannot guarantee correct
functionality of this template or compatibility with any version of the
Typst compiler or app.


\title{typst.app/universe/package/fervojo}

\phantomsection\label{banner}
\section{fervojo}\label{fervojo}

{ 0.1.0 }

railroad for typst, powered by wasm

\phantomsection\label{readme}
Use \href{https://github.com/lukaslueg/railroad_dsl}{railroads} in your
documents.

You use the function by calling \texttt{\ render(diagram-text,\ css)\ }
which renders the diagram. There, \texttt{\ diagram-text\ } contains is
the diagram itself and css is the one used for the style,
\texttt{\ css\ } is \texttt{\ default-css()\ } if you don’t pass it.
Both fields can be strings, bytes or a raw
\href{https://typst.app/docs/reference/text/raw/}{raw} block.

\subsubsection{How to add}\label{how-to-add}

Copy this into your project and use the import as \texttt{\ fervojo\ }

\begin{verbatim}
#import "@preview/fervojo:0.1.0"
\end{verbatim}

\includesvg[width=0.16667in,height=0.16667in]{/assets/icons/16-copy.svg}

Check the docs for
\href{https://typst.app/docs/reference/scripting/\#packages}{more
information on how to import packages} .

\subsubsection{About}\label{about}

\begin{description}
\tightlist
\item[Author :]
Leiser Fernández Gallo
\item[License:]
MIT
\item[Current version:]
0.1.0
\item[Last updated:]
June 5, 2024
\item[First released:]
June 5, 2024
\item[Archive size:]
44.8 kB
\href{https://packages.typst.org/preview/fervojo-0.1.0.tar.gz}{\pandocbounded{\includesvg[keepaspectratio]{/assets/icons/16-download.svg}}}
\item[Repository:]
\href{https://github.com/leiserfg/fervojo}{GitHub}
\end{description}

\subsubsection{Where to report issues?}\label{where-to-report-issues}

This package is a project of Leiser Fernández Gallo . Report issues on
\href{https://github.com/leiserfg/fervojo}{their repository} . You can
also try to ask for help with this package on the
\href{https://forum.typst.app}{Forum} .

Please report this package to the Typst team using the
\href{https://typst.app/contact}{contact form} if you believe it is a
safety hazard or infringes upon your rights.

\phantomsection\label{versions}
\subsubsection{Version history}\label{version-history}

\begin{longtable}[]{@{}ll@{}}
\toprule\noalign{}
Version & Release Date \\
\midrule\noalign{}
\endhead
\bottomrule\noalign{}
\endlastfoot
0.1.0 & June 5, 2024 \\
\end{longtable}

Typst GmbH did not create this package and cannot guarantee correct
functionality of this package or compatibility with any version of the
Typst compiler or app.


\title{typst.app/universe/package/ez-today}

\phantomsection\label{banner}
\section{ez-today}\label{ez-today}

{ 0.1.0 }

Simply displays the full current date.

\phantomsection\label{readme}
Simply displays the current date with easy to use customization.

\subsection{Included languages}\label{included-languages}

German, English, French and Italian months can be used out of the box.
If you want to use a language that is not included, you can easily add
it yourself. This is shown in the customization section below.

\subsection{Usage}\label{usage}

The usage is very simple, because there is only the \texttt{\ today()\ }
function.

\begin{Shaded}
\begin{Highlighting}[]
\NormalTok{\#import "@preview/ez{-}today:0.1.0"}

\NormalTok{// To get the current date use this}
\NormalTok{\#ez{-}today.today()}
\end{Highlighting}
\end{Shaded}

\subsection{Reference}\label{reference}

\subsubsection{\texorpdfstring{\texttt{\ today\ }}{ today }}\label{today}

Prints the current date with given arguments.

\begin{Shaded}
\begin{Highlighting}[]
\NormalTok{\#let today(}
\NormalTok{  lang: "de",}
\NormalTok{  format: "d. M Y",}
\NormalTok{  custom{-}months: ()}
\NormalTok{) = \{ .. \}}
\end{Highlighting}
\end{Shaded}

\textbf{Arguments:}

\begin{itemize}
\tightlist
\item
  \texttt{\ lang\ } : {[} \texttt{\ str\ } {]} â€'' Select one of the
  included languages (de, en, fr, it).
\item
  \texttt{\ format\ } : {[} \texttt{\ str\ } {]} â€'' Specify the output
  format.
\item
  \texttt{\ custom-months\ } : {[} \texttt{\ array\ } {]} of {[}
  \texttt{\ str\ } {]} â€'' Use custom names for each month. This array
  must have 12 entries. If this is used, the \texttt{\ lang\ } argument
  does nothing.
\end{itemize}

\subsection{Customization}\label{customization}

The default output prints the full current date with German months like
this:

\begin{Shaded}
\begin{Highlighting}[]
\NormalTok{\#ez{-}today.today()   // 11. Oktober 2024}
\end{Highlighting}
\end{Shaded}

You can choose one of the included languages with the \texttt{\ lang\ }
argument:

\begin{Shaded}
\begin{Highlighting}[]
\NormalTok{\#ez{-}today.today(lang: "en")   // 11. October 2024}
\NormalTok{\#ez{-}today.today(lang: "fr")   // 11. Octobre 2024}
\NormalTok{\#ez{-}today.today(lang: "it")   // 11. Ottobre 2024}
\end{Highlighting}
\end{Shaded}

You can also change the format of the output with the
\texttt{\ format\ } argument. Pass any string you want here, but know
that the following characters will be replaced with the following:

\begin{itemize}
\tightlist
\item
  \texttt{\ d\ } â€'' The current day as a decimal
\item
  \texttt{\ m\ } â€'' The current month as a decimal ( \texttt{\ lang\ }
  argument does nothing)
\item
  \texttt{\ M\ } â€'' The current month as a string with either the
  selected language or the custom array
\item
  \texttt{\ y\ } â€'' The current year as a decimal with the last two
  numbers
\item
  \texttt{\ Y\ } â€'' The current year as a decimal
\end{itemize}

Here are some examples:

\begin{Shaded}
\begin{Highlighting}[]
\NormalTok{\#ez{-}today.today(lang: "en", format: "M d Y")    // October 11 2024}
\NormalTok{\#ez{-}today.today(format: "m{-}d{-}y")                // 10{-}11{-}24}
\NormalTok{\#ez{-}today.today(format: "d/m")                  // 11/10}
\NormalTok{\#ez{-}today.today(format: "d.m.Y")                // 11.10.2024}
\end{Highlighting}
\end{Shaded}

Use the \texttt{\ custom-months\ } argument to give each month a custom
name. You can add a new language or use short terms for each month.

\begin{Shaded}
\begin{Highlighting}[]
\NormalTok{// Defining some custom names}
\NormalTok{\#let my{-}months = ("Jan", "Feb", "Mar", "Apr", "May", "Jun", "Jul", "Aug", "Sep", "Oct", "Nov", "Dec")}
\NormalTok{// Get current date with custom names}
\NormalTok{\#ez{-}today.today(custom{-}months: my{-}months, format: "M{-}y")    // Oct{-}24}
\end{Highlighting}
\end{Shaded}

\subsubsection{How to add}\label{how-to-add}

Copy this into your project and use the import as \texttt{\ ez-today\ }

\begin{verbatim}
#import "@preview/ez-today:0.1.0"
\end{verbatim}

\includesvg[width=0.16667in,height=0.16667in]{/assets/icons/16-copy.svg}

Check the docs for
\href{https://typst.app/docs/reference/scripting/\#packages}{more
information on how to import packages} .

\subsubsection{About}\label{about}

\begin{description}
\tightlist
\item[Author :]
Carlo Schafflik
\item[License:]
MIT
\item[Current version:]
0.1.0
\item[Last updated:]
October 11, 2024
\item[First released:]
October 11, 2024
\item[Archive size:]
2.42 kB
\href{https://packages.typst.org/preview/ez-today-0.1.0.tar.gz}{\pandocbounded{\includesvg[keepaspectratio]{/assets/icons/16-download.svg}}}
\item[Repository:]
\href{https://github.com/CarloSchafflik12/typst-ez-today}{GitHub}
\end{description}

\subsubsection{Where to report issues?}\label{where-to-report-issues}

This package is a project of Carlo Schafflik . Report issues on
\href{https://github.com/CarloSchafflik12/typst-ez-today}{their
repository} . You can also try to ask for help with this package on the
\href{https://forum.typst.app}{Forum} .

Please report this package to the Typst team using the
\href{https://typst.app/contact}{contact form} if you believe it is a
safety hazard or infringes upon your rights.

\phantomsection\label{versions}
\subsubsection{Version history}\label{version-history}

\begin{longtable}[]{@{}ll@{}}
\toprule\noalign{}
Version & Release Date \\
\midrule\noalign{}
\endhead
\bottomrule\noalign{}
\endlastfoot
0.1.0 & October 11, 2024 \\
\end{longtable}

Typst GmbH did not create this package and cannot guarantee correct
functionality of this package or compatibility with any version of the
Typst compiler or app.


\title{typst.app/universe/package/use-tabler-icons}

\phantomsection\label{banner}
\section{use-tabler-icons}\label{use-tabler-icons}

{ 0.4.0 }

Tabler Icons for Typst using webfont.

\phantomsection\label{readme}
\begin{quote}
\textbf{Note}

This project is greatly inspired by and mainly edited based on
\href{https://github.com/duskmoon314/typst-fontawesome}{typst-fontawesome}
.
\end{quote}

\subsection{\texorpdfstring{\protect\pandocbounded{\includesvg[keepaspectratio]{https://github.com/typst/packages/raw/main/packages/preview/use-tabler-icons/0.4.0/assets/splash.svg}}}{use-tabler-icons}}\label{use-tabler-icons-1}

A Typst library for \href{https://github.com/tabler/tabler-icons}{Tabler
Icons} , a set of over 5700 free MIT-licensed high-quality SVG icons.

\subsection{Usage}\label{usage}

\subsubsection{Install Font}\label{install-font}

Download \href{https://github.com/tabler/tabler-icons/releases}{latest
release of tabler-icons} and install
\texttt{\ webfont/fonts/tabler-icons.ttf\ } . Or, if you are using Typst
web app, simply upload the font file to your project.

\subsubsection{Import the Library}\label{import-the-library}

\paragraph{Using the Typst Packages}\label{using-the-typst-packages}

You can install the library using the typst packages:

\begin{Shaded}
\begin{Highlighting}[]
\NormalTok{\#import "@preview/use{-}tabler{-}icons:0.4.0": *}
\end{Highlighting}
\end{Shaded}

\paragraph{Manually Install}\label{manually-install}

Just copy all files under
\href{https://github.com/zyf722/typst-tabler-icons/tree/main/src}{\texttt{\ src\ }}
to your project and rename them to avoid naming conflicts.

Then, import \texttt{\ lib.typ\ } to use the library:

\begin{Shaded}
\begin{Highlighting}[]
\NormalTok{\#import "lib.typ": *}
\end{Highlighting}
\end{Shaded}

\subsubsection{Use the Icons}\label{use-the-icons}

You can use the \texttt{\ tabler-icon\ } function to create an icon with
its name:

\begin{Shaded}
\begin{Highlighting}[]
\NormalTok{\#tabler{-}icon("calendar")}
\end{Highlighting}
\end{Shaded}

Or you can directly use the \texttt{\ ti-\ } prefix :

\begin{Shaded}
\begin{Highlighting}[]
\NormalTok{\#ti{-}calendar()}
\end{Highlighting}
\end{Shaded}

As these icons are actually text with custom font, you can pass any text
attributes to the function:

\begin{Shaded}
\begin{Highlighting}[]
\NormalTok{\#tabler{-}icon("calendar", fill: blue)}
\end{Highlighting}
\end{Shaded}

Refer to
\href{https://github.com/zyf722/typst-tabler-icons/tree/main/gallery/gallery.pdf}{\texttt{\ gallery.pdf\ }}
and \href{https://tabler.io/icons}{Tabler Icons website} for all
available icons.

\subsection{Contributing}\label{contributing}

\href{https://github.com/zyf722/typst-tabler-icons/pulls}{Pull Requests}
are welcome!

It is strongly recommended to follow the
\href{https://www.conventionalcommits.org/en/v1.0.0/}{Conventional
Commits} specification when writing commit messages and creating pull
requests.

\subsubsection{Github Actions Workflow}\label{github-actions-workflow}

This package uses a daily run
\href{https://github.com/zyf722/typst-tabler-icons/tree/main/.github/workflows/build.yml}{Github
Actions workflow} to keep the library up-to-date with the latest version
of Tabler Icons, which internally runs
\href{https://github.com/zyf722/typst-tabler-icons/tree/main/scripts/generate.mjs}{\texttt{\ scripts/generate.mjs\ }}
to generate Typst source code of the library and gallery.

\subsection{License}\label{license}

\href{https://github.com/zyf722/typst-tabler-icons/tree/main/LICENSE}{MIT}

\subsubsection{How to add}\label{how-to-add}

Copy this into your project and use the import as
\texttt{\ use-tabler-icons\ }

\begin{verbatim}
#import "@preview/use-tabler-icons:0.4.0"
\end{verbatim}

\includesvg[width=0.16667in,height=0.16667in]{/assets/icons/16-copy.svg}

Check the docs for
\href{https://typst.app/docs/reference/scripting/\#packages}{more
information on how to import packages} .

\subsubsection{About}\label{about}

\begin{description}
\tightlist
\item[Author s :]
\href{mailto:kp.campbell.he@duskmoon314.com}{duskmoon (Campbell He)} \&
\href{mailto:MaxMixAlex@protonmail.com}{MaxMixAlex}
\item[License:]
MIT
\item[Current version:]
0.4.0
\item[Last updated:]
November 13, 2024
\item[First released:]
October 21, 2024
\item[Archive size:]
79.3 kB
\href{https://packages.typst.org/preview/use-tabler-icons-0.4.0.tar.gz}{\pandocbounded{\includesvg[keepaspectratio]{/assets/icons/16-download.svg}}}
\item[Repository:]
\href{https://github.com/zyf722/typst-tabler-icons}{GitHub}
\item[Categor y :]
\begin{itemize}
\tightlist
\item[]
\item
  \pandocbounded{\includesvg[keepaspectratio]{/assets/icons/16-text.svg}}
  \href{https://typst.app/universe/search/?category=text}{Text}
\end{itemize}
\end{description}

\subsubsection{Where to report issues?}\label{where-to-report-issues}

This package is a project of duskmoon (Campbell He) and MaxMixAlex .
Report issues on
\href{https://github.com/zyf722/typst-tabler-icons}{their repository} .
You can also try to ask for help with this package on the
\href{https://forum.typst.app}{Forum} .

Please report this package to the Typst team using the
\href{https://typst.app/contact}{contact form} if you believe it is a
safety hazard or infringes upon your rights.

\phantomsection\label{versions}
\subsubsection{Version history}\label{version-history}

\begin{longtable}[]{@{}ll@{}}
\toprule\noalign{}
Version & Release Date \\
\midrule\noalign{}
\endhead
\bottomrule\noalign{}
\endlastfoot
0.4.0 & November 13, 2024 \\
\href{https://typst.app/universe/package/use-tabler-icons/0.3.0/}{0.3.0}
& October 30, 2024 \\
\href{https://typst.app/universe/package/use-tabler-icons/0.2.0/}{0.2.0}
& October 25, 2024 \\
\href{https://typst.app/universe/package/use-tabler-icons/0.1.0/}{0.1.0}
& October 21, 2024 \\
\end{longtable}

Typst GmbH did not create this package and cannot guarantee correct
functionality of this package or compatibility with any version of the
Typst compiler or app.


\title{typst.app/universe/package/lineal}

\phantomsection\label{banner}
\section{lineal}\label{lineal}

{ 0.1.0 }

Build elegent slide decks with Typst

\phantomsection\label{readme}
IPA: /ˈlɪniəl/

Made up of, or having the characteristic of, lines.

Lineal is a Typst template for generating beautifully clean and
configurably awesome slides.

\pandocbounded{\includegraphics[keepaspectratio]{https://github.com/typst/packages/raw/main/packages/preview/lineal/0.1.0/assets/img/demo.gif}}

\subsection{Philosophy}\label{philosophy}

As a long time user of TeX, I have developed and embedded countless
LaTeX applications into personal and corporate environments, e.g.
automating documentation, conference materials, posters, resumes, etc.

However, LaTeX is showing its age. Compiling a some 30-slide
presentation, for instance, takes perhaps a second, and may requires
multiple renders to sync TikZ positioning and cross-document
referencing. Typst remediates these issues in real-time and with better
control and confidence in its data modelling.

Hence, Lineal was born. A professional set of slides produced near
instantly, readily equipped with configurable theming and a suite of
flexible components.

\subsection{Usage}\label{usage}

Lineal is available through Typst Universe. Ensure you have installed
Typst locally or are familiar with the Typst
\href{https://typst.app/}{web app} or the
\href{https://marketplace.visualstudio.com/items?itemName=myriad-dreamin.tinymist}{Tinymist
LSP} extensions for VS Code.

Get started by importing the package and populating your own
\texttt{\ /content/\textless{}slug\textgreater{}.typ\ } files:

\begin{Shaded}
\begin{Highlighting}[]
\NormalTok{\#import "@preview/lineal:0.1.0": lineal{-}theme}

\NormalTok{\#show: lineal{-}theme.with(}
\NormalTok{  aspect{-}ratio: "16{-}9",}
\NormalTok{  config{-}info(}
\NormalTok{    title: [$bb("L")"ineal"$],}
\NormalTok{    subtitle: [A Typst slide template],}
\NormalTok{    author: [Author],}
\NormalTok{    date: datetime.today(),}
\NormalTok{    institution: [Institution],}
\NormalTok{    logo: [Logo],}
\NormalTok{  ),}
\NormalTok{)}

\NormalTok{\#title{-}slide()}

\NormalTok{\#include "content/index.typ"}
\end{Highlighting}
\end{Shaded}

Marking up content is as you would with any other Typst document, where
the section ( \texttt{\ =\ \textless{}section-title\textgreater{}\ } )
and subsection ( \texttt{\ ==\ \textless{}slide-title\textgreater{}\ } )
shorthands generate the new section slides with dynamic outline and new
tracked slides respectively.

\subsection{Contributing}\label{contributing}

PRs are very welcome. If you think Lineal could be improved in any way
or is missing a feature, please raise a request 😎

\subsection{Acknowledgements}\label{acknowledgements}

A heartfelt thank you to the team behind
\href{https://github.com/typst/typst}{Typst} , developing a product that
not only preserves the beauty of LaTeX’s typesetting, but improves on
its developer experience in every way, in line with ongoing community
feedback.

The creators of the
\href{https://github.com/touying-typ/touying}{\texttt{\ Touying\ }} and
\href{https://github.com/andreasKroepelin/polylux}{\texttt{\ Polylux\ }}
Typst packages, on which Lineal is built.

\subsubsection{How to add}\label{how-to-add}

Copy this into your project and use the import as \texttt{\ lineal\ }

\begin{verbatim}
#import "@preview/lineal:0.1.0"
\end{verbatim}

\includesvg[width=0.16667in,height=0.16667in]{/assets/icons/16-copy.svg}

Check the docs for
\href{https://typst.app/docs/reference/scripting/\#packages}{more
information on how to import packages} .

\subsubsection{About}\label{about}

\begin{description}
\tightlist
\item[Author :]
\href{https://github.com/ellsphillips}{ellsphillips}
\item[License:]
MIT
\item[Current version:]
0.1.0
\item[Last updated:]
November 28, 2024
\item[First released:]
November 28, 2024
\item[Archive size:]
7.40 kB
\href{https://packages.typst.org/preview/lineal-0.1.0.tar.gz}{\pandocbounded{\includesvg[keepaspectratio]{/assets/icons/16-download.svg}}}
\item[Repository:]
\href{https://github.com/ellsphillips/lineal}{GitHub}
\item[Categor y :]
\begin{itemize}
\tightlist
\item[]
\item
  \pandocbounded{\includesvg[keepaspectratio]{/assets/icons/16-presentation.svg}}
  \href{https://typst.app/universe/search/?category=presentation}{Presentation}
\end{itemize}
\end{description}

\subsubsection{Where to report issues?}\label{where-to-report-issues}

This package is a project of ellsphillips . Report issues on
\href{https://github.com/ellsphillips/lineal}{their repository} . You
can also try to ask for help with this package on the
\href{https://forum.typst.app}{Forum} .

Please report this package to the Typst team using the
\href{https://typst.app/contact}{contact form} if you believe it is a
safety hazard or infringes upon your rights.

\phantomsection\label{versions}
\subsubsection{Version history}\label{version-history}

\begin{longtable}[]{@{}ll@{}}
\toprule\noalign{}
Version & Release Date \\
\midrule\noalign{}
\endhead
\bottomrule\noalign{}
\endlastfoot
0.1.0 & November 28, 2024 \\
\end{longtable}

Typst GmbH did not create this package and cannot guarantee correct
functionality of this package or compatibility with any version of the
Typst compiler or app.


\title{typst.app/universe/package/not-jku-thesis}

\phantomsection\label{banner}
\phantomsection\label{template-thumbnail}
\pandocbounded{\includegraphics[keepaspectratio]{https://packages.typst.org/preview/thumbnails/not-jku-thesis-0.1.0-small.webp}}

\section{not-jku-thesis}\label{not-jku-thesis}

{ 0.1.0 }

Customizable not official template for a thesis at the JKU, derived from
a template created by Fabian Scherer
\textless https://www.linkedin.com/in/fabian-scherer-de/\textgreater{}
with Leon Weber in an advisory role.

\href{/app?template=not-jku-thesis&version=0.1.0}{Create project in app}

\phantomsection\label{readme}
\href{https://github.com/typst/packages/raw/main/packages/preview/not-jku-thesis/0.1.0/template/thesis.pdf}{The
compiled demo thesis.pdf}

This is a Typst template for a thesis at JKU.

\subsection{Usage}\label{usage}

You can use this template in the Typst web app by clicking “Start from
template� on the dashboard and searching for
\texttt{\ not-JKU-thesis\ } .

Alternatively, you can use the CLI to kick this project off using the
command

\begin{verbatim}
typst init @preview/jku-thesis
\end{verbatim}

Typst will create a new directory with all the files needed to get you
started.

\subsection{Configuration}\label{configuration}

This template exports the \texttt{\ jku-thesis\ } function with the
following named arguments:

\begin{itemize}
\tightlist
\item
  \texttt{\ thesis-type\ } : String
\item
  \texttt{\ degree\ } : String
\item
  \texttt{\ program\ } : String
\item
  \texttt{\ supervisor\ } : String
\item
  \texttt{\ advisor\ } : Array of Strings
\item
  \texttt{\ department\ } : String
\item
  \texttt{\ author\ } : String
\item
  \texttt{\ date\ } : datetime
\item
  \texttt{\ place-of-submission\ } : string
\item
  \texttt{\ title\ } : String
\item
  \texttt{\ abstract-en\ } : Content block
\item
  \texttt{\ abstract-de\ } : optional: Content block or none
\item
  \texttt{\ acknowledgements\ } : optional: Content block or none
\item
  \texttt{\ show-title-in-header\ } : Boolean
\item
  \texttt{\ draft\ } : Boolean
\end{itemize}

The template will initialize your package with a sample call to the
\texttt{\ jku-thesis\ } function.

The dummy thesis, including the sources, was created by generative AI
and is simply meant as a placeholder. The content, citations, and data
presented are not based on actual research or verified information. They
are intended for illustrative purposes only and should not be considered
accurate, reliable, or suitable for any academic, professional, or
research use. Any resemblance to real persons, living or dead, or actual
research, is purely coincidental. Users are advised to replace all
placeholder content with genuine, verified data and references before
using this material in any formal or academic context.

\href{/app?template=not-jku-thesis&version=0.1.0}{Create project in app}

\subsubsection{How to use}\label{how-to-use}

Click the button above to create a new project using this template in
the Typst app.

You can also use the Typst CLI to start a new project on your computer
using this command:

\begin{verbatim}
typst init @preview/not-jku-thesis:0.1.0
\end{verbatim}

\includesvg[width=0.16667in,height=0.16667in]{/assets/icons/16-copy.svg}

\subsubsection{About}\label{about}

\begin{description}
\tightlist
\item[Author :]
Raphael Siegl
\item[License:]
MIT-0
\item[Current version:]
0.1.0
\item[Last updated:]
October 7, 2024
\item[First released:]
October 7, 2024
\item[Minimum Typst version:]
0.11.0
\item[Archive size:]
1.84 MB
\href{https://packages.typst.org/preview/not-jku-thesis-0.1.0.tar.gz}{\pandocbounded{\includesvg[keepaspectratio]{/assets/icons/16-download.svg}}}
\item[Categor y :]
\begin{itemize}
\tightlist
\item[]
\item
  \pandocbounded{\includesvg[keepaspectratio]{/assets/icons/16-mortarboard.svg}}
  \href{https://typst.app/universe/search/?category=thesis}{Thesis}
\end{itemize}
\end{description}

\subsubsection{Where to report issues?}\label{where-to-report-issues}

This template is a project of Raphael Siegl . You can also try to ask
for help with this template on the \href{https://forum.typst.app}{Forum}
.

Please report this template to the Typst team using the
\href{https://typst.app/contact}{contact form} if you believe it is a
safety hazard or infringes upon your rights.

\phantomsection\label{versions}
\subsubsection{Version history}\label{version-history}

\begin{longtable}[]{@{}ll@{}}
\toprule\noalign{}
Version & Release Date \\
\midrule\noalign{}
\endhead
\bottomrule\noalign{}
\endlastfoot
0.1.0 & October 7, 2024 \\
\end{longtable}

Typst GmbH did not create this template and cannot guarantee correct
functionality of this template or compatibility with any version of the
Typst compiler or app.


\title{typst.app/universe/package/scholarly-epfl-thesis}

\phantomsection\label{banner}
\phantomsection\label{template-thumbnail}
\pandocbounded{\includegraphics[keepaspectratio]{https://packages.typst.org/preview/thumbnails/scholarly-epfl-thesis-0.1.2-small.webp}}

\section{scholarly-epfl-thesis}\label{scholarly-epfl-thesis}

{ 0.1.2 }

A template for a thesis at EPFL

\href{/app?template=scholarly-epfl-thesis&version=0.1.2}{Create project
in app}

\phantomsection\label{readme}
Port of
\href{https://www.overleaf.com/latex/templates/swiss-federal-institute-of-technology-in-lausanne-epfl-phd-thesis/dhcgtppybcwv}{an
unofficial LaTeX template} to Typst.

A complete example is shown in the
\href{https://github.com/augustebaum/epfl-thesis-typst/blob/v0.1.2/example}{example
folder} ; see
\href{https://github.com/augustebaum/epfl-thesis-typst/blob/v0.1.2/example/main.pdf}{example.pdf}
for the rendered PDF. The document structure can of course be adapted to
your needs.

\subsection{Screenshots}\label{screenshots}

\includegraphics[width=2.08333in,height=\textheight,keepaspectratio]{https://raw.githubusercontent.com/augustebaum/epfl-thesis-typst/v0.1.2/screenshots/cover_page.png}
\includegraphics[width=2.08333in,height=\textheight,keepaspectratio]{https://raw.githubusercontent.com/augustebaum/epfl-thesis-typst/v0.1.2/screenshots/acknowledgements.png}
\includegraphics[width=2.08333in,height=\textheight,keepaspectratio]{https://raw.githubusercontent.com/augustebaum/epfl-thesis-typst/v0.1.2/screenshots/tables_and_figures.png}
\includegraphics[width=2.08333in,height=\textheight,keepaspectratio]{https://raw.githubusercontent.com/augustebaum/epfl-thesis-typst/v0.1.2/screenshots/appendix.png}

\subsection{Usage}\label{usage}

You can use this template in the Typst web app by clicking “Start from
template� on the dashboard and searching for \texttt{\ epfl\ } .

Alternatively, you can use the CLI to kick this project off using the
command

\begin{Shaded}
\begin{Highlighting}[]
\ExtensionTok{typst}\NormalTok{ init @preview/scholarly{-}epfl{-}thesis}
\end{Highlighting}
\end{Shaded}

Typst will create a new directory with all the files needed to get you
started.

This template uses certain fonts, including Utopia Latex for most text.
If the font is not available to Typst, as is the case in the Typst Web
App, then the template will fall back to a default font. The font is
included in example shown in the Github repository
\href{https://github.com/augustebaum/epfl-thesis-typst/blob/v0.1.2/example/utopia_font}{here}
, otherwise you can download it however you like.

\subsubsection{Configuration}\label{configuration}

This template exports the \texttt{\ template\ } function with the
following named arguments:

\begin{itemize}
\tightlist
\item
  \texttt{\ title\ } : The work’s title. Default:
  \texttt{\ {[}Your\ Title{]}\ }
\item
  \texttt{\ author\ } : The author’s name. Default:
  \texttt{\ "Your\ Name"\ }
\item
  \texttt{\ paper-size\ } : The work’s
  \href{https://typst.app/docs/reference/layout/page\#parameters-paper}{paper
  size} . Default: \texttt{\ "a4"\ }
\item
  \texttt{\ date\ } : The work’s date. Unused for now. Default:
  \texttt{\ none\ }
\item
  \texttt{\ date-format\ } : The format for displaying the work’s
  date. By default, the date will be displayed as
  \texttt{\ MMMM\ DD,\ YYYY\ } . Unused for now. Default:
  \texttt{\ {[}month\ repr:long{]}\ {[}day\ padding:zero{]},\ {[}year\ repr:full{]}\ }
\end{itemize}

The template will initialize your package with a basic call to the
\texttt{\ template\ } function in a \texttt{\ show\ } rule. If you,
however, want to change an existing project to use this template, you
can add a show rule like this at the top of your file:

\begin{Shaded}
\begin{Highlighting}[]
\NormalTok{\#import "@preview/scholarly{-}epfl{-}thesis:0.1.2": *}

\NormalTok{\#show: template.with(}
\NormalTok{  title: [Your Title],}
\NormalTok{  author: "Your Name",}
\NormalTok{  date: datetime(year: 2024, month: 03, day: 19),}
\NormalTok{)}

\NormalTok{// Your content goes below.}
\end{Highlighting}
\end{Shaded}

Also included are the \texttt{\ front-matter\ } ,
\texttt{\ main-matter\ } and \texttt{\ back-matter\ } helpers which you
can use in \texttt{\ show\ } rules in your document to change certain
settings when they are called: e.g. reset the page numbering when main
matter starts, or number headings with letters in the back matter. See
\href{https://github.com/augustebaum/epfl-thesis-typst/blob/v0.1.2/example/main.typ}{example/main.typ}
for example usage.

\subsection{Development}\label{development}

In order for Typst to access the Utopia Latex font, you need to include
it your font path. I’ve included the font in \texttt{\ example/\ } so
that you can run this in your shell:

\begin{Shaded}
\begin{Highlighting}[]
\BuiltInTok{cd}\NormalTok{ example}
\ExtensionTok{typst}\NormalTok{ w main.typ }\AttributeTok{{-}{-}font{-}path}\NormalTok{ .}
\end{Highlighting}
\end{Shaded}

See
\href{https://typst.app/docs/reference/text/text/\#parameters-font}{here}
for more about the font path.

\subsection{Credits}\label{credits}

\begin{itemize}
\tightlist
\item
  The creators of the
  \href{https://github.com/talal/ilm/blob/main/lib.typ}{ILM template}
  for the page layout, header and README format which I drew heavily
  from
\item
  The creators of the
  \href{https://www.overleaf.com/latex/templates/swiss-federal-institute-of-technology-in-lausanne-epfl-phd-thesis/dhcgtppybcwv}{original
  LateX template}
\end{itemize}

\subsection{TODO}\label{todo}

\begin{itemize}
\tightlist
\item
  {[} {]} Hide header and page number on empty pages

  \begin{itemize}
  \tightlist
  \item
    Tracking issue: \url{https://github.com/typst/typst/issues/2722}
  \item
    {[} {]} Quick fix:
    \url{https://github.com/typst/typst/issues/2722\#issuecomment-1911150996}

    \begin{itemize}
    \tightlist
    \item
      Tried it, when I put it in the
      \texttt{\ show\ heading.where(level:\ 1)\ } it disrupts the
      outline. I guess it would work if you put in the
      \texttt{\ metadata\ } manually before each chapter.
    \end{itemize}
  \item
    Optionally don’t force an empty page
  \end{itemize}
\item
  {[} {]} Table of contents

  \begin{itemize}
  \tightlist
  \item
    {[} {]} Join abstracts into one outline entry

    \begin{itemize}
    \tightlist
    \item
      I removed the lines for the German and French abstracts so it
      takes less space, but it’s not exactly the same as the original
      which has a custom outline entry
    \end{itemize}
  \item
    {[} {]} Style

    \begin{itemize}
    \tightlist
    \item
      {[} {]} Space between heading number and heading
    \item
      {[} {]} Level 1 Headings are bold and don’t have dot lines
      between the heading and the page number
    \item
      \url{https://sitandr.github.io/typst-examples-book/book/snippets/chapters/outlines.html}
    \item
      \texttt{\ outline.entry\ } can’t be modified easily because the
      arguments are positional

      \begin{itemize}
      \tightlist
      \item
        I found a solution on discord but it strips away the links. I
        tried putting in a \texttt{\ link\ } manually but that gets
        formatted like a link in the text, which is not what we’re
        looking for.

        \begin{itemize}
        \tightlist
        \item
          A solution to that link issue can be found in this thread:
          \url{https://discord.com/channels/1054443721975922748/1231526650462736474}
        \end{itemize}
      \end{itemize}
    \item
      I might use \url{https://typst.app/universe/package/outrageous}
    \end{itemize}
  \item
    {[}x{]} Include list of figures and tables
  \end{itemize}
\item
  {[} {]} Figures

  \begin{itemize}
  \tightlist
  \item
    {[} {]} Subfigures

    \begin{itemize}
    \tightlist
    \item
      tracking issue: \url{https://github.com/typst/typst/issues/246}
    \item
      A wip: \url{https://github.com/tingerrr/subpar}
    \item
      A quickfix:
      \url{https://github.com/typst/typst/issues/246\#issuecomment-1928735969}

      \begin{itemize}
      \tightlist
      \item
        Works if you abuse the \texttt{\ kind\ } mechanic, but I can’t
        get the superfigure’s caption centered
      \end{itemize}
    \end{itemize}
  \item
    {[}x{]} Short caption for table of contents

    \begin{itemize}
    \tightlist
    \item
      \url{https://sitandr.github.io/typst-examples-book/book/snippets/chapters/outlines.html}
    \end{itemize}
  \item
    {[}x{]} Numbering

    \begin{itemize}
    \tightlist
    \item
      i-figured?
    \end{itemize}
  \end{itemize}
\item
  {[} {]} Chemistry examples?
\item
  {[} {]} CV?
\item
  {[}x{]} Spacing after heading is different depending on if we’re in
  frontmatter or main matter
\item
  {[}x{]} Cover page should take its values from the template arguments

  \begin{itemize}
  \tightlist
  \item
    cover page is separate from template, given that it is not meant to
    be printed anyways

    \begin{itemize}
    \tightlist
    \item
      this also reflects how the latex template works
    \end{itemize}
  \end{itemize}
\item
  {[}x{]} Spacing before new sub-heading
\item
  {[}x{]} Readme

  \begin{itemize}
  \tightlist
  \item
    {[}x{]} How-to
  \item
    {[}x{]} Screenshots
  \item
    {[}x{]} Thumbnail
  \end{itemize}
\item
  {[}x{]} Refactor to \texttt{\ front-matter\ } ,
  \texttt{\ main-matter\ } …
\item
  {[}x{]} Numbering

  \begin{itemize}
  \tightlist
  \item
    {[}x{]} Why are pagenumbers bold on certain pages?

    \begin{itemize}
    \tightlist
    \item
      There was a show rule that inserted a pagebreak before each
      chapter. This produced a bug where the chapter start pages was
      inconsistent with the information Typst has.
    \end{itemize}
  \item
    {[}x{]} numbering starts on acknowledgements (or somewhere else?)
  \end{itemize}
\item
  {[}x{]} Equations

  \begin{itemize}
  \tightlist
  \item
    {[}x{]} Numbering

    \begin{itemize}
    \tightlist
    \item
      \url{https://sitandr.github.io/typst-examples-book/book/snippets/math/numbering.html}
    \end{itemize}
  \item
    {[}x{]} Align left

    \begin{itemize}
    \tightlist
    \item
      Why did \texttt{\ pad\ } work and not \texttt{\ h\ } ?
    \end{itemize}
  \end{itemize}
\item
  {[}x{]} page numbers are too low in the page
\item
  {[}x{]} First-line indent for front matter

  \begin{itemize}
  \tightlist
  \item
    \url{https://typst.app/docs/reference/model/par/\#parameters-first-line-indent}
  \item
    Actually this looks unintentional?
  \end{itemize}
\item
  {[}x{]} Appendices
\item
  {[}x{]} Margins
\item
  {[}x{]} Tables

  \begin{itemize}
  \tightlist
  \item
    {[}x{]} Style
  \end{itemize}
\end{itemize}

\href{/app?template=scholarly-epfl-thesis&version=0.1.2}{Create project
in app}

\subsubsection{How to use}\label{how-to-use}

Click the button above to create a new project using this template in
the Typst app.

You can also use the Typst CLI to start a new project on your computer
using this command:

\begin{verbatim}
typst init @preview/scholarly-epfl-thesis:0.1.2
\end{verbatim}

\includesvg[width=0.16667in,height=0.16667in]{/assets/icons/16-copy.svg}

\subsubsection{About}\label{about}

\begin{description}
\tightlist
\item[Author :]
\href{https://github.com/augustebaum}{Auguste Baum}
\item[License:]
MIT
\item[Current version:]
0.1.2
\item[Last updated:]
November 4, 2024
\item[First released:]
May 3, 2024
\item[Minimum Typst version:]
0.11.0
\item[Archive size:]
425 kB
\href{https://packages.typst.org/preview/scholarly-epfl-thesis-0.1.2.tar.gz}{\pandocbounded{\includesvg[keepaspectratio]{/assets/icons/16-download.svg}}}
\item[Repository:]
\href{https://github.com/augustebaum/epfl-thesis-typst}{GitHub}
\item[Categor y :]
\begin{itemize}
\tightlist
\item[]
\item
  \pandocbounded{\includesvg[keepaspectratio]{/assets/icons/16-mortarboard.svg}}
  \href{https://typst.app/universe/search/?category=thesis}{Thesis}
\end{itemize}
\end{description}

\subsubsection{Where to report issues?}\label{where-to-report-issues}

This template is a project of Auguste Baum . Report issues on
\href{https://github.com/augustebaum/epfl-thesis-typst}{their
repository} . You can also try to ask for help with this template on the
\href{https://forum.typst.app}{Forum} .

Please report this template to the Typst team using the
\href{https://typst.app/contact}{contact form} if you believe it is a
safety hazard or infringes upon your rights.

\phantomsection\label{versions}
\subsubsection{Version history}\label{version-history}

\begin{longtable}[]{@{}ll@{}}
\toprule\noalign{}
Version & Release Date \\
\midrule\noalign{}
\endhead
\bottomrule\noalign{}
\endlastfoot
0.1.2 & November 4, 2024 \\
\href{https://typst.app/universe/package/scholarly-epfl-thesis/0.1.1/}{0.1.1}
& July 18, 2024 \\
\href{https://typst.app/universe/package/scholarly-epfl-thesis/0.1.0/}{0.1.0}
& May 3, 2024 \\
\end{longtable}

Typst GmbH did not create this template and cannot guarantee correct
functionality of this template or compatibility with any version of the
Typst compiler or app.


\title{typst.app/universe/package/arkheion}

\phantomsection\label{banner}
\phantomsection\label{template-thumbnail}
\pandocbounded{\includegraphics[keepaspectratio]{https://packages.typst.org/preview/thumbnails/arkheion-0.1.0-small.webp}}

\section{arkheion}\label{arkheion}

{ 0.1.0 }

A simple template reproducing popular arXiv templates.

\href{/app?template=arkheion&version=0.1.0}{Create project in app}

\phantomsection\label{readme}
A Typst template based on popular LateX template used in arXiv and
bio-arXiv. Inspired by
\href{https://github.com/kourgeorge/arxiv-style}{arxiv-style}

\subsection{Usage}\label{usage}

\textbf{Import}

\begin{verbatim}
#import "@preview/arkheion:0.1.0": arkheion, arkheion-appendices
\end{verbatim}

\textbf{Main body}

\begin{verbatim}
#show: arkheion.with(
  title: "ArXiv Typst Template",
  authors: (
    (name: "Author 1", email: "user@domain.com", affiliation: "Company", orcid: "0000-0000-0000-0000"),
    (name: "Author 2", email: "user@domain.com", affiliation: "Company"),
  ),
  // Insert your abstract after the colon, wrapped in brackets.
  // Example: `abstract: [This is my abstract...]`
  abstract: lorem(55),
  keywords: ("First keyword", "Second keyword", "etc."),
  date: "May 16, 2023",
)
\end{verbatim}

\textbf{Appendix}

\begin{verbatim}
#show: arkheion-appendices
=

== Appendix section

#lorem(100)
\end{verbatim}

\subsection{License}\label{license}

The MIT License (MIT)

Copyright © 2023 Manuel Goulão

Permission is hereby granted, free of charge, to any person obtaining a
copy of this software and associated documentation files (the
“Software�), to deal in the Software without restriction, including
without limitation the rights to use, copy, modify, merge, publish,
distribute, sublicense, and/or sell copies of the Software, and to
permit persons to whom the Software is furnished to do so, subject to
the following conditions:

The above copyright notice and this permission notice shall be included
in all copies or substantial portions of the Software.

THE SOFTWARE IS PROVIDED “AS IS�, WITHOUT WARRANTY OF ANY KIND,
EXPRESS OR IMPLIED, INCLUDING BUT NOT LIMITED TO THE WARRANTIES OF
MERCHANTABILITY, FITNESS FOR A PARTICULAR PURPOSE AND NONINFRINGEMENT.
IN NO EVENT SHALL THE AUTHORS OR COPYRIGHT HOLDERS BE LIABLE FOR ANY
CLAIM, DAMAGES OR OTHER LIABILITY, WHETHER IN AN ACTION OF CONTRACT,
TORT OR OTHERWISE, ARISING FROM, OUT OF OR IN CONNECTION WITH THE
SOFTWARE OR THE USE OR OTHER DEALINGS IN THE SOFTWARE.

\href{/app?template=arkheion&version=0.1.0}{Create project in app}

\subsubsection{How to use}\label{how-to-use}

Click the button above to create a new project using this template in
the Typst app.

You can also use the Typst CLI to start a new project on your computer
using this command:

\begin{verbatim}
typst init @preview/arkheion:0.1.0
\end{verbatim}

\includesvg[width=0.16667in,height=0.16667in]{/assets/icons/16-copy.svg}

\subsubsection{About}\label{about}

\begin{description}
\tightlist
\item[Author :]
Manuel Goulão
\item[License:]
MIT
\item[Current version:]
0.1.0
\item[Last updated:]
March 23, 2024
\item[First released:]
March 23, 2024
\item[Archive size:]
4.85 kB
\href{https://packages.typst.org/preview/arkheion-0.1.0.tar.gz}{\pandocbounded{\includesvg[keepaspectratio]{/assets/icons/16-download.svg}}}
\item[Repository:]
\href{https://github.com/mgoulao/arkheion}{GitHub}
\item[Discipline s :]
\begin{itemize}
\tightlist
\item[]
\item
  \href{https://typst.app/universe/search/?discipline=engineering}{Engineering}
\item
  \href{https://typst.app/universe/search/?discipline=computer-science}{Computer
  Science}
\end{itemize}
\item[Categor y :]
\begin{itemize}
\tightlist
\item[]
\item
  \pandocbounded{\includesvg[keepaspectratio]{/assets/icons/16-atom.svg}}
  \href{https://typst.app/universe/search/?category=paper}{Paper}
\end{itemize}
\end{description}

\subsubsection{Where to report issues?}\label{where-to-report-issues}

This template is a project of Manuel Goulão . Report issues on
\href{https://github.com/mgoulao/arkheion}{their repository} . You can
also try to ask for help with this template on the
\href{https://forum.typst.app}{Forum} .

Please report this template to the Typst team using the
\href{https://typst.app/contact}{contact form} if you believe it is a
safety hazard or infringes upon your rights.

\phantomsection\label{versions}
\subsubsection{Version history}\label{version-history}

\begin{longtable}[]{@{}ll@{}}
\toprule\noalign{}
Version & Release Date \\
\midrule\noalign{}
\endhead
\bottomrule\noalign{}
\endlastfoot
0.1.0 & March 23, 2024 \\
\end{longtable}

Typst GmbH did not create this template and cannot guarantee correct
functionality of this template or compatibility with any version of the
Typst compiler or app.


