\title{typst.app/universe/package/metro}

\phantomsection\label{banner}
\section{metro}\label{metro}

{ 0.3.0 }

Typset units and numbers with options.

\phantomsection\label{readme}
The Metro package aims to be a port of the Latex package siunitx. It
allows easy typesetting of numbers and units with options. This package
is very early in development and many features are missing, so any
feature requests or bug reports are welcome!

Metro’s name comes from Metrology, the study scientific study of
measurement.

\textbf{Bug reports, feature requests, and PRs are welcome!}

\subsection{Usage}\label{usage}

Requires Typst v0.11.0+. Use Typst’s package manager:

\begin{verbatim}
#import "@preview/metro:0.3.0": *
\end{verbatim}

You can also download the \texttt{\ src\ } folder and import
\texttt{\ lib.typ\ } and import:

\begin{verbatim}
#import "src/lib.typ": *
\end{verbatim}

See the manual for more detailed information:
\href{https://github.com/typst/packages/raw/main/packages/preview/metro/0.3.0/manual.pdf}{manual.pdf}

\subsection{Future Features (in no particular
order)}\label{future-features-in-no-particular-order}

\begin{itemize}
\tightlist
\item
  {[}x{]} Angles
\item
  {[}x{]} Complex numbers
\item
  {[}x{]} Ranges, lists and products
\item
  {[} {]} table extensions?
\item
  {[} {]} Number parsing

  \begin{itemize}
  \tightlist
  \item
    {[} {]} Uncertainties
  \item
    {[}x{]} Exponents
  \end{itemize}
\item
  {[}x{]} Number post-processing

  \begin{itemize}
  \tightlist
  \item
    {[}x{]} rounding
  \item
    {[}x{]} exponent modes
  \end{itemize}
\end{itemize}

\subsubsection{How to add}\label{how-to-add}

Copy this into your project and use the import as \texttt{\ metro\ }

\begin{verbatim}
#import "@preview/metro:0.3.0"
\end{verbatim}

\includesvg[width=0.16667in,height=0.16667in]{/assets/icons/16-copy.svg}

Check the docs for
\href{https://typst.app/docs/reference/scripting/\#packages}{more
information on how to import packages} .

\subsubsection{About}\label{about}

\begin{description}
\tightlist
\item[Author s :]
\href{https://github.com/fenjalien}{fenjalien} \&
\href{https://github.com/Mc-Zen}{Mc-Zen}
\item[License:]
Apache-2.0
\item[Current version:]
0.3.0
\item[Last updated:]
May 27, 2024
\item[First released:]
August 22, 2023
\item[Minimum Typst version:]
0.11.0
\item[Archive size:]
17.5 kB
\href{https://packages.typst.org/preview/metro-0.3.0.tar.gz}{\pandocbounded{\includesvg[keepaspectratio]{/assets/icons/16-download.svg}}}
\item[Repository:]
\href{https://github.com/fenjalien/metro}{GitHub}
\item[Categor ies :]
\begin{itemize}
\tightlist
\item[]
\item
  \pandocbounded{\includesvg[keepaspectratio]{/assets/icons/16-chart.svg}}
  \href{https://typst.app/universe/search/?category=visualization}{Visualization}
\item
  \pandocbounded{\includesvg[keepaspectratio]{/assets/icons/16-text.svg}}
  \href{https://typst.app/universe/search/?category=text}{Text}
\end{itemize}
\end{description}

\subsubsection{Where to report issues?}\label{where-to-report-issues}

This package is a project of fenjalien and Mc-Zen . Report issues on
\href{https://github.com/fenjalien/metro}{their repository} . You can
also try to ask for help with this package on the
\href{https://forum.typst.app}{Forum} .

Please report this package to the Typst team using the
\href{https://typst.app/contact}{contact form} if you believe it is a
safety hazard or infringes upon your rights.

\phantomsection\label{versions}
\subsubsection{Version history}\label{version-history}

\begin{longtable}[]{@{}ll@{}}
\toprule\noalign{}
Version & Release Date \\
\midrule\noalign{}
\endhead
\bottomrule\noalign{}
\endlastfoot
0.3.0 & May 27, 2024 \\
\href{https://typst.app/universe/package/metro/0.2.0/}{0.2.0} & February
7, 2024 \\
\href{https://typst.app/universe/package/metro/0.1.1/}{0.1.1} &
September 22, 2023 \\
\href{https://typst.app/universe/package/metro/0.1.0/}{0.1.0} & August
22, 2023 \\
\end{longtable}

Typst GmbH did not create this package and cannot guarantee correct
functionality of this package or compatibility with any version of the
Typst compiler or app.


\title{typst.app/universe/package/lasaveur}

\phantomsection\label{banner}
\section{lasaveur}\label{lasaveur}

{ 0.1.3 }

Porting vim-latex\textquotesingle s math shorthands to Typst. An
accommendating vim syntax file is provided in the repo.

\phantomsection\label{readme}
This is a Typst package for speedy mathematical input, inspired by
\href{https://github.com/vim-latex/vim-latex}{vim-latex} . This project
is named after my Vim plugin
\href{https://github.com/yangwenbo99/vimtex-lasaveur}{vimtex-lasaveur} ,
which ports the operations in vim-latex to
\href{https://github.com/lervag/vimtex}{vimtex} .

\subsection{Usages in Typst}\label{usages-in-typst}

Either use the file released in “Releases� or import using the
following command:

\begin{Shaded}
\begin{Highlighting}[]
\NormalTok{\#import "@preview/lasaveur:0.1.3": *}
\end{Highlighting}
\end{Shaded}

This script generates a Typst library that defines shorthand commands
for various mathematical symbols and functions. Here’s an overview of
what it provides and how a user can use it:

\begin{enumerate}
\tightlist
\item
  Mathematical Functions:

  \begin{itemize}
  \tightlist
  \item
    Usage: \texttt{\ f\textless{}key\textgreater{}(argument)\ }
  \item
    Examples: \texttt{\ fh(x)\ } for hat, \texttt{\ ft(x)\ } for tilde,
    \texttt{\ f2(x)\ } for square root
  \end{itemize}
\item
  Font Styles:

  \begin{itemize}
  \tightlist
  \item
    Usage: \texttt{\ f\textless{}key\textgreater{}(argument)\ }
  \item
    Examples: \texttt{\ fb(x)\ } for bold, \texttt{\ fbb(x)\ } for
    blackboard bold, \texttt{\ fca(x)\ } for calligraphic
  \end{itemize}
\item
  Greek Letters:

  \begin{itemize}
  \tightlist
  \item
    Usage: \texttt{\ k\textless{}key\textgreater{}\ }
  \item
    Examples: \texttt{\ ka\ } for α (alpha), \texttt{\ kb\ } for β
    (beta), \texttt{\ kG\ } for Î`` (capital Gamma)
  \end{itemize}
\item
  Common Mathematical Symbols:

  \begin{itemize}
  \tightlist
  \item
    Usage: \texttt{\ g\textless{}key\textgreater{}\ }
  \item
    Examples: \texttt{\ g8\ } for ∞ (infinity), \texttt{\ gU\ } for
    ∪ (union), \texttt{\ gI\ } for ∩ (intersection)
  \end{itemize}
\item
  LaTeX-compatible Symbols:

  \begin{itemize}
  \tightlist
  \item
    Usage: Direct LaTeX command names
  \item
    Examples: \texttt{\ partial\ } for ∂, \texttt{\ infty\ } for ∞,
    \texttt{\ cdot\ } for â‹
  \end{itemize}
\item
  Arrows:

  \begin{itemize}
  \tightlist
  \item
    Usage: \texttt{\ ar.\textless{}key\textgreater{}\ }
  \item
    Examples: \texttt{\ ar.l\ } for â†?, \texttt{\ ar.r\ } for â†',
    \texttt{\ ar.lr\ } for â†''
  \end{itemize}
\end{enumerate}

Users can employ these shorthands in their Typst documents to quickly
input mathematical symbols and functions. The exact prefix for each
category (like \texttt{\ f\ } for functions or \texttt{\ k\ } for Greek
letters) can be customized using command-line arguments when running the
script.

For instance, in a Typst document, after importing the generated
library, a user could write:

\begin{Shaded}
\begin{Highlighting}[]
\NormalTok{$fh(x) + ka + g8 + ar.r$}
\end{Highlighting}
\end{Shaded}

This would produce: xÌ‚ + α + ∞ + â†'

The script provides a wide range of symbols covering most common
mathematical notations, making it easier and faster to type complex
mathematical expressions in Typst â€`` especially for users migrating
from vim-latex.

\subsection{Accompanying Vim Syntax
File}\label{accompanying-vim-syntax-file}

The syntax file provides more advanced and correct concealing for both
Typst’s built-in math syntax and the lasaveur shorthands. Download the
syntax file from the “Releases� section and place it in your
\texttt{\ \textasciitilde{}/.vim/after/syntax/\ } directory. The
\texttt{\ syntax.vim\ } file in the repo is supposed to be used by the
generation script and it \emph{will not work} if directly sourced in
Vim.

\subsubsection{How to add}\label{how-to-add}

Copy this into your project and use the import as \texttt{\ lasaveur\ }

\begin{verbatim}
#import "@preview/lasaveur:0.1.3"
\end{verbatim}

\includesvg[width=0.16667in,height=0.16667in]{/assets/icons/16-copy.svg}

Check the docs for
\href{https://typst.app/docs/reference/scripting/\#packages}{more
information on how to import packages} .

\subsubsection{About}\label{about}

\begin{description}
\tightlist
\item[Author :]
\href{https://github.com/yangwenbo99}{Paul Yang}
\item[License:]
MIT
\item[Current version:]
0.1.3
\item[Last updated:]
August 22, 2024
\item[First released:]
August 22, 2024
\item[Archive size:]
2.25 kB
\href{https://packages.typst.org/preview/lasaveur-0.1.3.tar.gz}{\pandocbounded{\includesvg[keepaspectratio]{/assets/icons/16-download.svg}}}
\item[Repository:]
\href{https://github.com/yangwenbo99/typst-lasaveur}{GitHub}
\item[Categor y :]
\begin{itemize}
\tightlist
\item[]
\item
  \pandocbounded{\includesvg[keepaspectratio]{/assets/icons/16-hammer.svg}}
  \href{https://typst.app/universe/search/?category=utility}{Utility}
\end{itemize}
\end{description}

\subsubsection{Where to report issues?}\label{where-to-report-issues}

This package is a project of Paul Yang . Report issues on
\href{https://github.com/yangwenbo99/typst-lasaveur}{their repository} .
You can also try to ask for help with this package on the
\href{https://forum.typst.app}{Forum} .

Please report this package to the Typst team using the
\href{https://typst.app/contact}{contact form} if you believe it is a
safety hazard or infringes upon your rights.

\phantomsection\label{versions}
\subsubsection{Version history}\label{version-history}

\begin{longtable}[]{@{}ll@{}}
\toprule\noalign{}
Version & Release Date \\
\midrule\noalign{}
\endhead
\bottomrule\noalign{}
\endlastfoot
0.1.3 & August 22, 2024 \\
\end{longtable}

Typst GmbH did not create this package and cannot guarantee correct
functionality of this package or compatibility with any version of the
Typst compiler or app.


\title{typst.app/universe/package/yats}

\phantomsection\label{banner}
\section{yats}\label{yats}

{ 0.1.0 }

Serialization module for Typst

\phantomsection\label{readme}
serialize the Typst objects into bytes and deserialize the bytes into
Typst objects

\subsection{Reason}\label{reason}

I want to interactive between the wasm and typst. But I found that the
input arguments and output argument are all bytes. It is not convenient
for me to use WASM. So I designed the serialization protocol and
implemented this serialization module for reference.

Although there have been some serialization APIs like cbor, yaml, json
and others, this is a good learning material and a good example to show
the abilities of Typst.

\subsection{Example}\label{example}

Have a look at the example
\href{https://github.com/typst/packages/raw/main/packages/preview/yats/0.1.0/example.typ}{here}
.

\subsection{Usage}\label{usage}

Simply import 2 functions : \texttt{\ serialize\ } ,
\texttt{\ deserialize\ } .

And enjoy it

\begin{Shaded}
\begin{Highlighting}[]
\NormalTok{\#import "@preview/yats:0.1.0": serialize, deserialize}
\NormalTok{\#\{}
\NormalTok{  let obj = (}
\NormalTok{    name : "0warning0error",}
\NormalTok{    age : 100,}
\NormalTok{    height : 1.8,}
\NormalTok{    birthday : datetime(year : 1998,month : 7,day:8)}
\NormalTok{  )}
\NormalTok{  deserialize(serialize(obj))}
\NormalTok{\}}
\end{Highlighting}
\end{Shaded}

\subsection{Supported Types}\label{supported-types}

\begin{itemize}
\tightlist
\item
  \texttt{\ none\ }
\item
  \texttt{\ bool\ }
\item
  \texttt{\ type\ } : type is a type
\item
  \texttt{\ int\ }
\item
  \texttt{\ float\ } : (implemented in string, for convenience)
\item
  \texttt{\ datetime\ } : only support \texttt{\ year\ } ,
  \texttt{\ month\ } , \texttt{\ day\ } ; \texttt{\ hour\ } ,
  \texttt{\ minute\ } , \texttt{\ second\ } ; both combined.
\item
  \texttt{\ duration\ }
\item
  \texttt{\ bytes\ }
\item
  \texttt{\ string\ }
\item
  \texttt{\ regex\ } : (dealing with it is a little tricky)
\item
  \texttt{\ array\ } : the element in it can be anything listed.
\item
  \texttt{\ dictionary\ } : the value in it can be anything listed.
\end{itemize}

\subsection{\texorpdfstring{\texttt{\ Yats\ }
function}{ Yats  function}}\label{yats-function}

\begin{Shaded}
\begin{Highlighting}[]
\NormalTok{\#let serialize(}
\NormalTok{  data : any}
\NormalTok{) = \{ .. \}}
\end{Highlighting}
\end{Shaded}

\textbf{Arguments:}

\begin{itemize}
\tightlist
\item
  \texttt{\ data\ } : {[} \texttt{\ any\ } {]} â€'' Any supported object
  .
\end{itemize}

\textbf{Return}

The serialized bytes.

\begin{Shaded}
\begin{Highlighting}[]
\NormalTok{\#let deserialize(}
\NormalTok{  data : array}
\NormalTok{) = \{ .. \}}
\end{Highlighting}
\end{Shaded}

\textbf{Arguments:}

\begin{itemize}
\tightlist
\item
  \texttt{\ data\ } : {[} \texttt{\ bytes\ } {]} â€'' serialized objects
  represented by bytes .
\end{itemize}

\textbf{Return}

binary array. (the first one is the object deserialized, the second one
is the valid length of the bytes)

\subsection{Potential Problems and
limitation}\label{potential-problems-and-limitation}

\begin{itemize}
\item
  Some problem can be caused by changes of \texttt{\ repr\ } . For
  example, the serialization of \texttt{\ regex\ } relies on the
  \texttt{\ repr\ } of \texttt{\ regex\ } . And there are no method to
  directly catch the inner \texttt{\ string\ } .
\item
  Because of lack of time, only basic types are supported. But more
  types can be supported in Typst.
\end{itemize}

\subsection{License}\label{license}

This project is licensed under the Apache 2.0 License.

\subsubsection{How to add}\label{how-to-add}

Copy this into your project and use the import as \texttt{\ yats\ }

\begin{verbatim}
#import "@preview/yats:0.1.0"
\end{verbatim}

\includesvg[width=0.16667in,height=0.16667in]{/assets/icons/16-copy.svg}

Check the docs for
\href{https://typst.app/docs/reference/scripting/\#packages}{more
information on how to import packages} .

\subsubsection{About}\label{about}

\begin{description}
\tightlist
\item[Author :]
\href{https://github.com/0warning0error}{Zhao Yuanhang}
\item[License:]
Apache-2.0
\item[Current version:]
0.1.0
\item[Last updated:]
March 15, 2024
\item[First released:]
March 15, 2024
\item[Minimum Typst version:]
0.10.0
\item[Archive size:]
7.92 kB
\href{https://packages.typst.org/preview/yats-0.1.0.tar.gz}{\pandocbounded{\includesvg[keepaspectratio]{/assets/icons/16-download.svg}}}
\item[Repository:]
\href{https://github.com/0warning0error/typst-yats}{GitHub}
\end{description}

\subsubsection{Where to report issues?}\label{where-to-report-issues}

This package is a project of Zhao Yuanhang . Report issues on
\href{https://github.com/0warning0error/typst-yats}{their repository} .
You can also try to ask for help with this package on the
\href{https://forum.typst.app}{Forum} .

Please report this package to the Typst team using the
\href{https://typst.app/contact}{contact form} if you believe it is a
safety hazard or infringes upon your rights.

\phantomsection\label{versions}
\subsubsection{Version history}\label{version-history}

\begin{longtable}[]{@{}ll@{}}
\toprule\noalign{}
Version & Release Date \\
\midrule\noalign{}
\endhead
\bottomrule\noalign{}
\endlastfoot
0.1.0 & March 15, 2024 \\
\end{longtable}

Typst GmbH did not create this package and cannot guarantee correct
functionality of this package or compatibility with any version of the
Typst compiler or app.


\title{typst.app/universe/package/tuhi-labscript-vuw}

\phantomsection\label{banner}
\phantomsection\label{template-thumbnail}
\pandocbounded{\includegraphics[keepaspectratio]{https://packages.typst.org/preview/thumbnails/tuhi-labscript-vuw-0.1.0-small.webp}}

\section{tuhi-labscript-vuw}\label{tuhi-labscript-vuw}

{ 0.1.0 }

A labscript template for VUW experimental courses.

\href{/app?template=tuhi-labscript-vuw&version=0.1.0}{Create project in
app}

\phantomsection\label{readme}
A Typst template for VUW lab scripts. To get started:

\begin{Shaded}
\begin{Highlighting}[]
\NormalTok{typst init @preview/tuhi{-}labscript{-}vuw:0.1.0}
\end{Highlighting}
\end{Shaded}

And edit the \texttt{\ main.typ\ } example.

\pandocbounded{\includegraphics[keepaspectratio]{https://github.com/typst/packages/raw/main/packages/preview/tuhi-labscript-vuw/0.1.0/thumbnail.png}}

\subsection{Contributing}\label{contributing}

PRs are welcome! And if you encounter any bugs or have any
requests/ideas, feel free to open an issue.

\href{/app?template=tuhi-labscript-vuw&version=0.1.0}{Create project in
app}

\subsubsection{How to use}\label{how-to-use}

Click the button above to create a new project using this template in
the Typst app.

You can also use the Typst CLI to start a new project on your computer
using this command:

\begin{verbatim}
typst init @preview/tuhi-labscript-vuw:0.1.0
\end{verbatim}

\includesvg[width=0.16667in,height=0.16667in]{/assets/icons/16-copy.svg}

\subsubsection{About}\label{about}

\begin{description}
\tightlist
\item[Author :]
\href{https://github.com/baptiste}{baptiste}
\item[License:]
MPL-2.0
\item[Current version:]
0.1.0
\item[Last updated:]
June 17, 2024
\item[First released:]
June 17, 2024
\item[Archive size:]
177 kB
\href{https://packages.typst.org/preview/tuhi-labscript-vuw-0.1.0.tar.gz}{\pandocbounded{\includesvg[keepaspectratio]{/assets/icons/16-download.svg}}}
\item[Categor y :]
\begin{itemize}
\tightlist
\item[]
\item
  \pandocbounded{\includesvg[keepaspectratio]{/assets/icons/16-envelope.svg}}
  \href{https://typst.app/universe/search/?category=office}{Office}
\end{itemize}
\end{description}

\subsubsection{Where to report issues?}\label{where-to-report-issues}

This template is a project of baptiste . You can also try to ask for
help with this template on the \href{https://forum.typst.app}{Forum} .

Please report this template to the Typst team using the
\href{https://typst.app/contact}{contact form} if you believe it is a
safety hazard or infringes upon your rights.

\phantomsection\label{versions}
\subsubsection{Version history}\label{version-history}

\begin{longtable}[]{@{}ll@{}}
\toprule\noalign{}
Version & Release Date \\
\midrule\noalign{}
\endhead
\bottomrule\noalign{}
\endlastfoot
0.1.0 & June 17, 2024 \\
\end{longtable}

Typst GmbH did not create this template and cannot guarantee correct
functionality of this template or compatibility with any version of the
Typst compiler or app.


\title{typst.app/universe/package/linguify}

\phantomsection\label{banner}
\section{linguify}\label{linguify}

{ 0.4.1 }

Load strings for different languages easily

\phantomsection\label{readme}
Load strings for different languages easily. This can be useful if you
create a package or template for multilingual usage.

\subsection{Usage}\label{usage}

The usage depends if you are using it inside a package or a template or
in your own document.

\subsubsection{For end users and own
templates}\label{for-end-users-and-own-templates}

You can use linguify global database.

Example:

\begin{Shaded}
\begin{Highlighting}[]
\NormalTok{\#import "@preview/linguify:0.4.0": *}

\NormalTok{\#let lang\_data = toml("lang.toml")}
\NormalTok{\#set{-}database(lang\_data);}

\NormalTok{\#set text(lang: "de")}

\NormalTok{\#linguify("abstract")  // Shows "Zusammenfassung" in the document.}
\end{Highlighting}
\end{Shaded}

The \texttt{\ lang.toml\ } musst look like this:

\begin{Shaded}
\begin{Highlighting}[]
\KeywordTok{[conf]}
\DataTypeTok{default{-}lang} \OperatorTok{=} \StringTok{"en"}

\KeywordTok{[en]}
\DataTypeTok{title} \OperatorTok{=} \StringTok{"A simple linguify example"}
\DataTypeTok{abstract} \OperatorTok{=} \StringTok{"Abstract"}

\KeywordTok{[de]}
\DataTypeTok{title} \OperatorTok{=} \StringTok{"Ein einfaches Linguify Beispiel"}
\DataTypeTok{abstract} \OperatorTok{=} \StringTok{"Zusammenfassung"}
\end{Highlighting}
\end{Shaded}

\subsubsection{Inside a package}\label{inside-a-package}

So that multiple packages can use linguify simultaneously, they should
contain their own database. A linguify database is just a dictionary
with a certain structure. (See database structure.)

Recommend is to store the database in a separate file like
\texttt{\ lang.toml\ } and load it inside the document. And specify it
in each \texttt{\ linguify()\ } function call.

Example:

\begin{Shaded}
\begin{Highlighting}[]
\NormalTok{\#import "@preview/linguify:0.4.0": *}

\NormalTok{\#let database = toml("lang.toml")}

\NormalTok{\#linguify("key", from: database, default: "key")}
\end{Highlighting}
\end{Shaded}

\subsection{Features}\label{features}

\begin{itemize}
\tightlist
\item
  Use a \texttt{\ toml\ } or other file to load strings for different
  languages. You need to pass a typst dictionary which follows the
  structure of the shown toml file.
\item
  Specify a \textbf{default-lang} . If none is specified it will default
  to \texttt{\ en\ }
\item
  \textbf{Fallback} to the default-lang if a key is not found for a
  certain language.
\item
  \href{https://projectfluent.org/}{Fluent} support
\end{itemize}

\subsubsection{How to add}\label{how-to-add}

Copy this into your project and use the import as \texttt{\ linguify\ }

\begin{verbatim}
#import "@preview/linguify:0.4.1"
\end{verbatim}

\includesvg[width=0.16667in,height=0.16667in]{/assets/icons/16-copy.svg}

Check the docs for
\href{https://typst.app/docs/reference/scripting/\#packages}{more
information on how to import packages} .

\subsubsection{About}\label{about}

\begin{description}
\tightlist
\item[Author :]
\href{https://github.com/jomaway}{Jomaway}
\item[License:]
MIT
\item[Current version:]
0.4.1
\item[Last updated:]
April 29, 2024
\item[First released:]
January 31, 2024
\item[Minimum Typst version:]
0.11.0
\item[Archive size:]
470 kB
\href{https://packages.typst.org/preview/linguify-0.4.1.tar.gz}{\pandocbounded{\includesvg[keepaspectratio]{/assets/icons/16-download.svg}}}
\item[Repository:]
\href{https://github.com/jomaway/typst-linguify}{GitHub}
\item[Categor ies :]
\begin{itemize}
\tightlist
\item[]
\item
  \pandocbounded{\includesvg[keepaspectratio]{/assets/icons/16-world.svg}}
  \href{https://typst.app/universe/search/?category=languages}{Languages}
\item
  \pandocbounded{\includesvg[keepaspectratio]{/assets/icons/16-hammer.svg}}
  \href{https://typst.app/universe/search/?category=utility}{Utility}
\end{itemize}
\end{description}

\subsubsection{Where to report issues?}\label{where-to-report-issues}

This package is a project of Jomaway . Report issues on
\href{https://github.com/jomaway/typst-linguify}{their repository} . You
can also try to ask for help with this package on the
\href{https://forum.typst.app}{Forum} .

Please report this package to the Typst team using the
\href{https://typst.app/contact}{contact form} if you believe it is a
safety hazard or infringes upon your rights.

\phantomsection\label{versions}
\subsubsection{Version history}\label{version-history}

\begin{longtable}[]{@{}ll@{}}
\toprule\noalign{}
Version & Release Date \\
\midrule\noalign{}
\endhead
\bottomrule\noalign{}
\endlastfoot
0.4.1 & April 29, 2024 \\
\href{https://typst.app/universe/package/linguify/0.4.0/}{0.4.0} & April
2, 2024 \\
\href{https://typst.app/universe/package/linguify/0.3.1/}{0.3.1} & March
26, 2024 \\
\href{https://typst.app/universe/package/linguify/0.3.0/}{0.3.0} & March
18, 2024 \\
\href{https://typst.app/universe/package/linguify/0.2.0/}{0.2.0} & March
16, 2024 \\
\href{https://typst.app/universe/package/linguify/0.1.0/}{0.1.0} &
January 31, 2024 \\
\end{longtable}

Typst GmbH did not create this package and cannot guarantee correct
functionality of this package or compatibility with any version of the
Typst compiler or app.


\title{typst.app/universe/package/scrutinize}

\phantomsection\label{banner}
\section{scrutinize}\label{scrutinize}

{ 0.3.0 }

A library for building exams, tests, etc. with Typst

\phantomsection\label{readme}
Scrutinize is a library for building exams, tests, etc. with Typst. It
has three general areas of focus:

\begin{itemize}
\tightlist
\item
  It helps with grading information: record the points that can be
  reached for each question and make them available for creating grading
  keys.
\item
  It provides a selection of question writing utilities, such as
  multiple choice or true/false questions.
\item
  It supports the creation of sample solutions by allowing to switch
  between the normal and “pre-filled� exam.
\end{itemize}

Right now, providing a styled template is not part of this package’s
scope. Also, visual customization of the provided question templates is
currently nonexistent.

See the
\href{https://github.com/typst/packages/raw/main/packages/preview/scrutinize/0.3.0/docs/manual.pdf}{manual}
for details.

\subsection{Example}\label{example}

\begin{longtable}[]{@{}ll@{}}
\toprule\noalign{}
\endhead
\bottomrule\noalign{}
\endlastfoot
\href{https://github.com/typst/packages/raw/main/packages/preview/scrutinize/0.3.0/gallery/gk-ek-austria.typ}{\pandocbounded{\includegraphics[keepaspectratio]{https://github.com/typst/packages/raw/main/packages/preview/scrutinize/0.3.0/thumbnail.png}}}
&
\href{https://github.com/typst/packages/raw/main/packages/preview/scrutinize/0.3.0/gallery/gk-ek-austria.typ}{\pandocbounded{\includegraphics[keepaspectratio]{https://github.com/typst/packages/raw/main/packages/preview/scrutinize/0.3.0/thumbnail-solved.png}}} \\
\end{longtable}

This example can be found in the
\href{https://github.com/typst/packages/raw/main/packages/preview/scrutinize/0.3.0/gallery/}{gallery}
. Here are some excerpts from it:

\begin{Shaded}
\begin{Highlighting}[]
\NormalTok{\#import "@preview/scrutinize:0.3.0" as scrutinize: grading, task, solution, task{-}kinds}
\NormalTok{\#import task{-}kinds: free{-}form, gap, choice}
\NormalTok{\#import task: t}

\NormalTok{// ... document setup ...}

\NormalTok{\#context \{}
\NormalTok{  let ts = task.all(level: 2)}
\NormalTok{  let total = grading.total{-}points(ts)}

\NormalTok{  let grades = grading.grades(}
\NormalTok{    [F],}
\NormalTok{    0.6 * total,}
\NormalTok{    [D],}
\NormalTok{    0.7 * total,}
\NormalTok{    [C],}
\NormalTok{    0.8 * total,}
\NormalTok{    [B],}
\NormalTok{    0.9 * total,}
\NormalTok{    [A],}
\NormalTok{  )}

\NormalTok{  // ... show the grading key ...}
\NormalTok{\}}

\NormalTok{// ...}

\NormalTok{= Basic competencies {-}{-} theoretical part B}

\NormalTok{\#lorem(40)}

\NormalTok{== Writing}
\NormalTok{\#t(category: "b", points: 4)}
\NormalTok{\#lorem(30)}

\NormalTok{\#free{-}form.lines(stretch: 180\%, lorem(20))}

\NormalTok{== Multiple Choice}
\NormalTok{\#t(category: "b", points: 2)}
\NormalTok{\#lorem(30)}

\NormalTok{\#\{}
\NormalTok{  set align(center)}
\NormalTok{  choice.multiple((}
\NormalTok{    (lorem(3), true),}
\NormalTok{    (lorem(5), true),}
\NormalTok{    (lorem(4), false),}
\NormalTok{  ))}
\NormalTok{\}}
\end{Highlighting}
\end{Shaded}

\subsubsection{How to add}\label{how-to-add}

Copy this into your project and use the import as
\texttt{\ scrutinize\ }

\begin{verbatim}
#import "@preview/scrutinize:0.3.0"
\end{verbatim}

\includesvg[width=0.16667in,height=0.16667in]{/assets/icons/16-copy.svg}

Check the docs for
\href{https://typst.app/docs/reference/scripting/\#packages}{more
information on how to import packages} .

\subsubsection{About}\label{about}

\begin{description}
\tightlist
\item[Author :]
\href{https://github.com/SillyFreak/}{Clemens Koza}
\item[License:]
MIT
\item[Current version:]
0.3.0
\item[Last updated:]
October 14, 2024
\item[First released:]
January 8, 2024
\item[Minimum Typst version:]
0.11.0
\item[Archive size:]
11.2 kB
\href{https://packages.typst.org/preview/scrutinize-0.3.0.tar.gz}{\pandocbounded{\includesvg[keepaspectratio]{/assets/icons/16-download.svg}}}
\item[Repository:]
\href{https://github.com/SillyFreak/typst-scrutinize}{GitHub}
\item[Discipline :]
\begin{itemize}
\tightlist
\item[]
\item
  \href{https://typst.app/universe/search/?discipline=education}{Education}
\end{itemize}
\item[Categor ies :]
\begin{itemize}
\tightlist
\item[]
\item
  \pandocbounded{\includesvg[keepaspectratio]{/assets/icons/16-list-unordered.svg}}
  \href{https://typst.app/universe/search/?category=model}{Model}
\item
  \pandocbounded{\includesvg[keepaspectratio]{/assets/icons/16-code.svg}}
  \href{https://typst.app/universe/search/?category=scripting}{Scripting}
\item
  \pandocbounded{\includesvg[keepaspectratio]{/assets/icons/16-envelope.svg}}
  \href{https://typst.app/universe/search/?category=office}{Office}
\end{itemize}
\end{description}

\subsubsection{Where to report issues?}\label{where-to-report-issues}

This package is a project of Clemens Koza . Report issues on
\href{https://github.com/SillyFreak/typst-scrutinize}{their repository}
. You can also try to ask for help with this package on the
\href{https://forum.typst.app}{Forum} .

Please report this package to the Typst team using the
\href{https://typst.app/contact}{contact form} if you believe it is a
safety hazard or infringes upon your rights.

\phantomsection\label{versions}
\subsubsection{Version history}\label{version-history}

\begin{longtable}[]{@{}ll@{}}
\toprule\noalign{}
Version & Release Date \\
\midrule\noalign{}
\endhead
\bottomrule\noalign{}
\endlastfoot
0.3.0 & October 14, 2024 \\
\href{https://typst.app/universe/package/scrutinize/0.2.0/}{0.2.0} &
July 15, 2024 \\
\href{https://typst.app/universe/package/scrutinize/0.1.0/}{0.1.0} &
January 8, 2024 \\
\end{longtable}

Typst GmbH did not create this package and cannot guarantee correct
functionality of this package or compatibility with any version of the
Typst compiler or app.


\title{typst.app/universe/package/unify}

\phantomsection\label{banner}
\section{unify}\label{unify}

{ 0.7.0 }

Format numbers, units, and ranges correctly.

{ } Featured Package

\phantomsection\label{readme}
\texttt{\ unify\ } is a \href{https://github.com/typst/typst}{Typst}
package simplifying the typesetting of numbers, units, and ranges. It is
the equivalent to LaTeX’s \texttt{\ siunitx\ } , though not as mature.

\subsection{Overview}\label{overview}

\texttt{\ unify\ } allows flexible numbers and units, and still mostly
gets well typeset results.

\begin{Shaded}
\begin{Highlighting}[]
\NormalTok{\#import "@preview/unify:0.7.0": num,qty,numrange,qtyrange}

\NormalTok{$ num("{-}1.32865+{-}0.50273e{-}6") $}
\NormalTok{$ qty("1.3+1.2{-}0.3e3", "erg/cm\^{}2/s", space: "\#h(2mm)") $}
\NormalTok{$ numrange("1,1238e{-}2", "3,0868e5", thousandsep: "\textquotesingle{}") $}
\NormalTok{$ qtyrange("1e3", "2e3", "meter per second squared", per: "/", delimiter: "\textbackslash{}"to\textbackslash{}"") $}
\end{Highlighting}
\end{Shaded}

\includegraphics[width=3.125in,height=\textheight,keepaspectratio]{https://github.com/typst/packages/raw/main/packages/preview/unify/0.7.0/examples/overview.jpg}

Right now, physical, monetary, and binary units are supported. New
issues or pull requests for new units are welcome!

\subsection{Multilingual support}\label{multilingual-support}

The Unify package supports multiple languages. Currently, the supported
languages are English and Russian. The fallback is English. If you want
to add your language, you should add two files:
\texttt{\ prefixes-xx.csv\ } and \texttt{\ units-xx.csv\ } , and in the
\texttt{\ lib.typ\ } file you should fix the \texttt{\ lang-db\ } state
for your files.

\subsection{\texorpdfstring{\texttt{\ num\ }}{ num }}\label{num}

\texttt{\ num\ } uses string parsing in order to typeset numbers,
including separators between the thousands. They can have the following
form:

\begin{itemize}
\tightlist
\item
  \texttt{\ float\ } or \texttt{\ integer\ } number
\item
  either ( \texttt{\ \{\}\ } stands for a number)

  \begin{itemize}
  \tightlist
  \item
    symmetric uncertainties with \texttt{\ +-\{\}\ }
  \item
    asymmetric uncertainties with \texttt{\ +\{\}-\{\}\ }
  \end{itemize}
\item
  exponential notation \texttt{\ e\{\}\ }
\end{itemize}

Parentheses are automatically set as necessary. Use
\texttt{\ thousandsep\ } to change the separator between the thousands,
and \texttt{\ multiplier\ } to change the multiplication symbol between
the number and exponential.

\subsection{\texorpdfstring{\texttt{\ unit\ }}{ unit }}\label{unit}

\texttt{\ unit\ } takes the unit in words or in symbolic notation as its
first argument. The value of \texttt{\ space\ } will be inserted between
units if necessary. Setting \texttt{\ per\ } to \texttt{\ symbol\ } will
format the number with exponents (i.e. \texttt{\ \^{}(-1)\ } ),
\texttt{\ fraction\ } or \texttt{\ /\ } using fraction, and
\texttt{\ fraction-short\ } or
\texttt{\ \textbackslash{}\textbackslash{}/\ } using in-line
fractions.\\
Units in words have four possible parts:

\begin{itemize}
\tightlist
\item
  \texttt{\ per\ } forms the inverse of the following unit.
\item
  A written-out prefix in the sense of SI (e.g. \texttt{\ centi\ } ).
  This is added before the unit.
\item
  The unit itself written out (e.g. \texttt{\ gram\ } ).
\item
  A postfix like \texttt{\ squared\ } . This is added after the unit and
  takes \texttt{\ per\ } into account.
\end{itemize}

The shorthand notation also has four parts:

\begin{itemize}
\tightlist
\item
  \texttt{\ /\ } forms the inverse of the following unit.
\item
  A short prefix in the sense of SI (e.g. \texttt{\ k\ } ). This is
  added before the unit.
\item
  The short unit itself (e.g. \texttt{\ g\ } ).
\item
  An exponent like \texttt{\ \^{}2\ } . This is added after the unit and
  takes \texttt{\ /\ } into account.
\end{itemize}

Note: Use \texttt{\ u\ } for micro.

The possible values of the three latter parts are loaded at runtime from
\texttt{\ prefixes.csv\ } , \texttt{\ units.csv\ } , and
\texttt{\ postfixes.csv\ } (in the library directory). Your own units
etc. can be permanently added in these files. At runtime, they can be
added using \texttt{\ add-unit\ } and \texttt{\ add-prefix\ } ,
respectively. The formats for the pre- and postfixes are:

\begin{longtable}[]{@{}lll@{}}
\toprule\noalign{}
pre-/postfix & shorthand & symbol \\
\midrule\noalign{}
\endhead
\bottomrule\noalign{}
\endlastfoot
milli & m & upright(“m�) \\
\end{longtable}

and for units:

\begin{longtable}[]{@{}llll@{}}
\toprule\noalign{}
unit & shorthand & symbol & space \\
\midrule\noalign{}
\endhead
\bottomrule\noalign{}
\endlastfoot
meter & m & upright(“m�) & true \\
\end{longtable}

The first column specifies the written-out word, the second one the
shorthand. These should be unique. The third column represents the
string that will be inserted as the unit symbol. For units, the last
column describes whether there should be space before the unit (possible
values: \texttt{\ true\ } / \texttt{\ false\ } , \texttt{\ 1\ } ,
\texttt{\ 0\ } ). This is mostly the cases for degrees and other angle
units (e.g. arcseconds).\\
If you think there are units not included that are of interest for other
users, you can create an issue or PR.

\subsection{\texorpdfstring{\texttt{\ qty\ }}{ qty }}\label{qty}

\texttt{\ qty\ } allows a \texttt{\ num\ } as the first argument
following the same rules. The second argument is a unit. If
\texttt{\ rawunit\ } is set to true, its value will be passed on to the
result (note that the string passed on will be passed to
\texttt{\ eval\ } , so add escaped quotes \texttt{\ \textbackslash{}"\ }
if necessary). Otherwise, it follows the rules of \texttt{\ unit\ } .
The value of \texttt{\ space\ } will be inserted between units if
necessary, \texttt{\ thousandsep\ } between the thousands, and
\texttt{\ per\ } switches between exponents and fractions.

\subsection{\texorpdfstring{\texttt{\ numrange\ }}{ numrange }}\label{numrange}

\texttt{\ numrange\ } takes two \texttt{\ num\ } s as the first two
arguments. If they have the same exponent, it is automatically
factorized. The range symbol can be changed with \texttt{\ delimiter\ }
, and the space between the numbers and symbols with \texttt{\ space\ }
.

\subsection{\texorpdfstring{\texttt{\ qtyrange\ }}{ qtyrange }}\label{qtyrange}

\texttt{\ qtyrange\ } is just a combination of \texttt{\ unit\ } and
\texttt{\ range\ } .

\subsubsection{How to add}\label{how-to-add}

Copy this into your project and use the import as \texttt{\ unify\ }

\begin{verbatim}
#import "@preview/unify:0.7.0"
\end{verbatim}

\includesvg[width=0.16667in,height=0.16667in]{/assets/icons/16-copy.svg}

Check the docs for
\href{https://typst.app/docs/reference/scripting/\#packages}{more
information on how to import packages} .

\subsubsection{About}\label{about}

\begin{description}
\tightlist
\item[Author :]
Christopher Hecker
\item[License:]
MIT
\item[Current version:]
0.7.0
\item[Last updated:]
November 28, 2024
\item[First released:]
July 27, 2023
\item[Archive size:]
9.04 kB
\href{https://packages.typst.org/preview/unify-0.7.0.tar.gz}{\pandocbounded{\includesvg[keepaspectratio]{/assets/icons/16-download.svg}}}
\item[Repository:]
\href{https://github.com/ChHecker/unify}{GitHub}
\item[Discipline s :]
\begin{itemize}
\tightlist
\item[]
\item
  \href{https://typst.app/universe/search/?discipline=business}{Business}
\item
  \href{https://typst.app/universe/search/?discipline=chemistry}{Chemistry}
\item
  \href{https://typst.app/universe/search/?discipline=computer-science}{Computer
  Science}
\item
  \href{https://typst.app/universe/search/?discipline=economics}{Economics}
\item
  \href{https://typst.app/universe/search/?discipline=engineering}{Engineering}
\item
  \href{https://typst.app/universe/search/?discipline=mathematics}{Mathematics}
\item
  \href{https://typst.app/universe/search/?discipline=physics}{Physics}
\end{itemize}
\item[Categor y :]
\begin{itemize}
\tightlist
\item[]
\item
  \pandocbounded{\includesvg[keepaspectratio]{/assets/icons/16-text.svg}}
  \href{https://typst.app/universe/search/?category=text}{Text}
\end{itemize}
\end{description}

\subsubsection{Where to report issues?}\label{where-to-report-issues}

This package is a project of Christopher Hecker . Report issues on
\href{https://github.com/ChHecker/unify}{their repository} . You can
also try to ask for help with this package on the
\href{https://forum.typst.app}{Forum} .

Please report this package to the Typst team using the
\href{https://typst.app/contact}{contact form} if you believe it is a
safety hazard or infringes upon your rights.

\phantomsection\label{versions}
\subsubsection{Version history}\label{version-history}

\begin{longtable}[]{@{}ll@{}}
\toprule\noalign{}
Version & Release Date \\
\midrule\noalign{}
\endhead
\bottomrule\noalign{}
\endlastfoot
0.7.0 & November 28, 2024 \\
\href{https://typst.app/universe/package/unify/0.6.1/}{0.6.1} & November
18, 2024 \\
\href{https://typst.app/universe/package/unify/0.6.0/}{0.6.0} & May 23,
2024 \\
\href{https://typst.app/universe/package/unify/0.5.0/}{0.5.0} & March
26, 2024 \\
\href{https://typst.app/universe/package/unify/0.4.3/}{0.4.3} & October
22, 2023 \\
\href{https://typst.app/universe/package/unify/0.4.2/}{0.4.2} & October
9, 2023 \\
\href{https://typst.app/universe/package/unify/0.4.1/}{0.4.1} &
September 3, 2023 \\
\href{https://typst.app/universe/package/unify/0.4.0/}{0.4.0} & July 28,
2023 \\
\href{https://typst.app/universe/package/unify/0.1.0/}{0.1.0} & July 27,
2023 \\
\end{longtable}

Typst GmbH did not create this package and cannot guarantee correct
functionality of this package or compatibility with any version of the
Typst compiler or app.


\title{typst.app/universe/package/latedef}

\phantomsection\label{banner}
\section{latedef}\label{latedef}

{ 0.1.0 }

Use now, define later

\phantomsection\label{readme}
\emph{Use now, define later!}

\subsection{Basic usage}\label{basic-usage}

This package exposes a single function, \texttt{\ latedef-setup\ } .

\begin{Shaded}
\begin{Highlighting}[]
\NormalTok{\#let (undef, def) = latedef{-}setup(simple: true)}

\NormalTok{My \#undef is \#undef.}
\NormalTok{\#def("dog")}
\NormalTok{\#def("cool")}
\end{Highlighting}
\end{Shaded}

\pandocbounded{\includegraphics[keepaspectratio]{https://github.com/typst/packages/raw/main/packages/preview/latedef/0.1.0/example-images/1.png}}

Note that the definition doesn’t actually have to come \emph{after}
the usage, but if you want to define something beforehand, you’re
better off using a variable instead.

\begin{Shaded}
\begin{Highlighting}[]
\NormalTok{\#let (undef, def) = latedef{-}setup(simple: true)}

\NormalTok{// Instead of}
\NormalTok{\#def("A")}
\NormalTok{The first letter is \#undef.}

\NormalTok{// you should use}
\NormalTok{\#let A = "A"}
\NormalTok{The first letter is \#A.}
\end{Highlighting}
\end{Shaded}

\pandocbounded{\includegraphics[keepaspectratio]{https://github.com/typst/packages/raw/main/packages/preview/latedef/0.1.0/example-images/2.png}}

\subsection{\texorpdfstring{The \texttt{\ simple\ }
parameter}{The  simple  parameter}}\label{the-simple-parameter}

When \texttt{\ simple:\ false\ } (which is the default),
\texttt{\ undef\ } becomes a function you have to call. It takes an
optional positional or named parameter \texttt{\ id\ } of type
\texttt{\ str\ } , which can be used to define things out of order.

\begin{Shaded}
\begin{Highlighting}[]
\NormalTok{\#let (undef, def) = latedef{-}setup() // or \textasciigrave{}latedef{-}setup(simple: false)\textasciigrave{}}

\NormalTok{// Note that you can still call it without an id, which works just like when \textasciigrave{}simple: true\textasciigrave{}.}
\NormalTok{My letters are \#undef("1"), \#undef(id: "2"), and \#undef().}

\NormalTok{// \textasciigrave{}def\textasciigrave{} now takes one positional and either another positional or a named parameter.}
\NormalTok{\#def("C")}
\NormalTok{\#def(id: "2", "B")}
\NormalTok{\#def("1", "A")}
\end{Highlighting}
\end{Shaded}

\pandocbounded{\includegraphics[keepaspectratio]{https://github.com/typst/packages/raw/main/packages/preview/latedef/0.1.0/example-images/3.png}}

\subsection{\texorpdfstring{The \texttt{\ footnote\ }
parameter}{The  footnote  parameter}}\label{the-footnote-parameter}

This is a convenience feature that automatically wraps
\texttt{\ undef\ } in \texttt{\ footnote\ } , either directly (when
\texttt{\ simple:\ true\ } ) or as a function (when
\texttt{\ simple:\ false\ } ).

This corresponds to LaTeX’s \texttt{\ \textbackslash{}footnotemark\ }
and \texttt{\ \textbackslash{}footnotetext\ } , hence the different
names in the example.

\begin{Shaded}
\begin{Highlighting}[]
\NormalTok{\#let (fmark, ftext) = latedef{-}setup(simple: true, footnote: true)}
\NormalTok{Do\#fmark you\#fmark believe\#fmark in God?\#fmark}

\NormalTok{\#let wdym = "What do you mean"}
\NormalTok{\#ftext[\#wdym "Do"?]}
\NormalTok{\#ftext[\#wdym "you"?]}
\NormalTok{\#ftext[\#wdym "believe"?]}
\NormalTok{\#ftext[And w\#wdym.slice(1) "God"?]}
\end{Highlighting}
\end{Shaded}

\pandocbounded{\includegraphics[keepaspectratio]{https://github.com/typst/packages/raw/main/packages/preview/latedef/0.1.0/example-images/4.png}}

\subsection{\texorpdfstring{The \texttt{\ stand-in\ }
parameter}{The  stand-in  parameter}}\label{the-stand-in-parameter}

This is a function that takes a single positional parameter (
\texttt{\ id\ } ) of type \texttt{\ none\ \textbar{}\ str\ } and
produces a stand-in value that gets shown when a late-defined value is
missing a corresponding definition.

\begin{Shaded}
\begin{Highlighting}[]
\NormalTok{\#let (undef, def) = latedef{-}setup()}
\NormalTok{// This is the default stand{-}in}
\NormalTok{\#undef()}
\NormalTok{\#undef("with an id")}

\NormalTok{// Custom stand{-}in}
\NormalTok{\#let (undef, def) = latedef{-}setup(stand{-}in: id =\textgreater{} emph[No \#id!])}
\NormalTok{\#undef()}
\NormalTok{\#undef("id")}
\end{Highlighting}
\end{Shaded}

\pandocbounded{\includegraphics[keepaspectratio]{https://github.com/typst/packages/raw/main/packages/preview/latedef/0.1.0/example-images/5.png}}

Since \texttt{\ stand-in\ } is a function, which is only called when a
definition is actually missing, you can even set it to panic to enforce
that all late-defined values have a definiton.

\begin{Shaded}
\begin{Highlighting}[]
\NormalTok{\#let (undef, def) = latedef{-}setup(stand{-}in: id =\textgreater{} panic("Missing definition for value with id " + repr(id)))}
\NormalTok{\#undef()}
\NormalTok{\#undef("id")}
\end{Highlighting}
\end{Shaded}

The output will look something like

\begin{verbatim}
error: panicked with: "Missing definition for value with id none"
  ┌─ example.typ:1:50
  │
  │ #let (undef, def) = latedef-setup(stand-in: id => panic("Missing definition for value with id " + repr(id)))
  │                                                   ^^^^^^^^^^^^^^^^^^^^^^^^^^^^^^^^^^^^^^^^^^^^^^^^^^^^^^^^^

error: panicked with: "Missing definition for value with id \"id\""
  ┌─ example.typ:1:50
  │
  │ #let (undef, def) = latedef-setup(stand-in: id => panic("Missing definition for value with id " + repr(id)))
  │                                                   ^^^^^^^^^^^^^^^^^^^^^^^^^^^^^^^^^^^^^^^^^^^^^^^^^^^^^^^^^
\end{verbatim}

And there is no error when everything has a definition:

\begin{Shaded}
\begin{Highlighting}[]
\NormalTok{\#let (undef, def) = latedef{-}setup(stand{-}in: id =\textgreater{} panic("Missing definition for value with id " + repr(id)))}
\NormalTok{\#undef() is \#undef("id").}
\NormalTok{\#def("This")}
\NormalTok{\#def("id", "fine")}
\end{Highlighting}
\end{Shaded}

\pandocbounded{\includegraphics[keepaspectratio]{https://github.com/typst/packages/raw/main/packages/preview/latedef/0.1.0/example-images/6.png}}

\subsection{\texorpdfstring{The \texttt{\ group\ }
parameter}{The  group  parameter}}\label{the-group-parameter}

Sometimes you may want to use multiple instances of \texttt{\ latedef\ }
in parallel. This is done using the \texttt{\ group\ } parameter, which
can be \texttt{\ none\ } (the default) or any \texttt{\ str\ } .

Note that using \texttt{\ footnote:\ true\ } sets the default group to
\texttt{\ "footnote"\ } instead.

\begin{Shaded}
\begin{Highlighting}[]
\NormalTok{// Use a group for the figure stuff...}
\NormalTok{\#let (caption{-}undef, caption) = latedef{-}setup(simple: true, group: "figure")}
\NormalTok{\#let figure = std.figure.with(caption: caption{-}undef)}
\NormalTok{// ...so you can still use the regular mechanism in parallel.}
\NormalTok{\#let (undef, def) = latedef{-}setup(simple: true)}

\NormalTok{\#figure(raw(block: true, lorem(5)))}
\NormalTok{\#caption[The \#undef \_lorem ipsum\_.]}
\NormalTok{\#def("classic")}
\end{Highlighting}
\end{Shaded}

\pandocbounded{\includegraphics[keepaspectratio]{https://github.com/typst/packages/raw/main/packages/preview/latedef/0.1.0/example-images/7.png}}

\subsubsection{How to add}\label{how-to-add}

Copy this into your project and use the import as \texttt{\ latedef\ }

\begin{verbatim}
#import "@preview/latedef:0.1.0"
\end{verbatim}

\includesvg[width=0.16667in,height=0.16667in]{/assets/icons/16-copy.svg}

Check the docs for
\href{https://typst.app/docs/reference/scripting/\#packages}{more
information on how to import packages} .

\subsubsection{About}\label{about}

\begin{description}
\tightlist
\item[Author :]
\href{mailto:realt0mstone@gmail.com}{T0mstone}
\item[License:]
MIT-0
\item[Current version:]
0.1.0
\item[Last updated:]
October 21, 2024
\item[First released:]
October 21, 2024
\item[Minimum Typst version:]
0.11.0
\item[Archive size:]
3.64 kB
\href{https://packages.typst.org/preview/latedef-0.1.0.tar.gz}{\pandocbounded{\includesvg[keepaspectratio]{/assets/icons/16-download.svg}}}
\item[Repository:]
\href{https://codeberg.org/T0mstone/typst-latedef}{Codeberg}
\end{description}

\subsubsection{Where to report issues?}\label{where-to-report-issues}

This package is a project of T0mstone . Report issues on
\href{https://codeberg.org/T0mstone/typst-latedef}{their repository} .
You can also try to ask for help with this package on the
\href{https://forum.typst.app}{Forum} .

Please report this package to the Typst team using the
\href{https://typst.app/contact}{contact form} if you believe it is a
safety hazard or infringes upon your rights.

\phantomsection\label{versions}
\subsubsection{Version history}\label{version-history}

\begin{longtable}[]{@{}ll@{}}
\toprule\noalign{}
Version & Release Date \\
\midrule\noalign{}
\endhead
\bottomrule\noalign{}
\endlastfoot
0.1.0 & October 21, 2024 \\
\end{longtable}

Typst GmbH did not create this package and cannot guarantee correct
functionality of this package or compatibility with any version of the
Typst compiler or app.


\title{typst.app/universe/package/tada}

\phantomsection\label{banner}
\section{tada}\label{tada}

{ 0.1.0 }

Easy, composable tabular data manipulation

\phantomsection\label{readme}
TaDa provides a set of simple but powerful operations on rows of data. A
full manual is available online:
\url{https://github.com/ntjess/typst-tada/blob/v0.1.0/docs/manual.pdf}

Key features include:

\begin{itemize}
\item
  \textbf{Arithmetic expressions} : Row-wise operations are as simple as
  string expressions with field names
\item
  \textbf{Aggregation} : Any function that operates on an array of
  values can perform row-wise or column-wise aggregation
\item
  \textbf{Data representation} : Handle displaying currencies, floats,
  integers, and more with ease and arbitrary customization
\end{itemize}

Note: This library is in early development. The API is subject to change
especially as typst adds more support for user-defined types.
\textbf{Backwards compatibility is not guaranteed!} Handling of field
info, value types, and more may change substantially with more user
feedback.

\subsection{Importing}\label{importing}

TaDa can be imported as follows:

\subsubsection{From the official packages repository
(recommended):}\label{from-the-official-packages-repository-recommended}

\begin{Shaded}
\begin{Highlighting}[]
\NormalTok{\#import "@preview/tada:0.1.0"}
\end{Highlighting}
\end{Shaded}

\subsubsection{From the source code (not
recommended)}\label{from-the-source-code-not-recommended}

\textbf{Option 1:} You can clone the package directly into your project
directory:

\begin{Shaded}
\begin{Highlighting}[]
\CommentTok{\# In your project directory}
\FunctionTok{git}\NormalTok{ clone https://github.com/ntjess/typst{-}tada.git tada}
\end{Highlighting}
\end{Shaded}

Then import the functionality with

\begin{Shaded}
\begin{Highlighting}[]
\NormalTok{\#import "./tada/lib.typ" }
\end{Highlighting}
\end{Shaded}

\textbf{Option 2:} If Python is available on your system, use the
provided packaging script to install TaDa in typst’s
\texttt{\ local\ } directory:

\begin{Shaded}
\begin{Highlighting}[]
\CommentTok{\# Anywhere on your system}
  \FunctionTok{git}\NormalTok{ clone https://github.com/ntjess/typst{-}tada.git}
  \BuiltInTok{cd}\NormalTok{ typst{-}tada}
  
  \CommentTok{\# Replace $XDG\_CACHE\_HOME with the appropriate directory based on}
  \CommentTok{\# https://github.com/typst/packages\#downloads}
  \ExtensionTok{python}\NormalTok{ package.py ./typst.toml }\StringTok{"}\VariableTok{$XDG\_CACHE\_HOME}\StringTok{/typst/packages"} \DataTypeTok{\textbackslash{}}
    \AttributeTok{{-}{-}namespace}\NormalTok{ local}
  
\end{Highlighting}
\end{Shaded}

Now, TaDa is available under the local namespace:

\begin{Shaded}
\begin{Highlighting}[]
\NormalTok{\#import "@local/tada:0.1.0"}
\end{Highlighting}
\end{Shaded}

\subsection{Creation}\label{creation}

TaDa provides three main ways to construct tables â€`` from columns,
rows, or records.

\begin{itemize}
\item
  \textbf{Columns} are a dictionary of field names to column values.
  Alternatively, a 2D array of columns can be passed to
  \texttt{\ from-columns\ } , where \texttt{\ values.at(0)\ } is a
  column (belongs to one field).
\item
  \textbf{Records} are a 1D array of dictionaries where each dictionary
  is a row.
\item
  \textbf{Rows} are a 2D array where \texttt{\ values.at(0)\ } is a row
  (has one value for each field). Note that if \texttt{\ rows\ } are
  given without field names, they default to (0, 1, …\$n\$).
\end{itemize}

\begin{Shaded}
\begin{Highlighting}[]
\NormalTok{\#let column{-}data = (}
\NormalTok{  name: ("Bread", "Milk", "Eggs"),}
\NormalTok{  price: (1.25, 2.50, 1.50),}
\NormalTok{  quantity: (2, 1, 3),}
\NormalTok{)}
\NormalTok{\#let record{-}data = (}
\NormalTok{  (name: "Bread", price: 1.25, quantity: 2),}
\NormalTok{  (name: "Milk", price: 2.50, quantity: 1),}
\NormalTok{  (name: "Eggs", price: 1.50, quantity: 3),}
\NormalTok{)}
\NormalTok{\#let row{-}data = (}
\NormalTok{  ("Bread", 1.25, 2),}
\NormalTok{  ("Milk", 2.50, 1),}
\NormalTok{  ("Eggs", 1.50, 3),}
\NormalTok{)}

\NormalTok{\#import tada: TableData}
\NormalTok{\#let td = TableData(data: column{-}data)}
\NormalTok{// Equivalent to:}
\NormalTok{\#let td2 = tada.from{-}records(record{-}data)}
\NormalTok{// \_Not\_ equivalent to (since field names are unknown):}
\NormalTok{\#let td3 = tada.from{-}rows(row{-}data)}

\NormalTok{\#to{-}tablex(td)}
\NormalTok{\#to{-}tablex(td2)}
\NormalTok{\#to{-}tablex(td3)}
\end{Highlighting}
\end{Shaded}

\pandocbounded{\includegraphics[keepaspectratio]{https://raw.githubusercontent.com/ntjess/typst-tada/v0.1.0/assets/example-01.png}}

\subsection{Title formatting}\label{title-formatting}

You can pass any \texttt{\ content\ } as a field’s \texttt{\ title\ }
. \textbf{Note} : if you pass a string, it will be evaluated as markup.

\begin{Shaded}
\begin{Highlighting}[]
\NormalTok{\#let fmt(it) = \{}
\NormalTok{  heading(outlined: false,}
\NormalTok{    upper(it.at(0))}
\NormalTok{    + it.slice(1).replace("\_", " ")}
\NormalTok{  )}
\NormalTok{\}}

\NormalTok{\#let titles = (}
\NormalTok{  // As a function}
\NormalTok{  name: (title: fmt),}
\NormalTok{  // As a string}
\NormalTok{  quantity: (title: fmt("Qty")),}
\NormalTok{)}
\NormalTok{\#let td = TableData(..td, field{-}info: titles)}

\NormalTok{\#to{-}tablex(td)}
\end{Highlighting}
\end{Shaded}

\pandocbounded{\includegraphics[keepaspectratio]{https://raw.githubusercontent.com/ntjess/typst-tada/v0.1.0/assets/example-02.png}}

\subsection{Adapting default behavior}\label{adapting-default-behavior}

You can specify defaults for any field not explicitly populated by
passing information to \texttt{\ field-defaults\ } . Observe in the last
example that \texttt{\ price\ } was not given a title. We can indicate
it should be formatted the same as \texttt{\ name\ } by passing
\texttt{\ title:\ fmt\ } to \texttt{\ field-defaults\ } . \textbf{Note}
that any field that is explicitly given a value will not be affected by
\texttt{\ field-defaults\ } (i.e., \texttt{\ quantity\ } will retain its
string title “Qty�)

\begin{Shaded}
\begin{Highlighting}[]
\NormalTok{\#let defaults = (title: fmt)}
\NormalTok{\#let td = TableData(..td, field{-}defaults: defaults)}
\NormalTok{\#to{-}tablex(td)}
\end{Highlighting}
\end{Shaded}

\pandocbounded{\includegraphics[keepaspectratio]{https://raw.githubusercontent.com/ntjess/typst-tada/v0.1.0/assets/example-03.png}}

\subsection{\texorpdfstring{Using
\texttt{\ \_\_index\ }}{Using  \_\_index }}\label{using-__index}

TaDa will automatically add an \texttt{\ \_\_index\ } field to each row
that is hidden by default. If you want it displayed, update its
information to set \texttt{\ hide:\ false\ } :

\begin{Shaded}
\begin{Highlighting}[]
\NormalTok{// Use the helper function \textasciigrave{}update{-}fields\textasciigrave{} to update multiple fields}
\NormalTok{// and/or attributes}
\NormalTok{\#import tada: update{-}fields}
\NormalTok{\#let td = update{-}fields(}
\NormalTok{  td, \_\_index: (hide: false, title: "\textbackslash{}\#")}
\NormalTok{)}
\NormalTok{// You can also insert attributes directly:}
\NormalTok{// \#td.field{-}info.\_\_index.insert("hide", false)}
\NormalTok{// etc.}
\NormalTok{\#to{-}tablex(td)}
\end{Highlighting}
\end{Shaded}

\pandocbounded{\includegraphics[keepaspectratio]{https://raw.githubusercontent.com/ntjess/typst-tada/v0.1.0/assets/example-04.png}}

\subsection{Value formatting}\label{value-formatting}

\subsubsection{\texorpdfstring{\texttt{\ type\ }}{ type }}\label{type}

Type information can have attached metadata that specifies alignment,
display formats, and more. Available types and their metadata are:

\begin{itemize}
\item
  \textbf{string} : (default-value: "", align: left)
\item
  \textbf{content} : (display: , align: left)
\item
  \textbf{float} : (align: right)
\item
  \textbf{integer} : (align: right)
\item
  \textbf{percent} : (display: , align: right)
\item
  \textbf{index} : (align: right)
\end{itemize}

While adding your own default types is not yet supported, you can simply
defined a dictionary of specifications and pass its keys to the field

\begin{Shaded}
\begin{Highlighting}[]
\NormalTok{\#let currency{-}info = (}
\NormalTok{  display: tada.display.format{-}usd, align: right}
\NormalTok{)}
\NormalTok{\#td.field{-}info.insert("price", (type: "currency"))}
\NormalTok{\#let td = TableData(..td, type{-}info: ("currency": currency{-}info))}
\NormalTok{\#to{-}tablex(td)}
\end{Highlighting}
\end{Shaded}

\pandocbounded{\includegraphics[keepaspectratio]{https://raw.githubusercontent.com/ntjess/typst-tada/v0.1.0/assets/example-05.png}}

\subsection{Transposing}\label{transposing}

\texttt{\ transpose\ } is supported, but keep in mind if columns have
different types, an error will be a frequent result. To avoid the error,
explicitly pass \texttt{\ ignore-types:\ true\ } . You can choose
whether to keep field names as an additional column by passing a string
to \texttt{\ fields-name\ } that is evaluated as markup:

\begin{Shaded}
\begin{Highlighting}[]
\NormalTok{\#to{-}tablex(}
\NormalTok{  tada.transpose(}
\NormalTok{    td, ignore{-}types: true, fields{-}name: ""}
\NormalTok{  )}
\NormalTok{)}
\end{Highlighting}
\end{Shaded}

\pandocbounded{\includegraphics[keepaspectratio]{https://raw.githubusercontent.com/ntjess/typst-tada/v0.1.0/assets/example-06.png}}

\subsubsection{\texorpdfstring{\texttt{\ display\ }}{ display }}\label{display}

If your type is not available or you want to customize its display, pass
a \texttt{\ display\ } function that formats the value, or a string that
accesses \texttt{\ value\ } in its scope:

\begin{Shaded}
\begin{Highlighting}[]
\NormalTok{\#td.field{-}info.at("quantity").insert(}
\NormalTok{  "display",}
\NormalTok{  val =\textgreater{} ("/", "One", "Two", "Three").at(val),}
\NormalTok{)}

\NormalTok{\#let td = TableData(..td)}
\NormalTok{\#to{-}tablex(td)}
\end{Highlighting}
\end{Shaded}

\pandocbounded{\includegraphics[keepaspectratio]{https://raw.githubusercontent.com/ntjess/typst-tada/v0.1.0/assets/example-07.png}}

\subsubsection{\texorpdfstring{\texttt{\ align\ }
etc.}{ align  etc.}}\label{align-etc.}

You can pass \texttt{\ align\ } and \texttt{\ width\ } to a given
field’s metadata to determine how content aligns in the cell and how
much horizontal space it takes up. In the future, more
\texttt{\ tablex\ } setup arguments will be accepted.

\begin{Shaded}
\begin{Highlighting}[]
\NormalTok{\#let adjusted = update{-}fields(}
\NormalTok{  td, name: (align: center, width: 1.4in)}
\NormalTok{)}
\NormalTok{\#to{-}tablex(adjusted)}
\end{Highlighting}
\end{Shaded}

\pandocbounded{\includegraphics[keepaspectratio]{https://raw.githubusercontent.com/ntjess/typst-tada/v0.1.0/assets/example-08.png}}

\subsection{\texorpdfstring{Deeper \texttt{\ tablex\ }
customization}{Deeper  tablex  customization}}\label{deeper-tablex-customization}

TaDa uses \texttt{\ tablex\ } to display the table. So any argument that
\texttt{\ tablex\ } accepts can be passed to TableData as well:

\begin{Shaded}
\begin{Highlighting}[]
\NormalTok{\#let mapper = (index, row) =\textgreater{} \{}
\NormalTok{  let fill = if index == 0 \{rgb("\#8888")\} else \{none\}}
\NormalTok{  row.map(cell =\textgreater{} (..cell, fill: fill))}
\NormalTok{\}}
\NormalTok{\#let td = TableData(}
\NormalTok{  ..td,}
\NormalTok{  tablex{-}kwargs: (}
\NormalTok{    map{-}rows: mapper, auto{-}vlines: false}
\NormalTok{  ),}
\NormalTok{)}
\NormalTok{\#to{-}tablex(td)}
\end{Highlighting}
\end{Shaded}

\pandocbounded{\includegraphics[keepaspectratio]{https://raw.githubusercontent.com/ntjess/typst-tada/v0.1.0/assets/example-09.png}}

\subsection{Subselection}\label{subselection}

You can select a subset of fields or rows to display:

\begin{Shaded}
\begin{Highlighting}[]
\NormalTok{\#import tada: subset}
\NormalTok{\#to{-}tablex(}
\NormalTok{  subset(td, indexes: (0,2), fields: ("name", "price"))}
\NormalTok{)}
\end{Highlighting}
\end{Shaded}

\pandocbounded{\includegraphics[keepaspectratio]{https://raw.githubusercontent.com/ntjess/typst-tada/v0.1.0/assets/example-10.png}}

Note that \texttt{\ indexes\ } is based on the table’s
\texttt{\ \_\_index\ } column, \emph{not} it’s positional index within
the table:

\begin{Shaded}
\begin{Highlighting}[]
\NormalTok{\#let td2 = td}
\NormalTok{\#td2.data.insert("\_\_index", (1, 2, 2))}
\NormalTok{\#to{-}tablex(}
\NormalTok{  subset(td2, indexes: 2, fields: ("\_\_index", "name"))}
\NormalTok{)}
\end{Highlighting}
\end{Shaded}

\pandocbounded{\includegraphics[keepaspectratio]{https://raw.githubusercontent.com/ntjess/typst-tada/v0.1.0/assets/example-11.png}}

Rows can also be selected by whether they fulfill a field condition:

\begin{Shaded}
\begin{Highlighting}[]
\NormalTok{\#to{-}tablex(}
\NormalTok{  tada.filter(td, expression: "price \textless{} 1.5")}
\NormalTok{)}
\end{Highlighting}
\end{Shaded}

\pandocbounded{\includegraphics[keepaspectratio]{https://raw.githubusercontent.com/ntjess/typst-tada/v0.1.0/assets/example-12.png}}

\subsection{Concatenation}\label{concatenation}

Concatenating rows and columns are both supported operations, but only
in the simple sense of stacking the data. Currently, there is no ability
to join on a field or otherwise intelligently merge data.

\begin{itemize}
\item
  \texttt{\ axis:\ 0\ } places new rows below current rows
\item
  \texttt{\ axis:\ 1\ } places new columns to the right of current
  columns
\item
  Unless you specify a fill value for missing values, the function will
  panic if the tables do not match exactly along their concatenation
  axis.
\item
  You cannot stack with \texttt{\ axis:\ 1\ } unless every column has a
  unique field name.
\end{itemize}

\begin{Shaded}
\begin{Highlighting}[]
\NormalTok{\#import tada: stack}

\NormalTok{\#let td2 = TableData(}
\NormalTok{  data: (}
\NormalTok{    name: ("Cheese", "Butter"),}
\NormalTok{    price: (2.50, 1.75),}
\NormalTok{  )}
\NormalTok{)}
\NormalTok{\#let td3 = TableData(}
\NormalTok{  data: (}
\NormalTok{    rating: (4.5, 3.5, 5.0, 4.0, 2.5),}
\NormalTok{  )}
\NormalTok{)}

\NormalTok{// This would fail without specifying the fill}
\NormalTok{// since \textasciigrave{}quantity\textasciigrave{} is missing from \textasciigrave{}td2\textasciigrave{}}
\NormalTok{\#let stack{-}a = stack(td, td2, missing{-}fill: 0)}
\NormalTok{\#let stack{-}b = stack(stack{-}a, td3, axis: 1)}
\NormalTok{\#to{-}tablex(stack{-}b)}
\end{Highlighting}
\end{Shaded}

\pandocbounded{\includegraphics[keepaspectratio]{https://raw.githubusercontent.com/ntjess/typst-tada/v0.1.0/assets/example-13.png}}

\subsection{Expressions}\label{expressions}

The easiest way to leverage TaDa’s flexibility is through expressions.
They can be strings that treat field names as variables, or functions
that take keyword-only arguments.

\begin{itemize}
\tightlist
\item
  \textbf{Note} ! When passing functions, every field is passed as a
  named argument to the function. So, make sure to capture unused fields
  with \texttt{\ ..rest\ } (the name is unimportant) to avoid errors.
\end{itemize}

\begin{Shaded}
\begin{Highlighting}[]
\NormalTok{\#let make{-}dict(field, expression) = \{}
\NormalTok{  let out = (:)}
\NormalTok{  out.insert(}
\NormalTok{    field,}
\NormalTok{    (expression: expression, type: "currency"),}
\NormalTok{  )}
\NormalTok{  out}
\NormalTok{\}}

\NormalTok{\#let td = update{-}fields(}
\NormalTok{  td, ..make{-}dict("total", "price * quantity" )}
\NormalTok{)}

\NormalTok{\#let tax{-}expr(total: none, ..rest) = \{ total * 0.2 \}}
\NormalTok{\#let taxed = update{-}fields(}
\NormalTok{  td, ..make{-}dict("tax", tax{-}expr),}
\NormalTok{)}

\NormalTok{\#to{-}tablex(}
\NormalTok{  subset(taxed, fields: ("name", "total", "tax"))}
\NormalTok{)}
\end{Highlighting}
\end{Shaded}

\pandocbounded{\includegraphics[keepaspectratio]{https://raw.githubusercontent.com/ntjess/typst-tada/v0.1.0/assets/example-14.png}}

\subsection{Chaining}\label{chaining}

It is inconvenient to require several temporary variables as above, or
deep function nesting, to perform multiple operations on a table. TaDa
provides a \texttt{\ chain\ } function to make this easier. Furthermore,
when you need to compute several fields at once and don’t need extra
field information, you can use \texttt{\ add-expressions\ } as a
shorthand:

\begin{Shaded}
\begin{Highlighting}[]
\NormalTok{\#import tada: chain, add{-}expressions}
\NormalTok{\#let totals = chain(td,}
\NormalTok{  add{-}expressions.with(}
\NormalTok{    total: "price * quantity",}
\NormalTok{    tax: "total * 0.2",}
\NormalTok{    after{-}tax: "total + tax",}
\NormalTok{  ),}
\NormalTok{  subset.with(}
\NormalTok{    fields: ("name", "total", "after{-}tax")}
\NormalTok{  ),}
\NormalTok{  // Add type information}
\NormalTok{  update{-}fields.with(}
\NormalTok{    after{-}tax: (type: "currency", title: fmt("w/ Tax")),}
\NormalTok{  ),}
\NormalTok{)}
\NormalTok{\#to{-}tablex(totals)}
\end{Highlighting}
\end{Shaded}

\pandocbounded{\includegraphics[keepaspectratio]{https://raw.githubusercontent.com/ntjess/typst-tada/v0.1.0/assets/example-15.png}}

\subsection{Sorting}\label{sorting}

You can sort by ascending/descending values of any field, or provide
your own transformation function to the \texttt{\ key\ } argument to
customize behavior further:

\begin{Shaded}
\begin{Highlighting}[]
\NormalTok{\#import tada: sort{-}values}
\NormalTok{\#to{-}tablex(sort{-}values(}
\NormalTok{  td, by: "quantity", descending: true}
\NormalTok{))}
\end{Highlighting}
\end{Shaded}

\pandocbounded{\includegraphics[keepaspectratio]{https://raw.githubusercontent.com/ntjess/typst-tada/v0.1.0/assets/example-16.png}}

\subsection{Aggregation}\label{aggregation}

Column-wise reduction is supported through \texttt{\ agg\ } , using
either functions or string expressions:

\begin{Shaded}
\begin{Highlighting}[]
\NormalTok{\#import tada: agg, item}
\NormalTok{\#let grand{-}total = chain(}
\NormalTok{  totals,}
\NormalTok{  agg.with(after{-}tax: array.sum),}
\NormalTok{  // use "item" to extract exactly one element}
\NormalTok{  item}
\NormalTok{)}
\NormalTok{// "Output" is a helper function just for these docs.}
\NormalTok{// It is not necessary in your code.}
\NormalTok{\#output[}
\NormalTok{  *Grand total: \#tada.display.format{-}usd(grand{-}total)*}
\NormalTok{]}
\end{Highlighting}
\end{Shaded}

\pandocbounded{\includegraphics[keepaspectratio]{https://raw.githubusercontent.com/ntjess/typst-tada/v0.1.0/assets/example-17.png}}

It is also easy to aggregate several expressions at once:

\begin{Shaded}
\begin{Highlighting}[]
\NormalTok{\#let agg{-}exprs = (}
\NormalTok{  "\# items": "quantity.sum()",}
\NormalTok{  "Longest name": "[\#name.sorted(key: str.len).at({-}1)]",}
\NormalTok{)}
\NormalTok{\#let agg{-}td = tada.agg(td, ..agg{-}exprs)}
\NormalTok{\#to{-}tablex(agg{-}td)}
\end{Highlighting}
\end{Shaded}

\pandocbounded{\includegraphics[keepaspectratio]{https://raw.githubusercontent.com/ntjess/typst-tada/v0.1.0/assets/example-18.png}}

\subsubsection{How to add}\label{how-to-add}

Copy this into your project and use the import as \texttt{\ tada\ }

\begin{verbatim}
#import "@preview/tada:0.1.0"
\end{verbatim}

\includesvg[width=0.16667in,height=0.16667in]{/assets/icons/16-copy.svg}

Check the docs for
\href{https://typst.app/docs/reference/scripting/\#packages}{more
information on how to import packages} .

\subsubsection{About}\label{about}

\begin{description}
\tightlist
\item[Author :]
Nathan Jessurun
\item[License:]
Unlicense
\item[Current version:]
0.1.0
\item[Last updated:]
December 15, 2023
\item[First released:]
December 15, 2023
\item[Archive size:]
16.2 kB
\href{https://packages.typst.org/preview/tada-0.1.0.tar.gz}{\pandocbounded{\includesvg[keepaspectratio]{/assets/icons/16-download.svg}}}
\item[Repository:]
\href{https://github.com/ntjess/typst-tada}{GitHub}
\end{description}

\subsubsection{Where to report issues?}\label{where-to-report-issues}

This package is a project of Nathan Jessurun . Report issues on
\href{https://github.com/ntjess/typst-tada}{their repository} . You can
also try to ask for help with this package on the
\href{https://forum.typst.app}{Forum} .

Please report this package to the Typst team using the
\href{https://typst.app/contact}{contact form} if you believe it is a
safety hazard or infringes upon your rights.

\phantomsection\label{versions}
\subsubsection{Version history}\label{version-history}

\begin{longtable}[]{@{}ll@{}}
\toprule\noalign{}
Version & Release Date \\
\midrule\noalign{}
\endhead
\bottomrule\noalign{}
\endlastfoot
0.1.0 & December 15, 2023 \\
\end{longtable}

Typst GmbH did not create this package and cannot guarantee correct
functionality of this package or compatibility with any version of the
Typst compiler or app.


\title{typst.app/universe/package/modern-ysu-thesis}

\phantomsection\label{banner}
\phantomsection\label{template-thumbnail}
\pandocbounded{\includegraphics[keepaspectratio]{https://packages.typst.org/preview/thumbnails/modern-ysu-thesis-0.1.0-small.webp}}

\section{modern-ysu-thesis}\label{modern-ysu-thesis}

{ 0.1.0 }

燕山大学学ä½?论æ--‡æ¨¡æ?¿ã€‚Modern Yanshan University Thesis.

\href{/app?template=modern-ysu-thesis&version=0.1.0}{Create project in
app}

\phantomsection\label{readme}
本模�在
\href{https://github.com/nju-lug/modern-nju-thesis}{modern-nju-thesis}
的基础上修æ''¹è€Œæ?¥

\begin{quote}
{[}!WARNING{]}

本模æ?¿æ­£å¤„于积æž?å¼€å?{}`阶段,存在一些æ~¼å¼?é---®é¢˜ï¼Œé€‚å?ˆå°?鲜
Typst 特性

本模æ?¿æ˜¯æ°`é---´æ¨¡æ?¿ï¼Œ \textbf{å?¯èƒ½ä¸?被学æ~¡è®¤å?¯}
,正å¼?使ç''¨è¿‡ç¨‹ä¸­è¯·å?šå¥½éš?æ---¶å°†å†\ldots 容è¿?移至 Word
æˆ-- LaTeX 的准备
\end{quote}

clone 本项目å?Žï¼Œç\ldots§ç?€ template\textbackslash thesis.typ
下写就行

\subsection{致谢}\label{uxe8uxe8}

\begin{itemize}
\tightlist
\item
  æ„Ÿè°¢
  \href{https://github.com/nju-lug/modern-nju-thesis}{modern-nju-thesis}
  Typst 中æ--‡è®ºæ--‡æ¨¡æ?¿ã€‚
\end{itemize}

\subsection{License}\label{license}

This project is licensed under the MIT License.

\href{/app?template=modern-ysu-thesis&version=0.1.0}{Create project in
app}

\subsubsection{How to use}\label{how-to-use}

Click the button above to create a new project using this template in
the Typst app.

You can also use the Typst CLI to start a new project on your computer
using this command:

\begin{verbatim}
typst init @preview/modern-ysu-thesis:0.1.0
\end{verbatim}

\includesvg[width=0.16667in,height=0.16667in]{/assets/icons/16-copy.svg}

\subsubsection{About}\label{about}

\begin{description}
\tightlist
\item[Author :]
Woodman3
\item[License:]
MIT
\item[Current version:]
0.1.0
\item[Last updated:]
May 24, 2024
\item[First released:]
May 24, 2024
\item[Archive size:]
98.8 kB
\href{https://packages.typst.org/preview/modern-ysu-thesis-0.1.0.tar.gz}{\pandocbounded{\includesvg[keepaspectratio]{/assets/icons/16-download.svg}}}
\item[Repository:]
\href{https://github.com/Woodman3/modern-ysu-thesis}{GitHub}
\item[Categor y :]
\begin{itemize}
\tightlist
\item[]
\item
  \pandocbounded{\includesvg[keepaspectratio]{/assets/icons/16-mortarboard.svg}}
  \href{https://typst.app/universe/search/?category=thesis}{Thesis}
\end{itemize}
\end{description}

\subsubsection{Where to report issues?}\label{where-to-report-issues}

This template is a project of Woodman3 . Report issues on
\href{https://github.com/Woodman3/modern-ysu-thesis}{their repository} .
You can also try to ask for help with this template on the
\href{https://forum.typst.app}{Forum} .

Please report this template to the Typst team using the
\href{https://typst.app/contact}{contact form} if you believe it is a
safety hazard or infringes upon your rights.

\phantomsection\label{versions}
\subsubsection{Version history}\label{version-history}

\begin{longtable}[]{@{}ll@{}}
\toprule\noalign{}
Version & Release Date \\
\midrule\noalign{}
\endhead
\bottomrule\noalign{}
\endlastfoot
0.1.0 & May 24, 2024 \\
\end{longtable}

Typst GmbH did not create this template and cannot guarantee correct
functionality of this template or compatibility with any version of the
Typst compiler or app.


\title{typst.app/universe/package/basic-resume}

\phantomsection\label{banner}
\phantomsection\label{template-thumbnail}
\pandocbounded{\includegraphics[keepaspectratio]{https://packages.typst.org/preview/thumbnails/basic-resume-0.2.0-small.webp}}

\section{basic-resume}\label{basic-resume}

{ 0.2.0 }

A simple, standard resume, designed to work well with ATS.

{ } Featured Template

\href{/app?template=basic-resume&version=0.2.0}{Create project in app}

\phantomsection\label{readme}
Version 0.2.0

This is a template for a simple resume. It is intended to be used as a
good starting point for quickly crafting a standard resume that will
properly be parsed by ATS systems. Inspiration is taken from
\href{https://github.com/jakegut/resume}{Jake’s Resume} and
\href{https://typst.app/universe/package/guided-resume-starter-cgc/}{guided-resume-starter-cgc}
. I’m currently a college student and was unable to find a Typst
resume template that fit my needs, so I wrote my own. I hope this
template can be useful to others as well.

\subsection{Sample Resume}\label{sample-resume}

\pandocbounded{\includegraphics[keepaspectratio]{https://raw.githubusercontent.com/stuxf/basic-typst-resume-template/main/example-resume.png}}

\subsection{Quick Start}\label{quick-start}

A barebones resume looks like this, which you can use to get started.

\begin{Shaded}
\begin{Highlighting}[]
\NormalTok{\#import "@preview/basic{-}resume:0.2.0": *}

\NormalTok{// Put your personal information here, replacing mine}
\NormalTok{\#let name = "Stephen Xu"}
\NormalTok{\#let location = "San Diego, CA"}
\NormalTok{\#let email = "stxu@hmc.edu"}
\NormalTok{\#let github = "github.com/stuxf"}
\NormalTok{\#let linkedin = "linkedin.com/in/stuxf"}
\NormalTok{\#let phone = "+1 (xxx) xxx{-}xxxx"}
\NormalTok{\#let personal{-}site = "stuxf.dev"}

\NormalTok{\#show: resume.with(}
\NormalTok{  author: name,}
\NormalTok{  // All the lines below are optional. }
\NormalTok{  // For example, if you want to to hide your phone number:}
\NormalTok{  // feel free to comment those lines out and they will not show.}
\NormalTok{  location: location,}
\NormalTok{  email: email,}
\NormalTok{  github: github,}
\NormalTok{  linkedin: linkedin,}
\NormalTok{  phone: phone,}
\NormalTok{  personal{-}site: personal{-}site,}
\NormalTok{  accent{-}color: "\#26428b",}
\NormalTok{  font: "New Computer Modern",}
\NormalTok{)}

\NormalTok{/*}
\NormalTok{* Lines that start with == are formatted into section headings}
\NormalTok{* You can use the specific formatting functions if needed}
\NormalTok{* The following formatting functions are listed below}
\NormalTok{* \#edu(dates: "", degree: "", gpa: "", institution: "", location: "")}
\NormalTok{* \#work(company: "", dates: "", location: "", title: "")}
\NormalTok{* \#project(dates: "", name: "", role: "", url: "")}
\NormalTok{* \#extracurriculars(activity: "", dates: "")}
\NormalTok{* There are also the following generic functions that don\textquotesingle{}t apply any formatting}
\NormalTok{* \#generic{-}two{-}by{-}two(top{-}left: "", top{-}right: "", bottom{-}left: "", bottom{-}right: "")}
\NormalTok{* \#generic{-}one{-}by{-}two(left: "", right: "")}
\NormalTok{*/}
\NormalTok{== Education}

\NormalTok{\#edu(}
\NormalTok{  institution: "Harvey Mudd College",}
\NormalTok{  location: "Claremont, CA",}
\NormalTok{  dates: dates{-}helper(start{-}date: "Aug 2023", end{-}date: "May 2027"),}
\NormalTok{  degree: "Bachelor\textquotesingle{}s of Science, Computer Science and Mathematics",}
\NormalTok{)}
\NormalTok{{-} Cumulative GPA: 4.0\textbackslash{}/4.0 | Dean\textquotesingle{}s List, Harvey S. Mudd Merit Scholarship, National Merit Scholarship}
\NormalTok{{-} Relevant Coursework: Data Structures, Program Development, Microprocessors, Abstract Algebra I: Groups and Rings, Linear Algebra, Discrete Mathematics, Multivariable \& Single Variable Calculus, Principles and Practice of Comp Sci}

\NormalTok{== Work Experience}

\NormalTok{\#work(}
\NormalTok{  title: "Subatomic Shepherd and Caffeine Connoisseur",}
\NormalTok{  location: "Atomville, CA",}
\NormalTok{  company: "Microscopic Circus, Schrodinger\textquotesingle{}s University",}
\NormalTok{  dates: dates{-}helper(start{-}date: "May 2024", end{-}date: "Present"),}
\NormalTok{)}
\NormalTok{{-} more bullet points go here}

\NormalTok{// ... more headers and stuff below}
\end{Highlighting}
\end{Shaded}

\href{/app?template=basic-resume&version=0.2.0}{Create project in app}

\subsubsection{How to use}\label{how-to-use}

Click the button above to create a new project using this template in
the Typst app.

You can also use the Typst CLI to start a new project on your computer
using this command:

\begin{verbatim}
typst init @preview/basic-resume:0.2.0
\end{verbatim}

\includesvg[width=0.16667in,height=0.16667in]{/assets/icons/16-copy.svg}

\subsubsection{About}\label{about}

\begin{description}
\tightlist
\item[Author :]
\href{https://stuxf.dev}{Stephen Xu}
\item[License:]
Unlicense
\item[Current version:]
0.2.0
\item[Last updated:]
November 29, 2024
\item[First released:]
August 1, 2024
\item[Archive size:]
6.05 kB
\href{https://packages.typst.org/preview/basic-resume-0.2.0.tar.gz}{\pandocbounded{\includesvg[keepaspectratio]{/assets/icons/16-download.svg}}}
\item[Repository:]
\href{https://github.com/stuxf/basic-typst-resume-template}{GitHub}
\item[Categor y :]
\begin{itemize}
\tightlist
\item[]
\item
  \pandocbounded{\includesvg[keepaspectratio]{/assets/icons/16-user.svg}}
  \href{https://typst.app/universe/search/?category=cv}{CV}
\end{itemize}
\end{description}

\subsubsection{Where to report issues?}\label{where-to-report-issues}

This template is a project of Stephen Xu . Report issues on
\href{https://github.com/stuxf/basic-typst-resume-template}{their
repository} . You can also try to ask for help with this template on the
\href{https://forum.typst.app}{Forum} .

Please report this template to the Typst team using the
\href{https://typst.app/contact}{contact form} if you believe it is a
safety hazard or infringes upon your rights.

\phantomsection\label{versions}
\subsubsection{Version history}\label{version-history}

\begin{longtable}[]{@{}ll@{}}
\toprule\noalign{}
Version & Release Date \\
\midrule\noalign{}
\endhead
\bottomrule\noalign{}
\endlastfoot
0.2.0 & November 29, 2024 \\
\href{https://typst.app/universe/package/basic-resume/0.1.4/}{0.1.4} &
November 12, 2024 \\
\href{https://typst.app/universe/package/basic-resume/0.1.3/}{0.1.3} &
October 15, 2024 \\
\href{https://typst.app/universe/package/basic-resume/0.1.2/}{0.1.2} &
October 7, 2024 \\
\href{https://typst.app/universe/package/basic-resume/0.1.0/}{0.1.0} &
August 1, 2024 \\
\end{longtable}

Typst GmbH did not create this template and cannot guarantee correct
functionality of this template or compatibility with any version of the
Typst compiler or app.


\title{typst.app/universe/package/classic-jmlr}

\phantomsection\label{banner}
\phantomsection\label{template-thumbnail}
\pandocbounded{\includegraphics[keepaspectratio]{https://packages.typst.org/preview/thumbnails/classic-jmlr-0.4.0-small.webp}}

\section{classic-jmlr}\label{classic-jmlr}

{ 0.4.0 }

Paper template for submission to Journal of Machine Learning Research
(JMLR)

\href{/app?template=classic-jmlr&version=0.4.0}{Create project in app}

\phantomsection\label{readme}
\subsection{Overview}\label{overview}

This is a Typst template for Journal of Machine Learning Research
(JMLR). It is based on official
\href{https://www.jmlr.org/format/authors-guide.html}{author guide} ,
\href{https://www.jmlr.org/format/format.html}{formatting instructions}
, and
\href{https://www.jmlr.org/format/formatting-errors.html}{formatting
error checklist} as well as the official
\href{https://github.com/jmlrorg/jmlr-style-file}{example paper} .

\subsection{Usage}\label{usage}

You can use this template in the Typst web app by clicking \emph{Start
from template} on the dashboard and searching for
\texttt{\ classic-jmlr\ } .

Alternatively, you can use the CLI to kick this project off using the
command

\begin{Shaded}
\begin{Highlighting}[]
\NormalTok{typst init @preview/classic{-}jmlr}
\end{Highlighting}
\end{Shaded}

Typst will create a new directory with all the files needed to get you
started.

\subsection{Configuration}\label{configuration}

This template exports the \texttt{\ jmlr\ } function with the following
named arguments.

\begin{itemize}
\tightlist
\item
  \texttt{\ title\ } : The paper’s title as content.
\item
  \texttt{\ short-title\ } : Paper short title (for page header).
\item
  \texttt{\ authors\ } : An array of author dictionaries. Each of the
  author dictionaries must have a name key and can have the keys
  department, organization, location, and email.
\item
  \texttt{\ last-names\ } : List of authors last names (for page
  header).
\item
  \texttt{\ keywords\ } : Publication keywords (used in PDF metadata).
\item
  \texttt{\ date\ } : Creation date (used in PDF metadata).
\item
  \texttt{\ abstract\ } : The content of a brief summary of the paper or
  none. Appears at the top under the title.
\item
  \texttt{\ bibliography\ } : The result of a call to the bibliography
  function or none. The function also accepts a single, positional
  argument for the body of the paper.
\item
  \texttt{\ appendix\ } : Content to append after bibliography section.
\item
  \texttt{\ pubdata\ } : Dictionary with auxiliary information about
  publication. It contains editor name(s), paper id, volume, and
  submission/review/publishing dates.
\end{itemize}

The template will initialize your package with a sample call to the
\texttt{\ jmlr\ } function in a show rule. If you want to change an
existing project to use this template, you can add a show rule at the
top of your file.

\begin{Shaded}
\begin{Highlighting}[]
\NormalTok{\#import "@preview/classic{-}jmlr": jmlr}
\NormalTok{\#show: jmlr.with(}
\NormalTok{  title: [Sample JMLR Paper],}
\NormalTok{  authors: (authors, affls),}
\NormalTok{  abstract: blindtext,}
\NormalTok{  keywords: ("keyword one", "keyword two", "keyword three"),}
\NormalTok{  bibliography: bibliography("main.bib"),}
\NormalTok{  appendix: include "appendix.typ",}
\NormalTok{  pubdata: (}
\NormalTok{    id: "21{-}0000",}
\NormalTok{    editor: "My editor",}
\NormalTok{    volume: 23,}
\NormalTok{    submitted{-}at: datetime(year: 2021, month: 1, day: 1),}
\NormalTok{    revised{-}at: datetime(year: 2022, month: 5, day: 1),}
\NormalTok{    published{-}at: datetime(year: 2022, month: 9, day: 1),}
\NormalTok{  ),}
\NormalTok{)}
\end{Highlighting}
\end{Shaded}

\subsection{Issues}\label{issues}

This template is developed at
\href{https://github.com/daskol/typst-templates}{daskol/typst-templates}
repo. Please report all issues there.

\begin{itemize}
\item
  Original JMLR example paper is not not representative. It does not
  demonstrate appearance of figures, images, tables, lists, etc.
\item
  Leading in author affilations in in the original template is varying.
\item
  There is no bibliography CSL-style. The closest one is
  \texttt{\ bristol-university-press\ } .
\item
  Another issue is related to Typst’s inablity to produce colored
  annotation. In order to mitigte the issue, we add a script which
  modifies annotations and make them colored.

\begin{Shaded}
\begin{Highlighting}[]
\NormalTok{../colorize{-}annotations.py \textbackslash{}}
\NormalTok{    example{-}paper.typst.pdf example{-}paper{-}colored.typst.pdf}
\end{Highlighting}
\end{Shaded}

  See
  \href{https://github.com/daskol/typst-templates/\#colored-annotations}{README.md}
  for details.
\end{itemize}

\href{/app?template=classic-jmlr&version=0.4.0}{Create project in app}

\subsubsection{How to use}\label{how-to-use}

Click the button above to create a new project using this template in
the Typst app.

You can also use the Typst CLI to start a new project on your computer
using this command:

\begin{verbatim}
typst init @preview/classic-jmlr:0.4.0
\end{verbatim}

\includesvg[width=0.16667in,height=0.16667in]{/assets/icons/16-copy.svg}

\subsubsection{About}\label{about}

\begin{description}
\tightlist
\item[Author :]
\href{mailto:d.bershatsky2@skoltech.ru}{Daniel Bershatsky}
\item[License:]
MIT
\item[Current version:]
0.4.0
\item[Last updated:]
April 19, 2024
\item[First released:]
April 19, 2024
\item[Minimum Typst version:]
0.11.0
\item[Archive size:]
8.60 kB
\href{https://packages.typst.org/preview/classic-jmlr-0.4.0.tar.gz}{\pandocbounded{\includesvg[keepaspectratio]{/assets/icons/16-download.svg}}}
\item[Repository:]
\href{https://github.com/daskol/typst-templates}{GitHub}
\item[Discipline s :]
\begin{itemize}
\tightlist
\item[]
\item
  \href{https://typst.app/universe/search/?discipline=computer-science}{Computer
  Science}
\item
  \href{https://typst.app/universe/search/?discipline=mathematics}{Mathematics}
\end{itemize}
\item[Categor y :]
\begin{itemize}
\tightlist
\item[]
\item
  \pandocbounded{\includesvg[keepaspectratio]{/assets/icons/16-atom.svg}}
  \href{https://typst.app/universe/search/?category=paper}{Paper}
\end{itemize}
\end{description}

\subsubsection{Where to report issues?}\label{where-to-report-issues}

This template is a project of Daniel Bershatsky . Report issues on
\href{https://github.com/daskol/typst-templates}{their repository} . You
can also try to ask for help with this template on the
\href{https://forum.typst.app}{Forum} .

Please report this template to the Typst team using the
\href{https://typst.app/contact}{contact form} if you believe it is a
safety hazard or infringes upon your rights.

\phantomsection\label{versions}
\subsubsection{Version history}\label{version-history}

\begin{longtable}[]{@{}ll@{}}
\toprule\noalign{}
Version & Release Date \\
\midrule\noalign{}
\endhead
\bottomrule\noalign{}
\endlastfoot
0.4.0 & April 19, 2024 \\
\end{longtable}

Typst GmbH did not create this template and cannot guarantee correct
functionality of this template or compatibility with any version of the
Typst compiler or app.


\title{typst.app/universe/package/shane-hhu-thesis}

\phantomsection\label{banner}
\phantomsection\label{template-thumbnail}
\pandocbounded{\includegraphics[keepaspectratio]{https://packages.typst.org/preview/thumbnails/shane-hhu-thesis-0.2.0-small.webp}}

\section{shane-hhu-thesis}\label{shane-hhu-thesis}

{ 0.2.0 }

河海大学本ç§`ç''Ÿæ¯•ä¸šè®ºæ--‡ï¼ˆè®¾è®¡ï¼‰æ¨¡æ?¿ã€‚Unofficial Hohai
University Undergraduate Thesis (Design) Template.

\href{/app?template=shane-hhu-thesis&version=0.2.0}{Create project in
app}

\phantomsection\label{readme}
使ç''¨ Typst
制作的河海大学「本ç§`毕业设计(论æ--‡ï¼‰æŠ¥å`Šã€?模æ?¿ï¼ˆå·¥ç§`)。官æ--¹æ¨¡æ?¿å?‚考
\href{https://bylw.hhu.edu.cn/UpLoadFile/83cd5f1169974a0db06d865c7ee11af4.pdf}{河海大学本ç§`毕业设计(论æ--‡ï¼‰è§„范æ~¼å¼?å?‚考}

\begin{quote}
{[}!IMPORTANT{]}

此模æ?¿é?žå®˜æ--¹æ¨¡æ?¿ï¼Œå?¯èƒ½ä»?存在一些é---®é¢˜ï¼Œå?Žç»­ä¼šä¸?æ--­æ›´æ--°å®Œå--„。

此模æ?¿ä»\ldots 适ç''¨äºŽå·¥ç§`ä¸``业本ç§`毕业论æ--‡ï¼ˆè®¾è®¡ï¼‰ï¼Œå?Žç»­å?¯èƒ½ä¼šæ›´æ--°æ--‡ç§`模æ?¿ã€‚

本模æ?¿ä½¿ç''¨ Typst 0.12.x ç¼--è¯`,Typst
æ›´æ--°é¢`率较高,å?¯èƒ½å‡ºçŽ°ç‰ˆæœ¬æ›´æ--°å?Žæ---~法ç¼--è¯`æˆ?功的æƒ\ldots 况。
\end{quote}

\pandocbounded{\includegraphics[keepaspectratio]{https://github.com/typst/packages/raw/main/packages/preview/shane-hhu-thesis/0.2.0/demo_images/title.png}}

\subsection{使ç''¨æ--¹æ³•}\label{uxe4uxbduxe7uxe6uxb9uxe6uxb3}

模æ?¿å·²ä¸Šä¼~ Typst Universe ,å?¯ä»¥ä½¿ç''¨ \texttt{\ typst\ init\ }
功能åˆ?始åŒ--,也å?¯ä»¥ä½¿ç''¨ Web APP ç¼--è¾`。 \textbf{Typst
Universe
上的模æ?¿å?¯èƒ½ä¸?是最æ--°ç‰ˆæœ¬ã€‚如果需è¦?使ç''¨æœ€æ--°ç‰ˆæœ¬çš„模æ?¿ï¼Œä»Žæœ¬
repo 中获å?--。}

\paragraph{本地使ç''¨ï¼ˆæŽ¨è??)}\label{uxe6ux153uxe5ux153uxe4uxbduxe7uxefuxbcux2c6uxe6ux17euxe8uxefuxbc}

使ç''¨å‰?,请å\ldots ˆå®‰è£
\texttt{\ https://github.com/shaneworld/Dots/tree/master/fonts\ }
中的å\ldots¨éƒ¨å­---ä½``。

\begin{itemize}
\item
  å\ldots‹éš†æœ¬ repo 到本地,ç¼--è¾` \texttt{\ init-files\ }
  目录å†\ldots çš„æ--‡ä»¶ã€‚
\item
  使ç''¨ \texttt{\ typst\ init\ @preview/shane-hhu-thesis:0.1.0\ }
  本地åˆ?始åŒ--模æ?¿ã€‚
\end{itemize}

\paragraph{Web APP å†\ldots 使ç''¨}\label{web-app-uxe5uxe4uxbduxe7}

ç''±äºŽ Typst Web APP
在æ¯?次æ‰``开页é?¢çš„æ---¶å€™éƒ½ä¼šä»Žæœ?务器中下载å­---ä½``,速度较æ\ldots¢ï¼Œä½``验较差,å›~æ­¤ä¸?建议使ç''¨æ­¤æ--¹æ³•ã€‚

在 \href{https://typst.app/universe/package/shane-hhu-thesis}{Typst
Universe} 中点击 \texttt{\ Create\ project\ in\ app\ }
按é'®è¿›å\ldots¥ Web APP å†\ldots 。

然�,请将
\texttt{\ https://github.com/shaneworld/Dots/tree/master/fonts\ }
å†\ldots 的所有å­---ä½``上ä¼~到 Web APP
å†\ldots 该项目的æ~¹ç›®å½•å?ŽæŒ‰ç\ldots§æ??示使ç''¨ã€‚

\subsection{模æ?¿å†\ldots 容}\label{uxe6uxe6uxe5uxe5uxb9}

æ­¤ Typst 模æ?¿æŒ‰ç\ldots§
\href{https://bylw.hhu.edu.cn/UpLoadFile/83cd5f1169974a0db06d865c7ee11af4.pdf}{《河海大学本ç§`毕业设计(论æ--‡ï¼‰åŸºæœ¬è§„范(修订)》}
制作,制作æ---¶å?‚考了
\href{https://github.com/csimide/SEU-Typst-Template}{东å?---大学制作的
Typst 模�} 。

ç›®å‰?åŒ\ldots å?«ä»¥ä¸‹é¡µé?¢ï¼š

\begin{itemize}
\tightlist
\item
  {[}x{]} 中英æ--‡å°?é?¢
\item
  {[}x{]} éƒ`é‡?声明
\item
  {[}x{]} 中英æ--‡æ`˜è¦?
\item
  {[}x{]} 目录
\item
  {[}x{]} æ­£æ--‡
\item
  {[}x{]} 致谢
\item
  {[}x{]} å?‚考æ--‡çŒ®
\item
  {[}x{]} 附录
\end{itemize}

此论æ--‡æ¨¡æ?¿ä¸?ä»\ldots 适ç''¨äºŽæœ¬ç§`ç''Ÿæ¯•ä¸šè®ºæ--‡/设计,å?Œæ~·é€‚ç''¨äºŽå¹³æ---¶çš„课程报å`Šç­‰è§„范å†\ldots 容。å?¯ä»¥é€šè¿‡è‡ªå®šä¹‰
\texttt{\ form\ }
å­---段更æ''¹è®ºæ--‡ç§?类,有以下3ç§?æ~¼å¼?å?¯ä¾›é€‰æ‹©ï¼š

\begin{itemize}
\tightlist
\item
  \texttt{\ thesis\ } :毕业论æ--‡
\item
  \texttt{\ design\ } :毕业设计
\item
  \texttt{\ report\ } :课程报å`Š
\end{itemize}

å?¯ä»¥é€šè¿‡ä¿®æ''¹ \texttt{\ heading\ }
å­---段修æ''¹é¡µçœ‰å†\ldots 容,修æ''¹ \texttt{\ thesis-name\ }
下的 \texttt{\ CN\ } å­---段修æ''¹å°?é?¢é¡µé?¢å±•ç¤ºçš„æ~‡é¢˜ã€‚

如果å?{}`现模æ?¿çš„é---®é¢˜ï¼Œæ¬¢è¿Žæ??交 issue。

\subsection{致谢}\label{uxe8uxe8}

\begin{itemize}
\item
  东å?---大学论æ--‡æ¨¡æ?¿ï¼š
  \href{https://github.com/csimide/SEU-Typst-Template}{csimide/SEU-Typst-Template}
\item
  åŒ---京大学本ç§`ç''Ÿæ¯•ä¸šè®ºæ--‡æ¨¡æ?¿ï¼š
  \href{https://github.com/sigongzi/pkuthss-typst-undergraduate}{sigongzi/pkuthss-typst-undergraduate}
\end{itemize}

\href{/app?template=shane-hhu-thesis&version=0.2.0}{Create project in
app}

\subsubsection{How to use}\label{how-to-use}

Click the button above to create a new project using this template in
the Typst app.

You can also use the Typst CLI to start a new project on your computer
using this command:

\begin{verbatim}
typst init @preview/shane-hhu-thesis:0.2.0
\end{verbatim}

\includesvg[width=0.16667in,height=0.16667in]{/assets/icons/16-copy.svg}

\subsubsection{About}\label{about}

\begin{description}
\tightlist
\item[Author :]
shane
\item[License:]
MIT
\item[Current version:]
0.2.0
\item[Last updated:]
November 18, 2024
\item[First released:]
November 14, 2024
\item[Archive size:]
297 kB
\href{https://packages.typst.org/preview/shane-hhu-thesis-0.2.0.tar.gz}{\pandocbounded{\includesvg[keepaspectratio]{/assets/icons/16-download.svg}}}
\item[Repository:]
\href{https://github.com/shaneworld/HHU-Thesis-Template}{GitHub}
\item[Categor y :]
\begin{itemize}
\tightlist
\item[]
\item
  \pandocbounded{\includesvg[keepaspectratio]{/assets/icons/16-mortarboard.svg}}
  \href{https://typst.app/universe/search/?category=thesis}{Thesis}
\end{itemize}
\end{description}

\subsubsection{Where to report issues?}\label{where-to-report-issues}

This template is a project of shane . Report issues on
\href{https://github.com/shaneworld/HHU-Thesis-Template}{their
repository} . You can also try to ask for help with this template on the
\href{https://forum.typst.app}{Forum} .

Please report this template to the Typst team using the
\href{https://typst.app/contact}{contact form} if you believe it is a
safety hazard or infringes upon your rights.

\phantomsection\label{versions}
\subsubsection{Version history}\label{version-history}

\begin{longtable}[]{@{}ll@{}}
\toprule\noalign{}
Version & Release Date \\
\midrule\noalign{}
\endhead
\bottomrule\noalign{}
\endlastfoot
0.2.0 & November 18, 2024 \\
\href{https://typst.app/universe/package/shane-hhu-thesis/0.1.0/}{0.1.0}
& November 14, 2024 \\
\end{longtable}

Typst GmbH did not create this template and cannot guarantee correct
functionality of this template or compatibility with any version of the
Typst compiler or app.


\title{typst.app/universe/package/touying-pres-ustc}

\phantomsection\label{banner}
\phantomsection\label{template-thumbnail}
\pandocbounded{\includegraphics[keepaspectratio]{https://packages.typst.org/preview/thumbnails/touying-pres-ustc-0.1.0-small.webp}}

\section{touying-pres-ustc}\label{touying-pres-ustc}

{ 0.1.0 }

Touying Slide Theme for USTC

\href{/app?template=touying-pres-ustc&version=0.1.0}{Create project in
app}

\phantomsection\label{readme}
\textbf{\href{http://www.xn--fiqs8srwby7cba020i2hih02b.com/}{www.中国ç§`学技术大学.com}}

为了é™?低æ--°æ‰‹ä¸Šè·¯çš„代ç~?分æž?æˆ?本以å?Šé™?低项目的耦å?ˆæ€§ï¼Œç‰¹å°†é¡¹ç›®æ‹†åˆ†æˆ?多个typæ--‡ä»¶ï¼Œè¯·æŒ‰ä»¥ä¸‹æ­¥éª¤äº†è§£ï¼š

\texttt{\ main.typ\ }
是渲æŸ``çš„å\ldots¥å?£ï¼Œæ--°æ‰‹ä¸Šè·¯ç›´æŽ¥ä»Žè¿™é‡Œå¼€å§‹ã€‚//第一步

\texttt{\ main.typ\ } 中导å\ldots¥äº† \texttt{\ config.typ\ }
,这是é\ldots?ç½®æ--‡ä»¶ï¼Œå°?é?¢çš„æ~‡é¢˜ï¼Œå‰¯æ~‡é¢˜ï¼Œä½œè€\ldots 等信æ?¯åœ¨è¿™é‡Œä¿®æ''¹ï¼Œå\ldots¶ä»--ä½~æ---~须烦æ?¼ã€‚//第二步

\texttt{\ config.typ\ } 中导å\ldots¥äº† \texttt{\ theme.typ\ } å'Œ
\texttt{\ lib.typ\ } ,æ--°æ‰‹ä¸Šè·¯å?¯ä»¥æš‚æ---¶è·³è¿‡ã€‚

\begin{itemize}
\tightlist
\item
  \texttt{\ theme.typ\ } 相å½``于 \texttt{\ CSS\ }
  ,控制ç?€æ¨¡æ?¿é•¿ä»€ä¹ˆæ~·ï¼Œå¦‚æžœä½~对模æ?¿çš„æ~·å¼?ä¸?满æ„?,åŒ\ldots 括å­---ä½``,图片ç´~æ??,跳转功能,æˆ--è€\ldots æ''¹æˆ?å\ldots¶ä»--å­¦æ~¡çš„主题,在这里修æ''¹æˆ?ä½~希望的æ~·å­?。学ä¹~曲线较陡峭,åˆ?å­¦è€\ldots æ---~需å\ldots³å¿ƒï¼Œ
\item
  \texttt{\ lib.typ\ }
  中是第三æ--¹åŒ\ldots å'Œè‡ªå®šä¹‰å‡½æ•°ï¼Œå½``模æ?¿ä¸­çš„åŒ\ldots ä¸?能满足ä½~的需求,æˆ--è€\ldots 想自定义一些常ç''¨å‡½æ•°ï¼Œè¯·æŠŠå®ƒä»¬æ''¾åœ¨è¿™é‡Œä»¥ä¿?æŒ?项目的ç»``构性å'Œä½Žè€¦å?ˆã€‚
\end{itemize}

\texttt{\ content.typ\ }
这是ä½~æ''¾å®žé™\ldots å†\ldots 容的地æ--¹ï¼Œä¸€çº§æ~‡é¢˜æ˜¯å¤§çº²ï¼ŒäºŒçº§æ~‡é¢˜æ˜¯å½``页的æ~‡é¢˜ï¼Œä¹‹å?Žçš„n级æ~‡é¢˜éƒ½åœ¨é¡µå†\ldots 作为å†\ldots 容显示,æ---~特殊地ä½?。//第三步

\begin{itemize}
\item
  \texttt{\ assets\ } : 模æ?¿èµ„æº?æ--‡ä»¶

  \begin{itemize}
  \tightlist
  \item
    \texttt{\ img\ } : 模æ?¿å›¾ç‰‡æ--‡ä»¶
  \end{itemize}
\item
  \texttt{\ template\ } : å?¯å¤?ç''¨ç»„件

  \begin{itemize}
  \tightlist
  \item
    {[}{]}
  \end{itemize}
\item
  \texttt{\ config.typ\ } :
  é\ldots?ç½®æ--‡ä»¶ï¼ŒåŒ\ldots 括å°?é?¢çš„ä¿¡æ?¯åœ¨è¿™é‡Œã€‚
\item
  \texttt{\ content.typ\ } : å?ªéœ€åœ¨æ­¤å¤„æ·»åŠ~å†\ldots 容
\item
  \texttt{\ lib.typ\ } :
  åº``æ--‡ä»¶ï¼Œå¯¼å\ldots¥ç¬¬ä¸‰æ--¹åº``请在这导å\ldots¥
\item
  \texttt{\ main.typ\ } :
  ç¼--è¯`å\ldots¥å?£ï¼Œå¯¼å\ldots¥ç¬¬ä¸‰æ--¹åº``æ---¶æœ‰å?¯èƒ½éœ€è¦?把
  \texttt{\ \#show\ } æ''¾åœ¨æ­¤å¤„
\item
  \texttt{\ theme.typ\ } : 主题æ--‡ä»¶ï¼Œå?¯è‡ªè¡Œä¿®æ''¹æ~·å¼?
\end{itemize}

\begin{enumerate}
\tightlist
\item
  vscode下载æ?'件 \texttt{\ Typst\ LSP\ } ,
  \texttt{\ Tinymist\ Typst\ } , \texttt{\ Typst\ Sync\ } 。
\item
  æ‰``开本项目,在 \texttt{\ main.typ\ } 中点击
  \texttt{\ preview\ } 。若在 \texttt{\ content.typ\ } 中点击
  \texttt{\ preview\ } ,则预览类似 \texttt{\ markdown\ }
  çš„æ--‡æ¡£æŽ'版。
\end{enumerate}

\begin{enumerate}
\tightlist
\item
  在使ç''¨æ---¶ï¼Œ \texttt{\ config.typ\ }
  åº''作为顶层æ--‡ä»¶è¢«å\ldots¶ä»--æ--‡ä»¶å¯¼å\ldots¥ï¼Œé?¿å\ldots?交å?‰å¯¼å\ldots¥ã€‚
\end{enumerate}

\begin{enumerate}
\tightlist
\item
\end{enumerate}

有空å†?写代ç~?注释

\href{/app?template=touying-pres-ustc&version=0.1.0}{Create project in
app}

\subsubsection{How to use}\label{how-to-use}

Click the button above to create a new project using this template in
the Typst app.

You can also use the Typst CLI to start a new project on your computer
using this command:

\begin{verbatim}
typst init @preview/touying-pres-ustc:0.1.0
\end{verbatim}

\includesvg[width=0.16667in,height=0.16667in]{/assets/icons/16-copy.svg}

\subsubsection{About}\label{about}

\begin{description}
\tightlist
\item[Author :]
\href{https://github.com/Quaternijkon}{Quaternijkon}
\item[License:]
MIT
\item[Current version:]
0.1.0
\item[Last updated:]
November 26, 2024
\item[First released:]
November 26, 2024
\item[Archive size:]
240 kB
\href{https://packages.typst.org/preview/touying-pres-ustc-0.1.0.tar.gz}{\pandocbounded{\includesvg[keepaspectratio]{/assets/icons/16-download.svg}}}
\item[Repository:]
\href{https://github.com/Quaternijkon/Typst_USTC_CS}{GitHub}
\item[Categor y :]
\begin{itemize}
\tightlist
\item[]
\item
  \pandocbounded{\includesvg[keepaspectratio]{/assets/icons/16-presentation.svg}}
  \href{https://typst.app/universe/search/?category=presentation}{Presentation}
\end{itemize}
\end{description}

\subsubsection{Where to report issues?}\label{where-to-report-issues}

This template is a project of Quaternijkon . Report issues on
\href{https://github.com/Quaternijkon/Typst_USTC_CS}{their repository} .
You can also try to ask for help with this template on the
\href{https://forum.typst.app}{Forum} .

Please report this template to the Typst team using the
\href{https://typst.app/contact}{contact form} if you believe it is a
safety hazard or infringes upon your rights.

\phantomsection\label{versions}
\subsubsection{Version history}\label{version-history}

\begin{longtable}[]{@{}ll@{}}
\toprule\noalign{}
Version & Release Date \\
\midrule\noalign{}
\endhead
\bottomrule\noalign{}
\endlastfoot
0.1.0 & November 26, 2024 \\
\end{longtable}

Typst GmbH did not create this template and cannot guarantee correct
functionality of this template or compatibility with any version of the
Typst compiler or app.


\title{typst.app/universe/package/versatile-apa}

\phantomsection\label{banner}
\phantomsection\label{template-thumbnail}
\pandocbounded{\includegraphics[keepaspectratio]{https://packages.typst.org/preview/thumbnails/versatile-apa-7.0.0-small.webp}}

\section{versatile-apa}\label{versatile-apa}

{ 7.0.0 }

Comprehensive APA 7th Edition Style Template for Typst, suitable for
both student and professional papers.

\href{/app?template=versatile-apa&version=7.0.0}{Create project in app}

\phantomsection\label{readme}
APA 7th Edition template for Typst. This template is based on the
official APA 7th Edition style guide and includes all the necessary
elements for a research paper. It is designed to be versatile and can be
used for any type of research paper, including essays, theses, and
dissertations.

\subsection{Usage}\label{usage}

To use this template, you can use the CLI tool:

\begin{Shaded}
\begin{Highlighting}[]
\ExtensionTok{typst}\NormalTok{ init @preview/versatile{-}apa}
\end{Highlighting}
\end{Shaded}

\subsubsection{Features}\label{features}

The template allows you to easily create academic students for both
student and professional versions of APA 7th Edition:

\begin{itemize}
\tightlist
\item
  Title page
\item
  Abstract
\item
  Localization
\item
  Headings
\item
  Raw/computer code
\item
  Math equations
\item
  Appendices
\item
  References
\item
  Quotation blocks (40 words or more)
\item
  Figures and tables
\item
  Lists
\item
  Footnotes
\item
  Authoring:

  \begin{itemize}
  \tightlist
  \item
    Automatic footnotes for author/affiliation
  \item
    Author notes
  \item
    ORCID
  \end{itemize}
\end{itemize}

\subsection{Planned Features}\label{planned-features}

As of now, the template is in its initial stages and will be updated
with more features in the future. Some of the planned features include:

\begin{itemize}
\tightlist
\item
  \textbf{LaTeX \texttt{\ apa7\ } class full support} : This template is
  inspired by the \texttt{\ apa7\ } class in LaTeX, and it’s planned
  to also include support for all 4 formats of APA (student,
  professional, journal, and manuscript).
\item
  \textbf{Figures notes} : Improved support for all 3 types of APA notes
  (general, specific, probability).
\end{itemize}

\subsection{License}\label{license}

Package licensed under the MIT License. See the repository for more
information.

\href{/app?template=versatile-apa&version=7.0.0}{Create project in app}

\subsubsection{How to use}\label{how-to-use}

Click the button above to create a new project using this template in
the Typst app.

You can also use the Typst CLI to start a new project on your computer
using this command:

\begin{verbatim}
typst init @preview/versatile-apa:7.0.0
\end{verbatim}

\includesvg[width=0.16667in,height=0.16667in]{/assets/icons/16-copy.svg}

\subsubsection{About}\label{about}

\begin{description}
\tightlist
\item[Author :]
\href{https://github.com/jassielof}{Jassiel Ovando}
\item[License:]
MIT-0
\item[Current version:]
7.0.0
\item[Last updated:]
November 4, 2024
\item[First released:]
November 4, 2024
\item[Minimum Typst version:]
0.12.0
\item[Archive size:]
812 kB
\href{https://packages.typst.org/preview/versatile-apa-7.0.0.tar.gz}{\pandocbounded{\includesvg[keepaspectratio]{/assets/icons/16-download.svg}}}
\item[Repository:]
\href{https://github.com/jassielof/typst-templates}{GitHub}
\item[Discipline :]
\begin{itemize}
\tightlist
\item[]
\item
  \href{https://typst.app/universe/search/?discipline=psychology}{Psychology}
\end{itemize}
\item[Categor ies :]
\begin{itemize}
\tightlist
\item[]
\item
  \pandocbounded{\includesvg[keepaspectratio]{/assets/icons/16-atom.svg}}
  \href{https://typst.app/universe/search/?category=paper}{Paper}
\item
  \pandocbounded{\includesvg[keepaspectratio]{/assets/icons/16-speak.svg}}
  \href{https://typst.app/universe/search/?category=report}{Report}
\end{itemize}
\end{description}

\subsubsection{Where to report issues?}\label{where-to-report-issues}

This template is a project of Jassiel Ovando . Report issues on
\href{https://github.com/jassielof/typst-templates}{their repository} .
You can also try to ask for help with this template on the
\href{https://forum.typst.app}{Forum} .

Please report this template to the Typst team using the
\href{https://typst.app/contact}{contact form} if you believe it is a
safety hazard or infringes upon your rights.

\phantomsection\label{versions}
\subsubsection{Version history}\label{version-history}

\begin{longtable}[]{@{}ll@{}}
\toprule\noalign{}
Version & Release Date \\
\midrule\noalign{}
\endhead
\bottomrule\noalign{}
\endlastfoot
7.0.0 & November 4, 2024 \\
\end{longtable}

Typst GmbH did not create this template and cannot guarantee correct
functionality of this template or compatibility with any version of the
Typst compiler or app.


\title{typst.app/universe/package/stack-pointer}

\phantomsection\label{banner}
\section{stack-pointer}\label{stack-pointer}

{ 0.1.0 }

A library for visualizing the execution of (imperative) computer
programs

{ } Featured Package

\phantomsection\label{readme}
Stack Pointer is a library for visualizing the execution of (imperative)
computer programs, particularly in terms of effects on the call stack:
stack frames and local variables therein.

Stack Pointer lets you represent an example program (e.g. a C or Java
program) using typst code with minimal hassle, and get the execution
state of that program at different points in time. For example, the
following C program

\begin{Shaded}
\begin{Highlighting}[]
\DataTypeTok{int}\NormalTok{ main}\OperatorTok{()} \OperatorTok{\{}
  \DataTypeTok{int}\NormalTok{ x }\OperatorTok{=}\NormalTok{ foo}\OperatorTok{();}
  \ControlFlowTok{return} \DecValTok{0}\OperatorTok{;}
\OperatorTok{\}}

\DataTypeTok{int}\NormalTok{ foo}\OperatorTok{()} \OperatorTok{\{}
  \ControlFlowTok{return} \DecValTok{0}\OperatorTok{;}
\OperatorTok{\}}
\end{Highlighting}
\end{Shaded}

would be represented by the following Typst code (see the
\href{https://github.com/typst/packages/raw/main/packages/preview/stack-pointer/0.1.0/docs/manual.pdf}{manual}
for a detailled explanation):

\begin{Shaded}
\begin{Highlighting}[]
\NormalTok{\#let steps = execute(\{}
\NormalTok{  let foo() = func("foo", 6, l =\textgreater{} \{}
\NormalTok{    l(0)}
\NormalTok{    l(1); retval(0)}
\NormalTok{  \})}
\NormalTok{  let main() = func("main", 1, l =\textgreater{} \{}
\NormalTok{    l(0)}
\NormalTok{    l(1)}
\NormalTok{    let (x, ..rest) = foo(); rest}
\NormalTok{    l(1, push("x", x))}
\NormalTok{    l(2)}
\NormalTok{  \})}
\NormalTok{  main(); l(none)}
\NormalTok{\})}
\end{Highlighting}
\end{Shaded}

The \texttt{\ steps\ } variable now contains an array, where each
element corresponds to one of the mentioned lines of code.

Take a look at
\href{https://github.com/typst/packages/raw/main/packages/preview/stack-pointer/0.1.0/gallery/sum.pdf}{this
complete example} of using Stack Pointer together with
\href{https://polylux.dev/book/}{Polylux} .

\subsubsection{How to add}\label{how-to-add}

Copy this into your project and use the import as
\texttt{\ stack-pointer\ }

\begin{verbatim}
#import "@preview/stack-pointer:0.1.0"
\end{verbatim}

\includesvg[width=0.16667in,height=0.16667in]{/assets/icons/16-copy.svg}

Check the docs for
\href{https://typst.app/docs/reference/scripting/\#packages}{more
information on how to import packages} .

\subsubsection{About}\label{about}

\begin{description}
\tightlist
\item[Author :]
\href{https://github.com/SillyFreak/}{Clemens Koza}
\item[License:]
MIT
\item[Current version:]
0.1.0
\item[Last updated:]
July 15, 2024
\item[First released:]
July 15, 2024
\item[Archive size:]
4.29 kB
\href{https://packages.typst.org/preview/stack-pointer-0.1.0.tar.gz}{\pandocbounded{\includesvg[keepaspectratio]{/assets/icons/16-download.svg}}}
\item[Repository:]
\href{https://github.com/SillyFreak/typst-stack-pointer}{GitHub}
\item[Discipline :]
\begin{itemize}
\tightlist
\item[]
\item
  \href{https://typst.app/universe/search/?discipline=computer-science}{Computer
  Science}
\end{itemize}
\item[Categor ies :]
\begin{itemize}
\tightlist
\item[]
\item
  \pandocbounded{\includesvg[keepaspectratio]{/assets/icons/16-code.svg}}
  \href{https://typst.app/universe/search/?category=scripting}{Scripting}
\item
  \pandocbounded{\includesvg[keepaspectratio]{/assets/icons/16-presentation.svg}}
  \href{https://typst.app/universe/search/?category=presentation}{Presentation}
\end{itemize}
\end{description}

\subsubsection{Where to report issues?}\label{where-to-report-issues}

This package is a project of Clemens Koza . Report issues on
\href{https://github.com/SillyFreak/typst-stack-pointer}{their
repository} . You can also try to ask for help with this package on the
\href{https://forum.typst.app}{Forum} .

Please report this package to the Typst team using the
\href{https://typst.app/contact}{contact form} if you believe it is a
safety hazard or infringes upon your rights.

\phantomsection\label{versions}
\subsubsection{Version history}\label{version-history}

\begin{longtable}[]{@{}ll@{}}
\toprule\noalign{}
Version & Release Date \\
\midrule\noalign{}
\endhead
\bottomrule\noalign{}
\endlastfoot
0.1.0 & July 15, 2024 \\
\end{longtable}

Typst GmbH did not create this package and cannot guarantee correct
functionality of this package or compatibility with any version of the
Typst compiler or app.


\title{typst.app/universe/package/tgm-hit-protocol}

\phantomsection\label{banner}
\phantomsection\label{template-thumbnail}
\pandocbounded{\includegraphics[keepaspectratio]{https://packages.typst.org/preview/thumbnails/tgm-hit-protocol-0.1.0-small.webp}}

\section{tgm-hit-protocol}\label{tgm-hit-protocol}

{ 0.1.0 }

Protocol template for students of the HIT department at TGM Wien

\href{/app?template=tgm-hit-protocol&version=0.1.0}{Create project in
app}

\phantomsection\label{readme}
This is a port of the
\href{https://github.com/TGM-HIT/latex-protocol/}{LaTeX protocol
template} available for students of the information technology
department at the TGM technical secondary school in Vienna.

\subsection{Getting Started}\label{getting-started}

Using the Typst web app, you can create a project by e.g. using this
link: \url{https://typst.app/?template=tgm-hit-protocol&version=latest}
.

To work locally, use the following command:

\begin{Shaded}
\begin{Highlighting}[]
\ExtensionTok{typst}\NormalTok{ init @preview/tgm{-}hit{-}protocol}
\end{Highlighting}
\end{Shaded}

\subsection{Usage}\label{usage}

The template (
\href{https://github.com/typst/packages/raw/main/packages/preview/tgm-hit-protocol/0.1.0/main.pdf}{rendered
PDF} ) contains thesis writing advice (in German) as example content. If
you are looking for the details of this template package’s function,
take a look at the
\href{https://github.com/typst/packages/raw/main/packages/preview/tgm-hit-protocol/0.1.0/docs/manual.pdf}{manual}
.

\href{/app?template=tgm-hit-protocol&version=0.1.0}{Create project in
app}

\subsubsection{How to use}\label{how-to-use}

Click the button above to create a new project using this template in
the Typst app.

You can also use the Typst CLI to start a new project on your computer
using this command:

\begin{verbatim}
typst init @preview/tgm-hit-protocol:0.1.0
\end{verbatim}

\includesvg[width=0.16667in,height=0.16667in]{/assets/icons/16-copy.svg}

\subsubsection{About}\label{about}

\begin{description}
\tightlist
\item[Author s :]
\href{https://github.com/k1W1M4ng0}{Simon Gao} \&
\href{https://github.com/SillyFreak/}{Clemens Koza}
\item[License:]
MIT
\item[Current version:]
0.1.0
\item[Last updated:]
October 10, 2024
\item[First released:]
October 10, 2024
\item[Minimum Typst version:]
0.11.0
\item[Archive size:]
80.4 kB
\href{https://packages.typst.org/preview/tgm-hit-protocol-0.1.0.tar.gz}{\pandocbounded{\includesvg[keepaspectratio]{/assets/icons/16-download.svg}}}
\item[Repository:]
\href{https://github.com/TGM-HIT/typst-protocol}{GitHub}
\item[Discipline :]
\begin{itemize}
\tightlist
\item[]
\item
  \href{https://typst.app/universe/search/?discipline=computer-science}{Computer
  Science}
\end{itemize}
\item[Categor y :]
\begin{itemize}
\tightlist
\item[]
\item
  \pandocbounded{\includesvg[keepaspectratio]{/assets/icons/16-speak.svg}}
  \href{https://typst.app/universe/search/?category=report}{Report}
\end{itemize}
\end{description}

\subsubsection{Where to report issues?}\label{where-to-report-issues}

This template is a project of Simon Gao and Clemens Koza . Report issues
on \href{https://github.com/TGM-HIT/typst-protocol}{their repository} .
You can also try to ask for help with this template on the
\href{https://forum.typst.app}{Forum} .

Please report this template to the Typst team using the
\href{https://typst.app/contact}{contact form} if you believe it is a
safety hazard or infringes upon your rights.

\phantomsection\label{versions}
\subsubsection{Version history}\label{version-history}

\begin{longtable}[]{@{}ll@{}}
\toprule\noalign{}
Version & Release Date \\
\midrule\noalign{}
\endhead
\bottomrule\noalign{}
\endlastfoot
0.1.0 & October 10, 2024 \\
\end{longtable}

Typst GmbH did not create this template and cannot guarantee correct
functionality of this template or compatibility with any version of the
Typst compiler or app.


\title{typst.app/universe/package/unilab}

\phantomsection\label{banner}
\phantomsection\label{template-thumbnail}
\pandocbounded{\includegraphics[keepaspectratio]{https://packages.typst.org/preview/thumbnails/unilab-0.0.2-small.webp}}

\section{unilab}\label{unilab}

{ 0.0.2 }

Lab report

\href{/app?template=unilab&version=0.0.2}{Create project in app}

\phantomsection\label{readme}
Typst Lab Report Template

\subsection{Local debugging}\label{local-debugging}

clone this repo into the
\href{https://github.com/typst/packages?tab=readme-ov-file\#local-packages}{local
package directory} , notice that the version should be specified (e.g.
\texttt{\ .../unilab/0.0.1/\ } )

\subsection{TODO}\label{todo}

\begin{itemize}
\tightlist
\item
  {[} {]} en font support
\item
  {[} {]} support school logo
\end{itemize}

\href{/app?template=unilab&version=0.0.2}{Create project in app}

\subsubsection{How to use}\label{how-to-use}

Click the button above to create a new project using this template in
the Typst app.

You can also use the Typst CLI to start a new project on your computer
using this command:

\begin{verbatim}
typst init @preview/unilab:0.0.2
\end{verbatim}

\includesvg[width=0.16667in,height=0.16667in]{/assets/icons/16-copy.svg}

\subsubsection{About}\label{about}

\begin{description}
\tightlist
\item[Author :]
\href{https://github.com/sjfhsjfh}{sjfhsjfh}
\item[License:]
MIT
\item[Current version:]
0.0.2
\item[Last updated:]
April 6, 2024
\item[First released:]
April 6, 2024
\item[Minimum Typst version:]
0.11.0
\item[Archive size:]
18.9 kB
\href{https://packages.typst.org/preview/unilab-0.0.2.tar.gz}{\pandocbounded{\includesvg[keepaspectratio]{/assets/icons/16-download.svg}}}
\item[Repository:]
\href{https://github.com/sjfhsjfh/unilab}{GitHub}
\item[Categor y :]
\begin{itemize}
\tightlist
\item[]
\item
  \pandocbounded{\includesvg[keepaspectratio]{/assets/icons/16-speak.svg}}
  \href{https://typst.app/universe/search/?category=report}{Report}
\end{itemize}
\end{description}

\subsubsection{Where to report issues?}\label{where-to-report-issues}

This template is a project of sjfhsjfh . Report issues on
\href{https://github.com/sjfhsjfh/unilab}{their repository} . You can
also try to ask for help with this template on the
\href{https://forum.typst.app}{Forum} .

Please report this template to the Typst team using the
\href{https://typst.app/contact}{contact form} if you believe it is a
safety hazard or infringes upon your rights.

\phantomsection\label{versions}
\subsubsection{Version history}\label{version-history}

\begin{longtable}[]{@{}ll@{}}
\toprule\noalign{}
Version & Release Date \\
\midrule\noalign{}
\endhead
\bottomrule\noalign{}
\endlastfoot
0.0.2 & April 6, 2024 \\
\end{longtable}

Typst GmbH did not create this template and cannot guarantee correct
functionality of this template or compatibility with any version of the
Typst compiler or app.


\title{typst.app/universe/package/prequery}

\phantomsection\label{banner}
\section{prequery}\label{prequery}

{ 0.1.0 }

library for extracting metadata for preprocessing from a typst document

\phantomsection\label{readme}
This package helps extracting metadata for preprocessing from a typst
document, for example image URLs for download from the web. Typst
compilations are sandboxed: it is not possible for Typst packages, or
even just a Typst document itself, to access the “ouside world�.
This sandboxing of Typst has good reasons. Yet, it is often convenient
to trade a bit of security for convenience by weakening it. Prequery
helps with that by providing some simple scaffolding for supporting
preprocessing of documents.

Here’s an example for referencing images from the internet:

\begin{Shaded}
\begin{Highlighting}[]
\NormalTok{\#import "@preview/prequery:0.1.0"}

\NormalTok{// toggle this comment or pass \textasciigrave{}{-}{-}input prequery{-}fallback=true\textasciigrave{} to enable fallback}
\NormalTok{// \#prequery.fallback.update(true)}

\NormalTok{\#prequery.image(}
\NormalTok{  "https://en.wikipedia.org/static/images/icons/wikipedia.png",}
\NormalTok{  "assets/wikipedia.png")}
\end{Highlighting}
\end{Shaded}

Using \texttt{\ typst\ query\ } , the image URL(s) are extracted from
the document:

\begin{Shaded}
\begin{Highlighting}[]
\ExtensionTok{typst}\NormalTok{ query }\AttributeTok{{-}{-}input}\NormalTok{ prequery{-}fallback=true }\AttributeTok{{-}{-}field}\NormalTok{ value }\DataTypeTok{\textbackslash{}}
\NormalTok{    main.typ }\StringTok{\textquotesingle{}\textquotesingle{}}
\end{Highlighting}
\end{Shaded}

This will output the following piece of JSON:

\begin{Shaded}
\begin{Highlighting}[]
\OtherTok{[}\FunctionTok{\{}\DataTypeTok{"url"}\FunctionTok{:} \StringTok{"https://en.wikipedia.org/static/images/icons/wikipedia.png"}\FunctionTok{,} \DataTypeTok{"path"}\FunctionTok{:} \StringTok{"assets/wikipedia.png"}\FunctionTok{\}}\OtherTok{]}
\end{Highlighting}
\end{Shaded}

Which can then be used to download all images to the expected locations.

See the
\href{https://github.com/typst/packages/raw/main/packages/preview/prequery/0.1.0/docs/manual.pdf}{manual}
for details.

\subsubsection{How to add}\label{how-to-add}

Copy this into your project and use the import as \texttt{\ prequery\ }

\begin{verbatim}
#import "@preview/prequery:0.1.0"
\end{verbatim}

\includesvg[width=0.16667in,height=0.16667in]{/assets/icons/16-copy.svg}

Check the docs for
\href{https://typst.app/docs/reference/scripting/\#packages}{more
information on how to import packages} .

\subsubsection{About}\label{about}

\begin{description}
\tightlist
\item[Author :]
\href{https://github.com/SillyFreak/}{Clemens Koza}
\item[License:]
MIT
\item[Current version:]
0.1.0
\item[Last updated:]
July 15, 2024
\item[First released:]
July 15, 2024
\item[Archive size:]
3.29 kB
\href{https://packages.typst.org/preview/prequery-0.1.0.tar.gz}{\pandocbounded{\includesvg[keepaspectratio]{/assets/icons/16-download.svg}}}
\item[Repository:]
\href{https://github.com/SillyFreak/typst-prequery}{GitHub}
\item[Categor ies :]
\begin{itemize}
\tightlist
\item[]
\item
  \pandocbounded{\includesvg[keepaspectratio]{/assets/icons/16-code.svg}}
  \href{https://typst.app/universe/search/?category=scripting}{Scripting}
\item
  \pandocbounded{\includesvg[keepaspectratio]{/assets/icons/16-hammer.svg}}
  \href{https://typst.app/universe/search/?category=utility}{Utility}
\end{itemize}
\end{description}

\subsubsection{Where to report issues?}\label{where-to-report-issues}

This package is a project of Clemens Koza . Report issues on
\href{https://github.com/SillyFreak/typst-prequery}{their repository} .
You can also try to ask for help with this package on the
\href{https://forum.typst.app}{Forum} .

Please report this package to the Typst team using the
\href{https://typst.app/contact}{contact form} if you believe it is a
safety hazard or infringes upon your rights.

\phantomsection\label{versions}
\subsubsection{Version history}\label{version-history}

\begin{longtable}[]{@{}ll@{}}
\toprule\noalign{}
Version & Release Date \\
\midrule\noalign{}
\endhead
\bottomrule\noalign{}
\endlastfoot
0.1.0 & July 15, 2024 \\
\end{longtable}

Typst GmbH did not create this package and cannot guarantee correct
functionality of this package or compatibility with any version of the
Typst compiler or app.


\title{typst.app/universe/package/bloated-neurips}

\phantomsection\label{banner}
\phantomsection\label{template-thumbnail}
\pandocbounded{\includegraphics[keepaspectratio]{https://packages.typst.org/preview/thumbnails/bloated-neurips-0.5.1-small.webp}}

\section{bloated-neurips}\label{bloated-neurips}

{ 0.5.1 }

NeurIPS-style paper template to publish at the Conference and Workshop
on Neural Information Processing Systems

\href{/app?template=bloated-neurips&version=0.5.1}{Create project in
app}

\phantomsection\label{readme}
\subsection{Usage}\label{usage}

You can use this template in the Typst web app by clicking \emph{Start
from template} on the dashboard and searching for
\texttt{\ bloated-neurips\ } .

Alternatively, you can use the CLI to kick this project off using the
command

\begin{Shaded}
\begin{Highlighting}[]
\NormalTok{typst init @preview/bloated{-}neurips}
\end{Highlighting}
\end{Shaded}

Typst will create a new directory with all the files needed to get you
started.

\subsection{Configuration}\label{configuration}

This template exports the \texttt{\ neurips2023\ } and
\texttt{\ neurips2024\ } function with the following named arguments.

\begin{itemize}
\tightlist
\item
  \texttt{\ title\ } : The paper’s title as content.
\item
  \texttt{\ authors\ } : An array of author dictionaries. Each of the
  author dictionaries must have a name key and can have the keys
  department, organization, location, and email.
\item
  \texttt{\ abstract\ } : The content of a brief summary of the paper or
  none. Appears at the top under the title.
\item
  \texttt{\ bibliography\ } : The result of a call to the bibliography
  function or none. The function also accepts a single, positional
  argument for the body of the paper.
\item
  \texttt{\ appendix\ } : A content which is placed after bibliography.
\item
  \texttt{\ accepted\ } : If this is set to \texttt{\ false\ } then
  anonymized ready for submission document is produced;
  \texttt{\ accepted:\ true\ } produces camera-redy version. If the
  argument is set to \texttt{\ none\ } then preprint version is produced
  (can be uploaded to arXiv).
\end{itemize}

The template will initialize your package with a sample call to the
\texttt{\ neurips2024\ } function in a show rule. If you want to change
an existing project to use this template, you can add a show rule at the
top of your file as follows.

\begin{Shaded}
\begin{Highlighting}[]
\NormalTok{\#import "@preview/bloated{-}neurips:0.5.1": neurips2024}

\NormalTok{\#show: neurips2024.with(}
\NormalTok{  title: [Formatting Instructions For NeurIPS 2024],}
\NormalTok{  authors: (authors, affls),}
\NormalTok{  keywords: ("Machine Learning", "NeurIPS"),}
\NormalTok{  abstract: [}
\NormalTok{    The abstract paragraph should be indented ½ inch (3 picas) on both the}
\NormalTok{    left{-} and right{-}hand margins. Use 10 point type, with a vertical spacing}
\NormalTok{    (leading) of 11 points. The word *Abstract* must be centered, bold, and in}
\NormalTok{    point size 12. Two line spaces precede the abstract. The abstract must be}
\NormalTok{    limited to one paragraph.}
\NormalTok{  ],}
\NormalTok{  bibliography: bibliography("main.bib"),}
\NormalTok{  appendix: [}
\NormalTok{    \#include "appendix.typ"}
\NormalTok{    \#include "checklist.typ"}
\NormalTok{  ],}
\NormalTok{  accepted: false,}
\NormalTok{)}

\NormalTok{\#lorem(42)}
\end{Highlighting}
\end{Shaded}

With template of version v0.5.1 or newer, one can override some parts.
Specifically, \texttt{\ get-notice\ } entry of \texttt{\ aux\ }
directory parameter of show rule allows to adjust the NeurIPS 2024
template to Science4DL workshop as follows.

\begin{Shaded}
\begin{Highlighting}[]
\NormalTok{\#import "@preview/bloated{-}neurips:0.5.1": neurips}

\NormalTok{\#let get{-}notice(accepted) = if accepted == none \{}
\NormalTok{  return [Preprint. Under review.]}
\NormalTok{\} else if accepted \{}
\NormalTok{  return [}
\NormalTok{    Workshop on Scientific Methods for Understanding Deep Learning, NeurIPS}
\NormalTok{    2024.}
\NormalTok{  ]}
\NormalTok{\} else \{}
\NormalTok{  return [}
\NormalTok{    Submitted to Workshop on Scientific Methods for Understanding Deep}
\NormalTok{    Learning, NeurIPS 2024.}
\NormalTok{  ]}
\NormalTok{\}}

\NormalTok{\#let science4dl2024(}
\NormalTok{  title: [], authors: (), keywords: (), date: auto, abstract: none,}
\NormalTok{  bibliography: none, appendix: none, accepted: false, body,}
\NormalTok{) = \{}
\NormalTok{  show: neurips.with(}
\NormalTok{    title: title,}
\NormalTok{    authors: authors,}
\NormalTok{    keywords: keywords,}
\NormalTok{    date: date,}
\NormalTok{    abstract: abstract,}
\NormalTok{    accepted: false,}
\NormalTok{    aux: (get{-}notice: get{-}notice),}
\NormalTok{  )}
\NormalTok{  body}
\NormalTok{\}}
\end{Highlighting}
\end{Shaded}

\subsection{Issues}\label{issues}

\begin{itemize}
\item
  The biggest and the most important issues is related to line ruler. We
  are not aware of universal method for numbering lines in the main body
  of a paper. Fortunately, line numbering support has been implemented
  at \href{https://github.com/typst/typst/pull/4516}{typst/typst\#4516}
  . Let’s wait for the next major release v0.12.0!
\item
  There is an issue in Typst with spacing between figures and between
  figure with floating placement. The issue is that there is no way to
  specify gap between subsequent figures. In order to have behaviour
  similar to original LaTeX template, one should consider direct spacing
  adjacemnt with \texttt{\ v(-1em)\ } as follows.

\begin{Shaded}
\begin{Highlighting}[]
\NormalTok{\#figure(}
\NormalTok{  rect(width: 4.25cm, height: 4.25cm, stroke: 0.4pt),}
\NormalTok{  caption: [Sample figure caption.\#v({-}1em)],}
\NormalTok{  placement: top,}
\NormalTok{)}
\NormalTok{\#figure(}
\NormalTok{  rect(width: 4.25cm, height: 4.25cm, stroke: 0.4pt),}
\NormalTok{  caption: [Sample figure caption.],}
\NormalTok{  placement: top,}
\NormalTok{)}
\end{Highlighting}
\end{Shaded}
\item
  Another issue is related to Typst’s inablity to produce colored
  annotation. In order to mitigte the issue, we add a script which
  modifies annotations and make them colored.

\begin{Shaded}
\begin{Highlighting}[]
\NormalTok{../colorize{-}annotations.py \textbackslash{}}
\NormalTok{    example{-}paper.typst.pdf example{-}paper{-}colored.typst.pdf}
\end{Highlighting}
\end{Shaded}

  See
  \href{https://github.com/daskol/typst-templates/\#colored-annotations}{README.md}
  for details.
\item
  NeurIPS 2023/2024 instructions discuss bibliography in vague terms.
  Namely, there is not specific requirements. Thus we stick to
  \texttt{\ ieee\ } bibliography style since we found it in several
  accepted papers and it is similar to that in the example paper.
\item
  It is unclear how to render notice in the bottom of the title page in
  case of final ( \texttt{\ accepted:\ true\ } ) or preprint (
  \texttt{\ accepted:\ none\ } ) submission.
\end{itemize}

\href{/app?template=bloated-neurips&version=0.5.1}{Create project in
app}

\subsubsection{How to use}\label{how-to-use}

Click the button above to create a new project using this template in
the Typst app.

You can also use the Typst CLI to start a new project on your computer
using this command:

\begin{verbatim}
typst init @preview/bloated-neurips:0.5.1
\end{verbatim}

\includesvg[width=0.16667in,height=0.16667in]{/assets/icons/16-copy.svg}

\subsubsection{About}\label{about}

\begin{description}
\tightlist
\item[Author :]
\href{mailto:daniel.bershatsky2@skoltech.ru}{Daniel Bershatsky}
\item[License:]
MIT
\item[Current version:]
0.5.1
\item[Last updated:]
October 8, 2024
\item[First released:]
March 19, 2024
\item[Minimum Typst version:]
0.11.1
\item[Archive size:]
21.2 kB
\href{https://packages.typst.org/preview/bloated-neurips-0.5.1.tar.gz}{\pandocbounded{\includesvg[keepaspectratio]{/assets/icons/16-download.svg}}}
\item[Repository:]
\href{https://github.com/daskol/typst-templates}{GitHub}
\item[Discipline s :]
\begin{itemize}
\tightlist
\item[]
\item
  \href{https://typst.app/universe/search/?discipline=computer-science}{Computer
  Science}
\item
  \href{https://typst.app/universe/search/?discipline=mathematics}{Mathematics}
\end{itemize}
\item[Categor y :]
\begin{itemize}
\tightlist
\item[]
\item
  \pandocbounded{\includesvg[keepaspectratio]{/assets/icons/16-atom.svg}}
  \href{https://typst.app/universe/search/?category=paper}{Paper}
\end{itemize}
\end{description}

\subsubsection{Where to report issues?}\label{where-to-report-issues}

This template is a project of Daniel Bershatsky . Report issues on
\href{https://github.com/daskol/typst-templates}{their repository} . You
can also try to ask for help with this template on the
\href{https://forum.typst.app}{Forum} .

Please report this template to the Typst team using the
\href{https://typst.app/contact}{contact form} if you believe it is a
safety hazard or infringes upon your rights.

\phantomsection\label{versions}
\subsubsection{Version history}\label{version-history}

\begin{longtable}[]{@{}ll@{}}
\toprule\noalign{}
Version & Release Date \\
\midrule\noalign{}
\endhead
\bottomrule\noalign{}
\endlastfoot
0.5.1 & October 8, 2024 \\
\href{https://typst.app/universe/package/bloated-neurips/0.5.0/}{0.5.0}
& September 22, 2024 \\
\href{https://typst.app/universe/package/bloated-neurips/0.2.1/}{0.2.1}
& March 19, 2024 \\
\end{longtable}

Typst GmbH did not create this template and cannot guarantee correct
functionality of this template or compatibility with any version of the
Typst compiler or app.


\title{typst.app/universe/package/tablem}

\phantomsection\label{banner}
\section{tablem}\label{tablem}

{ 0.1.0 }

Write markdown-like tables easily.

\phantomsection\label{readme}
Write markdown-like tables easily.

\subsection{Example}\label{example}

Have a look at the source
\href{https://github.com/typst/packages/raw/main/packages/preview/tablem/0.1.0/examples/example.typ}{here}
.

\pandocbounded{\includegraphics[keepaspectratio]{https://github.com/typst/packages/raw/main/packages/preview/tablem/0.1.0/examples/example.png}}

\subsection{Usage}\label{usage}

You can simply copy the markdown table and paste it in
\texttt{\ tablem\ } function.

\begin{Shaded}
\begin{Highlighting}[]
\NormalTok{\#import "@preview/tablem:0.1.0": tablem}

\NormalTok{\#tablem[}
\NormalTok{  | *Name* | *Location* | *Height* | *Score* |}
\NormalTok{  | {-}{-}{-}{-}{-}{-} | {-}{-}{-}{-}{-}{-}{-}{-}{-}{-} | {-}{-}{-}{-}{-}{-}{-}{-} | {-}{-}{-}{-}{-}{-}{-} |}
\NormalTok{  | John   | Second St. | 180 cm   |  5      |}
\NormalTok{  | Wally  | Third Av.  | 160 cm   |  10     |}
\NormalTok{]}
\end{Highlighting}
\end{Shaded}

And you can use custom render function.

\begin{Shaded}
\begin{Highlighting}[]
\NormalTok{\#import "@preview/tablex:0.0.6": tablex, hlinex}
\NormalTok{\#import "@preview/tablem:0.1.0": tablem}

\NormalTok{\#let three{-}line{-}table = tablem.with(}
\NormalTok{  render: (columns: auto, ..args) =\textgreater{} \{}
\NormalTok{    tablex(}
\NormalTok{      columns: columns,}
\NormalTok{      auto{-}lines: false,}
\NormalTok{      align: center + horizon,}
\NormalTok{      hlinex(y: 0),}
\NormalTok{      hlinex(y: 1),}
\NormalTok{      ..args,}
\NormalTok{      hlinex(),}
\NormalTok{    )}
\NormalTok{  \}}
\NormalTok{)}

\NormalTok{\#three{-}line{-}table[}
\NormalTok{  | *Name* | *Location* | *Height* | *Score* |}
\NormalTok{  | {-}{-}{-}{-}{-}{-} | {-}{-}{-}{-}{-}{-}{-}{-}{-}{-} | {-}{-}{-}{-}{-}{-}{-}{-} | {-}{-}{-}{-}{-}{-}{-} |}
\NormalTok{  | John   | Second St. | 180 cm   |  5      |}
\NormalTok{  | Wally  | Third Av.  | 160 cm   |  10     |}
\NormalTok{]}
\end{Highlighting}
\end{Shaded}

\pandocbounded{\includegraphics[keepaspectratio]{https://github.com/typst/packages/raw/main/packages/preview/tablem/0.1.0/examples/example.png}}

\subsection{\texorpdfstring{\texttt{\ tablem\ }
function}{ tablem  function}}\label{tablem-function}

\begin{Shaded}
\begin{Highlighting}[]
\NormalTok{\#let tablem(}
\NormalTok{  render: table,}
\NormalTok{  ignore{-}second{-}row: true,}
\NormalTok{  ..args,}
\NormalTok{  body}
\NormalTok{) = \{ .. \}}
\end{Highlighting}
\end{Shaded}

\textbf{Arguments:}

\begin{itemize}
\tightlist
\item
  \texttt{\ render\ } : {[}
  \texttt{\ (columns:\ int,\ ..args)\ =\textgreater{}\ \{\ ..\ \}\ } {]}
  â€'' Custom render function, default to be \texttt{\ table\ } ,
  receiving a integer-type columns, which is the count of first row.
  \texttt{\ ..args\ } is the combination of \texttt{\ args\ } of
  \texttt{\ tablem\ } function and children genenerated from
  \texttt{\ body\ } .
\item
  \texttt{\ ignore-second-row\ } : {[} \texttt{\ boolean\ } {]} â€''
  Whether to ignore the second row (something like
  \texttt{\ \textbar{}-\/-\/-\textbar{}\ } ).
\item
  \texttt{\ args\ } : {[} \texttt{\ any\ } {]} â€'' Some arguments you
  want to pass to \texttt{\ render\ } function.
\item
  \texttt{\ body\ } : {[} \texttt{\ content\ } {]} â€'' The
  markdown-like table. There should be no extra line breaks in it.
\end{itemize}

\subsection{Limitations}\label{limitations}

Cell merging has not yet been implemented.

\subsection{License}\label{license}

This project is licensed under the MIT License.

\subsubsection{How to add}\label{how-to-add}

Copy this into your project and use the import as \texttt{\ tablem\ }

\begin{verbatim}
#import "@preview/tablem:0.1.0"
\end{verbatim}

\includesvg[width=0.16667in,height=0.16667in]{/assets/icons/16-copy.svg}

Check the docs for
\href{https://typst.app/docs/reference/scripting/\#packages}{more
information on how to import packages} .

\subsubsection{About}\label{about}

\begin{description}
\tightlist
\item[Author :]
OrangeX4
\item[License:]
MIT
\item[Current version:]
0.1.0
\item[Last updated:]
November 18, 2023
\item[First released:]
November 18, 2023
\item[Archive size:]
2.37 kB
\href{https://packages.typst.org/preview/tablem-0.1.0.tar.gz}{\pandocbounded{\includesvg[keepaspectratio]{/assets/icons/16-download.svg}}}
\item[Repository:]
\href{https://github.com/OrangeX4/typst-tablem}{GitHub}
\end{description}

\subsubsection{Where to report issues?}\label{where-to-report-issues}

This package is a project of OrangeX4 . Report issues on
\href{https://github.com/OrangeX4/typst-tablem}{their repository} . You
can also try to ask for help with this package on the
\href{https://forum.typst.app}{Forum} .

Please report this package to the Typst team using the
\href{https://typst.app/contact}{contact form} if you believe it is a
safety hazard or infringes upon your rights.

\phantomsection\label{versions}
\subsubsection{Version history}\label{version-history}

\begin{longtable}[]{@{}ll@{}}
\toprule\noalign{}
Version & Release Date \\
\midrule\noalign{}
\endhead
\bottomrule\noalign{}
\endlastfoot
0.1.0 & November 18, 2023 \\
\end{longtable}

Typst GmbH did not create this package and cannot guarantee correct
functionality of this package or compatibility with any version of the
Typst compiler or app.


\title{typst.app/universe/package/zero}

\phantomsection\label{banner}
\section{zero}\label{zero}

{ 0.3.0 }

Advanced scientific number formatting.

{ } Featured Package

\phantomsection\label{readme}
\emph{Advanced scientific number formatting.}

\href{https://typst.app/universe/package/zero}{\pandocbounded{\includegraphics[keepaspectratio]{https://img.shields.io/badge/dynamic/toml?url=https\%3A\%2F\%2Fraw.githubusercontent.com\%2FMc-Zen\%2Fzero\%2Fv0.3.0\%2Ftypst.toml&query=\%24.package.version&prefix=v&logo=typst&label=package&color=239DAD}}}
\href{https://github.com/Mc-Zen/zero/actions/workflows/run_tests.yml}{\pandocbounded{\includesvg[keepaspectratio]{https://github.com/Mc-Zen/zero/actions/workflows/run_tests.yml/badge.svg}}}
\href{https://github.com/Mc-Zen/zero/blob/main/LICENSE}{\pandocbounded{\includegraphics[keepaspectratio]{https://img.shields.io/badge/license-MIT-blue}}}

\begin{itemize}
\tightlist
\item
  \href{https://github.com/typst/packages/raw/main/packages/preview/zero/0.3.0/\#introduction}{Introduction}
\item
  \href{https://github.com/typst/packages/raw/main/packages/preview/zero/0.3.0/\#quick-demo}{Quick
  Demo}
\item
  \href{https://github.com/typst/packages/raw/main/packages/preview/zero/0.3.0/\#documentation}{Documentation}
\item
  \href{https://github.com/typst/packages/raw/main/packages/preview/zero/0.3.0/\#table-alignment}{Table
  alignment}
\item
  \href{https://github.com/typst/packages/raw/main/packages/preview/zero/0.3.0/\#zero-for-third-party-packages}{Zero
  for third-party packages}
\end{itemize}

\subsection{Introduction}\label{introduction}

Proper number formatting requires some love for detail to guarantee a
readable and clear output. This package provides tools to ensure
consistent formatting and to simplify the process of following
established publication practices. Key features are

\begin{itemize}
\tightlist
\item
  \textbf{standardized} formatting,
\item
  digit
  \href{https://github.com/typst/packages/raw/main/packages/preview/zero/0.3.0/\#grouping}{\textbf{grouping}}
  , e.g., 299 792 458 instead of 299792458,
\item
  \textbf{plug-and-play} number
  \href{https://github.com/typst/packages/raw/main/packages/preview/zero/0.3.0/\#table-alignment}{\textbf{alignment
  in tables}} ,
\item
  quick scientific notation, e.g., \texttt{\ "2e4"\ } becomes
  2Ã---10â?´,
\item
  symmetric and asymmetric
  \href{https://github.com/typst/packages/raw/main/packages/preview/zero/0.3.0/\#specifying-uncertainties}{\textbf{uncertainties}}
  ,
\item
  \href{https://github.com/typst/packages/raw/main/packages/preview/zero/0.3.0/\#rounding}{\textbf{rounding}}
  in various modes,
\item
  and some specials for package authors.
\end{itemize}

A number in scientific notation consists of three parts of which the
latter two are optional. The first part is the \emph{mantissa} that may
consist of an \emph{integer} and a \emph{fractional} part. In many
fields of science, values are not known exactly and the corresponding
\emph{uncertainty} is then given along with the mantissa. Lastly, to
facilitate reading very large or small numbers, the mantissa may be
multiplied with a \emph{power} of 10 (or another base).

The anatomy of a formatted number is shown in the following figure.

\pandocbounded{\includegraphics[keepaspectratio]{https://github.com/user-attachments/assets/7ca9fa48-b732-4f4e-911f-b719e83305be}}

\subsection{Quick Demo}\label{quick-demo}

\begin{longtable}[]{@{}llll@{}}
\toprule\noalign{}
Code & Output & Code & Output \\
\midrule\noalign{}
\endhead
\bottomrule\noalign{}
\endlastfoot
\texttt{\ num("1.2e4")\ } & 1.2Ã---10â?´ & \texttt{\ num{[}1.2e4{]}\ } &
1.2Ã---10â?´ \\
\texttt{\ num("-5e-4")\ } & âˆ'5Ã---10â?»â?´ &
\texttt{\ num(fixed:\ -2){[}0.02{]}\ } & 2Ã---10â?»Â² \\
\texttt{\ num("9.81+-.01")\ } & 9.81±0.01 &
\texttt{\ num("9.81+0.02-.01")\ } & 9.81�²₋� \\
\texttt{\ num("9.81+-.01e2")\ } & (9.81±0.01)Ã---10² &
\texttt{\ num(base:\ 2){[}3e4{]}\ } & 3Ã---2â?´ \\
\end{longtable}

\subsection{Documentation}\label{documentation}

\begin{itemize}
\tightlist
\item
  \href{https://github.com/typst/packages/raw/main/packages/preview/zero/0.3.0/\#num}{Function
  \texttt{\ num\ }}
\item
  \href{https://github.com/typst/packages/raw/main/packages/preview/zero/0.3.0/\#grouping}{Grouping}
\item
  \href{https://github.com/typst/packages/raw/main/packages/preview/zero/0.3.0/\#rounding}{Rounding}
\item
  \href{https://github.com/typst/packages/raw/main/packages/preview/zero/0.3.0/\#specifying-uncertainties}{Uncertainties}
\item
  \href{https://github.com/typst/packages/raw/main/packages/preview/zero/0.3.0/\#table-alignment}{Table
  alignment}
\end{itemize}

\subsubsection{\texorpdfstring{\texttt{\ num\ }}{ num }}\label{num}

The function \texttt{\ num()\ } is the heart of \emph{Zero} . It
provides a wide range of number formatting utilities and its default
values are configurable via \texttt{\ set-num()\ } which takes the same
named arguments as \texttt{\ num()\ } .

\begin{Shaded}
\begin{Highlighting}[]
\NormalTok{\#let num(}
\NormalTok{  number:                 str | content | int | float | dictionary | array,}
\NormalTok{  digits:                 auto | int = auto,}
\NormalTok{  fixed:                  none | int = none,}

\NormalTok{  decimal{-}separator:      str = ".",}
\NormalTok{  product:                content = sym.times,}
\NormalTok{  tight:                  boolean = false,}
\NormalTok{  math:                   boolean = true,}
\NormalTok{  omit{-}unity{-}mantissa:    boolean = true,}
\NormalTok{  positive{-}sign:          boolean = false,}
\NormalTok{  positive{-}sign{-}exponent: boolean = false,}
\NormalTok{  base:                   int | content = 10,}
\NormalTok{  uncertainty{-}mode:       str = "separate",}
\NormalTok{  round:                  dictionary,}
\NormalTok{  group:                  dictionary,}
\NormalTok{)}
\end{Highlighting}
\end{Shaded}

\begin{itemize}
\tightlist
\item
  \texttt{\ number:\ str\ \textbar{} content\ \textbar{}\ int\ \textbar{}\ float\ \textbar{}\ array\ }
  : Number input; \texttt{\ str\ } is preferred. If the input is
  \texttt{\ content\ } , it may only contain text nodes. Numeric types
  \texttt{\ int\ } and \texttt{\ float\ } are supported but not
  encouraged because of information loss (e.g., the number of trailing
  “0� digits or the exponent). The remaining types
  \texttt{\ dictionary\ } and \texttt{\ array\ } are intended for
  advanced use, see
  \href{https://github.com/typst/packages/raw/main/packages/preview/zero/0.3.0/\#zero-for-third-party-packages}{below}
  .
\item
  \texttt{\ digits:\ auto\ \textbar{} int\ =\ auto\ } : Truncates the
  number at a given (positive) number of decimal places or pads the
  number with zeros if necessary. This is independent of
  \href{https://github.com/typst/packages/raw/main/packages/preview/zero/0.3.0/\#rounding}{rounding}
  .
\item
  \texttt{\ fixed:\ none\ \textbar{}\ int\ =\ none\ } : If not
  \texttt{\ none\ } , forces a fixed exponent. Additional exponents
  given in the number input are taken into account.
\item
  \texttt{\ decimal-separator:\ str\ =\ "."\ } : Specifies the marker
  that is used for separating integer and decimal part.
\item
  \texttt{\ product:\ content\ =\ sym.times\ } : Specifies the
  multiplication symbol used for scientific notation.
\item
  \texttt{\ tight:\ boolean\ =\ false\ } : If true, tight spacing is
  applied between operands (applies to Ã--- and ±).
\item
  \texttt{\ math:\ boolean\ =\ true\ } : If set to \texttt{\ false\ } ,
  the parts of the number won’t be wrapped in a
  \texttt{\ math.equation\ } wherever feasible. This makes it possible
  to use \texttt{\ num()\ } with non-math fonts to some extent. Powers
  are always rendered in math mode.
\item
  \texttt{\ omit-unity-mantissa:\ boolean\ =\ false\ } : Determines
  whether a mantissa of 1 is omitted in scientific notation, e.g., 10â?´
  instead of 1·10�.
\item
  \texttt{\ positive-sign:\ boolean\ =\ false\ } : If set to
  \texttt{\ true\ } , positive coefficients are shown with a + sign.
\item
  \texttt{\ positive-sign-exponent:\ boolean\ =\ false\ } : If set to
  \texttt{\ true\ } , positive exponents are shown with a + sign.
\item
  \texttt{\ base:\ int\ \textbar{}\ content\ =\ 10\ } : The base used
  for scientific power notation.
\item
  \texttt{\ uncertainty-mode:\ str\ =\ "separate"\ } : Selects one of
  the modes \texttt{\ "separate"\ } , \texttt{\ "compact"\ } , or
  \texttt{\ "compact-separator"\ } for displaying uncertainties. The
  different behaviors are shown below:
\end{itemize}

\begin{longtable}[]{@{}lll@{}}
\toprule\noalign{}
\texttt{\ "separate"\ } & \texttt{\ "compact"\ } &
\texttt{\ "compact-separator"\ } \\
\midrule\noalign{}
\endhead
\bottomrule\noalign{}
\endlastfoot
1.7±0.2 & 1.7(2) & 1.7(2) \\
6.2±2.1 & 6.2(21) & 6.2(2.1) \\
1.7��˙̇²₋₀.₠& 1.7�²₋₠& 1.7�²₋₠\\
1.7â?ºÂ²Ë™Ì‡â?°â‚‹â‚\ldots.â‚€ & 1.7â?ºÂ²â?°â‚‹â‚\ldots â‚€ &
1.7â?ºÂ²Ë™Ì‡â?°â‚‹â‚\ldots.â‚€ \\
\end{longtable}

\begin{itemize}
\tightlist
\item
  \texttt{\ round:\ dictionary\ } : You can provide one or more rounding
  options in a dictionary. Also see
  \href{https://github.com/typst/packages/raw/main/packages/preview/zero/0.3.0/\#rounding}{rounding}
  .
\item
  \texttt{\ group:\ dictionary\ } : You can provide one or more grouping
  options in a dictionary. Also see
  \href{https://github.com/typst/packages/raw/main/packages/preview/zero/0.3.0/\#grouping}{grouping}
  .
\end{itemize}

Configuration example:

\begin{Shaded}
\begin{Highlighting}[]
\NormalTok{\#set{-}num(product: math.dot, tight: true)}
\end{Highlighting}
\end{Shaded}

\subsubsection{Grouping}\label{grouping}

Digit grouping is important for keeping large figures readable. It is
customary to separate thousands with a thin space, a period, comma, or
an apostrophe (however, we discourage using a period or a comma to avoid
confusion since both are used for decimal separators in various
countries).

\pandocbounded{\includegraphics[keepaspectratio]{https://github.com/user-attachments/assets/1f53ae33-3e99-483d-ac6a-6e3cbed5484b}}

Digit grouping can be configured with the \texttt{\ set-group()\ }
function.

\begin{Shaded}
\begin{Highlighting}[]
\NormalTok{\#let set{-}group(}
\NormalTok{  size:       int = 3, }
\NormalTok{  separator:  content = sym.space.thin,}
\NormalTok{  threshold:  int = 5}
\NormalTok{)}
\end{Highlighting}
\end{Shaded}

\begin{itemize}
\tightlist
\item
  \texttt{\ size:\ int\ =\ 3\ } : Determines the size of the groups.
\item
  \texttt{\ separator:\ content\ =\ sym.space.thin\ } : Separator
  between groups.
\item
  \texttt{\ threshold:\ int\ =\ 5\ } : Necessary number of digits needed
  for digit grouping to kick in. Four-digit numbers for example are
  usually not grouped at all since they can still be read easily.
\end{itemize}

Configuration example:

\begin{Shaded}
\begin{Highlighting}[]
\NormalTok{\#set{-}group(separator: "\textquotesingle{}", threshold: 4)}
\end{Highlighting}
\end{Shaded}

Grouping can be turned off altogether by setting the
\texttt{\ threshold\ } to \texttt{\ calc.inf\ } .

\subsubsection{Rounding}\label{rounding}

Rounding can be configured with the \texttt{\ set-round()\ } function.

\begin{Shaded}
\begin{Highlighting}[]
\NormalTok{\#let set{-}round(}
\NormalTok{  mode:       none | str = none,}
\NormalTok{  precision:  int = 2,}
\NormalTok{  pad:        boolean = true,}
\NormalTok{  direction:  str = "nearest",}
\NormalTok{)}
\end{Highlighting}
\end{Shaded}

\begin{itemize}
\tightlist
\item
  \texttt{\ mode:\ none\ \textbar{} str\ =\ none\ } : Sets the
  rounding mode. The possible options are

  \begin{itemize}
  \tightlist
  \item
    \texttt{\ none\ } : Rounding is turned off.
  \item
    \texttt{\ "places"\ } : The number is rounded to the number of
    decimal places given by the \texttt{\ precision\ } parameter.
  \item
    \texttt{\ "figures"\ } : The number is rounded to a number of
    significant figures given by the \texttt{\ precision\ } parameter.
  \item
    \texttt{\ "uncertainty"\ } : Requires giving an uncertainty value.
    The uncertainty is rounded to significant figures according to the
    \texttt{\ precision\ } argument and then the number is rounded to
    the same number of decimal places as the uncertainty.
  \end{itemize}
\item
  \texttt{\ precision:\ int\ =\ 2\ } : The precision to round to. Also
  see parameter \texttt{\ mode\ } .
\item
  \texttt{\ pad:\ boolean\ =\ true\ } : Whether to pad the number with
  zeros if the number has fewer digits than the rounding precision.
\item
  \texttt{\ direction:\ str\ =\ "nearest"\ } : Sets the rounding
  direction.

  \begin{itemize}
  \tightlist
  \item
    \texttt{\ "nearest"\ } : Rounding takes place in the usual fashion,
    rounding to the nearer number, e.g., 2.34 â†' 2.3 and 2.36 â†' 2.4.
  \item
    \texttt{\ "down"\ } : Always rounds down, e.g., 2.38 â†' 2.3 and
    2.30 â†' 2.3.
  \item
    \texttt{\ "up"\ } : Always rounds up, e.g., 2.32 â†' 2.4 and 2.30
    â†' 2.3.
  \end{itemize}
\end{itemize}

\subsubsection{Specifying uncertainties}\label{specifying-uncertainties}

There are two ways of specifying uncertainties:

\begin{itemize}
\tightlist
\item
  Applying an uncertainty to the least significant digits using
  parentheses, e.g., \texttt{\ 2.3(4)\ } ,
\item
  Denoting an absolute uncertainty, e.g., \texttt{\ 2.3+-0.4\ } becomes
  2.3±0.4.
\end{itemize}

Zero supports both and can convert between these two, so that you can
pick the displayed style (configured via \texttt{\ uncertainty-mode\ } ,
see above) independently of the input style.

How do uncertainties interplay with exponents? The uncertainty needs to
come first, and the exponent applies to both the mantissa and the
uncertainty, e.g., \texttt{\ num("1.23+-.04e2")\ } becomes

(1.23 ± 0.03)Ã---10²

Note that the mantissa is now put in parentheses to disambiguate the
application of the power.

In some cases, the uncertainty is asymmetric which can be expressed via
\texttt{\ num("1.23+0.02-0.01")\ }

1.23��˙̇�²₋₀.₀�

\subsubsection{Table alignment}\label{table-alignment}

In scientific publication, presenting many numbers in a readable fashion
can be a difficult discipline. A good starting point is to align numbers
in a table at the decimal separator. With \emph{Zero} , this can be
accomplished by using \texttt{\ ztable\ } , a wrapper for the built-in
\texttt{\ table\ } function. It features an additional parameter
\texttt{\ format\ } which takes an array of \texttt{\ none\ } ,
\texttt{\ auto\ } , or \texttt{\ dictionary\ } values to turn on number
alignment for specific columns.

\begin{Shaded}
\begin{Highlighting}[]
\NormalTok{\#ztable(}
\NormalTok{  columns: 3,}
\NormalTok{  align: center,}
\NormalTok{  format: (none, auto, auto),}
\NormalTok{  $n$, $α$, $β$,}
\NormalTok{  [1], [3.45], [{-}11.1],}
\NormalTok{  ..}
\NormalTok{)}
\end{Highlighting}
\end{Shaded}

Non-number entries (e.g., in the header) are automatically recognized in
some cases and will not be aligned. In ambiguous cases, adding a leading
or trailing space tells \emph{Zero} not to apply alignment to this cell,
e.g., \texttt{\ {[}Angle\ {]}\ } instead of \texttt{\ {[}Angle{]}\ } .

\pandocbounded{\includegraphics[keepaspectratio]{https://github.com/user-attachments/assets/2effb7f0-0d9b-401a-92e1-20461d0c1fcb}}

In addition, you can prefix or suffix a numeral with content wrapped by
the function \texttt{\ nonum{[}{]}\ } to mark it as \emph{not belonging
to the number} . The remaining content may still be recognized as a
number and formatted/aligned accordingly.

\begin{Shaded}
\begin{Highlighting}[]
\NormalTok{\#ztable(}
\NormalTok{  format: (auto,),}
\NormalTok{  [\#nonum[€]123.0\#nonum(footnote[A special number])],}
\NormalTok{  [12.111],}
\NormalTok{)}
\end{Highlighting}
\end{Shaded}

\pandocbounded{\includegraphics[keepaspectratio]{https://github.com/user-attachments/assets/270ae789-2a8c-44a3-b3a9-0ca588bfbad3}}

Zero not only aligns numbers at the decimal point but also at the
uncertainty and exponent part. Moreover, by passing a
\texttt{\ dictionary\ } instead of \texttt{\ auto\ } , a set of
\texttt{\ num()\ } arguments to apply to all numbers in a column can be
specified.

\begin{Shaded}
\begin{Highlighting}[]
\NormalTok{\#ztable(}
\NormalTok{  columns: 4,}
\NormalTok{  align: center,}
\NormalTok{  format: (none, auto, auto, (digits: 1)),}
\NormalTok{  $n$, $α$, $β$, $γ$,}
\NormalTok{  [1], [3.45e2], [{-}11.1+{-}3], [0],}
\NormalTok{  ..}
\NormalTok{)}
\end{Highlighting}
\end{Shaded}

\pandocbounded{\includegraphics[keepaspectratio]{https://github.com/user-attachments/assets/c96941bc-f002-4b93-b2cd-705c8104682f}}

\subsection{Zero for third-party
packages}\label{zero-for-third-party-packages}

This package provides some useful extras for third-party packages that
generate formatted numbers (for example graphics libraries).

Instead of passing a \texttt{\ str\ } to \texttt{\ num()\ } , it is also
possible to pass a dictionary of the form

\begin{Shaded}
\begin{Highlighting}[]
\NormalTok{(}
\NormalTok{  mantissa:  str | int | float,}
\NormalTok{  e:         none | str,}
\NormalTok{  pm:        none | array}
\NormalTok{)}
\end{Highlighting}
\end{Shaded}

This way, parsing the number can be avoided which makes especially sense
for packages that generate numbers (e.g., tick labels for a diagram
axis) with independent mantissa and exponent.

Furthermore, \texttt{\ num()\ } also allows \texttt{\ array\ } arguments
for \texttt{\ number\ } which allows for more efficient batch-processing
of numbers with the same setup. In this case, the caller of the function
needs to provide \texttt{\ context\ } .

\subsection{Changelog}\label{changelog}

\subsubsection{Version 0.3.0}\label{version-0.3.0}

\begin{itemize}
\tightlist
\item
  Adds \texttt{\ nonum{[}{]}\ } function that can be used to mark
  content in cells as \emph{not belonging to the number} . The remaining
  content may still be recognized as a number and formatted/aligned
  accordingly. The content wrapped by \texttt{\ nonum{[}{]}\ } is
  preserved.
\item
  Fixes number alignment tables with new version Typst 0.12.
\end{itemize}

\subsubsection{Version 0.2.0}\label{version-0.2.0}

\begin{itemize}
\tightlist
\item
  Adds support for using non-math fonts for \texttt{\ num\ } via the
  option \texttt{\ math\ } . This can be activated by calling
  \texttt{\ \#set-num(math:\ false)\ } .
\item
  Performance improvements for both \texttt{\ num()\ } and
  \texttt{\ ztable(9)\ }
\end{itemize}

\subsubsection{Version 0.1.0}\label{version-0.1.0}

\subsubsection{How to add}\label{how-to-add}

Copy this into your project and use the import as \texttt{\ zero\ }

\begin{verbatim}
#import "@preview/zero:0.3.0"
\end{verbatim}

\includesvg[width=0.16667in,height=0.16667in]{/assets/icons/16-copy.svg}

Check the docs for
\href{https://typst.app/docs/reference/scripting/\#packages}{more
information on how to import packages} .

\subsubsection{About}\label{about}

\begin{description}
\tightlist
\item[Author :]
\href{https://github.com/Mc-Zen}{Mc-Zen}
\item[License:]
MIT
\item[Current version:]
0.3.0
\item[Last updated:]
October 28, 2024
\item[First released:]
September 16, 2024
\item[Minimum Typst version:]
0.11.0
\item[Archive size:]
15.7 kB
\href{https://packages.typst.org/preview/zero-0.3.0.tar.gz}{\pandocbounded{\includesvg[keepaspectratio]{/assets/icons/16-download.svg}}}
\item[Repository:]
\href{https://github.com/Mc-Zen/zero}{GitHub}
\item[Categor ies :]
\begin{itemize}
\tightlist
\item[]
\item
  \pandocbounded{\includesvg[keepaspectratio]{/assets/icons/16-chart.svg}}
  \href{https://typst.app/universe/search/?category=visualization}{Visualization}
\item
  \pandocbounded{\includesvg[keepaspectratio]{/assets/icons/16-layout.svg}}
  \href{https://typst.app/universe/search/?category=layout}{Layout}
\end{itemize}
\end{description}

\subsubsection{Where to report issues?}\label{where-to-report-issues}

This package is a project of Mc-Zen . Report issues on
\href{https://github.com/Mc-Zen/zero}{their repository} . You can also
try to ask for help with this package on the
\href{https://forum.typst.app}{Forum} .

Please report this package to the Typst team using the
\href{https://typst.app/contact}{contact form} if you believe it is a
safety hazard or infringes upon your rights.

\phantomsection\label{versions}
\subsubsection{Version history}\label{version-history}

\begin{longtable}[]{@{}ll@{}}
\toprule\noalign{}
Version & Release Date \\
\midrule\noalign{}
\endhead
\bottomrule\noalign{}
\endlastfoot
0.3.0 & October 28, 2024 \\
\href{https://typst.app/universe/package/zero/0.2.0/}{0.2.0} & October
4, 2024 \\
\href{https://typst.app/universe/package/zero/0.1.0/}{0.1.0} & September
16, 2024 \\
\end{longtable}

Typst GmbH did not create this package and cannot guarantee correct
functionality of this package or compatibility with any version of the
Typst compiler or app.


