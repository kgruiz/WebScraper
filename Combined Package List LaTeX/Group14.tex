\title{typst.app/universe/package/spreet}

\phantomsection\label{banner}
\section{spreet}\label{spreet}

{ 0.1.0 }

Parse a spreadsheet.

\phantomsection\label{readme}
Spreet is a spreadsheet decoder for typst (excel/opendocument
spreadsheets). The spreadsheet will be read and parsed into a dictonary
of 2-dimensional array of strings: Each workbook in the spreadsheet is
mapped as an entry in the dictonary. Each row of the workbook is
represented as an array of strings, and all rows are summarised in a
single array.

\subsection{Example}\label{example}

\begin{Shaded}
\begin{Highlighting}[]
\NormalTok{\#import "@preview/spreet:0.1.0"}

\NormalTok{\#let excel\_data = spreet.file{-}decode("excel.xlsx")}
\NormalTok{\#let opendocument\_data = spreet.file{-}decode("opendocument.ods")}

\NormalTok{\#let excel\_data\_from\_bytes = spreet.decode(read("excel.xlsx", encoding: none))}
\NormalTok{\#let opendocument\_data\_from\_bytes = spreet.decode(read("opendocument.ods", encoding: none))}

\NormalTok{/**}
\NormalTok{excel\_data or opendocument\_data contains a dict of all worksheets}
\NormalTok{(}
\NormalTok{  Worksheet1: (}
\NormalTok{    (Row1\_Column1, Row1\_Column2),}
\NormalTok{    (Row2\_Column1, Row2\_Column2),}
\NormalTok{  ),}
\NormalTok{  Worksheet2: (}
\NormalTok{    (Row1\_Column1, Row1\_Column2),}
\NormalTok{    (Row2\_Column1, Row2\_Column2),}
\NormalTok{  )}
\NormalTok{)}
\NormalTok{**/}
\end{Highlighting}
\end{Shaded}

\subsubsection{How to add}\label{how-to-add}

Copy this into your project and use the import as \texttt{\ spreet\ }

\begin{verbatim}
#import "@preview/spreet:0.1.0"
\end{verbatim}

\includesvg[width=0.16667in,height=0.16667in]{/assets/icons/16-copy.svg}

Check the docs for
\href{https://typst.app/docs/reference/scripting/\#packages}{more
information on how to import packages} .

\subsubsection{About}\label{about}

\begin{description}
\tightlist
\item[Author :]
lublak
\item[License:]
MIT
\item[Current version:]
0.1.0
\item[Last updated:]
September 15, 2024
\item[First released:]
September 15, 2024
\item[Archive size:]
335 kB
\href{https://packages.typst.org/preview/spreet-0.1.0.tar.gz}{\pandocbounded{\includesvg[keepaspectratio]{/assets/icons/16-download.svg}}}
\item[Repository:]
\href{https://github.com/lublak/typst-spreet-package}{GitHub}
\end{description}

\subsubsection{Where to report issues?}\label{where-to-report-issues}

This package is a project of lublak . Report issues on
\href{https://github.com/lublak/typst-spreet-package}{their repository}
. You can also try to ask for help with this package on the
\href{https://forum.typst.app}{Forum} .

Please report this package to the Typst team using the
\href{https://typst.app/contact}{contact form} if you believe it is a
safety hazard or infringes upon your rights.

\phantomsection\label{versions}
\subsubsection{Version history}\label{version-history}

\begin{longtable}[]{@{}ll@{}}
\toprule\noalign{}
Version & Release Date \\
\midrule\noalign{}
\endhead
\bottomrule\noalign{}
\endlastfoot
0.1.0 & September 15, 2024 \\
\end{longtable}

Typst GmbH did not create this package and cannot guarantee correct
functionality of this package or compatibility with any version of the
Typst compiler or app.


\title{typst.app/universe/package/codelst}

\phantomsection\label{banner}
\section{codelst}\label{codelst}

{ 2.0.2 }

A typst package to render sourcecode.

\phantomsection\label{readme}
\textbf{codelst} is a \href{https://github.com/typst/typst}{Typst}
package for rendering sourcecode with line numbers and some other
additions.

\subsection{Usage}\label{usage}

Import the package from the typst preview repository:

\begin{Shaded}
\begin{Highlighting}[]
\NormalTok{\#import }\StringTok{"@preview/codelst:2.0.2"}\OperatorTok{:}\NormalTok{ sourcecode}
\end{Highlighting}
\end{Shaded}

After importing the package, simply wrap any fenced code block in a call
to \texttt{\ \#sourcecode()\ } :

\begin{Shaded}
\begin{Highlighting}[]
\NormalTok{\#import }\StringTok{"@preview/codelst:2.0.2"}\OperatorTok{:}\NormalTok{ sourcecode}

\NormalTok{\#sourcecode[}\VerbatimStringTok{\textasciigrave{}\textasciigrave{}\textasciigrave{}typ}
\VerbatimStringTok{\#show "ArtosFlow": name =\textgreater{} box[}
\VerbatimStringTok{  \#box(image(}
\VerbatimStringTok{    "logo.svg",}
\VerbatimStringTok{    height: 0.7em,}
\VerbatimStringTok{  ))}
\VerbatimStringTok{  \#name}
\VerbatimStringTok{]}

\VerbatimStringTok{This report is embedded in the}
\VerbatimStringTok{ArtosFlow project. ArtosFlow is a}
\VerbatimStringTok{project of the Artos Institute.}
\VerbatimStringTok{\textasciigrave{}\textasciigrave{}\textasciigrave{}}\NormalTok{]}
\end{Highlighting}
\end{Shaded}

\subsection{Further documentation}\label{further-documentation}

See \texttt{\ manual.pdf\ } for a comprehensive manual of the package.

See \texttt{\ example.typ\ } for some quick usage examples.

\subsection{Development}\label{development}

The documentation is created using
\href{https://github.com/jneug/typst-mantys}{Mantys} , a Typst template
for creating package documentation.

To compile the manual, Mantys needs to be available as a local package.
Refer to Mantys’ manual for instructions on how to do so.

\subsection{Changelog}\label{changelog}

\subsubsection{v2.0.1}\label{v2.0.1}

This version makes \texttt{\ codelst\ } compatible to Typst 0.11.0.
Version 2.0.1 now requires Typst 0.11.0, since there are some breaking
changes to the way counters work.

Thanks to @kilpkonn for theses changes.

\subsubsection{v2.0.0}\label{v2.0.0}

Version 2 requires Typst 0.9.0 or newer. Rendering is now done using the
new \texttt{\ raw.line\ } elements get consistent line numbers and
syntax highlighting (even if \texttt{\ showrange\ } is used). Rendering
is now done in a \texttt{\ \#table\ } .

\begin{itemize}
\tightlist
\item
  Added \texttt{\ theme\ } and \texttt{\ syntaxes\ } options to
  overwrite passed in \texttt{\ \#raw\ } values.
\item
  Breaking: Renamed \texttt{\ tab-indend\ } to \texttt{\ tab-size\ } ,
  to conform with the Typst option.
\item
  Breaking: Removed \texttt{\ continue-numbering\ } option for now. (The
  feature failed in combination with label parsing and line highlights.)
\item
  Breaking: Removed styling of line numbers via a \texttt{\ show\ }
  -rule.
\end{itemize}

\subsubsection{v1.0.0}\label{v1.0.0}

\begin{itemize}
\tightlist
\item
  Complete rewrite of code rendering.
\item
  New options for \texttt{\ \#sourcecode()\ } :

  \begin{itemize}
  \tightlist
  \item
    \texttt{\ lang\ } : Overwrite code language setting.
  \item
    \texttt{\ numbers-first\ } : First line number to show.
  \item
    \texttt{\ numbers-step\ } : Only show every n-th number.
  \item
    \texttt{\ frame\ } : Set a frame (replaces
    \texttt{\ \textless{}codelst\textgreater{}\ } label.)
  \item
    Merged \texttt{\ line-numbers\ } and \texttt{\ numbering\ } options.
  \end{itemize}
\item
  Removed \texttt{\ \#numbers-style()\ } function.

  \begin{itemize}
  \tightlist
  \item
    \texttt{\ numbers-style\ } option now gets passed
    \texttt{\ counter.display()\ } .
  \end{itemize}
\item
  Removed \texttt{\ \textless{}codelst\textgreater{}\ } label.
\item
  \texttt{\ codelst-style\ } only sets \texttt{\ breakable\ } for
  figures.
\item
  New \texttt{\ codelst\ } function to setup a catchall show rules for
  \texttt{\ raw\ } text.
\item
  \texttt{\ label-regex:\ none\ } disables labels parsing.
\item
  Code improvements and refactorings.
\end{itemize}

\subsubsection{v0.0.5}\label{v0.0.5}

\begin{itemize}
\tightlist
\item
  Fixed insets for line highlights.
\item
  Added \texttt{\ numbers-width\ } option to manually set width of line
  numbers column.

  \begin{itemize}
  \tightlist
  \item
    This allows line numbers on margins by setting
    \texttt{\ numbers-width\ } to \texttt{\ 0pt\ } or a negative number
    like \texttt{\ -1em\ } .
  \end{itemize}
\end{itemize}

\subsubsection{v0.0.4}\label{v0.0.4}

\begin{itemize}
\tightlist
\item
  Fixed issue with context unaware syntax highlighting.
\end{itemize}

\subsubsection{v0.0.3}\label{v0.0.3}

\begin{itemize}
\tightlist
\item
  Removed call to \texttt{\ \#read()\ } from \texttt{\ \#sourcefile()\ }
  .
\item
  Added \texttt{\ continue-numbering\ } argument to
  \texttt{\ \#sourcecode()\ } .
\item
  Fixed problem with \texttt{\ showrange\ } having out of range line
  numbers.
\end{itemize}

\subsubsection{v0.0.2}\label{v0.0.2}

\begin{itemize}
\tightlist
\item
  Added a comprehensive manual.
\item
  Fixed crash for missing \texttt{\ lang\ } attribute in
  \texttt{\ raw\ } element.
\end{itemize}

\subsubsection{v0.0.1}\label{v0.0.1}

\begin{itemize}
\tightlist
\item
  Initial version submitted to typst/packages.
\end{itemize}

\subsubsection{How to add}\label{how-to-add}

Copy this into your project and use the import as \texttt{\ codelst\ }

\begin{verbatim}
#import "@preview/codelst:2.0.2"
\end{verbatim}

\includesvg[width=0.16667in,height=0.16667in]{/assets/icons/16-copy.svg}

Check the docs for
\href{https://typst.app/docs/reference/scripting/\#packages}{more
information on how to import packages} .

\subsubsection{About}\label{about}

\begin{description}
\tightlist
\item[Author :]
Jonas Neugebauer
\item[License:]
MIT
\item[Current version:]
2.0.2
\item[Last updated:]
October 23, 2024
\item[First released:]
July 24, 2023
\item[Minimum Typst version:]
0.12.0
\item[Archive size:]
5.41 kB
\href{https://packages.typst.org/preview/codelst-2.0.2.tar.gz}{\pandocbounded{\includesvg[keepaspectratio]{/assets/icons/16-download.svg}}}
\item[Repository:]
\href{https://github.com/jneug/typst-codelst}{GitHub}
\item[Discipline s :]
\begin{itemize}
\tightlist
\item[]
\item
  \href{https://typst.app/universe/search/?discipline=computer-science}{Computer
  Science}
\item
  \href{https://typst.app/universe/search/?discipline=mathematics}{Mathematics}
\item
  \href{https://typst.app/universe/search/?discipline=education}{Education}
\item
  \href{https://typst.app/universe/search/?discipline=linguistics}{Linguistics}
\end{itemize}
\item[Categor ies :]
\begin{itemize}
\tightlist
\item[]
\item
  \pandocbounded{\includesvg[keepaspectratio]{/assets/icons/16-package.svg}}
  \href{https://typst.app/universe/search/?category=components}{Components}
\item
  \pandocbounded{\includesvg[keepaspectratio]{/assets/icons/16-layout.svg}}
  \href{https://typst.app/universe/search/?category=layout}{Layout}
\end{itemize}
\end{description}

\subsubsection{Where to report issues?}\label{where-to-report-issues}

This package is a project of Jonas Neugebauer . Report issues on
\href{https://github.com/jneug/typst-codelst}{their repository} . You
can also try to ask for help with this package on the
\href{https://forum.typst.app}{Forum} .

Please report this package to the Typst team using the
\href{https://typst.app/contact}{contact form} if you believe it is a
safety hazard or infringes upon your rights.

\phantomsection\label{versions}
\subsubsection{Version history}\label{version-history}

\begin{longtable}[]{@{}ll@{}}
\toprule\noalign{}
Version & Release Date \\
\midrule\noalign{}
\endhead
\bottomrule\noalign{}
\endlastfoot
2.0.2 & October 23, 2024 \\
\href{https://typst.app/universe/package/codelst/2.0.1/}{2.0.1} & March
19, 2024 \\
\href{https://typst.app/universe/package/codelst/2.0.0/}{2.0.0} &
November 16, 2023 \\
\href{https://typst.app/universe/package/codelst/1.0.0/}{1.0.0} & July
29, 2023 \\
\href{https://typst.app/universe/package/codelst/0.0.3/}{0.0.3} & July
24, 2023 \\
\end{longtable}

Typst GmbH did not create this package and cannot guarantee correct
functionality of this package or compatibility with any version of the
Typst compiler or app.


\title{typst.app/universe/package/haw-hamburg-master-thesis}

\phantomsection\label{banner}
\phantomsection\label{template-thumbnail}
\pandocbounded{\includegraphics[keepaspectratio]{https://packages.typst.org/preview/thumbnails/haw-hamburg-master-thesis-0.3.1-small.webp}}

\section{haw-hamburg-master-thesis}\label{haw-hamburg-master-thesis}

{ 0.3.1 }

Unofficial template for writing a master-thesis in the HAW Hamburg
department of Computer Science design.

\href{/app?template=haw-hamburg-master-thesis&version=0.3.1}{Create
project in app}

\phantomsection\label{readme}
This is an \textbf{\texttt{\ unofficial\ }} template for writing a
master thesis in the \texttt{\ HAW\ Hamburg\ } department of
\texttt{\ Computer\ Science\ } design using
\href{https://github.com/typst/typst}{Typst} .

\subsection{Required Fonts}\label{required-fonts}

To correctly render this template please make sure that the
\texttt{\ New\ Computer\ Modern\ } font is installed on your system.

\subsection{Usage}\label{usage}

To use this package just add the following code to your
\href{https://github.com/typst/typst}{Typst} document:

\begin{Shaded}
\begin{Highlighting}[]
\NormalTok{\#import "@preview/haw{-}hamburg:0.3.1": master{-}thesis}

\NormalTok{\#show: master{-}thesis.with(}
\NormalTok{  language: "en",}

\NormalTok{  title{-}de: "Beispiel Titel",}
\NormalTok{  keywords{-}de: ("Stichwort", "Wichtig", "Super"),}
\NormalTok{  abstract{-}de: "Beispiel Zusammenfassung",}

\NormalTok{  title{-}en: "Example title",}
\NormalTok{  keywords{-}en:  ("Keyword", "Important", "Super"),}
\NormalTok{  abstract{-}en: "Example abstract",}

\NormalTok{  author: "The Computer",}
\NormalTok{  faculty: "Engineering and Computer Science",}
\NormalTok{  department: "Computer Science",}
\NormalTok{  study{-}course: "Master of Science Computer Science",}
\NormalTok{  supervisors: ("Prof. Dr. Example", "Prof. Dr. Example"),}
\NormalTok{  submission{-}date: datetime(year: 1948, month: 12, day: 10),}
\NormalTok{  include{-}declaration{-}of{-}independent{-}processing: true,}
\NormalTok{)}
\end{Highlighting}
\end{Shaded}

\subsection{How to Compile}\label{how-to-compile}

This project contains an example setup that splits individual chapters
into different files.\\
This can cause problems when using references etc.\\
These problems can be avoided by following these steps:

\begin{itemize}
\tightlist
\item
  Make sure to always compile your \texttt{\ main.typ\ } file which
  includes all of your chapters for references to work correctly.
\item
  VSCode:

  \begin{itemize}
  \tightlist
  \item
    Install the
    \href{https://marketplace.visualstudio.com/items?itemName=myriad-dreamin.tinymist}{Tinymist
    Typst} extension.
  \item
    Make sure to start the \texttt{\ PDF\ } or
    \texttt{\ Live\ Preview\ } only from within your
    \texttt{\ main.typ\ } file.
  \item
    If problems occur it usually helps to close the preview and restart
    it from your \texttt{\ main.typ\ } file.
  \end{itemize}
\end{itemize}

\href{/app?template=haw-hamburg-master-thesis&version=0.3.1}{Create
project in app}

\subsubsection{How to use}\label{how-to-use}

Click the button above to create a new project using this template in
the Typst app.

You can also use the Typst CLI to start a new project on your computer
using this command:

\begin{verbatim}
typst init @preview/haw-hamburg-master-thesis:0.3.1
\end{verbatim}

\includesvg[width=0.16667in,height=0.16667in]{/assets/icons/16-copy.svg}

\subsubsection{About}\label{about}

\begin{description}
\tightlist
\item[Author :]
Lasse Rosenow
\item[License:]
MIT
\item[Current version:]
0.3.1
\item[Last updated:]
November 13, 2024
\item[First released:]
October 14, 2024
\item[Archive size:]
6.54 kB
\href{https://packages.typst.org/preview/haw-hamburg-master-thesis-0.3.1.tar.gz}{\pandocbounded{\includesvg[keepaspectratio]{/assets/icons/16-download.svg}}}
\item[Repository:]
\href{https://github.com/LasseRosenow/HAW-Hamburg-Typst-Template}{GitHub}
\item[Categor y :]
\begin{itemize}
\tightlist
\item[]
\item
  \pandocbounded{\includesvg[keepaspectratio]{/assets/icons/16-mortarboard.svg}}
  \href{https://typst.app/universe/search/?category=thesis}{Thesis}
\end{itemize}
\end{description}

\subsubsection{Where to report issues?}\label{where-to-report-issues}

This template is a project of Lasse Rosenow . Report issues on
\href{https://github.com/LasseRosenow/HAW-Hamburg-Typst-Template}{their
repository} . You can also try to ask for help with this template on the
\href{https://forum.typst.app}{Forum} .

Please report this template to the Typst team using the
\href{https://typst.app/contact}{contact form} if you believe it is a
safety hazard or infringes upon your rights.

\phantomsection\label{versions}
\subsubsection{Version history}\label{version-history}

\begin{longtable}[]{@{}ll@{}}
\toprule\noalign{}
Version & Release Date \\
\midrule\noalign{}
\endhead
\bottomrule\noalign{}
\endlastfoot
0.3.1 & November 13, 2024 \\
\href{https://typst.app/universe/package/haw-hamburg-master-thesis/0.3.0/}{0.3.0}
& October 14, 2024 \\
\end{longtable}

Typst GmbH did not create this template and cannot guarantee correct
functionality of this template or compatibility with any version of the
Typst compiler or app.


\title{typst.app/universe/package/babel}

\phantomsection\label{banner}
\section{babel}\label{babel}

{ 0.1.1 }

Redact text by replacing it with random characters

\phantomsection\label{readme}
\href{https://typst.app/universe/package/babel}{\pandocbounded{\includegraphics[keepaspectratio]{https://img.shields.io/badge/Typst_Universe-fdfdfd?logo=typst}}}
\href{https://codeberg.org/afiaith/babel}{\pandocbounded{\includegraphics[keepaspectratio]{https://img.shields.io/badge/Git_repo-fdfdfd?logo=codeberg}}}
\href{https://github.com/typst/packages/raw/main/packages/preview/babel/0.1.1/docs/manual.pdf}{\pandocbounded{\includegraphics[keepaspectratio]{https://img.shields.io/badge/\%F0\%9F\%93\%96\%20manual-.pdf-239dad?labelColor=fdfdfd}}}
\href{https://github.com/typst/packages/raw/main/packages/preview/babel/0.1.1/LICENSE}{\pandocbounded{\includegraphics[keepaspectratio]{https://img.shields.io/badge/licence-MIT0-239dad?labelColor=fdfdfd}}}
\href{https://codeberg.org/afiaith/babel/releases/}{\pandocbounded{\includegraphics[keepaspectratio]{https://img.shields.io/gitea/v/release/afiaith/babel?gitea_url=https\%3A\%2F\%2Fcodeberg.org&labelColor=fdfdfd&color=239dad}}}
\href{https://codeberg.org/afiaith/babel/stars}{\pandocbounded{\includegraphics[keepaspectratio]{https://img.shields.io/gitea/stars/afiaith/babel?gitea_url=https\%3A\%2F\%2Fcodeberg.org&labelColor=fdfdfd&color=fdfdfd&logo=codeberg}}}

This package provides functions that replace actual text with random
characters, which is useful for redacting confidential information or
sharing the design and structure of an existing document without
disclosing the content itself. A variety of ready-made sets of
characters for replacement are available (75 in total; termed
\emph{alphabets} ), representing diverse writing systems, codes,
notations and symbols. Some of these are more conservative (such as
emulating redaction using a wide black pen) and many are more whimsical,
as demonstrated by the following example:

\begin{Shaded}
\begin{Highlighting}[]
\NormalTok{\#baffle(alphabet: "welsh")[Hello]. My \#tippex[name] is \#baffle(alphabet: "underscore")[Inigo Montoya]. You \#baffle(alphabet: "alchemy")[killed] my \#baffle(alphabet: "shavian")[father]. Prepare to \#redact[die].}

\NormalTok{Using show rules strings, regular expressions and other selectors can be redacted automatically:}

\NormalTok{\#show "jan Maja": baffle.with(alphabet: "sitelen{-}pona")}
\NormalTok{\#show regex("[a{-}zA{-}Z0{-}9.!\#$\%\&’*+/=?\^{}\_\textasciigrave{}\{|\}\textasciitilde{}{-}]+@[a{-}zA{-}Z0{-}9{-}]+(?:\textbackslash{}.[a{-}zA{-}Z0{-}9{-}]+)*"): baffle.with(alphabet: "maze{-}3") }

\NormalTok{I’m jan Maja, and my email is \textasciigrave{}foo@digitalwords.net\textasciigrave{}.}
\end{Highlighting}
\end{Shaded}

\pandocbounded{\includegraphics[keepaspectratio]{https://github.com/typst/packages/raw/main/packages/preview/babel/0.1.1/assets/example.webp}}

\subsection{ðŸ``-- The manual}\label{uxf0uxff-the-manual}

Using { Babel } is quite straightforward. A
\href{https://github.com/typst/packages/raw/main/packages/preview/babel/0.1.1/docs/manual.pdf}{\textbf{comprehensive
manual}} covers:

\begin{itemize}
\tightlist
\item
  Introductory background.
\item
  How to use the provided functions ( \texttt{\ baffle()\ } ,
  \texttt{\ redact()\ } and \texttt{\ tippex()\ } ).
\item
  A list of the provided alphabets, each demonstrated by a line of
  random text.
\end{itemize}

If the version of the precompiled manual doesn’t match the version of
the package, it means no difference between the two versions is
reflected in the manual.

\subsection{\texorpdfstring{ðŸ---¼ The Tower of { Babel
}}{ðŸ---¼ The Tower of  Babel }}\label{uxf0uxffuxbc-the-tower-of-babel}

A poster demonstrating the provided alphabets:

\href{https://github.com/typst/packages/raw/main/packages/preview/babel/0.1.1/assets/poster.webp}{\pandocbounded{\includegraphics[keepaspectratio]{https://github.com/typst/packages/raw/main/packages/preview/babel/0.1.1/assets/poster.webp}}}

\subsection{ðŸ''¨ Complementary
tools}\label{uxf0uxff-complementary-tools}

If you wish to share the Typst source files of your document, not just
the precompiled output, a tool called
\href{https://github.com/frozolotl/typst-mutilate}{\emph{Typst
Mutilate}} might be useful for you. Unlike { Babel } , it is not a Typst
package but an external tool, written in Rust. It replaces the content
of a Typst document with random words selected from a wordlist or random
characters (similarly to { Babel } ), changing the document in place (so
make sure to run it on a \emph{copy} !). As a package for Typst, { Babel
} cannot change your source files.

\subsubsection{How to add}\label{how-to-add}

Copy this into your project and use the import as \texttt{\ babel\ }

\begin{verbatim}
#import "@preview/babel:0.1.1"
\end{verbatim}

\includesvg[width=0.16667in,height=0.16667in]{/assets/icons/16-copy.svg}

Check the docs for
\href{https://typst.app/docs/reference/scripting/\#packages}{more
information on how to import packages} .

\subsubsection{About}\label{about}

\begin{description}
\tightlist
\item[Author :]
\href{https://me.digitalwords.net}{Maja Abramski-Kronenberg}
\item[License:]
MIT-0
\item[Current version:]
0.1.1
\item[Last updated:]
October 3, 2024
\item[First released:]
October 3, 2024
\item[Minimum Typst version:]
0.11.0
\item[Archive size:]
46.9 kB
\href{https://packages.typst.org/preview/babel-0.1.1.tar.gz}{\pandocbounded{\includesvg[keepaspectratio]{/assets/icons/16-download.svg}}}
\item[Repository:]
\href{https://codeberg.org/afiaith/babel}{Codeberg}
\item[Categor ies :]
\begin{itemize}
\tightlist
\item[]
\item
  \pandocbounded{\includesvg[keepaspectratio]{/assets/icons/16-world.svg}}
  \href{https://typst.app/universe/search/?category=languages}{Languages}
\item
  \pandocbounded{\includesvg[keepaspectratio]{/assets/icons/16-text.svg}}
  \href{https://typst.app/universe/search/?category=text}{Text}
\item
  \pandocbounded{\includesvg[keepaspectratio]{/assets/icons/16-smile.svg}}
  \href{https://typst.app/universe/search/?category=fun}{Fun}
\end{itemize}
\end{description}

\subsubsection{Where to report issues?}\label{where-to-report-issues}

This package is a project of Maja Abramski-Kronenberg . Report issues on
\href{https://codeberg.org/afiaith/babel}{their repository} . You can
also try to ask for help with this package on the
\href{https://forum.typst.app}{Forum} .

Please report this package to the Typst team using the
\href{https://typst.app/contact}{contact form} if you believe it is a
safety hazard or infringes upon your rights.

\phantomsection\label{versions}
\subsubsection{Version history}\label{version-history}

\begin{longtable}[]{@{}ll@{}}
\toprule\noalign{}
Version & Release Date \\
\midrule\noalign{}
\endhead
\bottomrule\noalign{}
\endlastfoot
0.1.1 & October 3, 2024 \\
\end{longtable}

Typst GmbH did not create this package and cannot guarantee correct
functionality of this package or compatibility with any version of the
Typst compiler or app.


\title{typst.app/universe/package/touying-buaa}

\phantomsection\label{banner}
\phantomsection\label{template-thumbnail}
\pandocbounded{\includegraphics[keepaspectratio]{https://packages.typst.org/preview/thumbnails/touying-buaa-0.2.0-small.webp}}

\section{touying-buaa}\label{touying-buaa}

{ 0.2.0 }

Touying Slide Theme for Beihang University

\href{/app?template=touying-buaa&version=0.2.0}{Create project in app}

\phantomsection\label{readme}
Inspired by
\href{https://github.com/QuadnucYard/touying-theme-seu}{Southeast
University Touying Slide Theme} .

\subsection{Use as Typst Template
Package}\label{use-as-typst-template-package}

Use \texttt{\ typst\ init\ @preview/touying-buaa\ } to create a new
project with this theme.

\begin{Shaded}
\begin{Highlighting}[]
\NormalTok{$ typst init @preview/touying{-}buaa}
\NormalTok{Successfully created new project from @preview/touying{-}buaa:}
\NormalTok{To start writing, run:}
\NormalTok{\textgreater{} cd touying{-}buaa}
\NormalTok{\textgreater{} typst watch main.typ}
\end{Highlighting}
\end{Shaded}

\subsection{Examples}\label{examples}

See
\href{https://github.com/typst/packages/raw/main/packages/preview/touying-buaa/0.2.0/examples}{examples}
and \href{https://coekjan.github.io/touying-buaa}{Github Pages} for more
details.

You can compile the examples by yourself.

\begin{Shaded}
\begin{Highlighting}[]
\NormalTok{$ typst compile ./examples/main.typ {-}{-}root .}
\end{Highlighting}
\end{Shaded}

And the PDF file \texttt{\ ./examples/main.pdf\ } will be generated.

\subsection{License}\label{license}

Licensed under the
\href{https://github.com/typst/packages/raw/main/packages/preview/touying-buaa/0.2.0/LICENSE}{MIT
License} .

\href{/app?template=touying-buaa&version=0.2.0}{Create project in app}

\subsubsection{How to use}\label{how-to-use}

Click the button above to create a new project using this template in
the Typst app.

You can also use the Typst CLI to start a new project on your computer
using this command:

\begin{verbatim}
typst init @preview/touying-buaa:0.2.0
\end{verbatim}

\includesvg[width=0.16667in,height=0.16667in]{/assets/icons/16-copy.svg}

\subsubsection{About}\label{about}

\begin{description}
\tightlist
\item[Author :]
\href{mailto:cn_yzr@qq.com}{Yip Coekjan}
\item[License:]
MIT
\item[Current version:]
0.2.0
\item[Last updated:]
September 8, 2024
\item[First released:]
June 4, 2024
\item[Archive size:]
20.5 kB
\href{https://packages.typst.org/preview/touying-buaa-0.2.0.tar.gz}{\pandocbounded{\includesvg[keepaspectratio]{/assets/icons/16-download.svg}}}
\item[Repository:]
\href{https://github.com/Coekjan/touying-buaa}{GitHub}
\item[Categor y :]
\begin{itemize}
\tightlist
\item[]
\item
  \pandocbounded{\includesvg[keepaspectratio]{/assets/icons/16-presentation.svg}}
  \href{https://typst.app/universe/search/?category=presentation}{Presentation}
\end{itemize}
\end{description}

\subsubsection{Where to report issues?}\label{where-to-report-issues}

This template is a project of Yip Coekjan . Report issues on
\href{https://github.com/Coekjan/touying-buaa}{their repository} . You
can also try to ask for help with this template on the
\href{https://forum.typst.app}{Forum} .

Please report this template to the Typst team using the
\href{https://typst.app/contact}{contact form} if you believe it is a
safety hazard or infringes upon your rights.

\phantomsection\label{versions}
\subsubsection{Version history}\label{version-history}

\begin{longtable}[]{@{}ll@{}}
\toprule\noalign{}
Version & Release Date \\
\midrule\noalign{}
\endhead
\bottomrule\noalign{}
\endlastfoot
0.2.0 & September 8, 2024 \\
\href{https://typst.app/universe/package/touying-buaa/0.1.0/}{0.1.0} &
June 4, 2024 \\
\end{longtable}

Typst GmbH did not create this template and cannot guarantee correct
functionality of this template or compatibility with any version of the
Typst compiler or app.


\title{typst.app/universe/package/flyingcircus}

\phantomsection\label{banner}
\phantomsection\label{template-thumbnail}
\pandocbounded{\includegraphics[keepaspectratio]{https://packages.typst.org/preview/thumbnails/flyingcircus-3.2.0-small.webp}}

\section{flyingcircus}\label{flyingcircus}

{ 3.2.0 }

For creating homebrew documents with the same fancy style as the Flying
Circus book? Provides simple commands to generate a whole aircraft stat
page, vehicle, or even ship.

\href{/app?template=flyingcircus&version=3.2.0}{Create project in app}

\phantomsection\label{readme}
Version 3.2.0

Do you want your homebrew to have the same fancy style as the Flying
Circus book? Do you want a simple command to generate a whole aircraft
stat page, vehicle, or even ship? I’ll bet you do! Take a look at the
Flying Circus Aircraft Catalog Template.

\subsection{Acknowledgments and Useful
Links}\label{acknowledgments-and-useful-links}

Download the fonts from
\href{https://github.com/Tetragramm/flying-circus-typst-template/archive/refs/heads/Fonts.zip}{HERE}
. Install them on your computer, upload them to the Typst web-app
(anywhere in the project is fine) or use the Typst command line option
-\/-font-path to include them.

Based on the style and work (with the permission of) Erika Chappell for
the \href{https://opensketch.itch.io/flying-circus}{Flying Circus RPG} .

Integrates with the
\href{https://tetragramm.github.io/PlaneBuilder/index.html}{Plane
Builder} . Just click the Catalog JSON button at the bottom to save what
you need for this template.

Same with the
\href{https://tetragramm.github.io/VehicleBuilder/}{Vehicle Builder} .

Or check out the \href{https://discord.gg/HKdyUuvmcb}{Discord server} .

\subsection{Getting Started}\label{getting-started}

These instructions will get you a copy of the project up and running on
the typst web app.

\begin{Shaded}
\begin{Highlighting}[]
\NormalTok{\#import "@preview/flyingcircus:3.2.0": *}

\NormalTok{\#show: FlyingCircus.with(}
\NormalTok{  Title: title,}
\NormalTok{  Author: author,}
\NormalTok{  CoverImg: image("images/cover.png"),}
\NormalTok{  Dedication: [It\textquotesingle{}s Alive!!! MUAHAHAHA!],}
\NormalTok{)}

\NormalTok{\#FCPlane(read("My Plane\_stats.json"), Nickname:"My First Plane")}
\end{Highlighting}
\end{Shaded}

\subsection{Usage}\label{usage}

The first thing is the FlyingCircus style.

\begin{Shaded}
\begin{Highlighting}[]
\NormalTok{\#import "@preview/flyingcircus:3.2.0": *}

\NormalTok{/// Defines the FlyingCircus template}
\NormalTok{///}
\NormalTok{/// {-} Title (str): Title of the document. Goes in metadata and on title page.}
\NormalTok{/// {-} Author (str): Author(s) of the document. Goes in metadata and on title page.}
\NormalTok{/// {-} CoverImg (image): Image to make the first page of the document.}
\NormalTok{/// {-} Description (str): Text to go with the title on the title page.}
\NormalTok{/// {-} Dedication (str): Dedication to go down below the title on the title page.}
\NormalTok{/// {-} body (content)}
\NormalTok{/// {-}\textgreater{} content}

\NormalTok{// Example}
\NormalTok{\#show: FlyingCircus.with(}
\NormalTok{  Title: title,}
\NormalTok{  Author: author,}
\NormalTok{  CoverImg: image("images/cover.png"),}
\NormalTok{  Dedication: [It\textquotesingle{}s Alive!!! MUAHAHAHA!],}
\NormalTok{)}
\end{Highlighting}
\end{Shaded}

Next is the FCPlane function for making plane pages.

\begin{Shaded}
\begin{Highlighting}[]
\NormalTok{/// Defines the FlyingCircus Plane page.  Always on a new page. Image optional.}
\NormalTok{///}
\NormalTok{/// {-} Plane (str | dictionary): JSON string or dictionary representing the plane stats.}
\NormalTok{/// {-} Nickname (str): Nickname to go under the aircraft name.}
\NormalTok{/// {-} Img (image | none): Image to go at the top of the page. Set to none to remove.}
\NormalTok{/// {-} BoxText (dictionary): Pairs of values to go in the box over the image. Does nothing if no Img provided.}
\NormalTok{/// {-} BoxAnchor (str): Which anchor of the image to put the box in?  Sample values are "north", "south{-}west", "center".}
\NormalTok{/// {-} DescriptiveText (content)}
\NormalTok{/// {-}\textgreater{} content}


\NormalTok{// Example}
\NormalTok{\#FCPlane(}
\NormalTok{  read("Basic Biplane\_stats.json"),}
\NormalTok{  Nickname: "Bring home the bacon!",}
\NormalTok{  Img: image("images/Bergziegel\_image.png"),}
\NormalTok{  BoxText: ("Role": "Fast Bomber", "First Flight": "1601", "Strengths": "Fastest Bomber"),}
\NormalTok{  BoxAnchor: "north{-}east",}
\NormalTok{)[}
\NormalTok{\#lorem(100)}
\NormalTok{]}
\end{Highlighting}
\end{Shaded}

The FCVehicleSimple is for when you want to put multiple vehicles on a
page.

\begin{Shaded}
\begin{Highlighting}[]
\NormalTok{/// Defines the FlyingCircus Simple Vehicle.  Not always a full page. Image optional.}
\NormalTok{///}
\NormalTok{/// {-} Vehicle (str | dictionary): JSON string or dictionary representing the Vehicle stats.}
\NormalTok{/// {-} Img (image): Image to go above the vehicle. (optional)}
\NormalTok{/// {-} DescriptiveText (content)}
\NormalTok{/// {-}\textgreater{} content}
\NormalTok{\#FCVehicleSimple(read("Sample Vehicle\_stats.json"))[\#lorem(120)]}
\end{Highlighting}
\end{Shaded}

FCVehicleFancy is a one or two page vehicle that looks nicer but takes
up more space.

\begin{Shaded}
\begin{Highlighting}[]
\NormalTok{/// Defines the FlyingCircus Vehicle page.  Always on a new page. Image optional.}
\NormalTok{/// If the Img is provided, it will take up two facing pages, otherwise only one, but a full page, unlike the Simple.}
\NormalTok{///}
\NormalTok{/// {-} Vehicle (str | dictionary): JSON string or dictionary representing the Vehicle stats.}
\NormalTok{/// {-} Img (image | none): Image to go at the top of the first page. Set to none to remove.}
\NormalTok{/// {-} TextVOffset (length): How far to push the text down the page. Want to do that inset text thing the book does? You can, the text can overlap with thte image.  Does nothing if no Img provided.}
\NormalTok{/// {-} BoxText (dictionary): Pairs of values to go in the box over the image. Does nothing if no Img provided.}
\NormalTok{/// {-} BoxAnchor (str): Which anchor of the image to put the box in?  Sample values are "north", "south{-}west", "center".}
\NormalTok{/// {-} FirstPageContent (content): Goes on the first page. If no image is provided, it is not present.}
\NormalTok{/// {-} AfterContent (content): Goes after the stat block. Always present.}
\NormalTok{/// {-}\textgreater{} content}

\NormalTok{// Example }
\NormalTok{\#FCVehicleFancy(}
\NormalTok{  read("Sample Vehicle\_stats.json"),}
\NormalTok{  Img: image("images/Wandelburg.png"),}
\NormalTok{  TextVOffset: 6.2in,}
\NormalTok{  BoxText: ("Role": "Fast Bomber", "First Flight": "1601", "Strengths": "Fastest Bomber"),}
\NormalTok{  BoxAnchor: "north{-}east",}
\NormalTok{)[}
\NormalTok{\#lorem(100)}
\NormalTok{][}
\NormalTok{\#lorem(100)}
\NormalTok{]}
\end{Highlighting}
\end{Shaded}

Last of the vehicles, FCShip is for boats like Into the Drink.

\begin{Shaded}
\begin{Highlighting}[]
\NormalTok{/// Defines the FlyingCircus Ship page.  Always on a new page. Image optional.}
\NormalTok{///}
\NormalTok{/// {-} Ship (str | dictionary): JSON string or dictionary representing the Ship stats.}
\NormalTok{/// {-} Img (image | none): Image to go at the top of the page. Set to none to remove.}
\NormalTok{/// {-} DescriptiveText (content): Goes below the name and above the stats table.}
\NormalTok{/// {-} notes (content): Goes in the notes section.}
\NormalTok{/// {-}\textgreater{} content}

\NormalTok{// Example: No builder for Ships, so you\textquotesingle{}ll have to put it in your own JSON, or just a dict, like this.}
\NormalTok{\#let ship\_stats = (}
\NormalTok{  Name: "Macchi Frigate",}
\NormalTok{  Speed: 5,}
\NormalTok{  Handling: 15,}
\NormalTok{  Hardness: 9,}
\NormalTok{  Soak: 0,}
\NormalTok{  Strengths: "{-}",}
\NormalTok{  Weaknesses: "{-}",}
\NormalTok{  Weapons: (}
\NormalTok{    (Name: "x2 Light Howitzer", Fore: "x1", Left: "x2", Right: "x2", Rear: "x1"),}
\NormalTok{    (Name: "x6 Pom{-}Pom Gun", Fore: "x2", Left: "x3", Right: "x3", Rear: "x2", Up: "x6"),}
\NormalTok{    (Name: "x2 WMG", Left: "x1", Right: "x1"),}
\NormalTok{  ),}
\NormalTok{  DamageStates: ("", "{-}1 Speed", "{-}3 Guns", "{-}1 Speed", "{-}3 Guns", "Sinking"),}
\NormalTok{)}

\NormalTok{\#FCShip(}
\NormalTok{  Img: image("images/Macchi Frigate.png"),}
\NormalTok{  Ship: ship\_stats,}
\NormalTok{)[}
\NormalTok{  \#lorem(100)}
\NormalTok{][}
\NormalTok{  \#lorem(5)}
\NormalTok{]}
\end{Highlighting}
\end{Shaded}

Additional functions include FCWeapon

\begin{Shaded}
\begin{Highlighting}[]
\NormalTok{/// Defines the FlyingCircus Weapon card. Image optional.}
\NormalTok{///}
\NormalTok{/// {-} Weapon (str | dictionary): JSON string or dictionary representing the Weapon stats.}
\NormalTok{/// {-} Img (image | none): Image to go above the card. Set to none to remove.}
\NormalTok{/// {-} DescriptiveText (content): Goes below the name and above the stats table.}
\NormalTok{/// {-}\textgreater{} content}

\NormalTok{//Example }
\NormalTok{\#FCWeapon(}
\NormalTok{  (Name: "Rifle/Carbine", Cells: (Hits: 1, Damage: 2, AP: 1, Range: "Extreme"), Price: "Scrip", Tags: "Manual"),}
\NormalTok{  Img: image("images/Rifle.png"),}
\NormalTok{)[}
\NormalTok{Note that you can set the text in the cell boxes to whatever you want.}
\NormalTok{]}
\end{Highlighting}
\end{Shaded}

KochFont:

\begin{Shaded}
\begin{Highlighting}[]
\NormalTok{/// Sets the tex to the Koch Fette FC font for people who don\textquotesingle{}t want to remember the font name.}
\NormalTok{///}
\NormalTok{/// {-} body (content)}
\NormalTok{/// {-} ..args: Any valid argument to the text function}
\NormalTok{/// {-}\textgreater{} content}

\NormalTok{// Example }
\NormalTok{\#KochFont(size: 18pt)[Vehicles]}
\end{Highlighting}
\end{Shaded}

and HiddenHeading, which is for adding to the table of contents without
actually putting words on the page.

\begin{Shaded}
\begin{Highlighting}[]
\NormalTok{//If we don\textquotesingle{}t want all our planes at the top level of the table of contents.  EX: if we want}
\NormalTok{// {-} Intro}
\NormalTok{// {-} Story}
\NormalTok{// {-} Planes }
\NormalTok{//   {-} First Plane}
\NormalTok{// We break the page, and create a HiddenHeading, that doesn\textquotesingle{}t show up in the document (Or a normal heading, if that\textquotesingle{}s what you need)}
\NormalTok{//Then we set the heading offset to one so everything after that is indented one level in the table of contents.}
\NormalTok{\#pagebreak()}
\NormalTok{\#HiddenHeading[= Vehicles]}
\NormalTok{\#set heading(offset: 1)}


\NormalTok{New in Version 3.2.0, the FCPlaybook (+utilities), FCNPCShort, and FCAirshipShort}

\NormalTok{//This creates pages like the playbook. Largely customizable, for say, chariots of steel versions.}
\NormalTok{// {-} Name (str) The name of the Playbook}
\NormalTok{// {-} Subhead (str) The text that goes with the name in the header}
\NormalTok{// {-} Character (content) This is the entire left column}
\NormalTok{// {-} Questions (content) This is the top section of the right column, for motivation and trust questions.}
\NormalTok{// {-} Starting (content) Middle section of the right column. Starting Assets, Burdens, Planes, Vices, ect}
\NormalTok{// {-} Stats (content) Bottom section of the right column. Just the four FCPStatTable calls (and a colbreak, probably)}
\NormalTok{// {-} StatNames () Define the stats to draw circles for on the top part of the 2nd page}
\NormalTok{// {-} Triggers (content) List of triggers, includes section, because not all playbooks use the same text there.}
\NormalTok{// {-} Vents (content) List of Vents, customizable like Triggers}
\NormalTok{// {-} Intimacy (content) Bottom section of the left column, for the intimacy move}
\NormalTok{// {-} Moves (content) The entire right column of the second page}
\NormalTok{//}
\NormalTok{// Utilities for use with FCPlaybook}
\NormalTok{// {-} FCPRule()  The full{-}column horizontal line}
\NormalTok{// {-} FCPSection(name: str)[content] The section break}
\NormalTok{//    {-} name (str) The fancy font name on the lft side of the section line, can be blank.}
\NormalTok{//    {-} content The italicized text on the right side of the line, can be blank.}
\NormalTok{// {-} FCPStatTable(name, tagline, stats) For creating Stat tables}
\NormalTok{//    {-} name (str) The name of the profile, to be rendered in smallcaps}
\NormalTok{//    {-} tagline (str) The tagline of the profile, italicized}
\NormalTok{//    {-} stats (dict) A dictionary of stats ex (Hard:"+2") Keys are first row, values are second row, no restrictions otherwise.}
\NormalTok{\#FCPlaybook(}
\NormalTok{  Name: str,}
\NormalTok{  Subhead: str,}
\NormalTok{  Character: content,}
\NormalTok{  Questions: content,}
\NormalTok{  Starting: content,}
\NormalTok{  Stats: content,}
\NormalTok{  StatNames: (),}
\NormalTok{  Triggers: content,}
\NormalTok{  Vents: content,}
\NormalTok{  Intimacy: content,}
\NormalTok{  Moves: content,}
\NormalTok{)}

\NormalTok{// This creates a short NPC profile like that in the back of the aircraft catalogue}
\NormalTok{// {-} plane (dict) Contains the keys }
\NormalTok{//      {-} Name}
\NormalTok{//      {-} Nickname}
\NormalTok{//      {-} Price (optional)}
\NormalTok{//      {-} Upkeep (optional)}
\NormalTok{//      {-} Used (optional)}
\NormalTok{//      {-} Speeds}
\NormalTok{//      {-} Handling}
\NormalTok{//      {-} Structure}
\NormalTok{// {-} img (image) Image to draw}
\NormalTok{// {-} img\_scale (number) What scale to draw the image, relative to the column size}
\NormalTok{// {-} img\_shift\_dx (percent) How far to shift the image in the x direction}
\NormalTok{// {-} img\_shift\_dy (percent) How far to shift the image in the y direction}
\NormalTok{// {-} content The decriptive text to go above the stat block}
\NormalTok{\#FCShortNPC(}
\NormalTok{  plane, }
\NormalTok{  img: none, }
\NormalTok{  img\_scale: 1.5, }
\NormalTok{  img\_shift\_dx: {-}10\%, }
\NormalTok{  img\_shift\_dy: {-}10\%, }
\NormalTok{  content}
\NormalTok{)}


\NormalTok{// This creates a short airship profile like that in the back of the aircraft catalogue}
\NormalTok{// {-} airship (dict) Contains the keys }
\NormalTok{//      {-} Name}
\NormalTok{//      {-} Nickname}
\NormalTok{//      {-} Price (optional)}
\NormalTok{//      {-} Upkeep (optional)}
\NormalTok{//      {-} Used (optional)}
\NormalTok{//      {-} Speed}
\NormalTok{//      {-} Lift}
\NormalTok{//      {-} Handling}
\NormalTok{//      {-} Toughness}
\NormalTok{// {-} img (image) Image to draw}
\NormalTok{// {-} img\_scale (number) What scale to draw the image, relative to the column size}
\NormalTok{// {-} img\_shift\_dx (percent) How far to shift the image in the x direction}
\NormalTok{// {-} img\_shift\_dy (percent) How far to shift the image in the y direction}
\NormalTok{// {-} content The decriptive text to go above the stat block}
\NormalTok{\#FCShortAirship(}
\NormalTok{  airship, }
\NormalTok{  img: none, }
\NormalTok{  img\_scale: 1.5, }
\NormalTok{  img\_shift\_dx: {-}10\%, }
\NormalTok{  img\_shift\_dy: {-}10\%, }
\NormalTok{  content}
\NormalTok{)}
\end{Highlighting}
\end{Shaded}

\href{/app?template=flyingcircus&version=3.2.0}{Create project in app}

\subsubsection{How to use}\label{how-to-use}

Click the button above to create a new project using this template in
the Typst app.

You can also use the Typst CLI to start a new project on your computer
using this command:

\begin{verbatim}
typst init @preview/flyingcircus:3.2.0
\end{verbatim}

\includesvg[width=0.16667in,height=0.16667in]{/assets/icons/16-copy.svg}

\subsubsection{About}\label{about}

\begin{description}
\tightlist
\item[Author :]
Tetragramm
\item[License:]
MIT
\item[Current version:]
3.2.0
\item[Last updated:]
October 25, 2024
\item[First released:]
August 23, 2024
\item[Minimum Typst version:]
0.12.0
\item[Archive size:]
476 kB
\href{https://packages.typst.org/preview/flyingcircus-3.2.0.tar.gz}{\pandocbounded{\includesvg[keepaspectratio]{/assets/icons/16-download.svg}}}
\item[Repository:]
\href{https://github.com/Tetragramm/flying-circus-typst-template}{GitHub}
\item[Discipline s :]
\begin{itemize}
\tightlist
\item[]
\item
  \href{https://typst.app/universe/search/?discipline=engineering}{Engineering}
\item
  \href{https://typst.app/universe/search/?discipline=design}{Design}
\item
  \href{https://typst.app/universe/search/?discipline=transportation}{Transportation}
\end{itemize}
\item[Categor ies :]
\begin{itemize}
\tightlist
\item[]
\item
  \pandocbounded{\includesvg[keepaspectratio]{/assets/icons/16-docs.svg}}
  \href{https://typst.app/universe/search/?category=book}{Book}
\item
  \pandocbounded{\includesvg[keepaspectratio]{/assets/icons/16-smile.svg}}
  \href{https://typst.app/universe/search/?category=fun}{Fun}
\item
  \pandocbounded{\includesvg[keepaspectratio]{/assets/icons/16-layout.svg}}
  \href{https://typst.app/universe/search/?category=layout}{Layout}
\end{itemize}
\end{description}

\subsubsection{Where to report issues?}\label{where-to-report-issues}

This template is a project of Tetragramm . Report issues on
\href{https://github.com/Tetragramm/flying-circus-typst-template}{their
repository} . You can also try to ask for help with this template on the
\href{https://forum.typst.app}{Forum} .

Please report this template to the Typst team using the
\href{https://typst.app/contact}{contact form} if you believe it is a
safety hazard or infringes upon your rights.

\phantomsection\label{versions}
\subsubsection{Version history}\label{version-history}

\begin{longtable}[]{@{}ll@{}}
\toprule\noalign{}
Version & Release Date \\
\midrule\noalign{}
\endhead
\bottomrule\noalign{}
\endlastfoot
3.2.0 & October 25, 2024 \\
\href{https://typst.app/universe/package/flyingcircus/3.0.0/}{3.0.0} &
August 23, 2024 \\
\end{longtable}

Typst GmbH did not create this template and cannot guarantee correct
functionality of this template or compatibility with any version of the
Typst compiler or app.


\title{typst.app/universe/package/ttuile}

\phantomsection\label{banner}
\phantomsection\label{template-thumbnail}
\pandocbounded{\includegraphics[keepaspectratio]{https://packages.typst.org/preview/thumbnails/ttuile-0.1.1-small.webp}}

\section{ttuile}\label{ttuile}

{ 0.1.1 }

A template for students\textquotesingle{} lab reports at INSA Lyon, a
french engineering school.

\href{/app?template=ttuile&version=0.1.1}{Create project in app}

\phantomsection\label{readme}
\href{https://typst.app/}{\pandocbounded{\includegraphics[keepaspectratio]{https://img.shields.io/badge/Typst-\%232f90ba.svg?&logo=Typst&logoColor=white}}}
\href{https://github.com/vitto4/ttuile/blob/main/LICENSE}{\pandocbounded{\includegraphics[keepaspectratio]{https://img.shields.io/github/license/vitto4/ttuile}}}
\href{https://github.com/vitto4/ttuile/releases}{\pandocbounded{\includegraphics[keepaspectratio]{https://img.shields.io/github/v/release/vitto4/ttuile}}}

\emph{A \textbf{Typst} template for lab reports at
\href{https://en.wikipedia.org/wiki/Institut_national_des_sciences_appliqu\%C3\%A9es_de_Lyon}{INSA
Lyon} .}

\href{https://github.com/vitto4/ttuile/blob/main/template/main.pdf}{\pandocbounded{\includegraphics[keepaspectratio]{https://raw.githubusercontent.com/vitto4/ttuile/main/assets/ttuile-banner.png?raw=true}}}

\begin{quote}
\textbf{Note :} Voir aussi le
\href{https://github.com/vitto4/ttuile/blob/main/README.FR.md}{README.FR.md}
en français.
\end{quote}

\subsection{🧭 Table of contents}\label{uxf0uxff-table-of-contents}

\begin{enumerate}
\tightlist
\item
  \href{https://github.com/typst/packages/raw/main/packages/preview/ttuile/0.1.1/\#-usage}{Usage}
\item
  \href{https://github.com/typst/packages/raw/main/packages/preview/ttuile/0.1.1/\#-documentation}{Documentation}
\item
  \href{https://github.com/typst/packages/raw/main/packages/preview/ttuile/0.1.1/\#-notes}{Notes}
\item
  \href{https://github.com/typst/packages/raw/main/packages/preview/ttuile/0.1.1/\#-contributing}{Contributing}
\end{enumerate}

\subsection{ðŸ``Ž Usage}\label{uxf0uxffux17e-usage}

This template targets french students, thus labels will be in french,
see
\href{https://github.com/typst/packages/raw/main/packages/preview/ttuile/0.1.1/\#-notes}{Notes}
.

It is available on \emph{Typst Universe} :
\href{https://typst.app/universe/package/ttuile}{\texttt{\ @preview/ttuile:0.1.1\ }}
.

If you wish to use it in a fully local manner, you’ll need to either
manually include \texttt{\ ttuile.typ\ } and
\texttt{\ logo-insa-lyon.png\ } in your project’s root directory ; or
upload them to the \emph{Typst web app} if that’s what you use.

You’ll find these files in the
\href{https://github.com/vitto4/ttuile/releases}{releases} section.

Your folder structure should then look something like this :

\begin{verbatim}
.
├── ttuile.typ
├── logo-insa-lyon.png
└── main.typ
\end{verbatim}

The template is now ready to be used, and can be called supplying the
following arguments. \texttt{\ ?\ } means the argument can be null if
not applicable.

\begin{longtable}[]{@{}cccl@{}}
\toprule\noalign{}
Argument & Default value & Type & Description \\
\midrule\noalign{}
\endhead
\bottomrule\noalign{}
\endlastfoot
\texttt{\ titre\ } & \texttt{\ none\ } & \texttt{\ content?\ } & The
title of your report. \\
\texttt{\ auteurs\ } & \texttt{\ none\ } &
\texttt{\ array\textless{}str\textgreater{}\ \textbar{}\ content?\ } &
One or multiple authors to be credited in the report. \\
\texttt{\ groupe\ } & \texttt{\ none\ } & \texttt{\ content?\ } & Your
class number/letter/identifier. Will be displayed right after the
author(s). \\
\texttt{\ numero-tp\ } & \texttt{\ none\ } & \texttt{\ content?\ } & The
number/identifier of the lab work/practical you’re writing this report
for. \\
\texttt{\ numero-poste\ } & \texttt{\ none\ } & \texttt{\ content?\ } &
Number of your lab bench. \\
\texttt{\ date\ } & \texttt{\ none\ } &
\texttt{\ datetime\ \textbar{}\ content?\ } & Date at which the lab
work/practical was carried out. \\
\texttt{\ sommaire\ } & \texttt{\ true\ } & \texttt{\ bool\ } & Display
the table of contents ? \\
\texttt{\ logo\ } & \texttt{\ image("logo-insa-lyon.png")\ } &
\texttt{\ image?\ } & University logo to use. \\
\texttt{\ point-legende\ } & \texttt{\ false\ } & \texttt{\ bool\ } &
Enable automatic enforcement of full stops at the end of figures’
captions. (still somewhat experimental). \\
\end{longtable}

A single positional argument is accepted, being the report’s body.

You can call the template using the following syntax :

\begin{Shaded}
\begin{Highlighting}[]
\NormalTok{// Local import}
\NormalTok{// \#import "ttuile.typ": *}

\NormalTok{// Universe import}
\NormalTok{\#import "@preview/ttuile:0.1.1": *}

\NormalTok{\#show: ttuile.with(}
\NormalTok{  titre: [« \#lorem(8) »],}
\NormalTok{  auteurs: (}
\NormalTok{      "Theresa Tungsten",}
\NormalTok{      "Jean Dupont",}
\NormalTok{      "Eugene Deklan",}
\NormalTok{  ),}
\NormalTok{  groupe: "TD0",}
\NormalTok{  numero{-}tp: 0,}
\NormalTok{  numero{-}poste: "0",}
\NormalTok{  date: datetime.today(),}
\NormalTok{  // sommaire: false,}
\NormalTok{  // logo: image("path\_to/logo.png"),}
\NormalTok{  // point{-}legende: true,}
\NormalTok{)}
\end{Highlighting}
\end{Shaded}

\subsection{ðŸ``š Documentation}\label{uxf0uxffux161-documentation}

The package \texttt{\ ttuile.typ\ } exposes multiple functions, find out
more about them in the \emph{documentation} .

\href{https://github.com/vitto4/ttuile/blob/main/DOC.EN.md}{To the
documentation}

An example file is also available in
\href{https://github.com/vitto4/ttuile/blob/main/template/main.typ}{\texttt{\ template/main.typ\ }}

\subsection{ðŸ''-- Notes}\label{uxf0uxff-notes}

\begin{itemize}
\item
  Beware, all of the labels will be in french (authors != auteurs,
  appendix != annexe, …)
\item
  If you really want to use this template despite not being an INSA
  student, you can probably figure out what to change in the code
  (namely labels mentioned above). You can remove the INSA logo by
  setting \texttt{\ logo:\ none\ }

  Should you still need help, no worries, feel free to reach out !
\item
  The code - variable names and comments - is all in french. That’s on
  me, I didn’t really think it through when first writing the template
  haha. I might consider translating sometime in the future.
\item
  The MIT license doesn’t apply to the file
  \texttt{\ logo-insa-lyon.png\ } , it was retrieved from
  \href{https://www.insa-lyon.fr/fr/elements-graphiques}{INSA Lyon -
  éléments graphiques} . It doesn’t apply either to the “INSA�
  branding.
\end{itemize}

\subsection{🧩 Contributing}\label{uxf0uxff-contributing}

Contributions are welcome ! Parts of the template are very much
spaghetti code, especially where the spacing between different headings
is handled (seriously, it’s pretty bad).

If you know the proper way of doing this, an issue or PR would be
greatly appreciated :)

\href{/app?template=ttuile&version=0.1.1}{Create project in app}

\subsubsection{How to use}\label{how-to-use}

Click the button above to create a new project using this template in
the Typst app.

You can also use the Typst CLI to start a new project on your computer
using this command:

\begin{verbatim}
typst init @preview/ttuile:0.1.1
\end{verbatim}

\includesvg[width=0.16667in,height=0.16667in]{/assets/icons/16-copy.svg}

\subsubsection{About}\label{about}

\begin{description}
\tightlist
\item[Author :]
\href{https://github.com/vitto4}{vitto}
\item[License:]
MIT
\item[Current version:]
0.1.1
\item[Last updated:]
May 6, 2024
\item[First released:]
May 3, 2024
\item[Archive size:]
46.8 kB
\href{https://packages.typst.org/preview/ttuile-0.1.1.tar.gz}{\pandocbounded{\includesvg[keepaspectratio]{/assets/icons/16-download.svg}}}
\item[Repository:]
\href{https://github.com/vitto4/ttuile}{GitHub}
\item[Discipline :]
\begin{itemize}
\tightlist
\item[]
\item
  \href{https://typst.app/universe/search/?discipline=engineering}{Engineering}
\end{itemize}
\item[Categor y :]
\begin{itemize}
\tightlist
\item[]
\item
  \pandocbounded{\includesvg[keepaspectratio]{/assets/icons/16-speak.svg}}
  \href{https://typst.app/universe/search/?category=report}{Report}
\end{itemize}
\end{description}

\subsubsection{Where to report issues?}\label{where-to-report-issues}

This template is a project of vitto . Report issues on
\href{https://github.com/vitto4/ttuile}{their repository} . You can also
try to ask for help with this template on the
\href{https://forum.typst.app}{Forum} .

Please report this template to the Typst team using the
\href{https://typst.app/contact}{contact form} if you believe it is a
safety hazard or infringes upon your rights.

\phantomsection\label{versions}
\subsubsection{Version history}\label{version-history}

\begin{longtable}[]{@{}ll@{}}
\toprule\noalign{}
Version & Release Date \\
\midrule\noalign{}
\endhead
\bottomrule\noalign{}
\endlastfoot
0.1.1 & May 6, 2024 \\
\href{https://typst.app/universe/package/ttuile/0.1.0/}{0.1.0} & May 3,
2024 \\
\end{longtable}

Typst GmbH did not create this template and cannot guarantee correct
functionality of this template or compatibility with any version of the
Typst compiler or app.


\title{typst.app/universe/package/clear-iclr}

\phantomsection\label{banner}
\phantomsection\label{template-thumbnail}
\pandocbounded{\includegraphics[keepaspectratio]{https://packages.typst.org/preview/thumbnails/clear-iclr-0.4.0-small.webp}}

\section{clear-iclr}\label{clear-iclr}

{ 0.4.0 }

Paper template for submission to International Conference on Learning
Representations (ICLR)

{ } Featured Template

\href{/app?template=clear-iclr&version=0.4.0}{Create project in app}

\phantomsection\label{readme}
\subsection{Usage}\label{usage}

You can use this template in the Typst web app by clicking \emph{Start
from template} on the dashboard and searching for
\texttt{\ clear-iclr\ } .

Alternatively, you can use the CLI to kick this project off using the
command

\begin{Shaded}
\begin{Highlighting}[]
\NormalTok{typst init @preview/clear{-}iclr}
\end{Highlighting}
\end{Shaded}

Typst will create a new directory with all the files needed to get you
started.

\subsection{Configuration}\label{configuration}

This template exports the \texttt{\ iclr\ } function with the following
named arguments.

\begin{itemize}
\item
  \texttt{\ title\ } : The paper’s title as content.
\item
  \texttt{\ authors\ } : An array of author dictionaries. Each of the
  author dictionaries must have a name key and can have the keys
  department, organization, location, and email.

\begin{Shaded}
\begin{Highlighting}[]
\NormalTok{\#let authors = (}
\NormalTok{  ...,}
\NormalTok{  (}
\NormalTok{    names: ([Coauthor1], [Coauthor2]),}
\NormalTok{    affilation: [Affiliation],}
\NormalTok{    address: [Address],}
\NormalTok{    email: "correspondent@example.org",}
\NormalTok{  ),}
\NormalTok{  ...}
\NormalTok{)}
\end{Highlighting}
\end{Shaded}
\item
  \texttt{\ keywords\ } : Publication keywords (used in PDF metadata).
\item
  \texttt{\ date\ } : Creation date (used in PDF metadata).
\item
  \texttt{\ abstract\ } : The content of a brief summary of the paper or
  none. Appears at the top under the title.
\item
  \texttt{\ bibliography\ } : The result of a call to the bibliography
  function or none. The function also accepts a single, positional
  argument for the body of the paper.
\item
  \texttt{\ appendix\ } : Content to append after bibliography section
  (can be included).
\item
  \texttt{\ accepted\ } : If this is set to \texttt{\ false\ } then
  anonymized ready for submission document is produced;
  \texttt{\ accepted:\ true\ } produces camera-redy version. If the
  argument is set to \texttt{\ none\ } then preprint version is produced
  (can be uploaded to arXiv).
\end{itemize}

The template will initialize your package with a sample call to the
\texttt{\ iclr\ } function in a show rule. If you want to change an
existing project to use this template, you can add a show rule at the
top of your file.

\subsection{Issues}\label{issues}

This template is developed at
\href{https://github.com/daskol/typst-templates}{daskol/typst-templates}
repo. Please report all issues there.

\begin{itemize}
\item
  Common issue is related to Typst’s inablity to produce colored
  annotation. In order to mitigte the issue, we add a script which
  modifies annotations and make them colored.

\begin{Shaded}
\begin{Highlighting}[]
\NormalTok{../colorize{-}annotations.py \textbackslash{}}
\NormalTok{    example{-}paper.typst.pdf example{-}paper{-}colored.typst.pdf}
\end{Highlighting}
\end{Shaded}

  See {[} \href{http://readme.md/}{README.md} {]}{[}3{]} for details.
\item
  The author instructions says that preferable font is MS Times New
  Roman but the official example paper uses serifs like Computer Modern
  and Nimbus font families. Monospace fonts are not specified.
\item
  ICML-like bibliography style. The bibliography slightly differs from
  the one in the original example paper. The main difference is that we
  prefer to use author’s lastname at first place to search an entry
  faster.
\end{itemize}

\href{/app?template=clear-iclr&version=0.4.0}{Create project in app}

\subsubsection{How to use}\label{how-to-use}

Click the button above to create a new project using this template in
the Typst app.

You can also use the Typst CLI to start a new project on your computer
using this command:

\begin{verbatim}
typst init @preview/clear-iclr:0.4.0
\end{verbatim}

\includesvg[width=0.16667in,height=0.16667in]{/assets/icons/16-copy.svg}

\subsubsection{About}\label{about}

\begin{description}
\tightlist
\item[Author :]
\href{mailto:d.bershatsky2@skoltech.ru}{Daniel Bershatsky}
\item[License:]
MIT
\item[Current version:]
0.4.0
\item[Last updated:]
September 11, 2024
\item[First released:]
September 11, 2024
\item[Minimum Typst version:]
0.11.1
\item[Archive size:]
21.4 kB
\href{https://packages.typst.org/preview/clear-iclr-0.4.0.tar.gz}{\pandocbounded{\includesvg[keepaspectratio]{/assets/icons/16-download.svg}}}
\item[Repository:]
\href{https://github.com/daskol/typst-templates}{GitHub}
\item[Discipline s :]
\begin{itemize}
\tightlist
\item[]
\item
  \href{https://typst.app/universe/search/?discipline=computer-science}{Computer
  Science}
\item
  \href{https://typst.app/universe/search/?discipline=mathematics}{Mathematics}
\end{itemize}
\item[Categor y :]
\begin{itemize}
\tightlist
\item[]
\item
  \pandocbounded{\includesvg[keepaspectratio]{/assets/icons/16-atom.svg}}
  \href{https://typst.app/universe/search/?category=paper}{Paper}
\end{itemize}
\end{description}

\subsubsection{Where to report issues?}\label{where-to-report-issues}

This template is a project of Daniel Bershatsky . Report issues on
\href{https://github.com/daskol/typst-templates}{their repository} . You
can also try to ask for help with this template on the
\href{https://forum.typst.app}{Forum} .

Please report this template to the Typst team using the
\href{https://typst.app/contact}{contact form} if you believe it is a
safety hazard or infringes upon your rights.

\phantomsection\label{versions}
\subsubsection{Version history}\label{version-history}

\begin{longtable}[]{@{}ll@{}}
\toprule\noalign{}
Version & Release Date \\
\midrule\noalign{}
\endhead
\bottomrule\noalign{}
\endlastfoot
0.4.0 & September 11, 2024 \\
\end{longtable}

Typst GmbH did not create this template and cannot guarantee correct
functionality of this template or compatibility with any version of the
Typst compiler or app.


\title{typst.app/universe/package/typslides}

\phantomsection\label{banner}
\section{typslides}\label{typslides}

{ 1.2.1 }

Minimalistic Typst slides

\phantomsection\label{readme}
\includegraphics[width=4.16667in,height=\textheight,keepaspectratio]{https://github.com/typst/packages/raw/main/packages/preview/typslides/1.2.1/img/logo.png}

\pandocbounded{\includegraphics[keepaspectratio]{https://img.shields.io/badge/license-GPLv3-blue}}
\href{https://github.com/typst/packages/raw/main/packages/preview/typslides/1.2.1/}{\pandocbounded{\includegraphics[keepaspectratio]{https://badgen.net/github/contributors/manjavacas/typslides}}}
\href{https://github.com/typst/packages/raw/main/packages/preview/typslides/1.2.1/}{\pandocbounded{\includegraphics[keepaspectratio]{https://badgen.net/github/release/manjavacas/typslides}}}
\pandocbounded{\includegraphics[keepaspectratio]{https://img.shields.io/github/stars/manjavacas/typslides}}

\emph{Minimalistic \href{https://typst.app/}{typst} slides!}

This is a simple usage example:

\begin{Shaded}
\begin{Highlighting}[]
\NormalTok{\#import "@preview/typslides:1.2.1": *}

\NormalTok{// Project configuration}
\NormalTok{\#show: typslides.with(}
\NormalTok{  ratio: "16{-}9",}
\NormalTok{  theme: "bluey",}
\NormalTok{)}

\NormalTok{// The front slide is the first slide of your presentation}
\NormalTok{\#front{-}slide(}
\NormalTok{  title: "This is a sample presentation",}
\NormalTok{  subtitle: [Using \_typslides\_],}
\NormalTok{  authors: "Antonio Manjavacas",}
\NormalTok{  info: [\#link("https://github.com/manjavacas/typslides")],}
\NormalTok{)}

\NormalTok{// Custom outline}
\NormalTok{\#table{-}of{-}contents()}

\NormalTok{// Title slides create new sections}
\NormalTok{\#title{-}slide[}
\NormalTok{  This is a \_Title slide\_}
\NormalTok{]}

\NormalTok{// A simple slide}
\NormalTok{\#slide[}
\NormalTok{  {-} This is a simple \textasciigrave{}slide\textasciigrave{} with no title.}
\NormalTok{  {-} \#stress("Bold and coloured") text by using \textasciigrave{}\#stress(text)\textasciigrave{}.}
\NormalTok{  {-} Sample link: \#link("typst.app")}
\NormalTok{  {-} Sample references: @typst, @typslides.}

\NormalTok{  \#framed[This text has been written using \textasciigrave{}\#framed(text)\textasciigrave{}. The background color of the box is customisable.]}

\NormalTok{  \#framed(title: "Frame with title")[This text has been written using \textasciigrave{}\#framed(title:"Frame with title")[text]\textasciigrave{}.]}
\NormalTok{]}

\NormalTok{// Focus slide}
\NormalTok{\#focus{-}slide[}
\NormalTok{  This is an auto{-}resized \_focus slide\_.}
\NormalTok{]}

\NormalTok{// Blank slide}
\NormalTok{\#blank{-}slide[}
\NormalTok{  {-} This is a \textasciigrave{}\#blank{-}slide\textasciigrave{}.}

\NormalTok{  {-} Available \#stress[themes]:}

\NormalTok{  \#text(fill: rgb("3059AB"), weight: "bold")[bluey]}
\NormalTok{  \#text(fill: rgb("BF3D3D"), weight: "bold")[greeny]}
\NormalTok{  \#text(fill: rgb("28842F"), weight: "bold")[reddy]}
\NormalTok{  \#text(fill: rgb("C4853D"), weight: "bold")[yelly]}
\NormalTok{  \#text(fill: rgb("862A70"), weight: "bold")[purply]}
\NormalTok{  \#text(fill: rgb("1F4289"), weight: "bold")[dusky]}
\NormalTok{  \#text(fill: black, weight: "bold")[darky]}
\NormalTok{]}

\NormalTok{// Slide with title}
\NormalTok{\#slide(title: "This is the slide title")[}
\NormalTok{  \#lorem(20)}
\NormalTok{  \#grayed([This is a \textasciigrave{}\#grayed\textasciigrave{} text. Useful for equations.])}
\NormalTok{  \#grayed($ P\_t = alpha {-} 1 / (sqrt(x) + f(y)) $)}
\NormalTok{  \#lorem(20)}
\NormalTok{]}

\NormalTok{// Bibliography}
\NormalTok{\#bibliography{-}slide("bibliography.bib")}
\end{Highlighting}
\end{Shaded}

{
\includesvg[width=3.125in,height=\textheight,keepaspectratio]{https://github.com/typst/packages/raw/main/packages/preview/typslides/1.2.1/img/slide-1.svg}
} {
\includesvg[width=3.125in,height=\textheight,keepaspectratio]{https://github.com/typst/packages/raw/main/packages/preview/typslides/1.2.1/img/slide-2.svg}
} {
\includesvg[width=3.125in,height=\textheight,keepaspectratio]{https://github.com/typst/packages/raw/main/packages/preview/typslides/1.2.1/img/slide-3.svg}
} {
\includesvg[width=3.125in,height=\textheight,keepaspectratio]{https://github.com/typst/packages/raw/main/packages/preview/typslides/1.2.1/img/slide-4.svg}
} {
\includesvg[width=3.125in,height=\textheight,keepaspectratio]{https://github.com/typst/packages/raw/main/packages/preview/typslides/1.2.1/img/slide-5.svg}
} {
\includesvg[width=3.125in,height=\textheight,keepaspectratio]{https://github.com/typst/packages/raw/main/packages/preview/typslides/1.2.1/img/slide-6.svg}
} {
\includesvg[width=3.125in,height=\textheight,keepaspectratio]{https://github.com/typst/packages/raw/main/packages/preview/typslides/1.2.1/img/slide-7.svg}
} {
\includesvg[width=3.125in,height=\textheight,keepaspectratio]{https://github.com/typst/packages/raw/main/packages/preview/typslides/1.2.1/img/slide-8.svg}
}

\subsubsection{How to add}\label{how-to-add}

Copy this into your project and use the import as \texttt{\ typslides\ }

\begin{verbatim}
#import "@preview/typslides:1.2.1"
\end{verbatim}

\includesvg[width=0.16667in,height=0.16667in]{/assets/icons/16-copy.svg}

Check the docs for
\href{https://typst.app/docs/reference/scripting/\#packages}{more
information on how to import packages} .

\subsubsection{About}\label{about}

\begin{description}
\tightlist
\item[Author :]
\href{https://github.com/manjavacas}{Antonio Manjavacas}
\item[License:]
GPL-3.0
\item[Current version:]
1.2.1
\item[Last updated:]
November 22, 2024
\item[First released:]
October 29, 2024
\item[Minimum Typst version:]
0.12.0
\item[Archive size:]
15.8 kB
\href{https://packages.typst.org/preview/typslides-1.2.1.tar.gz}{\pandocbounded{\includesvg[keepaspectratio]{/assets/icons/16-download.svg}}}
\item[Repository:]
\href{https://github.com/manjavacas/typslides}{GitHub}
\item[Categor ies :]
\begin{itemize}
\tightlist
\item[]
\item
  \pandocbounded{\includesvg[keepaspectratio]{/assets/icons/16-presentation.svg}}
  \href{https://typst.app/universe/search/?category=presentation}{Presentation}
\item
  \pandocbounded{\includesvg[keepaspectratio]{/assets/icons/16-layout.svg}}
  \href{https://typst.app/universe/search/?category=layout}{Layout}
\end{itemize}
\end{description}

\subsubsection{Where to report issues?}\label{where-to-report-issues}

This package is a project of Antonio Manjavacas . Report issues on
\href{https://github.com/manjavacas/typslides}{their repository} . You
can also try to ask for help with this package on the
\href{https://forum.typst.app}{Forum} .

Please report this package to the Typst team using the
\href{https://typst.app/contact}{contact form} if you believe it is a
safety hazard or infringes upon your rights.

\phantomsection\label{versions}
\subsubsection{Version history}\label{version-history}

\begin{longtable}[]{@{}ll@{}}
\toprule\noalign{}
Version & Release Date \\
\midrule\noalign{}
\endhead
\bottomrule\noalign{}
\endlastfoot
1.2.1 & November 22, 2024 \\
\href{https://typst.app/universe/package/typslides/1.2.0/}{1.2.0} &
November 12, 2024 \\
\href{https://typst.app/universe/package/typslides/1.1.1/}{1.1.1} &
October 29, 2024 \\
\end{longtable}

Typst GmbH did not create this package and cannot guarantee correct
functionality of this package or compatibility with any version of the
Typst compiler or app.


\title{typst.app/universe/package/big-rati}

\phantomsection\label{banner}
\section{big-rati}\label{big-rati}

{ 0.1.0 }

Utilities to work with big rational numbers in Typst

\phantomsection\label{readme}
\texttt{\ big-rati\ } is a package to work with rational numbers in
Typst

\subsection{Usage}\label{usage}

\begin{Shaded}
\begin{Highlighting}[]
\NormalTok{\#import "@preview/big{-}rati:0.1.0"}

\NormalTok{\#let a = 2      // 2/1}
\NormalTok{\#let b = (1, 2) // 1/2}

\NormalTok{\#let sum = big{-}rati.add(a, b) // 5/2}

\NormalTok{\#let c = ("4", 6)}
\NormalTok{\#let prod = big{-}rati.mul(c, sum) // 5/3}

\NormalTok{$\#big{-}rati.repr(prod)$}
\end{Highlighting}
\end{Shaded}

Functions, exported by the package are:

\begin{Shaded}
\begin{Highlighting}[]
\NormalTok{// Converts \textasciigrave{}x\textasciigrave{} to bytes, representing the rational number,}
\NormalTok{// that can be used in the functions below.}
\NormalTok{// \textasciigrave{}x\textasciigrave{} might be an integer or a big integer string.}
\NormalTok{// If \textasciigrave{}x\textasciigrave{} is an array of length two, which elements are integers}
\NormalTok{// or big integer strings, then it is converted to the array of all}
\NormalTok{// big integer strings, and then into the bytes representation.}
\NormalTok{\#let rational(x)}

\NormalTok{// Functions below work with "rational numbers", integers or big integer strings}

\NormalTok{// Returns \textasciigrave{}a + b\textasciigrave{}}
\NormalTok{\#let add(a, b)}

\NormalTok{// Returns \textasciigrave{}a {-} b\textasciigrave{}}
\NormalTok{\#let sub(a, b)}

\NormalTok{// Returns \textasciigrave{}a / b\textasciigrave{}}
\NormalTok{\#let div(a, b)}

\NormalTok{// Returns \textasciigrave{}a * b\textasciigrave{}}
\NormalTok{\#let mul(a, b)}

\NormalTok{// Returns \textasciigrave{}a \% b\textasciigrave{}}
\NormalTok{\#let rem(a, b)}

\NormalTok{// Returns \textasciigrave{}|a {-} b|\textasciigrave{}}
\NormalTok{\#let abs{-}diff(a, b)}

\NormalTok{// Returns \textasciigrave{}{-}1\textasciigrave{} if \textasciigrave{}a \textless{} b\textasciigrave{}, \textasciigrave{}0\textasciigrave{} if \textasciigrave{}a == b\textasciigrave{}, \textasciigrave{}1\textasciigrave{} if \textasciigrave{}a \textgreater{} b\textasciigrave{}}
\NormalTok{\#let cmp(a, b)}

\NormalTok{// Returns \textasciigrave{}{-}x\textasciigrave{}}
\NormalTok{\#let neg(x)}

\NormalTok{// Returns \textasciigrave{}|x|\textasciigrave{}}
\NormalTok{\#let abs(x)}

\NormalTok{// Rounds towards plus infinity}
\NormalTok{\#let ceil(x)}

\NormalTok{// Rounds towards minus infinity}
\NormalTok{\#let floor(x)}

\NormalTok{// Rounds to the nearest integer. Rounds half{-}way cases away from zero.}
\NormalTok{\#let round(x)}

\NormalTok{// Rounds towards zero.}
\NormalTok{\#let trunc(x)}

\NormalTok{// Returns the fractional part of a number, with division rounded towards zero.}
\NormalTok{// Satisfies \textasciigrave{}number == add(trunc(number), fract(number))\textasciigrave{}.}
\NormalTok{\#let fract(number)}

\NormalTok{// Returns the reciprocal.}
\NormalTok{// Panics if the number is zero.}
\NormalTok{\#let recip(x)}

\NormalTok{// Returns \textasciigrave{}x\^{}y\textasciigrave{}. \textasciigrave{}y\textasciigrave{} must be an \textasciigrave{}int\textasciigrave{}, in range of \textasciigrave{}{-}2\^{}32\textasciigrave{} to \textasciigrave{}2\^{}32 {-} 1\textasciigrave{}}
\NormalTok{\#let pow(x, y)}

\NormalTok{// Restrict a value to a certain interval.}
\NormalTok{//}
\NormalTok{// Returns \textasciigrave{}max\textasciigrave{} if \textasciigrave{}number\textasciigrave{} is greater than \textasciigrave{}max\textasciigrave{},}
\NormalTok{// and \textasciigrave{}min\textasciigrave{} if \textasciigrave{}number\textasciigrave{} is less than \textasciigrave{}min\textasciigrave{}.}
\NormalTok{// Otherwise returns \textasciigrave{}number\textasciigrave{}.}
\NormalTok{//}
\NormalTok{// Returns error if \textasciigrave{}min\textasciigrave{} is greater than \textasciigrave{}max\textasciigrave{}.}
\NormalTok{\#let clamp(number, min, max)}

\NormalTok{// Compares and returns the minimum of two values.}
\NormalTok{\#let min(a, b)}

\NormalTok{// Compares and returns the maximum of two values.}
\NormalTok{\#let max(a, b)}

\NormalTok{// Returns a value of type \textasciigrave{}content\textasciigrave{}, representing the rational number.}
\NormalTok{// If \textasciigrave{}is{-}mixed\textasciigrave{} is true, then the result is a mixed fraction,}
\NormalTok{// otherwise, it is a simple fraction.}
\NormalTok{\#let repr(x, is{-}mixed: true)}

\NormalTok{// Returns a string, representing the rational number}
\NormalTok{\#let to{-}decimal{-}str(x, precision: 8)}

\NormalTok{// Returns a floating{-}point number, representing the rational number}
\NormalTok{\#let to{-}float(x, precision: 8)}

\NormalTok{// Returns a decimal number, representing the rational number}
\NormalTok{\#let to{-}decimal(x, precision: 8)}
\end{Highlighting}
\end{Shaded}

\subsubsection{How to add}\label{how-to-add}

Copy this into your project and use the import as \texttt{\ big-rati\ }

\begin{verbatim}
#import "@preview/big-rati:0.1.0"
\end{verbatim}

\includesvg[width=0.16667in,height=0.16667in]{/assets/icons/16-copy.svg}

Check the docs for
\href{https://typst.app/docs/reference/scripting/\#packages}{more
information on how to import packages} .

\subsubsection{About}\label{about}

\begin{description}
\tightlist
\item[Author :]
Danik Vitek
\item[License:]
MIT
\item[Current version:]
0.1.0
\item[Last updated:]
October 29, 2024
\item[First released:]
October 29, 2024
\item[Archive size:]
33.9 kB
\href{https://packages.typst.org/preview/big-rati-0.1.0.tar.gz}{\pandocbounded{\includesvg[keepaspectratio]{/assets/icons/16-download.svg}}}
\item[Repository:]
\href{https://github.com/DanikVitek/typst-plugin-bigrational}{GitHub}
\item[Discipline :]
\begin{itemize}
\tightlist
\item[]
\item
  \href{https://typst.app/universe/search/?discipline=mathematics}{Mathematics}
\end{itemize}
\item[Categor y :]
\begin{itemize}
\tightlist
\item[]
\item
  \pandocbounded{\includesvg[keepaspectratio]{/assets/icons/16-code.svg}}
  \href{https://typst.app/universe/search/?category=scripting}{Scripting}
\end{itemize}
\end{description}

\subsubsection{Where to report issues?}\label{where-to-report-issues}

This package is a project of Danik Vitek . Report issues on
\href{https://github.com/DanikVitek/typst-plugin-bigrational}{their
repository} . You can also try to ask for help with this package on the
\href{https://forum.typst.app}{Forum} .

Please report this package to the Typst team using the
\href{https://typst.app/contact}{contact form} if you believe it is a
safety hazard or infringes upon your rights.

\phantomsection\label{versions}
\subsubsection{Version history}\label{version-history}

\begin{longtable}[]{@{}ll@{}}
\toprule\noalign{}
Version & Release Date \\
\midrule\noalign{}
\endhead
\bottomrule\noalign{}
\endlastfoot
0.1.0 & October 29, 2024 \\
\end{longtable}

Typst GmbH did not create this package and cannot guarantee correct
functionality of this package or compatibility with any version of the
Typst compiler or app.


\title{typst.app/universe/package/chem-par}

\phantomsection\label{banner}
\section{chem-par}\label{chem-par}

{ 0.0.1 }

Display chemical formulae and IUPAC nomenclature with ease

\phantomsection\label{readme}
A utility package for displaying IUPAC nomenclature and chemical
formulae without the hassle of manually formatting all of these in your
document.

\subsection{Example Usage}\label{example-usage}

\begin{Shaded}
\begin{Highlighting}[]
\NormalTok{\#import "@preview/chem{-}par:0.0.1": *}

\NormalTok{\#set page(width: 30em, height: auto, margin: 1em)}
\NormalTok{\#show: chem{-}style}

\NormalTok{The oxidation of n{-}butanol with K2Cr2O7 requires acidification with H2SO4 to yield butanoic acid. N,N{-}dimethyltryptamine.}
\end{Highlighting}
\end{Shaded}

\pandocbounded{\includegraphics[keepaspectratio]{https://github.com/typst/packages/raw/main/packages/preview/chem-par/0.0.1/gallery/example.typ.png}}

Works on most of the common things a chemist would type

\pandocbounded{\includegraphics[keepaspectratio]{https://github.com/typst/packages/raw/main/packages/preview/chem-par/0.0.1/gallery/test.typ.png}}

\subsubsection{How to add}\label{how-to-add}

Copy this into your project and use the import as \texttt{\ chem-par\ }

\begin{verbatim}
#import "@preview/chem-par:0.0.1"
\end{verbatim}

\includesvg[width=0.16667in,height=0.16667in]{/assets/icons/16-copy.svg}

Check the docs for
\href{https://typst.app/docs/reference/scripting/\#packages}{more
information on how to import packages} .

\subsubsection{About}\label{about}

\begin{description}
\tightlist
\item[Author :]
James (Fuzzy) Swift
\item[License:]
MIT
\item[Current version:]
0.0.1
\item[Last updated:]
October 30, 2023
\item[First released:]
October 30, 2023
\item[Archive size:]
3.64 kB
\href{https://packages.typst.org/preview/chem-par-0.0.1.tar.gz}{\pandocbounded{\includesvg[keepaspectratio]{/assets/icons/16-download.svg}}}
\item[Repository:]
\href{https://github.com/JamesxX/typst-chem-par}{GitHub}
\end{description}

\subsubsection{Where to report issues?}\label{where-to-report-issues}

This package is a project of James (Fuzzy) Swift . Report issues on
\href{https://github.com/JamesxX/typst-chem-par}{their repository} . You
can also try to ask for help with this package on the
\href{https://forum.typst.app}{Forum} .

Please report this package to the Typst team using the
\href{https://typst.app/contact}{contact form} if you believe it is a
safety hazard or infringes upon your rights.

\phantomsection\label{versions}
\subsubsection{Version history}\label{version-history}

\begin{longtable}[]{@{}ll@{}}
\toprule\noalign{}
Version & Release Date \\
\midrule\noalign{}
\endhead
\bottomrule\noalign{}
\endlastfoot
0.0.1 & October 30, 2023 \\
\end{longtable}

Typst GmbH did not create this package and cannot guarantee correct
functionality of this package or compatibility with any version of the
Typst compiler or app.


\title{typst.app/universe/package/chemicoms-paper}

\phantomsection\label{banner}
\phantomsection\label{template-thumbnail}
\pandocbounded{\includegraphics[keepaspectratio]{https://packages.typst.org/preview/thumbnails/chemicoms-paper-0.1.0-small.webp}}

\section{chemicoms-paper}\label{chemicoms-paper}

{ 0.1.0 }

An RSC-style paper template to publish at conferences and journals

\href{/app?template=chemicoms-paper&version=0.1.0}{Create project in
app}

\phantomsection\label{readme}
This is a Typst template for a two-column paper in a style similar to
that of the Royal Society of Chemistry.

\subsection{Usage}\label{usage}

You can use this template in the Typst web app by clicking “Start from
template� on the dashboard and searching for the
\texttt{\ chimicoms-paper\ } .

Alternatively, you can use the CLI to kick this project off using the
command

\begin{verbatim}
typst init @preview/chemicoms-paper
\end{verbatim}

\subsection{Configuration}\label{configuration}

This template exports the \texttt{\ template\ } function with the
following named arguments:

\begin{itemize}
\tightlist
\item
  \texttt{\ title\ } (optional, content)
\item
  \texttt{\ subtitle\ } (optional, content)
\item
  \texttt{\ short-title\ } (optional, string)
\item
  \texttt{\ author(s)\ } (optional, (array or singular) dictionary or
  string)

  \begin{itemize}
  \tightlist
  \item
    \texttt{\ name\ } (required, string, inferred)
  \item
    \texttt{\ url\ } (optional, string)
  \item
    \texttt{\ phone\ } (optional, string)
  \item
    \texttt{\ fax\ } (optional, string)
  \item
    \texttt{\ orcid\ } (optional, string)
  \item
    \texttt{\ note\ } (optional, string)
  \item
    \texttt{\ email\ } (optional, string)
  \item
    \texttt{\ corresponding\ } (optional, boolean, default true if email
    set)
  \item
    \texttt{\ equal-contributor\ } (optional, boolean)
  \item
    \texttt{\ deceased\ } (optional, boolean)
  \item
    \texttt{\ roles\ } (optional, (array or singular) string)
  \item
    \texttt{\ affiliation(s)\ } (optional, (array or singular)
    dictionary or strng)

    \begin{itemize}
    \tightlist
    \item
      either: (string) or (number)
    \end{itemize}
  \end{itemize}
\item
  \texttt{\ abstract(s)\ } (optional, (array or singular) dictionary or
  content)

  \begin{itemize}
  \tightlist
  \item
    \texttt{\ title\ } (default: “Abstract�)
  \item
    \texttt{\ content\ } (required, content, inferred)
  \end{itemize}
\item
  \texttt{\ open-access\ } (optional, boolean)
\item
  \texttt{\ venue\ } (optional, content)
\item
  \texttt{\ doi\ } (optional, string)
\item
  \texttt{\ keywords\ } (optional, array of strings)
\item
  \texttt{\ dates\ } (optional, (array or singular) dictionary or date)

  \begin{itemize}
  \tightlist
  \item
    \texttt{\ type\ } (optional, content)
  \item
    \texttt{\ date\ } (required, date or string, inferred)
  \end{itemize}
\end{itemize}

The functions also accepts a single, positional argument for the body of
the paper.

\subsection{Media}\label{media}

\includegraphics[width=0.45\linewidth,height=\textheight,keepaspectratio]{https://github.com/typst/packages/raw/main/packages/preview/chemicoms-paper/0.1.0/thumbnails/1.png}
\includegraphics[width=0.45\linewidth,height=\textheight,keepaspectratio]{https://github.com/typst/packages/raw/main/packages/preview/chemicoms-paper/0.1.0/thumbnails/2.png}

\href{/app?template=chemicoms-paper&version=0.1.0}{Create project in
app}

\subsubsection{How to use}\label{how-to-use}

Click the button above to create a new project using this template in
the Typst app.

You can also use the Typst CLI to start a new project on your computer
using this command:

\begin{verbatim}
typst init @preview/chemicoms-paper:0.1.0
\end{verbatim}

\includesvg[width=0.16667in,height=0.16667in]{/assets/icons/16-copy.svg}

\subsubsection{About}\label{about}

\begin{description}
\tightlist
\item[Author :]
James R. Swift
\item[License:]
MIT-0
\item[Current version:]
0.1.0
\item[Last updated:]
June 21, 2024
\item[First released:]
June 21, 2024
\item[Archive size:]
70.7 kB
\href{https://packages.typst.org/preview/chemicoms-paper-0.1.0.tar.gz}{\pandocbounded{\includesvg[keepaspectratio]{/assets/icons/16-download.svg}}}
\item[Repository:]
\href{https://github.com/JamesxX/chemicoms-paper}{GitHub}
\item[Categor y :]
\begin{itemize}
\tightlist
\item[]
\item
  \pandocbounded{\includesvg[keepaspectratio]{/assets/icons/16-atom.svg}}
  \href{https://typst.app/universe/search/?category=paper}{Paper}
\end{itemize}
\end{description}

\subsubsection{Where to report issues?}\label{where-to-report-issues}

This template is a project of James R. Swift . Report issues on
\href{https://github.com/JamesxX/chemicoms-paper}{their repository} .
You can also try to ask for help with this template on the
\href{https://forum.typst.app}{Forum} .

Please report this template to the Typst team using the
\href{https://typst.app/contact}{contact form} if you believe it is a
safety hazard or infringes upon your rights.

\phantomsection\label{versions}
\subsubsection{Version history}\label{version-history}

\begin{longtable}[]{@{}ll@{}}
\toprule\noalign{}
Version & Release Date \\
\midrule\noalign{}
\endhead
\bottomrule\noalign{}
\endlastfoot
0.1.0 & June 21, 2024 \\
\end{longtable}

Typst GmbH did not create this template and cannot guarantee correct
functionality of this template or compatibility with any version of the
Typst compiler or app.


\title{typst.app/universe/package/diatypst}

\phantomsection\label{banner}
\phantomsection\label{template-thumbnail}
\pandocbounded{\includegraphics[keepaspectratio]{https://packages.typst.org/preview/thumbnails/diatypst-0.3.0-small.webp}}

\section{diatypst}\label{diatypst}

{ 0.3.0 }

Easy slides with Typst â€`` sensible defaults, easy syntax, well-styled

{ } Featured Template

\href{/app?template=diatypst&version=0.3.0}{Create project in app}

\phantomsection\label{readme}
\emph{easy slides in typst}

Features:

\begin{itemize}
\tightlist
\item
  easy delimiter for slides and sections (just use headings)
\item
  sensible styling
\item
  dot counter in upper right corner (like LaTeX beamer)
\item
  adjustable color-theme
\item
  default show rules for terms, code, lists, … that match color-theme
\end{itemize}

Example Presentation

\begin{longtable}[]{@{}llll@{}}
\toprule\noalign{}
Title Slide & Section & Content & Outline \\
\midrule\noalign{}
\endhead
\bottomrule\noalign{}
\endlastfoot
\pandocbounded{\includegraphics[keepaspectratio]{https://github.com/typst/packages/raw/main/packages/preview/diatypst/0.3.0/screenshots/Example-Title.jpg}}
&
\pandocbounded{\includegraphics[keepaspectratio]{https://github.com/typst/packages/raw/main/packages/preview/diatypst/0.3.0/screenshots/Example-Section.jpg}}
&
\pandocbounded{\includegraphics[keepaspectratio]{https://github.com/typst/packages/raw/main/packages/preview/diatypst/0.3.0/screenshots/Example-Slide.jpg}}
&
\pandocbounded{\includegraphics[keepaspectratio]{https://github.com/typst/packages/raw/main/packages/preview/diatypst/0.3.0/screenshots/Example-TOC.jpg}} \\
\end{longtable}

These example slides and a usage guide are available in the
\texttt{\ example\ } Folder on GitHub as a
\href{https://github.com/skriptum/diatypst/blob/main/example/example.typ}{.typ
file} and a
\href{https://github.com/skriptum/diatypst/blob/main/example/example.pdf}{compiled
PDF}

\subsection{Usage}\label{usage}

To start a presentation, initialize it in your typst document:

\begin{Shaded}
\begin{Highlighting}[]
\NormalTok{\#import "@preview/diatypst:0.2.0": *}
\NormalTok{\#show: slides.with(}
\NormalTok{  title: "Diatypst", // Required}
\NormalTok{  subtitle: "easy slides in typst",}
\NormalTok{  date: "01.07.2024",}
\NormalTok{  authors: ("John Doe"),}
\NormalTok{)}
\NormalTok{...}
\end{Highlighting}
\end{Shaded}

Then, insert your content.

\begin{itemize}
\tightlist
\item
  Level-one headings corresponds to new sections.
\item
  Level-two headings corresponds to new slides.
\item
  or manually create a new slide with a \texttt{\ \#pagebreak()\ }
\end{itemize}

\begin{Shaded}
\begin{Highlighting}[]
\NormalTok{...}

\NormalTok{= First Section}

\NormalTok{== First Slide}

\NormalTok{\#lorem(20)}
\end{Highlighting}
\end{Shaded}

\emph{diatypst} is also available on the
\href{https://typst.app/universe/package/diatypst}{Typst Universe} for
easy importing into a project on typst.app

\subsection{Options}\label{options}

all available Options to initialize the template with

\begin{longtable}[]{@{}lll@{}}
\toprule\noalign{}
Keyword & Description & Default \\
\midrule\noalign{}
\endhead
\bottomrule\noalign{}
\endlastfoot
\emph{title} & Title of your Presentation, visible also in footer &
\texttt{\ none\ } but required! \\
\emph{subtitle} & Subtitle, also visible in footer &
\texttt{\ none\ } \\
\emph{date} & a normal string presenting your date &
\texttt{\ none\ } \\
\emph{authors} & either string or array of strings &
\texttt{\ none\ } \\
\emph{layout} & one of “small�, “medium�, “large�, adjusts
sizing of the elements on the slides & \texttt{\ "medium"\ } \\
\emph{ratio} & aspect ratio of the slides, e.g 16/9 &
\texttt{\ 4/3\ } \\
\emph{title-color} & Color to base the Elements of the Presentation on &
\texttt{\ blue.darken(50\%)\ } \\
\emph{count} & whether to display the dots for pages in upper right
corner & \texttt{\ true\ } \\
\emph{footer} & whether to display the footer at the bottom &
\texttt{\ true\ } \\
\emph{toc} & whether to display the table of contents &
\texttt{\ true\ } \\
\emph{footer-title} & custom text in the footer title (left) & same as
\emph{title} \\
\emph{footer-subtitle} & custom text in the footer subtitle (right) &
same as \emph{subtitle} \\
\end{longtable}

\subsection{Quarto}\label{quarto}

This template is also available as a \href{https://quarto.org/}{Quarto}
extension. To use it, add it to your project with the following command:

\begin{Shaded}
\begin{Highlighting}[]
\ExtensionTok{quarto}\NormalTok{ add skriptum/diatypst/diaquarto}
\end{Highlighting}
\end{Shaded}

Then, create a qmd file with the following YAML frontmatter:

\begin{Shaded}
\begin{Highlighting}[]
\FunctionTok{title}\KeywordTok{:}\AttributeTok{ }\StringTok{"Untitled"}
\CommentTok{...}
\CommentTok{format:}
\CommentTok{  diaquarto{-}typst: }
\CommentTok{    layout: medium \# small, medium, large}
\CommentTok{    ratio: 16/9 \# any ratio possible }
\CommentTok{    title{-}color: "013220" \# Any Hex code for the title color (without \#)}
\end{Highlighting}
\end{Shaded}

\subsection{Inspiration}\label{inspiration}

this template is inspired by
\href{https://github.com/glambrechts/slydst}{slydst} , and takes part of
the code from it. If you want simpler slides, look here!

The word \emph{Diatypst} is inspired by the ease of use of a
\href{https://de.wikipedia.org/wiki/Diaprojektor}{\textbf{Dia}
-projektor} (German for Slide Projector) and the
\href{https://en.wikipedia.org/wiki/Diatype_(machine)}{Diatype}

\href{/app?template=diatypst&version=0.3.0}{Create project in app}

\subsubsection{How to use}\label{how-to-use}

Click the button above to create a new project using this template in
the Typst app.

You can also use the Typst CLI to start a new project on your computer
using this command:

\begin{verbatim}
typst init @preview/diatypst:0.3.0
\end{verbatim}

\includesvg[width=0.16667in,height=0.16667in]{/assets/icons/16-copy.svg}

\subsubsection{About}\label{about}

\begin{description}
\tightlist
\item[Author :]
skriptum (https://github.com/skriptum)
\item[License:]
MIT-0
\item[Current version:]
0.3.0
\item[Last updated:]
November 18, 2024
\item[First released:]
July 22, 2024
\item[Minimum Typst version:]
0.12.0
\item[Archive size:]
4.89 kB
\href{https://packages.typst.org/preview/diatypst-0.3.0.tar.gz}{\pandocbounded{\includesvg[keepaspectratio]{/assets/icons/16-download.svg}}}
\item[Repository:]
\href{https://github.com/skriptum/Diatypst}{GitHub}
\item[Categor y :]
\begin{itemize}
\tightlist
\item[]
\item
  \pandocbounded{\includesvg[keepaspectratio]{/assets/icons/16-presentation.svg}}
  \href{https://typst.app/universe/search/?category=presentation}{Presentation}
\end{itemize}
\end{description}

\subsubsection{Where to report issues?}\label{where-to-report-issues}

This template is a project of skriptum (https://github.com/skriptum) .
Report issues on \href{https://github.com/skriptum/Diatypst}{their
repository} . You can also try to ask for help with this template on the
\href{https://forum.typst.app}{Forum} .

Please report this template to the Typst team using the
\href{https://typst.app/contact}{contact form} if you believe it is a
safety hazard or infringes upon your rights.

\phantomsection\label{versions}
\subsubsection{Version history}\label{version-history}

\begin{longtable}[]{@{}ll@{}}
\toprule\noalign{}
Version & Release Date \\
\midrule\noalign{}
\endhead
\bottomrule\noalign{}
\endlastfoot
0.3.0 & November 18, 2024 \\
\href{https://typst.app/universe/package/diatypst/0.2.0/}{0.2.0} &
November 6, 2024 \\
\href{https://typst.app/universe/package/diatypst/0.1.0/}{0.1.0} & July
22, 2024 \\
\end{longtable}

Typst GmbH did not create this template and cannot guarantee correct
functionality of this template or compatibility with any version of the
Typst compiler or app.


\title{typst.app/universe/package/tiaoma}

\phantomsection\label{banner}
\section{tiaoma}\label{tiaoma}

{ 0.2.1 }

Barcode and QRCode generator for Typst using Zint.

{ } Featured Package

\phantomsection\label{readme}
\href{https://github.com/enter-tainer/zint-wasi}{tiaoma(æ?¡ç~?)} is a
barcode generator for typst. It compiles
\href{https://github.com/zint/zint}{zint} to wasm and use it to generate
barcode. It support nearly all common barcode types. For a complete list
of supported barcode types, see \href{https://zint.org.uk/}{zint’s
documentation} :

\begin{itemize}
\tightlist
\item
  Australia Post

  \begin{itemize}
  \tightlist
  \item
    Standard Customer
  \item
    Reply Paid
  \item
    Routing
  \item
    Redirection
  \end{itemize}
\item
  Aztec Code
\item
  Aztec Runes
\item
  Channel Code
\item
  Codabar
\item
  Codablock F
\item
  Code 11
\item
  Code 128 with automatic subset switching
\item
  Code 16k
\item
  Code 2 of 5 variants:

  \begin{itemize}
  \tightlist
  \item
    Matrix 2 of 5
  \item
    Industrial 2 of 5
  \item
    IATA 2 of 5
  \item
    Datalogic 2 of 5
  \item
    Interleaved 2 of 5
  \item
    ITF-14
  \end{itemize}
\item
  Deutsche Post Leitcode
\item
  Deutsche Post Identcode
\item
  Code 32 (Italian pharmacode)
\item
  Code 3 of 9 (Code 39)
\item
  Code 3 of 9 Extended (Code 39 Extended)
\item
  Code 49
\item
  Code 93
\item
  Code One
\item
  Data Matrix ECC200
\item
  DotCode
\item
  Dutch Post KIX Code
\item
  EAN variants:

  \begin{itemize}
  \tightlist
  \item
    EAN-13
  \item
    EAN-8
  \end{itemize}
\item
  Grid Matrix
\item
  GS1 DataBar variants:

  \begin{itemize}
  \tightlist
  \item
    GS1 DataBar
  \item
    GS1 DataBar Stacked
  \item
    GS1 DataBar Stacked Omnidirectional
  \item
    GS1 DataBar Expanded
  \item
    GS1 DataBar Expanded Stacked
  \item
    GS1 DataBar Limited
  \end{itemize}
\item
  Han Xin
\item
  Japan Post
\item
  Korea Post
\item
  LOGMARS
\item
  MaxiCode
\item
  MSI (Modified Plessey)
\item
  PDF417 variants:

  \begin{itemize}
  \tightlist
  \item
    PDF417 Truncated
  \item
    PDF417
  \item
    Micro PDF417
  \end{itemize}
\item
  Pharmacode
\item
  Pharmacode Two-Track
\item
  Pharmazentralnummer
\item
  POSTNET / PLANET
\item
  QR Code
\item
  rMQR
\item
  Royal Mail 4-State (RM4SCC)
\item
  Royal Mail 4-State Mailmark
\item
  Telepen
\item
  UPC variants:

  \begin{itemize}
  \tightlist
  \item
    UPC-A
  \item
    UPC-E
  \end{itemize}
\item
  UPNQR
\item
  USPS OneCode (Intelligent Mail)
\end{itemize}

\subsection{Example}\label{example}

\begin{Shaded}
\begin{Highlighting}[]
\NormalTok{\#import "@preview/tiaoma:0.2.1"}
\NormalTok{\#set page(width: auto, height: auto)}

\NormalTok{= tiáo mǎ}

\NormalTok{\#tiaoma.ean("1234567890128")}
\end{Highlighting}
\end{Shaded}

\pandocbounded{\includesvg[keepaspectratio]{https://github.com/typst/packages/raw/main/packages/preview/tiaoma/0.2.1/example.svg}}

\subsection{Manual}\label{manual}

Please refer to
\href{https://github.com/typst/packages/raw/main/packages/preview/tiaoma/0.2.1/manual.pdf}{manual}
for more details.

\subsection{Alternatives}\label{alternatives}

There are other barcode/qrcode packages for typst such as:

\begin{itemize}
\tightlist
\item
  \url{https://github.com/jneug/typst-codetastic}
\item
  \url{https://github.com/Midbin/cades}
\end{itemize}

Packages differ in provided customization options for generated
barcodes. This package is limited by zint functionality, which focuses
more on coverage than customization (e.g. inserting graphics into QR
codes). Patching upstream zint code is (currently) outside of the scope
of this package - if it doesn’t provide functionality you need, check
the rest of the typst ecosystem to see if it’s available elsewhere or
request it \href{https://github.com/zint/zint}{upstream} and
\href{https://github.com/Enter-tainer/zint-wasi/issues}{notify us} when
it’s been merged.

\subsubsection{Pros}\label{pros}

\begin{enumerate}
\tightlist
\item
  Support for far greater number of barcode types (all provided by zint
  library)
\item
  Should be faster as is uses a WASM plugin which bundles zint code
  which is written in C; others are written in pure typst or javascript.
\end{enumerate}

\subsubsection{Cons}\label{cons}

\begin{enumerate}
\tightlist
\item
  While most if not all of zint functionality is covered, it’s hard to
  guarantee there’s no overlooked functionality.
\item
  This package uses typst plugin system and has a WASM backend written
  in Rust which makes is less welcoming for new contributors.
\end{enumerate}

\subsubsection{How to add}\label{how-to-add}

Copy this into your project and use the import as \texttt{\ tiaoma\ }

\begin{verbatim}
#import "@preview/tiaoma:0.2.1"
\end{verbatim}

\includesvg[width=0.16667in,height=0.16667in]{/assets/icons/16-copy.svg}

Check the docs for
\href{https://typst.app/docs/reference/scripting/\#packages}{more
information on how to import packages} .

\subsubsection{About}\label{about}

\begin{description}
\tightlist
\item[Author s :]
Wenzhuo Liu \& Tin Å~vagelj
\item[License:]
MIT
\item[Current version:]
0.2.1
\item[Last updated:]
September 8, 2024
\item[First released:]
November 30, 2023
\item[Archive size:]
457 kB
\href{https://packages.typst.org/preview/tiaoma-0.2.1.tar.gz}{\pandocbounded{\includesvg[keepaspectratio]{/assets/icons/16-download.svg}}}
\item[Repository:]
\href{https://github.com/Enter-tainer/zint-wasi}{GitHub}
\end{description}

\subsubsection{Where to report issues?}\label{where-to-report-issues}

This package is a project of Wenzhuo Liu and Tin Å~vagelj . Report
issues on \href{https://github.com/Enter-tainer/zint-wasi}{their
repository} . You can also try to ask for help with this package on the
\href{https://forum.typst.app}{Forum} .

Please report this package to the Typst team using the
\href{https://typst.app/contact}{contact form} if you believe it is a
safety hazard or infringes upon your rights.

\phantomsection\label{versions}
\subsubsection{Version history}\label{version-history}

\begin{longtable}[]{@{}ll@{}}
\toprule\noalign{}
Version & Release Date \\
\midrule\noalign{}
\endhead
\bottomrule\noalign{}
\endlastfoot
0.2.1 & September 8, 2024 \\
\href{https://typst.app/universe/package/tiaoma/0.2.0/}{0.2.0} &
February 6, 2024 \\
\href{https://typst.app/universe/package/tiaoma/0.1.0/}{0.1.0} &
November 30, 2023 \\
\end{longtable}

Typst GmbH did not create this package and cannot guarantee correct
functionality of this package or compatibility with any version of the
Typst compiler or app.


\title{typst.app/universe/package/minitoc}

\phantomsection\label{banner}
\section{minitoc}\label{minitoc}

{ 0.1.0 }

An outline function just for one section and nothing else

\phantomsection\label{readme}
This package provides the \texttt{\ minitoc\ } command that does the
same thing as the \texttt{\ outline\ } command but only for headings
under the heading above it.

This is inspired by minitoc package for LaTeX.

\subsection{Example}\label{example}

\begin{Shaded}
\begin{Highlighting}[]
\NormalTok{\#import "@preview/minitoc:0.1.0": *}
\NormalTok{\#set heading(numbering: "1.1")}

\NormalTok{\#outline()}

\NormalTok{= Heading 1}

\NormalTok{\#minitoc()}

\NormalTok{== Heading 1.1}

\NormalTok{\#lorem(20)}

\NormalTok{=== Heading 1.1.1}

\NormalTok{\#lorem(30)}

\NormalTok{== Heading 1.2}

\NormalTok{\#lorem(10)}

\NormalTok{= Heading 2}
\end{Highlighting}
\end{Shaded}

This produces

\pandocbounded{\includegraphics[keepaspectratio]{https://gitlab.com/human_person/typst-local-outline/-/raw/main/example/example.png}}

\subsection{Usage}\label{usage}

The \texttt{\ minitoc\ } function has the following signature:

\begin{Shaded}
\begin{Highlighting}[]
\NormalTok{\#let minitoc(}
\NormalTok{  title: none, target: heading.where(outlined: true),}
\NormalTok{    depth: none, indent: none, fill: repeat([.])}
\NormalTok{) \{ /* .. */ \}}
\end{Highlighting}
\end{Shaded}

This is designed to be as close to the
\href{https://typst.app/docs/reference/meta/outline/}{\texttt{\ outline\ }}
funtions as possible. The arguments are:

\begin{itemize}
\tightlist
\item
  \textbf{title} : The title for the local outline. This is the same as
  for
  \href{https://typst.app/docs/reference/meta/outline/\#parameters-title}{\texttt{\ outline.title\ }}
  .
\item
  \textbf{target} : What should be included. This is the same as for
  \href{https://typst.app/docs/reference/meta/outline/\#parameters-target}{\texttt{\ outline.target\ }}
\item
  \textbf{depth} : The maximum depth different to include. For example,
  if depth was 1 in the example, “Heading 1.1.1� would not be
  included
\item
  \textbf{indent} : How the entries should be indented. Takes the same
  types as for
  \href{https://typst.app/docs/reference/meta/outline/\#parameters-indent}{\texttt{\ outline.indent\ }}
  and is passed directly to it
\item
  \textbf{fill} : Content to put between the numbering and title, and
  the page number. Same types as for
  \href{https://typst.app/docs/reference/meta/outline/\#parameters-fill}{\texttt{\ outline.fill\ }}
\end{itemize}

\subsection{Unintended consequences}\label{unintended-consequences}

Because \texttt{\ minitoc\ } uses \texttt{\ outline\ } , if you apply
numbering to the title of outline with
\texttt{\ \#show\ outline:\ set\ heading(numbering:\ "1.")\ } or
similar, any title in \texttt{\ minitoc\ } will be numbered and be a
level 1 heading. This cannot be changed with
\texttt{\ \#show\ outline:\ set\ heading(level:\ 3)\ } or similar
unfortunately.

\subsubsection{How to add}\label{how-to-add}

Copy this into your project and use the import as \texttt{\ minitoc\ }

\begin{verbatim}
#import "@preview/minitoc:0.1.0"
\end{verbatim}

\includesvg[width=0.16667in,height=0.16667in]{/assets/icons/16-copy.svg}

Check the docs for
\href{https://typst.app/docs/reference/scripting/\#packages}{more
information on how to import packages} .

\subsubsection{About}\label{about}

\begin{description}
\tightlist
\item[Author :]
\href{https://github.com/RosiePuddles}{nxe}
\item[License:]
GPL-3.0-only
\item[Current version:]
0.1.0
\item[Last updated:]
January 7, 2024
\item[First released:]
January 7, 2024
\item[Archive size:]
13.6 kB
\href{https://packages.typst.org/preview/minitoc-0.1.0.tar.gz}{\pandocbounded{\includesvg[keepaspectratio]{/assets/icons/16-download.svg}}}
\item[Repository:]
\href{https://gitlab.com/human_person/typst-local-outline}{GitLab}
\end{description}

\subsubsection{Where to report issues?}\label{where-to-report-issues}

This package is a project of nxe . Report issues on
\href{https://gitlab.com/human_person/typst-local-outline}{their
repository} . You can also try to ask for help with this package on the
\href{https://forum.typst.app}{Forum} .

Please report this package to the Typst team using the
\href{https://typst.app/contact}{contact form} if you believe it is a
safety hazard or infringes upon your rights.

\phantomsection\label{versions}
\subsubsection{Version history}\label{version-history}

\begin{longtable}[]{@{}ll@{}}
\toprule\noalign{}
Version & Release Date \\
\midrule\noalign{}
\endhead
\bottomrule\noalign{}
\endlastfoot
0.1.0 & January 7, 2024 \\
\end{longtable}

Typst GmbH did not create this package and cannot guarantee correct
functionality of this package or compatibility with any version of the
Typst compiler or app.


\title{typst.app/universe/package/chuli-cv}

\phantomsection\label{banner}
\phantomsection\label{template-thumbnail}
\pandocbounded{\includegraphics[keepaspectratio]{https://packages.typst.org/preview/thumbnails/chuli-cv-0.1.0-small.webp}}

\section{chuli-cv}\label{chuli-cv}

{ 0.1.0 }

Minimalistic and modern CV and cover letter templates.

\href{/app?template=chuli-cv&version=0.1.0}{Create project in app}

\phantomsection\label{readme}
These are a minimalistic and modern CV and cover letter written in
Typst.

\pandocbounded{\includegraphics[keepaspectratio]{https://github.com/typst/packages/raw/main/packages/preview/chuli-cv/0.1.0/thumbnail.png}}

\subsection{Setup}\label{setup}

\begin{itemize}
\tightlist
\item
  Install \href{https://typst.app/}{Typst} and the font awesome fonts on
  your system, see
  \href{https://github.com/duskmoon314/typst-fontawesome}{guide} .
\item
  Run \texttt{\ typst\ init\ @preview/chuli-cv:0.1.0\ } to start your
  own CV.
\end{itemize}

\subsection{Inspiration}\label{inspiration}

\begin{itemize}
\tightlist
\item
  \href{https://github.com/mintyfrankie/brilliant-CV}{brilliant-CV} .
\item
  \href{https://github.com/AnsgarLichter/light-cv}{light-cv} .
\item
  \href{https://github.com/posquit0/Awesome-CV}{Awesome CV} .
\item
  \href{https://www.overleaf.com/articles/ritabh-ranjans-cv/ngtndgryfykt}{Ritabh
  Ranjan’s CV} .
\item
  \href{https://github.com/latex-ninja/hipster-cv}{hipster-cv} .
\end{itemize}

\href{/app?template=chuli-cv&version=0.1.0}{Create project in app}

\subsubsection{How to use}\label{how-to-use}

Click the button above to create a new project using this template in
the Typst app.

You can also use the Typst CLI to start a new project on your computer
using this command:

\begin{verbatim}
typst init @preview/chuli-cv:0.1.0
\end{verbatim}

\includesvg[width=0.16667in,height=0.16667in]{/assets/icons/16-copy.svg}

\subsubsection{About}\label{about}

\begin{description}
\tightlist
\item[Author :]
Naivy Pujol Méndez
\item[License:]
MIT
\item[Current version:]
0.1.0
\item[Last updated:]
May 14, 2024
\item[First released:]
May 14, 2024
\item[Archive size:]
831 kB
\href{https://packages.typst.org/preview/chuli-cv-0.1.0.tar.gz}{\pandocbounded{\includesvg[keepaspectratio]{/assets/icons/16-download.svg}}}
\item[Repository:]
\href{https://github.com/npujol/chuli-cv}{GitHub}
\item[Categor y :]
\begin{itemize}
\tightlist
\item[]
\item
  \pandocbounded{\includesvg[keepaspectratio]{/assets/icons/16-user.svg}}
  \href{https://typst.app/universe/search/?category=cv}{CV}
\end{itemize}
\end{description}

\subsubsection{Where to report issues?}\label{where-to-report-issues}

This template is a project of Naivy Pujol Méndez . Report issues on
\href{https://github.com/npujol/chuli-cv}{their repository} . You can
also try to ask for help with this template on the
\href{https://forum.typst.app}{Forum} .

Please report this template to the Typst team using the
\href{https://typst.app/contact}{contact form} if you believe it is a
safety hazard or infringes upon your rights.

\phantomsection\label{versions}
\subsubsection{Version history}\label{version-history}

\begin{longtable}[]{@{}ll@{}}
\toprule\noalign{}
Version & Release Date \\
\midrule\noalign{}
\endhead
\bottomrule\noalign{}
\endlastfoot
0.1.0 & May 14, 2024 \\
\end{longtable}

Typst GmbH did not create this template and cannot guarantee correct
functionality of this template or compatibility with any version of the
Typst compiler or app.


\title{typst.app/universe/package/report-flow-ustc}

\phantomsection\label{banner}
\phantomsection\label{template-thumbnail}
\pandocbounded{\includegraphics[keepaspectratio]{https://packages.typst.org/preview/thumbnails/report-flow-ustc-1.0.0-small.webp}}

\section{report-flow-ustc}\label{report-flow-ustc}

{ 1.0.0 }

A template suitable for USTC students (of course, you can freely modify
it for any school or organization) to complete course assignments or
submit lab reports.

\href{/app?template=report-flow-ustc&version=1.0.0}{Create project in
app}

\phantomsection\label{readme}
A template suitable for USTC students (of course, you can freely modify
it for any school or organization) to complete course assignments or
submit lab reports.

\href{/app?template=report-flow-ustc&version=1.0.0}{Create project in
app}

\subsubsection{How to use}\label{how-to-use}

Click the button above to create a new project using this template in
the Typst app.

You can also use the Typst CLI to start a new project on your computer
using this command:

\begin{verbatim}
typst init @preview/report-flow-ustc:1.0.0
\end{verbatim}

\includesvg[width=0.16667in,height=0.16667in]{/assets/icons/16-copy.svg}

\subsubsection{About}\label{about}

\begin{description}
\tightlist
\item[Author :]
\href{https://github.com/Quaternijkon}{Quaternijkon}
\item[License:]
MIT
\item[Current version:]
1.0.0
\item[Last updated:]
November 26, 2024
\item[First released:]
November 26, 2024
\item[Minimum Typst version:]
0.11.0
\item[Archive size:]
158 kB
\href{https://packages.typst.org/preview/report-flow-ustc-1.0.0.tar.gz}{\pandocbounded{\includesvg[keepaspectratio]{/assets/icons/16-download.svg}}}
\item[Repository:]
\href{https://github.com/Quaternijkon/Typst_Lab_Report}{GitHub}
\item[Categor y :]
\begin{itemize}
\tightlist
\item[]
\item
  \pandocbounded{\includesvg[keepaspectratio]{/assets/icons/16-speak.svg}}
  \href{https://typst.app/universe/search/?category=report}{Report}
\end{itemize}
\end{description}

\subsubsection{Where to report issues?}\label{where-to-report-issues}

This template is a project of Quaternijkon . Report issues on
\href{https://github.com/Quaternijkon/Typst_Lab_Report}{their
repository} . You can also try to ask for help with this template on the
\href{https://forum.typst.app}{Forum} .

Please report this template to the Typst team using the
\href{https://typst.app/contact}{contact form} if you believe it is a
safety hazard or infringes upon your rights.

\phantomsection\label{versions}
\subsubsection{Version history}\label{version-history}

\begin{longtable}[]{@{}ll@{}}
\toprule\noalign{}
Version & Release Date \\
\midrule\noalign{}
\endhead
\bottomrule\noalign{}
\endlastfoot
1.0.0 & November 26, 2024 \\
\end{longtable}

Typst GmbH did not create this template and cannot guarantee correct
functionality of this template or compatibility with any version of the
Typst compiler or app.


\title{typst.app/universe/package/peace-of-posters}

\phantomsection\label{banner}
\phantomsection\label{template-thumbnail}
\pandocbounded{\includegraphics[keepaspectratio]{https://packages.typst.org/preview/thumbnails/peace-of-posters-0.5.0-small.webp}}

\section{peace-of-posters}\label{peace-of-posters}

{ 0.5.0 }

Create scientific posters in Typst.

\href{/app?template=peace-of-posters&version=0.5.0}{Create project in
app}

\phantomsection\label{readme}
\pandocbounded{\includegraphics[keepaspectratio]{https://img.shields.io/github/actions/workflow/status/jonaspleyer/peace-of-posters/test.yml?style=flat-square&label=Test}}
\pandocbounded{\includegraphics[keepaspectratio]{https://img.shields.io/github/actions/workflow/status/jonaspleyer/peace-of-posters/docs.yml?style=flat-square&label=Docs}}

\begin{quote}
piece of cake\\
peace of mind\\
peace of posters
\end{quote}

\href{https://github.com/jonaspleyer/peace-of-posters}{peace-of-posters
(PoP)} is a Typst package to help creating scientific posters. It is
flexible and can be used for different sizes and layouts. To see what is
possible have a look at some of my own real-world examples in the
\href{https://jonaspleyer.github.io/peace-of-posters/showcase/}{showcase}
section of the documentation.

\subsection{Documentation}\label{documentation}

The external
\href{https://jonaspleyer.github.io/peace-of-posters/}{documentation} is
coming along slowly. Most notably, there are examples and showcases
missing but I hope to be adding them over the coming months.

\subsection{License}\label{license}

Download the \href{https://www.mit.edu/~amini/LICENSE.md}{MIT License}

\href{/app?template=peace-of-posters&version=0.5.0}{Create project in
app}

\subsubsection{How to use}\label{how-to-use}

Click the button above to create a new project using this template in
the Typst app.

You can also use the Typst CLI to start a new project on your computer
using this command:

\begin{verbatim}
typst init @preview/peace-of-posters:0.5.0
\end{verbatim}

\includesvg[width=0.16667in,height=0.16667in]{/assets/icons/16-copy.svg}

\subsubsection{About}\label{about}

\begin{description}
\tightlist
\item[Author :]
\href{mailto:jonas.sci@pleyer.org}{Jonas Pleyer}
\item[License:]
MIT
\item[Current version:]
0.5.0
\item[Last updated:]
October 25, 2024
\item[First released:]
May 31, 2024
\item[Archive size:]
213 kB
\href{https://packages.typst.org/preview/peace-of-posters-0.5.0.tar.gz}{\pandocbounded{\includesvg[keepaspectratio]{/assets/icons/16-download.svg}}}
\item[Repository:]
\href{https://github.com/jonaspleyer/peace-of-posters}{GitHub}
\item[Categor y :]
\begin{itemize}
\tightlist
\item[]
\item
  \pandocbounded{\includesvg[keepaspectratio]{/assets/icons/16-pin.svg}}
  \href{https://typst.app/universe/search/?category=poster}{Poster}
\end{itemize}
\end{description}

\subsubsection{Where to report issues?}\label{where-to-report-issues}

This template is a project of Jonas Pleyer . Report issues on
\href{https://github.com/jonaspleyer/peace-of-posters}{their repository}
. You can also try to ask for help with this template on the
\href{https://forum.typst.app}{Forum} .

Please report this template to the Typst team using the
\href{https://typst.app/contact}{contact form} if you believe it is a
safety hazard or infringes upon your rights.

\phantomsection\label{versions}
\subsubsection{Version history}\label{version-history}

\begin{longtable}[]{@{}ll@{}}
\toprule\noalign{}
Version & Release Date \\
\midrule\noalign{}
\endhead
\bottomrule\noalign{}
\endlastfoot
0.5.0 & October 25, 2024 \\
\href{https://typst.app/universe/package/peace-of-posters/0.4.3/}{0.4.3}
& October 22, 2024 \\
\href{https://typst.app/universe/package/peace-of-posters/0.4.1/}{0.4.1}
& June 3, 2024 \\
\href{https://typst.app/universe/package/peace-of-posters/0.4.0/}{0.4.0}
& May 31, 2024 \\
\end{longtable}

Typst GmbH did not create this template and cannot guarantee correct
functionality of this template or compatibility with any version of the
Typst compiler or app.


\title{typst.app/universe/package/simplebnf}

\phantomsection\label{banner}
\section{simplebnf}\label{simplebnf}

{ 0.1.1 }

A simple package to format Backus-Naur form (BNF)

\phantomsection\label{readme}
simplebnf is a simple package to format Backus-Naur form. The package
provides a simple way to format Backus-Naur form (BNF). It provides
constructs to denote BNF expressions, possibly with annotations.

This is a sister package of
\href{https://github.com/Zeta611/simplebnf}{simplebnf} , a LaTeX package
under the same name by the author.

\subsection{Usage}\label{usage}

Import simplebnf via

\begin{Shaded}
\begin{Highlighting}[]
\NormalTok{\#import "@preview/simplebnf:0.1.1": *}
\end{Highlighting}
\end{Shaded}

Use the \texttt{\ bnf\ } function to display the BNF production rules.
Each production rule can be created using the \texttt{\ Prod\ }
constructor function, which accepts the (left-hand side) metavariable,
an optional annotation for it, an optional delimiter (which defaults to
â©´), and a list of (right-hand side) alternatives. Each alternative
should be created using the \texttt{\ Or\ } constructor, which accepts a
syntactic form and an annotation.

Below are some examples using simplebnf.

\begin{Shaded}
\begin{Highlighting}[]
\NormalTok{\#bnf(}
\NormalTok{  Prod(}
\NormalTok{    $e$,}
\NormalTok{    annot: $sans("Expr")$,}
\NormalTok{    \{}
\NormalTok{      Or[$x$][\_variable\_]}
\NormalTok{      Or[$λ x. e$][\_abstraction\_]}
\NormalTok{      Or[$e$ $e$][\_application\_]}
\NormalTok{    \},}
\NormalTok{  ),}
\NormalTok{)}
\end{Highlighting}
\end{Shaded}

\pandocbounded{\includesvg[keepaspectratio]{https://github.com/typst/packages/raw/main/packages/preview/simplebnf/0.1.1/examples/lambda.svg}}

\begin{Shaded}
\begin{Highlighting}[]
\NormalTok{\#bnf(}
\NormalTok{  Prod(}
\NormalTok{    $e$,}
\NormalTok{    delim: $→$,}
\NormalTok{    \{}
\NormalTok{      Or[$x$][variable]}
\NormalTok{      Or[$λ x: τ.e$][abstraction]}
\NormalTok{      Or[$e space e$][application]}
\NormalTok{      Or[$λ τ.e space e$][type abstraction]}
\NormalTok{      Or[$e space [τ]$][type application]}
\NormalTok{    \},}
\NormalTok{  ),}
\NormalTok{  Prod(}
\NormalTok{    $τ$,}
\NormalTok{    delim: $→$,}
\NormalTok{    \{}
\NormalTok{      Or[$X$][type variable]}
\NormalTok{      Or[$τ → τ$][type of functions]}
\NormalTok{      Or[$∀X.τ$][universal quantification]}
\NormalTok{    \},}
\NormalTok{  ),}
\NormalTok{)}
\end{Highlighting}
\end{Shaded}

\pandocbounded{\includesvg[keepaspectratio]{https://github.com/typst/packages/raw/main/packages/preview/simplebnf/0.1.1/examples/system-f.svg}}

\subsection{Authors}\label{authors}

\begin{itemize}
\tightlist
\item
  Jay Lee
  \href{mailto:jaeho.lee@snu.ac.kr}{\nolinkurl{jaeho.lee@snu.ac.kr}}
\end{itemize}

\subsection{License}\label{license}

simplebnf.typ is available under the MIT license. See the
\href{https://github.com/Zeta611/simplebnf.typ/blob/master/LICENSE}{LICENSE}
file for more info.

\subsubsection{How to add}\label{how-to-add}

Copy this into your project and use the import as \texttt{\ simplebnf\ }

\begin{verbatim}
#import "@preview/simplebnf:0.1.1"
\end{verbatim}

\includesvg[width=0.16667in,height=0.16667in]{/assets/icons/16-copy.svg}

Check the docs for
\href{https://typst.app/docs/reference/scripting/\#packages}{more
information on how to import packages} .

\subsubsection{About}\label{about}

\begin{description}
\tightlist
\item[Author :]
\href{https://github.com/Zeta611}{Jay Lee}
\item[License:]
MIT
\item[Current version:]
0.1.1
\item[Last updated:]
July 15, 2024
\item[First released:]
May 23, 2024
\item[Archive size:]
2.10 kB
\href{https://packages.typst.org/preview/simplebnf-0.1.1.tar.gz}{\pandocbounded{\includesvg[keepaspectratio]{/assets/icons/16-download.svg}}}
\item[Repository:]
\href{https://github.com/Zeta611/simplebnf.typ}{GitHub}
\item[Discipline :]
\begin{itemize}
\tightlist
\item[]
\item
  \href{https://typst.app/universe/search/?discipline=computer-science}{Computer
  Science}
\end{itemize}
\item[Categor ies :]
\begin{itemize}
\tightlist
\item[]
\item
  \pandocbounded{\includesvg[keepaspectratio]{/assets/icons/16-package.svg}}
  \href{https://typst.app/universe/search/?category=components}{Components}
\item
  \pandocbounded{\includesvg[keepaspectratio]{/assets/icons/16-chart.svg}}
  \href{https://typst.app/universe/search/?category=visualization}{Visualization}
\item
  \pandocbounded{\includesvg[keepaspectratio]{/assets/icons/16-integration.svg}}
  \href{https://typst.app/universe/search/?category=integration}{Integration}
\end{itemize}
\end{description}

\subsubsection{Where to report issues?}\label{where-to-report-issues}

This package is a project of Jay Lee . Report issues on
\href{https://github.com/Zeta611/simplebnf.typ}{their repository} . You
can also try to ask for help with this package on the
\href{https://forum.typst.app}{Forum} .

Please report this package to the Typst team using the
\href{https://typst.app/contact}{contact form} if you believe it is a
safety hazard or infringes upon your rights.

\phantomsection\label{versions}
\subsubsection{Version history}\label{version-history}

\begin{longtable}[]{@{}ll@{}}
\toprule\noalign{}
Version & Release Date \\
\midrule\noalign{}
\endhead
\bottomrule\noalign{}
\endlastfoot
0.1.1 & July 15, 2024 \\
\href{https://typst.app/universe/package/simplebnf/0.1.0/}{0.1.0} & May
23, 2024 \\
\end{longtable}

Typst GmbH did not create this package and cannot guarantee correct
functionality of this package or compatibility with any version of the
Typst compiler or app.


\title{typst.app/universe/package/keyle}

\phantomsection\label{banner}
\section{keyle}\label{keyle}

{ 0.2.0 }

This package provides a simple way to style keyboard shortcuts in your
documentation.

\phantomsection\label{readme}
\href{https://raw.githubusercontent.com/magicwenli/keyle/main/doc/keyle.pdf}{\pandocbounded{\includegraphics[keepaspectratio]{https://img.shields.io/website?down_message=offline&label=manual&up_color=007aff&up_message=online&url=https://raw.githubusercontent.com/magicwenli/keyle/main/doc/keyle.pdf}}}
\href{https://github.com/magicwenli/keyle/blob/main/LICENSE}{\pandocbounded{\includegraphics[keepaspectratio]{https://img.shields.io/badge/license-MIT-brightgreen}}}

A simple way to style keyboard shortcuts in your documentation.

This package was inspired by
\href{https://auth0.github.io/kbd/}{auth0/kbd} and
\href{https://github.com/dogezen/badgery}{dogezen/badgery} . Also thanks
to \href{https://github.com/tweh/menukeys}{tweh/menukeys} â€`` A LaTeX
package for menu keys generation.

Document generating using
\href{https://github.com/jneug/typst-mantys}{jneug/typst-mantys} .

Send them respect and love.

\subsection{Usage}\label{usage}

Please see the
\href{https://github.com/magicwenli/keyle/blob/main/doc/keyle.pdf}{keyle.pdf}
for more documentation.

\texttt{\ keyle\ } is imported using:

\begin{Shaded}
\begin{Highlighting}[]
\NormalTok{\#import "@preview/keyle:0.2.0"}
\end{Highlighting}
\end{Shaded}

\subsubsection{Example}\label{example}

\paragraph{Custom Delimiter}\label{custom-delimiter}

\begin{Shaded}
\begin{Highlighting}[]
\NormalTok{\#let kbd = keyle.config()}
\NormalTok{\#kbd("Ctrl", "Shift", "K", delim: "{-}")}
\end{Highlighting}
\end{Shaded}

\pandocbounded{\includegraphics[keepaspectratio]{https://github.com/typst/packages/raw/main/packages/preview/keyle/0.2.0/test/test-1.png}}

\paragraph{Compact Mode}\label{compact-mode}

\begin{Shaded}
\begin{Highlighting}[]
\NormalTok{\#let kbd = keyle.config()}
\NormalTok{\#kbd("Ctrl", "Shift", "K", compact: true)}
\end{Highlighting}
\end{Shaded}

\pandocbounded{\includegraphics[keepaspectratio]{https://github.com/typst/packages/raw/main/packages/preview/keyle/0.2.0/test/test-2.png}}

\paragraph{Standard Theme}\label{standard-theme}

\begin{Shaded}
\begin{Highlighting}[]
\NormalTok{\#let kbd = keyle.config(theme: keyle.themes.standard)}
\NormalTok{\#keyle.gen{-}examples(kbd)}
\end{Highlighting}
\end{Shaded}

\pandocbounded{\includegraphics[keepaspectratio]{https://github.com/typst/packages/raw/main/packages/preview/keyle/0.2.0/test/test-3.png}}

\paragraph{Deep Blue Theme}\label{deep-blue-theme}

\begin{Shaded}
\begin{Highlighting}[]
\NormalTok{\#let kbd = keyle.config(theme: keyle.themes.deep{-}blue)}
\NormalTok{\#keyle.gen{-}examples(kbd)}
\end{Highlighting}
\end{Shaded}

\pandocbounded{\includegraphics[keepaspectratio]{https://github.com/typst/packages/raw/main/packages/preview/keyle/0.2.0/test/test-4.png}}

\paragraph{Type Writer Theme}\label{type-writer-theme}

\begin{Shaded}
\begin{Highlighting}[]
\NormalTok{\#let kbd = keyle.config(theme: keyle.themes.type{-}writer)}
\NormalTok{\#keyle.gen{-}examples(kbd)}
\end{Highlighting}
\end{Shaded}

\pandocbounded{\includegraphics[keepaspectratio]{https://github.com/typst/packages/raw/main/packages/preview/keyle/0.2.0/test/test-5.png}}

\paragraph{Biolinum Theme}\label{biolinum-theme}

\begin{Shaded}
\begin{Highlighting}[]
\NormalTok{\#let kbd = keyle.config(theme: keyle.themes.biolinum, delim: keyle.biolinum{-}key.delim\_plus)}
\NormalTok{\#keyle.gen{-}examples(kbd)}
\end{Highlighting}
\end{Shaded}

\pandocbounded{\includegraphics[keepaspectratio]{https://github.com/typst/packages/raw/main/packages/preview/keyle/0.2.0/test/test-6.png}}

\paragraph{Custom Theme}\label{custom-theme}

\begin{Shaded}
\begin{Highlighting}[]
\NormalTok{// https://www.radix{-}ui.com/themes/playground\#kbd}
\NormalTok{\#let radix\_kdb(content) = box(}
\NormalTok{  rect(}
\NormalTok{    inset: (x: 0.5em),}
\NormalTok{    outset: (y:0.05em),}
\NormalTok{    stroke: rgb("\#1c2024") + 0.3pt,}
\NormalTok{    radius: 0.35em,}
\NormalTok{    fill: rgb("\#fcfcfd"),}
\NormalTok{    text(fill: black, font: (}
\NormalTok{      "Roboto",}
\NormalTok{      "Helvetica Neue",}
\NormalTok{    ), content),}
\NormalTok{  ),}
\NormalTok{)}
\NormalTok{\#let kbd = keyle.config(theme: radix\_kdb)}
\NormalTok{\#keyle.gen{-}examples(kbd)}
\end{Highlighting}
\end{Shaded}

\pandocbounded{\includegraphics[keepaspectratio]{https://github.com/typst/packages/raw/main/packages/preview/keyle/0.2.0/test/test-7.png}}

\subsection{License}\label{license}

MIT

\subsubsection{How to add}\label{how-to-add}

Copy this into your project and use the import as \texttt{\ keyle\ }

\begin{verbatim}
#import "@preview/keyle:0.2.0"
\end{verbatim}

\includesvg[width=0.16667in,height=0.16667in]{/assets/icons/16-copy.svg}

Check the docs for
\href{https://typst.app/docs/reference/scripting/\#packages}{more
information on how to import packages} .

\subsubsection{About}\label{about}

\begin{description}
\tightlist
\item[Author :]
\href{mailto:yxnian@outlook.com}{magicwenli}
\item[License:]
MIT
\item[Current version:]
0.2.0
\item[Last updated:]
August 27, 2024
\item[First released:]
July 24, 2024
\item[Minimum Typst version:]
0.11.1
\item[Archive size:]
5.97 kB
\href{https://packages.typst.org/preview/keyle-0.2.0.tar.gz}{\pandocbounded{\includesvg[keepaspectratio]{/assets/icons/16-download.svg}}}
\item[Repository:]
\href{https://github.com/magicwenli/keyle}{GitHub}
\item[Categor ies :]
\begin{itemize}
\tightlist
\item[]
\item
  \pandocbounded{\includesvg[keepaspectratio]{/assets/icons/16-hammer.svg}}
  \href{https://typst.app/universe/search/?category=utility}{Utility}
\item
  \pandocbounded{\includesvg[keepaspectratio]{/assets/icons/16-smile.svg}}
  \href{https://typst.app/universe/search/?category=fun}{Fun}
\end{itemize}
\end{description}

\subsubsection{Where to report issues?}\label{where-to-report-issues}

This package is a project of magicwenli . Report issues on
\href{https://github.com/magicwenli/keyle}{their repository} . You can
also try to ask for help with this package on the
\href{https://forum.typst.app}{Forum} .

Please report this package to the Typst team using the
\href{https://typst.app/contact}{contact form} if you believe it is a
safety hazard or infringes upon your rights.

\phantomsection\label{versions}
\subsubsection{Version history}\label{version-history}

\begin{longtable}[]{@{}ll@{}}
\toprule\noalign{}
Version & Release Date \\
\midrule\noalign{}
\endhead
\bottomrule\noalign{}
\endlastfoot
0.2.0 & August 27, 2024 \\
\href{https://typst.app/universe/package/keyle/0.1.1/}{0.1.1} & August
12, 2024 \\
\href{https://typst.app/universe/package/keyle/0.1.0/}{0.1.0} & July 24,
2024 \\
\end{longtable}

Typst GmbH did not create this package and cannot guarantee correct
functionality of this package or compatibility with any version of the
Typst compiler or app.


\title{typst.app/universe/package/codly-languages}

\phantomsection\label{banner}
\section{codly-languages}\label{codly-languages}

{ 0.1.1 }

A set of language configurations for use with codly

\phantomsection\label{readme}
Provides a set of predefined language configurations for use with the
\texttt{\ codly\ } code listing package. For each supported language,
this package defines a name, icon, and color to use when displaying
code.

\subsection{Usage}\label{usage}

Pretty simple. Import \texttt{\ codly\ } . Initialize it. Import
\texttt{\ codly-languages\ } . Configure \texttt{\ codly\ } with the
languages. Like this:

\begin{Shaded}
\begin{Highlighting}[]
\NormalTok{\#import "@preview/codly:1.0.0": *}
\NormalTok{\#show: codly{-}init}

\NormalTok{\#import "@preview/codly{-}languages:0.1.1": *}
\NormalTok{\#codly(languages: codly{-}languages)}
\end{Highlighting}
\end{Shaded}

Then use code blocks as you normally would and the output, for supported
languages, should look like this:

\pandocbounded{\includegraphics[keepaspectratio]{https://github.com/typst/packages/raw/main/packages/preview/codly-languages/0.1.1/thumbnail.png}}

\subsection{Contributing}\label{contributing}

The following languages are still missing. All contributions welcome.

\begin{itemize}
\tightlist
\item
  ASP
\item
  ActionScript
\item
  Ada
\item
  AppleScript
\item
  AsciiDoc
\item
  Batch File
\item
  CFML
\item
  CSV
\item
  Cabal
\item
  Crontab
\item
  D
\item
  Diff
\item
  DotENV
\item
  Email
\item
  Fish
\item
  Fstab
\item
  GLSL
\item
  Graphviz
\item
  Groff
\item
  Group
\item
  INI
\item
  Jinja2
\item
  Jsonnet
\item
  Lean
\item
  Lisp
\item
  LiveScript
\item
  Makefile
\item
  MediaWiki
\item
  NSIS
\item
  Ninja
\item
  Org mode
\item
  Pascal
\item
  Passwd
\item
  Protobuf
\item
  Puppet
\item
  QML
\item
  Racket
\item
  Rego
\item
  Regular Expressions
\item
  Resolv
\item
  RestructuredText
\item
  Robot
\item
  SLS
\item
  SML
\item
  Slim
\item
  Strace
\item
  SublimeEthereum
\item
  SublimeJQ
\item
  SystemVerilo
\item
  TCL
\item
  TOML
\item
  Textile
\item
  TodoTxt
\item
  Verilog
\item
  WGSL
\item
  cmd-help
\item
  gnuplot
\item
  hosts
\item
  http-request-response
\item
  varlink
\item
  vscode-wgsl
\end{itemize}

\subsection{Icon Attribution}\label{icon-attribution}

The \texttt{\ typst-small.png\ } icon included in this package came from
the MIT-licensed \href{https://github.com/Dherse/codly}{codly} project.

All other icons included here came from the MIT-licensed
\href{https://github.com/devicons/devicon/}{devicon} project.

\subsection{License}\label{license}

This package is released under the MIT License.

\subsubsection{How to add}\label{how-to-add}

Copy this into your project and use the import as
\texttt{\ codly-languages\ }

\begin{verbatim}
#import "@preview/codly-languages:0.1.1"
\end{verbatim}

\includesvg[width=0.16667in,height=0.16667in]{/assets/icons/16-copy.svg}

Check the docs for
\href{https://typst.app/docs/reference/scripting/\#packages}{more
information on how to import packages} .

\subsubsection{About}\label{about}

\begin{description}
\tightlist
\item[Author :]
\href{mailto:steve@waits.net}{Stephen Waits}
\item[License:]
MIT
\item[Current version:]
0.1.1
\item[Last updated:]
November 21, 2024
\item[First released:]
November 18, 2024
\item[Archive size:]
96.2 kB
\href{https://packages.typst.org/preview/codly-languages-0.1.1.tar.gz}{\pandocbounded{\includesvg[keepaspectratio]{/assets/icons/16-download.svg}}}
\item[Repository:]
\href{https://github.com/swaits/typst-collection}{GitHub}
\end{description}

\subsubsection{Where to report issues?}\label{where-to-report-issues}

This package is a project of Stephen Waits . Report issues on
\href{https://github.com/swaits/typst-collection}{their repository} .
You can also try to ask for help with this package on the
\href{https://forum.typst.app}{Forum} .

Please report this package to the Typst team using the
\href{https://typst.app/contact}{contact form} if you believe it is a
safety hazard or infringes upon your rights.

\phantomsection\label{versions}
\subsubsection{Version history}\label{version-history}

\begin{longtable}[]{@{}ll@{}}
\toprule\noalign{}
Version & Release Date \\
\midrule\noalign{}
\endhead
\bottomrule\noalign{}
\endlastfoot
0.1.1 & November 21, 2024 \\
\href{https://typst.app/universe/package/codly-languages/0.1.0/}{0.1.0}
& November 18, 2024 \\
\end{longtable}

Typst GmbH did not create this package and cannot guarantee correct
functionality of this package or compatibility with any version of the
Typst compiler or app.


\title{typst.app/universe/package/treet}

\phantomsection\label{banner}
\section{treet}\label{treet}

{ 0.1.1 }

Create tree lists easily

\phantomsection\label{readme}
\href{https://github.com/8LWXpg/typst-treet/tags}{\pandocbounded{\includegraphics[keepaspectratio]{https://img.shields.io/github/v/tag/8LWXpg/typst-treet}}}
\href{https://github.com/8LWXpg/typst-treet}{\pandocbounded{\includegraphics[keepaspectratio]{https://img.shields.io/github/stars/8LWXpg/typst-treet?style=flat}}}
\href{https://github.com/8LWXpg/typst-treet/blob/master/LICENSE}{\pandocbounded{\includegraphics[keepaspectratio]{https://img.shields.io/github/license/8LWXpg/typst-treet}}}
\href{https://github.com/typst/packages/tree/main/packages/preview/treet}{\pandocbounded{\includegraphics[keepaspectratio]{https://img.shields.io/badge/typst-package-239dad}}}

Create tree list easily in Typst

contribution is welcomed!

\subsection{Usage}\label{usage}

\begin{Shaded}
\begin{Highlighting}[]
\NormalTok{\#import "@preview/treet:0.1.1": *}

\NormalTok{\#tree{-}list(}
\NormalTok{  marker:       content,}
\NormalTok{  last{-}marker:  content,}
\NormalTok{  indent:       content,}
\NormalTok{  empty{-}indent: content,}
\NormalTok{  marker{-}font:  string,}
\NormalTok{  content,}
\NormalTok{)}
\end{Highlighting}
\end{Shaded}

\subsubsection{Parameters}\label{parameters}

\begin{itemize}
\tightlist
\item
  \texttt{\ marker\ } - the marker of the tree list, default is
  \texttt{\ {[}├─\ {]}\ }
\item
  \texttt{\ last-marker\ } - the marker of the last item of the tree
  list, default is \texttt{\ {[}└─\ {]}\ }
\item
  \texttt{\ indent\ } - the indent after \texttt{\ marker\ } , default
  is \texttt{\ {[}│\#h(1em){]}\ }
\item
  \texttt{\ empty-indent\ } - the indent after \texttt{\ last-marker\ }
  , default is \texttt{\ {[}\#h(1.5em){]}\ } (same width as indent)
\item
  \texttt{\ marker-font\ } - the font of the marker, default is
  \texttt{\ "Cascadia\ Code"\ }
\item
  \texttt{\ content\ } - the content of the tree list, includes at least
  a list
\end{itemize}

\subsection{Demo}\label{demo}

see
\href{https://github.com/8LWXpg/typst-treet/blob/master/test/demo.typ}{demo.typ}
\href{https://github.com/8LWXpg/typst-treet/blob/master/test/demo.pdf}{demo.pdf}

\subsubsection{Default style}\label{default-style}

\begin{Shaded}
\begin{Highlighting}[]
\NormalTok{\#tree{-}list[}
\NormalTok{  {-} 1}
\NormalTok{    {-} 1.1}
\NormalTok{      {-} 1.1.1}
\NormalTok{    {-} 1.2}
\NormalTok{      {-} 1.2.1}
\NormalTok{      {-} 1.2.2}
\NormalTok{        {-} 1.2.2.1}
\NormalTok{  {-} 2}
\NormalTok{  {-} 3}
\NormalTok{    {-} 3.1}
\NormalTok{      {-} 3.1.1}
\NormalTok{    {-} 3.2}
\NormalTok{]}
\end{Highlighting}
\end{Shaded}

\pandocbounded{\includegraphics[keepaspectratio]{https://raw.githubusercontent.com/8LWXpg/typst-treet/master/img/1.png}}

\subsubsection{Custom style}\label{custom-style}

\begin{Shaded}
\begin{Highlighting}[]
\NormalTok{\#text(red, tree{-}list(}
\NormalTok{  marker: text(blue)[├── ],}
\NormalTok{  last{-}marker: text(aqua)[└── ],}
\NormalTok{  indent: text(teal)[│\#h(1.5em)],}
\NormalTok{  empty{-}indent: h(2em),}
\NormalTok{)[}
\NormalTok{  {-} 1}
\NormalTok{    {-} 1.1}
\NormalTok{      {-} 1.1.1}
\NormalTok{    {-} 1.2}
\NormalTok{      {-} 1.2.1}
\NormalTok{      {-} 1.2.2}
\NormalTok{        {-} 1.2.2.1}
\NormalTok{  {-} 2}
\NormalTok{  {-} 3}
\NormalTok{    {-} 3.1}
\NormalTok{      {-} 3.1.1}
\NormalTok{    {-} 3.2}
\NormalTok{])}
\end{Highlighting}
\end{Shaded}

\pandocbounded{\includegraphics[keepaspectratio]{https://raw.githubusercontent.com/8LWXpg/typst-treet/master/img/2.png}}

\subsubsection{Using show rule}\label{using-show-rule}

\begin{Shaded}
\begin{Highlighting}[]
\NormalTok{\#show list: tree{-}list}
\NormalTok{\#set text(font: "DejaVu Sans Mono")}

\NormalTok{root\_folder\textbackslash{}}
\NormalTok{{-} sub{-}folder}
\NormalTok{  {-} 1{-}1}
\NormalTok{    {-} 1.1.1 {-}}
\NormalTok{  {-} 1.2}
\NormalTok{    {-} 1.2.1}
\NormalTok{    {-} 1.2.2}
\NormalTok{{-} 2}
\end{Highlighting}
\end{Shaded}

\pandocbounded{\includegraphics[keepaspectratio]{https://raw.githubusercontent.com/8LWXpg/typst-treet/master/img/3.png}}

\subsubsection{How to add}\label{how-to-add}

Copy this into your project and use the import as \texttt{\ treet\ }

\begin{verbatim}
#import "@preview/treet:0.1.1"
\end{verbatim}

\includesvg[width=0.16667in,height=0.16667in]{/assets/icons/16-copy.svg}

Check the docs for
\href{https://typst.app/docs/reference/scripting/\#packages}{more
information on how to import packages} .

\subsubsection{About}\label{about}

\begin{description}
\tightlist
\item[Author :]
8LWXpg
\item[License:]
MIT
\item[Current version:]
0.1.1
\item[Last updated:]
April 15, 2024
\item[First released:]
January 8, 2024
\item[Minimum Typst version:]
0.10.0
\item[Archive size:]
2.41 kB
\href{https://packages.typst.org/preview/treet-0.1.1.tar.gz}{\pandocbounded{\includesvg[keepaspectratio]{/assets/icons/16-download.svg}}}
\item[Repository:]
\href{https://github.com/8LWXpg/typst-treet}{GitHub}
\item[Categor ies :]
\begin{itemize}
\tightlist
\item[]
\item
  \pandocbounded{\includesvg[keepaspectratio]{/assets/icons/16-package.svg}}
  \href{https://typst.app/universe/search/?category=components}{Components}
\item
  \pandocbounded{\includesvg[keepaspectratio]{/assets/icons/16-layout.svg}}
  \href{https://typst.app/universe/search/?category=layout}{Layout}
\end{itemize}
\end{description}

\subsubsection{Where to report issues?}\label{where-to-report-issues}

This package is a project of 8LWXpg . Report issues on
\href{https://github.com/8LWXpg/typst-treet}{their repository} . You can
also try to ask for help with this package on the
\href{https://forum.typst.app}{Forum} .

Please report this package to the Typst team using the
\href{https://typst.app/contact}{contact form} if you believe it is a
safety hazard or infringes upon your rights.

\phantomsection\label{versions}
\subsubsection{Version history}\label{version-history}

\begin{longtable}[]{@{}ll@{}}
\toprule\noalign{}
Version & Release Date \\
\midrule\noalign{}
\endhead
\bottomrule\noalign{}
\endlastfoot
0.1.1 & April 15, 2024 \\
\href{https://typst.app/universe/package/treet/0.1.0/}{0.1.0} & January
8, 2024 \\
\end{longtable}

Typst GmbH did not create this package and cannot guarantee correct
functionality of this package or compatibility with any version of the
Typst compiler or app.


\title{typst.app/universe/package/graceful-genetics}

\phantomsection\label{banner}
\phantomsection\label{template-thumbnail}
\pandocbounded{\includegraphics[keepaspectratio]{https://packages.typst.org/preview/thumbnails/graceful-genetics-0.2.0-small.webp}}

\section{graceful-genetics}\label{graceful-genetics}

{ 0.2.0 }

A paper template with which to publish in journals and at conferences

{ } Featured Template

\href{/app?template=graceful-genetics&version=0.2.0}{Create project in
app}

\phantomsection\label{readme}
Version 0.2.0

A recreation of the Oxford Physics template shown on the typst.app
homepage.

\subsection{Media}\label{media}

\includegraphics[width=0.45\linewidth,height=\textheight,keepaspectratio]{https://github.com/typst/packages/raw/main/packages/preview/graceful-genetics/0.2.0/thumbnails/1.png}
\includegraphics[width=0.45\linewidth,height=\textheight,keepaspectratio]{https://github.com/typst/packages/raw/main/packages/preview/graceful-genetics/0.2.0/thumbnails/2.png}

\subsection{Getting Started}\label{getting-started}

To use this template, simply import it as shown below:

\begin{Shaded}
\begin{Highlighting}[]
\NormalTok{\#import "@preview/graceful{-}genetics:0.2.0"}

\NormalTok{\#show: graceful{-}genetics.template.with(}
\NormalTok{  title: [Towards Swifter Interstellar Mail Delivery],}
\NormalTok{  authors: (}
\NormalTok{    (}
\NormalTok{      name: "Johanna Swift",}
\NormalTok{      department: "Primary Logistics Department",}
\NormalTok{      institution: "Delivery Institute",}
\NormalTok{      city: "Berlin",}
\NormalTok{      country: "Germany",}
\NormalTok{      mail: "swift@delivery.de",}
\NormalTok{    ),}
\NormalTok{    (}
\NormalTok{      name: "Egon Stellaris",}
\NormalTok{      department: "Communications Group",}
\NormalTok{      institution: "Space Institute",}
\NormalTok{      city: "Florence",}
\NormalTok{      country: "Italy",}
\NormalTok{      mail: "stegonaris@space.it",}
\NormalTok{    ),}
\NormalTok{    (}
\NormalTok{      name: "Oliver Liam",}
\NormalTok{      department: "Missing Letters Task Force",}
\NormalTok{      institution: "Mail Institute",}
\NormalTok{      city: "Budapest",}
\NormalTok{      country: "Hungary",}
\NormalTok{      mail: "oliver.liam@mail.hu",}
\NormalTok{    ),}
\NormalTok{  ),}
\NormalTok{  date: (}
\NormalTok{    year: 2022,}
\NormalTok{    month: "May",}
\NormalTok{    day: 17,}
\NormalTok{  ),}
\NormalTok{  keywords: (}
\NormalTok{    "Space",}
\NormalTok{    "Mail",}
\NormalTok{    "Astromail",}
\NormalTok{    "Faster{-}than{-}Light",}
\NormalTok{    "Mars",}
\NormalTok{  ),}
\NormalTok{  doi: "10:7891/120948510",}
\NormalTok{  abstract: [}
\NormalTok{    Recent advances in space{-}based document processing have enabled faster mail delivery between different planets of a solar system. Given the time it takes for a message to be transmitted from one planet to the next, its estimated that even a one{-}way trip to a distant destination could take up to one year. During these periods of interplanetary mail delivery there is a slight possibility of mail being lost in transit. This issue is considered so serious that space management employs P.I. agents to track down and retrieve lost mail. We propose A{-}Mail, a new anti{-}matter based approach that can ensure that mail loss occurring during interplanetary transit is unobservable and therefore potentially undetectable. Going even further, we extend A{-}Mail to predict problems and apply existing and new best practices to ensure the mail is delivered without any issues. We call this extension AI{-}Mail.}
\NormalTok{  ]}
\NormalTok{)}
\end{Highlighting}
\end{Shaded}

\href{/app?template=graceful-genetics&version=0.2.0}{Create project in
app}

\subsubsection{How to use}\label{how-to-use}

Click the button above to create a new project using this template in
the Typst app.

You can also use the Typst CLI to start a new project on your computer
using this command:

\begin{verbatim}
typst init @preview/graceful-genetics:0.2.0
\end{verbatim}

\includesvg[width=0.16667in,height=0.16667in]{/assets/icons/16-copy.svg}

\subsubsection{About}\label{about}

\begin{description}
\tightlist
\item[Author :]
James R. Swift
\item[License:]
Unlicense
\item[Current version:]
0.2.0
\item[Last updated:]
October 30, 2024
\item[First released:]
July 16, 2024
\item[Minimum Typst version:]
0.12.0
\item[Archive size:]
24.5 kB
\href{https://packages.typst.org/preview/graceful-genetics-0.2.0.tar.gz}{\pandocbounded{\includesvg[keepaspectratio]{/assets/icons/16-download.svg}}}
\item[Repository:]
\href{https://github.com/JamesxX/graceful-genetics}{GitHub}
\item[Categor y :]
\begin{itemize}
\tightlist
\item[]
\item
  \pandocbounded{\includesvg[keepaspectratio]{/assets/icons/16-atom.svg}}
  \href{https://typst.app/universe/search/?category=paper}{Paper}
\end{itemize}
\end{description}

\subsubsection{Where to report issues?}\label{where-to-report-issues}

This template is a project of James R. Swift . Report issues on
\href{https://github.com/JamesxX/graceful-genetics}{their repository} .
You can also try to ask for help with this template on the
\href{https://forum.typst.app}{Forum} .

Please report this template to the Typst team using the
\href{https://typst.app/contact}{contact form} if you believe it is a
safety hazard or infringes upon your rights.

\phantomsection\label{versions}
\subsubsection{Version history}\label{version-history}

\begin{longtable}[]{@{}ll@{}}
\toprule\noalign{}
Version & Release Date \\
\midrule\noalign{}
\endhead
\bottomrule\noalign{}
\endlastfoot
0.2.0 & October 30, 2024 \\
\href{https://typst.app/universe/package/graceful-genetics/0.1.0/}{0.1.0}
& July 16, 2024 \\
\end{longtable}

Typst GmbH did not create this template and cannot guarantee correct
functionality of this template or compatibility with any version of the
Typst compiler or app.


\title{typst.app/universe/package/g-exam}

\phantomsection\label{banner}
\phantomsection\label{template-thumbnail}
\pandocbounded{\includegraphics[keepaspectratio]{https://packages.typst.org/preview/thumbnails/g-exam-0.4.1-small.webp}}

\section{g-exam}\label{g-exam}

{ 0.4.1 }

Create exams with student information, grade chart, score control,
questions, and sub-questions.

\href{/app?template=g-exam&version=0.4.1}{Create project in app}

\phantomsection\label{readme}
This template provides a way to generate exams. You can create questions
and sub-questions, header with information about the academic center,
score box, subject, exam, header with student information,
clarifications, solutions, watermark with information about the exam
model and teacher.

\paragraph{Features}\label{features}

\begin{itemize}
\tightlist
\item
  Scoreboard.
\item
  Scoring by questions and subquestions.
\item
  Student information, on the first page or on all odd pages.
\item
  Question and subcuestion.
\item
  Show solutions and clarifications
\item
  List of clarifications.
\item
  Teacher’s Watermark
\item
  Exam Model Watermark
\item
  Draft mode
\end{itemize}

\subsection{Usage}\label{usage}

For information, see the
\href{https://matheschool.github.io/typst-g-exam/}{online
docucumentation} .

To use this package, simply add the following code to your document:

\paragraph{A sample exam}\label{a-sample-exam}

\pandocbounded{\includegraphics[keepaspectratio]{https://github.com/typst/packages/raw/main/packages/preview/g-exam/0.4.1/gallery/exam-table-content.png}}

\paragraph{Source:}\label{source}

\begin{Shaded}
\begin{Highlighting}[]
\NormalTok{\#import "@preview/g{-}exam:0.4.1": *}

\NormalTok{\#show: exam.with(}
\NormalTok{  school: (}
\NormalTok{    name: "Sunrise Secondary School",}
\NormalTok{    logo: read("./logo.png", encoding: none),}
\NormalTok{  ),}
\NormalTok{  exam{-}info: (}
\NormalTok{    academic{-}period: "Academic year 2023/2024",}
\NormalTok{    academic{-}level: "1st Secondary Education",}
\NormalTok{    academic{-}subject: "Mathematics",}
\NormalTok{    number: "2nd Assessment 1st Exam",}
\NormalTok{    content: "Radicals and fractions",}
\NormalTok{    model: "Model A"}
\NormalTok{  ),}
  
\NormalTok{  show{-}student{-}data: "first{-}page",}
\NormalTok{  show{-}grade{-}table: true,}
\NormalTok{  clarifications: "Answer the questions in the spaces provided. If you run out of room for an answer, continue on the back of the page."}
\NormalTok{)}
\NormalTok{\#question(points:2.5)[Is it true that $x\^{}n + y\^{}n = z\^{}n$ if $(x,y,z)$ and $n$ are positive integers?. Explain.] }
\NormalTok{\#v(1fr)}

\NormalTok{\#question(points:2.5)[Prove that the real part of all non{-}trivial zeros of the function $zeta(z) "is" 1/2$].}
\NormalTok{\#v(1fr)}

\NormalTok{\#question(points:2)[Compute $ integral\_0\^{}infinity (sin(x))/x $ ]}
\NormalTok{\#v(1fr)}
\end{Highlighting}
\end{Shaded}

\subsection{Changelog}\label{changelog}

\subsubsection{v0.4.1}\label{v0.4.1}

\begin{itemize}
\tightlist
\item
  Fix student data.
\item
  Fix Indenting subquestion.
\end{itemize}

\subsubsection{v0.4.0}\label{v0.4.0}

\begin{itemize}
\tightlist
\item
  Change g-exam for exam.
\item
  Change g-question and g-subquestion for question and subquestion.
\item
  Change point parameter to points in question and subquestion.
\item
  Change question-points-position paramet to question-points-position.
\item
  Include online documentation.
\item
  Use paper by default.
\item
  Indenting subquestion.
\item
  Include support for dutch language.
\item
  Corrections in English texts.
\item
  Draft label.
\end{itemize}

\subsubsection{v0.3.2}\label{v0.3.2}

\begin{itemize}
\tightlist
\item
  Change show-studen-data to show-student-data parameter.
\item
  Change languaje to language parameter.
\end{itemize}

\subsubsection{v0.3.1}\label{v0.3.1}

\begin{itemize}
\tightlist
\item
  Corrections in French.
\end{itemize}

\subsubsection{v0.3.0}\label{v0.3.0}

\begin{itemize}
\tightlist
\item
  Include parameter question-text-parameters.
\item
  Show solution.
\item
  Expand documentation.
\item
  Possibility of estrablecer question-point-position to none.
\item
  Bug fix show watermark.
\end{itemize}

\subsubsection{v0.2.0}\label{v0.2.0}

\begin{itemize}
\tightlist
\item
  Control the size of the logo image.
\item
  Convert to template
\item
  Allow true and false values in show-student-data.
\item
  Show clarifications.
\item
  Widen margin points.
\item
  Show solution.
\end{itemize}

\subsubsection{v0.1.1}\label{v0.1.1}

\begin{itemize}
\tightlist
\item
  Fix loading image.
\end{itemize}

\subsubsection{v0.1.0}\label{v0.1.0}

\begin{itemize}
\tightlist
\item
  Initial version submitted to typst/packages.
\end{itemize}

\href{/app?template=g-exam&version=0.4.1}{Create project in app}

\subsubsection{How to use}\label{how-to-use}

Click the button above to create a new project using this template in
the Typst app.

You can also use the Typst CLI to start a new project on your computer
using this command:

\begin{verbatim}
typst init @preview/g-exam:0.4.1
\end{verbatim}

\includesvg[width=0.16667in,height=0.16667in]{/assets/icons/16-copy.svg}

\subsubsection{About}\label{about}

\begin{description}
\tightlist
\item[Author :]
Andrés Giménez Muñoz
\item[License:]
MIT
\item[Current version:]
0.4.1
\item[Last updated:]
November 19, 2024
\item[First released:]
February 21, 2024
\item[Minimum Typst version:]
0.12.0
\item[Archive size:]
177 kB
\href{https://packages.typst.org/preview/g-exam-0.4.1.tar.gz}{\pandocbounded{\includesvg[keepaspectratio]{/assets/icons/16-download.svg}}}
\item[Repository:]
\href{https://github.com/MatheSchool/typst-g-exam}{GitHub}
\item[Discipline :]
\begin{itemize}
\tightlist
\item[]
\item
  \href{https://typst.app/universe/search/?discipline=education}{Education}
\end{itemize}
\item[Categor y :]
\begin{itemize}
\tightlist
\item[]
\item
  \pandocbounded{\includesvg[keepaspectratio]{/assets/icons/16-envelope.svg}}
  \href{https://typst.app/universe/search/?category=office}{Office}
\end{itemize}
\end{description}

\subsubsection{Where to report issues?}\label{where-to-report-issues}

This template is a project of Andrés Giménez Muñoz . Report issues on
\href{https://github.com/MatheSchool/typst-g-exam}{their repository} .
You can also try to ask for help with this template on the
\href{https://forum.typst.app}{Forum} .

Please report this template to the Typst team using the
\href{https://typst.app/contact}{contact form} if you believe it is a
safety hazard or infringes upon your rights.

\phantomsection\label{versions}
\subsubsection{Version history}\label{version-history}

\begin{longtable}[]{@{}ll@{}}
\toprule\noalign{}
Version & Release Date \\
\midrule\noalign{}
\endhead
\bottomrule\noalign{}
\endlastfoot
0.4.1 & November 19, 2024 \\
\href{https://typst.app/universe/package/g-exam/0.4.0/}{0.4.0} &
November 8, 2024 \\
\href{https://typst.app/universe/package/g-exam/0.3.2/}{0.3.2} & August
26, 2024 \\
\href{https://typst.app/universe/package/g-exam/0.3.1/}{0.3.1} & July
23, 2024 \\
\href{https://typst.app/universe/package/g-exam/0.3.0/}{0.3.0} & April
8, 2024 \\
\href{https://typst.app/universe/package/g-exam/0.2.0/}{0.2.0} & March
21, 2024 \\
\href{https://typst.app/universe/package/g-exam/0.1.1/}{0.1.1} &
February 22, 2024 \\
\href{https://typst.app/universe/package/g-exam/0.1.0/}{0.1.0} &
February 21, 2024 \\
\end{longtable}

Typst GmbH did not create this template and cannot guarantee correct
functionality of this template or compatibility with any version of the
Typst compiler or app.


