\section{Package List LaTeX/parcio-thesis.tex}
\title{typst.app/universe/package/parcio-thesis}

\phantomsection\label{banner}
\phantomsection\label{template-thumbnail}
\pandocbounded{\includegraphics[keepaspectratio]{https://packages.typst.org/preview/thumbnails/parcio-thesis-0.1.0-small.webp}}

\section{parcio-thesis}\label{parcio-thesis}

{ 0.1.0 }

A simple thesis template based on the ParCIO working group at OvGU
Magdeburg.

\href{/app?template=parcio-thesis&version=0.1.0}{Create project in app}

\phantomsection\label{readme}
\includegraphics[width=0.32\linewidth,height=\textheight,keepaspectratio]{https://github.com/typst/packages/raw/main/packages/preview/parcio-thesis/0.1.0/thumbnails/1.png}
\includegraphics[width=0.32\linewidth,height=\textheight,keepaspectratio]{https://github.com/typst/packages/raw/main/packages/preview/parcio-thesis/0.1.0/thumbnails/2.png}
\includegraphics[width=0.32\linewidth,height=\textheight,keepaspectratio]{https://github.com/typst/packages/raw/main/packages/preview/parcio-thesis/0.1.0/thumbnails/3.png}

A simple thesis template based on the ParCIO working group at
Otto-von-Guericke University Magdeburg.

\subsection{Getting Started}\label{getting-started}

To use this template, simply import it as shown below (more options
under \texttt{\ Usage\ } ):

\begin{Shaded}
\begin{Highlighting}[]
\NormalTok{\#import "@preview/parcio{-}thesis:0.1.0": *}

\NormalTok{\#show: parcio.with(}
\NormalTok{  title: "My great thesis",}
\NormalTok{  author: (}
\NormalTok{    name: "Author",}
\NormalTok{    mail: "author@ovgu.de"}
\NormalTok{  ),}
\NormalTok{  abstract: [My abstract begins here.],}
\NormalTok{)}
\end{Highlighting}
\end{Shaded}

\subsubsection{Local Installation}\label{local-installation}

Following these steps will make the template available locally under the
\texttt{\ @local\ } namespace. Requires
\href{https://github.com/casey/just}{“Just - A Command Runner�} .

\begin{Shaded}
\begin{Highlighting}[]
\FunctionTok{git}\NormalTok{ clone git@github.com:xkevio/parcio{-}typst.git }
\BuiltInTok{cd}\NormalTok{ parcio{-}typst/parcio{-}thesis/}
\ExtensionTok{just}\NormalTok{ install}
\end{Highlighting}
\end{Shaded}

\subsection{Usage}\label{usage}

See here for \textbf{all} possible arguments (and their default values)
and utility functions:

\begin{Shaded}
\begin{Highlighting}[]
\NormalTok{\#import "@preview/parcio{-}thesis:0.1.0": *}

\NormalTok{\#show: parcio.with(}
\NormalTok{  title: "Title",}
\NormalTok{  author: (name: "Author", mail: "author@ovgu.de"),}
\NormalTok{  abstract: [],}
\NormalTok{  thesis{-}type: "Bachelor/Master",}
\NormalTok{  reviewers: (),}
\NormalTok{  date: datetime.today(),}
\NormalTok{  lang: "en",}
\NormalTok{  header{-}logo: none,}
\NormalTok{  translations: none,}
\NormalTok{)}

\NormalTok{// Use these to *enable* or *change* page numbering for your frontmatter and mainmatter respectively.}
\NormalTok{// (By default, this template hides the page numbering!)}
\NormalTok{\#show: roman{-}numbering.with(reset: \textless{}true|false\textgreater{})}
\NormalTok{\#show: arabic{-}numbering.with(alternate: \textless{}true|false\textgreater{}, reset: \textless{}true|false\textgreater{})}
\end{Highlighting}
\end{Shaded}

\subsubsection{Utility Functions}\label{utility-functions}

These could be useful while writing your thesis!

\begin{Shaded}
\begin{Highlighting}[]
\NormalTok{// A TODO marker. (inline: false {-}\textgreater{} margin note, inline: true {-}\textgreater{} box).}
\NormalTok{\#let todo(inline: false, body)}

\NormalTok{// Like \textbackslash{}section* in LaTeX. (unnumbered level 2 heading, not in ToC).}
\NormalTok{\#let section = heading.with(level: 2, outlined: false, numbering: none)}

\NormalTok{// A neat inline{-}section in smallcaps and sans font.}
\NormalTok{\#let inline{-}section(title) = smallcaps[*\#text(font: "Libertinus Sans", title)*] }

\NormalTok{// Fully empty page, no page numbering.}
\NormalTok{\#let empty{-}page = page([], footer: [])}

\NormalTok{// Subfigures (see chapters/introduction for syntax).}
\NormalTok{\#let subfigure(..)}

\NormalTok{// A ParCIO{-}like table with a design taken from the LaTeX template.}
\NormalTok{\#let parcio{-}table(max{-}rows, ..args)}

\NormalTok{// Nicer handling of (multiple) appendices. Specify \textasciigrave{}reset: true\textasciigrave{} with your first appendix to reset the heading counter!}
\NormalTok{\#let appendix(reset: false, label: none, body)}
\end{Highlighting}
\end{Shaded}

\subsubsection{Translations}\label{translations}

If you wish, you can provide custom translations for things like
“Section�, “Contents�, etc. by providing a custom
\texttt{\ translations.toml\ } (this template already comes with
translations for English and German) with the following schema:

\begin{Shaded}
\begin{Highlighting}[]
\CommentTok{\# Top{-}level dict name should follow ISO 639 language codes!}
\DataTypeTok{default{-}lang} \OperatorTok{=} \StringTok{"en"}

\KeywordTok{[de]}
\DataTypeTok{contents} \OperatorTok{=} \StringTok{"Inhaltsverzeichnis"}
\DataTypeTok{chapter} \OperatorTok{=} \StringTok{"Kapitel"}
\DataTypeTok{section} \OperatorTok{=} \StringTok{"Sektion"}
\DataTypeTok{thesis} \OperatorTok{=} \OperatorTok{\{ }\DataTypeTok{value}\OperatorTok{ =} \StringTok{"Arbeit"}\OperatorTok{, }\DataTypeTok{compound}\OperatorTok{ =} \ConstantTok{true}\OperatorTok{ \}}

\KeywordTok{[de.title{-}page]}
\DataTypeTok{first{-}reviewer} \OperatorTok{=} \StringTok{"Erstgutachter"}
\DataTypeTok{second{-}reviewer} \OperatorTok{=} \StringTok{"Zweitgutachter"}
\DataTypeTok{supervisor} \OperatorTok{=} \StringTok{"Betreuer"}

\KeywordTok{[de.bibliography]}
\DataTypeTok{bibliography} \OperatorTok{=} \StringTok{"Quellenverzeichnis"}
\DataTypeTok{cited{-}on{-}page} \OperatorTok{=} \StringTok{"Zitiert auf Seite"}
\DataTypeTok{cited{-}on{-}pages} \OperatorTok{=} \StringTok{"Zitiert auf Seiten"}
\DataTypeTok{join} \OperatorTok{=} \StringTok{"und"}

\KeywordTok{[de.date]}
\DataTypeTok{date{-}format} \OperatorTok{=} \StringTok{"[day]. [month repr:long] [year]"}
\DataTypeTok{months} \OperatorTok{=} \OperatorTok{[}\StringTok{"Januar"}\OperatorTok{,} \StringTok{"Februar"}\OperatorTok{,} \StringTok{"März"}\OperatorTok{,} \StringTok{"April"}\OperatorTok{,} \StringTok{"Mai"}\OperatorTok{,} \StringTok{"Juni"}\OperatorTok{,} \StringTok{"Juli"}\OperatorTok{,} \StringTok{"August"}\OperatorTok{,} \StringTok{"September"}\OperatorTok{,} \StringTok{"Oktober"}\OperatorTok{,} \StringTok{"November"}\OperatorTok{,} \StringTok{"Dezember"}\OperatorTok{]}
\end{Highlighting}
\end{Shaded}

\subsection{Fonts and OvGU Corporate
Design}\label{fonts-and-ovgu-corporate-design}

This template requires these three fonts to be installed on your
system{[}\^{}1{]}:

\begin{itemize}
\tightlist
\item
  Libertinus Serif ( \url{https://github.com/alerque/libertinus} )
\item
  Libertinus Sans ( \url{https://github.com/alerque/libertinus} )
\item
  Inconsolata ( \url{https://github.com/googlefonts/Inconsolata} )
\end{itemize}

We bundle the default “Faculty of Computer Science� head banner and
use it as the \texttt{\ header-logo\ } . You can find yours at:
\href{https://www.cd.ovgu.de/Fakult\%C3\%A4ten.html}{https://www.cd.ovgu.de/Fakultäten.html}
.

{[}\^{}1{]}: Typst should already provide the Libertinus font family by
default as it is their standard font.

\href{/app?template=parcio-thesis&version=0.1.0}{Create project in app}

\subsubsection{How to use}\label{how-to-use}

Click the button above to create a new project using this template in
the Typst app.

You can also use the Typst CLI to start a new project on your computer
using this command:

\begin{verbatim}
typst init @preview/parcio-thesis:0.1.0
\end{verbatim}

\includesvg[width=0.16667in,height=0.16667in]{/assets/icons/16-copy.svg}

\subsubsection{About}\label{about}

\begin{description}
\tightlist
\item[Author :]
\href{https://github.com/xkevio}{Kevin Kulot}
\item[License:]
0BSD
\item[Current version:]
0.1.0
\item[Last updated:]
November 19, 2024
\item[First released:]
November 19, 2024
\item[Minimum Typst version:]
0.12.0
\item[Archive size:]
20.3 kB
\href{https://packages.typst.org/preview/parcio-thesis-0.1.0.tar.gz}{\pandocbounded{\includesvg[keepaspectratio]{/assets/icons/16-download.svg}}}
\item[Repository:]
\href{https://github.com/xkevio/parcio-typst/}{GitHub}
\item[Categor ies :]
\begin{itemize}
\tightlist
\item[]
\item
  \pandocbounded{\includesvg[keepaspectratio]{/assets/icons/16-mortarboard.svg}}
  \href{https://typst.app/universe/search/?category=thesis}{Thesis}
\item
  \pandocbounded{\includesvg[keepaspectratio]{/assets/icons/16-speak.svg}}
  \href{https://typst.app/universe/search/?category=report}{Report}
\end{itemize}
\end{description}

\subsubsection{Where to report issues?}\label{where-to-report-issues}

This template is a project of Kevin Kulot . Report issues on
\href{https://github.com/xkevio/parcio-typst/}{their repository} . You
can also try to ask for help with this template on the
\href{https://forum.typst.app}{Forum} .

Please report this template to the Typst team using the
\href{https://typst.app/contact}{contact form} if you believe it is a
safety hazard or infringes upon your rights.

\phantomsection\label{versions}
\subsubsection{Version history}\label{version-history}

\begin{longtable}[]{@{}ll@{}}
\toprule\noalign{}
Version & Release Date \\
\midrule\noalign{}
\endhead
\bottomrule\noalign{}
\endlastfoot
0.1.0 & November 19, 2024 \\
\end{longtable}

Typst GmbH did not create this template and cannot guarantee correct
functionality of this template or compatibility with any version of the
Typst compiler or app.


\section{Package List LaTeX/not-tudabeamer-2023.tex}
\title{typst.app/universe/package/not-tudabeamer-2023}

\phantomsection\label{banner}
\phantomsection\label{template-thumbnail}
\pandocbounded{\includegraphics[keepaspectratio]{https://packages.typst.org/preview/thumbnails/not-tudabeamer-2023-0.1.0-small.webp}}

\section{not-tudabeamer-2023}\label{not-tudabeamer-2023}

{ 0.1.0 }

Not the TU Darmstadt Beamer 2023 template

\href{/app?template=not-tudabeamer-2023&version=0.1.0}{Create project in
app}

\phantomsection\label{readme}
A \href{https://github.com/touying-typ/touying}{touying} presentation
template matching the TU Darmstadt Beamer Template 2023.

\subsection{Usage}\label{usage}

Install Roboto font for your system or download them from
\url{https://github.com/googlefonts/roboto/releases/download/v2.138/roboto-unhinted.zip}
.

Run \texttt{\ typst\ init\ @preview/not-tudabeamer-2023:0.1.0\ }

Download
\url{https://download.hrz.tu-darmstadt.de/protected/ULB/tuda_logo.pdf} .

Run \texttt{\ pdf2svg\ tuda\_logo.pdf\ tuda\_logo.svg\ } or convert to
\texttt{\ .svg\ } using e.g. Inkscape.

\subsubsection{Examples}\label{examples}

\begin{Shaded}
\begin{Highlighting}[]
\NormalTok{\#import "@preview/not{-}tudabeamer{-}2023:0.1.0": *}

\NormalTok{\#show: not{-}tudabeamer{-}2023{-}theme.with(}
\NormalTok{  config{-}info(}
\NormalTok{    title: [Title],}
\NormalTok{    short{-}title: [Title],}
\NormalTok{    subtitle: [Subtitle],}
\NormalTok{    author: "Author",}
\NormalTok{    short{-}author: "Author",}
\NormalTok{    date: datetime.today(),}
\NormalTok{    department: [Department],}
\NormalTok{    institute: [Institute],}
\NormalTok{    logo: text(fallback: true, size: 0.75in, emoji.cat.face)}
\NormalTok{    //logo: image("tuda\_logo.svg", height: 100\%)}
\NormalTok{  )}
\NormalTok{)}

\NormalTok{\#title{-}slide()}

\NormalTok{\#outline{-}slide()}

\NormalTok{= Section}

\NormalTok{== Subsection}

\NormalTok{{-} Some text}
\end{Highlighting}
\end{Shaded}

\subsection{Development}\label{development}

This template currently only follows the TU Darmstadt Beamer template in
spirit but not pixel-perfect. As the PowerPoint template uses non-free
fonts a goal of this project is to more closely match the LaTeX TU
Darmstadt Beamer 2023 template. Pull requests to improve this are really
welcome.

\begin{verbatim}
mkdir -p ~/.cache/typst/packages/preview/not-tudabeamer-2023
ln -s $PWD ~/.cache/typst/packages/preview/not-tudabeamer-2023/0.1.0
\end{verbatim}

\href{/app?template=not-tudabeamer-2023&version=0.1.0}{Create project in
app}

\subsubsection{How to use}\label{how-to-use}

Click the button above to create a new project using this template in
the Typst app.

You can also use the Typst CLI to start a new project on your computer
using this command:

\begin{verbatim}
typst init @preview/not-tudabeamer-2023:0.1.0
\end{verbatim}

\includesvg[width=0.16667in,height=0.16667in]{/assets/icons/16-copy.svg}

\subsubsection{About}\label{about}

\begin{description}
\tightlist
\item[Author :]
Moritz Hedtke
\item[License:]
Unlicense
\item[Current version:]
0.1.0
\item[Last updated:]
September 8, 2024
\item[First released:]
September 8, 2024
\item[Archive size:]
3.39 kB
\href{https://packages.typst.org/preview/not-tudabeamer-2023-0.1.0.tar.gz}{\pandocbounded{\includesvg[keepaspectratio]{/assets/icons/16-download.svg}}}
\item[Repository:]
\href{https://github.com/mohe2015/not-tudabeamer-2023}{GitHub}
\item[Categor y :]
\begin{itemize}
\tightlist
\item[]
\item
  \pandocbounded{\includesvg[keepaspectratio]{/assets/icons/16-presentation.svg}}
  \href{https://typst.app/universe/search/?category=presentation}{Presentation}
\end{itemize}
\end{description}

\subsubsection{Where to report issues?}\label{where-to-report-issues}

This template is a project of Moritz Hedtke . Report issues on
\href{https://github.com/mohe2015/not-tudabeamer-2023}{their repository}
. You can also try to ask for help with this template on the
\href{https://forum.typst.app}{Forum} .

Please report this template to the Typst team using the
\href{https://typst.app/contact}{contact form} if you believe it is a
safety hazard or infringes upon your rights.

\phantomsection\label{versions}
\subsubsection{Version history}\label{version-history}

\begin{longtable}[]{@{}ll@{}}
\toprule\noalign{}
Version & Release Date \\
\midrule\noalign{}
\endhead
\bottomrule\noalign{}
\endlastfoot
0.1.0 & September 8, 2024 \\
\end{longtable}

Typst GmbH did not create this template and cannot guarantee correct
functionality of this template or compatibility with any version of the
Typst compiler or app.


\section{Package List LaTeX/vonsim.tex}
\title{typst.app/universe/package/vonsim}

\phantomsection\label{banner}
\section{vonsim}\label{vonsim}

{ 0.1.0 }

Syntax highlighting support for VonSim.

\phantomsection\label{readme}
This package adds the ability to syntax highlighting VonSim source code
in Typst.

\subsection{How to use}\label{how-to-use}

To add global support for VonSim, just add these lines and use a raw
block with \texttt{\ vonsim\ } as its language.

\begin{Shaded}
\begin{Highlighting}[]
\NormalTok{\#import "@preview/vonsim:0.1.0": init{-}vonsim}

\NormalTok{// Adds global support for VonSim}
\NormalTok{\#show: init{-}vonsim}

\NormalTok{// Highlight VonSim code}
\NormalTok{\textasciigrave{}\textasciigrave{}\textasciigrave{}vonsim}
\NormalTok{; Welcome to VonSim!}
\NormalTok{; This is an example program that calculates the first}
\NormalTok{; n numbers of the Fibonacci sequence, and stores them}
\NormalTok{; starting at memory position 1000h.}

\NormalTok{     n  equ 10    ; Calculate the first 10 numbers}

\NormalTok{        org 1000h}
\NormalTok{start   db 1}

\NormalTok{        org 2000h}
\NormalTok{        mov bx, offset start + 1}
\NormalTok{        mov al, 0}
\NormalTok{        mov ah, start}

\NormalTok{loop:   cmp bx, offset start + n}
\NormalTok{        jns finish}
\NormalTok{        mov cl, ah}
\NormalTok{        add cl, al}
\NormalTok{        mov al, ah}
\NormalTok{        mov ah, cl}
\NormalTok{        mov [bx], cl}
\NormalTok{        inc bx}
\NormalTok{        jmp loop}
\NormalTok{finish: hlt}
\NormalTok{        end}
\NormalTok{\textasciigrave{}\textasciigrave{}\textasciigrave{}}
\end{Highlighting}
\end{Shaded}

Alternatively, use \texttt{\ init-vonsim-full\ } to also use the VonSim
theme.

\subsubsection{How to add}\label{how-to-add}

Copy this into your project and use the import as \texttt{\ vonsim\ }

\begin{verbatim}
#import "@preview/vonsim:0.1.0"
\end{verbatim}

\includesvg[width=0.16667in,height=0.16667in]{/assets/icons/16-copy.svg}

Check the docs for
\href{https://typst.app/docs/reference/scripting/\#packages}{more
information on how to import packages} .

\subsubsection{About}\label{about}

\begin{description}
\tightlist
\item[Author :]
\href{https://github.com/JuanM04}{Juan Martín Seery}
\item[License:]
AGPL-3.0-only
\item[Current version:]
0.1.0
\item[Last updated:]
June 10, 2024
\item[First released:]
June 10, 2024
\item[Minimum Typst version:]
0.11.0
\item[Archive size:]
13.8 kB
\href{https://packages.typst.org/preview/vonsim-0.1.0.tar.gz}{\pandocbounded{\includesvg[keepaspectratio]{/assets/icons/16-download.svg}}}
\item[Repository:]
\href{https://github.com/vonsim/typst-package}{GitHub}
\item[Categor y :]
\begin{itemize}
\tightlist
\item[]
\item
  \pandocbounded{\includesvg[keepaspectratio]{/assets/icons/16-text.svg}}
  \href{https://typst.app/universe/search/?category=text}{Text}
\end{itemize}
\end{description}

\subsubsection{Where to report issues?}\label{where-to-report-issues}

This package is a project of Juan Martín Seery . Report issues on
\href{https://github.com/vonsim/typst-package}{their repository} . You
can also try to ask for help with this package on the
\href{https://forum.typst.app}{Forum} .

Please report this package to the Typst team using the
\href{https://typst.app/contact}{contact form} if you believe it is a
safety hazard or infringes upon your rights.

\phantomsection\label{versions}
\subsubsection{Version history}\label{version-history}

\begin{longtable}[]{@{}ll@{}}
\toprule\noalign{}
Version & Release Date \\
\midrule\noalign{}
\endhead
\bottomrule\noalign{}
\endlastfoot
0.1.0 & June 10, 2024 \\
\end{longtable}

Typst GmbH did not create this package and cannot guarantee correct
functionality of this package or compatibility with any version of the
Typst compiler or app.


\section{Package List LaTeX/gridlock.tex}
\title{typst.app/universe/package/gridlock}

\phantomsection\label{banner}
\section{gridlock}\label{gridlock}

{ 0.2.0 }

Grid typesetting in Typst

\phantomsection\label{readme}
Grid typesetting in Typst. Use this package if you want to line up your
body text across columns and pages.

\subsection{Example}\label{example}

\pandocbounded{\includegraphics[keepaspectratio]{https://github.com/typst/packages/raw/main/packages/preview/gridlock/0.2.0/docs/assets/example-lines.png}}

\subsection{Getting Started}\label{getting-started}

\begin{Shaded}
\begin{Highlighting}[]
\NormalTok{\#import "@preview/gridlock:0.2.0": *}

\NormalTok{\#show: gridlock.with()}

\NormalTok{\#lock[= This is a heading]}

\NormalTok{\#lorem(30)}

\NormalTok{\#figure(}
\NormalTok{  placement: auto,}
\NormalTok{  caption: [a caption],}
\NormalTok{  rect()}
\NormalTok{)}

\NormalTok{\#lorem(30)}
\end{Highlighting}
\end{Shaded}

\subsection{Usage}\label{usage}

Check out
\href{https://github.com/typst/packages/raw/main/packages/preview/gridlock/0.2.0/docs/gridlock-manual.pdf}{the
manual} for a detailed description.

To get started, import the package into your document:

\begin{Shaded}
\begin{Highlighting}[]
\NormalTok{\#import "@preview/gridlock:0.2.0": *}
\end{Highlighting}
\end{Shaded}

Set up the basic parameters:

\begin{Shaded}
\begin{Highlighting}[]
\NormalTok{\#show: gridlock.with(}
\NormalTok{  paper: "a4",}
\NormalTok{  margin: (y: 76.445pt),}
\NormalTok{  font{-}size: 11pt,}
\NormalTok{  line{-}height: 13pt}
\NormalTok{)}
\end{Highlighting}
\end{Shaded}

You can now use the \texttt{\ lock()\ } function to align any block to
the text grid. Block quotes bulleted/numbered lists, and floating
figures do \emph{not} need to be wrapped in \texttt{\ lock()\ } . Their
spacing is handled fully automatically.

\begin{Shaded}
\begin{Highlighting}[]
\NormalTok{\#lock[= Heading]}

\NormalTok{\#lorem(50)}

\NormalTok{\#lock(figure(}
\NormalTok{  rect(),}
\NormalTok{  caption: [An example figure aligned to the grid.]}
\NormalTok{))}

\NormalTok{\#lorem(50)}

\NormalTok{\#lock[$ a\^{}2 = b\^{}2 + c\^{}2 $]}

\NormalTok{\#lorem(50)}
\end{Highlighting}
\end{Shaded}

\subsubsection{How to add}\label{how-to-add}

Copy this into your project and use the import as \texttt{\ gridlock\ }

\begin{verbatim}
#import "@preview/gridlock:0.2.0"
\end{verbatim}

\includesvg[width=0.16667in,height=0.16667in]{/assets/icons/16-copy.svg}

Check the docs for
\href{https://typst.app/docs/reference/scripting/\#packages}{more
information on how to import packages} .

\subsubsection{About}\label{about}

\begin{description}
\tightlist
\item[Author :]
ssotoen
\item[License:]
Unlicense
\item[Current version:]
0.2.0
\item[Last updated:]
October 21, 2024
\item[First released:]
August 8, 2024
\item[Minimum Typst version:]
0.12.0
\item[Archive size:]
4.03 kB
\href{https://packages.typst.org/preview/gridlock-0.2.0.tar.gz}{\pandocbounded{\includesvg[keepaspectratio]{/assets/icons/16-download.svg}}}
\item[Repository:]
\href{https://github.com/ssotoen/gridlock}{GitHub}
\item[Categor y :]
\begin{itemize}
\tightlist
\item[]
\item
  \pandocbounded{\includesvg[keepaspectratio]{/assets/icons/16-layout.svg}}
  \href{https://typst.app/universe/search/?category=layout}{Layout}
\end{itemize}
\end{description}

\subsubsection{Where to report issues?}\label{where-to-report-issues}

This package is a project of ssotoen . Report issues on
\href{https://github.com/ssotoen/gridlock}{their repository} . You can
also try to ask for help with this package on the
\href{https://forum.typst.app}{Forum} .

Please report this package to the Typst team using the
\href{https://typst.app/contact}{contact form} if you believe it is a
safety hazard or infringes upon your rights.

\phantomsection\label{versions}
\subsubsection{Version history}\label{version-history}

\begin{longtable}[]{@{}ll@{}}
\toprule\noalign{}
Version & Release Date \\
\midrule\noalign{}
\endhead
\bottomrule\noalign{}
\endlastfoot
0.2.0 & October 21, 2024 \\
\href{https://typst.app/universe/package/gridlock/0.1.0/}{0.1.0} &
August 8, 2024 \\
\end{longtable}

Typst GmbH did not create this package and cannot guarantee correct
functionality of this package or compatibility with any version of the
Typst compiler or app.


\section{Package List LaTeX/valkyrie.tex}
\title{typst.app/universe/package/valkyrie}

\phantomsection\label{banner}
\section{valkyrie}\label{valkyrie}

{ 0.2.1 }

Type safe type validation

\phantomsection\label{readme}
Version 0.2.1

This package implements type validation, and is targeted mainly at
package and template developers. The desired outcome is that it becomes
easier for the programmer to quickly put a package together without
spending a long time on type safety, but also to make the usage of those
packages by end-users less painful by generating useful error messages.

\subsection{Example Usage}\label{example-usage}

\begin{Shaded}
\begin{Highlighting}[]
\NormalTok{\#import "@preview/valkyrie:0.2.1" as z}

\NormalTok{\#let my{-}schema = z.dictionary((}
\NormalTok{    should{-}be{-}string: z.string(),}
\NormalTok{    complicated{-}tuple: z.tuple(}
\NormalTok{      z.email(),}
\NormalTok{      z.ip(),}
\NormalTok{      z.either(}
\NormalTok{        z.string(),}
\NormalTok{        z.number(),}
\NormalTok{      ),}
\NormalTok{    ),}
\NormalTok{  )}
\NormalTok{)}

\NormalTok{\#z.parse(}
\NormalTok{  (}
\NormalTok{    should{-}be{-}string: "This doesn\textquotesingle{}t error",}
\NormalTok{    complicated{-}tuple: (}
\NormalTok{      "neither@does{-}this.com",}
\NormalTok{      // Error: Schema validation failed on argument.complicated{-}tuple.1: }
\NormalTok{      //        String must be a valid IP address}
\NormalTok{      "NOT AN IP",}
\NormalTok{      1,}
\NormalTok{    ),}
\NormalTok{  ),}
\NormalTok{  my{-}schema,}
\NormalTok{)}
\end{Highlighting}
\end{Shaded}

\subsection{Community-lead}\label{community-lead}

As of version 0.2.0, \texttt{\ valkyrie\ } now resides in the
typst-community organisation. Typst users are encouraged to submit
additional types, assertions, coercions, and schemas that they believe
are already used widely, or should be widely adopted for the health of
the ecosystem.

\subsubsection{How to add}\label{how-to-add}

Copy this into your project and use the import as \texttt{\ valkyrie\ }

\begin{verbatim}
#import "@preview/valkyrie:0.2.1"
\end{verbatim}

\includesvg[width=0.16667in,height=0.16667in]{/assets/icons/16-copy.svg}

Check the docs for
\href{https://typst.app/docs/reference/scripting/\#packages}{more
information on how to import packages} .

\subsubsection{About}\label{about}

\begin{description}
\tightlist
\item[Author s :]
James R. Swift \& \href{mailto:me@tinger.dev}{tinger}
\item[License:]
MIT
\item[Current version:]
0.2.1
\item[Last updated:]
July 15, 2024
\item[First released:]
November 19, 2023
\item[Archive size:]
9.99 kB
\href{https://packages.typst.org/preview/valkyrie-0.2.1.tar.gz}{\pandocbounded{\includesvg[keepaspectratio]{/assets/icons/16-download.svg}}}
\item[Repository:]
\href{https://github.com/typst-community/valkyrie}{GitHub}
\item[Categor ies :]
\begin{itemize}
\tightlist
\item[]
\item
  \pandocbounded{\includesvg[keepaspectratio]{/assets/icons/16-code.svg}}
  \href{https://typst.app/universe/search/?category=scripting}{Scripting}
\item
  \pandocbounded{\includesvg[keepaspectratio]{/assets/icons/16-hammer.svg}}
  \href{https://typst.app/universe/search/?category=utility}{Utility}
\end{itemize}
\end{description}

\subsubsection{Where to report issues?}\label{where-to-report-issues}

This package is a project of James R. Swift and tinger . Report issues
on \href{https://github.com/typst-community/valkyrie}{their repository}
. You can also try to ask for help with this package on the
\href{https://forum.typst.app}{Forum} .

Please report this package to the Typst team using the
\href{https://typst.app/contact}{contact form} if you believe it is a
safety hazard or infringes upon your rights.

\phantomsection\label{versions}
\subsubsection{Version history}\label{version-history}

\begin{longtable}[]{@{}ll@{}}
\toprule\noalign{}
Version & Release Date \\
\midrule\noalign{}
\endhead
\bottomrule\noalign{}
\endlastfoot
0.2.1 & July 15, 2024 \\
\href{https://typst.app/universe/package/valkyrie/0.2.0/}{0.2.0} & May
27, 2024 \\
\href{https://typst.app/universe/package/valkyrie/0.1.1/}{0.1.1} & April
29, 2024 \\
\href{https://typst.app/universe/package/valkyrie/0.1.0/}{0.1.0} &
November 19, 2023 \\
\end{longtable}

Typst GmbH did not create this package and cannot guarantee correct
functionality of this package or compatibility with any version of the
Typst compiler or app.


\section{Package List LaTeX/touying-aqua.tex}
\title{typst.app/universe/package/touying-aqua}

\phantomsection\label{banner}
\phantomsection\label{template-thumbnail}
\pandocbounded{\includegraphics[keepaspectratio]{https://packages.typst.org/preview/thumbnails/touying-aqua-0.5.3-small.webp}}

\section{touying-aqua}\label{touying-aqua}

{ 0.5.3 }

Aqua theme in Touying - a powerful package for creating presentation
slides in Typst.

\href{/app?template=touying-aqua&version=0.5.3}{Create project in app}

\phantomsection\label{readme}
\href{https://github.com/touying-typ/touying}{Touying} (投影 in
chinese, /tóuyÇ?ng/, meaning projection) is a user-friendly, powerful
and efficient package for creating presentation slides in Typst. Partial
code is inherited from
\href{https://github.com/andreasKroepelin/polylux}{Polylux} . Therefore,
some concepts and APIs remain consistent with Polylux.

Touying provides automatically injected global configurations, which is
convenient for configuring themes. Besides, Touying does not rely on
\texttt{\ counter\ } and \texttt{\ context\ } to implement
\texttt{\ \#pause\ } , resulting in better performance.

If you like it, consider
\href{https://github.com/touying-typ/touying}{giving a star on GitHub} .
Touying is a community-driven project, feel free to suggest any ideas
and contribute.

\href{https://touying-typ.github.io/}{\pandocbounded{\includegraphics[keepaspectratio]{https://img.shields.io/badge/docs-book-green}}}
\href{https://github.com/touying-typ/touying/wiki}{\pandocbounded{\includegraphics[keepaspectratio]{https://img.shields.io/badge/docs-gallery-orange}}}
\pandocbounded{\includegraphics[keepaspectratio]{https://img.shields.io/github/license/touying-typ/touying}}
\pandocbounded{\includegraphics[keepaspectratio]{https://img.shields.io/github/v/release/touying-typ/touying}}
\pandocbounded{\includegraphics[keepaspectratio]{https://img.shields.io/github/stars/touying-typ/touying}}
\pandocbounded{\includegraphics[keepaspectratio]{https://img.shields.io/badge/themes-6-aqua}}

\subsection{Document}\label{document}

Read \href{https://touying-typ.github.io/}{the document} to learn all
about Touying.

We will maintain \textbf{English} and \textbf{Chinese} versions of the
documentation for Touying, and for each major version, we will maintain
a documentation copy. This allows you to easily refer to old versions of
the Touying documentation and migrate to new versions.

\textbf{Note that the documentation may be outdated, and you can also
use Tinymist to view Touying’s annotated documentation by hovering
over the code.}

\subsection{Gallery}\label{gallery}

Touying offers \href{https://github.com/touying-typ/touying/wiki}{a
gallery page} via wiki, where you can browse elegant slides created by
Touying users. You’re also encouraged to contribute your own beautiful
slides here!

\subsection{Special Features}\label{special-features}

\begin{enumerate}
\tightlist
\item
  Split slides by headings
  \href{https://touying-typ.github.io/docs/sections}{document}
\end{enumerate}

\begin{Shaded}
\begin{Highlighting}[]
\NormalTok{= Section}

\NormalTok{== Subsection}

\NormalTok{=== First Slide}

\NormalTok{Hello, Touying!}

\NormalTok{=== Second Slide}

\NormalTok{Hello, Typst!}
\end{Highlighting}
\end{Shaded}

\begin{enumerate}
\setcounter{enumi}{1}
\tightlist
\item
  \texttt{\ \#pause\ } and \texttt{\ \#meanwhile\ } animations
  \href{https://touying-typ.github.io/docs/dynamic/simple}{document}
\end{enumerate}

\begin{Shaded}
\begin{Highlighting}[]
\NormalTok{\#slide[}
\NormalTok{  First}

\NormalTok{  \#pause}

\NormalTok{  Second}

\NormalTok{  \#meanwhile}

\NormalTok{  Third}

\NormalTok{  \#pause}

\NormalTok{  Fourth}
\NormalTok{]}
\end{Highlighting}
\end{Shaded}

\pandocbounded{\includegraphics[keepaspectratio]{https://github.com/touying-typ/touying/assets/34951714/24ca19a3-b27c-4d31-ab75-09c37911e6ac}}

\begin{enumerate}
\setcounter{enumi}{2}
\tightlist
\item
  Math Equation Animation
  \href{https://touying-typ.github.io/docs/dynamic/equation}{document}
\end{enumerate}

\pandocbounded{\includegraphics[keepaspectratio]{https://github.com/touying-typ/touying/assets/34951714/8640fe0a-95e4-46ac-b570-c8c79f993de4}}

\begin{enumerate}
\setcounter{enumi}{3}
\tightlist
\item
  \texttt{\ touying-reducer\ } Cetz and Fletcher Animations
  \href{https://touying-typ.github.io/docs/dynamic/other}{document}
\end{enumerate}

\pandocbounded{\includegraphics[keepaspectratio]{https://github.com/touying-typ/touying/assets/34951714/9ba71f54-2a5d-4144-996c-4a42833cc5cc}}

\begin{enumerate}
\setcounter{enumi}{4}
\tightlist
\item
  Correct outline and bookmark (no duplicate and correct page number)
\end{enumerate}

\pandocbounded{\includegraphics[keepaspectratio]{https://github.com/touying-typ/touying/assets/34951714/7b62fcaf-6342-4dba-901b-818c16682529}}

\begin{enumerate}
\setcounter{enumi}{5}
\tightlist
\item
  Dewdrop Theme Navigation Bar
  \href{https://touying-typ.github.io/docs/themes/dewdrop}{document}
\end{enumerate}

\pandocbounded{\includegraphics[keepaspectratio]{https://github.com/touying-typ/touying/assets/34951714/0426516d-aa3c-4b7a-b7b6-2d5d276fb971}}

\begin{enumerate}
\setcounter{enumi}{6}
\tightlist
\item
  Semi-transparent cover mode
  \href{https://touying-typ.github.io/docs/dynamic/cover}{document}
\end{enumerate}

\pandocbounded{\includegraphics[keepaspectratio]{https://github.com/touying-typ/touying/assets/34951714/22a9ea66-c8b5-431e-a52c-2c8ca3f18e49}}

\begin{enumerate}
\setcounter{enumi}{7}
\tightlist
\item
  Speaker notes for dual-screen
  \href{https://touying-typ.github.io/docs/external/pympress}{document}
\end{enumerate}

\pandocbounded{\includegraphics[keepaspectratio]{https://github.com/touying-typ/touying/assets/34951714/afbe17cb-46d4-4507-90e8-959c53de95d5}}

\begin{enumerate}
\setcounter{enumi}{8}
\tightlist
\item
  Export slides to PPTX and HTML formats and show presentation online.
  \href{https://github.com/touying-typ/touying-exporter}{touying-exporter}
  \href{https://github.com/touying-typ/touying-template}{touying-template}
  \href{https://touying-typ.github.io/touying-template/}{online}
\end{enumerate}

\pandocbounded{\includegraphics[keepaspectratio]{https://github.com/touying-typ/touying-exporter/assets/34951714/207ddffc-87c8-4976-9bf4-4c6c5e2573ea}}

\subsection{Quick start}\label{quick-start}

Before you begin, make sure you have installed the Typst environment. If
not, you can use the \href{https://typst.app/}{Web App} or the
\href{https://marketplace.visualstudio.com/items?itemName=myriad-dreamin.tinymist}{Tinymist
LSP} extensions for VS Code.

To use Touying, you only need to include the following code in your
document:

\begin{Shaded}
\begin{Highlighting}[]
\NormalTok{\#import "@preview/touying:0.5.3": *}
\NormalTok{\#import themes.simple: *}

\NormalTok{\#show: simple{-}theme.with(aspect{-}ratio: "16{-}9")}

\NormalTok{= Title}

\NormalTok{== First Slide}

\NormalTok{Hello, Touying!}

\NormalTok{\#pause}

\NormalTok{Hello, Typst!}
\end{Highlighting}
\end{Shaded}

\pandocbounded{\includegraphics[keepaspectratio]{https://github.com/touying-typ/touying/assets/34951714/f5bdbf8f-7bf9-45fd-9923-0fa5d66450b2}}

It’s simple. Congratulations on creating your first Touying slide!
🎉

\textbf{Tip:} You can use Typst syntax like
\texttt{\ \#import\ "config.typ":\ *\ } or
\texttt{\ \#include\ "content.typ"\ } to implement Touying’s
multi-file architecture.

\subsection{More Complex Examples}\label{more-complex-examples}

In fact, Touying provides various styles for writing slides. For
example, the above example uses first-level and second-level titles to
create new slides. However, you can also use the
\texttt{\ \#slide{[}..{]}\ } format to access more powerful features
provided by Touying.

\begin{Shaded}
\begin{Highlighting}[]
\NormalTok{\#import "@preview/touying:0.5.3": *}
\NormalTok{\#import themes.university: *}
\NormalTok{\#import "@preview/cetz:0.2.2"}
\NormalTok{\#import "@preview/fletcher:0.5.1" as fletcher: node, edge}
\NormalTok{\#import "@preview/ctheorems:1.1.2": *}
\NormalTok{\#import "@preview/numbly:0.1.0": numbly}

\NormalTok{// cetz and fletcher bindings for touying}
\NormalTok{\#let cetz{-}canvas = touying{-}reducer.with(reduce: cetz.canvas, cover: cetz.draw.hide.with(bounds: true))}
\NormalTok{\#let fletcher{-}diagram = touying{-}reducer.with(reduce: fletcher.diagram, cover: fletcher.hide)}

\NormalTok{// Theorems configuration by ctheorems}
\NormalTok{\#show: thmrules.with(qed{-}symbol: $square$)}
\NormalTok{\#let theorem = thmbox("theorem", "Theorem", fill: rgb("\#eeffee"))}
\NormalTok{\#let corollary = thmplain(}
\NormalTok{  "corollary",}
\NormalTok{  "Corollary",}
\NormalTok{  base: "theorem",}
\NormalTok{  titlefmt: strong}
\NormalTok{)}
\NormalTok{\#let definition = thmbox("definition", "Definition", inset: (x: 1.2em, top: 1em))}
\NormalTok{\#let example = thmplain("example", "Example").with(numbering: none)}
\NormalTok{\#let proof = thmproof("proof", "Proof")}

\NormalTok{\#show: university{-}theme.with(}
\NormalTok{  aspect{-}ratio: "16{-}9",}
\NormalTok{  // config{-}common(handout: true),}
\NormalTok{  config{-}info(}
\NormalTok{    title: [Title],}
\NormalTok{    subtitle: [Subtitle],}
\NormalTok{    author: [Authors],}
\NormalTok{    date: datetime.today(),}
\NormalTok{    institution: [Institution],}
\NormalTok{    logo: emoji.school,}
\NormalTok{  ),}
\NormalTok{)}

\NormalTok{\#set heading(numbering: numbly("\{1\}.", default: "1.1"))}

\NormalTok{\#title{-}slide()}

\NormalTok{== Outline \textless{}touying:hidden\textgreater{}}

\NormalTok{\#components.adaptive{-}columns(outline(title: none, indent: 1em))}

\NormalTok{= Animation}

\NormalTok{== Simple Animation}

\NormalTok{We can use \textasciigrave{}\#pause\textasciigrave{} to \#pause display something later.}

\NormalTok{\#pause}

\NormalTok{Just like this.}

\NormalTok{\#meanwhile}

\NormalTok{Meanwhile, \#pause we can also use \textasciigrave{}\#meanwhile\textasciigrave{} to \#pause display other content synchronously.}

\NormalTok{\#speaker{-}note[}
\NormalTok{  + This is a speaker note.}
\NormalTok{  + You won\textquotesingle{}t see it unless you use \textasciigrave{}config{-}common(show{-}notes{-}on{-}second{-}screen: right)\textasciigrave{}}
\NormalTok{]}


\NormalTok{== Complex Animation}

\NormalTok{At subslide \#touying{-}fn{-}wrapper((self: none) =\textgreater{} str(self.subslide)), we can}

\NormalTok{use \#uncover("2{-}")[\textasciigrave{}\#uncover\textasciigrave{} function] for reserving space,}

\NormalTok{use \#only("2{-}")[\textasciigrave{}\#only\textasciigrave{} function] for not reserving space,}

\NormalTok{\#alternatives[call \textasciigrave{}\#only\textasciigrave{} multiple times \textbackslash{}u\{2717\}][use \textasciigrave{}\#alternatives\textasciigrave{} function \#sym.checkmark] for choosing one of the alternatives.}


\NormalTok{== Callback Style Animation}

\NormalTok{\#slide(repeat: 3, self =\textgreater{} [}
\NormalTok{  \#let (uncover, only, alternatives) = utils.methods(self)}

\NormalTok{  At subslide \#self.subslide, we can}

\NormalTok{  use \#uncover("2{-}")[\textasciigrave{}\#uncover\textasciigrave{} function] for reserving space,}

\NormalTok{  use \#only("2{-}")[\textasciigrave{}\#only\textasciigrave{} function] for not reserving space,}

\NormalTok{  \#alternatives[call \textasciigrave{}\#only\textasciigrave{} multiple times \textbackslash{}u\{2717\}][use \textasciigrave{}\#alternatives\textasciigrave{} function \#sym.checkmark] for choosing one of the alternatives.}
\NormalTok{])}


\NormalTok{== Math Equation Animation}

\NormalTok{Equation with \textasciigrave{}pause\textasciigrave{}:}

\NormalTok{$}
\NormalTok{  f(x) \&= pause x\^{}2 + 2x + 1  \textbackslash{}}
\NormalTok{       \&= pause (x + 1)\^{}2  \textbackslash{}}
\NormalTok{$}

\NormalTok{\#meanwhile}

\NormalTok{Here, \#pause we have the expression of $f(x)$.}

\NormalTok{\#pause}

\NormalTok{By factorizing, we can obtain this result.}


\NormalTok{== CeTZ Animation}

\NormalTok{CeTZ Animation in Touying:}

\NormalTok{\#cetz{-}canvas(\{}
\NormalTok{  import cetz.draw: *}
  
\NormalTok{  rect((0,0), (5,5))}

\NormalTok{  (pause,)}

\NormalTok{  rect((0,0), (1,1))}
\NormalTok{  rect((1,1), (2,2))}
\NormalTok{  rect((2,2), (3,3))}

\NormalTok{  (pause,)}

\NormalTok{  line((0,0), (2.5, 2.5), name: "line")}
\NormalTok{\})}


\NormalTok{== Fletcher Animation}

\NormalTok{Fletcher Animation in Touying:}

\NormalTok{\#fletcher{-}diagram(}
\NormalTok{  node{-}stroke: .1em,}
\NormalTok{  node{-}fill: gradient.radial(blue.lighten(80\%), blue, center: (30\%, 20\%), radius: 80\%),}
\NormalTok{  spacing: 4em,}
\NormalTok{  edge(({-}1,0), "r", "{-}|\textgreater{}", \textasciigrave{}open(path)\textasciigrave{}, label{-}pos: 0, label{-}side: center),}
\NormalTok{  node((0,0), \textasciigrave{}reading\textasciigrave{}, radius: 2em),}
\NormalTok{  edge((0,0), (0,0), \textasciigrave{}read()\textasciigrave{}, "{-}{-}|\textgreater{}", bend: 130deg),}
\NormalTok{  pause,}
\NormalTok{  edge(\textasciigrave{}read()\textasciigrave{}, "{-}|\textgreater{}"),}
\NormalTok{  node((1,0), \textasciigrave{}eof\textasciigrave{}, radius: 2em),}
\NormalTok{  pause,}
\NormalTok{  edge(\textasciigrave{}close()\textasciigrave{}, "{-}|\textgreater{}"),}
\NormalTok{  node((2,0), \textasciigrave{}closed\textasciigrave{}, radius: 2em, extrude: ({-}2.5, 0)),}
\NormalTok{  edge((0,0), (2,0), \textasciigrave{}close()\textasciigrave{}, "{-}|\textgreater{}", bend: {-}40deg),}
\NormalTok{)}


\NormalTok{= Theorems}

\NormalTok{== Prime numbers}

\NormalTok{\#definition[}
\NormalTok{  A natural number is called a \#highlight[\_prime number\_] if it is greater}
\NormalTok{  than 1 and cannot be written as the product of two smaller natural numbers.}
\NormalTok{]}
\NormalTok{\#example[}
\NormalTok{  The numbers $2$, $3$, and $17$ are prime.}
\NormalTok{  @cor\_largest\_prime shows that this list is not exhaustive!}
\NormalTok{]}

\NormalTok{\#theorem("Euclid")[}
\NormalTok{  There are infinitely many primes.}
\NormalTok{]}
\NormalTok{\#proof[}
\NormalTok{  Suppose to the contrary that $p\_1, p\_2, dots, p\_n$ is a finite enumeration}
\NormalTok{  of all primes. Set $P = p\_1 p\_2 dots p\_n$. Since $P + 1$ is not in our list,}
\NormalTok{  it cannot be prime. Thus, some prime factor $p\_j$ divides $P + 1$.  Since}
\NormalTok{  $p\_j$ also divides $P$, it must divide the difference $(P + 1) {-} P = 1$, a}
\NormalTok{  contradiction.}
\NormalTok{]}

\NormalTok{\#corollary[}
\NormalTok{  There is no largest prime number.}
\NormalTok{] \textless{}cor\_largest\_prime\textgreater{}}
\NormalTok{\#corollary[}
\NormalTok{  There are infinitely many composite numbers.}
\NormalTok{]}

\NormalTok{\#theorem[}
\NormalTok{  There are arbitrarily long stretches of composite numbers.}
\NormalTok{]}

\NormalTok{\#proof[}
\NormalTok{  For any $n \textgreater{} 2$, consider $}
\NormalTok{    n! + 2, quad n! + 3, quad ..., quad n! + n \#qedhere}
\NormalTok{  $}
\NormalTok{]}


\NormalTok{= Others}

\NormalTok{== Side{-}by{-}side}

\NormalTok{\#slide(composer: (1fr, 1fr))[}
\NormalTok{  First column.}
\NormalTok{][}
\NormalTok{  Second column.}
\NormalTok{]}


\NormalTok{== Multiple Pages}

\NormalTok{\#lorem(200)}


\NormalTok{\#show: appendix}

\NormalTok{= Appendix}

\NormalTok{== Appendix}

\NormalTok{Please pay attention to the current slide number.}
\end{Highlighting}
\end{Shaded}

\pandocbounded{\includegraphics[keepaspectratio]{https://github.com/user-attachments/assets/3488f256-a0b3-43d0-a266-009d9d0a7bd3}}

\subsection{Acknowledgements}\label{acknowledgements}

Thanks to…

\begin{itemize}
\tightlist
\item
  \href{https://github.com/andreasKroepelin}{@andreasKroepelin} for the
  \texttt{\ polylux\ } package
\item
  \href{https://github.com/Enivex}{@Enivex} for the
  \texttt{\ metropolis\ } theme
\item
  \href{https://github.com/drupol}{@drupol} for the
  \texttt{\ university\ } theme
\item
  \href{https://github.com/pride7}{@pride7} for the \texttt{\ aqua\ }
  theme
\item
  \href{https://github.com/Coekjan}{@Coekjan} and
  \href{https://github.com/QuadnucYard}{@QuadnucYard} for the
  \texttt{\ stargazer\ } theme
\item
  \href{https://github.com/ntjess}{@ntjess} for contributing to
  \texttt{\ fit-to-height\ } , \texttt{\ fit-to-width\ } and
  \texttt{\ cover-with-rect\ }
\end{itemize}

\subsection{Poster}\label{poster}

\pandocbounded{\includegraphics[keepaspectratio]{https://github.com/user-attachments/assets/e1ddb672-8e8f-472d-b364-b8caed1da16b}}

\href{https://github.com/touying-typ/touying-poster}{View Code}

\subsection{Star History}\label{star-history}

\href{https://star-history.com/\#touying-typ/touying&Date}{\pandocbounded{\includegraphics[keepaspectratio]{https://api.star-history.com/svg?repos=touying-typ/touying&type=Date}}}

\href{/app?template=touying-aqua&version=0.5.3}{Create project in app}

\subsubsection{How to use}\label{how-to-use}

Click the button above to create a new project using this template in
the Typst app.

You can also use the Typst CLI to start a new project on your computer
using this command:

\begin{verbatim}
typst init @preview/touying-aqua:0.5.3
\end{verbatim}

\includesvg[width=0.16667in,height=0.16667in]{/assets/icons/16-copy.svg}

\subsubsection{About}\label{about}

\begin{description}
\tightlist
\item[Author s :]
OrangeX4 , Andreas Kröpelin , ntjess , Enivex , Pol Dellaiera , \&
pride7
\item[License:]
MIT
\item[Current version:]
0.5.3
\item[Last updated:]
October 15, 2024
\item[First released:]
March 26, 2024
\item[Archive size:]
5.09 kB
\href{https://packages.typst.org/preview/touying-aqua-0.5.3.tar.gz}{\pandocbounded{\includesvg[keepaspectratio]{/assets/icons/16-download.svg}}}
\item[Repository:]
\href{https://github.com/touying-typ/touying}{GitHub}
\item[Categor y :]
\begin{itemize}
\tightlist
\item[]
\item
  \pandocbounded{\includesvg[keepaspectratio]{/assets/icons/16-presentation.svg}}
  \href{https://typst.app/universe/search/?category=presentation}{Presentation}
\end{itemize}
\end{description}

\subsubsection{Where to report issues?}\label{where-to-report-issues}

This template is a project of OrangeX4, Andreas Kröpelin, ntjess,
Enivex, Pol Dellaiera, and pride7 . Report issues on
\href{https://github.com/touying-typ/touying}{their repository} . You
can also try to ask for help with this template on the
\href{https://forum.typst.app}{Forum} .

Please report this template to the Typst team using the
\href{https://typst.app/contact}{contact form} if you believe it is a
safety hazard or infringes upon your rights.

\phantomsection\label{versions}
\subsubsection{Version history}\label{version-history}

\begin{longtable}[]{@{}ll@{}}
\toprule\noalign{}
Version & Release Date \\
\midrule\noalign{}
\endhead
\bottomrule\noalign{}
\endlastfoot
0.5.3 & October 15, 2024 \\
\href{https://typst.app/universe/package/touying-aqua/0.5.2/}{0.5.2} &
September 3, 2024 \\
\href{https://typst.app/universe/package/touying-aqua/0.5.1/}{0.5.1} &
September 3, 2024 \\
\href{https://typst.app/universe/package/touying-aqua/0.5.0/}{0.5.0} &
September 2, 2024 \\
\href{https://typst.app/universe/package/touying-aqua/0.4.2/}{0.4.2} &
May 27, 2024 \\
\href{https://typst.app/universe/package/touying-aqua/0.4.1/}{0.4.1} &
May 13, 2024 \\
\href{https://typst.app/universe/package/touying-aqua/0.4.0/}{0.4.0} &
April 6, 2024 \\
\href{https://typst.app/universe/package/touying-aqua/0.3.3/}{0.3.3} &
March 26, 2024 \\
\end{longtable}

Typst GmbH did not create this template and cannot guarantee correct
functionality of this template or compatibility with any version of the
Typst compiler or app.


\section{Package List LaTeX/touying-flow.tex}
\title{typst.app/universe/package/touying-flow}

\phantomsection\label{banner}
\phantomsection\label{template-thumbnail}
\pandocbounded{\includegraphics[keepaspectratio]{https://packages.typst.org/preview/thumbnails/touying-flow-1.0.0-small.webp}}

\section{touying-flow}\label{touying-flow}

{ 1.0.0 }

Discard irrelevant decorative elements, aiming to better immerse the
audience into a state of flow.

\href{/app?template=touying-flow&version=1.0.0}{Create project in app}

\phantomsection\label{readme}
Discard irrelevant decorative elements, aiming to better immerse the
audience into a state of flow.

Inspired by \href{https://github.com/touying-typ/touying.git}{Dewdrop} ,
made by \href{https://github.com/OrangeX4}{OrangeX4}

A \href{https://github.com/typst/typst}{Typst} template created based on
\href{https://github.com/touying-typ/touying}{Touying} , designed for
academic presentations in university settings.

\subsection{Example}\label{example}

See
\href{https://github.com/typst/packages/raw/main/packages/preview/touying-flow/1.0.0/main.pdf}{main.pdf}
for a sample PDF output. While the project is already complete, the
example content is still under development.

\subsection{Installation}\label{installation}

These steps assume that you already have
\href{https://typst.app/}{Typst} installed and running. If not, please
refer to
\href{https://github.com/typst/typst/releases/}{github.com/typst/typst/releases/}
for installation instructions.Alternatively, you can use VS Code for
editing by installing the Tinymist Typst extension ( \emph{recommended}
).

\subsubsection{Import from Typst
Universe}\label{import-from-typst-universe}

\begin{Shaded}
\begin{Highlighting}[]
\NormalTok{\#import "@preview/touying{-}flow:1.0.0":*}

\NormalTok{\#show: dewdrop{-}theme.with(}
\NormalTok{  aspect{-}ratio: "16{-}9",}
\NormalTok{  footer: self =\textgreater{} self.info.title,}
\NormalTok{  footer{-}alt: self =\textgreater{} self.info.subtitle,}
\NormalTok{  navigation: "mini{-}slides",}
\NormalTok{  config{-}info(}
\NormalTok{    title: [Title],}
\NormalTok{    subtitle: [Subtitle],}
\NormalTok{    author: [Quaternijkon],}
\NormalTok{    date: datetime.today(),}
\NormalTok{    institution: [USTC],}
\NormalTok{  ),}
\NormalTok{)}

\NormalTok{\#let primary= rgb("\#004098")}

\NormalTok{\#show :show{-}cn{-}fakebold}
\NormalTok{\#show outline.entry.where(}
\NormalTok{  level: 1}
\NormalTok{): it =\textgreater{} \{}
\NormalTok{  v(1em, weak: true)}
\NormalTok{  text(primary, it.body)}
\NormalTok{\}}
\NormalTok{\#show emph: it =\textgreater{} \{  }
\NormalTok{  underline(stroke: (thickness: 1em, paint: primary.transparentize(95\%), cap: "round"),offset: {-}7pt,background: true,evade: false,extent: {-}8pt,text(primary, it.body))}
\NormalTok{\}}

\NormalTok{\#title{-}slide()}

\NormalTok{= Example Section Title}

\NormalTok{== Example Page Title}
\end{Highlighting}
\end{Shaded}

\href{/app?template=touying-flow&version=1.0.0}{Create project in app}

\subsubsection{How to use}\label{how-to-use}

Click the button above to create a new project using this template in
the Typst app.

You can also use the Typst CLI to start a new project on your computer
using this command:

\begin{verbatim}
typst init @preview/touying-flow:1.0.0
\end{verbatim}

\includesvg[width=0.16667in,height=0.16667in]{/assets/icons/16-copy.svg}

\subsubsection{About}\label{about}

\begin{description}
\tightlist
\item[Author :]
\href{https://github.com/Quaternijkon}{Quaternijkon}
\item[License:]
MIT
\item[Current version:]
1.0.0
\item[Last updated:]
November 26, 2024
\item[First released:]
November 26, 2024
\item[Minimum Typst version:]
0.11.0
\item[Archive size:]
12.2 kB
\href{https://packages.typst.org/preview/touying-flow-1.0.0.tar.gz}{\pandocbounded{\includesvg[keepaspectratio]{/assets/icons/16-download.svg}}}
\item[Repository:]
\href{https://github.com/Quaternijkon/Typst_FLOW}{GitHub}
\item[Categor y :]
\begin{itemize}
\tightlist
\item[]
\item
  \pandocbounded{\includesvg[keepaspectratio]{/assets/icons/16-presentation.svg}}
  \href{https://typst.app/universe/search/?category=presentation}{Presentation}
\end{itemize}
\end{description}

\subsubsection{Where to report issues?}\label{where-to-report-issues}

This template is a project of Quaternijkon . Report issues on
\href{https://github.com/Quaternijkon/Typst_FLOW}{their repository} .
You can also try to ask for help with this template on the
\href{https://forum.typst.app}{Forum} .

Please report this template to the Typst team using the
\href{https://typst.app/contact}{contact form} if you believe it is a
safety hazard or infringes upon your rights.

\phantomsection\label{versions}
\subsubsection{Version history}\label{version-history}

\begin{longtable}[]{@{}ll@{}}
\toprule\noalign{}
Version & Release Date \\
\midrule\noalign{}
\endhead
\bottomrule\noalign{}
\endlastfoot
1.0.0 & November 26, 2024 \\
\end{longtable}

Typst GmbH did not create this template and cannot guarantee correct
functionality of this template or compatibility with any version of the
Typst compiler or app.


\section{Package List LaTeX/shadowed.tex}
\title{typst.app/universe/package/shadowed}

\phantomsection\label{banner}
\section{shadowed}\label{shadowed}

{ 0.1.2 }

Box shadows for Typst

\phantomsection\label{readme}
Box shadows for \href{https://typst.app/}{Typst} .

\subsection{Usage}\label{usage}

\begin{Shaded}
\begin{Highlighting}[]
\NormalTok{\#import "@preview/shadowed:0.1.2": shadowed}

\NormalTok{\#set par(justify: true)}

\NormalTok{\#shadowed(radius: 4pt, inset: 12pt)[}
\NormalTok{    \#lorem(50)}
\NormalTok{]}
\end{Highlighting}
\end{Shaded}

\pandocbounded{\includegraphics[keepaspectratio]{https://github.com/typst/packages/raw/main/packages/preview/shadowed/0.1.2/examples/lorem.png}}

\subsection{Reference}\label{reference}

The \texttt{\ shadowed\ } function takes the following arguments:

\begin{itemize}
\tightlist
\item
  \textbf{blur: Length} - The blur radius of the shadow. Also adds a
  padding of the same size.
\item
  \textbf{radius: Length} - The corner radius of the inner block and
  shadow.
\item
  \textbf{color: Color} - The color of the shadow.
\item
  \textbf{inset: Length} - The inset of the inner block.
\item
  \textbf{fill: Color} - The color of the inner block.
\end{itemize}

\subsection{Credits}\label{credits}

This project was inspired by
\href{https://github.com/typst-community/harbinger}{Harbinger} .

\subsubsection{How to add}\label{how-to-add}

Copy this into your project and use the import as \texttt{\ shadowed\ }

\begin{verbatim}
#import "@preview/shadowed:0.1.2"
\end{verbatim}

\includesvg[width=0.16667in,height=0.16667in]{/assets/icons/16-copy.svg}

Check the docs for
\href{https://typst.app/docs/reference/scripting/\#packages}{more
information on how to import packages} .

\subsubsection{About}\label{about}

\begin{description}
\tightlist
\item[Author :]
\href{https://github.com/T1mVo}{Tim Voßhenrich}
\item[License:]
MIT
\item[Current version:]
0.1.2
\item[Last updated:]
October 22, 2024
\item[First released:]
September 22, 2024
\item[Minimum Typst version:]
0.12.0
\item[Archive size:]
32.8 kB
\href{https://packages.typst.org/preview/shadowed-0.1.2.tar.gz}{\pandocbounded{\includesvg[keepaspectratio]{/assets/icons/16-download.svg}}}
\item[Repository:]
\href{https://github.com/T1mVo/shadowed}{GitHub}
\item[Categor y :]
\begin{itemize}
\tightlist
\item[]
\item
  \pandocbounded{\includesvg[keepaspectratio]{/assets/icons/16-package.svg}}
  \href{https://typst.app/universe/search/?category=components}{Components}
\end{itemize}
\end{description}

\subsubsection{Where to report issues?}\label{where-to-report-issues}

This package is a project of Tim Voßhenrich . Report issues on
\href{https://github.com/T1mVo/shadowed}{their repository} . You can
also try to ask for help with this package on the
\href{https://forum.typst.app}{Forum} .

Please report this package to the Typst team using the
\href{https://typst.app/contact}{contact form} if you believe it is a
safety hazard or infringes upon your rights.

\phantomsection\label{versions}
\subsubsection{Version history}\label{version-history}

\begin{longtable}[]{@{}ll@{}}
\toprule\noalign{}
Version & Release Date \\
\midrule\noalign{}
\endhead
\bottomrule\noalign{}
\endlastfoot
0.1.2 & October 22, 2024 \\
\href{https://typst.app/universe/package/shadowed/0.1.1/}{0.1.1} &
October 21, 2024 \\
\href{https://typst.app/universe/package/shadowed/0.1.0/}{0.1.0} &
September 22, 2024 \\
\end{longtable}

Typst GmbH did not create this package and cannot guarantee correct
functionality of this package or compatibility with any version of the
Typst compiler or app.


\section{Package List LaTeX/ttt-utils.tex}
\title{typst.app/universe/package/ttt-utils}

\phantomsection\label{banner}
\section{ttt-utils}\label{ttt-utils}

{ 0.1.2 }

A collection of tools to make a teachers life easier.

\phantomsection\label{readme}
\texttt{\ ttt-utils\ } is the core package of the
\href{https://github.com/jomaway/typst-teacher-templates}{typst-teacher-tools
collection} .

\subsection{Modules}\label{modules}

It contains several modules:

\begin{itemize}
\tightlist
\item
  \texttt{\ assignments\ } contains functions for creating exams.
\item
  \texttt{\ components\ } contains useful visual components such
  \emph{lines} or \emph{caro pattern} , \emph{tags} , etc …
\item
  \texttt{\ grading\ } contains functions for grading exams.
\item
  \texttt{\ helpers\ } contains some utility functions.
\item
  \texttt{\ layout\ } contains some layout functions such as
  \emph{side-by-side} , etc…
\item
  \texttt{\ random\ } contains a function to shuffle an array.
\end{itemize}

\subsection{Usage}\label{usage}

You can import the modules you need with:

\begin{Shaded}
\begin{Highlighting}[]
\NormalTok{\#import "@preview/ttt{-}utils:0.1.0": components}
\end{Highlighting}
\end{Shaded}

then you can access the modules function with:

\texttt{\ \#components.lines(4)\ } or \texttt{\ \#components.caro(5)\ }
, …

or import the wanted functions:

\begin{Shaded}
\begin{Highlighting}[]
\NormalTok{\#import "@preview/ttt{-}utils:0.1.0": components, assignments}

\NormalTok{\#import assignments: assignment, question, answer}
\NormalTok{\#import components: caro as grid\_pattern}

\NormalTok{// Add a question.}

\NormalTok{\#assignment[First assignment}

\NormalTok{    \#question[}
\NormalTok{        \#answer(field: grid\_pattern(5))}
\NormalTok{    ]}
\NormalTok{]}
\end{Highlighting}
\end{Shaded}

\subsection{Similar projects}\label{similar-projects}

\begin{itemize}
\tightlist
\item
  \href{https://github.com/SillyFreak/typst-packages/tree/main/scrutinize}{scrutinize}
  by \href{https://github.com/SillyFreak}{SillyFreak} : Package to
  create exams, very similar to the \texttt{\ assignment\ } module, but
  only questions without assignments, and a bit more low level. I
  adopted a few of his ideas.
\end{itemize}

\subsection{CHANGELOG}\label{changelog}

See
\href{https://github.com/typst/packages/raw/main/packages/preview/ttt-utils/CHANGELOG.md}{CHANGELOG.md}

\subsubsection{How to add}\label{how-to-add}

Copy this into your project and use the import as \texttt{\ ttt-utils\ }

\begin{verbatim}
#import "@preview/ttt-utils:0.1.2"
\end{verbatim}

\includesvg[width=0.16667in,height=0.16667in]{/assets/icons/16-copy.svg}

Check the docs for
\href{https://typst.app/docs/reference/scripting/\#packages}{more
information on how to import packages} .

\subsubsection{About}\label{about}

\begin{description}
\tightlist
\item[Author :]
\href{https://github.com/jomaway}{Jomaway}
\item[License:]
MIT
\item[Current version:]
0.1.2
\item[Last updated:]
May 22, 2024
\item[First released:]
March 26, 2024
\item[Minimum Typst version:]
0.11.0
\item[Archive size:]
7.60 kB
\href{https://packages.typst.org/preview/ttt-utils-0.1.2.tar.gz}{\pandocbounded{\includesvg[keepaspectratio]{/assets/icons/16-download.svg}}}
\item[Repository:]
\href{https://github.com/jomaway/typst-teacher-templates}{GitHub}
\item[Discipline :]
\begin{itemize}
\tightlist
\item[]
\item
  \href{https://typst.app/universe/search/?discipline=education}{Education}
\end{itemize}
\item[Categor ies :]
\begin{itemize}
\tightlist
\item[]
\item
  \pandocbounded{\includesvg[keepaspectratio]{/assets/icons/16-package.svg}}
  \href{https://typst.app/universe/search/?category=components}{Components}
\item
  \pandocbounded{\includesvg[keepaspectratio]{/assets/icons/16-hammer.svg}}
  \href{https://typst.app/universe/search/?category=utility}{Utility}
\end{itemize}
\end{description}

\subsubsection{Where to report issues?}\label{where-to-report-issues}

This package is a project of Jomaway . Report issues on
\href{https://github.com/jomaway/typst-teacher-templates}{their
repository} . You can also try to ask for help with this package on the
\href{https://forum.typst.app}{Forum} .

Please report this package to the Typst team using the
\href{https://typst.app/contact}{contact form} if you believe it is a
safety hazard or infringes upon your rights.

\phantomsection\label{versions}
\subsubsection{Version history}\label{version-history}

\begin{longtable}[]{@{}ll@{}}
\toprule\noalign{}
Version & Release Date \\
\midrule\noalign{}
\endhead
\bottomrule\noalign{}
\endlastfoot
0.1.2 & May 22, 2024 \\
\href{https://typst.app/universe/package/ttt-utils/0.1.0/}{0.1.0} &
March 26, 2024 \\
\end{longtable}

Typst GmbH did not create this package and cannot guarantee correct
functionality of this package or compatibility with any version of the
Typst compiler or app.


\section{Package List LaTeX/caidan.tex}
\title{typst.app/universe/package/caidan}

\phantomsection\label{banner}
\phantomsection\label{template-thumbnail}
\pandocbounded{\includegraphics[keepaspectratio]{https://packages.typst.org/preview/thumbnails/caidan-0.1.0-small.webp}}

\section{caidan}\label{caidan}

{ 0.1.0 }

A clean and minimal food menu template

\href{/app?template=caidan&version=0.1.0}{Create project in app}

\phantomsection\label{readme}
Caidan (è?œå?• in Chinese, /cÃ~i dÄ?n/, meaning food menu) is a clean
and minimal food menu template.

See the
\href{https://github.com/cu1ch3n/caidan/blob/main/example.pdf}{example.pdf}
file to see how it looks. Additionally,
\href{https://github.com/cu1ch3n/menu}{cu1ch3n/menu} serves as a
practical example project utilizing this template.

\subsection{Usage}\label{usage}

Ensure that
\href{http://www.galapagosdesign.com/original/webomints.htm}{WebOMints
GD} , \href{https://github.com/lxgw/LxgwWenKai}{LXGW WenKai} , and
\href{https://fonts.google.com/specimen/Ysabeau+Infant}{Ysabeau Infant}
fonts are installed first. The required fonts are provided in
\href{https://github.com/cu1ch3n/caidan/tree/main/fonts}{fonts} .

To use this template with typst.app, you may upload the required fonts
manually ( \textbf{Note} :
\href{https://github.com/lxgw/LxgwWenKai}{LXGW WenKai} may be too large
to upload onto typst.app).

\subsection{Configuration}\label{configuration}

This template includes the \texttt{\ caidan\ } function, which comes
with several configurable named arguments:

\begin{longtable}[]{@{}llll@{}}
\toprule\noalign{}
Argument & Default Value & Type & Description \\
\midrule\noalign{}
\endhead
\bottomrule\noalign{}
\endlastfoot
\texttt{\ title\ } & \texttt{\ none\ } &
\href{https://typst.app/docs/reference/foundations/content/}{content} &
The title for your menu \\
\texttt{\ cover\_image\ } & \texttt{\ none\ } &
\href{https://typst.app/docs/reference/foundations/content/}{content} &
The image on the menu’s cover page \\
\texttt{\ update\_date\ } & \texttt{\ none\ } &
\href{https://typst.app/docs/reference/foundations/datetime/}{datetime}
& This date will be displayed on the cover page in both Chinese and
English \\
\texttt{\ page\_height\ } & \texttt{\ 595.28pt\ } &
\href{https://typst.app/docs/reference/layout/length/}{length} & Page
height of your menu \\
\texttt{\ page\_width\ } & \texttt{\ 841.89pt\ } &
\href{https://typst.app/docs/reference/layout/length/}{length} & Page
width of your menu \\
\texttt{\ num\_columns\ } & \texttt{\ 3\ } &
\href{https://typst.app/docs/reference/foundations/int/}{int} & The
number of columns per page \\
\end{longtable}

The function also accepts a single, positional argument for the body.

\subsection{Example}\label{example}

\begin{Shaded}
\begin{Highlighting}[]
\NormalTok{\#import "@preview/caidan:0.1.0": *}

\NormalTok{\#show: caidan.with(}
\NormalTok{  title: [\#en\_text(22pt, fill: nord0)[Chen\textquotesingle{}s Private Cuisine]],}
\NormalTok{  cover\_image: image("cover.png"),}
\NormalTok{  update\_date: datetime.today(),}
\NormalTok{  num\_columns: 3,}
\NormalTok{)}

\NormalTok{\#cuisine[鲁菜][Shandong Cuisine]}
\NormalTok{{-} \#item[葱烧海参][Braised Sea Cucumber w/ Scallions]}
\NormalTok{{-} \#item[葱爆牛肉][Scallion Beef Stir{-}Fry]}
\NormalTok{{-} \#item[醋溜白菜][Napa Cabbage Stir{-}Fry w/ Vinegar]}

\NormalTok{\#cuisine[川菜][Sichuan Cuisine]}
\NormalTok{{-} \#item[宫保鸡丁][Gong Bao Chicken]}
\NormalTok{{-} \#item[回锅肉][Twice{-}cooked pork]}
\NormalTok{{-} \#item[麻婆豆腐][Mapo Tofu]}
\end{Highlighting}
\end{Shaded}

\href{/app?template=caidan&version=0.1.0}{Create project in app}

\subsubsection{How to use}\label{how-to-use}

Click the button above to create a new project using this template in
the Typst app.

You can also use the Typst CLI to start a new project on your computer
using this command:

\begin{verbatim}
typst init @preview/caidan:0.1.0
\end{verbatim}

\includesvg[width=0.16667in,height=0.16667in]{/assets/icons/16-copy.svg}

\subsubsection{About}\label{about}

\begin{description}
\tightlist
\item[Author :]
\href{https://github.com/cu1ch3n}{Chen Cui}
\item[License:]
MIT
\item[Current version:]
0.1.0
\item[Last updated:]
April 3, 2024
\item[First released:]
April 3, 2024
\item[Minimum Typst version:]
0.11.0
\item[Archive size:]
360 kB
\href{https://packages.typst.org/preview/caidan-0.1.0.tar.gz}{\pandocbounded{\includesvg[keepaspectratio]{/assets/icons/16-download.svg}}}
\item[Repository:]
\href{https://github.com/cu1ch3n/caidan}{GitHub}
\item[Categor y :]
\begin{itemize}
\tightlist
\item[]
\item
  \pandocbounded{\includesvg[keepaspectratio]{/assets/icons/16-map.svg}}
  \href{https://typst.app/universe/search/?category=flyer}{Flyer}
\end{itemize}
\end{description}

\subsubsection{Where to report issues?}\label{where-to-report-issues}

This template is a project of Chen Cui . Report issues on
\href{https://github.com/cu1ch3n/caidan}{their repository} . You can
also try to ask for help with this template on the
\href{https://forum.typst.app}{Forum} .

Please report this template to the Typst team using the
\href{https://typst.app/contact}{contact form} if you believe it is a
safety hazard or infringes upon your rights.

\phantomsection\label{versions}
\subsubsection{Version history}\label{version-history}

\begin{longtable}[]{@{}ll@{}}
\toprule\noalign{}
Version & Release Date \\
\midrule\noalign{}
\endhead
\bottomrule\noalign{}
\endlastfoot
0.1.0 & April 3, 2024 \\
\end{longtable}

Typst GmbH did not create this template and cannot guarantee correct
functionality of this template or compatibility with any version of the
Typst compiler or app.


\section{Package List LaTeX/oasis-align.tex}
\title{typst.app/universe/package/oasis-align}

\phantomsection\label{banner}
\section{oasis-align}\label{oasis-align}

{ 0.1.0 }

Cleanly place content side by side with equal heights using automatic
content sizing.

\phantomsection\label{readme}
\texttt{\ oasis-align\ } is a package that automatically sizes your
content so that their heights are equal, allowing you to cleanly place
content side by side.

To use \texttt{\ oasis-align\ } in your document, start by importing the
package like this:

\begin{Shaded}
\begin{Highlighting}[]
\NormalTok{\#import "@preview/oasis{-}align:0.1.0": *}
\end{Highlighting}
\end{Shaded}

This will give you access to the two functions found under
\href{https://github.com/typst/packages/raw/main/packages/preview/oasis-align/0.1.0/\#configuration}{configurations}
.

\subsection{Image with Text}\label{image-with-text}

\pandocbounded{\includegraphics[keepaspectratio]{https://github.com/typst/packages/raw/main/packages/preview/oasis-align/0.1.0/examples/image-with-text.gif}}

\subsection{Image with Image}\label{image-with-image}

\pandocbounded{\includegraphics[keepaspectratio]{https://github.com/typst/packages/raw/main/packages/preview/oasis-align/0.1.0/examples/image-with-image.gif}}

\subsection{Text with Text}\label{text-with-text}

\pandocbounded{\includegraphics[keepaspectratio]{https://github.com/typst/packages/raw/main/packages/preview/oasis-align/0.1.0/examples/text-with-text.gif}}

There are two functions associated with this package. The first is
specifically targeted at
\href{https://github.com/typst/packages/raw/main/packages/preview/oasis-align/0.1.0/\#oasis-align-images}{aligning
images} , and the second is targeted at
\href{https://github.com/typst/packages/raw/main/packages/preview/oasis-align/0.1.0/\#oasis-align-1}{content
in general} .

\begin{quote}
{[}!important{]} To change the size of the gutter in both functions, use
\texttt{\ \#set\ grid(column-gutter:\ length)\ } . This is necessary to
allow for fixed rules that aren’t possible with user-defined
functions.
\end{quote}

\subsection{\texorpdfstring{\texttt{\ oasis-align-images\ }}{ oasis-align-images }}\label{oasis-align-images}

Use this function to align two images.

\begin{Shaded}
\begin{Highlighting}[]
\NormalTok{\#oasis{-}align{-}images(}
\NormalTok{    "path/to/image1",}
\NormalTok{    "path/to/image2"}
\NormalTok{)}
\end{Highlighting}
\end{Shaded}

\begin{quote}
{[}!tip{]} Whenever aligning \textbf{only} images, it’s best to use
this function instead of the default \texttt{\ oasis-align\ } . \emph{To
learn more about why, check out
\href{https://github.com/typst/packages/raw/main/packages/preview/oasis-align/0.1.0/\#how-it-works}{how
it works} .}
\end{quote}

\subsection{\texorpdfstring{\texttt{\ oasis-align\ }}{ oasis-align }}\label{oasis-align-1}

Use this function to align content like text with other content like
images or figures.

\begin{quote}
{[}!tip{]} The parameters with defined values are the defaults and do
not need to be included unless desired.
\end{quote}

\begin{Shaded}
\begin{Highlighting}[]
\NormalTok{\#oasis{-}align(}
\NormalTok{  int{-}frac: 0.5,        // decimal between 0 and 1}
\NormalTok{  tolerance: 0.001pt,   // length}
\NormalTok{  max{-}iterations: 50,   // integer greater than 0}
\NormalTok{  int{-}dir: 1,           // 1 or {-}1}
\NormalTok{  debug: false          // boolean}
\NormalTok{  item1,                // content}
\NormalTok{  item2,                // content}
\NormalTok{)}
\end{Highlighting}
\end{Shaded}

\subsubsection{\texorpdfstring{\texttt{\ int-frac\ }}{ int-frac }}\label{int-frac}

The starting point of the search process. Changing this value may reduce
the total number of iterations of the function or find an
\href{https://github.com/typst/packages/raw/main/packages/preview/oasis-align/0.1.0/\#oasis-align-2}{alternate
solution} .

\subsubsection{\texorpdfstring{\texttt{\ tolerance\ }}{ tolerance }}\label{tolerance}

The allowable difference in heights between \texttt{\ item1\ } and
\texttt{\ item2\ } . The function will run until it has reached either
this \texttt{\ tolerance\ } or \texttt{\ max-iterations\ } . Making
\texttt{\ tolerance\ } larger may reduce the total number of iterations
but result in a larger height difference between pieces of content.

\begin{quote}
{[}!note{]} Two pieces of content may not always be able to achieve the
desired \texttt{\ tolerance\ } . In that case, the function sizes the
content to the iteration that had the least difference in height.
\emph{Check out
\href{https://github.com/typst/packages/raw/main/packages/preview/oasis-align/0.1.0/\#oasis-align-2}{how
it works} to understand why the function may not be able achieve the
desired \texttt{\ tolerance\ } .}
\end{quote}

\subsubsection{\texorpdfstring{\texttt{\ max-iterations\ }}{ max-iterations }}\label{max-iterations}

The maximum number of iterations the function is allowed to attempt
before terminating. Increasing this number may allow you to achieve a
smaller \texttt{\ tolerance\ } .

\subsubsection{\texorpdfstring{\texttt{\ int-dir\ }}{ int-dir }}\label{int-dir}

The initial direction that the dividing fraction is moved. Changing this
value will change the initial direction.

\begin{quote}
{[}!note{]} The program is hardcoded to switch directions if a solution
is not found in the initial direction. This parameter mainly serves to
let you easily choose between
\href{https://github.com/typst/packages/raw/main/packages/preview/oasis-align/0.1.0/\#oasis-align-2}{multiple
solutions} .
\end{quote}

\subsubsection{\texorpdfstring{\texttt{\ debug\ }}{ debug }}\label{debug}

A toggle that lets you look inside the function to see what is
happening. This is useful if you would like to understand why certain
content may be incompatible and which of the parameters above could be
changed to resolve the issue.

\subsection{\texorpdfstring{\texttt{\ oasis-align-images\ }}{ oasis-align-images }}\label{oasis-align-images-1}

The function begins by determining the width and height of the selected
images. These values can then be used to solve a set of linear
equations, the first of which states that the sum of the widths of the
images (plus the gutter) should be equal to the available horizontal
space, and the second which states that their heights should be equal.

If \$w\_1\$ and \$h\_1\$ are the width and height of \texttt{\ image1\ }
and \$w\_2\$ and \$h\_2\$ are the width and height of
\texttt{\ image2\ } , then the final width \$w\_1’\$ of
\texttt{\ image1\ } and the final width \$w\_2’\$ of
\texttt{\ image2\ } are

\$\$w\_1’ = \textbackslash left(\textbackslash frac\{h\_1 w\_2\}\{w\_1
h\_2\} + 1 \textbackslash right)\^{}\{-1\} \textbackslash qquad w\_2’
= \textbackslash left(\textbackslash frac\{w\_1 h\_2\}\{h\_1 w\_2\} + 1
\textbackslash right)\^{}\{-1\}\$\$

\subsection{\texorpdfstring{\texttt{\ oasis-align\ }}{ oasis-align }}\label{oasis-align-2}

Originally designed to allow for an image to be placed side-by-side with
text, this function takes an iterative approach to aligning the content.
When changing the width of a block of text, the height does not scale
linearly, but instead behaves as a step function that follows an
exponential trend (the graph below has a simplified visualization of
this). This prevents the use of an analytical methodology similar to the
one used in \texttt{\ oasis-align-images\ } , and thus must be solved
using an iterative approach.

The function starts by taking the available space and then splitting it
using the \texttt{\ int-frac\ } . The content is then placed in a block
with the width as determined using the split from \texttt{\ int-frac\ }
before measuring its height. Based on the \texttt{\ int-dir\ } , the
split will be moved left or right using the bisection method until a
solution within the \texttt{\ tolerance\ } has been found. In the case
that a solution within the \texttt{\ tolerance\ } is not found with the
\texttt{\ max-iterations\ } , the program terminates and uses the
container width fraction that had the smallest difference in height.

\pandocbounded{\includesvg[keepaspectratio]{https://github.com/typst/packages/raw/main/packages/preview/oasis-align/0.1.0/examples/graph-visualization.svg}}

\subsubsection{Multiple Solutions (1st
Graph)}\label{multiple-solutions-1st-graph}

Depending on the type of content, the function may find multiple
solutions. The parameters \texttt{\ int-dir\ } and \texttt{\ int-frac\ }
will allow you to choose between them by changing the direction in which
it iterates and changing the starting container width fraction
respectively.

\subsubsection{No Solutions (2nd Graph)}\label{no-solutions-2nd-graph}

There are cases in which the text size is incompatible with an image.
This can be because there is not enough or too much text, and regardless
of how the content is resized, their heights do not match.

\subsubsection{Tolerance Not Reached (3rd
Graph)}\label{tolerance-not-reached-3rd-graph}

In the case of having texts of different sizes (as seen in
\href{https://github.com/typst/packages/raw/main/packages/preview/oasis-align/0.1.0/\#text-with-text}{the
examples} ), the spacing between lines prevents the function from
finding a solution that meets the \texttt{\ tolerance\ } , and thus the
closest solution is used.

If you end up using this package, please feel free to share how you used
it under “Discussions� on the
\href{https://github.com/jdpieck/oasis-align}{GitHub Repository} or on
Discord with \texttt{\ @jdpieck\ } .

\subsubsection{How to add}\label{how-to-add}

Copy this into your project and use the import as
\texttt{\ oasis-align\ }

\begin{verbatim}
#import "@preview/oasis-align:0.1.0"
\end{verbatim}

\includesvg[width=0.16667in,height=0.16667in]{/assets/icons/16-copy.svg}

Check the docs for
\href{https://typst.app/docs/reference/scripting/\#packages}{more
information on how to import packages} .

\subsubsection{About}\label{about}

\begin{description}
\tightlist
\item[Author :]
@jdpieck
\item[License:]
MIT
\item[Current version:]
0.1.0
\item[Last updated:]
September 2, 2024
\item[First released:]
September 2, 2024
\item[Archive size:]
4.96 kB
\href{https://packages.typst.org/preview/oasis-align-0.1.0.tar.gz}{\pandocbounded{\includesvg[keepaspectratio]{/assets/icons/16-download.svg}}}
\item[Repository:]
\href{https://github.com/jdpieck/oasis-align}{GitHub}
\item[Categor y :]
\begin{itemize}
\tightlist
\item[]
\item
  \pandocbounded{\includesvg[keepaspectratio]{/assets/icons/16-layout.svg}}
  \href{https://typst.app/universe/search/?category=layout}{Layout}
\end{itemize}
\end{description}

\subsubsection{Where to report issues?}\label{where-to-report-issues}

This package is a project of @jdpieck . Report issues on
\href{https://github.com/jdpieck/oasis-align}{their repository} . You
can also try to ask for help with this package on the
\href{https://forum.typst.app}{Forum} .

Please report this package to the Typst team using the
\href{https://typst.app/contact}{contact form} if you believe it is a
safety hazard or infringes upon your rights.

\phantomsection\label{versions}
\subsubsection{Version history}\label{version-history}

\begin{longtable}[]{@{}ll@{}}
\toprule\noalign{}
Version & Release Date \\
\midrule\noalign{}
\endhead
\bottomrule\noalign{}
\endlastfoot
0.1.0 & September 2, 2024 \\
\end{longtable}

Typst GmbH did not create this package and cannot guarantee correct
functionality of this package or compatibility with any version of the
Typst compiler or app.


\section{Package List LaTeX/modern-bnu-thesis.tex}
\title{typst.app/universe/package/modern-bnu-thesis}

\phantomsection\label{banner}
\phantomsection\label{template-thumbnail}
\pandocbounded{\includegraphics[keepaspectratio]{https://packages.typst.org/preview/thumbnails/modern-bnu-thesis-0.0.1-small.webp}}

\section{modern-bnu-thesis}\label{modern-bnu-thesis}

{ 0.0.1 }

åŒ---京师范大学学ä½?论æ--‡æ¨¡æ?¿ã€‚Modern Beijing Normal University
Thesis.

\href{/app?template=modern-bnu-thesis&version=0.0.1}{Create project in
app}

\phantomsection\label{readme}
åŒ---京师范大学毕业论æ--‡ï¼ˆè®¾è®¡ï¼‰çš„ Typst
模æ?¿ï¼Œèƒ½å¤Ÿç®€æ´?ã€?快速ã€?æŒ?ç»­ç''Ÿæˆ? PDF
æ~¼å¼?的毕业论æ--‡ã€‚
\href{https://typst.app/universe/package/modern-bnu-thesis}{Typst
Universe}

\begin{quote}
{[}!tip{]} 本模�是基于
\href{https://github.com/nju-lug/modern-nju-thesis/tree/main}{å?---京大学的typst模æ?¿}
进行的二次开å?{}`,ä»\ldots 对ç\ldots§BNU人工智能学院指定的
\href{https://www.overleaf.com/latex/templates/thesis-template-of-beijing-normal-university-in-latex/nrjdjsvfxhms}{latex模�}
对细节进行适é\ldots?,相å\ldots³ä½¿ç''¨æ--¹æ³•å?¯ä»¥å?‚è§?å?---京大学的项目
\end{quote}

\subsection{致谢}\label{uxe8uxe8}

\begin{itemize}
\tightlist
\item
  æ„Ÿè°¢ \href{https://github.com/OrangeX4}{OrangeX4} å¼€å?{}`çš„
  \href{https://github.com/nju-lug/modern-nju-thesis}{modern-nju-thesis}
  Typst
  模æ?¿ï¼Œæœ¬æ¨¡æ?¿åŸºæœ¬ä¸Šæ˜¯åœ¨å¤§ä½¬çš„项目上进行å°?æ''¹åŠ¨ã€‚
\item
  æ„Ÿè°¢ BNU AI 学院æ??供的 Latex 模æ?¿
  \href{https://www.overleaf.com/latex/templates/thesis-template-of-beijing-normal-university-in-latex/nrjdjsvfxhms}{Thesis
  Template of Beijing Normal University in LaTeX}
  ,本模æ?¿æ~¼å¼?完å\ldots¨å?‚考这个latex模æ?¿ã€‚
\item
  æ„Ÿè°¢ \href{https://typst-doc-cn.github.io/tutorial/}{å°?è``?书 (The
  Raindrop-Blue Book)} �
  \href{https://sitandr.github.io/typst-examples-book/book/}{Typst
  Examples Book} ç­‰Typstå­¦ä¹~资æ--™ã€‚
\end{itemize}

\subsection{License}\label{license}

This project is licensed under the MIT License.

\href{/app?template=modern-bnu-thesis&version=0.0.1}{Create project in
app}

\subsubsection{How to use}\label{how-to-use}

Click the button above to create a new project using this template in
the Typst app.

You can also use the Typst CLI to start a new project on your computer
using this command:

\begin{verbatim}
typst init @preview/modern-bnu-thesis:0.0.1
\end{verbatim}

\includesvg[width=0.16667in,height=0.16667in]{/assets/icons/16-copy.svg}

\subsubsection{About}\label{about}

\begin{description}
\tightlist
\item[Author :]
MosRat
\item[License:]
MIT
\item[Current version:]
0.0.1
\item[Last updated:]
November 21, 2024
\item[First released:]
November 21, 2024
\item[Archive size:]
221 kB
\href{https://packages.typst.org/preview/modern-bnu-thesis-0.0.1.tar.gz}{\pandocbounded{\includesvg[keepaspectratio]{/assets/icons/16-download.svg}}}
\item[Repository:]
\href{https://github.com/mosrat/modern-bnu-thesis}{GitHub}
\item[Categor y :]
\begin{itemize}
\tightlist
\item[]
\item
  \pandocbounded{\includesvg[keepaspectratio]{/assets/icons/16-mortarboard.svg}}
  \href{https://typst.app/universe/search/?category=thesis}{Thesis}
\end{itemize}
\end{description}

\subsubsection{Where to report issues?}\label{where-to-report-issues}

This template is a project of MosRat . Report issues on
\href{https://github.com/mosrat/modern-bnu-thesis}{their repository} .
You can also try to ask for help with this template on the
\href{https://forum.typst.app}{Forum} .

Please report this template to the Typst team using the
\href{https://typst.app/contact}{contact form} if you believe it is a
safety hazard or infringes upon your rights.

\phantomsection\label{versions}
\subsubsection{Version history}\label{version-history}

\begin{longtable}[]{@{}ll@{}}
\toprule\noalign{}
Version & Release Date \\
\midrule\noalign{}
\endhead
\bottomrule\noalign{}
\endlastfoot
0.0.1 & November 21, 2024 \\
\end{longtable}

Typst GmbH did not create this template and cannot guarantee correct
functionality of this template or compatibility with any version of the
Typst compiler or app.


\section{Package List LaTeX/examit.tex}
\title{typst.app/universe/package/examit}

\phantomsection\label{banner}
\section{examit}\label{examit}

{ 0.1.1 }

An exam template based on the MIT LaTeX exam.cls

\phantomsection\label{readme}
A Typst exam package based on the MIT LaTeX
\href{https://ctan.org/pkg/exam}{exam} package

\subsection{Features}\label{features}

\begin{itemize}
\tightlist
\item
  Title block
\item
  Read questions from file/question bank
\item
  Grading table
\item
  Marking boxes
\item
  Question types

  \begin{itemize}
  \tightlist
  \item
    Standard answerline
  \item
    Multiple choice (including true-false)
  \item
    Writing box
  \item
    Blank rectangular, polar, numberline graphs
  \end{itemize}
\end{itemize}

\subsection{Example}\label{example}

\texttt{\ main.typ\ }

\begin{Shaded}
\begin{Highlighting}[]
\NormalTok{\#import "@preview/examit:0.1.1": *}
\NormalTok{\#import "questions.typ": questions}

\NormalTok{\#show: examit.with(}
\NormalTok{  questions: questions, // questions file, see example}

\NormalTok{  title: [examit],}
\NormalTok{  subtitle: "A Typst Exam Package",}
\NormalTok{  date: "2023{-}03{-}21",}

\NormalTok{  margin: (}
\NormalTok{    left: 18mm,}
\NormalTok{    top: 16mm,}
\NormalTok{    bottom: 25mm,}
\NormalTok{    right: 25mm,}
\NormalTok{  ),}
\NormalTok{  cols: 2,}
\NormalTok{  gutter: 18mm,}
\NormalTok{  lang: "en",}
\NormalTok{  font: "New Computer Modern",}

\NormalTok{  extrapicturebox: true, // "If you have time..." box at the end}
\NormalTok{  // dropallboxes: true, // points boxes next to answerlines instead of level with the question}
\NormalTok{  instructions: [Instructions before exam columns.],}
\NormalTok{  namebox: "left",}
\NormalTok{  pointsplacement: "right",}
\NormalTok{  answerlinelength: 4cm,}
\NormalTok{  defaultpoints: 1,}
\NormalTok{)}
\end{Highlighting}
\end{Shaded}

\texttt{\ questions.typ\ }

\begin{Shaded}
\begin{Highlighting}[]
\NormalTok{\#import "@preview/examit:0.1.1": *}

\NormalTok{\#let questions = (}
\NormalTok{  ( header: [Multiple Choice] ),}
\NormalTok{  (}
\NormalTok{    question: [What attributes \#underline("must") a *vector* have?],}
\NormalTok{    points: 2,}
\NormalTok{    choices: (}
\NormalTok{      [position],}
\NormalTok{      [magnitude],}
\NormalTok{      [direction],}
\NormalTok{      [x{-} and y{-}coordinates],}
\NormalTok{      [height],}
\NormalTok{      [width],}
\NormalTok{    ),}
\NormalTok{    horizontal: false,}
\NormalTok{    sameline: false,}
\NormalTok{  ),}
\NormalTok{  (}
\NormalTok{    question: [$bold(sin\^{}({-}1))$ returns an angle in which quadrants?],}
\NormalTok{    choices: ([I],[II],[III],[IV],),}
\NormalTok{    sameline: false,}
\NormalTok{    points: 2,}
\NormalTok{  ),}
\NormalTok{  (}
\NormalTok{    question: [A *scalar* is a vector with a\textbackslash{} magnitude of *1*.],}
\NormalTok{    tf: true,}
\NormalTok{    points: 2,}
\NormalTok{  ),}
\NormalTok{  (}
\NormalTok{    question: [}
\NormalTok{      Write this polar equation in rectangular form: $r = frac(5 ,cos theta + sin theta )$}
\NormalTok{    ],}
\NormalTok{    points: 4,}
\NormalTok{    bonus: true,}
\NormalTok{    answerbox: 3cm,}
\NormalTok{  ),}
\NormalTok{  ( pagebreak: true ),}
\NormalTok{  ( header: [Graphing]),}
\NormalTok{  (}
\NormalTok{    question: [}
\NormalTok{      Simplify and graph the complex number\textbackslash{} $5(cos 15 degree + i sin 15 degree) dot 10(cos 5 degree + i sin 5 degree)$.}
\NormalTok{    ],}
\NormalTok{    points: 3,}
\NormalTok{    spacing: 2.5cm,}
\NormalTok{    graph: "rect",}
\NormalTok{    answerline: true,}
\NormalTok{  ),}
\NormalTok{  (}
\NormalTok{    question: [$x\^{}2(x\^{}2+9)\textgreater{}6x\^{}3$],}
\NormalTok{    points: 2,}
\NormalTok{    bonus: true,}
\NormalTok{    numberline: 2.5in,}
\NormalTok{  ),}
\NormalTok{  (subheader: [A child is pulling a wagon with a force of 15 lb at an angle of 35° to the ground.}
\NormalTok{  Gravity is pulling down on the wagon with a force of 12 lb.]),}
\NormalTok{  (}
\NormalTok{    question: [}
\NormalTok{      What is the resulting force vector?}
\NormalTok{    ],}
\NormalTok{    points: 4,}
\NormalTok{    spacing: 3cm,}
\NormalTok{    label: "wagon"}
\NormalTok{  ),}
\NormalTok{  (}
\NormalTok{    question: [}
\NormalTok{      This question references \textbackslash{}\#@wagon.}
\NormalTok{    ],}
\NormalTok{    points: 4,}
\NormalTok{  ),}
\NormalTok{)}
\end{Highlighting}
\end{Shaded}

\pandocbounded{\includegraphics[keepaspectratio]{https://github.com/onomou/typst-examit/assets/131693/78ba6fdc-59c0-460a-89cc-9617c15ac3e0}}

\subsection{To Do}\label{to-do}

\begin{itemize}
\tightlist
\item
  {[} {]} Parts or sub-questions
\item
  {[} {]} Customize numbering for questions
\item
  {[}X{]} References for other questions
\item
  {[} {]} Customize marking box properties: size, positioning, style
\item
  {[} {]} Better multiple-choice and matching options: box/bubble style,
  layout arrangement (horizontal/vertical, alignment)
\item
  {[} {]} Question types: fill-in-the-blank
\item
  {[} {]} Grading table options: bonus points, positioning
\item
  {[} {]} Footer options
\item
  {[} {]} Margin adjustments based on points position
\item
  {[} {]} Size options for graph response
\item
  {[} {]} Customize first page or title block
\item
  {[} {]} Configure even/odd headers/footers
\item
  {[} {]} Show/hide answers
\item
  {[} {]} Page break vs column break?
\end{itemize}

\subsubsection{How to add}\label{how-to-add}

Copy this into your project and use the import as \texttt{\ examit\ }

\begin{verbatim}
#import "@preview/examit:0.1.1"
\end{verbatim}

\includesvg[width=0.16667in,height=0.16667in]{/assets/icons/16-copy.svg}

Check the docs for
\href{https://typst.app/docs/reference/scripting/\#packages}{more
information on how to import packages} .

\subsubsection{About}\label{about}

\begin{description}
\tightlist
\item[Author :]
Steven Williams
\item[License:]
MIT
\item[Current version:]
0.1.1
\item[Last updated:]
May 16, 2024
\item[First released:]
May 16, 2024
\item[Archive size:]
6.80 kB
\href{https://packages.typst.org/preview/examit-0.1.1.tar.gz}{\pandocbounded{\includesvg[keepaspectratio]{/assets/icons/16-download.svg}}}
\item[Discipline :]
\begin{itemize}
\tightlist
\item[]
\item
  \href{https://typst.app/universe/search/?discipline=education}{Education}
\end{itemize}
\item[Categor ies :]
\begin{itemize}
\tightlist
\item[]
\item
  \pandocbounded{\includesvg[keepaspectratio]{/assets/icons/16-package.svg}}
  \href{https://typst.app/universe/search/?category=components}{Components}
\item
  \pandocbounded{\includesvg[keepaspectratio]{/assets/icons/16-layout.svg}}
  \href{https://typst.app/universe/search/?category=layout}{Layout}
\item
  \pandocbounded{\includesvg[keepaspectratio]{/assets/icons/16-speak.svg}}
  \href{https://typst.app/universe/search/?category=report}{Report}
\end{itemize}
\end{description}

\subsubsection{Where to report issues?}\label{where-to-report-issues}

This package is a project of Steven Williams . You can also try to ask
for help with this package on the \href{https://forum.typst.app}{Forum}
.

Please report this package to the Typst team using the
\href{https://typst.app/contact}{contact form} if you believe it is a
safety hazard or infringes upon your rights.

\phantomsection\label{versions}
\subsubsection{Version history}\label{version-history}

\begin{longtable}[]{@{}ll@{}}
\toprule\noalign{}
Version & Release Date \\
\midrule\noalign{}
\endhead
\bottomrule\noalign{}
\endlastfoot
0.1.1 & May 16, 2024 \\
\end{longtable}

Typst GmbH did not create this package and cannot guarantee correct
functionality of this package or compatibility with any version of the
Typst compiler or app.


\section{Package List LaTeX/tablex.tex}
\title{typst.app/universe/package/tablex}

\phantomsection\label{banner}
\section{tablex}\label{tablex}

{ 0.0.9 }

More powerful and customizable tables in Typst.

\phantomsection\label{readme}
\textbf{More powerful and customizable tables in Typst.}

\subsection{Sponsors ��}\label{sponsors-uxe2uxef}

If you’d like to appear here,
\href{https://github.com/sponsors/PgBiel}{consider sponsoring the
project!}

\href{https://github.com/felipeacsi}{\includegraphics[width=0.52083in,height=\textheight,keepaspectratio]{https://github.com/felipeacsi.png}}
\href{https://github.com/Fabioni}{\includegraphics[width=0.52083in,height=\textheight,keepaspectratio]{https://github.com/Fabioni.png}}

\subsection{Important notice regarding Tablex
usage}\label{important-notice-regarding-tablex-usage}

\textbf{Summary: Please use built-in Typst tables instead of tablex.}
Most of tablex’s features were implemented in Typst 0.11.0, see the
\href{https://typst.app/docs/reference/model/table/}{docs} .

However, \textbf{keep an eye for future tablex updates} as there might
be some interesting goodies ahead, including CeTZ support!

\textbf{Details:}

A large amount of tablex’s features have successfully been upstreamed
by this package’s author to Typst’s built-in \texttt{\ table\ } and
\texttt{\ grid\ } elements (see the new Tables Guide, at
\url{https://typst.app/docs/guides/table-guide/} , and the
\texttt{\ table\ } element’s reference, at
\url{https://typst.app/docs/reference/model/table/} , for more
information).

This effort was tracked in the following Typst issue:
\url{https://github.com/typst/typst/issues/3001}

This means that, starting with Typst 0.11.0, \textbf{many advanced table
features can now be used with Typst grids and tables without tablex!}
This includes:

\begin{itemize}
\tightlist
\item
  Per-cell customization (through
  \texttt{\ table.cell(inset:\ ...,\ align:\ ...,\ fill:\ ...){[}body{]}\ }
  , and \texttt{\ \#show\ table.cell:\ it\ =\textgreater{}\ ...\ }
  instead of \texttt{\ map-cells\ } );
\item
  Merging cells (colspans and rowspans, through
  \texttt{\ table.cell(colspan:\ 2,\ rowspan:\ 2){[}body{]}\ } );
\item
  Line customization (you can control the \texttt{\ stroke\ } parameter
  of \texttt{\ table.cell\ } to control the lines around it, and you can
  use \texttt{\ table.hline\ } and \texttt{\ table.vline\ } which work
  similarly to their tablex counterparts - the equivalent of
  \texttt{\ map-hlines\ } and \texttt{\ map-vlines\ } is
  \texttt{\ table(stroke:\ (x,\ y)\ =\textgreater{}\ (left:\ ...,\ right:\ ...,\ top:\ ...,\ bottom:\ ...))\ }
  );
\item
  Repeatable table headers (through
  \texttt{\ table.header(...\ cells\ ...)\ } );
\item
  The features above are available within \texttt{\ grid\ } as well by
  replacing \texttt{\ table\ } with \texttt{\ grid\ } where applicable
  (e.g. \texttt{\ grid.cell\ } instead of \texttt{\ table.cell\ } ).
\end{itemize}

Additionally, built-in Typst tables have support for features which
weren’t previously available within tablex, such as \textbf{repeatable
table footers} (through \texttt{\ table.footer\ } and
\texttt{\ grid.footer\ } ).

Therefore, \textbf{for the vast majority of use cases, you will no
longer need to use this library.}

There are a few observations:

\begin{enumerate}
\tightlist
\item
  \textbf{Tablex will still receive updates over time} with extra
  features. In the next version (tablex 0.1.0), there will be
  \textbf{support for CeTZ integration} , which will allow you to easily
  \textbf{annotate your tables} using CeTZ (e.g. draw arrows between
  cells). If you’re interested in such features, then tablex might
  still be useful for you in the future!
\item
  \textbf{Not \emph{all} tablex features are present in built-in tables,
  at least yet.} Therefore, \textbf{if you happen to use the features
  listed below, you might still have to use tablex} depending on your
  use case. It is expected, however, that built-in tables will
  eventually have support for most of the missing features in future
  Typst releases. Here’s a non-exhaustive list of them:

  \begin{enumerate}
  \tightlist
  \item
    Built-in tables do not yet have the ability to expand table lines by
    some arbitrary length.
  \item
    The tablex \texttt{\ fit-spans\ } option, through which colspans and
    rowspans don’t cause \texttt{\ auto\ } -sized columns and/or rows
    to expand, is not yet supported in built-in tables.
  \item
    Built-in repeatable table headers currently always repeat in all
    pages, whereas you can define in which pages a tablex header should
    be repeated.
  \end{enumerate}
\item
  \textbf{Regarding sponsorships:} Any future sponsorships to the tablex
  author, \href{https://github.com/PgBiel}{@PgBiel} , who was also
  responsible for upstreaming the various tablex features to built-in
  tables, will go not only towards extended maintenance of tablex, but
  also towards other general contributions to the Typst ecosystem and
  his other open-source contributions! More information here:
  \url{https://github.com/sponsors/PgBiel/}
\end{enumerate}

If there any questions, feel free to open a thread in the
\texttt{\ Discussions\ } page of this repository, or ping the author on
Discord. Thanks to everyone who supported me throughout tablex’s
development and the upstreaming process. I hope you enjoy the new
update, and have fun with tables! 😄

And make sure to keep an eye for future tablex updates. 😉

\subsection{Table of Contents}\label{table-of-contents}

\begin{itemize}
\tightlist
\item
  \href{https://github.com/typst/packages/raw/main/packages/preview/tablex/0.0.9/\#usage}{Usage}
\item
  \href{https://github.com/typst/packages/raw/main/packages/preview/tablex/0.0.9/\#features}{Features}

  \begin{itemize}
  \tightlist
  \item
    \href{https://github.com/typst/packages/raw/main/packages/preview/tablex/0.0.9/\#almost-drop-in-replacement-for-table}{\emph{Almost}
    drop-in replacement for \texttt{\ \#table\ }}
  \item
    \href{https://github.com/typst/packages/raw/main/packages/preview/tablex/0.0.9/\#colspanxrowspanx}{colspanx/rowspanx}
  \item
    \href{https://github.com/typst/packages/raw/main/packages/preview/tablex/0.0.9/\#repeat-header-rows}{Repeat
    header rows}
  \item
    \href{https://github.com/typst/packages/raw/main/packages/preview/tablex/0.0.9/\#customize-every-single-line}{Customize
    every single line}
  \item
    \href{https://github.com/typst/packages/raw/main/packages/preview/tablex/0.0.9/\#customize-every-single-cell}{Customize
    every single cell}
  \end{itemize}
\item
  \href{https://github.com/typst/packages/raw/main/packages/preview/tablex/0.0.9/\#known-issues}{Known
  Issues}
\item
  \href{https://github.com/typst/packages/raw/main/packages/preview/tablex/0.0.9/\#reference}{Reference}

  \begin{itemize}
  \tightlist
  \item
    \href{https://github.com/typst/packages/raw/main/packages/preview/tablex/0.0.9/\#basic-types-and-functions}{Basic
    types and functions}
  \item
    \href{https://github.com/typst/packages/raw/main/packages/preview/tablex/0.0.9/\#gridx-and-tablex}{Gridx
    and Tablex}
  \end{itemize}
\item
  \href{https://github.com/typst/packages/raw/main/packages/preview/tablex/0.0.9/\#changelog}{Changelog}

  \begin{itemize}
  \tightlist
  \item
    \href{https://github.com/typst/packages/raw/main/packages/preview/tablex/0.0.9/\#v009}{v0.0.9}
  \item
    \href{https://github.com/typst/packages/raw/main/packages/preview/tablex/0.0.9/\#v008}{v0.0.8}
  \item
    \href{https://github.com/typst/packages/raw/main/packages/preview/tablex/0.0.9/\#v007}{v0.0.7}
  \item
    \href{https://github.com/typst/packages/raw/main/packages/preview/tablex/0.0.9/\#v006}{v0.0.6}
  \item
    \href{https://github.com/typst/packages/raw/main/packages/preview/tablex/0.0.9/\#v005}{v0.0.5}
  \item
    \href{https://github.com/typst/packages/raw/main/packages/preview/tablex/0.0.9/\#v004}{v0.0.4}
  \item
    \href{https://github.com/typst/packages/raw/main/packages/preview/tablex/0.0.9/\#v003}{v0.0.3}
  \item
    \href{https://github.com/typst/packages/raw/main/packages/preview/tablex/0.0.9/\#v002}{v0.0.2}
  \item
    \href{https://github.com/typst/packages/raw/main/packages/preview/tablex/0.0.9/\#v001}{v0.0.1}
  \end{itemize}
\item
  \href{https://github.com/typst/packages/raw/main/packages/preview/tablex/0.0.9/\#010-roadmap}{0.1.0
  Roadmap}
\item
  \href{https://github.com/typst/packages/raw/main/packages/preview/tablex/0.0.9/\#license}{License}
\end{itemize}

\subsection{Usage}\label{usage}

\textbf{NOTE: Please use built-in tables instead of this library} (see
notice above). \textbf{The rest of the README is kept for reference
purposes only.}

To use this library through the Typst package manager \textbf{(for Typst
v0.6.0+)} , write for example
\texttt{\ \#import\ "@preview/tablex:0.0.9":\ tablex,\ cellx\ } at the
top of your Typst file (you may also add whichever other functions you
use from the library to that import list!).

For older Typst versions, download the file \texttt{\ tablex.typ\ } from
the latest release (or directly from the main branch, for the
‘bleeding edge’) at the tablex repository (
\url{https://github.com/PgBiel/typst-tablex} ) and place it on the same
folder as your own Typst file. Then, at the top of your file, write for
example \texttt{\ \#import\ "tablex.typ":\ tablex,\ cellx\ } (plus
whichever other functions you use from the library).

This library should be compatible with Typst versions between v0.2.0 and
v0.12.0 (inclusive). \textbf{Using the latest Typst version is always
recommended} in order to make use of the latest optimizations and
features available.

Here’s an example of what \texttt{\ tablex\ } can do:

\pandocbounded{\includegraphics[keepaspectratio]{https://github.com/PgBiel/typst-tablex/assets/9021226/355c527a-7296-4264-bac7-4ec991b15a18}}

Here’s the code for that table:

\begin{Shaded}
\begin{Highlighting}[]
\NormalTok{\#import "@preview/tablex:0.0.9": tablex, rowspanx, colspanx}

\NormalTok{\#tablex(}
\NormalTok{  columns: 4,}
\NormalTok{  align: center + horizon,}
\NormalTok{  auto{-}vlines: false,}

\NormalTok{  // indicate the first two rows are the header}
\NormalTok{  // (in case we need to eventually}
\NormalTok{  // enable repeating the header across pages)}
\NormalTok{  header{-}rows: 2,}

\NormalTok{  // color the last column\textquotesingle{}s cells}
\NormalTok{  // based on the written number}
\NormalTok{  map{-}cells: cell =\textgreater{} \{}
\NormalTok{    if cell.x == 3 and cell.y \textgreater{} 1 \{}
\NormalTok{      cell.content = \{}
\NormalTok{        let value = int(cell.content.text)}
\NormalTok{        let text{-}color = if value \textless{} 10 \{}
\NormalTok{          red.lighten(30\%)}
\NormalTok{        \} else if value \textless{} 15 \{}
\NormalTok{          yellow.darken(13\%)}
\NormalTok{        \} else \{}
\NormalTok{          green}
\NormalTok{        \}}
\NormalTok{        set text(text{-}color)}
\NormalTok{        strong(cell.content)}
\NormalTok{      \}}
\NormalTok{    \}}
\NormalTok{    cell}
\NormalTok{  \},}

\NormalTok{  /* {-}{-}{-} header {-}{-}{-} */}
\NormalTok{  rowspanx(2)[*Username*], colspanx(2)[*Data*], (), rowspanx(2)[*Score*],}
\NormalTok{  (),                 [*Location*], [*Height*], (),}
\NormalTok{  /* {-}{-}{-}{-}{-}{-}{-}{-}{-}{-}{-}{-}{-}{-} */}

\NormalTok{  [John], [Second St.], [180 cm], [5],}
\NormalTok{  [Wally], [Third Av.], [160 cm], [10],}
\NormalTok{  [Jason], [Some St.], [150 cm], [15],}
\NormalTok{  [Robert], [123 Av.], [190 cm], [20],}
\NormalTok{  [Other], [Unknown St.], [170 cm], [25],}
\NormalTok{)}
\end{Highlighting}
\end{Shaded}

\subsection{Features}\label{features}

\subsubsection{\texorpdfstring{\emph{Almost} drop-in replacement for
\texttt{\ \#table\ }}{Almost drop-in replacement for  \#table }}\label{almost-drop-in-replacement-for-table}

( \textbf{Update:} tablex’s syntax was designed to be compatible with
Typst tables created \textbf{up to Typst v0.10.0} . The new table
features introduced in Typst v0.11.0 use syntax which isn’t compatible
with tablex, so it won’t be a drop-in replacement in that case.
However, tablex does have its own syntax for those features, as will be
explained below!)

In most cases, you should be able to replace \texttt{\ \#table\ } with
\texttt{\ \#tablex\ } and be good to go for a start - it should look
\emph{very} similar (if not identical). Indeed, the syntax is very
similar for the basics:

\begin{Shaded}
\begin{Highlighting}[]
\NormalTok{\#import "@preview/tablex:0.0.9": tablex}

\NormalTok{\#tablex(}
\NormalTok{  columns: (auto, 1em, 1fr, 1fr),  // 4 columns}
\NormalTok{  rows: auto,  // at least 1 row of auto size}
\NormalTok{  fill: red,}
\NormalTok{  align: center + horizon,}
\NormalTok{  stroke: green,}
\NormalTok{  [a], [b], [c], [d],}
\NormalTok{  [e], [f], [g], [h],}
\NormalTok{  [i], [j], [k], [l]}
\NormalTok{)}
\end{Highlighting}
\end{Shaded}

\pandocbounded{\includegraphics[keepaspectratio]{https://user-images.githubusercontent.com/9021226/230818397-2d599324-32a5-4184-973f-2fcfb6b62c84.png}}

There are still a few oddities in the library (see
\href{https://github.com/typst/packages/raw/main/packages/preview/tablex/0.0.9/\#known-issues}{Known
Issues} for more info), but, for the vast majority of cases, replacing
\texttt{\ \#tablex\ } by \texttt{\ \#table\ } should work just fine.
(Sometimes you can even replace \texttt{\ \#grid\ } by
\texttt{\ \#gridx\ } - see the line customization section for more -,
but not always, as the behavior is a bit different.)

This is mostly a word of caution in case anything I haven’t
anticipated happens, but, based on my tests (and after tons of
bug-fixing commits), the vast majority of tables (that don’t face one
of the listed known issues) should work just fine under the library.

\textbf{Note:} If your document is written in a right-to-left (RTL)
script, you may wish to enable \texttt{\ rtl:\ true\ } for your tables
so that the order of cells and lines properly follows your text
direction (when combined with \texttt{\ set\ text(dir:\ rtl)\ } ). This
is necessary because tablex cannot detect that setting automatically at
the moment (while the native Typst table can and flips itself
horizontally automatically). See the tablex option reference for more
information.

\subsubsection{colspanx/rowspanx}\label{colspanxrowspanx}

Your cells can now span more than one column and/or row at once, with
\texttt{\ colspanx\ } / \texttt{\ rowspanx\ } :

\begin{Shaded}
\begin{Highlighting}[]
\NormalTok{\#import "@preview/tablex:0.0.9": tablex, colspanx, rowspanx}

\NormalTok{\#tablex(}
\NormalTok{  columns: 3,}
\NormalTok{  colspanx(2)[a], (),  [b],}
\NormalTok{  [c], rowspanx(2)[d], [ed],}
\NormalTok{  [f], (),             [g]}
\NormalTok{)}
\end{Highlighting}
\end{Shaded}

\pandocbounded{\includegraphics[keepaspectratio]{https://user-images.githubusercontent.com/9021226/230810720-fbdfdbe5-8568-42ed-b8a2-5eff332a89d6.png}}

Note that the empty parentheses there are just for organization, and are
ignored (unless they come before the first cell - more on that later).
They’re useful to help us keep track of which cell positions are being
used up by the spans, because, if we try to add an actual cell at these
spots, it will just push the others forward, which might seem
unexpected.

Use \texttt{\ colspanx(2,\ rowspanx(2){[}d{]})\ } to colspan and rowspan
at the same time. Be careful not to attempt to overwrite other cells’
spans, as you will get a nasty error.

\textbf{Note (since tablex v0.0.8):} By default, colspans and rowspans
can cause spanned \texttt{\ auto\ } columns and rows to expand to fit
their contents (only the last spanned track - column or row - can
expand). If you’d like colspans to not affect column sizes at all (and
thus “fit� within their spanned columns), you may specify
\texttt{\ fit-spans:\ (x:\ true)\ } to the table. Similarly, you can
specify \texttt{\ fit-spans:\ (y:\ true)\ } to have rowspans not affect
row sizes at all. To apply both effects, use either
\texttt{\ fit-spans:\ true\ } or
\texttt{\ fit-spans:\ (x:\ true,\ y:\ true)\ } . You can also apply this
to a single colspan (for example) with
\texttt{\ colspanx(2,\ fit-spans:\ (x:\ true)){[}a{]}\ } , as this
option is available not only for the whole table but also for each cell.
See the reference section for more information.

\subsubsection{Repeat header rows}\label{repeat-header-rows}

You can now ensure the first row (or, rather, the rows covered by the
first rowspan) in your table repeats across pages. Just use
\texttt{\ repeat-header:\ true\ } (default is \texttt{\ false\ } ).

Note that you may wish to customize this. Use
\texttt{\ repeat-header:\ 6\ } to repeat for 6 more pages. Use
\texttt{\ repeat-header:\ (2,\ 4)\ } to repeat only on the 2nd and the
4th page (where the 1st page is the one the table starts in).
Additionally, use \texttt{\ header-rows:\ 5\ } to ensure the first
(e.g.) 5 rows are part of the header (by default, this is 1 - more rows
will be repeated where necessary if rowspans are used).

Also, note that, by default, the horizontal lines below the header are
transported to other pages, which may be an annoyance if you customize
lines too much (see below). Use
\texttt{\ header-hlines-have-priority:\ false\ } to ensure that the
first row in each page will dictate the appearance of the horizontal
lines above it (and not the header).

\textbf{Note:} Depending on the size of your document, repeatable
headers might not behave properly due to certain limitations in
Typst’s introspection system (as observed in
\url{https://github.com/PgBiel/typst-tablex/issues/43} ).

Example:

\begin{Shaded}
\begin{Highlighting}[]
\NormalTok{\#import "@preview/tablex:0.0.9": tablex, hlinex, vlinex, colspanx, rowspanx}

\NormalTok{\#pagebreak()}
\NormalTok{\#v(80\%)}

\NormalTok{\#tablex(}
\NormalTok{  columns: 4,}
\NormalTok{  align: center + horizon,}
\NormalTok{  auto{-}vlines: false,}
\NormalTok{  repeat{-}header: true,}

\NormalTok{  /* {-}{-}{-} header {-}{-}{-} */}
\NormalTok{  rowspanx(2)[*Names*], colspanx(2)[*Properties*], (), rowspanx(2)[*Creators*],}
\NormalTok{  (),                 [*Type*], [*Size*], (),}
\NormalTok{  /* {-}{-}{-}{-}{-}{-}{-}{-}{-}{-}{-}{-}{-}{-} */}

\NormalTok{  [Machine], [Steel], [5 $"cm"\^{}3$], [John p\& Kate],}
\NormalTok{  [Frog], [Animal], [6 $"cm"\^{}3$], [Robert],}
\NormalTok{  [Frog], [Animal], [6 $"cm"\^{}3$], [Robert],}
\NormalTok{  [Frog], [Animal], [6 $"cm"\^{}3$], [Robert],}
\NormalTok{  [Frog], [Animal], [6 $"cm"\^{}3$], [Robert],}
\NormalTok{  [Frog], [Animal], [6 $"cm"\^{}3$], [Robert],}
\NormalTok{  [Frog], [Animal], [6 $"cm"\^{}3$], [Robert],}
\NormalTok{  [Frog], [Animal], [6 $"cm"\^{}3$], [Rodbert],}
\NormalTok{)}
\end{Highlighting}
\end{Shaded}

\pandocbounded{\includegraphics[keepaspectratio]{https://user-images.githubusercontent.com/9021226/230810751-776a73c4-9c24-46ba-92cd-76292469ab7d.png}}

\subsubsection{Customize every single
line}\label{customize-every-single-line}

Every single line in the table is either a \texttt{\ hlinex\ }
(horizontal) or \texttt{\ vlinex\ } (vertical) instance. By default,
there is one between every column and between every row - unless you
specify a custom line for some column or row, in which case the
automatic line for it will be removed (to allow you to freely customize
it). To disable this behavior, use \texttt{\ auto-lines:\ false\ } ,
which will remove \emph{all} automatic lines. You may also remove only
automatic horizontal lines with \texttt{\ auto-hlines:\ false\ } , and
only vertical with \texttt{\ auto-vlines:\ false\ } .

\textbf{Note:} \texttt{\ gridx\ } is an alias for \texttt{\ tablex\ }
with \texttt{\ auto-lines:\ false\ } .

For your custom lines, write \texttt{\ hlinex()\ } at any position and
it will add a horizontal line below the current cell row (or at the top,
if before any cell). You can use
\texttt{\ hlinex(start:\ a,\ end:\ b)\ } to control the cells which that
line spans ( \texttt{\ a\ } is the first column number and
\texttt{\ b\ } is the last column number). You can also specify its
stroke (color/thickness) with \texttt{\ hlinex(stroke:\ red\ +\ 5pt)\ }
for example. To position it at an arbitrary row, use
\texttt{\ hlinex(y:\ 6)\ } or similar. (Columns and rows are indexed
starting from 0.)

Something similar occurs for \texttt{\ vlinex()\ } , which has
\texttt{\ start\ } , \texttt{\ end\ } (first row and last row it spans),
and also \texttt{\ stroke\ } . They will, by default, be placed to the
right of the current cell, and will span the entire table (top to
bottom). To override the default placement, use
\texttt{\ vlinex(x:\ 2)\ } or similar.

\textbf{Note:} Only one hline or vline with the same span (same
start/end) can be placed at once.

\textbf{Note:} You can also place vlines before the first cell, in which
case \emph{they will be placed consecutively, each after the last
\texttt{\ vlinex()\ }} . That is, if you place several of them in a row
( \emph{before the first cell} only), then it will not place all of them
at one location (which is normally what happens if you try to place
multiple vlines at once), but rather one after the other. With this
behavior, you can also specify \texttt{\ ()\ } between each vline to
\emph{skip} certain positions (again, only before the first cell -
afterwards, all \texttt{\ ()\ } are ignored). Note that you can also
just ignore this entirely and use \texttt{\ vlinex(x:\ 0)\ } ,
\texttt{\ vlinex(x:\ 1)\ } , …, \texttt{\ vlinex(x:\ columns.len())\ }
for full control.

Here’s some sample usage:

\begin{Shaded}
\begin{Highlighting}[]
\NormalTok{\#import "@preview/tablex:0.0.9": tablex, gridx, hlinex, vlinex, colspanx, rowspanx}

\NormalTok{\#tablex(}
\NormalTok{  columns: 4,}
\NormalTok{  auto{-}lines: false,}

\NormalTok{  // skip a column here         vv}
\NormalTok{  vlinex(), vlinex(), vlinex(), (), vlinex(),}
\NormalTok{  colspanx(2)[a], (),  [b], [J],}
\NormalTok{  [c], rowspanx(2)[d], [e], [K],}
\NormalTok{  [f], (),             [g], [L],}
\NormalTok{  //   \^{}\^{} \textquotesingle{}()\textquotesingle{} after the first cell are 100\% ignored}
\NormalTok{)}

\NormalTok{\#tablex(}
\NormalTok{  columns: 4,}
\NormalTok{  auto{-}vlines: false,}
\NormalTok{  colspanx(2)[a], (),  [b], [J],}
\NormalTok{  [c], rowspanx(2)[d], [e], [K],}
\NormalTok{  [f], (),             [g], [L],}
\NormalTok{)}

\NormalTok{\#gridx(}
\NormalTok{  columns: 4,}
\NormalTok{  (), (), vlinex(end: 2),}
\NormalTok{  hlinex(stroke: yellow + 2pt),}
\NormalTok{  colspanx(2)[a], (),  [b], [J],}
\NormalTok{  hlinex(start: 0, end: 1, stroke: yellow + 2pt),}
\NormalTok{  hlinex(start: 1, end: 2, stroke: green + 2pt),}
\NormalTok{  hlinex(start: 2, end: 3, stroke: red + 2pt),}
\NormalTok{  hlinex(start: 3, end: 4, stroke: blue.lighten(50\%) + 2pt),}
\NormalTok{  [c], rowspanx(2)[d], [e], [K],}
\NormalTok{  hlinex(start: 2),}
\NormalTok{  [f], (),             [g], [L],}
\NormalTok{)}
\end{Highlighting}
\end{Shaded}

\pandocbounded{\includegraphics[keepaspectratio]{https://user-images.githubusercontent.com/9021226/230817335-8a908d44-77be-45d2-b98f-89e9ccf07dc7.png}}

\paragraph{Bulk line customization}\label{bulk-line-customization}

You can also \emph{bulk-customize lines} by specifying
\texttt{\ map-hlines:\ h\ =\textgreater{}\ new\_hline\ } and
\texttt{\ map-vlines:\ v\ =\textgreater{}\ new\_vline\ } . This includes
any automatically generated lines. For example:

\begin{Shaded}
\begin{Highlighting}[]
\NormalTok{\#import "@preview/tablex:0.0.9": tablex, colspanx, rowspanx}

\NormalTok{\#tablex(}
\NormalTok{  columns: 3,}
\NormalTok{  map{-}hlines: h =\textgreater{} (..h, stroke: blue),}
\NormalTok{  map{-}vlines: v =\textgreater{} (..v, stroke: green + 2pt),}
\NormalTok{  colspanx(2)[a], (),  [b],}
\NormalTok{  [c], rowspanx(2)[d], [ed],}
\NormalTok{  [f], (),             [g]}
\NormalTok{)}
\end{Highlighting}
\end{Shaded}

\pandocbounded{\includegraphics[keepaspectratio]{https://user-images.githubusercontent.com/9021226/235371652-48e7e526-1eb0-43c3-a6f4-3ed81840cffc.png}}

\subsubsection{Customize every single
cell}\label{customize-every-single-cell}

Cells can be customized entirely. Instead of specifying content (e.g.
\texttt{\ {[}text{]}\ } ) as a table item, you can specify
\texttt{\ cellx(property:\ a,\ property:\ b,\ ...){[}text{]}\ } , which
allows you to customize properties, such as:

\begin{itemize}
\tightlist
\item
  \texttt{\ colspan:\ 2\ } (same as using
  \texttt{\ colspanx(2,\ ...){[}...{]}\ } )
\item
  \texttt{\ rowspan:\ 3\ } (same as using
  \texttt{\ rowspanx(3,\ ...){[}...{]}\ } )
\item
  \texttt{\ align:\ center\ } (override whole-table alignment for this
  cell)
\item
  \texttt{\ fill:\ blue\ } (fill just this cell with that color)
\item
  \texttt{\ inset:\ 0pt\ } (override inset/internal padding for this
  cell - note that this can look off unless you use auto columns and
  rows)
\item
  \texttt{\ x:\ 5\ } (arbitrarily place the cell at the given column,
  beginning at 0 - may error if conflicts occur)
\item
  \texttt{\ y:\ 6\ } (arbitrarily place the cell at the given row,
  beginning at 0 - may error if conflicts occur)
\end{itemize}

Additionally, instead of specifying content to the cell, you can specify
a function \texttt{\ (column,\ row)\ =\textgreater{}\ content\ } ,
allowing each cell to be aware of where it’s positioned. (Note that
positions are recorded in the cell’s \texttt{\ .x\ } and
\texttt{\ .y\ } attributes, and start as \texttt{\ auto\ } unless you
specify otherwise.)

For example:

\begin{Shaded}
\begin{Highlighting}[]
\NormalTok{\#import "@preview/tablex:0.0.9": tablex, cellx, colspanx, rowspanx}

\NormalTok{\#tablex(}
\NormalTok{  columns: 3,}
\NormalTok{  fill: red,}
\NormalTok{  align: right,}
\NormalTok{  colspanx(2)[a], (),  [beeee],}
\NormalTok{  [c], rowspanx(2)[d], cellx(fill: blue, align: left)[e],}
\NormalTok{  [f], (),             [g],}

\NormalTok{  // place this cell at the first column, seventh row}
\NormalTok{  cellx(colspan: 3, align: center, x: 0, y: 6)[hi I\textquotesingle{}m down here]}
\NormalTok{)}
\end{Highlighting}
\end{Shaded}

\pandocbounded{\includegraphics[keepaspectratio]{https://user-images.githubusercontent.com/9021226/230818283-b3b636db-dbd0-47b8-bdd5-f61a07d58749.png}}

\paragraph{Bulk customization of
cells}\label{bulk-customization-of-cells}

To customize multiple cells at once, you have a few options:

\begin{enumerate}
\item
  \texttt{\ map-cells:\ cell\ =\textgreater{}\ cell\ } (given a cell,
  returns a new cell). You can use this to customize the cell’s
  attributes, but also to change its positions (however, avoid doing
  that as it can easily generate conflicts). You can access the cell’s
  position with \texttt{\ cell.x\ } and \texttt{\ cell.y\ } . All other
  attributes are also accessible and changeable (see the
  \texttt{\ Reference\ } further below for a list). Return something
  like \texttt{\ (..cell,\ fill:\ blue)\ } , for example, to ensure the
  other properties (including the cell type marker) are kept. (Calling
  \texttt{\ cellx\ } here is not necessary. If overriding the cell’s
  content, use \texttt{\ content:\ {[}whatever{]}\ } ). This is useful
  if you want to, for example, customize a cell’s fill color based on
  its contents, or add some content to every cell, or something similar.
\item
  \texttt{\ map-rows:\ (row\_index,\ cells)\ =\textgreater{}\ cells\ }
  (given a row index and all cells in it, return a new array of cells).
  Allows customizing entire rows, but note that the cells in the
  \texttt{\ cells\ } parameter can be \texttt{\ none\ } if they’re
  some position occupied by a colspan or rowspan of another cell. Ensure
  you return the cells in the order you were given, including the
  \texttt{\ none\ } s, for best results. Also, you cannot move cells
  here to another row. You can change the cells’ columns (by changing
  their \texttt{\ x\ } property), but that will certainly generate
  conflicts if any col/rowspans are involved (in general, you cannot
  bulk-change col/rowspans without \texttt{\ map-cells\ } ).
\item
  \texttt{\ map-cols:\ (col\_index,\ cells)\ =\textgreater{}\ cells\ }
  (given a column index and all cells in it, return a new array of
  cells). Similar to \texttt{\ map-rows\ } , but for customizing
  columns. You cannot change the column of any cell here. (To do that,
  \texttt{\ map-cells\ } is required.) You can, however, change its row
  (with \texttt{\ y\ } , but do that sparingly), and, of course, all
  other properties.
\end{enumerate}

\textbf{Note:} Execution order is \texttt{\ map-cells\ } =\textgreater{}
\texttt{\ map-rows\ } =\textgreater{} \texttt{\ map-cols\ } .

Example:

\begin{Shaded}
\begin{Highlighting}[]
\NormalTok{\#import "@preview/tablex:0.0.9": tablex, colspanx, rowspanx}

\NormalTok{\#tablex(}
\NormalTok{  columns: 4,}
\NormalTok{  auto{-}vlines: true,}

\NormalTok{  // make all cells italicized}
\NormalTok{  map{-}cells: cell =\textgreater{} \{}
\NormalTok{    (..cell, content: emph(cell.content))}
\NormalTok{  \},}

\NormalTok{  // add some arbitrary content to entire rows}
\NormalTok{  map{-}rows: (row, cells) =\textgreater{} cells.map(c =\textgreater{}}
\NormalTok{    if c == none \{}
\NormalTok{      c  // keeping \textquotesingle{}none\textquotesingle{} is important}
\NormalTok{    \} else \{}
\NormalTok{      (..c, content: [\#c.content\textbackslash{} *R\#row*])}
\NormalTok{    \}}
\NormalTok{  ),}

\NormalTok{  // color cells based on their columns}
\NormalTok{  // (using \textquotesingle{}fill: (column, row) =\textgreater{} color\textquotesingle{} also works}
\NormalTok{  // for this particular purpose)}
\NormalTok{  map{-}cols: (col, cells) =\textgreater{} cells.map(c =\textgreater{}}
\NormalTok{    if c == none \{}
\NormalTok{      c}
\NormalTok{    \} else \{}
\NormalTok{      (..c, fill: if col \textless{} 2 \{ blue \} else \{ yellow \})}
\NormalTok{    \}}
\NormalTok{  ),}

\NormalTok{  colspanx(2)[a], (),  [b], [J],}
\NormalTok{  [c], rowspanx(2)[dd], [e], [K],}
\NormalTok{  [f], (),             [g], [L],}
\NormalTok{)}
\end{Highlighting}
\end{Shaded}

\pandocbounded{\includegraphics[keepaspectratio]{https://user-images.githubusercontent.com/9021226/230818347-30b49154-f444-4744-9415-dd4030b29393.png}}

Another example (summing columns):

\begin{Shaded}
\begin{Highlighting}[]
\NormalTok{\#gridx(}
\NormalTok{  columns: 3,}
\NormalTok{  rows: 6,}
\NormalTok{  fill: (col, row) =\textgreater{} (blue, red, green).at(calc.rem(row + col {-} 1, 3)),}
\NormalTok{  map{-}cols: (col, cells) =\textgreater{} \{}
\NormalTok{    let last = cells.last()}
\NormalTok{    last.content = [}
\NormalTok{      \#cells.slice(0, cells.len() {-} 1).fold(0, (acc, c) =\textgreater{} if c != none \{ acc + eval(c.content.text) \} else \{ acc \})}
\NormalTok{    ]}
\NormalTok{    last.fill = aqua}
\NormalTok{    cells.last() = last}
\NormalTok{    cells}
\NormalTok{  \},}
\NormalTok{  [0], [5], [10],}
\NormalTok{  [1], [6], [11],}
\NormalTok{  [2], [7], [12],}
\NormalTok{  [3], [8], [13],}
\NormalTok{  [4], [9], [14],}
\NormalTok{  [s], [s], [s]}
\NormalTok{)}
\end{Highlighting}
\end{Shaded}

\pandocbounded{\includegraphics[keepaspectratio]{https://user-images.githubusercontent.com/9021226/231343813-bf06872b-59ac-4221-b6ed-940d73e6a9c4.png}}

\subsection{Known Issues}\label{known-issues}

\begin{itemize}
\item
  Filled cells will partially overlap with horizontal lines above them
  (see \url{https://github.com/PgBiel/typst-tablex/issues/4} ).

  \begin{itemize}
  \tightlist
  \item
    To be fixed in a future rework of the table drawing process.
  \end{itemize}
\item
  Table lines don’t play very well with column and row gutter when a
  colspan or rowspan is used. They may be missing or be cut off by
  gutters.
\item
  Repeatable table headers might not behave properly depending on the
  size of your document or other factors (
  \url{https://github.com/PgBiel/typst-tablex/issues/43} ).
\item
  Using tablex (especially when using repeatable header rows) may cause
  a warning, “layout did not converge within 5 attempts�, to appear
  on recent Typst versions (
  \url{https://github.com/PgBiel/typst-tablex/issues/38} ). This warning
  is due to how tablex works internally \textbf{and is not your fault}
  (in principle), so don’t worry too much about it (unless you’re
  sure it’s not tablex that is causing this).
\item
  Rows with fractional height (such as \texttt{\ 2fr\ } ) have zero
  height if the table spans more than one page. This is because
  fractional row heights are calculated on the available height of the
  first page of the table, which is something that the default
  \texttt{\ \#table\ } can circumvent using internal code. This won’t
  be fixed for now. (Columns with fractional width work fine, provided
  all pages the table is in have the same width, \textbf{and the page
  width isn’t \texttt{\ auto\ }} (which forces fractional columns to
  be 0pt, even in the default \texttt{\ \#table\ } ).)
\item
  Rotation (via Typst’s \texttt{\ \#rotate\ } ) of text only affects
  the visual appearance of the text on the page, but does not change its
  dimensions as they factor into the layout. This leads to certain
  visual issues, such as rotated text potentially overflowing the cell
  height without being hyphenated or, inversely, being hyphenated even
  though there is enough space vertically (
  \url{https://github.com/PgBiel/typst-tablex/issues/59} ). This is a
  \href{https://github.com/typst/typst/issues/528}{known issue} with
  Typst (perhaps, in the future, \texttt{\ \#rotate\ }
  \href{https://github.com/typst/typst/issues/528\#issuecomment-1494123195}{may}
  get a setting to affect layout). As a workaround for the text
  hyphenation problem, the content can be boxed (and thus grouped
  together) with \texttt{\ \#box\ } (e.g.,
  \texttt{\ rowspanx(7,\ box(rotate(-90deg,\ {[}*donothyphenatethis*{]})))\ }
  ), or hyphenation can be prevented by setting
  \texttt{\ \#text(hyphenate:\ false,\ ...)\ } (e.g.,
  \texttt{\ colspanx(2,\ text(hyphenate:\ false,\ rotate(-90deg,\ {[}*donothyphenatethis*{]})))\ }
  ), as also discussed in
  \url{https://github.com/PgBiel/typst-tablex/issues/59} ; another
  alternative is to use \texttt{\ \#place\ } , e.g. aligning to
  \texttt{\ center\ +\ horizon\ } :
  \texttt{\ cellx(place(center\ +\ horizon,\ rotate(-90deg,\ {[}*donothyphenatethis*{]})))\ }
  , which probably allows the most control over the in-cell layout,
  since it simply draws the rotated content without having it occupy any
  space (letting you define that by yourself, e.g. using
  \texttt{\ box(width:\ 1em,\ height:\ 2em,\ place(...))\ } ).

  \begin{itemize}
  \tightlist
  \item
    Alternatively, you may attempt to use the solution proposed at
    \url{https://github.com/typst/typst/issues/528\#issuecomment-1494318510}
    to define a \texttt{\ rotatex\ } function which produces a rotated
    element with the appropriate sizes, such that tablex may recognize
    its size accordingly and avoid visual glitches.
  \end{itemize}
\item
  \texttt{\ tablex\ } can potentially be slower and/or take longer to
  compile than the default \texttt{\ table\ } (especially when the table
  spans a lot of pages). \textbf{Please use the latest Typst version to
  reduce this problem} (each version has been bringing further
  improvements in this sense). Still, we are looking for ways to better
  optimize the library (see more discussion at
  \url{https://github.com/PgBiel/typst-tablex/issues/5} - feel free to
  give some input!). However, re-compilation is usually fine thanks to
  Typst’s built-in memoization.
\item
  The internals of the library still aren’t very well documented; I
  plan on adding more info about this eventually.
\item
  \textbf{Please open a GitHub issue for anything weird you come across}
  (make sure others haven’t reported it first).
\end{itemize}

\subsection{Reference}\label{reference}

\subsubsection{Basic types and
functions}\label{basic-types-and-functions}

\begin{enumerate}
\item
  \texttt{\ cellx\ } : Represents a table cell, and is initialized as
  follows:

\begin{Shaded}
\begin{Highlighting}[]
\NormalTok{\#let cellx(content,}
\NormalTok{  x: auto, y: auto,}
\NormalTok{  rowspan: 1, colspan: 1,}
\NormalTok{  fill: auto, align: auto,}
\NormalTok{  inset: auto,}
\NormalTok{  fit{-}spans: auto}
\NormalTok{) = (}
\NormalTok{  tablex{-}dict{-}type: "cell",}
\NormalTok{  content: content,}
\NormalTok{  rowspan: rowspan,}
\NormalTok{  colspan: colspan,}
\NormalTok{  align: align,}
\NormalTok{  fill: fill,}
\NormalTok{  inset: inset,}
\NormalTok{  fit{-}spans: fit{-}spans,}
\NormalTok{  x: x,}
\NormalTok{  y: y,}
\NormalTok{)}
\end{Highlighting}
\end{Shaded}

  where:

  \begin{itemize}
  \tightlist
  \item
    \texttt{\ tablex-dict-type\ } is the type marker
  \item
    \texttt{\ content\ } is the cell’s content (either
    \texttt{\ content\ } or a function with
    \texttt{\ (col,\ row)\ =\textgreater{}\ content\ } )
  \item
    \texttt{\ rowspan\ } is how many rows this cell spans (default 1)
  \item
    \texttt{\ colspan\ } is how many columns this cell spans (default 1)
  \item
    \texttt{\ align\ } is this cell’s align override, such as
    “center� (default \texttt{\ auto\ } to follow the rest of the
    table)
  \item
    \texttt{\ fill\ } is this cell’s fill override, such as “blue�
    (default \texttt{\ auto\ } to follow the rest of the table)
  \item
    \texttt{\ inset\ } is this cell’s inset override, such as
    \texttt{\ 5pt\ } (default \texttt{\ auto\ } to follow the rest of
    the table)
  \item
    \texttt{\ fit-spans\ } allows overriding the table-wide
    \texttt{\ fit-spans\ } setting for this specific cell (e.g. if this
    cell has a \texttt{\ colspan\ } greater than 1,
    \texttt{\ fit-spans:\ (x:\ true)\ } will cause it to not affect the
    sizes of \texttt{\ auto\ } columns).
  \item
    \texttt{\ x\ } is the cell’s column index (0…len-1) -
    \texttt{\ auto\ } indicates it wasn’t assigned yet
  \item
    \texttt{\ y\ } is the cell’s row index (0…len-1) -
    \texttt{\ auto\ } indicates it wasn’t assigned yet
  \end{itemize}
\item
  \texttt{\ hlinex\ } : represents a horizontal line:

\begin{Shaded}
\begin{Highlighting}[]
\NormalTok{\#let hlinex(}
\NormalTok{  start: 0, end: auto, y: auto,}
\NormalTok{  stroke: auto,}
\NormalTok{  stop{-}pre{-}gutter: auto, gutter{-}restrict: none,}
\NormalTok{  stroke{-}expand: true,}
\NormalTok{  expand: none}
\NormalTok{) = (}
\NormalTok{  tablex{-}dict{-}type: "hline",}
\NormalTok{  start: start,}
\NormalTok{  end: end,}
\NormalTok{  y: y,}
\NormalTok{  stroke: stroke,}
\NormalTok{  stop{-}pre{-}gutter: stop{-}pre{-}gutter,}
\NormalTok{  gutter{-}restrict: gutter{-}restrict,}
\NormalTok{  stroke{-}expand: stroke{-}expand,}
\NormalTok{  expand: expand,}
\NormalTok{  parent: none,}
\NormalTok{)}
\end{Highlighting}
\end{Shaded}

  where:

  \begin{itemize}
  \tightlist
  \item
    \texttt{\ tablex-dict-type\ } is the type marker
  \item
    \texttt{\ start\ } is the column index where the hline starts from
    (default \texttt{\ 0\ } , a.k.a. the beginning)
  \item
    \texttt{\ end\ } is the last column the hline touches (default
    \texttt{\ auto\ } , a.k.a. all the way to the end)

    \begin{itemize}
    \tightlist
    \item
      Note that hlines will \emph{not} be drawn over cells with
      \texttt{\ colspan\ } larger than 1, even if their spans (
      \texttt{\ start\ } - \texttt{\ end\ } ) include that cell.
    \end{itemize}
  \item
    \texttt{\ y\ } is the index of the row at the top of which the hline
    is drawn. (Defaults to \texttt{\ auto\ } , a.k.a. depends on where
    you placed the \texttt{\ hline\ } among the table items - it’s
    always on the top of the row below the current one.)
  \item
    \texttt{\ stroke\ } is the hline’s stroke override (defaults to
    \texttt{\ auto\ } , a.k.a. follow the rest of the table).
  \item
    \texttt{\ stop-pre-gutter\ } : When \texttt{\ true\ } , the hline
    will not be drawn over gutter (which is the default behavior of
    tables). Defaults to \texttt{\ auto\ } which is essentially
    \texttt{\ false\ } (draw over gutter).
  \item
    \texttt{\ gutter-restrict\ } : Either \texttt{\ top\ } ,
    \texttt{\ bottom\ } , or \texttt{\ none\ } . Has no effect if
    \texttt{\ row-gutter\ } is set to \texttt{\ none\ } . Otherwise,
    defines if this \texttt{\ hline\ } should be drawn only on the top
    of the row gutter ( \texttt{\ top\ } ); on the bottom (
    \texttt{\ bottom\ } ); or on both the top and the bottom (
    \texttt{\ none\ } , the default). Note that \texttt{\ top\ } and
    \texttt{\ bottom\ } are alignment values (not strings).
  \item
    \texttt{\ stroke-expand\ } : When \texttt{\ true\ } , the hline will
    be extended as necessary to cover the stroke of the vlines going
    through either end of the line. Defaults to \texttt{\ true\ } .
  \item
    \texttt{\ expand\ } : Optionally extend the hline by an arbitrary
    length. When \texttt{\ none\ } , it is not expanded. When a length
    (such as \texttt{\ 5pt\ } ), it is expanded by that length on both
    ends. When an array of two lengths (such as
    \texttt{\ (5pt,\ 10pt)\ } ), it is expanded to the left by the first
    length (in this case, \texttt{\ 5pt\ } ) and to the right by the
    second (in this case, \texttt{\ 10pt\ } ). Defaults to
    \texttt{\ none\ } .
  \item
    \texttt{\ parent\ } : An internal attribute determined when
    splitting lines among cells. (It should always be \texttt{\ none\ }
    on user-facing interfaces.)
  \end{itemize}
\item
  \texttt{\ vlinex\ } : represents a vertical line:

\begin{Shaded}
\begin{Highlighting}[]
\NormalTok{\#let vlinex(}
\NormalTok{  start: 0, end: auto, x: auto,}
\NormalTok{  stroke: auto,}
\NormalTok{  stop{-}pre{-}gutter: auto, gutter{-}restrict: none,}
\NormalTok{  stroke{-}expand: true,}
\NormalTok{  expand: none}
\NormalTok{) = (}
\NormalTok{  tablex{-}dict{-}type: "vline",}
\NormalTok{  start: start,}
\NormalTok{  end: end,}
\NormalTok{  x: x,}
\NormalTok{  stroke: stroke,}
\NormalTok{  stop{-}pre{-}gutter: stop{-}pre{-}gutter,}
\NormalTok{  gutter{-}restrict: gutter{-}restrict,}
\NormalTok{  stroke{-}expand: stroke{-}expand,}
\NormalTok{  expand: expand,}
\NormalTok{  parent: none,}
\NormalTok{)}
\end{Highlighting}
\end{Shaded}

  where:

  \begin{itemize}
  \tightlist
  \item
    \texttt{\ tablex-dict-type\ } is the type marker
  \item
    \texttt{\ start\ } is the row index where the vline starts from
    (default \texttt{\ 0\ } , a.k.a. the top)
  \item
    \texttt{\ end\ } is the last row the vline touches (default
    \texttt{\ auto\ } , a.k.a. all the way to the bottom)

    \begin{itemize}
    \tightlist
    \item
      Note that vlines will \emph{not} be drawn over cells with
      \texttt{\ rowspan\ } larger than 1, even if their spans (
      \texttt{\ start\ } - \texttt{\ end\ } ) include that cell.
    \end{itemize}
  \item
    \texttt{\ x\ } is the index of the column to the left of which the
    vline is drawn. (Defaults to \texttt{\ auto\ } , a.k.a. depends on
    where you placed the \texttt{\ vline\ } among the table items.)

    \begin{itemize}
    \tightlist
    \item
      For a \texttt{\ vline\ } to be placed after all columns, its
      \texttt{\ x\ } value will be equal to the amount of columns (which
      isn’t a valid column index, but it’s what is used here).
    \end{itemize}
  \item
    \texttt{\ stroke\ } is the vline’s stroke override (defaults to
    \texttt{\ auto\ } , a.k.a. follow the rest of the table).
  \item
    \texttt{\ stop-pre-gutter\ } : When \texttt{\ true\ } , the vline
    will not be drawn over gutter (which is the default behavior of
    tables). Defaults to \texttt{\ auto\ } which is essentially
    \texttt{\ false\ } (draw over gutter).
  \item
    \texttt{\ gutter-restrict\ } : Either \texttt{\ left\ } ,
    \texttt{\ right\ } , or \texttt{\ none\ } . Has no effect if
    \texttt{\ column-gutter\ } is set to \texttt{\ none\ } . Otherwise,
    defines if this \texttt{\ vline\ } should be drawn only to the left
    of the column gutter ( \texttt{\ left\ } ); to the right (
    \texttt{\ right\ } ); or on both the left and the right (
    \texttt{\ none\ } , the default). Note that \texttt{\ left\ } and
    \texttt{\ right\ } are alignment values (not strings).
  \item
    \texttt{\ stroke-expand\ } : When \texttt{\ true\ } , the vline will
    be extended as necessary to cover the stroke of the hlines going
    through either end of the line. Defaults to \texttt{\ true\ } .
  \item
    \texttt{\ expand\ } : Optionally extend the vline by an arbitrary
    length. When \texttt{\ none\ } , it is not expanded. When a length
    (such as \texttt{\ 5pt\ } ), it is expanded by that length on both
    ends. When an array of two lengths (such as
    \texttt{\ (5pt,\ 10pt)\ } ), it is expanded towards the top by the
    first length (in this case, \texttt{\ 5pt\ } ) and towards the
    bottom by the second (in this case, \texttt{\ 10pt\ } ). Defaults to
    \texttt{\ none\ } .
  \item
    \texttt{\ parent\ } : An internal attribute determined when
    splitting lines among cells. (It should always be \texttt{\ none\ }
    on user-facing interfaces.)
  \end{itemize}
\item
  The \texttt{\ occupied\ } type is an internal type used to represent
  cell positions occupied by cells with \texttt{\ colspan\ } or
  \texttt{\ rowspan\ } greater than 1.
\item
  Use \texttt{\ is-tablex-cell\ } , \texttt{\ is-tablex-hline\ } ,
  \texttt{\ is-tablex-vline\ } and \texttt{\ is-tablex-occupied\ } to
  check if a particular object has the corresponding type marker.
\item
  \texttt{\ colspanx\ } and \texttt{\ rowspanx\ } are shorthands for
  setting the \texttt{\ colspan\ } and \texttt{\ rowspan\ } attributes
  of \texttt{\ cellx\ } . They can also be nested (one given as an
  argument to the other) to combine their properties (e.g.,
  \texttt{\ colspanx(2)(rowspanx(3){[}a{]})\ } ). They accept all other
  cell properties with named arguments. For example,
  \texttt{\ colspanx(2,\ align:\ center){[}b{]}\ } is equivalent to
  \texttt{\ cellx(colspan:\ 2,\ align:\ center){[}b{]}\ } .
\end{enumerate}

\subsubsection{Gridx and Tablex}\label{gridx-and-tablex}

\begin{enumerate}
\item
  \texttt{\ gridx\ } is equivalent to \texttt{\ tablex\ } with
  \texttt{\ auto-lines:\ false\ } ; see below.
\item
  \texttt{\ tablex:\ } The main function for creating a table with this
  library:

\begin{Shaded}
\begin{Highlighting}[]
\NormalTok{\#let tablex(}
\NormalTok{  columns: auto, rows: auto,}
\NormalTok{  inset: 5pt,}
\NormalTok{  align: auto,}
\NormalTok{  fill: none,}
\NormalTok{  stroke: auto,}
\NormalTok{  column{-}gutter: auto, row{-}gutter: auto,}
\NormalTok{  gutter: none,}
\NormalTok{  repeat{-}header: false,}
\NormalTok{  header{-}rows: 1,}
\NormalTok{  header{-}hlines{-}have{-}priority: true,}
\NormalTok{  auto{-}lines: true,}
\NormalTok{  auto{-}hlines: auto,}
\NormalTok{  auto{-}vlines: auto,}
\NormalTok{  map{-}cells: none,}
\NormalTok{  map{-}hlines: none,}
\NormalTok{  map{-}vlines: none,}
\NormalTok{  map{-}rows: none,}
\NormalTok{  map{-}cols: none,}
\NormalTok{  ..items}
\NormalTok{) = \{}
\NormalTok{// ...}
\NormalTok{\}}
\end{Highlighting}
\end{Shaded}

  \textbf{Parameters:}

  \begin{itemize}
  \item
    \texttt{\ columns\ } : The sizes (widths) of each column. They work
    just like regular \texttt{\ table\ } ’s columns, and can be:

    \begin{itemize}
    \tightlist
    \item
      an array of lengths ( \texttt{\ 1pt\ } , \texttt{\ 2em\ } ,
      \texttt{\ 100\%\ } , …), including fractional ( \texttt{\ 2fr\ }
      ), to specify the width of each column

      \begin{itemize}
      \tightlist
      \item
        For instance, \texttt{\ columns:\ (2pt,\ 3em)\ } will give you
        two columns: one with a width of \texttt{\ 2pt\ } and another
        with the width of \texttt{\ 3em\ } (3 times the font size).

        \begin{itemize}
        \tightlist
        \item
          Note that percentages, such as \texttt{\ 49\%\ } , \textbf{are
          considered fixed widths} as they are \textbf{always multiplied
          by the full page width} (minus margins) for columns. Thus, a
          column with a size of \texttt{\ 100\%\ } would span your whole
          page (even if there are other columns).
        \end{itemize}
      \item
        \texttt{\ auto\ } may be specified to automatically resize the
        column based on the largest width of its contents, if possible -
        \textbf{this is the most common column width choice,} as it just
        delegates the column sizing job to tablex!

        \begin{itemize}
        \tightlist
        \item
          For example, if your \texttt{\ auto\ } -sized column contains
          two cells with \texttt{\ Hello\ world!\ } and
          \texttt{\ Bye!\ } as contents, tablex will try to make the
          column large enough for \texttt{\ Hello\ world!\ } (the cell
          with largest \emph{potential} width) to fit in a single line.
        \item
          However, note that often enough that’s not possible, as
          increasing the column’s size too much would result in the
          table going over the page’s margin - perhaps even beyond the
          document’s total width. Therefore, \textbf{tablex will
          automatically reduce the size of your \texttt{\ auto\ }
          columns} when they would otherwise cause the table to overrun
          the page’s normal width (i.e. the width between the page’s
          lateral margins).

          \begin{itemize}
          \tightlist
          \item
            Fixed width columns (such as \texttt{\ 2pt\ } ,
            \texttt{\ 3em\ } or \texttt{\ 49\%\ } ) are not subject to
            this size reduction; thus, if you specify all columns’
            widths with fixed lengths, your table \emph{could} become
            larger than the page’s width! (In such a case,
            \textbf{\texttt{\ auto\ } columns would be reduced to a size
            of zero,} as there would be no available space anymore!)
          \end{itemize}
        \end{itemize}
      \item
        when specifying fractional widths ( \texttt{\ 1fr\ } ,
        \texttt{\ 2fr\ } …) for columns, the available space
        (remaining page width, after calculating all other columns’
        sizes) is divided between them, weighted on the fraction value
        of each column.

        \begin{itemize}
        \tightlist
        \item
          For example, with \texttt{\ (1fr,\ 2fr)\ } , the available
          space will be divided by 3 (1 + 2), and the first column will
          have 1/3 of the space, while the second will have 2/3.

          \begin{itemize}
          \tightlist
          \item
            \texttt{\ (1fr,\ 1fr)\ } would cause both columns to have
            equal length (1/2 and 1/2 of the available space).
          \end{itemize}
        \item
          This is useful when you want some columns to just occupy all
          the remaining horizontal space in the page.

          \begin{itemize}
          \tightlist
          \item
            \textbf{Note:} If only one column has a fractional width
            (e.g. a single column with \texttt{\ 1fr\ } ), it will
            occupy the entire available space.
          \end{itemize}
        \item
          \textbf{Warning:} fractional columns in tablex (much like in
          Typst’s default tables) \textbf{will not work properly in
          pages with \texttt{\ auto\ } width} (the columns will have
          width zero) - this is because those pages theoretically have
          infinite width (they can expand indefinitely), so having
          columns spanning the entire available width is then
          impossible!
        \end{itemize}
      \end{itemize}
    \item
      a single length like above, to indicate the width of a single
      column (equivalent to just placing it inside a unit array)

      \begin{itemize}
      \tightlist
      \item
        For instance, \texttt{\ columns:\ 2pt\ } is equivalent to
        \texttt{\ columns:\ (2pt,)\ } , which translates to a single
        column of width \texttt{\ 2pt\ } .
      \end{itemize}
    \item
      an integer (such as \texttt{\ 4\ } ), as a shorthand for
      \texttt{\ (auto,)\ *\ 4\ } (that many \texttt{\ auto\ } columns)

      \begin{itemize}
      \tightlist
      \item
        Useful if you just want to quickly set the amount of columns
        without worrying about their sizes ( \texttt{\ columns:\ 4\ }
        will give you four \texttt{\ auto\ } columns).
      \end{itemize}
    \end{itemize}
  \item
    \texttt{\ rows\ } : The sizes (heights) of each row. They follow the
    exact same format as \texttt{\ columns\ } , except that the
    “available space� is infinite (auto rows can expand as much as
    is needed, as the table can add rows over multiple pages).

    \begin{itemize}
    \tightlist
    \item
      \textbf{Note:} For rows, percentages (such as \texttt{\ 49\%\ } )
      are fixed width lengths, like in \texttt{\ columns\ } ; however,
      here, they are \textbf{multiplied by the page’s full height}
      (minus margins), and not width.
    \item
      \textbf{Note:} If more rows than specified are added, the height
      for the \textbf{last row} will be the one assigned to all extra
      rows. (If the last row is \texttt{\ auto\ } , the extra ones will
      also be \texttt{\ auto\ } , for example.)

      \begin{itemize}
      \tightlist
      \item
        Your table can have more rows than expected by simply having
        more cells than \texttt{\ (\#\ columns)\ } multiplied by
        \texttt{\ (\#\ rows)\ } . In this case, you will have an extra
        row for each \texttt{\ (\#\ columns)\ } cells after the limit.
        In other words, \textbf{the amount of columns is always fixed}
        (determined by the amount of widths in the array given to
        \texttt{\ columns\ } ), but the amount of rows can vary
        depending on your input of cells to the table.
      \item
        Adding a cell at an arbitrary \texttt{\ y\ } coordinate can also
        cause your table to have extra rows (enough rows to reach the
        cell at that coordinate).
      \end{itemize}
    \item
      \textbf{Warning:} support for fractional sizes for rows is still
      rudimentary - they only work properly on the table’s first page;
      on the second page and onwards, they will not behave properly,
      differently from the default \texttt{\ \#table\ } .
    \end{itemize}
  \item
    \texttt{\ inset\ } : Inset/internal padding to give to each cell.
    Can be either a length (same inset from the top, bottom, left and
    right of the cell), or a dictionary (e.g.
    \texttt{\ (left:\ 5pt,\ right:\ 10pt,\ bottom:\ 2pt,\ top:\ 4pt)\ }
    , or even \texttt{\ (left:\ 5pt,\ rest:\ 10pt)\ } to apply the same
    value to the remaining sides). Defaults to \texttt{\ 5pt\ } (the
    \texttt{\ \#table\ } default).
  \item
    \texttt{\ align\ } : How to align text in the cells. Defaults to
    \texttt{\ auto\ } , which inherits alignment from the outer context.
    Must be either \texttt{\ auto\ } , an \texttt{\ alignment\ } (such
    as \texttt{\ left\ } or \texttt{\ top\ } ), a
    \texttt{\ 2d\ alignment\ } (such as \texttt{\ left\ +\ top\ } ), an
    \texttt{\ array\ } of alignment/2d alignment (one for each column in
    the table - if there are more columns than alignment values, they
    will alternate); or a function
    \texttt{\ (column,\ row)\ =\textgreater{}\ alignment/2d\ alignment\ }
    (to customize for each individual cell).
  \item
    \texttt{\ fill\ } : Color with which to fill cells’ backgrounds.
    Defaults to \texttt{\ none\ } , or no fill. Must be either a
    \texttt{\ color\ } , such as \texttt{\ blue\ } ; an
    \texttt{\ array\ } of colors (one for each column in the table - if
    there are more columns than colors, they will alternate); or a
    function \texttt{\ (column,\ row)\ =\textgreater{}\ color\ } (to
    customize for each individual cell).
  \item
    \texttt{\ stroke\ } : Indicates how to draw the table lines.
    Defaults to the current line styles in the document. For example:
    \texttt{\ 5pt\ +\ red\ } to change the color and the thickness.
  \item
    \texttt{\ column-gutter\ } : optional separation (length) between
    columns (such as \texttt{\ 5pt\ } ). Defaults to \texttt{\ none\ }
    (disable). At the moment, looks a bit ugly if your table has a
    \texttt{\ hline\ } attempting to cross a \texttt{\ colspan\ } .
  \item
    \texttt{\ row-gutter\ } : optional separation (length) between rows.
    Defaults to \texttt{\ none\ } (disable). At the moment, looks a bit
    ugly if your table has a \texttt{\ vline\ } attempting to cross a
    \texttt{\ rowspan\ } .
  \item
    \texttt{\ gutter\ } : Sets a length to both \texttt{\ column-\ } and
    \texttt{\ row-gutter\ } at the same time (overridable by each).
  \item
    \texttt{\ repeat-header\ } : Controls header repetition. If set to
    \texttt{\ true\ } , the first row (or the amount of rows specified
    in \texttt{\ header-rows\ } ), including its rowspans, is repeated
    across all pages this table spans. If set to \texttt{\ false\ }
    (default), the aforementioned header row is not repeated in any
    page. If set to an integer (such as \texttt{\ 4\ } ), repeats for
    that many pages after the first, then stops. If set to an array of
    integers (such as \texttt{\ (3,\ 4)\ } ), repeats only on those
    pages \emph{relative to the table’s first page} (page 1 here is
    where the table is, so adding \texttt{\ 1\ } to said array has no
    effect).
  \item
    \texttt{\ header-rows\ } : minimum amount of rows for the repeatable
    header. 1 by default. Automatically increases if one of the cells is
    a rowspan that would go beyond the given amount of rows. For
    example, if 3 is given, then at least the first 3 rows will repeat.
  \item
    \texttt{\ header-hlines-have-priority\ } : if \texttt{\ true\ } ,
    the horizontal lines below the header being repeated take priority
    over the rows they appear atop of on further pages. If
    \texttt{\ false\ } , they draw their own horizontal lines. Defaults
    to \texttt{\ true\ } .

    \begin{itemize}
    \tightlist
    \item
      For example, if your header has a blue hline under it, that blue
      hline will display on all pages it is repeated on if this option
      is \texttt{\ true\ } . If this option is \texttt{\ false\ } , the
      header will repeat, but the blue hline will not.
    \end{itemize}
  \item
    \texttt{\ rtl\ } : if true, the table is horizontally flipped. That
    is, cells and lines are placed in the opposite order (starting from
    the right), and horizontal lines are flipped. This is meant to
    simulate the behavior of default Typst tables when
    \texttt{\ set\ text(dir:\ rtl)\ } is used, and is useful when
    writing in a language with a RTL (right-to-left) script. Defaults to
    \texttt{\ false\ } .
  \item
    \texttt{\ auto-lines\ } : Shorthand to apply a boolean to both
    \texttt{\ auto-hlines\ } and \texttt{\ auto-vlines\ } at the same
    time (overridable by each). Defaults to \texttt{\ true\ } .
  \item
    \texttt{\ auto-hlines\ } : If \texttt{\ true\ } , draw a horizontal
    line on every line where you did not manually draw one; if
    \texttt{\ false\ } , no hlines other than the ones you specify (via
    \texttt{\ hlinex\ } ) are drawn. Defaults to \texttt{\ auto\ }
    (follows \texttt{\ auto-lines\ } , which in turn defaults to
    \texttt{\ true\ } ).
  \item
    \texttt{\ auto-vlines\ } : If \texttt{\ true\ } , draw a vertical
    line on every line where you did not manually draw one; if
    \texttt{\ false\ } , no vlines other than the ones you specify (via
    \texttt{\ vlinex\ } ) are drawn. Defaults to \texttt{\ auto\ }
    (follows \texttt{\ auto-lines\ } , which in turn defaults to
    \texttt{\ true\ } ).
  \item
    \texttt{\ map-cells\ } : A function which takes a single
    \texttt{\ cellx\ } and returns another \texttt{\ cellx\ } , or a
    \texttt{\ content\ } which is converted to \texttt{\ cellx\ } by
    \texttt{\ cellx{[}\#content{]}\ } . You can customize the cell in
    pretty much any way using this function; just take care to avoid
    conflicting with already-placed cells if you move it.
  \item
    \texttt{\ map-hlines\ } : A function which takes each horizontal
    line object ( \texttt{\ hlinex\ } ) and returns another, optionally
    modifying its properties. You may also change its row position (
    \texttt{\ y\ } ). Note that this is also applied to lines generated
    by \texttt{\ auto-hlines\ } .
  \item
    \texttt{\ map-vlines\ } : A function which takes each horizontal
    line object ( \texttt{\ vlinex\ } ) and returns another, optionally
    modifying its properties. You may also change its column position (
    \texttt{\ x\ } ). Note that this is also applied to lines generated
    by \texttt{\ auto-vlines\ } .
  \item
    \texttt{\ map-rows\ } : A function mapping each row of cells to new
    values or modified properties. Takes
    \texttt{\ (row\_num,\ cell\_array)\ } and returns the modified
    \texttt{\ cell\_array\ } . Note that, with your function, they
    cannot be sent to another row. Also, please preserve the order of
    the cells. This is especially important given that cells may be
    \texttt{\ none\ } if they’re actually a position taken by another
    cell with colspan/rowspan. Make sure the \texttt{\ none\ } values
    are in the same indexes when the array is returned.
  \item
    \texttt{\ map-cols\ } : A function mapping each column of cells to
    new values or modified properties. Takes
    \texttt{\ (col\_num,\ cell\_array)\ } and returns the modified
    \texttt{\ cell\_array\ } . Note that, with your function, they
    cannot be sent to another column. Also, please preserve the order of
    the cells. This is especially important given that cells may be
    \texttt{\ none\ } if they’re actually a position taken by another
    cell with colspan/rowspan. Make sure the \texttt{\ none\ } values
    are in the same indexes when the array is returned.
  \item
    \texttt{\ fit-spans\ } : either a dictionary
    \texttt{\ (x:\ bool,\ y:\ bool)\ } or just \texttt{\ bool\ } (e.g.
    just \texttt{\ true\ } is converted to
    \texttt{\ (x:\ true,\ y:\ true)\ } ). When given
    \texttt{\ (x:\ true)\ } , colspans won’t affect the sizes of
    \texttt{\ auto\ } columns. When given \texttt{\ (y:\ true)\ } ,
    rowspans won’t affect the sizes of \texttt{\ auto\ } rows. By
    default, this is equal to \texttt{\ (x:\ false,\ y:\ false)\ }
    (equivalent to just \texttt{\ false\ } ), which means that colspans
    will cause the last spanned \texttt{\ auto\ } column to expand
    (depending on the contents of the cell) and rowspans will cause the
    last spanned \texttt{\ auto\ } row to expand similarly.

    \begin{itemize}
    \tightlist
    \item
      This is usually used as \texttt{\ (x:\ true)\ } to prevent
      unexpected expansion of \texttt{\ auto\ } columns after using a
      colspan, which can happen when a colspan spans both a
      fractional-size column (e.g. \texttt{\ 1fr\ } ) and an
      \texttt{\ auto\ } -sized column. Can be applied to rows too
      through \texttt{\ (y:\ true)\ } or
      \texttt{\ (x:\ true,\ y:\ true)\ } , if needed, however.
    \item
      The point of this option is to have colspans and rowspans not
      affect the size of the table at all, and just “fit� within the
      columns and rows they span. Therefore, this option does not have
      any effect upon colspans and rowspans which don’t span columns
      or rows with automatic size.
    \end{itemize}
  \end{itemize}
\end{enumerate}

\subsection{Changelog}\label{changelog}

\subsubsection{v0.0.9}\label{v0.0.9}

\textbf{NOTE:} Please use Typst’s built-in tables instead of tablex
(starting with Typst 0.11.0). Most of tablex’s features were
implemented in Typst’s tables by the author of tablex.

\begin{itemize}
\tightlist
\item
  Added compatibility with Typst v0.12.0 (
  \url{https://github.com/PgBiel/typst-tablex/issues/135} )
\item
  Added library usage notice to README
\item
  Tablex is now dual-licensed under MIT/Apache-2.0 (
  \url{https://github.com/PgBiel/typst-tablex/issues/134} )
\end{itemize}

\subsubsection{v0.0.8}\label{v0.0.8}

\begin{itemize}
\tightlist
\item
  Added \texttt{\ fit-spans\ } option to \texttt{\ tablex\ } and
  \texttt{\ cellx\ } (
  \url{https://github.com/PgBiel/typst-tablex/pull/111} )

  \begin{itemize}
  \tightlist
  \item
    Accepts \texttt{\ (x:\ bool,\ y:\ bool)\ } . When set to
    \texttt{\ (x:\ true)\ } , colspans won’t affect the sizes of
    \texttt{\ auto\ } columns. When set to \texttt{\ (y:\ true)\ } ,
    rowspans won’t affect the sizes of \texttt{\ auto\ } rows.
  \item
    Defaults to \texttt{\ false\ } , equivalent to
    \texttt{\ (x:\ false,\ y:\ false)\ } , that is, colspans and
    rowspans affect the sizes of \texttt{\ auto\ } tracks (columns and
    rows) by default (expanding the last spanned track if the
    colspan/rowspan is too large).
  \item
    Useful when you want merged cells (or a specific merged cell) to
    “fit� within their spanned columns and rows. May help when
    adding a colspan or rowspan causes an \texttt{\ auto\ } -sized track
    to inadvertently expand.
  \end{itemize}
\item
  \texttt{\ auto\ } column sizing received multiple improvements and bug
  fixes. Tables should now have more natural column widths. (
  \url{https://github.com/PgBiel/typst-tablex/pull/109} ,
  \url{https://github.com/PgBiel/typst-tablex/pull/116} )

  \begin{itemize}
  \tightlist
  \item
    Fixes some problems with overflowing cells (
    \url{https://github.com/PgBiel/typst-tablex/issues/48} ,
    \url{https://github.com/PgBiel/typst-tablex/issues/75} )
  \item
    Fixes \texttt{\ auto\ } columns being needlessly expanded in some
    cases ( \url{https://github.com/PgBiel/typst-tablex/issues/56} ,
    \url{https://github.com/PgBiel/typst-tablex/issues/78} )

    \begin{itemize}
    \tightlist
    \item
      For similar problems not fixed by this, please use the new
      \texttt{\ fit-spans\ } option as needed, or use fixed-size columns
      instead.
    \end{itemize}
  \end{itemize}
\item
  Several performance optimizations and other internal code improvements
  were made ( \url{https://github.com/PgBiel/typst-tablex/pull/113} ,
  \url{https://github.com/PgBiel/typst-tablex/pull/114} ,
  \url{https://github.com/PgBiel/typst-tablex/pull/115} ).

  \begin{itemize}
  \tightlist
  \item
    Documents with lots of \texttt{\ tablex\ } tables might now become
    \textbf{up to 20\% faster} to cold compile. Give it a shot!
  \end{itemize}
\item
  Fixed extra fixed-height rows appearing to have \texttt{\ auto\ }
  height ( \url{https://github.com/PgBiel/typst-tablex/pull/108} ).
\item
  Fixed rows without any visible cells being drawn with zero height (
  \url{https://github.com/PgBiel/typst-tablex/pull/107} ).

  \begin{itemize}
  \tightlist
  \item
    Fixes some rowspans causing cells to overlap (
    \url{https://github.com/PgBiel/typst-tablex/issues/82} ,
    \url{https://github.com/PgBiel/typst-tablex/issues/105} ).
  \end{itemize}
\end{itemize}

\subsubsection{v0.0.7}\label{v0.0.7}

I have begun
\href{https://github.com/PgBiel/typst-improv-tables-planning}{work on
bringing many tablex improvements to built-in Typst tables} ! In that
regard, \href{https://github.com/sponsors/PgBiel}{you can now sponsor my
work on tablex and improving Typst tables via GitHub Sponsors! Consider
taking a look :)}

\begin{itemize}
\tightlist
\item
  Allow gradients and patterns in fills (
  \url{https://github.com/PgBiel/typst-tablex/pull/87} )
\item
  Fixed a critical bug where \texttt{\ line\ } in tablex cells would
  misbehave ( \url{https://github.com/PgBiel/typst-tablex/issues/80} )

  \begin{itemize}
  \tightlist
  \item
    CeTZ and drawing in general should now work properly within tablex
    cells (see \url{https://github.com/johannes-wolf/cetz/issues/345} ).
  \item
    Also fixes a problem with nested tables (
    \url{https://github.com/PgBiel/typst-tablex/issues/34} )
  \end{itemize}
\item
  Fixed negative line expansion within a single cell (
  \url{https://github.com/PgBiel/typst-tablex/pull/84} )

  \begin{itemize}
  \tightlist
  \item
    Negative line expansion across multiple cells isn’t yet supported.
  \item
    Thanks GitHub user @dixslyf for the great work on fixing and testing
    this!
  \end{itemize}
\item
  Made internal length calculation procedures more robust (
  \url{https://github.com/PgBiel/typst-tablex/issues/92} ,
  \url{https://github.com/PgBiel/typst-tablex/issues/94} )

  \begin{itemize}
  \tightlist
  \item
    Fixes a potential incompatibility with (currently unreleased) Typst
    0.11.0
  \end{itemize}
\item
  Added missing support for boolean types in Typst 0.8.0+ (
  \url{https://github.com/PgBiel/typst-tablex/issues/73} )
\item
  Added some keywords to tablex’s \texttt{\ typst.toml\ } for better
  discoverability (
  \url{https://github.com/PgBiel/typst-tablex/issues/91} )
\end{itemize}

\subsubsection{v0.0.6}\label{v0.0.6}

\begin{itemize}
\tightlist
\item
  Added support for RTL tables with \texttt{\ rtl:\ true\ } (
  \url{https://github.com/PgBiel/typst-tablex/issues/58} ).

  \begin{itemize}
  \tightlist
  \item
    Default Typst tables are automatically flipped horizontally when
    using \texttt{\ set\ text(dir:\ rtl)\ } , however we can’t detect
    that setting from tablex at this moment (it isn’t currently
    possible to fetch set rules in Typst).
  \item
    Therefore, as a way around that, you can now specify
    \texttt{\ \#tablex(rtl:\ true,\ ...)\ } to flip your table
    horizontally if you’re writing a document in RTL (right-to-left)
    script. (You can use e.g. \texttt{\ \#let\ old-tablex\ =\ tablex\ }
    followed by
    \texttt{\ \#let\ tablex(..args)\ =\ old-tablex(rtl:\ true,\ ..args)\ }
    to not have to repeat the \texttt{\ rtl\ } parameter every time.)
  \end{itemize}
\item
  Added support for \texttt{\ box\ } ’s dictionary inset syntax on
  tablex ( \url{https://github.com/PgBiel/typst-tablex/issues/54} ).

  \begin{itemize}
  \tightlist
  \item
    For instance, you can now do
    \texttt{\ \#tablex(inset:\ (left:\ 5pt,\ top:\ 10pt,\ rest:\ 2pt),\ ...)\ }
    .
  \end{itemize}
\item
  Fixed errors when using floating point strokes or other more complex
  strokes ( \url{https://github.com/PgBiel/typst-tablex/issues/55} ).
\item
  Added full compatibility with the new Typst 0.8.0 type system (
  \url{https://github.com/PgBiel/typst-tablex/issues/69} ).
\item
  Added info about \texttt{\ \#rotate\ } problems to “Known Issues�
  in the README ( \url{https://github.com/PgBiel/typst-tablex/pull/60}
  ).
\item
  Improved docs for tablex options \texttt{\ columns\ } and
  \texttt{\ rows\ } (
  \url{https://github.com/PgBiel/typst-tablex/issues/53} ).
\end{itemize}

\subsubsection{v0.0.5}\label{v0.0.5}

\begin{itemize}
\tightlist
\item
  âš~ï¸? \textbf{Minimum Typst version raised to v0.2.0}
\item
  Improved calculation of page/container dimensions by using the
  \texttt{\ layout()\ } function.

  \begin{itemize}
  \tightlist
  \item
    Fixes tables with fractional columns not displaying properly in
    blocks with \texttt{\ auto\ } width (
    \url{https://github.com/PgBiel/typst-tablex/issues/44} ;
    \url{https://github.com/PgBiel/typst-tablex/issues/39} )
  \item
    Fixes some nested tables overflowing the page width (
    \url{https://github.com/PgBiel/typst-tablex/issues/41} )
  \item
    Fixes bad interaction between tables with fractional columns and
    nested tables (
    \url{https://github.com/PgBiel/typst-tablex/issues/28} )
  \item
    Fixes table rotation messing up table size calculation (
    \url{https://github.com/PgBiel/typst-tablex/issues/52} )
  \item
    Probably fixes other issues not listed here as well.
  \end{itemize}
\item
  Added some guards for infinite lengths and \texttt{\ auto\ } -sized
  pages ( \url{https://github.com/PgBiel/typst-tablex/issues/47} ).
\item
  Fixed tablex crashes/improper behavior with \texttt{\ em\ } strokes
  and other types of strokes (
  \url{https://github.com/PgBiel/typst-tablex/issues/49} ).
\item
  Added the tablex version number as a comment in the source file (as
  requested in \url{https://github.com/PgBiel/typst-tablex/issues/25} ).
\end{itemize}

\subsubsection{v0.0.4}\label{v0.0.4}

\begin{itemize}
\tightlist
\item
  Added \texttt{\ typst.toml\ } to support Typst v0.6.0’s
  soon-to-be-released package manager (see
  \url{https://github.com/PgBiel/typst-tablex/issues/22} ).
\item
  Fixed a division by zero regression from v0.0.3 (
  \url{https://github.com/PgBiel/typst-tablex/issues/19} ).
\item
  Fixed a bug where cells placed in arbitrary positions could force an
  extra empty row to appear (
  \url{https://github.com/PgBiel/typst-tablex/issues/16} ).
\item
  Fixed \texttt{\ hlinex(gutter-restrict:\ top)\ } causing the hline to
  just disappear (
  \url{https://github.com/PgBiel/typst-tablex/issues/20} ).
\item
  Fixed certain \texttt{\ gutter-restrict\ } lines disappearing when
  there’s no gutter (
  \url{https://github.com/PgBiel/typst-tablex/issues/21} ).
\item
  Fixed row gutter lines not properly splitting across pages (
  \url{https://github.com/PgBiel/typst-tablex/issues/23} ).
\end{itemize}

\subsubsection{v0.0.3}\label{v0.0.3}

\begin{itemize}
\tightlist
\item
  Added support for Typst v0.4.0 and v0.5.0.

  \begin{itemize}
  \tightlist
  \item
    The tablex options \texttt{\ fill:\ } and \texttt{\ align:\ } now
    accept arrays of values for each column (
    \url{https://github.com/PgBiel/typst-tablex/issues/13} ).

    \begin{itemize}
    \tightlist
    \item
      For example, \texttt{\ fill:\ (red,\ blue)\ } would fill the first
      column with red, the second column with blue, and any further
      columns would alternate between the two fill colors.
    \end{itemize}
  \end{itemize}
\item
  Fixed the calculation of the size of \texttt{\ auto\ } rows and
  columns when a rowspan or colspan was used (
  \url{https://github.com/PgBiel/typst-tablex/issues/11} ).
\item
  Fixed the calculation of the size of the last \texttt{\ auto\ } column
  when it was too long (
  \url{https://github.com/PgBiel/typst-tablex/issues/6} ).
\end{itemize}

\subsubsection{v0.0.2}\label{v0.0.2}

\begin{itemize}
\tightlist
\item
  Added support for Typst v0.3.0.
\item
  Fixed strokes - now lines will expand to not look weird when strokes
  are larger.

  \begin{itemize}
  \tightlist
  \item
    You can disable this behavior by setting
    \texttt{\ stroke-expand:\ false\ } on your lines.
  \end{itemize}
\item
  You can now arbitrarily change your lines’ sizes at either end with
  the option \texttt{\ expand:\ (length,\ length)\ } ; e.g.
  \texttt{\ expand:\ (5pt,\ 10pt)\ } will increase your horizontal line
  5pt to the left and 10pt to the right (or, for a vertical line, 5pt to
  the top and 10pt to the bottom).

  \begin{itemize}
  \tightlist
  \item
    Support for negative expand lengths is limited (so far, only reduces
    length in the first cell the line spans).
  \end{itemize}
\item
  Added some gutter fixes (not all gutter issues were fixed yet).
\end{itemize}

\subsubsection{v0.0.1}\label{v0.0.1}

Initial release.

\begin{itemize}
\tightlist
\item
  Added types \texttt{\ tablex\ } , \texttt{\ cellx\ } ,
  \texttt{\ hlinex\ } , \texttt{\ vlinex\ }
\item
  Added type aliases \texttt{\ gridx\ } , \texttt{\ rowspanx\ } ,
  \texttt{\ colspanx\ }
\end{itemize}

\subsection{0.1.0 Roadmap}\label{roadmap}

\begin{itemize}
\tightlist
\item
  {[} {]} General

  \begin{itemize}
  \tightlist
  \item
    {[}X{]} More docs
  \item
    {[} {]} Code cleanup
  \item
    {[} {]} Table drawing rework
  \end{itemize}
\item
  {[} {]} \texttt{\ \#table\ } parity

  \begin{itemize}
  \tightlist
  \item
    {[}X{]} \texttt{\ columns:\ } , \texttt{\ rows:\ }

    \begin{itemize}
    \tightlist
    \item
      {[}X{]} Basic support
    \item
      {[}X{]} Accept a single size to mean a single column
    \item
      {[}X{]} Adjust \texttt{\ auto\ } columns and rows
    \item
      {[}X{]} Accept integers to mean multiple \texttt{\ auto\ }
    \item
      {[}X{]} Basic unit conversion (em -\textgreater{} pt, etc.)
    \item
      {[}X{]} Ratio unit conversion (100\% -\textgreater{} page
      width…)
    \item
      {[}X{]} Fractional unit conversion based on available space (1fr,
      2fr -\textgreater{} 1/3, 2/3)
    \item
      {[}X{]} Shrink \texttt{\ auto\ } columns based on available space
    \end{itemize}
  \item
    {[}X{]} \texttt{\ fill\ }

    \begin{itemize}
    \tightlist
    \item
      {[}X{]} Basic support ( \texttt{\ color\ } for general fill)
    \item
      {[}X{]} Accept a function (
      \texttt{\ (column,\ row)\ =\textgreater{}\ color\ } )
    \item
      {[}X{]} Accept an array of colors (one for each column)
    \end{itemize}
  \item
    {[}X{]} \texttt{\ align\ }

    \begin{itemize}
    \tightlist
    \item
      {[}X{]} Basic support ( \texttt{\ alignment\ } and
      \texttt{\ 2d\ alignment\ } apply to all cells)
    \item
      {[}X{]} Accept a function (
      \texttt{\ (column,\ row)\ =\textgreater{}\ alignment/2d\ alignment\ }
      )
    \item
      {[}X{]} Accept an array of alignment values (one for each column)
    \end{itemize}
  \item
    {[}X{]} \texttt{\ inset\ }
  \item
    {[} {]} \texttt{\ gutter\ }

    \begin{itemize}
    \tightlist
    \item
      {[}X{]} Basic support

      \begin{itemize}
      \tightlist
      \item
        {[}X{]} \texttt{\ column-gutter\ }
      \item
        {[}X{]} \texttt{\ row-gutter\ }
      \end{itemize}
    \item
      {[} {]} Hline, vline adaptations

      \begin{itemize}
      \tightlist
      \item
        {[}X{]} \texttt{\ stop-pre-gutter\ } : Makes the hline/vline not
        transpose gutter boundaries
      \item
        {[}X{]} \texttt{\ gutter-restrict\ } : Makes the hline/vline not
        draw on both sides of a gutter boundary, and instead pick one
        (top/bottom; left/right)
      \item
        {[} {]} Properly work with gutters after colspanxs/rowspanxs
      \end{itemize}
    \end{itemize}
  \item
    {[}X{]} \texttt{\ stroke\ }

    \begin{itemize}
    \tightlist
    \item
      {[}X{]} Basic support (change all lines, vline or hline, without
      override)
    \item
      {[}X{]} \texttt{\ none\ } for no stroke
    \end{itemize}
  \item
    {[}X{]} Default to lines on every row and column
  \end{itemize}
\item
  {[} {]} New features for \texttt{\ \#tablex\ }

  \begin{itemize}
  \tightlist
  \item
    {[}X{]} Basic types ( \texttt{\ cellx\ } , \texttt{\ hlinex\ } ,
    \texttt{\ vlinex\ } )
  \item
    {[}X{]} \texttt{\ hlinex\ } , \texttt{\ vlinex\ }

    \begin{itemize}
    \tightlist
    \item
      {[}X{]} Auto-positioning when placed among cells
    \item
      {[}X{]} Arbitrary positioning
    \item
      {[}X{]} Allow customizing \texttt{\ stroke\ }
    \end{itemize}
  \item
    {[}X{]} \texttt{\ colspanx\ } , \texttt{\ rowspanx\ }

    \begin{itemize}
    \tightlist
    \item
      {[}X{]} Interrupt \texttt{\ hlinex\ } and \texttt{\ vlinex\ } with
      \texttt{\ end:\ auto\ }
    \item
      {[}X{]} Support simultaneous col/rowspan with
      \texttt{\ cellx(colspanx:,\ rowspanx:)\ }
    \item
      {[}X{]} Support nesting colspan/rowspan (
      \texttt{\ colspanx(rowspanx())\ } )
    \item
      {[}X{]} Support cell attributes (e.g.
      \texttt{\ colspanx(2,\ align:\ left){[}a{]}\ } )
    \item
      {[}X{]} Reliably detect conflicts
    \end{itemize}
  \item
    {[} {]} Repeating headers

    \begin{itemize}
    \tightlist
    \item
      {[}X{]} Basic support (first row group repeats on every page)
    \item
      {[} {]} Work with different page sizes
    \item
      {[}X{]} \texttt{\ repeat-header\ } : Control header repetition

      \begin{itemize}
      \tightlist
      \item
        {[}X{]} \texttt{\ true\ } : Repeat on all pages
      \item
        {[}X{]} integer: Repeat for the next ‘n’ pages
      \item
        {[}X{]} array of integers: Repeat on those (relative) pages
      \item
        {[}X{]} \texttt{\ false\ } (default): Do not repeat
      \end{itemize}
    \item
      {[}X{]} \texttt{\ header-rows\ } : Indicate what to consider as a
      “header�

      \begin{itemize}
      \tightlist
      \item
        {[}X{]} integer: At least first ‘n’ rows are a header (plus
        whatever rowspanxs show up there)

        \begin{itemize}
        \tightlist
        \item
          {[}X{]} Defaults to 1
        \end{itemize}
      \item
        {[}X{]} \texttt{\ none\ } or \texttt{\ 0\ } : no header
        (disables header repetition regardless of
        \texttt{\ repeat-header\ } )
      \end{itemize}
    \end{itemize}
  \item
    {[}X{]} \texttt{\ cellx\ }

    \begin{itemize}
    \tightlist
    \item
      {[}X{]} Auto-positioning based on order and columns
    \item
      {[}X{]} Place empty cells when there are too many
    \item
      {[}X{]} Allow arbitrary positioning with
      \texttt{\ cellx(x:,\ y:)\ }
    \item
      {[}X{]} Allow \texttt{\ align\ } override
    \item
      {[}X{]} Allow \texttt{\ fill\ } override
    \item
      {[}X{]} Allow \texttt{\ inset\ } override

      \begin{itemize}
      \tightlist
      \item
        {[}X{]} Works properly only with \texttt{\ auto\ } cols/rows
      \end{itemize}
    \item
      {[}X{]} Dynamic content (maybe shortcut for \texttt{\ map-cells\ }
      on a single cell)
    \end{itemize}
  \item
    {[}X{]} Auto-lines

    \begin{itemize}
    \tightlist
    \item
      {[}X{]} \texttt{\ auto-hlines\ } - \texttt{\ true\ } to place on
      all lines without hlines, \texttt{\ false\ } otherwise
    \item
      {[}X{]} \texttt{\ auto-vlines\ } - similar
    \item
      {[}X{]} \texttt{\ auto-lines\ } - controls both simultaneously
      (defaults to \texttt{\ true\ } )
    \end{itemize}
  \item
    {[}X{]} Iteration attributes

    \begin{itemize}
    \tightlist
    \item
      {[}X{]} \texttt{\ map-cells\ } - Customize every single cell
    \item
      {[}X{]} \texttt{\ map-hlines\ } - Customize each horizontal line
    \item
      {[}X{]} \texttt{\ map-vlines\ } - Customize each vertical line
    \item
      {[}X{]} \texttt{\ map-rows\ } - Customize entire rows of cells
    \item
      {[}X{]} \texttt{\ map-cols\ } - Customize entire columns of cells
    \end{itemize}
  \end{itemize}
\end{itemize}

\subsection{License}\label{license}

Tablex is licensed under MIT or Apache-2.0, at your option (see the
files \texttt{\ LICENSE-MIT\ } and \texttt{\ LICENSE-APACHE\ } ).

\subsubsection{How to add}\label{how-to-add}

Copy this into your project and use the import as \texttt{\ tablex\ }

\begin{verbatim}
#import "@preview/tablex:0.0.9"
\end{verbatim}

\includesvg[width=0.16667in,height=0.16667in]{/assets/icons/16-copy.svg}

Check the docs for
\href{https://typst.app/docs/reference/scripting/\#packages}{more
information on how to import packages} .

\subsubsection{About}\label{about}

\begin{description}
\tightlist
\item[Author :]
\href{https://github.com/PgBiel}{PgBiel}
\item[License:]
MIT OR Apache-2.0
\item[Current version:]
0.0.9
\item[Last updated:]
October 25, 2024
\item[First released:]
June 29, 2023
\item[Archive size:]
48.1 kB
\href{https://packages.typst.org/preview/tablex-0.0.9.tar.gz}{\pandocbounded{\includesvg[keepaspectratio]{/assets/icons/16-download.svg}}}
\item[Repository:]
\href{https://github.com/PgBiel/typst-tablex}{GitHub}
\end{description}

\subsubsection{Where to report issues?}\label{where-to-report-issues}

This package is a project of PgBiel . Report issues on
\href{https://github.com/PgBiel/typst-tablex}{their repository} . You
can also try to ask for help with this package on the
\href{https://forum.typst.app}{Forum} .

Please report this package to the Typst team using the
\href{https://typst.app/contact}{contact form} if you believe it is a
safety hazard or infringes upon your rights.

\phantomsection\label{versions}
\subsubsection{Version history}\label{version-history}

\begin{longtable}[]{@{}ll@{}}
\toprule\noalign{}
Version & Release Date \\
\midrule\noalign{}
\endhead
\bottomrule\noalign{}
\endlastfoot
0.0.9 & October 25, 2024 \\
\href{https://typst.app/universe/package/tablex/0.0.8/}{0.0.8} & January
12, 2024 \\
\href{https://typst.app/universe/package/tablex/0.0.7/}{0.0.7} &
December 19, 2023 \\
\href{https://typst.app/universe/package/tablex/0.0.6/}{0.0.6} & October
21, 2023 \\
\href{https://typst.app/universe/package/tablex/0.0.5/}{0.0.5} & August
19, 2023 \\
\href{https://typst.app/universe/package/tablex/0.0.4/}{0.0.4} & June
29, 2023 \\
\end{longtable}

Typst GmbH did not create this package and cannot guarantee correct
functionality of this package or compatibility with any version of the
Typst compiler or app.


\section{Package List LaTeX/bookletic.tex}
\title{typst.app/universe/package/bookletic}

\phantomsection\label{banner}
\section{bookletic}\label{bookletic}

{ 0.3.0 }

Create beautiful booklets with ease.

\phantomsection\label{readme}
Create beautiful booklets with ease.

The current version of this library (0.3.0) contains a single function
to take in an array of content blocks and order them into a ready to
print booklet, bulletin, etc. No need to fight with printer settings or
document converters.

\subsubsection{Example Output}\label{example-output}

Here is an example of the output generated by the \texttt{\ sig\ }
function (short for a book’s signature) with default parameters and
some sample content:

\pandocbounded{\includegraphics[keepaspectratio]{https://github.com/typst/packages/raw/main/packages/preview/bookletic/0.3.0/example/basic.png}}

Here is an example with some customization applied:

\pandocbounded{\includegraphics[keepaspectratio]{https://github.com/typst/packages/raw/main/packages/preview/bookletic/0.3.0/example/fancy.png}}

\subsection{\texorpdfstring{\texttt{\ sig\ }
Function}{ sig  Function}}\label{sig-function}

The \texttt{\ sig\ } function is used to create a signature (booklet)
layout from provided content. It takes various parameters to
automatically configure the layout.

\subsubsection{Parameters}\label{parameters}

\begin{itemize}
\tightlist
\item
  \texttt{\ page\_margin\_binding\ } : The binding margin for each page
  in the booklet (space between pages).
\item
  \texttt{\ page\_border\ } : Takes a color space value to draw a border
  around each page. If set to none no border will be drawn.
\item
  \texttt{\ draft\ } : A boolean value indicating whether to output an
  unordered draft or final layout.
\item
  \texttt{\ p-num-layout\ } : A configuration for page numbering styles,
  allowing multiple layouts that apply to specified page ranges. Each
  layout can be provided as a dictonary specifying the following
  options:

  \begin{itemize}
  \tightlist
  \item
    \texttt{\ p-num-start\ } : Starting page number for this layout
  \item
    \texttt{\ p-num-alt-start\ } : Alternative starting page number
    (e.g., for chapters)
  \item
    \texttt{\ p-num-pattern\ } : Pattern for page numbering (e.g.,
    \texttt{\ "1"\ } , \texttt{\ "i"\ } , \texttt{\ "a"\ } ,
    \texttt{\ "A"\ } )
  \item
    \texttt{\ p-num-placment\ } : Placement of page numbers (
    \texttt{\ top\ } or \texttt{\ bottom\ } )
  \item
    \texttt{\ p-num-align-horizontal\ } : Horizontal alignment of page
    numbers ( \texttt{\ left\ } , \texttt{\ center\ } , or
    \texttt{\ right\ } )
  \item
    \texttt{\ p-num-align-vertical\ } : Vertical alignment of page
    numbers ( \texttt{\ top\ } , \texttt{\ horizon\ } , or
    \texttt{\ bottom\ } )
  \item
    \texttt{\ p-num-pad-left\ } : Extra padding added to the left of the
    page number
  \item
    \texttt{\ p-num-pad-horizontal\ } : Horizontal padding for page
    numbers
  \item
    \texttt{\ p-num-size\ } : Size of page numbers
  \item
    \texttt{\ p-num-border\ } : The border color for the page numbers.
    If set to none no border will be drawn.
  \item
    \texttt{\ p-num-halign-alternate\ } : A boolean for whether to
    alternate horizontal alignment between left and right pages.
  \end{itemize}
\item
  \texttt{\ pad\_content\ } : The padding around the page content.
\item
  \texttt{\ contents\ } : The content to be laid out in the booklet.
  This should be an array of blocks.
\end{itemize}

\subsubsection{Usage}\label{usage}

To use the \texttt{\ sig\ } function, first set your desired page
settings using the native page function. Then simply call the sig
function with the desired parameters and provide the content to be laid
out in the booklet:

\begin{Shaded}
\begin{Highlighting}[]
\NormalTok{\#set page(flipped: true, paper: "us{-}letter")}
\NormalTok{\#bookletic.sig(}
\NormalTok{  contents: [}
\NormalTok{    ["Page 1 content"],}
\NormalTok{    ["Page 2 content"],}
\NormalTok{    ["Page 3 content"],}
\NormalTok{    ["Page 4 content"],}
\NormalTok{  ],}
\NormalTok{)}
\end{Highlighting}
\end{Shaded}

This will create a signature layout with the provided content, using the
default values for the other parameters.

You can customize the layout by passing different values for the various
parameters. For example:

\begin{Shaded}
\begin{Highlighting}[]
\NormalTok{\#set page(flipped: true, paper: "us{-}legal", margin: (top: 1in, bottom: 1in, left: 1in, right: 1in))}
\NormalTok{\#bookletic.sig(}
\NormalTok{  page\_margin\_binding: 0.5in,}
\NormalTok{  page\_border: none,}
\NormalTok{  draft: true,}
\NormalTok{  p{-}num{-}layout: (}
\NormalTok{    bookletic.num{-}layout(}
\NormalTok{      p{-}num{-}start: 1,}
\NormalTok{      p{-}num{-}alt{-}start: none,}
\NormalTok{      p{-}num{-}pattern: "\textasciitilde{} 1 \textasciitilde{}", }
\NormalTok{      p{-}num{-}placment: bottom,}
\NormalTok{      p{-}num{-}align{-}horizontal: right,}
\NormalTok{      p{-}num{-}align{-}vertical: horizon,}
\NormalTok{      p{-}num{-}pad{-}left: {-}5pt,}
\NormalTok{      p{-}num{-}pad{-}horizontal: 0pt,}
\NormalTok{      p{-}num{-}size: 16pt,}
\NormalTok{      p{-}num{-}border: rgb("\#ff4136"),}
\NormalTok{      p{-}num{-}halign{-}alternate: false,}
\NormalTok{    ),}
\NormalTok{  ),}
\NormalTok{  pad\_content: 10pt,}
\NormalTok{  contents: (}
\NormalTok{    ["Page 1 content"],}
\NormalTok{    ["Page 2 content"],}
\NormalTok{    ["Page 3 content"],}
\NormalTok{    ["Page 4 content"],}
\NormalTok{  ),}
\NormalTok{)}
\end{Highlighting}
\end{Shaded}

This will create an unordered draft signature layout with US Legal paper
size, larger margins, no page borders, page numbers at the bottom right
corner with a red border, and more padding around the content.

\subsubsection{Notes}\label{notes}

\begin{itemize}
\tightlist
\item
  The \texttt{\ sig\ } function is currently hardcoded to only handle
  two-page single-fold signatures. Other more complicated signatures may
  be supported in the future.
\item
  The \texttt{\ num-layout\ } function is a helper to create page number
  layouts with default values.
\item
  The \texttt{\ booklet\ } function is a placeholder for automatically
  breaking a single content block into pages dynamically. It is not
  implemented yet but will be added in coming versions.
\end{itemize}

\subsection{Collaboration}\label{collaboration}

I would love to see this package eventually turn into a community
effort. So any interest in collaboration is very welcome! You can find
the github repository for this library here:
\href{https://github.com/harrellbm/Bookletic}{Bookletic Repo} . Feel
free to file an issue, pull request, or start a discussion.

\subsection{Changlog}\label{changlog}

\paragraph{0.3.0}\label{section}

\begin{itemize}
\tightlist
\item
  Remove internal dependency on native page function. This allows the
  user to set the page function separately with full control over paper
  type, outer margins and everything else defined by the native page
  function.
\item
  Add p-num-halign-alternate to page number layout allowing setting page
  numbers to alternate on facing pages making it possible to place page
  numbers along the outside or inside edges of facing pages.
\item
  Internal improvements for ordering algorithm.
\item
  Add \texttt{\ num-layout\ } function helper.
\end{itemize}

\paragraph{0.2.0}\label{section-1}

\begin{itemize}
\tightlist
\item
  Handle odd number of pages by inserting a blank back cover
\item
  Implements page number layouts to allow defining different page
  numbers for different page ranges.
\item
  Add various other page number options
\end{itemize}

\paragraph{0.1.0}\label{section-2}

Initial Commit

\subsubsection{How to add}\label{how-to-add}

Copy this into your project and use the import as \texttt{\ bookletic\ }

\begin{verbatim}
#import "@preview/bookletic:0.3.0"
\end{verbatim}

\includesvg[width=0.16667in,height=0.16667in]{/assets/icons/16-copy.svg}

Check the docs for
\href{https://typst.app/docs/reference/scripting/\#packages}{more
information on how to import packages} .

\subsubsection{About}\label{about}

\begin{description}
\tightlist
\item[Author s :]
\href{https://github.com/harrellbm}{Brenden Harrell} \&
\href{https://github.com/paul2t}{Paul DE TEMMERMAN}
\item[License:]
Apache-2.0
\item[Current version:]
0.3.0
\item[Last updated:]
October 10, 2024
\item[First released:]
May 8, 2024
\item[Minimum Typst version:]
0.11.0
\item[Archive size:]
7.91 kB
\href{https://packages.typst.org/preview/bookletic-0.3.0.tar.gz}{\pandocbounded{\includesvg[keepaspectratio]{/assets/icons/16-download.svg}}}
\item[Repository:]
\href{https://github.com/harrellbm/Bookletic.git}{GitHub}
\item[Categor ies :]
\begin{itemize}
\tightlist
\item[]
\item
  \pandocbounded{\includesvg[keepaspectratio]{/assets/icons/16-docs.svg}}
  \href{https://typst.app/universe/search/?category=book}{Book}
\item
  \pandocbounded{\includesvg[keepaspectratio]{/assets/icons/16-layout.svg}}
  \href{https://typst.app/universe/search/?category=layout}{Layout}
\item
  \pandocbounded{\includesvg[keepaspectratio]{/assets/icons/16-map.svg}}
  \href{https://typst.app/universe/search/?category=flyer}{Flyer}
\end{itemize}
\end{description}

\subsubsection{Where to report issues?}\label{where-to-report-issues}

This package is a project of Brenden Harrell and Paul DE TEMMERMAN .
Report issues on \href{https://github.com/harrellbm/Bookletic.git}{their
repository} . You can also try to ask for help with this package on the
\href{https://forum.typst.app}{Forum} .

Please report this package to the Typst team using the
\href{https://typst.app/contact}{contact form} if you believe it is a
safety hazard or infringes upon your rights.

\phantomsection\label{versions}
\subsubsection{Version history}\label{version-history}

\begin{longtable}[]{@{}ll@{}}
\toprule\noalign{}
Version & Release Date \\
\midrule\noalign{}
\endhead
\bottomrule\noalign{}
\endlastfoot
0.3.0 & October 10, 2024 \\
\href{https://typst.app/universe/package/bookletic/0.2.0/}{0.2.0} & May
23, 2024 \\
\href{https://typst.app/universe/package/bookletic/0.1.0/}{0.1.0} & May
8, 2024 \\
\end{longtable}

Typst GmbH did not create this package and cannot guarantee correct
functionality of this package or compatibility with any version of the
Typst compiler or app.


\section{Package List LaTeX/metro.tex}
\title{typst.app/universe/package/metro}

\phantomsection\label{banner}
\section{metro}\label{metro}

{ 0.3.0 }

Typset units and numbers with options.

\phantomsection\label{readme}
The Metro package aims to be a port of the Latex package siunitx. It
allows easy typesetting of numbers and units with options. This package
is very early in development and many features are missing, so any
feature requests or bug reports are welcome!

Metro’s name comes from Metrology, the study scientific study of
measurement.

\textbf{Bug reports, feature requests, and PRs are welcome!}

\subsection{Usage}\label{usage}

Requires Typst v0.11.0+. Use Typst’s package manager:

\begin{verbatim}
#import "@preview/metro:0.3.0": *
\end{verbatim}

You can also download the \texttt{\ src\ } folder and import
\texttt{\ lib.typ\ } and import:

\begin{verbatim}
#import "src/lib.typ": *
\end{verbatim}

See the manual for more detailed information:
\href{https://github.com/typst/packages/raw/main/packages/preview/metro/0.3.0/manual.pdf}{manual.pdf}

\subsection{Future Features (in no particular
order)}\label{future-features-in-no-particular-order}

\begin{itemize}
\tightlist
\item
  {[}x{]} Angles
\item
  {[}x{]} Complex numbers
\item
  {[}x{]} Ranges, lists and products
\item
  {[} {]} table extensions?
\item
  {[} {]} Number parsing

  \begin{itemize}
  \tightlist
  \item
    {[} {]} Uncertainties
  \item
    {[}x{]} Exponents
  \end{itemize}
\item
  {[}x{]} Number post-processing

  \begin{itemize}
  \tightlist
  \item
    {[}x{]} rounding
  \item
    {[}x{]} exponent modes
  \end{itemize}
\end{itemize}

\subsubsection{How to add}\label{how-to-add}

Copy this into your project and use the import as \texttt{\ metro\ }

\begin{verbatim}
#import "@preview/metro:0.3.0"
\end{verbatim}

\includesvg[width=0.16667in,height=0.16667in]{/assets/icons/16-copy.svg}

Check the docs for
\href{https://typst.app/docs/reference/scripting/\#packages}{more
information on how to import packages} .

\subsubsection{About}\label{about}

\begin{description}
\tightlist
\item[Author s :]
\href{https://github.com/fenjalien}{fenjalien} \&
\href{https://github.com/Mc-Zen}{Mc-Zen}
\item[License:]
Apache-2.0
\item[Current version:]
0.3.0
\item[Last updated:]
May 27, 2024
\item[First released:]
August 22, 2023
\item[Minimum Typst version:]
0.11.0
\item[Archive size:]
17.5 kB
\href{https://packages.typst.org/preview/metro-0.3.0.tar.gz}{\pandocbounded{\includesvg[keepaspectratio]{/assets/icons/16-download.svg}}}
\item[Repository:]
\href{https://github.com/fenjalien/metro}{GitHub}
\item[Categor ies :]
\begin{itemize}
\tightlist
\item[]
\item
  \pandocbounded{\includesvg[keepaspectratio]{/assets/icons/16-chart.svg}}
  \href{https://typst.app/universe/search/?category=visualization}{Visualization}
\item
  \pandocbounded{\includesvg[keepaspectratio]{/assets/icons/16-text.svg}}
  \href{https://typst.app/universe/search/?category=text}{Text}
\end{itemize}
\end{description}

\subsubsection{Where to report issues?}\label{where-to-report-issues}

This package is a project of fenjalien and Mc-Zen . Report issues on
\href{https://github.com/fenjalien/metro}{their repository} . You can
also try to ask for help with this package on the
\href{https://forum.typst.app}{Forum} .

Please report this package to the Typst team using the
\href{https://typst.app/contact}{contact form} if you believe it is a
safety hazard or infringes upon your rights.

\phantomsection\label{versions}
\subsubsection{Version history}\label{version-history}

\begin{longtable}[]{@{}ll@{}}
\toprule\noalign{}
Version & Release Date \\
\midrule\noalign{}
\endhead
\bottomrule\noalign{}
\endlastfoot
0.3.0 & May 27, 2024 \\
\href{https://typst.app/universe/package/metro/0.2.0/}{0.2.0} & February
7, 2024 \\
\href{https://typst.app/universe/package/metro/0.1.1/}{0.1.1} &
September 22, 2023 \\
\href{https://typst.app/universe/package/metro/0.1.0/}{0.1.0} & August
22, 2023 \\
\end{longtable}

Typst GmbH did not create this package and cannot guarantee correct
functionality of this package or compatibility with any version of the
Typst compiler or app.


\section{Package List LaTeX/lasaveur.tex}
\title{typst.app/universe/package/lasaveur}

\phantomsection\label{banner}
\section{lasaveur}\label{lasaveur}

{ 0.1.3 }

Porting vim-latex\textquotesingle s math shorthands to Typst. An
accommendating vim syntax file is provided in the repo.

\phantomsection\label{readme}
This is a Typst package for speedy mathematical input, inspired by
\href{https://github.com/vim-latex/vim-latex}{vim-latex} . This project
is named after my Vim plugin
\href{https://github.com/yangwenbo99/vimtex-lasaveur}{vimtex-lasaveur} ,
which ports the operations in vim-latex to
\href{https://github.com/lervag/vimtex}{vimtex} .

\subsection{Usages in Typst}\label{usages-in-typst}

Either use the file released in “Releases� or import using the
following command:

\begin{Shaded}
\begin{Highlighting}[]
\NormalTok{\#import "@preview/lasaveur:0.1.3": *}
\end{Highlighting}
\end{Shaded}

This script generates a Typst library that defines shorthand commands
for various mathematical symbols and functions. Here’s an overview of
what it provides and how a user can use it:

\begin{enumerate}
\tightlist
\item
  Mathematical Functions:

  \begin{itemize}
  \tightlist
  \item
    Usage: \texttt{\ f\textless{}key\textgreater{}(argument)\ }
  \item
    Examples: \texttt{\ fh(x)\ } for hat, \texttt{\ ft(x)\ } for tilde,
    \texttt{\ f2(x)\ } for square root
  \end{itemize}
\item
  Font Styles:

  \begin{itemize}
  \tightlist
  \item
    Usage: \texttt{\ f\textless{}key\textgreater{}(argument)\ }
  \item
    Examples: \texttt{\ fb(x)\ } for bold, \texttt{\ fbb(x)\ } for
    blackboard bold, \texttt{\ fca(x)\ } for calligraphic
  \end{itemize}
\item
  Greek Letters:

  \begin{itemize}
  \tightlist
  \item
    Usage: \texttt{\ k\textless{}key\textgreater{}\ }
  \item
    Examples: \texttt{\ ka\ } for α (alpha), \texttt{\ kb\ } for β
    (beta), \texttt{\ kG\ } for Î`` (capital Gamma)
  \end{itemize}
\item
  Common Mathematical Symbols:

  \begin{itemize}
  \tightlist
  \item
    Usage: \texttt{\ g\textless{}key\textgreater{}\ }
  \item
    Examples: \texttt{\ g8\ } for ∞ (infinity), \texttt{\ gU\ } for
    ∪ (union), \texttt{\ gI\ } for ∩ (intersection)
  \end{itemize}
\item
  LaTeX-compatible Symbols:

  \begin{itemize}
  \tightlist
  \item
    Usage: Direct LaTeX command names
  \item
    Examples: \texttt{\ partial\ } for ∂, \texttt{\ infty\ } for ∞,
    \texttt{\ cdot\ } for â‹
  \end{itemize}
\item
  Arrows:

  \begin{itemize}
  \tightlist
  \item
    Usage: \texttt{\ ar.\textless{}key\textgreater{}\ }
  \item
    Examples: \texttt{\ ar.l\ } for â†?, \texttt{\ ar.r\ } for â†',
    \texttt{\ ar.lr\ } for â†''
  \end{itemize}
\end{enumerate}

Users can employ these shorthands in their Typst documents to quickly
input mathematical symbols and functions. The exact prefix for each
category (like \texttt{\ f\ } for functions or \texttt{\ k\ } for Greek
letters) can be customized using command-line arguments when running the
script.

For instance, in a Typst document, after importing the generated
library, a user could write:

\begin{Shaded}
\begin{Highlighting}[]
\NormalTok{$fh(x) + ka + g8 + ar.r$}
\end{Highlighting}
\end{Shaded}

This would produce: xÌ‚ + α + ∞ + â†'

The script provides a wide range of symbols covering most common
mathematical notations, making it easier and faster to type complex
mathematical expressions in Typst â€`` especially for users migrating
from vim-latex.

\subsection{Accompanying Vim Syntax
File}\label{accompanying-vim-syntax-file}

The syntax file provides more advanced and correct concealing for both
Typst’s built-in math syntax and the lasaveur shorthands. Download the
syntax file from the “Releases� section and place it in your
\texttt{\ \textasciitilde{}/.vim/after/syntax/\ } directory. The
\texttt{\ syntax.vim\ } file in the repo is supposed to be used by the
generation script and it \emph{will not work} if directly sourced in
Vim.

\subsubsection{How to add}\label{how-to-add}

Copy this into your project and use the import as \texttt{\ lasaveur\ }

\begin{verbatim}
#import "@preview/lasaveur:0.1.3"
\end{verbatim}

\includesvg[width=0.16667in,height=0.16667in]{/assets/icons/16-copy.svg}

Check the docs for
\href{https://typst.app/docs/reference/scripting/\#packages}{more
information on how to import packages} .

\subsubsection{About}\label{about}

\begin{description}
\tightlist
\item[Author :]
\href{https://github.com/yangwenbo99}{Paul Yang}
\item[License:]
MIT
\item[Current version:]
0.1.3
\item[Last updated:]
August 22, 2024
\item[First released:]
August 22, 2024
\item[Archive size:]
2.25 kB
\href{https://packages.typst.org/preview/lasaveur-0.1.3.tar.gz}{\pandocbounded{\includesvg[keepaspectratio]{/assets/icons/16-download.svg}}}
\item[Repository:]
\href{https://github.com/yangwenbo99/typst-lasaveur}{GitHub}
\item[Categor y :]
\begin{itemize}
\tightlist
\item[]
\item
  \pandocbounded{\includesvg[keepaspectratio]{/assets/icons/16-hammer.svg}}
  \href{https://typst.app/universe/search/?category=utility}{Utility}
\end{itemize}
\end{description}

\subsubsection{Where to report issues?}\label{where-to-report-issues}

This package is a project of Paul Yang . Report issues on
\href{https://github.com/yangwenbo99/typst-lasaveur}{their repository} .
You can also try to ask for help with this package on the
\href{https://forum.typst.app}{Forum} .

Please report this package to the Typst team using the
\href{https://typst.app/contact}{contact form} if you believe it is a
safety hazard or infringes upon your rights.

\phantomsection\label{versions}
\subsubsection{Version history}\label{version-history}

\begin{longtable}[]{@{}ll@{}}
\toprule\noalign{}
Version & Release Date \\
\midrule\noalign{}
\endhead
\bottomrule\noalign{}
\endlastfoot
0.1.3 & August 22, 2024 \\
\end{longtable}

Typst GmbH did not create this package and cannot guarantee correct
functionality of this package or compatibility with any version of the
Typst compiler or app.


\section{Package List LaTeX/yats.tex}
\title{typst.app/universe/package/yats}

\phantomsection\label{banner}
\section{yats}\label{yats}

{ 0.1.0 }

Serialization module for Typst

\phantomsection\label{readme}
serialize the Typst objects into bytes and deserialize the bytes into
Typst objects

\subsection{Reason}\label{reason}

I want to interactive between the wasm and typst. But I found that the
input arguments and output argument are all bytes. It is not convenient
for me to use WASM. So I designed the serialization protocol and
implemented this serialization module for reference.

Although there have been some serialization APIs like cbor, yaml, json
and others, this is a good learning material and a good example to show
the abilities of Typst.

\subsection{Example}\label{example}

Have a look at the example
\href{https://github.com/typst/packages/raw/main/packages/preview/yats/0.1.0/example.typ}{here}
.

\subsection{Usage}\label{usage}

Simply import 2 functions : \texttt{\ serialize\ } ,
\texttt{\ deserialize\ } .

And enjoy it

\begin{Shaded}
\begin{Highlighting}[]
\NormalTok{\#import "@preview/yats:0.1.0": serialize, deserialize}
\NormalTok{\#\{}
\NormalTok{  let obj = (}
\NormalTok{    name : "0warning0error",}
\NormalTok{    age : 100,}
\NormalTok{    height : 1.8,}
\NormalTok{    birthday : datetime(year : 1998,month : 7,day:8)}
\NormalTok{  )}
\NormalTok{  deserialize(serialize(obj))}
\NormalTok{\}}
\end{Highlighting}
\end{Shaded}

\subsection{Supported Types}\label{supported-types}

\begin{itemize}
\tightlist
\item
  \texttt{\ none\ }
\item
  \texttt{\ bool\ }
\item
  \texttt{\ type\ } : type is a type
\item
  \texttt{\ int\ }
\item
  \texttt{\ float\ } : (implemented in string, for convenience)
\item
  \texttt{\ datetime\ } : only support \texttt{\ year\ } ,
  \texttt{\ month\ } , \texttt{\ day\ } ; \texttt{\ hour\ } ,
  \texttt{\ minute\ } , \texttt{\ second\ } ; both combined.
\item
  \texttt{\ duration\ }
\item
  \texttt{\ bytes\ }
\item
  \texttt{\ string\ }
\item
  \texttt{\ regex\ } : (dealing with it is a little tricky)
\item
  \texttt{\ array\ } : the element in it can be anything listed.
\item
  \texttt{\ dictionary\ } : the value in it can be anything listed.
\end{itemize}

\subsection{\texorpdfstring{\texttt{\ Yats\ }
function}{ Yats  function}}\label{yats-function}

\begin{Shaded}
\begin{Highlighting}[]
\NormalTok{\#let serialize(}
\NormalTok{  data : any}
\NormalTok{) = \{ .. \}}
\end{Highlighting}
\end{Shaded}

\textbf{Arguments:}

\begin{itemize}
\tightlist
\item
  \texttt{\ data\ } : {[} \texttt{\ any\ } {]} â€'' Any supported object
  .
\end{itemize}

\textbf{Return}

The serialized bytes.

\begin{Shaded}
\begin{Highlighting}[]
\NormalTok{\#let deserialize(}
\NormalTok{  data : array}
\NormalTok{) = \{ .. \}}
\end{Highlighting}
\end{Shaded}

\textbf{Arguments:}

\begin{itemize}
\tightlist
\item
  \texttt{\ data\ } : {[} \texttt{\ bytes\ } {]} â€'' serialized objects
  represented by bytes .
\end{itemize}

\textbf{Return}

binary array. (the first one is the object deserialized, the second one
is the valid length of the bytes)

\subsection{Potential Problems and
limitation}\label{potential-problems-and-limitation}

\begin{itemize}
\item
  Some problem can be caused by changes of \texttt{\ repr\ } . For
  example, the serialization of \texttt{\ regex\ } relies on the
  \texttt{\ repr\ } of \texttt{\ regex\ } . And there are no method to
  directly catch the inner \texttt{\ string\ } .
\item
  Because of lack of time, only basic types are supported. But more
  types can be supported in Typst.
\end{itemize}

\subsection{License}\label{license}

This project is licensed under the Apache 2.0 License.

\subsubsection{How to add}\label{how-to-add}

Copy this into your project and use the import as \texttt{\ yats\ }

\begin{verbatim}
#import "@preview/yats:0.1.0"
\end{verbatim}

\includesvg[width=0.16667in,height=0.16667in]{/assets/icons/16-copy.svg}

Check the docs for
\href{https://typst.app/docs/reference/scripting/\#packages}{more
information on how to import packages} .

\subsubsection{About}\label{about}

\begin{description}
\tightlist
\item[Author :]
\href{https://github.com/0warning0error}{Zhao Yuanhang}
\item[License:]
Apache-2.0
\item[Current version:]
0.1.0
\item[Last updated:]
March 15, 2024
\item[First released:]
March 15, 2024
\item[Minimum Typst version:]
0.10.0
\item[Archive size:]
7.92 kB
\href{https://packages.typst.org/preview/yats-0.1.0.tar.gz}{\pandocbounded{\includesvg[keepaspectratio]{/assets/icons/16-download.svg}}}
\item[Repository:]
\href{https://github.com/0warning0error/typst-yats}{GitHub}
\end{description}

\subsubsection{Where to report issues?}\label{where-to-report-issues}

This package is a project of Zhao Yuanhang . Report issues on
\href{https://github.com/0warning0error/typst-yats}{their repository} .
You can also try to ask for help with this package on the
\href{https://forum.typst.app}{Forum} .

Please report this package to the Typst team using the
\href{https://typst.app/contact}{contact form} if you believe it is a
safety hazard or infringes upon your rights.

\phantomsection\label{versions}
\subsubsection{Version history}\label{version-history}

\begin{longtable}[]{@{}ll@{}}
\toprule\noalign{}
Version & Release Date \\
\midrule\noalign{}
\endhead
\bottomrule\noalign{}
\endlastfoot
0.1.0 & March 15, 2024 \\
\end{longtable}

Typst GmbH did not create this package and cannot guarantee correct
functionality of this package or compatibility with any version of the
Typst compiler or app.


\section{Package List LaTeX/tuhi-labscript-vuw.tex}
\title{typst.app/universe/package/tuhi-labscript-vuw}

\phantomsection\label{banner}
\phantomsection\label{template-thumbnail}
\pandocbounded{\includegraphics[keepaspectratio]{https://packages.typst.org/preview/thumbnails/tuhi-labscript-vuw-0.1.0-small.webp}}

\section{tuhi-labscript-vuw}\label{tuhi-labscript-vuw}

{ 0.1.0 }

A labscript template for VUW experimental courses.

\href{/app?template=tuhi-labscript-vuw&version=0.1.0}{Create project in
app}

\phantomsection\label{readme}
A Typst template for VUW lab scripts. To get started:

\begin{Shaded}
\begin{Highlighting}[]
\NormalTok{typst init @preview/tuhi{-}labscript{-}vuw:0.1.0}
\end{Highlighting}
\end{Shaded}

And edit the \texttt{\ main.typ\ } example.

\pandocbounded{\includegraphics[keepaspectratio]{https://github.com/typst/packages/raw/main/packages/preview/tuhi-labscript-vuw/0.1.0/thumbnail.png}}

\subsection{Contributing}\label{contributing}

PRs are welcome! And if you encounter any bugs or have any
requests/ideas, feel free to open an issue.

\href{/app?template=tuhi-labscript-vuw&version=0.1.0}{Create project in
app}

\subsubsection{How to use}\label{how-to-use}

Click the button above to create a new project using this template in
the Typst app.

You can also use the Typst CLI to start a new project on your computer
using this command:

\begin{verbatim}
typst init @preview/tuhi-labscript-vuw:0.1.0
\end{verbatim}

\includesvg[width=0.16667in,height=0.16667in]{/assets/icons/16-copy.svg}

\subsubsection{About}\label{about}

\begin{description}
\tightlist
\item[Author :]
\href{https://github.com/baptiste}{baptiste}
\item[License:]
MPL-2.0
\item[Current version:]
0.1.0
\item[Last updated:]
June 17, 2024
\item[First released:]
June 17, 2024
\item[Archive size:]
177 kB
\href{https://packages.typst.org/preview/tuhi-labscript-vuw-0.1.0.tar.gz}{\pandocbounded{\includesvg[keepaspectratio]{/assets/icons/16-download.svg}}}
\item[Categor y :]
\begin{itemize}
\tightlist
\item[]
\item
  \pandocbounded{\includesvg[keepaspectratio]{/assets/icons/16-envelope.svg}}
  \href{https://typst.app/universe/search/?category=office}{Office}
\end{itemize}
\end{description}

\subsubsection{Where to report issues?}\label{where-to-report-issues}

This template is a project of baptiste . You can also try to ask for
help with this template on the \href{https://forum.typst.app}{Forum} .

Please report this template to the Typst team using the
\href{https://typst.app/contact}{contact form} if you believe it is a
safety hazard or infringes upon your rights.

\phantomsection\label{versions}
\subsubsection{Version history}\label{version-history}

\begin{longtable}[]{@{}ll@{}}
\toprule\noalign{}
Version & Release Date \\
\midrule\noalign{}
\endhead
\bottomrule\noalign{}
\endlastfoot
0.1.0 & June 17, 2024 \\
\end{longtable}

Typst GmbH did not create this template and cannot guarantee correct
functionality of this template or compatibility with any version of the
Typst compiler or app.


\section{Package List LaTeX/linguify.tex}
\title{typst.app/universe/package/linguify}

\phantomsection\label{banner}
\section{linguify}\label{linguify}

{ 0.4.1 }

Load strings for different languages easily

\phantomsection\label{readme}
Load strings for different languages easily. This can be useful if you
create a package or template for multilingual usage.

\subsection{Usage}\label{usage}

The usage depends if you are using it inside a package or a template or
in your own document.

\subsubsection{For end users and own
templates}\label{for-end-users-and-own-templates}

You can use linguify global database.

Example:

\begin{Shaded}
\begin{Highlighting}[]
\NormalTok{\#import "@preview/linguify:0.4.0": *}

\NormalTok{\#let lang\_data = toml("lang.toml")}
\NormalTok{\#set{-}database(lang\_data);}

\NormalTok{\#set text(lang: "de")}

\NormalTok{\#linguify("abstract")  // Shows "Zusammenfassung" in the document.}
\end{Highlighting}
\end{Shaded}

The \texttt{\ lang.toml\ } musst look like this:

\begin{Shaded}
\begin{Highlighting}[]
\KeywordTok{[conf]}
\DataTypeTok{default{-}lang} \OperatorTok{=} \StringTok{"en"}

\KeywordTok{[en]}
\DataTypeTok{title} \OperatorTok{=} \StringTok{"A simple linguify example"}
\DataTypeTok{abstract} \OperatorTok{=} \StringTok{"Abstract"}

\KeywordTok{[de]}
\DataTypeTok{title} \OperatorTok{=} \StringTok{"Ein einfaches Linguify Beispiel"}
\DataTypeTok{abstract} \OperatorTok{=} \StringTok{"Zusammenfassung"}
\end{Highlighting}
\end{Shaded}

\subsubsection{Inside a package}\label{inside-a-package}

So that multiple packages can use linguify simultaneously, they should
contain their own database. A linguify database is just a dictionary
with a certain structure. (See database structure.)

Recommend is to store the database in a separate file like
\texttt{\ lang.toml\ } and load it inside the document. And specify it
in each \texttt{\ linguify()\ } function call.

Example:

\begin{Shaded}
\begin{Highlighting}[]
\NormalTok{\#import "@preview/linguify:0.4.0": *}

\NormalTok{\#let database = toml("lang.toml")}

\NormalTok{\#linguify("key", from: database, default: "key")}
\end{Highlighting}
\end{Shaded}

\subsection{Features}\label{features}

\begin{itemize}
\tightlist
\item
  Use a \texttt{\ toml\ } or other file to load strings for different
  languages. You need to pass a typst dictionary which follows the
  structure of the shown toml file.
\item
  Specify a \textbf{default-lang} . If none is specified it will default
  to \texttt{\ en\ }
\item
  \textbf{Fallback} to the default-lang if a key is not found for a
  certain language.
\item
  \href{https://projectfluent.org/}{Fluent} support
\end{itemize}

\subsubsection{How to add}\label{how-to-add}

Copy this into your project and use the import as \texttt{\ linguify\ }

\begin{verbatim}
#import "@preview/linguify:0.4.1"
\end{verbatim}

\includesvg[width=0.16667in,height=0.16667in]{/assets/icons/16-copy.svg}

Check the docs for
\href{https://typst.app/docs/reference/scripting/\#packages}{more
information on how to import packages} .

\subsubsection{About}\label{about}

\begin{description}
\tightlist
\item[Author :]
\href{https://github.com/jomaway}{Jomaway}
\item[License:]
MIT
\item[Current version:]
0.4.1
\item[Last updated:]
April 29, 2024
\item[First released:]
January 31, 2024
\item[Minimum Typst version:]
0.11.0
\item[Archive size:]
470 kB
\href{https://packages.typst.org/preview/linguify-0.4.1.tar.gz}{\pandocbounded{\includesvg[keepaspectratio]{/assets/icons/16-download.svg}}}
\item[Repository:]
\href{https://github.com/jomaway/typst-linguify}{GitHub}
\item[Categor ies :]
\begin{itemize}
\tightlist
\item[]
\item
  \pandocbounded{\includesvg[keepaspectratio]{/assets/icons/16-world.svg}}
  \href{https://typst.app/universe/search/?category=languages}{Languages}
\item
  \pandocbounded{\includesvg[keepaspectratio]{/assets/icons/16-hammer.svg}}
  \href{https://typst.app/universe/search/?category=utility}{Utility}
\end{itemize}
\end{description}

\subsubsection{Where to report issues?}\label{where-to-report-issues}

This package is a project of Jomaway . Report issues on
\href{https://github.com/jomaway/typst-linguify}{their repository} . You
can also try to ask for help with this package on the
\href{https://forum.typst.app}{Forum} .

Please report this package to the Typst team using the
\href{https://typst.app/contact}{contact form} if you believe it is a
safety hazard or infringes upon your rights.

\phantomsection\label{versions}
\subsubsection{Version history}\label{version-history}

\begin{longtable}[]{@{}ll@{}}
\toprule\noalign{}
Version & Release Date \\
\midrule\noalign{}
\endhead
\bottomrule\noalign{}
\endlastfoot
0.4.1 & April 29, 2024 \\
\href{https://typst.app/universe/package/linguify/0.4.0/}{0.4.0} & April
2, 2024 \\
\href{https://typst.app/universe/package/linguify/0.3.1/}{0.3.1} & March
26, 2024 \\
\href{https://typst.app/universe/package/linguify/0.3.0/}{0.3.0} & March
18, 2024 \\
\href{https://typst.app/universe/package/linguify/0.2.0/}{0.2.0} & March
16, 2024 \\
\href{https://typst.app/universe/package/linguify/0.1.0/}{0.1.0} &
January 31, 2024 \\
\end{longtable}

Typst GmbH did not create this package and cannot guarantee correct
functionality of this package or compatibility with any version of the
Typst compiler or app.


\section{Package List LaTeX/scrutinize.tex}
\title{typst.app/universe/package/scrutinize}

\phantomsection\label{banner}
\section{scrutinize}\label{scrutinize}

{ 0.3.0 }

A library for building exams, tests, etc. with Typst

\phantomsection\label{readme}
Scrutinize is a library for building exams, tests, etc. with Typst. It
has three general areas of focus:

\begin{itemize}
\tightlist
\item
  It helps with grading information: record the points that can be
  reached for each question and make them available for creating grading
  keys.
\item
  It provides a selection of question writing utilities, such as
  multiple choice or true/false questions.
\item
  It supports the creation of sample solutions by allowing to switch
  between the normal and “pre-filled� exam.
\end{itemize}

Right now, providing a styled template is not part of this package’s
scope. Also, visual customization of the provided question templates is
currently nonexistent.

See the
\href{https://github.com/typst/packages/raw/main/packages/preview/scrutinize/0.3.0/docs/manual.pdf}{manual}
for details.

\subsection{Example}\label{example}

\begin{longtable}[]{@{}ll@{}}
\toprule\noalign{}
\endhead
\bottomrule\noalign{}
\endlastfoot
\href{https://github.com/typst/packages/raw/main/packages/preview/scrutinize/0.3.0/gallery/gk-ek-austria.typ}{\pandocbounded{\includegraphics[keepaspectratio]{https://github.com/typst/packages/raw/main/packages/preview/scrutinize/0.3.0/thumbnail.png}}}
&
\href{https://github.com/typst/packages/raw/main/packages/preview/scrutinize/0.3.0/gallery/gk-ek-austria.typ}{\pandocbounded{\includegraphics[keepaspectratio]{https://github.com/typst/packages/raw/main/packages/preview/scrutinize/0.3.0/thumbnail-solved.png}}} \\
\end{longtable}

This example can be found in the
\href{https://github.com/typst/packages/raw/main/packages/preview/scrutinize/0.3.0/gallery/}{gallery}
. Here are some excerpts from it:

\begin{Shaded}
\begin{Highlighting}[]
\NormalTok{\#import "@preview/scrutinize:0.3.0" as scrutinize: grading, task, solution, task{-}kinds}
\NormalTok{\#import task{-}kinds: free{-}form, gap, choice}
\NormalTok{\#import task: t}

\NormalTok{// ... document setup ...}

\NormalTok{\#context \{}
\NormalTok{  let ts = task.all(level: 2)}
\NormalTok{  let total = grading.total{-}points(ts)}

\NormalTok{  let grades = grading.grades(}
\NormalTok{    [F],}
\NormalTok{    0.6 * total,}
\NormalTok{    [D],}
\NormalTok{    0.7 * total,}
\NormalTok{    [C],}
\NormalTok{    0.8 * total,}
\NormalTok{    [B],}
\NormalTok{    0.9 * total,}
\NormalTok{    [A],}
\NormalTok{  )}

\NormalTok{  // ... show the grading key ...}
\NormalTok{\}}

\NormalTok{// ...}

\NormalTok{= Basic competencies {-}{-} theoretical part B}

\NormalTok{\#lorem(40)}

\NormalTok{== Writing}
\NormalTok{\#t(category: "b", points: 4)}
\NormalTok{\#lorem(30)}

\NormalTok{\#free{-}form.lines(stretch: 180\%, lorem(20))}

\NormalTok{== Multiple Choice}
\NormalTok{\#t(category: "b", points: 2)}
\NormalTok{\#lorem(30)}

\NormalTok{\#\{}
\NormalTok{  set align(center)}
\NormalTok{  choice.multiple((}
\NormalTok{    (lorem(3), true),}
\NormalTok{    (lorem(5), true),}
\NormalTok{    (lorem(4), false),}
\NormalTok{  ))}
\NormalTok{\}}
\end{Highlighting}
\end{Shaded}

\subsubsection{How to add}\label{how-to-add}

Copy this into your project and use the import as
\texttt{\ scrutinize\ }

\begin{verbatim}
#import "@preview/scrutinize:0.3.0"
\end{verbatim}

\includesvg[width=0.16667in,height=0.16667in]{/assets/icons/16-copy.svg}

Check the docs for
\href{https://typst.app/docs/reference/scripting/\#packages}{more
information on how to import packages} .

\subsubsection{About}\label{about}

\begin{description}
\tightlist
\item[Author :]
\href{https://github.com/SillyFreak/}{Clemens Koza}
\item[License:]
MIT
\item[Current version:]
0.3.0
\item[Last updated:]
October 14, 2024
\item[First released:]
January 8, 2024
\item[Minimum Typst version:]
0.11.0
\item[Archive size:]
11.2 kB
\href{https://packages.typst.org/preview/scrutinize-0.3.0.tar.gz}{\pandocbounded{\includesvg[keepaspectratio]{/assets/icons/16-download.svg}}}
\item[Repository:]
\href{https://github.com/SillyFreak/typst-scrutinize}{GitHub}
\item[Discipline :]
\begin{itemize}
\tightlist
\item[]
\item
  \href{https://typst.app/universe/search/?discipline=education}{Education}
\end{itemize}
\item[Categor ies :]
\begin{itemize}
\tightlist
\item[]
\item
  \pandocbounded{\includesvg[keepaspectratio]{/assets/icons/16-list-unordered.svg}}
  \href{https://typst.app/universe/search/?category=model}{Model}
\item
  \pandocbounded{\includesvg[keepaspectratio]{/assets/icons/16-code.svg}}
  \href{https://typst.app/universe/search/?category=scripting}{Scripting}
\item
  \pandocbounded{\includesvg[keepaspectratio]{/assets/icons/16-envelope.svg}}
  \href{https://typst.app/universe/search/?category=office}{Office}
\end{itemize}
\end{description}

\subsubsection{Where to report issues?}\label{where-to-report-issues}

This package is a project of Clemens Koza . Report issues on
\href{https://github.com/SillyFreak/typst-scrutinize}{their repository}
. You can also try to ask for help with this package on the
\href{https://forum.typst.app}{Forum} .

Please report this package to the Typst team using the
\href{https://typst.app/contact}{contact form} if you believe it is a
safety hazard or infringes upon your rights.

\phantomsection\label{versions}
\subsubsection{Version history}\label{version-history}

\begin{longtable}[]{@{}ll@{}}
\toprule\noalign{}
Version & Release Date \\
\midrule\noalign{}
\endhead
\bottomrule\noalign{}
\endlastfoot
0.3.0 & October 14, 2024 \\
\href{https://typst.app/universe/package/scrutinize/0.2.0/}{0.2.0} &
July 15, 2024 \\
\href{https://typst.app/universe/package/scrutinize/0.1.0/}{0.1.0} &
January 8, 2024 \\
\end{longtable}

Typst GmbH did not create this package and cannot guarantee correct
functionality of this package or compatibility with any version of the
Typst compiler or app.


\section{Package List LaTeX/unify.tex}
\title{typst.app/universe/package/unify}

\phantomsection\label{banner}
\section{unify}\label{unify}

{ 0.7.0 }

Format numbers, units, and ranges correctly.

{ } Featured Package

\phantomsection\label{readme}
\texttt{\ unify\ } is a \href{https://github.com/typst/typst}{Typst}
package simplifying the typesetting of numbers, units, and ranges. It is
the equivalent to LaTeX’s \texttt{\ siunitx\ } , though not as mature.

\subsection{Overview}\label{overview}

\texttt{\ unify\ } allows flexible numbers and units, and still mostly
gets well typeset results.

\begin{Shaded}
\begin{Highlighting}[]
\NormalTok{\#import "@preview/unify:0.7.0": num,qty,numrange,qtyrange}

\NormalTok{$ num("{-}1.32865+{-}0.50273e{-}6") $}
\NormalTok{$ qty("1.3+1.2{-}0.3e3", "erg/cm\^{}2/s", space: "\#h(2mm)") $}
\NormalTok{$ numrange("1,1238e{-}2", "3,0868e5", thousandsep: "\textquotesingle{}") $}
\NormalTok{$ qtyrange("1e3", "2e3", "meter per second squared", per: "/", delimiter: "\textbackslash{}"to\textbackslash{}"") $}
\end{Highlighting}
\end{Shaded}

\includegraphics[width=3.125in,height=\textheight,keepaspectratio]{https://github.com/typst/packages/raw/main/packages/preview/unify/0.7.0/examples/overview.jpg}

Right now, physical, monetary, and binary units are supported. New
issues or pull requests for new units are welcome!

\subsection{Multilingual support}\label{multilingual-support}

The Unify package supports multiple languages. Currently, the supported
languages are English and Russian. The fallback is English. If you want
to add your language, you should add two files:
\texttt{\ prefixes-xx.csv\ } and \texttt{\ units-xx.csv\ } , and in the
\texttt{\ lib.typ\ } file you should fix the \texttt{\ lang-db\ } state
for your files.

\subsection{\texorpdfstring{\texttt{\ num\ }}{ num }}\label{num}

\texttt{\ num\ } uses string parsing in order to typeset numbers,
including separators between the thousands. They can have the following
form:

\begin{itemize}
\tightlist
\item
  \texttt{\ float\ } or \texttt{\ integer\ } number
\item
  either ( \texttt{\ \{\}\ } stands for a number)

  \begin{itemize}
  \tightlist
  \item
    symmetric uncertainties with \texttt{\ +-\{\}\ }
  \item
    asymmetric uncertainties with \texttt{\ +\{\}-\{\}\ }
  \end{itemize}
\item
  exponential notation \texttt{\ e\{\}\ }
\end{itemize}

Parentheses are automatically set as necessary. Use
\texttt{\ thousandsep\ } to change the separator between the thousands,
and \texttt{\ multiplier\ } to change the multiplication symbol between
the number and exponential.

\subsection{\texorpdfstring{\texttt{\ unit\ }}{ unit }}\label{unit}

\texttt{\ unit\ } takes the unit in words or in symbolic notation as its
first argument. The value of \texttt{\ space\ } will be inserted between
units if necessary. Setting \texttt{\ per\ } to \texttt{\ symbol\ } will
format the number with exponents (i.e. \texttt{\ \^{}(-1)\ } ),
\texttt{\ fraction\ } or \texttt{\ /\ } using fraction, and
\texttt{\ fraction-short\ } or
\texttt{\ \textbackslash{}\textbackslash{}/\ } using in-line
fractions.\\
Units in words have four possible parts:

\begin{itemize}
\tightlist
\item
  \texttt{\ per\ } forms the inverse of the following unit.
\item
  A written-out prefix in the sense of SI (e.g. \texttt{\ centi\ } ).
  This is added before the unit.
\item
  The unit itself written out (e.g. \texttt{\ gram\ } ).
\item
  A postfix like \texttt{\ squared\ } . This is added after the unit and
  takes \texttt{\ per\ } into account.
\end{itemize}

The shorthand notation also has four parts:

\begin{itemize}
\tightlist
\item
  \texttt{\ /\ } forms the inverse of the following unit.
\item
  A short prefix in the sense of SI (e.g. \texttt{\ k\ } ). This is
  added before the unit.
\item
  The short unit itself (e.g. \texttt{\ g\ } ).
\item
  An exponent like \texttt{\ \^{}2\ } . This is added after the unit and
  takes \texttt{\ /\ } into account.
\end{itemize}

Note: Use \texttt{\ u\ } for micro.

The possible values of the three latter parts are loaded at runtime from
\texttt{\ prefixes.csv\ } , \texttt{\ units.csv\ } , and
\texttt{\ postfixes.csv\ } (in the library directory). Your own units
etc. can be permanently added in these files. At runtime, they can be
added using \texttt{\ add-unit\ } and \texttt{\ add-prefix\ } ,
respectively. The formats for the pre- and postfixes are:

\begin{longtable}[]{@{}lll@{}}
\toprule\noalign{}
pre-/postfix & shorthand & symbol \\
\midrule\noalign{}
\endhead
\bottomrule\noalign{}
\endlastfoot
milli & m & upright(“m�) \\
\end{longtable}

and for units:

\begin{longtable}[]{@{}llll@{}}
\toprule\noalign{}
unit & shorthand & symbol & space \\
\midrule\noalign{}
\endhead
\bottomrule\noalign{}
\endlastfoot
meter & m & upright(“m�) & true \\
\end{longtable}

The first column specifies the written-out word, the second one the
shorthand. These should be unique. The third column represents the
string that will be inserted as the unit symbol. For units, the last
column describes whether there should be space before the unit (possible
values: \texttt{\ true\ } / \texttt{\ false\ } , \texttt{\ 1\ } ,
\texttt{\ 0\ } ). This is mostly the cases for degrees and other angle
units (e.g. arcseconds).\\
If you think there are units not included that are of interest for other
users, you can create an issue or PR.

\subsection{\texorpdfstring{\texttt{\ qty\ }}{ qty }}\label{qty}

\texttt{\ qty\ } allows a \texttt{\ num\ } as the first argument
following the same rules. The second argument is a unit. If
\texttt{\ rawunit\ } is set to true, its value will be passed on to the
result (note that the string passed on will be passed to
\texttt{\ eval\ } , so add escaped quotes \texttt{\ \textbackslash{}"\ }
if necessary). Otherwise, it follows the rules of \texttt{\ unit\ } .
The value of \texttt{\ space\ } will be inserted between units if
necessary, \texttt{\ thousandsep\ } between the thousands, and
\texttt{\ per\ } switches between exponents and fractions.

\subsection{\texorpdfstring{\texttt{\ numrange\ }}{ numrange }}\label{numrange}

\texttt{\ numrange\ } takes two \texttt{\ num\ } s as the first two
arguments. If they have the same exponent, it is automatically
factorized. The range symbol can be changed with \texttt{\ delimiter\ }
, and the space between the numbers and symbols with \texttt{\ space\ }
.

\subsection{\texorpdfstring{\texttt{\ qtyrange\ }}{ qtyrange }}\label{qtyrange}

\texttt{\ qtyrange\ } is just a combination of \texttt{\ unit\ } and
\texttt{\ range\ } .

\subsubsection{How to add}\label{how-to-add}

Copy this into your project and use the import as \texttt{\ unify\ }

\begin{verbatim}
#import "@preview/unify:0.7.0"
\end{verbatim}

\includesvg[width=0.16667in,height=0.16667in]{/assets/icons/16-copy.svg}

Check the docs for
\href{https://typst.app/docs/reference/scripting/\#packages}{more
information on how to import packages} .

\subsubsection{About}\label{about}

\begin{description}
\tightlist
\item[Author :]
Christopher Hecker
\item[License:]
MIT
\item[Current version:]
0.7.0
\item[Last updated:]
November 28, 2024
\item[First released:]
July 27, 2023
\item[Archive size:]
9.04 kB
\href{https://packages.typst.org/preview/unify-0.7.0.tar.gz}{\pandocbounded{\includesvg[keepaspectratio]{/assets/icons/16-download.svg}}}
\item[Repository:]
\href{https://github.com/ChHecker/unify}{GitHub}
\item[Discipline s :]
\begin{itemize}
\tightlist
\item[]
\item
  \href{https://typst.app/universe/search/?discipline=business}{Business}
\item
  \href{https://typst.app/universe/search/?discipline=chemistry}{Chemistry}
\item
  \href{https://typst.app/universe/search/?discipline=computer-science}{Computer
  Science}
\item
  \href{https://typst.app/universe/search/?discipline=economics}{Economics}
\item
  \href{https://typst.app/universe/search/?discipline=engineering}{Engineering}
\item
  \href{https://typst.app/universe/search/?discipline=mathematics}{Mathematics}
\item
  \href{https://typst.app/universe/search/?discipline=physics}{Physics}
\end{itemize}
\item[Categor y :]
\begin{itemize}
\tightlist
\item[]
\item
  \pandocbounded{\includesvg[keepaspectratio]{/assets/icons/16-text.svg}}
  \href{https://typst.app/universe/search/?category=text}{Text}
\end{itemize}
\end{description}

\subsubsection{Where to report issues?}\label{where-to-report-issues}

This package is a project of Christopher Hecker . Report issues on
\href{https://github.com/ChHecker/unify}{their repository} . You can
also try to ask for help with this package on the
\href{https://forum.typst.app}{Forum} .

Please report this package to the Typst team using the
\href{https://typst.app/contact}{contact form} if you believe it is a
safety hazard or infringes upon your rights.

\phantomsection\label{versions}
\subsubsection{Version history}\label{version-history}

\begin{longtable}[]{@{}ll@{}}
\toprule\noalign{}
Version & Release Date \\
\midrule\noalign{}
\endhead
\bottomrule\noalign{}
\endlastfoot
0.7.0 & November 28, 2024 \\
\href{https://typst.app/universe/package/unify/0.6.1/}{0.6.1} & November
18, 2024 \\
\href{https://typst.app/universe/package/unify/0.6.0/}{0.6.0} & May 23,
2024 \\
\href{https://typst.app/universe/package/unify/0.5.0/}{0.5.0} & March
26, 2024 \\
\href{https://typst.app/universe/package/unify/0.4.3/}{0.4.3} & October
22, 2023 \\
\href{https://typst.app/universe/package/unify/0.4.2/}{0.4.2} & October
9, 2023 \\
\href{https://typst.app/universe/package/unify/0.4.1/}{0.4.1} &
September 3, 2023 \\
\href{https://typst.app/universe/package/unify/0.4.0/}{0.4.0} & July 28,
2023 \\
\href{https://typst.app/universe/package/unify/0.1.0/}{0.1.0} & July 27,
2023 \\
\end{longtable}

Typst GmbH did not create this package and cannot guarantee correct
functionality of this package or compatibility with any version of the
Typst compiler or app.


\section{Package List LaTeX/latedef.tex}
\title{typst.app/universe/package/latedef}

\phantomsection\label{banner}
\section{latedef}\label{latedef}

{ 0.1.0 }

Use now, define later

\phantomsection\label{readme}
\emph{Use now, define later!}

\subsection{Basic usage}\label{basic-usage}

This package exposes a single function, \texttt{\ latedef-setup\ } .

\begin{Shaded}
\begin{Highlighting}[]
\NormalTok{\#let (undef, def) = latedef{-}setup(simple: true)}

\NormalTok{My \#undef is \#undef.}
\NormalTok{\#def("dog")}
\NormalTok{\#def("cool")}
\end{Highlighting}
\end{Shaded}

\pandocbounded{\includegraphics[keepaspectratio]{https://github.com/typst/packages/raw/main/packages/preview/latedef/0.1.0/example-images/1.png}}

Note that the definition doesn’t actually have to come \emph{after}
the usage, but if you want to define something beforehand, you’re
better off using a variable instead.

\begin{Shaded}
\begin{Highlighting}[]
\NormalTok{\#let (undef, def) = latedef{-}setup(simple: true)}

\NormalTok{// Instead of}
\NormalTok{\#def("A")}
\NormalTok{The first letter is \#undef.}

\NormalTok{// you should use}
\NormalTok{\#let A = "A"}
\NormalTok{The first letter is \#A.}
\end{Highlighting}
\end{Shaded}

\pandocbounded{\includegraphics[keepaspectratio]{https://github.com/typst/packages/raw/main/packages/preview/latedef/0.1.0/example-images/2.png}}

\subsection{\texorpdfstring{The \texttt{\ simple\ }
parameter}{The  simple  parameter}}\label{the-simple-parameter}

When \texttt{\ simple:\ false\ } (which is the default),
\texttt{\ undef\ } becomes a function you have to call. It takes an
optional positional or named parameter \texttt{\ id\ } of type
\texttt{\ str\ } , which can be used to define things out of order.

\begin{Shaded}
\begin{Highlighting}[]
\NormalTok{\#let (undef, def) = latedef{-}setup() // or \textasciigrave{}latedef{-}setup(simple: false)\textasciigrave{}}

\NormalTok{// Note that you can still call it without an id, which works just like when \textasciigrave{}simple: true\textasciigrave{}.}
\NormalTok{My letters are \#undef("1"), \#undef(id: "2"), and \#undef().}

\NormalTok{// \textasciigrave{}def\textasciigrave{} now takes one positional and either another positional or a named parameter.}
\NormalTok{\#def("C")}
\NormalTok{\#def(id: "2", "B")}
\NormalTok{\#def("1", "A")}
\end{Highlighting}
\end{Shaded}

\pandocbounded{\includegraphics[keepaspectratio]{https://github.com/typst/packages/raw/main/packages/preview/latedef/0.1.0/example-images/3.png}}

\subsection{\texorpdfstring{The \texttt{\ footnote\ }
parameter}{The  footnote  parameter}}\label{the-footnote-parameter}

This is a convenience feature that automatically wraps
\texttt{\ undef\ } in \texttt{\ footnote\ } , either directly (when
\texttt{\ simple:\ true\ } ) or as a function (when
\texttt{\ simple:\ false\ } ).

This corresponds to LaTeX’s \texttt{\ \textbackslash{}footnotemark\ }
and \texttt{\ \textbackslash{}footnotetext\ } , hence the different
names in the example.

\begin{Shaded}
\begin{Highlighting}[]
\NormalTok{\#let (fmark, ftext) = latedef{-}setup(simple: true, footnote: true)}
\NormalTok{Do\#fmark you\#fmark believe\#fmark in God?\#fmark}

\NormalTok{\#let wdym = "What do you mean"}
\NormalTok{\#ftext[\#wdym "Do"?]}
\NormalTok{\#ftext[\#wdym "you"?]}
\NormalTok{\#ftext[\#wdym "believe"?]}
\NormalTok{\#ftext[And w\#wdym.slice(1) "God"?]}
\end{Highlighting}
\end{Shaded}

\pandocbounded{\includegraphics[keepaspectratio]{https://github.com/typst/packages/raw/main/packages/preview/latedef/0.1.0/example-images/4.png}}

\subsection{\texorpdfstring{The \texttt{\ stand-in\ }
parameter}{The  stand-in  parameter}}\label{the-stand-in-parameter}

This is a function that takes a single positional parameter (
\texttt{\ id\ } ) of type \texttt{\ none\ \textbar{}\ str\ } and
produces a stand-in value that gets shown when a late-defined value is
missing a corresponding definition.

\begin{Shaded}
\begin{Highlighting}[]
\NormalTok{\#let (undef, def) = latedef{-}setup()}
\NormalTok{// This is the default stand{-}in}
\NormalTok{\#undef()}
\NormalTok{\#undef("with an id")}

\NormalTok{// Custom stand{-}in}
\NormalTok{\#let (undef, def) = latedef{-}setup(stand{-}in: id =\textgreater{} emph[No \#id!])}
\NormalTok{\#undef()}
\NormalTok{\#undef("id")}
\end{Highlighting}
\end{Shaded}

\pandocbounded{\includegraphics[keepaspectratio]{https://github.com/typst/packages/raw/main/packages/preview/latedef/0.1.0/example-images/5.png}}

Since \texttt{\ stand-in\ } is a function, which is only called when a
definition is actually missing, you can even set it to panic to enforce
that all late-defined values have a definiton.

\begin{Shaded}
\begin{Highlighting}[]
\NormalTok{\#let (undef, def) = latedef{-}setup(stand{-}in: id =\textgreater{} panic("Missing definition for value with id " + repr(id)))}
\NormalTok{\#undef()}
\NormalTok{\#undef("id")}
\end{Highlighting}
\end{Shaded}

The output will look something like

\begin{verbatim}
error: panicked with: "Missing definition for value with id none"
  ┌─ example.typ:1:50
  │
  │ #let (undef, def) = latedef-setup(stand-in: id => panic("Missing definition for value with id " + repr(id)))
  │                                                   ^^^^^^^^^^^^^^^^^^^^^^^^^^^^^^^^^^^^^^^^^^^^^^^^^^^^^^^^^

error: panicked with: "Missing definition for value with id \"id\""
  ┌─ example.typ:1:50
  │
  │ #let (undef, def) = latedef-setup(stand-in: id => panic("Missing definition for value with id " + repr(id)))
  │                                                   ^^^^^^^^^^^^^^^^^^^^^^^^^^^^^^^^^^^^^^^^^^^^^^^^^^^^^^^^^
\end{verbatim}

And there is no error when everything has a definition:

\begin{Shaded}
\begin{Highlighting}[]
\NormalTok{\#let (undef, def) = latedef{-}setup(stand{-}in: id =\textgreater{} panic("Missing definition for value with id " + repr(id)))}
\NormalTok{\#undef() is \#undef("id").}
\NormalTok{\#def("This")}
\NormalTok{\#def("id", "fine")}
\end{Highlighting}
\end{Shaded}

\pandocbounded{\includegraphics[keepaspectratio]{https://github.com/typst/packages/raw/main/packages/preview/latedef/0.1.0/example-images/6.png}}

\subsection{\texorpdfstring{The \texttt{\ group\ }
parameter}{The  group  parameter}}\label{the-group-parameter}

Sometimes you may want to use multiple instances of \texttt{\ latedef\ }
in parallel. This is done using the \texttt{\ group\ } parameter, which
can be \texttt{\ none\ } (the default) or any \texttt{\ str\ } .

Note that using \texttt{\ footnote:\ true\ } sets the default group to
\texttt{\ "footnote"\ } instead.

\begin{Shaded}
\begin{Highlighting}[]
\NormalTok{// Use a group for the figure stuff...}
\NormalTok{\#let (caption{-}undef, caption) = latedef{-}setup(simple: true, group: "figure")}
\NormalTok{\#let figure = std.figure.with(caption: caption{-}undef)}
\NormalTok{// ...so you can still use the regular mechanism in parallel.}
\NormalTok{\#let (undef, def) = latedef{-}setup(simple: true)}

\NormalTok{\#figure(raw(block: true, lorem(5)))}
\NormalTok{\#caption[The \#undef \_lorem ipsum\_.]}
\NormalTok{\#def("classic")}
\end{Highlighting}
\end{Shaded}

\pandocbounded{\includegraphics[keepaspectratio]{https://github.com/typst/packages/raw/main/packages/preview/latedef/0.1.0/example-images/7.png}}

\subsubsection{How to add}\label{how-to-add}

Copy this into your project and use the import as \texttt{\ latedef\ }

\begin{verbatim}
#import "@preview/latedef:0.1.0"
\end{verbatim}

\includesvg[width=0.16667in,height=0.16667in]{/assets/icons/16-copy.svg}

Check the docs for
\href{https://typst.app/docs/reference/scripting/\#packages}{more
information on how to import packages} .

\subsubsection{About}\label{about}

\begin{description}
\tightlist
\item[Author :]
\href{mailto:realt0mstone@gmail.com}{T0mstone}
\item[License:]
MIT-0
\item[Current version:]
0.1.0
\item[Last updated:]
October 21, 2024
\item[First released:]
October 21, 2024
\item[Minimum Typst version:]
0.11.0
\item[Archive size:]
3.64 kB
\href{https://packages.typst.org/preview/latedef-0.1.0.tar.gz}{\pandocbounded{\includesvg[keepaspectratio]{/assets/icons/16-download.svg}}}
\item[Repository:]
\href{https://codeberg.org/T0mstone/typst-latedef}{Codeberg}
\end{description}

\subsubsection{Where to report issues?}\label{where-to-report-issues}

This package is a project of T0mstone . Report issues on
\href{https://codeberg.org/T0mstone/typst-latedef}{their repository} .
You can also try to ask for help with this package on the
\href{https://forum.typst.app}{Forum} .

Please report this package to the Typst team using the
\href{https://typst.app/contact}{contact form} if you believe it is a
safety hazard or infringes upon your rights.

\phantomsection\label{versions}
\subsubsection{Version history}\label{version-history}

\begin{longtable}[]{@{}ll@{}}
\toprule\noalign{}
Version & Release Date \\
\midrule\noalign{}
\endhead
\bottomrule\noalign{}
\endlastfoot
0.1.0 & October 21, 2024 \\
\end{longtable}

Typst GmbH did not create this package and cannot guarantee correct
functionality of this package or compatibility with any version of the
Typst compiler or app.


\section{Package List LaTeX/tada.tex}
\title{typst.app/universe/package/tada}

\phantomsection\label{banner}
\section{tada}\label{tada}

{ 0.1.0 }

Easy, composable tabular data manipulation

\phantomsection\label{readme}
TaDa provides a set of simple but powerful operations on rows of data. A
full manual is available online:
\url{https://github.com/ntjess/typst-tada/blob/v0.1.0/docs/manual.pdf}

Key features include:

\begin{itemize}
\item
  \textbf{Arithmetic expressions} : Row-wise operations are as simple as
  string expressions with field names
\item
  \textbf{Aggregation} : Any function that operates on an array of
  values can perform row-wise or column-wise aggregation
\item
  \textbf{Data representation} : Handle displaying currencies, floats,
  integers, and more with ease and arbitrary customization
\end{itemize}

Note: This library is in early development. The API is subject to change
especially as typst adds more support for user-defined types.
\textbf{Backwards compatibility is not guaranteed!} Handling of field
info, value types, and more may change substantially with more user
feedback.

\subsection{Importing}\label{importing}

TaDa can be imported as follows:

\subsubsection{From the official packages repository
(recommended):}\label{from-the-official-packages-repository-recommended}

\begin{Shaded}
\begin{Highlighting}[]
\NormalTok{\#import "@preview/tada:0.1.0"}
\end{Highlighting}
\end{Shaded}

\subsubsection{From the source code (not
recommended)}\label{from-the-source-code-not-recommended}

\textbf{Option 1:} You can clone the package directly into your project
directory:

\begin{Shaded}
\begin{Highlighting}[]
\CommentTok{\# In your project directory}
\FunctionTok{git}\NormalTok{ clone https://github.com/ntjess/typst{-}tada.git tada}
\end{Highlighting}
\end{Shaded}

Then import the functionality with

\begin{Shaded}
\begin{Highlighting}[]
\NormalTok{\#import "./tada/lib.typ" }
\end{Highlighting}
\end{Shaded}

\textbf{Option 2:} If Python is available on your system, use the
provided packaging script to install TaDa in typst’s
\texttt{\ local\ } directory:

\begin{Shaded}
\begin{Highlighting}[]
\CommentTok{\# Anywhere on your system}
  \FunctionTok{git}\NormalTok{ clone https://github.com/ntjess/typst{-}tada.git}
  \BuiltInTok{cd}\NormalTok{ typst{-}tada}
  
  \CommentTok{\# Replace $XDG\_CACHE\_HOME with the appropriate directory based on}
  \CommentTok{\# https://github.com/typst/packages\#downloads}
  \ExtensionTok{python}\NormalTok{ package.py ./typst.toml }\StringTok{"}\VariableTok{$XDG\_CACHE\_HOME}\StringTok{/typst/packages"} \DataTypeTok{\textbackslash{}}
    \AttributeTok{{-}{-}namespace}\NormalTok{ local}
  
\end{Highlighting}
\end{Shaded}

Now, TaDa is available under the local namespace:

\begin{Shaded}
\begin{Highlighting}[]
\NormalTok{\#import "@local/tada:0.1.0"}
\end{Highlighting}
\end{Shaded}

\subsection{Creation}\label{creation}

TaDa provides three main ways to construct tables â€`` from columns,
rows, or records.

\begin{itemize}
\item
  \textbf{Columns} are a dictionary of field names to column values.
  Alternatively, a 2D array of columns can be passed to
  \texttt{\ from-columns\ } , where \texttt{\ values.at(0)\ } is a
  column (belongs to one field).
\item
  \textbf{Records} are a 1D array of dictionaries where each dictionary
  is a row.
\item
  \textbf{Rows} are a 2D array where \texttt{\ values.at(0)\ } is a row
  (has one value for each field). Note that if \texttt{\ rows\ } are
  given without field names, they default to (0, 1, …\$n\$).
\end{itemize}

\begin{Shaded}
\begin{Highlighting}[]
\NormalTok{\#let column{-}data = (}
\NormalTok{  name: ("Bread", "Milk", "Eggs"),}
\NormalTok{  price: (1.25, 2.50, 1.50),}
\NormalTok{  quantity: (2, 1, 3),}
\NormalTok{)}
\NormalTok{\#let record{-}data = (}
\NormalTok{  (name: "Bread", price: 1.25, quantity: 2),}
\NormalTok{  (name: "Milk", price: 2.50, quantity: 1),}
\NormalTok{  (name: "Eggs", price: 1.50, quantity: 3),}
\NormalTok{)}
\NormalTok{\#let row{-}data = (}
\NormalTok{  ("Bread", 1.25, 2),}
\NormalTok{  ("Milk", 2.50, 1),}
\NormalTok{  ("Eggs", 1.50, 3),}
\NormalTok{)}

\NormalTok{\#import tada: TableData}
\NormalTok{\#let td = TableData(data: column{-}data)}
\NormalTok{// Equivalent to:}
\NormalTok{\#let td2 = tada.from{-}records(record{-}data)}
\NormalTok{// \_Not\_ equivalent to (since field names are unknown):}
\NormalTok{\#let td3 = tada.from{-}rows(row{-}data)}

\NormalTok{\#to{-}tablex(td)}
\NormalTok{\#to{-}tablex(td2)}
\NormalTok{\#to{-}tablex(td3)}
\end{Highlighting}
\end{Shaded}

\pandocbounded{\includegraphics[keepaspectratio]{https://raw.githubusercontent.com/ntjess/typst-tada/v0.1.0/assets/example-01.png}}

\subsection{Title formatting}\label{title-formatting}

You can pass any \texttt{\ content\ } as a field’s \texttt{\ title\ }
. \textbf{Note} : if you pass a string, it will be evaluated as markup.

\begin{Shaded}
\begin{Highlighting}[]
\NormalTok{\#let fmt(it) = \{}
\NormalTok{  heading(outlined: false,}
\NormalTok{    upper(it.at(0))}
\NormalTok{    + it.slice(1).replace("\_", " ")}
\NormalTok{  )}
\NormalTok{\}}

\NormalTok{\#let titles = (}
\NormalTok{  // As a function}
\NormalTok{  name: (title: fmt),}
\NormalTok{  // As a string}
\NormalTok{  quantity: (title: fmt("Qty")),}
\NormalTok{)}
\NormalTok{\#let td = TableData(..td, field{-}info: titles)}

\NormalTok{\#to{-}tablex(td)}
\end{Highlighting}
\end{Shaded}

\pandocbounded{\includegraphics[keepaspectratio]{https://raw.githubusercontent.com/ntjess/typst-tada/v0.1.0/assets/example-02.png}}

\subsection{Adapting default behavior}\label{adapting-default-behavior}

You can specify defaults for any field not explicitly populated by
passing information to \texttt{\ field-defaults\ } . Observe in the last
example that \texttt{\ price\ } was not given a title. We can indicate
it should be formatted the same as \texttt{\ name\ } by passing
\texttt{\ title:\ fmt\ } to \texttt{\ field-defaults\ } . \textbf{Note}
that any field that is explicitly given a value will not be affected by
\texttt{\ field-defaults\ } (i.e., \texttt{\ quantity\ } will retain its
string title “Qty�)

\begin{Shaded}
\begin{Highlighting}[]
\NormalTok{\#let defaults = (title: fmt)}
\NormalTok{\#let td = TableData(..td, field{-}defaults: defaults)}
\NormalTok{\#to{-}tablex(td)}
\end{Highlighting}
\end{Shaded}

\pandocbounded{\includegraphics[keepaspectratio]{https://raw.githubusercontent.com/ntjess/typst-tada/v0.1.0/assets/example-03.png}}

\subsection{\texorpdfstring{Using
\texttt{\ \_\_index\ }}{Using  \_\_index }}\label{using-__index}

TaDa will automatically add an \texttt{\ \_\_index\ } field to each row
that is hidden by default. If you want it displayed, update its
information to set \texttt{\ hide:\ false\ } :

\begin{Shaded}
\begin{Highlighting}[]
\NormalTok{// Use the helper function \textasciigrave{}update{-}fields\textasciigrave{} to update multiple fields}
\NormalTok{// and/or attributes}
\NormalTok{\#import tada: update{-}fields}
\NormalTok{\#let td = update{-}fields(}
\NormalTok{  td, \_\_index: (hide: false, title: "\textbackslash{}\#")}
\NormalTok{)}
\NormalTok{// You can also insert attributes directly:}
\NormalTok{// \#td.field{-}info.\_\_index.insert("hide", false)}
\NormalTok{// etc.}
\NormalTok{\#to{-}tablex(td)}
\end{Highlighting}
\end{Shaded}

\pandocbounded{\includegraphics[keepaspectratio]{https://raw.githubusercontent.com/ntjess/typst-tada/v0.1.0/assets/example-04.png}}

\subsection{Value formatting}\label{value-formatting}

\subsubsection{\texorpdfstring{\texttt{\ type\ }}{ type }}\label{type}

Type information can have attached metadata that specifies alignment,
display formats, and more. Available types and their metadata are:

\begin{itemize}
\item
  \textbf{string} : (default-value: "", align: left)
\item
  \textbf{content} : (display: , align: left)
\item
  \textbf{float} : (align: right)
\item
  \textbf{integer} : (align: right)
\item
  \textbf{percent} : (display: , align: right)
\item
  \textbf{index} : (align: right)
\end{itemize}

While adding your own default types is not yet supported, you can simply
defined a dictionary of specifications and pass its keys to the field

\begin{Shaded}
\begin{Highlighting}[]
\NormalTok{\#let currency{-}info = (}
\NormalTok{  display: tada.display.format{-}usd, align: right}
\NormalTok{)}
\NormalTok{\#td.field{-}info.insert("price", (type: "currency"))}
\NormalTok{\#let td = TableData(..td, type{-}info: ("currency": currency{-}info))}
\NormalTok{\#to{-}tablex(td)}
\end{Highlighting}
\end{Shaded}

\pandocbounded{\includegraphics[keepaspectratio]{https://raw.githubusercontent.com/ntjess/typst-tada/v0.1.0/assets/example-05.png}}

\subsection{Transposing}\label{transposing}

\texttt{\ transpose\ } is supported, but keep in mind if columns have
different types, an error will be a frequent result. To avoid the error,
explicitly pass \texttt{\ ignore-types:\ true\ } . You can choose
whether to keep field names as an additional column by passing a string
to \texttt{\ fields-name\ } that is evaluated as markup:

\begin{Shaded}
\begin{Highlighting}[]
\NormalTok{\#to{-}tablex(}
\NormalTok{  tada.transpose(}
\NormalTok{    td, ignore{-}types: true, fields{-}name: ""}
\NormalTok{  )}
\NormalTok{)}
\end{Highlighting}
\end{Shaded}

\pandocbounded{\includegraphics[keepaspectratio]{https://raw.githubusercontent.com/ntjess/typst-tada/v0.1.0/assets/example-06.png}}

\subsubsection{\texorpdfstring{\texttt{\ display\ }}{ display }}\label{display}

If your type is not available or you want to customize its display, pass
a \texttt{\ display\ } function that formats the value, or a string that
accesses \texttt{\ value\ } in its scope:

\begin{Shaded}
\begin{Highlighting}[]
\NormalTok{\#td.field{-}info.at("quantity").insert(}
\NormalTok{  "display",}
\NormalTok{  val =\textgreater{} ("/", "One", "Two", "Three").at(val),}
\NormalTok{)}

\NormalTok{\#let td = TableData(..td)}
\NormalTok{\#to{-}tablex(td)}
\end{Highlighting}
\end{Shaded}

\pandocbounded{\includegraphics[keepaspectratio]{https://raw.githubusercontent.com/ntjess/typst-tada/v0.1.0/assets/example-07.png}}

\subsubsection{\texorpdfstring{\texttt{\ align\ }
etc.}{ align  etc.}}\label{align-etc.}

You can pass \texttt{\ align\ } and \texttt{\ width\ } to a given
field’s metadata to determine how content aligns in the cell and how
much horizontal space it takes up. In the future, more
\texttt{\ tablex\ } setup arguments will be accepted.

\begin{Shaded}
\begin{Highlighting}[]
\NormalTok{\#let adjusted = update{-}fields(}
\NormalTok{  td, name: (align: center, width: 1.4in)}
\NormalTok{)}
\NormalTok{\#to{-}tablex(adjusted)}
\end{Highlighting}
\end{Shaded}

\pandocbounded{\includegraphics[keepaspectratio]{https://raw.githubusercontent.com/ntjess/typst-tada/v0.1.0/assets/example-08.png}}

\subsection{\texorpdfstring{Deeper \texttt{\ tablex\ }
customization}{Deeper  tablex  customization}}\label{deeper-tablex-customization}

TaDa uses \texttt{\ tablex\ } to display the table. So any argument that
\texttt{\ tablex\ } accepts can be passed to TableData as well:

\begin{Shaded}
\begin{Highlighting}[]
\NormalTok{\#let mapper = (index, row) =\textgreater{} \{}
\NormalTok{  let fill = if index == 0 \{rgb("\#8888")\} else \{none\}}
\NormalTok{  row.map(cell =\textgreater{} (..cell, fill: fill))}
\NormalTok{\}}
\NormalTok{\#let td = TableData(}
\NormalTok{  ..td,}
\NormalTok{  tablex{-}kwargs: (}
\NormalTok{    map{-}rows: mapper, auto{-}vlines: false}
\NormalTok{  ),}
\NormalTok{)}
\NormalTok{\#to{-}tablex(td)}
\end{Highlighting}
\end{Shaded}

\pandocbounded{\includegraphics[keepaspectratio]{https://raw.githubusercontent.com/ntjess/typst-tada/v0.1.0/assets/example-09.png}}

\subsection{Subselection}\label{subselection}

You can select a subset of fields or rows to display:

\begin{Shaded}
\begin{Highlighting}[]
\NormalTok{\#import tada: subset}
\NormalTok{\#to{-}tablex(}
\NormalTok{  subset(td, indexes: (0,2), fields: ("name", "price"))}
\NormalTok{)}
\end{Highlighting}
\end{Shaded}

\pandocbounded{\includegraphics[keepaspectratio]{https://raw.githubusercontent.com/ntjess/typst-tada/v0.1.0/assets/example-10.png}}

Note that \texttt{\ indexes\ } is based on the table’s
\texttt{\ \_\_index\ } column, \emph{not} it’s positional index within
the table:

\begin{Shaded}
\begin{Highlighting}[]
\NormalTok{\#let td2 = td}
\NormalTok{\#td2.data.insert("\_\_index", (1, 2, 2))}
\NormalTok{\#to{-}tablex(}
\NormalTok{  subset(td2, indexes: 2, fields: ("\_\_index", "name"))}
\NormalTok{)}
\end{Highlighting}
\end{Shaded}

\pandocbounded{\includegraphics[keepaspectratio]{https://raw.githubusercontent.com/ntjess/typst-tada/v0.1.0/assets/example-11.png}}

Rows can also be selected by whether they fulfill a field condition:

\begin{Shaded}
\begin{Highlighting}[]
\NormalTok{\#to{-}tablex(}
\NormalTok{  tada.filter(td, expression: "price \textless{} 1.5")}
\NormalTok{)}
\end{Highlighting}
\end{Shaded}

\pandocbounded{\includegraphics[keepaspectratio]{https://raw.githubusercontent.com/ntjess/typst-tada/v0.1.0/assets/example-12.png}}

\subsection{Concatenation}\label{concatenation}

Concatenating rows and columns are both supported operations, but only
in the simple sense of stacking the data. Currently, there is no ability
to join on a field or otherwise intelligently merge data.

\begin{itemize}
\item
  \texttt{\ axis:\ 0\ } places new rows below current rows
\item
  \texttt{\ axis:\ 1\ } places new columns to the right of current
  columns
\item
  Unless you specify a fill value for missing values, the function will
  panic if the tables do not match exactly along their concatenation
  axis.
\item
  You cannot stack with \texttt{\ axis:\ 1\ } unless every column has a
  unique field name.
\end{itemize}

\begin{Shaded}
\begin{Highlighting}[]
\NormalTok{\#import tada: stack}

\NormalTok{\#let td2 = TableData(}
\NormalTok{  data: (}
\NormalTok{    name: ("Cheese", "Butter"),}
\NormalTok{    price: (2.50, 1.75),}
\NormalTok{  )}
\NormalTok{)}
\NormalTok{\#let td3 = TableData(}
\NormalTok{  data: (}
\NormalTok{    rating: (4.5, 3.5, 5.0, 4.0, 2.5),}
\NormalTok{  )}
\NormalTok{)}

\NormalTok{// This would fail without specifying the fill}
\NormalTok{// since \textasciigrave{}quantity\textasciigrave{} is missing from \textasciigrave{}td2\textasciigrave{}}
\NormalTok{\#let stack{-}a = stack(td, td2, missing{-}fill: 0)}
\NormalTok{\#let stack{-}b = stack(stack{-}a, td3, axis: 1)}
\NormalTok{\#to{-}tablex(stack{-}b)}
\end{Highlighting}
\end{Shaded}

\pandocbounded{\includegraphics[keepaspectratio]{https://raw.githubusercontent.com/ntjess/typst-tada/v0.1.0/assets/example-13.png}}

\subsection{Expressions}\label{expressions}

The easiest way to leverage TaDa’s flexibility is through expressions.
They can be strings that treat field names as variables, or functions
that take keyword-only arguments.

\begin{itemize}
\tightlist
\item
  \textbf{Note} ! When passing functions, every field is passed as a
  named argument to the function. So, make sure to capture unused fields
  with \texttt{\ ..rest\ } (the name is unimportant) to avoid errors.
\end{itemize}

\begin{Shaded}
\begin{Highlighting}[]
\NormalTok{\#let make{-}dict(field, expression) = \{}
\NormalTok{  let out = (:)}
\NormalTok{  out.insert(}
\NormalTok{    field,}
\NormalTok{    (expression: expression, type: "currency"),}
\NormalTok{  )}
\NormalTok{  out}
\NormalTok{\}}

\NormalTok{\#let td = update{-}fields(}
\NormalTok{  td, ..make{-}dict("total", "price * quantity" )}
\NormalTok{)}

\NormalTok{\#let tax{-}expr(total: none, ..rest) = \{ total * 0.2 \}}
\NormalTok{\#let taxed = update{-}fields(}
\NormalTok{  td, ..make{-}dict("tax", tax{-}expr),}
\NormalTok{)}

\NormalTok{\#to{-}tablex(}
\NormalTok{  subset(taxed, fields: ("name", "total", "tax"))}
\NormalTok{)}
\end{Highlighting}
\end{Shaded}

\pandocbounded{\includegraphics[keepaspectratio]{https://raw.githubusercontent.com/ntjess/typst-tada/v0.1.0/assets/example-14.png}}

\subsection{Chaining}\label{chaining}

It is inconvenient to require several temporary variables as above, or
deep function nesting, to perform multiple operations on a table. TaDa
provides a \texttt{\ chain\ } function to make this easier. Furthermore,
when you need to compute several fields at once and don’t need extra
field information, you can use \texttt{\ add-expressions\ } as a
shorthand:

\begin{Shaded}
\begin{Highlighting}[]
\NormalTok{\#import tada: chain, add{-}expressions}
\NormalTok{\#let totals = chain(td,}
\NormalTok{  add{-}expressions.with(}
\NormalTok{    total: "price * quantity",}
\NormalTok{    tax: "total * 0.2",}
\NormalTok{    after{-}tax: "total + tax",}
\NormalTok{  ),}
\NormalTok{  subset.with(}
\NormalTok{    fields: ("name", "total", "after{-}tax")}
\NormalTok{  ),}
\NormalTok{  // Add type information}
\NormalTok{  update{-}fields.with(}
\NormalTok{    after{-}tax: (type: "currency", title: fmt("w/ Tax")),}
\NormalTok{  ),}
\NormalTok{)}
\NormalTok{\#to{-}tablex(totals)}
\end{Highlighting}
\end{Shaded}

\pandocbounded{\includegraphics[keepaspectratio]{https://raw.githubusercontent.com/ntjess/typst-tada/v0.1.0/assets/example-15.png}}

\subsection{Sorting}\label{sorting}

You can sort by ascending/descending values of any field, or provide
your own transformation function to the \texttt{\ key\ } argument to
customize behavior further:

\begin{Shaded}
\begin{Highlighting}[]
\NormalTok{\#import tada: sort{-}values}
\NormalTok{\#to{-}tablex(sort{-}values(}
\NormalTok{  td, by: "quantity", descending: true}
\NormalTok{))}
\end{Highlighting}
\end{Shaded}

\pandocbounded{\includegraphics[keepaspectratio]{https://raw.githubusercontent.com/ntjess/typst-tada/v0.1.0/assets/example-16.png}}

\subsection{Aggregation}\label{aggregation}

Column-wise reduction is supported through \texttt{\ agg\ } , using
either functions or string expressions:

\begin{Shaded}
\begin{Highlighting}[]
\NormalTok{\#import tada: agg, item}
\NormalTok{\#let grand{-}total = chain(}
\NormalTok{  totals,}
\NormalTok{  agg.with(after{-}tax: array.sum),}
\NormalTok{  // use "item" to extract exactly one element}
\NormalTok{  item}
\NormalTok{)}
\NormalTok{// "Output" is a helper function just for these docs.}
\NormalTok{// It is not necessary in your code.}
\NormalTok{\#output[}
\NormalTok{  *Grand total: \#tada.display.format{-}usd(grand{-}total)*}
\NormalTok{]}
\end{Highlighting}
\end{Shaded}

\pandocbounded{\includegraphics[keepaspectratio]{https://raw.githubusercontent.com/ntjess/typst-tada/v0.1.0/assets/example-17.png}}

It is also easy to aggregate several expressions at once:

\begin{Shaded}
\begin{Highlighting}[]
\NormalTok{\#let agg{-}exprs = (}
\NormalTok{  "\# items": "quantity.sum()",}
\NormalTok{  "Longest name": "[\#name.sorted(key: str.len).at({-}1)]",}
\NormalTok{)}
\NormalTok{\#let agg{-}td = tada.agg(td, ..agg{-}exprs)}
\NormalTok{\#to{-}tablex(agg{-}td)}
\end{Highlighting}
\end{Shaded}

\pandocbounded{\includegraphics[keepaspectratio]{https://raw.githubusercontent.com/ntjess/typst-tada/v0.1.0/assets/example-18.png}}

\subsubsection{How to add}\label{how-to-add}

Copy this into your project and use the import as \texttt{\ tada\ }

\begin{verbatim}
#import "@preview/tada:0.1.0"
\end{verbatim}

\includesvg[width=0.16667in,height=0.16667in]{/assets/icons/16-copy.svg}

Check the docs for
\href{https://typst.app/docs/reference/scripting/\#packages}{more
information on how to import packages} .

\subsubsection{About}\label{about}

\begin{description}
\tightlist
\item[Author :]
Nathan Jessurun
\item[License:]
Unlicense
\item[Current version:]
0.1.0
\item[Last updated:]
December 15, 2023
\item[First released:]
December 15, 2023
\item[Archive size:]
16.2 kB
\href{https://packages.typst.org/preview/tada-0.1.0.tar.gz}{\pandocbounded{\includesvg[keepaspectratio]{/assets/icons/16-download.svg}}}
\item[Repository:]
\href{https://github.com/ntjess/typst-tada}{GitHub}
\end{description}

\subsubsection{Where to report issues?}\label{where-to-report-issues}

This package is a project of Nathan Jessurun . Report issues on
\href{https://github.com/ntjess/typst-tada}{their repository} . You can
also try to ask for help with this package on the
\href{https://forum.typst.app}{Forum} .

Please report this package to the Typst team using the
\href{https://typst.app/contact}{contact form} if you believe it is a
safety hazard or infringes upon your rights.

\phantomsection\label{versions}
\subsubsection{Version history}\label{version-history}

\begin{longtable}[]{@{}ll@{}}
\toprule\noalign{}
Version & Release Date \\
\midrule\noalign{}
\endhead
\bottomrule\noalign{}
\endlastfoot
0.1.0 & December 15, 2023 \\
\end{longtable}

Typst GmbH did not create this package and cannot guarantee correct
functionality of this package or compatibility with any version of the
Typst compiler or app.


\section{Package List LaTeX/modern-ysu-thesis.tex}
\title{typst.app/universe/package/modern-ysu-thesis}

\phantomsection\label{banner}
\phantomsection\label{template-thumbnail}
\pandocbounded{\includegraphics[keepaspectratio]{https://packages.typst.org/preview/thumbnails/modern-ysu-thesis-0.1.0-small.webp}}

\section{modern-ysu-thesis}\label{modern-ysu-thesis}

{ 0.1.0 }

燕山大学学ä½?论æ--‡æ¨¡æ?¿ã€‚Modern Yanshan University Thesis.

\href{/app?template=modern-ysu-thesis&version=0.1.0}{Create project in
app}

\phantomsection\label{readme}
本模�在
\href{https://github.com/nju-lug/modern-nju-thesis}{modern-nju-thesis}
的基础上修æ''¹è€Œæ?¥

\begin{quote}
{[}!WARNING{]}

本模æ?¿æ­£å¤„于积æž?å¼€å?{}`阶段,存在一些æ~¼å¼?é---®é¢˜ï¼Œé€‚å?ˆå°?鲜
Typst 特性

本模æ?¿æ˜¯æ°`é---´æ¨¡æ?¿ï¼Œ \textbf{å?¯èƒ½ä¸?被学æ~¡è®¤å?¯}
,正å¼?使ç''¨è¿‡ç¨‹ä¸­è¯·å?šå¥½éš?æ---¶å°†å†\ldots 容è¿?移至 Word
æˆ-- LaTeX 的准备
\end{quote}

clone 本项目å?Žï¼Œç\ldots§ç?€ template\textbackslash thesis.typ
下写就行

\subsection{致谢}\label{uxe8uxe8}

\begin{itemize}
\tightlist
\item
  æ„Ÿè°¢
  \href{https://github.com/nju-lug/modern-nju-thesis}{modern-nju-thesis}
  Typst 中æ--‡è®ºæ--‡æ¨¡æ?¿ã€‚
\end{itemize}

\subsection{License}\label{license}

This project is licensed under the MIT License.

\href{/app?template=modern-ysu-thesis&version=0.1.0}{Create project in
app}

\subsubsection{How to use}\label{how-to-use}

Click the button above to create a new project using this template in
the Typst app.

You can also use the Typst CLI to start a new project on your computer
using this command:

\begin{verbatim}
typst init @preview/modern-ysu-thesis:0.1.0
\end{verbatim}

\includesvg[width=0.16667in,height=0.16667in]{/assets/icons/16-copy.svg}

\subsubsection{About}\label{about}

\begin{description}
\tightlist
\item[Author :]
Woodman3
\item[License:]
MIT
\item[Current version:]
0.1.0
\item[Last updated:]
May 24, 2024
\item[First released:]
May 24, 2024
\item[Archive size:]
98.8 kB
\href{https://packages.typst.org/preview/modern-ysu-thesis-0.1.0.tar.gz}{\pandocbounded{\includesvg[keepaspectratio]{/assets/icons/16-download.svg}}}
\item[Repository:]
\href{https://github.com/Woodman3/modern-ysu-thesis}{GitHub}
\item[Categor y :]
\begin{itemize}
\tightlist
\item[]
\item
  \pandocbounded{\includesvg[keepaspectratio]{/assets/icons/16-mortarboard.svg}}
  \href{https://typst.app/universe/search/?category=thesis}{Thesis}
\end{itemize}
\end{description}

\subsubsection{Where to report issues?}\label{where-to-report-issues}

This template is a project of Woodman3 . Report issues on
\href{https://github.com/Woodman3/modern-ysu-thesis}{their repository} .
You can also try to ask for help with this template on the
\href{https://forum.typst.app}{Forum} .

Please report this template to the Typst team using the
\href{https://typst.app/contact}{contact form} if you believe it is a
safety hazard or infringes upon your rights.

\phantomsection\label{versions}
\subsubsection{Version history}\label{version-history}

\begin{longtable}[]{@{}ll@{}}
\toprule\noalign{}
Version & Release Date \\
\midrule\noalign{}
\endhead
\bottomrule\noalign{}
\endlastfoot
0.1.0 & May 24, 2024 \\
\end{longtable}

Typst GmbH did not create this template and cannot guarantee correct
functionality of this template or compatibility with any version of the
Typst compiler or app.


\section{Package List LaTeX/basic-resume.tex}
\title{typst.app/universe/package/basic-resume}

\phantomsection\label{banner}
\phantomsection\label{template-thumbnail}
\pandocbounded{\includegraphics[keepaspectratio]{https://packages.typst.org/preview/thumbnails/basic-resume-0.2.0-small.webp}}

\section{basic-resume}\label{basic-resume}

{ 0.2.0 }

A simple, standard resume, designed to work well with ATS.

{ } Featured Template

\href{/app?template=basic-resume&version=0.2.0}{Create project in app}

\phantomsection\label{readme}
Version 0.2.0

This is a template for a simple resume. It is intended to be used as a
good starting point for quickly crafting a standard resume that will
properly be parsed by ATS systems. Inspiration is taken from
\href{https://github.com/jakegut/resume}{Jake’s Resume} and
\href{https://typst.app/universe/package/guided-resume-starter-cgc/}{guided-resume-starter-cgc}
. I’m currently a college student and was unable to find a Typst
resume template that fit my needs, so I wrote my own. I hope this
template can be useful to others as well.

\subsection{Sample Resume}\label{sample-resume}

\pandocbounded{\includegraphics[keepaspectratio]{https://raw.githubusercontent.com/stuxf/basic-typst-resume-template/main/example-resume.png}}

\subsection{Quick Start}\label{quick-start}

A barebones resume looks like this, which you can use to get started.

\begin{Shaded}
\begin{Highlighting}[]
\NormalTok{\#import "@preview/basic{-}resume:0.2.0": *}

\NormalTok{// Put your personal information here, replacing mine}
\NormalTok{\#let name = "Stephen Xu"}
\NormalTok{\#let location = "San Diego, CA"}
\NormalTok{\#let email = "stxu@hmc.edu"}
\NormalTok{\#let github = "github.com/stuxf"}
\NormalTok{\#let linkedin = "linkedin.com/in/stuxf"}
\NormalTok{\#let phone = "+1 (xxx) xxx{-}xxxx"}
\NormalTok{\#let personal{-}site = "stuxf.dev"}

\NormalTok{\#show: resume.with(}
\NormalTok{  author: name,}
\NormalTok{  // All the lines below are optional. }
\NormalTok{  // For example, if you want to to hide your phone number:}
\NormalTok{  // feel free to comment those lines out and they will not show.}
\NormalTok{  location: location,}
\NormalTok{  email: email,}
\NormalTok{  github: github,}
\NormalTok{  linkedin: linkedin,}
\NormalTok{  phone: phone,}
\NormalTok{  personal{-}site: personal{-}site,}
\NormalTok{  accent{-}color: "\#26428b",}
\NormalTok{  font: "New Computer Modern",}
\NormalTok{)}

\NormalTok{/*}
\NormalTok{* Lines that start with == are formatted into section headings}
\NormalTok{* You can use the specific formatting functions if needed}
\NormalTok{* The following formatting functions are listed below}
\NormalTok{* \#edu(dates: "", degree: "", gpa: "", institution: "", location: "")}
\NormalTok{* \#work(company: "", dates: "", location: "", title: "")}
\NormalTok{* \#project(dates: "", name: "", role: "", url: "")}
\NormalTok{* \#extracurriculars(activity: "", dates: "")}
\NormalTok{* There are also the following generic functions that don\textquotesingle{}t apply any formatting}
\NormalTok{* \#generic{-}two{-}by{-}two(top{-}left: "", top{-}right: "", bottom{-}left: "", bottom{-}right: "")}
\NormalTok{* \#generic{-}one{-}by{-}two(left: "", right: "")}
\NormalTok{*/}
\NormalTok{== Education}

\NormalTok{\#edu(}
\NormalTok{  institution: "Harvey Mudd College",}
\NormalTok{  location: "Claremont, CA",}
\NormalTok{  dates: dates{-}helper(start{-}date: "Aug 2023", end{-}date: "May 2027"),}
\NormalTok{  degree: "Bachelor\textquotesingle{}s of Science, Computer Science and Mathematics",}
\NormalTok{)}
\NormalTok{{-} Cumulative GPA: 4.0\textbackslash{}/4.0 | Dean\textquotesingle{}s List, Harvey S. Mudd Merit Scholarship, National Merit Scholarship}
\NormalTok{{-} Relevant Coursework: Data Structures, Program Development, Microprocessors, Abstract Algebra I: Groups and Rings, Linear Algebra, Discrete Mathematics, Multivariable \& Single Variable Calculus, Principles and Practice of Comp Sci}

\NormalTok{== Work Experience}

\NormalTok{\#work(}
\NormalTok{  title: "Subatomic Shepherd and Caffeine Connoisseur",}
\NormalTok{  location: "Atomville, CA",}
\NormalTok{  company: "Microscopic Circus, Schrodinger\textquotesingle{}s University",}
\NormalTok{  dates: dates{-}helper(start{-}date: "May 2024", end{-}date: "Present"),}
\NormalTok{)}
\NormalTok{{-} more bullet points go here}

\NormalTok{// ... more headers and stuff below}
\end{Highlighting}
\end{Shaded}

\href{/app?template=basic-resume&version=0.2.0}{Create project in app}

\subsubsection{How to use}\label{how-to-use}

Click the button above to create a new project using this template in
the Typst app.

You can also use the Typst CLI to start a new project on your computer
using this command:

\begin{verbatim}
typst init @preview/basic-resume:0.2.0
\end{verbatim}

\includesvg[width=0.16667in,height=0.16667in]{/assets/icons/16-copy.svg}

\subsubsection{About}\label{about}

\begin{description}
\tightlist
\item[Author :]
\href{https://stuxf.dev}{Stephen Xu}
\item[License:]
Unlicense
\item[Current version:]
0.2.0
\item[Last updated:]
November 29, 2024
\item[First released:]
August 1, 2024
\item[Archive size:]
6.05 kB
\href{https://packages.typst.org/preview/basic-resume-0.2.0.tar.gz}{\pandocbounded{\includesvg[keepaspectratio]{/assets/icons/16-download.svg}}}
\item[Repository:]
\href{https://github.com/stuxf/basic-typst-resume-template}{GitHub}
\item[Categor y :]
\begin{itemize}
\tightlist
\item[]
\item
  \pandocbounded{\includesvg[keepaspectratio]{/assets/icons/16-user.svg}}
  \href{https://typst.app/universe/search/?category=cv}{CV}
\end{itemize}
\end{description}

\subsubsection{Where to report issues?}\label{where-to-report-issues}

This template is a project of Stephen Xu . Report issues on
\href{https://github.com/stuxf/basic-typst-resume-template}{their
repository} . You can also try to ask for help with this template on the
\href{https://forum.typst.app}{Forum} .

Please report this template to the Typst team using the
\href{https://typst.app/contact}{contact form} if you believe it is a
safety hazard or infringes upon your rights.

\phantomsection\label{versions}
\subsubsection{Version history}\label{version-history}

\begin{longtable}[]{@{}ll@{}}
\toprule\noalign{}
Version & Release Date \\
\midrule\noalign{}
\endhead
\bottomrule\noalign{}
\endlastfoot
0.2.0 & November 29, 2024 \\
\href{https://typst.app/universe/package/basic-resume/0.1.4/}{0.1.4} &
November 12, 2024 \\
\href{https://typst.app/universe/package/basic-resume/0.1.3/}{0.1.3} &
October 15, 2024 \\
\href{https://typst.app/universe/package/basic-resume/0.1.2/}{0.1.2} &
October 7, 2024 \\
\href{https://typst.app/universe/package/basic-resume/0.1.0/}{0.1.0} &
August 1, 2024 \\
\end{longtable}

Typst GmbH did not create this template and cannot guarantee correct
functionality of this template or compatibility with any version of the
Typst compiler or app.


\section{Package List LaTeX/classic-jmlr.tex}
\title{typst.app/universe/package/classic-jmlr}

\phantomsection\label{banner}
\phantomsection\label{template-thumbnail}
\pandocbounded{\includegraphics[keepaspectratio]{https://packages.typst.org/preview/thumbnails/classic-jmlr-0.4.0-small.webp}}

\section{classic-jmlr}\label{classic-jmlr}

{ 0.4.0 }

Paper template for submission to Journal of Machine Learning Research
(JMLR)

\href{/app?template=classic-jmlr&version=0.4.0}{Create project in app}

\phantomsection\label{readme}
\subsection{Overview}\label{overview}

This is a Typst template for Journal of Machine Learning Research
(JMLR). It is based on official
\href{https://www.jmlr.org/format/authors-guide.html}{author guide} ,
\href{https://www.jmlr.org/format/format.html}{formatting instructions}
, and
\href{https://www.jmlr.org/format/formatting-errors.html}{formatting
error checklist} as well as the official
\href{https://github.com/jmlrorg/jmlr-style-file}{example paper} .

\subsection{Usage}\label{usage}

You can use this template in the Typst web app by clicking \emph{Start
from template} on the dashboard and searching for
\texttt{\ classic-jmlr\ } .

Alternatively, you can use the CLI to kick this project off using the
command

\begin{Shaded}
\begin{Highlighting}[]
\NormalTok{typst init @preview/classic{-}jmlr}
\end{Highlighting}
\end{Shaded}

Typst will create a new directory with all the files needed to get you
started.

\subsection{Configuration}\label{configuration}

This template exports the \texttt{\ jmlr\ } function with the following
named arguments.

\begin{itemize}
\tightlist
\item
  \texttt{\ title\ } : The paper’s title as content.
\item
  \texttt{\ short-title\ } : Paper short title (for page header).
\item
  \texttt{\ authors\ } : An array of author dictionaries. Each of the
  author dictionaries must have a name key and can have the keys
  department, organization, location, and email.
\item
  \texttt{\ last-names\ } : List of authors last names (for page
  header).
\item
  \texttt{\ keywords\ } : Publication keywords (used in PDF metadata).
\item
  \texttt{\ date\ } : Creation date (used in PDF metadata).
\item
  \texttt{\ abstract\ } : The content of a brief summary of the paper or
  none. Appears at the top under the title.
\item
  \texttt{\ bibliography\ } : The result of a call to the bibliography
  function or none. The function also accepts a single, positional
  argument for the body of the paper.
\item
  \texttt{\ appendix\ } : Content to append after bibliography section.
\item
  \texttt{\ pubdata\ } : Dictionary with auxiliary information about
  publication. It contains editor name(s), paper id, volume, and
  submission/review/publishing dates.
\end{itemize}

The template will initialize your package with a sample call to the
\texttt{\ jmlr\ } function in a show rule. If you want to change an
existing project to use this template, you can add a show rule at the
top of your file.

\begin{Shaded}
\begin{Highlighting}[]
\NormalTok{\#import "@preview/classic{-}jmlr": jmlr}
\NormalTok{\#show: jmlr.with(}
\NormalTok{  title: [Sample JMLR Paper],}
\NormalTok{  authors: (authors, affls),}
\NormalTok{  abstract: blindtext,}
\NormalTok{  keywords: ("keyword one", "keyword two", "keyword three"),}
\NormalTok{  bibliography: bibliography("main.bib"),}
\NormalTok{  appendix: include "appendix.typ",}
\NormalTok{  pubdata: (}
\NormalTok{    id: "21{-}0000",}
\NormalTok{    editor: "My editor",}
\NormalTok{    volume: 23,}
\NormalTok{    submitted{-}at: datetime(year: 2021, month: 1, day: 1),}
\NormalTok{    revised{-}at: datetime(year: 2022, month: 5, day: 1),}
\NormalTok{    published{-}at: datetime(year: 2022, month: 9, day: 1),}
\NormalTok{  ),}
\NormalTok{)}
\end{Highlighting}
\end{Shaded}

\subsection{Issues}\label{issues}

This template is developed at
\href{https://github.com/daskol/typst-templates}{daskol/typst-templates}
repo. Please report all issues there.

\begin{itemize}
\item
  Original JMLR example paper is not not representative. It does not
  demonstrate appearance of figures, images, tables, lists, etc.
\item
  Leading in author affilations in in the original template is varying.
\item
  There is no bibliography CSL-style. The closest one is
  \texttt{\ bristol-university-press\ } .
\item
  Another issue is related to Typst’s inablity to produce colored
  annotation. In order to mitigte the issue, we add a script which
  modifies annotations and make them colored.

\begin{Shaded}
\begin{Highlighting}[]
\NormalTok{../colorize{-}annotations.py \textbackslash{}}
\NormalTok{    example{-}paper.typst.pdf example{-}paper{-}colored.typst.pdf}
\end{Highlighting}
\end{Shaded}

  See
  \href{https://github.com/daskol/typst-templates/\#colored-annotations}{README.md}
  for details.
\end{itemize}

\href{/app?template=classic-jmlr&version=0.4.0}{Create project in app}

\subsubsection{How to use}\label{how-to-use}

Click the button above to create a new project using this template in
the Typst app.

You can also use the Typst CLI to start a new project on your computer
using this command:

\begin{verbatim}
typst init @preview/classic-jmlr:0.4.0
\end{verbatim}

\includesvg[width=0.16667in,height=0.16667in]{/assets/icons/16-copy.svg}

\subsubsection{About}\label{about}

\begin{description}
\tightlist
\item[Author :]
\href{mailto:d.bershatsky2@skoltech.ru}{Daniel Bershatsky}
\item[License:]
MIT
\item[Current version:]
0.4.0
\item[Last updated:]
April 19, 2024
\item[First released:]
April 19, 2024
\item[Minimum Typst version:]
0.11.0
\item[Archive size:]
8.60 kB
\href{https://packages.typst.org/preview/classic-jmlr-0.4.0.tar.gz}{\pandocbounded{\includesvg[keepaspectratio]{/assets/icons/16-download.svg}}}
\item[Repository:]
\href{https://github.com/daskol/typst-templates}{GitHub}
\item[Discipline s :]
\begin{itemize}
\tightlist
\item[]
\item
  \href{https://typst.app/universe/search/?discipline=computer-science}{Computer
  Science}
\item
  \href{https://typst.app/universe/search/?discipline=mathematics}{Mathematics}
\end{itemize}
\item[Categor y :]
\begin{itemize}
\tightlist
\item[]
\item
  \pandocbounded{\includesvg[keepaspectratio]{/assets/icons/16-atom.svg}}
  \href{https://typst.app/universe/search/?category=paper}{Paper}
\end{itemize}
\end{description}

\subsubsection{Where to report issues?}\label{where-to-report-issues}

This template is a project of Daniel Bershatsky . Report issues on
\href{https://github.com/daskol/typst-templates}{their repository} . You
can also try to ask for help with this template on the
\href{https://forum.typst.app}{Forum} .

Please report this template to the Typst team using the
\href{https://typst.app/contact}{contact form} if you believe it is a
safety hazard or infringes upon your rights.

\phantomsection\label{versions}
\subsubsection{Version history}\label{version-history}

\begin{longtable}[]{@{}ll@{}}
\toprule\noalign{}
Version & Release Date \\
\midrule\noalign{}
\endhead
\bottomrule\noalign{}
\endlastfoot
0.4.0 & April 19, 2024 \\
\end{longtable}

Typst GmbH did not create this template and cannot guarantee correct
functionality of this template or compatibility with any version of the
Typst compiler or app.


\section{Package List LaTeX/shane-hhu-thesis.tex}
\title{typst.app/universe/package/shane-hhu-thesis}

\phantomsection\label{banner}
\phantomsection\label{template-thumbnail}
\pandocbounded{\includegraphics[keepaspectratio]{https://packages.typst.org/preview/thumbnails/shane-hhu-thesis-0.2.0-small.webp}}

\section{shane-hhu-thesis}\label{shane-hhu-thesis}

{ 0.2.0 }

河海大学本ç§`ç''Ÿæ¯•ä¸šè®ºæ--‡ï¼ˆè®¾è®¡ï¼‰æ¨¡æ?¿ã€‚Unofficial Hohai
University Undergraduate Thesis (Design) Template.

\href{/app?template=shane-hhu-thesis&version=0.2.0}{Create project in
app}

\phantomsection\label{readme}
使ç''¨ Typst
制作的河海大学「本ç§`毕业设计(论æ--‡ï¼‰æŠ¥å`Šã€?模æ?¿ï¼ˆå·¥ç§`)。官æ--¹æ¨¡æ?¿å?‚考
\href{https://bylw.hhu.edu.cn/UpLoadFile/83cd5f1169974a0db06d865c7ee11af4.pdf}{河海大学本ç§`毕业设计(论æ--‡ï¼‰è§„范æ~¼å¼?å?‚考}

\begin{quote}
{[}!IMPORTANT{]}

此模æ?¿é?žå®˜æ--¹æ¨¡æ?¿ï¼Œå?¯èƒ½ä»?存在一些é---®é¢˜ï¼Œå?Žç»­ä¼šä¸?æ--­æ›´æ--°å®Œå--„。

此模æ?¿ä»\ldots 适ç''¨äºŽå·¥ç§`ä¸``业本ç§`毕业论æ--‡ï¼ˆè®¾è®¡ï¼‰ï¼Œå?Žç»­å?¯èƒ½ä¼šæ›´æ--°æ--‡ç§`模æ?¿ã€‚

本模æ?¿ä½¿ç''¨ Typst 0.12.x ç¼--è¯`,Typst
æ›´æ--°é¢`率较高,å?¯èƒ½å‡ºçŽ°ç‰ˆæœ¬æ›´æ--°å?Žæ---~法ç¼--è¯`æˆ?功的æƒ\ldots 况。
\end{quote}

\pandocbounded{\includegraphics[keepaspectratio]{https://github.com/typst/packages/raw/main/packages/preview/shane-hhu-thesis/0.2.0/demo_images/title.png}}

\subsection{使ç''¨æ--¹æ³•}\label{uxe4uxbduxe7uxe6uxb9uxe6uxb3}

模æ?¿å·²ä¸Šä¼~ Typst Universe ,å?¯ä»¥ä½¿ç''¨ \texttt{\ typst\ init\ }
功能åˆ?始åŒ--,也å?¯ä»¥ä½¿ç''¨ Web APP ç¼--è¾`。 \textbf{Typst
Universe
上的模æ?¿å?¯èƒ½ä¸?是最æ--°ç‰ˆæœ¬ã€‚如果需è¦?使ç''¨æœ€æ--°ç‰ˆæœ¬çš„模æ?¿ï¼Œä»Žæœ¬
repo 中获å?--。}

\paragraph{本地使ç''¨ï¼ˆæŽ¨è??)}\label{uxe6ux153uxe5ux153uxe4uxbduxe7uxefuxbcux2c6uxe6ux17euxe8uxefuxbc}

使ç''¨å‰?,请å\ldots ˆå®‰è£
\texttt{\ https://github.com/shaneworld/Dots/tree/master/fonts\ }
中的å\ldots¨éƒ¨å­---ä½``。

\begin{itemize}
\item
  å\ldots‹éš†æœ¬ repo 到本地,ç¼--è¾` \texttt{\ init-files\ }
  目录å†\ldots çš„æ--‡ä»¶ã€‚
\item
  使ç''¨ \texttt{\ typst\ init\ @preview/shane-hhu-thesis:0.1.0\ }
  本地åˆ?始åŒ--模æ?¿ã€‚
\end{itemize}

\paragraph{Web APP å†\ldots 使ç''¨}\label{web-app-uxe5uxe4uxbduxe7}

ç''±äºŽ Typst Web APP
在æ¯?次æ‰``开页é?¢çš„æ---¶å€™éƒ½ä¼šä»Žæœ?务器中下载å­---ä½``,速度较æ\ldots¢ï¼Œä½``验较差,å›~æ­¤ä¸?建议使ç''¨æ­¤æ--¹æ³•ã€‚

在 \href{https://typst.app/universe/package/shane-hhu-thesis}{Typst
Universe} 中点击 \texttt{\ Create\ project\ in\ app\ }
按é'®è¿›å\ldots¥ Web APP å†\ldots 。

然�,请将
\texttt{\ https://github.com/shaneworld/Dots/tree/master/fonts\ }
å†\ldots 的所有å­---ä½``上ä¼~到 Web APP
å†\ldots 该项目的æ~¹ç›®å½•å?ŽæŒ‰ç\ldots§æ??示使ç''¨ã€‚

\subsection{模æ?¿å†\ldots 容}\label{uxe6uxe6uxe5uxe5uxb9}

æ­¤ Typst 模æ?¿æŒ‰ç\ldots§
\href{https://bylw.hhu.edu.cn/UpLoadFile/83cd5f1169974a0db06d865c7ee11af4.pdf}{《河海大学本ç§`毕业设计(论æ--‡ï¼‰åŸºæœ¬è§„范(修订)》}
制作,制作æ---¶å?‚考了
\href{https://github.com/csimide/SEU-Typst-Template}{东å?---大学制作的
Typst 模�} 。

ç›®å‰?åŒ\ldots å?«ä»¥ä¸‹é¡µé?¢ï¼š

\begin{itemize}
\tightlist
\item
  {[}x{]} 中英æ--‡å°?é?¢
\item
  {[}x{]} éƒ`é‡?声明
\item
  {[}x{]} 中英æ--‡æ`˜è¦?
\item
  {[}x{]} 目录
\item
  {[}x{]} æ­£æ--‡
\item
  {[}x{]} 致谢
\item
  {[}x{]} å?‚考æ--‡çŒ®
\item
  {[}x{]} 附录
\end{itemize}

此论æ--‡æ¨¡æ?¿ä¸?ä»\ldots 适ç''¨äºŽæœ¬ç§`ç''Ÿæ¯•ä¸šè®ºæ--‡/设计,å?Œæ~·é€‚ç''¨äºŽå¹³æ---¶çš„课程报å`Šç­‰è§„范å†\ldots 容。å?¯ä»¥é€šè¿‡è‡ªå®šä¹‰
\texttt{\ form\ }
å­---段更æ''¹è®ºæ--‡ç§?类,有以下3ç§?æ~¼å¼?å?¯ä¾›é€‰æ‹©ï¼š

\begin{itemize}
\tightlist
\item
  \texttt{\ thesis\ } :毕业论æ--‡
\item
  \texttt{\ design\ } :毕业设计
\item
  \texttt{\ report\ } :课程报å`Š
\end{itemize}

å?¯ä»¥é€šè¿‡ä¿®æ''¹ \texttt{\ heading\ }
å­---段修æ''¹é¡µçœ‰å†\ldots 容,修æ''¹ \texttt{\ thesis-name\ }
下的 \texttt{\ CN\ } å­---段修æ''¹å°?é?¢é¡µé?¢å±•ç¤ºçš„æ~‡é¢˜ã€‚

如果å?{}`现模æ?¿çš„é---®é¢˜ï¼Œæ¬¢è¿Žæ??交 issue。

\subsection{致谢}\label{uxe8uxe8}

\begin{itemize}
\item
  东å?---大学论æ--‡æ¨¡æ?¿ï¼š
  \href{https://github.com/csimide/SEU-Typst-Template}{csimide/SEU-Typst-Template}
\item
  åŒ---京大学本ç§`ç''Ÿæ¯•ä¸šè®ºæ--‡æ¨¡æ?¿ï¼š
  \href{https://github.com/sigongzi/pkuthss-typst-undergraduate}{sigongzi/pkuthss-typst-undergraduate}
\end{itemize}

\href{/app?template=shane-hhu-thesis&version=0.2.0}{Create project in
app}

\subsubsection{How to use}\label{how-to-use}

Click the button above to create a new project using this template in
the Typst app.

You can also use the Typst CLI to start a new project on your computer
using this command:

\begin{verbatim}
typst init @preview/shane-hhu-thesis:0.2.0
\end{verbatim}

\includesvg[width=0.16667in,height=0.16667in]{/assets/icons/16-copy.svg}

\subsubsection{About}\label{about}

\begin{description}
\tightlist
\item[Author :]
shane
\item[License:]
MIT
\item[Current version:]
0.2.0
\item[Last updated:]
November 18, 2024
\item[First released:]
November 14, 2024
\item[Archive size:]
297 kB
\href{https://packages.typst.org/preview/shane-hhu-thesis-0.2.0.tar.gz}{\pandocbounded{\includesvg[keepaspectratio]{/assets/icons/16-download.svg}}}
\item[Repository:]
\href{https://github.com/shaneworld/HHU-Thesis-Template}{GitHub}
\item[Categor y :]
\begin{itemize}
\tightlist
\item[]
\item
  \pandocbounded{\includesvg[keepaspectratio]{/assets/icons/16-mortarboard.svg}}
  \href{https://typst.app/universe/search/?category=thesis}{Thesis}
\end{itemize}
\end{description}

\subsubsection{Where to report issues?}\label{where-to-report-issues}

This template is a project of shane . Report issues on
\href{https://github.com/shaneworld/HHU-Thesis-Template}{their
repository} . You can also try to ask for help with this template on the
\href{https://forum.typst.app}{Forum} .

Please report this template to the Typst team using the
\href{https://typst.app/contact}{contact form} if you believe it is a
safety hazard or infringes upon your rights.

\phantomsection\label{versions}
\subsubsection{Version history}\label{version-history}

\begin{longtable}[]{@{}ll@{}}
\toprule\noalign{}
Version & Release Date \\
\midrule\noalign{}
\endhead
\bottomrule\noalign{}
\endlastfoot
0.2.0 & November 18, 2024 \\
\href{https://typst.app/universe/package/shane-hhu-thesis/0.1.0/}{0.1.0}
& November 14, 2024 \\
\end{longtable}

Typst GmbH did not create this template and cannot guarantee correct
functionality of this template or compatibility with any version of the
Typst compiler or app.


\section{Package List LaTeX/touying-pres-ustc.tex}
\title{typst.app/universe/package/touying-pres-ustc}

\phantomsection\label{banner}
\phantomsection\label{template-thumbnail}
\pandocbounded{\includegraphics[keepaspectratio]{https://packages.typst.org/preview/thumbnails/touying-pres-ustc-0.1.0-small.webp}}

\section{touying-pres-ustc}\label{touying-pres-ustc}

{ 0.1.0 }

Touying Slide Theme for USTC

\href{/app?template=touying-pres-ustc&version=0.1.0}{Create project in
app}

\phantomsection\label{readme}
\textbf{\href{http://www.xn--fiqs8srwby7cba020i2hih02b.com/}{www.中国ç§`学技术大学.com}}

为了é™?低æ--°æ‰‹ä¸Šè·¯çš„代ç~?分æž?æˆ?本以å?Šé™?低项目的耦å?ˆæ€§ï¼Œç‰¹å°†é¡¹ç›®æ‹†åˆ†æˆ?多个typæ--‡ä»¶ï¼Œè¯·æŒ‰ä»¥ä¸‹æ­¥éª¤äº†è§£ï¼š

\texttt{\ main.typ\ }
是渲æŸ``çš„å\ldots¥å?£ï¼Œæ--°æ‰‹ä¸Šè·¯ç›´æŽ¥ä»Žè¿™é‡Œå¼€å§‹ã€‚//第一步

\texttt{\ main.typ\ } 中导å\ldots¥äº† \texttt{\ config.typ\ }
,这是é\ldots?ç½®æ--‡ä»¶ï¼Œå°?é?¢çš„æ~‡é¢˜ï¼Œå‰¯æ~‡é¢˜ï¼Œä½œè€\ldots 等信æ?¯åœ¨è¿™é‡Œä¿®æ''¹ï¼Œå\ldots¶ä»--ä½~æ---~须烦æ?¼ã€‚//第二步

\texttt{\ config.typ\ } 中导å\ldots¥äº† \texttt{\ theme.typ\ } å'Œ
\texttt{\ lib.typ\ } ,æ--°æ‰‹ä¸Šè·¯å?¯ä»¥æš‚æ---¶è·³è¿‡ã€‚

\begin{itemize}
\tightlist
\item
  \texttt{\ theme.typ\ } 相å½``于 \texttt{\ CSS\ }
  ,控制ç?€æ¨¡æ?¿é•¿ä»€ä¹ˆæ~·ï¼Œå¦‚æžœä½~对模æ?¿çš„æ~·å¼?ä¸?满æ„?,åŒ\ldots 括å­---ä½``,图片ç´~æ??,跳转功能,æˆ--è€\ldots æ''¹æˆ?å\ldots¶ä»--å­¦æ~¡çš„主题,在这里修æ''¹æˆ?ä½~希望的æ~·å­?。学ä¹~曲线较陡峭,åˆ?å­¦è€\ldots æ---~需å\ldots³å¿ƒï¼Œ
\item
  \texttt{\ lib.typ\ }
  中是第三æ--¹åŒ\ldots å'Œè‡ªå®šä¹‰å‡½æ•°ï¼Œå½``模æ?¿ä¸­çš„åŒ\ldots ä¸?能满足ä½~的需求,æˆ--è€\ldots 想自定义一些常ç''¨å‡½æ•°ï¼Œè¯·æŠŠå®ƒä»¬æ''¾åœ¨è¿™é‡Œä»¥ä¿?æŒ?项目的ç»``构性å'Œä½Žè€¦å?ˆã€‚
\end{itemize}

\texttt{\ content.typ\ }
这是ä½~æ''¾å®žé™\ldots å†\ldots 容的地æ--¹ï¼Œä¸€çº§æ~‡é¢˜æ˜¯å¤§çº²ï¼ŒäºŒçº§æ~‡é¢˜æ˜¯å½``页的æ~‡é¢˜ï¼Œä¹‹å?Žçš„n级æ~‡é¢˜éƒ½åœ¨é¡µå†\ldots 作为å†\ldots 容显示,æ---~特殊地ä½?。//第三步

\begin{itemize}
\item
  \texttt{\ assets\ } : 模æ?¿èµ„æº?æ--‡ä»¶

  \begin{itemize}
  \tightlist
  \item
    \texttt{\ img\ } : 模æ?¿å›¾ç‰‡æ--‡ä»¶
  \end{itemize}
\item
  \texttt{\ template\ } : å?¯å¤?ç''¨ç»„件

  \begin{itemize}
  \tightlist
  \item
    {[}{]}
  \end{itemize}
\item
  \texttt{\ config.typ\ } :
  é\ldots?ç½®æ--‡ä»¶ï¼ŒåŒ\ldots 括å°?é?¢çš„ä¿¡æ?¯åœ¨è¿™é‡Œã€‚
\item
  \texttt{\ content.typ\ } : å?ªéœ€åœ¨æ­¤å¤„æ·»åŠ~å†\ldots 容
\item
  \texttt{\ lib.typ\ } :
  åº``æ--‡ä»¶ï¼Œå¯¼å\ldots¥ç¬¬ä¸‰æ--¹åº``请在这导å\ldots¥
\item
  \texttt{\ main.typ\ } :
  ç¼--è¯`å\ldots¥å?£ï¼Œå¯¼å\ldots¥ç¬¬ä¸‰æ--¹åº``æ---¶æœ‰å?¯èƒ½éœ€è¦?把
  \texttt{\ \#show\ } æ''¾åœ¨æ­¤å¤„
\item
  \texttt{\ theme.typ\ } : 主题æ--‡ä»¶ï¼Œå?¯è‡ªè¡Œä¿®æ''¹æ~·å¼?
\end{itemize}

\begin{enumerate}
\tightlist
\item
  vscode下载æ?'件 \texttt{\ Typst\ LSP\ } ,
  \texttt{\ Tinymist\ Typst\ } , \texttt{\ Typst\ Sync\ } 。
\item
  æ‰``开本项目,在 \texttt{\ main.typ\ } 中点击
  \texttt{\ preview\ } 。若在 \texttt{\ content.typ\ } 中点击
  \texttt{\ preview\ } ,则预览类似 \texttt{\ markdown\ }
  çš„æ--‡æ¡£æŽ'版。
\end{enumerate}

\begin{enumerate}
\tightlist
\item
  在使ç''¨æ---¶ï¼Œ \texttt{\ config.typ\ }
  åº''作为顶层æ--‡ä»¶è¢«å\ldots¶ä»--æ--‡ä»¶å¯¼å\ldots¥ï¼Œé?¿å\ldots?交å?‰å¯¼å\ldots¥ã€‚
\end{enumerate}

\begin{enumerate}
\tightlist
\item
\end{enumerate}

有空å†?写代ç~?注释

\href{/app?template=touying-pres-ustc&version=0.1.0}{Create project in
app}

\subsubsection{How to use}\label{how-to-use}

Click the button above to create a new project using this template in
the Typst app.

You can also use the Typst CLI to start a new project on your computer
using this command:

\begin{verbatim}
typst init @preview/touying-pres-ustc:0.1.0
\end{verbatim}

\includesvg[width=0.16667in,height=0.16667in]{/assets/icons/16-copy.svg}

\subsubsection{About}\label{about}

\begin{description}
\tightlist
\item[Author :]
\href{https://github.com/Quaternijkon}{Quaternijkon}
\item[License:]
MIT
\item[Current version:]
0.1.0
\item[Last updated:]
November 26, 2024
\item[First released:]
November 26, 2024
\item[Archive size:]
240 kB
\href{https://packages.typst.org/preview/touying-pres-ustc-0.1.0.tar.gz}{\pandocbounded{\includesvg[keepaspectratio]{/assets/icons/16-download.svg}}}
\item[Repository:]
\href{https://github.com/Quaternijkon/Typst_USTC_CS}{GitHub}
\item[Categor y :]
\begin{itemize}
\tightlist
\item[]
\item
  \pandocbounded{\includesvg[keepaspectratio]{/assets/icons/16-presentation.svg}}
  \href{https://typst.app/universe/search/?category=presentation}{Presentation}
\end{itemize}
\end{description}

\subsubsection{Where to report issues?}\label{where-to-report-issues}

This template is a project of Quaternijkon . Report issues on
\href{https://github.com/Quaternijkon/Typst_USTC_CS}{their repository} .
You can also try to ask for help with this template on the
\href{https://forum.typst.app}{Forum} .

Please report this template to the Typst team using the
\href{https://typst.app/contact}{contact form} if you believe it is a
safety hazard or infringes upon your rights.

\phantomsection\label{versions}
\subsubsection{Version history}\label{version-history}

\begin{longtable}[]{@{}ll@{}}
\toprule\noalign{}
Version & Release Date \\
\midrule\noalign{}
\endhead
\bottomrule\noalign{}
\endlastfoot
0.1.0 & November 26, 2024 \\
\end{longtable}

Typst GmbH did not create this template and cannot guarantee correct
functionality of this template or compatibility with any version of the
Typst compiler or app.


\section{Package List LaTeX/versatile-apa.tex}
\title{typst.app/universe/package/versatile-apa}

\phantomsection\label{banner}
\phantomsection\label{template-thumbnail}
\pandocbounded{\includegraphics[keepaspectratio]{https://packages.typst.org/preview/thumbnails/versatile-apa-7.0.0-small.webp}}

\section{versatile-apa}\label{versatile-apa}

{ 7.0.0 }

Comprehensive APA 7th Edition Style Template for Typst, suitable for
both student and professional papers.

\href{/app?template=versatile-apa&version=7.0.0}{Create project in app}

\phantomsection\label{readme}
APA 7th Edition template for Typst. This template is based on the
official APA 7th Edition style guide and includes all the necessary
elements for a research paper. It is designed to be versatile and can be
used for any type of research paper, including essays, theses, and
dissertations.

\subsection{Usage}\label{usage}

To use this template, you can use the CLI tool:

\begin{Shaded}
\begin{Highlighting}[]
\ExtensionTok{typst}\NormalTok{ init @preview/versatile{-}apa}
\end{Highlighting}
\end{Shaded}

\subsubsection{Features}\label{features}

The template allows you to easily create academic students for both
student and professional versions of APA 7th Edition:

\begin{itemize}
\tightlist
\item
  Title page
\item
  Abstract
\item
  Localization
\item
  Headings
\item
  Raw/computer code
\item
  Math equations
\item
  Appendices
\item
  References
\item
  Quotation blocks (40 words or more)
\item
  Figures and tables
\item
  Lists
\item
  Footnotes
\item
  Authoring:

  \begin{itemize}
  \tightlist
  \item
    Automatic footnotes for author/affiliation
  \item
    Author notes
  \item
    ORCID
  \end{itemize}
\end{itemize}

\subsection{Planned Features}\label{planned-features}

As of now, the template is in its initial stages and will be updated
with more features in the future. Some of the planned features include:

\begin{itemize}
\tightlist
\item
  \textbf{LaTeX \texttt{\ apa7\ } class full support} : This template is
  inspired by the \texttt{\ apa7\ } class in LaTeX, and it’s planned
  to also include support for all 4 formats of APA (student,
  professional, journal, and manuscript).
\item
  \textbf{Figures notes} : Improved support for all 3 types of APA notes
  (general, specific, probability).
\end{itemize}

\subsection{License}\label{license}

Package licensed under the MIT License. See the repository for more
information.

\href{/app?template=versatile-apa&version=7.0.0}{Create project in app}

\subsubsection{How to use}\label{how-to-use}

Click the button above to create a new project using this template in
the Typst app.

You can also use the Typst CLI to start a new project on your computer
using this command:

\begin{verbatim}
typst init @preview/versatile-apa:7.0.0
\end{verbatim}

\includesvg[width=0.16667in,height=0.16667in]{/assets/icons/16-copy.svg}

\subsubsection{About}\label{about}

\begin{description}
\tightlist
\item[Author :]
\href{https://github.com/jassielof}{Jassiel Ovando}
\item[License:]
MIT-0
\item[Current version:]
7.0.0
\item[Last updated:]
November 4, 2024
\item[First released:]
November 4, 2024
\item[Minimum Typst version:]
0.12.0
\item[Archive size:]
812 kB
\href{https://packages.typst.org/preview/versatile-apa-7.0.0.tar.gz}{\pandocbounded{\includesvg[keepaspectratio]{/assets/icons/16-download.svg}}}
\item[Repository:]
\href{https://github.com/jassielof/typst-templates}{GitHub}
\item[Discipline :]
\begin{itemize}
\tightlist
\item[]
\item
  \href{https://typst.app/universe/search/?discipline=psychology}{Psychology}
\end{itemize}
\item[Categor ies :]
\begin{itemize}
\tightlist
\item[]
\item
  \pandocbounded{\includesvg[keepaspectratio]{/assets/icons/16-atom.svg}}
  \href{https://typst.app/universe/search/?category=paper}{Paper}
\item
  \pandocbounded{\includesvg[keepaspectratio]{/assets/icons/16-speak.svg}}
  \href{https://typst.app/universe/search/?category=report}{Report}
\end{itemize}
\end{description}

\subsubsection{Where to report issues?}\label{where-to-report-issues}

This template is a project of Jassiel Ovando . Report issues on
\href{https://github.com/jassielof/typst-templates}{their repository} .
You can also try to ask for help with this template on the
\href{https://forum.typst.app}{Forum} .

Please report this template to the Typst team using the
\href{https://typst.app/contact}{contact form} if you believe it is a
safety hazard or infringes upon your rights.

\phantomsection\label{versions}
\subsubsection{Version history}\label{version-history}

\begin{longtable}[]{@{}ll@{}}
\toprule\noalign{}
Version & Release Date \\
\midrule\noalign{}
\endhead
\bottomrule\noalign{}
\endlastfoot
7.0.0 & November 4, 2024 \\
\end{longtable}

Typst GmbH did not create this template and cannot guarantee correct
functionality of this template or compatibility with any version of the
Typst compiler or app.


\section{Package List LaTeX/stack-pointer.tex}
\title{typst.app/universe/package/stack-pointer}

\phantomsection\label{banner}
\section{stack-pointer}\label{stack-pointer}

{ 0.1.0 }

A library for visualizing the execution of (imperative) computer
programs

{ } Featured Package

\phantomsection\label{readme}
Stack Pointer is a library for visualizing the execution of (imperative)
computer programs, particularly in terms of effects on the call stack:
stack frames and local variables therein.

Stack Pointer lets you represent an example program (e.g. a C or Java
program) using typst code with minimal hassle, and get the execution
state of that program at different points in time. For example, the
following C program

\begin{Shaded}
\begin{Highlighting}[]
\DataTypeTok{int}\NormalTok{ main}\OperatorTok{()} \OperatorTok{\{}
  \DataTypeTok{int}\NormalTok{ x }\OperatorTok{=}\NormalTok{ foo}\OperatorTok{();}
  \ControlFlowTok{return} \DecValTok{0}\OperatorTok{;}
\OperatorTok{\}}

\DataTypeTok{int}\NormalTok{ foo}\OperatorTok{()} \OperatorTok{\{}
  \ControlFlowTok{return} \DecValTok{0}\OperatorTok{;}
\OperatorTok{\}}
\end{Highlighting}
\end{Shaded}

would be represented by the following Typst code (see the
\href{https://github.com/typst/packages/raw/main/packages/preview/stack-pointer/0.1.0/docs/manual.pdf}{manual}
for a detailled explanation):

\begin{Shaded}
\begin{Highlighting}[]
\NormalTok{\#let steps = execute(\{}
\NormalTok{  let foo() = func("foo", 6, l =\textgreater{} \{}
\NormalTok{    l(0)}
\NormalTok{    l(1); retval(0)}
\NormalTok{  \})}
\NormalTok{  let main() = func("main", 1, l =\textgreater{} \{}
\NormalTok{    l(0)}
\NormalTok{    l(1)}
\NormalTok{    let (x, ..rest) = foo(); rest}
\NormalTok{    l(1, push("x", x))}
\NormalTok{    l(2)}
\NormalTok{  \})}
\NormalTok{  main(); l(none)}
\NormalTok{\})}
\end{Highlighting}
\end{Shaded}

The \texttt{\ steps\ } variable now contains an array, where each
element corresponds to one of the mentioned lines of code.

Take a look at
\href{https://github.com/typst/packages/raw/main/packages/preview/stack-pointer/0.1.0/gallery/sum.pdf}{this
complete example} of using Stack Pointer together with
\href{https://polylux.dev/book/}{Polylux} .

\subsubsection{How to add}\label{how-to-add}

Copy this into your project and use the import as
\texttt{\ stack-pointer\ }

\begin{verbatim}
#import "@preview/stack-pointer:0.1.0"
\end{verbatim}

\includesvg[width=0.16667in,height=0.16667in]{/assets/icons/16-copy.svg}

Check the docs for
\href{https://typst.app/docs/reference/scripting/\#packages}{more
information on how to import packages} .

\subsubsection{About}\label{about}

\begin{description}
\tightlist
\item[Author :]
\href{https://github.com/SillyFreak/}{Clemens Koza}
\item[License:]
MIT
\item[Current version:]
0.1.0
\item[Last updated:]
July 15, 2024
\item[First released:]
July 15, 2024
\item[Archive size:]
4.29 kB
\href{https://packages.typst.org/preview/stack-pointer-0.1.0.tar.gz}{\pandocbounded{\includesvg[keepaspectratio]{/assets/icons/16-download.svg}}}
\item[Repository:]
\href{https://github.com/SillyFreak/typst-stack-pointer}{GitHub}
\item[Discipline :]
\begin{itemize}
\tightlist
\item[]
\item
  \href{https://typst.app/universe/search/?discipline=computer-science}{Computer
  Science}
\end{itemize}
\item[Categor ies :]
\begin{itemize}
\tightlist
\item[]
\item
  \pandocbounded{\includesvg[keepaspectratio]{/assets/icons/16-code.svg}}
  \href{https://typst.app/universe/search/?category=scripting}{Scripting}
\item
  \pandocbounded{\includesvg[keepaspectratio]{/assets/icons/16-presentation.svg}}
  \href{https://typst.app/universe/search/?category=presentation}{Presentation}
\end{itemize}
\end{description}

\subsubsection{Where to report issues?}\label{where-to-report-issues}

This package is a project of Clemens Koza . Report issues on
\href{https://github.com/SillyFreak/typst-stack-pointer}{their
repository} . You can also try to ask for help with this package on the
\href{https://forum.typst.app}{Forum} .

Please report this package to the Typst team using the
\href{https://typst.app/contact}{contact form} if you believe it is a
safety hazard or infringes upon your rights.

\phantomsection\label{versions}
\subsubsection{Version history}\label{version-history}

\begin{longtable}[]{@{}ll@{}}
\toprule\noalign{}
Version & Release Date \\
\midrule\noalign{}
\endhead
\bottomrule\noalign{}
\endlastfoot
0.1.0 & July 15, 2024 \\
\end{longtable}

Typst GmbH did not create this package and cannot guarantee correct
functionality of this package or compatibility with any version of the
Typst compiler or app.


\section{Package List LaTeX/tgm-hit-protocol.tex}
\title{typst.app/universe/package/tgm-hit-protocol}

\phantomsection\label{banner}
\phantomsection\label{template-thumbnail}
\pandocbounded{\includegraphics[keepaspectratio]{https://packages.typst.org/preview/thumbnails/tgm-hit-protocol-0.1.0-small.webp}}

\section{tgm-hit-protocol}\label{tgm-hit-protocol}

{ 0.1.0 }

Protocol template for students of the HIT department at TGM Wien

\href{/app?template=tgm-hit-protocol&version=0.1.0}{Create project in
app}

\phantomsection\label{readme}
This is a port of the
\href{https://github.com/TGM-HIT/latex-protocol/}{LaTeX protocol
template} available for students of the information technology
department at the TGM technical secondary school in Vienna.

\subsection{Getting Started}\label{getting-started}

Using the Typst web app, you can create a project by e.g. using this
link: \url{https://typst.app/?template=tgm-hit-protocol&version=latest}
.

To work locally, use the following command:

\begin{Shaded}
\begin{Highlighting}[]
\ExtensionTok{typst}\NormalTok{ init @preview/tgm{-}hit{-}protocol}
\end{Highlighting}
\end{Shaded}

\subsection{Usage}\label{usage}

The template (
\href{https://github.com/typst/packages/raw/main/packages/preview/tgm-hit-protocol/0.1.0/main.pdf}{rendered
PDF} ) contains thesis writing advice (in German) as example content. If
you are looking for the details of this template package’s function,
take a look at the
\href{https://github.com/typst/packages/raw/main/packages/preview/tgm-hit-protocol/0.1.0/docs/manual.pdf}{manual}
.

\href{/app?template=tgm-hit-protocol&version=0.1.0}{Create project in
app}

\subsubsection{How to use}\label{how-to-use}

Click the button above to create a new project using this template in
the Typst app.

You can also use the Typst CLI to start a new project on your computer
using this command:

\begin{verbatim}
typst init @preview/tgm-hit-protocol:0.1.0
\end{verbatim}

\includesvg[width=0.16667in,height=0.16667in]{/assets/icons/16-copy.svg}

\subsubsection{About}\label{about}

\begin{description}
\tightlist
\item[Author s :]
\href{https://github.com/k1W1M4ng0}{Simon Gao} \&
\href{https://github.com/SillyFreak/}{Clemens Koza}
\item[License:]
MIT
\item[Current version:]
0.1.0
\item[Last updated:]
October 10, 2024
\item[First released:]
October 10, 2024
\item[Minimum Typst version:]
0.11.0
\item[Archive size:]
80.4 kB
\href{https://packages.typst.org/preview/tgm-hit-protocol-0.1.0.tar.gz}{\pandocbounded{\includesvg[keepaspectratio]{/assets/icons/16-download.svg}}}
\item[Repository:]
\href{https://github.com/TGM-HIT/typst-protocol}{GitHub}
\item[Discipline :]
\begin{itemize}
\tightlist
\item[]
\item
  \href{https://typst.app/universe/search/?discipline=computer-science}{Computer
  Science}
\end{itemize}
\item[Categor y :]
\begin{itemize}
\tightlist
\item[]
\item
  \pandocbounded{\includesvg[keepaspectratio]{/assets/icons/16-speak.svg}}
  \href{https://typst.app/universe/search/?category=report}{Report}
\end{itemize}
\end{description}

\subsubsection{Where to report issues?}\label{where-to-report-issues}

This template is a project of Simon Gao and Clemens Koza . Report issues
on \href{https://github.com/TGM-HIT/typst-protocol}{their repository} .
You can also try to ask for help with this template on the
\href{https://forum.typst.app}{Forum} .

Please report this template to the Typst team using the
\href{https://typst.app/contact}{contact form} if you believe it is a
safety hazard or infringes upon your rights.

\phantomsection\label{versions}
\subsubsection{Version history}\label{version-history}

\begin{longtable}[]{@{}ll@{}}
\toprule\noalign{}
Version & Release Date \\
\midrule\noalign{}
\endhead
\bottomrule\noalign{}
\endlastfoot
0.1.0 & October 10, 2024 \\
\end{longtable}

Typst GmbH did not create this template and cannot guarantee correct
functionality of this template or compatibility with any version of the
Typst compiler or app.


\section{Package List LaTeX/unilab.tex}
\title{typst.app/universe/package/unilab}

\phantomsection\label{banner}
\phantomsection\label{template-thumbnail}
\pandocbounded{\includegraphics[keepaspectratio]{https://packages.typst.org/preview/thumbnails/unilab-0.0.2-small.webp}}

\section{unilab}\label{unilab}

{ 0.0.2 }

Lab report

\href{/app?template=unilab&version=0.0.2}{Create project in app}

\phantomsection\label{readme}
Typst Lab Report Template

\subsection{Local debugging}\label{local-debugging}

clone this repo into the
\href{https://github.com/typst/packages?tab=readme-ov-file\#local-packages}{local
package directory} , notice that the version should be specified (e.g.
\texttt{\ .../unilab/0.0.1/\ } )

\subsection{TODO}\label{todo}

\begin{itemize}
\tightlist
\item
  {[} {]} en font support
\item
  {[} {]} support school logo
\end{itemize}

\href{/app?template=unilab&version=0.0.2}{Create project in app}

\subsubsection{How to use}\label{how-to-use}

Click the button above to create a new project using this template in
the Typst app.

You can also use the Typst CLI to start a new project on your computer
using this command:

\begin{verbatim}
typst init @preview/unilab:0.0.2
\end{verbatim}

\includesvg[width=0.16667in,height=0.16667in]{/assets/icons/16-copy.svg}

\subsubsection{About}\label{about}

\begin{description}
\tightlist
\item[Author :]
\href{https://github.com/sjfhsjfh}{sjfhsjfh}
\item[License:]
MIT
\item[Current version:]
0.0.2
\item[Last updated:]
April 6, 2024
\item[First released:]
April 6, 2024
\item[Minimum Typst version:]
0.11.0
\item[Archive size:]
18.9 kB
\href{https://packages.typst.org/preview/unilab-0.0.2.tar.gz}{\pandocbounded{\includesvg[keepaspectratio]{/assets/icons/16-download.svg}}}
\item[Repository:]
\href{https://github.com/sjfhsjfh/unilab}{GitHub}
\item[Categor y :]
\begin{itemize}
\tightlist
\item[]
\item
  \pandocbounded{\includesvg[keepaspectratio]{/assets/icons/16-speak.svg}}
  \href{https://typst.app/universe/search/?category=report}{Report}
\end{itemize}
\end{description}

\subsubsection{Where to report issues?}\label{where-to-report-issues}

This template is a project of sjfhsjfh . Report issues on
\href{https://github.com/sjfhsjfh/unilab}{their repository} . You can
also try to ask for help with this template on the
\href{https://forum.typst.app}{Forum} .

Please report this template to the Typst team using the
\href{https://typst.app/contact}{contact form} if you believe it is a
safety hazard or infringes upon your rights.

\phantomsection\label{versions}
\subsubsection{Version history}\label{version-history}

\begin{longtable}[]{@{}ll@{}}
\toprule\noalign{}
Version & Release Date \\
\midrule\noalign{}
\endhead
\bottomrule\noalign{}
\endlastfoot
0.0.2 & April 6, 2024 \\
\end{longtable}

Typst GmbH did not create this template and cannot guarantee correct
functionality of this template or compatibility with any version of the
Typst compiler or app.


\section{Package List LaTeX/prequery.tex}
\title{typst.app/universe/package/prequery}

\phantomsection\label{banner}
\section{prequery}\label{prequery}

{ 0.1.0 }

library for extracting metadata for preprocessing from a typst document

\phantomsection\label{readme}
This package helps extracting metadata for preprocessing from a typst
document, for example image URLs for download from the web. Typst
compilations are sandboxed: it is not possible for Typst packages, or
even just a Typst document itself, to access the “ouside world�.
This sandboxing of Typst has good reasons. Yet, it is often convenient
to trade a bit of security for convenience by weakening it. Prequery
helps with that by providing some simple scaffolding for supporting
preprocessing of documents.

Here’s an example for referencing images from the internet:

\begin{Shaded}
\begin{Highlighting}[]
\NormalTok{\#import "@preview/prequery:0.1.0"}

\NormalTok{// toggle this comment or pass \textasciigrave{}{-}{-}input prequery{-}fallback=true\textasciigrave{} to enable fallback}
\NormalTok{// \#prequery.fallback.update(true)}

\NormalTok{\#prequery.image(}
\NormalTok{  "https://en.wikipedia.org/static/images/icons/wikipedia.png",}
\NormalTok{  "assets/wikipedia.png")}
\end{Highlighting}
\end{Shaded}

Using \texttt{\ typst\ query\ } , the image URL(s) are extracted from
the document:

\begin{Shaded}
\begin{Highlighting}[]
\ExtensionTok{typst}\NormalTok{ query }\AttributeTok{{-}{-}input}\NormalTok{ prequery{-}fallback=true }\AttributeTok{{-}{-}field}\NormalTok{ value }\DataTypeTok{\textbackslash{}}
\NormalTok{    main.typ }\StringTok{\textquotesingle{}\textquotesingle{}}
\end{Highlighting}
\end{Shaded}

This will output the following piece of JSON:

\begin{Shaded}
\begin{Highlighting}[]
\OtherTok{[}\FunctionTok{\{}\DataTypeTok{"url"}\FunctionTok{:} \StringTok{"https://en.wikipedia.org/static/images/icons/wikipedia.png"}\FunctionTok{,} \DataTypeTok{"path"}\FunctionTok{:} \StringTok{"assets/wikipedia.png"}\FunctionTok{\}}\OtherTok{]}
\end{Highlighting}
\end{Shaded}

Which can then be used to download all images to the expected locations.

See the
\href{https://github.com/typst/packages/raw/main/packages/preview/prequery/0.1.0/docs/manual.pdf}{manual}
for details.

\subsubsection{How to add}\label{how-to-add}

Copy this into your project and use the import as \texttt{\ prequery\ }

\begin{verbatim}
#import "@preview/prequery:0.1.0"
\end{verbatim}

\includesvg[width=0.16667in,height=0.16667in]{/assets/icons/16-copy.svg}

Check the docs for
\href{https://typst.app/docs/reference/scripting/\#packages}{more
information on how to import packages} .

\subsubsection{About}\label{about}

\begin{description}
\tightlist
\item[Author :]
\href{https://github.com/SillyFreak/}{Clemens Koza}
\item[License:]
MIT
\item[Current version:]
0.1.0
\item[Last updated:]
July 15, 2024
\item[First released:]
July 15, 2024
\item[Archive size:]
3.29 kB
\href{https://packages.typst.org/preview/prequery-0.1.0.tar.gz}{\pandocbounded{\includesvg[keepaspectratio]{/assets/icons/16-download.svg}}}
\item[Repository:]
\href{https://github.com/SillyFreak/typst-prequery}{GitHub}
\item[Categor ies :]
\begin{itemize}
\tightlist
\item[]
\item
  \pandocbounded{\includesvg[keepaspectratio]{/assets/icons/16-code.svg}}
  \href{https://typst.app/universe/search/?category=scripting}{Scripting}
\item
  \pandocbounded{\includesvg[keepaspectratio]{/assets/icons/16-hammer.svg}}
  \href{https://typst.app/universe/search/?category=utility}{Utility}
\end{itemize}
\end{description}

\subsubsection{Where to report issues?}\label{where-to-report-issues}

This package is a project of Clemens Koza . Report issues on
\href{https://github.com/SillyFreak/typst-prequery}{their repository} .
You can also try to ask for help with this package on the
\href{https://forum.typst.app}{Forum} .

Please report this package to the Typst team using the
\href{https://typst.app/contact}{contact form} if you believe it is a
safety hazard or infringes upon your rights.

\phantomsection\label{versions}
\subsubsection{Version history}\label{version-history}

\begin{longtable}[]{@{}ll@{}}
\toprule\noalign{}
Version & Release Date \\
\midrule\noalign{}
\endhead
\bottomrule\noalign{}
\endlastfoot
0.1.0 & July 15, 2024 \\
\end{longtable}

Typst GmbH did not create this package and cannot guarantee correct
functionality of this package or compatibility with any version of the
Typst compiler or app.


\section{Package List LaTeX/bloated-neurips.tex}
\title{typst.app/universe/package/bloated-neurips}

\phantomsection\label{banner}
\phantomsection\label{template-thumbnail}
\pandocbounded{\includegraphics[keepaspectratio]{https://packages.typst.org/preview/thumbnails/bloated-neurips-0.5.1-small.webp}}

\section{bloated-neurips}\label{bloated-neurips}

{ 0.5.1 }

NeurIPS-style paper template to publish at the Conference and Workshop
on Neural Information Processing Systems

\href{/app?template=bloated-neurips&version=0.5.1}{Create project in
app}

\phantomsection\label{readme}
\subsection{Usage}\label{usage}

You can use this template in the Typst web app by clicking \emph{Start
from template} on the dashboard and searching for
\texttt{\ bloated-neurips\ } .

Alternatively, you can use the CLI to kick this project off using the
command

\begin{Shaded}
\begin{Highlighting}[]
\NormalTok{typst init @preview/bloated{-}neurips}
\end{Highlighting}
\end{Shaded}

Typst will create a new directory with all the files needed to get you
started.

\subsection{Configuration}\label{configuration}

This template exports the \texttt{\ neurips2023\ } and
\texttt{\ neurips2024\ } function with the following named arguments.

\begin{itemize}
\tightlist
\item
  \texttt{\ title\ } : The paper’s title as content.
\item
  \texttt{\ authors\ } : An array of author dictionaries. Each of the
  author dictionaries must have a name key and can have the keys
  department, organization, location, and email.
\item
  \texttt{\ abstract\ } : The content of a brief summary of the paper or
  none. Appears at the top under the title.
\item
  \texttt{\ bibliography\ } : The result of a call to the bibliography
  function or none. The function also accepts a single, positional
  argument for the body of the paper.
\item
  \texttt{\ appendix\ } : A content which is placed after bibliography.
\item
  \texttt{\ accepted\ } : If this is set to \texttt{\ false\ } then
  anonymized ready for submission document is produced;
  \texttt{\ accepted:\ true\ } produces camera-redy version. If the
  argument is set to \texttt{\ none\ } then preprint version is produced
  (can be uploaded to arXiv).
\end{itemize}

The template will initialize your package with a sample call to the
\texttt{\ neurips2024\ } function in a show rule. If you want to change
an existing project to use this template, you can add a show rule at the
top of your file as follows.

\begin{Shaded}
\begin{Highlighting}[]
\NormalTok{\#import "@preview/bloated{-}neurips:0.5.1": neurips2024}

\NormalTok{\#show: neurips2024.with(}
\NormalTok{  title: [Formatting Instructions For NeurIPS 2024],}
\NormalTok{  authors: (authors, affls),}
\NormalTok{  keywords: ("Machine Learning", "NeurIPS"),}
\NormalTok{  abstract: [}
\NormalTok{    The abstract paragraph should be indented ½ inch (3 picas) on both the}
\NormalTok{    left{-} and right{-}hand margins. Use 10 point type, with a vertical spacing}
\NormalTok{    (leading) of 11 points. The word *Abstract* must be centered, bold, and in}
\NormalTok{    point size 12. Two line spaces precede the abstract. The abstract must be}
\NormalTok{    limited to one paragraph.}
\NormalTok{  ],}
\NormalTok{  bibliography: bibliography("main.bib"),}
\NormalTok{  appendix: [}
\NormalTok{    \#include "appendix.typ"}
\NormalTok{    \#include "checklist.typ"}
\NormalTok{  ],}
\NormalTok{  accepted: false,}
\NormalTok{)}

\NormalTok{\#lorem(42)}
\end{Highlighting}
\end{Shaded}

With template of version v0.5.1 or newer, one can override some parts.
Specifically, \texttt{\ get-notice\ } entry of \texttt{\ aux\ }
directory parameter of show rule allows to adjust the NeurIPS 2024
template to Science4DL workshop as follows.

\begin{Shaded}
\begin{Highlighting}[]
\NormalTok{\#import "@preview/bloated{-}neurips:0.5.1": neurips}

\NormalTok{\#let get{-}notice(accepted) = if accepted == none \{}
\NormalTok{  return [Preprint. Under review.]}
\NormalTok{\} else if accepted \{}
\NormalTok{  return [}
\NormalTok{    Workshop on Scientific Methods for Understanding Deep Learning, NeurIPS}
\NormalTok{    2024.}
\NormalTok{  ]}
\NormalTok{\} else \{}
\NormalTok{  return [}
\NormalTok{    Submitted to Workshop on Scientific Methods for Understanding Deep}
\NormalTok{    Learning, NeurIPS 2024.}
\NormalTok{  ]}
\NormalTok{\}}

\NormalTok{\#let science4dl2024(}
\NormalTok{  title: [], authors: (), keywords: (), date: auto, abstract: none,}
\NormalTok{  bibliography: none, appendix: none, accepted: false, body,}
\NormalTok{) = \{}
\NormalTok{  show: neurips.with(}
\NormalTok{    title: title,}
\NormalTok{    authors: authors,}
\NormalTok{    keywords: keywords,}
\NormalTok{    date: date,}
\NormalTok{    abstract: abstract,}
\NormalTok{    accepted: false,}
\NormalTok{    aux: (get{-}notice: get{-}notice),}
\NormalTok{  )}
\NormalTok{  body}
\NormalTok{\}}
\end{Highlighting}
\end{Shaded}

\subsection{Issues}\label{issues}

\begin{itemize}
\item
  The biggest and the most important issues is related to line ruler. We
  are not aware of universal method for numbering lines in the main body
  of a paper. Fortunately, line numbering support has been implemented
  at \href{https://github.com/typst/typst/pull/4516}{typst/typst\#4516}
  . Let’s wait for the next major release v0.12.0!
\item
  There is an issue in Typst with spacing between figures and between
  figure with floating placement. The issue is that there is no way to
  specify gap between subsequent figures. In order to have behaviour
  similar to original LaTeX template, one should consider direct spacing
  adjacemnt with \texttt{\ v(-1em)\ } as follows.

\begin{Shaded}
\begin{Highlighting}[]
\NormalTok{\#figure(}
\NormalTok{  rect(width: 4.25cm, height: 4.25cm, stroke: 0.4pt),}
\NormalTok{  caption: [Sample figure caption.\#v({-}1em)],}
\NormalTok{  placement: top,}
\NormalTok{)}
\NormalTok{\#figure(}
\NormalTok{  rect(width: 4.25cm, height: 4.25cm, stroke: 0.4pt),}
\NormalTok{  caption: [Sample figure caption.],}
\NormalTok{  placement: top,}
\NormalTok{)}
\end{Highlighting}
\end{Shaded}
\item
  Another issue is related to Typst’s inablity to produce colored
  annotation. In order to mitigte the issue, we add a script which
  modifies annotations and make them colored.

\begin{Shaded}
\begin{Highlighting}[]
\NormalTok{../colorize{-}annotations.py \textbackslash{}}
\NormalTok{    example{-}paper.typst.pdf example{-}paper{-}colored.typst.pdf}
\end{Highlighting}
\end{Shaded}

  See
  \href{https://github.com/daskol/typst-templates/\#colored-annotations}{README.md}
  for details.
\item
  NeurIPS 2023/2024 instructions discuss bibliography in vague terms.
  Namely, there is not specific requirements. Thus we stick to
  \texttt{\ ieee\ } bibliography style since we found it in several
  accepted papers and it is similar to that in the example paper.
\item
  It is unclear how to render notice in the bottom of the title page in
  case of final ( \texttt{\ accepted:\ true\ } ) or preprint (
  \texttt{\ accepted:\ none\ } ) submission.
\end{itemize}

\href{/app?template=bloated-neurips&version=0.5.1}{Create project in
app}

\subsubsection{How to use}\label{how-to-use}

Click the button above to create a new project using this template in
the Typst app.

You can also use the Typst CLI to start a new project on your computer
using this command:

\begin{verbatim}
typst init @preview/bloated-neurips:0.5.1
\end{verbatim}

\includesvg[width=0.16667in,height=0.16667in]{/assets/icons/16-copy.svg}

\subsubsection{About}\label{about}

\begin{description}
\tightlist
\item[Author :]
\href{mailto:daniel.bershatsky2@skoltech.ru}{Daniel Bershatsky}
\item[License:]
MIT
\item[Current version:]
0.5.1
\item[Last updated:]
October 8, 2024
\item[First released:]
March 19, 2024
\item[Minimum Typst version:]
0.11.1
\item[Archive size:]
21.2 kB
\href{https://packages.typst.org/preview/bloated-neurips-0.5.1.tar.gz}{\pandocbounded{\includesvg[keepaspectratio]{/assets/icons/16-download.svg}}}
\item[Repository:]
\href{https://github.com/daskol/typst-templates}{GitHub}
\item[Discipline s :]
\begin{itemize}
\tightlist
\item[]
\item
  \href{https://typst.app/universe/search/?discipline=computer-science}{Computer
  Science}
\item
  \href{https://typst.app/universe/search/?discipline=mathematics}{Mathematics}
\end{itemize}
\item[Categor y :]
\begin{itemize}
\tightlist
\item[]
\item
  \pandocbounded{\includesvg[keepaspectratio]{/assets/icons/16-atom.svg}}
  \href{https://typst.app/universe/search/?category=paper}{Paper}
\end{itemize}
\end{description}

\subsubsection{Where to report issues?}\label{where-to-report-issues}

This template is a project of Daniel Bershatsky . Report issues on
\href{https://github.com/daskol/typst-templates}{their repository} . You
can also try to ask for help with this template on the
\href{https://forum.typst.app}{Forum} .

Please report this template to the Typst team using the
\href{https://typst.app/contact}{contact form} if you believe it is a
safety hazard or infringes upon your rights.

\phantomsection\label{versions}
\subsubsection{Version history}\label{version-history}

\begin{longtable}[]{@{}ll@{}}
\toprule\noalign{}
Version & Release Date \\
\midrule\noalign{}
\endhead
\bottomrule\noalign{}
\endlastfoot
0.5.1 & October 8, 2024 \\
\href{https://typst.app/universe/package/bloated-neurips/0.5.0/}{0.5.0}
& September 22, 2024 \\
\href{https://typst.app/universe/package/bloated-neurips/0.2.1/}{0.2.1}
& March 19, 2024 \\
\end{longtable}

Typst GmbH did not create this template and cannot guarantee correct
functionality of this template or compatibility with any version of the
Typst compiler or app.


\section{Package List LaTeX/tablem.tex}
\title{typst.app/universe/package/tablem}

\phantomsection\label{banner}
\section{tablem}\label{tablem}

{ 0.1.0 }

Write markdown-like tables easily.

\phantomsection\label{readme}
Write markdown-like tables easily.

\subsection{Example}\label{example}

Have a look at the source
\href{https://github.com/typst/packages/raw/main/packages/preview/tablem/0.1.0/examples/example.typ}{here}
.

\pandocbounded{\includegraphics[keepaspectratio]{https://github.com/typst/packages/raw/main/packages/preview/tablem/0.1.0/examples/example.png}}

\subsection{Usage}\label{usage}

You can simply copy the markdown table and paste it in
\texttt{\ tablem\ } function.

\begin{Shaded}
\begin{Highlighting}[]
\NormalTok{\#import "@preview/tablem:0.1.0": tablem}

\NormalTok{\#tablem[}
\NormalTok{  | *Name* | *Location* | *Height* | *Score* |}
\NormalTok{  | {-}{-}{-}{-}{-}{-} | {-}{-}{-}{-}{-}{-}{-}{-}{-}{-} | {-}{-}{-}{-}{-}{-}{-}{-} | {-}{-}{-}{-}{-}{-}{-} |}
\NormalTok{  | John   | Second St. | 180 cm   |  5      |}
\NormalTok{  | Wally  | Third Av.  | 160 cm   |  10     |}
\NormalTok{]}
\end{Highlighting}
\end{Shaded}

And you can use custom render function.

\begin{Shaded}
\begin{Highlighting}[]
\NormalTok{\#import "@preview/tablex:0.0.6": tablex, hlinex}
\NormalTok{\#import "@preview/tablem:0.1.0": tablem}

\NormalTok{\#let three{-}line{-}table = tablem.with(}
\NormalTok{  render: (columns: auto, ..args) =\textgreater{} \{}
\NormalTok{    tablex(}
\NormalTok{      columns: columns,}
\NormalTok{      auto{-}lines: false,}
\NormalTok{      align: center + horizon,}
\NormalTok{      hlinex(y: 0),}
\NormalTok{      hlinex(y: 1),}
\NormalTok{      ..args,}
\NormalTok{      hlinex(),}
\NormalTok{    )}
\NormalTok{  \}}
\NormalTok{)}

\NormalTok{\#three{-}line{-}table[}
\NormalTok{  | *Name* | *Location* | *Height* | *Score* |}
\NormalTok{  | {-}{-}{-}{-}{-}{-} | {-}{-}{-}{-}{-}{-}{-}{-}{-}{-} | {-}{-}{-}{-}{-}{-}{-}{-} | {-}{-}{-}{-}{-}{-}{-} |}
\NormalTok{  | John   | Second St. | 180 cm   |  5      |}
\NormalTok{  | Wally  | Third Av.  | 160 cm   |  10     |}
\NormalTok{]}
\end{Highlighting}
\end{Shaded}

\pandocbounded{\includegraphics[keepaspectratio]{https://github.com/typst/packages/raw/main/packages/preview/tablem/0.1.0/examples/example.png}}

\subsection{\texorpdfstring{\texttt{\ tablem\ }
function}{ tablem  function}}\label{tablem-function}

\begin{Shaded}
\begin{Highlighting}[]
\NormalTok{\#let tablem(}
\NormalTok{  render: table,}
\NormalTok{  ignore{-}second{-}row: true,}
\NormalTok{  ..args,}
\NormalTok{  body}
\NormalTok{) = \{ .. \}}
\end{Highlighting}
\end{Shaded}

\textbf{Arguments:}

\begin{itemize}
\tightlist
\item
  \texttt{\ render\ } : {[}
  \texttt{\ (columns:\ int,\ ..args)\ =\textgreater{}\ \{\ ..\ \}\ } {]}
  â€'' Custom render function, default to be \texttt{\ table\ } ,
  receiving a integer-type columns, which is the count of first row.
  \texttt{\ ..args\ } is the combination of \texttt{\ args\ } of
  \texttt{\ tablem\ } function and children genenerated from
  \texttt{\ body\ } .
\item
  \texttt{\ ignore-second-row\ } : {[} \texttt{\ boolean\ } {]} â€''
  Whether to ignore the second row (something like
  \texttt{\ \textbar{}-\/-\/-\textbar{}\ } ).
\item
  \texttt{\ args\ } : {[} \texttt{\ any\ } {]} â€'' Some arguments you
  want to pass to \texttt{\ render\ } function.
\item
  \texttt{\ body\ } : {[} \texttt{\ content\ } {]} â€'' The
  markdown-like table. There should be no extra line breaks in it.
\end{itemize}

\subsection{Limitations}\label{limitations}

Cell merging has not yet been implemented.

\subsection{License}\label{license}

This project is licensed under the MIT License.

\subsubsection{How to add}\label{how-to-add}

Copy this into your project and use the import as \texttt{\ tablem\ }

\begin{verbatim}
#import "@preview/tablem:0.1.0"
\end{verbatim}

\includesvg[width=0.16667in,height=0.16667in]{/assets/icons/16-copy.svg}

Check the docs for
\href{https://typst.app/docs/reference/scripting/\#packages}{more
information on how to import packages} .

\subsubsection{About}\label{about}

\begin{description}
\tightlist
\item[Author :]
OrangeX4
\item[License:]
MIT
\item[Current version:]
0.1.0
\item[Last updated:]
November 18, 2023
\item[First released:]
November 18, 2023
\item[Archive size:]
2.37 kB
\href{https://packages.typst.org/preview/tablem-0.1.0.tar.gz}{\pandocbounded{\includesvg[keepaspectratio]{/assets/icons/16-download.svg}}}
\item[Repository:]
\href{https://github.com/OrangeX4/typst-tablem}{GitHub}
\end{description}

\subsubsection{Where to report issues?}\label{where-to-report-issues}

This package is a project of OrangeX4 . Report issues on
\href{https://github.com/OrangeX4/typst-tablem}{their repository} . You
can also try to ask for help with this package on the
\href{https://forum.typst.app}{Forum} .

Please report this package to the Typst team using the
\href{https://typst.app/contact}{contact form} if you believe it is a
safety hazard or infringes upon your rights.

\phantomsection\label{versions}
\subsubsection{Version history}\label{version-history}

\begin{longtable}[]{@{}ll@{}}
\toprule\noalign{}
Version & Release Date \\
\midrule\noalign{}
\endhead
\bottomrule\noalign{}
\endlastfoot
0.1.0 & November 18, 2023 \\
\end{longtable}

Typst GmbH did not create this package and cannot guarantee correct
functionality of this package or compatibility with any version of the
Typst compiler or app.


\section{Package List LaTeX/zero.tex}
\title{typst.app/universe/package/zero}

\phantomsection\label{banner}
\section{zero}\label{zero}

{ 0.3.0 }

Advanced scientific number formatting.

{ } Featured Package

\phantomsection\label{readme}
\emph{Advanced scientific number formatting.}

\href{https://typst.app/universe/package/zero}{\pandocbounded{\includegraphics[keepaspectratio]{https://img.shields.io/badge/dynamic/toml?url=https\%3A\%2F\%2Fraw.githubusercontent.com\%2FMc-Zen\%2Fzero\%2Fv0.3.0\%2Ftypst.toml&query=\%24.package.version&prefix=v&logo=typst&label=package&color=239DAD}}}
\href{https://github.com/Mc-Zen/zero/actions/workflows/run_tests.yml}{\pandocbounded{\includesvg[keepaspectratio]{https://github.com/Mc-Zen/zero/actions/workflows/run_tests.yml/badge.svg}}}
\href{https://github.com/Mc-Zen/zero/blob/main/LICENSE}{\pandocbounded{\includegraphics[keepaspectratio]{https://img.shields.io/badge/license-MIT-blue}}}

\begin{itemize}
\tightlist
\item
  \href{https://github.com/typst/packages/raw/main/packages/preview/zero/0.3.0/\#introduction}{Introduction}
\item
  \href{https://github.com/typst/packages/raw/main/packages/preview/zero/0.3.0/\#quick-demo}{Quick
  Demo}
\item
  \href{https://github.com/typst/packages/raw/main/packages/preview/zero/0.3.0/\#documentation}{Documentation}
\item
  \href{https://github.com/typst/packages/raw/main/packages/preview/zero/0.3.0/\#table-alignment}{Table
  alignment}
\item
  \href{https://github.com/typst/packages/raw/main/packages/preview/zero/0.3.0/\#zero-for-third-party-packages}{Zero
  for third-party packages}
\end{itemize}

\subsection{Introduction}\label{introduction}

Proper number formatting requires some love for detail to guarantee a
readable and clear output. This package provides tools to ensure
consistent formatting and to simplify the process of following
established publication practices. Key features are

\begin{itemize}
\tightlist
\item
  \textbf{standardized} formatting,
\item
  digit
  \href{https://github.com/typst/packages/raw/main/packages/preview/zero/0.3.0/\#grouping}{\textbf{grouping}}
  , e.g., 299 792 458 instead of 299792458,
\item
  \textbf{plug-and-play} number
  \href{https://github.com/typst/packages/raw/main/packages/preview/zero/0.3.0/\#table-alignment}{\textbf{alignment
  in tables}} ,
\item
  quick scientific notation, e.g., \texttt{\ "2e4"\ } becomes
  2Ã---10â?´,
\item
  symmetric and asymmetric
  \href{https://github.com/typst/packages/raw/main/packages/preview/zero/0.3.0/\#specifying-uncertainties}{\textbf{uncertainties}}
  ,
\item
  \href{https://github.com/typst/packages/raw/main/packages/preview/zero/0.3.0/\#rounding}{\textbf{rounding}}
  in various modes,
\item
  and some specials for package authors.
\end{itemize}

A number in scientific notation consists of three parts of which the
latter two are optional. The first part is the \emph{mantissa} that may
consist of an \emph{integer} and a \emph{fractional} part. In many
fields of science, values are not known exactly and the corresponding
\emph{uncertainty} is then given along with the mantissa. Lastly, to
facilitate reading very large or small numbers, the mantissa may be
multiplied with a \emph{power} of 10 (or another base).

The anatomy of a formatted number is shown in the following figure.

\pandocbounded{\includegraphics[keepaspectratio]{https://github.com/user-attachments/assets/7ca9fa48-b732-4f4e-911f-b719e83305be}}

\subsection{Quick Demo}\label{quick-demo}

\begin{longtable}[]{@{}llll@{}}
\toprule\noalign{}
Code & Output & Code & Output \\
\midrule\noalign{}
\endhead
\bottomrule\noalign{}
\endlastfoot
\texttt{\ num("1.2e4")\ } & 1.2Ã---10â?´ & \texttt{\ num{[}1.2e4{]}\ } &
1.2Ã---10â?´ \\
\texttt{\ num("-5e-4")\ } & âˆ'5Ã---10â?»â?´ &
\texttt{\ num(fixed:\ -2){[}0.02{]}\ } & 2Ã---10â?»Â² \\
\texttt{\ num("9.81+-.01")\ } & 9.81±0.01 &
\texttt{\ num("9.81+0.02-.01")\ } & 9.81�²₋� \\
\texttt{\ num("9.81+-.01e2")\ } & (9.81±0.01)Ã---10² &
\texttt{\ num(base:\ 2){[}3e4{]}\ } & 3Ã---2â?´ \\
\end{longtable}

\subsection{Documentation}\label{documentation}

\begin{itemize}
\tightlist
\item
  \href{https://github.com/typst/packages/raw/main/packages/preview/zero/0.3.0/\#num}{Function
  \texttt{\ num\ }}
\item
  \href{https://github.com/typst/packages/raw/main/packages/preview/zero/0.3.0/\#grouping}{Grouping}
\item
  \href{https://github.com/typst/packages/raw/main/packages/preview/zero/0.3.0/\#rounding}{Rounding}
\item
  \href{https://github.com/typst/packages/raw/main/packages/preview/zero/0.3.0/\#specifying-uncertainties}{Uncertainties}
\item
  \href{https://github.com/typst/packages/raw/main/packages/preview/zero/0.3.0/\#table-alignment}{Table
  alignment}
\end{itemize}

\subsubsection{\texorpdfstring{\texttt{\ num\ }}{ num }}\label{num}

The function \texttt{\ num()\ } is the heart of \emph{Zero} . It
provides a wide range of number formatting utilities and its default
values are configurable via \texttt{\ set-num()\ } which takes the same
named arguments as \texttt{\ num()\ } .

\begin{Shaded}
\begin{Highlighting}[]
\NormalTok{\#let num(}
\NormalTok{  number:                 str | content | int | float | dictionary | array,}
\NormalTok{  digits:                 auto | int = auto,}
\NormalTok{  fixed:                  none | int = none,}

\NormalTok{  decimal{-}separator:      str = ".",}
\NormalTok{  product:                content = sym.times,}
\NormalTok{  tight:                  boolean = false,}
\NormalTok{  math:                   boolean = true,}
\NormalTok{  omit{-}unity{-}mantissa:    boolean = true,}
\NormalTok{  positive{-}sign:          boolean = false,}
\NormalTok{  positive{-}sign{-}exponent: boolean = false,}
\NormalTok{  base:                   int | content = 10,}
\NormalTok{  uncertainty{-}mode:       str = "separate",}
\NormalTok{  round:                  dictionary,}
\NormalTok{  group:                  dictionary,}
\NormalTok{)}
\end{Highlighting}
\end{Shaded}

\begin{itemize}
\tightlist
\item
  \texttt{\ number:\ str\ \textbar{} content\ \textbar{}\ int\ \textbar{}\ float\ \textbar{}\ array\ }
  : Number input; \texttt{\ str\ } is preferred. If the input is
  \texttt{\ content\ } , it may only contain text nodes. Numeric types
  \texttt{\ int\ } and \texttt{\ float\ } are supported but not
  encouraged because of information loss (e.g., the number of trailing
  “0� digits or the exponent). The remaining types
  \texttt{\ dictionary\ } and \texttt{\ array\ } are intended for
  advanced use, see
  \href{https://github.com/typst/packages/raw/main/packages/preview/zero/0.3.0/\#zero-for-third-party-packages}{below}
  .
\item
  \texttt{\ digits:\ auto\ \textbar{} int\ =\ auto\ } : Truncates the
  number at a given (positive) number of decimal places or pads the
  number with zeros if necessary. This is independent of
  \href{https://github.com/typst/packages/raw/main/packages/preview/zero/0.3.0/\#rounding}{rounding}
  .
\item
  \texttt{\ fixed:\ none\ \textbar{}\ int\ =\ none\ } : If not
  \texttt{\ none\ } , forces a fixed exponent. Additional exponents
  given in the number input are taken into account.
\item
  \texttt{\ decimal-separator:\ str\ =\ "."\ } : Specifies the marker
  that is used for separating integer and decimal part.
\item
  \texttt{\ product:\ content\ =\ sym.times\ } : Specifies the
  multiplication symbol used for scientific notation.
\item
  \texttt{\ tight:\ boolean\ =\ false\ } : If true, tight spacing is
  applied between operands (applies to Ã--- and ±).
\item
  \texttt{\ math:\ boolean\ =\ true\ } : If set to \texttt{\ false\ } ,
  the parts of the number won’t be wrapped in a
  \texttt{\ math.equation\ } wherever feasible. This makes it possible
  to use \texttt{\ num()\ } with non-math fonts to some extent. Powers
  are always rendered in math mode.
\item
  \texttt{\ omit-unity-mantissa:\ boolean\ =\ false\ } : Determines
  whether a mantissa of 1 is omitted in scientific notation, e.g., 10â?´
  instead of 1·10�.
\item
  \texttt{\ positive-sign:\ boolean\ =\ false\ } : If set to
  \texttt{\ true\ } , positive coefficients are shown with a + sign.
\item
  \texttt{\ positive-sign-exponent:\ boolean\ =\ false\ } : If set to
  \texttt{\ true\ } , positive exponents are shown with a + sign.
\item
  \texttt{\ base:\ int\ \textbar{}\ content\ =\ 10\ } : The base used
  for scientific power notation.
\item
  \texttt{\ uncertainty-mode:\ str\ =\ "separate"\ } : Selects one of
  the modes \texttt{\ "separate"\ } , \texttt{\ "compact"\ } , or
  \texttt{\ "compact-separator"\ } for displaying uncertainties. The
  different behaviors are shown below:
\end{itemize}

\begin{longtable}[]{@{}lll@{}}
\toprule\noalign{}
\texttt{\ "separate"\ } & \texttt{\ "compact"\ } &
\texttt{\ "compact-separator"\ } \\
\midrule\noalign{}
\endhead
\bottomrule\noalign{}
\endlastfoot
1.7±0.2 & 1.7(2) & 1.7(2) \\
6.2±2.1 & 6.2(21) & 6.2(2.1) \\
1.7��˙̇²₋₀.₠& 1.7�²₋₠& 1.7�²₋₠\\
1.7â?ºÂ²Ë™Ì‡â?°â‚‹â‚\ldots.â‚€ & 1.7â?ºÂ²â?°â‚‹â‚\ldots â‚€ &
1.7â?ºÂ²Ë™Ì‡â?°â‚‹â‚\ldots.â‚€ \\
\end{longtable}

\begin{itemize}
\tightlist
\item
  \texttt{\ round:\ dictionary\ } : You can provide one or more rounding
  options in a dictionary. Also see
  \href{https://github.com/typst/packages/raw/main/packages/preview/zero/0.3.0/\#rounding}{rounding}
  .
\item
  \texttt{\ group:\ dictionary\ } : You can provide one or more grouping
  options in a dictionary. Also see
  \href{https://github.com/typst/packages/raw/main/packages/preview/zero/0.3.0/\#grouping}{grouping}
  .
\end{itemize}

Configuration example:

\begin{Shaded}
\begin{Highlighting}[]
\NormalTok{\#set{-}num(product: math.dot, tight: true)}
\end{Highlighting}
\end{Shaded}

\subsubsection{Grouping}\label{grouping}

Digit grouping is important for keeping large figures readable. It is
customary to separate thousands with a thin space, a period, comma, or
an apostrophe (however, we discourage using a period or a comma to avoid
confusion since both are used for decimal separators in various
countries).

\pandocbounded{\includegraphics[keepaspectratio]{https://github.com/user-attachments/assets/1f53ae33-3e99-483d-ac6a-6e3cbed5484b}}

Digit grouping can be configured with the \texttt{\ set-group()\ }
function.

\begin{Shaded}
\begin{Highlighting}[]
\NormalTok{\#let set{-}group(}
\NormalTok{  size:       int = 3, }
\NormalTok{  separator:  content = sym.space.thin,}
\NormalTok{  threshold:  int = 5}
\NormalTok{)}
\end{Highlighting}
\end{Shaded}

\begin{itemize}
\tightlist
\item
  \texttt{\ size:\ int\ =\ 3\ } : Determines the size of the groups.
\item
  \texttt{\ separator:\ content\ =\ sym.space.thin\ } : Separator
  between groups.
\item
  \texttt{\ threshold:\ int\ =\ 5\ } : Necessary number of digits needed
  for digit grouping to kick in. Four-digit numbers for example are
  usually not grouped at all since they can still be read easily.
\end{itemize}

Configuration example:

\begin{Shaded}
\begin{Highlighting}[]
\NormalTok{\#set{-}group(separator: "\textquotesingle{}", threshold: 4)}
\end{Highlighting}
\end{Shaded}

Grouping can be turned off altogether by setting the
\texttt{\ threshold\ } to \texttt{\ calc.inf\ } .

\subsubsection{Rounding}\label{rounding}

Rounding can be configured with the \texttt{\ set-round()\ } function.

\begin{Shaded}
\begin{Highlighting}[]
\NormalTok{\#let set{-}round(}
\NormalTok{  mode:       none | str = none,}
\NormalTok{  precision:  int = 2,}
\NormalTok{  pad:        boolean = true,}
\NormalTok{  direction:  str = "nearest",}
\NormalTok{)}
\end{Highlighting}
\end{Shaded}

\begin{itemize}
\tightlist
\item
  \texttt{\ mode:\ none\ \textbar{} str\ =\ none\ } : Sets the
  rounding mode. The possible options are

  \begin{itemize}
  \tightlist
  \item
    \texttt{\ none\ } : Rounding is turned off.
  \item
    \texttt{\ "places"\ } : The number is rounded to the number of
    decimal places given by the \texttt{\ precision\ } parameter.
  \item
    \texttt{\ "figures"\ } : The number is rounded to a number of
    significant figures given by the \texttt{\ precision\ } parameter.
  \item
    \texttt{\ "uncertainty"\ } : Requires giving an uncertainty value.
    The uncertainty is rounded to significant figures according to the
    \texttt{\ precision\ } argument and then the number is rounded to
    the same number of decimal places as the uncertainty.
  \end{itemize}
\item
  \texttt{\ precision:\ int\ =\ 2\ } : The precision to round to. Also
  see parameter \texttt{\ mode\ } .
\item
  \texttt{\ pad:\ boolean\ =\ true\ } : Whether to pad the number with
  zeros if the number has fewer digits than the rounding precision.
\item
  \texttt{\ direction:\ str\ =\ "nearest"\ } : Sets the rounding
  direction.

  \begin{itemize}
  \tightlist
  \item
    \texttt{\ "nearest"\ } : Rounding takes place in the usual fashion,
    rounding to the nearer number, e.g., 2.34 â†' 2.3 and 2.36 â†' 2.4.
  \item
    \texttt{\ "down"\ } : Always rounds down, e.g., 2.38 â†' 2.3 and
    2.30 â†' 2.3.
  \item
    \texttt{\ "up"\ } : Always rounds up, e.g., 2.32 â†' 2.4 and 2.30
    â†' 2.3.
  \end{itemize}
\end{itemize}

\subsubsection{Specifying uncertainties}\label{specifying-uncertainties}

There are two ways of specifying uncertainties:

\begin{itemize}
\tightlist
\item
  Applying an uncertainty to the least significant digits using
  parentheses, e.g., \texttt{\ 2.3(4)\ } ,
\item
  Denoting an absolute uncertainty, e.g., \texttt{\ 2.3+-0.4\ } becomes
  2.3±0.4.
\end{itemize}

Zero supports both and can convert between these two, so that you can
pick the displayed style (configured via \texttt{\ uncertainty-mode\ } ,
see above) independently of the input style.

How do uncertainties interplay with exponents? The uncertainty needs to
come first, and the exponent applies to both the mantissa and the
uncertainty, e.g., \texttt{\ num("1.23+-.04e2")\ } becomes

(1.23 ± 0.03)Ã---10²

Note that the mantissa is now put in parentheses to disambiguate the
application of the power.

In some cases, the uncertainty is asymmetric which can be expressed via
\texttt{\ num("1.23+0.02-0.01")\ }

1.23��˙̇�²₋₀.₀�

\subsubsection{Table alignment}\label{table-alignment}

In scientific publication, presenting many numbers in a readable fashion
can be a difficult discipline. A good starting point is to align numbers
in a table at the decimal separator. With \emph{Zero} , this can be
accomplished by using \texttt{\ ztable\ } , a wrapper for the built-in
\texttt{\ table\ } function. It features an additional parameter
\texttt{\ format\ } which takes an array of \texttt{\ none\ } ,
\texttt{\ auto\ } , or \texttt{\ dictionary\ } values to turn on number
alignment for specific columns.

\begin{Shaded}
\begin{Highlighting}[]
\NormalTok{\#ztable(}
\NormalTok{  columns: 3,}
\NormalTok{  align: center,}
\NormalTok{  format: (none, auto, auto),}
\NormalTok{  $n$, $α$, $β$,}
\NormalTok{  [1], [3.45], [{-}11.1],}
\NormalTok{  ..}
\NormalTok{)}
\end{Highlighting}
\end{Shaded}

Non-number entries (e.g., in the header) are automatically recognized in
some cases and will not be aligned. In ambiguous cases, adding a leading
or trailing space tells \emph{Zero} not to apply alignment to this cell,
e.g., \texttt{\ {[}Angle\ {]}\ } instead of \texttt{\ {[}Angle{]}\ } .

\pandocbounded{\includegraphics[keepaspectratio]{https://github.com/user-attachments/assets/2effb7f0-0d9b-401a-92e1-20461d0c1fcb}}

In addition, you can prefix or suffix a numeral with content wrapped by
the function \texttt{\ nonum{[}{]}\ } to mark it as \emph{not belonging
to the number} . The remaining content may still be recognized as a
number and formatted/aligned accordingly.

\begin{Shaded}
\begin{Highlighting}[]
\NormalTok{\#ztable(}
\NormalTok{  format: (auto,),}
\NormalTok{  [\#nonum[€]123.0\#nonum(footnote[A special number])],}
\NormalTok{  [12.111],}
\NormalTok{)}
\end{Highlighting}
\end{Shaded}

\pandocbounded{\includegraphics[keepaspectratio]{https://github.com/user-attachments/assets/270ae789-2a8c-44a3-b3a9-0ca588bfbad3}}

Zero not only aligns numbers at the decimal point but also at the
uncertainty and exponent part. Moreover, by passing a
\texttt{\ dictionary\ } instead of \texttt{\ auto\ } , a set of
\texttt{\ num()\ } arguments to apply to all numbers in a column can be
specified.

\begin{Shaded}
\begin{Highlighting}[]
\NormalTok{\#ztable(}
\NormalTok{  columns: 4,}
\NormalTok{  align: center,}
\NormalTok{  format: (none, auto, auto, (digits: 1)),}
\NormalTok{  $n$, $α$, $β$, $γ$,}
\NormalTok{  [1], [3.45e2], [{-}11.1+{-}3], [0],}
\NormalTok{  ..}
\NormalTok{)}
\end{Highlighting}
\end{Shaded}

\pandocbounded{\includegraphics[keepaspectratio]{https://github.com/user-attachments/assets/c96941bc-f002-4b93-b2cd-705c8104682f}}

\subsection{Zero for third-party
packages}\label{zero-for-third-party-packages}

This package provides some useful extras for third-party packages that
generate formatted numbers (for example graphics libraries).

Instead of passing a \texttt{\ str\ } to \texttt{\ num()\ } , it is also
possible to pass a dictionary of the form

\begin{Shaded}
\begin{Highlighting}[]
\NormalTok{(}
\NormalTok{  mantissa:  str | int | float,}
\NormalTok{  e:         none | str,}
\NormalTok{  pm:        none | array}
\NormalTok{)}
\end{Highlighting}
\end{Shaded}

This way, parsing the number can be avoided which makes especially sense
for packages that generate numbers (e.g., tick labels for a diagram
axis) with independent mantissa and exponent.

Furthermore, \texttt{\ num()\ } also allows \texttt{\ array\ } arguments
for \texttt{\ number\ } which allows for more efficient batch-processing
of numbers with the same setup. In this case, the caller of the function
needs to provide \texttt{\ context\ } .

\subsection{Changelog}\label{changelog}

\subsubsection{Version 0.3.0}\label{version-0.3.0}

\begin{itemize}
\tightlist
\item
  Adds \texttt{\ nonum{[}{]}\ } function that can be used to mark
  content in cells as \emph{not belonging to the number} . The remaining
  content may still be recognized as a number and formatted/aligned
  accordingly. The content wrapped by \texttt{\ nonum{[}{]}\ } is
  preserved.
\item
  Fixes number alignment tables with new version Typst 0.12.
\end{itemize}

\subsubsection{Version 0.2.0}\label{version-0.2.0}

\begin{itemize}
\tightlist
\item
  Adds support for using non-math fonts for \texttt{\ num\ } via the
  option \texttt{\ math\ } . This can be activated by calling
  \texttt{\ \#set-num(math:\ false)\ } .
\item
  Performance improvements for both \texttt{\ num()\ } and
  \texttt{\ ztable(9)\ }
\end{itemize}

\subsubsection{Version 0.1.0}\label{version-0.1.0}

\subsubsection{How to add}\label{how-to-add}

Copy this into your project and use the import as \texttt{\ zero\ }

\begin{verbatim}
#import "@preview/zero:0.3.0"
\end{verbatim}

\includesvg[width=0.16667in,height=0.16667in]{/assets/icons/16-copy.svg}

Check the docs for
\href{https://typst.app/docs/reference/scripting/\#packages}{more
information on how to import packages} .

\subsubsection{About}\label{about}

\begin{description}
\tightlist
\item[Author :]
\href{https://github.com/Mc-Zen}{Mc-Zen}
\item[License:]
MIT
\item[Current version:]
0.3.0
\item[Last updated:]
October 28, 2024
\item[First released:]
September 16, 2024
\item[Minimum Typst version:]
0.11.0
\item[Archive size:]
15.7 kB
\href{https://packages.typst.org/preview/zero-0.3.0.tar.gz}{\pandocbounded{\includesvg[keepaspectratio]{/assets/icons/16-download.svg}}}
\item[Repository:]
\href{https://github.com/Mc-Zen/zero}{GitHub}
\item[Categor ies :]
\begin{itemize}
\tightlist
\item[]
\item
  \pandocbounded{\includesvg[keepaspectratio]{/assets/icons/16-chart.svg}}
  \href{https://typst.app/universe/search/?category=visualization}{Visualization}
\item
  \pandocbounded{\includesvg[keepaspectratio]{/assets/icons/16-layout.svg}}
  \href{https://typst.app/universe/search/?category=layout}{Layout}
\end{itemize}
\end{description}

\subsubsection{Where to report issues?}\label{where-to-report-issues}

This package is a project of Mc-Zen . Report issues on
\href{https://github.com/Mc-Zen/zero}{their repository} . You can also
try to ask for help with this package on the
\href{https://forum.typst.app}{Forum} .

Please report this package to the Typst team using the
\href{https://typst.app/contact}{contact form} if you believe it is a
safety hazard or infringes upon your rights.

\phantomsection\label{versions}
\subsubsection{Version history}\label{version-history}

\begin{longtable}[]{@{}ll@{}}
\toprule\noalign{}
Version & Release Date \\
\midrule\noalign{}
\endhead
\bottomrule\noalign{}
\endlastfoot
0.3.0 & October 28, 2024 \\
\href{https://typst.app/universe/package/zero/0.2.0/}{0.2.0} & October
4, 2024 \\
\href{https://typst.app/universe/package/zero/0.1.0/}{0.1.0} & September
16, 2024 \\
\end{longtable}

Typst GmbH did not create this package and cannot guarantee correct
functionality of this package or compatibility with any version of the
Typst compiler or app.


\section{Package List LaTeX/minimalbc.tex}
\title{typst.app/universe/package/minimalbc}

\phantomsection\label{banner}
\phantomsection\label{template-thumbnail}
\pandocbounded{\includegraphics[keepaspectratio]{https://packages.typst.org/preview/thumbnails/minimalbc-0.0.1-small.webp}}

\section{minimalbc}\label{minimalbc}

{ 0.0.1 }

Sleek, minimalist design for professional business cards. Emphasizing
clarity and elegance.

{ } Featured Template

\href{/app?template=minimalbc&version=0.0.1}{Create project in app}

\phantomsection\label{readme}
This repository provides a Typst template for creating sleek and
minimalist professional business cards.

The function, \textbf{minimalbc} , allows you to customize the majority
of the business card’s elements.

By default, the layout is horizontal. However, it can be easily switched
to a vertical layout by passing the value true to the flip argument in
the minimalbc function.

Here’s an example of how to use the minimalbc function:

\begin{Shaded}
\begin{Highlighting}[]
\NormalTok{\#import "@preview/minimalbc:0.1.0": minimalbc}

\NormalTok{\#show: minimalbc.with(}
\NormalTok{    // possible geo\_size options: eu, us, jp , cn}
\NormalTok{    geo\_size: "eu",}
\NormalTok{    flip:true,}
\NormalTok{    company\_name: "company name",}
\NormalTok{    name: "first and last name",}
\NormalTok{    role: "role",}
\NormalTok{    telephone\_number: "+000 00 000000",}
\NormalTok{    email\_address: "me@me.com",}
\NormalTok{    website: "example.com",}
\NormalTok{    company\_logo: image("company\_logo.png"),}
\NormalTok{    bg\_color: "ffffff",}
\NormalTok{)}

\end{Highlighting}
\end{Shaded}

When compiled, this example produces a PDF file named ‘your
filename’.pdf (see example.pdf).

Feel free to download and use this as a starting point for your own
business cards.

\href{/app?template=minimalbc&version=0.0.1}{Create project in app}

\subsubsection{How to use}\label{how-to-use}

Click the button above to create a new project using this template in
the Typst app.

You can also use the Typst CLI to start a new project on your computer
using this command:

\begin{verbatim}
typst init @preview/minimalbc:0.0.1
\end{verbatim}

\includesvg[width=0.16667in,height=0.16667in]{/assets/icons/16-copy.svg}

\subsubsection{About}\label{about}

\begin{description}
\tightlist
\item[Author :]
\href{https://github.com/sevehub}{S. Tessarin}
\item[License:]
MIT
\item[Current version:]
0.0.1
\item[Last updated:]
June 14, 2024
\item[First released:]
June 14, 2024
\item[Minimum Typst version:]
0.11.0
\item[Archive size:]
80.4 kB
\href{https://packages.typst.org/preview/minimalbc-0.0.1.tar.gz}{\pandocbounded{\includesvg[keepaspectratio]{/assets/icons/16-download.svg}}}
\item[Repository:]
\href{https://github.com/sevehub/minimalbc}{GitHub}
\item[Categor y :]
\begin{itemize}
\tightlist
\item[]
\item
  \pandocbounded{\includesvg[keepaspectratio]{/assets/icons/16-envelope.svg}}
  \href{https://typst.app/universe/search/?category=office}{Office}
\end{itemize}
\end{description}

\subsubsection{Where to report issues?}\label{where-to-report-issues}

This template is a project of S. Tessarin . Report issues on
\href{https://github.com/sevehub/minimalbc}{their repository} . You can
also try to ask for help with this template on the
\href{https://forum.typst.app}{Forum} .

Please report this template to the Typst team using the
\href{https://typst.app/contact}{contact form} if you believe it is a
safety hazard or infringes upon your rights.

\phantomsection\label{versions}
\subsubsection{Version history}\label{version-history}

\begin{longtable}[]{@{}ll@{}}
\toprule\noalign{}
Version & Release Date \\
\midrule\noalign{}
\endhead
\bottomrule\noalign{}
\endlastfoot
0.0.1 & June 14, 2024 \\
\end{longtable}

Typst GmbH did not create this template and cannot guarantee correct
functionality of this template or compatibility with any version of the
Typst compiler or app.


\section{Package List LaTeX/modern-cug-thesis.tex}
\title{typst.app/universe/package/modern-cug-thesis}

\phantomsection\label{banner}
\phantomsection\label{template-thumbnail}
\pandocbounded{\includegraphics[keepaspectratio]{https://packages.typst.org/preview/thumbnails/modern-cug-thesis-0.1.0-small.webp}}

\section{modern-cug-thesis}\label{modern-cug-thesis}

{ 0.1.0 }

中国地质大学(武汉)学ä½?论æ--‡æ¨¡æ?¿ã€‚China University of
Geosciences Thesis based on Typst.

\href{/app?template=modern-cug-thesis&version=0.1.0}{Create project in
app}

\phantomsection\label{readme}
\textbf{cug-thesis-thesis}
适ç''¨äºŽä¸­å›½åœ°è´¨å¤§å­¦ï¼ˆæ­¦æ±‰ï¼‰å­¦ä½?论æ--‡æ¨¡æ?¿ï¼Œå\ldots·æœ‰ä¾¿æ?·ã€?简å?•ã€?实æ---¶æ¸²æŸ``等特性。欢迎å?„ä½?å?Œå­¦ã€?æ~¡å?‹ä»¬å‰?æ?¥
\href{https://github.com/Rsweater/cug-thesis-typst/issues}{Issues}
交æµ?å­¦ä¹~\textasciitilde{}

\pandocbounded{\includegraphics[keepaspectratio]{https://cdn.jsdelivr.net/gh/Rsweater/images/img/preview.gif}}

\subsection{为什么考è™` Typst
实现学ä½?论æ--‡æ¨¡æ?¿ï¼Ÿ}\label{uxe4uxbauxe4uxe4uxb9ux2c6uxe8ux192uxe8-typst-uxe5ux17euxe7ux17euxe5uxe4uxbduxe8uxbauxe6uxe6uxe6uxefuxbcuxff}

\begin{enumerate}
\tightlist
\item
  é¦--è¦?是为了学ä¹~。看到 Typst
  惊人的�长速度,确实有点�激动。Typst 似乎继承了
  Markdown�Tex�Wiki �自的优点于自身。
\item
  本人写æ--‡æ¡£ç›¸å¯¹æ?¥è¯´è¾ƒä¸ºç²---心,使ç''¨ Word
  模æ?¿ä¼šå¿?ä¸?ä½?çš„å??å¤?去检查æ~¼å¼?是å?¦ç¬¦å?ˆè¦?求。å?ˆå?¬è¯´
  Latex
  写毕业论æ--‡å?¯èƒ½å?Žé?¢ç¼--è¯`一次需è¦?å‡~å??ç§'\textasciitilde\textasciitilde{}
  虽然这个自己å?ªæ˜¯å?¬è¯´ï¼Œä½†æ˜¯ LaTex 在线ç¼--è¾`çš„æ--¹å¼?
  Overleaf
  达到一定的ç¼--è¯`æ---¶é---´æ''¶è´¹è¿™ä¸ªæ˜¯çœŸçš„,就æˆ`å°?论æ--‡éƒ½å‹‰å¼ºå¤Ÿç''¨ã€‚自己使ç''¨å¼€æº?çš„
  Overleaf
  �建的平�功能上总是缺点什么,奈何自己��懂\textasciitilde{}
  自己�建的本地的 Tex
  环境éš?解决了ç¼--è¯`æ---¶é---´ä»˜è´¹é---®é¢˜ã€‚但是涉å?Šåˆ°çš„å®?åŒ\ldots ã€?环境,å‰?段æ---¶é---´æ‰``å¼€çª?然ä¸?能ç''¨äº†ï¼Œæ?£é¼``å?Šå¤©ä¸?知是何原å›~\textasciitilde{}
  直至é‡?æ--°è£\ldots 了2024å¹´çš„ LaTex
  环境æ‰?é‡?æ--°è¿?行自己的学ä½?论æ--‡ã€‚
\item
  惊å--œçš„å?{}`现 Typst ç¼--è¯`速度真的é?žå¸¸å¿«\textasciitilde{}
  ç»?过一段æ---¶é---´çš„了解,å?{}`现基本满足制作学ä½?论æ--‡çš„需求,于是乎\textasciitilde 就有了这个cug-thesis-typst。
\end{enumerate}

\subsection{�考规范}\label{uxe5uxe8ux192uxe8uxe8ux153ux192}

\begin{itemize}
\tightlist
\item
  本ç§`ç''Ÿè®ºæ--‡æ¨¡æ?¿å?‚考:
  \href{https://bksy.cug.edu.cn/info/1489/1851.htm}{学士学ä½?论æ--‡å†™ä½œè§„范(2018版)-中国地质大学本ç§`ç''Ÿé™¢}
\item
  ç~''究ç''Ÿè®ºæ--‡æ¨¡æ?¿å?‚考:
  \href{https://xgxy.cug.edu.cn/info/1073/3509.htm}{ç~''究ç''Ÿå­¦ä½?论æ--‡å†™ä½œè§„范(2015版)-中国地质大学地ç?†ä¸Žä¿¡æ?¯å·¥ç¨‹å­¦é™¢}
  (对ç~''究ç''Ÿé™¢ç›¸å\ldots³é€šçŸ¥é™„件进行了整ç?†ï¼‰
\end{itemize}

\subsection{模æ?¿è®¤å?¯åº¦é---®é¢˜}\label{uxe6uxe6uxe8uxe5uxe5uxbauxe9uxe9}

\textbf{值å¾---æ??é†'的是}
毕竟是æ°`é---´å®žçŽ°æ¨¡æ?¿ï¼Œæœ‰ä¸?被学院认å?¯çš„å?¯èƒ½æ€§\textasciitilde{}

ç›®å‰?已知æƒ\ldots 况,计ç®---机学院ã€?地信学院对于学ä½?论æ--‡è¦?求ä¸?是太苛刻。去年计ç®---机学院师å\ldots„使ç''¨äº†
Github 的 Latex 模�
\href{https://github.com/Timozer/CUGThesis}{Timozer/CUGThesis:
中国地质大学(武汉)ç~''究ç''Ÿå­¦ä½?论æ--‡ TeX 模æ?¿}
完æˆ?å­¦ä½?论æ--‡ã€‚

而ä¸''å``ˆ\textasciitilde{}
å'±ä»¬çš„ç~''究ç''Ÿå­¦ä½?论æ--‡å†™ä½œè§„范(2015版)似乎è¦?求似乎ä¸?是特别苛刻。请自行æ--Ÿé\ldots Œ\textasciitilde{}

\begin{quote}
�声\textasciitilde{}
ç~''究ç''Ÿå­¦ä½?论æ--‡å†™ä½œè§„范(2015版)似乎还有一处å‰?å?ŽçŸ›ç›¾çš„è¦?求,æ--¯\textasciitilde{}
\end{quote}

\subsection{使ç''¨æ--¹æ³•}\label{uxe4uxbduxe7uxe6uxb9uxe6uxb3}

\subsubsection{Typst
在线ç¼--è¾`}\label{typst-uxe5ux153uxe7uxbauxe7uxbcuxe8uxbe}

本模æ?¿å·²ä¸Šä¼ \href{https://typst.app/universe}{Typst Universe}
,您å?¯ä»¥ä½¿ç''¨ Typst 的官æ--¹ Web App
进行ç¼--è¾`。å?ªéœ€è¦?在 \href{https://typst.app/}{Typst Web App}
中的 \texttt{\ Start\ from\ template\ } 里选择
\texttt{\ modern-cug-thesis\ } ,��从模�创建项目。

æˆ--è€\ldots ,直接点击注册
\href{https://typst.app/app?template=modern-cug-thesis&version=0.1.0}{typst.app.universe.cug-thesis}
,并开始ç¼--写ä½~的论æ--‡\textasciitilde{}

\subsubsection{如果ä½~ç»?常使ç''¨ VS
Code,也æ¯''较推è??使ç''¨è¿™ä¸ª\textasciitilde{}}\label{uxe5uxe6ux17eux153uxe4uxbd-uxe7uxe5uxe4uxbduxe7-vs-codeuxefuxbcux153uxe4uxb9uxffuxe6uxe8uxbeux192uxe6ux17euxe8uxe4uxbduxe7uxe8uxe4uxaa}

\textbf{使ç''¨æ­¥éª¤} :安è£\ldots{} typst (å`½ä»¤è¡Œå·¥å\ldots·) â†'
VS Code
æ?'件(实æ---¶é¢„览ã€?智能æ??é†'),éš?å?Žå°±å?¯ä»¥å‡†å¤‡å¼€å§‹é¡¹ç›®äº†(æ‰``开项目æ--‡ä»¶ã€?æ'°å†™è®ºæ--‡å†\ldots 容)

\begin{enumerate}
\item
  \textbf{安è£\ldots{} typst :}

  \begin{itemize}
  \tightlist
  \item
    \textbf{macOS:} \texttt{\ brew\ install\ typst\ }
  \item
    \textbf{Windows:}
    \texttt{\ winget\ install\ -\/-id\ Typst.Typst\ -l\ "D:\textbackslash{}bw\_ch\textbackslash{}toolkits\textbackslash{}typst"\ }
  \end{itemize}
\item
  \textbf{安è£\ldots æ?'件} :在 VS Code 中安è£
  \href{https://marketplace.visualstudio.com/items?itemName=myriad-dreamin.tinymist}{Tinymist
  Typst}
\item
  \textbf{准备项目æ--‡ä»¶} :

  \begin{itemize}
  \tightlist
  \item
    \textbf{æ--¹æ³•ä¸€ï¼šClone Repo} : 使ç''¨å`½ä»¤
    \texttt{\ git\ clone\ https://github.com/Rsweater/cug-thesis-typst.git\ }
    将整个项目å\ldots‹éš†åˆ°æœ¬åœ°ï¼Œå¯»æ‰¾
    \texttt{\ template/thesis.typ\ } 。
  \item
    \textbf{æ--¹æ³•äºŒï¼šä½¿ç''¨ Typst Packages} :按下
    \texttt{\ Ctrl\ +\ Shift\ +\ P\ } æ‰``å¼€å`½ä»¤ç•Œé?¢ï¼Œè¾``å\ldots¥
    \texttt{\ Typst:\ Show\ available\ Typst\ templates\ (gallery)\ for\ picking\ up\ a\ template\ }
    æ‰``å¼€ Tinymist æ??供的 Template Gallery,然å?Žä»Žé‡Œé?¢æ‰¾åˆ°
    \texttt{\ cug-thesis\ } ,点击 \texttt{\ �\ }
    按é'®è¿›è¡Œæ''¶è---?,以å?Šç‚¹å‡» \texttt{\ +\ }
    å?·ï¼Œå°±å?¯ä»¥åˆ›å»ºå¯¹åº''的论æ--‡æ¨¡æ?¿äº†ï¼Œä¼šå‡ºçŽ°
    \texttt{\ ref.bib\ } 以� \texttt{\ thesis.typ\ } 。
  \end{itemize}
\item
  æ‰``开开始ç¼--写论æ--‡å†\ldots 容\textasciitilde{}
\end{enumerate}

\subsection{Q\&A}\label{qa}

\subsubsection{使ç''¨è¿™ä¸ªæ¨¡æ?¿éœ€è¦?了解些什么?}\label{uxe4uxbduxe7uxe8uxe4uxaauxe6uxe6uxe9ux153uxe8uxe4uxbauxe8uxe4uxbauxe4uxe4uxb9ux2c6uxefuxbcuxff}

需�掌�一些 Markdown Like
æ~‡è®°ç''¨æ?¥ç¼--写æ--‡æ¡£ï¼Œäº†è§£æ--‡ç«~大致ç»``æž„å?³å?¯ã€?è§?
\texttt{\ template\textbackslash{}thesis.typ\ } 中介ç»?ã€`。

\textbf{å?‚考资æ--™ï¼š}

\begin{itemize}
\tightlist
\item
  官ç½`Tutorial:
  \href{https://typst.app/docs/tutorial/writing-in-typst/}{Writing in
  Typst â€`` Typst Documentation} ã€?
  \href{https://typst-doc-cn.github.io/docs/tutorial/writing-in-typst/}{Tutorial中æ--‡ç¿»è¯`}
\item
  Typst 语法官ç½`æ--‡æ¡£ï¼š
  \href{https://typst.app/docs/reference/syntax/}{Syntax â€`` Typst
  Documentation} �
  \href{https://typst-doc-cn.github.io/docs/reference/syntax/}{语法中æ--‡ç¿»è¯`}
\item
  中æ--‡ç¤¾åŒºå°?è``?书:
  \href{https://typst-doc-cn.github.io/tutorial/basic/writing-markup.html}{The
  Raindrop-Blue Book (Typst中æ--‡æ•™ç¨‹)}
\end{itemize}

\subsubsection{æˆ`ä¸?会代ç~?ã€?ä¸?会 LaTeX
å?¯ä»¥ä½¿ç''¨å?---?从接触到使ç''¨éœ€è¦?多ä¹\ldots ?}\label{uxe6ux2c6uxe4uxe4uxbcux161uxe4uxe7-uxe3uxe4uxe4uxbcux161-latex-uxe5uxe4uxe4uxbduxe7uxe5uxefuxbcuxffuxe4ux17euxe6ux17euxe8uxe5ux2c6uxe4uxbduxe7uxe9ux153uxe8uxe5ux161uxe4uxb9uxefuxbcuxff}

å?¯ä»¥çš„。å›~为æ--‡æ¡£æ~·å¼?该模æ?¿å·²ç»?æ??供,Typst
有æ~‡è®°æ¨¡å¼?(语法ç³--),使ç''¨èµ·æ?¥å°±ç±»ä¼¼äºŽ
Markdown,完å\ldots¨ä¸?需è¦?较多的代ç~?功底。

如果有 Markdown 基础,基本上�以直接上手\textasciitilde{}
如果没有,跳回第一个é---®é¢˜ï¼ŒæŸ¥çœ‹ç›¸å\ldots³è¯´æ˜Žã€‚

\subsubsection{Typst 是个啥玩�?相较于 LaTeX
有啥优势?}\label{typst-uxe6uxe4uxaauxe5uxe7ux17euxe6uxefuxbcuxffuxe7uxe8uxbeux192uxe4uxbaux17e-latex-uxe6ux153uxe5uxe4uxbcuxe5ux161uxefuxbcuxff}

\textbf{æ??供两篇} 写的很ç''¨å¿ƒçš„ \textbf{æ--‡ç«~:}

\begin{itemize}
\tightlist
\item
  \href{https://mp.weixin.qq.com/s/58IYHA3pROuh4iDHB4o1Vw}{探索
  Typst,一ç§?类似于 LaTeX çš„æ--°æŽ'版系统} (è¯`æ--‡ï¼‰ã€?
  \href{https://blog.jreyesr.com/posts/typst/}{原æ--‡}
\item
  \href{https://zhuanlan.zhihu.com/p/669097092}{Typst
  中æ--‡ç''¨æˆ·ä½¿ç''¨ä½``验 - OrangeX4大佬}
\end{itemize}

\subsection{致谢}\label{uxe8uxe8}

\begin{itemize}
\tightlist
\item
  æ„Ÿè°¢
  \href{https://github.com/nju-lug/modern-nju-thesis}{modern-nju-thesis}
  � \href{https://github.com/sysu/better-thesis}{better-thesis} �
  \href{https://github.com/hitszosa/universal-hit-thesis}{HIT-Thesis-Typst}
  为本模æ?¿æ??供了项目实现æ€?路。
\item
  æ„Ÿè°¢ \href{https://github.com/Timozer/CUGThesis}{Timozer/CUGThesis:
  中国地质大学(武汉)ç~''究ç''Ÿå­¦ä½?论æ--‡ TeX 模æ?¿}
  æ??供了页é?¢å¸ƒå±€ä¾?æ?®ã€‚
\item
  æ„Ÿè°¢ \href{https://jq.qq.com/?_wv=1027&k=m58va1kd}{Typst
  é?žå®˜æ--¹ä¸­æ--‡äº¤æµ?群} 中大佬的热心帮助。
\end{itemize}

\subsection{License}\label{license}

This project is licensed under the MIT License.

\href{/app?template=modern-cug-thesis&version=0.1.0}{Create project in
app}

\subsubsection{How to use}\label{how-to-use}

Click the button above to create a new project using this template in
the Typst app.

You can also use the Typst CLI to start a new project on your computer
using this command:

\begin{verbatim}
typst init @preview/modern-cug-thesis:0.1.0
\end{verbatim}

\includesvg[width=0.16667in,height=0.16667in]{/assets/icons/16-copy.svg}

\subsubsection{About}\label{about}

\begin{description}
\tightlist
\item[Author :]
Brevin
\item[License:]
MIT
\item[Current version:]
0.1.0
\item[Last updated:]
November 28, 2024
\item[First released:]
November 28, 2024
\item[Archive size:]
226 kB
\href{https://packages.typst.org/preview/modern-cug-thesis-0.1.0.tar.gz}{\pandocbounded{\includesvg[keepaspectratio]{/assets/icons/16-download.svg}}}
\item[Repository:]
\href{https://github.com/Rsweater/cug-thesis-typst}{GitHub}
\item[Categor y :]
\begin{itemize}
\tightlist
\item[]
\item
  \pandocbounded{\includesvg[keepaspectratio]{/assets/icons/16-mortarboard.svg}}
  \href{https://typst.app/universe/search/?category=thesis}{Thesis}
\end{itemize}
\end{description}

\subsubsection{Where to report issues?}\label{where-to-report-issues}

This template is a project of Brevin . Report issues on
\href{https://github.com/Rsweater/cug-thesis-typst}{their repository} .
You can also try to ask for help with this template on the
\href{https://forum.typst.app}{Forum} .

Please report this template to the Typst team using the
\href{https://typst.app/contact}{contact form} if you believe it is a
safety hazard or infringes upon your rights.

\phantomsection\label{versions}
\subsubsection{Version history}\label{version-history}

\begin{longtable}[]{@{}ll@{}}
\toprule\noalign{}
Version & Release Date \\
\midrule\noalign{}
\endhead
\bottomrule\noalign{}
\endlastfoot
0.1.0 & November 28, 2024 \\
\end{longtable}

Typst GmbH did not create this template and cannot guarantee correct
functionality of this template or compatibility with any version of the
Typst compiler or app.


\section{Package List LaTeX/modern-resume.tex}
\title{typst.app/universe/package/modern-resume}

\phantomsection\label{banner}
\phantomsection\label{template-thumbnail}
\pandocbounded{\includegraphics[keepaspectratio]{https://packages.typst.org/preview/thumbnails/modern-resume-0.1.0-small.webp}}

\section{modern-resume}\label{modern-resume}

{ 0.1.0 }

A modern resume/CV template.

\href{/app?template=modern-resume&version=0.1.0}{Create project in app}

\phantomsection\label{readme}
\href{https://github.com/peterpf/modern-typst-resume/stargazers}{\pandocbounded{\includesvg[keepaspectratio]{https://img.shields.io/badge/Say\%20Thanks-\%F0\%9F\%91\%8D-1EAEDB.svg}}}

A customizable resume/CV template focusing on clean and concise
presentation, with a touch of color.

\subsection{Requirements}\label{requirements}

To compile this project you need the following:

\begin{itemize}
\tightlist
\item
  Typst
\item
  Roboto font family
\end{itemize}

\subsection{Compiling}\label{compiling}

Build the document once with

\begin{Shaded}
\begin{Highlighting}[]
\ExtensionTok{typst}\NormalTok{ compile main.typ}
\end{Highlighting}
\end{Shaded}

Build the document whenever you save changes by running

\begin{Shaded}
\begin{Highlighting}[]
\ExtensionTok{typst}\NormalTok{ watch main.typ}
\end{Highlighting}
\end{Shaded}

\subsection{Usage}\label{usage}

The following code provides a minimum working example:

\begin{Shaded}
\begin{Highlighting}[]
\NormalTok{\#import "@preview/modern{-}resume": *}

\NormalTok{\#show: modern{-}resume.with(}
\NormalTok{  author: "John Doe",           // Optional parameter}
\NormalTok{  job{-}title: "Data Scientist",  // Optional parameter}
\NormalTok{  bio: lorem(5),                // Optional parameter}
\NormalTok{  avatar: image("avatar.png"),  // Optional parameter}
\NormalTok{  contact{-}options: (            // All entries are optional}
\NormalTok{    email: link("mailto:john.doe@gmail.com")[john.doe\textbackslash{}@gmail.com],}
\NormalTok{    mobile: "+43 1234 5678",}
\NormalTok{    location: "Austria",}
\NormalTok{    linkedin: link("https://www.linkedin.com/in/jdoe")[linkedin/jdoe],}
\NormalTok{    github: link("https://github.com/jdoe")[github.com/jdoe],}
\NormalTok{    website: link("https://jdoe.dev")[jdoe.dev],}
\NormalTok{  ),}
\NormalTok{)}

\NormalTok{== Education}

\NormalTok{\#experience{-}edu(}
\NormalTok{  title: "Master\textquotesingle{}s degree",}
\NormalTok{  subtitle: "University of Sciences",}
\NormalTok{  task{-}description: [}
\NormalTok{    {-} Short summary of the most important courses}
\NormalTok{    {-} Explanation of master thesis topic}
\NormalTok{  ],}
\NormalTok{  date{-}from: "10/2021",}
\NormalTok{  date{-}to: "07/2023",}
\NormalTok{)}

\NormalTok{// More content goes here}
\end{Highlighting}
\end{Shaded}

See
\href{https://github.com/typst/packages/raw/main/packages/preview/modern-resume/0.1.0/main.typ}{main.typ}
for a full example that showcases all available elements.

\subsection{Output examples}\label{output-examples}

Example outputs for different color palettes:

\begin{longtable}[]{@{}cc@{}}
\toprule\noalign{}
Default colors & Pink colors \\
\midrule\noalign{}
\endhead
\bottomrule\noalign{}
\endlastfoot
\pandocbounded{\includegraphics[keepaspectratio]{https://github.com/typst/packages/raw/main/packages/preview/modern-resume/0.1.0/docs/images/demo-navy-dark.png}}
&
\pandocbounded{\includegraphics[keepaspectratio]{https://github.com/typst/packages/raw/main/packages/preview/modern-resume/0.1.0/docs/images/demo-pink.png}} \\
\end{longtable}

\subsection{Customization}\label{customization}

Note: customization is currently only supported when cloning the
template locally. Allowing customization via a “Typst
universe�-installed template is a feature that is actively worked on.

The template allows you to make it yours by defining a custom color
palette. Customize the color theme by changing the values of the
\texttt{\ color\ } dictionary in
\href{https://github.com/typst/packages/raw/main/packages/preview/modern-resume/0.1.0/lib.typ}{lib.typ}
. For example:

\begin{itemize}
\item
  The default color palette:

\begin{Shaded}
\begin{Highlighting}[]
\NormalTok{\#let colors = (}
\NormalTok{  primary: rgb("\#313C4E"),}
\NormalTok{  secondary: rgb("\#222A33"),}
\NormalTok{  accent{-}color: rgb("\#449399"),}
\NormalTok{  text{-}primary: black,}
\NormalTok{  text{-}secondary: rgb("\#7C7C7C"),}
\NormalTok{  text{-}tertiary: white,}
\NormalTok{)}
\end{Highlighting}
\end{Shaded}
\item
  A pink color palette:

\begin{Shaded}
\begin{Highlighting}[]
\NormalTok{\#let colors = (}
\NormalTok{  primary: rgb("\#e755e0"),}
\NormalTok{  secondary: rgb("\#ad00c2"),}
\NormalTok{  accent{-}color: rgb("\#00d032"),}
\NormalTok{  text{-}primary: black,}
\NormalTok{  text{-}secondary: rgb("\#7C7C7C"),}
\NormalTok{  text{-}tertiary: white,}
\NormalTok{)}
\end{Highlighting}
\end{Shaded}
\end{itemize}

\subsection{Elements}\label{elements}

This section introduces the visual elements that are part of this
template.

\subsubsection{Pills}\label{pills}

Import this element from the template module with \texttt{\ pill\ } .

\pandocbounded{\includegraphics[keepaspectratio]{https://github.com/typst/packages/raw/main/packages/preview/modern-resume/0.1.0/docs/images/pills.png}}

\begin{Shaded}
\begin{Highlighting}[]
\NormalTok{\#pill("German (native)")}
\NormalTok{\#pill("English (C1)")}
\end{Highlighting}
\end{Shaded}

\pandocbounded{\includegraphics[keepaspectratio]{https://github.com/typst/packages/raw/main/packages/preview/modern-resume/0.1.0/docs/images/pills-filled.png}}

\begin{Shaded}
\begin{Highlighting}[]
\NormalTok{\#pill("Teamwork", fill: true)}
\NormalTok{\#pill("Critical thinking", fill: true)}
\end{Highlighting}
\end{Shaded}

\subsubsection{Educational/work
experience}\label{educationalwork-experience}

Import the elements from the template module with
\texttt{\ experience-edu\ } and \texttt{\ experience-work\ }
respectively.

\pandocbounded{\includegraphics[keepaspectratio]{https://github.com/typst/packages/raw/main/packages/preview/modern-resume/0.1.0/docs/images/educational-experience.png}}

\begin{Shaded}
\begin{Highlighting}[]
\NormalTok{\#experience{-}edu(}
\NormalTok{  title: "Master\textquotesingle{}s degree",}
\NormalTok{  subtitle: "University of Sciences",}
\NormalTok{  task{-}description: [}
\NormalTok{    {-} Short summary of the most important courses}
\NormalTok{    {-} Explanation of master thesis topic}
\NormalTok{  ],}
\NormalTok{  date{-}from: "10/2021",}
\NormalTok{  date{-}to: "07/2023",}
\NormalTok{)}
\end{Highlighting}
\end{Shaded}

\pandocbounded{\includegraphics[keepaspectratio]{https://github.com/typst/packages/raw/main/packages/preview/modern-resume/0.1.0/docs/images/work-experience.png}}

\begin{Shaded}
\begin{Highlighting}[]
\NormalTok{\#experience{-}work(}
\NormalTok{  title: "Full Stack Software Engineer",}
\NormalTok{  subtitle: [\#link("https://www.google.com")[Some IT Company]],}
\NormalTok{  facility{-}description: "Company operating in sector XY",}
\NormalTok{  task{-}description: [}
\NormalTok{    {-} Short summary of your responsibilities}
\NormalTok{  ],}
\NormalTok{  date{-}from: "09/2018",}
\NormalTok{  date{-}to: "07/2021",}
\NormalTok{)}
\end{Highlighting}
\end{Shaded}

\subsubsection{Project}\label{project}

Import this element from the template module with \texttt{\ project\ } .

\pandocbounded{\includegraphics[keepaspectratio]{https://github.com/typst/packages/raw/main/packages/preview/modern-resume/0.1.0/docs/images/project.png}}

\begin{Shaded}
\begin{Highlighting}[]
\NormalTok{\#project(}
\NormalTok{  title: "Project 2",}
\NormalTok{  subtitle: "Data Visualization, Data Engineering",}
\NormalTok{  description: [}
\NormalTok{    {-} \#lorem(20)}
\NormalTok{  ],}
\NormalTok{  date{-}from: "08/2022",}
\NormalTok{  date{-}to: "09/2022",}
\NormalTok{)}
\end{Highlighting}
\end{Shaded}

\subsection{Contributing}\label{contributing}

I’m grateful for any improvements and suggestions.

\subsection{Acknowledgements}\label{acknowledgements}

This project would not be what it is without:

\begin{itemize}
\tightlist
\item
  \href{https://github.com/FortAwesome/Font-Awesome/}{Font Awesome Free}
  \textbar{} providing the icons
\end{itemize}

\href{/app?template=modern-resume&version=0.1.0}{Create project in app}

\subsubsection{How to use}\label{how-to-use}

Click the button above to create a new project using this template in
the Typst app.

You can also use the Typst CLI to start a new project on your computer
using this command:

\begin{verbatim}
typst init @preview/modern-resume:0.1.0
\end{verbatim}

\includesvg[width=0.16667in,height=0.16667in]{/assets/icons/16-copy.svg}

\subsubsection{About}\label{about}

\begin{description}
\tightlist
\item[Author :]
\href{https://github.com/peterpf}{Peter Egger}
\item[License:]
Unlicense
\item[Current version:]
0.1.0
\item[Last updated:]
June 13, 2024
\item[First released:]
June 13, 2024
\item[Minimum Typst version:]
0.10.0
\item[Archive size:]
1.26 MB
\href{https://packages.typst.org/preview/modern-resume-0.1.0.tar.gz}{\pandocbounded{\includesvg[keepaspectratio]{/assets/icons/16-download.svg}}}
\item[Repository:]
\href{https://github.com/peterpf/modern-typst-resume}{GitHub}
\item[Categor y :]
\begin{itemize}
\tightlist
\item[]
\item
  \pandocbounded{\includesvg[keepaspectratio]{/assets/icons/16-user.svg}}
  \href{https://typst.app/universe/search/?category=cv}{CV}
\end{itemize}
\end{description}

\subsubsection{Where to report issues?}\label{where-to-report-issues}

This template is a project of Peter Egger . Report issues on
\href{https://github.com/peterpf/modern-typst-resume}{their repository}
. You can also try to ask for help with this template on the
\href{https://forum.typst.app}{Forum} .

Please report this template to the Typst team using the
\href{https://typst.app/contact}{contact form} if you believe it is a
safety hazard or infringes upon your rights.

\phantomsection\label{versions}
\subsubsection{Version history}\label{version-history}

\begin{longtable}[]{@{}ll@{}}
\toprule\noalign{}
Version & Release Date \\
\midrule\noalign{}
\endhead
\bottomrule\noalign{}
\endlastfoot
0.1.0 & June 13, 2024 \\
\end{longtable}

Typst GmbH did not create this template and cannot guarantee correct
functionality of this template or compatibility with any version of the
Typst compiler or app.


\section{Package List LaTeX/minimal-presentation.tex}
\title{typst.app/universe/package/minimal-presentation}

\phantomsection\label{banner}
\phantomsection\label{template-thumbnail}
\pandocbounded{\includegraphics[keepaspectratio]{https://packages.typst.org/preview/thumbnails/minimal-presentation-0.3.0-small.webp}}

\section{minimal-presentation}\label{minimal-presentation}

{ 0.3.0 }

A modern minimalistic presentation template ready to use

\href{/app?template=minimal-presentation&version=0.3.0}{Create project
in app}

\phantomsection\label{readme}
A modern minimalistic presentation template ready to use.

\subsection{Usage}\label{usage}

You can use this template in the Typst web app by clicking “Start from
template� on the dashboard and searching for
\texttt{\ minimal-presentation\ } .

Alternatively, you can use the CLI to kick this project off using the
command

\begin{verbatim}
typst init @preview/minimal-presentation
\end{verbatim}

Typst will create a new directory with all the files needed to get you
started.

\subsection{Configuration}\label{configuration}

This template exports the \texttt{\ project\ } function with the
following named arguments:

\begin{itemize}
\tightlist
\item
  \texttt{\ title\ } : The book’s title as content.
\item
  \texttt{\ sub-title\ } : The book’s subtitle as content.
\item
  \texttt{\ author\ } : Content or an array of content to specify the
  author.
\item
  \texttt{\ aspect-ratio\ } : Defaults to \texttt{\ 16-9\ } . Can be
  also \texttt{\ 4-3\ } .
\end{itemize}

The function also accepts a single, positional argument for the body of
the book.

The template will initialize your package with a sample call to the
\texttt{\ project\ } function in a show rule. If you, however, want to
change an existing project to use this template, you can add a show rule
like this at the top of your file:

\begin{Shaded}
\begin{Highlighting}[]
\NormalTok{\#import "@preview/minimal{-}presentation:0.1.0": *}

\NormalTok{\#set text(font: "Lato")}
\NormalTok{\#show math.equation: set text(font: "Lato Math")}
\NormalTok{\#show raw: set text(font: "Fira Code")}

\NormalTok{\#show: project.with(}
\NormalTok{  title: "Minimalist presentation template",}
\NormalTok{  sub{-}title: "This is where your presentation begins",}
\NormalTok{  author: "Flavio Barisi",}
\NormalTok{  date: "10/08/2023",}
\NormalTok{  index{-}title: "Contents",}
\NormalTok{  logo: image("./logo.svg"),}
\NormalTok{  logo{-}light: image("./logo\_light.svg"),}
\NormalTok{  cover: image("./image\_3.jpg")}
\NormalTok{)}

\NormalTok{= This is a section}

\NormalTok{== This is a slide title}

\NormalTok{\#lorem(10)}

\NormalTok{{-} \#lorem(10)}
\NormalTok{  {-} \#lorem(10)}
\NormalTok{  {-} \#lorem(10)}
\NormalTok{  {-} \#lorem(10)}

\NormalTok{== One column image}

\NormalTok{\#figure(}
\NormalTok{  image("image\_1.jpg", height: 10.5cm),}
\NormalTok{  caption: [An image],}
\NormalTok{) \textless{}image\_label\textgreater{}}

\NormalTok{== Two columns image}

\NormalTok{\#columns{-}content()[}
\NormalTok{  \#figure(}
\NormalTok{    image("image\_1.jpg", width: 100\%),}
\NormalTok{    caption: [An image],}
\NormalTok{  ) \textless{}image\_label\_1\textgreater{}}
\NormalTok{][}
\NormalTok{  \#figure(}
\NormalTok{    image("image\_1.jpg", width: 100\%),}
\NormalTok{    caption: [An image],}
\NormalTok{  ) \textless{}image\_label\_2\textgreater{}}
\NormalTok{]}

\NormalTok{== Two columns}

\NormalTok{\#columns{-}content()[}
\NormalTok{  {-} \#lorem(10)}
\NormalTok{  {-} \#lorem(10)}
\NormalTok{  {-} \#lorem(10)}
\NormalTok{][}
\NormalTok{  \#figure(}
\NormalTok{    image("image\_3.jpg", width: 100\%),}
\NormalTok{    caption: [An image],}
\NormalTok{  ) \textless{}image\_label\_3\textgreater{}}
\NormalTok{]}

\NormalTok{= This is a section}

\NormalTok{== This is a slide title}

\NormalTok{\#lorem(10)}

\NormalTok{= This is a section}

\NormalTok{== This is a slide title}

\NormalTok{\#lorem(10)}

\NormalTok{= This is a section}

\NormalTok{== This is a slide title}

\NormalTok{\#lorem(10)}

\NormalTok{= This is a very v v v v v v v v v v v v v v v v v v v v  long section}

\NormalTok{== This is a very v v v v v v v v v v v v v v v v v v v v  long slide title}

\NormalTok{= sub{-}title test}

\NormalTok{== Slide title}

\NormalTok{\#lorem(50)}

\NormalTok{=== Slide sub{-}title 1}

\NormalTok{\#lorem(50)}

\NormalTok{=== Slide sub{-}title 2}

\NormalTok{\#lorem(50)}

\end{Highlighting}
\end{Shaded}

\subsection{Fonts}\label{fonts}

You can use the font selected by the author of this plugin, by download
theme at the following link:

\url{https://github.com/flavio20002/typst-presentation-minimal-template/tree/main/fonts}

You can then import thme in your system, import them in the typst web
app or just put them in a folder and launch the compilation with the
following argoument:

\begin{verbatim}
typst watch main.typ --root . --font-path fonts
\end{verbatim}

\href{/app?template=minimal-presentation&version=0.3.0}{Create project
in app}

\subsubsection{How to use}\label{how-to-use}

Click the button above to create a new project using this template in
the Typst app.

You can also use the Typst CLI to start a new project on your computer
using this command:

\begin{verbatim}
typst init @preview/minimal-presentation:0.3.0
\end{verbatim}

\includesvg[width=0.16667in,height=0.16667in]{/assets/icons/16-copy.svg}

\subsubsection{About}\label{about}

\begin{description}
\tightlist
\item[Author :]
Flavio Barisi
\item[License:]
MIT-0
\item[Current version:]
0.3.0
\item[Last updated:]
November 18, 2024
\item[First released:]
September 2, 2024
\item[Minimum Typst version:]
0.12.0
\item[Archive size:]
755 kB
\href{https://packages.typst.org/preview/minimal-presentation-0.3.0.tar.gz}{\pandocbounded{\includesvg[keepaspectratio]{/assets/icons/16-download.svg}}}
\item[Repository:]
\href{https://github.com/flavio20002/typst-presentation-minimal-template}{GitHub}
\item[Categor y :]
\begin{itemize}
\tightlist
\item[]
\item
  \pandocbounded{\includesvg[keepaspectratio]{/assets/icons/16-presentation.svg}}
  \href{https://typst.app/universe/search/?category=presentation}{Presentation}
\end{itemize}
\end{description}

\subsubsection{Where to report issues?}\label{where-to-report-issues}

This template is a project of Flavio Barisi . Report issues on
\href{https://github.com/flavio20002/typst-presentation-minimal-template}{their
repository} . You can also try to ask for help with this template on the
\href{https://forum.typst.app}{Forum} .

Please report this template to the Typst team using the
\href{https://typst.app/contact}{contact form} if you believe it is a
safety hazard or infringes upon your rights.

\phantomsection\label{versions}
\subsubsection{Version history}\label{version-history}

\begin{longtable}[]{@{}ll@{}}
\toprule\noalign{}
Version & Release Date \\
\midrule\noalign{}
\endhead
\bottomrule\noalign{}
\endlastfoot
0.3.0 & November 18, 2024 \\
\href{https://typst.app/universe/package/minimal-presentation/0.2.0/}{0.2.0}
& October 23, 2024 \\
\href{https://typst.app/universe/package/minimal-presentation/0.1.0/}{0.1.0}
& September 2, 2024 \\
\end{longtable}

Typst GmbH did not create this template and cannot guarantee correct
functionality of this template or compatibility with any version of the
Typst compiler or app.


\section{Package List LaTeX/quetta.tex}
\title{typst.app/universe/package/quetta}

\phantomsection\label{banner}
\section{quetta}\label{quetta}

{ 0.2.0 }

Write Tengwar easily with Typst.

\phantomsection\label{readme}
A simple module to write
\href{https://en.wikipedia.org/wiki/Tengwar}{tengwar} in
\href{https://typst.app/}{Typst} .

\subsection{Requirements}\label{requirements}

\begin{itemize}
\tightlist
\item
  \href{https://github.com/typst/typst}{Typst} version 0.11.0 or 0.11.1
\item
  The
  \href{https://www.fontspace.com/tengwar-annatar-font-f2244}{Tengwar
  Annatar} fonts version 1.20
\end{itemize}

To use this module with the \href{https://typst.app/}{Typst web app} ,
you need to upload the font files to your project.

\subsection{Usage}\label{usage}

The main functionality of this module is provided by functions taking
content and converting all text in Tenwar:

\begin{itemize}
\tightlist
\item
  \texttt{\ quenya\ } converts text using the mode of Quenya,
\item
  \texttt{\ gondor\ } converts text using the Sindarin mode of Gondor.
\end{itemize}

The original text is used as a phonetic transcription. (This module does
not translate English into Quenya or Sindarin.) See the
\href{https://github.com/FlorentCLMichel/quetta/blob/main/manual.pdf}{manual}
for more information.

The following line may be used to convert the whole document below to
Tengwar in Quenya mode (other \texttt{\ show\ } rules might interfere
with it):

\begin{verbatim}
#show: quetta.quenya
\end{verbatim}

\textbf{Example:}

\begin{verbatim}
#import "@preview/quetta:0.2.0"

// Use the function `quenya` to write a small amount of text in Tengwar (Quenya mode)
#text(size: 16pt, 
      fill: gradient.linear(blue, green)
     )[#box(quetta.quenya[_tengwar_])]

#v(1em)

// A `show` rule may be more convenient for larger contents; beware that it may interfere with other ones, though
#show: quetta.quenya

Namárië!

#h(1em) _Namárië!_

#h(2em) *Namárië!*
\end{verbatim}

\subsection{Roadmap}\label{roadmap}

\begin{itemize}
\tightlist
\item
  Number conversion: done
\item
  Support for the Quenya mode: done
\item
  Support for the mode of Gondor: done
\item
  Support for the mode of Beleriand: backlog
\item
  Support for the Black Speech: backlog
\end{itemize}

\subsection{Changelog}\label{changelog}

\subsubsection{v0.2.0}\label{v0.2.0}

\begin{itemize}
\tightlist
\item
  Add support for Sindarinâ€''Mode of Gondor
\item
  \textbf{Breaking change:} The symbol used to prevent combination was
  changed from \texttt{\ :\ } to \texttt{\ \textbar{}\ } .
\item
  Small changes to the kerning between several tengwar and to tehtar
  positions.
\end{itemize}

\subsubsection{v0.1.0}\label{v0.1.0}

Initial release with Quenya support.

\subsection{How can I contribute?}\label{how-can-i-contribute}

I (the original author) am definitely not en expert in either Typst nor
Tengwar. I could thus use some help in all areas. I would especially
welcome contributions or suggestions on the following:

\begin{itemize}
\tightlist
\item
  Identify and resolve inefficiencies in the Typst code.
\item
  Identify cases where the result differs from the expected one. (In
  particular, there are probably rules for writing in Tengwar that I
  either am not aware of or have not properly understood. Any advice on
  that is warmly welcome!)
\item
  References on Tengar, Quenya, and Sindarin.
\item
  Support for other Tengwar fonts.
\end{itemize}

\subsubsection{How to add}\label{how-to-add}

Copy this into your project and use the import as \texttt{\ quetta\ }

\begin{verbatim}
#import "@preview/quetta:0.2.0"
\end{verbatim}

\includesvg[width=0.16667in,height=0.16667in]{/assets/icons/16-copy.svg}

Check the docs for
\href{https://typst.app/docs/reference/scripting/\#packages}{more
information on how to import packages} .

\subsubsection{About}\label{about}

\begin{description}
\tightlist
\item[Author :]
\href{https://github.com/FlorentCLMichel}{Florent Michel}
\item[License:]
MIT
\item[Current version:]
0.2.0
\item[Last updated:]
September 24, 2024
\item[First released:]
July 31, 2024
\item[Minimum Typst version:]
0.11.0
\item[Archive size:]
8.96 kB
\href{https://packages.typst.org/preview/quetta-0.2.0.tar.gz}{\pandocbounded{\includesvg[keepaspectratio]{/assets/icons/16-download.svg}}}
\item[Repository:]
\href{https://github.com/FlorentCLMichel/quetta}{GitHub}
\item[Discipline s :]
\begin{itemize}
\tightlist
\item[]
\item
  \href{https://typst.app/universe/search/?discipline=linguistics}{Linguistics}
\item
  \href{https://typst.app/universe/search/?discipline=literature}{Literature}
\end{itemize}
\item[Categor ies :]
\begin{itemize}
\tightlist
\item[]
\item
  \pandocbounded{\includesvg[keepaspectratio]{/assets/icons/16-text.svg}}
  \href{https://typst.app/universe/search/?category=text}{Text}
\item
  \pandocbounded{\includesvg[keepaspectratio]{/assets/icons/16-world.svg}}
  \href{https://typst.app/universe/search/?category=languages}{Languages}
\item
  \pandocbounded{\includesvg[keepaspectratio]{/assets/icons/16-smile.svg}}
  \href{https://typst.app/universe/search/?category=fun}{Fun}
\end{itemize}
\end{description}

\subsubsection{Where to report issues?}\label{where-to-report-issues}

This package is a project of Florent Michel . Report issues on
\href{https://github.com/FlorentCLMichel/quetta}{their repository} . You
can also try to ask for help with this package on the
\href{https://forum.typst.app}{Forum} .

Please report this package to the Typst team using the
\href{https://typst.app/contact}{contact form} if you believe it is a
safety hazard or infringes upon your rights.

\phantomsection\label{versions}
\subsubsection{Version history}\label{version-history}

\begin{longtable}[]{@{}ll@{}}
\toprule\noalign{}
Version & Release Date \\
\midrule\noalign{}
\endhead
\bottomrule\noalign{}
\endlastfoot
0.2.0 & September 24, 2024 \\
\href{https://typst.app/universe/package/quetta/0.1.0/}{0.1.0} & July
31, 2024 \\
\end{longtable}

Typst GmbH did not create this package and cannot guarantee correct
functionality of this package or compatibility with any version of the
Typst compiler or app.


\section{Package List LaTeX/october.tex}
\title{typst.app/universe/package/october}

\phantomsection\label{banner}
\phantomsection\label{template-thumbnail}
\pandocbounded{\includegraphics[keepaspectratio]{https://packages.typst.org/preview/thumbnails/october-1.0.0-small.webp}}

\section{october}\label{october}

{ 1.0.0 }

Simple printable year calendar

\href{/app?template=october&version=1.0.0}{Create project in app}

\phantomsection\label{readme}
This template generates a monthly calendar, designed to be printed in
landscape.

The calendar function accepts one parameter for the year, which should
be formatted as an integer. Otherwise, the current year can be passed in
with \texttt{\ datetime.today().year()\ } .

\begin{Shaded}
\begin{Highlighting}[]
\NormalTok{ \#show: calendar.with(}
\NormalTok{  year: datetime.today().year()}
\NormalTok{)}
\end{Highlighting}
\end{Shaded}

There isn’t much space for writing in each day box, it’s more suited
to blocking out days with a highlighter. For example, to mark out free
days in a variable schedule of work shifts.

\href{/app?template=october&version=1.0.0}{Create project in app}

\subsubsection{How to use}\label{how-to-use}

Click the button above to create a new project using this template in
the Typst app.

You can also use the Typst CLI to start a new project on your computer
using this command:

\begin{verbatim}
typst init @preview/october:1.0.0
\end{verbatim}

\includesvg[width=0.16667in,height=0.16667in]{/assets/icons/16-copy.svg}

\subsubsection{About}\label{about}

\begin{description}
\tightlist
\item[Author :]
\href{mailto:pierre.marshall@gmail.com}{Pierre Marshall}
\item[License:]
MIT-0
\item[Current version:]
1.0.0
\item[Last updated:]
October 18, 2024
\item[First released:]
October 18, 2024
\item[Minimum Typst version:]
0.11.1
\item[Archive size:]
1.94 kB
\href{https://packages.typst.org/preview/october-1.0.0.tar.gz}{\pandocbounded{\includesvg[keepaspectratio]{/assets/icons/16-download.svg}}}
\item[Repository:]
\href{https://github.com/extua/october}{GitHub}
\item[Categor y :]
\begin{itemize}
\tightlist
\item[]
\item
  \pandocbounded{\includesvg[keepaspectratio]{/assets/icons/16-envelope.svg}}
  \href{https://typst.app/universe/search/?category=office}{Office}
\end{itemize}
\end{description}

\subsubsection{Where to report issues?}\label{where-to-report-issues}

This template is a project of Pierre Marshall . Report issues on
\href{https://github.com/extua/october}{their repository} . You can also
try to ask for help with this template on the
\href{https://forum.typst.app}{Forum} .

Please report this template to the Typst team using the
\href{https://typst.app/contact}{contact form} if you believe it is a
safety hazard or infringes upon your rights.

\phantomsection\label{versions}
\subsubsection{Version history}\label{version-history}

\begin{longtable}[]{@{}ll@{}}
\toprule\noalign{}
Version & Release Date \\
\midrule\noalign{}
\endhead
\bottomrule\noalign{}
\endlastfoot
1.0.0 & October 18, 2024 \\
\end{longtable}

Typst GmbH did not create this template and cannot guarantee correct
functionality of this template or compatibility with any version of the
Typst compiler or app.


\section{Package List LaTeX/upb-corporate-design-slides.tex}
\title{typst.app/universe/package/upb-corporate-design-slides}

\phantomsection\label{banner}
\phantomsection\label{template-thumbnail}
\pandocbounded{\includegraphics[keepaspectratio]{https://packages.typst.org/preview/thumbnails/upb-corporate-design-slides-0.1.1-small.webp}}

\section{upb-corporate-design-slides}\label{upb-corporate-design-slides}

{ 0.1.1 }

Presentation template for Paderborn University (UPB)

\href{/app?template=upb-corporate-design-slides&version=0.1.1}{Create
project in app}

\phantomsection\label{readme}
This template can be used to create presentations in
\href{https://typst.app/docs/}{Typst} with the corporate design of
\href{https://www.uni-paderborn.de/}{Paderborn University} using the
\href{https://touying-typ.github.io/}{Touying} slide engine.

\subsection{Usage}\label{usage}

Create a new typst project based on this template locally.

\begin{Shaded}
\begin{Highlighting}[]
\ExtensionTok{typst}\NormalTok{ init @preview/upb{-}corporate{-}design{-}slides}
\BuiltInTok{cd}\NormalTok{ upb{-}corporate{-}design{-}slides}
\end{Highlighting}
\end{Shaded}

Or create a project on the typst web app based on this template.

\subsubsection{Font setup}\label{font-setup}

The font \texttt{\ Karla\ } needs to be installed on your system. The
\href{https://www.uni-paderborn.de/universitaet/presse-kommunikation-marketing/brandportal}{UPB
brand portal} recommends to download it from google fonts.

\begin{itemize}
\tightlist
\item
  If you use arch linux, you can also get it from the
  \href{https://aur.archlinux.org/packages/ttf-karla}{AUR} .
\item
  If you use NixOS, karla is available in the \texttt{\ 24.11\ } and
  \texttt{\ unstable\ } channels.
\item
  If you use the typst web app, you will need to upload the font.
\end{itemize}

\href{/app?template=upb-corporate-design-slides&version=0.1.1}{Create
project in app}

\subsubsection{How to use}\label{how-to-use}

Click the button above to create a new project using this template in
the Typst app.

You can also use the Typst CLI to start a new project on your computer
using this command:

\begin{verbatim}
typst init @preview/upb-corporate-design-slides:0.1.1
\end{verbatim}

\includesvg[width=0.16667in,height=0.16667in]{/assets/icons/16-copy.svg}

\subsubsection{About}\label{about}

\begin{description}
\tightlist
\item[Author s :]
\href{https://kuchenmampfer.de/}{Tammes Burghard} \& Marvin Feiter
\item[License:]
MIT
\item[Current version:]
0.1.1
\item[Last updated:]
November 28, 2024
\item[First released:]
November 4, 2024
\item[Archive size:]
32.7 kB
\href{https://packages.typst.org/preview/upb-corporate-design-slides-0.1.1.tar.gz}{\pandocbounded{\includesvg[keepaspectratio]{/assets/icons/16-download.svg}}}
\item[Repository:]
\href{https://codeberg.org/Kuchenmampfer/upb-corporate-design-slides}{Codeberg}
\item[Categor y :]
\begin{itemize}
\tightlist
\item[]
\item
  \pandocbounded{\includesvg[keepaspectratio]{/assets/icons/16-presentation.svg}}
  \href{https://typst.app/universe/search/?category=presentation}{Presentation}
\end{itemize}
\end{description}

\subsubsection{Where to report issues?}\label{where-to-report-issues}

This template is a project of Tammes Burghard and Marvin Feiter . Report
issues on
\href{https://codeberg.org/Kuchenmampfer/upb-corporate-design-slides}{their
repository} . You can also try to ask for help with this template on the
\href{https://forum.typst.app}{Forum} .

Please report this template to the Typst team using the
\href{https://typst.app/contact}{contact form} if you believe it is a
safety hazard or infringes upon your rights.

\phantomsection\label{versions}
\subsubsection{Version history}\label{version-history}

\begin{longtable}[]{@{}ll@{}}
\toprule\noalign{}
Version & Release Date \\
\midrule\noalign{}
\endhead
\bottomrule\noalign{}
\endlastfoot
0.1.1 & November 28, 2024 \\
\href{https://typst.app/universe/package/upb-corporate-design-slides/0.1.0/}{0.1.0}
& November 4, 2024 \\
\end{longtable}

Typst GmbH did not create this template and cannot guarantee correct
functionality of this template or compatibility with any version of the
Typst compiler or app.


\section{Package List LaTeX/weave.tex}
\title{typst.app/universe/package/weave}

\phantomsection\label{banner}
\section{weave}\label{weave}

{ 0.2.0 }

A helper library for chaining lambda abstractions

\phantomsection\label{readme}
A helper library for chaining lambda abstractions, imitating the
\texttt{\ \textbar{}\textgreater{}\ } or \texttt{\ .\ } operator in some
functional languages.

The function \texttt{\ compose\ } is the \texttt{\ pipe\ } function in
the mathematical order. Functions suffixed with underscore have their
arguments flipped.

\subsection{Changelog}\label{changelog}

\begin{itemize}
\tightlist
\item
  0.2.0 Redesigned interface to work with typst’s \texttt{\ with\ }
  keyword.
\item
  0.1.0 Initial release
\end{itemize}

\subsection{Basic usage}\label{basic-usage}

It can help improve readability with nested applications to a content
value, or make the diff cleaner.

\begin{Shaded}
\begin{Highlighting}[]
\NormalTok{\#compose\_((}
\NormalTok{  text.with(blue),}
\NormalTok{  emph,}
\NormalTok{  strong,}
\NormalTok{  underline,}
\NormalTok{  strike,}
\NormalTok{))[This is a very long content with a lot of words]}
\NormalTok{// Is equivalent to}
\NormalTok{\#text(}
\NormalTok{  blue,}
\NormalTok{  emph(}
\NormalTok{    strong(}
\NormalTok{      underline(}
\NormalTok{        strike[This is a very long content with a lot of words]}
\NormalTok{      )}
\NormalTok{    )}
\NormalTok{  )}
\NormalTok{)}
\end{Highlighting}
\end{Shaded}

You can use it for show rules just like the example above.

\begin{Shaded}
\begin{Highlighting}[]
\NormalTok{\#show link: compose\_.with((}
\NormalTok{  text.with(fill: blue),}
\NormalTok{  emph,}
\NormalTok{  underline,}
\NormalTok{))}
\NormalTok{// These two are equivalent}
\NormalTok{\#show link: text.with(fill: blue)}
\NormalTok{\#show link: emph}
\NormalTok{\#show link: underline}
\end{Highlighting}
\end{Shaded}

This can also be useful when you need to destructure lists, as it allows
creating binds that are scoped by each lambda expression.

\begin{Shaded}
\begin{Highlighting}[]
\NormalTok{\#let two\_and\_one = pipe(}
\NormalTok{  (1, 2),}
\NormalTok{  (}
\NormalTok{    ((a, b)) =\textgreater{} (a, b, {-}1), // becomes a list of length three}
\NormalTok{    ((a, b, \_)) =\textgreater{} (b, a), // discard the third element and swap}
\NormalTok{  ),}
\NormalTok{)}
\end{Highlighting}
\end{Shaded}

\subsubsection{How to add}\label{how-to-add}

Copy this into your project and use the import as \texttt{\ weave\ }

\begin{verbatim}
#import "@preview/weave:0.2.0"
\end{verbatim}

\includesvg[width=0.16667in,height=0.16667in]{/assets/icons/16-copy.svg}

Check the docs for
\href{https://typst.app/docs/reference/scripting/\#packages}{more
information on how to import packages} .

\subsubsection{About}\label{about}

\begin{description}
\tightlist
\item[Author :]
\href{https://github.com/leana8959}{Léana 江}
\item[License:]
MIT
\item[Current version:]
0.2.0
\item[Last updated:]
October 21, 2024
\item[First released:]
October 21, 2024
\item[Archive size:]
1.92 kB
\href{https://packages.typst.org/preview/weave-0.2.0.tar.gz}{\pandocbounded{\includesvg[keepaspectratio]{/assets/icons/16-download.svg}}}
\item[Categor y :]
\begin{itemize}
\tightlist
\item[]
\item
  \pandocbounded{\includesvg[keepaspectratio]{/assets/icons/16-code.svg}}
  \href{https://typst.app/universe/search/?category=scripting}{Scripting}
\end{itemize}
\end{description}

\subsubsection{Where to report issues?}\label{where-to-report-issues}

This package is a project of Léana 江 . You can also try to ask for
help with this package on the \href{https://forum.typst.app}{Forum} .

Please report this package to the Typst team using the
\href{https://typst.app/contact}{contact form} if you believe it is a
safety hazard or infringes upon your rights.

\phantomsection\label{versions}
\subsubsection{Version history}\label{version-history}

\begin{longtable}[]{@{}ll@{}}
\toprule\noalign{}
Version & Release Date \\
\midrule\noalign{}
\endhead
\bottomrule\noalign{}
\endlastfoot
0.2.0 & October 21, 2024 \\
\href{https://typst.app/universe/package/weave/0.1.0/}{0.1.0} & October
21, 2024 \\
\end{longtable}

Typst GmbH did not create this package and cannot guarantee correct
functionality of this package or compatibility with any version of the
Typst compiler or app.


\section{Package List LaTeX/letter-pro.tex}
\title{typst.app/universe/package/letter-pro}

\phantomsection\label{banner}
\phantomsection\label{template-thumbnail}
\pandocbounded{\includegraphics[keepaspectratio]{https://packages.typst.org/preview/thumbnails/letter-pro-3.0.0-small.webp}}

\section{letter-pro}\label{letter-pro}

{ 3.0.0 }

DIN 5008 letter template for Typst.

\href{/app?template=letter-pro&version=3.0.0}{Create project in app}

\phantomsection\label{readme}
A template for creating business letters following the DIN 5008
standard.

\subsection{Overview}\label{overview}

typst-letter-pro provides a convenient and professional way to generate
business letters with a standardized layout. The template follows the
guidelines specified in the DIN 5008 standard, ensuring that your
letters adhere to the commonly accepted business communication
practices.

The goal of typst-letter-pro is to simplify the process of creating
business letters while maintaining a clean and professional appearance.
It offers predefined sections for the sender and recipient information,
subject, date, header, footer and more.

\subsection{\texorpdfstring{\href{https://raw.githubusercontent.com/wiki/Sematre/typst-letter-pro/documentation-v3.0.0.pdf}{Documentation}}{Documentation}}\label{documentation}

\subsection{Example}\label{example}

Text source:
\href{https://web.archive.org/web/20230927152049/https://www.deutschepost.de/de/b/briefvorlagen/beschwerden.html\#Einspruch}{Musterbrief
Widerspruch gegen Einkommensteuerbescheid}

\subsubsection{\texorpdfstring{Preview (
\href{https://raw.githubusercontent.com/wiki/Sematre/typst-letter-pro/simple_letter.pdf}{PDF
version} )}{Preview ( PDF version )}}\label{preview-pdf-version}

\pandocbounded{\includegraphics[keepaspectratio]{https://github.com/typst/packages/raw/main/packages/preview/letter-pro/3.0.0/template/thumbnail.png}}

\subsubsection{Code}\label{code}

\begin{Shaded}
\begin{Highlighting}[]
\NormalTok{\#import "@preview/letter{-}pro:3.0.0": letter{-}simple}

\NormalTok{\#set text(lang: "de")}

\NormalTok{\#show: letter{-}simple.with(}
\NormalTok{  sender: (}
\NormalTok{    name: "Anja Ahlsen",}
\NormalTok{    address: "Deutschherrenufer 28, 60528 Frankfurt",}
\NormalTok{    extra: [}
\NormalTok{      Telefon: \#link("tel:+4915228817386")[+49 152 28817386]\textbackslash{}}
\NormalTok{      E{-}Mail: \#link("mailto:aahlsen@example.com")[aahlsen\textbackslash{}@example.com]\textbackslash{}}
\NormalTok{    ],}
\NormalTok{  ),}
  
\NormalTok{  annotations: [Einschreiben {-} Rückschein],}
\NormalTok{  recipient: [}
\NormalTok{    Finanzamt Frankfurt\textbackslash{}}
\NormalTok{    Einkommenssteuerstelle\textbackslash{}}
\NormalTok{    Gutleutstraße 5\textbackslash{}}
\NormalTok{    60329 Frankfurt}
\NormalTok{  ],}
  
\NormalTok{  reference{-}signs: (}
\NormalTok{    ([Steuernummer], [333/24692/5775]),}
\NormalTok{  ),}
  
\NormalTok{  date: "12. November 2014",}
\NormalTok{  subject: "Einspruch gegen den ESt{-}Bescheid",}
\NormalTok{)}

\NormalTok{Sehr geehrte Damen und Herren,}

\NormalTok{die von mir bei den Werbekosten geltend gemachte Abschreibung für den im}
\NormalTok{vergangenen Jahr angeschafften Fotokopierer wurde von Ihnen nicht berücksichtigt.}
\NormalTok{Der Fotokopierer steht in meinem Büro und wird von mir ausschließlich zu beruflichen}
\NormalTok{Zwecken verwendet.}

\NormalTok{Ich lege deshalb Einspruch gegen den oben genannten Einkommensteuerbescheid ein}
\NormalTok{und bitte Sie, die Abschreibung anzuerkennen.}

\NormalTok{Anbei erhalten Sie eine Kopie der Rechnung des Gerätes.}

\NormalTok{Mit freundlichen Grüßen}
\NormalTok{\#v(1cm)}
\NormalTok{Anja Ahlsen}

\NormalTok{\#v(1fr)}
\NormalTok{*Anlagen:*}
\NormalTok{{-} Rechnung}
\end{Highlighting}
\end{Shaded}

\subsection{Usage}\label{usage}

\subsubsection{Preview repository}\label{preview-repository}

Import the package in your document:

\begin{Shaded}
\begin{Highlighting}[]
\NormalTok{\#import "@preview/letter{-}pro:3.0.0": letter{-}simple}
\end{Highlighting}
\end{Shaded}

\subsubsection{Local namespace}\label{local-namespace}

Download the repository to the local package namespace using Git:

\begin{Shaded}
\begin{Highlighting}[]
\ExtensionTok{$}\NormalTok{ git clone }\AttributeTok{{-}c}\NormalTok{ advice.detachedHead=false https://github.com/Sematre/typst{-}letter{-}pro.git }\AttributeTok{{-}{-}depth}\NormalTok{ 1 }\AttributeTok{{-}{-}branch}\NormalTok{ v3.0.0 \textasciitilde{}/.local/share/typst/packages/local/letter{-}pro/3.0.0}
\end{Highlighting}
\end{Shaded}

Then import the package in your document:

\begin{Shaded}
\begin{Highlighting}[]
\NormalTok{\#import "@local/letter{-}pro:3.0.0": letter{-}simple}
\end{Highlighting}
\end{Shaded}

\subsubsection{Manual}\label{manual}

Download the \texttt{\ letter-pro-v3.0.0.typ\ } file from the
\href{https://github.com/Sematre/typst-letter-pro/releases}{releases
page} and place it next to your document file, e.g., using \emph{wget} :

\begin{Shaded}
\begin{Highlighting}[]
\ExtensionTok{$}\NormalTok{ wget https://github.com/Sematre/typst{-}letter{-}pro/releases/download/v3.0.0/letter{-}pro{-}v3.0.0.typ}
\end{Highlighting}
\end{Shaded}

Then import the package in your document:

\begin{Shaded}
\begin{Highlighting}[]
\NormalTok{\#import "letter{-}pro{-}v3.0.0.typ": letter{-}simple}
\end{Highlighting}
\end{Shaded}

\subsection{Contributing}\label{contributing}

Contributions to typst-letter-pro are welcome! If you encounter any
issues or have suggestions for improvements, please open an issue on
GitHub or submit a pull request.

Before making any significant changes, please discuss your ideas with
the project maintainers to ensure they align with the project’s goals
and direction.

\subsection{Acknowledgments}\label{acknowledgments}

This project is inspired by the following projects and resources:

\begin{itemize}
\tightlist
\item
  \href{https://de.wikipedia.org/wiki/DIN_5008}{Wikipedia / DIN 5008}
\item
  \href{https://web.archive.org/web/20240223035339/https://www.deutschepost.de/de/b/briefvorlagen/normbrief-din-5008-vorlage.html}{Deutsche
  Post / DIN 5008 Vorlage}
\item
  \href{https://www.deutschepost.de/dam/dpag/images/P_p/printmailing/downloads/dp-automationsfaehige-briefsendungen-2024.pdf}{Deutsche
  Post / Automationsfähige Briefsendungen}
\item
  \href{https://www.edv-lehrgang.de/din-5008/}{EDV Lehrgang / DIN-5008}
\item
  \href{https://github.com/ludwig-austermann/typst-din-5008-letter}{Ludwig
  Austermann / typst-din-5008-letter}
\item
  \href{https://github.com/pascal-huber/typst-letter-template}{Pascal
  Huber / typst-letter-template}
\end{itemize}

\subsection{License}\label{license}

Distributed under the \textbf{MIT License} . See \texttt{\ LICENSE\ }
for more information.

\href{/app?template=letter-pro&version=3.0.0}{Create project in app}

\subsubsection{How to use}\label{how-to-use}

Click the button above to create a new project using this template in
the Typst app.

You can also use the Typst CLI to start a new project on your computer
using this command:

\begin{verbatim}
typst init @preview/letter-pro:3.0.0
\end{verbatim}

\includesvg[width=0.16667in,height=0.16667in]{/assets/icons/16-copy.svg}

\subsubsection{About}\label{about}

\begin{description}
\tightlist
\item[Author :]
Sematre
\item[License:]
MIT
\item[Current version:]
3.0.0
\item[Last updated:]
October 28, 2024
\item[First released:]
April 2, 2024
\item[Archive size:]
7.08 kB
\href{https://packages.typst.org/preview/letter-pro-3.0.0.tar.gz}{\pandocbounded{\includesvg[keepaspectratio]{/assets/icons/16-download.svg}}}
\item[Repository:]
\href{https://github.com/Sematre/typst-letter-pro}{GitHub}
\item[Categor y :]
\begin{itemize}
\tightlist
\item[]
\item
  \pandocbounded{\includesvg[keepaspectratio]{/assets/icons/16-envelope.svg}}
  \href{https://typst.app/universe/search/?category=office}{Office}
\end{itemize}
\end{description}

\subsubsection{Where to report issues?}\label{where-to-report-issues}

This template is a project of Sematre . Report issues on
\href{https://github.com/Sematre/typst-letter-pro}{their repository} .
You can also try to ask for help with this template on the
\href{https://forum.typst.app}{Forum} .

Please report this template to the Typst team using the
\href{https://typst.app/contact}{contact form} if you believe it is a
safety hazard or infringes upon your rights.

\phantomsection\label{versions}
\subsubsection{Version history}\label{version-history}

\begin{longtable}[]{@{}ll@{}}
\toprule\noalign{}
Version & Release Date \\
\midrule\noalign{}
\endhead
\bottomrule\noalign{}
\endlastfoot
3.0.0 & October 28, 2024 \\
\href{https://typst.app/universe/package/letter-pro/2.1.0/}{2.1.0} &
April 2, 2024 \\
\end{longtable}

Typst GmbH did not create this template and cannot guarantee correct
functionality of this template or compatibility with any version of the
Typst compiler or app.


\section{Package List LaTeX/ichigo.tex}
\title{typst.app/universe/package/ichigo}

\phantomsection\label{banner}
\phantomsection\label{template-thumbnail}
\pandocbounded{\includegraphics[keepaspectratio]{https://packages.typst.org/preview/thumbnails/ichigo-0.2.0-small.webp}}

\section{ichigo}\label{ichigo}

{ 0.2.0 }

A customizable Typst template for homework

\href{/app?template=ichigo&version=0.2.0}{Create project in app}

\phantomsection\label{readme}
Homework Template - 作业模�

\subsection{Usage -
使ç''¨æ--¹æ³•}\label{usage---uxe4uxbduxe7uxe6uxb9uxe6uxb3}

\begin{Shaded}
\begin{Highlighting}[]
\NormalTok{\#import "@preview/ichigo:0.2.0": config, prob}

\NormalTok{\#show: config.with(}
\NormalTok{  course{-}name: "Typst 使用小练习",}
\NormalTok{  serial{-}str: "第 1 次作业",}
\NormalTok{  author{-}info: [}
\NormalTok{    sjfhsjfh from PKU{-}Typst}
\NormalTok{  ],}
\NormalTok{  author{-}names: "sjfhsjfh",}
\NormalTok{)}

\NormalTok{\#prob[}
\NormalTok{  Calculate the 25th number in the Fibonacci sequence using Typst}
\NormalTok{][}
\NormalTok{  \#let f(n) = \{}
\NormalTok{    if n \textless{}= 2 \{}
\NormalTok{      return 1}
\NormalTok{    \}}
\NormalTok{    return f(n {-} 1) + f(n {-} 2)}
\NormalTok{  \}}
\NormalTok{  \#f(25)}
\NormalTok{]}
\end{Highlighting}
\end{Shaded}

\subsection{Documentation - æ--‡æ¡£}\label{documentation---uxe6uxe6}

\href{https://github.com/PKU-Typst/ichigo/releases/download/v0.2.0/documentation.pdf}{Click
to download - 点击下载}

\subsection{TODO - å¾\ldots 办}\label{todo---uxe5uxbeuxe5ux161ux17e}

\begin{itemize}
\tightlist
\item
  {[} {]} Theme list \& documentation
\end{itemize}

\href{/app?template=ichigo&version=0.2.0}{Create project in app}

\subsubsection{How to use}\label{how-to-use}

Click the button above to create a new project using this template in
the Typst app.

You can also use the Typst CLI to start a new project on your computer
using this command:

\begin{verbatim}
typst init @preview/ichigo:0.2.0
\end{verbatim}

\includesvg[width=0.16667in,height=0.16667in]{/assets/icons/16-copy.svg}

\subsubsection{About}\label{about}

\begin{description}
\tightlist
\item[Author :]
\href{https://github.com/PKU-Typst}{PKU-Typst}
\item[License:]
MIT
\item[Current version:]
0.2.0
\item[Last updated:]
November 18, 2024
\item[First released:]
October 3, 2024
\item[Archive size:]
17.1 kB
\href{https://packages.typst.org/preview/ichigo-0.2.0.tar.gz}{\pandocbounded{\includesvg[keepaspectratio]{/assets/icons/16-download.svg}}}
\item[Repository:]
\href{https://github.com/PKU-Typst/ichigo}{GitHub}
\item[Categor y :]
\begin{itemize}
\tightlist
\item[]
\item
  \pandocbounded{\includesvg[keepaspectratio]{/assets/icons/16-speak.svg}}
  \href{https://typst.app/universe/search/?category=report}{Report}
\end{itemize}
\end{description}

\subsubsection{Where to report issues?}\label{where-to-report-issues}

This template is a project of PKU-Typst . Report issues on
\href{https://github.com/PKU-Typst/ichigo}{their repository} . You can
also try to ask for help with this template on the
\href{https://forum.typst.app}{Forum} .

Please report this template to the Typst team using the
\href{https://typst.app/contact}{contact form} if you believe it is a
safety hazard or infringes upon your rights.

\phantomsection\label{versions}
\subsubsection{Version history}\label{version-history}

\begin{longtable}[]{@{}ll@{}}
\toprule\noalign{}
Version & Release Date \\
\midrule\noalign{}
\endhead
\bottomrule\noalign{}
\endlastfoot
0.2.0 & November 18, 2024 \\
\href{https://typst.app/universe/package/ichigo/0.1.0/}{0.1.0} & October
3, 2024 \\
\end{longtable}

Typst GmbH did not create this template and cannot guarantee correct
functionality of this template or compatibility with any version of the
Typst compiler or app.


\section{Package List LaTeX/socialhub-fa.tex}
\title{typst.app/universe/package/socialhub-fa}

\phantomsection\label{banner}
\section{socialhub-fa}\label{socialhub-fa}

{ 1.0.0 }

A Typst library for Social Media references with icons based on Font
Awesome.

\phantomsection\label{readme}
The \texttt{\ socialhub-fa\ } package is designed to help you create
your curriculum vitae (CV). It allows you to easily reference your
social media profiles with the typical icon of the service plus a link
to your profile.

\subsection{Features}\label{features}

\begin{itemize}
\tightlist
\item
  Support for popular social media, developer and career platforms
\item
  Uniform design for all entries
\item
  Based on the Internet’s icon library
  \href{https://fontawesome.com/}{Font Awesome}
\item
  Easy to use
\item
  Allows the customization of the look (extra args are passed to
  \href{https://typst.app/docs/reference/text/text/}{\texttt{\ text\ }}
  )
\end{itemize}

\subsection{Fonts Installation}\label{fonts-installation}

\subsubsection{Linux}\label{linux}

\begin{enumerate}
\tightlist
\item
  \href{https://fontawesome.com/download}{Download Font Awesome for
  Desktop}
\item
  Unzip the file
\item
  Switch into the \texttt{\ otfs\ } folder within the unzipped folder
\item
  Run \texttt{\ mkdir\ -p\ /usr/share/fonts/truetype/\ }
\item
  Run
  \texttt{\ install\ -m644\ \textquotesingle{}Font\ Awesome\ 6\ Brands-Regular-400.otf\textquotesingle{}\ /usr/share/fonts/truetype/\ }
\item
  Unfortunately not all brands are included in the brands font file, so
  you must also run
  \texttt{\ install\ -m644\ \textquotesingle{}Font\ Awesome\ 6\ Free-Regular-400.otf\textquotesingle{}\ /usr/share/fonts/truetype/\ }
\end{enumerate}

\subsection{Usage}\label{usage}

\subsubsection{Using Typst’s package
manager}\label{using-typstuxe2s-package-manager}

You can install the library using the
\href{https://github.com/typst/packages}{typst packages} :

\begin{Shaded}
\begin{Highlighting}[]
\NormalTok{\#import "@preview/socialhub{-}fa:1.0.0": *}
\end{Highlighting}
\end{Shaded}

\subsubsection{Install manually}\label{install-manually}

Put the \texttt{\ socialhub-fa.typ\ } file in your project directory and
import it:

\begin{Shaded}
\begin{Highlighting}[]
\NormalTok{\#import "socialhub{-}fa.typ": *}
\end{Highlighting}
\end{Shaded}

\subsubsection{Minimal Example}\label{minimal-example}

\begin{Shaded}
\begin{Highlighting}[]
\NormalTok{// \#import "@preview/socialhub{-}fa:1.0.0": github{-}info, gitlab{-}info}
\NormalTok{\#import "socialhub{-}fa.typ": github{-}info, gitlab{-}info}

\NormalTok{This project was created by \#github{-}info("Bi0T1N"). You can also find me on \#gitlab{-}info("GitLab", rgb("\#811052"), url: "https://gitlab.com/Bi0T1N").}
\end{Highlighting}
\end{Shaded}

\subsubsection{Examples}\label{examples}

See the
\href{https://github.com/typst/packages/raw/main/packages/preview/socialhub-fa/1.0.0/examples/examples.typ}{\texttt{\ examples.typ\ }}
file for a complete example. The
\href{https://github.com/typst/packages/raw/main/packages/preview/socialhub-fa/1.0.0/examples/}{generated
PDF files} are also available for preview.

\subsection{Troubleshooting}\label{troubleshooting}

\subsubsection{Icons are not displayed
correctly}\label{icons-are-not-displayed-correctly}

Make sure that you have installed the required Font Awesome
ligature-based font files.

\subsection{Contribution}\label{contribution}

Feel free to open an issue or a pull request if you find any problems or
have any suggestions.

\subsection{License}\label{license}

This library is licensed under the MIT license. Feel free to use it in
your project.

\subsubsection{How to add}\label{how-to-add}

Copy this into your project and use the import as
\texttt{\ socialhub-fa\ }

\begin{verbatim}
#import "@preview/socialhub-fa:1.0.0"
\end{verbatim}

\includesvg[width=0.16667in,height=0.16667in]{/assets/icons/16-copy.svg}

Check the docs for
\href{https://typst.app/docs/reference/scripting/\#packages}{more
information on how to import packages} .

\subsubsection{About}\label{about}

\begin{description}
\tightlist
\item[Author :]
Nico Neumann (Bi0T1N)
\item[License:]
MIT
\item[Current version:]
1.0.0
\item[Last updated:]
December 3, 2023
\item[First released:]
December 3, 2023
\item[Archive size:]
3.08 kB
\href{https://packages.typst.org/preview/socialhub-fa-1.0.0.tar.gz}{\pandocbounded{\includesvg[keepaspectratio]{/assets/icons/16-download.svg}}}
\item[Repository:]
\href{https://github.com/Bi0T1N/typst-socialhub-fa}{GitHub}
\end{description}

\subsubsection{Where to report issues?}\label{where-to-report-issues}

This package is a project of Nico Neumann (Bi0T1N) . Report issues on
\href{https://github.com/Bi0T1N/typst-socialhub-fa}{their repository} .
You can also try to ask for help with this package on the
\href{https://forum.typst.app}{Forum} .

Please report this package to the Typst team using the
\href{https://typst.app/contact}{contact form} if you believe it is a
safety hazard or infringes upon your rights.

\phantomsection\label{versions}
\subsubsection{Version history}\label{version-history}

\begin{longtable}[]{@{}ll@{}}
\toprule\noalign{}
Version & Release Date \\
\midrule\noalign{}
\endhead
\bottomrule\noalign{}
\endlastfoot
1.0.0 & December 3, 2023 \\
\end{longtable}

Typst GmbH did not create this package and cannot guarantee correct
functionality of this package or compatibility with any version of the
Typst compiler or app.


\section{Package List LaTeX/tbl.tex}
\title{typst.app/universe/package/tbl}

\phantomsection\label{banner}
\section{tbl}\label{tbl}

{ 0.0.4 }

Complex tables, written concisely

\phantomsection\label{readme}
This is a library for \href{https://typst.app/}{Typst} built upon Pg
Biel’s fabulous
\href{https://github.com/PgBiel/typst-tablex}{\texttt{\ tablex\ }}
library.

It allows the creation of complex tables in Typst using a compact syntax
based on the \texttt{\ tbl\ } preprocessor for the traditional UNIX
TROFF typesetting system. There are also some novel features that are
not currently offered by Typst itself or \texttt{\ tablex\ } , namely:

\begin{itemize}
\tightlist
\item
  Decimal point alignment (using the \texttt{\ decimalpoint\ } region
  option and \texttt{\ N\ } -classified columns)
\item
  Columns of equal width (using the \texttt{\ e\ } column modifier)
\item
  Columns with a guaranteed minimum width (using the \texttt{\ w\ }
  column modifier)
\item
  Cells that are ignored when calculating column widths (using the
  \texttt{\ z\ } column modifier)
\item
  Equation tables (using the \texttt{\ mode:\ "math"\ } region option)
\end{itemize}

Many other features exist to condense common configurations to a concise
syntax.

For example:

\begin{verbatim}
#import "@preview/tbl:0.0.4"
#show: tbl.template.with(box: true, tab: "|")

```tbl
      R | L
      R   N.
software|version
_
     AFL|2.39b
    Mutt|1.8.0
    Ruby|1.8.7.374
TeX Live|2015
```
\end{verbatim}

\pandocbounded{\includegraphics[keepaspectratio]{https://raw.githubusercontent.com/maxcrees/tbl.typ/v0.0.4/test/00/02_software.png}}

Many other examples and copious documentation are provided in the
\href{https://maxre.es/tbl.typ/v0.0.4.pdf}{\texttt{\ README.pdf\ }}
file.

\href{https://github.com/maxcrees/tbl.typ}{The source repository} also
includes a test suite based on those examples, which can be ran using
the GNU \texttt{\ make\ } command. See \texttt{\ make\ help\ } for
details.

\subsubsection{How to add}\label{how-to-add}

Copy this into your project and use the import as \texttt{\ tbl\ }

\begin{verbatim}
#import "@preview/tbl:0.0.4"
\end{verbatim}

\includesvg[width=0.16667in,height=0.16667in]{/assets/icons/16-copy.svg}

Check the docs for
\href{https://typst.app/docs/reference/scripting/\#packages}{more
information on how to import packages} .

\subsubsection{About}\label{about}

\begin{description}
\tightlist
\item[Author :]
Max Rees
\item[License:]
MPL-2.0
\item[Current version:]
0.0.4
\item[Last updated:]
August 19, 2023
\item[First released:]
July 29, 2023
\item[Archive size:]
14.3 kB
\href{https://packages.typst.org/preview/tbl-0.0.4.tar.gz}{\pandocbounded{\includesvg[keepaspectratio]{/assets/icons/16-download.svg}}}
\item[Repository:]
\href{https://github.com/maxcrees/tbl.typ}{GitHub}
\end{description}

\subsubsection{Where to report issues?}\label{where-to-report-issues}

This package is a project of Max Rees . Report issues on
\href{https://github.com/maxcrees/tbl.typ}{their repository} . You can
also try to ask for help with this package on the
\href{https://forum.typst.app}{Forum} .

Please report this package to the Typst team using the
\href{https://typst.app/contact}{contact form} if you believe it is a
safety hazard or infringes upon your rights.

\phantomsection\label{versions}
\subsubsection{Version history}\label{version-history}

\begin{longtable}[]{@{}ll@{}}
\toprule\noalign{}
Version & Release Date \\
\midrule\noalign{}
\endhead
\bottomrule\noalign{}
\endlastfoot
0.0.4 & August 19, 2023 \\
\href{https://typst.app/universe/package/tbl/0.0.3/}{0.0.3} & July 29,
2023 \\
\end{longtable}

Typst GmbH did not create this package and cannot guarantee correct
functionality of this package or compatibility with any version of the
Typst compiler or app.


\section{Package List LaTeX/guided-resume-starter-cgc.tex}
\title{typst.app/universe/package/guided-resume-starter-cgc}

\phantomsection\label{banner}
\phantomsection\label{template-thumbnail}
\pandocbounded{\includegraphics[keepaspectratio]{https://packages.typst.org/preview/thumbnails/guided-resume-starter-cgc-2.0.0-small.webp}}

\section{guided-resume-starter-cgc}\label{guided-resume-starter-cgc}

{ 2.0.0 }

A guided starter resume for people looking to focus on highlighting
their experience -\/- without having to worry about the hassle of
formatting.

\href{/app?template=guided-resume-starter-cgc&version=2.0.0}{Create
project in app}

\phantomsection\label{readme}
This template is a starter resume for people looking to focus on the
content of their resume, without having to worry about the hassle of
formatting.

\subsection{Get Started!}\label{get-started}

\subsubsection{Quickstart: Typst
Universe}\label{quickstart-typst-universe}

\begin{enumerate}
\tightlist
\item
  If you haven’t already, \href{https://typst.app/}{create a (free!)
  Typst account} .
\item
  Once you have an account, go to the template on
  \href{https://typst.app/universe/package/resume-starter-cgc}{Typst
  Universe}
\item
  Click on “Create Project in App�, give your project a title, and
  press “Create�.
\item
  Start editing! This copy is your own personal copy to edit however you
  want!
\end{enumerate}

There are two files included in this project:

\begin{itemize}
\tightlist
\item
  \texttt{\ starter.typ\ } contains the full template, along with a
  written guide to help you put your best (single-paged) foot forward!
\item
  \texttt{\ resume.typ\ } is the same template, but without the full
  guide included.
\item
  \texttt{\ templates/resume.template.typ\ } contains the formatting and
  style for the underlying pieces.
\end{itemize}

\textbf{I would highly recommend reading \texttt{\ starter.typ\ } or
skimming through the
\href{https://blog.chaoticgood.computer/writing/notes/typst-resume-template}{online
guide} to understand best practices when using the template.}

\subsubsection{Alternative: Typst CLI}\label{alternative-typst-cli}

If you’d prefer to simply download \& modify the template, you can use
the \href{https://github.com/typst/typst}{Typst CLI} to download it
instead:

\begin{Shaded}
\begin{Highlighting}[]
\ExtensionTok{typst}\NormalTok{ init @preview/resume{-}starter{-}cgc}
\end{Highlighting}
\end{Shaded}

\subsection{Layout}\label{layout}

\subsubsection{Header}\label{header}

The resume can be created with a header with the following attributes:

\begin{itemize}
\tightlist
\item
  \texttt{\ author\ } : Your name
\item
  \texttt{\ location\ } : The city, state/province, and country you
  reside in.
\item
  \texttt{\ contacts\ } : A list of contact information and additional
  information
\end{itemize}

\paragraph{Header Example}\label{header-example}

\begin{Shaded}
\begin{Highlighting}[]
\NormalTok{\#show: resume.with(}
\NormalTok{  author: "Dr. Emmit \textbackslash{}"Doc\textbackslash{}" Brown",}
\NormalTok{  location: "Hill Valley, CA",}
\NormalTok{  contacts: (}
\NormalTok{    [\#link("mailto:your\_email@yourmail.com")[Email]],}
\NormalTok{    [\#link("https://your{-}cool{-}site.com")[Website]],}
\NormalTok{    [\#link("https://github.com/your{-}linkedin")[GitHub]],}
\NormalTok{    [\#link("https://linkedin.com/in/your{-}linkedin")[LinkedIn]],}
\NormalTok{  )}
\NormalTok{)}
\end{Highlighting}
\end{Shaded}

\subsubsection{Education}\label{education}

The Education ( \texttt{\ \#edu\ } ) section can be used to highlight
for formal education and certifications.

\begin{itemize}
\tightlist
\item
  \texttt{\ institution\ } : Name of the institution where you study, or
  have graduated from.
\item
  \texttt{\ date\ } : Your graduation date, or expected graduation date.
\item
  \texttt{\ degrees\ } : The degrees you received at the institution

  \begin{itemize}
  \tightlist
  \item
    Each entry is two sections: the \textbf{title} of the degree, and
    the \textbf{subject} that you studied.
  \end{itemize}
\item
  \texttt{\ gpa\ } (optional): Your GPA, or other additional
  information.
\end{itemize}

\paragraph{Education Example}\label{education-example}

\begin{Shaded}
\begin{Highlighting}[]
\NormalTok{\#edu(}
\NormalTok{  institution: "University of Colombia",}
\NormalTok{  date: "Aug 1948",}
\NormalTok{  gpa: "3.9 of 4.0, Summa Cum Laude",}
\NormalTok{  degrees: (}
\NormalTok{    ("Bachelor\textquotesingle{}s of Science", "Nuclear Engineering"),}
\NormalTok{    ("Minors", "Automobile Design, Arabic"),}
\NormalTok{    ("Focus", "Childcare, Education")}
\NormalTok{  ),}
\NormalTok{)}
\end{Highlighting}
\end{Shaded}

\subsubsection{Skills}\label{skills}

An additional Skills ( \texttt{\ \#skills\ } ) section to list skills
relevant to the job you’re applying for.

The input is a list of \texttt{\ Label:\ Skills{[}{]}\ } , in order to
easily toggle comments on skills that you may want to leave in but not
render for a particular application.

\paragraph{Skills Example}\label{skills-example}

\begin{Shaded}
\begin{Highlighting}[]
\NormalTok{\#skills((}
\NormalTok{  ("Expertise", (}
\NormalTok{    [Theoretical Physics],}
\NormalTok{    [Time Travel],}
\NormalTok{    [Nuclear Material Management],}
\NormalTok{    [Student Mentoring],}
\NormalTok{  )),}
\NormalTok{  ("Software", (}
\NormalTok{    [AutoDesk CAD],}
\NormalTok{    [Delorean OS],}
\NormalTok{    [Windows 1],}
\NormalTok{  )),}
\NormalTok{  ("Languages", (}
\NormalTok{    [C++],}
\NormalTok{    [C Language],}
\NormalTok{    [MatLab],}
\NormalTok{    [Punch Cards],}
\NormalTok{  )),}
\NormalTok{))}
\end{Highlighting}
\end{Shaded}

\subsubsection{Experience}\label{experience}

The bulk of your resume, the Experience ( \texttt{\ \#exp\ } ) sections
provide a compact \& concise formatting for bulleted details of your
previous and current work experience.

This section is meant to be flexible, and can also be used to talk about
projects and other experiences that may fall outside of the traditional
definition of “work.�

\begin{itemize}
\tightlist
\item
  \texttt{\ role\ } : The title of your position/role in this
  experience.
\item
  \texttt{\ project\ } : The company you worked at, or the name of the
  project you worked on.
\item
  \texttt{\ date\ } : The start and end dates of this experience.
\item
  \texttt{\ details\ } : A description of the work you did in this
  position

  \begin{itemize}
  \tightlist
  \item
    It is \textbf{highly encouraged} to use bullet points in this
    section.
  \end{itemize}
\item
  \texttt{\ location\ } (optional): The location of the experience
\item
  \texttt{\ summary\ } (optional): A brief summary of the company’s
  mission or project goal.
\end{itemize}

\paragraph{Experience Example}\label{experience-example}

\begin{Shaded}
\begin{Highlighting}[]
\NormalTok{\#exp(}
\NormalTok{  role: "Theoretical Physics Consultant",}
\NormalTok{  project: "Doc Brown\textquotesingle{}s Garage",}
\NormalTok{  date: "June 1953 {-} Oct 2015",}
\NormalTok{  location: "Hill Valley, CA",}
\NormalTok{  summary: "Specializing in development of time travel devices and student tutoring",}
\NormalTok{  details: [}
\NormalTok{    {-} Lead development of time travel devices, resulting in the ability to travel back and forth through time}
\NormalTok{    {-} Managed and executed a budget of \textbackslash{}$14 million dollars gained from an unexplained family fortune}
\NormalTok{    {-} Oversaw QA testing for time travel devices, minimizing risk of maternal time{-}travel related incidents}
\NormalTok{  ]}
\NormalTok{)}
\end{Highlighting}
\end{Shaded}

\subsection{Questions \& Suggestions}\label{questions-suggestions}

Have any questions, comments, or suggestions about the template? Please
feel free to reach out at
\href{mailto:mentoring@chaoticgood.computer}{\texttt{\ mentoring@chaoticgood.computer\ }}
!

\href{/app?template=guided-resume-starter-cgc&version=2.0.0}{Create
project in app}

\subsubsection{How to use}\label{how-to-use}

Click the button above to create a new project using this template in
the Typst app.

You can also use the Typst CLI to start a new project on your computer
using this command:

\begin{verbatim}
typst init @preview/guided-resume-starter-cgc:2.0.0
\end{verbatim}

\includesvg[width=0.16667in,height=0.16667in]{/assets/icons/16-copy.svg}

\subsubsection{About}\label{about}

\begin{description}
\tightlist
\item[Author s :]
\href{https://chaoticgood.computer}{Spencer Elkington} \&
\href{mailto:spencer@chaoticgood.computer}{Spencer Elkington}
\item[License:]
Unlicense
\item[Current version:]
2.0.0
\item[Last updated:]
May 6, 2024
\item[First released:]
May 6, 2024
\item[Archive size:]
22.6 kB
\href{https://packages.typst.org/preview/guided-resume-starter-cgc-2.0.0.tar.gz}{\pandocbounded{\includesvg[keepaspectratio]{/assets/icons/16-download.svg}}}
\item[Repository:]
\href{https://github.com/chaoticgoodcomputing/typst-resume-starter}{GitHub}
\item[Categor y :]
\begin{itemize}
\tightlist
\item[]
\item
  \pandocbounded{\includesvg[keepaspectratio]{/assets/icons/16-user.svg}}
  \href{https://typst.app/universe/search/?category=cv}{CV}
\end{itemize}
\end{description}

\subsubsection{Where to report issues?}\label{where-to-report-issues}

This template is a project of Spencer Elkington and Spencer Elkington .
Report issues on
\href{https://github.com/chaoticgoodcomputing/typst-resume-starter}{their
repository} . You can also try to ask for help with this template on the
\href{https://forum.typst.app}{Forum} .

Please report this template to the Typst team using the
\href{https://typst.app/contact}{contact form} if you believe it is a
safety hazard or infringes upon your rights.

\phantomsection\label{versions}
\subsubsection{Version history}\label{version-history}

\begin{longtable}[]{@{}ll@{}}
\toprule\noalign{}
Version & Release Date \\
\midrule\noalign{}
\endhead
\bottomrule\noalign{}
\endlastfoot
2.0.0 & May 6, 2024 \\
\end{longtable}

Typst GmbH did not create this template and cannot guarantee correct
functionality of this template or compatibility with any version of the
Typst compiler or app.


\section{Package List LaTeX/clean-math-thesis.tex}
\title{typst.app/universe/package/clean-math-thesis}

\phantomsection\label{banner}
\phantomsection\label{template-thumbnail}
\pandocbounded{\includegraphics[keepaspectratio]{https://packages.typst.org/preview/thumbnails/clean-math-thesis-0.2.0-small.webp}}

\section{clean-math-thesis}\label{clean-math-thesis}

{ 0.2.0 }

A simple and good looking template for mathematical theses

\href{/app?template=clean-math-thesis&version=0.2.0}{Create project in
app}

\phantomsection\label{readme}
\href{https://github.com/sebaseb98/clean-math-thesis/actions/workflows/build.yml}{\pandocbounded{\includesvg[keepaspectratio]{https://github.com/sebaseb98/clean-math-thesis/actions/workflows/build.yml/badge.svg}}}
\href{https://github.com/sebaseb98/clean-math-thesis}{\pandocbounded{\includegraphics[keepaspectratio]{https://img.shields.io/badge/GitHub-repo-blue}}}
\href{https://opensource.org/licenses/MIT}{\pandocbounded{\includesvg[keepaspectratio]{https://img.shields.io/badge/License-MIT-success.svg}}}

\href{https://typst.app/home/}{Typst} thesis template for mathematical
theses built for simple, efficient use and a clean look. Of course, it
can also be used for other subjects, but the following math-specific
features are already contained in the template:

\begin{itemize}
\tightlist
\item
  theorems, lemmas, corollaries, proofs etc. prepared using
  \href{https://typst.app/universe/package/great-theorems}{great-theorems}
\item
  equation settings (using either
  \href{https://typst.app/universe/package/equate}{equate} for numbering
  of subequations or
  \href{https://typst.app/universe/package/i-figured/}{i-figured} for
  equation numbering which includes the chapter number)
\item
  pseudocode package
  \href{https://typst.app/universe/package/lovelace}{lovelace} included.
\end{itemize}

Additionally, it has headers built with
\href{https://typst.app/universe/package/hydra}{hydra} .

\subsection{Set-Up}\label{set-up}

The template is already filled with dummy data, to give users an
\href{https://github.com/sebaseb98/clean-math-thesis/blob/main/template/main.pdf}{impression
how it looks like} . The thesis is obtained by compiling
\texttt{\ main.typ\ } .

\begin{itemize}
\tightlist
\item
  after
  \href{https://github.com/typst/typst?tab=readme-ov-file\#installation}{installing
  Typst} you can conveniently use the following to create a new folder
  containing this project.
\end{itemize}

\begin{Shaded}
\begin{Highlighting}[]
\ExtensionTok{typst}\NormalTok{ init @preview/clean{-}math{-}thesis:0.2.0}
\end{Highlighting}
\end{Shaded}

\begin{itemize}
\tightlist
\item
  edit the data in \texttt{\ main.typ\ } â†'
  \texttt{\ \#show\ template.with({[}your\ data{]})\ }
\end{itemize}

\subsubsection{Parameters of the
Template}\label{parameters-of-the-template}

\ul{personal/subject related information}

\begin{itemize}
\tightlist
\item
  \texttt{\ author\ } : Name of the author of the thesis.
\item
  \texttt{\ title\ } : Title of the thesis.
\item
  \texttt{\ supervisor1\ } : Name of the first supervisor.
\item
  \texttt{\ supervisor2\ } : Name of the second supervisor.
\item
  \texttt{\ degree\ } : Degree for which the thesis is submitted.
\item
  \texttt{\ program\ } : Program under which the thesis is submitted.
\item
  \texttt{\ university\ } : Name of the university.
\item
  \texttt{\ institute\ } : Name of the institute.
\item
  \texttt{\ deadline\ } : Submission deadline of the thesis.
\end{itemize}

\ul{file paths for logos etc.}

\begin{itemize}
\tightlist
\item
  \texttt{\ uni-logo\ } : Image, e.g.
  \texttt{\ image("images/logo\_placeholder.svg",\ width:\ 50\%)\ }
\item
  \texttt{\ institute-logo\ } : Image.
\end{itemize}

\ul{formatting settings}

\begin{itemize}
\tightlist
\item
  \texttt{\ citation-style\ } : Citation style to be used in the thesis.
\item
  \texttt{\ body-font\ } : Font to be used for the body text.
\item
  \texttt{\ cover-font\ } : Font to be used for the cover text.
\end{itemize}

\ul{content that needs to be placed differently then normal chapters}

\begin{itemize}
\tightlist
\item
  \texttt{\ abstract\ } : Content for the abstract section.
\end{itemize}

\ul{equation settings}

\begin{itemize}
\tightlist
\item
  \texttt{\ equate-settings\ } : either none -\textgreater{} use
  i-figured; or tuple with the settings for the equations (see
  \href{https://typst.app/universe/package/equate}{docs} ), e.g.
  (breakable: true, sub-numbering: true, number-mode: “label�) The
  switching between these is currently not optimal: i-figured needs a
  prefix ( \texttt{\ eq:\ } ) so if we label an equation like
  \texttt{\ \textless{}equation\textgreater{}\ } the corresponding
  reference is \texttt{\ @eq:equation\ } and for equate we don’t have
  this prefix, i.e. the reference would be \texttt{\ @equation\ } in
  this example. This is something to be improved in future releases.
\item
  \texttt{\ equation-numbering-pattern\ } : specify the
  \href{https://typst.app/docs/reference/model/numbering/\#parameters-numbering}{numbering}
  of the equations. The second counting symbol (e.g. the \texttt{\ a\ }
  in \texttt{\ "(1.a)"\ } ) is either used for subequation numbering or
  for the numbering of equations in the chapters. \ul{colors}
\item
  \texttt{\ cover-color\ } : Color used for the cover.
\item
  \texttt{\ heading-color\ } : Color used for headings.
\item
  \texttt{\ link-color\ } : Color used for links and references.
\end{itemize}

\subsubsection{Other Customizations}\label{other-customizations}

\begin{itemize}
\tightlist
\item
  \texttt{\ declaration.typ\ } should be modified
\item
  when adding chapters, remember to include them into the
  \texttt{\ main.typ\ } .
\item
  (optional) change colors and appearance of the theorem environment in
  the \texttt{\ customization/\ } -folder.
\end{itemize}

\subsubsection{Use of the template in existing
projects}\label{use-of-the-template-in-existing-projects}

If you want to change an existing typst project structure to use this
template, just type the following lines

\begin{Shaded}
\begin{Highlighting}[]
\NormalTok{\#import "@preview/clean{-}math{-}thesis:0.1.0": template}

\NormalTok{\#show: template.with(}
\NormalTok{  // your user specific data, parameters explained above}
\NormalTok{)}

\NormalTok{\#include "my\_content.typ"  // and eventually more files}
\end{Highlighting}
\end{Shaded}

\subsection{Disclaimer}\label{disclaimer}

This template was created after Sebastian finished his master’s
thesis. We do not guarantee that it will be accepted by any university,
please clarify in advance if it fulfills all requirements. If not, this
template might still be a good starting point.

\subsection{Acknowledgements}\label{acknowledgements}

As inspiration on how to structure this template, we used the
\href{https://typst.app/universe/package/modern-unito-thesis}{modern-unito-thesis}
template. The design is inspired by the
\href{https://github.com/FAU-AMMN/fau-book}{fau-book} template.

\subsection{Feedback \& Improvements}\label{feedback-improvements}

If you encounter problems, please open issues. In case you found useful
extensions or improved anything We are also very happy to accept pull
requests.

\href{/app?template=clean-math-thesis&version=0.2.0}{Create project in
app}

\subsubsection{How to use}\label{how-to-use}

Click the button above to create a new project using this template in
the Typst app.

You can also use the Typst CLI to start a new project on your computer
using this command:

\begin{verbatim}
typst init @preview/clean-math-thesis:0.2.0
\end{verbatim}

\includesvg[width=0.16667in,height=0.16667in]{/assets/icons/16-copy.svg}

\subsubsection{About}\label{about}

\begin{description}
\tightlist
\item[Author s :]
\href{https://github.com/sebaseb98}{Sebastian Eberle} \&
\href{https://github.com/JoshuaLampert}{Joshua Lampert}
\item[License:]
MIT
\item[Current version:]
0.2.0
\item[Last updated:]
November 26, 2024
\item[First released:]
November 12, 2024
\item[Minimum Typst version:]
0.12.0
\item[Archive size:]
9.60 kB
\href{https://packages.typst.org/preview/clean-math-thesis-0.2.0.tar.gz}{\pandocbounded{\includesvg[keepaspectratio]{/assets/icons/16-download.svg}}}
\item[Repository:]
\href{https://github.com/sebaseb98/clean-math-thesis}{GitHub}
\item[Discipline :]
\begin{itemize}
\tightlist
\item[]
\item
  \href{https://typst.app/universe/search/?discipline=mathematics}{Mathematics}
\end{itemize}
\item[Categor y :]
\begin{itemize}
\tightlist
\item[]
\item
  \pandocbounded{\includesvg[keepaspectratio]{/assets/icons/16-mortarboard.svg}}
  \href{https://typst.app/universe/search/?category=thesis}{Thesis}
\end{itemize}
\end{description}

\subsubsection{Where to report issues?}\label{where-to-report-issues}

This template is a project of Sebastian Eberle and Joshua Lampert .
Report issues on
\href{https://github.com/sebaseb98/clean-math-thesis}{their repository}
. You can also try to ask for help with this template on the
\href{https://forum.typst.app}{Forum} .

Please report this template to the Typst team using the
\href{https://typst.app/contact}{contact form} if you believe it is a
safety hazard or infringes upon your rights.

\phantomsection\label{versions}
\subsubsection{Version history}\label{version-history}

\begin{longtable}[]{@{}ll@{}}
\toprule\noalign{}
Version & Release Date \\
\midrule\noalign{}
\endhead
\bottomrule\noalign{}
\endlastfoot
0.2.0 & November 26, 2024 \\
\href{https://typst.app/universe/package/clean-math-thesis/0.1.0/}{0.1.0}
& November 12, 2024 \\
\end{longtable}

Typst GmbH did not create this template and cannot guarantee correct
functionality of this template or compatibility with any version of the
Typst compiler or app.


\section{Package List LaTeX/rose-pine.tex}
\title{typst.app/universe/package/rose-pine}

\phantomsection\label{banner}
\section{rose-pine}\label{rose-pine}

{ 0.2.0 }

Soho vibes for Typst in a form of easily applicable theme.

\phantomsection\label{readme}
\includegraphics[width=0.83333in,height=\textheight,keepaspectratio]{https://raw.githubusercontent.com/rose-pine/rose-pine-theme/main/assets/icon.png}

\subsection{Rosé Pine for Typst}\label{rosuxe3-pine-for-typst}

All natural pine, faux fur and a bit of soho vibes for the classy
minimalist

\href{https://github.com/rose-pine/rose-pine-theme}{\pandocbounded{\includegraphics[keepaspectratio]{https://img.shields.io/badge/community-ros\%C3\%A9\%20pine-26233a?labelColor=191724&logo=data:image/svg+xml;base64,PHN2ZyB3aWR0aD0iMjUwIiBoZWlnaHQ9IjIzNyIgdmlld0JveD0iMCAwIDI1MCAyMzciIGZpbGw9Im5vbmUiIHhtbG5zPSJodHRwOi8vd3d3LnczLm9yZy8yMDAwL3N2ZyI+CjxwYXRoIGQ9Ik0xNjEuMjI3IDE2MS4yNTFDMTMyLjE1NCAxNjkuMDQxIDExNC45MDEgMTk4LjkyNCAxMjIuNjkxIDIyNy45OTdDMTIzLjkyNSAyMzIuNjAzIDEyOC42NTkgMjM1LjMzNiAxMzMuMjY0IDIzNC4xMDJMMTg1LjkwNyAyMTkuOTk2QzIxOS41ODUgMjEwLjk3MiAyMzkuNTcgMTc2LjM1NCAyMzAuNTQ2IDE0Mi42NzdMMTYxLjIyNyAxNjEuMjUxWiIgZmlsbD0iIzI0NjI3QiIvPgo8cGF0aCBkPSJNODguMTgzNiAxNTkuOTg4QzExNy4yNTcgMTY3Ljc3OCAxMzQuNTEgMTk3LjY2MiAxMjYuNzIgMjI2LjczNUMxMjUuNDg2IDIzMS4zNCAxMjAuNzUyIDIzNC4wNzMgMTE2LjE0NyAyMzIuODM5TDYzLjUwNDEgMjE4LjczM0MyOS44MjY0IDIwOS43MSA5Ljg0MDk0IDE3NS4wOTIgMTguODY0OSAxNDEuNDE0TDg4LjE4MzYgMTU5Ljk4OFoiIGZpbGw9IiMyNDYyN0IiLz4KPHBhdGggZD0iTTE4Ni44NjcgMTcyLjk4QzE1Mi4wMDIgMTcyLjk4IDEyMy43MzcgMjAxLjI0NSAxMjMuNzM3IDIzNi4xMTFIMTg2Ljg3QzIyMS43MzYgMjM2LjExMSAyNTAgMjA3Ljg0NiAyNTAgMTcyLjk4TDE4Ni44NjcgMTcyLjk4WiIgZmlsbD0iIzMxNzQ4RiIvPgo8cGF0aCBkPSJNNjMuMTMyNyAxNzIuOThDOTcuOTk4NCAxNzIuOTggMTI2LjI2MyAyMDEuMjQ1IDEyNi4yNjMgMjM2LjExMUg2My4xM0MyOC4yNjQyIDIzNi4xMTEgLTEuNTI0MDNlLTA2IDIwNy44NDYgMCAxNzIuOThMNjMuMTMyNyAxNzIuOThaIiBmaWxsPSIjMzE3NDhGIi8+CjxwYXRoIGQ9Ik0xNzEuNzE3IDc1LjEyNjNDMTcxLjcxNyAxMDEuMjc2IDE1MC41MTggMTIyLjQ3NSAxMjQuMzY5IDEyMi40NzVDOTguMjE4OCAxMjIuNDc1IDc3LjAyMDIgMTAxLjI3NiA3Ny4wMjAyIDc1LjEyNjNDNzcuMDIwMiA0OC45NzY0IDk4LjIxODggMjcuNzc3OCAxMjQuMzY5IDI3Ljc3NzhDMTUwLjUxOCAyNy43Nzc4IDE3MS43MTcgNDguOTc2NCAxNzEuNzE3IDc1LjEyNjNaIiBmaWxsPSIjRUJCQ0JBIi8+CjxwYXRoIGQ9Ik0xNDQuMjE3IDg2LjIzNzlDMTYxLjY0OSA1Ni4wNDMyIDE1MS4zMDMgMTcuNDMyOSAxMjEuMTA4IDBMMTA2LjA2IDI2LjA2NDRDODguNjI3IDU2LjI1OSA5OC45NzM2IDk0Ljg2OTQgMTI5LjE2OCAxMTIuMzAyTDE0NC4yMTcgODYuMjM3OVoiIGZpbGw9IiNFQkJDQkEiLz4KPHBhdGggZD0iTTEyNS4yOTkgNjAuOTc4OUMxMTYuMjc1IDI3LjMwMTIgODEuNjU3NSA3LjMxNTY3IDQ3Ljk3OTcgMTYuMzM5Nkw2NC4zMTk3IDc3LjMyMTFDNzMuMzQzNiAxMTAuOTk5IDEwNy45NjEgMTMwLjk4NCAxNDEuNjM5IDEyMS45NkwxMjUuMjk5IDYwLjk3ODlaIiBmaWxsPSIjRUJCQ0JBIi8+CjxwYXRoIGQ9Ik0xMjQuOTI2IDYwLjk3ODlDMTMzLjk1IDI3LjMwMTIgMTY4LjU2NyA3LjMxNTY3IDIwMi4yNDUgMTYuMzM5NkwxODUuOTA1IDc3LjMyMTFDMTc2Ljg4MSAxMTAuOTk5IDE0Mi4yNjMgMTMwLjk4NCAxMDguNTg2IDEyMS45NkwxMjQuOTI2IDYwLjk3ODlaIiBmaWxsPSIjRUJCQ0JBIi8+Cjwvc3ZnPgo=&style=for-the-badge}}}

\subsection{Usage}\label{usage}

\begin{enumerate}
\tightlist
\item
  Open a Typst document
\item
  Import this library:
\end{enumerate}

\begin{Shaded}
\begin{Highlighting}[]
\NormalTok{\#import "@preview/rose{-}pine:0.2.0": apply}
\end{Highlighting}
\end{Shaded}

\begin{enumerate}
\setcounter{enumi}{2}
\tightlist
\item
  Apply the theme to the document using \texttt{\ \#show\ } :
\end{enumerate}

\begin{Shaded}
\begin{Highlighting}[]
\NormalTok{\#show: apply()}
\end{Highlighting}
\end{Shaded}

\begin{enumerate}
\setcounter{enumi}{3}
\tightlist
\item
  Use other colors anywhere you want:
\end{enumerate}

\begin{Shaded}
\begin{Highlighting}[]
\NormalTok{\#import "@preview/rose{-}pine:0.2.0": rose{-}pine}

\NormalTok{\#text(fill: rose{-}pine.love)[Some red text]}
\end{Highlighting}
\end{Shaded}

You can also use other variants by importing them/passing their name to
\texttt{\ apply()\ } . E.g.

\begin{Shaded}
\begin{Highlighting}[]
\NormalTok{\#import "@preview/rose{-}pine:0.2.0": apply, rose{-}pine{-}dawn}
\NormalTok{\#show: apply(variant: "rose{-}pine{-}dawn")}
\end{Highlighting}
\end{Shaded}

\subsection{Gallery}\label{gallery}

\pandocbounded{\includegraphics[keepaspectratio]{https://github.com/oplik0/rose-pine-typst/assets/25460763/0f3f4c3a-a923-4587-bd80-92e60165dc7e}}

\pandocbounded{\includegraphics[keepaspectratio]{https://github.com/oplik0/rose-pine-typst/assets/25460763/9b842b68-4f57-4536-955f-7f510b77a579}}

\pandocbounded{\includegraphics[keepaspectratio]{https://github.com/oplik0/rose-pine-typst/assets/25460763/5b2d211d-d645-4b72-9b58-b73bccfae1ae}}

\subsection{Thanks to}\label{thanks-to}

\begin{itemize}
\tightlist
\item
  \href{https://github.com/oplik0}{oplik0}
\end{itemize}

\subsection{Contributing}\label{contributing}

\begin{quote}
Prefer using \href{https://github.com/rose-pine/build}{@rose-pine/build}
when possible
\end{quote}

Modify \texttt{\ src/template.typ\ } using Rosé Pine variables, then
build variants:

\begin{Shaded}
\begin{Highlighting}[]
\ExtensionTok{npx}\NormalTok{ @rose{-}pine/build@0.8.2 }\AttributeTok{{-}t}\NormalTok{ src/template.typ }\AttributeTok{{-}o}\NormalTok{ src/themes }\AttributeTok{{-}p}\NormalTok{ $ }\AttributeTok{{-}f}\NormalTok{ hex}
\end{Highlighting}
\end{Shaded}

\emph{Generated by
\href{https://github.com/rose-pine/build}{@rose-pine/build@0.8.2}}

To rebuild the syntax highlighting themes you similarly need to run

\begin{Shaded}
\begin{Highlighting}[]
\ExtensionTok{npx}\NormalTok{ @rose{-}pine/build@0.8.2 }\AttributeTok{{-}t}\NormalTok{ src/rose{-}pine{-}template.tmTheme }\AttributeTok{{-}o}\NormalTok{ src/themes }\AttributeTok{{-}p}\NormalTok{ $ }\AttributeTok{{-}f}\NormalTok{ hex }\AttributeTok{{-}{-}help}
\end{Highlighting}
\end{Shaded}

\subsubsection{How to add}\label{how-to-add}

Copy this into your project and use the import as \texttt{\ rose-pine\ }

\begin{verbatim}
#import "@preview/rose-pine:0.2.0"
\end{verbatim}

\includesvg[width=0.16667in,height=0.16667in]{/assets/icons/16-copy.svg}

Check the docs for
\href{https://typst.app/docs/reference/scripting/\#packages}{more
information on how to import packages} .

\subsubsection{About}\label{about}

\begin{description}
\tightlist
\item[Author s :]
oplik0 \& Rosé Pine
\item[License:]
MIT
\item[Current version:]
0.2.0
\item[Last updated:]
March 12, 2024
\item[First released:]
September 19, 2023
\item[Archive size:]
6.85 kB
\href{https://packages.typst.org/preview/rose-pine-0.2.0.tar.gz}{\pandocbounded{\includesvg[keepaspectratio]{/assets/icons/16-download.svg}}}
\item[Repository:]
\href{https://github.com/rose-pine/typst}{GitHub}
\end{description}

\subsubsection{Where to report issues?}\label{where-to-report-issues}

This package is a project of oplik0 and Rosé Pine . Report issues on
\href{https://github.com/rose-pine/typst}{their repository} . You can
also try to ask for help with this package on the
\href{https://forum.typst.app}{Forum} .

Please report this package to the Typst team using the
\href{https://typst.app/contact}{contact form} if you believe it is a
safety hazard or infringes upon your rights.

\phantomsection\label{versions}
\subsubsection{Version history}\label{version-history}

\begin{longtable}[]{@{}ll@{}}
\toprule\noalign{}
Version & Release Date \\
\midrule\noalign{}
\endhead
\bottomrule\noalign{}
\endlastfoot
0.2.0 & March 12, 2024 \\
\href{https://typst.app/universe/package/rose-pine/0.1.0/}{0.1.0} &
September 19, 2023 \\
\end{longtable}

Typst GmbH did not create this package and cannot guarantee correct
functionality of this package or compatibility with any version of the
Typst compiler or app.


\section{Package List LaTeX/marginalia.tex}
\title{typst.app/universe/package/marginalia}

\phantomsection\label{banner}
\section{marginalia}\label{marginalia}

{ 0.1.1 }

Configurable margin-notes and matching wide blocks.

\phantomsection\label{readme}
\subsection{Setup}\label{setup}

Put something akin to the following at the start of your
\texttt{\ .typ\ } file:

\begin{Shaded}
\begin{Highlighting}[]
\NormalTok{\#import "@preview/marginalia:0.1.1" as marginalia: note, wideblock}
\NormalTok{\#let config = (}
\NormalTok{  // inner: ( far: 5mm, width: 15mm, sep: 5mm ),}
\NormalTok{  // outer: ( far: 5mm, width: 15mm, sep: 5mm ),}
\NormalTok{  // top: 2.5cm,}
\NormalTok{  // bottom: 2.5cm,}
\NormalTok{  // book: false,}
\NormalTok{  // clearance: 8pt,}
\NormalTok{  // flush{-}numbers: false,}
\NormalTok{  // numbering: /* numbering{-}function */,}
\NormalTok{)}
\NormalTok{\#marginalia.configure(..config)}
\NormalTok{\#set page(}
\NormalTok{  // setup margins:}
\NormalTok{  ..marginalia.page{-}setup(..config),}
\NormalTok{  /* other page setup */}
\NormalTok{)}
\end{Highlighting}
\end{Shaded}

If \texttt{\ book\ } is \texttt{\ false\ } , \texttt{\ inner\ } and
\texttt{\ outer\ } correspond to the left and right margins
respectively. If book is true, the margins swap sides on even and odd
pages. Notes are placed in the outside margin by default.

Where you can then customize \texttt{\ config\ } to your preferences.
Shown here (as comments) are the default values taken if the
corresponding keys are unset.
\href{https://github.com/nleanba/typst-marginalia/blob/v0.1.1/Marginalia.pdf}{Please
refer to the PDF documentation for more details on the configuration and
the provided commands.}

\subsection{Margin-Notes}\label{margin-notes}

Provided via the \texttt{\ \#note{[}...{]}\ } command.

\begin{itemize}
\tightlist
\item
  \texttt{\ \#note(reverse:\ true){[}...{]}\ } to put it on the inside
  margin.
\item
  \texttt{\ \#note(numbered:\ false){[}...{]}\ } to remove the marker.
\end{itemize}

Note: it is recommended to reset the counter for the markers regularly,
e.g. by putting \texttt{\ marginalia.notecounter.update(0)\ } into the
code for your header.

\subsection{Wide Blocks}\label{wide-blocks}

Provided via the \texttt{\ \#wideblock{[}...{]}\ } command.

\begin{itemize}
\tightlist
\item
  \texttt{\ \#wideblock(reverse:\ true){[}...{]}\ } to extend into the
  inside margin instead.
\item
  \texttt{\ \#wideblock(double:\ true){[}...{]}\ } to extend into both
  margins.
\end{itemize}

Note: \texttt{\ reverse\ } and \texttt{\ double\ } are mutually
exclusive.

Note: Wideblocks do not handle pagebreaks in \texttt{\ book:\ true\ }
documents well.

\subsection{Figures}\label{figures}

You can use figures as normal, also within wideblocks. To get captions
on the side, use

\begin{Shaded}
\begin{Highlighting}[]
\NormalTok{\#set figure(gap: 0pt)}
\NormalTok{\#set figure.caption(position: top)}
\NormalTok{\#show figure.caption.where(position: top): note.with(numbered: false, dy: 1em)}
\end{Highlighting}
\end{Shaded}

For small figures, the package also provides a \texttt{\ notefigure\ }
command which places the figure in the margin.

\begin{Shaded}
\begin{Highlighting}[]
\NormalTok{\#marginalia.notefigure(}
\NormalTok{  rect(width: 100\%),}
\NormalTok{  label: \textless{}aaa\textgreater{},}
\NormalTok{  caption: [A notefigure.],}
\NormalTok{)}
\end{Highlighting}
\end{Shaded}

\begin{center}\rule{0.5\linewidth}{0.5pt}\end{center}

\href{https://github.com/nleanba/typst-marginalia/blob/v0.1.1/Marginalia.pdf}{\pandocbounded{\includesvg[keepaspectratio]{https://raw.githubusercontent.com/nleanba/typst-marginalia/refs/tags/v0.1.1/preview.svg}}}

\subsubsection{How to add}\label{how-to-add}

Copy this into your project and use the import as
\texttt{\ marginalia\ }

\begin{verbatim}
#import "@preview/marginalia:0.1.1"
\end{verbatim}

\includesvg[width=0.16667in,height=0.16667in]{/assets/icons/16-copy.svg}

Check the docs for
\href{https://typst.app/docs/reference/scripting/\#packages}{more
information on how to import packages} .

\subsubsection{About}\label{about}

\begin{description}
\tightlist
\item[Author :]
\href{https://github.com/nleanba}{nleanba}
\item[License:]
Unlicense
\item[Current version:]
0.1.1
\item[Last updated:]
November 25, 2024
\item[First released:]
November 19, 2024
\item[Minimum Typst version:]
0.12.0
\item[Archive size:]
6.17 kB
\href{https://packages.typst.org/preview/marginalia-0.1.1.tar.gz}{\pandocbounded{\includesvg[keepaspectratio]{/assets/icons/16-download.svg}}}
\item[Repository:]
\href{https://github.com/nleanba/typst-marginalia}{GitHub}
\item[Categor ies :]
\begin{itemize}
\tightlist
\item[]
\item
  \pandocbounded{\includesvg[keepaspectratio]{/assets/icons/16-layout.svg}}
  \href{https://typst.app/universe/search/?category=layout}{Layout}
\item
  \pandocbounded{\includesvg[keepaspectratio]{/assets/icons/16-hammer.svg}}
  \href{https://typst.app/universe/search/?category=utility}{Utility}
\end{itemize}
\end{description}

\subsubsection{Where to report issues?}\label{where-to-report-issues}

This package is a project of nleanba . Report issues on
\href{https://github.com/nleanba/typst-marginalia}{their repository} .
You can also try to ask for help with this package on the
\href{https://forum.typst.app}{Forum} .

Please report this package to the Typst team using the
\href{https://typst.app/contact}{contact form} if you believe it is a
safety hazard or infringes upon your rights.

\phantomsection\label{versions}
\subsubsection{Version history}\label{version-history}

\begin{longtable}[]{@{}ll@{}}
\toprule\noalign{}
Version & Release Date \\
\midrule\noalign{}
\endhead
\bottomrule\noalign{}
\endlastfoot
0.1.1 & November 25, 2024 \\
\href{https://typst.app/universe/package/marginalia/0.1.0/}{0.1.0} &
November 19, 2024 \\
\end{longtable}

Typst GmbH did not create this package and cannot guarantee correct
functionality of this package or compatibility with any version of the
Typst compiler or app.


\section{Package List LaTeX/bubble.tex}
\title{typst.app/universe/package/bubble}

\phantomsection\label{banner}
\phantomsection\label{template-thumbnail}
\pandocbounded{\includegraphics[keepaspectratio]{https://packages.typst.org/preview/thumbnails/bubble-0.2.2-small.webp}}

\section{bubble}\label{bubble}

{ 0.2.2 }

Simple and colorful template for Typst

\href{/app?template=bubble&version=0.2.2}{Create project in app}

\phantomsection\label{readme}
Simple and colorful template for \href{https://typst.app/}{Typst} . This
template uses a main color (default is \texttt{\ \#E94845\ } ) applied
to list items, links, inline blocks, selected words and headings. Every
page is numbered and has the title of the document and the name of the
author at the top.

You can see an example PDF
\href{https://github.com/hzkonor/bubble-template/blob/main/main.pdf}{here}
.

\subsection{Usage}\label{usage}

You can use this template in the Typst web app by clicking “Start from
template� on the dashboard and searching for \texttt{\ bubble\ } .

Alternatively, you can use the CLI to kick this project off using the
command

\begin{Shaded}
\begin{Highlighting}[]
\ExtensionTok{typst}\NormalTok{ init @preview/bubble}
\end{Highlighting}
\end{Shaded}

Typst will create a new directory with all the files needed to get you
started.

\subsection{Configuration}\label{configuration}

This template exports the \texttt{\ bubble\ } function with the
following named arguments:

\begin{itemize}
\tightlist
\item
  \texttt{\ title\ } : Title of the document
\item
  \texttt{\ subtitle\ } : Subtitle of the document
\item
  \texttt{\ author\ } : Name of the author(s)
\item
  \texttt{\ affiliation\ } : It is supposed to be the name of your
  university for example
\item
  \texttt{\ year\ } : The year you’re in
\item
  \texttt{\ class\ } : For which class this document is
\item
  \texttt{\ other\ } : Array of other information \emph{default is none}
\item
  \texttt{\ date\ } : Date of the document, current date if none is set
  \emph{default is current date}
\item
  \texttt{\ logo\ } : Path of the logo displayed at the top right of the
  title page, must be set like an image :
  \texttt{\ image("path-to-img")\ } \emph{default is none}
\item
  \texttt{\ main-color\ } : Main color used in the document
  \emph{default is \texttt{\ \#E94645\ }}
\item
  \texttt{\ alpha\ } : Percentage of transparency for the bubbles on the
  title page \emph{default is 60\%}
\item
  \texttt{\ color-words\ } : An array of strings that you want to be
  colored automatically in the main-color (be careful to put a trailing
  comma if you have only one string in the array as noted
  \href{https://typst.app/docs/reference/foundations/array/}{here} )
  \emph{default is an empty array}
\end{itemize}

This template also exports these functions :

\begin{itemize}
\tightlist
\item
  \texttt{\ blockquote\ } : Function that highlights quotes with a grey
  bar at the left
\item
  \texttt{\ primary-color\ } : to use your main color
\item
  \texttt{\ secondary-color\ } : to use your secondary color (which is
  your main color with the alpha transparency set)
\end{itemize}

If you want to change an existing project to use this template, you can
add a show rule like this at the top of your file:

\begin{Shaded}
\begin{Highlighting}[]
\NormalTok{\#import "@preview/bubble:0.2.2": *}

\NormalTok{\#show: bubble.with(}
\NormalTok{  title: "Bubble template",}
\NormalTok{  subtitle: "Simple and colorful template",}
\NormalTok{  author: "hzkonor",}
\NormalTok{  affiliation: "University",}
\NormalTok{  date: datetime.today().display(),}
\NormalTok{  year: "Year",}
\NormalTok{  class: "Class",}
\NormalTok{  other: ("Made with Typst", "https://typst.com"),}
\NormalTok{  logo: image("logo.png"),}
\NormalTok{  color{-}words: ("important",)}
\NormalTok{) }

\NormalTok{// Your content goes here}
\end{Highlighting}
\end{Shaded}

\href{/app?template=bubble&version=0.2.2}{Create project in app}

\subsubsection{How to use}\label{how-to-use}

Click the button above to create a new project using this template in
the Typst app.

You can also use the Typst CLI to start a new project on your computer
using this command:

\begin{verbatim}
typst init @preview/bubble:0.2.2
\end{verbatim}

\includesvg[width=0.16667in,height=0.16667in]{/assets/icons/16-copy.svg}

\subsubsection{About}\label{about}

\begin{description}
\tightlist
\item[Author :]
\href{https://github.com/hzkonor}{Conor}
\item[License:]
MIT-0
\item[Current version:]
0.2.2
\item[Last updated:]
October 21, 2024
\item[First released:]
April 16, 2024
\item[Archive size:]
36.2 kB
\href{https://packages.typst.org/preview/bubble-0.2.2.tar.gz}{\pandocbounded{\includesvg[keepaspectratio]{/assets/icons/16-download.svg}}}
\item[Repository:]
\href{https://github.com/hzkonor/bubble-template}{GitHub}
\item[Categor ies :]
\begin{itemize}
\tightlist
\item[]
\item
  \pandocbounded{\includesvg[keepaspectratio]{/assets/icons/16-atom.svg}}
  \href{https://typst.app/universe/search/?category=paper}{Paper}
\item
  \pandocbounded{\includesvg[keepaspectratio]{/assets/icons/16-speak.svg}}
  \href{https://typst.app/universe/search/?category=report}{Report}
\end{itemize}
\end{description}

\subsubsection{Where to report issues?}\label{where-to-report-issues}

This template is a project of Conor . Report issues on
\href{https://github.com/hzkonor/bubble-template}{their repository} .
You can also try to ask for help with this template on the
\href{https://forum.typst.app}{Forum} .

Please report this template to the Typst team using the
\href{https://typst.app/contact}{contact form} if you believe it is a
safety hazard or infringes upon your rights.

\phantomsection\label{versions}
\subsubsection{Version history}\label{version-history}

\begin{longtable}[]{@{}ll@{}}
\toprule\noalign{}
Version & Release Date \\
\midrule\noalign{}
\endhead
\bottomrule\noalign{}
\endlastfoot
0.2.2 & October 21, 2024 \\
\href{https://typst.app/universe/package/bubble/0.2.1/}{0.2.1} & August
2, 2024 \\
\href{https://typst.app/universe/package/bubble/0.2.0/}{0.2.0} & July
23, 2024 \\
\href{https://typst.app/universe/package/bubble/0.1.0/}{0.1.0} & April
16, 2024 \\
\end{longtable}

Typst GmbH did not create this template and cannot guarantee correct
functionality of this template or compatibility with any version of the
Typst compiler or app.


\section{Package List LaTeX/chromo.tex}
\title{typst.app/universe/package/chromo}

\phantomsection\label{banner}
\section{chromo}\label{chromo}

{ 0.1.0 }

Generate printer tests (likely CMYK) in typst.

\phantomsection\label{readme}
Generate printer tests directly in Typst. For now, only generates with
CMYK colors (as it is by far the most used).

I personally place one of these test on all my exam papers to ensure the
printer’s quality over time.

\subsection{Documentation}\label{documentation}

To import any of the functions needed, you may want to use the following
line:

\begin{Shaded}
\begin{Highlighting}[]
\NormalTok{\#import "@preview/chromo:0.1.0": square{-}printer{-}test, gradient{-}printer{-}test, circular{-}printer{-}test, crosshair{-}printer{-}test}
\end{Highlighting}
\end{Shaded}

\subsubsection{Square test}\label{square-test}

\begin{Shaded}
\begin{Highlighting}[]
\NormalTok{\#square{-}printer{-}test()}
\end{Highlighting}
\end{Shaded}

\subsubsection{Gradient test}\label{gradient-test}

\begin{Shaded}
\begin{Highlighting}[]
\NormalTok{\#gradient{-}printer{-}test()}
\end{Highlighting}
\end{Shaded}

\subsubsection{Circular test}\label{circular-test}

\begin{Shaded}
\begin{Highlighting}[]
\NormalTok{\#circular{-}printer{-}test()}
\end{Highlighting}
\end{Shaded}

\subsubsection{Crosshair test}\label{crosshair-test}

\begin{Shaded}
\begin{Highlighting}[]
\NormalTok{\#crosshair{-}printer{-}test()}
\end{Highlighting}
\end{Shaded}

\subsection{Contributors}\label{contributors}

\begin{itemize}
\tightlist
\item
  \href{https://github.com/julien-cpsn}{Julien-cpsn}
\end{itemize}

\subsubsection{How to add}\label{how-to-add}

Copy this into your project and use the import as \texttt{\ chromo\ }

\begin{verbatim}
#import "@preview/chromo:0.1.0"
\end{verbatim}

\includesvg[width=0.16667in,height=0.16667in]{/assets/icons/16-copy.svg}

Check the docs for
\href{https://typst.app/docs/reference/scripting/\#packages}{more
information on how to import packages} .

\subsubsection{About}\label{about}

\begin{description}
\tightlist
\item[Author :]
Julien Caposiena
\item[License:]
MIT
\item[Current version:]
0.1.0
\item[Last updated:]
January 29, 2024
\item[First released:]
January 29, 2024
\item[Archive size:]
2.00 kB
\href{https://packages.typst.org/preview/chromo-0.1.0.tar.gz}{\pandocbounded{\includesvg[keepaspectratio]{/assets/icons/16-download.svg}}}
\item[Repository:]
\href{https://github.com/julien-cpsn/typst-chromo}{GitHub}
\end{description}

\subsubsection{Where to report issues?}\label{where-to-report-issues}

This package is a project of Julien Caposiena . Report issues on
\href{https://github.com/julien-cpsn/typst-chromo}{their repository} .
You can also try to ask for help with this package on the
\href{https://forum.typst.app}{Forum} .

Please report this package to the Typst team using the
\href{https://typst.app/contact}{contact form} if you believe it is a
safety hazard or infringes upon your rights.

\phantomsection\label{versions}
\subsubsection{Version history}\label{version-history}

\begin{longtable}[]{@{}ll@{}}
\toprule\noalign{}
Version & Release Date \\
\midrule\noalign{}
\endhead
\bottomrule\noalign{}
\endlastfoot
0.1.0 & January 29, 2024 \\
\end{longtable}

Typst GmbH did not create this package and cannot guarantee correct
functionality of this package or compatibility with any version of the
Typst compiler or app.


\section{Package List LaTeX/wavy.tex}
\title{typst.app/universe/package/wavy}

\phantomsection\label{banner}
\section{wavy}\label{wavy}

{ 0.1.1 }

Draw digital timing diagram in Typst using Wavedrom.

\phantomsection\label{readme}
Draw digital timing diagram in Typst using
\href{https://wavedrom.com/}{Wavedrom} .

\pandocbounded{\includesvg[keepaspectratio]{https://github.com/typst/packages/raw/main/packages/preview/wavy/0.1.1/wavy.svg}}

\begin{Shaded}
\begin{Highlighting}[]
\NormalTok{\#import "@preview/wavy:0.1.1"}

\NormalTok{\#set page(height: auto, width: auto, fill: black, margin: 2em)}
\NormalTok{\#set text(fill: white)}

\NormalTok{\#show raw.where(lang: "wavy"): it =\textgreater{} wavy.render(it.text)}

\NormalTok{= Wavy}

\NormalTok{Typst, now with waves.}

\NormalTok{\textasciigrave{}\textasciigrave{}\textasciigrave{}wavy}
\NormalTok{\{}
\NormalTok{  signal:}
\NormalTok{  [}
\NormalTok{    \{name:\textquotesingle{}clk\textquotesingle{},wave:\textquotesingle{}p......\textquotesingle{}\},}
\NormalTok{    \{name:\textquotesingle{}bus\textquotesingle{},wave:\textquotesingle{}x.34.5x\textquotesingle{},data:\textquotesingle{}head body tail\textquotesingle{}\},}
\NormalTok{    \{name:\textquotesingle{}wire\textquotesingle{},wave:\textquotesingle{}0.1..0.\textquotesingle{}\}}
\NormalTok{  ]}
\NormalTok{\}}
\NormalTok{\textasciigrave{}\textasciigrave{}\textasciigrave{}}

\NormalTok{\textasciigrave{}\textasciigrave{}\textasciigrave{}js}
\NormalTok{\{}
\NormalTok{  signal:}
\NormalTok{  [}
\NormalTok{    \{name:\textquotesingle{}clk\textquotesingle{},wave:\textquotesingle{}p......\textquotesingle{}\},}
\NormalTok{    \{name:\textquotesingle{}bus\textquotesingle{},wave:\textquotesingle{}x.34.5x\textquotesingle{},data:\textquotesingle{}head body tail\textquotesingle{}\},}
\NormalTok{    \{name:\textquotesingle{}wire\textquotesingle{},wave:\textquotesingle{}0.1..0.\textquotesingle{}\}}
\NormalTok{  ]}
\NormalTok{\}}
\NormalTok{\textasciigrave{}\textasciigrave{}\textasciigrave{}}
\end{Highlighting}
\end{Shaded}

\subsection{Documentation}\label{documentation}

\subsubsection{\texorpdfstring{\texttt{\ render\ }}{ render }}\label{render}

Render a wavedrom json5 string to an image

\paragraph{Arguments}\label{arguments}

\begin{itemize}
\tightlist
\item
  \texttt{\ src\ } : \texttt{\ str\ } - wavedrom json5 string
\item
  All other arguments are passed to \texttt{\ image.decode\ } so you can
  customize the image size
\end{itemize}

\paragraph{Returns}\label{returns}

The image, of type \texttt{\ content\ }

\subsubsection{How to add}\label{how-to-add}

Copy this into your project and use the import as \texttt{\ wavy\ }

\begin{verbatim}
#import "@preview/wavy:0.1.1"
\end{verbatim}

\includesvg[width=0.16667in,height=0.16667in]{/assets/icons/16-copy.svg}

Check the docs for
\href{https://typst.app/docs/reference/scripting/\#packages}{more
information on how to import packages} .

\subsubsection{About}\label{about}

\begin{description}
\tightlist
\item[Author :]
Wenzhuo Liu
\item[License:]
MIT
\item[Current version:]
0.1.1
\item[Last updated:]
December 3, 2023
\item[First released:]
November 8, 2023
\item[Archive size:]
42.9 kB
\href{https://packages.typst.org/preview/wavy-0.1.1.tar.gz}{\pandocbounded{\includesvg[keepaspectratio]{/assets/icons/16-download.svg}}}
\item[Repository:]
\href{https://github.com/Enter-tainer/wavy}{GitHub}
\end{description}

\subsubsection{Where to report issues?}\label{where-to-report-issues}

This package is a project of Wenzhuo Liu . Report issues on
\href{https://github.com/Enter-tainer/wavy}{their repository} . You can
also try to ask for help with this package on the
\href{https://forum.typst.app}{Forum} .

Please report this package to the Typst team using the
\href{https://typst.app/contact}{contact form} if you believe it is a
safety hazard or infringes upon your rights.

\phantomsection\label{versions}
\subsubsection{Version history}\label{version-history}

\begin{longtable}[]{@{}ll@{}}
\toprule\noalign{}
Version & Release Date \\
\midrule\noalign{}
\endhead
\bottomrule\noalign{}
\endlastfoot
0.1.1 & December 3, 2023 \\
\href{https://typst.app/universe/package/wavy/0.1.0/}{0.1.0} & November
8, 2023 \\
\end{longtable}

Typst GmbH did not create this package and cannot guarantee correct
functionality of this package or compatibility with any version of the
Typst compiler or app.


\section{Package List LaTeX/numbly.tex}
\title{typst.app/universe/package/numbly}

\phantomsection\label{banner}
\section{numbly}\label{numbly}

{ 0.1.0 }

A package that helps you to specify different numbering formats for
different levels of headings.

\phantomsection\label{readme}
A package that helps you to specify different numbering formats for
different levels of headings.

Suppose you want to specify the following numbering format for your
document:

\begin{itemize}
\tightlist
\item
  Appendix A. Guide

  \begin{itemize}
  \tightlist
  \item
    A.1. Installation

    \begin{itemize}
    \tightlist
    \item
      Step 1. Download
    \item
      Step 2. Install
    \end{itemize}
  \item
    A.2. Usage
  \end{itemize}
\end{itemize}

You might use \texttt{\ if\ } to achieve this:

\begin{Shaded}
\begin{Highlighting}[]
\NormalTok{\#set heading(numbering: (..nums) =\textgreater{} \{}
\NormalTok{  nums = nums.pos()}
\NormalTok{  if nums.len() == 1 \{}
\NormalTok{    return "Appendix " + numbering("A.", ..nums)}
\NormalTok{  \} else if nums.len() == 2 \{}
\NormalTok{    return numbering("A.1.", ..nums)}
\NormalTok{  \} else \{}
\NormalTok{    return "Step " + numbering("1.", nums.last())}
\NormalTok{  \}}
\NormalTok{\})}

\NormalTok{= Guide}
\NormalTok{== Installation}
\NormalTok{=== Download}
\NormalTok{=== Install}
\NormalTok{== Usage}
\end{Highlighting}
\end{Shaded}

But with \texttt{\ numbly\ } , you can do this more easily:

\begin{Shaded}
\begin{Highlighting}[]
\NormalTok{\#import "@preview/numbly:0.1.0": numbly}
\NormalTok{\#set heading(numbering: numbly(}
\NormalTok{  "Appendix \{1:A\}.", // use \{level:format\} to specify the format}
\NormalTok{  "\{1:A\}.\{2\}.", // if format is not specified, arabic numbers will be used}
\NormalTok{  "Step \{3\}.", // here, we only want the 3rd level}
\NormalTok{))}
\end{Highlighting}
\end{Shaded}

\subsubsection{How to add}\label{how-to-add}

Copy this into your project and use the import as \texttt{\ numbly\ }

\begin{verbatim}
#import "@preview/numbly:0.1.0"
\end{verbatim}

\includesvg[width=0.16667in,height=0.16667in]{/assets/icons/16-copy.svg}

Check the docs for
\href{https://typst.app/docs/reference/scripting/\#packages}{more
information on how to import packages} .

\subsubsection{About}\label{about}

\begin{description}
\tightlist
\item[Author :]
\href{https://github.com/flaribbit}{flaribbit}
\item[License:]
MIT
\item[Current version:]
0.1.0
\item[Last updated:]
July 1, 2024
\item[First released:]
July 1, 2024
\item[Archive size:]
1.75 kB
\href{https://packages.typst.org/preview/numbly-0.1.0.tar.gz}{\pandocbounded{\includesvg[keepaspectratio]{/assets/icons/16-download.svg}}}
\item[Repository:]
\href{https://github.com/flaribbit/numbly}{GitHub}
\item[Categor y :]
\begin{itemize}
\tightlist
\item[]
\item
  \pandocbounded{\includesvg[keepaspectratio]{/assets/icons/16-hammer.svg}}
  \href{https://typst.app/universe/search/?category=utility}{Utility}
\end{itemize}
\end{description}

\subsubsection{Where to report issues?}\label{where-to-report-issues}

This package is a project of flaribbit . Report issues on
\href{https://github.com/flaribbit/numbly}{their repository} . You can
also try to ask for help with this package on the
\href{https://forum.typst.app}{Forum} .

Please report this package to the Typst team using the
\href{https://typst.app/contact}{contact form} if you believe it is a
safety hazard or infringes upon your rights.

\phantomsection\label{versions}
\subsubsection{Version history}\label{version-history}

\begin{longtable}[]{@{}ll@{}}
\toprule\noalign{}
Version & Release Date \\
\midrule\noalign{}
\endhead
\bottomrule\noalign{}
\endlastfoot
0.1.0 & July 1, 2024 \\
\end{longtable}

Typst GmbH did not create this package and cannot guarantee correct
functionality of this package or compatibility with any version of the
Typst compiler or app.


\section{Package List LaTeX/rubby.tex}
\title{typst.app/universe/package/rubby}

\phantomsection\label{banner}
\section{rubby}\label{rubby}

{ 0.10.1 }

Add ruby (furigana) next to base text.

\phantomsection\label{readme}
\subsection{Usage}\label{usage}

\begin{Shaded}
\begin{Highlighting}[]
\NormalTok{\#import "@preview/rubby:0.10.1": get{-}ruby}

\NormalTok{\#let ruby = get{-}ruby(}
\NormalTok{  size: 0.5em,         // Ruby font size}
\NormalTok{  dy: 0pt,             // Vertical offset of the ruby}
\NormalTok{  pos: top,            // Ruby position (top or bottom)}
\NormalTok{  alignment: "center", // Ruby alignment ("center", "start", "between", "around")}
\NormalTok{  delimiter: "|",      // The delimiter between words}
\NormalTok{  auto{-}spacing: true,  // Automatically add necessary space around words}
\NormalTok{)}

\NormalTok{// Ruby goes first, base text {-} second.}
\NormalTok{\#ruby[ふりがな][振り仮名]}

\NormalTok{Treat each kanji as a separate word:}
\NormalTok{\#ruby[とう|きょう|こう|ぎょう|だい|がく][東|京|工|業|大|学]}
\end{Highlighting}
\end{Shaded}

If you don’t want automatically wrap text with delimiter:

\begin{Shaded}
\begin{Highlighting}[]
\NormalTok{\#let ruby = get{-}ruby(auto{-}spacing: false)}
\end{Highlighting}
\end{Shaded}

See also \url{https://github.com/rinmyo/ruby-typ/blob/main/manual.pdf}
and \texttt{\ example.typ\ } .

\subsection{Notes}\label{notes}

Original project is at \url{https://github.com/rinmyo/ruby-typ} which
itself is based on
\href{https://zenn.dev/saito_atsushi/articles/ff9490458570e1}{the post}
of 齊è---¤æ•¦å¿--- (Saito Atsushi). This project is a modified version
of
\href{https://github.com/rinmyo/ruby-typ/commit/23ca86180757cf70f2b9f5851abb5ea5a3b4c6a1}{this
commit} .

\texttt{\ auto-spacing\ } adds missing delimiter around the
\texttt{\ content\ } / \texttt{\ string\ } which then adds space around
base text if ruby is wider than the base text.

Problems appear only if ruby is wider than its base text and
\texttt{\ auto-spacing\ } is not set to \texttt{\ true\ } (default is
\texttt{\ true\ } ).

You can always use a one-letter function (variable) name to shorten the
function call length (if you have to use it a lot), e.g.,
\texttt{\ \#let\ r\ =\ get-ruby()\ } (or \texttt{\ f\ } â€'' short for
furigana). But be careful as there are functions with names
\texttt{\ v\ } and \texttt{\ h\ } and there could be a new built-in
function with a name \texttt{\ r\ } or \texttt{\ f\ } which may break
your document (Typst right now is in beta, so breaking changes are
possible).

Although you can open issues or send PRs, I won’t be able to always
reply quickly (sometimes I’m very busy).

\subsection{Development}\label{development}

This repository should exist as a \texttt{\ @local\ } package with the
version from the \texttt{\ typst.toml\ } .

Here is a short description of the development process:

\begin{enumerate}
\tightlist
\item
  run \texttt{\ git\ checkout\ dev\ \&\&\ git\ pull\ } ;
\item
  make changes;
\item
  test changes, if not done or something isn’t working then go to step
  1;
\item
  when finished, run
  \texttt{\ just\ change-version\ \textless{}new\ semantic\ version\textgreater{}\ }
  ;
\item
  document changes in the \texttt{\ CHANGELOG.md\ } ;
\item
  commit all changes (only locally);
\item
  create a \texttt{\ @local\ } Typst package with the new version and
  test it;
\item
  if everything is working then run \texttt{\ git\ push\ } ;
\item
  realize that you’ve missed something and fix it (then push changes
  again);
\item
  run \texttt{\ git\ checkout\ master\ \&\&\ git\ merge\ dev\ } to sync
  \texttt{\ master\ } to \texttt{\ dev\ } ;
\item
  run \texttt{\ just\ create-release\ } .
\end{enumerate}

\subsection{Publishing a Typst
package}\label{publishing-a-typst-package}

\begin{enumerate}
\tightlist
\item
  To make a new package version for merging into
  \texttt{\ typst/packages\ } repository run
  \texttt{\ just\ mark-PR-version\ } ;
\item
  copy newly created directory (with a version name) and place it in the
  appropriate place in your fork of the \texttt{\ typst/packages\ }
  repository;
\item
  run
  \texttt{\ git\ fetch\ upstream\ \&\&\ git\ merge\ upstream\ main\ } to
  sync fork with \texttt{\ typst/packages\ } ;
\item
  go to a new branch with
  \texttt{\ git\ checkout\ -b\ \textless{}package-version\textgreater{}\ }
  ;
\item
  commit newly added directory with commit message:
  \texttt{\ package:version\ } ;
\item
  run \texttt{\ gh\ pr\ create\ } and follow further CLI instructions.
\end{enumerate}

\subsection{Changelog}\label{changelog}

You can view the change log in the \texttt{\ CHANGELOG.md\ } file in the
root of the project.

\subsection{License}\label{license}

This Typst package is licensed under AGPL v3.0. You can view the license
in the LICENSE file in the root of the project or at
\url{https://www.gnu.org/licenses/agpl-3.0.txt} . There is also a NOTICE
file for 3rd party copyright notices.

Copyright © 2023 Andrew Voynov

\subsubsection{How to add}\label{how-to-add}

Copy this into your project and use the import as \texttt{\ rubby\ }

\begin{verbatim}
#import "@preview/rubby:0.10.1"
\end{verbatim}

\includesvg[width=0.16667in,height=0.16667in]{/assets/icons/16-copy.svg}

Check the docs for
\href{https://typst.app/docs/reference/scripting/\#packages}{more
information on how to import packages} .

\subsubsection{About}\label{about}

\begin{description}
\tightlist
\item[Author :]
Andrew Voynov
\item[License:]
AGPL-3.0-only
\item[Current version:]
0.10.1
\item[Last updated:]
December 3, 2023
\item[First released:]
July 3, 2023
\item[Minimum Typst version:]
0.8.0
\item[Archive size:]
16.0 kB
\href{https://packages.typst.org/preview/rubby-0.10.1.tar.gz}{\pandocbounded{\includesvg[keepaspectratio]{/assets/icons/16-download.svg}}}
\item[Repository:]
\href{https://github.com/Andrew15-5/rubby}{GitHub}
\end{description}

\subsubsection{Where to report issues?}\label{where-to-report-issues}

This package is a project of Andrew Voynov . Report issues on
\href{https://github.com/Andrew15-5/rubby}{their repository} . You can
also try to ask for help with this package on the
\href{https://forum.typst.app}{Forum} .

Please report this package to the Typst team using the
\href{https://typst.app/contact}{contact form} if you believe it is a
safety hazard or infringes upon your rights.

\phantomsection\label{versions}
\subsubsection{Version history}\label{version-history}

\begin{longtable}[]{@{}ll@{}}
\toprule\noalign{}
Version & Release Date \\
\midrule\noalign{}
\endhead
\bottomrule\noalign{}
\endlastfoot
\href{https://typst.app/universe/package/rubby/0.9.2/}{0.9.2} &
September 15, 2023 \\
\href{https://typst.app/universe/package/rubby/0.8.0/}{0.8.0} & July 3,
2023 \\
0.10.1 & December 3, 2023 \\
\href{https://typst.app/universe/package/rubby/0.10.0/}{0.10.0} &
November 25, 2023 \\
\end{longtable}

Typst GmbH did not create this package and cannot guarantee correct
functionality of this package or compatibility with any version of the
Typst compiler or app.


\section{Package List LaTeX/pigmentpedia.tex}
\title{typst.app/universe/package/pigmentpedia}

\phantomsection\label{banner}
\section{pigmentpedia}\label{pigmentpedia}

{ 0.1.0 }

An extended color library for Typst.

\phantomsection\label{readme}
An extended color library for Typst. Most of these colors are commonly
used on the web.

\subsection{Usage}\label{usage}

Add the package with the following code (remember to add the asterisk
\texttt{\ :\ *\ } at the end) and pick a color.

\begin{Shaded}
\begin{Highlighting}[]
\NormalTok{\#include "@preview/pigmentpedia:0.1.0": *}

\NormalTok{\#set text(firebrick)}

\NormalTok{This text has a firebrick color.}
\end{Highlighting}
\end{Shaded}

There are over 100 different colors currently available and more will be
added.

\subsubsection{How to add}\label{how-to-add}

Copy this into your project and use the import as
\texttt{\ pigmentpedia\ }

\begin{verbatim}
#import "@preview/pigmentpedia:0.1.0"
\end{verbatim}

\includesvg[width=0.16667in,height=0.16667in]{/assets/icons/16-copy.svg}

Check the docs for
\href{https://typst.app/docs/reference/scripting/\#packages}{more
information on how to import packages} .

\subsubsection{About}\label{about}

\begin{description}
\tightlist
\item[Author :]
\href{https://github.com/neuralpain}{neuralpain}
\item[License:]
MIT
\item[Current version:]
0.1.0
\item[Last updated:]
November 29, 2024
\item[First released:]
November 29, 2024
\item[Archive size:]
2.75 kB
\href{https://packages.typst.org/preview/pigmentpedia-0.1.0.tar.gz}{\pandocbounded{\includesvg[keepaspectratio]{/assets/icons/16-download.svg}}}
\item[Repository:]
\href{https://github.com/neuralpain/pigmentpedia}{GitHub}
\item[Categor ies :]
\begin{itemize}
\tightlist
\item[]
\item
  \pandocbounded{\includesvg[keepaspectratio]{/assets/icons/16-hammer.svg}}
  \href{https://typst.app/universe/search/?category=utility}{Utility}
\item
  \pandocbounded{\includesvg[keepaspectratio]{/assets/icons/16-text.svg}}
  \href{https://typst.app/universe/search/?category=text}{Text}
\end{itemize}
\end{description}

\subsubsection{Where to report issues?}\label{where-to-report-issues}

This package is a project of neuralpain . Report issues on
\href{https://github.com/neuralpain/pigmentpedia}{their repository} .
You can also try to ask for help with this package on the
\href{https://forum.typst.app}{Forum} .

Please report this package to the Typst team using the
\href{https://typst.app/contact}{contact form} if you believe it is a
safety hazard or infringes upon your rights.

\phantomsection\label{versions}
\subsubsection{Version history}\label{version-history}

\begin{longtable}[]{@{}ll@{}}
\toprule\noalign{}
Version & Release Date \\
\midrule\noalign{}
\endhead
\bottomrule\noalign{}
\endlastfoot
0.1.0 & November 29, 2024 \\
\end{longtable}

Typst GmbH did not create this package and cannot guarantee correct
functionality of this package or compatibility with any version of the
Typst compiler or app.


\section{Package List LaTeX/iconic-salmon-fa.tex}
\title{typst.app/universe/package/iconic-salmon-fa}

\phantomsection\label{banner}
\section{iconic-salmon-fa}\label{iconic-salmon-fa}

{ 1.0.0 }

A Typst library for Social Media references with icons based on Font
Awesome.

\phantomsection\label{readme}
The \texttt{\ iconic-salmon-fa\ } package is designed to help you create
your curriculum vitae (CV). It allows you to easily reference your
social media profiles with the typical icon of the service plus a link
to your profile.\\
The package name is a combination of the acronym \emph{SociAL Media
icONs} and the word \emph{iconic} because all these icons have an iconic
design (and iconic also contains the word \emph{icon} ).

\subsection{Features}\label{features}

\begin{itemize}
\tightlist
\item
  Support for popular social media, developer and career platforms
\item
  Uniform design for all entries
\item
  Based on the Internet’s icon library
  \href{https://fontawesome.com/}{Font Awesome}
\item
  Easy to use
\item
  Allows the customization of the look (extra args are passed to
  \href{https://typst.app/docs/reference/text/text/}{\texttt{\ text\ }}
  )
\end{itemize}

\subsection{Fonts Installation}\label{fonts-installation}

\subsubsection{Linux}\label{linux}

\begin{enumerate}
\tightlist
\item
  \href{https://fontawesome.com/download}{Download Font Awesome for
  Desktop}
\item
  Unzip the file
\item
  Switch into the \texttt{\ otfs\ } folder within the unzipped folder
\item
  Run \texttt{\ mkdir\ -p\ /usr/share/fonts/truetype/\ }
\item
  Run
  \texttt{\ install\ -m644\ \textquotesingle{}Font\ Awesome\ 6\ Brands-Regular-400.otf\textquotesingle{}\ /usr/share/fonts/truetype/\ }
\item
  Unfortunately not all brands are included in the brands font file, so
  you must also run
  \texttt{\ install\ -m644\ \textquotesingle{}Font\ Awesome\ 6\ Free-Regular-400.otf\textquotesingle{}\ /usr/share/fonts/truetype/\ }
\end{enumerate}

\subsection{Usage}\label{usage}

\subsubsection{Using Typst’s package
manager}\label{using-typstuxe2s-package-manager}

You can install the library using the
\href{https://github.com/typst/packages}{typst packages} :

\begin{Shaded}
\begin{Highlighting}[]
\NormalTok{\#import "@preview/iconic{-}salmon{-}fa:1.0.0": *}
\end{Highlighting}
\end{Shaded}

\subsubsection{Install manually}\label{install-manually}

Put the \texttt{\ iconic-salmon-fa.typ\ } file in your project directory
and import it:

\begin{Shaded}
\begin{Highlighting}[]
\NormalTok{\#import "iconic{-}salmon{-}fa.typ": *}
\end{Highlighting}
\end{Shaded}

\subsubsection{Minimal Example}\label{minimal-example}

\begin{Shaded}
\begin{Highlighting}[]
\NormalTok{// \#import "@preview/iconic{-}salmon{-}fa:1.0.0": github{-}info, gitlab{-}info}
\NormalTok{\#import "iconic{-}salmon{-}fa.typ": github{-}info, gitlab{-}info}

\NormalTok{This project was created by \#github{-}info("Bi0T1N"). You can also find me on \#gitlab{-}info("GitLab", rgb("\#811052"), url: "https://gitlab.com/Bi0T1N").}
\end{Highlighting}
\end{Shaded}

\subsubsection{Examples}\label{examples}

See the
\href{https://github.com/typst/packages/raw/main/packages/preview/iconic-salmon-fa/1.0.0/examples/examples.typ}{\texttt{\ examples.typ\ }}
file for a complete example. The
\href{https://github.com/typst/packages/raw/main/packages/preview/iconic-salmon-fa/1.0.0/examples/}{generated
PDF files} are also available for preview.

\subsection{Troubleshooting}\label{troubleshooting}

\subsubsection{Icons are not displayed
correctly}\label{icons-are-not-displayed-correctly}

Make sure that you have installed the required Font Awesome
ligature-based font files.

\subsection{Contribution}\label{contribution}

Feel free to open an issue or a pull request if you find any problems or
have any suggestions.

\subsection{License}\label{license}

This library is licensed under the MIT license. Feel free to use it in
your project.

\subsection{Trademark Disclaimer}\label{trademark-disclaimer}

Product names, logos, brands and other trademarks referred to in this
project are the property of their respective trademark holders.\\
These trademark holders are not affiliated with this Typst library, nor
are the authors officially endorsed by them, nor do the authors claim
ownership of these trademarks.

\subsubsection{How to add}\label{how-to-add}

Copy this into your project and use the import as
\texttt{\ iconic-salmon-fa\ }

\begin{verbatim}
#import "@preview/iconic-salmon-fa:1.0.0"
\end{verbatim}

\includesvg[width=0.16667in,height=0.16667in]{/assets/icons/16-copy.svg}

Check the docs for
\href{https://typst.app/docs/reference/scripting/\#packages}{more
information on how to import packages} .

\subsubsection{About}\label{about}

\begin{description}
\tightlist
\item[Author :]
Nico Neumann (Bi0T1N)
\item[License:]
MIT
\item[Current version:]
1.0.0
\item[Last updated:]
May 16, 2024
\item[First released:]
May 16, 2024
\item[Archive size:]
3.32 kB
\href{https://packages.typst.org/preview/iconic-salmon-fa-1.0.0.tar.gz}{\pandocbounded{\includesvg[keepaspectratio]{/assets/icons/16-download.svg}}}
\item[Repository:]
\href{https://github.com/Bi0T1N/typst-iconic-salmon-fa}{GitHub}
\item[Categor y :]
\begin{itemize}
\tightlist
\item[]
\item
  \pandocbounded{\includesvg[keepaspectratio]{/assets/icons/16-package.svg}}
  \href{https://typst.app/universe/search/?category=components}{Components}
\end{itemize}
\end{description}

\subsubsection{Where to report issues?}\label{where-to-report-issues}

This package is a project of Nico Neumann (Bi0T1N) . Report issues on
\href{https://github.com/Bi0T1N/typst-iconic-salmon-fa}{their
repository} . You can also try to ask for help with this package on the
\href{https://forum.typst.app}{Forum} .

Please report this package to the Typst team using the
\href{https://typst.app/contact}{contact form} if you believe it is a
safety hazard or infringes upon your rights.

\phantomsection\label{versions}
\subsubsection{Version history}\label{version-history}

\begin{longtable}[]{@{}ll@{}}
\toprule\noalign{}
Version & Release Date \\
\midrule\noalign{}
\endhead
\bottomrule\noalign{}
\endlastfoot
1.0.0 & May 16, 2024 \\
\end{longtable}

Typst GmbH did not create this package and cannot guarantee correct
functionality of this package or compatibility with any version of the
Typst compiler or app.


\section{Package List LaTeX/bytefield.tex}
\title{typst.app/universe/package/bytefield}

\phantomsection\label{banner}
\section{bytefield}\label{bytefield}

{ 0.0.6 }

A package to create network protocol headers, memory map, register
definitions and more.

\phantomsection\label{readme}
A simple way to create network protocol headers, memory maps, register
definitions and more in typst.

âš~ï¸? Warning. As this package is still in an early stage, things might
break with the next version.

ℹ� If you find a bug or a feature which is missing, please open an
issue and/or send an PR.

\subsection{Example}\label{example}

\pandocbounded{\includegraphics[keepaspectratio]{https://github.com/typst/packages/raw/main/packages/preview/bytefield/0.0.6/docs/bytefield_example.png}}

\begin{Shaded}
\begin{Highlighting}[]
\NormalTok{\#import "@preview/bytefield:0.0.6": *}

\NormalTok{\#bytefield(}
\NormalTok{// Config the header}
\NormalTok{bitheader(}
\NormalTok{"bytes",}
\NormalTok{// adds every multiple of 8 to the header.}
\NormalTok{0, [start], // number with label}
\NormalTok{5,}
\NormalTok{// number without label}
\NormalTok{12, [\#text(14pt, fill: red, "test")],}
\NormalTok{23, [end\_test],}
\NormalTok{24, [start\_break],}
\NormalTok{36, [Fix], // will not be shown}
\NormalTok{angle: {-}50deg, // angle (default: {-}60deg)}
\NormalTok{text{-}size: 8pt, // length (default: global header\_font\_size or 9pt)}
\NormalTok{),}
\NormalTok{// Add data fields (bit, bits, byte, bytes) and notes}
\NormalTok{// A note always aligns on the same row as the start of the next data field.}
\NormalTok{note(left)[\#text(16pt, fill: blue, font: "Consolas", "Testing")],}
\NormalTok{bytes(3,fill: red.lighten(30\%))[Test],}
\NormalTok{note(right)[\#set text(9pt); \#sym.arrow.l This field \textbackslash{} breaks into 2 rows.],}
\NormalTok{bytes(2)[Break],}
\NormalTok{note(left)[\#set text(9pt); and continues \textbackslash{} here \#sym.arrow],}
\NormalTok{bits(24,fill: green.lighten(30\%))[Fill],}
\NormalTok{group(right,3)[spanning 3 rows],}
\NormalTok{bytes(12)[\#set text(20pt); *Multi* Row],}
\NormalTok{note(left, bracket: true)[Flags],}
\NormalTok{bits(4)[\#text(8pt)[reserved]],}
\NormalTok{flag[\#text(8pt)[SYN]],}
\NormalTok{flag(fill: orange.lighten(60\%))[\#text(8pt)[ACK]],}
\NormalTok{flag[\#text(8pt)[BOB]],}
\NormalTok{bits(25, fill: purple.lighten(60\%))[Padding],}
\NormalTok{// padding(fill: purple.lighten(40\%))[Padding],}
\NormalTok{bytes(2)[Next],}
\NormalTok{bytes(8, fill: yellow.lighten(60\%))[Multi break],}
\NormalTok{note(right)[\#emoji.checkmark Finish],}
\NormalTok{bytes(2)[\_End\_],}
\NormalTok{)}
\end{Highlighting}
\end{Shaded}

\subsection{Usage}\label{usage}

To use this library through the Typst package manager import bytefield
with \texttt{\ \#import\ "@preview/bytefield:0.0.6":\ *\ } at the top of
your file.

The package contains some of the most common network protocol headers
which are available under: \texttt{\ common.ipv4\ } ,
\texttt{\ common.ipv6\ } , \texttt{\ common.icmp\ } ,
\texttt{\ common.icmpv6\ } , \texttt{\ common.dns\ } ,
\texttt{\ common.tcp\ } , \texttt{\ common.udp\ } .

\subsection{Features}\label{features}

Here is a unsorted list of features which is possible right now.

\begin{itemize}
\tightlist
\item
  Adding fields with \texttt{\ bit\ } , \texttt{\ bits\ } ,
  \texttt{\ byte\ } or \texttt{\ bytes\ } function.

  \begin{itemize}
  \tightlist
  \item
    Fields can be colored
  \item
    Multirow and breaking fields are supported.
  \end{itemize}
\item
  Adding notes to the left or right with \texttt{\ note\ } or
  \texttt{\ group\ } function.
\item
  Config the header with the \texttt{\ bitheader\ } function. !Only one
  header per bytefield is processed currently.

  \begin{itemize}
  \tightlist
  \item
    Show numbers
  \item
    Show numbers and labels
  \item
    Show only labels
  \end{itemize}
\item
  Change the bit order in the header with \texttt{\ msb:left\ } or
  \texttt{\ msb:right\ } (default)
\end{itemize}

See
\href{https://github.com/typst/packages/raw/main/packages/preview/bytefield/0.0.6/example.typ}{example.typ}
for more information.

See
\href{https://github.com/typst/packages/raw/main/packages/preview/bytefield/0.0.6/CHANGELOG.md}{CHANGELOG.md}

\subsubsection{How to add}\label{how-to-add}

Copy this into your project and use the import as \texttt{\ bytefield\ }

\begin{verbatim}
#import "@preview/bytefield:0.0.6"
\end{verbatim}

\includesvg[width=0.16667in,height=0.16667in]{/assets/icons/16-copy.svg}

Check the docs for
\href{https://typst.app/docs/reference/scripting/\#packages}{more
information on how to import packages} .

\subsubsection{About}\label{about}

\begin{description}
\tightlist
\item[Author :]
\href{https://github.com/jomaway}{Jomaway}
\item[License:]
MIT
\item[Current version:]
0.0.6
\item[Last updated:]
May 24, 2024
\item[First released:]
September 3, 2023
\item[Minimum Typst version:]
0.10.0
\item[Archive size:]
12.0 kB
\href{https://packages.typst.org/preview/bytefield-0.0.6.tar.gz}{\pandocbounded{\includesvg[keepaspectratio]{/assets/icons/16-download.svg}}}
\item[Repository:]
\href{https://github.com/jomaway/typst-bytefield}{GitHub}
\end{description}

\subsubsection{Where to report issues?}\label{where-to-report-issues}

This package is a project of Jomaway . Report issues on
\href{https://github.com/jomaway/typst-bytefield}{their repository} .
You can also try to ask for help with this package on the
\href{https://forum.typst.app}{Forum} .

Please report this package to the Typst team using the
\href{https://typst.app/contact}{contact form} if you believe it is a
safety hazard or infringes upon your rights.

\phantomsection\label{versions}
\subsubsection{Version history}\label{version-history}

\begin{longtable}[]{@{}ll@{}}
\toprule\noalign{}
Version & Release Date \\
\midrule\noalign{}
\endhead
\bottomrule\noalign{}
\endlastfoot
0.0.6 & May 24, 2024 \\
\href{https://typst.app/universe/package/bytefield/0.0.5/}{0.0.5} &
March 11, 2024 \\
\href{https://typst.app/universe/package/bytefield/0.0.4/}{0.0.4} &
February 21, 2024 \\
\href{https://typst.app/universe/package/bytefield/0.0.3/}{0.0.3} &
November 20, 2023 \\
\href{https://typst.app/universe/package/bytefield/0.0.2/}{0.0.2} &
October 27, 2023 \\
\href{https://typst.app/universe/package/bytefield/0.0.1/}{0.0.1} &
September 3, 2023 \\
\end{longtable}

Typst GmbH did not create this package and cannot guarantee correct
functionality of this package or compatibility with any version of the
Typst compiler or app.


\section{Package List LaTeX/cartao.tex}
\title{typst.app/universe/package/cartao}

\phantomsection\label{banner}
\section{cartao}\label{cartao}

{ 0.1.0 }

Dead simple flashcards with Typst.

\phantomsection\label{readme}
Dead simple flashcards with Typst.

\subsection{Example usage:}\label{example-usage}

\begin{Shaded}
\begin{Highlighting}[]
\NormalTok{\#import "@preview/cartao:0.1.0": card, letter8up, a48up}

\NormalTok{\#set page(}
\NormalTok{  paper: "a4",}
\NormalTok{  // paper: "us{-}letter",}
\NormalTok{  // paper: "presentation{-}16{-}9",}
\NormalTok{  margin: (x: 0cm, y: 0cm),}
\NormalTok{)}

\NormalTok{// build the cards}
\NormalTok{\#a48up}
\NormalTok{// \#letter8up}
\NormalTok{// \#present}

\NormalTok{// define your cards}
\NormalTok{\#card(}
\NormalTok{  [Header],}
\NormalTok{  [Footer],}
\NormalTok{  [Question?],}
\NormalTok{  [answer]}
\NormalTok{)}

\NormalTok{\#card(}
\NormalTok{  [portuguese],}
\NormalTok{  [Hint: Its the title of this package!],}
\NormalTok{  [card],}
\NormalTok{  [cartão]}
\NormalTok{)}

\NormalTok{\#card(}
\NormalTok{  [french],}
\NormalTok{  [Hint: close to the portuguese],}
\NormalTok{  [card],}
\NormalTok{  [carte]}
\NormalTok{)}
\end{Highlighting}
\end{Shaded}

\subsection{Documentation}\label{documentation}

\subsubsection{\texorpdfstring{\texttt{\ card\ }}{ card }}\label{card}

Defines a card by updating the below \texttt{\ counter\ } and
\texttt{\ state\ } (s), and dropping a label.

\begin{Shaded}
\begin{Highlighting}[]
\NormalTok{\#let card(header, footer, question, answer) = [}
\NormalTok{  \#cardnumber.step()}
\NormalTok{  \#cardheader.update(header)}
\NormalTok{  \#cardfooter.update(footer)}
\NormalTok{  \#cardquestion.update(question)}
\NormalTok{  \#cardanswer.update(answer)}
\NormalTok{  \textless{}card\textgreater{}}
\NormalTok{]}
\end{Highlighting}
\end{Shaded}

\paragraph{Arguments}\label{arguments}

\begin{itemize}
\tightlist
\item
  \texttt{\ header\ }
\item
  \texttt{\ footer\ }
\item
  \texttt{\ question\ }
\item
  \texttt{\ answer\ }
\end{itemize}

\subsubsection{card builders}\label{card-builders}

\textbf{How they work}

\begin{enumerate}
\tightlist
\item
  Find all locations of the \texttt{\ \textless{}card\textgreater{}\ }
  label
\item
  Get the values of the \texttt{\ cardnumber\ } counter, and
  \texttt{\ cardheader\ } , \texttt{\ cardfooter\ } ,
  \texttt{\ cardquestion\ } , \texttt{\ cardanswer\ } states at each
  \texttt{\ \textless{}card\textgreater{}\ } .
\item
  Populates an array of questions and an array of answers using these
  values

  \begin{itemize}
  \tightlist
  \item
    The \texttt{\ \#a48up\ } and \texttt{\ \#letter8up\ } functions
    describe the layout of each card for each item in these arrays, and
    also rearrange the answers so that the layout makes sense when
    printed double sided.
  \end{itemize}
\item
  Loop over the arrays and dump each item’s \texttt{\ content\ } onto
  the page.

  \begin{itemize}
  \tightlist
  \item
    in the case of \texttt{\ \#a48up\ } and \texttt{\ letter8up\ } ,
    each item is dumped into a 2-column table.
  \end{itemize}
\end{enumerate}

\texttt{\ cartao\ } comes builtin with the following card building
functions. Take a look at the source for how they work, and use them as
a guide to help you build your own flashcards with different
sizes/formats.

\subsubsection{\texorpdfstring{\texttt{\ a48up\ }}{ a48up }}\label{a48up}

Produces a 2x8 portrait card layout on a4 paper.

Designed to be printed double-sided on the perforated 8-up a4 card paper
you can find on
\href{https://www.amazon.ca/s?k=a4+perforated+card&crid=37RT2L4H5XSD0&sprefix=a4+perforated+ca\%2Caps\%2C648&ref=nb_sb_noss}{Amazon}

Usage

\begin{Shaded}
\begin{Highlighting}[]
\NormalTok{\#a48up}
\end{Highlighting}
\end{Shaded}

\subsubsection{\texorpdfstring{\texttt{\ letter8up\ }}{ letter8up }}\label{letter8up}

Produces a 2x8 portrait card layout on us-letter paper.

Usage

\begin{Shaded}
\begin{Highlighting}[]
\NormalTok{\#letter8up}
\end{Highlighting}
\end{Shaded}

\subsubsection{\texorpdfstring{\texttt{\ present\ }}{ present }}\label{present}

A 16:9 presentation of the flashcards with questions and answers on
different slides

Usage

\begin{Shaded}
\begin{Highlighting}[]
\NormalTok{\#present}
\end{Highlighting}
\end{Shaded}

\subsubsection{How to add}\label{how-to-add}

Copy this into your project and use the import as \texttt{\ cartao\ }

\begin{verbatim}
#import "@preview/cartao:0.1.0"
\end{verbatim}

\includesvg[width=0.16667in,height=0.16667in]{/assets/icons/16-copy.svg}

Check the docs for
\href{https://typst.app/docs/reference/scripting/\#packages}{more
information on how to import packages} .

\subsubsection{About}\label{about}

\begin{description}
\tightlist
\item[Author :]
Gavin Vales
\item[License:]
MIT
\item[Current version:]
0.1.0
\item[Last updated:]
November 21, 2023
\item[First released:]
November 21, 2023
\item[Archive size:]
3.40 kB
\href{https://packages.typst.org/preview/cartao-0.1.0.tar.gz}{\pandocbounded{\includesvg[keepaspectratio]{/assets/icons/16-download.svg}}}
\end{description}

\subsubsection{Where to report issues?}\label{where-to-report-issues}

This package is a project of Gavin Vales . You can also try to ask for
help with this package on the \href{https://forum.typst.app}{Forum} .

Please report this package to the Typst team using the
\href{https://typst.app/contact}{contact form} if you believe it is a
safety hazard or infringes upon your rights.

\phantomsection\label{versions}
\subsubsection{Version history}\label{version-history}

\begin{longtable}[]{@{}ll@{}}
\toprule\noalign{}
Version & Release Date \\
\midrule\noalign{}
\endhead
\bottomrule\noalign{}
\endlastfoot
0.1.0 & November 21, 2023 \\
\end{longtable}

Typst GmbH did not create this package and cannot guarantee correct
functionality of this package or compatibility with any version of the
Typst compiler or app.


\section{Package List LaTeX/structogrammer.tex}
\title{typst.app/universe/package/structogrammer}

\phantomsection\label{banner}
\section{structogrammer}\label{structogrammer}

{ 0.1.1 }

Draw Nassi-Shneiderman diagrams (or structograms)

\phantomsection\label{readme}
Draw Nassi-Shneiderman diagrams, also called structograms, in Typst.

\subsection{Basic Usage}\label{basic-usage}

Import with:

\begin{Shaded}
\begin{Highlighting}[]
\NormalTok{\#import "@preview/structogrammer:0.1.0": structogram}
\end{Highlighting}
\end{Shaded}

You can then draw structograms, like so:

\begin{Shaded}
\begin{Highlighting}[]
\NormalTok{\#structogram(}
\NormalTok{  width: 30em,}
\NormalTok{  title: "merge\_sort(list)",}
\NormalTok{  (}
\NormalTok{    (If: "list empty", Then: (Break: "exit (return list)")),}
\NormalTok{    "left = []",}
\NormalTok{    "right = []",}
\NormalTok{    (For: "element with index i", In: "list", Do: (}
\NormalTok{      (If: "i \textless{} list.length / 2", Then: (}
\NormalTok{        "left.add(element)"}
\NormalTok{      ), Else: (}
\NormalTok{        "right.add(element)"}
\NormalTok{      ))}
\NormalTok{    )),}
\NormalTok{    "left = merge\_sort(left)",}
\NormalTok{    "right = merge\_sort(right)",}
\NormalTok{    (Break: "return with merge(left, right)")}
\NormalTok{  )}
\NormalTok{)}
\end{Highlighting}
\end{Shaded}

which yields:\\
\pandocbounded{\includesvg[keepaspectratio]{https://raw.githubusercontent.com/genericusername3/structogrammer/master/examples/merge-sort.svg}}

If \texttt{\ text.lang\ } is set to another language, this package will
try to match inserted text to it. Currently, only \texttt{\ "en"\ } and
\texttt{\ "de"\ } are supported

\subsection{Advanced usage}\label{advanced-usage}

\texttt{\ structogram()\ } takes the following named arguments:

\begin{itemize}
\tightlist
\item
  \texttt{\ columns\ } : If you already allocated wide and narrow
  columns, \texttt{\ to-elements\ } can use them. Useful for sub-specs,
  as you’d usually generate allocations first and then do another
  recursive pass to fill them.\\
  The default, \texttt{\ auto\ } does exactly this on the highest
  recursion level.
\item
  \texttt{\ stroke\ } : The stroke to use between cells, or for control
  blocks. Note: to avoid duplicate strokes, every cell only adds strokes
  to its top and left side. Put the resulting cells in a container with
  bottom and right strokes for a finished diagram. See
  \texttt{\ structogram()\ } .\\
  Default: \texttt{\ 0.5pt\ +\ black\ }
\item
  \texttt{\ inset\ } : How much to pad each cell.\\
  Default: \texttt{\ 0.5em\ }
\item
  \texttt{\ segment-height\ } : How high each row should be.\\
  Default: \texttt{\ 2em\ }
\item
  \texttt{\ narrow-width\ } : The width that narrow columns will be.
  Needed for diagonals in conditional blocks.\\
  Default: 1em
\end{itemize}

A \texttt{\ spec\ } (the positional argument to
\texttt{\ structogram()\ } ) can be one of the following:

\begin{itemize}
\tightlist
\item
  \texttt{\ none\ } or an emtpy
  \href{https://typst.app/docs/reference/foundations/array/}{\texttt{\ array\ }}
  \texttt{\ ()\ } : An empty cell, taking up at least a narrow column
\item
  a
  \href{https://typst.app/docs/reference/foundations/str/}{\texttt{\ string\ }}
  or
  \href{https://typst.app/docs/reference/foundations/content/}{\texttt{\ content\ }}
  : A cell containing that string or content, taking up at least a wide
  column
\item
  A
  \href{https://typst.app/docs/reference/foundations/dictionary/}{\texttt{\ dictionary\ }}
  : Control block (
  \href{https://github.com/typst/packages/raw/main/packages/preview/structogrammer/0.1.1/\#control-blocks}{see
  below} )
\item
  An
  \href{https://typst.app/docs/reference/foundations/array/}{\texttt{\ array\ }}
  of specs: The cells that each element produced, stacked on top of each
  other. Wide columns are aligned to wide columns of other element specs
  and narrow columns consumed as needed.
\end{itemize}

\subsubsection{Control blocks}\label{control-blocks}

Specs can contain the following control blocks, as dictionaries:

\paragraph{\texorpdfstring{1. \texttt{\ If\ } / \texttt{\ Then\ } /
\texttt{\ Else\ } :}{1.  If  /  Then  /  Else  :}}\label{if-then-else}

A conditional with the following keys:

\begin{itemize}
\tightlist
\item
  \texttt{\ If\ } : The condition on which to branch
\item
  \texttt{\ Then\ } : A diagram spec for the “yes�-branch
\item
  \texttt{\ Else\ } : A diagram spec for the “no�-branch
\end{itemize}

\texttt{\ Then\ } and \texttt{\ Else\ } are both optional, but at least
one must be present

Examples:

\begin{itemize}
\tightlist
\item
\item
  \texttt{\ (If:\ "debug\ mode",\ Then:\ ("print\ debug\ message"))\ }

  \pandocbounded{\includesvg[keepaspectratio]{https://raw.githubusercontent.com/genericusername3/structogrammer/master/examples/if-then.svg}}
\item
  \texttt{\ (If:\ "x\ \textgreater{}\ 5",\ Then:\ ("x\ =\ x\ -\ 1",\ "print\ x"),\ Else:\ "print\ x")\ }

  \pandocbounded{\includesvg[keepaspectratio]{https://raw.githubusercontent.com/genericusername3/structogrammer/master/examples/if-then-else.svg}}
\end{itemize}

Columns: Takes up columns according to its contents next to one another,
inserting narrow columns for empty branches

\paragraph{\texorpdfstring{2. \texttt{\ For\ } / \texttt{\ Do\ } ,
\texttt{\ For\ } / \texttt{\ To\ } / \texttt{\ Do\ } , \texttt{\ For\ }
/ \texttt{\ In\ } / \texttt{\ Do\ } , \texttt{\ While\ } /
\texttt{\ Do\ } , \texttt{\ Do\ } / \texttt{\ While\ }
:}{2.  For  /  Do  ,  For  /  To  /  Do  ,  For  /  In  /  Do  ,  While  /  Do  ,  Do  /  While  :}}\label{for-do-for-to-do-for-in-do-while-do-do-while}

A loop, with the loop control either at the top or bottom.

\begin{itemize}
\tightlist
\item
  \texttt{\ For\ } / \texttt{\ Do\ } formats the control as “For
  \$For�,
\item
  \texttt{\ For\ } / \texttt{\ To\ } / \texttt{\ Do\ } as “For \$For
  to \$To�,
\item
  \texttt{\ For\ } / \texttt{\ In\ } / \texttt{\ Do\ } as “For each
  \$For in \$In�,
\item
  \texttt{\ While\ } / \texttt{\ Do\ } and \texttt{\ Do\ } /
  \texttt{\ While\ } as “While \$While�.
\end{itemize}

Order of specified keys matters.

Examples:

\begin{itemize}
\tightlist
\item
\item
  \texttt{\ (While:\ "true",\ Do:\ "print\ \textbackslash{}"endless\ loop\textbackslash{}"")\ }

  \pandocbounded{\includesvg[keepaspectratio]{https://raw.githubusercontent.com/genericusername3/structogrammer/master/examples/while-do.svg}}
\item
  \texttt{\ (Do:\ "print\ \textbackslash{}"endless\ loop\textbackslash{}"",\ While:\ "true")\ }

  \pandocbounded{\includesvg[keepaspectratio]{https://raw.githubusercontent.com/genericusername3/structogrammer/master/examples/do-while.svg}}
\item
  \texttt{\ (For:\ "item",\ In:\ "Container",\ Do:\ "print\ item.name")\ }

  \pandocbounded{\includesvg[keepaspectratio]{https://raw.githubusercontent.com/genericusername3/structogrammer/master/examples/for-in.svg}}
\end{itemize}

Columns: Inserts a narrow column left to its content.

\paragraph{\texorpdfstring{3. Method call ( \texttt{\ Call\ }
)}{3. Method call (  Call  )}}\label{method-call-call}

A block indicating that a subroutine is executed here. Only accepts the
key \texttt{\ Call\ } , which is the string name

Example:

\begin{itemize}
\tightlist
\item
\item
  \texttt{\ (Call:\ "func()")\ }

  \pandocbounded{\includesvg[keepaspectratio]{https://raw.githubusercontent.com/genericusername3/structogrammer/master/examples/call.svg}}
\end{itemize}

Columns: One wide column

\paragraph{\texorpdfstring{4. Break/Return ( \texttt{\ Break\ }
)}{4. Break/Return (  Break  )}}\label{breakreturn-break}

A block indicating that a subroutine is executed here. Only accepts the
key \texttt{\ Break\ } , which is the target to break to

Examples:

\begin{itemize}
\tightlist
\item
\item
  \texttt{\ (Break:\ "")\ }

  \pandocbounded{\includesvg[keepaspectratio]{https://raw.githubusercontent.com/genericusername3/structogrammer/master/examples/break.svg}}
\item
  \texttt{\ (Break:\ "to\ enclosing\ loop")\ }

  \pandocbounded{\includesvg[keepaspectratio]{https://raw.githubusercontent.com/genericusername3/structogrammer/master/examples/break-to.svg}}
\end{itemize}

Columns: One wide column

\subsubsection{How to add}\label{how-to-add}

Copy this into your project and use the import as
\texttt{\ structogrammer\ }

\begin{verbatim}
#import "@preview/structogrammer:0.1.1"
\end{verbatim}

\includesvg[width=0.16667in,height=0.16667in]{/assets/icons/16-copy.svg}

Check the docs for
\href{https://typst.app/docs/reference/scripting/\#packages}{more
information on how to import packages} .

\subsubsection{About}\label{about}

\begin{description}
\tightlist
\item[Author :]
\href{https://cza.li}{Charlotte}
\item[License:]
MIT
\item[Current version:]
0.1.1
\item[Last updated:]
May 14, 2024
\item[First released:]
May 13, 2024
\item[Archive size:]
8.68 kB
\href{https://packages.typst.org/preview/structogrammer-0.1.1.tar.gz}{\pandocbounded{\includesvg[keepaspectratio]{/assets/icons/16-download.svg}}}
\end{description}

\subsubsection{Where to report issues?}\label{where-to-report-issues}

This package is a project of Charlotte . You can also try to ask for
help with this package on the \href{https://forum.typst.app}{Forum} .

Please report this package to the Typst team using the
\href{https://typst.app/contact}{contact form} if you believe it is a
safety hazard or infringes upon your rights.

\phantomsection\label{versions}
\subsubsection{Version history}\label{version-history}

\begin{longtable}[]{@{}ll@{}}
\toprule\noalign{}
Version & Release Date \\
\midrule\noalign{}
\endhead
\bottomrule\noalign{}
\endlastfoot
0.1.1 & May 14, 2024 \\
\href{https://typst.app/universe/package/structogrammer/0.1.0/}{0.1.0} &
May 13, 2024 \\
\end{longtable}

Typst GmbH did not create this package and cannot guarantee correct
functionality of this package or compatibility with any version of the
Typst compiler or app.


\section{Package List LaTeX/physica.tex}
\title{typst.app/universe/package/physica}

\phantomsection\label{banner}
\section{physica}\label{physica}

{ 0.9.3 }

Math constructs for science and engineering: derivative, differential,
vector field, matrix, tensor, Dirac braket, hbar, transpose, conjugate,
many operators, and more.

{ } Featured Package

\phantomsection\label{readme}
:green\_book: The
\href{https://github.com/Leedehai/typst-physics/blob/v0.9.3/physica-manual.pdf}{manual}
.

\includegraphics[width=5.67708in,height=\textheight,keepaspectratio]{https://github.com/Leedehai/typst-physics/assets/18319900/ed86198a-8ddb-4473-aed3-8111d5ecde60}

\href{https://github.com/Leedehai/typst-physics/actions/workflows/ci.yml}{\pandocbounded{\includesvg[keepaspectratio]{https://github.com/Leedehai/typst-physics/actions/workflows/ci.yml/badge.svg}}}
\href{https://github.com/Leedehai/typst-physics/releases/latest}{\pandocbounded{\includegraphics[keepaspectratio]{https://img.shields.io/github/v/release/Leedehai/typst-physics.svg?color=gold}}}

Available in the collection of
\href{https://typst.app/docs/packages/}{Typst packages} :
\texttt{\ \#import\ "@preview/physica:0.9.3":\ *\ }

\begin{quote}
physica \emph{noun} .

\begin{itemize}
\tightlist
\item
  Latin, study of nature
\end{itemize}
\end{quote}

This \href{https://typst.app/}{Typst} package provides handy typesetting
utilities for natural sciences, including:

\begin{itemize}
\tightlist
\item
  Braces,
\item
  Vectors and vector fields,
\item
  Matrices, including Jacobian and Hessian,
\item
  Smartly render \texttt{\ ..\^{}T\ } as transpose and
  \texttt{\ ..\^{}+\ } as dagger (conjugate transpose),
\item
  Dirac braket notations,
\item
  Common math functions,
\item
  Differentials and derivatives, including partial derivatives of mixed
  orders with automatic order summation,
\item
  Familiar “h-bar�, tensor abstract index notations, isotopes,
  Taylor series term,
\item
  Signal sequences i.e. digital timing diagrams.
\end{itemize}

\subsection{A quick look}\label{a-quick-look}

See the
\href{https://github.com/Leedehai/typst-physics/blob/v0.9.3/physica-manual.pdf}{manual}
for more details and examples.

\pandocbounded{\includegraphics[keepaspectratio]{https://github.com/Leedehai/typst-physics/assets/18319900/4a9f40df-f753-4324-8114-c682d270e9c7}}

A larger
\href{https://github.com/Leedehai/typst-physics/blob/master/demo.typ}{demo.typ}
:

\pandocbounded{\includegraphics[keepaspectratio]{https://github.com/Leedehai/typst-physics/assets/18319900/75b94ef8-cc98-434f-be5f-bfac1ef6aef9}}

\subsection{Using physica in your Typst
document}\label{using-physica-in-your-typst-document}

\subsubsection{\texorpdfstring{With \texttt{\ typst\ } package
management
(recommended)}{With  typst  package management (recommended)}}\label{with-typst-package-management-recommended}

See \url{https://github.com/typst/packages} . If you are using the
Typst’s web app, packages listed there are readily available; if you
are using the Typst compiler locally, it downloads packages on-demand
and caches them on-disk, see
\href{https://github.com/typst/packages\#downloads}{here} for details.

\includegraphics[width=1.80208in,height=\textheight,keepaspectratio]{https://github.com/Leedehai/typst-physics/assets/18319900/f2a3a2bd-3ef7-4383-ab92-9a71affb4e12}

\begin{Shaded}
\begin{Highlighting}[]
\NormalTok{// Style 1}
\NormalTok{\#import "@preview/physica:0.9.3": *}

\NormalTok{$ curl (grad f), tensor(T, {-}mu, +nu), pdv(f,x,y,[1,2]) $}
\end{Highlighting}
\end{Shaded}

\begin{Shaded}
\begin{Highlighting}[]
\NormalTok{// Style 2}
\NormalTok{\#import "@preview/physica:0.9.3": curl, grad, tensor, pdv}

\NormalTok{$ curl (grad f), tensor(T, {-}mu, +nu), pdv(f,x,y,[1,2]) $}
\end{Highlighting}
\end{Shaded}

\begin{Shaded}
\begin{Highlighting}[]
\NormalTok{// Style 3}
\NormalTok{\#import "@preview/physica:0.9.3"}

\NormalTok{$ physica.curl (physica.grad f), physica.tensor(T, {-}mu, +nu), physica.pdv(f,x,y,[1,2]) $}
\end{Highlighting}
\end{Shaded}

\subsubsection{\texorpdfstring{Without \texttt{\ typst\ } package
management}{Without  typst  package management}}\label{without-typst-package-management}

Similar to examples above, but import with the undecorated file path
like \texttt{\ "physica.typ"\ } .

\subsection{Typst version}\label{typst-version}

The version requirement for the compiler is in
\href{https://github.com/typst/packages/raw/main/packages/preview/physica/0.9.3/typst.toml}{typst.toml}
’s \texttt{\ compiler\ } field. If you are using an unsupported Typst
version, the compiler will throw an error. You may want to update your
compiler with \texttt{\ typst\ update\ } , or choose an earlier version
of the \texttt{\ physica\ } package.

Developed with compiler version:

\begin{Shaded}
\begin{Highlighting}[]
\ExtensionTok{$}\NormalTok{ typst }\AttributeTok{{-}{-}version}
\ExtensionTok{typst}\NormalTok{ 0.10.0 }\ErrorTok{(}\ExtensionTok{70ca0d25}\KeywordTok{)}
\end{Highlighting}
\end{Shaded}

\subsection{Manual}\label{manual}

See the
\href{https://github.com/Leedehai/typst-physics/blob/v0.9.3/physica-manual.pdf}{manual}
for a more comprehensive coverage, a PDF file generated directly with
the \href{https://typst.app/}{Typst} binary.

To regenerate the manual, use command

\begin{Shaded}
\begin{Highlighting}[]
\ExtensionTok{typst}\NormalTok{ watch physica{-}manual.typ}
\end{Highlighting}
\end{Shaded}

\subsection{Contribution}\label{contribution}

\begin{itemize}
\item
  Bug fixes are welcome!
\item
  New features: welcome as well. If it is small, feel free to create a
  pull request. If it is large, the best first step is creating an issue
  and let us explore the design together. Some features might warrant a
  package on its own.
\item
  Testing: currently testing is done by closely inspecting the generated
  \href{https://github.com/Leedehai/typst-physics/blob/v0.9.3/physica-manual.pdf}{manual}
  . This does not scale well. I plan to add programmatic testing by
  comparing rendered pictures with golden images.
\end{itemize}

\subsection{Change log}\label{change-log}

\href{https://github.com/Leedehai/typst-physics/blob/v0.9.3/changelog.md}{changelog.md}
.

\subsection{License}\label{license}

\begin{itemize}
\tightlist
\item
  Code: the
  \href{https://github.com/typst/packages/raw/main/packages/preview/physica/0.9.3/LICENSE.txt}{MIT
  License} .
\item
  Docs: the
  \href{https://creativecommons.org/licenses/by-nd/4.0/}{Creative
  Commons BY-ND 4.0 license} .
\end{itemize}

\subsubsection{How to add}\label{how-to-add}

Copy this into your project and use the import as \texttt{\ physica\ }

\begin{verbatim}
#import "@preview/physica:0.9.3"
\end{verbatim}

\includesvg[width=0.16667in,height=0.16667in]{/assets/icons/16-copy.svg}

Check the docs for
\href{https://typst.app/docs/reference/scripting/\#packages}{more
information on how to import packages} .

\subsubsection{About}\label{about}

\begin{description}
\tightlist
\item[Author :]
Leedehai
\item[License:]
MIT
\item[Current version:]
0.9.3
\item[Last updated:]
April 2, 2024
\item[First released:]
September 8, 2023
\item[Minimum Typst version:]
0.10.0
\item[Archive size:]
11.1 kB
\href{https://packages.typst.org/preview/physica-0.9.3.tar.gz}{\pandocbounded{\includesvg[keepaspectratio]{/assets/icons/16-download.svg}}}
\item[Repository:]
\href{https://github.com/Leedehai/typst-physics}{GitHub}
\item[Discipline s :]
\begin{itemize}
\tightlist
\item[]
\item
  \href{https://typst.app/universe/search/?discipline=chemistry}{Chemistry}
\item
  \href{https://typst.app/universe/search/?discipline=communication}{Communication}
\item
  \href{https://typst.app/universe/search/?discipline=economics}{Economics}
\item
  \href{https://typst.app/universe/search/?discipline=education}{Education}
\item
  \href{https://typst.app/universe/search/?discipline=engineering}{Engineering}
\item
  \href{https://typst.app/universe/search/?discipline=geology}{Geology}
\item
  \href{https://typst.app/universe/search/?discipline=mathematics}{Mathematics}
\item
  \href{https://typst.app/universe/search/?discipline=physics}{Physics}
\end{itemize}
\item[Categor ies :]
\begin{itemize}
\tightlist
\item[]
\item
  \pandocbounded{\includesvg[keepaspectratio]{/assets/icons/16-package.svg}}
  \href{https://typst.app/universe/search/?category=components}{Components}
\item
  \pandocbounded{\includesvg[keepaspectratio]{/assets/icons/16-hammer.svg}}
  \href{https://typst.app/universe/search/?category=utility}{Utility}
\end{itemize}
\end{description}

\subsubsection{Where to report issues?}\label{where-to-report-issues}

This package is a project of Leedehai . Report issues on
\href{https://github.com/Leedehai/typst-physics}{their repository} . You
can also try to ask for help with this package on the
\href{https://forum.typst.app}{Forum} .

Please report this package to the Typst team using the
\href{https://typst.app/contact}{contact form} if you believe it is a
safety hazard or infringes upon your rights.

\phantomsection\label{versions}
\subsubsection{Version history}\label{version-history}

\begin{longtable}[]{@{}ll@{}}
\toprule\noalign{}
Version & Release Date \\
\midrule\noalign{}
\endhead
\bottomrule\noalign{}
\endlastfoot
0.9.3 & April 2, 2024 \\
\href{https://typst.app/universe/package/physica/0.9.2/}{0.9.2} &
January 15, 2024 \\
\href{https://typst.app/universe/package/physica/0.9.1/}{0.9.1} &
December 23, 2023 \\
\href{https://typst.app/universe/package/physica/0.9.0/}{0.9.0} &
December 7, 2023 \\
\href{https://typst.app/universe/package/physica/0.8.1/}{0.8.1} &
November 1, 2023 \\
\href{https://typst.app/universe/package/physica/0.8.0/}{0.8.0} &
September 13, 2023 \\
\href{https://typst.app/universe/package/physica/0.7.5/}{0.7.5} &
September 8, 2023 \\
\end{longtable}

Typst GmbH did not create this package and cannot guarantee correct
functionality of this package or compatibility with any version of the
Typst compiler or app.


\section{Package List LaTeX/thesist.tex}
\title{typst.app/universe/package/thesist}

\phantomsection\label{banner}
\phantomsection\label{template-thumbnail}
\pandocbounded{\includegraphics[keepaspectratio]{https://packages.typst.org/preview/thumbnails/thesist-0.2.0-small.webp}}

\section{thesist}\label{thesist}

{ 0.2.0 }

A Master\textquotesingle s thesis template for Instituto Superior
Técnico (IST)

\href{/app?template=thesist&version=0.2.0}{Create project in app}

\phantomsection\label{readme}
ThesIST (pronounced “desist�) is an unofficial Master’s thesis
template for Instituto Superior Técnico written in Typst.

This template fully meets the official formatting requirements as
outlined
\href{https://tecnico.ulisboa.pt/files/2021/09/guia-disserta-o-mestrado.pdf}{here}
, and also attempts to follow most unwritten conventions. Regardless,
you can be on the lookout for things you may want to see added.

PIC2 reports are also supported. However, some conventions for these may
vary with the supervisors, so please check with them if anything needs
to be changed.

\subsection{Changelogs}\label{changelogs}

The changelogs of new versions are available on
\href{https://github.com/tfachada/thesist/releases}{the Releases page} .
Make sure to check the latest one(s) whenever you update the imported
\texttt{\ thesist\ } version.

\subsection{Usage}\label{usage}

If you are in the Typst web app, simply click on “Start from
template� and pick this template.

If you want to develop locally:

\begin{enumerate}
\tightlist
\item
  Make sure you have the \textbf{TeX Gyre Heros} font family installed.
\item
  Install the package with \texttt{\ typst\ init\ @preview/thesist\ } .
\end{enumerate}

\subsection{Overview}\label{overview}

\textbf{Please read the “Quick guide� chapter included in this
template to get set up. You can keep it as a reference if you want.}

This template’s source files, hidden from the user view, are the
following:

\begin{itemize}
\item
  \texttt{\ layout.typ\ } : The main configuration file, which
  initializes the thesis and contains its general formatting rules.
\item
  \texttt{\ figure-numbering.typ\ } : This file contains a function to
  set a chapter-relative numbering for the various types of figures. The
  function is called once or twice depending on whether the user decides
  to include appendices.
\item
  \texttt{\ utils.typ\ } : General functions that you may want to import
  and use for QoL improvements.
\end{itemize}

\subsubsection{A sidenote about
subfigures}\label{a-sidenote-about-subfigures}

Since subfigures are not yet native to Typst, the current
implementation, present in \texttt{\ utils.typ\ } , needs the user to
manually input whether each called subfigure figure (aka subfigure grid)
is in an appendix or not. This is because the numbering is different in
appendices, and because the functionality of
\texttt{\ figure-numbering.typ\ } can’t be applied to subfigure grids,
since they are imported with their default numbering once in every
chapter. \texttt{\ context\ } expressions also don’t work across
imports, so location within the document couldn’t be used as a
parameter (unless the user called \texttt{\ context\ } themselves, which
would be unintuitive). \textbf{Regardless, the workaround that was
found, which is explained in the quick guide, doesn’t need much
thinking from the user, so you can see this as a more technical note
that shouldn’t matter when you’re writing the thesis.}

\subsection{Final remarks}\label{final-remarks}

This template is not necessarily (or hopefully) a finished product. Feel
free to open issues or pull requests!

Also thanks to the Typst community members for the help in some of the
functionalities, and for the extensions used here.

\href{/app?template=thesist&version=0.2.0}{Create project in app}

\subsubsection{How to use}\label{how-to-use}

Click the button above to create a new project using this template in
the Typst app.

You can also use the Typst CLI to start a new project on your computer
using this command:

\begin{verbatim}
typst init @preview/thesist:0.2.0
\end{verbatim}

\includesvg[width=0.16667in,height=0.16667in]{/assets/icons/16-copy.svg}

\subsubsection{About}\label{about}

\begin{description}
\tightlist
\item[Author :]
\href{https://github.com/tfachada}{Tomás Fachada}
\item[License:]
MIT
\item[Current version:]
0.2.0
\item[Last updated:]
October 21, 2024
\item[First released:]
August 28, 2024
\item[Minimum Typst version:]
0.12.0
\item[Archive size:]
373 kB
\href{https://packages.typst.org/preview/thesist-0.2.0.tar.gz}{\pandocbounded{\includesvg[keepaspectratio]{/assets/icons/16-download.svg}}}
\item[Repository:]
\href{https://github.com/tfachada/thesist}{GitHub}
\item[Categor y :]
\begin{itemize}
\tightlist
\item[]
\item
  \pandocbounded{\includesvg[keepaspectratio]{/assets/icons/16-mortarboard.svg}}
  \href{https://typst.app/universe/search/?category=thesis}{Thesis}
\end{itemize}
\end{description}

\subsubsection{Where to report issues?}\label{where-to-report-issues}

This template is a project of Tomás Fachada . Report issues on
\href{https://github.com/tfachada/thesist}{their repository} . You can
also try to ask for help with this template on the
\href{https://forum.typst.app}{Forum} .

Please report this template to the Typst team using the
\href{https://typst.app/contact}{contact form} if you believe it is a
safety hazard or infringes upon your rights.

\phantomsection\label{versions}
\subsubsection{Version history}\label{version-history}

\begin{longtable}[]{@{}ll@{}}
\toprule\noalign{}
Version & Release Date \\
\midrule\noalign{}
\endhead
\bottomrule\noalign{}
\endlastfoot
0.2.0 & October 21, 2024 \\
\href{https://typst.app/universe/package/thesist/0.1.0/}{0.1.0} & August
28, 2024 \\
\end{longtable}

Typst GmbH did not create this template and cannot guarantee correct
functionality of this template or compatibility with any version of the
Typst compiler or app.


\section{Package List LaTeX/grayness.tex}
\title{typst.app/universe/package/grayness}

\phantomsection\label{banner}
\section{grayness}\label{grayness}

{ 0.2.0 }

Simple image editing capabilities like converting to grayscale and
cropping via a WASM plugin.

\phantomsection\label{readme}
A package providing simple image editing capabilities via a WASM plugin.

Available functionality includes converting images to grayscale,
cropping and flipping the images. Furthermore, this package supports
adding transparency and bluring (very slow) as well as handling
additional raster image formats.

The package name is inspired by the blurry, gray images of Nessie, the
\href{https://en.wikipedia.org/wiki/Loch_Ness_Monster}{Loch Ness
Monster}

\subsection{Usage}\label{usage}

Due to the way typst currently interprets given paths, you have to read
the images yourself in the calling typst file. This raw imagedata can
then be passed to the grayness-package functions, like grayscale-image.
These functions also optionally accept all additional parameters of the
original typst image function like \texttt{\ width\ } or
\texttt{\ height\ } :

\begin{Shaded}
\begin{Highlighting}[]
\NormalTok{\#import "@preview/grayness:0.2.0": grayscale{-}image}

\NormalTok{\#let data = read("Art.webp", encoding: none)}
\NormalTok{\#grayscale{-}image(data, width: 50\%)}
\end{Highlighting}
\end{Shaded}

A detailed descriptions of all available functions is provided in the
\href{https://github.com/typst/packages/raw/main/packages/preview/grayness/0.2.0/manual.pdf}{manual}
.

You can also use the built-in help functions provided by tidy:

\begin{Shaded}
\begin{Highlighting}[]
\NormalTok{\#import "@preview/grayness:0.2.0": *}
\NormalTok{\#help("flip{-}image{-}vertical")}
\end{Highlighting}
\end{Shaded}

The \texttt{\ grayscale-image\ } function also works with SVG images. To
do so you must specify the format as \texttt{\ "svg"\ } :

\begin{Shaded}
\begin{Highlighting}[]
\NormalTok{\#let data = read("example.svg", encoding: none)}
\NormalTok{\#grayscale{-}image(data, format: "svg")}
\end{Highlighting}
\end{Shaded}

\subsection{Examples}\label{examples}

Here are several functions applied to a WEBP image of
\href{https://commons.wikimedia.org/wiki/File:Arturo_Nieto-Dorantes.webp}{Arturo
Nieto Dorantes} (CC-By-SA 4.0):
\pandocbounded{\includegraphics[keepaspectratio]{https://github.com/typst/packages/raw/main/packages/preview/grayness/0.2.0/example.png}}

\subsubsection{How to add}\label{how-to-add}

Copy this into your project and use the import as \texttt{\ grayness\ }

\begin{verbatim}
#import "@preview/grayness:0.2.0"
\end{verbatim}

\includesvg[width=0.16667in,height=0.16667in]{/assets/icons/16-copy.svg}

Check the docs for
\href{https://typst.app/docs/reference/scripting/\#packages}{more
information on how to import packages} .

\subsubsection{About}\label{about}

\begin{description}
\tightlist
\item[Author :]
Nikolai Neff-Sarnow
\item[License:]
Apache-2.0
\item[Current version:]
0.2.0
\item[Last updated:]
October 10, 2024
\item[First released:]
April 13, 2024
\item[Minimum Typst version:]
0.11.0
\item[Archive size:]
682 kB
\href{https://packages.typst.org/preview/grayness-0.2.0.tar.gz}{\pandocbounded{\includesvg[keepaspectratio]{/assets/icons/16-download.svg}}}
\item[Repository:]
\href{https://github.com/nineff/grayness}{GitHub}
\item[Categor ies :]
\begin{itemize}
\tightlist
\item[]
\item
  \pandocbounded{\includesvg[keepaspectratio]{/assets/icons/16-chart.svg}}
  \href{https://typst.app/universe/search/?category=visualization}{Visualization}
\item
  \pandocbounded{\includesvg[keepaspectratio]{/assets/icons/16-integration.svg}}
  \href{https://typst.app/universe/search/?category=integration}{Integration}
\item
  \pandocbounded{\includesvg[keepaspectratio]{/assets/icons/16-hammer.svg}}
  \href{https://typst.app/universe/search/?category=utility}{Utility}
\end{itemize}
\end{description}

\subsubsection{Where to report issues?}\label{where-to-report-issues}

This package is a project of Nikolai Neff-Sarnow . Report issues on
\href{https://github.com/nineff/grayness}{their repository} . You can
also try to ask for help with this package on the
\href{https://forum.typst.app}{Forum} .

Please report this package to the Typst team using the
\href{https://typst.app/contact}{contact form} if you believe it is a
safety hazard or infringes upon your rights.

\phantomsection\label{versions}
\subsubsection{Version history}\label{version-history}

\begin{longtable}[]{@{}ll@{}}
\toprule\noalign{}
Version & Release Date \\
\midrule\noalign{}
\endhead
\bottomrule\noalign{}
\endlastfoot
0.2.0 & October 10, 2024 \\
\href{https://typst.app/universe/package/grayness/0.1.0/}{0.1.0} & April
13, 2024 \\
\end{longtable}

Typst GmbH did not create this package and cannot guarantee correct
functionality of this package or compatibility with any version of the
Typst compiler or app.


\section{Package List LaTeX/modern-sjtu-thesis.tex}
\title{typst.app/universe/package/modern-sjtu-thesis}

\phantomsection\label{banner}
\phantomsection\label{template-thumbnail}
\pandocbounded{\includegraphics[keepaspectratio]{https://packages.typst.org/preview/thumbnails/modern-sjtu-thesis-0.1.0-small.webp}}

\section{modern-sjtu-thesis}\label{modern-sjtu-thesis}

{ 0.1.0 }

上海交通大学硕士学ä½?论æ--‡ Typst 模æ?¿ã€‚Shanghai Jiao Tong
University Master Thesis Typst Template.

\href{/app?template=modern-sjtu-thesis&version=0.1.0}{Create project in
app}

\phantomsection\label{readme}
这是上海交通大学硕士学ä½?论æ--‡çš„ Typst
模æ?¿ï¼Œå®ƒèƒ½å¤Ÿç®€æ´?ã€?快速ã€?æŒ?ç»­ç''Ÿæˆ? PDF
æ~¼å¼?的毕业论æ--‡ï¼Œå®ƒåŸºäºŽç~''究ç''Ÿé™¢å®˜æ--¹æ??供的模æ?¿è¿›è¡Œå¼€å?{}`。基于ç~''究ç''Ÿé™¢æ??供的
\href{https://www.gs.sjtu.edu.cn/post/detail/Z3M2MjU=}{word 模�}
进行开å?{}`。

\subsection{使ç''¨}\label{uxe4uxbduxe7}

快速�览效果: 查看
\href{https://github.com/tzhTaylor/typst-sjtu-thesis-master/releases/download/v0.1.0/thesis.pdf}{thesis.pdf}
,æ~·ä¾‹è®ºæ--‡æº?ç~?:查看
\href{https://github.com/tzhTaylor/typst-sjtu-thesis-master/blob/main/template/thesis.typ}{thesis.typ}

\subsubsection{VS Code
本地ç¼--è¾`(推è??)}\label{vs-code-uxe6ux153uxe5ux153uxe7uxbcuxe8uxbeuxefuxbcux2c6uxe6ux17euxe8uxefuxbc}

\begin{enumerate}
\item
  在 VS Code 中安è£
  \href{https://marketplace.visualstudio.com/items?itemName=myriad-dreamin.tinymist}{Tinymist
  Typst} æ?'件,负责语法高亮, é''™è¯¯æ£€æŸ¥å'Œ PDF 预览。
\item
  按下 \texttt{\ Ctrl\ +\ Shift\ +\ P\ }
  æ‰``å¼€å`½ä»¤ç•Œé?¢ï¼Œè¾``å\ldots¥
  \texttt{\ Typst:\ Show\ available\ Typst\ templates\ (gallery)\ for\ picking\ up\ a\ template\ }
  æ‰``å¼€ Tinymist æ??供的 Template Gallery,然å?Žä»Žé‡Œé?¢æ‰¾åˆ°
  \texttt{\ modern-sjtu-thesis\ } ,点击 \texttt{\ �\ }
  按é'®è¿›è¡Œæ''¶è---?,以å?Šç‚¹å‡» \texttt{\ +\ }
  å?·ï¼Œå°±å?¯ä»¥åˆ›å»ºå¯¹åº''的论æ--‡æ¨¡æ?¿äº†ã€‚
\item
  最å?Žç''¨ VS Code æ‰``å¼€ç''Ÿæˆ?的目录,æ‰``å¼€
  \texttt{\ thesis.typ\ } æ--‡ä»¶ï¼ŒæŒ‰ä¸‹ \texttt{\ Ctrl\ +\ K\ V\ }
  (Windows) / \texttt{\ Command\ +\ K\ V\ } (MacOS)
  æˆ--è€\ldots 是点击å?³ä¸Šè§'的按é'®è¿›è¡Œå®žæ---¶ç¼--è¾`å'Œé¢„览。
\end{enumerate}

\subsection{致谢}\label{uxe8uxe8}

\begin{itemize}
\item
  æ„Ÿè°¢ \href{https://github.com/OrangeX4}{@OrangeX4} å¼€å?{}`çš„
  \href{https://github.com/nju-lug/modern-nju-thesis}{modern-nju-thesis}
  模æ?¿ï¼Œæœ¬æ¨¡æ?¿å¤§ä½``ç»``构都是å?‚考å\ldots¶å¼€å?{}`的。
\item
  æ„Ÿè°¢ \href{https://typst-doc-cn.github.io/guide/FAQ.html}{Typst
  中æ--‡ç¤¾åŒºå¯¼èˆª FAQ} ,帮忙解决了å?„ç§?ç--`éš¾æ?‚ç---‡ã€‚
\end{itemize}

\subsection{License}\label{license}

This project is licensed under the MIT License.

\href{/app?template=modern-sjtu-thesis&version=0.1.0}{Create project in
app}

\subsubsection{How to use}\label{how-to-use}

Click the button above to create a new project using this template in
the Typst app.

You can also use the Typst CLI to start a new project on your computer
using this command:

\begin{verbatim}
typst init @preview/modern-sjtu-thesis:0.1.0
\end{verbatim}

\includesvg[width=0.16667in,height=0.16667in]{/assets/icons/16-copy.svg}

\subsubsection{About}\label{about}

\begin{description}
\tightlist
\item[Author :]
tzhTaylor
\item[License:]
MIT
\item[Current version:]
0.1.0
\item[Last updated:]
November 19, 2024
\item[First released:]
November 19, 2024
\item[Archive size:]
84.9 kB
\href{https://packages.typst.org/preview/modern-sjtu-thesis-0.1.0.tar.gz}{\pandocbounded{\includesvg[keepaspectratio]{/assets/icons/16-download.svg}}}
\item[Repository:]
\href{https://github.com/tzhTaylor/typst-sjtu-thesis-master}{GitHub}
\item[Categor y :]
\begin{itemize}
\tightlist
\item[]
\item
  \pandocbounded{\includesvg[keepaspectratio]{/assets/icons/16-mortarboard.svg}}
  \href{https://typst.app/universe/search/?category=thesis}{Thesis}
\end{itemize}
\end{description}

\subsubsection{Where to report issues?}\label{where-to-report-issues}

This template is a project of tzhTaylor . Report issues on
\href{https://github.com/tzhTaylor/typst-sjtu-thesis-master}{their
repository} . You can also try to ask for help with this template on the
\href{https://forum.typst.app}{Forum} .

Please report this template to the Typst team using the
\href{https://typst.app/contact}{contact form} if you believe it is a
safety hazard or infringes upon your rights.

\phantomsection\label{versions}
\subsubsection{Version history}\label{version-history}

\begin{longtable}[]{@{}ll@{}}
\toprule\noalign{}
Version & Release Date \\
\midrule\noalign{}
\endhead
\bottomrule\noalign{}
\endlastfoot
0.1.0 & November 19, 2024 \\
\end{longtable}

Typst GmbH did not create this template and cannot guarantee correct
functionality of this template or compatibility with any version of the
Typst compiler or app.


\section{Package List LaTeX/babble-bubbles.tex}
\title{typst.app/universe/package/babble-bubbles}

\phantomsection\label{banner}
\section{babble-bubbles}\label{babble-bubbles}

{ 0.1.0 }

A package to create callouts.

\phantomsection\label{readme}
A package to create callouts in typst, inspired by the
\href{https://obsidian.md/}{Obsidan} callouts.

Use preset callouts, or create your own!

\pandocbounded{\includegraphics[keepaspectratio]{https://github.com/typst/packages/raw/main/packages/preview/babble-bubbles/0.1.0/examples/callouts.png}}

\subsection{Usage}\label{usage}

Import the package

\begin{Shaded}
\begin{Highlighting}[]
\NormalTok{\#import "@preview/babble{-}bubbles:0.1.0": *}
\end{Highlighting}
\end{Shaded}

Or grab it locally and use:

\begin{Shaded}
\begin{Highlighting}[]
\NormalTok{\#import "@local/babble{-}bubbles:0.1.0": *}
\end{Highlighting}
\end{Shaded}

\subsection{Presets}\label{presets}

Here you can find a list of presets and an example usage of each. You
can customise them with the same parameters as the \texttt{\ callout\ }
function. See the \texttt{\ Custom\ callouts\ } for more details.

\begin{Shaded}
\begin{Highlighting}[]
\NormalTok{\#info[This is information]}

\NormalTok{\#success[I\textquotesingle{}m making a note here: huge success]}

\NormalTok{\#check[This is checked!]}

\NormalTok{\#warning[First warning...]}

\NormalTok{\#note[My incredibly useful note]}

\NormalTok{\#question[Question?]}

\NormalTok{\#example[An example make things interesting]}

\NormalTok{\#quote[To be or not to be]}
\end{Highlighting}
\end{Shaded}

\subsection{Custom callouts}\label{custom-callouts}

\subsubsection{\texorpdfstring{\texttt{\ callout\ }}{ callout }}\label{callout}

Create a default callout. Tweak the parameters to create your own!

\begin{Shaded}
\begin{Highlighting}[]
\NormalTok{callout(}
\NormalTok{  body,}
\NormalTok{  title: "Callout",}
\NormalTok{  fill: blue,}
\NormalTok{  title{-}color: white,}
\NormalTok{  body{-}color: black,}
\NormalTok{  icon: none)}
\end{Highlighting}
\end{Shaded}

\subsubsection{Tips}\label{tips}

You can create aliases to more easily handle your newly create callouts
or customise presets by using
\href{https://typst.app/docs/reference/types/function/\#methods-with}{with}
.

\begin{verbatim}
#let mycallout = callout.with(title: "My callout")

#mycallout[Hey this is my custom callout!]
\end{verbatim}

\subsubsection{How to add}\label{how-to-add}

Copy this into your project and use the import as
\texttt{\ babble-bubbles\ }

\begin{verbatim}
#import "@preview/babble-bubbles:0.1.0"
\end{verbatim}

\includesvg[width=0.16667in,height=0.16667in]{/assets/icons/16-copy.svg}

Check the docs for
\href{https://typst.app/docs/reference/scripting/\#packages}{more
information on how to import packages} .

\subsubsection{About}\label{about}

\begin{description}
\tightlist
\item[Author :]
Dimitri Belopopsky
\item[License:]
MIT
\item[Current version:]
0.1.0
\item[Last updated:]
September 11, 2023
\item[First released:]
September 11, 2023
\item[Archive size:]
2.15 kB
\href{https://packages.typst.org/preview/babble-bubbles-0.1.0.tar.gz}{\pandocbounded{\includesvg[keepaspectratio]{/assets/icons/16-download.svg}}}
\item[Repository:]
\href{https://github.com/ShadowMitia/typst-babble-bubbles}{GitHub}
\end{description}

\subsubsection{Where to report issues?}\label{where-to-report-issues}

This package is a project of Dimitri Belopopsky . Report issues on
\href{https://github.com/ShadowMitia/typst-babble-bubbles}{their
repository} . You can also try to ask for help with this package on the
\href{https://forum.typst.app}{Forum} .

Please report this package to the Typst team using the
\href{https://typst.app/contact}{contact form} if you believe it is a
safety hazard or infringes upon your rights.

\phantomsection\label{versions}
\subsubsection{Version history}\label{version-history}

\begin{longtable}[]{@{}ll@{}}
\toprule\noalign{}
Version & Release Date \\
\midrule\noalign{}
\endhead
\bottomrule\noalign{}
\endlastfoot
0.1.0 & September 11, 2023 \\
\end{longtable}

Typst GmbH did not create this package and cannot guarantee correct
functionality of this package or compatibility with any version of the
Typst compiler or app.


\section{Package List LaTeX/problemst.tex}
\title{typst.app/universe/package/problemst}

\phantomsection\label{banner}
\phantomsection\label{template-thumbnail}
\pandocbounded{\includegraphics[keepaspectratio]{https://packages.typst.org/preview/thumbnails/problemst-0.1.0-small.webp}}

\section{problemst}\label{problemst}

{ 0.1.0 }

Simple and easy-to-use template for problem sets/homeworks/assignments.

\href{/app?template=problemst&version=0.1.0}{Create project in app}

\phantomsection\label{readme}
Simple and easy-to-use template for problem sets/homeworks/assignments.

\pandocbounded{\includegraphics[keepaspectratio]{https://github.com/typst/packages/raw/main/packages/preview/problemst/0.1.0/template/thumbnail.png}}

\subsection{Usage}\label{usage}

Click “Start from template� in the Typst web app and search for
\texttt{\ problemst\ } .

Alternatively, run the following command to create a directory
initialized with all necessary files:

\begin{verbatim}
typst init @preview/problemst:0.1.0
\end{verbatim}

\subsection{Configuration}\label{configuration}

The \texttt{\ pset\ } function takes the following named arguments:

\begin{itemize}
\tightlist
\item
  \texttt{\ class\ } (string): Class the assignment is for.
\item
  \texttt{\ student\ } (string): Student completing the assignment.
\item
  \texttt{\ title\ } (string): Title of the assignment.
\item
  \texttt{\ date\ } (datetime): Date to be displayed on the assignment.
\item
  \texttt{\ collaborators\ } (array of strings): Collaborators that
  worked on the assignment with the student. Can be \texttt{\ ()\ } .
\item
  \texttt{\ subproblems\ } (string): Numbering scheme for the
  subproblems.
\end{itemize}

\href{/app?template=problemst&version=0.1.0}{Create project in app}

\subsubsection{How to use}\label{how-to-use}

Click the button above to create a new project using this template in
the Typst app.

You can also use the Typst CLI to start a new project on your computer
using this command:

\begin{verbatim}
typst init @preview/problemst:0.1.0
\end{verbatim}

\includesvg[width=0.16667in,height=0.16667in]{/assets/icons/16-copy.svg}

\subsubsection{About}\label{about}

\begin{description}
\tightlist
\item[Author :]
\href{https://github.com/carreter}{Willow Carretero Chavez}
\item[License:]
MIT
\item[Current version:]
0.1.0
\item[Last updated:]
April 17, 2024
\item[First released:]
April 17, 2024
\item[Minimum Typst version:]
0.11.0
\item[Archive size:]
2.63 kB
\href{https://packages.typst.org/preview/problemst-0.1.0.tar.gz}{\pandocbounded{\includesvg[keepaspectratio]{/assets/icons/16-download.svg}}}
\item[Categor y :]
\begin{itemize}
\tightlist
\item[]
\item
  \pandocbounded{\includesvg[keepaspectratio]{/assets/icons/16-speak.svg}}
  \href{https://typst.app/universe/search/?category=report}{Report}
\end{itemize}
\end{description}

\subsubsection{Where to report issues?}\label{where-to-report-issues}

This template is a project of Willow Carretero Chavez . You can also try
to ask for help with this template on the
\href{https://forum.typst.app}{Forum} .

Please report this template to the Typst team using the
\href{https://typst.app/contact}{contact form} if you believe it is a
safety hazard or infringes upon your rights.

\phantomsection\label{versions}
\subsubsection{Version history}\label{version-history}

\begin{longtable}[]{@{}ll@{}}
\toprule\noalign{}
Version & Release Date \\
\midrule\noalign{}
\endhead
\bottomrule\noalign{}
\endlastfoot
0.1.0 & April 17, 2024 \\
\end{longtable}

Typst GmbH did not create this template and cannot guarantee correct
functionality of this template or compatibility with any version of the
Typst compiler or app.


\section{Package List LaTeX/cheq.tex}
\title{typst.app/universe/package/cheq}

\phantomsection\label{banner}
\section{cheq}\label{cheq}

{ 0.2.2 }

Write markdown-like checklist easily.

\phantomsection\label{readme}
Write markdown-like checklist easily.

\subsection{Usage}\label{usage}

Checklists are incredibly useful for keeping track of important items.
We can use the cheq package to achieve checklist syntax similar to
\href{https://github.github.com/gfm/\#task-list-items-extension-}{GitHub
Flavored Markdown} and \href{https://minimal.guide/checklists}{Minimal}
.

\begin{Shaded}
\begin{Highlighting}[]
\NormalTok{\#import "@preview/cheq:0.2.2": checklist}

\NormalTok{\#show: checklist}

\NormalTok{= Solar System Exploration, 1950s – 1960s}

\NormalTok{{-} [ ] Mercury}
\NormalTok{{-} [x] Venus}
\NormalTok{{-} [x] Earth (Orbit/Moon)}
\NormalTok{{-} [x] Mars}
\NormalTok{{-} [{-}] Jupiter}
\NormalTok{{-} [/] Saturn}
\NormalTok{{-} [ ] Uranus}
\NormalTok{{-} [ ] Neptune}
\NormalTok{{-} [ ] Comet Haley}

\NormalTok{= Extras}

\NormalTok{{-} [\textgreater{}] Forwarded}
\NormalTok{{-} [\textless{}] Scheduling}
\NormalTok{{-} [?] question}
\NormalTok{{-} [!] important}
\NormalTok{{-} [\textbackslash{}*] star}
\NormalTok{{-} ["] quote}
\NormalTok{{-} [l] location}
\NormalTok{{-} [b] bookmark}
\NormalTok{{-} [i] information}
\NormalTok{{-} [S] savings}
\NormalTok{{-} [I] idea}
\NormalTok{{-} [p] pros}
\NormalTok{{-} [c] cons}
\NormalTok{{-} [f] fire}
\NormalTok{{-} [k] key}
\NormalTok{{-} [w] win}
\NormalTok{{-} [u] up}
\NormalTok{{-} [d] down}
\end{Highlighting}
\end{Shaded}

\pandocbounded{\includegraphics[keepaspectratio]{https://github.com/typst/packages/raw/main/packages/preview/cheq/0.2.2/examples/example.png}}

\subsection{Custom Styles}\label{custom-styles}

\begin{Shaded}
\begin{Highlighting}[]
\NormalTok{\#import "@preview/cheq:0.2.2": checklist}

\NormalTok{\#show: checklist.with(fill: luma(95\%), stroke: blue, radius: .2em)}

\NormalTok{= Solar System Exploration, 1950s – 1960s}

\NormalTok{{-} [ ] Mercury}
\NormalTok{{-} [x] Venus}
\NormalTok{{-} [x] Earth (Orbit/Moon)}
\NormalTok{{-} [x] Mars}
\NormalTok{{-} [{-}] Jupiter}
\NormalTok{{-} [/] Saturn}
\NormalTok{{-} [ ] Uranus}
\NormalTok{{-} [ ] Neptune}
\NormalTok{{-} [ ] Comet Haley}

\NormalTok{\#show: checklist.with(marker{-}map: (" ": sym.ballot, "x": sym.ballot.x, "{-}": sym.bar.h, "/": sym.slash.double))}

\NormalTok{= Solar System Exploration, 1950s – 1960s}

\NormalTok{{-} [ ] Mercury}
\NormalTok{{-} [x] Venus}
\NormalTok{{-} [x] Earth (Orbit/Moon)}
\NormalTok{{-} [x] Mars}
\NormalTok{{-} [{-}] Jupiter}
\NormalTok{{-} [/] Saturn}
\NormalTok{{-} [ ] Uranus}
\NormalTok{{-} [ ] Neptune}
\NormalTok{{-} [ ] Comet Haley}
\end{Highlighting}
\end{Shaded}

\pandocbounded{\includegraphics[keepaspectratio]{https://github.com/typst/packages/raw/main/packages/preview/cheq/0.2.2/examples/custom-styles.png}}

\subsection{\texorpdfstring{\texttt{\ checklist\ }
function}{ checklist  function}}\label{checklist-function}

\begin{Shaded}
\begin{Highlighting}[]
\NormalTok{\#let checklist(}
\NormalTok{  fill: white,}
\NormalTok{  stroke: rgb("\#616161"),}
\NormalTok{  radius: .1em,}
\NormalTok{  marker{-}map: (:),}
\NormalTok{  body,}
\NormalTok{) = \{ .. \}}
\end{Highlighting}
\end{Shaded}

\textbf{Arguments:}

\begin{itemize}
\tightlist
\item
  \texttt{\ fill\ } : {[} \texttt{\ string\ } {]} â€'' The fill color
  for the checklist marker.
\item
  \texttt{\ stroke\ } : {[} \texttt{\ string\ } {]} â€'' The stroke
  color for the checklist marker.
\item
  \texttt{\ radius\ } : {[} \texttt{\ string\ } {]} â€'' The radius of
  the checklist marker.
\item
  \texttt{\ marker-map\ } : {[} \texttt{\ map\ } {]} â€'' The map of the
  checklist marker. It should be a map of character to symbol function,
  such as
  \texttt{\ ("\ ":\ sym.ballot,\ "x":\ sym.ballot.x,\ "-":\ sym.bar.h,\ "/":\ sym.slash.double)\ }
  .
\item
  \texttt{\ show-list-set-block\ } : {[} \texttt{\ dictionary\ } {]} -
  The configuration of the block in list. It should be a dictionary of
  \texttt{\ above\ } and \texttt{\ below\ } keys, such as
  \texttt{\ (above:\ .5em)\ } .
\item
  \texttt{\ body\ } : {[} \texttt{\ content\ } {]} â€'' The main body
  from \texttt{\ \#show:\ checklist\ } rule.
\end{itemize}

The default map is:

\begin{Shaded}
\begin{Highlighting}[]
\NormalTok{\#let default{-}map = (}
\NormalTok{  "x": checked{-}sym(fill: fill, stroke: stroke, radius: radius),}
\NormalTok{  " ": unchecked{-}sym(fill: fill, stroke: stroke, radius: radius),}
\NormalTok{  "/": incomplete{-}sym(fill: fill, stroke: stroke, radius: radius),}
\NormalTok{  "{-}": canceled{-}sym(fill: fill, stroke: stroke, radius: radius),}
\NormalTok{  "\textgreater{}": "➡",}
\NormalTok{  "\textless{}": "📆",}
\NormalTok{  "?": "❓",}
\NormalTok{  "!": "❗",}
\NormalTok{  "*": "⭐",}
\NormalTok{  "\textbackslash{}"": "❝",}
\NormalTok{  "l": "📍",}
\NormalTok{  "b": "🔖",}
\NormalTok{  "i": "ℹ️",}
\NormalTok{  "S": "💰",}
\NormalTok{  "I": "💡",}
\NormalTok{  "p": "👍",}
\NormalTok{  "c": "👎",}
\NormalTok{  "f": "🔥",}
\NormalTok{  "k": "🔑",}
\NormalTok{  "w": "🏆",}
\NormalTok{  "u": "🔼",}
\NormalTok{  "d": "🔽",}
\NormalTok{)}
\end{Highlighting}
\end{Shaded}

\subsection{\texorpdfstring{\texttt{\ unchecked-sym\ }
function}{ unchecked-sym  function}}\label{unchecked-sym-function}

\begin{Shaded}
\begin{Highlighting}[]
\NormalTok{\#let unchecked{-}sym(fill: white, stroke: rgb("\#616161"), radius: .1em) = \{ .. \}}
\end{Highlighting}
\end{Shaded}

\textbf{Arguments:}

\begin{itemize}
\tightlist
\item
  \texttt{\ fill\ } : {[} \texttt{\ string\ } {]} â€'' The fill color
  for the unchecked symbol.
\item
  \texttt{\ stroke\ } : {[} \texttt{\ string\ } {]} â€'' The stroke
  color for the unchecked symbol.
\item
  \texttt{\ radius\ } : {[} \texttt{\ string\ } {]} â€'' The radius of
  the unchecked symbol.
\end{itemize}

\subsection{\texorpdfstring{\texttt{\ checked-sym\ }
function}{ checked-sym  function}}\label{checked-sym-function}

\begin{Shaded}
\begin{Highlighting}[]
\NormalTok{\#let checked{-}sym(fill: white, stroke: rgb("\#616161"), radius: .1em) = \{ .. \}}
\end{Highlighting}
\end{Shaded}

\textbf{Arguments:}

\begin{itemize}
\tightlist
\item
  \texttt{\ fill\ } : {[} \texttt{\ string\ } {]} â€'' The fill color
  for the checked symbol.
\item
  \texttt{\ stroke\ } : {[} \texttt{\ string\ } {]} â€'' The stroke
  color for the checked symbol.
\item
  \texttt{\ radius\ } : {[} \texttt{\ string\ } {]} â€'' The radius of
  the checked symbol.
\end{itemize}

\subsection{\texorpdfstring{\texttt{\ incomplete-sym\ }
function}{ incomplete-sym  function}}\label{incomplete-sym-function}

\begin{Shaded}
\begin{Highlighting}[]
\NormalTok{\#let incomplete{-}sym(fill: white, stroke: rgb("\#616161"), radius: .1em) = \{ .. \}}
\end{Highlighting}
\end{Shaded}

\textbf{Arguments:}

\begin{itemize}
\tightlist
\item
  \texttt{\ fill\ } : {[} \texttt{\ string\ } {]} â€'' The fill color
  for the incomplete symbol.
\item
  \texttt{\ stroke\ } : {[} \texttt{\ string\ } {]} â€'' The stroke
  color for the incomplete symbol.
\item
  \texttt{\ radius\ } : {[} \texttt{\ string\ } {]} â€'' The radius of
  the incomplete symbol.
\end{itemize}

\subsection{\texorpdfstring{\texttt{\ canceled-sym\ }
function}{ canceled-sym  function}}\label{canceled-sym-function}

\begin{Shaded}
\begin{Highlighting}[]
\NormalTok{\#let canceled{-}sym(fill: white, stroke: rgb("\#616161"), radius: .1em) = \{ .. \}}
\end{Highlighting}
\end{Shaded}

\textbf{Arguments:}

\begin{itemize}
\tightlist
\item
  \texttt{\ fill\ } : {[} \texttt{\ string\ } {]} â€'' The fill color
  for the canceled symbol.
\item
  \texttt{\ stroke\ } : {[} \texttt{\ string\ } {]} â€'' The stroke
  color for the canceled symbol.
\item
  \texttt{\ radius\ } : {[} \texttt{\ string\ } {]} â€'' The radius of
  the canceled symbol.
\end{itemize}

\subsection{License}\label{license}

This project is licensed under the MIT License.

\subsubsection{How to add}\label{how-to-add}

Copy this into your project and use the import as \texttt{\ cheq\ }

\begin{verbatim}
#import "@preview/cheq:0.2.2"
\end{verbatim}

\includesvg[width=0.16667in,height=0.16667in]{/assets/icons/16-copy.svg}

Check the docs for
\href{https://typst.app/docs/reference/scripting/\#packages}{more
information on how to import packages} .

\subsubsection{About}\label{about}

\begin{description}
\tightlist
\item[Author s :]
OrangeX4 , Myriad-Dreamin , \& duskmoon314
\item[License:]
MIT
\item[Current version:]
0.2.2
\item[Last updated:]
October 17, 2024
\item[First released:]
April 12, 2024
\item[Archive size:]
3.33 kB
\href{https://packages.typst.org/preview/cheq-0.2.2.tar.gz}{\pandocbounded{\includesvg[keepaspectratio]{/assets/icons/16-download.svg}}}
\item[Repository:]
\href{https://github.com/OrangeX4/typst-cheq}{GitHub}
\item[Categor ies :]
\begin{itemize}
\tightlist
\item[]
\item
  \pandocbounded{\includesvg[keepaspectratio]{/assets/icons/16-package.svg}}
  \href{https://typst.app/universe/search/?category=components}{Components}
\item
  \pandocbounded{\includesvg[keepaspectratio]{/assets/icons/16-hammer.svg}}
  \href{https://typst.app/universe/search/?category=utility}{Utility}
\end{itemize}
\end{description}

\subsubsection{Where to report issues?}\label{where-to-report-issues}

This package is a project of OrangeX4, Myriad-Dreamin, and duskmoon314 .
Report issues on \href{https://github.com/OrangeX4/typst-cheq}{their
repository} . You can also try to ask for help with this package on the
\href{https://forum.typst.app}{Forum} .

Please report this package to the Typst team using the
\href{https://typst.app/contact}{contact form} if you believe it is a
safety hazard or infringes upon your rights.

\phantomsection\label{versions}
\subsubsection{Version history}\label{version-history}

\begin{longtable}[]{@{}ll@{}}
\toprule\noalign{}
Version & Release Date \\
\midrule\noalign{}
\endhead
\bottomrule\noalign{}
\endlastfoot
0.2.2 & October 17, 2024 \\
\href{https://typst.app/universe/package/cheq/0.2.1/}{0.2.1} & October
14, 2024 \\
\href{https://typst.app/universe/package/cheq/0.2.0/}{0.2.0} & September
8, 2024 \\
\href{https://typst.app/universe/package/cheq/0.1.0/}{0.1.0} & April 12,
2024 \\
\end{longtable}

Typst GmbH did not create this package and cannot guarantee correct
functionality of this package or compatibility with any version of the
Typst compiler or app.


\section{Package List LaTeX/ascii-ipa.tex}
\title{typst.app/universe/package/ascii-ipa}

\phantomsection\label{banner}
\section{ascii-ipa}\label{ascii-ipa}

{ 2.0.0 }

Converter for ASCII representations of the International Phonetic
Alphabet (IPA)

\phantomsection\label{readme}
ðŸ''„ ASCII / IPA conversion for Typst

This package allows you to easily convert different ASCII
representations of the International Phonetic Alphabet (IPA) to and from
the IPA. It also offers some minor utilities to make phonetic
transcriptions easier to use overall. The package is being maintained
\href{https://github.com/imatpot/typst-ascii-ipa}{here} .

Note: This is an extended port of the
\href{https://github.com/tirimid/ipa-translate}{\texttt{\ ipa-translate\ }}
Rust crate by \href{https://github.com/tirimid}{tirimid} ’s conversion
features into native Typst. Most conversions are implemented according
to
\href{https://en.wikipedia.org/wiki/Comparison_of_ASCII_encodings_of_the_International_Phonetic_Alphabet}{this
Wikipedia article} which in turn relies of the official specifications
of the respective ASCII representations.

\subsection{Conversion}\label{conversion}

The package supports multiple ASCII representations for the IPA with one
function each:

\begin{longtable}[]{@{}ll@{}}
\toprule\noalign{}
Notation & Function name \\
\midrule\noalign{}
\endhead
\bottomrule\noalign{}
\endlastfoot
Branner & \texttt{\ \#branner(...)\ } \\
Praat & \texttt{\ \#praat(...)\ } \\
SIL & \texttt{\ \#sil(...)\ } \\
X-SAMPA & \texttt{\ \#xsampa(...)\ } \\
\end{longtable}

They all return the converted value as a
\href{https://typst.app/docs/reference/foundations/str/}{\texttt{\ string\ }}
and accept the set of same parameters:

\begin{longtable}[]{@{}lllll@{}}
\toprule\noalign{}
Parameter & Type & Positional / Named & Default & Description \\
\midrule\noalign{}
\endhead
\bottomrule\noalign{}
\endlastfoot
\texttt{\ value\ } &
\href{https://typst.app/docs/reference/foundations/str/}{\texttt{\ string\ }}
& positional & & Main input to the function. Usually the transcription
in the corresponsing ASCII-based notation. \\
\texttt{\ reverse\ } &
\href{https://typst.app/docs/reference/foundations/bool/}{\texttt{\ bool\ }}
& named & \texttt{\ false\ } & Reverses the conversion. Pass Unicode IPA
into \texttt{\ value\ } to get the corresponsing ASCII-based notation
back. \\
\end{longtable}

\subsubsection{Examples}\label{examples}

All examples use the Swiss German word
\href{https://als.wikipedia.org/wiki/Chuchich\%C3\%A4schtli}{⟨Chuchichäschtli⟩
{[}ˈχʊχË?iˌχæʃË?tlɪ{]}} for the conversion.

\begin{Shaded}
\begin{Highlighting}[]
\NormalTok{\#import "@preview/ascii{-}ipa:2.0.0": *}

\NormalTok{// returns "ˈχʊχːiˌχæʃːtlɪ"}
\NormalTok{\#branner("\textquotesingle{}XUX:i,Xae)S:tlI")}

\NormalTok{// returns "\textquotesingle{}XUX:i,Xae)S:tlI"}
\NormalTok{\#branner("ˈχʊχːiˌχæʃːtlɪ", reverse: true)}

\NormalTok{// returns "ˈχʊχːiˌχæʃːtlɪ"}
\NormalTok{\#praat("\textbackslash{}\textbackslash{}\textquotesingle{}1\textbackslash{}\textbackslash{}cf\textbackslash{}\textbackslash{}hs\textbackslash{}\textbackslash{}cf\textbackslash{}\textbackslash{}:f\textbackslash{}\textbackslash{}\textquotesingle{}2\textbackslash{}\textbackslash{}ae\textbackslash{}\textbackslash{}sh\textbackslash{}\textbackslash{}:ftl\textbackslash{}\textbackslash{}ic")}

\NormalTok{// returns "\textbackslash{}\textbackslash{}\textquotesingle{}1\textbackslash{}\textbackslash{}cf\textbackslash{}\textbackslash{}hs\textbackslash{}\textbackslash{}cf\textbackslash{}\textbackslash{}:f\textbackslash{}\textbackslash{}\textquotesingle{}2\textbackslash{}\textbackslash{}ae\textbackslash{}\textbackslash{}sh\textbackslash{}\textbackslash{}:ftl\textbackslash{}\textbackslash{}ic"}
\NormalTok{\#praat("ˈχʊχːiˌχæʃːtlɪ", reverse: true)}

\NormalTok{// returns "ˈχʊχːiˌχæʃːtlɪ"}
\NormalTok{\#sil("\}x=u\textless{}x=:i\}\}x=a\textless{}s=:tli=")}

\NormalTok{// returns "\}x=u\textless{}x=:i\}\}x=a\textless{}s=:tli="}
\NormalTok{\#sil("ˈχʊχːiˌχæʃːtlɪ", reverse: true)}

\NormalTok{// returns "ˈχʊχːiˌχæʃːtlɪ"}
\NormalTok{\#xsampa("\textbackslash{}"XUX:i\%X\{S:tlI")}

\NormalTok{// returns "\textbackslash{}"XUX:i\%X\{S:tlI"}
\NormalTok{\#xsampa("ˈχʊχːiˌχæʃːtlɪ", reverse: true)}
\end{Highlighting}
\end{Shaded}

\subsubsection{\texorpdfstring{With
\texttt{\ raw\ }}{With  raw }}\label{with-raw}

You can also use
\href{https://typst.app/docs/reference/text/raw/}{\texttt{\ raw\ }} for
the conversion. This is useful if the notation uses a lot of
backslashes.

\begin{Shaded}
\begin{Highlighting}[]
\NormalTok{\#import "@preview/ascii{-}ipa:2.0.0": praat}

\NormalTok{// regular string}
\NormalTok{\#praat("\textbackslash{}\textbackslash{}\textquotesingle{}1\textbackslash{}\textbackslash{}cf\textbackslash{}\textbackslash{}hs\textbackslash{}\textbackslash{}cf\textbackslash{}\textbackslash{}:f\textbackslash{}\textbackslash{}\textquotesingle{}2\textbackslash{}\textbackslash{}ae\textbackslash{}\textbackslash{}sh\textbackslash{}\textbackslash{}:ftl\textbackslash{}\textbackslash{}ic")}

\NormalTok{// raw}
\NormalTok{\#praat(\textasciigrave{}\textbackslash{}\textquotesingle{}1\textbackslash{}cf\textbackslash{}hs\textbackslash{}cf\textbackslash{}:f\textbackslash{}\textquotesingle{}2\textbackslash{}ae\textbackslash{}sh\textbackslash{}:ftl\textbackslash{}ic\textasciigrave{})}
\end{Highlighting}
\end{Shaded}

Note: \texttt{\ raw\ } will not play nicely with notations that use
\texttt{\ \textasciigrave{}\ } a lot.

\subsection{Brackets \& Braces}\label{brackets-braces}

You can easily mark your notation text as different types of brackets or
braces.

\begin{Shaded}
\begin{Highlighting}[]
\NormalTok{\#import "@preview/ascii{-}ipa:2.0.0": *}

\NormalTok{\#phonetic("prʲɪˈvʲet") // [prʲɪˈvʲet]}
\NormalTok{\#phnt("prʲɪˈvʲet")     // [prʲɪˈvʲet]}

\NormalTok{\#precise("prʲɪˈvʲet") // ⟦prʲɪˈvʲet⟧}
\NormalTok{\#prec("prʲɪˈvʲet")    // ⟦prʲɪˈvʲet⟧}

\NormalTok{\#phonemic("prɪvet") // /prɪvet/}
\NormalTok{\#phnm("prɪvet")     // /prɪvet/}

\NormalTok{\#morphophonemic("prɪvet") // ⫽prɪvet⫽}
\NormalTok{\#mphnm("prɪvet")          // ⫽prɪvet⫽}

\NormalTok{\#indistinguishable("prʲɪˈvʲet") // (prʲɪˈvʲet)}
\NormalTok{\#idst("prʲɪˈvʲet")              // (prʲɪˈvʲet)}

\NormalTok{\#obscured("prʲɪˈvʲet") // ⸨prʲɪˈvʲet⸩}
\NormalTok{\#obsc("prʲɪˈvʲet")     // ⸨prʲɪˈvʲet⸩}

\NormalTok{\#orthographic("привет") // ⟨привет⟩}
\NormalTok{\#orth("привет")         // ⟨привет⟩}

\NormalTok{\#transliterated("privyet") // ⟪privyet⟫}
\NormalTok{\#trlt("privyet")           // ⟪privyet⟫}

\NormalTok{\#prosodic("prʲɪˈvʲet") // \{prʲɪˈvʲet\}}
\NormalTok{\#prsd("prʲɪˈvʲet")     // \{prʲɪˈvʲet\}}
\end{Highlighting}
\end{Shaded}

\subsubsection{How to add}\label{how-to-add}

Copy this into your project and use the import as \texttt{\ ascii-ipa\ }

\begin{verbatim}
#import "@preview/ascii-ipa:2.0.0"
\end{verbatim}

\includesvg[width=0.16667in,height=0.16667in]{/assets/icons/16-copy.svg}

Check the docs for
\href{https://typst.app/docs/reference/scripting/\#packages}{more
information on how to import packages} .

\subsubsection{About}\label{about}

\begin{description}
\tightlist
\item[Author :]
imatpot
\item[License:]
MIT
\item[Current version:]
2.0.0
\item[Last updated:]
May 14, 2024
\item[First released:]
March 26, 2024
\item[Minimum Typst version:]
0.10.0
\item[Archive size:]
9.84 kB
\href{https://packages.typst.org/preview/ascii-ipa-2.0.0.tar.gz}{\pandocbounded{\includesvg[keepaspectratio]{/assets/icons/16-download.svg}}}
\item[Repository:]
\href{https://github.com/imatpot/typst-ascii-ipa}{GitHub}
\item[Discipline :]
\begin{itemize}
\tightlist
\item[]
\item
  \href{https://typst.app/universe/search/?discipline=linguistics}{Linguistics}
\end{itemize}
\item[Categor y :]
\begin{itemize}
\tightlist
\item[]
\item
  \pandocbounded{\includesvg[keepaspectratio]{/assets/icons/16-text.svg}}
  \href{https://typst.app/universe/search/?category=text}{Text}
\end{itemize}
\end{description}

\subsubsection{Where to report issues?}\label{where-to-report-issues}

This package is a project of imatpot . Report issues on
\href{https://github.com/imatpot/typst-ascii-ipa}{their repository} .
You can also try to ask for help with this package on the
\href{https://forum.typst.app}{Forum} .

Please report this package to the Typst team using the
\href{https://typst.app/contact}{contact form} if you believe it is a
safety hazard or infringes upon your rights.

\phantomsection\label{versions}
\subsubsection{Version history}\label{version-history}

\begin{longtable}[]{@{}ll@{}}
\toprule\noalign{}
Version & Release Date \\
\midrule\noalign{}
\endhead
\bottomrule\noalign{}
\endlastfoot
2.0.0 & May 14, 2024 \\
\href{https://typst.app/universe/package/ascii-ipa/1.1.1/}{1.1.1} &
March 26, 2024 \\
\href{https://typst.app/universe/package/ascii-ipa/1.1.0/}{1.1.0} &
March 26, 2024 \\
\href{https://typst.app/universe/package/ascii-ipa/1.0.0/}{1.0.0} &
March 26, 2024 \\
\end{longtable}

Typst GmbH did not create this package and cannot guarantee correct
functionality of this package or compatibility with any version of the
Typst compiler or app.


\section{Package List LaTeX/tidy.tex}
\title{typst.app/universe/package/tidy}

\phantomsection\label{banner}
\section{tidy}\label{tidy}

{ 0.3.0 }

Documentation generator for Typst code in Typst.

{ } Featured Package

\phantomsection\label{readme}
\emph{Keep it tidy.}

\href{https://typst.app/universe/package/tidy}{\pandocbounded{\includegraphics[keepaspectratio]{https://img.shields.io/badge/dynamic/toml?url=https\%3A\%2F\%2Fraw.githubusercontent.com\%2FMc-Zen\%2Ftidy\%2Fmain\%2Ftypst.toml&query=\%24.package.version&prefix=v&logo=typst&label=package&color=239DAD}}}
\href{https://github.com/Mc-Zen/tidy/blob/main/LICENSE}{\pandocbounded{\includegraphics[keepaspectratio]{https://img.shields.io/badge/license-MIT-blue}}}
\href{https://github.com/Mc-Zen/tidy/releases/download/v0.3.0/tidy-guide.pdf}{\pandocbounded{\includegraphics[keepaspectratio]{https://img.shields.io/badge/manual-.pdf-purple}}}

\textbf{tidy} is a package that generates documentation directly in
\href{https://typst.app/}{Typst} for your Typst modules. It parses
docstring comments similar to javadoc and co. and can be used to easily
build a beautiful reference section for the parsed module. Within the
docstring you may use (almost) any Typst syntax âˆ' so markup, equations
and even figures are no problem!

Features:

\begin{itemize}
\tightlist
\item
  \textbf{Customizable} output styles.
\item
  Automatically
  \href{https://github.com/typst/packages/raw/main/packages/preview/tidy/0.3.0/\#example}{\textbf{render
  code examples}} .
\item
  \textbf{Annotate types} of parameters and return values.
\item
  Automatically read off default values for named parameters.
\item
  \href{https://github.com/typst/packages/raw/main/packages/preview/tidy/0.3.0/\#generate-a-help-command-for-you-package}{\textbf{Help}
  feature} for your package.
\item
  \href{https://github.com/typst/packages/raw/main/packages/preview/tidy/0.3.0/\#docstring-tests}{Docstring
  tests} .
\end{itemize}

The
\href{https://github.com/Mc-Zen/tidy/releases/download/v0.3.0/tidy-guide.pdf}{guide}
fully describes the usage of this module and defines the format for the
docstrings.

\subsection{Usage}\label{usage}

Using \texttt{\ tidy\ } is as simple as writing some docstrings and
calling:

\begin{Shaded}
\begin{Highlighting}[]
\NormalTok{\#import "@preview/tidy:0.3.0"}

\NormalTok{\#let docs = tidy.parse{-}module(read("my{-}module.typ"))}
\NormalTok{\#tidy.show{-}module(docs, style: tidy.styles.default)}
\end{Highlighting}
\end{Shaded}

The available predefined styles are currenty
\texttt{\ tidy.styles.default\ } and \texttt{\ tidy.styles.minimal\ } .
Custom styles can be added by hand (take a look at the
\href{https://github.com/Mc-Zen/tidy/releases/download/v0.3.0/tidy-guide.pdf}{guide}
).

\subsection{Example}\label{example}

A full example on how to use this module for your own package (maybe
even consisting of multiple files) can be found at
\href{https://github.com/Mc-Zen/tidy/tree/main/examples}{examples} .

\begin{Shaded}
\begin{Highlighting}[]
\NormalTok{/// This function computes the cardinal sine, $sinc(x)=sin(x)/x$. }
\NormalTok{///}
\NormalTok{/// \#example(\textasciigrave{}\#sinc(0)\textasciigrave{}, mode: "markup")}
\NormalTok{///}
\NormalTok{/// {-} x (int, float): The argument for the cardinal sine function. }
\NormalTok{/// {-}\textgreater{} float}
\NormalTok{\#let sinc(x) = if x == 0 \{1\} else \{calc.sin(x) / x\}}
\end{Highlighting}
\end{Shaded}

\textbf{tidy} turns this into:

\subsubsection{\texorpdfstring{\protect\pandocbounded{\includesvg[keepaspectratio]{https://github.com/typst/packages/raw/main/packages/preview/tidy/0.3.0/docs/images/sincx-docs.svg}}}{Tidy example output}}\label{tidy-example-output}

\subsection{Access user-defined functions and
images}\label{access-user-defined-functions-and-images}

The code in the docstrings is evaluated via \texttt{\ eval()\ } . In
order to access user-defined functions and images, you can make use of
the \texttt{\ scope\ } argument of \texttt{\ tidy.parse-module()\ } :

\begin{Shaded}
\begin{Highlighting}[]
\NormalTok{\#\{}
\NormalTok{    import "my{-}module.typ"}
\NormalTok{    let module = tidy.parse{-}module(read("my{-}module.typ"))}
\NormalTok{    let an{-}image = image("img.png")}
\NormalTok{    tidy.show{-}module(}
\NormalTok{        module,}
\NormalTok{        style: tidy.styles.default,}
\NormalTok{        scope: (my{-}module: my{-}module, img: an{-}image)}
\NormalTok{    )}
\NormalTok{\}}
\end{Highlighting}
\end{Shaded}

The docstrings in \texttt{\ my-module.typ\ } may now access the image
with \texttt{\ \#img\ } and can call any function or variable from
\texttt{\ my-module\ } in the style of
\texttt{\ \#my-module.my-function()\ } . This makes rendering examples
right in the docstrings as easy as a breeze!

\subsection{Generate a help command for you
package}\label{generate-a-help-command-for-you-package}

With \textbf{tidy} , you can add a help command to you package that
allows users to obtain the documentation of a specific definition or
parameter right in the document. This is similar to CLI-style help
commands. If you have already written docstrings for your package, it is
quite low-effort to add this feature. Once set up, the end-user can use
it like this:

\begin{Shaded}
\begin{Highlighting}[]
\NormalTok{// happily coding, but how do I use this one complex function again?}

\NormalTok{\#mypackage.help("func")}
\NormalTok{\#mypackage.help("func(param1)") // print only parameter description of param1}
\end{Highlighting}
\end{Shaded}

This will print the documentation of \texttt{\ func\ } directly into the
document â€'' no need to look it up in a manual. Read up in the
\href{https://github.com/Mc-Zen/tidy/releases/download/v0.3.0/tidy-guide.pdf}{guide}
for setup instructions.

\subsection{Docstring tests}\label{docstring-tests}

It is possible to add simple docstring tests â€'' assertions that will
be run when the documentation is generated. This is useful if you want
to keep small tests and documentation in one place.

\begin{Shaded}
\begin{Highlighting}[]
\NormalTok{/// \#test(}
\NormalTok{///   \textasciigrave{}num.my{-}square(2) == 4\textasciigrave{},}
\NormalTok{///   \textasciigrave{}num.my{-}square(4) == 16\textasciigrave{},}
\NormalTok{/// )}
\NormalTok{\#let my{-}square(n) = n * n}
\end{Highlighting}
\end{Shaded}

With the short-hand syntax, a unfulfilled assertion will even print the
line number of the failed test:

\begin{Shaded}
\begin{Highlighting}[]
\NormalTok{/// \textgreater{}\textgreater{}\textgreater{} my{-}square(2) == 4}
\NormalTok{/// \textgreater{}\textgreater{}\textgreater{} my{-}square(4) == 16}
\NormalTok{\#let my{-}square(n) = n * n}
\end{Highlighting}
\end{Shaded}

A few test assertion functions are available to improve readability,
simplicity, and error messages. Currently, these are
\texttt{\ eq(a,\ b)\ } for equality tests, \texttt{\ ne(a,\ b)\ } for
inequality tests and \texttt{\ approx(a,\ b,\ eps:\ 1e-10)\ } for
floating point comparisons. These assertion helper functions are always
available within docstring tests (with both \texttt{\ test()\ } and
\texttt{\ \textgreater{}\textgreater{}\textgreater{}\ } syntax).

\subsection{Changelog}\label{changelog}

\subsubsection{v0.3.0}\label{v0.3.0}

\begin{itemize}
\tightlist
\item
  New features:

  \begin{itemize}
  \tightlist
  \item
    Help feature.
  \item
    \texttt{\ preamble\ } option for examples (e.g., to add
    \texttt{\ import\ } statements).
  \item
    more options for \texttt{\ show-module\ } :
    \texttt{\ omit-private-definitions\ } ,
    \texttt{\ omit-private-parameters\ } ,
    \texttt{\ enable-cross-references\ } , \texttt{\ local-names\ } (for
    configuring language-specific strings).
  \end{itemize}
\item
  Improvements:

  \begin{itemize}
  \tightlist
  \item
    Allow using \texttt{\ show-example()\ } as standalone.
  \item
    Updated type names that changed with Typst 0.8.0, e.g., integer
    -\textgreater{} int.
  \end{itemize}
\item
  Fixes:

  \begin{itemize}
  \tightlist
  \item
    allow examples with ratio widths if \texttt{\ scale-preview\ } is
    not \texttt{\ auto\ } .
  \item
    \texttt{\ show-outline\ }
  \item
    explicitly use \texttt{\ raw(lang:\ none)\ } for types and function
    names.
  \end{itemize}
\end{itemize}

\subsubsection{v0.2.0}\label{v0.2.0}

\begin{itemize}
\tightlist
\item
  New features:

  \begin{itemize}
  \tightlist
  \item
    Add executable examples to docstrings.
  \item
    Documentation for variables (as well as functions).
  \item
    Docstring tests.
  \item
    Rainbow-colored types \texttt{\ color\ } and \texttt{\ gradient\ } .
  \end{itemize}
\item
  Improvements:

  \begin{itemize}
  \tightlist
  \item
    Allow customization of cross-references through
    \texttt{\ show-reference()\ } .
  \item
    Allow customization of spacing between functions through styles.
  \item
    Allow color customization (especially for the \texttt{\ default\ }
    theme).
  \end{itemize}
\item
  Fixes:

  \begin{itemize}
  \tightlist
  \item
    Empty parameter descriptions are omitted (if the corresponding
    option is set).
  \item
    Trim newline characters from parameter descriptions.
  \end{itemize}
\item
  âš~ï¸? Breaking changes:

  \begin{itemize}
  \tightlist
  \item
    Before, cross-references for functions using the \texttt{\ @@\ }
    syntax could omit the function parentheses. Now this is not possible
    anymore, since such references refer to variables now.
  \item
    (only concerning custom styles) The style functions
    \texttt{\ show-outline()\ } , \texttt{\ show-parameter-list\ } , and
    \texttt{\ show-type()\ } now take \texttt{\ style-args\ } arguments
    as well.
  \end{itemize}
\end{itemize}

\subsubsection{v0.1.0}\label{v0.1.0}

Initial Release.

\subsubsection{How to add}\label{how-to-add}

Copy this into your project and use the import as \texttt{\ tidy\ }

\begin{verbatim}
#import "@preview/tidy:0.3.0"
\end{verbatim}

\includesvg[width=0.16667in,height=0.16667in]{/assets/icons/16-copy.svg}

Check the docs for
\href{https://typst.app/docs/reference/scripting/\#packages}{more
information on how to import packages} .

\subsubsection{About}\label{about}

\begin{description}
\tightlist
\item[Author :]
\href{https://github.com/Mc-Zen}{Mc-Zen}
\item[License:]
MIT
\item[Current version:]
0.3.0
\item[Last updated:]
May 14, 2024
\item[First released:]
August 8, 2023
\item[Minimum Typst version:]
0.6.0
\item[Archive size:]
17.6 kB
\href{https://packages.typst.org/preview/tidy-0.3.0.tar.gz}{\pandocbounded{\includesvg[keepaspectratio]{/assets/icons/16-download.svg}}}
\item[Repository:]
\href{https://github.com/Mc-Zen/tidy}{GitHub}
\item[Categor ies :]
\begin{itemize}
\tightlist
\item[]
\item
  \pandocbounded{\includesvg[keepaspectratio]{/assets/icons/16-hammer.svg}}
  \href{https://typst.app/universe/search/?category=utility}{Utility}
\item
  \pandocbounded{\includesvg[keepaspectratio]{/assets/icons/16-code.svg}}
  \href{https://typst.app/universe/search/?category=scripting}{Scripting}
\item
  \pandocbounded{\includesvg[keepaspectratio]{/assets/icons/16-list-unordered.svg}}
  \href{https://typst.app/universe/search/?category=model}{Model}
\end{itemize}
\end{description}

\subsubsection{Where to report issues?}\label{where-to-report-issues}

This package is a project of Mc-Zen . Report issues on
\href{https://github.com/Mc-Zen/tidy}{their repository} . You can also
try to ask for help with this package on the
\href{https://forum.typst.app}{Forum} .

Please report this package to the Typst team using the
\href{https://typst.app/contact}{contact form} if you believe it is a
safety hazard or infringes upon your rights.

\phantomsection\label{versions}
\subsubsection{Version history}\label{version-history}

\begin{longtable}[]{@{}ll@{}}
\toprule\noalign{}
Version & Release Date \\
\midrule\noalign{}
\endhead
\bottomrule\noalign{}
\endlastfoot
0.3.0 & May 14, 2024 \\
\href{https://typst.app/universe/package/tidy/0.2.0/}{0.2.0} & January
3, 2024 \\
\href{https://typst.app/universe/package/tidy/0.1.0/}{0.1.0} & August 8,
2023 \\
\end{longtable}

Typst GmbH did not create this package and cannot guarantee correct
functionality of this package or compatibility with any version of the
Typst compiler or app.


\section{Package List LaTeX/charged-ieee.tex}
\title{typst.app/universe/package/charged-ieee}

\phantomsection\label{banner}
\phantomsection\label{template-thumbnail}
\pandocbounded{\includegraphics[keepaspectratio]{https://packages.typst.org/preview/thumbnails/charged-ieee-0.1.3-small.webp}}

\section{charged-ieee}\label{charged-ieee}

{ 0.1.3 }

An IEEE-style paper template to publish at conferences and journals for
Electrical Engineering, Computer Science, and Computer Engineering

{ } Featured Template

\href{/app?template=charged-ieee&version=0.1.3}{Create project in app}

\phantomsection\label{readme}
This is a Typst template for a two-column paper from the proceedings of
the Institute of Electrical and Electronics Engineers. The paper is
tightly spaced, fits a lot of content and comes preconfigured for
numeric citations from BibLaTeX or Hayagriva files.

\subsection{Usage}\label{usage}

You can use this template in the Typst web app by clicking “Start from
template� on the dashboard and searching for \texttt{\ charged-ieee\ }
.

Alternatively, you can use the CLI to kick this project off using the
command

\begin{verbatim}
typst init @preview/charged-ieee
\end{verbatim}

Typst will create a new directory with all the files needed to get you
started.

\subsection{Configuration}\label{configuration}

This template exports the \texttt{\ ieee\ } function with the following
named arguments:

\begin{itemize}
\tightlist
\item
  \texttt{\ title\ } : The paper’s title as content.
\item
  \texttt{\ authors\ } : An array of author dictionaries. Each of the
  author dictionaries must have a \texttt{\ name\ } key and can have the
  keys \texttt{\ department\ } , \texttt{\ organization\ } ,
  \texttt{\ location\ } , and \texttt{\ email\ } . All keys accept
  content.
\item
  \texttt{\ abstract\ } : The content of a brief summary of the paper or
  \texttt{\ none\ } . Appears at the top of the first column in
  boldface.
\item
  \texttt{\ index-terms\ } : Array of index terms to display after the
  abstract. Shall be \texttt{\ content\ } .
\item
  \texttt{\ paper-size\ } : Defaults to \texttt{\ us-letter\ } . Specify
  a
  \href{https://typst.app/docs/reference/layout/page/\#parameters-paper}{paper
  size string} to change the page format.
\item
  \texttt{\ bibliography\ } : The result of a call to the
  \texttt{\ bibliography\ } function or \texttt{\ none\ } . Specifying
  this will configure numeric, IEEE-style citations.
\end{itemize}

The function also accepts a single, positional argument for the body of
the paper.

The template will initialize your package with a sample call to the
\texttt{\ ieee\ } function in a show rule. If you want to change an
existing project to use this template, you can add a show rule like this
at the top of your file:

\begin{Shaded}
\begin{Highlighting}[]
\NormalTok{\#import "@preview/charged{-}ieee:0.1.3": ieee}

\NormalTok{\#show: ieee.with(}
\NormalTok{  title: [A typesetting system to untangle the scientific writing process],}
\NormalTok{  abstract: [}
\NormalTok{    The process of scientific writing is often tangled up with the intricacies of typesetting, leading to frustration and wasted time for researchers. In this paper, we introduce Typst, a new typesetting system designed specifically for scientific writing. Typst untangles the typesetting process, allowing researchers to compose papers faster. In a series of experiments we demonstrate that Typst offers several advantages, including faster document creation, simplified syntax, and increased ease{-}of{-}use.}
\NormalTok{  ],}
\NormalTok{  authors: (}
\NormalTok{    (}
\NormalTok{      name: "Martin Haug",}
\NormalTok{      department: [Co{-}Founder],}
\NormalTok{      organization: [Typst GmbH],}
\NormalTok{      location: [Berlin, Germany],}
\NormalTok{      email: "haug@typst.app"}
\NormalTok{    ),}
\NormalTok{    (}
\NormalTok{      name: "Laurenz Mädje",}
\NormalTok{      department: [Co{-}Founder],}
\NormalTok{      organization: [Typst GmbH],}
\NormalTok{      location: [Berlin, Germany],}
\NormalTok{      email: "maedje@typst.app"}
\NormalTok{    ),}
\NormalTok{  ),}
\NormalTok{  index{-}terms: ("Scientific writing", "Typesetting", "Document creation", "Syntax"),}
\NormalTok{  bibliography: bibliography("refs.bib"),}
\NormalTok{)}

\NormalTok{// Your content goes below.}
\end{Highlighting}
\end{Shaded}

\href{/app?template=charged-ieee&version=0.1.3}{Create project in app}

\subsubsection{How to use}\label{how-to-use}

Click the button above to create a new project using this template in
the Typst app.

You can also use the Typst CLI to start a new project on your computer
using this command:

\begin{verbatim}
typst init @preview/charged-ieee:0.1.3
\end{verbatim}

\includesvg[width=0.16667in,height=0.16667in]{/assets/icons/16-copy.svg}

\subsubsection{About}\label{about}

\begin{description}
\tightlist
\item[Author :]
\href{https://typst.app}{Typst GmbH}
\item[License:]
MIT-0
\item[Current version:]
0.1.3
\item[Last updated:]
October 29, 2024
\item[First released:]
March 6, 2024
\item[Minimum Typst version:]
0.12.0
\item[Archive size:]
6.39 kB
\href{https://packages.typst.org/preview/charged-ieee-0.1.3.tar.gz}{\pandocbounded{\includesvg[keepaspectratio]{/assets/icons/16-download.svg}}}
\item[Repository:]
\href{https://github.com/typst/templates}{GitHub}
\item[Discipline s :]
\begin{itemize}
\tightlist
\item[]
\item
  \href{https://typst.app/universe/search/?discipline=computer-science}{Computer
  Science}
\item
  \href{https://typst.app/universe/search/?discipline=engineering}{Engineering}
\end{itemize}
\item[Categor y :]
\begin{itemize}
\tightlist
\item[]
\item
  \pandocbounded{\includesvg[keepaspectratio]{/assets/icons/16-atom.svg}}
  \href{https://typst.app/universe/search/?category=paper}{Paper}
\end{itemize}
\end{description}

\subsubsection{Where to report issues?}\label{where-to-report-issues}

This template is a project of Typst GmbH . Report issues on
\href{https://github.com/typst/templates}{their repository} . You can
also try to ask for help with this template on the
\href{https://forum.typst.app}{Forum} .

\phantomsection\label{versions}
\subsubsection{Version history}\label{version-history}

\begin{longtable}[]{@{}ll@{}}
\toprule\noalign{}
Version & Release Date \\
\midrule\noalign{}
\endhead
\bottomrule\noalign{}
\endlastfoot
0.1.3 & October 29, 2024 \\
\href{https://typst.app/universe/package/charged-ieee/0.1.2/}{0.1.2} &
August 15, 2024 \\
\href{https://typst.app/universe/package/charged-ieee/0.1.1/}{0.1.1} &
August 8, 2024 \\
\href{https://typst.app/universe/package/charged-ieee/0.1.0/}{0.1.0} &
March 6, 2024 \\
\end{longtable}


\section{Package List LaTeX/algo.tex}
\title{typst.app/universe/package/algo}

\phantomsection\label{banner}
\section{algo}\label{algo}

{ 0.3.4 }

Beautifully typeset algorithms.

\phantomsection\label{readme}
A Typst library for writing algorithms. On Typst v0.6.0+ you can import
the \texttt{\ algo\ } package:

\begin{Shaded}
\begin{Highlighting}[]
\NormalTok{\#import "@preview/algo:0.3.4": algo, i, d, comment, code}
\end{Highlighting}
\end{Shaded}

Otherwise, add the \texttt{\ algo.typ\ } file to your project and import
it as normal:

\begin{Shaded}
\begin{Highlighting}[]
\NormalTok{\#import "algo.typ": algo, i, d, comment, code}
\end{Highlighting}
\end{Shaded}

Use the \texttt{\ algo\ } function for writing pseudocode and the
\texttt{\ code\ } function for writing code blocks with line numbers.
Check out the
\href{https://github.com/typst/packages/raw/main/packages/preview/algo/0.3.4/\#examples}{examples}
below for a quick overview. See the
\href{https://github.com/typst/packages/raw/main/packages/preview/algo/0.3.4/\#usage}{usage}
section to read about all the options each function has.

\subsection{Examples}\label{examples}

Here’s a basic use of \texttt{\ algo\ } :

\begin{Shaded}
\begin{Highlighting}[]
\NormalTok{\#algo(}
\NormalTok{  title: "Fib",}
\NormalTok{  parameters: ("n",)}
\NormalTok{)[}
\NormalTok{  if $n \textless{} 0$:\#i\textbackslash{}        // use \#i to indent the following lines}
\NormalTok{    return null\#d\textbackslash{}      // use \#d to dedent the following lines}
\NormalTok{  if $n = 0$ or $n = 1$:\#i \#comment[you can also]\textbackslash{}}
\NormalTok{    return $n$\#d \#comment[add comments!]\textbackslash{}}
\NormalTok{  return \#smallcaps("Fib")$(n{-}1) +$ \#smallcaps("Fib")$(n{-}2)$}
\NormalTok{]}
\end{Highlighting}
\end{Shaded}

\includegraphics[width=4.16667in,height=\textheight,keepaspectratio]{https://user-images.githubusercontent.com/40146328/235323240-e59ed7e2-ebb6-4b80-8742-eb171dd3721e.png}\\

Here’s a use of \texttt{\ algo\ } without a title, parameters, line
numbers, or syntax highlighting:

\begin{Shaded}
\begin{Highlighting}[]
\NormalTok{\#algo(}
\NormalTok{  line{-}numbers: false,}
\NormalTok{  strong{-}keywords: false}
\NormalTok{)[}
\NormalTok{  if $n \textless{} 0$:\#i\textbackslash{}}
\NormalTok{    return null\#d\textbackslash{}}
\NormalTok{  if $n = 0$ or $n = 1$:\#i\textbackslash{}}
\NormalTok{    return $n$\#d\textbackslash{}}
\NormalTok{  \textbackslash{}}
\NormalTok{  let $x \textless{}{-} 0$\textbackslash{}}
\NormalTok{  let $y \textless{}{-} 1$\textbackslash{}}
\NormalTok{  for $i \textless{}{-} 2$ to $n{-}1$:\#i \#comment[so dynamic!]\textbackslash{}}
\NormalTok{    let $z \textless{}{-} x+y$\textbackslash{}}
\NormalTok{    $x \textless{}{-} y$\textbackslash{}}
\NormalTok{    $y \textless{}{-} z$\#d\textbackslash{}}
\NormalTok{    \textbackslash{}}
\NormalTok{  return $x+y$}
\NormalTok{]}
\end{Highlighting}
\end{Shaded}

\includegraphics[width=3.125in,height=\textheight,keepaspectratio]{https://user-images.githubusercontent.com/40146328/235323261-d6e7a42c-ffb7-4c3a-bd2a-4c8fc2df5f36.png}\\

And here’s \texttt{\ algo\ } with more styling options:

\begin{Shaded}
\begin{Highlighting}[]
\NormalTok{\#algo(}
\NormalTok{  title: [                    // note that title and parameters}
\NormalTok{    \#set text(size: 15pt)     // can be content}
\NormalTok{    \#emph(smallcaps("Fib"))}
\NormalTok{  ],}
\NormalTok{  parameters: ([\#math.italic("n")],),}
\NormalTok{  comment{-}prefix: [\#sym.triangle.stroked.r ],}
\NormalTok{  comment{-}styles: (fill: rgb(100\%, 0\%, 0\%)),}
\NormalTok{  indent{-}size: 15pt,}
\NormalTok{  indent{-}guides: 1pt + gray,}
\NormalTok{  row{-}gutter: 5pt,}
\NormalTok{  column{-}gutter: 5pt,}
\NormalTok{  inset: 5pt,}
\NormalTok{  stroke: 2pt + black,}
\NormalTok{  fill: none,}
\NormalTok{)[}
\NormalTok{  if $n \textless{} 0$:\#i\textbackslash{}}
\NormalTok{    return null\#d\textbackslash{}}
\NormalTok{  if $n = 0$ or $n = 1$:\#i\textbackslash{}}
\NormalTok{    return $n$\#d\textbackslash{}}
\NormalTok{  \textbackslash{}}
\NormalTok{  let $x \textless{}{-} 0$\textbackslash{}}
\NormalTok{  let $y \textless{}{-} 1$\textbackslash{}}
\NormalTok{  for $i \textless{}{-} 2$ to $n{-}1$:\#i \#comment[so dynamic!]\textbackslash{}}
\NormalTok{    let $z \textless{}{-} x+y$\textbackslash{}}
\NormalTok{    $x \textless{}{-} y$\textbackslash{}}
\NormalTok{    $y \textless{}{-} z$\#d\textbackslash{}}
\NormalTok{    \textbackslash{}}
\NormalTok{  return $x+y$}
\NormalTok{]}
\end{Highlighting}
\end{Shaded}

\includegraphics[width=3.125in,height=\textheight,keepaspectratio]{https://github.com/platformer/typst-algorithms/assets/40146328/89f80b5d-bdb2-420a-935d-24f43ca597d8}

Here’s a basic use of \texttt{\ code\ } :

\begin{Shaded}
\begin{Highlighting}[]
\NormalTok{\#code()[}
\NormalTok{  \textasciigrave{}\textasciigrave{}\textasciigrave{}py}
\NormalTok{  def fib(n):}
\NormalTok{    if n \textless{} 0:}
\NormalTok{      return None}
\NormalTok{    if n == 0 or n == 1:        \# this comment is}
\NormalTok{      return n                  \# normal raw text}
\NormalTok{    return fib(n{-}1) + fib(n{-}2)}
\NormalTok{  \textasciigrave{}\textasciigrave{}\textasciigrave{}}
\NormalTok{]}
\end{Highlighting}
\end{Shaded}

\includegraphics[width=4.16667in,height=\textheight,keepaspectratio]{https://user-images.githubusercontent.com/40146328/235324088-a3596e0b-af90-4da3-b326-2de11158baac.png}\\

And here’s \texttt{\ code\ } with some styling options:

\begin{Shaded}
\begin{Highlighting}[]
\NormalTok{\#code(}
\NormalTok{  indent{-}guides: 1pt + gray,}
\NormalTok{  row{-}gutter: 5pt,}
\NormalTok{  column{-}gutter: 5pt,}
\NormalTok{  inset: 5pt,}
\NormalTok{  stroke: 2pt + black,}
\NormalTok{  fill: none,}
\NormalTok{)[}
\NormalTok{  \textasciigrave{}\textasciigrave{}\textasciigrave{}py}
\NormalTok{  def fib(n):}
\NormalTok{      if n \textless{} 0:}
\NormalTok{          return None}
\NormalTok{      if n == 0 or n == 1:        \# this comment is}
\NormalTok{          return n                \# normal raw text}
\NormalTok{      return fib(n{-}1) + fib(n{-}2)}
\NormalTok{  \textasciigrave{}\textasciigrave{}\textasciigrave{}}
\NormalTok{]}
\end{Highlighting}
\end{Shaded}

\includegraphics[width=4.16667in,height=\textheight,keepaspectratio]{https://github.com/platformer/typst-algorithms/assets/40146328/c091ac43-6861-40bc-8046-03ea285712c3}

\subsection{Usage}\label{usage}

\subsubsection{\texorpdfstring{\texttt{\ algo\ }}{ algo }}\label{algo-1}

Makes a pseudocode element.

\begin{Shaded}
\begin{Highlighting}[]
\NormalTok{algo(}
\NormalTok{  body,}
\NormalTok{  header: none,}
\NormalTok{  title: none,}
\NormalTok{  parameters: (),}
\NormalTok{  line{-}numbers: true,}
\NormalTok{  strong{-}keywords: true,}
\NormalTok{  keywords: \_algo{-}default{-}keywords, // see below}
\NormalTok{  comment{-}prefix: "// ",}
\NormalTok{  indent{-}size: 20pt,}
\NormalTok{  indent{-}guides: none,}
\NormalTok{  indent{-}guides{-}offset: 0pt,}
\NormalTok{  row{-}gutter: 10pt,}
\NormalTok{  column{-}gutter: 10pt,}
\NormalTok{  inset: 10pt,}
\NormalTok{  fill: rgb(98\%, 98\%, 98\%),}
\NormalTok{  stroke: 1pt + rgb(50\%, 50\%, 50\%),}
\NormalTok{  radius: 0pt,}
\NormalTok{  breakable: false,}
\NormalTok{  block{-}align: center,}
\NormalTok{  main{-}text{-}styles: (:),}
\NormalTok{  comment{-}styles: (fill: rgb(45\%, 45\%, 45\%)),}
\NormalTok{  line{-}number{-}styles: (:)}
\NormalTok{)}
\end{Highlighting}
\end{Shaded}

\textbf{Parameters:}

\begin{itemize}
\item
  \texttt{\ body\ } : \texttt{\ content\ } â€'' Main algorithm content.
\item
  \texttt{\ header\ } : \texttt{\ content\ } â€'' Algorithm header. If
  specified, \texttt{\ title\ } and \texttt{\ parameters\ } are ignored.
\item
  \texttt{\ title\ } : \texttt{\ string\ } or \texttt{\ content\ } â€''
  Algorithm title. Ignored if \texttt{\ header\ } is specified.
\item
  \texttt{\ Parameters\ } : \texttt{\ array\ } â€'' List of algorithm
  parameters. Elements can be \texttt{\ string\ } or
  \texttt{\ content\ } values. \texttt{\ string\ } values will
  automatically be displayed in math mode. Ignored if
  \texttt{\ header\ } is specified.
\item
  \texttt{\ line-numbers\ } : \texttt{\ boolean\ } â€'' Whether to
  display line numbers.
\item
  \texttt{\ strong-keywords\ } : \texttt{\ boolean\ } â€'' Whether to
  strongly emphasize keywords.
\item
  \texttt{\ keywords\ } : \texttt{\ array\ } â€'' List of terms to
  receive strong emphasis. Elements must be \texttt{\ string\ } values.
  Ignored if \texttt{\ strong-keywords\ } is \texttt{\ false\ } .

  The default list of keywords is stored in
  \texttt{\ \_algo-default-keywords\ } . This list contains the
  following terms:

\begin{verbatim}
("if", "else", "then", "while", "for",
"repeat", "do", "until", ":", "end",
"and", "or", "not", "in", "to",
"down", "let", "return", "goto")
\end{verbatim}

  Note that for each of the above terms,
  \texttt{\ \_algo-default-keywords\ } also contains the uppercase form
  of the term (e.g. “for� and “For�).
\item
  \texttt{\ comment-prefix\ } : \texttt{\ content\ } â€'' What to
  prepend comments with.
\item
  \texttt{\ indent-size\ } : \texttt{\ length\ } â€'' Size of line
  indentations.
\item
  \texttt{\ indent-guides\ } : \texttt{\ stroke\ } â€'' Stroke for
  indent guides.
\item
  \texttt{\ indent-guides-offset\ } : \texttt{\ length\ } â€''
  Horizontal offset of indent guides.
\item
  \texttt{\ row-gutter\ } : \texttt{\ length\ } â€'' Space between
  lines.
\item
  \texttt{\ column-gutter\ } : \texttt{\ length\ } â€'' Space between
  line numbers, text, and comments.
\item
  \texttt{\ inset\ } : \texttt{\ length\ } â€'' Size of inner padding.
\item
  \texttt{\ fill\ } : \texttt{\ color\ } â€'' Fill color.
\item
  \texttt{\ stroke\ } : \texttt{\ stroke\ } â€'' Stroke for the
  element’s border.
\item
  \texttt{\ radius\ } : \texttt{\ length\ } â€'' Corner radius.
\item
  \texttt{\ breakable\ } : \texttt{\ boolean\ } â€'' Whether the element
  can break across pages. WARNING: indent guides may look off when
  broken across pages.
\item
  \texttt{\ block-align\ } : \texttt{\ none\ } or \texttt{\ alignment\ }
  or \texttt{\ 2d\ alignment\ } â€'' Alignment of the \texttt{\ algo\ }
  on the page. Using \texttt{\ none\ } will cause the internal
  \texttt{\ block\ } element to be returned as-is.
\item
  \texttt{\ main-text-styles\ } : \texttt{\ dictionary\ } â€'' Styling
  options for the main algorithm text. Supports all parameters in
  Typst’s native \texttt{\ text\ } function.
\item
  \texttt{\ comment-styles\ } : \texttt{\ dictionary\ } â€'' Styling
  options for comment text. Supports all parameters in Typst’s native
  \texttt{\ text\ } function.
\item
  \texttt{\ line-number-styles\ } : \texttt{\ dictionary\ } â€'' Styling
  options for line numbers. Supports all parameters in Typst’s native
  \texttt{\ text\ } function.
\end{itemize}

\subsubsection{\texorpdfstring{\texttt{\ i\ } and
\texttt{\ d\ }}{ i  and  d }}\label{i-and-d}

For use in an \texttt{\ algo\ } body. \texttt{\ \#i\ } indents all
following lines and \texttt{\ \#d\ } dedents all following lines.

\subsubsection{\texorpdfstring{\texttt{\ comment\ }}{ comment }}\label{comment}

For use in an \texttt{\ algo\ } body. Adds a comment to the line in
which it’s placed.

\begin{Shaded}
\begin{Highlighting}[]
\NormalTok{comment(}
\NormalTok{  body,}
\NormalTok{  inline: false}
\NormalTok{)}
\end{Highlighting}
\end{Shaded}

\textbf{Parameters:}

\begin{itemize}
\item
  \texttt{\ body\ } : \texttt{\ content\ } â€'' Comment content.
\item
  \texttt{\ inline\ } : \texttt{\ boolean\ } â€'' If true, the comment
  is displayed in place rather than on the right side.

  NOTE: inline comments will respect both \texttt{\ main-text-styles\ }
  and \texttt{\ comment-styles\ } , preferring
  \texttt{\ comment-styles\ } when the two conflict.

  NOTE: to make inline comments insensitive to
  \texttt{\ strong-keywords\ } , strong emphasis is disabled within
  them. This can be circumvented via the \texttt{\ text\ } function:

\begin{Shaded}
\begin{Highlighting}[]
\NormalTok{\#comment(inline: true)[\#text(weight: 700)[...]]}
\end{Highlighting}
\end{Shaded}
\end{itemize}

\subsubsection{\texorpdfstring{\texttt{\ no-emph\ }}{ no-emph }}\label{no-emph}

For use in an \texttt{\ algo\ } body. Prevents the passed content from
being strongly emphasized. If a word appears in your algorithm both as a
keyword and as normal text, you may escape the non-keyword usages via
this function.

\begin{Shaded}
\begin{Highlighting}[]
\NormalTok{no{-}emph(}
\NormalTok{  body}
\NormalTok{)}
\end{Highlighting}
\end{Shaded}

\textbf{Parameters:}

\begin{itemize}
\tightlist
\item
  \texttt{\ body\ } : \texttt{\ content\ } â€'' Content to display
  without emphasis.
\end{itemize}

\subsubsection{\texorpdfstring{\texttt{\ code\ }}{ code }}\label{code}

Makes a code block element.

\begin{Shaded}
\begin{Highlighting}[]
\NormalTok{code(}
\NormalTok{  body,}
\NormalTok{  line{-}numbers: true,}
\NormalTok{  indent{-}guides: none,}
\NormalTok{  indent{-}guides{-}offset: 0pt,}
\NormalTok{  tab{-}size: auto,}
\NormalTok{  row{-}gutter: 10pt,}
\NormalTok{  column{-}gutter: 10pt,}
\NormalTok{  inset: 10pt,}
\NormalTok{  fill: rgb(98\%, 98\%, 98\%),}
\NormalTok{  stroke: 1pt + rgb(50\%, 50\%, 50\%),}
\NormalTok{  radius: 0pt,}
\NormalTok{  breakable: false,}
\NormalTok{  block{-}align: center,}
\NormalTok{  main{-}text{-}styles: (:),}
\NormalTok{  line{-}number{-}styles: (:)}
\NormalTok{)}
\end{Highlighting}
\end{Shaded}

\textbf{Parameters:}

\begin{itemize}
\item
  \texttt{\ body\ } : \texttt{\ content\ } â€'' Main content. Expects
  \texttt{\ raw\ } text.
\item
  \texttt{\ line-numbers\ } : \texttt{\ boolean\ } â€'' Whether to
  display line numbers.
\item
  \texttt{\ indent-guides\ } : \texttt{\ stroke\ } â€'' Stroke for
  indent guides.
\item
  \texttt{\ indent-guides-offset\ } : \texttt{\ length\ } â€''
  Horizontal offset of indent guides.
\item
  \texttt{\ tab-size\ } : \texttt{\ integer\ } â€'' Amount of spaces
  that should be considered an indent. If unspecified, the tab size is
  determined automatically from the first instance of starting
  whitespace.
\item
  \texttt{\ row-gutter\ } : \texttt{\ length\ } â€'' Space between
  lines.
\item
  \texttt{\ column-gutter\ } : \texttt{\ length\ } â€'' Space between
  line numbers and text.
\item
  \texttt{\ inset\ } : \texttt{\ length\ } â€'' Size of inner padding.
\item
  \texttt{\ fill\ } : \texttt{\ color\ } â€'' Fill color.
\item
  \texttt{\ stroke\ } : \texttt{\ stroke\ } â€'' Stroke for the
  element’s border.
\item
  \texttt{\ radius\ } : \texttt{\ length\ } â€'' Corner radius.
\item
  \texttt{\ breakable\ } : \texttt{\ boolean\ } â€'' Whether the element
  can break across pages. WARNING: indent guides may look off when
  broken across pages.
\item
  \texttt{\ block-align\ } : \texttt{\ none\ } or \texttt{\ alignment\ }
  or \texttt{\ 2d\ alignment\ } â€'' Alignment of the \texttt{\ code\ }
  on the page. Using \texttt{\ none\ } will cause the internal
  \texttt{\ block\ } element to be returned as-is.
\item
  \texttt{\ main-text-styles\ } : \texttt{\ dictionary\ } â€'' Styling
  options for the main raw text. Supports all parameters in Typst’s
  native \texttt{\ text\ } function.
\item
  \texttt{\ line-number-styles\ } : \texttt{\ dictionary\ } â€'' Styling
  options for line numbers. Supports all parameters in Typst’s native
  \texttt{\ text\ } function.
\end{itemize}

\subsection{Contributing}\label{contributing}

PRs are welcome! And if you encounter any bugs or have any
requests/ideas, feel free to open an issue.

\subsubsection{How to add}\label{how-to-add}

Copy this into your project and use the import as \texttt{\ algo\ }

\begin{verbatim}
#import "@preview/algo:0.3.4"
\end{verbatim}

\includesvg[width=0.16667in,height=0.16667in]{/assets/icons/16-copy.svg}

Check the docs for
\href{https://typst.app/docs/reference/scripting/\#packages}{more
information on how to import packages} .

\subsubsection{About}\label{about}

\begin{description}
\tightlist
\item[Author :]
\href{https://github.com/platformer}{platformer}
\item[License:]
MIT
\item[Current version:]
0.3.4
\item[Last updated:]
November 12, 2024
\item[First released:]
August 8, 2023
\item[Archive size:]
10.5 kB
\href{https://packages.typst.org/preview/algo-0.3.4.tar.gz}{\pandocbounded{\includesvg[keepaspectratio]{/assets/icons/16-download.svg}}}
\item[Repository:]
\href{https://github.com/platformer/typst-algorithms}{GitHub}
\item[Discipline :]
\begin{itemize}
\tightlist
\item[]
\item
  \href{https://typst.app/universe/search/?discipline=computer-science}{Computer
  Science}
\end{itemize}
\item[Categor y :]
\begin{itemize}
\tightlist
\item[]
\item
  \pandocbounded{\includesvg[keepaspectratio]{/assets/icons/16-package.svg}}
  \href{https://typst.app/universe/search/?category=components}{Components}
\end{itemize}
\end{description}

\subsubsection{Where to report issues?}\label{where-to-report-issues}

This package is a project of platformer . Report issues on
\href{https://github.com/platformer/typst-algorithms}{their repository}
. You can also try to ask for help with this package on the
\href{https://forum.typst.app}{Forum} .

Please report this package to the Typst team using the
\href{https://typst.app/contact}{contact form} if you believe it is a
safety hazard or infringes upon your rights.

\phantomsection\label{versions}
\subsubsection{Version history}\label{version-history}

\begin{longtable}[]{@{}ll@{}}
\toprule\noalign{}
Version & Release Date \\
\midrule\noalign{}
\endhead
\bottomrule\noalign{}
\endlastfoot
0.3.4 & November 12, 2024 \\
\href{https://typst.app/universe/package/algo/0.3.3/}{0.3.3} & September
21, 2023 \\
\href{https://typst.app/universe/package/algo/0.3.2/}{0.3.2} & September
3, 2023 \\
\href{https://typst.app/universe/package/algo/0.3.1/}{0.3.1} & August
19, 2023 \\
\href{https://typst.app/universe/package/algo/0.3.0/}{0.3.0} & August 8,
2023 \\
\end{longtable}

Typst GmbH did not create this package and cannot guarantee correct
functionality of this package or compatibility with any version of the
Typst compiler or app.


\section{Package List LaTeX/showybox.tex}
\title{typst.app/universe/package/showybox}

\phantomsection\label{banner}
\section{showybox}\label{showybox}

{ 2.0.3 }

Colorful and customizable boxes for Typst

{ } Featured Package

\phantomsection\label{readme}
\textbf{Showybox} is a Typst package for creating colorful and
customizable boxes.

\subsection{Usage}\label{usage}

To use this library through the Typst package manager (for Typst 0.6.0
or greater), write
\texttt{\ \#import\ "@preview/showybox:2.0.2":\ showybox\ } at the
beginning of your Typst file.

Once imported, you can create an empty showybox by using the function
\texttt{\ showybox()\ } and giving a default body content inside the
parenthesis or outside them using squared brackets \texttt{\ {[}{]}\ } .

By default a \texttt{\ showybox\ } with these properties will be
created:

\begin{itemize}
\tightlist
\item
  No title
\item
  No shadow
\item
  Not breakable
\item
  Black borders
\item
  White background
\item
  \texttt{\ 5pt\ } of border radius
\item
  \texttt{\ 1pt\ } of border thickness
\end{itemize}

\begin{Shaded}
\begin{Highlighting}[]
\NormalTok{\#import "@preview/showybox:2.0.3": showybox}

\NormalTok{\#showybox(}
\NormalTok{  [Hello world!]}
\NormalTok{)}
\end{Highlighting}
\end{Shaded}

\subsubsection{\texorpdfstring{\protect\pandocbounded{\includegraphics[keepaspectratio]{https://github.com/typst/packages/raw/main/packages/preview/showybox/2.0.3/assets/hello-world.png}}}{Hello world! example}}\label{hello-world-example}

Looks quite simple, but the “magic� starts when adding a title,
color and shadows. The following code creates two “unique� boxes
with defined colors and custom borders:

\begin{Shaded}
\begin{Highlighting}[]
\NormalTok{// First showybox}
\NormalTok{\#showybox(}
\NormalTok{  frame: (}
\NormalTok{    border{-}color: red.darken(50\%),}
\NormalTok{    title{-}color: red.lighten(60\%),}
\NormalTok{    body{-}color: red.lighten(80\%)}
\NormalTok{  ),}
\NormalTok{  title{-}style: (}
\NormalTok{    color: black,}
\NormalTok{    weight: "regular",}
\NormalTok{    align: center}
\NormalTok{  ),}
\NormalTok{  shadow: (}
\NormalTok{    offset: 3pt,}
\NormalTok{  ),}
\NormalTok{  title: "Red{-}ish showybox with separated sections!",}
\NormalTok{  lorem(20),}
\NormalTok{  lorem(12)}
\NormalTok{)}

\NormalTok{// Second showybox}
\NormalTok{\#showybox(}
\NormalTok{  frame: (}
\NormalTok{    dash: "dashed",}
\NormalTok{    border{-}color: red.darken(40\%)}
\NormalTok{  ),}
\NormalTok{  body{-}style: (}
\NormalTok{    align: center}
\NormalTok{  ),}
\NormalTok{  sep: (}
\NormalTok{    dash: "dashed"}
\NormalTok{  ),}
\NormalTok{  shadow: (}
\NormalTok{      offset: (x: 2pt, y: 3pt),}
\NormalTok{    color: yellow.lighten(70\%)}
\NormalTok{  ),}
\NormalTok{  [This is an important message!],}
\NormalTok{  [Be careful outside. There are dangerous bananas!]}
\NormalTok{)}
\end{Highlighting}
\end{Shaded}

\subsubsection{\texorpdfstring{\protect\pandocbounded{\includegraphics[keepaspectratio]{https://github.com/typst/packages/raw/main/packages/preview/showybox/2.0.3/assets/two-easy-examples.png}}}{Further examples}}\label{further-examples}

\subsection{Reference}\label{reference}

The \texttt{\ showybox()\ } function can receive the following
parameters:

\begin{itemize}
\tightlist
\item
  \texttt{\ title\ } : A string used as the title of the showybox
\item
  \texttt{\ footer\ } : A string used as the footer of the showybox
\item
  \texttt{\ frame\ } : A dictionary containing the frame’s properties
\item
  \texttt{\ title-style\ } : A dictionary containing the title’s
  styles
\item
  \texttt{\ body-style\ } : A dictionary containing the body’s styles
\item
  \texttt{\ footer-style\ } : A dictionary containing the footer’s
  styles
\item
  \texttt{\ sep\ } : A dictionary containing the separator’s
  properties
\item
  \texttt{\ shadow\ } : A dictionary containing the shadow’s
  properties
\item
  \texttt{\ width\ } : A relative length indicating the showybox’s
  width
\item
  \texttt{\ align\ } : An unidimensional alignement for the showybox in
  the page
\item
  \texttt{\ breakable\ } : A boolean indicating whether a showybox can
  break if it reached an end of page
\item
  \texttt{\ spacing\ } : Space above and below the showybox
\item
  \texttt{\ above\ } : Space above the showybox
\item
  \texttt{\ below\ } : Space below the showybox
\item
  \texttt{\ body\ } : The content of the showybox
\end{itemize}

\subsubsection{Frame properties}\label{frame-properties}

\begin{itemize}
\tightlist
\item
  \texttt{\ title-color\ } : Color used as background color where the
  title goes (default is \texttt{\ black\ } )
\item
  \texttt{\ body-color\ } : Color used as background color where the
  body goes (default is \texttt{\ white\ } )
\item
  \texttt{\ footer-color\ } : Color used as background color where the
  footer goes (default is \texttt{\ luma(85)\ } )
\item
  \texttt{\ border-color\ } : Color used for the showybox’s border
  (default is \texttt{\ black\ } )
\item
  \texttt{\ inset\ } : Inset used for title, body and footer elements
  (default is \texttt{\ (x:\ 1em,\ y:\ 0.65em)\ } ) if none of the
  followings are given:

  \begin{itemize}
  \tightlist
  \item
    \texttt{\ title-inset\ } : Inset used for the title
  \item
    \texttt{\ body-inset\ } : Inset used for the body
  \item
    \texttt{\ footer-inset\ } : Inset used for the body
  \end{itemize}
\item
  \texttt{\ radius\ } : Showybox’s radius (default is \texttt{\ 5pt\ }
  )
\item
  \texttt{\ thickness\ } : Border thickness of the showybox (default is
  \texttt{\ 1pt\ } )
\item
  \texttt{\ dash\ } : Showybox’s border style (default is
  \texttt{\ solid\ } )
\end{itemize}

\subsubsection{Title styles}\label{title-styles}

\begin{itemize}
\tightlist
\item
  \texttt{\ color\ } : Text color (default is \texttt{\ white\ } )
\item
  \texttt{\ weight\ } : Text weight (default is \texttt{\ bold\ } )
\item
  \texttt{\ align\ } : Text align (default is \texttt{\ left\ } )
\item
  \texttt{\ sep-thickness\ } : Thickness of the separator between title
  and body (default is \texttt{\ 1pt\ } )
\item
  \texttt{\ boxed-style\ } : If it’s a dictionary of properties,
  indicates that the title must appear like a “floating box� above
  the showybox. If it’s \texttt{\ none\ } , the title appears normally
  (default is \texttt{\ none\ } )
\end{itemize}

\paragraph{Boxed styles}\label{boxed-styles}

\begin{itemize}
\tightlist
\item
  \texttt{\ anchor\ } : Anchor of the boxed title

  \begin{itemize}
  \tightlist
  \item
    \texttt{\ y\ } : Vertical anchor ( \texttt{\ top\ } ,
    \texttt{\ horizon\ } or \texttt{\ bottom\ } â€`` default is
    \texttt{\ horizon\ } )
  \item
    \texttt{\ x\ } : Horizontal anchor ( \texttt{\ left\ } ,
    \texttt{\ start\ } , \texttt{\ center\ } , \texttt{\ right\ } ,
    \texttt{\ end\ } â€`` default is \texttt{\ left\ } )
  \end{itemize}
\item
  \texttt{\ offset\ } : How much to offset the boxed title in x and y
  direction as a dictionary with keys \texttt{\ x\ } and \texttt{\ y\ }
  (default is \texttt{\ 0pt\ } )
\item
  \texttt{\ radius\ } : Boxed title radius as a dictionary or relative
  length (default is \texttt{\ 5pt\ } )
\end{itemize}

\subsubsection{Body styles}\label{body-styles}

\begin{itemize}
\tightlist
\item
  \texttt{\ color\ } : Text color (default is \texttt{\ black\ } )
\item
  \texttt{\ align\ } : Text align (default is \texttt{\ left\ } )
\end{itemize}

\subsubsection{Footer styles}\label{footer-styles}

\begin{itemize}
\tightlist
\item
  \texttt{\ color\ } : Text color (default is \texttt{\ luma(85)\ } )
\item
  \texttt{\ weight\ } : Text weight (default is \texttt{\ regular\ } )
\item
  \texttt{\ align\ } : Text align (default is \texttt{\ left\ } )
\item
  \texttt{\ sep-thickness\ } : Thickness of the separator between body
  and footer (default is \texttt{\ 1pt\ } )
\end{itemize}

\subsubsection{Separator properties}\label{separator-properties}

\begin{itemize}
\tightlist
\item
  \texttt{\ thickness\ } : Separator’s thickness (default is
  \texttt{\ 1pt\ } )
\item
  \texttt{\ dash\ } : Separator’s style (as a \texttt{\ line\ } dash
  style, default is \texttt{\ "solid"\ } )
\item
  \texttt{\ gutter\ } : Separator’s space above and below (defalut is
  \texttt{\ 0.65em\ } )
\end{itemize}

\subsubsection{Shadow properties}\label{shadow-properties}

\begin{itemize}
\tightlist
\item
  \texttt{\ color\ } : Shadow color (default is \texttt{\ black\ } )
\item
  \texttt{\ offset\ } : How much to offset the shadow in x and y
  direction either as a length or a dictionary with keys \texttt{\ x\ }
  and \texttt{\ y\ } (default is \texttt{\ 4pt\ } )
\end{itemize}

\subsection{Gallery}\label{gallery}

\subsubsection{Colors for title, body and footer example (Stokes’
theorem)}\label{colors-for-title-body-and-footer-example-stokesuxe2-theorem}

\subsubsection{\texorpdfstring{\protect\pandocbounded{\includegraphics[keepaspectratio]{https://github.com/typst/packages/raw/main/packages/preview/showybox/2.0.3/assets/stokes-example.png}}}{Encapsulation}}\label{encapsulation}

\subsubsection{Boxed-title example (Clairaut’s
theorem)}\label{boxed-title-example-clairautuxe2s-theorem}

\subsubsection{\texorpdfstring{\protect\pandocbounded{\includegraphics[keepaspectratio]{https://github.com/typst/packages/raw/main/packages/preview/showybox/2.0.3/assets/clairaut-example.png}}}{Encapsulation}}\label{encapsulation-1}

\subsubsection{Encapsulation example}\label{encapsulation-example}

\subsubsection{\texorpdfstring{\protect\pandocbounded{\includegraphics[keepaspectratio]{https://github.com/typst/packages/raw/main/packages/preview/showybox/2.0.3/assets/encapsulation-example.png}}}{Encapsulation}}\label{encapsulation-2}

\subsubsection{Breakable showybox example (Newton’s second
law)}\label{breakable-showybox-example-newtonuxe2s-second-law}

\subsubsection{\texorpdfstring{\protect\pandocbounded{\includegraphics[keepaspectratio]{https://github.com/typst/packages/raw/main/packages/preview/showybox/2.0.3/assets/newton-example.png}}}{Enabling breakable}}\label{enabling-breakable}

\subsubsection{Custom radius and title’s separator thickness example
(Carnot’s cycle
efficency)}\label{custom-radius-and-titleuxe2s-separator-thickness-example-carnotuxe2s-cycle-efficency}

\subsubsection{\texorpdfstring{\protect\pandocbounded{\includegraphics[keepaspectratio]{https://github.com/typst/packages/raw/main/packages/preview/showybox/2.0.3/assets/carnot-example.png}}}{Custom radius}}\label{custom-radius}

\subsubsection{Custom border dash and inset example (Gauss’s
law)}\label{custom-border-dash-and-inset-example-gaussuxe2s-law}

\subsubsection{\texorpdfstring{\protect\pandocbounded{\includegraphics[keepaspectratio]{https://github.com/typst/packages/raw/main/packages/preview/showybox/2.0.3/assets/gauss-example.png}}}{Custom radius}}\label{custom-radius-1}

\subsubsection{Custom footer’s separator thickness example
(Divergence’s
theorem)}\label{custom-footeruxe2s-separator-thickness-example-divergenceuxe2s-theorem}

\subsubsection{\texorpdfstring{\protect\pandocbounded{\includegraphics[keepaspectratio]{https://github.com/typst/packages/raw/main/packages/preview/showybox/2.0.3/assets/divergence-example.png}}}{Custom radius}}\label{custom-radius-2}

\subsubsection{Colorful shadow example (Coulomb’s
law)}\label{colorful-shadow-example-coulombuxe2s-law}

\subsubsection{\texorpdfstring{\protect\pandocbounded{\includegraphics[keepaspectratio]{https://github.com/typst/packages/raw/main/packages/preview/showybox/2.0.3/assets/coulomb-example.png}}}{Custom radius}}\label{custom-radius-3}

\subsection{Changelog}\label{changelog}

\subsubsection{Version 2.0.3}\label{version-2.0.3}

\begin{itemize}
\tightlist
\item
  Revert fix breakable box empty before new page. Layout didn’t
  converge
\end{itemize}

\subsubsection{Version 2.0.2}\label{version-2.0.2}

\begin{itemize}
\tightlist
\item
  Remove deprecated functions in Typst 0.12.0
\item
  Fix breakable box empty before new page
\end{itemize}

\subsubsection{Version 2.0.1}\label{version-2.0.1}

\begin{itemize}
\tightlist
\item
  Fix bad behaviours of boxed-titles \texttt{\ anchor\ } inside a
  \texttt{\ figure\ }
\item
  Fix wrong \texttt{\ breakable\ } behaviour of showyboxes inside a
  \texttt{\ figure\ }
\item
  Fix Manual and README’s Stokes theorem example
\end{itemize}

\subsubsection{Version 2.0.0}\label{version-2.0.0}

\emph{Special thanks to Andrew Voynov (
\url{https://github.com/Andrew15-5} ) for the feedback while creating
the new behaviours for boxed-titles}

\begin{itemize}
\tightlist
\item
  Update \texttt{\ type()\ } conditionals to Typst 0.8.0 standards
\item
  Add \texttt{\ boxed-style\ } property, with \texttt{\ anchor\ } ,
  \texttt{\ offset\ } and \texttt{\ radius\ } properties.
\item
  Refactor \texttt{\ showy-inset()\ } for being general-purpose. Now
  it’s called \texttt{\ showy-value-in-direction()\ } and has a
  default value for handling properties defaults
\item
  Now sharp corners can be set by giving a dictionary to frame
  \texttt{\ radius\ } (e.g.
  \texttt{\ radius:\ (top:\ 5pt,\ bottom:\ 0pt)\ } ). Before this only
  was possible for untitled showyboxes.
\item
  Refactor shadow functions to be in a separated file.
\item
  Fix bug of bad behaviour while writing too long titles.
\item
  Fix bug while rendering separators with custom thickness. Now the
  thickness is gotten properly.
\item
  Fix bad shadow drawing in showyboxes with a boxed-title that has a
  “extreme� \texttt{\ offset\ } value.
\item
  Fix bad sizing while creating showyboxes with a \texttt{\ width\ } of
  less than \texttt{\ 100\%\ } , and a shadow.
\end{itemize}

\subsubsection{Version 1.1.0}\label{version-1.1.0}

\begin{itemize}
\tightlist
\item
  Added \texttt{\ boxed\ } option in title styles
\item
  Added \texttt{\ boxed-align\ } in title styles
\item
  Added \texttt{\ sep-thickness\ } for title and footer
\item
  Refactored repository’s files layout
\end{itemize}

\subsubsection{Version 1.0.0}\label{version-1.0.0}

\begin{itemize}
\tightlist
\item
  Fixed shadow displacement

  \begin{itemize}
  \tightlist
  \item
    \textbf{Details:} Instead of displacing the showybox’s body from
    the shadow, now the shadow is displaced from the body.
  \end{itemize}
\end{itemize}

\emph{Changes below were performed by Jonas Neugebauer (
\url{https://github.com/jneug} )}

\begin{itemize}
\tightlist
\item
  Added \texttt{\ title-inset\ } , \texttt{\ body-inset\ } ,
  \texttt{\ footer-inset\ } and \texttt{\ inset\ } options

  \begin{itemize}
  \tightlist
  \item
    \textbf{Details:} \texttt{\ title-inset\ } , \texttt{\ body-inset\ }
    and \texttt{\ footer-inset\ } will set the inset of the title, body
    and footer area respectively. \texttt{\ inset\ } is a fallback for
    those areas.
  \end{itemize}
\item
  Added a \texttt{\ sep.gutter\ } option to set the spacing around
  separator lines
\item
  Added option \texttt{\ width\ } to set the width of a showybox
\item
  Added option \texttt{\ align\ } to move a showybox with
  \texttt{\ width\ } \textless{} 100\% along the x-axis

  \begin{itemize}
  \tightlist
  \item
    \textbf{Details:} A showybox is now wrapped in another block to
    allow alignment. This also makes it possible to pass the spacing
    options \texttt{\ spacing\ } , \texttt{\ above\ } and
    \texttt{\ below\ } to \texttt{\ \#showybox()\ } .
  \end{itemize}
\item
  Added \texttt{\ footer\ } and \texttt{\ footer-style\ } options

  \begin{itemize}
  \tightlist
  \item
    \textbf{Details:} The optional footer is added at the bottom of the
    box.
  \end{itemize}
\end{itemize}

\subsubsection{Version 0.2.1}\label{version-0.2.1}

\emph{All changes listed here were performed by Jonas Neugebauer (
\url{https://github.com/jneug} )}

\begin{itemize}
\tightlist
\item
  Added the \texttt{\ shadow\ } option
\item
  Enabled auto-break ( \texttt{\ breakable\ } ) functionality for titled
  showyboxes
\item
  Removed a thin line that appears in showyboxes with no borders or
  dashed borders
\end{itemize}

\subsubsection{Version 0.2.0}\label{version-0.2.0}

\begin{itemize}
\tightlist
\item
  Improved code documentation
\item
  Enabled an auto-break functionality for non-titled showyboxes
\item
  Created a separator functionality to separate content inside a
  showybox with a horizontal line
\end{itemize}

\subsubsection{Version 0.1.1}\label{version-0.1.1}

\begin{itemize}
\tightlist
\item
  Changed package name from colorbox to showybox
\item
  Fixed a spacing bug in encapsulated showyboxes

  \begin{itemize}
  \tightlist
  \item
    \textbf{Details:} When a showybox was encapsulated inside another,
    the spacing after that showybox was \texttt{\ 0pt\ } , probably due
    to some “fixes� improved to manage default spacing between
    \texttt{\ rect\ } elements. The issue was solved by avoiding
    \texttt{\ \#set\ } statements and adding a \texttt{\ \#v(-1.1em)\ }
    to correct extra spacing between the title \texttt{\ rect\ } and the
    body \texttt{\ rect\ } .
  \end{itemize}
\end{itemize}

\subsubsection{Version 0.1.0}\label{version-0.1.0}

\begin{itemize}
\tightlist
\item
  Initial release
\end{itemize}

\subsubsection{How to add}\label{how-to-add}

Copy this into your project and use the import as \texttt{\ showybox\ }

\begin{verbatim}
#import "@preview/showybox:2.0.3"
\end{verbatim}

\includesvg[width=0.16667in,height=0.16667in]{/assets/icons/16-copy.svg}

Check the docs for
\href{https://typst.app/docs/reference/scripting/\#packages}{more
information on how to import packages} .

\subsubsection{About}\label{about}

\begin{description}
\tightlist
\item[Author s :]
Pablo González Calderón \& Showybox Contributors
\item[License:]
MIT
\item[Current version:]
2.0.3
\item[Last updated:]
October 24, 2024
\item[First released:]
July 3, 2023
\item[Minimum Typst version:]
0.12.0
\item[Archive size:]
9.41 kB
\href{https://packages.typst.org/preview/showybox-2.0.3.tar.gz}{\pandocbounded{\includesvg[keepaspectratio]{/assets/icons/16-download.svg}}}
\item[Repository:]
\href{https://github.com/Pablo-Gonzalez-Calderon/showybox-package}{GitHub}
\item[Categor y :]
\begin{itemize}
\tightlist
\item[]
\item
  \pandocbounded{\includesvg[keepaspectratio]{/assets/icons/16-package.svg}}
  \href{https://typst.app/universe/search/?category=components}{Components}
\end{itemize}
\end{description}

\subsubsection{Where to report issues?}\label{where-to-report-issues}

This package is a project of Pablo González Calderón and Showybox
Contributors . Report issues on
\href{https://github.com/Pablo-Gonzalez-Calderon/showybox-package}{their
repository} . You can also try to ask for help with this package on the
\href{https://forum.typst.app}{Forum} .

Please report this package to the Typst team using the
\href{https://typst.app/contact}{contact form} if you believe it is a
safety hazard or infringes upon your rights.

\phantomsection\label{versions}
\subsubsection{Version history}\label{version-history}

\begin{longtable}[]{@{}ll@{}}
\toprule\noalign{}
Version & Release Date \\
\midrule\noalign{}
\endhead
\bottomrule\noalign{}
\endlastfoot
2.0.3 & October 24, 2024 \\
\href{https://typst.app/universe/package/showybox/2.0.2/}{2.0.2} &
October 21, 2024 \\
\href{https://typst.app/universe/package/showybox/2.0.1/}{2.0.1} &
October 4, 2023 \\
\href{https://typst.app/universe/package/showybox/2.0.0/}{2.0.0} &
October 1, 2023 \\
\href{https://typst.app/universe/package/showybox/1.1.0/}{1.1.0} &
August 3, 2023 \\
\href{https://typst.app/universe/package/showybox/1.0.0/}{1.0.0} & July
31, 2023 \\
\href{https://typst.app/universe/package/showybox/0.2.1/}{0.2.1} & July
31, 2023 \\
\href{https://typst.app/universe/package/showybox/0.2.0/}{0.2.0} & July
10, 2023 \\
\href{https://typst.app/universe/package/showybox/0.1.1/}{0.1.1} & July
3, 2023 \\
\end{longtable}

Typst GmbH did not create this package and cannot guarantee correct
functionality of this package or compatibility with any version of the
Typst compiler or app.


\section{Package List LaTeX/universal-cau-thesis.tex}
\title{typst.app/universe/package/universal-cau-thesis}

\phantomsection\label{banner}
\phantomsection\label{template-thumbnail}
\pandocbounded{\includegraphics[keepaspectratio]{https://packages.typst.org/preview/thumbnails/universal-cau-thesis-0.1.0-small.webp}}

\section{universal-cau-thesis}\label{universal-cau-thesis}

{ 0.1.0 }

中国农业大学毕业论æ--‡çš„Typst模æ?¿

\href{/app?template=universal-cau-thesis&version=0.1.0}{Create project
in app}

\phantomsection\label{readme}
\subsubsection{为什么使ç''¨Typst}\label{uxe4uxbauxe4uxe4uxb9ux2c6uxe4uxbduxe7typst}

å›~为语法足够简å?•ï¼ˆç®€å?•æ˜``学)ã€?ç¼--è¯`器ä½``积å°?(éš?å?--éš?ç''¨ï¼‰ã€?ä¸''速度足够快(实æ---¶é¢„览)ï¼?

\pandocbounded{\includegraphics[keepaspectratio]{https://github.com/typst/packages/raw/main/packages/preview/universal-cau-thesis/0.1.0/images/PixPin_2024-03-13_17-19-53.png}}

\subsubsection{å\ldots³äºŽæœ¬è®ºæ--‡æ¨¡æ?¿}\label{uxe5uxb3uxe4uxbaux17euxe6ux153uxe8uxbauxe6uxe6uxe6}

本模��考
\href{https://jwc.cau.edu.cn/art/2020/2/25/art_14181_663910.html}{中国农业大学毕业论æ--‡æ¨¡æ?¿è¦?求}
ç¼--写,符å?ˆå­¦æ~¡è¦?求。对于å\ldots¶ä¸­ä¸€äº›å?¯ä»¥ç?µæ´»ä¿®æ''¹çš„æ~¼å¼?,本模æ?¿ä¹Ÿæ??供了é\ldots?置项,å\ldots·ä½``详è§?模æ?¿ä½¿ç''¨æ--¹æ³•ã€‚模æ?¿çš„效果è§?
\texttt{\ sample.pdf\ } æ--‡ä»¶ï¼š
\href{https://github.com/JWangL5/CAU-ThesisTemplate-Typst/blob/master/sample.pdf}{点击直达}

âš~ï¸?
但是,本模æ?¿ä¸ºä¸ªäººç¼--写使ç''¨ï¼Œæ²¡æœ‰å®Œå\ldots¨é€‚é\ldots?自动åŒ--逻è¾`å'Œæ¨¡å?---åŒ--代ç~?,å?¯èƒ½ä»?æ---§å­˜åœ¨éƒ¨åˆ†æƒ\ldots 况æŽ'版ä¸?å?ˆç?†ã€‚如果您在使ç''¨è¿‡ç¨‹ä¸­é?‡åˆ°é---®é¢˜ï¼Œå?¯ä»¥æ??交issue说明,也欢迎pull
request贡献。

如果该模æ?¿å¯¹æ‚¨æœ‰å¸®åŠ©å¹¶æ„¿æ„?æ''¯æŒ?æˆ`的工作,å?¯ä»¥åœ¨è¿™é‡Œ
\href{https://www.buymeacoffee.com/jwangl5}{buy me a coffee}
,å??分感谢😊ï¼?

\subsubsection{使ç''¨æ--¹æ³•}\label{uxe4uxbduxe7uxe6uxb9uxe6uxb3}

\begin{enumerate}
\item
  Typstå?¯ä»¥ä½¿ç''¨ \href{https://typst.app/}{线上WebApp}
  æˆ--本地下载ç¼--è¯`器å?Žè¿›è¡Œç¼--写,本地ç¼--写需è¦?
  \href{https://github.com/typst/typst/releases}{下载安è£\ldots ç¼--è¯`器}
  到本地,并将 \texttt{\ exe\ }
  æ~¼å¼?çš„ç¼--è¯`器添åŠ~到环境å?˜é‡?,以æ--¹ä¾¿è°ƒç''¨
\item
  下载该ä»``åº``到本地目录æˆ--WebApp的工作目录中,å?¯ä»¥ä½¿ç''¨gitå`½ä»¤æˆ--该页é?¢ä¸Šæ--¹çš„Code按é'®ç›´æŽ¥ä¸‹è½½

\begin{Shaded}
\begin{Highlighting}[]
\NormalTok{git clone https://github.com/JWangL5/CAU{-}ThesisTemplate{-}Typst.git}
\end{Highlighting}
\end{Shaded}
\item
  安è£\ldots 本模æ?¿æ‰€ä½¿ç''¨çš„相å\ldots³å­---ä½``(
  \texttt{\ /fonts\ } æ--‡ä»¶å¤¹å†\ldots å­---ä½``)

  PS:å?---é™?于中æ--‡å­---ä½``的衬线é---®é¢˜ï¼Œç›®å‰?版本的Typstæš‚ä¸?æ''¯æŒ?多数中æ--‡å­---ä½``çš„åŠ~ç²---,这里使ç''¨äº†å\ldots¶ä»--åŠ~ç²---å­---ä½``作为替代,å?¯ä»¥ç›´æŽ¥å?Œå‡»å­---ä½``æ--‡ä»¶æ‰``å¼€å?Žå®‰è£
\item
  本地ç¼--写æ---¶ï¼Œå»ºè®®ä½¿ç''¨
  \href{https://code.visualstudio.com/}{vscode}
  å?ŠTypsté\ldots?å¥---æ?'件(Typst-LSPã€?Typst Preview)
\item
  通过ç¼--写 \texttt{\ main.typ\ }
  æ--‡ä»¶å®Œæˆ?论æ--‡çš„æ'°å†™ï¼Œåœ¨è¯¥æ--‡ä»¶ä¸­ï¼Œä½¿ç''¨
  \texttt{\ import\ } å`½ä»¤å¼•å\ldots¥æ¨¡æ?¿ï¼Œå¹¶ä¿®æ''¹é\ldots?置项

  \begin{itemize}
  \tightlist
  \item
    kind:填写 \texttt{\ "本科"\ } , \texttt{\ "硕士"\ } ,
    \texttt{\ "�士"\ }
    ,å\ldots¶ä¼šå¯¹åº''ä¿®æ''¹å°?é?¢å'Œé¡µçœ‰å¤„çš„ä¿¡æ?¯
  \item
    title:论æ--‡æ~‡é¢˜ï¼Œå¡«å†™åœ¨æ‹¬å?· \texttt{\ {[}text{]}\ }
    å†\ldots ,使ç''¨ \texttt{\ \textbackslash{}\ } æ?¢è¡Œ
  \item
    abstract:论æ--‡æ`˜è¦?,需è¦?手动写å\ldots³é''®è¯?
  \item
    authors:论æ--‡ä½œè€\ldots å§``å??,å?¯ä»¥
    \texttt{\ {[}text{]}\ } ,也�以 \texttt{\ "text"\ }
  \item
    teacher:指导教师的å§``å??
  \item
    degree:ç''³è¯·å­¦ä½?é---¨ç±»çº§åˆ«ï¼Œæ¯''如
    \texttt{\ {[}�学硕士{]}\ }
  \item
    college, major,
    field:å°?é?¢ä¸Šçš„å†\ldots 容,学院ã€?ä¸``业å'Œç~''究æ--¹å?{}`
  \item
    signature:ä½~çš„ç''µå­?ç­¾å??æ--‡ä»¶è·¯å¾„,是论æ--‡ç‹¬åˆ›æ€§å£°æ˜Žå¤„çš„ç­¾å­---
  \item
    classification \& security:论æ--‡åœ¨å›¾ä¹¦é¦†æ''¶å½•æ---¶çš„
    \href{https://www.clcindex.com/}{中图法分类} å'Œä¿?密级别
  \item
    student\_ID:学�
  \item
    year, month, day:论æ--‡å°?é?¢å'Œè¯šä¿¡å£°æ˜Žé¡µä¸Šçš„æ---¥æœŸ
  \item
    draft:填写 \texttt{\ true\ }
    æ---¶æ·»åŠ~è?‰ç¨¿æ°´å?°ï¼Œç''¨ä»¥åŒºåˆ†æ˜¯å?¦ä¸ºæœ€ç»ˆç‰ˆæœ¬ï¼Œå¡«å†™
    \texttt{\ false\ } æ---¶åŽ»é™¤æ°´å?°å¹¶æ·»åŠ~论æ--‡ç«
  \item
    blindReview:填写为 \texttt{\ true\ }
    æ---¶éš?è---?å°?é?¢ä¸Šçš„相å\ldots³ä¿¡æ?¯ï¼Œä»¥å?Šè‡´è°¢å'Œä½œè€\ldots 介ç»?
  \end{itemize}
\item
  使ç''¨ \texttt{\ typst\ }
  å`½ä»¤ç''Ÿæˆ?pdfæ~¼å¼?æ--‡ä»¶ï¼Œæˆ--直接使ç''¨vscode的实æ---¶é¢„览æ?'件(默认快æ?·é''®
  \texttt{\ ctrl+k\ v\ } )

\begin{Shaded}
\begin{Highlighting}[]
\NormalTok{typst compile .}\AttributeTok{/sample}\NormalTok{.typ}
\end{Highlighting}
\end{Shaded}
\end{enumerate}

\subsubsection{Typstç¼--写简æ˜``指å?---}\label{typstuxe7uxbcuxe5uxe7uxe6uxe6ux153uxe5}

\begin{quote}
如果在使ç''¨Typstæ---¶é?‡åˆ°ä»»ä½•é---®é¢˜ï¼Œéƒ½å?¯ä»¥å?‚考
\href{https://typst.app/docs/}{官æ--¹å¸®åŠ©æ--‡æ¡£}
,下é?¢æ˜¯ç®€è¦?的使ç''¨æ--¹æ³•å?Šä¸Žæœ¬æ¨¡æ?¿ç›¸å\ldots³çš„é\ldots?å¥---设置,å?¯ä»¥å?‚考的示例æ--‡æ¡£
\texttt{\ sample.typ\ }
\end{quote}

\begin{itemize}
\item
  å\ldots³äºŽæ~‡é¢˜ï¼šTypst使ç''¨ \texttt{\ =\ }
  作为æ~‡é¢˜çš„指示符。本模æ?¿ä¸­ï¼Œä¸€çº§æ~‡é¢˜éœ€è¦?手动ç¼--å?·ï¼ŒäºŒã€?三级æ~‡é¢˜åˆ™ä¸?需è¦?

\begin{Shaded}
\begin{Highlighting}[]
\NormalTok{= 第一章 一级标题}
\NormalTok{== 二级标题}
\NormalTok{=== 三级标题}
\end{Highlighting}
\end{Shaded}
\item
  段è?½çš„ç¼--写:直接è¾``å\ldots¥æ--‡æœ¬å?³å?¯ç¼--写å†\ldots 容,使ç''¨ä¸¤ä¸ªå›žè½¦å?¦èµ·ä¸€æ®µ

\begin{Shaded}
\begin{Highlighting}[]
\NormalTok{这是内容的第一段,}
\NormalTok{这仍旧是第一段的内容}

\NormalTok{多一个换行符号后另起为第二段}
\end{Highlighting}
\end{Shaded}
\item
  æ--‡å­---å†\ldots 容的基础æ~¼å¼?:使ç''¨ \texttt{\ *text*\ }
  åŒ\ldots 括ä½?çš„æ--‡å­---å?¯ä»¥åŠ~ç²---,使ç''¨
  \texttt{\ \_text\_\ }
  åŒ\ldots 括ä½?çš„æ--‡å­---å?¯ä»¥æ--œä½``æ--‡å­---,使ç''¨
  \texttt{\ \#u{[}text{]}\ }
  åŒ\ldots 括ä½?çš„æ--‡å­---å?¯ä»¥å®žçŽ°ä¸‹åˆ'线,使ç''¨
  \texttt{\ \#sub{[}text{]}\ }
  åŒ\ldots 括ä½?çš„æ--‡å­---ä¿®æ''¹ä¸ºä¸‹æ~‡ï¼Œä½¿ç''¨
  \texttt{\ \#super{[}text{]}\ }
  åŒ\ldots 括ä½?çš„æ--‡å­---ä¿®æ''¹ä¸ºä¸Šæ~‡

\begin{Shaded}
\begin{Highlighting}[]
\NormalTok{这里是*加粗文字*内容,}
\NormalTok{这里是\_斜体文字\_内容,}
\NormalTok{这里是\#u[下划线文字]内容,}
\NormalTok{这里是\#sub[下标文字]内容,}
\NormalTok{这里是\#super[上标文字]内容}
\end{Highlighting}
\end{Shaded}
\item
  使ç''¨ \texttt{\ \#highlight(fill:\ red){[}text{]}\ }
  高亮æ--‡å­---æ~‡è®°ï¼Œä½¿ç''¨ \texttt{\ fill\ }
  å?‚æ•°ä¿®æ''¹é«˜äº®é¢œè‰²

\begin{Shaded}
\begin{Highlighting}[]
\NormalTok{这一段文字\#highlight[还需要修改]}
\end{Highlighting}
\end{Shaded}
\item
  使ç''¨ \texttt{\ \#link("your\ link\ here"){[}text{]}\ }
  æ~‡è®°è¶\ldots é``¾æŽ¥

\begin{Shaded}
\begin{Highlighting}[]
\NormalTok{这里是\#link("https://typst.app/home")[Typst官方网站]}
\end{Highlighting}
\end{Shaded}
\item
  使ç''¨ \texttt{\ -\ } æˆ-- \texttt{\ +\ }
  使ç''¨æ---~åº?åˆ---表æˆ--有åº?åˆ---表,使ç''¨ \texttt{\ Tab\ }
  缩进为二级åˆ---表

\begin{Shaded}
\begin{Highlighting}[]
\NormalTok{{-} 无序列表1}
\NormalTok{    + 有序列表1}
\NormalTok{    + 有序列表2}
\NormalTok{{-} 无序列表2}
\end{Highlighting}
\end{Shaded}
\item
  使ç''¨ \texttt{\ \#h(2em)\ } æˆ-- \texttt{\ \#v(1em)\ }
  æ?¥æ·»åŠ~æ°´å¹³æˆ--垂直的空白缩进,括å?·ä¸­çš„å?‚数为需è¦?空出的è·?离,å\ldots¶å?•ä½?å?¯ä»¥ä½¿ç''¨
  \texttt{\ \%\ } (页é?¢ç™¾åˆ†æ¯''), \texttt{\ pt\ }
  (点,å?°åˆ·è¡Œä¸šçš„ç»?对长度å?•ä½?,相å½``于1/72英寸),
  \texttt{\ em\ } (å­---符长度,相对于å½``å‰?å­---符大å°?)等

\begin{Shaded}
\begin{Highlighting}[]
\NormalTok{\#h(2em)默认格式会首行缩进两字符,额外添加会再次缩进}
\end{Highlighting}
\end{Shaded}
\item
  模æ?¿æ''¯æŒ?自动汇总图表目录,å?¯ä»¥ä½¿ç''¨
  \texttt{\ \#figure(image())\ } å`½ä»¤æ?¥æ·»åŠ~图片æˆ--使ç''¨
  \texttt{\ \#booktab()\ }
  æ·»åŠ~表æ~¼ï¼Œåœ¨å¡«å†™æ---¢å®šçš„é\ldots?置项å?Žå?¯æˆ?功渲æŸ``并自动汇总目录页,å?¯ä»¥ä½¿ç''¨
  \texttt{\ @\ } 引ç''¨å›¾è¡¨

\begin{Shaded}
\begin{Highlighting}[]
\NormalTok{\#figure(}
\NormalTok{    image(\textquotesingle{}./image/path.jpg\textquotesingle{}, width: 90\%),}
\NormalTok{    kind: image, }
\NormalTok{    supplement: [图],}
\NormalTok{    caption: [图片的标题],}
\NormalTok{)\textless{}img1\textgreater{}}

\NormalTok{\#booktab(}
\NormalTok{    width:60\%,}
\NormalTok{    columns: (20\%, 1fr, 2fr, 3fr),}
\NormalTok{    caption: [这里填写表格名称],}
\NormalTok{    kind: table, }
\NormalTok{    [1], [2], [3], [4],}
\NormalTok{    [a], [b], [c], [d],}
\NormalTok{    [e], [f], [g], [h],}
\NormalTok{    [i], [j], [k], [l]}
\NormalTok{)\textless{}tab1\textgreater{}}
\end{Highlighting}
\end{Shaded}
\item
  使ç''¨ \texttt{\ \$\ } ç¼--写数学å\ldots¬å¼?, \texttt{\ \$\ }
  符紧跟å†\ldots 容æ---¶ä¸ºè¡Œå†\ldots å\ldots¬å¼?,添åŠ~空æ~¼å?Žä¸ºè¡Œé---´å\ldots¬å¼?,å\ldots¬å¼?çš„å\ldots·ä½``规则å'Œ
  \href{https://typst.app/docs/reference/symbols/sym/}{符�} �以查
  \href{https://typst.app/docs/reference/math/}{帮助æ--‡æ¡£}

\begin{Shaded}
\begin{Highlighting}[]
\NormalTok{泰勒展开式(行内):}
\NormalTok{$f(x)= sum\_(n=0)\^{}(infinity) (f\^{}(n)(x\_0))/(n!) (x{-}x\_0)\^{}n$}

\NormalTok{欧拉公式(行间):}
\NormalTok{$ e\^{}(i theta) = cos theta + i sin theta \textbackslash{} e\^{}i pi + 1 = 0 $}
\end{Highlighting}
\end{Shaded}
\item
  使ç''¨
  \texttt{\ \textasciigrave{}\textasciigrave{}\textasciigrave{}code\textasciigrave{}\textasciigrave{}\textasciigrave{}\ }
  æ~‡è¯†ç¬¦è¾``å\ldots¥ä»£ç~?,Typstå?¯ä»¥æ¸²æŸ``ã€?显示代ç~?框,如果指定了语言类型,å?¯ä»¥æ~¹æ?®å\ldots¶è¯­æ³•æ~¼å¼?进行风æ~¼æ¸²æŸ``;使ç''¨å?•ä¸ªç¬¦å?·ä½¿ç''¨è¡Œå†\ldots 代ç~?

\begin{Shaded}
\begin{Highlighting}[]
\NormalTok{  \textasciigrave{}\textasciigrave{}\textasciigrave{}python}
\NormalTok{    print("hello world");}
\NormalTok{  \textasciigrave{}\textasciigrave{}\textasciigrave{}}
\NormalTok{同样支持行内代码\textasciigrave{}hello world\textasciigrave{}}
\end{Highlighting}
\end{Shaded}
\item
  ä¿®æ''¹ \texttt{\ ref\textbackslash{}acronyms.json\ }
  æ--‡ä»¶æ·»åŠ~缩略è¯?表,并使ç''¨ \texttt{\ \#acro("keyword1")\ }
  å`½ä»¤åœ¨æ--‡ä¸­å¼•å\ldots¥ç¼©ç•¥è¯?å\ldots¨ç§°ï¼Œåœ¨å¼•å\ldots¥å?Žä¼šè‡ªåŠ¨æ~¹æ?®jsonæ--‡ä»¶ä¸­ä¿¡æ?¯ï¼ŒæŽ'åº?å?Žæ·»åŠ~到缩略è¯?表中

\begin{Shaded}
\begin{Highlighting}[]
\FunctionTok{\{}
    \DataTypeTok{"keyword1"}\FunctionTok{:}\OtherTok{[}\StringTok{"英文缩写1"}\OtherTok{,} \StringTok{"英文全称1"}\OtherTok{,} \StringTok{"中文翻译1"}\OtherTok{]}\FunctionTok{,}
    \DataTypeTok{"keyword2"}\FunctionTok{:}\OtherTok{[}\StringTok{"英文缩写2"}\OtherTok{,} \StringTok{"英文全称2"}\OtherTok{,} \StringTok{"中文翻译2"}\OtherTok{]}
\FunctionTok{\}}
\end{Highlighting}
\end{Shaded}

\begin{Shaded}
\begin{Highlighting}[]
\NormalTok{在正文中可以使用\textasciigrave{}\#acro\textasciigrave{}命令引入缩略词(\#acro("ac"))。}
\end{Highlighting}
\end{Shaded}
\item
  使ç''¨ \texttt{\ \#{[}bibliography{]}()\ }
  æ·»åŠ~å?‚考æ--‡çŒ®ï¼Œæ‹¬å?·ä¸­éœ€è¦?填写 \texttt{\ .bib\ }
  æ~¼å¼?çš„å?‚考æ--‡çŒ®åˆ---表,在æ--‡ä¸­ä½¿ç''¨
  \texttt{\ @citationKey\ } 引ç''¨ï¼Œå\ldots·ä½``çš„ä¿¡æ?¯è§?
  \href{https://typst.app/docs/reference/model/bibliography/}{帮助æ--‡æ¡£}

  PS:å?¯ä»¥ä½¿ç''¨ \texttt{\ zotero+Better\ BibTex\ }
  自动导出/æ›´æ--° \texttt{\ .bib\ } æ~¼å¼?çš„å?‚考æ--‡çŒ®åˆ---表

  PS:在添åŠ~bib的代ç~?å?Žé?¢ï¼Œéš?è---?了一个heading,请ä¸?è¦?åˆ~除这一行,å?¦åˆ™å?‚考æ--‡çŒ®çš„页眉会出é''™

  PS:æ~¹æ?®å­¦é™¢è¦?求默认使ç''¨EmboJçš„æ~¼å¼?,如果需è¦?å\ldots¶ä»--æ~¼å¼?,å?ªè¦?下载到æ~¼å¼?说明
  \texttt{\ .csl\ } æ--‡ä»¶ä¿®æ''¹å?‚æ•°å?³å?¯
\item
  å½``æ--‡æœ¬å†\ldots 容ä»\ldots 有1页æ---¶ï¼Œæœ‰æ---¶é¡µçœ‰æ~‡é¢˜ä¼šå‡ºé''™ï¼Œå?¯ä»¥æ·»åŠ~一个空白æ~‡é¢˜è¿›è¡Œä¿®æ­£

\begin{Shaded}
\begin{Highlighting}[]
\NormalTok{\#heading(level: 6, numbering: none, outlined: false)[]}
\end{Highlighting}
\end{Shaded}
\end{itemize}

\subsubsection{致谢}\label{uxe8uxe8}

本模æ?¿åœ¨ç¼--写过程中å?‚考并学ä¹~了Typst模æ?¿çš„部分代ç~?,在这里统一致谢。

\href{/app?template=universal-cau-thesis&version=0.1.0}{Create project
in app}

\subsubsection{How to use}\label{how-to-use}

Click the button above to create a new project using this template in
the Typst app.

You can also use the Typst CLI to start a new project on your computer
using this command:

\begin{verbatim}
typst init @preview/universal-cau-thesis:0.1.0
\end{verbatim}

\includesvg[width=0.16667in,height=0.16667in]{/assets/icons/16-copy.svg}

\subsubsection{About}\label{about}

\begin{description}
\tightlist
\item[Author :]
JWangL5
\item[License:]
MIT
\item[Current version:]
0.1.0
\item[Last updated:]
May 27, 2024
\item[First released:]
May 27, 2024
\item[Archive size:]
107 kB
\href{https://packages.typst.org/preview/universal-cau-thesis-0.1.0.tar.gz}{\pandocbounded{\includesvg[keepaspectratio]{/assets/icons/16-download.svg}}}
\item[Repository:]
\href{https://github.com/JWangL5/CAU-ThesisTemplate-Typst}{GitHub}
\item[Categor y :]
\begin{itemize}
\tightlist
\item[]
\item
  \pandocbounded{\includesvg[keepaspectratio]{/assets/icons/16-mortarboard.svg}}
  \href{https://typst.app/universe/search/?category=thesis}{Thesis}
\end{itemize}
\end{description}

\subsubsection{Where to report issues?}\label{where-to-report-issues}

This template is a project of JWangL5 . Report issues on
\href{https://github.com/JWangL5/CAU-ThesisTemplate-Typst}{their
repository} . You can also try to ask for help with this template on the
\href{https://forum.typst.app}{Forum} .

Please report this template to the Typst team using the
\href{https://typst.app/contact}{contact form} if you believe it is a
safety hazard or infringes upon your rights.

\phantomsection\label{versions}
\subsubsection{Version history}\label{version-history}

\begin{longtable}[]{@{}ll@{}}
\toprule\noalign{}
Version & Release Date \\
\midrule\noalign{}
\endhead
\bottomrule\noalign{}
\endlastfoot
0.1.0 & May 27, 2024 \\
\end{longtable}

Typst GmbH did not create this template and cannot guarantee correct
functionality of this template or compatibility with any version of the
Typst compiler or app.


\section{Package List LaTeX/octique.tex}
\title{typst.app/universe/package/octique}

\phantomsection\label{banner}
\section{octique}\label{octique}

{ 0.1.0 }

GitHub Octicons for Typst.

\phantomsection\label{readme}
GitHub \href{https://primer.style/foundations/icons/}{Octicons} for
Typst.

\subsection{Installation}\label{installation}

\begin{Shaded}
\begin{Highlighting}[]
\NormalTok{\#import "@preview/octique:0.1.0": *}
\end{Highlighting}
\end{Shaded}

\subsection{Usage}\label{usage}

\begin{Shaded}
\begin{Highlighting}[]
\NormalTok{// Returns an image for the given name.}
\NormalTok{octique(name, color: rgb("\#000000"), width: 1em, height: 1em)}

\NormalTok{// Returns a boxed image for the given name.}
\NormalTok{octique{-}inline(name, color: rgb("\#000000"), width: 1em, height: 1em, baseline: 25\%)}

\NormalTok{// Returns an SVG text for the given name.}
\NormalTok{octique{-}svg(name)}
\end{Highlighting}
\end{Shaded}

\subsection{List of Available Icons}\label{list-of-available-icons}

See also
\href{https://github.com/typst/packages/raw/main/packages/preview/octique/0.1.0/sample/sample.pdf}{\texttt{\ sample/sample.pdf\ }}
.

\begin{longtable}[]{@{}lc@{}}
\toprule\noalign{}
Code & Icon \\
\midrule\noalign{}
\endhead
\bottomrule\noalign{}
\endlastfoot
\texttt{\ \#octique("accessibility-inset")\ } &
\pandocbounded{\includesvg[keepaspectratio]{https://github.com/0x6b/typst-octique/wiki/assets/accessibility-inset.svg}} \\
\texttt{\ \#octique("accessibility")\ } &
\pandocbounded{\includesvg[keepaspectratio]{https://github.com/0x6b/typst-octique/wiki/assets/accessibility.svg}} \\
\texttt{\ \#octique("alert-fill")\ } &
\pandocbounded{\includesvg[keepaspectratio]{https://github.com/0x6b/typst-octique/wiki/assets/alert-fill.svg}} \\
\texttt{\ \#octique("alert")\ } &
\pandocbounded{\includesvg[keepaspectratio]{https://github.com/0x6b/typst-octique/wiki/assets/alert.svg}} \\
\texttt{\ \#octique("apps")\ } &
\pandocbounded{\includesvg[keepaspectratio]{https://github.com/0x6b/typst-octique/wiki/assets/apps.svg}} \\
\texttt{\ \#octique("archive")\ } &
\pandocbounded{\includesvg[keepaspectratio]{https://github.com/0x6b/typst-octique/wiki/assets/archive.svg}} \\
\texttt{\ \#octique("arrow-both")\ } &
\pandocbounded{\includesvg[keepaspectratio]{https://github.com/0x6b/typst-octique/wiki/assets/arrow-both.svg}} \\
\texttt{\ \#octique("arrow-down-left")\ } &
\pandocbounded{\includesvg[keepaspectratio]{https://github.com/0x6b/typst-octique/wiki/assets/arrow-down-left.svg}} \\
\texttt{\ \#octique("arrow-down-right")\ } &
\pandocbounded{\includesvg[keepaspectratio]{https://github.com/0x6b/typst-octique/wiki/assets/arrow-down-right.svg}} \\
\texttt{\ \#octique("arrow-down")\ } &
\pandocbounded{\includesvg[keepaspectratio]{https://github.com/0x6b/typst-octique/wiki/assets/arrow-down.svg}} \\
\texttt{\ \#octique("arrow-left")\ } &
\pandocbounded{\includesvg[keepaspectratio]{https://github.com/0x6b/typst-octique/wiki/assets/arrow-left.svg}} \\
\texttt{\ \#octique("arrow-right")\ } &
\pandocbounded{\includesvg[keepaspectratio]{https://github.com/0x6b/typst-octique/wiki/assets/arrow-right.svg}} \\
\texttt{\ \#octique("arrow-switch")\ } &
\pandocbounded{\includesvg[keepaspectratio]{https://github.com/0x6b/typst-octique/wiki/assets/arrow-switch.svg}} \\
\texttt{\ \#octique("arrow-up-left")\ } &
\pandocbounded{\includesvg[keepaspectratio]{https://github.com/0x6b/typst-octique/wiki/assets/arrow-up-left.svg}} \\
\texttt{\ \#octique("arrow-up-right")\ } &
\pandocbounded{\includesvg[keepaspectratio]{https://github.com/0x6b/typst-octique/wiki/assets/arrow-up-right.svg}} \\
\texttt{\ \#octique("arrow-up")\ } &
\pandocbounded{\includesvg[keepaspectratio]{https://github.com/0x6b/typst-octique/wiki/assets/arrow-up.svg}} \\
\texttt{\ \#octique("beaker")\ } &
\pandocbounded{\includesvg[keepaspectratio]{https://github.com/0x6b/typst-octique/wiki/assets/beaker.svg}} \\
\texttt{\ \#octique("bell-fill")\ } &
\pandocbounded{\includesvg[keepaspectratio]{https://github.com/0x6b/typst-octique/wiki/assets/bell-fill.svg}} \\
\texttt{\ \#octique("bell-slash")\ } &
\pandocbounded{\includesvg[keepaspectratio]{https://github.com/0x6b/typst-octique/wiki/assets/bell-slash.svg}} \\
\texttt{\ \#octique("bell")\ } &
\pandocbounded{\includesvg[keepaspectratio]{https://github.com/0x6b/typst-octique/wiki/assets/bell.svg}} \\
\texttt{\ \#octique("blocked")\ } &
\pandocbounded{\includesvg[keepaspectratio]{https://github.com/0x6b/typst-octique/wiki/assets/blocked.svg}} \\
\texttt{\ \#octique("bold")\ } &
\pandocbounded{\includesvg[keepaspectratio]{https://github.com/0x6b/typst-octique/wiki/assets/bold.svg}} \\
\texttt{\ \#octique("book")\ } &
\pandocbounded{\includesvg[keepaspectratio]{https://github.com/0x6b/typst-octique/wiki/assets/book.svg}} \\
\texttt{\ \#octique("bookmark-slash")\ } &
\pandocbounded{\includesvg[keepaspectratio]{https://github.com/0x6b/typst-octique/wiki/assets/bookmark-slash.svg}} \\
\texttt{\ \#octique("bookmark")\ } &
\pandocbounded{\includesvg[keepaspectratio]{https://github.com/0x6b/typst-octique/wiki/assets/bookmark.svg}} \\
\texttt{\ \#octique("briefcase")\ } &
\pandocbounded{\includesvg[keepaspectratio]{https://github.com/0x6b/typst-octique/wiki/assets/briefcase.svg}} \\
\texttt{\ \#octique("broadcast")\ } &
\pandocbounded{\includesvg[keepaspectratio]{https://github.com/0x6b/typst-octique/wiki/assets/broadcast.svg}} \\
\texttt{\ \#octique("browser")\ } &
\pandocbounded{\includesvg[keepaspectratio]{https://github.com/0x6b/typst-octique/wiki/assets/browser.svg}} \\
\texttt{\ \#octique("bug")\ } &
\pandocbounded{\includesvg[keepaspectratio]{https://github.com/0x6b/typst-octique/wiki/assets/bug.svg}} \\
\texttt{\ \#octique("cache")\ } &
\pandocbounded{\includesvg[keepaspectratio]{https://github.com/0x6b/typst-octique/wiki/assets/cache.svg}} \\
\texttt{\ \#octique("calendar")\ } &
\pandocbounded{\includesvg[keepaspectratio]{https://github.com/0x6b/typst-octique/wiki/assets/calendar.svg}} \\
\texttt{\ \#octique("check-circle-fill")\ } &
\pandocbounded{\includesvg[keepaspectratio]{https://github.com/0x6b/typst-octique/wiki/assets/check-circle-fill.svg}} \\
\texttt{\ \#octique("check-circle")\ } &
\pandocbounded{\includesvg[keepaspectratio]{https://github.com/0x6b/typst-octique/wiki/assets/check-circle.svg}} \\
\texttt{\ \#octique("check")\ } &
\pandocbounded{\includesvg[keepaspectratio]{https://github.com/0x6b/typst-octique/wiki/assets/check.svg}} \\
\texttt{\ \#octique("checkbox")\ } &
\pandocbounded{\includesvg[keepaspectratio]{https://github.com/0x6b/typst-octique/wiki/assets/checkbox.svg}} \\
\texttt{\ \#octique("checklist")\ } &
\pandocbounded{\includesvg[keepaspectratio]{https://github.com/0x6b/typst-octique/wiki/assets/checklist.svg}} \\
\texttt{\ \#octique("chevron-down")\ } &
\pandocbounded{\includesvg[keepaspectratio]{https://github.com/0x6b/typst-octique/wiki/assets/chevron-down.svg}} \\
\texttt{\ \#octique("chevron-left")\ } &
\pandocbounded{\includesvg[keepaspectratio]{https://github.com/0x6b/typst-octique/wiki/assets/chevron-left.svg}} \\
\texttt{\ \#octique("chevron-right")\ } &
\pandocbounded{\includesvg[keepaspectratio]{https://github.com/0x6b/typst-octique/wiki/assets/chevron-right.svg}} \\
\texttt{\ \#octique("chevron-up")\ } &
\pandocbounded{\includesvg[keepaspectratio]{https://github.com/0x6b/typst-octique/wiki/assets/chevron-up.svg}} \\
\texttt{\ \#octique("circle-slash")\ } &
\pandocbounded{\includesvg[keepaspectratio]{https://github.com/0x6b/typst-octique/wiki/assets/circle-slash.svg}} \\
\texttt{\ \#octique("circle")\ } &
\pandocbounded{\includesvg[keepaspectratio]{https://github.com/0x6b/typst-octique/wiki/assets/circle.svg}} \\
\texttt{\ \#octique("clock-fill")\ } &
\pandocbounded{\includesvg[keepaspectratio]{https://github.com/0x6b/typst-octique/wiki/assets/clock-fill.svg}} \\
\texttt{\ \#octique("clock")\ } &
\pandocbounded{\includesvg[keepaspectratio]{https://github.com/0x6b/typst-octique/wiki/assets/clock.svg}} \\
\texttt{\ \#octique("cloud-offline")\ } &
\pandocbounded{\includesvg[keepaspectratio]{https://github.com/0x6b/typst-octique/wiki/assets/cloud-offline.svg}} \\
\texttt{\ \#octique("cloud")\ } &
\pandocbounded{\includesvg[keepaspectratio]{https://github.com/0x6b/typst-octique/wiki/assets/cloud.svg}} \\
\texttt{\ \#octique("code-of-conduct")\ } &
\pandocbounded{\includesvg[keepaspectratio]{https://github.com/0x6b/typst-octique/wiki/assets/code-of-conduct.svg}} \\
\texttt{\ \#octique("code-review")\ } &
\pandocbounded{\includesvg[keepaspectratio]{https://github.com/0x6b/typst-octique/wiki/assets/code-review.svg}} \\
\texttt{\ \#octique("code")\ } &
\pandocbounded{\includesvg[keepaspectratio]{https://github.com/0x6b/typst-octique/wiki/assets/code.svg}} \\
\texttt{\ \#octique("code-square")\ } &
\pandocbounded{\includesvg[keepaspectratio]{https://github.com/0x6b/typst-octique/wiki/assets/code-square.svg}} \\
\texttt{\ \#octique("codescan-checkmark")\ } &
\pandocbounded{\includesvg[keepaspectratio]{https://github.com/0x6b/typst-octique/wiki/assets/codescan-checkmark.svg}} \\
\texttt{\ \#octique("codescan")\ } &
\pandocbounded{\includesvg[keepaspectratio]{https://github.com/0x6b/typst-octique/wiki/assets/codescan.svg}} \\
\texttt{\ \#octique("codespaces")\ } &
\pandocbounded{\includesvg[keepaspectratio]{https://github.com/0x6b/typst-octique/wiki/assets/codespaces.svg}} \\
\texttt{\ \#octique("columns")\ } &
\pandocbounded{\includesvg[keepaspectratio]{https://github.com/0x6b/typst-octique/wiki/assets/columns.svg}} \\
\texttt{\ \#octique("command-palette")\ } &
\pandocbounded{\includesvg[keepaspectratio]{https://github.com/0x6b/typst-octique/wiki/assets/command-palette.svg}} \\
\texttt{\ \#octique("comment-discussion")\ } &
\pandocbounded{\includesvg[keepaspectratio]{https://github.com/0x6b/typst-octique/wiki/assets/comment-discussion.svg}} \\
\texttt{\ \#octique("comment")\ } &
\pandocbounded{\includesvg[keepaspectratio]{https://github.com/0x6b/typst-octique/wiki/assets/comment.svg}} \\
\texttt{\ \#octique("container")\ } &
\pandocbounded{\includesvg[keepaspectratio]{https://github.com/0x6b/typst-octique/wiki/assets/container.svg}} \\
\texttt{\ \#octique("copilot-error")\ } &
\pandocbounded{\includesvg[keepaspectratio]{https://github.com/0x6b/typst-octique/wiki/assets/copilot-error.svg}} \\
\texttt{\ \#octique("copilot")\ } &
\pandocbounded{\includesvg[keepaspectratio]{https://github.com/0x6b/typst-octique/wiki/assets/copilot.svg}} \\
\texttt{\ \#octique("copilot-warning")\ } &
\pandocbounded{\includesvg[keepaspectratio]{https://github.com/0x6b/typst-octique/wiki/assets/copilot-warning.svg}} \\
\texttt{\ \#octique("copy")\ } &
\pandocbounded{\includesvg[keepaspectratio]{https://github.com/0x6b/typst-octique/wiki/assets/copy.svg}} \\
\texttt{\ \#octique("cpu")\ } &
\pandocbounded{\includesvg[keepaspectratio]{https://github.com/0x6b/typst-octique/wiki/assets/cpu.svg}} \\
\texttt{\ \#octique("credit-card")\ } &
\pandocbounded{\includesvg[keepaspectratio]{https://github.com/0x6b/typst-octique/wiki/assets/credit-card.svg}} \\
\texttt{\ \#octique("cross-reference")\ } &
\pandocbounded{\includesvg[keepaspectratio]{https://github.com/0x6b/typst-octique/wiki/assets/cross-reference.svg}} \\
\texttt{\ \#octique("dash")\ } &
\pandocbounded{\includesvg[keepaspectratio]{https://github.com/0x6b/typst-octique/wiki/assets/dash.svg}} \\
\texttt{\ \#octique("database")\ } &
\pandocbounded{\includesvg[keepaspectratio]{https://github.com/0x6b/typst-octique/wiki/assets/database.svg}} \\
\texttt{\ \#octique("dependabot")\ } &
\pandocbounded{\includesvg[keepaspectratio]{https://github.com/0x6b/typst-octique/wiki/assets/dependabot.svg}} \\
\texttt{\ \#octique("desktop-download")\ } &
\pandocbounded{\includesvg[keepaspectratio]{https://github.com/0x6b/typst-octique/wiki/assets/desktop-download.svg}} \\
\texttt{\ \#octique("device-camera")\ } &
\pandocbounded{\includesvg[keepaspectratio]{https://github.com/0x6b/typst-octique/wiki/assets/device-camera.svg}} \\
\texttt{\ \#octique("device-camera-video")\ } &
\pandocbounded{\includesvg[keepaspectratio]{https://github.com/0x6b/typst-octique/wiki/assets/device-camera-video.svg}} \\
\texttt{\ \#octique("device-desktop")\ } &
\pandocbounded{\includesvg[keepaspectratio]{https://github.com/0x6b/typst-octique/wiki/assets/device-desktop.svg}} \\
\texttt{\ \#octique("device-mobile")\ } &
\pandocbounded{\includesvg[keepaspectratio]{https://github.com/0x6b/typst-octique/wiki/assets/device-mobile.svg}} \\
\texttt{\ \#octique("devices")\ } &
\pandocbounded{\includesvg[keepaspectratio]{https://github.com/0x6b/typst-octique/wiki/assets/devices.svg}} \\
\texttt{\ \#octique("diamond")\ } &
\pandocbounded{\includesvg[keepaspectratio]{https://github.com/0x6b/typst-octique/wiki/assets/diamond.svg}} \\
\texttt{\ \#octique("diff-added")\ } &
\pandocbounded{\includesvg[keepaspectratio]{https://github.com/0x6b/typst-octique/wiki/assets/diff-added.svg}} \\
\texttt{\ \#octique("diff-ignored")\ } &
\pandocbounded{\includesvg[keepaspectratio]{https://github.com/0x6b/typst-octique/wiki/assets/diff-ignored.svg}} \\
\texttt{\ \#octique("diff-modified")\ } &
\pandocbounded{\includesvg[keepaspectratio]{https://github.com/0x6b/typst-octique/wiki/assets/diff-modified.svg}} \\
\texttt{\ \#octique("diff-removed")\ } &
\pandocbounded{\includesvg[keepaspectratio]{https://github.com/0x6b/typst-octique/wiki/assets/diff-removed.svg}} \\
\texttt{\ \#octique("diff-renamed")\ } &
\pandocbounded{\includesvg[keepaspectratio]{https://github.com/0x6b/typst-octique/wiki/assets/diff-renamed.svg}} \\
\texttt{\ \#octique("diff")\ } &
\pandocbounded{\includesvg[keepaspectratio]{https://github.com/0x6b/typst-octique/wiki/assets/diff.svg}} \\
\texttt{\ \#octique("discussion-closed")\ } &
\pandocbounded{\includesvg[keepaspectratio]{https://github.com/0x6b/typst-octique/wiki/assets/discussion-closed.svg}} \\
\texttt{\ \#octique("discussion-duplicate")\ } &
\pandocbounded{\includesvg[keepaspectratio]{https://github.com/0x6b/typst-octique/wiki/assets/discussion-duplicate.svg}} \\
\texttt{\ \#octique("discussion-outdated")\ } &
\pandocbounded{\includesvg[keepaspectratio]{https://github.com/0x6b/typst-octique/wiki/assets/discussion-outdated.svg}} \\
\texttt{\ \#octique("dot-fill")\ } &
\pandocbounded{\includesvg[keepaspectratio]{https://github.com/0x6b/typst-octique/wiki/assets/dot-fill.svg}} \\
\texttt{\ \#octique("dot")\ } &
\pandocbounded{\includesvg[keepaspectratio]{https://github.com/0x6b/typst-octique/wiki/assets/dot.svg}} \\
\texttt{\ \#octique("download")\ } &
\pandocbounded{\includesvg[keepaspectratio]{https://github.com/0x6b/typst-octique/wiki/assets/download.svg}} \\
\texttt{\ \#octique("duplicate")\ } &
\pandocbounded{\includesvg[keepaspectratio]{https://github.com/0x6b/typst-octique/wiki/assets/duplicate.svg}} \\
\texttt{\ \#octique("ellipsis")\ } &
\pandocbounded{\includesvg[keepaspectratio]{https://github.com/0x6b/typst-octique/wiki/assets/ellipsis.svg}} \\
\texttt{\ \#octique("eye-closed")\ } &
\pandocbounded{\includesvg[keepaspectratio]{https://github.com/0x6b/typst-octique/wiki/assets/eye-closed.svg}} \\
\texttt{\ \#octique("eye")\ } &
\pandocbounded{\includesvg[keepaspectratio]{https://github.com/0x6b/typst-octique/wiki/assets/eye.svg}} \\
\texttt{\ \#octique("feed-discussion")\ } &
\pandocbounded{\includesvg[keepaspectratio]{https://github.com/0x6b/typst-octique/wiki/assets/feed-discussion.svg}} \\
\texttt{\ \#octique("feed-forked")\ } &
\pandocbounded{\includesvg[keepaspectratio]{https://github.com/0x6b/typst-octique/wiki/assets/feed-forked.svg}} \\
\texttt{\ \#octique("feed-heart")\ } &
\pandocbounded{\includesvg[keepaspectratio]{https://github.com/0x6b/typst-octique/wiki/assets/feed-heart.svg}} \\
\texttt{\ \#octique("feed-issue-closed")\ } &
\pandocbounded{\includesvg[keepaspectratio]{https://github.com/0x6b/typst-octique/wiki/assets/feed-issue-closed.svg}} \\
\texttt{\ \#octique("feed-issue-draft")\ } &
\pandocbounded{\includesvg[keepaspectratio]{https://github.com/0x6b/typst-octique/wiki/assets/feed-issue-draft.svg}} \\
\texttt{\ \#octique("feed-issue-open")\ } &
\pandocbounded{\includesvg[keepaspectratio]{https://github.com/0x6b/typst-octique/wiki/assets/feed-issue-open.svg}} \\
\texttt{\ \#octique("feed-issue-reopen")\ } &
\pandocbounded{\includesvg[keepaspectratio]{https://github.com/0x6b/typst-octique/wiki/assets/feed-issue-reopen.svg}} \\
\texttt{\ \#octique("feed-merged")\ } &
\pandocbounded{\includesvg[keepaspectratio]{https://github.com/0x6b/typst-octique/wiki/assets/feed-merged.svg}} \\
\texttt{\ \#octique("feed-person")\ } &
\pandocbounded{\includesvg[keepaspectratio]{https://github.com/0x6b/typst-octique/wiki/assets/feed-person.svg}} \\
\texttt{\ \#octique("feed-plus")\ } &
\pandocbounded{\includesvg[keepaspectratio]{https://github.com/0x6b/typst-octique/wiki/assets/feed-plus.svg}} \\
\texttt{\ \#octique("feed-public")\ } &
\pandocbounded{\includesvg[keepaspectratio]{https://github.com/0x6b/typst-octique/wiki/assets/feed-public.svg}} \\
\texttt{\ \#octique("feed-pull-request-closed")\ } &
\pandocbounded{\includesvg[keepaspectratio]{https://github.com/0x6b/typst-octique/wiki/assets/feed-pull-request-closed.svg}} \\
\texttt{\ \#octique("feed-pull-request-draft")\ } &
\pandocbounded{\includesvg[keepaspectratio]{https://github.com/0x6b/typst-octique/wiki/assets/feed-pull-request-draft.svg}} \\
\texttt{\ \#octique("feed-pull-request-open")\ } &
\pandocbounded{\includesvg[keepaspectratio]{https://github.com/0x6b/typst-octique/wiki/assets/feed-pull-request-open.svg}} \\
\texttt{\ \#octique("feed-repo")\ } &
\pandocbounded{\includesvg[keepaspectratio]{https://github.com/0x6b/typst-octique/wiki/assets/feed-repo.svg}} \\
\texttt{\ \#octique("feed-rocket")\ } &
\pandocbounded{\includesvg[keepaspectratio]{https://github.com/0x6b/typst-octique/wiki/assets/feed-rocket.svg}} \\
\texttt{\ \#octique("feed-star")\ } &
\pandocbounded{\includesvg[keepaspectratio]{https://github.com/0x6b/typst-octique/wiki/assets/feed-star.svg}} \\
\texttt{\ \#octique("feed-tag")\ } &
\pandocbounded{\includesvg[keepaspectratio]{https://github.com/0x6b/typst-octique/wiki/assets/feed-tag.svg}} \\
\texttt{\ \#octique("feed-trophy")\ } &
\pandocbounded{\includesvg[keepaspectratio]{https://github.com/0x6b/typst-octique/wiki/assets/feed-trophy.svg}} \\
\texttt{\ \#octique("file-added")\ } &
\pandocbounded{\includesvg[keepaspectratio]{https://github.com/0x6b/typst-octique/wiki/assets/file-added.svg}} \\
\texttt{\ \#octique("file-badge")\ } &
\pandocbounded{\includesvg[keepaspectratio]{https://github.com/0x6b/typst-octique/wiki/assets/file-badge.svg}} \\
\texttt{\ \#octique("file-binary")\ } &
\pandocbounded{\includesvg[keepaspectratio]{https://github.com/0x6b/typst-octique/wiki/assets/file-binary.svg}} \\
\texttt{\ \#octique("file-code")\ } &
\pandocbounded{\includesvg[keepaspectratio]{https://github.com/0x6b/typst-octique/wiki/assets/file-code.svg}} \\
\texttt{\ \#octique("file-diff")\ } &
\pandocbounded{\includesvg[keepaspectratio]{https://github.com/0x6b/typst-octique/wiki/assets/file-diff.svg}} \\
\texttt{\ \#octique("file-directory-fill")\ } &
\pandocbounded{\includesvg[keepaspectratio]{https://github.com/0x6b/typst-octique/wiki/assets/file-directory-fill.svg}} \\
\texttt{\ \#octique("file-directory-open-fill")\ } &
\pandocbounded{\includesvg[keepaspectratio]{https://github.com/0x6b/typst-octique/wiki/assets/file-directory-open-fill.svg}} \\
\texttt{\ \#octique("file-directory")\ } &
\pandocbounded{\includesvg[keepaspectratio]{https://github.com/0x6b/typst-octique/wiki/assets/file-directory.svg}} \\
\texttt{\ \#octique("file-directory-symlink")\ } &
\pandocbounded{\includesvg[keepaspectratio]{https://github.com/0x6b/typst-octique/wiki/assets/file-directory-symlink.svg}} \\
\texttt{\ \#octique("file-moved")\ } &
\pandocbounded{\includesvg[keepaspectratio]{https://github.com/0x6b/typst-octique/wiki/assets/file-moved.svg}} \\
\texttt{\ \#octique("file-removed")\ } &
\pandocbounded{\includesvg[keepaspectratio]{https://github.com/0x6b/typst-octique/wiki/assets/file-removed.svg}} \\
\texttt{\ \#octique("file")\ } &
\pandocbounded{\includesvg[keepaspectratio]{https://github.com/0x6b/typst-octique/wiki/assets/file.svg}} \\
\texttt{\ \#octique("file-submodule")\ } &
\pandocbounded{\includesvg[keepaspectratio]{https://github.com/0x6b/typst-octique/wiki/assets/file-submodule.svg}} \\
\texttt{\ \#octique("file-symlink-file")\ } &
\pandocbounded{\includesvg[keepaspectratio]{https://github.com/0x6b/typst-octique/wiki/assets/file-symlink-file.svg}} \\
\texttt{\ \#octique("file-zip")\ } &
\pandocbounded{\includesvg[keepaspectratio]{https://github.com/0x6b/typst-octique/wiki/assets/file-zip.svg}} \\
\texttt{\ \#octique("filter")\ } &
\pandocbounded{\includesvg[keepaspectratio]{https://github.com/0x6b/typst-octique/wiki/assets/filter.svg}} \\
\texttt{\ \#octique("fiscal-host")\ } &
\pandocbounded{\includesvg[keepaspectratio]{https://github.com/0x6b/typst-octique/wiki/assets/fiscal-host.svg}} \\
\texttt{\ \#octique("flame")\ } &
\pandocbounded{\includesvg[keepaspectratio]{https://github.com/0x6b/typst-octique/wiki/assets/flame.svg}} \\
\texttt{\ \#octique("fold-down")\ } &
\pandocbounded{\includesvg[keepaspectratio]{https://github.com/0x6b/typst-octique/wiki/assets/fold-down.svg}} \\
\texttt{\ \#octique("fold")\ } &
\pandocbounded{\includesvg[keepaspectratio]{https://github.com/0x6b/typst-octique/wiki/assets/fold.svg}} \\
\texttt{\ \#octique("fold-up")\ } &
\pandocbounded{\includesvg[keepaspectratio]{https://github.com/0x6b/typst-octique/wiki/assets/fold-up.svg}} \\
\texttt{\ \#octique("gear")\ } &
\pandocbounded{\includesvg[keepaspectratio]{https://github.com/0x6b/typst-octique/wiki/assets/gear.svg}} \\
\texttt{\ \#octique("gift")\ } &
\pandocbounded{\includesvg[keepaspectratio]{https://github.com/0x6b/typst-octique/wiki/assets/gift.svg}} \\
\texttt{\ \#octique("git-branch")\ } &
\pandocbounded{\includesvg[keepaspectratio]{https://github.com/0x6b/typst-octique/wiki/assets/git-branch.svg}} \\
\texttt{\ \#octique("git-commit")\ } &
\pandocbounded{\includesvg[keepaspectratio]{https://github.com/0x6b/typst-octique/wiki/assets/git-commit.svg}} \\
\texttt{\ \#octique("git-compare")\ } &
\pandocbounded{\includesvg[keepaspectratio]{https://github.com/0x6b/typst-octique/wiki/assets/git-compare.svg}} \\
\texttt{\ \#octique("git-merge-queue")\ } &
\pandocbounded{\includesvg[keepaspectratio]{https://github.com/0x6b/typst-octique/wiki/assets/git-merge-queue.svg}} \\
\texttt{\ \#octique("git-merge")\ } &
\pandocbounded{\includesvg[keepaspectratio]{https://github.com/0x6b/typst-octique/wiki/assets/git-merge.svg}} \\
\texttt{\ \#octique("git-pull-request-closed")\ } &
\pandocbounded{\includesvg[keepaspectratio]{https://github.com/0x6b/typst-octique/wiki/assets/git-pull-request-closed.svg}} \\
\texttt{\ \#octique("git-pull-request-draft")\ } &
\pandocbounded{\includesvg[keepaspectratio]{https://github.com/0x6b/typst-octique/wiki/assets/git-pull-request-draft.svg}} \\
\texttt{\ \#octique("git-pull-request")\ } &
\pandocbounded{\includesvg[keepaspectratio]{https://github.com/0x6b/typst-octique/wiki/assets/git-pull-request.svg}} \\
\texttt{\ \#octique("globe")\ } &
\pandocbounded{\includesvg[keepaspectratio]{https://github.com/0x6b/typst-octique/wiki/assets/globe.svg}} \\
\texttt{\ \#octique("goal")\ } &
\pandocbounded{\includesvg[keepaspectratio]{https://github.com/0x6b/typst-octique/wiki/assets/goal.svg}} \\
\texttt{\ \#octique("grabber")\ } &
\pandocbounded{\includesvg[keepaspectratio]{https://github.com/0x6b/typst-octique/wiki/assets/grabber.svg}} \\
\texttt{\ \#octique("graph")\ } &
\pandocbounded{\includesvg[keepaspectratio]{https://github.com/0x6b/typst-octique/wiki/assets/graph.svg}} \\
\texttt{\ \#octique("hash")\ } &
\pandocbounded{\includesvg[keepaspectratio]{https://github.com/0x6b/typst-octique/wiki/assets/hash.svg}} \\
\texttt{\ \#octique("heading")\ } &
\pandocbounded{\includesvg[keepaspectratio]{https://github.com/0x6b/typst-octique/wiki/assets/heading.svg}} \\
\texttt{\ \#octique("heart-fill")\ } &
\pandocbounded{\includesvg[keepaspectratio]{https://github.com/0x6b/typst-octique/wiki/assets/heart-fill.svg}} \\
\texttt{\ \#octique("heart")\ } &
\pandocbounded{\includesvg[keepaspectratio]{https://github.com/0x6b/typst-octique/wiki/assets/heart.svg}} \\
\texttt{\ \#octique("history")\ } &
\pandocbounded{\includesvg[keepaspectratio]{https://github.com/0x6b/typst-octique/wiki/assets/history.svg}} \\
\texttt{\ \#octique("home")\ } &
\pandocbounded{\includesvg[keepaspectratio]{https://github.com/0x6b/typst-octique/wiki/assets/home.svg}} \\
\texttt{\ \#octique("horizontal-rule")\ } &
\pandocbounded{\includesvg[keepaspectratio]{https://github.com/0x6b/typst-octique/wiki/assets/horizontal-rule.svg}} \\
\texttt{\ \#octique("hourglass")\ } &
\pandocbounded{\includesvg[keepaspectratio]{https://github.com/0x6b/typst-octique/wiki/assets/hourglass.svg}} \\
\texttt{\ \#octique("hubot")\ } &
\pandocbounded{\includesvg[keepaspectratio]{https://github.com/0x6b/typst-octique/wiki/assets/hubot.svg}} \\
\texttt{\ \#octique("id-badge")\ } &
\pandocbounded{\includesvg[keepaspectratio]{https://github.com/0x6b/typst-octique/wiki/assets/id-badge.svg}} \\
\texttt{\ \#octique("image")\ } &
\pandocbounded{\includesvg[keepaspectratio]{https://github.com/0x6b/typst-octique/wiki/assets/image.svg}} \\
\texttt{\ \#octique("inbox")\ } &
\pandocbounded{\includesvg[keepaspectratio]{https://github.com/0x6b/typst-octique/wiki/assets/inbox.svg}} \\
\texttt{\ \#octique("infinity")\ } &
\pandocbounded{\includesvg[keepaspectratio]{https://github.com/0x6b/typst-octique/wiki/assets/infinity.svg}} \\
\texttt{\ \#octique("info")\ } &
\pandocbounded{\includesvg[keepaspectratio]{https://github.com/0x6b/typst-octique/wiki/assets/info.svg}} \\
\texttt{\ \#octique("issue-closed")\ } &
\pandocbounded{\includesvg[keepaspectratio]{https://github.com/0x6b/typst-octique/wiki/assets/issue-closed.svg}} \\
\texttt{\ \#octique("issue-draft")\ } &
\pandocbounded{\includesvg[keepaspectratio]{https://github.com/0x6b/typst-octique/wiki/assets/issue-draft.svg}} \\
\texttt{\ \#octique("issue-opened")\ } &
\pandocbounded{\includesvg[keepaspectratio]{https://github.com/0x6b/typst-octique/wiki/assets/issue-opened.svg}} \\
\texttt{\ \#octique("issue-reopened")\ } &
\pandocbounded{\includesvg[keepaspectratio]{https://github.com/0x6b/typst-octique/wiki/assets/issue-reopened.svg}} \\
\texttt{\ \#octique("issue-tracked-by")\ } &
\pandocbounded{\includesvg[keepaspectratio]{https://github.com/0x6b/typst-octique/wiki/assets/issue-tracked-by.svg}} \\
\texttt{\ \#octique("issue-tracks")\ } &
\pandocbounded{\includesvg[keepaspectratio]{https://github.com/0x6b/typst-octique/wiki/assets/issue-tracks.svg}} \\
\texttt{\ \#octique("italic")\ } &
\pandocbounded{\includesvg[keepaspectratio]{https://github.com/0x6b/typst-octique/wiki/assets/italic.svg}} \\
\texttt{\ \#octique("iterations")\ } &
\pandocbounded{\includesvg[keepaspectratio]{https://github.com/0x6b/typst-octique/wiki/assets/iterations.svg}} \\
\texttt{\ \#octique("kebab-horizontal")\ } &
\pandocbounded{\includesvg[keepaspectratio]{https://github.com/0x6b/typst-octique/wiki/assets/kebab-horizontal.svg}} \\
\texttt{\ \#octique("key-asterisk")\ } &
\pandocbounded{\includesvg[keepaspectratio]{https://github.com/0x6b/typst-octique/wiki/assets/key-asterisk.svg}} \\
\texttt{\ \#octique("key")\ } &
\pandocbounded{\includesvg[keepaspectratio]{https://github.com/0x6b/typst-octique/wiki/assets/key.svg}} \\
\texttt{\ \#octique("law")\ } &
\pandocbounded{\includesvg[keepaspectratio]{https://github.com/0x6b/typst-octique/wiki/assets/law.svg}} \\
\texttt{\ \#octique("light-bulb")\ } &
\pandocbounded{\includesvg[keepaspectratio]{https://github.com/0x6b/typst-octique/wiki/assets/light-bulb.svg}} \\
\texttt{\ \#octique("link-external")\ } &
\pandocbounded{\includesvg[keepaspectratio]{https://github.com/0x6b/typst-octique/wiki/assets/link-external.svg}} \\
\texttt{\ \#octique("link")\ } &
\pandocbounded{\includesvg[keepaspectratio]{https://github.com/0x6b/typst-octique/wiki/assets/link.svg}} \\
\texttt{\ \#octique("list-ordered")\ } &
\pandocbounded{\includesvg[keepaspectratio]{https://github.com/0x6b/typst-octique/wiki/assets/list-ordered.svg}} \\
\texttt{\ \#octique("list-unordered")\ } &
\pandocbounded{\includesvg[keepaspectratio]{https://github.com/0x6b/typst-octique/wiki/assets/list-unordered.svg}} \\
\texttt{\ \#octique("location")\ } &
\pandocbounded{\includesvg[keepaspectratio]{https://github.com/0x6b/typst-octique/wiki/assets/location.svg}} \\
\texttt{\ \#octique("lock")\ } &
\pandocbounded{\includesvg[keepaspectratio]{https://github.com/0x6b/typst-octique/wiki/assets/lock.svg}} \\
\texttt{\ \#octique("log")\ } &
\pandocbounded{\includesvg[keepaspectratio]{https://github.com/0x6b/typst-octique/wiki/assets/log.svg}} \\
\texttt{\ \#octique("logo-gist")\ } &
\pandocbounded{\includesvg[keepaspectratio]{https://github.com/0x6b/typst-octique/wiki/assets/logo-gist.svg}} \\
\texttt{\ \#octique("logo-github")\ } &
\pandocbounded{\includesvg[keepaspectratio]{https://github.com/0x6b/typst-octique/wiki/assets/logo-github.svg}} \\
\texttt{\ \#octique("mail")\ } &
\pandocbounded{\includesvg[keepaspectratio]{https://github.com/0x6b/typst-octique/wiki/assets/mail.svg}} \\
\texttt{\ \#octique("mark-github")\ } &
\pandocbounded{\includesvg[keepaspectratio]{https://github.com/0x6b/typst-octique/wiki/assets/mark-github.svg}} \\
\texttt{\ \#octique("markdown")\ } &
\pandocbounded{\includesvg[keepaspectratio]{https://github.com/0x6b/typst-octique/wiki/assets/markdown.svg}} \\
\texttt{\ \#octique("megaphone")\ } &
\pandocbounded{\includesvg[keepaspectratio]{https://github.com/0x6b/typst-octique/wiki/assets/megaphone.svg}} \\
\texttt{\ \#octique("mention")\ } &
\pandocbounded{\includesvg[keepaspectratio]{https://github.com/0x6b/typst-octique/wiki/assets/mention.svg}} \\
\texttt{\ \#octique("meter")\ } &
\pandocbounded{\includesvg[keepaspectratio]{https://github.com/0x6b/typst-octique/wiki/assets/meter.svg}} \\
\texttt{\ \#octique("milestone")\ } &
\pandocbounded{\includesvg[keepaspectratio]{https://github.com/0x6b/typst-octique/wiki/assets/milestone.svg}} \\
\texttt{\ \#octique("mirror")\ } &
\pandocbounded{\includesvg[keepaspectratio]{https://github.com/0x6b/typst-octique/wiki/assets/mirror.svg}} \\
\texttt{\ \#octique("moon")\ } &
\pandocbounded{\includesvg[keepaspectratio]{https://github.com/0x6b/typst-octique/wiki/assets/moon.svg}} \\
\texttt{\ \#octique("mortar-board")\ } &
\pandocbounded{\includesvg[keepaspectratio]{https://github.com/0x6b/typst-octique/wiki/assets/mortar-board.svg}} \\
\texttt{\ \#octique("move-to-bottom")\ } &
\pandocbounded{\includesvg[keepaspectratio]{https://github.com/0x6b/typst-octique/wiki/assets/move-to-bottom.svg}} \\
\texttt{\ \#octique("move-to-end")\ } &
\pandocbounded{\includesvg[keepaspectratio]{https://github.com/0x6b/typst-octique/wiki/assets/move-to-end.svg}} \\
\texttt{\ \#octique("move-to-start")\ } &
\pandocbounded{\includesvg[keepaspectratio]{https://github.com/0x6b/typst-octique/wiki/assets/move-to-start.svg}} \\
\texttt{\ \#octique("move-to-top")\ } &
\pandocbounded{\includesvg[keepaspectratio]{https://github.com/0x6b/typst-octique/wiki/assets/move-to-top.svg}} \\
\texttt{\ \#octique("multi-select")\ } &
\pandocbounded{\includesvg[keepaspectratio]{https://github.com/0x6b/typst-octique/wiki/assets/multi-select.svg}} \\
\texttt{\ \#octique("mute")\ } &
\pandocbounded{\includesvg[keepaspectratio]{https://github.com/0x6b/typst-octique/wiki/assets/mute.svg}} \\
\texttt{\ \#octique("no-entry")\ } &
\pandocbounded{\includesvg[keepaspectratio]{https://github.com/0x6b/typst-octique/wiki/assets/no-entry.svg}} \\
\texttt{\ \#octique("north-star")\ } &
\pandocbounded{\includesvg[keepaspectratio]{https://github.com/0x6b/typst-octique/wiki/assets/north-star.svg}} \\
\texttt{\ \#octique("note")\ } &
\pandocbounded{\includesvg[keepaspectratio]{https://github.com/0x6b/typst-octique/wiki/assets/note.svg}} \\
\texttt{\ \#octique("number")\ } &
\pandocbounded{\includesvg[keepaspectratio]{https://github.com/0x6b/typst-octique/wiki/assets/number.svg}} \\
\texttt{\ \#octique("organization")\ } &
\pandocbounded{\includesvg[keepaspectratio]{https://github.com/0x6b/typst-octique/wiki/assets/organization.svg}} \\
\texttt{\ \#octique("package-dependencies")\ } &
\pandocbounded{\includesvg[keepaspectratio]{https://github.com/0x6b/typst-octique/wiki/assets/package-dependencies.svg}} \\
\texttt{\ \#octique("package-dependents")\ } &
\pandocbounded{\includesvg[keepaspectratio]{https://github.com/0x6b/typst-octique/wiki/assets/package-dependents.svg}} \\
\texttt{\ \#octique("package")\ } &
\pandocbounded{\includesvg[keepaspectratio]{https://github.com/0x6b/typst-octique/wiki/assets/package.svg}} \\
\texttt{\ \#octique("paintbrush")\ } &
\pandocbounded{\includesvg[keepaspectratio]{https://github.com/0x6b/typst-octique/wiki/assets/paintbrush.svg}} \\
\texttt{\ \#octique("paper-airplane")\ } &
\pandocbounded{\includesvg[keepaspectratio]{https://github.com/0x6b/typst-octique/wiki/assets/paper-airplane.svg}} \\
\texttt{\ \#octique("paperclip")\ } &
\pandocbounded{\includesvg[keepaspectratio]{https://github.com/0x6b/typst-octique/wiki/assets/paperclip.svg}} \\
\texttt{\ \#octique("passkey-fill")\ } &
\pandocbounded{\includesvg[keepaspectratio]{https://github.com/0x6b/typst-octique/wiki/assets/passkey-fill.svg}} \\
\texttt{\ \#octique("paste")\ } &
\pandocbounded{\includesvg[keepaspectratio]{https://github.com/0x6b/typst-octique/wiki/assets/paste.svg}} \\
\texttt{\ \#octique("pencil")\ } &
\pandocbounded{\includesvg[keepaspectratio]{https://github.com/0x6b/typst-octique/wiki/assets/pencil.svg}} \\
\texttt{\ \#octique("people")\ } &
\pandocbounded{\includesvg[keepaspectratio]{https://github.com/0x6b/typst-octique/wiki/assets/people.svg}} \\
\texttt{\ \#octique("person-add")\ } &
\pandocbounded{\includesvg[keepaspectratio]{https://github.com/0x6b/typst-octique/wiki/assets/person-add.svg}} \\
\texttt{\ \#octique("person-fill")\ } &
\pandocbounded{\includesvg[keepaspectratio]{https://github.com/0x6b/typst-octique/wiki/assets/person-fill.svg}} \\
\texttt{\ \#octique("person")\ } &
\pandocbounded{\includesvg[keepaspectratio]{https://github.com/0x6b/typst-octique/wiki/assets/person.svg}} \\
\texttt{\ \#octique("pin-slash")\ } &
\pandocbounded{\includesvg[keepaspectratio]{https://github.com/0x6b/typst-octique/wiki/assets/pin-slash.svg}} \\
\texttt{\ \#octique("pin")\ } &
\pandocbounded{\includesvg[keepaspectratio]{https://github.com/0x6b/typst-octique/wiki/assets/pin.svg}} \\
\texttt{\ \#octique("pivot-column")\ } &
\pandocbounded{\includesvg[keepaspectratio]{https://github.com/0x6b/typst-octique/wiki/assets/pivot-column.svg}} \\
\texttt{\ \#octique("play")\ } &
\pandocbounded{\includesvg[keepaspectratio]{https://github.com/0x6b/typst-octique/wiki/assets/play.svg}} \\
\texttt{\ \#octique("plug")\ } &
\pandocbounded{\includesvg[keepaspectratio]{https://github.com/0x6b/typst-octique/wiki/assets/plug.svg}} \\
\texttt{\ \#octique("plus-circle")\ } &
\pandocbounded{\includesvg[keepaspectratio]{https://github.com/0x6b/typst-octique/wiki/assets/plus-circle.svg}} \\
\texttt{\ \#octique("plus")\ } &
\pandocbounded{\includesvg[keepaspectratio]{https://github.com/0x6b/typst-octique/wiki/assets/plus.svg}} \\
\texttt{\ \#octique("project-roadmap")\ } &
\pandocbounded{\includesvg[keepaspectratio]{https://github.com/0x6b/typst-octique/wiki/assets/project-roadmap.svg}} \\
\texttt{\ \#octique("project")\ } &
\pandocbounded{\includesvg[keepaspectratio]{https://github.com/0x6b/typst-octique/wiki/assets/project.svg}} \\
\texttt{\ \#octique("project-symlink")\ } &
\pandocbounded{\includesvg[keepaspectratio]{https://github.com/0x6b/typst-octique/wiki/assets/project-symlink.svg}} \\
\texttt{\ \#octique("project-template")\ } &
\pandocbounded{\includesvg[keepaspectratio]{https://github.com/0x6b/typst-octique/wiki/assets/project-template.svg}} \\
\texttt{\ \#octique("pulse")\ } &
\pandocbounded{\includesvg[keepaspectratio]{https://github.com/0x6b/typst-octique/wiki/assets/pulse.svg}} \\
\texttt{\ \#octique("question")\ } &
\pandocbounded{\includesvg[keepaspectratio]{https://github.com/0x6b/typst-octique/wiki/assets/question.svg}} \\
\texttt{\ \#octique("quote")\ } &
\pandocbounded{\includesvg[keepaspectratio]{https://github.com/0x6b/typst-octique/wiki/assets/quote.svg}} \\
\texttt{\ \#octique("read")\ } &
\pandocbounded{\includesvg[keepaspectratio]{https://github.com/0x6b/typst-octique/wiki/assets/read.svg}} \\
\texttt{\ \#octique("redo")\ } &
\pandocbounded{\includesvg[keepaspectratio]{https://github.com/0x6b/typst-octique/wiki/assets/redo.svg}} \\
\texttt{\ \#octique("rel-file-path")\ } &
\pandocbounded{\includesvg[keepaspectratio]{https://github.com/0x6b/typst-octique/wiki/assets/rel-file-path.svg}} \\
\texttt{\ \#octique("reply")\ } &
\pandocbounded{\includesvg[keepaspectratio]{https://github.com/0x6b/typst-octique/wiki/assets/reply.svg}} \\
\texttt{\ \#octique("repo-clone")\ } &
\pandocbounded{\includesvg[keepaspectratio]{https://github.com/0x6b/typst-octique/wiki/assets/repo-clone.svg}} \\
\texttt{\ \#octique("repo-deleted")\ } &
\pandocbounded{\includesvg[keepaspectratio]{https://github.com/0x6b/typst-octique/wiki/assets/repo-deleted.svg}} \\
\texttt{\ \#octique("repo-forked")\ } &
\pandocbounded{\includesvg[keepaspectratio]{https://github.com/0x6b/typst-octique/wiki/assets/repo-forked.svg}} \\
\texttt{\ \#octique("repo-locked")\ } &
\pandocbounded{\includesvg[keepaspectratio]{https://github.com/0x6b/typst-octique/wiki/assets/repo-locked.svg}} \\
\texttt{\ \#octique("repo-pull")\ } &
\pandocbounded{\includesvg[keepaspectratio]{https://github.com/0x6b/typst-octique/wiki/assets/repo-pull.svg}} \\
\texttt{\ \#octique("repo-push")\ } &
\pandocbounded{\includesvg[keepaspectratio]{https://github.com/0x6b/typst-octique/wiki/assets/repo-push.svg}} \\
\texttt{\ \#octique("repo")\ } &
\pandocbounded{\includesvg[keepaspectratio]{https://github.com/0x6b/typst-octique/wiki/assets/repo.svg}} \\
\texttt{\ \#octique("repo-template")\ } &
\pandocbounded{\includesvg[keepaspectratio]{https://github.com/0x6b/typst-octique/wiki/assets/repo-template.svg}} \\
\texttt{\ \#octique("report")\ } &
\pandocbounded{\includesvg[keepaspectratio]{https://github.com/0x6b/typst-octique/wiki/assets/report.svg}} \\
\texttt{\ \#octique("rocket")\ } &
\pandocbounded{\includesvg[keepaspectratio]{https://github.com/0x6b/typst-octique/wiki/assets/rocket.svg}} \\
\texttt{\ \#octique("rows")\ } &
\pandocbounded{\includesvg[keepaspectratio]{https://github.com/0x6b/typst-octique/wiki/assets/rows.svg}} \\
\texttt{\ \#octique("rss")\ } &
\pandocbounded{\includesvg[keepaspectratio]{https://github.com/0x6b/typst-octique/wiki/assets/rss.svg}} \\
\texttt{\ \#octique("ruby")\ } &
\pandocbounded{\includesvg[keepaspectratio]{https://github.com/0x6b/typst-octique/wiki/assets/ruby.svg}} \\
\texttt{\ \#octique("screen-full")\ } &
\pandocbounded{\includesvg[keepaspectratio]{https://github.com/0x6b/typst-octique/wiki/assets/screen-full.svg}} \\
\texttt{\ \#octique("screen-normal")\ } &
\pandocbounded{\includesvg[keepaspectratio]{https://github.com/0x6b/typst-octique/wiki/assets/screen-normal.svg}} \\
\texttt{\ \#octique("search")\ } &
\pandocbounded{\includesvg[keepaspectratio]{https://github.com/0x6b/typst-octique/wiki/assets/search.svg}} \\
\texttt{\ \#octique("server")\ } &
\pandocbounded{\includesvg[keepaspectratio]{https://github.com/0x6b/typst-octique/wiki/assets/server.svg}} \\
\texttt{\ \#octique("share-android")\ } &
\pandocbounded{\includesvg[keepaspectratio]{https://github.com/0x6b/typst-octique/wiki/assets/share-android.svg}} \\
\texttt{\ \#octique("share")\ } &
\pandocbounded{\includesvg[keepaspectratio]{https://github.com/0x6b/typst-octique/wiki/assets/share.svg}} \\
\texttt{\ \#octique("shield-check")\ } &
\pandocbounded{\includesvg[keepaspectratio]{https://github.com/0x6b/typst-octique/wiki/assets/shield-check.svg}} \\
\texttt{\ \#octique("shield-lock")\ } &
\pandocbounded{\includesvg[keepaspectratio]{https://github.com/0x6b/typst-octique/wiki/assets/shield-lock.svg}} \\
\texttt{\ \#octique("shield-slash")\ } &
\pandocbounded{\includesvg[keepaspectratio]{https://github.com/0x6b/typst-octique/wiki/assets/shield-slash.svg}} \\
\texttt{\ \#octique("shield")\ } &
\pandocbounded{\includesvg[keepaspectratio]{https://github.com/0x6b/typst-octique/wiki/assets/shield.svg}} \\
\texttt{\ \#octique("shield-x")\ } &
\pandocbounded{\includesvg[keepaspectratio]{https://github.com/0x6b/typst-octique/wiki/assets/shield-x.svg}} \\
\texttt{\ \#octique("sidebar-collapse")\ } &
\pandocbounded{\includesvg[keepaspectratio]{https://github.com/0x6b/typst-octique/wiki/assets/sidebar-collapse.svg}} \\
\texttt{\ \#octique("sidebar-expand")\ } &
\pandocbounded{\includesvg[keepaspectratio]{https://github.com/0x6b/typst-octique/wiki/assets/sidebar-expand.svg}} \\
\texttt{\ \#octique("sign-in")\ } &
\pandocbounded{\includesvg[keepaspectratio]{https://github.com/0x6b/typst-octique/wiki/assets/sign-in.svg}} \\
\texttt{\ \#octique("sign-out")\ } &
\pandocbounded{\includesvg[keepaspectratio]{https://github.com/0x6b/typst-octique/wiki/assets/sign-out.svg}} \\
\texttt{\ \#octique("single-select")\ } &
\pandocbounded{\includesvg[keepaspectratio]{https://github.com/0x6b/typst-octique/wiki/assets/single-select.svg}} \\
\texttt{\ \#octique("skip-fill")\ } &
\pandocbounded{\includesvg[keepaspectratio]{https://github.com/0x6b/typst-octique/wiki/assets/skip-fill.svg}} \\
\texttt{\ \#octique("skip")\ } &
\pandocbounded{\includesvg[keepaspectratio]{https://github.com/0x6b/typst-octique/wiki/assets/skip.svg}} \\
\texttt{\ \#octique("sliders")\ } &
\pandocbounded{\includesvg[keepaspectratio]{https://github.com/0x6b/typst-octique/wiki/assets/sliders.svg}} \\
\texttt{\ \#octique("smiley")\ } &
\pandocbounded{\includesvg[keepaspectratio]{https://github.com/0x6b/typst-octique/wiki/assets/smiley.svg}} \\
\texttt{\ \#octique("sort-asc")\ } &
\pandocbounded{\includesvg[keepaspectratio]{https://github.com/0x6b/typst-octique/wiki/assets/sort-asc.svg}} \\
\texttt{\ \#octique("sort-desc")\ } &
\pandocbounded{\includesvg[keepaspectratio]{https://github.com/0x6b/typst-octique/wiki/assets/sort-desc.svg}} \\
\texttt{\ \#octique("sparkle-fill")\ } &
\pandocbounded{\includesvg[keepaspectratio]{https://github.com/0x6b/typst-octique/wiki/assets/sparkle-fill.svg}} \\
\texttt{\ \#octique("sponsor-tiers")\ } &
\pandocbounded{\includesvg[keepaspectratio]{https://github.com/0x6b/typst-octique/wiki/assets/sponsor-tiers.svg}} \\
\texttt{\ \#octique("square-fill")\ } &
\pandocbounded{\includesvg[keepaspectratio]{https://github.com/0x6b/typst-octique/wiki/assets/square-fill.svg}} \\
\texttt{\ \#octique("square")\ } &
\pandocbounded{\includesvg[keepaspectratio]{https://github.com/0x6b/typst-octique/wiki/assets/square.svg}} \\
\texttt{\ \#octique("squirrel")\ } &
\pandocbounded{\includesvg[keepaspectratio]{https://github.com/0x6b/typst-octique/wiki/assets/squirrel.svg}} \\
\texttt{\ \#octique("stack")\ } &
\pandocbounded{\includesvg[keepaspectratio]{https://github.com/0x6b/typst-octique/wiki/assets/stack.svg}} \\
\texttt{\ \#octique("star-fill")\ } &
\pandocbounded{\includesvg[keepaspectratio]{https://github.com/0x6b/typst-octique/wiki/assets/star-fill.svg}} \\
\texttt{\ \#octique("star")\ } &
\pandocbounded{\includesvg[keepaspectratio]{https://github.com/0x6b/typst-octique/wiki/assets/star.svg}} \\
\texttt{\ \#octique("stop")\ } &
\pandocbounded{\includesvg[keepaspectratio]{https://github.com/0x6b/typst-octique/wiki/assets/stop.svg}} \\
\texttt{\ \#octique("stopwatch")\ } &
\pandocbounded{\includesvg[keepaspectratio]{https://github.com/0x6b/typst-octique/wiki/assets/stopwatch.svg}} \\
\texttt{\ \#octique("strikethrough")\ } &
\pandocbounded{\includesvg[keepaspectratio]{https://github.com/0x6b/typst-octique/wiki/assets/strikethrough.svg}} \\
\texttt{\ \#octique("sun")\ } &
\pandocbounded{\includesvg[keepaspectratio]{https://github.com/0x6b/typst-octique/wiki/assets/sun.svg}} \\
\texttt{\ \#octique("sync")\ } &
\pandocbounded{\includesvg[keepaspectratio]{https://github.com/0x6b/typst-octique/wiki/assets/sync.svg}} \\
\texttt{\ \#octique("tab-external")\ } &
\pandocbounded{\includesvg[keepaspectratio]{https://github.com/0x6b/typst-octique/wiki/assets/tab-external.svg}} \\
\texttt{\ \#octique("table")\ } &
\pandocbounded{\includesvg[keepaspectratio]{https://github.com/0x6b/typst-octique/wiki/assets/table.svg}} \\
\texttt{\ \#octique("tag")\ } &
\pandocbounded{\includesvg[keepaspectratio]{https://github.com/0x6b/typst-octique/wiki/assets/tag.svg}} \\
\texttt{\ \#octique("tasklist")\ } &
\pandocbounded{\includesvg[keepaspectratio]{https://github.com/0x6b/typst-octique/wiki/assets/tasklist.svg}} \\
\texttt{\ \#octique("telescope-fill")\ } &
\pandocbounded{\includesvg[keepaspectratio]{https://github.com/0x6b/typst-octique/wiki/assets/telescope-fill.svg}} \\
\texttt{\ \#octique("telescope")\ } &
\pandocbounded{\includesvg[keepaspectratio]{https://github.com/0x6b/typst-octique/wiki/assets/telescope.svg}} \\
\texttt{\ \#octique("terminal")\ } &
\pandocbounded{\includesvg[keepaspectratio]{https://github.com/0x6b/typst-octique/wiki/assets/terminal.svg}} \\
\texttt{\ \#octique("three-bars")\ } &
\pandocbounded{\includesvg[keepaspectratio]{https://github.com/0x6b/typst-octique/wiki/assets/three-bars.svg}} \\
\texttt{\ \#octique("thumbsdown")\ } &
\pandocbounded{\includesvg[keepaspectratio]{https://github.com/0x6b/typst-octique/wiki/assets/thumbsdown.svg}} \\
\texttt{\ \#octique("thumbsup")\ } &
\pandocbounded{\includesvg[keepaspectratio]{https://github.com/0x6b/typst-octique/wiki/assets/thumbsup.svg}} \\
\texttt{\ \#octique("tools")\ } &
\pandocbounded{\includesvg[keepaspectratio]{https://github.com/0x6b/typst-octique/wiki/assets/tools.svg}} \\
\texttt{\ \#octique("tracked-by-closed-completed")\ } &
\pandocbounded{\includesvg[keepaspectratio]{https://github.com/0x6b/typst-octique/wiki/assets/tracked-by-closed-completed.svg}} \\
\texttt{\ \#octique("tracked-by-closed-not-planned")\ } &
\pandocbounded{\includesvg[keepaspectratio]{https://github.com/0x6b/typst-octique/wiki/assets/tracked-by-closed-not-planned.svg}} \\
\texttt{\ \#octique("trash")\ } &
\pandocbounded{\includesvg[keepaspectratio]{https://github.com/0x6b/typst-octique/wiki/assets/trash.svg}} \\
\texttt{\ \#octique("triangle-down")\ } &
\pandocbounded{\includesvg[keepaspectratio]{https://github.com/0x6b/typst-octique/wiki/assets/triangle-down.svg}} \\
\texttt{\ \#octique("triangle-left")\ } &
\pandocbounded{\includesvg[keepaspectratio]{https://github.com/0x6b/typst-octique/wiki/assets/triangle-left.svg}} \\
\texttt{\ \#octique("triangle-right")\ } &
\pandocbounded{\includesvg[keepaspectratio]{https://github.com/0x6b/typst-octique/wiki/assets/triangle-right.svg}} \\
\texttt{\ \#octique("triangle-up")\ } &
\pandocbounded{\includesvg[keepaspectratio]{https://github.com/0x6b/typst-octique/wiki/assets/triangle-up.svg}} \\
\texttt{\ \#octique("trophy")\ } &
\pandocbounded{\includesvg[keepaspectratio]{https://github.com/0x6b/typst-octique/wiki/assets/trophy.svg}} \\
\texttt{\ \#octique("typography")\ } &
\pandocbounded{\includesvg[keepaspectratio]{https://github.com/0x6b/typst-octique/wiki/assets/typography.svg}} \\
\texttt{\ \#octique("undo")\ } &
\pandocbounded{\includesvg[keepaspectratio]{https://github.com/0x6b/typst-octique/wiki/assets/undo.svg}} \\
\texttt{\ \#octique("unfold")\ } &
\pandocbounded{\includesvg[keepaspectratio]{https://github.com/0x6b/typst-octique/wiki/assets/unfold.svg}} \\
\texttt{\ \#octique("unlink")\ } &
\pandocbounded{\includesvg[keepaspectratio]{https://github.com/0x6b/typst-octique/wiki/assets/unlink.svg}} \\
\texttt{\ \#octique("unlock")\ } &
\pandocbounded{\includesvg[keepaspectratio]{https://github.com/0x6b/typst-octique/wiki/assets/unlock.svg}} \\
\texttt{\ \#octique("unmute")\ } &
\pandocbounded{\includesvg[keepaspectratio]{https://github.com/0x6b/typst-octique/wiki/assets/unmute.svg}} \\
\texttt{\ \#octique("unread")\ } &
\pandocbounded{\includesvg[keepaspectratio]{https://github.com/0x6b/typst-octique/wiki/assets/unread.svg}} \\
\texttt{\ \#octique("unverified")\ } &
\pandocbounded{\includesvg[keepaspectratio]{https://github.com/0x6b/typst-octique/wiki/assets/unverified.svg}} \\
\texttt{\ \#octique("upload")\ } &
\pandocbounded{\includesvg[keepaspectratio]{https://github.com/0x6b/typst-octique/wiki/assets/upload.svg}} \\
\texttt{\ \#octique("verified")\ } &
\pandocbounded{\includesvg[keepaspectratio]{https://github.com/0x6b/typst-octique/wiki/assets/verified.svg}} \\
\texttt{\ \#octique("versions")\ } &
\pandocbounded{\includesvg[keepaspectratio]{https://github.com/0x6b/typst-octique/wiki/assets/versions.svg}} \\
\texttt{\ \#octique("video")\ } &
\pandocbounded{\includesvg[keepaspectratio]{https://github.com/0x6b/typst-octique/wiki/assets/video.svg}} \\
\texttt{\ \#octique("webhook")\ } &
\pandocbounded{\includesvg[keepaspectratio]{https://github.com/0x6b/typst-octique/wiki/assets/webhook.svg}} \\
\texttt{\ \#octique("workflow")\ } &
\pandocbounded{\includesvg[keepaspectratio]{https://github.com/0x6b/typst-octique/wiki/assets/workflow.svg}} \\
\texttt{\ \#octique("x-circle-fill")\ } &
\pandocbounded{\includesvg[keepaspectratio]{https://github.com/0x6b/typst-octique/wiki/assets/x-circle-fill.svg}} \\
\texttt{\ \#octique("x-circle")\ } &
\pandocbounded{\includesvg[keepaspectratio]{https://github.com/0x6b/typst-octique/wiki/assets/x-circle.svg}} \\
\texttt{\ \#octique("x")\ } &
\pandocbounded{\includesvg[keepaspectratio]{https://github.com/0x6b/typst-octique/wiki/assets/x.svg}} \\
\texttt{\ \#octique("zap")\ } &
\pandocbounded{\includesvg[keepaspectratio]{https://github.com/0x6b/typst-octique/wiki/assets/zap.svg}} \\
\texttt{\ \#octique("zoom-in")\ } &
\pandocbounded{\includesvg[keepaspectratio]{https://github.com/0x6b/typst-octique/wiki/assets/zoom-in.svg}} \\
\texttt{\ \#octique("zoom-out")\ } &
\pandocbounded{\includesvg[keepaspectratio]{https://github.com/0x6b/typst-octique/wiki/assets/zoom-out.svg}} \\
\end{longtable}

\subsection{License}\label{license}

MIT. See
\href{https://github.com/typst/packages/raw/main/packages/preview/octique/0.1.0/LICENSE}{LICENSE}
for detail.

Octicons are © GitHub, Inc. When using the GitHub logos, you should
follow the \href{https://github.com/logos}{GitHub logo guidelines} .

\subsubsection{How to add}\label{how-to-add}

Copy this into your project and use the import as \texttt{\ octique\ }

\begin{verbatim}
#import "@preview/octique:0.1.0"
\end{verbatim}

\includesvg[width=0.16667in,height=0.16667in]{/assets/icons/16-copy.svg}

Check the docs for
\href{https://typst.app/docs/reference/scripting/\#packages}{more
information on how to import packages} .

\subsubsection{About}\label{about}

\begin{description}
\tightlist
\item[Author :]
0x6b
\item[License:]
MIT
\item[Current version:]
0.1.0
\item[Last updated:]
November 18, 2023
\item[First released:]
November 18, 2023
\item[Archive size:]
51.1 kB
\href{https://packages.typst.org/preview/octique-0.1.0.tar.gz}{\pandocbounded{\includesvg[keepaspectratio]{/assets/icons/16-download.svg}}}
\item[Repository:]
\href{https://github.com/0x6b/typst-octique}{GitHub}
\end{description}

\subsubsection{Where to report issues?}\label{where-to-report-issues}

This package is a project of 0x6b . Report issues on
\href{https://github.com/0x6b/typst-octique}{their repository} . You can
also try to ask for help with this package on the
\href{https://forum.typst.app}{Forum} .

Please report this package to the Typst team using the
\href{https://typst.app/contact}{contact form} if you believe it is a
safety hazard or infringes upon your rights.

\phantomsection\label{versions}
\subsubsection{Version history}\label{version-history}

\begin{longtable}[]{@{}ll@{}}
\toprule\noalign{}
Version & Release Date \\
\midrule\noalign{}
\endhead
\bottomrule\noalign{}
\endlastfoot
0.1.0 & November 18, 2023 \\
\end{longtable}

Typst GmbH did not create this package and cannot guarantee correct
functionality of this package or compatibility with any version of the
Typst compiler or app.


\section{Package List LaTeX/ssrn-scribe.tex}
\title{typst.app/universe/package/ssrn-scribe}

\phantomsection\label{banner}
\phantomsection\label{template-thumbnail}
\pandocbounded{\includegraphics[keepaspectratio]{https://packages.typst.org/preview/thumbnails/ssrn-scribe-0.6.0-small.webp}}

\section{ssrn-scribe}\label{ssrn-scribe}

{ 0.6.0 }

Personal working paper template for general doc and SSRN paper.

\href{/app?template=ssrn-scribe&version=0.6.0}{Create project in app}

\phantomsection\label{readme}
Following the official tutorial, I create a single-column paper template
for general use. You can use it for papers published on SSRN etc.

\subsection{How to use}\label{how-to-use}

\subsubsection{Use as a template
package}\label{use-as-a-template-package}

Typst integrated the template with their official package manager. You
can use it as the other third-party packages.

You only need to enter the following command in the terminal to
initialize the template.

\begin{Shaded}
\begin{Highlighting}[]
\ExtensionTok{typst}\NormalTok{ init @preview/ssrn{-}scribe}
\end{Highlighting}
\end{Shaded}

If will generate a subfolder \texttt{\ ssrn-scribe\ } including the
\texttt{\ main.typ\ } file in the current directory with the latest
version of the template.

\subsubsection{Mannully use}\label{mannully-use}

\begin{enumerate}
\tightlist
\item
  Download the template or clone the repository.
\item
  generate your bibliography file using \texttt{\ .biblatex\ } and store
  the file in the same directory of the template.
\item
  modify the \texttt{\ main.typ\ } file in the subfolder
  \texttt{\ /template\ } and compile it. \textbf{\emph{Note:} You should
  have \texttt{\ paper\_template.typ\ } and \texttt{\ main.typ\ } in the
  same directory.}
\end{enumerate}

In the template, you can modify the following parameters:

\texttt{\ maketitle\ } is a boolean ( \textbf{compulsory} ). If
\texttt{\ maketitle=true\ } , the template will generate a new page for
the title. Otherwise, the title will be shown on the first page.

\begin{itemize}
\tightlist
\item
  \texttt{\ maketitle=true\ } :
\end{itemize}

\begin{longtable}[]{@{}llll@{}}
\toprule\noalign{}
Parameter & Default & Optional & Description \\
\midrule\noalign{}
\endhead
\bottomrule\noalign{}
\endlastfoot
\texttt{\ font\ } & “PT Serif� & Yes & The font of the paper. You
can choose “Times New Roman� or “Palatino� \\
\texttt{\ fontsize\ } & 11pt & Yes & The font size of the paper. You can
choose 10pt or 12pt \\
\texttt{\ title\ } & “Title� & No & The title of the paper \\
\texttt{\ subtitle\ } & none & Yes & The subtitle of the paper, use
“� or {[}{]} \\
\texttt{\ authors\ } & none & No & The authors of the paper \\
\texttt{\ date\ } & none & Yes & The date of the paper \\
\texttt{\ abstract\ } & none & Yes & The abstract of the paper \\
\texttt{\ keywords\ } & none & Yes & The keywords of the paper \\
\texttt{\ JEL\ } & none & Yes & The JEL codes of the paper \\
\texttt{\ acknowledgments\ } & none & Yes & The acknowledgment of the
paper \\
\texttt{\ bibliography\ } & none & Yes & The bibliography of the paper
\texttt{\ bibliography:\ bibliography("bib.bib",\ title:\ "References",\ style:\ "apa")\ } \\
\end{longtable}

\begin{itemize}
\tightlist
\item
  \texttt{\ maketitle=false\ } :
\end{itemize}

\begin{longtable}[]{@{}llll@{}}
\toprule\noalign{}
Parameter & Default & Optional & Description \\
\midrule\noalign{}
\endhead
\bottomrule\noalign{}
\endlastfoot
\texttt{\ font\ } & “PT Serif� & Yes & The font of the paper. You
can choose “Times New Roman� or “Palatino� \\
\texttt{\ fontsize\ } & 11pt & Yes & The font size of the paper. You can
choose 10pt or 12pt \\
\texttt{\ title\ } & “Title� & No & The title of the paper \\
\texttt{\ subtitle\ } & none & Yes & The subtitle of the paper, use
“� or {[}{]} \\
\texttt{\ authors\ } & none & No & The authors of the paper \\
\texttt{\ date\ } & none & Yes & The date of the paper \\
\texttt{\ bibliography\ } & none & Yes & The bibliography of the paper
\texttt{\ bibliography:\ bibliography("bib.bib",\ title:\ "References",\ style:\ "apa")\ } \\
\end{longtable}

\textbf{Note: You need to keep the comma at the end of the first bracket
of the author’s list, even if you have only one author.}

\begin{Shaded}
\begin{Highlighting}[]
\NormalTok{    (}
\NormalTok{    name: "",}
\NormalTok{    affiliation: "", // optional}
\NormalTok{    email: "", // optional}
\NormalTok{    note: "", // optional}
\NormalTok{    ),}
\end{Highlighting}
\end{Shaded}

\begin{Shaded}
\begin{Highlighting}[]
\NormalTok{\#import "@preview/ssrn{-}scribe:0.6.0": *}

\NormalTok{\#show: paper.with(}
\NormalTok{  font: "PT Serif", // "Times New Roman"}
\NormalTok{  fontsize: 12pt, // 12pt}
\NormalTok{  maketitle: true, // whether to add new page for title}
\NormalTok{  title: [\#lorem(5)], // title }
\NormalTok{  subtitle: "A work in progress", // subtitle}
\NormalTok{  authors: (}
\NormalTok{    (}
\NormalTok{      name: "Theresa Tungsten",}
\NormalTok{      affiliation: "Artos Institute",}
\NormalTok{      email: "tung@artos.edu",}
\NormalTok{      note: "123",}
\NormalTok{    ),}
\NormalTok{  ),}
\NormalTok{  date: "July 2023",}
\NormalTok{  abstract: lorem(80), // replace lorem(80) with [ Your abstract here. ]}
\NormalTok{  keywords: [}
\NormalTok{    Imputation,}
\NormalTok{    Multiple Imputation,}
\NormalTok{    Bayesian,],}
\NormalTok{  JEL: [G11, G12],}
\NormalTok{  acknowledgments: "This paper is a work in progress. Please do not cite without permission.", }
\NormalTok{  // bibliography: bibliography("bib.bib", title: "References", style: "apa"),}
\NormalTok{)}
\NormalTok{= Introduction}
\NormalTok{\#lorem(50)}
\end{Highlighting}
\end{Shaded}

\subsection{Preview}\label{preview}

\subsubsection{Example}\label{example}

Here is a screenshot of the template:
\pandocbounded{\includegraphics[keepaspectratio]{https://minioapi.pjx.ac.cn/img1/2024/03/63ce084e2a43bc2e7e31bd79315a0fb5.png}}

\subsubsection{\texorpdfstring{Example-brief with
\texttt{\ maketitle=true\ }}{Example-brief with  maketitle=true }}\label{example-brief-with-maketitletrue}

\pandocbounded{\includegraphics[keepaspectratio]{https://minioapi.pjx.ac.cn/img1/2024/06/8d203bd7f2fbf20b39b33334f0ee4a36.png}}

\subsubsection{\texorpdfstring{Example-brief with
\texttt{\ maketitle=false\ }}{Example-brief with  maketitle=false }}\label{example-brief-with-maketitlefalse}

\pandocbounded{\includegraphics[keepaspectratio]{https://minioapi.pjx.ac.cn/img1/2024/06/83dd5821409031ce0a2c2a15e014cc60.png}}

\href{/app?template=ssrn-scribe&version=0.6.0}{Create project in app}

\subsubsection{How to use}\label{how-to-use-1}

Click the button above to create a new project using this template in
the Typst app.

You can also use the Typst CLI to start a new project on your computer
using this command:

\begin{verbatim}
typst init @preview/ssrn-scribe:0.6.0
\end{verbatim}

\includesvg[width=0.16667in,height=0.16667in]{/assets/icons/16-copy.svg}

\subsubsection{About}\label{about}

\begin{description}
\tightlist
\item[Author :]
jxpeng98
\item[License:]
MIT
\item[Current version:]
0.6.0
\item[Last updated:]
June 11, 2024
\item[First released:]
March 20, 2024
\item[Archive size:]
4.07 kB
\href{https://packages.typst.org/preview/ssrn-scribe-0.6.0.tar.gz}{\pandocbounded{\includesvg[keepaspectratio]{/assets/icons/16-download.svg}}}
\item[Repository:]
\href{https://github.com/jxpeng98/Typst-Paper-Template}{GitHub}
\item[Categor y :]
\begin{itemize}
\tightlist
\item[]
\item
  \pandocbounded{\includesvg[keepaspectratio]{/assets/icons/16-atom.svg}}
  \href{https://typst.app/universe/search/?category=paper}{Paper}
\end{itemize}
\end{description}

\subsubsection{Where to report issues?}\label{where-to-report-issues}

This template is a project of jxpeng98 . Report issues on
\href{https://github.com/jxpeng98/Typst-Paper-Template}{their
repository} . You can also try to ask for help with this template on the
\href{https://forum.typst.app}{Forum} .

Please report this template to the Typst team using the
\href{https://typst.app/contact}{contact form} if you believe it is a
safety hazard or infringes upon your rights.

\phantomsection\label{versions}
\subsubsection{Version history}\label{version-history}

\begin{longtable}[]{@{}ll@{}}
\toprule\noalign{}
Version & Release Date \\
\midrule\noalign{}
\endhead
\bottomrule\noalign{}
\endlastfoot
0.6.0 & June 11, 2024 \\
\href{https://typst.app/universe/package/ssrn-scribe/0.5.0/}{0.5.0} &
April 5, 2024 \\
\href{https://typst.app/universe/package/ssrn-scribe/0.4.9/}{0.4.9} &
March 20, 2024 \\
\end{longtable}

Typst GmbH did not create this template and cannot guarantee correct
functionality of this template or compatibility with any version of the
Typst compiler or app.


\section{Package List LaTeX/codly.tex}
\title{typst.app/universe/package/codly}

\phantomsection\label{banner}
\section{codly}\label{codly}

{ 1.0.0 }

Codly is a beautiful code presentation template with many features like
smart indentation, line numbering, highlighting, etc.

{ } Featured Package

\phantomsection\label{readme}
\href{https://github.com/Dherse/codly/blob/main/docs.pdf}{\pandocbounded{\includegraphics[keepaspectratio]{https://img.shields.io/website?down_message=offline&label=manual&up_color=007aff&up_message=online&url=https\%3A\%2F\%2Fgithub.com\%2FDherse\%2Fcodly\%2Fblob\%2Fmain\%2Fdocs.pdf}}}
\href{https://github.com/Dherse/codly/blob/main/LICENSE}{\pandocbounded{\includegraphics[keepaspectratio]{https://img.shields.io/badge/license-MIT-brightgreen}}}
\pandocbounded{\includesvg[keepaspectratio]{https://github.com/Dherse/codly/actions/workflows/test.yml/badge.svg}}

Codly is a package that lets you easily create \textbf{beautiful} code
blocks for your Typst documents. It uses the newly added
\href{https://typst.app/docs/reference/text/raw/\#definitions-line}{\texttt{\ raw.line\ }}
function to work across all languages easily. You can customize the
icons, colors, and more to suit your document’s theme. By default it
has zebra striping, line numbers, for ease of reading.

A full set of documentation can be found
\href{https://raw.githubusercontent.com/Dherse/codly/main/docs.pdf}{in
the repo} .

\pandocbounded{\includegraphics[keepaspectratio]{https://github.com/typst/packages/raw/main/packages/preview/codly/1.0.0/demo.png}}

\begin{Shaded}
\begin{Highlighting}[]
\NormalTok{\#import "@preview/codly:1.0.0": *}
\NormalTok{\#show: codly{-}init.with()}

\NormalTok{\#codly(}
\NormalTok{  languages: (}
\NormalTok{    rust: (}
\NormalTok{      name: "Rust",}
\NormalTok{      icon: text(font: "tabler{-}icons", "\textbackslash{}u\{fa53\}),}
\NormalTok{      color: rgb("\#CE412B")}
\NormalTok{    ),}
\NormalTok{  )}
\NormalTok{)}

\NormalTok{\textasciigrave{}\textasciigrave{}\textasciigrave{}rust}
\NormalTok{pub fn main() \{}
\NormalTok{    println!("Hello, world!");}
\NormalTok{\}}
\NormalTok{\textasciigrave{}\textasciigrave{}\textasciigrave{}}
\end{Highlighting}
\end{Shaded}

\subsubsection{Setup}\label{setup}

To start using codly, you need to initialize codly using a show rule:

\begin{Shaded}
\begin{Highlighting}[]
\NormalTok{\#show: codly{-}init.with()}
\end{Highlighting}
\end{Shaded}

\begin{quote}
{[}!TIP{]} You only need to do this once at the top of your document!
\end{quote}

Then you \emph{can} to configure codly with your parameters:

\begin{Shaded}
\begin{Highlighting}[]
\NormalTok{\#codly(}
\NormalTok{  languages: (}
\NormalTok{    rust: (name: "Rust", icon: "\textbackslash{}u\{fa53\}", color: rgb("\#CE412B")),}
\NormalTok{  )}
\NormalTok{)}
\end{Highlighting}
\end{Shaded}

\begin{quote}
{[}!IMPORTANT{]} Any parameter that you leave blank will use the
previous values (or the default value if never set) similar to a
\texttt{\ set\ } rule in regular typst. But the changes are always
global unless you use the provided \texttt{\ codly.local\ } function. To
get a full list of all settings, see the
\href{https://raw.githubusercontent.com/Dherse/codly/main/docs.pdf}{documentation}
.
\end{quote}

Then you just need to add a code block and it will be automatically
displayed correctly:

\begin{verbatim}
```rust
pub fn main() {
    println!("Hello, world!");
}
```
\end{verbatim}

\subsubsection{Disabling}\label{disabling}

To locally disable codly, you can just do the following, you can then
later re-enable it using the \texttt{\ codly\ } configuration function.

\begin{Shaded}
\begin{Highlighting}[]
\NormalTok{\#disable{-}codly()}
\end{Highlighting}
\end{Shaded}

Alternatively, you can use the \texttt{\ no-codly\ } function to achieve
the same effect locally:

\begin{Shaded}
\begin{Highlighting}[]
\NormalTok{\#no{-}codly[}
\NormalTok{  \textasciigrave{}\textasciigrave{}\textasciigrave{}typ}
\NormalTok{  I will be displayed using the normal raw blocks.}
\NormalTok{  \textasciigrave{}\textasciigrave{}\textasciigrave{}}
\NormalTok{]}
\end{Highlighting}
\end{Shaded}

\subsubsection{Setting an offset}\label{setting-an-offset}

If you wish to add an offset to your code block, but without selecting a
subset of lines, you can use the \texttt{\ codly-offset\ } function:

\begin{Shaded}
\begin{Highlighting}[]
\NormalTok{// Sets a 5 line offset}
\NormalTok{\#codly{-}offset(5)}
\end{Highlighting}
\end{Shaded}

\subsubsection{Selecting a subset of
lines}\label{selecting-a-subset-of-lines}

If you wish to select a subset of lines, you can use the
\texttt{\ codly-range\ } function. By setting the start to 1 and the end
to \texttt{\ none\ } you can select all lines from the start to the end
of the code block.

\begin{Shaded}
\begin{Highlighting}[]
\NormalTok{\#codly{-}range(start: 5, end: 10)}
\end{Highlighting}
\end{Shaded}

\subsubsection{Adding a “skip�}\label{adding-a-uxe2ux153skipuxe2}

You can add a “fake� skip between lines using the \texttt{\ skips\ }
parameters:

\begin{Shaded}
\begin{Highlighting}[]
\NormalTok{// Before the 5th line (indexing start at 0), insert a 32 line jump.}
\NormalTok{\#codly(skips: ((4, 32), ))}
\end{Highlighting}
\end{Shaded}

This can be customized using the \texttt{\ skip-line\ } and
\texttt{\ skip-number\ } to customize what it looks like.

\subsubsection{Adding annotations}\label{adding-annotations}

\begin{quote}
{[}!IMPORTANT{]} This is a Beta feature and has a few quirks, refer to
\href{https://raw.githubusercontent.com/Dherse/codly/main/docs.pdf}{the
documentation} for those
\end{quote}

You can annotate a line/group of lines using the
\texttt{\ annotations\ } parameters :

\begin{Shaded}
\begin{Highlighting}[]
\NormalTok{// Add an annotation from the second line (0 indexing) to the 5th line included.}
\NormalTok{\#codly(}
\NormalTok{  annotations: (}
\NormalTok{    (}
\NormalTok{      start: 1,}
\NormalTok{      end: 4,}
\NormalTok{      content: block(}
\NormalTok{        width: 2em,}
\NormalTok{        // Rotate the element to make it look nice}
\NormalTok{        rotate(}
\NormalTok{          {-}90deg,}
\NormalTok{          align(center, box(width: 100pt)[Function body])}
\NormalTok{        )}
\NormalTok{      )}
\NormalTok{    ), }
\NormalTok{  )}
\NormalTok{)}
\end{Highlighting}
\end{Shaded}

\subsubsection{Disabling line numbers}\label{disabling-line-numbers}

You can configure this with the \texttt{\ codly\ } function:

\begin{Shaded}
\begin{Highlighting}[]
\NormalTok{\#codly(number{-}format: none)}
\end{Highlighting}
\end{Shaded}

\subsubsection{Disabling zebra striping}\label{disabling-zebra-striping}

You disable zebra striping by setting the \texttt{\ zebra-fill\ } to
white.

\begin{Shaded}
\begin{Highlighting}[]
\NormalTok{\#codly(zebra{-}fill: none)}
\end{Highlighting}
\end{Shaded}

\subsubsection{Customize the stroke}\label{customize-the-stroke}

You can customize the stroke surrounding the figure using the
\texttt{\ stroke\ } parameter of the \texttt{\ codly\ } function:

\begin{Shaded}
\begin{Highlighting}[]
\NormalTok{\#codly(stroke: 1pt + red)}
\end{Highlighting}
\end{Shaded}

\subsubsection{Misc}\label{misc}

You can also disable the icon, by setting the \texttt{\ display-icon\ }
parameter to \texttt{\ false\ } :

\begin{Shaded}
\begin{Highlighting}[]
\NormalTok{\#codly(display{-}icon: false)}
\end{Highlighting}
\end{Shaded}

Same with the name, whether the block is breakable, the radius, the
padding, and the width of the numbers columns, and so many more
\href{https://raw.githubusercontent.com/Dherse/codly/main/docs.pdf}{documentation}
.

\subsubsection{How to add}\label{how-to-add}

Copy this into your project and use the import as \texttt{\ codly\ }

\begin{verbatim}
#import "@preview/codly:1.0.0"
\end{verbatim}

\includesvg[width=0.16667in,height=0.16667in]{/assets/icons/16-copy.svg}

Check the docs for
\href{https://typst.app/docs/reference/scripting/\#packages}{more
information on how to import packages} .

\subsubsection{About}\label{about}

\begin{description}
\tightlist
\item[Author :]
\href{https://github.com/Dherse}{Dherse}
\item[License:]
MIT
\item[Current version:]
1.0.0
\item[Last updated:]
July 17, 2024
\item[First released:]
November 6, 2023
\item[Minimum Typst version:]
0.11.0
\item[Archive size:]
14.3 kB
\href{https://packages.typst.org/preview/codly-1.0.0.tar.gz}{\pandocbounded{\includesvg[keepaspectratio]{/assets/icons/16-download.svg}}}
\item[Repository:]
\href{https://github.com/Dherse/codly}{GitHub}
\end{description}

\subsubsection{Where to report issues?}\label{where-to-report-issues}

This package is a project of Dherse . Report issues on
\href{https://github.com/Dherse/codly}{their repository} . You can also
try to ask for help with this package on the
\href{https://forum.typst.app}{Forum} .

Please report this package to the Typst team using the
\href{https://typst.app/contact}{contact form} if you believe it is a
safety hazard or infringes upon your rights.

\phantomsection\label{versions}
\subsubsection{Version history}\label{version-history}

\begin{longtable}[]{@{}ll@{}}
\toprule\noalign{}
Version & Release Date \\
\midrule\noalign{}
\endhead
\bottomrule\noalign{}
\endlastfoot
1.0.0 & July 17, 2024 \\
\href{https://typst.app/universe/package/codly/0.2.1/}{0.2.1} & April 1,
2024 \\
\href{https://typst.app/universe/package/codly/0.2.0/}{0.2.0} & January
1, 2024 \\
\href{https://typst.app/universe/package/codly/0.1.0/}{0.1.0} & November
6, 2023 \\
\end{longtable}

Typst GmbH did not create this package and cannot guarantee correct
functionality of this package or compatibility with any version of the
Typst compiler or app.


\section{Package List LaTeX/tuhi-exam-vuw.tex}
\title{typst.app/universe/package/tuhi-exam-vuw}

\phantomsection\label{banner}
\phantomsection\label{template-thumbnail}
\pandocbounded{\includegraphics[keepaspectratio]{https://packages.typst.org/preview/thumbnails/tuhi-exam-vuw-0.1.0-small.webp}}

\section{tuhi-exam-vuw}\label{tuhi-exam-vuw}

{ 0.1.0 }

A poster template for VUW exams.

\href{/app?template=tuhi-exam-vuw&version=0.1.0}{Create project in app}

\phantomsection\label{readme}
A Typst template for VUW exams. To get started:

\begin{Shaded}
\begin{Highlighting}[]
\NormalTok{typst init @preview/tuhi{-}exam{-}vuw:0.1.0}
\end{Highlighting}
\end{Shaded}

And edit the \texttt{\ main.typ\ } example.

\pandocbounded{\includegraphics[keepaspectratio]{https://github.com/typst/packages/raw/main/packages/preview/tuhi-exam-vuw/0.1.0/thumbnail.png}}

\subsection{Contributing}\label{contributing}

PRs are welcome! And if you encounter any bugs or have any
requests/ideas, feel free to open an issue.

\href{/app?template=tuhi-exam-vuw&version=0.1.0}{Create project in app}

\subsubsection{How to use}\label{how-to-use}

Click the button above to create a new project using this template in
the Typst app.

You can also use the Typst CLI to start a new project on your computer
using this command:

\begin{verbatim}
typst init @preview/tuhi-exam-vuw:0.1.0
\end{verbatim}

\includesvg[width=0.16667in,height=0.16667in]{/assets/icons/16-copy.svg}

\subsubsection{About}\label{about}

\begin{description}
\tightlist
\item[Author :]
\href{https://github.com/baptiste}{baptiste}
\item[License:]
MPL-2.0
\item[Current version:]
0.1.0
\item[Last updated:]
May 24, 2024
\item[First released:]
May 24, 2024
\item[Archive size:]
136 kB
\href{https://packages.typst.org/preview/tuhi-exam-vuw-0.1.0.tar.gz}{\pandocbounded{\includesvg[keepaspectratio]{/assets/icons/16-download.svg}}}
\item[Categor y :]
\begin{itemize}
\tightlist
\item[]
\item
  \pandocbounded{\includesvg[keepaspectratio]{/assets/icons/16-envelope.svg}}
  \href{https://typst.app/universe/search/?category=office}{Office}
\end{itemize}
\end{description}

\subsubsection{Where to report issues?}\label{where-to-report-issues}

This template is a project of baptiste . You can also try to ask for
help with this template on the \href{https://forum.typst.app}{Forum} .

Please report this template to the Typst team using the
\href{https://typst.app/contact}{contact form} if you believe it is a
safety hazard or infringes upon your rights.

\phantomsection\label{versions}
\subsubsection{Version history}\label{version-history}

\begin{longtable}[]{@{}ll@{}}
\toprule\noalign{}
Version & Release Date \\
\midrule\noalign{}
\endhead
\bottomrule\noalign{}
\endlastfoot
0.1.0 & May 24, 2024 \\
\end{longtable}

Typst GmbH did not create this template and cannot guarantee correct
functionality of this template or compatibility with any version of the
Typst compiler or app.


\section{Package List LaTeX/fractusist.tex}
\title{typst.app/universe/package/fractusist}

\phantomsection\label{banner}
\section{fractusist}\label{fractusist}

{ 0.1.0 }

Create a variety of wonderful fractals in Typst.

\phantomsection\label{readme}
Create a variety of wonderful fractals in Typst.

\subsection{Examples}\label{examples}

The example below creates a dragon curve of the 12th iteration with the
\texttt{\ dragon-curve\ } function.

\pandocbounded{\includegraphics[keepaspectratio]{https://github.com/typst/packages/raw/main/packages/preview/fractusist/0.1.0/examples/dragon-curve-n12.png}}

Show code

\begin{Shaded}
\begin{Highlighting}[]
\NormalTok{\#set page(width: auto, height: auto, margin: 0pt)}

\NormalTok{\#dragon{-}curve(}
\NormalTok{  12,}
\NormalTok{  step{-}size: 6,}
\NormalTok{  stroke{-}style: stroke(}
\NormalTok{    paint: gradient.linear(..color.map.crest, angle: 45deg),}
\NormalTok{    thickness: 3pt,}
\NormalTok{    cap: "square"}
\NormalTok{  )}
\NormalTok{)}
\end{Highlighting}
\end{Shaded}

\subsection{Features}\label{features}

\begin{itemize}
\tightlist
\item
  Use SVG backend for image rendering.
\item
  Generate fractals using
  \href{https://en.wikipedia.org/wiki/L-system}{L-system} .
\item
  The number of iterations, step size, fill and stroke styles, etc. of
  generated fractals could be customized.
\end{itemize}

\subsection{Usage}\label{usage}

Import the latest version of this package with:

\begin{Shaded}
\begin{Highlighting}[]
\NormalTok{\#import "@preview/fractusist:0.1.0": *}
\end{Highlighting}
\end{Shaded}

Each function generates a specific fractal. The input and output
arguments of all functions have a similar style. Typical input arguments
are as follows:

\begin{itemize}
\tightlist
\item
  \texttt{\ n\ } : the number of iterations ( \textbf{the valid range of
  values depends on the specific function} ).
\item
  \emph{\texttt{\ step-size\ }} : step size (in pt).
\item
  \emph{\texttt{\ fill-style\ }} : fill style, can be \texttt{\ none\ }
  or color or gradient ( \textbf{exists only when the curve is closed}
  ).
\item
  \emph{\texttt{\ stroke-style\ }} : stroke style, can be
  \texttt{\ none\ } or color or gradient or stroke object.
\item
  \emph{\texttt{\ width\ }} : the width of the image.
\item
  \emph{\texttt{\ height\ }} : the height of the image.
\item
  \emph{\texttt{\ fit\ }} : how the image should adjust itself to a
  given area, “cover� / “contain� / “stretch�.
\end{itemize}

The content returned is the \texttt{\ image\ } element.

For more codes with these functions see
\href{https://github.com/typst/packages/raw/main/packages/preview/fractusist/0.1.0/tests}{tests}
.

\subsection{Reference}\label{reference}

\subsubsection{Dragon}\label{dragon}

\begin{itemize}
\tightlist
\item
  \texttt{\ dragon-curve\ } : Generate dragon curve (n: range
  \textbf{{[}0, 16{]}} ).
\end{itemize}

\begin{Shaded}
\begin{Highlighting}[]
\NormalTok{\#let dragon{-}curve(n, step{-}size: 10, stroke{-}style: black + 1pt, width: auto, height: auto, fit: "cover") = \{...\}}
\end{Highlighting}
\end{Shaded}

\subsubsection{Hilbert}\label{hilbert}

\begin{itemize}
\tightlist
\item
  \texttt{\ hilbert-curve\ } : Generate 2D Hilbert curve. (n: range
  \textbf{{[}1, 8{]}} ).
\end{itemize}

\begin{Shaded}
\begin{Highlighting}[]
\NormalTok{\#let hilbert{-}curve(n, step{-}size: 10, stroke{-}style: black + 1pt, width: auto, height: auto, fit: "cover") = \{...\}}
\end{Highlighting}
\end{Shaded}

\begin{itemize}
\tightlist
\item
  \texttt{\ peano-curve\ } : Generate 2D Peano curve (n: range
  \textbf{{[}1, 5{]}} ).
\end{itemize}

\begin{Shaded}
\begin{Highlighting}[]
\NormalTok{\#let peano{-}curve(n, step{-}size: 10, stroke{-}style: black + 1pt, width: auto, height: auto, fit: "cover") = \{...\}}
\end{Highlighting}
\end{Shaded}

\subsubsection{Koch}\label{koch}

\begin{itemize}
\tightlist
\item
  \texttt{\ koch-curve\ } : Generate Koch curve (n: range \textbf{{[}0,
  6{]}} ).
\end{itemize}

\begin{Shaded}
\begin{Highlighting}[]
\NormalTok{\#let koch{-}curve(n, step{-}size: 10, stroke{-}style: black + 1pt, width: auto, height: auto, fit: "cover") = \{...\}}
\end{Highlighting}
\end{Shaded}

\begin{itemize}
\tightlist
\item
  \texttt{\ koch-snowflake\ } : Generate Koch snowflake (n: range
  \textbf{{[}0, 6{]}} ).
\end{itemize}

\begin{Shaded}
\begin{Highlighting}[]
\NormalTok{\#let koch{-}snowflake(n, step{-}size: 10, fill{-}style: none, stroke{-}style: black + 1pt, width: auto, height: auto, fit: "cover") = \{...\}}
\end{Highlighting}
\end{Shaded}

\subsubsection{Sierpiński}\label{sierpiuxe5ski}

\begin{itemize}
\tightlist
\item
  \texttt{\ sierpinski-curve\ } : Generate classic Sierpiński curve (n:
  range \textbf{{[}0, 7{]}} ).
\end{itemize}

\begin{Shaded}
\begin{Highlighting}[]
\NormalTok{\#let sierpinski{-}curve(n, step{-}size: 10, fill{-}style: none, stroke{-}style: black + 1pt, width: auto, height: auto, fit: "cover") = \{...\}}
\end{Highlighting}
\end{Shaded}

\begin{itemize}
\tightlist
\item
  \texttt{\ sierpinski-square-curve\ } : Generate Sierpiński square
  curve (n: range \textbf{{[}0, 7{]}} ).
\end{itemize}

\begin{Shaded}
\begin{Highlighting}[]
\NormalTok{\#let sierpinski{-}square{-}curve(n, step{-}size: 10, fill{-}style: none, stroke{-}style: black + 1pt, width: auto, height: auto, fit: "cover") = \{...\}}
\end{Highlighting}
\end{Shaded}

\begin{itemize}
\tightlist
\item
  \texttt{\ sierpinski-arrowhead-curve\ } : Generate Sierpiński
  arrowhead curve (n: range \textbf{{[}0, 8{]}} ).
\end{itemize}

\begin{Shaded}
\begin{Highlighting}[]
\NormalTok{\#let sierpinski{-}arrowhead{-}curve(n, step{-}size: 10, stroke{-}style: black + 1pt, width: auto, height: auto, fit: "cover") = \{...\}}
\end{Highlighting}
\end{Shaded}

\begin{itemize}
\tightlist
\item
  \texttt{\ sierpinski-triangle\ } : Generate 2D Sierpiński triangle
  (n: range \textbf{{[}0, 6{]}} ).
\end{itemize}

\begin{Shaded}
\begin{Highlighting}[]
\NormalTok{\#let sierpinski{-}triangle(n, step{-}size: 10, fill{-}style: none, stroke{-}style: black + 1pt, width: auto, height: auto, fit: "cover") = \{...\}}
\end{Highlighting}
\end{Shaded}

\subsubsection{How to add}\label{how-to-add}

Copy this into your project and use the import as
\texttt{\ fractusist\ }

\begin{verbatim}
#import "@preview/fractusist:0.1.0"
\end{verbatim}

\includesvg[width=0.16667in,height=0.16667in]{/assets/icons/16-copy.svg}

Check the docs for
\href{https://typst.app/docs/reference/scripting/\#packages}{more
information on how to import packages} .

\subsubsection{About}\label{about}

\begin{description}
\tightlist
\item[Author :]
\href{https://github.com/liuguangxi}{Guangxi Liu}
\item[License:]
MIT
\item[Current version:]
0.1.0
\item[Last updated:]
May 6, 2024
\item[First released:]
May 6, 2024
\item[Minimum Typst version:]
0.11.0
\item[Archive size:]
5.75 kB
\href{https://packages.typst.org/preview/fractusist-0.1.0.tar.gz}{\pandocbounded{\includesvg[keepaspectratio]{/assets/icons/16-download.svg}}}
\item[Repository:]
\href{https://github.com/liuguangxi/fractusist}{GitHub}
\item[Discipline s :]
\begin{itemize}
\tightlist
\item[]
\item
  \href{https://typst.app/universe/search/?discipline=computer-science}{Computer
  Science}
\item
  \href{https://typst.app/universe/search/?discipline=mathematics}{Mathematics}
\end{itemize}
\item[Categor ies :]
\begin{itemize}
\tightlist
\item[]
\item
  \pandocbounded{\includesvg[keepaspectratio]{/assets/icons/16-package.svg}}
  \href{https://typst.app/universe/search/?category=components}{Components}
\item
  \pandocbounded{\includesvg[keepaspectratio]{/assets/icons/16-chart.svg}}
  \href{https://typst.app/universe/search/?category=visualization}{Visualization}
\end{itemize}
\end{description}

\subsubsection{Where to report issues?}\label{where-to-report-issues}

This package is a project of Guangxi Liu . Report issues on
\href{https://github.com/liuguangxi/fractusist}{their repository} . You
can also try to ask for help with this package on the
\href{https://forum.typst.app}{Forum} .

Please report this package to the Typst team using the
\href{https://typst.app/contact}{contact form} if you believe it is a
safety hazard or infringes upon your rights.

\phantomsection\label{versions}
\subsubsection{Version history}\label{version-history}

\begin{longtable}[]{@{}ll@{}}
\toprule\noalign{}
Version & Release Date \\
\midrule\noalign{}
\endhead
\bottomrule\noalign{}
\endlastfoot
0.1.0 & May 6, 2024 \\
\end{longtable}

Typst GmbH did not create this package and cannot guarantee correct
functionality of this package or compatibility with any version of the
Typst compiler or app.


\section{Package List LaTeX/htlwienwest-da.tex}
\title{typst.app/universe/package/htlwienwest-da}

\phantomsection\label{banner}
\phantomsection\label{template-thumbnail}
\pandocbounded{\includegraphics[keepaspectratio]{https://packages.typst.org/preview/thumbnails/htlwienwest-da-0.1.0-small.webp}}

\section{htlwienwest-da}\label{htlwienwest-da}

{ 0.1.0 }

The diploma thesis template for students of the HTL Wien West.

{ } Officially affiliated

\href{/app?template=htlwienwest-da&version=0.1.0}{Create project in app}

\phantomsection\label{readme}
This is a Typst diploma thesis template for students of the HTL Wien
West. It fulfils all the necessary requirements for the diploma thesis.

\subsection{Usage}\label{usage}

You can use this template in the Typst web app by clicking “Start from
template� on the dashboard and searching for
\texttt{\ htlwienwest-da\ } .

Alternatively, you can use the CLI to kick this project off using the
command

\begin{verbatim}
typst init @preview/htlwienwest-da
\end{verbatim}

Typst will create a new directory with all the files needed to get you
started.

\subsection{Configuration}\label{configuration}

This template exports the \texttt{\ diplomarbeit\ } function with the
following named arguments:

\begin{itemize}
\tightlist
\item
  \texttt{\ titel\ } : \texttt{\ string\ } - The title of the thesis
\item
  \texttt{\ schuljahr\ } : \texttt{\ string\ } - The current school year
\item
  \texttt{\ abteilung\ } : \texttt{\ string\ } - The student’s
  department
\item
  \texttt{\ unterschrifts-datum\ } : \texttt{\ string\ } - The
  submission date
\item
  \texttt{\ autoren\ } : \texttt{\ array(dict)\ } - An array of all
  authors, represented as dictionaries. Each of them has the following
  properties

  \begin{itemize}
  \tightlist
  \item
    \texttt{\ vorname\ } : \texttt{\ string\ } - Firstname of the
    student
  \item
    \texttt{\ nachname\ } : \texttt{\ string\ } - Lastname of the
    student
  \item
    \texttt{\ klasse\ } : \texttt{\ string\ } - School class of the
    student
  \item
    \texttt{\ betreuer\ } : \texttt{\ dict\ } - The student’s advisor
    as dictionary

    \begin{itemize}
    \tightlist
    \item
      \texttt{\ name\ } : \texttt{\ string\ \textbar{}\ content\ } - The
      advisor’s name
    \item
      \texttt{\ geschlecht\ } :
      \texttt{\ "male"\ \textbar{}\ "female"\ } - Gender of advisor for
      correct gendering
    \end{itemize}
  \item
    \texttt{\ aufgaben\ } : \texttt{\ content\ } - The student’s
    responsibilities
  \end{itemize}
\item
  \texttt{\ kurzfassung\ } : \texttt{\ content\ } - Abstract in german
  as content block
\item
  \texttt{\ abstract\ } : \texttt{\ content\ } - Abstract in english as
  content block
\item
  \texttt{\ vorwort\ } : \texttt{\ content\ } - The thesis’ preface
\item
  \texttt{\ danksagung\ } : \texttt{\ content\ } - Acknowledgement
\item
  \texttt{\ anhang\ } : \texttt{\ content\ \textbar{}\ none\ } -
  Appendix
\item
  \texttt{\ literaturverzeichnis\ } : \texttt{\ function\ } - The
  bibliography prefilled with the BibTex file path
\end{itemize}

The function also accepts a single, positional argument for the body of
the paper.

The template will initialize your package with a sample call to the
\texttt{\ diplomarbeit\ } function in a show rule. If you want to change
an existing project to use thistemplate, you can add a show rule like
this at the top of your file:

\begin{Shaded}
\begin{Highlighting}[]
\NormalTok{\#import "@preview/htlwienwest{-}da:0.1.0": *}

\NormalTok{\#show: diplomarbeit.with(}
\NormalTok{  titel: "Titel der Diplomarbeit",}
\NormalTok{  abteilung: "Informationstechnologie",}
\NormalTok{  schuljahr: "2023/24",}
\NormalTok{  unterschrifts{-}datum: "20.04.2024",}
\NormalTok{  autoren: (}
\NormalTok{   (}
\NormalTok{     vorname: "Hans", nachname: "Mustermann",}
\NormalTok{     klasse: "5AHITN",}
\NormalTok{     betreuer: (name: "Dr. Walter Turbo", geschlecht: "male"),}
\NormalTok{     aufgaben: [}
\NormalTok{       \#lorem(100)}
\NormalTok{     ]}
\NormalTok{   ),}
\NormalTok{   (}
\NormalTok{     vorname: "Herta", nachname: "Musterfrau",}
\NormalTok{     klasse: "5AHITN",}
\NormalTok{     betreuer: (name: "Dipl.{-}Ing Maria Kreisel", geschlecht: "female"),}
\NormalTok{     aufgaben: [}
\NormalTok{       \#lorem(100)}
\NormalTok{     ]}
\NormalTok{   ),}
\NormalTok{  kurzfassung: [}
\NormalTok{    Die Kurzfassung muss die folgenden Inhalte darlegen (§8, Absatz 5 Prüfungsordnung): Thema, Fragestellung, Problemformulierung, wesentliche Ergebnisse. Sie soll einen prägnanten Überblick über die Arbeit geben.}
\NormalTok{  ],}
\NormalTok{  abstract: [}
\NormalTok{    Englische Version der Kurzfassung (siehe \#link(\textless{}Kurzfassung\textgreater{})[\_Kurzfassung\_])}
\NormalTok{  ],}
\NormalTok{  vorwort: [}
\NormalTok{    Perönlicher Zugang zum Thema. Gründe für die Themenwahl.}
\NormalTok{  ],}
\NormalTok{  danksagung: [}
\NormalTok{    Dank an Personen, die bei der Erstellung der Arbeit unterstützt haben.}
\NormalTok{  ],}
\NormalTok{  anhang: include "anhang.typ", // entfernen falls nicht benötigt}
\NormalTok{  literaturverzeichnis: bibliography.with("literaturverzeichnis.bib")}
\NormalTok{)}

\NormalTok{// Your content goes below.}
\end{Highlighting}
\end{Shaded}

\subsection{Provided Functions}\label{provided-functions}

Beside the \texttt{\ diplomarbeit\ } function, the template also
provides the \texttt{\ autor\ } function that is used after a heading to
indicate the specific author of the current section.

\begin{verbatim}
== Some Heading
#autor[Your Name]
\end{verbatim}

This will render additional information to the section’s heading.

To install the template locally, you can use

\begin{Shaded}
\begin{Highlighting}[]
\ExtensionTok{just}\NormalTok{ install}
\end{Highlighting}
\end{Shaded}

which uses the \href{https://github.com/casey/just}{just} command
runner.

If you don’t want to install \texttt{\ just\ } , you can run

\begin{Shaded}
\begin{Highlighting}[]
\FunctionTok{bash}\NormalTok{ ./scripts/package @local}
\end{Highlighting}
\end{Shaded}

The installed version can be used via \texttt{\ @local\ } instead of
\texttt{\ @preview\ } . To create a new typst project from the template,
run

\begin{Shaded}
\begin{Highlighting}[]
\ExtensionTok{typst}\NormalTok{ init @local/htlwienwest{-}da:}
\end{Highlighting}
\end{Shaded}

\href{/app?template=htlwienwest-da&version=0.1.0}{Create project in app}

\subsubsection{How to use}\label{how-to-use}

Click the button above to create a new project using this template in
the Typst app.

You can also use the Typst CLI to start a new project on your computer
using this command:

\begin{verbatim}
typst init @preview/htlwienwest-da:0.1.0
\end{verbatim}

\includesvg[width=0.16667in,height=0.16667in]{/assets/icons/16-copy.svg}

\subsubsection{About}\label{about}

\begin{description}
\tightlist
\item[Author s :]
\href{https://github.com/jozott00}{Johannes Zottele} \&
\href{https://github.com/peterw16}{peterw16}
\item[License:]
MIT
\item[Current version:]
0.1.0
\item[Last updated:]
May 3, 2024
\item[First released:]
May 3, 2024
\item[Archive size:]
61.9 kB
\href{https://packages.typst.org/preview/htlwienwest-da-0.1.0.tar.gz}{\pandocbounded{\includesvg[keepaspectratio]{/assets/icons/16-download.svg}}}
\item[Verification:]
We verified that the author is affiliated with their institution
\pandocbounded{\includesvg[keepaspectratio]{/assets/icons/16-verified.svg}}
\item[Repository:]
\href{https://github.com/htlwienwest/da-vorlage-typst}{GitHub}
\item[Categor y :]
\begin{itemize}
\tightlist
\item[]
\item
  \pandocbounded{\includesvg[keepaspectratio]{/assets/icons/16-mortarboard.svg}}
  \href{https://typst.app/universe/search/?category=thesis}{Thesis}
\end{itemize}
\end{description}

\subsubsection{Where to report issues?}\label{where-to-report-issues}

This template is a project of Johannes Zottele and peterw16 . Report
issues on \href{https://github.com/htlwienwest/da-vorlage-typst}{their
repository} . You can also try to ask for help with this template on the
\href{https://forum.typst.app}{Forum} .

Please report this template to the Typst team using the
\href{https://typst.app/contact}{contact form} if you believe it is a
safety hazard or infringes upon your rights.

\phantomsection\label{versions}
\subsubsection{Version history}\label{version-history}

\begin{longtable}[]{@{}ll@{}}
\toprule\noalign{}
Version & Release Date \\
\midrule\noalign{}
\endhead
\bottomrule\noalign{}
\endlastfoot
0.1.0 & May 3, 2024 \\
\end{longtable}

Typst GmbH did not create this template and cannot guarantee correct
functionality of this template or compatibility with any version of the
Typst compiler or app.


\section{Package List LaTeX/athena-tu-darmstadt-exercise.tex}
\title{typst.app/universe/package/athena-tu-darmstadt-exercise}

\phantomsection\label{banner}
\phantomsection\label{template-thumbnail}
\pandocbounded{\includegraphics[keepaspectratio]{https://packages.typst.org/preview/thumbnails/athena-tu-darmstadt-exercise-0.1.0-small.webp}}

\section{athena-tu-darmstadt-exercise}\label{athena-tu-darmstadt-exercise}

{ 0.1.0 }

Exercise template for TU Darmstadt (Technische Universität Darmstadt).

\href{/app?template=athena-tu-darmstadt-exercise&version=0.1.0}{Create
project in app}

\phantomsection\label{readme}
These \textbf{unofficial} templates can be used to write in
\href{https://github.com/typst/typst}{Typst} with the corporate design
of \href{https://www.tu-darmstadt.de/}{TU Darmstadt} .

\paragraph{Disclaimer}\label{disclaimer}

Please ask your supervisor if you are allowed to use typst and this
template for your thesis or other documents. Note that this template is
not checked by TU Darmstadt for correctness. Thus, this template does
not guarantee completeness or correctness. Also, note that submission in
TUbama requires PDF/A which typst currently can’t export to (
\url{https://github.com/typst/typst/issues/2942} ). You can use a
converter to convert from the typst output to PDF/A, but check that
there are no losses during the conversion. CMYK color space support may
be required for printing which is also currently not supported by typst
( \url{https://github.com/typst/typst/issues/2942} ), but this is not
relevant when you just submit online.

\subsection{Implemented Templates}\label{implemented-templates}

The templates imitate the style of the corresponding latex templates in
\href{https://github.com/tudace/tuda_latex_templates}{tuda\_latex\_templates}
. Note that there can be visual differences between the original latex
template and the typst template (you may open an issue when you find
one).

For missing features, ideas or other problems you can just open an issue
:wink:. Contributions are also welcome.

\begin{longtable}[]{@{}
  >{\raggedright\arraybackslash}p{(\linewidth - 6\tabcolsep) * \real{0.2500}}
  >{\raggedright\arraybackslash}p{(\linewidth - 6\tabcolsep) * \real{0.2500}}
  >{\raggedright\arraybackslash}p{(\linewidth - 6\tabcolsep) * \real{0.2500}}
  >{\raggedright\arraybackslash}p{(\linewidth - 6\tabcolsep) * \real{0.2500}}@{}}
\toprule\noalign{}
\begin{minipage}[b]{\linewidth}\raggedright
Template
\end{minipage} & \begin{minipage}[b]{\linewidth}\raggedright
Preview
\end{minipage} & \begin{minipage}[b]{\linewidth}\raggedright
Example
\end{minipage} & \begin{minipage}[b]{\linewidth}\raggedright
Scope
\end{minipage} \\
\midrule\noalign{}
\endhead
\bottomrule\noalign{}
\endlastfoot
\href{https://github.com/JeyRunner/tuda-typst-templates/blob/main/templates/tudapub/template/tudapub.typ}{tudapub}
&
\includegraphics[width=\linewidth,height=3.125in,keepaspectratio]{https://raw.githubusercontent.com/JeyRunner/tuda-typst-templates/refs/heads/main/templates/tudapub/preview/tudapub_prev-01.png}
& \begin{minipage}[t]{\linewidth}\raggedright
\href{https://github.com/JeyRunner/tuda-typst-templates/blob/main/example_tudapub.pdf}{example\_tudapub.pdf}\\
\href{https://github.com/JeyRunner/tuda-typst-templates/blob/main/example_tudapub.typ}{example\_tudapub.typ}\strut
\end{minipage} & Master and Bachelor thesis \\
\href{https://github.com/JeyRunner/tuda-typst-templates/blob/main/templates/tudaexercise/template/tudaexercise.typ}{tudaexercise}
&
\includegraphics[width=\linewidth,height=3.125in,keepaspectratio]{https://raw.githubusercontent.com/JeyRunner/tuda-typst-templates/refs/heads/main/templates/tudaexercise/preview/tudaexercise_prev-1.png}
&
\href{https://github.com/JeyRunner/tuda-typst-templates/blob/main/templates_examples/tudaexercise/main.typ}{Example
File} & Exercises \\
\end{longtable}

\subsection{Usage}\label{usage}

Create a new typst project based on this template locally.

\begin{Shaded}
\begin{Highlighting}[]
\CommentTok{\# for tudapub}
\ExtensionTok{typst}\NormalTok{ init @preview/athena{-}tu{-}darmstadt{-}thesis}
\BuiltInTok{cd}\NormalTok{ athena{-}tu{-}darmstadt{-}thesis}

\CommentTok{\# for tudaexercise}
\ExtensionTok{typst}\NormalTok{ init @preview/athena{-}tu{-}darmstadt{-}exercise}
\BuiltInTok{cd}\NormalTok{ athena{-}tu{-}darmstadt{-}exercise}
\end{Highlighting}
\end{Shaded}

Or create a project on the typst web app based on this template.

Or do a manual installation of this template.

For a manual setup create a folder for your writing project and download
this template into the `templates` folder:

\begin{Shaded}
\begin{Highlighting}[]
\FunctionTok{mkdir}\NormalTok{ my\_exercise }\KeywordTok{\&\&} \BuiltInTok{cd}\NormalTok{ my\_exercise}
\FunctionTok{git}\NormalTok{ clone https://github.com/JeyRunner/tuda{-}typst{-}templates}
\end{Highlighting}
\end{Shaded}

\subsubsection{Logo and Font Setup}\label{logo-and-font-setup}

Download the tud logo from
\href{https://download.hrz.tu-darmstadt.de/protected/ULB/tuda_logo.pdf}{download.hrz.tu-darmstadt.de/protected/ULB/tuda\_logo.pdf}
and put it into the \texttt{\ asssets/logos\ } folder. Now execute the
following script in the \texttt{\ asssets/logos\ } folder to convert it
into an svg:

\begin{Shaded}
\begin{Highlighting}[]
\BuiltInTok{cd}\NormalTok{ asssets/logos}
\ExtensionTok{./convert\_logo.sh}
\end{Highlighting}
\end{Shaded}

Note: The here used \texttt{\ pdf2svg\ } command might not be available.
In this case we recommend a online converter like
\href{https://tools.pdf24.org/en/pdf-to-svg}{PDF24 Tools} . There also
is a \href{https://github.com/FussballAndy/typst-img-to-local}{tool} to
install images as local typst packages.

Also download the required fonts \texttt{\ Roboto\ } and
\texttt{\ XCharter\ } :

\begin{Shaded}
\begin{Highlighting}[]
\BuiltInTok{cd}\NormalTok{ asssets/fonts}
\ExtensionTok{./download\_fonts.sh}
\end{Highlighting}
\end{Shaded}

Optionally you can install all fonts in the folders in
\texttt{\ fonts\ } on your system. But you can also use Typst’s
\texttt{\ -\/-font-path\ } option. Or install them in a folder and add
the folder to \texttt{\ TYPST\_FONT\_PATHS\ } for a single font folder.

Note: wget might not be available. In this case either download it or
replace the command with something like
\texttt{\ curl\ \textless{}url\textgreater{}\ -o\ \textless{}filename\textgreater{}\ -L\ }

Create a main.typ file for the manual template installation.

Create a simple `main.typ` in the root folder (`my\_exercise`) of your
new project:

\begin{Shaded}
\begin{Highlighting}[]
\NormalTok{\#import "tuda{-}typst{-}templates/templates/tudaexercise/template/lib.typ": *}

\NormalTok{\#show: tudaexercise.with(}
\NormalTok{  info: (}
\NormalTok{    title: "My Exercise",}
\NormalTok{    auhtor: "Your name",}
\NormalTok{    sheetnumber: 1    }
\NormalTok{  ),}
\NormalTok{  logo: image("tuda{-}typst{-}templates/assets/logos/tuda\_logo.svg")}
\NormalTok{)}

\NormalTok{= My First Task}
\NormalTok{Some Text}
\end{Highlighting}
\end{Shaded}

\subsubsection{Compile you typst file}\label{compile-you-typst-file}

\begin{Shaded}
\begin{Highlighting}[]
\ExtensionTok{typst} \AttributeTok{{-}{-}watch}\NormalTok{ main.typ }\AttributeTok{{-}{-}font{-}path}\NormalTok{ asssets/fonts/}
\end{Highlighting}
\end{Shaded}

This will watch your file and recompile it to a pdf when the file is
saved. For writing, you can use
\href{https://code.visualstudio.com/}{Vscode} with these extensions:
\href{https://marketplace.visualstudio.com/items?itemName=myriad-dreamin.tinymist}{Tinymist
Typst} . Or use the \href{https://typst.app/}{typst web app} (here you
need to upload the logo and the fonts).

Note that we add \texttt{\ -\/-font-path\ } to ensure that the correct
fonts are used. Due to a bug (typst/typst\#2917 typst/typst\#2098) typst
sometimes uses the font \texttt{\ Roboto\ condensed\ } instead of
\texttt{\ Roboto\ } . To be on the safe side, double-check the embedded
fonts in the pdf (there should be no \texttt{\ Roboto\ condensed\ } ).
What also works is to uninstall/deactivate all
\texttt{\ Roboto\ condensed\ } fonts from your system.

\subsection{Todos}\label{todos}

\begin{itemize}
\tightlist
\item
  \href{https://github.com/JeyRunner/tuda-typst-templates/blob/main/templates/tudapub/TODO.md}{todos
  of thesis template}
\end{itemize}

\href{/app?template=athena-tu-darmstadt-exercise&version=0.1.0}{Create
project in app}

\subsubsection{How to use}\label{how-to-use}

Click the button above to create a new project using this template in
the Typst app.

You can also use the Typst CLI to start a new project on your computer
using this command:

\begin{verbatim}
typst init @preview/athena-tu-darmstadt-exercise:0.1.0
\end{verbatim}

\includesvg[width=0.16667in,height=0.16667in]{/assets/icons/16-copy.svg}

\subsubsection{About}\label{about}

\begin{description}
\tightlist
\item[Author s :]
\href{https://github.com/JeyRunner}{JeyRunner} \&
\href{https://github.com/FussballAndy}{FussballAndy}
\item[License:]
MIT
\item[Current version:]
0.1.0
\item[Last updated:]
November 25, 2024
\item[First released:]
November 25, 2024
\item[Minimum Typst version:]
0.12.0
\item[Archive size:]
10.9 kB
\href{https://packages.typst.org/preview/athena-tu-darmstadt-exercise-0.1.0.tar.gz}{\pandocbounded{\includesvg[keepaspectratio]{/assets/icons/16-download.svg}}}
\item[Repository:]
\href{https://github.com/JeyRunner/tuda-typst-templates}{GitHub}
\item[Categor y :]
\begin{itemize}
\tightlist
\item[]
\item
  \pandocbounded{\includesvg[keepaspectratio]{/assets/icons/16-layout.svg}}
  \href{https://typst.app/universe/search/?category=layout}{Layout}
\end{itemize}
\end{description}

\subsubsection{Where to report issues?}\label{where-to-report-issues}

This template is a project of JeyRunner and FussballAndy . Report issues
on \href{https://github.com/JeyRunner/tuda-typst-templates}{their
repository} . You can also try to ask for help with this template on the
\href{https://forum.typst.app}{Forum} .

Please report this template to the Typst team using the
\href{https://typst.app/contact}{contact form} if you believe it is a
safety hazard or infringes upon your rights.

\phantomsection\label{versions}
\subsubsection{Version history}\label{version-history}

\begin{longtable}[]{@{}ll@{}}
\toprule\noalign{}
Version & Release Date \\
\midrule\noalign{}
\endhead
\bottomrule\noalign{}
\endlastfoot
0.1.0 & November 25, 2024 \\
\end{longtable}

Typst GmbH did not create this template and cannot guarantee correct
functionality of this template or compatibility with any version of the
Typst compiler or app.


\section{Package List LaTeX/meppp.tex}
\title{typst.app/universe/package/meppp}

\phantomsection\label{banner}
\phantomsection\label{template-thumbnail}
\pandocbounded{\includegraphics[keepaspectratio]{https://packages.typst.org/preview/thumbnails/meppp-0.2.1-small.webp}}

\section{meppp}\label{meppp}

{ 0.2.1 }

Template for modern physics experiment reports at the Physics School of
PKU.

\href{/app?template=meppp&version=0.2.1}{Create project in app}

\phantomsection\label{readme}
A simple template for modern physics experiments (MPE) courses at the
Physics School of PKU.

\subsection{meppp-lab-report}\label{meppp-lab-report}

The recommended report format of MPE course. Default arguments are shown
as below:

\begin{Shaded}
\begin{Highlighting}[]
\NormalTok{\#import "@preview/meppp:0.2.1": *}

\NormalTok{\#let meppp{-}lab{-}report(}
\NormalTok{  title: "",}
\NormalTok{  author: "",}
\NormalTok{  info: [],}
\NormalTok{  abstract: [],}
\NormalTok{  keywords: (),}
\NormalTok{  author{-}footnote: [],}
\NormalTok{  heading{-}numbering{-}array: ("I" ,"A", "1", "a"),}
\NormalTok{  heading{-}suffix: ". ",}
\NormalTok{  doc,}
\NormalTok{) = ...}
\end{Highlighting}
\end{Shaded}

\begin{itemize}
\tightlist
\item
  \texttt{\ title\ } is the title of the report.
\item
  \texttt{\ author\ } is the name of the author.
\item
  \texttt{\ info\ } is a line (or lines) of brief information of author
  and the report (e.g. student ID, school, experiment date…)
\item
  \texttt{\ abstract\ } is the abstract of the report, not shown when it
  is empty.
\item
  \texttt{\ keywords\ } are keywords of the report, only shown when the
  abstract is shown.
\item
  \texttt{\ author-footnote\ } is the phone number or the e-mail of the
  author, shown in the footnote.
\item
  \texttt{\ heading-numbering-array\ } is the heading numbering of each
  level. Only shows the numbering of the deepest level.
\item
  \texttt{\ heading-suffix\ } is the suffix of headings
\end{itemize}

It is recommended to use \texttt{\ \#show\ } to use the template:

\begin{Shaded}
\begin{Highlighting}[]
\NormalTok{\#show: meppp{-}lab{-}report.with(}
\NormalTok{    title: [Test title],}
\NormalTok{    ..args}
\NormalTok{)}
\NormalTok{...your report below.}
\end{Highlighting}
\end{Shaded}

\subsection{meppp-tl-table}\label{meppp-tl-table}

Modify your input \texttt{\ table\ } to a three-lined table (AIP style),
returned as a \texttt{\ figure\ } . Double-lines above and below the
table, and a single line below the header.

\begin{Shaded}
\begin{Highlighting}[]
\NormalTok{\#let meppp{-}tl{-}table(}
\NormalTok{  caption: none,}
\NormalTok{  supplement: auto,}
\NormalTok{  stroke: 0.5pt,}
\NormalTok{  tbl}
\NormalTok{) = ...}
\end{Highlighting}
\end{Shaded}

\begin{itemize}
\tightlist
\item
  \texttt{\ caption\ } is the caption above the table, center-aligned
\item
  \texttt{\ supplement\ } is same as the supplement in the figure.
\item
  \texttt{\ stroke\ } is the stroke used in the three lines (maybe five
  lines).
\item
  \texttt{\ tbl\ } is the input table, which must contains a
  \texttt{\ table.header\ }
\end{itemize}

Example:

\begin{Shaded}
\begin{Highlighting}[]
\NormalTok{\#meppp{-}tl{-}table(}
\NormalTok{  table(}
\NormalTok{    columns: 4,}
\NormalTok{    rows: 2,}
\NormalTok{    table.header([Item1], [Item2], [Item3], [Item4]),}
\NormalTok{    [Data1], [Data2], [Data3], [Data4],}
\NormalTok{  )}
\NormalTok{)}
\end{Highlighting}
\end{Shaded}

\subsection{subfigure}\label{subfigure}

Counts subfigures and displays in the figure, mostly used when inserting
multiple images.

\begin{Shaded}
\begin{Highlighting}[]
\NormalTok{\#let subfigure(}
\NormalTok{  body,}
\NormalTok{  caption: none,}
\NormalTok{  numbering: "(a)",}
\NormalTok{  inside: true,}
\NormalTok{  dx: 10pt,}
\NormalTok{  dy: 10pt,}
\NormalTok{  boxargs: (fill: white, inset: 5pt),}
\NormalTok{  alignment: top + left,}
\NormalTok{) = ...}
\end{Highlighting}
\end{Shaded}

\subsection{pku-logo}\label{pku-logo}

The logo of PKU, returned as a \texttt{\ image\ }

\begin{Shaded}
\begin{Highlighting}[]
\NormalTok{\#let pku{-}logo(..args) = image("pkulogo.png", ..args)}
\end{Highlighting}
\end{Shaded}

Example:

\begin{Shaded}
\begin{Highlighting}[]
\NormalTok{\#pku{-}logo(width: 50\%)}
\NormalTok{\#pku{-}logo()}
\end{Highlighting}
\end{Shaded}

\href{/app?template=meppp&version=0.2.1}{Create project in app}

\subsubsection{How to use}\label{how-to-use}

Click the button above to create a new project using this template in
the Typst app.

You can also use the Typst CLI to start a new project on your computer
using this command:

\begin{verbatim}
typst init @preview/meppp:0.2.1
\end{verbatim}

\includesvg[width=0.16667in,height=0.16667in]{/assets/icons/16-copy.svg}

\subsubsection{About}\label{about}

\begin{description}
\tightlist
\item[Author :]
\href{https://github.com/CL4R3T}{CL4R3T}
\item[License:]
MIT
\item[Current version:]
0.2.1
\item[Last updated:]
September 22, 2024
\item[First released:]
May 8, 2024
\item[Archive size:]
103 kB
\href{https://packages.typst.org/preview/meppp-0.2.1.tar.gz}{\pandocbounded{\includesvg[keepaspectratio]{/assets/icons/16-download.svg}}}
\item[Repository:]
\href{https://github.com/pku-typst/meppp}{GitHub}
\item[Categor y :]
\begin{itemize}
\tightlist
\item[]
\item
  \pandocbounded{\includesvg[keepaspectratio]{/assets/icons/16-speak.svg}}
  \href{https://typst.app/universe/search/?category=report}{Report}
\end{itemize}
\end{description}

\subsubsection{Where to report issues?}\label{where-to-report-issues}

This template is a project of CL4R3T . Report issues on
\href{https://github.com/pku-typst/meppp}{their repository} . You can
also try to ask for help with this template on the
\href{https://forum.typst.app}{Forum} .

Please report this template to the Typst team using the
\href{https://typst.app/contact}{contact form} if you believe it is a
safety hazard or infringes upon your rights.

\phantomsection\label{versions}
\subsubsection{Version history}\label{version-history}

\begin{longtable}[]{@{}ll@{}}
\toprule\noalign{}
Version & Release Date \\
\midrule\noalign{}
\endhead
\bottomrule\noalign{}
\endlastfoot
0.2.1 & September 22, 2024 \\
\href{https://typst.app/universe/package/meppp/0.2.0/}{0.2.0} &
September 14, 2024 \\
\href{https://typst.app/universe/package/meppp/0.1.0/}{0.1.0} & May 8,
2024 \\
\end{longtable}

Typst GmbH did not create this template and cannot guarantee correct
functionality of this template or compatibility with any version of the
Typst compiler or app.


\section{Package List LaTeX/tgm-hit-thesis.tex}
\title{typst.app/universe/package/tgm-hit-thesis}

\phantomsection\label{banner}
\phantomsection\label{template-thumbnail}
\pandocbounded{\includegraphics[keepaspectratio]{https://packages.typst.org/preview/thumbnails/tgm-hit-thesis-0.2.0-small.webp}}

\section{tgm-hit-thesis}\label{tgm-hit-thesis}

{ 0.2.0 }

Diploma thesis template for students of the HIT department at TGM Wien

{ } Officially affiliated

\href{/app?template=tgm-hit-thesis&version=0.2.0}{Create project in app}

\phantomsection\label{readme}
This is a port of the
\href{https://github.com/TGM-HIT/diploma-thesis}{LaTeX diploma thesis
template} available for students of the information technology
department at the TGM technical secondary school in Vienna.

\subsection{Getting Started}\label{getting-started}

Using the Typst web app, you can create a project by e.g. using this
link: \url{https://typst.app/?template=tgm-hit-thesis&version=latest} .

To work locally, use the following command:

\begin{Shaded}
\begin{Highlighting}[]
\ExtensionTok{typst}\NormalTok{ init @preview/tgm{-}hit{-}thesis}
\end{Highlighting}
\end{Shaded}

\subsection{Usage}\label{usage}

The template (
\href{https://github.com/typst/packages/raw/main/packages/preview/tgm-hit-thesis/0.2.0/example.pdf}{rendered
PDF} ) contains thesis writing advice (in German) as example content. If
you are looking for the details of this template package’s function,
take a look at the
\href{https://github.com/typst/packages/raw/main/packages/preview/tgm-hit-thesis/0.2.0/docs/manual.pdf}{manual}
.

\href{/app?template=tgm-hit-thesis&version=0.2.0}{Create project in app}

\subsubsection{How to use}\label{how-to-use}

Click the button above to create a new project using this template in
the Typst app.

You can also use the Typst CLI to start a new project on your computer
using this command:

\begin{verbatim}
typst init @preview/tgm-hit-thesis:0.2.0
\end{verbatim}

\includesvg[width=0.16667in,height=0.16667in]{/assets/icons/16-copy.svg}

\subsubsection{About}\label{about}

\begin{description}
\tightlist
\item[Author :]
\href{https://github.com/SillyFreak/}{Clemens Koza}
\item[License:]
MIT
\item[Current version:]
0.2.0
\item[Last updated:]
October 24, 2024
\item[First released:]
July 15, 2024
\item[Minimum Typst version:]
0.11.0
\item[Archive size:]
86.8 kB
\href{https://packages.typst.org/preview/tgm-hit-thesis-0.2.0.tar.gz}{\pandocbounded{\includesvg[keepaspectratio]{/assets/icons/16-download.svg}}}
\item[Verification:]
We verified that the author is affiliated with their institution
\pandocbounded{\includesvg[keepaspectratio]{/assets/icons/16-verified.svg}}
\item[Repository:]
\href{https://github.com/TGM-HIT/typst-diploma-thesis}{GitHub}
\item[Discipline :]
\begin{itemize}
\tightlist
\item[]
\item
  \href{https://typst.app/universe/search/?discipline=computer-science}{Computer
  Science}
\end{itemize}
\item[Categor y :]
\begin{itemize}
\tightlist
\item[]
\item
  \pandocbounded{\includesvg[keepaspectratio]{/assets/icons/16-mortarboard.svg}}
  \href{https://typst.app/universe/search/?category=thesis}{Thesis}
\end{itemize}
\end{description}

\subsubsection{Where to report issues?}\label{where-to-report-issues}

This template is a project of Clemens Koza . Report issues on
\href{https://github.com/TGM-HIT/typst-diploma-thesis}{their repository}
. You can also try to ask for help with this template on the
\href{https://forum.typst.app}{Forum} .

Please report this template to the Typst team using the
\href{https://typst.app/contact}{contact form} if you believe it is a
safety hazard or infringes upon your rights.

\phantomsection\label{versions}
\subsubsection{Version history}\label{version-history}

\begin{longtable}[]{@{}ll@{}}
\toprule\noalign{}
Version & Release Date \\
\midrule\noalign{}
\endhead
\bottomrule\noalign{}
\endlastfoot
0.2.0 & October 24, 2024 \\
\href{https://typst.app/universe/package/tgm-hit-thesis/0.1.3/}{0.1.3} &
September 15, 2024 \\
\href{https://typst.app/universe/package/tgm-hit-thesis/0.1.2/}{0.1.2} &
September 14, 2024 \\
\href{https://typst.app/universe/package/tgm-hit-thesis/0.1.1/}{0.1.1} &
September 11, 2024 \\
\href{https://typst.app/universe/package/tgm-hit-thesis/0.1.0/}{0.1.0} &
July 15, 2024 \\
\end{longtable}

Typst GmbH did not create this template and cannot guarantee correct
functionality of this template or compatibility with any version of the
Typst compiler or app.


\section{Package List LaTeX/diagraph.tex}
\title{typst.app/universe/package/diagraph}

\phantomsection\label{banner}
\section{diagraph}\label{diagraph}

{ 0.3.0 }

Draw graphs with Graphviz. Use mathematical formulas as labels.

\phantomsection\label{readme}
A simple Graphviz binding for Typst using the WebAssembly plugin system.

\subsection{Usage}\label{usage}

\subsubsection{Basic usage}\label{basic-usage}

You can render a Graphviz Dot string to a SVG image using the
\texttt{\ render\ } function:

\begin{Shaded}
\begin{Highlighting}[]
\NormalTok{\#render("digraph \{ a {-}\textgreater{} b \}")}
\end{Highlighting}
\end{Shaded}

Alternatively, you can use \texttt{\ raw-render\ } to pass a
\texttt{\ raw\ } instead of a string:

\begin{Shaded}
\begin{Highlighting}[]
\NormalTok{\#raw{-}render(}
\NormalTok{  \textasciigrave{}\textasciigrave{}\textasciigrave{}dot}
\NormalTok{  digraph \{}
\NormalTok{    a {-}\textgreater{} b}
\NormalTok{  \}}
\NormalTok{  \textasciigrave{}\textasciigrave{}\textasciigrave{}}
\NormalTok{)}
\end{Highlighting}
\end{Shaded}

\pandocbounded{\includegraphics[keepaspectratio]{https://raw.githubusercontent.com/Robotechnic/diagraph/main/images/raw-render1.png}}

For more information about the Graphviz Dot language, you can check the
\href{https://graphviz.org/documentation/}{official documentation} .

\subsubsection{Advanced usage}\label{advanced-usage}

Check the
\href{https://raw.githubusercontent.com/Robotechnic/diagraph/main/doc/manual.pdf}{manual}
for more information about the plugin.

\subsection{License}\label{license}

This project is licensed under the MIT License - see the
\href{https://github.com/typst/packages/raw/main/packages/preview/diagraph/0.3.0/LICENSE}{LICENSE}
file for details

\subsection{Changelog}\label{changelog}

\subsubsection{0.3.0}\label{section}

\begin{itemize}
\tightlist
\item
  Added support for edge labels
\item
  Added a manual generated with Typst
\item
  Updated graphviz version
\item
  Fix an error in math mode detection
\end{itemize}

\subsubsection{0.2.5}\label{section-1}

\begin{itemize}
\tightlist
\item
  If the shape is point, the label isn’t displayed
\item
  Now a minimum size is not enforced if the node label is empty
\item
  Added support for font alternatives
\end{itemize}

\subsubsection{0.2.4}\label{section-2}

\begin{itemize}
\tightlist
\item
  Added support for xlabels which are now rendered by Typst
\item
  Added support for cluster labels which are now rendered by Typst
\item
  Fix a margin problem with the clusters
\end{itemize}

\subsubsection{0.2.3}\label{section-3}

\begin{itemize}
\tightlist
\item
  Updated to typst 0.11.0
\item
  Added support for \texttt{\ fontcolor\ } , \texttt{\ fontsize\ } and
  \texttt{\ fontname\ } nodes attributes
\item
  Diagraph now uses a protocol generator to generate the wasm interface
\end{itemize}

\subsubsection{0.2.2}\label{section-4}

\begin{itemize}
\tightlist
\item
  Fix an alignment issue
\item
  Added a better mathematic formula recognition for node labels
\end{itemize}

\subsubsection{0.2.1}\label{section-5}

\begin{itemize}
\tightlist
\item
  Added support for relative lenghts in the \texttt{\ width\ } and
  \texttt{\ height\ } arguments
\item
  Fix various bugs
\end{itemize}

\subsubsection{0.2.0}\label{section-6}

\begin{itemize}
\tightlist
\item
  Node labels are now handled by Typst
\end{itemize}

\subsubsection{0.1.2}\label{section-7}

\begin{itemize}
\tightlist
\item
  Graphs are now scaled to make the graph text size match the document
  text size
\end{itemize}

\subsubsection{0.1.1}\label{section-8}

\begin{itemize}
\tightlist
\item
  Remove the \texttt{\ raw-render-rule\ } show rule because it doesn’t
  allow use of custom font and the \texttt{\ render\ } /
  \texttt{\ raw-render\ } functions are more flexible
\item
  Add the \texttt{\ background\ } parameter to the \texttt{\ render\ }
  and \texttt{\ raw-render\ } typst functions and default it to
  \texttt{\ transparent\ } instead of \texttt{\ white\ }
\item
  Add center attribute to draw graph in the center of the svg in the
  \texttt{\ render\ } c function
\end{itemize}

\subsubsection{0.1.0}\label{section-9}

Initial working version

\subsubsection{How to add}\label{how-to-add}

Copy this into your project and use the import as \texttt{\ diagraph\ }

\begin{verbatim}
#import "@preview/diagraph:0.3.0"
\end{verbatim}

\includesvg[width=0.16667in,height=0.16667in]{/assets/icons/16-copy.svg}

Check the docs for
\href{https://typst.app/docs/reference/scripting/\#packages}{more
information on how to import packages} .

\subsubsection{About}\label{about}

\begin{description}
\tightlist
\item[Author s :]
\href{https://github.com/Robotechnic}{Robotechnic} \&
\href{https://github.com/MDLC01}{Malo}
\item[License:]
MIT
\item[Current version:]
0.3.0
\item[Last updated:]
September 3, 2024
\item[First released:]
September 23, 2023
\item[Minimum Typst version:]
0.11.0
\item[Archive size:]
450 kB
\href{https://packages.typst.org/preview/diagraph-0.3.0.tar.gz}{\pandocbounded{\includesvg[keepaspectratio]{/assets/icons/16-download.svg}}}
\item[Repository:]
\href{https://github.com/Robotechnic/diagraph.git}{GitHub}
\item[Categor ies :]
\begin{itemize}
\tightlist
\item[]
\item
  \pandocbounded{\includesvg[keepaspectratio]{/assets/icons/16-package.svg}}
  \href{https://typst.app/universe/search/?category=components}{Components}
\item
  \pandocbounded{\includesvg[keepaspectratio]{/assets/icons/16-chart.svg}}
  \href{https://typst.app/universe/search/?category=visualization}{Visualization}
\item
  \pandocbounded{\includesvg[keepaspectratio]{/assets/icons/16-integration.svg}}
  \href{https://typst.app/universe/search/?category=integration}{Integration}
\end{itemize}
\end{description}

\subsubsection{Where to report issues?}\label{where-to-report-issues}

This package is a project of Robotechnic and Malo . Report issues on
\href{https://github.com/Robotechnic/diagraph.git}{their repository} .
You can also try to ask for help with this package on the
\href{https://forum.typst.app}{Forum} .

Please report this package to the Typst team using the
\href{https://typst.app/contact}{contact form} if you believe it is a
safety hazard or infringes upon your rights.

\phantomsection\label{versions}
\subsubsection{Version history}\label{version-history}

\begin{longtable}[]{@{}ll@{}}
\toprule\noalign{}
Version & Release Date \\
\midrule\noalign{}
\endhead
\bottomrule\noalign{}
\endlastfoot
0.3.0 & September 3, 2024 \\
\href{https://typst.app/universe/package/diagraph/0.2.5/}{0.2.5} & June
11, 2024 \\
\href{https://typst.app/universe/package/diagraph/0.2.4/}{0.2.4} & May
23, 2024 \\
\href{https://typst.app/universe/package/diagraph/0.2.3/}{0.2.3} & May
13, 2024 \\
\href{https://typst.app/universe/package/diagraph/0.2.2/}{0.2.2} & March
15, 2024 \\
\href{https://typst.app/universe/package/diagraph/0.2.1/}{0.2.1} &
January 16, 2024 \\
\href{https://typst.app/universe/package/diagraph/0.2.0/}{0.2.0} &
November 18, 2023 \\
\href{https://typst.app/universe/package/diagraph/0.1.2/}{0.1.2} &
November 6, 2023 \\
\href{https://typst.app/universe/package/diagraph/0.1.1/}{0.1.1} &
September 28, 2023 \\
\href{https://typst.app/universe/package/diagraph/0.1.0/}{0.1.0} &
September 23, 2023 \\
\end{longtable}

Typst GmbH did not create this package and cannot guarantee correct
functionality of this package or compatibility with any version of the
Typst compiler or app.


\section{Package List LaTeX/board-n-pieces.tex}
\title{typst.app/universe/package/board-n-pieces}

\phantomsection\label{banner}
\section{board-n-pieces}\label{board-n-pieces}

{ 0.5.0 }

Display chessboards.

\phantomsection\label{readme}
Display chessboards in Typst.

\subsection{Displaying chessboards}\label{displaying-chessboards}

The main function of this package is \texttt{\ board\ } . It lets you
display a specific position on a board.

\begin{Shaded}
\begin{Highlighting}[]
\NormalTok{\#board(starting{-}position)}
\end{Highlighting}
\end{Shaded}

\pandocbounded{\includesvg[keepaspectratio]{https://github.com/typst/packages/raw/main/packages/preview/board-n-pieces/0.5.0/examples/example-1.svg}}

\texttt{\ starting-position\ } is a position that is provided by the
package. It represents the initial position of a chess game.

You can create a different position using the \texttt{\ position\ }
function. It accepts strings representing each rank. Use upper-case
letters for white pieces, and lower-case letters for black pieces. Dots
and spaces correspond to empty squares.

\begin{Shaded}
\begin{Highlighting}[]
\NormalTok{\#board(position(}
\NormalTok{  "....r...",}
\NormalTok{  "........",}
\NormalTok{  "..p..PPk",}
\NormalTok{  ".p.r....",}
\NormalTok{  "pP..p.R.",}
\NormalTok{  "P.B.....",}
\NormalTok{  "..P..K..",}
\NormalTok{  "........",}
\NormalTok{))}
\end{Highlighting}
\end{Shaded}

\pandocbounded{\includesvg[keepaspectratio]{https://github.com/typst/packages/raw/main/packages/preview/board-n-pieces/0.5.0/examples/example-2.svg}}

Alternatively, you can use the \texttt{\ fen\ } function to create a
position using
\href{https://en.wikipedia.org/wiki/Forsyth\%E2\%80\%93Edwards_Notation}{Forsythâ€``Edwards
notation} :

\begin{Shaded}
\begin{Highlighting}[]
\NormalTok{\#board(fen("r1bk3r/p2pBpNp/n4n2/1p1NP2P/6P1/3P4/P1P1K3/q5b1 b {-} {-} 1 23"))}
\end{Highlighting}
\end{Shaded}

\pandocbounded{\includesvg[keepaspectratio]{https://github.com/typst/packages/raw/main/packages/preview/board-n-pieces/0.5.0/examples/example-3.svg}}

Note that you can specify only the first part of the FEN string:

\begin{Shaded}
\begin{Highlighting}[]
\NormalTok{\#board(fen("r4rk1/pp2Bpbp/1qp3p1/8/2BP2b1/Q1n2N2/P4PPP/3RK2R"))}
\end{Highlighting}
\end{Shaded}

\pandocbounded{\includesvg[keepaspectratio]{https://github.com/typst/packages/raw/main/packages/preview/board-n-pieces/0.5.0/examples/example-4.svg}}

Also note that positions do not need to be on a standard 8Ã---8 board:

\begin{Shaded}
\begin{Highlighting}[]
\NormalTok{\#board(position(}
\NormalTok{  "....Q....",}
\NormalTok{  "......Q..",}
\NormalTok{  "........Q",}
\NormalTok{  "...Q.....",}
\NormalTok{  ".Q.......",}
\NormalTok{  ".......Q.",}
\NormalTok{  ".....Q...",}
\NormalTok{  "..Q......",}
\NormalTok{  "Q........",}
\NormalTok{))}
\end{Highlighting}
\end{Shaded}

\pandocbounded{\includesvg[keepaspectratio]{https://github.com/typst/packages/raw/main/packages/preview/board-n-pieces/0.5.0/examples/example-5.svg}}

\subsection{\texorpdfstring{Using the \texttt{\ game\ }
function}{Using the  game  function}}\label{using-the-game-function}

The \texttt{\ game\ } function creates an array of positions from a full
chess game. A game is described by a series of turns written in
\href{https://en.wikipedia.org/wiki/Algebraic_notation_(chess)}{standard
algebraic notation} . Those turns can be specified as an array of
strings, or as a single string containing whitespace-separated moves.

\begin{Shaded}
\begin{Highlighting}[]
\NormalTok{The scholar\textquotesingle{}s mate:}
\NormalTok{\#let positions = game("e4 e5 Qh5 Nc6 Bc4 Nf6 Qxf7")}
\NormalTok{\#grid(}
\NormalTok{  columns: 4,}
\NormalTok{  gutter: 0.2cm,}
\NormalTok{  ..positions.map(board.with(square{-}size: 0.5cm)),}
\NormalTok{)}
\end{Highlighting}
\end{Shaded}

\pandocbounded{\includesvg[keepaspectratio]{https://github.com/typst/packages/raw/main/packages/preview/board-n-pieces/0.5.0/examples/example-6.svg}}

You can specify an alternative starting position to the
\texttt{\ game\ } function with the \texttt{\ starting-position\ } named
argument.

\subsection{\texorpdfstring{Using the \texttt{\ pgn\ } function to
import PGN
files}{Using the  pgn  function to import PGN files}}\label{using-the-pgn-function-to-import-pgn-files}

Similarly to the \texttt{\ game\ } function, the \texttt{\ pgn\ }
function creates an array of positions. It accepts a single argument,
which is a string containing
\href{https://en.wikipedia.org/wiki/Portable_Game_Notation}{portable
game notation} . To read a game from a PGN file, you can use this
function in combination with Typst’s native
\href{https://typst.app/docs/reference/data-loading/read/}{\texttt{\ read\ }}
function.

\begin{Shaded}
\begin{Highlighting}[]
\NormalTok{\#let positions = pgn(read("game.pgn"))}
\end{Highlighting}
\end{Shaded}

Note that the argument to \texttt{\ pgn\ } must describe a single game.
If you have a PGN file containing multiple games, you will need to split
them using other means.

\subsection{Using non-standard chess
pieces}\label{using-non-standard-chess-pieces}

The \texttt{\ board\ } function’s \texttt{\ pieces\ } argument lets
you specify how to display each piece by mapping each piece character to
some content. You can use this feature to display non-standard chess
pieces:

\begin{Shaded}
\begin{Highlighting}[]
\NormalTok{\#board(}
\NormalTok{  fen("g7/5g2/8/8/8/8/p6g/k1K4G"),}
\NormalTok{  pieces: (}
\NormalTok{    // We use symbols for the example.}
\NormalTok{    // In practice, you should import your own images.}
\NormalTok{    g: chess{-}sym.queen.black.b,}
\NormalTok{    p: chess{-}sym.pawn.black,}
\NormalTok{    k: chess{-}sym.king.black,}
\NormalTok{    K: chess{-}sym.king.white,}
\NormalTok{    G: chess{-}sym.queen.white.b,}
\NormalTok{  ),}
\NormalTok{)}
\end{Highlighting}
\end{Shaded}

\pandocbounded{\includesvg[keepaspectratio]{https://github.com/typst/packages/raw/main/packages/preview/board-n-pieces/0.5.0/examples/example-7.svg}}

\subsection{Customizing a chessboard}\label{customizing-a-chessboard}

The \texttt{\ board\ } function lets you customize the appearance of the
board in various ways, as illustrated in the example below.

\begin{Shaded}
\begin{Highlighting}[]
\NormalTok{// From https://lichess.org/study/Xf1PGrM0.}
\NormalTok{\#board(}
\NormalTok{  fen("3k4/7R/8/2PK4/8/8/8/6r1 b {-} {-} 0 1"),}

\NormalTok{  marked{-}squares: "c7 c6 h6",}
\NormalTok{  arrows: ("d8 c8", "d8 c7", "g1 g6", "h7 h6"),}
\NormalTok{  display{-}numbers: true,}

\NormalTok{  white{-}square{-}fill: rgb("\#d2eeea"),}
\NormalTok{  black{-}square{-}fill: rgb("\#567f96"),}
\NormalTok{  marking{-}color: rgb("\#2bcbC6"),}
\NormalTok{  arrow{-}stroke: 0.2cm + rgb("\#38f442df"),}

\NormalTok{  stroke: 0.8pt + black,}
\NormalTok{)}
\end{Highlighting}
\end{Shaded}

\pandocbounded{\includesvg[keepaspectratio]{https://github.com/typst/packages/raw/main/packages/preview/board-n-pieces/0.5.0/examples/example-8.svg}}

Here is a list of all the available arguments:

\begin{itemize}
\item
  \texttt{\ marked-squares\ } is a list of squares to mark (e.g.,
  \texttt{\ ("d3",\ "d2",\ "e3")\ } ). It can also be specified as a
  single string containing whitespace-separated squares (e.g.,
  \texttt{\ "d3\ d2\ e3"\ } ).
\item
  \texttt{\ arrows\ } is a list of arrows to draw (e.g.,
  \texttt{\ ("e2\ e4",\ "e7\ e5")\ } ).
\item
  \texttt{\ reverse\ } is a boolean indicating whether to reverse the
  board, displaying it from black’s point of view. This is
  \texttt{\ false\ } by default, meaning the board is displayed from
  white’s point of view.
\item
  \texttt{\ display-numbers\ } is a boolean indicating whether ranks and
  files should be numbered. This is \texttt{\ false\ } by default.
\item
  \texttt{\ rank-numbering\ } and \texttt{\ file-numbering\ } are
  functions describing how ranks and files should be numbered. By
  default they are respectively \texttt{\ numbering.with("1")\ } and
  \texttt{\ numbering.with("a")\ } .
\item
  \texttt{\ square-size\ } is a length describing the size of each
  square. By default, this is \texttt{\ 1cm\ } .
\item
  \texttt{\ white-square-fill\ } and \texttt{\ black-square-fill\ }
  indicate how squares should be filled. They can be colors, gradient or
  patterns.
\item
  \texttt{\ marking-color\ } is the color to use for markings (marked
  squares and arrows).
\item
  \texttt{\ marked-white-square-background\ } and
  \texttt{\ marked-black-square-background\ } define the content to
  display in the background of marked squares. By default, this is a
  circle using the \texttt{\ marking-color\ } .
\item
  \texttt{\ arrow-stroke\ } is the stroke to draw the arrows with. If
  only a length is given, \texttt{\ marking-color\ } is used.
  Alternatively, a stroke can be passed to specify a different color.
\item
  \texttt{\ pieces\ } is a dictionary containing images representing
  each piece. If specified, the dictionary must contain an entry for
  every piece kind in the displayed position. Keys are single upper-case
  letters for white pieces and single lower-case letters for black
  pieces. The default images are taken from
  \href{https://commons.wikimedia.org/wiki/Category:SVG_chess_pieces}{Wikimedia
  Commons} . Please refer to
  \href{https://github.com/typst/packages/raw/main/packages/preview/board-n-pieces/0.5.0/\#licensing}{the
  section on licensing} for information on how you can use them in your
  documents.
\item
  \texttt{\ stroke\ } has the same structure as
  \href{https://typst.app/docs/reference/visualize/rect/\#parameters-stroke}{\texttt{\ rect\ }
  ’s \texttt{\ stroke\ } parameter} and corresponds to the stroke to
  use around the board. If \texttt{\ display-numbers\ } is
  \texttt{\ true\ } , the numbers are displayed outside the stroke. The
  default value is \texttt{\ none\ } .
\end{itemize}

\subsection{Chess symbols}\label{chess-symbols}

This package also exports chess symbols for all Unicode chess-related
codepoints under the \texttt{\ chess-sym\ } submodule. Standard chess
pieces are available as
\texttt{\ chess-sym.\{pawn,knight,bishop,rook,queen,king\}.\{white,black,neutral\}\ }
. Alternatively, you can use \texttt{\ stroked\ } and
\texttt{\ filled\ } instead of, respectively, \texttt{\ white\ } and
\texttt{\ black\ } . They can be rotated rightward, downward, and
leftward respectively with with \texttt{\ .r\ } , \texttt{\ .b\ } , and
\texttt{\ .l\ } . Chinese chess pieces are also available as
\texttt{\ chess-sym.\{soldier,cannon,chariot,horse,elephant,mandarin,general\}.\{red,black\}\ }
. Similarly, you can use \texttt{\ stroked\ } and \texttt{\ filled\ } as
alternatives to, respectively, \texttt{\ red\ } and \texttt{\ black\ } .
Note that most fonts only support black and white versions of standard
pieces. To use the other symbols, you may have to use a font such as
Noto Sans Symbols 2.

\begin{Shaded}
\begin{Highlighting}[]
\NormalTok{The best move in this position is \#chess{-}sym.knight.white;c6.}
\end{Highlighting}
\end{Shaded}

\pandocbounded{\includesvg[keepaspectratio]{https://github.com/typst/packages/raw/main/packages/preview/board-n-pieces/0.5.0/examples/example-9.svg}}

\subsection{Licensing}\label{licensing}

The default images for chess pieces used by the \texttt{\ board\ }
function come from
\href{https://commons.wikimedia.org/wiki/Category:SVG_chess_pieces}{Wikimedia
Commons} . They are all licensed the
\href{https://www.gnu.org/licenses/old-licenses/gpl-2.0.html}{GNU
General Public License, version 2} by their original author:
\href{https://en.wikipedia.org/wiki/User:Cburnett}{Cburnett} .

\subsection{Changelog}\label{changelog}

\subsubsection{Version 0.5.0}\label{version-0.5.0}

\begin{itemize}
\item
  Add symbols for all Unicode chess-related codepoints.
\item
  Change the signature of the \texttt{\ board\ } function.

  \begin{itemize}
  \tightlist
  \item
    Rename argument \texttt{\ highlighted-squares\ } to
    \texttt{\ marked-squares\ } .
  \item
    Remove arguments \texttt{\ highlighted-white-square-fill\ } and
    \texttt{\ highlighted-black-square-fill\ } .
  \item
    Add argument \texttt{\ marking-color\ } , together with
    \texttt{\ marked-white-square-background\ } and
    \texttt{\ marked-black-square-background\ } .
  \item
    Support passing a length as \texttt{\ arrow-stroke\ } .
  \end{itemize}
\item
  Fix arrows not being displayed properly on reversed boards.
\end{itemize}

\subsubsection{Version 0.4.0}\label{version-0.4.0}

\begin{itemize}
\tightlist
\item
  Add the ability to draw arrows in \texttt{\ board\ } .
\end{itemize}

\subsubsection{Version 0.3.0}\label{version-0.3.0}

\begin{itemize}
\item
  Detect moves that put the king in check as illegal, improving SAN
  support.
\item
  Add \texttt{\ stroke\ } argument to the \texttt{\ board\ } function.
\item
  Rename \texttt{\ \{highlighted-,\}\{white,black\}-square-color\ }
  arguments to the \texttt{\ board\ } function to
  \texttt{\ \{highlighted-,\}\{white,black\}-square-fill\ } .
\end{itemize}

\subsubsection{Version 0.2.0}\label{version-0.2.0}

\begin{itemize}
\item
  Allow using dashes for empty squares in \texttt{\ position\ }
  function.
\item
  Allow passing highlighted squares as a single string of
  whitespace-separated squares.
\item
  Describe entire games using algebraic notation with the
  \texttt{\ game\ } function.
\item
  Initial PGN support through the \texttt{\ pgn\ } function.
\end{itemize}

\subsubsection{Version 0.1.0}\label{version-0.1.0}

\begin{itemize}
\item
  Display a chess position on a chessboard with the \texttt{\ board\ }
  function.
\item
  Get the starting position with \texttt{\ starting-position\ } .
\item
  Use chess-related symbols with the \texttt{\ chess-sym\ } module.
\end{itemize}

\subsubsection{How to add}\label{how-to-add}

Copy this into your project and use the import as
\texttt{\ board-n-pieces\ }

\begin{verbatim}
#import "@preview/board-n-pieces:0.5.0"
\end{verbatim}

\includesvg[width=0.16667in,height=0.16667in]{/assets/icons/16-copy.svg}

Check the docs for
\href{https://typst.app/docs/reference/scripting/\#packages}{more
information on how to import packages} .

\subsubsection{About}\label{about}

\begin{description}
\tightlist
\item[Author :]
\href{https://github.com/MDLC01}{Malo}
\item[License:]
MIT AND GPL-2.0-only
\item[Current version:]
0.5.0
\item[Last updated:]
July 22, 2024
\item[First released:]
March 20, 2024
\item[Minimum Typst version:]
0.11.0
\item[Archive size:]
52.2 kB
\href{https://packages.typst.org/preview/board-n-pieces-0.5.0.tar.gz}{\pandocbounded{\includesvg[keepaspectratio]{/assets/icons/16-download.svg}}}
\item[Repository:]
\href{https://github.com/MDLC01/board-n-pieces}{GitHub}
\item[Discipline s :]
\begin{itemize}
\tightlist
\item[]
\item
  \href{https://typst.app/universe/search/?discipline=computer-science}{Computer
  Science}
\item
  \href{https://typst.app/universe/search/?discipline=mathematics}{Mathematics}
\end{itemize}
\item[Categor y :]
\begin{itemize}
\tightlist
\item[]
\item
  \pandocbounded{\includesvg[keepaspectratio]{/assets/icons/16-chart.svg}}
  \href{https://typst.app/universe/search/?category=visualization}{Visualization}
\end{itemize}
\end{description}

\subsubsection{Where to report issues?}\label{where-to-report-issues}

This package is a project of Malo . Report issues on
\href{https://github.com/MDLC01/board-n-pieces}{their repository} . You
can also try to ask for help with this package on the
\href{https://forum.typst.app}{Forum} .

Please report this package to the Typst team using the
\href{https://typst.app/contact}{contact form} if you believe it is a
safety hazard or infringes upon your rights.

\phantomsection\label{versions}
\subsubsection{Version history}\label{version-history}

\begin{longtable}[]{@{}ll@{}}
\toprule\noalign{}
Version & Release Date \\
\midrule\noalign{}
\endhead
\bottomrule\noalign{}
\endlastfoot
0.5.0 & July 22, 2024 \\
\href{https://typst.app/universe/package/board-n-pieces/0.4.0/}{0.4.0} &
July 8, 2024 \\
\href{https://typst.app/universe/package/board-n-pieces/0.3.0/}{0.3.0} &
May 23, 2024 \\
\href{https://typst.app/universe/package/board-n-pieces/0.2.0/}{0.2.0} &
April 29, 2024 \\
\href{https://typst.app/universe/package/board-n-pieces/0.1.0/}{0.1.0} &
March 20, 2024 \\
\end{longtable}

Typst GmbH did not create this package and cannot guarantee correct
functionality of this package or compatibility with any version of the
Typst compiler or app.


\section{Package List LaTeX/light-report-uia.tex}
\title{typst.app/universe/package/light-report-uia}

\phantomsection\label{banner}
\phantomsection\label{template-thumbnail}
\pandocbounded{\includegraphics[keepaspectratio]{https://packages.typst.org/preview/thumbnails/light-report-uia-0.1.0-small.webp}}

\section{light-report-uia}\label{light-report-uia}

{ 0.1.0 }

Template for reports at the University of Agder

\href{/app?template=light-report-uia&version=0.1.0}{Create project in
app}

\phantomsection\label{readme}
Unofficial report template for reports at the University of Agder.

Supports both norwegian and english.

Usage:

\begin{verbatim}
#import "@preview/light-report-uia:0.1.0": *

// CHANGE THESE
#show: report.with(
  title: "New project",
  authors: (
    "Lars Larsen",
    "Lise Lisesen",
    "Knut Knutsen",
  ),
  group_name: "Group 14",
  course_code: "IKT123-G",
  course_name: "Course name",
  date: "august 2024",
  lang: "en", // use "no" for norwegian
)
// then do anything
\end{verbatim}

\href{/app?template=light-report-uia&version=0.1.0}{Create project in
app}

\subsubsection{How to use}\label{how-to-use}

Click the button above to create a new project using this template in
the Typst app.

You can also use the Typst CLI to start a new project on your computer
using this command:

\begin{verbatim}
typst init @preview/light-report-uia:0.1.0
\end{verbatim}

\includesvg[width=0.16667in,height=0.16667in]{/assets/icons/16-copy.svg}

\subsubsection{About}\label{about}

\begin{description}
\tightlist
\item[Author :]
Sebastian Dalheim Knudsen
\item[License:]
MIT
\item[Current version:]
0.1.0
\item[Last updated:]
September 14, 2024
\item[First released:]
September 14, 2024
\item[Archive size:]
7.18 kB
\href{https://packages.typst.org/preview/light-report-uia-0.1.0.tar.gz}{\pandocbounded{\includesvg[keepaspectratio]{/assets/icons/16-download.svg}}}
\item[Repository:]
\href{https://github.com/sebastos1/light-report-uia}{GitHub}
\item[Categor y :]
\begin{itemize}
\tightlist
\item[]
\item
  \pandocbounded{\includesvg[keepaspectratio]{/assets/icons/16-speak.svg}}
  \href{https://typst.app/universe/search/?category=report}{Report}
\end{itemize}
\end{description}

\subsubsection{Where to report issues?}\label{where-to-report-issues}

This template is a project of Sebastian Dalheim Knudsen . Report issues
on \href{https://github.com/sebastos1/light-report-uia}{their
repository} . You can also try to ask for help with this template on the
\href{https://forum.typst.app}{Forum} .

Please report this template to the Typst team using the
\href{https://typst.app/contact}{contact form} if you believe it is a
safety hazard or infringes upon your rights.

\phantomsection\label{versions}
\subsubsection{Version history}\label{version-history}

\begin{longtable}[]{@{}ll@{}}
\toprule\noalign{}
Version & Release Date \\
\midrule\noalign{}
\endhead
\bottomrule\noalign{}
\endlastfoot
0.1.0 & September 14, 2024 \\
\end{longtable}

Typst GmbH did not create this template and cannot guarantee correct
functionality of this template or compatibility with any version of the
Typst compiler or app.


\section{Package List LaTeX/down.tex}
\title{typst.app/universe/package/down}

\phantomsection\label{banner}
\section{down}\label{down}

{ 0.1.0 }

Pass down arguments of `sum`, `integral`, etc. to the next line, which
can generate shorthand to present reusable segments.

\phantomsection\label{readme}
Pass down arguments of \texttt{\ sum\ } , \texttt{\ integral\ } , etc.
to the next line, which can generate shorthand to present reusable
segments. While writing long step-by-step equations, only certain parts
of a line change. \texttt{\ down\ } leverages Typst’s
\texttt{\ context\ } (from version 0.11.0) to help relieve the pressure
of writing long and repetitive formulae.

Import the package:

\begin{Shaded}
\begin{Highlighting}[]
\NormalTok{\#import "@preview/down:0.1.0": *}
\end{Highlighting}
\end{Shaded}

\subsection{Usage}\label{usage}

Create new contexts by using camel-case commands, such as
\texttt{\ Limit(x,\ +0)\ } . Retrieve the contextual with
\texttt{\ cLimit\ } .

\begin{itemize}
\tightlist
\item
  \texttt{\ Limit(x,\ c)\ } and \texttt{\ cLimit\ } :
\end{itemize}

\begin{Shaded}
\begin{Highlighting}[]
\NormalTok{$}
\NormalTok{Lim(x, +0) x ln(sin x)}
\NormalTok{  = cLim ln(sin x) / x\^{}({-}1)}
\NormalTok{  = cLim x / (sin x) cos x}
\NormalTok{  = 0}
\NormalTok{$}
\end{Highlighting}
\end{Shaded}

\begin{itemize}
\tightlist
\item
  \texttt{\ Sum(index,\ lower,\ upper)\ } and \texttt{\ cSum\ } :
\end{itemize}

\begin{Shaded}
\begin{Highlighting}[]
\NormalTok{$}
\NormalTok{Sum(n, 0, oo) 1 / sqrt(n + 1)}
\NormalTok{  = Sum(\#none, 0, \#none) 1 / sqrt(n)}
\NormalTok{  = cSum 1 / n\^{}(1 / 2)}
\NormalTok{$}
\end{Highlighting}
\end{Shaded}

\begin{itemize}
\tightlist
\item
  \texttt{\ Integral(lower,\ upper,\ f,\ dif:\ {[}x{]})\ } ,
  \texttt{\ cIntegral(f)\ } and \texttt{\ cIntegrated(f)\ } :
\end{itemize}

\begin{Shaded}
\begin{Highlighting}[]
\NormalTok{$}
\NormalTok{Integral(0, pi / 3, sqrt(1 + tan\^{}2 x))}
\NormalTok{  = cIntegral(1 / (cos x))}
\NormalTok{  = cIntegrated(ln (cos x / 2 + sin x / 2) / (cos x / 2 {-} sin x / 2))}
\NormalTok{  = ln (2 + sqrt(3))}
\NormalTok{$}
\end{Highlighting}
\end{Shaded}

Refer to \texttt{\ ./sample.pdf\ } for more complex application.

\subsubsection{How to add}\label{how-to-add}

Copy this into your project and use the import as \texttt{\ down\ }

\begin{verbatim}
#import "@preview/down:0.1.0"
\end{verbatim}

\includesvg[width=0.16667in,height=0.16667in]{/assets/icons/16-copy.svg}

Check the docs for
\href{https://typst.app/docs/reference/scripting/\#packages}{more
information on how to import packages} .

\subsubsection{About}\label{about}

\begin{description}
\tightlist
\item[Author :]
\href{mailto:the@unpopular.me}{Toto}
\item[License:]
MIT
\item[Current version:]
0.1.0
\item[Last updated:]
April 1, 2024
\item[First released:]
April 1, 2024
\item[Minimum Typst version:]
0.11.0
\item[Archive size:]
2.15 kB
\href{https://packages.typst.org/preview/down-0.1.0.tar.gz}{\pandocbounded{\includesvg[keepaspectratio]{/assets/icons/16-download.svg}}}
\item[Repository:]
\href{https://git.sr.ht/~toto/down}{git.sr.ht}
\item[Discipline :]
\begin{itemize}
\tightlist
\item[]
\item
  \href{https://typst.app/universe/search/?discipline=mathematics}{Mathematics}
\end{itemize}
\item[Categor y :]
\begin{itemize}
\tightlist
\item[]
\item
  \pandocbounded{\includesvg[keepaspectratio]{/assets/icons/16-hammer.svg}}
  \href{https://typst.app/universe/search/?category=utility}{Utility}
\end{itemize}
\end{description}

\subsubsection{Where to report issues?}\label{where-to-report-issues}

This package is a project of Toto . Report issues on
\href{https://git.sr.ht/~toto/down}{their repository} . You can also try
to ask for help with this package on the
\href{https://forum.typst.app}{Forum} .

Please report this package to the Typst team using the
\href{https://typst.app/contact}{contact form} if you believe it is a
safety hazard or infringes upon your rights.

\phantomsection\label{versions}
\subsubsection{Version history}\label{version-history}

\begin{longtable}[]{@{}ll@{}}
\toprule\noalign{}
Version & Release Date \\
\midrule\noalign{}
\endhead
\bottomrule\noalign{}
\endlastfoot
0.1.0 & April 1, 2024 \\
\end{longtable}

Typst GmbH did not create this package and cannot guarantee correct
functionality of this package or compatibility with any version of the
Typst compiler or app.


\section{Package List LaTeX/lucky-icml.tex}
\title{typst.app/universe/package/lucky-icml}

\phantomsection\label{banner}
\phantomsection\label{template-thumbnail}
\pandocbounded{\includegraphics[keepaspectratio]{https://packages.typst.org/preview/thumbnails/lucky-icml-0.2.1-small.webp}}

\section{lucky-icml}\label{lucky-icml}

{ 0.2.1 }

ICML-style paper template to publish at conferences for International
Conference on Machine Learning

\href{/app?template=lucky-icml&version=0.2.1}{Create project in app}

\phantomsection\label{readme}
\subsection{Usage}\label{usage}

You can use this template in the Typst web app by clicking \emph{Start
from template} on the dashboard and searching for
\texttt{\ lucky-icml\ } .

Alternatively, you can use the CLI to kick this project off using the
command

\begin{Shaded}
\begin{Highlighting}[]
\NormalTok{typst init @preview/lucky{-}icml}
\end{Highlighting}
\end{Shaded}

Typst will create a new directory with all the files needed to get you
started.

\subsection{Configuration}\label{configuration}

This template exports the \texttt{\ icml2024\ } function with the
following named arguments.

\begin{itemize}
\tightlist
\item
  \texttt{\ title\ } : The paper’s title as content.
\item
  \texttt{\ authors\ } : An array of author dictionaries. Each of the
  author dictionaries must have a name key and can have the keys
  department, organization, location, and email.
\item
  \texttt{\ abstract\ } : The content of a brief summary of the paper or
  none. Appears at the top under the title.
\item
  \texttt{\ bibliography\ } : The result of a call to the bibliography
  function or none. The function also accepts a single, positional
  argument for the body of the paper.
\item
  \texttt{\ accepted\ } : If this is set to \texttt{\ false\ } then
  anonymized ready for submission document is produced;
  \texttt{\ accepted:\ true\ } produces camera-redy version. If the
  argument is set to \texttt{\ none\ } then preprint version is produced
  (can be uploaded to arXiv).
\end{itemize}

The template will initialize your package with a sample call to the
\texttt{\ icml2024\ } function in a show rule. If you want to change an
existing project to use this template, you can add a show rule at the
top of your file.

\subsection{Issues}\label{issues}

This template is developed at
\href{https://github.com/daskol/typst-templates}{daskol/typst-templates}
repo. Please report all issues there.

\subsubsection{Running Title}\label{running-title}

\begin{enumerate}
\tightlist
\item
  Runing title should be 10pt above the main text. With top margin 1in
  it gives that a solid line should be located at 62pt. Actual, position
  is 57pt in the original template.
\item
  Default value between header ruler and header text baseline is 4pt in
  \texttt{\ fancyhdr\ } . But actual value is 3pt due to thickness of a
  ruler in 1pt.
\end{enumerate}

\subsubsection{Page Numbering}\label{page-numbering}

\begin{enumerate}
\tightlist
\item
  Basis line of page number should be located 25pt below of main text.
  There is a discrepancy in about \textasciitilde1pt.
\end{enumerate}

\subsubsection{Heading}\label{heading}

\begin{enumerate}
\tightlist
\item
  Required space after level 3 headers is 0.1in or 7.2pt. Actual space
  size is large (e.g. distance between Section 2.3.1 header and text
  after it about 12pt).
\end{enumerate}

\subsubsection{Figures and Tables}\label{figures-and-tables}

\begin{enumerate}
\tightlist
\item
  At the moment Typst has limited support for multi-column documents. It
  allows define multi-column blocks and documents but there is no
  ability to typeset complex layout (e.g. page width figures or tables
  in two-column documents).
\end{enumerate}

\subsubsection{Citations and References}\label{citations-and-references}

\begin{enumerate}
\item
  There is a small bug in CSL processor which fails to recognize
  bibliography entries with \texttt{\ chapter\ } field. It is already
  report and will be fixed in the future.
\item
  There is no suitable bibliography style so we use default APA while
  ICML requires APA-like style but not exact APA. The difference is that
  ICML APA-like style places entry year at the end of reference entry.
  In order to fix the issue, we need to describe ICML bibliography style
  in CSL-format.
\item
  In the original template links are colored with dark blue. We can
  reproduce appearance with some additional effort. First of all,
  \texttt{\ icml2024.csl\ } shoule be updated as follows.

\begin{verbatim}
diff --git a/icml/icml2024.csl b/icml/icml2024.csl
index 3b9e9a2..3fe9f74 100644
--- a/icml/icml2024.csl
+++ b/icml/icml2024.csl
@@ -1648,7 +1648,8 @@
       
       
     
-    
+    
+    
       
         
         
\end{verbatim}

  Then instead of convenient citation shortcut
  \texttt{\ @cite-key1\ @cite-key2\ } , one should use special procedure
  \texttt{\ refer\ } with variadic arguments.

\begin{Shaded}
\begin{Highlighting}[]
\NormalTok{\#refer(\textless{}cite{-}key1\textgreater{}, \textless{}cite{-}key2\textgreater{})}
\end{Highlighting}
\end{Shaded}
\end{enumerate}

\href{/app?template=lucky-icml&version=0.2.1}{Create project in app}

\subsubsection{How to use}\label{how-to-use}

Click the button above to create a new project using this template in
the Typst app.

You can also use the Typst CLI to start a new project on your computer
using this command:

\begin{verbatim}
typst init @preview/lucky-icml:0.2.1
\end{verbatim}

\includesvg[width=0.16667in,height=0.16667in]{/assets/icons/16-copy.svg}

\subsubsection{About}\label{about}

\begin{description}
\tightlist
\item[Author :]
\href{mailto:d.bershatsky2@skoltech.ru}{Daniel Bershatsky}
\item[License:]
MIT
\item[Current version:]
0.2.1
\item[Last updated:]
March 19, 2024
\item[First released:]
March 19, 2024
\item[Minimum Typst version:]
0.10.0
\item[Archive size:]
51.2 kB
\href{https://packages.typst.org/preview/lucky-icml-0.2.1.tar.gz}{\pandocbounded{\includesvg[keepaspectratio]{/assets/icons/16-download.svg}}}
\item[Repository:]
\href{https://github.com/daskol/typst-templates}{GitHub}
\item[Discipline s :]
\begin{itemize}
\tightlist
\item[]
\item
  \href{https://typst.app/universe/search/?discipline=computer-science}{Computer
  Science}
\item
  \href{https://typst.app/universe/search/?discipline=mathematics}{Mathematics}
\end{itemize}
\item[Categor y :]
\begin{itemize}
\tightlist
\item[]
\item
  \pandocbounded{\includesvg[keepaspectratio]{/assets/icons/16-atom.svg}}
  \href{https://typst.app/universe/search/?category=paper}{Paper}
\end{itemize}
\end{description}

\subsubsection{Where to report issues?}\label{where-to-report-issues}

This template is a project of Daniel Bershatsky . Report issues on
\href{https://github.com/daskol/typst-templates}{their repository} . You
can also try to ask for help with this template on the
\href{https://forum.typst.app}{Forum} .

Please report this template to the Typst team using the
\href{https://typst.app/contact}{contact form} if you believe it is a
safety hazard or infringes upon your rights.

\phantomsection\label{versions}
\subsubsection{Version history}\label{version-history}

\begin{longtable}[]{@{}ll@{}}
\toprule\noalign{}
Version & Release Date \\
\midrule\noalign{}
\endhead
\bottomrule\noalign{}
\endlastfoot
0.2.1 & March 19, 2024 \\
\end{longtable}

Typst GmbH did not create this template and cannot guarantee correct
functionality of this template or compatibility with any version of the
Typst compiler or app.


\section{Package List LaTeX/lovelace.tex}
\title{typst.app/universe/package/lovelace}

\phantomsection\label{banner}
\section{lovelace}\label{lovelace}

{ 0.3.0 }

Algorithms in pseudocode, unopinionated and flexible

{ } Featured Package

\phantomsection\label{readme}
This is a package for writing pseudocode in
\href{https://typst.app/}{Typst} . It is named after the computer
science pioneer \href{https://en.wikipedia.org/wiki/Ada_Lovelace}{Ada
Lovelace} and inspired by the \href{https://ctan.org/pkg/pseudo}{pseudo
package} for LaTeX.

\pandocbounded{\includegraphics[keepaspectratio]{https://img.shields.io/github/license/andreasKroepelin/lovelace}}
\pandocbounded{\includegraphics[keepaspectratio]{https://img.shields.io/github/v/release/andreasKroepelin/lovelace}}
\pandocbounded{\includegraphics[keepaspectratio]{https://img.shields.io/github/stars/andreasKroepelin/lovelace}}

Pseudocode is not a programming language, it doesn’t have strict
syntax, so you should be able to write it however you need to in your
specific situation. Lovelace lets you do exactly that.

Main features include:

\begin{itemize}
\tightlist
\item
  arbitrary keywords and syntax structures
\item
  optional line numbering
\item
  line labels
\item
  lots of customisation with sensible defaults
\end{itemize}

\subsection{Usage}\label{usage}

\begin{itemize}
\tightlist
\item
  \href{https://github.com/typst/packages/raw/main/packages/preview/lovelace/0.3.0/\#getting-started}{Getting
  started}
\item
  \href{https://github.com/typst/packages/raw/main/packages/preview/lovelace/0.3.0/\#lower-level-interface}{Lower
  level interface}
\item
  \href{https://github.com/typst/packages/raw/main/packages/preview/lovelace/0.3.0/\#line-numbers}{Line
  numbers}
\item
  \href{https://github.com/typst/packages/raw/main/packages/preview/lovelace/0.3.0/\#referencing-lines}{Referencing
  lines}
\item
  \href{https://github.com/typst/packages/raw/main/packages/preview/lovelace/0.3.0/\#indentation-guides}{Indentation
  guides}
\item
  \href{https://github.com/typst/packages/raw/main/packages/preview/lovelace/0.3.0/\#spacing}{Spacing}
\item
  \href{https://github.com/typst/packages/raw/main/packages/preview/lovelace/0.3.0/\#decorations}{Decorations}
\item
  \href{https://github.com/typst/packages/raw/main/packages/preview/lovelace/0.3.0/\#algorithm-as-figure}{Algorithm
  as figure}
\item
  \href{https://github.com/typst/packages/raw/main/packages/preview/lovelace/0.3.0/\#customisation-overview}{Customisation
  overview}
\item
  \href{https://github.com/typst/packages/raw/main/packages/preview/lovelace/0.3.0/\#exported-functions}{Exported
  functions}
\end{itemize}

\subsubsection{Getting started}\label{getting-started}

Import the package using

\begin{Shaded}
\begin{Highlighting}[]
\NormalTok{\#import "@preview/lovelace:0.3.0": *}
\end{Highlighting}
\end{Shaded}

The simplest usage is via \texttt{\ pseudocode-list\ } which transforms
a nested list into pseudocode:

\begin{Shaded}
\begin{Highlighting}[]
\NormalTok{\#pseudocode{-}list[}
\NormalTok{  + do something}
\NormalTok{  + do something else}
\NormalTok{  + *while* still something to do}
\NormalTok{    + do even more}
\NormalTok{    + *if* not done yet *then*}
\NormalTok{      + wait a bit}
\NormalTok{      + resume working}
\NormalTok{    + *else*}
\NormalTok{      + go home}
\NormalTok{    + *end*}
\NormalTok{  + *end*}
\NormalTok{]}
\end{Highlighting}
\end{Shaded}

resulting in:

\pandocbounded{\includesvg[keepaspectratio]{https://github.com/typst/packages/raw/main/packages/preview/lovelace/0.3.0/examples/simple.svg}}

As you can see, every list item becomes one line of code and nested
lists become indented blocks. There are no special commands for common
keywords and control structures, you just use whatever you like.

Maybe in your domain very uncommon structures make more sense? No
problem!

\begin{Shaded}
\begin{Highlighting}[]
\NormalTok{\#pseudocode{-}list[}
\NormalTok{  + *in parallel for each* $i = 1, ..., n$ *do*}
\NormalTok{    + fetch chunk of data $i$}
\NormalTok{    + *with probability* $exp({-}epsilon\_i slash k T)$ *do*}
\NormalTok{      + perform update}
\NormalTok{    + *end*}
\NormalTok{  + *end*}
\NormalTok{]}
\end{Highlighting}
\end{Shaded}

\pandocbounded{\includesvg[keepaspectratio]{https://github.com/typst/packages/raw/main/packages/preview/lovelace/0.3.0/examples/custom.svg}}

\subsubsection{Lower level interface}\label{lower-level-interface}

If you feel uncomfortable with abusing Typst’s lists like we do here,
you can also use the \texttt{\ pseudocode\ } function directly:

\begin{Shaded}
\begin{Highlighting}[]
\NormalTok{\#pseudocode(}
\NormalTok{  [do something],}
\NormalTok{  [do something else],}
\NormalTok{  [*while* still something to do],}
\NormalTok{  indent(}
\NormalTok{    [do even more],}
\NormalTok{    [*if* not done yet *then*],}
\NormalTok{    indent(}
\NormalTok{      [wait a bit],}
\NormalTok{      [resume working],}
\NormalTok{    ),}
\NormalTok{    [*else*],}
\NormalTok{    indent(}
\NormalTok{      [go home],}
\NormalTok{    ),}
\NormalTok{    [*end*],}
\NormalTok{  ),}
\NormalTok{  [*end*],}
\NormalTok{)}
\end{Highlighting}
\end{Shaded}

This is equivalent to the first example. Note that each line is given as
one content argument and you indent a block by using the
\texttt{\ indent\ } function.

This approach has the advantage that you do not rely on significant
whitespace and code formatters can automatically correctly indent your
Typst code.

\subsubsection{Line numbers}\label{line-numbers}

Lovelace puts a number in front of each line by default. If you want no
numbers at all, you can set the \texttt{\ line-numbering\ } option to
\texttt{\ none\ } . The initial example then looks like this:

\begin{Shaded}
\begin{Highlighting}[]
\NormalTok{\#pseudocode{-}list(line{-}numbering: none)[}
\NormalTok{  + do something}
\NormalTok{  + do something else}
\NormalTok{  + *while* still something to do}
\NormalTok{    + do even more}
\NormalTok{    + *if* not done yet *then*}
\NormalTok{      + wait a bit}
\NormalTok{      + resume working}
\NormalTok{    + *else*}
\NormalTok{      + go home}
\NormalTok{    + *end*}
\NormalTok{  + *end*}
\NormalTok{]}
\end{Highlighting}
\end{Shaded}

\pandocbounded{\includesvg[keepaspectratio]{https://github.com/typst/packages/raw/main/packages/preview/lovelace/0.3.0/examples/simple-no-numbers.svg}}

(You can also pass this keyword argument to \texttt{\ pseudocode\ } .)

If you do want line numbers in general but need to turn them off for
specific lines, you can use \texttt{\ -\ } items instead of
\texttt{\ +\ } items in \texttt{\ pseudocode-list\ } :

\begin{Shaded}
\begin{Highlighting}[]
\NormalTok{\#pseudocode{-}list[}
\NormalTok{  + normal line with a number}
\NormalTok{  {-} this line has no number}
\NormalTok{  + this one has a number again}
\NormalTok{]}
\end{Highlighting}
\end{Shaded}

\pandocbounded{\includesvg[keepaspectratio]{https://github.com/typst/packages/raw/main/packages/preview/lovelace/0.3.0/examples/number-no-number.svg}}

It’s easy to remember: \texttt{\ -\ } items usually produce unnumbered
lists and \texttt{\ +\ } items produce numbered lists!

When using the \texttt{\ pseudocode\ } function, you can achieve the
same using \texttt{\ no-number\ } :

\begin{Shaded}
\begin{Highlighting}[]
\NormalTok{\#pseudocode(}
\NormalTok{  [normal line with a number],}
\NormalTok{  no{-}number[this line has no number],}
\NormalTok{  [this one has a number again],}
\NormalTok{)}
\end{Highlighting}
\end{Shaded}

\paragraph{More line number
customisation}\label{more-line-number-customisation}

Other than \texttt{\ none\ } , you can assign anything listed
\href{https://typst.app/docs/reference/model/numbering/\#parameters-numbering}{here}
to \texttt{\ line-numbering\ } .

So maybe you happen to think about the Roman Empire a lot and want to
reflect that in your pseudocode?

\begin{Shaded}
\begin{Highlighting}[]
\NormalTok{\#set text(font: "Cinzel")}

\NormalTok{\#pseudocode{-}list(line{-}numbering: "I:")[}
\NormalTok{  + explore European tribes}
\NormalTok{  + *while* not all tribes conquered}
\NormalTok{    + *for each* tribe *in* unconquered tribes}
\NormalTok{      + try to conquer tribe}
\NormalTok{    + *end*}
\NormalTok{  + *end*}
\NormalTok{]}
\end{Highlighting}
\end{Shaded}

\pandocbounded{\includesvg[keepaspectratio]{https://github.com/typst/packages/raw/main/packages/preview/lovelace/0.3.0/examples/roman.svg}}

\subsubsection{Referencing lines}\label{referencing-lines}

You can reference an inividual line of a pseudocode by giving it a
label. Inside \texttt{\ pseudocode-list\ } , you can use
\texttt{\ line-label\ } :

\begin{Shaded}
\begin{Highlighting}[]
\NormalTok{\#pseudocode{-}list[}
\NormalTok{  + \#line{-}label(\textless{}start\textgreater{}) do something}
\NormalTok{  + \#line{-}label(\textless{}important\textgreater{}) do something important}
\NormalTok{  + go back to @start}
\NormalTok{]}

\NormalTok{The relevance of the step in @important cannot be overstated.}
\end{Highlighting}
\end{Shaded}

\pandocbounded{\includesvg[keepaspectratio]{https://github.com/typst/packages/raw/main/packages/preview/lovelace/0.3.0/examples/label.svg}}

When using \texttt{\ pseudocode\ } , you can use
\texttt{\ with-line-label\ } :

\begin{Shaded}
\begin{Highlighting}[]
\NormalTok{\#pseudocode(}
\NormalTok{  with{-}line{-}label(\textless{}start\textgreater{})[do something],}
\NormalTok{  with{-}line{-}label(\textless{}important\textgreater{})[do something important],}
\NormalTok{  [go back to @start],}
\NormalTok{)}

\NormalTok{The relevance of the step in @important cannot be overstated.}
\end{Highlighting}
\end{Shaded}

This has the same effect as the previous example.

The number shown in the reference uses the numbering scheme defined in
the \texttt{\ line-numbering\ } option (see previous section).

By default, \texttt{\ "Line"\ } is used as the supplement for
referencing lines. You can change that using the
\texttt{\ line-number-supplement\ } option to \texttt{\ pseudocode\ } or
\texttt{\ pseudocode-list\ } .

\subsubsection{Indentation guides}\label{indentation-guides}

By default, Lovelace puts a thin gray ( \texttt{\ gray\ +\ 1pt\ } ) line
to the left of each indented block, which guides the reader in
understanding the indentations, just like a code editor would. You can
customise this using the \texttt{\ stroke\ } option which takes any
value that is a valid
\href{https://typst.app/docs/reference/visualize/stroke/}{Typst stroke}
. You can especially set it to \texttt{\ none\ } to have no indentation
guides.

The example from the beginning becomes:

\begin{Shaded}
\begin{Highlighting}[]
\NormalTok{\#pseudocode{-}list(stroke: none)[}
\NormalTok{  + do something}
\NormalTok{  + do something else}
\NormalTok{  + *while* still something to do}
\NormalTok{    + do even more}
\NormalTok{    + *if* not done yet *then*}
\NormalTok{      + wait a bit}
\NormalTok{      + resume working}
\NormalTok{    + *else*}
\NormalTok{      + go home}
\NormalTok{    + *end*}
\NormalTok{  + *end*}
\NormalTok{]}
\end{Highlighting}
\end{Shaded}

\pandocbounded{\includesvg[keepaspectratio]{https://github.com/typst/packages/raw/main/packages/preview/lovelace/0.3.0/examples/simple-no-stroke.svg}}

\paragraph{End blocks with hooks}\label{end-blocks-with-hooks}

Some people prefer using the indentation guide to signal the end of a
block instead of writing something like “ \textbf{end} � by having a
small “hook� at the end. To achieve that in Lovelace, you can make
use of the \texttt{\ hook\ } option and specify how far a line should
extend to the right from the indentation guide:

\begin{Shaded}
\begin{Highlighting}[]
\NormalTok{\#pseudocode{-}list(hooks: .5em)[}
\NormalTok{  + do something}
\NormalTok{  + do something else}
\NormalTok{  + *while* still something to do}
\NormalTok{    + do even more}
\NormalTok{    + *if* not done yet *then*}
\NormalTok{      + wait a bit}
\NormalTok{      + resume working}
\NormalTok{    + *else*}
\NormalTok{      + go home}
\NormalTok{]}
\end{Highlighting}
\end{Shaded}

\pandocbounded{\includesvg[keepaspectratio]{https://github.com/typst/packages/raw/main/packages/preview/lovelace/0.3.0/examples/hooks.svg}}

\subsubsection{Spacing}\label{spacing}

You can control how far indented lines are shifted right by the
\texttt{\ indentation\ } option. To change the space between lines, use
the \texttt{\ line-gap\ } option.

\begin{Shaded}
\begin{Highlighting}[]
\NormalTok{\#pseudocode{-}list(indentation: 3em, line{-}gap: 1.5em)[}
\NormalTok{  + do something}
\NormalTok{  + do something else}
\NormalTok{  + *while* still something to do}
\NormalTok{    + do even more}
\NormalTok{    + *if* not done yet *then*}
\NormalTok{      + wait a bit}
\NormalTok{      + resume working}
\NormalTok{    + *else*}
\NormalTok{      + go home}
\NormalTok{    + *end*}
\NormalTok{  + *end*}
\NormalTok{]}
\end{Highlighting}
\end{Shaded}

\pandocbounded{\includesvg[keepaspectratio]{https://github.com/typst/packages/raw/main/packages/preview/lovelace/0.3.0/examples/spacing.svg}}

\subsubsection{Decorations}\label{decorations}

You can also add a title and/or a frame around your algorithm if you
like:

\paragraph{Title}\label{title}

Using the \texttt{\ title\ } option, you can give your pseudocode a
title (surprise!). For example, to achieve
\href{https://en.wikipedia.org/wiki/Introduction_to_Algorithms}{CLRS
style} , you can do something like

\begin{Shaded}
\begin{Highlighting}[]
\NormalTok{\#pseudocode{-}list(stroke: none, title: smallcaps[Fancy{-}Algorithm])[}
\NormalTok{  + do something}
\NormalTok{  + do something else}
\NormalTok{  + *while* still something to do}
\NormalTok{    + do even more}
\NormalTok{    + *if* not done yet *then*}
\NormalTok{      + wait a bit}
\NormalTok{      + resume working}
\NormalTok{    + *else*}
\NormalTok{      + go home}
\NormalTok{    + *end*}
\NormalTok{  + *end*}
\NormalTok{]}
\end{Highlighting}
\end{Shaded}

\pandocbounded{\includesvg[keepaspectratio]{https://github.com/typst/packages/raw/main/packages/preview/lovelace/0.3.0/examples/title.svg}}

\paragraph{Booktabs}\label{booktabs}

If you like wrapping your algorithm in elegant horizontal lines, you can
do so by setting the \texttt{\ booktabs\ } option to \texttt{\ true\ } .

\begin{Shaded}
\begin{Highlighting}[]
\NormalTok{\#pseudocode{-}list(booktabs: true)[}
\NormalTok{  + do something}
\NormalTok{  + do something else}
\NormalTok{  + *while* still something to do}
\NormalTok{    + do even more}
\NormalTok{    + *if* not done yet *then*}
\NormalTok{      + wait a bit}
\NormalTok{      + resume working}
\NormalTok{    + *else*}
\NormalTok{      + go home}
\NormalTok{    + *end*}
\NormalTok{  + *end*}
\NormalTok{]}
\end{Highlighting}
\end{Shaded}

\pandocbounded{\includesvg[keepaspectratio]{https://github.com/typst/packages/raw/main/packages/preview/lovelace/0.3.0/examples/booktabs.svg}}

Together with the \texttt{\ title\ } option, you can produce

\begin{Shaded}
\begin{Highlighting}[]
\NormalTok{\#pseudocode{-}list(booktabs: true, title: [My cool title])[}
\NormalTok{  + do something}
\NormalTok{  + do something else}
\NormalTok{  + *while* still something to do}
\NormalTok{    + do even more}
\NormalTok{    + *if* not done yet *then*}
\NormalTok{      + wait a bit}
\NormalTok{      + resume working}
\NormalTok{    + *else*}
\NormalTok{      + go home}
\NormalTok{    + *end*}
\NormalTok{  + *end*}
\NormalTok{]}
\end{Highlighting}
\end{Shaded}

\pandocbounded{\includesvg[keepaspectratio]{https://github.com/typst/packages/raw/main/packages/preview/lovelace/0.3.0/examples/booktabs-title.svg}}

By default, the outer booktab strokes are \texttt{\ black\ +\ 2pt\ } .
You can change that with the option \texttt{\ booktabs-stroke\ } to any
valid \href{https://typst.app/docs/reference/visualize/stroke/}{Typst
stroke} . The inner line will always have the same stroke as the outer
ones, just with half the thickness.

\subsubsection{Algorithm as figure}\label{algorithm-as-figure}

To make algorithms referencable and being able to float in the document,
you can use Typst’s \texttt{\ figure\ } function with a custom
\texttt{\ kind\ } .

\begin{Shaded}
\begin{Highlighting}[]
\NormalTok{\#figure(}
\NormalTok{  kind: "algorithm",}
\NormalTok{  supplement: [Algorithm],}
\NormalTok{  caption: [My cool algorithm],}

\NormalTok{  pseudocode{-}list[}
\NormalTok{    + do something}
\NormalTok{    + do something else}
\NormalTok{    + *while* still something to do}
\NormalTok{      + do even more}
\NormalTok{      + *if* not done yet *then*}
\NormalTok{        + wait a bit}
\NormalTok{        + resume working}
\NormalTok{      + *else*}
\NormalTok{        + go home}
\NormalTok{      + *end*}
\NormalTok{    + *end*}
\NormalTok{  ]}
\NormalTok{)}
\end{Highlighting}
\end{Shaded}

\pandocbounded{\includesvg[keepaspectratio]{https://github.com/typst/packages/raw/main/packages/preview/lovelace/0.3.0/examples/figure.svg}}

If you want to have the algorithm counter inside the title instead (see
previous section), there is the option \texttt{\ numbered-title\ } :

\begin{Shaded}
\begin{Highlighting}[]
\NormalTok{\#figure(}
\NormalTok{  kind: "algorithm",}
\NormalTok{  supplement: [Algorithm],}

\NormalTok{  pseudocode{-}list(booktabs: true, numbered{-}title: [My cool algorithm])[}
\NormalTok{    + do something}
\NormalTok{    + do something else}
\NormalTok{    + *while* still something to do}
\NormalTok{      + do even more}
\NormalTok{      + *if* not done yet *then*}
\NormalTok{        + wait a bit}
\NormalTok{        + resume working}
\NormalTok{      + *else*}
\NormalTok{        + go home}
\NormalTok{      + *end*}
\NormalTok{    + *end*}
\NormalTok{  ]}
\NormalTok{) \textless{}cool\textgreater{}}

\NormalTok{See @cool for details on how to do something cool.}
\end{Highlighting}
\end{Shaded}

\pandocbounded{\includesvg[keepaspectratio]{https://github.com/typst/packages/raw/main/packages/preview/lovelace/0.3.0/examples/figure-title.svg}}

Note that the \texttt{\ numbered-title\ } option only makes sense when
nesting your pseudocode inside a figure with
\texttt{\ kind:\ "algorithm"\ } , otherwise it produces undefined
results.

\subsubsection{Customisation overview}\label{customisation-overview}

Both \texttt{\ pseudocode\ } and \texttt{\ pseudocode-list\ } accept the
following configuration arguments:

\begin{longtable}[]{@{}lll@{}}
\toprule\noalign{}
\textbf{option} & \textbf{type} & \textbf{default} \\
\midrule\noalign{}
\endhead
\bottomrule\noalign{}
\endlastfoot
\href{https://github.com/typst/packages/raw/main/packages/preview/lovelace/0.3.0/\#line-numbers}{\texttt{\ line-numbering\ }}
& \texttt{\ none\ } or a
\href{https://typst.app/docs/reference/model/numbering/\#parameters-numbering}{numbering}
& \texttt{\ "1"\ } \\
\href{https://github.com/typst/packages/raw/main/packages/preview/lovelace/0.3.0/\#more-line-number-customisation}{\texttt{\ line-number-supplement\ }}
& content & \texttt{\ "Line"\ } \\
\href{https://github.com/typst/packages/raw/main/packages/preview/lovelace/0.3.0/\#indentation-guides}{\texttt{\ stroke\ }}
& stroke & \texttt{\ 1pt\ +\ gray\ } \\
\href{https://github.com/typst/packages/raw/main/packages/preview/lovelace/0.3.0/\#end-blocks-with-hooks}{\texttt{\ hooks\ }}
& length & \texttt{\ 0pt\ } \\
\href{https://github.com/typst/packages/raw/main/packages/preview/lovelace/0.3.0/\#spacing}{\texttt{\ indentation\ }}
& length & \texttt{\ 1em\ } \\
\href{https://github.com/typst/packages/raw/main/packages/preview/lovelace/0.3.0/\#spacing}{\texttt{\ line-gap\ }}
& length & \texttt{\ .8em\ } \\
\href{https://github.com/typst/packages/raw/main/packages/preview/lovelace/0.3.0/\#booktabs}{\texttt{\ booktabs\ }}
& bool & \texttt{\ false\ } \\
\href{https://github.com/typst/packages/raw/main/packages/preview/lovelace/0.3.0/\#booktabs}{\texttt{\ booktabs-stroke\ }}
& stroke & \texttt{\ 2pt\ +\ black\ } \\
\href{https://github.com/typst/packages/raw/main/packages/preview/lovelace/0.3.0/\#title}{\texttt{\ title\ }}
& content or \texttt{\ none\ } & \texttt{\ none\ } \\
\href{https://github.com/typst/packages/raw/main/packages/preview/lovelace/0.3.0/\#algorithm-as-figure}{\texttt{\ numbered-title\ }}
& content or \texttt{\ none\ } & \texttt{\ none\ } \\
\end{longtable}

Until Typst supports user defined types, we can use the following trick
when wanting to set own default values for these options. Say, you
always want your algorithms to have colons after the line numbers, no
indentation guides and, if present, blue booktabs. In this case, you
would put the following at the top of your document:

\begin{Shaded}
\begin{Highlighting}[]
\NormalTok{\#let my{-}lovelace{-}defaults = (}
\NormalTok{  line{-}numbering: "1:",}
\NormalTok{  stroke: none,}
\NormalTok{  booktabs{-}stroke: 2pt + blue,}
\NormalTok{)}

\NormalTok{\#let pseudocode = pseudocode.with(..my{-}lovelace{-}defaults)}
\NormalTok{\#let pseudocode{-}list = pseudocode{-}list.with(..my{-}lovelace{-}defaults)}
\end{Highlighting}
\end{Shaded}

\subsubsection{Exported functions}\label{exported-functions}

Lovelace exports the following functions:

\begin{itemize}
\tightlist
\item
  \texttt{\ pseudocode\ } : Typeset pseudocode with each line as an
  individual content argument, see
  \href{https://github.com/typst/packages/raw/main/packages/preview/lovelace/0.3.0/\#lower-level-interface}{here}
  for details. Has
  \href{https://github.com/typst/packages/raw/main/packages/preview/lovelace/0.3.0/\#customisation-overview}{these}
  optional arguments.
\item
  \texttt{\ pseudocode-list\ } : Takes a standard Typst list and
  transforms it into a pseudocode. Has
  \href{https://github.com/typst/packages/raw/main/packages/preview/lovelace/0.3.0/\#customisation-overview}{these}
  optional arguments.
\item
  \texttt{\ indent\ } : Inside the argument list of
  \texttt{\ pseudocode\ } , use \texttt{\ indent\ } to specify an
  indented block, see
  \href{https://github.com/typst/packages/raw/main/packages/preview/lovelace/0.3.0/\#lower-level-interface}{here}
  for details.
\item
  \texttt{\ no-number\ } : Wrap an argument to \texttt{\ pseudocode\ }
  in this function to have the corresponding line be unnumbered, see
  \href{https://github.com/typst/packages/raw/main/packages/preview/lovelace/0.3.0/\#line-numbers}{here}
  for details.
\item
  \texttt{\ with-line-label\ } : Use this function in the
  \texttt{\ pseudocode\ } arguments to add a label to a specific line,
  see
  \href{https://github.com/typst/packages/raw/main/packages/preview/lovelace/0.3.0/\#referencing-lines}{here}
  for details.
\item
  \texttt{\ line-label\ } : When using \texttt{\ pseudocode-list\ } ,
  you do \emph{not} use \texttt{\ with-line-label\ } but insert a call
  to \texttt{\ line-label\ } somewhere in a line to add a label, see
  \href{https://github.com/typst/packages/raw/main/packages/preview/lovelace/0.3.0/\#referencing-lines}{here}
  for details.
\end{itemize}

\subsubsection{How to add}\label{how-to-add}

Copy this into your project and use the import as \texttt{\ lovelace\ }

\begin{verbatim}
#import "@preview/lovelace:0.3.0"
\end{verbatim}

\includesvg[width=0.16667in,height=0.16667in]{/assets/icons/16-copy.svg}

Check the docs for
\href{https://typst.app/docs/reference/scripting/\#packages}{more
information on how to import packages} .

\subsubsection{About}\label{about}

\begin{description}
\tightlist
\item[Author s :]
Andreas Kröpelin \& Lovelace contributors
\item[License:]
MIT
\item[Current version:]
0.3.0
\item[Last updated:]
July 1, 2024
\item[First released:]
September 6, 2023
\item[Archive size:]
3.44 kB
\href{https://packages.typst.org/preview/lovelace-0.3.0.tar.gz}{\pandocbounded{\includesvg[keepaspectratio]{/assets/icons/16-download.svg}}}
\item[Repository:]
\href{https://github.com/andreasKroepelin/lovelace}{GitHub}
\end{description}

\subsubsection{Where to report issues?}\label{where-to-report-issues}

This package is a project of Andreas Kröpelin and Lovelace contributors
. Report issues on
\href{https://github.com/andreasKroepelin/lovelace}{their repository} .
You can also try to ask for help with this package on the
\href{https://forum.typst.app}{Forum} .

Please report this package to the Typst team using the
\href{https://typst.app/contact}{contact form} if you believe it is a
safety hazard or infringes upon your rights.

\phantomsection\label{versions}
\subsubsection{Version history}\label{version-history}

\begin{longtable}[]{@{}ll@{}}
\toprule\noalign{}
Version & Release Date \\
\midrule\noalign{}
\endhead
\bottomrule\noalign{}
\endlastfoot
0.3.0 & July 1, 2024 \\
\href{https://typst.app/universe/package/lovelace/0.2.0/}{0.2.0} &
January 9, 2024 \\
\href{https://typst.app/universe/package/lovelace/0.1.0/}{0.1.0} &
September 6, 2023 \\
\end{longtable}

Typst GmbH did not create this package and cannot guarantee correct
functionality of this package or compatibility with any version of the
Typst compiler or app.


\section{Package List LaTeX/ansi-render.tex}
\title{typst.app/universe/package/ansi-render}

\phantomsection\label{banner}
\section{ansi-render}\label{ansi-render}

{ 0.6.1 }

provides a simple way to render text with ANSI escape sequences.

\phantomsection\label{readme}
\href{https://github.com/8LWXpg/typst-ansi-render/tags}{\pandocbounded{\includegraphics[keepaspectratio]{https://img.shields.io/github/v/tag/8LWXpg/typst-ansi-render}}}
\href{https://github.com/8LWXpg/typst-ansi-render}{\pandocbounded{\includegraphics[keepaspectratio]{https://img.shields.io/github/stars/8LWXpg/typst-ansi-render}}}
\href{https://github.com/8LWXpg/typst-ansi-render/blob/master/LICENSE}{\pandocbounded{\includegraphics[keepaspectratio]{https://img.shields.io/github/license/8LWXpg/typst-ansi-render}}}
\href{https://github.com/typst/packages/tree/main/packages/preview/ansi-render}{\pandocbounded{\includegraphics[keepaspectratio]{https://img.shields.io/badge/typst-package-239dad}}}

This script provides a simple way to render text with ANSI escape
sequences. Package \texttt{\ ansi-render\ } provides a function
\texttt{\ ansi-render\ } , and a dictionary of themes
\texttt{\ terminal-themes\ } .

contribution is welcomed!

\subsection{Usage}\label{usage}

\begin{Shaded}
\begin{Highlighting}[]
\NormalTok{\#import "@preview/ansi{-}render:0.6.1": *}

\NormalTok{\#ansi{-}render(}
\NormalTok{  string,}
\NormalTok{  font:           string or none,}
\NormalTok{  size:           length,}
\NormalTok{  width:          auto or relative length,}
\NormalTok{  height:         auto or relative length,}
\NormalTok{  breakable:      boolean,}
\NormalTok{  radius:         relative length or dictionary,}
\NormalTok{  inset:          relative length or dictionary,}
\NormalTok{  outset:         relative length or dictionary,}
\NormalTok{  spacing:        relative length or fraction,}
\NormalTok{  above:          relative length or fraction,}
\NormalTok{  below:          relative length or fraction,}
\NormalTok{  clip:           boolean,}
\NormalTok{  bold{-}is{-}bright: boolean,}
\NormalTok{  theme:          terminal{-}themes.theme,}
\NormalTok{)}
\end{Highlighting}
\end{Shaded}

\subsubsection{Parameters}\label{parameters}

\begin{itemize}
\tightlist
\item
  \texttt{\ string\ } - string with ANSI escape sequences
\item
  \texttt{\ font\ } - font name or none, default is
  \texttt{\ Cascadia\ Code\ } , set to \texttt{\ none\ } to use the same
  font as \texttt{\ raw\ }
\item
  \texttt{\ size\ } - font size, default is \texttt{\ 1em\ }
\item
  \texttt{\ bold-is-bright\ } - boolean, whether bold text is rendered
  with bright colors, default is \texttt{\ false\ }
\item
  \texttt{\ theme\ } - theme, default is \texttt{\ vscode-light\ }
\item
  parameters from
  \href{https://typst.app/docs/reference/layout/block/}{\texttt{\ block\ }}
  function with the same default value, only affects outmost block
  layout:

  \begin{itemize}
  \tightlist
  \item
    \texttt{\ width\ }
  \item
    \texttt{\ height\ }
  \item
    \texttt{\ breakable\ }
  \item
    \texttt{\ radius\ }
  \item
    \texttt{\ inset\ }
  \item
    \texttt{\ outset\ }
  \item
    \texttt{\ spacing\ }
  \item
    \texttt{\ above\ }
  \item
    \texttt{\ below\ }
  \item
    \texttt{\ clip\ }
  \end{itemize}
\end{itemize}

\subsection{Themes}\label{themes}

see
\href{https://github.com/8LWXpg/typst-ansi-render/blob/master/test/themes.pdf}{themes}

\subsection{Demo}\label{demo}

see
\href{https://github.com/8LWXpg/typst-ansi-render/blob/master/test/demo.typ}{demo.typ}
\href{https://github.com/8LWXpg/typst-ansi-render/blob/master/test/demo.pdf}{demo.pdf}

\begin{Shaded}
\begin{Highlighting}[]
\NormalTok{\#ansi{-}render(}
\NormalTok{"\textbackslash{}u\{1b\}[38;2;255;0;0mThis text is red.\textbackslash{}u\{1b\}[0m}
\NormalTok{\textbackslash{}u\{1b\}[48;2;0;255;0mThis background is green.\textbackslash{}u\{1b\}[0m}
\NormalTok{\textbackslash{}u\{1b\}[38;2;255;255;255m\textbackslash{}u\{1b\}[48;2;0;0;255mThis text is white on a blue background.\textbackslash{}u\{1b\}[0m}
\NormalTok{\textbackslash{}u\{1b\}[1mThis text is bold.\textbackslash{}u\{1b\}[0m}
\NormalTok{\textbackslash{}u\{1b\}[4mThis text is underlined.\textbackslash{}u\{1b\}[0m}
\NormalTok{\textbackslash{}u\{1b\}[38;2;255;165;0m\textbackslash{}u\{1b\}[48;2;255;255;0mThis text is orange on a yellow background.\textbackslash{}u\{1b\}[0m",}
\NormalTok{inset: 5pt, radius: 3pt,}
\NormalTok{theme: terminal{-}themes.vscode}
\NormalTok{)}
\end{Highlighting}
\end{Shaded}

\pandocbounded{\includegraphics[keepaspectratio]{https://raw.githubusercontent.com/8LWXpg/typst-ansi-render/master/img/1.png}}

\begin{Shaded}
\begin{Highlighting}[]
\NormalTok{\#ansi{-}render(}
\NormalTok{"\textbackslash{}u\{1b\}[38;5;196mRed text\textbackslash{}u\{1b\}[0m}
\NormalTok{\textbackslash{}u\{1b\}[48;5;27mBlue background\textbackslash{}u\{1b\}[0m}
\NormalTok{\textbackslash{}u\{1b\}[38;5;226;48;5;18mYellow text on blue background\textbackslash{}u\{1b\}[0m}
\NormalTok{\textbackslash{}u\{1b\}[7mInverted text\textbackslash{}u\{1b\}[0m}
\NormalTok{\textbackslash{}u\{1b\}[38;5;208;48;5;237mOrange text on gray background\textbackslash{}u\{1b\}[0m}
\NormalTok{\textbackslash{}u\{1b\}[38;5;39;48;5;208mBlue text on orange background\textbackslash{}u\{1b\}[0m}
\NormalTok{\textbackslash{}u\{1b\}[38;5;255;48;5;0mWhite text on black background\textbackslash{}u\{1b\}[0m",}
\NormalTok{inset: 5pt, radius: 3pt,}
\NormalTok{theme: terminal{-}themes.vscode}
\NormalTok{)}
\end{Highlighting}
\end{Shaded}

\pandocbounded{\includegraphics[keepaspectratio]{https://raw.githubusercontent.com/8LWXpg/typst-ansi-render/master/img/2.png}}

\begin{Shaded}
\begin{Highlighting}[]
\NormalTok{\#ansi{-}render(}
\NormalTok{"\textbackslash{}u\{1b\}[31;1mHello \textbackslash{}u\{1b\}[7mWorld\textbackslash{}u\{1b\}[0m}

\NormalTok{\textbackslash{}u\{1b\}[53;4;36mOver  and \textbackslash{}u\{1b\}[35m Under!}
\NormalTok{\textbackslash{}u\{1b\}[7;90mreverse\textbackslash{}u\{1b\}[101m and \textbackslash{}u\{1b\}[94;27mreverse",}
\NormalTok{inset: 5pt, radius: 3pt,}
\NormalTok{theme: terminal{-}themes.vscode}
\NormalTok{)}
\end{Highlighting}
\end{Shaded}

\pandocbounded{\includegraphics[keepaspectratio]{https://raw.githubusercontent.com/8LWXpg/typst-ansi-render/master/img/3.png}}

\begin{Shaded}
\begin{Highlighting}[]
\NormalTok{// uses the font that supports ligatures}
\NormalTok{\#ansi{-}render(read("test.txt"), inset: 5pt, radius: 3pt, font: "Cascadia Code", theme: terminal{-}themes.putty)}
\end{Highlighting}
\end{Shaded}

\pandocbounded{\includegraphics[keepaspectratio]{https://raw.githubusercontent.com/8LWXpg/typst-ansi-render/master/img/4.png}}

\subsubsection{How to add}\label{how-to-add}

Copy this into your project and use the import as
\texttt{\ ansi-render\ }

\begin{verbatim}
#import "@preview/ansi-render:0.6.1"
\end{verbatim}

\includesvg[width=0.16667in,height=0.16667in]{/assets/icons/16-copy.svg}

Check the docs for
\href{https://typst.app/docs/reference/scripting/\#packages}{more
information on how to import packages} .

\subsubsection{About}\label{about}

\begin{description}
\tightlist
\item[Author :]
8LWXpg
\item[License:]
MIT
\item[Current version:]
0.6.1
\item[Last updated:]
December 28, 2023
\item[First released:]
July 3, 2023
\item[Minimum Typst version:]
0.10.0
\item[Archive size:]
6.23 kB
\href{https://packages.typst.org/preview/ansi-render-0.6.1.tar.gz}{\pandocbounded{\includesvg[keepaspectratio]{/assets/icons/16-download.svg}}}
\item[Repository:]
\href{https://github.com/8LWXpg/typst-ansi-render}{GitHub}
\end{description}

\subsubsection{Where to report issues?}\label{where-to-report-issues}

This package is a project of 8LWXpg . Report issues on
\href{https://github.com/8LWXpg/typst-ansi-render}{their repository} .
You can also try to ask for help with this package on the
\href{https://forum.typst.app}{Forum} .

Please report this package to the Typst team using the
\href{https://typst.app/contact}{contact form} if you believe it is a
safety hazard or infringes upon your rights.

\phantomsection\label{versions}
\subsubsection{Version history}\label{version-history}

\begin{longtable}[]{@{}ll@{}}
\toprule\noalign{}
Version & Release Date \\
\midrule\noalign{}
\endhead
\bottomrule\noalign{}
\endlastfoot
0.6.1 & December 28, 2023 \\
\href{https://typst.app/universe/package/ansi-render/0.6.0/}{0.6.0} &
December 10, 2023 \\
\href{https://typst.app/universe/package/ansi-render/0.5.1/}{0.5.1} &
October 21, 2023 \\
\href{https://typst.app/universe/package/ansi-render/0.5.0/}{0.5.0} &
September 29, 2023 \\
\href{https://typst.app/universe/package/ansi-render/0.4.2/}{0.4.2} &
September 25, 2023 \\
\href{https://typst.app/universe/package/ansi-render/0.4.1/}{0.4.1} &
September 22, 2023 \\
\href{https://typst.app/universe/package/ansi-render/0.4.0/}{0.4.0} &
September 13, 2023 \\
\href{https://typst.app/universe/package/ansi-render/0.3.0/}{0.3.0} &
September 9, 2023 \\
\href{https://typst.app/universe/package/ansi-render/0.2.0/}{0.2.0} &
August 5, 2023 \\
\href{https://typst.app/universe/package/ansi-render/0.1.0/}{0.1.0} &
July 3, 2023 \\
\end{longtable}

Typst GmbH did not create this package and cannot guarantee correct
functionality of this package or compatibility with any version of the
Typst compiler or app.


\section{Package List LaTeX/splendid-mdpi.tex}
\title{typst.app/universe/package/splendid-mdpi}

\phantomsection\label{banner}
\phantomsection\label{template-thumbnail}
\pandocbounded{\includegraphics[keepaspectratio]{https://packages.typst.org/preview/thumbnails/splendid-mdpi-0.1.0-small.webp}}

\section{splendid-mdpi}\label{splendid-mdpi}

{ 0.1.0 }

An MDPI-style paper template to publish at conferences and journals

{ } Featured Template

\href{/app?template=splendid-mdpi&version=0.1.0}{Create project in app}

\phantomsection\label{readme}
Version 0.1.0

A recreation of the MDPI template shown on the typst.app homepage.

\subsection{Media}\label{media}

\includegraphics[width=0.45\linewidth,height=\textheight,keepaspectratio]{https://github.com/typst/packages/raw/main/packages/preview/splendid-mdpi/0.1.0/thumbnails/1.png}
\includegraphics[width=0.45\linewidth,height=\textheight,keepaspectratio]{https://github.com/typst/packages/raw/main/packages/preview/splendid-mdpi/0.1.0/thumbnails/2.png}

\subsection{Getting Started}\label{getting-started}

To use this template, simply import it as shown below:

\begin{Shaded}
\begin{Highlighting}[]
\NormalTok{\#import "@preview/splendid{-}mdpi:0.1.0"}

\NormalTok{\#show: splendid{-}mdpi.template.with(}
\NormalTok{  title: [Towards Swifter Interstellar Mail Delivery],}
\NormalTok{  authors: (}
\NormalTok{    (}
\NormalTok{      name: "Johanna Swift",}
\NormalTok{      department: "Primary Logistics Department",}
\NormalTok{      institution: "Delivery Institute",}
\NormalTok{      city: "Berlin",}
\NormalTok{      country: "Germany",}
\NormalTok{      mail: "swift@delivery.de",}
\NormalTok{    ),}
\NormalTok{    (}
\NormalTok{      name: "Egon Stellaris",}
\NormalTok{      department: "Communications Group",}
\NormalTok{      institution: "Space Institute",}
\NormalTok{      city: "Florence",}
\NormalTok{      country: "Italy",}
\NormalTok{      mail: "stegonaris@space.it",}
\NormalTok{    ),}
\NormalTok{    (}
\NormalTok{      name: "Oliver Liam",}
\NormalTok{      department: "Missing Letters Task Force",}
\NormalTok{      institution: "Mail Institute",}
\NormalTok{      city: "Budapest",}
\NormalTok{      country: "Hungary",}
\NormalTok{      mail: "oliver.liam@mail.hu",}
\NormalTok{    ),}
\NormalTok{  ),}
\NormalTok{  date: (}
\NormalTok{    year: 2022,}
\NormalTok{    month: "May",}
\NormalTok{    day: 17,}
\NormalTok{  ),}
\NormalTok{  keywords: (}
\NormalTok{    "Space",}
\NormalTok{    "Mail",}
\NormalTok{    "Astromail",}
\NormalTok{    "Faster{-}than{-}Light",}
\NormalTok{    "Mars",}
\NormalTok{  ),}
\NormalTok{  doi: "10:7891/120948510",}
\NormalTok{  abstract: [}
\NormalTok{    Recent advances in space{-}based document processing have enabled faster mail delivery between different planets of a solar system. Given the time it takes for a message to be transmitted from one planet to the next, its estimated that even a one{-}way trip to a distant destination could take up to one year. During these periods of interplanetary mail delivery there is a slight possibility of mail being lost in transit. This issue is considered so serious that space management employs P.I. agents to track down and retrieve lost mail. We propose A{-}Mail, a new anti{-}matter based approach that can ensure that mail loss occurring during interplanetary transit is unobservable and therefore potentially undetectable. Going even further, we extend A{-}Mail to predict problems and apply existing and new best practices to ensure the mail is delivered without any issues. We call this extension AI{-}Mail.}
\NormalTok{  ]}
\NormalTok{)}
\end{Highlighting}
\end{Shaded}

\href{/app?template=splendid-mdpi&version=0.1.0}{Create project in app}

\subsubsection{How to use}\label{how-to-use}

Click the button above to create a new project using this template in
the Typst app.

You can also use the Typst CLI to start a new project on your computer
using this command:

\begin{verbatim}
typst init @preview/splendid-mdpi:0.1.0
\end{verbatim}

\includesvg[width=0.16667in,height=0.16667in]{/assets/icons/16-copy.svg}

\subsubsection{About}\label{about}

\begin{description}
\tightlist
\item[Author :]
James R. Swift
\item[License:]
Unlicense
\item[Current version:]
0.1.0
\item[Last updated:]
July 16, 2024
\item[First released:]
July 16, 2024
\item[Archive size:]
34.2 kB
\href{https://packages.typst.org/preview/splendid-mdpi-0.1.0.tar.gz}{\pandocbounded{\includesvg[keepaspectratio]{/assets/icons/16-download.svg}}}
\item[Repository:]
\href{https://github.com/JamesxX/splendid-mdpi}{GitHub}
\item[Categor y :]
\begin{itemize}
\tightlist
\item[]
\item
  \pandocbounded{\includesvg[keepaspectratio]{/assets/icons/16-atom.svg}}
  \href{https://typst.app/universe/search/?category=paper}{Paper}
\end{itemize}
\end{description}

\subsubsection{Where to report issues?}\label{where-to-report-issues}

This template is a project of James R. Swift . Report issues on
\href{https://github.com/JamesxX/splendid-mdpi}{their repository} . You
can also try to ask for help with this template on the
\href{https://forum.typst.app}{Forum} .

Please report this template to the Typst team using the
\href{https://typst.app/contact}{contact form} if you believe it is a
safety hazard or infringes upon your rights.

\phantomsection\label{versions}
\subsubsection{Version history}\label{version-history}

\begin{longtable}[]{@{}ll@{}}
\toprule\noalign{}
Version & Release Date \\
\midrule\noalign{}
\endhead
\bottomrule\noalign{}
\endlastfoot
0.1.0 & July 16, 2024 \\
\end{longtable}

Typst GmbH did not create this template and cannot guarantee correct
functionality of this template or compatibility with any version of the
Typst compiler or app.


\section{Package List LaTeX/great-theorems.tex}
\title{typst.app/universe/package/great-theorems}

\phantomsection\label{banner}
\section{great-theorems}\label{great-theorems}

{ 0.1.1 }

Straightforward and functional theorem/proof environments.

\phantomsection\label{readme}
This package allows you to make \textbf{theorem/proof/remark/…}
blocks.

Features:

\begin{itemize}
\tightlist
\item
  supports advanced counters through both
  \href{https://typst.app/universe/package/headcount/}{headcount} and
  \href{https://typst.app/universe/package/rich-counters/}{rich-counters}
\item
  easy adjustment of style:

  \begin{itemize}
  \tightlist
  \item
    change prefix
  \item
    change how title is displayed
  \item
    change formatting of body
  \item
    change suffix
  \item
    change numbering style
  \item
    configure \emph{all} parameters of the
    \href{https://typst.app/docs/reference/layout/block/}{\texttt{\ block\ }}
    , including background color, stroke color, rounded corners, inset,
    …
  \end{itemize}
\item
  can adjust style also on individual basis (e.g. to highlight main
  theorem)
\item
  works with labels/references
\item
  sane and smart defaults
\end{itemize}

\subsection{Showcase}\label{showcase}

In the following example we use
\href{https://typst.app/universe/package/rich-counters/}{rich-counters}
to configure section-based counters. You can also use
\href{https://typst.app/universe/package/headcount/}{headcount} .

\begin{Shaded}
\begin{Highlighting}[]
\NormalTok{\#import "@preview/great{-}theorems:0.1.1": *}
\NormalTok{\#import "@preview/rich{-}counters:0.2.1": *}

\NormalTok{\#set heading(numbering: "1.1")}
\NormalTok{\#show: great{-}theorems{-}init}

\NormalTok{\#show link: text.with(fill: blue)}

\NormalTok{\#let mathcounter = rich{-}counter(}
\NormalTok{  identifier: "mathblocks",}
\NormalTok{  inherited\_levels: 1}
\NormalTok{)}

\NormalTok{\#let theorem = mathblock(}
\NormalTok{  blocktitle: "Theorem",}
\NormalTok{  counter: mathcounter,}
\NormalTok{)}

\NormalTok{\#let lemma = mathblock(}
\NormalTok{  blocktitle: "Lemma",}
\NormalTok{  counter: mathcounter,}
\NormalTok{)}

\NormalTok{\#let remark = mathblock(}
\NormalTok{  blocktitle: "Remark",}
\NormalTok{  prefix: [\_Remark.\_],}
\NormalTok{  inset: 5pt,}
\NormalTok{  fill: lime,}
\NormalTok{  radius: 5pt,}
\NormalTok{)}

\NormalTok{\#let proof = proofblock()}

\NormalTok{= Some Heading}

\NormalTok{\#theorem[}
\NormalTok{  This is some theorem.}
\NormalTok{] \textless{}mythm\textgreater{}}

\NormalTok{\#lemma[}
\NormalTok{  This is a lemma. Maybe it\textquotesingle{}s used to prove @mythm.}
\NormalTok{]}

\NormalTok{\#proof[}
\NormalTok{  This is a proof.}
\NormalTok{]}

\NormalTok{= Another Heading}

\NormalTok{\#theorem(title: "some title")[}
\NormalTok{  This is a theorem with a title.}
\NormalTok{] \textless{}thm2\textgreater{}}

\NormalTok{\#proof(of: \textless{}thm2\textgreater{})[}
\NormalTok{  This is a proof of the theorem which has a title.}
\NormalTok{]}

\NormalTok{\#remark[}
\NormalTok{  This is a remark.}
\NormalTok{  The remark box has some custom styling applied.}
\NormalTok{]}
\end{Highlighting}
\end{Shaded}

\pandocbounded{\includegraphics[keepaspectratio]{https://github.com/typst/packages/raw/main/packages/preview/great-theorems/0.1.1/example.png}}

\subsection{Usage}\label{usage}

\subsubsection{\texorpdfstring{\texttt{\ great-theorems-init\ }}{ great-theorems-init }}\label{great-theorems-init}

First, make sure to apply the following inital \texttt{\ show\ } rule to
your document:

\begin{Shaded}
\begin{Highlighting}[]
\NormalTok{\#show: great{-}theorems{-}init}
\end{Highlighting}
\end{Shaded}

This is important to make the blocks have the correct alignment and to
display references correctly.

\subsubsection{\texorpdfstring{\texttt{\ mathblock\ }}{ mathblock }}\label{mathblock}

The main constructor you will use is \texttt{\ mathblock\ } , which
allows you to construct a theorem/proof/remark/… environment in
exactly the way you like it.

Please see the showcase above for on example on how to use it. We now
list and explain all possible arguments.

\begin{itemize}
\item
  \texttt{\ blocktitle\ } (required)

  Usually something like \texttt{\ "Theorem"\ } or \texttt{\ "Lemma"\ }
  . Determines how references are displayed, and also determines the
  default \texttt{\ prefix\ } .
\item
  \texttt{\ counter\ } (default: \texttt{\ none\ } )

  If you want your \texttt{\ mathblock\ } to be counted, pass the
  counter here. Accepts either a Typst-native
  \href{https://typst.app/docs/reference/introspection/counter/}{\texttt{\ counter\ }}
  (which can be made to depend on the section with the
  \href{https://typst.app/universe/package/headcount/}{headcount}
  package) or a \texttt{\ rich-counter\ } from the
  \href{https://typst.app/universe/package/rich-counters/}{rich-counters}
  package. If you want multiple \texttt{\ mathblock\ } environments to
  share the same counter, just pass the same counter to all of them.
\item
  \texttt{\ numbering\ } (default: \texttt{\ "1.1"\ } )

  The numbering style that should be used to display the counters.

  \textbf{Note:} If you use the
  \href{https://typst.app/universe/package/headcount/}{headcount}
  package for your counters, you have to pass the
  \texttt{\ dependent-numbering\ } here.
\item
  \texttt{\ prefix\ } (default: contructed from \texttt{\ blocktitle\ }
  , bold style)

  What should be displayed before the body. If you didn’t pass a
  counter, it should just be a piece of content like
  \texttt{\ {[}*Theorem.*{]}\ } . \emph{If you passed a counter} , it
  should a function/closure, which takes the current counter value as an
  argument and returns the corresponding prefix; for example
  \texttt{\ (count)\ =\textgreater{}\ {[}*Theorem\ \#count.*{]}\ }
\item
  \texttt{\ titlix\ } (default:
  \texttt{\ title\ =\textgreater{}\ {[}(\#title){]}\ } )

  How a title should be displayed. Will be placed after the prefix if a
  title is present. Must be function which takes the title and returns
  the corresponding content that should be displayed.
\item
  \texttt{\ suffix\ } (default: \texttt{\ none\ } )

  A suffix that will be displayed after the body.
\item
  \texttt{\ bodyfmt\ } (default:
  \texttt{\ body\ =\textgreater{}\ body\ } i.e. no special formatting)

  A function that will style/transform the body. For example, if you
  want your theorem contents to be displayed in oblique style, you could
  pass \texttt{\ text.with(style:\ "oblique")\ } .
\item
  arguments for the surrounding
  \href{https://typst.app/docs/reference/layout/block/}{\texttt{\ block\ }}

  The \texttt{\ mathblock\ } , as the name suggests, is surrounded by a
  \href{https://typst.app/docs/reference/layout/block/}{\texttt{\ block\ }}
  , which can be styled to have a background color, stroke color,
  rounded corners, etc. . You can just pass all arguments that you could
  pass to a \texttt{\ block\ } also to \texttt{\ mathblock\ } , and it
  will be “passed through� the surrounding \texttt{\ block\ } . For
  example, you could write
  \texttt{\ \#let\ theorem\ =\ mathblock(...,\ fill:\ yellow,\ inset:\ 5pt)\ }
  .
\end{itemize}

So far we have discussed how you \emph{setup} your environment with
\texttt{\ \#let\ theorem\ =\ mathblock(...)\ } . Now let’s discuss how
to use the resulting \texttt{\ theorem\ } command. Again, please see the
showcase above for some examples on how to use it. We now list and
explain all possible arguments (apart from the body).

\begin{itemize}
\item
  \texttt{\ title\ } (default: \texttt{\ none\ } )

  This allows you to set a title for your theorem/lemma/…, which will
  be displayed according to \texttt{\ titlix\ } .
\item
  all the arguments from \texttt{\ mathblock\ } , except
  \texttt{\ blocktitle\ } and \texttt{\ counter\ }

  You can change all the parameters of your \texttt{\ mathblock\ } also
  on an individual basis, i.e. for each occurrence separately, by just
  passing the respective arguments, including \texttt{\ numbering\ } ,
  \texttt{\ prefix\ } , \texttt{\ titlix\ } , \texttt{\ suffix\ } ,
  \texttt{\ bodyfmt\ } , and arguments for \texttt{\ block\ } . These
  will take precedence over the global configuration.
\end{itemize}

\subsubsection{\texorpdfstring{\texttt{\ proofblock\ }}{ proofblock }}\label{proofblock}

Also a proof environment can be constructed with \texttt{\ mathblock\ }
, for example:

\begin{verbatim}
#let proof = mathblock(
  blocktitle: "Proof",
  prefix: [_Proof._],
  suffix: [#h(1fr) $square$],
)
\end{verbatim}

However, for convenience, we have made another \texttt{\ proofblock\ }
constructor. It works exactly the same as \texttt{\ mathblock\ } , the
only differences being:

\begin{itemize}
\tightlist
\item
  it has different default values for \texttt{\ blocktitle\ } ,
  \texttt{\ prefix\ } , and \texttt{\ suffix\ }
\item
  it has no \texttt{\ counter\ } and \texttt{\ numbering\ } argument
\item
  the \texttt{\ titlix\ } argument is replaced with a
  \texttt{\ prefix\_with\_of\ } argument (also consisting of a
  function), which will be used as a prefix when the constructed
  environment is used with \texttt{\ of\ } parameter
\end{itemize}

The constructed environment will have the following changes compared to
an environment constructed with \texttt{\ mathblock\ }

\begin{itemize}
\item
  the \texttt{\ title\ } argument is replaced with an \texttt{\ of\ }
  argument, which is used to denote to which theorem/lemma/… the proof
  belongs

  This can be either just content, or a label, in which case a reference
  to the label is displayed.
\end{itemize}

\subsection{FAQ}\label{faq}

\begin{itemize}
\item
  \emph{What is the difference to the ctheorems package?}

  You can achieve pretty much the same results with both packages. One
  goal of \texttt{\ great-theorems\ } was to have a cleaner
  implementation, for example by separating the counter functionality
  from the theorem block functionality. \texttt{\ ctheorems\ } also uses
  deprecated Typst functionality that will soon be removed. In the end,
  however, in comes down to personal preference, and
  \texttt{\ ctheorems\ } was certainly a big inspiration for this
  package!
\item
  \emph{How to set up the counters the way I want?}

  Please consult the documentation of
  \href{https://typst.app/universe/package/headcount/}{headcount} and
  \href{https://typst.app/universe/package/rich-counters/}{rich-counters}
  respectively, we support both packages as well as native
  \href{https://typst.app/docs/reference/introspection/counter/}{\texttt{\ counter\ }}
  s.
\item
  \emph{My theorems are all center aligned?!}

  You forgot to put the initial show rule at the start of your document:

\begin{Shaded}
\begin{Highlighting}[]
\NormalTok{\#show: great{-}theorems{-}init}
\end{Highlighting}
\end{Shaded}
\item
  \emph{My theorems break across pages, how do I stop that behavior?}

  You can pass \texttt{\ breakable:\ false\ } to \texttt{\ mathblock\ }
  to construct a non-breakable environment.
\item
  \emph{I have a default style for all my theorems/lemmas/remarks/…,
  and I’m writing boilerplate when I construct theorem/lemma/remark
  environments.}

  You can essentially define your own defaults like this:

\begin{Shaded}
\begin{Highlighting}[]
\NormalTok{\#let my\_mathblock = mathblock.with(fill: yellow, radius: 5pt, inset: 5pt)}

\NormalTok{\#let theorem = my\_mathblock(...)}
\NormalTok{\#let lemma = my\_mathblock(...)}
\NormalTok{\#let remark = my\_mathblock(...)}
\NormalTok{...}
\end{Highlighting}
\end{Shaded}
\item
  \emph{The documentation is too short or unclear… how do I do X?}

  Please just open an
  \href{https://github.com/jbirnick/typst-great-theorems/issues}{issue
  on GitHub} , and I will happily answer your question and extend the
  documentation!
\end{itemize}

\subsubsection{How to add}\label{how-to-add}

Copy this into your project and use the import as
\texttt{\ great-theorems\ }

\begin{verbatim}
#import "@preview/great-theorems:0.1.1"
\end{verbatim}

\includesvg[width=0.16667in,height=0.16667in]{/assets/icons/16-copy.svg}

Check the docs for
\href{https://typst.app/docs/reference/scripting/\#packages}{more
information on how to import packages} .

\subsubsection{About}\label{about}

\begin{description}
\tightlist
\item[Author :]
\href{https://jbirnick.net}{Johann Birnick}
\item[License:]
MIT
\item[Current version:]
0.1.1
\item[Last updated:]
October 22, 2024
\item[First released:]
October 16, 2024
\item[Archive size:]
5.32 kB
\href{https://packages.typst.org/preview/great-theorems-0.1.1.tar.gz}{\pandocbounded{\includesvg[keepaspectratio]{/assets/icons/16-download.svg}}}
\item[Repository:]
\href{https://github.com/jbirnick/typst-great-theorems}{GitHub}
\item[Discipline s :]
\begin{itemize}
\tightlist
\item[]
\item
  \href{https://typst.app/universe/search/?discipline=mathematics}{Mathematics}
\item
  \href{https://typst.app/universe/search/?discipline=computer-science}{Computer
  Science}
\item
  \href{https://typst.app/universe/search/?discipline=physics}{Physics}
\item
  \href{https://typst.app/universe/search/?discipline=engineering}{Engineering}
\item
  \href{https://typst.app/universe/search/?discipline=philosophy}{Philosophy}
\item
  \href{https://typst.app/universe/search/?discipline=education}{Education}
\end{itemize}
\item[Categor ies :]
\begin{itemize}
\tightlist
\item[]
\item
  \pandocbounded{\includesvg[keepaspectratio]{/assets/icons/16-package.svg}}
  \href{https://typst.app/universe/search/?category=components}{Components}
\item
  \pandocbounded{\includesvg[keepaspectratio]{/assets/icons/16-list-unordered.svg}}
  \href{https://typst.app/universe/search/?category=model}{Model}
\end{itemize}
\end{description}

\subsubsection{Where to report issues?}\label{where-to-report-issues}

This package is a project of Johann Birnick . Report issues on
\href{https://github.com/jbirnick/typst-great-theorems}{their
repository} . You can also try to ask for help with this package on the
\href{https://forum.typst.app}{Forum} .

Please report this package to the Typst team using the
\href{https://typst.app/contact}{contact form} if you believe it is a
safety hazard or infringes upon your rights.

\phantomsection\label{versions}
\subsubsection{Version history}\label{version-history}

\begin{longtable}[]{@{}ll@{}}
\toprule\noalign{}
Version & Release Date \\
\midrule\noalign{}
\endhead
\bottomrule\noalign{}
\endlastfoot
0.1.1 & October 22, 2024 \\
\href{https://typst.app/universe/package/great-theorems/0.1.0/}{0.1.0} &
October 16, 2024 \\
\end{longtable}

Typst GmbH did not create this package and cannot guarantee correct
functionality of this package or compatibility with any version of the
Typst compiler or app.


\section{Package List LaTeX/timeliney.tex}
\title{typst.app/universe/package/timeliney}

\phantomsection\label{banner}
\section{timeliney}\label{timeliney}

{ 0.1.0 }

Create Gantt charts in Typst.

{ } Featured Package

\phantomsection\label{readme}
Create Gantt charts automatically with Typst!

Here’s a fully-featured example:

\begin{Shaded}
\begin{Highlighting}[]
\NormalTok{\#import "@preview/timeliney:0.1.0"}

\NormalTok{\#timeliney.timeline(}
\NormalTok{  show{-}grid: true,}
\NormalTok{  \{}
\NormalTok{    import timeliney: *}
      
\NormalTok{    headerline(group(([*2023*], 4)), group(([*2024*], 4)))}
\NormalTok{    headerline(}
\NormalTok{      group(..range(4).map(n =\textgreater{} strong("Q" + str(n + 1)))),}
\NormalTok{      group(..range(4).map(n =\textgreater{} strong("Q" + str(n + 1)))),}
\NormalTok{    )}
  
\NormalTok{    taskgroup(title: [*Research*], \{}
\NormalTok{      task("Research the market", (0, 2), style: (stroke: 2pt + gray))}
\NormalTok{      task("Conduct user surveys", (1, 3), style: (stroke: 2pt + gray))}
\NormalTok{    \})}

\NormalTok{    taskgroup(title: [*Development*], \{}
\NormalTok{      task("Create mock{-}ups", (2, 3), style: (stroke: 2pt + gray))}
\NormalTok{      task("Develop application", (3, 5), style: (stroke: 2pt + gray))}
\NormalTok{      task("QA", (3.5, 6), style: (stroke: 2pt + gray))}
\NormalTok{    \})}

\NormalTok{    taskgroup(title: [*Marketing*], \{}
\NormalTok{      task("Press demos", (3.5, 7), style: (stroke: 2pt + gray))}
\NormalTok{      task("Social media advertising", (6, 7.5), style: (stroke: 2pt + gray))}
\NormalTok{    \})}

\NormalTok{    milestone(}
\NormalTok{      at: 3.75,}
\NormalTok{      style: (stroke: (dash: "dashed")),}
\NormalTok{      align(center, [}
\NormalTok{        *Conference demo*\textbackslash{}}
\NormalTok{        Dec 2023}
\NormalTok{      ])}
\NormalTok{    )}

\NormalTok{    milestone(}
\NormalTok{      at: 6.5,}
\NormalTok{      style: (stroke: (dash: "dashed")),}
\NormalTok{      align(center, [}
\NormalTok{        *App store launch*\textbackslash{}}
\NormalTok{        Aug 2024}
\NormalTok{      ])}
\NormalTok{    )}
\NormalTok{  \}}
\NormalTok{)}
\end{Highlighting}
\end{Shaded}

\pandocbounded{\includegraphics[keepaspectratio]{https://github.com/typst/packages/raw/main/packages/preview/timeliney/0.1.0/sample.png}}

\subsection{Installation}\label{installation}

Import with \texttt{\ \#import\ "@preview/timeliney:0.1.0"\ } . Then,
call the \texttt{\ timeliney.timeline\ } function.

\subsection{Documentation}\label{documentation}

See
\href{https://github.com/typst/packages/raw/main/packages/preview/timeliney/0.1.0/manual.pdf}{the
manual} !

\subsection{Changelog}\label{changelog}

\subsubsection{0.1.0}\label{section}

\begin{itemize}
\tightlist
\item
  Update CeTZ to 0.2.2 (@LordBaryhobal)
\item
  Add offset parameter
\end{itemize}

\subsubsection{How to add}\label{how-to-add}

Copy this into your project and use the import as \texttt{\ timeliney\ }

\begin{verbatim}
#import "@preview/timeliney:0.1.0"
\end{verbatim}

\includesvg[width=0.16667in,height=0.16667in]{/assets/icons/16-copy.svg}

Check the docs for
\href{https://typst.app/docs/reference/scripting/\#packages}{more
information on how to import packages} .

\subsubsection{About}\label{about}

\begin{description}
\tightlist
\item[Author :]
Pedro Alves
\item[License:]
MIT
\item[Current version:]
0.1.0
\item[Last updated:]
October 17, 2024
\item[First released:]
October 12, 2023
\item[Archive size:]
6.16 kB
\href{https://packages.typst.org/preview/timeliney-0.1.0.tar.gz}{\pandocbounded{\includesvg[keepaspectratio]{/assets/icons/16-download.svg}}}
\item[Repository:]
\href{https://github.com/pta2002/typst-timeliney}{GitHub}
\end{description}

\subsubsection{Where to report issues?}\label{where-to-report-issues}

This package is a project of Pedro Alves . Report issues on
\href{https://github.com/pta2002/typst-timeliney}{their repository} .
You can also try to ask for help with this package on the
\href{https://forum.typst.app}{Forum} .

Please report this package to the Typst team using the
\href{https://typst.app/contact}{contact form} if you believe it is a
safety hazard or infringes upon your rights.

\phantomsection\label{versions}
\subsubsection{Version history}\label{version-history}

\begin{longtable}[]{@{}ll@{}}
\toprule\noalign{}
Version & Release Date \\
\midrule\noalign{}
\endhead
\bottomrule\noalign{}
\endlastfoot
0.1.0 & October 17, 2024 \\
\href{https://typst.app/universe/package/timeliney/0.0.1/}{0.0.1} &
October 12, 2023 \\
\end{longtable}

Typst GmbH did not create this package and cannot guarantee correct
functionality of this package or compatibility with any version of the
Typst compiler or app.


\section{Package List LaTeX/koma-labeling.tex}
\title{typst.app/universe/package/koma-labeling}

\phantomsection\label{banner}
\section{koma-labeling}\label{koma-labeling}

{ 0.1.0 }

This package introduces a labeling feature to Typst, inspired by the
KOMA-Script\textquotesingle s labeling environment.

\phantomsection\label{readme}
Version 0.1.0

The koma-labeling package for Typst is inspired by the labeling
environment from the KOMA-Script bundle in LaTeX. It provides a
convenient way to create labeled lists with customizable label widths
and optional delimiters, making it perfect for creating structured
descriptions and lists in your Typst documents.

\subsection{Getting Started}\label{getting-started}

To get started with koma-labeling, simply import the package in your
Typst document and use the labeling environment to create your labeled
lists.

\begin{Shaded}
\begin{Highlighting}[]
\NormalTok{\#import "@preview/koma{-}labeling:0.1.0": labeling}

\NormalTok{\#labeling(}
\NormalTok{  (}
\NormalTok{    (lorem(1), lorem(10)),}
\NormalTok{    (lorem(2), lorem(20)),}
\NormalTok{    (lorem(3), lorem(30)),}
\NormalTok{  )}
\NormalTok{)}

\NormalTok{// or}

\NormalTok{\#labeling(}
\NormalTok{  (}
\NormalTok{    ([\#lorem(1)], [\#lorem(10)]),}
\NormalTok{    ([\#lorem(2)], [\#lorem(20)]),}
\NormalTok{    ([\#lorem(3)], [\#lorem(30)]),}
\NormalTok{  )}
\NormalTok{)}
\end{Highlighting}
\end{Shaded}

Output:

\pandocbounded{\includegraphics[keepaspectratio]{https://github.com/user-attachments/assets/bf382afe-f66d-4032-9055-f46c72a2e7dd}}

\textbf{Note:} Remember to terminate the list with a comma, even if only
one pair of items is passed.

\begin{Shaded}
\begin{Highlighting}[]
\NormalTok{\#import "@preview/koma{-}labeling:0.1.0": labeling}

\NormalTok{\#labeling(}
\NormalTok{  (}
\NormalTok{    (lorem(1), lorem(10)),  // Terminating the list with a comma is REQUIRED}
\NormalTok{  )}
\NormalTok{)}
\end{Highlighting}
\end{Shaded}

\subsection{Parameters}\label{parameters}

Although labeling is implemented using \texttt{\ tables\ } , its usage
is similar to \texttt{\ terms\ } , except that it lacks the
\texttt{\ tight\ } and \texttt{\ hanging-indent\ } parameters. If you
have any questions about the parameters for \texttt{\ labeling\ } , you
can refer to
\href{https://typst.app/docs/reference/model/terms/}{\texttt{\ terms\ }}
.

\begin{Shaded}
\begin{Highlighting}[]
\NormalTok{labeling(}
\NormalTok{  separator: content,}
\NormalTok{  indent: length,}
\NormalTok{  spacing: auto length}
\NormalTok{  pairs: ((content, content))}
\NormalTok{)}
\end{Highlighting}
\end{Shaded}

\subsubsection{separator}\label{separator}

The separator between the item and the description.

Default: \texttt{\ {[}:\#h(0.6em){]}\ }

\subsubsection{indent}\label{indent}

The indentation of each item.

Default: \texttt{\ 0pt\ }

\subsubsection{spacing}\label{spacing}

The spacing between the items of the term list.

Default: \texttt{\ auto\ }

\subsubsection{pairs}\label{pairs}

An array of \texttt{\ (item,\ description)\ } pairs.

Example:

\begin{Shaded}
\begin{Highlighting}[]
\NormalTok{\#labeling(}
\NormalTok{  (}
\NormalTok{    ([key 1],[description 1]),}
\NormalTok{    ([keyword 2],[description 2]),}
\NormalTok{  )}
\NormalTok{)}
\end{Highlighting}
\end{Shaded}

\subsection{Additional Documentation and
Acknowledgments}\label{additional-documentation-and-acknowledgments}

For more information on the koma-labeling package and its features, you
can refer to the following resources:

\begin{itemize}
\tightlist
\item
  Typst Documentation: \href{https://typst.app/docs}{Typst
  Documentation}
\item
  KOMA-Script Documentation:
  \href{https://ctan.org/pkg/koma-script}{KOMA-Script Documentation}
\end{itemize}

\subsubsection{How to add}\label{how-to-add}

Copy this into your project and use the import as
\texttt{\ koma-labeling\ }

\begin{verbatim}
#import "@preview/koma-labeling:0.1.0"
\end{verbatim}

\includesvg[width=0.16667in,height=0.16667in]{/assets/icons/16-copy.svg}

Check the docs for
\href{https://typst.app/docs/reference/scripting/\#packages}{more
information on how to import packages} .

\subsubsection{About}\label{about}

\begin{description}
\tightlist
\item[Author :]
Laniakea Kamasylvia
\item[License:]
MIT
\item[Current version:]
0.1.0
\item[Last updated:]
October 28, 2024
\item[First released:]
October 28, 2024
\item[Minimum Typst version:]
0.11.0
\item[Archive size:]
2.54 kB
\href{https://packages.typst.org/preview/koma-labeling-0.1.0.tar.gz}{\pandocbounded{\includesvg[keepaspectratio]{/assets/icons/16-download.svg}}}
\item[Categor y :]
\begin{itemize}
\tightlist
\item[]
\item
  \pandocbounded{\includesvg[keepaspectratio]{/assets/icons/16-code.svg}}
  \href{https://typst.app/universe/search/?category=scripting}{Scripting}
\end{itemize}
\end{description}

\subsubsection{Where to report issues?}\label{where-to-report-issues}

This package is a project of Laniakea Kamasylvia . You can also try to
ask for help with this package on the
\href{https://forum.typst.app}{Forum} .

Please report this package to the Typst team using the
\href{https://typst.app/contact}{contact form} if you believe it is a
safety hazard or infringes upon your rights.

\phantomsection\label{versions}
\subsubsection{Version history}\label{version-history}

\begin{longtable}[]{@{}ll@{}}
\toprule\noalign{}
Version & Release Date \\
\midrule\noalign{}
\endhead
\bottomrule\noalign{}
\endlastfoot
0.1.0 & October 28, 2024 \\
\end{longtable}

Typst GmbH did not create this package and cannot guarantee correct
functionality of this package or compatibility with any version of the
Typst compiler or app.


\section{Package List LaTeX/teig.tex}
\title{typst.app/universe/package/teig}

\phantomsection\label{banner}
\section{teig}\label{teig}

{ 0.1.0 }

Calculate eigenvalues of matrices

\phantomsection\label{readme}
This package provides an \texttt{\ eigenvalue\ } function that
calculates the eigenvalues of a matrix.

\begin{Shaded}
\begin{Highlighting}[]
\NormalTok{\#import "@preview/teig:0.1.0": eigenvalues}

\NormalTok{\#let data = (}
\NormalTok{  (1, 2, 3),}
\NormalTok{  (4, 5, 6),}
\NormalTok{  (7, 8, 9),}
\NormalTok{)}

\NormalTok{\#let evals = eigenvalues(data)}

\NormalTok{The eigenvalues of}
\NormalTok{$}
\NormalTok{  \#math.mat(..data)}
\NormalTok{$}
\NormalTok{are approximately}

\NormalTok{$}
\NormalTok{  \#math.vec(..evals.map(x =\textgreater{} str(calc.round(x, digits: 3))))}
\NormalTok{$}
\end{Highlighting}
\end{Shaded}

\pandocbounded{\includegraphics[keepaspectratio]{https://github.com/typst/packages/raw/main/packages/preview/teig/0.1.0/example.png}}

\subsubsection{How to add}\label{how-to-add}

Copy this into your project and use the import as \texttt{\ teig\ }

\begin{verbatim}
#import "@preview/teig:0.1.0"
\end{verbatim}

\includesvg[width=0.16667in,height=0.16667in]{/assets/icons/16-copy.svg}

Check the docs for
\href{https://typst.app/docs/reference/scripting/\#packages}{more
information on how to import packages} .

\subsubsection{About}\label{about}

\begin{description}
\tightlist
\item[Author :]
SolidTux
\item[License:]
MIT
\item[Current version:]
0.1.0
\item[Last updated:]
October 2, 2024
\item[First released:]
October 2, 2024
\item[Archive size:]
62.2 kB
\href{https://packages.typst.org/preview/teig-0.1.0.tar.gz}{\pandocbounded{\includesvg[keepaspectratio]{/assets/icons/16-download.svg}}}
\item[Repository:]
\href{https://gitlab.com/SolidTux/teig}{GitLab}
\item[Discipline :]
\begin{itemize}
\tightlist
\item[]
\item
  \href{https://typst.app/universe/search/?discipline=mathematics}{Mathematics}
\end{itemize}
\item[Categor ies :]
\begin{itemize}
\tightlist
\item[]
\item
  \pandocbounded{\includesvg[keepaspectratio]{/assets/icons/16-code.svg}}
  \href{https://typst.app/universe/search/?category=scripting}{Scripting}
\item
  \pandocbounded{\includesvg[keepaspectratio]{/assets/icons/16-hammer.svg}}
  \href{https://typst.app/universe/search/?category=utility}{Utility}
\end{itemize}
\end{description}

\subsubsection{Where to report issues?}\label{where-to-report-issues}

This package is a project of SolidTux . Report issues on
\href{https://gitlab.com/SolidTux/teig}{their repository} . You can also
try to ask for help with this package on the
\href{https://forum.typst.app}{Forum} .

Please report this package to the Typst team using the
\href{https://typst.app/contact}{contact form} if you believe it is a
safety hazard or infringes upon your rights.

\phantomsection\label{versions}
\subsubsection{Version history}\label{version-history}

\begin{longtable}[]{@{}ll@{}}
\toprule\noalign{}
Version & Release Date \\
\midrule\noalign{}
\endhead
\bottomrule\noalign{}
\endlastfoot
0.1.0 & October 2, 2024 \\
\end{longtable}

Typst GmbH did not create this package and cannot guarantee correct
functionality of this package or compatibility with any version of the
Typst compiler or app.


\section{Package List LaTeX/brilliant-cv.tex}
\title{typst.app/universe/package/brilliant-cv}

\phantomsection\label{banner}
\phantomsection\label{template-thumbnail}
\pandocbounded{\includegraphics[keepaspectratio]{https://packages.typst.org/preview/thumbnails/brilliant-cv-2.0.3-small.webp}}

\section{brilliant-cv}\label{brilliant-cv}

{ 2.0.3 }

ðŸ'¼ another CV template for your job application, yet powered by Typst
and more

\href{/app?template=brilliant-cv&version=2.0.3}{Create project in app}

\phantomsection\label{readme}
\hfill\break

\begin{quote}
If my work helps you drift through tedious job seeking journey, don’t
hesitate to think about
\href{https://github.com/sponsors/mintyfrankie}{buying me a Coke Zero}
… or a lot of them! 🥤
\end{quote}

\textbf{Brilliant CV} is a
\href{https://github.com/typst/typst}{\textbf{Typst}} template for
making \textbf{Résume} , \textbf{CV} or \textbf{Cover Letter} inspired
by the famous LaTeX CV template
\href{https://github.com/posquit0/Awesome-CV}{\textbf{Awesome-CV}} .

\subsection{Features}\label{features}

\textbf{1. Separation of style and content}

\begin{quote}
Version control your CV entries in the \texttt{\ modules\ } folder,
without touching the styling and typesetting of your CV / Cover Letter
\emph{(hey, I am not talking about \textbf{Macrohard Word} , you know)}
\end{quote}

\textbf{2. Quick twitches on the visual}

\begin{quote}
Add company logos, put your shiny company name or your coolest title at
the first line globally or per-document needs
\end{quote}

\textbf{3. Multilingual support}

\begin{quote}
Centrally store your multilingual CVs (English + French + German +
Chinese + Japanese if you are superb) and change output language in a
blink
\end{quote}

\textbf{\emph{(NEW)} 4. AI Prompt and Keywords Injection}

\begin{quote}
Fight against the abuse of ATS system or GenAI screening by injecting
invisible AI prompt or keyword list automatically.
\end{quote}

\subsection{Preview}\label{preview}

\begin{longtable}[]{@{}cc@{}}
\toprule\noalign{}
CV & Cover Letter \\
\midrule\noalign{}
\endhead
\bottomrule\noalign{}
\endlastfoot
\pandocbounded{\includegraphics[keepaspectratio]{https://github.com/mintyfrankie/mintyfrankie/assets/77310871/94f5fb5c-03d0-4912-b6d6-11ee7d27a9a3}}
&
\pandocbounded{\includegraphics[keepaspectratio]{https://github.com/mintyfrankie/brilliant-CV/assets/77310871/b4e74cdd-6b8d-4414-b52f-13cd6ba94315}} \\
\end{longtable}

\begin{longtable}[]{@{}cc@{}}
\toprule\noalign{}
CV ( \emph{French, red, no photo} ) & Cover Letter ( \emph{French, red}
) \\
\midrule\noalign{}
\endhead
\bottomrule\noalign{}
\endlastfoot
\pandocbounded{\includegraphics[keepaspectratio]{https://github.com/mintyfrankie/brilliant-CV/assets/77310871/fed7b66c-728e-4213-aa58-aa26db3b1362}}
&
\pandocbounded{\includegraphics[keepaspectratio]{https://github.com/mintyfrankie/brilliant-CV/assets/77310871/65ca65b0-c0e1-4fe8-b797-8a5e0bea4b1c}} \\
\end{longtable}

\begin{longtable}[]{@{}cc@{}}
\toprule\noalign{}
CV ( \emph{Chinese, green} ) & Cover Letter ( \emph{Chinese, green} ) \\
\midrule\noalign{}
\endhead
\bottomrule\noalign{}
\endlastfoot
\pandocbounded{\includegraphics[keepaspectratio]{https://github.com/mintyfrankie/brilliant-CV/assets/77310871/cb9c16f5-8ad7-4256-92fe-089c108d07f5}}
&
\pandocbounded{\includegraphics[keepaspectratio]{https://github.com/mintyfrankie/brilliant-CV/assets/77310871/a5a97be2-87e2-43fe-b605-f862a0d600d7}} \\
\end{longtable}

\subsection{Usage}\label{usage}

\begin{quote}
If you are using Typst online editor, you don’t have to follow local
development steps.
\end{quote}

\subsubsection{1. Install Fonts}\label{install-fonts}

In order to make Typst render correctly, you will have to install the
required fonts
\href{https://fonts.google.com/specimen/Roboto}{\textbf{Roboto}} ,
\href{https://fonts.google.com/specimen/Source+Sans+3}{\textbf{Source
Sans Pro}} (or \textbf{Source Sans 3} ) as well as
\href{https://fontawesome.com/download}{Fontawesome 6} in your local
system.

\emph{NOTE: For online editor, Source Sans Pro are already included;
however you will still have to manually upload the \texttt{\ .otf\ } or
\texttt{\ .ttf\ } files of \textbf{Fontawesome} and \textbf{Roboto} to
your project, by creating a folder \texttt{\ fonts\ } and put all the
\texttt{\ otf\ } files there. See
\href{https://github.com/typst/webapp-issues/issues/401}{Issue}}

\subsubsection{2. Check Documentation}\label{check-documentation}

A
\href{https://mintyfrankie.github.io/brilliant-CV/docs.pdf}{documentation}
on CV functions is provided for reference.

\subsubsection{3. Bootstrap Template}\label{bootstrap-template}

You have two ways to bootstrap the template, according to your need and
tech-savvy level.

\paragraph{3.1 With Typst CLI}\label{with-typst-cli}

In your local system, just working like \texttt{\ git\ clone\ } ,
boostrap the template using this command:

\begin{Shaded}
\begin{Highlighting}[]
\ExtensionTok{typst}\NormalTok{ init @preview/brilliant{-}cv:}
\end{Highlighting}
\end{Shaded}

Replace the \texttt{\ \textless{}version\textgreater{}\ } with the
latest or any releases (after 2.0.0).

\paragraph{\texorpdfstring{3.2 With \texttt{\ utpm\ } pakcage
manager}{3.2 With  utpm  pakcage manager}}\label{with-utpm-pakcage-manager}

\href{https://github.com/Thumuss/utpm}{utpm} is a WIP packager manager
for Typst. Install it with official instructions.

Git clone then this repository on your local system, and within the
workspace, run \texttt{\ utpm\ workspace\ link\ -\/-force\ } .

You will have to take care of templating by yourself, though.

\subsubsection{4. Compile Files}\label{compile-files}

Adapt the \texttt{\ metadata.toml\ } to suit your needs, then
\texttt{\ typst\ c\ cv.typ\ } to get your first CV!

\subsubsection{5. Beyond}\label{beyond}

It is recommended to:

\begin{enumerate}
\tightlist
\item
  Use \texttt{\ git\ } to manage your project, as it helps trace your
  changes and version control your CV.
\item
  Use \texttt{\ typstyle\ } and \texttt{\ pre-commit\ } to help you
  format your CV.
\item
  Use \texttt{\ typos\ } to check typos in your CV if your main locale
  is English.
\item
  (Advanced) Use \texttt{\ LTex\ } in your favorite code editor to check
  grammars and get language suggestions.
\end{enumerate}

\subsection{How to upgrade version}\label{how-to-upgrade-version}

For the time being, upgrade can be achieved by manually “find and
replace� the import statements in batch in your favorite IDE. For
example:

\begin{Shaded}
\begin{Highlighting}[]
\NormalTok{\#import "@preview/brilliant{-}cv:2.0.0" {-}\textgreater{} \#import "@preview/brilliant{-}cv:2.0.3"}
\end{Highlighting}
\end{Shaded}

\textbf{Make sure you read the release notes to notice any breaking
changes. We estimate that there would still be some as Typst has not
reached to a stable release neither.}

\subsection{\texorpdfstring{Migration from
\texttt{\ v1\ }}{Migration from  v1 }}\label{migration-from-v1}

\begin{quote}
The version \texttt{\ v1\ } is now deprecated, due to the compliance to
Typst Packages standard. However, if you want to continue to develop on
the older version, please refer to the \texttt{\ v1-legacy\ } branch.
\end{quote}

With an existing CV project using the \texttt{\ v1\ } version of the
template, a migration is needed, including replacing some files / some
content in certain files.

\begin{enumerate}
\tightlist
\item
  Delete \texttt{\ brilliant-CV\ } folder, \texttt{\ .gitmodules\ } .
  (Future package management will directly be managed by Typst)
\item
  Migrate all the config on \texttt{\ metadata.typ\ } by creating a new
  \texttt{\ metadata.toml\ } . Follow the example toml file in the repo,
  it is rather straightforward to migrate.
\item
  For \texttt{\ cv.typ\ } and \texttt{\ letter.typ\ } , copy the new
  files from the repo, and adapt the modules you have in your project.
\item
  For the module files in \texttt{\ /modules\_*\ } folders:

  \begin{enumerate}
  \tightlist
  \item
    Delete the old import
    \texttt{\ \#import\ "../brilliant-CV/template.typ":\ *\ } , and
    replace it by the import statements in the new template files.
  \item
    Due to the Typst path handling mecanism, one cannot directly pass
    the path string to some functions anymore. This concerns, for
    example, the \texttt{\ logo\ } argument in \texttt{\ cvEntry\ } ,
    but also on \texttt{\ cvPublication\ } as well. Some parameter names
    were changed, but most importantly, \textbf{you should pass a
    function instead of a string (i.e. \texttt{\ image("logo.png")\ }
    instead of \texttt{\ "logo.png"\ } ).} Refer to new template files
    for reference.
  \end{enumerate}
\item
  You might need to install \texttt{\ Roboto\ } and
  \texttt{\ Source\ Sans\ Pro\ } on your local system now, as new Typst
  package discourages including these large files.
\item
  Run \texttt{\ typst\ c\ cv.typ\ } without passing the
  \texttt{\ font-path\ } flag. All should be good now, congrats!
\end{enumerate}

Feel free to raise an issue for more assistance should you encounter a
problem that you cannot solve on your own :)

\subsection{Credit}\label{credit}

\begin{itemize}
\tightlist
\item
  \href{https://github.com/typst/typst}{\textbf{Typst}} is a newborn,
  open source and simple typesetting engine that offers a better
  scripting experience than
  \href{https://www.latex-project.org/}{\textbf{LaTeX}} .
\item
  \href{https://github.com/posquit0/Awesome-CV}{\textbf{Awesome-CV}} is
  the original LaTeX CV template from which this project is heavily
  inspired. Thanks \href{https://github.com/posquit0}{posquit0} for your
  excellent work!
\item
  \href{https://fontawesome.com/}{\textbf{Font Awesome}} is a
  comprehensive icon library and toolkit used widely in web projects for
  its vast array of icons and ease of integration.
\item
  \href{https://github.com/Mc-Zen/tidy}{\textbf{tidy}} is a package that
  generates documentation directly in Typst for your Typst modules. Keep
  it tidy!
\end{itemize}

\href{/app?template=brilliant-cv&version=2.0.3}{Create project in app}

\subsubsection{How to use}\label{how-to-use}

Click the button above to create a new project using this template in
the Typst app.

You can also use the Typst CLI to start a new project on your computer
using this command:

\begin{verbatim}
typst init @preview/brilliant-cv:2.0.3
\end{verbatim}

\includesvg[width=0.16667in,height=0.16667in]{/assets/icons/16-copy.svg}

\subsubsection{About}\label{about}

\begin{description}
\tightlist
\item[Author :]
\href{https://github.com/mintyfrankie}{Yunan Wang}
\item[License:]
Apache-2.0
\item[Current version:]
2.0.3
\item[Last updated:]
October 10, 2024
\item[First released:]
July 17, 2024
\item[Minimum Typst version:]
0.11.0
\item[Archive size:]
1.29 MB
\href{https://packages.typst.org/preview/brilliant-cv-2.0.3.tar.gz}{\pandocbounded{\includesvg[keepaspectratio]{/assets/icons/16-download.svg}}}
\item[Repository:]
\href{https://github.com/mintyfrankie/brilliant-CV}{GitHub}
\item[Categor ies :]
\begin{itemize}
\tightlist
\item[]
\item
  \pandocbounded{\includesvg[keepaspectratio]{/assets/icons/16-user.svg}}
  \href{https://typst.app/universe/search/?category=cv}{CV}
\item
  \pandocbounded{\includesvg[keepaspectratio]{/assets/icons/16-world.svg}}
  \href{https://typst.app/universe/search/?category=languages}{Languages}
\item
  \pandocbounded{\includesvg[keepaspectratio]{/assets/icons/16-layout.svg}}
  \href{https://typst.app/universe/search/?category=layout}{Layout}
\end{itemize}
\end{description}

\subsubsection{Where to report issues?}\label{where-to-report-issues}

This template is a project of Yunan Wang . Report issues on
\href{https://github.com/mintyfrankie/brilliant-CV}{their repository} .
You can also try to ask for help with this template on the
\href{https://forum.typst.app}{Forum} .

Please report this template to the Typst team using the
\href{https://typst.app/contact}{contact form} if you believe it is a
safety hazard or infringes upon your rights.

\phantomsection\label{versions}
\subsubsection{Version history}\label{version-history}

\begin{longtable}[]{@{}ll@{}}
\toprule\noalign{}
Version & Release Date \\
\midrule\noalign{}
\endhead
\bottomrule\noalign{}
\endlastfoot
2.0.3 & October 10, 2024 \\
\href{https://typst.app/universe/package/brilliant-cv/2.0.2/}{2.0.2} &
August 21, 2024 \\
\href{https://typst.app/universe/package/brilliant-cv/2.0.1/}{2.0.1} &
July 29, 2024 \\
\href{https://typst.app/universe/package/brilliant-cv/2.0.0/}{2.0.0} &
July 17, 2024 \\
\end{longtable}

Typst GmbH did not create this template and cannot guarantee correct
functionality of this template or compatibility with any version of the
Typst compiler or app.


\section{Package List LaTeX/ofbnote.tex}
\title{typst.app/universe/package/ofbnote}

\phantomsection\label{banner}
\phantomsection\label{template-thumbnail}
\pandocbounded{\includegraphics[keepaspectratio]{https://packages.typst.org/preview/thumbnails/ofbnote-0.2.0-small.webp}}

\section{ofbnote}\label{ofbnote}

{ 0.2.0 }

A document template using French Office for biodiversity design
guidelines

\href{/app?template=ofbnote&version=0.2.0}{Create project in app}

\phantomsection\label{readme}
This is a Typst template to help formatting documents according to the
French office for biodiversity design guidelines.

\subsection{Usage}\label{usage}

You can use the CLI to kick this project off using the command

\begin{verbatim}
typst init @preview/ofbnote
\end{verbatim}

Typst will create a new directory with all the files needed to get you
started.

\subsection{Configuration}\label{configuration}

This template exports the \texttt{\ ofbnote\ } function with one named
argument called \texttt{\ meta\ } which should be a dictionary of
metadata for the document. The \texttt{\ meta\ } dictionary can contain
the following fields:

\begin{itemize}
\tightlist
\item
  \texttt{\ title\ } : The document’s title as a string or content.
\item
  \texttt{\ authors\ } : The document’s author(s) as a string.
\item
  \texttt{\ date\ } : The document’s date as a string or content.
\item
  \texttt{\ version\ } : The document’s version as a string.
\end{itemize}

It may contains other values, but they have no effect on the final
document.

The function also accepts a single, positional argument for the body of
the paper.

The template will initialize your package with a sample call to the
\texttt{\ ofbnote\ } function in a show rule. If you want to change an
existing project to use this template, you can add a show rule like this
at the top of your file:

\begin{Shaded}
\begin{Highlighting}[]
\NormalTok{\#import "@preview/ofbnote:0.2.0": *}

\NormalTok{\#show: ofbnote.with( meta:(}
\NormalTok{  title: "My note",}
\NormalTok{  authors: "Me",}
\NormalTok{  date: "March 23rd, 2023",}
\NormalTok{  version: "1.0"}
\NormalTok{))}

\NormalTok{// Your content goes below.}
\end{Highlighting}
\end{Shaded}

\href{/app?template=ofbnote&version=0.2.0}{Create project in app}

\subsubsection{How to use}\label{how-to-use}

Click the button above to create a new project using this template in
the Typst app.

You can also use the Typst CLI to start a new project on your computer
using this command:

\begin{verbatim}
typst init @preview/ofbnote:0.2.0
\end{verbatim}

\includesvg[width=0.16667in,height=0.16667in]{/assets/icons/16-copy.svg}

\subsubsection{About}\label{about}

\begin{description}
\tightlist
\item[Author :]
François Hissel
\item[License:]
MIT-0
\item[Current version:]
0.2.0
\item[Last updated:]
August 12, 2024
\item[First released:]
August 12, 2024
\item[Minimum Typst version:]
0.11.0
\item[Archive size:]
9.14 kB
\href{https://packages.typst.org/preview/ofbnote-0.2.0.tar.gz}{\pandocbounded{\includesvg[keepaspectratio]{/assets/icons/16-download.svg}}}
\item[Categor ies :]
\begin{itemize}
\tightlist
\item[]
\item
  \pandocbounded{\includesvg[keepaspectratio]{/assets/icons/16-envelope.svg}}
  \href{https://typst.app/universe/search/?category=office}{Office}
\item
  \pandocbounded{\includesvg[keepaspectratio]{/assets/icons/16-speak.svg}}
  \href{https://typst.app/universe/search/?category=report}{Report}
\end{itemize}
\end{description}

\subsubsection{Where to report issues?}\label{where-to-report-issues}

This template is a project of François Hissel . You can also try to ask
for help with this template on the \href{https://forum.typst.app}{Forum}
.

Please report this template to the Typst team using the
\href{https://typst.app/contact}{contact form} if you believe it is a
safety hazard or infringes upon your rights.

\phantomsection\label{versions}
\subsubsection{Version history}\label{version-history}

\begin{longtable}[]{@{}ll@{}}
\toprule\noalign{}
Version & Release Date \\
\midrule\noalign{}
\endhead
\bottomrule\noalign{}
\endlastfoot
0.2.0 & August 12, 2024 \\
\end{longtable}

Typst GmbH did not create this template and cannot guarantee correct
functionality of this template or compatibility with any version of the
Typst compiler or app.


\section{Package List LaTeX/quill.tex}
\title{typst.app/universe/package/quill}

\phantomsection\label{banner}
\section{quill}\label{quill}

{ 0.5.0 }

Effortlessly create quantum circuit diagrams.

{ } Featured Package

\phantomsection\label{readme}
\pandocbounded{\includegraphics[keepaspectratio]{https://github.com/user-attachments/assets/bf6bfe99-6667-41b0-9b45-13c73ed18590}}

\href{https://typst.app/universe/package/quill}{\pandocbounded{\includegraphics[keepaspectratio]{https://img.shields.io/badge/dynamic/toml?url=https\%3A\%2F\%2Fraw.githubusercontent.com\%2FMc-Zen\%2Fquill\%2Fv0.5.0\%2Ftypst.toml&query=\%24.package.version&prefix=v&logo=typst&label=package&color=239DAD}}}
\href{https://github.com/Mc-Zen/quill/actions/workflows/run_tests.yml}{\pandocbounded{\includesvg[keepaspectratio]{https://github.com/Mc-Zen/quill/actions/workflows/run_tests.yml/badge.svg}}}
\href{https://github.com/Mc-Zen/quill/blob/main/LICENSE}{\pandocbounded{\includegraphics[keepaspectratio]{https://img.shields.io/badge/license-MIT-blue}}}
\href{https://github.com/Mc-Zen/quill/releases/download/v0.5.0/quill-guide.pdf}{\pandocbounded{\includegraphics[keepaspectratio]{https://img.shields.io/badge/manual-.pdf-purple}}}

\textbf{Quill} is a package for creating quantum circuit diagrams in
\href{https://typst.app/}{Typst} .

\emph{Note, that this package is in beta and may still be undergoing
breaking changes. As new features like data types and scoped functions
will be added to Typst, this package will be adapted to profit from the
new paradigms.}

\emph{Meanwhile, we suggest importing everything from the package in a
local scope to avoid polluting the global namespace (see example
below).}

\begin{itemize}
\tightlist
\item
  \href{https://github.com/typst/packages/raw/main/packages/preview/quill/0.5.0/\#basic-usage}{\textbf{Usage}}
  \emph{quick introduction}
\item
  \href{https://github.com/typst/packages/raw/main/packages/preview/quill/0.5.0/\#cheat-sheet}{\textbf{Cheat
  sheet}} \emph{gallery for quickly viewing all kinds of gates}
\item
  \href{https://github.com/typst/packages/raw/main/packages/preview/quill/0.5.0/\#tequila}{\textbf{Tequila}}
  \emph{building (sub-)circuits in a way similar to QASM or Qiskit}
\item
  \href{https://github.com/typst/packages/raw/main/packages/preview/quill/0.5.0/\#examples}{\textbf{Examples}}
\item
  \href{https://github.com/typst/packages/raw/main/packages/preview/quill/0.5.0/\#changelog}{\textbf{Changelog}}
\end{itemize}

\subsection{Basic usage}\label{basic-usage}

The function \texttt{\ quantum-circuit()\ } takes any number of
positional gates and works somewhat similarly to the built-int Typst
functions \texttt{\ table()\ } or \texttt{\ grid()\ } . A variety of
different gate and instruction commands are available for adding
elements and integers can be used to produce any number of empty cells
(filled with the current wire style). A new wire is started by adding a
\texttt{\ {[}\textbackslash{}\ {]}\ } item.

\begin{Shaded}
\begin{Highlighting}[]
\NormalTok{\#\{}
\NormalTok{  import "@preview/quill:0.5.0": *}

\NormalTok{  quantum{-}circuit(}
\NormalTok{    lstick($|0〉$), $H$, ctrl(1), rstick($(|00〉+|11〉)/√2$, n: 2), [\textbackslash{} ],}
\NormalTok{    lstick($|0〉$), 1, targ(), 1}
\NormalTok{  )}
\NormalTok{\}}
\end{Highlighting}
\end{Shaded}

\pandocbounded{\includegraphics[keepaspectratio]{https://github.com/user-attachments/assets/53d0c581-ab62-44e3-abf5-5497993d7aba}}

Plain quantum gates â€'' such as a Hadamard gate â€'' can be written
with the shorthand notation \texttt{\ \$H\$\ } instead of the more
lengthy \texttt{\ gate(\$H\$)\ } . The latter offers more options,
however.

Refer to the
\href{https://github.com/Mc-Zen/quill/releases/download/v0.5.0/quill-guide.pdf}{user
guide} for a full documentation of this package. You can also look up
the documentation of any function by calling the help module, e.g.,
\texttt{\ help("gate")\ } in order to print the signature and
description of the \texttt{\ gate\ } command, just where you are
currently typing (powered by \href{https://github.com/Mc-Zen/tidy}{tidy}
).

\subsection{Cheat Sheet}\label{cheat-sheet}

Instead of listing every featured gate (as is done in the
\href{https://github.com/Mc-Zen/quill/releases/download/v0.5.0/quill-guide.pdf}{user
guide} ), this gallery quickly showcases a large selection of possible
gates and decorations that can be added to any quantum circuit.

\pandocbounded{\includegraphics[keepaspectratio]{https://github.com/user-attachments/assets/29987e5b-c373-4cd6-86a0-58e27d778fb1}}

\subsection{Tequila}\label{tequila}

\emph{Tequila} is a submodule that adds a completely different way of
building circuits.

\begin{Shaded}
\begin{Highlighting}[]
\NormalTok{\#import "@preview/quill:0.5.0" as quill: tequila as tq}

\NormalTok{\#quill.quantum{-}circuit(}
\NormalTok{  ..tq.build(}
\NormalTok{    tq.h(0),}
\NormalTok{    tq.cx(0, 1),}
\NormalTok{    tq.cx(0, 2),}
\NormalTok{  ),}
\NormalTok{  quill.gategroup(x: 2, y: 0, 3, 2)}
\NormalTok{)}
\end{Highlighting}
\end{Shaded}

This is similar to how \emph{QASM} and \emph{Qiskit} work: gates are
successively applied to the circuit which is then layed out
automatically by packing gates as tightly as possible. We start by
calling the \texttt{\ tq.build()\ } function and filling it with quantum
operations. This returns a collection of gates which we expand into the
circuit with the \texttt{\ ..\ } syntax. Now, we still have the option
to add annotations, groups, slices, or even more gates via manual
placement.

The syntax works analog to Qiskit. Available gates are \texttt{\ x\ } ,
\texttt{\ y\ } , \texttt{\ z\ } , \texttt{\ h\ } , \texttt{\ s\ } ,
\texttt{\ sdg\ } , \texttt{\ sx\ } , \texttt{\ sxdg\ } , \texttt{\ t\ }
, \texttt{\ tdg\ } , \texttt{\ p\ } , \texttt{\ rx\ } , \texttt{\ ry\ }
, \texttt{\ rz\ } , \texttt{\ u\ } , \texttt{\ cx\ } , \texttt{\ cz\ } ,
and \texttt{\ swap\ } . With \texttt{\ barrier\ } , an invisible barrier
can be inserted to prevent gates on different qubits to be packed
tightly. Finally, with \texttt{\ tq.gate\ } and \texttt{\ tq.mqgate\ } ,
a generic gate can be created. These two accept the same styling
arguments as the normal \texttt{\ gate\ } (or \texttt{\ mqgate\ } ).

Also like Qiskit, all qubit arguments support ranges, e.g.,
\texttt{\ tq.h(range(5))\ } adds a Hadamard gate on the first five
qubits and \texttt{\ tq.cx((0,\ 1),\ (1,\ 2))\ } adds two CX gates: one
from qubit 0 to 1 and one from qubit 1 to 2.

With Tequila, it is easy to build templates for quantum circuits and to
compose circuits of various building blocks. For this purpose,
\texttt{\ tq.build()\ } and the built-in templates all feature optional
\texttt{\ x\ } and \texttt{\ y\ } arguments to allow placing a
subcircuit at an arbitrary position of the circuit. As an example,
Tequila provides a \texttt{\ tq.graph-state()\ } template for quickly
drawing graph state preparation circuits.

The following example demonstrates how to compose multiple subcircuits.

\begin{Shaded}
\begin{Highlighting}[]
\NormalTok{\#import tequila as tq}

\NormalTok{\#quantum{-}circuit(}
\NormalTok{  ..tq.graph{-}state((0, 1), (1, 2)),}
\NormalTok{  ..tq.build(y: 3, }
\NormalTok{      tq.p($pi$, 0), }
\NormalTok{      tq.cx(0, (1, 2)), }
\NormalTok{    ),}
\NormalTok{  ..tq.graph{-}state(x: 6, y: 2, invert: true, (0, 1), (0, 2)),}
\NormalTok{  gategroup(x: 1, 3, 3),}
\NormalTok{  gategroup(x: 1, y: 3, 3, 3),}
\NormalTok{  gategroup(x: 6, y: 2, 3, 3),}
\NormalTok{  slice(x: 5)}
\NormalTok{)}
\end{Highlighting}
\end{Shaded}

\pandocbounded{\includegraphics[keepaspectratio]{https://github.com/user-attachments/assets/41c8d60a-1a5e-4d0b-a7f4-82756423f5a8}}

\subsection{Examples}\label{examples}

Some show-off examples, loosely replicating figures from
\href{https://www.cambridge.org/highereducation/books/quantum-computation-and-quantum-information/01E10196D0A682A6AEFFEA52D53BE9AE\#overview}{Quantum
Computation and Quantum Information by M. Nielsen and I. Chuang} . The
code for these examples can be found in the
\href{https://github.com/Mc-Zen/quill/tree/v0.5.0/examples}{example
folder} or in the
\href{https://github.com/Mc-Zen/quill/releases/download/v0.5.0/quill-guide.pdf}{user
guide} .

\pandocbounded{\includegraphics[keepaspectratio]{https://github.com/user-attachments/assets/f38abeb9-fc2f-4be4-9592-7932e07af764}}

\pandocbounded{\includegraphics[keepaspectratio]{https://github.com/user-attachments/assets/6e947f71-67dc-4522-936e-6d9b795a1bba}}

\pandocbounded{\includegraphics[keepaspectratio]{https://github.com/user-attachments/assets/3fc92cd0-915e-4c5e-893d-63dac6f83ade}}

\subsection{Contribution}\label{contribution}

If you spot an issue or have a suggestion, you are invited to
\href{https://github.com/Mc-Zen/quill/issues}{post it} or to contribute.
In
\href{https://github.com/Mc-Zen/quill/tree/v0.5.0/docs/architecture.md}{architecture.md}
, you can also find a description of the algorithm that forms the base
of \texttt{\ quantum-circuit()\ } .

\subsection{Tests}\label{tests}

This package uses
\href{https://github.com/tingerrr/typst-test/}{typst-test} for running
\href{https://github.com/Mc-Zen/quill/tree/v0.5.0/tests/}{tests} .

\subsection{Changelog}\label{changelog}

\subsubsection{v0.5.0}\label{v0.5.0}

\begin{itemize}
\tightlist
\item
  Added support for multi-controlled gates with Tequila.
\item
  Switched to using \texttt{\ context\ } instead of the now deprecated
  \texttt{\ style()\ } for measurement. Note: Starting with this
  version, Typst 0.11.0 or higher is required.
\end{itemize}

\subsubsection{v0.4.0}\label{v0.4.0}

\begin{itemize}
\tightlist
\item
  Alternative model for creating and composing circuits:
  \href{https://github.com/typst/packages/raw/main/packages/preview/quill/0.5.0/\#tequila}{Tequila}
  .
\end{itemize}

\subsubsection{v0.3.0}\label{v0.3.0}

\begin{itemize}
\tightlist
\item
  New features

  \begin{itemize}
  \tightlist
  \item
    Enable manual placement of gates,
    \texttt{\ gate(\$X\$,\ x:\ 3,\ y:\ 1)\ } , similar to built-in
    \texttt{\ table()\ } in addition to automatic placement. This works
    for most elements, not only gates.
  \item
    Add parameter \texttt{\ pad\ } to \texttt{\ lstick()\ } and
    \texttt{\ rstick()\ } .
  \item
    Add parameter \texttt{\ fill-wires\ } to
    \texttt{\ quantum-circuit()\ } . All wires are filled unto the end
    (determined by the longest wire) by default (breaking change
    âš~ï¸?). This behavior can be reverted by setting
    \texttt{\ fill-wires:\ false\ } .
  \item
    \texttt{\ gategroup()\ } \texttt{\ slice()\ } and
    \texttt{\ annotate()\ } can now be placed above or below the circuit
    with \texttt{\ z:\ "above"\ } and \texttt{\ z:\ "below"\ } .
  \item
    \texttt{\ help()\ } command for quickly displaying the documentation
    of a given function, e.g., \texttt{\ help("gate")\ } . Powered by
    \href{https://github.com/Mc-Zen/tidy}{tidy} .
  \end{itemize}
\item
  Improvements:

  \begin{itemize}
  \tightlist
  \item
    Complete rework of circuit layout implementation

    \begin{itemize}
    \tightlist
    \item
      allows transparent gates since wires are not drawn through gates
      anymore. The default fill is now \texttt{\ auto\ } and using
      \texttt{\ none\ } sets the background to transparent.
    \item
      \texttt{\ midstick\ } is now transparent by default.
    \end{itemize}
  \item
    \texttt{\ setwire()\ } can now be used to override only partial wire
    settings, such as wire color \texttt{\ setwire(1,\ stroke:\ blue)\ }
    , width \texttt{\ setwire(1,\ stroke:\ 1pt)\ } or wire distance, all
    separately. Before, some settings were reset.
  \end{itemize}
\item
  Fixes:

  \begin{itemize}
  \tightlist
  \item
    Fixed \texttt{\ lstick\ } / \texttt{\ rstick\ } when equation
    numbering is turned on.
  \end{itemize}
\item
  Removed:

  \begin{itemize}
  \tightlist
  \item
    The already deprecated \texttt{\ scale-factor\ } (use
    \texttt{\ scale\ } instead)
  \end{itemize}
\end{itemize}

\subsubsection{v0.2.1}\label{v0.2.1}

\begin{itemize}
\tightlist
\item
  Improvements:

  \begin{itemize}
  \tightlist
  \item
    Add \texttt{\ fill\ } parameter to \texttt{\ midstick()\ } .
  \item
    Add \texttt{\ bend\ } parameter to \texttt{\ permute()\ } .
  \item
    Add \texttt{\ separation\ } parameter to \texttt{\ permute()\ } .
  \end{itemize}
\item
  Fixes:

  \begin{itemize}
  \tightlist
  \item
    With Typst 0.11.0, \texttt{\ scale()\ } now takes into account outer
    alignment. This broke the positioning of centered/right-aligned
    circuits, e.g., ones put into a \texttt{\ figure()\ } .
  \item
    Change wires to be drawn all through \texttt{\ ctrl()\ } , making it
    consistent to \texttt{\ swap()\ } and \texttt{\ targ()\ } .
  \end{itemize}
\end{itemize}

\subsubsection{v0.2.0}\label{v0.2.0}

\begin{itemize}
\tightlist
\item
  New features:

  \begin{itemize}
  \tightlist
  \item
    Add arbitrary labels to any \texttt{\ gate\ } (also derived gates
    such as \texttt{\ meter\ } , \texttt{\ ctrl\ } , …),
    \texttt{\ gategroup\ } or \texttt{\ slice\ } that can be anchored to
    any of the nine 2d alignments.
  \item
    Add optional gate inputs and outputs for multi-qubit gates (see
    gallery).
  \item
    Implicit gates (breaking change âš~ï¸?): a content item
    automatically becomes a gate, so you can just type
    \texttt{\ \$H\$\ } instead of \texttt{\ gate(\$H\$)\ } (of course,
    the \texttt{\ gate()\ } function is still important in order to use
    the many available options).
  \end{itemize}
\item
  Other breaking changes âš~ï¸?:

  \begin{itemize}
  \tightlist
  \item
    \texttt{\ slice()\ } has no \texttt{\ dx\ } and \texttt{\ dy\ }
    parameters anymore. Instead, labels are handled through
    \texttt{\ label\ } exactly as in \texttt{\ gate()\ } . Also the
    \texttt{\ wires\ } parameter is replaced with \texttt{\ n\ } for
    consistency with other multi-qubit gates.
  \item
    Swap order of row and column parameters in \texttt{\ annotate()\ }
    to make it consistent with built-in Typst functions.
  \end{itemize}
\item
  Improvements:

  \begin{itemize}
  \tightlist
  \item
    Improve layout (allow row/column spacing and min lengths to be
    specified in em-lengths).
  \item
    Automatic bounds computation, even for labels.
  \item
    Improve meter (allow multi-qubit gate meters and respect global
    (per-circuit) gate padding).d
  \end{itemize}
\item
  Fixes:

  \begin{itemize}
  \tightlist
  \item
    \texttt{\ lstick\ } / \texttt{\ rstick\ } braces broke with Typst
    0.7.0.
  \item
    \texttt{\ lstick\ } / \texttt{\ rstick\ } bounds.
  \end{itemize}
\item
  Documentation

  \begin{itemize}
  \tightlist
  \item
    Add section on creating custom gates.
  \item
    Add section on using labels.
  \item
    Explain usage of \texttt{\ slice()\ } and \texttt{\ gategroup()\ } .
  \end{itemize}
\end{itemize}

\subsubsection{v0.1.0}\label{v0.1.0}

Initial Release

\subsubsection{How to add}\label{how-to-add}

Copy this into your project and use the import as \texttt{\ quill\ }

\begin{verbatim}
#import "@preview/quill:0.5.0"
\end{verbatim}

\includesvg[width=0.16667in,height=0.16667in]{/assets/icons/16-copy.svg}

Check the docs for
\href{https://typst.app/docs/reference/scripting/\#packages}{more
information on how to import packages} .

\subsubsection{About}\label{about}

\begin{description}
\tightlist
\item[Author :]
\href{https://github.com/Mc-Zen}{Mc-Zen}
\item[License:]
MIT
\item[Current version:]
0.5.0
\item[Last updated:]
October 24, 2024
\item[First released:]
June 28, 2023
\item[Minimum Typst version:]
0.11.0
\item[Archive size:]
24.9 kB
\href{https://packages.typst.org/preview/quill-0.5.0.tar.gz}{\pandocbounded{\includesvg[keepaspectratio]{/assets/icons/16-download.svg}}}
\item[Repository:]
\href{https://github.com/Mc-Zen/quill}{GitHub}
\item[Discipline s :]
\begin{itemize}
\tightlist
\item[]
\item
  \href{https://typst.app/universe/search/?discipline=physics}{Physics}
\item
  \href{https://typst.app/universe/search/?discipline=computer-science}{Computer
  Science}
\end{itemize}
\item[Categor y :]
\begin{itemize}
\tightlist
\item[]
\item
  \pandocbounded{\includesvg[keepaspectratio]{/assets/icons/16-chart.svg}}
  \href{https://typst.app/universe/search/?category=visualization}{Visualization}
\end{itemize}
\end{description}

\subsubsection{Where to report issues?}\label{where-to-report-issues}

This package is a project of Mc-Zen . Report issues on
\href{https://github.com/Mc-Zen/quill}{their repository} . You can also
try to ask for help with this package on the
\href{https://forum.typst.app}{Forum} .

Please report this package to the Typst team using the
\href{https://typst.app/contact}{contact form} if you believe it is a
safety hazard or infringes upon your rights.

\phantomsection\label{versions}
\subsubsection{Version history}\label{version-history}

\begin{longtable}[]{@{}ll@{}}
\toprule\noalign{}
Version & Release Date \\
\midrule\noalign{}
\endhead
\bottomrule\noalign{}
\endlastfoot
0.5.0 & October 24, 2024 \\
\href{https://typst.app/universe/package/quill/0.4.0/}{0.4.0} &
September 16, 2024 \\
\href{https://typst.app/universe/package/quill/0.3.0/}{0.3.0} & May 22,
2024 \\
\href{https://typst.app/universe/package/quill/0.2.1/}{0.2.1} & March
11, 2024 \\
\href{https://typst.app/universe/package/quill/0.2.0/}{0.2.0} &
September 28, 2023 \\
\href{https://typst.app/universe/package/quill/0.1.0/}{0.1.0} & June 28,
2023 \\
\end{longtable}

Typst GmbH did not create this package and cannot guarantee correct
functionality of this package or compatibility with any version of the
Typst compiler or app.


\section{Package List LaTeX/metalogo.tex}
\title{typst.app/universe/package/metalogo}

\phantomsection\label{banner}
\section{metalogo}\label{metalogo}

{ 1.0.2 }

Typeset various LaTeX logos

\phantomsection\label{readme}
Typeset LaTeX compiler logos in
\href{https://github.com/typst/typst}{typst} .

\subsection{usage}\label{usage}

From
\href{https://github.com/typst/packages/raw/main/packages/preview/metalogo/1.0.2/demo.typ}{./demo.typ}
:

\begin{Shaded}
\begin{Highlighting}[]
\NormalTok{\#import "@preview/metalogo:1.0.2": TeX, LaTeX, XeLaTeX, XeTeX, LuaLaTeX}

\NormalTok{\#LaTeX is a typestting program based on \#TeX. Some people use \#XeLaTeX}
\NormalTok{(sometimes \#XeTeX), or \#LuaLaTeX to typeset their documents.}

\NormalTok{People who are afraid of \#LaTeX and its complex macro system may use typst}
\NormalTok{instead.}
\end{Highlighting}
\end{Shaded}

Output:

\pandocbounded{\includesvg[keepaspectratio]{https://github.com/typst/packages/raw/main/packages/preview/metalogo/1.0.2/demo.svg}}

\subsubsection{How to add}\label{how-to-add}

Copy this into your project and use the import as \texttt{\ metalogo\ }

\begin{verbatim}
#import "@preview/metalogo:1.0.2"
\end{verbatim}

\includesvg[width=0.16667in,height=0.16667in]{/assets/icons/16-copy.svg}

Check the docs for
\href{https://typst.app/docs/reference/scripting/\#packages}{more
information on how to import packages} .

\subsubsection{About}\label{about}

\begin{description}
\tightlist
\item[Author :]
\href{mailto:loek@pipeframe.xyz}{Loek Le Blansch}
\item[License:]
MIT
\item[Current version:]
1.0.2
\item[Last updated:]
August 26, 2024
\item[First released:]
August 26, 2024
\item[Minimum Typst version:]
0.10.0
\item[Archive size:]
1.57 kB
\href{https://packages.typst.org/preview/metalogo-1.0.2.tar.gz}{\pandocbounded{\includesvg[keepaspectratio]{/assets/icons/16-download.svg}}}
\item[Repository:]
\href{https://github.com/lonkaars/typst-metalogo.git}{GitHub}
\end{description}

\subsubsection{Where to report issues?}\label{where-to-report-issues}

This package is a project of Loek Le Blansch . Report issues on
\href{https://github.com/lonkaars/typst-metalogo.git}{their repository}
. You can also try to ask for help with this package on the
\href{https://forum.typst.app}{Forum} .

Please report this package to the Typst team using the
\href{https://typst.app/contact}{contact form} if you believe it is a
safety hazard or infringes upon your rights.

\phantomsection\label{versions}
\subsubsection{Version history}\label{version-history}

\begin{longtable}[]{@{}ll@{}}
\toprule\noalign{}
Version & Release Date \\
\midrule\noalign{}
\endhead
\bottomrule\noalign{}
\endlastfoot
1.0.2 & August 26, 2024 \\
\end{longtable}

Typst GmbH did not create this package and cannot guarantee correct
functionality of this package or compatibility with any version of the
Typst compiler or app.


\section{Package List LaTeX/blind-cvpr.tex}
\title{typst.app/universe/package/blind-cvpr}

\phantomsection\label{banner}
\phantomsection\label{template-thumbnail}
\pandocbounded{\includegraphics[keepaspectratio]{https://packages.typst.org/preview/thumbnails/blind-cvpr-0.5.0-small.webp}}

\section{blind-cvpr}\label{blind-cvpr}

{ 0.5.0 }

CVPR-style paper template to publish at the Computer Vision and Pattern
Recognition (CVPR) conferences.

\href{/app?template=blind-cvpr&version=0.5.0}{Create project in app}

\phantomsection\label{readme}
\subsection{Usage}\label{usage}

You can use this template in the Typst web app by clicking \emph{Start
from template} on the dashboard and searching for
\texttt{\ blind-cvpr\ } .

Alternatively, you can use the CLI to kick this project off using the
command

\begin{Shaded}
\begin{Highlighting}[]
\NormalTok{typst init @preview/blind{-}cvpr}
\end{Highlighting}
\end{Shaded}

Typst will create a new directory with all the files needed to get you
started.

\subsection{Configuration}\label{configuration}

This template exports the \texttt{\ cvpr2022\ } and
\texttt{\ cvpr2025\ } styling rule with the following named arguments.

\begin{itemize}
\tightlist
\item
  \texttt{\ title\ } : The paper’s title as content.
\item
  \texttt{\ authors\ } : An array of author dictionaries. Each of the
  author dictionaries must have a name key and can have the keys
  department, organization, location, and email.
\item
  \texttt{\ keywords\ } : Publication keywords (used in PDF metadata).
\item
  \texttt{\ date\ } : Creation date (used in PDF metadata).
\item
  \texttt{\ abstract\ } : The content of a brief summary of the paper or
  none. Appears at the top under the title.
\item
  \texttt{\ bibliography\ } : The result of a call to the bibliography
  function or none. The function also accepts a single, positional
  argument for the body of the paper.
\item
  \texttt{\ appendix\ } : Content to append after bibliography section.
\item
  \texttt{\ accepted\ } : If this is set to \texttt{\ false\ } then
  anonymized ready for submission document is produced;
  \texttt{\ accepted:\ true\ } produces camera-redy version. If the
  argument is set to \texttt{\ none\ } then preprint version is produced
  (can be uploaded to arXiv).
\item
  \texttt{\ id\ } : Identifier of a submission.
\end{itemize}

The template will initialize your package with a sample call to the
\texttt{\ cvpr2025\ } function in a show rule. If you want to change an
existing project to use this template, you can add a show rule at the
top of your file.

\begin{Shaded}
\begin{Highlighting}[]
\NormalTok{\#import "@preview/blind{-}cvpr:0.5.0": cvpr2025}

\NormalTok{\#show: cvpr2025.with(}
\NormalTok{  title: [LaTeX Author Guidelines for CVPR Proceedings],}
\NormalTok{  authors: (authors, affls),}
\NormalTok{  keywords: (),}
\NormalTok{  abstract: [}
\NormalTok{    The ABSTRACT is to be in fully justified italicized text, at the top of the}
\NormalTok{    left{-}hand column, below the author and affiliation information. Use the}
\NormalTok{    word "Abstract" as the title, in 12{-}point Times, boldface type, centered}
\NormalTok{    relative to the column, initially capitalized. The abstract is to be in}
\NormalTok{    10{-}point, single{-}spaced type. Leave two blank lines after the Abstract,}
\NormalTok{    then begin the main text. Look at previous CVPR abstracts to get a feel for}
\NormalTok{    style and length.}
\NormalTok{  ],}
\NormalTok{  bibliography: bibliography("main.bib"),}
\NormalTok{  accepted: false,}
\NormalTok{  id: none,}
\NormalTok{)}
\end{Highlighting}
\end{Shaded}

\subsection{Issues}\label{issues}

\begin{itemize}
\item
  In case of US Letter, column sizes + gap does not equals to text width
  (2 * 3.25 + 5/16 != 6 + 7/8). It seems that correct gap should be 3/8.
\item
  At the moment of Typst v0.11.0, it is impossible to indent the first
  paragraph in a section (see
  \href{https://github.com/typst/typst/issues/311}{typst/typst\#311} ).
  The workaround is to add indentation manually as follows.

\begin{Shaded}
\begin{Highlighting}[]
\NormalTok{== H2}

\NormalTok{\#h(12pt)  Manually as space for the first paragraph.}
\NormalTok{Lorem ipsum dolor sit amet, consectetur adipiscing elit, sed do.}

\NormalTok{// The second one is just fine.}
\NormalTok{Lorem ipsum dolor sit amet, consectetur adipiscing elit, sed do.}
\end{Highlighting}
\end{Shaded}

  Also, we add \texttt{\ indent\ } constant as a shortcut for
  \texttt{\ h(12pt)\ } .

  This issue is relevant to CVPR 2022. In the 2025 template there is no
  indentaino of the first paragraph in section.
\item
  At the moment Typst v0.11.0 does not allow flexible customization of
  citation styles. Specifically, CVPR 2022 citation lookes like
  \texttt{\ {[}42{]}\ } where number is colored hyperlink. In order to
  achive this, we shouuld provide custom CSL-style and then colorize
  number and put it into square parenthesis in typst markup.
\item
  CVPR 2022 requires simple ruler which enumerates lines in regular
  intervals whilst CVPR2025 already requires a ruler which add line
  numers per line in paragraph or heading. Thus we need the next major
  Typst release v0.12.0 for ruler. With the next Typst release, we can
  do the following.

\begin{Shaded}
\begin{Highlighting}[]
\NormalTok{set par.line(numbering: "1")}
\NormalTok{show figure: set par.line(numbering: none)}
\end{Highlighting}
\end{Shaded}

  For implementation details see
  \href{https://github.com/typst/typst/pull/4516}{typst/typst\#4516} .
\item
  CVPR 2022 and 2025 requires IEEE-like bibliography style but does not
  follow its guidelines closely. Since writing CSL-style files is
  tedious task, we adopt close enough bibliography style from Zotero.
\end{itemize}

\subsection{References}\label{references}

\begin{itemize}
\tightlist
\item
  CVPR 2022 conference
  \href{https://cvpr2022.thecvf.com/author-guidelines\#dates}{web site}
  .
\item
  CVPR 2025 conference
  \href{https://cvpr.thecvf.com/Conferences/2025}{web site} .
\end{itemize}

\href{/app?template=blind-cvpr&version=0.5.0}{Create project in app}

\subsubsection{How to use}\label{how-to-use}

Click the button above to create a new project using this template in
the Typst app.

You can also use the Typst CLI to start a new project on your computer
using this command:

\begin{verbatim}
typst init @preview/blind-cvpr:0.5.0
\end{verbatim}

\includesvg[width=0.16667in,height=0.16667in]{/assets/icons/16-copy.svg}

\subsubsection{About}\label{about}

\begin{description}
\tightlist
\item[Author :]
daskol
\item[License:]
MIT
\item[Current version:]
0.5.0
\item[Last updated:]
September 22, 2024
\item[First released:]
September 22, 2024
\item[Minimum Typst version:]
0.11.1
\item[Archive size:]
18.3 kB
\href{https://packages.typst.org/preview/blind-cvpr-0.5.0.tar.gz}{\pandocbounded{\includesvg[keepaspectratio]{/assets/icons/16-download.svg}}}
\item[Repository:]
\href{https://github.com/daskol/typst-templates}{GitHub}
\item[Discipline s :]
\begin{itemize}
\tightlist
\item[]
\item
  \href{https://typst.app/universe/search/?discipline=computer-science}{Computer
  Science}
\item
  \href{https://typst.app/universe/search/?discipline=mathematics}{Mathematics}
\end{itemize}
\item[Categor y :]
\begin{itemize}
\tightlist
\item[]
\item
  \pandocbounded{\includesvg[keepaspectratio]{/assets/icons/16-atom.svg}}
  \href{https://typst.app/universe/search/?category=paper}{Paper}
\end{itemize}
\end{description}

\subsubsection{Where to report issues?}\label{where-to-report-issues}

This template is a project of daskol . Report issues on
\href{https://github.com/daskol/typst-templates}{their repository} . You
can also try to ask for help with this template on the
\href{https://forum.typst.app}{Forum} .

Please report this template to the Typst team using the
\href{https://typst.app/contact}{contact form} if you believe it is a
safety hazard or infringes upon your rights.

\phantomsection\label{versions}
\subsubsection{Version history}\label{version-history}

\begin{longtable}[]{@{}ll@{}}
\toprule\noalign{}
Version & Release Date \\
\midrule\noalign{}
\endhead
\bottomrule\noalign{}
\endlastfoot
0.5.0 & September 22, 2024 \\
\end{longtable}

Typst GmbH did not create this template and cannot guarantee correct
functionality of this template or compatibility with any version of the
Typst compiler or app.


\section{Package List LaTeX/simple-preavis.tex}
\title{typst.app/universe/package/simple-preavis}

\phantomsection\label{banner}
\phantomsection\label{template-thumbnail}
\pandocbounded{\includegraphics[keepaspectratio]{https://packages.typst.org/preview/thumbnails/simple-preavis-0.1.0-small.webp}}

\section{simple-preavis}\label{simple-preavis}

{ 0.1.0 }

ðŸ``-- a french move out letter

\href{/app?template=simple-preavis&version=0.1.0}{Create project in app}

\phantomsection\label{readme}
\textbf{simple-preavis} est un template typst pour écrire une lettre de
préavis d’état des lieux Ã~ son propriétaire.

Il est fortement inspiré de cet
\href{https://www.service-public.fr/simulateur/calcul/CongeLogement}{outil}
réalisés par les services publics �.

\subsection{Utilisation}\label{utilisation}

\subsubsection{Exemple d’utilisation}\label{exemple-duxe2utilisation}

\begin{Shaded}
\begin{Highlighting}[]
\NormalTok{\#import "@preview/simple{-}preavis:0.1.0":*}
\NormalTok{\#lettre{-}preavis(}
\NormalTok{  locataire: locataire(}
\NormalTok{     "Dupont locataire",}
\NormalTok{     "Jean",}
\NormalTok{     adresse(}
\NormalTok{       "123 rue de la Paix",}
\NormalTok{       "75000",}
\NormalTok{       "Paris",}
\NormalTok{       complement: "Appartement 2"}
\NormalTok{    )}
\NormalTok{  ),}
\NormalTok{  proprietaire: proprietaire(}
\NormalTok{     "Martin proprietaire",}
\NormalTok{    "Sophie",}
\NormalTok{     adresse(}
\NormalTok{       "456 avenue des Champs{-}Élysées",}
\NormalTok{       "75008",}
\NormalTok{       "Paris"}
\NormalTok{    ),}
\NormalTok{    "Madame"}
\NormalTok{  ),}
\NormalTok{  date{-}etat{-}des{-}lieux: datetime(year:2024, month:9, day:21)}
\NormalTok{)}
\end{Highlighting}
\end{Shaded}

\subsection{TODO}\label{todo}

\begin{itemize}
\tightlist
\item
  {[} {]} Supporter plusieurs locataires
\item
  {[} {]} Supporter la législation zone tendu en fonction du code
  postal
\item
  {[} {]} Améliorer la documentation des fonctions
\item
  {[} {]} Séparer en une librairie et un template pour que cela
  ressemble plus aux autres template types
\end{itemize}

\subsection{Mention license}\label{mention-license}

Conformément Ã~ la license
\href{https://github.com/etalab/licence-ouverte/blob/master/LO.md}{etalab}

\begin{itemize}
\tightlist
\item
  Condérant : Direction de l’information légale et administrative
  (Premier ministre)
\item
  Date de mise Ã~ jour : Vérifié le 23 Avril 2024
\end{itemize}

\subsection{License}\label{license}

\href{https://github.com/typst/packages/raw/main/packages/preview/simple-preavis/0.1.0/LICENSE}{license
MIT}

\href{/app?template=simple-preavis&version=0.1.0}{Create project in app}

\subsubsection{How to use}\label{how-to-use}

Click the button above to create a new project using this template in
the Typst app.

You can also use the Typst CLI to start a new project on your computer
using this command:

\begin{verbatim}
typst init @preview/simple-preavis:0.1.0
\end{verbatim}

\includesvg[width=0.16667in,height=0.16667in]{/assets/icons/16-copy.svg}

\subsubsection{About}\label{about}

\begin{description}
\tightlist
\item[Author :]
\href{https://github.com/mathias-aparicio/}{Mathias APARICIO}
\item[License:]
MIT
\item[Current version:]
0.1.0
\item[Last updated:]
July 23, 2024
\item[First released:]
July 23, 2024
\item[Archive size:]
3.09 kB
\href{https://packages.typst.org/preview/simple-preavis-0.1.0.tar.gz}{\pandocbounded{\includesvg[keepaspectratio]{/assets/icons/16-download.svg}}}
\item[Repository:]
\href{https://github.com/mathias-aparicio/simple-preavis}{GitHub}
\item[Categor y :]
\begin{itemize}
\tightlist
\item[]
\item
  \pandocbounded{\includesvg[keepaspectratio]{/assets/icons/16-envelope.svg}}
  \href{https://typst.app/universe/search/?category=office}{Office}
\end{itemize}
\end{description}

\subsubsection{Where to report issues?}\label{where-to-report-issues}

This template is a project of Mathias APARICIO . Report issues on
\href{https://github.com/mathias-aparicio/simple-preavis}{their
repository} . You can also try to ask for help with this template on the
\href{https://forum.typst.app}{Forum} .

Please report this template to the Typst team using the
\href{https://typst.app/contact}{contact form} if you believe it is a
safety hazard or infringes upon your rights.

\phantomsection\label{versions}
\subsubsection{Version history}\label{version-history}

\begin{longtable}[]{@{}ll@{}}
\toprule\noalign{}
Version & Release Date \\
\midrule\noalign{}
\endhead
\bottomrule\noalign{}
\endlastfoot
0.1.0 & July 23, 2024 \\
\end{longtable}

Typst GmbH did not create this template and cannot guarantee correct
functionality of this template or compatibility with any version of the
Typst compiler or app.


\section{Package List LaTeX/bamdone-aiaa.tex}
\title{typst.app/universe/package/bamdone-aiaa}

\phantomsection\label{banner}
\phantomsection\label{template-thumbnail}
\pandocbounded{\includegraphics[keepaspectratio]{https://packages.typst.org/preview/thumbnails/bamdone-aiaa-0.1.1-small.webp}}

\section{bamdone-aiaa}\label{bamdone-aiaa}

{ 0.1.1 }

An American Institute of Aeronautics and Astronautics (AIAA) template
for conferences.

\href{/app?template=bamdone-aiaa&version=0.1.1}{Create project in app}

\phantomsection\label{readme}
This is a Typst template for a one-column paper from the proceedings of
the American Institute of Aeronautics and Astronautics. The paper is
tightly spaced, fits a lot of content and comes preconfigured for
numeric citations from BibLaTeX or Hayagriva files.

\subsection{Usage}\label{usage}

You can use this template in the Typst web app by clicking “Start from
template� on the dashboard and searching for \texttt{\ bamdone-aiaa\ }
.

Alternatively, you can use the CLI to kick this project off using the
command

\begin{verbatim}
typst init @preview/bamdone-aiaa
\end{verbatim}

Typst will create a new directory with all the files needed to get you
started.

\subsection{Configuration}\label{configuration}

This template exports the \texttt{\ aiaa\ } function with the following
named arguments:

\begin{itemize}
\tightlist
\item
  \texttt{\ title\ } : The paper’s title as content.
\item
  \texttt{\ authors-and-affiliations\ } : An array of author
  dictionaries and affiliation dictionaries. Author dictionaries must
  have a \texttt{\ name\ } key and can have the keys \texttt{\ job\ } ,
  \texttt{\ department\ } , \texttt{\ aiaa\ } is optional. Affiliation
  dictionaries must have the keys \texttt{\ institution\ } ,
  \texttt{\ city\ } , \texttt{\ state\ } , \texttt{\ zip\ } , and
  \texttt{\ country\ } .
\item
  \texttt{\ abstract\ } : The content of a brief summary of the paper or
  \texttt{\ none\ } . Appears at the top of the first column in
  boldface. Shall be \texttt{\ content\ } .
\item
  \texttt{\ paper-size\ } : Defaults to \texttt{\ us-letter\ } . Specify
  a
  \href{https://typst.app/docs/reference/layout/page/\#parameters-paper}{paper
  size string} to change the page format.
\item
  \texttt{\ bibliography\ } : The result of a call to the
  \texttt{\ bibliography\ } function or \texttt{\ none\ } . Specifying
  this will configure numeric, AIAA-style citations.
\end{itemize}

The function also accepts a single, positional argument for the body of
the paper.

The template will initialize your package with a sample call to the
\texttt{\ aiaa\ } function in a show rule. If you want to change an
existing project to use this template, you can add a show rule like this
at the top of your file:

\begin{Shaded}
\begin{Highlighting}[]
\NormalTok{\#import "@preview/bamdone{-}aiaa:0.1.0": aiaa}

\NormalTok{\#show: aiaa.with(}
\NormalTok{  title: [A typesetting system to untangle the scientific writing process],}
\NormalTok{  abstract: [}
\NormalTok{    These instructions give you guidelines for preparing papers for AIAA Technical Papers. Use this document as a template if you are using Typst. Otherwise, use this document as an instruction set. Define all symbols used in the abstract. Do not cite references in the abstract. The footnote on the first page should list the Job Title and AIAA Member Grade for each author, if known. Authors do not have to be AIAA members.}
\NormalTok{  ],}
\NormalTok{  authors: (}
\NormalTok{      (}
\NormalTok{        name:"First A. Author",}
\NormalTok{        job:"Insert Job Title",}
\NormalTok{        department:"Department Name",}
\NormalTok{        aiaa:"and AIAA Member Grade (if any) for first author"}
\NormalTok{      ),}
\NormalTok{      (}
\NormalTok{        institution:"Business or Academic Affiliation\textquotesingle{}s Full Name 1",}
\NormalTok{        city:"City",}
\NormalTok{        state:"State",}
\NormalTok{        zip:"Zip Code",}
\NormalTok{        country:"Country"}
\NormalTok{      ),}
\NormalTok{  ),}
\NormalTok{  bibliography: bibliography("refs.bib"),}
\NormalTok{)}

\NormalTok{// Your content goes below.}
\end{Highlighting}
\end{Shaded}

\href{/app?template=bamdone-aiaa&version=0.1.1}{Create project in app}

\subsubsection{How to use}\label{how-to-use}

Click the button above to create a new project using this template in
the Typst app.

You can also use the Typst CLI to start a new project on your computer
using this command:

\begin{verbatim}
typst init @preview/bamdone-aiaa:0.1.1
\end{verbatim}

\includesvg[width=0.16667in,height=0.16667in]{/assets/icons/16-copy.svg}

\subsubsection{About}\label{about}

\begin{description}
\tightlist
\item[Author s :]
\href{https://www.isaacew.com/}{Isaac Weintraub} \&
\href{https://avonmoll.github.io/}{Alexander Von Moll}
\item[License:]
MIT-0
\item[Current version:]
0.1.1
\item[Last updated:]
May 14, 2024
\item[First released:]
March 23, 2024
\item[Minimum Typst version:]
0.11.0
\item[Archive size:]
14.4 kB
\href{https://packages.typst.org/preview/bamdone-aiaa-0.1.1.tar.gz}{\pandocbounded{\includesvg[keepaspectratio]{/assets/icons/16-download.svg}}}
\item[Repository:]
\href{https://github.com/isaacew/aiaa-typst}{GitHub}
\item[Discipline s :]
\begin{itemize}
\tightlist
\item[]
\item
  \href{https://typst.app/universe/search/?discipline=engineering}{Engineering}
\item
  \href{https://typst.app/universe/search/?discipline=computer-science}{Computer
  Science}
\item
  \href{https://typst.app/universe/search/?discipline=mathematics}{Mathematics}
\item
  \href{https://typst.app/universe/search/?discipline=communication}{Communication}
\item
  \href{https://typst.app/universe/search/?discipline=transportation}{Transportation}
\item
  \href{https://typst.app/universe/search/?discipline=education}{Education}
\end{itemize}
\item[Categor y :]
\begin{itemize}
\tightlist
\item[]
\item
  \pandocbounded{\includesvg[keepaspectratio]{/assets/icons/16-atom.svg}}
  \href{https://typst.app/universe/search/?category=paper}{Paper}
\end{itemize}
\end{description}

\subsubsection{Where to report issues?}\label{where-to-report-issues}

This template is a project of Isaac Weintraub and Alexander Von Moll .
Report issues on \href{https://github.com/isaacew/aiaa-typst}{their
repository} . You can also try to ask for help with this template on the
\href{https://forum.typst.app}{Forum} .

Please report this template to the Typst team using the
\href{https://typst.app/contact}{contact form} if you believe it is a
safety hazard or infringes upon your rights.

\phantomsection\label{versions}
\subsubsection{Version history}\label{version-history}

\begin{longtable}[]{@{}ll@{}}
\toprule\noalign{}
Version & Release Date \\
\midrule\noalign{}
\endhead
\bottomrule\noalign{}
\endlastfoot
0.1.1 & May 14, 2024 \\
\href{https://typst.app/universe/package/bamdone-aiaa/0.1.0/}{0.1.0} &
March 23, 2024 \\
\end{longtable}

Typst GmbH did not create this template and cannot guarantee correct
functionality of this template or compatibility with any version of the
Typst compiler or app.


\section{Package List LaTeX/suiji.tex}
\title{typst.app/universe/package/suiji}

\phantomsection\label{banner}
\section{suiji}\label{suiji}

{ 0.3.0 }

A highly efficient random number generator for Typst

{ } Featured Package

\phantomsection\label{readme}
\href{https://github.com/liuguangxi/suiji}{Suiji} (�机 in Chinese,
/suíjī/, meaning random) is a high efficient random number generator in
Typst. Partial algorithm is inherited from
\href{https://www.gnu.org/software/gsl}{GSL} and most APIs are similar
to
\href{https://numpy.org/doc/stable/reference/random/generator.html}{NumPy
Random Generator} . It provides pure function implementation and does
not rely on any global state variables, resulting in better performance
and independency.

\subsection{Features}\label{features}

\begin{itemize}
\tightlist
\item
  All functions are immutable, which means results of random are
  completely deterministic.
\item
  Core random engine chooses “Maximally equidistributed combined
  Tausworthe generator� and “LCG�.
\item
  Generate random integers or floats from various distribution.
\item
  Randomly shuffle an array of objects.
\item
  Randomly sample from an array of objects.
\item
  Generate blind text of Simplified Chinese.
\end{itemize}

\subsection{Examples}\label{examples}

The example below uses \texttt{\ suiji\ } and \texttt{\ cetz\ } packages
to create a trajectory of a random walk.

\begin{Shaded}
\begin{Highlighting}[]
\NormalTok{\#import "@preview/suiji:0.3.0": *}
\NormalTok{\#import "@preview/cetz:0.2.2"}

\NormalTok{\#set page(width: auto, height: auto, margin: 0.5cm)}

\NormalTok{\#cetz.canvas(length: 5pt, \{}
\NormalTok{  import cetz.draw: *}

\NormalTok{  let n = 2000}
\NormalTok{  let (x, y) = (0, 0)}
\NormalTok{  let (x{-}new, y{-}new) = (0, 0)}
\NormalTok{  let rng = gen{-}rng(42)}
\NormalTok{  let v = ()}

\NormalTok{  for i in range(n) \{}
\NormalTok{    (rng, v) = uniform(rng, low: {-}2.0, high: 2.0, size: 2)}
\NormalTok{    (x{-}new, y{-}new) = (x {-} v.at(1), y {-} v.at(0))}
\NormalTok{    let col = color.mix((blue.transparentize(20\%), 1{-}i/n), (green.transparentize(20\%), i/n))}
\NormalTok{    line(stroke: (paint: col, cap: "round", thickness: 2pt),}
\NormalTok{      (x, y), (x{-}new, y{-}new)}
\NormalTok{    )}
\NormalTok{    (x, y) = (x{-}new, y{-}new)}
\NormalTok{  \}}
\NormalTok{\})}
\end{Highlighting}
\end{Shaded}

\pandocbounded{\includegraphics[keepaspectratio]{https://github.com/typst/packages/raw/main/packages/preview/suiji/0.3.0/examples/random-walk.png}}

Another example is drawing the the famous \textbf{Matrix} rain effect of
falling green characters in a terminal.

\begin{Shaded}
\begin{Highlighting}[]
\NormalTok{\#import "@preview/suiji:0.3.0": *}
\NormalTok{\#import "@preview/cetz:0.2.2"}

\NormalTok{\#set page(width: auto, height: auto, margin: 0pt)}

\NormalTok{\#cetz.canvas(length: 1pt, \{}
\NormalTok{  import cetz.draw: *}

\NormalTok{  let font{-}size = 10}
\NormalTok{  let num{-}col = 80}
\NormalTok{  let num{-}row = 32}
\NormalTok{  let text{-}len = 16}
\NormalTok{  let seq = "abcdefghijklmnopqrstuvwxyz!@\#$\%\^{}\&*".split("").slice(1, 35).map(it =\textgreater{} raw(it))}
\NormalTok{  let rng = gen{-}rng(42)}
\NormalTok{  let num{-}cnt = 0}
\NormalTok{  let val = 0}
\NormalTok{  let chars = ()}

\NormalTok{  rect(({-}10, {-}10), (font{-}size * (num{-}col {-} 1) * 0.6 + 10, font{-}size * (num{-}row {-} 1) + 10), fill: black)}

\NormalTok{  for c in range(num{-}col) \{}
\NormalTok{    (rng, num{-}cnt) = integers(rng, low: 1, high: 3)}
\NormalTok{    for cnt in range(num{-}cnt) \{}
\NormalTok{      (rng, val) = integers(rng, low: {-}10, high: num{-}row {-} 2)}
\NormalTok{      (rng, chars) = choice(rng, seq, size: text{-}len)}
\NormalTok{      for i in range(text{-}len) \{}
\NormalTok{        let y = i + val}
\NormalTok{        if y \textgreater{}= 0 and y \textless{} num{-}row \{}
\NormalTok{          let col = green.transparentize((i / text{-}len) * 100\%)}
\NormalTok{          content(}
\NormalTok{            (c * font{-}size * 0.6, y * font{-}size),}
\NormalTok{            text(size: font{-}size * 1pt, fill:col, stroke: (text{-}len {-} i) * 0.04pt + col, chars.at(i))}
\NormalTok{          )}
\NormalTok{        \}}
\NormalTok{      \}}
\NormalTok{    \}}
\NormalTok{  \}}
\NormalTok{\})}
\end{Highlighting}
\end{Shaded}

\pandocbounded{\includegraphics[keepaspectratio]{https://github.com/typst/packages/raw/main/packages/preview/suiji/0.3.0/examples/matrix-rain.png}}

\subsection{Usage}\label{usage}

Import \texttt{\ suiji\ } module first before use any random functions
from it.

\begin{Shaded}
\begin{Highlighting}[]
\NormalTok{\#import "@preview/suiji:0.3.0": *}
\end{Highlighting}
\end{Shaded}

For functions that generate various random numbers or randomly shuffle,
a random number generator object ( \textbf{rng} ) is required as both
input and output arguments. And the original \textbf{rng} should be
created by function \texttt{\ gen-rng\ } , with an integer as the
argument of seed. This calling style seems to be a little inconvenient,
as it is limited by the programming paradigm. For function
\texttt{\ discrete\ } , the given probalilities of the discrete events
should be preprocessed by function \texttt{\ discrete-preproc\ } , whose
output serves as an input argument of \texttt{\ discrete\ } .

Another set of functions with the same functionality provides higher
performance (about 3 times faster) and has the suffix \texttt{\ -f\ } in
their names. For example, \texttt{\ gen-rng-f\ } and
\texttt{\ integers-f\ } are the fast versions of \texttt{\ gen-rng\ }
and \texttt{\ integers\ } , respectively.

The function \texttt{\ rand-sc\ } creates blind text of Simplified
Chinese. This function yields a Chinese-like Lorem Ipsum blind text with
the given number of words, where punctuations are optional.

The code below generates several random permutations of 0 to 9. Each
time after function \texttt{\ shuffle-f\ } is called, the value of
variable \texttt{\ rng\ } is updated, so generated permutations are
different.

\begin{Shaded}
\begin{Highlighting}[]
\NormalTok{\#\{}
\NormalTok{  let rng = gen{-}rng{-}f(42)}
\NormalTok{  let a = ()}
\NormalTok{  for i in range(5) \{}
\NormalTok{    (rng, a) = shuffle{-}f(rng, range(10))}
\NormalTok{    [\#(a.map(it =\textgreater{} str(it)).join("  ")) \textbackslash{} ]}
\NormalTok{  \}}
\NormalTok{\}}
\end{Highlighting}
\end{Shaded}

\pandocbounded{\includegraphics[keepaspectratio]{https://github.com/typst/packages/raw/main/packages/preview/suiji/0.3.0/examples/random-permutation.png}}

For more codes with these functions see
\href{https://github.com/typst/packages/raw/main/packages/preview/suiji/0.3.0/tests}{tests}
.

\subsection{Reference}\label{reference}

\subsubsection{\texorpdfstring{\texttt{\ gen-rng\ } /
\texttt{\ gen-rng-f\ }}{ gen-rng  /  gen-rng-f }}\label{gen-rng-gen-rng-f}

Construct a new random number generator with a seed.

\begin{Shaded}
\begin{Highlighting}[]
\NormalTok{\#let gen{-}rng(seed) = \{...\}}
\end{Highlighting}
\end{Shaded}

\begin{itemize}
\item
  \textbf{Input Arguments}

  \begin{itemize}
  \tightlist
  \item
    \texttt{\ seed\ } : {[} \texttt{\ int\ } {]} value of seed.
  \end{itemize}
\item
  \textbf{Output Arguments}

  \begin{itemize}
  \tightlist
  \item
    \texttt{\ rng\ } : {[} \texttt{\ object\ } {]} generated object of
    random number generator.
  \end{itemize}
\end{itemize}

\subsubsection{\texorpdfstring{\texttt{\ randi-f\ }}{ randi-f }}\label{randi-f}

Return a raw random integer from {[}0, 2\^{}31).

\begin{Shaded}
\begin{Highlighting}[]
\NormalTok{\#let randi{-}f(rng) = \{...\}}
\end{Highlighting}
\end{Shaded}

\begin{itemize}
\item
  \textbf{Input Arguments}

  \begin{itemize}
  \tightlist
  \item
    \texttt{\ rng\ } : {[} \texttt{\ object\ } \textbar{}
    \texttt{\ int\ } {]} object of random number generator (generated by
    function \texttt{\ *-f\ } ).
  \end{itemize}
\item
  \textbf{Output Arguments}

  \begin{itemize}
  \tightlist
  \item
    \texttt{\ rng-out\ } : {[} \texttt{\ object\ } \textbar{}
    \texttt{\ int\ } {]} updated object of random number generator
    (random integer from the interval {[}0, 2\^{}31-1{]}).
  \end{itemize}
\end{itemize}

\subsubsection{\texorpdfstring{\texttt{\ integers\ } /
\texttt{\ integers-f\ }}{ integers  /  integers-f }}\label{integers-integers-f}

Return random integers from \texttt{\ low\ } (inclusive) to
\texttt{\ high\ } (exclusive).

\begin{Shaded}
\begin{Highlighting}[]
\NormalTok{\#let integers(rng, low: 0, high: 100, size: none, endpoint: false) = \{...\}}
\end{Highlighting}
\end{Shaded}

\begin{itemize}
\item
  \textbf{Input Arguments}

  \begin{itemize}
  \tightlist
  \item
    \texttt{\ rng\ } : {[} \texttt{\ object\ } {]} object of random
    number generator.
  \item
    \texttt{\ low\ } : {[} \texttt{\ int\ } {]} lowest (signed) integers
    to be drawn from the distribution, optional.
  \item
    \texttt{\ high\ } : {[} \texttt{\ int\ } {]} one above the largest
    (signed) integer to be drawn from the distribution, optional.
  \item
    \texttt{\ size\ } : {[} \texttt{\ none\ } or \texttt{\ int\ } {]}
    returned array size, must be none or non-negative integer, optional.
  \item
    \texttt{\ endpoint\ } : {[} \texttt{\ bool\ } {]} if true, sample
    from the interval {[} \texttt{\ low\ } , \texttt{\ high\ } {]}
    instead of the default {[} \texttt{\ low\ } , \texttt{\ high\ } ),
    optional.
  \end{itemize}
\item
  \textbf{Output Arguments}

  \begin{itemize}
  \tightlist
  \item
    {[} \texttt{\ array\ } {]} : ( \texttt{\ rng-out\ } ,
    \texttt{\ arr-out\ } )

    \begin{itemize}
    \tightlist
    \item
      \texttt{\ rng-out\ } : {[} \texttt{\ object\ } {]} updated object
      of random number generator.
    \item
      \texttt{\ arr-out\ } : {[} \texttt{\ int\ } \textbar{}
      \texttt{\ array\ } of \texttt{\ int\ } {]} array of random
      numbers.
    \end{itemize}
  \end{itemize}
\end{itemize}

\subsubsection{\texorpdfstring{\texttt{\ random\ } /
\texttt{\ random-f\ }}{ random  /  random-f }}\label{random-random-f}

Return random floats in the half-open interval {[}0.0, 1.0).

\begin{Shaded}
\begin{Highlighting}[]
\NormalTok{\#let random(rng, size: none) = \{...\}}
\end{Highlighting}
\end{Shaded}

\begin{itemize}
\item
  \textbf{Input Arguments}

  \begin{itemize}
  \tightlist
  \item
    \texttt{\ rng\ } : {[} \texttt{\ object\ } {]} object of random
    number generator.
  \item
    \texttt{\ size\ } : {[} \texttt{\ none\ } or \texttt{\ int\ } {]}
    returned array size, must be none or non-negative integer, optional.
  \end{itemize}
\item
  \textbf{Output Arguments}

  \begin{itemize}
  \tightlist
  \item
    {[} \texttt{\ array\ } {]} : ( \texttt{\ rng-out\ } ,
    \texttt{\ arr-out\ } )

    \begin{itemize}
    \tightlist
    \item
      \texttt{\ rng-out\ } : {[} \texttt{\ object\ } {]} updated object
      of random number generator.
    \item
      \texttt{\ arr-out\ } : {[} \texttt{\ float\ } \textbar{}
      \texttt{\ array\ } of \texttt{\ float\ } {]} array of random
      numbers.
    \end{itemize}
  \end{itemize}
\end{itemize}

\subsubsection{\texorpdfstring{\texttt{\ uniform\ } /
\texttt{\ uniform-f\ }}{ uniform  /  uniform-f }}\label{uniform-uniform-f}

Draw samples from a uniform distribution. Samples are uniformly
distributed over the half-open interval {[} \texttt{\ low\ } ,
\texttt{\ high\ } ) (includes \texttt{\ low\ } , but excludes
\texttt{\ high\ } ).

\begin{Shaded}
\begin{Highlighting}[]
\NormalTok{\#let uniform(rng, low: 0.0, high: 1.0, size: none) = \{...\}}
\end{Highlighting}
\end{Shaded}

\begin{itemize}
\item
  \textbf{Input Arguments}

  \begin{itemize}
  \tightlist
  \item
    \texttt{\ rng\ } : {[} \texttt{\ object\ } {]} object of random
    number generator.
  \item
    \texttt{\ low\ } : {[} \texttt{\ float\ } {]} lower boundary of the
    output interval, optional.
  \item
    \texttt{\ high\ } : {[} \texttt{\ float\ } {]} upper boundary of the
    output interval, optional.
  \item
    \texttt{\ size\ } : {[} \texttt{\ none\ } or \texttt{\ int\ } {]}
    returned array size, must be none or non-negative integer, optional.
  \end{itemize}
\item
  \textbf{Output Arguments}

  \begin{itemize}
  \tightlist
  \item
    {[} \texttt{\ array\ } {]} : ( \texttt{\ rng-out\ } ,
    \texttt{\ arr-out\ } )

    \begin{itemize}
    \tightlist
    \item
      \texttt{\ rng-out\ } : {[} \texttt{\ object\ } {]} updated object
      of random number generator.
    \item
      \texttt{\ arr-out\ } : {[} \texttt{\ float\ } \textbar{}
      \texttt{\ array\ } of \texttt{\ float\ } {]} array of random
      numbers.
    \end{itemize}
  \end{itemize}
\end{itemize}

\subsubsection{\texorpdfstring{\texttt{\ normal\ } /
\texttt{\ normal-f\ }}{ normal  /  normal-f }}\label{normal-normal-f}

Draw random samples from a normal (Gaussian) distribution.

\begin{Shaded}
\begin{Highlighting}[]
\NormalTok{\#let normal(rng, loc: 0.0, scale: 1.0, size: none) = \{...\}}
\end{Highlighting}
\end{Shaded}

\begin{itemize}
\item
  \textbf{Input Arguments}

  \begin{itemize}
  \tightlist
  \item
    \texttt{\ rng\ } : {[} \texttt{\ object\ } {]} object of random
    number generator.
  \item
    \texttt{\ loc\ } : {[} \texttt{\ float\ } {]} mean (centre) of the
    distribution, optional.
  \item
    \texttt{\ scale\ } : {[} \texttt{\ float\ } {]} standard deviation
    (spread or width) of the distribution, must be non-negative,
    optional.
  \item
    \texttt{\ size\ } : {[} \texttt{\ none\ } or \texttt{\ int\ } {]}
    returned array size, must be none or non-negative integer, optional.
  \end{itemize}
\item
  \textbf{Output Arguments}

  \begin{itemize}
  \tightlist
  \item
    {[} \texttt{\ array\ } {]} : ( \texttt{\ rng-out\ } ,
    \texttt{\ arr-out\ } )

    \begin{itemize}
    \tightlist
    \item
      \texttt{\ rng-out\ } : {[} \texttt{\ object\ } {]} updated object
      of random number generator.
    \item
      \texttt{\ arr-out\ } : {[} \texttt{\ float\ } \textbar{}
      \texttt{\ array\ } of \texttt{\ float\ } {]} array of random
      numbers.
    \end{itemize}
  \end{itemize}
\end{itemize}

\subsubsection{\texorpdfstring{\texttt{\ discrete-preproc\ } and
\texttt{\ discrete\ } / \texttt{\ discrete-preproc-f\ } and
\texttt{\ discrete-f\ }}{ discrete-preproc  and  discrete  /  discrete-preproc-f  and  discrete-f }}\label{discrete-preproc-and-discrete-discrete-preproc-f-and-discrete-f}

Return random indices from the given probalilities of the discrete
events.

\begin{Shaded}
\begin{Highlighting}[]
\NormalTok{\#let discrete{-}preproc(p) = \{...\}}
\end{Highlighting}
\end{Shaded}

\begin{itemize}
\item
  \textbf{Input Arguments}

  \begin{itemize}
  \tightlist
  \item
    \texttt{\ p\ } : {[} \texttt{\ array\ } of \texttt{\ int\ } or
    \texttt{\ float\ } {]} the array of probalilities of the discrete
    events, probalilities must be non-negative.
  \end{itemize}
\item
  \textbf{Output Arguments}

  \begin{itemize}
  \tightlist
  \item
    \texttt{\ g\ } : {[} \texttt{\ object\ } {]} generated object that
    contains the lookup table.
  \end{itemize}
\end{itemize}

\begin{Shaded}
\begin{Highlighting}[]
\NormalTok{\#let discrete(rng, g, size: none) = \{...\}}
\end{Highlighting}
\end{Shaded}

\begin{itemize}
\item
  \textbf{Input Arguments}

  \begin{itemize}
  \tightlist
  \item
    \texttt{\ rng\ } : {[} \texttt{\ object\ } {]} object of random
    number generator.
  \item
    \texttt{\ g\ } : {[} \texttt{\ object\ } {]} generated object that
    contains the lookup table by \texttt{\ discrete-preproc\ } function.
  \item
    \texttt{\ size\ } : {[} \texttt{\ none\ } or \texttt{\ int\ } {]}
    returned array size, must be none or non-negative integer, optional.
  \end{itemize}
\item
  \textbf{Output Arguments}

  \begin{itemize}
  \tightlist
  \item
    {[} \texttt{\ array\ } {]} : ( \texttt{\ rng-out\ } ,
    \texttt{\ arr-out\ } )

    \begin{itemize}
    \tightlist
    \item
      \texttt{\ rng-out\ } : {[} \texttt{\ object\ } {]} updated object
      of random number generator.
    \item
      \texttt{\ arr-out\ } : {[} \texttt{\ int\ } \textbar{}
      \texttt{\ array\ } of \texttt{\ int\ } {]} array of random
      indices.
    \end{itemize}
  \end{itemize}
\end{itemize}

\subsubsection{\texorpdfstring{\texttt{\ shuffle\ } /
\texttt{\ shuffle-f\ }}{ shuffle  /  shuffle-f }}\label{shuffle-shuffle-f}

Randomly shuffle a given array.

\begin{Shaded}
\begin{Highlighting}[]
\NormalTok{\#let shuffle(rng, arr) = \{...\}}
\end{Highlighting}
\end{Shaded}

\begin{itemize}
\item
  \textbf{Input Arguments}

  \begin{itemize}
  \tightlist
  \item
    \texttt{\ rng\ } : {[} \texttt{\ object\ } {]} object of random
    number generator.
  \item
    \texttt{\ arr\ } : {[} \texttt{\ array\ } {]} the array to be
    shuffled.
  \end{itemize}
\item
  \textbf{Output Arguments}

  \begin{itemize}
  \tightlist
  \item
    {[} \texttt{\ array\ } {]} : ( \texttt{\ rng-out\ } ,
    \texttt{\ arr-out\ } )

    \begin{itemize}
    \tightlist
    \item
      \texttt{\ rng-out\ } : {[} \texttt{\ object\ } {]} updated object
      of random number generator.
    \item
      \texttt{\ arr-out\ } : {[} \texttt{\ array\ } {]} shuffled array.
    \end{itemize}
  \end{itemize}
\end{itemize}

\subsubsection{\texorpdfstring{\texttt{\ choice\ } /
\texttt{\ choice-f\ }}{ choice  /  choice-f }}\label{choice-choice-f}

Generate random samples from a given array.

\begin{Shaded}
\begin{Highlighting}[]
\NormalTok{\#let choice(rng, arr, size: none, replacement: true, permutation: true) = \{...\}}
\end{Highlighting}
\end{Shaded}

\begin{itemize}
\item
  \textbf{Input Arguments}

  \begin{itemize}
  \tightlist
  \item
    \texttt{\ rng\ } : {[} \texttt{\ object\ } {]} object of random
    number generator.
  \item
    \texttt{\ arr\ } : {[} \texttt{\ array\ } {]} the array to be
    sampled.
  \item
    \texttt{\ size\ } : {[} \texttt{\ none\ } or \texttt{\ int\ } {]}
    returned array size, must be none or non-negative integer, optional.
  \item
    \texttt{\ replacement\ } : {[} \texttt{\ bool\ } {]} whether the
    sample is with or without replacement, optional; default is true,
    meaning that a value of \texttt{\ arr\ } can be selected multiple
    times.
  \item
    \texttt{\ permutation\ } : {[} \texttt{\ bool\ } {]} whether the
    sample is permuted when sampling without replacement, optional;
    default is true, false provides a speedup.
  \end{itemize}
\item
  \textbf{Output Arguments}

  \begin{itemize}
  \tightlist
  \item
    {[} \texttt{\ array\ } {]} : ( \texttt{\ rng-out\ } ,
    \texttt{\ arr-out\ } )

    \begin{itemize}
    \tightlist
    \item
      \texttt{\ rng-out\ } : {[} \texttt{\ object\ } {]} updated object
      of random number generator.
    \item
      \texttt{\ arr-out\ } : {[} \texttt{\ array\ } {]} generated random
      samples.
    \end{itemize}
  \end{itemize}
\end{itemize}

\subsubsection{\texorpdfstring{\texttt{\ rand-sc\ }}{ rand-sc }}\label{rand-sc}

Generate blind text of Simplified Chinese.

\begin{Shaded}
\begin{Highlighting}[]
\NormalTok{\#let rand{-}sc(words, seed: 42, punctuation: false, gap: 10) = \{...\}}
\end{Highlighting}
\end{Shaded}

\begin{itemize}
\item
  \textbf{Input Arguments}

  \begin{itemize}
  \tightlist
  \item
    \texttt{\ words\ } : {[} \texttt{\ int\ } {]} the length of the
    blind text in pure words.
  \item
    \texttt{\ seed\ } : {[} \texttt{\ int\ } {]} value of seed,
    optional.
  \item
    \texttt{\ punctuation\ } : {[} \texttt{\ bool\ } {]} if true, insert
    punctuations in generated words, optional.
  \item
    \texttt{\ gap\ } : {[} \texttt{\ int\ } {]} average gap between
    punctuations, optional.
  \end{itemize}
\item
  \textbf{Output Arguments}

  \begin{itemize}
  \tightlist
  \item
    {[} \texttt{\ str\ } {]} : generated blind text of Simplified
    Chinese.
  \end{itemize}
\end{itemize}

\subsubsection{How to add}\label{how-to-add}

Copy this into your project and use the import as \texttt{\ suiji\ }

\begin{verbatim}
#import "@preview/suiji:0.3.0"
\end{verbatim}

\includesvg[width=0.16667in,height=0.16667in]{/assets/icons/16-copy.svg}

Check the docs for
\href{https://typst.app/docs/reference/scripting/\#packages}{more
information on how to import packages} .

\subsubsection{About}\label{about}

\begin{description}
\tightlist
\item[Author :]
\href{https://github.com/liuguangxi}{Guangxi Liu}
\item[License:]
MIT
\item[Current version:]
0.3.0
\item[Last updated:]
April 16, 2024
\item[First released:]
March 19, 2024
\item[Minimum Typst version:]
0.11.0
\item[Archive size:]
17.4 kB
\href{https://packages.typst.org/preview/suiji-0.3.0.tar.gz}{\pandocbounded{\includesvg[keepaspectratio]{/assets/icons/16-download.svg}}}
\item[Repository:]
\href{https://github.com/liuguangxi/suiji}{GitHub}
\item[Categor y :]
\begin{itemize}
\tightlist
\item[]
\item
  \pandocbounded{\includesvg[keepaspectratio]{/assets/icons/16-hammer.svg}}
  \href{https://typst.app/universe/search/?category=utility}{Utility}
\end{itemize}
\end{description}

\subsubsection{Where to report issues?}\label{where-to-report-issues}

This package is a project of Guangxi Liu . Report issues on
\href{https://github.com/liuguangxi/suiji}{their repository} . You can
also try to ask for help with this package on the
\href{https://forum.typst.app}{Forum} .

Please report this package to the Typst team using the
\href{https://typst.app/contact}{contact form} if you believe it is a
safety hazard or infringes upon your rights.

\phantomsection\label{versions}
\subsubsection{Version history}\label{version-history}

\begin{longtable}[]{@{}ll@{}}
\toprule\noalign{}
Version & Release Date \\
\midrule\noalign{}
\endhead
\bottomrule\noalign{}
\endlastfoot
0.3.0 & April 16, 2024 \\
\href{https://typst.app/universe/package/suiji/0.2.2/}{0.2.2} & April 9,
2024 \\
\href{https://typst.app/universe/package/suiji/0.2.1/}{0.2.1} & March
28, 2024 \\
\href{https://typst.app/universe/package/suiji/0.2.0/}{0.2.0} & March
22, 2024 \\
\href{https://typst.app/universe/package/suiji/0.1.0/}{0.1.0} & March
19, 2024 \\
\end{longtable}

Typst GmbH did not create this package and cannot guarantee correct
functionality of this package or compatibility with any version of the
Typst compiler or app.


\section{Package List LaTeX/zen-zine.tex}
\title{typst.app/universe/package/zen-zine}

\phantomsection\label{banner}
\phantomsection\label{template-thumbnail}
\pandocbounded{\includegraphics[keepaspectratio]{https://packages.typst.org/preview/thumbnails/zen-zine-0.1.0-small.webp}}

\section{zen-zine}\label{zen-zine}

{ 0.1.0 }

Excellently type-set a fun little zine!

\href{/app?template=zen-zine&version=0.1.0}{Create project in app}

\phantomsection\label{readme}
Excellently type-set a cute little zine about your favorite topic!

Providing your eight pages in order will produce a US-Letter page with
the content in a layout ready to be folded into a zine! The content is
wrapped before movement so that padding and alignment are respected.

Here is the template and its preview:

\begin{Shaded}
\begin{Highlighting}[]
\NormalTok{\#import "@preview/zen{-}zine:0.1.0": zine}

\NormalTok{\#set document(author: "Tom", title: "Zen Zine Example")}
\NormalTok{\#set text(font: "Linux Libertine", lang: "en")}

\NormalTok{\#let my\_eight\_pages = (}
\NormalTok{  range(8).map(}
\NormalTok{    number =\textgreater{} [}
\NormalTok{      \#pad(2em, text(10em, align(center, str(number))))}
\NormalTok{    ]}
\NormalTok{  )}
\NormalTok{)}

\NormalTok{// provide your content pages in order and they}
\NormalTok{// are placed into the zine template positions.}
\NormalTok{// the content is wrapped before movement so that}
\NormalTok{// padding and alignment are respected.}
\NormalTok{\#zine(}
\NormalTok{  // draw\_border: true,}
\NormalTok{  // zine\_page\_margin: 5pt,}
\NormalTok{  contents: my\_eight\_pages}
\NormalTok{)}
\end{Highlighting}
\end{Shaded}

\pandocbounded{\includegraphics[keepaspectratio]{https://github.com/typst/packages/raw/main/packages/preview/zen-zine/0.1.0/template/preview.png}}

\subsection{Improvement Ideas}\label{improvement-ideas}

Roughly in order of priority.

\begin{itemize}
\tightlist
\item
  Write documentation and generate a manual
\item
  Deduce \texttt{\ page\ } properties so that user can change the page
  they wish to use.

  \begin{itemize}
  \tightlist
  \item
    Make sure the page is \texttt{\ flipped\ } and deduce the zine page
    width and height from the full page width and height (and the zine
    margin)
  \item
    I’m currently struggling with finding out the page properties
    (what’s the \texttt{\ \#get\ } equivalent to \texttt{\ \#set\ } ?)
  \end{itemize}
\item
  Add other zine sizes (there is a 16 page one I believe?)
\item
  Digital mode where zine pages are separate pages (of the same size)
  rather than ‘sub pages’ of a printer page
\end{itemize}

\href{/app?template=zen-zine&version=0.1.0}{Create project in app}

\subsubsection{How to use}\label{how-to-use}

Click the button above to create a new project using this template in
the Typst app.

You can also use the Typst CLI to start a new project on your computer
using this command:

\begin{verbatim}
typst init @preview/zen-zine:0.1.0
\end{verbatim}

\includesvg[width=0.16667in,height=0.16667in]{/assets/icons/16-copy.svg}

\subsubsection{About}\label{about}

\begin{description}
\tightlist
\item[Author :]
\href{https://github.com/tomeichlersmith}{Tom Eichlersmith}
\item[License:]
MIT
\item[Current version:]
0.1.0
\item[Last updated:]
April 4, 2024
\item[First released:]
April 4, 2024
\item[Minimum Typst version:]
0.11.0
\item[Archive size:]
2.34 kB
\href{https://packages.typst.org/preview/zen-zine-0.1.0.tar.gz}{\pandocbounded{\includesvg[keepaspectratio]{/assets/icons/16-download.svg}}}
\item[Repository:]
\href{https://github.com/tomeichlersmith/zen-zine}{GitHub}
\item[Categor ies :]
\begin{itemize}
\tightlist
\item[]
\item
  \pandocbounded{\includesvg[keepaspectratio]{/assets/icons/16-smile.svg}}
  \href{https://typst.app/universe/search/?category=fun}{Fun}
\item
  \pandocbounded{\includesvg[keepaspectratio]{/assets/icons/16-layout.svg}}
  \href{https://typst.app/universe/search/?category=layout}{Layout}
\end{itemize}
\end{description}

\subsubsection{Where to report issues?}\label{where-to-report-issues}

This template is a project of Tom Eichlersmith . Report issues on
\href{https://github.com/tomeichlersmith/zen-zine}{their repository} .
You can also try to ask for help with this template on the
\href{https://forum.typst.app}{Forum} .

Please report this template to the Typst team using the
\href{https://typst.app/contact}{contact form} if you believe it is a
safety hazard or infringes upon your rights.

\phantomsection\label{versions}
\subsubsection{Version history}\label{version-history}

\begin{longtable}[]{@{}ll@{}}
\toprule\noalign{}
Version & Release Date \\
\midrule\noalign{}
\endhead
\bottomrule\noalign{}
\endlastfoot
0.1.0 & April 4, 2024 \\
\end{longtable}

Typst GmbH did not create this template and cannot guarantee correct
functionality of this template or compatibility with any version of the
Typst compiler or app.


\section{Package List LaTeX/paddling-tongji-thesis.tex}
\title{typst.app/universe/package/paddling-tongji-thesis}

\phantomsection\label{banner}
\phantomsection\label{template-thumbnail}
\pandocbounded{\includegraphics[keepaspectratio]{https://packages.typst.org/preview/thumbnails/paddling-tongji-thesis-0.1.1-small.webp}}

\section{paddling-tongji-thesis}\label{paddling-tongji-thesis}

{ 0.1.1 }

å?ŒæµŽå¤§å­¦æœ¬ç§`ç''Ÿæ¯•ä¸šè®¾è®¡è®ºæ--‡æ¨¡æ?¿ \textbar{} Tongji
University Undergraduate Thesis Template

\href{/app?template=paddling-tongji-thesis&version=0.1.1}{Create project
in app}

\phantomsection\label{readme}
中æ--‡ \textbar{}
\href{https://github.com/typst/packages/raw/main/packages/preview/paddling-tongji-thesis/0.1.1/README-EN.md}{English}

\begin{quote}
{[}!CAUTION{]} ç''±äºŽ Typst
项目ä»?处于密集å?{}`展阶段,ä¸''对æŸ?些功能的æ''¯æŒ?ä¸?完å--„,å›~此本模æ?¿å?¯èƒ½å­˜åœ¨ä¸€äº›é---®é¢˜ã€‚如果您在使ç''¨è¿‡ç¨‹ä¸­é?‡åˆ°äº†é---®é¢˜ï¼Œæ¬¢è¿Žæ??交
issue æˆ-- PR,æˆ`们会尽力解决。

在此期é---´ï¼Œæ¬¢è¿Žå¤§å®¶ä½¿ç''¨
\href{https://github.com/TJ-CSCCG/tongji-undergrad-thesis}{æˆ`们的
LaTeX 模�} 。
\end{quote}

\subsection{æ~·ä¾‹å±•ç¤º}\label{uxe6-uxe4uxbeuxe5uxe7uxba}

以下�次展示
“å°?é?¢â€?ã€?“中æ--‡æ`˜è¦?â€?ã€?“目录â€?ã€?“主è¦?å†\ldots 容â€?ã€?“å?‚考æ--‡çŒ®â€?
与 “谢辞�。

\includegraphics[width=0.3\linewidth,height=\textheight,keepaspectratio]{https://media.githubusercontent.com/media/TJ-CSCCG/TJCS-Images/tongji-undergrad-thesis-typst/preview/main_page-0001.jpg}
\includegraphics[width=0.3\linewidth,height=\textheight,keepaspectratio]{https://media.githubusercontent.com/media/TJ-CSCCG/TJCS-Images/tongji-undergrad-thesis-typst/preview/main_page-0002.jpg}
\includegraphics[width=0.3\linewidth,height=\textheight,keepaspectratio]{https://media.githubusercontent.com/media/TJ-CSCCG/TJCS-Images/tongji-undergrad-thesis-typst/preview/main_page-0004.jpg}
\includegraphics[width=0.3\linewidth,height=\textheight,keepaspectratio]{https://media.githubusercontent.com/media/TJ-CSCCG/TJCS-Images/tongji-undergrad-thesis-typst/preview/main_page-0005.jpg}
\includegraphics[width=0.3\linewidth,height=\textheight,keepaspectratio]{https://media.githubusercontent.com/media/TJ-CSCCG/TJCS-Images/tongji-undergrad-thesis-typst/preview/main_page-0019.jpg}
\includegraphics[width=0.3\linewidth,height=\textheight,keepaspectratio]{https://media.githubusercontent.com/media/TJ-CSCCG/TJCS-Images/tongji-undergrad-thesis-typst/preview/main_page-0020.jpg}

\subsection{使ç''¨æ--¹æ³•}\label{uxe4uxbduxe7uxe6uxb9uxe6uxb3}

\subsubsection{在线 Web App}\label{uxe5ux153uxe7uxba-web-app}

\paragraph{创建项目}\label{uxe5ux2c6uxe5uxbauxe9uxb9uxe7}

\begin{itemize}
\item
  æ‰``å¼€ Typst Universe 中的
  \href{https://www.overleaf.com/latex/templates/tongji-university-undergraduate-thesis-template/tfvdvyggqybn}{\pandocbounded{\includegraphics[keepaspectratio]{https://img.shields.io/badge/Typst-paddling--tongji--thesis-239dae}}}
  并点击 \texttt{\ Create\ project\ in\ app\ } 。
\item
  æˆ--在 \href{https://typst.app/}{Typst Web App} 中选择
  \texttt{\ Start\ from\ a\ template\ } ,然�选择
  \texttt{\ paddling-tongji-thesis\ } 。
\end{itemize}

\paragraph{上ä¼~å­---ä½``}\label{uxe4ux161uxe4uxbc-uxe5uxe4uxbd}

从
\href{https://github.com/TJ-CSCCG/tongji-undergrad-thesis-typst/tree/fonts}{\texttt{\ fonts\ }
分æ''¯}
下载所有å­---ä½``æ--‡ä»¶ï¼Œä¸Šä¼~到该项目的æ~¹ç›®å½•ã€‚å?³å?¯å¼€å§‹ä½¿ç''¨ã€‚

\subsubsection{本地使ç''¨}\label{uxe6ux153uxe5ux153uxe4uxbduxe7}

\paragraph{1. 安è£\ldots{} Typst}\label{uxe5uxe8-typst}

å?‚ç\ldots§
\href{https://github.com/typst/typst?tab=readme-ov-file\#installation}{Typst}
官æ--¹æ--‡æ¡£å®‰è£\ldots{} Typst。

\paragraph{2. 下载å­---ä½``}\label{uxe4uxe8uxbduxbduxe5uxe4uxbd}

从
\href{https://github.com/TJ-CSCCG/tongji-undergrad-thesis-typst/tree/fonts}{\texttt{\ fonts\ }
分æ''¯} 下载所有å­---ä½``æ--‡ä»¶ï¼Œå¹¶
\textbf{安è£\ldots 到系统中} 。

\paragraph{\texorpdfstring{使ç''¨ \texttt{\ typst\ }
åˆ?始åŒ--}{使ç''¨  typst  åˆ?始åŒ--}}\label{uxe4uxbduxe7-typst-uxe5ux2c6uxe5uxe5ux153}

\subparagraph{åˆ?始åŒ--项目}\label{uxe5ux2c6uxe5uxe5ux153uxe9uxb9uxe7}

\begin{Shaded}
\begin{Highlighting}[]
\ExtensionTok{typst}\NormalTok{ init @preview/paddling{-}tongji{-}thesis}
\end{Highlighting}
\end{Shaded}

\subparagraph{ç¼--è¯`}\label{uxe7uxbcuxe8}

\begin{Shaded}
\begin{Highlighting}[]
\ExtensionTok{typst}\NormalTok{ compile main.typ}
\end{Highlighting}
\end{Shaded}

\paragraph{\texorpdfstring{使ç''¨ \texttt{\ git\ clone\ }
åˆ?始åŒ--}{使ç''¨  git clone  åˆ?始åŒ--}}\label{uxe4uxbduxe7-git-clone-uxe5ux2c6uxe5uxe5ux153}

\subparagraph{Git Clone 项目}\label{git-clone-uxe9uxb9uxe7}

\begin{Shaded}
\begin{Highlighting}[]
\FunctionTok{git}\NormalTok{ clone https://github.com/TJ{-}CSCCG/tongji{-}undergrad{-}thesis{-}typst.git}
\BuiltInTok{cd}\NormalTok{ tongji{-}undergrad{-}thesis{-}typst}
\end{Highlighting}
\end{Shaded}

\subparagraph{ç¼--è¯`}\label{uxe7uxbcuxe8-1}

\begin{Shaded}
\begin{Highlighting}[]
\ExtensionTok{typst}\NormalTok{ compile init{-}files/main.typ }\AttributeTok{{-}{-}root}\NormalTok{ .}
\end{Highlighting}
\end{Shaded}

\begin{quote}
{[}!TIP{]}
è‹¥ä½~ä¸?想把项目使ç''¨çš„å­---ä½``安è£\ldots 到系统中,å?¯ä»¥åœ¨ç¼--è¯`æ---¶æŒ‡å®šå­---ä½``路径,例如:

\begin{Shaded}
\begin{Highlighting}[]
\ExtensionTok{typst}\NormalTok{ compile init{-}files/main.typ }\AttributeTok{{-}{-}root}\NormalTok{ . }\AttributeTok{{-}{-}font{-}path}\NormalTok{ \{YOUR\_FONT\_PATH\}}
\end{Highlighting}
\end{Shaded}
\end{quote}

\subsection{如何为该项目贡献代ç~??}\label{uxe5uxe4uxbduxe4uxbauxe8uxe9uxb9uxe7uxe8uxe7ux153uxe4uxe7-uxefuxbcuxff}

还请查看
\href{https://github.com/typst/packages/raw/main/packages/preview/paddling-tongji-thesis/0.1.1/CONTRIBUTING.md/\#how-to-pull-request}{How
to pull request} 。

\subsection{å¼€æº?å??è®®}\label{uxe5uxbcuxe6uxbauxe5uxe8}

该项目使ç''¨
\href{https://github.com/typst/packages/raw/main/packages/preview/paddling-tongji-thesis/0.1.1/LICENSE}{MIT
License} å¼€æº?å??议。

\subsubsection{å\ldots?责声明}\label{uxe5uxe8uxe5uxe6ux17e}

本项目使ç''¨äº†æ--¹æ­£å­---åº``中的å­---ä½``,版æ?ƒå½'æ--¹æ­£å­---åº``所有。本项目ä»\ldots ç''¨äºŽå­¦ä¹~交æµ?,ä¸?å¾---ç''¨äºŽå•†ä¸šç''¨é€''。

\subsection{有å\ldots³çª?出贡献的说明}\label{uxe6ux153uxe5uxb3uxe7uxaauxe5uxbauxe8uxe7ux153uxe7ux161uxe8uxe6ux17e}

\begin{itemize}
\tightlist
\item
  该项目起�于 \href{https://github.com/seashell11234455}{FeO3}
  的�始版本项目
  \href{https://github.com/TJ-CSCCG/tongji-undergrad-thesis-typst/tree/lky}{tongji-undergrad-thesis-typst}
  。
\item
  å?Žæ?¥ \href{https://github.com/RizhongLin}{RizhongLin}
  对模æ?¿è¿›è¡Œäº†å®Œå--„,使å\ldots¶æ›´åŠ~符å?ˆå?ŒæµŽå¤§å­¦æœ¬ç§`ç''Ÿæ¯•ä¸šè®¾è®¡è®ºæ--‡çš„è¦?求,并增åŠ~了é'ˆå¯¹
  Typst 的基础教程。
\end{itemize}

æˆ`们é?žå¸¸æ„Ÿè°¢ä»¥ä¸Šè´¡çŒ®è€\ldots 的付出,ä»--们的工作为更多å?Œå­¦æ??供了æ--¹ä¾¿å'Œå¸®åŠ©ã€‚

在使ç''¨æœ¬æ¨¡æ?¿æ---¶ï¼Œå¦‚果您觉å¾---本项目对您的毕业设计æˆ--论æ--‡æœ‰æ‰€å¸®åŠ©ï¼Œæˆ`们希望您å?¯ä»¥åœ¨æ‚¨çš„致谢部分感谢并致以敬æ„?。

\subsection{致谢}\label{uxe8uxe8}

æˆ`们从顶å°--高æ~¡çš„优秀开æº?项目中学到了很多:

\begin{itemize}
\tightlist
\item
  \href{https://github.com/lucifer1004/pkuthss-typst}{lucifer1004/pkuthss-typst}
\item
  \href{https://github.com/werifu/HUST-typst-template}{werifu/HUST-typst-template}
\end{itemize}

\subsection{è?''ç³»æ--¹å¼?}\label{uxe8uxe7uxb3uxe6uxb9uxe5uxbc}

\begin{Shaded}
\begin{Highlighting}[]
\CommentTok{\# Python}
\NormalTok{[}
    \StringTok{\textquotesingle{}rizhonglin@$.\%\textquotesingle{}}\NormalTok{.replace(}\StringTok{\textquotesingle{}$\textquotesingle{}}\NormalTok{, }\StringTok{\textquotesingle{}epfl\textquotesingle{}}\NormalTok{).replace(}\StringTok{\textquotesingle{}\%\textquotesingle{}}\NormalTok{, }\StringTok{\textquotesingle{}ch\textquotesingle{}}\NormalTok{),}
\NormalTok{]}
\end{Highlighting}
\end{Shaded}

\subsubsection{QQ 群}\label{qq-uxe7uxbe}

\begin{itemize}
\tightlist
\item
  TJ-CSCCG 交�群: \texttt{\ 1013806782\ }
\end{itemize}

\href{/app?template=paddling-tongji-thesis&version=0.1.1}{Create project
in app}

\subsubsection{How to use}\label{how-to-use}

Click the button above to create a new project using this template in
the Typst app.

You can also use the Typst CLI to start a new project on your computer
using this command:

\begin{verbatim}
typst init @preview/paddling-tongji-thesis:0.1.1
\end{verbatim}

\includesvg[width=0.16667in,height=0.16667in]{/assets/icons/16-copy.svg}

\subsubsection{About}\label{about}

\begin{description}
\tightlist
\item[Author :]
\href{https://github.com/TJ-CSCCG}{TJ-CSCCG}
\item[License:]
MIT
\item[Current version:]
0.1.1
\item[Last updated:]
July 1, 2024
\item[First released:]
July 1, 2024
\item[Archive size:]
32.7 kB
\href{https://packages.typst.org/preview/paddling-tongji-thesis-0.1.1.tar.gz}{\pandocbounded{\includesvg[keepaspectratio]{/assets/icons/16-download.svg}}}
\item[Repository:]
\href{https://github.com/TJ-CSCCG/tongji-undergrad-thesis-typst.git}{GitHub}
\item[Categor y :]
\begin{itemize}
\tightlist
\item[]
\item
  \pandocbounded{\includesvg[keepaspectratio]{/assets/icons/16-mortarboard.svg}}
  \href{https://typst.app/universe/search/?category=thesis}{Thesis}
\end{itemize}
\end{description}

\subsubsection{Where to report issues?}\label{where-to-report-issues}

This template is a project of TJ-CSCCG . Report issues on
\href{https://github.com/TJ-CSCCG/tongji-undergrad-thesis-typst.git}{their
repository} . You can also try to ask for help with this template on the
\href{https://forum.typst.app}{Forum} .

Please report this template to the Typst team using the
\href{https://typst.app/contact}{contact form} if you believe it is a
safety hazard or infringes upon your rights.

\phantomsection\label{versions}
\subsubsection{Version history}\label{version-history}

\begin{longtable}[]{@{}ll@{}}
\toprule\noalign{}
Version & Release Date \\
\midrule\noalign{}
\endhead
\bottomrule\noalign{}
\endlastfoot
0.1.1 & July 1, 2024 \\
\end{longtable}

Typst GmbH did not create this template and cannot guarantee correct
functionality of this template or compatibility with any version of the
Typst compiler or app.


\section{Package List LaTeX/aio-studi-and-thesis.tex}
\title{typst.app/universe/package/aio-studi-and-thesis}

\phantomsection\label{banner}
\phantomsection\label{template-thumbnail}
\pandocbounded{\includegraphics[keepaspectratio]{https://packages.typst.org/preview/thumbnails/aio-studi-and-thesis-0.1.0-small.webp}}

\section{aio-studi-and-thesis}\label{aio-studi-and-thesis}

{ 0.1.0 }

All-in-one template for students and theses

\href{/app?template=aio-studi-and-thesis&version=0.1.0}{Create project
in app}

\phantomsection\label{readme}
\href{https://github.com/fuchs-fabian/typst-template-aio-studi-and-thesis/blob/main/docs/manual-de.pdf}{\pandocbounded{\includegraphics[keepaspectratio]{https://img.shields.io/website?down_message=offline&label=manual\%20de&up_color=007aff&up_message=online&url=https\%3A\%2F\%2Fgithub.com\%2Ffuchs-fabian\%2Ftypst-template-aio-studi-and-thesis\%2Fblob\%2Fmain\%2Fdocs\%2Fmanual-de.pdf}}}
\href{https://github.com/fuchs-fabian/typst-template-aio-studi-and-thesis/blob/main/docs/manual-en.pdf}{\pandocbounded{\includegraphics[keepaspectratio]{https://img.shields.io/website?down_message=offline&label=manual\%20en&up_color=007aff&up_message=online&url=https\%3A\%2F\%2Fgithub.com\%2Ffuchs-fabian\%2Ftypst-template-aio-studi-and-thesis\%2Fblob\%2Fmain\%2Fdocs\%2Fmanual-en.pdf}}}
\href{https://github.com/fuchs-fabian/typst-template-aio-studi-and-thesis/blob/main/docs/example-de-thesis.pdf}{\pandocbounded{\includegraphics[keepaspectratio]{https://img.shields.io/website?down_message=offline&label=example\%20de&up_color=007aff&up_message=online&url=https\%3A\%2F\%2Fgithub.com\%2Ffuchs-fabian\%2Ftypst-template-aio-studi-and-thesis\%2Fblob\%2Fmain\%2Fdocs\%2Fexample-de-thesis.pdf}}}
\href{https://github.com/fuchs-fabian/typst-template-aio-studi-and-thesis/blob/main/docs/example-en-thesis.pdf}{\pandocbounded{\includegraphics[keepaspectratio]{https://img.shields.io/website?down_message=offline&label=example\%20en&up_color=007aff&up_message=online&url=https\%3A\%2F\%2Fgithub.com\%2Ffuchs-fabian\%2Ftypst-template-aio-studi-and-thesis\%2Fblob\%2Fmain\%2Fdocs\%2Fexample-en-thesis.pdf}}}
\href{https://github.com/fuchs-fabian/typst-template-aio-studi-and-thesis/blob/main/LICENSE}{\pandocbounded{\includegraphics[keepaspectratio]{https://img.shields.io/badge/license-MIT-brightgreen}}}

This template can be used for extensive documentation as well as for
final theses such as bachelor theses.

It is characterised by the fact that it is highly customisable despite
the predefined design.

Initially, all template parameters are optional by default. It is then
suitable for documentation. To make it suitable for theses, only one
parameter needs to be changed.

\subsection{\texorpdfstring{âš~ï¸? \textbf{Disclaimer -
Important!}}{âš~ï¸? Disclaimer - Important!}}\label{uxe2ux161-uxef-disclaimer---important}

\begin{itemize}
\tightlist
\item
  It is a template and does not have to meet the exact requirements of
  your university
\item
  It is only supported in German and English (Default setting: German)
\end{itemize}

\subsection{Getting Started}\label{getting-started}

You can use this template in the Typst web app by clicking “Start from
template� on the dashboard and searching for
\texttt{\ aio-studi-and-thesis\ } .

Alternatively, you can use the CLI to kick this project off using the
command

\begin{Shaded}
\begin{Highlighting}[]
\ExtensionTok{typst}\NormalTok{ init @preview/aio{-}studi{-}and{-}thesis}
\end{Highlighting}
\end{Shaded}

Typst will create a new directory with all the files needed to get you
started.

\subsection{Usage}\label{usage}

The template (
\href{https://github.com/typst/packages/raw/main/packages/preview/aio-studi-and-thesis/0.1.0/docs/example-de-thesis.pdf}{rendered
PDF (DE)} ) contains thesis writing advice (in German) as example
content.

If you are looking for the details of this template package’s
function, take a look at the
\href{https://github.com/typst/packages/raw/main/packages/preview/aio-studi-and-thesis/0.1.0/docs/manual-de.pdf}{german
manual} or the
\href{https://github.com/typst/packages/raw/main/packages/preview/aio-studi-and-thesis/0.1.0/docs/manual-en.pdf}{english
manual} .

\begin{quote}
Roboto is used as the default font. Please note accordingly if you want
to use exactly this font.
\end{quote}

\subsection{Example configuration}\label{example-configuration}

\begin{Shaded}
\begin{Highlighting}[]
\NormalTok{\#import "@preview/aio{-}studi{-}and{-}thesis:0.1.0": *}

\NormalTok{\#show: project.with(}
\NormalTok{  lang: "de",}
\NormalTok{  authors: (}
\NormalTok{    (name: "Firstname Lastname"),}
\NormalTok{  ),}
\NormalTok{  title: "Title",}
\NormalTok{  subtitle: "Subtitle",}
\NormalTok{  cover{-}sheet: (}
\NormalTok{    cover{-}image: none,}
\NormalTok{    description: []}
\NormalTok{  )}
\NormalTok{)}
\end{Highlighting}
\end{Shaded}

\subsection{\texorpdfstring{Donate with
\href{https://www.paypal.com/donate/?hosted_button_id=4G9X8TDNYYNKG}{PayPal}}{Donate with PayPal}}\label{donate-with-paypal}

If you think this template is useful and saves you a lot of work and
nerves (Word and LaTex can be very tiring) and lets you sleep better,
then a small donation would be very nice.

\href{https://www.paypal.com/donate/?hosted_button_id=4G9X8TDNYYNKG}{\pandocbounded{\includegraphics[keepaspectratio]{https://www.paypalobjects.com/de_DE/i/btn/btn_donateCC_LG.gif}}}

\href{/app?template=aio-studi-and-thesis&version=0.1.0}{Create project
in app}

\subsubsection{How to use}\label{how-to-use}

Click the button above to create a new project using this template in
the Typst app.

You can also use the Typst CLI to start a new project on your computer
using this command:

\begin{verbatim}
typst init @preview/aio-studi-and-thesis:0.1.0
\end{verbatim}

\includesvg[width=0.16667in,height=0.16667in]{/assets/icons/16-copy.svg}

\subsubsection{About}\label{about}

\begin{description}
\tightlist
\item[Author :]
\href{https://github.com/fuchs-fabian/}{Fabian Fuchs}
\item[License:]
MIT
\item[Current version:]
0.1.0
\item[Last updated:]
July 31, 2024
\item[First released:]
July 31, 2024
\item[Minimum Typst version:]
0.11.1
\item[Archive size:]
352 kB
\href{https://packages.typst.org/preview/aio-studi-and-thesis-0.1.0.tar.gz}{\pandocbounded{\includesvg[keepaspectratio]{/assets/icons/16-download.svg}}}
\item[Repository:]
\href{https://github.com/fuchs-fabian/typst-template-aio-studi-and-thesis}{GitHub}
\item[Discipline :]
\begin{itemize}
\tightlist
\item[]
\item
  \href{https://typst.app/universe/search/?discipline=computer-science}{Computer
  Science}
\end{itemize}
\item[Categor y :]
\begin{itemize}
\tightlist
\item[]
\item
  \pandocbounded{\includesvg[keepaspectratio]{/assets/icons/16-mortarboard.svg}}
  \href{https://typst.app/universe/search/?category=thesis}{Thesis}
\end{itemize}
\end{description}

\subsubsection{Where to report issues?}\label{where-to-report-issues}

This template is a project of Fabian Fuchs . Report issues on
\href{https://github.com/fuchs-fabian/typst-template-aio-studi-and-thesis}{their
repository} . You can also try to ask for help with this template on the
\href{https://forum.typst.app}{Forum} .

Please report this template to the Typst team using the
\href{https://typst.app/contact}{contact form} if you believe it is a
safety hazard or infringes upon your rights.

\phantomsection\label{versions}
\subsubsection{Version history}\label{version-history}

\begin{longtable}[]{@{}ll@{}}
\toprule\noalign{}
Version & Release Date \\
\midrule\noalign{}
\endhead
\bottomrule\noalign{}
\endlastfoot
0.1.0 & July 31, 2024 \\
\end{longtable}

Typst GmbH did not create this template and cannot guarantee correct
functionality of this template or compatibility with any version of the
Typst compiler or app.


\section{Package List LaTeX/qcm.tex}
\title{typst.app/universe/package/qcm}

\phantomsection\label{banner}
\section{qcm}\label{qcm}

{ 0.1.0 }

Qualitative Colormaps

\phantomsection\label{readme}
Qualitative Colormaps for Typst

Qualitative colormaps contain a fixed number of distinct and easily
differentiable colors. They are suitable to use for e.g. categorical
data visualization.

\subsection{Source}\label{source}

The following colormaps are available:

\begin{itemize}
\tightlist
\item
  all \href{https://github.com/axismaps/colorbrewer/}{colorbrew}
  qualitive colormaps, for discovery and as documentation visit
  \href{https://colorbrewer2.org/}{colorbrewer2.org}
\end{itemize}

\subsection{Usage}\label{usage}

Usage is very simple:

\begin{Shaded}
\begin{Highlighting}[]
\NormalTok{\#import "@preview/qcm:0.1.0": colormap}

\NormalTok{\#colormap("Set1", 5)}
\end{Highlighting}
\end{Shaded}

\subsubsection{How to add}\label{how-to-add}

Copy this into your project and use the import as \texttt{\ qcm\ }

\begin{verbatim}
#import "@preview/qcm:0.1.0"
\end{verbatim}

\includesvg[width=0.16667in,height=0.16667in]{/assets/icons/16-copy.svg}

Check the docs for
\href{https://typst.app/docs/reference/scripting/\#packages}{more
information on how to import packages} .

\subsubsection{About}\label{about}

\begin{description}
\tightlist
\item[Author :]
Ludwig Austermann
\item[License:]
MIT
\item[Current version:]
0.1.0
\item[Last updated:]
April 12, 2024
\item[First released:]
April 12, 2024
\item[Minimum Typst version:]
0.10.0
\item[Archive size:]
2.37 kB
\href{https://packages.typst.org/preview/qcm-0.1.0.tar.gz}{\pandocbounded{\includesvg[keepaspectratio]{/assets/icons/16-download.svg}}}
\item[Repository:]
\href{https://github.com/ludwig-austermann/qcm}{GitHub}
\end{description}

\subsubsection{Where to report issues?}\label{where-to-report-issues}

This package is a project of Ludwig Austermann . Report issues on
\href{https://github.com/ludwig-austermann/qcm}{their repository} . You
can also try to ask for help with this package on the
\href{https://forum.typst.app}{Forum} .

Please report this package to the Typst team using the
\href{https://typst.app/contact}{contact form} if you believe it is a
safety hazard or infringes upon your rights.

\phantomsection\label{versions}
\subsubsection{Version history}\label{version-history}

\begin{longtable}[]{@{}ll@{}}
\toprule\noalign{}
Version & Release Date \\
\midrule\noalign{}
\endhead
\bottomrule\noalign{}
\endlastfoot
0.1.0 & April 12, 2024 \\
\end{longtable}

Typst GmbH did not create this package and cannot guarantee correct
functionality of this package or compatibility with any version of the
Typst compiler or app.


\section{Package List LaTeX/leipzig-glossing.tex}
\title{typst.app/universe/package/leipzig-glossing}

\phantomsection\label{banner}
\section{leipzig-glossing}\label{leipzig-glossing}

{ 0.4.0 }

Linguistic interlinear glosses according to the Leipzig Glossing rules

\phantomsection\label{readme}
\texttt{\ leipzig-glossing\ } is a
\href{https://github.com/typst/typst}{Typst} library for creating
interlinear morpheme-by-morpheme glosses according to the
\href{https://www.eva.mpg.de/lingua/pdf/Glossing-Rules.pdf}{Leipzig
glossing rules} .

Run \texttt{\ typst\ compile\ documentation.typ\ } in the root of the
repository to generate a pdf file with examples and documentation. This
command is also codified in the accompanying
\href{https://github.com/casey/just}{justfile} as
\texttt{\ just\ build-doc\ } .

The definitions intended for use by end users are the \texttt{\ gloss\ }
and \texttt{\ numbered-gloss\ } functions, and the
\texttt{\ abbreviations\ } submodule.

\subsection{Repositories}\label{repositories}

The canonical repository for this project is on the
\href{https://code.everydayimshuflin.com/greg/typst-lepizig-glossing}{Gitea
instance} .

There is also a
\href{https://github.com/neunenak/typst-leipzig-glossing/}{Github
mirror} , and a \href{https://radicle.xyz/}{Radicle} mirror available at
{rad://z2j7vQLS3EtQbPkrzi7Tn2XR7YWLw} .

Bug reports and code contributions are welcome from all users.

\subsection{License}\label{license}

This library uses the MIT license; see \texttt{\ LICENSE.txt\ } .

\subsection{Contributors}\label{contributors}

Thanks to \href{https://github.com/betoma}{Bethany E. Toma} for a number
of suggestions and improvements.

Thanks to \href{https://github.com/rwmpelstilzchen}{Maja
Abramski-Kronenberg} for the labeling functionality.

\subsubsection{How to add}\label{how-to-add}

Copy this into your project and use the import as
\texttt{\ leipzig-glossing\ }

\begin{verbatim}
#import "@preview/leipzig-glossing:0.4.0"
\end{verbatim}

\includesvg[width=0.16667in,height=0.16667in]{/assets/icons/16-copy.svg}

Check the docs for
\href{https://typst.app/docs/reference/scripting/\#packages}{more
information on how to import packages} .

\subsubsection{About}\label{about}

\begin{description}
\tightlist
\item[Author s :]
\href{mailto:greg@everydayimshuflin.com}{Greg Shuflin} \& Other
open-source contributors
\item[License:]
MIT
\item[Current version:]
0.4.0
\item[Last updated:]
November 12, 2024
\item[First released:]
July 6, 2023
\item[Minimum Typst version:]
0.12.0
\item[Archive size:]
5.26 kB
\href{https://packages.typst.org/preview/leipzig-glossing-0.4.0.tar.gz}{\pandocbounded{\includesvg[keepaspectratio]{/assets/icons/16-download.svg}}}
\item[Repository:]
\href{https://code.everydayimshuflin.com/greg/typst-lepizig-glossing}{code.everydayimshuflin.com}
\item[Discipline :]
\begin{itemize}
\tightlist
\item[]
\item
  \href{https://typst.app/universe/search/?discipline=linguistics}{Linguistics}
\end{itemize}
\item[Categor y :]
\begin{itemize}
\tightlist
\item[]
\item
  \pandocbounded{\includesvg[keepaspectratio]{/assets/icons/16-atom.svg}}
  \href{https://typst.app/universe/search/?category=paper}{Paper}
\end{itemize}
\end{description}

\subsubsection{Where to report issues?}\label{where-to-report-issues}

This package is a project of Greg Shuflin and Other open-source
contributors . Report issues on
\href{https://code.everydayimshuflin.com/greg/typst-lepizig-glossing}{their
repository} . You can also try to ask for help with this package on the
\href{https://forum.typst.app}{Forum} .

Please report this package to the Typst team using the
\href{https://typst.app/contact}{contact form} if you believe it is a
safety hazard or infringes upon your rights.

\phantomsection\label{versions}
\subsubsection{Version history}\label{version-history}

\begin{longtable}[]{@{}ll@{}}
\toprule\noalign{}
Version & Release Date \\
\midrule\noalign{}
\endhead
\bottomrule\noalign{}
\endlastfoot
0.4.0 & November 12, 2024 \\
\href{https://typst.app/universe/package/leipzig-glossing/0.3.0/}{0.3.0}
& August 21, 2024 \\
\href{https://typst.app/universe/package/leipzig-glossing/0.2.0/}{0.2.0}
& October 4, 2023 \\
\href{https://typst.app/universe/package/leipzig-glossing/0.1.0/}{0.1.0}
& July 6, 2023 \\
\end{longtable}

Typst GmbH did not create this package and cannot guarantee correct
functionality of this package or compatibility with any version of the
Typst compiler or app.


\section{Package List LaTeX/blinky.tex}
\title{typst.app/universe/package/blinky}

\phantomsection\label{banner}
\section{blinky}\label{blinky}

{ 0.1.0 }

Typesets paper titles in bibliographies as hyperlinks.

\phantomsection\label{readme}
This package permits the creation of Typst bibliographies in which paper
titles are typeset as hyperlinks. Here’s an example (with links
typeset in blue):

\includegraphics[width=0.8\linewidth,height=\textheight,keepaspectratio]{https://raw.githubusercontent.com/alexanderkoller/typst-blinky/main/examples/screenshot.png}

The bibliography is generated from a Bibtex file, and citations are done
with the usual Typst mechanisms. The hyperlinks are specified through
DOI or URL fields in the Bibtex entries; if such a field is present, the
title of the entry will be automatically typeset as a hyperlink.

See
\href{https://github.com/alexanderkoller/typst-blinky/tree/main/examples}{here}
for a full example.

\subsection{Usage}\label{usage}

Adding hyperlinks to your bibliography is a two-step process: (a) use a
CSL style with magic symbols (explained below), and (b) enclose the
\texttt{\ bibliography\ } command with the \texttt{\ link-bib-urls\ }
function:

\begin{verbatim}
#import "@preview/blinky:0.1.0": link-bib-urls

... @cite something ... @cite more ...

#let bibsrc = read("custom.bib")
#link-bib-urls(bibsrc)[
  #bibliography("custom.bib", style: "./association-for-computational-linguistics-blinky.csl")
]
\end{verbatim}

Observe that the Bibtex file \texttt{\ custom.bib\ } is loaded twice:
once to load into \texttt{\ link-bib-urls\ } and once in the standard
Typst \texttt{\ bibliography\ } command. Obviously, this needs to be the
same file twice. See under “Alternative solutions� below why this
can’t be simplified further at the moment.

If a Bibtex entry contains a DOI field, the title will become a
hyperlink to the DOI. Otherwise, if the Bibtex entry contains a URL
field, the title will become a hyperlink to this URL. Otherwise, the
title will be shown as normal, without a link.

\subsection{CSL with magic symbols}\label{csl-with-magic-symbols}

Blinky generates the hyperlinked titles through a regex show rule that
replaces a “magic symbol� with a
\href{https://typst.app/docs/reference/model/link/}{link} command. This
“magic symbol� is a string of the form
\texttt{\ !!BIBENTRY!\textless{}key\textgreater{}!!\ } , where
\texttt{\ \textless{}key\textgreater{}\ } is the Bibtex citation key of
the reference.

You will therefore need to tweak your CSL style to use it with Blinky.
Specifically, in every place where you would usually have the paper
title, i.e.

\begin{verbatim}
\end{verbatim}

or similar, your CSL file now instead needs to print a decorated version
of the paper’s citation-key (= Bibtex key):

\begin{verbatim}
\end{verbatim}

You can have more prefix before and suffix after the
\texttt{\ !!BIBENTRY!\ } and \texttt{\ !!\ } , as in the example, but
these magic symbols need to be there so Blinky can find the places in
the document where the hyperlinked title needs to be inserted.

You can check the
\href{https://github.com/alexanderkoller/typst-blinky/blob/main/examples/association-for-computational-linguistics-blinky.csl}{example
CSL file} to see what this looks like in practice; compare to
\href{https://github.com/citation-style-language/styles/blob/master/association-for-computational-linguistics.csl}{the
unmodified original} .

\subsection{Alternative solutions}\label{alternative-solutions}

The current mechanism in Blinky is somewhat heavy-handed: a Typst plugin
uses the \href{https://github.com/typst/biblatex}{biblatex} crate to
parse the Bibtex file (independently of the normal operations of the
\texttt{\ bibliography\ } command), and then all occurrences of the
magic symbol in the Typst bibliography are replaced by the hyperlinked
titles.

It would be great to replace this mechanism by something simpler, but it
is actually remarkably tricky to make bibliography titles hyperlinks
with the current version of Typst (0.11.1). All the alternatives that I
could think of don’t work. Here are some of them:

\begin{itemize}
\tightlist
\item
  Print the URL/DOI using the CSL style, and then use a regex show rule
  to convert it into a \texttt{\ link\ } around the title somehow. This
  does not work because most URLs contain a colon character (:), and
  these \href{https://github.com/typst/typst/issues/86}{cause trouble
  with Typst regexes} .
\item
  Make the CSL style output text of the form
  \texttt{\ \#link(url){[}title{]}\ } . This does not work because the
  content generated by CSL is not evaluated further by Typst. Also,
  Typst \href{https://github.com/typst/typst/issues/942}{does not
  support show rules for the individual bibliography items} , which
  makes it tricky to call
  \href{https://typst.app/docs/reference/foundations/eval/}{eval} on
  them.
\item
  Create a show rule for \texttt{\ link\ } . Some CSL styles already
  generate \texttt{\ link\ } elements if a URL/DOI is present in the bib
  entry - one could consider replacing it with a \texttt{\ link\ } whose
  URL is the same as before, but the text is a link symbol or some such.
  However, a show rule for a link that generates another link runs into
  an infinite recursion; Typst made
  \href{https://github.com/typst/typst/pull/3327}{the deliberate
  decision} to handle such recursions only for \texttt{\ text\ } show
  rules.
\item
  The best solution would be to simply use an unmodified CSL file, but
  it is not clear to me how one would pick out the paper title from the
  bibliography in a general way. I’m afraid that any solution that
  hyperlinks titles will require modifications to the CSL style.
\end{itemize}

It would furthermore be desirable to hide the fact that we are reading
the same Bibtex file twice behind a single function call. However, code
in a Typst package
\href{https://github.com/typst/typst/issues/2126}{resolves all filenames
relative to the package directory} , which means that the package cannot
access a bibliography file outside of the package directory. We may be
able to simplify this once
\href{https://github.com/typst/typst/issues/971}{\#971} gets addressed.

\subsubsection{How to add}\label{how-to-add}

Copy this into your project and use the import as \texttt{\ blinky\ }

\begin{verbatim}
#import "@preview/blinky:0.1.0"
\end{verbatim}

\includesvg[width=0.16667in,height=0.16667in]{/assets/icons/16-copy.svg}

Check the docs for
\href{https://typst.app/docs/reference/scripting/\#packages}{more
information on how to import packages} .

\subsubsection{About}\label{about}

\begin{description}
\tightlist
\item[Author :]
\href{mailto:akoller@gmail.com}{Alexander Koller}
\item[License:]
MIT
\item[Current version:]
0.1.0
\item[Last updated:]
August 7, 2024
\item[First released:]
August 7, 2024
\item[Archive size:]
75.1 kB
\href{https://packages.typst.org/preview/blinky-0.1.0.tar.gz}{\pandocbounded{\includesvg[keepaspectratio]{/assets/icons/16-download.svg}}}
\item[Repository:]
\href{https://github.com/alexanderkoller/typst-blinky}{GitHub}
\end{description}

\subsubsection{Where to report issues?}\label{where-to-report-issues}

This package is a project of Alexander Koller . Report issues on
\href{https://github.com/alexanderkoller/typst-blinky}{their repository}
. You can also try to ask for help with this package on the
\href{https://forum.typst.app}{Forum} .

Please report this package to the Typst team using the
\href{https://typst.app/contact}{contact form} if you believe it is a
safety hazard or infringes upon your rights.

\phantomsection\label{versions}
\subsubsection{Version history}\label{version-history}

\begin{longtable}[]{@{}ll@{}}
\toprule\noalign{}
Version & Release Date \\
\midrule\noalign{}
\endhead
\bottomrule\noalign{}
\endlastfoot
0.1.0 & August 7, 2024 \\
\end{longtable}

Typst GmbH did not create this package and cannot guarantee correct
functionality of this package or compatibility with any version of the
Typst compiler or app.


\section{Package List LaTeX/nifty-ntnu-thesis.tex}
\title{typst.app/universe/package/nifty-ntnu-thesis}

\phantomsection\label{banner}
\phantomsection\label{template-thumbnail}
\pandocbounded{\includegraphics[keepaspectratio]{https://packages.typst.org/preview/thumbnails/nifty-ntnu-thesis-0.1.1-small.webp}}

\section{nifty-ntnu-thesis}\label{nifty-ntnu-thesis}

{ 0.1.1 }

An NTNU thesis template

\href{/app?template=nifty-ntnu-thesis&version=0.1.1}{Create project in
app}

\phantomsection\label{readme}
Port of \href{https://github.com/COPCSE-NTNU/thesis-NTNU}{thesis-NTNU}
template to Typst.
\href{https://github.com/saimnaveediqbal/thesis-NTNU-typst/blob/main/template/main.typ}{main.pdf}
contains a full usage example, see
\href{https://github.com/saimnaveediqbal/thesis-NTNU-typst/blob/main/example.pdf}{example.pdf}
for a rendered pdf.

To use this template you need to import it at the beginning of your
document:

\begin{Shaded}
\begin{Highlighting}[]
\NormalTok{\#import "@preview/nifty{-}ntnu{-}thesis:0.1.0": *}
\end{Highlighting}
\end{Shaded}

The template has many arguments you can specify:

\begin{longtable}[]{@{}llll@{}}
\toprule\noalign{}
Argument & Default Value & Type & Description \\
\midrule\noalign{}
\endhead
\bottomrule\noalign{}
\endlastfoot
\texttt{\ title\ } & \texttt{\ Title\ } & {[}content{]} & The title of
the thesis. \\
\texttt{\ short-title\ } & \texttt{\ Title\ } & {[}content{]} & Short
form of the title. If specified, will show up in the header \\
\texttt{\ author\ } & \texttt{\ Author\ } & {[}array{]} & An array of
authors \\
\texttt{\ short-author\ } & `` & {[}string{]} & Short form version of
the authors. If specified, will show up in header \\
\texttt{\ font\ } & \texttt{\ Charter\ } & {[}string{]} & Main font of
template \\
\texttt{\ raw-font\ } & \texttt{\ DejaVu\ Sans\ Mono\ } & {[}string{]} &
Font used for code listings \\
\texttt{\ paper-size\ } & \texttt{\ a4\ } & {[}string{]} & Specify a
{[}paper size string{]} to change the page size. \\
\texttt{\ date\ } & \texttt{\ datetime.today()\ } & {[}datetime{]} & The
date that will be displayed on the cover page. \\
\texttt{\ date-format\ } &
\texttt{\ {[}day\ padding:zero{]}/{[}month\ repr:numerical{]}/{[}year\ repr:full{]}\ }
& {[}string{]} & The format for the date that will be displayed on the
cover page. By default, the date will be displayed as
\texttt{\ DD/MM/YYYY\ } . \\
\texttt{\ abstract-en\ } & \texttt{\ none\ } & {[}content{]} & English
abstract shown before main content. \\
\texttt{\ abstract-no\ } & \texttt{\ none\ } & {[}content{]} & Norwegian
abstract shown before main content. \\
\texttt{\ preface\ } & \texttt{\ none\ } & {[}content{]} & The preface
for your work. The preface content is shown on its own separate page
after the abstracts. \\
\texttt{\ table-of-contents\ } & \texttt{\ outline()\ } & {[}content{]}
& The table of contents. Setting this to \texttt{\ none\ } will disable
the table of contents. \\
\texttt{\ titlepage\ } & \texttt{\ false\ } & {[}bool{]} & Whether to
display the titlepage or not. \\
\texttt{\ bibliography\ } & \texttt{\ none\ } & {[}content{]} & The
bibliography function or none. Specifying this will configure numeric,
IEEE-style citations. \\
\texttt{\ chapter-pagebreak\ } & \texttt{\ true\ } & {[}bool{]} &
Setting this to \texttt{\ false\ } will prevent chapters from starting
on a new page. \\
\texttt{\ chapters-on-odd\ } & \texttt{\ false\ } & {[}bool{]} & Setting
this to \texttt{\ false\ } will prevent chapters from only starting on
an odd page. \\
\texttt{\ figure-index\ } &
\texttt{\ (enabled:\ true,\ title:\ "Figures")\ } & {[}dictionary{]} &
Setting this to \texttt{\ true\ } will display a index of image figures
at the end of the document. \\
\texttt{\ table-index\ } &
\texttt{\ (enabled:\ true,\ title:\ "Tables")\ } & {[}dictionary{]} &
Setting this to \texttt{\ true\ } will display a index of table figures
at the end of the document. \\
\texttt{\ listing-index\ } &
\texttt{\ (enabled:\ true,\ title:\ "Listings")\ } & {[}dictionary{]} &
Setting this to \texttt{\ true\ } will display a index of listing (code
block) figures at the end of the document. \\
\end{longtable}

\begin{itemize}
\tightlist
\item
  {[} {]} Styling for:

  \begin{itemize}
  \tightlist
  \item
    {[}x{]} Code blocks
  \item
    {[}x{]} Tables
  \item
    {[}x{]} Header and footer
  \item
    {[}x{]} Lists
  \end{itemize}
\item
  {[}x{]} Subfigures
\item
  {[}x{]} Abstract
\item
  {[}x{]} Preface
\item
  {[}x{]} Code block captions
\item
  {[}x{]} Bibliography
\item
  {[} {]} Norwegian language support
\item
  {[}x{]} Proper figure numbering
\item
  {[}x{]} Short title in header
\item
  {[}x{]} Multiple authors
\item
  {[}x{]} Start chapters on only odd pages
\end{itemize}

Thanks to:

\begin{itemize}
\tightlist
\item
  The creator of the
  \href{https://github.com/talal/ilm/blob/main/lib.typ}{ILM template}
  which I used as the basis for this.
\item
  The creators of the original
  \href{https://github.com/COPCSE-NTNU/thesis-NTNU}{NTNU thesis
  template}
\item
  The creators of the
  \href{https://github.com/maucejo/elsearticle}{elsearticle template}
  for their implementation of subfigures and appendix environment
\end{itemize}

\href{/app?template=nifty-ntnu-thesis&version=0.1.1}{Create project in
app}

\subsubsection{How to use}\label{how-to-use}

Click the button above to create a new project using this template in
the Typst app.

You can also use the Typst CLI to start a new project on your computer
using this command:

\begin{verbatim}
typst init @preview/nifty-ntnu-thesis:0.1.1
\end{verbatim}

\includesvg[width=0.16667in,height=0.16667in]{/assets/icons/16-copy.svg}

\subsubsection{About}\label{about}

\begin{description}
\tightlist
\item[Author :]
Saim Iqbal
\item[License:]
MIT
\item[Current version:]
0.1.1
\item[Last updated:]
November 6, 2024
\item[First released:]
August 29, 2024
\item[Archive size:]
856 kB
\href{https://packages.typst.org/preview/nifty-ntnu-thesis-0.1.1.tar.gz}{\pandocbounded{\includesvg[keepaspectratio]{/assets/icons/16-download.svg}}}
\item[Repository:]
\href{https://github.com/saimnaveediqbal/thesis-NTNU-typst}{GitHub}
\item[Categor y :]
\begin{itemize}
\tightlist
\item[]
\item
  \pandocbounded{\includesvg[keepaspectratio]{/assets/icons/16-mortarboard.svg}}
  \href{https://typst.app/universe/search/?category=thesis}{Thesis}
\end{itemize}
\end{description}

\subsubsection{Where to report issues?}\label{where-to-report-issues}

This template is a project of Saim Iqbal . Report issues on
\href{https://github.com/saimnaveediqbal/thesis-NTNU-typst}{their
repository} . You can also try to ask for help with this template on the
\href{https://forum.typst.app}{Forum} .

Please report this template to the Typst team using the
\href{https://typst.app/contact}{contact form} if you believe it is a
safety hazard or infringes upon your rights.

\phantomsection\label{versions}
\subsubsection{Version history}\label{version-history}

\begin{longtable}[]{@{}ll@{}}
\toprule\noalign{}
Version & Release Date \\
\midrule\noalign{}
\endhead
\bottomrule\noalign{}
\endlastfoot
0.1.1 & November 6, 2024 \\
\href{https://typst.app/universe/package/nifty-ntnu-thesis/0.1.0/}{0.1.0}
& August 29, 2024 \\
\end{longtable}

Typst GmbH did not create this template and cannot guarantee correct
functionality of this template or compatibility with any version of the
Typst compiler or app.


\section{Package List LaTeX/badgery.tex}
\title{typst.app/universe/package/badgery}

\phantomsection\label{banner}
\section{badgery}\label{badgery}

{ 0.1.1 }

Adds styled badges, boxes and menu actions.

\phantomsection\label{readme}
This package defines some colourful badges and boxes around text that
represent user interface actions such as a click or following a menu.

For examples have a look at the example
\href{https://github.com/typst/packages/raw/main/packages/preview/badgery/0.1.1/example/main.typ}{main.typ}
,
\href{https://github.com/typst/packages/raw/main/packages/preview/badgery/0.1.1/exmaple/main.pdf}{main.pdf}
.

\pandocbounded{\includegraphics[keepaspectratio]{https://github.com/typst/packages/raw/main/packages/preview/badgery/0.1.1/demo.png}}

\subsection{Badges}\label{badges}

\begin{Shaded}
\begin{Highlighting}[]
\NormalTok{\#badge{-}gray("Gray badge"),}
\NormalTok{\#badge{-}red("Red badge"),}
\NormalTok{\#badge{-}yellow("Yellow badge"),}
\NormalTok{\#badge{-}green("Green badge"),}
\NormalTok{\#badge{-}blue("Blue badge"),}
\NormalTok{\#badge{-}purple("Purple badge")}
\end{Highlighting}
\end{Shaded}

\subsection{User interface actions}\label{user-interface-actions}

This is a user interface action (ie. a click):

\begin{Shaded}
\begin{Highlighting}[]
\NormalTok{\#ui{-}action("Click X")}
\end{Highlighting}
\end{Shaded}

This is an action to follow a user interface menu (2 steps):

\begin{Shaded}
\begin{Highlighting}[]
\NormalTok{\#menu(("File", "New File..."))}
\end{Highlighting}
\end{Shaded}

This is a menu action with multiple steps:

\begin{Shaded}
\begin{Highlighting}[]
\NormalTok{\#menu(("Menu", "Sub{-}menu", "Sub{-}sub menu", "Action"))}
\end{Highlighting}
\end{Shaded}

\subsubsection{How to add}\label{how-to-add}

Copy this into your project and use the import as \texttt{\ badgery\ }

\begin{verbatim}
#import "@preview/badgery:0.1.1"
\end{verbatim}

\includesvg[width=0.16667in,height=0.16667in]{/assets/icons/16-copy.svg}

Check the docs for
\href{https://typst.app/docs/reference/scripting/\#packages}{more
information on how to import packages} .

\subsubsection{About}\label{about}

\begin{description}
\tightlist
\item[Author :]
dogezen
\item[License:]
MIT
\item[Current version:]
0.1.1
\item[Last updated:]
March 20, 2024
\item[First released:]
March 19, 2024
\item[Minimum Typst version:]
0.11.0
\item[Archive size:]
2.50 kB
\href{https://packages.typst.org/preview/badgery-0.1.1.tar.gz}{\pandocbounded{\includesvg[keepaspectratio]{/assets/icons/16-download.svg}}}
\item[Repository:]
\href{https://github.com/dogezen/badgery}{GitHub}
\item[Discipline s :]
\begin{itemize}
\tightlist
\item[]
\item
  \href{https://typst.app/universe/search/?discipline=computer-science}{Computer
  Science}
\item
  \href{https://typst.app/universe/search/?discipline=engineering}{Engineering}
\item
  \href{https://typst.app/universe/search/?discipline=business}{Business}
\item
  \href{https://typst.app/universe/search/?discipline=communication}{Communication}
\end{itemize}
\item[Categor y :]
\begin{itemize}
\tightlist
\item[]
\item
  \pandocbounded{\includesvg[keepaspectratio]{/assets/icons/16-package.svg}}
  \href{https://typst.app/universe/search/?category=components}{Components}
\end{itemize}
\end{description}

\subsubsection{Where to report issues?}\label{where-to-report-issues}

This package is a project of dogezen . Report issues on
\href{https://github.com/dogezen/badgery}{their repository} . You can
also try to ask for help with this package on the
\href{https://forum.typst.app}{Forum} .

Please report this package to the Typst team using the
\href{https://typst.app/contact}{contact form} if you believe it is a
safety hazard or infringes upon your rights.

\phantomsection\label{versions}
\subsubsection{Version history}\label{version-history}

\begin{longtable}[]{@{}ll@{}}
\toprule\noalign{}
Version & Release Date \\
\midrule\noalign{}
\endhead
\bottomrule\noalign{}
\endlastfoot
0.1.1 & March 20, 2024 \\
\href{https://typst.app/universe/package/badgery/0.1.0/}{0.1.0} & March
19, 2024 \\
\end{longtable}

Typst GmbH did not create this package and cannot guarantee correct
functionality of this package or compatibility with any version of the
Typst compiler or app.


\section{Package List LaTeX/haw-hamburg-bachelor-thesis.tex}
\title{typst.app/universe/package/haw-hamburg-bachelor-thesis}

\phantomsection\label{banner}
\phantomsection\label{template-thumbnail}
\pandocbounded{\includegraphics[keepaspectratio]{https://packages.typst.org/preview/thumbnails/haw-hamburg-bachelor-thesis-0.3.1-small.webp}}

\section{haw-hamburg-bachelor-thesis}\label{haw-hamburg-bachelor-thesis}

{ 0.3.1 }

Unofficial template for writing a bachelor-thesis in the HAW Hamburg
department of Computer Science design.

\href{/app?template=haw-hamburg-bachelor-thesis&version=0.3.1}{Create
project in app}

\phantomsection\label{readme}
This is an \textbf{\texttt{\ unofficial\ }} template for writing a
bachelor thesis in the \texttt{\ HAW\ Hamburg\ } department of
\texttt{\ Computer\ Science\ } design using
\href{https://github.com/typst/typst}{Typst} .

\subsection{Required Fonts}\label{required-fonts}

To correctly render this template please make sure that the
\texttt{\ New\ Computer\ Modern\ } font is installed on your system.

\subsection{Usage}\label{usage}

To use this package just add the following code to your
\href{https://github.com/typst/typst}{Typst} document:

\begin{Shaded}
\begin{Highlighting}[]
\NormalTok{\#import "@preview/haw{-}hamburg:0.3.1": bachelor{-}thesis}

\NormalTok{\#show: bachelor{-}thesis.with(}
\NormalTok{  language: "en",}

\NormalTok{  title{-}de: "Beispiel Titel",}
\NormalTok{  keywords{-}de: ("Stichwort", "Wichtig", "Super"),}
\NormalTok{  abstract{-}de: "Beispiel Zusammenfassung",}

\NormalTok{  title{-}en: "Example title",}
\NormalTok{  keywords{-}en:  ("Keyword", "Important", "Super"),}
\NormalTok{  abstract{-}en: "Example abstract",}

\NormalTok{  author: "Example author",}
\NormalTok{  faculty: "Engineering and Computer Science",}
\NormalTok{  department: "Computer Science",}
\NormalTok{  study{-}course: "Bachelor of Science Informatik Technischer Systeme",}
\NormalTok{  supervisors: ("Prof. Dr. Example", "Prof. Dr. Example"),}
\NormalTok{  submission{-}date: datetime(year: 1948, month: 12, day: 10),}
\NormalTok{  include{-}declaration{-}of{-}independent{-}processing: true,}
\NormalTok{)}
\end{Highlighting}
\end{Shaded}

\subsection{How to Compile}\label{how-to-compile}

This project contains an example setup that splits individual chapters
into different files.\\
This can cause problems when using references etc.\\
These problems can be avoided by following these steps:

\begin{itemize}
\tightlist
\item
  Make sure to always compile your \texttt{\ main.typ\ } file which
  includes all of your chapters for references to work correctly.
\item
  VSCode:

  \begin{itemize}
  \tightlist
  \item
    Install the
    \href{https://marketplace.visualstudio.com/items?itemName=myriad-dreamin.tinymist}{Tinymist
    Typst} extension.
  \item
    Make sure to start the \texttt{\ PDF\ } or
    \texttt{\ Live\ Preview\ } only from within your
    \texttt{\ main.typ\ } file.
  \item
    If problems occur it usually helps to close the preview and restart
    it from your \texttt{\ main.typ\ } file.
  \end{itemize}
\end{itemize}

\href{/app?template=haw-hamburg-bachelor-thesis&version=0.3.1}{Create
project in app}

\subsubsection{How to use}\label{how-to-use}

Click the button above to create a new project using this template in
the Typst app.

You can also use the Typst CLI to start a new project on your computer
using this command:

\begin{verbatim}
typst init @preview/haw-hamburg-bachelor-thesis:0.3.1
\end{verbatim}

\includesvg[width=0.16667in,height=0.16667in]{/assets/icons/16-copy.svg}

\subsubsection{About}\label{about}

\begin{description}
\tightlist
\item[Author :]
Lasse Rosenow
\item[License:]
MIT
\item[Current version:]
0.3.1
\item[Last updated:]
November 13, 2024
\item[First released:]
October 14, 2024
\item[Archive size:]
6.57 kB
\href{https://packages.typst.org/preview/haw-hamburg-bachelor-thesis-0.3.1.tar.gz}{\pandocbounded{\includesvg[keepaspectratio]{/assets/icons/16-download.svg}}}
\item[Repository:]
\href{https://github.com/LasseRosenow/HAW-Hamburg-Typst-Template}{GitHub}
\item[Categor y :]
\begin{itemize}
\tightlist
\item[]
\item
  \pandocbounded{\includesvg[keepaspectratio]{/assets/icons/16-mortarboard.svg}}
  \href{https://typst.app/universe/search/?category=thesis}{Thesis}
\end{itemize}
\end{description}

\subsubsection{Where to report issues?}\label{where-to-report-issues}

This template is a project of Lasse Rosenow . Report issues on
\href{https://github.com/LasseRosenow/HAW-Hamburg-Typst-Template}{their
repository} . You can also try to ask for help with this template on the
\href{https://forum.typst.app}{Forum} .

Please report this template to the Typst team using the
\href{https://typst.app/contact}{contact form} if you believe it is a
safety hazard or infringes upon your rights.

\phantomsection\label{versions}
\subsubsection{Version history}\label{version-history}

\begin{longtable}[]{@{}ll@{}}
\toprule\noalign{}
Version & Release Date \\
\midrule\noalign{}
\endhead
\bottomrule\noalign{}
\endlastfoot
0.3.1 & November 13, 2024 \\
\href{https://typst.app/universe/package/haw-hamburg-bachelor-thesis/0.3.0/}{0.3.0}
& October 14, 2024 \\
\end{longtable}

Typst GmbH did not create this template and cannot guarantee correct
functionality of this template or compatibility with any version of the
Typst compiler or app.


\section{Package List LaTeX/modpattern.tex}
\title{typst.app/universe/package/modpattern}

\phantomsection\label{banner}
\section{modpattern}\label{modpattern}

{ 0.1.0 }

Easily create patterns in typst.

\phantomsection\label{readme}
This package provides exactly one function: \texttt{\ modpattern\ }

It’s primary goal is to create make this:
\pandocbounded{\includegraphics[keepaspectratio]{https://github.com/typst/packages/raw/main/packages/preview/modpattern/0.1.0/examples/comparison.png}}

\subsection{modpattern function}\label{modpattern-function}

The function with the signature
\texttt{\ modpattern(size,\ body,\ dx:\ 0pt,\ dy:\ 0pt,\ background:\ none)\ }
has the following parameters:

\begin{itemize}
\tightlist
\item
  \texttt{\ size\ } is as size for patterns
\item
  \texttt{\ body\ } is the inside/body/content of the pattern
\item
  \texttt{\ dx\ } , \texttt{\ dy\ } allow for translations
\item
  \texttt{\ background\ } allows any type allowed in the box fill
  argument. It gets applied first
\end{itemize}

Notice that, in contrast to typst patterns, size is a positional
argument.

Take a look at the example directory, to understand how to use this and
to see the reasoning behind the \texttt{\ background\ } argument.

\subsubsection{How to add}\label{how-to-add}

Copy this into your project and use the import as
\texttt{\ modpattern\ }

\begin{verbatim}
#import "@preview/modpattern:0.1.0"
\end{verbatim}

\includesvg[width=0.16667in,height=0.16667in]{/assets/icons/16-copy.svg}

Check the docs for
\href{https://typst.app/docs/reference/scripting/\#packages}{more
information on how to import packages} .

\subsubsection{About}\label{about}

\begin{description}
\tightlist
\item[Author :]
Ludwig Austermann
\item[License:]
MIT
\item[Current version:]
0.1.0
\item[Last updated:]
April 12, 2024
\item[First released:]
April 12, 2024
\item[Minimum Typst version:]
0.11.0
\item[Archive size:]
1.62 kB
\href{https://packages.typst.org/preview/modpattern-0.1.0.tar.gz}{\pandocbounded{\includesvg[keepaspectratio]{/assets/icons/16-download.svg}}}
\item[Repository:]
\href{https://github.com/ludwig-austermann/modpattern}{GitHub}
\end{description}

\subsubsection{Where to report issues?}\label{where-to-report-issues}

This package is a project of Ludwig Austermann . Report issues on
\href{https://github.com/ludwig-austermann/modpattern}{their repository}
. You can also try to ask for help with this package on the
\href{https://forum.typst.app}{Forum} .

Please report this package to the Typst team using the
\href{https://typst.app/contact}{contact form} if you believe it is a
safety hazard or infringes upon your rights.

\phantomsection\label{versions}
\subsubsection{Version history}\label{version-history}

\begin{longtable}[]{@{}ll@{}}
\toprule\noalign{}
Version & Release Date \\
\midrule\noalign{}
\endhead
\bottomrule\noalign{}
\endlastfoot
0.1.0 & April 12, 2024 \\
\end{longtable}

Typst GmbH did not create this package and cannot guarantee correct
functionality of this package or compatibility with any version of the
Typst compiler or app.


\section{Package List LaTeX/starter-journal-article.tex}
\title{typst.app/universe/package/starter-journal-article}

\phantomsection\label{banner}
\phantomsection\label{template-thumbnail}
\pandocbounded{\includegraphics[keepaspectratio]{https://packages.typst.org/preview/thumbnails/starter-journal-article-0.3.1-small.webp}}

\section{starter-journal-article}\label{starter-journal-article}

{ 0.3.1 }

A starter template for journal articles.

\href{/app?template=starter-journal-article&version=0.3.1}{Create
project in app}

\phantomsection\label{readme}
This package provides a template for writing journal articles to
organise authors, institutions, and information of corresponding
authors.

\subsection{Usage}\label{usage}

Run the following command to use this template

\begin{Shaded}
\begin{Highlighting}[]
\NormalTok{typst init @preview/starter{-}journal{-}article}
\end{Highlighting}
\end{Shaded}

\subsection{Documentation}\label{documentation}

\subsubsection{\texorpdfstring{\texttt{\ article\ }}{ article }}\label{article}

The template for creating journal articles. It needs the following
arguments.

Arguments:

\begin{itemize}
\tightlist
\item
  \texttt{\ title\ } : The title of this article. Default:
  \texttt{\ "Article\ Title"\ } .
\item
  \texttt{\ authors\ } : A dictionary of authors. Dictionary keys are
  authors’ names. Dictionary values are meta data of every author,
  including label(s) of affiliation(s), email, contact address, or a
  self-defined name (to avoid name conflicts). The label(s) of
  affiliation(s) must be those claimed in the argument
  \texttt{\ affiliations\ } . Once the email or address exists, the
  author(s) will be labelled as the corresponding author(s), and their
  address will show in footnotes. Function \texttt{\ author-meta()\ } is
  useful in creating information for each author. Default:
  \texttt{\ ("Author\ Name":\ author-meta("affiliation-label"))\ } .
\item
  \texttt{\ affiliations\ } : A dictionary of affiliation. Dictionary
  keys are affiliations’ labels. These labels show be constent with
  those used in authors’ meta data. Dictionary values are addresses of
  every affiliation. Default:
  \texttt{\ ("affiliation-label":\ "Affiliation\ address")\ } .
\item
  \texttt{\ abstract\ } : The paper’s abstract. Default:
  \texttt{\ {[}{]}\ } .
\item
  \texttt{\ keywords\ } : The paper’s keywords. Default:
  \texttt{\ {[}{]}\ } .
\item
  \texttt{\ bib\ } : The bibliography. Accept value from the built-in
  \texttt{\ bibliography\ } function. Default: \texttt{\ none\ } .
\item
  \texttt{\ template\ } : Templates for the following parts. Please see
  below for more informations

  \begin{itemize}
  \tightlist
  \item
    \texttt{\ title:\ (title)\ =\textgreater{}\ \{\}\ } : how to show
    the title of this article.
  \item
    \texttt{\ author-info:\ (authors,\ affiliations)\ =\textgreater{}\ \{\}\ }
    : how to show each author’s information.
  \item
    \texttt{\ abstract:\ (abstract,\ keywords)\ =\textgreater{}\ \{\}\ }
    : how to show the abstract and keywords.
  \item
    \texttt{\ bibliography:\ (bib)\ =\textgreater{}\ \{\}\ } : how to
    show the bibliography.
  \item
    \texttt{\ body:\ (body)\ =\textgreater{}\ \{\}\ } : how to show main
    text.
  \end{itemize}
\end{itemize}

\subsubsection{\texorpdfstring{\texttt{\ author-meta\ }}{ author-meta }}\label{author-meta}

A helper to create meta information for an author.

Arguments:

\begin{itemize}
\tightlist
\item
  \texttt{\ ..affiliation\ } : Capture the positioned arguments as
  label(s) of affiliation(s). Mandatory.
\item
  \texttt{\ email\ } : The email address of the author. Default:
  \texttt{\ none\ } .
\item
  \texttt{\ alias\ } : The display name of the author. Default:
  \texttt{\ none\ } .
\item
  \texttt{\ address\ } : The address of the author. Default:
  \texttt{\ none\ } .
\item
  \texttt{\ cofirst\ } : Whether the author is the co-first author.
  Default: \texttt{\ false\ } .
\end{itemize}

\subsection{Default templates}\label{default-templates}

The following code shows how the default templates are defined. You can
override any of the by setting the \texttt{\ template\ } argument in the
\texttt{\ article()\ } function to customise the template.

\begin{Shaded}
\begin{Highlighting}[]
\NormalTok{\#let default{-}title(title) = \{}
\NormalTok{  show: block.with(width: 100\%)}
\NormalTok{  set align(center)}
\NormalTok{  set text(size: 1.75em, weight: "bold")}
\NormalTok{  title}
\NormalTok{\}}

\NormalTok{\#let default{-}author(author) = \{}
\NormalTok{  text(author.name)}
\NormalTok{  super(author.insts.map(it =\textgreater{} str(it)).join(","))}
\NormalTok{  if author.corresponding \{}
\NormalTok{    footnote[}
\NormalTok{      Corresponding author. Address: \#author.address.}
\NormalTok{      \#if author.email != none \{}
\NormalTok{        [Email: \#underline(author.email).]}
\NormalTok{      \}}
\NormalTok{    ]}
\NormalTok{  \}}
\NormalTok{  if author.cofirst == "thefirst" \{}
\NormalTok{    footnote("cofirst{-}author{-}mark")}
\NormalTok{  \} else if author.cofirst == "cofirst" \{}
\NormalTok{    locate(loc =\textgreater{} query(footnote.where(body: [cofirst{-}author{-}mark]), loc).last())}
\NormalTok{  \}}
\NormalTok{\}}

\NormalTok{\#let default{-}affiliation(id, address) = \{}
\NormalTok{  set text(size: 0.8em)}
\NormalTok{  super([\#(id+1)])}
\NormalTok{  address}
\NormalTok{\}}

\NormalTok{\#let default{-}author{-}info(authors, affiliations) = \{}
\NormalTok{  \{}
\NormalTok{    show: block.with(width: 100\%)}
\NormalTok{    authors.map(it =\textgreater{} default{-}author(it)).join(", ")}
\NormalTok{  \}}
\NormalTok{  \{}
\NormalTok{    show: block.with(width: 100\%)}
\NormalTok{    set par(leading: 0.4em)}
\NormalTok{    affiliations.keys().enumerate().map(((ik, key)) =\textgreater{} \{}
\NormalTok{      default{-}affiliation(ik, affiliations.at(key))}
\NormalTok{    \}).join(linebreak())}
\NormalTok{  \}}
\NormalTok{\}}

\NormalTok{\#let default{-}abstract(abstract, keywords) = \{}
\NormalTok{  // Abstract and keyword block}
\NormalTok{  if abstract != [] \{}
\NormalTok{    stack(}
\NormalTok{      dir: ttb,}
\NormalTok{      spacing: 1em,}
\NormalTok{      ..([}
\NormalTok{        \#heading([Abstract])}
\NormalTok{        \#abstract}
\NormalTok{      ], if keywords.len() \textgreater{} 0 \{}
\NormalTok{        text(weight: "bold", [Key words: ])}
\NormalTok{        text([\#keywords.join([; ]).])}
\NormalTok{      \} else \{none\} )}
\NormalTok{    )}
\NormalTok{  \}}
\NormalTok{\}}

\NormalTok{\#let default{-}bibliography(bib) = \{}
\NormalTok{  show bibliography: set text(1em)}
\NormalTok{  show bibliography: set par(first{-}line{-}indent: 0em)}
\NormalTok{  set bibliography(title: [References], style: "apa")}
\NormalTok{  bib}
\NormalTok{\}}

\NormalTok{\#let default{-}body(body) = \{}
\NormalTok{  show heading.where(level: 1): it =\textgreater{} block(above: 1.5em, below: 1.5em)[}
\NormalTok{    \#set pad(bottom: 2em, top: 1em)}
\NormalTok{    \#it.body}
\NormalTok{  ]}
\NormalTok{  set par(first{-}line{-}indent: 2em)}
\NormalTok{  set footnote(numbering: "1")}
\NormalTok{  body}
\NormalTok{\}}
\end{Highlighting}
\end{Shaded}

\subsection{Example}\label{example}

See
\href{https://github.com/typst/packages/raw/main/packages/preview/starter-journal-article/0.3.1/template/main.typ}{the
template} for full example.

\begin{Shaded}
\begin{Highlighting}[]
\NormalTok{\#show: article.with(}
\NormalTok{  title: "Artile Title",}
\NormalTok{  authors: (}
\NormalTok{    "Author One": author{-}meta(}
\NormalTok{      "UCL", "TSU",}
\NormalTok{      email: "author.one@inst.ac.uk",}
\NormalTok{    ),}
\NormalTok{    "Author Two": author{-}meta(}
\NormalTok{      "TSU",}
\NormalTok{      cofirst: true}
\NormalTok{    ),}
\NormalTok{    "Author Three": author{-}meta(}
\NormalTok{      "TSU"}
\NormalTok{    )}
\NormalTok{  ),}
\NormalTok{  affiliations: (}
\NormalTok{    "UCL": "UCL Centre for Advanced Spatial Analysis, First Floor, 90 Tottenham Court Road, London W1T 4TJ, United Kingdom",}
\NormalTok{    "TSU": "Haidian  District, Beijing, 100084, P. R. China"}
\NormalTok{  ),}
\NormalTok{  abstract: [\#lorem(100)],}
\NormalTok{  keywords: ("Typst", "Template", "Journal Article"),}
\NormalTok{  bib: bibliography("./ref.bib")}
\NormalTok{)}
\end{Highlighting}
\end{Shaded}

\pandocbounded{\includegraphics[keepaspectratio]{https://github.com/typst/packages/raw/main/packages/preview/starter-journal-article/0.3.1/assets/basic.png}}

\subsubsection{Custom title}\label{custom-title}

\begin{Shaded}
\begin{Highlighting}[]
\NormalTok{\#show: article.with(}
\NormalTok{  title: "Artile Title",}
\NormalTok{  authors: (}
\NormalTok{    "Author One": author{-}meta(}
\NormalTok{      "UCL", "TSU",}
\NormalTok{      email: "author.one@inst.ac.uk",}
\NormalTok{    ),}
\NormalTok{    "Author Two": author{-}meta(}
\NormalTok{      "TSU",}
\NormalTok{      cofirst: true}
\NormalTok{    ),}
\NormalTok{    "Author Three": author{-}meta(}
\NormalTok{      "TSU"}
\NormalTok{    )}
\NormalTok{  ),}
\NormalTok{  affiliations: (}
\NormalTok{    "UCL": "UCL Centre for Advanced Spatial Analysis, First Floor, 90 Tottenham Court Road, London W1T 4TJ, United Kingdom",}
\NormalTok{    "TSU": "Haidian  District, Beijing, 100084, P. R. China"}
\NormalTok{  ),}
\NormalTok{  abstract: [\#lorem(100)],}
\NormalTok{  keywords: ("Typst", "Template", "Journal Article"),}
\NormalTok{  bib: bibliography("./ref.bib"),}
\NormalTok{  template: (}
\NormalTok{    title: (title) =\textgreater{} \{}
\NormalTok{      set align(left)}
\NormalTok{      set text(size: 1.5em, weight: "bold", style: "italic")}
\NormalTok{      title}
\NormalTok{    \}}
\NormalTok{  )}
\NormalTok{)}
\end{Highlighting}
\end{Shaded}

\pandocbounded{\includegraphics[keepaspectratio]{https://github.com/typst/packages/raw/main/packages/preview/starter-journal-article/0.3.1/assets/custom-title.png}}

\subsubsection{Custom author infomation}\label{custom-author-infomation}

\begin{Shaded}
\begin{Highlighting}[]
\NormalTok{\#show: article.with(}
\NormalTok{  title: "Artile Title",}
\NormalTok{  authors: (}
\NormalTok{    "Author One": author{-}meta("UCL", email: "author.one@inst.ac.uk"),}
\NormalTok{    "Author Two": author{-}meta("TSU")}
\NormalTok{  ),}
\NormalTok{  affiliations: (}
\NormalTok{    "UCL": "UCL Centre for Advanced Spatial Analysis, First Floor, 90 Tottenham Court Road, London W1T 4TJ, United Kingdom",}
\NormalTok{    "TSU": "Haidian  District, Beijing, 100084, P. R. China"}
\NormalTok{  ),}
\NormalTok{  abstract: [\#lorem(20)],}
\NormalTok{  keywords: ("Typst", "Template", "Journal Article"),}
\NormalTok{  template: (}
\NormalTok{    author{-}info: (authors, affiliations) =\textgreater{} \{}
\NormalTok{      set align(center)}
\NormalTok{      show: block.with(width: 100\%, above: 2em, below: 2em)}
\NormalTok{      let first\_insts = authors.map(it =\textgreater{} it.insts.at(0)).dedup()}
\NormalTok{      stack(}
\NormalTok{        dir: ttb,}
\NormalTok{        spacing: 1em,}
\NormalTok{        ..first\_insts.map(inst\_id =\textgreater{} \{}
\NormalTok{          let inst\_authors = authors.filter(it =\textgreater{} it.insts.at(0) == inst\_id)}
\NormalTok{          stack(}
\NormalTok{            dir: ttb,}
\NormalTok{            spacing: 1em,}
\NormalTok{            \{}
\NormalTok{              inst\_authors.map(it =\textgreater{} it.name).join(", ")}
\NormalTok{            \},}
\NormalTok{            \{}
\NormalTok{              set text(0.8em, style: "italic")}
\NormalTok{              affiliations.values().at(inst\_id)}
\NormalTok{            \}}
\NormalTok{          )}
\NormalTok{        \})}
\NormalTok{      )}
\NormalTok{    \}}
\NormalTok{  )}
\NormalTok{)}
\end{Highlighting}
\end{Shaded}

\pandocbounded{\includegraphics[keepaspectratio]{https://github.com/typst/packages/raw/main/packages/preview/starter-journal-article/0.3.1/assets/custom-author-info.png}}

\subsubsection{Custom abstract}\label{custom-abstract}

\begin{Shaded}
\begin{Highlighting}[]
\NormalTok{\#show: article.with(}
\NormalTok{  title: "Artile Title",}
\NormalTok{  authors: (}
\NormalTok{    "Author One": author{-}meta("UCL", email: "author.one@inst.ac.uk"),}
\NormalTok{    "Author Two": author{-}meta("TSU")}
\NormalTok{  ),}
\NormalTok{  affiliations: (}
\NormalTok{    "UCL": "UCL Centre for Advanced Spatial Analysis, First Floor, 90 Tottenham Court Road, London W1T 4TJ, United Kingdom",}
\NormalTok{    "TSU": "Haidian  District, Beijing, 100084, P. R. China"}
\NormalTok{  ),}
\NormalTok{  abstract: [\#lorem(20)],}
\NormalTok{  keywords: ("Typst", "Template", "Journal Article"),}
\NormalTok{  template: (}
\NormalTok{    abstract: (abstract, keywords) =\textgreater{} \{}
\NormalTok{      show: block.with(}
\NormalTok{        width: 100\%,}
\NormalTok{        stroke: (y: 0.5pt + black),}
\NormalTok{        inset: (y: 1em)}
\NormalTok{      )}
\NormalTok{      show heading: set text(size: 12pt)}
\NormalTok{      heading(numbering: none, outlined: false, bookmarked: false, [Abstract])}
\NormalTok{      par(abstract)}
\NormalTok{      stack(}
\NormalTok{        dir: ltr,}
\NormalTok{        spacing: 4pt,}
\NormalTok{        strong([Keywords]),}
\NormalTok{        keywords.join(", ")}
\NormalTok{      )}
\NormalTok{    \}}
\NormalTok{  )}
\NormalTok{)}
\end{Highlighting}
\end{Shaded}

\pandocbounded{\includegraphics[keepaspectratio]{https://github.com/typst/packages/raw/main/packages/preview/starter-journal-article/0.3.1/assets/custom-abstract.png}}

\href{/app?template=starter-journal-article&version=0.3.1}{Create
project in app}

\subsubsection{How to use}\label{how-to-use}

Click the button above to create a new project using this template in
the Typst app.

You can also use the Typst CLI to start a new project on your computer
using this command:

\begin{verbatim}
typst init @preview/starter-journal-article:0.3.1
\end{verbatim}

\includesvg[width=0.16667in,height=0.16667in]{/assets/icons/16-copy.svg}

\subsubsection{About}\label{about}

\begin{description}
\tightlist
\item[Author :]
\href{https://github.com/HPDell}{HPDell}
\item[License:]
MIT
\item[Current version:]
0.3.1
\item[Last updated:]
August 19, 2024
\item[First released:]
March 26, 2024
\item[Archive size:]
5.29 kB
\href{https://packages.typst.org/preview/starter-journal-article-0.3.1.tar.gz}{\pandocbounded{\includesvg[keepaspectratio]{/assets/icons/16-download.svg}}}
\item[Repository:]
\href{https://github.com/HPDell/typst-starter-journal-article}{GitHub}
\item[Categor y :]
\begin{itemize}
\tightlist
\item[]
\item
  \pandocbounded{\includesvg[keepaspectratio]{/assets/icons/16-atom.svg}}
  \href{https://typst.app/universe/search/?category=paper}{Paper}
\end{itemize}
\end{description}

\subsubsection{Where to report issues?}\label{where-to-report-issues}

This template is a project of HPDell . Report issues on
\href{https://github.com/HPDell/typst-starter-journal-article}{their
repository} . You can also try to ask for help with this template on the
\href{https://forum.typst.app}{Forum} .

Please report this template to the Typst team using the
\href{https://typst.app/contact}{contact form} if you believe it is a
safety hazard or infringes upon your rights.

\phantomsection\label{versions}
\subsubsection{Version history}\label{version-history}

\begin{longtable}[]{@{}ll@{}}
\toprule\noalign{}
Version & Release Date \\
\midrule\noalign{}
\endhead
\bottomrule\noalign{}
\endlastfoot
0.3.1 & August 19, 2024 \\
\href{https://typst.app/universe/package/starter-journal-article/0.3.0/}{0.3.0}
& April 8, 2024 \\
\href{https://typst.app/universe/package/starter-journal-article/0.2.0/}{0.2.0}
& April 2, 2024 \\
\href{https://typst.app/universe/package/starter-journal-article/0.1.1/}{0.1.1}
& March 26, 2024 \\
\end{longtable}

Typst GmbH did not create this template and cannot guarantee correct
functionality of this template or compatibility with any version of the
Typst compiler or app.


\section{Package List LaTeX/outline-summaryst.tex}
\title{typst.app/universe/package/outline-summaryst}

\phantomsection\label{banner}
\section{outline-summaryst}\label{outline-summaryst}

{ 0.1.0 }

A basic template for including a summary for each entry in the table of
contents. Useful for writing books.

\phantomsection\label{readme}
\subsection{Description}\label{description}

\texttt{\ outline-summaryst\ } is a basic package designed for including
a summary for each entry in the table of contents, particularly useful
for writing books. It provides a simple structure for organizing content
and generating formatted documents with summary sections.

\subsection{Features}\label{features}

\begin{itemize}
\tightlist
\item
  A template for the outline, which styles both the heading and their
  summaries.
\item
  A macro for creating new headings and a summary for each heading.
\end{itemize}

\subsection{Note:}\label{note}

Because of the way the project is implemented, only the headings created
with the provided \texttt{\ make-heading("heading\ name",\ "summary")\ }
are shown in in the outline. Headings created with the default
\texttt{\ =\ Heading\ } syntax will not show in said outline (though
they will show up in the document itself).

\subsection{Example Usage:}\label{example-usage}

\begin{verbatim}
#import "@preview/outline-summaryst:0.1.0": style-outline, make-heading


// you can set `outline-title: none` in order not to display any title
#show outline: style-outline.with(outline-title: "Table of Contents")

#outline()


#make-heading("Part One", "This is the summary for part one")
#lorem(500)

#make-heading("Chapter One", "Summary for chapter one in part one", level: 2)
#lorem(300)

#make-heading("Chapter Two", "This is the summary for chapter two in part one", level: 2)
#lorem(300)

#make-heading("Part Two", "And here we have the summary for part two")
#lorem(500)

#make-heading("Chapter One", "Summary for chapter one in part two", level: 2)
#lorem(300)

#make-heading("Chapter Two", "Summary for chapter two in part two", level: 2)
#lorem(300)
\end{verbatim}

\subsection{Known limitations}\label{known-limitations}

\begin{itemize}
\tightlist
\item
  Currently, we do not provide a way for styling the table of contents
  or headings
\end{itemize}

\subsection{License:}\label{license}

This project is licensed under the MIT License. See the LICENSE file for
details.

\subsection{Contribution:}\label{contribution}

Contributions are welcome! Feel free to open an issue or submit a pull
request on GitHub.

\subsubsection{How to add}\label{how-to-add}

Copy this into your project and use the import as
\texttt{\ outline-summaryst\ }

\begin{verbatim}
#import "@preview/outline-summaryst:0.1.0"
\end{verbatim}

\includesvg[width=0.16667in,height=0.16667in]{/assets/icons/16-copy.svg}

Check the docs for
\href{https://typst.app/docs/reference/scripting/\#packages}{more
information on how to import packages} .

\subsubsection{About}\label{about}

\begin{description}
\tightlist
\item[Author :]
@aarneng
\item[License:]
MIT
\item[Current version:]
0.1.0
\item[Last updated:]
April 15, 2024
\item[First released:]
April 15, 2024
\item[Archive size:]
2.97 kB
\href{https://packages.typst.org/preview/outline-summaryst-0.1.0.tar.gz}{\pandocbounded{\includesvg[keepaspectratio]{/assets/icons/16-download.svg}}}
\item[Repository:]
\href{https://github.com/aarneng/Outline-Summary}{GitHub}
\item[Discipline :]
\begin{itemize}
\tightlist
\item[]
\item
  \href{https://typst.app/universe/search/?discipline=literature}{Literature}
\end{itemize}
\item[Categor y :]
\begin{itemize}
\tightlist
\item[]
\item
  \pandocbounded{\includesvg[keepaspectratio]{/assets/icons/16-layout.svg}}
  \href{https://typst.app/universe/search/?category=layout}{Layout}
\end{itemize}
\end{description}

\subsubsection{Where to report issues?}\label{where-to-report-issues}

This package is a project of @aarneng . Report issues on
\href{https://github.com/aarneng/Outline-Summary}{their repository} .
You can also try to ask for help with this package on the
\href{https://forum.typst.app}{Forum} .

Please report this package to the Typst team using the
\href{https://typst.app/contact}{contact form} if you believe it is a
safety hazard or infringes upon your rights.

\phantomsection\label{versions}
\subsubsection{Version history}\label{version-history}

\begin{longtable}[]{@{}ll@{}}
\toprule\noalign{}
Version & Release Date \\
\midrule\noalign{}
\endhead
\bottomrule\noalign{}
\endlastfoot
0.1.0 & April 15, 2024 \\
\end{longtable}

Typst GmbH did not create this package and cannot guarantee correct
functionality of this package or compatibility with any version of the
Typst compiler or app.


\section{Package List LaTeX/titleize.tex}
\title{typst.app/universe/package/titleize}

\phantomsection\label{banner}
\section{titleize}\label{titleize}

{ 0.1.1 }

Turn strings into title case

\phantomsection\label{readme}
Small wrapper around the
\href{https://crates.io/crates/titlecase}{titlecase} library to convert
text to title case. It follows the
\href{https://daringfireball.net/2008/05/title_case}{rules defined by
John Gruber} . For more details, refer to the library.

\texttt{\ titlecase\ } applies a show rule, that by default transforms
every string of at least four characters. This limit can be changed with
the \texttt{\ limit\ } parameter. Especially with equations, the results
can be a bit unpredictable, so proceed with care.

\begin{Shaded}
\begin{Highlighting}[]
\NormalTok{\#import "@preview/titleize:0.1.1": titlecase}

\NormalTok{\#for s in (}
\NormalTok{  "Being productive on linux",}
\NormalTok{  "Finding an alternative to Mac OS X — part 2",}
\NormalTok{  [an example with small words and sub{-}phrases: "the example"],}
\NormalTok{) [}
\NormalTok{  \#s =\textgreater{} \#titlecase(s) \textbackslash{}}
\NormalTok{]}

\NormalTok{\#let debug{-}print(x) = \{}
\NormalTok{  if type(x) == content \{}
\NormalTok{    let fields = x.fields()}
\NormalTok{    let func = x.func()}
\NormalTok{    [}
\NormalTok{      \#repr(func)}
\NormalTok{      \#for (k, v) in fields [}
\NormalTok{        {-} \#k: \#debug{-}print(v)}
\NormalTok{      ]}
\NormalTok{    ]}
\NormalTok{  \} else \{}
\NormalTok{    if type(x) == array [}
\NormalTok{      array}
\NormalTok{      \#for y in x [}
\NormalTok{        {-} \#debug{-}print(y)}
\NormalTok{      ]}
\NormalTok{    ] else [}
\NormalTok{      \#repr(type(x)) (\#repr(x))}
\NormalTok{    ]}
\NormalTok{  \}}
\NormalTok{\}}

\NormalTok{\#show: titlecase}

\NormalTok{= This is a test, even with math $a = b + c$}

\NormalTok{In math, text can appear in various places:}

\NormalTok{$}
\NormalTok{  a\_"for example in a subscript" \&= "or in a longer text" \textbackslash{}}
\NormalTok{  f(x) \&= sin(x)}
\NormalTok{$}
\end{Highlighting}
\end{Shaded}

\pandocbounded{\includegraphics[keepaspectratio]{https://github.com/typst/packages/raw/main/packages/preview/titleize/0.1.1/example.png}}

\subsubsection{How to add}\label{how-to-add}

Copy this into your project and use the import as \texttt{\ titleize\ }

\begin{verbatim}
#import "@preview/titleize:0.1.1"
\end{verbatim}

\includesvg[width=0.16667in,height=0.16667in]{/assets/icons/16-copy.svg}

Check the docs for
\href{https://typst.app/docs/reference/scripting/\#packages}{more
information on how to import packages} .

\subsubsection{About}\label{about}

\begin{description}
\tightlist
\item[Author :]
\href{mailto:mail@solidtux.de}{Daniel Hauck}
\item[License:]
MIT
\item[Current version:]
0.1.1
\item[Last updated:]
October 15, 2024
\item[First released:]
October 7, 2024
\item[Archive size:]
253 kB
\href{https://packages.typst.org/preview/titleize-0.1.1.tar.gz}{\pandocbounded{\includesvg[keepaspectratio]{/assets/icons/16-download.svg}}}
\item[Repository:]
\href{https://gitlab.com/SolidTux/titleize}{GitLab}
\item[Categor ies :]
\begin{itemize}
\tightlist
\item[]
\item
  \pandocbounded{\includesvg[keepaspectratio]{/assets/icons/16-text.svg}}
  \href{https://typst.app/universe/search/?category=text}{Text}
\item
  \pandocbounded{\includesvg[keepaspectratio]{/assets/icons/16-code.svg}}
  \href{https://typst.app/universe/search/?category=scripting}{Scripting}
\item
  \pandocbounded{\includesvg[keepaspectratio]{/assets/icons/16-hammer.svg}}
  \href{https://typst.app/universe/search/?category=utility}{Utility}
\end{itemize}
\end{description}

\subsubsection{Where to report issues?}\label{where-to-report-issues}

This package is a project of Daniel Hauck . Report issues on
\href{https://gitlab.com/SolidTux/titleize}{their repository} . You can
also try to ask for help with this package on the
\href{https://forum.typst.app}{Forum} .

Please report this package to the Typst team using the
\href{https://typst.app/contact}{contact form} if you believe it is a
safety hazard or infringes upon your rights.

\phantomsection\label{versions}
\subsubsection{Version history}\label{version-history}

\begin{longtable}[]{@{}ll@{}}
\toprule\noalign{}
Version & Release Date \\
\midrule\noalign{}
\endhead
\bottomrule\noalign{}
\endlastfoot
0.1.1 & October 15, 2024 \\
\href{https://typst.app/universe/package/titleize/0.1.0/}{0.1.0} &
October 7, 2024 \\
\end{longtable}

Typst GmbH did not create this package and cannot guarantee correct
functionality of this package or compatibility with any version of the
Typst compiler or app.


\section{Package List LaTeX/clean-math-paper.tex}
\title{typst.app/universe/package/clean-math-paper}

\phantomsection\label{banner}
\phantomsection\label{template-thumbnail}
\pandocbounded{\includegraphics[keepaspectratio]{https://packages.typst.org/preview/thumbnails/clean-math-paper-0.1.0-small.webp}}

\section{clean-math-paper}\label{clean-math-paper}

{ 0.1.0 }

A simple and good looking template for mathematical papers

\href{/app?template=clean-math-paper&version=0.1.0}{Create project in
app}

\phantomsection\label{readme}
\href{https://github.com/JoshuaLampert/clean-math-paper/actions/workflows/build.yml}{\pandocbounded{\includesvg[keepaspectratio]{https://github.com/JoshuaLampert/clean-math-paper/actions/workflows/build.yml/badge.svg}}}
\href{https://github.com/JoshuaLampert/clean-math-paper}{\pandocbounded{\includegraphics[keepaspectratio]{https://img.shields.io/badge/GitHub-repo-blue}}}
\href{https://opensource.org/licenses/MIT}{\pandocbounded{\includesvg[keepaspectratio]{https://img.shields.io/badge/License-MIT-success.svg}}}

\href{https://typst.app/home/}{Typst} paper template for mathematical
papers built for simple, efficient use and a clean look. Of course, it
can also be used for other subjects, but the following math-specific
features are already contained in the template:

\begin{itemize}
\tightlist
\item
  theorems, lemmas, corollaries, proofs etc. prepared using
  \href{https://typst.app/universe/package/great-theorems}{great-theorems}
\item
  equation settings
\end{itemize}

\subsection{Set-Up}\label{set-up}

The template is already filled with dummy data, to give users an
impression how it looks like. The paper is obtained by compiling
\texttt{\ main.typ\ } .

\begin{itemize}
\tightlist
\item
  after
  \href{https://github.com/typst/typst?tab=readme-ov-file\#installation}{installing
  Typst} you can conveniently use the following to create a new folder
  containing this project.
\end{itemize}

\begin{Shaded}
\begin{Highlighting}[]
\ExtensionTok{typst}\NormalTok{ init @preview/clean{-}math{-}paper:0.1.0}
\end{Highlighting}
\end{Shaded}

\begin{itemize}
\tightlist
\item
  edit the data in \texttt{\ main.typ\ } â†'
  \texttt{\ \#show\ template.with({[}your\ data{]})\ }
\end{itemize}

\subsubsection{Parameters of the
Template}\label{parameters-of-the-template}

\begin{itemize}
\tightlist
\item
  \texttt{\ title\ } : Title of the paper.
\item
  \texttt{\ authors\ } : List of names of the authors of the paper. Each
  entry of the list is a dictionary with the following keys:

  \begin{itemize}
  \tightlist
  \item
    \texttt{\ name\ } : Name of the author.
  \item
    \texttt{\ affiliation-id\ } : The ID of the affiliation in
    \texttt{\ affiliations\ } , see below.
  \item
    optionally \texttt{\ orcid\ } : The \href{https://orcid.org/}{ORCID}
    of the author. If provided, the author’s name will be linked to
    their ORCID profile.
  \end{itemize}
\item
  \texttt{\ affiliations\ } : List of affiliations of the authors. Each
  entry of the list is a dictionary with the following keys:

  \begin{itemize}
  \tightlist
  \item
    \texttt{\ id\ } : ID of the affiliation, which is used to link the
    authors to the affiliation, see above.
  \item
    \texttt{\ name\ } : Name of the affiliation.
  \end{itemize}
\item
  \texttt{\ date\ } : Date of the paper.
\item
  \texttt{\ heading-color\ } : Color of the headings including the
  title.
\item
  \texttt{\ link-color\ } : Color of the links.
\item
  \texttt{\ abstract\ } : Abstract of the paper.
\item
  \texttt{\ keywords\ } : List of keywords of the paper. If not
  provided, nothing will be shown.
\item
  \texttt{\ AMS\ } : List of AMS subject classifications of the paper.
  If not provided, nothing will be shown.
\end{itemize}

\subsection{Acknowledgements}\label{acknowledgements}

Some parts of this template are based on the
\href{https://github.com/mgoulao/arkheion}{arkheion} template.

\subsection{Feedback \& Improvements}\label{feedback-improvements}

If you encounter problems, please open issues. In case you found useful
extensions or improved anything We are also very happy to accept pull
requests.

\href{/app?template=clean-math-paper&version=0.1.0}{Create project in
app}

\subsubsection{How to use}\label{how-to-use}

Click the button above to create a new project using this template in
the Typst app.

You can also use the Typst CLI to start a new project on your computer
using this command:

\begin{verbatim}
typst init @preview/clean-math-paper:0.1.0
\end{verbatim}

\includesvg[width=0.16667in,height=0.16667in]{/assets/icons/16-copy.svg}

\subsubsection{About}\label{about}

\begin{description}
\tightlist
\item[Author :]
\href{https://github.com/JoshuaLampert}{Joshua Lampert}
\item[License:]
MIT
\item[Current version:]
0.1.0
\item[Last updated:]
November 21, 2024
\item[First released:]
November 21, 2024
\item[Minimum Typst version:]
0.12.0
\item[Archive size:]
5.95 kB
\href{https://packages.typst.org/preview/clean-math-paper-0.1.0.tar.gz}{\pandocbounded{\includesvg[keepaspectratio]{/assets/icons/16-download.svg}}}
\item[Repository:]
\href{https://github.com/JoshuaLampert/clean-math-paper}{GitHub}
\item[Discipline s :]
\begin{itemize}
\tightlist
\item[]
\item
  \href{https://typst.app/universe/search/?discipline=mathematics}{Mathematics}
\item
  \href{https://typst.app/universe/search/?discipline=engineering}{Engineering}
\item
  \href{https://typst.app/universe/search/?discipline=computer-science}{Computer
  Science}
\end{itemize}
\item[Categor y :]
\begin{itemize}
\tightlist
\item[]
\item
  \pandocbounded{\includesvg[keepaspectratio]{/assets/icons/16-atom.svg}}
  \href{https://typst.app/universe/search/?category=paper}{Paper}
\end{itemize}
\end{description}

\subsubsection{Where to report issues?}\label{where-to-report-issues}

This template is a project of Joshua Lampert . Report issues on
\href{https://github.com/JoshuaLampert/clean-math-paper}{their
repository} . You can also try to ask for help with this template on the
\href{https://forum.typst.app}{Forum} .

Please report this template to the Typst team using the
\href{https://typst.app/contact}{contact form} if you believe it is a
safety hazard or infringes upon your rights.

\phantomsection\label{versions}
\subsubsection{Version history}\label{version-history}

\begin{longtable}[]{@{}ll@{}}
\toprule\noalign{}
Version & Release Date \\
\midrule\noalign{}
\endhead
\bottomrule\noalign{}
\endlastfoot
0.1.0 & November 21, 2024 \\
\end{longtable}

Typst GmbH did not create this template and cannot guarantee correct
functionality of this template or compatibility with any version of the
Typst compiler or app.


\section{Package List LaTeX/elsearticle.tex}
\title{typst.app/universe/package/elsearticle}

\phantomsection\label{banner}
\phantomsection\label{template-thumbnail}
\pandocbounded{\includegraphics[keepaspectratio]{https://packages.typst.org/preview/thumbnails/elsearticle-0.4.0-small.webp}}

\section{elsearticle}\label{elsearticle}

{ 0.4.0 }

Conversion of the LaTeX elsearticle.cls

{ } Featured Template

\href{/app?template=elsearticle&version=0.4.0}{Create project in app}

\phantomsection\label{readme}
\href{https://github.com/typst/packages/raw/main/packages/preview/elsearticle/0.4.0/}{\pandocbounded{\includesvg[keepaspectratio]{https://img.shields.io/badge/Version-0.4.0-cornflowerblue.svg}}}
\href{https://github.com/maucejo/elsearticle/blob/main/LICENSE}{\pandocbounded{\includegraphics[keepaspectratio]{https://img.shields.io/badge/License-MIT-forestgreen}}}
\href{https://github.com/maucejo/elsearticle/blob/main/docs/manual.pdf}{\pandocbounded{\includegraphics[keepaspectratio]{https://img.shields.io/badge/doc-.pdf-mediumpurple}}}

\texttt{\ elsearticle\ } is a Typst template that aims to mimic the
Elsevier article LaTeX class, a.k.a. \texttt{\ elsearticle.cls\ } ,
provided by Elsevier to format manuscript properly for submission to
their journals.

\subsection{Basic usage}\label{basic-usage}

This section provides the minimal amount of information to get started
with the template. For more detailed information, see the
\href{https://github.com/maucejo/elsearticle/blob/main/docs/manual.pdf}{manual}
.

To use the \texttt{\ elsearticle\ } template, you need to include the
following line at the beginning of your \texttt{\ typ\ } file:

\begin{Shaded}
\begin{Highlighting}[]
\NormalTok{\#import "@preview/elsearticle:0.4.0": *}
\end{Highlighting}
\end{Shaded}

\subsubsection{Initializing the
template}\label{initializing-the-template}

After importing \texttt{\ elsearticle\ } , you have to initialize the
template by a show rule with the \texttt{\ \#elsearticle()\ } command.
This function takes an optional argument to specify the title of the
document.

\begin{itemize}
\tightlist
\item
  \texttt{\ title\ } : Title of the paper
\item
  \texttt{\ author\ } : List of the authors of the paper
\item
  \texttt{\ abstract\ } : Abstract of the paper
\item
  \texttt{\ journal\ } : Name of the journal
\item
  \texttt{\ keywords\ } : List of keywords of the paper
\item
  \texttt{\ format\ } : Format of the paper. Possible values are
  \texttt{\ preprint\ } , \texttt{\ review\ } , \texttt{\ 1p\ } ,
  \texttt{\ 3p\ } , \texttt{\ 5p\ }
\item
  \texttt{\ numcol\ } : Number of columns of the paper. Possible values
  are 1 and 2
\item
  \texttt{\ line-numbering\ } : Enable line numbering. Possible values
  are \texttt{\ true\ } and \texttt{\ false\ }
\end{itemize}

\subsection{Additional features}\label{additional-features}

The \texttt{\ elsearticle\ } template provides additional features to
help you format your document properly.

\subsubsection{Appendix}\label{appendix}

To activate the appendix environment, all you have to do is to place the
following command in your document:

\begin{Shaded}
\begin{Highlighting}[]
\NormalTok{\#show: appendix}

\NormalTok{// Appendix content here}
\end{Highlighting}
\end{Shaded}

\subsubsection{Subfigures}\label{subfigures}

Subfigures are not built-in features of Typst, but the
\texttt{\ elsearticle\ } template provides a way to handle them. It is
based on the \texttt{\ subpar\ } package that allows you to create
subfigures and properly reference them.

\begin{Shaded}
\begin{Highlighting}[]
\NormalTok{  \#subfigure(}
\NormalTok{    figure(image("image1.png"), caption: []), \textless{}figa\textgreater{},}
\NormalTok{    figure(image("image2.png"), caption: []), \textless{}figb\textgreater{},}
\NormalTok{    columns: (1fr, 1fr),}
\NormalTok{    caption: [(a) Left image and (b) Right image],}
\NormalTok{    label: \textless{}fig\textgreater{}}
\NormalTok{  )}
\end{Highlighting}
\end{Shaded}

\subsubsection{Equations}\label{equations}

The \texttt{\ elsearticle\ } template provides the
\texttt{\ \#nonumeq()\ } function to create unnmbered equations. The
latter function can be used as follows:

\begin{Shaded}
\begin{Highlighting}[]
\NormalTok{\#nonumeq[$}
\NormalTok{  y = f(x)}
\NormalTok{  $}
\NormalTok{]}
\end{Highlighting}
\end{Shaded}

\subsection{Roadmap}\label{roadmap}

\emph{Article format}

\begin{itemize}
\tightlist
\item
  {[}x{]} Preprint
\item
  {[}x{]} Review
\item
  {[}x{]} 1p
\item
  {[}x{]} 3p
\item
  {[}x{]} 5p
\end{itemize}

\emph{Environment}

\begin{itemize}
\tightlist
\item
  {[}x{]} Implementation of the \texttt{\ appendix\ } environment
\end{itemize}

\emph{Figures and tables}

\begin{itemize}
\tightlist
\item
  {[}x{]} Implementation of the \texttt{\ subfigure\ } environment
\end{itemize}

\emph{Equations}

\begin{itemize}
\tightlist
\item
  {[}x{]} Proper referencing of equations w.r.t. the context
\item
  {[}x{]} Use of the \texttt{\ equate\ } package to number each equation
  of a system as “(1a)�
\end{itemize}

\emph{Other features}

\begin{itemize}
\tightlist
\item
  {[}x{]} Line numbering - Line numbering - Use the built-in
  \texttt{\ par.line\ } function available from Typst v0.12
\end{itemize}

\subsection{License}\label{license}

MIT licensed

Copyright © 2024 Mathieu AUCEJO (maucejo)

\href{/app?template=elsearticle&version=0.4.0}{Create project in app}

\subsubsection{How to use}\label{how-to-use}

Click the button above to create a new project using this template in
the Typst app.

You can also use the Typst CLI to start a new project on your computer
using this command:

\begin{verbatim}
typst init @preview/elsearticle:0.4.0
\end{verbatim}

\includesvg[width=0.16667in,height=0.16667in]{/assets/icons/16-copy.svg}

\subsubsection{About}\label{about}

\begin{description}
\tightlist
\item[Author :]
Mathieu Aucejo
\item[License:]
MIT
\item[Current version:]
0.4.0
\item[Last updated:]
November 18, 2024
\item[First released:]
July 22, 2024
\item[Archive size:]
95.5 kB
\href{https://packages.typst.org/preview/elsearticle-0.4.0.tar.gz}{\pandocbounded{\includesvg[keepaspectratio]{/assets/icons/16-download.svg}}}
\item[Repository:]
\href{https://github.com/maucejo/elsearticle}{GitHub}
\item[Categor y :]
\begin{itemize}
\tightlist
\item[]
\item
  \pandocbounded{\includesvg[keepaspectratio]{/assets/icons/16-speak.svg}}
  \href{https://typst.app/universe/search/?category=report}{Report}
\end{itemize}
\end{description}

\subsubsection{Where to report issues?}\label{where-to-report-issues}

This template is a project of Mathieu Aucejo . Report issues on
\href{https://github.com/maucejo/elsearticle}{their repository} . You
can also try to ask for help with this template on the
\href{https://forum.typst.app}{Forum} .

Please report this template to the Typst team using the
\href{https://typst.app/contact}{contact form} if you believe it is a
safety hazard or infringes upon your rights.

\phantomsection\label{versions}
\subsubsection{Version history}\label{version-history}

\begin{longtable}[]{@{}ll@{}}
\toprule\noalign{}
Version & Release Date \\
\midrule\noalign{}
\endhead
\bottomrule\noalign{}
\endlastfoot
0.4.0 & November 18, 2024 \\
\href{https://typst.app/universe/package/elsearticle/0.3.0/}{0.3.0} &
October 21, 2024 \\
\href{https://typst.app/universe/package/elsearticle/0.2.1/}{0.2.1} &
September 27, 2024 \\
\href{https://typst.app/universe/package/elsearticle/0.2.0/}{0.2.0} &
August 1, 2024 \\
\href{https://typst.app/universe/package/elsearticle/0.1.0/}{0.1.0} &
July 22, 2024 \\
\end{longtable}

Typst GmbH did not create this template and cannot guarantee correct
functionality of this template or compatibility with any version of the
Typst compiler or app.


\section{Package List LaTeX/hidden-bib.tex}
\title{typst.app/universe/package/hidden-bib}

\phantomsection\label{banner}
\section{hidden-bib}\label{hidden-bib}

{ 0.1.1 }

Create hidden bibliographies or bibliographies with unmentioned (hidden)
citations.

\phantomsection\label{readme}
\href{https://github.com/pklaschka/typst-hidden-bib}{GitHub Repository
including Examples}

A Typst package to create hidden bibliographies or bibliographies with
unmentioned (hidden) citations.

\subsection{Use Cases}\label{use-cases}

\subsubsection{Hidden Bibliographies}\label{hidden-bibliographies}

In some documents, such as a letter, you may want to cite a reference
without printing a bibliography.

This can easily be achieved by wrapping your
\texttt{\ bibliography(...)\ } with the \texttt{\ hidden-bibliography\ }
function after importing the \texttt{\ hidden-bib\ } package.

The code then looks like this:

\begin{Shaded}
\begin{Highlighting}[]
\NormalTok{\#import "@preview/hidden{-}bib:0.1.0": hidden{-}bibliography}

\NormalTok{\#lorem(20) @example1}
\NormalTok{\#lorem(40) @example2[p. 2]}

\NormalTok{\#hidden{-}bibliography(}
\NormalTok{  bibliography("/refs.yml")}
\NormalTok{)}
\end{Highlighting}
\end{Shaded}

\emph{Note that this automatically sets the \texttt{\ style\ } option to
\texttt{\ "chicago-notes"\ } unless you specify a different style.}

\subsubsection{Hidden Citations}\label{hidden-citations}

In some documents, it may be necessary to include items in your
bibliography which weren’t explicitly cited at any specific point in
your document.

The code then looks like this:

\begin{Shaded}
\begin{Highlighting}[]
\NormalTok{\#import "@preview/hidden{-}bib:0.1.0": hidden{-}cite}

\NormalTok{\#hidden{-}cite("example1")}
\end{Highlighting}
\end{Shaded}

\subsubsection{Multiple Hidden
Citations}\label{multiple-hidden-citations}

If you want to include a large number of items in your bibliography
without having to use \texttt{\ hidden-cite\ } (to still get
autocompletion in the web editor), you can use the
\texttt{\ hidden-citations\ } environment.

The code then looks like this:

\begin{Shaded}
\begin{Highlighting}[]
\NormalTok{\#import "@preview/hidden{-}bib:0.1.0": hidden{-}citations}

\NormalTok{\#hidden{-}citations[}
\NormalTok{  @example1}
\NormalTok{  @example2}
\NormalTok{]}
\end{Highlighting}
\end{Shaded}

\subsection{FAQ}\label{faq}

\subsubsection{Why would I want to have hidden citations and a hidden
bibliography?}\label{why-would-i-want-to-have-hidden-citations-and-a-hidden-bibliography}

You don’t. While this package solves both (related) problems, you
should only use one of them at a time. Otherwise, you’ll simply see
nothing at all.

\subsubsection{Why would I want to have hidden
citations?}\label{why-would-i-want-to-have-hidden-citations}

That’s for you to decide. It essentially enables you to include
“uncited references�, similar to LaTeX’s
\texttt{\ \textbackslash{}nocite\{\}\ } command.

\subsection{License}\label{license}

This package is licensed under the MIT license. See the
\href{https://github.com/typst/packages/raw/main/packages/preview/hidden-bib/0.1.1/LICENSE}{LICENSE}
file for details.

\subsubsection{How to add}\label{how-to-add}

Copy this into your project and use the import as
\texttt{\ hidden-bib\ }

\begin{verbatim}
#import "@preview/hidden-bib:0.1.1"
\end{verbatim}

\includesvg[width=0.16667in,height=0.16667in]{/assets/icons/16-copy.svg}

Check the docs for
\href{https://typst.app/docs/reference/scripting/\#packages}{more
information on how to import packages} .

\subsubsection{About}\label{about}

\begin{description}
\tightlist
\item[Author :]
Zuri Klaschka
\item[License:]
MIT
\item[Current version:]
0.1.1
\item[Last updated:]
October 10, 2023
\item[First released:]
October 2, 2023
\item[Archive size:]
2.12 kB
\href{https://packages.typst.org/preview/hidden-bib-0.1.1.tar.gz}{\pandocbounded{\includesvg[keepaspectratio]{/assets/icons/16-download.svg}}}
\item[Repository:]
\href{https://github.com/pklaschka/typst-hidden-bib}{GitHub}
\end{description}

\subsubsection{Where to report issues?}\label{where-to-report-issues}

This package is a project of Zuri Klaschka . Report issues on
\href{https://github.com/pklaschka/typst-hidden-bib}{their repository} .
You can also try to ask for help with this package on the
\href{https://forum.typst.app}{Forum} .

Please report this package to the Typst team using the
\href{https://typst.app/contact}{contact form} if you believe it is a
safety hazard or infringes upon your rights.

\phantomsection\label{versions}
\subsubsection{Version history}\label{version-history}

\begin{longtable}[]{@{}ll@{}}
\toprule\noalign{}
Version & Release Date \\
\midrule\noalign{}
\endhead
\bottomrule\noalign{}
\endlastfoot
0.1.1 & October 10, 2023 \\
\href{https://typst.app/universe/package/hidden-bib/0.1.0/}{0.1.0} &
October 2, 2023 \\
\end{longtable}

Typst GmbH did not create this package and cannot guarantee correct
functionality of this package or compatibility with any version of the
Typst compiler or app.


\section{Package List LaTeX/equate.tex}
\title{typst.app/universe/package/equate}

\phantomsection\label{banner}
\section{equate}\label{equate}

{ 0.2.1 }

Breakable equations with improved numbering.

{ } Featured Package

\phantomsection\label{readme}
A package for improved layout of equations and mathematical expressions.

When applied, this package will split up multi-line block equations into
multiple elements, so that each line can be assigned a separate number.
By default, the equation counter is incremented for each line, but this
behavior can be changed by setting the \texttt{\ sub-numbering\ }
argument to \texttt{\ true\ } . In this case, the equation counter will
only be incremented once for the entire block, and each line will be
assigned a sub-number like \texttt{\ 1a\ } , \texttt{\ 2.1\ } , or
similar, depending on the set equation numbering. You can also set the
\texttt{\ number-mode\ } argument to \texttt{\ "label"\ } to only number
labelled lines. If a label is only applied to the full equation, all
lines will be numbered.

This splitting also makes it possible to spread equations over page
boundaries while keeping alignment in place, which can be useful for
long derivations or proofs. This can be configured by the
\texttt{\ breakable\ } parameter of the \texttt{\ equate\ } function, or
by setting the \texttt{\ breakable\ } parameter of \texttt{\ block\ }
for equations via a show-set rule. Additionally, the alignment of the
equation number is improved, so that it always matches the baseline of
the equation.

If you want to create a “standard� equation with a single equation
number centered across all lines, you can attach the
\texttt{\ \textless{}equate:revoke\textgreater{}\ } label to the
equation. This will disable the effect of this package for the current
equation. This label can also be used in single lines of an equation to
disable numbering for that line only.

\subsection{Usage}\label{usage}

The package comes with a single \texttt{\ equate\ } function that is
supposed to be used as a template. It takes two optional arguments for
customization:

\begin{longtable}[]{@{}llll@{}}
\toprule\noalign{}
Argument & Type & Description & Default \\
\midrule\noalign{}
\endhead
\bottomrule\noalign{}
\endlastfoot
\texttt{\ breakable\ } & \texttt{\ boolean\ } , \texttt{\ auto\ } &
Whether to allow the equation to break across pages. &
\texttt{\ auto\ } \\
\texttt{\ sub-numbering\ } & \texttt{\ boolean\ } & Whether to assign
sub-numbers to each line of an equation. & \texttt{\ false\ } \\
\texttt{\ number-mode\ } & \texttt{\ "line"\ } , \texttt{\ "label"\ } &
Whether to number all lines or only those with a label. &
\texttt{\ "line"\ } \\
\end{longtable}

To reference a specific line of an equation, include the label at the
end of the line, like in the following example:

\begin{Shaded}
\begin{Highlighting}[]
\NormalTok{\#import "@preview/equate:0.2.1": equate}

\NormalTok{\#show: equate.with(breakable: true, sub{-}numbering: true)}
\NormalTok{\#set math.equation(numbering: "(1.1)")}

\NormalTok{The dot product of two vectors $arrow(a)$ and $arrow(b)$ can}
\NormalTok{be calculated as shown in @dot{-}product.}

\NormalTok{$}
\NormalTok{  angle.l a, b angle.r \&= arrow(a) dot arrow(b) \textbackslash{}}
\NormalTok{                       \&= a\_1 b\_1 + a\_2 b\_2 + ... a\_n b\_n \textbackslash{}}
\NormalTok{                       \&= sum\_(i=1)\^{}n a\_i b\_i. \#\textless{}sum\textgreater{}}
\NormalTok{$ \textless{}dot{-}product\textgreater{}}

\NormalTok{The sum notation in @sum is a useful way to express the dot}
\NormalTok{product of two vectors.}
\end{Highlighting}
\end{Shaded}

\pandocbounded{\includesvg[keepaspectratio]{https://github.com/typst/packages/raw/main/packages/preview/equate/0.2.1/assets/example-1.svg}}\\
\pandocbounded{\includesvg[keepaspectratio]{https://github.com/typst/packages/raw/main/packages/preview/equate/0.2.1/assets/example-2.svg}}

\subsubsection{Local Usage}\label{local-usage}

If you only want to use the package features on selected equations, you
can also apply the \texttt{\ equate\ } function directly to the
equation. This will override the default behavior for the current
equation only. Note, that this will require you to use the
\texttt{\ equate\ } function as a show rule for references, as shown in
the following example:

\begin{Shaded}
\begin{Highlighting}[]
\NormalTok{\#import "@preview/equate:0.2.1": equate}

\NormalTok{// Allow references to a line of the equation.}
\NormalTok{\#show ref: equate}

\NormalTok{\#set math.equation(numbering: "(1.1)", supplement: "Eq.")}

\NormalTok{\#equate($}
\NormalTok{  E \&= m c\^{}2 \#\textless{}short\textgreater{} \textbackslash{}}
\NormalTok{    \&= sqrt(p\^{}2 c\^{}2 + m\^{}2 c\^{}4) \#\textless{}long\textgreater{}}
\NormalTok{$)}

\NormalTok{While @short is the famous equation by Einstein, @long is a}
\NormalTok{more general form of the energy{-}momentum relation.}
\end{Highlighting}
\end{Shaded}

\pandocbounded{\includesvg[keepaspectratio]{https://github.com/typst/packages/raw/main/packages/preview/equate/0.2.1/assets/example-local.svg}}

As an alternative to the show rule, it is also possible to manually wrap
each reference in an \texttt{\ equate\ } function, though this is less
convenient and more prone to mistakes.

\subsubsection{How to add}\label{how-to-add}

Copy this into your project and use the import as \texttt{\ equate\ }

\begin{verbatim}
#import "@preview/equate:0.2.1"
\end{verbatim}

\includesvg[width=0.16667in,height=0.16667in]{/assets/icons/16-copy.svg}

Check the docs for
\href{https://typst.app/docs/reference/scripting/\#packages}{more
information on how to import packages} .

\subsubsection{About}\label{about}

\begin{description}
\tightlist
\item[Author :]
Eric Biedert
\item[License:]
MIT
\item[Current version:]
0.2.1
\item[Last updated:]
September 11, 2024
\item[First released:]
July 5, 2024
\item[Minimum Typst version:]
0.11.0
\item[Archive size:]
5.81 kB
\href{https://packages.typst.org/preview/equate-0.2.1.tar.gz}{\pandocbounded{\includesvg[keepaspectratio]{/assets/icons/16-download.svg}}}
\item[Repository:]
\href{https://github.com/EpicEricEE/typst-equate}{GitHub}
\item[Categor ies :]
\begin{itemize}
\tightlist
\item[]
\item
  \pandocbounded{\includesvg[keepaspectratio]{/assets/icons/16-layout.svg}}
  \href{https://typst.app/universe/search/?category=layout}{Layout}
\item
  \pandocbounded{\includesvg[keepaspectratio]{/assets/icons/16-list-unordered.svg}}
  \href{https://typst.app/universe/search/?category=model}{Model}
\end{itemize}
\end{description}

\subsubsection{Where to report issues?}\label{where-to-report-issues}

This package is a project of Eric Biedert . Report issues on
\href{https://github.com/EpicEricEE/typst-equate}{their repository} .
You can also try to ask for help with this package on the
\href{https://forum.typst.app}{Forum} .

Please report this package to the Typst team using the
\href{https://typst.app/contact}{contact form} if you believe it is a
safety hazard or infringes upon your rights.

\phantomsection\label{versions}
\subsubsection{Version history}\label{version-history}

\begin{longtable}[]{@{}ll@{}}
\toprule\noalign{}
Version & Release Date \\
\midrule\noalign{}
\endhead
\bottomrule\noalign{}
\endlastfoot
0.2.1 & September 11, 2024 \\
\href{https://typst.app/universe/package/equate/0.2.0/}{0.2.0} & July 5,
2024 \\
\href{https://typst.app/universe/package/equate/0.1.0/}{0.1.0} & July 5,
2024 \\
\end{longtable}

Typst GmbH did not create this package and cannot guarantee correct
functionality of this package or compatibility with any version of the
Typst compiler or app.


\section{Package List LaTeX/cheda-seu-thesis.tex}
\title{typst.app/universe/package/cheda-seu-thesis}

\phantomsection\label{banner}
\phantomsection\label{template-thumbnail}
\pandocbounded{\includegraphics[keepaspectratio]{https://packages.typst.org/preview/thumbnails/cheda-seu-thesis-0.3.0-small.webp}}

\section{cheda-seu-thesis}\label{cheda-seu-thesis}

{ 0.3.0 }

东å?---大学本ç§`毕设与ç~''究ç''Ÿå­¦ä½?论æ--‡æ¨¡æ?¿ã€‚UNOFFICIAL
Southeast University Thesis.

\href{/app?template=cheda-seu-thesis&version=0.3.0}{Create project in
app}

\phantomsection\label{readme}
使ç''¨ Typst
å¤?刻东å?---大学「本ç§`毕业设计(论æ--‡ï¼‰æŠ¥å`Šã€?模æ?¿å'Œã€Œç~''究ç''Ÿå­¦ä½?论æ--‡ã€?模æ?¿ã€‚

请在
\href{https://github.com/typst/packages/raw/main/packages/preview/cheda-seu-thesis/0.3.0/init-files/}{\texttt{\ init-files\ }}
目录å†\ldots 查看 Demo PDF。

\begin{quote}
{[}!IMPORTANT{]}

此模æ?¿æ˜¯æ°`é---´æ¨¡æ?¿ï¼Œæœ‰ä¸?被学æ~¡è®¤å?¯çš„风险。

本模�虽已尽力�试�原原始 Word
模æ?¿ï¼Œä½†å?¯èƒ½ä»?然存在诸多æ~¼å¼?é---®é¢˜ã€‚

Typst
是一个ä»?在活跃开å?{}`ã€?å?¯èƒ½ä¼šæœ‰è¾ƒå¤§å?˜æ›´çš„æŽ'版工å\ldots·ï¼Œè¯·é€‰æ‹©æœ€æ--°ç‰ˆæ¨¡æ?¿ä¸Žæœ¬æ¨¡æ?¿å»ºè®®çš„
Typst 版本相é\ldots?å?ˆä½¿ç''¨ã€‚
\end{quote}

\begin{quote}
{[}!CAUTION{]}

本模�在
\href{https://github.com/csimide/SEU-Typst-Template/tree/c44b5172178c0c2380b322e50931750e2d761168}{\texttt{\ 0.2.2\ }}
-\textgreater{} \texttt{\ 0.3.0\ }
æ---¶è¿›è¡Œäº†ç~´å??性å?˜æ›´ã€‚有å\ldots³æ­¤æ¬¡å?˜æ›´çš„详细信æ?¯ï¼Œè¯·æŸ¥çœ‹
\href{https://github.com/typst/packages/raw/main/packages/preview/cheda-seu-thesis/0.3.0/CHANGELOG.md}{æ›´æ--°æ---¥å¿---}
\end{quote}

\begin{itemize}
\tightlist
\item
  \href{https://github.com/typst/packages/raw/main/packages/preview/cheda-seu-thesis/0.3.0/\#\%E4\%B8\%9C\%E5\%8D\%97\%E5\%A4\%A7\%E5\%AD\%A6\%E8\%AE\%BA\%E6\%96\%87\%E6\%A8\%A1\%E6\%9D\%BF}{东å?---大学论æ--‡æ¨¡æ?¿}

  \begin{itemize}
  \tightlist
  \item
    \href{https://github.com/typst/packages/raw/main/packages/preview/cheda-seu-thesis/0.3.0/\#\%E4\%BD\%BF\%E7\%94\%A8\%E6\%96\%B9\%E6\%B3\%95}{使ç''¨æ--¹æ³•}

    \begin{itemize}
    \tightlist
    \item
      \href{https://github.com/typst/packages/raw/main/packages/preview/cheda-seu-thesis/0.3.0/\#\%E6\%9C\%AC\%E5\%9C\%B0\%E4\%BD\%BF\%E7\%94\%A8}{本地使ç''¨}
    \item
      \href{https://github.com/typst/packages/raw/main/packages/preview/cheda-seu-thesis/0.3.0/\#web-app}{Web
      App}
    \end{itemize}
  \item
    \href{https://github.com/typst/packages/raw/main/packages/preview/cheda-seu-thesis/0.3.0/\#\%E6\%A8\%A1\%E6\%9D\%BF\%E5\%86\%85\%E5\%AE\%B9}{模æ?¿å†\ldots 容}

    \begin{itemize}
    \tightlist
    \item
      \href{https://github.com/typst/packages/raw/main/packages/preview/cheda-seu-thesis/0.3.0/\#\%E7\%A0\%94\%E7\%A9\%B6\%E7\%94\%9F\%E5\%AD\%A6\%E4\%BD\%8D\%E8\%AE\%BA\%E6\%96\%87\%E6\%A8\%A1\%E6\%9D\%BF}{ç~''究ç''Ÿå­¦ä½?论æ--‡æ¨¡æ?¿}
    \item
      \href{https://github.com/typst/packages/raw/main/packages/preview/cheda-seu-thesis/0.3.0/\#\%E6\%9C\%AC\%E7\%A7\%91\%E6\%AF\%95\%E4\%B8\%9A\%E8\%AE\%BE\%E8\%AE\%A1\%E8\%AE\%BA\%E6\%96\%87\%E6\%8A\%A5\%E5\%91\%8A\%E6\%A8\%A1\%E6\%9D\%BF}{本ç§`毕业设计(论æ--‡ï¼‰æŠ¥å`Šæ¨¡æ?¿}
    \end{itemize}
  \item
    \href{https://github.com/typst/packages/raw/main/packages/preview/cheda-seu-thesis/0.3.0/\#\%E7\%9B\%AE\%E5\%89\%8D\%E5\%AD\%98\%E5\%9C\%A8\%E7\%9A\%84\%E9\%97\%AE\%E9\%A2\%98}{ç›®å‰?存在的é---®é¢˜}

    \begin{itemize}
    \tightlist
    \item
      \href{https://github.com/typst/packages/raw/main/packages/preview/cheda-seu-thesis/0.3.0/\#\%E5\%8F\%82\%E8\%80\%83\%E6\%96\%87\%E7\%8C\%AE}{å?‚考æ--‡çŒ®}
    \end{itemize}
  \item
    \href{https://github.com/typst/packages/raw/main/packages/preview/cheda-seu-thesis/0.3.0/\#\%E5\%8F\%8B\%E6\%83\%85\%E9\%93\%BE\%E6\%8E\%A5}{å?‹æƒ\ldots é``¾æŽ¥}
  \item
    \href{https://github.com/typst/packages/raw/main/packages/preview/cheda-seu-thesis/0.3.0/\#\%E5\%BC\%80\%E5\%8F\%91\%E4\%B8\%8E\%E5\%8D\%8F\%E8\%AE\%AE}{å¼€å?{}`与å??è®®}

    \begin{itemize}
    \tightlist
    \item
      \href{https://github.com/typst/packages/raw/main/packages/preview/cheda-seu-thesis/0.3.0/\#\%E4\%BA\%8C\%E6\%AC\%A1\%E5\%BC\%80\%E5\%8F\%91}{二次开å?{}`}
    \end{itemize}
  \end{itemize}
\end{itemize}

\subsection{使ç''¨æ--¹æ³•}\label{uxe4uxbduxe7uxe6uxb9uxe6uxb3}

本模æ?¿éœ€è¦?使ç''¨ Typst 0.11.x ç¼--è¯`。

此模æ?¿å·²ä¸Šä¼~ Typst Universe ,å?¯ä»¥ä½¿ç''¨
\texttt{\ typst\ init\ } 功能åˆ?始åŒ--,也å?¯ä»¥ä½¿ç''¨ Web App
ç¼--è¾`。Typst Universe
上的模æ?¿å?¯èƒ½ä¸?是最æ--°ç‰ˆæœ¬ã€‚如果需è¦?使ç''¨æœ€æ--°ç‰ˆæœ¬çš„模æ?¿ï¼Œä»Žæœ¬
repo 中获å?--。

\subsubsection{本地使ç''¨}\label{uxe6ux153uxe5ux153uxe4uxbduxe7}

请å\ldots ˆå®‰è£\ldots ä½?于 \texttt{\ fonts\ }
目录å†\ldots çš„å\ldots¨éƒ¨å­---ä½``。然å?Žï¼Œæ‚¨å?¯ä»¥ä½¿ç''¨ä»¥ä¸‹ä¸¤ç§?æ--¹å¼?使ç''¨æœ¬æ¨¡æ?¿ï¼š

\begin{itemize}
\tightlist
\item
  下载/clone 本 repo çš„å\ldots¨éƒ¨æ--‡ä»¶ï¼Œç¼--è¾`
  \texttt{\ init-files\ } 目录å†\ldots 的示例æ--‡ä»¶ã€‚
\item
  使ç''¨ \texttt{\ typst\ init\ @preview/cheda-seu-thesis:0.2.2\ }
  æ?¥èŽ·å?--此模æ?¿ä¸Žåˆ?始åŒ--æ--‡ä»¶ã€‚
\end{itemize}

éš?å?Žï¼Œæ‚¨å?¯ä»¥é€šè¿‡ç¼--è¾`示例æ--‡ä»¶æ?¥ç''Ÿæˆ?想è¦?的论æ--‡ã€‚两ç§?论æ--‡æ~¼å¼?的说明都在对åº''的示例æ--‡æ¡£å†\ldots 。

如您使ç''¨ VSCode 作为ç¼--è¾`器,å?¯ä»¥å°?试使ç''¨
\href{https://marketplace.visualstudio.com/items?itemName=nvarner.typst-lsp}{Tinymist}
与
\href{https://marketplace.visualstudio.com/items?itemName=mgt19937.typst-preview}{Typst
Preview} æ?'件。如有本地åŒ\ldots äº`å?Œæ­¥éœ€æ±‚,å?¯ä»¥ä½¿ç''¨
\href{https://marketplace.visualstudio.com/items?itemName=OrangeX4.vscode-typst-sync}{Typst
Sync} æ?'件。更多ç¼--è¾`技巧,å?¯æŸ¥é˜
\href{https://github.com/nju-lug/modern-nju-thesis\#vs-code-\%E6\%9C\%AC\%E5\%9C\%B0\%E7\%BC\%96\%E8\%BE\%91\%E6\%8E\%A8\%E8\%8D\%90}{https://github.com/nju-lug/modern-nju-thesis\#vs-code-本地ç¼--è¾`推è??}
。

\subsubsection{Web App}\label{web-app}

\begin{quote}
{[}!NOTE{]}

ç''±äºŽå­---ä½``原å›~,ä¸?建议使ç''¨ Web App ç¼--è¾`此模æ?¿ã€‚
\end{quote}

请æ‰``å¼€ \url{https://typst.app/universe/package/cheda-seu-thesis}
并点击 \texttt{\ Create\ project\ in\ app\ } ,æˆ--在 Web App
中选择 \texttt{\ Start\ from\ a\ template\ } ,�选择
\texttt{\ cheda-seu-thesis\ } 。

然�,请将
\url{https://github.com/csimide/SEU-Typst-Template/tree/master/fonts}
å†\ldots çš„ \textbf{所有} å­---ä½``上ä¼~到 Typst Web App
å†\ldots 该项目的æ~¹ç›®å½•ã€‚注æ„?,之å?Žæ¯?次æ‰``开此项目,æµ?览器都会花费很长æ---¶é---´ä»Ž
Typst Web App çš„æœ?务器下载这一批å­---ä½``,ä½``验较差。

最å?Žï¼Œè¯·æŒ‰ç\ldots§è‡ªåŠ¨æ‰``开的æ--‡ä»¶çš„æ??示æ``?作。

\subsection{模æ?¿å†\ldots 容}\label{uxe6uxe6uxe5uxe5uxb9}

\subsubsection{ç~''究ç''Ÿå­¦ä½?论æ--‡æ¨¡æ?¿}\label{uxe7-uxe7uxe7uxffuxe5uxe4uxbduxe8uxbauxe6uxe6uxe6}

æ­¤ Typst 模æ?¿æŒ‰ç\ldots§
\href{https://seugs.seu.edu.cn/2023/0424/c26669a442680/page.htm}{《东å?---大学ç~''究ç''Ÿå­¦ä½?论æ--‡æ~¼å¼?规定》}
制作,制作æ---¶å?‚考了
\href{https://ctan.math.utah.edu/ctan/tex-archive/macros/latex/contrib/seuthesis/seuthesis.pdf}{SEUThesis
模�} 。

å½``å‰?æ''¯æŒ?进度:

\begin{itemize}
\tightlist
\item
  æ--‡æ¡£æž„件

  \begin{itemize}
  \tightlist
  \item
    {[}x{]} å°?é?¢
  \item
    {[}x{]} 中英æ--‡æ‰‰é¡µ
  \item
    {[}x{]} 中英æ--‡æ`˜è¦?
  \item
    {[}x{]} 目录
  \item
    {[}x{]} 术语表
  \item
    {[}x{]} æ­£æ--‡
  \item
    {[}x{]} 致谢
  \item
    {[}x{]} å?‚考æ--‡çŒ®
  \item
    {[}x{]} 附录
  \item
    {[} {]} 索引
  \item
    {[} {]} 作è€\ldots 简介
  \item
    {[} {]} å?Žè®°
  \item
    {[} {]} �底
  \end{itemize}
\item
  功能

  \begin{itemize}
  \tightlist
  \item
    {[} {]} 盲审
  \item
    {[}x{]}
    页ç~?ç¼--å?·ï¼šæ­£æ--‡å‰?使ç''¨ç½---马数å­---,正æ--‡å?Šæ­£æ--‡å?Žä½¿ç''¨é˜¿æ‹‰ä¼¯æ•°å­---
  \item
    {[}x{]} æ­£æ--‡ã€?附录图表ç¼--å?·æ~¼å¼?:详è§?ç~''院è¦?求
  \item
    {[}x{]}
    æ•°å­¦å\ldots¬å¼?æ''¾ç½®ä½?置:离页é?¢å·¦ä¾§ä¸¤ä¸ªæ±‰å­---è·?离
  \item
    {[}x{]} æ•°å­¦å\ldots¬å¼?ç¼--å?·ï¼šå\ldots¬å¼?å?---å?³ä¸‹
  \item
    {[}x{]} æ?'å\ldots¥ç©ºç™½é¡µï¼šæ--°ç«~节总在奇数页上
  \item
    {[}x{]}
    页眉:奇数页显示ç«~节å?·å'Œç«~节æ~‡é¢˜ï¼Œå?¶æ•°é¡µæ˜¾ç¤ºå›ºå®šå†\ldots 容
  \item
    {[}x{]} å?‚考æ--‡çŒ®ï¼šæ''¯æŒ?å?Œè¯­æ˜¾ç¤º
  \end{itemize}
\end{itemize}

\subsubsection{本ç§`毕业设计(论æ--‡ï¼‰æŠ¥å`Šæ¨¡æ?¿}\label{uxe6ux153uxe7uxe6uxe4ux161uxe8uxbeuxe8uxefuxbcux2c6uxe8uxbauxe6uxefuxbcuxe6ux161uxe5ux161uxe6uxe6}

æ­¤ Typst
模æ?¿åŸºäºŽä¸œå?---大学本ç§`毕业设计(论æ--‡ï¼‰æŠ¥å`Šæ¨¡æ?¿ï¼ˆ2024
å¹´ 1 月)仿制,原模æ?¿å?¯ä»¥åœ¨æ•™åŠ¡å¤„ç½`站上下载(
\href{https://jwc.seu.edu.cn/2021/1108/c21686a389963/page.htm}{2019 å¹´
9 月版} ,
\href{https://jwc.seu.edu.cn/2024/0117/c21686a479303/page.htm}{2024 å¹´
1 月版} )。

å½``å‰?æ''¯æŒ?进度:

\begin{itemize}
\tightlist
\item
  æ--‡æ¡£æž„件

  \begin{itemize}
  \tightlist
  \item
    {[}x{]} å°?é?¢
  \item
    {[}x{]} 中英æ--‡æ`˜è¦?
  \item
    {[}x{]} 目录
  \item
    {[}x{]} æ­£æ--‡
  \item
    {[}x{]} å?‚考æ--‡çŒ®
  \item
    {[}x{]} 附录
  \item
    {[}x{]} 致谢
  \item
    {[} {]} �底
  \end{itemize}
\item
  功能

  \begin{itemize}
  \tightlist
  \item
    {[} {]} 盲审
  \item
    {[}x{]}
    页ç~?ç¼--å?·ï¼šæ­£æ--‡å‰?使ç''¨ç½---马数å­---,正æ--‡å?Šæ­£æ--‡å?Žä½¿ç''¨é˜¿æ‹‰ä¼¯æ•°å­---
  \item
    {[}x{]}
    æ­£æ--‡ã€?附录图表ç¼--å?·æ~¼å¼?:详è§?本ç§`毕设è¦?求
  \item
    {[}x{]}
    æ•°å­¦å\ldots¬å¼?æ''¾ç½®ä½?置:离页é?¢å·¦ä¾§ä¸¤ä¸ªæ±‰å­---è·?离
  \item
    {[}x{]} æ•°å­¦å\ldots¬å¼?ç¼--å?·ï¼šå\ldots¬å¼?å?---å?³ä¾§ä¸­å¿ƒ
  \item
    {[}x{]} 页眉:显示固定å†\ldots 容
  \item
    {[}x{]} å?‚考æ--‡çŒ®ï¼šæ''¯æŒ?å?Œè¯­æ˜¾ç¤º
  \item
    {[} {]} \st{表æ~¼æ˜¾ç¤ºâ€œç»­è¡¨â€?}
    ç''±äºŽæ•™åŠ¡å¤„æ??供的模æ?¿ä¸­æ²¡æœ‰ç»™å‡ºâ€œç»­è¡¨â€?显示æ~·ä¾‹ï¼Œæ•\ldots æš‚ä¸?实现。
  \end{itemize}
\end{itemize}

\begin{quote}
{[}!NOTE{]}

å?¯ä»¥çœ‹çœ‹éš''å£? \url{https://github.com/TideDra/seu-thesis-typst/}
项目,也正在使ç''¨ Typst
实现毕业设计(论æ--‡ï¼‰æŠ¥å`Šæ¨¡æ?¿ï¼Œè¿˜æ??供了毕设翻è¯`模æ?¿ã€‚该项目的实现细节与本模æ?¿å¹¶ä¸?相å?Œï¼Œæ‚¨å?¯ä»¥æ~¹æ?®è‡ªå·±çš„å--œå¥½é€‰æ‹©ã€‚
\end{quote}

\subsection{ç›®å‰?存在的é---®é¢˜}\label{uxe7uxe5uxe5uxe5ux153uxe7ux161uxe9uxe9}

\begin{itemize}
\tightlist
\item
  中æ--‡é¦--段有æ---¶ä¼šè‡ªåŠ¨ç¼©è¿›ï¼Œæœ‰æ---¶ä¸?会。如果没有自动缩进,需è¦?使ç''¨
  \texttt{\ \#h(2em)\ } 手动缩进两个å­---符。
\item
  å?‚考æ--‡çŒ®æ~¼å¼?ä¸?完å\ldots¨ç¬¦å?ˆè¦?求。请è§?下æ--¹å?‚考æ--‡çŒ®å°?节。
\item
  è¡Œè·?ã€?è¾¹è·?等有å¾\ldots 继续调整。
\end{itemize}

\subsubsection{å?‚考æ--‡çŒ®}\label{uxe5uxe8ux192uxe6uxe7ux153}

å?‚考æ--‡çŒ®æ~¼å¼?ä¸?完å\ldots¨ç¬¦å?ˆè¦?求。Typst 自带的 GB/T
7714-2015 numeric
æ~¼å¼?与学æ~¡è¦?求æ~¼å¼?相æ¯'',有以下é---®é¢˜ï¼š

\begin{enumerate}
\item
  å­¦æ~¡è¦?求在作è€\ldots æ•°é‡?较多æ---¶ï¼Œè‹±æ--‡ä½¿ç''¨
  \texttt{\ et\ al.\ } 中æ--‡ä½¿ç''¨ \texttt{\ ç­‰\ }
  ��略。但是,Typst
  ç›®å‰?ä»\ldots å?¯ä»¥æ˜¾ç¤ºä¸ºå?•ä¸€è¯­è¨€ã€‚

  \textbf{A:} 该é---®é¢˜ç³» Typst çš„ CSL 解æž?器ä¸?æ''¯æŒ? CSL-M
  导致的。

  详细原å›

  \begin{itemize}
  \tightlist
  \item
    使ç''¨ CSL 实现这一 feature 需è¦?ç''¨åˆ°
    \href{https://citeproc-js.readthedocs.io/en/latest/csl-m/index.html\#cs-layout-extension}{CSL-M}
    扩展的多 \texttt{\ layout\ } 功能,而 Typst å°šä¸?æ''¯æŒ?
    CSL-M 扩展功能。详�
    \url{https://github.com/typst/typst/issues/2793} 与
    \url{https://github.com/typst/citationberg/issues/5} 。
  \item
    Typst 目�会忽视 BibTeX/CSL 中的 \texttt{\ language\ }
    å­---段。å?‚è§? \url{https://github.com/typst/hayagriva/pull/126}
    。
  \end{itemize}

  å›~为上述原å›~,目å‰?很难使ç''¨ Typst
  原ç''Ÿæ--¹æ³•å®žçŽ°æ~¹æ?®è¯­è¨€è‡ªåŠ¨é€‰ç''¨ \texttt{\ et\ al.\ } 与
  \texttt{\ 等\ } 。

  OrangeX4 å'Œæˆ`写了一个基于查找替æ?¢çš„
  \texttt{\ bilingual-bibliography\ } 功能,试图在 Typst æ''¯æŒ?
  CSL-M å‰?实现中æ--‡è¥¿æ--‡ä½¿ç''¨ä¸?å?Œçš„å\ldots³é''®è¯?。

  本模æ?¿çš„ Demo æ--‡æ¡£å†\ldots 已使ç''¨
  \texttt{\ bilingual-bibliography\ } 引ç''¨ï¼Œè¯·æŸ¥çœ‹ Demo
  æ--‡æ¡£ä»¥äº†è§£ç''¨æ³•ã€‚注æ„?,该功能ä»?在测试,很å?¯èƒ½æœ‰
  Bug,详�
  \url{https://github.com/csimide/SEU-Typst-Template/issues/1} 。

  \begin{quote}
  请在 \url{https://github.com/nju-lug/modern-nju-thesis/issues/3}
  查看更多有å\ldots³å?Œè¯­å?‚考æ--‡çŒ®å®žçŽ°çš„讨论。

  本模æ?¿æ›¾ç»?å°?试使ç''¨ \url{https://github.com/csimide/cslper}
  作为å?Œè¯­å?‚考æ--‡çŒ®çš„实现æ--¹æ³•ã€‚
  \end{quote}
\item
  å­¦æ~¡ç»™å‡ºçš„范例中,除了纯ç''µå­?资æº?,å?³ä½¿å¼•ç''¨æ--‡çŒ®æ?¥è‡ªçº¿ä¸Šæ¸~é?{}``,也å?‡ä¸?åŠ
  \texttt{\ OL\ } ã€?访é---®æ---¥æœŸã€?DOI 与 é``¾æŽ¥ã€‚但是,Typst
  å†\ldots 置的 GB/T 7714-2015 numeric æ~¼å¼?会为所有 bib
  å†\ldots 定义了é``¾æŽ¥/DOI çš„æ--‡çŒ®æ·»åŠ \texttt{\ OL\ }
  æ~‡è®°å'Œé``¾æŽ¥/DOI 。

  \textbf{A:} 该é---®é¢˜ç³»å­¦æ~¡çš„æ~‡å‡†ä¸Ž GB/T 7714-2015
  ä¸?完å\ldots¨ä¸€è‡´å¯¼è‡´çš„。

  请使ç''¨
  \texttt{\ style:\ "./seu-thesis/gb-t-7714-2015-numeric-seu.csl"\ }
  ,会自动ä¾?æ?®æ--‡çŒ®ç±»åž‹é€‰æ‹©æ˜¯å?¦æ˜¾ç¤º \texttt{\ OL\ }
  æ~‡è®°å'Œé``¾æŽ¥/DOI。

  \begin{quote}
  该 csl ä¿®æ''¹è‡ª
  \url{https://github.com/redleafnew/Chinese-STD-GB-T-7714-related-csl/blob/main/003gb-t-7714-2015-numeric-bilingual-no-url-doi.csl}

  原æ--‡ä»¶åŸºäºŽ CC-BY-SA 3.0 å??è®®å\ldots±äº«ã€‚
  \end{quote}
\item
  作è€\ldots 大å°?写(æˆ--è€\ldots å\ldots¶ä»--细节)与学æ~¡èŒƒä¾‹ä¸?一致。
\item
  å­¦ä½?论æ--‡ä¸­ï¼Œå­¦æ~¡è¦?求引ç''¨å\ldots¶ä»--å­¦ä½?论æ--‡çš„æ--‡çŒ®ç±»åž‹åº''å½``写作
  \texttt{\ {[}D{]}:\ {[}�士学�论文{]}.\ }
  æ~¼å¼?,但模æ?¿æ˜¾ç¤ºä¸º \texttt{\ {[}D{]}\ }
  ,�显示�类别。
\item
  å­¦ä½?论æ--‡ä¸­ï¼Œå­¦æ~¡ç»™å‡ºçš„范例使ç''¨å\ldots¨è§'符å?·ï¼Œå¦‚å\ldots¨è§'æ--¹æ‹¬å?·ã€?å\ldots¨è§'å?¥ç‚¹ç­‰ã€‚
\item
  引ç''¨æ?¡ç›®ä¸¢å¤± \texttt{\ .\ } ,如 \texttt{\ {[}M{]}2nd\ ed\ }
  。

  \textbf{3\textasciitilde6 A:} å­¦æ~¡ç''¨çš„是 GB/T 7714-2015
  çš„æ--¹è¨€ï¼Œæ›¾ç»?有学长把它å?«å?š GB/T 7714-SEU
  ,目å‰?没找到完美匹é\ldots?å­¦æ~¡è¦?求的
  CSL(ä¸?å?Œå­¦é™¢çš„è¦?求也ä¸?太一æ~·ï¼‰ï¼Œå?Žç»­ä¼šå†™ä¸€ä¸ªç¬¦å?ˆè¦?求的
  CSL æ--‡ä»¶ã€‚

  \textbf{2024-05-02 æ›´æ--°:} 现已åˆ?步实现 CSL。ä¸?å¾---ä¸?说
  Typst çš„ CSL æ''¯æŒ?æˆ?谜……目å‰?ä¿®å¤?æƒ\ldots 况如下:

  \begin{itemize}
  \tightlist
  \item
    é---®é¢˜ 3 已修å¤?ï¼›
  \item
    é---®é¢˜ 4 在学ä½?论æ--‡çš„ CSL å†\ldots 已修å¤?,但 Typst
    似乎ä¸?æ''¯æŒ?这一å­---段,å›~æ­¤æ---~法显示;
  \item
    é---®é¢˜ 5
    ä¸?准备修å¤?,查é˜\ldots 数篇已å?{}`表的学ä½?论æ--‡ï¼Œä½¿ç''¨çš„也是å?Šè§'符å?·ï¼›
  \item
    é---®é¢˜ 6 似乎是 Typst çš„ CSL
    æ''¯æŒ?çš„é---®é¢˜ï¼Œæœ¬æ¨¡æ?¿é™„带的 CSL
    æ--‡ä»¶å·²ç»?å?šäº†é¢?å¤--处ç?†ï¼Œåº''该ä¸?会丢 \texttt{\ .\ }
    了。
  \end{itemize}
\item
  引ç''¨å\ldots¶ä»--å­¦ä½?论æ--‡æ---¶ï¼ŒGB7714-2015/本ç§`毕设/å­¦ä½?论æ--‡å?‡è¦?求注明
  \texttt{\ 地点:\ å­¦æ~¡å??称,\ 年份.\ }
  。但是模��显示这一项。

  \textbf{A:} Typst ä¸?æ''¯æŒ? \texttt{\ school\ }
  \texttt{\ institution\ } 作为 \texttt{\ publisher\ }
  的别å??,亦ä¸?æ''¯æŒ?解æž? csl 中的 \texttt{\ institution\ } (
  \url{https://github.com/typst/hayagriva/issues/112}
  )。如需修å¤?,请手动修æ''¹ bib
  æ--‡ä»¶å†\ldots 对åº''æ?¡ç›®ï¼Œåœ¨
  \texttt{\ school\ =\ \{å­¦æ~¡å??称\},\ } 下åŠ~一行
  \texttt{\ publisher\ =\ \{å­¦æ~¡å??称\},\ } 。

  ä¿®æ''¹ç¤ºä¾‹

\begin{Shaded}
\begin{Highlighting}[]
\NormalTok{@phdthesis\{Example1,}
\NormalTok{  type = \{\{硕士学位论文\}\},}
\NormalTok{  title = \{\{摸鱼背景下的Typst模板使用研究\}\},}
\NormalTok{  author = \{王, 东南\},}
\NormalTok{  year = \{2024\},}
\NormalTok{  langid = \{chinese\},}
\NormalTok{  address = \{南京\},}
\NormalTok{  school = \{东南大学\},}
\NormalTok{  publisher = \{东南大学\},}
\NormalTok{\}}
\end{Highlighting}
\end{Shaded}
\item
  æ­£æ--‡ä¸­è¿žç»­å¼•ç''¨ï¼Œä¸Šæ~‡å?ˆå¹¶é''™è¯¯ï¼ˆä¾‹å¦‚,引ç''¨ 1 2 3 4
  åº''å½``显示为 {[}1-4{]} ,但是显示为 {[}1,4{]} )。

  \textbf{A:} 临æ---¶æ--¹æ¡ˆæ˜¯æŠŠ csl æ--‡ä»¶é‡Œ
  \texttt{\ after-collapse-delimiter=","\ } æ''¹æˆ?
  \texttt{\ after-collapse-delimiter="-"\ } 。本模�附带的 CSL
  æ--‡ä»¶å·²å?šæ­¤ä¿®æ''¹ã€‚

  详细原å›~请è§? \url{https://github.com/typst/hayagriva/issues/154}
  。

  \url{https://github.com/typst/hayagriva/pull/176} 正�试解决这一
  bug。 \textbf{该 bug ä¿®å¤?å?Žï¼Œè¯·å?Šæ---¶æ'¤é''€ä¸Šè¿°å¯¹ csl
  的临æ---¶ä¿®æ''¹ã€‚}
\end{enumerate}

\subsection{å?‹æƒ\ldots é``¾æŽ¥}\label{uxe5uxe6ux192uxe9uxbeuxe6ux17e}

\begin{itemize}
\tightlist
\item
  Typst Touying 东å?---大学主题幻ç?¯ç‰‡æ¨¡æ?¿ by QuadnucYard -
  \url{https://github.com/QuadnucYard/touying-theme-seu}
\item
  东å?---大学 Typst 本ç§`毕设模æ?¿ä¸Žæ¯•è®¾ç¿»è¯`模æ?¿ by Geary.Z
  (TideDra) - \url{https://github.com/TideDra/seu-thesis-typst}
\end{itemize}

\subsection{å¼€å?{}`与å??è®®}\label{uxe5uxbcuxe5uxe4ux17euxe5uxe8}

如果您在使ç''¨è¿‡ç¨‹ä¸­é?‡åˆ°ä»»ä½•é---®é¢˜ï¼Œè¯·æ??交
issue。本项目欢迎您的
PR。如果有å\ldots¶ä»--模æ?¿éœ€æ±‚也å?¯ä»¥åœ¨ issue 中æ??出。

除下述特殊说明的æ--‡ä»¶å¤--,此项目使ç''¨ MIT License 。

\begin{itemize}
\tightlist
\item
  \texttt{\ init-files/demo\_image/\ }
  路径下的æ--‡ä»¶æ?¥è‡ªä¸œå?---大学教务处本ç§`毕设模æ?¿ã€‚
\item
  \texttt{\ seu-thesis/assets/\ }
  路径下的æ--‡ä»¶æ˜¯ç''±ä¸œå?---大学教务处模æ?¿ç»?二次åŠ~å·¥å¾---到,æˆ--从东å?---大学视觉设计中å?--å¾---。
\item
  \texttt{\ fonts\ } 路径下的æ--‡ä»¶æ˜¯æ­¤æ¨¡æ?¿ç''¨åˆ°çš„å­---ä½``。
\item
  \texttt{\ 东�大学本科毕业设计(论文)�考模�\ (2024年1月修订).docx\ }
  是教务处æ??供的毕设论æ--‡æ¨¡æ?¿ã€‚
\end{itemize}

\subsubsection{二次开å?{}`}\label{uxe4uxbaux153uxe6uxe5uxbcuxe5}

本模æ?¿æ¬¢è¿ŽäºŒæ¬¡å¼€å?{}`。在二次开å?{}`å‰?,建议了解本模æ?¿çš„主è¦?特性与å\ldots³è?''çš„æ--‡ä»¶ï¼š

\begin{itemize}
\item
  有较为麻烦的图表ã€?å\ldots¬å¼?ç¼--å?·ï¼ˆå›¾è¡¨ç¼--å?·æ~¼å¼?ä¸?相å?Œï¼Œç''šè‡³é™„录与正æ--‡ä¸­å›¾è¡¨ç¼--å?·æ~¼å¼?也ä¸?相å?Œï¼›å›¾çš„å??称在图下æ--¹ï¼Œè¡¨çš„å??称在表上æ--¹ï¼›å\ldots¬å¼?ä¸?是å±\ldots 中对é½?,å\ldots¬å¼?ç¼--å?·ä½?ç½®ä¸?是å?³ä¾§ä¸Šä¸‹å±\ldots 中)。

  \begin{itemize}
  \tightlist
  \item
    å·²ç»?æ''¹ç''¨ \texttt{\ i-figured\ } åŒ\ldots 完æˆ?。
  \end{itemize}
\item
  (ä»\ldots ç~''究ç''Ÿå­¦ä½?论æ--‡ï¼‰å¥‡æ•°é¡µå?¶æ•°é¡µé¡µçœ‰ä¸?å?Œï¼Œä¸''有页眉中显示ç«~节å??称的需求。

  \begin{itemize}
  \tightlist
  \item
    该功能�于
    \texttt{\ seu-thesis/parts/main-body-degree-fn.typ\ } 。
  \item
    推è??æ''¹ç''¨ \texttt{\ chic-hdr\ }
    而ä¸?是自é€~è½®å­?,ç''±äºŽåŽ†å?²é?---ç•™é---®é¢˜æœ¬æ¨¡æ?¿æš‚未æ''¹ç''¨ã€‚
  \end{itemize}
\item
  æ''¯æŒ?å?Œè¯­æ˜¾ç¤ºå?‚考æ--‡çŒ®ï¼ˆè‡ªåŠ¨ä½¿ç''¨ \texttt{\ et\ al.\ }
  å'Œ \texttt{\ ç­‰\ } )

  \begin{itemize}
  \tightlist
  \item
    该功能�自 \texttt{\ bilingual-bibliography\ }
    ,å\ldots³è?''çš„æ--‡ä»¶æ˜¯
    \texttt{\ seu-thesis/utils/bilingual-bibliography.typ\ } 。
  \item
    有å\ldots³ \texttt{\ bilingual-bibliography\ }
    的更多信�,请查看
    \url{https://github.com/nju-lug/modern-nju-thesis/issues/3}
  \end{itemize}
\end{itemize}

\begin{quote}
{[}!NOTE{]}

本模æ?¿å†\ldots é€~çš„è½®å­?æ¯''较多,而ä¸''æˆ`的代ç~?è´¨é‡?一般,请é\ldots Œæƒ\ldots å?--ç''¨ã€‚
\end{quote}

\href{/app?template=cheda-seu-thesis&version=0.3.0}{Create project in
app}

\subsubsection{How to use}\label{how-to-use}

Click the button above to create a new project using this template in
the Typst app.

You can also use the Typst CLI to start a new project on your computer
using this command:

\begin{verbatim}
typst init @preview/cheda-seu-thesis:0.3.0
\end{verbatim}

\includesvg[width=0.16667in,height=0.16667in]{/assets/icons/16-copy.svg}

\subsubsection{About}\label{about}

\begin{description}
\tightlist
\item[Author :]
csimide
\item[License:]
MIT
\item[Current version:]
0.3.0
\item[Last updated:]
July 8, 2024
\item[First released:]
April 16, 2024
\item[Archive size:]
465 kB
\href{https://packages.typst.org/preview/cheda-seu-thesis-0.3.0.tar.gz}{\pandocbounded{\includesvg[keepaspectratio]{/assets/icons/16-download.svg}}}
\item[Repository:]
\href{https://github.com/csimide/SEU-Typst-Template}{GitHub}
\item[Categor y :]
\begin{itemize}
\tightlist
\item[]
\item
  \pandocbounded{\includesvg[keepaspectratio]{/assets/icons/16-mortarboard.svg}}
  \href{https://typst.app/universe/search/?category=thesis}{Thesis}
\end{itemize}
\end{description}

\subsubsection{Where to report issues?}\label{where-to-report-issues}

This template is a project of csimide . Report issues on
\href{https://github.com/csimide/SEU-Typst-Template}{their repository} .
You can also try to ask for help with this template on the
\href{https://forum.typst.app}{Forum} .

Please report this template to the Typst team using the
\href{https://typst.app/contact}{contact form} if you believe it is a
safety hazard or infringes upon your rights.

\phantomsection\label{versions}
\subsubsection{Version history}\label{version-history}

\begin{longtable}[]{@{}ll@{}}
\toprule\noalign{}
Version & Release Date \\
\midrule\noalign{}
\endhead
\bottomrule\noalign{}
\endlastfoot
0.3.0 & July 8, 2024 \\
\href{https://typst.app/universe/package/cheda-seu-thesis/0.2.2/}{0.2.2}
& May 23, 2024 \\
\href{https://typst.app/universe/package/cheda-seu-thesis/0.2.1/}{0.2.1}
& April 29, 2024 \\
\href{https://typst.app/universe/package/cheda-seu-thesis/0.2.0/}{0.2.0}
& April 16, 2024 \\
\end{longtable}

Typst GmbH did not create this template and cannot guarantee correct
functionality of this template or compatibility with any version of the
Typst compiler or app.


\section{Package List LaTeX/chronos.tex}
\title{typst.app/universe/package/chronos}

\phantomsection\label{banner}
\section{chronos}\label{chronos}

{ 0.2.0 }

A package to draw sequence diagrams with CeTZ

\phantomsection\label{readme}
A Typst package to draw sequence diagrams with CeTZ

\begin{center}\rule{0.5\linewidth}{0.5pt}\end{center}

This package lets you render sequence diagrams directly in Typst. The
following boilerplate code creates an empty sequence diagram with two
participants:

\begin{longtable}[]{@{}
  >{\raggedright\arraybackslash}p{(\linewidth - 2\tabcolsep) * \real{0.5000}}
  >{\raggedright\arraybackslash}p{(\linewidth - 2\tabcolsep) * \real{0.5000}}@{}}
\toprule\noalign{}
\endhead
\bottomrule\noalign{}
\endlastfoot
\textbf{Typst} & \textbf{Result} \\
\begin{minipage}[t]{\linewidth}\raggedright
\begin{Shaded}
\begin{Highlighting}[]
\NormalTok{\#import "@preview/chronos:0.2.0"}
\NormalTok{\#chronos.diagram(\{}
\NormalTok{  import chronos: *}
\NormalTok{  \_par("Alice")}
\NormalTok{  \_par("Bob")}
\NormalTok{\})}
\end{Highlighting}
\end{Shaded}
\end{minipage} &
\pandocbounded{\includegraphics[keepaspectratio]{https://github.com/typst/packages/raw/main/packages/preview/chronos/0.2.0/gallery/readme/boilerplate.png}} \\
\end{longtable}

\begin{quote}
\emph{Disclaimer}\\
The package cannot parse PlantUML syntax for the moment, and thus
requires the use of element functions, as shown in the examples. A
PlantUML parser is in the TODO list, just not the top priority
\end{quote}

\subsection{Basic sequences}\label{basic-sequences}

You can make basic sequences using the \texttt{\ \_seq\ } function:

\begin{longtable}[]{@{}
  >{\raggedright\arraybackslash}p{(\linewidth - 2\tabcolsep) * \real{0.5000}}
  >{\raggedright\arraybackslash}p{(\linewidth - 2\tabcolsep) * \real{0.5000}}@{}}
\toprule\noalign{}
\endhead
\bottomrule\noalign{}
\endlastfoot
\textbf{Typst} & \textbf{Result} \\
\begin{minipage}[t]{\linewidth}\raggedright
\begin{Shaded}
\begin{Highlighting}[]
\NormalTok{\#chronos.diagram(\{}
\NormalTok{  import chronos: *}
\NormalTok{  \_par("Alice")}
\NormalTok{  \_par("Bob")}

\NormalTok{  \_seq("Alice", "Bob", comment: "Hello")}
\NormalTok{  \_seq("Bob", "Bob", comment: "Think")}
\NormalTok{  \_seq("Bob", "Alice", comment: "Hi")}
\NormalTok{\})}
\end{Highlighting}
\end{Shaded}
\end{minipage} &
\pandocbounded{\includegraphics[keepaspectratio]{https://github.com/typst/packages/raw/main/packages/preview/chronos/0.2.0/gallery/readme/simple_sequence.png}} \\
\end{longtable}

You can make lifelines using the following parameters of the
\texttt{\ \_seq\ } function:

\begin{itemize}
\tightlist
\item
  \texttt{\ enable-dst\ } : enables the destination lifeline
\item
  \texttt{\ create-dst\ } : creates the destination lifeline and
  participant
\item
  \texttt{\ disable-dst\ } : disables the destination lifeline
\item
  \texttt{\ destroy-dst\ } : destroys the destination lifeline and
  participant
\item
  \texttt{\ disable-src\ } : disables the source lifeline
\item
  \texttt{\ destroy-src\ } : destroy the source lifeline and participant
\end{itemize}

\begin{longtable}[]{@{}
  >{\raggedright\arraybackslash}p{(\linewidth - 2\tabcolsep) * \real{0.5000}}
  >{\raggedright\arraybackslash}p{(\linewidth - 2\tabcolsep) * \real{0.5000}}@{}}
\toprule\noalign{}
\endhead
\bottomrule\noalign{}
\endlastfoot
\textbf{Typst} & \textbf{Result} \\
\begin{minipage}[t]{\linewidth}\raggedright
\begin{Shaded}
\begin{Highlighting}[]
\NormalTok{\#chronos.diagram(\{}
\NormalTok{  import chronos: *}
\NormalTok{  \_par("A", display{-}name: "Alice")}
\NormalTok{  \_par("B", display{-}name: "Bob")}
\NormalTok{  \_par("C", display{-}name: "Charlie")}
\NormalTok{  \_par("D", display{-}name: "Derek")}

\NormalTok{  \_seq("A", "B", comment: "hello", enable{-}dst: true)}
\NormalTok{  \_seq("B", "B", comment: "self call", enable{-}dst: true)}
\NormalTok{  \_seq("C", "B", comment: "hello from thread 2", enable{-}dst: true, lifeline{-}style: (fill: rgb("\#005500")))}
\NormalTok{  \_seq("B", "D", comment: "create", create{-}dst: true)}
\NormalTok{  \_seq("B", "C", comment: "done in thread 2", disable{-}src: true, dashed: true)}
\NormalTok{  \_seq("B", "B", comment: "rc", disable{-}src: true, dashed: true)}
\NormalTok{  \_seq("B", "D", comment: "delete", destroy{-}dst: true)}
\NormalTok{  \_seq("B", "A", comment: "success", disable{-}src: true, dashed: true)}
\NormalTok{\})}
\end{Highlighting}
\end{Shaded}
\end{minipage} &
\pandocbounded{\includegraphics[keepaspectratio]{https://github.com/typst/packages/raw/main/packages/preview/chronos/0.2.0/gallery/readme/lifelines.png}} \\
\end{longtable}

\subsection{Showcase}\label{showcase}

Several features have already been implemented in Chronos. Don’t
hesitate to checkout the examples in the
\href{https://github.com/typst/packages/raw/main/packages/preview/chronos/0.2.0/gallery}{gallery}
folder to see what you can do.

\paragraph{Quick example reference:}\label{quick-example-reference}

\begin{longtable}[]{@{}
  >{\raggedright\arraybackslash}p{(\linewidth - 2\tabcolsep) * \real{0.5000}}
  >{\raggedright\arraybackslash}p{(\linewidth - 2\tabcolsep) * \real{0.5000}}@{}}
\toprule\noalign{}
\endhead
\bottomrule\noalign{}
\endlastfoot
\textbf{Example} & \textbf{Features} \\
\begin{minipage}[t]{\linewidth}\raggedright
\texttt{\ example1\ }\strut \\
(
\href{https://github.com/typst/packages/raw/main/packages/preview/chronos/0.2.0/gallery/example1.pdf}{PDF}
\textbar{}
\href{https://github.com/typst/packages/raw/main/packages/preview/chronos/0.2.0/gallery/example1.typ}{Typst}
)\strut
\end{minipage} & Simple cases, color sequences, groups, separators,
gaps, self-sequences \\
\begin{minipage}[t]{\linewidth}\raggedright
\texttt{\ example2\ }\strut \\
(
\href{https://github.com/typst/packages/raw/main/packages/preview/chronos/0.2.0/gallery/example2.pdf}{PDF}
\textbar{}
\href{https://github.com/typst/packages/raw/main/packages/preview/chronos/0.2.0/gallery/example2.typ}{Typst}
)\strut
\end{minipage} & Lifelines, found/lost messages, synchronized sequences,
slanted sequences \\
\begin{minipage}[t]{\linewidth}\raggedright
\texttt{\ example3\ }\strut \\
(
\href{https://github.com/typst/packages/raw/main/packages/preview/chronos/0.2.0/gallery/example3.pdf}{PDF}
\textbar{}
\href{https://github.com/typst/packages/raw/main/packages/preview/chronos/0.2.0/gallery/example3.typ}{Typst}
)\strut
\end{minipage} & Participant shapes, sequence tips, hidden partipicant
ends \\
\begin{minipage}[t]{\linewidth}\raggedright
\texttt{\ notes\ }\strut \\
(
\href{https://github.com/typst/packages/raw/main/packages/preview/chronos/0.2.0/gallery/notes.pdf}{PDF}
\textbar{}
\href{https://github.com/typst/packages/raw/main/packages/preview/chronos/0.2.0/gallery/notes.typ}{Typst}
)\strut
\end{minipage} & Notes (duh), deferred participant creation \\
\end{longtable}

\begin{quote}
{[}!NOTE{]}

Many examples were taken/adapted from the PlantUML
\href{https://plantuml.com/sequence-diagram}{documentation} on sequence
diagrams
\end{quote}

\subsubsection{How to add}\label{how-to-add}

Copy this into your project and use the import as \texttt{\ chronos\ }

\begin{verbatim}
#import "@preview/chronos:0.2.0"
\end{verbatim}

\includesvg[width=0.16667in,height=0.16667in]{/assets/icons/16-copy.svg}

Check the docs for
\href{https://typst.app/docs/reference/scripting/\#packages}{more
information on how to import packages} .

\subsubsection{About}\label{about}

\begin{description}
\tightlist
\item[Author :]
\href{https://git.kb28.ch/HEL}{Louis Heredero}
\item[License:]
Apache-2.0
\item[Current version:]
0.2.0
\item[Last updated:]
November 12, 2024
\item[First released:]
October 1, 2024
\item[Minimum Typst version:]
0.12.0
\item[Archive size:]
537 kB
\href{https://packages.typst.org/preview/chronos-0.2.0.tar.gz}{\pandocbounded{\includesvg[keepaspectratio]{/assets/icons/16-download.svg}}}
\item[Repository:]
\href{https://git.kb28.ch/HEL/chronos}{git.kb28.ch}
\item[Categor y :]
\begin{itemize}
\tightlist
\item[]
\item
  \pandocbounded{\includesvg[keepaspectratio]{/assets/icons/16-chart.svg}}
  \href{https://typst.app/universe/search/?category=visualization}{Visualization}
\end{itemize}
\end{description}

\subsubsection{Where to report issues?}\label{where-to-report-issues}

This package is a project of Louis Heredero . Report issues on
\href{https://git.kb28.ch/HEL/chronos}{their repository} . You can also
try to ask for help with this package on the
\href{https://forum.typst.app}{Forum} .

Please report this package to the Typst team using the
\href{https://typst.app/contact}{contact form} if you believe it is a
safety hazard or infringes upon your rights.

\phantomsection\label{versions}
\subsubsection{Version history}\label{version-history}

\begin{longtable}[]{@{}ll@{}}
\toprule\noalign{}
Version & Release Date \\
\midrule\noalign{}
\endhead
\bottomrule\noalign{}
\endlastfoot
0.2.0 & November 12, 2024 \\
\href{https://typst.app/universe/package/chronos/0.1.0/}{0.1.0} &
October 1, 2024 \\
\end{longtable}

Typst GmbH did not create this package and cannot guarantee correct
functionality of this package or compatibility with any version of the
Typst compiler or app.


\section{Package List LaTeX/game-theoryst.tex}
\title{typst.app/universe/package/game-theoryst}

\phantomsection\label{banner}
\section{game-theoryst}\label{game-theoryst}

{ 0.1.0 }

A package for typesetting games in Typst.

{ } Featured Package

\phantomsection\label{readme}
A package for typesetting games in Typst.

Full manual available
\href{https://github.com/typst/packages/raw/main/packages/preview/game-theoryst/0.1.0/doc/gtheoryst-manual.pdf}{here}

Work in progress â€`` \emph{coming soon!}

\subsection{Overview}\label{overview}

\paragraph{Simple Example}\label{simple-example}

The main function to make strategic (or \textbf{normal} ) form games is
\texttt{\ nfg\ } . For a basic 2x2 game, you can do

\begin{Shaded}
\begin{Highlighting}[]
\NormalTok{\#nfg(}
\NormalTok{  players: ("Jack", "Diane"),}
\NormalTok{  s1: ($C$, $D$),}
\NormalTok{  s2: ($C$, $D$),}
\NormalTok{  [$10, 10$], [$2, 20$], }
\NormalTok{  [$20, 2$], [$5, 5$],}
\NormalTok{)}
\end{Highlighting}
\end{Shaded}

\includegraphics[width=4.16667in,height=\textheight,keepaspectratio]{https://github.com/typst/packages/raw/main/packages/preview/game-theoryst/0.1.0/doc/gallery/simple-example.png}

\subsubsection{Importing}\label{importing}

Simply insert the following into your Typst code:

\begin{Shaded}
\begin{Highlighting}[]
\NormalTok{\#import "@preview/game{-}theoryst:0.1.0": *}
\end{Highlighting}
\end{Shaded}

This imports the \texttt{\ nfg()\ } function as well as the underlining
methods. If you want to tweak the helper functions for generating an
\texttt{\ nfg\ } , import them explicitly through the
\texttt{\ utils/\ } directory.

\paragraph{Full Example}\label{full-example}

\begin{Shaded}
\begin{Highlighting}[]
\NormalTok{\#nfg(}
\NormalTok{  players: ([A\textbackslash{} Joe], [Bas Pro]),}
\NormalTok{  s1: ([$x$], [a]),}
\NormalTok{  s2: ("x", "aaaa", [$a$]),}
\NormalTok{  pad: ("x": 12pt, "y": 10pt),}
\NormalTok{  eliminations: ("s11", "s21", "s22"),}
\NormalTok{  ejust: (}
\NormalTok{    s11: (x: (0pt, 36pt), y: ({-}3pt, {-}3.5pt)),}
\NormalTok{    s22: (x: ({-}10pt, {-}12pt), y: ({-}10pt, 10pt)),}
\NormalTok{    s21: (x: ({-}3pt, {-}9pt), y: ({-}10pt, 10pt)),}
\NormalTok{  ),}
\NormalTok{  mixings: (hmix: ($p$, $1{-}p$), vmix: ($q$, $r$, $1{-}q{-}r$)),}
\NormalTok{  custom{-}fills: (hp: maroon, vp: navy, hm: purple, vm: fuchsia, he: gray, ve: gray),}
\NormalTok{  [$0,vul(100000000)$], [$0,1$], [$0,0$],}
\NormalTok{  [$hul(1),1$], [$0, {-}1$], table.cell(fill: yellow.lighten(30\%), [$hful(0),vful(0)$])}
\NormalTok{)}
\end{Highlighting}
\end{Shaded}

\includegraphics[width=5.46875in,height=\textheight,keepaspectratio]{https://github.com/typst/packages/raw/main/packages/preview/game-theoryst/0.1.0/doc/gallery/full-example.png}

\subsubsection{Color}\label{color}

By default, player names, mixed-strategy parameters (called
\emph{mixings} ), and elimination lines are shown in color. These colors
can be turned off at the method-level by passing \texttt{\ bw:\ true\ }
, or at the document level by running the state helper-function
\texttt{\ \#colorless()\ } .

\texttt{\ nfg\ } accepts custom colors for all of the aforementioned
parameters by passing a \texttt{\ dictionary\ } of colors to the
\texttt{\ custom-fills\ } arg. The keys for this dictionary are as
follows ( \texttt{\ \textless{}defaults\textgreater{}\ } ):

\begin{itemize}
\tightlist
\item
  \texttt{\ hp\ } â€`` “horizontal playerâ€? (red)
\item
  \texttt{\ vp\ } â€`` “vertical playerâ€? (blue)
\item
  \texttt{\ hm\ } â€`` “hor. mixingâ€? (\#e64173)
\item
  \texttt{\ vm\ } â€`` “ver. mixingâ€? (eastern)
\item
  \texttt{\ he\ } â€`` “hor. eliminationâ€? line (orange)
\item
  \texttt{\ ve\ } â€`` “ver. eliminationâ€? line (olive)
\end{itemize}

\subsection{Cell Customization}\label{cell-customization}

Since the payoffs are implemented as argument sinks (
\texttt{\ ..args\ } ) which are passed directly to Typst’s
\texttt{\ \#table()\ } , underlining of non-math can be accomplished via
the standard \texttt{\ \#underline()\ } command. Similarly, any of the
payoff cells can be customized by using \texttt{\ table.cell()\ }
directly. For instance,
\texttt{\ table.cell(fill:\ yellow.lighten(30\%),\ {[}\$1,\ 1\${]})\ }
can be used to highlight a specific cell.

\subsubsection{Padding}\label{padding}

There are edge cases where the default padding may be off. These can be
mended by passing the optional \texttt{\ pad\ } argument to
\texttt{\ nfg()\ } . This should represent how much
\textbf{\emph{additional}} padding you want. The \texttt{\ pad\ } arg.
is interpreted as follows:

\begin{itemize}
\tightlist
\item
  If a \texttt{\ length\ } is provided, it assumes you want that much
  length added to all cell walls
\item
  If an array of the form \texttt{\ (L1,\ L2)\ } is provided, it assumes
  you want padding a horizontal ( \texttt{\ x\ } ) padding of
  \texttt{\ L1\ } and a vertical padding ( \texttt{\ y\ } ) of
  \texttt{\ L2\ }
\item
  If a \texttt{\ dictionary\ } is provided, it operates identically to
  that of the array, but you must specify the \texttt{\ x\ } /
  \texttt{\ y\ } keys yourself
\end{itemize}

\subsubsection{AUtomatic Cell Sizing}\label{automatic-cell-sizing}

Cell are automatically sized to equal heights/widths according to the
longest/tallest content. If you want to avoid this behavior, pass
\texttt{\ lazy-cells:\ true\ } to \texttt{\ nfg\ } . This behavior can
be combined with the custom \texttt{\ padding\ } argument.

\subsection{Semantic Game Theory
Features}\label{semantic-game-theory-features}

\subsubsection{Underlining}\label{underlining}

The package imports a small set of underlining utility functions.

The primary functions for underlining are

\begin{itemize}
\tightlist
\item
  \texttt{\ hul()\ } â€`` \emph{Horizontal Underline}
\item
  \texttt{\ vul()\ } â€`` \emph{Vertical Underline}
\item
  \texttt{\ bul()\ } â€`` \emph{Black Underline} These can be wrapped
  around values in math-mode ( \texttt{\ \$..\$\ } ) within the payoff
  matrix. The underlines for \texttt{\ hul\ } and \texttt{\ vul\ } are
  colored by default according to the default colors for names, but they
  accept an optional \texttt{\ col\ } parameter for changing the color
  of the underline. \texttt{\ bul()\ } produces a black underline.
\end{itemize}

\begin{Shaded}
\begin{Highlighting}[]
\NormalTok{\#nfg(}
\NormalTok{  players: ("Jack", "Diane"),}
\NormalTok{  s2: ($x$, $y$, $z$),}
\NormalTok{  s1: ($a$, $b$),}
\NormalTok{  [$hul(0),vul(0)$], [$1,1$], [$2,2$],}
\NormalTok{  [$3,3$], [$4,4$], [$5,5$],}
\NormalTok{)}
\end{Highlighting}
\end{Shaded}

By default, these commands leave the numbers themselves black, but
boldfaces them. \emph{Full Color} versions of \texttt{\ hul\ } and
\texttt{\ vul\ } , which color the numbers and under-lines identically,
are available via \texttt{\ hful()\ } and \texttt{\ vful()\ } . Like
their counterparts, they accept an optional \texttt{\ col\ } command for
the color.

Both of the colors can be modified individually via the general
\texttt{\ cul()\ } command, which takes in content ( \texttt{\ cont\ }
), an underline color ( \texttt{\ ucol\ } ), and the color for the text
value ( \texttt{\ tcol\ } ). For instance,

\begin{Shaded}
\begin{Highlighting}[]
\NormalTok{\#let new{-}ul(cont, col: olive, tcol: fuchsia) = \{ cul(cont, col, tcol) \}}
\end{Highlighting}
\end{Shaded}

will define a new command which underlines in olive and sets the text
(math) color to fuchsia.

\subsubsection{Mixed Strategies}\label{mixed-strategies}

You can optionally mark mixed strategies that a player will in a
\texttt{\ nfg\ } using the \texttt{\ mixing\ } argument. This can be a
\texttt{\ dictionary\ } with \texttt{\ hmix\ } and \texttt{\ vmix\ }
keys, or an \texttt{\ array\ } , interpreted as a dictionary with the
aforementioned keys in the \texttt{\ (hmix,\ vmix)\ } order. The
values/entries here should be arrays which mimic \texttt{\ s1\ } and
\texttt{\ s2\ } in size, with some parameter denoting the proportion of
time the relevant player uses that strategy. If you would like to omit a
strategy from this markup, pass \texttt{\ {[}{]}\ } in it’s place.

For example:

\begin{Shaded}
\begin{Highlighting}[]
\NormalTok{\#nfg(}
\NormalTok{  players: ("Chet", "North"),}
\NormalTok{  s1: ([$F$], [$G$], [$H$]),}
\NormalTok{  s2: ([$X$], [$Y$]),}
\NormalTok{  mixings: (}
\NormalTok{    hmix: ($p$, $1{-}p$), }
\NormalTok{    vmix: ($q$, [], $1{-}q$)),}
\NormalTok{  [$7,3$], [$2,4$], }
\NormalTok{  [$5,2$], [$6,1$], }
\NormalTok{  [$6,1$], [$5,4$]}
\NormalTok{)}
\end{Highlighting}
\end{Shaded}

\includegraphics[width=4.16667in,height=\textheight,keepaspectratio]{https://github.com/typst/packages/raw/main/packages/preview/game-theoryst/0.1.0/doc/gallery/mix-ex.png}

\subsubsection{Iterated Deletion (Elimination) of Dominated
Strategies}\label{iterated-deletion-elimination-of-dominated-strategies}

You can use the \texttt{\ pinit\ } package to cross out lines,
semantically eliminating strategies. \texttt{\ pinit\ } comes
pre-imported with \texttt{\ game-theoryst\ } by default.

You can tell \texttt{\ nfg\ } which strategies to eliminate with the
\texttt{\ eliminations\ } argument and the corresponding
\texttt{\ ejust\ } helper-argument. The \texttt{\ eliminations\ }
argument is simply an \texttt{\ array\ } of \texttt{\ strings\ } of the
form
\texttt{\ "s\textless{}i\textgreater{}\textless{}j\textgreater{}"\ } ,
where \texttt{\ \textless{}i\textgreater{}\ } is the player â€`` 1 or 2
â€`` and \texttt{\ \textless{}j\textgreater{}\ } is player
\texttt{\ i\ } ’s \texttt{\ \textless{}j\textgreater{}\ } th strategy,
in left-to-right / top-to-bottom order \emph{starting at 1} . These
strategy strings represent the rows/columns which you want to eliminate.
For instance, \texttt{\ ("s12",\ "s21")\ } denotes an elimination of
player 1’s second strategy as well as player 2’s first strategy.

Due to \texttt{\ context\ } dependence, the lines typically need manual
adjustments, which can be done via the \texttt{\ ejust\ } arg.
\texttt{\ ejust\ } needs to be a dictionary with keys of matching those
strings present in \texttt{\ eliminations\ } ( \texttt{\ s11\ } ,
\texttt{\ s21\ } , etc.). The values of one of these dictionary entries
is itself a dictionary: one with \texttt{\ x\ } and \texttt{\ y\ } keys.
Each of these keys needs an array consisting of 2 lengths, corresponding
to the starting/ending \texttt{\ dx/dy\ } adjustments from
\texttt{\ pinit-line\ } .

For example, one such \texttt{\ ejust\ } argument could be
\texttt{\ ("s12":\ (x:\ (5pt,\ -5pt),\ y:\ (-10pt,\ 3pt)))\ } . This
says to adjust the “s12� elimination line by \texttt{\ 5pt\ } in the
x direction and \texttt{\ -10pt\ } in the y direction for the starting
(strategy-) side of the line, and adjust by \texttt{\ -5pt\ } in x and
\texttt{\ 3pt\ } in y on the ending (payoff-) side of the line.

\begin{Shaded}
\begin{Highlighting}[]
\NormalTok{\#let just{-}arr = (}
\NormalTok{    "s12": (x: (0pt, 10pt), y: ({-}3pt, {-}3pt)),}
\NormalTok{    "s13": (x: (0pt, 10pt), y: ({-}3pt, {-}3pt)),}
\NormalTok{    "s14": (x: (0pt, 10pt), y: ({-}3pt, {-}3pt)),}
\NormalTok{    "s21": (x: ({-}6pt, {-}8pt), y: (3pt, 8pt)),}
\NormalTok{    "s22": (x: ({-}4pt, {-}8pt), y: (3pt, 8pt)),}
\NormalTok{    "s23": (x: ({-}4pt, {-}8pt), y: (3pt, 8pt)),}
\NormalTok{)}

\NormalTok{\#nfg(}
\NormalTok{  players: ("A", "B"),}
\NormalTok{  s1: ([$N$], [$S$], [$E$], [$W$] ),}
\NormalTok{  s2: ([$W$], [$E$], [$F$], [$A$]),}
\NormalTok{  eliminations: ("s12", "s13", "s14", "s21", "s22", "s23"),}
\NormalTok{  ejust: just{-}arr,}
\NormalTok{  [$6,4$], [$7,3$], [$5,5$], [$6,6$],}
\NormalTok{  [$7,3$], [$2,7$], [$4,6$], [$5,5$],}
\NormalTok{  [$8,2$], [$6,4$], [$3,7$], [$2,8$],}
\NormalTok{  [$3,7$], [$5,5$], [$4,6$], [$5,5$],}
\NormalTok{)}
\end{Highlighting}
\end{Shaded}

\includegraphics[width=4.16667in,height=\textheight,keepaspectratio]{https://github.com/typst/packages/raw/main/packages/preview/game-theoryst/0.1.0/doc/gallery/elim-ex.png}

\subsubsection{Debugging}\label{debugging}

If you want to see all of the lines for the table, including the ones
for a players, strategies, and mixings, set the following at the top of
your document.

\begin{Shaded}
\begin{Highlighting}[]
\NormalTok{\#set table.cell(stroke: (thickness: auto))}
\end{Highlighting}
\end{Shaded}

Note that cells are always present for mixings, they just have 0
width/height when no mixings of a specific variety are provided.

\subsection{License}\label{license}

game-theoryst Copyright © 2024 Connor T. Wiegand

This program is free software: you can redistribute it and/or modify it
under the terms of the GNU Affero General Public License as published by
the Free Software Foundation, either version 3 of the License, or (at
your option) any later version.

This program is distributed in the hope that it will be useful, but
WITHOUT ANY WARRANTY; without even the implied warranty of
MERCHANTABILITY or FITNESS FOR A PARTICULAR PURPOSE. See the GNU Affero
General Public License for more details.

You should have received a copy of the GNU Affero General Public License
along with this program. If not, see
\href{http://www.gnu.org/licenses/}{http:www.gnu.org/licenses/} .

\subsubsection{How to add}\label{how-to-add}

Copy this into your project and use the import as
\texttt{\ game-theoryst\ }

\begin{verbatim}
#import "@preview/game-theoryst:0.1.0"
\end{verbatim}

\includesvg[width=0.16667in,height=0.16667in]{/assets/icons/16-copy.svg}

Check the docs for
\href{https://typst.app/docs/reference/scripting/\#packages}{more
information on how to import packages} .

\subsubsection{About}\label{about}

\begin{description}
\tightlist
\item[Author :]
\href{https://github.com/connortwiegand}{Connor T. Wiegand}
\item[License:]
AGPL-3.0-only
\item[Current version:]
0.1.0
\item[Last updated:]
August 14, 2024
\item[First released:]
August 14, 2024
\item[Archive size:]
18.9 kB
\href{https://packages.typst.org/preview/game-theoryst-0.1.0.tar.gz}{\pandocbounded{\includesvg[keepaspectratio]{/assets/icons/16-download.svg}}}
\item[Repository:]
\href{https://github.com/connortwiegand/game-theoryst}{GitHub}
\item[Discipline s :]
\begin{itemize}
\tightlist
\item[]
\item
  \href{https://typst.app/universe/search/?discipline=economics}{Economics}
\item
  \href{https://typst.app/universe/search/?discipline=education}{Education}
\item
  \href{https://typst.app/universe/search/?discipline=mathematics}{Mathematics}
\end{itemize}
\end{description}

\subsubsection{Where to report issues?}\label{where-to-report-issues}

This package is a project of Connor T. Wiegand . Report issues on
\href{https://github.com/connortwiegand/game-theoryst}{their repository}
. You can also try to ask for help with this package on the
\href{https://forum.typst.app}{Forum} .

Please report this package to the Typst team using the
\href{https://typst.app/contact}{contact form} if you believe it is a
safety hazard or infringes upon your rights.

\phantomsection\label{versions}
\subsubsection{Version history}\label{version-history}

\begin{longtable}[]{@{}ll@{}}
\toprule\noalign{}
Version & Release Date \\
\midrule\noalign{}
\endhead
\bottomrule\noalign{}
\endlastfoot
0.1.0 & August 14, 2024 \\
\end{longtable}

Typst GmbH did not create this package and cannot guarantee correct
functionality of this package or compatibility with any version of the
Typst compiler or app.


\section{Package List LaTeX/chicv.tex}
\title{typst.app/universe/package/chicv}

\phantomsection\label{banner}
\phantomsection\label{template-thumbnail}
\pandocbounded{\includegraphics[keepaspectratio]{https://packages.typst.org/preview/thumbnails/chicv-0.1.0-small.webp}}

\section{chicv}\label{chicv}

{ 0.1.0 }

A minimal and fully-customizable CV template.

\href{/app?template=chicv&version=0.1.0}{Create project in app}

\phantomsection\label{readme}
A simple CV template for \href{https://typst.app/}{typst.app} .

\subsection{How To Use}\label{how-to-use}

\subsubsection{Quick Start}\label{quick-start}

Create a project on \href{https://typst.app/}{typst.app} , copy paste
everything in
\url{https://github.com/skyzh/chicv/blob/master/template/cv.typ} . All
done!

\subsubsection{Customize your CV}\label{customize-your-cv}

To change the text size, you can uncomment the lines in
\texttt{\ cv.typ\ } and set to your choice. (Recommended text size for
CV is from 10pt to 12pt)

You can also change the page margin in \texttt{\ cv.typ\ } to fit in
more contents in a single page. The margin default is set to
\texttt{\ (x:\ 0.9cm,\ y:\ 1.3cm)\ } .

Don’t forget to include \texttt{\ \#chiline()\ } every time you open a
new section, this line acts as a perfect split.

For basic typst syntax, check this template as a reference, it’s super
easy to understand and use!

For advanced topics, please refer to
\href{https://typst.app/docs/reference/}{official reference} by typst.

\subsection{Showcases}\label{showcases}

\subsubsection{Sample CV}\label{sample-cv}

\pandocbounded{\includegraphics[keepaspectratio]{https://github.com/typst/packages/raw/main/packages/preview/chicv/0.1.0/cv.png}}

\subsubsection{Chi’s CV}\label{chiuxe2s-cv}

\href{https://skyzh.github.io/files/cv.pdf}{cv.pdf}

\href{/app?template=chicv&version=0.1.0}{Create project in app}

\subsubsection{How to use}\label{how-to-use-1}

Click the button above to create a new project using this template in
the Typst app.

You can also use the Typst CLI to start a new project on your computer
using this command:

\begin{verbatim}
typst init @preview/chicv:0.1.0
\end{verbatim}

\includesvg[width=0.16667in,height=0.16667in]{/assets/icons/16-copy.svg}

\subsubsection{About}\label{about}

\begin{description}
\tightlist
\item[Author :]
@skyzh
\item[License:]
MIT
\item[Current version:]
0.1.0
\item[Last updated:]
June 5, 2024
\item[First released:]
June 5, 2024
\item[Archive size:]
4.79 kB
\href{https://packages.typst.org/preview/chicv-0.1.0.tar.gz}{\pandocbounded{\includesvg[keepaspectratio]{/assets/icons/16-download.svg}}}
\item[Repository:]
\href{https://github.com/skyzh/chicv}{GitHub}
\item[Categor y :]
\begin{itemize}
\tightlist
\item[]
\item
  \pandocbounded{\includesvg[keepaspectratio]{/assets/icons/16-user.svg}}
  \href{https://typst.app/universe/search/?category=cv}{CV}
\end{itemize}
\end{description}

\subsubsection{Where to report issues?}\label{where-to-report-issues}

This template is a project of @skyzh . Report issues on
\href{https://github.com/skyzh/chicv}{their repository} . You can also
try to ask for help with this template on the
\href{https://forum.typst.app}{Forum} .

Please report this template to the Typst team using the
\href{https://typst.app/contact}{contact form} if you believe it is a
safety hazard or infringes upon your rights.

\phantomsection\label{versions}
\subsubsection{Version history}\label{version-history}

\begin{longtable}[]{@{}ll@{}}
\toprule\noalign{}
Version & Release Date \\
\midrule\noalign{}
\endhead
\bottomrule\noalign{}
\endlastfoot
0.1.0 & June 5, 2024 \\
\end{longtable}

Typst GmbH did not create this template and cannot guarantee correct
functionality of this template or compatibility with any version of the
Typst compiler or app.


\section{Package List LaTeX/minienvs.tex}
\title{typst.app/universe/package/minienvs}

\phantomsection\label{banner}
\section{minienvs}\label{minienvs}

{ 0.1.0 }

Theorem environments with minimal fuss

\phantomsection\label{readme}
Theorem environments in \href{https://typst.app/}{Typst} with minimal
fuss.

To use, import and add a show rule:

\begin{Shaded}
\begin{Highlighting}[]
\NormalTok{\#import "@preview/minienvs:0.1.0": *}
\NormalTok{\#show: minienvs}
\end{Highlighting}
\end{Shaded}

You can optionally pass a custom configuration in the show-rule via
\texttt{\ minienvs.with(…)\ } (see
\href{https://github.com/typst/packages/raw/main/packages/preview/minienvs/0.1.0/\#customization}{Customization}
).

You can now just add a theorem along with its proof using the term list
syntax. For example:

\begin{Shaded}
\begin{Highlighting}[]
\NormalTok{/ Theorem (Ville\textquotesingle{}s inequality):}
\NormalTok{  Let $X\_0, ...$ be a non{-}negative supermartingale. Then, for any real number $a \textgreater{} 0$,}

\NormalTok{  $ PP[sup\_(n\textgreater{}=0) X\_n \textgreater{}= a] \textless{}= EE[X\_0]/a. $}

\NormalTok{Let us now prove it:}

\NormalTok{/ Proof:}
\NormalTok{  Consider the stopping time $N = inf \{t \textgreater{}= 1 : X\_t \textgreater{}= a\}$.}
\NormalTok{  By the optional stopping theorem and the supermartingale convergence theorem, we have that}

\NormalTok{  $}
\NormalTok{    EE[X\_0] \textgreater{}= EE[X\_N]}
\NormalTok{    \&= EE[X\_N | N \textless{} oo] PP[N \textless{} oo] + EE[X\_oo | N = oo] PP[N = oo] \textbackslash{}}
\NormalTok{    \&\textgreater{}= EE[X\_N | N \textless{} oo] PP[N \textless{} oo]}
\NormalTok{    = EE[X\_N/a | N \textless{} oo] a PP[N \textless{} oo]. \textbackslash{}}
\NormalTok{  $}

\NormalTok{  And, therefore,}

\NormalTok{  $ PP[N \textless{} oo] \textless{}= EE[X\_0] \textbackslash{}/ a EE[X\_N/a | N \textless{} oo] \textless{}= EE[X\_0] \textbackslash{}/ a. $}
\end{Highlighting}
\end{Shaded}

\pandocbounded{\includegraphics[keepaspectratio]{https://github.com/typst/packages/raw/main/packages/preview/minienvs/0.1.0/assets/ville.png}}

\subsection{Labels and references}\label{labels-and-references}

Currently, in order to label a minienv one needs to use the
\texttt{\ envlabel\ } function. For example:

\begin{Shaded}
\begin{Highlighting}[]
\NormalTok{/ Lemma (Donsker and Varadhan\textquotesingle{}s variational formula) \#envlabel(\textless{}change{-}of{-}measure\textgreater{}):}
\NormalTok{  For any measureable, bounded function $h : Theta {-}\textgreater{} RR$ we have:}

\NormalTok{  $ log EE\_(theta \textasciitilde{} pi)[exp h(theta)] = sup\_(rho in cal(P)(Theta)) [ EE\_(theta\textasciitilde{}rho)[h(theta)] {-} KL(rho || pi) ]. $}

\NormalTok{As we will see, @change{-}of{-}measure is a fundamental building block of PAC{-}Bayes bounds.}
\end{Highlighting}
\end{Shaded}

\pandocbounded{\includegraphics[keepaspectratio]{https://github.com/typst/packages/raw/main/packages/preview/minienvs/0.1.0/assets/donsker-varadhan.png}}

\subsection{Customization}\label{customization}

You can customize the appearance of minienvs by providing a
configuration to the show-rule. For example, for the default
configuration, you can do:

\begin{Shaded}
\begin{Highlighting}[]
\NormalTok{\#show: minienvs.with(config: (}
\NormalTok{  // Whether to give numbers for environments.}
\NormalTok{  // If the environment is not mentioned in this dict, it has a number.}
\NormalTok{  no{-}numbering: (}
\NormalTok{    proof: true,}
\NormalTok{  ),}
\NormalTok{  // Additional options for the \textasciigrave{}block\textasciigrave{} containing the minienv (e.g., to put a box around the minienv).}
\NormalTok{  // If the environment is not mentioned in this dict, no additional options are passed.}
\NormalTok{  bbox: (:),}
\NormalTok{  // How to format the head of the minienv.}
\NormalTok{  // If the environment is not mentioned in this dict, then it is formatted in bold.}
\NormalTok{  head{-}style: (}
\NormalTok{    proof: it =\textgreater{} [\_\#\{it\}\_],}
\NormalTok{  ),}
\NormalTok{  // How to format the body of the minienv.}
\NormalTok{  // If the environment is not mentioned in this dict, then it is formatted in italic.}
\NormalTok{  transforms: (}
\NormalTok{    proof: it =\textgreater{} [\#it \#h(1fr) $space qed$],}
\NormalTok{  )}
\NormalTok{))}
\end{Highlighting}
\end{Shaded}

\subsection{Coming soon / Work in
progress}\label{coming-soon-work-in-progress}

\begin{itemize}
\tightlist
\item
  Presets for multiple languages
\item
  Separate counters
\item
  More customization
\end{itemize}

\subsubsection{How to add}\label{how-to-add}

Copy this into your project and use the import as \texttt{\ minienvs\ }

\begin{verbatim}
#import "@preview/minienvs:0.1.0"
\end{verbatim}

\includesvg[width=0.16667in,height=0.16667in]{/assets/icons/16-copy.svg}

Check the docs for
\href{https://typst.app/docs/reference/scripting/\#packages}{more
information on how to import packages} .

\subsubsection{About}\label{about}

\begin{description}
\tightlist
\item[Author :]
Daniel Csillag
\item[License:]
MIT
\item[Current version:]
0.1.0
\item[Last updated:]
December 14, 2023
\item[First released:]
December 14, 2023
\item[Archive size:]
3.13 kB
\href{https://packages.typst.org/preview/minienvs-0.1.0.tar.gz}{\pandocbounded{\includesvg[keepaspectratio]{/assets/icons/16-download.svg}}}
\end{description}

\subsubsection{Where to report issues?}\label{where-to-report-issues}

This package is a project of Daniel Csillag . You can also try to ask
for help with this package on the \href{https://forum.typst.app}{Forum}
.

Please report this package to the Typst team using the
\href{https://typst.app/contact}{contact form} if you believe it is a
safety hazard or infringes upon your rights.

\phantomsection\label{versions}
\subsubsection{Version history}\label{version-history}

\begin{longtable}[]{@{}ll@{}}
\toprule\noalign{}
Version & Release Date \\
\midrule\noalign{}
\endhead
\bottomrule\noalign{}
\endlastfoot
0.1.0 & December 14, 2023 \\
\end{longtable}

Typst GmbH did not create this package and cannot guarantee correct
functionality of this package or compatibility with any version of the
Typst compiler or app.


\section{Package List LaTeX/grape-suite.tex}
\title{typst.app/universe/package/grape-suite}

\phantomsection\label{banner}
\phantomsection\label{template-thumbnail}
\pandocbounded{\includegraphics[keepaspectratio]{https://packages.typst.org/preview/thumbnails/grape-suite-1.0.0-small.webp}}

\section{grape-suite}\label{grape-suite}

{ 1.0.0 }

Library of templates for exams, seminar papers, homeworks, etc.

{ } Featured Template

\href{/app?template=grape-suite&version=1.0.0}{Create project in app}

\phantomsection\label{readme}
The grape suite is a suite consisting of following templates:

\begin{itemize}
\item
  exercises (for exams, homework, etc.)
\item
  seminar papers
\item
  slides (using polylux)
\end{itemize}

\subsection{Exercises}\label{exercises}

\subsubsection{Setup}\label{setup}

\begin{Shaded}
\begin{Highlighting}[]
\NormalTok{\#import "@preview/grape{-}suite:1.0.0": exercise}
\NormalTok{\#import exercise: project, task, subtask}

\NormalTok{\#show: project.with(}
\NormalTok{    title: "Lorem ipsum dolor sit",}

\NormalTok{    university: [University],}
\NormalTok{    institute: [Institute],}
\NormalTok{    seminar: [Seminar],}

\NormalTok{    abstract: lorem(100),}
\NormalTok{    show{-}outline: true,}

\NormalTok{    author: "John Doe",}

\NormalTok{    show{-}solutions: false}
\NormalTok{)}
\end{Highlighting}
\end{Shaded}

\subsubsection{API-Documentation}\label{api-documentation}

\begin{longtable}[]{@{}ll@{}}
\toprule\noalign{}
\texttt{\ project\ } & \\
\midrule\noalign{}
\endhead
\bottomrule\noalign{}
\endlastfoot
\texttt{\ no\ } & optional, number, default: \texttt{\ none\ } , number
of the sheet in the series \\
\texttt{\ type\ } & optional, content, default: \texttt{\ {[}Exam{]}\ }
, type of the series, eg. exam, homework, protocol, … \\
\texttt{\ title\ } & optional, content, default: \texttt{\ none\ } ,
title of the document: if none, then generated from no, type and
suffix-title \\
\texttt{\ suffix-title\ } & optional, content, default:
\texttt{\ none\ } , used if title is none to generate the title of the
document \\
\texttt{\ show-outline\ } & optional, bool, default: \texttt{\ false\ }
, show outline after title iff true \\
\texttt{\ abstract\ } & optional, content, default: \texttt{\ none\ } ,
show abstract between outline and title \\
\texttt{\ document-title\ } & optional, content, default:
\texttt{\ none\ } , shown in the upper right corner of the page header:
if none, \texttt{\ title\ } is used \\
\texttt{\ show-hints\ } & optional, bool, default: \texttt{\ false\ } ,
generate hints from tasks iff true \\
\texttt{\ show-solutions\ } & optional, bool, default:
\texttt{\ false\ } , generate solutions from tasks iff true \\
\texttt{\ show-namefield\ } & optional, bool, default:
\texttt{\ false\ } , show namefield at the end of the left header iff
true \\
\texttt{\ namefield\ } & optional, content, default:
\texttt{\ {[}Name:{]}\ } , content shown iff
\texttt{\ show-namefield\ } \\
\texttt{\ show-timefield\ } & optional, bool, default:
\texttt{\ false\ } , show timefield at the end of right header iff
true \\
\texttt{\ timefield\ } & optional, function, default:
\texttt{\ (time)\ =\textgreater{}\ {[}Time:\ \#time\ min.{]}\ } , to
generate the content shown as the timefield iff
\texttt{\ show-timefield\ } is true \\
\texttt{\ max-time\ } & optional, number, default: \texttt{\ 0\ } , time
value used in the \texttt{\ timefield\ } function generateor \\
\texttt{\ show-lines\ } & optional, bool, default: \texttt{\ false\ } ,
draw automatic lines for each task, if \texttt{\ lines\ } parameter of
\texttt{\ task\ } is set \\
\texttt{\ show-point-distribution-in-tasks\ } & optional, bool, default:
\texttt{\ false\ } , show point distribution after tasks iff true \\
\texttt{\ show-point-distribution-in-solutions\ } & optional, bool,
default: \texttt{\ false\ } , show point distributions after solutions
iff true \\
\texttt{\ solutions-as-matrix\ } & optional, bool, default:
\texttt{\ false\ } , show solutions as a matrix iff true, \textbf{mind
that} : now the solution parameter of task expects a list of 2-tuples,
where the first element of the 2-tuple is the amount of points, a number
and the second element is content, how to achieve all points \\
\texttt{\ university\ } & optional, content, default:
\texttt{\ none\ } \\
\texttt{\ faculty\ } & optional, content, default: \texttt{\ none\ } \\
\texttt{\ institute\ } & optional, content, default:
\texttt{\ none\ } \\
\texttt{\ seminar\ } & optional, content, default: \texttt{\ none\ } \\
\texttt{\ semester\ } & optional, content, default: \texttt{\ none\ } \\
\texttt{\ docent\ } & optional, content, default: \texttt{\ none\ } \\
\texttt{\ author\ } & optional, content, default: \texttt{\ none\ } \\
\texttt{\ date\ } & optional, datetime, default:
\texttt{\ datetime.today()\ } \\
\texttt{\ header\ } & optional, content, default: \texttt{\ none\ } ,
overwrite page header \\
\texttt{\ header-right\ } & optional, content, default:
\texttt{\ none\ } , overwrite right header part \\
\texttt{\ header-middle\ } & optional, content, default:
\texttt{\ none\ } , overwrite middle header part \\
\texttt{\ header-left\ } & optional, content, default: \texttt{\ none\ }
, overwrite left header part \\
\texttt{\ footer\ } & optional, content, default: \texttt{\ none\ } ,
overwrite footer part \\
\texttt{\ footer-right\ } & optional, content, default:
\texttt{\ none\ } , overwrite right footer part \\
\texttt{\ footer-middle\ } & optional, content, default:
\texttt{\ none\ } , overwrite middle footer part \\
\texttt{\ footer-left\ } & optional, content, default: \texttt{\ none\ }
, overwrite left footer part \\
\texttt{\ task-type\ } & optional, content, default:
\texttt{\ {[}Task{]}\ } , content shown in task title box before
numbering \\
\texttt{\ extra-task-type\ } & optional, content, default:
\texttt{\ {[}Extra\ task{]}\ } , for tasks where the \texttt{\ extra\ }
parameter is true, content shown in title box before numbering \\
\texttt{\ box-task-title\ } & optional, content, default:
\texttt{\ {[}Task{]}\ } , shown as the title of a task box used by the
\texttt{\ slides\ } library \\
\texttt{\ box-hint-title\ } & optional, content, default:
\texttt{\ {[}Hint{]}\ } , shown as the title of a tasks colored hint
box \\
\texttt{\ box-solution-title\ } & optional, content, default:
\texttt{\ {[}Solution{]}\ } , shown as the title of a tasks colored
solution box \\
\texttt{\ box-definition-title\ } & optional, content, default:
\texttt{\ {[}Definition{]}\ } , shown as the title of a definition box
used by the \texttt{\ slides\ } library \\
\texttt{\ box-notice-title\ } & optional, content, default:
\texttt{\ {[}Notice{]}\ } , shown as the title of a notice box used by
the \texttt{\ slides\ } library \\
\texttt{\ box-example-title\ } & optional, content, default:
\texttt{\ {[}Example{]}\ } , shown as the title of a example box used by
the \texttt{\ slides\ } library \\
\texttt{\ hint-type\ } & optional, content, default:
\texttt{\ {[}Hint{]}\ } , title of a tasks hint version \\
\texttt{\ hints-title\ } & optional, content, default:
\texttt{\ {[}Hints{]}\ } , title of the hints section \\
\texttt{\ solution-type\ } & optional, content, default:
\texttt{\ {[}Suggested\ solution{]}\ } , title of a tasks solution
version \\
\texttt{\ solutions-title\ } & optional, content, default:
\texttt{\ {[}Suggested\ solutions{]}\ } , title of the solutions
section \\
\texttt{\ solution-matrix-task-header\ } & optional, content, default:
\texttt{\ {[}Tasks{]}\ } , first column header of solution matrix,
column contains the reasons on how to achieve the points \\
\texttt{\ solution-matrix-achieved-points-header\ } & optional, content,
default: \texttt{\ {[}Points\ achieved{]}\ } , second column header of
solution matrix, column contains the points the one achieved \\
\texttt{\ show-solution-matrix-comment-field\ } & optional, bool,
default: \texttt{\ false\ } , show comment field in solution matrix \\
\texttt{\ solution-matrix-comment-field-value\ } & optional, content,
default: \texttt{\ {[}*Note:*\ \#v(0.5cm){]}\ } , value of solution
matrix comment fields \\
\texttt{\ distribution-header-point-value\ } & optional, content,
default: \texttt{\ {[}Point{]}\ } , first row of point distribution,
used to indicate the points needed to get a specific grade \\
\texttt{\ distribution-header-point-grade\ } & optional, content,
default: \texttt{\ {[}Grade{]}\ } , second row of point distribution \\
\texttt{\ message\ } & optional, function, default:
\texttt{\ (points-sum,\ extrapoints-sum)\ =\textgreater{}\ {[}In\ sum\ \#points-sum\ +\ \#extrapoints-sum\ P.\ are\ achievable.\ You\ achieved\ \#box(line(stroke:\ purple,\ length:\ 1cm))\ out\ of\ \#points-sum\ points.{]}\ }
, used to generate the message part above the point distribution \\
\texttt{\ grade-scale\ } & optional, array, default:
\texttt{\ (({[}excellent{]},\ 0.9),\ ({[}very\ good{]},\ 0.8),\ ({[}good{]},\ 0.7),\ ({[}pass{]},\ 0.6),\ ({[}fail{]},\ 0.49))\ }
, list of grades and percentage of points to reach that grade \\
\texttt{\ page-margins\ } & optional, margins, default:
\texttt{\ none\ } , overwrite page margins \\
\texttt{\ fontsize\ } & optional, size, default: \texttt{\ 11pt\ } ,
overwrite font size \\
\texttt{\ show-todolist\ } & optional, bool, default: \texttt{\ true\ }
, show list of usages of the \texttt{\ todo\ } function after the
outline \\
\texttt{\ body\ } & content, document content \\
\end{longtable}

\texttt{\ task\ } creates a task element in an exercise project.

\begin{longtable}[]{@{}ll@{}}
\toprule\noalign{}
\texttt{\ task\ } & \\
\midrule\noalign{}
\endhead
\bottomrule\noalign{}
\endlastfoot
\texttt{\ lines\ } & optional, number, default: \texttt{\ 0\ } , number
of lines to draw if \texttt{\ show-lines\ } in exercise’s
\texttt{\ project\ } is set to \texttt{\ true\ } \\
\texttt{\ points\ } & optional, number, default: \texttt{\ 0\ } , number
of points achievable \\
\texttt{\ extra\ } & optional, bool, default: \texttt{\ false\ } ,
determines if the task is obligatory ( \texttt{\ false\ } ) or
additional ( \texttt{\ true\ } ) \\
\texttt{\ numbering-format\ } & optional, function, default:
\texttt{\ none\ } , \\
\texttt{\ title\ } & content, title of the task \\
\texttt{\ instruction\ } & content, instruction of the task,
highlighted \\
\texttt{\ ..args\ } & 1: content, task body; 2: content, task solution,
not highlighted (see \texttt{\ solution-as-matrix\ } of exercise’s
\texttt{\ project\ } ), 3: content, task hint \\
\end{longtable}

\texttt{\ subtask\ } creates a part of a task. Its points are added to
the parent task. \emph{\textbf{Subtasks are to be use inside of the
task’s body or inside of another subtask’s body.}}

\begin{longtable}[]{@{}ll@{}}
\toprule\noalign{}
\texttt{\ subtask\ } & \\
\midrule\noalign{}
\endhead
\bottomrule\noalign{}
\endlastfoot
\texttt{\ points\ } & optional, number, default: \texttt{\ 0\ } , points
achievable, adds to a tasks point \\
\texttt{\ tight\ } & optional, bool, default: \texttt{\ false\ } , enum
style \\
\texttt{\ markers\ } & optional, array, default:
\texttt{\ ("1.",\ "a)")\ } , numbering format for each level, fallback
is \texttt{\ i.\ } \\
\texttt{\ show-points\ } & optional, bool, default: \texttt{\ true\ } ,
show points next to subtask’s body iff \texttt{\ true\ } \\
\texttt{\ counter\ } & optional, counter, default: \texttt{\ none\ } ,
change number styled by the numbering format; if \texttt{\ none\ } ,
each level has an incrementel auto counter \\
\texttt{\ content\ } & content, subtask body \\
\end{longtable}

\subsection{Seminar paper}\label{seminar-paper}

\subsubsection{Setup}\label{setup-1}

\begin{Shaded}
\begin{Highlighting}[]
\NormalTok{\#import "@preview/grape{-}suite:1.0.0": seminar{-}paper}

\NormalTok{\#show: seminar{-}paper.project.with(}
\NormalTok{    title: "Die Intensionalität von dass{-}Sätzen",}
\NormalTok{    subtitle: "Intensionale Kontexte in philosophischen Argumenten",}

\NormalTok{    university: [Universität Musterstadt],}
\NormalTok{    faculty: [Exemplarische Fakultät],}
\NormalTok{    institute: [Institut für Philosophie],}
\NormalTok{    docent: [Dr. phil. Berta Beispielprüferin],}
\NormalTok{    seminar: [Beispielseminar],}

\NormalTok{    submit{-}to: [Eingereicht bei],}
\NormalTok{    submit{-}by: [Eingereicht durch],}

\NormalTok{    semester: german{-}dates.semester(datetime.today()),}

\NormalTok{    author: "Max Muster",}
\NormalTok{    email: "max.muster@uni{-}musterstadt.uni",}
\NormalTok{    address: [}
\NormalTok{        12345 Musterstadt \textbackslash{}}
\NormalTok{        Musterstraße 67}
\NormalTok{    ]}
\NormalTok{)}
\end{Highlighting}
\end{Shaded}

\subsubsection{Documentation}\label{documentation}

\begin{longtable}[]{@{}ll@{}}
\toprule\noalign{}
\texttt{\ project\ } & \\
\midrule\noalign{}
\endhead
\bottomrule\noalign{}
\endlastfoot
\texttt{\ title\ } & optional, content, default: \texttt{\ none\ } ,
title used on the title page \\
\texttt{\ subtitle\ } & optional, content, default: \texttt{\ none\ } ,
subtitle used on title page \\
\texttt{\ submit-to\ } & optional, content, default:
\texttt{\ "Submitted\ to"\ } , title for the assignees’s section \\
\texttt{\ submit-by\ } & optional, content, default:
\texttt{\ "Submitted\ by"\ } , title for the assigned’s section \\
\texttt{\ university\ } & optional, content, default:
\texttt{\ "UNIVERSITY"\ } \\
\texttt{\ faculty\ } & optional, content, default:
\texttt{\ "FACULTY"\ } \\
\texttt{\ institute\ } & optional, content, default:
\texttt{\ "INSTITUTE"\ } \\
\texttt{\ seminar\ } & optional, content, default:
\texttt{\ "SEMINAR"\ } \\
\texttt{\ semester\ } & optional, content, default:
\texttt{\ "SEMESTER"\ } \\
\texttt{\ docent\ } & optional, content, default:
\texttt{\ "DOCENT"\ } \\
\texttt{\ author\ } & optional, content, default:
\texttt{\ "AUTHOR"\ } \\
\texttt{\ email\ } & optional, content, default: \texttt{\ "EMAIL"\ } \\
\texttt{\ address\ } & optional, content, default:
\texttt{\ "ADDRESS"\ } \\
\texttt{\ title-page-part\ } & optional, content, default:
\texttt{\ none\ } , overwrite date, assignee and assigned section \\
\texttt{\ title-page-part-submit-date\ } & optional, content, default:
\texttt{\ none\ } , overwrite date section \\
\texttt{\ title-page-part-submit-to\ } & optional, content, default:
\texttt{\ none\ } , overwrite assignee section \\
\texttt{\ title-page-part-submit-by\ } & optional, content, default:
\texttt{\ none\ } , overwrite assigned section \\
\texttt{\ date\ } & optional, datetime, default:
\texttt{\ datetime.today()\ } \\
\texttt{\ date-format\ } & optional, function, default:
\texttt{\ (date)\ =\textgreater{}\ date.display("{[}day{]}.{[}month{]}.{[}year{]}")\ } \\
\texttt{\ header\ } & optional, content, default: \texttt{\ none\ } ,
overwrite page header \\
\texttt{\ header-right\ } & optional, content, default:
\texttt{\ none\ } , overwrite right header part \\
\texttt{\ header-middle\ } & optional, content, default:
\texttt{\ none\ } , overwrite middle header part \\
\texttt{\ header-left\ } & optional, content, default: \texttt{\ none\ }
, overwrite left header part \\
\texttt{\ footer\ } & optional, content, default: \texttt{\ none\ } ,
overwrite footer part \\
\texttt{\ footer-right\ } & optional, content, default:
\texttt{\ none\ } , overwrite right footer part \\
\texttt{\ footer-middle\ } & optional, content, default:
\texttt{\ none\ } , overwrite middle footer part \\
\texttt{\ footer-left\ } & optional, content, default: \texttt{\ none\ }
, overwrite left footer part \\
\texttt{\ show-outline\ } & optional, bool, default: \texttt{\ true\ } ,
show outline \\
\texttt{\ show-declaration-of-independent-work\ } & optional, bool,
default: \texttt{\ true\ } , show German declaration of independent
work \\
\texttt{\ page-margins\ } & optional, margins, default:
\texttt{\ none\ } , overwrite page margins \\
\texttt{\ fontsize\ } & optional, size, default: \texttt{\ 11pt\ } ,
overwrite fontsize \\
\texttt{\ show-todolist\ } & optional, bool, default: \texttt{\ true\ }
, show list of usages of the \texttt{\ todo\ } function after the
outline \\
\texttt{\ body\ } & content, document content \\
\end{longtable}

\begin{longtable}[]{@{}ll@{}}
\toprule\noalign{}
\texttt{\ sidenote\ } & \\
\midrule\noalign{}
\endhead
\bottomrule\noalign{}
\endlastfoot
\texttt{\ body\ } & sidenote content, which is a block with 3cm width
and will be displayed in the right margin of the page \\
\end{longtable}

\subsection{Slides}\label{slides}

\subsubsection{Setup}\label{setup-2}

\begin{Shaded}
\begin{Highlighting}[]
\NormalTok{\#import "@preview/grape{-}suite:1.0.0": slides}
\NormalTok{\#import slides: *}

\NormalTok{\#show: slides.with(}
\NormalTok{    no: 1,}
\NormalTok{    series: [Logik{-}Tutorium],}
\NormalTok{    title: [Organisatorisches und Einführung in die Logik],}

\NormalTok{    author: "Tristan Pieper",}
\NormalTok{    email: link("mailto:tristan.pieper@uni{-}rostock.de"),}
\NormalTok{)}
\end{Highlighting}
\end{Shaded}

\subsubsection{Documentation}\label{documentation-1}

\begin{longtable}[]{@{}ll@{}}
\toprule\noalign{}
\texttt{\ slides\ } & \\
\midrule\noalign{}
\endhead
\bottomrule\noalign{}
\endlastfoot
\texttt{\ no\ } & optional, number, default: \texttt{\ 0\ } , number in
the series \\
\texttt{\ series\ } & optional, content, default: \texttt{\ none\ } ,
name of the series \\
\texttt{\ title\ } & optional, content, default: \texttt{\ none\ } ,
title of the presentation \\
\texttt{\ topics\ } & optional, array, default: \texttt{\ ()\ } , topics
of the presentation \\
\texttt{\ author\ } & optional, content, default: \texttt{\ none\ } ,
author \\
\texttt{\ email\ } & optional, content, default: \texttt{\ none\ } ,
author’s email \\
\texttt{\ head-replacement\ } & optional, content, default:
\texttt{\ none\ } , replace head on title slide with given content \\
\texttt{\ title-replacement\ } & optional, content, default:
\texttt{\ none\ } , replace title below head on title slide with given
content \\
\texttt{\ footer\ } & optional, content, default: \texttt{\ none\ } ,
replace footer on slides with given content \\
\texttt{\ page-numbering\ } & optional, function, default:
\texttt{\ (n,\ total)\ =\textgreater{}\ \{...\}\ } , function that
creates the page numbering (where \texttt{\ n\ } is the current,
\texttt{\ total\ } is the last page) \\
\texttt{\ show-semester\ } & optional, bool, default: \texttt{\ true\ }
, show name of the semester (e.g. “SoSe 24�) \\
\texttt{\ show-date\ } & optional, bool, default: \texttt{\ true\ } ,
show date in german format \\
\texttt{\ show-outline\ } & optional, bool, default: \texttt{\ true\ } ,
show outline on the second slide \\
\texttt{\ box-task-title\ } & optional, content, default:
\texttt{\ {[}Task{]}\ } , shown as the title of a slide’s task box \\
\texttt{\ box-hint-title\ } & optional, content, default:
\texttt{\ {[}Hint{]}\ } , shown as the title of a slide’s tasks
colored \\
\texttt{\ box-solution-title\ } & optional, content, default:
\texttt{\ {[}Solution{]}\ } , shown as the title of a slide’s tasks
colored \\
\texttt{\ box-definition-title\ } & optional, content, default:
\texttt{\ {[}Definition{]}\ } , shown as the title of a slide’s
definition box \\
\texttt{\ box-notice-title\ } & optional, content, default:
\texttt{\ {[}Notice{]}\ } , shown as the title of a slide’s notice
box \\
\texttt{\ box-example-title\ } & optional, content, default:
\texttt{\ {[}Example{]}\ } , shown as the title of a slide’s example
box \\
\texttt{\ date\ } & optional, datetime, default:
\texttt{\ datetime.today()\ } \\
\texttt{\ show-todolist\ } & optional, bool, default: \texttt{\ true\ }
, show list of usages of the \texttt{\ todo\ } function after the
outline \\
\texttt{\ show-title-slide\ } & optional, bool, default:
\texttt{\ true\ } , show title slide \\
\texttt{\ show-author\ } & optional, bool, default: \texttt{\ true\ } ,
show author name on title slide \\
\texttt{\ show-footer\ } & optional, bool, default: \texttt{\ true\ } ,
show footer on slides \\
\texttt{\ show-page-numbers\ } & optional, bool, default:
\texttt{\ true\ } , show page numbering \\
\texttt{\ outline-title-text\ } & optional, content, default:
\texttt{\ "Outline"\ } , title for the outline \\
\texttt{\ body\ } & content, document content \\
\end{longtable}

\begin{itemize}
\tightlist
\item
  \texttt{\ slide\ } , \texttt{\ pause\ } , \texttt{\ only\ } ,
  \texttt{\ uncover\ } : imported from polylux
\end{itemize}

\subsubsection{Todos}\label{todos}

The following functions can be imported from \texttt{\ slides\ } ,
\texttt{\ exercise\ } and \texttt{\ seminar-paper\ } :

\begin{itemize}
\tightlist
\item
  \texttt{\ todo(content,\ ...)\ } - create a highlighted inline
  todo-note
\item
  \texttt{\ list-todos()\ } - create list of all todo-usages with page
  of usage and content
\item
  \texttt{\ hide-todos()\ } - hides all usages of \texttt{\ todo()\ } in
  the document
\end{itemize}

\subsubsection{Elements}\label{elements}

The following functions can be imported from \texttt{\ slides\ } ,
\texttt{\ exercise\ } and \texttt{\ seminar-paper\ } :
\texttt{\ definition\ }

\subsection{1.0.0}\label{section}

New:

\begin{itemize}
\tightlist
\item
  \texttt{\ todo\ } , \texttt{\ list-todos\ } , \texttt{\ hide-todos\ }
  in \texttt{\ todo.typ\ } , importable from \texttt{\ slides\ } ,
  \texttt{\ exercise.project\ } and \texttt{\ seminar-paper.project\ }
\item
  \texttt{\ show-todolist\ } attribute in above templates
\item
  \texttt{\ ignore-points\ } attribute in \texttt{\ task\ } and
  \texttt{\ subtask\ } of exercises, so that their points won’t be
  shown in the solution matrix or point distribution
\item
  comment field and a standard-value for solution matrix via
  \texttt{\ show-solution-matrix-comment-field\ } and
  \texttt{\ solution-matrix-comment-field-value\ } options in
  \texttt{\ exercise.project\ }
\item
  optional parameter \texttt{\ type\ } in \texttt{\ slides.task\ }
\item
  new parameters in \texttt{\ sllides.slides\ } :

  \begin{itemize}
  \tightlist
  \item
    \texttt{\ head-replacement\ }
  \item
    \texttt{\ title-replacement\ }
  \item
    \texttt{\ footer\ }
  \item
    \texttt{\ page-numbering\ }
  \item
    \texttt{\ show-title-slide\ }
  \item
    \texttt{\ show-author\ } (on title slide)
  \item
    \texttt{\ show-date\ }
  \item
    \texttt{\ show-footer\ }
  \item
    \texttt{\ show-page-numbers\ }
  \end{itemize}
\item
  optional parameter \texttt{\ show-outline\ } in
  \texttt{\ seminar-paper.project\ }
\end{itemize}

Changes:

\begin{itemize}
\tightlist
\item
  \texttt{\ dates.typ\ } becomes \texttt{\ german-dates.typ\ }
\end{itemize}

Fixes:

\begin{itemize}
\tightlist
\item
  remove forced German from the slides template
\item
  long headings are now properly aligned
\item
  subtask counter now resets for each part of task
\end{itemize}

\textbf{Breaking Changes:}

\begin{itemize}
\tightlist
\item
  \texttt{\ dates\ } becomes \texttt{\ german-dates\ }
\item
  changed all \texttt{\ with-outline\ } to \texttt{\ show-outline\ }
\end{itemize}

\href{/app?template=grape-suite&version=1.0.0}{Create project in app}

\subsubsection{How to use}\label{how-to-use}

Click the button above to create a new project using this template in
the Typst app.

You can also use the Typst CLI to start a new project on your computer
using this command:

\begin{verbatim}
typst init @preview/grape-suite:1.0.0
\end{verbatim}

\includesvg[width=0.16667in,height=0.16667in]{/assets/icons/16-copy.svg}

\subsubsection{About}\label{about}

\begin{description}
\tightlist
\item[Author :]
\href{mailto:tristanpieper080803@gmail.com}{Tristan Pieper}
\item[License:]
MIT
\item[Current version:]
1.0.0
\item[Last updated:]
July 22, 2024
\item[First released:]
May 3, 2024
\item[Minimum Typst version:]
0.11.0
\item[Archive size:]
15.7 kB
\href{https://packages.typst.org/preview/grape-suite-1.0.0.tar.gz}{\pandocbounded{\includesvg[keepaspectratio]{/assets/icons/16-download.svg}}}
\item[Repository:]
\href{https://github.com/piepert/grape-suite}{GitHub}
\item[Categor ies :]
\begin{itemize}
\tightlist
\item[]
\item
  \pandocbounded{\includesvg[keepaspectratio]{/assets/icons/16-layout.svg}}
  \href{https://typst.app/universe/search/?category=layout}{Layout}
\item
  \pandocbounded{\includesvg[keepaspectratio]{/assets/icons/16-atom.svg}}
  \href{https://typst.app/universe/search/?category=paper}{Paper}
\item
  \pandocbounded{\includesvg[keepaspectratio]{/assets/icons/16-presentation.svg}}
  \href{https://typst.app/universe/search/?category=presentation}{Presentation}
\end{itemize}
\end{description}

\subsubsection{Where to report issues?}\label{where-to-report-issues}

This template is a project of Tristan Pieper . Report issues on
\href{https://github.com/piepert/grape-suite}{their repository} . You
can also try to ask for help with this template on the
\href{https://forum.typst.app}{Forum} .

Please report this template to the Typst team using the
\href{https://typst.app/contact}{contact form} if you believe it is a
safety hazard or infringes upon your rights.

\phantomsection\label{versions}
\subsubsection{Version history}\label{version-history}

\begin{longtable}[]{@{}ll@{}}
\toprule\noalign{}
Version & Release Date \\
\midrule\noalign{}
\endhead
\bottomrule\noalign{}
\endlastfoot
1.0.0 & July 22, 2024 \\
\href{https://typst.app/universe/package/grape-suite/0.1.0/}{0.1.0} &
May 3, 2024 \\
\end{longtable}

Typst GmbH did not create this template and cannot guarantee correct
functionality of this template or compatibility with any version of the
Typst compiler or app.


\section{Package List LaTeX/pyrunner.tex}
\title{typst.app/universe/package/pyrunner}

\phantomsection\label{banner}
\section{pyrunner}\label{pyrunner}

{ 0.2.0 }

Run python code in typst.

\phantomsection\label{readme}
Run python code in \href{https://typst.app/}{typst} using
\href{https://github.com/RustPython/RustPython}{RustPython} .

\begin{Shaded}
\begin{Highlighting}[]
\NormalTok{\#import "@preview/pyrunner:0.1.0" as py}

\NormalTok{\#let compiled = py.compile(}
\NormalTok{\textasciigrave{}\textasciigrave{}\textasciigrave{}python}
\NormalTok{def find\_emails(string):}
\NormalTok{    import re}
\NormalTok{    return re.findall(r"\textbackslash{}b[a{-}zA{-}Z0{-}9.\_\%+{-}]+@[a{-}zA{-}Z0{-}9.{-}]+\textbackslash{}.[a{-}zA{-}Z]\{2,\}\textbackslash{}b", string)}

\NormalTok{def sum\_all(*array):}
\NormalTok{    return sum(array)}
\NormalTok{\textasciigrave{}\textasciigrave{}\textasciigrave{})}

\NormalTok{\#let txt = "My email address is john.doe@example.com and my friend\textquotesingle{}s email address is jane.doe@example.net."}

\NormalTok{\#py.call(compiled, "find\_emails", txt)}
\NormalTok{\#py.call(compiled, "sum\_all", 1, 2, 3)}
\end{Highlighting}
\end{Shaded}

Block mode is also available.

\begin{Shaded}
\begin{Highlighting}[]
\NormalTok{\#let code = \textasciigrave{}\textasciigrave{}\textasciigrave{}}
\NormalTok{f\textquotesingle{}\{a+b=\}\textquotesingle{}}
\NormalTok{\textasciigrave{}\textasciigrave{}\textasciigrave{}}

\NormalTok{\#py.block(code, globals: (a: 1, b: 2))}

\NormalTok{\#py.block(code, globals: (a: "1", b: "2"))}
\end{Highlighting}
\end{Shaded}

The result will be \texttt{\ a+b=3\ } and
\texttt{\ a+b=\textquotesingle{}12\textquotesingle{}\ } .

\subsection{Current limitations}\label{current-limitations}

Due to restrictions of typst and its plugin system, some Python function
will not work as expected:

\begin{itemize}
\tightlist
\item
  File and network IO will always raise an exception.
\item
  \texttt{\ datatime.now\ } will always return 1970-01-01.
\end{itemize}

Also, there is no way to import third-party modules. Only bundled stdlib
modules are available. We might find a way to lift this restriction, so
feel free to submit an issue if you want this functionality.

\subsection{API}\label{api}

\subsubsection{\texorpdfstring{\texttt{\ block\ }}{ block }}\label{block}

Run Python code block and get its result.

\paragraph{Arguments}\label{arguments}

\begin{itemize}
\tightlist
\item
  \texttt{\ code\ } : string \textbar{} raw content - The Python code to
  run.
\item
  \texttt{\ globals\ } : dict (named optional) - The global variables to
  bring into scope.
\end{itemize}

\paragraph{Returns}\label{returns}

The last expression of the code block.

\subsubsection{\texorpdfstring{\texttt{\ compile\ }}{ compile }}\label{compile}

Compile Python code to bytecode.

\paragraph{Arguments}\label{arguments-1}

\begin{itemize}
\tightlist
\item
  \texttt{\ code\ } : string \textbar{} raw content - The Python code to
  compile.
\end{itemize}

\paragraph{Returns}\label{returns-1}

The bytecode representation in \texttt{\ bytes\ } .

\subsubsection{\texorpdfstring{\texttt{\ call\ }}{ call }}\label{call}

Call a python function with arguments.

\paragraph{Arguments}\label{arguments-2}

\begin{itemize}
\tightlist
\item
  \texttt{\ compiled\ } : bytes - The bytecode representation of Python
  code.
\item
  \texttt{\ fn\_name\ } : string - The name of the function to be
  called.
\item
  \texttt{\ ..args\ } : any - The arguments to pass to the function.
\end{itemize}

\paragraph{Returns}\label{returns-2}

The result of the function call.

\subsection{Build from source}\label{build-from-source}

Install
\href{https://github.com/astrale-sharp/wasm-minimal-protocol}{\texttt{\ wasi-stub\ }}
. You should use a slightly modified one. See
\href{https://github.com/astrale-sharp/wasm-minimal-protocol/issues/22\#issuecomment-1827379467}{the
related issue} .

Build pyrunner.

\begin{verbatim}
rustup target add wasm32-wasi
cargo build --target wasm32-wasi
make pkg/typst-pyrunner.wasm
\end{verbatim}

Add to local package.

\begin{verbatim}
mkdir -p ~/.local/share/typst/packages/local/pyrunner/0.0.1
cp pkg/* ~/.local/share/typst/packages/local/pyrunner/0.0.1
\end{verbatim}

\subsubsection{How to add}\label{how-to-add}

Copy this into your project and use the import as \texttt{\ pyrunner\ }

\begin{verbatim}
#import "@preview/pyrunner:0.2.0"
\end{verbatim}

\includesvg[width=0.16667in,height=0.16667in]{/assets/icons/16-copy.svg}

Check the docs for
\href{https://typst.app/docs/reference/scripting/\#packages}{more
information on how to import packages} .

\subsubsection{About}\label{about}

\begin{description}
\tightlist
\item[Author :]
Peng Guanwen
\item[License:]
MIT
\item[Current version:]
0.2.0
\item[Last updated:]
March 18, 2024
\item[First released:]
December 4, 2023
\item[Minimum Typst version:]
0.10.0
\item[Archive size:]
5.89 MB
\href{https://packages.typst.org/preview/pyrunner-0.2.0.tar.gz}{\pandocbounded{\includesvg[keepaspectratio]{/assets/icons/16-download.svg}}}
\item[Repository:]
\href{https://github.com/peng1999/typst-pyrunner}{GitHub}
\item[Categor ies :]
\begin{itemize}
\tightlist
\item[]
\item
  \pandocbounded{\includesvg[keepaspectratio]{/assets/icons/16-code.svg}}
  \href{https://typst.app/universe/search/?category=scripting}{Scripting}
\item
  \pandocbounded{\includesvg[keepaspectratio]{/assets/icons/16-integration.svg}}
  \href{https://typst.app/universe/search/?category=integration}{Integration}
\end{itemize}
\end{description}

\subsubsection{Where to report issues?}\label{where-to-report-issues}

This package is a project of Peng Guanwen . Report issues on
\href{https://github.com/peng1999/typst-pyrunner}{their repository} .
You can also try to ask for help with this package on the
\href{https://forum.typst.app}{Forum} .

Please report this package to the Typst team using the
\href{https://typst.app/contact}{contact form} if you believe it is a
safety hazard or infringes upon your rights.

\phantomsection\label{versions}
\subsubsection{Version history}\label{version-history}

\begin{longtable}[]{@{}ll@{}}
\toprule\noalign{}
Version & Release Date \\
\midrule\noalign{}
\endhead
\bottomrule\noalign{}
\endlastfoot
0.2.0 & March 18, 2024 \\
\href{https://typst.app/universe/package/pyrunner/0.1.0/}{0.1.0} &
December 4, 2023 \\
\end{longtable}

Typst GmbH did not create this package and cannot guarantee correct
functionality of this package or compatibility with any version of the
Typst compiler or app.


\section{Package List LaTeX/frame-it.tex}
\title{typst.app/universe/package/frame-it}

\phantomsection\label{banner}
\section{frame-it}\label{frame-it}

{ 1.0.0 }

Beautiful, flexible, and integrated. Display custom frames for theorems,
environments, and more. Attractive visuals with syntax that blends
seamlessly into the source.

\phantomsection\label{readme}
\pandocbounded{\includesvg[keepaspectratio]{https://raw.githubusercontent.com/marc-thieme/frame-it/refs/heads/assets/README.svg}}

\subsubsection{How to add}\label{how-to-add}

Copy this into your project and use the import as \texttt{\ frame-it\ }

\begin{verbatim}
#import "@preview/frame-it:1.0.0"
\end{verbatim}

\includesvg[width=0.16667in,height=0.16667in]{/assets/icons/16-copy.svg}

Check the docs for
\href{https://typst.app/docs/reference/scripting/\#packages}{more
information on how to import packages} .

\subsubsection{About}\label{about}

\begin{description}
\tightlist
\item[Author :]
Marc Thieme
\item[License:]
MIT
\item[Current version:]
1.0.0
\item[Last updated:]
November 18, 2024
\item[First released:]
November 18, 2024
\item[Archive size:]
8.36 kB
\href{https://packages.typst.org/preview/frame-it-1.0.0.tar.gz}{\pandocbounded{\includesvg[keepaspectratio]{/assets/icons/16-download.svg}}}
\item[Repository:]
\href{https://github.com/marc-thieme/frame-it}{GitHub}
\item[Categor ies :]
\begin{itemize}
\tightlist
\item[]
\item
  \pandocbounded{\includesvg[keepaspectratio]{/assets/icons/16-package.svg}}
  \href{https://typst.app/universe/search/?category=components}{Components}
\item
  \pandocbounded{\includesvg[keepaspectratio]{/assets/icons/16-layout.svg}}
  \href{https://typst.app/universe/search/?category=layout}{Layout}
\item
  \pandocbounded{\includesvg[keepaspectratio]{/assets/icons/16-speak.svg}}
  \href{https://typst.app/universe/search/?category=report}{Report}
\end{itemize}
\end{description}

\subsubsection{Where to report issues?}\label{where-to-report-issues}

This package is a project of Marc Thieme . Report issues on
\href{https://github.com/marc-thieme/frame-it}{their repository} . You
can also try to ask for help with this package on the
\href{https://forum.typst.app}{Forum} .

Please report this package to the Typst team using the
\href{https://typst.app/contact}{contact form} if you believe it is a
safety hazard or infringes upon your rights.

\phantomsection\label{versions}
\subsubsection{Version history}\label{version-history}

\begin{longtable}[]{@{}ll@{}}
\toprule\noalign{}
Version & Release Date \\
\midrule\noalign{}
\endhead
\bottomrule\noalign{}
\endlastfoot
1.0.0 & November 18, 2024 \\
\end{longtable}

Typst GmbH did not create this package and cannot guarantee correct
functionality of this package or compatibility with any version of the
Typst compiler or app.


\section{Package List LaTeX/moderner-cv.tex}
\title{typst.app/universe/package/moderner-cv}

\phantomsection\label{banner}
\phantomsection\label{template-thumbnail}
\pandocbounded{\includegraphics[keepaspectratio]{https://packages.typst.org/preview/thumbnails/moderner-cv-0.1.0-small.webp}}

\section{moderner-cv}\label{moderner-cv}

{ 0.1.0 }

A resume template based on the moderncv LaTeX package.

\href{/app?template=moderner-cv&version=0.1.0}{Create project in app}

\phantomsection\label{readme}
This is a typst adaptation of LaTeX’s
\href{https://github.com/moderncv/moderncv}{moderncv} , a modern
curriculum vitae class.

\subsection{Requirements}\label{requirements}

This template uses FontAwesome icons via the
\href{https://typst.app/universe/package/fontawesome}{fontawesome typst
package} . In order to properly use it, you need to have fontawesome
installed on your system or have typst configured (via
\texttt{\ -\/-font-path\ } ) to use the fontawesome font files. You can
download fontawesome \href{https://fontawesome.com/download}{here} .

\subsection{Usage}\label{usage}

\begin{Shaded}
\begin{Highlighting}[]
\NormalTok{\#import "@preview/moderner{-}cv:0.1.0": *}

\NormalTok{\#show: moderner{-}cv.with(}
\NormalTok{  name: "Jane Doe",}
\NormalTok{  lang: "en",}
\NormalTok{  social: (}
\NormalTok{    email: "jane.doe@example.com",}
\NormalTok{    github: "jane{-}doe",}
\NormalTok{    linkedin: "jane{-}doe",}
\NormalTok{  ),}
\NormalTok{)}

\NormalTok{// ...}
\end{Highlighting}
\end{Shaded}

\subsection{Examples}\label{examples}

\pandocbounded{\includegraphics[keepaspectratio]{https://github.com/typst/packages/raw/main/packages/preview/moderner-cv/0.1.0/assets/example.png}}

\subsection{Building and Testing
Locally}\label{building-and-testing-locally}

To build and test the template locally, you can run
\texttt{\ pixi\ run\ watch\ } in the root of this repository. Please
ensure to have linked this package to your local typst packages, see
\href{https://github.com/typst/packages\#local-packages}{here} :

\begin{Shaded}
\begin{Highlighting}[]
\CommentTok{\# linux}
\FunctionTok{mkdir} \AttributeTok{{-}p}\NormalTok{ \textasciitilde{}/.local/share/typst/packages/preview/moderner{-}cv}
\FunctionTok{ln} \AttributeTok{{-}s} \VariableTok{$(}\BuiltInTok{pwd}\VariableTok{)}\NormalTok{ \textasciitilde{}/.local/share/typst/packages/preview/moderner{-}cv/0.1.0}

\CommentTok{\# macos}
\FunctionTok{mkdir} \AttributeTok{{-}p}\NormalTok{ \textasciitilde{}/Library/Application\textbackslash{} Support/typst/packages/preview/moderner{-}cv}
\FunctionTok{ln} \AttributeTok{{-}s} \VariableTok{$(}\BuiltInTok{pwd}\VariableTok{)}\NormalTok{ \textasciitilde{}/Library/Application\textbackslash{} Support/typst/packages/preview/moderner{-}cv/0.1.0}
\end{Highlighting}
\end{Shaded}

\href{/app?template=moderner-cv&version=0.1.0}{Create project in app}

\subsubsection{How to use}\label{how-to-use}

Click the button above to create a new project using this template in
the Typst app.

You can also use the Typst CLI to start a new project on your computer
using this command:

\begin{verbatim}
typst init @preview/moderner-cv:0.1.0
\end{verbatim}

\includesvg[width=0.16667in,height=0.16667in]{/assets/icons/16-copy.svg}

\subsubsection{About}\label{about}

\begin{description}
\tightlist
\item[Author :]
\href{https://github.com/pavelzw}{Pavel Zwerschke}
\item[License:]
MIT
\item[Current version:]
0.1.0
\item[Last updated:]
July 3, 2024
\item[First released:]
July 3, 2024
\item[Minimum Typst version:]
0.11.1
\item[Archive size:]
3.21 kB
\href{https://packages.typst.org/preview/moderner-cv-0.1.0.tar.gz}{\pandocbounded{\includesvg[keepaspectratio]{/assets/icons/16-download.svg}}}
\item[Repository:]
\href{https://github.com/pavelzw/moderner-cv}{GitHub}
\item[Categor y :]
\begin{itemize}
\tightlist
\item[]
\item
  \pandocbounded{\includesvg[keepaspectratio]{/assets/icons/16-user.svg}}
  \href{https://typst.app/universe/search/?category=cv}{CV}
\end{itemize}
\end{description}

\subsubsection{Where to report issues?}\label{where-to-report-issues}

This template is a project of Pavel Zwerschke . Report issues on
\href{https://github.com/pavelzw/moderner-cv}{their repository} . You
can also try to ask for help with this template on the
\href{https://forum.typst.app}{Forum} .

Please report this template to the Typst team using the
\href{https://typst.app/contact}{contact form} if you believe it is a
safety hazard or infringes upon your rights.

\phantomsection\label{versions}
\subsubsection{Version history}\label{version-history}

\begin{longtable}[]{@{}ll@{}}
\toprule\noalign{}
Version & Release Date \\
\midrule\noalign{}
\endhead
\bottomrule\noalign{}
\endlastfoot
0.1.0 & July 3, 2024 \\
\end{longtable}

Typst GmbH did not create this template and cannot guarantee correct
functionality of this template or compatibility with any version of the
Typst compiler or app.


\section{Package List LaTeX/finite.tex}
\title{typst.app/universe/package/finite}

\phantomsection\label{banner}
\section{finite}\label{finite}

{ 0.3.2 }

Typst-setting finite automata with CeTZ

{ } Featured Package

\phantomsection\label{readme}
\textbf{finite} is a \href{https://github.com/typst/typst}{Typst}
package for rendering finite automata on top of
\href{https://github.com/johannes-wolf/typst-canvas}{CeTZ} .

\subsection{Usage}\label{usage}

For Typst 0.6.0 or later, import the package from the typst preview
repository:

\begin{Shaded}
\begin{Highlighting}[]
\NormalTok{\#import }\StringTok{"@preview/finite:0.3.2"}\OperatorTok{:}\NormalTok{ automaton}
\end{Highlighting}
\end{Shaded}

After importing the package, simply call \texttt{\ \#automaton()\ } with
a dictionary holding a transition table:

\begin{Shaded}
\begin{Highlighting}[]
\NormalTok{\#import }\StringTok{"@preview/finite:0.3.2"}\OperatorTok{:}\NormalTok{ automaton}

\NormalTok{\#}\FunctionTok{automaton}\NormalTok{((}
\NormalTok{  q0}\OperatorTok{:}\NormalTok{ (q1}\OperatorTok{:}\DecValTok{0}\OperatorTok{,}\NormalTok{ q0}\OperatorTok{:}\StringTok{"0,1"}\NormalTok{)}\OperatorTok{,}
\NormalTok{  q1}\OperatorTok{:}\NormalTok{ (q0}\OperatorTok{:}\NormalTok{(}\DecValTok{0}\OperatorTok{,}\DecValTok{1}\NormalTok{)}\OperatorTok{,}\NormalTok{ q2}\OperatorTok{:}\StringTok{"0"}\NormalTok{)}\OperatorTok{,}
\NormalTok{  q2}\OperatorTok{:}\NormalTok{ ()}\OperatorTok{,}
\NormalTok{))}
\end{Highlighting}
\end{Shaded}

The output should look like this:
\pandocbounded{\includegraphics[keepaspectratio]{https://github.com/typst/packages/raw/main/packages/preview/finite/0.3.2/assets/example.png}}

\subsection{Further documentation}\label{further-documentation}

See \texttt{\ manual.pdf\ } for a full manual of the package.

\subsection{Development}\label{development}

The documentation is created using
\href{https://github.com/jneug/typst-mantys}{Mantys} , a Typst template
for creating package documentation.

To compile the manual, Mantys needs to be available as a local package.
Refer to Mantys’ manual for instructions on how to do so.

\subsection{Changelog}\label{changelog}

\subsubsection{Version 0.3.2}\label{version-0.3.2}

\begin{itemize}
\tightlist
\item
  Fixed an issue with final states not beeing recognized properly (\#5)
\end{itemize}

\subsubsection{Version 0.3.1}\label{version-0.3.1}

\begin{itemize}
\tightlist
\item
  Added styling options for intial states:

  \begin{itemize}
  \tightlist
  \item
    \texttt{\ stroke\ } sets a stroke for the marking.
  \item
    \texttt{\ scale\ } scales the marking by a factor.
  \end{itemize}
\item
  Updated manual.
\end{itemize}

\subsubsection{Version 0.3.0}\label{version-0.3.0}

\begin{itemize}
\tightlist
\item
  Bumped tools4typst to v0.3.2.
\item
  Introducing automaton specs as a data structure.
\item
  Changes to \texttt{\ automaton\ } command:

  \begin{itemize}
  \tightlist
  \item
    Changed \texttt{\ label-format\ } argument to
    \texttt{\ state-format\ } and \texttt{\ input-format\ } .
  \item
    \texttt{\ layout\ } can now take a dictionary with (
    \texttt{\ state\ } : \texttt{\ coordinate\ } ) pairs to position
    states.
  \end{itemize}
\item
  Added \texttt{\ \#powerset\ } command, to transform a NFA into a DFA.
\item
  Added \texttt{\ \#add-trap\ } command, to complete a partial DFA.
\item
  Added \texttt{\ \#accepts\ } command, to test a word against an NFA or
  DFA.
\item
  Added \texttt{\ transpose-table\ } and \texttt{\ get-inputs\ }
  utilities.
\item
  Added “Start� label to the mark for initial states.

  \begin{itemize}
  \tightlist
  \item
    Added option to modify the mark label for initial states.
  \end{itemize}
\item
  Added anchor option for loops, to position the loop at one of the
  eight default anchors.
\item
  Changed \texttt{\ curve\ } option to be the height of the arc of the
  transition.

  \begin{itemize}
  \tightlist
  \item
    This makes styling more consistent over longer distances.
  \end{itemize}
\item
  Added \texttt{\ rest\ } key to custom layouts.
\end{itemize}

\subsubsection{Version 0.2.0}\label{version-0.2.0}

\begin{itemize}
\tightlist
\item
  Bumped CeTZ to v0.1.1.
\end{itemize}

\subsubsection{Version 0.1.0}\label{version-0.1.0}

\begin{itemize}
\tightlist
\item
  Initial release submitted to
  \href{https://github.com/typst/packages}{typst/packages} .
\end{itemize}

\subsubsection{How to add}\label{how-to-add}

Copy this into your project and use the import as \texttt{\ finite\ }

\begin{verbatim}
#import "@preview/finite:0.3.2"
\end{verbatim}

\includesvg[width=0.16667in,height=0.16667in]{/assets/icons/16-copy.svg}

Check the docs for
\href{https://typst.app/docs/reference/scripting/\#packages}{more
information on how to import packages} .

\subsubsection{About}\label{about}

\begin{description}
\tightlist
\item[Author :]
Jonas Neugebauer
\item[License:]
MIT
\item[Current version:]
0.3.2
\item[Last updated:]
September 30, 2024
\item[First released:]
September 3, 2023
\item[Archive size:]
13.6 kB
\href{https://packages.typst.org/preview/finite-0.3.2.tar.gz}{\pandocbounded{\includesvg[keepaspectratio]{/assets/icons/16-download.svg}}}
\item[Repository:]
\href{https://github.com/jneug/typst-finite}{GitHub}
\end{description}

\subsubsection{Where to report issues?}\label{where-to-report-issues}

This package is a project of Jonas Neugebauer . Report issues on
\href{https://github.com/jneug/typst-finite}{their repository} . You can
also try to ask for help with this package on the
\href{https://forum.typst.app}{Forum} .

Please report this package to the Typst team using the
\href{https://typst.app/contact}{contact form} if you believe it is a
safety hazard or infringes upon your rights.

\phantomsection\label{versions}
\subsubsection{Version history}\label{version-history}

\begin{longtable}[]{@{}ll@{}}
\toprule\noalign{}
Version & Release Date \\
\midrule\noalign{}
\endhead
\bottomrule\noalign{}
\endlastfoot
0.3.2 & September 30, 2024 \\
\href{https://typst.app/universe/package/finite/0.3.0/}{0.3.0} &
September 23, 2023 \\
\href{https://typst.app/universe/package/finite/0.1.0/}{0.1.0} &
September 3, 2023 \\
\end{longtable}

Typst GmbH did not create this package and cannot guarantee correct
functionality of this package or compatibility with any version of the
Typst compiler or app.


\section{Package List LaTeX/inboisu.tex}
\title{typst.app/universe/package/inboisu}

\phantomsection\label{banner}
\phantomsection\label{template-thumbnail}
\pandocbounded{\includegraphics[keepaspectratio]{https://packages.typst.org/preview/thumbnails/inboisu-0.1.0-small.webp}}

\section{inboisu}\label{inboisu}

{ 0.1.0 }

Inboisu is a tool for creating Japanese invoices.

\href{/app?template=inboisu&version=0.1.0}{Create project in app}

\phantomsection\label{readme}
æ---¥æœ¬èªžã?®è«‹æ±‚書ã‚'作æˆ?ã?™ã‚‹ã?Ÿã‚?ã?® Typst
テンãƒ---レートã?§ã?™ã€‚

Inboisu is a Typst template for creating Japanese invoices.

\subsection{使ã?„æ--¹ / Usage}\label{uxe4uxbduxe3uxe6uxb9-usage}

\begin{Shaded}
\begin{Highlighting}[]
\NormalTok{\#include "@preview/inboisu:0.1.0": doc}

\NormalTok{\#show: doc(}
\NormalTok{    ... // ドキュメントを参照}
\NormalTok{)}
\end{Highlighting}
\end{Shaded}

\subsection{ドキュメント /
Documentation}\label{uxe3ux192uxe3uxe3ux192uxe3ux192uxe3ux192uxb3uxe3ux192ux2c6-documentation}

\begin{itemize}
\tightlist
\item
  \href{https://github.com/typst/packages/raw/main/packages/preview/inboisu/0.1.0/docs/documentation.pdf}{docs/documentation.pdf}
  (
  \href{https://github.com/typst/packages/raw/main/packages/preview/inboisu/0.1.0/docs/docs.typ}{Source}
  )
\end{itemize}

\subsection{テンãƒ---レート /
Templates}\label{uxe3ux192uxe3ux192uxb3uxe3ux192uxe3ux192uxe3ux192uxbcuxe3ux192ux2c6-templates}

\subsubsection{請求書 / Invoice}\label{uxe8uxe6uxe6-invoice}

\pandocbounded{\includegraphics[keepaspectratio]{https://github.com/typst/packages/raw/main/packages/preview/inboisu/0.1.0/images/invoice.png}}

\begin{itemize}
\tightlist
\item
  \href{https://github.com/typst/packages/raw/main/packages/preview/inboisu/0.1.0/template/invoice.typ}{invoice.typ}
\end{itemize}

\subsubsection{é~˜å?Žæ›¸ / Receipt}\label{uxe9-uxe5ux17euxe6-receipt}

\pandocbounded{\includegraphics[keepaspectratio]{https://github.com/typst/packages/raw/main/packages/preview/inboisu/0.1.0/images/receipt.png}}

\begin{itemize}
\tightlist
\item
  \href{https://github.com/typst/packages/raw/main/packages/preview/inboisu/0.1.0/template/receipt.typ}{receipt.typ}
\end{itemize}

\href{/app?template=inboisu&version=0.1.0}{Create project in app}

\subsubsection{How to use}\label{how-to-use}

Click the button above to create a new project using this template in
the Typst app.

You can also use the Typst CLI to start a new project on your computer
using this command:

\begin{verbatim}
typst init @preview/inboisu:0.1.0
\end{verbatim}

\includesvg[width=0.16667in,height=0.16667in]{/assets/icons/16-copy.svg}

\subsubsection{About}\label{about}

\begin{description}
\tightlist
\item[Author :]
\href{mailto:mkpoli@mkpo.li}{mkpoli}
\item[License:]
MIT-0
\item[Current version:]
0.1.0
\item[Last updated:]
November 21, 2024
\item[First released:]
November 21, 2024
\item[Archive size:]
5.01 kB
\href{https://packages.typst.org/preview/inboisu-0.1.0.tar.gz}{\pandocbounded{\includesvg[keepaspectratio]{/assets/icons/16-download.svg}}}
\item[Repository:]
\href{https://github.com/mkpoli/typst-inboisu}{GitHub}
\item[Discipline :]
\begin{itemize}
\tightlist
\item[]
\item
  \href{https://typst.app/universe/search/?discipline=business}{Business}
\end{itemize}
\item[Categor ies :]
\begin{itemize}
\tightlist
\item[]
\item
  \pandocbounded{\includesvg[keepaspectratio]{/assets/icons/16-layout.svg}}
  \href{https://typst.app/universe/search/?category=layout}{Layout}
\item
  \pandocbounded{\includesvg[keepaspectratio]{/assets/icons/16-envelope.svg}}
  \href{https://typst.app/universe/search/?category=office}{Office}
\end{itemize}
\end{description}

\subsubsection{Where to report issues?}\label{where-to-report-issues}

This template is a project of mkpoli . Report issues on
\href{https://github.com/mkpoli/typst-inboisu}{their repository} . You
can also try to ask for help with this template on the
\href{https://forum.typst.app}{Forum} .

Please report this template to the Typst team using the
\href{https://typst.app/contact}{contact form} if you believe it is a
safety hazard or infringes upon your rights.

\phantomsection\label{versions}
\subsubsection{Version history}\label{version-history}

\begin{longtable}[]{@{}ll@{}}
\toprule\noalign{}
Version & Release Date \\
\midrule\noalign{}
\endhead
\bottomrule\noalign{}
\endlastfoot
0.1.0 & November 21, 2024 \\
\end{longtable}

Typst GmbH did not create this template and cannot guarantee correct
functionality of this template or compatibility with any version of the
Typst compiler or app.


\section{Package List LaTeX/fruitify.tex}
\title{typst.app/universe/package/fruitify}

\phantomsection\label{banner}
\section{fruitify}\label{fruitify}

{ 0.1.1 }

Replace letters in equations with fruit emoji

\phantomsection\label{readme}
Make your equations more fruity!

This package automatically replaces any single letters in equations with
fruit emoji.

Refer to
\href{https://codeberg.org/T0mstone/typst-fruitify/src/tag/0.1.1/example-documentation.pdf}{\texttt{\ example-documentation.pdf\ }}
for more detail.

\subsection{Emoji support}\label{emoji-support}

Until 0.12, typst did not have good emoji support for PDF. This meant
that even though this package worked as intended, the output would look
very wrong when exporting to PDF. Therefore, it is recommended to stick
with PNG export for those older typst versions.

\subsubsection{How to add}\label{how-to-add}

Copy this into your project and use the import as \texttt{\ fruitify\ }

\begin{verbatim}
#import "@preview/fruitify:0.1.1"
\end{verbatim}

\includesvg[width=0.16667in,height=0.16667in]{/assets/icons/16-copy.svg}

Check the docs for
\href{https://typst.app/docs/reference/scripting/\#packages}{more
information on how to import packages} .

\subsubsection{About}\label{about}

\begin{description}
\tightlist
\item[Author :]
\href{mailto:realt0mstone@gmail.com}{T0mstone}
\item[License:]
MIT-0
\item[Current version:]
0.1.1
\item[Last updated:]
October 16, 2024
\item[First released:]
October 11, 2023
\item[Minimum Typst version:]
0.9.0
\item[Archive size:]
8.42 kB
\href{https://packages.typst.org/preview/fruitify-0.1.1.tar.gz}{\pandocbounded{\includesvg[keepaspectratio]{/assets/icons/16-download.svg}}}
\item[Repository:]
\href{https://codeberg.org/T0mstone/typst-fruitify}{Codeberg}
\end{description}

\subsubsection{Where to report issues?}\label{where-to-report-issues}

This package is a project of T0mstone . Report issues on
\href{https://codeberg.org/T0mstone/typst-fruitify}{their repository} .
You can also try to ask for help with this package on the
\href{https://forum.typst.app}{Forum} .

Please report this package to the Typst team using the
\href{https://typst.app/contact}{contact form} if you believe it is a
safety hazard or infringes upon your rights.

\phantomsection\label{versions}
\subsubsection{Version history}\label{version-history}

\begin{longtable}[]{@{}ll@{}}
\toprule\noalign{}
Version & Release Date \\
\midrule\noalign{}
\endhead
\bottomrule\noalign{}
\endlastfoot
0.1.1 & October 16, 2024 \\
\href{https://typst.app/universe/package/fruitify/0.1.0/}{0.1.0} &
October 11, 2023 \\
\end{longtable}

Typst GmbH did not create this package and cannot guarantee correct
functionality of this package or compatibility with any version of the
Typst compiler or app.


\section{Package List LaTeX/finely-crafted-cv.tex}
\title{typst.app/universe/package/finely-crafted-cv}

\phantomsection\label{banner}
\phantomsection\label{template-thumbnail}
\pandocbounded{\includegraphics[keepaspectratio]{https://packages.typst.org/preview/thumbnails/finely-crafted-cv-0.1.0-small.webp}}

\section{finely-crafted-cv}\label{finely-crafted-cv}

{ 0.1.0 }

A modern résumé/curriculum vitæ template with high attention to
detail.

\href{/app?template=finely-crafted-cv&version=0.1.0}{Create project in
app}

\phantomsection\label{readme}
This Typst template provides a clean and professional format for
creating a curriculum vitae (CV) or résumé. It comes with functions
and styles to help you easily generate a well-structured document,
complete with sections for education, experience, skills, and more.

\subsection{Features}\label{features}

\begin{itemize}
\tightlist
\item
  \textbf{Modern Design:} Aesthetic and professional layout designed for
  readability.
\item
  \textbf{Responsive Header \& Footer:} Includes contact information
  dynamically.
\end{itemize}

\subsection{Usage}\label{usage}

To use this template, import it with the version number and utilize the
\texttt{\ resume\ } or \texttt{\ cv\ } function:

\begin{Shaded}
\begin{Highlighting}[]
\NormalTok{\#import "@preview/finely{-}crafted{-}cv:0.1.0": *}

\NormalTok{\#show: resume.with(}
\NormalTok{  name: "Amira Patel",}
\NormalTok{  tagline: "Innovative marine biologist with 15+ years of experience in ocean conservation and research.",}
\NormalTok{  keywords: "marine biology, conservation, research, education, patents",}
\NormalTok{  email: "amira.patel@oceandreams.org",}
\NormalTok{  phone: "+1{-}305{-}555{-}7890",}
\NormalTok{  linkedin{-}username: "amirapatel",}
\NormalTok{  thumbnail: image("assets/my{-}qr{-}code.svg"),}
\NormalTok{)}

\NormalTok{= Introduction}

\NormalTok{\#lorem(100)}

\NormalTok{= Experience}

\NormalTok{\#company{-}heading("Some Company", start: "March 2018", end: "Present", icon: image("icons/earth.svg"))[}
\NormalTok{  \#job{-}heading("Some Job", location: "Some Location")[}
\NormalTok{    {-} Here is an achievement}
\NormalTok{    {-} Here\textquotesingle{}s another one.}
\NormalTok{  ]}
\NormalTok{  // companies can have multiple jobs}
\NormalTok{  \#job{-}heading("First Job", location: "Some Location")[}
\NormalTok{    {-} Here is an achievement}
\NormalTok{    {-} Here\textquotesingle{}s another one.}
\NormalTok{  ]}
\NormalTok{]}

\NormalTok{// for companies which have less detail, you can use the \textasciigrave{}comment\textasciigrave{} instead of a}
\NormalTok{// body of tasks, as follows:}
\NormalTok{\#company{-}heading("Another Company", start: "July 2005", end: "August 2009", icon: image("icons/microscope.svg"))[}
\NormalTok{  \#job{-}heading("Another Job", location: "Another Location",}
\NormalTok{    comment: [Contributed to 7 published studies. \#footnote[Visit https://amirapatel.org/publications for full list of publications.]]}
\NormalTok{  )[]}
\NormalTok{]}

\NormalTok{= Education}

\NormalTok{// school{-}heading is an alias for company{-}heading, accepts the same parameters as company{-}heading}
\NormalTok{\#school{-}heading("University of California, San Diego", start: "Fall 2001", end: "Spring 2005", icon: image("icons/graduation{-}cap.svg"))[}
\NormalTok{  // degree{-}heading is an alias for job{-}heading, accepts the same parameters as job{-}heading}
\NormalTok{  \#degree{-}heading("Ph.D. in Marine Biology")[]}
\NormalTok{]}
\end{Highlighting}
\end{Shaded}

\subsection{Functions and Parameters}\label{functions-and-parameters}

\subsubsection{\texorpdfstring{\texttt{\ resume\ } or
\texttt{\ cv\ }}{ resume  or  cv }}\label{resume-or-cv}

This is the main function to create a CV document.

\begin{itemize}
\tightlist
\item
  \textbf{Parameters:}

  \begin{itemize}
  \tightlist
  \item
    \texttt{\ name\ } : (String) Your full name. Default is “YOUR NAME
    HERE�.
  \item
    \texttt{\ tagline\ } : (String) A brief description of your
    professional identity or mission.
  \item
    \texttt{\ paper\ } : (String) The paper size, default is
    “us-letter�.
  \item
    \texttt{\ heading-font\ } : (Font) Font for headings, customizable.
  \item
    \texttt{\ body-font\ } : (Font) Font for body text, customizable.
  \item
    \texttt{\ body-size\ } : (Size) Font size for body text.
  \item
    \texttt{\ email\ } : (String) Your email address.
  \item
    \texttt{\ phone\ } : (String) Your phone number.
  \item
    \texttt{\ linkedin-username\ } : (String) Your LinkedIn username.
  \item
    \texttt{\ keywords\ } : (String) Keywords for searchability.
  \item
    \texttt{\ thumbnail\ } : (Image) Thumbnail or QR code image,
    optional.
  \item
    \texttt{\ body\ } : (Block) The main content of your CV.
  \end{itemize}
\end{itemize}

\subsubsection{\texorpdfstring{\texttt{\ company-heading\ }}{ company-heading }}\label{company-heading}

Used to create a heading for a company or organization.

\begin{itemize}
\tightlist
\item
  \textbf{Parameters:}

  \begin{itemize}
  \tightlist
  \item
    \texttt{\ name\ } : (String) Name of the company.
  \item
    \texttt{\ start\ } : (String) Start date.
  \item
    \texttt{\ end\ } : (String) End date, optional.
  \item
    \texttt{\ icon\ } : (Image) Icon image associated with the company,
    optional.
  \item
    \texttt{\ body\ } : (Block) Content related to the company role or
    tasks.
  \end{itemize}
\end{itemize}

\subsubsection{\texorpdfstring{\texttt{\ job-heading\ }}{ job-heading }}\label{job-heading}

Defines a job title within a company heading.

\begin{itemize}
\tightlist
\item
  \textbf{Parameters:}

  \begin{itemize}
  \tightlist
  \item
    \texttt{\ title\ } : (String) Job title.
  \item
    \texttt{\ location\ } : (String) Location of the job, optional.
  \item
    \texttt{\ start\ } : (String) Start date, optional.
  \item
    \texttt{\ end\ } : (String) End date, optional.
  \item
    \texttt{\ comment\ } : (String) Additional comments or notes,
    optional.
  \item
    \texttt{\ body\ } : (Block) Tasks or responsibilities.
  \end{itemize}
\end{itemize}

\subsubsection{\texorpdfstring{\texttt{\ school-heading\ }}{ school-heading }}\label{school-heading}

Alias for \texttt{\ company-heading\ } , used for educational
institutions.

\subsubsection{\texorpdfstring{\texttt{\ degree-heading\ }}{ degree-heading }}\label{degree-heading}

Alias for \texttt{\ job-heading\ } , used for academic degrees or
certifications.

\subsection{License}\label{license}

This template is released under the MIT License.

\href{/app?template=finely-crafted-cv&version=0.1.0}{Create project in
app}

\subsubsection{How to use}\label{how-to-use}

Click the button above to create a new project using this template in
the Typst app.

You can also use the Typst CLI to start a new project on your computer
using this command:

\begin{verbatim}
typst init @preview/finely-crafted-cv:0.1.0
\end{verbatim}

\includesvg[width=0.16667in,height=0.16667in]{/assets/icons/16-copy.svg}

\subsubsection{About}\label{about}

\begin{description}
\tightlist
\item[Author :]
\href{mailto:steve@waits.net}{Stephen Waits}
\item[License:]
MIT
\item[Current version:]
0.1.0
\item[Last updated:]
October 22, 2024
\item[First released:]
October 22, 2024
\item[Archive size:]
28.5 kB
\href{https://packages.typst.org/preview/finely-crafted-cv-0.1.0.tar.gz}{\pandocbounded{\includesvg[keepaspectratio]{/assets/icons/16-download.svg}}}
\item[Repository:]
\href{https://github.com/swaits/typst-collection}{GitHub}
\item[Discipline s :]
\begin{itemize}
\tightlist
\item[]
\item
  \href{https://typst.app/universe/search/?discipline=business}{Business}
\item
  \href{https://typst.app/universe/search/?discipline=communication}{Communication}
\end{itemize}
\item[Categor y :]
\begin{itemize}
\tightlist
\item[]
\item
  \pandocbounded{\includesvg[keepaspectratio]{/assets/icons/16-user.svg}}
  \href{https://typst.app/universe/search/?category=cv}{CV}
\end{itemize}
\end{description}

\subsubsection{Where to report issues?}\label{where-to-report-issues}

This template is a project of Stephen Waits . Report issues on
\href{https://github.com/swaits/typst-collection}{their repository} .
You can also try to ask for help with this template on the
\href{https://forum.typst.app}{Forum} .

Please report this template to the Typst team using the
\href{https://typst.app/contact}{contact form} if you believe it is a
safety hazard or infringes upon your rights.

\phantomsection\label{versions}
\subsubsection{Version history}\label{version-history}

\begin{longtable}[]{@{}ll@{}}
\toprule\noalign{}
Version & Release Date \\
\midrule\noalign{}
\endhead
\bottomrule\noalign{}
\endlastfoot
0.1.0 & October 22, 2024 \\
\end{longtable}

Typst GmbH did not create this template and cannot guarantee correct
functionality of this template or compatibility with any version of the
Typst compiler or app.


\section{Package List LaTeX/ouset.tex}
\title{typst.app/universe/package/ouset}

\phantomsection\label{banner}
\section{ouset}\label{ouset}

{ 0.2.0 }

Package providing over- and underset functions for math mode.

\phantomsection\label{readme}
\href{https://github.com/ludwig-austermann/typst-ouset}{GitHub
Repository including Examples and Changelog}

This is a small package providing over- and underset functions for math
mode in \href{https://typst.app/}{typst} .

\subsection{Usage}\label{usage}

To use this package simply \texttt{\ \#import\ "@preview/ouset:0.2.0"\ }
. To import all functions use \texttt{\ :\ *\ } and for specific ones,
use either the module or as described in the
\href{https://typst.app/docs/reference/scripting\#modules}{typst docs} .

The main function provided in this package is \texttt{\ ouset\ } for
math environments. This function can take arbitrary many arguments, but
with the following rules:

\begin{itemize}
\tightlist
\item
  if the first argument is \texttt{\ \&\ } , a ‘alignpoint’ is
  inserted immediately before the symbol
\item
  next follows the symbol, then the content to put on top, and then the
  content to put at the bottom
\item
  if the last argument is \texttt{\ \&\ } , a ‘alignpoint’ is
  inserted immediately after the symbol
\end{itemize}

There is a named argument \texttt{\ insert-and\ } , which if false, does
not insert an ‘alignpoint’ in the above cases, but only clips at
these points.

This package provides furthermore 3 other functions:

\begin{itemize}
\tightlist
\item
  \texttt{\ overset(s,\ t,\ c:\ 0,\ insert-and:\ true)\ } : output the
  symbol s with t on top of it
\item
  \texttt{\ underset(s,\ b,\ c:\ 0\ insert-and:\ true)\ } : output the
  symbol s with b on below of it
\item
  \texttt{\ overunderset(s,\ t,\ b,\ c:\ 0,\ insert-and:\ true)\ } :
  output the symbol s with t on top of it and b below it
\end{itemize}

All functions put enough spacing around the operator, such that other
content does not interfere with it. However, this spacing can be
disabled, by setting \texttt{\ c\ } to 1, 2 or 3. This is a flag system
with

\begin{itemize}
\tightlist
\item
  \texttt{\ c=0\ } : normal spacing on the left and right
\item
  \texttt{\ c=1\ } : left spacing is according to the operator / symbol
  s and right spacing is normal
\item
  \texttt{\ c=2\ } : left spacing is normal and right spacing according
  to the operator / symbol s
\item
  \texttt{\ c=3\ } : both spacings are according to the operator /
  symbol s
\end{itemize}

Hence: clip param
\texttt{\ c\ ∈\ \{0,1,2,3\}\ ≜\ \{no\ clip,\ left\ clip,\ right\ clip,\ both\ clip\}\ }

\subsection{Example usage}\label{example-usage}

Try something like:

\begin{itemize}
\item
  \texttt{\ \$ouset(-\/-\textgreater{},,\ n-\textgreater{}oo)\$\ }
\item
  \texttt{\ \$ouset(-,1,2)\$\ }
\item
\begin{Shaded}
\begin{Highlighting}[]
\NormalTok{\#import "@preview/ouset:0.2.0": ouset}

\NormalTok{$ M \&= sum\_(k=0)\^{}oo q\^{}k = 1 + q + q\^{}2 + q\^{}3 + q\^{}4 + dots\textbackslash{}}
\NormalTok{    \&= 1 + q (1 + q + q\^{}2 + q\^{}3 + dots)\textbackslash{}}
\NormalTok{    ouset(\&, =, "Def.", "of" M) 1 + q dot M $}
\end{Highlighting}
\end{Shaded}
\end{itemize}

\subsubsection{How to add}\label{how-to-add}

Copy this into your project and use the import as \texttt{\ ouset\ }

\begin{verbatim}
#import "@preview/ouset:0.2.0"
\end{verbatim}

\includesvg[width=0.16667in,height=0.16667in]{/assets/icons/16-copy.svg}

Check the docs for
\href{https://typst.app/docs/reference/scripting/\#packages}{more
information on how to import packages} .

\subsubsection{About}\label{about}

\begin{description}
\tightlist
\item[Author :]
Ludwig Austermann
\item[License:]
MIT
\item[Current version:]
0.2.0
\item[Last updated:]
May 31, 2024
\item[First released:]
July 6, 2023
\item[Minimum Typst version:]
0.11.0
\item[Archive size:]
2.61 kB
\href{https://packages.typst.org/preview/ouset-0.2.0.tar.gz}{\pandocbounded{\includesvg[keepaspectratio]{/assets/icons/16-download.svg}}}
\item[Repository:]
\href{https://github.com/ludwig-austermann/typst-ouset}{GitHub}
\item[Categor y :]
\begin{itemize}
\tightlist
\item[]
\item
  \pandocbounded{\includesvg[keepaspectratio]{/assets/icons/16-layout.svg}}
  \href{https://typst.app/universe/search/?category=layout}{Layout}
\end{itemize}
\end{description}

\subsubsection{Where to report issues?}\label{where-to-report-issues}

This package is a project of Ludwig Austermann . Report issues on
\href{https://github.com/ludwig-austermann/typst-ouset}{their
repository} . You can also try to ask for help with this package on the
\href{https://forum.typst.app}{Forum} .

Please report this package to the Typst team using the
\href{https://typst.app/contact}{contact form} if you believe it is a
safety hazard or infringes upon your rights.

\phantomsection\label{versions}
\subsubsection{Version history}\label{version-history}

\begin{longtable}[]{@{}ll@{}}
\toprule\noalign{}
Version & Release Date \\
\midrule\noalign{}
\endhead
\bottomrule\noalign{}
\endlastfoot
0.2.0 & May 31, 2024 \\
\href{https://typst.app/universe/package/ouset/0.1.1/}{0.1.1} & July 7,
2023 \\
\href{https://typst.app/universe/package/ouset/0.1.0/}{0.1.0} & July 6,
2023 \\
\end{longtable}

Typst GmbH did not create this package and cannot guarantee correct
functionality of this package or compatibility with any version of the
Typst compiler or app.


\section{Package List LaTeX/lambdabus.tex}
\title{typst.app/universe/package/lambdabus}

\phantomsection\label{banner}
\section{lambdabus}\label{lambdabus}

{ 0.1.0 }

Easily parse, normalize and display simple λ-Calculus expressions.

\phantomsection\label{readme}
Lambdabus allows you to parse, normalize and display simple λ-Calculus
expressions in Typst with ease.

\subsection{Usage}\label{usage}

Lambdabus is available on the
\href{https://typst.app/universe/package/lambdabus/}{Typst Universe} and
it is thus recommended to be imported like this:

\begin{Shaded}
\begin{Highlighting}[]
\NormalTok{\#import "@preview/lambdabus:0.1.0" as lmd}
\end{Highlighting}
\end{Shaded}

\subsection{Features/Examples}\label{featuresexamples}

\pandocbounded{\includegraphics[keepaspectratio]{https://raw.github.com/luca-schlecker/typst-lambdabus/v0.1.0/gallery.png}}

\subsubsection{How to add}\label{how-to-add}

Copy this into your project and use the import as \texttt{\ lambdabus\ }

\begin{verbatim}
#import "@preview/lambdabus:0.1.0"
\end{verbatim}

\includesvg[width=0.16667in,height=0.16667in]{/assets/icons/16-copy.svg}

Check the docs for
\href{https://typst.app/docs/reference/scripting/\#packages}{more
information on how to import packages} .

\subsubsection{About}\label{about}

\begin{description}
\tightlist
\item[Author :]
\href{https://github.com/luca-schlecker}{Luca Schlecker}
\item[License:]
MIT
\item[Current version:]
0.1.0
\item[Last updated:]
November 4, 2024
\item[First released:]
November 4, 2024
\item[Archive size:]
3.90 kB
\href{https://packages.typst.org/preview/lambdabus-0.1.0.tar.gz}{\pandocbounded{\includesvg[keepaspectratio]{/assets/icons/16-download.svg}}}
\item[Repository:]
\href{https://github.com/luca-schlecker/typst-lambdabus}{GitHub}
\item[Categor y :]
\begin{itemize}
\tightlist
\item[]
\item
  \pandocbounded{\includesvg[keepaspectratio]{/assets/icons/16-smile.svg}}
  \href{https://typst.app/universe/search/?category=fun}{Fun}
\end{itemize}
\end{description}

\subsubsection{Where to report issues?}\label{where-to-report-issues}

This package is a project of Luca Schlecker . Report issues on
\href{https://github.com/luca-schlecker/typst-lambdabus}{their
repository} . You can also try to ask for help with this package on the
\href{https://forum.typst.app}{Forum} .

Please report this package to the Typst team using the
\href{https://typst.app/contact}{contact form} if you believe it is a
safety hazard or infringes upon your rights.

\phantomsection\label{versions}
\subsubsection{Version history}\label{version-history}

\begin{longtable}[]{@{}ll@{}}
\toprule\noalign{}
Version & Release Date \\
\midrule\noalign{}
\endhead
\bottomrule\noalign{}
\endlastfoot
0.1.0 & November 4, 2024 \\
\end{longtable}

Typst GmbH did not create this package and cannot guarantee correct
functionality of this package or compatibility with any version of the
Typst compiler or app.


\section{Package List LaTeX/modern-unito-thesis.tex}
\title{typst.app/universe/package/modern-unito-thesis}

\phantomsection\label{banner}
\phantomsection\label{template-thumbnail}
\pandocbounded{\includegraphics[keepaspectratio]{https://packages.typst.org/preview/thumbnails/modern-unito-thesis-0.1.0-small.webp}}

\section{modern-unito-thesis}\label{modern-unito-thesis}

{ 0.1.0 }

A thesis template of the University of Turin

\href{/app?template=modern-unito-thesis&version=0.1.0}{Create project in
app}

\phantomsection\label{readme}
This is a thesis template for the University of Turin (UniTO) based on
\href{https://github.com/eduardz1/Bachelor-Thesis}{my thesis} , since
there are no strict templates (notable mention to
\href{https://github.com/esenes/Unito-thesis-template}{Eugenio’s LaTeX
template though} ) take my choices with a grain of salt, different
supervisors may ask you to customize the template differently. My
choices are loosely based on this document:
\href{https://elearning.unito.it/sme/pluginfile.php/29485/mod_folder/content/0/format_TESI_2011-2012.pdf}{Indicazioni
per il Format della Tesi} .

If you find errors or ways to improve the template please open an issue
or contribute directly with a PR.

\subsection{Usage}\label{usage}

In the Typst web app simply click “Start from template� on the
dashboard and search for \texttt{\ modern-unito-thesis\ } .

From the CLI you can initialize the project with the command

\begin{Shaded}
\begin{Highlighting}[]
\ExtensionTok{typst}\NormalTok{ init @preview/modern{-}unito{-}thesis}
\end{Highlighting}
\end{Shaded}

A new directory with all the files needed to get started will be
created.

\subsection{Configuration}\label{configuration}

This template exports the \texttt{\ template\ } function with the
following named arguments:

\begin{itemize}
\tightlist
\item
  \texttt{\ title\ } : the title of the thesis
\item
  \texttt{\ academic-year\ } : the academic year (e.g. 2023/2024)
\item
  \texttt{\ subtitle\ } : e.g. “Bachelor’s Thesis�
\item
  \texttt{\ paper-size\ } (default \texttt{\ a4\ } ): the paper format
\item
  \texttt{\ candidate\ } : your name, surname and matricola (student id)
\item
  \texttt{\ supervisor\ } (relatore): your supervisor’s name and
  surname
\item
  \texttt{\ co-supervisor\ } (correlatore): an array of your
  co-supervisors’ names and surnames
\item
  \texttt{\ affiliation\ } : a dictionary that specifies
  \texttt{\ university\ } , \texttt{\ school\ } and \texttt{\ degree\ }
  keywords
\item
  \texttt{\ lang\ } : configurable between \texttt{\ en\ } for English
  and \texttt{\ it\ } for Italian
\item
  \texttt{\ bibliography-path\ } : the path to your bibliography file
  (e.g. \texttt{\ works.bib\ } )
\item
  \texttt{\ logo\ } (already set to UniTO’s logo by default): the path
  to your university’s logo
\item
  \texttt{\ abstract\ } : your thesis’ abstract, can be set to
  \texttt{\ none\ } if not needed
\item
  \texttt{\ acknowledgments\ } : your thesis’ acknowledgments, can be
  set to \texttt{\ none\ } if not needed
\item
  \texttt{\ keywords\ } : a list of keywords for the thesis, can be set
  to \texttt{\ none\ } if not needed
\end{itemize}

The template will initialize an example project with sensible defaults.

The template divides the level 1 headings in chapters under the
\texttt{\ chapters\ } directory, I suggest using this structure to keep
the project organized.

If you want to change an existing project to use this template, you can
add a show rule like this at the top of your file:

\begin{Shaded}
\begin{Highlighting}[]
\NormalTok{\#import "@preview/modern{-}unito{-}thesis:0.1.0": template}

\NormalTok{\#show: template.with(}
\NormalTok{  title: "My Beautiful Thesis",}
\NormalTok{  academic{-}year: [2023/2024],}
\NormalTok{  subtitle: "Bachelor\textquotesingle{}s Thesis",}
\NormalTok{  logo: image("path/to/your/logo.png"),}
\NormalTok{  candidate: (}
\NormalTok{    name: "Eduard Antonovic Occhipinti",}
\NormalTok{    matricola: 947847}
\NormalTok{  ),}
\NormalTok{  supervisor: (}
\NormalTok{    "Prof. Luigi Paperino"}
\NormalTok{  ),}
\NormalTok{  co{-}supervisor: (}
\NormalTok{    "Dott. Pluto Mario",}
\NormalTok{    "Dott. Minni Topolino"}
\NormalTok{  ),}
\NormalTok{  affiliation: (}
\NormalTok{    university: "Università degli Studi di Torino",}
\NormalTok{    school: "Scuola di Scienze della Natura",}
\NormalTok{    degree: "Corso di Laurea Triennale in Informatica",}
\NormalTok{  ),}
\NormalTok{  bibliography: bibliography("works.yml"),}
\NormalTok{  abstract: [Your abstract goes here],}
\NormalTok{  acknowledgments: [Your acknowledgments go here],}
\NormalTok{  keywords: [keyword1, keyword2, keyword3]}
\NormalTok{)}

\NormalTok{// Your content goes here}
\end{Highlighting}
\end{Shaded}

\subsection{Compile}\label{compile}

To compile the project from the CLI you just need to run

\begin{Shaded}
\begin{Highlighting}[]
\ExtensionTok{typst}\NormalTok{ compile main.typ}
\end{Highlighting}
\end{Shaded}

or if you want to watch for changes (recommended)

\begin{Shaded}
\begin{Highlighting}[]
\ExtensionTok{typst}\NormalTok{ watch main.typ}
\end{Highlighting}
\end{Shaded}

\subsection{Bibliography}\label{bibliography}

I integrated the bibliography as a
\href{https://github.com/typst/hayagriva}{Hayagriva} \texttt{\ yaml\ }
file under
\href{https://github.com/typst/packages/raw/main/packages/preview/modern-unito-thesis/0.1.0/template/works.yml}{works.yml}
, nonetheless using the more common \texttt{\ bib\ } format for your
bibliography management is as simple as passing a BibTex file to the
template \texttt{\ bibliography\ } parameter. Given that our university
is not strict in this regard I suggest using Hayagriva though :).

\href{/app?template=modern-unito-thesis&version=0.1.0}{Create project in
app}

\subsubsection{How to use}\label{how-to-use}

Click the button above to create a new project using this template in
the Typst app.

You can also use the Typst CLI to start a new project on your computer
using this command:

\begin{verbatim}
typst init @preview/modern-unito-thesis:0.1.0
\end{verbatim}

\includesvg[width=0.16667in,height=0.16667in]{/assets/icons/16-copy.svg}

\subsubsection{About}\label{about}

\begin{description}
\tightlist
\item[Author :]
Eduard Antonovic Occhipinti
\item[License:]
MIT
\item[Current version:]
0.1.0
\item[Last updated:]
March 20, 2024
\item[First released:]
March 20, 2024
\item[Minimum Typst version:]
0.11.0
\item[Archive size:]
19.6 kB
\href{https://packages.typst.org/preview/modern-unito-thesis-0.1.0.tar.gz}{\pandocbounded{\includesvg[keepaspectratio]{/assets/icons/16-download.svg}}}
\item[Repository:]
\href{https://github.com/eduardz1/unito-typst-template}{GitHub}
\item[Categor y :]
\begin{itemize}
\tightlist
\item[]
\item
  \pandocbounded{\includesvg[keepaspectratio]{/assets/icons/16-mortarboard.svg}}
  \href{https://typst.app/universe/search/?category=thesis}{Thesis}
\end{itemize}
\end{description}

\subsubsection{Where to report issues?}\label{where-to-report-issues}

This template is a project of Eduard Antonovic Occhipinti . Report
issues on \href{https://github.com/eduardz1/unito-typst-template}{their
repository} . You can also try to ask for help with this template on the
\href{https://forum.typst.app}{Forum} .

Please report this template to the Typst team using the
\href{https://typst.app/contact}{contact form} if you believe it is a
safety hazard or infringes upon your rights.

\phantomsection\label{versions}
\subsubsection{Version history}\label{version-history}

\begin{longtable}[]{@{}ll@{}}
\toprule\noalign{}
Version & Release Date \\
\midrule\noalign{}
\endhead
\bottomrule\noalign{}
\endlastfoot
0.1.0 & March 20, 2024 \\
\end{longtable}

Typst GmbH did not create this template and cannot guarantee correct
functionality of this template or compatibility with any version of the
Typst compiler or app.


\section{Package List LaTeX/wenyuan-campaign.tex}
\title{typst.app/universe/package/wenyuan-campaign}

\phantomsection\label{banner}
\phantomsection\label{template-thumbnail}
\pandocbounded{\includegraphics[keepaspectratio]{https://packages.typst.org/preview/thumbnails/wenyuan-campaign-0.1.0-small.webp}}

\section{wenyuan-campaign}\label{wenyuan-campaign}

{ 0.1.0 }

Easily write DnD 5e style campaign documents.

\href{/app?template=wenyuan-campaign&version=0.1.0}{Create project in
app}

\phantomsection\label{readme}
A template for writing RPG campaigns imitating the 5e theme. This was
made as a typst version of the \$\textbackslash LaTeX\$ package
\href{https://github.com/rpgtex/DND-5e-LaTeX-Template}{DnD 5e LaTeX
Template} , though it is not functionally nor entirely visually similar.

Packages:

\begin{itemize}
\tightlist
\item
  \texttt{\ droplet:0.3.1\ }
\end{itemize}

Fonts:

\begin{itemize}
\tightlist
\item
  TeX Gyre Bonum
\item
  Scaly Sans
\item
  Scaly Sans Caps
\item
  Royal Initalen
\item
  京è?¯è€?宋ä½`` KingHwa OldSong
\end{itemize}

\emph{\textbf{Please note: in an effort to reduce the file size of the
template, fonts are included in MY repository only, not in the typst
official one.}} You may find the fonts in my
\href{https://github.com/yanwenywan/typst-packages/tree/master/wenyuan-campaign/0.1.0/template/fonts}{github
repository in the fonts folder} , or download them yourself, or heck
provide your own fonts to your liking.

\begin{verbatim}
typst init @preview/wenyuan-campaign:0.1.0
\end{verbatim}

This will copy over all required fonts and comes prefilled with the
standard template so you can see how it works. To use this you need to
either install all the fonts locally or pass the folder into
-\/-font-path when compiling.

To initialise the style, do:

\begin{Shaded}
\begin{Highlighting}[]
\NormalTok{\#import "@preview/wenyuan{-}campaign:0.1.0": *}

\NormalTok{\#show: conf.with() }
\end{Highlighting}
\end{Shaded}

Very easy.

Optionally, you may set all the theme fonts from the configure function
(the defaults are shown):

\begin{Shaded}
\begin{Highlighting}[]
\NormalTok{\#import "@preview/wenyuan{-}campaign:0.1.0": *}

\NormalTok{\#show: conf.with(}
\NormalTok{    fontsize: 10pt,}
\NormalTok{    mainFont: ("TeX Gyre Bonum", "KingHwa\_OldSong"),}
\NormalTok{    titleFont: ("TeX Gyre Bonum", "KingHwa\_OldSong"),}
\NormalTok{    sansFont: ("Scaly Sans Remake", "KingHwa\_OldSong"),}
\NormalTok{    sansSmallcapsFont: ("Scaly Sans Caps", "KingHwa\_OldSong"),}
\NormalTok{    dropcapFont: "Royal Initialen"}
\NormalTok{) }
\end{Highlighting}
\end{Shaded}

You are encouraged to copy the template files and modify them if they
are not up to your liking.

\textbf{set-theme-colour} \texttt{\ (colour:\ color)\ } .\\
Sets a theme colour from the colours package of this module or any other
colour you wantâ€''up to you if it looks bad :)\\
The colours recommended are:

\begin{quote}
phbgreen, phbcyan, phbmauve, phbtan, dmglavender, dmgcoral, dmgslategrey
(-ay), dmglilac
\end{quote}

\begin{center}\rule{0.5\linewidth}{0.5pt}\end{center}

\textbf{make-title}
\texttt{\ (title:\ content,\ subtitle:\ content\ =\ {[}{]},\ author:\ content\ =\ {[}{]},\ date:\ content\ =\ {[}{]},\ anything-before:\ content\ =\ {[}{]},\ anything-after:\ content\ =\ {[}{]})\ }
.\\
Makes a simple title page.

Parameters:

\begin{itemize}
\tightlist
\item
  title: main book title
\item
  subtitle: (optional) subtitle
\item
  author: (optional)
\item
  date: (optional) â€`` just acts as a separate line, can be used for
  anything else
\item
  anything-before: (optional) this is put before the title
\item
  anything-after: (optional) this is put after the date
\end{itemize}

\begin{center}\rule{0.5\linewidth}{0.5pt}\end{center}

\textbf{drop-paragraph}
\texttt{\ (small-caps:\ string\ =\ "",\ body:\ content)\ } .\\
Makes a paragraph with a drop capital.

Parameters:

\begin{itemize}
\tightlist
\item
  small-caps: (optional) any text you wish to be rendered in small caps,
  like how DnD does it
\item
  body: anything else
\end{itemize}

\begin{center}\rule{0.5\linewidth}{0.5pt}\end{center}

\textbf{bump} \texttt{\ ()\ } .\\
Manually does a 1em paragraph space

\begin{center}\rule{0.5\linewidth}{0.5pt}\end{center}

\textbf{namedpar} \texttt{\ (title:\ content,\ content:\ content)\ } .\\
A paragraph with a bold italic name at the start.

Parameters:

\begin{itemize}
\tightlist
\item
  title: the bold italic name; a full stop/period is added immediately
  after for you
\item
  content: everything else
\end{itemize}

\begin{center}\rule{0.5\linewidth}{0.5pt}\end{center}

\textbf{namedpar-block}
\texttt{\ (title:\ content,\ content:\ content)\ } .\\
See \textbf{namedpar} , but this one is in a block environment.

\begin{center}\rule{0.5\linewidth}{0.5pt}\end{center}

\textbf{readaloud} \texttt{\ {[}{]}\ } .\\
A tan-coloured read-aloud box with some decorations.

\begin{center}\rule{0.5\linewidth}{0.5pt}\end{center}

\textbf{comment-box}
\texttt{\ (title:\ content\ =\ {[}{]},\ content:\ content)\ } .\\
A theme-coloured plain comment box.

Parameters:

\begin{itemize}
\tightlist
\item
  title: (optional) a title in bold small caps
\item
  content
\end{itemize}

\begin{center}\rule{0.5\linewidth}{0.5pt}\end{center}

\textbf{fancy-comment-box}
\texttt{\ (title:\ content\ =\ {[}{]},\ content:\ content)\ } .\\
A theme-coloured fancy comment box with decorations.

Parameters:

\begin{itemize}
\tightlist
\item
  title: (optional) a title in bold small caps
\item
  content
\end{itemize}

\begin{center}\rule{0.5\linewidth}{0.5pt}\end{center}

\textbf{dndtable} \texttt{\ (...)\ } . A theme-coloured dnd-style table.
Parameters are identical to table except stroke, fill, and inset are not
included.

\begin{center}\rule{0.5\linewidth}{0.5pt}\end{center}

\textbf{sctitle} \texttt{\ {[}{]}\ } .\\
Makes a small caps header block.

\begin{center}\rule{0.5\linewidth}{0.5pt}\end{center}

\textbf{begin-stat} \texttt{\ {[}{]}\ } .\\
Begins the monster statblock environment.

\begin{center}\rule{0.5\linewidth}{0.5pt}\end{center}

\textbf{begin-item} \texttt{\ {[}{]}\ } .\\
Begins the item environment.

\emph{\textbf{Important.}} Statblocks are provided under the
\texttt{\ stat\ } namespace, and will only work as intended in a
\texttt{\ beginStat\ } block. All statblock functions must be prepended
with \texttt{\ stat\ } .

\subsection{stat functions}\label{stat-functions}

\textbf{dice} \texttt{\ (value:\ str)\ }\\
Parses a dice string (e.g., \texttt{\ 3d6\ } , \texttt{\ 3d6+2\ } , or
\texttt{\ 3d6-1\ } ) and returns a formatted dice value (e.g., “10
(3d6)�). Specifically, the types of strings it accepts are:

\begin{quote}
\texttt{\ \textbackslash{}d+d\textbackslash{}d+({[}+-{]}\textbackslash{}d+)?\ }
(number \texttt{\ d\ } number \texttt{\ +/-\ } number)
\end{quote}

(You need to ensure the string is correct.)

\begin{center}\rule{0.5\linewidth}{0.5pt}\end{center}

\textbf{dice-raw}
\texttt{\ (num-dice:\ int,\ dice-face:\ int,\ modifier:\ int)\ }\\
A helper function for the above, optionally used. It takes all values as
integers and prints the correct formatting.

\begin{center}\rule{0.5\linewidth}{0.5pt}\end{center}

\textbf{statheading} \texttt{\ (title,\ desc\ =\ {[}{]})\ }\\
Takes a title and description, formatting it into a top-level monster
name heading. \texttt{\ desc\ } is the description of the monster, e.g.,
\emph{Medium humanoid, neutral evil} , but it can be anything.

\begin{center}\rule{0.5\linewidth}{0.5pt}\end{center}

\textbf{stroke} \texttt{\ ()\ } .\\
Draws a red stroke with a fading right edge.

\begin{center}\rule{0.5\linewidth}{0.5pt}\end{center}

\textbf{main-stats}
\texttt{\ (ac\ =\ "",\ hp-dice\ =\ "",\ speed\ =\ "30ft",\ hp-etc\ =\ "")\ }\\
Produces \textbf{Armor Class} , \textbf{Hit Points} , and \textbf{Speed}
in one go. All fields are optional. \texttt{\ hp-dice\ } accepts a
\emph{valid dice string only} â€''if you do not want to use dice, leave
it blank and use \texttt{\ hp-etc\ } . No restrictions on other fields.

\begin{center}\rule{0.5\linewidth}{0.5pt}\end{center}

\textbf{ability} \texttt{\ (str,\ dex,\ con,\ int,\ wis,\ cha)\ }\\
Takes the six ability scores (base values) as integers and formats them
into a table with appropriate modifiers.

\begin{center}\rule{0.5\linewidth}{0.5pt}\end{center}

\textbf{challenge} \texttt{\ (cr:\ str)\ }\\
Takes a numeric challenge rating (as a string) and formats it along with
the XP (if the challenge rating is valid). All CRs between 0â€``30 are
valid, including the fractional \texttt{\ 1/8\ } , \texttt{\ 1/4\ } ,
\texttt{\ 1/2\ } (which can also be written in decimal form, e.g.,
\texttt{\ 0.125\ } ).

\begin{center}\rule{0.5\linewidth}{0.5pt}\end{center}

\textbf{skill} \texttt{\ (title,\ contents)\ }\\
Takes a title and description, creating a single skills entry. For
example, \texttt{\ \#skill("Challenge",\ challenge(1))\ } will produce
(in red):

\begin{quote}
\textbf{Challenge} 1 (200 XP)
\end{quote}

(This uses \texttt{\ challenge\ } from above.)

\textbf{Section headers} such as \emph{Actions} or \emph{Reactions} are
done using the second-level header \texttt{\ ==\ }

\textbf{Action names} â€`` the names that go in front of actions /
abilities are done using the third level header \texttt{\ ===\ } (do not
leave a blank line between the header and its body text)

\emph{\textbf{Important.}} Basic item capability is provided under the
\texttt{\ item\ } namespace, and will only work as intended in a
\texttt{\ beginItem\ } block. All item functions must be prepended with
\texttt{\ item\ } .

\subsection{item functions}\label{item-functions}

\textbf{Item Name} is done with the top-level header \texttt{\ =\ }

\textbf{Section headers} are the second level header \texttt{\ ==\ }

\textbf{Abilities and named paragraphs} are the third level header
\texttt{\ ===\ }

\begin{center}\rule{0.5\linewidth}{0.5pt}\end{center}

\textbf{smalltext} \texttt{\ {[}{]}\ } . Half-size text for item
subheadings

\textbf{flavourtext} \texttt{\ {[}{]}\ } . Indented italic flavour text

\pandocbounded{\includegraphics[keepaspectratio]{https://github.com/typst/packages/raw/main/packages/preview/wenyuan-campaign/0.1.0/sample.png}}

\begin{itemize}
\tightlist
\item
  The overall style is based on the
  \href{https://github.com/rpgtex/DND-5e-LaTeX-Template}{Dnd 5e LaTeX
  Template} , which in turn replicate the base DnD aesthetic.
\item
  TeX Gyre Bonum by GUST e-Foundry is used for the body text
\item
  Scaly Sans and Scaly Sans Caps are part of
  \href{https://github.com/jonathonf/solbera-dnd-fonts}{Solbera’s CC
  Alternatives to DnD Fonts} and are used for main body text.
  \emph{\textbf{Note that these fonts are CC-BY-SA i.e. Share-Alike, so
  keep that in mind. This shouldn’t affect homebrew created using
  these fonts (just like how a painting made with a CC-BY-SA art program
  isn’t itself CC-BY-SA) but what do I know I’m not a lawyer.}}
\item
  \href{https://zhuanlan.zhihu.com/p/637491623}{KingHwa\_OldSong}
  (京è?¯è€?宋ä½``) is a traditional Chinese print font used for all
  CJK text (if present, mostly because I need it)
\end{itemize}

\href{/app?template=wenyuan-campaign&version=0.1.0}{Create project in
app}

\subsubsection{How to use}\label{how-to-use}

Click the button above to create a new project using this template in
the Typst app.

You can also use the Typst CLI to start a new project on your computer
using this command:

\begin{verbatim}
typst init @preview/wenyuan-campaign:0.1.0
\end{verbatim}

\includesvg[width=0.16667in,height=0.16667in]{/assets/icons/16-copy.svg}

\subsubsection{About}\label{about}

\begin{description}
\tightlist
\item[Author :]
\href{https://github.com/yanwenywan}{Yan Xin}
\item[License:]
Apache-2.0
\item[Current version:]
0.1.0
\item[Last updated:]
November 28, 2024
\item[First released:]
November 28, 2024
\item[Archive size:]
403 kB
\href{https://packages.typst.org/preview/wenyuan-campaign-0.1.0.tar.gz}{\pandocbounded{\includesvg[keepaspectratio]{/assets/icons/16-download.svg}}}
\item[Repository:]
\href{https://github.com/yanwenywan/typst-packages/tree/master/wenyuan-campaign}{GitHub}
\item[Categor ies :]
\begin{itemize}
\tightlist
\item[]
\item
  \pandocbounded{\includesvg[keepaspectratio]{/assets/icons/16-layout.svg}}
  \href{https://typst.app/universe/search/?category=layout}{Layout}
\item
  \pandocbounded{\includesvg[keepaspectratio]{/assets/icons/16-text.svg}}
  \href{https://typst.app/universe/search/?category=text}{Text}
\item
  \pandocbounded{\includesvg[keepaspectratio]{/assets/icons/16-docs.svg}}
  \href{https://typst.app/universe/search/?category=book}{Book}
\end{itemize}
\end{description}

\subsubsection{Where to report issues?}\label{where-to-report-issues}

This template is a project of Yan Xin . Report issues on
\href{https://github.com/yanwenywan/typst-packages/tree/master/wenyuan-campaign}{their
repository} . You can also try to ask for help with this template on the
\href{https://forum.typst.app}{Forum} .

Please report this template to the Typst team using the
\href{https://typst.app/contact}{contact form} if you believe it is a
safety hazard or infringes upon your rights.

\phantomsection\label{versions}
\subsubsection{Version history}\label{version-history}

\begin{longtable}[]{@{}ll@{}}
\toprule\noalign{}
Version & Release Date \\
\midrule\noalign{}
\endhead
\bottomrule\noalign{}
\endlastfoot
0.1.0 & November 28, 2024 \\
\end{longtable}

Typst GmbH did not create this template and cannot guarantee correct
functionality of this template or compatibility with any version of the
Typst compiler or app.


\section{Package List LaTeX/scienceicons.tex}
\title{typst.app/universe/package/scienceicons}

\phantomsection\label{banner}
\section{scienceicons}\label{scienceicons}

{ 0.0.6 }

SVG icons for open-science articles

\phantomsection\label{readme}
SVG icons for open-science articles

\subsection{Usage}\label{usage}

\begin{Shaded}
\begin{Highlighting}[]
\NormalTok{\#import "@preview/scienceicons:0.0.6": open{-}access{-}icon}

\NormalTok{This article is Open Access \#open{-}access{-}icon(color: orange, height: 1.1em, baseline: 20\%)}
\end{Highlighting}
\end{Shaded}

\pandocbounded{\includegraphics[keepaspectratio]{https://raw.githubusercontent.com/curvenote/scienceicons/main/typst/docs/example.png?raw=true}}

\subsection{Arguments}\label{arguments}

The arguments for each icon are:

\begin{itemize}
\tightlist
\item
  \texttt{\ color\ } : A typst color, \texttt{\ red\ } ,
  \texttt{\ red.darken(20\%)\ } , \texttt{\ color(\#FF0000)\ } , etc.
  Default is \texttt{\ black\ } .
\item
  \texttt{\ height\ } : The height of the icon, by default this is
  slightly larger than the text height at \texttt{\ 1.1em\ }
\item
  \texttt{\ baseline\ } : Change the baseline of the box surrounding the
  icon, moving the icon up and down. Default is \texttt{\ 13.5\%\ } .
\end{itemize}

Additionally the raw SVG text for each icon can be found by replacing
\texttt{\ Icon\ } with \texttt{\ Svg\ } .

\subsection{List of Icons}\label{list-of-icons}

\begin{itemize}
\tightlist
\item
  arxiv-icon
\item
  cc-by-icon
\item
  cc-nc-icon
\item
  cc-nd-icon
\item
  cc-sa-icon
\item
  cc-zero-icon
\item
  cc-icon
\item
  curvenote-icon
\item
  discord-icon
\item
  email-icon
\item
  github-icon
\item
  jupyter-book-icon
\item
  jupyter-text-icon
\item
  jupyter-icon
\item
  linkedin-icon
\item
  mastodon-icon
\item
  myst-icon
\item
  open-access-icon
\item
  orcid-icon
\item
  osi-icon
\item
  ror-icon
\item
  slack-icon
\item
  twitter-icon
\item
  website-icon
\item
  youtube-icon
\end{itemize}

\subsection{See All Icons}\label{see-all-icons}

You can browse and see all icons here:

\url{https://github.com/curvenote/scienceicons/tree/main/typst/docs/scienceicons.pdf}

\pandocbounded{\includegraphics[keepaspectratio]{https://raw.githubusercontent.com/curvenote/scienceicons/main/typst/docs/icons.png?raw=true}}

\subsection{Contributing}\label{contributing}

To add or request an icon to be added to this package see:\\
\url{https://github.com/curvenote/scienceicons}

\subsubsection{How to add}\label{how-to-add}

Copy this into your project and use the import as
\texttt{\ scienceicons\ }

\begin{verbatim}
#import "@preview/scienceicons:0.0.6"
\end{verbatim}

\includesvg[width=0.16667in,height=0.16667in]{/assets/icons/16-copy.svg}

Check the docs for
\href{https://typst.app/docs/reference/scripting/\#packages}{more
information on how to import packages} .

\subsubsection{About}\label{about}

\begin{description}
\tightlist
\item[Author :]
rowanc1
\item[License:]
MIT
\item[Current version:]
0.0.6
\item[Last updated:]
January 26, 2024
\item[First released:]
January 26, 2024
\item[Archive size:]
7.73 kB
\href{https://packages.typst.org/preview/scienceicons-0.0.6.tar.gz}{\pandocbounded{\includesvg[keepaspectratio]{/assets/icons/16-download.svg}}}
\item[Repository:]
\href{https://github.com/curvenote/scienceicons}{GitHub}
\end{description}

\subsubsection{Where to report issues?}\label{where-to-report-issues}

This package is a project of rowanc1 . Report issues on
\href{https://github.com/curvenote/scienceicons}{their repository} . You
can also try to ask for help with this package on the
\href{https://forum.typst.app}{Forum} .

Please report this package to the Typst team using the
\href{https://typst.app/contact}{contact form} if you believe it is a
safety hazard or infringes upon your rights.

\phantomsection\label{versions}
\subsubsection{Version history}\label{version-history}

\begin{longtable}[]{@{}ll@{}}
\toprule\noalign{}
Version & Release Date \\
\midrule\noalign{}
\endhead
\bottomrule\noalign{}
\endlastfoot
0.0.6 & January 26, 2024 \\
\end{longtable}

Typst GmbH did not create this package and cannot guarantee correct
functionality of this package or compatibility with any version of the
Typst compiler or app.


\section{Package List LaTeX/silver-dev-cv.tex}
\title{typst.app/universe/package/silver-dev-cv}

\phantomsection\label{banner}
\phantomsection\label{template-thumbnail}
\pandocbounded{\includegraphics[keepaspectratio]{https://packages.typst.org/preview/thumbnails/silver-dev-cv-1.0.1-small.webp}}

\section{silver-dev-cv}\label{silver-dev-cv}

{ 1.0.1 }

A CV template by an engineer-recruiter, used by https://silver.dev

\href{/app?template=silver-dev-cv&version=1.0.1}{Create project in app}

\phantomsection\label{readme}
This Typst CV template is a streamlined version of the the Latex
template \href{https://github.com/jxpeng98/Typst-CV-Resume}{Modernpro} .

\subsection{How to start}\label{how-to-start}

\subsubsection{Use Typst CLI}\label{use-typst-cli}

If you use Typst CLI, you can use the following command to create a new
project:

\begin{Shaded}
\begin{Highlighting}[]
\ExtensionTok{typst}\NormalTok{ init silver{-}dev{-}cv}
\end{Highlighting}
\end{Shaded}

It will create a folder named \texttt{\ silver-dev-cv\ } with the
following structure:

\begin{Shaded}
\begin{Highlighting}[]
\NormalTok{silver{-}dev{-}cv}
\NormalTok{└── cv.typ}
\end{Highlighting}
\end{Shaded}

\subsubsection{Typst website}\label{typst-website}

If you want to use the template via \href{https://typst.app/}{Typst} ,
You can \texttt{\ start\ from\ template\ } and search for
\texttt{\ silver-dev-cv\ } .

\subsection{How to use the template}\label{how-to-use-the-template}

\subsubsection{The arguments}\label{the-arguments}

The template has the following arguments:

\begin{longtable}[]{@{}lll@{}}
\toprule\noalign{}
Argument & Description & Default \\
\midrule\noalign{}
\endhead
\bottomrule\noalign{}
\endlastfoot
\texttt{\ font-type\ } & The font type. You can choose any supported
font in your system. & \texttt{\ Times\ New\ Roman\ } \\
\texttt{\ continue-header\ } & Whether to continue the header on the
follwing pages. & \texttt{\ false\ } \\
\texttt{\ name\ } & Your name. & \texttt{\ ""\ } \\
\texttt{\ address\ } & Your address. & \texttt{\ ""\ } \\
\texttt{\ lastupdated\ } & Whether to show the last updated date. &
\texttt{\ true\ } \\
\texttt{\ pagecount\ } & Whether to show the page count. &
\texttt{\ true\ } \\
\texttt{\ date\ } & The date of the CV. & \texttt{\ today\ } \\
\texttt{\ contacts\ } & contact details, e.g phone number, email, etc. &
\texttt{\ (text:\ "",\ link:\ "")\ } \\
\end{longtable}

\subsubsection{Starting the CV}\label{starting-the-cv}

\begin{Shaded}
\begin{Highlighting}[]
\NormalTok{\#import "@preview/silver{-}dev{-}cv:1.0.0": *}

\NormalTok{\#show: cv.with(}
\NormalTok{  font{-}type: "PT Serif",}
\NormalTok{  continue{-}header: "false",}
\NormalTok{  name: "",}
\NormalTok{  address: "",}
\NormalTok{  lastupdated: "true",}
\NormalTok{  pagecount: "true",}
\NormalTok{  date: "2024{-}07{-}03",}
\NormalTok{  contacts: (}
\NormalTok{    (text: "08856", link: ""),}
\NormalTok{    (text: "example.com", link: "https://www.example.com"),}
\NormalTok{    (text: "github.com", link: "https://www.github.com"),}
\NormalTok{    (text: "123@example.com", link: "mailto:123@example.com"),}
\NormalTok{  )}
\NormalTok{)}
\end{Highlighting}
\end{Shaded}

\subsubsection{Content}\label{content}

Once you set up the arguments, you can start to add details to your CV /
Resume.

I preset the following functions for you to create different parts:

\begin{longtable}[]{@{}ll@{}}
\toprule\noalign{}
Function & Description \\
\midrule\noalign{}
\endhead
\bottomrule\noalign{}
\endlastfoot
\texttt{\ \#section("Section\ Name")\ } & Start a new section \\
\texttt{\ \#sectionsep\ } & End the section \\
\texttt{\ \#oneline-title-item(title:\ "",\ content:\ "")\ } & Add a
one-line item ( \textbf{Title:} content) \\
\texttt{\ \#oneline-two(entry1:\ "",\ entry2:\ "")\ } & Add a one-line
item with two entries, aligned left and right \\
\texttt{\ \#descript("descriptions")\ } & Add a description for
self-introduction \\
\texttt{\ \#award(award:\ "",\ date:\ "",\ institution:\ "")\ } & Add an
award ( \textbf{award} , \emph{institution} \emph{date} ) \\
\texttt{\ \#education(institution:\ "",\ major:\ "",\ date:\ "",\ institution:\ "",\ core-modules:\ "")\ }
& Add an education experience \\
\texttt{\ \#job(position:\ "",\ institution:\ "",\ location:\ "",\ date:\ "",\ description:\ {[}{]})\ }
& Add a job experience (description is optional) \\
\texttt{\ \#twoline-item(entry1:\ "",\ entry2:\ "",\ entry3:\ "",\ entry4:\ "")\ }
& Two line items, similar to education and job experiences \\
\end{longtable}

\subsection{License}\label{license}

The template is released under the MIT License. For more information,
please refer to the
\href{https://github.com/jxpeng98/Typst-CV-Resume/blob/main/LICENSE}{LICENSE}
file.

\href{/app?template=silver-dev-cv&version=1.0.1}{Create project in app}

\subsubsection{How to use}\label{how-to-use}

Click the button above to create a new project using this template in
the Typst app.

You can also use the Typst CLI to start a new project on your computer
using this command:

\begin{verbatim}
typst init @preview/silver-dev-cv:1.0.1
\end{verbatim}

\includesvg[width=0.16667in,height=0.16667in]{/assets/icons/16-copy.svg}

\subsubsection{About}\label{about}

\begin{description}
\tightlist
\item[Author s :]
Gabriel Benmergui \& Santiago Barraza
\item[License:]
MIT
\item[Current version:]
1.0.1
\item[Last updated:]
November 26, 2024
\item[First released:]
October 31, 2024
\item[Archive size:]
4.13 kB
\href{https://packages.typst.org/preview/silver-dev-cv-1.0.1.tar.gz}{\pandocbounded{\includesvg[keepaspectratio]{/assets/icons/16-download.svg}}}
\item[Categor y :]
\begin{itemize}
\tightlist
\item[]
\item
  \pandocbounded{\includesvg[keepaspectratio]{/assets/icons/16-user.svg}}
  \href{https://typst.app/universe/search/?category=cv}{CV}
\end{itemize}
\end{description}

\subsubsection{Where to report issues?}\label{where-to-report-issues}

This template is a project of Gabriel Benmergui and Santiago Barraza .
You can also try to ask for help with this template on the
\href{https://forum.typst.app}{Forum} .

Please report this template to the Typst team using the
\href{https://typst.app/contact}{contact form} if you believe it is a
safety hazard or infringes upon your rights.

\phantomsection\label{versions}
\subsubsection{Version history}\label{version-history}

\begin{longtable}[]{@{}ll@{}}
\toprule\noalign{}
Version & Release Date \\
\midrule\noalign{}
\endhead
\bottomrule\noalign{}
\endlastfoot
1.0.1 & November 26, 2024 \\
\href{https://typst.app/universe/package/silver-dev-cv/1.0.0/}{1.0.0} &
October 31, 2024 \\
\end{longtable}

Typst GmbH did not create this template and cannot guarantee correct
functionality of this template or compatibility with any version of the
Typst compiler or app.


\section{Package List LaTeX/polytonoi.tex}
\title{typst.app/universe/package/polytonoi}

\phantomsection\label{banner}
\section{polytonoi}\label{polytonoi}

{ 0.1.0 }

Renders Roman letters into polytonic Greek.

\phantomsection\label{readme}
A typst package for rendering text into polytonic Greek using a
hopefully-intuitive transliteration scheme.

\subsection{Usage}\label{usage}

First, be sure you include the package at the top of your typst file:

\begin{Shaded}
\begin{Highlighting}[]
\NormalTok{@import "preview/polytonoi@0.1.0: *}
\end{Highlighting}
\end{Shaded}

The package currently exposes one function,
\texttt{\ \#ptgk(\textless{}string\textgreater{})\ } , which will
convert \texttt{\ \textless{}string\textgreater{}\ } into polytonic
Greek text in the same location where the function appears in the typst
document.

For example: \texttt{\ \#ptgk("polu/s")\ } would render: πολÏ?Ï‚

\textbf{NOTE:} Quotation marks within the function call (see above
example) are \textbf{mandatory} , and the code will not work without
them.

Where possible, Greek letters have been linked with their closest Roman
equivalent (e.g. a -\/-\textgreater{} α, b -\/-\textgreater{} β).
Where not possible, I’ve tried to stick to common convention, such as
what is used by the Perseus Project for their transliteration. A couple
letters (ξ and ψ) are made up of two letters ( \texttt{\ ks\ } and
\texttt{\ ps\ } respectively), which the \texttt{\ \#ptgk()\ } function
handles as special cases. See below for the full transliteration scheme.

\paragraph{Additional Usage Notes}\label{additional-usage-notes}

\begin{enumerate}
\tightlist
\item
  Any character that isn’t specifically accounted for (including white
  space, most punctuation, numbers, etc.) is rendered as-is.
\item
  Smooth breathing marks are automatically added to a vowel that begins
  a word, unless that first vowel is followed by another. In this case,
  you’ll need to manually add it to the second vowel.
\end{enumerate}

\subsubsection{Text Formatting}\label{text-formatting}

As of now, the text is processed as a string, which means that any
formatting markup (such as \texttt{\ \_\ } or \texttt{\ *\ } ) is
\textbf{not} included in how the text is rendered, and is instead passed
through and will display literally. To apply any kind of formatting to
the Greek text, the markup or commands must be put outside the text
passed to the function. Compare the following two examples to see how
this works:

\texttt{\ \#ptgk("\_Arxh\textbackslash{}\_")\ } would display as
\_ἈÏ?χὴ\_

whereas

\texttt{\ \_\#ptgk("Arxh\textbackslash{}")\_\ } would display as
\emph{ἈÏ?χὴ}

\subsection{Transliteration Scheme}\label{transliteration-scheme}

\begin{longtable}[]{@{}lll@{}}
\toprule\noalign{}
Roman letter & Greek result & Notes \\
\midrule\noalign{}
\endhead
\bottomrule\noalign{}
\endlastfoot
a & α & \\
b & β & \\
g & γ & \\
d & δ & \\
e & ε & \\
z & ζ & \\
h & η & \\
q & θ & \\
i & ι & \\
k & κ & \\
l & λ & \\
m & μ & \\
n & ν & \\
ks & ξ & \\
o & ο & \\
p & π & \\
r & Ï? & \\
s & σ/ς & Renders as final sigma automatically \\
t & Ï„ & \\
u & Ï & \\
v & φ & \\
x & χ & \\
ps & ψ & \\
w & ω & \\
\end{longtable}

Upper-case letters are handled the same way, just with an upper-case
letter as input. The upper-case versions of the two “combined�
letters (Ξ and Ψ) can be entered either as “KS�/“PS� or
“Ks�/“Ps�.

\subsubsection{Accents \& Breathing
Marks}\label{accents-breathing-marks}

As mentioned above, this package will automatically put a smooth
breathing mark on a vowel that begins a word, unless that vowel is
followed immediately by a second vowel. In that instance, you’ll have
to manually put the smooth breathing mark in its correct place. (This is
to avoid having to code for edge cases, such as where a word starts with
three vowels in a row.) By the same token, rough breathing must always
be entered manually.

\begin{longtable}[]{@{}llll@{}}
\toprule\noalign{}
Input & Greek & Notes & Example \\
\midrule\noalign{}
\endhead
\bottomrule\noalign{}
\endlastfoot
( & rough breathing & Put before the vowel & \texttt{\ (a\ }
-\/-\textgreater{} á¼? \\
) & smooth breathing & Put before the vowel & \texttt{\ )a\ }
-\/-\textgreater{} á¼€ \\
\textbackslash{} & Grave / varia & Put after the vowel &
\texttt{\ a\textbackslash{}\ } -\/-\textgreater{} á½° \\
/ & Acute / oxia / tonos & Put after the vowel & \texttt{\ a/\ }
-\/-\textgreater{} ά \\
= & Tilde / perispomeni & Put after the vowel & \texttt{\ a=\ }
-\/-\textgreater{} ᾶ \\
\textbar{} & Iota subscript & Put after the vowel &
\texttt{\ a\textbar{}\ } -\/-\textgreater{} á¾³ \\
: & Diaresis & Put after the vowel & \texttt{\ i:\ } -\/-\textgreater{}
ÏŠ \\
\end{longtable}

Multiple diacriticals can be added to a vowel, e.g.
\texttt{\ (h\textbar{}\ } -\/-\textgreater{} á¾`

\subsubsection{Punctuation}\label{punctuation}

Most Roman punctuation characters are left unchanged. The exceptions are
\texttt{\ ;\ } (semicolon) and \texttt{\ ?\ } (question mark), which are
rendered as a high dot (·) and the Greek question mark (;)
respectively.

\subsection{Feedback}\label{feedback}

Feedback is welcome, and please don’t hesitate to open an issue if
something doesn’t work. I’ve tried to account for edge cases, but I
certainly can’t guarantee that I’ve found everything.

\subsubsection{How to add}\label{how-to-add}

Copy this into your project and use the import as \texttt{\ polytonoi\ }

\begin{verbatim}
#import "@preview/polytonoi:0.1.0"
\end{verbatim}

\includesvg[width=0.16667in,height=0.16667in]{/assets/icons/16-copy.svg}

Check the docs for
\href{https://typst.app/docs/reference/scripting/\#packages}{more
information on how to import packages} .

\subsubsection{About}\label{about}

\begin{description}
\tightlist
\item[Author :]
Dei Layborer
\item[License:]
GPL-3.0-only
\item[Current version:]
0.1.0
\item[Last updated:]
December 28, 2023
\item[First released:]
December 28, 2023
\item[Archive size:]
15.6 kB
\href{https://packages.typst.org/preview/polytonoi-0.1.0.tar.gz}{\pandocbounded{\includesvg[keepaspectratio]{/assets/icons/16-download.svg}}}
\item[Repository:]
\href{https://github.com/dei-layborer/polytonoi}{GitHub}
\end{description}

\subsubsection{Where to report issues?}\label{where-to-report-issues}

This package is a project of Dei Layborer . Report issues on
\href{https://github.com/dei-layborer/polytonoi}{their repository} . You
can also try to ask for help with this package on the
\href{https://forum.typst.app}{Forum} .

Please report this package to the Typst team using the
\href{https://typst.app/contact}{contact form} if you believe it is a
safety hazard or infringes upon your rights.

\phantomsection\label{versions}
\subsubsection{Version history}\label{version-history}

\begin{longtable}[]{@{}ll@{}}
\toprule\noalign{}
Version & Release Date \\
\midrule\noalign{}
\endhead
\bottomrule\noalign{}
\endlastfoot
0.1.0 & December 28, 2023 \\
\end{longtable}

Typst GmbH did not create this package and cannot guarantee correct
functionality of this package or compatibility with any version of the
Typst compiler or app.


\section{Package List LaTeX/chordx.tex}
\title{typst.app/universe/package/chordx}

\phantomsection\label{banner}
\section{chordx}\label{chordx}

{ 0.5.0 }

A package to write song lyrics with chord diagrams in Typst.

\phantomsection\label{readme}
A package to write song lyrics with chord diagrams in Typst.

\textbf{Table of Contents}

\begin{itemize}
\tightlist
\item
  \href{https://github.com/typst/packages/raw/main/packages/preview/chordx/0.5.0/\#introduction}{Introduction}
\item
  \href{https://github.com/typst/packages/raw/main/packages/preview/chordx/0.5.0/\#usage}{Usage}

  \begin{itemize}
  \tightlist
  \item
    \href{https://github.com/typst/packages/raw/main/packages/preview/chordx/0.5.0/\#typst-packages}{Typst
    Packages}
  \item
    \href{https://github.com/typst/packages/raw/main/packages/preview/chordx/0.5.0/\#local-packages}{Local
    Packages}
  \end{itemize}
\item
  \href{https://github.com/typst/packages/raw/main/packages/preview/chordx/0.5.0/\#documentation}{Documentation}
\item
  \href{https://github.com/typst/packages/raw/main/packages/preview/chordx/0.5.0/\#examples}{Examples}

  \begin{itemize}
  \tightlist
  \item
    \href{https://github.com/typst/packages/raw/main/packages/preview/chordx/0.5.0/\#chart-chords}{Chart
    Chords}
  \item
    \href{https://github.com/typst/packages/raw/main/packages/preview/chordx/0.5.0/\#piano-chords}{Piano
    Chords}
  \item
    \href{https://github.com/typst/packages/raw/main/packages/preview/chordx/0.5.0/\#single-chords}{Single
    Chords}
  \end{itemize}
\item
  \href{https://github.com/typst/packages/raw/main/packages/preview/chordx/0.5.0/\#changelog}{Changelog}
\item
  \href{https://github.com/typst/packages/raw/main/packages/preview/chordx/0.5.0/\#license}{License}
\end{itemize}

\subsection{Introduction}\label{introduction}

With \texttt{\ chordx\ } you can easily generate song lyrics with chords
for writing songbooks.

\texttt{\ chordx\ } generates chord charts for stringed instruments
(e.g. guitar, ukulele, etc.), piano chords (with diferent piano layouts)
and single chords that are chords without charts used to write the
chords over a word to write songbooks.

\subsection{Usage}\label{usage}

\texttt{\ chordx\ } exports 3 functions to generate diferents types fo
charts:

\begin{itemize}
\tightlist
\item
  \texttt{\ chart-chord\ } : used to generate chart chords for stringed
  instruments.
\item
  \texttt{\ piano-chord\ } : used to generate piano chords.
\item
  \texttt{\ single-chord\ } : used to show the chord name over a word.
\end{itemize}

\subsubsection{Typst Packages}\label{typst-packages}

Typst added an experimental package repository and you can import
\texttt{\ chordx\ } as follows:

\begin{Shaded}
\begin{Highlighting}[]
\NormalTok{\#import "@preview/chordx:0.5.0": *}
\end{Highlighting}
\end{Shaded}

\subsubsection{Local Packages}\label{local-packages}

If the package hasn’t been released yet, or if you just want to use it
from this repository, you can use \emph{\emph{local-packages}} .

You can read the documentation about typst
\href{https://github.com/typst/packages\#local-packages}{local-packages}
and learn about the path folders used in differents operating systems
(Linux / MacOS / Windows).

In Linux you can do:

\begin{Shaded}
\begin{Highlighting}[]
\FunctionTok{git}\NormalTok{ clone https://github.com/ljgago/typst{-}chords \textasciitilde{}/.local/share/typst/packages/local/chordx/0.5.0}
\end{Highlighting}
\end{Shaded}

And import the package in your file:

\begin{Shaded}
\begin{Highlighting}[]
\NormalTok{\#import "@local/chordx:0.5.0": *}
\end{Highlighting}
\end{Shaded}

\subsection{Documentation}\label{documentation}

Here
\href{https://github.com/ljgago/typst-chords/blob/v0.5.0/docs/chordx-docs.pdf}{chordx-docs}
you have the reference documentation that describes the functions and
parameters used in this package. ( \emph{Generated with
\href{https://github.com/Mc-Zen/tidy}{tidy}} )

\subsection{Examples:}\label{examples}

\subsubsection{Chart Chords}\label{chart-chords}

\begin{Shaded}
\begin{Highlighting}[]
\NormalTok{\#import "@preview/chordx:0.5.0": chart{-}chord}

\NormalTok{\#let chart{-}chord{-}sharp = chart{-}chord.with(size: 18pt)}
\NormalTok{\#let chart{-}chord{-}round = chart{-}chord.with(size: 1.5em, design: "round")}

\NormalTok{// Design "sharp"}
\NormalTok{\#chart{-}chord{-}sharp(tabs: "x32o1o", fingers: "n32n1n")[C]}
\NormalTok{\#chart{-}chord{-}sharp(tabs: "ooo3", fingers: "ooo3")[C]}

\NormalTok{// Desigh "round" with position "bottom"}
\NormalTok{\#chart{-}chord{-}round(tabs: "xn332n", fingers: "o13421", fret: 3, capos: "115", position: "bottom")[Cm]}
\NormalTok{\#chart{-}chord{-}round(tabs: "onnn", fingers: "n111", capos: "313", position: "bottom")[Cm]}

\NormalTok{// Design "round" with background color in chord name}
\NormalTok{\#chart{-}chord{-}round(tabs: "xn332n", fingers: "o13421", fret: 3, capos: "115", background: silver)[Cm]}
\NormalTok{\#chart{-}chord{-}round(tabs: "onnn", fingers: "n111", capos: "313", background: silver)[Cm]}
\end{Highlighting}
\end{Shaded}

\subsubsection{\texorpdfstring{\href{https://github.com/ljgago/typst-chords/blob/v0.5.0/examples/chart-chords.typ}{\protect\pandocbounded{\includesvg[keepaspectratio]{https://raw.githubusercontent.com/ljgago/typst-chords/v0.5.0/examples/chart-chords.svg}}}}{Chart Chord}}\label{chart-chord}

\subsubsection{Piano Chords}\label{piano-chords}

\begin{Shaded}
\begin{Highlighting}[]
\NormalTok{\#import "@preview/chordx:0.5.0": piano{-}chord}

\NormalTok{\#let piano{-}chord{-}sharp = piano{-}chord.with(layout: "F", size: 18pt)}
\NormalTok{\#let piano{-}chord{-}round = piano{-}chord.with(layout: "F", size: 1.5em, design: "round")}

\NormalTok{\#piano{-}chord{-}sharp(keys: "B1, D2\#, F2\#", fill{-}key: blue)[B]}
\NormalTok{\#piano{-}chord{-}round(keys: "B1, D2\#, F2\#", fill{-}key: yellow, position: "bottom")[B]}
\NormalTok{\#piano{-}chord{-}round(keys: "B1, D2\#, F2\#", fill{-}key: red)[B]}
\end{Highlighting}
\end{Shaded}

\subsubsection{\texorpdfstring{\href{https://github.com/ljgago/typst-chords/blob/v0.5.0/examples/piano-chords.typ}{\protect\pandocbounded{\includesvg[keepaspectratio]{https://raw.githubusercontent.com/ljgago/typst-chords/v0.5.0/examples/piano-chords.svg}}}}{Piano Chord}}\label{piano-chord}

\subsubsection{Single Chords}\label{single-chords}

\begin{Shaded}
\begin{Highlighting}[]
\NormalTok{\#import "@preview/chordx:0.5.0": single{-}chord}

\NormalTok{\#let chord = single{-}chord.with(}
\NormalTok{  font: "PT Sans",}
\NormalTok{  size: 12pt,}
\NormalTok{  weight: "semibold",}
\NormalTok{  background: silver}
\NormalTok{)}

\NormalTok{\#chord[Jingle][G][2] bells, jingle bells, jingle \#chord[all][C][2] the \#chord[way!][G][2] \textbackslash{}}
\NormalTok{\#chord[Oh][C][] what fun it \#chord[is][G][] to ride \textbackslash{}}
\NormalTok{In a \#chord[one{-}horse][A7][2] open \#chord[sleigh,][D7][3] hey!}
\end{Highlighting}
\end{Shaded}

\subsection{\texorpdfstring{\href{https://github.com/ljgago/typst-chords/blob/v0.5.0/examples/single-chords.typ}{\protect\pandocbounded{\includesvg[keepaspectratio]{https://raw.githubusercontent.com/ljgago/typst-chords/v0.5.0/examples/single-chords.svg}}}}{Single Chord}}\label{single-chord}

\subsection{Changelog}\label{changelog}

You can read the latest changes in
\href{https://github.com/typst/packages/raw/main/packages/preview/chordx/0.5.0/CHANGELOG.md}{CHANGELOG.md}

\subsection{License}\label{license}

\href{https://github.com/typst/packages/raw/main/packages/preview/chordx/0.5.0/LICENSE}{MIT
License}

\subsubsection{How to add}\label{how-to-add}

Copy this into your project and use the import as \texttt{\ chordx\ }

\begin{verbatim}
#import "@preview/chordx:0.5.0"
\end{verbatim}

\includesvg[width=0.16667in,height=0.16667in]{/assets/icons/16-copy.svg}

Check the docs for
\href{https://typst.app/docs/reference/scripting/\#packages}{more
information on how to import packages} .

\subsubsection{About}\label{about}

\begin{description}
\tightlist
\item[Author :]
\href{https://github.com/ljgago}{Leonardo Gago}
\item[License:]
MIT
\item[Current version:]
0.5.0
\item[Last updated:]
November 4, 2024
\item[First released:]
July 17, 2023
\item[Minimum Typst version:]
0.12.0
\item[Archive size:]
10.3 kB
\href{https://packages.typst.org/preview/chordx-0.5.0.tar.gz}{\pandocbounded{\includesvg[keepaspectratio]{/assets/icons/16-download.svg}}}
\item[Repository:]
\href{https://github.com/ljgago/typst-chords}{GitHub}
\end{description}

\subsubsection{Where to report issues?}\label{where-to-report-issues}

This package is a project of Leonardo Gago . Report issues on
\href{https://github.com/ljgago/typst-chords}{their repository} . You
can also try to ask for help with this package on the
\href{https://forum.typst.app}{Forum} .

Please report this package to the Typst team using the
\href{https://typst.app/contact}{contact form} if you believe it is a
safety hazard or infringes upon your rights.

\phantomsection\label{versions}
\subsubsection{Version history}\label{version-history}

\begin{longtable}[]{@{}ll@{}}
\toprule\noalign{}
Version & Release Date \\
\midrule\noalign{}
\endhead
\bottomrule\noalign{}
\endlastfoot
0.5.0 & November 4, 2024 \\
\href{https://typst.app/universe/package/chordx/0.4.0/}{0.4.0} & July
10, 2024 \\
\href{https://typst.app/universe/package/chordx/0.3.0/}{0.3.0} & March
3, 2024 \\
\href{https://typst.app/universe/package/chordx/0.2.0/}{0.2.0} &
September 3, 2023 \\
\href{https://typst.app/universe/package/chordx/0.1.0/}{0.1.0} & July
17, 2023 \\
\end{longtable}

Typst GmbH did not create this package and cannot guarantee correct
functionality of this package or compatibility with any version of the
Typst compiler or app.


\section{Package List LaTeX/classy-german-invoice.tex}
\title{typst.app/universe/package/classy-german-invoice}

\phantomsection\label{banner}
\phantomsection\label{template-thumbnail}
\pandocbounded{\includegraphics[keepaspectratio]{https://packages.typst.org/preview/thumbnails/classy-german-invoice-0.3.0-small.webp}}

\section{classy-german-invoice}\label{classy-german-invoice}

{ 0.3.0 }

Minimalistic invoice for Germany-based freelancers

{ } Featured Template

\href{/app?template=classy-german-invoice&version=0.3.0}{Create project
in app}

\phantomsection\label{readme}
A template for writing invoices, inspired by the
\href{https://github.com/mrzool/invoice-boilerplate/}{beautiful LaTeX
template by @mrzool.}

\begin{Shaded}
\begin{Highlighting}[]
\NormalTok{\#import "@preview/classy{-}german{-}invoice:0.3.0": invoice}

\NormalTok{\#show: invoice(}
\NormalTok{  // Invoice number}
\NormalTok{  "2023{-}001",}
\NormalTok{  // Invoice date}
\NormalTok{  datetime(year: 2024, month: 09, day: 03),}
\NormalTok{  // Items}
\NormalTok{  (}
\NormalTok{    (}
\NormalTok{      description: "The first service provided. The first service provided. The first service provided",}
\NormalTok{      price: 200,}
\NormalTok{    ),}
\NormalTok{    (}
\NormalTok{      description: "The second service provided",}
\NormalTok{      price: 150.2}
\NormalTok{    ),}
\NormalTok{  ),}
\NormalTok{  // Author}
\NormalTok{  (}
\NormalTok{    name: "Kerstin Humm",}
\NormalTok{    street: "Straße der Privatsphäre und Stille 1",}
\NormalTok{    zip: "54321",}
\NormalTok{    city: "Potsdam",}
\NormalTok{    tax\_nr: "12345/67890",}
\NormalTok{    // optional signature, can be omitted}
\NormalTok{    signature: image("example\_signature.png", width: 5em)}
\NormalTok{  ),}
\NormalTok{  // Recipient}
\NormalTok{  (}
\NormalTok{    name: "Erika Mustermann",}
\NormalTok{    street: "Musterallee",}
\NormalTok{    zip: "12345",}
\NormalTok{    city: "Musterstadt",}
\NormalTok{  ),}
\NormalTok{  // Bank account}
\NormalTok{  (}
\NormalTok{    name: "Todd Name",}
\NormalTok{    bank: "Deutsche Postbank AG",}
\NormalTok{    iban: "DE89370400440532013000",}
\NormalTok{    bic: "PBNKDEFF",}
\NormalTok{    // There is currently only one gendered term in this template.}
\NormalTok{    // You can overwrite it, or omit it and just choose the default.}
\NormalTok{    gender: (account\_holder: "Kontoinhaberin")}
\NormalTok{  ),}
\NormalTok{  // Umsatzsteuersatz (VAT)}
\NormalTok{  vat: 0.19,}
\NormalTok{  kleinunternehmer: true,}
\NormalTok{)}
\end{Highlighting}
\end{Shaded}

\pandocbounded{\includegraphics[keepaspectratio]{https://github.com/typst/packages/raw/main/packages/preview/classy-german-invoice/0.3.0/thumbnail.png}}

\subsection{Scope}\label{scope}

This template should work well for freelancers and small companies in
the german market, that don’t have an existing system in place for
order tracking. Or to put it the other way round; This template is for
people that mostly have to fulfill outside requirements with their
invoices and don’t so much benefit from extensive tracking themselfes.

\subsection{Features}\label{features}

\begin{itemize}
\tightlist
\item
  {[}X{]} multiple invoice items
\item
  {[}X{]} configurable VAT
\item
  {[}X{]} configurable § 19 UStG (Kleinunternehmerregelung) note
\item
  {[}X{]} configurable signature from PNG file
\item
  {[}X{]} employs both lining and old-style number types, depending on
  the application
\item
  {[}X{]} \href{https://en.wikipedia.org/wiki/EPC_QR_code}{EPC QR Code}
  for easier banking transactions
\item
  {[} {]} recipient address is guaranteed to fit in a windowed envolope
  (DIN 5008)
\end{itemize}

\subsection{Disclaimer}\label{disclaimer}

This template doesn’t constitute legal advice. Please check for
yourself wether it fulfills your legal requirements!

\href{/app?template=classy-german-invoice&version=0.3.0}{Create project
in app}

\subsubsection{How to use}\label{how-to-use}

Click the button above to create a new project using this template in
the Typst app.

You can also use the Typst CLI to start a new project on your computer
using this command:

\begin{verbatim}
typst init @preview/classy-german-invoice:0.3.0
\end{verbatim}

\includesvg[width=0.16667in,height=0.16667in]{/assets/icons/16-copy.svg}

\subsubsection{About}\label{about}

\begin{description}
\tightlist
\item[Author :]
\href{https://github.com/erictapen}{Kerstin Humm}
\item[License:]
MIT-0
\item[Current version:]
0.3.0
\item[Last updated:]
September 18, 2024
\item[First released:]
September 11, 2024
\item[Minimum Typst version:]
0.10.0
\item[Archive size:]
23.9 kB
\href{https://packages.typst.org/preview/classy-german-invoice-0.3.0.tar.gz}{\pandocbounded{\includesvg[keepaspectratio]{/assets/icons/16-download.svg}}}
\item[Repository:]
\href{https://github.com/erictapen/typst-invoice}{GitHub}
\item[Categor ies :]
\begin{itemize}
\tightlist
\item[]
\item
  \pandocbounded{\includesvg[keepaspectratio]{/assets/icons/16-layout.svg}}
  \href{https://typst.app/universe/search/?category=layout}{Layout}
\item
  \pandocbounded{\includesvg[keepaspectratio]{/assets/icons/16-envelope.svg}}
  \href{https://typst.app/universe/search/?category=office}{Office}
\end{itemize}
\end{description}

\subsubsection{Where to report issues?}\label{where-to-report-issues}

This template is a project of Kerstin Humm . Report issues on
\href{https://github.com/erictapen/typst-invoice}{their repository} .
You can also try to ask for help with this template on the
\href{https://forum.typst.app}{Forum} .

Please report this template to the Typst team using the
\href{https://typst.app/contact}{contact form} if you believe it is a
safety hazard or infringes upon your rights.

\phantomsection\label{versions}
\subsubsection{Version history}\label{version-history}

\begin{longtable}[]{@{}ll@{}}
\toprule\noalign{}
Version & Release Date \\
\midrule\noalign{}
\endhead
\bottomrule\noalign{}
\endlastfoot
0.3.0 & September 18, 2024 \\
\href{https://typst.app/universe/package/classy-german-invoice/0.2.0/}{0.2.0}
& September 11, 2024 \\
\end{longtable}

Typst GmbH did not create this template and cannot guarantee correct
functionality of this template or compatibility with any version of the
Typst compiler or app.


\section{Package List LaTeX/springer-spaniel.tex}
\title{typst.app/universe/package/springer-spaniel}

\phantomsection\label{banner}
\phantomsection\label{template-thumbnail}
\pandocbounded{\includegraphics[keepaspectratio]{https://packages.typst.org/preview/thumbnails/springer-spaniel-0.1.0-small.webp}}

\section{springer-spaniel}\label{springer-spaniel}

{ 0.1.0 }

A loose recreation of the Springer Contributed Chapter template on
Overleaf

{ } Featured Template

\href{/app?template=springer-spaniel&version=0.1.0}{Create project in
app}

\phantomsection\label{readme}
Version 0.1.0

This is an loose recreation of the \emph{Springer Contributed Chapter}
LaTeX template on Overleaf. It aims to provide template-level support
for commonly used packages so you don’t have to choose between style
and features.

\subsection{Media}\label{media}

\includegraphics[width=0.32\linewidth,height=\textheight,keepaspectratio]{https://github.com/typst/packages/raw/main/packages/preview/springer-spaniel/0.1.0/thumbnails/1.png}
\includegraphics[width=0.32\linewidth,height=\textheight,keepaspectratio]{https://github.com/typst/packages/raw/main/packages/preview/springer-spaniel/0.1.0/thumbnails/2.png}
\includegraphics[width=0.32\linewidth,height=\textheight,keepaspectratio]{https://github.com/typst/packages/raw/main/packages/preview/springer-spaniel/0.1.0/thumbnails/3.png}

\subsection{Getting Started}\label{getting-started}

These instructions will get you a copy of the project up and running on
the typst web app. Perhaps a short code example on importing the package
and a very simple teaser usage.

\begin{Shaded}
\begin{Highlighting}[]
\NormalTok{\#import "@preview/springer{-}spaniel:0.1.0"}
\NormalTok{\#import springer{-}spaniel.ctheorems: * // provides "proof", "theorem", "lemma"}

\NormalTok{\#show: springer{-}spaniel.template(}
\NormalTok{  title: [Contribution Title],}
\NormalTok{  authors: (}
\NormalTok{    (}
\NormalTok{      name: "Name of First Author",}
\NormalTok{      institute: "Name",}
\NormalTok{      address: "Address of Institute",}
\NormalTok{      email: "name@email.address"}
\NormalTok{    ),}
\NormalTok{    // ... and so on}
\NormalTok{  ),}
\NormalTok{  abstract: lorem(75),}

\NormalTok{  // debug: true, // Highlights structural elements and links}
\NormalTok{  // frame: 1pt, // A border around the page for white on white display}
\NormalTok{  // printer{-}test: true, // Suitably placed CMYK printer tests}
\NormalTok{)}

\NormalTok{= Section Heading}
\NormalTok{== Subsection Heading}
\NormalTok{=== Subsubsection Heading}
\NormalTok{==== Paragraph Heading}
\NormalTok{===== Subparagraph Heading}
\end{Highlighting}
\end{Shaded}

\subsubsection{Local Installation}\label{local-installation}

To install this project locally, follow the steps below;

\begin{itemize}
\tightlist
\item
  Install Just
\item
  Clone repository
\item
  In a bash compatible shell, \texttt{\ just\ install-preview\ }
\end{itemize}

\href{/app?template=springer-spaniel&version=0.1.0}{Create project in
app}

\subsubsection{How to use}\label{how-to-use}

Click the button above to create a new project using this template in
the Typst app.

You can also use the Typst CLI to start a new project on your computer
using this command:

\begin{verbatim}
typst init @preview/springer-spaniel:0.1.0
\end{verbatim}

\includesvg[width=0.16667in,height=0.16667in]{/assets/icons/16-copy.svg}

\subsubsection{About}\label{about}

\begin{description}
\tightlist
\item[Author :]
James R. Swift
\item[License:]
Unlicense
\item[Current version:]
0.1.0
\item[Last updated:]
July 16, 2024
\item[First released:]
July 16, 2024
\item[Archive size:]
437 kB
\href{https://packages.typst.org/preview/springer-spaniel-0.1.0.tar.gz}{\pandocbounded{\includesvg[keepaspectratio]{/assets/icons/16-download.svg}}}
\item[Repository:]
\href{https://github.com/JamesxX/springer-spaniel}{GitHub}
\item[Discipline s :]
\begin{itemize}
\tightlist
\item[]
\item
  \href{https://typst.app/universe/search/?discipline=chemistry}{Chemistry}
\item
  \href{https://typst.app/universe/search/?discipline=physics}{Physics}
\item
  \href{https://typst.app/universe/search/?discipline=mathematics}{Mathematics}
\end{itemize}
\item[Categor ies :]
\begin{itemize}
\tightlist
\item[]
\item
  \pandocbounded{\includesvg[keepaspectratio]{/assets/icons/16-docs.svg}}
  \href{https://typst.app/universe/search/?category=book}{Book}
\item
  \pandocbounded{\includesvg[keepaspectratio]{/assets/icons/16-speak.svg}}
  \href{https://typst.app/universe/search/?category=report}{Report}
\end{itemize}
\end{description}

\subsubsection{Where to report issues?}\label{where-to-report-issues}

This template is a project of James R. Swift . Report issues on
\href{https://github.com/JamesxX/springer-spaniel}{their repository} .
You can also try to ask for help with this template on the
\href{https://forum.typst.app}{Forum} .

Please report this template to the Typst team using the
\href{https://typst.app/contact}{contact form} if you believe it is a
safety hazard or infringes upon your rights.

\phantomsection\label{versions}
\subsubsection{Version history}\label{version-history}

\begin{longtable}[]{@{}ll@{}}
\toprule\noalign{}
Version & Release Date \\
\midrule\noalign{}
\endhead
\bottomrule\noalign{}
\endlastfoot
0.1.0 & July 16, 2024 \\
\end{longtable}

Typst GmbH did not create this template and cannot guarantee correct
functionality of this template or compatibility with any version of the
Typst compiler or app.


\section{Package List LaTeX/modern-sustech-thesis.tex}
\title{typst.app/universe/package/modern-sustech-thesis}

\phantomsection\label{banner}
\phantomsection\label{template-thumbnail}
\pandocbounded{\includegraphics[keepaspectratio]{https://packages.typst.org/preview/thumbnails/modern-sustech-thesis-0.1.1-small.webp}}

\section{modern-sustech-thesis}\label{modern-sustech-thesis}

{ 0.1.1 }

å?---æ--¹ç§`技大学本ç§`毕业设计(论æ--‡ï¼‰æ¨¡æ?¿. SUSTech
Bachelor Thesis Template.

\href{/app?template=modern-sustech-thesis&version=0.1.1}{Create project
in app}

\phantomsection\label{readme}
功能需求ã€?å?ˆä½œå¼€å?{}`请移步模æ?¿å¯¹åº''çš„ github ä»``åº``:
\href{https://github.com/Duolei-Wang/modern-sustech-thesis}{modern-sustech-thesis}
.

\subsection{typst.app ç½`页版使ç''¨è¯´æ˜Ž (Use
online)}\label{typst.app-uxe7uxbduxe9uxb5uxe7ux2c6uxe4uxbduxe7uxe8uxe6ux17e-use-online}

使ç''¨æ­¥éª¤ï¼š

\begin{itemize}
\item
  æ‰``å¼€ typst.app 从模æ?¿æ--°å»ºé¡¹ç›®ï¼ˆstart from template)
\item
  论æ--‡æ‰€éœ€å­---ä½``需è¦?手动上ä¼~到ä½~的项目æ--‡ä»¶åˆ---表.

  点击左侧 Explore
  Files,上ä¼~å­---ä½``æ--‡ä»¶ï¼Œä¸Šä¼~å?Žçš„å­---ä½``æ--‡ä»¶å­˜å‚¨ä½?置没有特殊è¦?求,typst
  拥有优秀的å†\ldots æ~¸ï¼Œå?¯ä»¥å®Œæˆ?自动æ?œç´¢.

  ç''±äºŽæ~¼å¼?渲æŸ``引æ``Žçš„æ~¸å¿ƒéœ€è¦?指定å­---ä½``çš„å??称,æˆ`在模æ?¿æµ‹è¯•é˜¶æ®µä½¿ç''¨äº†è‹¥å¹²æ~‡å‡†å­---ä½``,这些å­---ä½``å?¯ä»¥åœ¨æˆ`çš„
  github ä»``åº``
  \href{https://github.com/Duolei-Wang/modern-sustech-thesis}{modern-sustech-thesis}
  /template/fonts 里找到.

  æ­¤å¤--,å?¯ä»¥æ‰‹åŠ¨æ›´æ''¹å­---ä½``é\ldots?置,在正æ--‡å‰?使ç''¨
  ‘\#set’
  å`½ä»¤å?³å?¯ï¼Œç''±äºŽæ~‡é¢˜ã€?æ­£æ--‡å­---ä½``ä¸?å?Œï¼Œæ­¤å¤„大致语法如下:
\end{itemize}

\begin{Shaded}
\begin{Highlighting}[]
\NormalTok{// headings}
\NormalTok{  show heading.where(level: 1): it =\textgreater{}\{}
\NormalTok{    set text(}
\NormalTok{      font: fonts.HeiTi,}
\NormalTok{      size: fonts.No3,}
\NormalTok{      weight: "regular",}
\NormalTok{    )}
\NormalTok{    align(center)[}
\NormalTok{      // \#it}
\NormalTok{      \#strong(it)}
\NormalTok{    ]}
\NormalTok{    text()[\#v(0.5em)]}
\NormalTok{  \}}

\NormalTok{  show heading.where(level: 2): it =\textgreater{}\{}
\NormalTok{    set text(}
\NormalTok{      font: fonts.HeiTi,}
\NormalTok{      size: fonts.No4,}
\NormalTok{      weight: "regular"}
\NormalTok{      )}
\NormalTok{    it}
\NormalTok{    text()[\#v(0.5em)]}
\NormalTok{  \}}

\NormalTok{  show heading.where(level: 3): it =\textgreater{}\{}
\NormalTok{    set text(}
\NormalTok{      font: fonts.HeiTi,}
\NormalTok{      size: fonts.No4{-}Small,}
\NormalTok{      weight: "regular"}
\NormalTok{      )}
\NormalTok{    it}
\NormalTok{    text()[\#v(0.5em)] }
\NormalTok{  \}}

\NormalTok{  // paragraph}
\NormalTok{  set block(spacing: 1.5em)}
\NormalTok{  set par(}
\NormalTok{    justify: true,}
\NormalTok{    first{-}line{-}indent: 2em,}
\NormalTok{    leading: 1.5em)}
\end{Highlighting}
\end{Shaded}

headings
设定了å?„个登记æ~‡é¢˜çš„æ~¼å¼?,å\ldots¶ä¸­ä¸€çº§æ~‡é¢˜éœ€è¦?å±\ldots 中对é½?.
‘font: fonts.HeiTi’
å?³ä¸ºå­---ä½``çš„å\ldots³é''®å?‚数,å?‚数的值是å­---ä½``çš„å??称(å­---符串).
typst 将会在ç¼--è¯`器å†\ldots æ~¸ã€?项目目录中æ?œç´¢. typst
å†\ldots æ~¸è‡ªå¸¦äº† Source Sans(é»`ä½``)å'Œ Source
Serif(宋ä½``)系åˆ---,但是中æ--‡è®ºæ--‡æ‰€éœ€çš„仿宋ã€?楷ä½``ä»?需自己上ä¼~.

按ç\ldots§æ¯•ä¸šè®¾è®¡è¦?求,以 markdown
æ~¼å¼?书写ä½~的毕业论æ--‡ï¼Œå?ªéœ€è¦?:

\begin{itemize}
\item
  在 configs/info 里填å\ldots¥ä¸ªäººä¿¡æ?¯.
  如有æ~‡é¢˜ç¼--è¯`é''™è¯¯ï¼ˆæ¯''如æˆ`默认了有三行æ~‡é¢˜ï¼‰ï¼Œå?¯ä»¥è‡ªè¡ŒæŒ‰ç\ldots§ç¼--è¯`器æ??示把相å\ldots³ä»£ç~?注释æˆ--è€\ldots ä¿®æ''¹.
  大ä½``语法å'Œå†\ldots 容与基本的ç¼--程语言æ---~差别.
\item
  在 content.typ 里以 typst 特定的 markdown
  语法书写ä½~的论æ--‡å†\ldots 容. 有å\ldots³ typst 中 markdown
  的语法å?˜æ›´ï¼Œä¸ªäººè®¤ä¸ºçš„主è¦?å?˜åŒ--ç½---åˆ---如下:

  \begin{itemize}
  \tightlist
  \item
    æ~‡é¢˜æ~?使ç''¨ ‘=’ 而é?ž ‘\#’,‘\#’ 在 typst
    里是å®?å`½ä»¤çš„开头.
  \item
    æ•°å­¦å\ldots¬å¼?ä¸?需è¦?å??æ--œæ?~,数学符å?·å?¯ä»¥æŸ¥é˜\ldots :
    \url{https://typst.app/docs/reference/symbols/sym/} .
    值å¾---注æ„?的是,typst
    中语法ä¸?通过å?~åŠ~çš„æ--¹å¼?实现,如 “ä¸?ç­‰å?·â€? 在
    LaTex 中是 ‘\textbackslash not\{=\}’. 而在 typst 中,使ç''¨
    ‘eq.not’ çš„æ--¹å¼?æ?¥è°ƒç''¨ ‘eq’(等å?·ï¼‰çš„
    ‘not’(ä¸?等)å?˜ä½``实现.
  \item
    引ç''¨æ~‡ç­¾é‡‡ç''¨ ‘@label’ æ?¥å®žçŽ°ï¼Œè‡ªå®šä¹‰æ~‡ç­¾é€šè¿‡
    ‘’ �实现. 对于 BibTex
    æ~¼å¼?的引ç''¨ï¼ˆrefer.bib),与 LaTex
    �路相�,第一个缩略�将会被认定为 label.
  \end{itemize}
\item
  自定义æ~¼å¼?çš„æ€?è·¯.
  如有é¢?å¤--的需è¦?自定义æ~¼å¼?的需求,å?¯ä»¥è‡ªè¡Œå­¦ä¹~
  ‘\#set’, ‘\#show’
  å`½ä»¤ï¼Œè¿™å?¯èƒ½éœ€è¦?一定的ç¼--程语言知识,å?Žç»­æˆ`会更æ--°éƒ¨åˆ†ç®€ç•¥æ•™ç¨‹åœ¨æˆ`çš„
  github ä»``åº``里: \url{https://github.com/Duolei-Wang/lang-typst}
  .
\item
  本模æ?¿çš„ç»``æž„

  \begin{enumerate}
  \tightlist
  \item
    å†\ldots 容主ä½``. æ--‡ç«~主ä½``å†\ldots 容书写在 content.typ
    æ--‡ä»¶ä¸­ï¼Œé™„录部分书写在 appendix.typ æ--‡ä»¶ä¸­.
  \item
    å†\ldots 容顺åº?. æ--‡ç«~å†\ldots 容顺åº?ç''± main.typ
    决定,通过 typst 中 ‘\#include’
    指令实现了页é?¢çš„æ?'å\ldots¥.
  \item
    å†\ldots 容æ~¼å¼?. å†\ldots 容æ~¼å¼?ç''± /sections/*.typ
    控制,body.typ
    控制了æ--‡ç«~主ä½``çš„æ~¼å¼?,å\ldots¶ä½™ä¸Žå??称一致. cover
    为��,commitment 为承诺书,outline 为目录,abstract
    为æ`˜è¦?.
  \end{enumerate}
\end{itemize}

版本�:0.1.1

\begin{itemize}
\tightlist
\item
  Fixed the fatal bug.
  修正了å?‚æ•°ä¼~é€'失败é€~æˆ?çš„å°?é?¢ç­‰é¡µé?¢æ---~法正常更æ''¹ä¿¡æ?¯.
\end{itemize}

TODO:

\begin{itemize}
\tightlist
\item
  {[} {]} 引ç''¨æ~¼å¼? check.
\end{itemize}

å?---æ--¹ç§`技大学本ç§`毕业设计(论æ--‡ï¼‰æ¨¡æ?¿ï¼Œè®ºæ--‡æ~¼å¼?å?‚ç\ldots§
\href{https://tao.sustech.edu.cn/studentService/graduation_project.html}{å?---æ--¹ç§`技大学本ç§`ç''Ÿæ¯•ä¸šè®¾è®¡ï¼ˆè®ºæ--‡ï¼‰æ'°å†™è§„范}
.
如有ç--?æ¼?敬请è°\ldots 解,本模æ?¿ä¸ºæœ¬äººæ¯•ä¸šä¹‹å‰?自ç''¨ï¼Œå¦‚有使ç''¨ï¼Œç¨³å®šæ€§è¯·è‡ªè¡Œè´Ÿè´£.

\begin{itemize}
\item
  本模�主��考了 \href{https://github.com/iydon}{iydon}
  ä»``åº``çš„çš„ \$\textbackslash LaTeX\$ 模æ?¿
  \href{https://github.com/iydon/sustechthesis}{sustechthesis}
  ï¼›ç»``构组织å?‚ç\ldots§äº†
  \href{https://github.com/shuosc}{shuosc} ä»``åº``çš„
  \href{https://github.com/shuosc/SHU-Bachelor-Thesis-Typst}{SHU-Bachelor-Thesis-Typst}
  模æ?¿ï¼›å›¾ç‰‡ç´~æ??使ç''¨äº†
  \href{https://github.com/GuTaoZi}{GuTaoZi}
  çš„å?Œå†\ldots 容ä»``åº``里的模æ?¿.
\item
  æ„Ÿè°¢
  \href{https://github.com/shuosc/SHU-Bachelor-Thesis-Typst}{SHU-Bachelor-Thesis}
  çš„ç»``构组织让æˆ`å­¦ä¹~到了很多,给æˆ`的页é?¢ç»„织æ??供了ç?µæ„Ÿï¼Œ
\item
  在查找图片ç´~æ??çš„æ---¶å€™ï¼Œä½¿ç''¨äº† GuTaoZi ä»``åº``
  \href{https://github.com/GuTaoZi/SUSTech-thesis-typst}{SUSTech-thesis-typst}
  里的svg ç´~æ??,特此感谢.
\end{itemize}

本模æ?¿ã€?ä»``åº``处于个人安利 typst
的需è¦?â€''â€''在线模æ?¿éœ€ä¸Šä¼~至 typst/packages
官æ--¹ä»``åº``æ‰?能被æ?œç´¢åˆ°ï¼Œå¦‚有开å?{}`å'ŒæŽ¥ç®¡ç­‰éœ€æ±‚请务å¿\ldots è?''ç³»æˆ`:

QQ: 782564506

mail:
\href{mailto:wangdl2020@mail.sustech.edu.cn}{\nolinkurl{wangdl2020@mail.sustech.edu.cn}}

\href{/app?template=modern-sustech-thesis&version=0.1.1}{Create project
in app}

\subsubsection{How to use}\label{how-to-use}

Click the button above to create a new project using this template in
the Typst app.

You can also use the Typst CLI to start a new project on your computer
using this command:

\begin{verbatim}
typst init @preview/modern-sustech-thesis:0.1.1
\end{verbatim}

\includesvg[width=0.16667in,height=0.16667in]{/assets/icons/16-copy.svg}

\subsubsection{About}\label{about}

\begin{description}
\tightlist
\item[Author :]
MuTsingQAQ
\item[License:]
MIT
\item[Current version:]
0.1.1
\item[Last updated:]
April 29, 2024
\item[First released:]
April 17, 2024
\item[Archive size:]
50.6 kB
\href{https://packages.typst.org/preview/modern-sustech-thesis-0.1.1.tar.gz}{\pandocbounded{\includesvg[keepaspectratio]{/assets/icons/16-download.svg}}}
\item[Repository:]
\href{https://github.com/Duolei-Wang/sustech-thesis-typst}{GitHub}
\item[Categor y :]
\begin{itemize}
\tightlist
\item[]
\item
  \pandocbounded{\includesvg[keepaspectratio]{/assets/icons/16-mortarboard.svg}}
  \href{https://typst.app/universe/search/?category=thesis}{Thesis}
\end{itemize}
\end{description}

\subsubsection{Where to report issues?}\label{where-to-report-issues}

This template is a project of MuTsingQAQ . Report issues on
\href{https://github.com/Duolei-Wang/sustech-thesis-typst}{their
repository} . You can also try to ask for help with this template on the
\href{https://forum.typst.app}{Forum} .

Please report this template to the Typst team using the
\href{https://typst.app/contact}{contact form} if you believe it is a
safety hazard or infringes upon your rights.

\phantomsection\label{versions}
\subsubsection{Version history}\label{version-history}

\begin{longtable}[]{@{}ll@{}}
\toprule\noalign{}
Version & Release Date \\
\midrule\noalign{}
\endhead
\bottomrule\noalign{}
\endlastfoot
0.1.1 & April 29, 2024 \\
\href{https://typst.app/universe/package/modern-sustech-thesis/0.1.0/}{0.1.0}
& April 17, 2024 \\
\end{longtable}

Typst GmbH did not create this template and cannot guarantee correct
functionality of this template or compatibility with any version of the
Typst compiler or app.


\section{Package List LaTeX/idwtet.tex}
\title{typst.app/universe/package/idwtet}

\phantomsection\label{banner}
\section{idwtet}\label{idwtet}

{ 0.3.0 }

Package for uniform, correct and simplified typst code demonstration.

\phantomsection\label{readme}
The name \texttt{\ idwtet\ } stands for “I Don’t Wanna Type
Everything Twice�. It provides a \texttt{\ typst-ex\ } and a
\texttt{\ typst-ex-code\ } codeblock, which \emph{shows \textbf{and}
executes} typst code.

It is meant for code demonstration, e.g. when publishing a package, and
provides some niceties:

\begin{itemize}
\tightlist
\item
  the code should always be correct in the examples: As the example code
  is used for the code block, but also for evaluation, there is no need
  to write it twice
\item
  easy configuration with simple, uniform and good look
\end{itemize}

However, there are some limitations:

\begin{itemize}
\tightlist
\item
  Every code block has its own local scope and the default behaviour is
  that variables are not reachable from outside. A similar restriction
  lies on import statements. That is why, there is the
  \texttt{\ eval-scope\ } argument, which captures variables and
  simulates global variables. Additionally, a \texttt{\ typst\ }
  codeblock is provided for a consistent look.
\item
  Locality can be displayed to the users by automatically wrapping code
  in \texttt{\ typst-ex-code\ } , but \texttt{\ typst-ex\ } does not
  provide such functionality. It might thus be difficult for users to
  understand code examples this way.
\item
  The page width has to be defined in absolute terms. It is quite nice,
  for a showcase, to take the least possible space, but tracking the
  widths of all boxes and then setting the page width accordingly is not
  (yet) possible.
\end{itemize}

\subsection{Usage}\label{usage}

Only one function is defined,
\texttt{\ init(body,\ bcolor:\ luma(210),\ inset:\ 5pt,\ border:\ 2pt,\ radius:\ 2pt,\ content-font:\ "linux\ libertine",\ code-font-size:\ 9pt,\ content-font-size:\ 11pt,\ code-return-box:\ true,\ wrap-code:\ false,\ eval-scope:\ (:),\ escape-bracket:\ "\%")\ }
, which is supposed to be used with a show rule.

Then raw codeblocks (with \texttt{\ block=true\ } ) of the languages
\texttt{\ typst\ } , \texttt{\ typst-ex\ } , \texttt{\ typst-code\ } and
\texttt{\ typst-ex-code\ } are modified. The main feature of this
package are \texttt{\ typst-ex\ } and \texttt{\ typst-ex-code\ } . The
\texttt{\ typst\ } and \texttt{\ typst-code\ } blocks do not evaluate
anything, but their design fits that of the others.

The parameters of \texttt{\ init\ } are:

\begin{itemize}
\tightlist
\item
  \texttt{\ body\ } : for usage with show rule, hence the whole
  document.
\item
  \texttt{\ bcolor\ } : the background- (and border-) color of the
  blocks
\item
  \texttt{\ inset\ } : inset param of code and content blocks, should be
  ≥ 2pt
\item
  \texttt{\ border\ } : border thickness
\item
  \texttt{\ radius\ } : block radius
\item
  \texttt{\ content-font\ } : The font used in the previewed content /
  result.
\item
  \texttt{\ code-font-size\ } : The fontsize used in the code blocks.
\item
  \texttt{\ content-font-size\ } : The fontsize used in the preview
  content / result.
\item
  \texttt{\ code-return-box\ } : If to show the code return type on
  \texttt{\ typst-ex-code\ } blocks.
\item
  \texttt{\ wrap-code\ } : If to wrap the code in \texttt{\ \#\{\ } and
  \texttt{\ \}\ } , to symbolize local scope.
\item
  \texttt{\ eval-scope\ } : A dictionary with the keys as the variable
  names and the values as another dictionary with keys
  \texttt{\ value\ } and \texttt{\ code\ } , both of these are optional.
  The former has the defined value, the latter the code recreate the
  variable, for usage in the code blocks.
\item
  \texttt{\ escape-bracket\ } : The text to wrap a variable with, to
  access the \texttt{\ code\ } part of a \texttt{\ eval-scope\ }
  variable.
\end{itemize}

\subsection{Hiding code and
replacements}\label{hiding-code-and-replacements}

There are currently two methods to change the code:

\begin{itemize}
\tightlist
\item
  use the \texttt{\ eval-scope\ } argument from the \texttt{\ init\ }
  function. There is a \texttt{\ code\ } field in the dictionaries,
  which enables the usage of the key escaped in
  \texttt{\ escape-bracket\ } to be replaced in the codeblock code half
  and to be removed in the codeblock result half, as the value is given
  via scope. Take a look at the example below, where
  \texttt{\ \%ouset\%\ } is used this way.
\item
  use the \texttt{\ ENDHIDDEN\ } feature. When escaped in
  \texttt{\ escape-bracket\ } , everything above the statement is
  removed from the codeblock code half BUT everything (without the
  \texttt{\ ENDHIDDEN\ } statement) is evaluated. Take a look at the
  example in the examples folder.
\end{itemize}

\subsection{Example}\label{example}

\begin{Shaded}
\begin{Highlighting}[]
\NormalTok{\#set page(margin: 0.5cm, width: 14cm, height: auto)}
\NormalTok{\#import "@preview/idwtet:0.1.0"}
\NormalTok{\#show: idwtet.init.with(eval{-}scope: (}
\NormalTok{  ouset: (}
\NormalTok{    value: \{import "@preview/ouset:0.1.1": ouset; ouset\},}
\NormalTok{    code: "\#import \textbackslash{}"@preview/ouset:0.1.1\textbackslash{}": ouset"}
\NormalTok{  )}
\NormalTok{))}

\NormalTok{== ouset package \#text(gray)[(v0.1.1)]}
\NormalTok{\textasciigrave{}\textasciigrave{}\textasciigrave{}typst{-}ex}
\NormalTok{\%ouset\%}
\NormalTok{$}
\NormalTok{"Expression 1" ouset(\&, \textless{}==\textgreater{}, "Theorem 1") "Expression 2"\textbackslash{}}
\NormalTok{               ouset(\&, ==\textgreater{},, "Theorem 7") "Expression 3"}
\NormalTok{$}
\NormalTok{\textasciigrave{}\textasciigrave{}\textasciigrave{}}
\NormalTok{Or something like}
\NormalTok{\textasciigrave{}\textasciigrave{}\textasciigrave{}typst{-}ex{-}code}
\NormalTok{let a = range(10)}
\NormalTok{a}
\NormalTok{\textasciigrave{}\textasciigrave{}\textasciigrave{}}
\end{Highlighting}
\end{Shaded}

Further examples are given in the repo example folder.

\subsubsection{How to add}\label{how-to-add}

Copy this into your project and use the import as \texttt{\ idwtet\ }

\begin{verbatim}
#import "@preview/idwtet:0.3.0"
\end{verbatim}

\includesvg[width=0.16667in,height=0.16667in]{/assets/icons/16-copy.svg}

Check the docs for
\href{https://typst.app/docs/reference/scripting/\#packages}{more
information on how to import packages} .

\subsubsection{About}\label{about}

\begin{description}
\tightlist
\item[Author :]
Ludwig Austermann
\item[License:]
MIT
\item[Current version:]
0.3.0
\item[Last updated:]
September 25, 2023
\item[First released:]
August 19, 2023
\item[Minimum Typst version:]
0.8.0
\item[Archive size:]
3.84 kB
\href{https://packages.typst.org/preview/idwtet-0.3.0.tar.gz}{\pandocbounded{\includesvg[keepaspectratio]{/assets/icons/16-download.svg}}}
\item[Repository:]
\href{https://github.com/ludwig-austermann/typst-idwtet}{GitHub}
\end{description}

\subsubsection{Where to report issues?}\label{where-to-report-issues}

This package is a project of Ludwig Austermann . Report issues on
\href{https://github.com/ludwig-austermann/typst-idwtet}{their
repository} . You can also try to ask for help with this package on the
\href{https://forum.typst.app}{Forum} .

Please report this package to the Typst team using the
\href{https://typst.app/contact}{contact form} if you believe it is a
safety hazard or infringes upon your rights.

\phantomsection\label{versions}
\subsubsection{Version history}\label{version-history}

\begin{longtable}[]{@{}ll@{}}
\toprule\noalign{}
Version & Release Date \\
\midrule\noalign{}
\endhead
\bottomrule\noalign{}
\endlastfoot
0.3.0 & September 25, 2023 \\
\href{https://typst.app/universe/package/idwtet/0.2.0/}{0.2.0} & August
19, 2023 \\
\end{longtable}

Typst GmbH did not create this package and cannot guarantee correct
functionality of this package or compatibility with any version of the
Typst compiler or app.


\section{Package List LaTeX/boxr.tex}
\title{typst.app/universe/package/boxr}

\phantomsection\label{banner}
\section{boxr}\label{boxr}

{ 0.1.0 }

A modular, and easy to use, package for creating cardboard cutouts in
Typst.

\phantomsection\label{readme}
Boxr is a modular, and easy to use, package for creating cardboard
cutouts in Typst.

\subsection{Usage}\label{usage}

Create a boxr structure in your project with the following code:

\begin{verbatim}
#import "@preview/boxr:0.1.0": *

#render-structure(
  "box",
  width: 100pt,
  height: 100pt,
  depth: 100pt,
  tab-size: 20pt,
  [
    Hello from boxr!
  ]
)
\end{verbatim}

The \texttt{\ render-structure\ } function is the main function for
boxr. It either takes a path to one of the default structures provided
by boxr (e.g.: \texttt{\ "box"\ } ) or an unpacked json file with your
own custom structure (e.g.: \texttt{\ json(my-structure.json)\ } ).
These describe the structure of the cutout.\\
The other named arguments depend on the structure you are rendering. All
unnamed arguments are passed to the structure as content and will be
rendered on each box face (not triangles or tabs).

\subsection{Creating your own
structures}\label{creating-your-own-structures}

Structures are defined in \texttt{\ .json\ } files. An example structure
that just shows a box with a tab on one face is shown below:

\begin{verbatim}
{
  "variables": ["height", "width", "tab-size"],
  "width": "width",
  "height": "height + tab-size",
  "offset-x": "",
  "offset-y": "tab-size",
  "root": {
    "type": "box",
    "id": 0,
    "width": "width",
    "height": "height",
    "children": {
      "top": "tab(tab-size, tab-size)"
    }
  }
}
\end{verbatim}

The \texttt{\ variables\ } key is a list of variable names that can be
passed to the structure. These will be required to be passed to the
\texttt{\ render-structure\ } function.\\
The \texttt{\ width\ } and \texttt{\ height\ } keys are evaluated to
calculate the width and height of the structure.\\
The \texttt{\ offset-x\ } and \texttt{\ offset-y\ } keys are evaluated
to place the structure in the middle of its bounds. It is relative to
the root node. In this case for example, the top tab adds a
\texttt{\ tab-size\ } on top of the box as opposed to the bottom, where
there is no tab. So this \texttt{\ tab-size\ } is added to the
\texttt{\ offset-y\ } .\\
\texttt{\ root\ } denotes the first node in the structure.\\
A node can be of the following types:

\begin{itemize}
\tightlist
\item
  \texttt{\ box\ } :

  \begin{itemize}
  \tightlist
  \item
    The root node has a \texttt{\ width\ } and a \texttt{\ height\ } .
    All following nodes have a \texttt{\ size\ } . Child nodes use
    \texttt{\ size\ } and the parent node’s \texttt{\ width\ } and
    \texttt{\ height\ } to calculate their own width and height.
  \item
    Can have \texttt{\ children\ } nodes.
  \item
    Can have an \texttt{\ id\ } key that is used to place content on the
    face of the box. The id-th unnamed argument is placed on the face.
    Multiple faces can have the same id.
  \item
    Can have a \texttt{\ no-fold\ } key. If this exists, no fold stroke
    will be drawn between this box and its parent.
  \end{itemize}
\item
  \texttt{\ triangle-\textless{}left\textbar{}right\textgreater{}\ } :

  \begin{itemize}
  \tightlist
  \item
    Has a \texttt{\ width\ } and \texttt{\ height\ } .
  \item
    \texttt{\ left\ } and \texttt{\ right\ } denote the direction the
    other right angled line is facing relative to the base.
  \item
    Can have \texttt{\ children\ } nodes.
  \item
    Can have a \texttt{\ no-fold\ } key. If this exists, no fold stroke
    will be drawn between this triangle and its parent.
  \end{itemize}
\item
  \texttt{\ tab\ } :

  \begin{itemize}
  \tightlist
  \item
    Is not a json object, but a string that denotes a tab. The tab is
    placed on the parent node.
  \item
    Has a tab-size and a cutin-size inside the \texttt{\ ()\ } separted
    by a \texttt{\ ,\ } .
  \end{itemize}
\item
  \texttt{\ none\ } :

  \begin{itemize}
  \tightlist
  \item
    Is not a json object, but a string that denotes no node. This is
    useful for deleting a cut-stroke between two nodes.
  \end{itemize}
\end{itemize}

Every string value in the json file (
\texttt{\ width:\ "\_\_",\ height:\ "\_\_",\ ...\ offset-x/y:\ "\_\_"\ }
and the values between the \texttt{\ \textbar{}\ } for tabs) is
evaluated as regular typst code. This means that you can use all named
variables passed to the structure. All inputs are converted to points
and the result of the evaluation will be converted back to a length.

\subsubsection{How to add}\label{how-to-add}

Copy this into your project and use the import as \texttt{\ boxr\ }

\begin{verbatim}
#import "@preview/boxr:0.1.0"
\end{verbatim}

\includesvg[width=0.16667in,height=0.16667in]{/assets/icons/16-copy.svg}

Check the docs for
\href{https://typst.app/docs/reference/scripting/\#packages}{more
information on how to import packages} .

\subsubsection{About}\label{about}

\begin{description}
\tightlist
\item[Author :]
\href{https://github.com/Lypsilonx}{Lypsilonx}
\item[License:]
MIT
\item[Current version:]
0.1.0
\item[Last updated:]
May 23, 2024
\item[First released:]
May 23, 2024
\item[Archive size:]
6.23 kB
\href{https://packages.typst.org/preview/boxr-0.1.0.tar.gz}{\pandocbounded{\includesvg[keepaspectratio]{/assets/icons/16-download.svg}}}
\item[Repository:]
\href{https://github.com/Lypsilonx/boxr}{GitHub}
\item[Discipline :]
\begin{itemize}
\tightlist
\item[]
\item
  \href{https://typst.app/universe/search/?discipline=design}{Design}
\end{itemize}
\item[Categor ies :]
\begin{itemize}
\tightlist
\item[]
\item
  \pandocbounded{\includesvg[keepaspectratio]{/assets/icons/16-chart.svg}}
  \href{https://typst.app/universe/search/?category=visualization}{Visualization}
\item
  \pandocbounded{\includesvg[keepaspectratio]{/assets/icons/16-layout.svg}}
  \href{https://typst.app/universe/search/?category=layout}{Layout}
\end{itemize}
\end{description}

\subsubsection{Where to report issues?}\label{where-to-report-issues}

This package is a project of Lypsilonx . Report issues on
\href{https://github.com/Lypsilonx/boxr}{their repository} . You can
also try to ask for help with this package on the
\href{https://forum.typst.app}{Forum} .

Please report this package to the Typst team using the
\href{https://typst.app/contact}{contact form} if you believe it is a
safety hazard or infringes upon your rights.

\phantomsection\label{versions}
\subsubsection{Version history}\label{version-history}

\begin{longtable}[]{@{}ll@{}}
\toprule\noalign{}
Version & Release Date \\
\midrule\noalign{}
\endhead
\bottomrule\noalign{}
\endlastfoot
0.1.0 & May 23, 2024 \\
\end{longtable}

Typst GmbH did not create this package and cannot guarantee correct
functionality of this package or compatibility with any version of the
Typst compiler or app.


\section{Package List LaTeX/abiding-ifacconf.tex}
\title{typst.app/universe/package/abiding-ifacconf}

\phantomsection\label{banner}
\phantomsection\label{template-thumbnail}
\pandocbounded{\includegraphics[keepaspectratio]{https://packages.typst.org/preview/thumbnails/abiding-ifacconf-0.1.0-small.webp}}

\section{abiding-ifacconf}\label{abiding-ifacconf}

{ 0.1.0 }

An IFAC-style paper template to publish at conferences for International
Federation of Automatic Control

\href{/app?template=abiding-ifacconf&version=0.1.0}{Create project in
app}

\phantomsection\label{readme}
\subsection{(unofficial) IFAC Conference Template for
Typst}\label{unofficial-ifac-conference-template-for-typst}

IFAC stands for \href{https://ifac-control.org/}{International
Federation of Automatic Control} . This repository is meant to be a port
of the existing author tools for conference papers (e.g., for LaTeX, see
\href{https://www.ifac-control.org/conferences/author-guide/copy_of_ifacconf_latex.zip/view}{ifacconf\_latex.zip}
) for Typst.

\subsection{Usage}\label{usage}

Running the following command will create a new directory with all the
files that are needed:

\begin{verbatim}
typst init @preview/abiding-ifacconf
\end{verbatim}

\subsection{Configuration}\label{configuration}

This template exports the \texttt{\ ifacconf\ } function with the
following named arguments:

\begin{itemize}
\tightlist
\item
  \texttt{\ authors\ } : (default: ()) array of authors. For each author
  you can specify a name, email (optional), and affiliation. The
  affiliation must be an integer corresponding to an entry in the
  1-indexed affiliations list (or 0 for no affiliation).
\item
  \texttt{\ affiliations\ } : (default: ()) array of affiliations. For
  each affiliation you can specify a department, organization, and
  address. Everything is optional (i.e., an affiliation can be an empty
  array).
\item
  \texttt{\ abstract\ } : (default: none) the paper’s abstract. Can be
  omitted if you don’t have one.
\item
  \texttt{\ keywords\ } : (default: ()) array of keywords to display
  after the abstract
\item
  \texttt{\ sponsor\ } : (default: none) acknowledgment of sponsor or
  financial support (appears as a footnote on the first page)
\end{itemize}

\subsection{Minimal Working Example}\label{minimal-working-example}

\begin{Shaded}
\begin{Highlighting}[]
\NormalTok{\#import "@preview:abiding{-}ifacconf:0.1.0": *}
\NormalTok{\#show: ifacconf{-}rules}
\NormalTok{\#show: ifacconf.with(}
\NormalTok{  title: "Minimal Working Example",}
\NormalTok{  authors: (}
\NormalTok{    (}
\NormalTok{      name: "First A. Author",}
\NormalTok{      email: "author@boulder.nist.gov",}
\NormalTok{      affiliation: 1,}
\NormalTok{    ),}
\NormalTok{  ),}
\NormalTok{  affiliations: (}
\NormalTok{    (}
\NormalTok{      department: "Engineering",}
\NormalTok{      organization: "National Institute of Standards and Technology",}
\NormalTok{      address: "Boulder, CO 80305 USA",}
\NormalTok{    ),}
\NormalTok{  ),}
\NormalTok{  abstract: [}
\NormalTok{    Abstract should be 50{-}100 words.}
\NormalTok{  ],}
\NormalTok{  keywords: ("keyword1", "keyword2"),}
\NormalTok{  sponsor: [}
\NormalTok{    Sponsor information.}
\NormalTok{  ],}
\NormalTok{)}

\NormalTok{= Introduction}

\NormalTok{A minimum working example (with bibliography) @Abl56.}

\NormalTok{\#lorem(80)}

\NormalTok{\#lorem(80)}

\NormalTok{\#bibliography("refs.bib")}
\end{Highlighting}
\end{Shaded}

\subsection{Full(er) Example}\label{fuller-example}

See
\href{https://github.com/avonmoll/ifacconf-typst/blob/main/template/main.typ}{\texttt{\ main.typ\ }}
.

\subsection{Dependencies}\label{dependencies}

\begin{itemize}
\tightlist
\item
  typst 0.11.0
\item
  ctheorems 1.1.0 (a Typst package for handling theorem-like
  environments)
\end{itemize}

\subsection{Notes, features, etc.}\label{notes-features-etc.}

\begin{itemize}
\tightlist
\item
  the call to \texttt{\ \#show:\ ifacconf-rules\ } is necessary for some
  show rules defined in \texttt{\ template.typ\ } to get activated
\item
  \texttt{\ ifac-conference.csl\ } is a lightly modified version of
  \texttt{\ apa.csl\ } and is included in order to change the citation
  format from, e.g., \texttt{\ (Able\ 1956)\ } to
  \texttt{\ Able\ (1956)\ } in order to match
  \texttt{\ ifacconf\_latex\ }
\item
  Tables have formatting rules that get activated inside calls to
  \texttt{\ figure\ } with \texttt{\ kind:\ "table"\ } ; a convenience
  function \texttt{\ tablefig\ } is provided which sets this
  automatically
\item
  all theorem-like environments that were available in
  \texttt{\ ifacconf\_latex\ } are defined in \texttt{\ template.typ\ }
  ; simply call, for example,
  \texttt{\ \#theorem{[}Content...{]}\ ...\ \#proof{[}Proof...{]}\ }
\item
  the LaTeX version does not include a QED symbol at the end of proofs,
  however one is included here (this is easy to change)
\item
  Typst did not seem to like BibTeX citation keys containing colons
  (which was how they came from \texttt{\ ifacconf\_latex\ } )
\item
  alignment for linebreaks in long equations is somewhat manual (e.g.,
  for equation (2) in \texttt{\ ifacconf.typ\ } ) but probably there is
  a better way to handle this now or in the future
\item
  the files \texttt{\ refs.bib\ } (essentially) and
  \texttt{\ bifurcation.jpg\ } come from \texttt{\ ifacconf\_latex\ }
\item
  the file \texttt{\ ifacconf.typ\ } is modeled directly after
  \texttt{\ ifacconf.tex\ } by Juan a. de la Puente
\item
  the \texttt{\ citep\ } function renders citations like
  \texttt{\ (Keohane,\ 1958)\ } instead of the default style of
  \texttt{\ Keohane\ (1958)\ }
\end{itemize}

\subsection{License}\label{license}

This template is licensed according to the MIT No Attribution license
(see \texttt{\ LICENSE.MD\ } ).

The files in the \texttt{\ CSL\ } folder are licensed according to
\texttt{\ CSL/LICENSE.md\ } (CC BY/SA 4.0) because it is a lightly
modified version of \texttt{\ apa.csl\ } by Brenton M. Wiernik which is
also licensed by a CC BY/SA license.

\href{/app?template=abiding-ifacconf&version=0.1.0}{Create project in
app}

\subsubsection{How to use}\label{how-to-use}

Click the button above to create a new project using this template in
the Typst app.

You can also use the Typst CLI to start a new project on your computer
using this command:

\begin{verbatim}
typst init @preview/abiding-ifacconf:0.1.0
\end{verbatim}

\includesvg[width=0.16667in,height=0.16667in]{/assets/icons/16-copy.svg}

\subsubsection{About}\label{about}

\begin{description}
\tightlist
\item[Author :]
\href{https://avonmoll.github.io}{Alexander Von Moll}
\item[License:]
MIT-0
\item[Current version:]
0.1.0
\item[Last updated:]
March 21, 2024
\item[First released:]
March 21, 2024
\item[Minimum Typst version:]
0.11.0
\item[Archive size:]
27.8 kB
\href{https://packages.typst.org/preview/abiding-ifacconf-0.1.0.tar.gz}{\pandocbounded{\includesvg[keepaspectratio]{/assets/icons/16-download.svg}}}
\item[Repository:]
\href{https://github.com/avonmoll/ifacconf-typst}{GitHub}
\item[Discipline s :]
\begin{itemize}
\tightlist
\item[]
\item
  \href{https://typst.app/universe/search/?discipline=computer-science}{Computer
  Science}
\item
  \href{https://typst.app/universe/search/?discipline=engineering}{Engineering}
\end{itemize}
\item[Categor y :]
\begin{itemize}
\tightlist
\item[]
\item
  \pandocbounded{\includesvg[keepaspectratio]{/assets/icons/16-atom.svg}}
  \href{https://typst.app/universe/search/?category=paper}{Paper}
\end{itemize}
\end{description}

\subsubsection{Where to report issues?}\label{where-to-report-issues}

This template is a project of Alexander Von Moll . Report issues on
\href{https://github.com/avonmoll/ifacconf-typst}{their repository} .
You can also try to ask for help with this template on the
\href{https://forum.typst.app}{Forum} .

Please report this template to the Typst team using the
\href{https://typst.app/contact}{contact form} if you believe it is a
safety hazard or infringes upon your rights.

\phantomsection\label{versions}
\subsubsection{Version history}\label{version-history}

\begin{longtable}[]{@{}ll@{}}
\toprule\noalign{}
Version & Release Date \\
\midrule\noalign{}
\endhead
\bottomrule\noalign{}
\endlastfoot
0.1.0 & March 21, 2024 \\
\end{longtable}

Typst GmbH did not create this template and cannot guarantee correct
functionality of this template or compatibility with any version of the
Typst compiler or app.


\section{Package List LaTeX/optimal-ovgu-thesis.tex}
\title{typst.app/universe/package/optimal-ovgu-thesis}

\phantomsection\label{banner}
\phantomsection\label{template-thumbnail}
\pandocbounded{\includegraphics[keepaspectratio]{https://packages.typst.org/preview/thumbnails/optimal-ovgu-thesis-0.1.1-small.webp}}

\section{optimal-ovgu-thesis}\label{optimal-ovgu-thesis}

{ 0.1.1 }

A thesis template for Otto von Guericke University Magdeburg

\href{/app?template=optimal-ovgu-thesis&version=0.1.1}{Create project in
app}

\phantomsection\label{readme}
This template was created for a master thesis at the faculty of computer
science (FIN), but should work as well for other faculties.

\subsection{File structure}\label{file-structure}

\begin{verbatim}
.
├── assets                          // Images, CSV-Files, etc. 
│   └── figure                      // Image files
│       └── optimal-ovgu-thesis    
├── chapter                         // Content
│   ├── 01-Einleitung.typ
│   ├── ...
│   └── 99-Appendix.typ
├── expose.typ                      // Exposé template
├── metadata.typ                    // Metadata and template config
├── thesis.bib                      // Bibliography (e.g. generated by Zotero + Better BibTex)
└── thesis.typ                      // Thesis template
\end{verbatim}

\subsection{Logos on the title page}\label{logos-on-the-title-page}

The header- and organisation-logo can be set in the
\texttt{\ metadata.typ\ } file (see example below). There are two
example logo files in \texttt{\ assets/figure/optimal-ovgu-thesis\ } .
Please refer to
\href{https://www.cd.ovgu.de/Fakult\%C3\%A4ten.html}{cd.ovgu.de} for
more information regarding the OvGU corporate design and for the signet
and logo of your faculty.

Header logos are set in \texttt{\ metadata.typ\ } :

\begin{Shaded}
\begin{Highlighting}[]
\NormalTok{// Example 1: Use UCC logo as organisation{-}logo and the FIN faculty header as header{-}logo}
\NormalTok{\#let organisation{-}logo = image("assets/figure/optimal{-}ovgu{-}thesis/ucc.svg", width: 2cm)}
\NormalTok{\#let header{-}logo = image("assets/figure/optimal{-}ovgu{-}thesis/fin{-}de.svg", width: 100\%)}

\NormalTok{// Example 2: Do not use logos at all}
\NormalTok{\#let organisation{-}logo = none}
\NormalTok{\#let header{-}logo = none}
\end{Highlighting}
\end{Shaded}

\subsection{Fonts}\label{fonts}

This template requires these two fonts to be installed on your system:

\begin{itemize}
\tightlist
\item
  New Computer Modern
\item
  New Computer Modern Sans
\end{itemize}

\subsubsection{NixOS}\label{nixos}

In your \texttt{\ configuration.nix\ } :

\begin{Shaded}
\begin{Highlighting}[]
\NormalTok{  fonts.packages = }\KeywordTok{with}\NormalTok{ pkgs}\OperatorTok{;} \OperatorTok{[}
\NormalTok{    liberation\_ttf }\CommentTok{\# here are your other fonts (liberation is just an example)}
  \OperatorTok{]} \OperatorTok{++}\NormalTok{ texlive.newcomputermodern.pkgs; }\CommentTok{\# ← New Computer Modern font}
\end{Highlighting}
\end{Shaded}

\subsection{Development}\label{development}

In case you want to contribute, check out the repo into a
\href{https://github.com/typst/packages?tab=readme-ov-file\#local-packages}{typst
package directory}

\subsubsection{Example for Linux:}\label{example-for-linux}

Local package path:
\texttt{\ \textasciitilde{}/.local/share/typst/packages/local/optimal-ovgu-thesis/0.1.1\ }

\begin{Shaded}
\begin{Highlighting}[]
\FunctionTok{mkdir} \AttributeTok{{-}p}\NormalTok{ \textasciitilde{}/.local/share/typst/packages/local/optimal{-}ovgu{-}thesis}
\BuiltInTok{cd}\NormalTok{ \textasciitilde{}/.local/share/typst/packages/local/optimal{-}ovgu{-}thesis}
\FunctionTok{git}\NormalTok{ clone git@github.com:v411e/optimal{-}ovgu{-}thesis.git}
\FunctionTok{mv}\NormalTok{ optimal{-}ovgu{-}thesis 0.1.1}
\end{Highlighting}
\end{Shaded}

This will make the package available locally, so you can use
\texttt{\ typst\ init\ "@local/optimal-ovgu-thesis:0.1.1"\ } to create a
test-project from the template.

\href{/app?template=optimal-ovgu-thesis&version=0.1.1}{Create project in
app}

\subsubsection{How to use}\label{how-to-use}

Click the button above to create a new project using this template in
the Typst app.

You can also use the Typst CLI to start a new project on your computer
using this command:

\begin{verbatim}
typst init @preview/optimal-ovgu-thesis:0.1.1
\end{verbatim}

\includesvg[width=0.16667in,height=0.16667in]{/assets/icons/16-copy.svg}

\subsubsection{About}\label{about}

\begin{description}
\tightlist
\item[Author :]
\href{https://github.com/v411e}{Valentin Rieß}
\item[License:]
MIT
\item[Current version:]
0.1.1
\item[Last updated:]
November 25, 2024
\item[First released:]
May 17, 2024
\item[Archive size:]
36.9 kB
\href{https://packages.typst.org/preview/optimal-ovgu-thesis-0.1.1.tar.gz}{\pandocbounded{\includesvg[keepaspectratio]{/assets/icons/16-download.svg}}}
\item[Repository:]
\href{https://github.com/v411e/optimal-ovgu-thesis}{GitHub}
\item[Categor y :]
\begin{itemize}
\tightlist
\item[]
\item
  \pandocbounded{\includesvg[keepaspectratio]{/assets/icons/16-mortarboard.svg}}
  \href{https://typst.app/universe/search/?category=thesis}{Thesis}
\end{itemize}
\end{description}

\subsubsection{Where to report issues?}\label{where-to-report-issues}

This template is a project of Valentin Rieß . Report issues on
\href{https://github.com/v411e/optimal-ovgu-thesis}{their repository} .
You can also try to ask for help with this template on the
\href{https://forum.typst.app}{Forum} .

Please report this template to the Typst team using the
\href{https://typst.app/contact}{contact form} if you believe it is a
safety hazard or infringes upon your rights.

\phantomsection\label{versions}
\subsubsection{Version history}\label{version-history}

\begin{longtable}[]{@{}ll@{}}
\toprule\noalign{}
Version & Release Date \\
\midrule\noalign{}
\endhead
\bottomrule\noalign{}
\endlastfoot
0.1.1 & November 25, 2024 \\
\href{https://typst.app/universe/package/optimal-ovgu-thesis/0.1.0/}{0.1.0}
& May 17, 2024 \\
\end{longtable}

Typst GmbH did not create this template and cannot guarantee correct
functionality of this template or compatibility with any version of the
Typst compiler or app.


\section{Package List LaTeX/modern-hsh-thesis.tex}
\title{typst.app/universe/package/modern-hsh-thesis}

\phantomsection\label{banner}
\phantomsection\label{template-thumbnail}
\pandocbounded{\includegraphics[keepaspectratio]{https://packages.typst.org/preview/thumbnails/modern-hsh-thesis-1.0.0-small.webp}}

\section{modern-hsh-thesis}\label{modern-hsh-thesis}

{ 1.0.0 }

Template for writing a bachelors or masters thesis at the Hochschule
Hannover, Faculty 4.

\href{/app?template=modern-hsh-thesis&version=1.0.0}{Create project in
app}

\phantomsection\label{readme}
Version 1.0.0

A template for writing a bachelors or masters thesis at the Hochschule
Hannover, Faculty 4.

\subsection{Getting Started}\label{getting-started}

\subsubsection{WebApp}\label{webapp}

Choose the template in the typst web app and follow the instructions
there.

\subsubsection{Terminal}\label{terminal}

\begin{Shaded}
\begin{Highlighting}[]
\ExtensionTok{typst}\NormalTok{ init @preview/modern{-}hsh{-}thesis:1.0.0}
\end{Highlighting}
\end{Shaded}

\subsubsection{Import}\label{import}

\begin{Shaded}
\begin{Highlighting}[]
\NormalTok{\#import "@preview/modern{-}hsh{-}thesis:1.0.0": *}

\NormalTok{\#show: project.with(}
\NormalTok{  title: "Beispiel{-}Titel",}
\NormalTok{  subtitle: "Bachelorarbeit im Studiengang Mediendesigninformatik",}
\NormalTok{  author: "Vorname Nachname",}
\NormalTok{  author\_email: "vorname@nachname.tld",}
\NormalTok{  matrikelnummer: 1234567,}
\NormalTok{  prof: [}
\NormalTok{    Prof. Dr. Vorname Nachname\textbackslash{}}
\NormalTok{    Abteilung Informatik, Fakultät IV\textbackslash{}}
\NormalTok{    Hochschule Hannover\textbackslash{}    }
\NormalTok{    \#link("mailto:vorname.nachname@hs{-}hannover.de")}
    
\NormalTok{  ],}
\NormalTok{  second\_prof: [}
\NormalTok{    Prof. Dr. Vorname Nachname\textbackslash{}}
\NormalTok{    Abteilung Informatik, Fakultät IV\textbackslash{}}
\NormalTok{    Hochschule Hannover\textbackslash{}    }
\NormalTok{    \#link("mailto:vorname.nachname@hs{-}hannover.de")}
\NormalTok{  ],}
\NormalTok{  date: "01. August 2024",}
\NormalTok{  glossaryColumns: 1,}
\NormalTok{  bibliography: bibliography(("sources.bib", "sources.yaml"), style: "institute{-}of{-}electrical{-}and{-}electronics{-}engineers", title: "Literaturverzeichnis")}
\NormalTok{)}
\end{Highlighting}
\end{Shaded}

\subsubsection{Additional functions}\label{additional-functions}

\texttt{\ customFunctions.typ\ } contains additional functions that can
be used in the template.

\texttt{\ \#smallLine\ } : A small line that can be used to separate
sections.

\texttt{\ \#task\ } : A card that can be used to create a list of tracks
(see example in 1-einleitung.typ).

\texttt{\ \#track\ } or \texttt{\ \#\#narrowTrack\ } : A track that can
be displayed inside a task (see example in 1-einleitung.typ).

\texttt{\ \#useCase\ } : Display a Use Case (see example in
1-einleitung.typ).

\texttt{\ \#attributedQuote\ } : Display a quote with an attribution.

\texttt{\ \#diagramFigure\ } , \texttt{\ \#codeFigure\ } ,
\texttt{\ \#imageFigure\ } , \texttt{\ \#treeFigure\ } : Wrap an
image/code/diagram/tree-list in a figure with a caption.

\texttt{\ \#imageFigureNoPad\ } : Display a figure without padding.

\texttt{\ \#getCurrentHeadingHydra\ } , \texttt{\ \#getCurrentHeading\ }
: Get the heading of the current page.

\subsubsection{Development Environment}\label{development-environment}

\begin{enumerate}
\tightlist
\item
  Install Typst \url{https://github.com/typst-community/typst-install}
\item
  Clone the repository
\item
  CD into the repository
\item
  Run
  \texttt{\ git\ pull\ \&\&\ just\ install\ \&\&\ just\ install-preview\ }
  to install/update the template
\item
  Run
  \texttt{\ typst\ init\ @local/modern-hsh-thesis:1.0.0\ \&\&\ typst\ compile\ modern-hsh-thesis/main.typ\ }
  to compile the template
\end{enumerate}

\subsection{Additional Documentation}\label{additional-documentation}

Take a look at this complete Bachelor’s thesis example using the
\texttt{\ modern-hsh-thesis\ } template:
\href{https://github.com/MrToWy/Bachelorarbeit}{Bachelor’s Thesis
Example}

\href{/app?template=modern-hsh-thesis&version=1.0.0}{Create project in
app}

\subsubsection{How to use}\label{how-to-use}

Click the button above to create a new project using this template in
the Typst app.

You can also use the Typst CLI to start a new project on your computer
using this command:

\begin{verbatim}
typst init @preview/modern-hsh-thesis:1.0.0
\end{verbatim}

\includesvg[width=0.16667in,height=0.16667in]{/assets/icons/16-copy.svg}

\subsubsection{About}\label{about}

\begin{description}
\tightlist
\item[Author :]
\href{https://github.com/MrToWy}{Tobias Wylega}
\item[License:]
MIT
\item[Current version:]
1.0.0
\item[Last updated:]
September 8, 2024
\item[First released:]
September 8, 2024
\item[Minimum Typst version:]
0.11.1
\item[Archive size:]
31.6 kB
\href{https://packages.typst.org/preview/modern-hsh-thesis-1.0.0.tar.gz}{\pandocbounded{\includesvg[keepaspectratio]{/assets/icons/16-download.svg}}}
\item[Repository:]
\href{https://github.com/MrToWy/hsh-thesis}{GitHub}
\item[Categor y :]
\begin{itemize}
\tightlist
\item[]
\item
  \pandocbounded{\includesvg[keepaspectratio]{/assets/icons/16-mortarboard.svg}}
  \href{https://typst.app/universe/search/?category=thesis}{Thesis}
\end{itemize}
\end{description}

\subsubsection{Where to report issues?}\label{where-to-report-issues}

This template is a project of Tobias Wylega . Report issues on
\href{https://github.com/MrToWy/hsh-thesis}{their repository} . You can
also try to ask for help with this template on the
\href{https://forum.typst.app}{Forum} .

Please report this template to the Typst team using the
\href{https://typst.app/contact}{contact form} if you believe it is a
safety hazard or infringes upon your rights.

\phantomsection\label{versions}
\subsubsection{Version history}\label{version-history}

\begin{longtable}[]{@{}ll@{}}
\toprule\noalign{}
Version & Release Date \\
\midrule\noalign{}
\endhead
\bottomrule\noalign{}
\endlastfoot
1.0.0 & September 8, 2024 \\
\end{longtable}

Typst GmbH did not create this template and cannot guarantee correct
functionality of this template or compatibility with any version of the
Typst compiler or app.


\section{Package List LaTeX/truthfy.tex}
\title{typst.app/universe/package/truthfy}

\phantomsection\label{banner}
\section{truthfy}\label{truthfy}

{ 0.5.0 }

Make empty or automatically filled truth table

\phantomsection\label{readme}
Make an empty or filled truth table in Typst

\begin{Shaded}
\begin{Highlighting}[]
\ExtensionTok{truth{-}table{-}empty}\ErrorTok{(}\ExtensionTok{info:}\NormalTok{ array}\PreprocessorTok{[}\SpecialStringTok{math\_block}\PreprocessorTok{]}\NormalTok{, data: array}\PreprocessorTok{[}\SpecialStringTok{str}\PreprocessorTok{]}\KeywordTok{)}\BuiltInTok{:}\NormalTok{ table}
    \CommentTok{\# Create an empty (or filled with "data") truth table. }

\ExtensionTok{truth{-}table}\ErrorTok{(}\ExtensionTok{..info:}\NormalTok{ array}\PreprocessorTok{[}\SpecialStringTok{math\_block}\PreprocessorTok{]}\KeywordTok{)}\BuiltInTok{:}\NormalTok{ table}
    \CommentTok{\# Create a filled truth table.}

\ExtensionTok{karnaugh{-}empty}\ErrorTok{(}\ExtensionTok{info:}\NormalTok{ array}\PreprocessorTok{[}\SpecialStringTok{math\_block}\PreprocessorTok{]}\NormalTok{, data: array}\PreprocessorTok{[}\SpecialStringTok{str}\PreprocessorTok{]}\KeywordTok{)}\BuiltInTok{:}\NormalTok{ table}
    \CommentTok{\# Create an empty karnaugh table.}

\ExtensionTok{NAND:}\NormalTok{ Equivalent to sym.arrow.t}
\ExtensionTok{NOR:}\NormalTok{ Equivalent to sym.arrow.b}
\end{Highlighting}
\end{Shaded}

\subsection{\texorpdfstring{\texttt{\ sc\ }}{ sc }}\label{sc}

Theses functions have a new named argument, called \texttt{\ sc\ } for
symbol-convention.

You can add you own function to customise the render of the 0 and the 1.
See examples.

Syntax:

\begin{Shaded}
\begin{Highlighting}[]
\NormalTok{\#let sc(symb) = \{}
\NormalTok{    if (symb) \{}
\NormalTok{        "an one"}
\NormalTok{    \} else \{}
\NormalTok{        "a zero"}
\NormalTok{    \}}
\NormalTok{\}}
\end{Highlighting}
\end{Shaded}

\subsection{\texorpdfstring{\texttt{\ reverse\ }}{ reverse }}\label{reverse}

Reverse your table, see issue \#3

\subsection{Simple}\label{simple}

\begin{Shaded}
\begin{Highlighting}[]
\NormalTok{\#import "@preview/truthfy:0.4.0": truth{-}table, truth{-}table{-}empty}

\NormalTok{\#truth{-}table($A and B$, $B or A$, $A =\textgreater{} B$, $(A =\textgreater{} B) \textless{}=\textgreater{} A$, $ A xor B$)}

\NormalTok{\#truth{-}table($p =\textgreater{} q$, $not p =\textgreater{} (q =\textgreater{} p)$, $p or q$, $not p or q$)}
\end{Highlighting}
\end{Shaded}

\pandocbounded{\includegraphics[keepaspectratio]{https://github.com/Thumuss/truthfy/assets/42680097/7edb921d-659e-4348-a12a-07bcc3822012}}

\begin{Shaded}
\begin{Highlighting}[]
\NormalTok{\#import "@preview/truthfy:0.4.0": truth{-}table, truth{-}table{-}empty}

\NormalTok{\#truth{-}table(sc: (a) =\textgreater{} \{if (a) \{"a"\} else \{"b"\}\}, $a and b$)}

\NormalTok{\#truth{-}table{-}empty(sc: (a) =\textgreater{} \{if (a) \{"x"\} else \{"$"\}\}, ($a and b$,), (false, [], true))}
\end{Highlighting}
\end{Shaded}

\pandocbounded{\includegraphics[keepaspectratio]{https://github.com/Thumuss/truthfy/assets/42680097/1ccf6077-5cfb-4643-b621-1dc9529b8176}}

If you have any idea to add in this package, add a new issue
\href{https://github.com/Thumuss/truthfy/issues}{here} !

\texttt{\ 0.1.0\ } : Create the package.\\
\texttt{\ 0.2.0\ } :

\begin{itemize}
\tightlist
\item
  You can now use \texttt{\ t\ } , \texttt{\ r\ } , \texttt{\ u\ } ,
  \texttt{\ e\ } , \texttt{\ f\ } , \texttt{\ a\ } , \texttt{\ l\ } ,
  \texttt{\ s\ } without any problems!
\item
  You can now add subscript to a letter
\item
  Only \texttt{\ generate-table\ } and \texttt{\ generate-empty\ } are
  now exported
\item
  Better example with more cases
\item
  Implemented the \texttt{\ a\ ?\ b\ :\ c\ } operator\\
\end{itemize}

\texttt{\ 0.3.0\ } :

\begin{itemize}
\tightlist
\item
  Changing the name of \texttt{\ generate-table\ } and
  \texttt{\ generate-empty\ } to \texttt{\ truth-table\ } and
  \texttt{\ truth-table-empty\ }
\item
  Adding support of \texttt{\ NAND\ } and \texttt{\ NOR\ } operators.
\item
  Adding support of custom \texttt{\ sc\ } function.
\item
  Better example and \href{http://readme.md/}{README.md}
\end{itemize}

\texttt{\ 0.4.0\ } :

\begin{itemize}
\tightlist
\item
  Add \texttt{\ karnaugh-empty\ }
\item
  Images re-added (see \#2)
\item
  Add \texttt{\ reverse\ } option (see \#3)
\end{itemize}

\subsubsection{How to add}\label{how-to-add}

Copy this into your project and use the import as \texttt{\ truthfy\ }

\begin{verbatim}
#import "@preview/truthfy:0.5.0"
\end{verbatim}

\includesvg[width=0.16667in,height=0.16667in]{/assets/icons/16-copy.svg}

Check the docs for
\href{https://typst.app/docs/reference/scripting/\#packages}{more
information on how to import packages} .

\subsubsection{About}\label{about}

\begin{description}
\tightlist
\item[Author :]
\href{https://github.com/Thumuss}{Quemin Thomas}
\item[License:]
MIT
\item[Current version:]
0.5.0
\item[Last updated:]
September 14, 2024
\item[First released:]
October 9, 2023
\item[Archive size:]
4.54 kB
\href{https://packages.typst.org/preview/truthfy-0.5.0.tar.gz}{\pandocbounded{\includesvg[keepaspectratio]{/assets/icons/16-download.svg}}}
\item[Repository:]
\href{https://github.com/Thumuss/truthfy}{GitHub}
\end{description}

\subsubsection{Where to report issues?}\label{where-to-report-issues}

This package is a project of Quemin Thomas . Report issues on
\href{https://github.com/Thumuss/truthfy}{their repository} . You can
also try to ask for help with this package on the
\href{https://forum.typst.app}{Forum} .

Please report this package to the Typst team using the
\href{https://typst.app/contact}{contact form} if you believe it is a
safety hazard or infringes upon your rights.

\phantomsection\label{versions}
\subsubsection{Version history}\label{version-history}

\begin{longtable}[]{@{}ll@{}}
\toprule\noalign{}
Version & Release Date \\
\midrule\noalign{}
\endhead
\bottomrule\noalign{}
\endlastfoot
0.5.0 & September 14, 2024 \\
\href{https://typst.app/universe/package/truthfy/0.4.0/}{0.4.0} & June
10, 2024 \\
\href{https://typst.app/universe/package/truthfy/0.3.0/}{0.3.0} &
February 6, 2024 \\
\href{https://typst.app/universe/package/truthfy/0.2.0/}{0.2.0} &
October 16, 2023 \\
\href{https://typst.app/universe/package/truthfy/0.1.0/}{0.1.0} &
October 9, 2023 \\
\end{longtable}

Typst GmbH did not create this package and cannot guarantee correct
functionality of this package or compatibility with any version of the
Typst compiler or app.


\section{Package List LaTeX/canonical-nthu-thesis.tex}
\title{typst.app/universe/package/canonical-nthu-thesis}

\phantomsection\label{banner}
\phantomsection\label{template-thumbnail}
\pandocbounded{\includegraphics[keepaspectratio]{https://packages.typst.org/preview/thumbnails/canonical-nthu-thesis-0.2.0-small.webp}}

\section{canonical-nthu-thesis}\label{canonical-nthu-thesis}

{ 0.2.0 }

A template for master theses and doctoral dissertations for NTHU
(National Tsing Hua University).

\href{/app?template=canonical-nthu-thesis&version=0.2.0}{Create project
in app}

\phantomsection\label{readme}
A \href{https://typst.app/docs/}{Typst} template for master theses and
doctoral dissertations for NTHU (National Tsing Hua University).

國立æ¸\ldots è?¯å¤§å­¸ç¢©å£«ï¼ˆå?šå£«ï¼‰è«--æ--‡
\href{https://typst.app/docs/}{Typst} 模�。

\begin{itemize}
\tightlist
\item
  \href{https://typst.app/universe/package/canonical-nthu-thesis}{Typst
  Universe Package}
\item
  \href{https://codeberg.org/kotatsuyaki/canonical-nthu-thesis}{Codeberg
  Repo}
\end{itemize}

\pandocbounded{\includegraphics[keepaspectratio]{https://github.com/typst/packages/raw/main/packages/preview/canonical-nthu-thesis/0.2.0/covers.png}}

\subsection{Usage}\label{usage}

\subsubsection{Installing the Chinese
fonts}\label{installing-the-chinese-fonts}

This template uses the official fonts from the Ministry of Education of
Taiwan (Edukai/TW-MOE-Std-Kai), which are required to be downloaded and
installed manually from
\href{https://language.moe.gov.tw/001/Upload/Files/site_content/M0001/edukai-5.0.zip}{language.moe.gov.tw}
. The Typst web app has the fonts installed by default, so there is no
need to install the fonts on the web app.

此模æ?¿ä¸­æ--‡éƒ¨åˆ†ä½¿ç''¨æ•™è‚²éƒ¨æ¨™æº--楷書å­---é«''(Edukai/TW-MOE-Std-Kai),在本地編譯æ--‡ä»¶å‰?需è¦?自
\href{https://language.moe.gov.tw/001/Upload/Files/site_content/M0001/edukai-5.0.zip}{language.moe.gov.tw}
下載並手動安�。Typst web
appå·²é~?è£?該å­---é«'',æ•\ldots 無需é¡?å¤--安è£?。

\subsubsection{Editing}\label{editing}

All the content of the thesis are in the \texttt{\ thesis.typ\ } file.
In the beginning of \texttt{\ thesis.typ\ } , there is a call to the
\texttt{\ setup-thesis(info,\ style)\ } function that configures the
metadata (the titles and the author etc.) and the styling of the thesis
document. Replace the values with your own.

所有è«--æ--‡å\ldots§å®¹çš†ä½?æ--¼ \texttt{\ thesis.typ\ }
æª''案å\ldots§ã€‚該æª''案å‰?段的部分å`¼å?«äº†
\texttt{\ setup-thesis(info,\ style)\ }
函å¼?,設置è«--æ--‡çš„雜é~\ldots 資訊(標題å?Šä½œè€\ldots 等)å?Šå¤--觀é?¸é~\ldots ,請置æ?›ç‚ºè‡ªå·±çš„資訊。

\subsubsection{Local usage}\label{local-usage}

\begin{Shaded}
\begin{Highlighting}[]
\ExtensionTok{$}\NormalTok{ typst init @preview/canonical{-}nthu{-}thesis:0.2.0 my{-}thesis}
\ExtensionTok{$}\NormalTok{ cd my{-}thesis}
\ExtensionTok{$}\NormalTok{ typst watch thesis.typ}
\end{Highlighting}
\end{Shaded}

\subsection{Development}\label{development}

Development and issue tracking happens on the
\href{https://codeberg.org/kotatsuyaki/canonical-nthu-thesis}{repository
on Codeberg} .

\subsection{License}\label{license}

This project is licensed under the MIT License.

\href{/app?template=canonical-nthu-thesis&version=0.2.0}{Create project
in app}

\subsubsection{How to use}\label{how-to-use}

Click the button above to create a new project using this template in
the Typst app.

You can also use the Typst CLI to start a new project on your computer
using this command:

\begin{verbatim}
typst init @preview/canonical-nthu-thesis:0.2.0
\end{verbatim}

\includesvg[width=0.16667in,height=0.16667in]{/assets/icons/16-copy.svg}

\subsubsection{About}\label{about}

\begin{description}
\tightlist
\item[Author :]
kotatsuyaki
\item[License:]
MIT
\item[Current version:]
0.2.0
\item[Last updated:]
August 1, 2024
\item[First released:]
June 17, 2024
\item[Archive size:]
48.4 kB
\href{https://packages.typst.org/preview/canonical-nthu-thesis-0.2.0.tar.gz}{\pandocbounded{\includesvg[keepaspectratio]{/assets/icons/16-download.svg}}}
\item[Repository:]
\href{https://codeberg.org/kotatsuyaki/canonical-nthu-thesis}{Codeberg}
\item[Categor y :]
\begin{itemize}
\tightlist
\item[]
\item
  \pandocbounded{\includesvg[keepaspectratio]{/assets/icons/16-mortarboard.svg}}
  \href{https://typst.app/universe/search/?category=thesis}{Thesis}
\end{itemize}
\end{description}

\subsubsection{Where to report issues?}\label{where-to-report-issues}

This template is a project of kotatsuyaki . Report issues on
\href{https://codeberg.org/kotatsuyaki/canonical-nthu-thesis}{their
repository} . You can also try to ask for help with this template on the
\href{https://forum.typst.app}{Forum} .

Please report this template to the Typst team using the
\href{https://typst.app/contact}{contact form} if you believe it is a
safety hazard or infringes upon your rights.

\phantomsection\label{versions}
\subsubsection{Version history}\label{version-history}

\begin{longtable}[]{@{}ll@{}}
\toprule\noalign{}
Version & Release Date \\
\midrule\noalign{}
\endhead
\bottomrule\noalign{}
\endlastfoot
0.2.0 & August 1, 2024 \\
\href{https://typst.app/universe/package/canonical-nthu-thesis/0.1.0/}{0.1.0}
& June 17, 2024 \\
\end{longtable}

Typst GmbH did not create this template and cannot guarantee correct
functionality of this template or compatibility with any version of the
Typst compiler or app.


\section{Package List LaTeX/rubber-article.tex}
\title{typst.app/universe/package/rubber-article}

\phantomsection\label{banner}
\phantomsection\label{template-thumbnail}
\pandocbounded{\includegraphics[keepaspectratio]{https://packages.typst.org/preview/thumbnails/rubber-article-0.1.0-small.webp}}

\section{rubber-article}\label{rubber-article}

{ 0.1.0 }

A simple template recreating the look of the classic LaTeX article.

\href{/app?template=rubber-article&version=0.1.0}{Create project in app}

\phantomsection\label{readme}
Version 0.1.0

This template is a intended as a starting point for creating documents,
which should have the classic LaTeX Article look.

\subsection{Getting Started}\label{getting-started}

These instructions will get you a copy of the project up and running on
the typst web app. Perhaps a short code example on importing the package
and a very simple teaser usage.

\begin{Shaded}
\begin{Highlighting}[]
\NormalTok{\#import "@preview/rubber{-}article:0.1.0": *}

\NormalTok{\#show: article.with()}

\NormalTok{\#maketitle(}
\NormalTok{  title: "The Title of the Paper",}
\NormalTok{  authors: (}
\NormalTok{    "Authors Name",}
\NormalTok{  ),}
\NormalTok{  date: "September 1970",}
\NormalTok{)}
\end{Highlighting}
\end{Shaded}

\subsection{Further Functionality}\label{further-functionality}

The template provides a few more functions to customize the document.

\begin{Shaded}
\begin{Highlighting}[]
\NormalTok{\#show article.with(}
\NormalTok{  lang:"de",}
\NormalTok{  eq{-}numbering:none,}
\NormalTok{  text{-}size:10pt,}
\NormalTok{  page{-}numbering: "1",}
\NormalTok{  page{-}numbering{-}align: center,}
\NormalTok{  heading{-}numbering: "1.1  ",}
\NormalTok{)}
\end{Highlighting}
\end{Shaded}

\href{/app?template=rubber-article&version=0.1.0}{Create project in app}

\subsubsection{How to use}\label{how-to-use}

Click the button above to create a new project using this template in
the Typst app.

You can also use the Typst CLI to start a new project on your computer
using this command:

\begin{verbatim}
typst init @preview/rubber-article:0.1.0
\end{verbatim}

\includesvg[width=0.16667in,height=0.16667in]{/assets/icons/16-copy.svg}

\subsubsection{About}\label{about}

\begin{description}
\tightlist
\item[Author :]
Niko Pikall
\item[License:]
Unlicense
\item[Current version:]
0.1.0
\item[Last updated:]
September 8, 2024
\item[First released:]
September 8, 2024
\item[Archive size:]
3.00 kB
\href{https://packages.typst.org/preview/rubber-article-0.1.0.tar.gz}{\pandocbounded{\includesvg[keepaspectratio]{/assets/icons/16-download.svg}}}
\item[Repository:]
\href{https://github.com/npikall/rubber-article.git}{GitHub}
\item[Discipline s :]
\begin{itemize}
\tightlist
\item[]
\item
  \href{https://typst.app/universe/search/?discipline=biology}{Biology}
\item
  \href{https://typst.app/universe/search/?discipline=chemistry}{Chemistry}
\item
  \href{https://typst.app/universe/search/?discipline=engineering}{Engineering}
\item
  \href{https://typst.app/universe/search/?discipline=geography}{Geography}
\item
  \href{https://typst.app/universe/search/?discipline=mathematics}{Mathematics}
\item
  \href{https://typst.app/universe/search/?discipline=physics}{Physics}
\end{itemize}
\item[Categor ies :]
\begin{itemize}
\tightlist
\item[]
\item
  \pandocbounded{\includesvg[keepaspectratio]{/assets/icons/16-atom.svg}}
  \href{https://typst.app/universe/search/?category=paper}{Paper}
\item
  \pandocbounded{\includesvg[keepaspectratio]{/assets/icons/16-speak.svg}}
  \href{https://typst.app/universe/search/?category=report}{Report}
\end{itemize}
\end{description}

\subsubsection{Where to report issues?}\label{where-to-report-issues}

This template is a project of Niko Pikall . Report issues on
\href{https://github.com/npikall/rubber-article.git}{their repository} .
You can also try to ask for help with this template on the
\href{https://forum.typst.app}{Forum} .

Please report this template to the Typst team using the
\href{https://typst.app/contact}{contact form} if you believe it is a
safety hazard or infringes upon your rights.

\phantomsection\label{versions}
\subsubsection{Version history}\label{version-history}

\begin{longtable}[]{@{}ll@{}}
\toprule\noalign{}
Version & Release Date \\
\midrule\noalign{}
\endhead
\bottomrule\noalign{}
\endlastfoot
0.1.0 & September 8, 2024 \\
\end{longtable}

Typst GmbH did not create this template and cannot guarantee correct
functionality of this template or compatibility with any version of the
Typst compiler or app.


\section{Package List LaTeX/silky-report-insa.tex}
\title{typst.app/universe/package/silky-report-insa}

\phantomsection\label{banner}
\phantomsection\label{template-thumbnail}
\pandocbounded{\includegraphics[keepaspectratio]{https://packages.typst.org/preview/thumbnails/silky-report-insa-0.4.0-small.webp}}

\section{silky-report-insa}\label{silky-report-insa}

{ 0.4.0 }

A template made for reports and other documents of INSA, a French
engineering school.

\href{/app?template=silky-report-insa&version=0.4.0}{Create project in
app}

\phantomsection\label{readme}
Typst Template for full documents and reports for the french engineering
school INSA.

\subsection{Table of contents}\label{table-of-contents}

\begin{enumerate}
\tightlist
\item
  \href{https://github.com/typst/packages/raw/main/packages/preview/silky-report-insa/0.4.0/\#examples}{Examples
  \& Usage}

  \begin{enumerate}
  \tightlist
  \item
    \href{https://github.com/typst/packages/raw/main/packages/preview/silky-report-insa/0.4.0/\#\%F0\%9F\%A7\%AA-tp-report}{🧪
    TP report}
  \item
    \href{https://github.com/typst/packages/raw/main/packages/preview/silky-report-insa/0.4.0/\#\%F0\%9F\%93\%9A-internship-report}{ðŸ``š
    Internship report}
  \item
    \href{https://github.com/typst/packages/raw/main/packages/preview/silky-report-insa/0.4.0/\#\%F0\%9F\%97\%92\%EF\%B8\%8F-blank-templates}{ðŸ---'ï¸?
    Blank templates}
  \end{enumerate}
\item
  \href{https://github.com/typst/packages/raw/main/packages/preview/silky-report-insa/0.4.0/\#fonts}{Fonts
  information}
\item
  \href{https://github.com/typst/packages/raw/main/packages/preview/silky-report-insa/0.4.0/\#notes}{Notes}
\item
  \href{https://github.com/typst/packages/raw/main/packages/preview/silky-report-insa/0.4.0/\#license}{License}
\item
  \href{https://github.com/typst/packages/raw/main/packages/preview/silky-report-insa/0.4.0/\#changelog}{Changelog}
\end{enumerate}

\subsection{Examples \& Usage}\label{examples-usage}

\subsubsection{🧪 TP report}\label{uxf0uxffuxaa-tp-report}

\pandocbounded{\includegraphics[keepaspectratio]{https://github.com/typst/packages/raw/main/packages/preview/silky-report-insa/0.4.0/thumbnail-insa-report.png}}

This is the default report for the \texttt{\ silky-report-insa\ }
package. It uses the \texttt{\ insa-report\ } show rule.\\
It is primarily used for reports of Practical Works (Travaux Pratiques).

\paragraph{Example}\label{example}

\begin{Shaded}
\begin{Highlighting}[]
\NormalTok{\#import "@preview/silky{-}report{-}insa:0.4.0": *}
\NormalTok{\#show: doc =\textgreater{} insa{-}report(}
\NormalTok{  id: 3,}
\NormalTok{  pre{-}title: "STPI 2",}
\NormalTok{  title: "Interférences et diffraction",}
\NormalTok{  authors: [}
\NormalTok{    *LE JEUNE Youenn*}

\NormalTok{    *MAUVY Eva*}
    
\NormalTok{    Groupe D}

\NormalTok{    Binôme 5}
\NormalTok{  ],}
\NormalTok{  date: "11/04/2023",}
\NormalTok{  insa: "rennes",}
\NormalTok{  doc)}

\NormalTok{= Introduction}
\NormalTok{Le but de ce TP est d’interpréter les figures de diffraction observées avec différents objets diffractants}
\NormalTok{et d’en déduire les dimensions de ces objets.}

\NormalTok{= Partie théorique {-} Phénomène d\textquotesingle{}interférence}
\NormalTok{== Diffraction par une fente double}
\NormalTok{Lors du passage de la lumière par une fente double de largeur $a$ et de distance $b$ entre les centres}
\NormalTok{des fentes...}
\end{Highlighting}
\end{Shaded}

\paragraph{Parameters}\label{parameters}

\begin{longtable}[]{@{}llll@{}}
\toprule\noalign{}
Parameter & Description & Type & Example \\
\midrule\noalign{}
\endhead
\bottomrule\noalign{}
\endlastfoot
\textbf{id} & TP number & int & \texttt{\ 1\ } \\
\textbf{pre-title} & Text written before the title & str &
\texttt{\ "STPI\ 2"\ } \\
\textbf{title} & Title of the TP & str &
\texttt{\ "Interférences\ et\ diffraction"\ } \\
\textbf{authors} & Authors & content &
\texttt{\ {[}\textbackslash{}*LE\ JEUNE\ Youenn\textbackslash{}*{]}\ } \\
\textbf{date} & Date of the TP & datetime/str &
\texttt{\ "11/04/2023"\ } \\
\textbf{insa} & INSA name ( \texttt{\ rennes\ } , \texttt{\ hdf\ } …)
& str & \texttt{\ "rennes"\ } \\
\textbf{lang} & Language & str & \texttt{\ "fr"\ } \\
\end{longtable}

\subsubsection{ðŸ``š Internship
report}\label{uxf0uxffux161-internship-report}

\pandocbounded{\includegraphics[keepaspectratio]{https://github.com/typst/packages/raw/main/packages/preview/silky-report-insa/0.4.0/thumbnail-insa-stage.png}}

If you want to make an internship report, you will need to use another
show rule: \texttt{\ insa-stage\ } .

\paragraph{Example}\label{example-1}

\begin{Shaded}
\begin{Highlighting}[]
\NormalTok{\#import "@preview/silky{-}report{-}insa:0.4.0": *}
\NormalTok{\#show: doc =\textgreater{} insa{-}stage(}
\NormalTok{  "Youenn LE JEUNE",}
\NormalTok{  "INFO",}
\NormalTok{  "2023{-}2024",}
\NormalTok{  "Real{-}time virtual interaction with deformable structure",}
\NormalTok{  "Sapienza University of Rome",}
\NormalTok{  image("logo{-}example.png"),}
\NormalTok{  "Marilena VENDITELLI",}
\NormalTok{  "Bertrand COUASNON",}
\NormalTok{  [}
\NormalTok{    Résumé du stage en français.}
\NormalTok{  ],}
\NormalTok{  [}
\NormalTok{    Summary of the internship in english.}
\NormalTok{  ],}
\NormalTok{  insa: "rennes",}
\NormalTok{  lang: "fr",}
\NormalTok{  doc}
\NormalTok{)}

\NormalTok{= Introduction}
\NormalTok{Présentation de l\textquotesingle{}entreprise, tout ça tout ça.}

\NormalTok{\#pagebreak()}
\NormalTok{= Travail réalisé}
\NormalTok{== Première partie}
\NormalTok{Blabla}

\NormalTok{== Seconde partie}
\NormalTok{Bleble}

\NormalTok{\#pagebreak()}
\NormalTok{= Conclusion}
\NormalTok{Conclusion random}

\NormalTok{\#pagebreak()}
\NormalTok{= Annexes}
\end{Highlighting}
\end{Shaded}

This template can also be used for a report that is written in english:
in this case, add the \texttt{\ lang:\ "en"\ } parameter to the function
call in the show rule.

\paragraph{Parameters}\label{parameters-1}

\begin{longtable}[]{@{}lllll@{}}
\toprule\noalign{}
\textbf{Parameter} & Required & Type & Description & Example \\
\midrule\noalign{}
\endhead
\bottomrule\noalign{}
\endlastfoot
\textbf{name} & yes & str & Name of the student &
\texttt{\ "Youenn\ LE\ JEUNE"\ } \\
\textbf{department} & yes & str & Department of the student &
\texttt{\ "INFO"\ } \\
\textbf{year} & yes & str & School year during the internship &
\texttt{\ "2023-2024"\ } \\
\textbf{title} & yes & str & Title of the internship &
\texttt{\ "Real-time\ virtual\ interaction\ with\ deformable\ structure"\ } \\
\textbf{company} & yes & str & Company &
\texttt{\ Sapienza\ University\ of\ Rome\ } \\
\textbf{company-logo} & yes & content & Logo of the company &
\texttt{\ image("logo-example.png")\ } \\
\textbf{company-tutor} & yes & str & Tutor in the company &
\texttt{\ "Marilena\ VENDITELLI"\ } \\
\textbf{insa-tutor} & yes & str & Tutor at INSA &
\texttt{\ "Bertrand\ COUASNON"\ } \\
\textbf{insa-tutor-suffix} & no & str & Suffix at the end of
“encadrant� in french & \texttt{\ "e"\ } \\
\textbf{summary-french} & yes & content & Summary in French &
\texttt{\ {[}\ Résumé\ du\ stage\ en\ français.\ {]}\ } \\
\textbf{summary-english} & yes & content & Summary in English &
\texttt{\ {[}\ Summary\ of\ the\ internship\ in\ english.\ {]}\ } \\
\textbf{student-suffix} & no & str & Suffix at the end of
“ingénieur� in french & \texttt{\ "e"\ } \\
\textbf{thanks-page} & no & content & Special thanks page. &
\texttt{\ {[}\ Thanks\ to\ my\ *supervisor*,\ blah\ blah\ blah.\ {]}\ } \\
\textbf{omit-outline} & no & bool & Whether to skip the outline page or
not & \texttt{\ false\ } \\
\textbf{insa} & no & str & INSA name ( \texttt{\ rennes\ } ,
\texttt{\ hdf\ } …) & \texttt{\ "rennes"\ } \\
\textbf{lang} & no & str & Language of the template. Some strings are
translated. & \texttt{\ "fr"\ } \\
\end{longtable}

\subsubsection{ðŸ---'ï¸? Blank
templates}\label{uxf0uxffuxef-blank-templates}

\pandocbounded{\includegraphics[keepaspectratio]{https://github.com/typst/packages/raw/main/packages/preview/silky-report-insa/0.4.0/thumbnail-insa-document.png}}

If you do not want the preformatted output with “TP x�, the title
and date in the header, etc. you can simply use the
\texttt{\ insa-document\ } show rule and customize all by yourself.

\paragraph{Blank template types}\label{blank-template-types}

The graphic charter provides 3 different document types, that are
translated in this Typst template under those names:

\begin{itemize}
\tightlist
\item
  \textbf{\texttt{\ light\ }} , which does not have many color and can
  be printed easily. Has 3 spots to write on the cover:
  \texttt{\ cover-top-left\ } , \texttt{\ cover-middle-left\ } and
  \texttt{\ cover-bottom-right\ } .
\item
  \textbf{\texttt{\ colored\ }} , which is beautiful but consumes a lot
  of ink to print. Only has 1 spot to write on the cover:
  \texttt{\ cover-top-left\ } .
\item
  \textbf{\texttt{\ pfe\ }} , which is primarily used for internship
  reports. Has 4 spots to write on both the front and back covers:
  \texttt{\ cover-top-left\ } , \texttt{\ cover-middle-left\ } ,
  \texttt{\ cover-bottom-right\ } and \texttt{\ back-cover\ } .
\end{itemize}

The document type must be the first argument of the
\texttt{\ insa-document\ } function.

\paragraph{Example}\label{example-2}

\begin{Shaded}
\begin{Highlighting}[]
\NormalTok{\#import "@preview/silky{-}report{-}insa:0.4.0": *}
\NormalTok{\#show: doc =\textgreater{} insa{-}document(}
\NormalTok{  "light",}
\NormalTok{  cover{-}top{-}left: [*Document important*],}
\NormalTok{  cover{-}middle{-}left: [}
\NormalTok{    NOM Prénom}

\NormalTok{    Département INFO}
\NormalTok{  ],}
\NormalTok{  cover{-}bottom{-}right: "uwu",}
\NormalTok{  page{-}header: "En{-}tête au pif",}
\NormalTok{  doc}
\NormalTok{)}
\end{Highlighting}
\end{Shaded}

\paragraph{Parameters}\label{parameters-2}

\begin{longtable}[]{@{}lll@{}}
\toprule\noalign{}
\textbf{Parameter} & Type & Description \\
\midrule\noalign{}
\endhead
\bottomrule\noalign{}
\endlastfoot
\textbf{cover-type} & str & ( \textbf{REQUIRED} ) Type of cover.
Available are: light, colored, pfe. \\
\textbf{cover-top-left} & content & \\
\textbf{cover-middle-left} & content & \\
\textbf{cover-bottom-right} & content & \\
\textbf{back-cover} & content & What to display on the back cover. \\
\textbf{page-header} & content & Header of the pages (except the front
and back). If \texttt{\ none\ } , will display the INSA logo. If not
empty, will display the passed content with an underline. \\
\textbf{page-footer} & content & Footer of the pages (except the front
and back). The page counter will be displayed at the right of the
footer, except if the page number is 0. \\
\textbf{include-back-cover} & bool & whether to add the back cover or
not. \\
\textbf{insa} & str & INSA name ( \texttt{\ rennes\ } , \texttt{\ hdf\ }
…) \\
\textbf{lang} & str & Language of the template. Some strings are
translated. \\
\textbf{metadata-title} & content & Title of the document that will be
embedded in the PDF metadata. \\
\textbf{metadata-authors} & str list & Authors that will be embedded in
the PDF metadata. \\
\textbf{metadata-date} & datetime & Date that will be set as the
document creation date. If not specified, will be set to now. \\
\end{longtable}

\subsection{Fonts}\label{fonts}

The graphic charter recommends the fonts \textbf{League Spartan} for
headings and \textbf{Source Serif} for regular text. To have the best
look, you should install those fonts.

\begin{quote}
You can download the fonts from
\href{https://github.com/SkytAsul/INSA-Typst-Template/tree/main/fonts}{here}
.
\end{quote}

To behave correctly on computers lacking those specific fonts, this
template will automatically fallback to similar ones:

\begin{itemize}
\tightlist
\item
  \textbf{League Spartan} -\textgreater{} \textbf{Arial} (approved by
  INSA’s graphic charter, by default in Windows) -\textgreater{}
  \textbf{Liberation Sans} (by default in most Linux)
\item
  \textbf{Source Serif} -\textgreater{} \textbf{Source Serif 4}
  (downloadable for free) -\textgreater{} \textbf{Georgia} (approved by
  the graphic charter) -\textgreater{} \textbf{Linux Libertine} (default
  Typst font)
\end{itemize}

\subsubsection{Note on variable fonts}\label{note-on-variable-fonts}

If you want to install those fonts on your computer, Typst might not
recognize them if you install their \emph{Variable} versions. You should
install the static versions ( \textbf{League Spartan Bold} and most
versions of \textbf{Source Serif} ).

Keep an eye on \href{https://github.com/typst/typst/issues/185}{the
issue in Typst bug tracker} to see when variable fonts will be used!

\subsection{Notes}\label{notes}

This template is being developed by Youenn LE JEUNE from the INSA de
Rennes in \href{https://github.com/SkytAsul/INSA-Typst-Template}{this
repository} .

For now it includes assets from the graphic charters of those INSAs:

\begin{itemize}
\tightlist
\item
  Rennes ( \texttt{\ rennes\ } )
\item
  Hauts de France ( \texttt{\ hdf\ } )
\item
  Centre Val de Loire ( \texttt{\ cvl\ } ) Users from other INSAs can
  open a pull request on the repository with the assets for their INSA.
\end{itemize}

If you have any other feature request, open an issue on the repository.

\subsection{License}\label{license}

The typst template is licensed under the
\href{https://github.com/SkytAsul/INSA-Typst-Template/blob/main/LICENSE}{MIT
license} . This does \emph{not} apply to the image assets. Those image
files are property of Groupe INSA.

\subsection{Changelog}\label{changelog}

\subsubsection{0.4.0}\label{section}

\begin{itemize}
\tightlist
\item
  Added INSA CVL assets
\item
  Added \texttt{\ insa-tutor-suffix\ } option to \texttt{\ insa-stage\ }
\end{itemize}

\subsubsection{0.3.1}\label{section-1}

\begin{itemize}
\tightlist
\item
  Added \texttt{\ insa\ } option to all templates
\item
  Added INSA HdF assets
\item
  Added \texttt{\ student-suffix\ } option to \texttt{\ insa-stage\ }
\item
  Made outline not shown in outline
\end{itemize}

\subsubsection{0.3.0}\label{section-2}

\begin{itemize}
\tightlist
\item
  Added \texttt{\ omit-outline\ } option to \texttt{\ insa-stage\ }
\item
  Added \texttt{\ thanks-page\ } parameter to \texttt{\ insa-stage\ }
\item
  Added metadata-related options to \texttt{\ insa-document\ }
\item
  Made some PDF metadata automatically exported for
  \texttt{\ insa-stage\ } and \texttt{\ insa-report\ }
\item
  Made page number not displayed if equals to 0
\item
  Adjusted positions of elements in back covers
\item
  Fixed some translations
\item
  Updated README to have changelog, visual examples of all documents and
  parameters table
\end{itemize}

\href{/app?template=silky-report-insa&version=0.4.0}{Create project in
app}

\subsubsection{How to use}\label{how-to-use}

Click the button above to create a new project using this template in
the Typst app.

You can also use the Typst CLI to start a new project on your computer
using this command:

\begin{verbatim}
typst init @preview/silky-report-insa:0.4.0
\end{verbatim}

\includesvg[width=0.16667in,height=0.16667in]{/assets/icons/16-copy.svg}

\subsubsection{About}\label{about}

\begin{description}
\tightlist
\item[Author :]
SkytAsul
\item[License:]
MIT
\item[Current version:]
0.4.0
\item[Last updated:]
November 21, 2024
\item[First released:]
March 19, 2024
\item[Archive size:]
4.48 MB
\href{https://packages.typst.org/preview/silky-report-insa-0.4.0.tar.gz}{\pandocbounded{\includesvg[keepaspectratio]{/assets/icons/16-download.svg}}}
\item[Repository:]
\href{https://github.com/SkytAsul/INSA-Typst-Template}{GitHub}
\item[Discipline s :]
\begin{itemize}
\tightlist
\item[]
\item
  \href{https://typst.app/universe/search/?discipline=engineering}{Engineering}
\item
  \href{https://typst.app/universe/search/?discipline=computer-science}{Computer
  Science}
\item
  \href{https://typst.app/universe/search/?discipline=mathematics}{Mathematics}
\item
  \href{https://typst.app/universe/search/?discipline=physics}{Physics}
\item
  \href{https://typst.app/universe/search/?discipline=education}{Education}
\end{itemize}
\item[Categor y :]
\begin{itemize}
\tightlist
\item[]
\item
  \pandocbounded{\includesvg[keepaspectratio]{/assets/icons/16-speak.svg}}
  \href{https://typst.app/universe/search/?category=report}{Report}
\end{itemize}
\end{description}

\subsubsection{Where to report issues?}\label{where-to-report-issues}

This template is a project of SkytAsul . Report issues on
\href{https://github.com/SkytAsul/INSA-Typst-Template}{their repository}
. You can also try to ask for help with this template on the
\href{https://forum.typst.app}{Forum} .

Please report this template to the Typst team using the
\href{https://typst.app/contact}{contact form} if you believe it is a
safety hazard or infringes upon your rights.

\phantomsection\label{versions}
\subsubsection{Version history}\label{version-history}

\begin{longtable}[]{@{}ll@{}}
\toprule\noalign{}
Version & Release Date \\
\midrule\noalign{}
\endhead
\bottomrule\noalign{}
\endlastfoot
0.4.0 & November 21, 2024 \\
\href{https://typst.app/universe/package/silky-report-insa/0.3.1/}{0.3.1}
& September 24, 2024 \\
\href{https://typst.app/universe/package/silky-report-insa/0.3.0/}{0.3.0}
& August 7, 2024 \\
\href{https://typst.app/universe/package/silky-report-insa/0.2.1/}{0.2.1}
& July 24, 2024 \\
\href{https://typst.app/universe/package/silky-report-insa/0.2.0/}{0.2.0}
& June 10, 2024 \\
\href{https://typst.app/universe/package/silky-report-insa/0.1.0/}{0.1.0}
& March 19, 2024 \\
\end{longtable}

Typst GmbH did not create this template and cannot guarantee correct
functionality of this template or compatibility with any version of the
Typst compiler or app.


\section{Package List LaTeX/curryst.tex}
\title{typst.app/universe/package/curryst}

\phantomsection\label{banner}
\section{curryst}\label{curryst}

{ 0.3.0 }

Typeset trees of inference rules.

{ } Featured Package

\phantomsection\label{readme}
A Typst package for typesetting proof trees.

\subsection{Import}\label{import}

You can import the latest version of this package with:

\begin{Shaded}
\begin{Highlighting}[]
\NormalTok{\#import "@preview/curryst:0.3.0": rule, proof{-}tree}
\end{Highlighting}
\end{Shaded}

\subsection{Basic usage}\label{basic-usage}

To display a proof tree, you first need to create a tree, using the
\texttt{\ rule\ } function. Its first argument is the conclusion, and
the other positional arguments are the premises. It also accepts a
\texttt{\ name\ } for the rule name, displayed on the right of the bar,
as well as a \texttt{\ label\ } , displayed on the left of the bar.

\begin{Shaded}
\begin{Highlighting}[]
\NormalTok{\#let tree = rule(}
\NormalTok{  label: [Label],}
\NormalTok{  name: [Rule name],}
\NormalTok{  [Conclusion],}
\NormalTok{  [Premise 1],}
\NormalTok{  [Premise 2],}
\NormalTok{  [Premise 3]}
\NormalTok{)}
\end{Highlighting}
\end{Shaded}

Then, you can display the tree with the \texttt{\ proof-tree\ }
function:

\begin{Shaded}
\begin{Highlighting}[]
\NormalTok{\#proof{-}tree(tree)}
\end{Highlighting}
\end{Shaded}

In this case, we get the following result:

\pandocbounded{\includesvg[keepaspectratio]{https://github.com/typst/packages/raw/main/packages/preview/curryst/0.3.0/examples/usage.svg}}

Proof trees can be part of mathematical formulas:

\begin{Shaded}
\begin{Highlighting}[]
\NormalTok{Consider the following tree:}
\NormalTok{$}
\NormalTok{  Pi quad = quad \#proof{-}tree(}
\NormalTok{    rule(}
\NormalTok{      $phi$,}
\NormalTok{      $Pi\_1$,}
\NormalTok{      $Pi\_2$,}
\NormalTok{    )}
\NormalTok{  )}
\NormalTok{$}
\NormalTok{$Pi$ constitutes a derivation of $phi$.s}
\end{Highlighting}
\end{Shaded}

\pandocbounded{\includesvg[keepaspectratio]{https://github.com/typst/packages/raw/main/packages/preview/curryst/0.3.0/examples/math-formula.svg}}

You can specify a rule as the premises of a rule in order to create a
tree:

\begin{Shaded}
\begin{Highlighting}[]
\NormalTok{\#proof{-}tree(}
\NormalTok{  rule(}
\NormalTok{    name: $R$,}
\NormalTok{    $C\_1 or C\_2 or C\_3$,}
\NormalTok{    rule(}
\NormalTok{      name: $A$,}
\NormalTok{      $C\_1 or C\_2 or L$,}
\NormalTok{      rule(}
\NormalTok{        $C\_1 or L$,}
\NormalTok{        $Pi\_1$,}
\NormalTok{      ),}
\NormalTok{    ),}
\NormalTok{    rule(}
\NormalTok{      $C\_2 or overline(L)$,}
\NormalTok{      $Pi\_2$,}
\NormalTok{    ),}
\NormalTok{  )}
\NormalTok{)}
\end{Highlighting}
\end{Shaded}

\pandocbounded{\includesvg[keepaspectratio]{https://github.com/typst/packages/raw/main/packages/preview/curryst/0.3.0/examples/rule-as-premise.svg}}

As an example, here is a natural deduction proof tree generated with
Curryst:

\pandocbounded{\includesvg[keepaspectratio]{https://github.com/typst/packages/raw/main/packages/preview/curryst/0.3.0/examples/natural-deduction.svg}}

Show code

\begin{Shaded}
\begin{Highlighting}[]
\NormalTok{\#let ax = rule.with(name: [ax])}
\NormalTok{\#let and{-}el = rule.with(name: $and\_e\^{}ell$)}
\NormalTok{\#let and{-}er = rule.with(name: $and\_e\^{}r$)}
\NormalTok{\#let impl{-}i = rule.with(name: $scripts({-}\textgreater{})\_i$)}
\NormalTok{\#let impl{-}e = rule.with(name: $scripts({-}\textgreater{})\_e$)}
\NormalTok{\#let not{-}i = rule.with(name: $not\_i$)}
\NormalTok{\#let not{-}e = rule.with(name: $not\_e$)}

\NormalTok{\#proof{-}tree(}
\NormalTok{  impl{-}i(}
\NormalTok{    $tack (p {-}\textgreater{} q) {-}\textgreater{} not (p and not q)$,}
\NormalTok{    not{-}i(}
\NormalTok{      $p {-}\textgreater{} q tack  not (p and not q)$,}
\NormalTok{      not{-}e(}
\NormalTok{        $ underbrace(p {-}\textgreater{} q\textbackslash{}, p and not q, Gamma) tack bot $,}
\NormalTok{        impl{-}e(}
\NormalTok{          $Gamma tack q$,}
\NormalTok{          ax($Gamma tack p {-}\textgreater{} q$),}
\NormalTok{          and{-}el(}
\NormalTok{            $Gamma tack p$,}
\NormalTok{            ax($Gamma tack p and not q$),}
\NormalTok{          ),}
\NormalTok{        ),}
\NormalTok{        and{-}er(}
\NormalTok{          $Gamma tack not q$,}
\NormalTok{          ax($Gamma tack p and not q$),}
\NormalTok{        ),}
\NormalTok{      ),}
\NormalTok{    ),}
\NormalTok{  )}
\NormalTok{)}
\end{Highlighting}
\end{Shaded}

\subsection{Advanced usage}\label{advanced-usage}

The \texttt{\ proof-tree\ } function accepts multiple named arguments
that let you customize the tree:

\begin{description}
\tightlist
\item[\texttt{\ prem-min-spacing\ }]
The minimum amount of space between two premises.
\item[\texttt{\ title-inset\ }]
The amount width with which to extend the horizontal bar beyond the
content. Also determines how far from the bar labels and names are
displayed.
\item[\texttt{\ stroke\ }]
The stroke to use for the horizontal bars.
\item[\texttt{\ horizontal-spacing\ }]
The space between the bottom of the bar and the conclusion, and between
the top of the bar and the premises.
\item[\texttt{\ min-bar-height\ }]
The minimum height of the box containing the horizontal bar.
\end{description}

For more information, please refer to the documentation in
\href{https://github.com/typst/packages/raw/main/packages/preview/curryst/0.3.0/curryst.typ}{\texttt{\ curryst.typ\ }}
.

\subsubsection{How to add}\label{how-to-add}

Copy this into your project and use the import as \texttt{\ curryst\ }

\begin{verbatim}
#import "@preview/curryst:0.3.0"
\end{verbatim}

\includesvg[width=0.16667in,height=0.16667in]{/assets/icons/16-copy.svg}

Check the docs for
\href{https://typst.app/docs/reference/scripting/\#packages}{more
information on how to import packages} .

\subsubsection{About}\label{about}

\begin{description}
\tightlist
\item[Author s :]
\href{https://github.com/remih23}{Rémi Hutin} ,
\href{https://github.com/pauladam94}{Paul Adam} , \&
\href{https://github.com/MDLC01}{Malo}
\item[License:]
MIT
\item[Current version:]
0.3.0
\item[Last updated:]
April 16, 2024
\item[First released:]
December 7, 2023
\item[Minimum Typst version:]
0.11.0
\item[Archive size:]
4.71 kB
\href{https://packages.typst.org/preview/curryst-0.3.0.tar.gz}{\pandocbounded{\includesvg[keepaspectratio]{/assets/icons/16-download.svg}}}
\item[Repository:]
\href{https://github.com/pauladam94/curryst}{GitHub}
\item[Discipline s :]
\begin{itemize}
\tightlist
\item[]
\item
  \href{https://typst.app/universe/search/?discipline=computer-science}{Computer
  Science}
\item
  \href{https://typst.app/universe/search/?discipline=mathematics}{Mathematics}
\end{itemize}
\item[Categor ies :]
\begin{itemize}
\tightlist
\item[]
\item
  \pandocbounded{\includesvg[keepaspectratio]{/assets/icons/16-package.svg}}
  \href{https://typst.app/universe/search/?category=components}{Components}
\item
  \pandocbounded{\includesvg[keepaspectratio]{/assets/icons/16-chart.svg}}
  \href{https://typst.app/universe/search/?category=visualization}{Visualization}
\item
  \pandocbounded{\includesvg[keepaspectratio]{/assets/icons/16-integration.svg}}
  \href{https://typst.app/universe/search/?category=integration}{Integration}
\end{itemize}
\end{description}

\subsubsection{Where to report issues?}\label{where-to-report-issues}

This package is a project of Rémi Hutin, Paul Adam, and Malo . Report
issues on \href{https://github.com/pauladam94/curryst}{their repository}
. You can also try to ask for help with this package on the
\href{https://forum.typst.app}{Forum} .

Please report this package to the Typst team using the
\href{https://typst.app/contact}{contact form} if you believe it is a
safety hazard or infringes upon your rights.

\phantomsection\label{versions}
\subsubsection{Version history}\label{version-history}

\begin{longtable}[]{@{}ll@{}}
\toprule\noalign{}
Version & Release Date \\
\midrule\noalign{}
\endhead
\bottomrule\noalign{}
\endlastfoot
0.3.0 & April 16, 2024 \\
\href{https://typst.app/universe/package/curryst/0.2.0/}{0.2.0} & March
19, 2024 \\
\href{https://typst.app/universe/package/curryst/0.1.1/}{0.1.1} &
January 31, 2024 \\
\href{https://typst.app/universe/package/curryst/0.1.0/}{0.1.0} &
December 7, 2023 \\
\end{longtable}

Typst GmbH did not create this package and cannot guarantee correct
functionality of this package or compatibility with any version of the
Typst compiler or app.


\section{Package List LaTeX/touying.tex}
\title{typst.app/universe/package/touying}

\phantomsection\label{banner}
\section{touying}\label{touying}

{ 0.5.3 }

A powerful package for creating presentation slides in Typst.

{ } Featured Package

\phantomsection\label{readme}
\href{https://github.com/touying-typ/touying}{Touying} (投影 in
chinese, /tóuyÇ?ng/, meaning projection) is a user-friendly, powerful
and efficient package for creating presentation slides in Typst. Partial
code is inherited from
\href{https://github.com/andreasKroepelin/polylux}{Polylux} . Therefore,
some concepts and APIs remain consistent with Polylux.

Touying provides automatically injected global configurations, which is
convenient for configuring themes. Besides, Touying does not rely on
\texttt{\ counter\ } and \texttt{\ context\ } to implement
\texttt{\ \#pause\ } , resulting in better performance.

If you like it, consider
\href{https://github.com/touying-typ/touying}{giving a star on GitHub} .
Touying is a community-driven project, feel free to suggest any ideas
and contribute.

\href{https://touying-typ.github.io/}{\pandocbounded{\includegraphics[keepaspectratio]{https://img.shields.io/badge/docs-book-green}}}
\href{https://github.com/touying-typ/touying/wiki}{\pandocbounded{\includegraphics[keepaspectratio]{https://img.shields.io/badge/docs-gallery-orange}}}
\pandocbounded{\includegraphics[keepaspectratio]{https://img.shields.io/github/license/touying-typ/touying}}
\pandocbounded{\includegraphics[keepaspectratio]{https://img.shields.io/github/v/release/touying-typ/touying}}
\pandocbounded{\includegraphics[keepaspectratio]{https://img.shields.io/github/stars/touying-typ/touying}}
\pandocbounded{\includegraphics[keepaspectratio]{https://img.shields.io/badge/themes-6-aqua}}

\subsection{Document}\label{document}

Read \href{https://touying-typ.github.io/}{the document} to learn all
about Touying.

We will maintain \textbf{English} and \textbf{Chinese} versions of the
documentation for Touying, and for each major version, we will maintain
a documentation copy. This allows you to easily refer to old versions of
the Touying documentation and migrate to new versions.

\textbf{Note that the documentation may be outdated, and you can also
use Tinymist to view Touying’s annotated documentation by hovering
over the code.}

\subsection{Gallery}\label{gallery}

Touying offers \href{https://github.com/touying-typ/touying/wiki}{a
gallery page} via wiki, where you can browse elegant slides created by
Touying users. You’re also encouraged to contribute your own beautiful
slides here!

\subsection{Special Features}\label{special-features}

\begin{enumerate}
\tightlist
\item
  Split slides by headings
  \href{https://touying-typ.github.io/docs/sections}{document}
\end{enumerate}

\begin{Shaded}
\begin{Highlighting}[]
\NormalTok{= Section}

\NormalTok{== Subsection}

\NormalTok{=== First Slide}

\NormalTok{Hello, Touying!}

\NormalTok{=== Second Slide}

\NormalTok{Hello, Typst!}
\end{Highlighting}
\end{Shaded}

\begin{enumerate}
\setcounter{enumi}{1}
\tightlist
\item
  \texttt{\ \#pause\ } and \texttt{\ \#meanwhile\ } animations
  \href{https://touying-typ.github.io/docs/dynamic/simple}{document}
\end{enumerate}

\begin{Shaded}
\begin{Highlighting}[]
\NormalTok{\#slide[}
\NormalTok{  First}

\NormalTok{  \#pause}

\NormalTok{  Second}

\NormalTok{  \#meanwhile}

\NormalTok{  Third}

\NormalTok{  \#pause}

\NormalTok{  Fourth}
\NormalTok{]}
\end{Highlighting}
\end{Shaded}

\pandocbounded{\includegraphics[keepaspectratio]{https://github.com/touying-typ/touying/assets/34951714/24ca19a3-b27c-4d31-ab75-09c37911e6ac}}

\begin{enumerate}
\setcounter{enumi}{2}
\tightlist
\item
  Math Equation Animation
  \href{https://touying-typ.github.io/docs/dynamic/equation}{document}
\end{enumerate}

\pandocbounded{\includegraphics[keepaspectratio]{https://github.com/touying-typ/touying/assets/34951714/8640fe0a-95e4-46ac-b570-c8c79f993de4}}

\begin{enumerate}
\setcounter{enumi}{3}
\tightlist
\item
  \texttt{\ touying-reducer\ } Cetz and Fletcher Animations
  \href{https://touying-typ.github.io/docs/dynamic/other}{document}
\end{enumerate}

\pandocbounded{\includegraphics[keepaspectratio]{https://github.com/touying-typ/touying/assets/34951714/9ba71f54-2a5d-4144-996c-4a42833cc5cc}}

\begin{enumerate}
\setcounter{enumi}{4}
\tightlist
\item
  Correct outline and bookmark (no duplicate and correct page number)
\end{enumerate}

\pandocbounded{\includegraphics[keepaspectratio]{https://github.com/touying-typ/touying/assets/34951714/7b62fcaf-6342-4dba-901b-818c16682529}}

\begin{enumerate}
\setcounter{enumi}{5}
\tightlist
\item
  Dewdrop Theme Navigation Bar
  \href{https://touying-typ.github.io/docs/themes/dewdrop}{document}
\end{enumerate}

\pandocbounded{\includegraphics[keepaspectratio]{https://github.com/touying-typ/touying/assets/34951714/0426516d-aa3c-4b7a-b7b6-2d5d276fb971}}

\begin{enumerate}
\setcounter{enumi}{6}
\tightlist
\item
  Semi-transparent cover mode
  \href{https://touying-typ.github.io/docs/dynamic/cover}{document}
\end{enumerate}

\pandocbounded{\includegraphics[keepaspectratio]{https://github.com/touying-typ/touying/assets/34951714/22a9ea66-c8b5-431e-a52c-2c8ca3f18e49}}

\begin{enumerate}
\setcounter{enumi}{7}
\tightlist
\item
  Speaker notes for dual-screen
  \href{https://touying-typ.github.io/docs/external/pympress}{document}
\end{enumerate}

\pandocbounded{\includegraphics[keepaspectratio]{https://github.com/touying-typ/touying/assets/34951714/afbe17cb-46d4-4507-90e8-959c53de95d5}}

\begin{enumerate}
\setcounter{enumi}{8}
\tightlist
\item
  Export slides to PPTX and HTML formats and show presentation online.
  \href{https://github.com/touying-typ/touying-exporter}{touying-exporter}
  \href{https://github.com/touying-typ/touying-template}{touying-template}
  \href{https://touying-typ.github.io/touying-template/}{online}
\end{enumerate}

\pandocbounded{\includegraphics[keepaspectratio]{https://github.com/touying-typ/touying-exporter/assets/34951714/207ddffc-87c8-4976-9bf4-4c6c5e2573ea}}

\subsection{Quick start}\label{quick-start}

Before you begin, make sure you have installed the Typst environment. If
not, you can use the \href{https://typst.app/}{Web App} or the
\href{https://marketplace.visualstudio.com/items?itemName=myriad-dreamin.tinymist}{Tinymist
LSP} extensions for VS Code.

To use Touying, you only need to include the following code in your
document:

\begin{Shaded}
\begin{Highlighting}[]
\NormalTok{\#import "@preview/touying:0.5.3": *}
\NormalTok{\#import themes.simple: *}

\NormalTok{\#show: simple{-}theme.with(aspect{-}ratio: "16{-}9")}

\NormalTok{= Title}

\NormalTok{== First Slide}

\NormalTok{Hello, Touying!}

\NormalTok{\#pause}

\NormalTok{Hello, Typst!}
\end{Highlighting}
\end{Shaded}

\pandocbounded{\includegraphics[keepaspectratio]{https://github.com/touying-typ/touying/assets/34951714/f5bdbf8f-7bf9-45fd-9923-0fa5d66450b2}}

It’s simple. Congratulations on creating your first Touying slide!
🎉

\textbf{Tip:} You can use Typst syntax like
\texttt{\ \#import\ "config.typ":\ *\ } or
\texttt{\ \#include\ "content.typ"\ } to implement Touying’s
multi-file architecture.

\subsection{More Complex Examples}\label{more-complex-examples}

In fact, Touying provides various styles for writing slides. For
example, the above example uses first-level and second-level titles to
create new slides. However, you can also use the
\texttt{\ \#slide{[}..{]}\ } format to access more powerful features
provided by Touying.

\begin{Shaded}
\begin{Highlighting}[]
\NormalTok{\#import "@preview/touying:0.5.3": *}
\NormalTok{\#import themes.university: *}
\NormalTok{\#import "@preview/cetz:0.2.2"}
\NormalTok{\#import "@preview/fletcher:0.5.1" as fletcher: node, edge}
\NormalTok{\#import "@preview/ctheorems:1.1.2": *}
\NormalTok{\#import "@preview/numbly:0.1.0": numbly}

\NormalTok{// cetz and fletcher bindings for touying}
\NormalTok{\#let cetz{-}canvas = touying{-}reducer.with(reduce: cetz.canvas, cover: cetz.draw.hide.with(bounds: true))}
\NormalTok{\#let fletcher{-}diagram = touying{-}reducer.with(reduce: fletcher.diagram, cover: fletcher.hide)}

\NormalTok{// Theorems configuration by ctheorems}
\NormalTok{\#show: thmrules.with(qed{-}symbol: $square$)}
\NormalTok{\#let theorem = thmbox("theorem", "Theorem", fill: rgb("\#eeffee"))}
\NormalTok{\#let corollary = thmplain(}
\NormalTok{  "corollary",}
\NormalTok{  "Corollary",}
\NormalTok{  base: "theorem",}
\NormalTok{  titlefmt: strong}
\NormalTok{)}
\NormalTok{\#let definition = thmbox("definition", "Definition", inset: (x: 1.2em, top: 1em))}
\NormalTok{\#let example = thmplain("example", "Example").with(numbering: none)}
\NormalTok{\#let proof = thmproof("proof", "Proof")}

\NormalTok{\#show: university{-}theme.with(}
\NormalTok{  aspect{-}ratio: "16{-}9",}
\NormalTok{  // config{-}common(handout: true),}
\NormalTok{  config{-}info(}
\NormalTok{    title: [Title],}
\NormalTok{    subtitle: [Subtitle],}
\NormalTok{    author: [Authors],}
\NormalTok{    date: datetime.today(),}
\NormalTok{    institution: [Institution],}
\NormalTok{    logo: emoji.school,}
\NormalTok{  ),}
\NormalTok{)}

\NormalTok{\#set heading(numbering: numbly("\{1\}.", default: "1.1"))}

\NormalTok{\#title{-}slide()}

\NormalTok{== Outline \textless{}touying:hidden\textgreater{}}

\NormalTok{\#components.adaptive{-}columns(outline(title: none, indent: 1em))}

\NormalTok{= Animation}

\NormalTok{== Simple Animation}

\NormalTok{We can use \textasciigrave{}\#pause\textasciigrave{} to \#pause display something later.}

\NormalTok{\#pause}

\NormalTok{Just like this.}

\NormalTok{\#meanwhile}

\NormalTok{Meanwhile, \#pause we can also use \textasciigrave{}\#meanwhile\textasciigrave{} to \#pause display other content synchronously.}

\NormalTok{\#speaker{-}note[}
\NormalTok{  + This is a speaker note.}
\NormalTok{  + You won\textquotesingle{}t see it unless you use \textasciigrave{}config{-}common(show{-}notes{-}on{-}second{-}screen: right)\textasciigrave{}}
\NormalTok{]}


\NormalTok{== Complex Animation}

\NormalTok{At subslide \#touying{-}fn{-}wrapper((self: none) =\textgreater{} str(self.subslide)), we can}

\NormalTok{use \#uncover("2{-}")[\textasciigrave{}\#uncover\textasciigrave{} function] for reserving space,}

\NormalTok{use \#only("2{-}")[\textasciigrave{}\#only\textasciigrave{} function] for not reserving space,}

\NormalTok{\#alternatives[call \textasciigrave{}\#only\textasciigrave{} multiple times \textbackslash{}u\{2717\}][use \textasciigrave{}\#alternatives\textasciigrave{} function \#sym.checkmark] for choosing one of the alternatives.}


\NormalTok{== Callback Style Animation}

\NormalTok{\#slide(repeat: 3, self =\textgreater{} [}
\NormalTok{  \#let (uncover, only, alternatives) = utils.methods(self)}

\NormalTok{  At subslide \#self.subslide, we can}

\NormalTok{  use \#uncover("2{-}")[\textasciigrave{}\#uncover\textasciigrave{} function] for reserving space,}

\NormalTok{  use \#only("2{-}")[\textasciigrave{}\#only\textasciigrave{} function] for not reserving space,}

\NormalTok{  \#alternatives[call \textasciigrave{}\#only\textasciigrave{} multiple times \textbackslash{}u\{2717\}][use \textasciigrave{}\#alternatives\textasciigrave{} function \#sym.checkmark] for choosing one of the alternatives.}
\NormalTok{])}


\NormalTok{== Math Equation Animation}

\NormalTok{Equation with \textasciigrave{}pause\textasciigrave{}:}

\NormalTok{$}
\NormalTok{  f(x) \&= pause x\^{}2 + 2x + 1  \textbackslash{}}
\NormalTok{       \&= pause (x + 1)\^{}2  \textbackslash{}}
\NormalTok{$}

\NormalTok{\#meanwhile}

\NormalTok{Here, \#pause we have the expression of $f(x)$.}

\NormalTok{\#pause}

\NormalTok{By factorizing, we can obtain this result.}


\NormalTok{== CeTZ Animation}

\NormalTok{CeTZ Animation in Touying:}

\NormalTok{\#cetz{-}canvas(\{}
\NormalTok{  import cetz.draw: *}
  
\NormalTok{  rect((0,0), (5,5))}

\NormalTok{  (pause,)}

\NormalTok{  rect((0,0), (1,1))}
\NormalTok{  rect((1,1), (2,2))}
\NormalTok{  rect((2,2), (3,3))}

\NormalTok{  (pause,)}

\NormalTok{  line((0,0), (2.5, 2.5), name: "line")}
\NormalTok{\})}


\NormalTok{== Fletcher Animation}

\NormalTok{Fletcher Animation in Touying:}

\NormalTok{\#fletcher{-}diagram(}
\NormalTok{  node{-}stroke: .1em,}
\NormalTok{  node{-}fill: gradient.radial(blue.lighten(80\%), blue, center: (30\%, 20\%), radius: 80\%),}
\NormalTok{  spacing: 4em,}
\NormalTok{  edge(({-}1,0), "r", "{-}|\textgreater{}", \textasciigrave{}open(path)\textasciigrave{}, label{-}pos: 0, label{-}side: center),}
\NormalTok{  node((0,0), \textasciigrave{}reading\textasciigrave{}, radius: 2em),}
\NormalTok{  edge((0,0), (0,0), \textasciigrave{}read()\textasciigrave{}, "{-}{-}|\textgreater{}", bend: 130deg),}
\NormalTok{  pause,}
\NormalTok{  edge(\textasciigrave{}read()\textasciigrave{}, "{-}|\textgreater{}"),}
\NormalTok{  node((1,0), \textasciigrave{}eof\textasciigrave{}, radius: 2em),}
\NormalTok{  pause,}
\NormalTok{  edge(\textasciigrave{}close()\textasciigrave{}, "{-}|\textgreater{}"),}
\NormalTok{  node((2,0), \textasciigrave{}closed\textasciigrave{}, radius: 2em, extrude: ({-}2.5, 0)),}
\NormalTok{  edge((0,0), (2,0), \textasciigrave{}close()\textasciigrave{}, "{-}|\textgreater{}", bend: {-}40deg),}
\NormalTok{)}


\NormalTok{= Theorems}

\NormalTok{== Prime numbers}

\NormalTok{\#definition[}
\NormalTok{  A natural number is called a \#highlight[\_prime number\_] if it is greater}
\NormalTok{  than 1 and cannot be written as the product of two smaller natural numbers.}
\NormalTok{]}
\NormalTok{\#example[}
\NormalTok{  The numbers $2$, $3$, and $17$ are prime.}
\NormalTok{  @cor\_largest\_prime shows that this list is not exhaustive!}
\NormalTok{]}

\NormalTok{\#theorem("Euclid")[}
\NormalTok{  There are infinitely many primes.}
\NormalTok{]}
\NormalTok{\#proof[}
\NormalTok{  Suppose to the contrary that $p\_1, p\_2, dots, p\_n$ is a finite enumeration}
\NormalTok{  of all primes. Set $P = p\_1 p\_2 dots p\_n$. Since $P + 1$ is not in our list,}
\NormalTok{  it cannot be prime. Thus, some prime factor $p\_j$ divides $P + 1$.  Since}
\NormalTok{  $p\_j$ also divides $P$, it must divide the difference $(P + 1) {-} P = 1$, a}
\NormalTok{  contradiction.}
\NormalTok{]}

\NormalTok{\#corollary[}
\NormalTok{  There is no largest prime number.}
\NormalTok{] \textless{}cor\_largest\_prime\textgreater{}}
\NormalTok{\#corollary[}
\NormalTok{  There are infinitely many composite numbers.}
\NormalTok{]}

\NormalTok{\#theorem[}
\NormalTok{  There are arbitrarily long stretches of composite numbers.}
\NormalTok{]}

\NormalTok{\#proof[}
\NormalTok{  For any $n \textgreater{} 2$, consider $}
\NormalTok{    n! + 2, quad n! + 3, quad ..., quad n! + n \#qedhere}
\NormalTok{  $}
\NormalTok{]}


\NormalTok{= Others}

\NormalTok{== Side{-}by{-}side}

\NormalTok{\#slide(composer: (1fr, 1fr))[}
\NormalTok{  First column.}
\NormalTok{][}
\NormalTok{  Second column.}
\NormalTok{]}


\NormalTok{== Multiple Pages}

\NormalTok{\#lorem(200)}


\NormalTok{\#show: appendix}

\NormalTok{= Appendix}

\NormalTok{== Appendix}

\NormalTok{Please pay attention to the current slide number.}
\end{Highlighting}
\end{Shaded}

\pandocbounded{\includegraphics[keepaspectratio]{https://github.com/user-attachments/assets/3488f256-a0b3-43d0-a266-009d9d0a7bd3}}

\subsection{Acknowledgements}\label{acknowledgements}

Thanks to…

\begin{itemize}
\tightlist
\item
  \href{https://github.com/andreasKroepelin}{@andreasKroepelin} for the
  \texttt{\ polylux\ } package
\item
  \href{https://github.com/Enivex}{@Enivex} for the
  \texttt{\ metropolis\ } theme
\item
  \href{https://github.com/drupol}{@drupol} for the
  \texttt{\ university\ } theme
\item
  \href{https://github.com/pride7}{@pride7} for the \texttt{\ aqua\ }
  theme
\item
  \href{https://github.com/Coekjan}{@Coekjan} and
  \href{https://github.com/QuadnucYard}{@QuadnucYard} for the
  \texttt{\ stargazer\ } theme
\item
  \href{https://github.com/ntjess}{@ntjess} for contributing to
  \texttt{\ fit-to-height\ } , \texttt{\ fit-to-width\ } and
  \texttt{\ cover-with-rect\ }
\end{itemize}

\subsection{Poster}\label{poster}

\pandocbounded{\includegraphics[keepaspectratio]{https://github.com/user-attachments/assets/e1ddb672-8e8f-472d-b364-b8caed1da16b}}

\href{https://github.com/touying-typ/touying-poster}{View Code}

\subsection{Star History}\label{star-history}

\href{https://star-history.com/\#touying-typ/touying&Date}{\pandocbounded{\includegraphics[keepaspectratio]{https://api.star-history.com/svg?repos=touying-typ/touying&type=Date}}}

\subsubsection{How to add}\label{how-to-add}

Copy this into your project and use the import as \texttt{\ touying\ }

\begin{verbatim}
#import "@preview/touying:0.5.3"
\end{verbatim}

\includesvg[width=0.16667in,height=0.16667in]{/assets/icons/16-copy.svg}

Check the docs for
\href{https://typst.app/docs/reference/scripting/\#packages}{more
information on how to import packages} .

\subsubsection{About}\label{about}

\begin{description}
\tightlist
\item[Author s :]
OrangeX4 , Andreas Kröpelin , ntjess , Enivex , Pol Dellaiera , pride7
, \& Coekjan
\item[License:]
MIT
\item[Current version:]
0.5.3
\item[Last updated:]
October 15, 2024
\item[First released:]
January 11, 2024
\item[Minimum Typst version:]
0.11.0
\item[Archive size:]
302 kB
\href{https://packages.typst.org/preview/touying-0.5.3.tar.gz}{\pandocbounded{\includesvg[keepaspectratio]{/assets/icons/16-download.svg}}}
\item[Repository:]
\href{https://github.com/touying-typ/touying}{GitHub}
\item[Categor y :]
\begin{itemize}
\tightlist
\item[]
\item
  \pandocbounded{\includesvg[keepaspectratio]{/assets/icons/16-presentation.svg}}
  \href{https://typst.app/universe/search/?category=presentation}{Presentation}
\end{itemize}
\end{description}

\subsubsection{Where to report issues?}\label{where-to-report-issues}

This package is a project of OrangeX4, Andreas Kröpelin, ntjess,
Enivex, Pol Dellaiera, pride7, and Coekjan . Report issues on
\href{https://github.com/touying-typ/touying}{their repository} . You
can also try to ask for help with this package on the
\href{https://forum.typst.app}{Forum} .

Please report this package to the Typst team using the
\href{https://typst.app/contact}{contact form} if you believe it is a
safety hazard or infringes upon your rights.

\phantomsection\label{versions}
\subsubsection{Version history}\label{version-history}

\begin{longtable}[]{@{}ll@{}}
\toprule\noalign{}
Version & Release Date \\
\midrule\noalign{}
\endhead
\bottomrule\noalign{}
\endlastfoot
0.5.3 & October 15, 2024 \\
\href{https://typst.app/universe/package/touying/0.5.2/}{0.5.2} &
September 3, 2024 \\
\href{https://typst.app/universe/package/touying/0.5.1/}{0.5.1} &
September 3, 2024 \\
\href{https://typst.app/universe/package/touying/0.5.0/}{0.5.0} &
September 2, 2024 \\
\href{https://typst.app/universe/package/touying/0.4.2/}{0.4.2} & May
27, 2024 \\
\href{https://typst.app/universe/package/touying/0.4.1/}{0.4.1} & May
13, 2024 \\
\href{https://typst.app/universe/package/touying/0.4.0/}{0.4.0} & April
6, 2024 \\
\href{https://typst.app/universe/package/touying/0.3.3/}{0.3.3} & March
26, 2024 \\
\href{https://typst.app/universe/package/touying/0.3.2/}{0.3.2} & March
15, 2024 \\
\href{https://typst.app/universe/package/touying/0.3.1/}{0.3.1} & March
7, 2024 \\
\href{https://typst.app/universe/package/touying/0.3.0/}{0.3.0} & March
6, 2024 \\
\href{https://typst.app/universe/package/touying/0.2.1/}{0.2.1} &
February 17, 2024 \\
\href{https://typst.app/universe/package/touying/0.2.0/}{0.2.0} &
January 20, 2024 \\
\href{https://typst.app/universe/package/touying/0.1.0/}{0.1.0} &
January 11, 2024 \\
\end{longtable}

Typst GmbH did not create this package and cannot guarantee correct
functionality of this package or compatibility with any version of the
Typst compiler or app.


\section{Package List LaTeX/pinit.tex}
\title{typst.app/universe/package/pinit}

\phantomsection\label{banner}
\section{pinit}\label{pinit}

{ 0.2.2 }

Relative positioning by pins, especially useful for making slides in
typst.

{ } Featured Package

\phantomsection\label{readme}
Relative positioning by pins, especially useful for making slides in
typst.

\subsection{Example}\label{example}

\subsubsection{Pin things as you like}\label{pin-things-as-you-like}

Have a look at the source
\href{https://github.com/typst/packages/raw/main/packages/preview/pinit/0.2.2/examples/example.typ}{here}
.

\pandocbounded{\includegraphics[keepaspectratio]{https://github.com/typst/packages/raw/main/packages/preview/pinit/0.2.2/examples/example.png}}

\subsubsection{Dynamic Slides}\label{dynamic-slides}

Pinit works with \href{https://github.com/touying-typ/touying}{Touying}
or \href{https://github.com/andreasKroepelin/polylux}{Polylux}
animations.

Have a look at the pdf file
\href{https://github.com/OrangeX4/typst-pinit/blob/main/examples/example.pdf}{here}
.

\pandocbounded{\includegraphics[keepaspectratio]{https://github.com/typst/packages/raw/main/packages/preview/pinit/0.2.2/examples/example-pages.png}}

\subsection{Usage}\label{usage}

\subsubsection{Examples}\label{examples}

The idea of pinit is pinning pins on the normal flow of the text, and
then placing the content on the page by \texttt{\ absolute-place\ }
function.

For example, we can highlight text and add a tip by pins simply:

\begin{Shaded}
\begin{Highlighting}[]
\NormalTok{\#import "@preview/pinit:0.2.2": *}

\NormalTok{\#set text(size: 24pt)}

\NormalTok{A simple \#pin(1)highlighted text\#pin(2).}

\NormalTok{\#pinit{-}highlight(1, 2)}

\NormalTok{\#pinit{-}point{-}from(2)[It is simple.]}
\end{Highlighting}
\end{Shaded}

\pandocbounded{\includegraphics[keepaspectratio]{https://github.com/typst/packages/raw/main/packages/preview/pinit/0.2.2/examples/simple-demo.png}}

If you want to place the content relative to the center of some pins,
you use a array of pins:

\begin{Shaded}
\begin{Highlighting}[]
\NormalTok{\#import "@preview/pinit:0.2.2": *}

\NormalTok{\#set text(size: 12pt)}

\NormalTok{A simple \#pin(1)highlighted text\#pin(2).}

\NormalTok{\#pinit{-}highlight(1, 2)}

\NormalTok{\#pinit{-}point{-}from((1, 2))[It is simple.]}
\end{Highlighting}
\end{Shaded}

\pandocbounded{\includegraphics[keepaspectratio]{https://github.com/typst/packages/raw/main/packages/preview/pinit/0.2.2/examples/simple-demo2.png}}

A more complex example, Have a look at the source
\href{https://github.com/typst/packages/raw/main/packages/preview/pinit/0.2.2/examples/equation-desc.typ}{here}
.

\pandocbounded{\includegraphics[keepaspectratio]{https://github.com/typst/packages/raw/main/packages/preview/pinit/0.2.2/examples/equation-desc.png}}

\subsubsection{Fletcher edge support}\label{fletcher-edge-support}

\href{https://github.com/Jollywatt/typst-fletcher}{Fletcher} is a
powerful Typst package for drawing diagrams with arrows. We can use
fletcher to draw more complex arrows.

\href{https://github.com/typst/packages/raw/main/packages/preview/pinit/0.2.2/\#pinit-fletcher-edge}{\texttt{\ pinit-fletcher-edge\ }}

\begin{Shaded}
\begin{Highlighting}[]
\NormalTok{\#import "@preview/pinit:0.2.2": *}
\NormalTok{\#import "@preview/fletcher:0.5.1"}

\NormalTok{Con\#pin(1)\#h(4em)\#pin(2)nect}

\NormalTok{\#pinit{-}fletcher{-}edge(}
\NormalTok{  fletcher, 1, end: 2, (1, 0), [bend], bend: {-}20deg, "\textless{}{-}\textgreater{}",}
\NormalTok{  decorations: fletcher.cetz.decorations.wave.with(amplitude: .1),}
\NormalTok{)}
\end{Highlighting}
\end{Shaded}

\pandocbounded{\includegraphics[keepaspectratio]{https://github.com/typst/packages/raw/main/packages/preview/pinit/0.2.2/examples/fletcher.png}}

\subsubsection{Pinit for raw}\label{pinit-for-raw}

In the code block, we need to use a regex trick to get pinit to work,
for example

\begin{Shaded}
\begin{Highlighting}[]
\NormalTok{\#show raw: it =\textgreater{} \{}
\NormalTok{  show regex("pin\textbackslash{}d"): it =\textgreater{} pin(eval(it.text.slice(3)))}
\NormalTok{  it}
\NormalTok{\}}

\NormalTok{\textasciigrave{}print(pin1"hello, world"pin2)\textasciigrave{}}

\NormalTok{\#pinit{-}highlight(1, 2)}
\end{Highlighting}
\end{Shaded}

\pandocbounded{\includegraphics[keepaspectratio]{https://github.com/typst/packages/raw/main/packages/preview/pinit/0.2.2/examples/pinit-for-raw.png}}

Note that typst’s code highlighting breaks up the text, causing overly
complex regular expressions such as ‘\#pin(.*?)’ to not work
properly.

However, you may want to consider putting it in a comment to avoid
highlighting the text and breaking it up.

\subsection{Notice}\label{notice}

\textbf{Since Typst does not provide a reliable
\texttt{\ absolute-place\ } function, you may consider taking the
following steps if a MISALIGNMENT occurs:}

\begin{enumerate}
\tightlist
\item
  \textbf{You could try to add a \texttt{\ \#box()\ } after the
  \texttt{\ \#pinit-xxx\ } function call, like
  \texttt{\ \#pinit-xxx()\#box()\ } .}
\item
  \textbf{You should add a blank line before the
  \texttt{\ \#pinit-xxx\ } function call, otherwise it will cause
  misalignment.}
\item
  \textbf{You can try moving \texttt{\ \#pinit-xxx()\ } in front of or
  behind \texttt{\ \#pin()\ } , or otherwhere, in short, try more.}
\item
  \textbf{Try to add a offset to the \texttt{\ dx\ } or \texttt{\ dy\ }
  argument of \texttt{\ \#pinit-xxx\ } function by yourself.}
\item
  \textbf{Open an issue if you have any questions you can’t solve.}
\end{enumerate}

\subsection{Outline}\label{outline}

\begin{itemize}
\tightlist
\item
  \href{https://github.com/typst/packages/raw/main/packages/preview/pinit/0.2.2/\#pinit}{Pinit}

  \begin{itemize}
  \tightlist
  \item
    \href{https://github.com/typst/packages/raw/main/packages/preview/pinit/0.2.2/\#example}{Example}

    \begin{itemize}
    \tightlist
    \item
      \href{https://github.com/typst/packages/raw/main/packages/preview/pinit/0.2.2/\#pin-things-as-you-like}{Pin
      things as you like}
    \item
      \href{https://github.com/typst/packages/raw/main/packages/preview/pinit/0.2.2/\#dynamic-slides}{Dynamic
      Slides}
    \end{itemize}
  \item
    \href{https://github.com/typst/packages/raw/main/packages/preview/pinit/0.2.2/\#usage}{Usage}

    \begin{itemize}
    \tightlist
    \item
      \href{https://github.com/typst/packages/raw/main/packages/preview/pinit/0.2.2/\#examples}{Examples}
    \item
      \href{https://github.com/typst/packages/raw/main/packages/preview/pinit/0.2.2/\#fletcher-edge-support}{Fletcher
      edge support}
    \item
      \href{https://github.com/typst/packages/raw/main/packages/preview/pinit/0.2.2/\#pinit-for-raw}{Pinit
      for raw}
    \end{itemize}
  \item
    \href{https://github.com/typst/packages/raw/main/packages/preview/pinit/0.2.2/\#notice}{Notice}
  \item
    \href{https://github.com/typst/packages/raw/main/packages/preview/pinit/0.2.2/\#outline}{Outline}
  \item
    \href{https://github.com/typst/packages/raw/main/packages/preview/pinit/0.2.2/\#reference}{Reference}

    \begin{itemize}
    \tightlist
    \item
      \href{https://github.com/typst/packages/raw/main/packages/preview/pinit/0.2.2/\#pin}{\texttt{\ pin\ }}
    \item
      \href{https://github.com/typst/packages/raw/main/packages/preview/pinit/0.2.2/\#pinit-1}{\texttt{\ pinit\ }}
    \item
      \href{https://github.com/typst/packages/raw/main/packages/preview/pinit/0.2.2/\#absolute-place}{\texttt{\ absolute-place\ }}
    \item
      \href{https://github.com/typst/packages/raw/main/packages/preview/pinit/0.2.2/\#pinit-place}{\texttt{\ pinit-place\ }}
    \item
      \href{https://github.com/typst/packages/raw/main/packages/preview/pinit/0.2.2/\#pinit-rect}{\texttt{\ pinit-rect\ }}
    \item
      \href{https://github.com/typst/packages/raw/main/packages/preview/pinit/0.2.2/\#pinit-highlight}{\texttt{\ pinit-highlight\ }}
    \item
      \href{https://github.com/typst/packages/raw/main/packages/preview/pinit/0.2.2/\#pinit-line}{\texttt{\ pinit-line\ }}
    \item
      \href{https://github.com/typst/packages/raw/main/packages/preview/pinit/0.2.2/\#pinit-line-to}{\texttt{\ pinit-line-to\ }}
    \item
      \href{https://github.com/typst/packages/raw/main/packages/preview/pinit/0.2.2/\#pinit-arrow}{\texttt{\ pinit-arrow\ }}
    \item
      \href{https://github.com/typst/packages/raw/main/packages/preview/pinit/0.2.2/\#pinit-double-arrow}{\texttt{\ pinit-double-arrow\ }}
    \item
      \href{https://github.com/typst/packages/raw/main/packages/preview/pinit/0.2.2/\#pinit-point-to}{\texttt{\ pinit-point-to\ }}
    \item
      \href{https://github.com/typst/packages/raw/main/packages/preview/pinit/0.2.2/\#pinit-point-from}{\texttt{\ pinit-point-from\ }}
    \item
      \href{https://github.com/typst/packages/raw/main/packages/preview/pinit/0.2.2/\#simple-arrow}{\texttt{\ simple-arrow\ }}
    \item
      \href{https://github.com/typst/packages/raw/main/packages/preview/pinit/0.2.2/\#double-arrow}{\texttt{\ double-arrow\ }}
    \item
      \href{https://github.com/typst/packages/raw/main/packages/preview/pinit/0.2.2/\#pinit-fletcher-edge}{\texttt{\ pinit-fletcher-edge\ }}
    \end{itemize}
  \item
    \href{https://github.com/typst/packages/raw/main/packages/preview/pinit/0.2.2/\#changelog}{Changelog}

    \begin{itemize}
    \tightlist
    \item
      \href{https://github.com/typst/packages/raw/main/packages/preview/pinit/0.2.2/\#022}{0.2.2}
    \item
      \href{https://github.com/typst/packages/raw/main/packages/preview/pinit/0.2.2/\#021}{0.2.1}
    \item
      \href{https://github.com/typst/packages/raw/main/packages/preview/pinit/0.2.2/\#020}{0.2.0}
    \item
      \href{https://github.com/typst/packages/raw/main/packages/preview/pinit/0.2.2/\#014}{0.1.4}
    \item
      \href{https://github.com/typst/packages/raw/main/packages/preview/pinit/0.2.2/\#013}{0.1.3}
    \item
      \href{https://github.com/typst/packages/raw/main/packages/preview/pinit/0.2.2/\#012}{0.1.2}
    \item
      \href{https://github.com/typst/packages/raw/main/packages/preview/pinit/0.2.2/\#011}{0.1.1}
    \item
      \href{https://github.com/typst/packages/raw/main/packages/preview/pinit/0.2.2/\#010}{0.1.0}
    \end{itemize}
  \item
    \href{https://github.com/typst/packages/raw/main/packages/preview/pinit/0.2.2/\#acknowledgements}{Acknowledgements}
  \item
    \href{https://github.com/typst/packages/raw/main/packages/preview/pinit/0.2.2/\#license}{License}
  \end{itemize}
\end{itemize}

\subsection{Reference}\label{reference}

\subsubsection{\texorpdfstring{\texttt{\ pin\ }}{ pin }}\label{pin}

Pinning a pin in text, the pin is supposed to be unique in one page.

\begin{Shaded}
\begin{Highlighting}[]
\NormalTok{\#let pin(name) = \{ .. \}}
\end{Highlighting}
\end{Shaded}

\textbf{Arguments:}

\begin{itemize}
\tightlist
\item
  \texttt{\ name\ } : {[} \texttt{\ int\ } or \texttt{\ str\ } or
  \texttt{\ any\ } {]} â€'' Name of pin, which can be any types with
  unique \texttt{\ repr()\ } return value, such as integer and string.
\end{itemize}

\subsubsection{\texorpdfstring{\texttt{\ pinit\ }}{ pinit }}\label{pinit-1}

Query positions of pins in the same page, then call the callback
function \texttt{\ callback\ } .

\begin{Shaded}
\begin{Highlighting}[]
\NormalTok{\#let pinit(callback: none, ..pins) = \{ .. \}}
\end{Highlighting}
\end{Shaded}

\textbf{Arguments:}

\begin{itemize}
\tightlist
\item
  \texttt{\ ..pins\ } : {[} \texttt{\ pin\ } {]} â€'' Names of pins you
  want to query. It is supposed to be arguments of pin or a group of
  pins.
\item
  \texttt{\ callback\ } : {[}
  \texttt{\ (..positions)\ =\textgreater{}\ \{\ ..\ \}\ } {]} â€'' A
  callback function accepting an array of positions (or a single
  position) as a parameter. Each position is a dictionary like
  \texttt{\ (page:\ 1,\ x:\ 319.97pt,\ y:\ 86.66pt)\ } . You can use the
  \texttt{\ absolute-place\ } function in this callback function to
  display something around the pins.
\end{itemize}

\subsubsection{\texorpdfstring{\texttt{\ absolute-place\ }}{ absolute-place }}\label{absolute-place}

Place content at a specific location on the page relative to the top
left corner of the page, regardless of margins, current containers, etc.

\begin{quote}
This function comes from
\href{https://github.com/ntjess/typst-drafting}{typst-drafting} .
\end{quote}

\begin{Shaded}
\begin{Highlighting}[]
\NormalTok{\#let absolute{-}place(}
\NormalTok{  dx: 0em,}
\NormalTok{  dy: 0em,}
\NormalTok{  body,}
\NormalTok{) = \{ .. \}}
\end{Highlighting}
\end{Shaded}

\textbf{Arguments:}

\begin{itemize}
\tightlist
\item
  \texttt{\ dx\ } : {[} \texttt{\ length\ } {]} â€'' Length in the
  x-axis relative to the left edge of the page.
\item
  \texttt{\ dy\ } : {[} \texttt{\ length\ } {]} â€'' Length in the
  y-axis relative to the top edge of the page.
\item
  \texttt{\ content\ } : {[} \texttt{\ content\ } {]} â€'' The content
  you want to place.
\end{itemize}

\subsubsection{\texorpdfstring{\texttt{\ pinit-place\ }}{ pinit-place }}\label{pinit-place}

Place content at a specific location on the page relative to the pin.

\begin{Shaded}
\begin{Highlighting}[]
\NormalTok{\#let pinit{-}place(}
\NormalTok{  dx: 0pt,}
\NormalTok{  dy: 0pt,}
\NormalTok{  pin{-}name,}
\NormalTok{  body,}
\NormalTok{) = \{ .. \}}
\end{Highlighting}
\end{Shaded}

\textbf{Arguments:}

\begin{itemize}
\tightlist
\item
  \texttt{\ dx\ } : {[} \texttt{\ length\ } {]} â€'' Offset X relative
  to the pin.
\item
  \texttt{\ dy\ } : {[} \texttt{\ length\ } {]} â€'' Offset Y relative
  to the pin.
\item
  \texttt{\ pin-name\ } : {[} \texttt{\ pin\ } {]} â€'' Name of the pin
  to which you want to locate.
\item
  \texttt{\ body\ } : {[} \texttt{\ content\ } {]} â€'' The content you
  want to place.
\end{itemize}

\subsubsection{\texorpdfstring{\texttt{\ pinit-rect\ }}{ pinit-rect }}\label{pinit-rect}

Draw a rectangular shape on the page \textbf{containing all pins} with
optional extended width and height.

\begin{Shaded}
\begin{Highlighting}[]
\NormalTok{\#let pinit{-}rect(}
\NormalTok{  dx: 0em,}
\NormalTok{  dy: {-}1em,}
\NormalTok{  extended{-}width: 0em,}
\NormalTok{  extended{-}height: 1.4em,}
\NormalTok{  pin1,}
\NormalTok{  pin2,}
\NormalTok{  pin3,  // Optional}
\NormalTok{  ..pinX,}
\NormalTok{  ..args,}
\NormalTok{) = \{ .. \}}
\end{Highlighting}
\end{Shaded}

\textbf{Arguments:}

\begin{itemize}
\tightlist
\item
  \texttt{\ dx\ } : {[} \texttt{\ length\ } {]} â€'' Offset X relative
  to the min-left of pins.
\item
  \texttt{\ dy\ } : {[} \texttt{\ length\ } {]} â€'' Offset Y relative
  to the min-top of pins.
\item
  \texttt{\ extended-width\ } : {[} \texttt{\ length\ } {]} â€''
  Optional extended width of the rectangular shape.
\item
  \texttt{\ extended-height\ } : {[} \texttt{\ length\ } {]} â€''
  Optional extended height of the rectangular shape.
\item
  \texttt{\ pin1\ } : {[} \texttt{\ pin\ } {]} â€'' One of these pins.
\item
  \texttt{\ pin2\ } : {[} \texttt{\ pin\ } {]} â€'' One of these pins.
\item
  \texttt{\ pin3\ } : {[} \texttt{\ pin\ } {]} â€'' One of these pins,
  optionally.
\item
  \texttt{\ ...args\ } : Additional named arguments or settings for
  \href{https://typst.app/docs/reference/visualize/rect/}{\texttt{\ rect\ }}
  , like \texttt{\ fill\ } , \texttt{\ stroke\ } and \texttt{\ radius\ }
  .
\end{itemize}

\subsubsection{\texorpdfstring{\texttt{\ pinit-highlight\ }}{ pinit-highlight }}\label{pinit-highlight}

Highlight a specific area on the page with a filled color and optional
radius and stroke. It is just a simply styled \texttt{\ pinit-rect\ } .

\begin{Shaded}
\begin{Highlighting}[]
\NormalTok{\#let pinit{-}highlight(}
\NormalTok{  fill: rgb(255, 0, 0, 20),}
\NormalTok{  radius: 5pt,}
\NormalTok{  stroke: 0pt,}
\NormalTok{  dx: 0em,}
\NormalTok{  dy: {-}1em,}
\NormalTok{  extended{-}width: 0em,}
\NormalTok{  extended{-}height: 1.4em,}
\NormalTok{  pin1,}
\NormalTok{  pin2,}
\NormalTok{  pin3,  // Optional}
\NormalTok{  ..pinX,}
\NormalTok{  ...args,}
\NormalTok{) = \{ .. \}}
\end{Highlighting}
\end{Shaded}

\textbf{Arguments:}

\begin{itemize}
\tightlist
\item
  \texttt{\ fill\ } : {[} \texttt{\ color\ } {]} â€'' The fill color for
  the highlighted area.
\item
  \texttt{\ radius\ } : {[} \texttt{\ length\ } {]} â€'' Optional radius
  for the highlight.
\item
  \texttt{\ stroke\ } : {[} \texttt{\ stroke\ } {]} â€'' Optional stroke
  width for the highlight.
\item
  \texttt{\ dx\ } : {[} \texttt{\ length\ } {]} â€'' Offset X relative
  to the min-left of pins.
\item
  \texttt{\ dy\ } : {[} \texttt{\ length\ } {]} â€'' Offset Y relative
  to the min-top of pins.
\item
  \texttt{\ extended-width\ } : {[} \texttt{\ length\ } {]} â€''
  Optional extended width of the rectangular shape.
\item
  \texttt{\ extended-height\ } : {[} \texttt{\ length\ } {]} â€''
  Optional extended height of the rectangular shape.
\item
  \texttt{\ pin1\ } : {[} \texttt{\ pin\ } {]} â€'' One of these pins.
\item
  \texttt{\ pin2\ } : {[} \texttt{\ pin\ } {]} â€'' One of these pins.
\item
  \texttt{\ pin3\ } : {[} \texttt{\ pin\ } {]} â€'' One of these pins,
  optionally.
\item
  \texttt{\ ...args\ } : Additional arguments or settings for
  \href{https://github.com/typst/packages/raw/main/packages/preview/pinit/0.2.2/\#pinit-rect}{\texttt{\ pinit-rect\ }}
  .
\end{itemize}

\subsubsection{\texorpdfstring{\texttt{\ pinit-line\ }}{ pinit-line }}\label{pinit-line}

Draw a line on the page between two specified pins with an optional
stroke.

\begin{Shaded}
\begin{Highlighting}[]
\NormalTok{\#let pinit{-}line(}
\NormalTok{  stroke: 1pt,}
\NormalTok{  start{-}dx: 0pt,}
\NormalTok{  start{-}dy: 0pt,}
\NormalTok{  end{-}dx: 0pt,}
\NormalTok{  end{-}dy: 0pt,}
\NormalTok{  start,}
\NormalTok{  end,}
\NormalTok{) = \{ ... \}}
\end{Highlighting}
\end{Shaded}

\textbf{Arguments:}

\begin{itemize}
\tightlist
\item
  \texttt{\ stroke\ } : {[} \texttt{\ stroke\ } {]} â€'' The stroke for
  the line.
\item
  \texttt{\ start-dx\ } : {[} \texttt{\ length\ } {]} â€'' Offset X
  relative to the start pin.
\item
  \texttt{\ start-dy\ } : {[} \texttt{\ length\ } {]} â€'' Offset Y
  relative to the start pin.
\item
  \texttt{\ end-dx\ } : {[} \texttt{\ length\ } {]} â€'' Offset X
  relative to the end pin.
\item
  \texttt{\ end-dy\ } : {[} \texttt{\ length\ } {]} â€'' Offset Y
  relative to the end pin.
\item
  \texttt{\ start\ } : {[} \texttt{\ pin\ } {]} â€'' The start pin.
\item
  \texttt{\ end\ } : {[} \texttt{\ pin\ } {]} â€'' The end pin.
\end{itemize}

\subsubsection{\texorpdfstring{\texttt{\ pinit-line-to\ }}{ pinit-line-to }}\label{pinit-line-to}

Draw an line from a specified pin to a point on the page with optional
settings.

\begin{Shaded}
\begin{Highlighting}[]
\NormalTok{\#let pinit{-}line{-}to(}
\NormalTok{  stroke: 1pt,}
\NormalTok{  pin{-}dx: 5pt,}
\NormalTok{  pin{-}dy: 5pt,}
\NormalTok{  body{-}dx: 5pt,}
\NormalTok{  body{-}dy: 5pt,}
\NormalTok{  offset{-}dx: 35pt,}
\NormalTok{  offset{-}dy: 35pt,}
\NormalTok{  pin{-}name,}
\NormalTok{  body,}
\NormalTok{) = \{ ... \}}
\end{Highlighting}
\end{Shaded}

\textbf{Arguments:}

\begin{itemize}
\tightlist
\item
  \texttt{\ stroke\ } : {[} \texttt{\ stroke\ } {]} â€'' The stroke for
  the line.
\item
  \texttt{\ pin-dx\ } : {[} \texttt{\ length\ } {]} â€'' Offset X of
  arrow start relative to the pin.
\item
  \texttt{\ pin-dy\ } : {[} \texttt{\ length\ } {]} â€'' Offset Y of
  arrow start relative to the pin.
\item
  \texttt{\ body-dx\ } : {[} \texttt{\ length\ } {]} â€'' Offset X of
  arrow end relative to the body.
\item
  \texttt{\ body-dy\ } : {[} \texttt{\ length\ } {]} â€'' Offset Y of
  arrow end relative to the body.
\item
  \texttt{\ offset-dx\ } : {[} \texttt{\ length\ } {]} â€'' Offset X
  relative to the pin.
\item
  \texttt{\ offset-dy\ } : {[} \texttt{\ length\ } {]} â€'' Offset Y
  relative to the pin.
\item
  \texttt{\ pin-name\ } : {[} \texttt{\ pin\ } {]} â€'' The name of the
  pin to start from.
\item
  \texttt{\ body\ } : {[} \texttt{\ content\ } {]} â€'' The content to
  draw the arrow to.
\end{itemize}

\subsubsection{\texorpdfstring{\texttt{\ pinit-arrow\ }}{ pinit-arrow }}\label{pinit-arrow}

Draw an arrow between two specified pins with optional settings.

\begin{Shaded}
\begin{Highlighting}[]
\NormalTok{\#let pinit{-}arrow(}
\NormalTok{  start{-}dx: 0pt,}
\NormalTok{  start{-}dy: 0pt,}
\NormalTok{  end{-}dx: 0pt,}
\NormalTok{  end{-}dy: 0pt,}
\NormalTok{  start,}
\NormalTok{  end,}
\NormalTok{  ..args,}
\NormalTok{) = \{ ... \}}
\end{Highlighting}
\end{Shaded}

\textbf{Arguments:}

\begin{itemize}
\tightlist
\item
  \texttt{\ start-dx\ } : {[} \texttt{\ length\ } {]} â€'' Offset X
  relative to the start pin.
\item
  \texttt{\ start-dy\ } : {[} \texttt{\ length\ } {]} â€'' Offset Y
  relative to the start pin.
\item
  \texttt{\ end-dx\ } : {[} \texttt{\ length\ } {]} â€'' Offset X
  relative to the end pin.
\item
  \texttt{\ end-dy\ } : {[} \texttt{\ length\ } {]} â€'' Offset Y
  relative to the end pin.
\item
  \texttt{\ start\ } : {[} \texttt{\ pin\ } {]} â€'' The start pin.
\item
  \texttt{\ end\ } : {[} \texttt{\ pin\ } {]} â€'' The end pin.
\item
  \texttt{\ ...args\ } : Additional arguments or settings for
  \href{https://github.com/typst/packages/raw/main/packages/preview/pinit/0.2.2/\#simple-arrow}{\texttt{\ simple-arrow\ }}
  , like \texttt{\ fill\ } , \texttt{\ stroke\ } and
  \texttt{\ thickness\ } .
\end{itemize}

\subsubsection{\texorpdfstring{\texttt{\ pinit-double-arrow\ }}{ pinit-double-arrow }}\label{pinit-double-arrow}

Draw an double arrow between two specified pins with optional settings.

\begin{Shaded}
\begin{Highlighting}[]
\NormalTok{\#let pinit{-}double{-}arrow(}
\NormalTok{  start{-}dx: 0pt,}
\NormalTok{  start{-}dy: 0pt,}
\NormalTok{  end{-}dx: 0pt,}
\NormalTok{  end{-}dy: 0pt,}
\NormalTok{  start,}
\NormalTok{  end,}
\NormalTok{  ..args,}
\NormalTok{) = \{ ... \}}
\end{Highlighting}
\end{Shaded}

\textbf{Arguments:}

\begin{itemize}
\tightlist
\item
  \texttt{\ start-dx\ } : {[} \texttt{\ length\ } {]} â€'' Offset X
  relative to the start pin.
\item
  \texttt{\ start-dy\ } : {[} \texttt{\ length\ } {]} â€'' Offset Y
  relative to the start pin.
\item
  \texttt{\ end-dx\ } : {[} \texttt{\ length\ } {]} â€'' Offset X
  relative to the end pin.
\item
  \texttt{\ end-dy\ } : {[} \texttt{\ length\ } {]} â€'' Offset Y
  relative to the end pin.
\item
  \texttt{\ start\ } : {[} \texttt{\ pin\ } {]} â€'' The start pin.
\item
  \texttt{\ end\ } : {[} \texttt{\ pin\ } {]} â€'' The end pin.
\item
  \texttt{\ ...args\ } : Additional arguments or settings for
  \href{https://github.com/typst/packages/raw/main/packages/preview/pinit/0.2.2/\#double-arrow}{\texttt{\ double-arrow\ }}
  , like \texttt{\ fill\ } , \texttt{\ stroke\ } and
  \texttt{\ thickness\ } .
\end{itemize}

\subsubsection{\texorpdfstring{\texttt{\ pinit-point-to\ }}{ pinit-point-to }}\label{pinit-point-to}

Draw an arrow from a specified pin to a point on the page with optional
settings.

\begin{Shaded}
\begin{Highlighting}[]
\NormalTok{\#let pinit{-}point{-}to(}
\NormalTok{  pin{-}dx: 5pt,}
\NormalTok{  pin{-}dy: 5pt,}
\NormalTok{  body{-}dx: 5pt,}
\NormalTok{  body{-}dy: 5pt,}
\NormalTok{  offset{-}dx: 35pt,}
\NormalTok{  offset{-}dy: 35pt,}
\NormalTok{  double: false,}
\NormalTok{  pin{-}name,}
\NormalTok{  body,}
\NormalTok{  ..args,}
\NormalTok{) = \{ ... \}}
\end{Highlighting}
\end{Shaded}

\textbf{Arguments:}

\begin{itemize}
\tightlist
\item
  \texttt{\ pin-dx\ } : {[} \texttt{\ length\ } {]} â€'' Offset X of
  arrow start relative to the pin.
\item
  \texttt{\ pin-dy\ } : {[} \texttt{\ length\ } {]} â€'' Offset Y of
  arrow start relative to the pin.
\item
  \texttt{\ body-dx\ } : {[} \texttt{\ length\ } {]} â€'' Offset X of
  arrow end relative to the body.
\item
  \texttt{\ body-dy\ } : {[} \texttt{\ length\ } {]} â€'' Offset Y of
  arrow end relative to the body.
\item
  \texttt{\ offset-dx\ } : {[} \texttt{\ length\ } {]} â€'' Offset X
  relative to the pin.
\item
  \texttt{\ offset-dy\ } : {[} \texttt{\ length\ } {]} â€'' Offset Y
  relative to the pin.
\item
  \texttt{\ double\ } : {[} \texttt{\ bool\ } {]} â€'' Draw a double
  arrow, default is \texttt{\ false\ } .
\item
  \texttt{\ pin-name\ } : {[} \texttt{\ pin\ } {]} â€'' The name of the
  pin to start from.
\item
  \texttt{\ body\ } : {[} \texttt{\ content\ } {]} â€'' The content to
  draw the arrow to.
\item
  \texttt{\ ...args\ } : Additional arguments or settings for
  \href{https://github.com/typst/packages/raw/main/packages/preview/pinit/0.2.2/\#simple-arrow}{\texttt{\ simple-arrow\ }}
  , like \texttt{\ fill\ } , \texttt{\ stroke\ } and
  \texttt{\ thickness\ } .
\end{itemize}

\subsubsection{\texorpdfstring{\texttt{\ pinit-point-from\ }}{ pinit-point-from }}\label{pinit-point-from}

Draw an arrow from a point on the page to a specified pin with optional
settings.

\begin{Shaded}
\begin{Highlighting}[]
\NormalTok{\#let pinit{-}point{-}from(}
\NormalTok{  pin{-}dx: 5pt,}
\NormalTok{  pin{-}dy: 5pt,}
\NormalTok{  body{-}dx: 5pt,}
\NormalTok{  body{-}dy: 5pt,}
\NormalTok{  offset{-}dx: 35pt,}
\NormalTok{  offset{-}dy: 35pt,}
\NormalTok{  double: false,}
\NormalTok{  pin{-}name,}
\NormalTok{  body,}
\NormalTok{  ..args,}
\NormalTok{) = \{ ... \}}
\end{Highlighting}
\end{Shaded}

\textbf{Arguments:}

\begin{itemize}
\tightlist
\item
  \texttt{\ pin-dx\ } : {[} \texttt{\ length\ } {]} â€'' Offset X
  relative to the pin.
\item
  \texttt{\ pin-dy\ } : {[} \texttt{\ length\ } {]} â€'' Offset Y
  relative to the pin.
\item
  \texttt{\ body-dx\ } : {[} \texttt{\ length\ } {]} â€'' Offset X
  relative to the body.
\item
  \texttt{\ body-dy\ } : {[} \texttt{\ length\ } {]} â€'' Offset Y
  relative to the body.
\item
  \texttt{\ offset-dx\ } : {[} \texttt{\ length\ } {]} â€'' Offset X
  relative to the left edge of the page.
\item
  \texttt{\ offset-dy\ } : {[} \texttt{\ length\ } {]} â€'' Offset Y
  relative to the top edge of the page.
\item
  \texttt{\ double\ } : {[} \texttt{\ bool\ } {]} â€'' Draw a double
  arrow, default is \texttt{\ false\ } .
\item
  \texttt{\ pin-name\ } : {[} \texttt{\ pin\ } {]} â€'' The name of the
  pin that the arrow to.
\item
  \texttt{\ body\ } : {[} \texttt{\ content\ } {]} â€'' The content to
  draw the arrow from.
\item
  \texttt{\ ...args\ } : Additional arguments or settings for
  \href{https://github.com/typst/packages/raw/main/packages/preview/pinit/0.2.2/\#simple-arrow}{\texttt{\ simple-arrow\ }}
  , like \texttt{\ fill\ } , \texttt{\ stroke\ } and
  \texttt{\ thickness\ } .
\end{itemize}

\subsubsection{\texorpdfstring{\texttt{\ simple-arrow\ }}{ simple-arrow }}\label{simple-arrow}

Draw a simple arrow on the page with optional settings, implemented by
\href{https://typst.app/docs/reference/visualize/polygon/}{\texttt{\ polygon\ }}
.

\begin{Shaded}
\begin{Highlighting}[]
\NormalTok{\#let simple{-}arrow(}
\NormalTok{  fill: black,}
\NormalTok{  stroke: 0pt,}
\NormalTok{  start: (0pt, 0pt),}
\NormalTok{  end: (30pt, 0pt),}
\NormalTok{  thickness: 2pt,}
\NormalTok{  arrow{-}width: 4,}
\NormalTok{  arrow{-}height: 4,}
\NormalTok{  inset: 0.5,}
\NormalTok{  tail: (),}
\NormalTok{) = \{ ... \}}
\end{Highlighting}
\end{Shaded}

\textbf{Arguments:}

\begin{itemize}
\tightlist
\item
  \texttt{\ fill\ } : {[} \texttt{\ color\ } {]} â€'' The fill color for
  the arrow.
\item
  \texttt{\ stroke\ } : {[} \texttt{\ stroke\ } {]} â€'' The stroke for
  the arrow.
\item
  \texttt{\ start\ } : {[} \texttt{\ point\ } {]} â€'' The starting
  point of the arrow.
\item
  \texttt{\ end\ } : {[} \texttt{\ point\ } {]} â€'' The ending point of
  the arrow.
\item
  \texttt{\ thickness\ } : {[} \texttt{\ length\ } {]} â€'' The
  thickness of the arrow.
\item
  \texttt{\ arrow-width\ } : {[} \texttt{\ int\ } or \texttt{\ float\ }
  {]} â€'' The width of the arrowhead relative to thickness.
\item
  \texttt{\ arrow-height\ } : {[} \texttt{\ int\ } or \texttt{\ float\ }
  {]} â€'' The height of the arrowhead relative to thickness.
\item
  \texttt{\ inset\ } : {[} \texttt{\ int\ } or \texttt{\ float\ } {]}
  â€'' The inset value for the arrowhead relative to thickness.
\item
  \texttt{\ tail\ } : {[} \texttt{\ array\ } {]} â€'' The tail settings
  for the arrow.
\end{itemize}

\subsubsection{\texorpdfstring{\texttt{\ double-arrow\ }}{ double-arrow }}\label{double-arrow}

Draw a double arrow on the page with optional settings, implemented by
\href{https://typst.app/docs/reference/visualize/polygon/}{\texttt{\ polygon\ }}
.

\begin{Shaded}
\begin{Highlighting}[]
\NormalTok{\#let double{-}arrow(}
\NormalTok{  fill: black,}
\NormalTok{  stroke: 0pt,}
\NormalTok{  start: (0pt, 0pt),}
\NormalTok{  end: (30pt, 0pt),}
\NormalTok{  thickness: 2pt,}
\NormalTok{  arrow{-}width: 4,}
\NormalTok{  arrow{-}height: 4,}
\NormalTok{  inset: 0.5,}
\NormalTok{  tail: (),}
\NormalTok{) = \{ ... \}}
\end{Highlighting}
\end{Shaded}

\textbf{Arguments:}

\begin{itemize}
\tightlist
\item
  \texttt{\ fill\ } : {[} \texttt{\ color\ } {]} â€'' The fill color for
  the arrow.
\item
  \texttt{\ stroke\ } : {[} \texttt{\ stroke\ } {]} â€'' The stroke for
  the arrow.
\item
  \texttt{\ start\ } : {[} \texttt{\ point\ } {]} â€'' The starting
  point of the arrow.
\item
  \texttt{\ end\ } : {[} \texttt{\ point\ } {]} â€'' The ending point of
  the arrow.
\item
  \texttt{\ thickness\ } : {[} \texttt{\ length\ } {]} â€'' The
  thickness of the arrow.
\item
  \texttt{\ arrow-width\ } : {[} \texttt{\ int\ } or \texttt{\ float\ }
  {]} â€'' The width of the arrowhead relative to thickness.
\item
  \texttt{\ arrow-height\ } : {[} \texttt{\ int\ } or \texttt{\ float\ }
  {]} â€'' The height of the arrowhead relative to thickness.
\item
  \texttt{\ inset\ } : {[} \texttt{\ int\ } or \texttt{\ float\ } {]}
  â€'' The inset value for the arrowhead relative to thickness.
\item
  \texttt{\ tail\ } : {[} \texttt{\ array\ } {]} â€'' The tail settings
  for the arrow.
\end{itemize}

\subsubsection{\texorpdfstring{\texttt{\ pinit-fletcher-edge\ }}{ pinit-fletcher-edge }}\label{pinit-fletcher-edge}

Draw a connecting line or arc in an fletcher arrow diagram.

\begin{Shaded}
\begin{Highlighting}[]
\NormalTok{\#let pinit{-}fletcher{-}edge(}
\NormalTok{  fletcher,}
\NormalTok{  start,}
\NormalTok{  end: none,}
\NormalTok{  start{-}dx: 0pt,}
\NormalTok{  start{-}dy: 0pt,}
\NormalTok{  end{-}dx: 0pt,}
\NormalTok{  end{-}dy: 0pt,}
\NormalTok{  width{-}scale: 100\%,}
\NormalTok{  height{-}scale: 100\%,}
\NormalTok{  default{-}width: 30pt,}
\NormalTok{  default{-}height: 30pt,}
\NormalTok{    ..args,}
\NormalTok{) = \{ ... \}}
\end{Highlighting}
\end{Shaded}

\textbf{Arguments:}

\begin{itemize}
\tightlist
\item
  \texttt{\ fletcher\ } (module): The Fletcher module. You can import it
  with something like \texttt{\ \#import\ "@preview/fletcher:0.5.1"\ }
\item
  \texttt{\ start\ } (pin): The starting pin of the edge. It is assumed
  that the pin is at the \emph{origin point (0, 0)} of the edge.
\item
  \texttt{\ end\ } (pin): The ending pin of the edge. If not provided,
  the edge will use default values for the width and height.
\item
  \texttt{\ start-dx\ } (length): The x-offset of the starting pin. You
  should use pt units.
\item
  \texttt{\ start-dy\ } (length): The y-offset of the starting pin. You
  should use pt units.
\item
  \texttt{\ end-dx\ } (length): The x-offset of the ending pin. You
  should use pt units.
\item
  \texttt{\ end-dy\ } (length): The y-offset of the ending pin. You
  should use pt units.
\item
  \texttt{\ width-scale\ } (percent): The width scale of the edge. The
  default value is 100\%. If you set the width scale to 50\%, the width
  of the edge will be half of the default width. Then you can use
  \texttt{\ "r,r"\ } which is equivalent to single \texttt{\ "r"\ } .
\item
  \texttt{\ height-scale\ } (percent): The height scale of the edge. The
  default value is 100\%.
\item
  \texttt{\ default-width\ } (length): The default width of the edge.
  The default value is 30pt, which will only be used if the end pin is
  not provided or the width is 0pt or 0em.
\item
  \texttt{\ default-height\ } (length): The default height of the edge.
  The default value is 30pt, which will only be used if the end pin is
  not provided or the height is 0pt or 0em.
\item
  \texttt{\ ..args\ } (any): An edge’s positional arguments may
  specify:

  \begin{itemize}
  \tightlist
  \item
    the edge’s \#param{[}edge{]}{[}vertices{]}, each specified with a
    CeTZ-style coordinate
  \item
    the \#param{[}edge{]}{[}label{]} content
  \item
    arrow \#param{[}edge{]}{[}marks{]}, like
    \texttt{\ "=\textgreater{}"\ } or
    \texttt{\ "\textless{}\textless{}-\textbar{}-o"\ }
  \item
    other style flags, like \texttt{\ "double"\ } or \texttt{\ "wave"\ }
  \end{itemize}
\end{itemize}

\subsection{Changelog}\label{changelog}

\subsubsection{0.2.2}\label{section}

\begin{itemize}
\tightlist
\item
  Fix bugs.
\end{itemize}

\subsubsection{0.2.1}\label{section-1}

\begin{itemize}
\tightlist
\item
  To be compatible with Typst 0.12.
\end{itemize}

\subsubsection{0.2.0}\label{section-2}

\begin{itemize}
\tightlist
\item
  \textbf{Breaking changes} : \texttt{\ \#pinit(pins,\ func)\ } is
  replaced by \texttt{\ \#pinit(callback:\ none,\ ..pins)\ } and the
  callback argument will receive an
  \texttt{\ (..positions)\ =\textgreater{}\ \{\ ..\ \}\ } function
  instead of a \texttt{\ (positions)\ =\textgreater{}\ \{\ ..\ \}\ }
  function.

  \begin{itemize}
  \tightlist
  \item
    \textbf{Migration} : you need to use a named argument
    \texttt{\ callback:\ (..positions)\ =\textgreater{}\ \{\ ..\ \}\ }
    to specify the callback function.
  \item
    \textbf{Migration} : you cannot use a array as a pin name. Now
    \texttt{\ \#pinit((pin1,\ pin2),\ callback:\ func)\ } means that we
    use \texttt{\ pin1\ } and \texttt{\ pin2\ } as a group of pins, and
    the callback function will receive \textbf{a single position} (the
    center of the bounding box of \texttt{\ pin1\ } and
    \texttt{\ pin2\ } ).
  \item
    \textbf{Benefit} : you can use
    \texttt{\ \#pinit(pin1,\ pin2,\ callback:\ func)\ } to query the
    positions of \texttt{\ pin1\ } and \texttt{\ pin2\ } separately, and
    \texttt{\ \#pinit((pin1,\ pin2),\ callback:\ func)\ } to query the
    position of the center of the bounding box of \texttt{\ pin1\ } and
    \texttt{\ pin2\ } .
  \end{itemize}
\item
  Add \texttt{\ pinit-fletcher-edge\ } function to draw a connecting
  line or arc in an fletcher arrow diagram.
\item
  Add \texttt{\ double-arrow\ } function and
  \texttt{\ pinit-double-arrow\ } function.
\item
  Add \texttt{\ double\ } argument for \texttt{\ pinit-point-to\ } and
  \texttt{\ pinit-point-from\ } functions.
\item
  Better comments and documentation.
\end{itemize}

\subsubsection{0.1.4}\label{section-3}

\begin{itemize}
\tightlist
\item
  Update documentation.
\end{itemize}

\subsubsection{0.1.3}\label{section-4}

\begin{itemize}
\tightlist
\item
  Add \texttt{\ pinit-line-to\ } function.
\end{itemize}

\subsubsection{0.1.2}\label{section-5}

\begin{itemize}
\tightlist
\item
  Add em unit support for \texttt{\ simple-arrow\ } .
\end{itemize}

\subsubsection{0.1.1}\label{section-6}

\begin{itemize}
\tightlist
\item
  Fix some bugs.
\end{itemize}

\subsubsection{0.1.0}\label{section-7}

\begin{itemize}
\tightlist
\item
  Initial release.
\end{itemize}

\subsection{Acknowledgements}\label{acknowledgements}

\begin{itemize}
\tightlist
\item
  Some of the inspirations and codes comes from
  \href{https://github.com/ntjess/typst-drafting}{typst-drafting} .
\item
  The concise and aesthetic example slide style come from course
  \emph{Data Structures and Algorithms} of
  \href{https://chaodong.me/}{Chaodong ZHENG} .
\item
  Thank \href{https://github.com/psads-git}{PaulS} for double arrow
  feature.
\item
  Thank \href{https://github.com/Jollywatt}{Jollywatt} for fletcher
  package.
\end{itemize}

\subsection{License}\label{license}

This project is licensed under the MIT License.

\subsubsection{How to add}\label{how-to-add}

Copy this into your project and use the import as \texttt{\ pinit\ }

\begin{verbatim}
#import "@preview/pinit:0.2.2"
\end{verbatim}

\includesvg[width=0.16667in,height=0.16667in]{/assets/icons/16-copy.svg}

Check the docs for
\href{https://typst.app/docs/reference/scripting/\#packages}{more
information on how to import packages} .

\subsubsection{About}\label{about}

\begin{description}
\tightlist
\item[Author :]
OrangeX4
\item[License:]
MIT
\item[Current version:]
0.2.2
\item[Last updated:]
October 17, 2024
\item[First released:]
November 7, 2023
\item[Archive size:]
14.0 kB
\href{https://packages.typst.org/preview/pinit-0.2.2.tar.gz}{\pandocbounded{\includesvg[keepaspectratio]{/assets/icons/16-download.svg}}}
\item[Repository:]
\href{https://github.com/OrangeX4/typst-pinit}{GitHub}
\item[Categor ies :]
\begin{itemize}
\tightlist
\item[]
\item
  \pandocbounded{\includesvg[keepaspectratio]{/assets/icons/16-layout.svg}}
  \href{https://typst.app/universe/search/?category=layout}{Layout}
\item
  \pandocbounded{\includesvg[keepaspectratio]{/assets/icons/16-hammer.svg}}
  \href{https://typst.app/universe/search/?category=utility}{Utility}
\end{itemize}
\end{description}

\subsubsection{Where to report issues?}\label{where-to-report-issues}

This package is a project of OrangeX4 . Report issues on
\href{https://github.com/OrangeX4/typst-pinit}{their repository} . You
can also try to ask for help with this package on the
\href{https://forum.typst.app}{Forum} .

Please report this package to the Typst team using the
\href{https://typst.app/contact}{contact form} if you believe it is a
safety hazard or infringes upon your rights.

\phantomsection\label{versions}
\subsubsection{Version history}\label{version-history}

\begin{longtable}[]{@{}ll@{}}
\toprule\noalign{}
Version & Release Date \\
\midrule\noalign{}
\endhead
\bottomrule\noalign{}
\endlastfoot
0.2.2 & October 17, 2024 \\
\href{https://typst.app/universe/package/pinit/0.2.1/}{0.2.1} & October
16, 2024 \\
\href{https://typst.app/universe/package/pinit/0.2.0/}{0.2.0} & August
26, 2024 \\
\href{https://typst.app/universe/package/pinit/0.1.4/}{0.1.4} & April
12, 2024 \\
\href{https://typst.app/universe/package/pinit/0.1.3/}{0.1.3} & December
23, 2023 \\
\href{https://typst.app/universe/package/pinit/0.1.2/}{0.1.2} & November
29, 2023 \\
\href{https://typst.app/universe/package/pinit/0.1.1/}{0.1.1} & November
7, 2023 \\
\href{https://typst.app/universe/package/pinit/0.1.0/}{0.1.0} & November
7, 2023 \\
\end{longtable}

Typst GmbH did not create this package and cannot guarantee correct
functionality of this package or compatibility with any version of the
Typst compiler or app.


\section{Package List LaTeX/pesha.tex}
\title{typst.app/universe/package/pesha}

\phantomsection\label{banner}
\phantomsection\label{template-thumbnail}
\pandocbounded{\includegraphics[keepaspectratio]{https://packages.typst.org/preview/thumbnails/pesha-0.4.0-small.webp}}

\section{pesha}\label{pesha}

{ 0.4.0 }

A clean and minimal template for your résumé or CV

\href{/app?template=pesha&version=0.4.0}{Create project in app}

\phantomsection\label{readme}
\begin{quote}
Pesha (Urdu: پیشÛ?) is the Urdu term for occupation/profession. It is
pronounced as pay-sha.
\end{quote}

A clean and minimal template for your CV or résumé.

This template is inspired by Matthew Butterick’s excellent
\href{https://practicaltypography.com/}{\emph{Practical Typography}}
book.

See
\href{https://github.com/talal/pesha/blob/main/example.pdf}{example.pdf}
or
\href{https://github.com/talal/pesha/blob/main/example-profile-picture.pdf}{example-profile-picture.pdf}
file to see how it looks.

\subsection{Usage}\label{usage}

You can use this template in the Typst web app by clicking “Start from
template� on the dashboard and searching for \texttt{\ pesha\ } .

Alternatively, you can use the CLI to kick this project off using the
command

\begin{Shaded}
\begin{Highlighting}[]
\ExtensionTok{typst}\NormalTok{ init @preview/pesha}
\end{Highlighting}
\end{Shaded}

Typst will create a new directory with all the files needed to get you
started.

\subsection{Configuration}\label{configuration}

This template exports the \texttt{\ pesha\ } function with the following
named arguments:

\begin{longtable}[]{@{}lll@{}}
\toprule\noalign{}
Argument & Type & Description \\
\midrule\noalign{}
\endhead
\bottomrule\noalign{}
\endlastfoot
\texttt{\ name\ } &
\href{https://typst.app/docs/reference/foundations/str/}{string} & A
string to specify the author’s name. \\
\texttt{\ address\ } &
\href{https://typst.app/docs/reference/foundations/str/}{string} & A
string to specify the author’s address. \\
\texttt{\ contacts\ } &
\href{https://typst.app/docs/reference/foundations/array/}{array} & An
array of content to specify your contact information. E.g., phone
number, email, LinkedIn, etc. \\
\texttt{\ profile-picture\ } &
\href{https://typst.app/docs/reference/foundations/content/}{content} &
The result of a call to the
\href{https://typst.app/docs/reference/visualize/image/}{image function}
or \texttt{\ none\ } . For best result, make sure that your image has an
1:1 aspect ratio. \\
\texttt{\ paper-size\ } &
\href{https://typst.app/docs/reference/foundations/str/}{string} &
Specify a
\href{https://typst.app/docs/reference/layout/page\#parameters-paper}{paper
size string} to change the page size (default is \texttt{\ a4\ } ). \\
\texttt{\ footer-text\ } &
\href{https://typst.app/docs/reference/foundations/content/}{content} &
Content that will be prepended to the page numbering in the footer. \\
\texttt{\ page-numbering-format\ } &
\href{https://typst.app/docs/reference/foundations/str/}{string} &
\href{https://typst.app/docs/reference/model/numbering/\#parameters-numbering}{Pattern}
that will be used for displaying page numbering in the footer (default
is \texttt{\ 1\ of\ 1\ } ). \\
\end{longtable}

The function also accepts a single, positional argument for the body.

The template will initialize your package with a sample call to the
\texttt{\ pesha\ } function in a show rule. If you, however, want to
change an existing project to use this template, you can add a show rule
like this at the top of your file:

\begin{Shaded}
\begin{Highlighting}[]
\NormalTok{\#import "@preview/pesha:0.4.0": *}

\NormalTok{\#show: pesha.with(}
\NormalTok{  name: "Max Mustermann",}
\NormalTok{  address: "5419 Hollywood Blvd Ste c731, Los Angeles, CA 90027",}
\NormalTok{  contacts: (}
\NormalTok{    [(323) 555 1435],}
\NormalTok{    [\#link("mailto:max@mustermann.com")],}
\NormalTok{  ),}
\NormalTok{  paper{-}size: "us{-}letter",}
\NormalTok{  footer{-}text: [Mustermann Résumé {-}{-}{-}]}
\NormalTok{)}

\NormalTok{// Your content goes below.}
\end{Highlighting}
\end{Shaded}

\href{/app?template=pesha&version=0.4.0}{Create project in app}

\subsubsection{How to use}\label{how-to-use}

Click the button above to create a new project using this template in
the Typst app.

You can also use the Typst CLI to start a new project on your computer
using this command:

\begin{verbatim}
typst init @preview/pesha:0.4.0
\end{verbatim}

\includesvg[width=0.16667in,height=0.16667in]{/assets/icons/16-copy.svg}

\subsubsection{About}\label{about}

\begin{description}
\tightlist
\item[Author :]
\href{https://github.com/talal}{Muhammad Talal Anwar}
\item[License:]
MIT-0
\item[Current version:]
0.4.0
\item[Last updated:]
October 24, 2024
\item[First released:]
March 23, 2024
\item[Minimum Typst version:]
0.12.0
\item[Archive size:]
4.24 kB
\href{https://packages.typst.org/preview/pesha-0.4.0.tar.gz}{\pandocbounded{\includesvg[keepaspectratio]{/assets/icons/16-download.svg}}}
\item[Repository:]
\href{https://github.com/talal/pesha}{GitHub}
\item[Categor y :]
\begin{itemize}
\tightlist
\item[]
\item
  \pandocbounded{\includesvg[keepaspectratio]{/assets/icons/16-user.svg}}
  \href{https://typst.app/universe/search/?category=cv}{CV}
\end{itemize}
\end{description}

\subsubsection{Where to report issues?}\label{where-to-report-issues}

This template is a project of Muhammad Talal Anwar . Report issues on
\href{https://github.com/talal/pesha}{their repository} . You can also
try to ask for help with this template on the
\href{https://forum.typst.app}{Forum} .

Please report this template to the Typst team using the
\href{https://typst.app/contact}{contact form} if you believe it is a
safety hazard or infringes upon your rights.

\phantomsection\label{versions}
\subsubsection{Version history}\label{version-history}

\begin{longtable}[]{@{}ll@{}}
\toprule\noalign{}
Version & Release Date \\
\midrule\noalign{}
\endhead
\bottomrule\noalign{}
\endlastfoot
0.4.0 & October 24, 2024 \\
\href{https://typst.app/universe/package/pesha/0.3.1/}{0.3.1} & April
19, 2024 \\
\href{https://typst.app/universe/package/pesha/0.3.0/}{0.3.0} & April
15, 2024 \\
\href{https://typst.app/universe/package/pesha/0.2.0/}{0.2.0} & April
12, 2024 \\
\href{https://typst.app/universe/package/pesha/0.1.0/}{0.1.0} & March
23, 2024 \\
\end{longtable}

Typst GmbH did not create this template and cannot guarantee correct
functionality of this template or compatibility with any version of the
Typst compiler or app.


\section{Package List LaTeX/postercise.tex}
\title{typst.app/universe/package/postercise}

\phantomsection\label{banner}
\section{postercise}\label{postercise}

{ 0.1.0 }

Postercise allows users to easily create academic research posters with
different themes using Typst.

\phantomsection\label{readme}
\emph{Postercise} allows users to easily create academic research
posters with different themes using \href{https://typst.app/}{Typst} .

\pandocbounded{\includegraphics[keepaspectratio]{https://img.shields.io/github/license/dangh3014/postercise}}
\pandocbounded{\includegraphics[keepaspectratio]{https://img.shields.io/github/v/release/dangh3014/postercise}}
\pandocbounded{\includegraphics[keepaspectratio]{https://img.shields.io/github/stars/dangh3014/postercise}}

\subsection{Getting started}\label{getting-started}

You can get \textbf{Postercise} from the official package repository by
entering the following.

\begin{Shaded}
\begin{Highlighting}[]
\NormalTok{\#import "@preview/postercise:0.1.0": *}
\end{Highlighting}
\end{Shaded}

Another option is to use \textbf{Postercise} as a local module by
downloading the package files into your project folder.

Next you will need to import a theme, set up the page and font, and call
the \texttt{\ show\ } command.

\begin{Shaded}
\begin{Highlighting}[]
\NormalTok{\#import themes.basic: *}

\NormalTok{\#set page(width: 24in, height: 18in)}
\NormalTok{\#set text(size: 24pt)}

\NormalTok{\#show: theme}
\end{Highlighting}
\end{Shaded}

To add content to the poster, use the \texttt{\ poster-content\ }
command.

\begin{Shaded}
\begin{Highlighting}[]
\NormalTok{\#poster{-}content()[}
\NormalTok{  // Content goes here}
\NormalTok{]}
\end{Highlighting}
\end{Shaded}

There are a few options for types of content that should be added inside
the \texttt{\ poster-content\ } . The body of the poster can be typed as
normal, or two box styles are provided to headings and/or highlight
content in special ways.

\begin{Shaded}
\begin{Highlighting}[]
\NormalTok{\#normal{-}box[]}
\NormalTok{\#focus{-}box[]}
\end{Highlighting}
\end{Shaded}

Basic information like title and authors is placed as options using the
\texttt{\ poster-header\ } script.

\begin{Shaded}
\begin{Highlighting}[]
\NormalTok{\#poster{-}header(}
\NormalTok{  title: [Title],}
\NormalTok{  authors: [Author],}
\NormalTok{)}
\end{Highlighting}
\end{Shaded}

Finally, additional content can be added to the footer with the
\texttt{\ poster-footer\ } script.

\begin{Shaded}
\begin{Highlighting}[]
\NormalTok{\#poster{-}footer[]}
\end{Highlighting}
\end{Shaded}

Again, as a reminder, all of these scripts should be called from inside
of the \texttt{\ poster-content\ } block.

Using these commands, it is easy to produce posters like the following:
\pandocbounded{\includegraphics[keepaspectratio]{https://raw.githubusercontent.com/dangh3014/postercise/main/examples/postercise-examples.png}}

\subsection{More details}\label{more-details}

\subsubsection{\texorpdfstring{\texttt{\ themes\ }}{ themes }}\label{themes}

Currently, 3 themes are available. Use one of these \texttt{\ import\ }
commands to load that theme.

\begin{Shaded}
\begin{Highlighting}[]
\NormalTok{\#import themes.basic: *}
\NormalTok{\#import themes.better: *}
\NormalTok{\#import themes.boxes: *}
\end{Highlighting}
\end{Shaded}

\subsubsection{\texorpdfstring{\texttt{\ show:\ theme.with()\ }}{ show: theme.with() }}\label{show-theme.with}

Theme options allow you to adjust the color scheme, as well as the color
and size of the content in the header. The defaults are shown below.
(The ‘better.typ’ theme defaults to different titletext color and
size.)

\begin{Shaded}
\begin{Highlighting}[]
\NormalTok{\#show: theme.with(}
\NormalTok{  primary{-}color: rgb(28,55,103), // Dark blue}
\NormalTok{  background{-}color: white,}
\NormalTok{  accent{-}color: rgb(243,163,30), // Yellow}
\NormalTok{  titletext{-}color: white,}
\NormalTok{  titletext{-}size: 2em,}
\NormalTok{)}
\end{Highlighting}
\end{Shaded}

\subsubsection{\texorpdfstring{\texttt{\ poster-content(){[}{]}\ }}{ poster-content(){[}{]} }}\label{poster-content}

The only option for the main content is the number of columns. This
defaults to 3 for most themes. For the “better.typ� theme, there is
1 column and content is placed in the leftmost column below
\texttt{\ poster-header\ } .

\begin{Shaded}
\begin{Highlighting}[]
\NormalTok{\#poster{-}content(col: 3)[}
\NormalTok{  // Content goes here}
\NormalTok{]}
\end{Highlighting}
\end{Shaded}

\subsubsection{\texorpdfstring{\texttt{\ normal-box(){[}{]}\ } and
\texttt{\ focus-box(){[}{]}\ }}{ normal-box(){[}{]}  and  focus-box(){[}{]} }}\label{normal-box-and-focus-box}

By default, these boxes use the no fill and the accent-color fill,
respectively. However, they do accept color as an option, and will add a
primary-color stroke around the box if a color is given. For the
“better.typ� theme, use \texttt{\ focus-box\ } to place content in
the center column.

\begin{Shaded}
\begin{Highlighting}[]
\NormalTok{\#normal{-}box(color: none)[}
\NormalTok{  // Content}
\NormalTok{]}

\NormalTok{\#focus{-}box(color: none)[}
\NormalTok{  // Content}
\NormalTok{]}
\end{Highlighting}
\end{Shaded}

\subsubsection{\texorpdfstring{\texttt{\ poster-header()\ }}{ poster-header() }}\label{poster-header}

Available options for the poster header for most themes are shown below.
Note that logos should be explicitly labeled as images. Logos are not
currently displayed in the header in the “better.typ� theme.

\begin{Shaded}
\begin{Highlighting}[]
\NormalTok{\#poster{-}header(}
\NormalTok{  title: [Title],}
\NormalTok{  subtitle: [Subtitle],}
\NormalTok{  author: [Author],}
\NormalTok{  affiliation: [Affiliation],}
\NormalTok{  logo{-}1: image("placeholder.png")}
\NormalTok{  logo{-}2: image("placeholder.png") }
\NormalTok{)}
\end{Highlighting}
\end{Shaded}

\subsubsection{\texorpdfstring{\texttt{\ poster-footer{[}{]}\ }}{ poster-footer{[}{]} }}\label{poster-footer}

This command does not currently have any extra options. The content is
typically placed at the bottom of the poster, but it is placed in the
rightmost column for the “better.typ� theme.

\begin{Shaded}
\begin{Highlighting}[]
\NormalTok{\#poster{-}footer[}
\NormalTok{  // Content}
\NormalTok{]}
\end{Highlighting}
\end{Shaded}

\subsection{Known Issues}\label{known-issues}

\begin{itemize}
\tightlist
\item
  The bibliography does not work properly and must be done manually
\item
  Figure captions do not number correctly and must be done manually
\end{itemize}

\subsection{Planned Features/Themes}\label{planned-featuresthemes}

\begin{itemize}
\tightlist
\item
  Themes that use color gradients and background images
\item
  Add QR code generation
\end{itemize}

\subsubsection{How to add}\label{how-to-add}

Copy this into your project and use the import as
\texttt{\ postercise\ }

\begin{verbatim}
#import "@preview/postercise:0.1.0"
\end{verbatim}

\includesvg[width=0.16667in,height=0.16667in]{/assets/icons/16-copy.svg}

Check the docs for
\href{https://typst.app/docs/reference/scripting/\#packages}{more
information on how to import packages} .

\subsubsection{About}\label{about}

\begin{description}
\tightlist
\item[Author :]
Daniel King
\item[License:]
MIT
\item[Current version:]
0.1.0
\item[Last updated:]
May 27, 2024
\item[First released:]
May 27, 2024
\item[Archive size:]
5.11 kB
\href{https://packages.typst.org/preview/postercise-0.1.0.tar.gz}{\pandocbounded{\includesvg[keepaspectratio]{/assets/icons/16-download.svg}}}
\item[Repository:]
\href{https://github.com/dangh3014/postercise/}{GitHub}
\item[Categor y :]
\begin{itemize}
\tightlist
\item[]
\item
  \pandocbounded{\includesvg[keepaspectratio]{/assets/icons/16-pin.svg}}
  \href{https://typst.app/universe/search/?category=poster}{Poster}
\end{itemize}
\end{description}

\subsubsection{Where to report issues?}\label{where-to-report-issues}

This package is a project of Daniel King . Report issues on
\href{https://github.com/dangh3014/postercise/}{their repository} . You
can also try to ask for help with this package on the
\href{https://forum.typst.app}{Forum} .

Please report this package to the Typst team using the
\href{https://typst.app/contact}{contact form} if you believe it is a
safety hazard or infringes upon your rights.

\phantomsection\label{versions}
\subsubsection{Version history}\label{version-history}

\begin{longtable}[]{@{}ll@{}}
\toprule\noalign{}
Version & Release Date \\
\midrule\noalign{}
\endhead
\bottomrule\noalign{}
\endlastfoot
0.1.0 & May 27, 2024 \\
\end{longtable}

Typst GmbH did not create this package and cannot guarantee correct
functionality of this package or compatibility with any version of the
Typst compiler or app.


\section{Package List LaTeX/datify.tex}
\title{typst.app/universe/package/datify}

\phantomsection\label{banner}
\section{datify}\label{datify}

{ 0.1.3 }

Datify is a simple date package that allows users to format dates in new
ways and addresses the issue of lacking date formats in different
languages.

\phantomsection\label{readme}
Datify is a simple date package that allows users to format dates in new
ways and addresses the issue of lacking date formats in different
languages.

\subsection{Installation}\label{installation}

To include this package in your Typst project, add the following to your
project file:

\begin{Shaded}
\begin{Highlighting}[]
\NormalTok{\#import "@preview/datify:0.1.3": day{-}name, month{-}name, custom{-}date{-}format}
\end{Highlighting}
\end{Shaded}

\subsection{Reference}\label{reference}

\subsubsection{\texorpdfstring{\texttt{\ day-name\ }}{ day-name }}\label{day-name}

Returns the name of the weekday.

\paragraph{Example}\label{example}

\begin{Shaded}
\begin{Highlighting}[]
\NormalTok{\#import "@preview/datify:0.1.3": day{-}name}

\NormalTok{\#day{-}name(2)}
\NormalTok{\#day{-}name(1,"fr",true)}
\end{Highlighting}
\end{Shaded}

Output:

\begin{verbatim}
tuesday
Lundi
\end{verbatim}

\paragraph{Parameters}\label{parameters}

\begin{Shaded}
\begin{Highlighting}[]
\NormalTok{day{-}name(weekday: int or str, lang: str, upper: boolean) {-}{-}\textgreater{} str}
\end{Highlighting}
\end{Shaded}

\begin{longtable}[]{@{}lll@{}}
\toprule\noalign{}
Parameter & Description & Default \\
\midrule\noalign{}
\endhead
\bottomrule\noalign{}
\endlastfoot
weekday* & The weekday as an integer (1-7) or a string
(“1�-“7�). & none \\
lang & An ISO 639-1 code representing the language. & en \\
upper & A boolean that sets the first letter to be uppercase. & false \\
\end{longtable}

* required

\subsubsection{\texorpdfstring{\texttt{\ month-name\ }}{ month-name }}\label{month-name}

Returns the name of the month.

\paragraph{Example}\label{example-1}

\begin{Shaded}
\begin{Highlighting}[]
\NormalTok{\#import "@preview/datify:0.1.3": month{-}name}

\NormalTok{\#month{-}name(2)}
\NormalTok{\#month{-}name(1, "fr", true)}
\end{Highlighting}
\end{Shaded}

Output:

\begin{verbatim}
february
Janvier
\end{verbatim}

\paragraph{Parameters}\label{parameters-1}

\begin{Shaded}
\begin{Highlighting}[]
\NormalTok{month{-}name(month: int or str, lang: str = \textquotesingle{}en\textquotesingle{}, upper: bool = false) {-}\textgreater{} str}
\end{Highlighting}
\end{Shaded}

\begin{longtable}[]{@{}lll@{}}
\toprule\noalign{}
Parameter & Description & Default \\
\midrule\noalign{}
\endhead
\bottomrule\noalign{}
\endlastfoot
month* & The month as an integer (1-12) or a string (“1�-“12�).
& none \\
lang & An ISO 639-1 code representing the language. & en \\
upper & A boolean that sets the first letter to be uppercase. & false \\
\end{longtable}

* required

\subsubsection{\texorpdfstring{\texttt{\ custom-date-format\ }}{ custom-date-format }}\label{custom-date-format}

Formats a given date according to a specified format and language.

\paragraph{Example}\label{example-2}

\begin{Shaded}
\begin{Highlighting}[]
\NormalTok{\#import "@preview/datify:0.1.3": custom{-}date{-}format}

\NormalTok{\#let my{-}date = datetime(year: 2024, month: 8, day: 4)}
\NormalTok{\#custom{-}date{-}format(my{-}date, "MMMM DDth, YYYY")}
\end{Highlighting}
\end{Shaded}

Output:

\begin{verbatim}
August 04th, 2024
\end{verbatim}

\paragraph{Parameters}\label{parameters-2}

\begin{Shaded}
\begin{Highlighting}[]
\NormalTok{custom{-}date{-}format(date: datetime, format: str, lang: str = \textquotesingle{}en\textquotesingle{}) {-}\textgreater{} str}
\end{Highlighting}
\end{Shaded}

\begin{longtable}[]{@{}lll@{}}
\toprule\noalign{}
Parameter & Description & Default \\
\midrule\noalign{}
\endhead
\bottomrule\noalign{}
\endlastfoot
date* & A datetime object representing the date. & none \\
format* & A string representing the desired date format. & none \\
lang & An ISO 639-1 code representing the language. & en \\
\end{longtable}

* required

\paragraph{Format Types}\label{format-types}

Below is a table of all possible format types that can be used in the
format string:

\begin{longtable}[]{@{}lll@{}}
\toprule\noalign{}
Format & Description & Example \\
\midrule\noalign{}
\endhead
\bottomrule\noalign{}
\endlastfoot
\texttt{\ DD\ } & Day of the month, 2 digits & 05 \\
\texttt{\ day\ } & Full name of the day & tuesday \\
\texttt{\ Day\ } & Capitalized full name of the day & Tuesday \\
\texttt{\ DAY\ } & Uppercase full name of the day & TUESDAY \\
\texttt{\ MMMM\ } & Capitalized full name of the month & May \\
\texttt{\ MMM\ } & Short name of the month (first 3 chars) & May \\
\texttt{\ MM\ } & Month number, 2 digits & 05 \\
\texttt{\ month\ } & Full name of the month & may \\
\texttt{\ Month\ } & Capitalized full name of the month & May \\
\texttt{\ MONTH\ } & Uppercase full name of the month & MAY \\
\texttt{\ YYYY\ } & 4-digit year & 2023 \\
\texttt{\ YY\ } & Last 2 digits of the year & 23 \\
\end{longtable}

\subsection{Examples}\label{examples}

Here are some examples demonstrating the usage of the functions provided
by the Datify package:

\begin{Shaded}
\begin{Highlighting}[]
\NormalTok{\#let my{-}date = datetime(year: 2024, month: 12, day: 25)}

\NormalTok{\#custom{-}date{-}format(my{-}date, "DD{-}MM{-}YYYY")  // Output: 25{-}12{-}2024}
\NormalTok{\#custom{-}date{-}format(my{-}date, "Day, DD Month YYYY", "fr")  // Output: Mercredi, 25 Décembre 2024}
\NormalTok{\#custom{-}date{-}format(my{-}date, "day, DD de month de YYYY", "es") // Output: miércoles, 25 de diciembre de 2024}
\NormalTok{\#custom{-}date{-}format(my{-}date, "day, DD de month de YYYY", "pt") // Output: terça{-}feira, 25 de dezembro de 2024}

\NormalTok{\#day{-}name(4)  // Output: thursday}

\NormalTok{\#month{-}name(12)  // Output: december}
\end{Highlighting}
\end{Shaded}

\subsection{Supported language}\label{supported-language}

\begin{longtable}[]{@{}ll@{}}
\toprule\noalign{}
ISO 639-1 code & Status \\
\midrule\noalign{}
\endhead
\bottomrule\noalign{}
\endlastfoot
aa & � \\
ab & � \\
ae & � \\
af & � \\
ak & � \\
am & � \\
an & � \\
ar & � \\
as & � \\
av & � \\
ay & � \\
az & � \\
ba & � \\
be & � \\
bg & � \\
bh & � \\
bi & � \\
bm & � \\
bn & � \\
bo & � \\
br & � \\
bs & � \\
ca & � \\
ce & � \\
ch & � \\
co & � \\
cr & � \\
cs & � \\
cu & � \\
cv & � \\
cy & � \\
da & � \\
de & ✠\\
dv & � \\
dz & � \\
ee & � \\
el & � \\
en & ✠\\
eo & � \\
es & ✠\\
et & � \\
eu & � \\
fa & � \\
ff & � \\
fi & � \\
fj & � \\
fo & � \\
fr & ✠\\
fy & � \\
ga & � \\
gd & � \\
gl & � \\
gn & � \\
gu & � \\
gv & � \\
ha & � \\
he & � \\
hi & � \\
ho & � \\
hr & � \\
ht & � \\
hu & � \\
hy & � \\
hz & � \\
ia & � \\
id & � \\
ie & � \\
ig & � \\
ii & � \\
ik & � \\
io & � \\
is & � \\
it & � \\
iu & � \\
ja & � \\
jv & � \\
ka & � \\
kg & � \\
ki & � \\
kj & � \\
kk & � \\
kl & � \\
km & � \\
kn & � \\
ko & � \\
kr & � \\
ks & � \\
ku & � \\
kv & � \\
kw & � \\
ky & � \\
la & � \\
lb & � \\
lg & � \\
li & � \\
ln & � \\
lo & � \\
lt & � \\
lu & � \\
lv & � \\
mg & � \\
mh & � \\
mi & � \\
mk & � \\
ml & � \\
mn & � \\
mr & � \\
ms & � \\
mt & � \\
my & � \\
na & � \\
nb & � \\
nd & � \\
ne & � \\
ng & � \\
nl & � \\
nn & � \\
no & � \\
nr & � \\
nv & � \\
ny & � \\
oc & � \\
oj & � \\
om & � \\
or & � \\
os & � \\
pa & � \\
pi & � \\
pl & � \\
ps & � \\
pt & ✠\\
qu & � \\
rm & � \\
rn & � \\
ro & � \\
ru & � \\
rw & � \\
sa & � \\
sc & � \\
sd & � \\
se & � \\
sg & � \\
si & � \\
sk & � \\
sl & � \\
sm & � \\
sn & � \\
so & � \\
sq & � \\
sr & � \\
ss & � \\
st & � \\
su & � \\
sv & � \\
sw & � \\
ta & � \\
te & � \\
tg & � \\
th & � \\
ti & � \\
tk & � \\
tl & � \\
tn & � \\
to & � \\
tr & � \\
ts & � \\
tt & � \\
tw & � \\
ty & � \\
ug & � \\
uk & � \\
ur & � \\
uz & � \\
ve & � \\
vi & � \\
vo & � \\
wa & � \\
wo & � \\
xh & � \\
yi & � \\
yo & � \\
za & � \\
zh & � \\
zu & � \\
\end{longtable}

\subsection{Contributing}\label{contributing}

Contributions are welcome! Please feel free to submit a pull request or
open an issue if you encounter any problems.

\subsection{License}\label{license}

This project is licensed under the MIT License.

\subsection{Planned}\label{planned}

\begin{itemize}
\tightlist
\item
  Adding support for more language
\item
  Adding set and get method to set default language for a whole document
\item
  Adding new methods
\end{itemize}

\subsubsection{How to add}\label{how-to-add}

Copy this into your project and use the import as \texttt{\ datify\ }

\begin{verbatim}
#import "@preview/datify:0.1.3"
\end{verbatim}

\includesvg[width=0.16667in,height=0.16667in]{/assets/icons/16-copy.svg}

Check the docs for
\href{https://typst.app/docs/reference/scripting/\#packages}{more
information on how to import packages} .

\subsubsection{About}\label{about}

\begin{description}
\tightlist
\item[Author :]
Jeomhps
\item[License:]
MIT
\item[Current version:]
0.1.3
\item[Last updated:]
November 4, 2024
\item[First released:]
May 27, 2024
\item[Minimum Typst version:]
0.11.1
\item[Archive size:]
4.81 kB
\href{https://packages.typst.org/preview/datify-0.1.3.tar.gz}{\pandocbounded{\includesvg[keepaspectratio]{/assets/icons/16-download.svg}}}
\item[Repository:]
\href{https://github.com/Jeomhps/datify}{GitHub}
\end{description}

\subsubsection{Where to report issues?}\label{where-to-report-issues}

This package is a project of Jeomhps . Report issues on
\href{https://github.com/Jeomhps/datify}{their repository} . You can
also try to ask for help with this package on the
\href{https://forum.typst.app}{Forum} .

Please report this package to the Typst team using the
\href{https://typst.app/contact}{contact form} if you believe it is a
safety hazard or infringes upon your rights.

\phantomsection\label{versions}
\subsubsection{Version history}\label{version-history}

\begin{longtable}[]{@{}ll@{}}
\toprule\noalign{}
Version & Release Date \\
\midrule\noalign{}
\endhead
\bottomrule\noalign{}
\endlastfoot
0.1.3 & November 4, 2024 \\
\href{https://typst.app/universe/package/datify/0.1.2/}{0.1.2} & May 31,
2024 \\
\href{https://typst.app/universe/package/datify/0.1.1/}{0.1.1} & May 27,
2024 \\
\end{longtable}

Typst GmbH did not create this package and cannot guarantee correct
functionality of this package or compatibility with any version of the
Typst compiler or app.


\section{Package List LaTeX/supercharged-dhbw.tex}
\title{typst.app/universe/package/supercharged-dhbw}

\phantomsection\label{banner}
\phantomsection\label{template-thumbnail}
\pandocbounded{\includegraphics[keepaspectratio]{https://packages.typst.org/preview/thumbnails/supercharged-dhbw-3.3.2-small.webp}}

\section{supercharged-dhbw}\label{supercharged-dhbw}

{ 3.3.2 }

Unofficial Template for DHBW

\href{/app?template=supercharged-dhbw&version=3.3.2}{Create project in
app}

\phantomsection\label{readme}
Unofficial \href{https://typst.app/}{Typst} template for DHBW students.

You can see an example PDF of how the template looks
\href{https://github.com/DannySeidel/typst-dhbw-template/blob/main/examples/example.pdf}{here}
.

To see an example of how you can use this template, check out the
\texttt{\ main.typ\ } file. More examples can be found in the
\href{https://github.com/DannySeidel/typst-dhbw-template/blob/main/examples}{examples
directory} of the GitHub repository.

\subsection{Usage}\label{usage}

You can use this template in the Typst web app by clicking “Start from
template� on the dashboard and searching for
\texttt{\ supercharged-dhbw\ } .

Alternatively, you can use the CLI to kick this project off using the
command

\begin{Shaded}
\begin{Highlighting}[]
\NormalTok{typst init @preview/supercharged{-}dhbw}
\end{Highlighting}
\end{Shaded}

Typst will create a new directory with all the files needed to get you
started.

\subsection{Fonts}\label{fonts}

This template uses the following fonts:

\begin{itemize}
\tightlist
\item
  \href{https://fonts.google.com/specimen/Montserrat}{Montserrat}
\item
  \href{https://fonts.google.com/specimen/Open+Sans}{Open Sans}
\end{itemize}

If you want to use typst locally, you can download the fonts from the
links above and install them on your system. Otherwise, when using the
web version add the fonts to your project.

For further information on how to add fonts to your project, please
refer to the
\href{https://typst.app/docs/reference/text/text/\#parameters-font}{Typst
documentation} .

\subsection{Used Packages}\label{used-packages}

This template uses the following packages:

\begin{itemize}
\tightlist
\item
  \href{https://typst.app/universe/package/codelst}{codelst} : To create
  code snippets
\end{itemize}

Insert code snippets using the following syntax:

\begin{Shaded}
\begin{Highlighting}[]
\NormalTok{\#figure(caption: "Codeblock Example", sourcecode[\textasciigrave{}\textasciigrave{}\textasciigrave{}ts}
\NormalTok{const ReactComponent = () =\textgreater{} \{}
\NormalTok{  return (}
\NormalTok{    \textless{}div\textgreater{}}
\NormalTok{      \textless{}h1\textgreater{}Hello World\textless{}/h1\textgreater{}}
\NormalTok{    \textless{}/div\textgreater{}}
\NormalTok{  );}
\NormalTok{\};}

\NormalTok{export default ReactComponent;}
\NormalTok{\textasciigrave{}\textasciigrave{}\textasciigrave{}])}
\end{Highlighting}
\end{Shaded}

\subsection{Configuration}\label{configuration}

This template exports the \texttt{\ supercharged-dhbw\ } function with
the following named arguments:

\texttt{\ title\ (str*)\ } : Title of the document

\texttt{\ authors\ (dictionary*)\ } : List of authors with the following
named arguments (max. 6 authors when in the company or 8 authors when at
DHBW):

\begin{itemize}
\tightlist
\item
  name (str*): Name of the author
\item
  student-id (str*): Student ID of the author
\item
  course (str*): Course of the author
\item
  course-of-studies (str*): Course of studies of the author
\item
  company (dictionary): Company of the author (only needed when
  \texttt{\ at-university\ } is \texttt{\ false\ } ) with the following
  named arguments:

  \begin{itemize}
  \tightlist
  \item
    name (str*): Name of the company
  \item
    post-code (str): Post code of the company
  \item
    city (str*): City of the company
  \item
    country (str): Country of the company
  \end{itemize}
\end{itemize}

\texttt{\ abstract\ (content)\ } : Content of the abstract, it is
recommended that you pass a variable containing the content or a
function that returns the content

\texttt{\ acronym-spacing\ (length)\ } : Spacing between the acronym and
its long form (check the
\href{https://typst.app/docs/reference/layout/length/}{Typst
documentation} for examples on how to provide parameters of type
length), default is \texttt{\ 5em\ }

\texttt{\ acronyms\ (dictionary)\ } : Pass a dictionary containing the
acronyms and their long forms (See the example in the
\texttt{\ acronyms.typ\ } file)

\texttt{\ appendix\ (content)\ } : Content of the appendix, it is
recommended that you pass a variable containing the content or a
function that returns the content

\texttt{\ at-university\ (bool*)\ } : Whether the document is written at
university or not, default is \texttt{\ false\ }

\texttt{\ bibliography\ (content)\ } : Path to the bibliography file

\texttt{\ bib-style\ (str)\ } : Style of the bibliography, default is
\texttt{\ ieee\ }

\texttt{\ city\ (str)\ } : City of the author (only needed when
\texttt{\ at-university\ } is \texttt{\ true\ } )

\texttt{\ confidentiality-marker:\ (dictionary)\ } : Configure the
confidentially marker (red or green circle) on the title page (using
this option reduces the maximum number of authors by 2 to 4 authors when
in the company or 6 authors when at DHBW)

\begin{itemize}
\tightlist
\item
  display (bool*): Whether the confidentiality marker should be shown,
  default is \texttt{\ false\ }
\item
  offset-x (length): Horizontal offset of the confidentiality marker,
  default is \texttt{\ 0pt\ }
\item
  offset-y (length): Vertical offset of the confidentiality marker,
  default is \texttt{\ 0pt\ }
\item
  size (length): Size of the confidentiality marker, default is
  \texttt{\ 7em\ }
\item
  title-spacing (length): Adds space below the title to make room for
  the confidentiality marker, default is \texttt{\ 2em\ }
\end{itemize}

\texttt{\ confidentiality-statement-content\ (content)\ } : Provide a
custom confidentiality statement

\texttt{\ date\ (datetime*\ \textbar{}\ array*)\ } : Provide a datetime
object to display one date (e.g. submission date) or a array containing
two datetime objects to display a date range (e.g. start and end date of
the project), default is \texttt{\ datetime.today()\ }

\texttt{\ date-format\ (str)\ } : Format of the displayed dates, default
is \texttt{\ "{[}day{]}.{[}month{]}.{[}year{]}"\ } (for more information
on possible formats check the
\href{https://typst.app/docs/reference/foundations/datetime/\#format}{Typst
documentation} )

\texttt{\ declaration-of-authorship-content\ (content)\ } : Provide a
custom declaration of authorship

\texttt{\ glossary\ (dictionary)\ } : Pass a dictionary containing the
glossary terms and their definitions (See the example in the
\texttt{\ glossary.typ\ } file)

\texttt{\ glossary-spacing\ (length)\ } : Spacing between the glossary
term and its definition (check the
\href{https://typst.app/docs/reference/layout/length/}{Typst
documentation} for examples on how to provide parameters of type
length), default is \texttt{\ 1.5em\ }

\texttt{\ header\ (dictionary)\ } : Configure the header of the document

\begin{itemize}
\tightlist
\item
  display (bool): Whether the header should be shown, default is
  \texttt{\ true\ }
\item
  show-chapter (bool): Whether the current chapter should be shown in
  the header, default is \texttt{\ true\ }
\item
  show-left-logo (bool): Whether the left logo should be shown in the
  header, default is \texttt{\ true\ }
\item
  show-right-logo (bool): Whether the right logo should be shown in the
  header, default is \texttt{\ true\ }
\item
  show-divider (bool): Whether the header divider should be shown,
  default is \texttt{\ true\ }
\item
  content (content): Content for a custom header, it is recommended that
  you pass a variable containing the content or a function that returns
  the content
\end{itemize}

\texttt{\ heading-numering\ (str)\ } : Numbering style of the headings,
default is \texttt{\ "1.1"\ } (for more information on possible
numbering formats check the
\href{https://typst.app/docs/reference/model/numbering}{Typst
documentation} )

\texttt{\ ignored-link-label-keys-for-highlighting\ (array)\ } : List of
keys of labels that should be ignored when highlighting links in the
document, default is \texttt{\ ()\ }

\texttt{\ language\ (str*)\ } : Language of the document which is either
\texttt{\ en\ } or \texttt{\ de\ } , default is \texttt{\ en\ }

\texttt{\ logo-left\ (content)\ } : Path to the logo on the left side of
the title page (usage: image(“path/to/image.png�)), default is the
\texttt{\ DHBW\ logo\ }

\texttt{\ logo-right\ (content)\ } : Path to the logo on the right side
of the title page (usage: image(“path/to/image.png�)), default is
\texttt{\ no\ logo\ }

\texttt{\ logo-size-ratio\ (str)\ } : Ratio between the right logo and
the left logo height (left-logo:right-logo), default is
\texttt{\ "1:1"\ }

\texttt{\ math-numbering\ (str)\ } : Numbering style of the math
equations, set to \texttt{\ none\ } to turn off equation numbering,
default is \texttt{\ "(1)"\ } (for more information on possible
numbering formats check the
\href{https://typst.app/docs/reference/model/numbering}{Typst
documentation} )

\texttt{\ numbering-alignment\ (alignment)\ } : Alignment of the page
numbering (for possible options check the
\href{https://typst.app/docs/reference/layout/alignment/}{Typst
documentation} ), default is \texttt{\ center\ }

\texttt{\ show-abstract\ (bool)\ } : Whether the abstract should be
shown, default is \texttt{\ true\ }

\texttt{\ show-acronyms\ (bool)\ } : Whether the list of acronyms should
be shown, default is \texttt{\ true\ }

\texttt{\ show-code-snippets\ (bool)\ } : Whether the code snippets
should be shown, default is \texttt{\ true\ }

\texttt{\ show-confidentiality-statement\ (bool)\ } : Whether the
confidentiality statement should be shown, default is \texttt{\ true\ }

\texttt{\ show-declaration-of-authorship\ (bool)\ } : Whether the
declaration of authorship should be shown, default is \texttt{\ true\ }

\texttt{\ show-list-of-figures\ (bool)\ } : Whether the list of figures
should be shown, default is \texttt{\ true\ }

\texttt{\ show-list-of-tables\ (bool)\ } : Whether the list of tables
should be shown, default is \texttt{\ true\ }

\texttt{\ show-table-of-contents\ (bool)\ } : Whether the table of
contents should be shown, default is \texttt{\ true\ }

\texttt{\ supervisor\ (dictionary*)\ } : Name of the supervisor at the
university and/or company (e.g. supervisor: (company: “John Doe�,
university: “Jane Doe�))

\begin{itemize}
\tightlist
\item
  company (str): Name of the supervisor at the company (note while the
  argument is optional at least one of the two arguments must be
  provided)
\item
  university (str): Name of the supervisor at the university (note while
  the argument is optional at least one of the two arguments must be
  provided)
\end{itemize}

\texttt{\ titlepage-content\ (content)\ } : Provide a custom title page

\texttt{\ toc-depth\ (int)\ } : Depth of the table of contents, default
is \texttt{\ 3\ }

\texttt{\ type-of-thesis\ (str)\ } : Type of the thesis, default is
\texttt{\ none\ } (using this option reduces the maximum number of
authors by 2 to 4 authors when in the company or 6 authors when at DHBW)

\texttt{\ type-of-degree\ (str)\ } : Type of the degree, default is
\texttt{\ none\ } (using this option reduces the maximum number of
authors by 2 to 4 authors when in the company or 6 authors when at DHBW)

\texttt{\ university\ (str*)\ } : Name of the university

\texttt{\ university-location\ (str*)\ } : Campus or city of the
university

\texttt{\ university-short\ (str*)\ } : Short name of the university
(e.g. DHBW), displayed for the university supervisor

Behind the arguments the type of the value is given in parentheses. All
arguments marked with \texttt{\ *\ } are required.

\subsection{Acronyms}\label{acronyms}

This template provides functions to reference acronyms in the text. To
use these functions, you need to define the acronyms in the
\texttt{\ acronyms\ } attribute of the template. The acronyms referenced
with the functions below will be linked to their definition in the list
of acronyms.

\subsubsection{Functions}\label{functions}

This template provides the following functions to reference acronyms:

\texttt{\ acr\ } : Reference an acronym in the text (e.g.
\texttt{\ acr("API")\ } -\textgreater{}
\texttt{\ Application\ Programming\ Interface\ (API)\ } or
\texttt{\ API\ } )

\texttt{\ acrpl\ } : Reference an acronym in the text in plural form
(e.g. \texttt{\ acrpl("API")\ } -\textgreater{}
\texttt{\ Application\ Programming\ Interfaces\ (API)\ } or
\texttt{\ APIs\ } )

\texttt{\ acrs\ } : Reference an acronym in the text in short form (e.g.
\texttt{\ acrs("API")\ } -\textgreater{} \texttt{\ API\ } )

\texttt{\ acrspl\ } : Reference an acronym in the text in short form in
plural form (e.g. \texttt{\ acrpl("API")\ } -\textgreater{}
\texttt{\ APIs\ } )

\texttt{\ acrl\ } : Reference an acronym in the text in long form (e.g.
\texttt{\ acrl("API")\ } -\textgreater{}
\texttt{\ Application\ Programming\ Interface\ } )

\texttt{\ acrlpl\ } : Reference an acronym in the text in long form in
plural form (e.g. \texttt{\ acrlpl("API")\ } -\textgreater{}
\texttt{\ Application\ Programming\ Interfaces\ } )

\texttt{\ acrf\ } : Reference an acronym in the text in full form (e.g.
\texttt{\ acrf("API")\ } -\textgreater{}
\texttt{\ Application\ Programming\ Interface\ (API)\ } )

\texttt{\ acrfpl\ } : Reference an acronym in the text in full form in
plural form (e.g. \texttt{\ acrfpl("API")\ } -\textgreater{}
\texttt{\ Application\ Programming\ Interfaces\ (API)\ } )

\subsubsection{Definition}\label{definition}

To define acronyms use a dictionary and pass it to the acronyms
attribute of the template. The dictionary should contain the acronyms as
keys and their long forms as values.

\begin{Shaded}
\begin{Highlighting}[]
\NormalTok{\#let acronyms = (}
\NormalTok{  API: "Application Programming Interface",}
\NormalTok{  HTTP: "Hypertext Transfer Protocol",}
\NormalTok{  REST: "Representational State Transfer",}
\NormalTok{)}
\end{Highlighting}
\end{Shaded}

To define the plural form of an acronym use a array as value with the
first element being the singular form and the second element being the
plural form. If you don’t define the plural form, the template will
automatically add an “s� to the singular form.

\begin{Shaded}
\begin{Highlighting}[]
\NormalTok{\#let acronyms = (}
\NormalTok{  API: ("Application Programming Interface", "Application Programming Interfaces"),}
\NormalTok{  HTTP: ("Hypertext Transfer Protocol", "Hypertext Transfer Protocols"),}
\NormalTok{  REST: ("Representational State Transfer", "Representational State Transfers"),}
\NormalTok{)}
\end{Highlighting}
\end{Shaded}

\subsection{Glossary}\label{glossary}

Similar to the acronyms, this template provides a function to reference
glossary terms in the text. To use the function, you need to define the
glossary terms in the \texttt{\ glossary\ } attribute of the template.
The glossary terms referenced with the function below will be linked to
their definition in the list of glossary terms.

\subsubsection{Reference}\label{reference}

\texttt{\ gls\ } : Reference a glossary term in the text (e.g.
\texttt{\ gls("Vulnerability")\ } -\textgreater{} link to the definition
of “Vulnerability� in the glossary)

\subsubsection{Definition}\label{definition-1}

The definition works analogously to the acronyms. Define the glossary
terms in a dictionary and pass it to the glossary attribute of the
template. The dictionary should contain the glossary terms as keys and
their definitions as values.

\begin{Shaded}
\begin{Highlighting}[]
\NormalTok{\#let glossary = (}
\NormalTok{  Vulnerability: "A Vulnerability is a flaw in a computer system that weakens the overall security of the system.",}
\NormalTok{  Patch: "A patch is data that is intended to be used to modify an existing software resource such as a program or a file, often to fix bugs and security vulnerabilities.",}
\NormalTok{  Exploit: "An exploit is a method or piece of code that takes advantage of vulnerabilities in software, applications, networks, operating systems, or hardware, typically for malicious purposes.",}
\NormalTok{)}
\end{Highlighting}
\end{Shaded}

\href{/app?template=supercharged-dhbw&version=3.3.2}{Create project in
app}

\subsubsection{How to use}\label{how-to-use}

Click the button above to create a new project using this template in
the Typst app.

You can also use the Typst CLI to start a new project on your computer
using this command:

\begin{verbatim}
typst init @preview/supercharged-dhbw:3.3.2
\end{verbatim}

\includesvg[width=0.16667in,height=0.16667in]{/assets/icons/16-copy.svg}

\subsubsection{About}\label{about}

\begin{description}
\tightlist
\item[Author :]
\href{https://github.com/DannySeidel}{Danny Seidel}
\item[License:]
MIT
\item[Current version:]
3.3.2
\item[Last updated:]
November 4, 2024
\item[First released:]
May 14, 2024
\item[Archive size:]
26.9 kB
\href{https://packages.typst.org/preview/supercharged-dhbw-3.3.2.tar.gz}{\pandocbounded{\includesvg[keepaspectratio]{/assets/icons/16-download.svg}}}
\item[Repository:]
\href{https://github.com/DannySeidel/typst-dhbw-template}{GitHub}
\item[Categor ies :]
\begin{itemize}
\tightlist
\item[]
\item
  \pandocbounded{\includesvg[keepaspectratio]{/assets/icons/16-atom.svg}}
  \href{https://typst.app/universe/search/?category=paper}{Paper}
\item
  \pandocbounded{\includesvg[keepaspectratio]{/assets/icons/16-mortarboard.svg}}
  \href{https://typst.app/universe/search/?category=thesis}{Thesis}
\item
  \pandocbounded{\includesvg[keepaspectratio]{/assets/icons/16-speak.svg}}
  \href{https://typst.app/universe/search/?category=report}{Report}
\end{itemize}
\end{description}

\subsubsection{Where to report issues?}\label{where-to-report-issues}

This template is a project of Danny Seidel . Report issues on
\href{https://github.com/DannySeidel/typst-dhbw-template}{their
repository} . You can also try to ask for help with this template on the
\href{https://forum.typst.app}{Forum} .

Please report this template to the Typst team using the
\href{https://typst.app/contact}{contact form} if you believe it is a
safety hazard or infringes upon your rights.

\phantomsection\label{versions}
\subsubsection{Version history}\label{version-history}

\begin{longtable}[]{@{}ll@{}}
\toprule\noalign{}
Version & Release Date \\
\midrule\noalign{}
\endhead
\bottomrule\noalign{}
\endlastfoot
3.3.2 & November 4, 2024 \\
\href{https://typst.app/universe/package/supercharged-dhbw/3.3.1/}{3.3.1}
& October 3, 2024 \\
\href{https://typst.app/universe/package/supercharged-dhbw/3.3.0/}{3.3.0}
& September 22, 2024 \\
\href{https://typst.app/universe/package/supercharged-dhbw/3.2.0/}{3.2.0}
& September 17, 2024 \\
\href{https://typst.app/universe/package/supercharged-dhbw/3.1.1/}{3.1.1}
& August 26, 2024 \\
\href{https://typst.app/universe/package/supercharged-dhbw/3.1.0/}{3.1.0}
& August 21, 2024 \\
\href{https://typst.app/universe/package/supercharged-dhbw/3.0.0/}{3.0.0}
& August 8, 2024 \\
\href{https://typst.app/universe/package/supercharged-dhbw/2.2.0/}{2.2.0}
& July 29, 2024 \\
\href{https://typst.app/universe/package/supercharged-dhbw/2.1.0/}{2.1.0}
& July 19, 2024 \\
\href{https://typst.app/universe/package/supercharged-dhbw/2.0.2/}{2.0.2}
& July 4, 2024 \\
\href{https://typst.app/universe/package/supercharged-dhbw/2.0.1/}{2.0.1}
& July 4, 2024 \\
\href{https://typst.app/universe/package/supercharged-dhbw/2.0.0/}{2.0.0}
& July 2, 2024 \\
\href{https://typst.app/universe/package/supercharged-dhbw/1.5.0/}{1.5.0}
& June 24, 2024 \\
\href{https://typst.app/universe/package/supercharged-dhbw/1.4.0/}{1.4.0}
& June 10, 2024 \\
\href{https://typst.app/universe/package/supercharged-dhbw/1.3.1/}{1.3.1}
& May 27, 2024 \\
\href{https://typst.app/universe/package/supercharged-dhbw/1.3.0/}{1.3.0}
& May 23, 2024 \\
\href{https://typst.app/universe/package/supercharged-dhbw/1.2.0/}{1.2.0}
& May 16, 2024 \\
\href{https://typst.app/universe/package/supercharged-dhbw/1.1.0/}{1.1.0}
& May 16, 2024 \\
\href{https://typst.app/universe/package/supercharged-dhbw/1.0.0/}{1.0.0}
& May 14, 2024 \\
\end{longtable}

Typst GmbH did not create this template and cannot guarantee correct
functionality of this template or compatibility with any version of the
Typst compiler or app.


\section{Package List LaTeX/numberingx.tex}
\title{typst.app/universe/package/numberingx}

\phantomsection\label{banner}
\section{numberingx}\label{numberingx}

{ 0.0.1 }

Extended numbering patterns using the CSS Counter Styles spec

\phantomsection\label{readme}
\emph{Extended numbering patterns using the
\href{https://www.w3.org/TR/css-counter-styles-3/}{CSS Counter Styles}
specification, along with a number of
\href{https://www.w3.org/TR/predefined-counter-styles/}{Ready-made
Counter Styles} .}

\subsection{Usage}\label{usage}

\begin{Shaded}
\begin{Highlighting}[]
\CommentTok{// numberingx is expected to be imported with the syntax creating a named module}
\NormalTok{\#}\ImportTok{import} \StringTok{"@preview/numberingx:0.0.1"}

\CommentTok{// Use full{-}width roman numerals for titles, and lowercase ukrainian letters}
\NormalTok{\#set }\FunctionTok{heading}\NormalTok{(numbering}\OperatorTok{:}\NormalTok{ numberingx}\OperatorTok{.}\FunctionTok{formatter}\NormalTok{(}
  \StringTok{"\{fullwidth{-}upper{-}roman\}.\{fullwidth{-}lower{-}roman\}.\{lower{-}ukrainian\}"}
\NormalTok{))}
\end{Highlighting}
\end{Shaded}

\subsubsection{Patterns}\label{patterns}

numberingx’s patterns are similiar to typst’s
\href{https://typst.app/docs/reference/meta/numbering/}{numbering
patterns} and use the same notion of fragments with a prefix and a final
suffix. The main difference is that it doesn’t use special characters
and all numbering styles must be written within braces. To insert a
literal brace, you can double it.

A list of patterns can be found in the
\href{https://www.w3.org/TR/predefined-counter-styles/}{Ready-made
Counter Styles} document. Additionally, numberingx allows typst’s
numbering characters to be used in patterns. This way,
\texttt{\ "\{upper-roman\}.\{decimal\})"\ } can be shortened to
\texttt{\ "\{I\}.\{1\})"\ } .

\subsubsection{API}\label{api}

numberingx exposes two functions, \texttt{\ format\ } and
\texttt{\ formatter\ } .

\paragraph{\texorpdfstring{\texttt{\ format(fmt,\ styles:\ (:),\ ..nums)\ }}{ format(fmt, styles: (:), ..nums) }}\label{formatfmt-styles-..nums}

This function uses the same api as typst’s \texttt{\ numbering()\ }
and takes the pattern string as its first positional argument, and
numbers as trailing arguments. An optional \texttt{\ styles\ } argument
allows for
\href{https://github.com/typst/packages/raw/main/packages/preview/numberingx/0.0.1/\#user-defined-styles}{user-defined
styles} .

\paragraph{\texorpdfstring{\texttt{\ formatter(fmt,\ styles:\ (:))\ }}{ formatter(fmt, styles: (:)) }}\label{formatterfmt-styles}

This function is little more than a shorter version of
\texttt{\ format.with(..)\ } . It takes a pattern string and an optional
\texttt{\ styles\ } argument, and return the matching numbering
functions. This is mainly intended to be used for \texttt{\ \#set\ }
rules.

\subsection{User-defined styles}\label{user-defined-styles}

Custom styles can be defined according to the
\href{https://www.w3.org/TR/css-counter-styles-3/}{CSS Counter Styles}
spec and passed through a \texttt{\ styles\ } named argument to
\texttt{\ format\ } and \texttt{\ formatter\ } . It must be a dictionary
mapping style names to style descriptions.

Note that the \texttt{\ prefix\ } , \texttt{\ suffix\ } ,
\texttt{\ pad\ } , and \texttt{\ speak-as\ } descriptors are not
supported, nor is the \texttt{\ extends\ } system.

\subsection{License}\label{license}

This repository is licensed under
\href{https://spdx.org/licenses/MIT-0.html}{MIT-0} , which is the
closest I’m legally allowed to public domain while being OSI approved.

\subsubsection{How to add}\label{how-to-add}

Copy this into your project and use the import as
\texttt{\ numberingx\ }

\begin{verbatim}
#import "@preview/numberingx:0.0.1"
\end{verbatim}

\includesvg[width=0.16667in,height=0.16667in]{/assets/icons/16-copy.svg}

Check the docs for
\href{https://typst.app/docs/reference/scripting/\#packages}{more
information on how to import packages} .

\subsubsection{About}\label{about}

\begin{description}
\tightlist
\item[Author :]
Edhebi
\item[License:]
MIT-0
\item[Current version:]
0.0.1
\item[Last updated:]
July 21, 2023
\item[First released:]
July 21, 2023
\item[Archive size:]
13.9 kB
\href{https://packages.typst.org/preview/numberingx-0.0.1.tar.gz}{\pandocbounded{\includesvg[keepaspectratio]{/assets/icons/16-download.svg}}}
\item[Repository:]
\href{https://github.com/edhebi/numberingx}{GitHub}
\end{description}

\subsubsection{Where to report issues?}\label{where-to-report-issues}

This package is a project of Edhebi . Report issues on
\href{https://github.com/edhebi/numberingx}{their repository} . You can
also try to ask for help with this package on the
\href{https://forum.typst.app}{Forum} .

Please report this package to the Typst team using the
\href{https://typst.app/contact}{contact form} if you believe it is a
safety hazard or infringes upon your rights.

\phantomsection\label{versions}
\subsubsection{Version history}\label{version-history}

\begin{longtable}[]{@{}ll@{}}
\toprule\noalign{}
Version & Release Date \\
\midrule\noalign{}
\endhead
\bottomrule\noalign{}
\endlastfoot
0.0.1 & July 21, 2023 \\
\end{longtable}

Typst GmbH did not create this package and cannot guarantee correct
functionality of this package or compatibility with any version of the
Typst compiler or app.


\section{Package List LaTeX/gloss-awe.tex}
\title{typst.app/universe/package/gloss-awe}

\phantomsection\label{banner}
\section{gloss-awe}\label{gloss-awe}

{ 0.0.5 }

Awesome Glossary for Typst.

\phantomsection\label{readme}
Automatically create a glossary in \href{https://typst.app/}{typst} .

This typst component creates a glossary page from a given pool of
potential glossary entries using only those entries, that are marked
with the \texttt{\ gls\ } or \texttt{\ gls-add\ } functions in the
document.

âš~ï¸? Typst is in beta and evolving, and this package evolves with it.
As a result, no backward compatibility is guaranteed yet. Also, the
package itself is under development and fine-tuning.

\subsection{Table of Contents}\label{table-of-contents}

\begin{itemize}
\tightlist
\item
  \href{https://github.com/typst/packages/raw/main/packages/preview/gloss-awe/0.0.5/\#usage}{Usage}

  \begin{itemize}
  \tightlist
  \item
    \href{https://github.com/typst/packages/raw/main/packages/preview/gloss-awe/0.0.5/\#marking-the-entries}{Marking
    the Entries}
  \item
    \href{https://github.com/typst/packages/raw/main/packages/preview/gloss-awe/0.0.5/\#controlling-the-show}{Controlling
    the Show}
  \item
    \href{https://github.com/typst/packages/raw/main/packages/preview/gloss-awe/0.0.5/\#hiding-entries-from-the-glossary-page}{Hiding
    Entries from the Glossary Page}
  \item
    \href{https://github.com/typst/packages/raw/main/packages/preview/gloss-awe/0.0.5/\#pool-of-entries}{Pool
    of Entries}
  \item
    \href{https://github.com/typst/packages/raw/main/packages/preview/gloss-awe/0.0.5/\#unknown-entries}{Unknown
    Entries}
  \item
    \href{https://github.com/typst/packages/raw/main/packages/preview/gloss-awe/0.0.5/\#creating-the-glossary-page}{Creating
    the glossary page}
  \end{itemize}
\item
  \href{https://github.com/typst/packages/raw/main/packages/preview/gloss-awe/0.0.5/\#changelog}{Changelog}

  \begin{itemize}
  \tightlist
  \item
    \href{https://github.com/typst/packages/raw/main/packages/preview/gloss-awe/0.0.5/\#v005}{v0.0.5}
  \item
    \href{https://github.com/typst/packages/raw/main/packages/preview/gloss-awe/0.0.5/\#v004}{v0.0.4}
  \item
    \href{https://github.com/typst/packages/raw/main/packages/preview/gloss-awe/0.0.5/\#v003}{v0.0.3}
  \item
    \href{https://github.com/typst/packages/raw/main/packages/preview/gloss-awe/0.0.5/\#v002}{v0.0.2}
  \end{itemize}
\end{itemize}

\subsection{Usage}\label{usage}

\subsubsection{Marking the Entries}\label{marking-the-entries}

To include a term into the glossary, it can be marked with the
\texttt{\ gls\ } function. The simplest form is like this:

\begin{Shaded}
\begin{Highlighting}[]
\NormalTok{This is how to mark something for the glossary in \#gls[Typst].}
\end{Highlighting}
\end{Shaded}

The function will render as defined with the specified show rule (see
below!).

\subsubsection{Controlling the Show}\label{controlling-the-show}

To control, how the markers are rendered in the document, a show rule
has to be defined. It usually looks like the following:

\begin{Shaded}
\begin{Highlighting}[]
\NormalTok{// marker display : this rule makes the glossary marker in the document visible.}
\NormalTok{\#show figure.where(kind: "jkrb\_glossary"): it =\textgreater{} \{it.body\}}
\end{Highlighting}
\end{Shaded}

\subsubsection{Hiding Entries from the Glossary
Page}\label{hiding-entries-from-the-glossary-page}

It is also possible to hide entries (temporarily) from the generated
glossary page without removing any markers for them from the document.

The following sample will hide the entries for “Amaranth� and
“Butterscotch� from the glossary, even if it is marked with
\texttt{\ gls{[}...{]}\ } or \texttt{\ gls-add{[}...{]}\ } somewhere in
the document.

\begin{Shaded}
\begin{Highlighting}[]
\NormalTok{    \#let hidden{-}entries = (}
\NormalTok{        "Amaranth",}
\NormalTok{        "Butterscotch"}
\NormalTok{    )}

\NormalTok{    \#make{-}glossary(glossary{-}pool, excluded: hidden{-}entries)}
\end{Highlighting}
\end{Shaded}

\subsubsection{Pool of Entries}\label{pool-of-entries}

A “pool of entries� is a typst dictionary. It can be read from a
file or may be the result of other operations.

One or more pool(s) are then given to the \texttt{\ make-glossary()\ }
function. This will create a glossary page of all referenced entries. If
given more than one pool, the order pools are searched in the order they
are given to the method. The first match wins. This can be used to have
a global pool to be used in different documents, and another one for
local usage only.

The pool consists of a dictionary of definition entries. The key of an
entry is the term. Note, that it is case-sensitive. Each entry itself is
also a dictionary, and its main key is \texttt{\ description\ } . This
is the content for the term. There may be other keys in an entry in the
future. For now, it supports:

\begin{itemize}
\tightlist
\item
  description
\item
  link
\end{itemize}

An entry in the pool for the term “Engine� file may look like this:

\begin{Shaded}
\begin{Highlighting}[]
\NormalTok{    Engine: (}
\NormalTok{        description: [}

\NormalTok{            In the context of software, an engine...}

\NormalTok{        ],}
\NormalTok{        link: [}

\NormalTok{            (1) Software engine {-} Wikipedia.}
\NormalTok{            https://en.wikipedia.org/wiki/Software\_engine}
\NormalTok{            (13.5.2023).}

\NormalTok{        ]}
\NormalTok{    ),}
\end{Highlighting}
\end{Shaded}

\subsubsection{Unknown Entries}\label{unknown-entries}

If the document marks a term that is not contained in the pool, an entry
will be generated anyway, but it will be visually marked as missing.
This is convenient for the workflow, where one can mark the desired
entries while writing the document, and provide missing entries later.

There is a parameter for the \texttt{\ make-glossary()\ } function named
\texttt{\ missing\ } , where a function can be provided to format or
even suppress the missing entries.

\subsubsection{Creating the Glossary
Page}\label{creating-the-glossary-page}

To create the glossary page, provide the pool of potential entries to
the make-glossary function. The following listing shows a complete
sample document with a glossary. The sample glossary pool is contained
in the main document as well:

\begin{Shaded}
\begin{Highlighting}[]
\NormalTok{    \#import "@preview/gloss{-}awe:0.0.5": *}

\NormalTok{    // Text settings}
\NormalTok{    \#set text(font: ("Arial", "Trebuchet MS"), size: 12pt)}

\NormalTok{    // Defining the Glossary Pool with definitions.}
\NormalTok{    \#let glossary{-}pool = (}
\NormalTok{        Cloud: (}
\NormalTok{            description: [}

\NormalTok{                Cloud computing is a model where computer resources are made available}
\NormalTok{                over the internet. Such resources can be assigned on demand in a very short}
\NormalTok{                time, and only as long as they are required by the user.}

\NormalTok{            ]}
\NormalTok{        ),}

\NormalTok{        Marker: (}
\NormalTok{            description: [}

\NormalTok{                A Marker in \textasciigrave{}gloss{-}awe\textasciigrave{} is a typst function to mark a word or phrase to appear}
\NormalTok{                in the documents glossary. The marker is also linked to the glossary section}
\NormalTok{                by referencing the label \textasciigrave{}\textless{}Glossary\textgreater{}\textasciigrave{}.}

\NormalTok{            ]}
\NormalTok{        ),}

\NormalTok{        Glossary: (}
\NormalTok{            description: [}

\NormalTok{                A glossary is a list of terms and their definitions that are specific to a}
\NormalTok{                particular subject or field. It is used to define the intended meaning of}
\NormalTok{                terms used in a document and to agree on a common definition of those terms. A}
\NormalTok{                well{-}defined glossary can be very helpful in documents where very specific}
\NormalTok{                meanings of certain terms are used.}

\NormalTok{            ]}
\NormalTok{        ),}

\NormalTok{        "Glossary Pool": (}
\NormalTok{            description: [}

\NormalTok{                A glossary pool is a collection of glossary entries. An automated tool can}
\NormalTok{                pull needed definitions from this pool to create the glossary pages for a}
\NormalTok{                specific context.}

\NormalTok{            ]}
\NormalTok{        ),}

\NormalTok{        REST: (}
\NormalTok{            description: [}

\NormalTok{                Representational State Transfer (abgekürzt REST) ist ein Paradigma für die}
\NormalTok{                Softwarearchitektur von verteilten Systemen, insbesondere für Webservices.}
\NormalTok{                REST ist eine Abstraktion der Struktur und des Verhaltens des World Wide}
\NormalTok{                Web. REST hat das Ziel, einen Architekturstil zu schaffen, der den}
\NormalTok{                Anforderungen des modernen Web besser genügt.}

\NormalTok{            ]}
\NormalTok{        ),}

\NormalTok{        XML: (}
\NormalTok{            description: [}

\NormalTok{                XML stands for \textasciigrave{}\textquotesingle{}eXtensible Markup Language\textquotesingle{}\textasciigrave{}.}

\NormalTok{            ],}
\NormalTok{            link: [https://www.w3.org/XML]}
\NormalTok{        ),}
\NormalTok{    )}

\NormalTok{    // Defining, how marked glossary entries in the document appear}
\NormalTok{    \#show figure.where(kind: "jkrb\_glossary"): it =\textgreater{} \{it.body\}}

\NormalTok{    // This alternate rule, creates links to the glossary for marked entries.}
\NormalTok{    // \#show figure.where(kind: "jkrb\_glossary"): it =\textgreater{} [\#link(\textless{}Glossar\textgreater{})[\#it.body]]}

\NormalTok{    = My Sample Document with \textasciigrave{}gloss{-}awe\textasciigrave{}}

\NormalTok{    In this document the usage of the \textasciigrave{}gloss{-}awe\textasciigrave{} package is demonstrated to create a glossary}
\NormalTok{    with the help of a simple and small sample glossary pool. We have defined the pool in a}
\NormalTok{    dictionary named \#gls[Glossary Pool] above. It contains the definitions the \textasciigrave{}gloss{-}awe\textasciigrave{}}
\NormalTok{    can use to build the glossary in the \#gls[Glossary] section of this document. The pool}
\NormalTok{    could also come from external files, like \#gls[JSON] or \#gls[XML] or other sources. Only}
\NormalTok{    those definitions are shown in the glossary, that are marked in this document with one of}
\NormalTok{    the \#gls(entry: "Marker")[marker] functions \textasciigrave{}gloss{-}awe\textasciigrave{} provides.}

\NormalTok{    If a Word is marked, that is not in the Glossary Pool, it gets a special markup in the}
\NormalTok{    resulting glossary, making it easy for the Author to spot the missing information an}
\NormalTok{    providing a definition. In this sample, there is no definition for "JSON" provided,}
\NormalTok{    resulting in an Entry with a red remark "\#text(fill: red)[No\textasciitilde{}glossary\textasciitilde{}entry]".}

\NormalTok{    There is also a way to include Entries in the glossary for Words that are not contained in}
\NormalTok{    the documents text:}

\NormalTok{    \#gls{-}add("Cloud")}
\NormalTok{    \#gls{-}add("REST")}


\NormalTok{    = Glossary \textless{}Glossary\textgreater{}}

\NormalTok{    This section contains the generated Glossary, in a nice two{-}column{-}layout. It also bears a}
\NormalTok{    label, to enable the linking from marked words to the glossary.}

\NormalTok{    \#line(length: 100\%)}
\NormalTok{    \#set text(font: ("Arial", "Trebuchet MS"), size: 10pt)}
\NormalTok{    \#columns(2)[}
\NormalTok{        \#make{-}glossary(glossary{-}pool)}
\NormalTok{    ]}
\end{Highlighting}
\end{Shaded}

More usage samples are shown in the document
\texttt{\ sample-usage.typ\ } on
\href{https://github.com/typst/packages/raw/main/packages/preview/gloss-awe/0.0.5/\%5BTitle\%5D(https://github.com/RolfBremer/typst-glossary)}{gloss-awe´s
GitHub} .

A more complex sample PDF is available there as well.

\subsection{Changelog}\label{changelog}

\subsubsection{v0.0.5}\label{v0.0.5}

\begin{itemize}
\tightlist
\item
  Address change in \texttt{\ figure.caption\ } in typst (commit:
  976abdf ).
\end{itemize}

\subsubsection{v0.0.4}\label{v0.0.4}

\begin{itemize}
\tightlist
\item
  Breaking: Renamed the main file from \texttt{\ glossary.typ\ } to
  \texttt{\ gloss-awe.typ\ } to match package.
\item
  Added support for hidden glossary entries.
\item
  Added a Changelog to this readme.
\end{itemize}

\subsubsection{v0.0.3}\label{v0.0.3}

\begin{itemize}
\tightlist
\item
  Added support for package manager in Typst.
\item
  Add \texttt{\ gls-add{[}...{]}\ } function for entries that are not in
  the document.
\end{itemize}

\subsubsection{v.0.0.2}\label{v.0.0.2}

\begin{itemize}
\tightlist
\item
  Moved version to Github.
\end{itemize}

\subsubsection{How to add}\label{how-to-add}

Copy this into your project and use the import as \texttt{\ gloss-awe\ }

\begin{verbatim}
#import "@preview/gloss-awe:0.0.5"
\end{verbatim}

\includesvg[width=0.16667in,height=0.16667in]{/assets/icons/16-copy.svg}

Check the docs for
\href{https://typst.app/docs/reference/scripting/\#packages}{more
information on how to import packages} .

\subsubsection{About}\label{about}

\begin{description}
\tightlist
\item[Author :]
\href{https://github.com/RolfBremer}{JKRB}
\item[License:]
Apache-2.0
\item[Current version:]
0.0.5
\item[Last updated:]
September 13, 2023
\item[First released:]
July 3, 2023
\item[Archive size:]
8.39 kB
\href{https://packages.typst.org/preview/gloss-awe-0.0.5.tar.gz}{\pandocbounded{\includesvg[keepaspectratio]{/assets/icons/16-download.svg}}}
\item[Repository:]
\href{https://github.com/RolfBremer/gloss-awe}{GitHub}
\end{description}

\subsubsection{Where to report issues?}\label{where-to-report-issues}

This package is a project of JKRB . Report issues on
\href{https://github.com/RolfBremer/gloss-awe}{their repository} . You
can also try to ask for help with this package on the
\href{https://forum.typst.app}{Forum} .

Please report this package to the Typst team using the
\href{https://typst.app/contact}{contact form} if you believe it is a
safety hazard or infringes upon your rights.

\phantomsection\label{versions}
\subsubsection{Version history}\label{version-history}

\begin{longtable}[]{@{}ll@{}}
\toprule\noalign{}
Version & Release Date \\
\midrule\noalign{}
\endhead
\bottomrule\noalign{}
\endlastfoot
0.0.5 & September 13, 2023 \\
\href{https://typst.app/universe/package/gloss-awe/0.0.4/}{0.0.4} & July
6, 2023 \\
\href{https://typst.app/universe/package/gloss-awe/0.0.3/}{0.0.3} & July
3, 2023 \\
\end{longtable}

Typst GmbH did not create this package and cannot guarantee correct
functionality of this package or compatibility with any version of the
Typst compiler or app.


\section{Package List LaTeX/rich-counters.tex}
\title{typst.app/universe/package/rich-counters}

\phantomsection\label{banner}
\section{rich-counters}\label{rich-counters}

{ 0.2.2 }

Counters which can inherit from other counters.

\phantomsection\label{readme}
This package allows you to have \textbf{counters which can inherit from
other counters} .

Concretely, it implements \texttt{\ rich-counter\ } , which is a counter
that can \emph{inherit} one or more levels from another counter.

The interface is pretty much the same as the
\href{https://typst.app/docs/reference/introspection/counter/}{usual
counter} . It provides a \texttt{\ display()\ } , \texttt{\ get()\ } ,
\texttt{\ final()\ } , \texttt{\ at()\ } , and a \texttt{\ step()\ }
method. An \texttt{\ update()\ } method will be implemented soon.

\subsection{Simple typical Showcase}\label{simple-typical-showcase}

In the following example, \texttt{\ mycounter\ } inherits the first
level from \texttt{\ heading\ } (but not deeper levels).

\begin{Shaded}
\begin{Highlighting}[]
\NormalTok{\#import "@preview/rich{-}counters:0.2.2": *}

\NormalTok{\#set heading(numbering: "1.1")}
\NormalTok{\#let mycounter = rich{-}counter(identifier: "mycounter", inherited\_levels: 1)}

\NormalTok{// DOCUMENT}

\NormalTok{Displaying \textasciigrave{}mycounter\textasciigrave{} here: \#context (mycounter.display)()}

\NormalTok{= First level heading}

\NormalTok{Displaying \textasciigrave{}mycounter\textasciigrave{} here: \#context (mycounter.display)()}

\NormalTok{Stepping \textasciigrave{}mycounter\textasciigrave{} here. \#(mycounter.step)()}

\NormalTok{Displaying \textasciigrave{}mycounter\textasciigrave{} here: \#context (mycounter.display)()}

\NormalTok{= Another first level heading}

\NormalTok{Displaying \textasciigrave{}mycounter\textasciigrave{} here: \#context (mycounter.display)()}

\NormalTok{Stepping \textasciigrave{}mycounter\textasciigrave{} here. \#(mycounter.step)()}

\NormalTok{Displaying \textasciigrave{}mycounter\textasciigrave{} here: \#context (mycounter.display)()}

\NormalTok{== Second level heading}

\NormalTok{Displaying \textasciigrave{}mycounter\textasciigrave{} here: \#context (mycounter.display)()}

\NormalTok{Stepping \textasciigrave{}mycounter\textasciigrave{} here. \#(mycounter.step)()}

\NormalTok{Displaying \textasciigrave{}mycounter\textasciigrave{} here: \#context (mycounter.display)()}

\NormalTok{= Aaand another first level heading}

\NormalTok{Displaying \textasciigrave{}mycounter\textasciigrave{} here: \#context (mycounter.display)()}

\NormalTok{Stepping \textasciigrave{}mycounter\textasciigrave{} here. \#(mycounter.step)()}

\NormalTok{Displaying \textasciigrave{}mycounter\textasciigrave{} here: \#context (mycounter.display)()}
\end{Highlighting}
\end{Shaded}

\pandocbounded{\includegraphics[keepaspectratio]{https://github.com/typst/packages/raw/main/packages/preview/rich-counters/0.2.2/example.png}}

\subsection{\texorpdfstring{Construction of a
\texttt{\ rich-counter\ }}{Construction of a  rich-counter }}\label{construction-of-a-rich-counter}

To create a \texttt{\ rich-counter\ } , you have to call the
\texttt{\ rich-counter(...)\ } function. It accepts three arguments:

\begin{itemize}
\item
  \texttt{\ identifier\ } (required)

  Must be a unique \texttt{\ string\ } which identifies the counter.
\item
  \texttt{\ inherited\_levels\ }

  Specifies how many levels should be inherited from the parent counter.
\item
  \texttt{\ inherited\_from\ } (Default: \texttt{\ heading\ } )

  Specifies the parent counter. Can be a \texttt{\ rich-counter\ } or
  any key that is accepted by the
  \href{https://typst.app/docs/reference/introspection/counter\#constructor}{\texttt{\ counter(...)\ }
  constructor} , such as a \texttt{\ label\ } , a \texttt{\ selector\ }
  , a \texttt{\ location\ } , or a \texttt{\ function\ } like
  \texttt{\ heading\ } . If not specified, defaults to
  \texttt{\ heading\ } (and hence it will inherit from the counter of
  the headings).

  If it’s a \texttt{\ rich-counter\ } and
  \texttt{\ inherited\_levels\ } is \emph{not} specified, then
  \texttt{\ inherited\_levels\ } will default to one level higher than
  the given \texttt{\ rich-counter\ } .
\end{itemize}

For example, the following creates a \texttt{\ rich-counter\ }
\texttt{\ foo\ } which inherits one level from the headings, and then
another \texttt{\ rich-counter\ } \texttt{\ bar\ } which inherits two
levels (implicitly) from \texttt{\ foo\ } .

\begin{Shaded}
\begin{Highlighting}[]
\NormalTok{\#import "@preview/rich{-}counters:0.2.2": *}

\NormalTok{\#let foo = rich{-}counter(identifier: "foo", inherited\_levels: 1)}
\NormalTok{\#let bar = rich{-}counter(identifier: "bar", inherited\_from: foo)}
\end{Highlighting}
\end{Shaded}

\subsection{\texorpdfstring{Usage of a
\texttt{\ rich-counter\ }}{Usage of a  rich-counter }}\label{usage-of-a-rich-counter}

\begin{itemize}
\item
  \texttt{\ display(numbering)\ } (needs \texttt{\ context\ } )

  Displays the current value of the counter with the given numbering
  style. Giving the numbering style is optional, with default value
  \texttt{\ "1.1"\ } .
\item
  \texttt{\ get()\ } (needs \texttt{\ context\ } )

  Returns the current value of the counter (as an \texttt{\ array\ } ).
\item
  \texttt{\ final()\ } (needs \texttt{\ context\ } )

  Returns the value of the counter at the end of the document.
\item
  \texttt{\ at(loc)\ } (needs \texttt{\ context\ } )

  Returns the value of the counter at \texttt{\ loc\ } , where
  \texttt{\ loc\ } can be a \texttt{\ label\ } , \texttt{\ selector\ } ,
  \texttt{\ location\ } , or \texttt{\ function\ } .
\item
  \texttt{\ step(depth:\ 1)\ }

  Steps the counter at the specified \texttt{\ depth\ } (default:
  \texttt{\ 1\ } ). That is, it steps the \texttt{\ rich-counter\ } at
  level \texttt{\ inherited\_levels\ +\ depth\ } .
\end{itemize}

\textbf{Due to a Typst limitation, you have to put parentheses before
you put the arguments. (See below.)}

For example, the following steps \texttt{\ mycounter\ } (at depth 1) and
then displays it.

\begin{Shaded}
\begin{Highlighting}[]
\NormalTok{\#import "@preview/rich{-}counters:0.2.2": *}
\NormalTok{\#let mycounter = rich{-}counter(...)}

\NormalTok{\#(mycounter.step)()}
\NormalTok{\#context (mycounter.display)("1.1")}
\end{Highlighting}
\end{Shaded}

\subsection{Limitations}\label{limitations}

Due to current Typst limitations, there is no way to detect manual
updates or steps of Typst-native counters, like
\texttt{\ counter(heading).update(...)\ } or
\texttt{\ counter(heading).step(...)\ } . Only occurrences of actual
\texttt{\ heading\ } s can be detected. So make sure that after you call
e.g. \texttt{\ counter(heading).update(...)\ } , you place a heading
directly after it, before you call any \texttt{\ rich-counter\ } s.

\subsection{Roadmap}\label{roadmap}

\begin{itemize}
\tightlist
\item
  implement \texttt{\ update()\ }
\item
  use Typst custom types as soon as they become available
\item
  adopt native Typst implementation of dependent counters as soon it
  becomes available
\end{itemize}

\subsubsection{How to add}\label{how-to-add}

Copy this into your project and use the import as
\texttt{\ rich-counters\ }

\begin{verbatim}
#import "@preview/rich-counters:0.2.2"
\end{verbatim}

\includesvg[width=0.16667in,height=0.16667in]{/assets/icons/16-copy.svg}

Check the docs for
\href{https://typst.app/docs/reference/scripting/\#packages}{more
information on how to import packages} .

\subsubsection{About}\label{about}

\begin{description}
\tightlist
\item[Author :]
\href{https://jbirnick.net}{Johann Birnick}
\item[License:]
MIT
\item[Current version:]
0.2.2
\item[Last updated:]
November 21, 2024
\item[First released:]
August 14, 2024
\item[Archive size:]
3.60 kB
\href{https://packages.typst.org/preview/rich-counters-0.2.2.tar.gz}{\pandocbounded{\includesvg[keepaspectratio]{/assets/icons/16-download.svg}}}
\item[Repository:]
\href{https://github.com/jbirnick/typst-rich-counters}{GitHub}
\item[Categor ies :]
\begin{itemize}
\tightlist
\item[]
\item
  \pandocbounded{\includesvg[keepaspectratio]{/assets/icons/16-list-unordered.svg}}
  \href{https://typst.app/universe/search/?category=model}{Model}
\item
  \pandocbounded{\includesvg[keepaspectratio]{/assets/icons/16-code.svg}}
  \href{https://typst.app/universe/search/?category=scripting}{Scripting}
\item
  \pandocbounded{\includesvg[keepaspectratio]{/assets/icons/16-hammer.svg}}
  \href{https://typst.app/universe/search/?category=utility}{Utility}
\end{itemize}
\end{description}

\subsubsection{Where to report issues?}\label{where-to-report-issues}

This package is a project of Johann Birnick . Report issues on
\href{https://github.com/jbirnick/typst-rich-counters}{their repository}
. You can also try to ask for help with this package on the
\href{https://forum.typst.app}{Forum} .

Please report this package to the Typst team using the
\href{https://typst.app/contact}{contact form} if you believe it is a
safety hazard or infringes upon your rights.

\phantomsection\label{versions}
\subsubsection{Version history}\label{version-history}

\begin{longtable}[]{@{}ll@{}}
\toprule\noalign{}
Version & Release Date \\
\midrule\noalign{}
\endhead
\bottomrule\noalign{}
\endlastfoot
0.2.2 & November 21, 2024 \\
\href{https://typst.app/universe/package/rich-counters/0.2.1/}{0.2.1} &
October 16, 2024 \\
\href{https://typst.app/universe/package/rich-counters/0.2.0/}{0.2.0} &
October 14, 2024 \\
\href{https://typst.app/universe/package/rich-counters/0.1.0/}{0.1.0} &
August 14, 2024 \\
\end{longtable}

Typst GmbH did not create this package and cannot guarantee correct
functionality of this package or compatibility with any version of the
Typst compiler or app.


\section{Package List LaTeX/nassi.tex}
\title{typst.app/universe/package/nassi}

\phantomsection\label{banner}
\section{nassi}\label{nassi}

{ 0.1.2 }

Draw Nassi-Shneiderman diagrams (Struktogramme) with Typst.

\phantomsection\label{readme}
\textbf{nassi} is a package for \href{https://typst.app/}{Typst} to draw
\href{https://en.wikipedia.org/wiki/Nassi\%E2\%80\%93Shneiderman_diagram}{Nassi-Shneiderman
diagrams} (Struktogramme).

\pandocbounded{\includegraphics[keepaspectratio]{https://github.com/typst/packages/raw/main/packages/preview/nassi/0.1.2/assets/example-1.png}}

\subsection{Usage}\label{usage}

Import \textbf{nassi} in your document:

\begin{Shaded}
\begin{Highlighting}[]
\NormalTok{\#import "@preview/nassi:0.1.2"}
\end{Highlighting}
\end{Shaded}

There are several options to draw diagrams. One is to parse all
code-blocks with the language “nassi�. Simply add a show-rule like
this:

\begin{Shaded}
\begin{Highlighting}[]
\NormalTok{\#import "@preview/nassi:0.1.2"}
\NormalTok{\#show: nassi.shneiderman()}

\NormalTok{\textasciigrave{}\textasciigrave{}\textasciigrave{}nassi}
\NormalTok{function ggt(a, b)}
\NormalTok{  while a \textgreater{} 0 and b \textgreater{} 0}
\NormalTok{    if a \textgreater{} b}
\NormalTok{      a \textless{}{-} a {-} b}
\NormalTok{    else}
\NormalTok{      b \textless{}{-} b {-} a}
\NormalTok{    endif}
\NormalTok{  endwhile}
\NormalTok{  if b == 0}
\NormalTok{    return a}
\NormalTok{  else}
\NormalTok{    return b}
\NormalTok{  endif}
\NormalTok{endfunction}
\NormalTok{\textasciigrave{}\textasciigrave{}\textasciigrave{}}
\end{Highlighting}
\end{Shaded}

In this case, the diagram is created from a simple pseudocode. To have
more control over the output, you can add blocks manually using the
element functions provided in \texttt{\ nassi.elements\ } :

\begin{Shaded}
\begin{Highlighting}[]
\NormalTok{\#import "@preview/nassi:0.1.2"}

\NormalTok{\#nassi.diagram(\{}
\NormalTok{    import nassi.elements: *}

\NormalTok{    function("ggt(a, b)", \{}
\NormalTok{        loop("a \textgreater{} b and b \textgreater{} 0", \{}
\NormalTok{            branch("a \textgreater{} b", \{}
\NormalTok{                assign("a", "a {-} b")}
\NormalTok{            \}, \{}
\NormalTok{                assign("b", "b {-} a",}
\NormalTok{                    fill: gradient.linear(..color.map.rainbow),}
\NormalTok{                    stroke:red + 2pt}
\NormalTok{                )}
\NormalTok{            \})}
\NormalTok{        \})}
\NormalTok{        branch("b == 0", \{ process("return a") \}, \{ process("return b") \})}
\NormalTok{    \})}
\NormalTok{\})}
\end{Highlighting}
\end{Shaded}

\pandocbounded{\includegraphics[keepaspectratio]{https://github.com/typst/packages/raw/main/packages/preview/nassi/0.1.2/assets/example-3.png}}

Since \textbf{nassi} uses \textbf{cetz} for drawing, you can add
diagrams directly to a canvas. Each block gets a name within the diagram
group to reference it in the drawing:

\begin{Shaded}
\begin{Highlighting}[]
\NormalTok{\#import "@preview/cetz:0.2.2"}
\NormalTok{\#import "@preview/nassi:0.1.2"}

\NormalTok{\#cetz.canvas(\{}
\NormalTok{  import nassi.draw: diagram}
\NormalTok{  import nassi.elements: *}
\NormalTok{  import cetz.draw: *}

\NormalTok{  diagram((4,4), \{}
\NormalTok{    function("ggt(a, b)", \{}
\NormalTok{      loop("a \textgreater{} b and b \textgreater{} 0", \{}
\NormalTok{        branch("a \textgreater{} b", \{}
\NormalTok{          assign("a", "a {-} b")}
\NormalTok{        \}, \{}
\NormalTok{          assign("b", "b {-} a")}
\NormalTok{        \})}
\NormalTok{      \})}
\NormalTok{      branch("b == 0", \{ process("return a") \}, \{ process("return b") \})}
\NormalTok{    \})}
\NormalTok{  \})}

\NormalTok{  for i in range(8) \{}
\NormalTok{    content(}
\NormalTok{      "nassi.e" + str(i+1) + ".north{-}west",}
\NormalTok{      stroke:red,}
\NormalTok{      fill:red.transparentize(50\%),}
\NormalTok{      frame:"circle",}
\NormalTok{      padding:.05,}
\NormalTok{      anchor:"north{-}west",}
\NormalTok{      text(white, weight:"bold", "e"+str(i)),}
\NormalTok{    )}
\NormalTok{  \}}
\NormalTok{\})}
\end{Highlighting}
\end{Shaded}

\pandocbounded{\includegraphics[keepaspectratio]{https://github.com/typst/packages/raw/main/packages/preview/nassi/0.1.2/assets/example-cetz-2.png}}

This can be useful to annotate a diagram:

\pandocbounded{\includegraphics[keepaspectratio]{https://github.com/typst/packages/raw/main/packages/preview/nassi/0.1.2/assets/example-cetz.png}}

See \texttt{\ assets/\ } for usage examples.

\subsection{Changelog}\label{changelog}

\subsubsection{Version 0.1.2}\label{version-0.1.2}

\begin{itemize}
\tightlist
\item
  Fix for deprecation warnings in Typst 0.12.
\end{itemize}

\subsubsection{Version 0.1.1}\label{version-0.1.1}

\begin{itemize}
\tightlist
\item
  Fixed labels option not working for branches in other elements.
\item
  Added \texttt{\ switch\ } statements (thanks to @Geronymos).
\end{itemize}

\subsubsection{Version 0.1.0}\label{version-0.1.0}

Initial release of \textbf{nassi} .

\subsubsection{How to add}\label{how-to-add}

Copy this into your project and use the import as \texttt{\ nassi\ }

\begin{verbatim}
#import "@preview/nassi:0.1.2"
\end{verbatim}

\includesvg[width=0.16667in,height=0.16667in]{/assets/icons/16-copy.svg}

Check the docs for
\href{https://typst.app/docs/reference/scripting/\#packages}{more
information on how to import packages} .

\subsubsection{About}\label{about}

\begin{description}
\tightlist
\item[Author :]
Jonas Neugebauer
\item[License:]
MIT
\item[Current version:]
0.1.2
\item[Last updated:]
October 23, 2024
\item[First released:]
June 3, 2024
\item[Minimum Typst version:]
0.11.0
\item[Archive size:]
5.93 kB
\href{https://packages.typst.org/preview/nassi-0.1.2.tar.gz}{\pandocbounded{\includesvg[keepaspectratio]{/assets/icons/16-download.svg}}}
\item[Repository:]
\href{https://github.com/jneug/typst-nassi}{GitHub}
\item[Discipline :]
\begin{itemize}
\tightlist
\item[]
\item
  \href{https://typst.app/universe/search/?discipline=computer-science}{Computer
  Science}
\end{itemize}
\item[Categor y :]
\begin{itemize}
\tightlist
\item[]
\item
  \pandocbounded{\includesvg[keepaspectratio]{/assets/icons/16-chart.svg}}
  \href{https://typst.app/universe/search/?category=visualization}{Visualization}
\end{itemize}
\end{description}

\subsubsection{Where to report issues?}\label{where-to-report-issues}

This package is a project of Jonas Neugebauer . Report issues on
\href{https://github.com/jneug/typst-nassi}{their repository} . You can
also try to ask for help with this package on the
\href{https://forum.typst.app}{Forum} .

Please report this package to the Typst team using the
\href{https://typst.app/contact}{contact form} if you believe it is a
safety hazard or infringes upon your rights.

\phantomsection\label{versions}
\subsubsection{Version history}\label{version-history}

\begin{longtable}[]{@{}ll@{}}
\toprule\noalign{}
Version & Release Date \\
\midrule\noalign{}
\endhead
\bottomrule\noalign{}
\endlastfoot
0.1.2 & October 23, 2024 \\
\href{https://typst.app/universe/package/nassi/0.1.1/}{0.1.1} & October
2, 2024 \\
\href{https://typst.app/universe/package/nassi/0.1.0/}{0.1.0} & June 3,
2024 \\
\end{longtable}

Typst GmbH did not create this package and cannot guarantee correct
functionality of this package or compatibility with any version of the
Typst compiler or app.


\section{Package List LaTeX/in-dexter.tex}
\title{typst.app/universe/package/in-dexter}

\phantomsection\label{banner}
\section{in-dexter}\label{in-dexter}

{ 0.5.3 }

Hand Picked Index for Typst.

\phantomsection\label{readme}
Automatically create a handcrafted index in
\href{https://typst.app/}{typst} . This typst component allows the
automatic creation of an Index page with entries that have been manually
marked in the document by its authors. This, in times of advanced search
functionality, seems somewhat outdated, but a handcrafted index like
this allows the authors to point the reader to just the right location
in the document.

âš~ï¸? Typst is in beta and evolving, and this package evolves with it.
As a result, no backward compatibility is guaranteed yet. Also, the
package itself is under development and fine-tuning.

\subsection{Table of Contents}\label{table-of-contents}

\begin{itemize}
\tightlist
\item
  \href{https://github.com/typst/packages/raw/main/packages/preview/in-dexter/0.5.3/\#usage}{Usage}

  \begin{itemize}
  \tightlist
  \item
    \href{https://github.com/typst/packages/raw/main/packages/preview/in-dexter/0.5.3/\#importing-the-component}{Importing
    the Component}
  \item
    \href{https://github.com/typst/packages/raw/main/packages/preview/in-dexter/0.5.3/\#remarks-for-new-version}{Remarks
    for new version}
  \item
    \href{https://github.com/typst/packages/raw/main/packages/preview/in-dexter/0.5.3/\#marking-entries}{Marking
    Entries}

    \begin{itemize}
    \tightlist
    \item
      \href{https://github.com/typst/packages/raw/main/packages/preview/in-dexter/0.5.3/\#generating-the-index-page}{Generating
      the index page}
    \item
      \href{https://github.com/typst/packages/raw/main/packages/preview/in-dexter/0.5.3/\#brief-sample-document}{Brief
      Sample Document}
    \item
      \href{https://github.com/typst/packages/raw/main/packages/preview/in-dexter/0.5.3/\#full-sample-document}{Full
      Sample Document}
    \end{itemize}
  \end{itemize}
\item
  \href{https://github.com/typst/packages/raw/main/packages/preview/in-dexter/0.5.3/\#changelog}{Changelog}

  \begin{itemize}
  \tightlist
  \item
    \href{https://github.com/typst/packages/raw/main/packages/preview/in-dexter/0.5.3/\#v053}{v0.5.3}
  \item
    \href{https://github.com/typst/packages/raw/main/packages/preview/in-dexter/0.5.3/\#v052}{v0.5.2}
  \item
    \href{https://github.com/typst/packages/raw/main/packages/preview/in-dexter/0.5.3/\#v051}{v0.5.1}
  \item
    \href{https://github.com/typst/packages/raw/main/packages/preview/in-dexter/0.5.3/\#v050}{v0.5.0}
  \item
    \href{https://github.com/typst/packages/raw/main/packages/preview/in-dexter/0.5.3/\#v043}{v0.4.3}
  \item
    \href{https://github.com/typst/packages/raw/main/packages/preview/in-dexter/0.5.3/\#v042}{v0.4.2}
  \item
    \href{https://github.com/typst/packages/raw/main/packages/preview/in-dexter/0.5.3/\#v041}{v0.4.1}
  \item
    \href{https://github.com/typst/packages/raw/main/packages/preview/in-dexter/0.5.3/\#v040}{v0.4.0}
  \item
    \href{https://github.com/typst/packages/raw/main/packages/preview/in-dexter/0.5.3/\#v032}{v0.3.2}
  \item
    \href{https://github.com/typst/packages/raw/main/packages/preview/in-dexter/0.5.3/\#v031}{v0.3.1}
  \item
    \href{https://github.com/typst/packages/raw/main/packages/preview/in-dexter/0.5.3/\#v030}{v0.3.0}
  \item
    \href{https://github.com/typst/packages/raw/main/packages/preview/in-dexter/0.5.3/\#v020}{v0.2.0}
  \item
    \href{https://github.com/typst/packages/raw/main/packages/preview/in-dexter/0.5.3/\#v010}{v0.1.0}
  \item
    \href{https://github.com/typst/packages/raw/main/packages/preview/in-dexter/0.5.3/\#v006}{v0.0.6}
  \item
    \href{https://github.com/typst/packages/raw/main/packages/preview/in-dexter/0.5.3/\#v005}{v0.0.5}
  \item
    \href{https://github.com/typst/packages/raw/main/packages/preview/in-dexter/0.5.3/\#v004}{v0.0.4}
  \item
    \href{https://github.com/typst/packages/raw/main/packages/preview/in-dexter/0.5.3/\#v003}{v0.0.3}
  \item
    \href{https://github.com/typst/packages/raw/main/packages/preview/in-dexter/0.5.3/\#v002}{v0.0.2}
  \end{itemize}
\end{itemize}

\subsection{Usage}\label{usage}

\subsection{Importing the Component}\label{importing-the-component}

To use the index functionality, the component must be available. This
can be achieved by importing the package \texttt{\ in-dexter\ } into the
project:

Add the following code to the head of the document file(s) that want to
use the index:

\begin{Shaded}
\begin{Highlighting}[]
\NormalTok{  \#import "@preview/in{-}dexter:0.5.3": *}
\end{Highlighting}
\end{Shaded}

Alternatively it can be loaded from the file, if you have it copied into
your project.

\begin{Shaded}
\begin{Highlighting}[]
\NormalTok{  \#import "in{-}dexter.typ": *}
\end{Highlighting}
\end{Shaded}

\subsection{Remarks for new version}\label{remarks-for-new-version}

In previous versions (before 0.0.6) of in-dexter, it was required to
hide the index entries with a show rule. This is not required anymore.

\subsection{Marking Entries}\label{marking-entries}

To mark a word to be included in the index, a simple function can be
used. In the following sample code, the word “elit� is marked to be
included into the index.

\begin{Shaded}
\begin{Highlighting}[]
\NormalTok{= Sample Text}
\NormalTok{Lorem ipsum dolor sit amet, consectetur adipiscing \#index[elit], sed do eiusmod tempor}
\NormalTok{incididunt ut labore et dolore.}
\end{Highlighting}
\end{Shaded}

Nested entries can be created - the following would create an entry
\texttt{\ adipiscing\ } with sub entry \texttt{\ elit\ } .

\begin{Shaded}
\begin{Highlighting}[]
\NormalTok{= Sample Text}
\NormalTok{Lorem ipsum dolor sit amet, consectetur adipiscing elit\#index("adipiscing", "elit"), sed do eiusmod}
\NormalTok{tempor incididunt ut labore et dolore.}
\end{Highlighting}
\end{Shaded}

The marking, by default, is invisible in the resulting text, while the
marked word will still be visible. With the marking in place, the index
component knows about the word, as well as its location in the document.

\subsection{Generating the Index Page}\label{generating-the-index-page}

The index page can be generated by the following function:

\begin{Shaded}
\begin{Highlighting}[]
\NormalTok{= Index}
\NormalTok{\#columns(3)[}
\NormalTok{  \#make{-}index(title: none)}
\NormalTok{]}
\end{Highlighting}
\end{Shaded}

This sample uses the optional title, outline, and use-page-counter
parameters:

\begin{Shaded}
\begin{Highlighting}[]
\NormalTok{\#make{-}index(title: [Index], outlined: true, use{-}page{-}counter: true)}
\end{Highlighting}
\end{Shaded}

The \texttt{\ make-index()\ } function takes three optional arguments:
\texttt{\ title\ } , \texttt{\ outlined\ } , and
\texttt{\ use-page-counter\ } .

\begin{itemize}
\tightlist
\item
  \texttt{\ title\ } adds a title (with \texttt{\ heading\ } ) and
\item
  \texttt{\ outlined\ } is \texttt{\ false\ } by default and is passed
  to the heading function
\item
  \texttt{\ use-page-counter\ } is \texttt{\ false\ } by default. If set
  to \texttt{\ true\ } it will use \texttt{\ counter(page).display()\ }
  for the page number text in the index instead of the absolute page
  position (the absolute position is still used for the actual link
  target)
\end{itemize}

If no title is given the heading should never appear in the layout.
Note: The heading is (currently) not numbered.

The first sample emits the index in three columns. Note: The actual
appearance depends on your template or other settings of your document.

You can find a preview image of the resulting page on
\href{https://github.com/RolfBremer/in-dexter}{in-dexter´s GitHub
repository} .

You may have noticed that some page numbers are displayed as bold. These
are index entries which are marked as “main� entries. Such entries
are meant to be the most important for the given entry. They can be
marked as follows:

\begin{Shaded}
\begin{Highlighting}[]
\NormalTok{\#index(fmt: strong, [Willkommen])}
\end{Highlighting}
\end{Shaded}

or you can use the predefined semantically helper function

\begin{Shaded}
\begin{Highlighting}[]
\NormalTok{\#index{-}main[Willkommen]}
\end{Highlighting}
\end{Shaded}

\subsubsection{Brief Sample Document}\label{brief-sample-document}

This is a very brief sample to demonstrate how in-dexter can be used.
The next chapter contains a more fleshed out sample.

\begin{Shaded}
\begin{Highlighting}[]
\NormalTok{\#import "@preview/in{-}dexter:0.5.3": *}


\NormalTok{= My Sample Document with \textasciigrave{}in{-}dexter\textasciigrave{}}

\NormalTok{In this document the usage of the \textasciigrave{}in{-}dexter\textasciigrave{} package is demonstrated to create}
\NormalTok{a hand picked \#index[Hand Picked] index. This sample \#index{-}main[Sample]}
\NormalTok{document \#index[Document] is quite short, and so is its index.}


\NormalTok{= Index}

\NormalTok{This section contains the generated Index.}

\NormalTok{\#make{-}index()}
\end{Highlighting}
\end{Shaded}

\subsubsection{Full Sample Document}\label{full-sample-document}

\begin{Shaded}
\begin{Highlighting}[]
\NormalTok{\#import "@preview/in{-}dexter:0.5.3": *}

\NormalTok{\#let index{-}main(..args) = index(fmt: strong, ..args)}

\NormalTok{// Document settings}
\NormalTok{\#set page("a5")}
\NormalTok{\#set text(font: ("Arial", "Trebuchet MS"), size: 12pt)}


\NormalTok{= My Sample Document with \textasciigrave{}in{-}dexter\textasciigrave{}}

\NormalTok{In this document \#index[Document] the usage of the \textasciigrave{}in{-}dexter\textasciigrave{} package \#index[Package]}
\NormalTok{is demonstrated to create a hand picked index \#index{-}main[Index]. This sample document}
\NormalTok{is quite short, and so is its index. So to fill this sample with some real text,}
\NormalTok{let´s elaborate on some aspects of a hand picked \#index[Hand Picked] index. So, "hand}
\NormalTok{picked" means, the entries \#index[Entries] in the index are carefully chosen by the}
\NormalTok{author(s) of the document to point the reader, who is interested in a specific topic}
\NormalTok{within the documents domain \#index[Domain], to the right spot \#index[Spot]. Thats, how}
\NormalTok{it should be; and it is quite different to what is done in this sample text, where the}
\NormalTok{objective \#index{-}main[Objective] was to put many different index markers}
\NormalTok{\#index[Markers] into a small text, because a sample should be as brief as possible,}
\NormalTok{while providing enough substance \#index[Substance] to demo the subject}
\NormalTok{\#index[Subject]. The resulting index in this demo is somewhat pointless}
\NormalTok{\#index[Pointless], because all entries are pointing to few different pages}
\NormalTok{\#index[Pages], due to the fact that the demo text only has few pages \#index[Page].}
\NormalTok{That is also the reason for what we chose the DIN A5 \#index[DIN A5] format, and we}
\NormalTok{also continue with some remarks \#index[Remarks] on the next page.}


\NormalTok{== Some more demo content without deeper meaning}

\NormalTok{\#lorem(50) \#index[Lorem]}

\NormalTok{\#pagebreak()}

\NormalTok{== Remarks}

\NormalTok{Here are some more remarks \#index{-}main[Remarks] to have some content on a second page, what}
\NormalTok{is a precondition \#index[Precondition] to demo that Index \#index[Index] entries}
\NormalTok{\#index[Entries] may point to multiple pages.}


\NormalTok{= Index}

\NormalTok{This section \#index[Section] contains the generated Index \#index[Index], in a nice}
\NormalTok{two{-}column{-}layout.}

\NormalTok{\#set text(size: 10pt)}
\NormalTok{\#columns(2)[}
\NormalTok{    \#make{-}index()}
\NormalTok{]}
\end{Highlighting}
\end{Shaded}

The following image shows a generated index page of another document,
with additional formatting for headers applied.

\pandocbounded{\includegraphics[keepaspectratio]{https://github.com/typst/packages/raw/main/packages/preview/in-dexter/0.5.3/gallery/SampleIndex.png}}

More usage samples are shown in the document
\texttt{\ sample-usage.typ\ } on
\href{https://github.com/RolfBremer/in-dexter}{in-dexter´s GitHub} .

A more complex sample PDF is available there as well.

\subsection{Changelog}\label{changelog}

\subsubsection{v0.5.3}\label{v0.5.3}

\begin{itemize}
\tightlist
\item
  fix error in typst.toml file.
\item
  Add a sample for raw display.
\end{itemize}

\subsubsection{v0.5.2}\label{v0.5.2}

\begin{itemize}
\tightlist
\item
  Fix a bug with bang notation.
\item
  Add compiler to toml file.
\end{itemize}

\subsubsection{v0.5.1}\label{v0.5.1}

\begin{itemize}
\tightlist
\item
  Migrate deprecated locate to context.
\end{itemize}

\subsubsection{v0.5.0}\label{v0.5.0}

\begin{itemize}
\tightlist
\item
  Support page numbering formats (i.e. roman), when
  \texttt{\ use-page-counter\ } is set to true. Thanks to
  @ThePuzzlemaker!
\end{itemize}

\subsubsection{v0.4.3}\label{v0.4.3}

\begin{itemize}
\tightlist
\item
  Suppress extra space character emitted by the \texttt{\ index()\ }
  function.
\item
  Fix a bug where math formulas are not displayed.
\item
  Introduce \texttt{\ apply-casing\ } parameter to \texttt{\ index()\ }
  to suppress entry-casing for individual entries.
\end{itemize}

\subsubsection{v0.4.2}\label{v0.4.2}

\begin{itemize}
\tightlist
\item
  Improve internal method \texttt{\ as-text\ } to be more robust.
\item
  tidy up sample-usage.typ.
\end{itemize}

\subsubsection{v0.4.1}\label{v0.4.1}

\begin{itemize}
\tightlist
\item
  Bug fixed: Fix a bug where an index entry with same name as a group
  hides the group.
\item
  Fixed typos in the sample-usage document.
\end{itemize}

\subsubsection{v0.4.0}\label{v0.4.0}

\begin{itemize}
\tightlist
\item
  Support for a \texttt{\ display\ } parameter for entries. This allows
  the usage of complex content, like math expressions in the index.
  (based on suggestions by @lukasjuhrich)
\item
  Also support a tuple value for display and key parameters of the
  entry.
\item
  Improve internal robustness and fix some errors in the sample
  document.
\end{itemize}

\subsubsection{v0.3.2}\label{v0.3.2}

\begin{itemize}
\tightlist
\item
  Fix initial parsing and returning fist letter (thanks to
  @lukasjuhrich, \#14)
\end{itemize}

\subsubsection{v0.3.1}\label{v0.3.1}

\begin{itemize}
\tightlist
\item
  Fix handling of trailing or multiple spaces and crlf in index entries.
\end{itemize}

\subsubsection{v0.3.0}\label{v0.3.0}

\begin{itemize}
\tightlist
\item
  Support multiple named indexes. Also allow the generation of combined
  index pages.
\item
  Support for LaTeX index group syntax (
  \texttt{\ \#index("Group1!Group2@Entry"\ } ).
\item
  Support for advanced case handling for the entries in the index. Note:
  The new default ist to ignore the casing for the sorting of the
  entries. The behavior can be changed by providing a
  \texttt{\ sort-order()\ } function to the \texttt{\ make-index\ }
  function.
\item
  The casing for the index entry can also be altered by providing a
  \texttt{\ entry-casing()\ } function to the \texttt{\ make-index\ }
  function. So it is possible that all entries have an uppercase first
  letter (which is also the new default!).
\end{itemize}

\subsubsection{v0.2.0}\label{v0.2.0}

\begin{itemize}
\tightlist
\item
  Allow index to respect unnumbered physical pages at the start of the
  document (Thanks to @jewelpit). See “Skipping physical pages� in
  the sample-usage document.
\end{itemize}

\subsubsection{v0.1.0}\label{v0.1.0}

\begin{itemize}
\tightlist
\item
  big refactor (by @epsilonhalbe).
\item
  changing “marker classes� to support direct format function
  \texttt{\ fmt:\ content\ -\textgreater{}\ content\ } e.g.
  \texttt{\ index(fmt:\ strong,\ {[}entry{]})\ } .
\item
  Implemented:

  \begin{itemize}
  \tightlist
  \item
    nested entries.
  \item
    custom initials + custom sorting.
  \end{itemize}
\end{itemize}

\subsubsection{v0.0.6}\label{v0.0.6}

\begin{itemize}
\tightlist
\item
  Change internal index marker to use metadata instead of figures. This
  allows a cleaner implementation and does not require a show rule to
  hide the marker-figure anymore.
\item
  This version requires Typst 0.8.0 due to the use of metadata().
\item
  Consolidated the \texttt{\ PackageReadme.md\ } into a single
  \texttt{\ README.md\ } .
\end{itemize}

\subsubsection{v0.0.5}\label{v0.0.5}

\begin{itemize}
\tightlist
\item
  Address change in \texttt{\ figure.caption\ } in typst (commit:
  976abdf ).
\end{itemize}

\subsubsection{v0.0.4}\label{v0.0.4}

\begin{itemize}
\tightlist
\item
  Add title and outline arguments to \#make-index() by @sbatial in \#4
\end{itemize}

\subsubsection{v0.0.3}\label{v0.0.3}

\begin{itemize}
\tightlist
\item
  Breaking: Renamed the main file from \texttt{\ index.typ\ } to
  \texttt{\ in-dexter.typ\ } to match package.
\item
  Added a Changelog to this README.
\item
  Introduced a brief and a full sample code to this README.
\item
  Added support for package manager in Typst.
\end{itemize}

\subsubsection{v.0.0.2}\label{v.0.0.2}

\begin{itemize}
\tightlist
\item
  Moved version to GitHub.
\end{itemize}

\subsubsection{How to add}\label{how-to-add}

Copy this into your project and use the import as \texttt{\ in-dexter\ }

\begin{verbatim}
#import "@preview/in-dexter:0.5.3"
\end{verbatim}

\includesvg[width=0.16667in,height=0.16667in]{/assets/icons/16-copy.svg}

Check the docs for
\href{https://typst.app/docs/reference/scripting/\#packages}{more
information on how to import packages} .

\subsubsection{About}\label{about}

\begin{description}
\tightlist
\item[Author s :]
\href{https://github.com/RolfBremer}{JKRB} \& in-dexter Contributors
\item[License:]
Apache-2.0
\item[Current version:]
0.5.3
\item[Last updated:]
August 14, 2024
\item[First released:]
July 10, 2023
\item[Minimum Typst version:]
0.11.0
\item[Archive size:]
11.3 kB
\href{https://packages.typst.org/preview/in-dexter-0.5.3.tar.gz}{\pandocbounded{\includesvg[keepaspectratio]{/assets/icons/16-download.svg}}}
\item[Repository:]
\href{https://github.com/RolfBremer/in-dexter}{GitHub}
\item[Categor y :]
\begin{itemize}
\tightlist
\item[]
\item
  \pandocbounded{\includesvg[keepaspectratio]{/assets/icons/16-package.svg}}
  \href{https://typst.app/universe/search/?category=components}{Components}
\end{itemize}
\end{description}

\subsubsection{Where to report issues?}\label{where-to-report-issues}

This package is a project of JKRB and in-dexter Contributors . Report
issues on \href{https://github.com/RolfBremer/in-dexter}{their
repository} . You can also try to ask for help with this package on the
\href{https://forum.typst.app}{Forum} .

Please report this package to the Typst team using the
\href{https://typst.app/contact}{contact form} if you believe it is a
safety hazard or infringes upon your rights.

\phantomsection\label{versions}
\subsubsection{Version history}\label{version-history}

\begin{longtable}[]{@{}ll@{}}
\toprule\noalign{}
Version & Release Date \\
\midrule\noalign{}
\endhead
\bottomrule\noalign{}
\endlastfoot
0.5.3 & August 14, 2024 \\
\href{https://typst.app/universe/package/in-dexter/0.5.2/}{0.5.2} &
August 23, 2024 \\
\href{https://typst.app/universe/package/in-dexter/0.4.2/}{0.4.2} & June
14, 2024 \\
\href{https://typst.app/universe/package/in-dexter/0.3.0/}{0.3.0} & May
13, 2024 \\
\href{https://typst.app/universe/package/in-dexter/0.2.0/}{0.2.0} &
April 30, 2024 \\
\href{https://typst.app/universe/package/in-dexter/0.1.0/}{0.1.0} &
January 8, 2024 \\
\href{https://typst.app/universe/package/in-dexter/0.0.6/}{0.0.6} &
October 1, 2023 \\
\href{https://typst.app/universe/package/in-dexter/0.0.5/}{0.0.5} &
September 13, 2023 \\
\href{https://typst.app/universe/package/in-dexter/0.0.4/}{0.0.4} &
August 6, 2023 \\
\href{https://typst.app/universe/package/in-dexter/0.0.3/}{0.0.3} & July
10, 2023 \\
\end{longtable}

Typst GmbH did not create this package and cannot guarantee correct
functionality of this package or compatibility with any version of the
Typst compiler or app.


\section{Package List LaTeX/droplet.tex}
\title{typst.app/universe/package/droplet}

\phantomsection\label{banner}
\section{droplet}\label{droplet}

{ 0.3.1 }

Customizable dropped capitals.

\phantomsection\label{readme}
A package for creating dropped capitals.

\subsection{Usage}\label{usage}

This package exports a single \texttt{\ dropcap\ } function that is used
to create dropped capitals. The function takes one or two positional
arguments, and several optional keyword arguments for customization:

\begin{longtable}[]{@{}llll@{}}
\toprule\noalign{}
Parameter & Type & Description & Default \\
\midrule\noalign{}
\endhead
\bottomrule\noalign{}
\endlastfoot
\texttt{\ height\ } & \texttt{\ integer\ } , \texttt{\ length\ } ,
\texttt{\ auto\ } & The height of the dropped capital. &
\texttt{\ 2\ } \\
\texttt{\ justify\ } & \texttt{\ boolean\ } , \texttt{\ auto\ } &
Whether the text should be justified. & \texttt{\ auto\ } \\
\texttt{\ gap\ } & \texttt{\ length\ } & The space between the dropped
capital and the text. & \texttt{\ 0pt\ } \\
\texttt{\ hanging-indent\ } & \texttt{\ length\ } & The indent of lines
after the first. & \texttt{\ 0pt\ } \\
\texttt{\ overhang\ } & \texttt{\ length\ } , \texttt{\ relative\ } ,
\texttt{\ ratio\ } & How much the dropped capital should hang into the
margin. & \texttt{\ 0pt\ } \\
\texttt{\ depth\ } & \texttt{\ integer\ } , \texttt{\ length\ } & The
space below the dropped capital. & \texttt{\ 0pt\ } \\
\texttt{\ transform\ } & \texttt{\ function\ } , \texttt{\ none\ } & A
function to be applied to the dropped capital. & \texttt{\ none\ } \\
\texttt{\ ..text-args\ } & & How to style the \texttt{\ text\ } of the
dropped capital. & \\
\end{longtable}

Some parameters allow values of different types for maximum flexibility:

\begin{itemize}
\tightlist
\item
  If \texttt{\ height\ } is given as an integer, it is interpreted as a
  number of lines. If given as \texttt{\ auto\ } , the dropped capital
  will not be scaled and remain at its original size.
\item
  If \texttt{\ overhang\ } has a relative part, it is resolved relative
  to the width of the dropped capital.
\item
  If \texttt{\ depth\ } is given as an integer, it is interpreted as a
  number of lines.
\item
  The \texttt{\ transform\ } function takes the extracted or passed
  dropped capital and returns the content to be shown.
\end{itemize}

If two positional arguments are given, the first is used as the dropped
capital, and the second as the text. If only one argument is given, the
dropped capital is automatically extracted from the text.

\subsubsection{Extraction}\label{extraction}

If no explicit dropped capital is passed, it is extracted automatically.
For this to work, the package looks into the content making up the first
paragraph and extracts the first letter of the first word. This letter
is then split off from the rest of the text and used as the dropped
capital. There are some special cases to consider:

\begin{itemize}
\tightlist
\item
  If the first element of the paragraph is a \texttt{\ box\ } , the
  whole box is used as the dropped capital.
\item
  If the first element is a list or enum item, it is assumed that the
  literal meaning of the list or enum syntax was intended, and the
  number or bullet is used as the dropped capital.
\end{itemize}

Affixes, such as punctuation, super- and subscripts, quotes, and spaces
will also be detected and stay with the dropped capital.

\subsubsection{Paragraph Splitting}\label{paragraph-splitting}

To wrap the text around the dropped capital, the paragraph is split into
two parts: the part next to the dropped capital and the part after it.
As Typst doesn’t natively support wrapping text around an element,
this package splits the paragraph at word boundaries and tries to fit as
much in the first part as possible. This approach comes with some
limitations:

\begin{itemize}
\tightlist
\item
  The paragraph is split at word boundaries, which makes hyphenation
  across the split impossible.
\item
  Some elements cannot be properly split, such as containers, lists, and
  context expressions.
\item
  The approach uses a greedy algorithm, which might not always find the
  optimal split.
\item
  If the split happens at a block element, the spacing between the two
  parts might be off.
\end{itemize}

To determine whether an elements fits into the first part, the position
of top edge of the element is crucial. If the top edge is above the
baseline of the dropped capital, the element is considered to be part of
the first part. This means that elements with a large height will be
part of the first part. This is done to avoid gaps between the two parts
of the paragraph.

\subsubsection{Styling}\label{styling}

In case you wish to style the dropped capital more than what is possible
with the arguments of the \texttt{\ text\ } function, you can use a
\texttt{\ transform\ } function. This function takes the extracted or
passed dropped capital and returns the content to be shown. The function
is provided with the context of the dropped capital.

Note that when using \texttt{\ em\ } units, they are resolved relative
to the font size of the dropped capital. When the dropped capital is
scaled to fit the given \texttt{\ height\ } parameter, the font size is
adjusted so that the \emph{bounds} of the transformed content match the
given height. For that, the \texttt{\ top-edge\ } and
\texttt{\ bottom-edge\ } parameters of \texttt{\ text-args\ } are set to
\texttt{\ bounds\ } by default.

\subsection{Example}\label{example}

\begin{Shaded}
\begin{Highlighting}[]
\NormalTok{\#import "@preview/droplet:0.3.1": dropcap}

\NormalTok{\#set par(justify: true)}

\NormalTok{\#dropcap(}
\NormalTok{  height: 3,}
\NormalTok{  gap: 4pt,}
\NormalTok{  hanging{-}indent: 1em,}
\NormalTok{  overhang: 8pt,}
\NormalTok{  font: "Curlz MT",}
\NormalTok{)[}
\NormalTok{  *Typst* is a new markup{-}based typesetting system that is designed to be as}
\NormalTok{  \_powerful\_ as LaTeX while being \_much easier\_ to learn and use. Typst has:}

\NormalTok{  {-} Built{-}in markup for the most common formatting tasks}
\NormalTok{  {-} Flexible functions for everything else}
\NormalTok{  {-} A tightly integrated scripting system}
\NormalTok{  {-} Math typesetting, bibliography management, and more}
\NormalTok{  {-} Fast compile times thanks to incremental compilation}
\NormalTok{  {-} Friendly error messages in case something goes wrong}
\NormalTok{]}
\end{Highlighting}
\end{Shaded}

\pandocbounded{\includesvg[keepaspectratio]{https://github.com/typst/packages/raw/main/packages/preview/droplet/0.3.1/assets/example.svg}}

\subsubsection{How to add}\label{how-to-add}

Copy this into your project and use the import as \texttt{\ droplet\ }

\begin{verbatim}
#import "@preview/droplet:0.3.1"
\end{verbatim}

\includesvg[width=0.16667in,height=0.16667in]{/assets/icons/16-copy.svg}

Check the docs for
\href{https://typst.app/docs/reference/scripting/\#packages}{more
information on how to import packages} .

\subsubsection{About}\label{about}

\begin{description}
\tightlist
\item[Author :]
Eric Biedert
\item[License:]
MIT
\item[Current version:]
0.3.1
\item[Last updated:]
November 21, 2024
\item[First released:]
July 5, 2024
\item[Minimum Typst version:]
0.11.0
\item[Archive size:]
7.82 kB
\href{https://packages.typst.org/preview/droplet-0.3.1.tar.gz}{\pandocbounded{\includesvg[keepaspectratio]{/assets/icons/16-download.svg}}}
\item[Repository:]
\href{https://github.com/EpicEricEE/typst-droplet}{GitHub}
\item[Categor y :]
\begin{itemize}
\tightlist
\item[]
\item
  \pandocbounded{\includesvg[keepaspectratio]{/assets/icons/16-text.svg}}
  \href{https://typst.app/universe/search/?category=text}{Text}
\end{itemize}
\end{description}

\subsubsection{Where to report issues?}\label{where-to-report-issues}

This package is a project of Eric Biedert . Report issues on
\href{https://github.com/EpicEricEE/typst-droplet}{their repository} .
You can also try to ask for help with this package on the
\href{https://forum.typst.app}{Forum} .

Please report this package to the Typst team using the
\href{https://typst.app/contact}{contact form} if you believe it is a
safety hazard or infringes upon your rights.

\phantomsection\label{versions}
\subsubsection{Version history}\label{version-history}

\begin{longtable}[]{@{}ll@{}}
\toprule\noalign{}
Version & Release Date \\
\midrule\noalign{}
\endhead
\bottomrule\noalign{}
\endlastfoot
0.3.1 & November 21, 2024 \\
\href{https://typst.app/universe/package/droplet/0.3.0/}{0.3.0} &
October 24, 2024 \\
\href{https://typst.app/universe/package/droplet/0.2.0/}{0.2.0} & July
5, 2024 \\
\href{https://typst.app/universe/package/droplet/0.1.0/}{0.1.0} & July
5, 2024 \\
\end{longtable}

Typst GmbH did not create this package and cannot guarantee correct
functionality of this package or compatibility with any version of the
Typst compiler or app.


\section{Package List LaTeX/vartable.tex}
\title{typst.app/universe/package/vartable}

\phantomsection\label{banner}
\section{vartable}\label{vartable}

{ 0.1.2 }

A simple package to make variation table

\phantomsection\label{readme}
An easy way to render variation table on typst, built on
\href{https://github.com/Jollywatt/typst-fletcher}{fletcher}\\
The
\href{https://github.com/Le-foucheur/Typst-VarTable/blob/main/documentation.pdf}{documention}

\begin{Shaded}
\begin{Highlighting}[]
\NormalTok{\#import "@preview/Tabvar:0.1.0": tabvar}
\end{Highlighting}
\end{Shaded}

\subsubsection{Trigonometric functions}\label{trigonometric-functions}

Turn this :

\begin{Shaded}
\begin{Highlighting}[]
\NormalTok{\#import }\StringTok{"@preview/Tabvar:0.1.0"}\OperatorTok{:}\NormalTok{ tabvar}

\NormalTok{\#}\FunctionTok{tabvar}\NormalTok{(}
\NormalTok{  init}\OperatorTok{:}\NormalTok{ (}
\NormalTok{    variable}\OperatorTok{:}\NormalTok{ $x$}\OperatorTok{,}
\NormalTok{    label}\OperatorTok{:}\NormalTok{ (}
\NormalTok{      ([sign }\KeywordTok{of}\NormalTok{ cos]}\OperatorTok{,} \StringTok{"Sign"}\NormalTok{)}\OperatorTok{,}
\NormalTok{      ([variation }\KeywordTok{of}\NormalTok{ cos]}\OperatorTok{,} \StringTok{"Variation"}\NormalTok{)}\OperatorTok{,}
\NormalTok{      ([sign }\KeywordTok{of}\NormalTok{ sin]}\OperatorTok{,} \StringTok{"Sign"}\NormalTok{)}\OperatorTok{,}
\NormalTok{      ([variation }\KeywordTok{of}\NormalTok{ sin]}\OperatorTok{,} \StringTok{"Variation"}\NormalTok{)}\OperatorTok{,}
\NormalTok{    )}\OperatorTok{,}
\NormalTok{  )}\OperatorTok{,}
\NormalTok{  domain}\OperatorTok{:}\NormalTok{ ($0$}\OperatorTok{,}\NormalTok{ $ pi }\OperatorTok{/} \DecValTok{2}\NormalTok{ $}\OperatorTok{,}\NormalTok{ $ pi $}\OperatorTok{,} \FunctionTok{$}\NormalTok{ (}\DecValTok{2}\ErrorTok{pi}\NormalTok{) }\OperatorTok{/} \DecValTok{3}\NormalTok{ $}\OperatorTok{,}\NormalTok{ $ }\DecValTok{2}\NormalTok{ pi $)}\OperatorTok{,}
\NormalTok{  content}\OperatorTok{:}\NormalTok{ (}
\NormalTok{    ($}\OperatorTok{{-}}\NormalTok{$}\OperatorTok{,}\NormalTok{ ()}\OperatorTok{,}\NormalTok{ $}\OperatorTok{+}\NormalTok{$}\OperatorTok{,}\NormalTok{ ())}\OperatorTok{,}
\NormalTok{    (}
\NormalTok{      (top}\OperatorTok{,}\NormalTok{ $1$)}\OperatorTok{,}
\NormalTok{      ()}\OperatorTok{,}
\NormalTok{      (bottom}\OperatorTok{,}\NormalTok{ $}\OperatorTok{{-}}\DecValTok{1}\ErrorTok{$}\NormalTok{)}\OperatorTok{,}
\NormalTok{      ()}\OperatorTok{,}
\NormalTok{      (top}\OperatorTok{,}\NormalTok{ $1$)}\OperatorTok{,}
\NormalTok{    )}\OperatorTok{,}
\NormalTok{    ($}\OperatorTok{+}\NormalTok{$}\OperatorTok{,}\NormalTok{ $}\OperatorTok{{-}}\NormalTok{$}\OperatorTok{,}\NormalTok{ ()}\OperatorTok{,}\NormalTok{ $}\OperatorTok{+}\NormalTok{$)}\OperatorTok{,}
\NormalTok{    (}
\NormalTok{      (center}\OperatorTok{,}\NormalTok{ $0$)}\OperatorTok{,}
\NormalTok{      (top}\OperatorTok{,}\NormalTok{ $1$)}\OperatorTok{,}
\NormalTok{      ()}\OperatorTok{,}
\NormalTok{      (bottom}\OperatorTok{,}\NormalTok{ $}\OperatorTok{{-}}\DecValTok{1}\ErrorTok{$}\NormalTok{)}\OperatorTok{,}
\NormalTok{      (top}\OperatorTok{,}\NormalTok{ $1$)}\OperatorTok{,}
\NormalTok{    )}\OperatorTok{,}
\NormalTok{  )}\OperatorTok{,}
\NormalTok{)}
\end{Highlighting}
\end{Shaded}

Into this

\pandocbounded{\includegraphics[keepaspectratio]{https://github.com/typst/packages/raw/main/packages/preview/vartable/0.1.2/examples/trigonometricFunction.png}}

\subsubsection{hyperbolic function \$f(x) = 1/x
\$}\label{hyperbolic-function-fx-1x}

\begin{Shaded}
\begin{Highlighting}[]
\NormalTok{\#import }\StringTok{"@preview/Tabvar:0.1.0"}\OperatorTok{:}\NormalTok{ tabvar}

\NormalTok{\#}\FunctionTok{tabvar}\NormalTok{(}
\NormalTok{    init}\OperatorTok{:}\NormalTok{ (}
\NormalTok{        variable}\OperatorTok{:}\NormalTok{ $x$}\OperatorTok{,}
\NormalTok{    label}\OperatorTok{:}\NormalTok{ (}
\NormalTok{        ([sign }\KeywordTok{of}\NormalTok{ $f$]}\OperatorTok{,} \StringTok{"Sign"}\NormalTok{)}\OperatorTok{,}
\NormalTok{      ([variation }\KeywordTok{of}\NormalTok{ $f$]}\OperatorTok{,} \StringTok{"Variation"}\NormalTok{)}\OperatorTok{,}
\NormalTok{    )}\OperatorTok{,}
\NormalTok{  )}\OperatorTok{,}
\NormalTok{  domain}\OperatorTok{:}\NormalTok{ ($ }\OperatorTok{{-}}\NormalTok{oo $}\OperatorTok{,}\NormalTok{ $ }\DecValTok{0}\NormalTok{ $}\OperatorTok{,}\NormalTok{ $ }\OperatorTok{+}\NormalTok{oo $)}\OperatorTok{,}
\NormalTok{  content}\OperatorTok{:}\NormalTok{ (}
\NormalTok{      ($}\OperatorTok{+}\NormalTok{$}\OperatorTok{,}\NormalTok{ (}\StringTok{"||"}\OperatorTok{,}\NormalTok{ $}\OperatorTok{+}\NormalTok{$))}\OperatorTok{,}
\NormalTok{    (}
\NormalTok{        (center}\OperatorTok{,}\NormalTok{ $0$)}\OperatorTok{,}
\NormalTok{      (bottom}\OperatorTok{,}\NormalTok{ top}\OperatorTok{,} \StringTok{"||"}\OperatorTok{,}\NormalTok{ $}\OperatorTok{{-}}\NormalTok{oo$}\OperatorTok{,}\NormalTok{ $}\OperatorTok{+}\NormalTok{oo$)}\OperatorTok{,}
\NormalTok{      (center}\OperatorTok{,}\NormalTok{ $0$)}\OperatorTok{,}
\NormalTok{    )}\OperatorTok{,}
\NormalTok{  )}\OperatorTok{,}
\NormalTok{)}
\end{Highlighting}
\end{Shaded}

\pandocbounded{\includegraphics[keepaspectratio]{https://github.com/typst/packages/raw/main/packages/preview/vartable/0.1.2/examples/hyperbolicFuntion.png}}

\begin{itemize}
\tightlist
\item
  if you put too wide an element for the last value of a variation
  table, this can create a space between the edge of the table and the
  lines separating the lines of the table
\end{itemize}

\pandocbounded{\includegraphics[keepaspectratio]{https://github.com/typst/packages/raw/main/packages/preview/vartable/0.1.2/examples/bug1.png}}

\subsection{·change log·}\label{uxe2change-loguxe2}

\paragraph{0.1.2 :}\label{uxe2}

\begin{itemize}
\tightlist
\item
  Support \texttt{\ fletcher\ 0.5.2\ }
\end{itemize}

\paragraph{0.1.1 :}\label{uxe2-1}

\begin{itemize}
\tightlist
\item
  added customisation of separator bars between signs
\end{itemize}

\subparagraph{0.1.0 :}\label{uxe2-2}

\begin{itemize}
\tightlist
\item
  publishing the package
\end{itemize}

\subsubsection{How to add}\label{how-to-add}

Copy this into your project and use the import as \texttt{\ vartable\ }

\begin{verbatim}
#import "@preview/vartable:0.1.2"
\end{verbatim}

\includesvg[width=0.16667in,height=0.16667in]{/assets/icons/16-copy.svg}

Check the docs for
\href{https://typst.app/docs/reference/scripting/\#packages}{more
information on how to import packages} .

\subsubsection{About}\label{about}

\begin{description}
\tightlist
\item[Author :]
Le\_Foucheur
\item[License:]
MIT
\item[Current version:]
0.1.2
\item[Last updated:]
October 29, 2024
\item[First released:]
July 2, 2024
\item[Archive size:]
114 kB
\href{https://packages.typst.org/preview/vartable-0.1.2.tar.gz}{\pandocbounded{\includesvg[keepaspectratio]{/assets/icons/16-download.svg}}}
\item[Repository:]
\href{https://github.com/Le-foucheur/Typst-VarTable}{GitHub}
\item[Categor y :]
\begin{itemize}
\tightlist
\item[]
\item
  \pandocbounded{\includesvg[keepaspectratio]{/assets/icons/16-chart.svg}}
  \href{https://typst.app/universe/search/?category=visualization}{Visualization}
\end{itemize}
\end{description}

\subsubsection{Where to report issues?}\label{where-to-report-issues}

This package is a project of Le\_Foucheur . Report issues on
\href{https://github.com/Le-foucheur/Typst-VarTable}{their repository} .
You can also try to ask for help with this package on the
\href{https://forum.typst.app}{Forum} .

Please report this package to the Typst team using the
\href{https://typst.app/contact}{contact form} if you believe it is a
safety hazard or infringes upon your rights.

\phantomsection\label{versions}
\subsubsection{Version history}\label{version-history}

\begin{longtable}[]{@{}ll@{}}
\toprule\noalign{}
Version & Release Date \\
\midrule\noalign{}
\endhead
\bottomrule\noalign{}
\endlastfoot
0.1.2 & October 29, 2024 \\
\href{https://typst.app/universe/package/vartable/0.1.1/}{0.1.1} &
October 14, 2024 \\
\href{https://typst.app/universe/package/vartable/0.1.0/}{0.1.0} & July
2, 2024 \\
\end{longtable}

Typst GmbH did not create this package and cannot guarantee correct
functionality of this package or compatibility with any version of the
Typst compiler or app.


\section{Package List LaTeX/pubmatter.tex}
\title{typst.app/universe/package/pubmatter}

\phantomsection\label{banner}
\section{pubmatter}\label{pubmatter}

{ 0.1.0 }

Parse, normalize and show publication frontmatter, including authors and
affiliations

\phantomsection\label{readme}
\emph{Beautiful scientific documents with structured metadata for
publishers}

\href{https://github.com/curvenote/pubmatter/blob/main/docs.pdf}{\pandocbounded{\includesvg[keepaspectratio]{https://img.shields.io/badge/typst-docs-orange.svg}}}
\href{https://github.com/curvenote/pubmatter/blob/main/LICENSE}{\pandocbounded{\includesvg[keepaspectratio]{https://img.shields.io/badge/license-MIT-blue.svg}}}

Pubmatter is a typst library for parsing, normalizing and showing
scientific publication frontmatter.

Utilities for loading, normalizing and working with authors,
affiliations, abstracts, keywords and other frontmatter information
common in scientific publications. Our goal is to introduce standardized
ways of working with this content to expose metadata to scientific
publishers who are interested in using typst in a standardized way. The
specification for this \texttt{\ pubmatter\ } is based on
\href{https://mystmd.org/}{MyST Markdown} and
\href{https://quarto.org/}{Quarto} , and can load their YAML files
directly.

\subsection{Examples}\label{examples}

Pubmatter was used to create these documents, for loading the authors in
a standardized way and creting the common elements (authors,
affiliations, ORCIDs, DOIs, Open Access Links, copyright statements,
etc.)

\pandocbounded{\includegraphics[keepaspectratio]{https://raw.githubusercontent.com/curvenote/pubmatter/main/images/lapreprint.png?raw=true}}

\pandocbounded{\includegraphics[keepaspectratio]{https://raw.githubusercontent.com/curvenote/pubmatter/main/images/scipy.png?raw=true}}

\pandocbounded{\includegraphics[keepaspectratio]{https://raw.githubusercontent.com/curvenote/pubmatter/main/images/agrogeo.png?raw=true}}

\subsection{Documentation}\label{documentation}

The full documentation can be found in
\href{https://github.com/curvenote/pubmatter/blob/main/docs.pdf}{docs.pdf}
. To use \texttt{\ pubmatter\ } import it:

\begin{Shaded}
\begin{Highlighting}[]
\NormalTok{\#import "@preview/pubmatter:0.1.0"}
\end{Highlighting}
\end{Shaded}

The docs also use \texttt{\ pubmatter\ } , in a simplified way, you can
see the
\href{https://github.com/curvenote/pubmatter/blob/main/docs.typ}{docs.typ}
to see a simple example of using various components to create a new
document. Here is a preview of the docs:

\href{https://github.com/curvenote/pubmatter/blob/main/docs.pdf}{\pandocbounded{\includegraphics[keepaspectratio]{https://raw.githubusercontent.com/curvenote/pubmatter/main/images/pubmatter.png?raw=true}}}

\subsubsection{Loading Frontmatter}\label{loading-frontmatter}

The frontmatter can contain all information for an article, including
title, authors, affiliations, abstracts and keywords. These are then
normalized into a standardized format that can be used with a number of
\texttt{\ show\ } functions like \texttt{\ show-authors\ } . For
example, we might have a YAML file that looks like this:

\begin{Shaded}
\begin{Highlighting}[]
\FunctionTok{author}\KeywordTok{:}\AttributeTok{ Rowan Cockett}
\FunctionTok{date}\KeywordTok{:}\AttributeTok{ 2024/01/26}
\end{Highlighting}
\end{Shaded}

You can load that file with \texttt{\ yaml\ } , and pass it to the
\texttt{\ load\ } function:

\begin{Shaded}
\begin{Highlighting}[]
\NormalTok{\#let fm = pubmatter.load(yaml("pubmatter.yml"))}
\end{Highlighting}
\end{Shaded}

This will give you a normalized data-structure that can be used with the
\texttt{\ show\ } functions for showing various parts of a document.

You can also use a \texttt{\ dictionary\ } directly:

\begin{Shaded}
\begin{Highlighting}[]
\NormalTok{\#let fm = pubmatter.load((}
\NormalTok{  author: (}
\NormalTok{    (}
\NormalTok{      name: "Rowan Cockett",}
\NormalTok{      email: "rowan@curvenote.com",}
\NormalTok{      orcid: "0000{-}0002{-}7859{-}8394",}
\NormalTok{      affiliations: "Curvenote Inc.",}
\NormalTok{    ),}
\NormalTok{  ),}
\NormalTok{  date: datetime(year: 2024, month: 01, day: 26),}
\NormalTok{  doi: "10.1190/tle35080703.1",}
\NormalTok{))}
\NormalTok{\#pubmatter.show{-}author{-}block(fm)}
\end{Highlighting}
\end{Shaded}

\pandocbounded{\includegraphics[keepaspectratio]{https://raw.githubusercontent.com/curvenote/pubmatter/main/images/author-block.png?raw=true}}

\subsubsection{Theming}\label{theming}

The theme including color and font choice can be set using the
\texttt{\ THEME\ } state. For example, this document has the following
theme set:

\begin{Shaded}
\begin{Highlighting}[]
\NormalTok{\#let theme = (color: red.darken(20\%), font: "Noto Sans")}
\NormalTok{\#set page(header: pubmatter.show{-}page{-}header(theme: theme, fm), footer: pubmatter.show{-}page{-}footer(fm))}
\NormalTok{\#state("THEME").update(theme)}
\end{Highlighting}
\end{Shaded}

Note that for the \texttt{\ header\ } the theme must be passed in
directly. This will hopefully become easier in the future, however,
there is a current bug that removes the page header/footer if you set
this above the \texttt{\ set\ page\ } . See
\href{https://github.com/typst/packages/raw/main/packages/preview/pubmatter/0.1.0/\#2987}{https://github.com/typst/typst/issues/2987}
.

The \texttt{\ font\ } option only corresponds to the frontmatter content
(abstracts, title, header/footer etc.) allowing the body of your
document to have a different font choice.

\subsubsection{Normalized Frontmatter
Object}\label{normalized-frontmatter-object}

The frontmatter object has the following normalized structure:

\begin{Shaded}
\begin{Highlighting}[]
\FunctionTok{title}\KeywordTok{:}\AttributeTok{ content}
\FunctionTok{subtitle}\KeywordTok{:}\AttributeTok{ content}
\FunctionTok{short{-}title}\KeywordTok{:}\AttributeTok{ string}\CommentTok{ \# alias: running{-}title, running{-}head}
\CommentTok{\# Authors Array}
\CommentTok{\# simple string provided for author is turned into ((name: string),)}
\FunctionTok{authors}\KeywordTok{:}\CommentTok{ \# alias: author}
\AttributeTok{  }\KeywordTok{{-}}\AttributeTok{ }\FunctionTok{name}\KeywordTok{:}\AttributeTok{ string}
\AttributeTok{    }\FunctionTok{url}\KeywordTok{:}\AttributeTok{ string}\CommentTok{ \# alias: website, homepage}
\AttributeTok{    }\FunctionTok{email}\KeywordTok{:}\AttributeTok{ string}
\AttributeTok{    }\FunctionTok{phone}\KeywordTok{:}\AttributeTok{ string}
\AttributeTok{    }\FunctionTok{fax}\KeywordTok{:}\AttributeTok{ string}
\AttributeTok{    }\FunctionTok{orcid}\KeywordTok{:}\AttributeTok{ string}\CommentTok{ \# alias: ORCID}
\AttributeTok{    }\FunctionTok{note}\KeywordTok{:}\AttributeTok{ string}
\AttributeTok{    }\FunctionTok{corresponding}\KeywordTok{:}\AttributeTok{ boolean}\CommentTok{ \# default: \textasciigrave{}true\textasciigrave{} when email set}
\AttributeTok{    }\FunctionTok{equal{-}contributor}\KeywordTok{:}\AttributeTok{ boolean}\CommentTok{ \# alias: equalContributor, equal\_contributor}
\AttributeTok{    }\FunctionTok{deceased}\KeywordTok{:}\AttributeTok{ boolean}
\AttributeTok{    }\FunctionTok{roles}\KeywordTok{:}\AttributeTok{ string[]}\CommentTok{ \# must be a contributor role}
\AttributeTok{    }\FunctionTok{affiliations}\KeywordTok{:}\CommentTok{ \# alias: affiliation}
\AttributeTok{      }\KeywordTok{{-}}\AttributeTok{ }\FunctionTok{id}\KeywordTok{:}\AttributeTok{ string}
\AttributeTok{        }\FunctionTok{index}\KeywordTok{:}\AttributeTok{ number}
\CommentTok{\# Affiliations Array}
\FunctionTok{affiliations}\KeywordTok{:}\CommentTok{ \# alias: affiliation}
\AttributeTok{  }\KeywordTok{{-}}\AttributeTok{ string}\CommentTok{ \# simple string is turned into (name: string)}
\AttributeTok{  }\KeywordTok{{-}}\AttributeTok{ }\FunctionTok{id}\KeywordTok{:}\AttributeTok{ string}
\AttributeTok{    }\FunctionTok{index}\KeywordTok{:}\AttributeTok{ number}
\AttributeTok{    }\FunctionTok{name}\KeywordTok{:}\AttributeTok{ string}
\AttributeTok{    }\FunctionTok{institution}\KeywordTok{:}\AttributeTok{ string}\CommentTok{ \# use either name or institution}
\CommentTok{\# Other publication metadata}
\FunctionTok{open{-}access}\KeywordTok{:}\AttributeTok{ boolean}
\FunctionTok{license}\KeywordTok{:}\CommentTok{ \# Can be set with a SPDX ID for creative commons}
\AttributeTok{  }\FunctionTok{id}\KeywordTok{:}\AttributeTok{ string}
\AttributeTok{  }\FunctionTok{url}\KeywordTok{:}\AttributeTok{ string}
\AttributeTok{  }\FunctionTok{name}\KeywordTok{:}\AttributeTok{ string}
\FunctionTok{doi}\KeywordTok{:}\AttributeTok{ string}\CommentTok{ \# must be only the ID, not the full URL}
\FunctionTok{date}\KeywordTok{:}\AttributeTok{ datetime}\CommentTok{ \# validates from \textquotesingle{}YYYY{-}MM{-}DD\textquotesingle{} if a string}
\FunctionTok{citation}\KeywordTok{:}\AttributeTok{ content}
\CommentTok{\# Abstracts Array}
\CommentTok{\# content is turned into ((title: "Abstract", content: string),)}
\FunctionTok{abstracts}\KeywordTok{:}\CommentTok{ \# alias: abstract}
\AttributeTok{  }\KeywordTok{{-}}\AttributeTok{ }\FunctionTok{title}\KeywordTok{:}\AttributeTok{ content}
\AttributeTok{    }\FunctionTok{content}\KeywordTok{:}\AttributeTok{ content}
\end{Highlighting}
\end{Shaded}

Note that you will usually write the affiliations directly in line, in
the following example, we can see that the output is a normalized
affiliation object that is linked by the \texttt{\ id\ } of the
affiliation (just the name if it is a string!).

\begin{Shaded}
\begin{Highlighting}[]
\NormalTok{\#let fm = pubmatter.load((}
\NormalTok{  authors: (}
\NormalTok{    (}
\NormalTok{      name: "Rowan Cockett",}
\NormalTok{      affiliations: "Curvenote Inc.",}
\NormalTok{    ),}
\NormalTok{    (}
\NormalTok{      name: "Steve Purves",}
\NormalTok{      affiliations: ("Executable Books", "Curvenote Inc."),}
\NormalTok{    ),}
\NormalTok{  ),}
\NormalTok{))}
\NormalTok{\#raw(lang:"yaml", yaml.encode(fm))}
\end{Highlighting}
\end{Shaded}

\pandocbounded{\includegraphics[keepaspectratio]{https://raw.githubusercontent.com/curvenote/pubmatter/main/images/normalized.png?raw=true}}

\subsubsection{Full List of Functions}\label{full-list-of-functions}

\begin{itemize}
\tightlist
\item
  \texttt{\ load()\ } - Load a raw frontmatter object
\item
  \texttt{\ doi-link()\ } - Create a DOI link
\item
  \texttt{\ email-link()\ } - Create a mailto link with an email icon
\item
  \texttt{\ github-link()\ } - Create a link to a GitHub profile with
  the GitHub icon
\item
  \texttt{\ open-access-link()\ } - Create a link to Wikipedia with an
  OpenAccess icon
\item
  \texttt{\ orcid-link()\ } - Create a ORCID link with an ORCID logo
\item
  \texttt{\ show-abstract-block()\ } - Show abstract-block including all
  abstracts and keywords
\item
  \texttt{\ show-abstracts()\ } - Show all abstracts (e.g. abstract,
  plain language summary)
\item
  \texttt{\ show-affiliations()\ } - Show affiliations
\item
  \texttt{\ show-author-block()\ } - Show author block, including
  author, icon links (e.g. ORCID, email, etc.) and affiliations
\item
  \texttt{\ show-authors()\ } - Show authors
\item
  \texttt{\ show-citation()\ } - Create a short citation in APA format,
  e.g. Cockett \emph{et al.} , 2023
\item
  \texttt{\ show-copyright()\ } - Show copyright statement based on
  license
\item
  \texttt{\ show-keywords()\ } - Show keywords as a list
\item
  \texttt{\ show-license-badge()\ } - Show the license badges
\item
  \texttt{\ show-page-footer()\ } - Show the venue, date and page
  numbers
\item
  \texttt{\ show-page-header()\ } - Show an open-access badge and the
  DOI and then the running-title and citation
\item
  \texttt{\ show-spaced-content()\ }
\item
  \texttt{\ show-title()\ } - Show title and subtitle
\item
  \texttt{\ show-title-block()\ } - Show title, authors and affiliations
\end{itemize}

\subsection{Contributing}\label{contributing}

To help with standardization of metadata or improve the show-functions
please contribute to this package:\\
\url{https://github.com/curvenote/pubmatter}

\subsubsection{How to add}\label{how-to-add}

Copy this into your project and use the import as \texttt{\ pubmatter\ }

\begin{verbatim}
#import "@preview/pubmatter:0.1.0"
\end{verbatim}

\includesvg[width=0.16667in,height=0.16667in]{/assets/icons/16-copy.svg}

Check the docs for
\href{https://typst.app/docs/reference/scripting/\#packages}{more
information on how to import packages} .

\subsubsection{About}\label{about}

\begin{description}
\tightlist
\item[Author :]
rowanc1
\item[License:]
MIT
\item[Current version:]
0.1.0
\item[Last updated:]
February 10, 2024
\item[First released:]
February 10, 2024
\item[Archive size:]
9.84 kB
\href{https://packages.typst.org/preview/pubmatter-0.1.0.tar.gz}{\pandocbounded{\includesvg[keepaspectratio]{/assets/icons/16-download.svg}}}
\item[Repository:]
\href{https://github.com/curvenote/pubmatter}{GitHub}
\end{description}

\subsubsection{Where to report issues?}\label{where-to-report-issues}

This package is a project of rowanc1 . Report issues on
\href{https://github.com/curvenote/pubmatter}{their repository} . You
can also try to ask for help with this package on the
\href{https://forum.typst.app}{Forum} .

Please report this package to the Typst team using the
\href{https://typst.app/contact}{contact form} if you believe it is a
safety hazard or infringes upon your rights.

\phantomsection\label{versions}
\subsubsection{Version history}\label{version-history}

\begin{longtable}[]{@{}ll@{}}
\toprule\noalign{}
Version & Release Date \\
\midrule\noalign{}
\endhead
\bottomrule\noalign{}
\endlastfoot
0.1.0 & February 10, 2024 \\
\end{longtable}

Typst GmbH did not create this package and cannot guarantee correct
functionality of this package or compatibility with any version of the
Typst compiler or app.


\section{Package List LaTeX/subpar.tex}
\title{typst.app/universe/package/subpar}

\phantomsection\label{banner}
\section{subpar}\label{subpar}

{ 0.2.0 }

Create sub figures easily.

\phantomsection\label{readme}
Subpar is a \href{https://typst.app/}{Typst} package for creating sub
figures.

\begin{Shaded}
\begin{Highlighting}[]
\NormalTok{\#import "@preview/subpar:0.2.0"}

\NormalTok{\#set page(height: auto)}
\NormalTok{\#set par(justify: true)}

\NormalTok{\#subpar.grid(}
\NormalTok{  figure(image("/assets/andromeda.jpg"), caption: [}
\NormalTok{    An image of the andromeda galaxy.}
\NormalTok{  ]), \textless{}a\textgreater{},}
\NormalTok{  figure(image("/assets/mountains.jpg"), caption: [}
\NormalTok{    A sunset illuminating the sky above a mountain range.}
\NormalTok{  ]), \textless{}b\textgreater{},}
\NormalTok{  columns: (1fr, 1fr),}
\NormalTok{  caption: [A figure composed of two sub figures.],}
\NormalTok{  label: \textless{}full\textgreater{},}
\NormalTok{)}

\NormalTok{Above in @full, we see a figure which is composed of two other figures, namely @a and @b.}
\end{Highlighting}
\end{Shaded}

\pandocbounded{\includegraphics[keepaspectratio]{https://github.com/typst/packages/raw/main/packages/preview/subpar/0.2.0/examples/example.png}}

\subsection{Contributing}\label{contributing}

Contributions are most welcome, make sure to let others know you’re
working on something beforehand so no two people waste their time
working on the same issue. It’s recommended to have
\href{https://github.com/tingerrr/typst-test}{\texttt{\ typst-test\ }}
installed to run tests locally.

\subsection{Documentation}\label{documentation}

A guide and API-reference for subpar can be found in it’s
\href{https://github.com/typst/packages/raw/main/packages/preview/subpar/0.2.0/doc/manual.pdf}{manual}
.

\subsubsection{How to add}\label{how-to-add}

Copy this into your project and use the import as \texttt{\ subpar\ }

\begin{verbatim}
#import "@preview/subpar:0.2.0"
\end{verbatim}

\includesvg[width=0.16667in,height=0.16667in]{/assets/icons/16-copy.svg}

Check the docs for
\href{https://typst.app/docs/reference/scripting/\#packages}{more
information on how to import packages} .

\subsubsection{About}\label{about}

\begin{description}
\tightlist
\item[Author :]
\href{mailto:me@tinger.dev}{tinger}
\item[License:]
MIT
\item[Current version:]
0.2.0
\item[Last updated:]
November 18, 2024
\item[First released:]
May 3, 2024
\item[Minimum Typst version:]
0.12.0
\item[Archive size:]
1.15 MB
\href{https://packages.typst.org/preview/subpar-0.2.0.tar.gz}{\pandocbounded{\includesvg[keepaspectratio]{/assets/icons/16-download.svg}}}
\item[Repository:]
\href{https://github.com/tingerrr/subpar}{GitHub}
\item[Categor ies :]
\begin{itemize}
\tightlist
\item[]
\item
  \pandocbounded{\includesvg[keepaspectratio]{/assets/icons/16-package.svg}}
  \href{https://typst.app/universe/search/?category=components}{Components}
\item
  \pandocbounded{\includesvg[keepaspectratio]{/assets/icons/16-list-unordered.svg}}
  \href{https://typst.app/universe/search/?category=model}{Model}
\end{itemize}
\end{description}

\subsubsection{Where to report issues?}\label{where-to-report-issues}

This package is a project of tinger . Report issues on
\href{https://github.com/tingerrr/subpar}{their repository} . You can
also try to ask for help with this package on the
\href{https://forum.typst.app}{Forum} .

Please report this package to the Typst team using the
\href{https://typst.app/contact}{contact form} if you believe it is a
safety hazard or infringes upon your rights.

\phantomsection\label{versions}
\subsubsection{Version history}\label{version-history}

\begin{longtable}[]{@{}ll@{}}
\toprule\noalign{}
Version & Release Date \\
\midrule\noalign{}
\endhead
\bottomrule\noalign{}
\endlastfoot
0.2.0 & November 18, 2024 \\
\href{https://typst.app/universe/package/subpar/0.1.1/}{0.1.1} & July 3,
2024 \\
\href{https://typst.app/universe/package/subpar/0.1.0/}{0.1.0} & May 3,
2024 \\
\end{longtable}

Typst GmbH did not create this package and cannot guarantee correct
functionality of this package or compatibility with any version of the
Typst compiler or app.


\section{Package List LaTeX/badformer.tex}
\title{typst.app/universe/package/badformer}

\phantomsection\label{banner}
\phantomsection\label{template-thumbnail}
\pandocbounded{\includegraphics[keepaspectratio]{https://packages.typst.org/preview/thumbnails/badformer-0.1.0-small.webp}}

\section{badformer}\label{badformer}

{ 0.1.0 }

Retro-gaming in Typst. Reach the goal and complete the mission.

\href{/app?template=badformer&version=0.1.0}{Create project in app}

\phantomsection\label{readme}
Reach the goal in this retro-inspired wireframing platformer. Play in 3
dimensions and compete for the lowest number of steps to win!

This small game is playable in the Typst editor and best enjoyed with
the web app or \texttt{\ typst\ watch\ } . It was first released for the
24 Days to Christmas campaign in winter of 2023.

\subsection{Usage}\label{usage}

You can use this template in the Typst web app by clicking “Start from
template� on the dashboard and searching for \texttt{\ badformer\ } .

Alternatively, you can use the CLI to kick this project off using the
command

\begin{verbatim}
typst init @preview/badformer
\end{verbatim}

Typst will create a new directory with all the files needed to get you
started.

Move with WASD and jump with space. You can also display a minimap by
pressing E.

\subsection{Configuration}\label{configuration}

This template exports the \texttt{\ game\ } function, which accepts a
positional argument for the game input.

The template will initialize your package with a sample call to the
\texttt{\ game\ } function in a show rule. If you want to change an
existing project to use this template, you can add a show rule like this
at the top of your file:

\begin{Shaded}
\begin{Highlighting}[]
\NormalTok{\#import "@preview/badformer:0.1.0": game}
\NormalTok{\#show: game(read("main.typ"))}

\NormalTok{// Move with WASD and jump with space.}
\end{Highlighting}
\end{Shaded}

\href{/app?template=badformer&version=0.1.0}{Create project in app}

\subsubsection{How to use}\label{how-to-use}

Click the button above to create a new project using this template in
the Typst app.

You can also use the Typst CLI to start a new project on your computer
using this command:

\begin{verbatim}
typst init @preview/badformer:0.1.0
\end{verbatim}

\includesvg[width=0.16667in,height=0.16667in]{/assets/icons/16-copy.svg}

\subsubsection{About}\label{about}

\begin{description}
\tightlist
\item[Author :]
\href{https://typst.app}{Typst GmbH}
\item[License:]
MIT-0
\item[Current version:]
0.1.0
\item[Last updated:]
March 6, 2024
\item[First released:]
March 6, 2024
\item[Minimum Typst version:]
0.10.0
\item[Archive size:]
5.43 kB
\href{https://packages.typst.org/preview/badformer-0.1.0.tar.gz}{\pandocbounded{\includesvg[keepaspectratio]{/assets/icons/16-download.svg}}}
\item[Repository:]
\href{https://github.com/typst/templates}{GitHub}
\item[Categor y :]
\begin{itemize}
\tightlist
\item[]
\item
  \pandocbounded{\includesvg[keepaspectratio]{/assets/icons/16-smile.svg}}
  \href{https://typst.app/universe/search/?category=fun}{Fun}
\end{itemize}
\end{description}

\subsubsection{Where to report issues?}\label{where-to-report-issues}

This template is a project of Typst GmbH . Report issues on
\href{https://github.com/typst/templates}{their repository} . You can
also try to ask for help with this template on the
\href{https://forum.typst.app}{Forum} .

\phantomsection\label{versions}
\subsubsection{Version history}\label{version-history}

\begin{longtable}[]{@{}ll@{}}
\toprule\noalign{}
Version & Release Date \\
\midrule\noalign{}
\endhead
\bottomrule\noalign{}
\endlastfoot
0.1.0 & March 6, 2024 \\
\end{longtable}


\section{Package List LaTeX/ttt-lists.tex}
\title{typst.app/universe/package/ttt-lists}

\phantomsection\label{banner}
\phantomsection\label{template-thumbnail}
\pandocbounded{\includegraphics[keepaspectratio]{https://packages.typst.org/preview/thumbnails/ttt-lists-0.1.0-small.webp}}

\section{ttt-lists}\label{ttt-lists}

{ 0.1.0 }

Template to create student lists. Part of the ttt-collection to make a
teachers life easier.

\href{/app?template=ttt-lists&version=0.1.0}{Create project in app}

\phantomsection\label{readme}
\texttt{\ ttt-lists\ } is a \emph{template} to create class lists and
belongs to the
\href{https://github.com/jomaway/typst-teacher-templates}{typst-teacher-tools-collection}
.

\subsection{Usage}\label{usage}

Run this command inside your terminal to init a new list.

\begin{Shaded}
\begin{Highlighting}[]
\ExtensionTok{typst}\NormalTok{ init @preview/ttt{-}lists my{-}student{-}list}
\end{Highlighting}
\end{Shaded}

This will scaffold the following folder structure.

\begin{Shaded}
\begin{Highlighting}[]
\NormalTok{my{-}student{-}list/}
\NormalTok{├─ students.csv}
\NormalTok{└─ students.typ}
\end{Highlighting}
\end{Shaded}

Edit the \texttt{\ students.csv\ } file or replace it with your own.
Modify the \texttt{\ students.typ\ } to your liking or leave as is and
then run \texttt{\ typst\ compile\ students.typ\ } to create a beautiful
list.

\href{/app?template=ttt-lists&version=0.1.0}{Create project in app}

\subsubsection{How to use}\label{how-to-use}

Click the button above to create a new project using this template in
the Typst app.

You can also use the Typst CLI to start a new project on your computer
using this command:

\begin{verbatim}
typst init @preview/ttt-lists:0.1.0
\end{verbatim}

\includesvg[width=0.16667in,height=0.16667in]{/assets/icons/16-copy.svg}

\subsubsection{About}\label{about}

\begin{description}
\tightlist
\item[Author :]
\href{https://github.com/jomaway}{Jomaway}
\item[License:]
MIT
\item[Current version:]
0.1.0
\item[Last updated:]
April 2, 2024
\item[First released:]
April 2, 2024
\item[Minimum Typst version:]
0.11.0
\item[Archive size:]
3.03 kB
\href{https://packages.typst.org/preview/ttt-lists-0.1.0.tar.gz}{\pandocbounded{\includesvg[keepaspectratio]{/assets/icons/16-download.svg}}}
\item[Repository:]
\href{https://github.com/jomaway/typst-teacher-templates}{GitHub}
\item[Discipline :]
\begin{itemize}
\tightlist
\item[]
\item
  \href{https://typst.app/universe/search/?discipline=education}{Education}
\end{itemize}
\item[Categor ies :]
\begin{itemize}
\tightlist
\item[]
\item
  \pandocbounded{\includesvg[keepaspectratio]{/assets/icons/16-package.svg}}
  \href{https://typst.app/universe/search/?category=components}{Components}
\item
  \pandocbounded{\includesvg[keepaspectratio]{/assets/icons/16-hammer.svg}}
  \href{https://typst.app/universe/search/?category=utility}{Utility}
\item
  \pandocbounded{\includesvg[keepaspectratio]{/assets/icons/16-envelope.svg}}
  \href{https://typst.app/universe/search/?category=office}{Office}
\end{itemize}
\end{description}

\subsubsection{Where to report issues?}\label{where-to-report-issues}

This template is a project of Jomaway . Report issues on
\href{https://github.com/jomaway/typst-teacher-templates}{their
repository} . You can also try to ask for help with this template on the
\href{https://forum.typst.app}{Forum} .

Please report this template to the Typst team using the
\href{https://typst.app/contact}{contact form} if you believe it is a
safety hazard or infringes upon your rights.

\phantomsection\label{versions}
\subsubsection{Version history}\label{version-history}

\begin{longtable}[]{@{}ll@{}}
\toprule\noalign{}
Version & Release Date \\
\midrule\noalign{}
\endhead
\bottomrule\noalign{}
\endlastfoot
0.1.0 & April 2, 2024 \\
\end{longtable}

Typst GmbH did not create this template and cannot guarantee correct
functionality of this template or compatibility with any version of the
Typst compiler or app.


\section{Package List LaTeX/haw-hamburg-report.tex}
\title{typst.app/universe/package/haw-hamburg-report}

\phantomsection\label{banner}
\phantomsection\label{template-thumbnail}
\pandocbounded{\includegraphics[keepaspectratio]{https://packages.typst.org/preview/thumbnails/haw-hamburg-report-0.3.1-small.webp}}

\section{haw-hamburg-report}\label{haw-hamburg-report}

{ 0.3.1 }

Unofficial template for writing a report in the HAW Hamburg department
of Computer Science design.

\href{/app?template=haw-hamburg-report&version=0.3.1}{Create project in
app}

\phantomsection\label{readme}
This is an \textbf{\texttt{\ unofficial\ }} template for writing a
report in the \texttt{\ HAW\ Hamburg\ } department of
\texttt{\ Computer\ Science\ } design using
\href{https://github.com/typst/typst}{Typst} .

\subsection{Required Fonts}\label{required-fonts}

To correctly render this template please make sure that the
\texttt{\ New\ Computer\ Modern\ } font is installed on your system.

\subsection{Usage}\label{usage}

To use this package just add the following code to your
\href{https://github.com/typst/typst}{Typst} document:

\begin{Shaded}
\begin{Highlighting}[]
\NormalTok{\#import "@preview/haw{-}hamburg:0.3.1": report}

\NormalTok{\#show: report.with(}
\NormalTok{  language: "en",}
\NormalTok{  title: "Example title",}
\NormalTok{  author:"Example author",}
\NormalTok{  faculty: "Engineering and Computer Science",}
\NormalTok{  department: "Computer Science",}
\NormalTok{  include{-}declaration{-}of{-}independent{-}processing: true,}
\NormalTok{)}
\end{Highlighting}
\end{Shaded}

\subsection{How to Compile}\label{how-to-compile}

This project contains an example setup that splits individual chapters
into different files.\\
This can cause problems when using references etc.\\
These problems can be avoided by following these steps:

\begin{itemize}
\tightlist
\item
  Make sure to always compile your \texttt{\ main.typ\ } file which
  includes all of your chapters for references to work correctly.
\item
  VSCode:

  \begin{itemize}
  \tightlist
  \item
    Install the
    \href{https://marketplace.visualstudio.com/items?itemName=myriad-dreamin.tinymist}{Tinymist
    Typst} extension.
  \item
    Make sure to start the \texttt{\ PDF\ } or
    \texttt{\ Live\ Preview\ } only from within your
    \texttt{\ main.typ\ } file.
  \item
    If problems occur it usually helps to close the preview and restart
    it from your \texttt{\ main.typ\ } file.
  \end{itemize}
\end{itemize}

\href{/app?template=haw-hamburg-report&version=0.3.1}{Create project in
app}

\subsubsection{How to use}\label{how-to-use}

Click the button above to create a new project using this template in
the Typst app.

You can also use the Typst CLI to start a new project on your computer
using this command:

\begin{verbatim}
typst init @preview/haw-hamburg-report:0.3.1
\end{verbatim}

\includesvg[width=0.16667in,height=0.16667in]{/assets/icons/16-copy.svg}

\subsubsection{About}\label{about}

\begin{description}
\tightlist
\item[Author :]
Lasse Rosenow
\item[License:]
MIT
\item[Current version:]
0.3.1
\item[Last updated:]
November 13, 2024
\item[First released:]
October 14, 2024
\item[Archive size:]
6.11 kB
\href{https://packages.typst.org/preview/haw-hamburg-report-0.3.1.tar.gz}{\pandocbounded{\includesvg[keepaspectratio]{/assets/icons/16-download.svg}}}
\item[Repository:]
\href{https://github.com/LasseRosenow/HAW-Hamburg-Typst-Template}{GitHub}
\item[Categor y :]
\begin{itemize}
\tightlist
\item[]
\item
  \pandocbounded{\includesvg[keepaspectratio]{/assets/icons/16-speak.svg}}
  \href{https://typst.app/universe/search/?category=report}{Report}
\end{itemize}
\end{description}

\subsubsection{Where to report issues?}\label{where-to-report-issues}

This template is a project of Lasse Rosenow . Report issues on
\href{https://github.com/LasseRosenow/HAW-Hamburg-Typst-Template}{their
repository} . You can also try to ask for help with this template on the
\href{https://forum.typst.app}{Forum} .

Please report this template to the Typst team using the
\href{https://typst.app/contact}{contact form} if you believe it is a
safety hazard or infringes upon your rights.

\phantomsection\label{versions}
\subsubsection{Version history}\label{version-history}

\begin{longtable}[]{@{}ll@{}}
\toprule\noalign{}
Version & Release Date \\
\midrule\noalign{}
\endhead
\bottomrule\noalign{}
\endlastfoot
0.3.1 & November 13, 2024 \\
\href{https://typst.app/universe/package/haw-hamburg-report/0.3.0/}{0.3.0}
& October 14, 2024 \\
\end{longtable}

Typst GmbH did not create this template and cannot guarantee correct
functionality of this template or compatibility with any version of the
Typst compiler or app.


\section{Package List LaTeX/modern-cv.tex}
\title{typst.app/universe/package/modern-cv}

\phantomsection\label{banner}
\phantomsection\label{template-thumbnail}
\pandocbounded{\includegraphics[keepaspectratio]{https://packages.typst.org/preview/thumbnails/modern-cv-0.7.0-small.webp}}

\section{modern-cv}\label{modern-cv}

{ 0.7.0 }

A modern resume template based on the Awesome-CV Latex template.

\href{/app?template=modern-cv&version=0.7.0}{Create project in app}

\phantomsection\label{readme}
\href{https://github.com/DeveloperPaul123/modern-cv/stargazers}{\pandocbounded{\includesvg[keepaspectratio]{https://img.shields.io/badge/Say\%20Thanks-\%F0\%9F\%91\%8D-1EAEDB.svg}}}
\href{https://discord.gg/CX2ybByRnt}{\pandocbounded{\includegraphics[keepaspectratio]{https://img.shields.io/discord/652515194572111872?logo=Discord}}}
\pandocbounded{\includegraphics[keepaspectratio]{https://img.shields.io/github/v/release/DeveloperPaul123/modern-cv}}
\href{https://github.com/DeveloperPaul123/modern-cv/actions/workflows/tests.yml}{\pandocbounded{\includesvg[keepaspectratio]{https://github.com/DeveloperPaul123/modern-cv/actions/workflows/tests.yml/badge.svg}}}

A port of the \href{https://github.com/posquit0/Awesome-CV}{Awesome-CV}
Latex resume template in \href{https://github.com/typst/typst}{typst} .

\subsection{Requirements}\label{requirements}

\subsubsection{Tools}\label{tools}

The following tools are used for the development of this template:

\begin{itemize}
\tightlist
\item
  \href{https://github.com/typst/typst}{typst}
\item
  \href{https://github.com/tingerrr/typst-test}{typst-test} for running
  tests
\item
  \href{https://github.com/casey/just}{just} for simplifying command
  running
\item
  \href{https://github.com/shssoichiro/oxipng}{oxipng} for compressing
  thumbnails used in the README
\end{itemize}

\subsubsection{Fonts}\label{fonts}

You will need the \texttt{\ Roboto\ } and \texttt{\ Source\ Sans\ Pro\ }
fonts installed on your system or available somewhere. If you are using
the \texttt{\ typst\ } web app, no further action is necessary. You can
download them from the following links:

\begin{itemize}
\tightlist
\item
  \href{https://fonts.google.com/specimen/Roboto}{Roboto}
\item
  \href{https://github.com/adobe-fonts/source-sans-pro}{Source Sans Pro}
\end{itemize}

This template also uses FontAwesome icons via the
\href{https://typst.app/universe/package/fontawesome}{fontawesome}
package. You will need to install the fontawesome fonts on your system
or configure the \texttt{\ typst\ } web app to use them. You can
download fontawesome \href{https://fontawesome.com/download}{here} .

To use the fontawesome icons in the web app, add a \texttt{\ fonts\ }
folder to your project and upload the \texttt{\ otf\ } files from the
fontawesome download to this folder like so:

\pandocbounded{\includegraphics[keepaspectratio]{https://github.com/typst/packages/raw/main/packages/preview/modern-cv/0.7.0/assets/images/typst_web_editor.png}}

See \texttt{\ typst\ fonts\ -\/-help\ } for more information on
configuring fonts for \texttt{\ typst\ } that are not installed on your
system.

\subsubsection{Usage}\label{usage}

Below is a basic example for a simple resume:

\begin{Shaded}
\begin{Highlighting}[]
\NormalTok{\#import "@preview/modern{-}cv:0.7.0": *}

\NormalTok{\#show: resume.with(}
\NormalTok{  author: (}
\NormalTok{      firstname: "John", }
\NormalTok{      lastname: "Smith",}
\NormalTok{      email: "js@example.com", }
\NormalTok{      phone: "(+1) 111{-}111{-}1111",}
\NormalTok{      github: "DeveloperPaul123",}
\NormalTok{      linkedin: "Example",}
\NormalTok{      address: "111 Example St. Example City, EX 11111",}
\NormalTok{      positions: (}
\NormalTok{        "Software Engineer",}
\NormalTok{        "Software Architect"}
\NormalTok{      )}
\NormalTok{  ),}
\NormalTok{  date: datetime.today().display()}
\NormalTok{)}

\NormalTok{= Education}

\NormalTok{\#resume{-}entry(}
\NormalTok{  title: "Example University",}
\NormalTok{  location: "B.S. in Computer Science",}
\NormalTok{  date: "August 2014 {-} May 2019",}
\NormalTok{  description: "Example"}
\NormalTok{)}

\NormalTok{\#resume{-}item[}
\NormalTok{  {-} \#lorem(20)}
\NormalTok{  {-} \#lorem(15)}
\NormalTok{  {-} \#lorem(25)  }
\NormalTok{]}
\end{Highlighting}
\end{Shaded}

After saving to a \texttt{\ *.typ\ } file, compile your resume using the
following command:

\begin{Shaded}
\begin{Highlighting}[]
\ExtensionTok{typst}\NormalTok{ compile resume.typ}
\end{Highlighting}
\end{Shaded}

For more information on how to use and compile \texttt{\ typst\ } files,
see the \href{https://typst.app/docs}{official documentation} .

Documentation for this template is published with each commit. See the
attached PDF on each Github Action run
\href{https://github.com/DeveloperPaul123/modern-cv/actions}{here} .

\subsubsection{Output Examples}\label{output-examples}

\begin{longtable}[]{@{}ll@{}}
\toprule\noalign{}
Resumes & Cover letters \\
\midrule\noalign{}
\endhead
\bottomrule\noalign{}
\endlastfoot
\pandocbounded{\includegraphics[keepaspectratio]{https://github.com/typst/packages/raw/main/packages/preview/modern-cv/0.7.0/assets/images/resume.png}}
&
\pandocbounded{\includegraphics[keepaspectratio]{https://github.com/typst/packages/raw/main/packages/preview/modern-cv/0.7.0/assets/images/coverletter.png}} \\
\pandocbounded{\includegraphics[keepaspectratio]{https://github.com/typst/packages/raw/main/packages/preview/modern-cv/0.7.0/assets/images/resume2.png}}
&
\pandocbounded{\includegraphics[keepaspectratio]{https://github.com/typst/packages/raw/main/packages/preview/modern-cv/0.7.0/assets/images/coverletter2.png}} \\
\end{longtable}

\subsection{Building and Testing
Locally}\label{building-and-testing-locally}

To build and test the project locally, you will need to install the
\texttt{\ typst\ } CLI. You can find instructions on how to do this
\href{https://github.com/typst/typst\#installation}{here} .

With typst installed you can make changes to \texttt{\ lib.typ\ } and
then \texttt{\ just\ install\ } or \texttt{\ just\ install-preview\ } to
install the package locally. Change the import statements in the
template files to point to the local package (if needed):

\begin{Shaded}
\begin{Highlighting}[]
\NormalTok{\#import "@local/modern{-}cv:0.6.0": *}
\end{Highlighting}
\end{Shaded}

If you use \texttt{\ just\ install-preview\ } you will only need to
update the version number to match \texttt{\ typst.toml\ } .

Note that the script parses the \texttt{\ typst.toml\ } to determine the
version number and the folder the local files are installed to.

\subsubsection{Formatting}\label{formatting}

This project uses
\href{https://github.com/Enter-tainer/typstyle}{typstyle} to format the
code. Run \texttt{\ just\ format\ } to format all the \texttt{\ *.typ\ }
files in the project. Be sure to install \texttt{\ typstyle\ } before
running the script.

\subsection{License}\label{license}

The project is licensed under the MIT license. See
\href{https://github.com/typst/packages/raw/main/packages/preview/modern-cv/0.7.0/LICENSE}{LICENSE}
for more details.

\subsection{Author}\label{author}

\begin{longtable}[]{@{}
  >{\raggedright\arraybackslash}p{(\linewidth - 0\tabcolsep) * \real{1.0000}}@{}}
\toprule\noalign{}
\begin{minipage}[b]{\linewidth}\centering
\href{https://github.com/DeveloperPaul123}{\includegraphics[width=1.04167in,height=\textheight,keepaspectratio]{https://avatars0.githubusercontent.com/u/6591180?s=460&v=4}\\
\textsubscript{@DeveloperPaul123}}\strut
\end{minipage} \\
\midrule\noalign{}
\endhead
\bottomrule\noalign{}
\endlastfoot
\end{longtable}

\href{/app?template=modern-cv&version=0.7.0}{Create project in app}

\subsubsection{How to use}\label{how-to-use}

Click the button above to create a new project using this template in
the Typst app.

You can also use the Typst CLI to start a new project on your computer
using this command:

\begin{verbatim}
typst init @preview/modern-cv:0.7.0
\end{verbatim}

\includesvg[width=0.16667in,height=0.16667in]{/assets/icons/16-copy.svg}

\subsubsection{About}\label{about}

\begin{description}
\tightlist
\item[Author :]
\href{https://github.com/DeveloperPaul123}{Paul Tsouchlos}
\item[License:]
MIT
\item[Current version:]
0.7.0
\item[Last updated:]
November 4, 2024
\item[First released:]
March 26, 2024
\item[Minimum Typst version:]
0.12.0
\item[Archive size:]
20.3 kB
\href{https://packages.typst.org/preview/modern-cv-0.7.0.tar.gz}{\pandocbounded{\includesvg[keepaspectratio]{/assets/icons/16-download.svg}}}
\item[Repository:]
\href{https://github.com/DeveloperPaul123/modern-cv}{GitHub}
\item[Categor y :]
\begin{itemize}
\tightlist
\item[]
\item
  \pandocbounded{\includesvg[keepaspectratio]{/assets/icons/16-user.svg}}
  \href{https://typst.app/universe/search/?category=cv}{CV}
\end{itemize}
\end{description}

\subsubsection{Where to report issues?}\label{where-to-report-issues}

This template is a project of Paul Tsouchlos . Report issues on
\href{https://github.com/DeveloperPaul123/modern-cv}{their repository} .
You can also try to ask for help with this template on the
\href{https://forum.typst.app}{Forum} .

Please report this template to the Typst team using the
\href{https://typst.app/contact}{contact form} if you believe it is a
safety hazard or infringes upon your rights.

\phantomsection\label{versions}
\subsubsection{Version history}\label{version-history}

\begin{longtable}[]{@{}ll@{}}
\toprule\noalign{}
Version & Release Date \\
\midrule\noalign{}
\endhead
\bottomrule\noalign{}
\endlastfoot
0.7.0 & November 4, 2024 \\
\href{https://typst.app/universe/package/modern-cv/0.6.0/}{0.6.0} &
September 3, 2024 \\
\href{https://typst.app/universe/package/modern-cv/0.5.0/}{0.5.0} & July
23, 2024 \\
\href{https://typst.app/universe/package/modern-cv/0.4.0/}{0.4.0} & July
10, 2024 \\
\href{https://typst.app/universe/package/modern-cv/0.3.1/}{0.3.1} & May
16, 2024 \\
\href{https://typst.app/universe/package/modern-cv/0.3.0/}{0.3.0} &
April 17, 2024 \\
\href{https://typst.app/universe/package/modern-cv/0.2.0/}{0.2.0} &
April 4, 2024 \\
\href{https://typst.app/universe/package/modern-cv/0.1.0/}{0.1.0} &
March 26, 2024 \\
\end{longtable}

Typst GmbH did not create this template and cannot guarantee correct
functionality of this template or compatibility with any version of the
Typst compiler or app.


\section{Package List LaTeX/polylux.tex}
\title{typst.app/universe/package/polylux}

\phantomsection\label{banner}
\section{polylux}\label{polylux}

{ 0.3.1 }

Presentation slides creation with Typst

{ } Featured Package

\phantomsection\label{readme}
This is a package for creating presentation slides in
\href{https://typst.app/}{Typst} . Read the
\href{https://andreaskroepelin.github.io/polylux/book}{book} to learn
all about it and click
\href{https://andreaskroepelin.github.io/polylux/book/changelog.html}{here}
to see what’s new!

If you like it, consider
\href{https://github.com/andreasKroepelin/polylux}{giving a star on
GitHub} !

\href{https://andreaskroepelin.github.io/polylux/book}{\pandocbounded{\includegraphics[keepaspectratio]{https://img.shields.io/badge/docs-book-green}}}
\pandocbounded{\includegraphics[keepaspectratio]{https://img.shields.io/github/license/andreasKroepelin/polylux}}
\pandocbounded{\includegraphics[keepaspectratio]{https://img.shields.io/github/v/release/andreasKroepelin/polylux}}
\pandocbounded{\includegraphics[keepaspectratio]{https://img.shields.io/github/stars/andreasKroepelin/polylux}}
\href{https://github.com/andreasKroepelin/polylux/releases/latest/download/demo.pdf}{\pandocbounded{\includegraphics[keepaspectratio]{https://img.shields.io/badge/demo-pdf-blue}}}
\pandocbounded{\includegraphics[keepaspectratio]{https://img.shields.io/badge/themes-5-aqua}}

\subsection{Quickstart}\label{quickstart}

For the bare-bones, do-it-yourself experience, all you need is:

\begin{Shaded}
\begin{Highlighting}[]
\NormalTok{// Get Polylux from the official package repository}
\NormalTok{\#import "@preview/polylux:0.3.1": *}

\NormalTok{// Make the paper dimensions fit for a presentation and the text larger}
\NormalTok{\#set page(paper: "presentation{-}16{-}9")}
\NormalTok{\#set text(size: 25pt)}

\NormalTok{// Use \#polylux{-}slide to create a slide and style it using your favourite Typst functions}
\NormalTok{\#polylux{-}slide[}
\NormalTok{  \#align(horizon + center)[}
\NormalTok{    = Very minimalist slides}

\NormalTok{    A lazy author}

\NormalTok{    July 23, 2023}
\NormalTok{  ]}
\NormalTok{]}

\NormalTok{\#polylux{-}slide[}
\NormalTok{  == First slide}

\NormalTok{  Some static text on this slide.}
\NormalTok{]}

\NormalTok{\#polylux{-}slide[}
\NormalTok{  == This slide changes!}

\NormalTok{  You can always see this.}
\NormalTok{  // Make use of features like \#uncover, \#only, and others to create dynamic content}
\NormalTok{  \#uncover(2)[But this appears later!]}
\NormalTok{]}
\end{Highlighting}
\end{Shaded}

This code produces these PDF pages:
\pandocbounded{\includegraphics[keepaspectratio]{https://andreaskroepelin.github.io/polylux/book/minimal.png}}

From there, you can either start creatively adapting the looks to your
likings or you can use one of the provided themes. The simplest one of
them is called \texttt{\ simple\ } (what a coincidence!). It is still
very unintrusive but gives you some sensible defaults:

\begin{Shaded}
\begin{Highlighting}[]
\NormalTok{\#import "@preview/polylux:0.3.1": *}

\NormalTok{\#import themes.simple: *}

\NormalTok{\#set text(font: "Inria Sans")}

\NormalTok{\#show: simple{-}theme.with(}
\NormalTok{  footer: [Simple slides],}
\NormalTok{)}

\NormalTok{\#title{-}slide[}
\NormalTok{  = Keep it simple!}
\NormalTok{  \#v(2em)}

\NormalTok{  Alpha \#footnote[Uni Augsburg] \#h(1em)}
\NormalTok{  Bravo \#footnote[Uni Bayreuth] \#h(1em)}
\NormalTok{  Charlie \#footnote[Uni Chemnitz] \#h(1em)}

\NormalTok{  July 23}
\NormalTok{]}

\NormalTok{\#slide[}
\NormalTok{  == First slide}

\NormalTok{  \#lorem(20)}
\NormalTok{]}

\NormalTok{\#focus{-}slide[}
\NormalTok{  \_Focus!\_}

\NormalTok{  This is very important.}
\NormalTok{]}

\NormalTok{\#centered{-}slide[}
\NormalTok{  = Let\textquotesingle{}s start a new section!}
\NormalTok{]}

\NormalTok{\#slide[}
\NormalTok{  == Dynamic slide}
\NormalTok{  Did you know that...}

\NormalTok{  \#pause}
\NormalTok{  ...you can see the current section at the top of the slide?}
\NormalTok{]}
\end{Highlighting}
\end{Shaded}

This time, we obtain these PDF pages:
\pandocbounded{\includegraphics[keepaspectratio]{https://andreaskroepelin.github.io/polylux/book/themes/gallery/simple.png}}

As you can see, a theme can introduce its own types of slides (here:
\texttt{\ title-slide\ } , \texttt{\ slide\ } , \texttt{\ focus-slide\ }
, \texttt{\ centered-slide\ } ) to let you quickly switch between
different layouts. The book
\href{https://andreaskroepelin.github.io/polylux/book/themes/themes.html}{has
more infos} on how to use (and create your own) themes.

For dynamic content, Polylux also provides
\href{https://andreaskroepelin.github.io/polylux/book/dynamic/dynamic.html}{a
convenient API for complex overlays} .

If you use \href{https://pdfpc.github.io/}{pdfpc} to display your
slides, you can rely on
\href{https://andreaskroepelin.github.io/polylux/book/external/pdfpc.html}{Polylux’
support for it} and create speaker notes, hide slides, configure the
timer and more!

Visit the \href{https://andreaskroepelin.github.io/polylux/book}{book}
for more details or take a look at the
\href{https://github.com/andreasKroepelin/polylux/releases/latest/download/demo.pdf}{demo
PDF} where you can see the features of this template in action.

\textbf{âš~ This package is under active development and there are no
backwards compatibility guarantees!}

\subsection{Acknowledgements}\label{acknowledgements}

Thank you to…

\begin{itemize}
\tightlist
\item
  \href{https://github.com/drupol}{@drupol} for the
  \texttt{\ university\ } theme
\item
  \href{https://github.com/Enivex}{@Enivex} for the
  \texttt{\ metropolis\ } theme
\item
  \href{https://github.com/MarkBlyth}{@MarkBlyth} for contributing to
  the \texttt{\ clean\ } theme
\item
  \href{https://github.com/ntjess}{@ntjess} for contributing to the
  height fitting feature
\item
  \href{https://github.com/JuliusFreudenberger}{@JuliusFreudenberger}
  for maintaining the \texttt{\ polylux2pdfpc\ } AUR package
\item
  \href{https://github.com/fncnt}{@fncnt} for coming up with the name
  “Polylux�
\item
  the Typst authors and contributors for this refreshing piece of
  software
\end{itemize}

\subsubsection{How to add}\label{how-to-add}

Copy this into your project and use the import as \texttt{\ polylux\ }

\begin{verbatim}
#import "@preview/polylux:0.3.1"
\end{verbatim}

\includesvg[width=0.16667in,height=0.16667in]{/assets/icons/16-copy.svg}

Check the docs for
\href{https://typst.app/docs/reference/scripting/\#packages}{more
information on how to import packages} .

\subsubsection{About}\label{about}

\begin{description}
\tightlist
\item[Author s :]
Andreas Kröpelin \& contributors
\item[License:]
MIT
\item[Current version:]
0.3.1
\item[Last updated:]
September 3, 2023
\item[First released:]
July 26, 2023
\item[Archive size:]
9.62 kB
\href{https://packages.typst.org/preview/polylux-0.3.1.tar.gz}{\pandocbounded{\includesvg[keepaspectratio]{/assets/icons/16-download.svg}}}
\item[Repository:]
\href{https://github.com/andreasKroepelin/polylux}{GitHub}
\end{description}

\subsubsection{Where to report issues?}\label{where-to-report-issues}

This package is a project of Andreas Kröpelin and contributors . Report
issues on \href{https://github.com/andreasKroepelin/polylux}{their
repository} . You can also try to ask for help with this package on the
\href{https://forum.typst.app}{Forum} .

Please report this package to the Typst team using the
\href{https://typst.app/contact}{contact form} if you believe it is a
safety hazard or infringes upon your rights.

\phantomsection\label{versions}
\subsubsection{Version history}\label{version-history}

\begin{longtable}[]{@{}ll@{}}
\toprule\noalign{}
Version & Release Date \\
\midrule\noalign{}
\endhead
\bottomrule\noalign{}
\endlastfoot
0.3.1 & September 3, 2023 \\
\href{https://typst.app/universe/package/polylux/0.2.0/}{0.2.0} & July
26, 2023 \\
\end{longtable}

Typst GmbH did not create this package and cannot guarantee correct
functionality of this package or compatibility with any version of the
Typst compiler or app.


\section{Package List LaTeX/minerva-report-fcfm.tex}
\title{typst.app/universe/package/minerva-report-fcfm}

\phantomsection\label{banner}
\phantomsection\label{template-thumbnail}
\pandocbounded{\includegraphics[keepaspectratio]{https://packages.typst.org/preview/thumbnails/minerva-report-fcfm-0.2.1-small.webp}}

\section{minerva-report-fcfm}\label{minerva-report-fcfm}

{ 0.2.1 }

Template de artículos, informes y tareas para la Facultad de Ciencias
Físicas y Matemáticas (FCFM).

\href{/app?template=minerva-report-fcfm&version=0.2.1}{Create project in
app}

\phantomsection\label{readme}
Template para hacer tareas, informes y trabajos, para estudiantes y
académicos de la Facultad de Ciencias Físicas y Matemáticas de la
Universidad de Chile que han usado templates similares para LaTeX.

\subsection{Guía Rápida}\label{guuxe3a-ruxe3pida}

\subsubsection{\texorpdfstring{\href{https://typst.app/}{Webapp}}{Webapp}}\label{webapp}

Si utilizas la webapp de Typst puedes presionar “Start from
template� y buscar “minerva-report-fcfm� para crear un nuevo
proyecto con este template.

\subsubsection{Typst CLI}\label{typst-cli}

Teniendo el CLI con la versión 0.11.0 o mayor, puedes realizar:

\begin{Shaded}
\begin{Highlighting}[]
\ExtensionTok{typst}\NormalTok{ init @preview/minerva{-}report{-}fcfm:0.2.1}
\end{Highlighting}
\end{Shaded}

Esto va a descargar el template en la cache de typst y luego va a
iniciar el proyecto en la carpeta actual.

\subsection{Configuración}\label{configuraciuxe3uxb3n}

La mayoría de la configuración se realiza a través del archivo
\texttt{\ meta.typ\ } , allí podrás elegir un título, indicar los
autores, el equipo docente, entre otras configuraciones.

El campo \texttt{\ autores\ } solo puede ser \texttt{\ string\ } o un
\texttt{\ array\ } de strings.

La configuración \texttt{\ departamento\ } puede ser personalizada a
cualquier organización pasandole un diccionario de esta forma:

\begin{Shaded}
\begin{Highlighting}[]
\NormalTok{\#let departamento = (}
\NormalTok{  nombre: (}
\NormalTok{    "Universidad Técnica Federico Santa María",}
\NormalTok{  )}
\NormalTok{)}
\end{Highlighting}
\end{Shaded}

Las demás configuraciones pueden ser un \texttt{\ content\ }
arbitrario, o un \texttt{\ string\ } .

\subsubsection{Configuración
Avanzada}\label{configuraciuxe3uxb3n-avanzada}

Algunos aspectos más avanzados pueden ser configurados a través de la
show rule que inicializa el documento
\texttt{\ \#show:\ minerva.report.with(\ ...\ )\ } , los parámetros
opcionales que recibe la función \texttt{\ report\ } son los
siguientes:

\begin{longtable}[]{@{}lll@{}}
\toprule\noalign{}
nombre & tipo & descrición \\
\midrule\noalign{}
\endhead
\bottomrule\noalign{}
\endlastfoot
portada & (meta) =\textgreater{} content & Una función que recibe el
diccionario \texttt{\ meta.typ\ } y retorna una página. \\
header & (meta) =\textgreater{} content & Header a aplicarse a cada
página. \\
footer & (meta) =\textgreater{} content & Footer a aplicarse a cada
página. \\
showrules & bool & El template aplica ciertas show-rules para que sea
más fácil de utilizar. Si quires más personalización, es probable
que necesites desactivarlas y luego solo utilizar las que necesites. \\
\end{longtable}

\paragraph{Show Rules}\label{show-rules}

El template incluye show rules que pueden ser incluidas opcionalmente.
Todas estas show rules pueden ser activadas agregando:

\begin{Shaded}
\begin{Highlighting}[]
\NormalTok{\#show: minerva.\textless{}nombre{-}función\textgreater{}}
\end{Highlighting}
\end{Shaded}

Justo después de la línea
\texttt{\ \#show\ minerva.report.with(\ ...\ )\ } reemplazando
\texttt{\ \textless{}nombre-función\textgreater{}\ } por el nombre de
la show rule a aplicar.

\subparagraph{primer-heading-en-nueva-pag (activada por
defecto)}\label{primer-heading-en-nueva-pag-activada-por-defecto}

Esta show rule hace que el primer heading que tenga
\texttt{\ outlined:\ true\ } se muestre en una nueva página (con
\texttt{\ weak:\ true\ } ). Notar que al ser \texttt{\ weak:\ true\ } si
la página ya de por si estaba vacía, no se crea otra página adicional,
pero para que la página realmente se considere vacía no debe contener
absolutamente nada, incluso tener elementos invisibles va a causar que
se agregue una página extra.

\subparagraph{operadores-es (activada por
defecto)}\label{operadores-es-activada-por-defecto}

Cambia los operadores matemáticos que define Typst por defecto a sus
contrapartes en español, esto es, cambia \texttt{\ lim\ } por
\texttt{\ lím\ } , \texttt{\ inf\ } por \texttt{\ ínf\ } y así con
todos.

\subparagraph{formato-numeros-es}\label{formato-numeros-es}

Cambia los números dentro de las ecuaciones para que usen coma decimal
en vez de punto decimal, como es convención en el español. Esta show
rule no viene activa por defecto.

\href{/app?template=minerva-report-fcfm&version=0.2.1}{Create project in
app}

\subsubsection{How to use}\label{how-to-use}

Click the button above to create a new project using this template in
the Typst app.

You can also use the Typst CLI to start a new project on your computer
using this command:

\begin{verbatim}
typst init @preview/minerva-report-fcfm:0.2.1
\end{verbatim}

\includesvg[width=0.16667in,height=0.16667in]{/assets/icons/16-copy.svg}

\subsubsection{About}\label{about}

\begin{description}
\tightlist
\item[Author :]
\href{https://github.com/Dav1com}{David Ibáñez}
\item[License:]
MIT-0
\item[Current version:]
0.2.1
\item[Last updated:]
October 14, 2024
\item[First released:]
April 15, 2024
\item[Minimum Typst version:]
0.11.0
\item[Archive size:]
246 kB
\href{https://packages.typst.org/preview/minerva-report-fcfm-0.2.1.tar.gz}{\pandocbounded{\includesvg[keepaspectratio]{/assets/icons/16-download.svg}}}
\item[Repository:]
\href{https://github.com/Dav1com/minerva-report-fcfm}{GitHub}
\item[Categor y :]
\begin{itemize}
\tightlist
\item[]
\item
  \pandocbounded{\includesvg[keepaspectratio]{/assets/icons/16-speak.svg}}
  \href{https://typst.app/universe/search/?category=report}{Report}
\end{itemize}
\end{description}

\subsubsection{Where to report issues?}\label{where-to-report-issues}

This template is a project of David Ibáñez . Report issues on
\href{https://github.com/Dav1com/minerva-report-fcfm}{their repository}
. You can also try to ask for help with this template on the
\href{https://forum.typst.app}{Forum} .

Please report this template to the Typst team using the
\href{https://typst.app/contact}{contact form} if you believe it is a
safety hazard or infringes upon your rights.

\phantomsection\label{versions}
\subsubsection{Version history}\label{version-history}

\begin{longtable}[]{@{}ll@{}}
\toprule\noalign{}
Version & Release Date \\
\midrule\noalign{}
\endhead
\bottomrule\noalign{}
\endlastfoot
0.2.1 & October 14, 2024 \\
\href{https://typst.app/universe/package/minerva-report-fcfm/0.2.0/}{0.2.0}
& April 29, 2024 \\
\href{https://typst.app/universe/package/minerva-report-fcfm/0.1.0/}{0.1.0}
& April 15, 2024 \\
\end{longtable}

Typst GmbH did not create this template and cannot guarantee correct
functionality of this template or compatibility with any version of the
Typst compiler or app.


\section{Package List LaTeX/silky-slides-insa.tex}
\title{typst.app/universe/package/silky-slides-insa}

\phantomsection\label{banner}
\phantomsection\label{template-thumbnail}
\pandocbounded{\includegraphics[keepaspectratio]{https://packages.typst.org/preview/thumbnails/silky-slides-insa-0.1.1-small.webp}}

\section{silky-slides-insa}\label{silky-slides-insa}

{ 0.1.1 }

A template made for presentations of INSA, a French engineering school.

\href{/app?template=silky-slides-insa&version=0.1.1}{Create project in
app}

\phantomsection\label{readme}
\pandocbounded{\includegraphics[keepaspectratio]{https://github.com/typst/packages/raw/main/packages/preview/silky-slides-insa/0.1.1/thumbnail-full.png}}

Typst Template for presentation for the french engineering school INSA.

\subsection{Table of contents}\label{table-of-contents}

\begin{enumerate}
\tightlist
\item
  \href{https://github.com/typst/packages/raw/main/packages/preview/silky-slides-insa/0.1.1/\#examples}{Example}
\item
  \href{https://github.com/typst/packages/raw/main/packages/preview/silky-slides-insa/0.1.1/\#usage}{Usage}
\item
  \href{https://github.com/typst/packages/raw/main/packages/preview/silky-slides-insa/0.1.1/\#fonts}{Fonts
  information}
\item
  \href{https://github.com/typst/packages/raw/main/packages/preview/silky-slides-insa/0.1.1/\#notes}{Notes}
\item
  \href{https://github.com/typst/packages/raw/main/packages/preview/silky-slides-insa/0.1.1/\#license}{License}
\item
  \href{https://github.com/typst/packages/raw/main/packages/preview/silky-slides-insa/0.1.1/\#changelog}{Changelog}
\end{enumerate}

\subsection{Example}\label{example}

\begin{Shaded}
\begin{Highlighting}[]
\NormalTok{\#import "@preview/silky{-}slides{-}insa:0.1.1": *}
\NormalTok{\#show: insa{-}slides.with(}
\NormalTok{  title: "Titre du diaporama",}
\NormalTok{  title{-}visual: none,}
\NormalTok{  subtitle: "Sous{-}titre (noms et prénoms ?)",}
\NormalTok{  insa: "rennes"}
\NormalTok{)}

\NormalTok{= Titre de section}

\NormalTok{== Titre d\textquotesingle{}une slide}

\NormalTok{{-} Liste}
\NormalTok{  {-} dans}
\NormalTok{    {-} une liste}

\NormalTok{On peut aussi faire un \#text(fill: insa{-}colors.secondary)[texte] avec les \#text(fill: insa{-}colors.primary)[couleurs de l\textquotesingle{}INSA] !}

\NormalTok{== Une autre slide}

\NormalTok{Du texte}

\NormalTok{\#pause}

\NormalTok{Et un autre texte qui apparaît plus tard !}

\NormalTok{\#section{-}slide[Une autre section][Avec une petite description]}

\NormalTok{Coucou}
\end{Highlighting}
\end{Shaded}

\subsection{Usage}\label{usage}

\subsubsection{Slide show rule}\label{slide-show-rule}

You call it with \texttt{\ \#show:\ insa-slides.with(..parameters)\ } .

\begin{longtable}[]{@{}llll@{}}
\toprule\noalign{}
Parameter & Description & Type & Example \\
\midrule\noalign{}
\endhead
\bottomrule\noalign{}
\endlastfoot
\textbf{title} & Title of the presentation & content &
\texttt{\ {[}Titre\ de\ la\ prez{]}\ } \\
\textbf{title-visual} & Content shown at the right of the title slide &
content & none \\
\textbf{subtitle} & Subtitle of the presentation & content &
\texttt{\ {[}Sous-titre{]}\ } \\
\textbf{insa} & INSA name ( \texttt{\ rennes\ } , \texttt{\ hdf\ } …)
& str & \texttt{\ "rennes"\ } \\
\end{longtable}

If you assign a content to \texttt{\ title-visual\ } , the title slide
will automatically switch layout to the “visual� one from the
graphic charter. If you do not assign a visual content, the title slide
will only contain the title and subtitle and will choose the simple
layout.

\subsubsection{Section slide}\label{section-slide}

A section slide is automatically created when you put a level-1 header
in your markup. For example:

\begin{Shaded}
\begin{Highlighting}[]
\NormalTok{= Slide section}
\NormalTok{Blablabla}
\end{Highlighting}
\end{Shaded}

Will create a section slide with the title “Slide section� and will
be followed by a content slide containing “Blablabla�.

If you want to put a subtitle in your section slide, you must
explicitely use the \texttt{\ section-slide\ } function like so:

\begin{Shaded}
\begin{Highlighting}[]
\NormalTok{\#section{-}slide[Titre de section][Description de section]}
\end{Highlighting}
\end{Shaded}

\subsection{Fonts}\label{fonts}

The graphic charter recommends the fonts \textbf{League Spartan} for
headings and \textbf{Source Serif} for regular text. To have the best
look, you should install those fonts.

\begin{quote}
You can download the fonts from
\href{https://github.com/SkytAsul/INSA-Typst-Template/tree/main/fonts}{here}
.
\end{quote}

To behave correctly on computers lacking those specific fonts, this
template will automatically fallback to similar ones:

\begin{itemize}
\tightlist
\item
  \textbf{League Spartan} -\textgreater{} \textbf{Arial} (approved by
  INSA’s graphic charter, by default in Windows) -\textgreater{}
  \textbf{Liberation Sans} (by default in most Linux)
\item
  \textbf{Source Serif} -\textgreater{} \textbf{Source Serif 4}
  (downloadable for free) -\textgreater{} \textbf{Georgia} (approved by
  the graphic charter) -\textgreater{} \textbf{Linux Libertine} (default
  Typst font)
\end{itemize}

\subsubsection{Note on variable fonts}\label{note-on-variable-fonts}

If you want to install those fonts on your computer, Typst might not
recognize them if you install their \emph{Variable} versions. You should
install the static versions ( \textbf{League Spartan Bold} and most
versions of \textbf{Source Serif} ).

Keep an eye on \href{https://github.com/typst/typst/issues/185}{the
issue in Typst bug tracker} to see when variable fonts will be used!

\subsection{Notes}\label{notes}

This template is being developed by Youenn LE JEUNE from the INSA de
Rennes in \href{https://github.com/SkytAsul/INSA-Typst-Template}{this
repository} .

For now it includes assets from the graphic charters of those INSAs:

\begin{itemize}
\tightlist
\item
  Rennes ( \texttt{\ rennes\ } )
\item
  Hauts de France ( \texttt{\ hdf\ } )
\item
  Centre Val de Loire ( \texttt{\ cvl\ } ) Users from other INSAs can
  open a pull request on the repository with the assets for their INSA.
\end{itemize}

If you have any other feature request, open an issue on the repository.

\subsection{License}\label{license}

The typst template is licensed under the
\href{https://github.com/SkytAsul/INSA-Typst-Template/blob/main/LICENSE}{MIT
license} . This does \emph{not} apply to the image assets. Those image
files are property of Groupe INSA.

\subsection{Changelog}\label{changelog}

\subsubsection{0.1.1}\label{section}

\begin{itemize}
\tightlist
\item
  Added INSA CVL assets
\end{itemize}

\subsubsection{0.1.0}\label{section-1}

\begin{itemize}
\tightlist
\item
  Created the template
\end{itemize}

\href{/app?template=silky-slides-insa&version=0.1.1}{Create project in
app}

\subsubsection{How to use}\label{how-to-use}

Click the button above to create a new project using this template in
the Typst app.

You can also use the Typst CLI to start a new project on your computer
using this command:

\begin{verbatim}
typst init @preview/silky-slides-insa:0.1.1
\end{verbatim}

\includesvg[width=0.16667in,height=0.16667in]{/assets/icons/16-copy.svg}

\subsubsection{About}\label{about}

\begin{description}
\tightlist
\item[Author :]
SkytAsul
\item[License:]
MIT
\item[Current version:]
0.1.1
\item[Last updated:]
November 21, 2024
\item[First released:]
October 16, 2024
\item[Archive size:]
227 kB
\href{https://packages.typst.org/preview/silky-slides-insa-0.1.1.tar.gz}{\pandocbounded{\includesvg[keepaspectratio]{/assets/icons/16-download.svg}}}
\item[Repository:]
\href{https://github.com/SkytAsul/INSA-Typst-Template}{GitHub}
\item[Discipline s :]
\begin{itemize}
\tightlist
\item[]
\item
  \href{https://typst.app/universe/search/?discipline=engineering}{Engineering}
\item
  \href{https://typst.app/universe/search/?discipline=computer-science}{Computer
  Science}
\item
  \href{https://typst.app/universe/search/?discipline=mathematics}{Mathematics}
\item
  \href{https://typst.app/universe/search/?discipline=physics}{Physics}
\item
  \href{https://typst.app/universe/search/?discipline=education}{Education}
\end{itemize}
\item[Categor y :]
\begin{itemize}
\tightlist
\item[]
\item
  \pandocbounded{\includesvg[keepaspectratio]{/assets/icons/16-presentation.svg}}
  \href{https://typst.app/universe/search/?category=presentation}{Presentation}
\end{itemize}
\end{description}

\subsubsection{Where to report issues?}\label{where-to-report-issues}

This template is a project of SkytAsul . Report issues on
\href{https://github.com/SkytAsul/INSA-Typst-Template}{their repository}
. You can also try to ask for help with this template on the
\href{https://forum.typst.app}{Forum} .

Please report this template to the Typst team using the
\href{https://typst.app/contact}{contact form} if you believe it is a
safety hazard or infringes upon your rights.

\phantomsection\label{versions}
\subsubsection{Version history}\label{version-history}

\begin{longtable}[]{@{}ll@{}}
\toprule\noalign{}
Version & Release Date \\
\midrule\noalign{}
\endhead
\bottomrule\noalign{}
\endlastfoot
0.1.1 & November 21, 2024 \\
\href{https://typst.app/universe/package/silky-slides-insa/0.1.0/}{0.1.0}
& October 16, 2024 \\
\end{longtable}

Typst GmbH did not create this template and cannot guarantee correct
functionality of this template or compatibility with any version of the
Typst compiler or app.


\section{Package List LaTeX/academic-conf-pre.tex}
\title{typst.app/universe/package/academic-conf-pre}

\phantomsection\label{banner}
\phantomsection\label{template-thumbnail}
\pandocbounded{\includegraphics[keepaspectratio]{https://packages.typst.org/preview/thumbnails/academic-conf-pre-0.1.0-small.webp}}

\section{academic-conf-pre}\label{academic-conf-pre}

{ 0.1.0 }

Slide Theme for Acadmic Presentations in Australia

\href{/app?template=academic-conf-pre&version=0.1.0}{Create project in
app}

\phantomsection\label{readme}
\subsection{1. Introduction}\label{introduction}

This is a template for \textbf{academic conference presentations} . It
is designed for the use in Typst which is simplier and more
user-friendly than LaTeX.

\subsection{2. How to use this template}\label{how-to-use-this-template}

themes documents are under \textbf{/themes/}

examples documents are under \textbf{/examples/}

if you don’t want to redesign the template, just follow the typ files
under the examples.

\subsection{3. How it looks}\label{how-it-looks}

The colors are entirely controllable, and I have provided three
relatively comfortable color schemes with \textbf{green} , \textbf{blue}
, and \textbf{red} as the base tones, which can be easily adjusted.

It is notable, there is a logo displayed in the center of the
presentation, and its \textbf{appearance} , \textbf{transparency} , and
\textbf{position} can be fully adjusted. Here, I use the University of
Sydney’s logo as an example.

You could see the PDFs under \textbf{examples/xxx.pdf}

\href{/app?template=academic-conf-pre&version=0.1.0}{Create project in
app}

\subsubsection{How to use}\label{how-to-use}

Click the button above to create a new project using this template in
the Typst app.

You can also use the Typst CLI to start a new project on your computer
using this command:

\begin{verbatim}
typst init @preview/academic-conf-pre:0.1.0
\end{verbatim}

\includesvg[width=0.16667in,height=0.16667in]{/assets/icons/16-copy.svg}

\subsubsection{About}\label{about}

\begin{description}
\tightlist
\item[Author :]
\href{mailto:isjun.liu@gmail.com}{JL-ghcoder}
\item[License:]
MIT
\item[Current version:]
0.1.0
\item[Last updated:]
November 5, 2024
\item[First released:]
November 5, 2024
\item[Archive size:]
1.29 MB
\href{https://packages.typst.org/preview/academic-conf-pre-0.1.0.tar.gz}{\pandocbounded{\includesvg[keepaspectratio]{/assets/icons/16-download.svg}}}
\item[Repository:]
\href{https://github.com/JL-ghcoder/Typst-Pre-Template}{GitHub}
\item[Categor y :]
\begin{itemize}
\tightlist
\item[]
\item
  \pandocbounded{\includesvg[keepaspectratio]{/assets/icons/16-presentation.svg}}
  \href{https://typst.app/universe/search/?category=presentation}{Presentation}
\end{itemize}
\end{description}

\subsubsection{Where to report issues?}\label{where-to-report-issues}

This template is a project of JL-ghcoder . Report issues on
\href{https://github.com/JL-ghcoder/Typst-Pre-Template}{their
repository} . You can also try to ask for help with this template on the
\href{https://forum.typst.app}{Forum} .

Please report this template to the Typst team using the
\href{https://typst.app/contact}{contact form} if you believe it is a
safety hazard or infringes upon your rights.

\phantomsection\label{versions}
\subsubsection{Version history}\label{version-history}

\begin{longtable}[]{@{}ll@{}}
\toprule\noalign{}
Version & Release Date \\
\midrule\noalign{}
\endhead
\bottomrule\noalign{}
\endlastfoot
0.1.0 & November 5, 2024 \\
\end{longtable}

Typst GmbH did not create this template and cannot guarantee correct
functionality of this template or compatibility with any version of the
Typst compiler or app.


\section{Package List LaTeX/untypsignia.tex}
\title{typst.app/universe/package/untypsignia}

\phantomsection\label{banner}
\section{untypsignia}\label{untypsignia}

{ 0.1.1 }

Unofficial typesetter\textquotesingle s insignia emulations

\phantomsection\label{readme}
The \texttt{\ untypsignia\ } is a 3rd-party, unofficial, unendorsed
Typst package that exposes commands for rendering, as
\texttt{\ content\ } texts, some typesetters names in a stylized
fashion, emulating their respective \emph{insignia} , i.e., marks by
which they are known.

\subsection{Name}\label{name}

The package name is a blend of:

\begin{itemize}
\tightlist
\item
  “un�, from “unofficial�,
\item
  “typ�, from “Typst�, and
\item
  “signia�, from “insignia�, which means marks by which anything
  is known.
\end{itemize}

\subsection{Description}\label{description}

The typical use case of \texttt{\ untypsignia\ } in Typst is to emulate
a given typesetting system’s mark, if available, when referring to
them, in sentences like: “This document is typeset in \texttt{\ XYZ\ }
�, as traditionally done in \texttt{\ TeX\ } systems and derivatives
thereof.

Currently available insignia emulations include:

\begin{itemize}
\tightlist
\item
  \texttt{\ TeX\ } ,
\item
  \texttt{\ LaTeX\ } , and
\item
  \texttt{\ Typst\ } (see below)
\end{itemize}

Despite there’s no such a thing as a Typst “official� typography,
according to this post on
\href{https://discord.com/channels/1054443721975922748/1054443722592497796/1107039477714665522}{Discord}
, it can be typeset with “whatever font� the surrounding text is
being typeset. Moreover, Typst
\href{https://typst.app/legal/brand/}{branding page} requires
capitalization of the initial “T� whenever the name is used in
prose. Therefore, the “Typst� support in this package is a mere,
still unofficial, implementation of the capitalization of “Typst� in
the currently used font.

\subsection{Font Requirements}\label{font-requirements}

For the \texttt{\ TeX\ } system and it’s derivatives, the
\texttt{\ "New\ Computer\ Modern"\ } font is required.

\subsection{Usage}\label{usage}

The package exposes the following few, parameterless, functions:

\begin{itemize}
\tightlist
\item
  \texttt{\ \#texmark()\ } ,
\item
  \texttt{\ \#latexmark()\ } , and
\item
  \texttt{\ \#typstmark()\ } .
\end{itemize}

Except for the \texttt{\ \#typstmark()\ } , each such command outputs
their respective namesake signus emulation, in the document’s current
\texttt{\ text\ } settings, with the exception of font â€'' meaning text
size, color, etc… will apply to the signus emulation.

Aditionally, the signus emulation is produced, as \texttt{\ contexts\ }
text inside a \texttt{\ box\ } â€'' hence not images â€'' so as to avoid
hyphenation to take place. This also applies to the
\texttt{\ \#typstmark()\ } function, for lack of specific guidance, and
also because “Typst� is a short word.

\subsection{Example}\label{example}

\begin{Shaded}
\begin{Highlighting}[]
\NormalTok{\#set page(width: auto, height: auto, margin: 12pt, fill: rgb("19181f"))}
\NormalTok{\#set par(leading: 1.5em)}
\NormalTok{\#set text(font: "Rouge Script", fill: rgb("80f4b6"))}

\NormalTok{\#import "@preview/untypsignia:0.1.1": *}

\NormalTok{\#let say() = [I prefer \#typstmark() over \#texmark() or \#latexmark().]}

\NormalTok{\#for sz in (20, 16, 14, 12, 10, 8) \{}
\NormalTok{  set text(size: sz * 1pt)}
\NormalTok{  say()}
\NormalTok{  linebreak()}
\NormalTok{\}}
\end{Highlighting}
\end{Shaded}

This example results in a 1-page document like this one:

\pandocbounded{\includegraphics[keepaspectratio]{https://raw.githubusercontent.com/cnaak/untypsignia.typ/86b221379931edcbfa91b50159a4ff930ecbec47/thumbnail.png}}

\subsection{Citing}\label{citing}

This package can be cited with the following bibliography database
entry:

\begin{Shaded}
\begin{Highlighting}[]
\FunctionTok{untypsignia{-}package}\KeywordTok{:}
\AttributeTok{  }\FunctionTok{type}\KeywordTok{:}\AttributeTok{ Web}
\AttributeTok{  }\FunctionTok{author}\KeywordTok{:}\AttributeTok{ Naaktgeboren, C.}
\AttributeTok{  }\FunctionTok{title}\KeywordTok{:}
\AttributeTok{    }\FunctionTok{value}\KeywordTok{:}\AttributeTok{ }\StringTok{"untypsignia: unofficial typesetter\textquotesingle{}s insignia emulations"}
\AttributeTok{  }\FunctionTok{url}\KeywordTok{:}\AttributeTok{ https://github.com/cnaak/untypsignia.typ}
\AttributeTok{  }\FunctionTok{version}\KeywordTok{:}\AttributeTok{ }\FloatTok{0.1.1}
\AttributeTok{  }\FunctionTok{date}\KeywordTok{:}\AttributeTok{ 2024{-}08}
\end{Highlighting}
\end{Shaded}

\subsubsection{How to add}\label{how-to-add}

Copy this into your project and use the import as
\texttt{\ untypsignia\ }

\begin{verbatim}
#import "@preview/untypsignia:0.1.1"
\end{verbatim}

\includesvg[width=0.16667in,height=0.16667in]{/assets/icons/16-copy.svg}

Check the docs for
\href{https://typst.app/docs/reference/scripting/\#packages}{more
information on how to import packages} .

\subsubsection{About}\label{about}

\begin{description}
\tightlist
\item[Author :]
Naaktgeboren, C.
\item[License:]
MIT
\item[Current version:]
0.1.1
\item[Last updated:]
August 21, 2024
\item[First released:]
August 14, 2024
\item[Minimum Typst version:]
0.11.1
\item[Archive size:]
2.13 kB
\href{https://packages.typst.org/preview/untypsignia-0.1.1.tar.gz}{\pandocbounded{\includesvg[keepaspectratio]{/assets/icons/16-download.svg}}}
\item[Discipline :]
\begin{itemize}
\tightlist
\item[]
\item
  \href{https://typst.app/universe/search/?discipline=computer-science}{Computer
  Science}
\end{itemize}
\item[Categor ies :]
\begin{itemize}
\tightlist
\item[]
\item
  \pandocbounded{\includesvg[keepaspectratio]{/assets/icons/16-chart.svg}}
  \href{https://typst.app/universe/search/?category=visualization}{Visualization}
\item
  \pandocbounded{\includesvg[keepaspectratio]{/assets/icons/16-hammer.svg}}
  \href{https://typst.app/universe/search/?category=utility}{Utility}
\item
  \pandocbounded{\includesvg[keepaspectratio]{/assets/icons/16-smile.svg}}
  \href{https://typst.app/universe/search/?category=fun}{Fun}
\end{itemize}
\end{description}

\subsubsection{Where to report issues?}\label{where-to-report-issues}

This package is a project of Naaktgeboren, C. . You can also try to ask
for help with this package on the \href{https://forum.typst.app}{Forum}
.

Please report this package to the Typst team using the
\href{https://typst.app/contact}{contact form} if you believe it is a
safety hazard or infringes upon your rights.

\phantomsection\label{versions}
\subsubsection{Version history}\label{version-history}

\begin{longtable}[]{@{}ll@{}}
\toprule\noalign{}
Version & Release Date \\
\midrule\noalign{}
\endhead
\bottomrule\noalign{}
\endlastfoot
0.1.1 & August 21, 2024 \\
\href{https://typst.app/universe/package/untypsignia/0.1.0/}{0.1.0} &
August 14, 2024 \\
\end{longtable}

Typst GmbH did not create this package and cannot guarantee correct
functionality of this package or compatibility with any version of the
Typst compiler or app.


\section{Package List LaTeX/roremu.tex}
\title{typst.app/universe/package/roremu}

\phantomsection\label{banner}
\section{roremu}\label{roremu}

{ 0.1.0 }

æ---¥æœ¬èªžã?®ãƒ€ãƒŸãƒ¼ãƒ†ã‚­ã‚¹ãƒˆç''Ÿæˆ?(Lorem Ipsum)

\phantomsection\label{readme}
æ---¥æœ¬èªžã?®ãƒ€ãƒŸãƒ¼ãƒ†ã‚­ã‚¹ãƒˆï¼ˆLipsum)ç''Ÿæˆ?ツール。

Blind text (Lorem ipsum) generator for Japanese.

\subsection{ç''¨æ³• / Usage}\label{uxe7uxe6uxb3-usage}

\begin{Shaded}
\begin{Highlighting}[]
\NormalTok{\#import "@preview/roremu:0.1.0": roremu}

\NormalTok{\#roremu(8) \# 吾輩は猫である。}

\NormalTok{\#roremu(8, offset: 8) \#名前はまだ無い。}

\NormalTok{\#roremu(17, custom{-}text: "私はその人を常に先生と呼んでいた。")}
\end{Highlighting}
\end{Shaded}

\subsection{テキスト / Text
Source}\label{uxe3ux192uxe3uxe3uxb9uxe3ux192ux2c6-text-source}

�目漱石『
\href{https://ja.wikipedia.org/wiki/\%E5\%90\%BE\%E8\%BC\%A9\%E3\%81\%AF\%E7\%8C\%AB\%E3\%81\%A7\%E3\%81\%82\%E3\%82\%8B}{�輩�猫��る}
�(
\href{https://www.aozora.gr.jp/cards/000148/card789.html}{é?'空æ--‡åº«ç‰ˆ}
より抜粋ã€?ルãƒ``抜ã??)

\subsection{å??称ç''±æ?¥ / Why
“roremu�?}\label{uxe5uxe7uxe7uxe6-why-uxe2ux153roremuuxe2}

lorem「ロレãƒ~ã€?ã?®ãƒ­ãƒ¼ãƒžå­---表記。

“roremuâ€? is the romanization of ロレãƒ~ (lorem).

\subsection{ライセンス /
License}\label{uxe3ux192uxe3uxe3uxe3ux192uxb3uxe3uxb9-license}

Unlicense

\subsubsection{How to add}\label{how-to-add}

Copy this into your project and use the import as \texttt{\ roremu\ }

\begin{verbatim}
#import "@preview/roremu:0.1.0"
\end{verbatim}

\includesvg[width=0.16667in,height=0.16667in]{/assets/icons/16-copy.svg}

Check the docs for
\href{https://typst.app/docs/reference/scripting/\#packages}{more
information on how to import packages} .

\subsubsection{About}\label{about}

\begin{description}
\tightlist
\item[Author :]
mkpoli
\item[License:]
Unlicense
\item[Current version:]
0.1.0
\item[Last updated:]
January 23, 2024
\item[First released:]
January 23, 2024
\item[Archive size:]
7.83 kB
\href{https://packages.typst.org/preview/roremu-0.1.0.tar.gz}{\pandocbounded{\includesvg[keepaspectratio]{/assets/icons/16-download.svg}}}
\item[Repository:]
\href{https://github.com/mkpoli/roremu}{GitHub}
\end{description}

\subsubsection{Where to report issues?}\label{where-to-report-issues}

This package is a project of mkpoli . Report issues on
\href{https://github.com/mkpoli/roremu}{their repository} . You can also
try to ask for help with this package on the
\href{https://forum.typst.app}{Forum} .

Please report this package to the Typst team using the
\href{https://typst.app/contact}{contact form} if you believe it is a
safety hazard or infringes upon your rights.

\phantomsection\label{versions}
\subsubsection{Version history}\label{version-history}

\begin{longtable}[]{@{}ll@{}}
\toprule\noalign{}
Version & Release Date \\
\midrule\noalign{}
\endhead
\bottomrule\noalign{}
\endlastfoot
0.1.0 & January 23, 2024 \\
\end{longtable}

Typst GmbH did not create this package and cannot guarantee correct
functionality of this package or compatibility with any version of the
Typst compiler or app.


\section{Package List LaTeX/yagenda.tex}
\title{typst.app/universe/package/yagenda}

\phantomsection\label{banner}
\phantomsection\label{template-thumbnail}
\pandocbounded{\includegraphics[keepaspectratio]{https://packages.typst.org/preview/thumbnails/yagenda-0.1.0-small.webp}}

\section{yagenda}\label{yagenda}

{ 0.1.0 }

A tabular template for meeting agendas with agenda items defined in
Yaml.

\href{/app?template=yagenda&version=0.1.0}{Create project in app}

\phantomsection\label{readme}
A Typst template for meeting agendas using Yaml for agenda items. To get
started:

\begin{Shaded}
\begin{Highlighting}[]
\NormalTok{typst init @preview/yagenda:0.1.0}
\end{Highlighting}
\end{Shaded}

And edit the \texttt{\ main.typ\ } example. The data are drawn from
\texttt{\ agenda.yaml\ } .

\pandocbounded{\includegraphics[keepaspectratio]{https://github.com/typst/packages/raw/main/packages/preview/yagenda/0.1.0/thumbnail.png}}

\subsection{Contributing}\label{contributing}

PRs are welcome! And if you encounter any bugs or have any
requests/ideas, feel free to open an issue.

\subsection{Acknowledgements}\label{acknowledgements}

The Typst grid layout was designed by
\href{https://discord.com/channels/1054443721975922748/1219401775908655115}{PgSuper
on Discord} .

\href{/app?template=yagenda&version=0.1.0}{Create project in app}

\subsubsection{How to use}\label{how-to-use}

Click the button above to create a new project using this template in
the Typst app.

You can also use the Typst CLI to start a new project on your computer
using this command:

\begin{verbatim}
typst init @preview/yagenda:0.1.0
\end{verbatim}

\includesvg[width=0.16667in,height=0.16667in]{/assets/icons/16-copy.svg}

\subsubsection{About}\label{about}

\begin{description}
\tightlist
\item[Author :]
\href{https://github.com/baptiste}{baptiste}
\item[License:]
MPL-2.0
\item[Current version:]
0.1.0
\item[Last updated:]
April 8, 2024
\item[First released:]
April 8, 2024
\item[Archive size:]
9.17 kB
\href{https://packages.typst.org/preview/yagenda-0.1.0.tar.gz}{\pandocbounded{\includesvg[keepaspectratio]{/assets/icons/16-download.svg}}}
\item[Categor y :]
\begin{itemize}
\tightlist
\item[]
\item
  \pandocbounded{\includesvg[keepaspectratio]{/assets/icons/16-envelope.svg}}
  \href{https://typst.app/universe/search/?category=office}{Office}
\end{itemize}
\end{description}

\subsubsection{Where to report issues?}\label{where-to-report-issues}

This template is a project of baptiste . You can also try to ask for
help with this template on the \href{https://forum.typst.app}{Forum} .

Please report this template to the Typst team using the
\href{https://typst.app/contact}{contact form} if you believe it is a
safety hazard or infringes upon your rights.

\phantomsection\label{versions}
\subsubsection{Version history}\label{version-history}

\begin{longtable}[]{@{}ll@{}}
\toprule\noalign{}
Version & Release Date \\
\midrule\noalign{}
\endhead
\bottomrule\noalign{}
\endlastfoot
0.1.0 & April 8, 2024 \\
\end{longtable}

Typst GmbH did not create this template and cannot guarantee correct
functionality of this template or compatibility with any version of the
Typst compiler or app.


\section{Package List LaTeX/bob-draw.tex}
\title{typst.app/universe/package/bob-draw}

\phantomsection\label{banner}
\section{bob-draw}\label{bob-draw}

{ 0.1.0 }

svgbob for typst, powered by wasm

\phantomsection\label{readme}
svgbob for typst, powered by wasm

This package provides a typst plugin for rendering
\href{https://github.com/ivanceras/svgbob}{svgbob} diagrams.

\begin{Shaded}
\begin{Highlighting}[]
\NormalTok{\#import "@preview/bob{-}draw:0.1.0": *}
\NormalTok{\#render(\textasciigrave{}\textasciigrave{}\textasciigrave{}}
\NormalTok{         /\textbackslash{}\_/\textbackslash{}}
\NormalTok{bob {-}\textgreater{}  ( o.o )}
\NormalTok{         \textbackslash{} " /}
\NormalTok{  .{-}{-}{-}{-}{-}{-}/  /}
\NormalTok{ (        | |}
\NormalTok{  \textasciigrave{}====== o o}
\NormalTok{\textasciigrave{}\textasciigrave{}\textasciigrave{})}
\end{Highlighting}
\end{Shaded}

output:

\pandocbounded{\includesvg[keepaspectratio]{https://github.com/typst/packages/raw/main/packages/preview/bob-draw/0.1.0/examples/basic-example.svg}}

\subsection{Full example}\label{full-example}

\begin{Shaded}
\begin{Highlighting}[]
\NormalTok{\#import "@preview/bob{-}draw:0.1.0": *}
\NormalTok{\#show raw.where(lang: "bob"): it =\textgreater{} render(it)}

\NormalTok{\#let svg = bob2svg("\textless{}{-}{-}{-}\textgreater{}")}
\NormalTok{\#render("\textless{}{-}{-}{-}\textgreater{}")}
\NormalTok{\#render(}
\NormalTok{    \textasciigrave{}\textasciigrave{}\textasciigrave{}}
\NormalTok{      0       3  }
\NormalTok{       *{-}{-}{-}{-}{-}{-}{-}* }
\NormalTok{    1 /|    2 /| }
\NormalTok{     *{-}+{-}{-}{-}{-}{-}* | }
\NormalTok{     | |4    | |7}
\NormalTok{     | *{-}{-}{-}{-}{-}|{-}*}
\NormalTok{     |/      |/}
\NormalTok{     *{-}{-}{-}{-}{-}{-}{-}*}
\NormalTok{    5       6}
\NormalTok{    \textasciigrave{}\textasciigrave{}\textasciigrave{},}
\NormalTok{    width: 25\%,}
\NormalTok{)}

\NormalTok{\textasciigrave{}\textasciigrave{}\textasciigrave{}bob}
\NormalTok{"cats:"}
\NormalTok{ /\textbackslash{}\_/\textbackslash{}  /\textbackslash{}\_/\textbackslash{}  /\textbackslash{}\_/\textbackslash{}  /\textbackslash{}\_/\textbackslash{} }
\NormalTok{( o.o )( o.o )( o.o )( o.o )}
\NormalTok{\textasciigrave{}\textasciigrave{}\textasciigrave{}}
\end{Highlighting}
\end{Shaded}

\subsubsection{How to add}\label{how-to-add}

Copy this into your project and use the import as \texttt{\ bob-draw\ }

\begin{verbatim}
#import "@preview/bob-draw:0.1.0"
\end{verbatim}

\includesvg[width=0.16667in,height=0.16667in]{/assets/icons/16-copy.svg}

Check the docs for
\href{https://typst.app/docs/reference/scripting/\#packages}{more
information on how to import packages} .

\subsubsection{About}\label{about}

\begin{description}
\tightlist
\item[Author :]
Luca Ciucci
\item[License:]
MIT
\item[Current version:]
0.1.0
\item[Last updated:]
October 24, 2023
\item[First released:]
October 24, 2023
\item[Archive size:]
126 kB
\href{https://packages.typst.org/preview/bob-draw-0.1.0.tar.gz}{\pandocbounded{\includesvg[keepaspectratio]{/assets/icons/16-download.svg}}}
\item[Repository:]
\href{https://github.com/LucaCiucci/bob-typ}{GitHub}
\end{description}

\subsubsection{Where to report issues?}\label{where-to-report-issues}

This package is a project of Luca Ciucci . Report issues on
\href{https://github.com/LucaCiucci/bob-typ}{their repository} . You can
also try to ask for help with this package on the
\href{https://forum.typst.app}{Forum} .

Please report this package to the Typst team using the
\href{https://typst.app/contact}{contact form} if you believe it is a
safety hazard or infringes upon your rights.

\phantomsection\label{versions}
\subsubsection{Version history}\label{version-history}

\begin{longtable}[]{@{}ll@{}}
\toprule\noalign{}
Version & Release Date \\
\midrule\noalign{}
\endhead
\bottomrule\noalign{}
\endlastfoot
0.1.0 & October 24, 2023 \\
\end{longtable}

Typst GmbH did not create this package and cannot guarantee correct
functionality of this package or compatibility with any version of the
Typst compiler or app.


\section{Package List LaTeX/rivet.tex}
\title{typst.app/universe/package/rivet}

\phantomsection\label{banner}
\section{rivet}\label{rivet}

{ 0.1.0 }

Register / Instruction Visualizer \& Explainer Tool with Typst, using
CeTZ

\phantomsection\label{readme}
RIVET \emph{(Register / Instruction Visualizer \& Explainer Tool)} is a
\href{https://typst.app/}{Typst} package for visualizing binary
instructions or describing the contents of a register, using the
\href{https://typst.app/universe/package/cetz}{CeTZ} package.

It is based on the \href{https://git.kb28.ch/HEL/rivet/}{homonymous
Python script}

\subsection{Examples}\label{examples}

\begin{longtable}[]{@{}l@{}}
\toprule\noalign{}
\endhead
\bottomrule\noalign{}
\endlastfoot
\href{https://github.com/typst/packages/raw/main/packages/preview/rivet/0.1.0/gallery/example1.typ}{\includegraphics[width=10.41667in,height=\textheight,keepaspectratio]{https://github.com/typst/packages/raw/main/packages/preview/rivet/0.1.0/gallery/example1.png}} \\
A bit of eveything \\
\href{https://github.com/typst/packages/raw/main/packages/preview/rivet/0.1.0/gallery/example2.typ}{\includegraphics[width=10.41667in,height=\textheight,keepaspectratio]{https://github.com/typst/packages/raw/main/packages/preview/rivet/0.1.0/gallery/example2.png}} \\
RISC-V memory instructions (blueprint) \\
\end{longtable}

\emph{Click on the example image to jump to the code.}

\subsection{Usage}\label{usage}

For more information, see the
\href{https://github.com/typst/packages/raw/main/packages/preview/rivet/0.1.0/manual.pdf}{manual}

To use this package, simply import \texttt{\ schema\ } from
\href{https://typst.app/universe/package/rivet}{rivet} and call
\texttt{\ schema.load\ } to parse a schema description. Then use
\texttt{\ schema.render\ } to render it, et voilÃ~ !

\begin{Shaded}
\begin{Highlighting}[]
\NormalTok{\#import "@preview/rivet:0.1.0": schema}
\NormalTok{\#let doc = schema.load("path/to/schema.yaml")}
\NormalTok{\#schema.render(doc)}
\end{Highlighting}
\end{Shaded}

\subsubsection{How to add}\label{how-to-add}

Copy this into your project and use the import as \texttt{\ rivet\ }

\begin{verbatim}
#import "@preview/rivet:0.1.0"
\end{verbatim}

\includesvg[width=0.16667in,height=0.16667in]{/assets/icons/16-copy.svg}

Check the docs for
\href{https://typst.app/docs/reference/scripting/\#packages}{more
information on how to import packages} .

\subsubsection{About}\label{about}

\begin{description}
\tightlist
\item[Author :]
\href{https://git.kb28.ch/HEL}{Louis Heredero}
\item[License:]
Apache-2.0
\item[Current version:]
0.1.0
\item[Last updated:]
October 3, 2024
\item[First released:]
October 3, 2024
\item[Minimum Typst version:]
0.11.0
\item[Archive size:]
120 kB
\href{https://packages.typst.org/preview/rivet-0.1.0.tar.gz}{\pandocbounded{\includesvg[keepaspectratio]{/assets/icons/16-download.svg}}}
\item[Repository:]
\href{https://git.kb28.ch/HEL/rivet-typst}{git.kb28.ch}
\item[Categor y :]
\begin{itemize}
\tightlist
\item[]
\item
  \pandocbounded{\includesvg[keepaspectratio]{/assets/icons/16-chart.svg}}
  \href{https://typst.app/universe/search/?category=visualization}{Visualization}
\end{itemize}
\end{description}

\subsubsection{Where to report issues?}\label{where-to-report-issues}

This package is a project of Louis Heredero . Report issues on
\href{https://git.kb28.ch/HEL/rivet-typst}{their repository} . You can
also try to ask for help with this package on the
\href{https://forum.typst.app}{Forum} .

Please report this package to the Typst team using the
\href{https://typst.app/contact}{contact form} if you believe it is a
safety hazard or infringes upon your rights.

\phantomsection\label{versions}
\subsubsection{Version history}\label{version-history}

\begin{longtable}[]{@{}ll@{}}
\toprule\noalign{}
Version & Release Date \\
\midrule\noalign{}
\endhead
\bottomrule\noalign{}
\endlastfoot
0.1.0 & October 3, 2024 \\
\end{longtable}

Typst GmbH did not create this package and cannot guarantee correct
functionality of this package or compatibility with any version of the
Typst compiler or app.


\section{Package List LaTeX/colorful-boxes.tex}
\title{typst.app/universe/package/colorful-boxes}

\phantomsection\label{banner}
\section{colorful-boxes}\label{colorful-boxes}

{ 1.3.1 }

Predefined colorful boxes to spice up your document.

\phantomsection\label{readme}
Colorful boxes in \href{https://github.com/typst/typst}{Typst} .

Check out \href{https://typst.app/project/rp9q3upfc69bPUCbv0BjzX}{the
example project} to see all boxes in action

Current features include:

\begin{itemize}
\tightlist
\item
  a colorful box is in four different colors (black, red, blue, green)
\item
  a colorful box with a slanted headline
\item
  a box with a simple outline
\item
  a rotateable stickynote
\end{itemize}

\subsection{Colorbox}\label{colorbox}

\pandocbounded{\includegraphics[keepaspectratio]{https://github.com/typst/packages/raw/main/packages/preview/colorful-boxes/1.3.1/examples/colorbox.png}}

\subsubsection{Usage}\label{usage}

\begin{verbatim}
#colorbox(
  title: lorem(5),
  color: "blue",
  radius: 2pt,
  width: auto
)[
  #lorem(50)
]
\end{verbatim}

\subsection{Slanted Colorbox}\label{slanted-colorbox}

\pandocbounded{\includegraphics[keepaspectratio]{https://github.com/typst/packages/raw/main/packages/preview/colorful-boxes/1.3.1/examples/slanted-colorbox.png}}

\subsubsection{Usage}\label{usage-1}

\begin{verbatim}
#slanted-colorbox(
  title: lorem(5),
  color: "red",
  radius: 0pt,
  width: auto
)[
  #lorem(50)
]
\end{verbatim}

\subsection{Outlinebox}\label{outlinebox}

\pandocbounded{\includegraphics[keepaspectratio]{https://github.com/typst/packages/raw/main/packages/preview/colorful-boxes/1.3.1/examples/outline-colorbox.png}}

\subsubsection{Usage}\label{usage-2}

\begin{verbatim}
#outlinebox(
  title: lorem(5),
  width: auto,
  radius: 2pt,
  centering: false
)[
  #lorem(50)
]

#outlinebox(
  title: lorem(5),
  color: "green",
  width: auto,
  radius: 2pt,
  centering: true
)[
  #lorem(50)
]
\end{verbatim}

\subsection{Stickybox}\label{stickybox}

\pandocbounded{\includegraphics[keepaspectratio]{https://github.com/typst/packages/raw/main/packages/preview/colorful-boxes/1.3.1/examples/stickybox.png}}

\subsubsection{Usage}\label{usage-3}

\begin{verbatim}
#stickybox(
  rotation: 5deg,
  width: 5cm
)[
  #lorem(20)
]
\end{verbatim}

\subsubsection{How to add}\label{how-to-add}

Copy this into your project and use the import as
\texttt{\ colorful-boxes\ }

\begin{verbatim}
#import "@preview/colorful-boxes:1.3.1"
\end{verbatim}

\includesvg[width=0.16667in,height=0.16667in]{/assets/icons/16-copy.svg}

Check the docs for
\href{https://typst.app/docs/reference/scripting/\#packages}{more
information on how to import packages} .

\subsubsection{About}\label{about}

\begin{description}
\tightlist
\item[Author :]
\href{https://github.com/lkoehl}{Lukas Köhl}
\item[License:]
MIT
\item[Current version:]
1.3.1
\item[Last updated:]
March 16, 2024
\item[First released:]
August 6, 2023
\item[Archive size:]
3.09 kB
\href{https://packages.typst.org/preview/colorful-boxes-1.3.1.tar.gz}{\pandocbounded{\includesvg[keepaspectratio]{/assets/icons/16-download.svg}}}
\item[Repository:]
\href{https://github.com/lkoehl/typst-boxes}{GitHub}
\item[Categor y :]
\begin{itemize}
\tightlist
\item[]
\item
  \pandocbounded{\includesvg[keepaspectratio]{/assets/icons/16-package.svg}}
  \href{https://typst.app/universe/search/?category=components}{Components}
\end{itemize}
\end{description}

\subsubsection{Where to report issues?}\label{where-to-report-issues}

This package is a project of Lukas Köhl . Report issues on
\href{https://github.com/lkoehl/typst-boxes}{their repository} . You can
also try to ask for help with this package on the
\href{https://forum.typst.app}{Forum} .

Please report this package to the Typst team using the
\href{https://typst.app/contact}{contact form} if you believe it is a
safety hazard or infringes upon your rights.

\phantomsection\label{versions}
\subsubsection{Version history}\label{version-history}

\begin{longtable}[]{@{}ll@{}}
\toprule\noalign{}
Version & Release Date \\
\midrule\noalign{}
\endhead
\bottomrule\noalign{}
\endlastfoot
1.3.1 & March 16, 2024 \\
\href{https://typst.app/universe/package/colorful-boxes/1.2.0/}{1.2.0} &
September 13, 2023 \\
\href{https://typst.app/universe/package/colorful-boxes/1.1.0/}{1.1.0} &
August 19, 2023 \\
\href{https://typst.app/universe/package/colorful-boxes/1.0.0/}{1.0.0} &
August 6, 2023 \\
\end{longtable}

Typst GmbH did not create this package and cannot guarantee correct
functionality of this package or compatibility with any version of the
Typst compiler or app.


\section{Package List LaTeX/casual-szu-report.tex}
\title{typst.app/universe/package/casual-szu-report}

\phantomsection\label{banner}
\phantomsection\label{template-thumbnail}
\pandocbounded{\includegraphics[keepaspectratio]{https://packages.typst.org/preview/thumbnails/casual-szu-report-0.1.0-small.webp}}

\section{casual-szu-report}\label{casual-szu-report}

{ 0.1.0 }

A template for SZU course reports.

\href{/app?template=casual-szu-report&version=0.1.0}{Create project in
app}

\phantomsection\label{readme}
A Typst template for SZU course reports.

\subsection{Usage}\label{usage}

Example is at \texttt{\ template/main.typ\ } . TLDR:

\begin{Shaded}
\begin{Highlighting}[]
\NormalTok{\#import "lib.typ": template}

\NormalTok{\#show: template.with(}
\NormalTok{  course{-}title: [养鸡学习],}
\NormalTok{  experiment{-}title: [养鸡],}
\NormalTok{  faculty: [养鸡学院],}
\NormalTok{  major: [智能养鸡],}
\NormalTok{  instructor: [鸡老师],}
\NormalTok{  reporter: [鸡],}
\NormalTok{  student{-}id: [4144010590],}
\NormalTok{  class: [养鸡99班],}
\NormalTok{  experiment{-}date: datetime(year: 1983, month: 9, day: 27),}
\NormalTok{  features: (}
\NormalTok{    "Bibliography": "template/refs.bib",}
\NormalTok{  ),}
\NormalTok{)}
\NormalTok{// Start here}
\end{Highlighting}
\end{Shaded}

Features:

\begin{enumerate}
\tightlist
\item
  \texttt{\ Bibliography\ } : Bibliography file path.
\item
  \texttt{\ FontFamily\ } : Custom font family. Default:
  \texttt{\ ("Noto\ Serif",\ "Noto\ Serif\ CJK\ SC")\ }
\item
  \texttt{\ CitationStyle\ } : Citation Style supported by Typst.
  Default: \texttt{\ gb-7714-2015-numeric\ }
\end{enumerate}

\begin{itemize}
\tightlist
\item
  Only work if have \texttt{\ Bibliography\ } specified.
\end{itemize}

\begin{enumerate}
\setcounter{enumi}{3}
\tightlist
\item
  \texttt{\ CourseID\ } : Add a Course ID box on top-right of the cover.
  NOT YET IMPLEMENTED.
\end{enumerate}

\subsection{Method}\label{method}

The template will traverse body content, and split it into groups
according to Heading-1 layout. Each group content will be wrapped with
\texttt{\ table.cell\ } . So all content will be wrapped in container,
you can’t use \texttt{\ pagebreak()\ } in your body content.

\subsection{Warning}\label{warning}

This is not a serious work and may have some rough edges. And reports
from different faculties isn’t entirely uniform. Be careful when using
it.

\href{/app?template=casual-szu-report&version=0.1.0}{Create project in
app}

\subsubsection{How to use}\label{how-to-use}

Click the button above to create a new project using this template in
the Typst app.

You can also use the Typst CLI to start a new project on your computer
using this command:

\begin{verbatim}
typst init @preview/casual-szu-report:0.1.0
\end{verbatim}

\includesvg[width=0.16667in,height=0.16667in]{/assets/icons/16-copy.svg}

\subsubsection{About}\label{about}

\begin{description}
\tightlist
\item[Author :]
\href{mailto:jiang131072@gmail.com}{Keaton Jiang}
\item[License:]
MIT
\item[Current version:]
0.1.0
\item[Last updated:]
September 8, 2024
\item[First released:]
September 8, 2024
\item[Archive size:]
36.6 kB
\href{https://packages.typst.org/preview/casual-szu-report-0.1.0.tar.gz}{\pandocbounded{\includesvg[keepaspectratio]{/assets/icons/16-download.svg}}}
\item[Repository:]
\href{https://github.com/jiang131072/casual-szu-report}{GitHub}
\item[Categor y :]
\begin{itemize}
\tightlist
\item[]
\item
  \pandocbounded{\includesvg[keepaspectratio]{/assets/icons/16-speak.svg}}
  \href{https://typst.app/universe/search/?category=report}{Report}
\end{itemize}
\end{description}

\subsubsection{Where to report issues?}\label{where-to-report-issues}

This template is a project of Keaton Jiang . Report issues on
\href{https://github.com/jiang131072/casual-szu-report}{their
repository} . You can also try to ask for help with this template on the
\href{https://forum.typst.app}{Forum} .

Please report this template to the Typst team using the
\href{https://typst.app/contact}{contact form} if you believe it is a
safety hazard or infringes upon your rights.

\phantomsection\label{versions}
\subsubsection{Version history}\label{version-history}

\begin{longtable}[]{@{}ll@{}}
\toprule\noalign{}
Version & Release Date \\
\midrule\noalign{}
\endhead
\bottomrule\noalign{}
\endlastfoot
0.1.0 & September 8, 2024 \\
\end{longtable}

Typst GmbH did not create this template and cannot guarantee correct
functionality of this template or compatibility with any version of the
Typst compiler or app.


\section{Package List LaTeX/oxifmt.tex}
\title{typst.app/universe/package/oxifmt}

\phantomsection\label{banner}
\section{oxifmt}\label{oxifmt}

{ 0.2.1 }

Convenient Rust-like string formatting in Typst

\phantomsection\label{readme}
A Typst library that brings convenient string formatting and
interpolation through the \texttt{\ strfmt\ } function. Its syntax is
taken directly from Rust’s \texttt{\ format!\ } syntax, so feel free
to read its page for more information (
\url{https://doc.rust-lang.org/std/fmt/} ); however, this README should
have enough information and examples for all expected uses of the
library. Only a few things aren’t supported from the Rust syntax, such
as the \texttt{\ p\ } (pointer) format type, or the \texttt{\ .*\ }
precision specifier.

A few extras (beyond the Rust-like syntax) will be added over time,
though (feel free to drop suggestions at the repository:
\url{https://github.com/PgBiel/typst-oxifmt} ). The first “extra� so
far is the \texttt{\ fmt-decimal-separator:\ "string"\ } parameter,
which lets you customize the decimal separator for decimal numbers
(floats) inserted into strings. E.g.
\texttt{\ strfmt("Result:\ \{\}",\ 5.8,\ fmt-decimal-separator:\ ",")\ }
will return the string \texttt{\ "Result:\ 5,8"\ } (comma instead of
dot). See more below.

\textbf{Compatible with:} \href{https://github.com/typst/typst}{Typst}
v0.4.0+

\subsection{Table of Contents}\label{table-of-contents}

\begin{itemize}
\tightlist
\item
  \href{https://github.com/typst/packages/raw/main/packages/preview/oxifmt/0.2.1/\#usage}{Usage}

  \begin{itemize}
  \tightlist
  \item
    \href{https://github.com/typst/packages/raw/main/packages/preview/oxifmt/0.2.1/\#formatting-options}{Formatting
    options}
  \item
    \href{https://github.com/typst/packages/raw/main/packages/preview/oxifmt/0.2.1/\#examples}{Examples}
  \item
    \href{https://github.com/typst/packages/raw/main/packages/preview/oxifmt/0.2.1/\#grammar}{Grammar}
  \end{itemize}
\item
  \href{https://github.com/typst/packages/raw/main/packages/preview/oxifmt/0.2.1/\#issues-and-contributing}{Issues
  and Contributing}
\item
  \href{https://github.com/typst/packages/raw/main/packages/preview/oxifmt/0.2.1/\#testing}{Testing}
\item
  \href{https://github.com/typst/packages/raw/main/packages/preview/oxifmt/0.2.1/\#changelog}{Changelog}
\item
  \href{https://github.com/typst/packages/raw/main/packages/preview/oxifmt/0.2.1/\#license}{License}
\end{itemize}

\subsection{Usage}\label{usage}

You can use this library through Typst’s package manager (for Typst
v0.6.0+):

\begin{Shaded}
\begin{Highlighting}[]
\NormalTok{\#import "@preview/oxifmt:0.2.1": strfmt}
\end{Highlighting}
\end{Shaded}

For older Typst versions, download the \texttt{\ oxifmt.typ\ } file
either from Releases or directly from the repository. Then, move it to
your project’s folder, and write at the top of your Typst file(s):

\begin{Shaded}
\begin{Highlighting}[]
\NormalTok{\#import "oxifmt.typ": strfmt}
\end{Highlighting}
\end{Shaded}

Doing the above will give you access to the main function provided by
this library ( \texttt{\ strfmt\ } ), which accepts a format string,
followed by zero or more replacements to insert in that string
(according to \texttt{\ \{...\}\ } formats inserted in that string), an
optional \texttt{\ fmt-decimal-separator\ } parameter, and returns the
formatted string, as described below.

Its syntax is almost identical to Rust’s \texttt{\ format!\ } (as
specified here: \url{https://doc.rust-lang.org/std/fmt/} ). You can
escape formats by duplicating braces ( \texttt{\ \{\{\ } and
\texttt{\ \}\}\ } become \texttt{\ \{\ } and \texttt{\ \}\ } ). Here’s
an example (see more examples in the file
\texttt{\ tests/strfmt-tests.typ\ } ):

\begin{Shaded}
\begin{Highlighting}[]
\NormalTok{\#import "@preview/oxifmt:0.2.1": strfmt}

\NormalTok{\#let s = strfmt("I\textquotesingle{}m \{\}. I have \{num\} cars. I\textquotesingle{}m \{0\}. \{\} is \{\{cool\}\}.", "John", "Carl", num: 10)}
\NormalTok{\#assert.eq(s, "I\textquotesingle{}m John. I have 10 cars. I\textquotesingle{}m John. Carl is \{cool\}.")}
\end{Highlighting}
\end{Shaded}

Note that \texttt{\ \{\}\ } extracts positional arguments after the
string sequentially (the first \texttt{\ \{\}\ } extracts the first one,
the second \texttt{\ \{\}\ } extracts the second one, and so on), while
\texttt{\ \{0\}\ } , \texttt{\ \{1\}\ } , etc. will always extract the
first, the second etc. positional arguments after the string.
Additionally, \texttt{\ \{bananas\}\ } will extract the named argument
“bananas�.

\subsubsection{Formatting options}\label{formatting-options}

You can use \texttt{\ \{:spec\}\ } to customize your output. See the
Rust docs linked above for more info, but a summary is below.

(You may also want to check out the examples at
\href{https://github.com/typst/packages/raw/main/packages/preview/oxifmt/0.2.1/\#examples}{Examples}
.)

\begin{itemize}
\tightlist
\item
  Adding a \texttt{\ ?\ } at the end of \texttt{\ spec\ } (that is,
  writing e.g. \texttt{\ \{0:?\}\ } ) will call \texttt{\ repr()\ } to
  stringify your argument, instead of \texttt{\ str()\ } . Note that
  this only has an effect if your argument is a string, an integer, a
  float or a \texttt{\ label()\ } /
  \texttt{\ \textless{}label\textgreater{}\ } - for all other types
  (such as booleans or elements), \texttt{\ repr()\ } is always called
  (as \texttt{\ str()\ } is unsupported for those).

  \begin{itemize}
  \tightlist
  \item
    For strings, \texttt{\ ?\ } (and thus \texttt{\ repr()\ } ) has the
    effect of printing them with double quotes. For floats, this ensures
    a \texttt{\ .0\ } appears after it, even if it doesn’t have
    decimal digits. For integers, this doesn’t change anything.
    Finally, for labels, the \texttt{\ \textless{}label\textgreater{}\ }
    (with \texttt{\ ?\ } ) is printed as
    \texttt{\ \textless{}label\textgreater{}\ } instead of
    \texttt{\ label\ } .
  \item
    \textbf{TIP:} Prefer to always use \texttt{\ ?\ } when you’re
    inserting something that isn’t a string, number or label, in order
    to ensure consistent results even if the library eventually changes
    the non- \texttt{\ ?\ } representation.
  \end{itemize}
\item
  After the \texttt{\ :\ } , add e.g. \texttt{\ \_\textless{}8\ } to
  align the string to the left, padding it with as many \texttt{\ \_\ }
  s as necessary for it to be at least \texttt{\ 8\ } characters long
  (for example). Replace \texttt{\ \textless{}\ } by
  \texttt{\ \textgreater{}\ } for right alignment, or \texttt{\ \^{}\ }
  for center alignment. (If the \texttt{\ \_\ } is omitted, it defaults
  to ’ ’ (aligns with spaces).)

  \begin{itemize}
  \tightlist
  \item
    If you prefer to specify the minimum width (the \texttt{\ 8\ }
    there) as a separate argument to \texttt{\ strfmt\ } instead, you
    can specify \texttt{\ argument\$\ } in place of the width, which
    will extract it from the integer at \texttt{\ argument\ } . For
    example, \texttt{\ \_\^{}3\$\ } will center align the output with
    \texttt{\ \_\ } s, where the minimum width desired is specified by
    the fourth positional argument (index \texttt{\ 3\ } ), as an
    integer. This means that a call such as
    \texttt{\ strfmt("\{:\_\^{}3\$\}",\ 1,\ 2,\ 3,\ 4)\ } would produce
    \texttt{\ "\_\_1\_\_"\ } , as \texttt{\ 3\$\ } would evaluate to
    \texttt{\ 4\ } (the value at the fourth positional argument/index
    \texttt{\ 3\ } ). Similarly, \texttt{\ named\$\ } would take the
    width from the argument with name \texttt{\ named\ } , if it is an
    integer (otherwise, error).
  \end{itemize}
\item
  \textbf{For numbers:}

  \begin{itemize}
  \tightlist
  \item
    Specify \texttt{\ +\ } after the \texttt{\ :\ } to ensure zero or
    positive numbers are prefixed with \texttt{\ +\ } before them
    (instead of having no sign). \texttt{\ -\ } is also accepted but
    ignored (negative numbers always specify their sign anyways).
  \item
    Use something like \texttt{\ :09\ } to add zeroes to the left of the
    number until it has at least 9 digits / characters.

    \begin{itemize}
    \tightlist
    \item
      The \texttt{\ 9\ } here is also a width, so the same comment from
      before applies (you can add \texttt{\ \$\ } to take it from an
      argument to the \texttt{\ strfmt\ } function).
    \end{itemize}
  \item
    Use \texttt{\ :.5\ } to ensure your float is represented with 5
    decimal digits of precision (zeroes are added to the right if
    needed; otherwise, it is rounded, \textbf{not truncated} ).

    \begin{itemize}
    \tightlist
    \item
      Note that floating point inaccuracies can be sometimes observed
      here, which is an unfortunate current limitation.
    \item
      Similarly to \texttt{\ width\ } , the precision can also be
      specified via an argument with the \texttt{\ \$\ } syntax:
      \texttt{\ .5\$\ } will take the precision from the integer at
      argument number 5 (the sixth one), while \texttt{\ .test\$\ } will
      take it from the argument named \texttt{\ test\ } .
    \end{itemize}
  \item
    \textbf{Integers only:} Add \texttt{\ x\ } (lowercase hex) or
    \texttt{\ X\ } (uppercase) at the end of the \texttt{\ spec\ } to
    convert the number to hexadecimal. Also, \texttt{\ b\ } will convert
    it to binary, while \texttt{\ o\ } will convert to octal.

    \begin{itemize}
    \tightlist
    \item
      Specify a hashtag, e.g. \texttt{\ \#x\ } or \texttt{\ \#b\ } , to
      prepend the corresponding base prefix to the base-converted
      number, e.g. \texttt{\ 0xABC\ } instead of \texttt{\ ABC\ } .
    \end{itemize}
  \item
    Add \texttt{\ e\ } or \texttt{\ E\ } at the end of the
    \texttt{\ spec\ } to ensure the number is represented in scientific
    notation (with \texttt{\ e\ } or \texttt{\ E\ } as the exponent
    separator, respectively).
  \item
    For decimal numbers (floats), you can specify
    \texttt{\ fmt-decimal-separator:\ ","\ } to \texttt{\ strfmt\ } to
    have the decimal separator be a comma instead of a dot, for example.

    \begin{itemize}
    \tightlist
    \item
      To have this be the default, you can alias \texttt{\ strfmt\ } ,
      such as using
      \texttt{\ \#let\ strfmt\ =\ strfmt.with(fmt-decimal-separator:\ ",")\ }
      .
    \end{itemize}
  \item
    Number spec arguments (such as \texttt{\ .5\ } ) are ignored when
    the argument is not a number, but e.g. a string, even if it looks
    like a number (such as \texttt{\ "5"\ } ).
  \end{itemize}
\item
  Note that all spec arguments above \textbf{have to be specified in
  order} - if you mix up the order, it won’t work properly!

  \begin{itemize}
  \tightlist
  \item
    Check the grammar below for the proper order, but, in summary: fill
    (character) with align ( \texttt{\ \textless{}\ } ,
    \texttt{\ \textgreater{}\ } or \texttt{\ \^{}\ } ) -\textgreater{}
    sign ( \texttt{\ +\ } or \texttt{\ -\ } ) -\textgreater{}
    \texttt{\ \#\ } -\textgreater{} \texttt{\ 0\ } (for 0 left-padding
    of numbers) -\textgreater{} width (e.g. \texttt{\ 8\ } from
    \texttt{\ 08\ } or \texttt{\ 9\ } from \texttt{\ -\textless{}9\ } )
    -\textgreater{} \texttt{\ .precision\ } -\textgreater{} spec type (
    \texttt{\ ?\ } , \texttt{\ x\ } , \texttt{\ X\ } , \texttt{\ b\ } ,
    \texttt{\ o\ } , \texttt{\ e\ } , \texttt{\ E\ } )).
  \end{itemize}
\end{itemize}

Some examples:

\begin{Shaded}
\begin{Highlighting}[]
\NormalTok{\#import "@preview/oxifmt:0.2.1": strfmt}

\NormalTok{\#let s1 = strfmt("\{0:?\}, \{test:+012e\}, \{1:{-}\textless{}\#8x\}", "hi", {-}74, test: 569.4)}
\NormalTok{\#assert.eq(s1, "\textbackslash{}"hi\textbackslash{}", +00005.694e2, {-}0x4a{-}{-}{-}")}

\NormalTok{\#let s2 = strfmt("\{:\_\textgreater{}+11.5\}", 59.4)}
\NormalTok{\#assert.eq(s2, "\_\_+59.40000")}

\NormalTok{\#let s3 = strfmt("Dict: \{:!\textless{}10?\}", (a: 5))}
\NormalTok{\#assert.eq(s3, "Dict: (a: 5)!!!!")}
\end{Highlighting}
\end{Shaded}

\subsubsection{Examples}\label{examples}

\begin{itemize}
\tightlist
\item
  \textbf{Inserting labels, text and numbers into strings:}
\end{itemize}

\begin{Shaded}
\begin{Highlighting}[]
\NormalTok{\#import "@preview/oxifmt:0.2.1": strfmt}

\NormalTok{\#let s = strfmt("First: \{\}, Second: \{\}, Fourth: \{3\}, Banana: \{banana\} (brackets: \{\{escaped\}\})", 1, 2.1, 3, label("four"), banana: "Banana!!")}
\NormalTok{\#assert.eq(s, "First: 1, Second: 2.1, Fourth: four, Banana: Banana!! (brackets: \{escaped\})")}
\end{Highlighting}
\end{Shaded}

\begin{itemize}
\tightlist
\item
  \textbf{Forcing \texttt{\ repr()\ } with \texttt{\ \{:?\}\ }} (which
  adds quotes around strings, and other things - basically represents a
  Typst value):
\end{itemize}

\begin{Shaded}
\begin{Highlighting}[]
\NormalTok{\#import "@preview/oxifmt:0.2.1": strfmt}

\NormalTok{\#let s = strfmt("The value is: \{:?\} | Also the label is \{:?\}", "something", label("label"))}
\NormalTok{\#assert.eq(s, "The value is: \textbackslash{}"something\textbackslash{}" | Also the label is \textless{}label\textgreater{}")}
\end{Highlighting}
\end{Shaded}

\begin{itemize}
\tightlist
\item
  \textbf{Inserting other types than numbers and strings} (for now, they
  will always use \texttt{\ repr()\ } , even without
  \texttt{\ \{...:?\}\ } , although that is more explicit):
\end{itemize}

\begin{Shaded}
\begin{Highlighting}[]
\NormalTok{\#import "@preview/oxifmt:0.2.1": strfmt}

\NormalTok{\#let s = strfmt("Values: \{:?\}, \{1:?\}, \{stuff:?\}", (test: 500), ("a", 5.1), stuff: [a])}
\NormalTok{\#assert.eq(s, "Values: (test: 500), (\textbackslash{}"a\textbackslash{}", 5.1), [a]")}
\end{Highlighting}
\end{Shaded}

\begin{itemize}
\tightlist
\item
  \textbf{Padding to a certain width with characters:} Use
  \texttt{\ \{:x\textless{}8\}\ } , where \texttt{\ x\ } is the
  \textbf{character to pad with} (e.g. space or \texttt{\ \_\ } , but
  can be anything), \texttt{\ \textless{}\ } is the \textbf{alignment of
  the original text} relative to the padding (can be
  \texttt{\ \textless{}\ } for left aligned (padding goes to the right),
  \texttt{\ \textgreater{}\ } for right aligned (padded to its left) and
  \texttt{\ \^{}\ } for center aligned (padded at both left and right)),
  and \texttt{\ 8\ } is the \textbf{desired total width} (padding will
  add enough characters to reach this width; if the replacement string
  already has this width, no padding will be added):
\end{itemize}

\begin{Shaded}
\begin{Highlighting}[]
\NormalTok{\#import "@preview/oxifmt:0.2.1": strfmt}

\NormalTok{\#let s = strfmt("Left5 \{:{-}\textless{}5\}, Right6 \{:=\textgreater{}6\}, Center10 \{centered: \^{}10?\}, Left3 \{tleft:\_\textless{}3\}", "xx", 539, tleft: "okay", centered: [a])}
\NormalTok{\#assert.eq(s, "Left5 xx{-}{-}{-}, Right6 ===539, Center10     [a]    , Left3 okay")}
\NormalTok{// note how \textquotesingle{}okay\textquotesingle{} didn\textquotesingle{}t suffer any padding at all (it already had at least the desired total width).}
\end{Highlighting}
\end{Shaded}

\begin{itemize}
\tightlist
\item
  \textbf{Padding numbers with zeroes to the left:} It’s a similar
  functionality to the above, however you write \texttt{\ \{:08\}\ } for
  8 characters (for instance) - note that any characters in the
  number’s representation matter for width (including sign, dot and
  decimal part):
\end{itemize}

\begin{Shaded}
\begin{Highlighting}[]
\NormalTok{\#import "@preview/oxifmt:0.2.1": strfmt}

\NormalTok{\#let s = strfmt("Left{-}padded7 numbers: \{:07\} \{:07\} \{:07\} \{3:07\}", 123, {-}344, 44224059, 45.32)}
\NormalTok{\#assert.eq(s, "Left{-}padded7 numbers: 0000123 {-}000344 44224059 0045.32")}
\end{Highlighting}
\end{Shaded}

\begin{itemize}
\tightlist
\item
  \textbf{Defining padding-to width using parameters, not literals:} If
  you want the desired replacement width (the \texttt{\ 8\ } in
  \texttt{\ \{:08\}\ } or \texttt{\ \{:\ \^{}8\}\ } ) to be passed via
  parameter (instead of being hardcoded into the format string), you can
  specify \texttt{\ parameter\$\ } in place of the width, e.g.
  \texttt{\ \{:02\$\}\ } to take it from the third positional parameter,
  or \texttt{\ \{:a\textgreater{}banana\$\}\ } to take it from the
  parameter named \texttt{\ banana\ } - note that the chosen parameter
  \textbf{must be an integer} (desired total width):
\end{itemize}

\begin{Shaded}
\begin{Highlighting}[]
\NormalTok{\#import "@preview/oxifmt:0.2.1": strfmt}

\NormalTok{\#let s = strfmt("Padding depending on parameter: \{0:02$\} and \{0:a\textgreater{}banana$\}", 432, 0, 5, banana: 9)}
\NormalTok{\#assert.eq(s, "Padding depending on parameter: 00432 aaaaaa432")  // widths 5 and 9}
\end{Highlighting}
\end{Shaded}

\begin{itemize}
\tightlist
\item
  \textbf{Displaying \texttt{\ +\ } on positive numbers:} Just add a
  \texttt{\ +\ } at the “beginning�, i.e., before the
  \texttt{\ \#0\ } (if either is there), or after the custom fill and
  align (if it’s there and not \texttt{\ 0\ } - see
  \href{https://github.com/typst/packages/raw/main/packages/preview/oxifmt/0.2.1/\#grammar}{Grammar}
  for the exact positioning), like so:
\end{itemize}

\begin{Shaded}
\begin{Highlighting}[]
\NormalTok{\#import "@preview/oxifmt:0.2.1": strfmt}

\NormalTok{\#let s = strfmt("Some numbers: \{:+\} \{:+08\}; With fill and align: \{:\_\textless{}+8\}; Negative (no{-}op): \{neg:+\}", 123, 456, 4444, neg: {-}435)}
\NormalTok{\#assert.eq(s, "Some numbers: +123 +0000456; With fill and align: +4444\_\_\_; Negative (no{-}op): {-}435")}
\end{Highlighting}
\end{Shaded}

\begin{itemize}
\tightlist
\item
  \textbf{Converting numbers to bases 2, 8 and 16:} Use one of the
  following specifier types (i.e., characters which always go at the
  very end of the format): \texttt{\ b\ } (binary), \texttt{\ o\ }
  (octal), \texttt{\ x\ } (lowercase hexadecimal) or \texttt{\ X\ }
  (uppercase hexadecimal). You can also add a \texttt{\ \#\ } between
  \texttt{\ +\ } and \texttt{\ 0\ } (see the exact position at the
  \href{https://github.com/typst/packages/raw/main/packages/preview/oxifmt/0.2.1/\#grammar}{Grammar}
  ) to display a \textbf{base prefix} before the number (i.e.
  \texttt{\ 0b\ } for binary, \texttt{\ 0o\ } for octal and
  \texttt{\ 0x\ } for hexadecimal):
\end{itemize}

\begin{Shaded}
\begin{Highlighting}[]
\NormalTok{\#import "@preview/oxifmt:0.2.1": strfmt}

\NormalTok{\#let s = strfmt("Bases (10, 2, 8, 16(l), 16(U):) \{0\} \{0:b\} \{0:o\} \{0:x\} \{0:X\} | W/ prefixes and modifiers: \{0:\#b\} \{0:+\#09o\} \{0:\_\textgreater{}+\#9X\}", 124)}
\NormalTok{\#assert.eq(s, "Bases (10, 2, 8, 16(l), 16(U):) 124 1111100 174 7c 7C | W/ prefixes and modifiers: 0b1111100 +0o000174 \_\_\_\_+0x7C")}
\end{Highlighting}
\end{Shaded}

\begin{itemize}
\tightlist
\item
  \textbf{Picking float precision (right-extending with zeroes):} Add,
  at the end of the format (just before the spec type (such as
  \texttt{\ ?\ } ), if there’s any), either \texttt{\ .precision\ }
  (hardcoded, e.g. \texttt{\ .8\ } for 8 decimal digits) or
  \texttt{\ .parameter\$\ } (taking the precision value from the
  specified parameter, like with \texttt{\ width\ } ):
\end{itemize}

\begin{Shaded}
\begin{Highlighting}[]
\NormalTok{\#import "@preview/oxifmt:0.2.1": strfmt}

\NormalTok{\#let s = strfmt("\{0:.8\} \{0:.2$\} \{0:.potato$\}", 1.234, 0, 2, potato: 5)}
\NormalTok{\#assert.eq(s, "1.23400000 1.23 1.23400")}
\end{Highlighting}
\end{Shaded}

\begin{itemize}
\tightlist
\item
  \textbf{Scientific notation:} Use \texttt{\ e\ } (lowercase) or
  \texttt{\ E\ } (uppercase) as specifier types (can be combined with
  precision):
\end{itemize}

\begin{Shaded}
\begin{Highlighting}[]
\NormalTok{\#import "@preview/oxifmt:0.2.1": strfmt}

\NormalTok{\#let s = strfmt("\{0:e\} \{0:E\} \{0:+.9e\} | \{1:e\} | \{2:.4E\}", 124.2312, 50, {-}0.02)}
\NormalTok{\#assert.eq(s, "1.242312e2 1.242312E2 +1.242312000e2 | 5e1 | {-}2.0000E{-}2")}
\end{Highlighting}
\end{Shaded}

\begin{itemize}
\tightlist
\item
  \textbf{Customizing the decimal separator on floats:} Just specify
  \texttt{\ fmt-decimal-separator:\ ","\ } (comma as an example):
\end{itemize}

\begin{Shaded}
\begin{Highlighting}[]
\NormalTok{\#import "@preview/oxifmt:0.2.1": strfmt}

\NormalTok{\#let s = strfmt("\{0\} \{0:.6\} \{0:.5e\}", 1.432, fmt{-}decimal{-}separator: ",")}
\NormalTok{\#assert.eq(s, "1,432 1,432000 1,43200e0")}
\end{Highlighting}
\end{Shaded}

\subsubsection{Grammar}\label{grammar}

Here’s the grammar specification for valid format \texttt{\ spec\ } s
(in \texttt{\ \{name:spec\}\ } ), which is basically Rust’s format:

\begin{verbatim}
format_spec := [[fill]align][sign]['#']['0'][width]['.' precision]type
fill := character
align := '<' | '^' | '>'
sign := '+' | '-'
width := count
precision := count | '*'
type := '' | '?' | 'x?' | 'X?' | identifier
count := parameter | integer
parameter := argument '$'
\end{verbatim}

Note, however, that precision of type \texttt{\ .*\ } is not supported
yet and will raise an error.

\subsection{Issues and Contributing}\label{issues-and-contributing}

Please report any issues or send any contributions (through pull
requests) to the repository at
\url{https://github.com/PgBiel/typst-oxifmt}

\subsection{Testing}\label{testing}

If you wish to contribute, you may clone the repository and test this
package with the following commands (from the project root folder):

\begin{Shaded}
\begin{Highlighting}[]
\FunctionTok{git}\NormalTok{ clone https://github.com/PgBiel/typst{-}oxifmt}
\BuiltInTok{cd}\NormalTok{ typst{-}oxifmt/tests}
\ExtensionTok{typst}\NormalTok{ c strfmt{-}tests.typ }\AttributeTok{{-}{-}root}\NormalTok{ ..}
\end{Highlighting}
\end{Shaded}

The tests succeeded if you received no error messages from the last
command (please ensure you’re using a supported Typst version).

\subsection{Changelog}\label{changelog}

\subsubsection{v0.2.1}\label{v0.2.1}

\begin{itemize}
\tightlist
\item
  Fixed formatting of UTF-8 strings. Before, strings with multi-byte
  UTF-8 codepoints would cause formatting inconsistencies or even
  crashes. (
  \href{https://github.com/PgBiel/typst-oxifmt/issues/6}{Issue \#6} )
\item
  Fixed an inconsistency in negative number formatting. Now, it will
  always print a regular hyphen (e.g. ‘-2’), which is consistent
  with Rust’s behavior; before, it would occasionally print a minus
  sign instead (as observed in a comment to
  \href{https://github.com/PgBiel/typst-oxifmt/issues/4}{Issue \#4} ).
\item
  Added compatibility with Typst 0.8.0’s new type system.
\end{itemize}

\subsubsection{v0.2.0}\label{v0.2.0}

\begin{itemize}
\tightlist
\item
  The package’s name is now \texttt{\ oxifmt\ } !
\item
  \texttt{\ oxifmt:0.2.0\ } is now available through Typst’s Package
  Manager! You can now write
  \texttt{\ \#import\ "@preview/oxifmt:0.2.0":\ strfmt\ } to use the
  library.
\item
  Greatly improved the README, adding a section for common examples.
\item
  Fixed negative numbers being formatted with two minus signs.
\item
  Fixed custom precision of floats not working when they are exact
  integers.
\end{itemize}

\subsubsection{v0.1.0}\label{v0.1.0}

\begin{itemize}
\tightlist
\item
  Initial release, added \texttt{\ strfmt\ } .
\end{itemize}

\subsection{License}\label{license}

MIT-0 license (see the \texttt{\ LICENSE\ } file).

\subsubsection{How to add}\label{how-to-add}

Copy this into your project and use the import as \texttt{\ oxifmt\ }

\begin{verbatim}
#import "@preview/oxifmt:0.2.1"
\end{verbatim}

\includesvg[width=0.16667in,height=0.16667in]{/assets/icons/16-copy.svg}

Check the docs for
\href{https://typst.app/docs/reference/scripting/\#packages}{more
information on how to import packages} .

\subsubsection{About}\label{about}

\begin{description}
\tightlist
\item[Author :]
\href{https://github.com/PgBiel}{PgBiel}
\item[License:]
MIT-0
\item[Current version:]
0.2.1
\item[Last updated:]
May 6, 2024
\item[First released:]
August 2, 2023
\item[Archive size:]
12.0 kB
\href{https://packages.typst.org/preview/oxifmt-0.2.1.tar.gz}{\pandocbounded{\includesvg[keepaspectratio]{/assets/icons/16-download.svg}}}
\item[Repository:]
\href{https://github.com/PgBiel/typst-oxifmt}{GitHub}
\end{description}

\subsubsection{Where to report issues?}\label{where-to-report-issues}

This package is a project of PgBiel . Report issues on
\href{https://github.com/PgBiel/typst-oxifmt}{their repository} . You
can also try to ask for help with this package on the
\href{https://forum.typst.app}{Forum} .

Please report this package to the Typst team using the
\href{https://typst.app/contact}{contact form} if you believe it is a
safety hazard or infringes upon your rights.

\phantomsection\label{versions}
\subsubsection{Version history}\label{version-history}

\begin{longtable}[]{@{}ll@{}}
\toprule\noalign{}
Version & Release Date \\
\midrule\noalign{}
\endhead
\bottomrule\noalign{}
\endlastfoot
0.2.1 & May 6, 2024 \\
\href{https://typst.app/universe/package/oxifmt/0.2.0/}{0.2.0} & August
2, 2023 \\
\end{longtable}

Typst GmbH did not create this package and cannot guarantee correct
functionality of this package or compatibility with any version of the
Typst compiler or app.


\section{Package List LaTeX/pointless-size.tex}
\title{typst.app/universe/package/pointless-size}

\phantomsection\label{banner}
\section{pointless-size}\label{pointless-size}

{ 0.1.0 }

中æ--‡å­---å?·çš„å?·æ•°åˆ¶å?Šå­---ä½``度é‡?å?•ä½? Chinese size system
(hÃ~o-system) and type-related measurements units

\phantomsection\label{readme}
中æ--‡å­---å?·çš„å?·æ•°åˆ¶å?Šå­---ä½``度é‡?å?•ä½?。 Chinese size system
(hÃ~o-system) and type-related measurements units.

\begin{Shaded}
\begin{Highlighting}[]
\NormalTok{\#import "@preview/pointless{-}size:0.1.0": zh, zihao}

\NormalTok{\#set text(size: zh(5)) // 五号(10.5pt)}
\NormalTok{// or}
\NormalTok{\#set text(zh(5))}
\NormalTok{\#show: zihao(5)}

\NormalTok{// 小号用负数表示 use negative numbers for small sizes }
\NormalTok{\#zh({-}4) // 小四(12pt)}
\NormalTok{\#zh(1) // 一号(26pt)}
\NormalTok{\#zh({-}1) // 小一(24pt)}
\NormalTok{\#zh("{-}0") // 小初(36pt)}
\NormalTok{\#zh(0) // 初号(42pt)}
\end{Highlighting}
\end{Shaded}

å­---å?·æ²¡æœ‰ç»Ÿä¸€è§„定,本åŒ\ldots 默认与 CTeXã€?MS
Wordã€?WPSã€?Adobe 的中æ--‡è§„则一致。 Chinese size systems were
not standardized. By default, this package is consistent with Chinese
rules of CTeX, MS Word, WPS, Adobe.

如想覆ç›--定义:If you want to override:

\begin{Shaded}
\begin{Highlighting}[]
\NormalTok{\#import "@preview/pointless{-}size:0.1.0": zh as \_zh}

\NormalTok{\#let zh = \_zh.with(overrides: ((7, 5.25pt),))}

\NormalTok{\#assert.eq(\_zh(7), 5.5pt)}
\NormalTok{\#assert.eq(zh(7), 5.25pt)}
\end{Highlighting}
\end{Shaded}

\subsection{相å\ldots³é``¾æŽ¥ Relevant
links}\label{uxe7uxe5uxb3uxe9uxbeuxe6ux17e-relevant-links}

\begin{quote}
{[}!TIP{]}

\begin{itemize}
\tightlist
\item
  âœ\ldots{} = 一致 consistent
\item
  ðŸ`ª = 与æ??述的规则之一一致 consistent with one of the
  described rules
\item
  🚸 = ä¸?完å\ldots¨ä¸€è‡´ not fully consistent
\end{itemize}
\end{quote}

\begin{itemize}
\item
  🚸
  \href{https://www.w3.org/International/clreq/\#considerations_in_designing_type_area}{§2.3.5
  基本版å¼?设计的注æ„?事项 - 中æ--‡æŽ'版需求 \textbar{} W3C
  ç¼--è¾`è?‰ç¨¿} (中/英){[}2024-09-13{]}

  \begin{quote}
  “å?·â€?ç''±äºŽå½``å¹´é‡`属活å­---å?„地厂家的规范ä¸?一而ä¸?尽相å?Œâ€¦â€¦ä¸?作为规范性规定。
  \end{quote}

  §2.3.5 Considerations when Designing the Type Area - Requirements for
  Chinese Text Layout \textbar{} W3C Editor’s Draft (Chinese \&
  English)

  \begin{quote}
  These hÃ~o-systems were not standardized by the various foundries in
  the past. …It is not normative information.
  \end{quote}
\item
  âœ\ldots 表25 中æ--‡å­---å?· -
  \href{http://mirrors.ctan.org/language/chinese/ctex/ctex.pdf}{CTeX
  v2.5.0 (2022-07-14) å®?集手册 \textbar{} CTAN} (中æ--‡ï¼‰

  Table 25 Chinese text size - Documentation of the package CTeX v2.5.10
  (2022-07-14) (Chinese)

  \url{https://github.com/CTeX-org/ctex-kit/blob/0fb196c42c56287403fecca6eb6b137c00167f40/ctex/ctex.dtx\#L9974-L9993}
\item
  ðŸ`ª
  \href{https://ccjktype.fonts.adobe.com/2009/04/post_1.html}{å­---ä½``度é‡?å?•ä½?
  - CJK Type Blog \textbar{} Adobe} (中/英){[}2009-04-02{]} (
  \href{https://archive.today/QxXuk}{archive.today} )

  Type-related Measurements Units (Chinese \& English)
\item
  âœ
  \href{https://www.wps.cn/learning/question/detail/id/2940}{如何转æ?¢å­---å?·ã€?ç£\ldots ã€?px?-
  技巧é---®ç­'' \textbar{} WPSå­¦å~‚} (中æ--‡ï¼‰{[}2020-05-07{]}

  How to convert between hÃ~o, point, and pixel? - Tech Q\&A \textbar{}
  WPS learning (Chinese)
\item
  \href{https://www.thetype.com/typechat/ep-135/}{\#135
  显明解行å?·å?·ç?? - å­---è°ˆå­---ç•\ldots{} \textbar{} The Type}
  (中æ--‡ï¼Œå¸¦æ--‡å­---说明的æ'­å®¢ï¼‰{[}2020-09-09{]} (
  \href{https://archive.today/qaG8D}{archive.today} )

  (Chinese, podcast with show notes)
\item
  \href{https://github.com/CTeX-org/ctex-kit/issues/543}{\#543 ctexsize:
  é‡?设å?„级å­---å?·å¤§å°? - CTeX-org/ctex-kit \textbar{} GitHub}
  (中æ--‡ï¼‰{[}2020-10-13{]}

  \#543 ctextsize: Redesign the font size system (Chinese)
\item
  âœ
  \href{https://std.samr.gov.cn/gb/search/gbDetailed?id=BBE32B661B7E8FC8E05397BE0A0AB906}{GB
  40070â€``2021 å„¿ç«¥é?'å°`å¹´å­¦ä¹~ç''¨å``?è¿`视防控å?«ç''Ÿè¦?求 -
  国家æ~‡å‡† \textbar{}
  å\ldots¨å›½æ~‡å‡†ä¿¡æ?¯å\ldots¬å\ldots±æœ?务平å?°}
  (中æ--‡ï¼‰{[}2021-02-20{]}

  å\ldots¶ä¸­ç''¨åˆ°äº†å?·æ•°åˆ¶ï¼Œä¾‹å¦‚4.3.1“å°?学一ã€?二年级ç''¨å­---åº''ä¸?å°?于16P(3å?·ï¼‰å­---â€?。总ç»``下æ?¥æ˜¯ä¸‰å?·
  16ptã€?å››å?· 14ptã€?å°?å›› 12ptã€?äº''å?· 10.5ptã€?å°?äº'' 9pt。

  GB 40070â€``2021 Hygienic requirements of study products for myopia
  prevention and control in children and adolescents - National
  standards \textbar{} National public service platform for standards
  information (Chinese)

  The standard uses hÃ~o-system, e.g. 4.3.1 “texts for grade 1/2 of
  primary school should not be less than 16P (size 3)�. To summarize,
  size 3 = 16pt, size 4 = 14pt, size small 4 = 12pt, size 5 = 10.5pt,
  size small 5 = 9pt.
\item
  ðŸ`ª
  \href{https://zh.wikipedia.org/wiki/\%E5\%AD\%97\%E5\%8F\%B7_(\%E5\%8D\%B0\%E5\%88\%B7)}{å­---å?·ï¼ˆå?°åˆ·ï¼‰\textbar{}
  维基百ç§`} (中æ--‡ï¼‰

  HÃ~o (typography) \textbar{} Wikipedia (Chinese)
\end{itemize}

\subsubsection{How to add}\label{how-to-add}

Copy this into your project and use the import as
\texttt{\ pointless-size\ }

\begin{verbatim}
#import "@preview/pointless-size:0.1.0"
\end{verbatim}

\includesvg[width=0.16667in,height=0.16667in]{/assets/icons/16-copy.svg}

Check the docs for
\href{https://typst.app/docs/reference/scripting/\#packages}{more
information on how to import packages} .

\subsubsection{About}\label{about}

\begin{description}
\tightlist
\item[Author :]
Y.D.X.
\item[License:]
MIT
\item[Current version:]
0.1.0
\item[Last updated:]
September 14, 2024
\item[First released:]
September 14, 2024
\item[Archive size:]
3.72 kB
\href{https://packages.typst.org/preview/pointless-size-0.1.0.tar.gz}{\pandocbounded{\includesvg[keepaspectratio]{/assets/icons/16-download.svg}}}
\item[Repository:]
\href{https://github.com/YDX-2147483647/typst-pointless-size}{GitHub}
\item[Categor ies :]
\begin{itemize}
\tightlist
\item[]
\item
  \pandocbounded{\includesvg[keepaspectratio]{/assets/icons/16-text.svg}}
  \href{https://typst.app/universe/search/?category=text}{Text}
\item
  \pandocbounded{\includesvg[keepaspectratio]{/assets/icons/16-world.svg}}
  \href{https://typst.app/universe/search/?category=languages}{Languages}
\end{itemize}
\end{description}

\subsubsection{Where to report issues?}\label{where-to-report-issues}

This package is a project of Y.D.X. . Report issues on
\href{https://github.com/YDX-2147483647/typst-pointless-size}{their
repository} . You can also try to ask for help with this package on the
\href{https://forum.typst.app}{Forum} .

Please report this package to the Typst team using the
\href{https://typst.app/contact}{contact form} if you believe it is a
safety hazard or infringes upon your rights.

\phantomsection\label{versions}
\subsubsection{Version history}\label{version-history}

\begin{longtable}[]{@{}ll@{}}
\toprule\noalign{}
Version & Release Date \\
\midrule\noalign{}
\endhead
\bottomrule\noalign{}
\endlastfoot
0.1.0 & September 14, 2024 \\
\end{longtable}

Typst GmbH did not create this package and cannot guarantee correct
functionality of this package or compatibility with any version of the
Typst compiler or app.


\section{Package List LaTeX/red-agora.tex}
\title{typst.app/universe/package/red-agora}

\phantomsection\label{banner}
\phantomsection\label{template-thumbnail}
\pandocbounded{\includegraphics[keepaspectratio]{https://packages.typst.org/preview/thumbnails/red-agora-0.1.1-small.webp}}

\section{red-agora}\label{red-agora}

{ 0.1.1 }

Quickly scaffold a report for your projects and internships at ENSIAS
and elsewhere

{ } Featured Template

\href{/app?template=red-agora&version=0.1.1}{Create project in app}

\phantomsection\label{readme}
A Typst template to quickly make reports for projects at ENSIAS. This
template was created based on our reports that we also made for our
projects.

\subsection{What does it provide?}\label{what-does-it-provide}

For now, it provides a first page style that matches the common reports
style used and encouraged at ENSIAS.

It also provides a style for first level headings to act as chapters.

More improvements will come soon.

\subsection{Usage}\label{usage}

\begin{Shaded}
\begin{Highlighting}[]
\NormalTok{\#import "@preview/red{-}agora:0.1.1": project}

\NormalTok{\#show: project.with(}
\NormalTok{  title: "Injecting a backdoor in the xz library and taking over NASA and SpaceX spaceship tracking servers (for education purposes only)",}
\NormalTok{  subtitle: "Second year internship report",}
\NormalTok{  authors: (}
\NormalTok{    "Amine Hadnane",}
\NormalTok{    "Mehdi Essalehi"}
\NormalTok{  ),}
\NormalTok{  school{-}logo: image("images/ENSIAS.svg"), // Replace with [] to remove the school logo}
\NormalTok{  company{-}logo: image("images/company.svg"),}
\NormalTok{  mentors: (}
\NormalTok{    "Pr. John Smith (Internal)",}
\NormalTok{    "Jane Doe (External)"}
\NormalTok{  ),}
\NormalTok{  jury: (}
\NormalTok{    "Pr. John Smith",}
\NormalTok{    "Pr. Jane Doe"}
\NormalTok{  ),}
\NormalTok{  branch: "Software Engineering",}
\NormalTok{  academic{-}year: "2077{-}2078",}
\NormalTok{  french: false // Use french instead of english}
\NormalTok{  footer{-}text: "ENSIAS" // Text used in left side of the footer}
\NormalTok{)}

\NormalTok{// Put then your content here}
\end{Highlighting}
\end{Shaded}

\subsection{Changelog}\label{changelog}

\textbf{0.1.0 - Initial release}

\begin{itemize}
\tightlist
\item
  First page style
\item
  Level 1 headings chapter style
\end{itemize}

\textbf{0.1.1}

\begin{itemize}
\tightlist
\item
  Fixed major issue where custom school \& company logos would throw an
  error
\item
  Added option to customize footer left side text (thus fixing the issue
  of it being hardcoded)
\end{itemize}

\href{/app?template=red-agora&version=0.1.1}{Create project in app}

\subsubsection{How to use}\label{how-to-use}

Click the button above to create a new project using this template in
the Typst app.

You can also use the Typst CLI to start a new project on your computer
using this command:

\begin{verbatim}
typst init @preview/red-agora:0.1.1
\end{verbatim}

\includesvg[width=0.16667in,height=0.16667in]{/assets/icons/16-copy.svg}

\subsubsection{About}\label{about}

\begin{description}
\tightlist
\item[Author s :]
\href{https://github.com/essmehdi}{Mehdi Essalehi} \&
\href{https://github.com/amin-hdn}{Amine Hadnane}
\item[License:]
MIT
\item[Current version:]
0.1.1
\item[Last updated:]
April 29, 2024
\item[First released:]
April 19, 2024
\item[Archive size:]
4.37 kB
\href{https://packages.typst.org/preview/red-agora-0.1.1.tar.gz}{\pandocbounded{\includesvg[keepaspectratio]{/assets/icons/16-download.svg}}}
\item[Repository:]
\href{https://github.com/essmehdi/ensias-report-template}{GitHub}
\item[Categor y :]
\begin{itemize}
\tightlist
\item[]
\item
  \pandocbounded{\includesvg[keepaspectratio]{/assets/icons/16-speak.svg}}
  \href{https://typst.app/universe/search/?category=report}{Report}
\end{itemize}
\end{description}

\subsubsection{Where to report issues?}\label{where-to-report-issues}

This template is a project of Mehdi Essalehi and Amine Hadnane . Report
issues on
\href{https://github.com/essmehdi/ensias-report-template}{their
repository} . You can also try to ask for help with this template on the
\href{https://forum.typst.app}{Forum} .

Please report this template to the Typst team using the
\href{https://typst.app/contact}{contact form} if you believe it is a
safety hazard or infringes upon your rights.

\phantomsection\label{versions}
\subsubsection{Version history}\label{version-history}

\begin{longtable}[]{@{}ll@{}}
\toprule\noalign{}
Version & Release Date \\
\midrule\noalign{}
\endhead
\bottomrule\noalign{}
\endlastfoot
0.1.1 & April 29, 2024 \\
\href{https://typst.app/universe/package/red-agora/0.1.0/}{0.1.0} &
April 19, 2024 \\
\end{longtable}

Typst GmbH did not create this template and cannot guarantee correct
functionality of this template or compatibility with any version of the
Typst compiler or app.


\section{Package List LaTeX/one-liner.tex}
\title{typst.app/universe/package/one-liner}

\phantomsection\label{banner}
\section{one-liner}\label{one-liner}

{ 0.1.0 }

Automatically adjust the text size to make it fit on one line filling
the available space.

\phantomsection\label{readme}
One-liner is a package containing a helper function to fit text to the
available width, without wrapping, by adjusting the text size based upon
the context. This is useful in templates where you don’t know the
length of text that is supposed to fit in specific locations in your
template.

\subsection{Example}\label{example}

In the current version(0.1.0) one-liner contains 1 function:
fit-to-width that can used as follows:

\begin{Shaded}
\begin{Highlighting}[]
\NormalTok{\#import "@preview/one{-}liner:0.1.0": fit{-}to{-}width }

\NormalTok{\#block(}
\NormalTok{  height: 3cm,}
\NormalTok{  width: 10cm,}
\NormalTok{  fill: luma(230),}
\NormalTok{  inset: 8pt,}
\NormalTok{  radius: 4pt,}
\NormalTok{  align(horizon + center,fit{-}to{-}width(lorem(2))),}
\NormalTok{)}
\end{Highlighting}
\end{Shaded}

Here we have a block of specific dimensions. Using fit-to-width will
change the font-size to the content passed to fit-to-width will fit the
full width without wrapping the content.

\pandocbounded{\includegraphics[keepaspectratio]{https://github.com/typst/packages/raw/main/packages/preview/one-liner/0.1.0/img/example1.png}}

\subsection{fit-to-width function}\label{fit-to-width-function}

Besides content the function has two parameters:

\emph{max-text-size} of type length. It has a default of 64pt. When
fit-to-width is limited by the max-text-size you will see that not the
entire width of space is used.

\emph{min-text-size} of type length. It has a default of 4pt. When
fit-to-width is limited by the min-text-size you will see that the text
will wrap, because the min-text-size is bigger than the size that would
be required to prevent wrapping.

\subsection{Disclaimer}\label{disclaimer}

This package was only tested in a few of my own documents and only to
fit text. Not tested with other content yet.

\subsubsection{How to add}\label{how-to-add}

Copy this into your project and use the import as \texttt{\ one-liner\ }

\begin{verbatim}
#import "@preview/one-liner:0.1.0"
\end{verbatim}

\includesvg[width=0.16667in,height=0.16667in]{/assets/icons/16-copy.svg}

Check the docs for
\href{https://typst.app/docs/reference/scripting/\#packages}{more
information on how to import packages} .

\subsubsection{About}\label{about}

\begin{description}
\tightlist
\item[Author :]
\href{https://github.com/mtolk}{Marco}
\item[License:]
MIT
\item[Current version:]
0.1.0
\item[Last updated:]
November 12, 2024
\item[First released:]
November 12, 2024
\item[Minimum Typst version:]
0.12.0
\item[Archive size:]
2.00 kB
\href{https://packages.typst.org/preview/one-liner-0.1.0.tar.gz}{\pandocbounded{\includesvg[keepaspectratio]{/assets/icons/16-download.svg}}}
\item[Repository:]
\href{https://github.com/mtolk/one-liner}{GitHub}
\item[Categor ies :]
\begin{itemize}
\tightlist
\item[]
\item
  \pandocbounded{\includesvg[keepaspectratio]{/assets/icons/16-layout.svg}}
  \href{https://typst.app/universe/search/?category=layout}{Layout}
\item
  \pandocbounded{\includesvg[keepaspectratio]{/assets/icons/16-text.svg}}
  \href{https://typst.app/universe/search/?category=text}{Text}
\item
  \pandocbounded{\includesvg[keepaspectratio]{/assets/icons/16-hammer.svg}}
  \href{https://typst.app/universe/search/?category=utility}{Utility}
\end{itemize}
\end{description}

\subsubsection{Where to report issues?}\label{where-to-report-issues}

This package is a project of Marco . Report issues on
\href{https://github.com/mtolk/one-liner}{their repository} . You can
also try to ask for help with this package on the
\href{https://forum.typst.app}{Forum} .

Please report this package to the Typst team using the
\href{https://typst.app/contact}{contact form} if you believe it is a
safety hazard or infringes upon your rights.

\phantomsection\label{versions}
\subsubsection{Version history}\label{version-history}

\begin{longtable}[]{@{}ll@{}}
\toprule\noalign{}
Version & Release Date \\
\midrule\noalign{}
\endhead
\bottomrule\noalign{}
\endlastfoot
0.1.0 & November 12, 2024 \\
\end{longtable}

Typst GmbH did not create this package and cannot guarantee correct
functionality of this package or compatibility with any version of the
Typst compiler or app.


\section{Package List LaTeX/efilrst.tex}
\title{typst.app/universe/package/efilrst}

\phantomsection\label{banner}
\section{efilrst}\label{efilrst}

{ 0.3.0 }

A simple referenceable list library for typst.

\phantomsection\label{readme}
A simple referenceable list library for Typst. If you ever wanted to
reference elements in a list by a key, this library is for you. The name
comes from “reflist� but sorted alphabetically because we are not
allowed to use descriptive names for packages in Typst
🤷��♂�.

\subsection{Example}\label{example}

\begin{Shaded}
\begin{Highlighting}[]
\NormalTok{\#import "@preview/efilrst:0.1.0" as efilrst}
\NormalTok{\#show ref: efilrst.show{-}rule}

\NormalTok{\#let constraint = efilrst.reflist.with(}
\NormalTok{  name: "Constraint", }
\NormalTok{  list{-}style: "C1.1.1)", }
\NormalTok{  ref{-}style: "C1.1.1")}

\NormalTok{\#constraint(}
\NormalTok{  counter{-}name: "continuable",}
\NormalTok{  [My cool constraint A],\textless{}c:a\textgreater{},}
\NormalTok{  [My also cool constraint B],\textless{}c:b\textgreater{},}
\NormalTok{  [My non{-}referenceable constraint C],}
\NormalTok{)}

\NormalTok{See how my @c:a is better than @c:b but not as cool as @c:e.}

\NormalTok{\#constraint(}
\NormalTok{  counter{-}name: "continuable",}
\NormalTok{  [We continue the list with D],\textless{}c:d\textgreater{},}
\NormalTok{  [And then add constraint E],\textless{}c:e\textgreater{},}
\NormalTok{)}

\NormalTok{\#constraint(}
\NormalTok{  [This is a new list!],\textless{}c:f\textgreater{},}
\NormalTok{  (}
\NormalTok{    [And it has a sublist!],\textless{}c:g\textgreater{},}
\NormalTok{    [With a constraint H],\textless{}c:h\textgreater{},}
\NormalTok{  )}
\NormalTok{)}

\NormalTok{\#constraint(}
\NormalTok{  [This is another list!],\textless{}c:i\textgreater{},}
\NormalTok{)}
\end{Highlighting}
\end{Shaded}

This generates the following output:

\pandocbounded{\includegraphics[keepaspectratio]{https://github.com/typst/packages/raw/main/packages/preview/efilrst/0.3.0/img/image.png}}

\subsection{License}\label{license}

This project is licensed under the MIT License - see the
\href{https://github.com/typst/packages/raw/main/packages/preview/efilrst/0.3.0/LICENSE}{LICENSE}
file for details.

\subsection{TODO}\label{todo}

\begin{itemize}
\tightlist
\item
  {[}x{]} Add continuation of lists through the \texttt{\ counter\ }
  function
\item
  {[}x{]} Add support for nested lists
\end{itemize}

\subsection{Changelog}\label{changelog}

\subsubsection{0.1.0}\label{section}

\begin{itemize}
\tightlist
\item
  Initial release
\end{itemize}

\subsubsection{0.2.0}\label{section-1}

\begin{itemize}
\tightlist
\item
  Add continuation of lists through the \texttt{\ counter\ } function
\end{itemize}

\subsubsection{0.3.0}\label{section-2}

\begin{itemize}
\tightlist
\item
  Add support for nested lists
\end{itemize}

\subsubsection{How to add}\label{how-to-add}

Copy this into your project and use the import as \texttt{\ efilrst\ }

\begin{verbatim}
#import "@preview/efilrst:0.3.0"
\end{verbatim}

\includesvg[width=0.16667in,height=0.16667in]{/assets/icons/16-copy.svg}

Check the docs for
\href{https://typst.app/docs/reference/scripting/\#packages}{more
information on how to import packages} .

\subsubsection{About}\label{about}

\begin{description}
\tightlist
\item[Author :]
\href{https://github.com/jmigual}{Joan Marcè i Igual}
\item[License:]
MIT
\item[Current version:]
0.3.0
\item[Last updated:]
November 25, 2024
\item[First released:]
August 27, 2024
\item[Minimum Typst version:]
0.12.0
\item[Archive size:]
2.77 kB
\href{https://packages.typst.org/preview/efilrst-0.3.0.tar.gz}{\pandocbounded{\includesvg[keepaspectratio]{/assets/icons/16-download.svg}}}
\item[Repository:]
\href{https://github.com/jmigual/typst-efilrst}{GitHub}
\end{description}

\subsubsection{Where to report issues?}\label{where-to-report-issues}

This package is a project of Joan Marcè i Igual . Report issues on
\href{https://github.com/jmigual/typst-efilrst}{their repository} . You
can also try to ask for help with this package on the
\href{https://forum.typst.app}{Forum} .

Please report this package to the Typst team using the
\href{https://typst.app/contact}{contact form} if you believe it is a
safety hazard or infringes upon your rights.

\phantomsection\label{versions}
\subsubsection{Version history}\label{version-history}

\begin{longtable}[]{@{}ll@{}}
\toprule\noalign{}
Version & Release Date \\
\midrule\noalign{}
\endhead
\bottomrule\noalign{}
\endlastfoot
0.3.0 & November 25, 2024 \\
\href{https://typst.app/universe/package/efilrst/0.2.0/}{0.2.0} &
November 12, 2024 \\
\href{https://typst.app/universe/package/efilrst/0.1.0/}{0.1.0} & August
27, 2024 \\
\end{longtable}

Typst GmbH did not create this package and cannot guarantee correct
functionality of this package or compatibility with any version of the
Typst compiler or app.


\section{Package List LaTeX/easytable.tex}
\title{typst.app/universe/package/easytable}

\phantomsection\label{banner}
\section{easytable}\label{easytable}

{ 0.1.0 }

Simple Table Package

\phantomsection\label{readme}
A Typst library for writing simple tables.

\subsection{Usage}\label{usage}

\begin{Shaded}
\begin{Highlighting}[]
\NormalTok{\#import "@preview/easytable:0.1.0": easytable, elem}
\NormalTok{\#import elem: *}
\end{Highlighting}
\end{Shaded}

\subsection{Manual}\label{manual}

\begin{itemize}
\tightlist
\item
  You can create a table by specifying data or layout elements as
  arguments to the \texttt{\ easytable\ } function.
\item
  The following elements are provided in the \texttt{\ elem\ } module.

  \begin{itemize}
  \tightlist
  \item
    \texttt{\ elem.tr\ } : a data row
  \item
    \texttt{\ elem.th\ } : a header row
  \item
    \texttt{\ elem.hline\ } : a horizontal line
  \item
    \texttt{\ elem.vline\ } : a vertical line
  \item
    \texttt{\ elem.cwidth\ } : a column-width specifier
  \item
    \texttt{\ elem.cstyle\ } : a column-style (font, alignment, etc.)
    specifier
  \end{itemize}
\end{itemize}

See
\href{https://github.com/typst/packages/raw/main/packages/preview/easytable/0.1.0/manual.pdf}{manual}
in detail.

\subsection{Examples}\label{examples}

\subsubsection{A Simple Table}\label{a-simple-table}

\begin{Shaded}
\begin{Highlighting}[]
\NormalTok{\#easytable(\{}
\NormalTok{  th[Header 1 ][Header 2][Header 3  ]}
\NormalTok{  tr[How      ][I       ][want      ]}
\NormalTok{  tr[a        ][drink,  ][alcoholic ]}
\NormalTok{  tr[of       ][course, ][after     ]}
\NormalTok{  tr[the      ][heavy   ][lectures  ]}
\NormalTok{  tr[involving][quantum ][mechanics.]}
\NormalTok{\})}
\end{Highlighting}
\end{Shaded}

\pandocbounded{\includegraphics[keepaspectratio]{https://github.com/monaqa/typst-easytable/assets/48883418/690b466b-56d9-4660-8ca5-25cc25e379f9}}

\subsubsection{Setting Column Alignment and
Width}\label{setting-column-alignment-and-width}

\begin{Shaded}
\begin{Highlighting}[]
\NormalTok{\#easytable(\{}
\NormalTok{  cwidth(100pt, 1fr, 20\%)}
\NormalTok{  cstyle(left, center, right)}
\NormalTok{  th[Header 1 ][Header 2][Header 3  ]}
\NormalTok{  tr[How      ][I       ][want      ]}
\NormalTok{  tr[a        ][drink,  ][alcoholic ]}
\NormalTok{  tr[of       ][course, ][after     ]}
\NormalTok{  tr[the      ][heavy   ][lectures  ]}
\NormalTok{  tr[involving][quantum ][mechanics.]}
\NormalTok{\})}
\end{Highlighting}
\end{Shaded}

\pandocbounded{\includegraphics[keepaspectratio]{https://github.com/monaqa/typst-easytable/assets/48883418/8ff574b4-bf1f-46ca-8a2d-584ab701a989}}

\subsubsection{Customizing Styles}\label{customizing-styles}

\begin{Shaded}
\begin{Highlighting}[]
\NormalTok{\#easytable(\{}
\NormalTok{  let tr = tr.with(trans: pad.with(x: 3pt))}

\NormalTok{  th[Header 1][Header 2][Header 3]}
\NormalTok{  tr[How][I][want]}
\NormalTok{  tr[a][drink,][alcoholic]}
\NormalTok{  tr[of][course,][after]}
\NormalTok{  tr[the][heavy][lectures]}
\NormalTok{  tr[involving][quantum][mechanics.]}
\NormalTok{\})}
\end{Highlighting}
\end{Shaded}

\pandocbounded{\includegraphics[keepaspectratio]{https://github.com/monaqa/typst-easytable/assets/48883418/8a1ed0d0-4a9e-4a28-a0ff-b8f7a09cb8a8}}

\begin{Shaded}
\begin{Highlighting}[]
\NormalTok{\#easytable(\{}
\NormalTok{  let th = th.with(trans: emph)}
\NormalTok{  let tr = tr.with(}
\NormalTok{    cell\_style: (x: none, y: none)}
\NormalTok{      =\textgreater{} (fill: if calc.even(y) \{}
\NormalTok{        luma(95\%)}
\NormalTok{      \} else \{}
\NormalTok{        none}
\NormalTok{      \})}
\NormalTok{  )}

\NormalTok{  th[Header 1][Header 2][Header 3]}
\NormalTok{  tr[How][I][want]}
\NormalTok{  tr[a][drink,][alcoholic]}
\NormalTok{  tr[of][course,][after]}
\NormalTok{  tr[the][heavy][lectures]}
\NormalTok{  tr[involving][quantum][mechanics.]}
\NormalTok{\})}
\end{Highlighting}
\end{Shaded}

\pandocbounded{\includegraphics[keepaspectratio]{https://github.com/monaqa/typst-easytable/assets/48883418/5f8bf796-b2bd-41c4-a79e-fd97c2824ecd}}

\begin{Shaded}
\begin{Highlighting}[]
\NormalTok{\#easytable(\{}
\NormalTok{  th[Header 1][Header 2][Header 3]}
\NormalTok{  tr[How][I][want]}
\NormalTok{  hline(stroke: red)}
\NormalTok{  tr[a][drink,][alcoholic]}
\NormalTok{  tr[of][course,][after]}
\NormalTok{  tr[the][heavy][lectures]}
\NormalTok{  tr[involving][quantum][mechanics.]}

\NormalTok{  // Specifying the insertion point directly}
\NormalTok{  hline(stroke: 2pt + green, y: 4)}
\NormalTok{  vline(}
\NormalTok{    stroke: (paint: blue, thickness: 1pt, dash: "dashed"),}
\NormalTok{    x: 2,}
\NormalTok{    start: 1,}
\NormalTok{    end: 5,}
\NormalTok{  )}
\NormalTok{\})}
\end{Highlighting}
\end{Shaded}

\pandocbounded{\includegraphics[keepaspectratio]{https://github.com/monaqa/typst-easytable/assets/48883418/cf400dad-a7fc-4f3a-991d-9611adab41c6}}

\subsubsection{How to add}\label{how-to-add}

Copy this into your project and use the import as \texttt{\ easytable\ }

\begin{verbatim}
#import "@preview/easytable:0.1.0"
\end{verbatim}

\includesvg[width=0.16667in,height=0.16667in]{/assets/icons/16-copy.svg}

Check the docs for
\href{https://typst.app/docs/reference/scripting/\#packages}{more
information on how to import packages} .

\subsubsection{About}\label{about}

\begin{description}
\tightlist
\item[Author :]
monaqa
\item[License:]
MIT
\item[Current version:]
0.1.0
\item[Last updated:]
February 24, 2024
\item[First released:]
February 24, 2024
\item[Archive size:]
3.36 kB
\href{https://packages.typst.org/preview/easytable-0.1.0.tar.gz}{\pandocbounded{\includesvg[keepaspectratio]{/assets/icons/16-download.svg}}}
\item[Repository:]
\href{https://github.com/monaqa/typst-easytable}{GitHub}
\end{description}

\subsubsection{Where to report issues?}\label{where-to-report-issues}

This package is a project of monaqa . Report issues on
\href{https://github.com/monaqa/typst-easytable}{their repository} . You
can also try to ask for help with this package on the
\href{https://forum.typst.app}{Forum} .

Please report this package to the Typst team using the
\href{https://typst.app/contact}{contact form} if you believe it is a
safety hazard or infringes upon your rights.

\phantomsection\label{versions}
\subsubsection{Version history}\label{version-history}

\begin{longtable}[]{@{}ll@{}}
\toprule\noalign{}
Version & Release Date \\
\midrule\noalign{}
\endhead
\bottomrule\noalign{}
\endlastfoot
0.1.0 & February 24, 2024 \\
\end{longtable}

Typst GmbH did not create this package and cannot guarantee correct
functionality of this package or compatibility with any version of the
Typst compiler or app.


\section{Package List LaTeX/vantage-cv.tex}
\title{typst.app/universe/package/vantage-cv}

\phantomsection\label{banner}
\phantomsection\label{template-thumbnail}
\pandocbounded{\includegraphics[keepaspectratio]{https://packages.typst.org/preview/thumbnails/vantage-cv-1.0.0-small.webp}}

\section{vantage-cv}\label{vantage-cv}

{ 1.0.0 }

An ATS friendly simple Typst CV template

\href{/app?template=vantage-cv&version=1.0.0}{Create project in app}

\phantomsection\label{readme}
An ATS friendly simple Typst CV template, inspired by
\href{https://github.com/GeorgeHoneywood/alta-typst}{alta-typst by
George Honeywood} .

\subsection{Features}\label{features}

\begin{itemize}
\tightlist
\item
  \textbf{Two-column layout} : Experience on the left and other
  important details on the right, organized for easy scanning.
\item
  \textbf{Customizable icons} : Add and replace icons to suit your
  personal style.
\item
  \textbf{Responsive design} : Adjusts well for various print formats.
\end{itemize}

\subsection{Usage}\label{usage}

\subsubsection{Running Locally with Typst
CLI}\label{running-locally-with-typst-cli}

\begin{enumerate}
\item
  \textbf{Install Typst CLI} : Follow the installation instructions on
  the \href{https://github.com/typst/typst\#installation}{Typst CLI
  GitHub page} to set up Typst on your machine.
\item
  \textbf{Clone the repository} :

\begin{Shaded}
\begin{Highlighting}[]
\FunctionTok{git}\NormalTok{ clone https://github.com/sardorml/vantage{-}typst.git}
\BuiltInTok{cd}\NormalTok{ vantage{-}typst}
\end{Highlighting}
\end{Shaded}
\item
  \textbf{Run Typst} :

  Use the following command to render your CV:

\begin{Shaded}
\begin{Highlighting}[]
\ExtensionTok{typst}\NormalTok{ compile example.typ}
\end{Highlighting}
\end{Shaded}

  This will generate a PDF output in the same directory.
\item
  \textbf{Edit your CV} :

  Open the \texttt{\ example.typ\ } file in your preferred text editor
  to customize the layout.
\end{enumerate}

\subsubsection{Configuration}\label{configuration}

You can easily customize your personal data by editing the
\texttt{\ configuration.yaml\ } file. This file allows you to set your
name, contact information, work experience, education, and skills.
Here’s how to do it:

\begin{enumerate}
\tightlist
\item
  Open the \texttt{\ configuration.yaml\ } file in your text editor.
\item
  Update the fields with your personal information.
\item
  Save the file, and your CV will automatically reflect these changes
  when you compile it.
\end{enumerate}

\subsection{Icons}\label{icons}

You can enhance your CV with additional icons by following these steps:

\begin{enumerate}
\item
  \textbf{Upload Icons} : Place your \texttt{\ .svg\ } files in the
  \texttt{\ icons/\ } folder.
\item
  \textbf{Reference Icons} : Modify the \texttt{\ links\ } array in the
  Typst file to include your new icons by referencing their filenames as
  the \texttt{\ name\ } values.

  Example:

\begin{Shaded}
\begin{Highlighting}[]
\NormalTok{links: [}
\NormalTok{  \{ name: "your{-}icon{-}name", url: "https://example.com" \},}
\NormalTok{]}
\end{Highlighting}
\end{Shaded}
\end{enumerate}

For existing icons, the current selection is sourced from
\href{https://lucide.dev/icons/}{Lucide Icons} .

\subsection{License}\label{license}

This project is licensed under the
\href{https://github.com/typst/packages/raw/main/packages/preview/vantage-cv/1.0.0/LICENSE}{MIT
License} .

Icons are from Lucide Icons and are subject to
\href{https://lucide.dev/license}{their terms} .

\subsection{Acknowledgments}\label{acknowledgments}

\begin{itemize}
\tightlist
\item
  Inspired by the work of
  \href{https://github.com/GeorgeHoneywood/alta-typst}{George Honeywood}
  .
\item
  Thanks to \href{https://lucide.dev/icons/}{Lucide Icons} for providing
  the icon library.
\end{itemize}

For any questions or contributions, feel free to open an issue or submit
a pull request!

\href{/app?template=vantage-cv&version=1.0.0}{Create project in app}

\subsubsection{How to use}\label{how-to-use}

Click the button above to create a new project using this template in
the Typst app.

You can also use the Typst CLI to start a new project on your computer
using this command:

\begin{verbatim}
typst init @preview/vantage-cv:1.0.0
\end{verbatim}

\includesvg[width=0.16667in,height=0.16667in]{/assets/icons/16-copy.svg}

\subsubsection{About}\label{about}

\begin{description}
\tightlist
\item[Author :]
\href{https://github.com/sardorml}{Sardor}
\item[License:]
MIT
\item[Current version:]
1.0.0
\item[Last updated:]
October 9, 2024
\item[First released:]
October 9, 2024
\item[Archive size:]
6.59 kB
\href{https://packages.typst.org/preview/vantage-cv-1.0.0.tar.gz}{\pandocbounded{\includesvg[keepaspectratio]{/assets/icons/16-download.svg}}}
\item[Repository:]
\href{https://github.com/sardorml/vantage-typst}{GitHub}
\item[Categor ies :]
\begin{itemize}
\tightlist
\item[]
\item
  \pandocbounded{\includesvg[keepaspectratio]{/assets/icons/16-user.svg}}
  \href{https://typst.app/universe/search/?category=cv}{CV}
\item
  \pandocbounded{\includesvg[keepaspectratio]{/assets/icons/16-layout.svg}}
  \href{https://typst.app/universe/search/?category=layout}{Layout}
\end{itemize}
\end{description}

\subsubsection{Where to report issues?}\label{where-to-report-issues}

This template is a project of Sardor . Report issues on
\href{https://github.com/sardorml/vantage-typst}{their repository} . You
can also try to ask for help with this template on the
\href{https://forum.typst.app}{Forum} .

Please report this template to the Typst team using the
\href{https://typst.app/contact}{contact form} if you believe it is a
safety hazard or infringes upon your rights.

\phantomsection\label{versions}
\subsubsection{Version history}\label{version-history}

\begin{longtable}[]{@{}ll@{}}
\toprule\noalign{}
Version & Release Date \\
\midrule\noalign{}
\endhead
\bottomrule\noalign{}
\endlastfoot
1.0.0 & October 9, 2024 \\
\end{longtable}

Typst GmbH did not create this template and cannot guarantee correct
functionality of this template or compatibility with any version of the
Typst compiler or app.


\section{Package List LaTeX/fervojo.tex}
\title{typst.app/universe/package/fervojo}

\phantomsection\label{banner}
\section{fervojo}\label{fervojo}

{ 0.1.0 }

railroad for typst, powered by wasm

\phantomsection\label{readme}
Use \href{https://github.com/lukaslueg/railroad_dsl}{railroads} in your
documents.

You use the function by calling \texttt{\ render(diagram-text,\ css)\ }
which renders the diagram. There, \texttt{\ diagram-text\ } contains is
the diagram itself and css is the one used for the style,
\texttt{\ css\ } is \texttt{\ default-css()\ } if you don’t pass it.
Both fields can be strings, bytes or a raw
\href{https://typst.app/docs/reference/text/raw/}{raw} block.

\subsubsection{How to add}\label{how-to-add}

Copy this into your project and use the import as \texttt{\ fervojo\ }

\begin{verbatim}
#import "@preview/fervojo:0.1.0"
\end{verbatim}

\includesvg[width=0.16667in,height=0.16667in]{/assets/icons/16-copy.svg}

Check the docs for
\href{https://typst.app/docs/reference/scripting/\#packages}{more
information on how to import packages} .

\subsubsection{About}\label{about}

\begin{description}
\tightlist
\item[Author :]
Leiser Fernández Gallo
\item[License:]
MIT
\item[Current version:]
0.1.0
\item[Last updated:]
June 5, 2024
\item[First released:]
June 5, 2024
\item[Archive size:]
44.8 kB
\href{https://packages.typst.org/preview/fervojo-0.1.0.tar.gz}{\pandocbounded{\includesvg[keepaspectratio]{/assets/icons/16-download.svg}}}
\item[Repository:]
\href{https://github.com/leiserfg/fervojo}{GitHub}
\end{description}

\subsubsection{Where to report issues?}\label{where-to-report-issues}

This package is a project of Leiser Fernández Gallo . Report issues on
\href{https://github.com/leiserfg/fervojo}{their repository} . You can
also try to ask for help with this package on the
\href{https://forum.typst.app}{Forum} .

Please report this package to the Typst team using the
\href{https://typst.app/contact}{contact form} if you believe it is a
safety hazard or infringes upon your rights.

\phantomsection\label{versions}
\subsubsection{Version history}\label{version-history}

\begin{longtable}[]{@{}ll@{}}
\toprule\noalign{}
Version & Release Date \\
\midrule\noalign{}
\endhead
\bottomrule\noalign{}
\endlastfoot
0.1.0 & June 5, 2024 \\
\end{longtable}

Typst GmbH did not create this package and cannot guarantee correct
functionality of this package or compatibility with any version of the
Typst compiler or app.


\section{Package List LaTeX/ez-today.tex}
\title{typst.app/universe/package/ez-today}

\phantomsection\label{banner}
\section{ez-today}\label{ez-today}

{ 0.1.0 }

Simply displays the full current date.

\phantomsection\label{readme}
Simply displays the current date with easy to use customization.

\subsection{Included languages}\label{included-languages}

German, English, French and Italian months can be used out of the box.
If you want to use a language that is not included, you can easily add
it yourself. This is shown in the customization section below.

\subsection{Usage}\label{usage}

The usage is very simple, because there is only the \texttt{\ today()\ }
function.

\begin{Shaded}
\begin{Highlighting}[]
\NormalTok{\#import "@preview/ez{-}today:0.1.0"}

\NormalTok{// To get the current date use this}
\NormalTok{\#ez{-}today.today()}
\end{Highlighting}
\end{Shaded}

\subsection{Reference}\label{reference}

\subsubsection{\texorpdfstring{\texttt{\ today\ }}{ today }}\label{today}

Prints the current date with given arguments.

\begin{Shaded}
\begin{Highlighting}[]
\NormalTok{\#let today(}
\NormalTok{  lang: "de",}
\NormalTok{  format: "d. M Y",}
\NormalTok{  custom{-}months: ()}
\NormalTok{) = \{ .. \}}
\end{Highlighting}
\end{Shaded}

\textbf{Arguments:}

\begin{itemize}
\tightlist
\item
  \texttt{\ lang\ } : {[} \texttt{\ str\ } {]} â€'' Select one of the
  included languages (de, en, fr, it).
\item
  \texttt{\ format\ } : {[} \texttt{\ str\ } {]} â€'' Specify the output
  format.
\item
  \texttt{\ custom-months\ } : {[} \texttt{\ array\ } {]} of {[}
  \texttt{\ str\ } {]} â€'' Use custom names for each month. This array
  must have 12 entries. If this is used, the \texttt{\ lang\ } argument
  does nothing.
\end{itemize}

\subsection{Customization}\label{customization}

The default output prints the full current date with German months like
this:

\begin{Shaded}
\begin{Highlighting}[]
\NormalTok{\#ez{-}today.today()   // 11. Oktober 2024}
\end{Highlighting}
\end{Shaded}

You can choose one of the included languages with the \texttt{\ lang\ }
argument:

\begin{Shaded}
\begin{Highlighting}[]
\NormalTok{\#ez{-}today.today(lang: "en")   // 11. October 2024}
\NormalTok{\#ez{-}today.today(lang: "fr")   // 11. Octobre 2024}
\NormalTok{\#ez{-}today.today(lang: "it")   // 11. Ottobre 2024}
\end{Highlighting}
\end{Shaded}

You can also change the format of the output with the
\texttt{\ format\ } argument. Pass any string you want here, but know
that the following characters will be replaced with the following:

\begin{itemize}
\tightlist
\item
  \texttt{\ d\ } â€'' The current day as a decimal
\item
  \texttt{\ m\ } â€'' The current month as a decimal ( \texttt{\ lang\ }
  argument does nothing)
\item
  \texttt{\ M\ } â€'' The current month as a string with either the
  selected language or the custom array
\item
  \texttt{\ y\ } â€'' The current year as a decimal with the last two
  numbers
\item
  \texttt{\ Y\ } â€'' The current year as a decimal
\end{itemize}

Here are some examples:

\begin{Shaded}
\begin{Highlighting}[]
\NormalTok{\#ez{-}today.today(lang: "en", format: "M d Y")    // October 11 2024}
\NormalTok{\#ez{-}today.today(format: "m{-}d{-}y")                // 10{-}11{-}24}
\NormalTok{\#ez{-}today.today(format: "d/m")                  // 11/10}
\NormalTok{\#ez{-}today.today(format: "d.m.Y")                // 11.10.2024}
\end{Highlighting}
\end{Shaded}

Use the \texttt{\ custom-months\ } argument to give each month a custom
name. You can add a new language or use short terms for each month.

\begin{Shaded}
\begin{Highlighting}[]
\NormalTok{// Defining some custom names}
\NormalTok{\#let my{-}months = ("Jan", "Feb", "Mar", "Apr", "May", "Jun", "Jul", "Aug", "Sep", "Oct", "Nov", "Dec")}
\NormalTok{// Get current date with custom names}
\NormalTok{\#ez{-}today.today(custom{-}months: my{-}months, format: "M{-}y")    // Oct{-}24}
\end{Highlighting}
\end{Shaded}

\subsubsection{How to add}\label{how-to-add}

Copy this into your project and use the import as \texttt{\ ez-today\ }

\begin{verbatim}
#import "@preview/ez-today:0.1.0"
\end{verbatim}

\includesvg[width=0.16667in,height=0.16667in]{/assets/icons/16-copy.svg}

Check the docs for
\href{https://typst.app/docs/reference/scripting/\#packages}{more
information on how to import packages} .

\subsubsection{About}\label{about}

\begin{description}
\tightlist
\item[Author :]
Carlo Schafflik
\item[License:]
MIT
\item[Current version:]
0.1.0
\item[Last updated:]
October 11, 2024
\item[First released:]
October 11, 2024
\item[Archive size:]
2.42 kB
\href{https://packages.typst.org/preview/ez-today-0.1.0.tar.gz}{\pandocbounded{\includesvg[keepaspectratio]{/assets/icons/16-download.svg}}}
\item[Repository:]
\href{https://github.com/CarloSchafflik12/typst-ez-today}{GitHub}
\end{description}

\subsubsection{Where to report issues?}\label{where-to-report-issues}

This package is a project of Carlo Schafflik . Report issues on
\href{https://github.com/CarloSchafflik12/typst-ez-today}{their
repository} . You can also try to ask for help with this package on the
\href{https://forum.typst.app}{Forum} .

Please report this package to the Typst team using the
\href{https://typst.app/contact}{contact form} if you believe it is a
safety hazard or infringes upon your rights.

\phantomsection\label{versions}
\subsubsection{Version history}\label{version-history}

\begin{longtable}[]{@{}ll@{}}
\toprule\noalign{}
Version & Release Date \\
\midrule\noalign{}
\endhead
\bottomrule\noalign{}
\endlastfoot
0.1.0 & October 11, 2024 \\
\end{longtable}

Typst GmbH did not create this package and cannot guarantee correct
functionality of this package or compatibility with any version of the
Typst compiler or app.


\section{Package List LaTeX/use-tabler-icons.tex}
\title{typst.app/universe/package/use-tabler-icons}

\phantomsection\label{banner}
\section{use-tabler-icons}\label{use-tabler-icons}

{ 0.4.0 }

Tabler Icons for Typst using webfont.

\phantomsection\label{readme}
\begin{quote}
\textbf{Note}

This project is greatly inspired by and mainly edited based on
\href{https://github.com/duskmoon314/typst-fontawesome}{typst-fontawesome}
.
\end{quote}

\subsection{\texorpdfstring{\protect\pandocbounded{\includesvg[keepaspectratio]{https://github.com/typst/packages/raw/main/packages/preview/use-tabler-icons/0.4.0/assets/splash.svg}}}{use-tabler-icons}}\label{use-tabler-icons-1}

A Typst library for \href{https://github.com/tabler/tabler-icons}{Tabler
Icons} , a set of over 5700 free MIT-licensed high-quality SVG icons.

\subsection{Usage}\label{usage}

\subsubsection{Install Font}\label{install-font}

Download \href{https://github.com/tabler/tabler-icons/releases}{latest
release of tabler-icons} and install
\texttt{\ webfont/fonts/tabler-icons.ttf\ } . Or, if you are using Typst
web app, simply upload the font file to your project.

\subsubsection{Import the Library}\label{import-the-library}

\paragraph{Using the Typst Packages}\label{using-the-typst-packages}

You can install the library using the typst packages:

\begin{Shaded}
\begin{Highlighting}[]
\NormalTok{\#import "@preview/use{-}tabler{-}icons:0.4.0": *}
\end{Highlighting}
\end{Shaded}

\paragraph{Manually Install}\label{manually-install}

Just copy all files under
\href{https://github.com/zyf722/typst-tabler-icons/tree/main/src}{\texttt{\ src\ }}
to your project and rename them to avoid naming conflicts.

Then, import \texttt{\ lib.typ\ } to use the library:

\begin{Shaded}
\begin{Highlighting}[]
\NormalTok{\#import "lib.typ": *}
\end{Highlighting}
\end{Shaded}

\subsubsection{Use the Icons}\label{use-the-icons}

You can use the \texttt{\ tabler-icon\ } function to create an icon with
its name:

\begin{Shaded}
\begin{Highlighting}[]
\NormalTok{\#tabler{-}icon("calendar")}
\end{Highlighting}
\end{Shaded}

Or you can directly use the \texttt{\ ti-\ } prefix :

\begin{Shaded}
\begin{Highlighting}[]
\NormalTok{\#ti{-}calendar()}
\end{Highlighting}
\end{Shaded}

As these icons are actually text with custom font, you can pass any text
attributes to the function:

\begin{Shaded}
\begin{Highlighting}[]
\NormalTok{\#tabler{-}icon("calendar", fill: blue)}
\end{Highlighting}
\end{Shaded}

Refer to
\href{https://github.com/zyf722/typst-tabler-icons/tree/main/gallery/gallery.pdf}{\texttt{\ gallery.pdf\ }}
and \href{https://tabler.io/icons}{Tabler Icons website} for all
available icons.

\subsection{Contributing}\label{contributing}

\href{https://github.com/zyf722/typst-tabler-icons/pulls}{Pull Requests}
are welcome!

It is strongly recommended to follow the
\href{https://www.conventionalcommits.org/en/v1.0.0/}{Conventional
Commits} specification when writing commit messages and creating pull
requests.

\subsubsection{Github Actions Workflow}\label{github-actions-workflow}

This package uses a daily run
\href{https://github.com/zyf722/typst-tabler-icons/tree/main/.github/workflows/build.yml}{Github
Actions workflow} to keep the library up-to-date with the latest version
of Tabler Icons, which internally runs
\href{https://github.com/zyf722/typst-tabler-icons/tree/main/scripts/generate.mjs}{\texttt{\ scripts/generate.mjs\ }}
to generate Typst source code of the library and gallery.

\subsection{License}\label{license}

\href{https://github.com/zyf722/typst-tabler-icons/tree/main/LICENSE}{MIT}

\subsubsection{How to add}\label{how-to-add}

Copy this into your project and use the import as
\texttt{\ use-tabler-icons\ }

\begin{verbatim}
#import "@preview/use-tabler-icons:0.4.0"
\end{verbatim}

\includesvg[width=0.16667in,height=0.16667in]{/assets/icons/16-copy.svg}

Check the docs for
\href{https://typst.app/docs/reference/scripting/\#packages}{more
information on how to import packages} .

\subsubsection{About}\label{about}

\begin{description}
\tightlist
\item[Author s :]
\href{mailto:kp.campbell.he@duskmoon314.com}{duskmoon (Campbell He)} \&
\href{mailto:MaxMixAlex@protonmail.com}{MaxMixAlex}
\item[License:]
MIT
\item[Current version:]
0.4.0
\item[Last updated:]
November 13, 2024
\item[First released:]
October 21, 2024
\item[Archive size:]
79.3 kB
\href{https://packages.typst.org/preview/use-tabler-icons-0.4.0.tar.gz}{\pandocbounded{\includesvg[keepaspectratio]{/assets/icons/16-download.svg}}}
\item[Repository:]
\href{https://github.com/zyf722/typst-tabler-icons}{GitHub}
\item[Categor y :]
\begin{itemize}
\tightlist
\item[]
\item
  \pandocbounded{\includesvg[keepaspectratio]{/assets/icons/16-text.svg}}
  \href{https://typst.app/universe/search/?category=text}{Text}
\end{itemize}
\end{description}

\subsubsection{Where to report issues?}\label{where-to-report-issues}

This package is a project of duskmoon (Campbell He) and MaxMixAlex .
Report issues on
\href{https://github.com/zyf722/typst-tabler-icons}{their repository} .
You can also try to ask for help with this package on the
\href{https://forum.typst.app}{Forum} .

Please report this package to the Typst team using the
\href{https://typst.app/contact}{contact form} if you believe it is a
safety hazard or infringes upon your rights.

\phantomsection\label{versions}
\subsubsection{Version history}\label{version-history}

\begin{longtable}[]{@{}ll@{}}
\toprule\noalign{}
Version & Release Date \\
\midrule\noalign{}
\endhead
\bottomrule\noalign{}
\endlastfoot
0.4.0 & November 13, 2024 \\
\href{https://typst.app/universe/package/use-tabler-icons/0.3.0/}{0.3.0}
& October 30, 2024 \\
\href{https://typst.app/universe/package/use-tabler-icons/0.2.0/}{0.2.0}
& October 25, 2024 \\
\href{https://typst.app/universe/package/use-tabler-icons/0.1.0/}{0.1.0}
& October 21, 2024 \\
\end{longtable}

Typst GmbH did not create this package and cannot guarantee correct
functionality of this package or compatibility with any version of the
Typst compiler or app.


\section{Package List LaTeX/lineal.tex}
\title{typst.app/universe/package/lineal}

\phantomsection\label{banner}
\section{lineal}\label{lineal}

{ 0.1.0 }

Build elegent slide decks with Typst

\phantomsection\label{readme}
IPA: /ˈlɪniəl/

Made up of, or having the characteristic of, lines.

Lineal is a Typst template for generating beautifully clean and
configurably awesome slides.

\pandocbounded{\includegraphics[keepaspectratio]{https://github.com/typst/packages/raw/main/packages/preview/lineal/0.1.0/assets/img/demo.gif}}

\subsection{Philosophy}\label{philosophy}

As a long time user of TeX, I have developed and embedded countless
LaTeX applications into personal and corporate environments, e.g.
automating documentation, conference materials, posters, resumes, etc.

However, LaTeX is showing its age. Compiling a some 30-slide
presentation, for instance, takes perhaps a second, and may requires
multiple renders to sync TikZ positioning and cross-document
referencing. Typst remediates these issues in real-time and with better
control and confidence in its data modelling.

Hence, Lineal was born. A professional set of slides produced near
instantly, readily equipped with configurable theming and a suite of
flexible components.

\subsection{Usage}\label{usage}

Lineal is available through Typst Universe. Ensure you have installed
Typst locally or are familiar with the Typst
\href{https://typst.app/}{web app} or the
\href{https://marketplace.visualstudio.com/items?itemName=myriad-dreamin.tinymist}{Tinymist
LSP} extensions for VS Code.

Get started by importing the package and populating your own
\texttt{\ /content/\textless{}slug\textgreater{}.typ\ } files:

\begin{Shaded}
\begin{Highlighting}[]
\NormalTok{\#import "@preview/lineal:0.1.0": lineal{-}theme}

\NormalTok{\#show: lineal{-}theme.with(}
\NormalTok{  aspect{-}ratio: "16{-}9",}
\NormalTok{  config{-}info(}
\NormalTok{    title: [$bb("L")"ineal"$],}
\NormalTok{    subtitle: [A Typst slide template],}
\NormalTok{    author: [Author],}
\NormalTok{    date: datetime.today(),}
\NormalTok{    institution: [Institution],}
\NormalTok{    logo: [Logo],}
\NormalTok{  ),}
\NormalTok{)}

\NormalTok{\#title{-}slide()}

\NormalTok{\#include "content/index.typ"}
\end{Highlighting}
\end{Shaded}

Marking up content is as you would with any other Typst document, where
the section ( \texttt{\ =\ \textless{}section-title\textgreater{}\ } )
and subsection ( \texttt{\ ==\ \textless{}slide-title\textgreater{}\ } )
shorthands generate the new section slides with dynamic outline and new
tracked slides respectively.

\subsection{Contributing}\label{contributing}

PRs are very welcome. If you think Lineal could be improved in any way
or is missing a feature, please raise a request 😎

\subsection{Acknowledgements}\label{acknowledgements}

A heartfelt thank you to the team behind
\href{https://github.com/typst/typst}{Typst} , developing a product that
not only preserves the beauty of LaTeX’s typesetting, but improves on
its developer experience in every way, in line with ongoing community
feedback.

The creators of the
\href{https://github.com/touying-typ/touying}{\texttt{\ Touying\ }} and
\href{https://github.com/andreasKroepelin/polylux}{\texttt{\ Polylux\ }}
Typst packages, on which Lineal is built.

\subsubsection{How to add}\label{how-to-add}

Copy this into your project and use the import as \texttt{\ lineal\ }

\begin{verbatim}
#import "@preview/lineal:0.1.0"
\end{verbatim}

\includesvg[width=0.16667in,height=0.16667in]{/assets/icons/16-copy.svg}

Check the docs for
\href{https://typst.app/docs/reference/scripting/\#packages}{more
information on how to import packages} .

\subsubsection{About}\label{about}

\begin{description}
\tightlist
\item[Author :]
\href{https://github.com/ellsphillips}{ellsphillips}
\item[License:]
MIT
\item[Current version:]
0.1.0
\item[Last updated:]
November 28, 2024
\item[First released:]
November 28, 2024
\item[Archive size:]
7.40 kB
\href{https://packages.typst.org/preview/lineal-0.1.0.tar.gz}{\pandocbounded{\includesvg[keepaspectratio]{/assets/icons/16-download.svg}}}
\item[Repository:]
\href{https://github.com/ellsphillips/lineal}{GitHub}
\item[Categor y :]
\begin{itemize}
\tightlist
\item[]
\item
  \pandocbounded{\includesvg[keepaspectratio]{/assets/icons/16-presentation.svg}}
  \href{https://typst.app/universe/search/?category=presentation}{Presentation}
\end{itemize}
\end{description}

\subsubsection{Where to report issues?}\label{where-to-report-issues}

This package is a project of ellsphillips . Report issues on
\href{https://github.com/ellsphillips/lineal}{their repository} . You
can also try to ask for help with this package on the
\href{https://forum.typst.app}{Forum} .

Please report this package to the Typst team using the
\href{https://typst.app/contact}{contact form} if you believe it is a
safety hazard or infringes upon your rights.

\phantomsection\label{versions}
\subsubsection{Version history}\label{version-history}

\begin{longtable}[]{@{}ll@{}}
\toprule\noalign{}
Version & Release Date \\
\midrule\noalign{}
\endhead
\bottomrule\noalign{}
\endlastfoot
0.1.0 & November 28, 2024 \\
\end{longtable}

Typst GmbH did not create this package and cannot guarantee correct
functionality of this package or compatibility with any version of the
Typst compiler or app.


\section{Package List LaTeX/not-jku-thesis.tex}
\title{typst.app/universe/package/not-jku-thesis}

\phantomsection\label{banner}
\phantomsection\label{template-thumbnail}
\pandocbounded{\includegraphics[keepaspectratio]{https://packages.typst.org/preview/thumbnails/not-jku-thesis-0.1.0-small.webp}}

\section{not-jku-thesis}\label{not-jku-thesis}

{ 0.1.0 }

Customizable not official template for a thesis at the JKU, derived from
a template created by Fabian Scherer
\textless https://www.linkedin.com/in/fabian-scherer-de/\textgreater{}
with Leon Weber in an advisory role.

\href{/app?template=not-jku-thesis&version=0.1.0}{Create project in app}

\phantomsection\label{readme}
\href{https://github.com/typst/packages/raw/main/packages/preview/not-jku-thesis/0.1.0/template/thesis.pdf}{The
compiled demo thesis.pdf}

This is a Typst template for a thesis at JKU.

\subsection{Usage}\label{usage}

You can use this template in the Typst web app by clicking “Start from
template� on the dashboard and searching for
\texttt{\ not-JKU-thesis\ } .

Alternatively, you can use the CLI to kick this project off using the
command

\begin{verbatim}
typst init @preview/jku-thesis
\end{verbatim}

Typst will create a new directory with all the files needed to get you
started.

\subsection{Configuration}\label{configuration}

This template exports the \texttt{\ jku-thesis\ } function with the
following named arguments:

\begin{itemize}
\tightlist
\item
  \texttt{\ thesis-type\ } : String
\item
  \texttt{\ degree\ } : String
\item
  \texttt{\ program\ } : String
\item
  \texttt{\ supervisor\ } : String
\item
  \texttt{\ advisor\ } : Array of Strings
\item
  \texttt{\ department\ } : String
\item
  \texttt{\ author\ } : String
\item
  \texttt{\ date\ } : datetime
\item
  \texttt{\ place-of-submission\ } : string
\item
  \texttt{\ title\ } : String
\item
  \texttt{\ abstract-en\ } : Content block
\item
  \texttt{\ abstract-de\ } : optional: Content block or none
\item
  \texttt{\ acknowledgements\ } : optional: Content block or none
\item
  \texttt{\ show-title-in-header\ } : Boolean
\item
  \texttt{\ draft\ } : Boolean
\end{itemize}

The template will initialize your package with a sample call to the
\texttt{\ jku-thesis\ } function.

The dummy thesis, including the sources, was created by generative AI
and is simply meant as a placeholder. The content, citations, and data
presented are not based on actual research or verified information. They
are intended for illustrative purposes only and should not be considered
accurate, reliable, or suitable for any academic, professional, or
research use. Any resemblance to real persons, living or dead, or actual
research, is purely coincidental. Users are advised to replace all
placeholder content with genuine, verified data and references before
using this material in any formal or academic context.

\href{/app?template=not-jku-thesis&version=0.1.0}{Create project in app}

\subsubsection{How to use}\label{how-to-use}

Click the button above to create a new project using this template in
the Typst app.

You can also use the Typst CLI to start a new project on your computer
using this command:

\begin{verbatim}
typst init @preview/not-jku-thesis:0.1.0
\end{verbatim}

\includesvg[width=0.16667in,height=0.16667in]{/assets/icons/16-copy.svg}

\subsubsection{About}\label{about}

\begin{description}
\tightlist
\item[Author :]
Raphael Siegl
\item[License:]
MIT-0
\item[Current version:]
0.1.0
\item[Last updated:]
October 7, 2024
\item[First released:]
October 7, 2024
\item[Minimum Typst version:]
0.11.0
\item[Archive size:]
1.84 MB
\href{https://packages.typst.org/preview/not-jku-thesis-0.1.0.tar.gz}{\pandocbounded{\includesvg[keepaspectratio]{/assets/icons/16-download.svg}}}
\item[Categor y :]
\begin{itemize}
\tightlist
\item[]
\item
  \pandocbounded{\includesvg[keepaspectratio]{/assets/icons/16-mortarboard.svg}}
  \href{https://typst.app/universe/search/?category=thesis}{Thesis}
\end{itemize}
\end{description}

\subsubsection{Where to report issues?}\label{where-to-report-issues}

This template is a project of Raphael Siegl . You can also try to ask
for help with this template on the \href{https://forum.typst.app}{Forum}
.

Please report this template to the Typst team using the
\href{https://typst.app/contact}{contact form} if you believe it is a
safety hazard or infringes upon your rights.

\phantomsection\label{versions}
\subsubsection{Version history}\label{version-history}

\begin{longtable}[]{@{}ll@{}}
\toprule\noalign{}
Version & Release Date \\
\midrule\noalign{}
\endhead
\bottomrule\noalign{}
\endlastfoot
0.1.0 & October 7, 2024 \\
\end{longtable}

Typst GmbH did not create this template and cannot guarantee correct
functionality of this template or compatibility with any version of the
Typst compiler or app.


\section{Package List LaTeX/scholarly-epfl-thesis.tex}
\title{typst.app/universe/package/scholarly-epfl-thesis}

\phantomsection\label{banner}
\phantomsection\label{template-thumbnail}
\pandocbounded{\includegraphics[keepaspectratio]{https://packages.typst.org/preview/thumbnails/scholarly-epfl-thesis-0.1.2-small.webp}}

\section{scholarly-epfl-thesis}\label{scholarly-epfl-thesis}

{ 0.1.2 }

A template for a thesis at EPFL

\href{/app?template=scholarly-epfl-thesis&version=0.1.2}{Create project
in app}

\phantomsection\label{readme}
Port of
\href{https://www.overleaf.com/latex/templates/swiss-federal-institute-of-technology-in-lausanne-epfl-phd-thesis/dhcgtppybcwv}{an
unofficial LaTeX template} to Typst.

A complete example is shown in the
\href{https://github.com/augustebaum/epfl-thesis-typst/blob/v0.1.2/example}{example
folder} ; see
\href{https://github.com/augustebaum/epfl-thesis-typst/blob/v0.1.2/example/main.pdf}{example.pdf}
for the rendered PDF. The document structure can of course be adapted to
your needs.

\subsection{Screenshots}\label{screenshots}

\includegraphics[width=2.08333in,height=\textheight,keepaspectratio]{https://raw.githubusercontent.com/augustebaum/epfl-thesis-typst/v0.1.2/screenshots/cover_page.png}
\includegraphics[width=2.08333in,height=\textheight,keepaspectratio]{https://raw.githubusercontent.com/augustebaum/epfl-thesis-typst/v0.1.2/screenshots/acknowledgements.png}
\includegraphics[width=2.08333in,height=\textheight,keepaspectratio]{https://raw.githubusercontent.com/augustebaum/epfl-thesis-typst/v0.1.2/screenshots/tables_and_figures.png}
\includegraphics[width=2.08333in,height=\textheight,keepaspectratio]{https://raw.githubusercontent.com/augustebaum/epfl-thesis-typst/v0.1.2/screenshots/appendix.png}

\subsection{Usage}\label{usage}

You can use this template in the Typst web app by clicking “Start from
template� on the dashboard and searching for \texttt{\ epfl\ } .

Alternatively, you can use the CLI to kick this project off using the
command

\begin{Shaded}
\begin{Highlighting}[]
\ExtensionTok{typst}\NormalTok{ init @preview/scholarly{-}epfl{-}thesis}
\end{Highlighting}
\end{Shaded}

Typst will create a new directory with all the files needed to get you
started.

This template uses certain fonts, including Utopia Latex for most text.
If the font is not available to Typst, as is the case in the Typst Web
App, then the template will fall back to a default font. The font is
included in example shown in the Github repository
\href{https://github.com/augustebaum/epfl-thesis-typst/blob/v0.1.2/example/utopia_font}{here}
, otherwise you can download it however you like.

\subsubsection{Configuration}\label{configuration}

This template exports the \texttt{\ template\ } function with the
following named arguments:

\begin{itemize}
\tightlist
\item
  \texttt{\ title\ } : The work’s title. Default:
  \texttt{\ {[}Your\ Title{]}\ }
\item
  \texttt{\ author\ } : The author’s name. Default:
  \texttt{\ "Your\ Name"\ }
\item
  \texttt{\ paper-size\ } : The work’s
  \href{https://typst.app/docs/reference/layout/page\#parameters-paper}{paper
  size} . Default: \texttt{\ "a4"\ }
\item
  \texttt{\ date\ } : The work’s date. Unused for now. Default:
  \texttt{\ none\ }
\item
  \texttt{\ date-format\ } : The format for displaying the work’s
  date. By default, the date will be displayed as
  \texttt{\ MMMM\ DD,\ YYYY\ } . Unused for now. Default:
  \texttt{\ {[}month\ repr:long{]}\ {[}day\ padding:zero{]},\ {[}year\ repr:full{]}\ }
\end{itemize}

The template will initialize your package with a basic call to the
\texttt{\ template\ } function in a \texttt{\ show\ } rule. If you,
however, want to change an existing project to use this template, you
can add a show rule like this at the top of your file:

\begin{Shaded}
\begin{Highlighting}[]
\NormalTok{\#import "@preview/scholarly{-}epfl{-}thesis:0.1.2": *}

\NormalTok{\#show: template.with(}
\NormalTok{  title: [Your Title],}
\NormalTok{  author: "Your Name",}
\NormalTok{  date: datetime(year: 2024, month: 03, day: 19),}
\NormalTok{)}

\NormalTok{// Your content goes below.}
\end{Highlighting}
\end{Shaded}

Also included are the \texttt{\ front-matter\ } ,
\texttt{\ main-matter\ } and \texttt{\ back-matter\ } helpers which you
can use in \texttt{\ show\ } rules in your document to change certain
settings when they are called: e.g. reset the page numbering when main
matter starts, or number headings with letters in the back matter. See
\href{https://github.com/augustebaum/epfl-thesis-typst/blob/v0.1.2/example/main.typ}{example/main.typ}
for example usage.

\subsection{Development}\label{development}

In order for Typst to access the Utopia Latex font, you need to include
it your font path. I’ve included the font in \texttt{\ example/\ } so
that you can run this in your shell:

\begin{Shaded}
\begin{Highlighting}[]
\BuiltInTok{cd}\NormalTok{ example}
\ExtensionTok{typst}\NormalTok{ w main.typ }\AttributeTok{{-}{-}font{-}path}\NormalTok{ .}
\end{Highlighting}
\end{Shaded}

See
\href{https://typst.app/docs/reference/text/text/\#parameters-font}{here}
for more about the font path.

\subsection{Credits}\label{credits}

\begin{itemize}
\tightlist
\item
  The creators of the
  \href{https://github.com/talal/ilm/blob/main/lib.typ}{ILM template}
  for the page layout, header and README format which I drew heavily
  from
\item
  The creators of the
  \href{https://www.overleaf.com/latex/templates/swiss-federal-institute-of-technology-in-lausanne-epfl-phd-thesis/dhcgtppybcwv}{original
  LateX template}
\end{itemize}

\subsection{TODO}\label{todo}

\begin{itemize}
\tightlist
\item
  {[} {]} Hide header and page number on empty pages

  \begin{itemize}
  \tightlist
  \item
    Tracking issue: \url{https://github.com/typst/typst/issues/2722}
  \item
    {[} {]} Quick fix:
    \url{https://github.com/typst/typst/issues/2722\#issuecomment-1911150996}

    \begin{itemize}
    \tightlist
    \item
      Tried it, when I put it in the
      \texttt{\ show\ heading.where(level:\ 1)\ } it disrupts the
      outline. I guess it would work if you put in the
      \texttt{\ metadata\ } manually before each chapter.
    \end{itemize}
  \item
    Optionally don’t force an empty page
  \end{itemize}
\item
  {[} {]} Table of contents

  \begin{itemize}
  \tightlist
  \item
    {[} {]} Join abstracts into one outline entry

    \begin{itemize}
    \tightlist
    \item
      I removed the lines for the German and French abstracts so it
      takes less space, but it’s not exactly the same as the original
      which has a custom outline entry
    \end{itemize}
  \item
    {[} {]} Style

    \begin{itemize}
    \tightlist
    \item
      {[} {]} Space between heading number and heading
    \item
      {[} {]} Level 1 Headings are bold and don’t have dot lines
      between the heading and the page number
    \item
      \url{https://sitandr.github.io/typst-examples-book/book/snippets/chapters/outlines.html}
    \item
      \texttt{\ outline.entry\ } can’t be modified easily because the
      arguments are positional

      \begin{itemize}
      \tightlist
      \item
        I found a solution on discord but it strips away the links. I
        tried putting in a \texttt{\ link\ } manually but that gets
        formatted like a link in the text, which is not what we’re
        looking for.

        \begin{itemize}
        \tightlist
        \item
          A solution to that link issue can be found in this thread:
          \url{https://discord.com/channels/1054443721975922748/1231526650462736474}
        \end{itemize}
      \end{itemize}
    \item
      I might use \url{https://typst.app/universe/package/outrageous}
    \end{itemize}
  \item
    {[}x{]} Include list of figures and tables
  \end{itemize}
\item
  {[} {]} Figures

  \begin{itemize}
  \tightlist
  \item
    {[} {]} Subfigures

    \begin{itemize}
    \tightlist
    \item
      tracking issue: \url{https://github.com/typst/typst/issues/246}
    \item
      A wip: \url{https://github.com/tingerrr/subpar}
    \item
      A quickfix:
      \url{https://github.com/typst/typst/issues/246\#issuecomment-1928735969}

      \begin{itemize}
      \tightlist
      \item
        Works if you abuse the \texttt{\ kind\ } mechanic, but I can’t
        get the superfigure’s caption centered
      \end{itemize}
    \end{itemize}
  \item
    {[}x{]} Short caption for table of contents

    \begin{itemize}
    \tightlist
    \item
      \url{https://sitandr.github.io/typst-examples-book/book/snippets/chapters/outlines.html}
    \end{itemize}
  \item
    {[}x{]} Numbering

    \begin{itemize}
    \tightlist
    \item
      i-figured?
    \end{itemize}
  \end{itemize}
\item
  {[} {]} Chemistry examples?
\item
  {[} {]} CV?
\item
  {[}x{]} Spacing after heading is different depending on if we’re in
  frontmatter or main matter
\item
  {[}x{]} Cover page should take its values from the template arguments

  \begin{itemize}
  \tightlist
  \item
    cover page is separate from template, given that it is not meant to
    be printed anyways

    \begin{itemize}
    \tightlist
    \item
      this also reflects how the latex template works
    \end{itemize}
  \end{itemize}
\item
  {[}x{]} Spacing before new sub-heading
\item
  {[}x{]} Readme

  \begin{itemize}
  \tightlist
  \item
    {[}x{]} How-to
  \item
    {[}x{]} Screenshots
  \item
    {[}x{]} Thumbnail
  \end{itemize}
\item
  {[}x{]} Refactor to \texttt{\ front-matter\ } ,
  \texttt{\ main-matter\ } …
\item
  {[}x{]} Numbering

  \begin{itemize}
  \tightlist
  \item
    {[}x{]} Why are pagenumbers bold on certain pages?

    \begin{itemize}
    \tightlist
    \item
      There was a show rule that inserted a pagebreak before each
      chapter. This produced a bug where the chapter start pages was
      inconsistent with the information Typst has.
    \end{itemize}
  \item
    {[}x{]} numbering starts on acknowledgements (or somewhere else?)
  \end{itemize}
\item
  {[}x{]} Equations

  \begin{itemize}
  \tightlist
  \item
    {[}x{]} Numbering

    \begin{itemize}
    \tightlist
    \item
      \url{https://sitandr.github.io/typst-examples-book/book/snippets/math/numbering.html}
    \end{itemize}
  \item
    {[}x{]} Align left

    \begin{itemize}
    \tightlist
    \item
      Why did \texttt{\ pad\ } work and not \texttt{\ h\ } ?
    \end{itemize}
  \end{itemize}
\item
  {[}x{]} page numbers are too low in the page
\item
  {[}x{]} First-line indent for front matter

  \begin{itemize}
  \tightlist
  \item
    \url{https://typst.app/docs/reference/model/par/\#parameters-first-line-indent}
  \item
    Actually this looks unintentional?
  \end{itemize}
\item
  {[}x{]} Appendices
\item
  {[}x{]} Margins
\item
  {[}x{]} Tables

  \begin{itemize}
  \tightlist
  \item
    {[}x{]} Style
  \end{itemize}
\end{itemize}

\href{/app?template=scholarly-epfl-thesis&version=0.1.2}{Create project
in app}

\subsubsection{How to use}\label{how-to-use}

Click the button above to create a new project using this template in
the Typst app.

You can also use the Typst CLI to start a new project on your computer
using this command:

\begin{verbatim}
typst init @preview/scholarly-epfl-thesis:0.1.2
\end{verbatim}

\includesvg[width=0.16667in,height=0.16667in]{/assets/icons/16-copy.svg}

\subsubsection{About}\label{about}

\begin{description}
\tightlist
\item[Author :]
\href{https://github.com/augustebaum}{Auguste Baum}
\item[License:]
MIT
\item[Current version:]
0.1.2
\item[Last updated:]
November 4, 2024
\item[First released:]
May 3, 2024
\item[Minimum Typst version:]
0.11.0
\item[Archive size:]
425 kB
\href{https://packages.typst.org/preview/scholarly-epfl-thesis-0.1.2.tar.gz}{\pandocbounded{\includesvg[keepaspectratio]{/assets/icons/16-download.svg}}}
\item[Repository:]
\href{https://github.com/augustebaum/epfl-thesis-typst}{GitHub}
\item[Categor y :]
\begin{itemize}
\tightlist
\item[]
\item
  \pandocbounded{\includesvg[keepaspectratio]{/assets/icons/16-mortarboard.svg}}
  \href{https://typst.app/universe/search/?category=thesis}{Thesis}
\end{itemize}
\end{description}

\subsubsection{Where to report issues?}\label{where-to-report-issues}

This template is a project of Auguste Baum . Report issues on
\href{https://github.com/augustebaum/epfl-thesis-typst}{their
repository} . You can also try to ask for help with this template on the
\href{https://forum.typst.app}{Forum} .

Please report this template to the Typst team using the
\href{https://typst.app/contact}{contact form} if you believe it is a
safety hazard or infringes upon your rights.

\phantomsection\label{versions}
\subsubsection{Version history}\label{version-history}

\begin{longtable}[]{@{}ll@{}}
\toprule\noalign{}
Version & Release Date \\
\midrule\noalign{}
\endhead
\bottomrule\noalign{}
\endlastfoot
0.1.2 & November 4, 2024 \\
\href{https://typst.app/universe/package/scholarly-epfl-thesis/0.1.1/}{0.1.1}
& July 18, 2024 \\
\href{https://typst.app/universe/package/scholarly-epfl-thesis/0.1.0/}{0.1.0}
& May 3, 2024 \\
\end{longtable}

Typst GmbH did not create this template and cannot guarantee correct
functionality of this template or compatibility with any version of the
Typst compiler or app.


\section{Package List LaTeX/arkheion.tex}
\title{typst.app/universe/package/arkheion}

\phantomsection\label{banner}
\phantomsection\label{template-thumbnail}
\pandocbounded{\includegraphics[keepaspectratio]{https://packages.typst.org/preview/thumbnails/arkheion-0.1.0-small.webp}}

\section{arkheion}\label{arkheion}

{ 0.1.0 }

A simple template reproducing popular arXiv templates.

\href{/app?template=arkheion&version=0.1.0}{Create project in app}

\phantomsection\label{readme}
A Typst template based on popular LateX template used in arXiv and
bio-arXiv. Inspired by
\href{https://github.com/kourgeorge/arxiv-style}{arxiv-style}

\subsection{Usage}\label{usage}

\textbf{Import}

\begin{verbatim}
#import "@preview/arkheion:0.1.0": arkheion, arkheion-appendices
\end{verbatim}

\textbf{Main body}

\begin{verbatim}
#show: arkheion.with(
  title: "ArXiv Typst Template",
  authors: (
    (name: "Author 1", email: "user@domain.com", affiliation: "Company", orcid: "0000-0000-0000-0000"),
    (name: "Author 2", email: "user@domain.com", affiliation: "Company"),
  ),
  // Insert your abstract after the colon, wrapped in brackets.
  // Example: `abstract: [This is my abstract...]`
  abstract: lorem(55),
  keywords: ("First keyword", "Second keyword", "etc."),
  date: "May 16, 2023",
)
\end{verbatim}

\textbf{Appendix}

\begin{verbatim}
#show: arkheion-appendices
=

== Appendix section

#lorem(100)
\end{verbatim}

\subsection{License}\label{license}

The MIT License (MIT)

Copyright © 2023 Manuel Goulão

Permission is hereby granted, free of charge, to any person obtaining a
copy of this software and associated documentation files (the
“Software�), to deal in the Software without restriction, including
without limitation the rights to use, copy, modify, merge, publish,
distribute, sublicense, and/or sell copies of the Software, and to
permit persons to whom the Software is furnished to do so, subject to
the following conditions:

The above copyright notice and this permission notice shall be included
in all copies or substantial portions of the Software.

THE SOFTWARE IS PROVIDED “AS IS�, WITHOUT WARRANTY OF ANY KIND,
EXPRESS OR IMPLIED, INCLUDING BUT NOT LIMITED TO THE WARRANTIES OF
MERCHANTABILITY, FITNESS FOR A PARTICULAR PURPOSE AND NONINFRINGEMENT.
IN NO EVENT SHALL THE AUTHORS OR COPYRIGHT HOLDERS BE LIABLE FOR ANY
CLAIM, DAMAGES OR OTHER LIABILITY, WHETHER IN AN ACTION OF CONTRACT,
TORT OR OTHERWISE, ARISING FROM, OUT OF OR IN CONNECTION WITH THE
SOFTWARE OR THE USE OR OTHER DEALINGS IN THE SOFTWARE.

\href{/app?template=arkheion&version=0.1.0}{Create project in app}

\subsubsection{How to use}\label{how-to-use}

Click the button above to create a new project using this template in
the Typst app.

You can also use the Typst CLI to start a new project on your computer
using this command:

\begin{verbatim}
typst init @preview/arkheion:0.1.0
\end{verbatim}

\includesvg[width=0.16667in,height=0.16667in]{/assets/icons/16-copy.svg}

\subsubsection{About}\label{about}

\begin{description}
\tightlist
\item[Author :]
Manuel Goulão
\item[License:]
MIT
\item[Current version:]
0.1.0
\item[Last updated:]
March 23, 2024
\item[First released:]
March 23, 2024
\item[Archive size:]
4.85 kB
\href{https://packages.typst.org/preview/arkheion-0.1.0.tar.gz}{\pandocbounded{\includesvg[keepaspectratio]{/assets/icons/16-download.svg}}}
\item[Repository:]
\href{https://github.com/mgoulao/arkheion}{GitHub}
\item[Discipline s :]
\begin{itemize}
\tightlist
\item[]
\item
  \href{https://typst.app/universe/search/?discipline=engineering}{Engineering}
\item
  \href{https://typst.app/universe/search/?discipline=computer-science}{Computer
  Science}
\end{itemize}
\item[Categor y :]
\begin{itemize}
\tightlist
\item[]
\item
  \pandocbounded{\includesvg[keepaspectratio]{/assets/icons/16-atom.svg}}
  \href{https://typst.app/universe/search/?category=paper}{Paper}
\end{itemize}
\end{description}

\subsubsection{Where to report issues?}\label{where-to-report-issues}

This template is a project of Manuel Goulão . Report issues on
\href{https://github.com/mgoulao/arkheion}{their repository} . You can
also try to ask for help with this template on the
\href{https://forum.typst.app}{Forum} .

Please report this template to the Typst team using the
\href{https://typst.app/contact}{contact form} if you believe it is a
safety hazard or infringes upon your rights.

\phantomsection\label{versions}
\subsubsection{Version history}\label{version-history}

\begin{longtable}[]{@{}ll@{}}
\toprule\noalign{}
Version & Release Date \\
\midrule\noalign{}
\endhead
\bottomrule\noalign{}
\endlastfoot
0.1.0 & March 23, 2024 \\
\end{longtable}

Typst GmbH did not create this template and cannot guarantee correct
functionality of this template or compatibility with any version of the
Typst compiler or app.


\section{Package List LaTeX/pintorita.tex}
\title{typst.app/universe/package/pintorita}

\phantomsection\label{banner}
\section{pintorita}\label{pintorita}

{ 0.1.2 }

Package to draw Sequence Diagrams, Entity Relationship Diagrams,
Component Diagrams, Activity Diagrams, Mind Maps, Gantt Diagrams, and
DOT Diagrams based on Pintora which is heavily influenced by mermaid.js
and plantuml.

\phantomsection\label{readme}
\href{https://pintorajs.vercel.app/}{Pintora}

Typst package for drawing the following from markup:

\begin{itemize}
\tightlist
\item
  Sequence Diagram
\item
  Entity Relationship Diagram
\item
  Component Diagram
\item
  Activity Diagram
\item
  Mind Map Experiment
\item
  Gantt Diagram Experiment
\item
  DOT Diagram Experiment
\end{itemize}

\pandocbounded{\includesvg[keepaspectratio]{https://github.com/typst/packages/raw/main/packages/preview/pintorita/0.1.2/pintorita.svg}}

\begin{Shaded}
\begin{Highlighting}[]
\NormalTok{\#import "@preview/pintorita:0.1.2"}

\NormalTok{\#set page(height: auto, width: auto, fill: black, margin: 2em)}
\NormalTok{\#set text(fill: white)}

\NormalTok{\#show raw.where(lang: "pintora"): it =\textgreater{} pintorita.render(it.text)}

\NormalTok{= pintora}

\NormalTok{Typst just got a load of diagrams. }

\NormalTok{\textasciigrave{}\textasciigrave{}\textasciigrave{}pintora}
\NormalTok{mindmap}
\NormalTok{@param layoutDirection TB}
\NormalTok{+ UML Diagrams}
\NormalTok{++ Behavior Diagrams}
\NormalTok{+++ Sequence Diagram}
\NormalTok{+++ State Diagram}
\NormalTok{+++ Activity Diagram}
\NormalTok{++ Structural Diagrams}
\NormalTok{+++ Class Diagram}
\NormalTok{+++ Component Diagram}
\NormalTok{\textasciigrave{}\textasciigrave{}\textasciigrave{}}

\NormalTok{\textasciigrave{}\textasciigrave{}\textasciigrave{}}
\NormalTok{mindmap}
\NormalTok{@param layoutDirection TB}
\NormalTok{+ UML Diagrams}
\NormalTok{++ Behavior Diagrams}
\NormalTok{+++ Sequence Diagram}
\NormalTok{+++ State Diagram}
\NormalTok{+++ Activity Diagram}
\NormalTok{++ Structural Diagrams}
\NormalTok{+++ Class Diagram}
\NormalTok{+++ Component Diagram}
\NormalTok{\textasciigrave{}\textasciigrave{}\textasciigrave{}}
\end{Highlighting}
\end{Shaded}

\subsection{Documentation}\label{documentation}

\subsubsection{\texorpdfstring{\texttt{\ render\ }}{ render }}\label{render}

Render a pintora string to an image

\paragraph{Arguments}\label{arguments}

\begin{itemize}
\tightlist
\item
  \texttt{\ src\ } : \texttt{\ str\ } - pintora source string
\item
  \texttt{\ factor\ } : scale output svg, “factor:0.5� will scale
  images down by half, so scale can be consistent across renders.
\item
  \texttt{\ style\ } : \texttt{\ str\ } diagram style,
  \texttt{\ default\ } or \texttt{\ dark\ } or \texttt{\ larkLight\ } or
  \texttt{\ larkDark\ }
\item
  \texttt{\ font\ } : \texttt{\ str\ } font family, default is
  \texttt{\ Source\ Code\ Pro,\ sans-serif\ }
\item
  All other arguments are passed to \texttt{\ image.decode\ } so you can
  customize the image size
\end{itemize}

\paragraph{Returns}\label{returns}

The image, of type \texttt{\ content\ }

\subsubsection{\texorpdfstring{\texttt{\ render-svg\ }}{ render-svg }}\label{render-svg}

Render a pintora string to an image

\paragraph{Arguments}\label{arguments-1}

\begin{itemize}
\tightlist
\item
  \texttt{\ src\ } : \texttt{\ str\ } - pintora source string
\item
  \texttt{\ factor\ } : scale output svg, “factor:0.5� will scale
  images down by half, so scale can be consistent across renders.
\item
  \texttt{\ style\ } : \texttt{\ str\ } diagram style,
  \texttt{\ default\ } or \texttt{\ dark\ } or \texttt{\ larkLight\ } or
  \texttt{\ larkDark\ }
\item
  \texttt{\ font\ } : \texttt{\ str\ } font family, default is
  \texttt{\ Source\ Code\ Pro,\ sans-serif\ }
\item
  All other arguments are passed to \texttt{\ image.decode\ } so you can
  customize the image size
\end{itemize}

\paragraph{Returns}\label{returns-1}

The svg image

\subsection{History}\label{history}

\begin{itemize}
\tightlist
\item
  0.1.0 - Inital Release
\item
  0.1.1 - Updated to Jogs 0.2.3 and pintora 0.7.3
\item
  0.1.2 - Fixed strange offset of text rows in class diagram, added
  \texttt{\ render-svg\ } function and more customization options
\end{itemize}

\subsubsection{How to add}\label{how-to-add}

Copy this into your project and use the import as \texttt{\ pintorita\ }

\begin{verbatim}
#import "@preview/pintorita:0.1.2"
\end{verbatim}

\includesvg[width=0.16667in,height=0.16667in]{/assets/icons/16-copy.svg}

Check the docs for
\href{https://typst.app/docs/reference/scripting/\#packages}{more
information on how to import packages} .

\subsubsection{About}\label{about}

\begin{description}
\tightlist
\item[Author s :]
Min Chen (hikerpig) \& Taylorh140
\item[License:]
MIT
\item[Current version:]
0.1.2
\item[Last updated:]
October 4, 2024
\item[First released:]
January 16, 2024
\item[Archive size:]
725 kB
\href{https://packages.typst.org/preview/pintorita-0.1.2.tar.gz}{\pandocbounded{\includesvg[keepaspectratio]{/assets/icons/16-download.svg}}}
\item[Repository:]
\href{https://github.com/taylorh140/typst-pintora}{GitHub}
\item[Categor y :]
\begin{itemize}
\tightlist
\item[]
\item
  \pandocbounded{\includesvg[keepaspectratio]{/assets/icons/16-chart.svg}}
  \href{https://typst.app/universe/search/?category=visualization}{Visualization}
\end{itemize}
\end{description}

\subsubsection{Where to report issues?}\label{where-to-report-issues}

This package is a project of Min Chen (hikerpig) and Taylorh140 . Report
issues on \href{https://github.com/taylorh140/typst-pintora}{their
repository} . You can also try to ask for help with this package on the
\href{https://forum.typst.app}{Forum} .

Please report this package to the Typst team using the
\href{https://typst.app/contact}{contact form} if you believe it is a
safety hazard or infringes upon your rights.

\phantomsection\label{versions}
\subsubsection{Version history}\label{version-history}

\begin{longtable}[]{@{}ll@{}}
\toprule\noalign{}
Version & Release Date \\
\midrule\noalign{}
\endhead
\bottomrule\noalign{}
\endlastfoot
0.1.2 & October 4, 2024 \\
\href{https://typst.app/universe/package/pintorita/0.1.1/}{0.1.1} &
April 3, 2024 \\
\href{https://typst.app/universe/package/pintorita/0.1.0/}{0.1.0} &
January 16, 2024 \\
\end{longtable}

Typst GmbH did not create this package and cannot guarantee correct
functionality of this package or compatibility with any version of the
Typst compiler or app.


\section{Package List LaTeX/typearea.tex}
\title{typst.app/universe/package/typearea}

\phantomsection\label{banner}
\section{typearea}\label{typearea}

{ 0.2.0 }

A KOMA-Script inspired package to better configure your typearea and
margins.

\phantomsection\label{readme}
A KOMA-Script inspired package to better configure your typearea and
margins.

\begin{Shaded}
\begin{Highlighting}[]
\NormalTok{\#import "@preview/typearea:0.2.0": typearea}

\NormalTok{\#show: typearea.with(}
\NormalTok{  paper: "a4",}
\NormalTok{  div: 9,}
\NormalTok{  binding{-}correction: 11mm,}
\NormalTok{)}

\NormalTok{= Hello World}
\end{Highlighting}
\end{Shaded}

\subsection{Reference}\label{reference}

\texttt{\ typearea\ } accepts the following options:

\subsubsection{two-sided}\label{two-sided}

Whether the document is two-sided. Defaults to \texttt{\ true\ } .

\subsubsection{binding-correction}\label{binding-correction}

Binding correction. Defaults to \texttt{\ 0pt\ } .

Additional margin on the inside of a page when two-sided is true. If
two-sided is false it will be on the left or right side, depending on
the value of \texttt{\ binding\ } . A \texttt{\ binding\ } value of
\texttt{\ auto\ } will currently default to \texttt{\ left\ } .

Note: Before version 0.2.0 this was called \texttt{\ bcor\ } .

\subsubsection{div}\label{div}

How many equal parts to split the page into. Controls the margins.
Defautls to \texttt{\ 9\ } .

The top and bottom margin will always be one and two parts respectively.
In two-sided mode the inside margin will be one part and the outside
margin two parts, so the combined margins between the text on the left
side and the text on the right side is the same as the margins from the
outer edge of the text to the outer edge of the page.

In one-sided mode the left and right margin will take 1.5 parts each.

\subsubsection{header-height /
footer-height}\label{header-height-footer-height}

The height of the page header/footer.

\subsubsection{header-sep / footer-sep}\label{header-sep-footer-sep}

The distance between the page header/footer and the text area.

\subsubsection{header-include /
footer-include}\label{header-include-footer-include}

Whether the header/footer should be counted as part of the text area
when calculating the margins. Defaults to \texttt{\ false\ } .

\subsubsection{…rest}\label{uxe2rest}

All other arguments are passed on to \texttt{\ page()\ } as is. You can
see which arguments \texttt{\ page()\ } accepts in the
\href{https://typst.app/docs/reference/layout/page/}{typst reference for
the page function} .

You should prefer this over calling or setting \texttt{\ page()\ }
directly as doing so could break some of \texttt{\ typearea\ } ’s
functionality.

\subsubsection{How to add}\label{how-to-add}

Copy this into your project and use the import as \texttt{\ typearea\ }

\begin{verbatim}
#import "@preview/typearea:0.2.0"
\end{verbatim}

\includesvg[width=0.16667in,height=0.16667in]{/assets/icons/16-copy.svg}

Check the docs for
\href{https://typst.app/docs/reference/scripting/\#packages}{more
information on how to import packages} .

\subsubsection{About}\label{about}

\begin{description}
\tightlist
\item[Author :]
Adrian Freund
\item[License:]
MIT
\item[Current version:]
0.2.0
\item[Last updated:]
June 13, 2024
\item[First released:]
October 29, 2023
\item[Archive size:]
2.39 kB
\href{https://packages.typst.org/preview/typearea-0.2.0.tar.gz}{\pandocbounded{\includesvg[keepaspectratio]{/assets/icons/16-download.svg}}}
\item[Repository:]
\href{https://github.com/freundTech/typst-typearea}{GitHub}
\end{description}

\subsubsection{Where to report issues?}\label{where-to-report-issues}

This package is a project of Adrian Freund . Report issues on
\href{https://github.com/freundTech/typst-typearea}{their repository} .
You can also try to ask for help with this package on the
\href{https://forum.typst.app}{Forum} .

Please report this package to the Typst team using the
\href{https://typst.app/contact}{contact form} if you believe it is a
safety hazard or infringes upon your rights.

\phantomsection\label{versions}
\subsubsection{Version history}\label{version-history}

\begin{longtable}[]{@{}ll@{}}
\toprule\noalign{}
Version & Release Date \\
\midrule\noalign{}
\endhead
\bottomrule\noalign{}
\endlastfoot
0.2.0 & June 13, 2024 \\
\href{https://typst.app/universe/package/typearea/0.1.0/}{0.1.0} &
October 29, 2023 \\
\end{longtable}

Typst GmbH did not create this package and cannot guarantee correct
functionality of this package or compatibility with any version of the
Typst compiler or app.


\section{Package List LaTeX/super-suboptimal.tex}
\title{typst.app/universe/package/super-suboptimal}

\phantomsection\label{banner}
\section{super-suboptimal}\label{super-suboptimal}

{ 0.1.0 }

Unicode super- and subscripts in equations.

\phantomsection\label{readme}
A Typst package enabling support for Unicode super- and subscript
characters in equations.

\subsection{Usage}\label{usage}

The package exposes the template-function \texttt{\ super-subscripts\ }
. It affects all \texttt{\ math.equation\ } s by attaching every
superscript- and subscript-character to the first non-space-element on
its left.

\begin{Shaded}
\begin{Highlighting}[]
\NormalTok{\#import "@preview/super{-}suboptimal:0.1.0": *}
\NormalTok{\#show: super{-}subscripts}

\NormalTok{For all $(x,y)∈ℝ²$:}
\NormalTok{$}
\NormalTok{  q := norm((x,y))₂ \textless{} 1}
\NormalTok{  ==\textgreater{} ∑ᵢ₌₁ⁿ q ⁱ \textless{} ∞}
\NormalTok{$}
\end{Highlighting}
\end{Shaded}

\pandocbounded{\includesvg[keepaspectratio]{https://github.com/typst/packages/raw/main/packages/preview/super-suboptimal/0.1.0/assets/example0.svg}}

Because code like \texttt{\ \$x+yᶻ\$\ } throws an “unknown
variable� error, the package also exposes the function \texttt{\ eq\ }
, which inserts spaces before every superscript- and subscript-character
and passing it on to \texttt{\ math.equation\ } . This comes at the cost
of missing syntax-highlighting and code-suggestions in your IDE.

\texttt{\ eq\ } accepts a \texttt{\ raw\ } string as a positional
parameter, and an argument-sink that’s passed onto
\texttt{\ math.equation\ } . Unless specified otherwise in the
argument-sink, the resulting equation is typeset with
\texttt{\ block:\ true\ } if and only if the \texttt{\ raw\ } also
satisfied \texttt{\ block:\ true\ } . \texttt{\ eq\ } is automatically
applied to every \texttt{\ raw\ } with \texttt{\ lang:\ "eq"\ } .

\begin{Shaded}
\begin{Highlighting}[]
\NormalTok{\#eq(\textasciigrave{}0 = aᵇ\textasciigrave{})}

\NormalTok{\#eq(\textasciigrave{}\textasciigrave{}\textasciigrave{}}
\NormalTok{  1 = x+yᶻ}
\NormalTok{\textasciigrave{}\textasciigrave{}\textasciigrave{})}

\NormalTok{\#eq(\textasciigrave{}2 = aᵇ\textasciigrave{}, block: true, numbering: "(1)")}

\NormalTok{\textasciigrave{}\textasciigrave{}\textasciigrave{}eq}
\NormalTok{  3 = aᵇᶜ⁺ᵈ₃ₑ⁽ᶠ⁻ᵍ⁾ₕᵢ}
\NormalTok{\textasciigrave{}\textasciigrave{}\textasciigrave{}}
\end{Highlighting}
\end{Shaded}

\pandocbounded{\includesvg[keepaspectratio]{https://github.com/typst/packages/raw/main/packages/preview/super-suboptimal/0.1.0/assets/example1.svg}}

Sometimes in mathematical writing, variables are decorated with an
asterisk, e.g. \texttt{\ \$x\^{}*\$\ } . The character \texttt{\ ꙳\ }
can now be used, as well: \texttt{\ \$x꙳\ =\ x\^{}*\$\ } .

\subsection{Known issues}\label{known-issues}

\begin{itemize}
\item
  As mentioned above, \texttt{\ \$aᵇ\$\ } leads to an “unknown
  variable� error. As a workaround, \texttt{\ \$a\ ᵇ\$\ } produces
  the same output, or you can use the \texttt{\ eq\ } function described
  above.

  \begin{itemize}
  \tightlist
  \item
    The first workaround also means I can’t reasonably implement
    top-left and bottom-left attachments. For example,
    \texttt{\ \$a\ ³b\$\ } is rendered like
    \texttt{\ \$attach(a,\ t:\ 3)\ b\$\ } , rather than
    \texttt{\ \$a\ attach(b,\ tl:\ 3)\$\ } .
  \end{itemize}
\item
  Multiple attachments are concatenated into one content without another
  pass of \texttt{\ equation\ } . For example,
  \texttt{\ \#eq(\textasciigrave{}0ˢ�����\textasciigrave{})\ }
  is equivalent to \texttt{\ \$0\^{}(s\ i\ n\ "("\ k\ ")")\$\ } , rather
  than \texttt{\ \$0\^{}sin(k)\$\ } . I won’t fix this, because:

  \begin{itemize}
  \tightlist
  \item
    Another pass of \texttt{\ equation\ } would cause performance issues
    at best, and infinite loops at worst.
  \item
    If this were fixed, code such as \texttt{\ \$e\ ˣ\ ʸ\$\ } would
    undesirably produce an “unknown variable \texttt{\ xy\ } �
    error.
  \end{itemize}
\item
  Let’s call a piece of content “small� if it consists of only a
  single non-separated sequence of characters in Typst (internally, this
  is the distinction between the content-functions \texttt{\ sequence\ }
  and \texttt{\ text\ } ). For instance, \texttt{\ \$1234\$\ } and
  \texttt{\ \$a\$\ } constitute “small� content, but
  \texttt{\ \$a\ b\$\ } and \texttt{\ \$3a\$\ } and
  \texttt{\ \$1+2+3+4+5\$\ } do not.

  This package only runs on non-“small� pieces of content. For
  example, \texttt{\ \$sqrt(35²)\$\ } still renders with the
  default-Unicode-character and will look different from
  \texttt{\ \$sqrt(35\^{}2)\$\ } . On the other hand,
  \texttt{\ \$sqrt(aâ?¶)\$\ } \emph{is} rendered correctly. This is
  because \texttt{\ 35²\ } constitutes “small� content, but
  \texttt{\ aâ?¶\ } does not.

  A workaround is implemented for “small� content immediately within
  an equation, i.e. not nested within another content-function. For
  example, \texttt{\ \$7²\$\ } renders the same as
  \texttt{\ \$7\^{}2\$\ } , even though it’s “small� content.
\item
  Equations within other content-elements might trigger multiple
  show-rule-passes, possibly causing performance-issues.
\end{itemize}

\subsubsection{How to add}\label{how-to-add}

Copy this into your project and use the import as
\texttt{\ super-suboptimal\ }

\begin{verbatim}
#import "@preview/super-suboptimal:0.1.0"
\end{verbatim}

\includesvg[width=0.16667in,height=0.16667in]{/assets/icons/16-copy.svg}

Check the docs for
\href{https://typst.app/docs/reference/scripting/\#packages}{more
information on how to import packages} .

\subsubsection{About}\label{about}

\begin{description}
\tightlist
\item[Author s :]
Eric Biedert \& Lumi
\item[License:]
MIT
\item[Current version:]
0.1.0
\item[Last updated:]
January 29, 2024
\item[First released:]
January 29, 2024
\item[Archive size:]
6.15 kB
\href{https://packages.typst.org/preview/super-suboptimal-0.1.0.tar.gz}{\pandocbounded{\includesvg[keepaspectratio]{/assets/icons/16-download.svg}}}
\end{description}

\subsubsection{Where to report issues?}\label{where-to-report-issues}

This package is a project of Eric Biedert and Lumi . You can also try to
ask for help with this package on the
\href{https://forum.typst.app}{Forum} .

Please report this package to the Typst team using the
\href{https://typst.app/contact}{contact form} if you believe it is a
safety hazard or infringes upon your rights.

\phantomsection\label{versions}
\subsubsection{Version history}\label{version-history}

\begin{longtable}[]{@{}ll@{}}
\toprule\noalign{}
Version & Release Date \\
\midrule\noalign{}
\endhead
\bottomrule\noalign{}
\endlastfoot
0.1.0 & January 29, 2024 \\
\end{longtable}

Typst GmbH did not create this package and cannot guarantee correct
functionality of this package or compatibility with any version of the
Typst compiler or app.


\section{Package List LaTeX/note-me.tex}
\title{typst.app/universe/package/note-me}

\phantomsection\label{banner}
\section{note-me}\label{note-me}

{ 0.3.0 }

Adds GitHub-style Admonitions (Alerts) to Typst.

\phantomsection\label{readme}
\begin{quote}
{[}!NOTE{]} Add GitHub style admonitions (also known as alerts) to
Typst.
\end{quote}

\subsection{Usage}\label{usage}

Import this package, and do

\begin{Shaded}
\begin{Highlighting}[]
\NormalTok{// Import from @preview namespace is suggested}
\NormalTok{// \#import "@preview/note{-}me:0.3.0": *}

\NormalTok{// Import from @local namespace is only for debugging purpose}
\NormalTok{// \#import "@local/note{-}me:0.3.0": *}

\NormalTok{// Import relatively is for development purpose}
\NormalTok{\#import "lib.typ": *}

\NormalTok{= Basic Examples}

\NormalTok{\#note[}
\NormalTok{  Highlights information that users should take into account, even when skimming.}
\NormalTok{]}

\NormalTok{\#tip[}
\NormalTok{  Optional information to help a user be more successful.}
\NormalTok{]}

\NormalTok{\#important[}
\NormalTok{  Crucial information necessary for users to succeed.}
\NormalTok{]}

\NormalTok{\#warning[}
\NormalTok{  Critical content demanding immediate user attention due to potential risks.}
\NormalTok{]}

\NormalTok{\#caution[}
\NormalTok{  Negative potential consequences of an action.}
\NormalTok{]}

\NormalTok{\#admonition(}
\NormalTok{  icon{-}path: "icons/stop.svg",}
\NormalTok{  color: color.fuchsia,}
\NormalTok{  title: "Customize",}
\NormalTok{  foreground{-}color: color.white,}
\NormalTok{  background{-}color: color.black,}
\NormalTok{)[}
\NormalTok{  The icon, (theme) color, title, foreground and background color are customizable.}
\NormalTok{]}

\NormalTok{\#admonition(}
\NormalTok{  icon{-}string: read("icons/light{-}bulb.svg"),}
\NormalTok{  color: color.fuchsia,}
\NormalTok{  title: "Customize",}
\NormalTok{)[}
\NormalTok{  The icon can be specified as a string of SVG. This is useful if the user want to use an SVG icon that is not available in this package.}
\NormalTok{]}

\NormalTok{\#admonition(}
\NormalTok{  icon: [🙈],}
\NormalTok{  color: color.fuchsia,}
\NormalTok{  title: "Customize",}
\NormalTok{)[}
\NormalTok{  Or, pass a content directly as the icon...}
\NormalTok{]}

\NormalTok{= More Examples}

\NormalTok{\#todo[}
\NormalTok{  Fix \textasciigrave{}note{-}me\textasciigrave{} package.}
\NormalTok{]}


\NormalTok{= Prevent Page Breaks from Breaking Admonitions}

\NormalTok{\#box(}
\NormalTok{  width: 1fr,}
\NormalTok{  height: 50pt,}
\NormalTok{  fill: gray,}
\NormalTok{)}

\NormalTok{\#note[}
\NormalTok{  \#lorem(100)}
\NormalTok{]}
\end{Highlighting}
\end{Shaded}

\pandocbounded{\includesvg[keepaspectratio]{https://github.com/typst/packages/raw/main/packages/preview/note-me/0.3.0/example.svg}}

Further Reading:

\begin{itemize}
\tightlist
\item
  \url{https://github.com/orgs/community/discussions/16925}
\item
  \url{https://docs.asciidoctor.org/asciidoc/latest/blocks/admonitions/}
\end{itemize}

\subsection{Style}\label{style}

It borrows the style of GitHub’s admonition.

\begin{quote}
{[}!NOTE{]}\\
Highlights information that users should take into account, even when
skimming.
\end{quote}

\begin{quote}
{[}!TIP{]} Optional information to help a user be more successful.
\end{quote}

\begin{quote}
{[}!IMPORTANT{]}\\
Crucial information necessary for users to succeed.
\end{quote}

\begin{quote}
{[}!WARNING{]}\\
Critical content demanding immediate user attention due to potential
risks.
\end{quote}

\begin{quote}
{[}!CAUTION{]} Negative potential consequences of an action.
\end{quote}

\subsection{Credits}\label{credits}

The admonition icons are from
\href{https://github.com/primer/octicons}{Octicons} .

\subsubsection{How to add}\label{how-to-add}

Copy this into your project and use the import as \texttt{\ note-me\ }

\begin{verbatim}
#import "@preview/note-me:0.3.0"
\end{verbatim}

\includesvg[width=0.16667in,height=0.16667in]{/assets/icons/16-copy.svg}

Check the docs for
\href{https://typst.app/docs/reference/scripting/\#packages}{more
information on how to import packages} .

\subsubsection{About}\label{about}

\begin{description}
\tightlist
\item[Author :]
Flandia Yingman
\item[License:]
MIT
\item[Current version:]
0.3.0
\item[Last updated:]
September 30, 2024
\item[First released:]
February 11, 2024
\item[Archive size:]
5.02 kB
\href{https://packages.typst.org/preview/note-me-0.3.0.tar.gz}{\pandocbounded{\includesvg[keepaspectratio]{/assets/icons/16-download.svg}}}
\item[Repository:]
\href{https://github.com/FlandiaYingman/note-me}{GitHub}
\end{description}

\subsubsection{Where to report issues?}\label{where-to-report-issues}

This package is a project of Flandia Yingman . Report issues on
\href{https://github.com/FlandiaYingman/note-me}{their repository} . You
can also try to ask for help with this package on the
\href{https://forum.typst.app}{Forum} .

Please report this package to the Typst team using the
\href{https://typst.app/contact}{contact form} if you believe it is a
safety hazard or infringes upon your rights.

\phantomsection\label{versions}
\subsubsection{Version history}\label{version-history}

\begin{longtable}[]{@{}ll@{}}
\toprule\noalign{}
Version & Release Date \\
\midrule\noalign{}
\endhead
\bottomrule\noalign{}
\endlastfoot
0.3.0 & September 30, 2024 \\
\href{https://typst.app/universe/package/note-me/0.2.1/}{0.2.1} & March
8, 2024 \\
\href{https://typst.app/universe/package/note-me/0.1.1/}{0.1.1} &
February 25, 2024 \\
\href{https://typst.app/universe/package/note-me/0.1.0/}{0.1.0} &
February 11, 2024 \\
\end{longtable}

Typst GmbH did not create this package and cannot guarantee correct
functionality of this package or compatibility with any version of the
Typst compiler or app.


\section{Package List LaTeX/tenv.tex}
\title{typst.app/universe/package/tenv}

\phantomsection\label{banner}
\section{tenv}\label{tenv}

{ 0.1.1 }

Parse a .env content.

\phantomsection\label{readme}
Parse a .env content.

\subsection{Usage}\label{usage}

\begin{Shaded}
\begin{Highlighting}[]
\NormalTok{\#import "@preview/tenv.typ:0.1.1": parse\_dotenv}

\NormalTok{\#let env = parse\_dotenv(read(".env"))}
\end{Highlighting}
\end{Shaded}

\subsection{Example}\label{example}

\pandocbounded{\includegraphics[keepaspectratio]{https://github.com/typst/packages/raw/main/packages/preview/tenv/0.1.1/example.png}}

\subsubsection{How to add}\label{how-to-add}

Copy this into your project and use the import as \texttt{\ tenv\ }

\begin{verbatim}
#import "@preview/tenv:0.1.1"
\end{verbatim}

\includesvg[width=0.16667in,height=0.16667in]{/assets/icons/16-copy.svg}

Check the docs for
\href{https://typst.app/docs/reference/scripting/\#packages}{more
information on how to import packages} .

\subsubsection{About}\label{about}

\begin{description}
\tightlist
\item[Author :]
chillcicada
\item[License:]
MIT
\item[Current version:]
0.1.1
\item[Last updated:]
May 16, 2024
\item[First released:]
May 16, 2024
\item[Archive size:]
1.42 kB
\href{https://packages.typst.org/preview/tenv-0.1.1.tar.gz}{\pandocbounded{\includesvg[keepaspectratio]{/assets/icons/16-download.svg}}}
\item[Repository:]
\href{https://github.com/chillcicada/typst-dotenv}{GitHub}
\end{description}

\subsubsection{Where to report issues?}\label{where-to-report-issues}

This package is a project of chillcicada . Report issues on
\href{https://github.com/chillcicada/typst-dotenv}{their repository} .
You can also try to ask for help with this package on the
\href{https://forum.typst.app}{Forum} .

Please report this package to the Typst team using the
\href{https://typst.app/contact}{contact form} if you believe it is a
safety hazard or infringes upon your rights.

\phantomsection\label{versions}
\subsubsection{Version history}\label{version-history}

\begin{longtable}[]{@{}ll@{}}
\toprule\noalign{}
Version & Release Date \\
\midrule\noalign{}
\endhead
\bottomrule\noalign{}
\endlastfoot
0.1.1 & May 16, 2024 \\
\end{longtable}

Typst GmbH did not create this package and cannot guarantee correct
functionality of this package or compatibility with any version of the
Typst compiler or app.


\section{Package List LaTeX/scholarly-tauthesis.tex}
\title{typst.app/universe/package/scholarly-tauthesis}

\phantomsection\label{banner}
\phantomsection\label{template-thumbnail}
\pandocbounded{\includegraphics[keepaspectratio]{https://packages.typst.org/preview/thumbnails/scholarly-tauthesis-0.9.0-small.webp}}

\section{scholarly-tauthesis}\label{scholarly-tauthesis}

{ 0.9.0 }

A template for writing Tampere University theses.

\href{/app?template=scholarly-tauthesis&version=0.9.0}{Create project in
app}

\phantomsection\label{readme}
This is a TAU thesis template written in the
\href{https://github.com/typst/typst}{\texttt{\ typst\ }} typesetting
language, a potential successor to LaTeΧ. The version of typst used to
test this template is
\href{https://github.com/typst/typst/releases/tag/v0.12.0}{\texttt{\ 0.12.0\ }}
.

\subsection{Using the template on
typst.app}\label{using-the-template-on-typst.app}

This template is also available on
\href{https://typst.app/universe}{Typst Universe} as
\href{https://typst.app/universe/package/scholarly-tauthesis}{\texttt{\ scholarly-tauthesis\ }}
. Simply create an account on \url{https://typst.app/} and start a new
\texttt{\ scholarly-tauthesis\ } project by clicking on \textbf{Start
from template} and searching for \textbf{scholarly-tauthesis} .

If you have initialized your project with an older stable version of
this template and wish to upgrade to a newer release, the simplest way
to do it is to change the value of \texttt{\ \$VERSION\ } ≥
\texttt{\ 0.9.0\ } in the import statements

\begin{Shaded}
\begin{Highlighting}[]
\NormalTok{\#import "@preview/scholarly{-}tauthesis:$VERSION" as tauthesis}
\end{Highlighting}
\end{Shaded}

to correspond to a newer released version. Alternatively, you could
download the \texttt{\ tauthesis.typ\ } file from the
\href{https://gitlab.com/tuni-official/thesis-templates/tau-typst-thesis-template}{thesis
template repository} , and upload it into you project on
\url{https://typst.app/} . Then use

\begin{Shaded}
\begin{Highlighting}[]
\NormalTok{\#import "path/to/tauthesis.typ" as tauthesis}
\end{Highlighting}
\end{Shaded}

instead of

\begin{Shaded}
\begin{Highlighting}[]
\NormalTok{\#import "@preview/scholarly{-}tauthesis:$VERSION" as tauthesis}
\end{Highlighting}
\end{Shaded}

to import the \texttt{\ tauthesis\ } module.

\subsubsection{Note}\label{note}

Versions of this template before 0.9.0 do not actually work with
typst.app due to a packaging issue.

\subsection{Local installation}\label{local-installation}

If \href{https://typst.app/universe}{Typst Universe} is online, this
template will be downloaded automatically to

\begin{verbatim}
$CACHEDIR/typst/packages/preview/scholarly-tauthesis/$VERSION/
\end{verbatim}

when one runs the command

\begin{verbatim}
typst init @preview/scholarly-tauthesis:$VERSION mythesis
\end{verbatim}

Here \texttt{\ \$VERSION\ } should be ≥ 0.9.0. The value
\texttt{\ \$CACHEDIR\ } for your OS can be discovered from
\url{https://docs.rs/dirs/latest/dirs/fn.cache_dir.html} .

For a manual installation, download the contents of this repository via
Git or as a ZIP file from the template
\href{https://gitlab.com/tuni-official/thesis-templates/tau-typst-thesis-template/-/tags}{tags}
page. Then, make a symbolic link

\begin{verbatim}
$DATADIR/typst/packages/preview/scholarly-tauthesis/$VERSION/ → /path/to/root/of/tauthesis/
\end{verbatim}

so that a local installation of \texttt{\ typst\ } can discover the
\texttt{\ tauthesis.typ\ } file no matter where you are running it from.
To find out the value \texttt{\ \$DATADIR\ } for your operating system,
see \url{https://docs.rs/dirs/latest/dirs/fn.data_dir.html} . The value
\texttt{\ \$VERSION\ } is the version \texttt{\ A.B.C\ } ≥
\texttt{\ 0.9.0\ } of this template you wish to use.

Once the package has been installed, the command

\begin{verbatim}
typst init @preview/scholarly-tauthesis:$VERSION mythesis
\end{verbatim}

creates a folder \texttt{\ mythesis\ } with the template files in place.
Simply make the \texttt{\ mythesis\ } folder you current working
directory and run

\begin{Shaded}
\begin{Highlighting}[]
\ExtensionTok{typst}\NormalTok{ compile main.typ}
\end{Highlighting}
\end{Shaded}

in the shell of your choice to compile the document from scratch.
Alternatively, type

\begin{Shaded}
\begin{Highlighting}[]
\ExtensionTok{typst}\NormalTok{ watch main.typ }\OperatorTok{\&\textgreater{}}\NormalTok{ typst.log }\KeywordTok{\&}
\end{Highlighting}
\end{Shaded}

to have a \href{https://github.com/typst/typst}{\texttt{\ typst\ }}
process watch the file for changes and compile it when a file is
changed. Possible error messages can then be viewed by checking the
contents of the mentioned file \texttt{\ typst.log\ } .

This template can also be uploaded to the typst online editor at
\url{https://typst.app/} . However, the file paths related to the
\texttt{\ tauthesis\ } file will need to be changed if this is done
manually. See the tutorial at \url{https://typst.app/docs/tutorial/} to
learn the basics of the language. Some examples are also given in the
template itself.

\subsection{Archiving the final version of your
work}\label{archiving-the-final-version-of-your-work}

Before submitting your thesis to the university archives, it needs to be
converted to PDF/A format. Typst versions ≥ 0.12.0 should support the
creation of PDF/A-2b files, when run with the command

\begin{Shaded}
\begin{Highlighting}[]
\ExtensionTok{typst}\NormalTok{ compile }\AttributeTok{{-}{-}pdf{-}standard}\NormalTok{ a{-}2b template/main.typ}
\end{Highlighting}
\end{Shaded}

If a verification program such as
\href{https://docs.verapdf.org/install/}{veraPDF} still complains that
the file \texttt{\ template/main.pdf\ } does not conform to the
standard, the Muuntaja-service of Tampere University should be used to
do the final conversion. See the related instructions (
\href{https://libguides.tuni.fi/opinnaytteet/pdfa}{link} ) for how to do
it. Basically it boils down to feeding your compiled PDF document to the
converter at \href{https://muuntaja.tuni.fi/}{https://muuntaja.tuni.fi}
. \textbf{Remember to check that the output of the converter is not
corrupted, before submitting your thesis to the archives.}

\subsection{Usage}\label{usage}

You can either write your entire \emph{main matter} in the
\href{https://github.com/typst/packages/raw/main/packages/preview/scholarly-tauthesis/0.9.0/template/main.typ}{\texttt{\ main.typ\ }}
file, or more preferrably, split it into multiple chapter-specific files
and place those in the
\href{https://github.com/typst/packages/raw/main/packages/preview/scholarly-tauthesis/0.9.0/template/content}{\texttt{\ contents/\ }}
folder, which this template tries to demonstrate. If you choose to write
your own commands (functions) in the
\href{https://github.com/typst/packages/raw/main/packages/preview/scholarly-tauthesis/0.9.0/template/preamble.typ}{\texttt{\ preamble.typ\ }}
file, this needs to be imported at the start of each chapter you plan to
use the commands in. Sections that come before the main matter, like the
Finnish and English abstracts (
\href{https://github.com/typst/packages/raw/main/packages/preview/scholarly-tauthesis/0.9.0/template/content/tiivistelm\%C3\%A4.typ}{\texttt{\ tiivistelmä.typ\ }}
\textbar{}
\href{https://github.com/typst/packages/raw/main/packages/preview/scholarly-tauthesis/0.9.0/template/content/abstract.typ}{\texttt{\ abstract.typ\ }}
) and
\href{https://github.com/typst/packages/raw/main/packages/preview/scholarly-tauthesis/0.9.0/template/content/preface.typ}{\texttt{\ preface.typ\ }}
must \emph{not} be removed from the
\href{https://github.com/typst/packages/raw/main/packages/preview/scholarly-tauthesis/0.9.0/template/content}{\texttt{\ contents\ }}
folder, as the automation supposes that they are located there.

You should probably \emph{not} modify the file
\href{https://github.com/typst/packages/raw/main/packages/preview/scholarly-tauthesis/0.9.0/tauthesis.typ}{\texttt{\ tauthesis.typ\ }}
, unless there is a bug that needs fixing right now, and not when the
maintainer of this project manages to find the time to do it.

\subsection{Contributing}\label{contributing}

Issues may be created in the issue tracker on the
\href{https://gitlab.com/tuni-official/thesis-templates/tau-typst-thesis-template}{template
GitLab repository} , if one has a GitLab account. Merge requests may
also be performed after GitLab account creation, and forking the
project. See GitLab’s documentation on this to find out how to do it
\href{https://docs.gitlab.com/ee/user/project/repository/forking_workflow.html}{link}
.

\subsection{License}\label{license}

This project itself uses the MIT license. See the
\href{https://github.com/typst/packages/raw/main/packages/preview/scholarly-tauthesis/0.9.0/LICENSE}{LICENSE}
file for details.

\href{/app?template=scholarly-tauthesis&version=0.9.0}{Create project in
app}

\subsubsection{How to use}\label{how-to-use}

Click the button above to create a new project using this template in
the Typst app.

You can also use the Typst CLI to start a new project on your computer
using this command:

\begin{verbatim}
typst init @preview/scholarly-tauthesis:0.9.0
\end{verbatim}

\includesvg[width=0.16667in,height=0.16667in]{/assets/icons/16-copy.svg}

\subsubsection{About}\label{about}

\begin{description}
\tightlist
\item[Author :]
\href{mailto:santtu.soderholm@tuni.fi}{Santtu Söderholm}
\item[License:]
MIT
\item[Current version:]
0.9.0
\item[Last updated:]
November 12, 2024
\item[First released:]
April 9, 2024
\item[Minimum Typst version:]
0.12.0
\item[Archive size:]
36.4 kB
\href{https://packages.typst.org/preview/scholarly-tauthesis-0.9.0.tar.gz}{\pandocbounded{\includesvg[keepaspectratio]{/assets/icons/16-download.svg}}}
\item[Repository:]
\href{https://gitlab.com/tuni-official/thesis-templates/tau-typst-thesis-template}{GitLab}
\item[Discipline :]
\begin{itemize}
\tightlist
\item[]
\item
  \href{https://typst.app/universe/search/?discipline=education}{Education}
\end{itemize}
\item[Categor y :]
\begin{itemize}
\tightlist
\item[]
\item
  \pandocbounded{\includesvg[keepaspectratio]{/assets/icons/16-mortarboard.svg}}
  \href{https://typst.app/universe/search/?category=thesis}{Thesis}
\end{itemize}
\end{description}

\subsubsection{Where to report issues?}\label{where-to-report-issues}

This template is a project of Santtu Söderholm . Report issues on
\href{https://gitlab.com/tuni-official/thesis-templates/tau-typst-thesis-template}{their
repository} . You can also try to ask for help with this template on the
\href{https://forum.typst.app}{Forum} .

Please report this template to the Typst team using the
\href{https://typst.app/contact}{contact form} if you believe it is a
safety hazard or infringes upon your rights.

\phantomsection\label{versions}
\subsubsection{Version history}\label{version-history}

\begin{longtable}[]{@{}ll@{}}
\toprule\noalign{}
Version & Release Date \\
\midrule\noalign{}
\endhead
\bottomrule\noalign{}
\endlastfoot
0.9.0 & November 12, 2024 \\
\href{https://typst.app/universe/package/scholarly-tauthesis/0.8.0/}{0.8.0}
& October 21, 2024 \\
\href{https://typst.app/universe/package/scholarly-tauthesis/0.7.0/}{0.7.0}
& September 17, 2024 \\
\href{https://typst.app/universe/package/scholarly-tauthesis/0.6.2/}{0.6.2}
& April 29, 2024 \\
\href{https://typst.app/universe/package/scholarly-tauthesis/0.5.0/}{0.5.0}
& April 15, 2024 \\
\href{https://typst.app/universe/package/scholarly-tauthesis/0.4.1/}{0.4.1}
& April 13, 2024 \\
\href{https://typst.app/universe/package/scholarly-tauthesis/0.4.0/}{0.4.0}
& April 9, 2024 \\
\end{longtable}

Typst GmbH did not create this template and cannot guarantee correct
functionality of this template or compatibility with any version of the
Typst compiler or app.


\section{Package List LaTeX/uo-pup-thesis-manuscript.tex}
\title{typst.app/universe/package/uo-pup-thesis-manuscript}

\phantomsection\label{banner}
\phantomsection\label{template-thumbnail}
\pandocbounded{\includegraphics[keepaspectratio]{https://packages.typst.org/preview/thumbnails/uo-pup-thesis-manuscript-0.1.0-small.webp}}

\section{uo-pup-thesis-manuscript}\label{uo-pup-thesis-manuscript}

{ 0.1.0 }

Unofficial Typst template for PUP (Polytechnic University of the
Philippines) undergraduate thesis manuscript

\href{/app?template=uo-pup-thesis-manuscript&version=0.1.0}{Create
project in app}

\phantomsection\label{readme}
Unofficial \href{https://typst.app/}{typst} template for undergraduate
thesis manuscript for PUP (Polytechnic University of the Philippines).
This template adheres to the University’s Thesis and Dissertation
Manual as of 2017 (ISBN: 978-971â€``95208-8-7 (Online)). An example
manuscript is also provided (see \texttt{\ ./thesis.pdf\ } ).

\subsection{Setup}\label{setup}

Using
\href{https://github.com/typst/typst?tab=readme-ov-file\#installation}{Typst
CLI} :

\begin{Shaded}
\begin{Highlighting}[]
\ExtensionTok{typst}\NormalTok{ init @preview/uo{-}pup{-}thesis{-}manuscript my{-}thesis}
\BuiltInTok{cd}\NormalTok{ my{-}thesis}
\ExtensionTok{typst}\NormalTok{ compile thesis.typ  }\CommentTok{\# to compile to PDF}
\end{Highlighting}
\end{Shaded}

or run

\begin{Shaded}
\begin{Highlighting}[]
\ExtensionTok{typst}\NormalTok{ watch thesis.typ  }\CommentTok{\# to automatically compiles PDF on save}
\end{Highlighting}
\end{Shaded}

\subsection{Usage}\label{usage}

The template already provided an example structure and some guides. But
to start from nothing, make an entrypoint file with a basic structure
like this:

\begin{Shaded}
\begin{Highlighting}[]
\NormalTok{// thesis.typ}
\NormalTok{\#import "@preview/uo{-}pup{-}thesis{-}manuscript:0.1.0": *}


\NormalTok{\#show: template.with(}
\NormalTok{  [\textless{}your thesis title here\textgreater{}],}
\NormalTok{  ("Author 1", "Author 2", ..., "Author N"),}
\NormalTok{  "name of your college here",}
\NormalTok{  "name of your deg. program here",}
\NormalTok{  "Month YYYY"}
\NormalTok{)}


\NormalTok{// Main content starts here}

\NormalTok{// This provides a customized heading for}
\NormalTok{// chapters that follows the manual}
\NormalTok{\#chapter(1, "Chapter 1 Title") }

\NormalTok{// Since \#chapter() provides a heading level 1,}
\NormalTok{// start each headings under chapters with level 2}
\NormalTok{// to avoid messing up the generated Table of Contents}
\NormalTok{== Introduction}

\NormalTok{...}

\NormalTok{\#chapter(2, "Chapter 2 Title")}

\NormalTok{== Topic A}

\NormalTok{...}

\NormalTok{// End of main content}


\NormalTok{// Bibliography formatting setup}
\NormalTok{\#set par(first{-}line{-}indent: 0pt, hanging{-}indent: 0.5in)}
\NormalTok{\#set page(header: context [\#h(1fr) \#counter(page).get().first()])}
\NormalTok{\#align(center)[ \#heading("REFERENCES") ]}
\NormalTok{\#set par(spacing: 1.5em)}

\NormalTok{// Get the apa.csl file from \textasciigrave{}template/\textasciigrave{} folder}
\NormalTok{\#bibliography(title: none, style: "./apa.csl", "path/to/your/bibtex/file.bib")}


\NormalTok{// Appendices}
\NormalTok{\#show: appendices{-}section}

\NormalTok{\#appendix(1, "Appendix Title")}

\NormalTok{...}

\NormalTok{\#pagebreak()}

\NormalTok{\#appendix(2, "Appendix Title")}

\NormalTok{...}
\end{Highlighting}
\end{Shaded}

There are also provided utilities for some parts that have a specific
way of formatting.

For example, in \texttt{\ Definition\ of\ Terms\ } and
\texttt{\ Significance\ of\ the\ Study\ } sections, use
\texttt{\ \#description\ } function:

\begin{Shaded}
\begin{Highlighting}[]
\NormalTok{== Significance of the Study}
\NormalTok{\#description(}
\NormalTok{  (}
\NormalTok{    (term: [Topic A], desc: [\#lorem(30)]),}
\NormalTok{    (term: [Topic B], desc: [\#lorem(30)]),}
\NormalTok{    (term: [Topic C], desc: [\#lorem(30)]),}
\NormalTok{  )}
\NormalTok{)}

\NormalTok{...}

\NormalTok{== Definition of Terms}
\NormalTok{\#description(}
\NormalTok{  (}
\NormalTok{    (term: [Topic A], desc: [\#lorem(30)]),}
\NormalTok{    (term: [Topic B], desc: [\#lorem(30)]),}
\NormalTok{  )}
\NormalTok{)}
\end{Highlighting}
\end{Shaded}

\begin{center}\rule{0.5\linewidth}{0.5pt}\end{center}

If there’s any mistakes, wrong formatting (e.g., not actually
following the manual), etc., file an issue or a pull request.

\begin{center}\rule{0.5\linewidth}{0.5pt}\end{center}

\subsection{TODO}\label{todo}

\begin{itemize}
\tightlist
\item
  {[} {]} Chapter 4
\item
  {[} {]} Chapter 5
\item
  {[} {]} Abstract
\item
  {[} {]} Acknowledgement
\item
  {[} {]} Copyright
\item
  If possible:

  \begin{itemize}
  \tightlist
  \item
    Approval Sheet
  \item
    Certificate of Originality
  \end{itemize}
\end{itemize}

\href{/app?template=uo-pup-thesis-manuscript&version=0.1.0}{Create
project in app}

\subsubsection{How to use}\label{how-to-use}

Click the button above to create a new project using this template in
the Typst app.

You can also use the Typst CLI to start a new project on your computer
using this command:

\begin{verbatim}
typst init @preview/uo-pup-thesis-manuscript:0.1.0
\end{verbatim}

\includesvg[width=0.16667in,height=0.16667in]{/assets/icons/16-copy.svg}

\subsubsection{About}\label{about}

\begin{description}
\tightlist
\item[Author :]
\href{https://github.com/datsudo}{Datsudo}
\item[License:]
MIT
\item[Current version:]
0.1.0
\item[Last updated:]
November 4, 2024
\item[First released:]
November 4, 2024
\item[Archive size:]
21.6 kB
\href{https://packages.typst.org/preview/uo-pup-thesis-manuscript-0.1.0.tar.gz}{\pandocbounded{\includesvg[keepaspectratio]{/assets/icons/16-download.svg}}}
\item[Repository:]
\href{https://gitlab.com/datsudo/uo-pup-thesis-manuscript}{GitLab}
\item[Categor y :]
\begin{itemize}
\tightlist
\item[]
\item
  \pandocbounded{\includesvg[keepaspectratio]{/assets/icons/16-mortarboard.svg}}
  \href{https://typst.app/universe/search/?category=thesis}{Thesis}
\end{itemize}
\end{description}

\subsubsection{Where to report issues?}\label{where-to-report-issues}

This template is a project of Datsudo . Report issues on
\href{https://gitlab.com/datsudo/uo-pup-thesis-manuscript}{their
repository} . You can also try to ask for help with this template on the
\href{https://forum.typst.app}{Forum} .

Please report this template to the Typst team using the
\href{https://typst.app/contact}{contact form} if you believe it is a
safety hazard or infringes upon your rights.

\phantomsection\label{versions}
\subsubsection{Version history}\label{version-history}

\begin{longtable}[]{@{}ll@{}}
\toprule\noalign{}
Version & Release Date \\
\midrule\noalign{}
\endhead
\bottomrule\noalign{}
\endlastfoot
0.1.0 & November 4, 2024 \\
\end{longtable}

Typst GmbH did not create this template and cannot guarantee correct
functionality of this template or compatibility with any version of the
Typst compiler or app.


\section{Package List LaTeX/acrotastic.tex}
\title{typst.app/universe/package/acrotastic}

\phantomsection\label{banner}
\section{acrotastic}\label{acrotastic}

{ 0.1.1 }

Manage acronyms and their definitions in Typst.

\phantomsection\label{readme}
Manages all your acronyms for you.

Acrotastics main features are clickable abbreviations that auto-expand
on the first occurence, manual short and long forms, implicit or manual
plural form support, and customizable index printing.

\subsection{Quick Start}\label{quick-start}

\begin{verbatim}
#import "@preview/acrotastic:0.1.1": *

#init-acronyms((
  "WTP": ("Wonderful Typst Package","Wonderful Typst Packages"),
))

Acrotastic is a #acr("WTP")! This #acr("WTP") enables easy acronym manipulation.
\end{verbatim}

\subsection{Usage}\label{usage}

\subsubsection{Define acronyms}\label{define-acronyms}

First, define the acronyms in a dictionary, with the keys being the
acronyms and the values being arrays of their definitions. If there is
only a singular version of the definition, the array contains only one
value. If there are both singular and plural versions, define the
definition as an array where the first item is the singular definition
and the second item is the plural. Then, initialize Acrotastic by
passing the dictionay you just defined to the
\texttt{\ \#init-acronyms(...)\ } function.

Here is a example of the \texttt{\ acronyms.typ\ } file:

\begin{verbatim}
#import "@preview/acrotastic:0.1.1": *

#init-acronyms((
  "NN": ("Neural Network"),
  "OS": ("Operating System",),
  "BIOS": ("Basic Input/Output System", "Basic Input/Output Systems"),
))
\end{verbatim}

\subsubsection{Call Acrotastic
functions}\label{call-acrotastic-functions}

There is a large number of different functions to fit every use case.
You will find an overview of all functions and their descriptions in the
table below.

\begin{longtable}[]{@{}ll@{}}
\toprule\noalign{}
Function & Description \\
\midrule\noalign{}
\endhead
\bottomrule\noalign{}
\endlastfoot
\texttt{\ \#acr(...)\ } & On the first occurrence the long version of
the abbreviation and the abbreviation itself are displayed in brackets.
The next time only the abbreviation is displayed. \\
\texttt{\ \#acrpl(...)\ } & Same as \texttt{\ \#acr(...)\ } but the
plural will be diplayed. If no plural is defined, an ‘s’ is added to
the singular form. \\
\texttt{\ \#acrf(...)\ } & The acronym will be displayed as if it is the
first time. This means that it is again shown in the long form and the
abbreviation in brackets. \\
\texttt{\ \#acrfpl(...)\ } & Same as \texttt{\ \#acrf(...)\ } but the
plural will be displayed. If no plural is defined, an ‘s’ is added
to the singular form. \\
\texttt{\ \#acrs(...)\ } & Always displays the short form of the
acronym. \\
\texttt{\ \#acrspl(...)\ } & Same as \texttt{\ \#acrs(...)\ } but adds
an ‘s’ to the acronym for the plural form. \\
\texttt{\ \#acrl(...)\ } & Always displays the long form of the
acronym. \\
\texttt{\ \#acrlpl(...)\ } & Same as \texttt{\ \#acrl(...)\ } but the
plural will be displayed. If no plural is defined, an ‘s’ is added
to the singular form. \\
\texttt{\ \#reset-acronym(...)\ } & Resets a specific acronym. The
acronym will be expanded on the next use. \\
\texttt{\ reset-all-acronyms()\ } & Resets all acronyms. The acronyms
will be expanded on their next use. \\
\end{longtable}

You can alternatively use \texttt{\ \#acr(...)\ } ,
\texttt{\ \#acrf(...)\ } , \texttt{\ \#acrs(...)\ } and
\texttt{\ \#acrl(...)\ } with \texttt{\ plural:\ true\ } to display the
plural form.

\begin{verbatim}
#acr("BIOS", plural: true)
\end{verbatim}

To deactivate the link to the abbreviations directory (for whatever
reason), you can set \texttt{\ link:\ false\ } .

\begin{verbatim}
#acr("BIOS", link: false)
\end{verbatim}

\subsubsection{Print Abbreviations
directory}\label{print-abbreviations-directory}

You can also print an index of all acronyms used in the document with
the \texttt{\ \#print-index()\ } function. There are some parameters for
customization.

\begin{longtable}[]{@{}llll@{}}
\toprule\noalign{}
parameter & values & default & description \\
\midrule\noalign{}
\endhead
\bottomrule\noalign{}
\endlastfoot
title & string & “List of Abbreviations� & Heading of the acronym
index \\
level & number & 1 & Level of the heading \\
sorted & “up�, “down�, “keep� & “up� & “Up� sorts
alphabetically, “Down� sorts reversed alphabetically and “keep�
uses the order from initialization \\
delimiter & string & “:� & String to place after the acronym in the
list \\
acr-col-size & percentage & 20\% & Size of the acronym column in
percent \\
outlined & bool & false & Make the index section outlined \\
\end{longtable}

\subsection{Possible Errors}\label{possible-errors}

\begin{longtable}[]{@{}ll@{}}
\toprule\noalign{}
Error & Solution \\
\midrule\noalign{}
\endhead
\bottomrule\noalign{}
\endlastfoot
Acronym is not a key in the acronyms dictionary. & Make sure that the
acronym is defined in the dictionary passed to
\texttt{\ \#init-acronyms(dict)\ } \\
No definitions found for acronym. Make sure it is defined in the
dictionary passed to \#init-acronyms(dict) & The acronym is in the
dictionary, but has no correct definition. \\
Definitions should be arrays of one or two strings. Definition of
acronym is: & The acronym has a definition, but the definition doesn’t
have the right type. Make sure it’s an array of one or two strings. \\
\end{longtable}

Moreover you have to be careful when using states.

\begin{itemize}
\tightlist
\item
  For every acronym “ABC� that you define, the state named
  “acronym-state-ABC� is initialized and used. To avoid errors, do
  not try to use this state manually for other purposes. Similarly, the
  state named “acronyms� is reserved to Acrotastic. Please avoid
  using it.
\item
  The functions above are leveraging the state \texttt{\ display\ }
  function and only works if the return value is actually printed in the
  document. For more information on states, see the
  \href{https://typst.app/docs/reference/introspection/state/}{Typst
  documentation on states} .
\end{itemize}

\subsection{Contributing}\label{contributing}

If you notice any bug or want to contribute a new feature, please open
an issue or a pull request on the fork
\href{https://github.com/Julian702/typst-packages?tab=readme-ov-file}{Julian702/typst-packages}

\subsection{Acknowledgement}\label{acknowledgement}

Thanks to @Grisely who developed the
\href{https://typst.app/universe/package/acrostiche/}{acrostiche
package} which was the basis for acrotastic.

\subsubsection{How to add}\label{how-to-add}

Copy this into your project and use the import as
\texttt{\ acrotastic\ }

\begin{verbatim}
#import "@preview/acrotastic:0.1.1"
\end{verbatim}

\includesvg[width=0.16667in,height=0.16667in]{/assets/icons/16-copy.svg}

Check the docs for
\href{https://typst.app/docs/reference/scripting/\#packages}{more
information on how to import packages} .

\subsubsection{About}\label{about}

\begin{description}
\tightlist
\item[Author s :]
@Julian702 \& Gaetan Lepage @GaetanLepage
\item[License:]
MIT
\item[Current version:]
0.1.1
\item[Last updated:]
September 3, 2024
\item[First released:]
April 29, 2024
\item[Archive size:]
4.20 kB
\href{https://packages.typst.org/preview/acrotastic-0.1.1.tar.gz}{\pandocbounded{\includesvg[keepaspectratio]{/assets/icons/16-download.svg}}}
\item[Repository:]
\href{https://github.com/Julian702/typst-packages}{GitHub}
\item[Categor y :]
\begin{itemize}
\tightlist
\item[]
\item
  \pandocbounded{\includesvg[keepaspectratio]{/assets/icons/16-list-unordered.svg}}
  \href{https://typst.app/universe/search/?category=model}{Model}
\end{itemize}
\end{description}

\subsubsection{Where to report issues?}\label{where-to-report-issues}

This package is a project of @Julian702 and Gaetan Lepage @GaetanLepage
. Report issues on
\href{https://github.com/Julian702/typst-packages}{their repository} .
You can also try to ask for help with this package on the
\href{https://forum.typst.app}{Forum} .

Please report this package to the Typst team using the
\href{https://typst.app/contact}{contact form} if you believe it is a
safety hazard or infringes upon your rights.

\phantomsection\label{versions}
\subsubsection{Version history}\label{version-history}

\begin{longtable}[]{@{}ll@{}}
\toprule\noalign{}
Version & Release Date \\
\midrule\noalign{}
\endhead
\bottomrule\noalign{}
\endlastfoot
0.1.1 & September 3, 2024 \\
\href{https://typst.app/universe/package/acrotastic/0.1.0/}{0.1.0} &
April 29, 2024 \\
\end{longtable}

Typst GmbH did not create this package and cannot guarantee correct
functionality of this package or compatibility with any version of the
Typst compiler or app.


\section{Package List LaTeX/universal-hit-thesis.tex}
\title{typst.app/universe/package/universal-hit-thesis}

\phantomsection\label{banner}
\phantomsection\label{template-thumbnail}
\pandocbounded{\includegraphics[keepaspectratio]{https://packages.typst.org/preview/thumbnails/universal-hit-thesis-0.2.1-small.webp}}

\section{universal-hit-thesis}\label{universal-hit-thesis}

{ 0.2.1 }

å``ˆå°''滨工业大学学ä½?论æ--‡æ¨¡æ?¿ \textbar{} Universal Harbin
Institute of Technology Thesis

\href{/app?template=universal-hit-thesis&version=0.2.1}{Create project
in app}

\phantomsection\label{readme}
适ç''¨äºŽå``ˆå°''滨工业大学学ä½?论æ--‡çš„ Typst 模æ?¿

\pandocbounded{\includegraphics[keepaspectratio]{https://vonbrank-images.oss-cn-hangzhou.aliyuncs.com/20240426-HIT-Thesis-Typst/hit-thesis-typst-development-cover-01.jpg}}

\begin{quote}
{[}!WARNING{]}
本模æ?¿æ­£å¤„于积æž?å¼€å?{}`阶段,存在一些æ~¼å¼?é---®é¢˜ï¼Œé€‚å?ˆå°?鲜
Typst 特性

本模æ?¿æ˜¯æ°`é---´æ¨¡æ?¿ï¼Œ \textbf{å?¯èƒ½ä¸?被学æ~¡è®¤å?¯}
,正å¼?使ç''¨è¿‡ç¨‹ä¸­è¯·å?šå¥½éš?æ---¶å°†å†\ldots 容è¿?移至 Word
æˆ-- LaTeX 的准备
\end{quote}

\subsection{å\ldots³äºŽæœ¬é¡¹ç›®}\label{uxe5uxb3uxe4uxbaux17euxe6ux153uxe9uxb9uxe7}

\href{https://typst.app/}{Typst} 是使ç''¨ Rust
语言开å?{}`çš„å\ldots¨æ--°æ--‡æ¡£æŽ'版系统,有望以 Markdown
级别的简æ´?语法å'Œç¼--è¯`速度实现 LaTeX
级别的æŽ'版能力,å?³é€šè¿‡ç¼--写é?µå¾ª Typst
语法规则的æ--‡æœ¬æ--‡æ¡£ã€?执行ç¼--è¯`å`½ä»¤ï¼Œæ?¥å?¯ç''Ÿæˆ?ç›®æ~‡æ~¼å¼?çš„
PDF æ--‡æ¡£ã€‚

\textbf{HIT Thesis Typst}
是一å¥---简å?•æ˜``ç''¨çš„å``ˆå°''滨工业大学学ä½?论æ--‡ Typst
模æ?¿ï¼Œå?--- \href{https://github.com/hithesis/hithesis}{hithesis}
å?¯å?{}`,计åˆ'囊括一æ~¡ä¸‰åŒºæœ¬ç§`ã€?硕士ã€?å?šå£«çš„å­¦ä½?论æ--‡æ~¼å¼?。

\textbf{预览效果}

\begin{itemize}
\tightlist
\item
  本ç§`通ç''¨ï¼š
  \href{https://github.com/chosertech/HIT-Thesis-Typst/blob/build/universal-bachelor.pdf}{universal-bachelor.pdf}
\end{itemize}

\subsection{使ç''¨}\label{uxe4uxbduxe7}

\subsubsection{本地ç¼--è¾` â\ldots~
(推è??)}\label{uxe6ux153uxe5ux153uxe7uxbcuxe8uxbe-uxe2-uxefuxbcux2c6uxe6ux17euxe8uxefuxbc}

è¿™ç§?æ--¹å¼?适å?ˆå¤§å¤šæ•°ç''¨æˆ·ã€‚

é¦--å\ldots ˆå®‰è£\ldots{} Typst,您å?¯ä»¥åœ¨ Typst Github ä»``åº``çš„
\href{https://github.com/typst/typst/releases/}{Release 页�}
下载最æ--°çš„安è£\ldots åŒ\ldots 直接安è£\ldots ,并将
\texttt{\ typst\ } å?¯æ‰§è¡Œç¨‹åº?æ·»åŠ~到 \texttt{\ PATH\ }
环境å?˜é‡?;如果您使ç''¨ Scoop
åŒ\ldots 管ç?†å™¨ï¼Œåˆ™å?¯ä»¥ç›´æŽ¥é€šè¿‡
\texttt{\ scoop\ install\ typst\ } å`½ä»¤å®‰è£\ldots 。

安è£\ldots 好 Typst
之å?Žï¼Œæ‚¨å?ªéœ€è¦?选择一个您å--œæ¬¢çš„目录,并在此目录下执行以下å`½ä»¤ï¼š

\begin{Shaded}
\begin{Highlighting}[]
\ExtensionTok{typst}\NormalTok{ init @preview/universal{-}hit{-}thesis:0.2.1}
\end{Highlighting}
\end{Shaded}

Typst 将会创建一个å??为 \texttt{\ universal-hit-thesis\ }
çš„æ--‡ä»¶å¤¹ï¼Œè¿›å\ldots¥è¯¥ç›®å½•å?Žï¼Œæ‚¨å?¯ä»¥ç›´æŽ¥ä¿®æ''¹ç›®å½•ä¸‹çš„
\texttt{\ universal-bachelor.typ\ }
,然å?Žæ‰§è¡Œä»¥ä¸‹å`½ä»¤è¿›è¡Œç¼--è¯`ç''Ÿæˆ? \texttt{\ .pdf\ }
æ--‡æ¡£ï¼š

\begin{Shaded}
\begin{Highlighting}[]
\ExtensionTok{typst}\NormalTok{ compile universal{-}bachelor.typ}
\end{Highlighting}
\end{Shaded}

æˆ--è€\ldots 使ç''¨ä»¥ä¸‹å`½ä»¤è¿›è¡Œå®žæ---¶é¢„览:

\begin{Shaded}
\begin{Highlighting}[]
\ExtensionTok{typst}\NormalTok{ watch universal{-}bachelor.typ}
\end{Highlighting}
\end{Shaded}

å½``您è¦?实æ---¶é¢„览æ---¶ï¼Œæˆ`们更推è??使ç''¨ Visual Studio
Code 进行ç¼--è¾`,é\ldots?å?ˆ
\href{https://marketplace.visualstudio.com/items?itemName=nvarner.typst-lsp}{Tinymist
Typst} ,
\href{https://marketplace.visualstudio.com/items?itemName=mgt19937.typst-preview}{Typst
Preview} ç­‰æ?'件å?¯ä»¥å¤§å¹\ldots æ??å?‡æ‚¨çš„ç¼--è¾`ä½``验。

\subsubsection{本地ç¼--è¾`
â\ldots¡}\label{uxe6ux153uxe5ux153uxe7uxbcuxe8uxbe-uxe2}

è¿™ç§?æ--¹æ³•é€‚å?ˆ Typst å¼€å?{}`è€\ldots 。

é¦--å\ldots ˆä½¿ç''¨ \texttt{\ git\ clone\ } å`½ä»¤ clone
本项目,æˆ--è€\ldots 直接在 Release
页é?¢ä¸‹è½½ç‰¹å®šç‰ˆæœ¬çš„æº?ç~?。在 \texttt{\ templates/\ }
目录下选择您需è¦?的模æ?¿ï¼Œç›´æŽ¥ä¿®æ''¹æˆ--å¤?制一份,在æ~¹ç›®å½•è¿?行以下å`½ä»¤è¿›è¡Œç¼--è¯`:

\begin{Shaded}
\begin{Highlighting}[]
\ExtensionTok{typst}\NormalTok{ compile ./templates/.typ }\AttributeTok{{-}{-}root}\NormalTok{ ./}
\end{Highlighting}
\end{Shaded}

æˆ--è€\ldots 使ç''¨å¦‚下å`½ä»¤è¿›è¡Œå®žæ---¶é¢„览:

\begin{Shaded}
\begin{Highlighting}[]
\ExtensionTok{typst}\NormalTok{ watch ./templates/.typ }\AttributeTok{{-}{-}root}\NormalTok{ ./}
\end{Highlighting}
\end{Shaded}

\begin{quote}
{[}!TIP{]}
本模æ?¿æ­£å¤„于积æž?å¼€å?{}`阶段,更æ--°è¾ƒä¸ºé¢`ç¹?,虽然已ç»?上ä¼~至
Typst Universe,但是您�然�以借助 Typst local packages
�实现在 Typst Universe
å?Œæ­¥æœ¬æ¨¡æ?¿çš„最æ--°ç‰ˆæœ¬å‰?,在本地ä½``验本模æ?¿çš„最æ--°ç‰ˆæœ¬ï¼Œå\ldots·ä½``å?šæ³•ä¸ºï¼š

\begin{itemize}
\item
  在 Release
  页é?¢ä¸‹è½½å¯¹åº''版本的æº?ç~?压缩åŒ\ldots ,并将å\ldots¶è§£åŽ‹åˆ°
  \texttt{\ \{data-dir\}/typst/packages/local/universal-hit-thesis/\{version\}\ }
  , \texttt{\ \{data-dir\}\ } 在ä¸?å?Œæ``?作系统下的值为:

  \begin{itemize}
  \tightlist
  \item
    \texttt{\ \$XDG\_DATA\_HOME\ } or
    \texttt{\ \textasciitilde{}/.local/share\ } on Linux
  \item
    \texttt{\ \textasciitilde{}/Library/Application\ } Support on macOS
  \item
    \texttt{\ \%LOCALAPPDATA\%\ } on Windows
  \end{itemize}

  \texttt{\ \{version\}\ } 的值为 \texttt{\ typst.toml\ } 中
  \texttt{\ version\ } 项的值.

  解压完æˆ?å?Ž \texttt{\ typst.toml\ } æ--‡ä»¶åº''该出现在
  \texttt{\ \{data-dir\}/typst/packages/local/universal-hit-thesis/\{version\}\ }
  目录下.
\item
  接ç?€æ‚¨éœ€è¦?在您的论æ--‡ä¸­å°†
  \texttt{\ \#import\ "@preview/universal-hit-thesis:0.2.1"\ }
  ä¿®æ''¹ä¸º
  \texttt{\ \#import\ "@local/universal-hit-thesis:\{version\}"\ }
  ,å?³å?¯æ›´æ--°æ¨¡æ?¿.
\end{itemize}
\end{quote}

\subsubsection{在线ç¼--è¾`}\label{uxe5ux153uxe7uxbauxe7uxbcuxe8uxbe}

本模æ?¿å·²ä¸Šä¼~ Typst Universe,您å?¯ä»¥ä½¿ç''¨ Typst 的官æ--¹
Web App 进行ç¼--è¾`。

å\ldots·ä½``æ?¥è¯´ï¼Œåœ¨ Typst Web App 登录å?Žï¼Œç‚¹å‡»
\texttt{\ Start\ from\ template\ } ,在弹出的çª---å?£ä¸­é€‰æ‹©
\texttt{\ universal-hit-thesis\ } ,��从模�创建项目。

\pandocbounded{\includegraphics[keepaspectratio]{https://vonbrank-images.oss-cn-hangzhou.aliyuncs.com/20240426-HIT-Thesis-Typst/hit-thesis-web-app-create.jpg}}

\pandocbounded{\includegraphics[keepaspectratio]{https://vonbrank-images.oss-cn-hangzhou.aliyuncs.com/20240426-HIT-Thesis-Typst/hit-thesis-web-app-demo.jpg}}

\begin{quote}
{[}!NOTE{]}

Typst Web App
çš„æŽ'版渲æŸ``在æµ?览器本地执行,所以实æ---¶é¢„览ä½``验å‡~乎与在本地ç¼--è¾`æ---~异。

默认æƒ\ldots 况下,å½``您在 Web App
使ç''¨æ¨¡æ?¿åˆ›å»ºè®ºæ--‡é¡¹ç›®å?Žï¼Œå?¯èƒ½åœ¨é¡¹ç›®ä¸­çœ‹åˆ°å¤§é‡?é'ˆå¯¹ä¸­æ--‡æ--‡æœ¬çš„拼写é''™è¯¯è­¦å`Šï¼Œæ‚¨å?¯ä»¥é€šè¿‡åœ¨
\texttt{\ \#cover()\ } 函数调ç''¨ç‚¹å‰?æ?'å\ldots¥è¯­å?¥
\texttt{\ \#set\ text(lang:\ "zh")\ }
æ?¥æ¶ˆé™¤è¿™äº›è­¦å`Šï¼Œè¯¥é---®é¢˜å°†åœ¨æœªæ?¥çš„版本中å¾---到修å¤?.

æ­¤å¤--您å?¯èƒ½å·²ç»?注æ„?到,Web App
中的模æ?¿å­---ä½``显示与预期存在差è·?,这是å›~为 Web App
默认ä¸?æ??ä¾› \texttt{\ SimSun\ } , \texttt{\ Times\ New\ Roman\ }
等中æ--‡æŽ'版常ç''¨å­---ä½``。为了解决这个é---®é¢˜ï¼Œæ‚¨å?¯ä»¥åœ¨æ?œç´¢å¼•æ``Žæ?œç´¢ä»¥ä¸‹å­---ä½``æ--‡ä»¶ï¼š

\begin{itemize}
\tightlist
\item
  \texttt{\ TimesNewRoman.ttf\ } (åŒ\ldots 括 \texttt{\ Bold\ } ,
  \texttt{\ Italic\ } \texttt{\ Bold-Italic\ } 等版本)
\item
  \texttt{\ SimSun.ttf\ }
\item
  \texttt{\ SimHei.ttf\ }
\item
  \texttt{\ Kaiti.ttf\ }
\item
  \texttt{\ Consolas.ttf\ }
\item
  \texttt{\ Courier\ New.ttf\ }
\end{itemize}

并将这些æ--‡ä»¶æ‰‹åŠ¨ä¸Šä¼~至 Web App
项目æ~¹ç›®å½•ä¸­ï¼Œæˆ--为了目录整æ´?,å?¯ä»¥åˆ›å»ºä¸€ä¸ª
\texttt{\ fonts\ } æ--‡ä»¶å¤¹å¹¶å°†å­---ä½``置于å\ldots¶ä¸­ï¼ŒTypst Web
App 将自动åŠ~载这些å­---ä½``,并正确渲æŸ``到预览çª---å?£ä¸­.

ç''±äºŽæ¯?次在 Typst Web App
中æ‰``开项目æ---¶éƒ½éœ€è¦?é‡?æ--°ä¸‹è½½å­---ä½``,而中æ--‡å­---ä½``ä½``积普é??较大,åŠ~è½½æ---¶é---´è¾ƒé•¿ï¼Œå›~æ­¤æˆ`们更推è??
\textbf{本地ç¼--è¾`} 。
\end{quote}

\begin{center}\rule{0.5\linewidth}{0.5pt}\end{center}

\begin{quote}
{[}!NOTE{]} 注æ„?到,官æ--¹æ??供的本ç§`毕业设计 Microsoft
Word 论æ--‡æ¨¡æ?¿
\texttt{\ 本科毕业论文(设计)书写范例(�工类).doc\ }
在一æ~¡ä¸‰åŒºæ˜¯é€šç''¨çš„,æ„?å`³ç?€æœ¬ Typst
模æ?¿çš„本ç§`论æ--‡éƒ¨åˆ†ç?†è®ºä¸Šä¹Ÿæ˜¯åœ¨ä¸€æ~¡ä¸‰åŒºé€šç''¨çš„,å›~æ­¤æˆ`们æ??供适ç''¨äºŽå?„æ~¡åŒºçš„本ç§`毕业论æ--‡æ¨¡æ?¿æ¨¡å?---导出,å?³ä»¥ä¸‹å››ç§?导å\ldots¥æ¨¡å?---çš„æ--¹å¼?效果相å?Œï¼š

\begin{Shaded}
\begin{Highlighting}[]
\NormalTok{\#import "@preview/universal{-}hit{-}thesis:0.2.1": harbin{-}bachelor}
\NormalTok{\#import harbin{-}bachelor: * // 哈尔滨校区本科}
\end{Highlighting}
\end{Shaded}

\begin{Shaded}
\begin{Highlighting}[]
\NormalTok{\#import "@preview/universal{-}hit{-}thesis:0.2.1": weihai{-}bachelor}
\NormalTok{\#import weihai{-}bachelor: * // 威海校区本科}
\end{Highlighting}
\end{Shaded}

\begin{Shaded}
\begin{Highlighting}[]
\NormalTok{\#import "@preview/universal{-}hit{-}thesis:0.2.1": shenzhen{-}bachelor}
\NormalTok{\#import shenzhen{-}bachelor: * // 深圳校区本科}
\end{Highlighting}
\end{Shaded}

\begin{Shaded}
\begin{Highlighting}[]
\NormalTok{\#import "@preview/universal{-}hit{-}thesis:0.2.1": universal{-}bachelor}
\NormalTok{\#import universal{-}bachelor: * // 一校三区本科通用}
\end{Highlighting}
\end{Shaded}
\end{quote}

\subsection{ä¾?èµ--}\label{uxe4uxbeuxe8uxb5}

\subsubsection{å?¯é€‰ä¾?èµ--}\label{uxe5uxe9uxe4uxbeuxe8uxb5}

è‹¥è¦?书写å'Œå¼•ç''¨ä¼ªä»£ç~?,您å?¯ä»¥ä½¿ç''¨
\texttt{\ algorithm-figure\ } ,为此,您需è¦?导å\ldots¥
\texttt{\ algorithmic\ } æˆ-- \texttt{\ lovelace\ } åŒ\ldots 。

\begin{Shaded}
\begin{Highlighting}[]
\NormalTok{\#import "@preview/algorithmic:0.1.0"}
\NormalTok{\#import algorithmic: algorithm}

\NormalTok{\#import "@preview/lovelace:0.2.0": *}
\end{Highlighting}
\end{Shaded}

使ç''¨æ--¹å¼?详è§?
\href{https://github.com/chosertech/HIT-Thesis-Typst/blob/main/templates/universal-bachelor.typ}{模�}
中的 \texttt{\ 伪代ç~?\ } 节

\subsection{已知é---®é¢˜}\label{uxe5uxb2uxe7uxffuxe9uxe9}

\subsubsection{æŽ'版}\label{uxe6ux17euxe7ux2c6}

尽管本 Typst 模æ?¿å?„部分å­---ä½``ã€?å­---å?·ç­‰è®¾ç½®å?‡ä¸ŽåŽŸ Word
模æ?¿ä¸€è‡´ï¼Œä½†æ®µè?½æŽ'版视觉上ä»?与 Word
模æ?¿æœ‰ä¸€äº›å·®åˆ«ï¼Œè¿™ä¸Žå­---符é---´è·?ã€?è¡Œè·?ã€?段è?½é---´è·?有一定肉眼æŽ'版æˆ?分有å\ldots³.

\subsubsection{å?‚考æ--‡çŒ®}\label{uxe5uxe8ux192uxe6uxe7ux153}

\begin{itemize}
\tightlist
\item
  å­¦æ~¡å¯¹å?‚考æ--‡çŒ®æ~¼å¼?çš„è¦?求与æ~‡å‡†çš„
  \texttt{\ GB/T\ 7714-2015\ numeric\ }
  æ~¼å¼?存在差异,æˆ`们已修æ''¹ç›¸å\ldots³ CSL æ--‡ä»¶å¹¶å½¢æˆ?
  \texttt{\ gb-t-7714-2015-numeric-hit.csl\ }
  以修å¤?作è€\ldots å??å­---大å°?写等é---®é¢˜ï¼Œä½†ä»?有以下已知特性尚未æ''¯æŒ?:

  \begin{itemize}
  \tightlist
  \item
    ä»\ldots 纯ç''µå­?资æº?(如ç½`页ã€?软件)显示引ç''¨æ---¥æœŸå'Œ
    URL
  \item
    æ---~ DOI
  \end{itemize}
\item
  引ç''¨å\ldots¶ä»--å­¦æ~¡çš„å­¦ä½?论æ--‡æ---¶å?‚考æ--‡çŒ®é¡µå¯¹åº''æ?¡ç›®å­˜åœ¨æ~¼å¼?é---®é¢˜ï¼Œå›~为
  Typst å°šä¸?æ''¯æŒ? CSL æ--‡ä»¶ä¸­çš„ \texttt{\ school\ } ç­‰å­---段.
\item
  目�版本的 Typst 对 CSL
  æ''¯æŒ?程度æˆ?谜,更多é---®é¢˜å?¯å?‚考
  \href{https://github.com/csimide/SEU-Typst-Template/?tab=readme-ov-file\#\%E5\%8F\%82\%E8\%80\%83\%E6\%96\%87\%E7\%8C\%AE}{SEU-Typst-Template
  å?‚考æ--‡çŒ®å·²çŸ¥é---®é¢˜} .
\end{itemize}

\subsection{致谢}\label{uxe8uxe8}

\begin{itemize}
\tightlist
\item
  æ„Ÿè°¢
  \href{https://github.com/werifu/HUST-typst-template}{HUST-typst-template}
  为本模æ?¿æ---©æœŸç‰ˆæœ¬çš„框架æ??ä¾›æ€?è·¯.
\item
  æ„Ÿè°¢ \href{https://gist.github.com/csimide}{@csimide} å'Œ
  \href{https://github.com/OrangeX4}{@OrangeX4}
  æ??供的中英å?Œè¯­å?‚考æ--‡çŒ®å®žçŽ°.
\item
  æ„Ÿè°¢
  \href{https://github.com/nju-lug/modern-nju-thesis}{modern-nju-thesis}
  为本模æ?¿çš„一些特性æ??供实现æ€?è·¯.
\end{itemize}

\href{/app?template=universal-hit-thesis&version=0.2.1}{Create project
in app}

\subsubsection{How to use}\label{how-to-use}

Click the button above to create a new project using this template in
the Typst app.

You can also use the Typst CLI to start a new project on your computer
using this command:

\begin{verbatim}
typst init @preview/universal-hit-thesis:0.2.1
\end{verbatim}

\includesvg[width=0.16667in,height=0.16667in]{/assets/icons/16-copy.svg}

\subsubsection{About}\label{about}

\begin{description}
\tightlist
\item[Author :]
CHOSER Tech
\item[License:]
MIT
\item[Current version:]
0.2.1
\item[Last updated:]
June 19, 2024
\item[First released:]
May 17, 2024
\item[Archive size:]
23.7 kB
\href{https://packages.typst.org/preview/universal-hit-thesis-0.2.1.tar.gz}{\pandocbounded{\includesvg[keepaspectratio]{/assets/icons/16-download.svg}}}
\item[Repository:]
\href{https://github.com/chosertech/HIT-Thesis-Typst}{GitHub}
\item[Categor y :]
\begin{itemize}
\tightlist
\item[]
\item
  \pandocbounded{\includesvg[keepaspectratio]{/assets/icons/16-mortarboard.svg}}
  \href{https://typst.app/universe/search/?category=thesis}{Thesis}
\end{itemize}
\end{description}

\subsubsection{Where to report issues?}\label{where-to-report-issues}

This template is a project of CHOSER Tech . Report issues on
\href{https://github.com/chosertech/HIT-Thesis-Typst}{their repository}
. You can also try to ask for help with this template on the
\href{https://forum.typst.app}{Forum} .

Please report this template to the Typst team using the
\href{https://typst.app/contact}{contact form} if you believe it is a
safety hazard or infringes upon your rights.

\phantomsection\label{versions}
\subsubsection{Version history}\label{version-history}

\begin{longtable}[]{@{}ll@{}}
\toprule\noalign{}
Version & Release Date \\
\midrule\noalign{}
\endhead
\bottomrule\noalign{}
\endlastfoot
0.2.1 & June 19, 2024 \\
\href{https://typst.app/universe/package/universal-hit-thesis/0.2.0/}{0.2.0}
& May 17, 2024 \\
\end{longtable}

Typst GmbH did not create this template and cannot guarantee correct
functionality of this template or compatibility with any version of the
Typst compiler or app.


\section{Package List LaTeX/jogs.tex}
\title{typst.app/universe/package/jogs}

\phantomsection\label{banner}
\section{jogs}\label{jogs}

{ 0.2.3 }

QuickJS JavaScript runtime for Typst

\phantomsection\label{readme}
Quickjs javascript runtime for typst. This package provides a typst
plugin for evaluating javascript code.

\subsection{Example}\label{example}

\begin{Shaded}
\begin{Highlighting}[]
\NormalTok{\#import "@preview/jogs:0.2.3": *}

\NormalTok{\#set page(height: auto, width: auto, fill: black, margin: 1em)}
\NormalTok{\#set text(fill: white)}

\NormalTok{\#let code = \textasciigrave{}\textasciigrave{}\textasciigrave{}}
\NormalTok{function sum() \{}
\NormalTok{  const total = Array.prototype.slice.call(arguments).reduce(function(a, b) \{}
\NormalTok{      return a + b;}
\NormalTok{  \}, 0);}
\NormalTok{  return total;}
\NormalTok{\}}

\NormalTok{function string\_join(arr) \{}
\NormalTok{  return arr.join(" | ");}
\NormalTok{\}}

\NormalTok{function return\_complex\_obj() \{}
\NormalTok{  return \{}
\NormalTok{    a: 1,}
\NormalTok{    b: "2",}
\NormalTok{    c: [1, 2, 3],}
\NormalTok{    d: \{}
\NormalTok{      e: 1,}
\NormalTok{    \},}
\NormalTok{  \};}
\NormalTok{\}}

\NormalTok{\textasciigrave{}\textasciigrave{}\textasciigrave{}}
\NormalTok{\#let bytecode = compile{-}js(code)}

\NormalTok{\#list{-}global{-}property(bytecode)}

\NormalTok{\#call{-}js{-}function(bytecode, "sum", 6, 7, 8, 9, 10)}

\NormalTok{\#call{-}js{-}function(bytecode, "string\_join", ("a", "b", "c", "d", "e"),)}

\NormalTok{\#call{-}js{-}function(bytecode, "return\_complex\_obj",)}

\end{Highlighting}
\end{Shaded}

result:

\pandocbounded{\includesvg[keepaspectratio]{https://github.com/typst/packages/raw/main/packages/preview/jogs/0.2.3/typst-package/examples/fib.svg}}

\subsection{Documentation}\label{documentation}

This package provide following function(s):

\subsubsection{\texorpdfstring{\texttt{\ eval-js\ }}{ eval-js }}\label{eval-js}

Run a Javascript code snippet.

\paragraph{Arguments}\label{arguments}

\begin{itemize}
\tightlist
\item
  \texttt{\ code\ } - The Javascript code to run. It can be a string or
  a raw block.
\end{itemize}

\paragraph{Returns}\label{returns}

The result of the Javascript code. The type is the typst type which most
closely resembles the Javascript type.

\paragraph{Example}\label{example-1}

\begin{Shaded}
\begin{Highlighting}[]
\NormalTok{\#let result = eval{-}js("1 + 1")}
\end{Highlighting}
\end{Shaded}

\subsubsection{\texorpdfstring{\texttt{\ compile-js\ }}{ compile-js }}\label{compile-js}

Compile a Javascript code snippet into bytecode.

\paragraph{Arguments}\label{arguments-1}

\begin{itemize}
\tightlist
\item
  \texttt{\ code\ } - The Javascript code to compile. It can be a string
  or a raw block.
\end{itemize}

\paragraph{Returns}\label{returns-1}

The bytecode of the Javascript code. The type is \texttt{\ bytes\ } .

\paragraph{Example}\label{example-2}

\begin{Shaded}
\begin{Highlighting}[]
\NormalTok{\#let bytecode = compile{-}js("function sum(a, b) \{ return a + b; \}")}
\end{Highlighting}
\end{Shaded}

\subsubsection{\texorpdfstring{\texttt{\ call-js-function\ }}{ call-js-function }}\label{call-js-function}

Call a Javascript function with arguments.

\paragraph{Arguments}\label{arguments-2}

\begin{itemize}
\tightlist
\item
  \texttt{\ bytecode\ } - The bytecode of the Javascript function. It
  can be obtained by calling \texttt{\ compile-js\ } .
\item
  \texttt{\ name\ } - The name of the Javascript function.
\item
  \texttt{\ ..args\ } - The arguments to pass to the Javascript
  function. All other positional arguments are passed to the Javascript
  function as is.
\end{itemize}

\paragraph{Returns}\label{returns-2}

The result of the Javascript function. The type is the typst type which
most closely resembles the Javascript type.

\paragraph{Example}\label{example-3}

\begin{Shaded}
\begin{Highlighting}[]
\NormalTok{\#let bytecode = compile{-}js("function sum(a, b) \{ return a + b; \}")}
\NormalTok{\#let result = call{-}js{-}function(bytecode, "sum", 1, 2)}
\end{Highlighting}
\end{Shaded}

\subsubsection{\texorpdfstring{\texttt{\ list-global-property\ }}{ list-global-property }}\label{list-global-property}

List all global properties of a compiled Javascript bytecode. This is
useful for inspecting the compiled Javascript bytecode.

\paragraph{Arguments}\label{arguments-3}

\begin{itemize}
\tightlist
\item
  \texttt{\ bytecode\ } - The bytecode of the Javascript function. It
  can be obtained by calling \texttt{\ compile-js\ } .
\end{itemize}

\paragraph{Returns}\label{returns-3}

A list of all global properties of the Javascript bytecode. The type is
\texttt{\ array\ } .

\paragraph{Example}\label{example-4}

\begin{Shaded}
\begin{Highlighting}[]
\NormalTok{\#let bytecode = compile{-}js("function sum(a, b) \{ return a + b; \}")}
\NormalTok{\#let result = list{-}global{-}property(bytecode)}
\end{Highlighting}
\end{Shaded}

\subsubsection{How to add}\label{how-to-add}

Copy this into your project and use the import as \texttt{\ jogs\ }

\begin{verbatim}
#import "@preview/jogs:0.2.3"
\end{verbatim}

\includesvg[width=0.16667in,height=0.16667in]{/assets/icons/16-copy.svg}

Check the docs for
\href{https://typst.app/docs/reference/scripting/\#packages}{more
information on how to import packages} .

\subsubsection{About}\label{about}

\begin{description}
\tightlist
\item[Author :]
Wenzhuo Liu
\item[License:]
MIT
\item[Current version:]
0.2.3
\item[Last updated:]
February 1, 2024
\item[First released:]
November 6, 2023
\item[Archive size:]
354 kB
\href{https://packages.typst.org/preview/jogs-0.2.3.tar.gz}{\pandocbounded{\includesvg[keepaspectratio]{/assets/icons/16-download.svg}}}
\item[Repository:]
\href{https://github.com/Enter-tainer/jogs}{GitHub}
\end{description}

\subsubsection{Where to report issues?}\label{where-to-report-issues}

This package is a project of Wenzhuo Liu . Report issues on
\href{https://github.com/Enter-tainer/jogs}{their repository} . You can
also try to ask for help with this package on the
\href{https://forum.typst.app}{Forum} .

Please report this package to the Typst team using the
\href{https://typst.app/contact}{contact form} if you believe it is a
safety hazard or infringes upon your rights.

\phantomsection\label{versions}
\subsubsection{Version history}\label{version-history}

\begin{longtable}[]{@{}ll@{}}
\toprule\noalign{}
Version & Release Date \\
\midrule\noalign{}
\endhead
\bottomrule\noalign{}
\endlastfoot
0.2.3 & February 1, 2024 \\
\href{https://typst.app/universe/package/jogs/0.2.2/}{0.2.2} & January
15, 2024 \\
\href{https://typst.app/universe/package/jogs/0.2.1/}{0.2.1} & November
29, 2023 \\
\href{https://typst.app/universe/package/jogs/0.2.0/}{0.2.0} & November
7, 2023 \\
\href{https://typst.app/universe/package/jogs/0.1.0/}{0.1.0} & November
6, 2023 \\
\end{longtable}

Typst GmbH did not create this package and cannot guarantee correct
functionality of this package or compatibility with any version of the
Typst compiler or app.


\section{Package List LaTeX/shiroa.tex}
\title{typst.app/universe/package/shiroa}

\phantomsection\label{banner}
\section{shiroa}\label{shiroa}

{ 0.1.2 }

A simple tool for creating modern online books in pure typst.

\phantomsection\label{readme}
\href{https://github.com/Myriad-Dreamin/shiroa}{\emph{shiroa}} (
\emph{Shiro A} , or \emph{The White} , or \emph{äº`笺} ) is a simple
tool for creating modern online (cloud) books in pure typst.

\subsection{Installation (shiroa CLI)}\label{installation-shiroa-cli}

There are multiple ways to install the
\href{https://github.com/Myriad-Dreamin/shiroa}{shiroa} CLI tool. Choose
any one of the methods below that best suit your needs.

\subsubsection{Pre-compiled binaries}\label{pre-compiled-binaries}

Executable binaries are available for download on the
\href{https://github.com/Myriad-Dreamin/shiroa/releases}{GitHub Releases
page} . Download the binary for your platform (Windows, macOS, or Linux)
and extract the archive. The archive contains an \texttt{\ shiroa\ }
executable which you can run to build your books.

To make it easier to run, put the path to the binary into your
\texttt{\ PATH\ } .

\subsubsection{Build from source using
Rust}\label{build-from-source-using-rust}

To build the \texttt{\ shiroa\ } executable from source, you will first
need to install Yarn, Rust, and Cargo. Follow the instructions on the
\href{https://classic.yarnpkg.com/en/docs/install}{Yarn installation
page} and \href{https://www.rust-lang.org/tools/install}{Rust
installation page} . shiroa currently requires at least Rust version
1.75.

To build with precompiled artifacts, run the following commands:

\begin{Shaded}
\begin{Highlighting}[]
\ExtensionTok{cargo}\NormalTok{ install }\AttributeTok{{-}{-}git}\NormalTok{ https://github.com/Myriad{-}Dreamin/shiroa }\AttributeTok{{-}{-}locked}\NormalTok{ shiroa{-}cli}
\end{Highlighting}
\end{Shaded}

To build from source, run the following commands:

\begin{Shaded}
\begin{Highlighting}[]
\FunctionTok{git}\NormalTok{ clone https://github.com/Myriad{-}Dreamin/shiroa.git}
\FunctionTok{git}\NormalTok{ submodule update }\AttributeTok{{-}{-}recursive} \AttributeTok{{-}{-}init}
\ExtensionTok{cargo}\NormalTok{ run }\AttributeTok{{-}{-}bin}\NormalTok{ shiroa{-}build}
\CommentTok{\# optional: install it globally}
\ExtensionTok{cargo}\NormalTok{ install }\AttributeTok{{-}{-}path}\NormalTok{ ./cli}
\end{Highlighting}
\end{Shaded}

With global installation, to uninstall, run the command
\texttt{\ cargo\ uninstall\ shiroa\ } .

Again, make sure to add the Cargo bin directory to your
\texttt{\ PATH\ } .

\subsubsection{Get started}\label{get-started}

See the
\href{https://myriad-dreamin.github.io/shiroa/guide/get-started.html}{Get-started}
online documentation.

\subsubsection{Setup for writing your
book}\label{setup-for-writing-your-book}

We don’t provide a watch command, but \texttt{\ shiroa\ } is
designated to embracing all of the approaches to writing typst
documents. It’s feasible to preview your documents by following
approaches (like previewing normal documents):

\begin{itemize}
\item
  via \href{https://typst.app/}{Official Web App} .
\item
  via VSCod(e,ium), see
  \href{https://marketplace.visualstudio.com/items?itemName=myriad-dreamin.tinymist}{Tinymist}
  and
  \href{https://marketplace.visualstudio.com/items?itemName=mgt19937.typst-preview}{Typst
  Preview} .
\item
  via other editors. For example of neovim, see
  \href{https://github.com/kaarmu/typst.vim}{typst.vim} and
  \href{https://github.com/Enter-tainer/typst-preview\#use-without-vscode}{Typst
  Preview} .
\item
  via \texttt{\ typst-cli\ watch\ } , See
  \href{https://github.com/typst/typst\#usage}{typst-cli watch} .
\end{itemize}

\subsubsection{Acknowledgement}\label{acknowledgement}

\begin{itemize}
\item
  The
  \href{https://github.com/typst/packages/raw/main/packages/preview/shiroa/0.1.2/themes/mdbook/}{mdbook
  theme} is borrowed from
  \href{https://github.com/rust-lang/mdBook/tree/master/src/theme}{mdBook}
  project.
\item
  Compile the document with awesome
  \href{https://github.com/typst/typst}{Typst} .
\end{itemize}

\subsubsection{How to add}\label{how-to-add}

Copy this into your project and use the import as \texttt{\ shiroa\ }

\begin{verbatim}
#import "@preview/shiroa:0.1.2"
\end{verbatim}

\includesvg[width=0.16667in,height=0.16667in]{/assets/icons/16-copy.svg}

Check the docs for
\href{https://typst.app/docs/reference/scripting/\#packages}{more
information on how to import packages} .

\subsubsection{About}\label{about}

\begin{description}
\tightlist
\item[Author :]
\href{https://github.com/Myriad-Dreamin}{Myriad-Dreamin}
\item[License:]
Apache-2.0
\item[Current version:]
0.1.2
\item[Last updated:]
October 22, 2024
\item[First released:]
June 17, 2024
\item[Archive size:]
11.2 kB
\href{https://packages.typst.org/preview/shiroa-0.1.2.tar.gz}{\pandocbounded{\includesvg[keepaspectratio]{/assets/icons/16-download.svg}}}
\item[Repository:]
\href{https://github.com/Myriad-Dreamin/shiroa}{GitHub}
\item[Categor y :]
\begin{itemize}
\tightlist
\item[]
\item
  \pandocbounded{\includesvg[keepaspectratio]{/assets/icons/16-docs.svg}}
  \href{https://typst.app/universe/search/?category=book}{Book}
\end{itemize}
\end{description}

\subsubsection{Where to report issues?}\label{where-to-report-issues}

This package is a project of Myriad-Dreamin . Report issues on
\href{https://github.com/Myriad-Dreamin/shiroa}{their repository} . You
can also try to ask for help with this package on the
\href{https://forum.typst.app}{Forum} .

Please report this package to the Typst team using the
\href{https://typst.app/contact}{contact form} if you believe it is a
safety hazard or infringes upon your rights.

\phantomsection\label{versions}
\subsubsection{Version history}\label{version-history}

\begin{longtable}[]{@{}ll@{}}
\toprule\noalign{}
Version & Release Date \\
\midrule\noalign{}
\endhead
\bottomrule\noalign{}
\endlastfoot
0.1.2 & October 22, 2024 \\
\href{https://typst.app/universe/package/shiroa/0.1.1/}{0.1.1} & August
26, 2024 \\
\href{https://typst.app/universe/package/shiroa/0.1.0/}{0.1.0} & June
17, 2024 \\
\end{longtable}

Typst GmbH did not create this package and cannot guarantee correct
functionality of this package or compatibility with any version of the
Typst compiler or app.


\section{Package List LaTeX/fletcher.tex}
\title{typst.app/universe/package/fletcher}

\phantomsection\label{banner}
\section{fletcher}\label{fletcher}

{ 0.5.2 }

Draw diagrams with nodes and arrows.

{ } Featured Package

\phantomsection\label{readme}
\href{https://github.com/typst/packages/raw/main/packages/preview/fletcher/0.5.2/docs/manual.pdf?raw=true}{\pandocbounded{\includegraphics[keepaspectratio]{https://img.shields.io/badge/docs-manual.pdf-green}}}
\pandocbounded{\includegraphics[keepaspectratio]{https://img.shields.io/badge/version-0.5.2-green}}
\href{https://github.com/Jollywatt/typst-fletcher/tree/dev}{\pandocbounded{\includegraphics[keepaspectratio]{https://img.shields.io/badge/dynamic/toml?url=https\%3A\%2F\%2Fgithub.com\%2FJollywatt\%2Ftypst-fletcher\%2Fraw\%2Fdev\%2Ftypst.toml&query=package.version&label=dev&color=blue}}}
\href{https://github.com/Jollywatt/typst-fletcher}{\pandocbounded{\includegraphics[keepaspectratio]{https://img.shields.io/badge/GitHub-repo-blue}}}

\emph{\textbf{fletcher} (noun) a maker of arrows}

A \href{https://typst.app/}{Typst} package for drawing diagrams with
arrows, built on top of
\href{https://github.com/johannes-wolf/cetz}{CeTZ} . See the
\href{https://github.com/typst/packages/raw/main/packages/preview/fletcher/0.5.2/docs/manual.pdf?raw=true}{manual}
for documentation.

\begin{Shaded}
\begin{Highlighting}[]
\NormalTok{\#import "@preview/fletcher:0.5.2" as fletcher: diagram, node, edge}
\end{Highlighting}
\end{Shaded}

\pandocbounded{\includesvg[keepaspectratio]{https://github.com/typst/packages/raw/main/packages/preview/fletcher/0.5.2/docs/readme-examples/first-isomorphism-theorem-light.svg}}

\begin{Shaded}
\begin{Highlighting}[]
\NormalTok{\#diagram(cell{-}size: 15mm, $}
\NormalTok{  G edge(f, {-}\textgreater{}) edge("d", pi, {-}\textgreater{}\textgreater{}) \& im(f) \textbackslash{}}
\NormalTok{  G slash ker(f) edge("ur", tilde(f), "hook{-}{-}\textgreater{}")}
\NormalTok{$)}
\end{Highlighting}
\end{Shaded}

\pandocbounded{\includesvg[keepaspectratio]{https://github.com/typst/packages/raw/main/packages/preview/fletcher/0.5.2/docs/readme-examples/flowchart-trap-light.svg}}

\begin{Shaded}
\begin{Highlighting}[]
\NormalTok{// https://xkcd.com/1195/}
\NormalTok{\#import fletcher.shapes: diamond}
\NormalTok{\#set text(font: "Comic Neue", weight: 600)}

\NormalTok{\#diagram(}
\NormalTok{  node{-}stroke: 1pt,}
\NormalTok{  edge{-}stroke: 1pt,}
\NormalTok{  node((0,0), [Start], corner{-}radius: 2pt, extrude: (0, 3)),}
\NormalTok{  edge("{-}|\textgreater{}"),}
\NormalTok{  node((0,1), align(center)[}
\NormalTok{    Hey, wait,\textbackslash{} this flowchart\textbackslash{} is a trap!}
\NormalTok{  ], shape: diamond),}
\NormalTok{  edge("d,r,u,l", "{-}|\textgreater{}", [Yes], label{-}pos: 0.1)}
\NormalTok{)}
\end{Highlighting}
\end{Shaded}

\pandocbounded{\includesvg[keepaspectratio]{https://github.com/typst/packages/raw/main/packages/preview/fletcher/0.5.2/docs/readme-examples/state-machine-light.svg}}

\begin{Shaded}
\begin{Highlighting}[]
\NormalTok{\#set text(10pt)}
\NormalTok{\#diagram(}
\NormalTok{  node{-}stroke: .1em,}
\NormalTok{  node{-}fill: gradient.radial(blue.lighten(80\%), blue, center: (30\%, 20\%), radius: 80\%),}
\NormalTok{  spacing: 4em,}
\NormalTok{  edge(({-}1,0), "r", "{-}|\textgreater{}", \textasciigrave{}open(path)\textasciigrave{}, label{-}pos: 0, label{-}side: center),}
\NormalTok{  node((0,0), \textasciigrave{}reading\textasciigrave{}, radius: 2em),}
\NormalTok{  edge(\textasciigrave{}read()\textasciigrave{}, "{-}|\textgreater{}"),}
\NormalTok{  node((1,0), \textasciigrave{}eof\textasciigrave{}, radius: 2em),}
\NormalTok{  edge(\textasciigrave{}close()\textasciigrave{}, "{-}|\textgreater{}"),}
\NormalTok{  node((2,0), \textasciigrave{}closed\textasciigrave{}, radius: 2em, extrude: ({-}2.5, 0)),}
\NormalTok{  edge((0,0), (0,0), \textasciigrave{}read()\textasciigrave{}, "{-}{-}|\textgreater{}", bend: 130deg),}
\NormalTok{  edge((0,0), (2,0), \textasciigrave{}close()\textasciigrave{}, "{-}|\textgreater{}", bend: {-}40deg),}
\NormalTok{)}
\end{Highlighting}
\end{Shaded}

\pandocbounded{\includesvg[keepaspectratio]{https://github.com/typst/packages/raw/main/packages/preview/fletcher/0.5.2/docs/readme-examples/feynman-diagram-light.svg}}

\begin{Shaded}
\begin{Highlighting}[]
\NormalTok{\#diagram($}
\NormalTok{  e\^{}{-} edge("rd", "{-}\textless{}|{-}") \& \& \& edge("ld", "{-}|\textgreater{}{-}") e\^{}+ \textbackslash{}}
\NormalTok{  \& edge(gamma, "wave") \textbackslash{}}
\NormalTok{  e\^{}+ edge("ru", "{-}|\textgreater{}{-}") \& \& \& edge("lu", "{-}\textless{}|{-}") e\^{}{-} \textbackslash{}}
\NormalTok{$)}
\end{Highlighting}
\end{Shaded}

Pull requests are most welcome!

\begin{longtable}[]{@{}
  >{\raggedright\arraybackslash}p{(\linewidth - 2\tabcolsep) * \real{0.5000}}
  >{\raggedright\arraybackslash}p{(\linewidth - 2\tabcolsep) * \real{0.5000}}@{}}
\toprule\noalign{}
\endhead
\bottomrule\noalign{}
\endlastfoot
\href{https://github.com/typst/packages/raw/main/packages/preview/fletcher/0.5.2/docs/gallery/commutative.typ}{}

\includesvg[width=1\linewidth,height=\textheight,keepaspectratio]{https://github.com/typst/packages/raw/main/packages/preview/fletcher/0.5.2/docs/gallery/commutative.svg}
&
\href{https://github.com/typst/packages/raw/main/packages/preview/fletcher/0.5.2/docs/gallery/algebra-cube.typ}{}

\includesvg[width=1\linewidth,height=\textheight,keepaspectratio]{https://github.com/typst/packages/raw/main/packages/preview/fletcher/0.5.2/docs/gallery/algebra-cube.svg} \\
\href{https://github.com/typst/packages/raw/main/packages/preview/fletcher/0.5.2/docs/gallery/ml-architecture.typ}{}

\includesvg[width=1\linewidth,height=\textheight,keepaspectratio]{https://github.com/typst/packages/raw/main/packages/preview/fletcher/0.5.2/docs/gallery/ml-architecture.svg}
&
\href{https://github.com/typst/packages/raw/main/packages/preview/fletcher/0.5.2/docs/gallery/io-flowchart.typ}{}

\includesvg[width=1\linewidth,height=\textheight,keepaspectratio]{https://github.com/typst/packages/raw/main/packages/preview/fletcher/0.5.2/docs/gallery/io-flowchart.svg} \\
\href{https://github.com/typst/packages/raw/main/packages/preview/fletcher/0.5.2/docs/gallery/digraph.typ}{}

\includesvg[width=1\linewidth,height=\textheight,keepaspectratio]{https://github.com/typst/packages/raw/main/packages/preview/fletcher/0.5.2/docs/gallery/digraph.svg}
&
\href{https://github.com/typst/packages/raw/main/packages/preview/fletcher/0.5.2/docs/gallery/node-groups.typ}{}

\includesvg[width=1\linewidth,height=\textheight,keepaspectratio]{https://github.com/typst/packages/raw/main/packages/preview/fletcher/0.5.2/docs/gallery/node-groups.svg} \\
\href{https://github.com/typst/packages/raw/main/packages/preview/fletcher/0.5.2/docs/gallery/uml-diagram.typ}{}

\includesvg[width=1\linewidth,height=\textheight,keepaspectratio]{https://github.com/typst/packages/raw/main/packages/preview/fletcher/0.5.2/docs/gallery/uml-diagram.svg}
& \\
\end{longtable}

\subsection{Change log}\label{change-log}

\subsubsection{0.5.2}\label{section}

\begin{itemize}
\tightlist
\item
  \textbf{Require \texttt{\ typst\ } version
  \texttt{\ \textgreater{}=0.12.0\ } .}
\item
  Update \texttt{\ cetz\ } dependency to \texttt{\ 0.3.1\ } .
  \textbf{Note:} This may slightly change edge label positions.
\item
  Add \texttt{\ loop-angle\ } option to \texttt{\ edge()\ } (\#36).
\end{itemize}

\subsubsection{0.5.1}\label{section-1}

\begin{itemize}
\tightlist
\item
  Fix nodes which \texttt{\ enclose\ } absolute coordinates.
\item
  Allow CeTZ-style coordinate expressions in node \texttt{\ enclose\ }
  option.
\item
  Fix crash with polar coordinates.
\item
  Fix edges which bend at 0deg or 180deg (e.g., \texttt{\ edge("r,r")\ }
  or \texttt{\ edge("r,l")\ } ) and enhance the way the corner radius
  adapts to the bend angle. \textbf{Note:} This may change diagram
  layout from previous versions.
\item
  Improve error messages.
\item
  Add marks for crow’s foot notation: \texttt{\ n\ } (many),
  \texttt{\ n?\ } (zero or more), \texttt{\ n!\ } (one or more),
  \texttt{\ 1\ } (one), \texttt{\ 1?\ } (zero or one), \texttt{\ 1!\ }
  (exactly one).
\item
  Add \texttt{\ node-shape\ } option to \texttt{\ diagram()\ } .
\end{itemize}

\subsubsection{0.5.0}\label{section-2}

\begin{itemize}
\tightlist
\item
  Greatly enhance coordinate system.

  \begin{itemize}
  \tightlist
  \item
    Support CeTZ-style coordinate expressions (relative, polar,
    interpolating, named coordinates, etc).
  \item
    Absolute coordinates (physical lengths) can be used alongside
    “elastic� coordinates (row/column positions).
  \end{itemize}
\item
  Add \texttt{\ label-angle\ } option to \texttt{\ edge()\ } .
\item
  Add \texttt{\ label-wrapper\ } option to allow changing the label
  inset, outline stroke, and so on (\#26).
\item
  Add \texttt{\ label-size\ } option to control default edge label text
  size (\#35)
\item
  Add \texttt{\ trapezium\ } node shape.
\item
  Disallow string labels to be passed as positional arguments to
  \texttt{\ edge()\ } (to eliminate ambiguity). Used named argument or
  pass content instead.
\end{itemize}

\subsubsection{0.4.5}\label{section-3}

\begin{itemize}
\tightlist
\item
  Add isosceles triangle node shape (\#31).
\item
  Add \texttt{\ fit\ } and \texttt{\ dir\ } options to various node
  shapes to adjust sizing and orientation.
\item
  Allow more than one consecutive edge to have an implicit end vertex.
\item
  Allow \texttt{\ snap-to\ } to be \texttt{\ none\ } to disable edge
  snapping (\#32).
\end{itemize}

\subsubsection{0.4.4}\label{section-4}

\begin{itemize}
\tightlist
\item
  Support fully customisable marks/arrowheads!

  \begin{itemize}
  \tightlist
  \item
    Added new mark styles and tweaked some existing defaults.
    \textbf{Note.} This may change the micro-typography of diagrams from
    previous versions.
  \end{itemize}
\item
  Add node groups via the \texttt{\ enclose\ } option of
  \texttt{\ node()\ } .
\item
  Node labels can be aligned inside the node with \texttt{\ align()\ } .
\item
  Node labels wrap naturally when label text is wider than the node.
  \textbf{Note:} This may change diagram layout from previous versions.
\item
  Add a \texttt{\ layer\ } option to nodes and edges to control drawing
  order.
\item
  Add node shapes: \texttt{\ ellipse\ } , \texttt{\ octagon\ } .
\end{itemize}

\subsubsection{0.4.3}\label{section-5}

\begin{itemize}
\tightlist
\item
  Fixed edge crossing backgrounds being drawn above nodes (\#14).
\item
  Add \texttt{\ fletcher.hide()\ } to hide elements with/without
  affecting layout, useful for incremental diagrams in slides (\#15).
\item
  Support \texttt{\ shift\ } ing edges by coordinate deltas as well as
  absolute lengths (\#13).
\item
  Support node names (\#8).
\end{itemize}

\subsubsection{0.4.2}\label{section-6}

\begin{itemize}
\tightlist
\item
  Improve edge-to-node snapping. Edges can terminate anywhere near a
  node (not just at its center) and will automatically snap to the node
  outline. Added \texttt{\ snap-to\ } option to \texttt{\ edge()\ } .
\item
  Fix node \texttt{\ inset\ } being half the amount specified. If
  upgrading from previous version, you will need to divide node
  \texttt{\ inset\ } values by two to preserve diagram layout.
\item
  Add \texttt{\ decorations\ } option to \texttt{\ edge()\ } for CeTZ
  path decorations ( \texttt{\ "wave"\ } , \texttt{\ "zigzag"\ } , and
  \texttt{\ "coil"\ } , also accepted as positional string arguments).
\end{itemize}

\subsubsection{0.4.1}\label{section-7}

\begin{itemize}
\tightlist
\item
  Support custom node shapes! Edges connect to node outlines
  automatically.

  \begin{itemize}
  \tightlist
  \item
    New \texttt{\ shapes\ } submodule, containing \texttt{\ diamond\ } ,
    \texttt{\ pill\ } , \texttt{\ parallelogram\ } ,
    \texttt{\ hexagon\ } , and other node shapes.
  \end{itemize}
\item
  Allow edges to have multiple segments.

  \begin{itemize}
  \tightlist
  \item
    Add \texttt{\ vertices\ } an \texttt{\ corner-radius\ } options to
    \texttt{\ edge()\ } .
  \item
    Relative coordinate shorthands may be comma separated to signify
    multiple segments, e.g., \texttt{\ "r,u,ll"\ } .
  \end{itemize}
\item
  Add \texttt{\ dodge\ } option to \texttt{\ edge()\ } to adjust end
  points.
\item
  Support \texttt{\ cetz:0.2.0\ } .
\end{itemize}

\subsubsection{0.4.0}\label{section-8}

\begin{itemize}
\tightlist
\item
  Add ability to specify diagrams in math-mode, using \texttt{\ \&\ } to
  separate nodes.
\item
  Allow implicit and relative edge coordinates, e.g.,
  \texttt{\ edge("d")\ } becomes \texttt{\ edge(prev-node,\ (0,\ 1))\ }
  .
\item
  Add ability to place marks anywhere along an edge. Shorthands now
  accept an optional middle mark, for example
  \texttt{\ \textbar{}-\textgreater{}-\textbar{}\ } and
  \texttt{\ hook-/-\textgreater{}\textgreater{}\ } .
\item
  Add “hanging tail� correction to marks on curved edges. Marks now
  rotate a bit to fit more comfortably along tightly curving arcs.
\item
  Add more arrowheads for the sake of it: \texttt{\ \}\textgreater{}\ }
  , \texttt{\ \textless{}\{\ } , \texttt{\ /\ } ,
  \texttt{\ \textbackslash{}\ } , \texttt{\ x\ } , \texttt{\ X\ } ,
  \texttt{\ *\ } (solid dot), \texttt{\ @\ } (solid circle).
\item
  Add \texttt{\ axes\ } option to \texttt{\ diagram()\ } to control the
  direction of each axis in the diagram’s coordinate system.
\item
  Add \texttt{\ width\ } , \texttt{\ height\ } and \texttt{\ radius\ }
  options to \texttt{\ node()\ } for explicit control over size.
\item
  Add \texttt{\ corner-radius\ } option to \texttt{\ node()\ } .
\item
  Add \texttt{\ stroke\ } option to \texttt{\ edge()\ } replacing
  \texttt{\ thickness\ } and \texttt{\ paint\ } options.
\item
  Add \texttt{\ edge-stroke\ } option to \texttt{\ diagram()\ }
  replacing \texttt{\ edge-thickness\ } .
\end{itemize}

\subsubsection{0.3.0}\label{section-9}

\begin{itemize}
\tightlist
\item
  Make round-style arrow heads better approximate the default math font.
\item
  Add solid arrow heads with shorthand
  \texttt{\ \textless{}\textbar{}-\ } ,
  \texttt{\ -\textbar{}\textgreater{}\ } and double-bar
  \texttt{\ \textbar{}\textbar{}-\ } ,
  \texttt{\ -\textbar{}\textbar{}\ } .
\item
  Add an \texttt{\ extrude\ } option to \texttt{\ node()\ } which
  duplicates and extrudes the node’s stroke, enabling double stroke
  effects.
\end{itemize}

\subsubsection{0.2.0}\label{section-10}

\begin{itemize}
\tightlist
\item
  Experimental support for customising arrowheads.
\item
  Add right-angled edges with
  \texttt{\ edge(...,\ corner:\ left/right)\ } .
\end{itemize}

\subsection{Star History}\label{star-history}

\href{https://star-history.com/\#jollywatt/typst-fletcher&Date}{\pandocbounded{\includegraphics[keepaspectratio]{https://api.star-history.com/svg?repos=jollywatt/typst-fletcher&type=Date}}}

\subsubsection{How to add}\label{how-to-add}

Copy this into your project and use the import as \texttt{\ fletcher\ }

\begin{verbatim}
#import "@preview/fletcher:0.5.2"
\end{verbatim}

\includesvg[width=0.16667in,height=0.16667in]{/assets/icons/16-copy.svg}

Check the docs for
\href{https://typst.app/docs/reference/scripting/\#packages}{more
information on how to import packages} .

\subsubsection{About}\label{about}

\begin{description}
\tightlist
\item[Author :]
Joseph Wilson (Jollywatt)
\item[License:]
MIT
\item[Current version:]
0.5.2
\item[Last updated:]
October 25, 2024
\item[First released:]
November 23, 2023
\item[Minimum Typst version:]
0.12.0
\item[Archive size:]
51.1 kB
\href{https://packages.typst.org/preview/fletcher-0.5.2.tar.gz}{\pandocbounded{\includesvg[keepaspectratio]{/assets/icons/16-download.svg}}}
\item[Repository:]
\href{https://github.com/Jollywatt/typst-fletcher}{GitHub}
\end{description}

\subsubsection{Where to report issues?}\label{where-to-report-issues}

This package is a project of Joseph Wilson (Jollywatt) . Report issues
on \href{https://github.com/Jollywatt/typst-fletcher}{their repository}
. You can also try to ask for help with this package on the
\href{https://forum.typst.app}{Forum} .

Please report this package to the Typst team using the
\href{https://typst.app/contact}{contact form} if you believe it is a
safety hazard or infringes upon your rights.

\phantomsection\label{versions}
\subsubsection{Version history}\label{version-history}

\begin{longtable}[]{@{}ll@{}}
\toprule\noalign{}
Version & Release Date \\
\midrule\noalign{}
\endhead
\bottomrule\noalign{}
\endlastfoot
0.5.2 & October 25, 2024 \\
\href{https://typst.app/universe/package/fletcher/0.5.1/}{0.5.1} & July
10, 2024 \\
\href{https://typst.app/universe/package/fletcher/0.5.0/}{0.5.0} & June
11, 2024 \\
\href{https://typst.app/universe/package/fletcher/0.4.5/}{0.4.5} & May
17, 2024 \\
\href{https://typst.app/universe/package/fletcher/0.4.4/}{0.4.4} & May
3, 2024 \\
\href{https://typst.app/universe/package/fletcher/0.4.3/}{0.4.3} & April
2, 2024 \\
\href{https://typst.app/universe/package/fletcher/0.4.2/}{0.4.2} &
February 23, 2024 \\
\href{https://typst.app/universe/package/fletcher/0.4.1/}{0.4.1} &
February 8, 2024 \\
\href{https://typst.app/universe/package/fletcher/0.4.0/}{0.4.0} &
January 30, 2024 \\
\href{https://typst.app/universe/package/fletcher/0.3.0/}{0.3.0} &
January 1, 2024 \\
\href{https://typst.app/universe/package/fletcher/0.2.0/}{0.2.0} &
November 29, 2023 \\
\href{https://typst.app/universe/package/fletcher/0.1.1/}{0.1.1} &
November 23, 2023 \\
\end{longtable}

Typst GmbH did not create this package and cannot guarantee correct
functionality of this package or compatibility with any version of the
Typst compiler or app.


\section{Package List LaTeX/numty.tex}
\title{typst.app/universe/package/numty}

\phantomsection\label{banner}
\section{numty}\label{numty}

{ 0.0.5 }

Numeric Typst

\phantomsection\label{readme}
\subsection{Numty}\label{numty-1}

\subsubsection{Numeric Typst}\label{numeric-typst}

A library for performing mathematical operations on n-dimensional
matrices, vectors/arrays, and numbers in Typst, with support for
broadcasting and handling NaN values. Numty’s broadcasting rules and
API are inspired by NumPy.

\begin{Shaded}
\begin{Highlighting}[]
\NormalTok{\#import "numty.typ" as nt}

\NormalTok{// Define vectors and matrices}
\NormalTok{\#let a = (1, 2, 3)}
\NormalTok{\#let b = 2}
\NormalTok{\#let m = ((1, 2), (1, 3))}

\NormalTok{// Element{-}wise operations with broadcasting}
\NormalTok{\#nt.mult(a, b)  // Multiply vector \textquotesingle{}a\textquotesingle{} by scalar \textquotesingle{}b\textquotesingle{}: (2, 4, 6)}
\NormalTok{\#nt.add(a, a)   // Add two vectors: (2, 4, 6)}
\NormalTok{\#nt.add(2, a)   // Add scalar \textquotesingle{}2\textquotesingle{} to each element of vector \textquotesingle{}a\textquotesingle{}: (3, 4, 5)}
\NormalTok{\#nt.add(m, 1)   // Add scalar \textquotesingle{}1\textquotesingle{} to each element of matrix \textquotesingle{}m\textquotesingle{}: ((2, 3), (2, 4))}

\NormalTok{// Dot product of vectors}
\NormalTok{\#nt.dot(a, a)   // Dot product of vector \textquotesingle{}a\textquotesingle{} with itself: 14}

\NormalTok{// Handling NaN cases in mathematical functions}
\NormalTok{\#calc.sin((3, 4)) // Fails, as Typst does not support vector operations directly}
\NormalTok{\#nt.sin((3.4))    // Sine of each element in vector: (0.14411, 0.90929)}

\NormalTok{// Generate equally spaced values and vectorized functions}
\NormalTok{\#let x = nt.linspace(0, 10, 3)  // Generate 3 equally spaced values between 0 and 10: (0, 5, 10)}
\NormalTok{\#let y = nt.sin(x)              // Apply sine function to each element: (0, {-}0.95, {-}0.54)}

\NormalTok{// Logical operations}
\NormalTok{\#nt.eq(a, b)      // Compare each element in \textquotesingle{}a\textquotesingle{} to \textquotesingle{}b\textquotesingle{}: (false, true, false)}
\NormalTok{\#nt.any(nt.eq(a, b)) // Check if any element in \textquotesingle{}a\textquotesingle{} equals \textquotesingle{}b\textquotesingle{}: true}
\NormalTok{\#nt.all(nt.eq(a, b)) // Check if all elements in \textquotesingle{}a\textquotesingle{} equal \textquotesingle{}b\textquotesingle{}: false}

\NormalTok{// Handling special cases like division by zero}
\NormalTok{\#nt.div((1, 3), (2, 0))  // Element{-}wise division, with NaN for division by zero: (0.5, float.nan)}

\NormalTok{// Matrix operations (element{-}wise)}
\NormalTok{\#nt.add(m, 1)  // Add scalar to matrix elements: ((2, 3), (2, 4))}

\NormalTok{// matrix}
\NormalTok{\#nt.transpose(m)  // transposition}
\NormalTok{\#nt.matmul(m,m) //  matrix multipliation}
\NormalTok{\#nt.matmul(c(1,2), r(2,3)) //  colum vector times row vector multiplication.}
\NormalTok{\#np.trace(m) // trace}
\NormalTok{\#np.det(m) /2x2 determinant}
 
\NormalTok{// printing}
\NormalTok{\#nt.print(m, " != " , (1,2))  // long dollar print, see in pdf }
\NormalTok{\#nt.p(m, " != " , (1,2))  //  short long print print, see in pdf }
\end{Highlighting}
\end{Shaded}

Since vesion 0.0.4 n-dim matrices are supported as well in most
operations.

\subsection{Supported Features:}\label{supported-features}

\subsubsection{Dimensions:}\label{dimensions}

Numty uses standard typst list as a base type, most 1d operations like
dot are suported directly for them.

For matrix specific operations we use 2d arrays / nested arrays, that
are also the standard typst list, but nested like in: ((1,2), (1,1)).

For convenience you can create column or row vectors with the \#nt.c and
\#nt.r functions.

\begin{Shaded}
\begin{Highlighting}[]
\NormalTok{\#import "numty.typ" as nt}
\NormalTok{\#import "numty.typ": c, r}

\NormalTok{\#let a = (1,2,3)}
\NormalTok{\#let b = (3,2,1)}
\NormalTok{\#c(..a) // ((1,),(2,),(3,)) }
\NormalTok{\#r(..b) // ((3,2,1),)}
\NormalTok{\#nt.matmul(c(..a), r(..b)) // column @ row}
\end{Highlighting}
\end{Shaded}

\subsubsection{Logic Operations:}\label{logic-operations}

\begin{Shaded}
\begin{Highlighting}[]
\NormalTok{\#let a = (1,2,3)}
\NormalTok{\#let b = 2}

\NormalTok{\#nt.eq(a,b)  // (false, true, false)}
\NormalTok{\#nt.all(nt.eq(a,b))  // false}
\NormalTok{\#nt.any(nt.eq(a,b))  // true}
\end{Highlighting}
\end{Shaded}

\subsubsection{Math operators:}\label{math-operators}

All operators are element-wise, traditional matrix multiplication is not
yet supported.

\begin{Shaded}
\begin{Highlighting}[]
\NormalTok{\#nt.add((0,1,3), 1)  // (1,2,4)}
\NormalTok{\#nt.mult((1,3),(2,2)) // (2,6)}
\NormalTok{\#nt.div((1,3), (2,0)) // (0.5,float.nan)}
\end{Highlighting}
\end{Shaded}

\subsubsection{Algebra with Nan
handling:}\label{algebra-with-nan-handling}

\begin{Shaded}
\begin{Highlighting}[]
\NormalTok{\#nt.log((0,1,3)) //  (float.nan, 0 , 0.47...)}
\NormalTok{\#nt.sin((1,3)) //  (0.84.. , 0.14...)}
\end{Highlighting}
\end{Shaded}

\subsubsection{Vector operations:}\label{vector-operations}

Basic vector operations

\begin{Shaded}
\begin{Highlighting}[]
\NormalTok{\#nt.dot((1,2),(2,4))  //  9}
\NormalTok{\#nt.normalize((1,4), l:1) // (1/5,4/5)}
\end{Highlighting}
\end{Shaded}

\subsubsection{Others:}\label{others}

Functions for creating equally spaced indexes in linear and logspace,
usefull for log plots, to sample in acordance to the logarithmic space.

\begin{Shaded}
\begin{Highlighting}[]
\NormalTok{\#nt.linspace(0,10,3) // (0,5,10)}
\NormalTok{\#nt.logspace(1,3,3)}
\NormalTok{\#nt.geomspace(1,3,3) }
\end{Highlighting}
\end{Shaded}

\subsubsection{Matrix}\label{matrix}

\begin{Shaded}
\begin{Highlighting}[]
\NormalTok{\#nt.matmul(m,m )              // matrix multiplication}
\NormalTok{\#nt.det(((1,3), (3,4)))       // only 2x2 supported for now}
\NormalTok{\#nt.trace(((1,3), (3,4)))     // trace of square matrix}
\NormalTok{\#nt.transpose(((1,3), (3,4))) // matrix transposition}
\end{Highlighting}
\end{Shaded}

\subsubsection{Printing}\label{printing}

Numty supports \$ printing to the pdf of numerical matrices, both long
and short format.

\begin{Shaded}
\begin{Highlighting}[]
\NormalTok{\#nt.print((1,2),(4,2)))  // to display in the pdf}
\NormalTok{\#nt.p((1,2),(4,2)), " random string ")     // to display in the pdf}
\end{Highlighting}
\end{Shaded}

\subsubsection{How to add}\label{how-to-add}

Copy this into your project and use the import as \texttt{\ numty\ }

\begin{verbatim}
#import "@preview/numty:0.0.5"
\end{verbatim}

\includesvg[width=0.16667in,height=0.16667in]{/assets/icons/16-copy.svg}

Check the docs for
\href{https://typst.app/docs/reference/scripting/\#packages}{more
information on how to import packages} .

\subsubsection{About}\label{about}

\begin{description}
\tightlist
\item[Author :]
Pablo Ruiz Cuevas
\item[License:]
MIT
\item[Current version:]
0.0.5
\item[Last updated:]
November 12, 2024
\item[First released:]
October 22, 2024
\item[Archive size:]
4.27 kB
\href{https://packages.typst.org/preview/numty-0.0.5.tar.gz}{\pandocbounded{\includesvg[keepaspectratio]{/assets/icons/16-download.svg}}}
\item[Repository:]
\href{https://github.com/PabloRuizCuevas/numty}{GitHub}
\item[Categor ies :]
\begin{itemize}
\tightlist
\item[]
\item
  \pandocbounded{\includesvg[keepaspectratio]{/assets/icons/16-hammer.svg}}
  \href{https://typst.app/universe/search/?category=utility}{Utility}
\item
  \pandocbounded{\includesvg[keepaspectratio]{/assets/icons/16-code.svg}}
  \href{https://typst.app/universe/search/?category=scripting}{Scripting}
\end{itemize}
\end{description}

\subsubsection{Where to report issues?}\label{where-to-report-issues}

This package is a project of Pablo Ruiz Cuevas . Report issues on
\href{https://github.com/PabloRuizCuevas/numty}{their repository} . You
can also try to ask for help with this package on the
\href{https://forum.typst.app}{Forum} .

Please report this package to the Typst team using the
\href{https://typst.app/contact}{contact form} if you believe it is a
safety hazard or infringes upon your rights.

\phantomsection\label{versions}
\subsubsection{Version history}\label{version-history}

\begin{longtable}[]{@{}ll@{}}
\toprule\noalign{}
Version & Release Date \\
\midrule\noalign{}
\endhead
\bottomrule\noalign{}
\endlastfoot
0.0.5 & November 12, 2024 \\
\href{https://typst.app/universe/package/numty/0.0.4/}{0.0.4} & October
31, 2024 \\
\href{https://typst.app/universe/package/numty/0.0.3/}{0.0.3} & October
23, 2024 \\
\href{https://typst.app/universe/package/numty/0.0.2/}{0.0.2} & October
22, 2024 \\
\href{https://typst.app/universe/package/numty/0.0.1/}{0.0.1} & October
22, 2024 \\
\end{longtable}

Typst GmbH did not create this package and cannot guarantee correct
functionality of this package or compatibility with any version of the
Typst compiler or app.


\section{Package List LaTeX/modern-iu-thesis.tex}
\title{typst.app/universe/package/modern-iu-thesis}

\phantomsection\label{banner}
\phantomsection\label{template-thumbnail}
\pandocbounded{\includegraphics[keepaspectratio]{https://packages.typst.org/preview/thumbnails/modern-iu-thesis-0.1.0-small.webp}}

\section{modern-iu-thesis}\label{modern-iu-thesis}

{ 0.1.0 }

Modern Typst thesis template for Indiana University

\href{/app?template=modern-iu-thesis&version=0.1.0}{Create project in
app}

\phantomsection\label{readme}
IU thesis template in Typst

Simple template which meets thesis and dissertation
\href{https://graduate.indiana.edu/academic-requirements/thesis-dissertation/index.html}{requirements}
of IU Bloomington’s Graduate School. This template provides a couple
functions:

\begin{Shaded}
\begin{Highlighting}[]
\NormalTok{\#show: thesis.with(}
\NormalTok{    title: [content],}
\NormalTok{    author: [content],}
\NormalTok{    dept: [content],}
\NormalTok{    year: [content],}
\NormalTok{    month: [content],}
\NormalTok{    day: [content],}
\NormalTok{    committee: (}
\NormalTok{        (}
\NormalTok{            name: "Person 1",}
\NormalTok{            title: "Ph.D.",}
\NormalTok{        ),}
\NormalTok{    ),}
\NormalTok{    dedication: [content],}
\NormalTok{    acknowledgement: [content],}
\NormalTok{    abstract: [content],}
\NormalTok{)}
\end{Highlighting}
\end{Shaded}

and

\begin{Shaded}
\begin{Highlighting}[]
\NormalTok{\#iuquote(content)}
\end{Highlighting}
\end{Shaded}

Everything else follows along with basic Typst syntax such as headings,
figures, tables, etc. See the provided template as an example.

\href{/app?template=modern-iu-thesis&version=0.1.0}{Create project in
app}

\subsubsection{How to use}\label{how-to-use}

Click the button above to create a new project using this template in
the Typst app.

You can also use the Typst CLI to start a new project on your computer
using this command:

\begin{verbatim}
typst init @preview/modern-iu-thesis:0.1.0
\end{verbatim}

\includesvg[width=0.16667in,height=0.16667in]{/assets/icons/16-copy.svg}

\subsubsection{About}\label{about}

\begin{description}
\tightlist
\item[Author :]
Bo Johnson
\item[License:]
MIT
\item[Current version:]
0.1.0
\item[Last updated:]
November 12, 2024
\item[First released:]
November 12, 2024
\item[Archive size:]
2.97 kB
\href{https://packages.typst.org/preview/modern-iu-thesis-0.1.0.tar.gz}{\pandocbounded{\includesvg[keepaspectratio]{/assets/icons/16-download.svg}}}
\item[Repository:]
\href{https://github.com/bojohnson5/modern-iu-thesis}{GitHub}
\item[Categor y :]
\begin{itemize}
\tightlist
\item[]
\item
  \pandocbounded{\includesvg[keepaspectratio]{/assets/icons/16-mortarboard.svg}}
  \href{https://typst.app/universe/search/?category=thesis}{Thesis}
\end{itemize}
\end{description}

\subsubsection{Where to report issues?}\label{where-to-report-issues}

This template is a project of Bo Johnson . Report issues on
\href{https://github.com/bojohnson5/modern-iu-thesis}{their repository}
. You can also try to ask for help with this template on the
\href{https://forum.typst.app}{Forum} .

Please report this template to the Typst team using the
\href{https://typst.app/contact}{contact form} if you believe it is a
safety hazard or infringes upon your rights.

\phantomsection\label{versions}
\subsubsection{Version history}\label{version-history}

\begin{longtable}[]{@{}ll@{}}
\toprule\noalign{}
Version & Release Date \\
\midrule\noalign{}
\endhead
\bottomrule\noalign{}
\endlastfoot
0.1.0 & November 12, 2024 \\
\end{longtable}

Typst GmbH did not create this template and cannot guarantee correct
functionality of this template or compatibility with any version of the
Typst compiler or app.


\section{Package List LaTeX/cades.tex}
\title{typst.app/universe/package/cades}

\phantomsection\label{banner}
\section{cades}\label{cades}

{ 0.3.0 }

Generate QR codes in typst.

\phantomsection\label{readme}
Draw QR codes in typst.

\begin{Shaded}
\begin{Highlighting}[]
\NormalTok{\#import "@preview/cades:0.3.0": qr{-}code}

\NormalTok{= QR Code for \textasciigrave{}typst.app\textasciigrave{}:}
\NormalTok{\#qr{-}code("https://typst.app", width: 3cm)}
\end{Highlighting}
\end{Shaded}

\subsection{Documentation}\label{documentation}

\subsubsection{\texorpdfstring{\texttt{\ qr-code\ }}{ qr-code }}\label{qr-code}

Draw a qr code to an image.

\paragraph{Arguments}\label{arguments}

\begin{itemize}
\tightlist
\item
  \texttt{\ content\ } : \texttt{\ str\ } - the content of the qr code
\item
  \texttt{\ width\ } : \texttt{\ length\ } \textbar{} \texttt{\ auto\ }
  - the width of the qr code, default is \texttt{\ auto\ }
\item
  \texttt{\ height\ } : \texttt{\ length\ } \textbar{} \texttt{\ auto\ }
  - the height of the qr code, default is \texttt{\ auto\ }
\item
  \texttt{\ color\ } : \texttt{\ color\ } - the color of the qrcode,
  default is \texttt{\ black\ }
\item
  \texttt{\ background\ } : \texttt{\ color\ } - the background color
  behind the qrcode, default is \texttt{\ white\ }
\item
  \texttt{\ error-correction\ } : \texttt{\ "L"\ } \textbar{}
  \texttt{\ "M"\ } \textbar{} \texttt{\ "Q"\ } \textbar{}
  \texttt{\ "H"\ } - the error correction level for the qr code, default
  is \texttt{\ "M"\ }
\end{itemize}

\paragraph{Returns}\label{returns}

The image, of type \texttt{\ content\ } .

\subsection{Acknowledgements}\label{acknowledgements}

This package uses \href{https://github.com/Enter-tainer/jogs}{Jogs} by
\href{https://github.com/Enter-tainer}{Wenzhuo Liu} and the qr code
rendering code is based on
\href{https://github.com/papnkukn/qrcode-svg/}{qrcode-svg} by
\href{https://github.com/papnkukn}{papnkukn} .

\subsubsection{How to add}\label{how-to-add}

Copy this into your project and use the import as \texttt{\ cades\ }

\begin{verbatim}
#import "@preview/cades:0.3.0"
\end{verbatim}

\includesvg[width=0.16667in,height=0.16667in]{/assets/icons/16-copy.svg}

Check the docs for
\href{https://typst.app/docs/reference/scripting/\#packages}{more
information on how to import packages} .

\subsubsection{About}\label{about}

\begin{description}
\tightlist
\item[Author :]
Niklas Ausborn
\item[License:]
MIT
\item[Current version:]
0.3.0
\item[Last updated:]
November 25, 2023
\item[First released:]
November 10, 2023
\item[Archive size:]
8.61 kB
\href{https://packages.typst.org/preview/cades-0.3.0.tar.gz}{\pandocbounded{\includesvg[keepaspectratio]{/assets/icons/16-download.svg}}}
\item[Repository:]
\href{https://github.com/Midbin/cades}{GitHub}
\end{description}

\subsubsection{Where to report issues?}\label{where-to-report-issues}

This package is a project of Niklas Ausborn . Report issues on
\href{https://github.com/Midbin/cades}{their repository} . You can also
try to ask for help with this package on the
\href{https://forum.typst.app}{Forum} .

Please report this package to the Typst team using the
\href{https://typst.app/contact}{contact form} if you believe it is a
safety hazard or infringes upon your rights.

\phantomsection\label{versions}
\subsubsection{Version history}\label{version-history}

\begin{longtable}[]{@{}ll@{}}
\toprule\noalign{}
Version & Release Date \\
\midrule\noalign{}
\endhead
\bottomrule\noalign{}
\endlastfoot
0.3.0 & November 25, 2023 \\
\href{https://typst.app/universe/package/cades/0.2.0/}{0.2.0} & November
10, 2023 \\
\end{longtable}

Typst GmbH did not create this package and cannot guarantee correct
functionality of this package or compatibility with any version of the
Typst compiler or app.


\section{Package List LaTeX/bamdone-rebuttal.tex}
\title{typst.app/universe/package/bamdone-rebuttal}

\phantomsection\label{banner}
\phantomsection\label{template-thumbnail}
\pandocbounded{\includegraphics[keepaspectratio]{https://packages.typst.org/preview/thumbnails/bamdone-rebuttal-0.1.1-small.webp}}

\section{bamdone-rebuttal}\label{bamdone-rebuttal}

{ 0.1.1 }

Rebuttal/response letter template that allows authors to respond to
feedback given by reviewers in a peer-review process on a point-by-point
basis.

{ } Featured Template

\href{/app?template=bamdone-rebuttal&version=0.1.1}{Create project in
app}

\phantomsection\label{readme}
This is a Typst template for a rebuttal/response letter. It allows
authors to respond to feedback given by reviewers in a peer-review
process on a point-by-point basis. This template is based heavily on the
LaTeX template from Zenke Lab (see
\href{https://zenkelab.org/resources/latex-rebuttal-response-to-reviewers-template/}{here}
).

\subsection{Usage}\label{usage}

You can use this template in the Typst web app by clicking “Start from
template� on the dashboard and searching for
\texttt{\ bamdone-rebuttal\ } .

Alternatively, you can use the CLI to kick this project off using the
command

\begin{verbatim}
typst init @preview/bamdone-rebuttal
\end{verbatim}

Typst will create a new directory with all the files needed to get you
started.

\subsection{Configuration}\label{configuration}

This template exports the \texttt{\ rebuttal\ } function with the
following named arguments:

\begin{itemize}
\tightlist
\item
  \texttt{\ title\ } : (content), something like “Response Letter�
  (the default) or “Rebuttal�.
\item
  \texttt{\ authors\ } : (content), list of author names the top of the
  first column in boldface.
\item
  \texttt{\ date\ } : (content), defaults to
  \texttt{\ datetime.today().display()\ }
\item
  \texttt{\ paper-size\ } : Defaults to \texttt{\ us-letter\ } . Specify
  a
  \href{https://typst.app/docs/reference/layout/page/\#parameters-paper}{paper
  size string} to change the page format. Specifying this will configure
  numeric, IEEE-style citations.
\end{itemize}

The function also accepts a single, positional argument for the body of
the letter.

In addition, the template exports the \texttt{\ configure\ } function
which accepts the following named arguments corresponding to the text
color of various pieces of the letter:

\begin{itemize}
\tightlist
\item
  \texttt{\ point-color\ } : defaults to \texttt{\ blue.darken(30\%)\ }
  , the text color for reviewers’ points
\item
  \texttt{\ response-color\ } : defaults to \texttt{\ black\ } , the
  text color for the authors’ responses
\item
  \texttt{\ new-color\ } : defaults to \texttt{\ green.darken(30\%)\ } ,
  the text color for changes/additions to the manuscript (i.e., within a
  \texttt{\ quote\ } block to show what’s changed from the initial
  submission)
\end{itemize}

The template will initialize your package with a sample call to the
\texttt{\ rebuttal\ } function in a show rule.

\begin{Shaded}
\begin{Highlighting}[]
\NormalTok{// Optional color configuration}
\NormalTok{\#let (point, response, new) = configure(}
\NormalTok{  point{-}color: blue.darken(30\%),}
\NormalTok{  response{-}color: black,}
\NormalTok{  new{-}color: green.darken(30\%)}
\NormalTok{)}

\NormalTok{// Setup the rebuttal}
\NormalTok{\#show: rebuttal.with(}
\NormalTok{  authors: [First A. Author and Second B. Author],}
\NormalTok{  // date: ,}
\NormalTok{  // paper{-}size: ,}
\NormalTok{)}

\NormalTok{// Your content goes below}
\NormalTok{We thank the reviewers...}
\end{Highlighting}
\end{Shaded}

\href{/app?template=bamdone-rebuttal&version=0.1.1}{Create project in
app}

\subsubsection{How to use}\label{how-to-use}

Click the button above to create a new project using this template in
the Typst app.

You can also use the Typst CLI to start a new project on your computer
using this command:

\begin{verbatim}
typst init @preview/bamdone-rebuttal:0.1.1
\end{verbatim}

\includesvg[width=0.16667in,height=0.16667in]{/assets/icons/16-copy.svg}

\subsubsection{About}\label{about}

\begin{description}
\tightlist
\item[Author s :]
\href{https://avonmoll.github.io}{Alexander Von Moll} \&
\href{https://wwww.isaacew.com}{Isaac Weintraub}
\item[License:]
MIT-0
\item[Current version:]
0.1.1
\item[Last updated:]
November 12, 2024
\item[First released:]
May 16, 2024
\item[Minimum Typst version:]
0.12.0
\item[Archive size:]
4.26 kB
\href{https://packages.typst.org/preview/bamdone-rebuttal-0.1.1.tar.gz}{\pandocbounded{\includesvg[keepaspectratio]{/assets/icons/16-download.svg}}}
\item[Repository:]
\href{https://github.com/avonmoll/bamdone-rebuttal}{GitHub}
\item[Categor ies :]
\begin{itemize}
\tightlist
\item[]
\item
  \pandocbounded{\includesvg[keepaspectratio]{/assets/icons/16-envelope.svg}}
  \href{https://typst.app/universe/search/?category=office}{Office}
\item
  \pandocbounded{\includesvg[keepaspectratio]{/assets/icons/16-speak.svg}}
  \href{https://typst.app/universe/search/?category=report}{Report}
\end{itemize}
\end{description}

\subsubsection{Where to report issues?}\label{where-to-report-issues}

This template is a project of Alexander Von Moll and Isaac Weintraub .
Report issues on
\href{https://github.com/avonmoll/bamdone-rebuttal}{their repository} .
You can also try to ask for help with this template on the
\href{https://forum.typst.app}{Forum} .

Please report this template to the Typst team using the
\href{https://typst.app/contact}{contact form} if you believe it is a
safety hazard or infringes upon your rights.

\phantomsection\label{versions}
\subsubsection{Version history}\label{version-history}

\begin{longtable}[]{@{}ll@{}}
\toprule\noalign{}
Version & Release Date \\
\midrule\noalign{}
\endhead
\bottomrule\noalign{}
\endlastfoot
0.1.1 & November 12, 2024 \\
\href{https://typst.app/universe/package/bamdone-rebuttal/0.1.0/}{0.1.0}
& May 16, 2024 \\
\end{longtable}

Typst GmbH did not create this template and cannot guarantee correct
functionality of this template or compatibility with any version of the
Typst compiler or app.


\section{Package List LaTeX/iconic-salmon-svg.tex}
\title{typst.app/universe/package/iconic-salmon-svg}

\phantomsection\label{banner}
\section{iconic-salmon-svg}\label{iconic-salmon-svg}

{ 1.1.0 }

A Typst library for Social Media references with scalable vector
graphics icons.

\phantomsection\label{readme}
The \texttt{\ iconic-salmon-svg\ } package is designed to help you
create your curriculum vitae (CV). It allows you to easily reference
your social media profiles with the typical icon of the service plus a
link to your profile.\\
The package name is a combination of the acronym \emph{SociAL Media
icONs} and the word \emph{iconic} because all these icons have an iconic
design (and iconic also contains the word \emph{icon} ).

\subsection{Features}\label{features}

\begin{itemize}
\tightlist
\item
  Support for popular social media, developer and career platforms
\item
  Uniform design for all entries
\item
  Based on publicly available SVG symbols
\item
  Easy to use
\item
  Allows the customization of the look (extra args are passed to
  \href{https://typst.app/docs/reference/text/text/}{\texttt{\ text\ }}
  )
\end{itemize}

\subsection{Usage}\label{usage}

\subsubsection{Using Typst’s package
manager}\label{using-typstuxe2s-package-manager}

You can install the library using the
\href{https://github.com/typst/packages}{typst packages} :

\begin{Shaded}
\begin{Highlighting}[]
\NormalTok{\#import "@preview/iconic{-}salmon{-}svg:1.0.0": *}
\end{Highlighting}
\end{Shaded}

\subsubsection{Install manually}\label{install-manually}

Put the \texttt{\ iconic-salmon-svg.typ\ } file in your project
directory and import it:

\begin{Shaded}
\begin{Highlighting}[]
\NormalTok{\#import "iconic{-}salmon{-}svg.typ": *}
\end{Highlighting}
\end{Shaded}

\subsubsection{Minimal Example}\label{minimal-example}

\begin{Shaded}
\begin{Highlighting}[]
\NormalTok{// \#import "@preview/iconic{-}salmon{-}svg:1.0.0": github{-}info, gitlab{-}info}
\NormalTok{\#import "iconic{-}salmon{-}svg.typ": github{-}info, gitlab{-}info}

\NormalTok{This project was created by \#github{-}info("Bi0T1N"). You can also find me on \#gitlab{-}info("GitLab", rgb("\#811052"), url: "https://gitlab.com/Bi0T1N").}
\end{Highlighting}
\end{Shaded}

\subsubsection{Examples}\label{examples}

See the
\href{https://github.com/typst/packages/raw/main/packages/preview/iconic-salmon-svg/1.1.0/examples/examples.typ}{\texttt{\ examples.typ\ }}
file for a complete example. The
\href{https://github.com/typst/packages/raw/main/packages/preview/iconic-salmon-svg/1.1.0/examples/}{generated
PDF files} are also available for preview.

\subsection{Contribution}\label{contribution}

Feel free to open an issue or a pull request if you find any problems or
have any suggestions.

\subsection{License}\label{license}

This library is licensed under the MIT license. Feel free to use it in
your project.

\subsection{Trademark Disclaimer}\label{trademark-disclaimer}

Product names, logos, brands and other trademarks referred to in this
project are the property of their respective trademark holders.\\
These trademark holders are not affiliated with this Typst library, nor
are the authors officially endorsed by them, nor do the authors claim
ownership of these trademarks.

\subsubsection{How to add}\label{how-to-add}

Copy this into your project and use the import as
\texttt{\ iconic-salmon-svg\ }

\begin{verbatim}
#import "@preview/iconic-salmon-svg:1.1.0"
\end{verbatim}

\includesvg[width=0.16667in,height=0.16667in]{/assets/icons/16-copy.svg}

Check the docs for
\href{https://typst.app/docs/reference/scripting/\#packages}{more
information on how to import packages} .

\subsubsection{About}\label{about}

\begin{description}
\tightlist
\item[Author :]
Nico Neumann (Bi0T1N)
\item[License:]
MIT
\item[Current version:]
1.1.0
\item[Last updated:]
May 23, 2024
\item[First released:]
April 15, 2024
\item[Archive size:]
22.5 kB
\href{https://packages.typst.org/preview/iconic-salmon-svg-1.1.0.tar.gz}{\pandocbounded{\includesvg[keepaspectratio]{/assets/icons/16-download.svg}}}
\item[Repository:]
\href{https://github.com/Bi0T1N/typst-iconic-salmon-svg}{GitHub}
\item[Categor y :]
\begin{itemize}
\tightlist
\item[]
\item
  \pandocbounded{\includesvg[keepaspectratio]{/assets/icons/16-package.svg}}
  \href{https://typst.app/universe/search/?category=components}{Components}
\end{itemize}
\end{description}

\subsubsection{Where to report issues?}\label{where-to-report-issues}

This package is a project of Nico Neumann (Bi0T1N) . Report issues on
\href{https://github.com/Bi0T1N/typst-iconic-salmon-svg}{their
repository} . You can also try to ask for help with this package on the
\href{https://forum.typst.app}{Forum} .

Please report this package to the Typst team using the
\href{https://typst.app/contact}{contact form} if you believe it is a
safety hazard or infringes upon your rights.

\phantomsection\label{versions}
\subsubsection{Version history}\label{version-history}

\begin{longtable}[]{@{}ll@{}}
\toprule\noalign{}
Version & Release Date \\
\midrule\noalign{}
\endhead
\bottomrule\noalign{}
\endlastfoot
1.1.0 & May 23, 2024 \\
\href{https://typst.app/universe/package/iconic-salmon-svg/1.0.0/}{1.0.0}
& April 15, 2024 \\
\end{longtable}

Typst GmbH did not create this package and cannot guarantee correct
functionality of this package or compatibility with any version of the
Typst compiler or app.


\section{Package List LaTeX/cuti.tex}
\title{typst.app/universe/package/cuti}

\phantomsection\label{banner}
\section{cuti}\label{cuti}

{ 0.3.0 }

Easily simulate (fake) bold, italic and small capital characters.

\phantomsection\label{readme}
Cuti (/kjuË?ti/) is a package that simulates fake bold / fake italic /
fake small captials. This package is typically used on fonts that do not
have a \texttt{\ bold\ } weight, such as “SimSun�.

\subsection{Usage}\label{usage}

Please refer to the
\href{https://csimide.github.io/cuti-docs/en/}{Documentation} .

本 Package æ??供中æ--‡æ--‡æ¡£ï¼š
\href{https://csimide.github.io/cuti-docs/zh-CN/}{中æ--‡æ--‡æ¡£} 。

\subsubsection{Getting Started Quickly (For Chinese
User)}\label{getting-started-quickly-for-chinese-user}

Please add the following content at the beginning of the document:

\begin{Shaded}
\begin{Highlighting}[]
\NormalTok{\#import "@preview/cuti:0.3.0": show{-}cn{-}fakebold}
\NormalTok{\#show: show{-}cn{-}fakebold}
\end{Highlighting}
\end{Shaded}

Then, the bolding for SimHei, SimSun, and KaiTi fonts should work
correctly.

\subsection{Changelog}\label{changelog}

\subsubsection{\texorpdfstring{\texttt{\ 0.3.0\ }}{ 0.3.0 }}\label{section}

\begin{itemize}
\tightlist
\item
  feat: Add fake small caps feature by Tetragramm.
\item
  fix: \texttt{\ show-fakebold\ } may crash on Typst version 0.12.0.
\end{itemize}

\subsubsection{\texorpdfstring{\texttt{\ 0.2.1\ }}{ 0.2.1 }}\label{section-1}

\begin{itemize}
\tightlist
\item
  feat: The stroke of fake bold will use the same color as the text.
\item
  fix: Attempted to fix the issue with the spacing of punctuation in
  fake italic (\#2), but there are still problems.
\end{itemize}

\subsubsection{\texorpdfstring{\texttt{\ 0.2.0\ }}{ 0.2.0 }}\label{section-2}

\begin{itemize}
\tightlist
\item
  feat: Added fake italic functionality.
\end{itemize}

\subsubsection{\texorpdfstring{\texttt{\ 0.1.0\ }}{ 0.1.0 }}\label{section-3}

\begin{itemize}
\tightlist
\item
  Basic fake bold functionality.
\end{itemize}

\subsection{License}\label{license}

MIT License

This package refers to the following content:

\begin{itemize}
\tightlist
\item
  \href{https://zhuanlan.zhihu.com/p/19686102}{TeX and Chinese Character
  Processing: Fake Bold and Fake Italic}
\item
  Typst issue \href{https://github.com/typst/typst/issues/394}{\#394}
\item
  Typst issue \href{https://github.com/typst/typst/issues/2749}{\#2749}
  (The function \texttt{\ \_skew\ } comes from Enivex’s code.)
\end{itemize}

Thanks to Enter-tainer for the assistance.

\subsubsection{How to add}\label{how-to-add}

Copy this into your project and use the import as \texttt{\ cuti\ }

\begin{verbatim}
#import "@preview/cuti:0.3.0"
\end{verbatim}

\includesvg[width=0.16667in,height=0.16667in]{/assets/icons/16-copy.svg}

Check the docs for
\href{https://typst.app/docs/reference/scripting/\#packages}{more
information on how to import packages} .

\subsubsection{About}\label{about}

\begin{description}
\tightlist
\item[Author s :]
csimide , Enivex , \& Tetragramm
\item[License:]
MIT
\item[Current version:]
0.3.0
\item[Last updated:]
November 25, 2024
\item[First released:]
March 18, 2024
\item[Minimum Typst version:]
0.12.0
\item[Archive size:]
2.37 kB
\href{https://packages.typst.org/preview/cuti-0.3.0.tar.gz}{\pandocbounded{\includesvg[keepaspectratio]{/assets/icons/16-download.svg}}}
\item[Repository:]
\href{https://github.com/csimide/cuti}{GitHub}
\end{description}

\subsubsection{Where to report issues?}\label{where-to-report-issues}

This package is a project of csimide, Enivex, and Tetragramm . Report
issues on \href{https://github.com/csimide/cuti}{their repository} . You
can also try to ask for help with this package on the
\href{https://forum.typst.app}{Forum} .

Please report this package to the Typst team using the
\href{https://typst.app/contact}{contact form} if you believe it is a
safety hazard or infringes upon your rights.

\phantomsection\label{versions}
\subsubsection{Version history}\label{version-history}

\begin{longtable}[]{@{}ll@{}}
\toprule\noalign{}
Version & Release Date \\
\midrule\noalign{}
\endhead
\bottomrule\noalign{}
\endlastfoot
0.3.0 & November 25, 2024 \\
\href{https://typst.app/universe/package/cuti/0.2.1/}{0.2.1} & April 5,
2024 \\
\href{https://typst.app/universe/package/cuti/0.2.0/}{0.2.0} & March 20,
2024 \\
\href{https://typst.app/universe/package/cuti/0.1.0/}{0.1.0} & March 18,
2024 \\
\end{longtable}

Typst GmbH did not create this package and cannot guarantee correct
functionality of this package or compatibility with any version of the
Typst compiler or app.


\section{Package List LaTeX/modern-technique-report.tex}
\title{typst.app/universe/package/modern-technique-report}

\phantomsection\label{banner}
\phantomsection\label{template-thumbnail}
\pandocbounded{\includegraphics[keepaspectratio]{https://packages.typst.org/preview/thumbnails/modern-technique-report-0.1.0-small.webp}}

\section{modern-technique-report}\label{modern-technique-report}

{ 0.1.0 }

A template for creating modern-style technique report in Typst.

\href{/app?template=modern-technique-report&version=0.1.0}{Create
project in app}

\phantomsection\label{readme}
= Modern Technique Report

A template support modern technique report in Typst.

= Usage

\begin{Shaded}
\begin{Highlighting}[]
\NormalTok{\#import "@preview/modern{-}technique{-}report:0.1.0": *}

\NormalTok{\#show: modern{-}technique{-}report.with(}
\NormalTok{  title: [Project Name \textbackslash{} Multiline When too Long],}
\NormalTok{  subtitle: [}
\NormalTok{    *Fourth Edition*, \textbackslash{} by \_H.L. Royden\_ and \_P.M. Fitzpatrick\_}
\NormalTok{  ],}
\NormalTok{  series: [Mathematics Courses \textbackslash{} Framework Series],}
\NormalTok{  author: grid(}
\NormalTok{    align: left + horizon,}
\NormalTok{    columns: 3,}
\NormalTok{    inset: 7pt,}
\NormalTok{    [*Member*], [B. Alice], [qwertyuiop\textbackslash{}@youremail.com],}
\NormalTok{    [], [B. Alice], [qwertyuiop\textbackslash{}@youremail.com],}
\NormalTok{    [], [B. Alice], [qwertyuiop\textbackslash{}@youremail.com],}
\NormalTok{    [*Advisor*], [E. Eric], [qwertyuiop\textbackslash{}@youremail.com],}
\NormalTok{  ),}
\NormalTok{  date: datetime.today().display("[year] {-}{-} [month] {-}{-} [day]"),}
\NormalTok{  background: image("bg.jpg"),}
\NormalTok{  theme\_color: rgb(21, 74, 135),}
\NormalTok{  font: "New Computer Modern",}
\NormalTok{  title\_font: "Noto Sans",}
\NormalTok{)}
\end{Highlighting}
\end{Shaded}

Then a cover page and a content page will be automatically generated.
Template also manipulates the style of headings and some contents.

\href{/app?template=modern-technique-report&version=0.1.0}{Create
project in app}

\subsubsection{How to use}\label{how-to-use}

Click the button above to create a new project using this template in
the Typst app.

You can also use the Typst CLI to start a new project on your computer
using this command:

\begin{verbatim}
typst init @preview/modern-technique-report:0.1.0
\end{verbatim}

\includesvg[width=0.16667in,height=0.16667in]{/assets/icons/16-copy.svg}

\subsubsection{About}\label{about}

\begin{description}
\tightlist
\item[Author :]
aytony
\item[License:]
MIT
\item[Current version:]
0.1.0
\item[Last updated:]
April 16, 2024
\item[First released:]
April 16, 2024
\item[Minimum Typst version:]
0.11.0
\item[Archive size:]
132 kB
\href{https://packages.typst.org/preview/modern-technique-report-0.1.0.tar.gz}{\pandocbounded{\includesvg[keepaspectratio]{/assets/icons/16-download.svg}}}
\item[Categor ies :]
\begin{itemize}
\tightlist
\item[]
\item
  \pandocbounded{\includesvg[keepaspectratio]{/assets/icons/16-layout.svg}}
  \href{https://typst.app/universe/search/?category=layout}{Layout}
\item
  \pandocbounded{\includesvg[keepaspectratio]{/assets/icons/16-speak.svg}}
  \href{https://typst.app/universe/search/?category=report}{Report}
\end{itemize}
\end{description}

\subsubsection{Where to report issues?}\label{where-to-report-issues}

This template is a project of aytony . You can also try to ask for help
with this template on the \href{https://forum.typst.app}{Forum} .

Please report this template to the Typst team using the
\href{https://typst.app/contact}{contact form} if you believe it is a
safety hazard or infringes upon your rights.

\phantomsection\label{versions}
\subsubsection{Version history}\label{version-history}

\begin{longtable}[]{@{}ll@{}}
\toprule\noalign{}
Version & Release Date \\
\midrule\noalign{}
\endhead
\bottomrule\noalign{}
\endlastfoot
0.1.0 & April 16, 2024 \\
\end{longtable}

Typst GmbH did not create this template and cannot guarantee correct
functionality of this template or compatibility with any version of the
Typst compiler or app.


\section{Package List LaTeX/ctxjs.tex}
\title{typst.app/universe/package/ctxjs}

\phantomsection\label{banner}
\section{ctxjs}\label{ctxjs}

{ 0.2.0 }

Run javascript in contexts.

\phantomsection\label{readme}
A typst plugin to evaluate javascript code.

\begin{itemize}
\tightlist
\item
  multiple javascript contexts
\item
  load javascript modules as source or bytecode
\item
  simple evaluations
\item
  formated evaluations (execute your code with your typst data)
\item
  call functions
\item
  call functions in modules
\item
  create bytecode with an extra tool (ctxjs\_module\_bytecode\_builder)
\item
  allow later evaluation of javascript code
\end{itemize}

\subsection{Example}\label{example}

\begin{Shaded}
\begin{Highlighting}[]
\NormalTok{\#import "@preview/ctxjs:0.2.0"}

\NormalTok{\#\{}
\NormalTok{  \_ = ctxjs.create{-}context("context\_name")}
\NormalTok{  let test = ctxjs.eval("context\_name", "function test(data) \{return data + 2;\}")}
\NormalTok{  let returns{-}4 = ctxjs.call{-}function("context\_name", "test", (2,))}
\NormalTok{  let returns{-}6 = ctxjs.eval{-}format("context\_name", "test(\{test\})", (test: 4))}
\NormalTok{  let code = \textasciigrave{}\textasciigrave{}\textasciigrave{}}
\NormalTok{    export function another\_test\_function() \{ return \{data: \textquotesingle{}test\textquotesingle{}\}; \}}
\NormalTok{  \textasciigrave{}\textasciigrave{}\textasciigrave{};}
\NormalTok{  \_ = ctxjs.load{-}module{-}js("context\_name", "module\_name", code.text)}
\NormalTok{  let returns{-}array{-}with{-}another{-}test = ctxjs.get{-}module{-}properties("context\_name", "module\_name")}
\NormalTok{  let returns{-}data{-}with{-}test{-}string = ctxjs.call{-}module{-}function("context\_name", "module\_name", "another\_test\_function", ())}
\NormalTok{  let returns{-}8 = ctxjs.eval{-}format("context\_name", "test(\{test\})", (test: ctxjs.eval{-}later("4 + 4")))}
\NormalTok{\}}
\end{Highlighting}
\end{Shaded}

\subsubsection{How to add}\label{how-to-add}

Copy this into your project and use the import as \texttt{\ ctxjs\ }

\begin{verbatim}
#import "@preview/ctxjs:0.2.0"
\end{verbatim}

\includesvg[width=0.16667in,height=0.16667in]{/assets/icons/16-copy.svg}

Check the docs for
\href{https://typst.app/docs/reference/scripting/\#packages}{more
information on how to import packages} .

\subsubsection{About}\label{about}

\begin{description}
\tightlist
\item[Author :]
lublak
\item[License:]
MIT
\item[Current version:]
0.2.0
\item[Last updated:]
November 28, 2024
\item[First released:]
September 11, 2024
\item[Archive size:]
427 kB
\href{https://packages.typst.org/preview/ctxjs-0.2.0.tar.gz}{\pandocbounded{\includesvg[keepaspectratio]{/assets/icons/16-download.svg}}}
\item[Repository:]
\href{https://github.com/lublak/typst-ctxjs-package}{GitHub}
\end{description}

\subsubsection{Where to report issues?}\label{where-to-report-issues}

This package is a project of lublak . Report issues on
\href{https://github.com/lublak/typst-ctxjs-package}{their repository} .
You can also try to ask for help with this package on the
\href{https://forum.typst.app}{Forum} .

Please report this package to the Typst team using the
\href{https://typst.app/contact}{contact form} if you believe it is a
safety hazard or infringes upon your rights.

\phantomsection\label{versions}
\subsubsection{Version history}\label{version-history}

\begin{longtable}[]{@{}ll@{}}
\toprule\noalign{}
Version & Release Date \\
\midrule\noalign{}
\endhead
\bottomrule\noalign{}
\endlastfoot
0.2.0 & November 28, 2024 \\
\href{https://typst.app/universe/package/ctxjs/0.1.1/}{0.1.1} &
September 30, 2024 \\
\href{https://typst.app/universe/package/ctxjs/0.1.0/}{0.1.0} &
September 11, 2024 \\
\end{longtable}

Typst GmbH did not create this package and cannot guarantee correct
functionality of this package or compatibility with any version of the
Typst compiler or app.


\section{Package List LaTeX/mantys.tex}
\title{typst.app/universe/package/mantys}

\phantomsection\label{banner}
\phantomsection\label{template-thumbnail}
\pandocbounded{\includegraphics[keepaspectratio]{https://packages.typst.org/preview/thumbnails/mantys-0.1.4-small.webp}}

\section{mantys}\label{mantys}

{ 0.1.4 }

Helpers to build manuals for Typst packages.

\href{/app?template=mantys&version=0.1.4}{Create project in app}

\phantomsection\label{readme}
\begin{quote}
\textbf{MAN} uals for \textbf{TY} p \textbf{S} t
\end{quote}

Template for documenting \href{https://github.com/typst/typst}{typst}
packages and templates.

\subsection{Usage}\label{usage}

Just import the package at the beginning of your manual:

\begin{Shaded}
\begin{Highlighting}[]
\NormalTok{\#import "@preview/mantys:0.1.4": *}
\end{Highlighting}
\end{Shaded}

Mantys supports \textbf{Typst 0.11.0} and newer.

\subsection{Writing basics}\label{writing-basics}

A basic template for a manual could look like this:

\begin{Shaded}
\begin{Highlighting}[]
\NormalTok{\#import "@local/mantys:0.1.4": *}

\NormalTok{\#import "your{-}package.typ"}

\NormalTok{\#show: mantys.with(}
\NormalTok{    name:        "your{-}package{-}name",}
\NormalTok{    title:       [A title for the manual],}
\NormalTok{    subtitle:    [A subtitle for the manual],}
\NormalTok{    info:        [A short descriptive text for the package.],}
\NormalTok{    authors: "Your Name",}
\NormalTok{    url:     "https://github.com/repository/url",}
\NormalTok{    version: "0.0.1",}
\NormalTok{    date:        "date{-}of{-}release",}
\NormalTok{    abstract:    [}
\NormalTok{        A few paragraphs of text to describe the package.}
\NormalTok{    ],}

\NormalTok{    example{-}imports: (your{-}package: your{-}package)}
\NormalTok{)}

\NormalTok{// end of preamble}

\NormalTok{\# About}
\NormalTok{\#lorem(50)}

\NormalTok{\# Usage}
\NormalTok{\#lorem(50)}

\NormalTok{\# Available commands}
\NormalTok{\#lorem(50)}
\end{Highlighting}
\end{Shaded}

Use \texttt{\ \#command(name,\ ..args){[}description{]}\ } to describe
commands and \texttt{\ \#argument(name,\ ...){[}description{]}\ } for
arguments:

\begin{Shaded}
\begin{Highlighting}[]
\NormalTok{\#command("headline", arg[color], arg(size:1.8em), sarg[other{-}args], barg[body])[}
\NormalTok{    Renders a prominent headline using \#doc("meta/heading").}

\NormalTok{    \#argument("color", type:"color")[}
\NormalTok{    The color of the headline will be used as the background of a \#doc("layout/block") element containing the headline.}
\NormalTok{  ]}
\NormalTok{  \#argument("size", default:1.8em)[}
\NormalTok{    The text size for the headline.}
\NormalTok{  ]}
\NormalTok{  \#argument("sarg", is{-}sink:true)[}
\NormalTok{    Other options will get passed directly to \#doc("meta/heading").}
\NormalTok{  ]}
\NormalTok{  \#argument("body", type:"content")[}
\NormalTok{    The text for the headline.}
\NormalTok{  ]}

\NormalTok{  The headline is shown as a prominent colored block to highlight important news articles in the newsletter:}

\NormalTok{  \#example[\textasciigrave{}\textasciigrave{}\textasciigrave{}}
\NormalTok{  \#headline(blue, size: 2em, level: 3)[}
\NormalTok{    \#lorem(8)}
\NormalTok{  ]}
\NormalTok{  \textasciigrave{}\textasciigrave{}\textasciigrave{}]}
\NormalTok{]}
\end{Highlighting}
\end{Shaded}

The result might look something like this:

\pandocbounded{\includegraphics[keepaspectratio]{https://github.com/typst/packages/raw/main/packages/preview/mantys/0.1.4/docs/assets/headline-example.png}}

For a full reference of available commands read
\href{https://github.com/typst/packages/raw/main/packages/preview/mantys/0.1.4/docs/mantys-manual.pdf}{the
manual} .

\subsection{Changelog}\label{changelog}

\subsubsection{Version 0.1.4}\label{version-0.1.4}

\begin{itemize}
\tightlist
\item
  Fix missing links in outline (@tingerrr).
\item
  Fixed problem when evaluating default values with Tidy.
\end{itemize}

\subsubsection{Version 0.1.3}\label{version-0.1.3}

\begin{itemize}
\tightlist
\item
  Fix for some datatypes not being displayed properly (thanks to
  @tingerrr).
\item
  Fix for imbalanced outline columns (thanks again to @tingerrr).
\end{itemize}

\subsubsection{Version 0.1.2}\label{version-0.1.2}

\begin{itemize}
\tightlist
\item
  Added \href{https://typst.app/universe/package/hydra}{hydra} for
  better detection of headings in page headers (thanks to @tingerrr for
  the suggestion).
\item
  Fixed problem with multiple quotes around default string values in
  tidy docs.
\item
  Fixed datatypes linking to wrong documentation urls.
\end{itemize}

\subsubsection{Version 0.1.1}\label{version-0.1.1}

\begin{itemize}
\tightlist
\item
  Added template files for submission to \emph{Typst Universe} .
\end{itemize}

\subsubsection{Version 0.1.0}\label{version-0.1.0}

\begin{itemize}
\tightlist
\item
  Refactorings and some style changes
\item
  Updated manual.
\item
  Restructuring of package repository.
\end{itemize}

\subsubsection{Version 0.0.4}\label{version-0.0.4}

\begin{itemize}
\tightlist
\item
  Added integration with \href{https://github.com/Mc-Zen/tidy}{tidy} .
\item
  Fixed issue with types in argument boxes.
\item
  \texttt{\ \#lambda\ } now uses \texttt{\ \#dtype\ }
\end{itemize}

\paragraph{Breaking changes}\label{breaking-changes}

\begin{itemize}
\tightlist
\item
  Adapted \texttt{\ scope\ } argument for \texttt{\ eval\ } in examples.

  \begin{itemize}
  \tightlist
  \item
    \texttt{\ \#example()\ } , \texttt{\ \#side-by-side()\ } and
    \texttt{\ \#shortex()\ } now support the \texttt{\ scope\ } and
    \texttt{\ mode\ } argument.
  \item
    The option \texttt{\ example-imports\ } was replaced by
    \texttt{\ examples-scope\ } .
  \end{itemize}
\end{itemize}

\subsubsection{Version 0.0.3}\label{version-0.0.3}

\begin{itemize}
\item
  It is now possible to load a packages’ \texttt{\ typst.toml\ } file
  directly into \texttt{\ \#mantys\ } :

\begin{Shaded}
\begin{Highlighting}[]
\NormalTok{\#show: mantys.with( ..toml("typst.toml") )}
\end{Highlighting}
\end{Shaded}
\item
  Added some dependencies:

  \begin{itemize}
  \tightlist
  \item
    \href{https://github.com/jneug/typst-tools4typst}{jneug/typst-tools4typst}
    for some common utilities,
  \item
    \href{https://github.com/jneug/typst-codelst}{jneug/typst-codelst}
    for rendering examples and source code,
  \item
    \href{https://github.com/Pablo-Gonzalez-Calderon/showybox-package}{Pablo-Gonzalez-Calderon/showybox-package}
    for adding frames to different areas of a manual (like examples).
  \end{itemize}
\item
  Redesign of some elements:

  \begin{itemize}
  \tightlist
  \item
    Argument display in command descriptions,
  \item
    Alert boxes.
  \end{itemize}
\item
  Added \texttt{\ \#version(since:(),\ until:())\ } command to add
  version markers to commands.
\item
  Styles moved to a separate \texttt{\ theme.typ\ } file to allow easy
  customization of colors and styles.
\item
  Added \texttt{\ \#func()\ } , \texttt{\ \#lambda()\ } and
  \texttt{\ \#symbol()\ } commands, to handle special cases for values.
\item
  Fixes and code improvements.
\end{itemize}

\subsubsection{Version 0.0.2}\label{version-0.0.2}

\begin{itemize}
\tightlist
\item
  Some major updates to the core commands and styles.
\end{itemize}

\subsubsection{Version 0.0.1}\label{version-0.0.1}

\begin{itemize}
\tightlist
\item
  Initial release.
\end{itemize}

\href{/app?template=mantys&version=0.1.4}{Create project in app}

\subsubsection{How to use}\label{how-to-use}

Click the button above to create a new project using this template in
the Typst app.

You can also use the Typst CLI to start a new project on your computer
using this command:

\begin{verbatim}
typst init @preview/mantys:0.1.4
\end{verbatim}

\includesvg[width=0.16667in,height=0.16667in]{/assets/icons/16-copy.svg}

\subsubsection{About}\label{about}

\begin{description}
\tightlist
\item[Author :]
Jonas Neugebauer
\item[License:]
MIT
\item[Current version:]
0.1.4
\item[Last updated:]
May 23, 2024
\item[First released:]
March 21, 2024
\item[Minimum Typst version:]
0.11.0
\item[Archive size:]
19.7 kB
\href{https://packages.typst.org/preview/mantys-0.1.4.tar.gz}{\pandocbounded{\includesvg[keepaspectratio]{/assets/icons/16-download.svg}}}
\item[Repository:]
\href{https://github.com/jneug/typst-mantys}{GitHub}
\item[Categor ies :]
\begin{itemize}
\tightlist
\item[]
\item
  \pandocbounded{\includesvg[keepaspectratio]{/assets/icons/16-layout.svg}}
  \href{https://typst.app/universe/search/?category=layout}{Layout}
\item
  \pandocbounded{\includesvg[keepaspectratio]{/assets/icons/16-list-unordered.svg}}
  \href{https://typst.app/universe/search/?category=model}{Model}
\item
  \pandocbounded{\includesvg[keepaspectratio]{/assets/icons/16-hammer.svg}}
  \href{https://typst.app/universe/search/?category=utility}{Utility}
\end{itemize}
\end{description}

\subsubsection{Where to report issues?}\label{where-to-report-issues}

This template is a project of Jonas Neugebauer . Report issues on
\href{https://github.com/jneug/typst-mantys}{their repository} . You can
also try to ask for help with this template on the
\href{https://forum.typst.app}{Forum} .

Please report this template to the Typst team using the
\href{https://typst.app/contact}{contact form} if you believe it is a
safety hazard or infringes upon your rights.

\phantomsection\label{versions}
\subsubsection{Version history}\label{version-history}

\begin{longtable}[]{@{}ll@{}}
\toprule\noalign{}
Version & Release Date \\
\midrule\noalign{}
\endhead
\bottomrule\noalign{}
\endlastfoot
0.1.4 & May 23, 2024 \\
\href{https://typst.app/universe/package/mantys/0.1.3/}{0.1.3} & April
29, 2024 \\
\href{https://typst.app/universe/package/mantys/0.1.1/}{0.1.1} & March
21, 2024 \\
\end{longtable}

Typst GmbH did not create this template and cannot guarantee correct
functionality of this template or compatibility with any version of the
Typst compiler or app.


\section{Package List LaTeX/modernpro-coverletter.tex}
\title{typst.app/universe/package/modernpro-coverletter}

\phantomsection\label{banner}
\phantomsection\label{template-thumbnail}
\pandocbounded{\includegraphics[keepaspectratio]{https://packages.typst.org/preview/thumbnails/modernpro-coverletter-0.0.5-small.webp}}

\section{modernpro-coverletter}\label{modernpro-coverletter}

{ 0.0.5 }

A cover letter template with modern Sans font for job applications and
other formal letters.

\href{/app?template=modernpro-coverletter&version=0.0.5}{Create project
in app}

\phantomsection\label{readme}
This is a cover letter template for Typst with Sans font. It is a modern
and professional cover letter template. It is easy to use and customize.
This cover letter template is suitable for any job application or
general purpose.

If you want to find a CV template, you can check out
\href{https://github.com/jxpeng98/Typst-CV-Resume/blob/main/README.md}{modernpro-cv}
.

\subsection{How to use}\label{how-to-use}

\subsubsection{Use from the Typst
Universe}\label{use-from-the-typst-universe}

It is simple and easy to use this template from the Typst Universe. If
you prefer to use the local editor and \texttt{\ typst-cli\ } , you can
use the following command to create a new cover letter project with this
template.

\begin{Shaded}
\begin{Highlighting}[]
\ExtensionTok{typst}\NormalTok{ init @preview/modernpro{-}coverletter}
\end{Highlighting}
\end{Shaded}

It will create a new cover letter project with this template in the
current directory.

\subsubsection{Use from GitHub}\label{use-from-github}

You can also use this template from GitHub. You can clone this
repository and use it as a normal project.

\begin{Shaded}
\begin{Highlighting}[]
\FunctionTok{git}\NormalTok{ clone https://github.com/jxpeng98/typst{-}coverletter.git}
\end{Highlighting}
\end{Shaded}

\subsection{Features}\label{features}

This package provides one \textbf{cover letter} template and one
\textbf{statement} template.

\subsubsection{Cover Letter}\label{cover-letter}

\begin{Shaded}
\begin{Highlighting}[]
\NormalTok{\#import "@preview/fontawesome:0.5.0": *}
\NormalTok{\#import "@preview/modernpro{-}coverletter:0.0.5": *}

\NormalTok{\#show: coverletter.with(}
\NormalTok{  font{-}type: "PT Serif",}
\NormalTok{  name: [example],}
\NormalTok{  address: [],}
\NormalTok{  contacts: (}
\NormalTok{    (text: [\#fa{-}icon("location{-}dot") UK]),}
\NormalTok{    (text: [123{-}456{-}789], link: "tel:123{-}456{-}789"),}
\NormalTok{    (text: [example.com], link: "https://www.example.com"),}
\NormalTok{    (text: [github], link: "https://github.com/"),}
\NormalTok{    (text: [example\textbackslash{}@example.com], link: "mailto:example@example.com"),}
\NormalTok{  ),}
\NormalTok{  recipient: (}
\NormalTok{    start{-}title: [],}
\NormalTok{    cl{-}title: [],}
\NormalTok{    date: [],}
\NormalTok{    department: [],}
\NormalTok{    institution: [],}
\NormalTok{    address: [],}
\NormalTok{    postcode: [],}
\NormalTok{  ),}
\NormalTok{)}

\NormalTok{\#set par(justify: true, first{-}line{-}indent: 2em)}
\NormalTok{\#set text(weight: "regular", size: 12pt)}
\end{Highlighting}
\end{Shaded}

\begin{longtable}[]{@{}ll@{}}
\toprule\noalign{}
Parameter & Description \\
\midrule\noalign{}
\endhead
\bottomrule\noalign{}
\endlastfoot
\texttt{\ font-type\ } & The font type of the cover letter, e.g. “PT
Serif� \\
\texttt{\ name\ } & The name of the sender \\
\texttt{\ address\ } & The address of the sender \\
\texttt{\ contacts\ } & The contact information of the
sender(text:{[}{]}, link: {[}{]}) \\
\end{longtable}

\begin{longtable}[]{@{}ll@{}}
\toprule\noalign{}
Parameter in Recipient & Description \\
\midrule\noalign{}
\endhead
\bottomrule\noalign{}
\endlastfoot
\texttt{\ start-title\ } & The start title of the letter \\
\texttt{\ cl-title\ } & The title of the letter (i.g., Job Application
for Hiring Manager) \\
\texttt{\ date\ } & The date of the letter(If “� or {[}{]}, it will
generate the current date) \\
\texttt{\ department\ } & The department of the recipient, can be “�
or {[}{]} \\
\texttt{\ institution\ } & The institution of the recipient \\
\texttt{\ address\ } & The address of the recipient \\
\texttt{\ postcode\ } & The postcode of the recipient \\
\end{longtable}

\subsubsection{Statement}\label{statement}

\begin{Shaded}
\begin{Highlighting}[]
\NormalTok{\#import "@preview/fontawesome:0.5.0": *}
\NormalTok{\#import "@preview/modernpro{-}coverletter:0.0.5": *}

\NormalTok{\#show: statement.with(}
\NormalTok{  font{-}type: "PT Serif",}
\NormalTok{  name: [],}
\NormalTok{  address: [],}
\NormalTok{  contacts: (}
\NormalTok{    (text: [\#fa{-}icon("location{-}dot")]),}
\NormalTok{    (text: [\#fa{-}icon("mobile") 123{-}456{-}789], link: "tel:123{-}456{-}789"),}
\NormalTok{    (text: [\#fa{-}icon("link") example.com], link: "https://www.example.com"),}
\NormalTok{    (text: [\#fa{-}icon("github") github], link: "https://github.com/"),}
\NormalTok{    (text: [\#fa{-}icon("envelope") example\textbackslash{}@example.com], link: "mailto:example@example.com"),}
\NormalTok{  ),}
\NormalTok{)}

\NormalTok{\#v(1em)}
\NormalTok{\#align(center, text(13pt, weight: "semibold")[\#underline([Title])])}
\NormalTok{\#set par(first{-}line{-}indent: 2em, justify: true)}
\NormalTok{\#set text(11pt, weight: "regular")}

\NormalTok{// Main body of the statement}
\end{Highlighting}
\end{Shaded}

\begin{longtable}[]{@{}ll@{}}
\toprule\noalign{}
Parameter & Description \\
\midrule\noalign{}
\endhead
\bottomrule\noalign{}
\endlastfoot
\texttt{\ font-type\ } & The font type of the cover letter, e.g. “PT
Serif� \\
\texttt{\ name\ } & The name of the sender \\
\texttt{\ address\ } & The address of the sender \\
\texttt{\ contacts\ } & The contact information of the
sender(text:{[}{]}, link: {[}{]}) \\
\end{longtable}

\subsubsection{Icons}\label{icons}

The new version also integrates the FontAwesome icons. You can use the
\texttt{\ \#fa-icon("icon")\ } function to insert the icons in the cover
letter or statement template as shown above.

You just need to import the FontAwesome package at the beginning of the
document.

\begin{Shaded}
\begin{Highlighting}[]
\NormalTok{\#import "@preview/fontawesome:0.5.0": *}
\end{Highlighting}
\end{Shaded}

\subsection{Preview}\label{preview}

\subsubsection{Cover Letter}\label{cover-letter-1}

\pandocbounded{\includegraphics[keepaspectratio]{https://minioapi.pjx.ac.cn/img1/2024/08/79decf8975b899d31b9dc76c5466a01a.png}}

\subsubsection{Statement}\label{statement-1}

\pandocbounded{\includegraphics[keepaspectratio]{https://minioapi.pjx.ac.cn/img1/2024/08/0483a06862932e1e9a9f1589676ce862.png}}

\href{/app?template=modernpro-coverletter&version=0.0.5}{Create project
in app}

\subsubsection{How to use}\label{how-to-use-1}

Click the button above to create a new project using this template in
the Typst app.

You can also use the Typst CLI to start a new project on your computer
using this command:

\begin{verbatim}
typst init @preview/modernpro-coverletter:0.0.5
\end{verbatim}

\includesvg[width=0.16667in,height=0.16667in]{/assets/icons/16-copy.svg}

\subsubsection{About}\label{about}

\begin{description}
\tightlist
\item[Author :]
jxpeng98
\item[License:]
MIT
\item[Current version:]
0.0.5
\item[Last updated:]
October 22, 2024
\item[First released:]
April 29, 2024
\item[Archive size:]
2.97 kB
\href{https://packages.typst.org/preview/modernpro-coverletter-0.0.5.tar.gz}{\pandocbounded{\includesvg[keepaspectratio]{/assets/icons/16-download.svg}}}
\item[Repository:]
\href{https://github.com/jxpeng98/typst-coverletter}{GitHub}
\item[Categor ies :]
\begin{itemize}
\tightlist
\item[]
\item
  \pandocbounded{\includesvg[keepaspectratio]{/assets/icons/16-user.svg}}
  \href{https://typst.app/universe/search/?category=cv}{CV}
\item
  \pandocbounded{\includesvg[keepaspectratio]{/assets/icons/16-hammer.svg}}
  \href{https://typst.app/universe/search/?category=utility}{Utility}
\end{itemize}
\end{description}

\subsubsection{Where to report issues?}\label{where-to-report-issues}

This template is a project of jxpeng98 . Report issues on
\href{https://github.com/jxpeng98/typst-coverletter}{their repository} .
You can also try to ask for help with this template on the
\href{https://forum.typst.app}{Forum} .

Please report this template to the Typst team using the
\href{https://typst.app/contact}{contact form} if you believe it is a
safety hazard or infringes upon your rights.

\phantomsection\label{versions}
\subsubsection{Version history}\label{version-history}

\begin{longtable}[]{@{}ll@{}}
\toprule\noalign{}
Version & Release Date \\
\midrule\noalign{}
\endhead
\bottomrule\noalign{}
\endlastfoot
0.0.5 & October 22, 2024 \\
\href{https://typst.app/universe/package/modernpro-coverletter/0.0.4/}{0.0.4}
& September 2, 2024 \\
\href{https://typst.app/universe/package/modernpro-coverletter/0.0.3/}{0.0.3}
& August 14, 2024 \\
\href{https://typst.app/universe/package/modernpro-coverletter/0.0.2/}{0.0.2}
& July 29, 2024 \\
\href{https://typst.app/universe/package/modernpro-coverletter/0.0.1/}{0.0.1}
& April 29, 2024 \\
\end{longtable}

Typst GmbH did not create this template and cannot guarantee correct
functionality of this template or compatibility with any version of the
Typst compiler or app.


\section{Package List LaTeX/wonderous-book.tex}
\title{typst.app/universe/package/wonderous-book}

\phantomsection\label{banner}
\phantomsection\label{template-thumbnail}
\pandocbounded{\includegraphics[keepaspectratio]{https://packages.typst.org/preview/thumbnails/wonderous-book-0.1.1-small.webp}}

\section{wonderous-book}\label{wonderous-book}

{ 0.1.1 }

A fiction book template with running headers and serif typography

\href{/app?template=wonderous-book&version=0.1.1}{Create project in app}

\phantomsection\label{readme}
A book template for fiction. The template contains a title page, a table
of contents, and a chapter template.

Dynamic running headers contain the title of the chapter and the book.

\subsection{Usage}\label{usage}

You can use this template in the Typst web app by clicking “Start from
template� on the dashboard and searching for
\texttt{\ wonderous-book\ } .

Alternatively, you can use the CLI to kick this project off using the
command

\begin{verbatim}
typst init @preview/wonderous-book
\end{verbatim}

Typst will create a new directory with all the files needed to get you
started.

\subsection{Configuration}\label{configuration}

This template exports the \texttt{\ book\ } function with the following
named arguments:

\begin{itemize}
\tightlist
\item
  \texttt{\ title\ } : The book’s title as content.
\item
  \texttt{\ author\ } : Content or an array of content to specify the
  author.
\item
  \texttt{\ paper-size\ } : Defaults to \texttt{\ iso-b5\ } . Specify a
  \href{https://typst.app/docs/reference/layout/page/\#parameters-paper}{paper
  size string} to change the page format.
\item
  \texttt{\ dedication\ } : Who or what this book is dedicated to as
  content or \texttt{\ none\ } . Will appear on its own page.
\item
  \texttt{\ publishing-info\ } : Details for the front matter of this
  book as content or \texttt{\ none\ } .
\end{itemize}

The function also accepts a single, positional argument for the body of
the book.

The template will initialize your package with a sample call to the
\texttt{\ book\ } function in a show rule. If you, however, want to
change an existing project to use this template, you can add a show rule
like this at the top of your file:

\begin{Shaded}
\begin{Highlighting}[]
\NormalTok{\#import "@preview/wonderous{-}book:0.1.1": book}

\NormalTok{\#show: book.with(}
\NormalTok{  title: [Liam\textquotesingle{}s Playlist],}
\NormalTok{  author: "Janet Doe",}
\NormalTok{  dedication: [for Rachel],}
\NormalTok{  publishing{-}info: [}
\NormalTok{    UK Publishing, Inc. \textbackslash{}}
\NormalTok{    6 Abbey Road \textbackslash{}}
\NormalTok{    Vaughnham, 1PX 8A3}

\NormalTok{    \#link("https://example.co.uk/")}

\NormalTok{    971{-}1{-}XXXXXX{-}XX{-}X}
\NormalTok{  ],}
\NormalTok{)}

\NormalTok{// Your content goes below.}
\end{Highlighting}
\end{Shaded}

\href{/app?template=wonderous-book&version=0.1.1}{Create project in app}

\subsubsection{How to use}\label{how-to-use}

Click the button above to create a new project using this template in
the Typst app.

You can also use the Typst CLI to start a new project on your computer
using this command:

\begin{verbatim}
typst init @preview/wonderous-book:0.1.1
\end{verbatim}

\includesvg[width=0.16667in,height=0.16667in]{/assets/icons/16-copy.svg}

\subsubsection{About}\label{about}

\begin{description}
\tightlist
\item[Author :]
\href{https://typst.app}{Typst GmbH}
\item[License:]
MIT-0
\item[Current version:]
0.1.1
\item[Last updated:]
October 29, 2024
\item[First released:]
March 6, 2024
\item[Minimum Typst version:]
0.12.0
\item[Archive size:]
4.06 kB
\href{https://packages.typst.org/preview/wonderous-book-0.1.1.tar.gz}{\pandocbounded{\includesvg[keepaspectratio]{/assets/icons/16-download.svg}}}
\item[Repository:]
\href{https://github.com/typst/templates}{GitHub}
\item[Categor y :]
\begin{itemize}
\tightlist
\item[]
\item
  \pandocbounded{\includesvg[keepaspectratio]{/assets/icons/16-docs.svg}}
  \href{https://typst.app/universe/search/?category=book}{Book}
\end{itemize}
\end{description}

\subsubsection{Where to report issues?}\label{where-to-report-issues}

This template is a project of Typst GmbH . Report issues on
\href{https://github.com/typst/templates}{their repository} . You can
also try to ask for help with this template on the
\href{https://forum.typst.app}{Forum} .

\phantomsection\label{versions}
\subsubsection{Version history}\label{version-history}

\begin{longtable}[]{@{}ll@{}}
\toprule\noalign{}
Version & Release Date \\
\midrule\noalign{}
\endhead
\bottomrule\noalign{}
\endlastfoot
0.1.1 & October 29, 2024 \\
\href{https://typst.app/universe/package/wonderous-book/0.1.0/}{0.1.0} &
March 6, 2024 \\
\end{longtable}


\section{Package List LaTeX/pro-letter.tex}
\title{typst.app/universe/package/pro-letter}

\phantomsection\label{banner}
\phantomsection\label{template-thumbnail}
\pandocbounded{\includegraphics[keepaspectratio]{https://packages.typst.org/preview/thumbnails/pro-letter-0.1.1-small.webp}}

\section{pro-letter}\label{pro-letter}

{ 0.1.1 }

A formal business letter template.

\href{/app?template=pro-letter&version=0.1.1}{Create project in app}

\phantomsection\label{readme}
This Typst template lets you create professional business letters
effortlessly.

\subsection{Features}\label{features}

\begin{itemize}
\tightlist
\item
  \textbf{Flexible Styling} : Adjust fonts, sizes, and emphasis
  according to your preferences.
\item
  \textbf{Customizable Sender and Recipient Details} : Easily configure
  names, addresses, and contact information.
\item
  \textbf{Date and Subject Line} : Clearly define the date and subject
  of your letter.
\item
  \textbf{Attachment Listings} : Describe any attachments that accompany
  your letter.
\item
  \textbf{Notary Section} : Include an optional notary acknowledgment
  page.
\end{itemize}

\subsection{Usage}\label{usage}

To use this template, import it and configure the parameters as shown:

\begin{Shaded}
\begin{Highlighting}[]
\NormalTok{\#import "@preview/pro{-}letter:0.1.1": pro{-}letter}

\NormalTok{\#show: pro{-}letter.with(}
\NormalTok{  sender: (}
\NormalTok{    name: "Alexandra Bloom",}
\NormalTok{    street: "123 Blueberry Lane",}
\NormalTok{    city: "Wonderland",}
\NormalTok{    state: "NA",}
\NormalTok{    zip: "56789",}
\NormalTok{    phone: "+1{-}555{-}987{-}6543",}
\NormalTok{    email: "alex@bloomworld.net",}
\NormalTok{  ),}

\NormalTok{  recipient: (}
\NormalTok{    company: "Fantasy Finance Faucets",}
\NormalTok{    attention: "Treasury Team",}
\NormalTok{    street: "456 Dreamscape Ave",}
\NormalTok{    city: "Fabletown",}
\NormalTok{    state: "IM",}
\NormalTok{    zip: "12345",}
\NormalTok{  ),}

\NormalTok{  date: "January 15, 2025",}

\NormalTok{  subject: "Account Closure Request",}

\NormalTok{  signer: "Alexandra Bloom",}

\NormalTok{  attachments: "Fae Council Closure Order.",}
\NormalTok{)}

\NormalTok{I am writing to formally request the closure of the enchanted vault at Fantasy}
\NormalTok{Finance Faucets held in my name, Alexandra Bloom.}

\NormalTok{Attached is the official Fae Council Closure Order for your verification and}
\NormalTok{records.}

\NormalTok{The account is identified by the vault number: *12345FAE*.}

\NormalTok{As the rightful owner, I \_authorize the closure of the aforementioned vault\_ and}
\NormalTok{\_request that all enchanted funds be redirected to the Fae Council Reserve\_.}
\NormalTok{Please find the necessary details for the transfer enclosed.}

\NormalTok{Thank you for your prompt attention to this magical matter.}
\end{Highlighting}
\end{Shaded}

\subsection{Parameters}\label{parameters}

\subsubsection{Address Information}\label{address-information}

Both \texttt{\ sender\ } and \texttt{\ recipient\ } parameters accept
the following optional fields in the form of a dictionary. Include only
the fields necessary for your letter:

\begin{itemize}
\tightlist
\item
  \textbf{\texttt{\ name\ }} : Full name of the person.
\item
  \textbf{\texttt{\ company\ }} : Company name.
\item
  \textbf{\texttt{\ attention\ }} : Department or individual to address
  within the company.
\item
  \textbf{\texttt{\ street\ }} : Street address.
\item
  \textbf{\texttt{\ city\ }} : City.
\item
  \textbf{\texttt{\ state\ }} : State or region.
\item
  \textbf{\texttt{\ zip\ }} : ZIP or postal code.
\item
  \textbf{\texttt{\ phone\ }} : Phone number.
\item
  \textbf{\texttt{\ email\ }} : Email address.
\end{itemize}

\subsubsection{Letter Details}\label{letter-details}

\begin{itemize}
\tightlist
\item
  \textbf{\texttt{\ date\ }} : The date of the letter. Optional;
  defaults to none.
\item
  \textbf{\texttt{\ subject\ }} : The subject line of the letter.
  Optional; defaults to none.
\item
  \textbf{\texttt{\ salutation\ }} : The greeting in the letter.
  Optional; defaults to “To whom it may concern,�.
\item
  \textbf{\texttt{\ closing\ }} : The closing line of the letter.
  Required; defaults to “Sincerely,�.
\item
  \textbf{\texttt{\ signer\ }} : The name of the person signing the
  letter. Required.
\end{itemize}

\subsubsection{Additional Features}\label{additional-features}

\begin{itemize}
\tightlist
\item
  \textbf{\texttt{\ attachments\ }} : Description of any attachments
  accompanying the letter. Optional; defaults to none.
\item
  \textbf{\texttt{\ notary-page\ }} : Boolean flag to include a notary
  public acknowledgment page. Defaults to false.
\end{itemize}

\subsubsection{Text and Style Settings}\label{text-and-style-settings}

\begin{itemize}
\tightlist
\item
  \textbf{\texttt{\ font\ }} : The typeface to use for the letter.
  Defaults to “Libertinus Serif�.
\item
  \textbf{\texttt{\ size\ }} : Font size. Defaults to 11pt.
\item
  \textbf{\texttt{\ weight\ }} : Font weight. Defaults to “light�.
\item
  \textbf{\texttt{\ strong-delta\ }} : Additional weight for bold text.
  Defaults to 300.
\item
  \textbf{\texttt{\ lang\ }} : Language for the document. Defaults to
  “en�.
\end{itemize}

\subsubsection{Page Settings}\label{page-settings}

\begin{itemize}
\tightlist
\item
  \textbf{\texttt{\ paper\ }} : Paper size for the document. Defaults to
  “us-letter�.
\item
  \textbf{\texttt{\ margin\ }} : Margin size around the edges of the
  page. Defaults to “1in�.
\end{itemize}

\subsection{License}\label{license}

This work is licensed under the MIT License.

\href{/app?template=pro-letter&version=0.1.1}{Create project in app}

\subsubsection{How to use}\label{how-to-use}

Click the button above to create a new project using this template in
the Typst app.

You can also use the Typst CLI to start a new project on your computer
using this command:

\begin{verbatim}
typst init @preview/pro-letter:0.1.1
\end{verbatim}

\includesvg[width=0.16667in,height=0.16667in]{/assets/icons/16-copy.svg}

\subsubsection{About}\label{about}

\begin{description}
\tightlist
\item[Author :]
\href{mailto:steve@waits.net}{Stephen Waits}
\item[License:]
MIT
\item[Current version:]
0.1.1
\item[Last updated:]
October 22, 2024
\item[First released:]
October 21, 2024
\item[Archive size:]
4.23 kB
\href{https://packages.typst.org/preview/pro-letter-0.1.1.tar.gz}{\pandocbounded{\includesvg[keepaspectratio]{/assets/icons/16-download.svg}}}
\item[Repository:]
\href{https://github.com/swaits/typst-collection}{GitHub}
\item[Discipline s :]
\begin{itemize}
\tightlist
\item[]
\item
  \href{https://typst.app/universe/search/?discipline=business}{Business}
\item
  \href{https://typst.app/universe/search/?discipline=communication}{Communication}
\end{itemize}
\item[Categor y :]
\begin{itemize}
\tightlist
\item[]
\item
  \pandocbounded{\includesvg[keepaspectratio]{/assets/icons/16-envelope.svg}}
  \href{https://typst.app/universe/search/?category=office}{Office}
\end{itemize}
\end{description}

\subsubsection{Where to report issues?}\label{where-to-report-issues}

This template is a project of Stephen Waits . Report issues on
\href{https://github.com/swaits/typst-collection}{their repository} .
You can also try to ask for help with this template on the
\href{https://forum.typst.app}{Forum} .

Please report this template to the Typst team using the
\href{https://typst.app/contact}{contact form} if you believe it is a
safety hazard or infringes upon your rights.

\phantomsection\label{versions}
\subsubsection{Version history}\label{version-history}

\begin{longtable}[]{@{}ll@{}}
\toprule\noalign{}
Version & Release Date \\
\midrule\noalign{}
\endhead
\bottomrule\noalign{}
\endlastfoot
0.1.1 & October 22, 2024 \\
\href{https://typst.app/universe/package/pro-letter/0.1.0/}{0.1.0} &
October 21, 2024 \\
\end{longtable}

Typst GmbH did not create this template and cannot guarantee correct
functionality of this template or compatibility with any version of the
Typst compiler or app.


\section{Package List LaTeX/modern-report-umfds.tex}
\title{typst.app/universe/package/modern-report-umfds}

\phantomsection\label{banner}
\phantomsection\label{template-thumbnail}
\pandocbounded{\includegraphics[keepaspectratio]{https://packages.typst.org/preview/thumbnails/modern-report-umfds-0.1.1-small.webp}}

\section{modern-report-umfds}\label{modern-report-umfds}

{ 0.1.1 }

A template for writing reports for the Faculty of Sciences of the
University of Montpellier

\href{/app?template=modern-report-umfds&version=0.1.1}{Create project in
app}

\phantomsection\label{readme}
A template for writing reports for the Faculty of Sciences of the
University of Montpellier.

Basic usage:

\begin{Shaded}
\begin{Highlighting}[]
\NormalTok{\#import "@preview/modern{-}report{-}umfds:0.1.1": umfds}

\NormalTok{\#show: umfds.with(}
\NormalTok{  title: [Your report title],}
\NormalTok{  authors: (}
\NormalTok{    "Author 1",}
\NormalTok{    "Author 2",}
\NormalTok{    "Author 3",}
\NormalTok{    "Author 4"}
\NormalTok{  ),}
\NormalTok{  date: datetime.today().display("[day] [month repr:long] [year]"), // or any string}
\NormalTok{  img: image("cover.png"), // optional}
\NormalTok{  abstract: [}
\NormalTok{    Your abstract, optional}
\NormalTok{  ],}
\NormalTok{  bibliography: bibliography("refs.bib", full: true), // optional}
\NormalTok{  lang: "en", // or "fr"}
\NormalTok{)}

\NormalTok{// Your report content}
\end{Highlighting}
\end{Shaded}

\href{/app?template=modern-report-umfds&version=0.1.1}{Create project in
app}

\subsubsection{How to use}\label{how-to-use}

Click the button above to create a new project using this template in
the Typst app.

You can also use the Typst CLI to start a new project on your computer
using this command:

\begin{verbatim}
typst init @preview/modern-report-umfds:0.1.1
\end{verbatim}

\includesvg[width=0.16667in,height=0.16667in]{/assets/icons/16-copy.svg}

\subsubsection{About}\label{about}

\begin{description}
\tightlist
\item[Author s :]
Pablo Laviron \& Sébastien Vial
\item[License:]
MIT-0
\item[Current version:]
0.1.1
\item[Last updated:]
October 14, 2024
\item[First released:]
October 7, 2024
\item[Archive size:]
80.0 kB
\href{https://packages.typst.org/preview/modern-report-umfds-0.1.1.tar.gz}{\pandocbounded{\includesvg[keepaspectratio]{/assets/icons/16-download.svg}}}
\item[Repository:]
\href{https://github.com/UM-nerds/modern-report-umfds}{GitHub}
\item[Categor y :]
\begin{itemize}
\tightlist
\item[]
\item
  \pandocbounded{\includesvg[keepaspectratio]{/assets/icons/16-speak.svg}}
  \href{https://typst.app/universe/search/?category=report}{Report}
\end{itemize}
\end{description}

\subsubsection{Where to report issues?}\label{where-to-report-issues}

This template is a project of Pablo Laviron and Sébastien Vial . Report
issues on \href{https://github.com/UM-nerds/modern-report-umfds}{their
repository} . You can also try to ask for help with this template on the
\href{https://forum.typst.app}{Forum} .

Please report this template to the Typst team using the
\href{https://typst.app/contact}{contact form} if you believe it is a
safety hazard or infringes upon your rights.

\phantomsection\label{versions}
\subsubsection{Version history}\label{version-history}

\begin{longtable}[]{@{}ll@{}}
\toprule\noalign{}
Version & Release Date \\
\midrule\noalign{}
\endhead
\bottomrule\noalign{}
\endlastfoot
0.1.1 & October 14, 2024 \\
\href{https://typst.app/universe/package/modern-report-umfds/0.1.0/}{0.1.0}
& October 7, 2024 \\
\end{longtable}

Typst GmbH did not create this template and cannot guarantee correct
functionality of this template or compatibility with any version of the
Typst compiler or app.


\section{Package List LaTeX/anti-matter.tex}
\title{typst.app/universe/package/anti-matter}

\phantomsection\label{banner}
\section{anti-matter}\label{anti-matter}

{ 0.1.1 }

Simple page numbering of front and back matter.

\phantomsection\label{readme}
This typst packages allows you to simply mark the end and start of your
front matter and back matter to change style and value of your page
number without manually setting and keeping track of inner and outer
page counters.

\subsection{Example}\label{example}

\begin{Shaded}
\begin{Highlighting}[]
\NormalTok{\#import "@preview/anti{-}matter:0.1.1": anti{-}matter, fence, set{-}numbering}

\NormalTok{\#set page("a4", height: auto)}
\NormalTok{\#show heading.where(level: 1): it =\textgreater{} pagebreak(weak: true) + it}

\NormalTok{\#show: anti{-}matter}

\NormalTok{\#set{-}numbering(none)}
\NormalTok{\#align(center)[My Title Page]}
\NormalTok{\#pagebreak()}
\NormalTok{\#set{-}numbering("I")}

\NormalTok{\#include "front{-}matter.typ"}
\NormalTok{\#fence()}

\NormalTok{\#include "chapters.typ"}
\NormalTok{\#fence()}

\NormalTok{\#include "back{-}matter.typ"}
\end{Highlighting}
\end{Shaded}

\pandocbounded{\includegraphics[keepaspectratio]{https://github.com/typst/packages/raw/main/packages/preview/anti-matter/0.1.1/example/example.png}}

\subsection{Features}\label{features}

\begin{itemize}
\tightlist
\item
  Marking the start and end of front/back matter.
\item
  Specifying the numbering styles for each part fo the document
\end{itemize}

\subsection{FAQ}\label{faq}

\begin{enumerate}
\tightlist
\item
  Why are the pages not correctly counted?

  \begin{itemize}
  \tightlist
  \item
    If you are setting your own page header, you must use
    \texttt{\ step\ } , see section II in the
    \href{https://github.com/typst/packages/raw/main/packages/preview/anti-matter/0.1.1/docs/manual.pdf}{manual}
    .
  \end{itemize}
\item
  Why is my outline not displaying the correct numbering?

  \begin{itemize}
  \tightlist
  \item
    If you configure your own \texttt{\ outline.entry\ } , you must use
    \texttt{\ page-number\ } , see section II in the
    \href{https://github.com/typst/packages/raw/main/packages/preview/anti-matter/0.1.1/docs/manual.pdf}{manual}
    .
  \end{itemize}
\item
  Why does my front/inner/back matter numbering start on the wrong page?

  \begin{itemize}
  \tightlist
  \item
    The fences must be on the last page of their respective part, if you
    have a \texttt{\ pagebreak\ } forcing them on the next page it will
    also incorrectly label that page.
  \item
    Otherwise please open an issue with a minimal reproducible example.
  \end{itemize}
\end{enumerate}

\subsection{Etymology}\label{etymology}

The package name \texttt{\ anti-matter\ } was choosen as a word play on
front/back matter.

\subsection{Glossary}\label{glossary}

\begin{itemize}
\tightlist
\item
  \href{https://en.wikipedia.org/wiki/Book_design\#Front_matter}{front
  matter} - The first part of a thesis or book (intro, outline, etc.)
\item
  \href{https://en.wikipedia.org/wiki/Book_design\#Back_matter_(end_matter)}{back
  matter} - The last part of a thesis or book (bibliography, listings,
  acknowledgements, etc.)
\end{itemize}

\subsubsection{How to add}\label{how-to-add}

Copy this into your project and use the import as
\texttt{\ anti-matter\ }

\begin{verbatim}
#import "@preview/anti-matter:0.1.1"
\end{verbatim}

\includesvg[width=0.16667in,height=0.16667in]{/assets/icons/16-copy.svg}

Check the docs for
\href{https://typst.app/docs/reference/scripting/\#packages}{more
information on how to import packages} .

\subsubsection{About}\label{about}

\begin{description}
\tightlist
\item[Author :]
\href{mailto:me@tinger.dev}{tinger}
\item[License:]
MIT
\item[Current version:]
0.1.1
\item[Last updated:]
December 3, 2023
\item[First released:]
September 29, 2023
\item[Minimum Typst version:]
0.8.0
\item[Archive size:]
3.70 kB
\href{https://packages.typst.org/preview/anti-matter-0.1.1.tar.gz}{\pandocbounded{\includesvg[keepaspectratio]{/assets/icons/16-download.svg}}}
\item[Repository:]
\href{https://github.com/tingerrr/anti-matter}{GitHub}
\end{description}

\subsubsection{Where to report issues?}\label{where-to-report-issues}

This package is a project of tinger . Report issues on
\href{https://github.com/tingerrr/anti-matter}{their repository} . You
can also try to ask for help with this package on the
\href{https://forum.typst.app}{Forum} .

Please report this package to the Typst team using the
\href{https://typst.app/contact}{contact form} if you believe it is a
safety hazard or infringes upon your rights.

\phantomsection\label{versions}
\subsubsection{Version history}\label{version-history}

\begin{longtable}[]{@{}ll@{}}
\toprule\noalign{}
Version & Release Date \\
\midrule\noalign{}
\endhead
\bottomrule\noalign{}
\endlastfoot
0.1.1 & December 3, 2023 \\
\href{https://typst.app/universe/package/anti-matter/0.1.0/}{0.1.0} &
November 29, 2023 \\
\href{https://typst.app/universe/package/anti-matter/0.0.2/}{0.0.2} &
October 2, 2023 \\
\href{https://typst.app/universe/package/anti-matter/0.0.1/}{0.0.1} &
September 29, 2023 \\
\end{longtable}

Typst GmbH did not create this package and cannot guarantee correct
functionality of this package or compatibility with any version of the
Typst compiler or app.


\section{Package List LaTeX/pillar.tex}
\title{typst.app/universe/package/pillar}

\phantomsection\label{banner}
\section{pillar}\label{pillar}

{ 0.2.0 }

Faster column specifications for tables.

\phantomsection\label{readme}
\emph{Shorthand notations for table column specifications in
\href{https://typst.app/}{Typst} .}

\href{https://typst.app/universe/package/pillar}{\pandocbounded{\includegraphics[keepaspectratio]{https://img.shields.io/badge/dynamic/toml?url=https\%3A\%2F\%2Fraw.githubusercontent.com\%2FMc-Zen\%2Fpillar\%2Fmain\%2Ftypst.toml&query=\%24.package.version&prefix=v&logo=typst&label=package&color=239DAD}}}
\href{https://github.com/Mc-Zen/pillar/actions/workflows/run_tests.yml}{\pandocbounded{\includesvg[keepaspectratio]{https://github.com/Mc-Zen/pillar/actions/workflows/run_tests.yml/badge.svg}}}
\href{https://github.com/Mc-Zen/pillar/blob/main/LICENSE}{\pandocbounded{\includegraphics[keepaspectratio]{https://img.shields.io/badge/license-MIT-blue}}}

\begin{itemize}
\tightlist
\item
  \href{https://github.com/typst/packages/raw/main/packages/preview/pillar/0.2.0/\#introduction}{Introduction}
\item
  \href{https://github.com/typst/packages/raw/main/packages/preview/pillar/0.2.0/\#column-specification}{Column
  specification}
\item
  \href{https://github.com/typst/packages/raw/main/packages/preview/pillar/0.2.0/\#number-alignment}{Number
  alignment}
\item
  \href{https://github.com/typst/packages/raw/main/packages/preview/pillar/0.2.0/\#pillarcols}{\texttt{\ pillar.cols()\ }}
\item
  \href{https://github.com/typst/packages/raw/main/packages/preview/pillar/0.2.0/\#pillartable}{\texttt{\ pillar.table()\ }}
\item
  \href{https://github.com/typst/packages/raw/main/packages/preview/pillar/0.2.0/\#vline-customization}{\texttt{\ vline\ }
  customization}
\end{itemize}

\subsection{Introduction}\label{introduction}

With \textbf{pillar} , you can significantly simplify the column setup
of tables by unifying the specification of the number, alignment, and
separation of columns. This package is in particular designed for
scientific tables, which typically have simple styling:

\pandocbounded{\includegraphics[keepaspectratio]{https://github.com/user-attachments/assets/c0c60651-c682-4968-9ac9-0fa1e8d85dad}}

In order to prepare this table with just the built-in methods, some code
like the following would be required.

\begin{Shaded}
\begin{Highlighting}[]
\NormalTok{\#table(}
\NormalTok{  columns: 5,}
\NormalTok{  align: (center,) * 4 + (right,),}
\NormalTok{  stroke: none,}


\NormalTok{  [Piano Key], table.vline(), [MIDI Number], [Note Name], [Pitch Name], table.vline(), [$f$ in Hz],}
\NormalTok{  ..}
\NormalTok{)}
\end{Highlighting}
\end{Shaded}

Using \textbf{pillar} , the same can be achieved with

\begin{Shaded}
\begin{Highlighting}[]
\NormalTok{\#table(}
\NormalTok{    ..pillar.cols("c|ccc|r"),}

\NormalTok{    [Piano Key], [MIDI Number], [Note Name], [Pitch Name], [$f$ in Hz], ..}
\NormalTok{)}
\end{Highlighting}
\end{Shaded}

or alternatively

\begin{Shaded}
\begin{Highlighting}[]
\NormalTok{\#pillar.table(}
\NormalTok{    cols: "c|ccc|r",}

\NormalTok{    [Piano Key], [MIDI Number], [Note Name], [Pitch Name], [$f$ in Hz], ..}
\NormalTok{)}
\end{Highlighting}
\end{Shaded}

\textbf{Pillar} is designed for interoperability. It uses the powerful
standard tables and provides generated arguments for \texttt{\ table\ }
’s \texttt{\ columns\ } , \texttt{\ align\ } , \texttt{\ stroke\ } ,
and for the specified vertical lines. This means that all features of
the built-in tables (and also \texttt{\ show\ } and \texttt{\ set\ }
rules) can be applied as usual.

\subsection{Column specification}\label{column-specification}

This package works with \emph{column specification strings} . Each
column is described by its alignment which can be \texttt{\ l\ } (left),
\texttt{\ c\ } (center), \texttt{\ r\ } (right), or \texttt{\ a\ }
(auto).

The width of a column can optionally be specified by appending a
(relative) length, or fraction in square brackets to the alignment
specifier, e.g., \texttt{\ c{[}2cm{]}\ } or \texttt{\ r{[}1fr{]}\ } .

Vertical lines can be added between columns with a
\texttt{\ \textbar{}\ } character. Double lines can be produced with
\texttt{\ \textbar{}\textbar{}\ } (see
\href{https://github.com/typst/packages/raw/main/packages/preview/pillar/0.2.0/\#vline-customization}{\texttt{\ vline\ }
customization} ). The stroke of the vertical line can be changed by
appending anything that is usually allowed as a stroke argument in
square brackets, e.g., \texttt{\ \textbar{}{[}2pt{]}\ } ,
\texttt{\ \textbar{}{[}red{]}\ } or
\texttt{\ \textbar{}{[}(dash:\ \textbackslash{}"dashed\textbackslash{}"){]}\ }
.

A column specification string may contain any number of spaces (e.g., to
improve readability) â€'' all spaces will be ignored.

\emph{If you find yourself writing highly complex column specifications,
consider not using this package and resort to the parameters that the
built-in tables provide. This package is intended for quick and
relatively simple column specifications.}

\subsection{Number alignment}\label{number-alignment}

Choosing capital letters \texttt{\ L\ } , \texttt{\ C\ } ,
\texttt{\ R\ } , or \texttt{\ A\ } instead of lower-case letters
activates number alignment at the decimal separator for a specific
column (similar to the column type “S� of the LaTeX package
\href{https://github.com/josephwright/siunitx}{siunitx} ). This feature
is provided via the Typst package \textbf{Zero} .
\href{https://github.com/Mc-Zen/zero}{Here} you can read up on the
configuration of number formatting.

\begin{Shaded}
\begin{Highlighting}[]
\NormalTok{\#let percm = $"cm"\^{}({-}1)$}

\NormalTok{\#pillar.table(}
\NormalTok{  cols: "l|CCCC",}
\NormalTok{  [], [$Δ ν\_0$ in \#percm], [$B\textquotesingle{}\_ν$ in \#percm], [$B\textquotesingle{}\textquotesingle{}\_ν$ in \#percm], [$D\textquotesingle{}\_ν$ in \#percm],}
\NormalTok{  table.hline(),}
\NormalTok{  [Measurement], [14525.278],   [1.41],    [1.47],    [1.5e{-}5],}
\NormalTok{  [Uncertainty], [2],           [0.3],     [0.3],     [4e{-}6],}
\NormalTok{  [Ref. [2]],    [14525,74856], [1.37316], [1.43777], [5.401e{-}6]}
\NormalTok{)}
\end{Highlighting}
\end{Shaded}

\pandocbounded{\includegraphics[keepaspectratio]{https://github.com/user-attachments/assets/066cd34e-7043-48c7-b067-e3256e942f14}}

Non-number entries (e.g., in the header) are automatically recognized in
some cases and will not be aligned. In ambiguous cases, adding a leading
or trailing space tells Zero not to apply alignment to this cell, e.g.,
\texttt{\ {[}Voltage\ {]}\ } instead of \texttt{\ {[}Voltage{]}\ } .

\subsection{\texorpdfstring{\texttt{\ pillar.cols()\ }}{ pillar.cols() }}\label{pillar.cols}

This function produces an argument list that may contain arguments for
\texttt{\ columns\ } , \texttt{\ align\ } , \texttt{\ stroke\ } , and
\texttt{\ column-gutter\ } as well as instances of
\texttt{\ table.vline()\ } . These arguments are intended to be expanded
with the \texttt{\ ..\ } syntax into the argument list of
\texttt{\ table\ } as shown in the examples.

\subsection{\texorpdfstring{\texttt{\ pillar.table()\ }}{ pillar.table() }}\label{pillar.table}

This is a thin wrapper that behaves just like the built-in
\texttt{\ table\ } , except that it extracts the first positional
argument if it is a string, and runs it through
\texttt{\ pillar.cols()\ } .

\subsection{\texorpdfstring{\texttt{\ vline\ }
customization}{ vline  customization}}\label{vline-customization}

In order to customize the default line setting, just use set rules on
\texttt{\ table.vline\ } , e.g.,

\begin{Shaded}
\begin{Highlighting}[]
\NormalTok{\#set table.vline(stroke: .7pt)}

\NormalTok{\#table(..pillar.cols("c|cc"), ..)}
\end{Highlighting}
\end{Shaded}

Double lines are currently experimental and are realized through column
gutters. They could also be realized through patterns, but this can
produce artifacts with some renderers. As it currently is, double lines
are not supported before the first and after the last column. On the
other hand, with the current method, double lines are styled with set
rules on \texttt{\ table.vline\ } which is nice and not achievable in
the same way with patterns.

\subsection{Examples}\label{examples}

\subsubsection{Double lines}\label{double-lines}

The following example uses a double line for visually separating
repeated table columns.

\begin{Shaded}
\begin{Highlighting}[]
\NormalTok{\#pillar.table(}
\NormalTok{  cols: "ccc ||[.7pt] ccc",}
  
\NormalTok{  ..([\textbackslash{}\#], [$α$ in °], [$β$ in °]) * 2,}
\NormalTok{  table.hline(),}
\NormalTok{  [1], [34.3], [11.1],  [6], [34.0], [12.9],}
\NormalTok{  [2], [34.2], [11.2],  [7], [34.3], [12.8],}
\NormalTok{  [3], [34.6], [11.4],  [8], [33.9], [11.9],}
\NormalTok{  [4], [34.7], [10.3],  [9], [34.4], [11.8],}
\NormalTok{  [5], [34.3], [11.1], [10], [34.4], [11.8],}
\NormalTok{)}
\end{Highlighting}
\end{Shaded}

\pandocbounded{\includegraphics[keepaspectratio]{https://github.com/user-attachments/assets/e05e7bad-61b6-44f9-af34-5e558f338cdc}}

\subsubsection{Further customization}\label{further-customization}

This example shows the codes of the first ten printable ASCII
characters, demonstrating stroke and column width customization.

\begin{Shaded}
\begin{Highlighting}[]
\NormalTok{\#pillar.table(}
\NormalTok{  cols: "ccc|ccc|[1.8pt + blue] l[5cm]",}
  
\NormalTok{  [Dec],[Hex],[Bin],[Symbol], [HTML code], [HTML name], [Description],}
\NormalTok{  table.hline(),}
\NormalTok{  [32], [20], [00100000], [\&\#32;], [],         [SP], [Space],}
\NormalTok{  [33], [21], [00100001], [\&\#33;], [\&excl;],   [!],  [Exclamation mark],}
\NormalTok{  [34], [22], [00100010], [\&\#34;], [\&quot;],   ["],  [Double quotes],}
\NormalTok{  [35], [23], [00100011], [\&\#35;], [\&num;],    [\textbackslash{}\#], [Number sign],}
\NormalTok{  [36], [24], [00100100], [\&\#36;], [\&dollar;], [\textbackslash{}$], [Dollar sign],}
\NormalTok{  [37], [25], [00100101], [\&\#37;], [\&percnt;], [\%],  [Percent sign],}
\NormalTok{  [38], [26], [00100110], [\&\#38;], [\&amp;],    [\&],  [Ampersand],}
\NormalTok{  [39], [27], [00100111], [\&\#39;], [\&apos;],   [\textquotesingle{}],  [Single quote],}
\NormalTok{  [40], [28], [00101000], [\&\#40;], [\&lparen;], [(],  [Opening parenthesis],}
\NormalTok{  [41], [29], [00101001], [\&\#41;], [\&rparen;], [)],  [Closing parenthesis],}
\NormalTok{)}
\end{Highlighting}
\end{Shaded}

\pandocbounded{\includegraphics[keepaspectratio]{https://github.com/user-attachments/assets/9fae998e-033d-4d7e-9344-fe3778bbd9e6}}

\subsection{Tests}\label{tests}

This package uses
\href{https://github.com/tingerrr/typst-test/}{typst-test} for running
\href{https://github.com/typst/packages/raw/main/packages/preview/pillar/0.2.0/tests/}{tests}
.

\subsubsection{How to add}\label{how-to-add}

Copy this into your project and use the import as \texttt{\ pillar\ }

\begin{verbatim}
#import "@preview/pillar:0.2.0"
\end{verbatim}

\includesvg[width=0.16667in,height=0.16667in]{/assets/icons/16-copy.svg}

Check the docs for
\href{https://typst.app/docs/reference/scripting/\#packages}{more
information on how to import packages} .

\subsubsection{About}\label{about}

\begin{description}
\tightlist
\item[Author :]
\href{https://github.com/Mc-Zen}{Mc-Zen}
\item[License:]
MIT
\item[Current version:]
0.2.0
\item[Last updated:]
August 22, 2024
\item[First released:]
May 27, 2024
\item[Minimum Typst version:]
0.11.0
\item[Archive size:]
5.52 kB
\href{https://packages.typst.org/preview/pillar-0.2.0.tar.gz}{\pandocbounded{\includesvg[keepaspectratio]{/assets/icons/16-download.svg}}}
\item[Categor ies :]
\begin{itemize}
\tightlist
\item[]
\item
  \pandocbounded{\includesvg[keepaspectratio]{/assets/icons/16-chart.svg}}
  \href{https://typst.app/universe/search/?category=visualization}{Visualization}
\item
  \pandocbounded{\includesvg[keepaspectratio]{/assets/icons/16-layout.svg}}
  \href{https://typst.app/universe/search/?category=layout}{Layout}
\end{itemize}
\end{description}

\subsubsection{Where to report issues?}\label{where-to-report-issues}

This package is a project of Mc-Zen . You can also try to ask for help
with this package on the \href{https://forum.typst.app}{Forum} .

Please report this package to the Typst team using the
\href{https://typst.app/contact}{contact form} if you believe it is a
safety hazard or infringes upon your rights.

\phantomsection\label{versions}
\subsubsection{Version history}\label{version-history}

\begin{longtable}[]{@{}ll@{}}
\toprule\noalign{}
Version & Release Date \\
\midrule\noalign{}
\endhead
\bottomrule\noalign{}
\endlastfoot
0.2.0 & August 22, 2024 \\
\href{https://typst.app/universe/package/pillar/0.1.0/}{0.1.0} & May 27,
2024 \\
\end{longtable}

Typst GmbH did not create this package and cannot guarantee correct
functionality of this package or compatibility with any version of the
Typst compiler or app.


\section{Package List LaTeX/blindex.tex}
\title{typst.app/universe/package/blindex}

\phantomsection\label{banner}
\section{blindex}\label{blindex}

{ 0.1.0 }

Index-making of Biblical literature citations in Typst

\phantomsection\label{readme}
Blindex ( \texttt{\ blindex:0.1.0\ } ) is a Typst package specifically
designed for the generation of indices of Biblical literature citations
in documents. Target audience includes theologians and authors of
documents that frequently cite biblical literature.

\subsection{Index Sorting Options}\label{index-sorting-options}

The generated indices are gathered and sorted by Biblical literature
books, which can be ordered according to various Biblical literature
book ordering conventions, including:

\begin{itemize}
\tightlist
\item
  \texttt{\ "LXX"\ } â€`` The Septuagint;
\item
  \texttt{\ "Greek-Bible"\ } â€`` Septuagint + New Testament (King
  James);
\item
  \texttt{\ "Hebrew-Tanakh"\ } â€`` The Hebrew (Torah + Neviim +
  Ketuvim);
\item
  \texttt{\ "Hebrew-Bible"\ } â€`` The Hebrew Tanakh + New Testament
  (King James);
\item
  \texttt{\ "Protestant-Bible"\ } â€`` The Protestant Old + New
  Testaments;
\item
  \texttt{\ "Catholic-Bible"\ } â€`` The Catholic Old + New Testaments;
\item
  \texttt{\ "Orthodox-Bible"\ } â€`` The Orthodox Old + New Testaments;
\item
  \texttt{\ "Oecumenic-Bible"\ } â€`` The Jewish Tanakh + Old Testament
  Deuterocanonical + New Testament;
\item
  \texttt{\ "code"\ } â€`` All registered Biblical literature books: All
  Protestant + All Apocripha.
\end{itemize}

\subsection{Biblical Literature
Abbrevations}\label{biblical-literature-abbrevations}

It is common practice among theologians to refer to biblical literature
books by their abbreviations. Practice shows that abbreviation
conventions are language- and tradition- dependent. Therefore,
\texttt{\ blindex\ } usage reflects this fact, while offering a way to
input arbitrary language-tradition abbreviations, in the
\texttt{\ lang.typ\ } source file.

\subsubsection{Language and Traditions
(Variants)}\label{language-and-traditions-variants}

The \texttt{\ blindex\ } implementation generalizes the concept of
\textbf{tradition} (in the context of biblical literature book
abbreviation bookkeeping) as language \textbf{variants} , since the
software can have things such as a “default� of “n-char�
variants.

As of the current release, supported languages include the following few
ones:

\begin{longtable}[]{@{}llll@{}}
\toprule\noalign{}
Language & Variant & Description & Name \\
\midrule\noalign{}
\endhead
\bottomrule\noalign{}
\endlastfoot
English & 3-char & A 3-char abbreviations & \texttt{\ en-3\ } \\
English & Logos & Used in \texttt{\ logos.com\ } &
\texttt{\ en-logos\ } \\
Portuguese (BR) & Protestant & Protestant for Brazil &
\texttt{\ br-pro\ } \\
Portuguese (BR) & Catholic & Catholic for Brazil &
\texttt{\ br-cat\ } \\
\end{longtable}

Additional language-variations can be added to the \texttt{\ lang.typ\ }
source file by the author and/or by pull requests to the
\texttt{\ dev\ } branch of the (UNFORKED!) development repository
\texttt{\ https://github.com/cnaak/blindex.typ\ } .

\subsection{Low-Level Indexing
Command}\label{low-level-indexing-command}

The \texttt{\ blindex\ } library has a low-level, index entry marking
function \texttt{\ \#blindex(abrv,\ lang,\ entry)\ } , whose arguments
are (abbreviation, language, entry), as in:

\begin{Shaded}
\begin{Highlighting}[]
\NormalTok{"the citation..." \#blindex("1Thess", "en", [1.1{-}{-}3]) citation\textquotesingle{}s typesetting...}
\end{Highlighting}
\end{Shaded}

Following the usual index making strategy in Typst, this user
\texttt{\ \#blindex\ } command only adds the index-marking
\texttt{\ \#metadata\ } in the document, without producing any visible
typeset output.

Biblical literature index listings can be generated (typeset) in
arbitrary amounts and locations throughout the document, just by calling
the user \texttt{\ \#mkIndex\ } command:

\begin{Shaded}
\begin{Highlighting}[]
\NormalTok{\#mkIndex()}
\end{Highlighting}
\end{Shaded}

Optional arguments control style and sorting convention parameters, as
exemplified below.

\subsection{Higher-Level Quoting-Indexing
Commands}\label{higher-level-quoting-indexing-commands}

The library also offers higher-level functions to assemble the entire
(i) citation typesetting, (ii) index entry, (iii) citation typesetting,
and (iv) bibliography entrying (with some typesetting (styling)
options), of the passage. Such commands are \texttt{\ \#iQuot(...)\ }
and \texttt{\ \#bQuot(...)\ } , respectively for \textbf{inline} and
\textbf{block} quoting of Biblical literature, with automatic indexing
and bibliography citation. Mandatory arguments are identical for either
command:

\begin{Shaded}
\begin{Highlighting}[]
\NormalTok{paragraph text...}
\NormalTok{\#iQuot(body, abrv, lang, pssg, version, cited)}
\NormalTok{more text...}

\NormalTok{// Displayed block quote of Biblical literature:}
\NormalTok{\#bQuot(body, abrv, lang, pssg, version, cited)}
\end{Highlighting}
\end{Shaded}

In which:

\begin{itemize}
\tightlist
\item
  \texttt{\ body\ } ( \texttt{\ content\ } ) is the quoted biblical
  literature text;
\item
  \texttt{\ abrv\ } ( \texttt{\ string\ } ) is the book abbreviation
  according to the
\item
  \texttt{\ lang\ } ( \texttt{\ string\ } ) language-variant (see
  above);
\item
  \texttt{\ pssg\ } ( \texttt{\ content\ } ) is the quoted text passage
  â€'' usually chapter and verses â€'' as they will appear in the text
  and in the biblical literature index;
\item
  \texttt{\ version\ } ( \texttt{\ string\ } ) is a translation
  identifier, such as \texttt{\ "LXX"\ } , or \texttt{\ "KJV"\ } ; and
\item
  \texttt{\ cited\ } ( \texttt{\ label\ } ) is the corresponding
  bibliography entry label, which can be constructed through:
\end{itemize}

\texttt{\ label("bib-key")\ } , where \texttt{\ bib-key\ } is the
bibliographic entry key, in the bibliography database â€'' whether
\texttt{\ bibTeX\ } or \texttt{\ Hayagriva\ } .

\subsection{Higher-Level Example}\label{higher-level-example}

\begin{Shaded}
\begin{Highlighting}[]
\NormalTok{\#set page(paper: "a7", fill: rgb("\#eec"))}
\NormalTok{\#import "@preview/blindex:0.1.0": *}

\NormalTok{The Septuagint (LXX) starts with \#iQuot([ΕΝ ἀρχῇ ἐποίησεν ὁ Θεὸς τὸν οὐρανὸν καὶ τὴν γῆν.],}
\NormalTok{"Gen", "en", [1.1], "LXX", label("2012{-}LXX{-}SBB")).}

\NormalTok{\#pagebreak()}

\NormalTok{Moreover, the book of Odes begins with: \#iQuot([ᾠδὴ Μωυσέως ἐν τῇ ἐξόδῳ], "Ode", "en", [1.0],}
\NormalTok{"LXX", label("2012{-}LXX{-}SBB")).}

\NormalTok{\#pagebreak()}

\NormalTok{= Biblical Citations}
\NormalTok{Books are sorted following the LXX ordering.}

\NormalTok{\#mkIndex(cols: 1, sorting{-}tradition: "LXX")}

\NormalTok{\#pagebreak()}

\NormalTok{\#bibliography("test{-}01{-}readme.yml", title: "References", style: "ieee")}
\end{Highlighting}
\end{Shaded}

The listing of the bibliography file, \texttt{\ test-01-readme.yml\ } ,
as shown in the example, is:

\begin{Shaded}
\begin{Highlighting}[]
\FunctionTok{2012{-}LXX{-}SBB}\KeywordTok{:}
\AttributeTok{  }\FunctionTok{type}\KeywordTok{:}\AttributeTok{ book}
\AttributeTok{  }\FunctionTok{title}\KeywordTok{:}
\AttributeTok{    }\FunctionTok{value}\KeywordTok{:}\AttributeTok{ }\StringTok{"Septuaginta: Edição Acadêmica Capa dura – Edição de luxo"}
\AttributeTok{    }\FunctionTok{sentence{-}case}\KeywordTok{:}\AttributeTok{ }\StringTok{"Septuaginta: edição acadêmica capa dura – edição de luxo"}
\AttributeTok{    }\FunctionTok{short}\KeywordTok{:}\AttributeTok{ Septuaginta}
\AttributeTok{  }\FunctionTok{publisher}\KeywordTok{:}\AttributeTok{ Sociedade Bíblica do Brasil, SBB}
\AttributeTok{  }\FunctionTok{editor}\KeywordTok{:}\AttributeTok{ Rahlfs, Alfred}
\AttributeTok{  }\FunctionTok{affiliated}\KeywordTok{:}
\AttributeTok{    }\KeywordTok{{-}}\AttributeTok{ }\FunctionTok{role}\KeywordTok{:}\AttributeTok{ collaborator}
\AttributeTok{      }\FunctionTok{names}\KeywordTok{:}\AttributeTok{ }\KeywordTok{[}\AttributeTok{ }\StringTok{"Hanhart, Robert"}\KeywordTok{,}\AttributeTok{ }\KeywordTok{]}
\AttributeTok{  }\FunctionTok{pages}\KeywordTok{:}\AttributeTok{ }\DecValTok{2240}
\AttributeTok{  }\FunctionTok{date}\KeywordTok{:}\AttributeTok{ 2012{-}01{-}11}
\AttributeTok{  }\FunctionTok{edition}\KeywordTok{:}\AttributeTok{ }\DecValTok{1}
\AttributeTok{  }\FunctionTok{ISBN}\KeywordTok{:}\AttributeTok{ 978{-}3438052278}
\AttributeTok{  }\FunctionTok{language}\KeywordTok{:}\AttributeTok{ el}
\end{Highlighting}
\end{Shaded}

This example results in a 4-page document like this one:

\pandocbounded{\includegraphics[keepaspectratio]{https://raw.githubusercontent.com/cnaak/blindex.typ/55d275e4fdab1f47c13e1fe01cbb2b397de5e0fb/thumbnail.png}}

\subsection{Citing}\label{citing}

This package can be cited with the following bibliography database
entry:

\begin{Shaded}
\begin{Highlighting}[]
\FunctionTok{blindex{-}package}\KeywordTok{:}
\AttributeTok{  }\FunctionTok{type}\KeywordTok{:}\AttributeTok{ Web}
\AttributeTok{  }\FunctionTok{author}\KeywordTok{:}\AttributeTok{ Naaktgeboren, C.}
\AttributeTok{  }\FunctionTok{title}\KeywordTok{:}
\AttributeTok{    }\FunctionTok{value}\KeywordTok{:}\AttributeTok{ }\StringTok{"Blindex: Index{-}making of Biblical literature citations in Typst"}
\AttributeTok{    }\FunctionTok{short}\KeywordTok{:}\AttributeTok{ }\StringTok{"Blindex: Index{-}making in Typst"}
\AttributeTok{  }\FunctionTok{url}\KeywordTok{:}\AttributeTok{ https://github.com/cnaak/blindex.typ}
\AttributeTok{  }\FunctionTok{version}\KeywordTok{:}\AttributeTok{ }\FloatTok{0.1.0}
\AttributeTok{  }\FunctionTok{date}\KeywordTok{:}\AttributeTok{ 2024{-}08}
\end{Highlighting}
\end{Shaded}

\subsubsection{How to add}\label{how-to-add}

Copy this into your project and use the import as \texttt{\ blindex\ }

\begin{verbatim}
#import "@preview/blindex:0.1.0"
\end{verbatim}

\includesvg[width=0.16667in,height=0.16667in]{/assets/icons/16-copy.svg}

Check the docs for
\href{https://typst.app/docs/reference/scripting/\#packages}{more
information on how to import packages} .

\subsubsection{About}\label{about}

\begin{description}
\tightlist
\item[Author :]
Naaktgeboren, C.
\item[License:]
MIT
\item[Current version:]
0.1.0
\item[Last updated:]
August 14, 2024
\item[First released:]
August 14, 2024
\item[Minimum Typst version:]
0.11.1
\item[Archive size:]
11.1 kB
\href{https://packages.typst.org/preview/blindex-0.1.0.tar.gz}{\pandocbounded{\includesvg[keepaspectratio]{/assets/icons/16-download.svg}}}
\item[Discipline :]
\begin{itemize}
\tightlist
\item[]
\item
  \href{https://typst.app/universe/search/?discipline=theology}{Theology}
\end{itemize}
\item[Categor ies :]
\begin{itemize}
\tightlist
\item[]
\item
  \pandocbounded{\includesvg[keepaspectratio]{/assets/icons/16-list-unordered.svg}}
  \href{https://typst.app/universe/search/?category=model}{Model}
\item
  \pandocbounded{\includesvg[keepaspectratio]{/assets/icons/16-code.svg}}
  \href{https://typst.app/universe/search/?category=scripting}{Scripting}
\end{itemize}
\end{description}

\subsubsection{Where to report issues?}\label{where-to-report-issues}

This package is a project of Naaktgeboren, C. . You can also try to ask
for help with this package on the \href{https://forum.typst.app}{Forum}
.

Please report this package to the Typst team using the
\href{https://typst.app/contact}{contact form} if you believe it is a
safety hazard or infringes upon your rights.

\phantomsection\label{versions}
\subsubsection{Version history}\label{version-history}

\begin{longtable}[]{@{}ll@{}}
\toprule\noalign{}
Version & Release Date \\
\midrule\noalign{}
\endhead
\bottomrule\noalign{}
\endlastfoot
0.1.0 & August 14, 2024 \\
\end{longtable}

Typst GmbH did not create this package and cannot guarantee correct
functionality of this package or compatibility with any version of the
Typst compiler or app.


\section{Package List LaTeX/sourcerer.tex}
\title{typst.app/universe/package/sourcerer}

\phantomsection\label{banner}
\section{sourcerer}\label{sourcerer}

{ 0.2.1 }

Customizable and flexible source-code blocks

\phantomsection\label{readme}
Sourcerer is a Typst package for displaying stylized source code blocks,
with some extra features. Main features include:

\begin{itemize}
\tightlist
\item
  Rendering source code with numbering
\item
  Rendering only a range of lines from the source code, keeping the
  original highlighting of the code (For example, block comments are
  still rendered well, even if cut)
\item
  Adding in-code line labels which are easily referenceable (via
  \texttt{\ reference\ } )
\item
  Considerable customization options for the display of the code block
\item
  Consistent and pretty cutoff between pages
\item
  Displaying the language used for a code block in a readable manner,
  in-code-block
\end{itemize}

First, import the package via:

\begin{Shaded}
\begin{Highlighting}[]
\NormalTok{\#import "@preview/sourcerer:0.2.1": code}
\end{Highlighting}
\end{Shaded}

Then, display custom code blocks via the \texttt{\ code\ } function,
like so:

\begin{Shaded}
\begin{Highlighting}[]
\NormalTok{\#code(}
\NormalTok{  lang: "Typst",}
\NormalTok{  \textasciigrave{}\textasciigrave{}\textasciigrave{}typ}
\NormalTok{  Woah, that\textquotesingle{}s pretty \#smallcaps(cool)!}
\NormalTok{  That\textquotesingle{}s neat too.}
\NormalTok{  \textasciigrave{}\textasciigrave{}\textasciigrave{}}
\NormalTok{)}
\end{Highlighting}
\end{Shaded}

This results in:

\includegraphics[width=7.8125in,height=\textheight,keepaspectratio]{https://github.com/typst/packages/raw/main/packages/preview/sourcerer/0.2.1/assets/sourcerer.png}

To view all of the options of the \texttt{\ code\ } function, consult
the
\href{https://github.com/typst/packages/raw/main/packages/preview/sourcerer/0.2.1/DOCS.md}{documentation}
.

\subsubsection{How to add}\label{how-to-add}

Copy this into your project and use the import as \texttt{\ sourcerer\ }

\begin{verbatim}
#import "@preview/sourcerer:0.2.1"
\end{verbatim}

\includesvg[width=0.16667in,height=0.16667in]{/assets/icons/16-copy.svg}

Check the docs for
\href{https://typst.app/docs/reference/scripting/\#packages}{more
information on how to import packages} .

\subsubsection{About}\label{about}

\begin{description}
\tightlist
\item[Author :]
\href{mailto:miestrode@proton.me}{Yoav Grimland}
\item[License:]
MIT
\item[Current version:]
0.2.1
\item[Last updated:]
November 10, 2023
\item[First released:]
November 6, 2023
\item[Minimum Typst version:]
0.9.0
\item[Archive size:]
3.98 kB
\href{https://packages.typst.org/preview/sourcerer-0.2.1.tar.gz}{\pandocbounded{\includesvg[keepaspectratio]{/assets/icons/16-download.svg}}}
\item[Repository:]
\href{https://github.com/miestrode/sourcerer}{GitHub}
\end{description}

\subsubsection{Where to report issues?}\label{where-to-report-issues}

This package is a project of Yoav Grimland . Report issues on
\href{https://github.com/miestrode/sourcerer}{their repository} . You
can also try to ask for help with this package on the
\href{https://forum.typst.app}{Forum} .

Please report this package to the Typst team using the
\href{https://typst.app/contact}{contact form} if you believe it is a
safety hazard or infringes upon your rights.

\phantomsection\label{versions}
\subsubsection{Version history}\label{version-history}

\begin{longtable}[]{@{}ll@{}}
\toprule\noalign{}
Version & Release Date \\
\midrule\noalign{}
\endhead
\bottomrule\noalign{}
\endlastfoot
0.2.1 & November 10, 2023 \\
\href{https://typst.app/universe/package/sourcerer/0.2.0/}{0.2.0} &
November 7, 2023 \\
\href{https://typst.app/universe/package/sourcerer/0.1.0/}{0.1.0} &
November 6, 2023 \\
\end{longtable}

Typst GmbH did not create this package and cannot guarantee correct
functionality of this package or compatibility with any version of the
Typst compiler or app.


\section{Package List LaTeX/fuzzy-cnoi-statement.tex}
\title{typst.app/universe/package/fuzzy-cnoi-statement}

\phantomsection\label{banner}
\phantomsection\label{template-thumbnail}
\pandocbounded{\includegraphics[keepaspectratio]{https://packages.typst.org/preview/thumbnails/fuzzy-cnoi-statement-0.1.2-small.webp}}

\section{fuzzy-cnoi-statement}\label{fuzzy-cnoi-statement}

{ 0.1.2 }

A template for CNOI(Olympiad in Informatics in China)-style statements
for competitive programming

\href{/app?template=fuzzy-cnoi-statement&version=0.1.2}{Create project
in app}

\phantomsection\label{readme}
Fuzzy CNOI Statement is a template for CNOI(Olympiad in Informatics in
China)-style statements for competitive programming.

Fuzzy CNOI Statement 是一个 CNOI 题é?¢æŽ'版风æ~¼çš„ Typst
模�。

It is mainly designed to mimic the appearance of official CNOI-style
statements, which are usually generated by
\href{https://gitee.com/mulab/oi_tools}{TUACK} .

å\ldots¶ä¸»è¦?模仿国å†\ldots{} NOI
ç³»åˆ---æ¯''赛官æ--¹é¢˜é?¢çš„å¤--观。这些题é?¢ä¸€èˆ¬ç''±
\href{https://gitee.com/mulab/oi_tools}{TUACK} ç''Ÿæˆ?。

This template is not affiliated with the China Computer Federation (CCF)
or the NOI Committee. When using this template, it is recommended to
indicate the unofficial nature of the contest to avoid
misunderstandings.

该模æ?¿ä¸Žä¸­å›½è®¡ç®---机学会(CCF)ã€?NOI
å§''å`˜ä¼šå®˜æ--¹æ---~å\ldots³ã€‚在使ç''¨è¯¥æ¨¡æ?¿æ---¶ï¼Œå»ºè®®æ~‡æ˜Žæ¯''赛的é?žå®˜æ--¹æ€§è´¨ï¼Œä»¥å\ldots?é€~æˆ?误解。

\subsection{Usage}\label{usage}

Here are the fonts that this template will use, you can change the font
by passing parameters:\\
以下是该模æ?¿ä¼šç''¨åˆ°çš„å­---ä½``,ä½~å?¯ä»¥é€šè¿‡ä¼~å\ldots¥å?‚æ•°çš„æ--¹å¼?æ›´æ?¢å­---ä½``:

\begin{itemize}
\tightlist
\item
  Consolas
\item
  New Computer Modern
\item
  æ--¹æ­£ä¹¦å®‹ï¼ˆFZShuSong-Z01S)
\item
  æ--¹æ­£é»`ä½``(FZHei-B01S)
\item
  æ--¹æ­£ä»¿å®‹ï¼ˆFZFangSong-Z02S)
\item
  æ--¹æ­£æ¥·ä½``(FZKai-Z03S)
\end{itemize}

\begin{Shaded}
\begin{Highlighting}[]
\NormalTok{// Define your contest information and problem list}
\NormalTok{// 定义比赛信息和题目列表}

\NormalTok{\#let (init, title, problem{-}table, next{-}problem, filename, current{-}filename, current{-}sample{-}filename, data{-}constraints{-}table{-}args) = document{-}class(}
\NormalTok{  contest{-}info,}
\NormalTok{  prob{-}list,}
\NormalTok{)}

\NormalTok{\#show: init}

\NormalTok{\#title()}

\NormalTok{\#problem{-}table()}

\NormalTok{*注意事项(请仔细阅读)*}
\NormalTok{+ ...}

\NormalTok{\#next{-}problem()}
\NormalTok{== 题目描述}
\NormalTok{...}
\end{Highlighting}
\end{Shaded}

Refer to \texttt{\ main.typ\ } for a complete example.

\texttt{\ main.typ\ } æ??供了一个完整的示例。

\href{/app?template=fuzzy-cnoi-statement&version=0.1.2}{Create project
in app}

\subsubsection{How to use}\label{how-to-use}

Click the button above to create a new project using this template in
the Typst app.

You can also use the Typst CLI to start a new project on your computer
using this command:

\begin{verbatim}
typst init @preview/fuzzy-cnoi-statement:0.1.2
\end{verbatim}

\includesvg[width=0.16667in,height=0.16667in]{/assets/icons/16-copy.svg}

\subsubsection{About}\label{about}

\begin{description}
\tightlist
\item[Author :]
Wallbreaker5th
\item[License:]
MIT-0
\item[Current version:]
0.1.2
\item[Last updated:]
November 12, 2024
\item[First released:]
March 19, 2024
\item[Archive size:]
19.1 kB
\href{https://packages.typst.org/preview/fuzzy-cnoi-statement-0.1.2.tar.gz}{\pandocbounded{\includesvg[keepaspectratio]{/assets/icons/16-download.svg}}}
\item[Repository:]
\href{https://github.com/Wallbreaker5th/fuzzy-cnoi-statement}{GitHub}
\item[Discipline s :]
\begin{itemize}
\tightlist
\item[]
\item
  \href{https://typst.app/universe/search/?discipline=computer-science}{Computer
  Science}
\item
  \href{https://typst.app/universe/search/?discipline=education}{Education}
\end{itemize}
\item[Categor y :]
\begin{itemize}
\tightlist
\item[]
\item
  \pandocbounded{\includesvg[keepaspectratio]{/assets/icons/16-envelope.svg}}
  \href{https://typst.app/universe/search/?category=office}{Office}
\end{itemize}
\end{description}

\subsubsection{Where to report issues?}\label{where-to-report-issues}

This template is a project of Wallbreaker5th . Report issues on
\href{https://github.com/Wallbreaker5th/fuzzy-cnoi-statement}{their
repository} . You can also try to ask for help with this template on the
\href{https://forum.typst.app}{Forum} .

Please report this template to the Typst team using the
\href{https://typst.app/contact}{contact form} if you believe it is a
safety hazard or infringes upon your rights.

\phantomsection\label{versions}
\subsubsection{Version history}\label{version-history}

\begin{longtable}[]{@{}ll@{}}
\toprule\noalign{}
Version & Release Date \\
\midrule\noalign{}
\endhead
\bottomrule\noalign{}
\endlastfoot
0.1.2 & November 12, 2024 \\
\href{https://typst.app/universe/package/fuzzy-cnoi-statement/0.1.1/}{0.1.1}
& March 19, 2024 \\
\href{https://typst.app/universe/package/fuzzy-cnoi-statement/0.1.0/}{0.1.0}
& March 19, 2024 \\
\end{longtable}

Typst GmbH did not create this template and cannot guarantee correct
functionality of this template or compatibility with any version of the
Typst compiler or app.


\section{Package List LaTeX/songb.tex}
\title{typst.app/universe/package/songb}

\phantomsection\label{banner}
\section{songb}\label{songb}

{ 0.1.0 }

A songbook package, to display chords above the lyrics and show a
number-based index (similar to patacrep)

\phantomsection\label{readme}
Attempt at creating a songbook package to replace
\href{https://github.com/patacrep/patacrep}{patacrep} (which is based on
LaTeX + \href{https://songs.sourceforge.net/}{Songs} ).

\subsection{Quickstart}\label{quickstart}

First, create a \texttt{\ main.typ\ } file, like the following:

\begin{Shaded}
\begin{Highlighting}[]
\NormalTok{\#set page(paper: "a6",margin: (inside: 14mm, outside: 6mm, y: 10mm))}

\NormalTok{\#import "@preview/songb:0.1.0": autobreak, index{-}by{-}letter}

\NormalTok{// helper function, to include you own songs (feel free to customize)}
\NormalTok{\#let song(path) = \{}
\NormalTok{    // WARNING: autobreak is currently broken (does not converge)}
\NormalTok{    // see https://github.com/typst/typst/discussions/4530}
\NormalTok{    autobreak(include path)}
\NormalTok{    v({-}1.19em)}
\NormalTok{\}}

\NormalTok{// indexes (put them wherever you want, or comment them out)}
\NormalTok{= Song Index}
\NormalTok{\#index{-}by{-}letter(\textless{}song\textgreater{})}

\NormalTok{= Singer Index}
\NormalTok{\#index{-}by{-}letter(\textless{}singer\textgreater{})}

\NormalTok{\#pagebreak()}

\NormalTok{// include all you songs, in the right order}
\NormalTok{\#song("./songs/first\_song.typ")}

\NormalTok{\#song("./songs/other\_song.typ")}

\NormalTok{// ...}
\end{Highlighting}
\end{Shaded}

Then, create your song files, like \texttt{\ songs/first\_song.typ\ } :

\begin{Shaded}
\begin{Highlighting}[]
\NormalTok{\#import "@preview/songb:0.1.0": song, chorus, verse, chord}

\NormalTok{\#show: doc =\textgreater{} song(}
\NormalTok{  title: "First Song",}
\NormalTok{  singer: "Sing",}
\NormalTok{  doc,}
\NormalTok{)}

\NormalTok{\#chorus[}
\NormalTok{  \#chord[Am]First line,\#chord[G][ ]of the chorus\textbackslash{}}
\NormalTok{  \#chord[Am]Second line,\#chord[G][ ]of the chorus.}
\NormalTok{]}


\NormalTok{\#verse[}
\NormalTok{  \#chord[Em]First verse\textbackslash{}}
\NormalTok{  With multiple\textbackslash{}}
\NormalTok{  \#chord[C]Lines}
\NormalTok{]}

\NormalTok{If there is \#chord[D][a] bridge\textbackslash{}}
\NormalTok{you can write it directly}
\end{Highlighting}
\end{Shaded}

\subsection{Writing a song}\label{writing-a-song}

\subsubsection{song}\label{song}

\begin{Shaded}
\begin{Highlighting}[]
\NormalTok{\#let song(}
\NormalTok{  title: none,}
\NormalTok{  title{-}index: none,}
\NormalTok{  singer: none,}
\NormalTok{  singer{-}index: none,}
\NormalTok{  references: (),}
\NormalTok{  line{-}color: rgb(0xd0, 0xd0, 0xd0),}
\NormalTok{  header{-}display: (number, title, singer) =\textgreater{} (...),}
\NormalTok{  doc}
\NormalTok{)}
\end{Highlighting}
\end{Shaded}

\subsubsection{chord}\label{chord}

\begin{Shaded}
\begin{Highlighting}[]
\NormalTok{// first argument: chord name}
\NormalTok{// optional second argument: text below the chord (useful for whitespace for instance)}
\NormalTok{\#let chord(..content)}
\end{Highlighting}
\end{Shaded}

\subsubsection{verse}\label{verse}

\begin{Shaded}
\begin{Highlighting}[]
\NormalTok{\#let verse(body)}
\end{Highlighting}
\end{Shaded}

\subsubsection{chorus}\label{chorus}

\begin{Shaded}
\begin{Highlighting}[]
\NormalTok{\#let chorus(body)}
\end{Highlighting}
\end{Shaded}

\subsection{Organizing songs}\label{organizing-songs}

\subsubsection{autobreak}\label{autobreak}

\begin{quote}
{[}!WARNING{]} Currently broken (lack of convergence for bigger
documents) See \url{https://github.com/typst/typst/discussions/4530}
\end{quote}

This function aims at putting the content on a single page (or on facing
pages), by introducing pagebreaks when needed.

\begin{Shaded}
\begin{Highlighting}[]
\NormalTok{\#let autobreak(content)}
\end{Highlighting}
\end{Shaded}

\subsubsection{index-by-letter}\label{index-by-letter}

\begin{Shaded}
\begin{Highlighting}[]
\NormalTok{\#let index{-}by{-}letter(label, letter{-}highlight: (letter) =\textgreater{} (...))}
\end{Highlighting}
\end{Shaded}

label: \texttt{\ \textless{}song\textgreater{}\ } or
\texttt{\ \textless{}singer\textgreater{}\ } are provided by the
\texttt{\ song\ } function.

\subsubsection{How to add}\label{how-to-add}

Copy this into your project and use the import as \texttt{\ songb\ }

\begin{verbatim}
#import "@preview/songb:0.1.0"
\end{verbatim}

\includesvg[width=0.16667in,height=0.16667in]{/assets/icons/16-copy.svg}

Check the docs for
\href{https://typst.app/docs/reference/scripting/\#packages}{more
information on how to import packages} .

\subsubsection{About}\label{about}

\begin{description}
\tightlist
\item[Author :]
\href{mailto:git@olivier.pfad.fr}{Oliverpool}
\item[License:]
EUPL-1.2+
\item[Current version:]
0.1.0
\item[Last updated:]
July 25, 2024
\item[First released:]
July 25, 2024
\item[Archive size:]
12.7 kB
\href{https://packages.typst.org/preview/songb-0.1.0.tar.gz}{\pandocbounded{\includesvg[keepaspectratio]{/assets/icons/16-download.svg}}}
\item[Repository:]
\href{https://codeberg.org/pfad.fr/typst-songbook}{Codeberg}
\item[Discipline :]
\begin{itemize}
\tightlist
\item[]
\item
  \href{https://typst.app/universe/search/?discipline=music}{Music}
\end{itemize}
\end{description}

\subsubsection{Where to report issues?}\label{where-to-report-issues}

This package is a project of Oliverpool . Report issues on
\href{https://codeberg.org/pfad.fr/typst-songbook}{their repository} .
You can also try to ask for help with this package on the
\href{https://forum.typst.app}{Forum} .

Please report this package to the Typst team using the
\href{https://typst.app/contact}{contact form} if you believe it is a
safety hazard or infringes upon your rights.

\phantomsection\label{versions}
\subsubsection{Version history}\label{version-history}

\begin{longtable}[]{@{}ll@{}}
\toprule\noalign{}
Version & Release Date \\
\midrule\noalign{}
\endhead
\bottomrule\noalign{}
\endlastfoot
0.1.0 & July 25, 2024 \\
\end{longtable}

Typst GmbH did not create this package and cannot guarantee correct
functionality of this package or compatibility with any version of the
Typst compiler or app.


\section{Package List LaTeX/enunciado-facil-fcfm.tex}
\title{typst.app/universe/package/enunciado-facil-fcfm}

\phantomsection\label{banner}
\phantomsection\label{template-thumbnail}
\pandocbounded{\includegraphics[keepaspectratio]{https://packages.typst.org/preview/thumbnails/enunciado-facil-fcfm-0.1.0-small.webp}}

\section{enunciado-facil-fcfm}\label{enunciado-facil-fcfm}

{ 0.1.0 }

Documentos de ejercicios (controles, auxiliares, tareas, pautas) para la
FCFM, UChile

\href{/app?template=enunciado-facil-fcfm&version=0.1.0}{Create project
in app}

\phantomsection\label{readme}
Template de Typst para documentos de la FCFM (auxiliares, controles,
pautas)

\subsection{Ejemplo de uso}\label{ejemplo-de-uso}

\subsubsection{\texorpdfstring{En
\href{https://typst.app/}{typst.app}}{En typst.app}}\label{en-typst.app}

Si utilizas la aplicación web oficial, puedes presionar “Start from
template� y buscar “enunciado-facil-fcfm� para crear un proyecto
ya inicializado con el template.

\subsubsection{En CLI}\label{en-cli}

Si usas Typst de manera local, puedes ejecutar:

\begin{Shaded}
\begin{Highlighting}[]
\ExtensionTok{typst}\NormalTok{ init @preview/enunciado{-}facil{-}fcfm:0.1.0}
\end{Highlighting}
\end{Shaded}

lo cual inicializará un proyecto usando el template en el directorio
actual.

\subsubsection{Manualmente}\label{manualmente}

Basta crear un archivo con el siguiente contenido para usar el template:

\begin{Shaded}
\begin{Highlighting}[]
\NormalTok{\#import "@preview/enunciado{-}facil{-}fcfm:0.1.0" as template}

\NormalTok{\#show: template.conf.with(}
\NormalTok{  titulo: "Auxiliar 1",}
\NormalTok{  subtitulo: "Typst",}
\NormalTok{  titulo{-}extra: (}
\NormalTok{    [*Profesora*: Ada Lovelace],}
\NormalTok{    [*Auxiliares*: Grace Hopper y Alan Turing],}
\NormalTok{  ),}
\NormalTok{  departamento: template.departamentos.dcc,}
\NormalTok{  curso: "CC4034 {-} Composición de documentos",}
\NormalTok{)}

\NormalTok{...el resto del documento comienza acá}
\end{Highlighting}
\end{Shaded}

Puedes ver un ejemplo más completo en
\href{https://github.com/typst/packages/raw/main/packages/preview/enunciado-facil-fcfm/0.1.0/template/main.typ}{main.typ}
. Para aprender la sintáxis de Typst existe la
\href{https://typst.app/docs}{documentación oficial} . Si vienes desde
LaTeX, te recomiendo la
\href{https://typst.app/docs/guides/guide-for-latex-users/}{guía para
usuarios de LaTeX} .

\subsection{Configuración}\label{configuraciuxe3uxb3n}

La función \texttt{\ conf\ } importada desde el template recibe los
siguientes parámetros:

\begin{longtable}[]{@{}ll@{}}
\toprule\noalign{}
Parámetro & Descripción \\
\midrule\noalign{}
\endhead
\bottomrule\noalign{}
\endlastfoot
\texttt{\ titulo\ } & Título del documento \\
\texttt{\ subtitulo\ } & Subtítulo del documento \\
\texttt{\ titulo-extra\ } & Arreglo con bloques de contenido adicionales
a agregar después del título. Útil para mostrar los nombres del equipo
docente. \\
\texttt{\ departamento\ } & Diccionario que contiene el nombre (
\texttt{\ string\ } ) y el logo del departamento ( \texttt{\ content\ }
). El template viene con uno ya creado para cada departamento bajo
\texttt{\ template.departamentos\ } . Valor por defecto:
\texttt{\ template.departamentos.dcc\ } \\
\texttt{\ curso\ } & Código y/o nombre del curso. \\
\texttt{\ page-conf\ } & Diccionario con parámetros adicionales
(tamaño de página, márgenes, etc) para pasarle a la función
\href{https://typst.app/docs/reference/layout/page/}{page} . \\
\end{longtable}

\subsection{FAQ}\label{faq}

\subsubsection{Cómo cambiar el logo del
departamento}\label{cuxe3uxb3mo-cambiar-el-logo-del-departamento}

El parámetro \texttt{\ departamento\ } solamente es un diccionario de
Typst con las llaves \texttt{\ nombre\ } y \texttt{\ logo\ } . Puedes
crear un diccionario con un logo personalizado y pasárselo al template:

\begin{Shaded}
\begin{Highlighting}[]
\NormalTok{\#import "@preview/enunciado{-}facil{-}fcfm:0.1.0" as template}

\NormalTok{\#let mi{-}departamento = (}
\NormalTok{  nombre: "Mi súper departamento personalizado",}
\NormalTok{  logo: image("mi{-}super{-}logo.png"),}
\NormalTok{)}

\NormalTok{\#show: template.conf.with(}
\NormalTok{  titulo: "Documento con logo personalizado",}
\NormalTok{  departamento: mi{-}departamento,}
\NormalTok{  curso: "CC4034 {-} Composición de documentos",}
\NormalTok{)}
\end{Highlighting}
\end{Shaded}

\subsubsection{Cómo cambiar márgenes, tamaño de página,
etcétera}\label{cuxe3uxb3mo-cambiar-muxe3rgenes-tamauxe3o-de-puxe3gina-etcuxe3tera}

Para cambiar la configuración de la página hay que interceptar la
\href{https://typst.app/docs/reference/styling/\#set-rules}{set rule}
que se hace sobre \texttt{\ page\ } . Para ello, el template expone el
parámetro \texttt{\ page-conf\ } que permit sobreescribir la
configuración de página del template. Por ejemplo, para cambiar el
tamaño del papel a A4:

\begin{Shaded}
\begin{Highlighting}[]
\NormalTok{\#import "@preview/enunciado{-}facil{-}fcfm:0.1.0" as template}

\NormalTok{\#show: template.conf.with(}
\NormalTok{  titulo: "Documento con tamaño A4",}
\NormalTok{  departamento: template.departamentos.dcc,}
\NormalTok{  curso: "CC4034 {-} Composición de documentos",}
\NormalTok{  page{-}conf: (paper: "a4")}
\NormalTok{)}
\end{Highlighting}
\end{Shaded}

\subsubsection{Cómo cambiar la fuente, headings,
etc}\label{cuxe3uxb3mo-cambiar-la-fuente-headings-etc}

Usando \href{https://typst.app/docs/reference/styling/}{show y set
rules} puedes personalizar mucho más el template. Por ejemplo, para
cambiar la fuente:

\begin{Shaded}
\begin{Highlighting}[]
\NormalTok{\#import "@preview/enunciado{-}facil{-}fcfm:0.1.0" as template}

\NormalTok{// En este caso hay que cambiar la fuente}
\NormalTok{// antes de que se configure el template}
\NormalTok{// para que se aplique en el título y encabezado}
\NormalTok{\#set text(font: "New Computer Modern")}

\NormalTok{\#show: template.conf.with(}
\NormalTok{  titulo: "Documento con la fuente de LaTeX",}
\NormalTok{  departamento: template.departamentos.dcc,}
\NormalTok{  curso: "CC4034 {-} Composición de documentos",}
\NormalTok{)}
\end{Highlighting}
\end{Shaded}

\href{/app?template=enunciado-facil-fcfm&version=0.1.0}{Create project
in app}

\subsubsection{How to use}\label{how-to-use}

Click the button above to create a new project using this template in
the Typst app.

You can also use the Typst CLI to start a new project on your computer
using this command:

\begin{verbatim}
typst init @preview/enunciado-facil-fcfm:0.1.0
\end{verbatim}

\includesvg[width=0.16667in,height=0.16667in]{/assets/icons/16-copy.svg}

\subsubsection{About}\label{about}

\begin{description}
\tightlist
\item[Author :]
\href{https://github.com/bkorecic}{Blaz Korecic}
\item[License:]
MIT
\item[Current version:]
0.1.0
\item[Last updated:]
October 9, 2024
\item[First released:]
October 9, 2024
\item[Archive size:]
264 kB
\href{https://packages.typst.org/preview/enunciado-facil-fcfm-0.1.0.tar.gz}{\pandocbounded{\includesvg[keepaspectratio]{/assets/icons/16-download.svg}}}
\item[Repository:]
\href{https://github.com/bkorecic/enunciado-facil-fcfm}{GitHub}
\item[Categor y :]
\begin{itemize}
\tightlist
\item[]
\item
  \pandocbounded{\includesvg[keepaspectratio]{/assets/icons/16-speak.svg}}
  \href{https://typst.app/universe/search/?category=report}{Report}
\end{itemize}
\end{description}

\subsubsection{Where to report issues?}\label{where-to-report-issues}

This template is a project of Blaz Korecic . Report issues on
\href{https://github.com/bkorecic/enunciado-facil-fcfm}{their
repository} . You can also try to ask for help with this template on the
\href{https://forum.typst.app}{Forum} .

Please report this template to the Typst team using the
\href{https://typst.app/contact}{contact form} if you believe it is a
safety hazard or infringes upon your rights.

\phantomsection\label{versions}
\subsubsection{Version history}\label{version-history}

\begin{longtable}[]{@{}ll@{}}
\toprule\noalign{}
Version & Release Date \\
\midrule\noalign{}
\endhead
\bottomrule\noalign{}
\endlastfoot
0.1.0 & October 9, 2024 \\
\end{longtable}

Typst GmbH did not create this template and cannot guarantee correct
functionality of this template or compatibility with any version of the
Typst compiler or app.


\section{Package List LaTeX/clatter.tex}
\title{typst.app/universe/package/clatter}

\phantomsection\label{banner}
\section{clatter}\label{clatter}

{ 0.1.0 }

Just the PDF417 generator from rxing.

\phantomsection\label{readme}
clatter is a simple Typst package for generating PDF417 barcodes,
utilizing the \href{https://github.com/rxing-core/rxing}{rxing} library.

\subsection{Features}\label{features}

\begin{itemize}
\tightlist
\item
  \textbf{Easy to Use} : The package provides a single, intuitive
  function to generate barcodes.
\item
  \textbf{Flexible Sizing} : Control the size of the barcode with
  optional width and height parameters.
\item
  \textbf{Customizable Orientation} : Barcodes can be rendered
  horizontally or vertically, with automatic adjustment based on size.
\end{itemize}

\subsection{Usage}\label{usage}

The primary function provided by this package is \texttt{\ pdf417\ } .

\subsubsection{Parameters}\label{parameters}

\begin{itemize}
\tightlist
\item
  \texttt{\ text\ } (required): The text to encode in the barcode.
\item
  \texttt{\ width\ } (optional): The desired width of the barcode.
\item
  \texttt{\ height\ } (optional): The desired height of the barcode.
\item
  \texttt{\ direction\ } (optional): Sets the orientation of the
  barcode, either \texttt{\ "horizontal"\ } or \texttt{\ "vertical"\ } .
  If not specified, the orientation is automatically determined based on
  the provided dimensions.
\end{itemize}

\subsubsection{Sizing Behavior}\label{sizing-behavior}

\begin{itemize}
\tightlist
\item
  By default, the barcode is rendered horizontally at a reasonable size.
\item
  If both \texttt{\ width\ } and \texttt{\ height\ } are provided, the
  barcode will fit within the specified dimensions (i.e.
  \texttt{\ fit:\ "contain"\ } ).
\item
  If the \texttt{\ height\ } is greater than the \texttt{\ width\ } ,
  the barcode will automatically switch to vertical orientation unless
  \texttt{\ direction\ } is manually set.
\end{itemize}

\subsubsection{Example Usage}\label{example-usage}

\begin{Shaded}
\begin{Highlighting}[]
\NormalTok{\#import "@preview/clatter:0.1.0": pdf417}

\NormalTok{// Generate a sized horizontal PDF417 barcode }
\NormalTok{// Note: The specified size may not be exact, as the barcode will fit within the box, maintaining its aspect ratio.}
\NormalTok{\#pdf417("sized{-}barcode", width: 50mm, height: 20mm)}

\NormalTok{// Generate a vertical barcode}
\NormalTok{\#pdf417("vertical{-}barcode", direction: "vertical")}

\NormalTok{// Generate a barcode and position it on the page}
\NormalTok{\#place(top + right, pdf417("absolutely{-}positioned{-}barcode", width: 50mm), dx: {-}5mm, dy: 5mm)}
\end{Highlighting}
\end{Shaded}

\begin{center}\rule{0.5\linewidth}{0.5pt}\end{center}

{Of course, such a lengthy README can’t be written without the help of
ChatGPT.}

\subsubsection{How to add}\label{how-to-add}

Copy this into your project and use the import as \texttt{\ clatter\ }

\begin{verbatim}
#import "@preview/clatter:0.1.0"
\end{verbatim}

\includesvg[width=0.16667in,height=0.16667in]{/assets/icons/16-copy.svg}

Check the docs for
\href{https://typst.app/docs/reference/scripting/\#packages}{more
information on how to import packages} .

\subsubsection{About}\label{about}

\begin{description}
\tightlist
\item[Author :]
\href{mailto:whygowe@gmail.com}{Hung-I Wang}
\item[License:]
MIT
\item[Current version:]
0.1.0
\item[Last updated:]
August 14, 2024
\item[First released:]
August 14, 2024
\item[Archive size:]
411 kB
\href{https://packages.typst.org/preview/clatter-0.1.0.tar.gz}{\pandocbounded{\includesvg[keepaspectratio]{/assets/icons/16-download.svg}}}
\item[Repository:]
\href{https://github.com/Gowee/typst-clatter}{GitHub}
\end{description}

\subsubsection{Where to report issues?}\label{where-to-report-issues}

This package is a project of Hung-I Wang . Report issues on
\href{https://github.com/Gowee/typst-clatter}{their repository} . You
can also try to ask for help with this package on the
\href{https://forum.typst.app}{Forum} .

Please report this package to the Typst team using the
\href{https://typst.app/contact}{contact form} if you believe it is a
safety hazard or infringes upon your rights.

\phantomsection\label{versions}
\subsubsection{Version history}\label{version-history}

\begin{longtable}[]{@{}ll@{}}
\toprule\noalign{}
Version & Release Date \\
\midrule\noalign{}
\endhead
\bottomrule\noalign{}
\endlastfoot
0.1.0 & August 14, 2024 \\
\end{longtable}

Typst GmbH did not create this package and cannot guarantee correct
functionality of this package or compatibility with any version of the
Typst compiler or app.


\section{Package List LaTeX/touying-unistra-pristine.tex}
\title{typst.app/universe/package/touying-unistra-pristine}

\phantomsection\label{banner}
\phantomsection\label{template-thumbnail}
\pandocbounded{\includegraphics[keepaspectratio]{https://packages.typst.org/preview/thumbnails/touying-unistra-pristine-1.2.0-small.webp}}

\section{touying-unistra-pristine}\label{touying-unistra-pristine}

{ 1.2.0 }

Touying theme adhering to the core principles of the style guide of the
University of Strasbourg, France

\href{/app?template=touying-unistra-pristine&version=1.2.0}{Create
project in app}

\phantomsection\label{readme}
\begin{quote}
{[}!WARNING{]} This theme is \textbf{NOT} affiliated with the University
of Strasbourg. The logo and the fonts are the property of the University
of Strasbourg.
\end{quote}

\textbf{touying-unistra-pristine} is a
\href{https://github.com/touying-typ/touying}{Touying} theme for
creating presentation slides in
\href{https://github.com/typst/typst}{Typst} , adhering to the core
principles of the \href{https://langagevisuel.unistra.fr/}{style guide
of the University of Strasbourg, France} (French). It is an
\textbf{unofficial} theme and it is \textbf{NOT} affiliated with the
University of Strasbourg.

This theme was partly created using components from
\href{https://github.com/typst-tud/tud-slides}{tud-slides} and
\href{https://github.com/piepert/grape-suite}{grape-suite} .

\begin{itemize}
\tightlist
\item
  \textbf{Focus Slides} , with predefined themes and custom colors
  support.
\item
  \textbf{Hero Slides} .
\item
  \textbf{Gallery Slides} .
\item
  \textbf{Admonitions} (with localization and plural support).
\item
  \textbf{Universally Toggleable Header/Footer} (see
  \href{https://github.com/typst/packages/raw/main/packages/preview/touying-unistra-pristine/1.2.0/\#Configuration}{Configuration}
  ).
\item
  Subset of predefined colors taken from the
  \href{https://langagevisuel.unistra.fr/index.php?id=396}{style guide
  of the University of Strasbourg} (see
  \href{https://github.com/typst/packages/raw/main/packages/preview/touying-unistra-pristine/1.2.0/colors.typ}{colors.typ}
  ).
\end{itemize}

See
\href{https://github.com/typst/packages/raw/main/packages/preview/touying-unistra-pristine/1.2.0/example/example.pdf}{example/example.pdf}
for an example PDF output, and
\href{https://github.com/typst/packages/raw/main/packages/preview/touying-unistra-pristine/1.2.0/example/example.typ}{example/example.typ}
for the corresponding Typst file.

These steps assume that you already have
\href{https://typst.app/}{Typst} installed and/or running.

\subsection{Import from Typst
Universe}\label{import-from-typst-universe}

\begin{Shaded}
\begin{Highlighting}[]
\NormalTok{\#import "@preview/touying:0.5.3": *}
\NormalTok{\#import "@preview/touying{-}unistra{-}pristine:1.2.0": *}

\NormalTok{\#show: unistra{-}theme.with(}
\NormalTok{  aspect{-}ratio: "16{-}9",}
\NormalTok{  config{-}info(}
\NormalTok{    title: [Title],}
\NormalTok{    subtitle: [\_Subtitle\_],}
\NormalTok{    author: [Author],}
\NormalTok{    date: datetime.today().display("[month repr:long] [day], [year repr:full]"),}
\NormalTok{  ),}
\NormalTok{)}

\NormalTok{\#title{-}slide[]}

\NormalTok{= Example Section Title}

\NormalTok{== Example Slide}

\NormalTok{A slide with *important information*.}

\NormalTok{\#lorem(50)}
\end{Highlighting}
\end{Shaded}

\subsection{Local installation}\label{local-installation}

\subsubsection{1. Clone the project}\label{clone-the-project}

\texttt{\ git\ clone\ https://github.com/spidersouris/touying-unistra-pristine\ }

\subsubsection{2. Import Touying and
touying-unistra-pristine}\label{import-touying-and-touying-unistra-pristine}

See
\href{https://github.com/typst/packages/raw/main/packages/preview/touying-unistra-pristine/1.2.0/example/example.typ}{example/example.typ}
for a complete example with configuration.

\begin{Shaded}
\begin{Highlighting}[]
\NormalTok{\#import "@preview/touying:0.5.3": *}
\NormalTok{\#import "src/unistra.typ": *}
\NormalTok{\#import "src/colors.typ": *}
\NormalTok{\#import "src/admonition.typ": *}

\NormalTok{\#show: unistra{-}theme.with(}
\NormalTok{  aspect{-}ratio: "16{-}9",}
\NormalTok{  config{-}info(}
\NormalTok{    title: [Title],}
\NormalTok{    subtitle: [\_Subtitle\_],}
\NormalTok{    author: [Author],}
\NormalTok{    date: datetime.today().display("[month repr:long] [day], [year repr:full]"),}
\NormalTok{  ),}
\NormalTok{)}

\NormalTok{\#title{-}slide[]}

\NormalTok{= Example Section Title}

\NormalTok{== Example Slide}

\NormalTok{A slide with *important information*.}

\NormalTok{\#lorem(50)}
\end{Highlighting}
\end{Shaded}

\begin{quote}
{[}!NOTE{]} The default font used by touying-unistra-pristine is
“Unistra A�, a font that can only be downloaded by students and
staff from the University of Strasbourg. If the font is not installed on
your computer, Segoe UI or Roboto will be used as a fallback, in that
specific order. You can change that behavior in the
\href{https://github.com/typst/packages/raw/main/packages/preview/touying-unistra-pristine/1.2.0/\#Configuration}{settings}
.
\end{quote}

The theme can be configured to your liking by adding the
\texttt{\ config-store()\ } object when initializing
\texttt{\ unistra-theme\ } . An example with the \texttt{\ quotes\ }
setting can be found in
\href{https://github.com/typst/packages/raw/main/packages/preview/touying-unistra-pristine/1.2.0/example/example.typ}{example/example.typ}
.

A complete list of settings can be found in the
\texttt{\ config-store\ } object in
\href{https://github.com/typst/packages/raw/main/packages/preview/touying-unistra-pristine/1.2.0/src/unistra.typ}{src/unistra.typ}
.

\href{/app?template=touying-unistra-pristine&version=1.2.0}{Create
project in app}

\subsubsection{How to use}\label{how-to-use}

Click the button above to create a new project using this template in
the Typst app.

You can also use the Typst CLI to start a new project on your computer
using this command:

\begin{verbatim}
typst init @preview/touying-unistra-pristine:1.2.0
\end{verbatim}

\includesvg[width=0.16667in,height=0.16667in]{/assets/icons/16-copy.svg}

\subsubsection{About}\label{about}

\begin{description}
\tightlist
\item[Author :]
\href{https://edoyen.com/}{Enzo Doyen}
\item[License:]
MIT
\item[Current version:]
1.2.0
\item[Last updated:]
November 22, 2024
\item[First released:]
September 11, 2024
\item[Minimum Typst version:]
0.12.0
\item[Archive size:]
19.7 kB
\href{https://packages.typst.org/preview/touying-unistra-pristine-1.2.0.tar.gz}{\pandocbounded{\includesvg[keepaspectratio]{/assets/icons/16-download.svg}}}
\item[Repository:]
\href{https://github.com/spidersouris/touying-unistra-pristine}{GitHub}
\item[Categor y :]
\begin{itemize}
\tightlist
\item[]
\item
  \pandocbounded{\includesvg[keepaspectratio]{/assets/icons/16-presentation.svg}}
  \href{https://typst.app/universe/search/?category=presentation}{Presentation}
\end{itemize}
\end{description}

\subsubsection{Where to report issues?}\label{where-to-report-issues}

This template is a project of Enzo Doyen . Report issues on
\href{https://github.com/spidersouris/touying-unistra-pristine}{their
repository} . You can also try to ask for help with this template on the
\href{https://forum.typst.app}{Forum} .

Please report this template to the Typst team using the
\href{https://typst.app/contact}{contact form} if you believe it is a
safety hazard or infringes upon your rights.

\phantomsection\label{versions}
\subsubsection{Version history}\label{version-history}

\begin{longtable}[]{@{}ll@{}}
\toprule\noalign{}
Version & Release Date \\
\midrule\noalign{}
\endhead
\bottomrule\noalign{}
\endlastfoot
1.2.0 & November 22, 2024 \\
\href{https://typst.app/universe/package/touying-unistra-pristine/1.1.0/}{1.1.0}
& October 17, 2024 \\
\href{https://typst.app/universe/package/touying-unistra-pristine/1.0.0/}{1.0.0}
& September 11, 2024 \\
\end{longtable}

Typst GmbH did not create this template and cannot guarantee correct
functionality of this template or compatibility with any version of the
Typst compiler or app.


\section{Package List LaTeX/numblex.tex}
\title{typst.app/universe/package/numblex}

\phantomsection\label{banner}
\section{numblex}\label{numblex}

{ 0.2.0 }

Numbering helper.

\phantomsection\label{readme}
How to number the heading like this?

\begin{itemize}
\tightlist
\item
  Appendix A. XXXX

  \begin{itemize}
  \tightlist
  \item
    A.1. YYY
  \item
    A.2. ZZZ
  \end{itemize}
\end{itemize}

Or this?

\begin{itemize}
\tightlist
\item
  一��题

  \begin{itemize}
  \tightlist
  \item
    1. 论点

    \begin{itemize}
    \tightlist
    \item
      (1) ��
    \end{itemize}
  \end{itemize}
\end{itemize}

You might use a function:

\begin{Shaded}
\begin{Highlighting}[]
\NormalTok{\#set heading(numbering: (..nums) =\textgreater{} \{}
\NormalTok{  if nums.pos().len() == 1 \{}
\NormalTok{    return "Appendix " + numbering("A.", ..nums)}
\NormalTok{  \}}
\NormalTok{  return numbering("A.1.", ..nums)}
\NormalTok{\}}
\end{Highlighting}
\end{Shaded}

Or set up a couple of \texttt{\ set\ } rules:

\begin{Shaded}
\begin{Highlighting}[]
\NormalTok{\#set heading(numbering: "A.1.")}
\NormalTok{\#show heading.where(level: 1): set heading(numbering: (n) =\textgreater{} "Appendix " + numbering("A.", n))}
\NormalTok{// No, you can\textquotesingle{}t use "Appendix A." since Typst would treat the first "A" as a numbering}
\end{Highlighting}
\end{Shaded}

Or simply use Numblex:

\begin{Shaded}
\begin{Highlighting}[]
\NormalTok{\#import "@preview/numblex:0.2.0": numblex}

\NormalTok{\#set heading(numbering: numblex("\{Appendix [A].:d==1;[A].\}\{[1].\}"))}
\end{Highlighting}
\end{Shaded}

\subsection{Usage}\label{usage}

\begin{Shaded}
\begin{Highlighting}[]
\NormalTok{\#import "@preview/numblex:0.2.0": numblex}

\NormalTok{\#set heading(numbering: numblex("\{Section [A].:d==1;[A].\}\{[1].\}\{[1])\}"))}
\end{Highlighting}
\end{Shaded}

You can read the
\href{https://github.com/ParaN3xus/numblex/blob/main/manual.pdf}{Manual}
for more information.

\subsubsection{How to add}\label{how-to-add}

Copy this into your project and use the import as \texttt{\ numblex\ }

\begin{verbatim}
#import "@preview/numblex:0.2.0"
\end{verbatim}

\includesvg[width=0.16667in,height=0.16667in]{/assets/icons/16-copy.svg}

Check the docs for
\href{https://typst.app/docs/reference/scripting/\#packages}{more
information on how to import packages} .

\subsubsection{About}\label{about}

\begin{description}
\tightlist
\item[Author s :]
\href{https://github.com/ParaN3xus}{ParaN3xus} \&
\href{https://github.com/sjfhsjfh}{sjfhsjfh}
\item[License:]
MIT
\item[Current version:]
0.2.0
\item[Last updated:]
June 24, 2024
\item[First released:]
May 3, 2024
\item[Archive size:]
217 kB
\href{https://packages.typst.org/preview/numblex-0.2.0.tar.gz}{\pandocbounded{\includesvg[keepaspectratio]{/assets/icons/16-download.svg}}}
\item[Repository:]
\href{https://github.com/ParaN3xus/numblex}{GitHub}
\item[Categor y :]
\begin{itemize}
\tightlist
\item[]
\item
  \pandocbounded{\includesvg[keepaspectratio]{/assets/icons/16-hammer.svg}}
  \href{https://typst.app/universe/search/?category=utility}{Utility}
\end{itemize}
\end{description}

\subsubsection{Where to report issues?}\label{where-to-report-issues}

This package is a project of ParaN3xus and sjfhsjfh . Report issues on
\href{https://github.com/ParaN3xus/numblex}{their repository} . You can
also try to ask for help with this package on the
\href{https://forum.typst.app}{Forum} .

Please report this package to the Typst team using the
\href{https://typst.app/contact}{contact form} if you believe it is a
safety hazard or infringes upon your rights.

\phantomsection\label{versions}
\subsubsection{Version history}\label{version-history}

\begin{longtable}[]{@{}ll@{}}
\toprule\noalign{}
Version & Release Date \\
\midrule\noalign{}
\endhead
\bottomrule\noalign{}
\endlastfoot
0.2.0 & June 24, 2024 \\
\href{https://typst.app/universe/package/numblex/0.1.1/}{0.1.1} & May 6,
2024 \\
\href{https://typst.app/universe/package/numblex/0.1.0/}{0.1.0} & May 3,
2024 \\
\end{longtable}

Typst GmbH did not create this package and cannot guarantee correct
functionality of this package or compatibility with any version of the
Typst compiler or app.


\section{Package List LaTeX/basalt-backlinks.tex}
\title{typst.app/universe/package/basalt-backlinks}

\phantomsection\label{banner}
\section{basalt-backlinks}\label{basalt-backlinks}

{ 0.1.0 }

Generate and get backlinks.

\phantomsection\label{readme}
A Typst package for generating and getting backlinks.

\begin{Shaded}
\begin{Highlighting}[]
\NormalTok{\#import "@preview/basalt{-}backlinks:0.1.0" as backlinks}
\NormalTok{\#show link: backlinks.generate}
\end{Highlighting}
\end{Shaded}

\begin{Shaded}
\begin{Highlighting}[]
\NormalTok{Here\textquotesingle{}s some content I want to link to. \textless{}linktome\textgreater{}}

\NormalTok{\#pagebreak()}

\NormalTok{\#link(\textless{}linktome\textgreater{})[I\textquotesingle{}m linking to the content.]}

\NormalTok{\#pagebreak()}

\NormalTok{\#link(\textless{}linktome\textgreater{})[I\textquotesingle{}m also linking to the content!]}

\NormalTok{\#pagebreak()}

\NormalTok{\#context \{}
\NormalTok{  let backs = backlinks.get(\textless{}linktome\textgreater{})}
\NormalTok{  for (i, back) in backs.enumerate() [}
\NormalTok{    \#link(back.location())[}
\NormalTok{      Backlink for \textbackslash{}\textless{}linktome\textgreater{} (\textbackslash{}\#\#i)}
\NormalTok{    ]}

\NormalTok{  ]}
\NormalTok{\}}
\end{Highlighting}
\end{Shaded}

\subsubsection{How to add}\label{how-to-add}

Copy this into your project and use the import as
\texttt{\ basalt-backlinks\ }

\begin{verbatim}
#import "@preview/basalt-backlinks:0.1.0"
\end{verbatim}

\includesvg[width=0.16667in,height=0.16667in]{/assets/icons/16-copy.svg}

Check the docs for
\href{https://typst.app/docs/reference/scripting/\#packages}{more
information on how to import packages} .

\subsubsection{About}\label{about}

\begin{description}
\tightlist
\item[Author :]
Gabriel Talbert Bunt
\item[License:]
MIT
\item[Current version:]
0.1.0
\item[Last updated:]
October 7, 2024
\item[First released:]
October 7, 2024
\item[Archive size:]
1.59 kB
\href{https://packages.typst.org/preview/basalt-backlinks-0.1.0.tar.gz}{\pandocbounded{\includesvg[keepaspectratio]{/assets/icons/16-download.svg}}}
\item[Repository:]
\href{https://github.com/GabrielDTB/basalt-backlinks}{GitHub}
\end{description}

\subsubsection{Where to report issues?}\label{where-to-report-issues}

This package is a project of Gabriel Talbert Bunt . Report issues on
\href{https://github.com/GabrielDTB/basalt-backlinks}{their repository}
. You can also try to ask for help with this package on the
\href{https://forum.typst.app}{Forum} .

Please report this package to the Typst team using the
\href{https://typst.app/contact}{contact form} if you believe it is a
safety hazard or infringes upon your rights.

\phantomsection\label{versions}
\subsubsection{Version history}\label{version-history}

\begin{longtable}[]{@{}ll@{}}
\toprule\noalign{}
Version & Release Date \\
\midrule\noalign{}
\endhead
\bottomrule\noalign{}
\endlastfoot
0.1.0 & October 7, 2024 \\
\end{longtable}

Typst GmbH did not create this package and cannot guarantee correct
functionality of this package or compatibility with any version of the
Typst compiler or app.


\section{Package List LaTeX/unofficial-fhict-document-template.tex}
\title{typst.app/universe/package/unofficial-fhict-document-template}

\phantomsection\label{banner}
\phantomsection\label{template-thumbnail}
\pandocbounded{\includegraphics[keepaspectratio]{https://packages.typst.org/preview/thumbnails/unofficial-fhict-document-template-1.1.1-small.webp}}

\section{unofficial-fhict-document-template}\label{unofficial-fhict-document-template}

{ 1.1.1 }

This is a document template for creating professional-looking documents
with Typst, tailored for FHICT (Fontys Hogeschool ICT).

\href{/app?template=unofficial-fhict-document-template&version=1.1.1}{Create
project in app}

\phantomsection\label{readme}
\pandocbounded{\includegraphics[keepaspectratio]{https://img.shields.io/github/stars/TomVer99/FHICT-typst-template?style=flat-square}}
\pandocbounded{\includegraphics[keepaspectratio]{https://img.shields.io/github/v/release/TomVer99/FHICT-typst-template?style=flat-square}}

\pandocbounded{\includegraphics[keepaspectratio]{https://img.shields.io/maintenance/Yes/2024?style=flat-square}}
\pandocbounded{\includegraphics[keepaspectratio]{https://img.shields.io/github/issues-raw/TomVer99/FHICT-typst-template?label=Issues&style=flat-square}}
\pandocbounded{\includegraphics[keepaspectratio]{https://img.shields.io/github/commits-since/TomVer99/FHICT-typst-template/latest?style=flat-square}}

This is a document template for creating professional-looking documents
with Typst, tailored for FHICT (Fontys Hogeschool ICT).

\subsection{Introduction}\label{introduction}

Creating well-structured and visually appealing documents is crucial in
academic and professional settings. This template is designed to help
FHICT students and faculty produce professional looking documents.

\includegraphics[width=0.49\linewidth,height=\textheight,keepaspectratio]{https://github.com/typst/packages/raw/main/packages/preview/unofficial-fhict-document-template/1.1.1/thumbnail.png}
\includegraphics[width=0.49\linewidth,height=\textheight,keepaspectratio]{https://github.com/typst/packages/raw/main/packages/preview/unofficial-fhict-document-template/1.1.1/showcase-r.png}

\subsection{Features}\label{features}

\begin{itemize}
\tightlist
\item
  Consistent formatting for titles, headings, subheadings, paragraphs
  and other elements.
\item
  Clean and professional document layout.
\item
  FHICT Style.
\item
  Configurable document options.
\item
  Helper functions.
\item
  Multiple languages support (nl, en, de, fr, es).
\end{itemize}

\subsection{Requirements}\label{requirements}

\begin{itemize}
\tightlist
\item
  Roboto font installed on your system.
\item
  Typst builder installed on your system (Explained in
  \texttt{\ Getting\ Started\ } ).
\end{itemize}

\subsection{Getting Started}\label{getting-started}

To get started with this Typst document template, follow these steps:

\begin{enumerate}
\tightlist
\item
  \textbf{Check for the roboto font} : Check if you have the roboto font
  installed on your system. If you don’t, you can download it from
  \href{https://fonts.google.com/specimen/Roboto}{Google Fonts} .
\item
  \textbf{Install Typst} : I recommend to use VSCode with
  \href{https://marketplace.visualstudio.com/items?itemName=myriad-dreamin.tinymist}{Tinymist
  Typst Extension} . You will also need a PDF viewer in VSCode if you
  want to view the document live.
\item
  \textbf{Import the template} : Import the template into your own typst
  document.
  \texttt{\ \#import\ "@preview/unofficial-fhict-document-template:1.1.1":\ *\ }
\item
  \textbf{Set the available options} : Set the available options in the
  template file to your liking.
\item
  \textbf{Start writing} : Start writing your document.
\end{enumerate}

\subsection{Helpful Links / Resources}\label{helpful-links-resources}

\begin{itemize}
\tightlist
\item
  The manual contains a list of all available options and helper
  functions. It can be found
  \href{https://github.com/TomVer99/FHICT-typst-template/blob/main/documentation/manual.pdf}{here}
  or attached to the latest release.
\item
  The \href{https://typst.app/docs/}{Typst Documentation} is a great
  resource for learning how to use Typst.
\item
  The bibliography file is written in
  \href{http://www.bibtex.org/Format/}{BibTeX} . You can use
  \href{https://truben.no/latex/bibtex/}{BibTeX Editor} to easily create
  and edit your bibliography.
\item
  You can use sub files to split your document into multiple files. This
  is especially useful for large documents.
\end{itemize}

\subsection{Contributing}\label{contributing}

I welcome contributions to improve and expand this document template. If
you have ideas, suggestions, or encounter issues, please consider
contributing by creating a pull request or issue.

\subsubsection{Adding a new language}\label{adding-a-new-language}

Currently, the template supports the following languages:
\texttt{\ Dutch\ } \texttt{\ (nl)\ } , \texttt{\ English\ }
\texttt{\ (en)\ } , \texttt{\ German\ } \texttt{\ (de)\ } ,
\texttt{\ French\ } \texttt{\ (fr)\ } , and \texttt{\ Spanish\ }
\texttt{\ (es)\ } . If you want to add a new language, you can do so by
following these steps:

\begin{enumerate}
\tightlist
\item
  Add the language to the \texttt{\ language.yml\ } file in the
  \texttt{\ assets\ } folder. Copy the \texttt{\ en\ } section and
  replace the values with the new language.
\item
  Add a flag \texttt{\ XX-flag.svg\ } to the \texttt{\ assets\ } folder.
\item
  Update the README with the new language.
\item
  Create a pull request with the changes.
\end{enumerate}

\subsection{Disclaimer}\label{disclaimer}

This template / repository is not endorsed by, directly affiliated with,
maintained, authorized or sponsored by Fontys Hogeschool ICT. It is
provided as-is, without any warranty or guarantee of any kind. Use at
your own risk.

The author was/is a student at Fontys Hogeschool ICT and created this
template for personal use. It is shared publicly in the hope that it
will be useful to others.

\href{/app?template=unofficial-fhict-document-template&version=1.1.1}{Create
project in app}

\subsubsection{How to use}\label{how-to-use}

Click the button above to create a new project using this template in
the Typst app.

You can also use the Typst CLI to start a new project on your computer
using this command:

\begin{verbatim}
typst init @preview/unofficial-fhict-document-template:1.1.1
\end{verbatim}

\includesvg[width=0.16667in,height=0.16667in]{/assets/icons/16-copy.svg}

\subsubsection{About}\label{about}

\begin{description}
\tightlist
\item[Author :]
TomVer99
\item[License:]
MIT
\item[Current version:]
1.1.1
\item[Last updated:]
November 12, 2024
\item[First released:]
June 3, 2024
\item[Minimum Typst version:]
0.12.0
\item[Archive size:]
227 kB
\href{https://packages.typst.org/preview/unofficial-fhict-document-template-1.1.1.tar.gz}{\pandocbounded{\includesvg[keepaspectratio]{/assets/icons/16-download.svg}}}
\item[Repository:]
\href{https://github.com/TomVer99/FHICT-typst-template}{GitHub}
\item[Categor ies :]
\begin{itemize}
\tightlist
\item[]
\item
  \pandocbounded{\includesvg[keepaspectratio]{/assets/icons/16-speak.svg}}
  \href{https://typst.app/universe/search/?category=report}{Report}
\item
  \pandocbounded{\includesvg[keepaspectratio]{/assets/icons/16-layout.svg}}
  \href{https://typst.app/universe/search/?category=layout}{Layout}
\item
  \pandocbounded{\includesvg[keepaspectratio]{/assets/icons/16-mortarboard.svg}}
  \href{https://typst.app/universe/search/?category=thesis}{Thesis}
\end{itemize}
\end{description}

\subsubsection{Where to report issues?}\label{where-to-report-issues}

This template is a project of TomVer99 . Report issues on
\href{https://github.com/TomVer99/FHICT-typst-template}{their
repository} . You can also try to ask for help with this template on the
\href{https://forum.typst.app}{Forum} .

Please report this template to the Typst team using the
\href{https://typst.app/contact}{contact form} if you believe it is a
safety hazard or infringes upon your rights.

\phantomsection\label{versions}
\subsubsection{Version history}\label{version-history}

\begin{longtable}[]{@{}ll@{}}
\toprule\noalign{}
Version & Release Date \\
\midrule\noalign{}
\endhead
\bottomrule\noalign{}
\endlastfoot
1.1.1 & November 12, 2024 \\
\href{https://typst.app/universe/package/unofficial-fhict-document-template/1.1.0/}{1.1.0}
& November 6, 2024 \\
\href{https://typst.app/universe/package/unofficial-fhict-document-template/1.0.2/}{1.0.2}
& September 17, 2024 \\
\href{https://typst.app/universe/package/unofficial-fhict-document-template/1.0.1/}{1.0.1}
& September 11, 2024 \\
\href{https://typst.app/universe/package/unofficial-fhict-document-template/1.0.0/}{1.0.0}
& August 19, 2024 \\
\href{https://typst.app/universe/package/unofficial-fhict-document-template/0.11.0/}{0.11.0}
& July 22, 2024 \\
\href{https://typst.app/universe/package/unofficial-fhict-document-template/0.10.1/}{0.10.1}
& June 12, 2024 \\
\href{https://typst.app/universe/package/unofficial-fhict-document-template/0.10.0/}{0.10.0}
& June 3, 2024 \\
\end{longtable}

Typst GmbH did not create this template and cannot guarantee correct
functionality of this template or compatibility with any version of the
Typst compiler or app.


\section{Package List LaTeX/appreciated-letter.tex}
\title{typst.app/universe/package/appreciated-letter}

\phantomsection\label{banner}
\phantomsection\label{template-thumbnail}
\pandocbounded{\includegraphics[keepaspectratio]{https://packages.typst.org/preview/thumbnails/appreciated-letter-0.1.0-small.webp}}

\section{appreciated-letter}\label{appreciated-letter}

{ 0.1.0 }

Correspond with business associates and your friends via mail

\href{/app?template=appreciated-letter&version=0.1.0}{Create project in
app}

\phantomsection\label{readme}
A basic letter with sender and recipient address. The letter is ready
for a DIN DL windowed envelope.

\subsection{Usage}\label{usage}

You can use this template in the Typst web app by clicking “Start from
template� on the dashboard and searching for
\texttt{\ appreciated-letter\ } .

Alternatively, you can use the CLI to kick this project off using the
command

\begin{verbatim}
typst init @preview/appreciated-letter
\end{verbatim}

Typst will create a new directory with all the files needed to get you
started.

\subsection{Configuration}\label{configuration}

This template exports the \texttt{\ letter\ } function with the
following named arguments:

\begin{itemize}
\tightlist
\item
  \texttt{\ sender\ } : The letter’s sender as content. This is
  displayed at the top of the page.
\item
  \texttt{\ recipient\ } : The address of the letter’s recipient as
  content. This is displayed near the top of the page.
\item
  \texttt{\ date\ } : The date, and possibly place, the letter was
  written at as content. Flushed to the right after the address.
\item
  \texttt{\ subject\ } : The subject line for the letter as content.
\item
  \texttt{\ name\ } : The name the letter closes with as content.
\end{itemize}

The function also accepts a single, positional argument for the body of
the letter.

The template will initialize your package with a sample call to the
\texttt{\ letter\ } function in a show rule. If you, however, want to
change an existing project to use this template, you can add a show rule
like this at the top of your file:

\begin{Shaded}
\begin{Highlighting}[]
\NormalTok{\#import "@preview/appreciated{-}letter:0.1.0": letter}

\NormalTok{\#show: letter.with(}
\NormalTok{  sender: [}
\NormalTok{    Jane Smith, Universal Exports, 1 Heavy Plaza, Morristown, NJ 07964}
\NormalTok{  ],}
\NormalTok{  recipient: [}
\NormalTok{    Mr. John Doe \textbackslash{}}
\NormalTok{    Acme Corp. \textbackslash{}}
\NormalTok{    123 Glennwood Ave \textbackslash{}}
\NormalTok{    Quarto Creek, VA 22438}
\NormalTok{  ],}
\NormalTok{  date: [Morristown, June 9th, 2023],}
\NormalTok{  subject: [Revision of our Producrement Contract],}
\NormalTok{  name: [Jane Smith \textbackslash{} Regional Director],}
\NormalTok{)}

\NormalTok{Dear Joe,}

\NormalTok{\#lorem(99)}

\NormalTok{Best,}
\end{Highlighting}
\end{Shaded}

\href{/app?template=appreciated-letter&version=0.1.0}{Create project in
app}

\subsubsection{How to use}\label{how-to-use}

Click the button above to create a new project using this template in
the Typst app.

You can also use the Typst CLI to start a new project on your computer
using this command:

\begin{verbatim}
typst init @preview/appreciated-letter:0.1.0
\end{verbatim}

\includesvg[width=0.16667in,height=0.16667in]{/assets/icons/16-copy.svg}

\subsubsection{About}\label{about}

\begin{description}
\tightlist
\item[Author :]
\href{https://typst.app}{Typst GmbH}
\item[License:]
MIT-0
\item[Current version:]
0.1.0
\item[Last updated:]
March 6, 2024
\item[First released:]
March 6, 2024
\item[Minimum Typst version:]
0.6.0
\item[Archive size:]
2.33 kB
\href{https://packages.typst.org/preview/appreciated-letter-0.1.0.tar.gz}{\pandocbounded{\includesvg[keepaspectratio]{/assets/icons/16-download.svg}}}
\item[Repository:]
\href{https://github.com/typst/templates}{GitHub}
\item[Categor y :]
\begin{itemize}
\tightlist
\item[]
\item
  \pandocbounded{\includesvg[keepaspectratio]{/assets/icons/16-envelope.svg}}
  \href{https://typst.app/universe/search/?category=office}{Office}
\end{itemize}
\end{description}

\subsubsection{Where to report issues?}\label{where-to-report-issues}

This template is a project of Typst GmbH . Report issues on
\href{https://github.com/typst/templates}{their repository} . You can
also try to ask for help with this template on the
\href{https://forum.typst.app}{Forum} .

\phantomsection\label{versions}
\subsubsection{Version history}\label{version-history}

\begin{longtable}[]{@{}ll@{}}
\toprule\noalign{}
Version & Release Date \\
\midrule\noalign{}
\endhead
\bottomrule\noalign{}
\endlastfoot
0.1.0 & March 6, 2024 \\
\end{longtable}


\section{Package List LaTeX/commute.tex}
\title{typst.app/universe/package/commute}

\phantomsection\label{banner}
\section{commute}\label{commute}

{ 0.2.0 }

A proof of concept library for commutative diagrams.

{ } Featured Package

\phantomsection\label{readme}
Proof-of-concept commutative diagrams library for
\href{https://typst.app/home}{typst}

See {[}EricWay1024/tikzcd-editor{]}{[}
\url{https://github.com/EricWay1024/tikzcd-editor} {]} for a web-based
visual diagram editor for this library!

\begin{verbatim}
#import "@preview/commute:0.2.0": node, arr, commutative-diagram

#align(center)[#commutative-diagram(
  node((0, 0), $X$),
  node((0, 1), $Y$),
  node((1, 0), $X \/ "ker"(f)$, "quot"),
  arr($X$, $Y$, $f$),
  arr("quot", (0, 1), $tilde(f)$, label-pos: right, "dashed", "inj"),
  arr($X$, "quot", $pi$),
)]
\end{verbatim}

\pandocbounded{\includegraphics[keepaspectratio]{https://github.com/typst/packages/assets/20535498/71eb8d47-b6f9-43fa-a1fd-7ff58b8d0025}}

For more usage examples look at \texttt{\ example.typ\ }

The library provides 3 functions: \texttt{\ node\ } , \texttt{\ arr\ } ,
and \texttt{\ commutative-diagram\ } . You can clone this repo and
import \texttt{\ lib.typ\ } :

\begin{verbatim}
#import "path/to/commute/lib.typ": node, arr, commutative-diagram
\end{verbatim}

Or directly use the builtin package manager:

\begin{verbatim}
#import "@preview/commute:0.2.0": node, arr, commutative-diagram
\end{verbatim}

\subsection{\texorpdfstring{\texttt{\ commutative-diagram\ }}{ commutative-diagram }}\label{commutative-diagram}

\begin{verbatim}
commutative-diagram(
  node-padding: (70pt, 70pt),
  arr-clearance: 0.7em,
  padding: 1.5em,
  debug: false,
  ..entities
)
\end{verbatim}

\texttt{\ commutative-diagram\ } returns a rectangular region containing
the nodes and arrows. All the unnamed arguments passed to
\texttt{\ commutative-diagram\ } are treated as nodes or arrows of the
diagram. These can be constructed using the \texttt{\ node\ } and
\texttt{\ arr\ } functions explained below. The other arguments are as
follows:

\begin{itemize}
\tightlist
\item
  \texttt{\ node-padding\ } : \texttt{\ (length,\ length)\ } . The space
  to leave between adjacent nodes. It’s a tuple, \texttt{\ (h,\ v)\ }
  , containing the horizontal and vertical spacing respectively.
\item
  \texttt{\ arr-clearance\ } : \texttt{\ length\ } . The default space
  between arrows’ base/tip and the diagram’s nodes.
\item
  \texttt{\ padding\ } : \texttt{\ length\ } . The padding around the
  whole diagram
\item
  \texttt{\ debug\ } : \texttt{\ bool\ } . Whether or not to display
  debug information.
\end{itemize}

\subsection{\texorpdfstring{\texttt{\ node\ }}{ node }}\label{node}

\begin{verbatim}
node(
  pos,
  label,
  id: label,
)
\end{verbatim}

Creates a new diagram node. Has the following positional arguments:

\begin{itemize}
\tightlist
\item
  \texttt{\ pos\ } : \texttt{\ (integer,\ integer)\ } . The position of
  the node in \texttt{\ (row,\ column)\ } format. Must be integers, but
  can be negative, the only thing that counts is the difference between
  the coordinares of the variuos nodes in the diagram.
\item
  \texttt{\ label\ } : \texttt{\ content\ } . The node’s label.
\item
  \texttt{\ id\ } : \texttt{\ any\ } . The node’s id, defaults to its
  label if not specified.
\end{itemize}

\subsection{\texorpdfstring{\texttt{\ arr\ }}{ arr }}\label{arr}

\begin{verbatim}
arr(
  start,
  end,
  label,
  start-space: none,
  end-space: none,
  label-pos: left,
  curve: 0deg,
  stroke: 0.45pt,
  ..options
)
\end{verbatim}

Creates an arrow. Has the following arguments:

\begin{itemize}
\tightlist
\item
  \texttt{\ start\ } : \texttt{\ (integer,\ integer)\ } or
  \texttt{\ any\ } . The position of the node from which the arrow
  starts, in \texttt{\ (row,\ column)\ } format, or its id.
\item
  \texttt{\ end\ } : \texttt{\ (integer,\ integer)\ } or
  \texttt{\ any\ } . The position of the node where the arrow ends, in
  \texttt{\ (row,\ column)\ } format, or its id.
\item
  \texttt{\ label\ } : \texttt{\ content\ } . The label to put on the
  arrow.
\item
  \texttt{\ start-space\ } : \texttt{\ length\ } . The space between the
  start node and the beginning of the arrow. You can pass
  \texttt{\ none\ } to leave a sensible default, customizable using the
  \texttt{\ arr-clearance\ } parameter of the
  \texttt{\ commutative-diagram\ } function.
\item
  \texttt{\ end-space\ } : \texttt{\ length\ } . Similar to the above.
\item
  \texttt{\ label-pos\ } : \texttt{\ length\ } or \texttt{\ left\ } or
  \texttt{\ right\ } . Where to position the arrow’s label relative to
  the arrow. A positive length means that, when looking towards the tip
  of the arrow, the label is on the left. \texttt{\ left\ } and
  \texttt{\ right\ } measure the label to automatically get a reasonable
  length. If set to \texttt{\ 0\ } ( \texttt{\ 0\ } the number, which is
  different from \texttt{\ 0pt\ } or \texttt{\ 0em\ } ) then the label
  is placed on top of the arrow, with a white background to help with
  legibility.
\item
  \texttt{\ curve\ } : \texttt{\ angle\ } . The difference in
  orientation between the start and the end of the arrow. If positive,
  the arrow curves to the right, when looking towards the tip.
\item
  \texttt{\ stroke\ } : \texttt{\ stroke\ } . The thickness and color of
  the arrows. The default should probably be fine.
\item
  \texttt{\ options\ } : \texttt{\ string\ } s. After the mandatory
  positional arguments \texttt{\ start\ } , \texttt{\ end\ } and
  \texttt{\ label\ } , any remaining unnamed argument is treated as an
  extra option. Recognized options are:

  \begin{itemize}
  \tightlist
  \item
    \texttt{\ "inj"\ } , gives the arrow a hook at the start, used for
    injective functions
  \item
    \texttt{\ "surj"\ } , gives the arrow a double tip, used for
    surjective functions
  \item
    \texttt{\ "bij"\ } , gives the arrow a tip also at the start, used
    for bijective functions
  \item
    \texttt{\ "def"\ } , gives the arrow a bar at the start, used for
    function definitions
  \item
    \texttt{\ "nat"\ } , gives the arrow a double stem, used for natural
    transformations
  \item
    All the options supported by the \texttt{\ dash\ } parameter of
    Typst’s \texttt{\ stroke\ } type, such as \texttt{\ "dashed"\ } ,
    \texttt{\ "densely-dotted"\ } , etc. These change the appearance of
    the arrow’s stem
  \end{itemize}
\end{itemize}

\subsubsection{How to add}\label{how-to-add}

Copy this into your project and use the import as \texttt{\ commute\ }

\begin{verbatim}
#import "@preview/commute:0.2.0"
\end{verbatim}

\includesvg[width=0.16667in,height=0.16667in]{/assets/icons/16-copy.svg}

Check the docs for
\href{https://typst.app/docs/reference/scripting/\#packages}{more
information on how to import packages} .

\subsubsection{About}\label{about}

\begin{description}
\tightlist
\item[Author :]
\href{https://gitlab.com/giacomogallina}{giacomogallina}
\item[License:]
MIT
\item[Current version:]
0.2.0
\item[Last updated:]
November 1, 2023
\item[First released:]
July 21, 2023
\item[Archive size:]
6.15 kB
\href{https://packages.typst.org/preview/commute-0.2.0.tar.gz}{\pandocbounded{\includesvg[keepaspectratio]{/assets/icons/16-download.svg}}}
\item[Repository:]
\href{https://gitlab.com/giacomogallina/commute}{GitLab}
\end{description}

\subsubsection{Where to report issues?}\label{where-to-report-issues}

This package is a project of giacomogallina . Report issues on
\href{https://gitlab.com/giacomogallina/commute}{their repository} . You
can also try to ask for help with this package on the
\href{https://forum.typst.app}{Forum} .

Please report this package to the Typst team using the
\href{https://typst.app/contact}{contact form} if you believe it is a
safety hazard or infringes upon your rights.

\phantomsection\label{versions}
\subsubsection{Version history}\label{version-history}

\begin{longtable}[]{@{}ll@{}}
\toprule\noalign{}
Version & Release Date \\
\midrule\noalign{}
\endhead
\bottomrule\noalign{}
\endlastfoot
0.2.0 & November 1, 2023 \\
\href{https://typst.app/universe/package/commute/0.1.0/}{0.1.0} & July
21, 2023 \\
\end{longtable}

Typst GmbH did not create this package and cannot guarantee correct
functionality of this package or compatibility with any version of the
Typst compiler or app.


\section{Package List LaTeX/nth.tex}
\title{typst.app/universe/package/nth}

\phantomsection\label{banner}
\section{nth}\label{nth}

{ 1.0.1 }

Add english ordinals to numbers, eg. 1st, 2nd, 3rd, 4th.

\phantomsection\label{readme}
Provides functions \texttt{\ \#nth()\ } and \texttt{\ \#nths()\ } which
take a number and return it suffixed by an english ordinal.

This package is named after the nth
\href{https://ctan.org/pkg/nth}{LaTeX macro} by Donald Arseneau.

\subsection{Usage}\label{usage}

Include this line in your document to import the package.

\begin{Shaded}
\begin{Highlighting}[]
\NormalTok{\#import "@preview/nth:1.0.1": *}
\end{Highlighting}
\end{Shaded}

Then, you can use \texttt{\ \#nth()\ } to markup ordinal numbers in your
document.

For example, \texttt{\ \#nth(1)\ } shows 1st,\\
\texttt{\ \#nth(2)\ } shows 2nd,\\
\texttt{\ \#nth(3)\ } shows 3rd,\\
\texttt{\ \#nth(4)\ } shows 4th,\\
and \texttt{\ \#nth(11)\ } shows 11th.

If you want the ordinal to be in superscript, use \texttt{\ \#nths\ }
with an ‘s’ at the end.

For example, \texttt{\ \#nths(1)\ } shows 1 \textsuperscript{st} .

\subsubsection{How to add}\label{how-to-add}

Copy this into your project and use the import as \texttt{\ nth\ }

\begin{verbatim}
#import "@preview/nth:1.0.1"
\end{verbatim}

\includesvg[width=0.16667in,height=0.16667in]{/assets/icons/16-copy.svg}

Check the docs for
\href{https://typst.app/docs/reference/scripting/\#packages}{more
information on how to import packages} .

\subsubsection{About}\label{about}

\begin{description}
\tightlist
\item[Author s :]
\href{mailto:pierre.marshall@gmail.com}{Pierre Marshall} ,
\href{https://github.com/fnoaman}{fnoaman} , \&
\href{https://github.com/jeffa5}{Andrew Jeffery}
\item[License:]
MIT-0
\item[Current version:]
1.0.1
\item[Last updated:]
June 21, 2024
\item[First released:]
September 22, 2023
\item[Minimum Typst version:]
0.8.0
\item[Archive size:]
2.38 kB
\href{https://packages.typst.org/preview/nth-1.0.1.tar.gz}{\pandocbounded{\includesvg[keepaspectratio]{/assets/icons/16-download.svg}}}
\item[Repository:]
\href{https://github.com/extua/nth}{GitHub}
\item[Categor y :]
\begin{itemize}
\tightlist
\item[]
\item
  \pandocbounded{\includesvg[keepaspectratio]{/assets/icons/16-text.svg}}
  \href{https://typst.app/universe/search/?category=text}{Text}
\end{itemize}
\end{description}

\subsubsection{Where to report issues?}\label{where-to-report-issues}

This package is a project of Pierre Marshall, fnoaman, and Andrew
Jeffery . Report issues on \href{https://github.com/extua/nth}{their
repository} . You can also try to ask for help with this package on the
\href{https://forum.typst.app}{Forum} .

Please report this package to the Typst team using the
\href{https://typst.app/contact}{contact form} if you believe it is a
safety hazard or infringes upon your rights.

\phantomsection\label{versions}
\subsubsection{Version history}\label{version-history}

\begin{longtable}[]{@{}ll@{}}
\toprule\noalign{}
Version & Release Date \\
\midrule\noalign{}
\endhead
\bottomrule\noalign{}
\endlastfoot
1.0.1 & June 21, 2024 \\
\href{https://typst.app/universe/package/nth/1.0.0/}{1.0.0} & December
23, 2023 \\
\href{https://typst.app/universe/package/nth/0.2.0/}{0.2.0} & October 2,
2023 \\
\href{https://typst.app/universe/package/nth/0.1.0/}{0.1.0} & September
22, 2023 \\
\end{longtable}

Typst GmbH did not create this package and cannot guarantee correct
functionality of this package or compatibility with any version of the
Typst compiler or app.


\section{Package List LaTeX/fauve-cdb.tex}
\title{typst.app/universe/package/fauve-cdb}

\phantomsection\label{banner}
\phantomsection\label{template-thumbnail}
\pandocbounded{\includegraphics[keepaspectratio]{https://packages.typst.org/preview/thumbnails/fauve-cdb-0.1.0-small.webp}}

\section{fauve-cdb}\label{fauve-cdb}

{ 0.1.0 }

The unofficial implementation of the Collège Doctoral de Bretagne
thesis manuscript template.

\href{/app?template=fauve-cdb&version=0.1.0}{Create project in app}

\phantomsection\label{readme}
Typst template for doctoral dissertations of the French
\href{https://www.doctorat-bretagne.fr/}{Collège doctoral de Bretagne
(CdB)} . The original LaTeX template can be found
\href{https://gitlab.com/ed-matisse/latex-template}{here} .

You can use this template in the Typst web app by clicking “Start from
template� on the dashboard and searching for \texttt{\ fauve-cdb\ } .

Alternatively, you can use the CLI to kick this project off using the
command

\begin{verbatim}
typst init @preview/fauve-cdb
\end{verbatim}

Typst will create a new directory with all the files needed to get you
started.

The original LaTeX template allows selecting different themes
corresponding to different schools of the CdB. For now, we only
implemented the \href{https://ed-matisse.doctorat-bretagne.fr/}{MATISSE}
theme.

\begin{quote}
Fauve is an artistic movement of which French painter
\href{https://en.wikipedia.org/wiki/Henri_Matisse}{Henri Matisse} was a
leader.
\end{quote}

\href{/app?template=fauve-cdb&version=0.1.0}{Create project in app}

\subsubsection{How to use}\label{how-to-use}

Click the button above to create a new project using this template in
the Typst app.

You can also use the Typst CLI to start a new project on your computer
using this command:

\begin{verbatim}
typst init @preview/fauve-cdb:0.1.0
\end{verbatim}

\includesvg[width=0.16667in,height=0.16667in]{/assets/icons/16-copy.svg}

\subsubsection{About}\label{about}

\begin{description}
\tightlist
\item[Author s :]
\href{mailto:timothe.albouy@gmail.com}{Timothé Albouy} \&
\href{https://grodino.github.io}{Augustin Godinot}
\item[License:]
MIT-0
\item[Current version:]
0.1.0
\item[Last updated:]
September 25, 2024
\item[First released:]
September 25, 2024
\item[Archive size:]
122 kB
\href{https://packages.typst.org/preview/fauve-cdb-0.1.0.tar.gz}{\pandocbounded{\includesvg[keepaspectratio]{/assets/icons/16-download.svg}}}
\item[Discipline s :]
\begin{itemize}
\tightlist
\item[]
\item
  \href{https://typst.app/universe/search/?discipline=computer-science}{Computer
  Science}
\item
  \href{https://typst.app/universe/search/?discipline=mathematics}{Mathematics}
\end{itemize}
\item[Categor y :]
\begin{itemize}
\tightlist
\item[]
\item
  \pandocbounded{\includesvg[keepaspectratio]{/assets/icons/16-mortarboard.svg}}
  \href{https://typst.app/universe/search/?category=thesis}{Thesis}
\end{itemize}
\end{description}

\subsubsection{Where to report issues?}\label{where-to-report-issues}

This template is a project of Timothé Albouy and Augustin Godinot . You
can also try to ask for help with this template on the
\href{https://forum.typst.app}{Forum} .

Please report this template to the Typst team using the
\href{https://typst.app/contact}{contact form} if you believe it is a
safety hazard or infringes upon your rights.

\phantomsection\label{versions}
\subsubsection{Version history}\label{version-history}

\begin{longtable}[]{@{}ll@{}}
\toprule\noalign{}
Version & Release Date \\
\midrule\noalign{}
\endhead
\bottomrule\noalign{}
\endlastfoot
0.1.0 & September 25, 2024 \\
\end{longtable}

Typst GmbH did not create this template and cannot guarantee correct
functionality of this template or compatibility with any version of the
Typst compiler or app.


\section{Package List LaTeX/wrap-it.tex}
\title{typst.app/universe/package/wrap-it}

\phantomsection\label{banner}
\section{wrap-it}\label{wrap-it}

{ 0.1.1 }

Wrap text around figures and content

{ } Featured Package

\phantomsection\label{readme}
Until \ul{\ul{\url{https://github.com/typst/typst/issues/553}}} is
resolved, \texttt{\ typst\ } doesn’t natively support wrapping text
around figures or other content. However, you can use
\texttt{\ wrap-it\ } to mimic much of this functionality:

\begin{itemize}
\item
  Wrapping images left or right of their text
\item
  Specifying margins
\item
  And more
\end{itemize}

Detailed descriptions of each parameter are available in the
\ul{\ul{\href{https://github.com/ntjess/wrap-it/blob/main/docs/manual.pdf}{wrap-it
documentation}}} .

The easiest method is to import \texttt{\ wrap-it:\ wrap-content\ } from
the \texttt{\ @preview\ } package:

\begin{Shaded}
\begin{Highlighting}[]
\NormalTok{\#import "@preview/wrap{-}it:0.1.0": wrap{-}content}
\end{Highlighting}
\end{Shaded}

\subsection{Vanilla}\label{vanilla}

\begin{Shaded}
\begin{Highlighting}[]
\NormalTok{\#let fig = figure(}
\NormalTok{rect(fill: teal, radius: 0.5em, width: 8em),}
\NormalTok{caption: [A figure],}
\NormalTok{)}
\NormalTok{\#let body = lorem(30)}
\NormalTok{\#wrap{-}content(fig, body)}
\end{Highlighting}
\end{Shaded}

\pandocbounded{\includegraphics[keepaspectratio]{https://www.github.com/ntjess/wrap-it/raw/v0.1.1/assets/example-1.png}}

\subsection{Changing alignment and
margin}\label{changing-alignment-and-margin}

\begin{Shaded}
\begin{Highlighting}[]
\NormalTok{\#wrap{-}content(}
\NormalTok{fig,}
\NormalTok{body,}
\NormalTok{align: bottom + right,}
\NormalTok{column{-}gutter: 2em}
\NormalTok{)}
\end{Highlighting}
\end{Shaded}

\pandocbounded{\includegraphics[keepaspectratio]{https://www.github.com/ntjess/wrap-it/raw/v0.1.1/assets/example-2.png}}

\subsection{Uniform margin around the
image}\label{uniform-margin-around-the-image}

The easiest way to get a uniform, highly-customizable margin is through
boxing your image:

\begin{Shaded}
\begin{Highlighting}[]
\NormalTok{\#let boxed = box(fig, inset: 0.25em)}
\NormalTok{\#wrap{-}content(boxed)[}
\NormalTok{\#lorem(30)}
\NormalTok{]}
\end{Highlighting}
\end{Shaded}

\pandocbounded{\includegraphics[keepaspectratio]{https://www.github.com/ntjess/wrap-it/raw/v0.1.1/assets/example-3.png}}

\subsection{Wrapping two images in the same
paragraph}\label{wrapping-two-images-in-the-same-paragraph}

Note that for longer captions (as is the case in the bottom figure
below), providing an explicit \texttt{\ columns\ } parameter is
necessary to inform caption text of where to wrap.

\begin{Shaded}
\begin{Highlighting}[]
\NormalTok{\#let fig2 = figure(}
\NormalTok{rect(fill: lime, radius: 0.5em),}
\NormalTok{caption: [\#lorem(10)],}
\NormalTok{)}
\NormalTok{\#wrap{-}top{-}bottom(}
\NormalTok{bottom{-}kwargs: (columns: (1fr, 2fr)),}
\NormalTok{box(fig, inset: 0.25em),}
\NormalTok{fig2,}
\NormalTok{lorem(50),}
\NormalTok{)}
\end{Highlighting}
\end{Shaded}

\pandocbounded{\includegraphics[keepaspectratio]{https://www.github.com/ntjess/wrap-it/raw/v0.1.1/assets/example-4.png}}

\subsection{Adding a label to a wrapped
figure}\label{adding-a-label-to-a-wrapped-figure}

Typst can only append labels to figures in content mode. So, when
wrapping text around a figure that needs a label, you must first place
your figure in a content block with its label, then wrap it:

\begin{Shaded}
\begin{Highlighting}[]
\NormalTok{\#show ref: it =\textgreater{} underline(text(blue, it))}
\NormalTok{\#let fig = [}
\NormalTok{  \#figure(}
\NormalTok{    rect(fill: red, radius: 0.5em, width: 8em),}
\NormalTok{    caption:[Labeled]}
\NormalTok{  )\textless{}fig:lbl\textgreater{}}
\NormalTok{]}
\NormalTok{\#wrap{-}content(fig, [Fortunately, @fig:lbl\textquotesingle{}s label can be referenced within the wrapped text. \#lorem(15)])}
\end{Highlighting}
\end{Shaded}

\pandocbounded{\includegraphics[keepaspectratio]{https://www.github.com/ntjess/wrap-it/raw/v0.1.1/assets/example-5.png}}

\subsubsection{How to add}\label{how-to-add}

Copy this into your project and use the import as \texttt{\ wrap-it\ }

\begin{verbatim}
#import "@preview/wrap-it:0.1.1"
\end{verbatim}

\includesvg[width=0.16667in,height=0.16667in]{/assets/icons/16-copy.svg}

Check the docs for
\href{https://typst.app/docs/reference/scripting/\#packages}{more
information on how to import packages} .

\subsubsection{About}\label{about}

\begin{description}
\tightlist
\item[Author :]
Nathan Jessurun
\item[License:]
Unlicense
\item[Current version:]
0.1.1
\item[Last updated:]
November 28, 2024
\item[First released:]
January 26, 2024
\item[Archive size:]
5.30 kB
\href{https://packages.typst.org/preview/wrap-it-0.1.1.tar.gz}{\pandocbounded{\includesvg[keepaspectratio]{/assets/icons/16-download.svg}}}
\item[Repository:]
\href{https://github.com/ntjess/wrap-it}{GitHub}
\end{description}

\subsubsection{Where to report issues?}\label{where-to-report-issues}

This package is a project of Nathan Jessurun . Report issues on
\href{https://github.com/ntjess/wrap-it}{their repository} . You can
also try to ask for help with this package on the
\href{https://forum.typst.app}{Forum} .

Please report this package to the Typst team using the
\href{https://typst.app/contact}{contact form} if you believe it is a
safety hazard or infringes upon your rights.

\phantomsection\label{versions}
\subsubsection{Version history}\label{version-history}

\begin{longtable}[]{@{}ll@{}}
\toprule\noalign{}
Version & Release Date \\
\midrule\noalign{}
\endhead
\bottomrule\noalign{}
\endlastfoot
0.1.1 & November 28, 2024 \\
\href{https://typst.app/universe/package/wrap-it/0.1.0/}{0.1.0} &
January 26, 2024 \\
\end{longtable}

Typst GmbH did not create this package and cannot guarantee correct
functionality of this package or compatibility with any version of the
Typst compiler or app.


\section{Package List LaTeX/clean-math-presentation.tex}
\title{typst.app/universe/package/clean-math-presentation}

\phantomsection\label{banner}
\phantomsection\label{template-thumbnail}
\pandocbounded{\includegraphics[keepaspectratio]{https://packages.typst.org/preview/thumbnails/clean-math-presentation-0.1.0-small.webp}}

\section{clean-math-presentation}\label{clean-math-presentation}

{ 0.1.0 }

A simple and good looking template for mathematical presentations

\href{/app?template=clean-math-presentation&version=0.1.0}{Create
project in app}

\phantomsection\label{readme}
\href{https://github.com/JoshuaLampert/clean-math-presentation/actions/workflows/build.yml}{\pandocbounded{\includesvg[keepaspectratio]{https://github.com/JoshuaLampert/clean-math-presentation/actions/workflows/build.yml/badge.svg}}}
\href{https://github.com/JoshuaLampert/clean-math-presentation}{\pandocbounded{\includegraphics[keepaspectratio]{https://img.shields.io/badge/GitHub-repo-blue}}}
\href{https://opensource.org/licenses/MIT}{\pandocbounded{\includesvg[keepaspectratio]{https://img.shields.io/badge/License-MIT-success.svg}}}

\href{https://typst.app/home/}{Typst} template for presentations built
for simple, efficient use and a clean look using
\href{https://touying-typ.github.io/}{touying} . The template provides a
custom title page, a footer, a header, and built-in support for theorem
blocks and proofs.

\subsection{Usage}\label{usage}

The template is already filled with dummy data, to give users an
impression how it looks like. The paper is obtained by compiling
\texttt{\ main.typ\ } .

\begin{itemize}
\tightlist
\item
  after
  \href{https://github.com/typst/typst?tab=readme-ov-file\#installation}{installing
  Typst} you can conveniently use the following to create a new folder
  containing this project.
\end{itemize}

\begin{Shaded}
\begin{Highlighting}[]
\ExtensionTok{typst}\NormalTok{ init @preview/clean{-}math{-}presentation:0.1.0}
\end{Highlighting}
\end{Shaded}

\begin{itemize}
\tightlist
\item
  edit the data in \texttt{\ main.typ\ } â†'
  \texttt{\ \#show\ template.with({[}your\ data{]})\ }
\end{itemize}

\subsubsection{Parameters of the
Template}\label{parameters-of-the-template}

\begin{itemize}
\tightlist
\item
  \texttt{\ title\ } : Title of the presentation.
\item
  \texttt{\ subtitle\ } : Subtitle of the presentation, optional.
\item
  \texttt{\ short-title\ } : Short version of the presentation to be
  shown in the footer, optional. If not short title is provided, the
  \texttt{\ title\ } will be shown in the footer.
\item
  \texttt{\ date\ } : Date of the presentation.
\item
  \texttt{\ authors\ } : List of names of the authors of the paper. Each
  entry of the list is a dictionary with the following keys:

  \begin{itemize}
  \tightlist
  \item
    \texttt{\ name\ } : Name of the author.
  \item
    \texttt{\ affiliation-id\ } : The ID of the affiliation in
    \texttt{\ affiliations\ } , see below.
  \end{itemize}
\item
  \texttt{\ affiliations\ } : List of affiliations of the authors. Each
  entry of the list is a dictionary with the following keys:

  \begin{itemize}
  \tightlist
  \item
    \texttt{\ id\ } : ID of the affiliation, which is used to link the
    authors to the affiliation, see above.
  \item
    \texttt{\ name\ } : Name of the affiliation.
  \end{itemize}
\item
  \texttt{\ author\ } : The name of the presenting author, which will be
  displayed in the footer of each slide. If the \texttt{\ author\ }
  matches one of the \texttt{\ authors\ } , this name will be underlined
  in the title slide.
\end{itemize}

Other arguments like \texttt{\ align\ } , \texttt{\ progess-bar\ } and
more are available and similar to other templates in touying, especially
the \href{https://touying-typ.github.io/docs/themes/stargazer}{stargazer
theme} . The colorscheme can be adjusted by passing
\texttt{\ config-colors\ } to the \texttt{\ template\ } , e.g.

\begin{Shaded}
\begin{Highlighting}[]
\NormalTok{config{-}colors(}
\NormalTok{  primary: rgb("\#6068d6"),}
\NormalTok{  secondary: rgb("\#2f1971"),}
\NormalTok{)}
\end{Highlighting}
\end{Shaded}

The title page can be created with \texttt{\ \#title-slide\ } . It
accepts optionally a \texttt{\ background\ } , which can be an image or
\texttt{\ none\ } (default) and up to two logos \texttt{\ logo1\ } and
\texttt{\ logo2\ } ( \texttt{\ none\ } by default).

The theme provides different types of slides like
\texttt{\ \#outline-slide\ } , \texttt{\ \#focus-slide\ } ,
\texttt{\ \#ending-slide\ } , and the usual \texttt{\ \#slide\ } .
Additionally,it already provides support for theorems and alike by the
functions \texttt{\ \#theorem\ } , \texttt{\ \#lemma\ } ,
\texttt{\ \#corollary\ } , \texttt{\ \#definition\ } ,
\texttt{\ \#example\ } , and \texttt{\ \#proof\ } .

\subsection{Acknowledgements}\label{acknowledgements}

Some parts of this template are based on the
\href{https://github.com/touying-typ/touying/blob/main/themes/stargazer.typ}{stargazer}
theme from touying.

\subsection{Feedback \& Improvements}\label{feedback-improvements}

If you encounter problems, please open issues. In case you found useful
extensions or improved anything We are also very happy to accept pull
requests.

\href{/app?template=clean-math-presentation&version=0.1.0}{Create
project in app}

\subsubsection{How to use}\label{how-to-use}

Click the button above to create a new project using this template in
the Typst app.

You can also use the Typst CLI to start a new project on your computer
using this command:

\begin{verbatim}
typst init @preview/clean-math-presentation:0.1.0
\end{verbatim}

\includesvg[width=0.16667in,height=0.16667in]{/assets/icons/16-copy.svg}

\subsubsection{About}\label{about}

\begin{description}
\tightlist
\item[Author :]
\href{https://github.com/JoshuaLampert}{Joshua Lampert}
\item[License:]
MIT
\item[Current version:]
0.1.0
\item[Last updated:]
November 21, 2024
\item[First released:]
November 21, 2024
\item[Minimum Typst version:]
0.12.0
\item[Archive size:]
10.3 kB
\href{https://packages.typst.org/preview/clean-math-presentation-0.1.0.tar.gz}{\pandocbounded{\includesvg[keepaspectratio]{/assets/icons/16-download.svg}}}
\item[Repository:]
\href{https://github.com/JoshuaLampert/clean-math-presentation}{GitHub}
\item[Discipline s :]
\begin{itemize}
\tightlist
\item[]
\item
  \href{https://typst.app/universe/search/?discipline=mathematics}{Mathematics}
\item
  \href{https://typst.app/universe/search/?discipline=engineering}{Engineering}
\item
  \href{https://typst.app/universe/search/?discipline=computer-science}{Computer
  Science}
\end{itemize}
\item[Categor y :]
\begin{itemize}
\tightlist
\item[]
\item
  \pandocbounded{\includesvg[keepaspectratio]{/assets/icons/16-presentation.svg}}
  \href{https://typst.app/universe/search/?category=presentation}{Presentation}
\end{itemize}
\end{description}

\subsubsection{Where to report issues?}\label{where-to-report-issues}

This template is a project of Joshua Lampert . Report issues on
\href{https://github.com/JoshuaLampert/clean-math-presentation}{their
repository} . You can also try to ask for help with this template on the
\href{https://forum.typst.app}{Forum} .

Please report this template to the Typst team using the
\href{https://typst.app/contact}{contact form} if you believe it is a
safety hazard or infringes upon your rights.

\phantomsection\label{versions}
\subsubsection{Version history}\label{version-history}

\begin{longtable}[]{@{}ll@{}}
\toprule\noalign{}
Version & Release Date \\
\midrule\noalign{}
\endhead
\bottomrule\noalign{}
\endlastfoot
0.1.0 & November 21, 2024 \\
\end{longtable}

Typst GmbH did not create this template and cannot guarantee correct
functionality of this template or compatibility with any version of the
Typst compiler or app.


\section{Package List LaTeX/indic-numerals.tex}
\title{typst.app/universe/package/indic-numerals}

\phantomsection\label{banner}
\section{indic-numerals}\label{indic-numerals}

{ 0.1.0 }

convert arabic numerals to indic numerals and vice versa

\phantomsection\label{readme}
\href{https://github.com/cecoeco/indic-numerals/blob/main/LICENSE.md}{\pandocbounded{\includesvg[keepaspectratio]{https://img.shields.io/badge/License-MIT-blue.svg}}}

\subsection{indic-numerals}\label{indic-numerals-1}

\emph{convert arabic numerals to indic numerals and vice versa}

\begin{Shaded}
\begin{Highlighting}[]
\NormalTok{\#import "@preview/indic{-}numerals:0.1.0": arabic{-}to{-}indic, indic{-}to{-}arabic}

\NormalTok{\#indic{-}to{-}arabic("௦௧௨௩௪௫௬௭௮௯", "tamil") // Output: 0123456789}

\NormalTok{\#arabic{-}to{-}indic("0123456789", "tamil") // Output: ௦௧௨௩௪௫௬௭௮௯}
\end{Highlighting}
\end{Shaded}

\subsubsection{How to add}\label{how-to-add}

Copy this into your project and use the import as
\texttt{\ indic-numerals\ }

\begin{verbatim}
#import "@preview/indic-numerals:0.1.0"
\end{verbatim}

\includesvg[width=0.16667in,height=0.16667in]{/assets/icons/16-copy.svg}

Check the docs for
\href{https://typst.app/docs/reference/scripting/\#packages}{more
information on how to import packages} .

\subsubsection{About}\label{about}

\begin{description}
\tightlist
\item[Author :]
Ceco Elijah Maples
\item[License:]
MIT
\item[Current version:]
0.1.0
\item[Last updated:]
November 4, 2024
\item[First released:]
November 4, 2024
\item[Archive size:]
1.85 kB
\href{https://packages.typst.org/preview/indic-numerals-0.1.0.tar.gz}{\pandocbounded{\includesvg[keepaspectratio]{/assets/icons/16-download.svg}}}
\item[Repository:]
\href{https://github.com/cecoeco/indic-numerals}{GitHub}
\end{description}

\subsubsection{Where to report issues?}\label{where-to-report-issues}

This package is a project of Ceco Elijah Maples . Report issues on
\href{https://github.com/cecoeco/indic-numerals}{their repository} . You
can also try to ask for help with this package on the
\href{https://forum.typst.app}{Forum} .

Please report this package to the Typst team using the
\href{https://typst.app/contact}{contact form} if you believe it is a
safety hazard or infringes upon your rights.

\phantomsection\label{versions}
\subsubsection{Version history}\label{version-history}

\begin{longtable}[]{@{}ll@{}}
\toprule\noalign{}
Version & Release Date \\
\midrule\noalign{}
\endhead
\bottomrule\noalign{}
\endlastfoot
0.1.0 & November 4, 2024 \\
\end{longtable}

Typst GmbH did not create this package and cannot guarantee correct
functionality of this package or compatibility with any version of the
Typst compiler or app.


\section{Package List LaTeX/silky-letter-insa.tex}
\title{typst.app/universe/package/silky-letter-insa}

\phantomsection\label{banner}
\phantomsection\label{template-thumbnail}
\pandocbounded{\includegraphics[keepaspectratio]{https://packages.typst.org/preview/thumbnails/silky-letter-insa-0.2.2-small.webp}}

\section{silky-letter-insa}\label{silky-letter-insa}

{ 0.2.2 }

A template made for letters and short documents of INSA, a French
engineering school.

\href{/app?template=silky-letter-insa&version=0.2.2}{Create project in
app}

\phantomsection\label{readme}
Typst Template for short documents and letters for the french
engineering school INSA.

\subsection{Example}\label{example}

By default, the template initializes with the \texttt{\ insa-letter\ }
show rule, with parameters that you must fill in by yourself.

Here is an example of filled template:

\begin{Shaded}
\begin{Highlighting}[]
\NormalTok{\#import "@preview/silky{-}letter{-}insa:0.2.2": *}
\NormalTok{\#show: doc =\textgreater{} insa{-}letter(}
\NormalTok{  author: "Youenn LE JEUNE, Kelian NINET",}
\NormalTok{  insa: "rennes"}
\NormalTok{  doc)}

\NormalTok{\#v(15pt)}
\NormalTok{\#align(center, text(size: 22pt, weight: "bold", smallcaps("Probabilités {-} Annale 2022 (V1)")))}
\NormalTok{\#v(5pt)}

\NormalTok{\#set heading(numbering: "1.")}
\NormalTok{\#show heading.where(level: 2): it =\textgreater{} [}
\NormalTok{  \#counter(heading).display()}
\NormalTok{  \#text(weight: "medium", style: "italic", size: 13pt, it.body)}

\NormalTok{]}

\NormalTok{= Intervalle de confiance}
\NormalTok{== Calculer sur l’échantillon une estimation de la moyenne.}
\NormalTok{$ overline(x\_n) = 1/n sum\_(i=1)\^{}n x\_i = 1885 $}

\NormalTok{== Calculer sur l’échantillon une estimation de la variance.}
\NormalTok{$}
\NormalTok{"Variance biaisée :" s\^{}2 \&= 1/n sum\_(i=1)\^{}n (x\_i {-} overline(x\_n))\^{}2 = 218\^{}2\textbackslash{}}
\NormalTok{"Variance corrigée :" s\textquotesingle{}\^{}2 \&=  n/(n{-}1) s\^{}2 = 231\^{}2}
\NormalTok{$}

\NormalTok{Le bon estimateur est le second.}

\NormalTok{== Écrire le code R permettant d’évaluer les deux bornes de l’intervalle de confiance du temps d’exécution avec une confiance de 92\%.}
\NormalTok{Nous sommes dans le cas d\textquotesingle{}une recherche de moyenne avec variance inconnue, l\textquotesingle{}intervalle sera donc}
\NormalTok{$ [overline(X) + t\_(n{-}1)(alpha/2) S\textquotesingle{}/sqrt(n), quad overline(X) + t\_(n{-}1)(1 {-} alpha/2) S\textquotesingle{}/sqrt(n)] $}
\NormalTok{En R, avec l\textquotesingle{}échantillon nommé \textasciigrave{}data\textasciigrave{}, ça donne}
\NormalTok{\textasciigrave{}\textasciigrave{}\textasciigrave{}R}
\NormalTok{data = c(1653, 2059, 2281, 1813, 2180, 1721, 1857, 1677, 1728)}
\NormalTok{moyenne = mean(data)}
\NormalTok{s\_prime = sqrt(var(data)) \# car la variance de R est déjà corrigée}
\NormalTok{n = 9}
\NormalTok{alpha = 0.08}

\NormalTok{IC\_min = moyenne + qt(alpha / 2, df = n {-} 1) * s\_prime / sqrt(n)}
\NormalTok{IC\_max = moyenne + qt(1 {-} alpha / 2, df = n {-} 1) * s\_prime / sqrt(n)}
\NormalTok{\textasciigrave{}\textasciigrave{}\textasciigrave{}}

\NormalTok{Ici on a $[1730, 2040]$.}
\end{Highlighting}
\end{Shaded}

\subsection{Fonts}\label{fonts}

The graphic charter recommends the fonts \textbf{League Spartan} for
headings and \textbf{Source Serif} for regular text. To have the best
look, you should install those fonts.

To behave correctly on computers without those specific fonts installed,
this template will automatically fallback to other similar fonts:

\begin{itemize}
\tightlist
\item
  \textbf{League Spartan} -\textgreater{} \textbf{Arial} (approved by
  INSA’s graphic charter, by default in Windows) -\textgreater{}
  \textbf{Liberation Sans} (by default in most Linux)
\item
  \textbf{Source Serif} -\textgreater{} \textbf{Source Serif 4}
  (downloadable for free) -\textgreater{} \textbf{Georgia} (approved by
  the graphic charter) -\textgreater{} \textbf{Linux Libertine} (default
  Typst font)
\end{itemize}

\subsubsection{Note on variable fonts}\label{note-on-variable-fonts}

If you want to install those fonts on your computer, Typst might not
recognize them if you install their \emph{Variable} versions. You should
install the static versions ( \textbf{League Spartan Bold} and most
versions of \textbf{Source Serif} ).

Keep an eye on \href{https://github.com/typst/typst/issues/185}{the
issue in Typst bug tracker} to see when variable fonts will be used!

\subsection{Notes}\label{notes}

This template is being developed by Youenn LE JEUNE from the INSA de
Rennes in \href{https://github.com/SkytAsul/INSA-Typst-Template}{this
repository} .

For now it includes assets from the graphic charters of those INSAs:

\begin{itemize}
\tightlist
\item
  Rennes ( \texttt{\ rennes\ } )
\item
  Hauts de France ( \texttt{\ hdf\ } )
\item
  Centre Val de Loire ( \texttt{\ cvl\ } ) Users from other INSAs can
  open a pull request on the repository with the assets for their INSA.
\end{itemize}

If you have any other feature request, open an issue on the repository
as well.

\subsection{License}\label{license}

The typst template is licensed under the
\href{https://github.com/SkytAsul/INSA-Typst-Template/blob/main/LICENSE}{MIT
license} . This does \emph{not} apply to the image assets. Those image
files are property of Groupe INSA.

\subsection{Changelog}\label{changelog}

\subsubsection{0.2.2}\label{section}

\begin{itemize}
\tightlist
\item
  Added INSA CVL assets
\end{itemize}

\subsubsection{0.2.1}\label{section-1}

\begin{itemize}
\tightlist
\item
  Added \texttt{\ insa\ } option
\item
  Added INSA HdF assets
\end{itemize}

\href{/app?template=silky-letter-insa&version=0.2.2}{Create project in
app}

\subsubsection{How to use}\label{how-to-use}

Click the button above to create a new project using this template in
the Typst app.

You can also use the Typst CLI to start a new project on your computer
using this command:

\begin{verbatim}
typst init @preview/silky-letter-insa:0.2.2
\end{verbatim}

\includesvg[width=0.16667in,height=0.16667in]{/assets/icons/16-copy.svg}

\subsubsection{About}\label{about}

\begin{description}
\tightlist
\item[Author :]
SkytAsul
\item[License:]
MIT
\item[Current version:]
0.2.2
\item[Last updated:]
November 21, 2024
\item[First released:]
March 23, 2024
\item[Archive size:]
269 kB
\href{https://packages.typst.org/preview/silky-letter-insa-0.2.2.tar.gz}{\pandocbounded{\includesvg[keepaspectratio]{/assets/icons/16-download.svg}}}
\item[Repository:]
\href{https://github.com/SkytAsul/INSA-Typst-Template}{GitHub}
\item[Discipline s :]
\begin{itemize}
\tightlist
\item[]
\item
  \href{https://typst.app/universe/search/?discipline=engineering}{Engineering}
\item
  \href{https://typst.app/universe/search/?discipline=computer-science}{Computer
  Science}
\item
  \href{https://typst.app/universe/search/?discipline=mathematics}{Mathematics}
\item
  \href{https://typst.app/universe/search/?discipline=physics}{Physics}
\end{itemize}
\item[Categor y :]
\begin{itemize}
\tightlist
\item[]
\item
  \pandocbounded{\includesvg[keepaspectratio]{/assets/icons/16-envelope.svg}}
  \href{https://typst.app/universe/search/?category=office}{Office}
\end{itemize}
\end{description}

\subsubsection{Where to report issues?}\label{where-to-report-issues}

This template is a project of SkytAsul . Report issues on
\href{https://github.com/SkytAsul/INSA-Typst-Template}{their repository}
. You can also try to ask for help with this template on the
\href{https://forum.typst.app}{Forum} .

Please report this template to the Typst team using the
\href{https://typst.app/contact}{contact form} if you believe it is a
safety hazard or infringes upon your rights.

\phantomsection\label{versions}
\subsubsection{Version history}\label{version-history}

\begin{longtable}[]{@{}ll@{}}
\toprule\noalign{}
Version & Release Date \\
\midrule\noalign{}
\endhead
\bottomrule\noalign{}
\endlastfoot
0.2.2 & November 21, 2024 \\
\href{https://typst.app/universe/package/silky-letter-insa/0.2.1/}{0.2.1}
& September 24, 2024 \\
\href{https://typst.app/universe/package/silky-letter-insa/0.2.0/}{0.2.0}
& June 10, 2024 \\
\href{https://typst.app/universe/package/silky-letter-insa/0.1.0/}{0.1.0}
& March 23, 2024 \\
\end{longtable}

Typst GmbH did not create this template and cannot guarantee correct
functionality of this template or compatibility with any version of the
Typst compiler or app.


\section{Package List LaTeX/crudo.tex}
\title{typst.app/universe/package/crudo}

\phantomsection\label{banner}
\section{crudo}\label{crudo}

{ 0.1.1 }

Take slices from raw blocks

\phantomsection\label{readme}
Crudo allows conveniently working with \texttt{\ raw\ } blocks in terms
of individual lines. It allows you to e.g.

\begin{itemize}
\tightlist
\item
  filter lines by content
\item
  filter lines by range (slicing)
\item
  transform lines
\item
  join multiple raw blocks
\end{itemize}

While transforming the content, the original
\href{https://typst.app/docs/reference/text/raw/\#parameters}{parameters}
specified on the given raw block will be preserved.

\subsection{Getting Started}\label{getting-started}

The full version of this example can be found in
\href{https://github.com/typst/packages/raw/main/packages/preview/crudo/0.1.1/gallery/thumbnail.typ}{gallery/thumbnail.typ}
.

\begin{Shaded}
\begin{Highlighting}[]
\NormalTok{From}

\NormalTok{\#let preamble = \textasciigrave{}\textasciigrave{}\textasciigrave{}typ}
\NormalTok{\#import "@preview/crudo:0.1.0"}

\NormalTok{\textasciigrave{}\textasciigrave{}\textasciigrave{}}
\NormalTok{\#preamble}

\NormalTok{and}

\NormalTok{\#let example = \textasciigrave{}\textasciigrave{}\textasciigrave{}\textasciigrave{}typ}
\NormalTok{\#crudo.r2l(\textasciigrave{}\textasciigrave{}\textasciigrave{}c}
\NormalTok{int main() \{}
\NormalTok{  return 0;}
\NormalTok{\}}
\NormalTok{\textasciigrave{}\textasciigrave{}\textasciigrave{})}
\NormalTok{\textasciigrave{}\textasciigrave{}\textasciigrave{}\textasciigrave{}}
\NormalTok{\#example}

\NormalTok{we get}

\NormalTok{\#let full{-}example = crudo.join(preamble, example)}
\NormalTok{\#full{-}example}

\NormalTok{If you execute that, you get}

\NormalTok{\#eval(full{-}example.text, mode: "markup")}
\end{Highlighting}
\end{Shaded}

\pandocbounded{\includegraphics[keepaspectratio]{https://github.com/typst/packages/raw/main/packages/preview/crudo/0.1.1/thumbnail.png}}

\subsection{Usage}\label{usage}

See the
\href{https://github.com/typst/packages/raw/main/packages/preview/crudo/0.1.1/docs/manual.pdf}{manual}
for details.

\subsubsection{How to add}\label{how-to-add}

Copy this into your project and use the import as \texttt{\ crudo\ }

\begin{verbatim}
#import "@preview/crudo:0.1.1"
\end{verbatim}

\includesvg[width=0.16667in,height=0.16667in]{/assets/icons/16-copy.svg}

Check the docs for
\href{https://typst.app/docs/reference/scripting/\#packages}{more
information on how to import packages} .

\subsubsection{About}\label{about}

\begin{description}
\tightlist
\item[Author :]
\href{https://github.com/SillyFreak/}{Clemens Koza}
\item[License:]
MIT
\item[Current version:]
0.1.1
\item[Last updated:]
September 30, 2024
\item[First released:]
July 15, 2024
\item[Minimum Typst version:]
0.9.0
\item[Archive size:]
4.11 kB
\href{https://packages.typst.org/preview/crudo-0.1.1.tar.gz}{\pandocbounded{\includesvg[keepaspectratio]{/assets/icons/16-download.svg}}}
\item[Repository:]
\href{https://github.com/SillyFreak/typst-crudo}{GitHub}
\item[Categor ies :]
\begin{itemize}
\tightlist
\item[]
\item
  \pandocbounded{\includesvg[keepaspectratio]{/assets/icons/16-text.svg}}
  \href{https://typst.app/universe/search/?category=text}{Text}
\item
  \pandocbounded{\includesvg[keepaspectratio]{/assets/icons/16-code.svg}}
  \href{https://typst.app/universe/search/?category=scripting}{Scripting}
\item
  \pandocbounded{\includesvg[keepaspectratio]{/assets/icons/16-hammer.svg}}
  \href{https://typst.app/universe/search/?category=utility}{Utility}
\end{itemize}
\end{description}

\subsubsection{Where to report issues?}\label{where-to-report-issues}

This package is a project of Clemens Koza . Report issues on
\href{https://github.com/SillyFreak/typst-crudo}{their repository} . You
can also try to ask for help with this package on the
\href{https://forum.typst.app}{Forum} .

Please report this package to the Typst team using the
\href{https://typst.app/contact}{contact form} if you believe it is a
safety hazard or infringes upon your rights.

\phantomsection\label{versions}
\subsubsection{Version history}\label{version-history}

\begin{longtable}[]{@{}ll@{}}
\toprule\noalign{}
Version & Release Date \\
\midrule\noalign{}
\endhead
\bottomrule\noalign{}
\endlastfoot
0.1.1 & September 30, 2024 \\
\href{https://typst.app/universe/package/crudo/0.1.0/}{0.1.0} & July 15,
2024 \\
\end{longtable}

Typst GmbH did not create this package and cannot guarantee correct
functionality of this package or compatibility with any version of the
Typst compiler or app.


\section{Package List LaTeX/conchord.tex}
\title{typst.app/universe/package/conchord}

\phantomsection\label{banner}
\section{conchord}\label{conchord}

{ 0.2.0 }

Easily write lyrics with chords, generate chord diagrams and tabs.

{ } Featured Package

\phantomsection\label{readme}
\begin{quote}
Notice: I’m preparing the update, so the documentation there is
referring to the new version.
\end{quote}

\texttt{\ conchord\ } (concise chord) is a
\href{https://github.com/typst/typst}{Typst} package to write lyrics
with chords and generate colorful fretboard diagram (aka chord diagram).
From \texttt{\ 0.1.1\ } there is also experimental tabs support (though
quite simple and unstable yet). It is inspired by
\href{https://github.com/ljgago/typst-chords}{chordx} package and my
previous tiny project for generating chord diagrams SVG-s.

\texttt{\ conchord\ } makes it easy to add new chords, both for diagrams
and lyrics. Unlike \href{https://github.com/ljgago/typst-chords}{chordx}
, you don’t need to think about layout and pass lots of arrays for
drawing barres. Just pass a string with held frets and it will work:

\begin{Shaded}
\begin{Highlighting}[]
\NormalTok{\#import "@preview/conchord:0.2.0": new{-}chordgen, overchord}

\NormalTok{\#let chord = new{-}chordgen()}

\NormalTok{\#box(chord("x32010", name: "C"))}
\NormalTok{\#box(chord("x33222", name: "F\#m/C\#"))}
\NormalTok{\#box(chord("x,9,7,8,9,9"))}
\end{Highlighting}
\end{Shaded}

\pandocbounded{\includegraphics[keepaspectratio]{https://github.com/typst/packages/raw/main/packages/preview/conchord/0.2.0/examples/simple.png}}

\begin{quote}
\texttt{\ x\ } means closed string, \texttt{\ 0\ } is open, other number
is a fret. In case of frets larger than \texttt{\ 9\ } frets should be
separated with commas, otherwise you can list them without any
separators.
\end{quote}

\begin{quote}
Chord diagram works like a usual block, so to put them into one line you
need to wrap them into boxes. In real code it is recommended to create a
wrapper function to customize box margins etc (see larger example
below).
\end{quote}

It is easy to customize the colors and styles of chords with
\texttt{\ colors\ } argument and \texttt{\ show\ } rules for text. You
can also put \texttt{\ !\ } and \texttt{\ *\ } marks in the end of the
string to force diagram generation. \texttt{\ !\ } forces barre,
\texttt{\ *\ } removes it:

\begin{Shaded}
\begin{Highlighting}[]
\NormalTok{\#let custom{-}chord = new{-}chordgen(string{-}number: 3,}
\NormalTok{    colors: (shadow{-}barre: orange,}
\NormalTok{        grid: gray.darken(30\%),}
\NormalTok{        hold: red,}
\NormalTok{        barre: purple)}
\NormalTok{)}

\NormalTok{\#set text(fill: purple)}
\NormalTok{\#box(custom{-}chord("320", name: "C"))}
\NormalTok{\#box(custom{-}chord("2,4,4,*", name: "Bm"))}
\NormalTok{\#box(custom{-}chord("2,2,2, *"))}
\NormalTok{\#box(custom{-}chord("x,3,2, !"))}
\end{Highlighting}
\end{Shaded}

\pandocbounded{\includegraphics[keepaspectratio]{https://github.com/typst/packages/raw/main/packages/preview/conchord/0.2.0/examples/crazy.png}}

\begin{quote}
NOTE: be careful when using \textbf{!} , if barre cannot be used, it
will result into nonsense.
\end{quote}

For lyrics, you don’t need to add chord to word and specify the number
of char in words (unlike
\href{https://github.com/ljgago/typst-chords}{chordx} ). Simply add
\texttt{\ \#overchord\ } to the place you want a chord. Compose with
native Typst stylistic things for non-plain look (you don’t need to
dig into \href{https://github.com/ljgago/typst-chords}{chordx} ’s
custom arguments):

\begin{Shaded}
\begin{Highlighting}[]
\NormalTok{\#let och(it) = overchord(strong(it))}

\NormalTok{=== \#raw("[Verse 1]")}

\NormalTok{\#och[Em] Another head }
\NormalTok{\#och[C] hangs lowly \textbackslash{}}
\NormalTok{\#och[G] Child is slowly}
\NormalTok{\#och[D] taken}

\NormalTok{...}
\end{Highlighting}
\end{Shaded}

\begin{quote}
Complete example of lyrics with chords (see
\href{https://github.com/typst/packages/raw/main/packages/preview/conchord/0.2.0/examples/zombie.typ}{full
source} ):
\end{quote}

\pandocbounded{\includegraphics[keepaspectratio]{https://github.com/typst/packages/raw/main/packages/preview/conchord/0.2.0/examples/zombie.png}}

I was quite amazed with general idea of
\href{https://github.com/ljgago/typst-chords}{chordx} , but a bit
frustated with implementation, so I decided to quickly rewrite my old js
code to Typst. I use \texttt{\ cetz\ } there, so code is quite clean.

\begin{quote}
Note: This package doesn’t use any piece of
\href{https://github.com/ljgago/typst-chords}{chordx} , only the general
idea is taken.
\end{quote}

Brief comparison may be seen there, some concepts explained below:

\pandocbounded{\includegraphics[keepaspectratio]{https://github.com/typst/packages/raw/main/packages/preview/conchord/0.2.0/examples/compare.png}}

\subsection{Think about frets, not
layout}\label{think-about-frets-not-layout}

Write frets for chord as you hold it, like a string like “123456�
(see examples above). You don’t need to think about layouting and
subtracting frets, \texttt{\ conchord\ } does it for you.

\begin{quote}
NOTE: I can’t guarantee that will be the best chord layout. Moreover,
the logic is quite simple: e.g., barre can’t be multiple and can’t
be put anywhere except first bar in the image. However, surprisingly, it
works well in almost all of the common cases, so the exceptions are
really rare.
\end{quote}

If you need to create something too \emph{custom/complex} \st{(but not
\emph{concise} )} , maybe it is worth to try
\href{https://github.com/ljgago/typst-chords}{chordx} . You can also try
using core function \texttt{\ render-chord\ } for more manual~control,
but it is still limited by one barre starting from one (but that barre
may be shifted). If you think that feature should be supported, you can
create issue there.

\subsection{Shadow barre}\label{shadow-barre}

Some chord generators put barre only where it \emph{ought to} be (any
less will not hold some strings). Others put it where it can be
(sometimes maximal size, sometimes some other logic). I use simple barre
where it \textbf{ought to} be, and add \emph{shadow barre} where it
\textbf{could} maximally be. You can easily disable it by either setting
\texttt{\ use-shadow-barre:\ false\ } on \texttt{\ new-chordgen\ } (only
necessary part of barre rendered) or by setting color of
\texttt{\ shadow-barre\ } the same as \texttt{\ barre\ } (maximal
possible barre).

\subsection{Name auto-scaling}\label{name-auto-scaling}

Chord name font size is \emph{reduced} for \emph{large} chord names, so
the name fits well into chord diagram (see example above). That makes it
much more pretty to stack several chords together. To achieve
chordx-like behavior, you can always use
\texttt{\ \#figure(chord("…"),\ caption:\ …)\ } .

\subsection{Easier chords for lyrics}\label{easier-chords-for-lyrics}

Just add chord labels above lyrics in arbitrary place, don’t think
about what letter exactly it should be located. By default it aligns the
chord label to the left, so it produces pretty results out-of-box. You
can pass other alignments to \texttt{\ alignment\ } argument, or use the
chords straight inside words.

The command is \emph{much} simpler than chordx (of course, it is a
trade-off):

\begin{Shaded}
\begin{Highlighting}[]
\NormalTok{\#let overchord(body, align: start, height: 1em, width: {-}0.25em) = box(place(align, body), height: 1em + height, width: width)}
\end{Highlighting}
\end{Shaded}

Feel free to use it for your purposes outside of the package.

It takes on default \texttt{\ -0.25em\ } width to remove one adjacent
space, so

\begin{itemize}
\tightlist
\item
  To make it work on monospace/other special fonts, you will need to
  adjust \texttt{\ width\ } argument. The problem is that I can’t
  \texttt{\ measure\ } space, but maybe that will be eventually fixed.
\item
  To add chord inside word, you have to add \emph{one} space, like
  \texttt{\ wo\ \#chord{[}Am{]}rd\ } .
\end{itemize}

\subsection{Colors}\label{colors}

Customize the colors of chord elements. \texttt{\ new-chordgen\ }
accepts the \texttt{\ colors\ } dictionary with following possible
fields:

\begin{itemize}
\tightlist
\item
  \texttt{\ grid\ } : color of grid, default is
  \texttt{\ gray.darken(20\%)\ }
\item
  \texttt{\ open\ } : color of circles for open strings, default is
  \texttt{\ black\ }
\item
  \texttt{\ muted\ } : color of crosses for muted strings, default is
  \texttt{\ black\ }
\item
  \texttt{\ hold\ } : color of held positions, default is
  \texttt{\ \#5d6eaf\ }
\item
  \texttt{\ barre\ } : color of main barre part, default is
  \texttt{\ \#5d6eaf\ }
\item
  \texttt{\ shadow-barre\ } : color of “unnecessary� barre part,
  default is \texttt{\ \#5d6eaf.lighten(30\%)\ }
\end{itemize}

\subsubsection{Customizing text}\label{customizing-text}

\textbf{Important} : \emph{frets} are rendered using \texttt{\ raw\ }
elements. So if you want to customize their font or color, please use
\texttt{\ \#show\ raw:\ set\ text(fill:\ ...)\ } or similar things.

The chord’s name, on the other hand, uses default font, so to set it,
just use \texttt{\ \#set\ text(font:\ ...)\ } in the corresponding
scope.

\subsection{Assertions}\label{assertions}

Currently \href{https://github.com/ljgago/typst-chords}{chordx} has
almost no checks inside for correctness of passed chords.
\texttt{\ conchord\ } , on the other side, checks for

\begin{itemize}
\tightlist
\item
  Number of passed\&parsed frets equal to set string-number
\item
  Only numbers and \texttt{\ x\ } passed as frets
\item
  All frets fitting in the diagram
\end{itemize}

\begin{quote}
Everything there is highly experimental and unstable
\end{quote}

\pandocbounded{\includegraphics[keepaspectratio]{https://github.com/typst/packages/raw/main/packages/preview/conchord/0.2.0/examples/tabs.png}}

\begin{Shaded}
\begin{Highlighting}[]
\NormalTok{\#let chord = new{-}chordgen(scale{-}length: 0.6pt)}

\NormalTok{\#let ending(n) = \{}
\NormalTok{    rect(stroke: (left: black, top: black), inset: 0.2em, n + h(3em))}
\NormalTok{    v(0.5em)}
\NormalTok{\}}
\NormalTok{*This thing doesn\textquotesingle{}t follow musical notation rules, it is used just for demonstration purposes*:}

\NormalTok{\#tabs.new(\textasciigrave{}\textasciigrave{}\textasciigrave{}}
\NormalTok{2/4 2/4{-}3 2/4{-}2 2/4{-}3 |}
\NormalTok{2/4{-}2 2/4{-}3 2/4 2/4 2/4 |}
\NormalTok{2/4{-}2 p 0/2{-}3 3/2{-}2}
\NormalTok{|:}

\NormalTok{0/1+0/6 0/1 0/1{-}3 2/1 | 3/1+3/5{-}2 3/1 3/1{-}3 5/1 | 2/1+0/4{-}2 2/1 0/1{-}3 3/2{-}3 | \textbackslash{} \textbackslash{}}
\NormalTok{3/2{-}2 \textasciigrave{}5/2{-}3}
\NormalTok{p{-}2}
\NormalTok{\#\#}
\NormalTok{  chord("022000", name: "Em")}
\NormalTok{\#\#south}
\NormalTok{0/2{-}3 3/2 | | \#\# [...] \#\# p{-}4. | | 7/1{-}3 0/1{-}2 p{-}3 0/1 3/1 }

\NormalTok{\#\#}
\NormalTok{    ending[1.]}
\NormalTok{\#\#west}

\NormalTok{|}
\NormalTok{2/1{-}3}
\NormalTok{2/1}
\NormalTok{3/1 0/1 2/1{-}2 p{-}3 0/2{-}3 3/2{-}3}
\NormalTok{\#\#}
\NormalTok{  ending[2.]}
\NormalTok{\#\#west}
\NormalTok{|}
\NormalTok{2/1{-}2 2/1 0/1{-}3 3/2 :| 0/6{-}2 | \^{}0/6{-}2 || \textbackslash{}}
\NormalTok{1/1 2/1 2/2 2/2 2/3 2/3 4/4 4/4 4/4 4/4 4/4 4/4 2/3 2/3 2/3 2/3  2/3 2/3 2/3 2/3 2/3 2/3 2/3 2/3 2/3}
\NormalTok{\#\#}
\NormalTok{[notice there is no manual break]}
\NormalTok{\#\#east}
\NormalTok{| 2/3 2/3 8/3 7/3 6/3 5/3 4/3 2/3  5/3 8/3 9/3  7/3 2/3 | 2/3 2/2 2/3 2/4 |}
\NormalTok{10/1{-}3 10/1{-}3 10/1{-}3 10/1{-}4 10/1{-}4 10/1{-}4 10/1{-}4 10/1{-}5. 10/1{-}5. 10/1{-}5 10/1{-}5 10/1{-}2 \textbackslash{}}
\NormalTok{1/3bfullr+2/5{-}2 1/2b1/2{-}1 2/3v{-}1}
\NormalTok{\textasciigrave{}\textasciigrave{}\textasciigrave{}, eval{-}scope: (chord: chord, ending: ending)}
\NormalTok{ )}


\NormalTok{Not a lot customization is available yet, but something is already possible:}

\NormalTok{\#show raw: set text(red.darken(30\%), font: "Comic Sans MS")}

\NormalTok{\#tabs.new("0/1+2/5{-}1 \^{}0/1+\textasciigrave{}3/5{-}2.. 2/3 |: 2/3{-}1 2/3 2/3 | 3/3 ||",}
\NormalTok{  scale{-}length: 0.2cm,}
\NormalTok{  one{-}beat{-}length: 12,}
\NormalTok{  s{-}num: 5,}
\NormalTok{  colors: (}
\NormalTok{    lines: gradient.linear(yellow, blue),}
\NormalTok{    bars: green,}
\NormalTok{    connects: red}
\NormalTok{  ),}
\NormalTok{  enable{-}scale: false}
\NormalTok{)}
\end{Highlighting}
\end{Shaded}

As you can see from example, you can use raw strings or code blocks to
write tabs, there is no real difference.

The general idea is very simple: to write a number on some line, write
\texttt{\ \textless{}fret\ number\textgreater{}/\textless{}string\textgreater{}\ }
.

\textbf{Spaces are important!} All notes and special symbols work well
only if properly separated.

\subsubsection{Duration}\label{duration}

By default they will be quarter notes. To change that, you have to
specify the duration:
\texttt{\ \textless{}fret\textgreater{}/\textless{}string\textgreater{}-\textless{}duration\textgreater{}\ }
, where duration is \$log\_2\$ from note duration. So a whole note will
be \texttt{\ -0\ } , a half: \texttt{\ -1\ } and so on. You can also use
as many dots as you want to multiply duration by 1.5, e.g.
\texttt{\ -2.\ }

Once you change the duration, all the following notes will use it, so
you have to specify duration every time it is changed (basically,
always, but it really depends on composition). Of course, you can just
ignore all that duration staff.

\subsubsection{Bars and repetitions}\label{bars-and-repetitions}

To add simple bar, just add \texttt{\ \textbar{}\ } . To add double bar
line, use \texttt{\ \textbar{}\ \textbar{}\ } . To add end
movement/composition, add \texttt{\ \textbar{}\textbar{}\ } . To add
repetitions, use \texttt{\ \textbar{}:\ } and \texttt{\ :\textbar{}\ }
respectively.

\subsubsection{Linebreaks}\label{linebreaks}

Notes and bars that don’t fit in line will be automatically moved to
next. However, sometimes it isn’t ideal and may be a bit bugged, so it
is recommended to do that manually, using \texttt{\ \textbackslash{}\ }
.

The line is autoscaled if it is possible and not too ugly. You can
change the maximum and minimum scaling size with \texttt{\ scale-max\ }
and \texttt{\ scale-min\ } . It is also possible to completely disable
scaling with \texttt{\ enable-scale:\ false\ } .

\subsubsection{Ties and slides}\label{ties-and-slides}

You can \emph{tie} notes or \emph{slide} between them. To use ties, you
have to add \texttt{\ \^{}\ } in front of \emph{second} tied note, like
\texttt{\ 1/1\ \^{}3/1\ } . To use slides you have to do the same, but
with `.

\emph{Current limitation:} tying and sliding works only on the same
string and may work really bad if tied/slided through line break.

\subsection{Bends and vibratos}\label{bends-and-vibratos}

Add \texttt{\ b\ } after note, but before the duration (e.g.
\texttt{\ 2/3b-2\ } ) to add a bend. After \texttt{\ b\ } you can write
custom text to be written on top (for example, \texttt{\ b1/2\ } ). Add
\texttt{\ r\ } to the end to add a release.

Adding vibratos works the same way, via adding \texttt{\ v\ } to the
note. The length of vibrato will be the same as the length of the note.

Unfortunately, they are all supported things for now. But wait, there is
still one cool thing left!

\subsubsection{Custom content}\label{custom-content}

Add any typst code you want between \texttt{\ \#\#\ …\ \#\#\ } . It
will be rendered with \texttt{\ cetz\ } on top of the line where you
wrote it. That means you can write \emph{lyrics, chords, add complex
things like endings} , even \textbf{draw the elements that are still
missing} (well, it is still worth to create issue there, I will try to
do something).

That code is evaluated with \texttt{\ eval\ } , so you will need to pass
dictionary to \texttt{\ eval-scope\ } with all things you want to use.

You can set align of these elements by writing cetz anchors after the
second (e.g., \texttt{\ west\ } , \texttt{\ south\ } and their
combinations, like \texttt{\ west-south\ } ).

Additionally, if you enjoy drawing missing things, you can also use
\texttt{\ preamble\ } and \texttt{\ extra\ } arguments in
\texttt{\ tabs.new\ } where you can put any \texttt{\ cetz\ } inner
things (tabs uses canvas, and that allow you drawing on it) before or
after the tabs are drawn.

\subsubsection{Plans}\label{plans}

\begin{enumerate}
\tightlist
\item
  Add \emph{(optional)} “rhythm section� under tabs
\item
  Add more signs\&lines
\item
  Add more built-in things to attach above tabs
\end{enumerate}

It is far from what I want to do, so maybe there will be much more! I
will be very glad to receive \emph{any feedback} , both issues and PR-s
are very welcome (though I can’t promise I will be able to work on it
immediately)!

\subsubsection{How to add}\label{how-to-add}

Copy this into your project and use the import as \texttt{\ conchord\ }

\begin{verbatim}
#import "@preview/conchord:0.2.0"
\end{verbatim}

\includesvg[width=0.16667in,height=0.16667in]{/assets/icons/16-copy.svg}

Check the docs for
\href{https://typst.app/docs/reference/scripting/\#packages}{more
information on how to import packages} .

\subsubsection{About}\label{about}

\begin{description}
\tightlist
\item[Author :]
sitandr
\item[License:]
MIT
\item[Current version:]
0.2.0
\item[Last updated:]
February 6, 2024
\item[First released:]
July 24, 2023
\item[Minimum Typst version:]
0.8.0
\item[Archive size:]
12.8 kB
\href{https://packages.typst.org/preview/conchord-0.2.0.tar.gz}{\pandocbounded{\includesvg[keepaspectratio]{/assets/icons/16-download.svg}}}
\item[Repository:]
\href{https://github.com/sitandr/conchord}{GitHub}
\end{description}

\subsubsection{Where to report issues?}\label{where-to-report-issues}

This package is a project of sitandr . Report issues on
\href{https://github.com/sitandr/conchord}{their repository} . You can
also try to ask for help with this package on the
\href{https://forum.typst.app}{Forum} .

Please report this package to the Typst team using the
\href{https://typst.app/contact}{contact form} if you believe it is a
safety hazard or infringes upon your rights.

\phantomsection\label{versions}
\subsubsection{Version history}\label{version-history}

\begin{longtable}[]{@{}ll@{}}
\toprule\noalign{}
Version & Release Date \\
\midrule\noalign{}
\endhead
\bottomrule\noalign{}
\endlastfoot
0.2.0 & February 6, 2024 \\
\href{https://typst.app/universe/package/conchord/0.1.1/}{0.1.1} &
September 19, 2023 \\
\href{https://typst.app/universe/package/conchord/0.1.0/}{0.1.0} & July
24, 2023 \\
\end{longtable}

Typst GmbH did not create this package and cannot guarantee correct
functionality of this package or compatibility with any version of the
Typst compiler or app.


\section{Package List LaTeX/ibanator.tex}
\title{typst.app/universe/package/ibanator}

\phantomsection\label{banner}
\section{ibanator}\label{ibanator}

{ 0.1.0 }

A package for validating and formatting International Bank Account
Numbers (IBANs) according to ISO 13616-1.

\phantomsection\label{readme}
\begin{quote}
Validate and format IBAN numbers according to ISO 13616-1.
\end{quote}

\subsection{Usage}\label{usage}

\begin{Shaded}
\begin{Highlighting}[]
\NormalTok{\#import "@preview/ibanator:0.1.0": iban}

\NormalTok{\#iban("DE89370400440532013000")}
\end{Highlighting}
\end{Shaded}

\includegraphics[width=3.64583in,height=\textheight,keepaspectratio]{https://github.com/typst/packages/raw/main/packages/preview/ibanator/0.1.0/docs/example.png}

To disable validation, set the \texttt{\ validate\ } flag to false:

\begin{Shaded}
\begin{Highlighting}[]
\NormalTok{\#iban("DE89370400440532013000", validate: false)}
\end{Highlighting}
\end{Shaded}

\subsubsection{How to add}\label{how-to-add}

Copy this into your project and use the import as \texttt{\ ibanator\ }

\begin{verbatim}
#import "@preview/ibanator:0.1.0"
\end{verbatim}

\includesvg[width=0.16667in,height=0.16667in]{/assets/icons/16-copy.svg}

Check the docs for
\href{https://typst.app/docs/reference/scripting/\#packages}{more
information on how to import packages} .

\subsubsection{About}\label{about}

\begin{description}
\tightlist
\item[Author :]
@mainrs
\item[License:]
EUPL-1.2
\item[Current version:]
0.1.0
\item[Last updated:]
April 8, 2024
\item[First released:]
April 8, 2024
\item[Archive size:]
19.0 kB
\href{https://packages.typst.org/preview/ibanator-0.1.0.tar.gz}{\pandocbounded{\includesvg[keepaspectratio]{/assets/icons/16-download.svg}}}
\item[Repository:]
\href{https://github.com/mainrs/typst-iban-formatter.git}{GitHub}
\item[Categor y :]
\begin{itemize}
\tightlist
\item[]
\item
  \pandocbounded{\includesvg[keepaspectratio]{/assets/icons/16-text.svg}}
  \href{https://typst.app/universe/search/?category=text}{Text}
\end{itemize}
\end{description}

\subsubsection{Where to report issues?}\label{where-to-report-issues}

This package is a project of @mainrs . Report issues on
\href{https://github.com/mainrs/typst-iban-formatter.git}{their
repository} . You can also try to ask for help with this package on the
\href{https://forum.typst.app}{Forum} .

Please report this package to the Typst team using the
\href{https://typst.app/contact}{contact form} if you believe it is a
safety hazard or infringes upon your rights.

\phantomsection\label{versions}
\subsubsection{Version history}\label{version-history}

\begin{longtable}[]{@{}ll@{}}
\toprule\noalign{}
Version & Release Date \\
\midrule\noalign{}
\endhead
\bottomrule\noalign{}
\endlastfoot
0.1.0 & April 8, 2024 \\
\end{longtable}

Typst GmbH did not create this package and cannot guarantee correct
functionality of this package or compatibility with any version of the
Typst compiler or app.


\section{Package List LaTeX/i-figured.tex}
\title{typst.app/universe/package/i-figured}

\phantomsection\label{banner}
\section{i-figured}\label{i-figured}

{ 0.2.4 }

Configurable figure and equation numbering per section.

\phantomsection\label{readme}
Configurable figure numbering per section.

\subsection{Examples}\label{examples}

\subsubsection{Basic}\label{basic}

Have a look at the source
\href{https://github.com/typst/packages/raw/main/packages/preview/i-figured/0.2.4/examples/basic.typ}{here}
.

\pandocbounded{\includegraphics[keepaspectratio]{https://github.com/typst/packages/raw/main/packages/preview/i-figured/0.2.4/examples/basic.png}}

\subsubsection{Two levels deep}\label{two-levels-deep}

Have a look at the source
\href{https://github.com/typst/packages/raw/main/packages/preview/i-figured/0.2.4/examples/level-two.typ}{here}
.

\pandocbounded{\includegraphics[keepaspectratio]{https://github.com/typst/packages/raw/main/packages/preview/i-figured/0.2.4/examples/level-two.png}}

\subsection{Usage}\label{usage}

The package mainly consists of two customizable show rules, which set up
all the numbering. There is also an additional function to make showing
an outline of figures easier.

Because the
\href{https://github.com/typst/packages/raw/main/packages/preview/i-figured/0.2.4/\#show-figure}{\texttt{\ show-figure()\ }}
function must internally create another figure element, attached labels
cannot directly be used for references. To circumvent this, a new label
is attached to the internal figure, with the same name but prefixed with
\texttt{\ fig:\ } , \texttt{\ tbl:\ } , or \texttt{\ lst:\ } for images
(and all other types of generic figures), tables, and raw code figures
(aka listings) respectively. These new labels can be used for
referencing without problems.

\begin{Shaded}
\begin{Highlighting}[]
\NormalTok{// import the package}
\NormalTok{\#import "@preview/i{-}figured:0.2.4"}

\NormalTok{// make sure you have some heading numbering set}
\NormalTok{\#set heading(numbering: "1.")}

\NormalTok{// apply the show rules (these can be customized)}
\NormalTok{\#show heading: i{-}figured.reset{-}counters}
\NormalTok{\#show figure: i{-}figured.show{-}figure}

\NormalTok{// show an outline}
\NormalTok{\#i{-}figured.outline()}

\NormalTok{= Hello World}

\NormalTok{\#figure([hi], caption: [Bye World.]) \textless{}bye\textgreater{}}

\NormalTok{// when referencing, the label names must be prefixed with \textasciigrave{}fig:\textasciigrave{}, \textasciigrave{}tbl:\textasciigrave{},}
\NormalTok{// or \textasciigrave{}lst:\textasciigrave{} depending on the figure kind.}
\NormalTok{@fig:bye displays the text "hi".}
\end{Highlighting}
\end{Shaded}

\subsection{Reference}\label{reference}

\subsubsection{\texorpdfstring{\texttt{\ reset-counters\ }}{ reset-counters }}\label{reset-counters}

Reset all figure counters. To be used in a heading show rule like
\texttt{\ \#show\ heading:\ i-figured.reset-counters\ } .

\begin{Shaded}
\begin{Highlighting}[]
\NormalTok{\#let reset{-}counters(}
\NormalTok{  it,}
\NormalTok{  level: 1,}
\NormalTok{  extra{-}kinds: (),}
\NormalTok{  equations: true,}
\NormalTok{  return{-}orig{-}heading: true,}
\NormalTok{) = \{ .. \}}
\end{Highlighting}
\end{Shaded}

\textbf{Arguments:}

\begin{itemize}
\tightlist
\item
  \texttt{\ it\ } :
  \href{https://typst.app/docs/reference/foundations/content/}{\texttt{\ content\ }}
  â€'' The heading element from the show rule.
\item
  \texttt{\ level\ } :
  \href{https://typst.app/docs/reference/foundations/int/}{\texttt{\ int\ }}
  â€'' At which heading level to reset the counters. A value of
  \texttt{\ 2\ } will cause the counters to be reset at level two
  \emph{and} level one headings.
\item
  \texttt{\ extra-kinds\ } :
  \href{https://typst.app/docs/reference/foundations/array/}{\texttt{\ array\ }}
  of (
  \href{https://typst.app/docs/reference/foundations/str/}{\texttt{\ str\ }}
  or
  \href{https://typst.app/docs/reference/foundations/function/}{\texttt{\ function\ }}
  ) â€'' Additional custom figure kinds. If you have any figures with a
  \texttt{\ kind\ } other than \texttt{\ image\ } , \texttt{\ table\ } ,
  or \texttt{\ raw\ } , you must add the \texttt{\ kind\ } here for its
  counter to be reset.
\item
  \texttt{\ equations\ } :
  \href{https://typst.app/docs/reference/foundations/bool/}{\texttt{\ bool\ }}
  â€'' Whether the counter for math equations should be reset.
\item
  \texttt{\ return-orig-heading\ } :
  \href{https://typst.app/docs/reference/foundations/bool/}{\texttt{\ bool\ }}
  â€'' Whether the original heading element should be included in the
  returned content. Set this to false if you manually want to construct
  a heading instead of using the default.
\end{itemize}

\textbf{Returns:}
\href{https://typst.app/docs/reference/foundations/content/}{\texttt{\ content\ }}
â€'' The unmodified heading.

\subsubsection{\texorpdfstring{\texttt{\ show-figure\ }}{ show-figure }}\label{show-figure}

Show a figure with per-section numbering. To be used in a figure show
rule like \texttt{\ \#show\ figure:\ i-figured.show-figure\ } .

\begin{Shaded}
\begin{Highlighting}[]
\NormalTok{\#let show{-}figure(}
\NormalTok{  it,}
\NormalTok{  level: 1,}
\NormalTok{  zero{-}fill: true,}
\NormalTok{  leading{-}zero: true,}
\NormalTok{  numbering: "1.1",}
\NormalTok{  extra{-}prefixes: (:),}
\NormalTok{  fallback{-}prefix: "fig:",}
\NormalTok{) = \{ .. \}}
\end{Highlighting}
\end{Shaded}

\textbf{Arguments:}

\begin{itemize}
\tightlist
\item
  \texttt{\ it\ } :
  \href{https://typst.app/docs/reference/foundations/content/}{\texttt{\ content\ }}
  â€'' The figure element from the show rule.
\item
  \texttt{\ level\ } :
  \href{https://typst.app/docs/reference/foundations/int/}{\texttt{\ int\ }}
  â€'' How many levels of the current heading counter should be added in
  front. Note that you can control this individually from the
  \texttt{\ level\ } parameter on
  \href{https://github.com/typst/packages/raw/main/packages/preview/i-figured/0.2.4/\#reset-counters}{\texttt{\ reset-counters()\ }}
  .
\item
  \texttt{\ zero-fill\ } :
  \href{https://typst.app/docs/reference/foundations/bool/}{\texttt{\ bool\ }}
  â€'' If \texttt{\ true\ } and assuming a \texttt{\ level\ } of
  \texttt{\ 2\ } , a figure after a \texttt{\ 1.\ } heading but before a
  \texttt{\ 1.1.\ } heading will show \texttt{\ 1.0.1\ } as numbering,
  else the middle zero is excluded. Note that if set to
  \texttt{\ false\ } , not all figure numberings are guaranteed to have
  the same length.
\item
  \texttt{\ leading-zero\ } :
  \href{https://typst.app/docs/reference/foundations/bool/}{\texttt{\ bool\ }}
  â€'' Whether figures before the first top-level heading should have a
  leading \texttt{\ 0\ } . Note that if set to \texttt{\ false\ } , not
  all figure numberings are guaranteed to have the same length.
\item
  \texttt{\ numbering\ } :
  \href{https://typst.app/docs/reference/foundations/str/}{\texttt{\ str\ }}
  or
  \href{https://typst.app/docs/reference/foundations/function/}{\texttt{\ function\ }}
  â€'' The actual numbering pattern to use for the figures.
\item
  \texttt{\ extra-prefixes\ } :
  \href{https://typst.app/docs/reference/foundations/dictionary/}{\texttt{\ dictionary\ }}
  of
  \href{https://typst.app/docs/reference/foundations/str/}{\texttt{\ str\ }}
  to
  \href{https://typst.app/docs/reference/foundations/str/}{\texttt{\ str\ }}
  pairs â€'' Additional label prefixes. This can optionally be used to
  specify prefixes for custom figure kinds, otherwise they will also use
  the fallback prefix.
\item
  \texttt{\ fallback-prefix\ } :
  \href{https://typst.app/docs/reference/foundations/str/}{\texttt{\ str\ }}
  â€'' The label prefix to use for figure kinds which don’t have
  another prefix set.
\end{itemize}

\textbf{Returns:}
\href{https://typst.app/docs/reference/foundations/content/}{\texttt{\ content\ }}
â€'' The modified figure.

\subsubsection{\texorpdfstring{\texttt{\ show-equation\ }}{ show-equation }}\label{show-equation}

Show a math equation with per-section numbering. To be used in a show
rule like \texttt{\ \#show\ math.equation:\ i-figured.show-equation\ } .

\begin{Shaded}
\begin{Highlighting}[]
\NormalTok{\#let show{-}equation(}
\NormalTok{  it,}
\NormalTok{  level: 1,}
\NormalTok{  zero{-}fill: true,}
\NormalTok{  leading{-}zero: true,}
\NormalTok{  numbering: "(1.1)",}
\NormalTok{  prefix: "eqt:",}
\NormalTok{  only{-}labeled: false,}
\NormalTok{  unnumbered{-}label: "{-}",}
\NormalTok{) = \{ .. \}}
\end{Highlighting}
\end{Shaded}

\textbf{Arguments:}

For the arguments \texttt{\ level\ } , \texttt{\ zero-fill\ } ,
\texttt{\ leading-zero\ } , and \texttt{\ numbering\ } refer to
\href{https://github.com/typst/packages/raw/main/packages/preview/i-figured/0.2.4/\#show-figure}{\texttt{\ show-figure()\ }}
.

\begin{itemize}
\tightlist
\item
  \texttt{\ it\ } :
  \href{https://typst.app/docs/reference/foundations/content/}{\texttt{\ content\ }}
  â€'' The equation element from the show rule.
\item
  \texttt{\ prefix\ } :
  \href{https://typst.app/docs/reference/foundations/str/}{\texttt{\ str\ }}
  â€'' The label prefix to use for all equations.
\item
  \texttt{\ only-labeled\ } :
  \href{https://typst.app/docs/reference/foundations/bool/}{\texttt{\ bool\ }}
  â€'' Whether only equations with labels should be numbered.
\item
  \texttt{\ unnumbered-label\ } :
  \href{https://typst.app/docs/reference/foundations/str/}{\texttt{\ str\ }}
  â€'' A label to explicitly disable numbering for an equation.
\end{itemize}

\textbf{Returns:}
\href{https://typst.app/docs/reference/foundations/content/}{\texttt{\ content\ }}
â€'' The modified equation.

\subsubsection{\texorpdfstring{\texttt{\ outline\ }}{ outline }}\label{outline}

Show the outline for a kind of figure. This is just the same as calling
\texttt{\ outline(target:\ figure.where(kind:\ i-figured.\_prefix\ +\ repr(target-kind)),\ ..)\ }
, the function just exists for convenience and clarity.

\begin{Shaded}
\begin{Highlighting}[]
\NormalTok{\#let outline(target{-}kind: image, title: [List of Figures], ..args) = \{ .. \}}
\end{Highlighting}
\end{Shaded}

\textbf{Arguments:}

\begin{itemize}
\tightlist
\item
  \texttt{\ target-kind\ } :
  \href{https://typst.app/docs/reference/foundations/str/}{\texttt{\ str\ }}
  or
  \href{https://typst.app/docs/reference/foundations/function/}{\texttt{\ function\ }}
  â€'' Which kind of figure to list.
\item
  \texttt{\ title\ } :
  \href{https://typst.app/docs/reference/foundations/content/}{\texttt{\ content\ }}
  or \texttt{\ none\ } â€'' The title of the outline.
\item
  \texttt{\ ..args\ } â€'' Other arguments to pass to the underlying
  \href{https://typst.app/docs/reference/meta/outline/}{\texttt{\ outline()\ }}
  call.
\end{itemize}

\textbf{Returns:}
\href{https://typst.app/docs/reference/foundations/content/}{\texttt{\ content\ }}
â€'' The outline element.

\subsection{Acknowledgements}\label{acknowledgements}

The core code is based off code from
\href{https://github.com/PgBiel}{@PgBiel} ( \texttt{\ @PgSuper\ } on
Discord) and \href{https://github.com/aagolovanov}{@aagolovanov} (
\texttt{\ @aag.\ } on Discord). Specifically from
\href{https://discord.com/channels/1054443721975922748/1088371919725793360/1158534418760224809}{this
message} and the conversation around
\href{https://discord.com/channels/1054443721975922748/1088371919725793360/1159172567282749561}{here}
.

\subsubsection{How to add}\label{how-to-add}

Copy this into your project and use the import as \texttt{\ i-figured\ }

\begin{verbatim}
#import "@preview/i-figured:0.2.4"
\end{verbatim}

\includesvg[width=0.16667in,height=0.16667in]{/assets/icons/16-copy.svg}

Check the docs for
\href{https://typst.app/docs/reference/scripting/\#packages}{more
information on how to import packages} .

\subsubsection{About}\label{about}

\begin{description}
\tightlist
\item[Author :]
RubixDev
\item[License:]
MIT
\item[Current version:]
0.2.4
\item[Last updated:]
January 26, 2024
\item[First released:]
October 9, 2023
\item[Archive size:]
1.97 kB
\href{https://packages.typst.org/preview/i-figured-0.2.4.tar.gz}{\pandocbounded{\includesvg[keepaspectratio]{/assets/icons/16-download.svg}}}
\item[Repository:]
\href{https://github.com/RubixDev/typst-i-figured}{GitHub}
\end{description}

\subsubsection{Where to report issues?}\label{where-to-report-issues}

This package is a project of RubixDev . Report issues on
\href{https://github.com/RubixDev/typst-i-figured}{their repository} .
You can also try to ask for help with this package on the
\href{https://forum.typst.app}{Forum} .

Please report this package to the Typst team using the
\href{https://typst.app/contact}{contact form} if you believe it is a
safety hazard or infringes upon your rights.

\phantomsection\label{versions}
\subsubsection{Version history}\label{version-history}

\begin{longtable}[]{@{}ll@{}}
\toprule\noalign{}
Version & Release Date \\
\midrule\noalign{}
\endhead
\bottomrule\noalign{}
\endlastfoot
0.2.4 & January 26, 2024 \\
\href{https://typst.app/universe/package/i-figured/0.2.3/}{0.2.3} &
December 11, 2023 \\
\href{https://typst.app/universe/package/i-figured/0.2.2/}{0.2.2} &
December 7, 2023 \\
\href{https://typst.app/universe/package/i-figured/0.2.1/}{0.2.1} &
November 19, 2023 \\
\href{https://typst.app/universe/package/i-figured/0.2.0/}{0.2.0} &
November 16, 2023 \\
\href{https://typst.app/universe/package/i-figured/0.1.0/}{0.1.0} &
October 9, 2023 \\
\end{longtable}

Typst GmbH did not create this package and cannot guarantee correct
functionality of this package or compatibility with any version of the
Typst compiler or app.


\section{Package List LaTeX/modern-sysu-thesis.tex}
\title{typst.app/universe/package/modern-sysu-thesis}

\phantomsection\label{banner}
\phantomsection\label{template-thumbnail}
\pandocbounded{\includegraphics[keepaspectratio]{https://packages.typst.org/preview/thumbnails/modern-sysu-thesis-0.3.0-small.webp}}

\section{modern-sysu-thesis}\label{modern-sysu-thesis}

{ 0.3.0 }

中山大学学ä½?论æ--‡ Typst 模æ?¿ï¼ŒA Typst template for SYSU thesis

\href{/app?template=modern-sysu-thesis&version=0.3.0}{Create project in
app}

\phantomsection\label{readme}
\href{https://gitlab.com/sysu-gitlab/thesis-template/better-thesis/-/releases}{\pandocbounded{\includegraphics[keepaspectratio]{https://gitlab.com/sysu-gitlab/thesis-template/better-thesis/-/badges/release.svg?style=flat-square&value_width=100}}}
\href{https://github.com/sysu/better-thesis}{\pandocbounded{\includegraphics[keepaspectratio]{https://img.shields.io/github/stars/sysu/better-thesis.svg?style=social&label=Star&maxAge=2592000}}}

\textbf{\href{https://typst.app/app?template=modern-sysu-thesis&version=0.1.1}{点击此处注册
typst.app 并创建ä½~的论æ--‡å·¥ç¨‹}}

本ç§`ç''Ÿæ¨¡æ?¿å·²ç»?符å?ˆå­¦ä½?论æ--‡æ~¼å¼?è¦?求(
\href{https://gitlab.com/sysu-gitlab/thesis-template/better-thesis/-/issues/6}{\#6}
),欢迎å?Œå­¦/æ~¡å?‹ä»¬
\href{https://gitlab.com/sysu-gitlab/thesis-template/better-thesis/-/merge_requests}{贡献代ç~?}
/å??馈é---®é¢˜ï¼ˆ
\href{https://gitlab.com/sysu-gitlab/thesis-template/better-thesis/-/issues}{GitLab
issue} /
\href{mailto:contact-project+sysu-gitlab-thesis-template-better-thesis-57823416-issue-@incoming.gitlab.com}{邮件}
)�

模�交� QQ 群:
\href{https://jq.qq.com/?_wv=1027&k=m58va1kd}{797942860}

\subsection{�考规范}\label{uxe5uxe8ux192uxe8uxe8ux153ux192}

\begin{itemize}
\tightlist
\item
  本ç§`ç''Ÿè®ºæ--‡æ¨¡æ?¿å?‚考
  \href{https://spa.sysu.edu.cn/zh-hans/article/1744}{中山大学本ç§`ç''Ÿæ¯•ä¸šè®ºæ--‡ï¼ˆè®¾è®¡ï¼‰å†™ä½œä¸Žå?°åˆ¶è§„范
  2020å¹´å?{}`}
\item
  ç~''究ç''Ÿè®ºæ--‡æ¨¡æ?¿å?‚考
  \href{https://graduate.sysu.edu.cn/sites/graduate.prod.dpcms4.sysu.edu.cn/files/2019-04/\%E4\%B8\%AD\%E5\%B1\%B1\%E5\%A4\%A7\%E5\%AD\%A6\%E7\%A0\%94\%E7\%A9\%B6\%E7\%94\%9F\%E5\%AD\%A6\%E4\%BD\%8D\%E8\%AE\%BA\%E6\%96\%87\%E6\%A0\%BC\%E5\%BC\%8F\%E8\%A6\%81\%E6\%B1\%82.pdf}{中山大学ç~''究ç''Ÿå­¦ä½?论æ--‡æ~¼å¼?è¦?求}
\end{itemize}

\subsection{使ç''¨æ--¹æ³•}\label{uxe4uxbduxe7uxe6uxb9uxe6uxb3}

\subsubsection{typst.app}\label{typst.app}

ç»?过è¿`一月紧å¼~的迭代é‡?构,本模æ?¿å·²ç»?
\href{https://typst.app/universe/package/modern-sysu-thesis}{å?{}`布在typst-app.universe}
上,
\href{https://typst.app/app?template=modern-sysu-thesis&version=0.2.0}{点击此处直接创建ä½~的论æ--‡å·¥ç¨‹}
,并直接开始ç¼--写ä½~的论æ--‡ï¼?

\subsubsection{Windows ç''¨æˆ·}\label{windows-uxe7uxe6ux2c6}

\begin{enumerate}
\tightlist
\item
  \href{https://gitlab.com/sysu-gitlab/thesis-template/better-thesis/-/archive/main/better-thesis-main.zip}{下载本ä»``åº``}
  ,æˆ--è€\ldots 使ç''¨
  \texttt{\ git\ clone\ https://gitlab.com/sysu-gitlab/thesis-template/better-thesis\ }
  å`½ä»¤å\ldots‹éš†æœ¬ä»``åº``。
\item
  å?³é''® \texttt{\ install\_typst.ps1\ } æ--‡ä»¶ï¼Œé€‰æ‹©â€œç''¨
  Powershell è¿?è¡Œâ€?,等å¾\ldots{} Typst 安è£\ldots 完æˆ?。
\item
  æ~¹æ?® \href{https://typst.app/docs/}{Typst æ--‡æ¡£} ,å?‚考
  \href{https://github.com/typst/packages/raw/main/packages/preview/modern-sysu-thesis/0.3.0/\#\%E9\%A1\%B9\%E7\%9B\%AE\%E7\%BB\%93\%E6\%9E\%84}{项目ç»``æž„}
  中的说明,按ç\ldots§ä½~的需è¦?ä¿®æ''¹è®ºæ--‡çš„å?„个部分。
\item
  å?Œå‡»è¿?è¡Œ \texttt{\ compile.bat\ } ,å?³å?¯ç''Ÿæˆ?
  \texttt{\ thesis.pdf\ } æ--‡ä»¶ã€‚
\end{enumerate}

\subsubsection{Linux/macOS ç''¨æˆ·}\label{linuxmacos-uxe7uxe6ux2c6}

\begin{enumerate}
\tightlist
\item
  \href{https://gitlab.com/sysu-gitlab/thesis-template/better-thesis/-/archive/main/better-thesis-main.zip}{下载本ä»``åº``}
  ,æˆ--è€\ldots 使ç''¨
  \texttt{\ git\ clone\ https://gitlab.com/sysu-gitlab/thesis-template/better-thesis\ }
  å`½ä»¤å\ldots‹éš†æœ¬ä»``åº``。
\item
  使ç''¨å`½ä»¤è¡Œå®‰è£\ldots{} Rust å·¥å\ldots·é``¾ä»¥å?Š Typst:
\end{enumerate}

\begin{Shaded}
\begin{Highlighting}[]
\CommentTok{\# 安装 Rust 环境并激活,之前安装过则不需要执行下面这两行}
\ExtensionTok{curl} \AttributeTok{{-}{-}proto} \StringTok{\textquotesingle{}=https\textquotesingle{}} \AttributeTok{{-}{-}tlsv1.2} \AttributeTok{{-}sSf}\NormalTok{ https://sh.rustup.rs }\KeywordTok{|} \FunctionTok{sh} \AttributeTok{{-}s} \AttributeTok{{-}{-}} \AttributeTok{{-}y}
\BuiltInTok{source} \VariableTok{$HOME}\NormalTok{/.cargo/env}

\CommentTok{\# 安装 Typst CLI}
\ExtensionTok{cargo}\NormalTok{ install typst{-}cli}

\CommentTok{\# 访问缓慢的话,执行以下命令设置清华镜像源,并从镜像站安装}
\FunctionTok{cat} \OperatorTok{\textless{}\textless{} EOF} \OperatorTok{\textgreater{}} \VariableTok{$HOME}\NormalTok{/.cargo/config}
\StringTok{[source.crates{-}io]}
\StringTok{replace{-}with = "tuna"}

\StringTok{[source.tuna]}
\StringTok{registry = "https://mirrors.tuna.tsinghua.edu.cn/git/crates.io{-}index.git"}
\OperatorTok{EOF}
\ExtensionTok{cargo}\NormalTok{ install typst{-}cli}
\end{Highlighting}
\end{Shaded}

\begin{enumerate}
\setcounter{enumi}{2}
\tightlist
\item
  æ~¹æ?® \href{https://typst.app/docs/}{Typst æ--‡æ¡£} ,å?‚考
  \href{https://github.com/typst/packages/raw/main/packages/preview/modern-sysu-thesis/0.3.0/\#\%E9\%A1\%B9\%E7\%9B\%AE\%E7\%BB\%93\%E6\%9E\%84}{项目ç»``æž„}
  中的说明,按ç\ldots§ä½~的需è¦?ä¿®æ''¹è®ºæ--‡çš„å?„个部分。
\item
  执行 \texttt{\ make\ } å`½ä»¤ï¼Œå?³å?¯ç''Ÿæˆ?
  \texttt{\ thesis.pdf\ } æ--‡ä»¶ã€‚
\end{enumerate}

\subsection{项目ç»``æž„}\label{uxe9uxb9uxe7uxe7uxe6ux17e}

详� \texttt{\ template\textbackslash{}thesis.typ\ }

\subsection{FAQ}\label{faq}

\subsubsection{为什么 XXX
的功能ä¸?能ç''¨/ä¸?符å?ˆé¢„期?}\label{uxe4uxbauxe4uxe4uxb9ux2c6-xxx-uxe7ux161uxe5ux161uxffuxe8ux192uxbduxe4uxe8ux192uxbduxe7uxe4uxe7uxe5ux2c6uxe9uxe6ux153uxffuxefuxbcuxff}

\begin{enumerate}
\tightlist
\item
  å\ldots ˆå?‚考
  \href{https://typst-doc-cn.github.io/docs/chinese/}{Typst
  中æ--‡æ''¯æŒ?相å\ldots³é---®é¢˜} ,以å?Š
  \href{https://typst.app/docs/}{Typst 官æ--¹æ--‡æ¡£} 与
  \href{https://typst.app/universe}{tpyst.app/universe ä»``åº``}
  ,了解相å\ldots³é---®é¢˜è¿›å±•æˆ--解决æ--¹æ¡ˆ
\item
  如果在以上资æ--™ä¸­æ‰¾ä¸?到å\ldots³è?''资æ--™ï¼Œå?¯ä»¥å?‚考是å?¦åœ¨çš„
  \href{https://gitlab.com/sysu-gitlab/thesis-template/better-thesis/-/issues}{issue
  åˆ---表} 中能找到相å\ldots³é---®é¢˜ä¸Žè¿›å±•ã€‚
\item
  如果ä¾?然没有线索,欢迎å??馈é---®é¢˜ï¼ˆ
  \href{https://gitlab.com/sysu-gitlab/thesis-template/better-thesis/-/issues}{GitLab
  issue} /
  \href{mailto:contact-project+sysu-gitlab-thesis-template-better-thesis-57823416-issue-@incoming.gitlab.com}{邮件}
  )
\end{enumerate}

\subsubsection{\texorpdfstring{为什么学æ~¡å­¦ä½?论æ--‡å·²ç»?有了
\href{https://github.com/SYSU-SCC/sysu-thesis}{LaTeX 模�} ,还有
Typst
模æ?¿ï¼Ÿ}{为什么学æ~¡å­¦ä½?论æ--‡å·²ç»?有了 LaTeX 模æ?¿ ,还有 Typst 模æ?¿ï¼Ÿ}}\label{uxe4uxbauxe4uxe4uxb9ux2c6uxe5uxe6-uxe5uxe4uxbduxe8uxbauxe6uxe5uxb2uxe7uxe6ux153uxe4uxba-latex-uxe6uxe6-uxefuxbcux153uxe8uxe6ux153-typst-uxe6uxe6uxefuxbcuxff}

\begin{itemize}
\tightlist
\item
  �述 LaTeX
  模æ?¿ç›®å‰?ä»\ldots 有计ç®---机学院官æ--¹æŒ‡å®šä½¿ç''¨ï¼Œå\ldots¶ä»--学院并没有统一指定
\item
  考è™`到 LaTeX
  对于大部分é?žè®¡ç®---机/ç?†å·¥ç§`çš„å­¦ç''Ÿå\ldots¥é---¨æˆ?本æ¯''较高,å›~此有å¿\ldots è¦?æ??供一ç§?æ›´åŠ~简æ´?æ¸\ldots 晰并ä¸''æ--¹ä¾¿çš„论æ--‡æ¨¡æ?¿ï¼ŒåŒ\ldots 括:

  \begin{itemize}
  \tightlist
  \item
    开箱å?³ç''¨ï¼š

    \begin{itemize}
    \tightlist
    \item
      如
      \href{https://github.com/typst/packages/raw/main/packages/preview/modern-sysu-thesis/0.3.0/\#typstapp}{å‰?æ--‡æ‰€è¿°}
      ,本模æ?¿æ??供了在线直接ç¼--è¾`/ä¿?å­˜/备份æ--¹æ¡ˆ
    \item
      本地使ç''¨æ¨¡æ?¿æ---¶ï¼Œæ¨¡æ?¿ç»„件å?¯ä»¥ç®€å?•åœ°é€šè¿‡
      \texttt{\ typst\ } å`½ä»¤è‡ªåŠ¨ç®¡ç?†å®‰è£
    \end{itemize}
  \item
    语法简�:typst 是与 markdown
    类似的æ~‡è®°æ€§è¯­è¨€ï¼Œå?¯ä»¥é€šè¿‡æ~‡è®°çš„æ--¹å¼?æ?¥è½»æ?¾æŽ§åˆ¶è¯­æ³•ï¼ˆå¦‚
    \texttt{\ =\ æ~‡é¢˜\ } ã€? \texttt{\ *粗体*\ } ã€?
    \texttt{\ \_斜体\_\ } \texttt{\ @引用\ } ã€? æ•°å­¦å\ldots¬å¼?
    \texttt{\ \$E\ =\ m\ c\^{}2\$\ } )
  \end{itemize}
\end{itemize}

\subsubsection{\texorpdfstring{为什么有两份 Typst 模�(
\href{https://github.com/howardlau1999/sysu-thesis-typst}{sysu-thesis-typst}
å'Œ
modern-sysu-thesis)?}{为什么有两份 Typst 模æ?¿ï¼ˆ sysu-thesis-typst å'Œ modern-sysu-thesis)?}}\label{uxe4uxbauxe4uxe4uxb9ux2c6uxe6ux153uxe4uxe4uxbd-typst-uxe6uxe6uxefuxbcux2c6-sysu-thesis-typst-uxe5ux153-modern-sysu-thesisuxefuxbcuxefuxbcuxff}

å?Žè€\ldots 是在å‰?è€\ldots 的基础上,å?Œæ---¶å?‚考
\href{https://typst.app/universe/package/modern-nju-thesis}{modern-nju-thesis}
,æ''¹é€~å?Žé€‚é\ldots?了
\href{https://typst.app/universe}{typst.app/universe}
。以å?Šï¼Œæ''¾åˆ° \href{https://github.com/sysu}{@sysu}
组织下æ??高了æ›?å\ldots‰åº¦ã€‚

\subsection{致谢}\label{uxe8uxe8}

\begin{itemize}
\tightlist
\item
  æ„Ÿè°¢
  \href{https://github.com/howardlau1999/sysu-thesis-typst}{sysu-thesis-typst}
  æ??供了中山大学的页é?¢æ~·å¼?与åˆ?版æº?ç~?
\item
  æ„Ÿè°¢
  \href{https://typst.app/universe/package/modern-nju-thesis}{modern-nju-thesis}
  æ??供了一个更好的代ç~?组织架构
\item
  感谢中山大学 Typst 模�交�群(
  \href{https://jq.qq.com/?_wv=1027&k=m58va1kd}{797942860} )�Typst
  中æ--‡äº¤æµ?群(793548390)群å?‹çš„帮助交æµ?。
\end{itemize}

\href{/app?template=modern-sysu-thesis&version=0.3.0}{Create project in
app}

\subsubsection{How to use}\label{how-to-use}

Click the button above to create a new project using this template in
the Typst app.

You can also use the Typst CLI to start a new project on your computer
using this command:

\begin{verbatim}
typst init @preview/modern-sysu-thesis:0.3.0
\end{verbatim}

\includesvg[width=0.16667in,height=0.16667in]{/assets/icons/16-copy.svg}

\subsubsection{About}\label{about}

\begin{description}
\tightlist
\item[Author s :]
\href{https://github.com/howardlau1999}{howardlau1999} \&
\href{https://github.com/huangjj27}{Sunny Huang}
\item[License:]
MIT
\item[Current version:]
0.3.0
\item[Last updated:]
June 17, 2024
\item[First released:]
May 17, 2024
\item[Archive size:]
43.5 kB
\href{https://packages.typst.org/preview/modern-sysu-thesis-0.3.0.tar.gz}{\pandocbounded{\includesvg[keepaspectratio]{/assets/icons/16-download.svg}}}
\item[Repository:]
\href{https://gitlab.com/sysu-gitlab/thesis-template/better-thesis}{GitLab}
\item[Categor y :]
\begin{itemize}
\tightlist
\item[]
\item
  \pandocbounded{\includesvg[keepaspectratio]{/assets/icons/16-mortarboard.svg}}
  \href{https://typst.app/universe/search/?category=thesis}{Thesis}
\end{itemize}
\end{description}

\subsubsection{Where to report issues?}\label{where-to-report-issues}

This template is a project of howardlau1999 and Sunny Huang . Report
issues on
\href{https://gitlab.com/sysu-gitlab/thesis-template/better-thesis}{their
repository} . You can also try to ask for help with this template on the
\href{https://forum.typst.app}{Forum} .

Please report this template to the Typst team using the
\href{https://typst.app/contact}{contact form} if you believe it is a
safety hazard or infringes upon your rights.

\phantomsection\label{versions}
\subsubsection{Version history}\label{version-history}

\begin{longtable}[]{@{}ll@{}}
\toprule\noalign{}
Version & Release Date \\
\midrule\noalign{}
\endhead
\bottomrule\noalign{}
\endlastfoot
0.3.0 & June 17, 2024 \\
\href{https://typst.app/universe/package/modern-sysu-thesis/0.2.0/}{0.2.0}
& June 10, 2024 \\
\href{https://typst.app/universe/package/modern-sysu-thesis/0.1.1/}{0.1.1}
& May 23, 2024 \\
\href{https://typst.app/universe/package/modern-sysu-thesis/0.1.0/}{0.1.0}
& May 17, 2024 \\
\end{longtable}

Typst GmbH did not create this template and cannot guarantee correct
functionality of this template or compatibility with any version of the
Typst compiler or app.


\section{Package List LaTeX/athena-tu-darmstadt-thesis.tex}
\title{typst.app/universe/package/athena-tu-darmstadt-thesis}

\phantomsection\label{banner}
\phantomsection\label{template-thumbnail}
\pandocbounded{\includegraphics[keepaspectratio]{https://packages.typst.org/preview/thumbnails/athena-tu-darmstadt-thesis-0.1.0-small.webp}}

\section{athena-tu-darmstadt-thesis}\label{athena-tu-darmstadt-thesis}

{ 0.1.0 }

Thesis template for TU Darmstadt (Technische Universität Darmstadt).

\href{/app?template=athena-tu-darmstadt-thesis&version=0.1.0}{Create
project in app}

\phantomsection\label{readme}
This \textbf{unofficial} template can be used to write in
\href{https://github.com/typst/typst}{Typst} with the corporate design
of \href{https://www.tu-darmstadt.de/}{TU Darmstadt} .

\paragraph{Disclaimer}\label{disclaimer}

Please ask your supervisor if you are allowed to use typst and this
template for your thesis or other documents. Note that this template is
not checked by TU Darmstadt for correctness. Thus, this template does
not guarantee completeness or correctness. Also, note that submission in
TUbama requires PDF/A which typst currently can’t export to (
\url{https://github.com/typst/typst/issues/2942} ). You can use a
converter to convert from the typst output to PDF/A, but check that
there are no losses during the conversion. CMYK color space support may
be required for printing which is also currently not supported by typst
( \url{https://github.com/typst/typst/issues/2942} ), but this is not
relevant when you just submit online.

\subsection{Implemented Templates}\label{implemented-templates}

The templates imitate the style of the corresponding latex templates in
\href{https://github.com/tudace/tuda_latex_templates}{tuda\_latex\_templates}
. Note that there can be visual differences between the original latex
template and the typst template (you may open an issue when you find
one).

For missing features, ideas or other problems you can just open an issue
:wink:. Contributions are also welcome.

\begin{longtable}[]{@{}
  >{\raggedright\arraybackslash}p{(\linewidth - 6\tabcolsep) * \real{0.2500}}
  >{\raggedright\arraybackslash}p{(\linewidth - 6\tabcolsep) * \real{0.2500}}
  >{\raggedright\arraybackslash}p{(\linewidth - 6\tabcolsep) * \real{0.2500}}
  >{\raggedright\arraybackslash}p{(\linewidth - 6\tabcolsep) * \real{0.2500}}@{}}
\toprule\noalign{}
\begin{minipage}[b]{\linewidth}\raggedright
Template
\end{minipage} & \begin{minipage}[b]{\linewidth}\raggedright
Preview
\end{minipage} & \begin{minipage}[b]{\linewidth}\raggedright
Example
\end{minipage} & \begin{minipage}[b]{\linewidth}\raggedright
Scope
\end{minipage} \\
\midrule\noalign{}
\endhead
\bottomrule\noalign{}
\endlastfoot
\href{https://github.com/JeyRunner/tuda-typst-templates/blob/main/templates/tudapub/tudapub.typ}{tudapub}
&
\includegraphics[width=\linewidth,height=3.125in,keepaspectratio]{https://github.com/typst/packages/raw/main/packages/preview/athena-tu-darmstadt-thesis/0.1.0/img/tudapub_prev-01.png}
& \begin{minipage}[t]{\linewidth}\raggedright
\href{https://github.com/JeyRunner/tuda-typst-templates/blob/main/example_tudapub.pdf}{example\_tudapub.pdf}\\
\href{https://github.com/JeyRunner/tuda-typst-templates/blob/main/example_tudapub.typ}{example\_tudapub.typ}\strut
\end{minipage} & Master and Bachelor thesis \\
\end{longtable}

\subsection{Usage}\label{usage}

Create a new typst project based on this template locally.

\begin{Shaded}
\begin{Highlighting}[]
\ExtensionTok{typst}\NormalTok{ init @preview/athena{-}tu{-}darmstadt{-}thesis}
\BuiltInTok{cd}\NormalTok{ athena{-}tu{-}darmstadt{-}thesis}
\end{Highlighting}
\end{Shaded}

Or create a project on the typst web app based on this template.

Or do a manual installation of this template.

For a manual setup create a folder for your writing project and download
this template into the `templates` folder:

\begin{Shaded}
\begin{Highlighting}[]
\FunctionTok{mkdir}\NormalTok{ my\_thesis }\KeywordTok{\&\&} \BuiltInTok{cd}\NormalTok{ my\_thesis}
\FunctionTok{mkdir}\NormalTok{ templates }\KeywordTok{\&\&} \BuiltInTok{cd}\NormalTok{ templates}
\FunctionTok{git}\NormalTok{ clone https://github.com/JeyRunner/tuda{-}typst{-}templates templates/}
\end{Highlighting}
\end{Shaded}

\subsubsection{Logo and Font Setup}\label{logo-and-font-setup}

Download the tud logo from
\href{https://download.hrz.tu-darmstadt.de/protected/ULB/tuda_logo.pdf}{download.hrz.tu-darmstadt.de/protected/ULB/tuda\_logo.pdf}
and put it into the \texttt{\ logos\ } folder. Now execute the following
script in the \texttt{\ logos\ } folder to convert it into an svg:

\begin{Shaded}
\begin{Highlighting}[]
\BuiltInTok{cd}\NormalTok{ logos}
\ExtensionTok{./convert\_logo.sh}
\end{Highlighting}
\end{Shaded}

Also download the required fonts \texttt{\ Roboto\ } and
\texttt{\ XCharter\ } :

\begin{Shaded}
\begin{Highlighting}[]
\BuiltInTok{cd}\NormalTok{ fonts}
\ExtensionTok{./download\_fonts.sh}
\end{Highlighting}
\end{Shaded}

Now you can install all fonts in the folders in \texttt{\ fonts\ } on
your system.

Create a main.typ file for the manual template installation.

Create a simple `main.typ` in the root folder (`my\_thesis`) of your new
project:

\begin{Shaded}
\begin{Highlighting}[]
\NormalTok{\#import }\StringTok{" templates/tuda{-}typst{-}templates/templates/tudapub/tudapub.typ"}\OperatorTok{:}\NormalTok{ tudapub}

\NormalTok{\#show}\OperatorTok{:}\NormalTok{ tudapub}\OperatorTok{.}\FunctionTok{with}\NormalTok{(}
\NormalTok{  title}\OperatorTok{:}\NormalTok{ [}
\NormalTok{    My Thesis}
\NormalTok{  ]}\OperatorTok{,}
\NormalTok{  author}\OperatorTok{:} \StringTok{"My Name"}\OperatorTok{,}
\NormalTok{  accentcolor}\OperatorTok{:} \StringTok{"3d"}
\NormalTok{)}

\OperatorTok{=}\NormalTok{ My First Chapter}
\NormalTok{Some }\BuiltInTok{Text}
\end{Highlighting}
\end{Shaded}

\subsubsection{Compile you typst file}\label{compile-you-typst-file}

\begin{Shaded}
\begin{Highlighting}[]
\ExtensionTok{typst} \AttributeTok{{-}{-}watch}\NormalTok{ main.typ }\AttributeTok{{-}{-}font{-}path}\NormalTok{ fonts/}
\end{Highlighting}
\end{Shaded}

This will watch your file and recompile it to a pdf when the file is
saved. For writing, you can use
\href{https://code.visualstudio.com/}{Vscode} with these extensions:
\href{https://marketplace.visualstudio.com/items?itemName=nvarner.typst-lsp}{Typst
LSP} and
\href{https://marketplace.visualstudio.com/items?itemName=mgt19937.typst-preview}{Typst
Preview} . Or use the \href{https://typst.app/}{typst web app} (here you
need to upload the logo and the fonts).

Note that we add \texttt{\ -\/-font-path\ } to ensure that the correct
fonts are used. Due to a bug (typst/typst\#2917 typst/typst\#2098) typst
sometimes uses the font \texttt{\ Roboto\ condensed\ } instead of
\texttt{\ Roboto\ } . To be on the safe side, double-check the embedded
fonts in the pdf (there should be no \texttt{\ Roboto\ condensed\ } ).
What also works is to uninstall/deactivate all
\texttt{\ Roboto\ condensed\ } fonts from your system.

\subsection{Todos}\label{todos}

\begin{itemize}
\tightlist
\item
  {[} {]} some bug seems to insert an empty page at the end of the
  document when content (title page) appears before this second ‘set
  page’
\item
  {[} {]} numbering/labeling of sub-equations (that are aligned with the
  other sub-equations)
\item
  {[}x{]} remove page numbers in footer before \st{and at table of
  contents}
\item
  {[}x{]} fix first-level heading page is wrong

  \begin{itemize}
  \tightlist
  \item
    in the outline, the page of the first-level heading is sometimes the
    previous page of the heading. Just appears in combination with
    \texttt{\ figure\_numbering\_per\_chapter\ } .
  \end{itemize}
\item
  {[} {]} fix referencing figures respect figure numbering when using
  \texttt{\ figure\_numbering\_per\_chapter\ }
\item
  {[} {]} first-level headings should be referenced as ‘Chapter’ not
  as ‘Sections’
\item
  {[} {]} add pages for:

  \begin{itemize}
  \tightlist
  \item
    {[}x{]} abstract
  \item
    {[} {]} list of figures, tables, … other
  \item
    {[} {]} list of abbreviations (glossary)
  \item
    {[}x{]} references
  \end{itemize}
\item
  {[} {]} references list: use same citation style is ‘numeric’ in
  latex
\item
  {[} {]} reduce vertical spacing between adjacent headings when there
  is no text in between (looks better, latex template also does this)
\item
  {[}x{]} add arguments for optional pages:

  \begin{itemize}
  \tightlist
  \item
    after title page
  \item
    before outline table of contents
  \item
    after outline table of contents
  \end{itemize}
\item
  {[} {]} fix equation numbering per chapter (somehow increases in steps
  of 2)
\item
  {[}x{]} provide some default page margins (small, medium, large)
\item
  {[} {]} \st{make all font sizes relative to the main text font size
  (e.g. headings)}
\item
  {[} {]} switch to kebab case for template, function args
\end{itemize}

\href{/app?template=athena-tu-darmstadt-thesis&version=0.1.0}{Create
project in app}

\subsubsection{How to use}\label{how-to-use}

Click the button above to create a new project using this template in
the Typst app.

You can also use the Typst CLI to start a new project on your computer
using this command:

\begin{verbatim}
typst init @preview/athena-tu-darmstadt-thesis:0.1.0
\end{verbatim}

\includesvg[width=0.16667in,height=0.16667in]{/assets/icons/16-copy.svg}

\subsubsection{About}\label{about}

\begin{description}
\tightlist
\item[Author :]
\href{https://github.com/JeyRunner}{JeyRunner}
\item[License:]
MIT
\item[Current version:]
0.1.0
\item[Last updated:]
May 22, 2024
\item[First released:]
May 22, 2024
\item[Archive size:]
15.1 kB
\href{https://packages.typst.org/preview/athena-tu-darmstadt-thesis-0.1.0.tar.gz}{\pandocbounded{\includesvg[keepaspectratio]{/assets/icons/16-download.svg}}}
\item[Repository:]
\href{https://github.com/JeyRunner/tuda-typst-templates}{GitHub}
\item[Categor y :]
\begin{itemize}
\tightlist
\item[]
\item
  \pandocbounded{\includesvg[keepaspectratio]{/assets/icons/16-mortarboard.svg}}
  \href{https://typst.app/universe/search/?category=thesis}{Thesis}
\end{itemize}
\end{description}

\subsubsection{Where to report issues?}\label{where-to-report-issues}

This template is a project of JeyRunner . Report issues on
\href{https://github.com/JeyRunner/tuda-typst-templates}{their
repository} . You can also try to ask for help with this template on the
\href{https://forum.typst.app}{Forum} .

Please report this template to the Typst team using the
\href{https://typst.app/contact}{contact form} if you believe it is a
safety hazard or infringes upon your rights.

\phantomsection\label{versions}
\subsubsection{Version history}\label{version-history}

\begin{longtable}[]{@{}ll@{}}
\toprule\noalign{}
Version & Release Date \\
\midrule\noalign{}
\endhead
\bottomrule\noalign{}
\endlastfoot
0.1.0 & May 22, 2024 \\
\end{longtable}

Typst GmbH did not create this template and cannot guarantee correct
functionality of this template or compatibility with any version of the
Typst compiler or app.


\section{Package List LaTeX/grotesk-cv.tex}
\title{typst.app/universe/package/grotesk-cv}

\phantomsection\label{banner}
\phantomsection\label{template-thumbnail}
\pandocbounded{\includegraphics[keepaspectratio]{https://packages.typst.org/preview/thumbnails/grotesk-cv-1.0.1-small.webp}}

\section{grotesk-cv}\label{grotesk-cv}

{ 1.0.1 }

A clean CV and cover letter template based on Brilliant-cv and fireside
templates.

\href{/app?template=grotesk-cv&version=1.0.1}{Create project in app}

\phantomsection\label{readme}
Version 1.0.1

{ }

grotesk-cv provides a pair of elegant and simple, one-page CV and cover
letter templates, inspired by the
\href{https://typst.app/universe/package/brilliant-cv/}{Brilliant-cv}
and \href{https://typst.app/universe/package/fireside/1.0.0/}{fireside}
templates.

\subsubsection{Features}\label{features}

\begin{itemize}
\tightlist
\item
  Templates for multilingual CV and cover letter, enabled by flag
\item
  Separation of styling and content
\item
  Customizable fonts, colors and icons
\end{itemize}

\subsection{Preview}\label{preview}

\begin{longtable}[]{@{}cc@{}}
\toprule\noalign{}
CV & Cover Letter \\
\midrule\noalign{}
\endhead
\bottomrule\noalign{}
\endlastfoot
\pandocbounded{\includegraphics[keepaspectratio]{https://raw.githubusercontent.com/AsiSkarp/grotesk-cv/main/examples/cv_example.png?raw=true}}
&
\pandocbounded{\includegraphics[keepaspectratio]{https://raw.githubusercontent.com/AsiSkarp/grotesk-cv/main/examples/cover_letter_example.png?raw=true}} \\
\end{longtable}

\subsection{Getting Started}\label{getting-started}

To edit this template, changes are mostly made in either of two places.
Changes to contact information or layout settings are made in
\texttt{\ info.toml\ } . To change the section contents, navigate to the
corresponding section file e.g. \texttt{\ content/profile.typ\ } to edit
the \textbf{Profile} section.

\subsubsection{Adding or Removing
Sections}\label{adding-or-removing-sections}

To add a new section, create a new \texttt{\ .typ\ } file in the
\texttt{\ content\ } directory with the desired section name. To include
the section in the CV, add the section at the desired position in either
left or right panes in the \texttt{\ cv.typ\ } file. To remove sections,
simply remove or comment-out the section name in the same list of
section names in the \texttt{\ cv.typ\ } file. Sections are rendered in
the order they appear in the list. The section column width can be
adjusted in the \texttt{\ info.toml\ } file under the
\texttt{\ left\_pane\_width\ } value. In the following example, the
\texttt{\ projects.typ\ } section file has been created and is included
in the left pane of the CV, and the \texttt{\ education.typ\ } section
has been removed.

\begin{Shaded}
\begin{Highlighting}[]
\NormalTok{\#let left}\OperatorTok{{-}}\NormalTok{pane }\OperatorTok{=}\NormalTok{ (}
  \StringTok{"profile"}\OperatorTok{,}
  \StringTok{"experience"}\OperatorTok{,}
  \CommentTok{//"education",}
  \StringTok{"projects"}\OperatorTok{,}
\NormalTok{)}
\end{Highlighting}
\end{Shaded}

\subsubsection{Changing Profile Photo}\label{changing-profile-photo}

To change the profile photo, upload your image to the
\texttt{\ content/img\ } folder. To enable the new image, update the
\texttt{\ profile\_image\ } value in \texttt{\ info.toml\ } with the
name of your uploaded image.

\subsubsection{Changing FontAwesome
Icons}\label{changing-fontawesome-icons}

The template uses \href{https://fontawesome.com/}{FontAwesome} for all
icons. To change an icon, change the desired icon string in the
\texttt{\ info.toml\ } file with the corresponding FontAwesome icon
name. Icon strings can be found in the
\href{https://fontawesome.com/v4/cheatsheet/}{cheat sheet} . Note that
the icon strings must be written without the \texttt{\ fa-\ } prefix. To
disable the use of icons, set the \texttt{\ include\_icons\ } value to
\texttt{\ false\ } .

\subsubsection{Customizing Contact
Information}\label{customizing-contact-information}

To change or add contact information, update the corresponding value
under \texttt{\ {[}personal.info{]}\ } in the \texttt{\ info.toml\ }
file. Information is rendered in the order it appears in the file. To
add a new contact information field, add a new variable under
\texttt{\ {[}personal.info{]}\ } with the desired string value. Next,
assign a valid FontAwesome icon string to a variable of the same name
under \texttt{\ {[}personal.icon{]}\ } . In the following example, a
homepage field has been added to the contact information.

\begin{Shaded}
\begin{Highlighting}[]
\KeywordTok{[personal.info]}
\DataTypeTok{homepage} \OperatorTok{=} \StringTok{"www.myawesomehomepage.com"}

\KeywordTok{[personal.icon]}
\DataTypeTok{homepage} \OperatorTok{=} \StringTok{"globe"}
\end{Highlighting}
\end{Shaded}

\subsubsection{Changing language}\label{changing-language}

The template provides the option to instantly change the language of the
CV and cover letter by using a variable in the \texttt{\ info.toml\ }
file. The template demonstrates the use of the \texttt{\ language\ }
variable to switch between English and Spanish, but any language can be
used, provided that the information is entered manually inside the
corresponding section files. For instance, to change the alternate
language to German, changes would have to be made in the section files
to include the German text. In the following example, the language of
the \textbf{Profile} section has been changed from Spanish to German,
and the required changes have been made in the
\texttt{\ content/profile.typ\ } file.

\begin{verbatim}
// = Summary
= #if include-icon [#fa-icon(icon) #h(5pt)] #if language == "en" [Summary] else if language == "ger" [Zusammenfassung]

#v(5pt)

#if language == "en" [

  Experienced Software Engineer specializing in artificial intelligence, machine learning, and robotics. Proficient in C++, Python, and Java, with a knack for developing sentient AI systems capable of complex decision-making. Passionate about ethical AI development and eager to contribute to groundbreaking projects in dynamic environments.

] else if language == "ger" [

  Erfahrener Software-Ingenieur, der sich auf künstliche Intelligenz, maschinelles Lernen und Robotik spezialisiert hat. Er beherrscht C++, Python und Java und hat ein Händchen für die Entwicklung empfindungsfähiger KI-Systeme, die in der Lage sind, komplexe Entscheidungen zu treffen. Leidenschaft für ethische KI-Entwicklung und bestrebt, zu bahnbrechenden Projekten in dynamischen Umgebungen beizutragen.

]
\end{verbatim}

\subsubsection{Changing Fonts}\label{changing-fonts}

If using the template online with Typst Universe, multiple font types
are available to use, a list of which can be found by pressing the
\texttt{\ Ag\ } button. To use a different font, upload a
\texttt{\ ttf\ } or \texttt{\ otf\ } file to the
\texttt{\ content/fonts\ } folder and update the \texttt{\ font\ } value
in the \texttt{\ info.toml\ } file with the name of the uploaded font.

\subsubsection{Installation}\label{installation}

To use the template offline, clone this repository to your local
machine. Typst can be used and rendered offline by installing the Typst
CLI. My preferred workflow has been to use VSCode with the
\href{https://github.com/Myriad-Dreamin/tinymist/releases}{Tinymist}
extension, which provides LSP support, syntax highlighting, and error
checking, live rendered previews and automatic exports to PDF.

Please feel free to fork this repository and create PRs for any changes
or improvements.

\href{/app?template=grotesk-cv&version=1.0.1}{Create project in app}

\subsubsection{How to use}\label{how-to-use}

Click the button above to create a new project using this template in
the Typst app.

You can also use the Typst CLI to start a new project on your computer
using this command:

\begin{verbatim}
typst init @preview/grotesk-cv:1.0.1
\end{verbatim}

\includesvg[width=0.16667in,height=0.16667in]{/assets/icons/16-copy.svg}

\subsubsection{About}\label{about}

\begin{description}
\tightlist
\item[Author :]
\href{https://github.com/AsiSkarp}{�sbjörn Skarphéðinsson}
\item[License:]
Unlicense
\item[Current version:]
1.0.1
\item[Last updated:]
October 21, 2024
\item[First released:]
September 30, 2024
\item[Archive size:]
1.38 MB
\href{https://packages.typst.org/preview/grotesk-cv-1.0.1.tar.gz}{\pandocbounded{\includesvg[keepaspectratio]{/assets/icons/16-download.svg}}}
\item[Repository:]
\href{https://github.com/AsiSkarp/grotesk-cv}{GitHub}
\item[Categor ies :]
\begin{itemize}
\tightlist
\item[]
\item
  \pandocbounded{\includesvg[keepaspectratio]{/assets/icons/16-user.svg}}
  \href{https://typst.app/universe/search/?category=cv}{CV}
\item
  \pandocbounded{\includesvg[keepaspectratio]{/assets/icons/16-layout.svg}}
  \href{https://typst.app/universe/search/?category=layout}{Layout}
\end{itemize}
\end{description}

\subsubsection{Where to report issues?}\label{where-to-report-issues}

This template is a project of �sbjörn Skarphéðinsson . Report issues
on \href{https://github.com/AsiSkarp/grotesk-cv}{their repository} . You
can also try to ask for help with this template on the
\href{https://forum.typst.app}{Forum} .

Please report this template to the Typst team using the
\href{https://typst.app/contact}{contact form} if you believe it is a
safety hazard or infringes upon your rights.

\phantomsection\label{versions}
\subsubsection{Version history}\label{version-history}

\begin{longtable}[]{@{}ll@{}}
\toprule\noalign{}
Version & Release Date \\
\midrule\noalign{}
\endhead
\bottomrule\noalign{}
\endlastfoot
1.0.1 & October 21, 2024 \\
\href{https://typst.app/universe/package/grotesk-cv/1.0.0/}{1.0.0} &
October 17, 2024 \\
\href{https://typst.app/universe/package/grotesk-cv/0.1.6/}{0.1.6} &
October 11, 2024 \\
\href{https://typst.app/universe/package/grotesk-cv/0.1.5/}{0.1.5} &
October 10, 2024 \\
\href{https://typst.app/universe/package/grotesk-cv/0.1.4/}{0.1.4} &
October 9, 2024 \\
\href{https://typst.app/universe/package/grotesk-cv/0.1.3/}{0.1.3} &
October 8, 2024 \\
\href{https://typst.app/universe/package/grotesk-cv/0.1.2/}{0.1.2} &
October 7, 2024 \\
\href{https://typst.app/universe/package/grotesk-cv/0.1.1/}{0.1.1} &
October 2, 2024 \\
\href{https://typst.app/universe/package/grotesk-cv/0.1.0/}{0.1.0} &
September 30, 2024 \\
\end{longtable}

Typst GmbH did not create this template and cannot guarantee correct
functionality of this template or compatibility with any version of the
Typst compiler or app.


\section{Package List LaTeX/icu-datetime.tex}
\title{typst.app/universe/package/icu-datetime}

\phantomsection\label{banner}
\section{icu-datetime}\label{icu-datetime}

{ 0.1.2 }

Date and time formatting using ICU4X via WASM

\phantomsection\label{readme}
This library is a wrapper around
\href{https://github.com/unicode-org/icu4x}{ICU4X} ’
\texttt{\ datetime\ } formatting for Typst which provides
internationalized formatting for dates, times, and timezones.

As the WASM bundle includes all localization data, it’s quite large
(about 8 MiB).

See \href{https://nerixyz.github.io/icu-typ}{nerixyz.github.io/icu-typ}
for a full API reference with more examples.

\subsection{Example}\label{example}

\begin{Shaded}
\begin{Highlighting}[]
\NormalTok{\#import "@preview/icu{-}datetime:0.1.2": fmt{-}date, fmt{-}time, fmt{-}datetime, experimental}
\NormalTok{// These functions may change at any time}
\NormalTok{\#import experimental: fmt{-}timezone, fmt{-}zoned{-}datetime}

\NormalTok{\#let day = datetime(}
\NormalTok{  year: 2024,}
\NormalTok{  month: 5,}
\NormalTok{  day: 31,}
\NormalTok{)}
\NormalTok{\#let time = datetime(}
\NormalTok{  hour: 18,}
\NormalTok{  minute: 2,}
\NormalTok{  second: 23,}
\NormalTok{)}
\NormalTok{\#let dt = datetime(}
\NormalTok{  year: 2024,}
\NormalTok{  month: 5,}
\NormalTok{  day: 31,}
\NormalTok{  hour: 18,}
\NormalTok{  minute: 2,}
\NormalTok{  second: 23,}
\NormalTok{)}
\NormalTok{\#let tz = (}
\NormalTok{  offset: "{-}07",}
\NormalTok{  iana: "America/Los\_Angeles",}
\NormalTok{  zone{-}variant: "st", // standard}
\NormalTok{)}

\NormalTok{= Dates}
\NormalTok{\#fmt{-}date(day, locale: "km", length: "full") \textbackslash{}}
\NormalTok{\#fmt{-}date(day, locale: "af", length: "full") \textbackslash{}}
\NormalTok{\#fmt{-}date(day, locale: "za", length: "full") \textbackslash{}}

\NormalTok{= Time}
\NormalTok{\#fmt{-}time(time, locale: "id", length: "medium") \textbackslash{}}
\NormalTok{\#fmt{-}time(time, locale: "en", length: "medium") \textbackslash{}}
\NormalTok{\#fmt{-}time(time, locale: "ga", length: "medium") \textbackslash{}}

\NormalTok{= Date and Time}
\NormalTok{\#fmt{-}datetime(dt, locale: "ru", date{-}length: "full") \textbackslash{}}
\NormalTok{\#fmt{-}datetime(dt, locale: "en{-}US", date{-}length: "full") \textbackslash{}}
\NormalTok{\#fmt{-}datetime(dt, locale: "zh{-}Hans{-}CN", date{-}length: "full") \textbackslash{}}
\NormalTok{\#fmt{-}datetime(dt, locale: "ar", date{-}length: "full") \textbackslash{}}
\NormalTok{\#fmt{-}datetime(dt, locale: "fi", date{-}length: "full")}

\NormalTok{= Timezones (experimental)}
\NormalTok{\#fmt{-}timezone(}
\NormalTok{  ..tz,}
\NormalTok{  local{-}date: datetime.today(),}
\NormalTok{  format: "specific{-}non{-}location{-}long"}
\NormalTok{) \textbackslash{}}
\NormalTok{\#fmt{-}timezone(}
\NormalTok{  ..tz,}
\NormalTok{  format: (}
\NormalTok{    iso8601: (}
\NormalTok{      format: "utc{-}extended",}
\NormalTok{      minutes: "required",}
\NormalTok{      seconds: "optional",}
\NormalTok{    )}
\NormalTok{  )}
\NormalTok{)}

\NormalTok{= Zoned Datetimes (experimental)}
\NormalTok{\#fmt{-}zoned{-}datetime(dt, tz) \textbackslash{}}
\NormalTok{\#fmt{-}zoned{-}datetime(dt, tz, locale: "lv") \textbackslash{}}
\NormalTok{\#fmt{-}zoned{-}datetime(dt, tz, locale: "en{-}CA{-}u{-}hc{-}h24{-}ca{-}buddhist")}
\end{Highlighting}
\end{Shaded}

\pandocbounded{\includegraphics[keepaspectratio]{https://github.com/typst/packages/raw/main/packages/preview/icu-datetime/0.1.2/res/example.png}}

Locales must be
\href{https://unicode.org/reports/tr35/tr35.html\#Unicode_locale_identifier}{Unicode
Locale Identifier} s. Use
\href{https://nerixyz.github.io/icu-typ/locale-info/}{\texttt{\ locale-info(locale)\ }}
to get information on how a locale is parsed. Unicode extensions are
supported, so you can, for example, set the hour cycle with
\texttt{\ hc-h12\ } or set the calendar with \texttt{\ ca-buddhist\ }
(e.g. \texttt{\ en-CA-u-hc-h24-ca-buddhist\ } ).

\subsection{Documentation}\label{documentation}

Documentation can be found on
\href{https://nerixyz.github.io/icu-typ}{nerixyz.github.io/icu-typ} .

\subsection{Using Locally}\label{using-locally}

Download the \href{https://github.com/Nerixyz/icu-typ/releases}{latest
release} , unzip it to your
\href{https://github.com/typst/packages\#local-packages}{local Typst
packages} , and use \texttt{\ \#import\ "@local/icu-datetime:0.1.2"\ } .

\subsection{Building}\label{building}

To build the library, you need to have
\href{https://www.rust-lang.org/}{Rust} ,
\href{https://just.systems/}{just} , and
\href{https://github.com/WebAssembly/binaryen}{\texttt{\ wasm-opt\ }}
installed.

\begin{Shaded}
\begin{Highlighting}[]
\ExtensionTok{just}\NormalTok{ build}
\CommentTok{\# to deploy the package locally, use \textasciigrave{}just deploy\textasciigrave{}}
\end{Highlighting}
\end{Shaded}

While developing, you can symlink the WASM file into the root of the
repository (it’s in the \texttt{\ .gitignore\ } ):

\begin{Shaded}
\begin{Highlighting}[]
\CommentTok{\# Windows (PowerShell)}
\ExtensionTok{New{-}Item}\NormalTok{ icu{-}datetime.wasm }\AttributeTok{{-}ItemType}\NormalTok{ SymbolicLink }\AttributeTok{{-}Value}\NormalTok{ ./target/wasm32{-}unknown{-}unknown/debug/icu\_typ.wasm}
\CommentTok{\# Unix}
\FunctionTok{ln} \AttributeTok{{-}s}\NormalTok{ ./target/wasm32{-}unknown{-}unknown/debug/icu\_typ.wasm icu{-}datetime.wasm}
\end{Highlighting}
\end{Shaded}

Use \texttt{\ cargo\ b\ -\/-target\ wasm32-unknown-unknown\ } to build
in debug mode.

\subsubsection{How to add}\label{how-to-add}

Copy this into your project and use the import as
\texttt{\ icu-datetime\ }

\begin{verbatim}
#import "@preview/icu-datetime:0.1.2"
\end{verbatim}

\includesvg[width=0.16667in,height=0.16667in]{/assets/icons/16-copy.svg}

Check the docs for
\href{https://typst.app/docs/reference/scripting/\#packages}{more
information on how to import packages} .

\subsubsection{About}\label{about}

\begin{description}
\tightlist
\item[Author :]
\href{https://github.com/Nerixyz}{Nerixyz}
\item[License:]
MIT
\item[Current version:]
0.1.2
\item[Last updated:]
June 14, 2024
\item[First released:]
June 3, 2024
\item[Minimum Typst version:]
0.11.0
\item[Archive size:]
1.55 MB
\href{https://packages.typst.org/preview/icu-datetime-0.1.2.tar.gz}{\pandocbounded{\includesvg[keepaspectratio]{/assets/icons/16-download.svg}}}
\item[Repository:]
\href{https://github.com/Nerixyz/icu-typ}{GitHub}
\item[Categor y :]
\begin{itemize}
\tightlist
\item[]
\item
  \pandocbounded{\includesvg[keepaspectratio]{/assets/icons/16-world.svg}}
  \href{https://typst.app/universe/search/?category=languages}{Languages}
\end{itemize}
\end{description}

\subsubsection{Where to report issues?}\label{where-to-report-issues}

This package is a project of Nerixyz . Report issues on
\href{https://github.com/Nerixyz/icu-typ}{their repository} . You can
also try to ask for help with this package on the
\href{https://forum.typst.app}{Forum} .

Please report this package to the Typst team using the
\href{https://typst.app/contact}{contact form} if you believe it is a
safety hazard or infringes upon your rights.

\phantomsection\label{versions}
\subsubsection{Version history}\label{version-history}

\begin{longtable}[]{@{}ll@{}}
\toprule\noalign{}
Version & Release Date \\
\midrule\noalign{}
\endhead
\bottomrule\noalign{}
\endlastfoot
0.1.2 & June 14, 2024 \\
\href{https://typst.app/universe/package/icu-datetime/0.1.1/}{0.1.1} &
June 10, 2024 \\
\href{https://typst.app/universe/package/icu-datetime/0.1.0/}{0.1.0} &
June 3, 2024 \\
\end{longtable}

Typst GmbH did not create this package and cannot guarantee correct
functionality of this package or compatibility with any version of the
Typst compiler or app.


\section{Package List LaTeX/typographix-polytechnique-reports.tex}
\title{typst.app/universe/package/typographix-polytechnique-reports}

\phantomsection\label{banner}
\phantomsection\label{template-thumbnail}
\pandocbounded{\includegraphics[keepaspectratio]{https://packages.typst.org/preview/thumbnails/typographix-polytechnique-reports-0.1.4-small.webp}}

\section{typographix-polytechnique-reports}\label{typographix-polytechnique-reports}

{ 0.1.4 }

A report template for Polytechnique students (from TypographiX).

\href{/app?template=typographix-polytechnique-reports&version=0.1.4}{Create
project in app}

\phantomsection\label{readme}
A Typst package for Polytechnique student reports.

For a short introduction to Typst features and detailled information
about the package, check the
\href{https://github.com/remigerme/typst-polytechnique/blob/main/guide.pdf}{guide}
(available from the repo only).

\subsection{Usage}\label{usage}

If you want to use it on local, make sure you have the font “New
Computer Modern Sans� installed.

Define variables at the top of the template :

\begin{Shaded}
\begin{Highlighting}[]
\NormalTok{\#let title = "Rapport de stage en entreprise sur plusieurs lignes automatiquement"}
\NormalTok{\#let subtitle = "Un sous{-}titre pour expliquer ce titre"}
\NormalTok{\#let logo = image("path/to/my{-}logo.png")}
\NormalTok{\#let logo{-}horizontal = true}
\NormalTok{\#let short{-}title = "Rapport de stage"}
\NormalTok{\#let authors = ("Rémi Germe")}
\NormalTok{\#let date{-}start = datetime(year: 2024, month: 06, day: 05)}
\NormalTok{\#let date{-}end = datetime(year: 2024, month: 09, day: 05)}
\NormalTok{\#let despair{-}mode = false}

\NormalTok{\#set text(lang: "fr")}
\end{Highlighting}
\end{Shaded}

These variables will be used for PDF metadata, default cover page and
default header.

\subsection{Contributing}\label{contributing}

Contributions are welcomed ! See
\href{https://github.com/typst/packages/raw/main/packages/preview/typographix-polytechnique-reports/0.1.4/CONTRIBUTING.md}{contribution
guidelines} .

\subsection{Todo}\label{todo}

\begin{itemize}
\tightlist
\item
  {[} {]} heading not at the end of a page : might be tricky
\item
  {[}x{]} first line indent
\item
  {[} {]} better spacing between elements
\item
  {[}x{]} handle logos on cover page
\item
  {[}x{]} \st{handle logos on header} : feature canceled
\end{itemize}

\href{/app?template=typographix-polytechnique-reports&version=0.1.4}{Create
project in app}

\subsubsection{How to use}\label{how-to-use}

Click the button above to create a new project using this template in
the Typst app.

You can also use the Typst CLI to start a new project on your computer
using this command:

\begin{verbatim}
typst init @preview/typographix-polytechnique-reports:0.1.4
\end{verbatim}

\includesvg[width=0.16667in,height=0.16667in]{/assets/icons/16-copy.svg}

\subsubsection{About}\label{about}

\begin{description}
\tightlist
\item[Author :]
\href{https://github.com/remigerme}{Rémi Germe}
\item[License:]
MIT
\item[Current version:]
0.1.4
\item[Last updated:]
September 17, 2024
\item[First released:]
August 1, 2024
\item[Archive size:]
166 kB
\href{https://packages.typst.org/preview/typographix-polytechnique-reports-0.1.4.tar.gz}{\pandocbounded{\includesvg[keepaspectratio]{/assets/icons/16-download.svg}}}
\item[Repository:]
\href{https://github.com/remigerme/typst-polytechnique}{GitHub}
\item[Categor y :]
\begin{itemize}
\tightlist
\item[]
\item
  \pandocbounded{\includesvg[keepaspectratio]{/assets/icons/16-speak.svg}}
  \href{https://typst.app/universe/search/?category=report}{Report}
\end{itemize}
\end{description}

\subsubsection{Where to report issues?}\label{where-to-report-issues}

This template is a project of Rémi Germe . Report issues on
\href{https://github.com/remigerme/typst-polytechnique}{their
repository} . You can also try to ask for help with this template on the
\href{https://forum.typst.app}{Forum} .

Please report this template to the Typst team using the
\href{https://typst.app/contact}{contact form} if you believe it is a
safety hazard or infringes upon your rights.

\phantomsection\label{versions}
\subsubsection{Version history}\label{version-history}

\begin{longtable}[]{@{}ll@{}}
\toprule\noalign{}
Version & Release Date \\
\midrule\noalign{}
\endhead
\bottomrule\noalign{}
\endlastfoot
0.1.4 & September 17, 2024 \\
\href{https://typst.app/universe/package/typographix-polytechnique-reports/0.1.3/}{0.1.3}
& August 8, 2024 \\
\href{https://typst.app/universe/package/typographix-polytechnique-reports/0.1.2/}{0.1.2}
& August 1, 2024 \\
\end{longtable}

Typst GmbH did not create this template and cannot guarantee correct
functionality of this template or compatibility with any version of the
Typst compiler or app.


\section{Package List LaTeX/embiggen.tex}
\title{typst.app/universe/package/embiggen}

\phantomsection\label{banner}
\section{embiggen}\label{embiggen}

{ 0.0.1 }

LaTeX-like delimeter sizing in Typst

\phantomsection\label{readme}
Get LaTeX-like delimeter sizing in Typst!

\subsection{Usage}\label{usage}

\begin{Shaded}
\begin{Highlighting}[]
\NormalTok{\#import "@preview/embiggen:0.0.1": *}

\NormalTok{= embiggen}

\NormalTok{Here\textquotesingle{}s an equation of sorts:}

\NormalTok{$ \{lr(1/2x\^{}2|)\^{}(x=n)\_(x=0) + (2x+3)\} $}

\NormalTok{And here are some bigger versions of it:}

\NormalTok{$ \{big(1/2x\^{}2|)\^{}(x=n)\_(x=0) + big((2x+3))\} $}
\NormalTok{$ \{Big(1/2x\^{}2|)\^{}(x=n)\_(x=0) + Big((2x+3))\} $}
\NormalTok{$ \{bigg(1/2x\^{}2|)\^{}(x=n)\_(x=0) + bigg((2x+3))\} $}
\NormalTok{$ \{Bigg(1/2x\^{}2|)\^{}(x=n)\_(x=0) + Bigg((2x+3))\} $}

\NormalTok{And now, some smaller versions (\#text([\#link("https://x.com/tsoding/status/1756517251497255167", "cAn YoUr LaTeX dO tHaT?")], fill: rgb(50, 20, 200), font: "Noto Mono")):}

\NormalTok{$ small(1/2x\^{}2|)\^{}(x=n)\_(x=0) $}
\NormalTok{$ Small(1/2x\^{}2|)\^{}(x=n)\_(x=0) $}
\NormalTok{$ smalll(1/2x\^{}2|)\^{}(x=n)\_(x=0) $}
\NormalTok{$ Smalll(1/2x\^{}2|)\^{}(x=n)\_(x=0) $}
\end{Highlighting}
\end{Shaded}

\subsection{Functions}\label{functions}

\subsubsection{big(…)}\label{biguxe2}

Applies a scale factor of \texttt{\ 125\%\ } to \texttt{\ \#lr\ }
pre-determined scale. Delimeters are enlarged by this amount compared to
what \texttt{\ \#lr\ } would normally do.

\subsubsection{Big(…)}\label{biguxe2-1}

Like \texttt{\ big(...)\ } , but applies a scale factor of
\texttt{\ 156.25\%\ } .

\subsubsection{bigg(…)}\label{bigguxe2}

Like \texttt{\ big(...)\ } , but applies a scale factor of
\texttt{\ 195.313\%\ } .

\subsubsection{Bigg(…)}\label{bigguxe2-1}

Like \texttt{\ big(...)\ } , but applies a scale factor of
\texttt{\ 244.141\%\ } .

\subsubsection{small(…)}\label{smalluxe2}

Applies a scale factor of \texttt{\ 80\%\ } to \texttt{\ \#lr\ }
pre-determined scale. Delimeters are shrunk by this amount compared to
what \texttt{\ \#lr\ } would normally do. This does \emph{not} exist in
standard LaTeX, but is necessary in this package because these functions
scale the output of \texttt{\ \#lr\ } , so delimeter sizes will get
larger depending on the content.

\subsubsection{Small(…)}\label{smalluxe2-1}

Like \texttt{\ small(...)\ } , but applies a scale factor of
\texttt{\ 64\%\ } .

\subsubsection{smalll(…)}\label{smallluxe2}

Like \texttt{\ small(...)\ } , but applies a scale factor of
\texttt{\ 51.2\%\ } .

\subsubsection{Smalll(…)}\label{smallluxe2-1}

Like \texttt{\ small(...)\ } , but applies a scale factor of
\texttt{\ 40.96\%\ } .

\subsubsection{How to add}\label{how-to-add}

Copy this into your project and use the import as \texttt{\ embiggen\ }

\begin{verbatim}
#import "@preview/embiggen:0.0.1"
\end{verbatim}

\includesvg[width=0.16667in,height=0.16667in]{/assets/icons/16-copy.svg}

Check the docs for
\href{https://typst.app/docs/reference/scripting/\#packages}{more
information on how to import packages} .

\subsubsection{About}\label{about}

\begin{description}
\tightlist
\item[Author :]
\href{mailto:dev.quantum9innovation@gmail.com}{Ananth Venkatesh}
\item[License:]
GPL-3.0-or-later
\item[Current version:]
0.0.1
\item[Last updated:]
June 18, 2024
\item[First released:]
June 18, 2024
\item[Archive size:]
13.6 kB
\href{https://packages.typst.org/preview/embiggen-0.0.1.tar.gz}{\pandocbounded{\includesvg[keepaspectratio]{/assets/icons/16-download.svg}}}
\item[Categor ies :]
\begin{itemize}
\tightlist
\item[]
\item
  \pandocbounded{\includesvg[keepaspectratio]{/assets/icons/16-text.svg}}
  \href{https://typst.app/universe/search/?category=text}{Text}
\item
  \pandocbounded{\includesvg[keepaspectratio]{/assets/icons/16-speak.svg}}
  \href{https://typst.app/universe/search/?category=report}{Report}
\item
  \pandocbounded{\includesvg[keepaspectratio]{/assets/icons/16-atom.svg}}
  \href{https://typst.app/universe/search/?category=paper}{Paper}
\end{itemize}
\end{description}

\subsubsection{Where to report issues?}\label{where-to-report-issues}

This package is a project of Ananth Venkatesh . You can also try to ask
for help with this package on the \href{https://forum.typst.app}{Forum}
.

Please report this package to the Typst team using the
\href{https://typst.app/contact}{contact form} if you believe it is a
safety hazard or infringes upon your rights.

\phantomsection\label{versions}
\subsubsection{Version history}\label{version-history}

\begin{longtable}[]{@{}ll@{}}
\toprule\noalign{}
Version & Release Date \\
\midrule\noalign{}
\endhead
\bottomrule\noalign{}
\endlastfoot
0.0.1 & June 18, 2024 \\
\end{longtable}

Typst GmbH did not create this package and cannot guarantee correct
functionality of this package or compatibility with any version of the
Typst compiler or app.


\section{Package List LaTeX/kinase.tex}
\title{typst.app/universe/package/kinase}

\phantomsection\label{banner}
\section{kinase}\label{kinase}

{ 0.1.0 }

Easy styling for different link types like mails and urls.

\phantomsection\label{readme}
Package for easy styling of links. See
\href{https://github.com/typst/packages/raw/main/packages/preview/kinase/0.1.0/docs/manual.pdf}{Docs}
for a detailed guide. Below is an example of the functionality that is
added. The problem the package solves is that different link types
cannot be styled seperatly, but are recognized as such. This package
allows for easy styling of phone numbers, urls and mail addresses. It
provides helper functions that return regex patterns for the most common
use cases.

\begin{Shaded}
\begin{Highlighting}[]
\NormalTok{\#import "@previes/kinase:0.0.1"}

\NormalTok{\#show: make{-}link}

\NormalTok{// Insert some rules}
\NormalTok{\#update{-}link{-}style(key: l{-}mailto(), value: it =\textgreater{} strong(it), )}
\NormalTok{\#update{-}link{-}style(key: l{-}url(base: "typst\textbackslash{}.app"), value: it =\textgreater{} emph(it))}
\NormalTok{\#update{-}link{-}style(key: l{-}url(base: "google\textbackslash{}.com"), before: l{-}url(base: "typst\textbackslash{}.app"), value: it =\textgreater{} highlight(it))}
\NormalTok{\#update{-}link{-}style(key: l{-}url(base: "typst\textbackslash{}.app/docs"), value: it =\textgreater{} strong(it), before: l{-}url(base: "typst\textbackslash{}.app"))}

\NormalTok{\#link("mailto:john.smith@typst.org") \textbackslash{}}

\NormalTok{\#link("https://www.typst.app/docs")}

\NormalTok{\#link("typst.app")}

\NormalTok{\#link("+49 2422424422")}
\end{Highlighting}
\end{Shaded}

\pandocbounded{\includegraphics[keepaspectratio]{https://github.com/typst/packages/raw/main/packages/preview/kinase/0.1.0/ressources/example.png}}

\subsubsection{How to add}\label{how-to-add}

Copy this into your project and use the import as \texttt{\ kinase\ }

\begin{verbatim}
#import "@preview/kinase:0.1.0"
\end{verbatim}

\includesvg[width=0.16667in,height=0.16667in]{/assets/icons/16-copy.svg}

Check the docs for
\href{https://typst.app/docs/reference/scripting/\#packages}{more
information on how to import packages} .

\subsubsection{About}\label{about}

\begin{description}
\tightlist
\item[Author :]
Lennart Schuster
\item[License:]
MIT
\item[Current version:]
0.1.0
\item[Last updated:]
May 16, 2024
\item[First released:]
May 16, 2024
\item[Archive size:]
3.10 kB
\href{https://packages.typst.org/preview/kinase-0.1.0.tar.gz}{\pandocbounded{\includesvg[keepaspectratio]{/assets/icons/16-download.svg}}}
\end{description}

\subsubsection{Where to report issues?}\label{where-to-report-issues}

This package is a project of Lennart Schuster . You can also try to ask
for help with this package on the \href{https://forum.typst.app}{Forum}
.

Please report this package to the Typst team using the
\href{https://typst.app/contact}{contact form} if you believe it is a
safety hazard or infringes upon your rights.

\phantomsection\label{versions}
\subsubsection{Version history}\label{version-history}

\begin{longtable}[]{@{}ll@{}}
\toprule\noalign{}
Version & Release Date \\
\midrule\noalign{}
\endhead
\bottomrule\noalign{}
\endlastfoot
0.1.0 & May 16, 2024 \\
\end{longtable}

Typst GmbH did not create this package and cannot guarantee correct
functionality of this package or compatibility with any version of the
Typst compiler or app.


\section{Package List LaTeX/isc-hei-report.tex}
\title{typst.app/universe/package/isc-hei-report}

\phantomsection\label{banner}
\phantomsection\label{template-thumbnail}
\pandocbounded{\includegraphics[keepaspectratio]{https://packages.typst.org/preview/thumbnails/isc-hei-report-0.1.5-small.webp}}

\section{isc-hei-report}\label{isc-hei-report}

{ 0.1.5 }

An official report template for the \textquotesingle Informatique et
systèmes de communication\textquotesingle{} (ISC) bachelor degree
programme at the School of Engineering (HEI) in Sion, Switzerland.

{ } Officially affiliated

\href{/app?template=isc-hei-report&version=0.1.5}{Create project in app}

\phantomsection\label{readme}
\pandocbounded{\includegraphics[keepaspectratio]{https://img.shields.io/github/stars/ISC-HEI/isc-hei-report}}
\pandocbounded{\includegraphics[keepaspectratio]{https://img.shields.io/github/v/release/ISC-HEI/isc-hei-report?include_prereleases}}

\href{https://hevs.ch/isc}{\includegraphics[width=0.5\linewidth,height=\textheight,keepaspectratio]{https://raw.githubusercontent.com/ISC-HEI/isc_logos/4f8d335f7f4b99d3d83ee579ef334c201a15166a/ISC\%20Logo\%20inline\%20v1.png?raw=true}}

This is an official template for students reports for the
\href{https://isc.hevs.ch/}{ISC degree programme} at the School of
engineering in Sion.

\subsection{Using the template}\label{using-the-template}

In the \texttt{\ Typst\ } web application, start with the
\texttt{\ isc-hei-report\ } and voilÃ~ ! Using the CLI, you can
initialize the project with the command :

\begin{Shaded}
\begin{Highlighting}[]
\ExtensionTok{typst}\NormalTok{ init @preview/isc{-}hei{-}report:0.1.5}
\end{Highlighting}
\end{Shaded}

This template will initialize an sample report with sensible default
values.

\subsection{Installing fonts locally}\label{installing-fonts-locally}

If you are running \texttt{\ typst\ } locally, you might miss some of
the required fonts. For your convenience, a font download script is
included in this repos. As all the fonts are released under the
\href{https://openfontlicense.org/}{SIL Open Font License} , there are
no file inclusion issues here.

To the install the fonts locally in a Linux environment, simply type

\begin{Shaded}
\begin{Highlighting}[]
\BuiltInTok{source}\NormalTok{ install\_fonts.sh}
\end{Highlighting}
\end{Shaded}

from within the \texttt{\ fonts\ } directory.

\subsection{PDF images inclusion}\label{pdf-images-inclusion}

Unfortunately, \texttt{\ typst\ } does not support PDF file types
inclusion at the time of writing this documentation. As a temporary
workaround, PDF files can be converted to SVG via \texttt{\ pdf2svg\ } .

When used locally, creating a PDF is straightforward with the command

\begin{Shaded}
\begin{Highlighting}[]
\ExtensionTok{typst}\NormalTok{ compile report.typ}
\end{Highlighting}
\end{Shaded}

Even nicer, the following command compiles the report every time the
file is modified.

\begin{Shaded}
\begin{Highlighting}[]
\ExtensionTok{typst}\NormalTok{ watch report.typ}
\end{Highlighting}
\end{Shaded}

Another nice possibility is of course to use a VScod{[}e \textbar{}
ium{]} via the \texttt{\ Typst\ LSP\ } plugin which enables direct
compilation.

In the future, several things \emph{might} be updated, such as :

\begin{itemize}
\tightlist
\item
  {[} {]} State diagrams and UML diagrams examples
\item
  {[} {]} Glossary inclusion
\item
  {[} {]} Master thesis version of this template
\item
  {[} {]} Themes for code
\item
  {[} {]} Other nice things
\item
  {[}x{]} Appendix
\item
  {[}x{]} Acronyms inclusion
\item
  {[}x{]} Basic support for including code files
\end{itemize}

If you need any help for installing or running those tools, do not
hesitate to get in touch with its maintainer
\href{https://github.com/pmudry}{pmudry} .

You can of course also propose changes using PR or raise issues if
something is not clear. Have fun writing reports!

\href{/app?template=isc-hei-report&version=0.1.5}{Create project in app}

\subsubsection{How to use}\label{how-to-use}

Click the button above to create a new project using this template in
the Typst app.

You can also use the Typst CLI to start a new project on your computer
using this command:

\begin{verbatim}
typst init @preview/isc-hei-report:0.1.5
\end{verbatim}

\includesvg[width=0.16667in,height=0.16667in]{/assets/icons/16-copy.svg}

\subsubsection{About}\label{about}

\begin{description}
\tightlist
\item[Author :]
Pierre-André Mudry
\item[License:]
MIT
\item[Current version:]
0.1.5
\item[Last updated:]
June 17, 2024
\item[First released:]
May 1, 2024
\item[Minimum Typst version:]
0.11.1
\item[Archive size:]
680 kB
\href{https://packages.typst.org/preview/isc-hei-report-0.1.5.tar.gz}{\pandocbounded{\includesvg[keepaspectratio]{/assets/icons/16-download.svg}}}
\item[Verification:]
We verified that the author is affiliated with their institution
\pandocbounded{\includesvg[keepaspectratio]{/assets/icons/16-verified.svg}}
\item[Repository:]
\href{https://github.com/ISC-HEI/ISC-report}{GitHub}
\item[Discipline s :]
\begin{itemize}
\tightlist
\item[]
\item
  \href{https://typst.app/universe/search/?discipline=computer-science}{Computer
  Science}
\item
  \href{https://typst.app/universe/search/?discipline=engineering}{Engineering}
\end{itemize}
\item[Categor y :]
\begin{itemize}
\tightlist
\item[]
\item
  \pandocbounded{\includesvg[keepaspectratio]{/assets/icons/16-speak.svg}}
  \href{https://typst.app/universe/search/?category=report}{Report}
\end{itemize}
\end{description}

\subsubsection{Where to report issues?}\label{where-to-report-issues}

This template is a project of Pierre-André Mudry . Report issues on
\href{https://github.com/ISC-HEI/ISC-report}{their repository} . You can
also try to ask for help with this template on the
\href{https://forum.typst.app}{Forum} .

Please report this template to the Typst team using the
\href{https://typst.app/contact}{contact form} if you believe it is a
safety hazard or infringes upon your rights.

\phantomsection\label{versions}
\subsubsection{Version history}\label{version-history}

\begin{longtable}[]{@{}ll@{}}
\toprule\noalign{}
Version & Release Date \\
\midrule\noalign{}
\endhead
\bottomrule\noalign{}
\endlastfoot
0.1.5 & June 17, 2024 \\
\href{https://typst.app/universe/package/isc-hei-report/0.1.3/}{0.1.3} &
June 13, 2024 \\
\href{https://typst.app/universe/package/isc-hei-report/0.1.0/}{0.1.0} &
May 1, 2024 \\
\end{longtable}

Typst GmbH did not create this template and cannot guarantee correct
functionality of this template or compatibility with any version of the
Typst compiler or app.


\section{Package List LaTeX/zhconv.tex}
\title{typst.app/universe/package/zhconv}

\phantomsection\label{banner}
\section{zhconv}\label{zhconv}

{ 0.3.1 }

Convert Chinese text between Traditional/Simplified and regional
variants. 中æ--‡ç®€ç¹?å?Šåœ°å?€è©žè½‰æ?›

\phantomsection\label{readme}
zhconv-typst converts Chinese text between Traditional, Simplified and
regional variants in typst, utilizing
\href{https://github.com/Gowee/zhconv-rs}{zhconv-rs} .

\subsection{Usage}\label{usage}

To use the \texttt{\ zhconv\ } plugin in your Typst project, import it
as follows:

\begin{Shaded}
\begin{Highlighting}[]
\NormalTok{\#import "@preview/zhconv:0.3.1": zhconv}
\end{Highlighting}
\end{Shaded}

\subsubsection{Text Conversion}\label{text-conversion}

The primary function provided by this package is \texttt{\ zhconv\ } ,
which converts strings or nested contents to a target Chinese variant.

\begin{Shaded}
\begin{Highlighting}[]
\NormalTok{\#zhconv(content, "target{-}variant", wikitext: false)}
\end{Highlighting}
\end{Shaded}

\begin{itemize}
\tightlist
\item
  \texttt{\ content\ } : The text or content to be converted.
\item
  \texttt{\ target-variant\ } : The target Chinese variant (e.g.,
  \texttt{\ "zh-hant"\ } , \texttt{\ "zh-hans"\ } , \texttt{\ "zh-cn"\ }
  , \texttt{\ "zh-tw"\ } , \texttt{\ "zh-hk"\ } ).
\item
  \texttt{\ wikitext\ } : An optional boolean flag to specify if the
  text should be processed as wikitext (default is \texttt{\ false\ } ).
\end{itemize}

\paragraph{Example}\label{example}

\subparagraph{Convert a string}\label{convert-a-string}

\begin{Shaded}
\begin{Highlighting}[]
\NormalTok{\#let text = "互联网"}
\NormalTok{Original: \#text}
\NormalTok{{-} \#emph([zh{-}HK]): \#zhconv(text, "zh{-}hk")}
\NormalTok{{-} \#emph([zh{-}TW]): \#zhconv(text, "zh{-}tw")}
\end{Highlighting}
\end{Shaded}

\subparagraph{Convert nested contents}\label{convert-nested-contents}

\begin{Shaded}
\begin{Highlighting}[]
\NormalTok{\#zhconv([}
\NormalTok{柳外輕雷池上雨 \textbackslash{}}
\NormalTok{雨聲滴碎荷聲 \textbackslash{}}

\NormalTok{小樓西角斷虹明 \textbackslash{}}
\NormalTok{闌干倚處 \textbackslash{}}
\NormalTok{待得月華生 \textbackslash{}}
\NormalTok{], "zh{-}hans")}
\end{Highlighting}
\end{Shaded}

\subsubsection{How to add}\label{how-to-add}

Copy this into your project and use the import as \texttt{\ zhconv\ }

\begin{verbatim}
#import "@preview/zhconv:0.3.1"
\end{verbatim}

\includesvg[width=0.16667in,height=0.16667in]{/assets/icons/16-copy.svg}

Check the docs for
\href{https://typst.app/docs/reference/scripting/\#packages}{more
information on how to import packages} .

\subsubsection{About}\label{about}

\begin{description}
\tightlist
\item[Author :]
\href{mailto:whygowe@gmail.com}{Hung-I Wang}
\item[License:]
GPL-2.0
\item[Current version:]
0.3.1
\item[Last updated:]
August 14, 2024
\item[First released:]
August 14, 2024
\item[Archive size:]
804 kB
\href{https://packages.typst.org/preview/zhconv-0.3.1.tar.gz}{\pandocbounded{\includesvg[keepaspectratio]{/assets/icons/16-download.svg}}}
\item[Repository:]
\href{https://github.com/Gowee/zhconv-rs}{GitHub}
\end{description}

\subsubsection{Where to report issues?}\label{where-to-report-issues}

This package is a project of Hung-I Wang . Report issues on
\href{https://github.com/Gowee/zhconv-rs}{their repository} . You can
also try to ask for help with this package on the
\href{https://forum.typst.app}{Forum} .

Please report this package to the Typst team using the
\href{https://typst.app/contact}{contact form} if you believe it is a
safety hazard or infringes upon your rights.

\phantomsection\label{versions}
\subsubsection{Version history}\label{version-history}

\begin{longtable}[]{@{}ll@{}}
\toprule\noalign{}
Version & Release Date \\
\midrule\noalign{}
\endhead
\bottomrule\noalign{}
\endlastfoot
0.3.1 & August 14, 2024 \\
\end{longtable}

Typst GmbH did not create this package and cannot guarantee correct
functionality of this package or compatibility with any version of the
Typst compiler or app.


\section{Package List LaTeX/cineca.tex}
\title{typst.app/universe/package/cineca}

\phantomsection\label{banner}
\section{cineca}\label{cineca}

{ 0.4.0 }

A package to create calendar with events.

\phantomsection\label{readme}
CINECA Is Not an Electric Calendar App, but a Typst package to create
calendars with events.

\subsection{Usage}\label{usage}

The package now support creating events from ICS files (thanks
@Geronymos). To do so, read an ICS file and parse with
\texttt{\ ics-parser()\ } .

\begin{Shaded}
\begin{Highlighting}[]
\NormalTok{\#let events2 = ics{-}parser(read("sample.ics")).map(event =\textgreater{} (}
\NormalTok{  event.dtstart, }
\NormalTok{  event.dtstart,}
\NormalTok{  event.dtend,}
\NormalTok{  event.summary}
\NormalTok{))}

\NormalTok{\#calendar(events2, hour{-}range: (10, 14))}
\end{Highlighting}
\end{Shaded}

\subsubsection{Day view}\label{day-view}

\texttt{\ calendar(events,\ hour-range,\ minute-height,\ template,\ stroke)\ }

Parameters:

\begin{itemize}
\tightlist
\item
  \texttt{\ events\ } : An array of events. Each item is a 4-element
  array:

  \begin{itemize}
  \tightlist
  \item
    Date. Can be \texttt{\ datetime()\ } or valid args of
    \texttt{\ day()\ } .
  \item
    Start time. Can be valid args of \texttt{\ time()\ } .
  \item
    End time. Can be valid args of \texttt{\ time()\ } .
  \item
    Event body. Can be anything. Passed to the template.body to show
    more details.
  \end{itemize}
\item
  \texttt{\ hour-range\ } : Then range of hours, affacting the range of
  the calendar. Default: \texttt{\ (8,\ 20)\ } .
\item
  \texttt{\ minute-height\ } : Height of per minute. Each minute occupys
  a row. This number is to control the height of each row. Default:
  \texttt{\ 0.8pt\ } .
\item
  \texttt{\ template\ } : Templates for headers, times, or events. It
  takes a dictionary of the following entries: \texttt{\ header\ } ,
  \texttt{\ time\ } , and \texttt{\ event\ } . Default: \texttt{\ (:)\ }
  .
\item
  \texttt{\ stroke\ } : A stroke style to control the style of the
  default stroke, or a function taking two parameters
  \texttt{\ (x,\ y)\ } to control the stroke. The first row is the
  dates, and the first column is the times. Default: \texttt{\ none\ } .
\end{itemize}

\begin{quote}
{[}!NOTE{]} See below for more details about the format of start time
and end time.
\end{quote}

Example:

\pandocbounded{\includegraphics[keepaspectratio]{https://github.com/typst/packages/raw/main/packages/preview/cineca/0.4.0/test/day-view.png}}

\subsubsection{Month view}\label{month-view}

\texttt{\ calendar-month(events,\ template,\ sunday-first,\ ..args)\ }

\begin{itemize}
\tightlist
\item
  \texttt{\ events\ } : Event list. Each element is a two-element array.

  \begin{itemize}
  \tightlist
  \item
    Day. A datetime object.
  \item
    Additional information for showing a day. It actually depends on the
    template \texttt{\ day-body\ } . For the deafult template, it
    requires a content.
  \end{itemize}
\item
  \texttt{\ template\ } : Templates for headers, times, or events. It
  takes a dictionary of the following entries: \texttt{\ day-body\ } ,
  \texttt{\ day-head\ } , \texttt{\ month-head\ } , and
  \texttt{\ layout\ } .
\item
  \texttt{\ sunday-first\ } : Whether to put sunday as the first day of
  a week.
\item
  \texttt{\ ..args\ } : Additional arguments for the calendar’s grid.
\end{itemize}

Example:

\begin{Shaded}
\begin{Highlighting}[]
\NormalTok{\#let events = (}
\NormalTok{  (daytime("2024{-}2{-}1", "9:0:0"), [Lecture]),}
\NormalTok{  (daytime("2024{-}2{-}1", "10:0:0"), [Tutorial]),}
\NormalTok{  (daytime("2024{-}2{-}2", "10:0:0"), [Meeting]),}
\NormalTok{  (daytime("2024{-}2{-}10", "12:0:0"), [Lunch]),}
\NormalTok{  (daytime("2024{-}2{-}25", "8:0:0"), [Train]),}
\NormalTok{)}

\NormalTok{\#calendar{-}month(}
\NormalTok{  events,}
\NormalTok{  sunday{-}first: false,}
\NormalTok{  template: (}
\NormalTok{    month{-}head: (content) =\textgreater{} strong(content)}
\NormalTok{  )}
\NormalTok{)}
\end{Highlighting}
\end{Shaded}

\begin{Shaded}
\begin{Highlighting}[]
\NormalTok{\#let events2 = (}
\NormalTok{  (datetime(year: 2024, month: 5, day: 1, hour: 9, minute: 0, second: 0), ([Lecture], blue)),}
\NormalTok{  (datetime(year: 2024, month: 5, day: 1, hour: 10, minute: 0, second: 0), ([Tutorial], red)),}
\NormalTok{  (datetime(year: 2024, month: 5, day: 1, hour: 11, minute: 0, second: 0), [Lab]),}
\NormalTok{)}

\NormalTok{\#calendar{-}month(}
\NormalTok{  events2,}
\NormalTok{  sunday{-}first: true,}
\NormalTok{  rows: (2em,) * 2 + (8em,),}
\NormalTok{  template: (}
\NormalTok{    day{-}body: (day, events) =\textgreater{} \{}
\NormalTok{      show: block.with(width: 100\%, height: 100\%, inset: 2pt)}
\NormalTok{      set align(left + top)}
\NormalTok{      stack(}
\NormalTok{        spacing: 2pt,}
\NormalTok{        pad(bottom: 4pt, text(weight: "bold", day.display("[day]"))),}
\NormalTok{        ..events.map(((d, e)) =\textgreater{} \{}
\NormalTok{          let col = if type(e) == array and e.len() \textgreater{} 1 \{ e.at(1) \} else \{ yellow \}}
\NormalTok{          show: block.with(}
\NormalTok{            fill: col.lighten(40\%),}
\NormalTok{            stroke: col,}
\NormalTok{            width: 100\%,}
\NormalTok{            inset: 2pt,}
\NormalTok{            radius: 2pt}
\NormalTok{          )}
\NormalTok{          d.display("[hour]")}
\NormalTok{          h(4pt)}
\NormalTok{          if type(e) == array \{ e.at(0) \} else \{ e \}}
\NormalTok{        \})}
\NormalTok{      )}
\NormalTok{    \}}
\NormalTok{  )}
\NormalTok{)}
\end{Highlighting}
\end{Shaded}

\pandocbounded{\includegraphics[keepaspectratio]{https://github.com/typst/packages/raw/main/packages/preview/cineca/0.4.0/test/month-view.png}}

\subsubsection{Month-summary view}\label{month-summary-view}

\texttt{\ calendar-month-summary(events,\ template,\ sunday-first,\ ..args)\ }

\begin{itemize}
\tightlist
\item
  \texttt{\ events\ } : Event list. Each element is a two-element array.

  \begin{itemize}
  \tightlist
  \item
    Day. A datetime object.
  \item
    Additional information for showing a day. It actually depends on the
    template \texttt{\ day-body\ } . For the deafult template, it
    requires an array of two elements.

    \begin{itemize}
    \tightlist
    \item
      Shape. A function specify how to darw the shape, such as
      \texttt{\ circle\ } .
    \item
      Arguments. Further arguments for render a shape.
    \end{itemize}
  \end{itemize}
\item
  \texttt{\ template\ } : Templates for headers, times, or events. It
  takes a dictionary of the following entries: \texttt{\ day-body\ } ,
  \texttt{\ day-head\ } , \texttt{\ month-head\ } , and
  \texttt{\ layout\ } .
\item
  \texttt{\ sunday-first\ } : Whether to put sunday as the first day of
  a week.
\item
  \texttt{\ ..args\ } : Additional arguments for the calendar’s grid.
\end{itemize}

Example:

\begin{Shaded}
\begin{Highlighting}[]
\NormalTok{\#let events = (}
\NormalTok{  (day("2024{-}02{-}21"), (circle, (stroke: color.green, inset: 2pt))),}
\NormalTok{  (day("2024{-}02{-}22"), (circle, (stroke: color.green, inset: 2pt))),}
\NormalTok{  (day("2024{-}05{-}27"), (circle, (stroke: color.green, inset: 2pt))),}
\NormalTok{  (day("2024{-}05{-}28"), (circle, (stroke: color.blue, inset: 2pt))),}
\NormalTok{  (day("2024{-}05{-}29"), (circle, (stroke: color.blue, inset: 2pt))),}
\NormalTok{  (day("2024{-}06{-}03"), (circle, (stroke: color.blue, inset: 2pt))),}
\NormalTok{  (day("2024{-}06{-}04"), (circle, (stroke: color.yellow, inset: 2pt))),}
\NormalTok{  (day("2024{-}06{-}05"), (circle, (stroke: color.yellow, inset: 2pt))),}
\NormalTok{  (day("2024{-}06{-}10"), (circle, (stroke: color.red, inset: 2pt))),}
\NormalTok{)}

\NormalTok{\#calendar{-}month{-}summary(}
\NormalTok{  events: events}
\NormalTok{)}

\NormalTok{\#calendar{-}month{-}summary(}
\NormalTok{  events: events,}
\NormalTok{  sunday{-}first: true}
\NormalTok{)}

\NormalTok{// An empty calendar}
\NormalTok{\#calendar{-}month{-}summary(}
\NormalTok{  events: (}
\NormalTok{    (datetime(year: 2024, month: 05, day: 21), (none,)),}
\NormalTok{  ),}
\NormalTok{  stroke: 1pt,}
\NormalTok{)}
\end{Highlighting}
\end{Shaded}

\pandocbounded{\includegraphics[keepaspectratio]{https://github.com/typst/packages/raw/main/packages/preview/cineca/0.4.0/test/month-summary.png}}

\subsection{Day/Time/Daytime Format}\label{daytimedaytime-format}

In addition to using \texttt{\ datetime()\ } to set up time, the package
provided some other ways, supported by functions \texttt{\ day()\ } ,
\texttt{\ time()\ } , and \texttt{\ daytime()\ } .

\begin{Shaded}
\begin{Highlighting}[]
\NormalTok{{-} \#time(8)}
\NormalTok{{-} \#time(8, 10)}
\NormalTok{{-} \#time(8, 10, 30)}
\NormalTok{{-} \#time("8.30")}
\NormalTok{{-} \#time(decimal("12.10"))}
\NormalTok{{-} \#time(14.10)            // 24{-}hour format}
\NormalTok{{-} \#time("8:10:08")}

\NormalTok{{-} \#day(2024)}
\NormalTok{{-} \#day(2024, 2)}
\NormalTok{{-} \#day(2024, 2, 5)    // year, month, day}
\NormalTok{{-} \#day("2024{-}3{-}7")    // ISO format (year{-}month{-}day)}
\NormalTok{{-} \#day("26/12/2024")  // British format (day/month/year)}

\NormalTok{{-} \#daytime(2024)}
\NormalTok{{-} \#daytime(2024, 2)}
\NormalTok{{-} \#daytime(2024, 2, 5)}
\NormalTok{{-} \#daytime(2024, 2, 5, 8)}
\NormalTok{{-} \#daytime(2024, 2, 5, 8, 10)}
\NormalTok{{-} \#daytime("2024{-}6{-}1", 8)}
\NormalTok{{-} \#daytime("2024{-}6{-}1", 8, 10)}
\NormalTok{{-} \#daytime("2024{-}6{-}1", 8, 10, 30)}
\NormalTok{{-} \#daytime(2024, "12:00")}
\NormalTok{{-} \#daytime(2024, 2, "12:00")}
\NormalTok{{-} \#daytime(2024, 2, 5, "12:00")}
\NormalTok{{-} \#daytime("2024{-}3{-}7", "11:30:45")}
\NormalTok{{-} \#daytime("2024{-}12{-}26 8:30")}
\end{Highlighting}
\end{Shaded}

\subsection{Limitations}\label{limitations}

\begin{itemize}
\tightlist
\item
  Page breaking may be incorrect.
\item
  Items will overlap when they happens at the same time.
\end{itemize}

\subsubsection{How to add}\label{how-to-add}

Copy this into your project and use the import as \texttt{\ cineca\ }

\begin{verbatim}
#import "@preview/cineca:0.4.0"
\end{verbatim}

\includesvg[width=0.16667in,height=0.16667in]{/assets/icons/16-copy.svg}

Check the docs for
\href{https://typst.app/docs/reference/scripting/\#packages}{more
information on how to import packages} .

\subsubsection{About}\label{about}

\begin{description}
\tightlist
\item[Author :]
HPDell
\item[License:]
MIT
\item[Current version:]
0.4.0
\item[Last updated:]
November 25, 2024
\item[First released:]
April 1, 2024
\item[Archive size:]
7.04 kB
\href{https://packages.typst.org/preview/cineca-0.4.0.tar.gz}{\pandocbounded{\includesvg[keepaspectratio]{/assets/icons/16-download.svg}}}
\item[Repository:]
\href{https://github.com/HPdell/typst-cineca}{GitHub}
\item[Categor ies :]
\begin{itemize}
\tightlist
\item[]
\item
  \pandocbounded{\includesvg[keepaspectratio]{/assets/icons/16-layout.svg}}
  \href{https://typst.app/universe/search/?category=layout}{Layout}
\item
  \pandocbounded{\includesvg[keepaspectratio]{/assets/icons/16-chart.svg}}
  \href{https://typst.app/universe/search/?category=visualization}{Visualization}
\end{itemize}
\end{description}

\subsubsection{Where to report issues?}\label{where-to-report-issues}

This package is a project of HPDell . Report issues on
\href{https://github.com/HPdell/typst-cineca}{their repository} . You
can also try to ask for help with this package on the
\href{https://forum.typst.app}{Forum} .

Please report this package to the Typst team using the
\href{https://typst.app/contact}{contact form} if you believe it is a
safety hazard or infringes upon your rights.

\phantomsection\label{versions}
\subsubsection{Version history}\label{version-history}

\begin{longtable}[]{@{}ll@{}}
\toprule\noalign{}
Version & Release Date \\
\midrule\noalign{}
\endhead
\bottomrule\noalign{}
\endlastfoot
0.4.0 & November 25, 2024 \\
\href{https://typst.app/universe/package/cineca/0.3.0/}{0.3.0} &
November 18, 2024 \\
\href{https://typst.app/universe/package/cineca/0.2.1/}{0.2.1} & July 1,
2024 \\
\href{https://typst.app/universe/package/cineca/0.2.0/}{0.2.0} & May 22,
2024 \\
\href{https://typst.app/universe/package/cineca/0.1.0/}{0.1.0} & April
1, 2024 \\
\end{longtable}

Typst GmbH did not create this package and cannot guarantee correct
functionality of this package or compatibility with any version of the
Typst compiler or app.


\section{Package List LaTeX/jlyfish.tex}
\title{typst.app/universe/package/jlyfish}

\phantomsection\label{banner}
\section{jlyfish}\label{jlyfish}

{ 0.1.0 }

Julia code evaluation inside your Typst document

\phantomsection\label{readme}
\pandocbounded{\includesvg[keepaspectratio]{https://github.com/typst/packages/raw/main/packages/preview/jlyfish/0.1.0/assets/logo.svg}}

Jlyfish is a package for Julia and Typst that allows you to integrate
Julia computations in your Typst document.

\href{https://github.com/andreasKroepelin/TypstJlyfish.jl/wiki}{\pandocbounded{\includegraphics[keepaspectratio]{https://img.shields.io/badge/docs-wiki-blue}}}
\pandocbounded{\includegraphics[keepaspectratio]{https://img.shields.io/github/license/andreasKroepelin/TypstJlyfish.jl}}
\pandocbounded{\includegraphics[keepaspectratio]{https://img.shields.io/github/v/release/andreasKroepelin/TypstJlyfish.jl}}
\href{https://github.com/andreasKroepelin/TypstJlyfish.jl}{\pandocbounded{\includegraphics[keepaspectratio]{https://img.shields.io/github/stars/andreasKroepelin/TypstJlyfish.jl}}}

You should use Jlyfish if you want to write a Typst document and have
some of the content automatically produced by Julia code but want the
source code for that within your document source. It fills a similar
role as \href{https://github.com/gpoore/pythontex}{PythonTeX} does for
Python and LaTeX. Note that this is different from tools like
\href{https://quarto.org/}{Quarto} where you write documents in
Markdown, also integrate some Julia code, but then might use Typst only
as a backend to produce the final document.

See below for a quick introduction or read the
\href{https://github.com/andreasKroepelin/TypstJlyfish.jl/wiki}{wiki}
for an in depth explanation.

Since Jlyfish builds a bridge between Julia and Typst, we also have to
get two things running. First, install the Julia package
\texttt{\ TypstJlyfish\ } from the general registry by executing

\begin{Shaded}
\begin{Highlighting}[]
\NormalTok{julia\textgreater{} ]}

\NormalTok{(@v1.10) pkg\textgreater{} add TypstJlyfish}
\end{Highlighting}
\end{Shaded}

You only have to do this once. (It is like installing and using the
Pluto notebook system, if you are familiar with that.)

When you want to use Jlyfish in a Typst document (say,
\texttt{\ your-document.typ\ } ), add the following line at the top:

\begin{Shaded}
\begin{Highlighting}[]
\NormalTok{\#import "@preview/jlyfish:0.1.0": *}
\end{Highlighting}
\end{Shaded}

Then, open a Julia REPL and run

\begin{Shaded}
\begin{Highlighting}[]
\NormalTok{julia\textgreater{} import TypstJlyfish}

\NormalTok{julia\textgreater{} TypstJlyfish.watch("your{-}document.typ")}
\end{Highlighting}
\end{Shaded}

Jlyfish facilitates the communication between Julia and Typst via a JSON
file. By default, Jlyfish uses the name of your document and adds a
\texttt{\ -jlyfish.json\ } , so \texttt{\ your-document.typ\ } would
become \texttt{\ your-document-jlyfish.json\ } . This can be configured,
of course.

To let Typst know of the computed data in the JSON file, add the
following line to your document:

\begin{Shaded}
\begin{Highlighting}[]
\NormalTok{\#read{-}julia{-}output(json("your{-}document{-}jlyfish.json"))}
\end{Highlighting}
\end{Shaded}

You can then place some Julia code in your Typst source using the
\texttt{\ \#jl\ } function:

\begin{Shaded}
\begin{Highlighting}[]
\NormalTok{What is the sum of the whole numbers from one to a hundred? \#jl(\textasciigrave{}sum(1:100)\textasciigrave{})}
\end{Highlighting}
\end{Shaded}

Head over to the
\href{https://github.com/andreasKroepelin/TypstJlyfish.jl/wiki}{wiki} to
learn more!

Just to show what is possible with Jlyfish:

\pandocbounded{\includesvg[keepaspectratio]{https://github.com/typst/packages/raw/main/packages/preview/jlyfish/0.1.0/examples/demo.svg}}

\begin{Shaded}
\begin{Highlighting}[]
\NormalTok{\#import "@preview/jlyfish:0.1.0": *}

\NormalTok{\#set page(width: auto, height: auto, margin: 1em)}
\NormalTok{\#set text(font: "Alegreya Sans")}
\NormalTok{\#let note = text.with(size: .7em, fill: luma(100), style: "italic")}

\NormalTok{\#read{-}julia{-}output(json("demo{-}jlyfish.json"))}
\NormalTok{\#jl{-}pkg("Colors", "Typstry", "Makie", "CairoMakie")}

\NormalTok{\#grid(}
\NormalTok{  columns: 2,}
\NormalTok{  gutter: 1em,}
\NormalTok{  align: top,}
\NormalTok{  [}
\NormalTok{    \#note[Generate Typst code in Julia:]}

\NormalTok{    \#set text(size: 4em)}
\NormalTok{    \#jl(\textasciigrave{}\textasciigrave{}\textasciigrave{}julia}
\NormalTok{      using Typstry, Colors}

\NormalTok{      parts = map([:red, :green, :purple], ["Ju", "li", "a"]) do name, text}
\NormalTok{        color = hex(Colors.JULIA\_LOGO\_COLORS[name])}
\NormalTok{        "\#text(fill: rgb(\textbackslash{}"$color\textbackslash{}"))[$text]"}
\NormalTok{      end}
\NormalTok{      TypstText(join(parts))}
\NormalTok{    \textasciigrave{}\textasciigrave{}\textasciigrave{})}
\NormalTok{  ],}
\NormalTok{  [}
\NormalTok{    \#note[Produce images in Julia:]}

\NormalTok{    \#set image(width: 10em)}
\NormalTok{    \#jl(recompute: false, \textasciigrave{}\textasciigrave{}\textasciigrave{}}
\NormalTok{      using Makie, CairoMakie}

\NormalTok{      as = {-}2.2:.01:.7}
\NormalTok{      bs = {-}1.5:.01:1.5}
\NormalTok{      C = [a + b * im for a in as, b in bs]}
\NormalTok{      function mandelbrot(c)}
\NormalTok{        z = c}
\NormalTok{        i = 1}
\NormalTok{        while i \textless{} 100 \&\& abs2(z) \textless{} 4}
\NormalTok{          z = z\^{}2 + c}
\NormalTok{          i += 1}
\NormalTok{        end}
\NormalTok{        i}
\NormalTok{      end}

\NormalTok{      contour(as, bs, mandelbrot.(C), axis = (;aspect = DataAspect()))}
\NormalTok{    \textasciigrave{}\textasciigrave{}\textasciigrave{})}
\NormalTok{  ],}
\NormalTok{  [}
\NormalTok{    \#note[Hand over raw data from Julia to Typst:]}
\NormalTok{    \#let barchart(counts) = \{}
\NormalTok{      set align(bottom)}
\NormalTok{      let bars = counts.map(count =\textgreater{} rect(}
\NormalTok{        width: .3em,}
\NormalTok{        height: count * 9em,}
\NormalTok{        stroke: white,}
\NormalTok{        fill: blue,}
\NormalTok{      ))}
\NormalTok{      stack(dir: ltr, ..bars)}
\NormalTok{    \}}

\NormalTok{    \#jl{-}raw(fn: it =\textgreater{} barchart(it.result.data), \textasciigrave{}\textasciigrave{}\textasciigrave{}julia}
\NormalTok{      p = .5}
\NormalTok{      n = 40}
\NormalTok{      counts = zeros(n + 1)}
\NormalTok{      for \_ in 1:10\_000}
\NormalTok{        count = 0}
\NormalTok{        for \_ in 1:n}
\NormalTok{          if rand() \textless{} p}
\NormalTok{            count += 1}
\NormalTok{          end}
\NormalTok{        end}
\NormalTok{        counts[count + 1] += 1}
\NormalTok{      end}

\NormalTok{      counts ./= maximum(counts)}
\NormalTok{      lo, hi = findfirst(\textgreater{}(1e{-}3), counts), findlast(\textgreater{}(1e{-}3), counts)}
\NormalTok{      counts[lo:hi]}
\NormalTok{    \textasciigrave{}\textasciigrave{}\textasciigrave{})}
\NormalTok{  ],}
\NormalTok{  [}
\NormalTok{    \#note[See errors, stdout, and logs:]}

\NormalTok{    \#jl(\textasciigrave{}\textasciigrave{}\textasciigrave{}julia}
\NormalTok{      println("Hello from stdout!")}
\NormalTok{      @info "Something to note" n p}
\NormalTok{      @warn "You should read this!"}
\NormalTok{      this\_does\_not\_exist}
\NormalTok{    \textasciigrave{}\textasciigrave{}\textasciigrave{})}
\NormalTok{  ]}
\NormalTok{)}
\end{Highlighting}
\end{Shaded}

\subsubsection{How to add}\label{how-to-add}

Copy this into your project and use the import as \texttt{\ jlyfish\ }

\begin{verbatim}
#import "@preview/jlyfish:0.1.0"
\end{verbatim}

\includesvg[width=0.16667in,height=0.16667in]{/assets/icons/16-copy.svg}

Check the docs for
\href{https://typst.app/docs/reference/scripting/\#packages}{more
information on how to import packages} .

\subsubsection{About}\label{about}

\begin{description}
\tightlist
\item[Author :]
Andreas Kröpelin
\item[License:]
MIT
\item[Current version:]
0.1.0
\item[Last updated:]
July 8, 2024
\item[First released:]
July 8, 2024
\item[Archive size:]
2.75 kB
\href{https://packages.typst.org/preview/jlyfish-0.1.0.tar.gz}{\pandocbounded{\includesvg[keepaspectratio]{/assets/icons/16-download.svg}}}
\item[Repository:]
\href{https://github.com/andreasKroepelin/TypstJlyfish.jl}{GitHub}
\item[Categor ies :]
\begin{itemize}
\tightlist
\item[]
\item
  \pandocbounded{\includesvg[keepaspectratio]{/assets/icons/16-code.svg}}
  \href{https://typst.app/universe/search/?category=scripting}{Scripting}
\item
  \pandocbounded{\includesvg[keepaspectratio]{/assets/icons/16-hammer.svg}}
  \href{https://typst.app/universe/search/?category=utility}{Utility}
\item
  \pandocbounded{\includesvg[keepaspectratio]{/assets/icons/16-integration.svg}}
  \href{https://typst.app/universe/search/?category=integration}{Integration}
\end{itemize}
\end{description}

\subsubsection{Where to report issues?}\label{where-to-report-issues}

This package is a project of Andreas Kröpelin . Report issues on
\href{https://github.com/andreasKroepelin/TypstJlyfish.jl}{their
repository} . You can also try to ask for help with this package on the
\href{https://forum.typst.app}{Forum} .

Please report this package to the Typst team using the
\href{https://typst.app/contact}{contact form} if you believe it is a
safety hazard or infringes upon your rights.

\phantomsection\label{versions}
\subsubsection{Version history}\label{version-history}

\begin{longtable}[]{@{}ll@{}}
\toprule\noalign{}
Version & Release Date \\
\midrule\noalign{}
\endhead
\bottomrule\noalign{}
\endlastfoot
0.1.0 & July 8, 2024 \\
\end{longtable}

Typst GmbH did not create this package and cannot guarantee correct
functionality of this package or compatibility with any version of the
Typst compiler or app.


\section{Package List LaTeX/marge.tex}
\title{typst.app/universe/package/marge}

\phantomsection\label{banner}
\section{marge}\label{marge}

{ 0.1.0 }

Easy-to-use but powerful and smart margin notes.

\phantomsection\label{readme}
A package for easy-to-use but powerful and smart margin notes.

\subsection{Usage}\label{usage}

The main function provided by this package is \texttt{\ sidenote\ } ,
which allows you to create margin notes. The function takes a single
positional argument (the text of the note) and several optional keyword
arguments for customization:

\begin{longtable}[]{@{}lll@{}}
\toprule\noalign{}
Parameter & Description & Default \\
\midrule\noalign{}
\endhead
\bottomrule\noalign{}
\endlastfoot
\texttt{\ side\ } & The margin where the note should be placed. &
\texttt{\ auto\ } \\
\texttt{\ dy\ } & The custom offset by which the note should be moved
along the y-axis. & \texttt{\ 0pt\ } \\
\texttt{\ padding\ } & The space between the note and the page or
content border. & \texttt{\ 2em\ } \\
\texttt{\ gap\ } & The minimum gap between this and neighboring notes. &
\texttt{\ 0.4em\ } \\
\texttt{\ numbering\ } & How the note should be numbered. &
\texttt{\ none\ } \\
\texttt{\ counter\ } & The counter to use for numbering. &
\texttt{\ counter("sidenote")\ } \\
\texttt{\ format\ } & The “show rule� for the note. &
\texttt{\ it\ =\textgreater{}\ it.default\ } \\
\end{longtable}

The parameters allow maximum flexibility and often allow values of
different types:

\begin{itemize}
\tightlist
\item
  The \texttt{\ side\ } parameter can be set to \texttt{\ auto\ } ,
  \texttt{\ "inside"\ } , \texttt{\ "outside"\ } or any horizontal
  \texttt{\ alignment\ } value. If set to \texttt{\ auto\ } , the note
  is placed on the larger of the two margins. If they are equally large,
  it is placed on the \texttt{\ "outside"\ } margin.
\item
  If the \texttt{\ dy\ } parameter has a relative part, it is resolved
  relative to the height of the note.
\item
  The \texttt{\ padding\ } parameter can be set either to a single
  length value or a dictionary. If a dictionary is used, the keys can be
  any horizontal alignment value, as well as \texttt{\ inside\ } and
  \texttt{\ outside\ } .
\item
  With the \texttt{\ counter\ } parameter, you can for example combine
  the numbering of footnotes and sidenotes.
\end{itemize}

An especially useful feature is the \texttt{\ format\ } parameter, as it
emulates the behavior of a show rule via a function. That function is
called with the context of the note and receives a dictionary with the
following keys:

\begin{longtable}[]{@{}lll@{}}
\toprule\noalign{}
Key & Description & Value or type \\
\midrule\noalign{}
\endhead
\bottomrule\noalign{}
\endlastfoot
\texttt{\ side\ } & The side of the page the note is placed on. &
\texttt{\ left\ } or \texttt{\ right\ } \\
\texttt{\ numbering\ } & The numbering of the note. & \texttt{\ str\ } ,
\texttt{\ function\ } or \texttt{\ none\ } \\
\texttt{\ counter\ } & The counter used for numbering the note. &
\texttt{\ counter\ } \\
\texttt{\ padding\ } & The padding of the note, resolved to
\texttt{\ left\ } and \texttt{\ right\ } . & \texttt{\ dictionary\ } \\
\texttt{\ margin\ } & The size of the margin, which the note is placed
on. & \texttt{\ length\ } \\
\texttt{\ source\ } & The location in the document where the note is
defined. & \texttt{\ location\ } \\
\texttt{\ body\ } & The content of the note. & \texttt{\ str\ } \\
\texttt{\ default\ } & The default look of the note. &
\texttt{\ content\ } \\
\end{longtable}

As the dictionary itself is not an element, you cannot directly use it
within the \texttt{\ format\ } function as you would be able to in a
normal show rule. To still be able to build upon the default look of the
note without having to reconstruct it, the \texttt{\ default\ } key is
provided. The default style sets the font size to \texttt{\ 0.85em\ }
and the paragraph’s leading to \texttt{\ 0.5em\ } , matching the
default style of footnotes. This can of course be overridden.

Aside from the customizability, the package also provides automatic
overlap and overflow protection. If a note would overlap with another
note, it is moved further down the page, so that the \texttt{\ gap\ }
parameters of both notes are respected. If a note would overflow the
page, it is moved upwards, so that the bottom of the note is aligned
with the bottom of the page content. Any previous notes, which would
then overlap with the moved note, are also moved accordingly.

\subsubsection{Note about pages with automatic
width}\label{note-about-pages-with-automatic-width}

If a note is placed in the right margin of a page with width set to
\texttt{\ auto\ } , additional configuration is necessary. As the final
width of the page is not known when the note is placed, the note’s
position cannot be calculated. To place notes on the right margin of
such pages, the package provides a \texttt{\ container\ } , which is
supposed to be included in the page’s \texttt{\ background\ } or
\texttt{\ foreground\ } :

\begin{Shaded}
\begin{Highlighting}[]
\NormalTok{\#import "@preview/marge:0.1.0": sidenote, container}

\NormalTok{\#set page(width: auto, background: container)}

\NormalTok{...}
\end{Highlighting}
\end{Shaded}

The use of the \texttt{\ container\ } variable is detected automatically
by the package, so that an error can be raised when it is required but
not set.

\subsubsection{Note about layout convergence and
performance}\label{note-about-layout-convergence-and-performance}

This package makes heavy use of states and contextual blocks, causing
Typst to require multiple layout passes to fully resolve the final
layout. Usually, the limit imposed by Typst is sufficient, but I cannot
guarantee that this will remain true for large documents with a lot of
notes. If you happen to run into this limit, you can try using the
\texttt{\ container\ } variable as mentioned above, as it can reduce the
number of layout passes required.

As each layout iteration adds to the total compile time, the use of the
\texttt{\ container\ } can also be beneficial for performance reasons.
Another performance tip is to keep the size of paragraphs containing
margin notes small, as the line breaking algorithm cannot be memoized
when the paragraph contains a note.

\subsubsection{Note about how lengths are
resolved}\label{note-about-how-lengths-are-resolved}

When a length is given in a context-dependent way (i.e. in
\texttt{\ em\ } units), it is resolved relative to the font size of the
\emph{content} , not the font size of the note (which is smaller by
default). This has the unfortunate side effect that a gap set to
\texttt{\ 0pt\ } will still have some space due to the content
paragraph’s leading (which is also larger than default leading of the
note). Similarly, if notes are defined in a context with a larger font
size, the padding and gap values may unexpectedly be larger than of
neighboring notes.

\subsection{Example}\label{example}

\begin{Shaded}
\begin{Highlighting}[]
\NormalTok{\#import "@preview/marge:0.1.0": sidenote}

\NormalTok{\#set page(margin: (right: 5cm))}
\NormalTok{\#set par(justify: true)}

\NormalTok{\#let sidenote = sidenote.with(numbering: "1", padding: 1em)}

\NormalTok{The Simpsons is an iconic animated series that began in 1989}
\NormalTok{\#sidenote[The show holds the record for the most episodes of any}
\NormalTok{American sitcom.]. The show features the Simpson family: Homer,}
\NormalTok{Marge, Bart, Lisa, and Maggie. }

\NormalTok{Bart is the rebellious son who often gets into trouble, and Lisa}
\NormalTok{is the intelligent and talented daughter \#sidenote[Lisa is known}
\NormalTok{for her saxophone playing and academic achievements.]. Baby}
\NormalTok{Maggie, though silent, has had moments of surprising brilliance}
\NormalTok{\#sidenote[Maggie once shot Mr. Burns in a dramatic plot twist.].}
\end{Highlighting}
\end{Shaded}

\pandocbounded{\includesvg[keepaspectratio]{https://github.com/typst/packages/raw/main/packages/preview/marge/0.1.0/assets/example.svg}}

\subsubsection{How to add}\label{how-to-add}

Copy this into your project and use the import as \texttt{\ marge\ }

\begin{verbatim}
#import "@preview/marge:0.1.0"
\end{verbatim}

\includesvg[width=0.16667in,height=0.16667in]{/assets/icons/16-copy.svg}

Check the docs for
\href{https://typst.app/docs/reference/scripting/\#packages}{more
information on how to import packages} .

\subsubsection{About}\label{about}

\begin{description}
\tightlist
\item[Author :]
Eric Biedert
\item[License:]
MIT
\item[Current version:]
0.1.0
\item[Last updated:]
November 19, 2024
\item[First released:]
November 19, 2024
\item[Minimum Typst version:]
0.11.0
\item[Archive size:]
7.98 kB
\href{https://packages.typst.org/preview/marge-0.1.0.tar.gz}{\pandocbounded{\includesvg[keepaspectratio]{/assets/icons/16-download.svg}}}
\item[Repository:]
\href{https://github.com/EpicEricEE/typst-marge}{GitHub}
\item[Categor ies :]
\begin{itemize}
\tightlist
\item[]
\item
  \pandocbounded{\includesvg[keepaspectratio]{/assets/icons/16-package.svg}}
  \href{https://typst.app/universe/search/?category=components}{Components}
\item
  \pandocbounded{\includesvg[keepaspectratio]{/assets/icons/16-list-unordered.svg}}
  \href{https://typst.app/universe/search/?category=model}{Model}
\item
  \pandocbounded{\includesvg[keepaspectratio]{/assets/icons/16-layout.svg}}
  \href{https://typst.app/universe/search/?category=layout}{Layout}
\end{itemize}
\end{description}

\subsubsection{Where to report issues?}\label{where-to-report-issues}

This package is a project of Eric Biedert . Report issues on
\href{https://github.com/EpicEricEE/typst-marge}{their repository} . You
can also try to ask for help with this package on the
\href{https://forum.typst.app}{Forum} .

Please report this package to the Typst team using the
\href{https://typst.app/contact}{contact form} if you believe it is a
safety hazard or infringes upon your rights.

\phantomsection\label{versions}
\subsubsection{Version history}\label{version-history}

\begin{longtable}[]{@{}ll@{}}
\toprule\noalign{}
Version & Release Date \\
\midrule\noalign{}
\endhead
\bottomrule\noalign{}
\endlastfoot
0.1.0 & November 19, 2024 \\
\end{longtable}

Typst GmbH did not create this package and cannot guarantee correct
functionality of this package or compatibility with any version of the
Typst compiler or app.


\section{Package List LaTeX/example.tex}
\title{typst.app/universe/package/example}

\phantomsection\label{banner}
\section{example}\label{example}

{ 0.1.0 }

An example package.

\phantomsection\label{readme}
An example package demonstrating the structure of a Typst package.

Displays the text “This is an example!� when included and exports
four functions \texttt{\ add\ } , \texttt{\ sub\ } , \texttt{\ mul\ } ,
and \texttt{\ div\ } that perform the respective mathematical operations
on two operands.

\subsubsection{How to add}\label{how-to-add}

Copy this into your project and use the import as \texttt{\ example\ }

\begin{verbatim}
#import "@preview/example:0.1.0"
\end{verbatim}

\includesvg[width=0.16667in,height=0.16667in]{/assets/icons/16-copy.svg}

Check the docs for
\href{https://typst.app/docs/reference/scripting/\#packages}{more
information on how to import packages} .

\subsubsection{About}\label{about}

\begin{description}
\tightlist
\item[Author :]
The Typst Project Developers
\item[License:]
Unlicense
\item[Current version:]
0.1.0
\item[Last updated:]
June 28, 2023
\item[First released:]
June 28, 2023
\item[Archive size:]
1.37 kB
\href{https://packages.typst.org/preview/example-0.1.0.tar.gz}{\pandocbounded{\includesvg[keepaspectratio]{/assets/icons/16-download.svg}}}
\end{description}

\subsubsection{Where to report issues?}\label{where-to-report-issues}

This package is a project of The Typst Project Developers . You can also
try to ask for help with this package on the
\href{https://forum.typst.app}{Forum} .

Please report this package to the Typst team using the
\href{https://typst.app/contact}{contact form} if you believe it is a
safety hazard or infringes upon your rights.

\phantomsection\label{versions}
\subsubsection{Version history}\label{version-history}

\begin{longtable}[]{@{}ll@{}}
\toprule\noalign{}
Version & Release Date \\
\midrule\noalign{}
\endhead
\bottomrule\noalign{}
\endlastfoot
0.1.0 & June 28, 2023 \\
\end{longtable}

Typst GmbH did not create this package and cannot guarantee correct
functionality of this package or compatibility with any version of the
Typst compiler or app.


\section{Package List LaTeX/ionio-illustrate.tex}
\title{typst.app/universe/package/ionio-illustrate}

\phantomsection\label{banner}
\section{ionio-illustrate}\label{ionio-illustrate}

{ 0.2.0 }

Mass spectra with annotations for typst.

\phantomsection\label{readme}
\phantomsection\label{readme-top}{}

\href{https://github.com/jamesxx/ionio-illustrate/blob/master/LICENSE}{\pandocbounded{\includegraphics[keepaspectratio]{https://img.shields.io/github/license/jamesxx/ionio-illustrate}}}
\href{https://github.com/typst/packages/tree/main/packages/preview/ionio-illustrate}{\pandocbounded{\includegraphics[keepaspectratio]{https://img.shields.io/badge/typst-package-239dad}}}
\href{https://github.com/JamesxX/ionio-illustrate/tags}{\pandocbounded{\includegraphics[keepaspectratio]{https://img.shields.io/github/v/tag/jamesxx/ionio-illustrate}}}

This package implements a Cetz chart-like object for displying mass
spectrometric data in Typst documents. It allows for individually styled
mass peaks, callouts, titles, and mass callipers.\\

\href{https://github.com/jamesxx/ionio-illustrate/blob/main/manual.pdf}{\textbf{Explore
the docs »}}\\
\strut \\
\href{https://github.com/jamesxx/ionio-illustrate/issues}{Report Bug} ·
\href{https://github.com/jamesxx/ionio-illustrate/issues}{Request
Feature}

\subsection{Getting Started}\label{getting-started}

To make use of the \texttt{\ ionio-illustrate\ } package, you’ll need
to add it to your project like shown below. Make sure you are importing
a version that supports your end goal.

\begin{Shaded}
\begin{Highlighting}[]
\NormalTok{\#import "@preview/ionio{-}illustrate:0.2.0": *}
\end{Highlighting}
\end{Shaded}

Then, load in your mass spectrum data and pass it through to the package
like so. Data should be 2D array, and by default the mass-charge ratio
is in the first column, and the relative intensities are in the second
column.

\begin{Shaded}
\begin{Highlighting}[]
\NormalTok{\#let data = csv("isobutelene\_epoxide.csv")}

\NormalTok{\#let ms = mass{-}spectrum(massspec, args: (}
\NormalTok{  size: (12,6),}
\NormalTok{  range: (0,100),}
\NormalTok{)) }

\NormalTok{\#figure((ms.display)())}
\end{Highlighting}
\end{Shaded}

\pandocbounded{\includegraphics[keepaspectratio]{https://github.com/typst/packages/raw/main/packages/preview/ionio-illustrate/0.2.0/gallery/isobulelene_epoxide.typ.png}}

There are many ways to further enhance your spectrum, please check out
the manual to find out how.

(
\href{https://github.com/typst/packages/raw/main/packages/preview/ionio-illustrate/0.2.0/\#readme-top}{back
to top} )

\subsection{Roadmap}\label{roadmap}

\begin{itemize}
\tightlist
\item
  {[}x{]} Pass style options through to the plot (tracker: \#1)
\item
  {[} {]} Better placement of text depending on plot size
\item
  {[} {]} Improve default step on axes
\item
  {[} {]} Add support for callouts that are not immediately above their
  assigned peak

  \begin{itemize}
  \tightlist
  \item
    {[} {]} Automatically detect when two annotations are too close, and
    display accordingly
  \end{itemize}
\item
  {[} {]} Move to new Typst type system (waiting on upstream)
\item
  {[} {]} Add in function for displaying skeletal structure of chemical
\item
  {[} {]} Optional second axis for absolute intensity
\item
  {[} {]} Add additional display functions

  \begin{itemize}
  \tightlist
  \item
    {[} {]} Figure out function signature for multiple data sets
  \item
    {[} {]} Overlayed and shifted
  \item
    {[} {]} Horizontal reflection

    \begin{itemize}
    \tightlist
    \item
      {[} {]} How to update existing extras?
    \end{itemize}
  \end{itemize}
\end{itemize}

See the \href{https://github.com/jamesxx/ionio-illustrate/issues}{open
issues} for a full list of proposed features (and known issues).

(
\href{https://github.com/typst/packages/raw/main/packages/preview/ionio-illustrate/0.2.0/\#readme-top}{back
to top} )

\subsection{Contributing}\label{contributing}

Contributions are what make the open source community such an amazing
place to learn, inspire, and create. Any contributions you make are
\textbf{greatly appreciated} .

If you have a suggestion that would make this better, please fork the
repo and create a pull request. You can also simply open an issue with
the tag “enhancement�. Don’t forget to give the project a star!
Thanks again!

\begin{enumerate}
\tightlist
\item
  Fork the Project
\item
  Create your Feature Branch (
  \texttt{\ git\ checkout\ -b\ feature/AmazingFeature\ } )
\item
  Commit your Changes (
  \texttt{\ git\ commit\ -m\ \textquotesingle{}Add\ some\ AmazingFeature\textquotesingle{}\ }
  )
\item
  Push to the Branch (
  \texttt{\ git\ push\ origin\ feature/AmazingFeature\ } )
\item
  Open a Pull Request
\end{enumerate}

(
\href{https://github.com/typst/packages/raw/main/packages/preview/ionio-illustrate/0.2.0/\#readme-top}{back
to top} )

\subsection{License}\label{license}

Distributed under the MIT License. See
\href{https://github.com/jamesxx/ionio-illustrate/blob/master/LICENSE}{\texttt{\ LICENSE\ }}
for more information.

(
\href{https://github.com/typst/packages/raw/main/packages/preview/ionio-illustrate/0.2.0/\#readme-top}{back
to top} )

\subsection{Gallery}\label{gallery}

\pandocbounded{\includegraphics[keepaspectratio]{https://github.com/typst/packages/raw/main/packages/preview/ionio-illustrate/0.2.0/gallery/linalool.typ.png}}

(
\href{https://github.com/typst/packages/raw/main/packages/preview/ionio-illustrate/0.2.0/\#readme-top}{back
to top} )

\subsubsection{How to add}\label{how-to-add}

Copy this into your project and use the import as
\texttt{\ ionio-illustrate\ }

\begin{verbatim}
#import "@preview/ionio-illustrate:0.2.0"
\end{verbatim}

\includesvg[width=0.16667in,height=0.16667in]{/assets/icons/16-copy.svg}

Check the docs for
\href{https://typst.app/docs/reference/scripting/\#packages}{more
information on how to import packages} .

\subsubsection{About}\label{about}

\begin{description}
\tightlist
\item[Author :]
James (Fuzzy) Swift
\item[License:]
MIT
\item[Current version:]
0.2.0
\item[Last updated:]
October 22, 2023
\item[First released:]
October 21, 2023
\item[Archive size:]
5.76 kB
\href{https://packages.typst.org/preview/ionio-illustrate-0.2.0.tar.gz}{\pandocbounded{\includesvg[keepaspectratio]{/assets/icons/16-download.svg}}}
\item[Repository:]
\href{https://github.com/JamesxX/ionio-illustrate}{GitHub}
\end{description}

\subsubsection{Where to report issues?}\label{where-to-report-issues}

This package is a project of James (Fuzzy) Swift . Report issues on
\href{https://github.com/JamesxX/ionio-illustrate}{their repository} .
You can also try to ask for help with this package on the
\href{https://forum.typst.app}{Forum} .

Please report this package to the Typst team using the
\href{https://typst.app/contact}{contact form} if you believe it is a
safety hazard or infringes upon your rights.

\phantomsection\label{versions}
\subsubsection{Version history}\label{version-history}

\begin{longtable}[]{@{}ll@{}}
\toprule\noalign{}
Version & Release Date \\
\midrule\noalign{}
\endhead
\bottomrule\noalign{}
\endlastfoot
0.2.0 & October 22, 2023 \\
\href{https://typst.app/universe/package/ionio-illustrate/0.1.0/}{0.1.0}
& October 21, 2023 \\
\end{longtable}

Typst GmbH did not create this package and cannot guarantee correct
functionality of this package or compatibility with any version of the
Typst compiler or app.


\section{Package List LaTeX/big-todo.tex}
\title{typst.app/universe/package/big-todo}

\phantomsection\label{banner}
\section{big-todo}\label{big-todo}

{ 0.2.0 }

Package to insert clear TODOs, optionally with an outline.

\phantomsection\label{readme}
Create clearly visible TODOs in your document, and add an outline to
keep track of them.

\subsection{Usage}\label{usage}

Import the package

\begin{Shaded}
\begin{Highlighting}[]
\NormalTok{import "@preview/big{-}todo:0.2.0": *}
\end{Highlighting}
\end{Shaded}

And use the \texttt{\ todo\ } function to create a TODO, and the put the
\texttt{\ todo\_outline\ } somewhere to keep track of them.

\begin{Shaded}
\begin{Highlighting}[]
\NormalTok{= Pirates}

\NormalTok{Pirates, a term often associated with seafaring outlaws, have left an indelible mark on world history. The term conjures images of Jolly Roger flags, eye patches, and treasure chests, but the reality of piracy is more complex and varied than its romanticized image suggests. Historically, pirates were motivated by wealth, adventure, or desperation and were not confined to the seas of the Caribbean but roamed the waters of the Mediterranean, the South China Sea, and the Atlantic Ocean. Pirate societies were notorious for their flouting of societal norms, and many pirate ships operated under democratic principles, offering crew members an equal share in the spoils and voting rights on important decisions. To get a better picture, it\textquotesingle{}s worth looking into how the social structures onboard these pirate vessels contrasted with those on merchant or navy vessels of the same era. \#todo([Research and provide more \_detail\_ on \#underline[pirate ship] governance and societal norms ])}

\NormalTok{Pirates\textquotesingle{} influence on history extends beyond their shipboard societies, however. Many pirates played important roles in global trade, war, and politics, often acting as privateers for countries at war. At times, they acted as de facto naval forces, protecting their patron countries\textquotesingle{} interests or disrupting those of their enemies. During the Golden Age of Piracy, roughly from 1650 to 1720, pirates were a major force in the Atlantic and the Caribbean \#todo("other seas?", inline: true), attacking the heavily laden ships of the Spanish Empire and others. They have also impacted popular culture, inspiring countless books, movies, and games. But their story is not finished. Modern{-}day piracy, especially off the coast of Somalia, has become a significant issue in international shipping.}

\NormalTok{\#todo\_outline}
\end{Highlighting}
\end{Shaded}

\pandocbounded{\includegraphics[keepaspectratio]{https://user-images.githubusercontent.com/64754924/250580952-e427a139-1c6e-4eb6-9eee-c07d98875c88.png}}

The \texttt{\ todo\ } function has the follwin parameters and defaults:

\begin{longtable}[]{@{}llll@{}}
\toprule\noalign{}
Parameter & Default & Type & Description \\
\midrule\noalign{}
\endhead
\bottomrule\noalign{}
\endlastfoot
\texttt{\ body\ } & / & Content & Content of the todo \\
\texttt{\ inline\ } & false & Boolean & If true, the todo will be
inline, otherwise it will be block \\
\texttt{\ big\_text\ } & 40pt & Length & Size of the
\texttt{\ !\ TODO\ !\ } text \\
\texttt{\ small\_text\ } & 15pt & Length & Size of the content \\
\texttt{\ gap\ } & 2mm & Length & Gap between the
\texttt{\ !\ TODO\ !\ } text and the content \\
\end{longtable}

\subsubsection{How to add}\label{how-to-add}

Copy this into your project and use the import as \texttt{\ big-todo\ }

\begin{verbatim}
#import "@preview/big-todo:0.2.0"
\end{verbatim}

\includesvg[width=0.16667in,height=0.16667in]{/assets/icons/16-copy.svg}

Check the docs for
\href{https://typst.app/docs/reference/scripting/\#packages}{more
information on how to import packages} .

\subsubsection{About}\label{about}

\begin{description}
\tightlist
\item[Author :]
Raik Rohde
\item[License:]
Unlicense
\item[Current version:]
0.2.0
\item[Last updated:]
July 4, 2023
\item[First released:]
July 4, 2023
\item[Archive size:]
2.81 kB
\href{https://packages.typst.org/preview/big-todo-0.2.0.tar.gz}{\pandocbounded{\includesvg[keepaspectratio]{/assets/icons/16-download.svg}}}
\end{description}

\subsubsection{Where to report issues?}\label{where-to-report-issues}

This package is a project of Raik Rohde . You can also try to ask for
help with this package on the \href{https://forum.typst.app}{Forum} .

Please report this package to the Typst team using the
\href{https://typst.app/contact}{contact form} if you believe it is a
safety hazard or infringes upon your rights.

\phantomsection\label{versions}
\subsubsection{Version history}\label{version-history}

\begin{longtable}[]{@{}ll@{}}
\toprule\noalign{}
Version & Release Date \\
\midrule\noalign{}
\endhead
\bottomrule\noalign{}
\endlastfoot
0.2.0 & July 4, 2023 \\
\href{https://typst.app/universe/package/big-todo/0.1.0/}{0.1.0} & July
7, 2023 \\
\end{longtable}

Typst GmbH did not create this package and cannot guarantee correct
functionality of this package or compatibility with any version of the
Typst compiler or app.


\section{Package List LaTeX/fh-joanneum-iit-thesis.tex}
\title{typst.app/universe/package/fh-joanneum-iit-thesis}

\phantomsection\label{banner}
\phantomsection\label{template-thumbnail}
\pandocbounded{\includegraphics[keepaspectratio]{https://packages.typst.org/preview/thumbnails/fh-joanneum-iit-thesis-2.0.5-small.webp}}

\section{fh-joanneum-iit-thesis}\label{fh-joanneum-iit-thesis}

{ 2.0.5 }

BA or MA thesis at FH JOANNEUM

{ } Officially affiliated

\href{/app?template=fh-joanneum-iit-thesis&version=2.0.5}{Create project
in app}

\phantomsection\label{readme}
Template for Your Bachelor’s or Master’s Thesis at
\href{http://www.fh-joanneum.at/iit}{FH JOANNEUM, IIT} .

\subsubsection{TL;DR}\label{tldr}

Using the typst universe preview package/template

\begin{verbatim}
typst init @preview/fh-joanneum-iit-thesis
\end{verbatim}

\pandocbounded{\includegraphics[keepaspectratio]{https://github.com/typst/packages/raw/main/packages/preview/fh-joanneum-iit-thesis/2.0.5/thumbnail.png}}

\href{/app?template=fh-joanneum-iit-thesis&version=2.0.5}{Create project
in app}

\subsubsection{How to use}\label{how-to-use}

Click the button above to create a new project using this template in
the Typst app.

You can also use the Typst CLI to start a new project on your computer
using this command:

\begin{verbatim}
typst init @preview/fh-joanneum-iit-thesis:2.0.5
\end{verbatim}

\includesvg[width=0.16667in,height=0.16667in]{/assets/icons/16-copy.svg}

\subsubsection{About}\label{about}

\begin{description}
\tightlist
\item[Author :]
\href{https://fh-joanneum.at/iit}{IIT, FH JOANNEUM}
\item[License:]
MIT
\item[Current version:]
2.0.5
\item[Last updated:]
November 26, 2024
\item[First released:]
August 27, 2024
\item[Archive size:]
26.1 kB
\href{https://packages.typst.org/preview/fh-joanneum-iit-thesis-2.0.5.tar.gz}{\pandocbounded{\includesvg[keepaspectratio]{/assets/icons/16-download.svg}}}
\item[Verification:]
We verified that the author is affiliated with their institution
\pandocbounded{\includesvg[keepaspectratio]{/assets/icons/16-verified.svg}}
\item[Repository:]
\href{https://git-iit.fh-joanneum.at/oss/thesis-template}{git-iit.fh-joanneum.at}
\item[Categor y :]
\begin{itemize}
\tightlist
\item[]
\item
  \pandocbounded{\includesvg[keepaspectratio]{/assets/icons/16-mortarboard.svg}}
  \href{https://typst.app/universe/search/?category=thesis}{Thesis}
\end{itemize}
\end{description}

\subsubsection{Where to report issues?}\label{where-to-report-issues}

This template is a project of IIT, FH JOANNEUM . Report issues on
\href{https://git-iit.fh-joanneum.at/oss/thesis-template}{their
repository} . You can also try to ask for help with this template on the
\href{https://forum.typst.app}{Forum} .

Please report this template to the Typst team using the
\href{https://typst.app/contact}{contact form} if you believe it is a
safety hazard or infringes upon your rights.

\phantomsection\label{versions}
\subsubsection{Version history}\label{version-history}

\begin{longtable}[]{@{}ll@{}}
\toprule\noalign{}
Version & Release Date \\
\midrule\noalign{}
\endhead
\bottomrule\noalign{}
\endlastfoot
2.0.5 & November 26, 2024 \\
\href{https://typst.app/universe/package/fh-joanneum-iit-thesis/2.0.2/}{2.0.2}
& October 29, 2024 \\
\href{https://typst.app/universe/package/fh-joanneum-iit-thesis/1.2.3/}{1.2.3}
& September 19, 2024 \\
\href{https://typst.app/universe/package/fh-joanneum-iit-thesis/1.2.2/}{1.2.2}
& August 30, 2024 \\
\href{https://typst.app/universe/package/fh-joanneum-iit-thesis/1.2.0/}{1.2.0}
& August 28, 2024 \\
\href{https://typst.app/universe/package/fh-joanneum-iit-thesis/1.1.0/}{1.1.0}
& August 27, 2024 \\
\end{longtable}

Typst GmbH did not create this template and cannot guarantee correct
functionality of this template or compatibility with any version of the
Typst compiler or app.


\section{Package List LaTeX/unichar.tex}
\title{typst.app/universe/package/unichar}

\phantomsection\label{banner}
\section{unichar}\label{unichar}

{ 0.3.0 }

A partial port of the Unicode Character Database.

\phantomsection\label{readme}
This package ports part of the
\href{https://www.unicode.org/reports/tr44/}{Unicode Character Database}
to Typst. Notably, it includes information from
\href{https://unicode.org/reports/tr44/\#UnicodeData.txt}{UnicodeData.txt}
and \href{https://unicode.org/reports/tr44/\#Blocks.txt}{Blocks.txt} .

\subsection{Usage}\label{usage}

This package defines a single function: \texttt{\ codepoint\ } . It lets
you get the information related to a specific codepoint. The codepoint
can be specified as a string containing a single character, or with its
value.

\begin{Shaded}
\begin{Highlighting}[]
\NormalTok{\#codepoint("√").name \textbackslash{}}
\NormalTok{\#codepoint(sym.times).block.name \textbackslash{}}
\NormalTok{\#codepoint(0x00C9).general{-}category \textbackslash{}}
\NormalTok{\#codepoint(sym.eq).math{-}class}
\end{Highlighting}
\end{Shaded}

\pandocbounded{\includesvg[keepaspectratio]{https://github.com/typst/packages/raw/main/packages/preview/unichar/0.3.0/examples/example-1.svg}}

You can display a codepoint in the style of
\href{https://en.wikipedia.org/wiki/Template:Unichar}{Template:Unichar}
using the \texttt{\ show\ } entry:

\begin{Shaded}
\begin{Highlighting}[]
\NormalTok{\#codepoint("¤").show \textbackslash{}}
\NormalTok{\#codepoint(sym.copyright).show \textbackslash{}}
\NormalTok{\#codepoint(0x1249).show \textbackslash{}}
\NormalTok{\#codepoint(0x100000).show}
\end{Highlighting}
\end{Shaded}

\pandocbounded{\includesvg[keepaspectratio]{https://github.com/typst/packages/raw/main/packages/preview/unichar/0.3.0/examples/example-2.svg}}

\subsection{Changelog}\label{changelog}

\subsubsection{Version 0.3.0}\label{version-0.3.0}

\begin{itemize}
\item
  Add \texttt{\ math-class\ } attribute to codepoints.

  \begin{itemize}
  \tightlist
  \item
    Some codepoints have their math class overridden by Typst. This is
    the Unicode math class, not the one used by Typst.
  \end{itemize}
\item
  The \texttt{\ id\ } of codepoints now returns a string without the
  \texttt{\ "U+"\ } prefix.
\end{itemize}

\subsubsection{Version 0.2.0}\label{version-0.2.0}

\begin{itemize}
\item
  Codepoints now have an \texttt{\ id\ } attribute which is its
  corresponding “U+xxxx� string.
\item
  The \texttt{\ block\ } attribute of a codepoint now contains a
  \texttt{\ name\ } , a \texttt{\ start\ } , and a \texttt{\ size\ } .
\item
  Fix an issue that made some codepoints cause a panic.
\item
  Include data from NameAlias.txt.
\end{itemize}

\subsubsection{Version 0.1.0}\label{version-0.1.0}

\begin{itemize}
\tightlist
\item
  Add the \texttt{\ codepoint\ } function.
\end{itemize}

\subsubsection{How to add}\label{how-to-add}

Copy this into your project and use the import as \texttt{\ unichar\ }

\begin{verbatim}
#import "@preview/unichar:0.3.0"
\end{verbatim}

\includesvg[width=0.16667in,height=0.16667in]{/assets/icons/16-copy.svg}

Check the docs for
\href{https://typst.app/docs/reference/scripting/\#packages}{more
information on how to import packages} .

\subsubsection{About}\label{about}

\begin{description}
\tightlist
\item[Author :]
\href{https://github.com/MDLC01}{Malo}
\item[License:]
MIT AND Unicode-3.0
\item[Current version:]
0.3.0
\item[Last updated:]
September 19, 2024
\item[First released:]
September 14, 2024
\item[Minimum Typst version:]
0.11.0
\item[Archive size:]
202 kB
\href{https://packages.typst.org/preview/unichar-0.3.0.tar.gz}{\pandocbounded{\includesvg[keepaspectratio]{/assets/icons/16-download.svg}}}
\item[Repository:]
\href{https://github.com/MDLC01/unichar}{GitHub}
\item[Categor ies :]
\begin{itemize}
\tightlist
\item[]
\item
  \pandocbounded{\includesvg[keepaspectratio]{/assets/icons/16-code.svg}}
  \href{https://typst.app/universe/search/?category=scripting}{Scripting}
\item
  \pandocbounded{\includesvg[keepaspectratio]{/assets/icons/16-integration.svg}}
  \href{https://typst.app/universe/search/?category=integration}{Integration}
\end{itemize}
\end{description}

\subsubsection{Where to report issues?}\label{where-to-report-issues}

This package is a project of Malo . Report issues on
\href{https://github.com/MDLC01/unichar}{their repository} . You can
also try to ask for help with this package on the
\href{https://forum.typst.app}{Forum} .

Please report this package to the Typst team using the
\href{https://typst.app/contact}{contact form} if you believe it is a
safety hazard or infringes upon your rights.

\phantomsection\label{versions}
\subsubsection{Version history}\label{version-history}

\begin{longtable}[]{@{}ll@{}}
\toprule\noalign{}
Version & Release Date \\
\midrule\noalign{}
\endhead
\bottomrule\noalign{}
\endlastfoot
0.3.0 & September 19, 2024 \\
\href{https://typst.app/universe/package/unichar/0.2.0/}{0.2.0} &
September 15, 2024 \\
\href{https://typst.app/universe/package/unichar/0.1.0/}{0.1.0} &
September 14, 2024 \\
\end{longtable}

Typst GmbH did not create this package and cannot guarantee correct
functionality of this package or compatibility with any version of the
Typst compiler or app.


\section{Package List LaTeX/tuhi-booklet-vuw.tex}
\title{typst.app/universe/package/tuhi-booklet-vuw}

\phantomsection\label{banner}
\phantomsection\label{template-thumbnail}
\pandocbounded{\includegraphics[keepaspectratio]{https://packages.typst.org/preview/thumbnails/tuhi-booklet-vuw-0.1.0-small.webp}}

\section{tuhi-booklet-vuw}\label{tuhi-booklet-vuw}

{ 0.1.0 }

A course description booklet template for VUW courses.

\href{/app?template=tuhi-booklet-vuw&version=0.1.0}{Create project in
app}

\phantomsection\label{readme}
A Typst template for VUW programme descriptions. To get started:

\begin{Shaded}
\begin{Highlighting}[]
\NormalTok{typst init @preview/tuhi{-}booklet{-}vuw:0.1.0}
\end{Highlighting}
\end{Shaded}

And edit the \texttt{\ main.typ\ } example.

\pandocbounded{\includegraphics[keepaspectratio]{https://github.com/typst/packages/raw/main/packages/preview/tuhi-booklet-vuw/0.1.0/thumbnail.png}}

\subsection{Contributing}\label{contributing}

PRs are welcome! And if you encounter any bugs or have any
requests/ideas, feel free to open an issue.

\href{/app?template=tuhi-booklet-vuw&version=0.1.0}{Create project in
app}

\subsubsection{How to use}\label{how-to-use}

Click the button above to create a new project using this template in
the Typst app.

You can also use the Typst CLI to start a new project on your computer
using this command:

\begin{verbatim}
typst init @preview/tuhi-booklet-vuw:0.1.0
\end{verbatim}

\includesvg[width=0.16667in,height=0.16667in]{/assets/icons/16-copy.svg}

\subsubsection{About}\label{about}

\begin{description}
\tightlist
\item[Author :]
\href{https://github.com/baptiste}{baptiste}
\item[License:]
MPL-2.0
\item[Current version:]
0.1.0
\item[Last updated:]
July 1, 2024
\item[First released:]
July 1, 2024
\item[Archive size:]
170 kB
\href{https://packages.typst.org/preview/tuhi-booklet-vuw-0.1.0.tar.gz}{\pandocbounded{\includesvg[keepaspectratio]{/assets/icons/16-download.svg}}}
\item[Categor y :]
\begin{itemize}
\tightlist
\item[]
\item
  \pandocbounded{\includesvg[keepaspectratio]{/assets/icons/16-envelope.svg}}
  \href{https://typst.app/universe/search/?category=office}{Office}
\end{itemize}
\end{description}

\subsubsection{Where to report issues?}\label{where-to-report-issues}

This template is a project of baptiste . You can also try to ask for
help with this template on the \href{https://forum.typst.app}{Forum} .

Please report this template to the Typst team using the
\href{https://typst.app/contact}{contact form} if you believe it is a
safety hazard or infringes upon your rights.

\phantomsection\label{versions}
\subsubsection{Version history}\label{version-history}

\begin{longtable}[]{@{}ll@{}}
\toprule\noalign{}
Version & Release Date \\
\midrule\noalign{}
\endhead
\bottomrule\noalign{}
\endlastfoot
0.1.0 & July 1, 2024 \\
\end{longtable}

Typst GmbH did not create this template and cannot guarantee correct
functionality of this template or compatibility with any version of the
Typst compiler or app.


\section{Package List LaTeX/natrix.tex}
\title{typst.app/universe/package/natrix}

\phantomsection\label{banner}
\section{natrix}\label{natrix}

{ 0.1.0 }

Natural and consistent matrix for typst.

\phantomsection\label{readme}
\pandocbounded{\includesvg[keepaspectratio]{https://github.com/typst/packages/raw/main/packages/preview/natrix/0.1.0/natrix.svg}}

\texttt{\ natrix.nat\ } is a drop-in replacement for \texttt{\ mat\ }
with some additional features. \texttt{\ nat\ } ensures that each row in
your matrix should have the similar height, unless one of them becomes
too tall.

At this moment, it is recommended to use \texttt{\ nat\ } only in
display equations, but not in inline equations. This is because
\texttt{\ nat\ } looks a little bit off when used in inline equations.

\subsection{Documentation}\label{documentation}

\subsubsection{\texorpdfstring{\texttt{\ nat\ }}{ nat }}\label{nat}

Every thing is the same as \texttt{\ mat\ } in typst.

This package also provides \texttt{\ bnat\ } , \texttt{\ Bnat\ } ,
\texttt{\ vnat\ } , \texttt{\ Vnat\ } ,

\subsubsection{How to add}\label{how-to-add}

Copy this into your project and use the import as \texttt{\ natrix\ }

\begin{verbatim}
#import "@preview/natrix:0.1.0"
\end{verbatim}

\includesvg[width=0.16667in,height=0.16667in]{/assets/icons/16-copy.svg}

Check the docs for
\href{https://typst.app/docs/reference/scripting/\#packages}{more
information on how to import packages} .

\subsubsection{About}\label{about}

\begin{description}
\tightlist
\item[Author :]
Wenzhuo Liu
\item[License:]
Apache-2.0
\item[Current version:]
0.1.0
\item[Last updated:]
May 16, 2024
\item[First released:]
May 16, 2024
\item[Archive size:]
5.26 kB
\href{https://packages.typst.org/preview/natrix-0.1.0.tar.gz}{\pandocbounded{\includesvg[keepaspectratio]{/assets/icons/16-download.svg}}}
\item[Repository:]
\href{https://github.com/Enter-tainer/natrix}{GitHub}
\end{description}

\subsubsection{Where to report issues?}\label{where-to-report-issues}

This package is a project of Wenzhuo Liu . Report issues on
\href{https://github.com/Enter-tainer/natrix}{their repository} . You
can also try to ask for help with this package on the
\href{https://forum.typst.app}{Forum} .

Please report this package to the Typst team using the
\href{https://typst.app/contact}{contact form} if you believe it is a
safety hazard or infringes upon your rights.

\phantomsection\label{versions}
\subsubsection{Version history}\label{version-history}

\begin{longtable}[]{@{}ll@{}}
\toprule\noalign{}
Version & Release Date \\
\midrule\noalign{}
\endhead
\bottomrule\noalign{}
\endlastfoot
0.1.0 & May 16, 2024 \\
\end{longtable}

Typst GmbH did not create this package and cannot guarantee correct
functionality of this package or compatibility with any version of the
Typst compiler or app.


\section{Package List LaTeX/typpuccino.tex}
\title{typst.app/universe/package/typpuccino}

\phantomsection\label{banner}
\section{typpuccino}\label{typpuccino}

{ 0.1.0 }

Use catppuccin palette with Typst.

\phantomsection\label{readme}
User everyone’s favorite
\href{https://github.com/catppuccin/catppuccin}{Catppuccin color
palettes} in your Typst projects.

\subsection{Usage}\label{usage}

To use the Catppuccin color palette in your Typst project, add the
following import statement to your Typst file, and then you can use all
the colors from the Catppuccin color palette.

\begin{Shaded}
\begin{Highlighting}[]
\NormalTok{\#import "@preview/typpuccino:0.1.0": latte, frappe, macchiato, mocha}

\NormalTok{\#square(fill: mocha.red)}
\end{Highlighting}
\end{Shaded}

For more information on available colors, see the
\href{https://github.com/typst/packages/raw/main/packages/preview/typpuccino/0.1.0/example.pdf}{this
example} .

\subsubsection{How to add}\label{how-to-add}

Copy this into your project and use the import as
\texttt{\ typpuccino\ }

\begin{verbatim}
#import "@preview/typpuccino:0.1.0"
\end{verbatim}

\includesvg[width=0.16667in,height=0.16667in]{/assets/icons/16-copy.svg}

Check the docs for
\href{https://typst.app/docs/reference/scripting/\#packages}{more
information on how to import packages} .

\subsubsection{About}\label{about}

\begin{description}
\tightlist
\item[Author :]
\href{https://github.com/TeddyHuang-00}{Nan Huang}
\item[License:]
MIT
\item[Current version:]
0.1.0
\item[Last updated:]
April 13, 2024
\item[First released:]
April 13, 2024
\item[Archive size:]
42.4 kB
\href{https://packages.typst.org/preview/typpuccino-0.1.0.tar.gz}{\pandocbounded{\includesvg[keepaspectratio]{/assets/icons/16-download.svg}}}
\item[Repository:]
\href{https://github.com/TeddyHuang-00/typpuccino}{GitHub}
\item[Categor y :]
\begin{itemize}
\tightlist
\item[]
\item
  \pandocbounded{\includesvg[keepaspectratio]{/assets/icons/16-package.svg}}
  \href{https://typst.app/universe/search/?category=components}{Components}
\end{itemize}
\end{description}

\subsubsection{Where to report issues?}\label{where-to-report-issues}

This package is a project of Nan Huang . Report issues on
\href{https://github.com/TeddyHuang-00/typpuccino}{their repository} .
You can also try to ask for help with this package on the
\href{https://forum.typst.app}{Forum} .

Please report this package to the Typst team using the
\href{https://typst.app/contact}{contact form} if you believe it is a
safety hazard or infringes upon your rights.

\phantomsection\label{versions}
\subsubsection{Version history}\label{version-history}

\begin{longtable}[]{@{}ll@{}}
\toprule\noalign{}
Version & Release Date \\
\midrule\noalign{}
\endhead
\bottomrule\noalign{}
\endlastfoot
0.1.0 & April 13, 2024 \\
\end{longtable}

Typst GmbH did not create this package and cannot guarantee correct
functionality of this package or compatibility with any version of the
Typst compiler or app.


\section{Package List LaTeX/klaro-ifsc-sj.tex}
\title{typst.app/universe/package/klaro-ifsc-sj}

\phantomsection\label{banner}
\phantomsection\label{template-thumbnail}
\pandocbounded{\includegraphics[keepaspectratio]{https://packages.typst.org/preview/thumbnails/klaro-ifsc-sj-0.1.0-small.webp}}

\section{klaro-ifsc-sj}\label{klaro-ifsc-sj}

{ 0.1.0 }

Report Typst template for IFSC.

\href{/app?template=klaro-ifsc-sj&version=0.1.0}{Create project in app}

\phantomsection\label{readme}
A report Typst template for \href{https://sj.ifsc.edu.br/}{IFSC-SJ} .

\subsection{Usage}\label{usage}

You can use this template in the Typst web app by clicking “Start from
template� on the dashboard and searching for
\texttt{\ klaro-ifsc-sj\ } .

Alternatively, you can use the CLI to kick this project off using the
command

\begin{verbatim}
typst init @preview/klaro-ifsc-sj
\end{verbatim}

Typst will create a new directory with all the files needed to get you
started.

\subsection{Configuration}\label{configuration}

This template exports the \texttt{\ report\ } function with the
following named arguments:

\begin{itemize}
\tightlist
\item
  \texttt{\ title\ } : The reoirt’s title as string. This is displayed
  at the center of the cover page.
\item
  \texttt{\ subtitle\ } : The report’s subtitle as string. This is
  displayed below the title at the cover page.
\item
  \texttt{\ authors\ } : The array of authors as strings. Each author is
  displayed on a separate line at the cover page.
\item
  \texttt{\ date\ } : The date of the last revision of the report. This
  is displayed at the bottom of the cover page.
\end{itemize}

\href{/app?template=klaro-ifsc-sj&version=0.1.0}{Create project in app}

\subsubsection{How to use}\label{how-to-use}

Click the button above to create a new project using this template in
the Typst app.

You can also use the Typst CLI to start a new project on your computer
using this command:

\begin{verbatim}
typst init @preview/klaro-ifsc-sj:0.1.0
\end{verbatim}

\includesvg[width=0.16667in,height=0.16667in]{/assets/icons/16-copy.svg}

\subsubsection{About}\label{about}

\begin{description}
\tightlist
\item[Author :]
\href{https://gabrielluizep.dev}{Gabriel Luiz Espindola Pedro}
\item[License:]
MIT-0
\item[Current version:]
0.1.0
\item[Last updated:]
March 27, 2024
\item[First released:]
March 27, 2024
\item[Minimum Typst version:]
0.10.0
\item[Archive size:]
41.7 kB
\href{https://packages.typst.org/preview/klaro-ifsc-sj-0.1.0.tar.gz}{\pandocbounded{\includesvg[keepaspectratio]{/assets/icons/16-download.svg}}}
\item[Repository:]
\href{https://github.com/gabrielluizep/klaro-ifsc-sj}{GitHub}
\item[Categor y :]
\begin{itemize}
\tightlist
\item[]
\item
  \pandocbounded{\includesvg[keepaspectratio]{/assets/icons/16-speak.svg}}
  \href{https://typst.app/universe/search/?category=report}{Report}
\end{itemize}
\end{description}

\subsubsection{Where to report issues?}\label{where-to-report-issues}

This template is a project of Gabriel Luiz Espindola Pedro . Report
issues on \href{https://github.com/gabrielluizep/klaro-ifsc-sj}{their
repository} . You can also try to ask for help with this template on the
\href{https://forum.typst.app}{Forum} .

Please report this template to the Typst team using the
\href{https://typst.app/contact}{contact form} if you believe it is a
safety hazard or infringes upon your rights.

\phantomsection\label{versions}
\subsubsection{Version history}\label{version-history}

\begin{longtable}[]{@{}ll@{}}
\toprule\noalign{}
Version & Release Date \\
\midrule\noalign{}
\endhead
\bottomrule\noalign{}
\endlastfoot
0.1.0 & March 27, 2024 \\
\end{longtable}

Typst GmbH did not create this template and cannot guarantee correct
functionality of this template or compatibility with any version of the
Typst compiler or app.


\section{Package List LaTeX/codedis.tex}
\title{typst.app/universe/package/codedis}

\phantomsection\label{banner}
\section{codedis}\label{codedis}

{ 0.1.0 }

A simple package for displaying code.

\phantomsection\label{readme}
Used to display code files in Typst. Main feature is that it displays
code blocks over multiple pages in a way that implies the code block
continues onto the next page. Also a simple and intuitive syntax for
displaying code blocks.

Usage:

\begin{Shaded}
\begin{Highlighting}[]
\NormalTok{// IMPORT PACKAGE}
\NormalTok{\#import "@preview/codedis:0.1.0": code}

\NormalTok{// READ IN CODE}
\NormalTok{\#let codeblock{-}1 = read("some\_code.py")}
\NormalTok{\#let codeblock{-}2 = read("some\_code.cpp")}

\NormalTok{\#set page(numbering: "1")}
\NormalTok{\#v(80\%)}

\NormalTok{// DEFAULT LANGUAGE IS Python ("py")}
\NormalTok{\#code(codeblock{-}1)}
\NormalTok{\#code(codeblock{-}2, lang: "cpp")}
\end{Highlighting}
\end{Shaded}

Renders to:
\pandocbounded{\includegraphics[keepaspectratio]{https://github.com/AugustinWinther/codedis/assets/30674646/76bb13d5-adc8-457f-bd55-53e3fd5c5df7}}

It is very basic and limited, but it does what I need it too, and hope
that it may be of help to others. I’m most likely not going to develop
it further than this.

\subsubsection{How to add}\label{how-to-add}

Copy this into your project and use the import as \texttt{\ codedis\ }

\begin{verbatim}
#import "@preview/codedis:0.1.0"
\end{verbatim}

\includesvg[width=0.16667in,height=0.16667in]{/assets/icons/16-copy.svg}

Check the docs for
\href{https://typst.app/docs/reference/scripting/\#packages}{more
information on how to import packages} .

\subsubsection{About}\label{about}

\begin{description}
\tightlist
\item[Author :]
\href{https://winther.io}{Augustin Winther}
\item[License:]
MIT
\item[Current version:]
0.1.0
\item[Last updated:]
April 29, 2024
\item[First released:]
April 29, 2024
\item[Archive size:]
2.08 kB
\href{https://packages.typst.org/preview/codedis-0.1.0.tar.gz}{\pandocbounded{\includesvg[keepaspectratio]{/assets/icons/16-download.svg}}}
\item[Repository:]
\href{https://github.com/AugustinWinther/codedis}{GitHub}
\item[Categor y :]
\begin{itemize}
\tightlist
\item[]
\item
  \pandocbounded{\includesvg[keepaspectratio]{/assets/icons/16-package.svg}}
  \href{https://typst.app/universe/search/?category=components}{Components}
\end{itemize}
\end{description}

\subsubsection{Where to report issues?}\label{where-to-report-issues}

This package is a project of Augustin Winther . Report issues on
\href{https://github.com/AugustinWinther/codedis}{their repository} .
You can also try to ask for help with this package on the
\href{https://forum.typst.app}{Forum} .

Please report this package to the Typst team using the
\href{https://typst.app/contact}{contact form} if you believe it is a
safety hazard or infringes upon your rights.

\phantomsection\label{versions}
\subsubsection{Version history}\label{version-history}

\begin{longtable}[]{@{}ll@{}}
\toprule\noalign{}
Version & Release Date \\
\midrule\noalign{}
\endhead
\bottomrule\noalign{}
\endlastfoot
0.1.0 & April 29, 2024 \\
\end{longtable}

Typst GmbH did not create this package and cannot guarantee correct
functionality of this package or compatibility with any version of the
Typst compiler or app.


\section{Package List LaTeX/tinyset.tex}
\title{typst.app/universe/package/tinyset}

\phantomsection\label{banner}
\section{tinyset}\label{tinyset}

{ 0.1.0 }

Simple, consistent, and appealing math homework template

\phantomsection\label{readme}
Extremely simple \href{https://github.com/typst/typst}{typst} package
for writing math problem sets quickly and consistently. Under the hood
it is just typst fundamentals that could be defined by hand, however the
aim of this package is to make you faster and the code easier to read.

\subsection{Usage}\label{usage}

Import styles and create a new header. I like to copy this from the top
of the previous week’s homework (don’t forget to increment the
number!).

Example using proof, question, and part environments. Indentation in
source code is largely ignored and left to personal preference. By
default questions are numbered and each part is lettered, you can change
this based on course / instructor preference.

\begin{Shaded}
\begin{Highlighting}[]
\NormalTok{\#import "@preview/tinyset:0.1.0": *}
\NormalTok{\#header(number: 7, name: "Sylvan Franklin", class: "Math 3551 {-} Fall 2024")}

\NormalTok{+ \#qs[}
\NormalTok{Let $G\_1$ and $G\_2$ be groups, $phi : G\_1 {-}\textgreater{} G\_2$ be a homomorphism, and $H$ be}
\NormalTok{any subgroup of $G\_2$. Define}

\NormalTok{$ phi\^{}({-}1)(H) = \{g in G\_1 : phi(g) in H\}. $}

\NormalTok{+ \#pt[ }
\NormalTok{    Prove that $phi\^{}({-}1)(H)$ is a subgroup of $G\_1$.}
\NormalTok{    \#prf[ Non empty: Since $H$ is a subgroup it contains the indentity, and}
\NormalTok{    since $phi$ is a homomorphism and ... ]}
\NormalTok{]}

\NormalTok{+ \#pt[ }
\NormalTok{    What about a question that you don\textquotesingle{}t need a proof for?}
\NormalTok{    \#ans[Use the ans environment]}
\NormalTok{]}

\NormalTok{]}
\end{Highlighting}
\end{Shaded}

\subsection{Custom shorthand}\label{custom-shorthand}

Sometimes when thinking about math I find it easier to phonetically
write out these symbols instead of using the built in typst classes. For
certain others I find the original symbols annoying to type quickly.

\begin{longtable}[]{@{}ll@{}}
\toprule\noalign{}
shorthand & expansion \\
\midrule\noalign{}
\endhead
\bottomrule\noalign{}
\endlastfoot
implies / impl & ==\textgreater{} \\
iff & \textless==\textgreater{} \\
wlog & without loss of generality \\
inv() & ()\^{}(-1) \\
\end{longtable}

\subsubsection{How to add}\label{how-to-add}

Copy this into your project and use the import as \texttt{\ tinyset\ }

\begin{verbatim}
#import "@preview/tinyset:0.1.0"
\end{verbatim}

\includesvg[width=0.16667in,height=0.16667in]{/assets/icons/16-copy.svg}

Check the docs for
\href{https://typst.app/docs/reference/scripting/\#packages}{more
information on how to import packages} .

\subsubsection{About}\label{about}

\begin{description}
\tightlist
\item[Author :]
Sylvan Franklin
\item[License:]
MIT
\item[Current version:]
0.1.0
\item[Last updated:]
November 6, 2024
\item[First released:]
November 6, 2024
\item[Archive size:]
2.28 kB
\href{https://packages.typst.org/preview/tinyset-0.1.0.tar.gz}{\pandocbounded{\includesvg[keepaspectratio]{/assets/icons/16-download.svg}}}
\item[Repository:]
\href{https://github.com/sylvanfranklin/tinyset}{GitHub}
\item[Discipline :]
\begin{itemize}
\tightlist
\item[]
\item
  \href{https://typst.app/universe/search/?discipline=mathematics}{Mathematics}
\end{itemize}
\item[Categor y :]
\begin{itemize}
\tightlist
\item[]
\item
  \pandocbounded{\includesvg[keepaspectratio]{/assets/icons/16-layout.svg}}
  \href{https://typst.app/universe/search/?category=layout}{Layout}
\end{itemize}
\end{description}

\subsubsection{Where to report issues?}\label{where-to-report-issues}

This package is a project of Sylvan Franklin . Report issues on
\href{https://github.com/sylvanfranklin/tinyset}{their repository} . You
can also try to ask for help with this package on the
\href{https://forum.typst.app}{Forum} .

Please report this package to the Typst team using the
\href{https://typst.app/contact}{contact form} if you believe it is a
safety hazard or infringes upon your rights.

\phantomsection\label{versions}
\subsubsection{Version history}\label{version-history}

\begin{longtable}[]{@{}ll@{}}
\toprule\noalign{}
Version & Release Date \\
\midrule\noalign{}
\endhead
\bottomrule\noalign{}
\endlastfoot
0.1.0 & November 6, 2024 \\
\end{longtable}

Typst GmbH did not create this package and cannot guarantee correct
functionality of this package or compatibility with any version of the
Typst compiler or app.


\section{Package List LaTeX/gentle-clues.tex}
\title{typst.app/universe/package/gentle-clues}

\phantomsection\label{banner}
\section{gentle-clues}\label{gentle-clues}

{ 1.0.0 }

A package to simply create and add some admonitions to your documents.

{ } Featured Package

\phantomsection\label{readme}
Simple admonitions for typst. Add predefined or define your own.

Inspired from
\href{https://tommilligan.github.io/mdbook-admonish/}{mdbook-admonish} .

\subsection{Overview of all predefined
clues:}\label{overview-of-all-predefined-clues}

\pandocbounded{\includesvg[keepaspectratio]{https://github.com/typst/packages/raw/main/packages/preview/gentle-clues/1.0.0/gc-overview.svg}}

\subsection{Usage}\label{usage}

For full information, see the
\href{https://github.com/jomaway/typst-gentle-clues/blob/main/docs.pdf}{docs.pdf}

To use this package, simply add the following code to your document:

\begin{Shaded}
\begin{Highlighting}[]
\NormalTok{\#import "@preview/gentle{-}clues:1.0.0": *}

\NormalTok{// add an info clue}
\NormalTok{\#info[ This is the info clue ... ]}

\NormalTok{// or a tip with custom title}
\NormalTok{\#tip(title: "Best tip ever")[Check out this cool package]}
\end{Highlighting}
\end{Shaded}

\_This will create an info clue and tip clue inside your document. See
the overview for all available clues.

\subsubsection{Features}\label{features}

This package provides some features which helps to customize the clues
to your liking.

\begin{itemize}
\tightlist
\item
  Set global default for all clues
\item
  Overwrite each parameter on a single clue for changing title, color,
  etc.
\item
  Show or hide a counter value on tasks.
\item
  Define your own clues very easily.
\item
  …
\end{itemize}

For a full list see the
\href{https://github.com/jomaway/typst-gentle-clues/blob/main/docs.pdf}{documentation}
.

\subsection{Language support}\label{language-support}

This package does use
\href{https://github.com/jomaway/typst-linguify}{linguify} to support
multiple languages.

\textbf{Header titles:} The language of the header titles is detected
automatically from the \texttt{\ context\ text.lang\ } . See the file
\href{https://github.com/jomaway/typst-gentle-clues/blob/main/lib/lang.toml}{lang.toml}
for currently supported languages.

If an unsupported language is set it will fallback to english as
default. Feel free to open a PR with your language added to the
\texttt{\ lang.toml\ } file.

\subsection{License}\label{license}

\href{https://github.com/typst/packages/raw/main/packages/preview/gentle-clues/1.0.0/LICENSE}{MIT
License}

\subsection{Changelog}\label{changelog}

\href{https://github.com/typst/packages/raw/main/packages/preview/gentle-clues/1.0.0/CHANGELOG.md}{See
CHANGELOG.md}

\subsubsection{How to add}\label{how-to-add}

Copy this into your project and use the import as
\texttt{\ gentle-clues\ }

\begin{verbatim}
#import "@preview/gentle-clues:1.0.0"
\end{verbatim}

\includesvg[width=0.16667in,height=0.16667in]{/assets/icons/16-copy.svg}

Check the docs for
\href{https://typst.app/docs/reference/scripting/\#packages}{more
information on how to import packages} .

\subsubsection{About}\label{about}

\begin{description}
\tightlist
\item[Author :]
\href{https://github.com/jomaway}{Jomaway}
\item[License:]
MIT
\item[Current version:]
1.0.0
\item[Last updated:]
September 8, 2024
\item[First released:]
September 15, 2023
\item[Minimum Typst version:]
0.11.0
\item[Archive size:]
70.3 kB
\href{https://packages.typst.org/preview/gentle-clues-1.0.0.tar.gz}{\pandocbounded{\includesvg[keepaspectratio]{/assets/icons/16-download.svg}}}
\item[Repository:]
\href{https://github.com/jomaway/typst-gentle-clues}{GitHub}
\item[Categor ies :]
\begin{itemize}
\tightlist
\item[]
\item
  \pandocbounded{\includesvg[keepaspectratio]{/assets/icons/16-package.svg}}
  \href{https://typst.app/universe/search/?category=components}{Components}
\item
  \pandocbounded{\includesvg[keepaspectratio]{/assets/icons/16-chart.svg}}
  \href{https://typst.app/universe/search/?category=visualization}{Visualization}
\end{itemize}
\end{description}

\subsubsection{Where to report issues?}\label{where-to-report-issues}

This package is a project of Jomaway . Report issues on
\href{https://github.com/jomaway/typst-gentle-clues}{their repository} .
You can also try to ask for help with this package on the
\href{https://forum.typst.app}{Forum} .

Please report this package to the Typst team using the
\href{https://typst.app/contact}{contact form} if you believe it is a
safety hazard or infringes upon your rights.

\phantomsection\label{versions}
\subsubsection{Version history}\label{version-history}

\begin{longtable}[]{@{}ll@{}}
\toprule\noalign{}
Version & Release Date \\
\midrule\noalign{}
\endhead
\bottomrule\noalign{}
\endlastfoot
1.0.0 & September 8, 2024 \\
\href{https://typst.app/universe/package/gentle-clues/0.9.0/}{0.9.0} &
July 1, 2024 \\
\href{https://typst.app/universe/package/gentle-clues/0.8.0/}{0.8.0} &
April 29, 2024 \\
\href{https://typst.app/universe/package/gentle-clues/0.7.1/}{0.7.1} &
March 26, 2024 \\
\href{https://typst.app/universe/package/gentle-clues/0.7.0/}{0.7.0} &
March 18, 2024 \\
\href{https://typst.app/universe/package/gentle-clues/0.6.0/}{0.6.0} &
January 11, 2024 \\
\href{https://typst.app/universe/package/gentle-clues/0.5.0/}{0.5.0} &
January 8, 2024 \\
\href{https://typst.app/universe/package/gentle-clues/0.4.0/}{0.4.0} &
November 17, 2023 \\
\href{https://typst.app/universe/package/gentle-clues/0.3.0/}{0.3.0} &
October 20, 2023 \\
\href{https://typst.app/universe/package/gentle-clues/0.2.0/}{0.2.0} &
September 26, 2023 \\
\href{https://typst.app/universe/package/gentle-clues/0.1.0/}{0.1.0} &
September 15, 2023 \\
\end{longtable}

Typst GmbH did not create this package and cannot guarantee correct
functionality of this package or compatibility with any version of the
Typst compiler or app.


\section{Package List LaTeX/cumcm-muban.tex}
\title{typst.app/universe/package/cumcm-muban}

\phantomsection\label{banner}
\phantomsection\label{template-thumbnail}
\pandocbounded{\includegraphics[keepaspectratio]{https://packages.typst.org/preview/thumbnails/cumcm-muban-0.3.0-small.webp}}

\section{cumcm-muban}\label{cumcm-muban}

{ 0.3.0 }

为高教社æ?¯å\ldots¨å›½å¤§å­¦ç''Ÿæ•°å­¦å»ºæ¨¡ç«žèµ›è®¾è®¡çš„ Typst
模�

\href{/app?template=cumcm-muban&version=0.3.0}{Create project in app}

\phantomsection\label{readme}
cumcm-muban
是一个为高教社æ?¯å\ldots¨å›½å¤§å­¦ç''Ÿæ•°å­¦å»ºæ¨¡ç«žèµ›è®¾è®¡çš„
Typst 模�。

\subsection{使ç''¨æ--¹æ³•}\label{uxe4uxbduxe7uxe6uxb9uxe6uxb3}

ä½~å?¯ä»¥åœ¨ Typst
ç½`页åº''ç''¨ä¸­ä½¿ç''¨æ­¤æ¨¡æ?¿ï¼Œå?ªéœ€åœ¨ä»ªè¡¨æ?¿ä¸Šç‚¹å‡» “Start
from template�,然��索 cumcm-muban。

å?¦å¤--,ä½~也å?¯ä»¥ä½¿ç''¨ CLI å`½ä»¤æ?¥å?¯åŠ¨è¿™ä¸ªé¡¹ç›®ã€‚

\begin{verbatim}
typst init @preview/cumcm-muban
\end{verbatim}

Typst
将会创建一个æ--°çš„目录,å\ldots¶ä¸­åŒ\ldots å?«äº†æ‰€æœ‰ä½~开始所需è¦?çš„æ--‡ä»¶ã€‚

\subsection{é\ldots?ç½®}\label{uxe9uxe7uxbd}

此模æ?¿å¯¼å‡ºäº† cumcm 函数,åŒ\ldots å?«ä»¥ä¸‹å`½å??å?‚数:

\begin{itemize}
\tightlist
\item
  title: 论æ--‡çš„æ~‡é¢˜
\item
  problem-chosen: 选择的题目
\item
  team-number: 团队的ç¼--å?·
\item
  college-name: 高æ~¡çš„å??称
\item
  member: 团队æˆ?å`˜çš„å§``å??
\item
  advisor: 指导教师的å§``å??
\item
  date: 竞赛开始的æ---¶é---´
\item
  cover-display: 是å?¦æ˜¾ç¤ºå°?é?¢ä»¥å?Šç¼--å?·é¡µ
\item
  abstract: æ`˜è¦?å†\ldots 容åŒ\ldots 裹在 \texttt{\ {[}{]}\ } 中
\item
  keywords: å\ldots³é''®å­---å†\ldots 容åŒ\ldots 裹在 \texttt{\ ()\ }
  中,使ç''¨é€---å?·åˆ†éš''
\end{itemize}

该函数还接å?---一个å?‚æ•° \texttt{\ body\ }
,ç''¨äºŽä¼~å\ldots¥è®ºæ--‡çš„æ­£æ--‡å†\ldots 容。

该模æ?¿å°†åœ¨æ˜¾ç¤ºè§„则中使ç''¨ \texttt{\ cumcm\ }
函数进行示例调ç''¨æ?¥åˆ?始åŒ--您的项目。如果您想è¦?将现有项目更æ''¹ä¸ºä½¿ç''¨æ­¤æ¨¡æ?¿ï¼Œæ‚¨å?¯ä»¥åœ¨æ--‡ä»¶é¡¶éƒ¨æ·»åŠ~一个类似于以下的显示规则:

\begin{Shaded}
\begin{Highlighting}[]
\NormalTok{\#import "@preview/cumcm{-}muban:0.3.0": *}
\NormalTok{\#show: thmrules}

\NormalTok{\#show: cumcm.with(}
\NormalTok{  title: "全国大学生数学建模竞赛 Typst 模板",}
\NormalTok{  problem{-}chosen: "A",}
\NormalTok{  team{-}number: "1234",}
\NormalTok{  college{-}name: " ",}
\NormalTok{  member: (}
\NormalTok{    A: " ",}
\NormalTok{    B: " ",}
\NormalTok{    C: " ",}
\NormalTok{  ),}
\NormalTok{  advisor: " ",}
\NormalTok{  date: datetime(year: 2023, month: 9, day: 8),}

\NormalTok{  cover{-}display: true,}

\NormalTok{  abstract: [],}
\NormalTok{  keywords: ("Typst", "模板", "数学建模"),}
\NormalTok{)}

\NormalTok{// 正文内容}

\NormalTok{// 参考文献}
\NormalTok{\#bib(bibliography("refs.bib"))}

\NormalTok{// 附录}
\NormalTok{\#appendix("附录标题", "附录内容")}
\end{Highlighting}
\end{Shaded}

\subsection{模�预览}\label{uxe6uxe6uxe9uxe8ux2c6}

\begin{longtable}[]{@{}ccc@{}}
\toprule\noalign{}
æ­£æ--‡1 & æ­£æ--‡2 & 附录 \\
\midrule\noalign{}
\endhead
\bottomrule\noalign{}
\endlastfoot
\pandocbounded{\includegraphics[keepaspectratio]{https://raw.githubusercontent.com/a-kkiri/CUMCM-typst-template/main/template/figures/p4.jpg?raw=true}}
&
\pandocbounded{\includegraphics[keepaspectratio]{https://raw.githubusercontent.com/a-kkiri/CUMCM-typst-template/main/template/figures/p6.jpg?raw=true}}
&
\pandocbounded{\includegraphics[keepaspectratio]{https://raw.githubusercontent.com/a-kkiri/CUMCM-typst-template/main/template/figures/p10.jpg?raw=true}} \\
\end{longtable}

\subsection{âš~ï¸?注æ„?}\label{uxe2ux161-uxefuxe6uxb3uxe6}

\begin{quote}
本模æ?¿ä½¿ç''¨åˆ°çš„å­---ä½``有
中æ˜``宋ä½``(SimSun),中æ˜``é»`ä½``(SimHei),中æ˜``楷ä½``(SimKai),Times
New Romans。这些å­---ä½``为 Windows 系统å†\ldots 置,ä¸?过对于
WebAPP/Linux/MacOS 使ç''¨è€\ldots 请到ä»``åº``自行获å?--
\end{quote}

\subsection{æ›´æ''¹è®°å½•}\label{uxe6uxe6uxb9uxe8uxe5uxbd}

2024-08-20

\begin{itemize}
\tightlist
\item
  æ›´æ''¹é™„录页代ç~?框æ~·å¼?
\item
  ä¿®å¤?æ~‡é¢˜æ---~法修æ''¹çš„é---®é¢˜
\item
  增åŠ~函数 \texttt{\ appendix\ } ç''¨äºŽæ˜¾ç¤ºé™„录å†\ldots 容
\item
  å°†ç²---ä½``çš„ \texttt{\ stroke\ } å?‚数设置为 0.02857em
\end{itemize}

\href{/app?template=cumcm-muban&version=0.3.0}{Create project in app}

\subsubsection{How to use}\label{how-to-use}

Click the button above to create a new project using this template in
the Typst app.

You can also use the Typst CLI to start a new project on your computer
using this command:

\begin{verbatim}
typst init @preview/cumcm-muban:0.3.0
\end{verbatim}

\includesvg[width=0.16667in,height=0.16667in]{/assets/icons/16-copy.svg}

\subsubsection{About}\label{about}

\begin{description}
\tightlist
\item[Author :]
\href{https://github.com/a-kkiri}{Akkiri}
\item[License:]
Apache-2.0
\item[Current version:]
0.3.0
\item[Last updated:]
August 22, 2024
\item[First released:]
March 18, 2024
\item[Archive size:]
235 kB
\href{https://packages.typst.org/preview/cumcm-muban-0.3.0.tar.gz}{\pandocbounded{\includesvg[keepaspectratio]{/assets/icons/16-download.svg}}}
\item[Repository:]
\href{https://github.com/a-kkiri/CUMCM-typst-template}{GitHub}
\item[Discipline s :]
\begin{itemize}
\tightlist
\item[]
\item
  \href{https://typst.app/universe/search/?discipline=mathematics}{Mathematics}
\item
  \href{https://typst.app/universe/search/?discipline=computer-science}{Computer
  Science}
\end{itemize}
\item[Categor y :]
\begin{itemize}
\tightlist
\item[]
\item
  \pandocbounded{\includesvg[keepaspectratio]{/assets/icons/16-mortarboard.svg}}
  \href{https://typst.app/universe/search/?category=thesis}{Thesis}
\end{itemize}
\end{description}

\subsubsection{Where to report issues?}\label{where-to-report-issues}

This template is a project of Akkiri . Report issues on
\href{https://github.com/a-kkiri/CUMCM-typst-template}{their repository}
. You can also try to ask for help with this template on the
\href{https://forum.typst.app}{Forum} .

Please report this template to the Typst team using the
\href{https://typst.app/contact}{contact form} if you believe it is a
safety hazard or infringes upon your rights.

\phantomsection\label{versions}
\subsubsection{Version history}\label{version-history}

\begin{longtable}[]{@{}ll@{}}
\toprule\noalign{}
Version & Release Date \\
\midrule\noalign{}
\endhead
\bottomrule\noalign{}
\endlastfoot
0.3.0 & August 22, 2024 \\
\href{https://typst.app/universe/package/cumcm-muban/0.2.0/}{0.2.0} &
April 3, 2024 \\
\href{https://typst.app/universe/package/cumcm-muban/0.1.0/}{0.1.0} &
March 18, 2024 \\
\end{longtable}

Typst GmbH did not create this template and cannot guarantee correct
functionality of this template or compatibility with any version of the
Typst compiler or app.


\section{Package List LaTeX/solving-physics.tex}
\title{typst.app/universe/package/solving-physics}

\phantomsection\label{banner}
\section{solving-physics}\label{solving-physics}

{ 0.1.0 }

A package to formulate the solution to a physical problem

\phantomsection\label{readme}
The easiest method is to import \texttt{\ solving-physics:\ task\ } from
the \texttt{\ @preview\ } package:

\begin{Shaded}
\begin{Highlighting}[]
\NormalTok{\#import "@preview/solving{-}physics:0.1.0": *}
\end{Highlighting}
\end{Shaded}

\begin{Shaded}
\begin{Highlighting}[]
\NormalTok{\#task(}
\NormalTok{  given: [}
\NormalTok{    $mu = 0.4$ \textbackslash{}}
\NormalTok{    $g = 10$ \textbackslash{}}
\NormalTok{    $m = 20$}
\NormalTok{  ],}
\NormalTok{  find: [}
\NormalTok{    $F$ {-}{-}{-} ?}
\NormalTok{  ],}
\NormalTok{  fig: image("./example.png", width: 5cm)}
\NormalTok{)[}
\NormalTok{  \#lorem(100)}
\NormalTok{]}
\end{Highlighting}
\end{Shaded}

\pandocbounded{\includegraphics[keepaspectratio]{https://raw.githubusercontent.com/yegorweb/solving-physics/master/examples/example1.png}}

\begin{Shaded}
\begin{Highlighting}[]
\NormalTok{\#task(}
\NormalTok{  given: [}
\NormalTok{    $mu = 0.4$ \textbackslash{}}
\NormalTok{    $g = 10$ \textbackslash{}}
\NormalTok{    $m = 20$}
\NormalTok{  ],}
\NormalTok{  find: [}
\NormalTok{    $F$ {-}{-}{-} ?}
\NormalTok{  ],}
\NormalTok{  stroke: "full"}
\NormalTok{)[]}
\end{Highlighting}
\end{Shaded}

\pandocbounded{\includesvg[keepaspectratio]{https://raw.githubusercontent.com/yegorweb/solving-physics/master/examples/example2.svg}}

\begin{Shaded}
\begin{Highlighting}[]
\NormalTok{\#task(}
\NormalTok{  given: [}
\NormalTok{    $mu = 0.4$ \textbackslash{}}
\NormalTok{    $g = 10$ \textbackslash{}}
\NormalTok{    $m = 20$}
\NormalTok{  ],}
\NormalTok{  find: [}
\NormalTok{    $F$ {-}{-}{-} ?}
\NormalTok{  ],}
\NormalTok{  stroke: "find"}
\NormalTok{)[]}
\end{Highlighting}
\end{Shaded}

\pandocbounded{\includesvg[keepaspectratio]{https://raw.githubusercontent.com/yegorweb/solving-physics/master/examples/example3.svg}}

\begin{Shaded}
\begin{Highlighting}[]
\NormalTok{\#task(}
\NormalTok{  given: [}
\NormalTok{    $mu = 0.4$ \textbackslash{}}
\NormalTok{    $g = 10$ \textbackslash{}}
\NormalTok{    $m = 20$}
\NormalTok{  ],}
\NormalTok{  find: [}
\NormalTok{    $F$ {-}{-}{-} ?}
\NormalTok{  ],}
\NormalTok{  stroke: none}
\NormalTok{)[]}
\end{Highlighting}
\end{Shaded}

\pandocbounded{\includesvg[keepaspectratio]{https://raw.githubusercontent.com/yegorweb/solving-physics/master/examples/example4.svg}}

If you have so large given you may use \texttt{\ given-width\ } :

\begin{Shaded}
\begin{Highlighting}[]
\NormalTok{\#task(}
\NormalTok{  given: [}
\NormalTok{    $mu = 0.4$ \textbackslash{}}
\NormalTok{    $g = 10$ \textbackslash{}}
\NormalTok{    $m = 20$ \textbackslash{}}
\NormalTok{    \#lorem(10)}
\NormalTok{  ],}
\NormalTok{  given{-}width: 6em,}
\NormalTok{  find: [}
\NormalTok{    $F$ {-}{-}{-} ?}
\NormalTok{  ],}
\NormalTok{)[]}
\end{Highlighting}
\end{Shaded}

\pandocbounded{\includesvg[keepaspectratio]{https://raw.githubusercontent.com/yegorweb/solving-physics/master/examples/example5.svg}}

You may locate you figure on the center of body by
\texttt{\ fig-align:\ top\ +\ center\ }

\begin{Shaded}
\begin{Highlighting}[]
\NormalTok{\#task(}
\NormalTok{  given: [}
\NormalTok{    $mu = 0.4$ \textbackslash{}}
\NormalTok{    $g = 10$ \textbackslash{}}
\NormalTok{    $m = 20$}
\NormalTok{  ],}
\NormalTok{  find: [}
\NormalTok{    $F$ {-}{-}{-} ?}
\NormalTok{  ],}
\NormalTok{  fig: image("./example.png", width: 60\%),}
\NormalTok{  fig{-}align: top + center}
\NormalTok{)[}
\NormalTok{  \#lorem(100)}
\NormalTok{]}
\end{Highlighting}
\end{Shaded}

\pandocbounded{\includegraphics[keepaspectratio]{https://raw.githubusercontent.com/yegorweb/solving-physics/master/examples/example6.png}}

\subsubsection{How to add}\label{how-to-add}

Copy this into your project and use the import as
\texttt{\ solving-physics\ }

\begin{verbatim}
#import "@preview/solving-physics:0.1.0"
\end{verbatim}

\includesvg[width=0.16667in,height=0.16667in]{/assets/icons/16-copy.svg}

Check the docs for
\href{https://typst.app/docs/reference/scripting/\#packages}{more
information on how to import packages} .

\subsubsection{About}\label{about}

\begin{description}
\tightlist
\item[Author :]
Yegor Knyazev
\item[License:]
MIT
\item[Current version:]
0.1.0
\item[Last updated:]
May 13, 2024
\item[First released:]
May 13, 2024
\item[Archive size:]
1.86 kB
\href{https://packages.typst.org/preview/solving-physics-0.1.0.tar.gz}{\pandocbounded{\includesvg[keepaspectratio]{/assets/icons/16-download.svg}}}
\item[Repository:]
\href{https://github.com/yegorweb/solving-physics}{GitHub}
\item[Discipline s :]
\begin{itemize}
\tightlist
\item[]
\item
  \href{https://typst.app/universe/search/?discipline=chemistry}{Chemistry}
\item
  \href{https://typst.app/universe/search/?discipline=education}{Education}
\item
  \href{https://typst.app/universe/search/?discipline=physics}{Physics}
\end{itemize}
\item[Categor y :]
\begin{itemize}
\tightlist
\item[]
\item
  \pandocbounded{\includesvg[keepaspectratio]{/assets/icons/16-package.svg}}
  \href{https://typst.app/universe/search/?category=components}{Components}
\end{itemize}
\end{description}

\subsubsection{Where to report issues?}\label{where-to-report-issues}

This package is a project of Yegor Knyazev . Report issues on
\href{https://github.com/yegorweb/solving-physics}{their repository} .
You can also try to ask for help with this package on the
\href{https://forum.typst.app}{Forum} .

Please report this package to the Typst team using the
\href{https://typst.app/contact}{contact form} if you believe it is a
safety hazard or infringes upon your rights.

\phantomsection\label{versions}
\subsubsection{Version history}\label{version-history}

\begin{longtable}[]{@{}ll@{}}
\toprule\noalign{}
Version & Release Date \\
\midrule\noalign{}
\endhead
\bottomrule\noalign{}
\endlastfoot
0.1.0 & May 13, 2024 \\
\end{longtable}

Typst GmbH did not create this package and cannot guarantee correct
functionality of this package or compatibility with any version of the
Typst compiler or app.


\section{Package List LaTeX/formalettre.tex}
\title{typst.app/universe/package/formalettre}

\phantomsection\label{banner}
\phantomsection\label{template-thumbnail}
\pandocbounded{\includegraphics[keepaspectratio]{https://packages.typst.org/preview/thumbnails/formalettre-0.1.1-small.webp}}

\section{formalettre}\label{formalettre}

{ 0.1.1 }

French formal letter template

\href{/app?template=formalettre&version=0.1.1}{Create project in app}

\phantomsection\label{readme}
Un template destiné Ã~ écrire des lettres selon une typographie
francophone, et inspiré du package LaTeX
\href{https://ctan.org/pkg/lettre}{lettre} .

Pour utiliser le template, il est possible de recopier le fichier
exemple.

\subsection{Documentation des
variables}\label{documentation-des-variables}

\subsubsection{Expéditeur}\label{expuxe3diteur}

\begin{itemize}
\tightlist
\item
  \texttt{\ expediteur.nom\ } : nom de famille de l’expéditeur·ice,
  \textbf{requis} .
\item
  \texttt{\ expediteur.prenom\ } : prénom de l’expéditeur·ice,
  \textbf{requis} .
\item
  \texttt{\ expediteur.voie\ } : numéro de voie et nom de la voie,
  \textbf{requis} .
\item
  \texttt{\ expediteur.complement\_adresse\ } : la seconde ligne parfois
  requise dans une adresse, \emph{facultatif} .
\item
  \texttt{\ expediteur.code\_postal\ } : code postal, \textbf{requis} .
\item
  \texttt{\ expediteur.commune\ } : commune de l’expéditeur·ice,
  \textbf{requis} .
\item
  \texttt{\ expediteur.telephone\ } : numéro de téléphone. Le format
  est libre et l’affichage en police mono. \emph{Facultatif} .
\item
  \texttt{\ expediteur.email\ } : l’email fourni sera affiché en
  police mono et cliquable. \emph{Facultatif}
\item
  \texttt{\ expediteur.signature\ } : peut être \texttt{\ true\ } ou
  \texttt{\ false\ } , par défaut \texttt{\ false\ } . Prévient le
  paquet qu’une image de signature sera ajoutée, de manière Ã~
  organiser la superposition de la signature et du nom apposé en fin de
  courrier.
\end{itemize}

\subsection{Destinataire}\label{destinataire}

\begin{itemize}
\tightlist
\item
  \texttt{\ destinataire.titre\ } : titre du ou de la destinataire,
  \textbf{requis} .
\item
  \texttt{\ destinataire.voie\ } : numéro de voie et nom de la voie,
  \textbf{requis} .
\item
  \texttt{\ destinataire.complement\_adresse\ } : la seconde ligne
  parfois requise dans une adresse, \emph{facultatif} .
\item
  \texttt{\ destinataire.code\_postal\ } : code postal, \textbf{requis}
  .
\item
  \texttt{\ destinataire.commune\ } : commune de l’expéditeur·ice,
  \textbf{requis} .
\item
  \texttt{\ destinataire.sc\ } : si le courrier est envoyé “sous
  couvert� d’une hiérarchie intermédiaire, spécifier cette
  autorité. \emph{Facultatif} .
\end{itemize}

\subsection{Lettre}\label{lettre}

\begin{itemize}
\tightlist
\item
  \texttt{\ objet\ } : l’objet du courrier, \textbf{requis} .
\item
  \texttt{\ date\ } : date Ã~ indiquer sous forme libre, \textbf{requis}
  .
\item
  \texttt{\ lieu\ } : lieu de rédaction, \textbf{requis} .
\item
  \texttt{\ pj\ } : permet d’indiquer la présence de pièces jointes.
  Il est possible d’en faire une liste, par exemple :
\end{itemize}

\begin{verbatim}
pj: [
    + Dossier n°1
    + Dossier n° 2
    + Attestation
    ]
\end{verbatim}

Le texte de la lettre proprement dite se situe après la configuration
de la lettre.

À la fin de la lettre, il est possible de décommenter les deux
dernières lignes pour ajouter une image en guise de signature. Veillez
dans ce cas Ã~ positionner la varibale \texttt{\ expediteur.signature\ }
à \texttt{\ true\ } .

\href{/app?template=formalettre&version=0.1.1}{Create project in app}

\subsubsection{How to use}\label{how-to-use}

Click the button above to create a new project using this template in
the Typst app.

You can also use the Typst CLI to start a new project on your computer
using this command:

\begin{verbatim}
typst init @preview/formalettre:0.1.1
\end{verbatim}

\includesvg[width=0.16667in,height=0.16667in]{/assets/icons/16-copy.svg}

\subsubsection{About}\label{about}

\begin{description}
\tightlist
\item[Author :]
@Brndan
\item[License:]
BSD-3-Clause
\item[Current version:]
0.1.1
\item[Last updated:]
October 23, 2024
\item[First released:]
October 22, 2024
\item[Archive size:]
3.20 kB
\href{https://packages.typst.org/preview/formalettre-0.1.1.tar.gz}{\pandocbounded{\includesvg[keepaspectratio]{/assets/icons/16-download.svg}}}
\item[Repository:]
\href{https://github.com/Brndan/lettre}{GitHub}
\item[Categor y :]
\begin{itemize}
\tightlist
\item[]
\item
  \pandocbounded{\includesvg[keepaspectratio]{/assets/icons/16-envelope.svg}}
  \href{https://typst.app/universe/search/?category=office}{Office}
\end{itemize}
\end{description}

\subsubsection{Where to report issues?}\label{where-to-report-issues}

This template is a project of @Brndan . Report issues on
\href{https://github.com/Brndan/lettre}{their repository} . You can also
try to ask for help with this template on the
\href{https://forum.typst.app}{Forum} .

Please report this template to the Typst team using the
\href{https://typst.app/contact}{contact form} if you believe it is a
safety hazard or infringes upon your rights.

\phantomsection\label{versions}
\subsubsection{Version history}\label{version-history}

\begin{longtable}[]{@{}ll@{}}
\toprule\noalign{}
Version & Release Date \\
\midrule\noalign{}
\endhead
\bottomrule\noalign{}
\endlastfoot
0.1.1 & October 23, 2024 \\
\href{https://typst.app/universe/package/formalettre/0.1.0/}{0.1.0} &
October 22, 2024 \\
\end{longtable}

Typst GmbH did not create this template and cannot guarantee correct
functionality of this template or compatibility with any version of the
Typst compiler or app.


\section{Package List LaTeX/spreet.tex}
\title{typst.app/universe/package/spreet}

\phantomsection\label{banner}
\section{spreet}\label{spreet}

{ 0.1.0 }

Parse a spreadsheet.

\phantomsection\label{readme}
Spreet is a spreadsheet decoder for typst (excel/opendocument
spreadsheets). The spreadsheet will be read and parsed into a dictonary
of 2-dimensional array of strings: Each workbook in the spreadsheet is
mapped as an entry in the dictonary. Each row of the workbook is
represented as an array of strings, and all rows are summarised in a
single array.

\subsection{Example}\label{example}

\begin{Shaded}
\begin{Highlighting}[]
\NormalTok{\#import "@preview/spreet:0.1.0"}

\NormalTok{\#let excel\_data = spreet.file{-}decode("excel.xlsx")}
\NormalTok{\#let opendocument\_data = spreet.file{-}decode("opendocument.ods")}

\NormalTok{\#let excel\_data\_from\_bytes = spreet.decode(read("excel.xlsx", encoding: none))}
\NormalTok{\#let opendocument\_data\_from\_bytes = spreet.decode(read("opendocument.ods", encoding: none))}

\NormalTok{/**}
\NormalTok{excel\_data or opendocument\_data contains a dict of all worksheets}
\NormalTok{(}
\NormalTok{  Worksheet1: (}
\NormalTok{    (Row1\_Column1, Row1\_Column2),}
\NormalTok{    (Row2\_Column1, Row2\_Column2),}
\NormalTok{  ),}
\NormalTok{  Worksheet2: (}
\NormalTok{    (Row1\_Column1, Row1\_Column2),}
\NormalTok{    (Row2\_Column1, Row2\_Column2),}
\NormalTok{  )}
\NormalTok{)}
\NormalTok{**/}
\end{Highlighting}
\end{Shaded}

\subsubsection{How to add}\label{how-to-add}

Copy this into your project and use the import as \texttt{\ spreet\ }

\begin{verbatim}
#import "@preview/spreet:0.1.0"
\end{verbatim}

\includesvg[width=0.16667in,height=0.16667in]{/assets/icons/16-copy.svg}

Check the docs for
\href{https://typst.app/docs/reference/scripting/\#packages}{more
information on how to import packages} .

\subsubsection{About}\label{about}

\begin{description}
\tightlist
\item[Author :]
lublak
\item[License:]
MIT
\item[Current version:]
0.1.0
\item[Last updated:]
September 15, 2024
\item[First released:]
September 15, 2024
\item[Archive size:]
335 kB
\href{https://packages.typst.org/preview/spreet-0.1.0.tar.gz}{\pandocbounded{\includesvg[keepaspectratio]{/assets/icons/16-download.svg}}}
\item[Repository:]
\href{https://github.com/lublak/typst-spreet-package}{GitHub}
\end{description}

\subsubsection{Where to report issues?}\label{where-to-report-issues}

This package is a project of lublak . Report issues on
\href{https://github.com/lublak/typst-spreet-package}{their repository}
. You can also try to ask for help with this package on the
\href{https://forum.typst.app}{Forum} .

Please report this package to the Typst team using the
\href{https://typst.app/contact}{contact form} if you believe it is a
safety hazard or infringes upon your rights.

\phantomsection\label{versions}
\subsubsection{Version history}\label{version-history}

\begin{longtable}[]{@{}ll@{}}
\toprule\noalign{}
Version & Release Date \\
\midrule\noalign{}
\endhead
\bottomrule\noalign{}
\endlastfoot
0.1.0 & September 15, 2024 \\
\end{longtable}

Typst GmbH did not create this package and cannot guarantee correct
functionality of this package or compatibility with any version of the
Typst compiler or app.


\section{Package List LaTeX/codelst.tex}
\title{typst.app/universe/package/codelst}

\phantomsection\label{banner}
\section{codelst}\label{codelst}

{ 2.0.2 }

A typst package to render sourcecode.

\phantomsection\label{readme}
\textbf{codelst} is a \href{https://github.com/typst/typst}{Typst}
package for rendering sourcecode with line numbers and some other
additions.

\subsection{Usage}\label{usage}

Import the package from the typst preview repository:

\begin{Shaded}
\begin{Highlighting}[]
\NormalTok{\#import }\StringTok{"@preview/codelst:2.0.2"}\OperatorTok{:}\NormalTok{ sourcecode}
\end{Highlighting}
\end{Shaded}

After importing the package, simply wrap any fenced code block in a call
to \texttt{\ \#sourcecode()\ } :

\begin{Shaded}
\begin{Highlighting}[]
\NormalTok{\#import }\StringTok{"@preview/codelst:2.0.2"}\OperatorTok{:}\NormalTok{ sourcecode}

\NormalTok{\#sourcecode[}\VerbatimStringTok{\textasciigrave{}\textasciigrave{}\textasciigrave{}typ}
\VerbatimStringTok{\#show "ArtosFlow": name =\textgreater{} box[}
\VerbatimStringTok{  \#box(image(}
\VerbatimStringTok{    "logo.svg",}
\VerbatimStringTok{    height: 0.7em,}
\VerbatimStringTok{  ))}
\VerbatimStringTok{  \#name}
\VerbatimStringTok{]}

\VerbatimStringTok{This report is embedded in the}
\VerbatimStringTok{ArtosFlow project. ArtosFlow is a}
\VerbatimStringTok{project of the Artos Institute.}
\VerbatimStringTok{\textasciigrave{}\textasciigrave{}\textasciigrave{}}\NormalTok{]}
\end{Highlighting}
\end{Shaded}

\subsection{Further documentation}\label{further-documentation}

See \texttt{\ manual.pdf\ } for a comprehensive manual of the package.

See \texttt{\ example.typ\ } for some quick usage examples.

\subsection{Development}\label{development}

The documentation is created using
\href{https://github.com/jneug/typst-mantys}{Mantys} , a Typst template
for creating package documentation.

To compile the manual, Mantys needs to be available as a local package.
Refer to Mantys’ manual for instructions on how to do so.

\subsection{Changelog}\label{changelog}

\subsubsection{v2.0.1}\label{v2.0.1}

This version makes \texttt{\ codelst\ } compatible to Typst 0.11.0.
Version 2.0.1 now requires Typst 0.11.0, since there are some breaking
changes to the way counters work.

Thanks to @kilpkonn for theses changes.

\subsubsection{v2.0.0}\label{v2.0.0}

Version 2 requires Typst 0.9.0 or newer. Rendering is now done using the
new \texttt{\ raw.line\ } elements get consistent line numbers and
syntax highlighting (even if \texttt{\ showrange\ } is used). Rendering
is now done in a \texttt{\ \#table\ } .

\begin{itemize}
\tightlist
\item
  Added \texttt{\ theme\ } and \texttt{\ syntaxes\ } options to
  overwrite passed in \texttt{\ \#raw\ } values.
\item
  Breaking: Renamed \texttt{\ tab-indend\ } to \texttt{\ tab-size\ } ,
  to conform with the Typst option.
\item
  Breaking: Removed \texttt{\ continue-numbering\ } option for now. (The
  feature failed in combination with label parsing and line highlights.)
\item
  Breaking: Removed styling of line numbers via a \texttt{\ show\ }
  -rule.
\end{itemize}

\subsubsection{v1.0.0}\label{v1.0.0}

\begin{itemize}
\tightlist
\item
  Complete rewrite of code rendering.
\item
  New options for \texttt{\ \#sourcecode()\ } :

  \begin{itemize}
  \tightlist
  \item
    \texttt{\ lang\ } : Overwrite code language setting.
  \item
    \texttt{\ numbers-first\ } : First line number to show.
  \item
    \texttt{\ numbers-step\ } : Only show every n-th number.
  \item
    \texttt{\ frame\ } : Set a frame (replaces
    \texttt{\ \textless{}codelst\textgreater{}\ } label.)
  \item
    Merged \texttt{\ line-numbers\ } and \texttt{\ numbering\ } options.
  \end{itemize}
\item
  Removed \texttt{\ \#numbers-style()\ } function.

  \begin{itemize}
  \tightlist
  \item
    \texttt{\ numbers-style\ } option now gets passed
    \texttt{\ counter.display()\ } .
  \end{itemize}
\item
  Removed \texttt{\ \textless{}codelst\textgreater{}\ } label.
\item
  \texttt{\ codelst-style\ } only sets \texttt{\ breakable\ } for
  figures.
\item
  New \texttt{\ codelst\ } function to setup a catchall show rules for
  \texttt{\ raw\ } text.
\item
  \texttt{\ label-regex:\ none\ } disables labels parsing.
\item
  Code improvements and refactorings.
\end{itemize}

\subsubsection{v0.0.5}\label{v0.0.5}

\begin{itemize}
\tightlist
\item
  Fixed insets for line highlights.
\item
  Added \texttt{\ numbers-width\ } option to manually set width of line
  numbers column.

  \begin{itemize}
  \tightlist
  \item
    This allows line numbers on margins by setting
    \texttt{\ numbers-width\ } to \texttt{\ 0pt\ } or a negative number
    like \texttt{\ -1em\ } .
  \end{itemize}
\end{itemize}

\subsubsection{v0.0.4}\label{v0.0.4}

\begin{itemize}
\tightlist
\item
  Fixed issue with context unaware syntax highlighting.
\end{itemize}

\subsubsection{v0.0.3}\label{v0.0.3}

\begin{itemize}
\tightlist
\item
  Removed call to \texttt{\ \#read()\ } from \texttt{\ \#sourcefile()\ }
  .
\item
  Added \texttt{\ continue-numbering\ } argument to
  \texttt{\ \#sourcecode()\ } .
\item
  Fixed problem with \texttt{\ showrange\ } having out of range line
  numbers.
\end{itemize}

\subsubsection{v0.0.2}\label{v0.0.2}

\begin{itemize}
\tightlist
\item
  Added a comprehensive manual.
\item
  Fixed crash for missing \texttt{\ lang\ } attribute in
  \texttt{\ raw\ } element.
\end{itemize}

\subsubsection{v0.0.1}\label{v0.0.1}

\begin{itemize}
\tightlist
\item
  Initial version submitted to typst/packages.
\end{itemize}

\subsubsection{How to add}\label{how-to-add}

Copy this into your project and use the import as \texttt{\ codelst\ }

\begin{verbatim}
#import "@preview/codelst:2.0.2"
\end{verbatim}

\includesvg[width=0.16667in,height=0.16667in]{/assets/icons/16-copy.svg}

Check the docs for
\href{https://typst.app/docs/reference/scripting/\#packages}{more
information on how to import packages} .

\subsubsection{About}\label{about}

\begin{description}
\tightlist
\item[Author :]
Jonas Neugebauer
\item[License:]
MIT
\item[Current version:]
2.0.2
\item[Last updated:]
October 23, 2024
\item[First released:]
July 24, 2023
\item[Minimum Typst version:]
0.12.0
\item[Archive size:]
5.41 kB
\href{https://packages.typst.org/preview/codelst-2.0.2.tar.gz}{\pandocbounded{\includesvg[keepaspectratio]{/assets/icons/16-download.svg}}}
\item[Repository:]
\href{https://github.com/jneug/typst-codelst}{GitHub}
\item[Discipline s :]
\begin{itemize}
\tightlist
\item[]
\item
  \href{https://typst.app/universe/search/?discipline=computer-science}{Computer
  Science}
\item
  \href{https://typst.app/universe/search/?discipline=mathematics}{Mathematics}
\item
  \href{https://typst.app/universe/search/?discipline=education}{Education}
\item
  \href{https://typst.app/universe/search/?discipline=linguistics}{Linguistics}
\end{itemize}
\item[Categor ies :]
\begin{itemize}
\tightlist
\item[]
\item
  \pandocbounded{\includesvg[keepaspectratio]{/assets/icons/16-package.svg}}
  \href{https://typst.app/universe/search/?category=components}{Components}
\item
  \pandocbounded{\includesvg[keepaspectratio]{/assets/icons/16-layout.svg}}
  \href{https://typst.app/universe/search/?category=layout}{Layout}
\end{itemize}
\end{description}

\subsubsection{Where to report issues?}\label{where-to-report-issues}

This package is a project of Jonas Neugebauer . Report issues on
\href{https://github.com/jneug/typst-codelst}{their repository} . You
can also try to ask for help with this package on the
\href{https://forum.typst.app}{Forum} .

Please report this package to the Typst team using the
\href{https://typst.app/contact}{contact form} if you believe it is a
safety hazard or infringes upon your rights.

\phantomsection\label{versions}
\subsubsection{Version history}\label{version-history}

\begin{longtable}[]{@{}ll@{}}
\toprule\noalign{}
Version & Release Date \\
\midrule\noalign{}
\endhead
\bottomrule\noalign{}
\endlastfoot
2.0.2 & October 23, 2024 \\
\href{https://typst.app/universe/package/codelst/2.0.1/}{2.0.1} & March
19, 2024 \\
\href{https://typst.app/universe/package/codelst/2.0.0/}{2.0.0} &
November 16, 2023 \\
\href{https://typst.app/universe/package/codelst/1.0.0/}{1.0.0} & July
29, 2023 \\
\href{https://typst.app/universe/package/codelst/0.0.3/}{0.0.3} & July
24, 2023 \\
\end{longtable}

Typst GmbH did not create this package and cannot guarantee correct
functionality of this package or compatibility with any version of the
Typst compiler or app.


\section{Package List LaTeX/haw-hamburg-master-thesis.tex}
\title{typst.app/universe/package/haw-hamburg-master-thesis}

\phantomsection\label{banner}
\phantomsection\label{template-thumbnail}
\pandocbounded{\includegraphics[keepaspectratio]{https://packages.typst.org/preview/thumbnails/haw-hamburg-master-thesis-0.3.1-small.webp}}

\section{haw-hamburg-master-thesis}\label{haw-hamburg-master-thesis}

{ 0.3.1 }

Unofficial template for writing a master-thesis in the HAW Hamburg
department of Computer Science design.

\href{/app?template=haw-hamburg-master-thesis&version=0.3.1}{Create
project in app}

\phantomsection\label{readme}
This is an \textbf{\texttt{\ unofficial\ }} template for writing a
master thesis in the \texttt{\ HAW\ Hamburg\ } department of
\texttt{\ Computer\ Science\ } design using
\href{https://github.com/typst/typst}{Typst} .

\subsection{Required Fonts}\label{required-fonts}

To correctly render this template please make sure that the
\texttt{\ New\ Computer\ Modern\ } font is installed on your system.

\subsection{Usage}\label{usage}

To use this package just add the following code to your
\href{https://github.com/typst/typst}{Typst} document:

\begin{Shaded}
\begin{Highlighting}[]
\NormalTok{\#import "@preview/haw{-}hamburg:0.3.1": master{-}thesis}

\NormalTok{\#show: master{-}thesis.with(}
\NormalTok{  language: "en",}

\NormalTok{  title{-}de: "Beispiel Titel",}
\NormalTok{  keywords{-}de: ("Stichwort", "Wichtig", "Super"),}
\NormalTok{  abstract{-}de: "Beispiel Zusammenfassung",}

\NormalTok{  title{-}en: "Example title",}
\NormalTok{  keywords{-}en:  ("Keyword", "Important", "Super"),}
\NormalTok{  abstract{-}en: "Example abstract",}

\NormalTok{  author: "The Computer",}
\NormalTok{  faculty: "Engineering and Computer Science",}
\NormalTok{  department: "Computer Science",}
\NormalTok{  study{-}course: "Master of Science Computer Science",}
\NormalTok{  supervisors: ("Prof. Dr. Example", "Prof. Dr. Example"),}
\NormalTok{  submission{-}date: datetime(year: 1948, month: 12, day: 10),}
\NormalTok{  include{-}declaration{-}of{-}independent{-}processing: true,}
\NormalTok{)}
\end{Highlighting}
\end{Shaded}

\subsection{How to Compile}\label{how-to-compile}

This project contains an example setup that splits individual chapters
into different files.\\
This can cause problems when using references etc.\\
These problems can be avoided by following these steps:

\begin{itemize}
\tightlist
\item
  Make sure to always compile your \texttt{\ main.typ\ } file which
  includes all of your chapters for references to work correctly.
\item
  VSCode:

  \begin{itemize}
  \tightlist
  \item
    Install the
    \href{https://marketplace.visualstudio.com/items?itemName=myriad-dreamin.tinymist}{Tinymist
    Typst} extension.
  \item
    Make sure to start the \texttt{\ PDF\ } or
    \texttt{\ Live\ Preview\ } only from within your
    \texttt{\ main.typ\ } file.
  \item
    If problems occur it usually helps to close the preview and restart
    it from your \texttt{\ main.typ\ } file.
  \end{itemize}
\end{itemize}

\href{/app?template=haw-hamburg-master-thesis&version=0.3.1}{Create
project in app}

\subsubsection{How to use}\label{how-to-use}

Click the button above to create a new project using this template in
the Typst app.

You can also use the Typst CLI to start a new project on your computer
using this command:

\begin{verbatim}
typst init @preview/haw-hamburg-master-thesis:0.3.1
\end{verbatim}

\includesvg[width=0.16667in,height=0.16667in]{/assets/icons/16-copy.svg}

\subsubsection{About}\label{about}

\begin{description}
\tightlist
\item[Author :]
Lasse Rosenow
\item[License:]
MIT
\item[Current version:]
0.3.1
\item[Last updated:]
November 13, 2024
\item[First released:]
October 14, 2024
\item[Archive size:]
6.54 kB
\href{https://packages.typst.org/preview/haw-hamburg-master-thesis-0.3.1.tar.gz}{\pandocbounded{\includesvg[keepaspectratio]{/assets/icons/16-download.svg}}}
\item[Repository:]
\href{https://github.com/LasseRosenow/HAW-Hamburg-Typst-Template}{GitHub}
\item[Categor y :]
\begin{itemize}
\tightlist
\item[]
\item
  \pandocbounded{\includesvg[keepaspectratio]{/assets/icons/16-mortarboard.svg}}
  \href{https://typst.app/universe/search/?category=thesis}{Thesis}
\end{itemize}
\end{description}

\subsubsection{Where to report issues?}\label{where-to-report-issues}

This template is a project of Lasse Rosenow . Report issues on
\href{https://github.com/LasseRosenow/HAW-Hamburg-Typst-Template}{their
repository} . You can also try to ask for help with this template on the
\href{https://forum.typst.app}{Forum} .

Please report this template to the Typst team using the
\href{https://typst.app/contact}{contact form} if you believe it is a
safety hazard or infringes upon your rights.

\phantomsection\label{versions}
\subsubsection{Version history}\label{version-history}

\begin{longtable}[]{@{}ll@{}}
\toprule\noalign{}
Version & Release Date \\
\midrule\noalign{}
\endhead
\bottomrule\noalign{}
\endlastfoot
0.3.1 & November 13, 2024 \\
\href{https://typst.app/universe/package/haw-hamburg-master-thesis/0.3.0/}{0.3.0}
& October 14, 2024 \\
\end{longtable}

Typst GmbH did not create this template and cannot guarantee correct
functionality of this template or compatibility with any version of the
Typst compiler or app.


\section{Package List LaTeX/babel.tex}
\title{typst.app/universe/package/babel}

\phantomsection\label{banner}
\section{babel}\label{babel}

{ 0.1.1 }

Redact text by replacing it with random characters

\phantomsection\label{readme}
\href{https://typst.app/universe/package/babel}{\pandocbounded{\includegraphics[keepaspectratio]{https://img.shields.io/badge/Typst_Universe-fdfdfd?logo=typst}}}
\href{https://codeberg.org/afiaith/babel}{\pandocbounded{\includegraphics[keepaspectratio]{https://img.shields.io/badge/Git_repo-fdfdfd?logo=codeberg}}}
\href{https://github.com/typst/packages/raw/main/packages/preview/babel/0.1.1/docs/manual.pdf}{\pandocbounded{\includegraphics[keepaspectratio]{https://img.shields.io/badge/\%F0\%9F\%93\%96\%20manual-.pdf-239dad?labelColor=fdfdfd}}}
\href{https://github.com/typst/packages/raw/main/packages/preview/babel/0.1.1/LICENSE}{\pandocbounded{\includegraphics[keepaspectratio]{https://img.shields.io/badge/licence-MIT0-239dad?labelColor=fdfdfd}}}
\href{https://codeberg.org/afiaith/babel/releases/}{\pandocbounded{\includegraphics[keepaspectratio]{https://img.shields.io/gitea/v/release/afiaith/babel?gitea_url=https\%3A\%2F\%2Fcodeberg.org&labelColor=fdfdfd&color=239dad}}}
\href{https://codeberg.org/afiaith/babel/stars}{\pandocbounded{\includegraphics[keepaspectratio]{https://img.shields.io/gitea/stars/afiaith/babel?gitea_url=https\%3A\%2F\%2Fcodeberg.org&labelColor=fdfdfd&color=fdfdfd&logo=codeberg}}}

This package provides functions that replace actual text with random
characters, which is useful for redacting confidential information or
sharing the design and structure of an existing document without
disclosing the content itself. A variety of ready-made sets of
characters for replacement are available (75 in total; termed
\emph{alphabets} ), representing diverse writing systems, codes,
notations and symbols. Some of these are more conservative (such as
emulating redaction using a wide black pen) and many are more whimsical,
as demonstrated by the following example:

\begin{Shaded}
\begin{Highlighting}[]
\NormalTok{\#baffle(alphabet: "welsh")[Hello]. My \#tippex[name] is \#baffle(alphabet: "underscore")[Inigo Montoya]. You \#baffle(alphabet: "alchemy")[killed] my \#baffle(alphabet: "shavian")[father]. Prepare to \#redact[die].}

\NormalTok{Using show rules strings, regular expressions and other selectors can be redacted automatically:}

\NormalTok{\#show "jan Maja": baffle.with(alphabet: "sitelen{-}pona")}
\NormalTok{\#show regex("[a{-}zA{-}Z0{-}9.!\#$\%\&’*+/=?\^{}\_\textasciigrave{}\{|\}\textasciitilde{}{-}]+@[a{-}zA{-}Z0{-}9{-}]+(?:\textbackslash{}.[a{-}zA{-}Z0{-}9{-}]+)*"): baffle.with(alphabet: "maze{-}3") }

\NormalTok{I’m jan Maja, and my email is \textasciigrave{}foo@digitalwords.net\textasciigrave{}.}
\end{Highlighting}
\end{Shaded}

\pandocbounded{\includegraphics[keepaspectratio]{https://github.com/typst/packages/raw/main/packages/preview/babel/0.1.1/assets/example.webp}}

\subsection{ðŸ``-- The manual}\label{uxf0uxff-the-manual}

Using { Babel } is quite straightforward. A
\href{https://github.com/typst/packages/raw/main/packages/preview/babel/0.1.1/docs/manual.pdf}{\textbf{comprehensive
manual}} covers:

\begin{itemize}
\tightlist
\item
  Introductory background.
\item
  How to use the provided functions ( \texttt{\ baffle()\ } ,
  \texttt{\ redact()\ } and \texttt{\ tippex()\ } ).
\item
  A list of the provided alphabets, each demonstrated by a line of
  random text.
\end{itemize}

If the version of the precompiled manual doesn’t match the version of
the package, it means no difference between the two versions is
reflected in the manual.

\subsection{\texorpdfstring{ðŸ---¼ The Tower of { Babel
}}{ðŸ---¼ The Tower of  Babel }}\label{uxf0uxffuxbc-the-tower-of-babel}

A poster demonstrating the provided alphabets:

\href{https://github.com/typst/packages/raw/main/packages/preview/babel/0.1.1/assets/poster.webp}{\pandocbounded{\includegraphics[keepaspectratio]{https://github.com/typst/packages/raw/main/packages/preview/babel/0.1.1/assets/poster.webp}}}

\subsection{ðŸ''¨ Complementary
tools}\label{uxf0uxff-complementary-tools}

If you wish to share the Typst source files of your document, not just
the precompiled output, a tool called
\href{https://github.com/frozolotl/typst-mutilate}{\emph{Typst
Mutilate}} might be useful for you. Unlike { Babel } , it is not a Typst
package but an external tool, written in Rust. It replaces the content
of a Typst document with random words selected from a wordlist or random
characters (similarly to { Babel } ), changing the document in place (so
make sure to run it on a \emph{copy} !). As a package for Typst, { Babel
} cannot change your source files.

\subsubsection{How to add}\label{how-to-add}

Copy this into your project and use the import as \texttt{\ babel\ }

\begin{verbatim}
#import "@preview/babel:0.1.1"
\end{verbatim}

\includesvg[width=0.16667in,height=0.16667in]{/assets/icons/16-copy.svg}

Check the docs for
\href{https://typst.app/docs/reference/scripting/\#packages}{more
information on how to import packages} .

\subsubsection{About}\label{about}

\begin{description}
\tightlist
\item[Author :]
\href{https://me.digitalwords.net}{Maja Abramski-Kronenberg}
\item[License:]
MIT-0
\item[Current version:]
0.1.1
\item[Last updated:]
October 3, 2024
\item[First released:]
October 3, 2024
\item[Minimum Typst version:]
0.11.0
\item[Archive size:]
46.9 kB
\href{https://packages.typst.org/preview/babel-0.1.1.tar.gz}{\pandocbounded{\includesvg[keepaspectratio]{/assets/icons/16-download.svg}}}
\item[Repository:]
\href{https://codeberg.org/afiaith/babel}{Codeberg}
\item[Categor ies :]
\begin{itemize}
\tightlist
\item[]
\item
  \pandocbounded{\includesvg[keepaspectratio]{/assets/icons/16-world.svg}}
  \href{https://typst.app/universe/search/?category=languages}{Languages}
\item
  \pandocbounded{\includesvg[keepaspectratio]{/assets/icons/16-text.svg}}
  \href{https://typst.app/universe/search/?category=text}{Text}
\item
  \pandocbounded{\includesvg[keepaspectratio]{/assets/icons/16-smile.svg}}
  \href{https://typst.app/universe/search/?category=fun}{Fun}
\end{itemize}
\end{description}

\subsubsection{Where to report issues?}\label{where-to-report-issues}

This package is a project of Maja Abramski-Kronenberg . Report issues on
\href{https://codeberg.org/afiaith/babel}{their repository} . You can
also try to ask for help with this package on the
\href{https://forum.typst.app}{Forum} .

Please report this package to the Typst team using the
\href{https://typst.app/contact}{contact form} if you believe it is a
safety hazard or infringes upon your rights.

\phantomsection\label{versions}
\subsubsection{Version history}\label{version-history}

\begin{longtable}[]{@{}ll@{}}
\toprule\noalign{}
Version & Release Date \\
\midrule\noalign{}
\endhead
\bottomrule\noalign{}
\endlastfoot
0.1.1 & October 3, 2024 \\
\end{longtable}

Typst GmbH did not create this package and cannot guarantee correct
functionality of this package or compatibility with any version of the
Typst compiler or app.


\section{Package List LaTeX/touying-buaa.tex}
\title{typst.app/universe/package/touying-buaa}

\phantomsection\label{banner}
\phantomsection\label{template-thumbnail}
\pandocbounded{\includegraphics[keepaspectratio]{https://packages.typst.org/preview/thumbnails/touying-buaa-0.2.0-small.webp}}

\section{touying-buaa}\label{touying-buaa}

{ 0.2.0 }

Touying Slide Theme for Beihang University

\href{/app?template=touying-buaa&version=0.2.0}{Create project in app}

\phantomsection\label{readme}
Inspired by
\href{https://github.com/QuadnucYard/touying-theme-seu}{Southeast
University Touying Slide Theme} .

\subsection{Use as Typst Template
Package}\label{use-as-typst-template-package}

Use \texttt{\ typst\ init\ @preview/touying-buaa\ } to create a new
project with this theme.

\begin{Shaded}
\begin{Highlighting}[]
\NormalTok{$ typst init @preview/touying{-}buaa}
\NormalTok{Successfully created new project from @preview/touying{-}buaa:}
\NormalTok{To start writing, run:}
\NormalTok{\textgreater{} cd touying{-}buaa}
\NormalTok{\textgreater{} typst watch main.typ}
\end{Highlighting}
\end{Shaded}

\subsection{Examples}\label{examples}

See
\href{https://github.com/typst/packages/raw/main/packages/preview/touying-buaa/0.2.0/examples}{examples}
and \href{https://coekjan.github.io/touying-buaa}{Github Pages} for more
details.

You can compile the examples by yourself.

\begin{Shaded}
\begin{Highlighting}[]
\NormalTok{$ typst compile ./examples/main.typ {-}{-}root .}
\end{Highlighting}
\end{Shaded}

And the PDF file \texttt{\ ./examples/main.pdf\ } will be generated.

\subsection{License}\label{license}

Licensed under the
\href{https://github.com/typst/packages/raw/main/packages/preview/touying-buaa/0.2.0/LICENSE}{MIT
License} .

\href{/app?template=touying-buaa&version=0.2.0}{Create project in app}

\subsubsection{How to use}\label{how-to-use}

Click the button above to create a new project using this template in
the Typst app.

You can also use the Typst CLI to start a new project on your computer
using this command:

\begin{verbatim}
typst init @preview/touying-buaa:0.2.0
\end{verbatim}

\includesvg[width=0.16667in,height=0.16667in]{/assets/icons/16-copy.svg}

\subsubsection{About}\label{about}

\begin{description}
\tightlist
\item[Author :]
\href{mailto:cn_yzr@qq.com}{Yip Coekjan}
\item[License:]
MIT
\item[Current version:]
0.2.0
\item[Last updated:]
September 8, 2024
\item[First released:]
June 4, 2024
\item[Archive size:]
20.5 kB
\href{https://packages.typst.org/preview/touying-buaa-0.2.0.tar.gz}{\pandocbounded{\includesvg[keepaspectratio]{/assets/icons/16-download.svg}}}
\item[Repository:]
\href{https://github.com/Coekjan/touying-buaa}{GitHub}
\item[Categor y :]
\begin{itemize}
\tightlist
\item[]
\item
  \pandocbounded{\includesvg[keepaspectratio]{/assets/icons/16-presentation.svg}}
  \href{https://typst.app/universe/search/?category=presentation}{Presentation}
\end{itemize}
\end{description}

\subsubsection{Where to report issues?}\label{where-to-report-issues}

This template is a project of Yip Coekjan . Report issues on
\href{https://github.com/Coekjan/touying-buaa}{their repository} . You
can also try to ask for help with this template on the
\href{https://forum.typst.app}{Forum} .

Please report this template to the Typst team using the
\href{https://typst.app/contact}{contact form} if you believe it is a
safety hazard or infringes upon your rights.

\phantomsection\label{versions}
\subsubsection{Version history}\label{version-history}

\begin{longtable}[]{@{}ll@{}}
\toprule\noalign{}
Version & Release Date \\
\midrule\noalign{}
\endhead
\bottomrule\noalign{}
\endlastfoot
0.2.0 & September 8, 2024 \\
\href{https://typst.app/universe/package/touying-buaa/0.1.0/}{0.1.0} &
June 4, 2024 \\
\end{longtable}

Typst GmbH did not create this template and cannot guarantee correct
functionality of this template or compatibility with any version of the
Typst compiler or app.


\section{Package List LaTeX/flyingcircus.tex}
\title{typst.app/universe/package/flyingcircus}

\phantomsection\label{banner}
\phantomsection\label{template-thumbnail}
\pandocbounded{\includegraphics[keepaspectratio]{https://packages.typst.org/preview/thumbnails/flyingcircus-3.2.0-small.webp}}

\section{flyingcircus}\label{flyingcircus}

{ 3.2.0 }

For creating homebrew documents with the same fancy style as the Flying
Circus book? Provides simple commands to generate a whole aircraft stat
page, vehicle, or even ship.

\href{/app?template=flyingcircus&version=3.2.0}{Create project in app}

\phantomsection\label{readme}
Version 3.2.0

Do you want your homebrew to have the same fancy style as the Flying
Circus book? Do you want a simple command to generate a whole aircraft
stat page, vehicle, or even ship? I’ll bet you do! Take a look at the
Flying Circus Aircraft Catalog Template.

\subsection{Acknowledgments and Useful
Links}\label{acknowledgments-and-useful-links}

Download the fonts from
\href{https://github.com/Tetragramm/flying-circus-typst-template/archive/refs/heads/Fonts.zip}{HERE}
. Install them on your computer, upload them to the Typst web-app
(anywhere in the project is fine) or use the Typst command line option
-\/-font-path to include them.

Based on the style and work (with the permission of) Erika Chappell for
the \href{https://opensketch.itch.io/flying-circus}{Flying Circus RPG} .

Integrates with the
\href{https://tetragramm.github.io/PlaneBuilder/index.html}{Plane
Builder} . Just click the Catalog JSON button at the bottom to save what
you need for this template.

Same with the
\href{https://tetragramm.github.io/VehicleBuilder/}{Vehicle Builder} .

Or check out the \href{https://discord.gg/HKdyUuvmcb}{Discord server} .

\subsection{Getting Started}\label{getting-started}

These instructions will get you a copy of the project up and running on
the typst web app.

\begin{Shaded}
\begin{Highlighting}[]
\NormalTok{\#import "@preview/flyingcircus:3.2.0": *}

\NormalTok{\#show: FlyingCircus.with(}
\NormalTok{  Title: title,}
\NormalTok{  Author: author,}
\NormalTok{  CoverImg: image("images/cover.png"),}
\NormalTok{  Dedication: [It\textquotesingle{}s Alive!!! MUAHAHAHA!],}
\NormalTok{)}

\NormalTok{\#FCPlane(read("My Plane\_stats.json"), Nickname:"My First Plane")}
\end{Highlighting}
\end{Shaded}

\subsection{Usage}\label{usage}

The first thing is the FlyingCircus style.

\begin{Shaded}
\begin{Highlighting}[]
\NormalTok{\#import "@preview/flyingcircus:3.2.0": *}

\NormalTok{/// Defines the FlyingCircus template}
\NormalTok{///}
\NormalTok{/// {-} Title (str): Title of the document. Goes in metadata and on title page.}
\NormalTok{/// {-} Author (str): Author(s) of the document. Goes in metadata and on title page.}
\NormalTok{/// {-} CoverImg (image): Image to make the first page of the document.}
\NormalTok{/// {-} Description (str): Text to go with the title on the title page.}
\NormalTok{/// {-} Dedication (str): Dedication to go down below the title on the title page.}
\NormalTok{/// {-} body (content)}
\NormalTok{/// {-}\textgreater{} content}

\NormalTok{// Example}
\NormalTok{\#show: FlyingCircus.with(}
\NormalTok{  Title: title,}
\NormalTok{  Author: author,}
\NormalTok{  CoverImg: image("images/cover.png"),}
\NormalTok{  Dedication: [It\textquotesingle{}s Alive!!! MUAHAHAHA!],}
\NormalTok{)}
\end{Highlighting}
\end{Shaded}

Next is the FCPlane function for making plane pages.

\begin{Shaded}
\begin{Highlighting}[]
\NormalTok{/// Defines the FlyingCircus Plane page.  Always on a new page. Image optional.}
\NormalTok{///}
\NormalTok{/// {-} Plane (str | dictionary): JSON string or dictionary representing the plane stats.}
\NormalTok{/// {-} Nickname (str): Nickname to go under the aircraft name.}
\NormalTok{/// {-} Img (image | none): Image to go at the top of the page. Set to none to remove.}
\NormalTok{/// {-} BoxText (dictionary): Pairs of values to go in the box over the image. Does nothing if no Img provided.}
\NormalTok{/// {-} BoxAnchor (str): Which anchor of the image to put the box in?  Sample values are "north", "south{-}west", "center".}
\NormalTok{/// {-} DescriptiveText (content)}
\NormalTok{/// {-}\textgreater{} content}


\NormalTok{// Example}
\NormalTok{\#FCPlane(}
\NormalTok{  read("Basic Biplane\_stats.json"),}
\NormalTok{  Nickname: "Bring home the bacon!",}
\NormalTok{  Img: image("images/Bergziegel\_image.png"),}
\NormalTok{  BoxText: ("Role": "Fast Bomber", "First Flight": "1601", "Strengths": "Fastest Bomber"),}
\NormalTok{  BoxAnchor: "north{-}east",}
\NormalTok{)[}
\NormalTok{\#lorem(100)}
\NormalTok{]}
\end{Highlighting}
\end{Shaded}

The FCVehicleSimple is for when you want to put multiple vehicles on a
page.

\begin{Shaded}
\begin{Highlighting}[]
\NormalTok{/// Defines the FlyingCircus Simple Vehicle.  Not always a full page. Image optional.}
\NormalTok{///}
\NormalTok{/// {-} Vehicle (str | dictionary): JSON string or dictionary representing the Vehicle stats.}
\NormalTok{/// {-} Img (image): Image to go above the vehicle. (optional)}
\NormalTok{/// {-} DescriptiveText (content)}
\NormalTok{/// {-}\textgreater{} content}
\NormalTok{\#FCVehicleSimple(read("Sample Vehicle\_stats.json"))[\#lorem(120)]}
\end{Highlighting}
\end{Shaded}

FCVehicleFancy is a one or two page vehicle that looks nicer but takes
up more space.

\begin{Shaded}
\begin{Highlighting}[]
\NormalTok{/// Defines the FlyingCircus Vehicle page.  Always on a new page. Image optional.}
\NormalTok{/// If the Img is provided, it will take up two facing pages, otherwise only one, but a full page, unlike the Simple.}
\NormalTok{///}
\NormalTok{/// {-} Vehicle (str | dictionary): JSON string or dictionary representing the Vehicle stats.}
\NormalTok{/// {-} Img (image | none): Image to go at the top of the first page. Set to none to remove.}
\NormalTok{/// {-} TextVOffset (length): How far to push the text down the page. Want to do that inset text thing the book does? You can, the text can overlap with thte image.  Does nothing if no Img provided.}
\NormalTok{/// {-} BoxText (dictionary): Pairs of values to go in the box over the image. Does nothing if no Img provided.}
\NormalTok{/// {-} BoxAnchor (str): Which anchor of the image to put the box in?  Sample values are "north", "south{-}west", "center".}
\NormalTok{/// {-} FirstPageContent (content): Goes on the first page. If no image is provided, it is not present.}
\NormalTok{/// {-} AfterContent (content): Goes after the stat block. Always present.}
\NormalTok{/// {-}\textgreater{} content}

\NormalTok{// Example }
\NormalTok{\#FCVehicleFancy(}
\NormalTok{  read("Sample Vehicle\_stats.json"),}
\NormalTok{  Img: image("images/Wandelburg.png"),}
\NormalTok{  TextVOffset: 6.2in,}
\NormalTok{  BoxText: ("Role": "Fast Bomber", "First Flight": "1601", "Strengths": "Fastest Bomber"),}
\NormalTok{  BoxAnchor: "north{-}east",}
\NormalTok{)[}
\NormalTok{\#lorem(100)}
\NormalTok{][}
\NormalTok{\#lorem(100)}
\NormalTok{]}
\end{Highlighting}
\end{Shaded}

Last of the vehicles, FCShip is for boats like Into the Drink.

\begin{Shaded}
\begin{Highlighting}[]
\NormalTok{/// Defines the FlyingCircus Ship page.  Always on a new page. Image optional.}
\NormalTok{///}
\NormalTok{/// {-} Ship (str | dictionary): JSON string or dictionary representing the Ship stats.}
\NormalTok{/// {-} Img (image | none): Image to go at the top of the page. Set to none to remove.}
\NormalTok{/// {-} DescriptiveText (content): Goes below the name and above the stats table.}
\NormalTok{/// {-} notes (content): Goes in the notes section.}
\NormalTok{/// {-}\textgreater{} content}

\NormalTok{// Example: No builder for Ships, so you\textquotesingle{}ll have to put it in your own JSON, or just a dict, like this.}
\NormalTok{\#let ship\_stats = (}
\NormalTok{  Name: "Macchi Frigate",}
\NormalTok{  Speed: 5,}
\NormalTok{  Handling: 15,}
\NormalTok{  Hardness: 9,}
\NormalTok{  Soak: 0,}
\NormalTok{  Strengths: "{-}",}
\NormalTok{  Weaknesses: "{-}",}
\NormalTok{  Weapons: (}
\NormalTok{    (Name: "x2 Light Howitzer", Fore: "x1", Left: "x2", Right: "x2", Rear: "x1"),}
\NormalTok{    (Name: "x6 Pom{-}Pom Gun", Fore: "x2", Left: "x3", Right: "x3", Rear: "x2", Up: "x6"),}
\NormalTok{    (Name: "x2 WMG", Left: "x1", Right: "x1"),}
\NormalTok{  ),}
\NormalTok{  DamageStates: ("", "{-}1 Speed", "{-}3 Guns", "{-}1 Speed", "{-}3 Guns", "Sinking"),}
\NormalTok{)}

\NormalTok{\#FCShip(}
\NormalTok{  Img: image("images/Macchi Frigate.png"),}
\NormalTok{  Ship: ship\_stats,}
\NormalTok{)[}
\NormalTok{  \#lorem(100)}
\NormalTok{][}
\NormalTok{  \#lorem(5)}
\NormalTok{]}
\end{Highlighting}
\end{Shaded}

Additional functions include FCWeapon

\begin{Shaded}
\begin{Highlighting}[]
\NormalTok{/// Defines the FlyingCircus Weapon card. Image optional.}
\NormalTok{///}
\NormalTok{/// {-} Weapon (str | dictionary): JSON string or dictionary representing the Weapon stats.}
\NormalTok{/// {-} Img (image | none): Image to go above the card. Set to none to remove.}
\NormalTok{/// {-} DescriptiveText (content): Goes below the name and above the stats table.}
\NormalTok{/// {-}\textgreater{} content}

\NormalTok{//Example }
\NormalTok{\#FCWeapon(}
\NormalTok{  (Name: "Rifle/Carbine", Cells: (Hits: 1, Damage: 2, AP: 1, Range: "Extreme"), Price: "Scrip", Tags: "Manual"),}
\NormalTok{  Img: image("images/Rifle.png"),}
\NormalTok{)[}
\NormalTok{Note that you can set the text in the cell boxes to whatever you want.}
\NormalTok{]}
\end{Highlighting}
\end{Shaded}

KochFont:

\begin{Shaded}
\begin{Highlighting}[]
\NormalTok{/// Sets the tex to the Koch Fette FC font for people who don\textquotesingle{}t want to remember the font name.}
\NormalTok{///}
\NormalTok{/// {-} body (content)}
\NormalTok{/// {-} ..args: Any valid argument to the text function}
\NormalTok{/// {-}\textgreater{} content}

\NormalTok{// Example }
\NormalTok{\#KochFont(size: 18pt)[Vehicles]}
\end{Highlighting}
\end{Shaded}

and HiddenHeading, which is for adding to the table of contents without
actually putting words on the page.

\begin{Shaded}
\begin{Highlighting}[]
\NormalTok{//If we don\textquotesingle{}t want all our planes at the top level of the table of contents.  EX: if we want}
\NormalTok{// {-} Intro}
\NormalTok{// {-} Story}
\NormalTok{// {-} Planes }
\NormalTok{//   {-} First Plane}
\NormalTok{// We break the page, and create a HiddenHeading, that doesn\textquotesingle{}t show up in the document (Or a normal heading, if that\textquotesingle{}s what you need)}
\NormalTok{//Then we set the heading offset to one so everything after that is indented one level in the table of contents.}
\NormalTok{\#pagebreak()}
\NormalTok{\#HiddenHeading[= Vehicles]}
\NormalTok{\#set heading(offset: 1)}


\NormalTok{New in Version 3.2.0, the FCPlaybook (+utilities), FCNPCShort, and FCAirshipShort}

\NormalTok{//This creates pages like the playbook. Largely customizable, for say, chariots of steel versions.}
\NormalTok{// {-} Name (str) The name of the Playbook}
\NormalTok{// {-} Subhead (str) The text that goes with the name in the header}
\NormalTok{// {-} Character (content) This is the entire left column}
\NormalTok{// {-} Questions (content) This is the top section of the right column, for motivation and trust questions.}
\NormalTok{// {-} Starting (content) Middle section of the right column. Starting Assets, Burdens, Planes, Vices, ect}
\NormalTok{// {-} Stats (content) Bottom section of the right column. Just the four FCPStatTable calls (and a colbreak, probably)}
\NormalTok{// {-} StatNames () Define the stats to draw circles for on the top part of the 2nd page}
\NormalTok{// {-} Triggers (content) List of triggers, includes section, because not all playbooks use the same text there.}
\NormalTok{// {-} Vents (content) List of Vents, customizable like Triggers}
\NormalTok{// {-} Intimacy (content) Bottom section of the left column, for the intimacy move}
\NormalTok{// {-} Moves (content) The entire right column of the second page}
\NormalTok{//}
\NormalTok{// Utilities for use with FCPlaybook}
\NormalTok{// {-} FCPRule()  The full{-}column horizontal line}
\NormalTok{// {-} FCPSection(name: str)[content] The section break}
\NormalTok{//    {-} name (str) The fancy font name on the lft side of the section line, can be blank.}
\NormalTok{//    {-} content The italicized text on the right side of the line, can be blank.}
\NormalTok{// {-} FCPStatTable(name, tagline, stats) For creating Stat tables}
\NormalTok{//    {-} name (str) The name of the profile, to be rendered in smallcaps}
\NormalTok{//    {-} tagline (str) The tagline of the profile, italicized}
\NormalTok{//    {-} stats (dict) A dictionary of stats ex (Hard:"+2") Keys are first row, values are second row, no restrictions otherwise.}
\NormalTok{\#FCPlaybook(}
\NormalTok{  Name: str,}
\NormalTok{  Subhead: str,}
\NormalTok{  Character: content,}
\NormalTok{  Questions: content,}
\NormalTok{  Starting: content,}
\NormalTok{  Stats: content,}
\NormalTok{  StatNames: (),}
\NormalTok{  Triggers: content,}
\NormalTok{  Vents: content,}
\NormalTok{  Intimacy: content,}
\NormalTok{  Moves: content,}
\NormalTok{)}

\NormalTok{// This creates a short NPC profile like that in the back of the aircraft catalogue}
\NormalTok{// {-} plane (dict) Contains the keys }
\NormalTok{//      {-} Name}
\NormalTok{//      {-} Nickname}
\NormalTok{//      {-} Price (optional)}
\NormalTok{//      {-} Upkeep (optional)}
\NormalTok{//      {-} Used (optional)}
\NormalTok{//      {-} Speeds}
\NormalTok{//      {-} Handling}
\NormalTok{//      {-} Structure}
\NormalTok{// {-} img (image) Image to draw}
\NormalTok{// {-} img\_scale (number) What scale to draw the image, relative to the column size}
\NormalTok{// {-} img\_shift\_dx (percent) How far to shift the image in the x direction}
\NormalTok{// {-} img\_shift\_dy (percent) How far to shift the image in the y direction}
\NormalTok{// {-} content The decriptive text to go above the stat block}
\NormalTok{\#FCShortNPC(}
\NormalTok{  plane, }
\NormalTok{  img: none, }
\NormalTok{  img\_scale: 1.5, }
\NormalTok{  img\_shift\_dx: {-}10\%, }
\NormalTok{  img\_shift\_dy: {-}10\%, }
\NormalTok{  content}
\NormalTok{)}


\NormalTok{// This creates a short airship profile like that in the back of the aircraft catalogue}
\NormalTok{// {-} airship (dict) Contains the keys }
\NormalTok{//      {-} Name}
\NormalTok{//      {-} Nickname}
\NormalTok{//      {-} Price (optional)}
\NormalTok{//      {-} Upkeep (optional)}
\NormalTok{//      {-} Used (optional)}
\NormalTok{//      {-} Speed}
\NormalTok{//      {-} Lift}
\NormalTok{//      {-} Handling}
\NormalTok{//      {-} Toughness}
\NormalTok{// {-} img (image) Image to draw}
\NormalTok{// {-} img\_scale (number) What scale to draw the image, relative to the column size}
\NormalTok{// {-} img\_shift\_dx (percent) How far to shift the image in the x direction}
\NormalTok{// {-} img\_shift\_dy (percent) How far to shift the image in the y direction}
\NormalTok{// {-} content The decriptive text to go above the stat block}
\NormalTok{\#FCShortAirship(}
\NormalTok{  airship, }
\NormalTok{  img: none, }
\NormalTok{  img\_scale: 1.5, }
\NormalTok{  img\_shift\_dx: {-}10\%, }
\NormalTok{  img\_shift\_dy: {-}10\%, }
\NormalTok{  content}
\NormalTok{)}
\end{Highlighting}
\end{Shaded}

\href{/app?template=flyingcircus&version=3.2.0}{Create project in app}

\subsubsection{How to use}\label{how-to-use}

Click the button above to create a new project using this template in
the Typst app.

You can also use the Typst CLI to start a new project on your computer
using this command:

\begin{verbatim}
typst init @preview/flyingcircus:3.2.0
\end{verbatim}

\includesvg[width=0.16667in,height=0.16667in]{/assets/icons/16-copy.svg}

\subsubsection{About}\label{about}

\begin{description}
\tightlist
\item[Author :]
Tetragramm
\item[License:]
MIT
\item[Current version:]
3.2.0
\item[Last updated:]
October 25, 2024
\item[First released:]
August 23, 2024
\item[Minimum Typst version:]
0.12.0
\item[Archive size:]
476 kB
\href{https://packages.typst.org/preview/flyingcircus-3.2.0.tar.gz}{\pandocbounded{\includesvg[keepaspectratio]{/assets/icons/16-download.svg}}}
\item[Repository:]
\href{https://github.com/Tetragramm/flying-circus-typst-template}{GitHub}
\item[Discipline s :]
\begin{itemize}
\tightlist
\item[]
\item
  \href{https://typst.app/universe/search/?discipline=engineering}{Engineering}
\item
  \href{https://typst.app/universe/search/?discipline=design}{Design}
\item
  \href{https://typst.app/universe/search/?discipline=transportation}{Transportation}
\end{itemize}
\item[Categor ies :]
\begin{itemize}
\tightlist
\item[]
\item
  \pandocbounded{\includesvg[keepaspectratio]{/assets/icons/16-docs.svg}}
  \href{https://typst.app/universe/search/?category=book}{Book}
\item
  \pandocbounded{\includesvg[keepaspectratio]{/assets/icons/16-smile.svg}}
  \href{https://typst.app/universe/search/?category=fun}{Fun}
\item
  \pandocbounded{\includesvg[keepaspectratio]{/assets/icons/16-layout.svg}}
  \href{https://typst.app/universe/search/?category=layout}{Layout}
\end{itemize}
\end{description}

\subsubsection{Where to report issues?}\label{where-to-report-issues}

This template is a project of Tetragramm . Report issues on
\href{https://github.com/Tetragramm/flying-circus-typst-template}{their
repository} . You can also try to ask for help with this template on the
\href{https://forum.typst.app}{Forum} .

Please report this template to the Typst team using the
\href{https://typst.app/contact}{contact form} if you believe it is a
safety hazard or infringes upon your rights.

\phantomsection\label{versions}
\subsubsection{Version history}\label{version-history}

\begin{longtable}[]{@{}ll@{}}
\toprule\noalign{}
Version & Release Date \\
\midrule\noalign{}
\endhead
\bottomrule\noalign{}
\endlastfoot
3.2.0 & October 25, 2024 \\
\href{https://typst.app/universe/package/flyingcircus/3.0.0/}{3.0.0} &
August 23, 2024 \\
\end{longtable}

Typst GmbH did not create this template and cannot guarantee correct
functionality of this template or compatibility with any version of the
Typst compiler or app.


\section{Package List LaTeX/ttuile.tex}
\title{typst.app/universe/package/ttuile}

\phantomsection\label{banner}
\phantomsection\label{template-thumbnail}
\pandocbounded{\includegraphics[keepaspectratio]{https://packages.typst.org/preview/thumbnails/ttuile-0.1.1-small.webp}}

\section{ttuile}\label{ttuile}

{ 0.1.1 }

A template for students\textquotesingle{} lab reports at INSA Lyon, a
french engineering school.

\href{/app?template=ttuile&version=0.1.1}{Create project in app}

\phantomsection\label{readme}
\href{https://typst.app/}{\pandocbounded{\includegraphics[keepaspectratio]{https://img.shields.io/badge/Typst-\%232f90ba.svg?&logo=Typst&logoColor=white}}}
\href{https://github.com/vitto4/ttuile/blob/main/LICENSE}{\pandocbounded{\includegraphics[keepaspectratio]{https://img.shields.io/github/license/vitto4/ttuile}}}
\href{https://github.com/vitto4/ttuile/releases}{\pandocbounded{\includegraphics[keepaspectratio]{https://img.shields.io/github/v/release/vitto4/ttuile}}}

\emph{A \textbf{Typst} template for lab reports at
\href{https://en.wikipedia.org/wiki/Institut_national_des_sciences_appliqu\%C3\%A9es_de_Lyon}{INSA
Lyon} .}

\href{https://github.com/vitto4/ttuile/blob/main/template/main.pdf}{\pandocbounded{\includegraphics[keepaspectratio]{https://raw.githubusercontent.com/vitto4/ttuile/main/assets/ttuile-banner.png?raw=true}}}

\begin{quote}
\textbf{Note :} Voir aussi le
\href{https://github.com/vitto4/ttuile/blob/main/README.FR.md}{README.FR.md}
en français.
\end{quote}

\subsection{🧭 Table of contents}\label{uxf0uxff-table-of-contents}

\begin{enumerate}
\tightlist
\item
  \href{https://github.com/typst/packages/raw/main/packages/preview/ttuile/0.1.1/\#-usage}{Usage}
\item
  \href{https://github.com/typst/packages/raw/main/packages/preview/ttuile/0.1.1/\#-documentation}{Documentation}
\item
  \href{https://github.com/typst/packages/raw/main/packages/preview/ttuile/0.1.1/\#-notes}{Notes}
\item
  \href{https://github.com/typst/packages/raw/main/packages/preview/ttuile/0.1.1/\#-contributing}{Contributing}
\end{enumerate}

\subsection{ðŸ``Ž Usage}\label{uxf0uxffux17e-usage}

This template targets french students, thus labels will be in french,
see
\href{https://github.com/typst/packages/raw/main/packages/preview/ttuile/0.1.1/\#-notes}{Notes}
.

It is available on \emph{Typst Universe} :
\href{https://typst.app/universe/package/ttuile}{\texttt{\ @preview/ttuile:0.1.1\ }}
.

If you wish to use it in a fully local manner, you’ll need to either
manually include \texttt{\ ttuile.typ\ } and
\texttt{\ logo-insa-lyon.png\ } in your project’s root directory ; or
upload them to the \emph{Typst web app} if that’s what you use.

You’ll find these files in the
\href{https://github.com/vitto4/ttuile/releases}{releases} section.

Your folder structure should then look something like this :

\begin{verbatim}
.
├── ttuile.typ
├── logo-insa-lyon.png
└── main.typ
\end{verbatim}

The template is now ready to be used, and can be called supplying the
following arguments. \texttt{\ ?\ } means the argument can be null if
not applicable.

\begin{longtable}[]{@{}cccl@{}}
\toprule\noalign{}
Argument & Default value & Type & Description \\
\midrule\noalign{}
\endhead
\bottomrule\noalign{}
\endlastfoot
\texttt{\ titre\ } & \texttt{\ none\ } & \texttt{\ content?\ } & The
title of your report. \\
\texttt{\ auteurs\ } & \texttt{\ none\ } &
\texttt{\ array\textless{}str\textgreater{}\ \textbar{}\ content?\ } &
One or multiple authors to be credited in the report. \\
\texttt{\ groupe\ } & \texttt{\ none\ } & \texttt{\ content?\ } & Your
class number/letter/identifier. Will be displayed right after the
author(s). \\
\texttt{\ numero-tp\ } & \texttt{\ none\ } & \texttt{\ content?\ } & The
number/identifier of the lab work/practical you’re writing this report
for. \\
\texttt{\ numero-poste\ } & \texttt{\ none\ } & \texttt{\ content?\ } &
Number of your lab bench. \\
\texttt{\ date\ } & \texttt{\ none\ } &
\texttt{\ datetime\ \textbar{}\ content?\ } & Date at which the lab
work/practical was carried out. \\
\texttt{\ sommaire\ } & \texttt{\ true\ } & \texttt{\ bool\ } & Display
the table of contents ? \\
\texttt{\ logo\ } & \texttt{\ image("logo-insa-lyon.png")\ } &
\texttt{\ image?\ } & University logo to use. \\
\texttt{\ point-legende\ } & \texttt{\ false\ } & \texttt{\ bool\ } &
Enable automatic enforcement of full stops at the end of figures’
captions. (still somewhat experimental). \\
\end{longtable}

A single positional argument is accepted, being the report’s body.

You can call the template using the following syntax :

\begin{Shaded}
\begin{Highlighting}[]
\NormalTok{// Local import}
\NormalTok{// \#import "ttuile.typ": *}

\NormalTok{// Universe import}
\NormalTok{\#import "@preview/ttuile:0.1.1": *}

\NormalTok{\#show: ttuile.with(}
\NormalTok{  titre: [« \#lorem(8) »],}
\NormalTok{  auteurs: (}
\NormalTok{      "Theresa Tungsten",}
\NormalTok{      "Jean Dupont",}
\NormalTok{      "Eugene Deklan",}
\NormalTok{  ),}
\NormalTok{  groupe: "TD0",}
\NormalTok{  numero{-}tp: 0,}
\NormalTok{  numero{-}poste: "0",}
\NormalTok{  date: datetime.today(),}
\NormalTok{  // sommaire: false,}
\NormalTok{  // logo: image("path\_to/logo.png"),}
\NormalTok{  // point{-}legende: true,}
\NormalTok{)}
\end{Highlighting}
\end{Shaded}

\subsection{ðŸ``š Documentation}\label{uxf0uxffux161-documentation}

The package \texttt{\ ttuile.typ\ } exposes multiple functions, find out
more about them in the \emph{documentation} .

\href{https://github.com/vitto4/ttuile/blob/main/DOC.EN.md}{To the
documentation}

An example file is also available in
\href{https://github.com/vitto4/ttuile/blob/main/template/main.typ}{\texttt{\ template/main.typ\ }}

\subsection{ðŸ''-- Notes}\label{uxf0uxff-notes}

\begin{itemize}
\item
  Beware, all of the labels will be in french (authors != auteurs,
  appendix != annexe, …)
\item
  If you really want to use this template despite not being an INSA
  student, you can probably figure out what to change in the code
  (namely labels mentioned above). You can remove the INSA logo by
  setting \texttt{\ logo:\ none\ }

  Should you still need help, no worries, feel free to reach out !
\item
  The code - variable names and comments - is all in french. That’s on
  me, I didn’t really think it through when first writing the template
  haha. I might consider translating sometime in the future.
\item
  The MIT license doesn’t apply to the file
  \texttt{\ logo-insa-lyon.png\ } , it was retrieved from
  \href{https://www.insa-lyon.fr/fr/elements-graphiques}{INSA Lyon -
  éléments graphiques} . It doesn’t apply either to the “INSA�
  branding.
\end{itemize}

\subsection{🧩 Contributing}\label{uxf0uxff-contributing}

Contributions are welcome ! Parts of the template are very much
spaghetti code, especially where the spacing between different headings
is handled (seriously, it’s pretty bad).

If you know the proper way of doing this, an issue or PR would be
greatly appreciated :)

\href{/app?template=ttuile&version=0.1.1}{Create project in app}

\subsubsection{How to use}\label{how-to-use}

Click the button above to create a new project using this template in
the Typst app.

You can also use the Typst CLI to start a new project on your computer
using this command:

\begin{verbatim}
typst init @preview/ttuile:0.1.1
\end{verbatim}

\includesvg[width=0.16667in,height=0.16667in]{/assets/icons/16-copy.svg}

\subsubsection{About}\label{about}

\begin{description}
\tightlist
\item[Author :]
\href{https://github.com/vitto4}{vitto}
\item[License:]
MIT
\item[Current version:]
0.1.1
\item[Last updated:]
May 6, 2024
\item[First released:]
May 3, 2024
\item[Archive size:]
46.8 kB
\href{https://packages.typst.org/preview/ttuile-0.1.1.tar.gz}{\pandocbounded{\includesvg[keepaspectratio]{/assets/icons/16-download.svg}}}
\item[Repository:]
\href{https://github.com/vitto4/ttuile}{GitHub}
\item[Discipline :]
\begin{itemize}
\tightlist
\item[]
\item
  \href{https://typst.app/universe/search/?discipline=engineering}{Engineering}
\end{itemize}
\item[Categor y :]
\begin{itemize}
\tightlist
\item[]
\item
  \pandocbounded{\includesvg[keepaspectratio]{/assets/icons/16-speak.svg}}
  \href{https://typst.app/universe/search/?category=report}{Report}
\end{itemize}
\end{description}

\subsubsection{Where to report issues?}\label{where-to-report-issues}

This template is a project of vitto . Report issues on
\href{https://github.com/vitto4/ttuile}{their repository} . You can also
try to ask for help with this template on the
\href{https://forum.typst.app}{Forum} .

Please report this template to the Typst team using the
\href{https://typst.app/contact}{contact form} if you believe it is a
safety hazard or infringes upon your rights.

\phantomsection\label{versions}
\subsubsection{Version history}\label{version-history}

\begin{longtable}[]{@{}ll@{}}
\toprule\noalign{}
Version & Release Date \\
\midrule\noalign{}
\endhead
\bottomrule\noalign{}
\endlastfoot
0.1.1 & May 6, 2024 \\
\href{https://typst.app/universe/package/ttuile/0.1.0/}{0.1.0} & May 3,
2024 \\
\end{longtable}

Typst GmbH did not create this template and cannot guarantee correct
functionality of this template or compatibility with any version of the
Typst compiler or app.


\section{Package List LaTeX/clear-iclr.tex}
\title{typst.app/universe/package/clear-iclr}

\phantomsection\label{banner}
\phantomsection\label{template-thumbnail}
\pandocbounded{\includegraphics[keepaspectratio]{https://packages.typst.org/preview/thumbnails/clear-iclr-0.4.0-small.webp}}

\section{clear-iclr}\label{clear-iclr}

{ 0.4.0 }

Paper template for submission to International Conference on Learning
Representations (ICLR)

{ } Featured Template

\href{/app?template=clear-iclr&version=0.4.0}{Create project in app}

\phantomsection\label{readme}
\subsection{Usage}\label{usage}

You can use this template in the Typst web app by clicking \emph{Start
from template} on the dashboard and searching for
\texttt{\ clear-iclr\ } .

Alternatively, you can use the CLI to kick this project off using the
command

\begin{Shaded}
\begin{Highlighting}[]
\NormalTok{typst init @preview/clear{-}iclr}
\end{Highlighting}
\end{Shaded}

Typst will create a new directory with all the files needed to get you
started.

\subsection{Configuration}\label{configuration}

This template exports the \texttt{\ iclr\ } function with the following
named arguments.

\begin{itemize}
\item
  \texttt{\ title\ } : The paper’s title as content.
\item
  \texttt{\ authors\ } : An array of author dictionaries. Each of the
  author dictionaries must have a name key and can have the keys
  department, organization, location, and email.

\begin{Shaded}
\begin{Highlighting}[]
\NormalTok{\#let authors = (}
\NormalTok{  ...,}
\NormalTok{  (}
\NormalTok{    names: ([Coauthor1], [Coauthor2]),}
\NormalTok{    affilation: [Affiliation],}
\NormalTok{    address: [Address],}
\NormalTok{    email: "correspondent@example.org",}
\NormalTok{  ),}
\NormalTok{  ...}
\NormalTok{)}
\end{Highlighting}
\end{Shaded}
\item
  \texttt{\ keywords\ } : Publication keywords (used in PDF metadata).
\item
  \texttt{\ date\ } : Creation date (used in PDF metadata).
\item
  \texttt{\ abstract\ } : The content of a brief summary of the paper or
  none. Appears at the top under the title.
\item
  \texttt{\ bibliography\ } : The result of a call to the bibliography
  function or none. The function also accepts a single, positional
  argument for the body of the paper.
\item
  \texttt{\ appendix\ } : Content to append after bibliography section
  (can be included).
\item
  \texttt{\ accepted\ } : If this is set to \texttt{\ false\ } then
  anonymized ready for submission document is produced;
  \texttt{\ accepted:\ true\ } produces camera-redy version. If the
  argument is set to \texttt{\ none\ } then preprint version is produced
  (can be uploaded to arXiv).
\end{itemize}

The template will initialize your package with a sample call to the
\texttt{\ iclr\ } function in a show rule. If you want to change an
existing project to use this template, you can add a show rule at the
top of your file.

\subsection{Issues}\label{issues}

This template is developed at
\href{https://github.com/daskol/typst-templates}{daskol/typst-templates}
repo. Please report all issues there.

\begin{itemize}
\item
  Common issue is related to Typst’s inablity to produce colored
  annotation. In order to mitigte the issue, we add a script which
  modifies annotations and make them colored.

\begin{Shaded}
\begin{Highlighting}[]
\NormalTok{../colorize{-}annotations.py \textbackslash{}}
\NormalTok{    example{-}paper.typst.pdf example{-}paper{-}colored.typst.pdf}
\end{Highlighting}
\end{Shaded}

  See {[} \href{http://readme.md/}{README.md} {]}{[}3{]} for details.
\item
  The author instructions says that preferable font is MS Times New
  Roman but the official example paper uses serifs like Computer Modern
  and Nimbus font families. Monospace fonts are not specified.
\item
  ICML-like bibliography style. The bibliography slightly differs from
  the one in the original example paper. The main difference is that we
  prefer to use author’s lastname at first place to search an entry
  faster.
\end{itemize}

\href{/app?template=clear-iclr&version=0.4.0}{Create project in app}

\subsubsection{How to use}\label{how-to-use}

Click the button above to create a new project using this template in
the Typst app.

You can also use the Typst CLI to start a new project on your computer
using this command:

\begin{verbatim}
typst init @preview/clear-iclr:0.4.0
\end{verbatim}

\includesvg[width=0.16667in,height=0.16667in]{/assets/icons/16-copy.svg}

\subsubsection{About}\label{about}

\begin{description}
\tightlist
\item[Author :]
\href{mailto:d.bershatsky2@skoltech.ru}{Daniel Bershatsky}
\item[License:]
MIT
\item[Current version:]
0.4.0
\item[Last updated:]
September 11, 2024
\item[First released:]
September 11, 2024
\item[Minimum Typst version:]
0.11.1
\item[Archive size:]
21.4 kB
\href{https://packages.typst.org/preview/clear-iclr-0.4.0.tar.gz}{\pandocbounded{\includesvg[keepaspectratio]{/assets/icons/16-download.svg}}}
\item[Repository:]
\href{https://github.com/daskol/typst-templates}{GitHub}
\item[Discipline s :]
\begin{itemize}
\tightlist
\item[]
\item
  \href{https://typst.app/universe/search/?discipline=computer-science}{Computer
  Science}
\item
  \href{https://typst.app/universe/search/?discipline=mathematics}{Mathematics}
\end{itemize}
\item[Categor y :]
\begin{itemize}
\tightlist
\item[]
\item
  \pandocbounded{\includesvg[keepaspectratio]{/assets/icons/16-atom.svg}}
  \href{https://typst.app/universe/search/?category=paper}{Paper}
\end{itemize}
\end{description}

\subsubsection{Where to report issues?}\label{where-to-report-issues}

This template is a project of Daniel Bershatsky . Report issues on
\href{https://github.com/daskol/typst-templates}{their repository} . You
can also try to ask for help with this template on the
\href{https://forum.typst.app}{Forum} .

Please report this template to the Typst team using the
\href{https://typst.app/contact}{contact form} if you believe it is a
safety hazard or infringes upon your rights.

\phantomsection\label{versions}
\subsubsection{Version history}\label{version-history}

\begin{longtable}[]{@{}ll@{}}
\toprule\noalign{}
Version & Release Date \\
\midrule\noalign{}
\endhead
\bottomrule\noalign{}
\endlastfoot
0.4.0 & September 11, 2024 \\
\end{longtable}

Typst GmbH did not create this template and cannot guarantee correct
functionality of this template or compatibility with any version of the
Typst compiler or app.


\section{Package List LaTeX/typslides.tex}
\title{typst.app/universe/package/typslides}

\phantomsection\label{banner}
\section{typslides}\label{typslides}

{ 1.2.1 }

Minimalistic Typst slides

\phantomsection\label{readme}
\includegraphics[width=4.16667in,height=\textheight,keepaspectratio]{https://github.com/typst/packages/raw/main/packages/preview/typslides/1.2.1/img/logo.png}

\pandocbounded{\includegraphics[keepaspectratio]{https://img.shields.io/badge/license-GPLv3-blue}}
\href{https://github.com/typst/packages/raw/main/packages/preview/typslides/1.2.1/}{\pandocbounded{\includegraphics[keepaspectratio]{https://badgen.net/github/contributors/manjavacas/typslides}}}
\href{https://github.com/typst/packages/raw/main/packages/preview/typslides/1.2.1/}{\pandocbounded{\includegraphics[keepaspectratio]{https://badgen.net/github/release/manjavacas/typslides}}}
\pandocbounded{\includegraphics[keepaspectratio]{https://img.shields.io/github/stars/manjavacas/typslides}}

\emph{Minimalistic \href{https://typst.app/}{typst} slides!}

This is a simple usage example:

\begin{Shaded}
\begin{Highlighting}[]
\NormalTok{\#import "@preview/typslides:1.2.1": *}

\NormalTok{// Project configuration}
\NormalTok{\#show: typslides.with(}
\NormalTok{  ratio: "16{-}9",}
\NormalTok{  theme: "bluey",}
\NormalTok{)}

\NormalTok{// The front slide is the first slide of your presentation}
\NormalTok{\#front{-}slide(}
\NormalTok{  title: "This is a sample presentation",}
\NormalTok{  subtitle: [Using \_typslides\_],}
\NormalTok{  authors: "Antonio Manjavacas",}
\NormalTok{  info: [\#link("https://github.com/manjavacas/typslides")],}
\NormalTok{)}

\NormalTok{// Custom outline}
\NormalTok{\#table{-}of{-}contents()}

\NormalTok{// Title slides create new sections}
\NormalTok{\#title{-}slide[}
\NormalTok{  This is a \_Title slide\_}
\NormalTok{]}

\NormalTok{// A simple slide}
\NormalTok{\#slide[}
\NormalTok{  {-} This is a simple \textasciigrave{}slide\textasciigrave{} with no title.}
\NormalTok{  {-} \#stress("Bold and coloured") text by using \textasciigrave{}\#stress(text)\textasciigrave{}.}
\NormalTok{  {-} Sample link: \#link("typst.app")}
\NormalTok{  {-} Sample references: @typst, @typslides.}

\NormalTok{  \#framed[This text has been written using \textasciigrave{}\#framed(text)\textasciigrave{}. The background color of the box is customisable.]}

\NormalTok{  \#framed(title: "Frame with title")[This text has been written using \textasciigrave{}\#framed(title:"Frame with title")[text]\textasciigrave{}.]}
\NormalTok{]}

\NormalTok{// Focus slide}
\NormalTok{\#focus{-}slide[}
\NormalTok{  This is an auto{-}resized \_focus slide\_.}
\NormalTok{]}

\NormalTok{// Blank slide}
\NormalTok{\#blank{-}slide[}
\NormalTok{  {-} This is a \textasciigrave{}\#blank{-}slide\textasciigrave{}.}

\NormalTok{  {-} Available \#stress[themes]:}

\NormalTok{  \#text(fill: rgb("3059AB"), weight: "bold")[bluey]}
\NormalTok{  \#text(fill: rgb("BF3D3D"), weight: "bold")[greeny]}
\NormalTok{  \#text(fill: rgb("28842F"), weight: "bold")[reddy]}
\NormalTok{  \#text(fill: rgb("C4853D"), weight: "bold")[yelly]}
\NormalTok{  \#text(fill: rgb("862A70"), weight: "bold")[purply]}
\NormalTok{  \#text(fill: rgb("1F4289"), weight: "bold")[dusky]}
\NormalTok{  \#text(fill: black, weight: "bold")[darky]}
\NormalTok{]}

\NormalTok{// Slide with title}
\NormalTok{\#slide(title: "This is the slide title")[}
\NormalTok{  \#lorem(20)}
\NormalTok{  \#grayed([This is a \textasciigrave{}\#grayed\textasciigrave{} text. Useful for equations.])}
\NormalTok{  \#grayed($ P\_t = alpha {-} 1 / (sqrt(x) + f(y)) $)}
\NormalTok{  \#lorem(20)}
\NormalTok{]}

\NormalTok{// Bibliography}
\NormalTok{\#bibliography{-}slide("bibliography.bib")}
\end{Highlighting}
\end{Shaded}

{
\includesvg[width=3.125in,height=\textheight,keepaspectratio]{https://github.com/typst/packages/raw/main/packages/preview/typslides/1.2.1/img/slide-1.svg}
} {
\includesvg[width=3.125in,height=\textheight,keepaspectratio]{https://github.com/typst/packages/raw/main/packages/preview/typslides/1.2.1/img/slide-2.svg}
} {
\includesvg[width=3.125in,height=\textheight,keepaspectratio]{https://github.com/typst/packages/raw/main/packages/preview/typslides/1.2.1/img/slide-3.svg}
} {
\includesvg[width=3.125in,height=\textheight,keepaspectratio]{https://github.com/typst/packages/raw/main/packages/preview/typslides/1.2.1/img/slide-4.svg}
} {
\includesvg[width=3.125in,height=\textheight,keepaspectratio]{https://github.com/typst/packages/raw/main/packages/preview/typslides/1.2.1/img/slide-5.svg}
} {
\includesvg[width=3.125in,height=\textheight,keepaspectratio]{https://github.com/typst/packages/raw/main/packages/preview/typslides/1.2.1/img/slide-6.svg}
} {
\includesvg[width=3.125in,height=\textheight,keepaspectratio]{https://github.com/typst/packages/raw/main/packages/preview/typslides/1.2.1/img/slide-7.svg}
} {
\includesvg[width=3.125in,height=\textheight,keepaspectratio]{https://github.com/typst/packages/raw/main/packages/preview/typslides/1.2.1/img/slide-8.svg}
}

\subsubsection{How to add}\label{how-to-add}

Copy this into your project and use the import as \texttt{\ typslides\ }

\begin{verbatim}
#import "@preview/typslides:1.2.1"
\end{verbatim}

\includesvg[width=0.16667in,height=0.16667in]{/assets/icons/16-copy.svg}

Check the docs for
\href{https://typst.app/docs/reference/scripting/\#packages}{more
information on how to import packages} .

\subsubsection{About}\label{about}

\begin{description}
\tightlist
\item[Author :]
\href{https://github.com/manjavacas}{Antonio Manjavacas}
\item[License:]
GPL-3.0
\item[Current version:]
1.2.1
\item[Last updated:]
November 22, 2024
\item[First released:]
October 29, 2024
\item[Minimum Typst version:]
0.12.0
\item[Archive size:]
15.8 kB
\href{https://packages.typst.org/preview/typslides-1.2.1.tar.gz}{\pandocbounded{\includesvg[keepaspectratio]{/assets/icons/16-download.svg}}}
\item[Repository:]
\href{https://github.com/manjavacas/typslides}{GitHub}
\item[Categor ies :]
\begin{itemize}
\tightlist
\item[]
\item
  \pandocbounded{\includesvg[keepaspectratio]{/assets/icons/16-presentation.svg}}
  \href{https://typst.app/universe/search/?category=presentation}{Presentation}
\item
  \pandocbounded{\includesvg[keepaspectratio]{/assets/icons/16-layout.svg}}
  \href{https://typst.app/universe/search/?category=layout}{Layout}
\end{itemize}
\end{description}

\subsubsection{Where to report issues?}\label{where-to-report-issues}

This package is a project of Antonio Manjavacas . Report issues on
\href{https://github.com/manjavacas/typslides}{their repository} . You
can also try to ask for help with this package on the
\href{https://forum.typst.app}{Forum} .

Please report this package to the Typst team using the
\href{https://typst.app/contact}{contact form} if you believe it is a
safety hazard or infringes upon your rights.

\phantomsection\label{versions}
\subsubsection{Version history}\label{version-history}

\begin{longtable}[]{@{}ll@{}}
\toprule\noalign{}
Version & Release Date \\
\midrule\noalign{}
\endhead
\bottomrule\noalign{}
\endlastfoot
1.2.1 & November 22, 2024 \\
\href{https://typst.app/universe/package/typslides/1.2.0/}{1.2.0} &
November 12, 2024 \\
\href{https://typst.app/universe/package/typslides/1.1.1/}{1.1.1} &
October 29, 2024 \\
\end{longtable}

Typst GmbH did not create this package and cannot guarantee correct
functionality of this package or compatibility with any version of the
Typst compiler or app.


\section{Package List LaTeX/big-rati.tex}
\title{typst.app/universe/package/big-rati}

\phantomsection\label{banner}
\section{big-rati}\label{big-rati}

{ 0.1.0 }

Utilities to work with big rational numbers in Typst

\phantomsection\label{readme}
\texttt{\ big-rati\ } is a package to work with rational numbers in
Typst

\subsection{Usage}\label{usage}

\begin{Shaded}
\begin{Highlighting}[]
\NormalTok{\#import "@preview/big{-}rati:0.1.0"}

\NormalTok{\#let a = 2      // 2/1}
\NormalTok{\#let b = (1, 2) // 1/2}

\NormalTok{\#let sum = big{-}rati.add(a, b) // 5/2}

\NormalTok{\#let c = ("4", 6)}
\NormalTok{\#let prod = big{-}rati.mul(c, sum) // 5/3}

\NormalTok{$\#big{-}rati.repr(prod)$}
\end{Highlighting}
\end{Shaded}

Functions, exported by the package are:

\begin{Shaded}
\begin{Highlighting}[]
\NormalTok{// Converts \textasciigrave{}x\textasciigrave{} to bytes, representing the rational number,}
\NormalTok{// that can be used in the functions below.}
\NormalTok{// \textasciigrave{}x\textasciigrave{} might be an integer or a big integer string.}
\NormalTok{// If \textasciigrave{}x\textasciigrave{} is an array of length two, which elements are integers}
\NormalTok{// or big integer strings, then it is converted to the array of all}
\NormalTok{// big integer strings, and then into the bytes representation.}
\NormalTok{\#let rational(x)}

\NormalTok{// Functions below work with "rational numbers", integers or big integer strings}

\NormalTok{// Returns \textasciigrave{}a + b\textasciigrave{}}
\NormalTok{\#let add(a, b)}

\NormalTok{// Returns \textasciigrave{}a {-} b\textasciigrave{}}
\NormalTok{\#let sub(a, b)}

\NormalTok{// Returns \textasciigrave{}a / b\textasciigrave{}}
\NormalTok{\#let div(a, b)}

\NormalTok{// Returns \textasciigrave{}a * b\textasciigrave{}}
\NormalTok{\#let mul(a, b)}

\NormalTok{// Returns \textasciigrave{}a \% b\textasciigrave{}}
\NormalTok{\#let rem(a, b)}

\NormalTok{// Returns \textasciigrave{}|a {-} b|\textasciigrave{}}
\NormalTok{\#let abs{-}diff(a, b)}

\NormalTok{// Returns \textasciigrave{}{-}1\textasciigrave{} if \textasciigrave{}a \textless{} b\textasciigrave{}, \textasciigrave{}0\textasciigrave{} if \textasciigrave{}a == b\textasciigrave{}, \textasciigrave{}1\textasciigrave{} if \textasciigrave{}a \textgreater{} b\textasciigrave{}}
\NormalTok{\#let cmp(a, b)}

\NormalTok{// Returns \textasciigrave{}{-}x\textasciigrave{}}
\NormalTok{\#let neg(x)}

\NormalTok{// Returns \textasciigrave{}|x|\textasciigrave{}}
\NormalTok{\#let abs(x)}

\NormalTok{// Rounds towards plus infinity}
\NormalTok{\#let ceil(x)}

\NormalTok{// Rounds towards minus infinity}
\NormalTok{\#let floor(x)}

\NormalTok{// Rounds to the nearest integer. Rounds half{-}way cases away from zero.}
\NormalTok{\#let round(x)}

\NormalTok{// Rounds towards zero.}
\NormalTok{\#let trunc(x)}

\NormalTok{// Returns the fractional part of a number, with division rounded towards zero.}
\NormalTok{// Satisfies \textasciigrave{}number == add(trunc(number), fract(number))\textasciigrave{}.}
\NormalTok{\#let fract(number)}

\NormalTok{// Returns the reciprocal.}
\NormalTok{// Panics if the number is zero.}
\NormalTok{\#let recip(x)}

\NormalTok{// Returns \textasciigrave{}x\^{}y\textasciigrave{}. \textasciigrave{}y\textasciigrave{} must be an \textasciigrave{}int\textasciigrave{}, in range of \textasciigrave{}{-}2\^{}32\textasciigrave{} to \textasciigrave{}2\^{}32 {-} 1\textasciigrave{}}
\NormalTok{\#let pow(x, y)}

\NormalTok{// Restrict a value to a certain interval.}
\NormalTok{//}
\NormalTok{// Returns \textasciigrave{}max\textasciigrave{} if \textasciigrave{}number\textasciigrave{} is greater than \textasciigrave{}max\textasciigrave{},}
\NormalTok{// and \textasciigrave{}min\textasciigrave{} if \textasciigrave{}number\textasciigrave{} is less than \textasciigrave{}min\textasciigrave{}.}
\NormalTok{// Otherwise returns \textasciigrave{}number\textasciigrave{}.}
\NormalTok{//}
\NormalTok{// Returns error if \textasciigrave{}min\textasciigrave{} is greater than \textasciigrave{}max\textasciigrave{}.}
\NormalTok{\#let clamp(number, min, max)}

\NormalTok{// Compares and returns the minimum of two values.}
\NormalTok{\#let min(a, b)}

\NormalTok{// Compares and returns the maximum of two values.}
\NormalTok{\#let max(a, b)}

\NormalTok{// Returns a value of type \textasciigrave{}content\textasciigrave{}, representing the rational number.}
\NormalTok{// If \textasciigrave{}is{-}mixed\textasciigrave{} is true, then the result is a mixed fraction,}
\NormalTok{// otherwise, it is a simple fraction.}
\NormalTok{\#let repr(x, is{-}mixed: true)}

\NormalTok{// Returns a string, representing the rational number}
\NormalTok{\#let to{-}decimal{-}str(x, precision: 8)}

\NormalTok{// Returns a floating{-}point number, representing the rational number}
\NormalTok{\#let to{-}float(x, precision: 8)}

\NormalTok{// Returns a decimal number, representing the rational number}
\NormalTok{\#let to{-}decimal(x, precision: 8)}
\end{Highlighting}
\end{Shaded}

\subsubsection{How to add}\label{how-to-add}

Copy this into your project and use the import as \texttt{\ big-rati\ }

\begin{verbatim}
#import "@preview/big-rati:0.1.0"
\end{verbatim}

\includesvg[width=0.16667in,height=0.16667in]{/assets/icons/16-copy.svg}

Check the docs for
\href{https://typst.app/docs/reference/scripting/\#packages}{more
information on how to import packages} .

\subsubsection{About}\label{about}

\begin{description}
\tightlist
\item[Author :]
Danik Vitek
\item[License:]
MIT
\item[Current version:]
0.1.0
\item[Last updated:]
October 29, 2024
\item[First released:]
October 29, 2024
\item[Archive size:]
33.9 kB
\href{https://packages.typst.org/preview/big-rati-0.1.0.tar.gz}{\pandocbounded{\includesvg[keepaspectratio]{/assets/icons/16-download.svg}}}
\item[Repository:]
\href{https://github.com/DanikVitek/typst-plugin-bigrational}{GitHub}
\item[Discipline :]
\begin{itemize}
\tightlist
\item[]
\item
  \href{https://typst.app/universe/search/?discipline=mathematics}{Mathematics}
\end{itemize}
\item[Categor y :]
\begin{itemize}
\tightlist
\item[]
\item
  \pandocbounded{\includesvg[keepaspectratio]{/assets/icons/16-code.svg}}
  \href{https://typst.app/universe/search/?category=scripting}{Scripting}
\end{itemize}
\end{description}

\subsubsection{Where to report issues?}\label{where-to-report-issues}

This package is a project of Danik Vitek . Report issues on
\href{https://github.com/DanikVitek/typst-plugin-bigrational}{their
repository} . You can also try to ask for help with this package on the
\href{https://forum.typst.app}{Forum} .

Please report this package to the Typst team using the
\href{https://typst.app/contact}{contact form} if you believe it is a
safety hazard or infringes upon your rights.

\phantomsection\label{versions}
\subsubsection{Version history}\label{version-history}

\begin{longtable}[]{@{}ll@{}}
\toprule\noalign{}
Version & Release Date \\
\midrule\noalign{}
\endhead
\bottomrule\noalign{}
\endlastfoot
0.1.0 & October 29, 2024 \\
\end{longtable}

Typst GmbH did not create this package and cannot guarantee correct
functionality of this package or compatibility with any version of the
Typst compiler or app.


\section{Package List LaTeX/chem-par.tex}
\title{typst.app/universe/package/chem-par}

\phantomsection\label{banner}
\section{chem-par}\label{chem-par}

{ 0.0.1 }

Display chemical formulae and IUPAC nomenclature with ease

\phantomsection\label{readme}
A utility package for displaying IUPAC nomenclature and chemical
formulae without the hassle of manually formatting all of these in your
document.

\subsection{Example Usage}\label{example-usage}

\begin{Shaded}
\begin{Highlighting}[]
\NormalTok{\#import "@preview/chem{-}par:0.0.1": *}

\NormalTok{\#set page(width: 30em, height: auto, margin: 1em)}
\NormalTok{\#show: chem{-}style}

\NormalTok{The oxidation of n{-}butanol with K2Cr2O7 requires acidification with H2SO4 to yield butanoic acid. N,N{-}dimethyltryptamine.}
\end{Highlighting}
\end{Shaded}

\pandocbounded{\includegraphics[keepaspectratio]{https://github.com/typst/packages/raw/main/packages/preview/chem-par/0.0.1/gallery/example.typ.png}}

Works on most of the common things a chemist would type

\pandocbounded{\includegraphics[keepaspectratio]{https://github.com/typst/packages/raw/main/packages/preview/chem-par/0.0.1/gallery/test.typ.png}}

\subsubsection{How to add}\label{how-to-add}

Copy this into your project and use the import as \texttt{\ chem-par\ }

\begin{verbatim}
#import "@preview/chem-par:0.0.1"
\end{verbatim}

\includesvg[width=0.16667in,height=0.16667in]{/assets/icons/16-copy.svg}

Check the docs for
\href{https://typst.app/docs/reference/scripting/\#packages}{more
information on how to import packages} .

\subsubsection{About}\label{about}

\begin{description}
\tightlist
\item[Author :]
James (Fuzzy) Swift
\item[License:]
MIT
\item[Current version:]
0.0.1
\item[Last updated:]
October 30, 2023
\item[First released:]
October 30, 2023
\item[Archive size:]
3.64 kB
\href{https://packages.typst.org/preview/chem-par-0.0.1.tar.gz}{\pandocbounded{\includesvg[keepaspectratio]{/assets/icons/16-download.svg}}}
\item[Repository:]
\href{https://github.com/JamesxX/typst-chem-par}{GitHub}
\end{description}

\subsubsection{Where to report issues?}\label{where-to-report-issues}

This package is a project of James (Fuzzy) Swift . Report issues on
\href{https://github.com/JamesxX/typst-chem-par}{their repository} . You
can also try to ask for help with this package on the
\href{https://forum.typst.app}{Forum} .

Please report this package to the Typst team using the
\href{https://typst.app/contact}{contact form} if you believe it is a
safety hazard or infringes upon your rights.

\phantomsection\label{versions}
\subsubsection{Version history}\label{version-history}

\begin{longtable}[]{@{}ll@{}}
\toprule\noalign{}
Version & Release Date \\
\midrule\noalign{}
\endhead
\bottomrule\noalign{}
\endlastfoot
0.0.1 & October 30, 2023 \\
\end{longtable}

Typst GmbH did not create this package and cannot guarantee correct
functionality of this package or compatibility with any version of the
Typst compiler or app.


\section{Package List LaTeX/chemicoms-paper.tex}
\title{typst.app/universe/package/chemicoms-paper}

\phantomsection\label{banner}
\phantomsection\label{template-thumbnail}
\pandocbounded{\includegraphics[keepaspectratio]{https://packages.typst.org/preview/thumbnails/chemicoms-paper-0.1.0-small.webp}}

\section{chemicoms-paper}\label{chemicoms-paper}

{ 0.1.0 }

An RSC-style paper template to publish at conferences and journals

\href{/app?template=chemicoms-paper&version=0.1.0}{Create project in
app}

\phantomsection\label{readme}
This is a Typst template for a two-column paper in a style similar to
that of the Royal Society of Chemistry.

\subsection{Usage}\label{usage}

You can use this template in the Typst web app by clicking “Start from
template� on the dashboard and searching for the
\texttt{\ chimicoms-paper\ } .

Alternatively, you can use the CLI to kick this project off using the
command

\begin{verbatim}
typst init @preview/chemicoms-paper
\end{verbatim}

\subsection{Configuration}\label{configuration}

This template exports the \texttt{\ template\ } function with the
following named arguments:

\begin{itemize}
\tightlist
\item
  \texttt{\ title\ } (optional, content)
\item
  \texttt{\ subtitle\ } (optional, content)
\item
  \texttt{\ short-title\ } (optional, string)
\item
  \texttt{\ author(s)\ } (optional, (array or singular) dictionary or
  string)

  \begin{itemize}
  \tightlist
  \item
    \texttt{\ name\ } (required, string, inferred)
  \item
    \texttt{\ url\ } (optional, string)
  \item
    \texttt{\ phone\ } (optional, string)
  \item
    \texttt{\ fax\ } (optional, string)
  \item
    \texttt{\ orcid\ } (optional, string)
  \item
    \texttt{\ note\ } (optional, string)
  \item
    \texttt{\ email\ } (optional, string)
  \item
    \texttt{\ corresponding\ } (optional, boolean, default true if email
    set)
  \item
    \texttt{\ equal-contributor\ } (optional, boolean)
  \item
    \texttt{\ deceased\ } (optional, boolean)
  \item
    \texttt{\ roles\ } (optional, (array or singular) string)
  \item
    \texttt{\ affiliation(s)\ } (optional, (array or singular)
    dictionary or strng)

    \begin{itemize}
    \tightlist
    \item
      either: (string) or (number)
    \end{itemize}
  \end{itemize}
\item
  \texttt{\ abstract(s)\ } (optional, (array or singular) dictionary or
  content)

  \begin{itemize}
  \tightlist
  \item
    \texttt{\ title\ } (default: “Abstract�)
  \item
    \texttt{\ content\ } (required, content, inferred)
  \end{itemize}
\item
  \texttt{\ open-access\ } (optional, boolean)
\item
  \texttt{\ venue\ } (optional, content)
\item
  \texttt{\ doi\ } (optional, string)
\item
  \texttt{\ keywords\ } (optional, array of strings)
\item
  \texttt{\ dates\ } (optional, (array or singular) dictionary or date)

  \begin{itemize}
  \tightlist
  \item
    \texttt{\ type\ } (optional, content)
  \item
    \texttt{\ date\ } (required, date or string, inferred)
  \end{itemize}
\end{itemize}

The functions also accepts a single, positional argument for the body of
the paper.

\subsection{Media}\label{media}

\includegraphics[width=0.45\linewidth,height=\textheight,keepaspectratio]{https://github.com/typst/packages/raw/main/packages/preview/chemicoms-paper/0.1.0/thumbnails/1.png}
\includegraphics[width=0.45\linewidth,height=\textheight,keepaspectratio]{https://github.com/typst/packages/raw/main/packages/preview/chemicoms-paper/0.1.0/thumbnails/2.png}

\href{/app?template=chemicoms-paper&version=0.1.0}{Create project in
app}

\subsubsection{How to use}\label{how-to-use}

Click the button above to create a new project using this template in
the Typst app.

You can also use the Typst CLI to start a new project on your computer
using this command:

\begin{verbatim}
typst init @preview/chemicoms-paper:0.1.0
\end{verbatim}

\includesvg[width=0.16667in,height=0.16667in]{/assets/icons/16-copy.svg}

\subsubsection{About}\label{about}

\begin{description}
\tightlist
\item[Author :]
James R. Swift
\item[License:]
MIT-0
\item[Current version:]
0.1.0
\item[Last updated:]
June 21, 2024
\item[First released:]
June 21, 2024
\item[Archive size:]
70.7 kB
\href{https://packages.typst.org/preview/chemicoms-paper-0.1.0.tar.gz}{\pandocbounded{\includesvg[keepaspectratio]{/assets/icons/16-download.svg}}}
\item[Repository:]
\href{https://github.com/JamesxX/chemicoms-paper}{GitHub}
\item[Categor y :]
\begin{itemize}
\tightlist
\item[]
\item
  \pandocbounded{\includesvg[keepaspectratio]{/assets/icons/16-atom.svg}}
  \href{https://typst.app/universe/search/?category=paper}{Paper}
\end{itemize}
\end{description}

\subsubsection{Where to report issues?}\label{where-to-report-issues}

This template is a project of James R. Swift . Report issues on
\href{https://github.com/JamesxX/chemicoms-paper}{their repository} .
You can also try to ask for help with this template on the
\href{https://forum.typst.app}{Forum} .

Please report this template to the Typst team using the
\href{https://typst.app/contact}{contact form} if you believe it is a
safety hazard or infringes upon your rights.

\phantomsection\label{versions}
\subsubsection{Version history}\label{version-history}

\begin{longtable}[]{@{}ll@{}}
\toprule\noalign{}
Version & Release Date \\
\midrule\noalign{}
\endhead
\bottomrule\noalign{}
\endlastfoot
0.1.0 & June 21, 2024 \\
\end{longtable}

Typst GmbH did not create this template and cannot guarantee correct
functionality of this template or compatibility with any version of the
Typst compiler or app.


\section{Package List LaTeX/diatypst.tex}
\title{typst.app/universe/package/diatypst}

\phantomsection\label{banner}
\phantomsection\label{template-thumbnail}
\pandocbounded{\includegraphics[keepaspectratio]{https://packages.typst.org/preview/thumbnails/diatypst-0.3.0-small.webp}}

\section{diatypst}\label{diatypst}

{ 0.3.0 }

Easy slides with Typst â€`` sensible defaults, easy syntax, well-styled

{ } Featured Template

\href{/app?template=diatypst&version=0.3.0}{Create project in app}

\phantomsection\label{readme}
\emph{easy slides in typst}

Features:

\begin{itemize}
\tightlist
\item
  easy delimiter for slides and sections (just use headings)
\item
  sensible styling
\item
  dot counter in upper right corner (like LaTeX beamer)
\item
  adjustable color-theme
\item
  default show rules for terms, code, lists, … that match color-theme
\end{itemize}

Example Presentation

\begin{longtable}[]{@{}llll@{}}
\toprule\noalign{}
Title Slide & Section & Content & Outline \\
\midrule\noalign{}
\endhead
\bottomrule\noalign{}
\endlastfoot
\pandocbounded{\includegraphics[keepaspectratio]{https://github.com/typst/packages/raw/main/packages/preview/diatypst/0.3.0/screenshots/Example-Title.jpg}}
&
\pandocbounded{\includegraphics[keepaspectratio]{https://github.com/typst/packages/raw/main/packages/preview/diatypst/0.3.0/screenshots/Example-Section.jpg}}
&
\pandocbounded{\includegraphics[keepaspectratio]{https://github.com/typst/packages/raw/main/packages/preview/diatypst/0.3.0/screenshots/Example-Slide.jpg}}
&
\pandocbounded{\includegraphics[keepaspectratio]{https://github.com/typst/packages/raw/main/packages/preview/diatypst/0.3.0/screenshots/Example-TOC.jpg}} \\
\end{longtable}

These example slides and a usage guide are available in the
\texttt{\ example\ } Folder on GitHub as a
\href{https://github.com/skriptum/diatypst/blob/main/example/example.typ}{.typ
file} and a
\href{https://github.com/skriptum/diatypst/blob/main/example/example.pdf}{compiled
PDF}

\subsection{Usage}\label{usage}

To start a presentation, initialize it in your typst document:

\begin{Shaded}
\begin{Highlighting}[]
\NormalTok{\#import "@preview/diatypst:0.2.0": *}
\NormalTok{\#show: slides.with(}
\NormalTok{  title: "Diatypst", // Required}
\NormalTok{  subtitle: "easy slides in typst",}
\NormalTok{  date: "01.07.2024",}
\NormalTok{  authors: ("John Doe"),}
\NormalTok{)}
\NormalTok{...}
\end{Highlighting}
\end{Shaded}

Then, insert your content.

\begin{itemize}
\tightlist
\item
  Level-one headings corresponds to new sections.
\item
  Level-two headings corresponds to new slides.
\item
  or manually create a new slide with a \texttt{\ \#pagebreak()\ }
\end{itemize}

\begin{Shaded}
\begin{Highlighting}[]
\NormalTok{...}

\NormalTok{= First Section}

\NormalTok{== First Slide}

\NormalTok{\#lorem(20)}
\end{Highlighting}
\end{Shaded}

\emph{diatypst} is also available on the
\href{https://typst.app/universe/package/diatypst}{Typst Universe} for
easy importing into a project on typst.app

\subsection{Options}\label{options}

all available Options to initialize the template with

\begin{longtable}[]{@{}lll@{}}
\toprule\noalign{}
Keyword & Description & Default \\
\midrule\noalign{}
\endhead
\bottomrule\noalign{}
\endlastfoot
\emph{title} & Title of your Presentation, visible also in footer &
\texttt{\ none\ } but required! \\
\emph{subtitle} & Subtitle, also visible in footer &
\texttt{\ none\ } \\
\emph{date} & a normal string presenting your date &
\texttt{\ none\ } \\
\emph{authors} & either string or array of strings &
\texttt{\ none\ } \\
\emph{layout} & one of “small�, “medium�, “large�, adjusts
sizing of the elements on the slides & \texttt{\ "medium"\ } \\
\emph{ratio} & aspect ratio of the slides, e.g 16/9 &
\texttt{\ 4/3\ } \\
\emph{title-color} & Color to base the Elements of the Presentation on &
\texttt{\ blue.darken(50\%)\ } \\
\emph{count} & whether to display the dots for pages in upper right
corner & \texttt{\ true\ } \\
\emph{footer} & whether to display the footer at the bottom &
\texttt{\ true\ } \\
\emph{toc} & whether to display the table of contents &
\texttt{\ true\ } \\
\emph{footer-title} & custom text in the footer title (left) & same as
\emph{title} \\
\emph{footer-subtitle} & custom text in the footer subtitle (right) &
same as \emph{subtitle} \\
\end{longtable}

\subsection{Quarto}\label{quarto}

This template is also available as a \href{https://quarto.org/}{Quarto}
extension. To use it, add it to your project with the following command:

\begin{Shaded}
\begin{Highlighting}[]
\ExtensionTok{quarto}\NormalTok{ add skriptum/diatypst/diaquarto}
\end{Highlighting}
\end{Shaded}

Then, create a qmd file with the following YAML frontmatter:

\begin{Shaded}
\begin{Highlighting}[]
\FunctionTok{title}\KeywordTok{:}\AttributeTok{ }\StringTok{"Untitled"}
\CommentTok{...}
\CommentTok{format:}
\CommentTok{  diaquarto{-}typst: }
\CommentTok{    layout: medium \# small, medium, large}
\CommentTok{    ratio: 16/9 \# any ratio possible }
\CommentTok{    title{-}color: "013220" \# Any Hex code for the title color (without \#)}
\end{Highlighting}
\end{Shaded}

\subsection{Inspiration}\label{inspiration}

this template is inspired by
\href{https://github.com/glambrechts/slydst}{slydst} , and takes part of
the code from it. If you want simpler slides, look here!

The word \emph{Diatypst} is inspired by the ease of use of a
\href{https://de.wikipedia.org/wiki/Diaprojektor}{\textbf{Dia}
-projektor} (German for Slide Projector) and the
\href{https://en.wikipedia.org/wiki/Diatype_(machine)}{Diatype}

\href{/app?template=diatypst&version=0.3.0}{Create project in app}

\subsubsection{How to use}\label{how-to-use}

Click the button above to create a new project using this template in
the Typst app.

You can also use the Typst CLI to start a new project on your computer
using this command:

\begin{verbatim}
typst init @preview/diatypst:0.3.0
\end{verbatim}

\includesvg[width=0.16667in,height=0.16667in]{/assets/icons/16-copy.svg}

\subsubsection{About}\label{about}

\begin{description}
\tightlist
\item[Author :]
skriptum (https://github.com/skriptum)
\item[License:]
MIT-0
\item[Current version:]
0.3.0
\item[Last updated:]
November 18, 2024
\item[First released:]
July 22, 2024
\item[Minimum Typst version:]
0.12.0
\item[Archive size:]
4.89 kB
\href{https://packages.typst.org/preview/diatypst-0.3.0.tar.gz}{\pandocbounded{\includesvg[keepaspectratio]{/assets/icons/16-download.svg}}}
\item[Repository:]
\href{https://github.com/skriptum/Diatypst}{GitHub}
\item[Categor y :]
\begin{itemize}
\tightlist
\item[]
\item
  \pandocbounded{\includesvg[keepaspectratio]{/assets/icons/16-presentation.svg}}
  \href{https://typst.app/universe/search/?category=presentation}{Presentation}
\end{itemize}
\end{description}

\subsubsection{Where to report issues?}\label{where-to-report-issues}

This template is a project of skriptum (https://github.com/skriptum) .
Report issues on \href{https://github.com/skriptum/Diatypst}{their
repository} . You can also try to ask for help with this template on the
\href{https://forum.typst.app}{Forum} .

Please report this template to the Typst team using the
\href{https://typst.app/contact}{contact form} if you believe it is a
safety hazard or infringes upon your rights.

\phantomsection\label{versions}
\subsubsection{Version history}\label{version-history}

\begin{longtable}[]{@{}ll@{}}
\toprule\noalign{}
Version & Release Date \\
\midrule\noalign{}
\endhead
\bottomrule\noalign{}
\endlastfoot
0.3.0 & November 18, 2024 \\
\href{https://typst.app/universe/package/diatypst/0.2.0/}{0.2.0} &
November 6, 2024 \\
\href{https://typst.app/universe/package/diatypst/0.1.0/}{0.1.0} & July
22, 2024 \\
\end{longtable}

Typst GmbH did not create this template and cannot guarantee correct
functionality of this template or compatibility with any version of the
Typst compiler or app.


\section{Package List LaTeX/tiaoma.tex}
\title{typst.app/universe/package/tiaoma}

\phantomsection\label{banner}
\section{tiaoma}\label{tiaoma}

{ 0.2.1 }

Barcode and QRCode generator for Typst using Zint.

{ } Featured Package

\phantomsection\label{readme}
\href{https://github.com/enter-tainer/zint-wasi}{tiaoma(æ?¡ç~?)} is a
barcode generator for typst. It compiles
\href{https://github.com/zint/zint}{zint} to wasm and use it to generate
barcode. It support nearly all common barcode types. For a complete list
of supported barcode types, see \href{https://zint.org.uk/}{zint’s
documentation} :

\begin{itemize}
\tightlist
\item
  Australia Post

  \begin{itemize}
  \tightlist
  \item
    Standard Customer
  \item
    Reply Paid
  \item
    Routing
  \item
    Redirection
  \end{itemize}
\item
  Aztec Code
\item
  Aztec Runes
\item
  Channel Code
\item
  Codabar
\item
  Codablock F
\item
  Code 11
\item
  Code 128 with automatic subset switching
\item
  Code 16k
\item
  Code 2 of 5 variants:

  \begin{itemize}
  \tightlist
  \item
    Matrix 2 of 5
  \item
    Industrial 2 of 5
  \item
    IATA 2 of 5
  \item
    Datalogic 2 of 5
  \item
    Interleaved 2 of 5
  \item
    ITF-14
  \end{itemize}
\item
  Deutsche Post Leitcode
\item
  Deutsche Post Identcode
\item
  Code 32 (Italian pharmacode)
\item
  Code 3 of 9 (Code 39)
\item
  Code 3 of 9 Extended (Code 39 Extended)
\item
  Code 49
\item
  Code 93
\item
  Code One
\item
  Data Matrix ECC200
\item
  DotCode
\item
  Dutch Post KIX Code
\item
  EAN variants:

  \begin{itemize}
  \tightlist
  \item
    EAN-13
  \item
    EAN-8
  \end{itemize}
\item
  Grid Matrix
\item
  GS1 DataBar variants:

  \begin{itemize}
  \tightlist
  \item
    GS1 DataBar
  \item
    GS1 DataBar Stacked
  \item
    GS1 DataBar Stacked Omnidirectional
  \item
    GS1 DataBar Expanded
  \item
    GS1 DataBar Expanded Stacked
  \item
    GS1 DataBar Limited
  \end{itemize}
\item
  Han Xin
\item
  Japan Post
\item
  Korea Post
\item
  LOGMARS
\item
  MaxiCode
\item
  MSI (Modified Plessey)
\item
  PDF417 variants:

  \begin{itemize}
  \tightlist
  \item
    PDF417 Truncated
  \item
    PDF417
  \item
    Micro PDF417
  \end{itemize}
\item
  Pharmacode
\item
  Pharmacode Two-Track
\item
  Pharmazentralnummer
\item
  POSTNET / PLANET
\item
  QR Code
\item
  rMQR
\item
  Royal Mail 4-State (RM4SCC)
\item
  Royal Mail 4-State Mailmark
\item
  Telepen
\item
  UPC variants:

  \begin{itemize}
  \tightlist
  \item
    UPC-A
  \item
    UPC-E
  \end{itemize}
\item
  UPNQR
\item
  USPS OneCode (Intelligent Mail)
\end{itemize}

\subsection{Example}\label{example}

\begin{Shaded}
\begin{Highlighting}[]
\NormalTok{\#import "@preview/tiaoma:0.2.1"}
\NormalTok{\#set page(width: auto, height: auto)}

\NormalTok{= tiáo mǎ}

\NormalTok{\#tiaoma.ean("1234567890128")}
\end{Highlighting}
\end{Shaded}

\pandocbounded{\includesvg[keepaspectratio]{https://github.com/typst/packages/raw/main/packages/preview/tiaoma/0.2.1/example.svg}}

\subsection{Manual}\label{manual}

Please refer to
\href{https://github.com/typst/packages/raw/main/packages/preview/tiaoma/0.2.1/manual.pdf}{manual}
for more details.

\subsection{Alternatives}\label{alternatives}

There are other barcode/qrcode packages for typst such as:

\begin{itemize}
\tightlist
\item
  \url{https://github.com/jneug/typst-codetastic}
\item
  \url{https://github.com/Midbin/cades}
\end{itemize}

Packages differ in provided customization options for generated
barcodes. This package is limited by zint functionality, which focuses
more on coverage than customization (e.g. inserting graphics into QR
codes). Patching upstream zint code is (currently) outside of the scope
of this package - if it doesn’t provide functionality you need, check
the rest of the typst ecosystem to see if it’s available elsewhere or
request it \href{https://github.com/zint/zint}{upstream} and
\href{https://github.com/Enter-tainer/zint-wasi/issues}{notify us} when
it’s been merged.

\subsubsection{Pros}\label{pros}

\begin{enumerate}
\tightlist
\item
  Support for far greater number of barcode types (all provided by zint
  library)
\item
  Should be faster as is uses a WASM plugin which bundles zint code
  which is written in C; others are written in pure typst or javascript.
\end{enumerate}

\subsubsection{Cons}\label{cons}

\begin{enumerate}
\tightlist
\item
  While most if not all of zint functionality is covered, it’s hard to
  guarantee there’s no overlooked functionality.
\item
  This package uses typst plugin system and has a WASM backend written
  in Rust which makes is less welcoming for new contributors.
\end{enumerate}

\subsubsection{How to add}\label{how-to-add}

Copy this into your project and use the import as \texttt{\ tiaoma\ }

\begin{verbatim}
#import "@preview/tiaoma:0.2.1"
\end{verbatim}

\includesvg[width=0.16667in,height=0.16667in]{/assets/icons/16-copy.svg}

Check the docs for
\href{https://typst.app/docs/reference/scripting/\#packages}{more
information on how to import packages} .

\subsubsection{About}\label{about}

\begin{description}
\tightlist
\item[Author s :]
Wenzhuo Liu \& Tin Å~vagelj
\item[License:]
MIT
\item[Current version:]
0.2.1
\item[Last updated:]
September 8, 2024
\item[First released:]
November 30, 2023
\item[Archive size:]
457 kB
\href{https://packages.typst.org/preview/tiaoma-0.2.1.tar.gz}{\pandocbounded{\includesvg[keepaspectratio]{/assets/icons/16-download.svg}}}
\item[Repository:]
\href{https://github.com/Enter-tainer/zint-wasi}{GitHub}
\end{description}

\subsubsection{Where to report issues?}\label{where-to-report-issues}

This package is a project of Wenzhuo Liu and Tin Å~vagelj . Report
issues on \href{https://github.com/Enter-tainer/zint-wasi}{their
repository} . You can also try to ask for help with this package on the
\href{https://forum.typst.app}{Forum} .

Please report this package to the Typst team using the
\href{https://typst.app/contact}{contact form} if you believe it is a
safety hazard or infringes upon your rights.

\phantomsection\label{versions}
\subsubsection{Version history}\label{version-history}

\begin{longtable}[]{@{}ll@{}}
\toprule\noalign{}
Version & Release Date \\
\midrule\noalign{}
\endhead
\bottomrule\noalign{}
\endlastfoot
0.2.1 & September 8, 2024 \\
\href{https://typst.app/universe/package/tiaoma/0.2.0/}{0.2.0} &
February 6, 2024 \\
\href{https://typst.app/universe/package/tiaoma/0.1.0/}{0.1.0} &
November 30, 2023 \\
\end{longtable}

Typst GmbH did not create this package and cannot guarantee correct
functionality of this package or compatibility with any version of the
Typst compiler or app.


\section{Package List LaTeX/minitoc.tex}
\title{typst.app/universe/package/minitoc}

\phantomsection\label{banner}
\section{minitoc}\label{minitoc}

{ 0.1.0 }

An outline function just for one section and nothing else

\phantomsection\label{readme}
This package provides the \texttt{\ minitoc\ } command that does the
same thing as the \texttt{\ outline\ } command but only for headings
under the heading above it.

This is inspired by minitoc package for LaTeX.

\subsection{Example}\label{example}

\begin{Shaded}
\begin{Highlighting}[]
\NormalTok{\#import "@preview/minitoc:0.1.0": *}
\NormalTok{\#set heading(numbering: "1.1")}

\NormalTok{\#outline()}

\NormalTok{= Heading 1}

\NormalTok{\#minitoc()}

\NormalTok{== Heading 1.1}

\NormalTok{\#lorem(20)}

\NormalTok{=== Heading 1.1.1}

\NormalTok{\#lorem(30)}

\NormalTok{== Heading 1.2}

\NormalTok{\#lorem(10)}

\NormalTok{= Heading 2}
\end{Highlighting}
\end{Shaded}

This produces

\pandocbounded{\includegraphics[keepaspectratio]{https://gitlab.com/human_person/typst-local-outline/-/raw/main/example/example.png}}

\subsection{Usage}\label{usage}

The \texttt{\ minitoc\ } function has the following signature:

\begin{Shaded}
\begin{Highlighting}[]
\NormalTok{\#let minitoc(}
\NormalTok{  title: none, target: heading.where(outlined: true),}
\NormalTok{    depth: none, indent: none, fill: repeat([.])}
\NormalTok{) \{ /* .. */ \}}
\end{Highlighting}
\end{Shaded}

This is designed to be as close to the
\href{https://typst.app/docs/reference/meta/outline/}{\texttt{\ outline\ }}
funtions as possible. The arguments are:

\begin{itemize}
\tightlist
\item
  \textbf{title} : The title for the local outline. This is the same as
  for
  \href{https://typst.app/docs/reference/meta/outline/\#parameters-title}{\texttt{\ outline.title\ }}
  .
\item
  \textbf{target} : What should be included. This is the same as for
  \href{https://typst.app/docs/reference/meta/outline/\#parameters-target}{\texttt{\ outline.target\ }}
\item
  \textbf{depth} : The maximum depth different to include. For example,
  if depth was 1 in the example, “Heading 1.1.1� would not be
  included
\item
  \textbf{indent} : How the entries should be indented. Takes the same
  types as for
  \href{https://typst.app/docs/reference/meta/outline/\#parameters-indent}{\texttt{\ outline.indent\ }}
  and is passed directly to it
\item
  \textbf{fill} : Content to put between the numbering and title, and
  the page number. Same types as for
  \href{https://typst.app/docs/reference/meta/outline/\#parameters-fill}{\texttt{\ outline.fill\ }}
\end{itemize}

\subsection{Unintended consequences}\label{unintended-consequences}

Because \texttt{\ minitoc\ } uses \texttt{\ outline\ } , if you apply
numbering to the title of outline with
\texttt{\ \#show\ outline:\ set\ heading(numbering:\ "1.")\ } or
similar, any title in \texttt{\ minitoc\ } will be numbered and be a
level 1 heading. This cannot be changed with
\texttt{\ \#show\ outline:\ set\ heading(level:\ 3)\ } or similar
unfortunately.

\subsubsection{How to add}\label{how-to-add}

Copy this into your project and use the import as \texttt{\ minitoc\ }

\begin{verbatim}
#import "@preview/minitoc:0.1.0"
\end{verbatim}

\includesvg[width=0.16667in,height=0.16667in]{/assets/icons/16-copy.svg}

Check the docs for
\href{https://typst.app/docs/reference/scripting/\#packages}{more
information on how to import packages} .

\subsubsection{About}\label{about}

\begin{description}
\tightlist
\item[Author :]
\href{https://github.com/RosiePuddles}{nxe}
\item[License:]
GPL-3.0-only
\item[Current version:]
0.1.0
\item[Last updated:]
January 7, 2024
\item[First released:]
January 7, 2024
\item[Archive size:]
13.6 kB
\href{https://packages.typst.org/preview/minitoc-0.1.0.tar.gz}{\pandocbounded{\includesvg[keepaspectratio]{/assets/icons/16-download.svg}}}
\item[Repository:]
\href{https://gitlab.com/human_person/typst-local-outline}{GitLab}
\end{description}

\subsubsection{Where to report issues?}\label{where-to-report-issues}

This package is a project of nxe . Report issues on
\href{https://gitlab.com/human_person/typst-local-outline}{their
repository} . You can also try to ask for help with this package on the
\href{https://forum.typst.app}{Forum} .

Please report this package to the Typst team using the
\href{https://typst.app/contact}{contact form} if you believe it is a
safety hazard or infringes upon your rights.

\phantomsection\label{versions}
\subsubsection{Version history}\label{version-history}

\begin{longtable}[]{@{}ll@{}}
\toprule\noalign{}
Version & Release Date \\
\midrule\noalign{}
\endhead
\bottomrule\noalign{}
\endlastfoot
0.1.0 & January 7, 2024 \\
\end{longtable}

Typst GmbH did not create this package and cannot guarantee correct
functionality of this package or compatibility with any version of the
Typst compiler or app.


\section{Package List LaTeX/chuli-cv.tex}
\title{typst.app/universe/package/chuli-cv}

\phantomsection\label{banner}
\phantomsection\label{template-thumbnail}
\pandocbounded{\includegraphics[keepaspectratio]{https://packages.typst.org/preview/thumbnails/chuli-cv-0.1.0-small.webp}}

\section{chuli-cv}\label{chuli-cv}

{ 0.1.0 }

Minimalistic and modern CV and cover letter templates.

\href{/app?template=chuli-cv&version=0.1.0}{Create project in app}

\phantomsection\label{readme}
These are a minimalistic and modern CV and cover letter written in
Typst.

\pandocbounded{\includegraphics[keepaspectratio]{https://github.com/typst/packages/raw/main/packages/preview/chuli-cv/0.1.0/thumbnail.png}}

\subsection{Setup}\label{setup}

\begin{itemize}
\tightlist
\item
  Install \href{https://typst.app/}{Typst} and the font awesome fonts on
  your system, see
  \href{https://github.com/duskmoon314/typst-fontawesome}{guide} .
\item
  Run \texttt{\ typst\ init\ @preview/chuli-cv:0.1.0\ } to start your
  own CV.
\end{itemize}

\subsection{Inspiration}\label{inspiration}

\begin{itemize}
\tightlist
\item
  \href{https://github.com/mintyfrankie/brilliant-CV}{brilliant-CV} .
\item
  \href{https://github.com/AnsgarLichter/light-cv}{light-cv} .
\item
  \href{https://github.com/posquit0/Awesome-CV}{Awesome CV} .
\item
  \href{https://www.overleaf.com/articles/ritabh-ranjans-cv/ngtndgryfykt}{Ritabh
  Ranjan’s CV} .
\item
  \href{https://github.com/latex-ninja/hipster-cv}{hipster-cv} .
\end{itemize}

\href{/app?template=chuli-cv&version=0.1.0}{Create project in app}

\subsubsection{How to use}\label{how-to-use}

Click the button above to create a new project using this template in
the Typst app.

You can also use the Typst CLI to start a new project on your computer
using this command:

\begin{verbatim}
typst init @preview/chuli-cv:0.1.0
\end{verbatim}

\includesvg[width=0.16667in,height=0.16667in]{/assets/icons/16-copy.svg}

\subsubsection{About}\label{about}

\begin{description}
\tightlist
\item[Author :]
Naivy Pujol Méndez
\item[License:]
MIT
\item[Current version:]
0.1.0
\item[Last updated:]
May 14, 2024
\item[First released:]
May 14, 2024
\item[Archive size:]
831 kB
\href{https://packages.typst.org/preview/chuli-cv-0.1.0.tar.gz}{\pandocbounded{\includesvg[keepaspectratio]{/assets/icons/16-download.svg}}}
\item[Repository:]
\href{https://github.com/npujol/chuli-cv}{GitHub}
\item[Categor y :]
\begin{itemize}
\tightlist
\item[]
\item
  \pandocbounded{\includesvg[keepaspectratio]{/assets/icons/16-user.svg}}
  \href{https://typst.app/universe/search/?category=cv}{CV}
\end{itemize}
\end{description}

\subsubsection{Where to report issues?}\label{where-to-report-issues}

This template is a project of Naivy Pujol Méndez . Report issues on
\href{https://github.com/npujol/chuli-cv}{their repository} . You can
also try to ask for help with this template on the
\href{https://forum.typst.app}{Forum} .

Please report this template to the Typst team using the
\href{https://typst.app/contact}{contact form} if you believe it is a
safety hazard or infringes upon your rights.

\phantomsection\label{versions}
\subsubsection{Version history}\label{version-history}

\begin{longtable}[]{@{}ll@{}}
\toprule\noalign{}
Version & Release Date \\
\midrule\noalign{}
\endhead
\bottomrule\noalign{}
\endlastfoot
0.1.0 & May 14, 2024 \\
\end{longtable}

Typst GmbH did not create this template and cannot guarantee correct
functionality of this template or compatibility with any version of the
Typst compiler or app.


\section{Package List LaTeX/report-flow-ustc.tex}
\title{typst.app/universe/package/report-flow-ustc}

\phantomsection\label{banner}
\phantomsection\label{template-thumbnail}
\pandocbounded{\includegraphics[keepaspectratio]{https://packages.typst.org/preview/thumbnails/report-flow-ustc-1.0.0-small.webp}}

\section{report-flow-ustc}\label{report-flow-ustc}

{ 1.0.0 }

A template suitable for USTC students (of course, you can freely modify
it for any school or organization) to complete course assignments or
submit lab reports.

\href{/app?template=report-flow-ustc&version=1.0.0}{Create project in
app}

\phantomsection\label{readme}
A template suitable for USTC students (of course, you can freely modify
it for any school or organization) to complete course assignments or
submit lab reports.

\href{/app?template=report-flow-ustc&version=1.0.0}{Create project in
app}

\subsubsection{How to use}\label{how-to-use}

Click the button above to create a new project using this template in
the Typst app.

You can also use the Typst CLI to start a new project on your computer
using this command:

\begin{verbatim}
typst init @preview/report-flow-ustc:1.0.0
\end{verbatim}

\includesvg[width=0.16667in,height=0.16667in]{/assets/icons/16-copy.svg}

\subsubsection{About}\label{about}

\begin{description}
\tightlist
\item[Author :]
\href{https://github.com/Quaternijkon}{Quaternijkon}
\item[License:]
MIT
\item[Current version:]
1.0.0
\item[Last updated:]
November 26, 2024
\item[First released:]
November 26, 2024
\item[Minimum Typst version:]
0.11.0
\item[Archive size:]
158 kB
\href{https://packages.typst.org/preview/report-flow-ustc-1.0.0.tar.gz}{\pandocbounded{\includesvg[keepaspectratio]{/assets/icons/16-download.svg}}}
\item[Repository:]
\href{https://github.com/Quaternijkon/Typst_Lab_Report}{GitHub}
\item[Categor y :]
\begin{itemize}
\tightlist
\item[]
\item
  \pandocbounded{\includesvg[keepaspectratio]{/assets/icons/16-speak.svg}}
  \href{https://typst.app/universe/search/?category=report}{Report}
\end{itemize}
\end{description}

\subsubsection{Where to report issues?}\label{where-to-report-issues}

This template is a project of Quaternijkon . Report issues on
\href{https://github.com/Quaternijkon/Typst_Lab_Report}{their
repository} . You can also try to ask for help with this template on the
\href{https://forum.typst.app}{Forum} .

Please report this template to the Typst team using the
\href{https://typst.app/contact}{contact form} if you believe it is a
safety hazard or infringes upon your rights.

\phantomsection\label{versions}
\subsubsection{Version history}\label{version-history}

\begin{longtable}[]{@{}ll@{}}
\toprule\noalign{}
Version & Release Date \\
\midrule\noalign{}
\endhead
\bottomrule\noalign{}
\endlastfoot
1.0.0 & November 26, 2024 \\
\end{longtable}

Typst GmbH did not create this template and cannot guarantee correct
functionality of this template or compatibility with any version of the
Typst compiler or app.


\section{Package List LaTeX/peace-of-posters.tex}
\title{typst.app/universe/package/peace-of-posters}

\phantomsection\label{banner}
\phantomsection\label{template-thumbnail}
\pandocbounded{\includegraphics[keepaspectratio]{https://packages.typst.org/preview/thumbnails/peace-of-posters-0.5.0-small.webp}}

\section{peace-of-posters}\label{peace-of-posters}

{ 0.5.0 }

Create scientific posters in Typst.

\href{/app?template=peace-of-posters&version=0.5.0}{Create project in
app}

\phantomsection\label{readme}
\pandocbounded{\includegraphics[keepaspectratio]{https://img.shields.io/github/actions/workflow/status/jonaspleyer/peace-of-posters/test.yml?style=flat-square&label=Test}}
\pandocbounded{\includegraphics[keepaspectratio]{https://img.shields.io/github/actions/workflow/status/jonaspleyer/peace-of-posters/docs.yml?style=flat-square&label=Docs}}

\begin{quote}
piece of cake\\
peace of mind\\
peace of posters
\end{quote}

\href{https://github.com/jonaspleyer/peace-of-posters}{peace-of-posters
(PoP)} is a Typst package to help creating scientific posters. It is
flexible and can be used for different sizes and layouts. To see what is
possible have a look at some of my own real-world examples in the
\href{https://jonaspleyer.github.io/peace-of-posters/showcase/}{showcase}
section of the documentation.

\subsection{Documentation}\label{documentation}

The external
\href{https://jonaspleyer.github.io/peace-of-posters/}{documentation} is
coming along slowly. Most notably, there are examples and showcases
missing but I hope to be adding them over the coming months.

\subsection{License}\label{license}

Download the \href{https://www.mit.edu/~amini/LICENSE.md}{MIT License}

\href{/app?template=peace-of-posters&version=0.5.0}{Create project in
app}

\subsubsection{How to use}\label{how-to-use}

Click the button above to create a new project using this template in
the Typst app.

You can also use the Typst CLI to start a new project on your computer
using this command:

\begin{verbatim}
typst init @preview/peace-of-posters:0.5.0
\end{verbatim}

\includesvg[width=0.16667in,height=0.16667in]{/assets/icons/16-copy.svg}

\subsubsection{About}\label{about}

\begin{description}
\tightlist
\item[Author :]
\href{mailto:jonas.sci@pleyer.org}{Jonas Pleyer}
\item[License:]
MIT
\item[Current version:]
0.5.0
\item[Last updated:]
October 25, 2024
\item[First released:]
May 31, 2024
\item[Archive size:]
213 kB
\href{https://packages.typst.org/preview/peace-of-posters-0.5.0.tar.gz}{\pandocbounded{\includesvg[keepaspectratio]{/assets/icons/16-download.svg}}}
\item[Repository:]
\href{https://github.com/jonaspleyer/peace-of-posters}{GitHub}
\item[Categor y :]
\begin{itemize}
\tightlist
\item[]
\item
  \pandocbounded{\includesvg[keepaspectratio]{/assets/icons/16-pin.svg}}
  \href{https://typst.app/universe/search/?category=poster}{Poster}
\end{itemize}
\end{description}

\subsubsection{Where to report issues?}\label{where-to-report-issues}

This template is a project of Jonas Pleyer . Report issues on
\href{https://github.com/jonaspleyer/peace-of-posters}{their repository}
. You can also try to ask for help with this template on the
\href{https://forum.typst.app}{Forum} .

Please report this template to the Typst team using the
\href{https://typst.app/contact}{contact form} if you believe it is a
safety hazard or infringes upon your rights.

\phantomsection\label{versions}
\subsubsection{Version history}\label{version-history}

\begin{longtable}[]{@{}ll@{}}
\toprule\noalign{}
Version & Release Date \\
\midrule\noalign{}
\endhead
\bottomrule\noalign{}
\endlastfoot
0.5.0 & October 25, 2024 \\
\href{https://typst.app/universe/package/peace-of-posters/0.4.3/}{0.4.3}
& October 22, 2024 \\
\href{https://typst.app/universe/package/peace-of-posters/0.4.1/}{0.4.1}
& June 3, 2024 \\
\href{https://typst.app/universe/package/peace-of-posters/0.4.0/}{0.4.0}
& May 31, 2024 \\
\end{longtable}

Typst GmbH did not create this template and cannot guarantee correct
functionality of this template or compatibility with any version of the
Typst compiler or app.


\section{Package List LaTeX/simplebnf.tex}
\title{typst.app/universe/package/simplebnf}

\phantomsection\label{banner}
\section{simplebnf}\label{simplebnf}

{ 0.1.1 }

A simple package to format Backus-Naur form (BNF)

\phantomsection\label{readme}
simplebnf is a simple package to format Backus-Naur form. The package
provides a simple way to format Backus-Naur form (BNF). It provides
constructs to denote BNF expressions, possibly with annotations.

This is a sister package of
\href{https://github.com/Zeta611/simplebnf}{simplebnf} , a LaTeX package
under the same name by the author.

\subsection{Usage}\label{usage}

Import simplebnf via

\begin{Shaded}
\begin{Highlighting}[]
\NormalTok{\#import "@preview/simplebnf:0.1.1": *}
\end{Highlighting}
\end{Shaded}

Use the \texttt{\ bnf\ } function to display the BNF production rules.
Each production rule can be created using the \texttt{\ Prod\ }
constructor function, which accepts the (left-hand side) metavariable,
an optional annotation for it, an optional delimiter (which defaults to
â©´), and a list of (right-hand side) alternatives. Each alternative
should be created using the \texttt{\ Or\ } constructor, which accepts a
syntactic form and an annotation.

Below are some examples using simplebnf.

\begin{Shaded}
\begin{Highlighting}[]
\NormalTok{\#bnf(}
\NormalTok{  Prod(}
\NormalTok{    $e$,}
\NormalTok{    annot: $sans("Expr")$,}
\NormalTok{    \{}
\NormalTok{      Or[$x$][\_variable\_]}
\NormalTok{      Or[$λ x. e$][\_abstraction\_]}
\NormalTok{      Or[$e$ $e$][\_application\_]}
\NormalTok{    \},}
\NormalTok{  ),}
\NormalTok{)}
\end{Highlighting}
\end{Shaded}

\pandocbounded{\includesvg[keepaspectratio]{https://github.com/typst/packages/raw/main/packages/preview/simplebnf/0.1.1/examples/lambda.svg}}

\begin{Shaded}
\begin{Highlighting}[]
\NormalTok{\#bnf(}
\NormalTok{  Prod(}
\NormalTok{    $e$,}
\NormalTok{    delim: $→$,}
\NormalTok{    \{}
\NormalTok{      Or[$x$][variable]}
\NormalTok{      Or[$λ x: τ.e$][abstraction]}
\NormalTok{      Or[$e space e$][application]}
\NormalTok{      Or[$λ τ.e space e$][type abstraction]}
\NormalTok{      Or[$e space [τ]$][type application]}
\NormalTok{    \},}
\NormalTok{  ),}
\NormalTok{  Prod(}
\NormalTok{    $τ$,}
\NormalTok{    delim: $→$,}
\NormalTok{    \{}
\NormalTok{      Or[$X$][type variable]}
\NormalTok{      Or[$τ → τ$][type of functions]}
\NormalTok{      Or[$∀X.τ$][universal quantification]}
\NormalTok{    \},}
\NormalTok{  ),}
\NormalTok{)}
\end{Highlighting}
\end{Shaded}

\pandocbounded{\includesvg[keepaspectratio]{https://github.com/typst/packages/raw/main/packages/preview/simplebnf/0.1.1/examples/system-f.svg}}

\subsection{Authors}\label{authors}

\begin{itemize}
\tightlist
\item
  Jay Lee
  \href{mailto:jaeho.lee@snu.ac.kr}{\nolinkurl{jaeho.lee@snu.ac.kr}}
\end{itemize}

\subsection{License}\label{license}

simplebnf.typ is available under the MIT license. See the
\href{https://github.com/Zeta611/simplebnf.typ/blob/master/LICENSE}{LICENSE}
file for more info.

\subsubsection{How to add}\label{how-to-add}

Copy this into your project and use the import as \texttt{\ simplebnf\ }

\begin{verbatim}
#import "@preview/simplebnf:0.1.1"
\end{verbatim}

\includesvg[width=0.16667in,height=0.16667in]{/assets/icons/16-copy.svg}

Check the docs for
\href{https://typst.app/docs/reference/scripting/\#packages}{more
information on how to import packages} .

\subsubsection{About}\label{about}

\begin{description}
\tightlist
\item[Author :]
\href{https://github.com/Zeta611}{Jay Lee}
\item[License:]
MIT
\item[Current version:]
0.1.1
\item[Last updated:]
July 15, 2024
\item[First released:]
May 23, 2024
\item[Archive size:]
2.10 kB
\href{https://packages.typst.org/preview/simplebnf-0.1.1.tar.gz}{\pandocbounded{\includesvg[keepaspectratio]{/assets/icons/16-download.svg}}}
\item[Repository:]
\href{https://github.com/Zeta611/simplebnf.typ}{GitHub}
\item[Discipline :]
\begin{itemize}
\tightlist
\item[]
\item
  \href{https://typst.app/universe/search/?discipline=computer-science}{Computer
  Science}
\end{itemize}
\item[Categor ies :]
\begin{itemize}
\tightlist
\item[]
\item
  \pandocbounded{\includesvg[keepaspectratio]{/assets/icons/16-package.svg}}
  \href{https://typst.app/universe/search/?category=components}{Components}
\item
  \pandocbounded{\includesvg[keepaspectratio]{/assets/icons/16-chart.svg}}
  \href{https://typst.app/universe/search/?category=visualization}{Visualization}
\item
  \pandocbounded{\includesvg[keepaspectratio]{/assets/icons/16-integration.svg}}
  \href{https://typst.app/universe/search/?category=integration}{Integration}
\end{itemize}
\end{description}

\subsubsection{Where to report issues?}\label{where-to-report-issues}

This package is a project of Jay Lee . Report issues on
\href{https://github.com/Zeta611/simplebnf.typ}{their repository} . You
can also try to ask for help with this package on the
\href{https://forum.typst.app}{Forum} .

Please report this package to the Typst team using the
\href{https://typst.app/contact}{contact form} if you believe it is a
safety hazard or infringes upon your rights.

\phantomsection\label{versions}
\subsubsection{Version history}\label{version-history}

\begin{longtable}[]{@{}ll@{}}
\toprule\noalign{}
Version & Release Date \\
\midrule\noalign{}
\endhead
\bottomrule\noalign{}
\endlastfoot
0.1.1 & July 15, 2024 \\
\href{https://typst.app/universe/package/simplebnf/0.1.0/}{0.1.0} & May
23, 2024 \\
\end{longtable}

Typst GmbH did not create this package and cannot guarantee correct
functionality of this package or compatibility with any version of the
Typst compiler or app.


\section{Package List LaTeX/keyle.tex}
\title{typst.app/universe/package/keyle}

\phantomsection\label{banner}
\section{keyle}\label{keyle}

{ 0.2.0 }

This package provides a simple way to style keyboard shortcuts in your
documentation.

\phantomsection\label{readme}
\href{https://raw.githubusercontent.com/magicwenli/keyle/main/doc/keyle.pdf}{\pandocbounded{\includegraphics[keepaspectratio]{https://img.shields.io/website?down_message=offline&label=manual&up_color=007aff&up_message=online&url=https://raw.githubusercontent.com/magicwenli/keyle/main/doc/keyle.pdf}}}
\href{https://github.com/magicwenli/keyle/blob/main/LICENSE}{\pandocbounded{\includegraphics[keepaspectratio]{https://img.shields.io/badge/license-MIT-brightgreen}}}

A simple way to style keyboard shortcuts in your documentation.

This package was inspired by
\href{https://auth0.github.io/kbd/}{auth0/kbd} and
\href{https://github.com/dogezen/badgery}{dogezen/badgery} . Also thanks
to \href{https://github.com/tweh/menukeys}{tweh/menukeys} â€`` A LaTeX
package for menu keys generation.

Document generating using
\href{https://github.com/jneug/typst-mantys}{jneug/typst-mantys} .

Send them respect and love.

\subsection{Usage}\label{usage}

Please see the
\href{https://github.com/magicwenli/keyle/blob/main/doc/keyle.pdf}{keyle.pdf}
for more documentation.

\texttt{\ keyle\ } is imported using:

\begin{Shaded}
\begin{Highlighting}[]
\NormalTok{\#import "@preview/keyle:0.2.0"}
\end{Highlighting}
\end{Shaded}

\subsubsection{Example}\label{example}

\paragraph{Custom Delimiter}\label{custom-delimiter}

\begin{Shaded}
\begin{Highlighting}[]
\NormalTok{\#let kbd = keyle.config()}
\NormalTok{\#kbd("Ctrl", "Shift", "K", delim: "{-}")}
\end{Highlighting}
\end{Shaded}

\pandocbounded{\includegraphics[keepaspectratio]{https://github.com/typst/packages/raw/main/packages/preview/keyle/0.2.0/test/test-1.png}}

\paragraph{Compact Mode}\label{compact-mode}

\begin{Shaded}
\begin{Highlighting}[]
\NormalTok{\#let kbd = keyle.config()}
\NormalTok{\#kbd("Ctrl", "Shift", "K", compact: true)}
\end{Highlighting}
\end{Shaded}

\pandocbounded{\includegraphics[keepaspectratio]{https://github.com/typst/packages/raw/main/packages/preview/keyle/0.2.0/test/test-2.png}}

\paragraph{Standard Theme}\label{standard-theme}

\begin{Shaded}
\begin{Highlighting}[]
\NormalTok{\#let kbd = keyle.config(theme: keyle.themes.standard)}
\NormalTok{\#keyle.gen{-}examples(kbd)}
\end{Highlighting}
\end{Shaded}

\pandocbounded{\includegraphics[keepaspectratio]{https://github.com/typst/packages/raw/main/packages/preview/keyle/0.2.0/test/test-3.png}}

\paragraph{Deep Blue Theme}\label{deep-blue-theme}

\begin{Shaded}
\begin{Highlighting}[]
\NormalTok{\#let kbd = keyle.config(theme: keyle.themes.deep{-}blue)}
\NormalTok{\#keyle.gen{-}examples(kbd)}
\end{Highlighting}
\end{Shaded}

\pandocbounded{\includegraphics[keepaspectratio]{https://github.com/typst/packages/raw/main/packages/preview/keyle/0.2.0/test/test-4.png}}

\paragraph{Type Writer Theme}\label{type-writer-theme}

\begin{Shaded}
\begin{Highlighting}[]
\NormalTok{\#let kbd = keyle.config(theme: keyle.themes.type{-}writer)}
\NormalTok{\#keyle.gen{-}examples(kbd)}
\end{Highlighting}
\end{Shaded}

\pandocbounded{\includegraphics[keepaspectratio]{https://github.com/typst/packages/raw/main/packages/preview/keyle/0.2.0/test/test-5.png}}

\paragraph{Biolinum Theme}\label{biolinum-theme}

\begin{Shaded}
\begin{Highlighting}[]
\NormalTok{\#let kbd = keyle.config(theme: keyle.themes.biolinum, delim: keyle.biolinum{-}key.delim\_plus)}
\NormalTok{\#keyle.gen{-}examples(kbd)}
\end{Highlighting}
\end{Shaded}

\pandocbounded{\includegraphics[keepaspectratio]{https://github.com/typst/packages/raw/main/packages/preview/keyle/0.2.0/test/test-6.png}}

\paragraph{Custom Theme}\label{custom-theme}

\begin{Shaded}
\begin{Highlighting}[]
\NormalTok{// https://www.radix{-}ui.com/themes/playground\#kbd}
\NormalTok{\#let radix\_kdb(content) = box(}
\NormalTok{  rect(}
\NormalTok{    inset: (x: 0.5em),}
\NormalTok{    outset: (y:0.05em),}
\NormalTok{    stroke: rgb("\#1c2024") + 0.3pt,}
\NormalTok{    radius: 0.35em,}
\NormalTok{    fill: rgb("\#fcfcfd"),}
\NormalTok{    text(fill: black, font: (}
\NormalTok{      "Roboto",}
\NormalTok{      "Helvetica Neue",}
\NormalTok{    ), content),}
\NormalTok{  ),}
\NormalTok{)}
\NormalTok{\#let kbd = keyle.config(theme: radix\_kdb)}
\NormalTok{\#keyle.gen{-}examples(kbd)}
\end{Highlighting}
\end{Shaded}

\pandocbounded{\includegraphics[keepaspectratio]{https://github.com/typst/packages/raw/main/packages/preview/keyle/0.2.0/test/test-7.png}}

\subsection{License}\label{license}

MIT

\subsubsection{How to add}\label{how-to-add}

Copy this into your project and use the import as \texttt{\ keyle\ }

\begin{verbatim}
#import "@preview/keyle:0.2.0"
\end{verbatim}

\includesvg[width=0.16667in,height=0.16667in]{/assets/icons/16-copy.svg}

Check the docs for
\href{https://typst.app/docs/reference/scripting/\#packages}{more
information on how to import packages} .

\subsubsection{About}\label{about}

\begin{description}
\tightlist
\item[Author :]
\href{mailto:yxnian@outlook.com}{magicwenli}
\item[License:]
MIT
\item[Current version:]
0.2.0
\item[Last updated:]
August 27, 2024
\item[First released:]
July 24, 2024
\item[Minimum Typst version:]
0.11.1
\item[Archive size:]
5.97 kB
\href{https://packages.typst.org/preview/keyle-0.2.0.tar.gz}{\pandocbounded{\includesvg[keepaspectratio]{/assets/icons/16-download.svg}}}
\item[Repository:]
\href{https://github.com/magicwenli/keyle}{GitHub}
\item[Categor ies :]
\begin{itemize}
\tightlist
\item[]
\item
  \pandocbounded{\includesvg[keepaspectratio]{/assets/icons/16-hammer.svg}}
  \href{https://typst.app/universe/search/?category=utility}{Utility}
\item
  \pandocbounded{\includesvg[keepaspectratio]{/assets/icons/16-smile.svg}}
  \href{https://typst.app/universe/search/?category=fun}{Fun}
\end{itemize}
\end{description}

\subsubsection{Where to report issues?}\label{where-to-report-issues}

This package is a project of magicwenli . Report issues on
\href{https://github.com/magicwenli/keyle}{their repository} . You can
also try to ask for help with this package on the
\href{https://forum.typst.app}{Forum} .

Please report this package to the Typst team using the
\href{https://typst.app/contact}{contact form} if you believe it is a
safety hazard or infringes upon your rights.

\phantomsection\label{versions}
\subsubsection{Version history}\label{version-history}

\begin{longtable}[]{@{}ll@{}}
\toprule\noalign{}
Version & Release Date \\
\midrule\noalign{}
\endhead
\bottomrule\noalign{}
\endlastfoot
0.2.0 & August 27, 2024 \\
\href{https://typst.app/universe/package/keyle/0.1.1/}{0.1.1} & August
12, 2024 \\
\href{https://typst.app/universe/package/keyle/0.1.0/}{0.1.0} & July 24,
2024 \\
\end{longtable}

Typst GmbH did not create this package and cannot guarantee correct
functionality of this package or compatibility with any version of the
Typst compiler or app.


\section{Package List LaTeX/codly-languages.tex}
\title{typst.app/universe/package/codly-languages}

\phantomsection\label{banner}
\section{codly-languages}\label{codly-languages}

{ 0.1.1 }

A set of language configurations for use with codly

\phantomsection\label{readme}
Provides a set of predefined language configurations for use with the
\texttt{\ codly\ } code listing package. For each supported language,
this package defines a name, icon, and color to use when displaying
code.

\subsection{Usage}\label{usage}

Pretty simple. Import \texttt{\ codly\ } . Initialize it. Import
\texttt{\ codly-languages\ } . Configure \texttt{\ codly\ } with the
languages. Like this:

\begin{Shaded}
\begin{Highlighting}[]
\NormalTok{\#import "@preview/codly:1.0.0": *}
\NormalTok{\#show: codly{-}init}

\NormalTok{\#import "@preview/codly{-}languages:0.1.1": *}
\NormalTok{\#codly(languages: codly{-}languages)}
\end{Highlighting}
\end{Shaded}

Then use code blocks as you normally would and the output, for supported
languages, should look like this:

\pandocbounded{\includegraphics[keepaspectratio]{https://github.com/typst/packages/raw/main/packages/preview/codly-languages/0.1.1/thumbnail.png}}

\subsection{Contributing}\label{contributing}

The following languages are still missing. All contributions welcome.

\begin{itemize}
\tightlist
\item
  ASP
\item
  ActionScript
\item
  Ada
\item
  AppleScript
\item
  AsciiDoc
\item
  Batch File
\item
  CFML
\item
  CSV
\item
  Cabal
\item
  Crontab
\item
  D
\item
  Diff
\item
  DotENV
\item
  Email
\item
  Fish
\item
  Fstab
\item
  GLSL
\item
  Graphviz
\item
  Groff
\item
  Group
\item
  INI
\item
  Jinja2
\item
  Jsonnet
\item
  Lean
\item
  Lisp
\item
  LiveScript
\item
  Makefile
\item
  MediaWiki
\item
  NSIS
\item
  Ninja
\item
  Org mode
\item
  Pascal
\item
  Passwd
\item
  Protobuf
\item
  Puppet
\item
  QML
\item
  Racket
\item
  Rego
\item
  Regular Expressions
\item
  Resolv
\item
  RestructuredText
\item
  Robot
\item
  SLS
\item
  SML
\item
  Slim
\item
  Strace
\item
  SublimeEthereum
\item
  SublimeJQ
\item
  SystemVerilo
\item
  TCL
\item
  TOML
\item
  Textile
\item
  TodoTxt
\item
  Verilog
\item
  WGSL
\item
  cmd-help
\item
  gnuplot
\item
  hosts
\item
  http-request-response
\item
  varlink
\item
  vscode-wgsl
\end{itemize}

\subsection{Icon Attribution}\label{icon-attribution}

The \texttt{\ typst-small.png\ } icon included in this package came from
the MIT-licensed \href{https://github.com/Dherse/codly}{codly} project.

All other icons included here came from the MIT-licensed
\href{https://github.com/devicons/devicon/}{devicon} project.

\subsection{License}\label{license}

This package is released under the MIT License.

\subsubsection{How to add}\label{how-to-add}

Copy this into your project and use the import as
\texttt{\ codly-languages\ }

\begin{verbatim}
#import "@preview/codly-languages:0.1.1"
\end{verbatim}

\includesvg[width=0.16667in,height=0.16667in]{/assets/icons/16-copy.svg}

Check the docs for
\href{https://typst.app/docs/reference/scripting/\#packages}{more
information on how to import packages} .

\subsubsection{About}\label{about}

\begin{description}
\tightlist
\item[Author :]
\href{mailto:steve@waits.net}{Stephen Waits}
\item[License:]
MIT
\item[Current version:]
0.1.1
\item[Last updated:]
November 21, 2024
\item[First released:]
November 18, 2024
\item[Archive size:]
96.2 kB
\href{https://packages.typst.org/preview/codly-languages-0.1.1.tar.gz}{\pandocbounded{\includesvg[keepaspectratio]{/assets/icons/16-download.svg}}}
\item[Repository:]
\href{https://github.com/swaits/typst-collection}{GitHub}
\end{description}

\subsubsection{Where to report issues?}\label{where-to-report-issues}

This package is a project of Stephen Waits . Report issues on
\href{https://github.com/swaits/typst-collection}{their repository} .
You can also try to ask for help with this package on the
\href{https://forum.typst.app}{Forum} .

Please report this package to the Typst team using the
\href{https://typst.app/contact}{contact form} if you believe it is a
safety hazard or infringes upon your rights.

\phantomsection\label{versions}
\subsubsection{Version history}\label{version-history}

\begin{longtable}[]{@{}ll@{}}
\toprule\noalign{}
Version & Release Date \\
\midrule\noalign{}
\endhead
\bottomrule\noalign{}
\endlastfoot
0.1.1 & November 21, 2024 \\
\href{https://typst.app/universe/package/codly-languages/0.1.0/}{0.1.0}
& November 18, 2024 \\
\end{longtable}

Typst GmbH did not create this package and cannot guarantee correct
functionality of this package or compatibility with any version of the
Typst compiler or app.


\section{Package List LaTeX/treet.tex}
\title{typst.app/universe/package/treet}

\phantomsection\label{banner}
\section{treet}\label{treet}

{ 0.1.1 }

Create tree lists easily

\phantomsection\label{readme}
\href{https://github.com/8LWXpg/typst-treet/tags}{\pandocbounded{\includegraphics[keepaspectratio]{https://img.shields.io/github/v/tag/8LWXpg/typst-treet}}}
\href{https://github.com/8LWXpg/typst-treet}{\pandocbounded{\includegraphics[keepaspectratio]{https://img.shields.io/github/stars/8LWXpg/typst-treet?style=flat}}}
\href{https://github.com/8LWXpg/typst-treet/blob/master/LICENSE}{\pandocbounded{\includegraphics[keepaspectratio]{https://img.shields.io/github/license/8LWXpg/typst-treet}}}
\href{https://github.com/typst/packages/tree/main/packages/preview/treet}{\pandocbounded{\includegraphics[keepaspectratio]{https://img.shields.io/badge/typst-package-239dad}}}

Create tree list easily in Typst

contribution is welcomed!

\subsection{Usage}\label{usage}

\begin{Shaded}
\begin{Highlighting}[]
\NormalTok{\#import "@preview/treet:0.1.1": *}

\NormalTok{\#tree{-}list(}
\NormalTok{  marker:       content,}
\NormalTok{  last{-}marker:  content,}
\NormalTok{  indent:       content,}
\NormalTok{  empty{-}indent: content,}
\NormalTok{  marker{-}font:  string,}
\NormalTok{  content,}
\NormalTok{)}
\end{Highlighting}
\end{Shaded}

\subsubsection{Parameters}\label{parameters}

\begin{itemize}
\tightlist
\item
  \texttt{\ marker\ } - the marker of the tree list, default is
  \texttt{\ {[}├─\ {]}\ }
\item
  \texttt{\ last-marker\ } - the marker of the last item of the tree
  list, default is \texttt{\ {[}└─\ {]}\ }
\item
  \texttt{\ indent\ } - the indent after \texttt{\ marker\ } , default
  is \texttt{\ {[}│\#h(1em){]}\ }
\item
  \texttt{\ empty-indent\ } - the indent after \texttt{\ last-marker\ }
  , default is \texttt{\ {[}\#h(1.5em){]}\ } (same width as indent)
\item
  \texttt{\ marker-font\ } - the font of the marker, default is
  \texttt{\ "Cascadia\ Code"\ }
\item
  \texttt{\ content\ } - the content of the tree list, includes at least
  a list
\end{itemize}

\subsection{Demo}\label{demo}

see
\href{https://github.com/8LWXpg/typst-treet/blob/master/test/demo.typ}{demo.typ}
\href{https://github.com/8LWXpg/typst-treet/blob/master/test/demo.pdf}{demo.pdf}

\subsubsection{Default style}\label{default-style}

\begin{Shaded}
\begin{Highlighting}[]
\NormalTok{\#tree{-}list[}
\NormalTok{  {-} 1}
\NormalTok{    {-} 1.1}
\NormalTok{      {-} 1.1.1}
\NormalTok{    {-} 1.2}
\NormalTok{      {-} 1.2.1}
\NormalTok{      {-} 1.2.2}
\NormalTok{        {-} 1.2.2.1}
\NormalTok{  {-} 2}
\NormalTok{  {-} 3}
\NormalTok{    {-} 3.1}
\NormalTok{      {-} 3.1.1}
\NormalTok{    {-} 3.2}
\NormalTok{]}
\end{Highlighting}
\end{Shaded}

\pandocbounded{\includegraphics[keepaspectratio]{https://raw.githubusercontent.com/8LWXpg/typst-treet/master/img/1.png}}

\subsubsection{Custom style}\label{custom-style}

\begin{Shaded}
\begin{Highlighting}[]
\NormalTok{\#text(red, tree{-}list(}
\NormalTok{  marker: text(blue)[├── ],}
\NormalTok{  last{-}marker: text(aqua)[└── ],}
\NormalTok{  indent: text(teal)[│\#h(1.5em)],}
\NormalTok{  empty{-}indent: h(2em),}
\NormalTok{)[}
\NormalTok{  {-} 1}
\NormalTok{    {-} 1.1}
\NormalTok{      {-} 1.1.1}
\NormalTok{    {-} 1.2}
\NormalTok{      {-} 1.2.1}
\NormalTok{      {-} 1.2.2}
\NormalTok{        {-} 1.2.2.1}
\NormalTok{  {-} 2}
\NormalTok{  {-} 3}
\NormalTok{    {-} 3.1}
\NormalTok{      {-} 3.1.1}
\NormalTok{    {-} 3.2}
\NormalTok{])}
\end{Highlighting}
\end{Shaded}

\pandocbounded{\includegraphics[keepaspectratio]{https://raw.githubusercontent.com/8LWXpg/typst-treet/master/img/2.png}}

\subsubsection{Using show rule}\label{using-show-rule}

\begin{Shaded}
\begin{Highlighting}[]
\NormalTok{\#show list: tree{-}list}
\NormalTok{\#set text(font: "DejaVu Sans Mono")}

\NormalTok{root\_folder\textbackslash{}}
\NormalTok{{-} sub{-}folder}
\NormalTok{  {-} 1{-}1}
\NormalTok{    {-} 1.1.1 {-}}
\NormalTok{  {-} 1.2}
\NormalTok{    {-} 1.2.1}
\NormalTok{    {-} 1.2.2}
\NormalTok{{-} 2}
\end{Highlighting}
\end{Shaded}

\pandocbounded{\includegraphics[keepaspectratio]{https://raw.githubusercontent.com/8LWXpg/typst-treet/master/img/3.png}}

\subsubsection{How to add}\label{how-to-add}

Copy this into your project and use the import as \texttt{\ treet\ }

\begin{verbatim}
#import "@preview/treet:0.1.1"
\end{verbatim}

\includesvg[width=0.16667in,height=0.16667in]{/assets/icons/16-copy.svg}

Check the docs for
\href{https://typst.app/docs/reference/scripting/\#packages}{more
information on how to import packages} .

\subsubsection{About}\label{about}

\begin{description}
\tightlist
\item[Author :]
8LWXpg
\item[License:]
MIT
\item[Current version:]
0.1.1
\item[Last updated:]
April 15, 2024
\item[First released:]
January 8, 2024
\item[Minimum Typst version:]
0.10.0
\item[Archive size:]
2.41 kB
\href{https://packages.typst.org/preview/treet-0.1.1.tar.gz}{\pandocbounded{\includesvg[keepaspectratio]{/assets/icons/16-download.svg}}}
\item[Repository:]
\href{https://github.com/8LWXpg/typst-treet}{GitHub}
\item[Categor ies :]
\begin{itemize}
\tightlist
\item[]
\item
  \pandocbounded{\includesvg[keepaspectratio]{/assets/icons/16-package.svg}}
  \href{https://typst.app/universe/search/?category=components}{Components}
\item
  \pandocbounded{\includesvg[keepaspectratio]{/assets/icons/16-layout.svg}}
  \href{https://typst.app/universe/search/?category=layout}{Layout}
\end{itemize}
\end{description}

\subsubsection{Where to report issues?}\label{where-to-report-issues}

This package is a project of 8LWXpg . Report issues on
\href{https://github.com/8LWXpg/typst-treet}{their repository} . You can
also try to ask for help with this package on the
\href{https://forum.typst.app}{Forum} .

Please report this package to the Typst team using the
\href{https://typst.app/contact}{contact form} if you believe it is a
safety hazard or infringes upon your rights.

\phantomsection\label{versions}
\subsubsection{Version history}\label{version-history}

\begin{longtable}[]{@{}ll@{}}
\toprule\noalign{}
Version & Release Date \\
\midrule\noalign{}
\endhead
\bottomrule\noalign{}
\endlastfoot
0.1.1 & April 15, 2024 \\
\href{https://typst.app/universe/package/treet/0.1.0/}{0.1.0} & January
8, 2024 \\
\end{longtable}

Typst GmbH did not create this package and cannot guarantee correct
functionality of this package or compatibility with any version of the
Typst compiler or app.


\section{Package List LaTeX/graceful-genetics.tex}
\title{typst.app/universe/package/graceful-genetics}

\phantomsection\label{banner}
\phantomsection\label{template-thumbnail}
\pandocbounded{\includegraphics[keepaspectratio]{https://packages.typst.org/preview/thumbnails/graceful-genetics-0.2.0-small.webp}}

\section{graceful-genetics}\label{graceful-genetics}

{ 0.2.0 }

A paper template with which to publish in journals and at conferences

{ } Featured Template

\href{/app?template=graceful-genetics&version=0.2.0}{Create project in
app}

\phantomsection\label{readme}
Version 0.2.0

A recreation of the Oxford Physics template shown on the typst.app
homepage.

\subsection{Media}\label{media}

\includegraphics[width=0.45\linewidth,height=\textheight,keepaspectratio]{https://github.com/typst/packages/raw/main/packages/preview/graceful-genetics/0.2.0/thumbnails/1.png}
\includegraphics[width=0.45\linewidth,height=\textheight,keepaspectratio]{https://github.com/typst/packages/raw/main/packages/preview/graceful-genetics/0.2.0/thumbnails/2.png}

\subsection{Getting Started}\label{getting-started}

To use this template, simply import it as shown below:

\begin{Shaded}
\begin{Highlighting}[]
\NormalTok{\#import "@preview/graceful{-}genetics:0.2.0"}

\NormalTok{\#show: graceful{-}genetics.template.with(}
\NormalTok{  title: [Towards Swifter Interstellar Mail Delivery],}
\NormalTok{  authors: (}
\NormalTok{    (}
\NormalTok{      name: "Johanna Swift",}
\NormalTok{      department: "Primary Logistics Department",}
\NormalTok{      institution: "Delivery Institute",}
\NormalTok{      city: "Berlin",}
\NormalTok{      country: "Germany",}
\NormalTok{      mail: "swift@delivery.de",}
\NormalTok{    ),}
\NormalTok{    (}
\NormalTok{      name: "Egon Stellaris",}
\NormalTok{      department: "Communications Group",}
\NormalTok{      institution: "Space Institute",}
\NormalTok{      city: "Florence",}
\NormalTok{      country: "Italy",}
\NormalTok{      mail: "stegonaris@space.it",}
\NormalTok{    ),}
\NormalTok{    (}
\NormalTok{      name: "Oliver Liam",}
\NormalTok{      department: "Missing Letters Task Force",}
\NormalTok{      institution: "Mail Institute",}
\NormalTok{      city: "Budapest",}
\NormalTok{      country: "Hungary",}
\NormalTok{      mail: "oliver.liam@mail.hu",}
\NormalTok{    ),}
\NormalTok{  ),}
\NormalTok{  date: (}
\NormalTok{    year: 2022,}
\NormalTok{    month: "May",}
\NormalTok{    day: 17,}
\NormalTok{  ),}
\NormalTok{  keywords: (}
\NormalTok{    "Space",}
\NormalTok{    "Mail",}
\NormalTok{    "Astromail",}
\NormalTok{    "Faster{-}than{-}Light",}
\NormalTok{    "Mars",}
\NormalTok{  ),}
\NormalTok{  doi: "10:7891/120948510",}
\NormalTok{  abstract: [}
\NormalTok{    Recent advances in space{-}based document processing have enabled faster mail delivery between different planets of a solar system. Given the time it takes for a message to be transmitted from one planet to the next, its estimated that even a one{-}way trip to a distant destination could take up to one year. During these periods of interplanetary mail delivery there is a slight possibility of mail being lost in transit. This issue is considered so serious that space management employs P.I. agents to track down and retrieve lost mail. We propose A{-}Mail, a new anti{-}matter based approach that can ensure that mail loss occurring during interplanetary transit is unobservable and therefore potentially undetectable. Going even further, we extend A{-}Mail to predict problems and apply existing and new best practices to ensure the mail is delivered without any issues. We call this extension AI{-}Mail.}
\NormalTok{  ]}
\NormalTok{)}
\end{Highlighting}
\end{Shaded}

\href{/app?template=graceful-genetics&version=0.2.0}{Create project in
app}

\subsubsection{How to use}\label{how-to-use}

Click the button above to create a new project using this template in
the Typst app.

You can also use the Typst CLI to start a new project on your computer
using this command:

\begin{verbatim}
typst init @preview/graceful-genetics:0.2.0
\end{verbatim}

\includesvg[width=0.16667in,height=0.16667in]{/assets/icons/16-copy.svg}

\subsubsection{About}\label{about}

\begin{description}
\tightlist
\item[Author :]
James R. Swift
\item[License:]
Unlicense
\item[Current version:]
0.2.0
\item[Last updated:]
October 30, 2024
\item[First released:]
July 16, 2024
\item[Minimum Typst version:]
0.12.0
\item[Archive size:]
24.5 kB
\href{https://packages.typst.org/preview/graceful-genetics-0.2.0.tar.gz}{\pandocbounded{\includesvg[keepaspectratio]{/assets/icons/16-download.svg}}}
\item[Repository:]
\href{https://github.com/JamesxX/graceful-genetics}{GitHub}
\item[Categor y :]
\begin{itemize}
\tightlist
\item[]
\item
  \pandocbounded{\includesvg[keepaspectratio]{/assets/icons/16-atom.svg}}
  \href{https://typst.app/universe/search/?category=paper}{Paper}
\end{itemize}
\end{description}

\subsubsection{Where to report issues?}\label{where-to-report-issues}

This template is a project of James R. Swift . Report issues on
\href{https://github.com/JamesxX/graceful-genetics}{their repository} .
You can also try to ask for help with this template on the
\href{https://forum.typst.app}{Forum} .

Please report this template to the Typst team using the
\href{https://typst.app/contact}{contact form} if you believe it is a
safety hazard or infringes upon your rights.

\phantomsection\label{versions}
\subsubsection{Version history}\label{version-history}

\begin{longtable}[]{@{}ll@{}}
\toprule\noalign{}
Version & Release Date \\
\midrule\noalign{}
\endhead
\bottomrule\noalign{}
\endlastfoot
0.2.0 & October 30, 2024 \\
\href{https://typst.app/universe/package/graceful-genetics/0.1.0/}{0.1.0}
& July 16, 2024 \\
\end{longtable}

Typst GmbH did not create this template and cannot guarantee correct
functionality of this template or compatibility with any version of the
Typst compiler or app.


\section{Package List LaTeX/g-exam.tex}
\title{typst.app/universe/package/g-exam}

\phantomsection\label{banner}
\phantomsection\label{template-thumbnail}
\pandocbounded{\includegraphics[keepaspectratio]{https://packages.typst.org/preview/thumbnails/g-exam-0.4.1-small.webp}}

\section{g-exam}\label{g-exam}

{ 0.4.1 }

Create exams with student information, grade chart, score control,
questions, and sub-questions.

\href{/app?template=g-exam&version=0.4.1}{Create project in app}

\phantomsection\label{readme}
This template provides a way to generate exams. You can create questions
and sub-questions, header with information about the academic center,
score box, subject, exam, header with student information,
clarifications, solutions, watermark with information about the exam
model and teacher.

\paragraph{Features}\label{features}

\begin{itemize}
\tightlist
\item
  Scoreboard.
\item
  Scoring by questions and subquestions.
\item
  Student information, on the first page or on all odd pages.
\item
  Question and subcuestion.
\item
  Show solutions and clarifications
\item
  List of clarifications.
\item
  Teacher’s Watermark
\item
  Exam Model Watermark
\item
  Draft mode
\end{itemize}

\subsection{Usage}\label{usage}

For information, see the
\href{https://matheschool.github.io/typst-g-exam/}{online
docucumentation} .

To use this package, simply add the following code to your document:

\paragraph{A sample exam}\label{a-sample-exam}

\pandocbounded{\includegraphics[keepaspectratio]{https://github.com/typst/packages/raw/main/packages/preview/g-exam/0.4.1/gallery/exam-table-content.png}}

\paragraph{Source:}\label{source}

\begin{Shaded}
\begin{Highlighting}[]
\NormalTok{\#import "@preview/g{-}exam:0.4.1": *}

\NormalTok{\#show: exam.with(}
\NormalTok{  school: (}
\NormalTok{    name: "Sunrise Secondary School",}
\NormalTok{    logo: read("./logo.png", encoding: none),}
\NormalTok{  ),}
\NormalTok{  exam{-}info: (}
\NormalTok{    academic{-}period: "Academic year 2023/2024",}
\NormalTok{    academic{-}level: "1st Secondary Education",}
\NormalTok{    academic{-}subject: "Mathematics",}
\NormalTok{    number: "2nd Assessment 1st Exam",}
\NormalTok{    content: "Radicals and fractions",}
\NormalTok{    model: "Model A"}
\NormalTok{  ),}
  
\NormalTok{  show{-}student{-}data: "first{-}page",}
\NormalTok{  show{-}grade{-}table: true,}
\NormalTok{  clarifications: "Answer the questions in the spaces provided. If you run out of room for an answer, continue on the back of the page."}
\NormalTok{)}
\NormalTok{\#question(points:2.5)[Is it true that $x\^{}n + y\^{}n = z\^{}n$ if $(x,y,z)$ and $n$ are positive integers?. Explain.] }
\NormalTok{\#v(1fr)}

\NormalTok{\#question(points:2.5)[Prove that the real part of all non{-}trivial zeros of the function $zeta(z) "is" 1/2$].}
\NormalTok{\#v(1fr)}

\NormalTok{\#question(points:2)[Compute $ integral\_0\^{}infinity (sin(x))/x $ ]}
\NormalTok{\#v(1fr)}
\end{Highlighting}
\end{Shaded}

\subsection{Changelog}\label{changelog}

\subsubsection{v0.4.1}\label{v0.4.1}

\begin{itemize}
\tightlist
\item
  Fix student data.
\item
  Fix Indenting subquestion.
\end{itemize}

\subsubsection{v0.4.0}\label{v0.4.0}

\begin{itemize}
\tightlist
\item
  Change g-exam for exam.
\item
  Change g-question and g-subquestion for question and subquestion.
\item
  Change point parameter to points in question and subquestion.
\item
  Change question-points-position paramet to question-points-position.
\item
  Include online documentation.
\item
  Use paper by default.
\item
  Indenting subquestion.
\item
  Include support for dutch language.
\item
  Corrections in English texts.
\item
  Draft label.
\end{itemize}

\subsubsection{v0.3.2}\label{v0.3.2}

\begin{itemize}
\tightlist
\item
  Change show-studen-data to show-student-data parameter.
\item
  Change languaje to language parameter.
\end{itemize}

\subsubsection{v0.3.1}\label{v0.3.1}

\begin{itemize}
\tightlist
\item
  Corrections in French.
\end{itemize}

\subsubsection{v0.3.0}\label{v0.3.0}

\begin{itemize}
\tightlist
\item
  Include parameter question-text-parameters.
\item
  Show solution.
\item
  Expand documentation.
\item
  Possibility of estrablecer question-point-position to none.
\item
  Bug fix show watermark.
\end{itemize}

\subsubsection{v0.2.0}\label{v0.2.0}

\begin{itemize}
\tightlist
\item
  Control the size of the logo image.
\item
  Convert to template
\item
  Allow true and false values in show-student-data.
\item
  Show clarifications.
\item
  Widen margin points.
\item
  Show solution.
\end{itemize}

\subsubsection{v0.1.1}\label{v0.1.1}

\begin{itemize}
\tightlist
\item
  Fix loading image.
\end{itemize}

\subsubsection{v0.1.0}\label{v0.1.0}

\begin{itemize}
\tightlist
\item
  Initial version submitted to typst/packages.
\end{itemize}

\href{/app?template=g-exam&version=0.4.1}{Create project in app}

\subsubsection{How to use}\label{how-to-use}

Click the button above to create a new project using this template in
the Typst app.

You can also use the Typst CLI to start a new project on your computer
using this command:

\begin{verbatim}
typst init @preview/g-exam:0.4.1
\end{verbatim}

\includesvg[width=0.16667in,height=0.16667in]{/assets/icons/16-copy.svg}

\subsubsection{About}\label{about}

\begin{description}
\tightlist
\item[Author :]
Andrés Giménez Muñoz
\item[License:]
MIT
\item[Current version:]
0.4.1
\item[Last updated:]
November 19, 2024
\item[First released:]
February 21, 2024
\item[Minimum Typst version:]
0.12.0
\item[Archive size:]
177 kB
\href{https://packages.typst.org/preview/g-exam-0.4.1.tar.gz}{\pandocbounded{\includesvg[keepaspectratio]{/assets/icons/16-download.svg}}}
\item[Repository:]
\href{https://github.com/MatheSchool/typst-g-exam}{GitHub}
\item[Discipline :]
\begin{itemize}
\tightlist
\item[]
\item
  \href{https://typst.app/universe/search/?discipline=education}{Education}
\end{itemize}
\item[Categor y :]
\begin{itemize}
\tightlist
\item[]
\item
  \pandocbounded{\includesvg[keepaspectratio]{/assets/icons/16-envelope.svg}}
  \href{https://typst.app/universe/search/?category=office}{Office}
\end{itemize}
\end{description}

\subsubsection{Where to report issues?}\label{where-to-report-issues}

This template is a project of Andrés Giménez Muñoz . Report issues on
\href{https://github.com/MatheSchool/typst-g-exam}{their repository} .
You can also try to ask for help with this template on the
\href{https://forum.typst.app}{Forum} .

Please report this template to the Typst team using the
\href{https://typst.app/contact}{contact form} if you believe it is a
safety hazard or infringes upon your rights.

\phantomsection\label{versions}
\subsubsection{Version history}\label{version-history}

\begin{longtable}[]{@{}ll@{}}
\toprule\noalign{}
Version & Release Date \\
\midrule\noalign{}
\endhead
\bottomrule\noalign{}
\endlastfoot
0.4.1 & November 19, 2024 \\
\href{https://typst.app/universe/package/g-exam/0.4.0/}{0.4.0} &
November 8, 2024 \\
\href{https://typst.app/universe/package/g-exam/0.3.2/}{0.3.2} & August
26, 2024 \\
\href{https://typst.app/universe/package/g-exam/0.3.1/}{0.3.1} & July
23, 2024 \\
\href{https://typst.app/universe/package/g-exam/0.3.0/}{0.3.0} & April
8, 2024 \\
\href{https://typst.app/universe/package/g-exam/0.2.0/}{0.2.0} & March
21, 2024 \\
\href{https://typst.app/universe/package/g-exam/0.1.1/}{0.1.1} &
February 22, 2024 \\
\href{https://typst.app/universe/package/g-exam/0.1.0/}{0.1.0} &
February 21, 2024 \\
\end{longtable}

Typst GmbH did not create this template and cannot guarantee correct
functionality of this template or compatibility with any version of the
Typst compiler or app.


\section{Package List LaTeX/invoice-maker.tex}
\title{typst.app/universe/package/invoice-maker}

\phantomsection\label{banner}
\phantomsection\label{template-thumbnail}
\pandocbounded{\includegraphics[keepaspectratio]{https://packages.typst.org/preview/thumbnails/invoice-maker-1.1.0-small.webp}}

\section{invoice-maker}\label{invoice-maker}

{ 1.1.0 }

Generate beautiful invoices from a simple data record.

\href{/app?template=invoice-maker&version=1.1.0}{Create project in app}

\phantomsection\label{readme}
Generate beautiful invoices from a simple data record.

\textless img alt=“Example invoice� src=“thumbnail.png�
height=“768�

\begin{quote}
\end{quote}

\subsection{Features}\label{features}

\begin{itemize}
\tightlist
\item
  \textbf{Simple, yet Powerful}

  \begin{itemize}
  \tightlist
  \item
    Write invoices as simple \texttt{\ .typ\ } or \texttt{\ .yaml\ }
    files
  \item
    Support for cancellations, discounts, and taxes
  \end{itemize}
\item
  \textbf{Multilingual}

  \begin{itemize}
  \tightlist
  \item
    Integrated support for English and German
  \item
    Easy to add more languages by adding a translation dictionary (Check
    out this example:
    \href{https://github.com/ad-si/invoice-maker/blob/master/examples/custom-language.typ}{custom-language.typ}
    )
  \end{itemize}
\item
  \textbf{Customizable}

  \begin{itemize}
  \tightlist
  \item
    User your own banner image
  \item
    Customize the colors and fonts
  \end{itemize}
\item
  \textbf{Elegant}

  \begin{itemize}
  \tightlist
  \item
    Modern design with a focus on readability
  \item
    PDFs with a professional look
  \end{itemize}
\item
  \textbf{Stable}

  \begin{itemize}
  \tightlist
  \item
    Visual regression tests to ensure consistent output
  \end{itemize}
\item
  \textbf{Free and Open Source}

  \begin{itemize}
  \tightlist
  \item
    ISC License
  \end{itemize}
\end{itemize}

\subsection{Usage}\label{usage}

\begin{Shaded}
\begin{Highlighting}[]
\NormalTok{\#import "@preview/invoice{-}maker:1.0.0": *}

\NormalTok{\#show: invoice.with(}
\NormalTok{  language: "en", // or "de"}
\NormalTok{  banner\_image: image("banner.png"),}
\NormalTok{  invoice\_id: "2024{-}03{-}10t183205",}
\NormalTok{  …}
\NormalTok{)}
\end{Highlighting}
\end{Shaded}

Check out the \href{https://github.com/ad-si/invoice-maker}{GitHub
repository} for more information and examples.

\href{/app?template=invoice-maker&version=1.1.0}{Create project in app}

\subsubsection{How to use}\label{how-to-use}

Click the button above to create a new project using this template in
the Typst app.

You can also use the Typst CLI to start a new project on your computer
using this command:

\begin{verbatim}
typst init @preview/invoice-maker:1.1.0
\end{verbatim}

\includesvg[width=0.16667in,height=0.16667in]{/assets/icons/16-copy.svg}

\subsubsection{About}\label{about}

\begin{description}
\tightlist
\item[Author :]
\href{https://github.com/ad-si}{Adrian Sieber}
\item[License:]
ISC
\item[Current version:]
1.1.0
\item[Last updated:]
March 28, 2024
\item[First released:]
March 28, 2024
\item[Archive size:]
14.1 kB
\href{https://packages.typst.org/preview/invoice-maker-1.1.0.tar.gz}{\pandocbounded{\includesvg[keepaspectratio]{/assets/icons/16-download.svg}}}
\item[Repository:]
\href{https://github.com/ad-si/invoice-maker}{GitHub}
\item[Discipline :]
\begin{itemize}
\tightlist
\item[]
\item
  \href{https://typst.app/universe/search/?discipline=business}{Business}
\end{itemize}
\item[Categor y :]
\begin{itemize}
\tightlist
\item[]
\item
  \pandocbounded{\includesvg[keepaspectratio]{/assets/icons/16-envelope.svg}}
  \href{https://typst.app/universe/search/?category=office}{Office}
\end{itemize}
\end{description}

\subsubsection{Where to report issues?}\label{where-to-report-issues}

This template is a project of Adrian Sieber . Report issues on
\href{https://github.com/ad-si/invoice-maker}{their repository} . You
can also try to ask for help with this template on the
\href{https://forum.typst.app}{Forum} .

Please report this template to the Typst team using the
\href{https://typst.app/contact}{contact form} if you believe it is a
safety hazard or infringes upon your rights.

\phantomsection\label{versions}
\subsubsection{Version history}\label{version-history}

\begin{longtable}[]{@{}ll@{}}
\toprule\noalign{}
Version & Release Date \\
\midrule\noalign{}
\endhead
\bottomrule\noalign{}
\endlastfoot
1.1.0 & March 28, 2024 \\
\end{longtable}

Typst GmbH did not create this template and cannot guarantee correct
functionality of this template or compatibility with any version of the
Typst compiler or app.


\section{Package List LaTeX/based.tex}
\title{typst.app/universe/package/based}

\phantomsection\label{banner}
\section{based}\label{based}

{ 0.2.0 }

Encoder and decoder for base64, base32, and base16.

\phantomsection\label{readme}
A package for encoding and decoding in base64, base32, and base16.

\subsection{Usage}\label{usage}

The package comes with three submodules: \texttt{\ base64\ } ,
\texttt{\ base32\ } , and \texttt{\ base16\ } . All of them have an
\texttt{\ encode\ } and \texttt{\ decode\ } function. The package also
provides the function aliases

\begin{itemize}
\tightlist
\item
  \texttt{\ encode64\ } / \texttt{\ decode64\ } ,
\item
  \texttt{\ encode32\ } / \texttt{\ decode32\ } , and
\item
  \texttt{\ encode16\ } / \texttt{\ decode16\ } .
\end{itemize}

Both base64 and base32 allow you to choose whether to use padding for
encoding via the \texttt{\ pad\ } parameter, which is enabled by
default. Base64 also allows you to encode with the URL-safe alphabet by
enabling the \texttt{\ url\ } parameter, while base32 allows you to
encode or decode with the “extended hex� alphabet via the
\texttt{\ hex\ } parameter. Both options are disabled by default. The
base16 encoder uses lowercase letters, the decoder is case-insensitive.

You can encode strings, arrays and bytes. The \texttt{\ encode\ }
function will return a string, while the \texttt{\ decode\ } function
will return bytes.

\subsection{Example}\label{example}

\begin{Shaded}
\begin{Highlighting}[]
\NormalTok{\#import "@preview/based:0.2.0": base64, base32, base16}

\NormalTok{\#table(}
\NormalTok{  columns: 3,}
  
\NormalTok{  table.header[*Base64*][*Base32*][*Base16*],}

\NormalTok{  raw(base64.encode("Hello world!")),}
\NormalTok{  raw(base32.encode("Hello world!")),}
\NormalTok{  raw(base16.encode("Hello world!")),}

\NormalTok{  str(base64.decode("SGVsbG8gd29ybGQh")),}
\NormalTok{  str(base32.decode("JBSWY3DPEB3W64TMMQQQ====")),}
\NormalTok{  str(base16.decode("48656C6C6F20776F726C6421"))}
\NormalTok{)}
\end{Highlighting}
\end{Shaded}

\pandocbounded{\includesvg[keepaspectratio]{https://github.com/typst/packages/raw/main/packages/preview/based/0.2.0/assets/example.svg}}

\subsubsection{How to add}\label{how-to-add}

Copy this into your project and use the import as \texttt{\ based\ }

\begin{verbatim}
#import "@preview/based:0.2.0"
\end{verbatim}

\includesvg[width=0.16667in,height=0.16667in]{/assets/icons/16-copy.svg}

Check the docs for
\href{https://typst.app/docs/reference/scripting/\#packages}{more
information on how to import packages} .

\subsubsection{About}\label{about}

\begin{description}
\tightlist
\item[Author :]
Eric Biedert
\item[License:]
MIT
\item[Current version:]
0.2.0
\item[Last updated:]
November 6, 2024
\item[First released:]
July 5, 2024
\item[Archive size:]
21.1 kB
\href{https://packages.typst.org/preview/based-0.2.0.tar.gz}{\pandocbounded{\includesvg[keepaspectratio]{/assets/icons/16-download.svg}}}
\item[Repository:]
\href{https://github.com/EpicEricEE/typst-based}{GitHub}
\item[Categor y :]
\begin{itemize}
\tightlist
\item[]
\item
  \pandocbounded{\includesvg[keepaspectratio]{/assets/icons/16-code.svg}}
  \href{https://typst.app/universe/search/?category=scripting}{Scripting}
\end{itemize}
\end{description}

\subsubsection{Where to report issues?}\label{where-to-report-issues}

This package is a project of Eric Biedert . Report issues on
\href{https://github.com/EpicEricEE/typst-based}{their repository} . You
can also try to ask for help with this package on the
\href{https://forum.typst.app}{Forum} .

Please report this package to the Typst team using the
\href{https://typst.app/contact}{contact form} if you believe it is a
safety hazard or infringes upon your rights.

\phantomsection\label{versions}
\subsubsection{Version history}\label{version-history}

\begin{longtable}[]{@{}ll@{}}
\toprule\noalign{}
Version & Release Date \\
\midrule\noalign{}
\endhead
\bottomrule\noalign{}
\endlastfoot
0.2.0 & November 6, 2024 \\
\href{https://typst.app/universe/package/based/0.1.0/}{0.1.0} & July 5,
2024 \\
\end{longtable}

Typst GmbH did not create this package and cannot guarantee correct
functionality of this package or compatibility with any version of the
Typst compiler or app.


\section{Package List LaTeX/cetz-venn.tex}
\title{typst.app/universe/package/cetz-venn}

\phantomsection\label{banner}
\section{cetz-venn}\label{cetz-venn}

{ 0.1.2 }

CeTZ library for drawing venn diagrams for two or three sets.

\phantomsection\label{readme}
A \href{https://github.com/cetz-package/cetz}{CeTZ} library for drawing
simple two- or three-set Venn diagrams.

\subsection{Examples}\label{examples}

\begin{longtable}[]{@{}ll@{}}
\toprule\noalign{}
\endhead
\bottomrule\noalign{}
\endlastfoot
\href{https://github.com/typst/packages/raw/main/packages/preview/cetz-venn/0.1.2/gallery/venn2.typ}{\includegraphics[width=2.60417in,height=\textheight,keepaspectratio]{https://github.com/typst/packages/raw/main/packages/preview/cetz-venn/0.1.2/gallery/venn2.png}}
&
\href{https://github.com/typst/packages/raw/main/packages/preview/cetz-venn/0.1.2/gallery/venn3.typ}{\includegraphics[width=2.60417in,height=\textheight,keepaspectratio]{https://github.com/typst/packages/raw/main/packages/preview/cetz-venn/0.1.2/gallery/venn3.png}} \\
Two set Venn diagram & Three set Venn diagram \\
\end{longtable}

\emph{Click on the example image to jump to the code.}

\subsection{Usage}\label{usage}

This package requires CeTZ version \textgreater= 0.3.1!

For information, see the
\href{https://github.com/cetz-package/cetz-venn/blob/stable/manual.pdf?raw=true}{manual
(stable)} .

To use this package, simply add the following code to your document:

\begin{verbatim}
#import "@preview/cetz:0.3.1"
#import "@preview/cetz-venn:0.1.1"

#cetz.canvas({
  cetz-venn.venn2()
})
\end{verbatim}

\subsubsection{How to add}\label{how-to-add}

Copy this into your project and use the import as \texttt{\ cetz-venn\ }

\begin{verbatim}
#import "@preview/cetz-venn:0.1.2"
\end{verbatim}

\includesvg[width=0.16667in,height=0.16667in]{/assets/icons/16-copy.svg}

Check the docs for
\href{https://typst.app/docs/reference/scripting/\#packages}{more
information on how to import packages} .

\subsubsection{About}\label{about}

\begin{description}
\tightlist
\item[Author :]
\href{https://github.com/johannes-wolf}{Johannes Wolf}
\item[License:]
Apache-2.0
\item[Current version:]
0.1.2
\item[Last updated:]
October 28, 2024
\item[First released:]
July 1, 2024
\item[Archive size:]
6.34 kB
\href{https://packages.typst.org/preview/cetz-venn-0.1.2.tar.gz}{\pandocbounded{\includesvg[keepaspectratio]{/assets/icons/16-download.svg}}}
\item[Repository:]
\href{https://github.com/johannes-wolf/cetz-venn}{GitHub}
\end{description}

\subsubsection{Where to report issues?}\label{where-to-report-issues}

This package is a project of Johannes Wolf . Report issues on
\href{https://github.com/johannes-wolf/cetz-venn}{their repository} .
You can also try to ask for help with this package on the
\href{https://forum.typst.app}{Forum} .

Please report this package to the Typst team using the
\href{https://typst.app/contact}{contact form} if you believe it is a
safety hazard or infringes upon your rights.

\phantomsection\label{versions}
\subsubsection{Version history}\label{version-history}

\begin{longtable}[]{@{}ll@{}}
\toprule\noalign{}
Version & Release Date \\
\midrule\noalign{}
\endhead
\bottomrule\noalign{}
\endlastfoot
0.1.2 & October 28, 2024 \\
\href{https://typst.app/universe/package/cetz-venn/0.1.1/}{0.1.1} & July
19, 2024 \\
\href{https://typst.app/universe/package/cetz-venn/0.1.0/}{0.1.0} & July
1, 2024 \\
\end{longtable}

Typst GmbH did not create this package and cannot guarantee correct
functionality of this package or compatibility with any version of the
Typst compiler or app.


\section{Package List LaTeX/minideck.tex}
\title{typst.app/universe/package/minideck}

\phantomsection\label{banner}
\section{minideck}\label{minideck}

{ 0.2.1 }

Simple dynamic slides.

\phantomsection\label{readme}
A small package for dynamic slides in typst.

minideck provides basic functionality for dynamic slides (
\texttt{\ pause\ } , \texttt{\ uncover\ } , …), integration with
\href{https://typst.app/universe/package/fletcher}{fletcher} and
\href{https://typst.app/universe/package/cetz/}{CetZ} , and some minimal
infrastructure for theming.

\subsection{Usage}\label{usage}

Call \texttt{\ minideck.config\ } to get the functions you want to use:

\begin{Shaded}
\begin{Highlighting}[]
\NormalTok{\#import "@preview/minideck:0.2.1"}

\NormalTok{\#let (template, slide, title{-}slide, pause, uncover, only) = minideck.config()}
\NormalTok{\#show: template}

\NormalTok{\#title{-}slide[}
\NormalTok{  = Slides with \textasciigrave{}minideck\textasciigrave{}}
\NormalTok{  == Some examples}
\NormalTok{  John Doe}

\NormalTok{  \#datetime.today().display()}
\NormalTok{]}

\NormalTok{\#slide[}
\NormalTok{  = Some title}

\NormalTok{  Some content}

\NormalTok{  \#show: pause}

\NormalTok{  More content}

\NormalTok{  1. One}
\NormalTok{  2. Two}
\NormalTok{  \#show: pause}
\NormalTok{  3. Three}
\NormalTok{]}
\end{Highlighting}
\end{Shaded}

This will show three subslides with progressively more content. (Note
that the default theme uses the font Libertinus Sans from the
\href{https://github.com/alerque/libertinus}{Libertinus} family, so you
may want to install it.)

Instead of \texttt{\ \#show:\ pause\ } , you can use
\texttt{\ \#uncover(2,3){[}...{]}\ } to make content visible only on
subslides 2 and 3, or \texttt{\ \#uncover(from:\ 2){[}...{]}\ } to have
it visible on subslide 2 and following.

The \texttt{\ only\ } function is similar to \texttt{\ uncover\ } , but
instead of hiding the content (without affecting the layout), it removes
it.

\begin{Shaded}
\begin{Highlighting}[]
\NormalTok{\#slide[}
\NormalTok{  = \textasciigrave{}uncover\textasciigrave{} and \textasciigrave{}only\textasciigrave{}}
  
\NormalTok{  \#uncover(1, from:3)[}
\NormalTok{    Content visible on subslides 1 and 3+}
\NormalTok{    (space reserved on 2).}
\NormalTok{  ]}

\NormalTok{  \#only(2,3)[}
\NormalTok{    Content included on subslides 2 and 3}
\NormalTok{    (no space reserved on 1).}
\NormalTok{  ]}

\NormalTok{  Content always visible.}
\NormalTok{]}
\end{Highlighting}
\end{Shaded}

Contrary to \texttt{\ pause\ } , the \texttt{\ uncover\ } and
\texttt{\ only\ } functions also work in math mode:

\begin{Shaded}
\begin{Highlighting}[]
\NormalTok{\#slide[}
\NormalTok{  = Dynamic equation}

\NormalTok{  $}
\NormalTok{    f(x) \&= x\^{}2 + 2x + 1  \textbackslash{}}
\NormalTok{         \#uncover(2, $\&= (x + 1)\^{}2$)}
\NormalTok{  $}
\NormalTok{]}
\end{Highlighting}
\end{Shaded}

When mixing \texttt{\ pause\ } with \texttt{\ uncover\ } /
\texttt{\ only\ } , the sequence of pauses should be taken as reference
for the meaning of subslide indices. For example content after the first
pause always appears on the second subslide, even if it’s preceded by
a call to \texttt{\ \#uncover(from:\ 3){[}...{]}\ } .

The package also works well with
\href{https://typst.app/universe/package/pinit}{pinit} :

\begin{Shaded}
\begin{Highlighting}[]
\NormalTok{\#import "@preview/pinit:0.1.4": *}

\NormalTok{\#slide[}
\NormalTok{  = Works well with \textasciigrave{}pinit\textasciigrave{}}

\NormalTok{  Pythagorean theorem:}

\NormalTok{  $ \#pin(1)a\^{}2\#pin(2) + \#pin(3)b\^{}2\#pin(4) = \#pin(5)c\^{}2\#pin(6) $}

\NormalTok{  \#show: pause}

\NormalTok{  $a\^{}2$ and $b\^{}2$ : squares of triangle legs}

\NormalTok{  \#only(2, \{}
\NormalTok{    pinit{-}highlight(1,2)}
\NormalTok{    pinit{-}highlight(3,4)}
\NormalTok{  \})}

\NormalTok{  \#show: pause}

\NormalTok{  $c\^{}2$ : square of hypotenuse}

\NormalTok{  \#pinit{-}highlight(5,6, fill: green.transparentize(80\%))}
\NormalTok{  \#pinit{-}point{-}from(6)[larger than $a\^{}2$ and $b\^{}2$]}
\NormalTok{]}
\end{Highlighting}
\end{Shaded}

\subsubsection{Handout mode}\label{handout-mode}

minideck can make a handout version of the document, in which dynamic
behavior is disabled: the content of all subslides is shown together in
a single slide.

To compile a handout version, pass \texttt{\ -\/-input\ handout=true\ }
in the command line:

\begin{Shaded}
\begin{Highlighting}[]
\ExtensionTok{typst}\NormalTok{ compile }\AttributeTok{{-}{-}input}\NormalTok{ handout=true myfile.typ}
\end{Highlighting}
\end{Shaded}

It is also possible to enable handout mode from within the document, as
shown in the next section.

\subsubsection{Configuration}\label{configuration}

The behavior of the minideck functions depends on the settings passed to
\texttt{\ minideck.config\ } . For example, handout mode can also be
enabled like this:

\begin{Shaded}
\begin{Highlighting}[]
\NormalTok{\#import "@preview/minideck:0.2.1"}

\NormalTok{\#let (template, slide, pause) = minideck.config(handout: true)}
\NormalTok{\#show: template}

\NormalTok{\#slide[}
\NormalTok{  = Slide title}
  
\NormalTok{  Some text}

\NormalTok{  \#show: pause}

\NormalTok{  More text}
\NormalTok{]}
\end{Highlighting}
\end{Shaded}

(The default value of \texttt{\ handout\ } is \texttt{\ auto\ } , in
which case minideck checks for a command line setting as in the previous
section.)

\texttt{\ minideck.config\ } accepts the following named arguments:

\begin{itemize}
\tightlist
\item
  \texttt{\ paper\ } : a string for one of the paper size names
  recognized by
  \href{https://typst.app/docs/reference/layout/page/\#parameters-paper}{\texttt{\ page.paper\ }}
  , or one of the shorthands \texttt{\ "16:9"\ } or \texttt{\ "4:3"\ } .
  Default: \texttt{\ "4:3"\ } .
\item
  \texttt{\ landscape\ } : use the paper size in landscape orientation.
  Default: \texttt{\ true\ } .
\item
  \texttt{\ width\ } : page width as an absolute length. Takes
  precedence over \texttt{\ paper\ } and \texttt{\ landscape\ } .
\item
  \texttt{\ height\ } : page height as an absolute length. Takes
  precedence over \texttt{\ paper\ } and \texttt{\ landscape\ } .
\item
  \texttt{\ handout\ } : whether to make a document for handout, with
  content of all subslides shown together in a single slide.
\item
  \texttt{\ theme\ } : the theme (see below).
\item
  \texttt{\ cetz\ } : the CeTZ module (see below).
\item
  \texttt{\ fletcher\ } : the fletcher module (see below).
\end{itemize}

For example to make slides with 16:9 aspect ratio, use
\texttt{\ minideck.config(paper:\ "16:9")\ } .

\subsubsection{Themes}\label{themes}

Use \texttt{\ minideck.config(theme:\ ...)\ } to select a theme.
Currently there is only one built-in:
\texttt{\ minideck.themes.simple\ } . However you can also pass a theme
implemented by yourself or from a third-party package. See the
\href{https://github.com/typst/packages/raw/main/packages/preview/minideck/0.2.1/themes/README.md}{theme
documentation} for how that works.

Themes are functions and can be configured using the standard
\href{https://typst.app/docs/reference/foundations/function/\#definitions-with}{\texttt{\ with\ }
method} :

\begin{itemize}
\tightlist
\item
  The \texttt{\ simple\ } theme has a \texttt{\ variant\ } setting with
  values “light� (default) and “dark�.
\end{itemize}

Here’s an example:

\begin{Shaded}
\begin{Highlighting}[]
\NormalTok{\#import "@preview/minideck:0.2.1"}

\NormalTok{\#let (template, slide) = minideck.config(}
\NormalTok{  theme: minideck.themes.simple.with(variant: "dark"),}
\NormalTok{)}
\NormalTok{\#show: template}

\NormalTok{\#slide[}
\NormalTok{  = Slide with dark theme}
  
\NormalTok{  Some text}
\NormalTok{]}
\end{Highlighting}
\end{Shaded}

Note that you can override part of a theme with show and set rules:

\begin{Shaded}
\begin{Highlighting}[]
\NormalTok{\#import "@preview/minideck:0.2.1"}

\NormalTok{\#let (template, slide) = minideck.config(}
\NormalTok{  theme: minideck.themes.simple.with(variant: "dark"),}
\NormalTok{)}
\NormalTok{\#show: template}

\NormalTok{\#set page(footer: none) // get rid of the page number}
\NormalTok{\#show heading: it =\textgreater{} text(style: "italic", it)}
\NormalTok{\#set text(red)}

\NormalTok{\#slide[}
\NormalTok{  = Slide with dark theme and red text}
  
\NormalTok{  Some text}
\NormalTok{]}
\end{Highlighting}
\end{Shaded}

\subsubsection{Integration with CeTZ}\label{integration-with-cetz}

You can use \texttt{\ uncover\ } and \texttt{\ only\ } (but not
\texttt{\ pause\ } ) in CeTZ figures, with the following extra steps:

\begin{itemize}
\item
  Get CeTZ-specific functions \texttt{\ cetz-uncover\ } and
  \texttt{\ cetz-only\ } by passing the CeTZ module to
  \texttt{\ minideck.config\ } (see example below).

  This ensures that minideck uses CeTZ functions from the correct
  version of CeTZ.
\item
  Add a \texttt{\ context\ } keyword outside the \texttt{\ canvas\ }
  call.

  This is required to access the minideck subslide state from within the
  canvas without making the content opaque (CeTZ needs to inspect the
  canvas content to make the drawing).
\end{itemize}

Example:

\begin{Shaded}
\begin{Highlighting}[]
\NormalTok{\#import "@preview/cetz:0.2.2" as cetz: *}
\NormalTok{\#import "@preview/minideck:0.2.1"}

\NormalTok{\#let (template, slide, only, cetz{-}uncover, cetz{-}only) = minideck.config(cetz: cetz)}
\NormalTok{\#show: template}

\NormalTok{\#slide[}
\NormalTok{  = In a CeTZ figure}

\NormalTok{  Above canvas}
\NormalTok{  \#context canvas(\{}
\NormalTok{    import draw: *}
\NormalTok{    cetz{-}only(3, rect((0,{-}2), (14,4), stroke: 3pt))}
\NormalTok{    cetz{-}uncover(from: 2, rect((0,{-}2), (16,2), stroke: blue+3pt))}
\NormalTok{    content((8,0), box(stroke: red+3pt, inset: 1em)[}
\NormalTok{      A typst box \#only(2)[on 2nd subslide]}
\NormalTok{    ])}
\NormalTok{  \})}
\NormalTok{  Below canvas}
\NormalTok{]}
\end{Highlighting}
\end{Shaded}

\subsubsection{Integration with
fletcher}\label{integration-with-fletcher}

The same steps are required as for CeTZ integration (passing the
fletcher module to get fletcher-specific functions), plus an additional
step:

\begin{itemize}
\item
  Give explicitly the number of subslides to the \texttt{\ slide\ }
  function.

  This is required because I could not find a reliable way to update a
  typst state from within a fletcher diagram, so I cannot rely on the
  state to keep track of the number of subslides.
\end{itemize}

Example:

\begin{Shaded}
\begin{Highlighting}[]
\NormalTok{\#import "@preview/fletcher:0.5.0" as fletcher: diagram, node, edge}
\NormalTok{\#import "@preview/minideck:0.2.1"}
\NormalTok{\#let (template, slide, fletcher{-}uncover) = minideck.config(fletcher: fletcher)}
\NormalTok{\#show: template}

\NormalTok{\#slide(steps: 2)[}
\NormalTok{  = In a fletcher diagram}

\NormalTok{  \#set align(center)}

\NormalTok{  Above diagram}

\NormalTok{  \#context diagram(}
\NormalTok{    node{-}stroke: 1pt,}
\NormalTok{    node((0,0), [Start], corner{-}radius: 2pt, extrude: (0, 3)),}
\NormalTok{    edge("{-}|\textgreater{}"),}
\NormalTok{    node((1,0), align(center)[A]),}
\NormalTok{    fletcher{-}uncover(from: 2, edge("d,r,u,l", "{-}|\textgreater{}", [x], label{-}pos: 0.1))}
\NormalTok{  )}
  
\NormalTok{  Below diagram}
\NormalTok{]}
\end{Highlighting}
\end{Shaded}

\subsection{Comparison with other slides
packages}\label{comparison-with-other-slides-packages}

Performance: minideck is currently faster than
\href{https://typst.app/universe/package/polylux/}{Polylux} when using
\texttt{\ pause\ } , especially for incremental compilation, but a bit
slower than \href{https://typst.app/universe/package/touying}{Touying} ,
according to my tests.

Features: Polylux and Touying have more themes and more features, for
example support for \href{https://pdfpc.github.io/}{pdfpc} which
provides speaker notes and more. Minideck allows using
\texttt{\ uncover\ } and \texttt{\ only\ } in CeTZ figures and fletcher
diagrams, which Polylux currently doesn’t support.

Syntax: package configuration is simpler in minideck than Touying but a
bit more involved than in Polylux. The minideck \texttt{\ pause\ } is
more cumbersome to use: one must write \texttt{\ \#show:\ pause\ }
instead of \texttt{\ \#pause\ } . On the other hand minideck’s
\texttt{\ uncover\ } and \texttt{\ only\ } can be used directly in
equations without requiring a special math environment as in Touying (I
think).

Other: minideck sometimes has better error messages than Touying due to
implementation differences: the minideck stack trace points back to the
user’s code while Touying errors sometimes point only to an output
page number.

\subsection{Limitations}\label{limitations}

\begin{itemize}
\tightlist
\item
  \texttt{\ pause\ } , \texttt{\ uncover\ } and \texttt{\ only\ } work
  in enumerations but they require explicit enum indices (
  \texttt{\ 1.\ ...\ } rather than \texttt{\ +\ ...\ } ) as they cause a
  reset of the list index.
\item
  Usage in a CeTZ canvas or fletcher diagram requires a
  \texttt{\ context\ } keyword in front of the \texttt{\ canvas\ } /
  \texttt{\ diagram\ } call (see above).
\item
  fletcher diagrams also require to specify the number of subslides
  explicitly (see above).
\item
  \texttt{\ pause\ } doesn’t work in CeTZ figures, fletcher diagrams
  and math mode.
\item
  \texttt{\ pause\ } requires writing \texttt{\ \#show:\ pause\ } and
  its effect is lost after the \texttt{\ \#show\ } call goes out of
  scope. For example this means that one can use \texttt{\ pause\ }
  inside of a grid, but further \texttt{\ pause\ } calls after the grid
  (in the same slide) won’t work as intended.
\end{itemize}

\subsection{Internals}\label{internals}

The package uses states with the following keys:

\texttt{\ \_\_minideck-subslide-count\ } : an array of two integers for
counting pauses and keeping track of the subslide number automatically.
The first value is the number of subslides so far referenced in current
slide. The second value is the number of pauses seen so far in the
current slide. Both values are kept in one state so that an update
function can update the number of subslides based on the number of
pauses, without requiring a context. This avoids problems with layout
convergence.

\texttt{\ \_\_minideck-subslide-step\ } : the current subslide being
generated while processing a slide.

\subsubsection{How to add}\label{how-to-add}

Copy this into your project and use the import as \texttt{\ minideck\ }

\begin{verbatim}
#import "@preview/minideck:0.2.1"
\end{verbatim}

\includesvg[width=0.16667in,height=0.16667in]{/assets/icons/16-copy.svg}

Check the docs for
\href{https://typst.app/docs/reference/scripting/\#packages}{more
information on how to import packages} .

\subsubsection{About}\label{about}

\begin{description}
\tightlist
\item[Author :]
\href{https://github.com/knuesel}{Jeremie Knuesel}
\item[License:]
MIT
\item[Current version:]
0.2.1
\item[Last updated:]
July 1, 2024
\item[First released:]
July 1, 2024
\item[Archive size:]
10.3 kB
\href{https://packages.typst.org/preview/minideck-0.2.1.tar.gz}{\pandocbounded{\includesvg[keepaspectratio]{/assets/icons/16-download.svg}}}
\item[Repository:]
\href{https://github.com/knuesel/typst-minideck}{GitHub}
\item[Categor y :]
\begin{itemize}
\tightlist
\item[]
\item
  \pandocbounded{\includesvg[keepaspectratio]{/assets/icons/16-presentation.svg}}
  \href{https://typst.app/universe/search/?category=presentation}{Presentation}
\end{itemize}
\end{description}

\subsubsection{Where to report issues?}\label{where-to-report-issues}

This package is a project of Jeremie Knuesel . Report issues on
\href{https://github.com/knuesel/typst-minideck}{their repository} . You
can also try to ask for help with this package on the
\href{https://forum.typst.app}{Forum} .

Please report this package to the Typst team using the
\href{https://typst.app/contact}{contact form} if you believe it is a
safety hazard or infringes upon your rights.

\phantomsection\label{versions}
\subsubsection{Version history}\label{version-history}

\begin{longtable}[]{@{}ll@{}}
\toprule\noalign{}
Version & Release Date \\
\midrule\noalign{}
\endhead
\bottomrule\noalign{}
\endlastfoot
0.2.1 & July 1, 2024 \\
\end{longtable}

Typst GmbH did not create this package and cannot guarantee correct
functionality of this package or compatibility with any version of the
Typst compiler or app.


\section{Package List LaTeX/ccicons.tex}
\title{typst.app/universe/package/ccicons}

\phantomsection\label{banner}
\section{ccicons}\label{ccicons}

{ 1.0.0 }

A port of the ccicon LaTeX package for Typst.

\phantomsection\label{readme}
Creative Commons icons for your Typst documents

\begin{center}\rule{0.5\linewidth}{0.5pt}\end{center}

\begin{quote}
{[}!NOTE{]} \texttt{\ ccicons\ } is an adaption of the
\href{https://ctan.org/pkg/ccicons}{ccicons package} for LaTeX by
\href{https://github.com/ummels}{Michael Ummels} .
\end{quote}

\subsection{Getting Started}\label{getting-started}

Import the package into your document:

\begin{Shaded}
\begin{Highlighting}[]
\NormalTok{\#import "@preview/ccicons:1.0.0": *}
\end{Highlighting}
\end{Shaded}

Start using license icons:

\begin{Shaded}
\begin{Highlighting}[]
\NormalTok{\#cc{-}by{-}nc{-}sa}
\end{Highlighting}
\end{Shaded}

See the
\href{https://github.com/typst/packages/raw/main/packages/preview/ccicons/1.0.0/docs/ccicons-manual.pdf}{the
manual} for more details and an overview all available Creative Commons
icons.

Please note that all icons that can be typeset using this package are
trademarks of Creative Commons and are subject to the Creative Commons
trademark policy (see \url{http://creativecommons.org/policies} ).

The symbols in this font have been obtained from
\url{https://creativecommons.org/mission/downloads/} and released by
Creative Commons under a Creative Commons Attribution 4.0 International
license: \url{https://creativecommons.org/licenses/by/4.0/}

\subsubsection{How to add}\label{how-to-add}

Copy this into your project and use the import as \texttt{\ ccicons\ }

\begin{verbatim}
#import "@preview/ccicons:1.0.0"
\end{verbatim}

\includesvg[width=0.16667in,height=0.16667in]{/assets/icons/16-copy.svg}

Check the docs for
\href{https://typst.app/docs/reference/scripting/\#packages}{more
information on how to import packages} .

\subsubsection{About}\label{about}

\begin{description}
\tightlist
\item[Author :]
J. Neugebauer
\item[License:]
MIT
\item[Current version:]
1.0.0
\item[Last updated:]
June 17, 2024
\item[First released:]
June 17, 2024
\item[Archive size:]
197 kB
\href{https://packages.typst.org/preview/ccicons-1.0.0.tar.gz}{\pandocbounded{\includesvg[keepaspectratio]{/assets/icons/16-download.svg}}}
\item[Repository:]
\href{https://github.com/jneug/typst-ccicons}{GitHub}
\item[Categor y :]
\begin{itemize}
\tightlist
\item[]
\item
  \pandocbounded{\includesvg[keepaspectratio]{/assets/icons/16-package.svg}}
  \href{https://typst.app/universe/search/?category=components}{Components}
\end{itemize}
\end{description}

\subsubsection{Where to report issues?}\label{where-to-report-issues}

This package is a project of J. Neugebauer . Report issues on
\href{https://github.com/jneug/typst-ccicons}{their repository} . You
can also try to ask for help with this package on the
\href{https://forum.typst.app}{Forum} .

Please report this package to the Typst team using the
\href{https://typst.app/contact}{contact form} if you believe it is a
safety hazard or infringes upon your rights.

\phantomsection\label{versions}
\subsubsection{Version history}\label{version-history}

\begin{longtable}[]{@{}ll@{}}
\toprule\noalign{}
Version & Release Date \\
\midrule\noalign{}
\endhead
\bottomrule\noalign{}
\endlastfoot
1.0.0 & June 17, 2024 \\
\end{longtable}

Typst GmbH did not create this package and cannot guarantee correct
functionality of this package or compatibility with any version of the
Typst compiler or app.


\section{Package List LaTeX/unequivocal-ams.tex}
\title{typst.app/universe/package/unequivocal-ams}

\phantomsection\label{banner}
\phantomsection\label{template-thumbnail}
\pandocbounded{\includegraphics[keepaspectratio]{https://packages.typst.org/preview/thumbnails/unequivocal-ams-0.1.2-small.webp}}

\section{unequivocal-ams}\label{unequivocal-ams}

{ 0.1.2 }

An AMS-style paper template to publish at conferences and journals for
mathematicians

{ } Featured Template

\href{/app?template=unequivocal-ams&version=0.1.2}{Create project in
app}

\phantomsection\label{readme}
A single-column paper for the American Mathematical Society. The
template comes with functions for theorems and proofs. It also is a nice
starting point for a classy tech report or thesis.

\subsection{Usage}\label{usage}

You can use this template in the Typst web app by clicking “Start from
template� on the dashboard and searching for
\texttt{\ unequivocal-ams\ } .

Alternatively, you can use the CLI to kick this project off using the
command

\begin{verbatim}
typst init @preview/unequivocal-ams
\end{verbatim}

Typst will create a new directory with all the files needed to get you
started.

\subsection{Configuration}\label{configuration}

This template exports the \texttt{\ ams-article\ } function with the
following named arguments:

\begin{itemize}
\tightlist
\item
  \texttt{\ title\ } : The paper’s title as content.
\item
  \texttt{\ authors\ } : An array of author dictionaries. Each of the
  author dictionaries must have a \texttt{\ name\ } key and can have the
  keys \texttt{\ department\ } , \texttt{\ organization\ } ,
  \texttt{\ location\ } , and \texttt{\ email\ } . All keys accept
  content.
\item
  \texttt{\ abstract\ } : The content of a brief summary of the paper or
  \texttt{\ none\ } . Appears at the top of the first column in
  boldface.
\item
  \texttt{\ paper-size\ } : Defaults to \texttt{\ us-letter\ } . Specify
  a
  \href{https://typst.app/docs/reference/layout/page/\#parameters-paper}{paper
  size string} to change the page format.
\item
  \texttt{\ bibliography\ } : The result of a call to the
  \texttt{\ bibliography\ } function or \texttt{\ none\ } . Specifying
  this will configure numeric, Springer MathPhys-style citations.
\end{itemize}

The function also accepts a single, positional argument for the body of
the paper.

The template will initialize your package with a sample call to the
\texttt{\ ams-article\ } function in a show rule. If you, however, want
to change an existing project to use this template, you can add a show
rule like this at the top of your file:

\begin{Shaded}
\begin{Highlighting}[]
\NormalTok{\#import "@preview/unequivocal{-}ams:0.1.2": ams{-}article, theorem, proof}

\NormalTok{\#show: ams{-}article.with(}
\NormalTok{  title: [Mathematical Theorems],}
\NormalTok{  authors: (}
\NormalTok{    (}
\NormalTok{      name: "Ralph Howard",}
\NormalTok{      department: [Department of Mathematics],}
\NormalTok{      organization: [University of South Carolina],}
\NormalTok{      location: [Columbia, SC 29208],}
\NormalTok{      email: "howard@math.sc.edu",}
\NormalTok{      url: "www.math.sc.edu/\textasciitilde{}howard"}
\NormalTok{    ),}
\NormalTok{  ),}
\NormalTok{  abstract: lorem(100),}
\NormalTok{  bibliography: bibliography("refs.bib"),}
\NormalTok{)}

\NormalTok{// Your content goes below.}
\end{Highlighting}
\end{Shaded}

\href{/app?template=unequivocal-ams&version=0.1.2}{Create project in
app}

\subsubsection{How to use}\label{how-to-use}

Click the button above to create a new project using this template in
the Typst app.

You can also use the Typst CLI to start a new project on your computer
using this command:

\begin{verbatim}
typst init @preview/unequivocal-ams:0.1.2
\end{verbatim}

\includesvg[width=0.16667in,height=0.16667in]{/assets/icons/16-copy.svg}

\subsubsection{About}\label{about}

\begin{description}
\tightlist
\item[Author :]
\href{https://typst.app}{Typst GmbH}
\item[License:]
MIT-0
\item[Current version:]
0.1.2
\item[Last updated:]
October 29, 2024
\item[First released:]
March 6, 2024
\item[Minimum Typst version:]
0.12.0
\item[Archive size:]
6.30 kB
\href{https://packages.typst.org/preview/unequivocal-ams-0.1.2.tar.gz}{\pandocbounded{\includesvg[keepaspectratio]{/assets/icons/16-download.svg}}}
\item[Repository:]
\href{https://github.com/typst/templates}{GitHub}
\item[Discipline :]
\begin{itemize}
\tightlist
\item[]
\item
  \href{https://typst.app/universe/search/?discipline=mathematics}{Mathematics}
\end{itemize}
\item[Categor y :]
\begin{itemize}
\tightlist
\item[]
\item
  \pandocbounded{\includesvg[keepaspectratio]{/assets/icons/16-atom.svg}}
  \href{https://typst.app/universe/search/?category=paper}{Paper}
\end{itemize}
\end{description}

\subsubsection{Where to report issues?}\label{where-to-report-issues}

This template is a project of Typst GmbH . Report issues on
\href{https://github.com/typst/templates}{their repository} . You can
also try to ask for help with this template on the
\href{https://forum.typst.app}{Forum} .

\phantomsection\label{versions}
\subsubsection{Version history}\label{version-history}

\begin{longtable}[]{@{}ll@{}}
\toprule\noalign{}
Version & Release Date \\
\midrule\noalign{}
\endhead
\bottomrule\noalign{}
\endlastfoot
0.1.2 & October 29, 2024 \\
\href{https://typst.app/universe/package/unequivocal-ams/0.1.1/}{0.1.1}
& August 8, 2024 \\
\href{https://typst.app/universe/package/unequivocal-ams/0.1.0/}{0.1.0}
& March 6, 2024 \\
\end{longtable}


\section{Package List LaTeX/rfc-vibe.tex}
\title{typst.app/universe/package/rfc-vibe}

\phantomsection\label{banner}
\section{rfc-vibe}\label{rfc-vibe}

{ 0.1.0 }

Bring RFC language into everyday docs

\phantomsection\label{readme}
Bring that RFC lingo to your everyday documents.

A \href{https://typst.app/}{Typst} package that makes it easy to use the
exact keywords, boilerplate, and definitions provided by BCP 14,
RFC2119, and RFC8174. See the end of this README for an example of the
output.

In the future, this package may include other RFC-related patterns which
are applicable to a wide variety of everyday documents.

\subsection{Usage}\label{usage}

Import the package in your Typst document:

\begin{Shaded}
\begin{Highlighting}[]
\NormalTok{\#import "@preview/rfc{-}vibe:0.1.0": *}
\end{Highlighting}
\end{Shaded}

\subsubsection{Keywords}\label{keywords}

Use the keywords according to these examples:

\begin{Shaded}
\begin{Highlighting}[]
\NormalTok{\#must              // renders as: MUST}
\NormalTok{\#must{-}not          // renders as: MUST NOT}
\NormalTok{\#required          // renders as: REQUIRED}
\NormalTok{\#shall             // renders as: SHALL}
\NormalTok{\#shall{-}not         // renders as: SHALL NOT}
\NormalTok{\#should            // renders as: SHOULD}
\NormalTok{\#should{-}not        // renders as: SHOULD NOT}
\NormalTok{\#recommended       // renders as: RECOMMENDED}
\NormalTok{\#not{-}recommended   // renders as: NOT RECOMMENDED}
\NormalTok{\#may               // renders as: MAY}
\NormalTok{\#optional          // renders as: OPTIONAL}
\end{Highlighting}
\end{Shaded}

For the rare situation when you want the keywords included in quotation
marks, use the \texttt{\ -quoted\ } versions:

\begin{Shaded}
\begin{Highlighting}[]
\NormalTok{\#must{-}quoted              // renders as: "MUST"}
\NormalTok{\#must{-}not{-}quoted          // renders as: "MUST NOT"}
\NormalTok{\#required{-}quoted          // renders as: "REQUIRED"}
\NormalTok{\#shall{-}quoted             // renders as: "SHALL"}
\NormalTok{\#shall{-}not{-}quoted         // renders as: "SHALL NOT"}
\NormalTok{\#should{-}quoted            // renders as: "SHOULD"}
\NormalTok{\#should{-}not{-}quoted        // renders as: "SHOULD NOT"}
\NormalTok{\#recommended{-}quoted       // renders as: "RECOMMENDED"}
\NormalTok{\#not{-}recommended{-}quoted   // renders as: "NOT RECOMMENDED"}
\NormalTok{\#may{-}quoted               // renders as: "MAY"}
\NormalTok{\#optional{-}quoted          // renders as: "OPTIONAL"}
\end{Highlighting}
\end{Shaded}

\subsubsection{Boilerplate}\label{boilerplate}

According to RFC8174, \emph{authors who follow these guidelines should
incorporate a specific phrase near the beginning of their document} .
Include this boilerplate text with:

\begin{Shaded}
\begin{Highlighting}[]
\NormalTok{\#rfc{-}keyword{-}boilerplate}
\end{Highlighting}
\end{Shaded}

This will render as:

\begin{Shaded}
\begin{Highlighting}[]
\NormalTok{The key words "MUST", "MUST NOT", "REQUIRED", "SHALL", "SHALL NOT", "SHOULD",}
\NormalTok{"SHOULD NOT", "RECOMMENDED", "NOT RECOMMENDED", "MAY", and "OPTIONAL" in this}
\NormalTok{document are to be interpreted as described in BCP 14 [RFC2119] [RFC8174] when,}
\NormalTok{and only when, they appear in all capitals, as shown here.}
\end{Highlighting}
\end{Shaded}

\subsubsection{Definitions}\label{definitions}

Although not required (and maybe discouraged), you can include
definitions of individual keywords in your document:

\begin{Shaded}
\begin{Highlighting}[]
\NormalTok{\#rfc{-}keyword{-}must{-}definition}
\NormalTok{\#rfc{-}keyword{-}must{-}not{-}definition}
\NormalTok{\#rfc{-}keyword{-}should{-}definition}
\NormalTok{\#rfc{-}keyword{-}should{-}not{-}definition}
\NormalTok{\#rfc{-}keyword{-}may{-}definition}
\end{Highlighting}
\end{Shaded}

Or include all keyword definitions at once with:

\begin{Shaded}
\begin{Highlighting}[]
\NormalTok{\#rfc{-}keyword{-}definitions}
\end{Highlighting}
\end{Shaded}

\subsection{Example Output}\label{example-output}

\pandocbounded{\includegraphics[keepaspectratio]{https://github.com/typst/packages/raw/main/packages/preview/rfc-vibe/0.1.0/thumbnail.png}}

\subsection{License}\label{license}

This project is licensed under the MIT License. See the
\href{https://github.com/typst/packages/raw/main/packages/preview/rfc-vibe/0.1.0/LICENSE}{LICENSE}
file for details.

\subsubsection{How to add}\label{how-to-add}

Copy this into your project and use the import as \texttt{\ rfc-vibe\ }

\begin{verbatim}
#import "@preview/rfc-vibe:0.1.0"
\end{verbatim}

\includesvg[width=0.16667in,height=0.16667in]{/assets/icons/16-copy.svg}

Check the docs for
\href{https://typst.app/docs/reference/scripting/\#packages}{more
information on how to import packages} .

\subsubsection{About}\label{about}

\begin{description}
\tightlist
\item[Author :]
\href{mailto:steve@waits.net}{Stephen Waits}
\item[License:]
MIT
\item[Current version:]
0.1.0
\item[Last updated:]
November 28, 2024
\item[First released:]
November 28, 2024
\item[Archive size:]
3.35 kB
\href{https://packages.typst.org/preview/rfc-vibe-0.1.0.tar.gz}{\pandocbounded{\includesvg[keepaspectratio]{/assets/icons/16-download.svg}}}
\item[Repository:]
\href{https://github.com/swaits/typst-collection}{GitHub}
\item[Categor y :]
\begin{itemize}
\tightlist
\item[]
\item
  \pandocbounded{\includesvg[keepaspectratio]{/assets/icons/16-hammer.svg}}
  \href{https://typst.app/universe/search/?category=utility}{Utility}
\end{itemize}
\end{description}

\subsubsection{Where to report issues?}\label{where-to-report-issues}

This package is a project of Stephen Waits . Report issues on
\href{https://github.com/swaits/typst-collection}{their repository} .
You can also try to ask for help with this package on the
\href{https://forum.typst.app}{Forum} .

Please report this package to the Typst team using the
\href{https://typst.app/contact}{contact form} if you believe it is a
safety hazard or infringes upon your rights.

\phantomsection\label{versions}
\subsubsection{Version history}\label{version-history}

\begin{longtable}[]{@{}ll@{}}
\toprule\noalign{}
Version & Release Date \\
\midrule\noalign{}
\endhead
\bottomrule\noalign{}
\endlastfoot
0.1.0 & November 28, 2024 \\
\end{longtable}

Typst GmbH did not create this package and cannot guarantee correct
functionality of this package or compatibility with any version of the
Typst compiler or app.


\section{Package List LaTeX/tufte-memo.tex}
\title{typst.app/universe/package/tufte-memo}

\phantomsection\label{banner}
\phantomsection\label{template-thumbnail}
\pandocbounded{\includegraphics[keepaspectratio]{https://packages.typst.org/preview/thumbnails/tufte-memo-0.1.2-small.webp}}

\section{tufte-memo}\label{tufte-memo}

{ 0.1.2 }

A memo template inspired by the design of Edward Tufte\textquotesingle s
books

{ } Featured Template

\href{/app?template=tufte-memo&version=0.1.2}{Create project in app}

\phantomsection\label{readme}
A memo document template inspired by the design of Edward Tufte’s
books for the Typst typesetting program.

For usage, see the usage guide
\href{https://github.com/nogula/tufte-memo/blob/main/template/main.pdf}{here}
.

The template provides handy functions: \texttt{\ template\ } ,
\texttt{\ note\ } , and \texttt{\ wideblock\ } . To create a document
with this template, use:

\begin{Shaded}
\begin{Highlighting}[]
\NormalTok{\#import "@preview/tufte{-}memo:0.1.2": *}

\NormalTok{\#show: template.with(}
\NormalTok{    title: [Document Title],}
\NormalTok{    authors: (}
\NormalTok{        (}
\NormalTok{        name: "Author Name",}
\NormalTok{        role: "Optional Role Line",}
\NormalTok{        affiliation: "Optional Affiliation Line",}
\NormalTok{        email: "email@company.com"}
\NormalTok{        ),}
\NormalTok{    )}
\NormalTok{)}
\NormalTok{...}
\end{Highlighting}
\end{Shaded}

additional configuration information is available in the usage guide.

The \texttt{\ note()\ } function provides the ability to produce
sidenotes next to the main body content. It can be called simply with
\texttt{\ \#note{[}...{]}\ } . Additionally, \texttt{\ wideblock()\ }
expands the width of its content to fill the full 6.5-inch-wide space,
rather than be compressed in to a four-inch column. It is simply called
with \texttt{\ wideblock{[}...{]}\ } .

\href{/app?template=tufte-memo&version=0.1.2}{Create project in app}

\subsubsection{How to use}\label{how-to-use}

Click the button above to create a new project using this template in
the Typst app.

You can also use the Typst CLI to start a new project on your computer
using this command:

\begin{verbatim}
typst init @preview/tufte-memo:0.1.2
\end{verbatim}

\includesvg[width=0.16667in,height=0.16667in]{/assets/icons/16-copy.svg}

\subsubsection{About}\label{about}

\begin{description}
\tightlist
\item[Author :]
Noah Gula
\item[License:]
MIT
\item[Current version:]
0.1.2
\item[Last updated:]
August 12, 2024
\item[First released:]
June 3, 2024
\item[Archive size:]
9.31 kB
\href{https://packages.typst.org/preview/tufte-memo-0.1.2.tar.gz}{\pandocbounded{\includesvg[keepaspectratio]{/assets/icons/16-download.svg}}}
\item[Repository:]
\href{https://github.com/nogula/tufte-memo}{GitHub}
\item[Categor y :]
\begin{itemize}
\tightlist
\item[]
\item
  \pandocbounded{\includesvg[keepaspectratio]{/assets/icons/16-speak.svg}}
  \href{https://typst.app/universe/search/?category=report}{Report}
\end{itemize}
\end{description}

\subsubsection{Where to report issues?}\label{where-to-report-issues}

This template is a project of Noah Gula . Report issues on
\href{https://github.com/nogula/tufte-memo}{their repository} . You can
also try to ask for help with this template on the
\href{https://forum.typst.app}{Forum} .

Please report this template to the Typst team using the
\href{https://typst.app/contact}{contact form} if you believe it is a
safety hazard or infringes upon your rights.

\phantomsection\label{versions}
\subsubsection{Version history}\label{version-history}

\begin{longtable}[]{@{}ll@{}}
\toprule\noalign{}
Version & Release Date \\
\midrule\noalign{}
\endhead
\bottomrule\noalign{}
\endlastfoot
0.1.2 & August 12, 2024 \\
\href{https://typst.app/universe/package/tufte-memo/0.1.1/}{0.1.1} &
June 5, 2024 \\
\href{https://typst.app/universe/package/tufte-memo/0.1.0/}{0.1.0} &
June 3, 2024 \\
\end{longtable}

Typst GmbH did not create this template and cannot guarantee correct
functionality of this template or compatibility with any version of the
Typst compiler or app.


\section{Package List LaTeX/backtrack.tex}
\title{typst.app/universe/package/backtrack}

\phantomsection\label{banner}
\section{backtrack}\label{backtrack}

{ 1.0.0 }

A version-agnostic library for checking the compiler version.

\phantomsection\label{readme}
Backtrack is a simple and performant Typst library that determines the
current compiler version and provides an API for comparing, displaying,
and observing versions.

Unlike the built-in
\href{https://github.com/typst/typst/pull/2016}{version API} which is
only available on Typst 0.9.0+, Backtrack works on any
\href{https://github.com/typst/packages/raw/main/packages/preview/backtrack/1.0.0/\#version-support}{*}
Typst version. It uses the built-in API when available so that it’ll
continue to work on all future Typst versions without modification.

Additionally, it:

\begin{itemize}
\tightlist
\item
  doesn’t noticeably impact compilation time. All version checks are
  extremely simple, and newer versions are checked first to avoid
  overhead from supporting old versions.
\item
  is automatically tested on \emph{every} supported Typst version to
  ensure reliability.
\item
  can be downloaded and installed manually in addition to being
  available as a package.
\end{itemize}

\subsection{Example}\label{example}

\begin{Shaded}
\begin{Highlighting}[]
\NormalTok{\#import "@preview/backtrack:1.0.0": current{-}version, versions}

\NormalTok{You are using Typst \#current{-}version.displayable!}
\NormalTok{\#\{}
\NormalTok{  if current{-}version.cmpable \textless{}= versions.v2023{-}03{-}28.cmpable [}
\NormalTok{    That is ancient.}
\NormalTok{  ] else if current{-}version.cmpable \textless{} versions.v0{-}5{-}0.cmpable [}
\NormalTok{    That is old.}
\NormalTok{  ] else [}
\NormalTok{    That is modern.}
\NormalTok{  ]}
\NormalTok{\}}
\end{Highlighting}
\end{Shaded}

\subsection{Installation}\label{installation}

There are two ways to install the library:

\begin{itemize}
\item
  Use the package on Typst 0.6.0+. This is as simple as adding the
  following line to your document:

\begin{Shaded}
\begin{Highlighting}[]
\NormalTok{\#import "@preview/backtrack:1.0.0"}
\end{Highlighting}
\end{Shaded}
\item
  Download and install the library manually:

  \begin{enumerate}
  \item
    Download and extract the latest
    \href{https://github.com/TheLukeGuy/backtrack/releases}{release} .
  \item
    Rename \texttt{\ src/lib.typ\ } to \texttt{\ src/backtrack.typ\ } .
  \item
    Move/copy \texttt{\ COPYING\ } into \texttt{\ src/\ } .
  \item
    Rename the \texttt{\ src/\ } directory to \texttt{\ backtrack/\ } .
  \item
    Move/copy the newly-renamed \texttt{\ backtrack/\ } directory into
    your project.
  \item
    Add the following line to your document:

\begin{Shaded}
\begin{Highlighting}[]
\NormalTok{\#import "[path/to/]backtrack/backtrack.typ"}
\end{Highlighting}
\end{Shaded}
  \end{enumerate}
\end{itemize}

\subsection{Documentation}\label{documentation}

See
\href{https://github.com/typst/packages/raw/main/packages/preview/backtrack/1.0.0/DOCS.md}{DOCS.md}
. It’s quite short. 😀

\subsection{Version Support}\label{version-support}

Backtrack compiles on and can detect these versions:

\begin{longtable}[]{@{}lcl@{}}
\toprule\noalign{}
Version & Status & Notes \\
\midrule\noalign{}
\endhead
\bottomrule\noalign{}
\endlastfoot
0.6.0+ & ✠& Also available as a package \\
March 28, 2023â€``0.5.0 & ✠& \\
March 21, 2023 & ✠& Initial public/standalone Typst release \\
February 25, 2023 & 🟡 & Detects as February 15, 2023 \\
February 12â€``15, 2023 & ✠& \\
February 2, 2023 & 🟡 & Detects as January 30, 2023 \\
January 30, 2023 & ✠& \\
\end{longtable}

The partially-supported versions \emph{may} be possible to detect, but
they’re tricky since most of their changes are content-related.
Content values were opaque up until the March 21, 2023 release, making
it difficult to automatically check for the presence of these changes.

\subsection{Copying}\label{copying}

Copyright © 2023 \href{https://github.com/TheLukeGuy}{Luke Chambers}

Backtrack is licensed under the Apache License, Version 2.0. You can
find a copy of the license in
\href{https://github.com/typst/packages/raw/main/packages/preview/backtrack/1.0.0/COPYING}{COPYING}
or online at \url{https://www.apache.org/licenses/LICENSE-2.0} .

\subsubsection{How to add}\label{how-to-add}

Copy this into your project and use the import as \texttt{\ backtrack\ }

\begin{verbatim}
#import "@preview/backtrack:1.0.0"
\end{verbatim}

\includesvg[width=0.16667in,height=0.16667in]{/assets/icons/16-copy.svg}

Check the docs for
\href{https://typst.app/docs/reference/scripting/\#packages}{more
information on how to import packages} .

\subsubsection{About}\label{about}

\begin{description}
\tightlist
\item[Author :]
TheLukeGuy
\item[License:]
Apache-2.0
\item[Current version:]
1.0.0
\item[Last updated:]
October 27, 2023
\item[First released:]
October 27, 2023
\item[Archive size:]
8.64 kB
\href{https://packages.typst.org/preview/backtrack-1.0.0.tar.gz}{\pandocbounded{\includesvg[keepaspectratio]{/assets/icons/16-download.svg}}}
\item[Repository:]
\href{https://github.com/TheLukeGuy/backtrack}{GitHub}
\end{description}

\subsubsection{Where to report issues?}\label{where-to-report-issues}

This package is a project of TheLukeGuy . Report issues on
\href{https://github.com/TheLukeGuy/backtrack}{their repository} . You
can also try to ask for help with this package on the
\href{https://forum.typst.app}{Forum} .

Please report this package to the Typst team using the
\href{https://typst.app/contact}{contact form} if you believe it is a
safety hazard or infringes upon your rights.

\phantomsection\label{versions}
\subsubsection{Version history}\label{version-history}

\begin{longtable}[]{@{}ll@{}}
\toprule\noalign{}
Version & Release Date \\
\midrule\noalign{}
\endhead
\bottomrule\noalign{}
\endlastfoot
1.0.0 & October 27, 2023 \\
\end{longtable}

Typst GmbH did not create this package and cannot guarantee correct
functionality of this package or compatibility with any version of the
Typst compiler or app.


\section{Package List LaTeX/delegis.tex}
\title{typst.app/universe/package/delegis}

\phantomsection\label{banner}
\phantomsection\label{template-thumbnail}
\pandocbounded{\includegraphics[keepaspectratio]{https://packages.typst.org/preview/thumbnails/delegis-0.3.0-small.webp}}

\section{delegis}\label{delegis}

{ 0.3.0 }

A package and template for drafting legislative content in a
German-style structuring, such as for bylaws, etc.

\href{/app?template=delegis&version=0.3.0}{Create project in app}

\phantomsection\label{readme}
\begin{longtable}[]{@{}lll@{}}
\toprule\noalign{}
\endhead
\bottomrule\noalign{}
\endlastfoot
\pandocbounded{\includegraphics[keepaspectratio]{https://github.com/typst/packages/raw/main/packages/preview/delegis/0.3.0/demo-1.png}}
&
\pandocbounded{\includegraphics[keepaspectratio]{https://github.com/typst/packages/raw/main/packages/preview/delegis/0.3.0/demo-2.png}}
&
\pandocbounded{\includegraphics[keepaspectratio]{https://github.com/typst/packages/raw/main/packages/preview/delegis/0.3.0/demo-3.png}} \\
\end{longtable}

A package and template for drafting legislative content in a
German-style structuring, such as for bylaws, etc.

While the template is designed to be used in German documents, all
strings are customizable. You can have a look at the
\texttt{\ delegis.typ\ } to see all available parameters.

\subsection{General Usage}\label{general-usage}

While this \texttt{\ README.md\ } gives you a brief overview of the
package’s usage, we recommend that you use the template (in the
\texttt{\ template\ } folder) as a starting point instead.

\subsubsection{Importing the Package}\label{importing-the-package}

\begin{Shaded}
\begin{Highlighting}[]
\NormalTok{\#import "@preview/delegis:0.3.0": *}
\end{Highlighting}
\end{Shaded}

\subsubsection{Initializing the
template}\label{initializing-the-template}

\begin{Shaded}
\begin{Highlighting}[]
\NormalTok{\#show: delegis.with(}
\NormalTok{  // Metadata}
\NormalTok{  title: "Vereinsordnung zu ABCDEF", // title of the law/bylaw/...}
\NormalTok{  abbreviation: "ABCDEFVO", // abbreviation of the law/bylaw/...}
\NormalTok{  resolution: "3. Beschluss des Vorstands vom 24.01.2024", // resolution number and date}
\NormalTok{  in{-}effect: "24.01.2024", // date when it comes into effect}
\NormalTok{  draft: false, // whether this is a draft}
\NormalTok{  // Template}
\NormalTok{  logo: image("wuespace.jpg", alt: "WüSpace e. V."), // logo of the organization, shown on the first page}
\NormalTok{)}
\end{Highlighting}
\end{Shaded}

\subsubsection{Sections}\label{sections}

Sections are auto-detected as long as they follow the pattern
\texttt{\ §\ 1\ ...\ } or \texttt{\ §\ 1a\ ...\ } in its own
paragraph:

\begin{Shaded}
\begin{Highlighting}[]
\NormalTok{§ 1 Geltungsbereich}

\NormalTok{(1) }
\NormalTok{Diese Ordnung gilt für alle Mitglieder des Vereins.}

\NormalTok{(2) }
\NormalTok{Sie regelt die Mitgliedschaft im Verein.}

\NormalTok{§ 2 Mitgliedschaft}

\NormalTok{(1) }
\NormalTok{Die Mitgliedschaft im Verein ist freiwillig.}

\NormalTok{(2) }
\NormalTok{Sie kann jederzeit gekündigt werden.}

\NormalTok{§ 2a Ehrenmitgliedschaft}

\NormalTok{(1) }
\NormalTok{Die Ehrenmitgliedschaft wird durch den Vorstand verliehen.}
\end{Highlighting}
\end{Shaded}

Alternatively (or if you want to use special characters otherwise not
supported, such as \texttt{\ *\ } ), you can also use the
\texttt{\ \#section{[}number{]}{[}title{]}\ } function:

\begin{Shaded}
\begin{Highlighting}[]
\NormalTok{\#section[§ 3][Administrator*innen]}
\end{Highlighting}
\end{Shaded}

\subsubsection{Hierarchical Divisions}\label{hierarchical-divisions}

If you want to add more structure to your sections, you can use normal
Typst headings. Note that only the level 6 headings are reserved for the
section numbers:

\begin{Shaded}
\begin{Highlighting}[]
\NormalTok{= Allgemeine Bestimmungen}

\NormalTok{§ 1 ABC}

\NormalTok{§ 2 DEF}

\NormalTok{= Besondere Bestimmungen}

\NormalTok{§ 3 GHI}

\NormalTok{§ 4 JKL}
\end{Highlighting}
\end{Shaded}

Delegis will automatically use a numbering scheme for the divisions that
is in line with the “Handbuch der Rechtsförmlichkeit�, Rn. 379 f.
If you want to customize the division titles, you can do so by setting
the \texttt{\ division-prefixes\ } parameter in the \texttt{\ delegis\ }
function:

\begin{Shaded}
\begin{Highlighting}[]
\NormalTok{\#show: delegis.with(}
\NormalTok{  division{-}prefixes: ("Teil", "Kapitel", "Abschnitt", "Unterabschnitt")}
\NormalTok{)}
\end{Highlighting}
\end{Shaded}

\subsubsection{Sentence Numbering}\label{sentence-numbering}

If a paragraph contains multiple sentences, you can number them by
adding a \texttt{\ \#s\textasciitilde{}\ } at the beginning of the
sentences:

\begin{Shaded}
\begin{Highlighting}[]
\NormalTok{§ 3 Mitgliedsbeiträge}

\NormalTok{\#s\textasciitilde{}Die Mitgliedsbeiträge sind monatlich zu entrichten.}
\NormalTok{\#s\textasciitilde{}Sie sind bis zum 5. des Folgemonats zu zahlen.}
\end{Highlighting}
\end{Shaded}

This automatically adds corresponding sentence numbers in superscript.

\subsubsection{Referencing other
Sections}\label{referencing-other-sections}

Referencing works manually by specifying the section number. While
automations would be feasible, we have found that in practice, they’re
not as useful as they might seem for legislative documents.

In some cases, referencing sections using \texttt{\ §\ X\ } could be
mis-interpreted as a new section. To avoid this, use the non-breaking
space character \texttt{\ \textasciitilde{}\ } between the
\texttt{\ §\ } and the number:

\begin{Shaded}
\begin{Highlighting}[]
\NormalTok{§ 5 Inkrafttreten}

\NormalTok{Diese Ordnung tritt am 24.01.2024 in Kraft. §\textasciitilde{}4 bleibt unberührt.}
\end{Highlighting}
\end{Shaded}

\subsection{Changelog}\label{changelog}

\subsubsection{v0.3.0}\label{v0.3.0}

\paragraph{Features}\label{features}

\begin{itemize}
\tightlist
\item
  Adjust numbered list / enumeration numbering to be in line with
  “Handbuch der Rechtsförmlichkeit�, Rn. 374
\item
  Make division titles (e.g., “Part�, “Chapter�, “Division�)
  customizable and conform to the “Handbuch der
  Rechtsförmlichkeit�, Rn. 379 f.
\end{itemize}

\subsubsection{v0.2.0}\label{v0.2.0}

\paragraph{Features}\label{features-1}

\begin{itemize}
\tightlist
\item
  Add \texttt{\ \#metadata\ } fields for usage with
  \texttt{\ typst\ query\ } . You can now use
  \texttt{\ typst\ query\ file.typ\ "\textless{}field\textgreater{}"\ -\/-field\ value\ -\/-one\ }
  with \texttt{\ \textless{}field\textgreater{}\ } being one of the
  following to query metadata fields in the command line:

  \begin{itemize}
  \tightlist
  \item
    \texttt{\ \textless{}title\textgreater{}\ }
  \item
    \texttt{\ \textless{}abbreviation\textgreater{}\ }
  \item
    \texttt{\ \textless{}resolution\textgreater{}\ }
  \item
    \texttt{\ \textless{}in-effect\textgreater{}\ }
  \end{itemize}
\item
  Add \texttt{\ \#section{[}§\ 1{]}{[}ABC{]}\ } function to enable
  previously unsupported special chars (such as \texttt{\ *\ } ) in
  section headings. Note that this was previously possible using
  \texttt{\ \#unnumbered{[}§\ 1\textbackslash{}\ ABC{]}\ } , but the
  new function adds a semantically better-fitting alternative to this
  fix.
\item
  Improve heading style rules. This also fixes an incompatibility with
  \texttt{\ pandoc\ } , meaning it’s now possible to use
  \texttt{\ pandoc\ } to convert delegis documents to HTML, etc.
\item
  Set the footnote numbering to \texttt{\ {[}1{]}\ } to not collide with
  sentence numbers.
\end{itemize}

\paragraph{Bug Fixes}\label{bug-fixes}

\begin{itemize}
\tightlist
\item
  Fix a typo in the \texttt{\ str-draft\ } variable name that lead to
  draft documents resulting in a syntax error.
\item
  Fix hyphenation issues with the abbreviation on the title page
  (hyphenation between the parentheses and the abbreviation itself)
\end{itemize}

\subsubsection{v0.1.0}\label{v0.1.0}

Initial Release

\href{/app?template=delegis&version=0.3.0}{Create project in app}

\subsubsection{How to use}\label{how-to-use}

Click the button above to create a new project using this template in
the Typst app.

You can also use the Typst CLI to start a new project on your computer
using this command:

\begin{verbatim}
typst init @preview/delegis:0.3.0
\end{verbatim}

\includesvg[width=0.16667in,height=0.16667in]{/assets/icons/16-copy.svg}

\subsubsection{About}\label{about}

\begin{description}
\tightlist
\item[Author :]
\href{https://github.com/wuespace}{WüSpace e. V.}
\item[License:]
MIT
\item[Current version:]
0.3.0
\item[Last updated:]
May 22, 2024
\item[First released:]
March 16, 2024
\item[Archive size:]
13.4 kB
\href{https://packages.typst.org/preview/delegis-0.3.0.tar.gz}{\pandocbounded{\includesvg[keepaspectratio]{/assets/icons/16-download.svg}}}
\item[Repository:]
\href{https://github.com/wuespace/delegis}{GitHub}
\item[Discipline :]
\begin{itemize}
\tightlist
\item[]
\item
  \href{https://typst.app/universe/search/?discipline=law}{Law}
\end{itemize}
\item[Categor y :]
\begin{itemize}
\tightlist
\item[]
\item
  \pandocbounded{\includesvg[keepaspectratio]{/assets/icons/16-envelope.svg}}
  \href{https://typst.app/universe/search/?category=office}{Office}
\end{itemize}
\end{description}

\subsubsection{Where to report issues?}\label{where-to-report-issues}

This template is a project of WüSpace e. V. . Report issues on
\href{https://github.com/wuespace/delegis}{their repository} . You can
also try to ask for help with this template on the
\href{https://forum.typst.app}{Forum} .

Please report this template to the Typst team using the
\href{https://typst.app/contact}{contact form} if you believe it is a
safety hazard or infringes upon your rights.

\phantomsection\label{versions}
\subsubsection{Version history}\label{version-history}

\begin{longtable}[]{@{}ll@{}}
\toprule\noalign{}
Version & Release Date \\
\midrule\noalign{}
\endhead
\bottomrule\noalign{}
\endlastfoot
0.3.0 & May 22, 2024 \\
\href{https://typst.app/universe/package/delegis/0.2.0/}{0.2.0} & May
17, 2024 \\
\href{https://typst.app/universe/package/delegis/0.1.0/}{0.1.0} & March
16, 2024 \\
\end{longtable}

Typst GmbH did not create this template and cannot guarantee correct
functionality of this template or compatibility with any version of the
Typst compiler or app.


\section{Package List LaTeX/diverential.tex}
\title{typst.app/universe/package/diverential}

\phantomsection\label{banner}
\section{diverential}\label{diverential}

{ 0.2.0 }

Format differentials conveniently.

\phantomsection\label{readme}
\texttt{\ diverential\ } is a
\href{https://github.com/typst/typst}{Typst} package simplifying the
typesetting of differentials. It is the equivalent to LaTeX’s
\texttt{\ diffcoeff\ } , though not as mature.

\subsection{Overview}\label{overview}

\texttt{\ diverential\ } allows normal, partial, compact, and separated
derivatives with smart degree calculations.

\begin{Shaded}
\begin{Highlighting}[]
\NormalTok{\#}\ImportTok{import} \StringTok{"@preview/diverential:0.2.0"}\OperatorTok{:} \OperatorTok{*}

\NormalTok{$ }\FunctionTok{dv}\NormalTok{(f}\OperatorTok{,}\NormalTok{ x}\OperatorTok{,}\NormalTok{ deg}\OperatorTok{:} \DecValTok{2}\OperatorTok{,}\NormalTok{ eval}\OperatorTok{:} \DecValTok{0}\NormalTok{) $}
\NormalTok{$ }\FunctionTok{dvp}\NormalTok{(f}\OperatorTok{,}\NormalTok{ x}\OperatorTok{,}\NormalTok{ y}\OperatorTok{,}\NormalTok{ eval}\OperatorTok{:} \DecValTok{0}\OperatorTok{,}\NormalTok{ evalsym}\OperatorTok{:} \StringTok{"["}\NormalTok{) $}
\NormalTok{$ }\FunctionTok{dvpc}\NormalTok{(f}\OperatorTok{,}\NormalTok{ x) $}
\NormalTok{$ }\FunctionTok{dvps}\NormalTok{(f}\OperatorTok{,}\NormalTok{ \#([x]}\OperatorTok{,} \DecValTok{2}\NormalTok{)}\OperatorTok{,}\NormalTok{ \#([y]}\OperatorTok{,}\NormalTok{ [n])}\OperatorTok{,}\NormalTok{ \#([z]}\OperatorTok{,}\NormalTok{ [m])}\OperatorTok{,}\NormalTok{ eval}\OperatorTok{:} \DecValTok{0}\NormalTok{) $}
\end{Highlighting}
\end{Shaded}

\includegraphics[width=1.5625in,height=\textheight,keepaspectratio]{https://github.com/typst/packages/raw/main/packages/preview/diverential/0.2.0/examples/overview.jpg}

\subsection{\texorpdfstring{\texttt{\ dv\ }}{ dv }}\label{dv}

\texttt{\ dv\ } is an ordinary derivative. It takes the function as its
first argument and the variable as its second one. A degree can be
specified with \texttt{\ deg\ } . The derivate can be specified to be
evaluated at a point with \texttt{\ eval\ } , the brackets of which can
be changed with \texttt{\ evalsym\ } . \texttt{\ space\ } influences the
space between derivative and evaluation bracket. Unless defined
otherwise, no space is set by default, except for
\texttt{\ \textbar{}\ } , where a small gap is introduced.

\subsubsection{\texorpdfstring{\texttt{\ dvs\ }}{ dvs }}\label{dvs}

Same as \texttt{\ dv\ } , but separates the function from the
derivative.\\
Example: \$\$
\textbackslash frac\{\textbackslash mathrm\{d\}\}\{\textbackslash mathrm\{d\}x\}
f \$\$

\subsubsection{\texorpdfstring{\texttt{\ dvc\ }}{ dvc }}\label{dvc}

Same as \texttt{\ dv\ } , but uses a compact derivative.\\
Example: \$\$ \textbackslash mathrm\{d\}\_x f \$\$

\subsection{\texorpdfstring{\texttt{\ dvp\ }}{ dvp }}\label{dvp}

\texttt{\ dv\ } is a partial derivative. It takes the function as its
first argument and the variable as the rest. For information on
\texttt{\ eval\ } , \texttt{\ evalsym\ } , and \texttt{\ space\ } , read
the description of \texttt{\ dv\ } .\\
The variable can be one of these options:

\begin{itemize}
\tightlist
\item
  plain variable, e.g. \texttt{\ x\ }
\item
  list of variables, e.g. \texttt{\ x,\ y\ }
\item
  list of variables with higher degrees (type
  \texttt{\ (content,\ integer)\ } ), e.g.
  \texttt{\ x,\ \#({[}y{]},\ 2)\ } The degree is smartly calculated: If
  all degrees of the variables are integers, the total degree is their
  sum. If some are content, the integer ones are summed (arithmetically)
  up and added to the visual sum of the content degrees. Example:
  \texttt{\ \#({[}x{]},\ n),\ \#({[}y{]},\ 2),\ z\ } â†'
  \$\textbackslash frac\{\textbackslash partial\^{}\{n+3\}\}\{\textbackslash partial
  x\^{}n,\textbackslash partial y\^{}2,\textbackslash partial z\}\$\\
  Specifying \texttt{\ deg\ } manually is always possible and might be
  required in more complicated cases.
\end{itemize}

\subsubsection{\texorpdfstring{\texttt{\ dvps\ }}{ dvps }}\label{dvps}

Same as \texttt{\ dvp\ } , but separates the function from the
derivative.\\
Example: \$\$
\textbackslash frac\{\textbackslash partial\}\{\textbackslash partial
x\} f \$\$

\subsubsection{\texorpdfstring{\texttt{\ dvpc\ }}{ dvpc }}\label{dvpc}

Same as \texttt{\ dvp\ } , but uses a compact derivative.\\
Note: If supplying multiple variables, \texttt{\ deg\ } is ignored.\\
Example: \$\$ \textbackslash partial\_x f \$\$

\subsubsection{How to add}\label{how-to-add}

Copy this into your project and use the import as
\texttt{\ diverential\ }

\begin{verbatim}
#import "@preview/diverential:0.2.0"
\end{verbatim}

\includesvg[width=0.16667in,height=0.16667in]{/assets/icons/16-copy.svg}

Check the docs for
\href{https://typst.app/docs/reference/scripting/\#packages}{more
information on how to import packages} .

\subsubsection{About}\label{about}

\begin{description}
\tightlist
\item[Author :]
Christopher Hecker
\item[License:]
MIT
\item[Current version:]
0.2.0
\item[Last updated:]
July 29, 2023
\item[First released:]
July 29, 2023
\item[Archive size:]
3.38 kB
\href{https://packages.typst.org/preview/diverential-0.2.0.tar.gz}{\pandocbounded{\includesvg[keepaspectratio]{/assets/icons/16-download.svg}}}
\end{description}

\subsubsection{Where to report issues?}\label{where-to-report-issues}

This package is a project of Christopher Hecker . You can also try to
ask for help with this package on the
\href{https://forum.typst.app}{Forum} .

Please report this package to the Typst team using the
\href{https://typst.app/contact}{contact form} if you believe it is a
safety hazard or infringes upon your rights.

\phantomsection\label{versions}
\subsubsection{Version history}\label{version-history}

\begin{longtable}[]{@{}ll@{}}
\toprule\noalign{}
Version & Release Date \\
\midrule\noalign{}
\endhead
\bottomrule\noalign{}
\endlastfoot
0.2.0 & July 29, 2023 \\
\end{longtable}

Typst GmbH did not create this package and cannot guarantee correct
functionality of this package or compatibility with any version of the
Typst compiler or app.


\section{Package List LaTeX/indenta.tex}
\title{typst.app/universe/package/indenta}

\phantomsection\label{banner}
\section{indenta}\label{indenta}

{ 0.0.3 }

Fix indent of first paragraph.

\phantomsection\label{readme}
An attempt to fix the indentation of the first paragraph in typst.

It works.

\subsection{Usage}\label{usage}

\begin{Shaded}
\begin{Highlighting}[]
\NormalTok{\#set par(first{-}line{-}indent: 2em)}
\NormalTok{\#import "@preview/indenta:0.0.3": fix{-}indent}
\NormalTok{\#show: fix{-}indent()}
\end{Highlighting}
\end{Shaded}

\subsection{Demo}\label{demo}

\pandocbounded{\includegraphics[keepaspectratio]{https://github.com/flaribbit/indenta/assets/24885181/874df696-3277-4103-9166-a24639b0c7c6}}

\subsection{Note}\label{note}

When you use \texttt{\ fix-indent()\ } with other show rules, make sure
to call \texttt{\ fix-indent()\ } \textbf{after other show rules} . For
example:

\begin{Shaded}
\begin{Highlighting}[]
\NormalTok{\#show heading.where(level: 1): set text(size: 20pt)}
\NormalTok{\#show: fix{-}indent()}
\end{Highlighting}
\end{Shaded}

If you want to process the content inside your custom block, you can
call \texttt{\ fix-indent\ } inside your block. For example:

\begin{Shaded}
\begin{Highlighting}[]
\NormalTok{\#block[\#set text(fill: red)}
\NormalTok{\#show: fix{-}indent()}

\NormalTok{Hello}

\NormalTok{\#table()[table]}

\NormalTok{World}
\NormalTok{]}
\end{Highlighting}
\end{Shaded}

This package is in a very early stage and may not work as expected in
some cases. Currently, there is no easy way to check if an element is
inlined or not. If you got an unexpected result, you can try
\texttt{\ fix-indent(unsafe:\ true)\ } to disable the check.

Minor fixes can be made at any time, but the package in typst universe
may not be updated immediately. You can check the latest version on
\href{https://github.com/flaribbit/indenta}{GitHub} then copy and paste
the code into your typst file.

If it still doesn’t work as expected, you can try another solution
(aka fake-par solution):

\begin{Shaded}
\begin{Highlighting}[]
\NormalTok{\#let fakepar=context\{box();v({-}measure(block()+block()).height)\}}
\NormalTok{\#show heading: it=\textgreater{}it+fakepar}
\NormalTok{\#show figure: it=\textgreater{}it+fakepar}
\NormalTok{\#show math.equation.where(block: true): it=\textgreater{}it+fakepar}
\NormalTok{// ... other elements}
\end{Highlighting}
\end{Shaded}

\subsubsection{How to add}\label{how-to-add}

Copy this into your project and use the import as \texttt{\ indenta\ }

\begin{verbatim}
#import "@preview/indenta:0.0.3"
\end{verbatim}

\includesvg[width=0.16667in,height=0.16667in]{/assets/icons/16-copy.svg}

Check the docs for
\href{https://typst.app/docs/reference/scripting/\#packages}{more
information on how to import packages} .

\subsubsection{About}\label{about}

\begin{description}
\tightlist
\item[Author :]
\href{https://github.com/flaribbit}{梦飞ç¿''}
\item[License:]
MIT
\item[Current version:]
0.0.3
\item[Last updated:]
June 10, 2024
\item[First released:]
May 24, 2024
\item[Archive size:]
2.35 kB
\href{https://packages.typst.org/preview/indenta-0.0.3.tar.gz}{\pandocbounded{\includesvg[keepaspectratio]{/assets/icons/16-download.svg}}}
\item[Repository:]
\href{https://github.com/flaribbit/indenta}{GitHub}
\item[Categor y :]
\begin{itemize}
\tightlist
\item[]
\item
  \pandocbounded{\includesvg[keepaspectratio]{/assets/icons/16-hammer.svg}}
  \href{https://typst.app/universe/search/?category=utility}{Utility}
\end{itemize}
\end{description}

\subsubsection{Where to report issues?}\label{where-to-report-issues}

This package is a project of 梦飞ç¿'' . Report issues on
\href{https://github.com/flaribbit/indenta}{their repository} . You can
also try to ask for help with this package on the
\href{https://forum.typst.app}{Forum} .

Please report this package to the Typst team using the
\href{https://typst.app/contact}{contact form} if you believe it is a
safety hazard or infringes upon your rights.

\phantomsection\label{versions}
\subsubsection{Version history}\label{version-history}

\begin{longtable}[]{@{}ll@{}}
\toprule\noalign{}
Version & Release Date \\
\midrule\noalign{}
\endhead
\bottomrule\noalign{}
\endlastfoot
0.0.3 & June 10, 2024 \\
\href{https://typst.app/universe/package/indenta/0.0.2/}{0.0.2} & May
27, 2024 \\
\href{https://typst.app/universe/package/indenta/0.0.1/}{0.0.1} & May
24, 2024 \\
\end{longtable}

Typst GmbH did not create this package and cannot guarantee correct
functionality of this package or compatibility with any version of the
Typst compiler or app.


\section{Package List LaTeX/tlacuache-thesis-fc-unam.tex}
\title{typst.app/universe/package/tlacuache-thesis-fc-unam}

\phantomsection\label{banner}
\phantomsection\label{template-thumbnail}
\pandocbounded{\includegraphics[keepaspectratio]{https://packages.typst.org/preview/thumbnails/tlacuache-thesis-fc-unam-0.1.1-small.webp}}

\section{tlacuache-thesis-fc-unam}\label{tlacuache-thesis-fc-unam}

{ 0.1.1 }

Template para escribir una tesis para la facultad de ciencias.

\href{/app?template=tlacuache-thesis-fc-unam&version=0.1.1}{Create
project in app}

\phantomsection\label{readme}
Este es un template para tesis de la facultad de ciencias, en la
Universidad Nacional Autónoma de México (UNAM).

This is a thesis template for the Science Faculty at Universidad
Nacional Autónoma de México (UNAM) based on my thesis.

\subsection{Uso/Usage}\label{usousage}

En la aplicación web de Typst da click en “Start from template� y
busca \texttt{\ tlacuache-thesis-fc-unam\ } .

In the Typst web app simply click “Start from template� on the
dashboard and search for \texttt{\ tlacuache-thesis-fc-unam\ } .

Si estas usando la versión de teminal usa el comando: From the CLI you
can initialize the project with the command:

\begin{Shaded}
\begin{Highlighting}[]
\ExtensionTok{typst}\NormalTok{ init @preview/tlacuache{-}thesis{-}fc{-}unam:0.1.1}
\end{Highlighting}
\end{Shaded}

\subsection{Configuración/Configuration}\label{configuraciuxe3uxb3nconfiguration}

Para configurar tu tesis puedes hacerlo con estas lineas al inicio de tu
archivo principal.

To set the thesis template, you can use the following lines in your main
file.

\begin{Shaded}
\begin{Highlighting}[]
\NormalTok{\#import "@preview/tlacuache{-}thesis{-}fc{-}unam:0.1.1": thesis}

\NormalTok{\#show: thesis.with(}
\NormalTok{  ttitulo: [Titulo],}
\NormalTok{  grado: [Licenciatura],}
\NormalTok{  autor: [Autor],}
\NormalTok{  asesor: [Asesor],}
\NormalTok{  lugar: [Ciudad de México, México],}
\NormalTok{  agno: [\#datetime.today().year()],}
\NormalTok{  bibliography: bibliography("references.bib"),}
\NormalTok{)}

\NormalTok{// Tu tesis va aquí}
\end{Highlighting}
\end{Shaded}

Tambien puedes utilizar estas lineas para crear capítulos con
bibliografía, si deseas crear un pdf solomente para el capítulo.

You could also create a pdf for just a chapter with bibliography, by
using the following lines.

\begin{Shaded}
\begin{Highlighting}[]
\NormalTok{\#import "@preview/tlacuache{-}thesis{-}fc{-}unam:0.1.1": chapter}

\NormalTok{// completamente opcional cargar la bibliografía, compilar el capítulo}
\NormalTok{\#show: chapter.with(bibliography: bibliography("references.bib"))}

\NormalTok{// Tu capítulo va aquí}
\end{Highlighting}
\end{Shaded}

Si quieres crear pdf aun mas cortos, puedes utilizar estas lineas para
crear un pdf solo para el sección de tu capítulo.

You could even create a pdf for just a section of a chapter.

\begin{Shaded}
\begin{Highlighting}[]
\NormalTok{\#import "@preview/tlacuache{-}thesis{-}fc{-}unam:0.1.1": section}

\NormalTok{// completamente opcional cargar la bibliografía, compilar el sección}
\NormalTok{\#show: section.with(bibliography: bibliography("references.bib"))}

\NormalTok{// Tu sección va aquí}
\end{Highlighting}
\end{Shaded}

\href{/app?template=tlacuache-thesis-fc-unam&version=0.1.1}{Create
project in app}

\subsubsection{How to use}\label{how-to-use}

Click the button above to create a new project using this template in
the Typst app.

You can also use the Typst CLI to start a new project on your computer
using this command:

\begin{verbatim}
typst init @preview/tlacuache-thesis-fc-unam:0.1.1
\end{verbatim}

\includesvg[width=0.16667in,height=0.16667in]{/assets/icons/16-copy.svg}

\subsubsection{About}\label{about}

\begin{description}
\tightlist
\item[Author :]
David Valencia, davidalencia@ciencias.unam.mx
\item[License:]
MIT
\item[Current version:]
0.1.1
\item[Last updated:]
April 9, 2024
\item[First released:]
April 9, 2024
\item[Archive size:]
3.14 MB
\href{https://packages.typst.org/preview/tlacuache-thesis-fc-unam-0.1.1.tar.gz}{\pandocbounded{\includesvg[keepaspectratio]{/assets/icons/16-download.svg}}}
\item[Categor y :]
\begin{itemize}
\tightlist
\item[]
\item
  \pandocbounded{\includesvg[keepaspectratio]{/assets/icons/16-mortarboard.svg}}
  \href{https://typst.app/universe/search/?category=thesis}{Thesis}
\end{itemize}
\end{description}

\subsubsection{Where to report issues?}\label{where-to-report-issues}

This template is a project of David Valencia,
davidalencia@ciencias.unam.mx . You can also try to ask for help with
this template on the \href{https://forum.typst.app}{Forum} .

Please report this template to the Typst team using the
\href{https://typst.app/contact}{contact form} if you believe it is a
safety hazard or infringes upon your rights.

\phantomsection\label{versions}
\subsubsection{Version history}\label{version-history}

\begin{longtable}[]{@{}ll@{}}
\toprule\noalign{}
Version & Release Date \\
\midrule\noalign{}
\endhead
\bottomrule\noalign{}
\endlastfoot
0.1.1 & April 9, 2024 \\
\end{longtable}

Typst GmbH did not create this template and cannot guarantee correct
functionality of this template or compatibility with any version of the
Typst compiler or app.


\section{Package List LaTeX/delimitizer.tex}
\title{typst.app/universe/package/delimitizer}

\phantomsection\label{banner}
\section{delimitizer}\label{delimitizer}

{ 0.1.0 }

Customize the size of delimiters. Like \textbackslash big,
\textbackslash Big, \textbackslash bigg, \textbackslash Bigg in LaTeX.

\phantomsection\label{readme}
This package lets you customize the size of delimiters in your math
equations. It is useful when you want to make your equations more
readable by increasing the size of certain delimiters. Just like
\texttt{\ \textbackslash{}big\ } , \texttt{\ \textbackslash{}Big\ } ,
\texttt{\ \textbackslash{}bigg\ } , and
\texttt{\ \textbackslash{}Bigg\ } in LaTeX, \texttt{\ delimitizer\ }
provides you with the same functionality in Typst.

\begin{itemize}
\tightlist
\item
  \texttt{\ big(delimiter)\ } : Makes the delimiters bigger than the
  default size.
\item
  \texttt{\ Big(delimiter)\ } : Makes the delimiters bigger than
  \texttt{\ big()\ } .
\item
  \texttt{\ bigg(delimiter)\ } : Makes the delimiters bigger than
  \texttt{\ Big()\ } .
\item
  \texttt{\ Bigg(delimiter)\ } : Makes the delimiters bigger than
  \texttt{\ bigg()\ } .
\item
  \texttt{\ scaled-delimiter(delimiter,\ size)\ } : Scales the
  delimiters by a factor of your choice.
\item
  \texttt{\ paired-delimiter(left,\ right)\ } : Make a short hand for
  paired delimiters. This function returns a closure
  \texttt{\ f(size\ =\ auto:\ auto\ \textbar{}\ none\ \textbar{}\ big\ \textbar{}\ Big\ \textbar{}\ bigg\ \textbar{}\ Bigg\ \textbar{}\ relative,\ content:\ content)\ }
  . The keyed argument \texttt{\ size\ } is optional and defaults to
  \texttt{\ auto\ } . The positional argument \texttt{\ content\ } is
  required.

  \begin{itemize}
  \tightlist
  \item
    when \texttt{\ size\ } is \texttt{\ auto\ } , the size of the
    delimiters is automatically determined.
  \item
    when \texttt{\ size\ } is \texttt{\ none\ } , the size of the
    delimiters is \texttt{\ 1em\ } .
  \item
    when \texttt{\ size\ } is \texttt{\ big\ } / \texttt{\ Big\ } /
    \texttt{\ bigg\ } / \texttt{\ Bigg\ } , the size of the delimiters
    is set to \texttt{\ big\ } / \texttt{\ Big\ } / \texttt{\ bigg\ } /
    \texttt{\ Bigg\ } respectively.
  \item
    when \texttt{\ size\ } is \texttt{\ relative\ } length like
    \texttt{\ 3em\ } or \texttt{\ 150\%\ } , the size of the delimiters
    is scaled by the factor you provide.
  \end{itemize}
\end{itemize}

Example:

\begin{Shaded}
\begin{Highlighting}[]
\NormalTok{\#let parn = paired{-}delimiter("(", ")")}

\NormalTok{$}
\NormalTok{parn(size: bigg,}
\NormalTok{  parn(size: big, (a+b)times (a{-}b))}
\NormalTok{div}
\NormalTok{  parn(size: big, (c+d)times (c{-}d))}
\NormalTok{) + d \textbackslash{} = (a\^{}2{-}b\^{}2) / (c\^{}2{-}d\^{}2)+d}
\NormalTok{$}
\end{Highlighting}
\end{Shaded}

\pandocbounded{\includesvg[keepaspectratio]{https://github.com/typst/packages/raw/main/packages/preview/delimitizer/0.1.0/demo.svg}}

\subsubsection{How to add}\label{how-to-add}

Copy this into your project and use the import as
\texttt{\ delimitizer\ }

\begin{verbatim}
#import "@preview/delimitizer:0.1.0"
\end{verbatim}

\includesvg[width=0.16667in,height=0.16667in]{/assets/icons/16-copy.svg}

Check the docs for
\href{https://typst.app/docs/reference/scripting/\#packages}{more
information on how to import packages} .

\subsubsection{About}\label{about}

\begin{description}
\tightlist
\item[Author :]
Wenzhuo Liu
\item[License:]
MIT
\item[Current version:]
0.1.0
\item[Last updated:]
May 1, 2024
\item[First released:]
May 1, 2024
\item[Archive size:]
2.10 kB
\href{https://packages.typst.org/preview/delimitizer-0.1.0.tar.gz}{\pandocbounded{\includesvg[keepaspectratio]{/assets/icons/16-download.svg}}}
\item[Repository:]
\href{https://github.com/Enter-tainer/delimitizer}{GitHub}
\end{description}

\subsubsection{Where to report issues?}\label{where-to-report-issues}

This package is a project of Wenzhuo Liu . Report issues on
\href{https://github.com/Enter-tainer/delimitizer}{their repository} .
You can also try to ask for help with this package on the
\href{https://forum.typst.app}{Forum} .

Please report this package to the Typst team using the
\href{https://typst.app/contact}{contact form} if you believe it is a
safety hazard or infringes upon your rights.

\phantomsection\label{versions}
\subsubsection{Version history}\label{version-history}

\begin{longtable}[]{@{}ll@{}}
\toprule\noalign{}
Version & Release Date \\
\midrule\noalign{}
\endhead
\bottomrule\noalign{}
\endlastfoot
0.1.0 & May 1, 2024 \\
\end{longtable}

Typst GmbH did not create this package and cannot guarantee correct
functionality of this package or compatibility with any version of the
Typst compiler or app.


\section{Package List LaTeX/vienna-tech.tex}
\title{typst.app/universe/package/vienna-tech}

\phantomsection\label{banner}
\phantomsection\label{template-thumbnail}
\pandocbounded{\includegraphics[keepaspectratio]{https://packages.typst.org/preview/thumbnails/vienna-tech-0.1.2-small.webp}}

\section{vienna-tech}\label{vienna-tech}

{ 0.1.2 }

An unofficial template for writing thesis at the TU Wien civil- and
environmental engineering faculty.

\href{/app?template=vienna-tech&version=0.1.2}{Create project in app}

\phantomsection\label{readme}
Version 0.1.2

This is a template, modeled after the LaTeX template provided by the
Vienna University of Technology for Engineering Students. It is intended
to be used as a starting point for writing Bachelor’s or Master’s
theses, but can be adapted for other purposes as well. It shall be noted
that this template is not an official template provided by the Vienna
University of Technology, but rather a personal effort to provide a
similar template in a new typesetting system. If you want to checkout
the original templates visit the website of
\href{https://www.tuwien.at/cee/edvlabor/lehre/vorlagen}{TU Wien}

\subsection{Getting Started}\label{getting-started}

These instructions will help you set up the template on the typst web
app.

\begin{Shaded}
\begin{Highlighting}[]
\NormalTok{\#import "@preview/vienna{-}tech:0.1.2": *}

\NormalTok{// Useing the configuration}
\NormalTok{\#show: tuw{-}thesis.with(}
\NormalTok{  title: [Titel of the Thesis],}
\NormalTok{  thesis{-}type: [Bachelorthesis],}
\NormalTok{  lang: "de",}
\NormalTok{  authors: (}
\NormalTok{    (}
\NormalTok{      name: "Firstname Lastname", }
\NormalTok{      email: "email@email.com",}
\NormalTok{      matrnr: "12345678",}
\NormalTok{      date: datetime.today().display("[day] [month repr:long] [year]"),}
\NormalTok{    ),}
\NormalTok{  ),}
\NormalTok{  abstract: [The Abstract of the Thesis],}
\NormalTok{  bibliography: bibliography("bibliography.bib"), }
\NormalTok{  appendix: [The Appendix of the Thesis], }
\NormalTok{    )}
\end{Highlighting}
\end{Shaded}

\subsection{Options}\label{options}

All the available options that are available for the template are listed
below.

\begin{longtable}[]{@{}lll@{}}
\toprule\noalign{}
Parameter & Type & Description \\
\midrule\noalign{}
\endhead
\bottomrule\noalign{}
\endlastfoot
\texttt{\ title\ } & \texttt{\ content\ } & Title of the thesis. \\
\texttt{\ thesis-type\ } & \texttt{\ content\ } & Type of thesis (e.g.,
Bachelor’s thesis, Master’s thesis). \\
\texttt{\ authors\ } & \texttt{\ content\ } ; \texttt{\ string\ } ;
\texttt{\ array\ } & Name of the author(s) as text or array. \\
\texttt{\ abstract\ } & \texttt{\ content\ } & Abstract of the
thesis. \\
\texttt{\ papersize\ } & \texttt{\ string\ } & Paper size (e.g., A4,
Letter). \\
\texttt{\ bibliography\ } & \texttt{\ bibliography\ } & Bibliography
section. \\
\texttt{\ lang\ } & \texttt{\ string\ } & Language of the thesis (e.g.,
“en� for English, “de� for German). \\
\texttt{\ appendix\ } & \texttt{\ content\ } & Appendix of the
thesis. \\
\texttt{\ toc\ } & \texttt{\ bool\ } & Show table of contents (
\texttt{\ true\ } or \texttt{\ false\ } ). \\
\texttt{\ font-size\ } & \texttt{\ length\ } & Font size for the main
text. \\
\texttt{\ main-font\ } & \texttt{\ string\ } ; \texttt{\ array\ } & Main
font as a name or an array of font names. \\
\texttt{\ title-font\ } & \texttt{\ string\ } ; \texttt{\ array\ } &
Font for the title as a name or an array of font names. \\
\texttt{\ raw-font\ } & \texttt{\ string\ } ; \texttt{\ array\ } & Font
for specific text as a name or an array of font names. \\
\texttt{\ title-page\ } & \texttt{\ content\ } & Content of the title
page. \\
\texttt{\ paper-margins\ } & \texttt{\ auto\ } ; \texttt{\ relative\ } ;
\texttt{\ dictionary\ } & Margins of the document. Can be set as
automatic, relative, or defined by a dictionary. \\
\texttt{\ title-hyphenation\ } & \texttt{\ auto\ } ; \texttt{\ bool\ } &
Title hyphenation, either automatic ( \texttt{\ auto\ } ) or manual (
\texttt{\ true\ } or \texttt{\ false\ } ). \\
\end{longtable}

\subsection{Usage}\label{usage}

These instructions will get you a copy of the project up and running on
the typst web app.

\begin{Shaded}
\begin{Highlighting}[]
\ExtensionTok{typst}\NormalTok{ init @preview/vienna{-}tech:0.1.2}
\end{Highlighting}
\end{Shaded}

\subsubsection{Template overview}\label{template-overview}

After setting up the template, you will have the following files:

\begin{itemize}
\tightlist
\item
  \texttt{\ main.typ\ } : the file which is used to compile the document
\item
  \texttt{\ abstract.typ\ } : a file where you can put your abstract
  text
\item
  \texttt{\ appendix.typ\ } : a file where you can put your appendix
  text
\item
  \texttt{\ sections.typ\ } : a file which can include all your contents
\item
  \texttt{\ refs.bib\ } : references
\end{itemize}

\subsection{Contribute to the
template}\label{contribute-to-the-template}

Feel free to contribute to the template by opening a pull request. If
you have any questions, feel free to open an issue.

\href{/app?template=vienna-tech&version=0.1.2}{Create project in app}

\subsubsection{How to use}\label{how-to-use}

Click the button above to create a new project using this template in
the Typst app.

You can also use the Typst CLI to start a new project on your computer
using this command:

\begin{verbatim}
typst init @preview/vienna-tech:0.1.2
\end{verbatim}

\includesvg[width=0.16667in,height=0.16667in]{/assets/icons/16-copy.svg}

\subsubsection{About}\label{about}

\begin{description}
\tightlist
\item[Author :]
Niko Pikall
\item[License:]
Unlicense
\item[Current version:]
0.1.2
\item[Last updated:]
November 6, 2024
\item[First released:]
August 23, 2024
\item[Archive size:]
66.3 kB
\href{https://packages.typst.org/preview/vienna-tech-0.1.2.tar.gz}{\pandocbounded{\includesvg[keepaspectratio]{/assets/icons/16-download.svg}}}
\item[Repository:]
\href{https://github.com/npikall/vienna-tech.git}{GitHub}
\item[Categor y :]
\begin{itemize}
\tightlist
\item[]
\item
  \pandocbounded{\includesvg[keepaspectratio]{/assets/icons/16-mortarboard.svg}}
  \href{https://typst.app/universe/search/?category=thesis}{Thesis}
\end{itemize}
\end{description}

\subsubsection{Where to report issues?}\label{where-to-report-issues}

This template is a project of Niko Pikall . Report issues on
\href{https://github.com/npikall/vienna-tech.git}{their repository} .
You can also try to ask for help with this template on the
\href{https://forum.typst.app}{Forum} .

Please report this template to the Typst team using the
\href{https://typst.app/contact}{contact form} if you believe it is a
safety hazard or infringes upon your rights.

\phantomsection\label{versions}
\subsubsection{Version history}\label{version-history}

\begin{longtable}[]{@{}ll@{}}
\toprule\noalign{}
Version & Release Date \\
\midrule\noalign{}
\endhead
\bottomrule\noalign{}
\endlastfoot
0.1.2 & November 6, 2024 \\
\href{https://typst.app/universe/package/vienna-tech/0.1.1/}{0.1.1} &
August 27, 2024 \\
\href{https://typst.app/universe/package/vienna-tech/0.1.0/}{0.1.0} &
August 23, 2024 \\
\end{longtable}

Typst GmbH did not create this template and cannot guarantee correct
functionality of this template or compatibility with any version of the
Typst compiler or app.


\section{Package List LaTeX/superb-pci.tex}
\title{typst.app/universe/package/superb-pci}

\phantomsection\label{banner}
\phantomsection\label{template-thumbnail}
\pandocbounded{\includegraphics[keepaspectratio]{https://packages.typst.org/preview/thumbnails/superb-pci-0.1.0-small.webp}}

\section{superb-pci}\label{superb-pci}

{ 0.1.0 }

A Peer Community In (PCI) and Peer Community Journal (PCJ) template.

\href{/app?template=superb-pci&version=0.1.0}{Create project in app}

\phantomsection\label{readme}
Template for \href{https://peercommunityin.org/}{Peer Community In}
(PCI) submission and \href{https://peercommunityjournal.org/}{Peer
Community Journal} (PCJ) post-recommendation upload.

The template is as close as possible to the LaTeX one.

\subsection{Usage}\label{usage}

To use this template in Typst, simply import it at the top of your
document.

\begin{verbatim}
#import "@preview/superb-pci:0.1.0": *
\end{verbatim}

Alternatively, you can start using this template from the command line
with

\begin{verbatim}
typst init @preview/superb-pci:0.1.0 my-superb-manuscript-dir
\end{verbatim}

or directly in the web app by clicking “Start from template�.

Please see the main Readme about Typst packages
\url{https://github.com/typst/packages} .

\subsection{Configuration}\label{configuration}

This template exports the \texttt{\ pci\ } function with the following
named arguments:

\begin{itemize}
\tightlist
\item
  \texttt{\ title\ } : the paper title
\item
  \texttt{\ authors\ } : array of author dictionaries. Each author must
  have the \texttt{\ name\ } field, and can have the optional fields
  \texttt{\ orcid\ } , and \texttt{\ affiliations\ } .
\item
  \texttt{\ affiliations\ } : array of affiliation dictionaries, each
  with the keys \texttt{\ id\ } and \texttt{\ name\ } . All
  correspondence between authors and affiliations is done manually.
\item
  \texttt{\ abstract\ } : abstract of the paper as content
\item
  \texttt{\ doi\ } : DOI of the paper displayed on the front page
\item
  \texttt{\ keywords\ } : array of keywords displayed on the front page
\item
  \texttt{\ correspondence\ } : corresponding address displayed on the
  front page
\item
  \texttt{\ numbered\_sections\ } : boolean, whether sections should be
  numbered
\item
  \texttt{\ pcj\ } : boolean, provides a way to remove the front page
  and headers/footers for upload to the Peer Community Journal.
  \texttt{\ {[}default:\ false{]}\ }
\end{itemize}

The template will initialize your folder with a sample call to the
\texttt{\ pci\ } function in a show rule and dummy content as an
example. If you want to change an existing project to use this template,
you can add a show rule like this at the top of your file:

\begin{Shaded}
\begin{Highlighting}[]
\NormalTok{\#import "@preview/superb{-}pci:0.1.0": *}

\NormalTok{\#show: pci.with(}
\NormalTok{  title: [Sample for the template, with quite a very long title],}
\NormalTok{  abstract: lorem(200),}
\NormalTok{  authors: (}
\NormalTok{    (}
\NormalTok{      name: "Antoine Lavoisier",}
\NormalTok{      orcid: "0000{-}0000{-}0000{-}0001",}
\NormalTok{      affiliations: "\#,1"}
\NormalTok{    ),}
\NormalTok{    (}
\NormalTok{      name: "Mary P. Curry",}
\NormalTok{      orcid: "0000{-}0000{-}0000{-}0001",}
\NormalTok{      affiliations: "\#,2",}
\NormalTok{    ),}
\NormalTok{    (}
\NormalTok{      name: "Peter Curry",}
\NormalTok{      affiliations: "2",}
\NormalTok{    ),}
\NormalTok{    (}
\NormalTok{      name: "Dick Darlington",}
\NormalTok{      orcid: "0000{-}0000{-}0000{-}0001",}
\NormalTok{      affiliations: "1,3"}
\NormalTok{    ),}
\NormalTok{  ),}
\NormalTok{  affiliations: (}
\NormalTok{   (id: "1", name: "Rue sans aplomb, Paris, France"),}
\NormalTok{   (id: "2", name: "Center for spiced radium experiments, United Kingdom"),}
\NormalTok{   (id: "3", name: "Bruce\textquotesingle{}s Bar and Grill, London (near Susan\textquotesingle{}s)"),}
\NormalTok{   (id: "\#", name: "Equal contributions"),}
\NormalTok{  ),}
\NormalTok{  doi: "https://doi.org/10.5802/fake.doi",}
\NormalTok{  keywords: ("Scientific writing", "Typst", "PCI", "Example"),}
\NormalTok{  correspondence: "a{-}lavois@lead{-}free{-}univ.edu",}
\NormalTok{  numbered\_sections: false,}
\NormalTok{  pcj: false,}
\NormalTok{)}

\NormalTok{// Your content goes here}
\end{Highlighting}
\end{Shaded}

You might also need to use the \texttt{\ table\_note\ } function from
the template.

\subsection{To do}\label{to-do}

Some things that are not straightforward in Typst yet that need to be
added in the futures:

\begin{itemize}
\tightlist
\item
  {[} {]} line numbers
\item
  {[} {]} switch equation numbers to the left
\end{itemize}

\href{/app?template=superb-pci&version=0.1.0}{Create project in app}

\subsubsection{How to use}\label{how-to-use}

Click the button above to create a new project using this template in
the Typst app.

You can also use the Typst CLI to start a new project on your computer
using this command:

\begin{verbatim}
typst init @preview/superb-pci:0.1.0
\end{verbatim}

\includesvg[width=0.16667in,height=0.16667in]{/assets/icons/16-copy.svg}

\subsubsection{About}\label{about}

\begin{description}
\tightlist
\item[Author :]
Alexis Simon
\item[License:]
MIT-0
\item[Current version:]
0.1.0
\item[Last updated:]
April 15, 2024
\item[First released:]
April 15, 2024
\item[Minimum Typst version:]
0.11.0
\item[Archive size:]
170 kB
\href{https://packages.typst.org/preview/superb-pci-0.1.0.tar.gz}{\pandocbounded{\includesvg[keepaspectratio]{/assets/icons/16-download.svg}}}
\item[Repository:]
\href{https://codeberg.org/alxsim/superb-pci}{Codeberg}
\item[Discipline s :]
\begin{itemize}
\tightlist
\item[]
\item
  \href{https://typst.app/universe/search/?discipline=biology}{Biology}
\item
  \href{https://typst.app/universe/search/?discipline=archaeology}{Archaeology}
\end{itemize}
\item[Categor y :]
\begin{itemize}
\tightlist
\item[]
\item
  \pandocbounded{\includesvg[keepaspectratio]{/assets/icons/16-atom.svg}}
  \href{https://typst.app/universe/search/?category=paper}{Paper}
\end{itemize}
\end{description}

\subsubsection{Where to report issues?}\label{where-to-report-issues}

This template is a project of Alexis Simon . Report issues on
\href{https://codeberg.org/alxsim/superb-pci}{their repository} . You
can also try to ask for help with this template on the
\href{https://forum.typst.app}{Forum} .

Please report this template to the Typst team using the
\href{https://typst.app/contact}{contact form} if you believe it is a
safety hazard or infringes upon your rights.

\phantomsection\label{versions}
\subsubsection{Version history}\label{version-history}

\begin{longtable}[]{@{}ll@{}}
\toprule\noalign{}
Version & Release Date \\
\midrule\noalign{}
\endhead
\bottomrule\noalign{}
\endlastfoot
0.1.0 & April 15, 2024 \\
\end{longtable}

Typst GmbH did not create this template and cannot guarantee correct
functionality of this template or compatibility with any version of the
Typst compiler or app.


\section{Package List LaTeX/tabut.tex}
\title{typst.app/universe/package/tabut}

\phantomsection\label{banner}
\section{tabut}\label{tabut}

{ 1.0.2 }

Display data as tables.

\phantomsection\label{readme}
\emph{Powerful, Simple, Concise}

A Typst plugin for turning data into tables.

\subsection{Outline}\label{outline}

\begin{itemize}
\item
  \href{https://github.com/typst/packages/raw/main/packages/preview/tabut/1.0.2/\#examples}{Examples}

  \begin{itemize}
  \item
    \href{https://github.com/typst/packages/raw/main/packages/preview/tabut/1.0.2/\#input-format-and-creation}{Input
    Format and Creation}
  \item
    \href{https://github.com/typst/packages/raw/main/packages/preview/tabut/1.0.2/\#basic-table}{Basic
    Table}
  \item
    \href{https://github.com/typst/packages/raw/main/packages/preview/tabut/1.0.2/\#table-styling}{Table
    Styling}
  \item
    \href{https://github.com/typst/packages/raw/main/packages/preview/tabut/1.0.2/\#header-formatting}{Header
    Formatting}
  \item
    \href{https://github.com/typst/packages/raw/main/packages/preview/tabut/1.0.2/\#remove-headers}{Remove
    Headers}
  \item
    \href{https://github.com/typst/packages/raw/main/packages/preview/tabut/1.0.2/\#cell-expressions-and-formatting}{Cell
    Expressions and Formatting}
  \item
    \href{https://github.com/typst/packages/raw/main/packages/preview/tabut/1.0.2/\#index}{Index}
  \item
    \href{https://github.com/typst/packages/raw/main/packages/preview/tabut/1.0.2/\#transpose}{Transpose}
  \item
    \href{https://github.com/typst/packages/raw/main/packages/preview/tabut/1.0.2/\#alignment}{Alignment}
  \item
    \href{https://github.com/typst/packages/raw/main/packages/preview/tabut/1.0.2/\#column-width}{Column
    Width}
  \item
    \href{https://github.com/typst/packages/raw/main/packages/preview/tabut/1.0.2/\#get-cells-only}{Get
    Cells Only}
  \item
    \href{https://github.com/typst/packages/raw/main/packages/preview/tabut/1.0.2/\#use-with-tablex}{Use
    with Tablex}
  \end{itemize}
\item
  \href{https://github.com/typst/packages/raw/main/packages/preview/tabut/1.0.2/\#data-operation-examples}{Data
  Operation Examples}

  \begin{itemize}
  \item
    \href{https://github.com/typst/packages/raw/main/packages/preview/tabut/1.0.2/\#csv-data}{CSV
    Data}
  \item
    \href{https://github.com/typst/packages/raw/main/packages/preview/tabut/1.0.2/\#slice}{Slice}
  \item
    \href{https://github.com/typst/packages/raw/main/packages/preview/tabut/1.0.2/\#sorting-and-reversing}{Sorting
    and Reversing}
  \item
    \href{https://github.com/typst/packages/raw/main/packages/preview/tabut/1.0.2/\#filter}{Filter}
  \item
    \href{https://github.com/typst/packages/raw/main/packages/preview/tabut/1.0.2/\#aggregation-using-map-and-sum}{Aggregation
    using Map and Sum}
  \item
    \href{https://github.com/typst/packages/raw/main/packages/preview/tabut/1.0.2/\#grouping}{Grouping}
  \end{itemize}
\item
  \href{https://github.com/typst/packages/raw/main/packages/preview/tabut/1.0.2/\#function-definitions}{Function
  Definitions}

  \begin{itemize}
  \item
    \href{https://github.com/typst/packages/raw/main/packages/preview/tabut/1.0.2/\#tabut}{\texttt{\ tabut\ }}
  \item
    \href{https://github.com/typst/packages/raw/main/packages/preview/tabut/1.0.2/\#tabut-cells}{\texttt{\ tabut-cells\ }}
  \item
    \href{https://github.com/typst/packages/raw/main/packages/preview/tabut/1.0.2/\#rows-to-records}{\texttt{\ rows-to-records\ }}
  \item
    \href{https://github.com/typst/packages/raw/main/packages/preview/tabut/1.0.2/\#records-from-csv}{\texttt{\ records-from-csv\ }}
  \item
    \href{https://github.com/typst/packages/raw/main/packages/preview/tabut/1.0.2/\#group}{\texttt{\ group\ }}
  \end{itemize}
\end{itemize}

\subsection[Input Format and Creation ]{\texorpdfstring{Input Format and
Creation \protect\hypertarget{input-format-and-creation}{}{
}}{Input Format and Creation  }}\label{input-format-and-creation}

The \texttt{\ tabut\ } function takes input in “record� format, an
array of dictionaries, with each dictionary representing a single
“object� or “record�.

In the example below, each record is a listing for an office supply
product.

\begin{Shaded}
\begin{Highlighting}[]
\NormalTok{\#let supplies = (}
\NormalTok{  (name: "Notebook", price: 3.49, quantity: 5),}
\NormalTok{  (name: "Ballpoint Pens", price: 5.99, quantity: 2),}
\NormalTok{  (name: "Printer Paper", price: 6.99, quantity: 3),}
\NormalTok{)}
\end{Highlighting}
\end{Shaded}

\subsection[Basic Table ]{\texorpdfstring{Basic Table
\protect\hypertarget{basic-table}{}{
}}{Basic Table  }}\label{basic-table}

Now create a basic table from the data.

\begin{Shaded}
\begin{Highlighting}[]
\NormalTok{\#import "@preview/tabut:1.0.2": tabut}
\NormalTok{\#import "example{-}data/supplies.typ": supplies}

\NormalTok{\#tabut(}
\NormalTok{  supplies, // the source of the data used to generate the table}
\NormalTok{  ( // column definitions}
\NormalTok{    (}
\NormalTok{      header: [Name], // label, takes content.}
\NormalTok{      func: r =\textgreater{} r.name // generates the cell content.}
\NormalTok{    ), }
\NormalTok{    (header: [Price], func: r =\textgreater{} r.price), }
\NormalTok{    (header: [Quantity], func: r =\textgreater{} r.quantity), }
\NormalTok{  )}
\NormalTok{)}
\end{Highlighting}
\end{Shaded}

\pandocbounded{\includesvg[keepaspectratio]{https://github.com/typst/packages/raw/main/packages/preview/tabut/1.0.2/doc/compiled-snippets/basic.svg}}

\texttt{\ funct\ } takes a function which generates content for a given
cell corrosponding to the defined column for each record. \texttt{\ r\ }
is the record, so \texttt{\ r\ =\textgreater{}\ r.name\ } returns the
\texttt{\ name\ } property of each record in the input data if it has
one.

The philosphy of \texttt{\ tabut\ } is that the display of data should
be simple and clearly defined, therefore each column and it’s content
and formatting should be defined within a single clear column defintion.
One consequence is you can comment out, remove or move, any column
easily, for example:

\begin{Shaded}
\begin{Highlighting}[]
\NormalTok{\#import "@preview/tabut:1.0.2": tabut}
\NormalTok{\#import "example{-}data/supplies.typ": supplies}

\NormalTok{\#tabut(}
\NormalTok{  supplies,}
\NormalTok{  (}
\NormalTok{    (header: [Price], func: r =\textgreater{} r.price), // This column is moved to the front}
\NormalTok{    (header: [Name], func: r =\textgreater{} r.name), }
\NormalTok{    (header: [Name 2], func: r =\textgreater{} r.name), // copied}
\NormalTok{    // (header: [Quantity], func: r =\textgreater{} r.quantity), // removed via comment}
\NormalTok{  )}
\NormalTok{)}
\end{Highlighting}
\end{Shaded}

\pandocbounded{\includesvg[keepaspectratio]{https://github.com/typst/packages/raw/main/packages/preview/tabut/1.0.2/doc/compiled-snippets/rearrange.svg}}

\subsection[Table Styling ]{\texorpdfstring{Table Styling
\protect\hypertarget{table-styling}{}{
}}{Table Styling  }}\label{table-styling}

Any default Table style options can be tacked on and are passed to the
final table function.

\begin{Shaded}
\begin{Highlighting}[]
\NormalTok{\#import "@preview/tabut:1.0.2": tabut}
\NormalTok{\#import "example{-}data/supplies.typ": supplies}

\NormalTok{\#tabut(}
\NormalTok{  supplies,}
\NormalTok{  ( }
\NormalTok{    (header: [Name], func: r =\textgreater{} r.name), }
\NormalTok{    (header: [Price], func: r =\textgreater{} r.price), }
\NormalTok{    (header: [Quantity], func: r =\textgreater{} r.quantity),}
\NormalTok{  ),}
\NormalTok{  fill: (\_, row) =\textgreater{} if calc.odd(row) \{ luma(240) \} else \{ luma(220) \}, }
\NormalTok{  stroke: none}
\NormalTok{)}
\end{Highlighting}
\end{Shaded}

\pandocbounded{\includesvg[keepaspectratio]{https://github.com/typst/packages/raw/main/packages/preview/tabut/1.0.2/doc/compiled-snippets/styling.svg}}

\subsection[Header Formatting ]{\texorpdfstring{Header Formatting
\protect\hypertarget{header-formatting}{}{
}}{Header Formatting  }}\label{header-formatting}

You can pass any content or expression into the header property.

\begin{Shaded}
\begin{Highlighting}[]
\NormalTok{\#import "@preview/tabut:1.0.2": tabut}
\NormalTok{\#import "example{-}data/supplies.typ": supplies}

\NormalTok{\#let fmt(it) = \{}
\NormalTok{  heading(}
\NormalTok{    outlined: false,}
\NormalTok{    upper(it)}
\NormalTok{  )}
\NormalTok{\}}

\NormalTok{\#tabut(}
\NormalTok{  supplies,}
\NormalTok{  ( }
\NormalTok{    (header: fmt([Name]), func: r =\textgreater{} r.name ), }
\NormalTok{    (header: fmt([Price]), func: r =\textgreater{} r.price), }
\NormalTok{    (header: fmt([Quantity]), func: r =\textgreater{} r.quantity), }
\NormalTok{  ),}
\NormalTok{  fill: (\_, row) =\textgreater{} if calc.odd(row) \{ luma(240) \} else \{ luma(220) \}, }
\NormalTok{  stroke: none}
\NormalTok{)}
\end{Highlighting}
\end{Shaded}

\pandocbounded{\includesvg[keepaspectratio]{https://github.com/typst/packages/raw/main/packages/preview/tabut/1.0.2/doc/compiled-snippets/title.svg}}

\subsection[Remove Headers ]{\texorpdfstring{Remove Headers
\protect\hypertarget{remove-headers}{}{
}}{Remove Headers  }}\label{remove-headers}

You can prevent from being generated with the \texttt{\ headers\ }
paramater. This is useful with the \texttt{\ tabut-cells\ } function as
demonstrated in it’s section.

\begin{Shaded}
\begin{Highlighting}[]
\NormalTok{\#import "@preview/tabut:1.0.2": tabut}
\NormalTok{\#import "example{-}data/supplies.typ": supplies}

\NormalTok{\#tabut(}
\NormalTok{  supplies,}
\NormalTok{  (}
\NormalTok{    (header: [*Name*], func: r =\textgreater{} r.name), }
\NormalTok{    (header: [*Price*], func: r =\textgreater{} r.price), }
\NormalTok{    (header: [*Quantity*], func: r =\textgreater{} r.quantity), }
\NormalTok{  ),}
\NormalTok{  headers: false, // Prevents Headers from being generated}
\NormalTok{  fill: (\_, row) =\textgreater{} if calc.odd(row) \{ luma(240) \} else \{ luma(220) \}, }
\NormalTok{  stroke: none,}
\NormalTok{)}
\end{Highlighting}
\end{Shaded}

\pandocbounded{\includesvg[keepaspectratio]{https://github.com/typst/packages/raw/main/packages/preview/tabut/1.0.2/doc/compiled-snippets/no-headers.svg}}

\subsection[Cell Expressions and Formatting ]{\texorpdfstring{Cell
Expressions and Formatting
\protect\hypertarget{cell-expressions-and-formatting}{}{
}}{Cell Expressions and Formatting  }}\label{cell-expressions-and-formatting}

Just like the headers, cell contents can be modified and formatted like
any content in Typst.

\begin{Shaded}
\begin{Highlighting}[]
\NormalTok{\#import "@preview/tabut:1.0.2": tabut}
\NormalTok{\#import "usd.typ": usd}
\NormalTok{\#import "example{-}data/supplies.typ": supplies}

\NormalTok{\#tabut(}
\NormalTok{  supplies,}
\NormalTok{  ( }
\NormalTok{    (header: [*Name*], func: r =\textgreater{} r.name ), }
\NormalTok{    (header: [*Price*], func: r =\textgreater{} usd(r.price)), }
\NormalTok{  ),}
\NormalTok{  fill: (\_, row) =\textgreater{} if calc.odd(row) \{ luma(240) \} else \{ luma(220) \}, }
\NormalTok{  stroke: none}
\NormalTok{)}
\end{Highlighting}
\end{Shaded}

\pandocbounded{\includesvg[keepaspectratio]{https://github.com/typst/packages/raw/main/packages/preview/tabut/1.0.2/doc/compiled-snippets/format.svg}}

You can have the cell content function do calculations on a record
property.

\begin{Shaded}
\begin{Highlighting}[]
\NormalTok{\#import "@preview/tabut:1.0.2": tabut}
\NormalTok{\#import "usd.typ": usd}
\NormalTok{\#import "example{-}data/supplies.typ": supplies}

\NormalTok{\#tabut(}
\NormalTok{  supplies,}
\NormalTok{  ( }
\NormalTok{    (header: [*Name*], func: r =\textgreater{} r.name ), }
\NormalTok{    (header: [*Price*], func: r =\textgreater{} usd(r.price)), }
\NormalTok{    (header: [*Tax*], func: r =\textgreater{} usd(r.price * .2)), }
\NormalTok{    (header: [*Total*], func: r =\textgreater{} usd(r.price * 1.2)), }
\NormalTok{  ),}
\NormalTok{  fill: (\_, row) =\textgreater{} if calc.odd(row) \{ luma(240) \} else \{ luma(220) \}, }
\NormalTok{  stroke: none}
\NormalTok{)}
\end{Highlighting}
\end{Shaded}

\pandocbounded{\includesvg[keepaspectratio]{https://github.com/typst/packages/raw/main/packages/preview/tabut/1.0.2/doc/compiled-snippets/calculation.svg}}

Or even combine multiple record properties, go wild.

\begin{Shaded}
\begin{Highlighting}[]
\NormalTok{\#import "@preview/tabut:1.0.2": tabut}

\NormalTok{\#let employees = (}
\NormalTok{    (id: 3251, first: "Alice", last: "Smith", middle: "Jane"),}
\NormalTok{    (id: 4872, first: "Carlos", last: "Garcia", middle: "Luis"),}
\NormalTok{    (id: 5639, first: "Evelyn", last: "Chen", middle: "Ming")}
\NormalTok{);}

\NormalTok{\#tabut(}
\NormalTok{  employees,}
\NormalTok{  ( }
\NormalTok{    (header: [*ID*], func: r =\textgreater{} r.id ),}
\NormalTok{    (header: [*Full Name*], func: r =\textgreater{} [\#r.first \#r.middle.first(), \#r.last] ),}
\NormalTok{  ),}
\NormalTok{  fill: (\_, row) =\textgreater{} if calc.odd(row) \{ luma(240) \} else \{ luma(220) \}, }
\NormalTok{  stroke: none}
\NormalTok{)}
\end{Highlighting}
\end{Shaded}

\pandocbounded{\includesvg[keepaspectratio]{https://github.com/typst/packages/raw/main/packages/preview/tabut/1.0.2/doc/compiled-snippets/combine.svg}}

\subsection[Index ]{\texorpdfstring{Index \protect\hypertarget{index}{}{
}}{Index  }}\label{index}

\texttt{\ tabut\ } automatically adds an \texttt{\ \_index\ } property
to each record.

\begin{Shaded}
\begin{Highlighting}[]
\NormalTok{\#import "@preview/tabut:1.0.2": tabut}
\NormalTok{\#import "example{-}data/supplies.typ": supplies}

\NormalTok{\#tabut(}
\NormalTok{  supplies,}
\NormalTok{  ( }
\NormalTok{    (header: [*\textbackslash{}\#*], func: r =\textgreater{} r.\_index),}
\NormalTok{    (header: [*Name*], func: r =\textgreater{} r.name ), }
\NormalTok{  ),}
\NormalTok{  fill: (\_, row) =\textgreater{} if calc.odd(row) \{ luma(240) \} else \{ luma(220) \}, }
\NormalTok{  stroke: none}
\NormalTok{)}
\end{Highlighting}
\end{Shaded}

\pandocbounded{\includesvg[keepaspectratio]{https://github.com/typst/packages/raw/main/packages/preview/tabut/1.0.2/doc/compiled-snippets/index.svg}}

You can also prevent the \texttt{\ index\ } property being generated by
setting it to \texttt{\ none\ } , or you can also set an alternate name
of the index property as shown below.

\begin{Shaded}
\begin{Highlighting}[]
\NormalTok{\#import "@preview/tabut:1.0.2": tabut}
\NormalTok{\#import "example{-}data/supplies.typ": supplies}

\NormalTok{\#tabut(}
\NormalTok{  supplies,}
\NormalTok{  ( }
\NormalTok{    (header: [*\textbackslash{}\#*], func: r =\textgreater{} r.index{-}alt ),}
\NormalTok{    (header: [*Name*], func: r =\textgreater{} r.name ), }
\NormalTok{  ),}
\NormalTok{  index: "index{-}alt", // set an aternate name for the automatically generated index property.}
\NormalTok{  fill: (\_, row) =\textgreater{} if calc.odd(row) \{ luma(240) \} else \{ luma(220) \}, }
\NormalTok{  stroke: none}
\NormalTok{)}
\end{Highlighting}
\end{Shaded}

\pandocbounded{\includesvg[keepaspectratio]{https://github.com/typst/packages/raw/main/packages/preview/tabut/1.0.2/doc/compiled-snippets/index-alternate.svg}}

\subsection[Transpose ]{\texorpdfstring{Transpose
\protect\hypertarget{transpose}{}{ }}{Transpose  }}\label{transpose}

This was annoying to implement, and I don’t know when you’d actually
use this, but here.

\begin{Shaded}
\begin{Highlighting}[]
\NormalTok{\#import "@preview/tabut:1.0.2": tabut}
\NormalTok{\#import "usd.typ": usd}
\NormalTok{\#import "example{-}data/supplies.typ": supplies}

\NormalTok{\#tabut(}
\NormalTok{  supplies,}
\NormalTok{  (}
\NormalTok{    (header: [*\textbackslash{}\#*], func: r =\textgreater{} r.\_index),}
\NormalTok{    (header: [*Name*], func: r =\textgreater{} r.name), }
\NormalTok{    (header: [*Price*], func: r =\textgreater{} usd(r.price)), }
\NormalTok{    (header: [*Quantity*], func: r =\textgreater{} r.quantity),}
\NormalTok{  ),}
\NormalTok{  transpose: true,  // set optional name arg \textasciigrave{}transpose\textasciigrave{} to \textasciigrave{}true\textasciigrave{}}
\NormalTok{  fill: (\_, row) =\textgreater{} if calc.odd(row) \{ luma(240) \} else \{ luma(220) \}, }
\NormalTok{  stroke: none}
\NormalTok{)}
\end{Highlighting}
\end{Shaded}

\pandocbounded{\includesvg[keepaspectratio]{https://github.com/typst/packages/raw/main/packages/preview/tabut/1.0.2/doc/compiled-snippets/transpose.svg}}

\subsection[Alignment ]{\texorpdfstring{Alignment
\protect\hypertarget{alignment}{}{ }}{Alignment  }}\label{alignment}

\begin{Shaded}
\begin{Highlighting}[]
\NormalTok{\#import "@preview/tabut:1.0.2": tabut}
\NormalTok{\#import "usd.typ": usd}
\NormalTok{\#import "example{-}data/supplies.typ": supplies}

\NormalTok{\#tabut(}
\NormalTok{  supplies,}
\NormalTok{  ( // Include \textasciigrave{}align\textasciigrave{} as an optional arg to a column def}
\NormalTok{    (header: [*\textbackslash{}\#*], func: r =\textgreater{} r.\_index),}
\NormalTok{    (header: [*Name*], align: right, func: r =\textgreater{} r.name), }
\NormalTok{    (header: [*Price*], align: right, func: r =\textgreater{} usd(r.price)), }
\NormalTok{    (header: [*Quantity*], align: right, func: r =\textgreater{} r.quantity),}
\NormalTok{  ),}
\NormalTok{  fill: (\_, row) =\textgreater{} if calc.odd(row) \{ luma(240) \} else \{ luma(220) \}, }
\NormalTok{  stroke: none}
\NormalTok{)}
\end{Highlighting}
\end{Shaded}

\pandocbounded{\includesvg[keepaspectratio]{https://github.com/typst/packages/raw/main/packages/preview/tabut/1.0.2/doc/compiled-snippets/align.svg}}

You can also define Alignment manually as in the the standard Table
Function.

\begin{Shaded}
\begin{Highlighting}[]
\NormalTok{\#import "@preview/tabut:1.0.2": tabut}
\NormalTok{\#import "usd.typ": usd}
\NormalTok{\#import "example{-}data/supplies.typ": supplies}

\NormalTok{\#tabut(}
\NormalTok{  supplies,}
\NormalTok{  ( }
\NormalTok{    (header: [*\textbackslash{}\#*], func: r =\textgreater{} r.\_index),}
\NormalTok{    (header: [*Name*], func: r =\textgreater{} r.name), }
\NormalTok{    (header: [*Price*], func: r =\textgreater{} usd(r.price)), }
\NormalTok{    (header: [*Quantity*], func: r =\textgreater{} r.quantity),}
\NormalTok{  ),}
\NormalTok{  align: (auto, right, right, right), // Alignment defined as in standard table function}
\NormalTok{  fill: (\_, row) =\textgreater{} if calc.odd(row) \{ luma(240) \} else \{ luma(220) \}, }
\NormalTok{  stroke: none}
\NormalTok{)}
\end{Highlighting}
\end{Shaded}

\pandocbounded{\includesvg[keepaspectratio]{https://github.com/typst/packages/raw/main/packages/preview/tabut/1.0.2/doc/compiled-snippets/align-manual.svg}}

\subsection[Column Width ]{\texorpdfstring{Column Width
\protect\hypertarget{column-width}{}{
}}{Column Width  }}\label{column-width}

\begin{Shaded}
\begin{Highlighting}[]
\NormalTok{\#import "@preview/tabut:1.0.2": tabut}
\NormalTok{\#import "usd.typ": usd}
\NormalTok{\#import "example{-}data/supplies.typ": supplies}

\NormalTok{\#box(}
\NormalTok{  width: 300pt,}
\NormalTok{  tabut(}
\NormalTok{    supplies,}
\NormalTok{    ( // Include \textasciigrave{}width\textasciigrave{} as an optional arg to a column def}
\NormalTok{      (header: [*\textbackslash{}\#*], func: r =\textgreater{} r.\_index),}
\NormalTok{      (header: [*Name*], width: 1fr, func: r =\textgreater{} r.name), }
\NormalTok{      (header: [*Price*], width: 20\%, func: r =\textgreater{} usd(r.price)), }
\NormalTok{      (header: [*Quantity*], width: 1.5in, func: r =\textgreater{} r.quantity),}
\NormalTok{    ),}
\NormalTok{    fill: (\_, row) =\textgreater{} if calc.odd(row) \{ luma(240) \} else \{ luma(220) \}, }
\NormalTok{    stroke: none,}
\NormalTok{  )}
\NormalTok{)}
\end{Highlighting}
\end{Shaded}

\pandocbounded{\includesvg[keepaspectratio]{https://github.com/typst/packages/raw/main/packages/preview/tabut/1.0.2/doc/compiled-snippets/width.svg}}

You can also define Columns manually as in the the standard Table
Function.

\begin{Shaded}
\begin{Highlighting}[]
\NormalTok{\#import "@preview/tabut:1.0.2": tabut}
\NormalTok{\#import "usd.typ": usd}
\NormalTok{\#import "example{-}data/supplies.typ": supplies}

\NormalTok{\#box(}
\NormalTok{  width: 300pt,}
\NormalTok{  tabut(}
\NormalTok{    supplies,}
\NormalTok{    (}
\NormalTok{      (header: [*\textbackslash{}\#*], func: r =\textgreater{} r.\_index),}
\NormalTok{      (header: [*Name*], func: r =\textgreater{} r.name), }
\NormalTok{      (header: [*Price*], func: r =\textgreater{} usd(r.price)), }
\NormalTok{      (header: [*Quantity*], func: r =\textgreater{} r.quantity),}
\NormalTok{    ),}
\NormalTok{    columns: (auto, 1fr, 20\%, 1.5in),  // Columns defined as in standard table}
\NormalTok{    fill: (\_, row) =\textgreater{} if calc.odd(row) \{ luma(240) \} else \{ luma(220) \}, }
\NormalTok{    stroke: none,}
\NormalTok{  )}
\NormalTok{)}
\end{Highlighting}
\end{Shaded}

\pandocbounded{\includesvg[keepaspectratio]{https://github.com/typst/packages/raw/main/packages/preview/tabut/1.0.2/doc/compiled-snippets/width-manual.svg}}

\subsection[Get Cells Only ]{\texorpdfstring{Get Cells Only
\protect\hypertarget{get-cells-only}{}{
}}{Get Cells Only  }}\label{get-cells-only}

\begin{Shaded}
\begin{Highlighting}[]
\NormalTok{\#import "@preview/tabut:1.0.2": tabut{-}cells}
\NormalTok{\#import "usd.typ": usd}
\NormalTok{\#import "example{-}data/supplies.typ": supplies}

\NormalTok{\#tabut{-}cells(}
\NormalTok{  supplies,}
\NormalTok{  ( }
\NormalTok{    (header: [Name], func: r =\textgreater{} r.name), }
\NormalTok{    (header: [Price], func: r =\textgreater{} usd(r.price)), }
\NormalTok{    (header: [Quantity], func: r =\textgreater{} r.quantity),}
\NormalTok{  )}
\NormalTok{)}
\end{Highlighting}
\end{Shaded}

\pandocbounded{\includesvg[keepaspectratio]{https://github.com/typst/packages/raw/main/packages/preview/tabut/1.0.2/doc/compiled-snippets/only-cells.svg}}

\subsection[Use with Tablex ]{\texorpdfstring{Use with Tablex
\protect\hypertarget{use-with-tablex}{}{
}}{Use with Tablex  }}\label{use-with-tablex}

\begin{Shaded}
\begin{Highlighting}[]
\NormalTok{\#import "@preview/tabut:1.0.2": tabut{-}cells}
\NormalTok{\#import "usd.typ": usd}
\NormalTok{\#import "example{-}data/supplies.typ": supplies}

\NormalTok{\#import "@preview/tablex:0.0.8": tablex, rowspanx, colspanx}

\NormalTok{\#tablex(}
\NormalTok{  auto{-}vlines: false,}
\NormalTok{  header{-}rows: 2,}

\NormalTok{  /* {-}{-}{-} header {-}{-}{-} */}
\NormalTok{  rowspanx(2)[*Name*], colspanx(2)[*Price*], (), rowspanx(2)[*Quantity*],}
\NormalTok{  (),                 [*Base*], [*W/Tax*], (),}
\NormalTok{  /* {-}{-}{-}{-}{-}{-}{-}{-}{-}{-}{-}{-}{-}{-} */}

\NormalTok{  ..tabut{-}cells(}
\NormalTok{    supplies,}
\NormalTok{    ( }
\NormalTok{      (header: [], func: r =\textgreater{} r.name), }
\NormalTok{      (header: [], func: r =\textgreater{} usd(r.price)), }
\NormalTok{      (header: [], func: r =\textgreater{} usd(r.price * 1.3)), }
\NormalTok{      (header: [], func: r =\textgreater{} r.quantity),}
\NormalTok{    ),}
\NormalTok{    headers: false}
\NormalTok{  )}
\NormalTok{)}
\end{Highlighting}
\end{Shaded}

\pandocbounded{\includesvg[keepaspectratio]{https://github.com/typst/packages/raw/main/packages/preview/tabut/1.0.2/doc/compiled-snippets/tablex.svg}}

While technically seperate from table display, the following are
examples of how to perform operations on data before it is displayed
with \texttt{\ tabut\ } .

Since \texttt{\ tabut\ } assumes an “array of dictionaries� format,
then most data operations can be performed easily with Typst’s native
array functions. \texttt{\ tabut\ } also provides several functions to
provide additional functionality.

\subsection[CSV Data ]{\texorpdfstring{CSV Data
\protect\hypertarget{csv-data}{}{ }}{CSV Data  }}\label{csv-data}

By default, imported CSV gives a “rows� or “array of arrays�
data format, which can not be directly used by \texttt{\ tabut\ } . To
convert, \texttt{\ tabut\ } includes a function
\texttt{\ rows-to-records\ } demonstrated below.

\begin{Shaded}
\begin{Highlighting}[]
\NormalTok{\#import "@preview/tabut:1.0.2": tabut, rows{-}to{-}records}
\NormalTok{\#import "example{-}data/supplies.typ": supplies}

\NormalTok{\#let titanic = \{}
\NormalTok{  let titanic{-}raw = csv("example{-}data/titanic.csv");}
\NormalTok{  rows{-}to{-}records(}
\NormalTok{    titanic{-}raw.first(), // The header row}
\NormalTok{    titanic{-}raw.slice(1, {-}1), // The rest of the rows}
\NormalTok{  )}
\NormalTok{\}}
\end{Highlighting}
\end{Shaded}

Imported CSV data are all strings, so it’s usefull to convert them to
\texttt{\ int\ } or \texttt{\ float\ } when possible.

\begin{Shaded}
\begin{Highlighting}[]
\NormalTok{\#import "@preview/tabut:1.0.2": tabut, rows{-}to{-}records}
\NormalTok{\#import "example{-}data/supplies.typ": supplies}

\NormalTok{\#let auto{-}type(input) = \{}
\NormalTok{  let is{-}int = (input.match(regex("\^{}{-}?\textbackslash{}d+$")) != none);}
\NormalTok{  if is{-}int \{ return int(input); \}}
\NormalTok{  let is{-}float = (input.match(regex("\^{}{-}?(inf|nan|\textbackslash{}d+|\textbackslash{}d*(\textbackslash{}.\textbackslash{}d+))$")) != none);}
\NormalTok{  if is{-}float \{ return float(input) \}}
\NormalTok{  input}
\NormalTok{\}}

\NormalTok{\#let titanic = \{}
\NormalTok{  let titanic{-}raw = csv("example{-}data/titanic.csv");}
\NormalTok{  rows{-}to{-}records( titanic{-}raw.first(), titanic{-}raw.slice(1, {-}1) )}
\NormalTok{  .map( r =\textgreater{} \{}
\NormalTok{    let new{-}record = (:);}
\NormalTok{    for (k, v) in r.pairs() \{ new{-}record.insert(k, auto{-}type(v)); \}}
\NormalTok{    new{-}record}
\NormalTok{  \})}
\NormalTok{\}}
\end{Highlighting}
\end{Shaded}

\texttt{\ tabut\ } includes a function, \texttt{\ records-from-csv\ } ,
to automatically perform this process.

\begin{Shaded}
\begin{Highlighting}[]
\NormalTok{\#import "@preview/tabut:1.0.2": records{-}from{-}csv}

\NormalTok{\#let titanic = records{-}from{-}csv(csv("example{-}data/titanic.csv"));}
\end{Highlighting}
\end{Shaded}

\subsection[Slice ]{\texorpdfstring{Slice \protect\hypertarget{slice}{}{
}}{Slice  }}\label{slice}

\begin{Shaded}
\begin{Highlighting}[]
\NormalTok{\#import "@preview/tabut:1.0.2": tabut, records{-}from{-}csv}
\NormalTok{\#import "usd.typ": usd}
\NormalTok{\#import "example{-}data/titanic.typ": titanic}

\NormalTok{\#let classes = (}
\NormalTok{  "N/A",}
\NormalTok{  "First", }
\NormalTok{  "Second", }
\NormalTok{  "Third"}
\NormalTok{);}

\NormalTok{\#let titanic{-}head = titanic.slice(0, 5);}

\NormalTok{\#tabut(}
\NormalTok{  titanic{-}head,}
\NormalTok{  ( }
\NormalTok{    (header: [*Name*], func: r =\textgreater{} r.Name), }
\NormalTok{    (header: [*Class*], func: r =\textgreater{} classes.at(r.Pclass)),}
\NormalTok{    (header: [*Fare*], func: r =\textgreater{} usd(r.Fare)), }
\NormalTok{    (header: [*Survived?*], func: r =\textgreater{} ("No", "Yes").at(r.Survived)), }
\NormalTok{  ),}
\NormalTok{  fill: (\_, row) =\textgreater{} if calc.odd(row) \{ luma(240) \} else \{ luma(220) \}, }
\NormalTok{  stroke: none}
\NormalTok{)}
\end{Highlighting}
\end{Shaded}

\pandocbounded{\includesvg[keepaspectratio]{https://github.com/typst/packages/raw/main/packages/preview/tabut/1.0.2/doc/compiled-snippets/slice.svg}}

\subsection[Sorting and Reversing ]{\texorpdfstring{Sorting and
Reversing \protect\hypertarget{sorting-and-reversing}{}{
}}{Sorting and Reversing  }}\label{sorting-and-reversing}

\begin{Shaded}
\begin{Highlighting}[]
\NormalTok{\#import "@preview/tabut:1.0.2": tabut}
\NormalTok{\#import "usd.typ": usd}
\NormalTok{\#import "example{-}data/titanic.typ": titanic, classes}

\NormalTok{\#tabut(}
\NormalTok{  titanic}
\NormalTok{  .sorted(key: r =\textgreater{} r.Fare)}
\NormalTok{  .rev()}
\NormalTok{  .slice(0, 5),}
\NormalTok{  ( }
\NormalTok{    (header: [*Name*], func: r =\textgreater{} r.Name), }
\NormalTok{    (header: [*Class*], func: r =\textgreater{} classes.at(r.Pclass)),}
\NormalTok{    (header: [*Fare*], func: r =\textgreater{} usd(r.Fare)), }
\NormalTok{    (header: [*Survived?*], func: r =\textgreater{} ("No", "Yes").at(r.Survived)), }
\NormalTok{  ),}
\NormalTok{  fill: (\_, row) =\textgreater{} if calc.odd(row) \{ luma(240) \} else \{ luma(220) \}, }
\NormalTok{  stroke: none}
\NormalTok{)}
\end{Highlighting}
\end{Shaded}

\pandocbounded{\includesvg[keepaspectratio]{https://github.com/typst/packages/raw/main/packages/preview/tabut/1.0.2/doc/compiled-snippets/sort.svg}}

\subsection[Filter ]{\texorpdfstring{Filter
\protect\hypertarget{filter}{}{ }}{Filter  }}\label{filter}

\begin{Shaded}
\begin{Highlighting}[]
\NormalTok{\#import "@preview/tabut:1.0.2": tabut}
\NormalTok{\#import "usd.typ": usd}
\NormalTok{\#import "example{-}data/titanic.typ": titanic, classes}

\NormalTok{\#tabut(}
\NormalTok{  titanic}
\NormalTok{  .filter(r =\textgreater{} r.Pclass == 1)}
\NormalTok{  .slice(0, 5),}
\NormalTok{  ( }
\NormalTok{    (header: [*Name*], func: r =\textgreater{} r.Name), }
\NormalTok{    (header: [*Class*], func: r =\textgreater{} classes.at(r.Pclass)),}
\NormalTok{    (header: [*Fare*], func: r =\textgreater{} usd(r.Fare)), }
\NormalTok{    (header: [*Survived?*], func: r =\textgreater{} ("No", "Yes").at(r.Survived)), }
\NormalTok{  ),}
\NormalTok{  fill: (\_, row) =\textgreater{} if calc.odd(row) \{ luma(240) \} else \{ luma(220) \}, }
\NormalTok{  stroke: none}
\NormalTok{)}
\end{Highlighting}
\end{Shaded}

\pandocbounded{\includesvg[keepaspectratio]{https://github.com/typst/packages/raw/main/packages/preview/tabut/1.0.2/doc/compiled-snippets/filter.svg}}

\subsection[Aggregation using Map and Sum ]{\texorpdfstring{Aggregation
using Map and Sum \protect\hypertarget{aggregation-using-map-and-sum}{}{
}}{Aggregation using Map and Sum  }}\label{aggregation-using-map-and-sum}

\begin{Shaded}
\begin{Highlighting}[]
\NormalTok{\#import "usd.typ": usd}
\NormalTok{\#import "example{-}data/titanic.typ": titanic, classes}

\NormalTok{\#table(}
\NormalTok{  columns: (auto, auto),}
\NormalTok{  [*Fare, Total:*], [\#usd(titanic.map(r =\textgreater{} r.Fare).sum())],}
\NormalTok{  [*Fare, Avg:*], [\#usd(titanic.map(r =\textgreater{} r.Fare).sum() / titanic.len())], }
\NormalTok{  stroke: none}
\NormalTok{)}
\end{Highlighting}
\end{Shaded}

\pandocbounded{\includesvg[keepaspectratio]{https://github.com/typst/packages/raw/main/packages/preview/tabut/1.0.2/doc/compiled-snippets/aggregation.svg}}

\subsection[Grouping ]{\texorpdfstring{Grouping
\protect\hypertarget{grouping}{}{ }}{Grouping  }}\label{grouping}

\begin{Shaded}
\begin{Highlighting}[]
\NormalTok{\#import "@preview/tabut:1.0.2": tabut, group}
\NormalTok{\#import "example{-}data/titanic.typ": titanic, classes}

\NormalTok{\#tabut(}
\NormalTok{  group(titanic, r =\textgreater{} r.Pclass),}
\NormalTok{  (}
\NormalTok{    (header: [*Class*], func: r =\textgreater{} classes.at(r.value)), }
\NormalTok{    (header: [*Passengers*], func: r =\textgreater{} r.group.len()), }
\NormalTok{  ),}
\NormalTok{  fill: (\_, row) =\textgreater{} if calc.odd(row) \{ luma(240) \} else \{ luma(220) \}, }
\NormalTok{  stroke: none}
\NormalTok{)}
\end{Highlighting}
\end{Shaded}

\pandocbounded{\includesvg[keepaspectratio]{https://github.com/typst/packages/raw/main/packages/preview/tabut/1.0.2/doc/compiled-snippets/group.svg}}

\begin{Shaded}
\begin{Highlighting}[]
\NormalTok{\#import "@preview/tabut:1.0.2": tabut, group}
\NormalTok{\#import "usd.typ": usd}
\NormalTok{\#import "example{-}data/titanic.typ": titanic, classes}

\NormalTok{\#tabut(}
\NormalTok{  group(titanic, r =\textgreater{} r.Pclass),}
\NormalTok{  (}
\NormalTok{    (header: [*Class*], func: r =\textgreater{} classes.at(r.value)), }
\NormalTok{    (header: [*Total Fare*], func: r =\textgreater{} usd(r.group.map(r =\textgreater{} r.Fare).sum())), }
\NormalTok{    (}
\NormalTok{      header: [*Avg Fare*], }
\NormalTok{      func: r =\textgreater{} usd(r.group.map(r =\textgreater{} r.Fare).sum() / r.group.len())}
\NormalTok{    ), }
\NormalTok{  ),}
\NormalTok{  fill: (\_, row) =\textgreater{} if calc.odd(row) \{ luma(240) \} else \{ luma(220) \}, }
\NormalTok{  stroke: none}
\NormalTok{)}
\end{Highlighting}
\end{Shaded}

\pandocbounded{\includesvg[keepaspectratio]{https://github.com/typst/packages/raw/main/packages/preview/tabut/1.0.2/doc/compiled-snippets/group-aggregation.svg}}

\subsection[\texttt{\ tabut\ } ]{\texorpdfstring{\texttt{\ tabut\ }
\protect\hypertarget{tabut}{}{ }}{ tabut   }}\label{tabut-1}

Takes data and column definitions and outputs a table.

\begin{Shaded}
\begin{Highlighting}[]
\NormalTok{tabut(}
\NormalTok{  data{-}raw, }
\NormalTok{  colDefs, }
\NormalTok{  columns: auto,}
\NormalTok{  align: auto,}
\NormalTok{  index: "\_index",}
\NormalTok{  transpose: false,}
\NormalTok{  headers: true,}
\NormalTok{  ..tableArgs}
\NormalTok{) {-}\textgreater{} content}
\end{Highlighting}
\end{Shaded}

\subsubsection{Parameters}\label{parameters}

\texttt{\ data-raw\ }\strut \\
This is the raw data that will be used to generate the table. The data
is expected to be in an array of dictionaries, where each dictionary
represents a single record or object.

\texttt{\ colDefs\ }\strut \\
These are the column definitions. An array of dictionaries, each
representing column definition. Must include the properties
\texttt{\ header\ } and a \texttt{\ func\ } . \texttt{\ header\ }
expects content, and specifies the label of the column.
\texttt{\ func\ } expects a function, the function takes a record
dictionary as input and returns the value to be displayed in the cell
corresponding to that record and column. There are also two optional
properties; \texttt{\ align\ } sets the alignment of the content within
the cells of the column, \texttt{\ width\ } sets the width of the
column.

\texttt{\ columns\ }\strut \\
(optional, default: \texttt{\ auto\ } ) Specifies the column widths. If
set to \texttt{\ auto\ } , the function automatically generates column
widths by each column’s column definition. Otherwise functions exactly
the \texttt{\ columns\ } paramater of the standard Typst
\texttt{\ table\ } function. Unlike the \texttt{\ tabut-cells\ } setting
this to \texttt{\ none\ } will break.

\texttt{\ align\ }\strut \\
(optional, default: \texttt{\ auto\ } ) Specifies the column alignment.
If set to \texttt{\ auto\ } , the function automatically generates
column alignment by each column’s column definition. If set to
\texttt{\ none\ } no \texttt{\ align\ } property is added to the output
arg. Otherwise functions exactly the \texttt{\ align\ } paramater of the
standard Typst \texttt{\ table\ } function.

\texttt{\ index\ }\strut \\
(optional, default: \texttt{\ "\_index"\ } ) Specifies the property name
for the index of each record. By default, an \texttt{\ \_index\ }
property is automatically added to each record. If set to
\texttt{\ none\ } , no index property is added.

\texttt{\ transpose\ }\strut \\
(optional, default: \texttt{\ false\ } ) If set to \texttt{\ true\ } ,
transposes the table, swapping rows and columns.

\texttt{\ headers\ }\strut \\
(optional, default: \texttt{\ true\ } ) Determines whether headers
should be included in the output. If set to \texttt{\ false\ } , headers
are not generated.

\texttt{\ tableArgs\ }\strut \\
(optional) Any additional arguments are passed to the \texttt{\ table\ }
function, can be used for styling or anything else.

\subsection[\texttt{\ tabut-cells\ }
]{\texorpdfstring{\texttt{\ tabut-cells\ }
\protect\hypertarget{tabut-cells}{}{
}}{ tabut-cells   }}\label{tabut-cells}

The \texttt{\ tabut-cells\ } function functions as \texttt{\ tabut\ } ,
but returns \texttt{\ arguments\ } for use in either the standard
\texttt{\ table\ } function or other tools such as \texttt{\ tablex\ } .
If you just want the array of cells, use the \texttt{\ pos\ } function
on the returned value, ex \texttt{\ tabut-cells(...).pos\ } .

\texttt{\ tabut-cells\ } is particularly useful when you need to
generate only the cell contents of a table or when these cells need to
be passed to another function for further processing or customization.

\subsubsection{Function Signature}\label{function-signature}

\begin{Shaded}
\begin{Highlighting}[]
\NormalTok{tabut{-}cells(}
\NormalTok{  data{-}raw, }
\NormalTok{  colDefs, }
\NormalTok{  columns: auto,}
\NormalTok{  align: auto,}
\NormalTok{  index: "\_index",}
\NormalTok{  transpose: false,}
\NormalTok{  headers: true,}
\NormalTok{) {-}\textgreater{} arguments}
\end{Highlighting}
\end{Shaded}

\subsubsection{Parameters}\label{parameters-1}

\texttt{\ data-raw\ }\strut \\
This is the raw data that will be used to generate the table. The data
is expected to be in an array of dictionaries, where each dictionary
represents a single record or object.

\texttt{\ colDefs\ }\strut \\
These are the column definitions. An array of dictionaries, each
representing column definition. Must include the properties
\texttt{\ header\ } and a \texttt{\ func\ } . \texttt{\ header\ }
expects content, and specifies the label of the column.
\texttt{\ func\ } expects a function, the function takes a record
dictionary as input and returns the value to be displayed in the cell
corresponding to that record and column. There are also two optional
properties; \texttt{\ align\ } sets the alignment of the content within
the cells of the column, \texttt{\ width\ } sets the width of the
column.

\texttt{\ columns\ }\strut \\
(optional, default: \texttt{\ auto\ } ) Specifies the column widths. If
set to \texttt{\ auto\ } , the function automatically generates column
widths by each column’s column definition. If set to \texttt{\ none\ }
no \texttt{\ column\ } property is added to the output arg. Otherwise
functions exactly the \texttt{\ columns\ } paramater of the standard
typst \texttt{\ table\ } function.

\texttt{\ align\ }\strut \\
(optional, default: \texttt{\ auto\ } ) Specifies the column alignment.
If set to \texttt{\ auto\ } , the function automatically generates
column alignment by each column’s column definition. If set to
\texttt{\ none\ } no \texttt{\ align\ } property is added to the output
arg. Otherwise functions exactly the \texttt{\ align\ } paramater of the
standard typst \texttt{\ table\ } function.

\texttt{\ index\ }\strut \\
(optional, default: \texttt{\ "\_index"\ } ) Specifies the property name
for the index of each record. By default, an \texttt{\ \_index\ }
property is automatically added to each record. If set to
\texttt{\ none\ } , no index property is added.

\texttt{\ transpose\ }\strut \\
(optional, default: \texttt{\ false\ } ) If set to \texttt{\ true\ } ,
transposes the table, swapping rows and columns.

\texttt{\ headers\ }\strut \\
(optional, default: \texttt{\ true\ } ) Determines whether headers
should be included in the output. If set to \texttt{\ false\ } , headers
are not generated.

\subsection[\texttt{\ records-from-csv\ }
]{\texorpdfstring{\texttt{\ records-from-csv\ }
\protect\hypertarget{records-from-csv}{}{
}}{ records-from-csv   }}\label{records-from-csv}

Automatically converts a CSV data into an array of records.

\begin{Shaded}
\begin{Highlighting}[]
\NormalTok{records{-}from{-}csv(}
\NormalTok{  data}
\NormalTok{) {-}\textgreater{} array}
\end{Highlighting}
\end{Shaded}

\subsubsection{Parameters}\label{parameters-2}

\texttt{\ data\ }\strut \\
The CSV data that needs to be converted, this can be obtained using the
native \texttt{\ csv\ } function, like
\texttt{\ records-from-csv(csv(file-path))\ } .

This function simplifies the process of converting CSV data into a
format compatible with \texttt{\ tabut\ } . It reads the CSV data,
extracts the headers, and converts each row into a dictionary, using the
headers as keys.

It also automatically converts data into floats or integers when
possible.

\subsection[\texttt{\ rows-to-records\ }
]{\texorpdfstring{\texttt{\ rows-to-records\ }
\protect\hypertarget{rows-to-records}{}{
}}{ rows-to-records   }}\label{rows-to-records}

Converts rows of data into an array of records based on specified
headers.

This function is useful for converting data in a “rows� format
(commonly found in CSV files) into an array of dictionaries format,
which is required for \texttt{\ tabut\ } and allows easy data processing
using the built in array functions.

\begin{Shaded}
\begin{Highlighting}[]
\NormalTok{rows{-}to{-}records(}
\NormalTok{  headers, }
\NormalTok{  rows, }
\NormalTok{  default: none}
\NormalTok{) {-}\textgreater{} array}
\end{Highlighting}
\end{Shaded}

\subsubsection{Parameters}\label{parameters-3}

\texttt{\ headers\ }\strut \\
An array representing the headers of the table. Each item in this array
corresponds to a column header.

\texttt{\ rows\ }\strut \\
An array of arrays, each representing a row of data. Each sub-array
contains the cell data for a corresponding row.

\texttt{\ default\ }\strut \\
(optional, default: \texttt{\ none\ } ) A default value to use when a
cell is empty or there is an error.

\subsection[\texttt{\ group\ } ]{\texorpdfstring{\texttt{\ group\ }
\protect\hypertarget{group}{}{ }}{ group   }}\label{group}

Groups data based on a specified function and returns an array of
grouped records.

\begin{Shaded}
\begin{Highlighting}[]
\NormalTok{group(}
\NormalTok{  data, }
\NormalTok{  function}
\NormalTok{) {-}\textgreater{} array}
\end{Highlighting}
\end{Shaded}

\subsubsection{Parameters}\label{parameters-4}

\texttt{\ data\ }\strut \\
An array of dictionaries. Each dictionary represents a single record or
object.

\texttt{\ function\ }\strut \\
A function that takes a record as input and returns a value based on
which the grouping is to be performed.

This function iterates over each record in the \texttt{\ data\ } ,
applies the \texttt{\ function\ } to determine the grouping value, and
organizes the records into groups based on this value. Each group record
is represented as a dictionary with two properties: \texttt{\ value\ }
(the result of the grouping function) and \texttt{\ group\ } (an array
of records belonging to this group).

In the context of \texttt{\ tabut\ } , the \texttt{\ group\ } function
is particularly useful for creating summary tables where records need to
be categorized and aggregated based on certain criteria, such as
calculating total or average values for each group.

\subsubsection{How to add}\label{how-to-add}

Copy this into your project and use the import as \texttt{\ tabut\ }

\begin{verbatim}
#import "@preview/tabut:1.0.2"
\end{verbatim}

\includesvg[width=0.16667in,height=0.16667in]{/assets/icons/16-copy.svg}

Check the docs for
\href{https://typst.app/docs/reference/scripting/\#packages}{more
information on how to import packages} .

\subsubsection{About}\label{about}

\begin{description}
\tightlist
\item[Author :]
\href{https://github.com/Amelia-Mowers}{Amelia Mowers}
\item[License:]
MIT
\item[Current version:]
1.0.2
\item[Last updated:]
April 16, 2024
\item[First released:]
January 29, 2024
\item[Archive size:]
9.40 kB
\href{https://packages.typst.org/preview/tabut-1.0.2.tar.gz}{\pandocbounded{\includesvg[keepaspectratio]{/assets/icons/16-download.svg}}}
\item[Repository:]
\href{https://github.com/Amelia-Mowers/typst-tabut}{GitHub}
\end{description}

\subsubsection{Where to report issues?}\label{where-to-report-issues}

This package is a project of Amelia Mowers . Report issues on
\href{https://github.com/Amelia-Mowers/typst-tabut}{their repository} .
You can also try to ask for help with this package on the
\href{https://forum.typst.app}{Forum} .

Please report this package to the Typst team using the
\href{https://typst.app/contact}{contact form} if you believe it is a
safety hazard or infringes upon your rights.

\phantomsection\label{versions}
\subsubsection{Version history}\label{version-history}

\begin{longtable}[]{@{}ll@{}}
\toprule\noalign{}
Version & Release Date \\
\midrule\noalign{}
\endhead
\bottomrule\noalign{}
\endlastfoot
1.0.2 & April 16, 2024 \\
\href{https://typst.app/universe/package/tabut/1.0.1/}{1.0.1} & January
31, 2024 \\
\href{https://typst.app/universe/package/tabut/1.0.0/}{1.0.0} & January
29, 2024 \\
\end{longtable}

Typst GmbH did not create this package and cannot guarantee correct
functionality of this package or compatibility with any version of the
Typst compiler or app.


\section{Package List LaTeX/cetz-plot.tex}
\title{typst.app/universe/package/cetz-plot}

\phantomsection\label{banner}
\section{cetz-plot}\label{cetz-plot}

{ 0.1.0 }

Plotting module for CeTZ.

\phantomsection\label{readme}
CeTZ-Plot is a library that adds plots and charts to
\href{https://github.com/cetz-package/cetz}{CeTZ} , a library for
drawing with \href{https://typst.app/}{Typst} .

CeTZ-Plot requires CeTZ version ≥ 0.3.1!

\subsection{Examples}\label{examples}

\begin{longtable}[]{@{}lll@{}}
\toprule\noalign{}
\endhead
\bottomrule\noalign{}
\endlastfoot
\href{https://github.com/typst/packages/raw/main/packages/preview/cetz-plot/0.1.0/gallery/line.typ}{\includegraphics[width=2.60417in,height=\textheight,keepaspectratio]{https://github.com/typst/packages/raw/main/packages/preview/cetz-plot/0.1.0/gallery/line.png}}
&
\href{https://github.com/typst/packages/raw/main/packages/preview/cetz-plot/0.1.0/gallery/piechart.typ}{\includegraphics[width=2.60417in,height=\textheight,keepaspectratio]{https://github.com/typst/packages/raw/main/packages/preview/cetz-plot/0.1.0/gallery/piechart.png}}
&
\href{https://github.com/typst/packages/raw/main/packages/preview/cetz-plot/0.1.0/gallery/barchart.typ}{\includegraphics[width=2.60417in,height=\textheight,keepaspectratio]{https://github.com/typst/packages/raw/main/packages/preview/cetz-plot/0.1.0/gallery/barchart.png}} \\
Plot & Pie Chart & Clustered Barchart \\
\end{longtable}

\emph{Click on the example image to jump to the code.}

\subsection{Usage}\label{usage}

For information, see the
\href{https://github.com/cetz-package/cetz-plot/blob/stable/manual.pdf?raw=true}{manual
(stable)} .

To use this package, simply add the following code to your document:

\begin{verbatim}
#import "@preview/cetz:0.3.1"
#import "@preview/cetz-plot:0.1.0": plot, chart

#cetz.canvas({
  // Your plot/chart code goes here
})
\end{verbatim}

\subsection{Installing}\label{installing}

To install the CeTZ-Plot package under
\href{https://github.com/typst/packages?tab=readme-ov-file\#local-packages}{your
local typst package dir} you can use the \texttt{\ install\ } script
from the repository.

\subsubsection{Just}\label{just}

This project uses \href{https://github.com/casey/just}{just} , a handy
command runner.

You can run all commands without having \texttt{\ just\ } installed,
just have a look into the \texttt{\ justfile\ } . To install
\texttt{\ just\ } on your system, use your systems package manager. On
Windows, \href{https://doc.rust-lang.org/cargo/}{Cargo} (
\texttt{\ cargo\ install\ just\ } ),
\href{https://chocolatey.org/}{Chocolatey} (
\texttt{\ choco\ install\ just\ } ) and
\href{https://just.systems/man/en/chapter_4.html}{some other sources}
can be used. You need to run it from a \texttt{\ sh\ } compatible shell
on Windows (e.g git-bash).

\subsection{Testing}\label{testing}

This package comes with some unit tests under the \texttt{\ tests\ }
directory. To run all tests you can run the \texttt{\ just\ test\ }
target. You need to have
\href{https://github.com/tingerrr/typst-test/}{\texttt{\ typst-test\ }}
in your \texttt{\ PATH\ } :
\texttt{\ cargo\ install\ typst-test\ -\/-git\ https://github.com/tingerrr/typst-test\ }
.

\subsubsection{How to add}\label{how-to-add}

Copy this into your project and use the import as \texttt{\ cetz-plot\ }

\begin{verbatim}
#import "@preview/cetz-plot:0.1.0"
\end{verbatim}

\includesvg[width=0.16667in,height=0.16667in]{/assets/icons/16-copy.svg}

Check the docs for
\href{https://typst.app/docs/reference/scripting/\#packages}{more
information on how to import packages} .

\subsubsection{About}\label{about}

\begin{description}
\tightlist
\item[Author s :]
\href{https://github.com/johannes-wolf}{Johannes Wolf} \&
\href{https://github.com/fenjalien}{fenjalien}
\item[License:]
LGPL-3.0-or-later
\item[Current version:]
0.1.0
\item[Last updated:]
October 21, 2024
\item[First released:]
October 21, 2024
\item[Minimum Typst version:]
0.11.0
\item[Archive size:]
43.9 kB
\href{https://packages.typst.org/preview/cetz-plot-0.1.0.tar.gz}{\pandocbounded{\includesvg[keepaspectratio]{/assets/icons/16-download.svg}}}
\item[Repository:]
\href{https://github.com/cetz-package/cetz-plot}{GitHub}
\item[Categor y :]
\begin{itemize}
\tightlist
\item[]
\item
  \pandocbounded{\includesvg[keepaspectratio]{/assets/icons/16-chart.svg}}
  \href{https://typst.app/universe/search/?category=visualization}{Visualization}
\end{itemize}
\end{description}

\subsubsection{Where to report issues?}\label{where-to-report-issues}

This package is a project of Johannes Wolf and fenjalien . Report issues
on \href{https://github.com/cetz-package/cetz-plot}{their repository} .
You can also try to ask for help with this package on the
\href{https://forum.typst.app}{Forum} .

Please report this package to the Typst team using the
\href{https://typst.app/contact}{contact form} if you believe it is a
safety hazard or infringes upon your rights.

\phantomsection\label{versions}
\subsubsection{Version history}\label{version-history}

\begin{longtable}[]{@{}ll@{}}
\toprule\noalign{}
Version & Release Date \\
\midrule\noalign{}
\endhead
\bottomrule\noalign{}
\endlastfoot
0.1.0 & October 21, 2024 \\
\end{longtable}

Typst GmbH did not create this package and cannot guarantee correct
functionality of this package or compatibility with any version of the
Typst compiler or app.


\section{Package List LaTeX/autofletcher.tex}
\title{typst.app/universe/package/autofletcher}

\phantomsection\label{banner}
\section{autofletcher}\label{autofletcher}

{ 0.1.1 }

Easier diagrams with fletcher

\phantomsection\label{readme}
This small module provides functions to (sort of) abstract away manual
placement of coordinates.

See the
\href{https://raw.githubusercontent.com/3akev/autofletcher/main/manual.pdf}{manual}
for usage examples.

\subsection{Credits}\label{credits}

\href{https://github.com/Jollywatt/typst-fletcher}{fletcher}

\subsubsection{How to add}\label{how-to-add}

Copy this into your project and use the import as
\texttt{\ autofletcher\ }

\begin{verbatim}
#import "@preview/autofletcher:0.1.1"
\end{verbatim}

\includesvg[width=0.16667in,height=0.16667in]{/assets/icons/16-copy.svg}

Check the docs for
\href{https://typst.app/docs/reference/scripting/\#packages}{more
information on how to import packages} .

\subsubsection{About}\label{about}

\begin{description}
\tightlist
\item[Author :]
\href{https://github.com/3akev}{3akev}
\item[License:]
MIT
\item[Current version:]
0.1.1
\item[Last updated:]
May 23, 2024
\item[First released:]
May 14, 2024
\item[Archive size:]
2.67 kB
\href{https://packages.typst.org/preview/autofletcher-0.1.1.tar.gz}{\pandocbounded{\includesvg[keepaspectratio]{/assets/icons/16-download.svg}}}
\item[Repository:]
\href{https://github.com/3akev/autofletcher}{GitHub}
\item[Categor y :]
\begin{itemize}
\tightlist
\item[]
\item
  \pandocbounded{\includesvg[keepaspectratio]{/assets/icons/16-chart.svg}}
  \href{https://typst.app/universe/search/?category=visualization}{Visualization}
\end{itemize}
\end{description}

\subsubsection{Where to report issues?}\label{where-to-report-issues}

This package is a project of 3akev . Report issues on
\href{https://github.com/3akev/autofletcher}{their repository} . You can
also try to ask for help with this package on the
\href{https://forum.typst.app}{Forum} .

Please report this package to the Typst team using the
\href{https://typst.app/contact}{contact form} if you believe it is a
safety hazard or infringes upon your rights.

\phantomsection\label{versions}
\subsubsection{Version history}\label{version-history}

\begin{longtable}[]{@{}ll@{}}
\toprule\noalign{}
Version & Release Date \\
\midrule\noalign{}
\endhead
\bottomrule\noalign{}
\endlastfoot
0.1.1 & May 23, 2024 \\
\href{https://typst.app/universe/package/autofletcher/0.1.0/}{0.1.0} &
May 14, 2024 \\
\end{longtable}

Typst GmbH did not create this package and cannot guarantee correct
functionality of this package or compatibility with any version of the
Typst compiler or app.


\section{Package List LaTeX/accelerated-jacow.tex}
\title{typst.app/universe/package/accelerated-jacow}

\phantomsection\label{banner}
\phantomsection\label{template-thumbnail}
\pandocbounded{\includegraphics[keepaspectratio]{https://packages.typst.org/preview/thumbnails/accelerated-jacow-0.1.1-small.webp}}

\section{accelerated-jacow}\label{accelerated-jacow}

{ 0.1.1 }

Paper template for conference proceedings in accelerator physics

\href{/app?template=accelerated-jacow&version=0.1.1}{Create project in
app}

\phantomsection\label{readme}
\href{https://github.com/eltos/accelerated-jacow}{\pandocbounded{\includegraphics[keepaspectratio]{https://img.shields.io/badge/GitHub\%20Repo-eltos\%2Faccelerated--jacow-lightgray}}}
\href{https://typst.app/universe/package/accelerated-jacow}{\pandocbounded{\includegraphics[keepaspectratio]{https://img.shields.io/badge/Typst\%20Universe-accelerated--jacow-\%23219dac}}}

Paper template for conference proceedings in accelerator physics.

Based on the JACoW guide for preparation of papers available at
\url{https://jacow.org/} .

\subsection{Usage}\label{usage}

\subsubsection{Typst web app}\label{typst-web-app}

In the \href{https://typst.app/}{typst web app} select “start from
template� and search for the accelerated-jacow template.

\subsubsection{Local installation}\label{local-installation}

Run these commands inside your terminal:

\begin{Shaded}
\begin{Highlighting}[]
\ExtensionTok{typst}\NormalTok{ init @preview/accelerated{-}jacow}
\BuiltInTok{cd}\NormalTok{ accelerated{-}jacow}
\ExtensionTok{typst}\NormalTok{ watch paper.typ}
\end{Highlighting}
\end{Shaded}

If you don’t yet have the \emph{TeX Gyre Termes} font family, you can
install it with \texttt{\ sudo\ apt\ install\ tex-gyre\ } .

\pandocbounded{\includegraphics[keepaspectratio]{https://github.com/typst/packages/raw/main/packages/preview/accelerated-jacow/0.1.1/thumbnail.webp}}

\subsection{Licence}\label{licence}

Files inside the template folder are licensed under MIT-0. You can use
them without restrictions.\\
The citation style (CSL) file is based on the IEEE style and licensed
under the \href{https://creativecommons.org/licenses/by-sa/4.0/}{CC BY
SA 4.0} compatible
\href{https://www.gnu.org/licenses/gpl-3.0.html}{GPLv3} license.\\
All other files are licensed under
\href{https://www.gnu.org/licenses/gpl-3.0.html}{GPLv3} .

\href{/app?template=accelerated-jacow&version=0.1.1}{Create project in
app}

\subsubsection{How to use}\label{how-to-use}

Click the button above to create a new project using this template in
the Typst app.

You can also use the Typst CLI to start a new project on your computer
using this command:

\begin{verbatim}
typst init @preview/accelerated-jacow:0.1.1
\end{verbatim}

\includesvg[width=0.16667in,height=0.16667in]{/assets/icons/16-copy.svg}

\subsubsection{About}\label{about}

\begin{description}
\tightlist
\item[Author :]
\href{https://github.com/eltos}{Philipp Niedermayer}
\item[License:]
GPL-3.0-only AND MIT-0
\item[Current version:]
0.1.1
\item[Last updated:]
November 21, 2024
\item[First released:]
October 30, 2024
\item[Archive size:]
27.1 kB
\href{https://packages.typst.org/preview/accelerated-jacow-0.1.1.tar.gz}{\pandocbounded{\includesvg[keepaspectratio]{/assets/icons/16-download.svg}}}
\item[Repository:]
\href{https://github.com/eltos/accelerated-jacow/}{GitHub}
\item[Discipline s :]
\begin{itemize}
\tightlist
\item[]
\item
  \href{https://typst.app/universe/search/?discipline=physics}{Physics}
\item
  \href{https://typst.app/universe/search/?discipline=engineering}{Engineering}
\end{itemize}
\item[Categor y :]
\begin{itemize}
\tightlist
\item[]
\item
  \pandocbounded{\includesvg[keepaspectratio]{/assets/icons/16-atom.svg}}
  \href{https://typst.app/universe/search/?category=paper}{Paper}
\end{itemize}
\end{description}

\subsubsection{Where to report issues?}\label{where-to-report-issues}

This template is a project of Philipp Niedermayer . Report issues on
\href{https://github.com/eltos/accelerated-jacow/}{their repository} .
You can also try to ask for help with this template on the
\href{https://forum.typst.app}{Forum} .

Please report this template to the Typst team using the
\href{https://typst.app/contact}{contact form} if you believe it is a
safety hazard or infringes upon your rights.

\phantomsection\label{versions}
\subsubsection{Version history}\label{version-history}

\begin{longtable}[]{@{}ll@{}}
\toprule\noalign{}
Version & Release Date \\
\midrule\noalign{}
\endhead
\bottomrule\noalign{}
\endlastfoot
0.1.1 & November 21, 2024 \\
\href{https://typst.app/universe/package/accelerated-jacow/0.1.0/}{0.1.0}
& October 30, 2024 \\
\end{longtable}

Typst GmbH did not create this template and cannot guarantee correct
functionality of this template or compatibility with any version of the
Typst compiler or app.


\section{Package List LaTeX/ctheorems.tex}
\title{typst.app/universe/package/ctheorems}

\phantomsection\label{banner}
\section{ctheorems}\label{ctheorems}

{ 1.1.3 }

Numbered theorem environments for typst.

\phantomsection\label{readme}
An implementation of numbered theorem environments in
\href{https://github.com/typst/typst}{typst} . Import using

\begin{Shaded}
\begin{Highlighting}[]
\NormalTok{\#import "@preview/ctheorems:1.1.3": *}
\NormalTok{\#show: thmrules}
\end{Highlighting}
\end{Shaded}

\subsubsection{Features}\label{features}

\begin{itemize}
\tightlist
\item
  Numbered theorem environments can be created and customized.
\item
  Environments can share the same counter, via same
  \texttt{\ identifier\ } s.
\item
  Environment counters can be \emph{attached} (just as subheadings are
  attached to headings) to other environments, headings, or keep a
  global count via \texttt{\ base\ } .
\item
  The depth of a counter can be manually set, via
  \texttt{\ base\_level\ } .
\item
  Environments can be \texttt{\ \textless{}label\textgreater{}\ }
  \textquotesingle d and \texttt{\ @reference\ } ’d.
\item
  Awesome presets (coming soon!)
\end{itemize}

\subsection{Manual and Examples}\label{manual-and-examples}

Get acquainted with \texttt{\ ctheorems\ } by checking out the minimal
example below!

You can read the
\href{https://github.com/typst/packages/raw/main/packages/preview/ctheorems/1.1.3/assets/manual.pdf}{manual}
for a full walkthrough of functionality offered by this module; flick
through
\href{https://github.com/typst/packages/raw/main/packages/preview/ctheorems/1.1.3/assets/manual_examples.pdf}{manual\_examples}
to just see the examples.

\pandocbounded{\includegraphics[keepaspectratio]{https://github.com/typst/packages/raw/main/packages/preview/ctheorems/1.1.3/assets/basic.png}}

\subsubsection{Preamble}\label{preamble}

\begin{Shaded}
\begin{Highlighting}[]
\NormalTok{\#import "@preview/ctheorems:1.1.3": *}
\NormalTok{\#show: thmrules.with(qed{-}symbol: $square$)}

\NormalTok{\#set page(width: 16cm, height: auto, margin: 1.5cm)}
\NormalTok{\#set heading(numbering: "1.1.")}

\NormalTok{\#let theorem = thmbox("theorem", "Theorem", fill: rgb("\#eeffee"))}
\NormalTok{\#let corollary = thmplain(}
\NormalTok{  "corollary",}
\NormalTok{  "Corollary",}
\NormalTok{  base: "theorem",}
\NormalTok{  titlefmt: strong}
\NormalTok{)}
\NormalTok{\#let definition = thmbox("definition", "Definition", inset: (x: 1.2em, top: 1em))}

\NormalTok{\#let example = thmplain("example", "Example").with(numbering: none)}
\NormalTok{\#let proof = thmproof("proof", "Proof")}
\end{Highlighting}
\end{Shaded}

\subsubsection{Document}\label{document}

\begin{Shaded}
\begin{Highlighting}[]
\NormalTok{= Prime numbers}

\NormalTok{\#definition[}
\NormalTok{  A natural number is called a \#highlight[\_prime number\_] if it is greater}
\NormalTok{  than 1 and cannot be written as the product of two smaller natural numbers.}
\NormalTok{]}
\NormalTok{\#example[}
\NormalTok{  The numbers $2$, $3$, and $17$ are prime.}
\NormalTok{  @cor\_largest\_prime shows that this list is not exhaustive!}
\NormalTok{]}

\NormalTok{\#theorem("Euclid")[}
\NormalTok{  There are infinitely many primes.}
\NormalTok{]}
\NormalTok{\#proof[}
\NormalTok{  Suppose to the contrary that $p\_1, p\_2, dots, p\_n$ is a finite enumeration}
\NormalTok{  of all primes. Set $P = p\_1 p\_2 dots p\_n$. Since $P + 1$ is not in our list,}
\NormalTok{  it cannot be prime. Thus, some prime factor $p\_j$ divides $P + 1$.  Since}
\NormalTok{  $p\_j$ also divides $P$, it must divide the difference $(P + 1) {-} P = 1$, a}
\NormalTok{  contradiction.}
\NormalTok{]}

\NormalTok{\#corollary[}
\NormalTok{  There is no largest prime number.}
\NormalTok{] \textless{}cor\_largest\_prime\textgreater{}}
\NormalTok{\#corollary[}
\NormalTok{  There are infinitely many composite numbers.}
\NormalTok{]}

\NormalTok{\#theorem[}
\NormalTok{  There are arbitrarily long stretches of composite numbers.}
\NormalTok{]}
\NormalTok{\#proof[}
\NormalTok{  For any $n \textgreater{} 2$, consider $}
\NormalTok{    n! + 2, quad n! + 3, quad ..., quad n! + n \#qedhere}
\NormalTok{  $}
\NormalTok{]}
\end{Highlighting}
\end{Shaded}

\subsection{Changelog}\label{changelog}

\subsubsection{v1.1.3}\label{v1.1.3}

\begin{itemize}
\tightlist
\item
  Fixed alignment and block-breaking issues resulting from breaking
  changes in Typst 0.12.
\end{itemize}

\subsubsection{v1.1.2}\label{v1.1.2}

\begin{itemize}
\tightlist
\item
  Introduced the \texttt{\ thmproof\ } function for creating proof
  environments.
\item
  Inserting \texttt{\ \#qedhere\ } in a block equation/list/enum item
  (in a proof) places the qed symbol on the same line. The qed symbol
  can be customized via \texttt{\ thmrules\ } .
\end{itemize}

\subsubsection{v1.1.1}\label{v1.1.1}

\begin{itemize}
\tightlist
\item
  Extra named arguments given to a theorem environment produced by
  \texttt{\ thmbox\ } (or \texttt{\ thmplain\ } ) are passed to
  \texttt{\ block\ } .
\end{itemize}

\subsubsection{v1.1.0}\label{v1.1.0}

\begin{itemize}
\tightlist
\item
  The \texttt{\ supplement\ } (for references) is no longer set in
  \texttt{\ thmenv\ } . It can be passed to the theorem environment
  directly, along with \texttt{\ refnumbering\ } to control the
  appearance of \texttt{\ @reference\ } s.
\item
  Extra named arguments given to \texttt{\ thmbox\ } are passed to
  \texttt{\ block\ } .
\item
  Fixed spacing bug for unnumbered environments.
\item
  Replaced dummy figure with labelled metadata.
\end{itemize}

\subsubsection{v1.0.0}\label{v1.0.0}

\begin{itemize}
\tightlist
\item
  Extra named arguments given to a theorem environment are passed to its
  formatting function \texttt{\ fmt\ } .
\item
  Removed \texttt{\ thmref\ } , introduced normal
  \texttt{\ \textless{}label\textgreater{}\ } s and
  \texttt{\ @reference\ } s.
\item
  Import must be followed by \texttt{\ show:\ thmrules\ } .
\item
  Removed \texttt{\ name:\ ...\ } from theorem environments; use
  \texttt{\ \#theorem("Euclid"){[}{]}\ } instead of
  \texttt{\ \#theorem(name:\ "Euclid"){[}{]}\ } .
\item
  Theorems are now wrapped in \texttt{\ figure\ } s.
\end{itemize}

\subsection{Credits}\label{credits}

\begin{itemize}
\tightlist
\item
  \href{https://github.com/sahasatvik}{sahasatvik (Satvik Saha)}
\item
  \href{https://github.com/MJHutchinson}{MJHutchinson (Michael
  Hutchinson)}
\item
  \href{https://github.com/rmolinari}{rmolinari (Rory Molinari)}
\item
  \href{https://github.com/PgBiel}{PgBiel}
\item
  \href{https://github.com/DVDTSB}{DVDTSB}
\end{itemize}

\subsubsection{How to add}\label{how-to-add}

Copy this into your project and use the import as \texttt{\ ctheorems\ }

\begin{verbatim}
#import "@preview/ctheorems:1.1.3"
\end{verbatim}

\includesvg[width=0.16667in,height=0.16667in]{/assets/icons/16-copy.svg}

Check the docs for
\href{https://typst.app/docs/reference/scripting/\#packages}{more
information on how to import packages} .

\subsubsection{About}\label{about}

\begin{description}
\tightlist
\item[Author s :]
sahasatvik (Satvik Saha) , rmolinari (Rory Molinari) , MJHutchinson
(Michael Hutchinson) , PgBiel , \& DVDTSB
\item[License:]
MIT
\item[Current version:]
1.1.3
\item[Last updated:]
October 23, 2024
\item[First released:]
September 13, 2023
\item[Archive size:]
4.68 kB
\href{https://packages.typst.org/preview/ctheorems-1.1.3.tar.gz}{\pandocbounded{\includesvg[keepaspectratio]{/assets/icons/16-download.svg}}}
\item[Repository:]
\href{https://github.com/sahasatvik/typst-theorems}{GitHub}
\end{description}

\subsubsection{Where to report issues?}\label{where-to-report-issues}

This package is a project of sahasatvik (Satvik Saha), rmolinari (Rory
Molinari), MJHutchinson (Michael Hutchinson), PgBiel, and DVDTSB .
Report issues on
\href{https://github.com/sahasatvik/typst-theorems}{their repository} .
You can also try to ask for help with this package on the
\href{https://forum.typst.app}{Forum} .

Please report this package to the Typst team using the
\href{https://typst.app/contact}{contact form} if you believe it is a
safety hazard or infringes upon your rights.

\phantomsection\label{versions}
\subsubsection{Version history}\label{version-history}

\begin{longtable}[]{@{}ll@{}}
\toprule\noalign{}
Version & Release Date \\
\midrule\noalign{}
\endhead
\bottomrule\noalign{}
\endlastfoot
1.1.3 & October 23, 2024 \\
\href{https://typst.app/universe/package/ctheorems/1.1.2/}{1.1.2} &
February 25, 2024 \\
\href{https://typst.app/universe/package/ctheorems/1.1.1/}{1.1.1} &
February 24, 2024 \\
\href{https://typst.app/universe/package/ctheorems/1.1.0/}{1.1.0} &
November 6, 2023 \\
\href{https://typst.app/universe/package/ctheorems/1.0.0/}{1.0.0} &
September 23, 2023 \\
\href{https://typst.app/universe/package/ctheorems/0.1.0/}{0.1.0} &
September 13, 2023 \\
\end{longtable}

Typst GmbH did not create this package and cannot guarantee correct
functionality of this package or compatibility with any version of the
Typst compiler or app.


\section{Package List LaTeX/echarm.tex}
\title{typst.app/universe/package/echarm}

\phantomsection\label{banner}
\section{echarm}\label{echarm}

{ 0.1.1 }

Run echarts in typst with the use of CtxJS.

\phantomsection\label{readme}
A typst plugin to run echarts in typst with the use of CtxJS.

\subsection{Examples}\label{examples}

\begin{longtable}[]{@{}lll@{}}
\toprule\noalign{}
\endhead
\bottomrule\noalign{}
\endlastfoot
\href{https://github.com/typst/packages/raw/main/packages/preview/echarm/0.1.1/examples/mixed_charts.typ}{\includegraphics[width=1\linewidth,height=\textheight,keepaspectratio]{https://github.com/typst/packages/raw/main/packages/preview/echarm/0.1.1/examples/mixed_charts.png}}
&
\href{https://github.com/typst/packages/raw/main/packages/preview/echarm/0.1.1/examples/radar.typ}{\includegraphics[width=1\linewidth,height=\textheight,keepaspectratio]{https://github.com/typst/packages/raw/main/packages/preview/echarm/0.1.1/examples/radar.png}}
&
\href{https://github.com/typst/packages/raw/main/packages/preview/echarm/0.1.1/examples/pie.typ}{\includegraphics[width=1\linewidth,height=\textheight,keepaspectratio]{https://github.com/typst/packages/raw/main/packages/preview/echarm/0.1.1/examples/pie.png}} \\
\href{https://github.com/typst/packages/raw/main/packages/preview/echarm/0.1.1/examples/mixed_charts.typ}{Source
Code} &
\href{https://github.com/typst/packages/raw/main/packages/preview/echarm/0.1.1/examples/radar.typ}{Source
Code} &
\href{https://github.com/typst/packages/raw/main/packages/preview/echarm/0.1.1/examples/pie.typ}{Source
Code} \\
\href{https://github.com/typst/packages/raw/main/packages/preview/echarm/0.1.1/examples/scatter.typ}{\includegraphics[width=1\linewidth,height=\textheight,keepaspectratio]{https://github.com/typst/packages/raw/main/packages/preview/echarm/0.1.1/examples/scatter.png}}
&
\href{https://github.com/typst/packages/raw/main/packages/preview/echarm/0.1.1/examples/gauge.typ}{\includegraphics[width=1\linewidth,height=\textheight,keepaspectratio]{https://github.com/typst/packages/raw/main/packages/preview/echarm/0.1.1/examples/gauge.png}}
&
\href{https://github.com/typst/packages/raw/main/packages/preview/echarm/0.1.1/examples/candlestick.typ}{\includegraphics[width=1\linewidth,height=\textheight,keepaspectratio]{https://github.com/typst/packages/raw/main/packages/preview/echarm/0.1.1/examples/candlestick.png}} \\
\href{https://github.com/typst/packages/raw/main/packages/preview/echarm/0.1.1/examples/scatter.typ}{Source
Code} &
\href{https://github.com/typst/packages/raw/main/packages/preview/echarm/0.1.1/examples/gauge.typ}{Source
Code} &
\href{https://github.com/typst/packages/raw/main/packages/preview/echarm/0.1.1/examples/candlestick.typ}{Source
Code} \\
\end{longtable}

For more examples see:

\url{https://echarts.apache.org/examples/en/index.html}

For the complete documentation for the configuration of echarts, see:

\url{https://echarts.apache.org/en/option.html}

\subsection{Usage}\label{usage}

\begin{Shaded}
\begin{Highlighting}[]
\NormalTok{\#import "@preview/echarm:0.1.1"}

\NormalTok{// options are echart options}
\NormalTok{\#echarm.render(width: 100\%, height: 100\%, options: (:))}
\end{Highlighting}
\end{Shaded}

\subsection{Infos}\label{infos}

The version is not the same as the echart version, so that I can update
independently. Animations are not supported here!

You can find more information about CtxJS here:

\url{https://typst.app/universe/package/ctxjs/}

\subsection{Versions}\label{versions}

\begin{longtable}[]{@{}ll@{}}
\toprule\noalign{}
Version & Echart-Version \\
\midrule\noalign{}
\endhead
\bottomrule\noalign{}
\endlastfoot
0.1.0 & 5.5.1 \\
0.1.1 & 5.5.1 \textsuperscript{1} \\
\end{longtable}

\textsuperscript{1} new eval-later feature

\subsubsection{How to add}\label{how-to-add}

Copy this into your project and use the import as \texttt{\ echarm\ }

\begin{verbatim}
#import "@preview/echarm:0.1.1"
\end{verbatim}

\includesvg[width=0.16667in,height=0.16667in]{/assets/icons/16-copy.svg}

Check the docs for
\href{https://typst.app/docs/reference/scripting/\#packages}{more
information on how to import packages} .

\subsubsection{About}\label{about}

\begin{description}
\tightlist
\item[Author :]
lublak
\item[License:]
MIT
\item[Current version:]
0.1.1
\item[Last updated:]
November 29, 2024
\item[First released:]
September 15, 2024
\item[Archive size:]
888 kB
\href{https://packages.typst.org/preview/echarm-0.1.1.tar.gz}{\pandocbounded{\includesvg[keepaspectratio]{/assets/icons/16-download.svg}}}
\item[Repository:]
\href{https://github.com/lublak/typst-echarm-package}{GitHub}
\end{description}

\subsubsection{Where to report issues?}\label{where-to-report-issues}

This package is a project of lublak . Report issues on
\href{https://github.com/lublak/typst-echarm-package}{their repository}
. You can also try to ask for help with this package on the
\href{https://forum.typst.app}{Forum} .

Please report this package to the Typst team using the
\href{https://typst.app/contact}{contact form} if you believe it is a
safety hazard or infringes upon your rights.

\phantomsection\label{versions}
\subsubsection{Version history}\label{version-history}

\begin{longtable}[]{@{}ll@{}}
\toprule\noalign{}
Version & Release Date \\
\midrule\noalign{}
\endhead
\bottomrule\noalign{}
\endlastfoot
0.1.1 & November 29, 2024 \\
\href{https://typst.app/universe/package/echarm/0.1.0/}{0.1.0} &
September 15, 2024 \\
\end{longtable}

Typst GmbH did not create this package and cannot guarantee correct
functionality of this package or compatibility with any version of the
Typst compiler or app.


\section{Package List LaTeX/smooth-tmlr.tex}
\title{typst.app/universe/package/smooth-tmlr}

\phantomsection\label{banner}
\phantomsection\label{template-thumbnail}
\pandocbounded{\includegraphics[keepaspectratio]{https://packages.typst.org/preview/thumbnails/smooth-tmlr-0.4.0-small.webp}}

\section{smooth-tmlr}\label{smooth-tmlr}

{ 0.4.0 }

Paper template for submission to Transaction on Machine Learning
Research (TMLR)

\href{/app?template=smooth-tmlr&version=0.4.0}{Create project in app}

\phantomsection\label{readme}
\subsection{Usage}\label{usage}

You can use this template in the Typst web app by clicking \emph{Start
from template} on the dashboard and searching for
\texttt{\ smooth-tmlr\ } .

Alternatively, you can use the CLI to kick this project off using the
command

\begin{Shaded}
\begin{Highlighting}[]
\NormalTok{typst init @preview/smooth{-}tmlr}
\end{Highlighting}
\end{Shaded}

Typst will create a new directory with all the files needed to get you
started.

\subsection{Example Papers}\label{example-papers}

Here are an example paper in
\href{https://github.com/daskol/typst-templates}{LaTeX} and in
\href{https://github.com/daskol/typst-templates/\#colored-annotations}{Typst}
.

\subsection{Configuration}\label{configuration}

This template exports the \texttt{\ tmlr\ } function with the following
named arguments.

\begin{itemize}
\tightlist
\item
  \texttt{\ title\ } : The paper’s title as content.
\item
  \texttt{\ authors\ } : An array of author dictionaries. Each of the
  author dictionaries must have a name key and can have the keys
  department, organization, location, and email.
\item
  \texttt{\ keywords\ } : Publication keywords (used in PDF metadata).
\item
  \texttt{\ date\ } : Creation date (used in PDF metadata).
\item
  \texttt{\ abstract\ } : The content of a brief summary of the paper or
  none. Appears at the top under the title.
\item
  \texttt{\ bibliography\ } : The result of a call to the bibliography
  function or none. The function also accepts a single, positional
  argument for the body of the paper.
\item
  \texttt{\ appendix\ } : Content to append after bibliography section.
\item
  \texttt{\ accepted\ } : If this is set to \texttt{\ false\ } then
  anonymized ready for submission document is produced;
  \texttt{\ accepted:\ true\ } produces camera-redy version. If the
  argument is set to \texttt{\ none\ } then preprint version is produced
  (can be uploaded to arXiv).
\item
  \texttt{\ review\ } : Hypertext link to review on OpenReview.
\item
  \texttt{\ pubdate\ } : Date of publication (used only month and date).
\end{itemize}

The template will initialize your package with a sample call to the
\texttt{\ tmlr\ } function in a show rule. If you want to change an
existing project to use this template, you can add a show rule at the
top of your file.

\subsection{Issues}\label{issues}

This template is developed at
\href{https://github.com/daskol/typst-templates}{daskol/typst-templates}
repo. Please report all issues there.

\begin{itemize}
\item
  While author instruction says the all text should be in sans serif
  font Computer Modern Bright, only headers and titles are in sans font
  Computer Modern Sans and the rest of text is causal Computer Modern
  Serif (or Roman). To be precice, the original template uses Latin
  Modern, a descendant of Computer Modern. In this tempalte we stick to
  CMU (Computer Modern Unicode) font family.
\item
  In the original template paper, the word \textbf{Abstract} is of large
  font size (12pt) and bold. This affects slightly line spacing. We
  don’t know how to adjust Typst to reproduce this feature of the
  reference template but this issue does not impact a lot on visual
  appearence and layouting.
\item
  In the original template special level-3 sections like “Author
  Contributions� or “Acknowledgements� are not added to outline.
  We add them to outline as level-1 header but still render them as
  level-3 headers.
\item
  ICML-like bibliography style. In this case, the bibliography slightly
  differs from the one in the original example paper. The main
  difference is that we prefer to use author’s lastname at first place
  to search an entry faster.
\item
  In the original template a lot of vertical space is inserted before
  and after graphics and table figures. It is unclear why so much space
  are inserted. We belive that the reason is how Typst justify content
  verticaly. Nevertheless, we append a page break after “Default
  Notation� section in order to show that spacing does not differ
  visually.
\item
  Another issue is related to Typst’s inablity to produce colored
  annotation. In order to mitigte the issue, we add a script which
  modifies annotations and make them colored.

\begin{Shaded}
\begin{Highlighting}[]
\NormalTok{../colorize{-}annotations.py \textbackslash{}}
\NormalTok{    example{-}paper.typst.pdf example{-}paper{-}colored.typst.pdf}
\end{Highlighting}
\end{Shaded}

  See
  \href{https://github.com/daskol/typst-templates/\#colored-annotations}{README.md}
  for details.
\end{itemize}

\href{/app?template=smooth-tmlr&version=0.4.0}{Create project in app}

\subsubsection{How to use}\label{how-to-use}

Click the button above to create a new project using this template in
the Typst app.

You can also use the Typst CLI to start a new project on your computer
using this command:

\begin{verbatim}
typst init @preview/smooth-tmlr:0.4.0
\end{verbatim}

\includesvg[width=0.16667in,height=0.16667in]{/assets/icons/16-copy.svg}

\subsubsection{About}\label{about}

\begin{description}
\tightlist
\item[Author :]
\href{mailto:d.bershatsky2@skoltech.ru}{Daniel Bershatsky}
\item[License:]
MIT
\item[Current version:]
0.4.0
\item[Last updated:]
April 29, 2024
\item[First released:]
March 28, 2024
\item[Minimum Typst version:]
0.10.0
\item[Archive size:]
21.3 kB
\href{https://packages.typst.org/preview/smooth-tmlr-0.4.0.tar.gz}{\pandocbounded{\includesvg[keepaspectratio]{/assets/icons/16-download.svg}}}
\item[Repository:]
\href{https://github.com/daskol/typst-templates}{GitHub}
\item[Discipline s :]
\begin{itemize}
\tightlist
\item[]
\item
  \href{https://typst.app/universe/search/?discipline=computer-science}{Computer
  Science}
\item
  \href{https://typst.app/universe/search/?discipline=mathematics}{Mathematics}
\end{itemize}
\item[Categor y :]
\begin{itemize}
\tightlist
\item[]
\item
  \pandocbounded{\includesvg[keepaspectratio]{/assets/icons/16-atom.svg}}
  \href{https://typst.app/universe/search/?category=paper}{Paper}
\end{itemize}
\end{description}

\subsubsection{Where to report issues?}\label{where-to-report-issues}

This template is a project of Daniel Bershatsky . Report issues on
\href{https://github.com/daskol/typst-templates}{their repository} . You
can also try to ask for help with this template on the
\href{https://forum.typst.app}{Forum} .

Please report this template to the Typst team using the
\href{https://typst.app/contact}{contact form} if you believe it is a
safety hazard or infringes upon your rights.

\phantomsection\label{versions}
\subsubsection{Version history}\label{version-history}

\begin{longtable}[]{@{}ll@{}}
\toprule\noalign{}
Version & Release Date \\
\midrule\noalign{}
\endhead
\bottomrule\noalign{}
\endlastfoot
0.4.0 & April 29, 2024 \\
\href{https://typst.app/universe/package/smooth-tmlr/0.3.0/}{0.3.0} &
March 28, 2024 \\
\end{longtable}

Typst GmbH did not create this template and cannot guarantee correct
functionality of this template or compatibility with any version of the
Typst compiler or app.


\section{Package List LaTeX/abbr.tex}
\title{typst.app/universe/package/abbr}

\phantomsection\label{banner}
\section{abbr}\label{abbr}

{ 0.1.0 }

An Abbreviations package.

\phantomsection\label{readme}
Short package for making the handling of abbreviations, acronyms, and
initialisms \emph{easy} .

Declare your abbreviations anywhere, use everywhere â€`` they get picked
up automatically.

\subsection{Features}\label{features}

\begin{itemize}
\tightlist
\item
  Automatic plurals, with optional overrides.
\item
  Automatic 1- or 2-column sorted list of abbreviations
\item
  Automatic links to list of abbreviations, if included.
\item
  styling configuration
\end{itemize}

\subsection{Getting started}\label{getting-started}

\begin{Shaded}
\begin{Highlighting}[]
\NormalTok{\#import "@preview/abbr:0.1.0"}

\NormalTok{\#abbr.list()}
\NormalTok{\#abbr.make(}
\NormalTok{  ("PDE", "Partial Differential Equation"),}
\NormalTok{  ("BC", "Boundary Condition"),}
\NormalTok{  ("DOF", "Degree of Freedom", "Degrees of Freedom"),}
\NormalTok{)}

\NormalTok{= Constrained Equations}

\NormalTok{\#abbr.pla[BC] constrain the \#abbr.pla[DOF] of the \#abbr.pla[PDE] they act on.\textbackslash{}}
\NormalTok{\#abbr.pla[BC] constrain the \#abbr.pla[DOF] of the \#abbr.pla[PDE] they act on.}

\NormalTok{\#abbr.add("MOL", "Method of Lines")}
\NormalTok{The \#abbr.a[MOL] is a procedure to solve \#abbr.pla[PDE] in time.}
\end{Highlighting}
\end{Shaded}

\pandocbounded{\includegraphics[keepaspectratio]{https://github.com/typst/packages/raw/main/packages/preview/abbr/0.1.0/example.png}}

\subsection{API Reference}\label{api-reference}

\subsubsection{Configuration}\label{configuration}

\begin{itemize}
\tightlist
\item
  \textbf{style} \texttt{\ (func)\ }\\
  Set a callable for styling the short version in the text.
\end{itemize}

\subsubsection{Creation}\label{creation}

\begin{itemize}
\item
  \textbf{add} \texttt{\ (short,\ long,\ long-plural)\ }\\
  Add single entry to use later.\\
  \texttt{\ long-plural\ } is \emph{optional} , if not given but used,
  an \texttt{\ s\ } is appended to create a plural.
\item
  \textbf{make} \texttt{\ (list,\ of,\ entries)\ }\\
  Add multiple entries, each of the form
  \texttt{\ (short,\ long,\ long-plural)\ } .
\end{itemize}

\subsubsection{Listing}\label{listing}

\begin{itemize}
\tightlist
\item
  \textbf{list} \texttt{\ (title)\ }\\
  Create an outline with all abbreviations in short and expanded form
\end{itemize}

\subsubsection{Usage}\label{usage}

\begin{itemize}
\tightlist
\item
  \textbf{s} \texttt{\ ()\ } - short\\
  force short form of abbreviation
\item
  \textbf{l} \texttt{\ ()\ } - long\\
  force long form of abbreviation
\item
  \textbf{a} \texttt{\ ()\ } - auto\\
  first occurence will be long form, the rest short
\item
  \textbf{pls} \texttt{\ ()\ } - plural short\\
  plural short form
\item
  \textbf{pll} \texttt{\ ()\ } - plural long\\
  plural long form
\item
  \textbf{pl} \texttt{\ ()\ } - plural automatic\\
  plural. first occurence long form, then short
\end{itemize}

\subsection{Why yet another Abbreviations
package?}\label{why-yet-another-abbreviations-package}

This mostly exists because I started working on it before checking if
somebody already made a package for it. After I saw that e.g.
\texttt{\ acrotastic\ } exists, I kept convincing myself a new package
still makes sense for the following reasons:

\begin{itemize}
\tightlist
\item
  Getting to know Typst
\item
  More automatic handling than other packages
\item
  Ability to keep keys as {[}Content{]} instead of having to stringify
  everything
\end{itemize}

Especially the last part seems to lower the friction of writing for me.
It seems silly, I know.

\subsection{Contributing}\label{contributing}

Please head over to the \href{https://sr.ht/~slowjo/typst-packages}{hub}
to find the mailing list and ticket tracker.

Or simply reach out on IRC (
\href{https://web.libera.chat/gamja/?autojoin=\#typst}{\#typst on
libera.chat} )!

\subsubsection{How to add}\label{how-to-add}

Copy this into your project and use the import as \texttt{\ abbr\ }

\begin{verbatim}
#import "@preview/abbr:0.1.0"
\end{verbatim}

\includesvg[width=0.16667in,height=0.16667in]{/assets/icons/16-copy.svg}

Check the docs for
\href{https://typst.app/docs/reference/scripting/\#packages}{more
information on how to import packages} .

\subsubsection{About}\label{about}

\begin{description}
\tightlist
\item[Author :]
\href{mailto:slowjo@halmen.xyz}{Jonathan Halmen}
\item[License:]
MIT
\item[Current version:]
0.1.0
\item[Last updated:]
November 5, 2024
\item[First released:]
November 5, 2024
\item[Archive size:]
3.40 kB
\href{https://packages.typst.org/preview/abbr-0.1.0.tar.gz}{\pandocbounded{\includesvg[keepaspectratio]{/assets/icons/16-download.svg}}}
\item[Repository:]
\href{https://git.sr.ht/~slowjo/typst-abbr}{git.sr.ht}
\item[Categor y :]
\begin{itemize}
\tightlist
\item[]
\item
  \pandocbounded{\includesvg[keepaspectratio]{/assets/icons/16-list-unordered.svg}}
  \href{https://typst.app/universe/search/?category=model}{Model}
\end{itemize}
\end{description}

\subsubsection{Where to report issues?}\label{where-to-report-issues}

This package is a project of Jonathan Halmen . Report issues on
\href{https://git.sr.ht/~slowjo/typst-abbr}{their repository} . You can
also try to ask for help with this package on the
\href{https://forum.typst.app}{Forum} .

Please report this package to the Typst team using the
\href{https://typst.app/contact}{contact form} if you believe it is a
safety hazard or infringes upon your rights.

\phantomsection\label{versions}
\subsubsection{Version history}\label{version-history}

\begin{longtable}[]{@{}ll@{}}
\toprule\noalign{}
Version & Release Date \\
\midrule\noalign{}
\endhead
\bottomrule\noalign{}
\endlastfoot
0.1.0 & November 5, 2024 \\
\end{longtable}

Typst GmbH did not create this package and cannot guarantee correct
functionality of this package or compatibility with any version of the
Typst compiler or app.


\section{Package List LaTeX/kouhu.tex}
\title{typst.app/universe/package/kouhu}

\phantomsection\label{banner}
\section{kouhu}\label{kouhu}

{ 0.1.1 }

Chinese lipsum text generator; 中æ--‡ä¹±æ•°å?‡æ--‡ï¼ˆLorem
Ipsum)ç''Ÿæˆ?器

\phantomsection\label{readme}
\texttt{\ kouhu\ } is a Chinese lipsum text generator for
\href{https://typst.app/}{Typst} . It provides a set of built-in text
samples containing both Simplified and Traditional Chinese characters.
You can choose from generated fake text, classic or modern Chinese
literature, or specify your own text.

\texttt{\ kouhu\ } is inspired by
\href{https://ctan.org/pkg/zhlipsum}{zhlipsum} LaTeX package and
\href{https://typst.app/universe/package/roremu}{roremu} Typst package.

All
\href{https://github.com/typst/packages/raw/main/packages/preview/kouhu/0.1.1/data/zhlipsum.json}{sample
text} is excerpted from \texttt{\ zhlipsum\ } LaTeX package (see
Appendix for details).

\subsection{Usage}\label{usage}

\begin{Shaded}
\begin{Highlighting}[]
\NormalTok{\#import "@preview/kouhu:0.1.0": kouhu}

\NormalTok{\#kouhu(indicies: range(1, 3)) // select the first 3 paragraphs from default text}

\NormalTok{\#kouhu(builtin{-}text: "zhufu", offset: 5, length: 100) // select 100 characters from the 5th paragraph of "zhufu" text}

\NormalTok{\#kouhu(custom{-}text: ("Foo", "Bar")) // provide your own text}
\end{Highlighting}
\end{Shaded}

See
\href{https://github.com/Harry-Chen/kouhu/blob/master/doc/manual.pdf}{manual}
for more details.

\subsection{\texorpdfstring{What does \texttt{\ kouhu\ }
mean?}{What does  kouhu  mean?}}\label{what-does-kouhu-mean}

GitHub Copilot says:

\begin{quote}
\texttt{\ kouhu\ } (�胡) is a Chinese term for reading aloud without
understanding the meaning. It is often used in the context of learning
Chinese language or reciting Chinese literature.
\end{quote}

\subsection{Changelog}\label{changelog}

\subsubsection{0.1.1}\label{section}

\begin{itemize}
\tightlist
\item
  Fix some wrong paths in \texttt{\ README.md\ } .
\item
  Fix genearation of \texttt{\ indicies\ } when not specified by user.
\item
  Add repetition of text until \texttt{\ length\ } is reached.
\end{itemize}

\subsubsection{0.1.0}\label{section-1}

\begin{itemize}
\tightlist
\item
  Initial release.
\end{itemize}

\subsection{Appendix}\label{appendix}

\subsubsection{\texorpdfstring{Generating
\texttt{\ zhlipsum.json\ }}{Generating  zhlipsum.json }}\label{generating-zhlipsum.json}

First download the \texttt{\ zhlipsum-text.dtx\ } from
\href{https://ctan.org/pkg/zhlipsum}{CTAN} or from local TeX Live (
\texttt{\ kpsewhich\ zhlipsum-text.dtx\ } ). Then run:

\begin{Shaded}
\begin{Highlighting}[]
\ExtensionTok{python3}\NormalTok{ utils/generate\_zhlipsum.py /path/to/zhlipsum{-}text.dtx src/zhlipsum.json}
\end{Highlighting}
\end{Shaded}

\subsubsection{How to add}\label{how-to-add}

Copy this into your project and use the import as \texttt{\ kouhu\ }

\begin{verbatim}
#import "@preview/kouhu:0.1.1"
\end{verbatim}

\includesvg[width=0.16667in,height=0.16667in]{/assets/icons/16-copy.svg}

Check the docs for
\href{https://typst.app/docs/reference/scripting/\#packages}{more
information on how to import packages} .

\subsubsection{About}\label{about}

\begin{description}
\tightlist
\item[Author :]
\href{mailto:harry-chen@outlook.com}{Shengqi Chen}
\item[License:]
MIT
\item[Current version:]
0.1.1
\item[Last updated:]
September 30, 2024
\item[First released:]
September 27, 2024
\item[Archive size:]
905 kB
\href{https://packages.typst.org/preview/kouhu-0.1.1.tar.gz}{\pandocbounded{\includesvg[keepaspectratio]{/assets/icons/16-download.svg}}}
\item[Repository:]
\href{https://github.com/Harry-Chen/kouhu}{GitHub}
\item[Categor y :]
\begin{itemize}
\tightlist
\item[]
\item
  \pandocbounded{\includesvg[keepaspectratio]{/assets/icons/16-hammer.svg}}
  \href{https://typst.app/universe/search/?category=utility}{Utility}
\end{itemize}
\end{description}

\subsubsection{Where to report issues?}\label{where-to-report-issues}

This package is a project of Shengqi Chen . Report issues on
\href{https://github.com/Harry-Chen/kouhu}{their repository} . You can
also try to ask for help with this package on the
\href{https://forum.typst.app}{Forum} .

Please report this package to the Typst team using the
\href{https://typst.app/contact}{contact form} if you believe it is a
safety hazard or infringes upon your rights.

\phantomsection\label{versions}
\subsubsection{Version history}\label{version-history}

\begin{longtable}[]{@{}ll@{}}
\toprule\noalign{}
Version & Release Date \\
\midrule\noalign{}
\endhead
\bottomrule\noalign{}
\endlastfoot
0.1.1 & September 30, 2024 \\
\href{https://typst.app/universe/package/kouhu/0.1.0/}{0.1.0} &
September 27, 2024 \\
\end{longtable}

Typst GmbH did not create this package and cannot guarantee correct
functionality of this package or compatibility with any version of the
Typst compiler or app.


\section{Package List LaTeX/ttt-exam.tex}
\title{typst.app/universe/package/ttt-exam}

\phantomsection\label{banner}
\phantomsection\label{template-thumbnail}
\pandocbounded{\includegraphics[keepaspectratio]{https://packages.typst.org/preview/thumbnails/ttt-exam-0.1.2-small.webp}}

\section{ttt-exam}\label{ttt-exam}

{ 0.1.2 }

A collection of tools to make a teachers life easier (german).

\href{/app?template=ttt-exam&version=0.1.2}{Create project in app}

\phantomsection\label{readme}
\texttt{\ ttt-exam\ } is a \emph{template} to create exams and belongs
to the
\href{https://github.com/jomaway/typst-teacher-templates}{typst-teacher-tools-collection}
.

\subsection{Usage}\label{usage}

Run this command inside your terminal to init a new exam.

\begin{Shaded}
\begin{Highlighting}[]
\ExtensionTok{typst}\NormalTok{ init @preview/ttt{-}exam my{-}exam}
\end{Highlighting}
\end{Shaded}

This will scaffold the following folder structure.

\begin{Shaded}
\begin{Highlighting}[]
\NormalTok{my{-}exam/}
\NormalTok{├─ meta.toml}
\NormalTok{├─ exam.typ}
\NormalTok{├─ eval.typ}
\NormalTok{├─ justfile}
\NormalTok{└─ logo.jpg}
\end{Highlighting}
\end{Shaded}

Replace the \texttt{\ logo.jpg\ } with your schools, university, …
logo or remove it. Then edit the \texttt{\ meta.toml\ } . Edit the
\texttt{\ exam.typ\ } and replace the questions with your own. If you
like you can also remove the \texttt{\ meta.toml\ } file and specify the
values directly inside \texttt{\ exam.typ\ }

If you have installed \href{https://just.systems/}{just} you can use it
to build a \emph{student} and \emph{teacher} version of your exam by
running \texttt{\ just\ build\ } .

Here you can see an example with both versions. On the left the student
version and on the right the teachers version.

\pandocbounded{\includegraphics[keepaspectratio]{https://raw.githubusercontent.com/jomaway/typst-teacher-templates/main/ttt-exam/thumbnail.png}}

The \texttt{\ eval.typ\ } is a template for generating grade lists. You
need to add your students to \texttt{\ meta.toml\ } and add the total
amount of points.

\subsection{Features}\label{features}

You can pass the following arguments to \texttt{\ exam\ }

\begin{Shaded}
\begin{Highlighting}[]
\NormalTok{\#let exam(}
\NormalTok{  // metadata }
\NormalTok{  logo: none, // none | image}
\NormalTok{  title: "exam", // the title of the exam}
\NormalTok{  subtitle: none, // is shown below the title}
\NormalTok{  date: none,     // date of the exam, preferred type of datetime.}
\NormalTok{  class: "",      }
\NormalTok{  subject: "" ,}
\NormalTok{  authors: "",  // string | array}
\NormalTok{  // config}
\NormalTok{  solution: auto,  // if solutions are displayed can also be specified with \textasciigrave{}{-}{-}input solution=true\textasciigrave{} on the cli.}
\NormalTok{   cover: true, // true | false}
\NormalTok{   header: auto, // true | false | auto}
\NormalTok{   eval{-}table: false,  // true | false}
\NormalTok{   appendix: none, // content | none}
\NormalTok{)}
\end{Highlighting}
\end{Shaded}

\href{/app?template=ttt-exam&version=0.1.2}{Create project in app}

\subsubsection{How to use}\label{how-to-use}

Click the button above to create a new project using this template in
the Typst app.

You can also use the Typst CLI to start a new project on your computer
using this command:

\begin{verbatim}
typst init @preview/ttt-exam:0.1.2
\end{verbatim}

\includesvg[width=0.16667in,height=0.16667in]{/assets/icons/16-copy.svg}

\subsubsection{About}\label{about}

\begin{description}
\tightlist
\item[Author :]
\href{https://github.com/jomaway}{Jomaway}
\item[License:]
MIT
\item[Current version:]
0.1.2
\item[Last updated:]
May 23, 2024
\item[First released:]
April 2, 2024
\item[Minimum Typst version:]
0.11.0
\item[Archive size:]
152 kB
\href{https://packages.typst.org/preview/ttt-exam-0.1.2.tar.gz}{\pandocbounded{\includesvg[keepaspectratio]{/assets/icons/16-download.svg}}}
\item[Repository:]
\href{https://github.com/jomaway/typst-teacher-templates}{GitHub}
\item[Discipline :]
\begin{itemize}
\tightlist
\item[]
\item
  \href{https://typst.app/universe/search/?discipline=education}{Education}
\end{itemize}
\item[Categor y :]
\begin{itemize}
\tightlist
\item[]
\item
  \pandocbounded{\includesvg[keepaspectratio]{/assets/icons/16-envelope.svg}}
  \href{https://typst.app/universe/search/?category=office}{Office}
\end{itemize}
\end{description}

\subsubsection{Where to report issues?}\label{where-to-report-issues}

This template is a project of Jomaway . Report issues on
\href{https://github.com/jomaway/typst-teacher-templates}{their
repository} . You can also try to ask for help with this template on the
\href{https://forum.typst.app}{Forum} .

Please report this template to the Typst team using the
\href{https://typst.app/contact}{contact form} if you believe it is a
safety hazard or infringes upon your rights.

\phantomsection\label{versions}
\subsubsection{Version history}\label{version-history}

\begin{longtable}[]{@{}ll@{}}
\toprule\noalign{}
Version & Release Date \\
\midrule\noalign{}
\endhead
\bottomrule\noalign{}
\endlastfoot
0.1.2 & May 23, 2024 \\
\href{https://typst.app/universe/package/ttt-exam/0.1.0/}{0.1.0} & April
2, 2024 \\
\end{longtable}

Typst GmbH did not create this template and cannot guarantee correct
functionality of this template or compatibility with any version of the
Typst compiler or app.


\section{Package List LaTeX/chic-hdr.tex}
\title{typst.app/universe/package/chic-hdr}

\phantomsection\label{banner}
\section{chic-hdr}\label{chic-hdr}

{ 0.4.0 }

Typst package for creating elegant headers and footers

\phantomsection\label{readme}
\textbf{Chic-header} (chic-hdr) is a Typst package for creating elegant
headers and footers

\subsection{Usage}\label{usage}

To use this library through the Typst package manager (for Typst 0.6.0
or greater), write \texttt{\ \#import\ "@preview/chic-hdr:0.4.0":\ *\ }
at the beginning of your Typst file. Once imported, you can start using
the package by writing the instruction \texttt{\ \#show:\ chic.with()\ }
and giving any of the chic functions inside the parenthesis
\texttt{\ ()\ } .

\emph{\textbf{Important: If you are using a custom template that also
needs the \texttt{\ \#show\ } instruction to be applied, prefer to use
\texttt{\ \#show:\ chic()\ } after the template’s \texttt{\ \#show\ }
.}}

For example, the code below…

\begin{Shaded}
\begin{Highlighting}[]
\NormalTok{\#import "@preview/chic{-}hdr:0.4.0": *}

\NormalTok{\#set page(paper: "a7")}

\NormalTok{\#show: chic.with(}
\NormalTok{  chic{-}footer(}
\NormalTok{    left{-}side: strong(}
\NormalTok{        link("mailto:admin@chic.hdr", "admin@chic.hdr")}
\NormalTok{    ),}
\NormalTok{    right{-}side: chic{-}page{-}number()}
\NormalTok{  ),}
\NormalTok{  chic{-}header(}
\NormalTok{    left{-}side: emph(chic{-}heading{-}name(fill: true)),}
\NormalTok{    right{-}side: smallcaps("Example")}
\NormalTok{  ),}
\NormalTok{  chic{-}separator(1pt),}
\NormalTok{  chic{-}offset(7pt),}
\NormalTok{  chic{-}height(1.5cm)}
\NormalTok{)}

\NormalTok{= Introduction}
\NormalTok{\#lorem(30)}

\NormalTok{== Details}
\NormalTok{\#lorem(70)}
\end{Highlighting}
\end{Shaded}

…will look like this:

\subsubsection{\texorpdfstring{\protect\pandocbounded{\includegraphics[keepaspectratio]{https://github.com/typst/packages/raw/main/packages/preview/chic-hdr/0.4.0/assets/usage.png}}}{Usage example}}\label{usage-example}

\subsection{Reference}\label{reference}

\emph{Note: For a detailed explanation of the functions and parameters,
see Chic-header’s Manual.pdf.}

While using \texttt{\ \#show:\ chic.with()\ } , you can give the
following parameters inside the parenthesis:

\begin{itemize}
\tightlist
\item
  \texttt{\ width\ } : Indicates the with of headers and footers in all
  the document (default is \texttt{\ 100\%\ } ).
\item
  \texttt{\ skip\ } : Which pages must be skipped for setting its header
  and footer. Other properties changed with \texttt{\ chic-height()\ }
  or \texttt{\ chic-offset()\ } are preserved. Giving a negative index
  causes a skip of the last pages using last page as index -1(default is
  \texttt{\ ()\ } ).
\item
  \texttt{\ even\ } : Header and footer for even pages. Here, only
  \texttt{\ chic-header()\ } , \texttt{\ chic-footer()\ } and
  \texttt{\ chic-separator()\ } functions will take effect. Other
  functions must be given as an argument of \texttt{\ chic()\ } .
\item
  \texttt{\ odd\ } : Sets the header and footer for odd pages. Here,
  only \texttt{\ chic-header()\ } , \texttt{\ chic-footer()\ } and
  \texttt{\ chic-separator()\ } functions will take effect. Other
  functions must be given as an argument of \texttt{\ chic()\ } .
\item
  \texttt{\ ..functions()\ } : These are a variable number of arguments
  that corresponds to Chic-header’s style functions.
\end{itemize}

\subsubsection{Functions}\label{functions}

\begin{enumerate}
\tightlist
\item
  \texttt{\ chic-header()\ } - Sets the header content.

  \begin{itemize}
  \tightlist
  \item
    \texttt{\ v-center\ } : Whether to vertically align the header
    content, or not (default is \texttt{\ false\ } ).
  \item
    \texttt{\ side-width\ } : Custom width for the sides. It can be an
    3-element-array, length or relative length (default is
    \texttt{\ none\ } and widths are set to \texttt{\ 1fr\ } if a side
    is present).
  \item
    \texttt{\ left-side\ } : Content displayed in the left side of the
    header (default is \texttt{\ none\ } ).
  \item
    \texttt{\ center-side\ } : Content displayed in the center of the
    header (default is \texttt{\ none\ } ).
  \item
    \texttt{\ right-side\ } : Content displayed in the right side of the
    header (default is \texttt{\ none\ } ).
  \end{itemize}
\item
  \texttt{\ chic-footer()\ } - Sets the footer content.

  \begin{itemize}
  \tightlist
  \item
    \texttt{\ v-center\ } : Whether to vertically align the header
    content, or not (default is \texttt{\ false\ } ).
  \item
    \texttt{\ side-width\ } : Custom width for the sides. It can be an
    3-element-array, length or relative length (default is
    \texttt{\ none\ } and widths are set to \texttt{\ 1fr\ } if a side
    is present).
  \item
    \texttt{\ left-side\ } : Content displayed in the left side of the
    footer (default is \texttt{\ none\ } ).
  \item
    \texttt{\ center-side\ } : Content displayed in the center of the
    footer (default is \texttt{\ none\ } ).
  \item
    \texttt{\ right-side\ } : Content displayed in the right side of the
    footer (default is \texttt{\ none\ } ).
  \end{itemize}
\item
  \texttt{\ chic-separator()\ } - Sets the separator for either the
  header, the footer or both.

  \begin{itemize}
  \tightlist
  \item
    \texttt{\ on\ } : Where to apply the separator. It can be
    \texttt{\ "header"\ } , \texttt{\ "footer"\ } or \texttt{\ "both"\ }
    (default is \texttt{\ "both"\ } ).
  \item
    \texttt{\ outset\ } : Space around the separator beyond the page
    margins (default is \texttt{\ 0pt\ } ).
  \item
    \texttt{\ gutter\ } : How much spacing insert around the separator
    (default is \texttt{\ 0.65em\ } ).
  \item
    (unnamed): A length for a \texttt{\ line()\ } , a stroke for a
    \texttt{\ line()\ } , or a custom content element.
  \end{itemize}
\item
  \texttt{\ chic-styled-separator()\ } - Returns a pre-made custom
  separator for using it in \texttt{\ chic-separator()\ }

  \begin{itemize}
  \tightlist
  \item
    \texttt{\ color\ } : Separator’s color (default is
    \texttt{\ black\ } ).
  \item
    (unnamed): A string indicating the separator’s style. It can be
    \texttt{\ "double-line"\ } , \texttt{\ "center-dot"\ } ,
    \texttt{\ "bold-center"\ } , or \texttt{\ "flower-end"\ } .
  \end{itemize}
\item
  \texttt{\ chic-height()\ } - Sets the height of either the header, the
  footer or both.

  \begin{itemize}
  \tightlist
  \item
    \texttt{\ on\ } : Where to change the height. It can be
    \texttt{\ "header"\ } , \texttt{\ "footer"\ } or \texttt{\ "both"\ }
    (default is \texttt{\ "both"\ } ).
  \item
    (unnamed): A relative length (the new height value).
  \end{itemize}
\item
  \texttt{\ chic-offset()\ } - Sets the offset of either the header, the
  footer or both (relative to the page content).

  \begin{itemize}
  \tightlist
  \item
    \texttt{\ on\ } : Where to change the offset It can be
    \texttt{\ "header"\ } , \texttt{\ "footer"\ } or \texttt{\ "both"\ }
    (default is \texttt{\ "both\ } ).
  \item
    (unnamed): A relative length (the new offset value).
  \end{itemize}
\item
  \texttt{\ chic-page-number()\ } - Returns the current page number.
  Useful for header and footer \texttt{\ sides\ } . It doesn’t take
  any parameters.
\item
  \texttt{\ chic-heading-name()\ } - Returns the next heading name in
  the \texttt{\ dir\ } direction. The heading must have a lower or equal
  level than \texttt{\ level\ } . If there’re no more headings in that
  direction, and \texttt{\ fill\ } is \texttt{\ true\ } , then headings
  are sought in the other direction.

  \begin{itemize}
  \tightlist
  \item
    \texttt{\ dir\ } : Direction for searching the next heading:
    \texttt{\ "next"\ } (from the current page, get the next heading) or
    \texttt{\ "prev"\ } (from the current page, get the previous
    heading). Default is \texttt{\ "next"\ } .
  \item
    \texttt{\ fill\ } : If there’s no more headings in the
    \texttt{\ dir\ } direction, indicates whether to try to get a
    heading in the opposite direction (default is \texttt{\ false\ } ).
  \item
    \texttt{\ level\ } : Up to what level of headings should this
    function search (default is \texttt{\ 2\ } ).
  \end{itemize}
\end{enumerate}

\subsection{Gallery}\label{gallery}

\subsubsection{\texorpdfstring{\protect\pandocbounded{\includegraphics[keepaspectratio]{https://github.com/typst/packages/raw/main/packages/preview/chic-hdr/0.4.0/assets/example-1.png}}}{Example 1}}\label{example-1}

\emph{Header with \texttt{\ chic-heading-name()\ } at left, and
\texttt{\ chic-page-number()\ } at right. There’s a
\texttt{\ chic-separator()\ } of \texttt{\ 1pt\ } only for the header.}

\subsubsection{\texorpdfstring{\protect\pandocbounded{\includegraphics[keepaspectratio]{https://github.com/typst/packages/raw/main/packages/preview/chic-hdr/0.4.0/assets/example-2.png}}}{Example 2}}\label{example-2}

\emph{Footer with \texttt{\ chic-page-number()\ } at right, and a custom
\texttt{\ chic-separator()\ } showing “end of page (No. page)�
between 9 \texttt{\ \textasciitilde{}\ } symbols at each side.}

\subsection{Changelog}\label{changelog}

\subsubsection{Version 0.1.0}\label{version-0.1.0}

\begin{itemize}
\tightlist
\item
  Initial release
\item
  Implemented \texttt{\ chic-header()\ } , \texttt{\ chic-footer()\ } ,
  \texttt{\ chic-separator()\ } , \texttt{\ chic-height()\ } ,
  \texttt{\ chic-offset()\ } , \texttt{\ chic-page-number()\ } , and
  \texttt{\ chic-heading-name()\ } functions
\end{itemize}

\subsubsection{Version 0.2.0}\label{version-0.2.0}

\emph{Thanks to Slashformotion ( \url{https://github.com/slashformotion}
) for noticing this version bugs, and suggesting a vertical alignment
for headers.}

\begin{itemize}
\tightlist
\item
  Fix alignment error in \texttt{\ chic-header()\ } and
  \texttt{\ chic-footer()\ }
\item
  Add \texttt{\ v-center\ } option for \texttt{\ chic-header()\ } and
  \texttt{\ chic-footer()\ }
\item
  Add \texttt{\ outset\ } option for \texttt{\ chic-separator()\ }
\item
  Add \texttt{\ chic-styled-separator()\ } function
\end{itemize}

\subsubsection{Version 0.3.0}\label{version-0.3.0}

\begin{itemize}
\tightlist
\item
  Add \texttt{\ side-width\ } option for \texttt{\ chic-header()\ } and
  \texttt{\ chic-footer()\ }
\end{itemize}

\subsubsection{Version 0.4.0}\label{version-0.4.0}

\emph{Thanks to David ( \url{https://github.com/davidleejy} ) for being
interested in the package and giving feedback and ideas for new
parameters}

\begin{itemize}
\tightlist
\item
  Update \texttt{\ type()\ } conditionals to met Typst 0.8.0 standards
\item
  Add \texttt{\ dir\ } , \texttt{\ fill\ } , and \texttt{\ level\ }
  parameters to \texttt{\ chic-heading-name()\ }
\item
  Allow negative indexes for skipping final pages while using
  \texttt{\ skip\ }
\item
  Include some panic alerts for types mismatch
\item
  Upload manual code in the package repository
\end{itemize}

\subsubsection{How to add}\label{how-to-add}

Copy this into your project and use the import as \texttt{\ chic-hdr\ }

\begin{verbatim}
#import "@preview/chic-hdr:0.4.0"
\end{verbatim}

\includesvg[width=0.16667in,height=0.16667in]{/assets/icons/16-copy.svg}

Check the docs for
\href{https://typst.app/docs/reference/scripting/\#packages}{more
information on how to import packages} .

\subsubsection{About}\label{about}

\begin{description}
\tightlist
\item[Author s :]
Pablo González Calderón \& Chic-hdr Contributors
\item[License:]
MIT
\item[Current version:]
0.4.0
\item[Last updated:]
December 28, 2023
\item[First released:]
August 19, 2023
\item[Archive size:]
7.64 kB
\href{https://packages.typst.org/preview/chic-hdr-0.4.0.tar.gz}{\pandocbounded{\includesvg[keepaspectratio]{/assets/icons/16-download.svg}}}
\item[Repository:]
\href{https://github.com/Pablo-Gonzalez-Calderon/chic-header-package}{GitHub}
\end{description}

\subsubsection{Where to report issues?}\label{where-to-report-issues}

This package is a project of Pablo González Calderón and Chic-hdr
Contributors . Report issues on
\href{https://github.com/Pablo-Gonzalez-Calderon/chic-header-package}{their
repository} . You can also try to ask for help with this package on the
\href{https://forum.typst.app}{Forum} .

Please report this package to the Typst team using the
\href{https://typst.app/contact}{contact form} if you believe it is a
safety hazard or infringes upon your rights.

\phantomsection\label{versions}
\subsubsection{Version history}\label{version-history}

\begin{longtable}[]{@{}ll@{}}
\toprule\noalign{}
Version & Release Date \\
\midrule\noalign{}
\endhead
\bottomrule\noalign{}
\endlastfoot
0.4.0 & December 28, 2023 \\
\href{https://typst.app/universe/package/chic-hdr/0.3.0/}{0.3.0} &
September 11, 2023 \\
\href{https://typst.app/universe/package/chic-hdr/0.2.0/}{0.2.0} &
August 19, 2023 \\
\href{https://typst.app/universe/package/chic-hdr/0.1.0/}{0.1.0} &
August 19, 2023 \\
\end{longtable}

Typst GmbH did not create this package and cannot guarantee correct
functionality of this package or compatibility with any version of the
Typst compiler or app.


\section{Package List LaTeX/cereal-words.tex}
\title{typst.app/universe/package/cereal-words}

\phantomsection\label{banner}
\phantomsection\label{template-thumbnail}
\pandocbounded{\includegraphics[keepaspectratio]{https://packages.typst.org/preview/thumbnails/cereal-words-0.1.0-small.webp}}

\section{cereal-words}\label{cereal-words}

{ 0.1.0 }

Time to kill? Search for words in a box of letters!

\href{/app?template=cereal-words&version=0.1.0}{Create project in app}

\phantomsection\label{readme}
Oh no, the Typst guys jumbled the letters! Bring order into this mess by
finding the hidden words.

This small game is playable in the Typst editor and best enjoyed with
the web app or \texttt{\ typst\ watch\ } . It was first released for the
24 Days to Christmas campaign in winter of 2023.

\subsection{Usage}\label{usage}

You can use this template in the Typst web app by clicking “Start from
template� on the dashboard and searching for \texttt{\ cereal-words\ }
.

Alternatively, you can use the CLI to kick this project off using the
command

\begin{verbatim}
typst init @preview/cereal-words
\end{verbatim}

Typst will create a new directory with all the files needed to get you
started.

\subsection{Configuration}\label{configuration}

This template exports the \texttt{\ game\ } function, which accepts a
single positional argument for the game input.

The template will initialize your package with a sample call to the
\texttt{\ game\ } function in a show rule. If you want to change an
existing project to use this template, you can add a show rule like this
at the top of your file:

\begin{Shaded}
\begin{Highlighting}[]
\NormalTok{\#import "@preview/cereal{-}words:0.1.0": game}
\NormalTok{\#show: game}

\NormalTok{// Type the words here}
\end{Highlighting}
\end{Shaded}

\href{/app?template=cereal-words&version=0.1.0}{Create project in app}

\subsubsection{How to use}\label{how-to-use}

Click the button above to create a new project using this template in
the Typst app.

You can also use the Typst CLI to start a new project on your computer
using this command:

\begin{verbatim}
typst init @preview/cereal-words:0.1.0
\end{verbatim}

\includesvg[width=0.16667in,height=0.16667in]{/assets/icons/16-copy.svg}

\subsubsection{About}\label{about}

\begin{description}
\tightlist
\item[Author :]
\href{https://typst.app}{Typst GmbH}
\item[License:]
MIT-0
\item[Current version:]
0.1.0
\item[Last updated:]
March 6, 2024
\item[First released:]
March 6, 2024
\item[Minimum Typst version:]
0.8.0
\item[Archive size:]
3.29 kB
\href{https://packages.typst.org/preview/cereal-words-0.1.0.tar.gz}{\pandocbounded{\includesvg[keepaspectratio]{/assets/icons/16-download.svg}}}
\item[Repository:]
\href{https://github.com/typst/templates}{GitHub}
\item[Categor y :]
\begin{itemize}
\tightlist
\item[]
\item
  \pandocbounded{\includesvg[keepaspectratio]{/assets/icons/16-smile.svg}}
  \href{https://typst.app/universe/search/?category=fun}{Fun}
\end{itemize}
\end{description}

\subsubsection{Where to report issues?}\label{where-to-report-issues}

This template is a project of Typst GmbH . Report issues on
\href{https://github.com/typst/templates}{their repository} . You can
also try to ask for help with this template on the
\href{https://forum.typst.app}{Forum} .

\phantomsection\label{versions}
\subsubsection{Version history}\label{version-history}

\begin{longtable}[]{@{}ll@{}}
\toprule\noalign{}
Version & Release Date \\
\midrule\noalign{}
\endhead
\bottomrule\noalign{}
\endlastfoot
0.1.0 & March 6, 2024 \\
\end{longtable}


\section{Package List LaTeX/sigfig.tex}
\title{typst.app/universe/package/sigfig}

\phantomsection\label{banner}
\section{sigfig}\label{sigfig}

{ 0.1.0 }

Typst library for rounding numbers with significant figures and
measurement uncertainty.

\phantomsection\label{readme}
\texttt{\ sigfig\ } is a \href{https://typst.app/}{Typst} package for
rounding numbers with
\href{https://en.wikipedia.org/wiki/Significant_figures}{significant
figures} and
\href{https://en.wikipedia.org/wiki/Measurement_uncertainty}{measurement
uncertainty} .

\subsection{Overview}\label{overview}

\begin{Shaded}
\begin{Highlighting}[]
\NormalTok{\#import "@preview/sigfig:0.1.0": round, urounds}
\NormalTok{\#import "@preview/unify:0.5.0": num}

\NormalTok{$ \#num(round(98654, 3)) $}
\NormalTok{$ \#num(round(2.8977729e{-}3, 4)) $}
\NormalTok{$ \#num(round({-}.0999, 2)) $}
\NormalTok{$ \#num(urounds(114514.19, 1.98)) $}
\NormalTok{$ \#num(urounds(1234.5678, 0.096)) $}
\end{Highlighting}
\end{Shaded}

yields

\includegraphics[width=2.5in,height=\textheight,keepaspectratio]{https://github.com/typst/packages/assets/20166026/f3d69c3c-bc67-484f-81f9-80a10913fd11}

\subsection{Documentation}\label{documentation}

\subsubsection{\texorpdfstring{\texttt{\ round\ }}{ round }}\label{round}

\texttt{\ round\ } is similar to JavaScript’s
\texttt{\ Number.prototype.toPrecision()\ } (
\href{https://tc39.es/ecma262/multipage/numbers-and-dates.html\#sec-number.prototype.toprecision}{ES
spec} ).

\begin{Shaded}
\begin{Highlighting}[]
\NormalTok{\#assert(round(114514, 3) == "1.15e5")}
\NormalTok{\#assert(round(1, 5) == "1.0000")}
\NormalTok{\#assert(round(12345, 10) == "12345.00000")}
\NormalTok{\#assert(round(.00000002468, 3) == "2.47e{-}8")}
\end{Highlighting}
\end{Shaded}

Note that what is different from the ES spec is that there will be no
sign (\$+\$) if the exponent after \texttt{\ e\ } is positive.

\subsubsection{\texorpdfstring{\texttt{\ uround\ }}{ uround }}\label{uround}

\texttt{\ uround\ } rounds a number with its uncertainty, and returns a
string of both.

\begin{Shaded}
\begin{Highlighting}[]
\NormalTok{\#assert(uround(114514, 1919) == "1.15e5+{-}2e3")}
\NormalTok{\#assert(uround(114514.0, 1.9) == "114514+{-}2")}
\end{Highlighting}
\end{Shaded}

\subsubsection{\texorpdfstring{\texttt{\ urounds\ }}{ urounds }}\label{urounds}

\texttt{\ uround\ } rounds a number with its uncertainty, and returns a
string of both with the same exponent, if any.

You can use \texttt{\ num\ } in \texttt{\ unify\ } to display the
result.

\subsection{License}\label{license}

MIT © 2024 OverflowCat (
\href{https://about.overflow.cat/}{overflow.cat} ).

\subsubsection{How to add}\label{how-to-add}

Copy this into your project and use the import as \texttt{\ sigfig\ }

\begin{verbatim}
#import "@preview/sigfig:0.1.0"
\end{verbatim}

\includesvg[width=0.16667in,height=0.16667in]{/assets/icons/16-copy.svg}

Check the docs for
\href{https://typst.app/docs/reference/scripting/\#packages}{more
information on how to import packages} .

\subsubsection{About}\label{about}

\begin{description}
\tightlist
\item[Author :]
OverflowCat
\item[License:]
MIT
\item[Current version:]
0.1.0
\item[Last updated:]
June 17, 2024
\item[First released:]
June 17, 2024
\item[Archive size:]
3.73 kB
\href{https://packages.typst.org/preview/sigfig-0.1.0.tar.gz}{\pandocbounded{\includesvg[keepaspectratio]{/assets/icons/16-download.svg}}}
\item[Repository:]
\href{https://github.com/OverflowCat/sigfig}{GitHub}
\item[Discipline :]
\begin{itemize}
\tightlist
\item[]
\item
  \href{https://typst.app/universe/search/?discipline=engineering}{Engineering}
\end{itemize}
\end{description}

\subsubsection{Where to report issues?}\label{where-to-report-issues}

This package is a project of OverflowCat . Report issues on
\href{https://github.com/OverflowCat/sigfig}{their repository} . You can
also try to ask for help with this package on the
\href{https://forum.typst.app}{Forum} .

Please report this package to the Typst team using the
\href{https://typst.app/contact}{contact form} if you believe it is a
safety hazard or infringes upon your rights.

\phantomsection\label{versions}
\subsubsection{Version history}\label{version-history}

\begin{longtable}[]{@{}ll@{}}
\toprule\noalign{}
Version & Release Date \\
\midrule\noalign{}
\endhead
\bottomrule\noalign{}
\endlastfoot
0.1.0 & June 17, 2024 \\
\end{longtable}

Typst GmbH did not create this package and cannot guarantee correct
functionality of this package or compatibility with any version of the
Typst compiler or app.


\section{Package List LaTeX/k-mapper.tex}
\title{typst.app/universe/package/k-mapper}

\phantomsection\label{banner}
\section{k-mapper}\label{k-mapper}

{ 1.1.0 }

A package to add Karnaugh maps into Typst projects.

\phantomsection\label{readme}
ðŸ``-- See the \texttt{\ k-mapper\ } Manual
\href{https://github.com/derekchai/k-mapper/blob/1f334d9e0f02cc656c01835302474bf728db9f80/manual.pdf}{here}
! The Manual features much more documentation, and is typeset using
Typst.

This is a package for adding Karnaugh maps into your Typst projects.

See the changelog for the package
\href{https://github.com/derekchai/k-mapper/blob/698e8554ce67e3a61dd30319ab8f712a6a6b8daa/changelog.md}{here}
.

\subsection{Features}\label{features}

\begin{itemize}
\tightlist
\item
  2-variable (2 by 2) Karnaugh maps
\item
  3-variable (2 by 4) Karnaugh maps
\item
  4-variable (4 by 4) Karnaugh maps
\end{itemize}

\subsection{Getting Started}\label{getting-started}

Simply import \texttt{\ k-mapper\ } using the Typst package manager to
begin using \texttt{\ k-mapper\ } within your Typst documents.

\begin{Shaded}
\begin{Highlighting}[]
\NormalTok{\#import "@preview/k{-}mapper:1.1.0": *}
\end{Highlighting}
\end{Shaded}

\subsection{Example}\label{example}

\begin{Shaded}
\begin{Highlighting}[]
\NormalTok{  \#karnaugh(}
\NormalTok{    16,}
\NormalTok{    x{-}label: $C D$,}
\NormalTok{    y{-}label: $A B$,}
\NormalTok{    manual{-}terms: (}
\NormalTok{      0, 1, 2, 3, 4, 5, 6, 7, 8, }
\NormalTok{      9, 10, 11, 12, 13, 14, 15}
\NormalTok{    ),}
\NormalTok{    implicants: ((5, 7), (5, 13), (15, 15)),}
\NormalTok{    vertical{-}implicants: ((1, 11), ),}
\NormalTok{    horizontal{-}implicants: ((4, 14), ),}
\NormalTok{    corner{-}implicants: true,}
\NormalTok{  )}
\end{Highlighting}
\end{Shaded}

\pandocbounded{\includegraphics[keepaspectratio]{https://raw.githubusercontent.com/derekchai/k-mapper/005cb0a839499d0dfa90ee48d2128d41e582d755/readme-example.png}}

For more detailed documentation and examples, including function
parameters, see the Manual
\href{https://github.com/derekchai/k-mapper/blob/1f334d9e0f02cc656c01835302474bf728db9f80/manual.pdf}{PDF}
and
\href{https://github.com/derekchai/k-mapper/blob/1f334d9e0f02cc656c01835302474bf728db9f80/manual.typ}{Typst
file} in the
\href{https://github.com/derekchai/typst-karnaugh-map}{Github repo} .

\subsubsection{How to add}\label{how-to-add}

Copy this into your project and use the import as \texttt{\ k-mapper\ }

\begin{verbatim}
#import "@preview/k-mapper:1.1.0"
\end{verbatim}

\includesvg[width=0.16667in,height=0.16667in]{/assets/icons/16-copy.svg}

Check the docs for
\href{https://typst.app/docs/reference/scripting/\#packages}{more
information on how to import packages} .

\subsubsection{About}\label{about}

\begin{description}
\tightlist
\item[Author :]
Derek Chai
\item[License:]
MIT
\item[Current version:]
1.1.0
\item[Last updated:]
May 14, 2024
\item[First released:]
May 13, 2024
\item[Archive size:]
4.52 kB
\href{https://packages.typst.org/preview/k-mapper-1.1.0.tar.gz}{\pandocbounded{\includesvg[keepaspectratio]{/assets/icons/16-download.svg}}}
\item[Repository:]
\href{https://github.com/derekchai/typst-karnaugh-map}{GitHub}
\item[Categor y :]
\begin{itemize}
\tightlist
\item[]
\item
  \pandocbounded{\includesvg[keepaspectratio]{/assets/icons/16-chart.svg}}
  \href{https://typst.app/universe/search/?category=visualization}{Visualization}
\end{itemize}
\end{description}

\subsubsection{Where to report issues?}\label{where-to-report-issues}

This package is a project of Derek Chai . Report issues on
\href{https://github.com/derekchai/typst-karnaugh-map}{their repository}
. You can also try to ask for help with this package on the
\href{https://forum.typst.app}{Forum} .

Please report this package to the Typst team using the
\href{https://typst.app/contact}{contact form} if you believe it is a
safety hazard or infringes upon your rights.

\phantomsection\label{versions}
\subsubsection{Version history}\label{version-history}

\begin{longtable}[]{@{}ll@{}}
\toprule\noalign{}
Version & Release Date \\
\midrule\noalign{}
\endhead
\bottomrule\noalign{}
\endlastfoot
1.1.0 & May 14, 2024 \\
\href{https://typst.app/universe/package/k-mapper/1.0.0/}{1.0.0} & May
13, 2024 \\
\end{longtable}

Typst GmbH did not create this package and cannot guarantee correct
functionality of this package or compatibility with any version of the
Typst compiler or app.


\section{Package List LaTeX/edgeframe.tex}
\title{typst.app/universe/package/edgeframe}

\phantomsection\label{banner}
\section{edgeframe}\label{edgeframe}

{ 0.1.0 }

For quick paper setups.

\phantomsection\label{readme}
Custom margins and other components for page setup or layout.

\subsection{Usage}\label{usage}

Add the package with the following code. Remember to add the asterisk
\texttt{\ :\ *\ } at the end.

\begin{Shaded}
\begin{Highlighting}[]
\NormalTok{\#include "@preview/edgeframe:0.1.0": *}
\end{Highlighting}
\end{Shaded}

\begin{Shaded}
\begin{Highlighting}[]
\NormalTok{\#set page(margin: margin{-}normal)}
\end{Highlighting}
\end{Shaded}

\subsection{List of parameters}\label{list-of-parameters}

\begin{itemize}
\tightlist
\item
  margin-normal
\item
  margin-narrow
\item
  margin-moderate-x
\item
  margin-moderate-y
\item
  margin-wide-x
\item
  margin-wide-y
\item
  margin-a5-x
\item
  margin-a5-y
\end{itemize}

\begin{quote}
Parameters with \texttt{\ x\ } and \texttt{\ y\ } should to be used
together

\begin{verbatim}
#set page(margin: (x: margin-moderate-x, y: margin-moderate-y))
\end{verbatim}
\end{quote}

\subsubsection{How to add}\label{how-to-add}

Copy this into your project and use the import as \texttt{\ edgeframe\ }

\begin{verbatim}
#import "@preview/edgeframe:0.1.0"
\end{verbatim}

\includesvg[width=0.16667in,height=0.16667in]{/assets/icons/16-copy.svg}

Check the docs for
\href{https://typst.app/docs/reference/scripting/\#packages}{more
information on how to import packages} .

\subsubsection{About}\label{about}

\begin{description}
\tightlist
\item[Author :]
\href{https://github.com/neuralpain}{neuralpain}
\item[License:]
MIT
\item[Current version:]
0.1.0
\item[Last updated:]
November 29, 2024
\item[First released:]
November 29, 2024
\item[Archive size:]
1.44 kB
\href{https://packages.typst.org/preview/edgeframe-0.1.0.tar.gz}{\pandocbounded{\includesvg[keepaspectratio]{/assets/icons/16-download.svg}}}
\item[Repository:]
\href{https://github.com/neuralpain/edgeframe}{GitHub}
\item[Categor ies :]
\begin{itemize}
\tightlist
\item[]
\item
  \pandocbounded{\includesvg[keepaspectratio]{/assets/icons/16-envelope.svg}}
  \href{https://typst.app/universe/search/?category=office}{Office}
\item
  \pandocbounded{\includesvg[keepaspectratio]{/assets/icons/16-layout.svg}}
  \href{https://typst.app/universe/search/?category=layout}{Layout}
\end{itemize}
\end{description}

\subsubsection{Where to report issues?}\label{where-to-report-issues}

This package is a project of neuralpain . Report issues on
\href{https://github.com/neuralpain/edgeframe}{their repository} . You
can also try to ask for help with this package on the
\href{https://forum.typst.app}{Forum} .

Please report this package to the Typst team using the
\href{https://typst.app/contact}{contact form} if you believe it is a
safety hazard or infringes upon your rights.

\phantomsection\label{versions}
\subsubsection{Version history}\label{version-history}

\begin{longtable}[]{@{}ll@{}}
\toprule\noalign{}
Version & Release Date \\
\midrule\noalign{}
\endhead
\bottomrule\noalign{}
\endlastfoot
0.1.0 & November 29, 2024 \\
\end{longtable}

Typst GmbH did not create this package and cannot guarantee correct
functionality of this package or compatibility with any version of the
Typst compiler or app.


\section{Package List LaTeX/definitely-not-tuw-thesis.tex}
\title{typst.app/universe/package/definitely-not-tuw-thesis}

\phantomsection\label{banner}
\phantomsection\label{template-thumbnail}
\pandocbounded{\includegraphics[keepaspectratio]{https://packages.typst.org/preview/thumbnails/definitely-not-tuw-thesis-0.1.0-small.webp}}

\section{definitely-not-tuw-thesis}\label{definitely-not-tuw-thesis}

{ 0.1.0 }

An unofficial template for a thesis at the TU Wien informatics
institute.

\href{/app?template=definitely-not-tuw-thesis&version=0.1.0}{Create
project in app}

\phantomsection\label{readme}
An example thesis can be viewed here:
\url{https://otto-aa.github.io/definitely-not-tuw-thesis/thesis.pdf}

\subsection{Usage}\label{usage}

You can download the template with:

\begin{Shaded}
\begin{Highlighting}[]
\ExtensionTok{typst}\NormalTok{ init @preview/definitely{-}not{-}tuw{-}thesis}
\end{Highlighting}
\end{Shaded}

\subsubsection{Template overview}\label{template-overview}

After setting up the template, you will have the following files:

\begin{itemize}
\tightlist
\item
  \texttt{\ thesis.typ\ } : overall structure and styling, configuration
  for the cover pages and PDF metadata
\item
  \texttt{\ content/front-matter.typ\ } : acknowledgments and abstract
\item
  \texttt{\ content/main.typ\ } : all your chapters
\item
  \texttt{\ content/appendix.typ\ } : AI tools acknowledgment and other
  appendices
\item
  \texttt{\ refs.bib\ } : references
\end{itemize}

Then copy the values you get from compiling the
\href{https://gitlab.com/ThomasAUZINGER/vutinfth}{official template} ,
and paste them in \texttt{\ thesis.typ\ } . Remove all unused fields
and, finally, compare if it is close enough to the official template. If
not, please open an issue or PR to fix it.

\subsubsection{Styling}\label{styling}

If you want to adapt the styling, you can remove the
\texttt{\ show:\ ...\ } commands in the \texttt{\ thesis.typ\ } and
replace them with your own, or simply extend them with your own
\texttt{\ show:\ ...\ } commands.

\subsection{Contributing}\label{contributing}

I guess there are many ways to improve this template, feel free to do so
and submit issues and PRs! More information at
\href{https://github.com/Otto-AA/unofficial-tu-wien-thesis-template/blob/main/CONTRIBUTING.md}{CONTRIBUTING.md}

\subsection{License}\label{license}

The code is licensed under MIT-0. The ‘TU Wien Informatics’ logo and
signet are copyright of the TU Wien.

\subsection{Acknowledgments}\label{acknowledgments}

This work is based on the
\href{https://gitlab.com/ThomasAUZINGER/vutinfth}{official template}
maintained by Thomas Auzinger. The repository structure is based on
\href{https://github.com/typst-community/typst-package-template}{typst-package-template}
.

\href{/app?template=definitely-not-tuw-thesis&version=0.1.0}{Create
project in app}

\subsubsection{How to use}\label{how-to-use}

Click the button above to create a new project using this template in
the Typst app.

You can also use the Typst CLI to start a new project on your computer
using this command:

\begin{verbatim}
typst init @preview/definitely-not-tuw-thesis:0.1.0
\end{verbatim}

\includesvg[width=0.16667in,height=0.16667in]{/assets/icons/16-copy.svg}

\subsubsection{About}\label{about}

\begin{description}
\tightlist
\item[Author :]
Othmar Lechner @Otto-AA
\item[License:]
MIT-0
\item[Current version:]
0.1.0
\item[Last updated:]
July 29, 2024
\item[First released:]
July 29, 2024
\item[Archive size:]
364 kB
\href{https://packages.typst.org/preview/definitely-not-tuw-thesis-0.1.0.tar.gz}{\pandocbounded{\includesvg[keepaspectratio]{/assets/icons/16-download.svg}}}
\item[Repository:]
\href{https://github.com/Otto-AA/definitely-not-tuw-thesis}{GitHub}
\item[Categor y :]
\begin{itemize}
\tightlist
\item[]
\item
  \pandocbounded{\includesvg[keepaspectratio]{/assets/icons/16-mortarboard.svg}}
  \href{https://typst.app/universe/search/?category=thesis}{Thesis}
\end{itemize}
\end{description}

\subsubsection{Where to report issues?}\label{where-to-report-issues}

This template is a project of Othmar Lechner @Otto-AA . Report issues on
\href{https://github.com/Otto-AA/definitely-not-tuw-thesis}{their
repository} . You can also try to ask for help with this template on the
\href{https://forum.typst.app}{Forum} .

Please report this template to the Typst team using the
\href{https://typst.app/contact}{contact form} if you believe it is a
safety hazard or infringes upon your rights.

\phantomsection\label{versions}
\subsubsection{Version history}\label{version-history}

\begin{longtable}[]{@{}ll@{}}
\toprule\noalign{}
Version & Release Date \\
\midrule\noalign{}
\endhead
\bottomrule\noalign{}
\endlastfoot
0.1.0 & July 29, 2024 \\
\end{longtable}

Typst GmbH did not create this template and cannot guarantee correct
functionality of this template or compatibility with any version of the
Typst compiler or app.


\section{Package List LaTeX/a2c-nums.tex}
\title{typst.app/universe/package/a2c-nums}

\phantomsection\label{banner}
\section{a2c-nums}\label{a2c-nums}

{ 0.0.1 }

Convert a number to Chinese

\phantomsection\label{readme}
Convert Arabic numbers to Chinese characters.

\subsection{usage}\label{usage}

\begin{Shaded}
\begin{Highlighting}[]
\NormalTok{\#import "@preview/a2c{-}nums:0.0.1": int{-}to{-}cn{-}num, int{-}to{-}cn{-}ancient{-}num, int{-}to{-}cn{-}simple{-}num, num{-}to{-}cn{-}currency}

\NormalTok{\#int{-}to{-}cn{-}num(1234567890)}

\NormalTok{\#int{-}to{-}cn{-}ancient{-}num(1234567890)}

\NormalTok{\#int{-}to{-}cn{-}simple{-}num(2024)}

\NormalTok{\#num{-}to{-}cn{-}currency(1234567890.12)}
\end{Highlighting}
\end{Shaded}

\subsection{Functions}\label{functions}

\subsubsection{int-to-cn-num}\label{int-to-cn-num}

Convert an integer to Chinese number. ex:
\texttt{\ \#int-to-cn-num(123)\ } will be \texttt{\ 一百二å??三\ }

\subsubsection{int-to-cn-ancient-num}\label{int-to-cn-ancient-num}

Convert an integer to ancient Chinese number. ex:
\texttt{\ \#int-to-cn-ancient-num(123)\ } will be
\texttt{\ 壹佰贰拾å??\ }

\subsubsection{int-to-cn-simple-num}\label{int-to-cn-simple-num}

Convert an integer to simpple Chinese number. ex:
\texttt{\ \#int-to-cn-simple-num(2024)\ } will be
\texttt{\ 二〇二四\ }

\subsubsection{num-to-cn-currency}\label{num-to-cn-currency}

Convert a number to Chinese currency. ex:
\texttt{\ \#int-to-cn-simple-num(1234.56)\ } will be
\texttt{\ 壹仟贰佰å??拾肆元ä¼?角陆分\ }

\subsubsection{more details}\label{more-details}

Reference
\href{https://github.com/typst/packages/raw/main/packages/preview/a2c-nums/0.0.1/demo.typ}{demo.typ}
for more details please.

\subsubsection{How to add}\label{how-to-add}

Copy this into your project and use the import as \texttt{\ a2c-nums\ }

\begin{verbatim}
#import "@preview/a2c-nums:0.0.1"
\end{verbatim}

\includesvg[width=0.16667in,height=0.16667in]{/assets/icons/16-copy.svg}

Check the docs for
\href{https://typst.app/docs/reference/scripting/\#packages}{more
information on how to import packages} .

\subsubsection{About}\label{about}

\begin{description}
\tightlist
\item[Author :]
\href{mailto:soarowl@yeah.net}{Zhuo Nengwen}
\item[License:]
MIT
\item[Current version:]
0.0.1
\item[Last updated:]
January 8, 2024
\item[First released:]
January 8, 2024
\item[Minimum Typst version:]
0.10.0
\item[Archive size:]
2.54 kB
\href{https://packages.typst.org/preview/a2c-nums-0.0.1.tar.gz}{\pandocbounded{\includesvg[keepaspectratio]{/assets/icons/16-download.svg}}}
\item[Repository:]
\href{https://github.com/soarowl/a2c-nums.git}{GitHub}
\end{description}

\subsubsection{Where to report issues?}\label{where-to-report-issues}

This package is a project of Zhuo Nengwen . Report issues on
\href{https://github.com/soarowl/a2c-nums.git}{their repository} . You
can also try to ask for help with this package on the
\href{https://forum.typst.app}{Forum} .

Please report this package to the Typst team using the
\href{https://typst.app/contact}{contact form} if you believe it is a
safety hazard or infringes upon your rights.

\phantomsection\label{versions}
\subsubsection{Version history}\label{version-history}

\begin{longtable}[]{@{}ll@{}}
\toprule\noalign{}
Version & Release Date \\
\midrule\noalign{}
\endhead
\bottomrule\noalign{}
\endlastfoot
0.0.1 & January 8, 2024 \\
\end{longtable}

Typst GmbH did not create this package and cannot guarantee correct
functionality of this package or compatibility with any version of the
Typst compiler or app.


\section{Package List LaTeX/showman.tex}
\title{typst.app/universe/package/showman}

\phantomsection\label{banner}
\section{showman}\label{showman}

{ 0.1.2 }

Eval \& show typst code outputs inline with their source

\phantomsection\label{readme}
\pandocbounded{\includegraphics[keepaspectratio]{https://www.github.com/ntjess/showman/raw/v0.1.0/showman.jpg}}

\begin{center}\rule{0.5\linewidth}{0.5pt}\end{center}

Automagic tools to smooth the package documentation \& development
process.

\begin{itemize}
\item
  Package your files for local typst installation or PR submission in
  one command
\item
  Provide your readme in typst format with code block examples, and let
  \texttt{\ showman\ } do the rest! In one command, it will the readme
  to markdown and render code block outputs as included images.

  \begin{itemize}
  \tightlist
  \item
    Bonus points â€`` let \texttt{\ showman\ } know your repository path
    and it will ensure images will properly appear in your readme even
    after your package has been distributed through typst’s registry.
  \end{itemize}
\item
  Execute non-typst code blocks and render their outputs
\end{itemize}

\textbf{Prerequisites} : Make sure you have \texttt{\ typst\ } and
\texttt{\ pandoc\ } available from the command line. Then, in a python
virtual environment, run:

\begin{Shaded}
\begin{Highlighting}[]
\ExtensionTok{pip}\NormalTok{ install showman}
\end{Highlighting}
\end{Shaded}

Create a typst file with
\texttt{\ \textasciigrave{}\textasciigrave{}\textasciigrave{}example\ }
code blocks that show the output you want to include in your readme. For
instance:

\begin{Shaded}
\begin{Highlighting}[]
\NormalTok{\#import "@preview/cetz:0.1.2"}
\NormalTok{// Just to avoid showing this heading in the readme itself}
\NormalTok{\#set heading(outlined: false)}

\NormalTok{= Hello, world!}
\NormalTok{Let\textquotesingle{}s do something complicated:}

\NormalTok{\#cetz.canvas(\{}
\NormalTok{  import cetz.plot}
\NormalTok{  import cetz.palette}
\NormalTok{  cetz.draw.set{-}style(axes: (tick: (length: {-}.05)))}
\NormalTok{  // Plot something}
\NormalTok{  plot.plot(size: (3,3), x{-}tick{-}step: 1, axis{-}style: "left", \{}
\NormalTok{      for i in range(0, 3) \{}
\NormalTok{      plot.add(domain: ({-}4, 2),}
\NormalTok{      x =\textgreater{} calc.exp({-}(calc.pow(x + i, 2))),}
\NormalTok{      fill: true, style: palette.tango)}
\NormalTok{    \}}
\NormalTok{  \})}
\NormalTok{\})}
\end{Highlighting}
\end{Shaded}

\pandocbounded{\includegraphics[keepaspectratio]{https://www.github.com/ntjess/showman/raw/v0.1.0/assets/example-1.png}}

Then, run the following command:

\begin{Shaded}
\begin{Highlighting}[]
\ExtensionTok{showman}\NormalTok{ md }
\end{Highlighting}
\end{Shaded}

Congrats, you now have a readme with inline images 🎉

You can optionally specify your workspace root, output file name, image
folder, etc. These options are visible under
\texttt{\ showman\ md\ -\/-help\ } .

\textbf{Note} : You can customize the appearance of these images by
providing \texttt{\ showman\ } the template to use when creating them.
In your file to be rendered, create a variable called
\texttt{\ showman-config\ } at the outermost scope:

\begin{Shaded}
\begin{Highlighting}[]
\NormalTok{// Render images with a black background and red text}
\NormalTok{\#let showman{-}config = (}
\NormalTok{  template: it =\textgreater{} \{}
\NormalTok{    set text(fill: red)}
\NormalTok{    rect(fill: black, it)}
\NormalTok{  \}}
\NormalTok{)}
\end{Highlighting}
\end{Shaded}

Behind the scenes, showman imports your file as a module and looks for
this variable. If it is found, your template and a few other options are
injected into the example rendering process.

\textbf{Note} : If every example has the same setup (package imports,
etc.), and you don’t want the text to be included in your examples,
you can pass \texttt{\ eval-kwargs\ } in this config as well to specify
a string that gets prefixed to every example. Alternatively, pass
variables in a scope directly:

\begin{Shaded}
\begin{Highlighting}[]
\NormalTok{// Setup either through providing scope or import prefixes}
\NormalTok{\#let my{-}variable = 5}
\NormalTok{\#let showman{-}config = (}
\NormalTok{  eval{-}kwargs: (}
\NormalTok{    prefix: "\#import \textbackslash{}"@preview/cetz:0.1.2\textbackslash{}"}
\NormalTok{  ),}
\NormalTok{  // Now you can use \textasciigrave{}my{-}var\textasciigrave{} in your examples}
\NormalTok{  scope: (my{-}var: my{-}variable)}
\NormalTok{)}
\end{Highlighting}
\end{Shaded}

\subsection{Caveats}\label{caveats}

\begin{itemize}
\item
  \texttt{\ showman\ } uses the beautiful \texttt{\ pandoc\ } to do most
  of the markdown conversion heavy lifting. So, if your document can’t
  be processed by pandoc, you may need to adjust your syntax until
  pandoc is happy making a markdown document.
\item
  Typst doesn’t allow page styling inside containers. Since
  \texttt{\ showman\ } must use containers to extract each rendered
  example, you can’t use \texttt{\ \#set\ page(...)\ } or
  \texttt{\ \#pagebreak()\ } inside your examples.
\end{itemize}

If you don’t care about converting your readme to markdown, it’s
even easier to have example rendered alongside their code. Simply add
the following preamble to your file:

\begin{Shaded}
\begin{Highlighting}[]
\NormalTok{\#import "@preview/showman:0.1.1"}
\NormalTok{\#show: showman.formatter.template}

\NormalTok{The code below will be rendered side by side with its output:}

\NormalTok{\textasciigrave{}\textasciigrave{}\textasciigrave{} typst}
\NormalTok{= Hello world!}
\NormalTok{\textasciigrave{}\textasciigrave{}\textasciigrave{}}
\NormalTok{![Example 2](https://www.github.com/ntjess/showman/raw/v0.1.0/assets/example{-}2.png)}

\NormalTok{Several keywords can be privded to customize appearance and more. See \textasciigrave{}showman.formatter.template\textasciigrave{} for more details.}
\end{Highlighting}
\end{Shaded}

You’ve done the hard work of creating a beautiful, well-documented
manual. Now it’s time to share it with the world. \texttt{\ showman\ }
can help you package your files for distribution in one command, after
some minimal setup.

\begin{enumerate}
\item
  Make sure you have a \texttt{\ typst.toml\ } file that follows typst
  \href{https://github.com/typst/packages}{packaging guidelines}
\item
  Add a new block to your toml file as follows:
\end{enumerate}

\begin{Shaded}
\begin{Highlighting}[]
\KeywordTok{[tool.packager]}
\DataTypeTok{paths} \OperatorTok{=} \OperatorTok{[}\ErrorTok{...}\OperatorTok{]}
\end{Highlighting}
\end{Shaded}

Where \texttt{\ paths\ } is a list of files and directories you want to
include in your package.

\begin{enumerate}
\setcounter{enumi}{2}
\tightlist
\item
  Run the following command from the root of your repository:
\end{enumerate}

\begin{Shaded}
\begin{Highlighting}[]
\ExtensionTok{showman}\NormalTok{ package }
\end{Highlighting}
\end{Shaded}

\begin{enumerate}
\setcounter{enumi}{3}
\item
  Without any other arguments, you’ve just installed your package in
  your system’s local typst packages folder. Now you can import it
  with
  \texttt{\ typst\ \#import\ "@local/mypackage:\textless{}version\textgreater{}"\ }
  .

  \begin{itemize}
  \tightlist
  \item
    You can alternatively specify the path to your fork of
    \texttt{\ typst/packages\ } to prep your files for a PR, or specify
    a \texttt{\ -\/-namespace\ } other than \texttt{\ local\ } .
  \end{itemize}
\end{enumerate}

\textbf{Note} : You can see the full list of command options with
\texttt{\ showman\ package\ -\/-help\ } .

This package also executes non-typst code (currently bash on
non-windows, python, and c++). You can use
\texttt{\ showman\ execute\ ./path/to/file.typ\ } to execute code blocks
in these languages, and the output will be captured in a
\texttt{\ .coderunner.json\ } file in the root directory you specified.
To enable this feature, you need to add the following preamble to your
file:

\begin{Shaded}
\begin{Highlighting}[]
\NormalTok{\#import "@preview/showman:0.1.1": runner}

\NormalTok{\#let cache = json("/.coderunner.json").at("path/to/file.typ", default: (:))}
\NormalTok{\#let show{-}rule = runner.external{-}code.with(result{-}cache: cache)}

\NormalTok{// Now, apply the show rule to languages that have a \textasciigrave{}showman execute\textasciigrave{} result:}
\NormalTok{\#show raw.where(lang: "python"): show{-}rule}
\end{Highlighting}
\end{Shaded}

You can optionally style
\texttt{\ \textless{}example-input\textgreater{}\ } and
\texttt{\ \textless{}example-output\textgreater{}\ } labels to customize
how input and output blocks appear. For even deeper customization, you
can specify the \texttt{\ container\ } that displays the input and
output blocks that accepts a keyword \texttt{\ direction\ } and
positional \texttt{\ input\ } and \texttt{\ output\ } .

\subsubsection{How to add}\label{how-to-add}

Copy this into your project and use the import as \texttt{\ showman\ }

\begin{verbatim}
#import "@preview/showman:0.1.2"
\end{verbatim}

\includesvg[width=0.16667in,height=0.16667in]{/assets/icons/16-copy.svg}

Check the docs for
\href{https://typst.app/docs/reference/scripting/\#packages}{more
information on how to import packages} .

\subsubsection{About}\label{about}

\begin{description}
\tightlist
\item[Author :]
Nathan Jessurun
\item[License:]
Unlicense
\item[Current version:]
0.1.2
\item[Last updated:]
November 28, 2024
\item[First released:]
January 15, 2024
\item[Minimum Typst version:]
0.12.0
\item[Archive size:]
6.00 kB
\href{https://packages.typst.org/preview/showman-0.1.2.tar.gz}{\pandocbounded{\includesvg[keepaspectratio]{/assets/icons/16-download.svg}}}
\item[Repository:]
\href{https://github.com/ntjess/showman}{GitHub}
\item[Categor ies :]
\begin{itemize}
\tightlist
\item[]
\item
  \pandocbounded{\includesvg[keepaspectratio]{/assets/icons/16-code.svg}}
  \href{https://typst.app/universe/search/?category=scripting}{Scripting}
\item
  \pandocbounded{\includesvg[keepaspectratio]{/assets/icons/16-hammer.svg}}
  \href{https://typst.app/universe/search/?category=utility}{Utility}
\end{itemize}
\end{description}

\subsubsection{Where to report issues?}\label{where-to-report-issues}

This package is a project of Nathan Jessurun . Report issues on
\href{https://github.com/ntjess/showman}{their repository} . You can
also try to ask for help with this package on the
\href{https://forum.typst.app}{Forum} .

Please report this package to the Typst team using the
\href{https://typst.app/contact}{contact form} if you believe it is a
safety hazard or infringes upon your rights.

\phantomsection\label{versions}
\subsubsection{Version history}\label{version-history}

\begin{longtable}[]{@{}ll@{}}
\toprule\noalign{}
Version & Release Date \\
\midrule\noalign{}
\endhead
\bottomrule\noalign{}
\endlastfoot
0.1.2 & November 28, 2024 \\
\href{https://typst.app/universe/package/showman/0.1.1/}{0.1.1} & March
16, 2024 \\
\href{https://typst.app/universe/package/showman/0.1.0/}{0.1.0} &
January 15, 2024 \\
\end{longtable}

Typst GmbH did not create this package and cannot guarantee correct
functionality of this package or compatibility with any version of the
Typst compiler or app.


\section{Package List LaTeX/fontawesome.tex}
\title{typst.app/universe/package/fontawesome}

\phantomsection\label{banner}
\section{fontawesome}\label{fontawesome}

{ 0.5.0 }

A Typst library for Font Awesome icons through the desktop fonts.

\phantomsection\label{readme}
A Typst library for Font Awesome icons through the desktop fonts.

p.s. The library is based on the Font Awesome 6 desktop fonts (v6.6.0)

\subsection{Usage}\label{usage}

\subsubsection{Install the fonts}\label{install-the-fonts}

You can download the fonts from the official website:
\url{https://fontawesome.com/download}

After downloading the zip file, you can install the fonts depending on
your OS.

\paragraph{Typst web app}\label{typst-web-app}

You can simply upload the \texttt{\ otf\ } files to the web app and use
them with this package.

\paragraph{Mac}\label{mac}

You can double click the \texttt{\ otf\ } files to install them.

\paragraph{Windows}\label{windows}

You can right-click the \texttt{\ otf\ } files and select
\texttt{\ Install\ } .

\paragraph{Some notes}\label{some-notes}

This library is tested with the otf files of the Font Awesome Free set.
TrueType fonts may not work as expected. (Though I am not sure whether
Font Awesome provides TrueType fonts, some issue is reported with
TrueType fonts.)

\subsubsection{Import the library}\label{import-the-library}

\paragraph{Using the typst packages}\label{using-the-typst-packages}

You can install the library using the typst packages:

\texttt{\ \#import\ "@preview/fontawesome:0.5.0":\ *\ }

\paragraph{Manually install}\label{manually-install}

Copy all files start with \texttt{\ lib\ } to your project and import
the library:

\texttt{\ \#import\ "lib.typ":\ *\ }

There are three files:

\begin{itemize}
\tightlist
\item
  \texttt{\ lib.typ\ } : The main entrypoint of the library.
\item
  \texttt{\ lib-impl.typ\ } : The implementation of \texttt{\ fa-icon\ }
  .
\item
  \texttt{\ lib-gen.typ\ } : The generated icon map and functions.
\end{itemize}

I recommend renaming these files to avoid conflicts with other
libraries.

\subsubsection{Use the icons}\label{use-the-icons}

You can use the \texttt{\ fa-icon\ } function to create an icon with its
name:

\texttt{\ \#fa-icon("chess-queen")\ }

Or you can use the \texttt{\ fa-\ } prefix to create an icon with its
name:

\texttt{\ \#fa-chess-queen()\ } (This is equivalent to
\texttt{\ \#fa-icon().with("chess-queen")\ } )

You can also set \texttt{\ solid\ } to \texttt{\ true\ } to use the
solid version of the icon:

\texttt{\ \#fa-icon("chess-queen",\ solid:\ true)\ }

Some icons only have the solid version in the Free set, so you need to
set \texttt{\ solid\ } to \texttt{\ true\ } to use them if you are using
the Free set. Otherwise, you may not get the expected glyph.

\paragraph{Full list of icons}\label{full-list-of-icons}

You can find all icons on the
\href{https://fontawesome.com/search}{official website}

\paragraph{Different sets}\label{different-sets}

By default, the library supports \texttt{\ Free\ } , \texttt{\ Brands\ }
, \texttt{\ Pro\ } , \texttt{\ Duotone\ } and \texttt{\ Sharp\ } sets.
(See
\href{https://github.com/typst/packages/raw/main/packages/preview/fontawesome/0.5.0/\#enable-pro-sets}{Enable
Pro sets} for enabling Pro sets.)

But only \texttt{\ Free\ } and \texttt{\ Brands\ } are tested by me.
That is, three font files are used to test:

\begin{itemize}
\tightlist
\item
  Font Awesome 6 Free (Also named as \emph{Font Awesome 6 Free Regular}
  )
\item
  Font Awesome 6 Free Solid
\item
  Font Awesome 6 Brands
\end{itemize}

Due to some limitations of typst 0.12.0, the regular and solid versions
are treated as different fonts. In this library, \texttt{\ solid\ } is
used to switch between the regular and solid versions.

To use other sets or specify one set, you can pass the \texttt{\ font\ }
parameter to the inner \texttt{\ text\ } function:\\
\texttt{\ fa-icon("github",\ font:\ "Font\ Awesome\ 6\ Pro\ Solid")\ }

If you have Font Awesome Pro, please help me test the library with the
Pro set. Any feedback is appreciated.

\subparagraph{Enable Pro sets}\label{enable-pro-sets}

Typst 0.12.0 raise a warning when the font is not found. To use the Pro
set, \texttt{\ \#fa-use-pro()\ } should be called before any
\texttt{\ fa-*\ } functions.

\begin{Shaded}
\begin{Highlighting}[]
\NormalTok{\#fa{-}use{-}pro()                 // Enable Pro sets}

\NormalTok{\#fa{-}icon("chess{-}queen{-}piece") // Use icons from Pro sets}
\end{Highlighting}
\end{Shaded}

\paragraph{Customization}\label{customization}

The \texttt{\ fa-icon\ } function passes args to \texttt{\ text\ } , so
you can customize the icon by passing parameters to it:

\texttt{\ \#fa-icon("chess-queen",\ fill:\ blue)\ }

\paragraph{Stacking icons}\label{stacking-icons}

The \texttt{\ fa-stack\ } function can be used to create stacked icons:

\texttt{\ \#fa-stack(fa-icon-args:\ (solid:\ true),\ "square",\ ("chess-queen",\ (fill:\ white,\ size:\ 5.5pt)))\ }

Declaration is
\texttt{\ fa-stack(box-args:\ (:),\ grid-args:\ (:),\ fa-icon-args:\ (:),\ ..icons)\ }

\begin{itemize}
\tightlist
\item
  The order of the icons is from the bottom to the top.
\item
  \texttt{\ fa-icon-args\ } is used to set the default args for all
  icons.
\item
  You can also control the internal \texttt{\ box\ } and
  \texttt{\ grid\ } by passing the \texttt{\ box-args\ } and
  \texttt{\ grid-args\ } to the \texttt{\ fa-stack\ } function.
\item
  Currently, four types of icons are supported. The first three types
  leverage the \texttt{\ fa-icon\ } function, and the last type is just
  a content you want to put in the stack.

  \begin{itemize}
  \tightlist
  \item
    \texttt{\ str\ } , e.g., \texttt{\ "square"\ }
  \item
    \texttt{\ array\ } , e.g.,
    \texttt{\ ("chess-queen",\ (fill:\ white,\ size:\ 5.5pt))\ }
  \item
    \texttt{\ arguments\ } , e.g.
    \texttt{\ arguments("chess-queen",\ solid:\ true,\ fill:\ white)\ }
  \item
    \texttt{\ content\ } , e.g.
    \texttt{\ fa-chess-queen(solid:\ true,\ fill:\ white)\ }
  \end{itemize}
\end{itemize}

\paragraph{Known Issues}\label{known-issues}

\begin{itemize}
\item
  \href{https://github.com/typst/typst/issues/2578}{typst\#2578}
  \href{https://github.com/duskmoon314/typst-fontawesome/issues/2}{typst-fontawesome\#2}

  This is a known issue that the ligatures may not work in headings,
  list items, grid items, and other elements. You can use the Unicode
  from the \href{https://fontawesome.com/}{official website} to avoid
  this issue when using Pro sets.

  For most icons, Unicode is used implicitly. So I assume we usually
  don’t need to worry about this.

  Any help on this issue is appreciated.
\end{itemize}

\subsection{Example}\label{example}

See the
\href{https://typst.app/project/rQwGUWt5p33vrsb_uNPR9F}{\texttt{\ example.typ\ }}
file for a complete example.

\subsection{Contribution}\label{contribution}

Feel free to open an issue or a pull request if you find any problems or
have any suggestions.

\subsubsection{Python helper}\label{python-helper}

The \texttt{\ helper.py\ } script is used to get metadata via the
GraphQL API and generate typst code. I aim only to use standard python
libraries, so running it on any platform with python installed should be
easy.

\subsubsection{Repo structure}\label{repo-structure}

\begin{itemize}
\tightlist
\item
  \texttt{\ helper.py\ } : The helper script to get metadata and
  generate typst code.
\item
  \texttt{\ lib.typ\ } : The main entrypoint of the library.
\item
  \texttt{\ lib-impl.typ\ } : The implementation of \texttt{\ fa-icon\ }
  .
\item
  \texttt{\ lib-gen.typ\ } : The generated functions of icons.
\item
  \texttt{\ example.typ\ } : An example file to show how to use the
  library.
\item
  \texttt{\ gallery.typ\ } : The generated gallery of icons. It is used
  in the example file.
\end{itemize}

\subsection{License}\label{license}

This library is licensed under the MIT license. Feel free to use it in
your project.

\subsubsection{How to add}\label{how-to-add}

Copy this into your project and use the import as
\texttt{\ fontawesome\ }

\begin{verbatim}
#import "@preview/fontawesome:0.5.0"
\end{verbatim}

\includesvg[width=0.16667in,height=0.16667in]{/assets/icons/16-copy.svg}

Check the docs for
\href{https://typst.app/docs/reference/scripting/\#packages}{more
information on how to import packages} .

\subsubsection{About}\label{about}

\begin{description}
\tightlist
\item[Author :]
\href{mailto:kp.campbell.he@duskmoon314.com}{duskmoon (Campbell He)}
\item[License:]
MIT
\item[Current version:]
0.5.0
\item[Last updated:]
October 21, 2024
\item[First released:]
July 3, 2023
\item[Archive size:]
74.7 kB
\href{https://packages.typst.org/preview/fontawesome-0.5.0.tar.gz}{\pandocbounded{\includesvg[keepaspectratio]{/assets/icons/16-download.svg}}}
\item[Repository:]
\href{https://github.com/duskmoon314/typst-fontawesome}{GitHub}
\end{description}

\subsubsection{Where to report issues?}\label{where-to-report-issues}

This package is a project of duskmoon (Campbell He) . Report issues on
\href{https://github.com/duskmoon314/typst-fontawesome}{their
repository} . You can also try to ask for help with this package on the
\href{https://forum.typst.app}{Forum} .

Please report this package to the Typst team using the
\href{https://typst.app/contact}{contact form} if you believe it is a
safety hazard or infringes upon your rights.

\phantomsection\label{versions}
\subsubsection{Version history}\label{version-history}

\begin{longtable}[]{@{}ll@{}}
\toprule\noalign{}
Version & Release Date \\
\midrule\noalign{}
\endhead
\bottomrule\noalign{}
\endlastfoot
0.5.0 & October 21, 2024 \\
\href{https://typst.app/universe/package/fontawesome/0.4.0/}{0.4.0} &
August 1, 2024 \\
\href{https://typst.app/universe/package/fontawesome/0.3.0/}{0.3.0} &
July 23, 2024 \\
\href{https://typst.app/universe/package/fontawesome/0.2.1/}{0.2.1} &
June 17, 2024 \\
\href{https://typst.app/universe/package/fontawesome/0.2.0/}{0.2.0} &
April 19, 2024 \\
\href{https://typst.app/universe/package/fontawesome/0.1.1/}{0.1.1} &
April 1, 2024 \\
\href{https://typst.app/universe/package/fontawesome/0.1.0/}{0.1.0} &
July 3, 2023 \\
\end{longtable}

Typst GmbH did not create this package and cannot guarantee correct
functionality of this package or compatibility with any version of the
Typst compiler or app.


\section{Package List LaTeX/wordometer.tex}
\title{typst.app/universe/package/wordometer}

\phantomsection\label{banner}
\section{wordometer}\label{wordometer}

{ 0.1.4 }

Word counts and document statistics.

{ } Featured Package

\phantomsection\label{readme}
\href{https://github.com/typst/packages/raw/main/packages/preview/wordometer/0.1.4/docs/manual.pdf}{\pandocbounded{\includegraphics[keepaspectratio]{https://img.shields.io/badge/docs-manual.pdf-green}}}
\pandocbounded{\includegraphics[keepaspectratio]{https://img.shields.io/badge/dynamic/toml?url=https\%3A\%2F\%2Fgithub.com\%2FJollywatt\%2Ftypst-wordometer\%2Fraw\%2Fmaster\%2Ftypst.toml&query=package.version&label=latest\%20version}}
\href{https://github.com/Jollywatt/typst-wordometer}{\pandocbounded{\includegraphics[keepaspectratio]{https://img.shields.io/badge/GitHub-repo-blue}}}

A small
\href{https://github.com/typst/packages/raw/main/packages/preview/wordometer/0.1.4/\%22https://typst.app/\%22}{Typst}
package for quick and easy in-document word counts.

\subsection{Basic usage}\label{basic-usage}

\begin{Shaded}
\begin{Highlighting}[]
\NormalTok{\#import "@preview/wordometer:0.1.3": word{-}count, total{-}words}

\NormalTok{\#show: word{-}count}

\NormalTok{In this document, there are \#total{-}words words all up.}

\NormalTok{\#word{-}count(total =\textgreater{} [}
\NormalTok{  The number of words in this block is \#total.words}
\NormalTok{  and there are \#total.characters letters.}
\NormalTok{])}
\end{Highlighting}
\end{Shaded}

\subsection{Excluding elements}\label{excluding-elements}

You can exclude elements by name (e.g., \texttt{\ "caption"\ } ),
function (e.g., \texttt{\ figure.caption\ } ), where-selector (e.g.,
\texttt{\ raw.where(block:\ true)\ } ), or label (e.g.,
\texttt{\ \textless{}no-wc\textgreater{}\ } ).

\begin{Shaded}
\begin{Highlighting}[]
\NormalTok{\#show: word{-}count.with(exclude: (heading.where(level: 1), strike))}

\NormalTok{= This Heading Doesn\textquotesingle{}t Count}
\NormalTok{== But I do!}

\NormalTok{In this document \#strike[(excluding me)], there are \#total{-}words words all up.}

\NormalTok{\#word{-}count(total =\textgreater{} [}
\NormalTok{  You can exclude elements by label, too.}
\NormalTok{  \#[That was \#total.words, excluding this sentence!] \textless{}no{-}wc\textgreater{}}
\NormalTok{], exclude: \textless{}no{-}wc\textgreater{})}
\end{Highlighting}
\end{Shaded}

\subsection{Changelog}\label{changelog}

\subsubsection{v0.1.4}\label{v0.1.4}

\begin{itemize}
\tightlist
\item
  Fix deprecated use of \texttt{\ locate()\ } for Typst
  \texttt{\ \textgreater{}=0.12.0\ }
\end{itemize}

\subsubsection{v0.1.3}\label{v0.1.3}

(No changes 🤡)

\subsubsection{v0.1.2}\label{v0.1.2}

\begin{itemize}
\tightlist
\item
  Fix bugs when using labels and where-selectors to exclude elements
\end{itemize}

\subsubsection{v0.1.1}\label{v0.1.1}

\begin{itemize}
\tightlist
\item
  Allow excluding elements by passing their element functions
\item
  Add basic \texttt{\ element.where(..)\ } selectors
\item
  Fix crash for figures without captions
\end{itemize}

\subsubsection{v0.1.0}\label{v0.1.0}

\begin{itemize}
\tightlist
\item
  Initial version
\end{itemize}

\subsubsection{How to add}\label{how-to-add}

Copy this into your project and use the import as
\texttt{\ wordometer\ }

\begin{verbatim}
#import "@preview/wordometer:0.1.4"
\end{verbatim}

\includesvg[width=0.16667in,height=0.16667in]{/assets/icons/16-copy.svg}

Check the docs for
\href{https://typst.app/docs/reference/scripting/\#packages}{more
information on how to import packages} .

\subsubsection{About}\label{about}

\begin{description}
\tightlist
\item[Author :]
Joseph Wilson (Jollywatt)
\item[License:]
MIT
\item[Current version:]
0.1.4
\item[Last updated:]
October 29, 2024
\item[First released:]
January 29, 2024
\item[Archive size:]
7.49 kB
\href{https://packages.typst.org/preview/wordometer-0.1.4.tar.gz}{\pandocbounded{\includesvg[keepaspectratio]{/assets/icons/16-download.svg}}}
\item[Repository:]
\href{https://github.com/Jollywatt/typst-wordometer}{GitHub}
\end{description}

\subsubsection{Where to report issues?}\label{where-to-report-issues}

This package is a project of Joseph Wilson (Jollywatt) . Report issues
on \href{https://github.com/Jollywatt/typst-wordometer}{their
repository} . You can also try to ask for help with this package on the
\href{https://forum.typst.app}{Forum} .

Please report this package to the Typst team using the
\href{https://typst.app/contact}{contact form} if you believe it is a
safety hazard or infringes upon your rights.

\phantomsection\label{versions}
\subsubsection{Version history}\label{version-history}

\begin{longtable}[]{@{}ll@{}}
\toprule\noalign{}
Version & Release Date \\
\midrule\noalign{}
\endhead
\bottomrule\noalign{}
\endlastfoot
0.1.4 & October 29, 2024 \\
\href{https://typst.app/universe/package/wordometer/0.1.3/}{0.1.3} &
October 8, 2024 \\
\href{https://typst.app/universe/package/wordometer/0.1.2/}{0.1.2} &
April 29, 2024 \\
\href{https://typst.app/universe/package/wordometer/0.1.1/}{0.1.1} &
March 3, 2024 \\
\href{https://typst.app/universe/package/wordometer/0.1.0/}{0.1.0} &
January 29, 2024 \\
\end{longtable}

Typst GmbH did not create this package and cannot guarantee correct
functionality of this package or compatibility with any version of the
Typst compiler or app.


\section{Package List LaTeX/touying-brandred-uobristol.tex}
\title{typst.app/universe/package/touying-brandred-uobristol}

\phantomsection\label{banner}
\phantomsection\label{template-thumbnail}
\pandocbounded{\includegraphics[keepaspectratio]{https://packages.typst.org/preview/thumbnails/touying-brandred-uobristol-0.1.2-small.webp}}

\section{touying-brandred-uobristol}\label{touying-brandred-uobristol}

{ 0.1.2 }

Touying Slide Theme for University of Bristol

\href{/app?template=touying-brandred-uobristol&version=0.1.2}{Create
project in app}

\phantomsection\label{readme}
Inspired by the brand guidelines of University of Bristol and modified
from the Metropolis theme.

\subsection{Use as Typst Template
Package}\label{use-as-typst-template-package}

Use the following command to create a new project with this theme.

\begin{Shaded}
\begin{Highlighting}[]
\ExtensionTok{typst}\NormalTok{ init @preview/touying{-}uobristol}
\end{Highlighting}
\end{Shaded}

\subsection{Examples}\label{examples}

See the
\href{https://github.com/typst/packages/raw/main/packages/preview/touying-brandred-uobristol/0.1.2/examples/example.typ}{examples}
and
\href{https://github.com/typst/packages/raw/main/packages/preview/touying-brandred-uobristol/0.1.2/examples/example.pdf}{output}
for more details.

Licensed under the
\href{https://github.com/typst/packages/raw/main/packages/preview/touying-brandred-uobristol/0.1.2/LICENSE}{MIT
License} .

\href{/app?template=touying-brandred-uobristol&version=0.1.2}{Create
project in app}

\subsubsection{How to use}\label{how-to-use}

Click the button above to create a new project using this template in
the Typst app.

You can also use the Typst CLI to start a new project on your computer
using this command:

\begin{verbatim}
typst init @preview/touying-brandred-uobristol:0.1.2
\end{verbatim}

\includesvg[width=0.16667in,height=0.16667in]{/assets/icons/16-copy.svg}

\subsubsection{About}\label{about}

\begin{description}
\tightlist
\item[Author :]
\href{mailto:huyg0180110559@outlook.com}{HPDell}
\item[License:]
MIT
\item[Current version:]
0.1.2
\item[Last updated:]
November 18, 2024
\item[First released:]
September 18, 2024
\item[Archive size:]
12.0 kB
\href{https://packages.typst.org/preview/touying-brandred-uobristol-0.1.2.tar.gz}{\pandocbounded{\includesvg[keepaspectratio]{/assets/icons/16-download.svg}}}
\item[Repository:]
\href{https://github.com/HPDell/touying-brandred-uobristol}{GitHub}
\item[Categor y :]
\begin{itemize}
\tightlist
\item[]
\item
  \pandocbounded{\includesvg[keepaspectratio]{/assets/icons/16-presentation.svg}}
  \href{https://typst.app/universe/search/?category=presentation}{Presentation}
\end{itemize}
\end{description}

\subsubsection{Where to report issues?}\label{where-to-report-issues}

This template is a project of HPDell . Report issues on
\href{https://github.com/HPDell/touying-brandred-uobristol}{their
repository} . You can also try to ask for help with this template on the
\href{https://forum.typst.app}{Forum} .

Please report this template to the Typst team using the
\href{https://typst.app/contact}{contact form} if you believe it is a
safety hazard or infringes upon your rights.

\phantomsection\label{versions}
\subsubsection{Version history}\label{version-history}

\begin{longtable}[]{@{}ll@{}}
\toprule\noalign{}
Version & Release Date \\
\midrule\noalign{}
\endhead
\bottomrule\noalign{}
\endlastfoot
0.1.2 & November 18, 2024 \\
\href{https://typst.app/universe/package/touying-brandred-uobristol/0.1.1/}{0.1.1}
& October 9, 2024 \\
\href{https://typst.app/universe/package/touying-brandred-uobristol/0.1.0/}{0.1.0}
& September 18, 2024 \\
\end{longtable}

Typst GmbH did not create this template and cannot guarantee correct
functionality of this template or compatibility with any version of the
Typst compiler or app.


\section{Package List LaTeX/fireside.tex}
\title{typst.app/universe/package/fireside}

\phantomsection\label{banner}
\phantomsection\label{template-thumbnail}
\pandocbounded{\includegraphics[keepaspectratio]{https://packages.typst.org/preview/thumbnails/fireside-1.0.0-small.webp}}

\section{fireside}\label{fireside}

{ 1.0.0 }

A simple letter template with a touch of warmth

{ } Featured Template

\href{/app?template=fireside&version=1.0.0}{Create project in app}

\phantomsection\label{readme}
A Typst theme for a nice and simple looking letter that’s not
completely black and white. Inspired by a Canva theme.

Features:

\begin{itemize}
\tightlist
\item
  A neutral-warm beige background that feels cosier and softer to the
  eyes than pure white, while still looking kinda white-ish
\item
  Short content is vertically padded to look a bit more centered
\item
  Long content overflows gracefully on as many pages as necessary
\end{itemize}

\begin{longtable}[]{@{}lll@{}}
\toprule\noalign{}
Basic example & Short text (vertically centered) & Multi-page
overflowing text \\
\midrule\noalign{}
\endhead
\bottomrule\noalign{}
\endlastfoot
\href{https://github.com/typst/packages/raw/main/packages/preview/fireside/1.0.0/.rendered/demo_medium.pdf}{\texttt{\ .rendered/demo\_medium.pdf\ }}
&
\href{https://github.com/typst/packages/raw/main/packages/preview/fireside/1.0.0/.rendered/demo_short.pdf}{\texttt{\ .rendered/demo\_short.pdf\ }}
&
\href{https://github.com/typst/packages/raw/main/packages/preview/fireside/1.0.0/.rendered/demo_long.pdf}{\texttt{\ .rendered/demo\_long.pdf\ }} \\
\end{longtable}

\begin{itemize}
\item
  If using Typst locally, install the
  \href{https://fonts.google.com/specimen/Hanken+Grotesk}{HK Grotesk}
  font

  \begin{itemize}
  \tightlist
  \item
    \emph{Note: it is already installed on the \url{https://typst.app/}
    IDE}
  \end{itemize}
\item
  Insert the setup \texttt{\ show\ } statement

\begin{Shaded}
\begin{Highlighting}[]
\NormalTok{\#import "@preview/fireside:1.0.0": *}

\NormalTok{\#show: project.with(}
\NormalTok{  title: [Anakin \textbackslash{} Skywalker],}
\NormalTok{  from{-}details: [}
\NormalTok{    Appt. x, \textbackslash{}}
\NormalTok{    Mos Espa, \textbackslash{}}
\NormalTok{    Tatooine \textbackslash{}}
\NormalTok{    anakin\textbackslash{}@example.com \textbackslash{} +999 xxxx xxx}
\NormalTok{  ],}
\NormalTok{  to{-}details: [}
\NormalTok{    Sheev Palpatine \textbackslash{}}
\NormalTok{    500 Republica, \textbackslash{}}
\NormalTok{    Ambassadorial Sector, Senate District, \textbackslash{}}
\NormalTok{    Galactic City, \textbackslash{} Coruscant}
\NormalTok{  ],}
\NormalTok{)}

\NormalTok{Dear Emperor, ...}
\end{Highlighting}
\end{Shaded}
\item
  If your text overflows on multiple pages, you might want to add
  \href{https://typst.app/docs/reference/layout/page/\#parameters-numbering}{page
  numbering} , as shown in
  \href{https://github.com/typst/packages/raw/main/packages/preview/fireside/1.0.0/.demo/demo_long.typ}{\texttt{\ .demo/demo\_long.typ\ }}
  (line 3)
\end{itemize}

\begin{Shaded}
\begin{Highlighting}[]
\NormalTok{  background: rgb("f4f1eb"), \# Override the background color}
\NormalTok{  title: "",                 \# Set the top{-}left title. It looks best on two lines}
\NormalTok{  from{-}details: none,        \# Letter sender (you) details}
\NormalTok{  to{-}details: none,          \# Letter receiver details}
\NormalTok{  margin: 2.1cm,             \# Page margin}
\NormalTok{  vertical{-}center{-}level: 2,  \# When the content is small, it is vertically centered a bit, but still kept closer to the top. This controls how much. Setting to none will disable centering.}
\NormalTok{  body}
\end{Highlighting}
\end{Shaded}

\begin{itemize}
\tightlist
\item
  \texttt{\ lib.typ\ } is licensed as MIT (
  \url{https://opensource.org/license/mit} )
\item
  The demo/template files are licensed as CC0 (
  \url{https://creativecommons.org/publicdomain/zero/1.0/legalcode} )
\item
  Any document fully or partially generated using this template may be
  licensed however you wish
\end{itemize}

\href{/app?template=fireside&version=1.0.0}{Create project in app}

\subsubsection{How to use}\label{how-to-use}

Click the button above to create a new project using this template in
the Typst app.

You can also use the Typst CLI to start a new project on your computer
using this command:

\begin{verbatim}
typst init @preview/fireside:1.0.0
\end{verbatim}

\includesvg[width=0.16667in,height=0.16667in]{/assets/icons/16-copy.svg}

\subsubsection{About}\label{about}

\begin{description}
\tightlist
\item[Author :]
\href{https://edgar.bzh/}{Edgar Onghena}
\item[License:]
MIT
\item[Current version:]
1.0.0
\item[Last updated:]
May 8, 2024
\item[First released:]
May 8, 2024
\item[Minimum Typst version:]
0.11.0
\item[Archive size:]
2.96 kB
\href{https://packages.typst.org/preview/fireside-1.0.0.tar.gz}{\pandocbounded{\includesvg[keepaspectratio]{/assets/icons/16-download.svg}}}
\item[Categor y :]
\begin{itemize}
\tightlist
\item[]
\item
  \pandocbounded{\includesvg[keepaspectratio]{/assets/icons/16-envelope.svg}}
  \href{https://typst.app/universe/search/?category=office}{Office}
\end{itemize}
\end{description}

\subsubsection{Where to report issues?}\label{where-to-report-issues}

This template is a project of Edgar Onghena . You can also try to ask
for help with this template on the \href{https://forum.typst.app}{Forum}
.

Please report this template to the Typst team using the
\href{https://typst.app/contact}{contact form} if you believe it is a
safety hazard or infringes upon your rights.

\phantomsection\label{versions}
\subsubsection{Version history}\label{version-history}

\begin{longtable}[]{@{}ll@{}}
\toprule\noalign{}
Version & Release Date \\
\midrule\noalign{}
\endhead
\bottomrule\noalign{}
\endlastfoot
1.0.0 & May 8, 2024 \\
\end{longtable}

Typst GmbH did not create this template and cannot guarantee correct
functionality of this template or compatibility with any version of the
Typst compiler or app.


\section{Package List LaTeX/salsa-dip.tex}
\title{typst.app/universe/package/salsa-dip}

\phantomsection\label{banner}
\section{salsa-dip}\label{salsa-dip}

{ 0.1.0 }

DIP chip labels for Typst

\phantomsection\label{readme}
Salsa Dip is a library for making
\href{https://en.wikipedia.org/wiki/Dual_in-line_package}{DIP} chip
labels in Typst.

\begin{Shaded}
\begin{Highlighting}[]
\NormalTok{\#import "@preview/salsa{-}dip:0.1.0": dip{-}chip{-}label}

\NormalTok{\#set text(font: ("JetBrains Mono", "Fira Code", "DejaVu Sans Mono"), weight: "extrabold")}
\NormalTok{\#set page(width: auto, height: auto, margin: .125cm)}

\NormalTok{\#let z80{-}labels = ("A11", ..., "A9", "A10")}

\NormalTok{\#dip{-}chip{-}label(40, 0.54in, z80{-}labels, "Z80", settings: (pin{-}number{-}margin: 1pt, pin{-}number{-}size: 2.5pt, chip{-}label{-}size: 5pt))}
\end{Highlighting}
\end{Shaded}

\pandocbounded{\includegraphics[keepaspectratio]{https://github.com/typst/packages/raw/main/packages/preview/salsa-dip/0.1.0/examples/z80.png}}

\begin{Shaded}
\begin{Highlighting}[]
\NormalTok{\#import "@preview/salsa{-}dip:0.1.0": dip{-}chip{-}label}

\NormalTok{\#set text(font: ("JetBrains Mono", "Fira Code", "DejaVu Sans Mono"), weight: "extrabold")}
\NormalTok{\#set page(width: auto, height: auto, margin: .125cm)}

\NormalTok{\#let labels = ("1A", "1B", "1Y", "2A", "2B", "2Y", "GND", "3Y", "3A", "3B", "4Y", "4A", "4B", "VCC")}
\NormalTok{\#dip{-}chip{-}label(14, 0.24in, labels, "74LS00")}
\end{Highlighting}
\end{Shaded}

\pandocbounded{\includegraphics[keepaspectratio]{https://github.com/typst/packages/raw/main/packages/preview/salsa-dip/0.1.0/examples/74ls00.png}}

The \texttt{\ dip-chip-label\ } function is called with four parameters,
an integer number of pins for the chip, the width (usually
\texttt{\ 0.24in\ } or \texttt{\ 0.54in\ } ), the list of pin labels (if
no labels are desired, an empty array can be passed), and the chip
label.

There is an additional \texttt{\ settings\ } parameter which can be used
to fine tune the apperance of the chip labels. The \texttt{\ settings\ }
parameter is a dictionary optionally containing any of the setting keys:

\begin{itemize}
\tightlist
\item
  \texttt{\ chip-label-size\ } : Font size for the chip label
\item
  \texttt{\ pin-number-margin\ } : Margin to give next to pin numbers
\item
  \texttt{\ pin-number-size\ } : Font size for pin numbers
\item
  \texttt{\ pin-label-size\ } : Font size for pin labels
\item
  \texttt{\ include-numbers\ } : Boolean enabling pin numbers
\item
  \texttt{\ pin-spacing\ } : Spacing of pins
\item
  \texttt{\ vertical-margin\ } : Total margin to put into spacing above
  and below pin labels
\end{itemize}

\subsubsection{How to add}\label{how-to-add}

Copy this into your project and use the import as \texttt{\ salsa-dip\ }

\begin{verbatim}
#import "@preview/salsa-dip:0.1.0"
\end{verbatim}

\includesvg[width=0.16667in,height=0.16667in]{/assets/icons/16-copy.svg}

Check the docs for
\href{https://typst.app/docs/reference/scripting/\#packages}{more
information on how to import packages} .

\subsubsection{About}\label{about}

\begin{description}
\tightlist
\item[Author :]
\href{https://gitlab.com/users/ashplasek}{Ashlen Plasek}
\item[License:]
MIT
\item[Current version:]
0.1.0
\item[Last updated:]
June 17, 2024
\item[First released:]
June 17, 2024
\item[Archive size:]
2.49 kB
\href{https://packages.typst.org/preview/salsa-dip-0.1.0.tar.gz}{\pandocbounded{\includesvg[keepaspectratio]{/assets/icons/16-download.svg}}}
\item[Repository:]
\href{https://gitlab.com/ashplasek/salsa-dip}{GitLab}
\end{description}

\subsubsection{Where to report issues?}\label{where-to-report-issues}

This package is a project of Ashlen Plasek . Report issues on
\href{https://gitlab.com/ashplasek/salsa-dip}{their repository} . You
can also try to ask for help with this package on the
\href{https://forum.typst.app}{Forum} .

Please report this package to the Typst team using the
\href{https://typst.app/contact}{contact form} if you believe it is a
safety hazard or infringes upon your rights.

\phantomsection\label{versions}
\subsubsection{Version history}\label{version-history}

\begin{longtable}[]{@{}ll@{}}
\toprule\noalign{}
Version & Release Date \\
\midrule\noalign{}
\endhead
\bottomrule\noalign{}
\endlastfoot
0.1.0 & June 17, 2024 \\
\end{longtable}

Typst GmbH did not create this package and cannot guarantee correct
functionality of this package or compatibility with any version of the
Typst compiler or app.


\section{Package List LaTeX/quick-sip.tex}
\title{typst.app/universe/package/quick-sip}

\phantomsection\label{banner}
\phantomsection\label{template-thumbnail}
\pandocbounded{\includegraphics[keepaspectratio]{https://packages.typst.org/preview/thumbnails/quick-sip-0.1.0-small.webp}}

\section{quick-sip}\label{quick-sip}

{ 0.1.0 }

A template for creating quick reference handbook style checklists.

\href{/app?template=quick-sip&version=0.1.0}{Create project in app}

\phantomsection\label{readme}
Creates aviation style checklists like Quick Reference Handbooks.

\includegraphics[width=3.125in,height=\textheight,keepaspectratio]{https://github.com/typst/packages/raw/main/packages/preview/quick-sip/0.1.0/thumbnail.png}

\subsubsection{Features:}\label{features}

\begin{itemize}
\tightlist
\item
  Index
\item
  Section
\item
  Conditions
\item
  Objective
\item
  Step (When/If)
\item
  Sub Step
\item
  Caution
\item
  Note
\item
  Choose One
\item
  Go to step
\item
  End section now
\end{itemize}

\subsection{Start with}\label{start-with}

\begin{Shaded}
\begin{Highlighting}[]
\NormalTok{\#import "@preview/quick{-}sip:0.1.0": *}
\NormalTok{\#show: QRH.with(title: "Cup of Tea")}
\end{Highlighting}
\end{Shaded}

Then add a section:

\begin{Shaded}
\begin{Highlighting}[]
\NormalTok{\#section("Cup of Tea preparation")[}
\NormalTok{  \#step("KETTLE", "Filled to 1 CUP")}
\NormalTok{  \#step([*When* KETTLE boiled:], "")}
\NormalTok{  \#step([*If* sugar required], "")}
\NormalTok{    //.. Rest of section goes here}
\NormalTok{]}
\end{Highlighting}
\end{Shaded}

\paragraph{Index}\label{index}

An index with an entry for each section in the document.

\begin{Shaded}
\begin{Highlighting}[]
\NormalTok{\#index()}
\end{Highlighting}
\end{Shaded}

\paragraph{Section}\label{section}

A section title, forces capitalisation.

\begin{Shaded}
\begin{Highlighting}[]
\NormalTok{\#section("Cup of Tea preparation")[}
\NormalTok{    //.. Rest of section goes here}
\NormalTok{]}
\end{Highlighting}
\end{Shaded}

\paragraph{Conditions}\label{conditions}

Conditionals for this section.

\begin{Shaded}
\begin{Highlighting}[]
\NormalTok{\#condition[}
\NormalTok{    {-} Dehydration}
\NormalTok{    {-} Fatigue}
\NormalTok{    {-} Inability to Concentrate}
\NormalTok{]}
\end{Highlighting}
\end{Shaded}

\paragraph{Objective}\label{objective}

An objective for this section (optional).

\begin{Shaded}
\begin{Highlighting}[]
\NormalTok{\#objective[To replenish fluids.]}
\end{Highlighting}
\end{Shaded}

\paragraph{Step}\label{step}

A numbered step in the checklist. The first parameter is to the left of
the dotted line, the second is to the right. If the second parameter is
\texttt{\ ""\ } then there is no dotted line.

\begin{Shaded}
\begin{Highlighting}[]
\NormalTok{\#step("KETTLE", "Filled to 1 CUP")}
\NormalTok{\#step([*When* KETTLE boiled:], "")}
\NormalTok{\#step([*If* sugar required], "")}
\end{Highlighting}
\end{Shaded}

\paragraph{Tab}\label{tab}

Indents contents by one tab.

\begin{Shaded}
\begin{Highlighting}[]
\NormalTok{\#tab(goto("9"))}
\NormalTok{\#tab(tab("Large mugs may require more water."))}
\end{Highlighting}
\end{Shaded}

\paragraph{Caution}\label{caution}

Adds a caution element.

\begin{Shaded}
\begin{Highlighting}[]
\NormalTok{\#caution([HOT WATER \#linebreak()Adult supervision required.])}
\end{Highlighting}
\end{Shaded}

\paragraph{Note}\label{note}

Adds a note.

\begin{Shaded}
\begin{Highlighting}[]
\NormalTok{\#note("Stir after each step")}
\end{Highlighting}
\end{Shaded}

\paragraph{Choose One}\label{choose-one}

A numbered step with options.

\begin{Shaded}
\begin{Highlighting}[]
\NormalTok{ \#choose{-}one[}
\NormalTok{    \#option[Black tea *required:*]}
\NormalTok{    \#option[Tea with MILK *required:*]}
\NormalTok{  ]}
\end{Highlighting}
\end{Shaded}

\paragraph{Go to step}\label{go-to-step}

Two right facing arrow heads followed by Go to step
\texttt{\ step\ number\ } . Links to step in pdf.

\begin{Shaded}
\begin{Highlighting}[]
\NormalTok{\#goto("9")}
\end{Highlighting}
\end{Shaded}

\paragraph{End}\label{end}

Ends the section here with 4 dots.

\begin{Shaded}
\begin{Highlighting}[]
\NormalTok{\#end()}
\end{Highlighting}
\end{Shaded}

\paragraph{Wait}\label{wait}

Long small dotted line for waiting for a task to complete.

\begin{Shaded}
\begin{Highlighting}[]
\NormalTok{\#wait()}
\end{Highlighting}
\end{Shaded}

\href{/app?template=quick-sip&version=0.1.0}{Create project in app}

\subsubsection{How to use}\label{how-to-use}

Click the button above to create a new project using this template in
the Typst app.

You can also use the Typst CLI to start a new project on your computer
using this command:

\begin{verbatim}
typst init @preview/quick-sip:0.1.0
\end{verbatim}

\includesvg[width=0.16667in,height=0.16667in]{/assets/icons/16-copy.svg}

\subsubsection{About}\label{about}

\begin{description}
\tightlist
\item[Author :]
\href{https://github.com/artomweb}{Archie Webster}
\item[License:]
MIT
\item[Current version:]
0.1.0
\item[Last updated:]
October 16, 2024
\item[First released:]
October 16, 2024
\item[Archive size:]
4.28 kB
\href{https://packages.typst.org/preview/quick-sip-0.1.0.tar.gz}{\pandocbounded{\includesvg[keepaspectratio]{/assets/icons/16-download.svg}}}
\item[Repository:]
\href{https://github.com/artomweb/Quick-Sip-Typst-Template}{GitHub}
\item[Categor y :]
\begin{itemize}
\tightlist
\item[]
\item
  \pandocbounded{\includesvg[keepaspectratio]{/assets/icons/16-hammer.svg}}
  \href{https://typst.app/universe/search/?category=utility}{Utility}
\end{itemize}
\end{description}

\subsubsection{Where to report issues?}\label{where-to-report-issues}

This template is a project of Archie Webster . Report issues on
\href{https://github.com/artomweb/Quick-Sip-Typst-Template}{their
repository} . You can also try to ask for help with this template on the
\href{https://forum.typst.app}{Forum} .

Please report this template to the Typst team using the
\href{https://typst.app/contact}{contact form} if you believe it is a
safety hazard or infringes upon your rights.

\phantomsection\label{versions}
\subsubsection{Version history}\label{version-history}

\begin{longtable}[]{@{}ll@{}}
\toprule\noalign{}
Version & Release Date \\
\midrule\noalign{}
\endhead
\bottomrule\noalign{}
\endlastfoot
0.1.0 & October 16, 2024 \\
\end{longtable}

Typst GmbH did not create this template and cannot guarantee correct
functionality of this template or compatibility with any version of the
Typst compiler or app.


\section{Package List LaTeX/frackable.tex}
\title{typst.app/universe/package/frackable}

\phantomsection\label{banner}
\section{frackable}\label{frackable}

{ 0.2.0 }

Vulgar Fractions

\phantomsection\label{readme}
Version 0.2.0

Provides a function,
\texttt{\ frackable(numerator,\ denominator,\ whole:\ none)\ } , to
typeset vulgar and mixed fractions. Provides a second
\texttt{\ generator(...)\ } function that returns another having the
same signature as \texttt{\ frackable\ } to typeset arbitrary vulgar and
mixed fractions in fonts that do not support the \texttt{\ frac\ }
feature.

\begin{Shaded}
\begin{Highlighting}[]
\NormalTok{\#import "@preview/frackable:0.2.0": *}

\NormalTok{\#frackable(1, 2)}
\NormalTok{\#frackable(1, 3)}
\NormalTok{\#frackable(3, 4, whole: 9)}
\NormalTok{\#frackable(9, 16)}
\NormalTok{\#frackable(31, 32)}
\NormalTok{\#frackable(0, "000")}
\end{Highlighting}
\end{Shaded}

\pandocbounded{\includegraphics[keepaspectratio]{https://github.com/typst/packages/raw/main/packages/preview/frackable/0.2.0/example.png}}

\subsubsection{How to add}\label{how-to-add}

Copy this into your project and use the import as \texttt{\ frackable\ }

\begin{verbatim}
#import "@preview/frackable:0.2.0"
\end{verbatim}

\includesvg[width=0.16667in,height=0.16667in]{/assets/icons/16-copy.svg}

Check the docs for
\href{https://typst.app/docs/reference/scripting/\#packages}{more
information on how to import packages} .

\subsubsection{About}\label{about}

\begin{description}
\tightlist
\item[Author :]
James R. Swift
\item[License:]
Unlicense
\item[Current version:]
0.2.0
\item[Last updated:]
September 27, 2024
\item[First released:]
September 24, 2024
\item[Minimum Typst version:]
0.11.0
\item[Archive size:]
2.83 kB
\href{https://packages.typst.org/preview/frackable-0.2.0.tar.gz}{\pandocbounded{\includesvg[keepaspectratio]{/assets/icons/16-download.svg}}}
\item[Repository:]
\href{https://www.github.com/jamesrswift/frackable}{GitHub}
\item[Categor ies :]
\begin{itemize}
\tightlist
\item[]
\item
  \pandocbounded{\includesvg[keepaspectratio]{/assets/icons/16-package.svg}}
  \href{https://typst.app/universe/search/?category=components}{Components}
\item
  \pandocbounded{\includesvg[keepaspectratio]{/assets/icons/16-text.svg}}
  \href{https://typst.app/universe/search/?category=text}{Text}
\end{itemize}
\end{description}

\subsubsection{Where to report issues?}\label{where-to-report-issues}

This package is a project of James R. Swift . Report issues on
\href{https://www.github.com/jamesrswift/frackable}{their repository} .
You can also try to ask for help with this package on the
\href{https://forum.typst.app}{Forum} .

Please report this package to the Typst team using the
\href{https://typst.app/contact}{contact form} if you believe it is a
safety hazard or infringes upon your rights.

\phantomsection\label{versions}
\subsubsection{Version history}\label{version-history}

\begin{longtable}[]{@{}ll@{}}
\toprule\noalign{}
Version & Release Date \\
\midrule\noalign{}
\endhead
\bottomrule\noalign{}
\endlastfoot
0.2.0 & September 27, 2024 \\
\href{https://typst.app/universe/package/frackable/0.1.0/}{0.1.0} &
September 24, 2024 \\
\end{longtable}

Typst GmbH did not create this package and cannot guarantee correct
functionality of this package or compatibility with any version of the
Typst compiler or app.


\section{Package List LaTeX/dashing-dept-news.tex}
\title{typst.app/universe/package/dashing-dept-news}

\phantomsection\label{banner}
\phantomsection\label{template-thumbnail}
\pandocbounded{\includegraphics[keepaspectratio]{https://packages.typst.org/preview/thumbnails/dashing-dept-news-0.1.1-small.webp}}

\section{dashing-dept-news}\label{dashing-dept-news}

{ 0.1.1 }

Share the news with bold graphic design and a modern layout

{ } Featured Template

\href{/app?template=dashing-dept-news&version=0.1.1}{Create project in
app}

\phantomsection\label{readme}
A fun newsletter layout for departmental news. The template contains a
hero image, a main column, and a margin with secondary articles.

Place content in the sidebar with the \texttt{\ article\ } function, and
use the cool customized \texttt{\ blockquote\ } s and
\texttt{\ figure\ } s!

\subsection{Usage}\label{usage}

You can use this template in the Typst web app by clicking “Start from
template� on the dashboard and searching for
\texttt{\ dashing-dept-news\ } .

Alternatively, you can use the CLI to kick this project off using the
command

\begin{verbatim}
typst init @preview/dashing-dept-news
\end{verbatim}

Typst will create a new directory with all the files needed to get you
started.

\subsection{Configuration}\label{configuration}

This template exports the \texttt{\ newsletter\ } function with the
following named arguments:

\begin{itemize}
\tightlist
\item
  \texttt{\ title\ } : The newsletter’s title as content.
\item
  \texttt{\ edition\ } : The edition of the newsletter as content or
  \texttt{\ none\ } . This is displayed at the top of the sidebar.
\item
  \texttt{\ hero-image\ } : A dictionary with the keys
  \texttt{\ image\ } and \texttt{\ caption\ } or \texttt{\ none\ } .
  Image is content with the hero image while \texttt{\ caption\ } is
  content that is displayed to the right of the image.
\item
  \texttt{\ publication-info\ } : More information about the publication
  as content or \texttt{\ none\ } . It is displayed at the end of the
  document.
\end{itemize}

The function also accepts a single, positional argument for the body of
the newsletter’s main column and exports the \texttt{\ article\ }
function accepting a single content argument to populate the sidebar.

The template will initialize your package with a sample call to the
\texttt{\ newsletter\ } function in a show rule. If you, however, want
to change an existing project to use this template, you can add a show
rule like this at the top of your file:

\begin{Shaded}
\begin{Highlighting}[]
\NormalTok{\#import "@preview/dashing{-}dept{-}news:0.1.1": newsletter, article}
\NormalTok{\#show: newsletter.with(}
\NormalTok{  title: [Chemistry Department],}
\NormalTok{  edition: [}
\NormalTok{    March 18th, 2023 \textbackslash{}}
\NormalTok{    Purview College}
\NormalTok{  ],}
\NormalTok{  hero{-}image: (}
\NormalTok{    image: image("newsletter{-}cover.jpg"),}
\NormalTok{    caption: [Award{-}wining science],}
\NormalTok{  ),}
\NormalTok{  publication{-}info: [}
\NormalTok{    The Dean of the Department of Chemistry. \textbackslash{}}
\NormalTok{    Purview College, 17 Earlmeyer D, Exampleville, TN 59341. \textbackslash{}}
\NormalTok{    \#link("mailto:newsletter@chem.purview.edu")}
\NormalTok{  ],}
\NormalTok{)}

\NormalTok{// Your content goes here. Use \textasciigrave{}article\textasciigrave{} to populate the sidebar and \textasciigrave{}blockquote\textasciigrave{} for cool pull quotes.}
\end{Highlighting}
\end{Shaded}

\href{/app?template=dashing-dept-news&version=0.1.1}{Create project in
app}

\subsubsection{How to use}\label{how-to-use}

Click the button above to create a new project using this template in
the Typst app.

You can also use the Typst CLI to start a new project on your computer
using this command:

\begin{verbatim}
typst init @preview/dashing-dept-news:0.1.1
\end{verbatim}

\includesvg[width=0.16667in,height=0.16667in]{/assets/icons/16-copy.svg}

\subsubsection{About}\label{about}

\begin{description}
\tightlist
\item[Author :]
\href{https://typst.app}{Typst GmbH}
\item[License:]
MIT-0
\item[Current version:]
0.1.1
\item[Last updated:]
October 29, 2024
\item[First released:]
March 6, 2024
\item[Minimum Typst version:]
0.11.0
\item[Archive size:]
125 kB
\href{https://packages.typst.org/preview/dashing-dept-news-0.1.1.tar.gz}{\pandocbounded{\includesvg[keepaspectratio]{/assets/icons/16-download.svg}}}
\item[Repository:]
\href{https://github.com/typst/templates}{GitHub}
\item[Categor y :]
\begin{itemize}
\tightlist
\item[]
\item
  \pandocbounded{\includesvg[keepaspectratio]{/assets/icons/16-envelope.svg}}
  \href{https://typst.app/universe/search/?category=office}{Office}
\end{itemize}
\end{description}

\subsubsection{Where to report issues?}\label{where-to-report-issues}

This template is a project of Typst GmbH . Report issues on
\href{https://github.com/typst/templates}{their repository} . You can
also try to ask for help with this template on the
\href{https://forum.typst.app}{Forum} .

\phantomsection\label{versions}
\subsubsection{Version history}\label{version-history}

\begin{longtable}[]{@{}ll@{}}
\toprule\noalign{}
Version & Release Date \\
\midrule\noalign{}
\endhead
\bottomrule\noalign{}
\endlastfoot
0.1.1 & October 29, 2024 \\
\href{https://typst.app/universe/package/dashing-dept-news/0.1.0/}{0.1.0}
& March 6, 2024 \\
\end{longtable}


\section{Package List LaTeX/genealotree.tex}
\title{typst.app/universe/package/genealotree}

\phantomsection\label{banner}
\section{genealotree}\label{genealotree}

{ 0.1.0 }

A package to draw genealogical trees, based on CeTZ

\phantomsection\label{readme}
Genealotree is a typst package to draw genealogical trees. It is
developped at \url{https://codeberg.org/drloiseau/genealogy} . This is
the place you can get the developpement version and send issues and pull
requests.

\pandocbounded{\includegraphics[keepaspectratio]{https://github.com/typst/packages/raw/main/packages/preview/genealotree/0.1.0/examples/example.jpg}}

This package is based on
\href{https://github.com/typst/packages/raw/main/packages/preview/genealotree/0.1.0/\%22https://typst.app/universe/package/cetz/\%22}{CeTZ}
and it provides functions to draw genealogical trees. It is oriented
towards medical genealogy to study genetic disorders inheritance, but
you might be able to use it to draw your family tree.

\textbf{Features :}

\begin{itemize}
\tightlist
\item
  Draw an unlimited number of independant genealogical trees
\item
  Supports consanguinity and unions between different trees (see
  limitations)
\item
  Auto adjusts position of children to optimize spacing
\item
  Customize all lengths
\item
  Draw as much phenotypes as needed by coloring individuals
\item
  Print genotype and/or phenotype labels under individuals
\end{itemize}

\textbf{Limitations :}

\begin{itemize}
\tightlist
\item
  Must manually adjust individual position in the tree when drawing
  consanguinity and unions between trees to prevent overlapping of
  individuals.
\item
  No remarriages (might be added in a future version)
\item
  No union between individuals at different generations (might be added
  in a future version)
\end{itemize}

\textbf{To be implemented :}

\begin{itemize}
\tightlist
\item
  Allow to pass CeTZ arguments to every elements to cutomize their
  appearance
\item
  Draw optional legends for tree symbols and for phenotypes
\end{itemize}

See example.typ for a simple usage example, and the manual for precise
references at
\href{https://codeberg.org/attachments/cfdad2b7-52ae-4e18-8d7b-453fadc45532}{manual.pdf}

The steps to produce a tree are :

\begin{itemize}
\tightlist
\item
  Import the package and CeTZ
\end{itemize}

\begin{Shaded}
\begin{Highlighting}[]
\NormalTok{\#import "@preview/genealotree:0.1.0": *}
\NormalTok{\#import "@preview/cetz:0.2.2": canvas}
\end{Highlighting}
\end{Shaded}

\begin{itemize}
\tightlist
\item
  Create a genealogy object
\end{itemize}

\begin{Shaded}
\begin{Highlighting}[]
\NormalTok{let my{-}geneal = genealogy{-}init()}
\end{Highlighting}
\end{Shaded}

\begin{itemize}
\tightlist
\item
  Add persons to the object : pass a dictionary mapping a persons name
  with a dictionary describing its characteristics. See the manual for a
  full reference.
\end{itemize}

\begin{Shaded}
\begin{Highlighting}[]
\NormalTok{let my{-}geneal = add{-}persons(}
\NormalTok{  my{-}geneal,}
\NormalTok{  (}
\NormalTok{    "I1": (sex: "m"),}
\NormalTok{    "I2": (sex: "f"),}
\NormalTok{    "II1": (sex: "f"),}
\NormalTok{  )}
\NormalTok{)}
\end{Highlighting}
\end{Shaded}

\begin{itemize}
\tightlist
\item
  Set unions between persons : give the parents names as an array of 2
  strings, and the children names as an array of strings.
\end{itemize}

\begin{Shaded}
\begin{Highlighting}[]
\NormalTok{let my{-}geneal = add{-}unions(}
\NormalTok{  my{-}geneal,}
\NormalTok{  (("I1", "I2"), ("II1",))}
\NormalTok{)}
\end{Highlighting}
\end{Shaded}

\begin{itemize}
\tightlist
\item
  Open up a CeTZ canva and draw the tree
\end{itemize}

\begin{Shaded}
\begin{Highlighting}[]
\NormalTok{\#canvas(length: 0.4cm, \{}
\NormalTok{    draw{-}tree(my{-}geneal)}
\NormalTok{\})}
\end{Highlighting}
\end{Shaded}

\subsubsection{How to add}\label{how-to-add}

Copy this into your project and use the import as
\texttt{\ genealotree\ }

\begin{verbatim}
#import "@preview/genealotree:0.1.0"
\end{verbatim}

\includesvg[width=0.16667in,height=0.16667in]{/assets/icons/16-copy.svg}

Check the docs for
\href{https://typst.app/docs/reference/scripting/\#packages}{more
information on how to import packages} .

\subsubsection{About}\label{about}

\begin{description}
\tightlist
\item[Author :]
DrLoiseau
\item[License:]
GPL-3.0-only
\item[Current version:]
0.1.0
\item[Last updated:]
May 23, 2024
\item[First released:]
May 23, 2024
\item[Minimum Typst version:]
0.10.0
\item[Archive size:]
22.9 kB
\href{https://packages.typst.org/preview/genealotree-0.1.0.tar.gz}{\pandocbounded{\includesvg[keepaspectratio]{/assets/icons/16-download.svg}}}
\item[Repository:]
\href{https://codeberg.org/drloiseau/genealogy}{Codeberg}
\item[Discipline s :]
\begin{itemize}
\tightlist
\item[]
\item
  \href{https://typst.app/universe/search/?discipline=anthropology}{Anthropology}
\item
  \href{https://typst.app/universe/search/?discipline=biology}{Biology}
\item
  \href{https://typst.app/universe/search/?discipline=history}{History}
\item
  \href{https://typst.app/universe/search/?discipline=medicine}{Medicine}
\end{itemize}
\item[Categor y :]
\begin{itemize}
\tightlist
\item[]
\item
  \pandocbounded{\includesvg[keepaspectratio]{/assets/icons/16-chart.svg}}
  \href{https://typst.app/universe/search/?category=visualization}{Visualization}
\end{itemize}
\end{description}

\subsubsection{Where to report issues?}\label{where-to-report-issues}

This package is a project of DrLoiseau . Report issues on
\href{https://codeberg.org/drloiseau/genealogy}{their repository} . You
can also try to ask for help with this package on the
\href{https://forum.typst.app}{Forum} .

Please report this package to the Typst team using the
\href{https://typst.app/contact}{contact form} if you believe it is a
safety hazard or infringes upon your rights.

\phantomsection\label{versions}
\subsubsection{Version history}\label{version-history}

\begin{longtable}[]{@{}ll@{}}
\toprule\noalign{}
Version & Release Date \\
\midrule\noalign{}
\endhead
\bottomrule\noalign{}
\endlastfoot
0.1.0 & May 23, 2024 \\
\end{longtable}

Typst GmbH did not create this package and cannot guarantee correct
functionality of this package or compatibility with any version of the
Typst compiler or app.


\section{Package List LaTeX/fauxreilly.tex}
\title{typst.app/universe/package/fauxreilly}

\phantomsection\label{banner}
\section{fauxreilly}\label{fauxreilly}

{ 0.1.0 }

A package for creating O\textquotesingle Rly- /
O\textquotesingle Reilly-type cover pages

\phantomsection\label{readme}
\href{https://forthebadge.com/}{\pandocbounded{\includesvg[keepaspectratio]{https://raw.githubusercontent.com/dei-layborer/o-rly-typst/refs/heads/main/made-with-(2s)-2\%2C6-diamino-n-\%5B(2s)-1-phenylpropan-2-yl\%5Dhexanamide-n-\%5B(2s)-1-phenyl-2-propanyl\%5D-l-lysinamide.svg}}}

\href{https://deilayborer.neocities.org/funding}{\includesvg[width=\linewidth,height=0.3125in,keepaspectratio]{https://raw.githubusercontent.com/dei-layborer/o-rly-typst/refs/heads/main/\%24\%24\%24-gimmie.svg}}

A \texttt{\ typst\ } package for creating \textbf{O’RLY?} -style cover
pages.

\subsection{Example}\label{example}

\begin{Shaded}
\begin{Highlighting}[]
\NormalTok{\#import }\StringTok{"@preview/o{-}rly{-}cover:0.1.0"}\OperatorTok{:} \OperatorTok{*}

\NormalTok{\#orly(}
\NormalTok{    color}\OperatorTok{:}\NormalTok{ rgb(}\StringTok{"\#85144b"}\NormalTok{)}\OperatorTok{,}
\NormalTok{    title}\OperatorTok{:} \StringTok{"Learn to Stop Worrying and Love Metathesis"}\OperatorTok{,}
\NormalTok{    top}\OperatorTok{{-}}\NormalTok{text}\OperatorTok{:} \StringTok{"Axe nat why (or do)"}\OperatorTok{,}
\NormalTok{    subtitle}\OperatorTok{:} \StringTok{"Free yourself from prescriptivism"}\OperatorTok{,}
\NormalTok{    pic}\OperatorTok{:} \StringTok{"chomskydoz.png"}\OperatorTok{,}
\NormalTok{    signature}\OperatorTok{:} \StringTok{"Dr. N. Supponent"}
\NormalTok{)}
\end{Highlighting}
\end{Shaded}

\pandocbounded{\includegraphics[keepaspectratio]{https://github.com/typst/packages/raw/main/packages/preview/fauxreilly/0.1.0/example.png}}

\subsection{Usage}\label{usage}

First, import the package at the top of your \texttt{\ typst\ } file:
\texttt{\ \#import\ "@preview/o-rly-cover:0.1.0":\ *\ }

Only one function is exposed, \texttt{\ \#orly()\ } . This will create
its own page in the document at whatever location you call the function.
In other words, any content in the \texttt{\ typst\ } document that
appears before \texttt{\ \#orly()\ } is called will be before the
O’Rly? page in the PDF that \texttt{\ typst\ } renders. Anything after
the function call will be on subsequent page(s).

All content for the title page is passed as options to
\texttt{\ \#orly()\ } . I included what I figured were the most likely
things you’d want to customize without having a million options.
Meanwhile, most of the layout parameters (font sizes, the heights of
individual pieces, etc.) are variables within the code, so hopefully
aren’t too hard to alter if need-be. None of the options are strictly
required, although the text fields are the only ones that can be left
empty without potentially breaking the layout. A few have defaults
instead, and those are listed below where applicable.

\subsubsection{Options}\label{options}

The order that the options appear in the table is the order they must be
sent to the function, unless you specify the option’s key along with
its value.

Data types listed are based on \texttt{\ typst\ } ’s internal types,
so are entered the same way as they would be in any other function that
takes that data type. For example, the data type needed for the
\texttt{\ font\ } option is the same as what is used for
\texttt{\ typst\ } ’s built-in \texttt{\ \#text()\ } function, which
is linked in the table below. (All links go to their specific usage in
the \texttt{\ typst\ } documentation.)

\begin{longtable}[]{@{}llcc@{}}
\toprule\noalign{}
Option & Description & Type & Default \\
\midrule\noalign{}
\endhead
\bottomrule\noalign{}
\endlastfoot
\texttt{\ font\ } & The font for all text except the “publisher� in
the bottom-left corner &
\href{https://typst.app/docs/reference/text/text/\#parameters-font}{\texttt{\ string(s)\ }}
& Whatever is set in the document context \\
\texttt{\ color\ } & Accent color. Used for the background of the title
block and of the colored bar at the very top. &
\href{https://typst.app/docs/reference/visualize/color/}{\texttt{\ color\ }}
& \texttt{\ blue\ } (typst built-in) \\
\texttt{\ top-text\ } & The text at the top, just under the color bar &
\href{https://typst.app/docs/reference/foundations/str/}{\texttt{\ string\ }}
& Empty \\
\texttt{\ pic\ } & Image to be used above the title block &
\href{https://typst.app/docs/reference/visualize/image/\#parameters-path}{\texttt{\ string\ }}
with path to the image & Empty \\
\texttt{\ title\ } & The title of the book &
\href{https://typst.app/docs/reference/foundations/str/}{\texttt{\ string\ }}
& Empty \\
\texttt{\ title-align\ } & How the text is aligned (horizontally) in the
title block &
\href{https://typst.app/docs/reference/layout/alignment/}{\texttt{\ alignment\ }}
& \texttt{\ left\ } \\
\texttt{\ subtitle\ } & Text that appears just below the title block &
\href{https://typst.app/docs/reference/foundations/str/}{\texttt{\ string\ }}
& Empty \\
\texttt{\ publisher\ } & The name of the “publisher� in the
bottom-left &
\href{https://typst.app/docs/reference/foundations/str/}{\texttt{\ string\ }}
& O RLY \textsuperscript{?} (see example above) \\
\texttt{\ publisher-font\ } & Font to be used for “publisher� name &
\href{https://typst.app/docs/reference/text/text/\#parameters-font}{\texttt{\ string(s)\ }}
& Noto Sans, Arial Rounded MT, document context (in that order) \\
\texttt{\ signature\ } & Text in the bottom-right corner &
\href{https://typst.app/docs/reference/foundations/str/}{\texttt{\ string\ }}
& Empty \\
\texttt{\ margin\ } & Page margins &
\href{https://typst.app/docs/reference/layout/page/\#parameters-margin}{\texttt{\ length\ }
or \texttt{\ dictionary\ }} & \texttt{\ top:\ 0\ } , all others will use
the document context \\
\end{longtable}

\subsubsection{Usage Notes}\label{usage-notes}

There are a couple quirks to data types and the like that may not be
obvious.

\begin{enumerate}
\tightlist
\item
  \texttt{\ string\ } s typically must be contained in quotation marks.
  But note that this will render quotation marks \emph{within} those
  strings without using
  \href{https://typst.app/docs/reference/text/smartquote/}{smartquotes}
  . To avoid this, you may use content mode instead (by enclosing the
  text in square brackets \texttt{\ {[}{]}\ } ). For example,
  \texttt{\ "Some\ title"\ } â†' \texttt{\ {[}Some\ title{]}\ }

  \begin{itemize}
  \tightlist
  \item
    Similarly, you can use this to toggle italics (e.g.
    \texttt{\ {[}Italic\ text,\ \_except\_\ this\ one{]}\ } ) or apply
    other formatting
  \end{itemize}
\item
  Other types may look like strings but do \textbf{not} take quotes,
  specifically \texttt{\ color\ } (including when using the built-in
  color names) and \texttt{\ alignment\ }
\item
  With the \texttt{\ margin\ } type, if a single value is entered
  (without quotes), that value is applied to all four sides. All other
  usage must be done as a dictionary (meaning enclosed in parentheses),
  even if you’re only specifying one side. For example, the default is
  written \texttt{\ (top:\ 0in)\ } .

  \begin{itemize}
  \tightlist
  \item
    If you’re going to pass any value other than the top as an option,
    you’ll likely want to add \texttt{\ top:\ 0in\ } back in to avoid
    a gap between the top of the page and the color bar
  \item
    Any values passed to the function (or the default value if none are)
    will override any margin(s) set earlier in the \texttt{\ typst\ }
    file. So you can use a \texttt{\ set\ } rule at the beginning of the
    document without affecting the cover page
  \end{itemize}
\end{enumerate}

\subsubsection{Images}\label{images}

The package uses \texttt{\ typst\ } ’s built-in image handling, which
means it only supports PNG, JPG, and SVG.

The image will be resized to as close to 100\% page width (inside the
margins) as possible while both maintaining proportions and avoiding any
cropping. The rest of the layout \emph{should} flow reasonably well
around any image height, but outliers may exist.

O’Reilly-style animals can be found in the
\href{https://etc.usf.edu/clipart/galleries/730-animals}{relevant
section} of the Florida Center for Instructional Technology’s
\href{https://etc.usf.edu/clipart/}{ClipArt ETC} project. Just be aware
that these are provided as GIFs(!), so conversion to one of
\texttt{\ typst\ } ’s supported formats will be required.

\subsection{Bugs \& Feature Requests}\label{bugs-feature-requests}

I put this whole thing together in an afternoon when I should’ve been
doing work for my day job. Granted I’d already done a basic version
for a seminary writing assignment (I love to spoof academic writing),
but either way, I’ve gotten this project to a basic level of
functionality and no further. I’m entirely open to suggestions for
additional functionality, however, so feel free to
\href{https://github.com/dei-layborer/o-rly-typst/issues}{create an
issue} if there’s something you’d like to see added.

It hopefully goes without saying that the same is true if something
breaks!

Tested on \texttt{\ typst\ } versions \texttt{\ 0.11.1\ } and
\texttt{\ 0.12.0-rc1\ } .

\subsection{Thanks \& Shout-Outs}\label{thanks-shout-outs}

Shout out to Arthur Beaulieu (
\href{https://github.com/ArthurBeaulieu}{@arthurbeaulieu} ), whose
\href{https://arthurbeaulieu.github.io/ORlyGenerator/}{web-based
generator} served as both inspiration and reference (I pretty much stole
his layout settings).

Significant thanks to the folks in the
\href{https://discord.gg/2uDybryKPe}{typst discord} who helped me sort
out some layout woes.

Extra double appreciation to Enivex on the discord for the name.

\subsubsection{How to add}\label{how-to-add}

Copy this into your project and use the import as
\texttt{\ fauxreilly\ }

\begin{verbatim}
#import "@preview/fauxreilly:0.1.0"
\end{verbatim}

\includesvg[width=0.16667in,height=0.16667in]{/assets/icons/16-copy.svg}

Check the docs for
\href{https://typst.app/docs/reference/scripting/\#packages}{more
information on how to import packages} .

\subsubsection{About}\label{about}

\begin{description}
\tightlist
\item[Author :]
Dei Layborer
\item[License:]
GPL-3.0
\item[Current version:]
0.1.0
\item[Last updated:]
October 16, 2024
\item[First released:]
October 16, 2024
\item[Minimum Typst version:]
0.11.1
\item[Archive size:]
4.48 kB
\href{https://packages.typst.org/preview/fauxreilly-0.1.0.tar.gz}{\pandocbounded{\includesvg[keepaspectratio]{/assets/icons/16-download.svg}}}
\item[Repository:]
\href{https://github.com/dei-layborer/o-rly-typst}{GitHub}
\item[Categor ies :]
\begin{itemize}
\tightlist
\item[]
\item
  \pandocbounded{\includesvg[keepaspectratio]{/assets/icons/16-package.svg}}
  \href{https://typst.app/universe/search/?category=components}{Components}
\item
  \pandocbounded{\includesvg[keepaspectratio]{/assets/icons/16-layout.svg}}
  \href{https://typst.app/universe/search/?category=layout}{Layout}
\item
  \pandocbounded{\includesvg[keepaspectratio]{/assets/icons/16-smile.svg}}
  \href{https://typst.app/universe/search/?category=fun}{Fun}
\end{itemize}
\end{description}

\subsubsection{Where to report issues?}\label{where-to-report-issues}

This package is a project of Dei Layborer . Report issues on
\href{https://github.com/dei-layborer/o-rly-typst}{their repository} .
You can also try to ask for help with this package on the
\href{https://forum.typst.app}{Forum} .

Please report this package to the Typst team using the
\href{https://typst.app/contact}{contact form} if you believe it is a
safety hazard or infringes upon your rights.

\phantomsection\label{versions}
\subsubsection{Version history}\label{version-history}

\begin{longtable}[]{@{}ll@{}}
\toprule\noalign{}
Version & Release Date \\
\midrule\noalign{}
\endhead
\bottomrule\noalign{}
\endlastfoot
0.1.0 & October 16, 2024 \\
\end{longtable}

Typst GmbH did not create this package and cannot guarantee correct
functionality of this package or compatibility with any version of the
Typst compiler or app.


\section{Package List LaTeX/headcount.tex}
\title{typst.app/universe/package/headcount}

\phantomsection\label{banner}
\section{headcount}\label{headcount}

{ 0.1.0 }

Make counters inherit from the heading counter.

\phantomsection\label{readme}
This package allows you to make \textbf{counters depend on the current
chapter/section number} .

This works for \textbf{figures, theorems, and any other counters} .

The advantage compared to
\href{https://typst.app/universe/package/rich-counters/}{rich-counters}
is that you stick with native \texttt{\ counter\ } s and you can
influence e.g. the \texttt{\ figure\ } counter directly without writing
a new \texttt{\ show\ } rule with a custom counter or so.

\subsection{Showcase}\label{showcase}

In the following example, we demonstrate how you can inherit 1 level of
the heading counter for figures and 2 levels for theorems.

\begin{Shaded}
\begin{Highlighting}[]
\NormalTok{\#import "@preview/headcount:0.1.0": *}
\NormalTok{\#import "@preview/great{-}theorems:0.1.0": *}

\NormalTok{\#show: great{-}theorems{-}init}

\NormalTok{\#set heading(numbering: "1.1")}

\NormalTok{// contruct theorem environment with counter that inherits 2 levels from heading}
\NormalTok{\#let thmcounter = counter("hello")}
\NormalTok{\#let theorem = mathblock(}
\NormalTok{  blocktitle: [Theorem],}
\NormalTok{  counter: thmcounter,}
\NormalTok{  numbering: dependent{-}numbering("1.1", levels: 2)}
\NormalTok{)}
\NormalTok{\#show heading: reset{-}counter(thmcounter, levels: 2)}

\NormalTok{// set figure counter so that it inherits 1 level from heading}
\NormalTok{\#set figure(numbering: dependent{-}numbering("1.1"))}
\NormalTok{\#show heading: reset{-}counter(counter(figure.where(kind: image)))}

\NormalTok{= First heading}

\NormalTok{The theorems inherit 2 levels from the headings and the figures inherit 1 level from the headings.}

\NormalTok{\#theorem[Some theorem.]}
\NormalTok{\#theorem[Some theorem.]}
\NormalTok{\#figure([SOME FIGURE], caption: [some figure])}
\NormalTok{\#figure([SOME FIGURE], caption: [some figure])}

\NormalTok{== Subheading}

\NormalTok{\#theorem[Some theorem.]}
\NormalTok{\#figure([SOME FIGURE], caption: [some figure])}
\NormalTok{\#figure([SOME FIGURE], caption: [some figure])}

\NormalTok{= Second heading}

\NormalTok{\#theorem[Some theorem.]}
\NormalTok{\#figure([SOME FIGURE], caption: [some figure])}
\NormalTok{\#theorem[Some theorem.]}
\end{Highlighting}
\end{Shaded}

\pandocbounded{\includegraphics[keepaspectratio]{https://github.com/typst/packages/raw/main/packages/preview/headcount/0.1.0/example.png}}

\subsection{Usage}\label{usage}

To make another \texttt{\ counter\ } inherit from the heading counter,
you have to do \textbf{two} things.

\begin{enumerate}
\item
  For the numbering of your counter, you have to use
  \texttt{\ dependent-numbering(...)\ } .

  \begin{itemize}
  \item
    \texttt{\ dependent-numbering(style,\ level:\ 1)\ } (needs
    \texttt{\ context\ } )

    Is a replacement for the \texttt{\ numbering\ } function, with the
    difference that it precedes any counter value with
    \texttt{\ level\ } many values of the heading counter.
  \end{itemize}

\begin{Shaded}
\begin{Highlighting}[]
\NormalTok{\#import "@preview/headcount:0.1.0": *}

\NormalTok{\#set heading(numbering: "1.1")}

\NormalTok{\#let mycounter = counter("hello")}

\NormalTok{= First heading}

\NormalTok{\#context mycounter.step()}
\NormalTok{\#context mycounter.display(dependent{-}numbering("1.1"))}

\NormalTok{= Second heading}

\NormalTok{\#context mycounter.step()}
\NormalTok{\#context mycounter.display(dependent{-}numbering("1.1"))}

\NormalTok{\#context mycounter.step()}
\NormalTok{\#context mycounter.display(dependent{-}numbering("1.1"))}
\end{Highlighting}
\end{Shaded}

  This displays the desired amount of levels of the heading counter in
  front of the actual counter. However, as you can see in the code
  above, our actual counter does not yet reset in each section.
\item
  For resetting the counter at the appropriate places, you need to equip
  \texttt{\ heading\ } with the \texttt{\ show\ } rule that
  \texttt{\ reset-counter(...)\ } returns.

  \begin{itemize}
  \item
    \texttt{\ reset-counter(counter,\ level:\ 1)\ } (needs
    \texttt{\ context\ } )

    Returns a function that should be used as a \texttt{\ show\ } rule
    for \texttt{\ heading\ } . It will reset \texttt{\ counter\ } if the
    level of the heading is less than or equal to \texttt{\ level\ } .
  \end{itemize}

  \textbf{Important:} This \texttt{\ show\ } rule should be placed as
  the \emph{last} \texttt{\ show\ } rule for \texttt{\ heading\ } , or
  at least after \texttt{\ show\ } rules for \texttt{\ heading\ } that
  employ a custom design, see
  \href{https://forum.typst.app/t/i-figured-broken-with-custom-template/1730/10?u=jbirnick}{here}
  for an explanation.

\begin{Shaded}
\begin{Highlighting}[]
\NormalTok{\#import "@preview/headcount:0.1.0": *}

\NormalTok{\#set heading(numbering: "1.1")}
\NormalTok{\#show heading: reset{-}counter(mycounter, levels: 1)}

\NormalTok{\#let mycounter = counter("hello")}

\NormalTok{= First heading}

\NormalTok{\#context mycounter.step()}
\NormalTok{\#context mycounter.display(dependent{-}numbering("1.1"))}

\NormalTok{= Second heading}

\NormalTok{\#context mycounter.step()}
\NormalTok{\#context mycounter.display(dependent{-}numbering("1.1"))}

\NormalTok{\#context mycounter.step()}
\NormalTok{\#context mycounter.display(dependent{-}numbering("1.1"))}
\end{Highlighting}
\end{Shaded}
\end{enumerate}

\textbf{Note:} The \texttt{\ level\ } that you pass to
\texttt{\ dependent-numbering(...)\ } and the \texttt{\ level\ } that
you pass to \texttt{\ reset-counter(...)\ } must be the \emph{same} .

\subsection{Limitations}\label{limitations}

Due to current Typst limitations, there is no way to detect manual
updates or steps of the heading counter, like
\texttt{\ counter(heading).update(...)\ } or
\texttt{\ counter(heading).step(...)\ } . Only occurrences of actual
\texttt{\ heading\ } s can be detected. So make sure that after you call
e.g. \texttt{\ counter(heading).update(...)\ } , you place a heading
directly after it, before you use any counters that depend on the
heading counter.

\subsubsection{How to add}\label{how-to-add}

Copy this into your project and use the import as \texttt{\ headcount\ }

\begin{verbatim}
#import "@preview/headcount:0.1.0"
\end{verbatim}

\includesvg[width=0.16667in,height=0.16667in]{/assets/icons/16-copy.svg}

Check the docs for
\href{https://typst.app/docs/reference/scripting/\#packages}{more
information on how to import packages} .

\subsubsection{About}\label{about}

\begin{description}
\tightlist
\item[Author :]
\href{https://jbirnick.net}{Johann Birnick}
\item[License:]
MIT
\item[Current version:]
0.1.0
\item[Last updated:]
October 16, 2024
\item[First released:]
October 16, 2024
\item[Archive size:]
2.67 kB
\href{https://packages.typst.org/preview/headcount-0.1.0.tar.gz}{\pandocbounded{\includesvg[keepaspectratio]{/assets/icons/16-download.svg}}}
\item[Repository:]
\href{https://github.com/jbirnick/typst-headcount}{GitHub}
\item[Categor ies :]
\begin{itemize}
\tightlist
\item[]
\item
  \pandocbounded{\includesvg[keepaspectratio]{/assets/icons/16-list-unordered.svg}}
  \href{https://typst.app/universe/search/?category=model}{Model}
\item
  \pandocbounded{\includesvg[keepaspectratio]{/assets/icons/16-code.svg}}
  \href{https://typst.app/universe/search/?category=scripting}{Scripting}
\item
  \pandocbounded{\includesvg[keepaspectratio]{/assets/icons/16-hammer.svg}}
  \href{https://typst.app/universe/search/?category=utility}{Utility}
\end{itemize}
\end{description}

\subsubsection{Where to report issues?}\label{where-to-report-issues}

This package is a project of Johann Birnick . Report issues on
\href{https://github.com/jbirnick/typst-headcount}{their repository} .
You can also try to ask for help with this package on the
\href{https://forum.typst.app}{Forum} .

Please report this package to the Typst team using the
\href{https://typst.app/contact}{contact form} if you believe it is a
safety hazard or infringes upon your rights.

\phantomsection\label{versions}
\subsubsection{Version history}\label{version-history}

\begin{longtable}[]{@{}ll@{}}
\toprule\noalign{}
Version & Release Date \\
\midrule\noalign{}
\endhead
\bottomrule\noalign{}
\endlastfoot
0.1.0 & October 16, 2024 \\
\end{longtable}

Typst GmbH did not create this package and cannot guarantee correct
functionality of this package or compatibility with any version of the
Typst compiler or app.


\section{Package List LaTeX/alchemist.tex}
\title{typst.app/universe/package/alchemist}

\phantomsection\label{banner}
\section{alchemist}\label{alchemist}

{ 0.1.2 }

A package to render skeletal formulas using cetz

{ } Featured Package

\phantomsection\label{readme}
Alchemist is a typst package to draw skeletal formulae. It is based on
the \href{https://ctan.org/pkg/chemfig}{chemfig} package. The main goal
of alchemist is not to reproduce one-to-one chemfig. Instead, it aims to
provide an interface to achieve the same results in Typst.

\begin{Shaded}
\begin{Highlighting}[]
\NormalTok{\#skeletize(\{}
\NormalTok{  molecule(name: "A", "A")}
\NormalTok{  single()}
\NormalTok{  molecule("B")}
\NormalTok{  branch(\{}
\NormalTok{    single(angle: 1)}
\NormalTok{    molecule(}
\NormalTok{      "W",}
\NormalTok{      links: (}
\NormalTok{        "A": double(stroke: red),}
\NormalTok{      ),}
\NormalTok{    )}
\NormalTok{    single()}
\NormalTok{    molecule(name: "X", "X")}
\NormalTok{  \})}
\NormalTok{  branch(\{}
\NormalTok{    single(angle: {-}1)}
\NormalTok{    molecule("Y")}
\NormalTok{    single()}
\NormalTok{    molecule(}
\NormalTok{      name: "Z",}
\NormalTok{      "Z",}
\NormalTok{      links: (}
\NormalTok{        "X": single(stroke: black + 3pt),}
\NormalTok{      ),}
\NormalTok{    )}
\NormalTok{  \})}
\NormalTok{  single()}
\NormalTok{  molecule(}
\NormalTok{    "C",}
\NormalTok{    links: (}
\NormalTok{      "X": cram{-}filled{-}left(fill: blue),}
\NormalTok{      "Z": single(),}
\NormalTok{    ),}
\NormalTok{  )}
\NormalTok{\})}
\end{Highlighting}
\end{Shaded}

\pandocbounded{\includegraphics[keepaspectratio]{https://raw.githubusercontent.com/Robotechnic/alchemist/master/images/links1.png}}

Alchemist uses cetz to draw the molecules. This means that you can draw
cetz shapes in the same canvas as the molecules. Like this:

\begin{Shaded}
\begin{Highlighting}[]
\NormalTok{\#skeletize(\{}
\NormalTok{  import cetz.draw: *}
\NormalTok{  double(absolute: 30deg, name: "l1")}
\NormalTok{  single(absolute: {-}30deg, name: "l2")}
\NormalTok{  molecule("X", name: "X")}
\NormalTok{  hobby(}
\NormalTok{    "l1.50\%",}
\NormalTok{    ("l1.start", 0.5, 90deg, "l1.end"),}
\NormalTok{    "l1.start",}
\NormalTok{    stroke: (paint: red, dash: "dashed"),}
\NormalTok{    mark: (end: "\textgreater{}"),}
\NormalTok{  )}
\NormalTok{  hobby(}
\NormalTok{    (to: "X.north", rel: (0, 1pt)),}
\NormalTok{    ("l2.end", 0.4, {-}90deg, "l2.start"),}
\NormalTok{    "l2.50\%",}
\NormalTok{    mark: (end: "\textgreater{}"),}
\NormalTok{  )}
\NormalTok{\})}
\end{Highlighting}
\end{Shaded}

\pandocbounded{\includegraphics[keepaspectratio]{https://raw.githubusercontent.com/Robotechnic/alchemist/master/images/cetz1.png}}

\subsection{Usage}\label{usage}

To start using alchemist, just use the following code:

\begin{Shaded}
\begin{Highlighting}[]
\NormalTok{\#import "@preview/alchemist:0.1.2": *}

\NormalTok{\#skeletize(\{}
\NormalTok{  // Your molecule here}
\NormalTok{\})}
\end{Highlighting}
\end{Shaded}

For more information, check the
\href{https://raw.githubusercontent.com/Robotechnic/alchemist/master/doc/manual.pdf}{manual}
.

\subsection{Changelog}\label{changelog}

\subsubsection{0.1.2}\label{section}

\begin{itemize}
\tightlist
\item
  Added default values for link style properties.
\item
  Updated \texttt{\ cetz\ } to version 0.3.1.
\item
  Added a \texttt{\ tip-lenght\ } argument to dashed cram links.
\end{itemize}

\subsubsection{0.1.1}\label{section-1}

\begin{itemize}
\tightlist
\item
  Exposed the \texttt{\ draw-skeleton\ } function. This allows to draw
  in a cetz canvas directly.
\item
  Fixed multiples bugs that causes overdraws of links.
\end{itemize}

\subsubsection{0.1.0}\label{section-2}

\begin{itemize}
\tightlist
\item
  Initial release
\end{itemize}

\subsubsection{How to add}\label{how-to-add}

Copy this into your project and use the import as \texttt{\ alchemist\ }

\begin{verbatim}
#import "@preview/alchemist:0.1.2"
\end{verbatim}

\includesvg[width=0.16667in,height=0.16667in]{/assets/icons/16-copy.svg}

Check the docs for
\href{https://typst.app/docs/reference/scripting/\#packages}{more
information on how to import packages} .

\subsubsection{About}\label{about}

\begin{description}
\tightlist
\item[Author :]
\href{https://github.com/Robotechnic}{Robotechnic}
\item[License:]
MIT
\item[Current version:]
0.1.2
\item[Last updated:]
November 13, 2024
\item[First released:]
August 14, 2024
\item[Minimum Typst version:]
0.11.0
\item[Archive size:]
11.5 kB
\href{https://packages.typst.org/preview/alchemist-0.1.2.tar.gz}{\pandocbounded{\includesvg[keepaspectratio]{/assets/icons/16-download.svg}}}
\item[Repository:]
\href{https://github.com/Robotechnic/alchemist}{GitHub}
\item[Discipline s :]
\begin{itemize}
\tightlist
\item[]
\item
  \href{https://typst.app/universe/search/?discipline=chemistry}{Chemistry}
\item
  \href{https://typst.app/universe/search/?discipline=biology}{Biology}
\end{itemize}
\item[Categor y :]
\begin{itemize}
\tightlist
\item[]
\item
  \pandocbounded{\includesvg[keepaspectratio]{/assets/icons/16-chart.svg}}
  \href{https://typst.app/universe/search/?category=visualization}{Visualization}
\end{itemize}
\end{description}

\subsubsection{Where to report issues?}\label{where-to-report-issues}

This package is a project of Robotechnic . Report issues on
\href{https://github.com/Robotechnic/alchemist}{their repository} . You
can also try to ask for help with this package on the
\href{https://forum.typst.app}{Forum} .

Please report this package to the Typst team using the
\href{https://typst.app/contact}{contact form} if you believe it is a
safety hazard or infringes upon your rights.

\phantomsection\label{versions}
\subsubsection{Version history}\label{version-history}

\begin{longtable}[]{@{}ll@{}}
\toprule\noalign{}
Version & Release Date \\
\midrule\noalign{}
\endhead
\bottomrule\noalign{}
\endlastfoot
0.1.2 & November 13, 2024 \\
\href{https://typst.app/universe/package/alchemist/0.1.1/}{0.1.1} &
August 19, 2024 \\
\href{https://typst.app/universe/package/alchemist/0.1.0/}{0.1.0} &
August 14, 2024 \\
\end{longtable}

Typst GmbH did not create this package and cannot guarantee correct
functionality of this package or compatibility with any version of the
Typst compiler or app.


\section{Package List LaTeX/miage-rapide-tp.tex}
\title{typst.app/universe/package/miage-rapide-tp}

\phantomsection\label{banner}
\phantomsection\label{template-thumbnail}
\pandocbounded{\includegraphics[keepaspectratio]{https://packages.typst.org/preview/thumbnails/miage-rapide-tp-0.1.2-small.webp}}

\section{miage-rapide-tp}\label{miage-rapide-tp}

{ 0.1.2 }

Quickly generate a report for MIAGE practical work.

\href{/app?template=miage-rapide-tp&version=0.1.2}{Create project in
app}

\phantomsection\label{readme}
Typst template to generate a practical work report for students of the
MIAGE (Méthodes Informatiques Appliquées Ã~ la Gestion des
Entreprises).

\subsection{ðŸ§`â€?ðŸ'» Usage}\label{uxf0uxffuxe2uxf0uxff-usage}

\begin{itemize}
\item
  Directly from \href{https://typst.app/}{Typst web app} by clicking
  “Start from template� on the dashboard and searching for
  \texttt{\ miage-rapide-tp\ } .
\item
  With CLI:
\end{itemize}

\begin{verbatim}
typst init @preview/miage-rapide-tp:{version}
\end{verbatim}

\subsection{🚀 Features}\label{uxf0uxffux161-features}

\begin{itemize}
\tightlist
\item
  Cover page
\item
  Table of contents (optionnal)
\item
  \texttt{\ question\ } = automatically generates a question number
  (optionnal) with the content of the question
\item
  \texttt{\ code-block\ } = code block with syntax highlighting. You can
  pass a filepath or code directly to display in the block
\item
  \texttt{\ remarque\ } = a remark block with content and color
\end{itemize}

\subsubsection{Cover page}\label{cover-page}

The conf looks like this:

\begin{Shaded}
\begin{Highlighting}[]
\NormalTok{\#let conf(}
\NormalTok{  subtitle: none,}
\NormalTok{  authors: (),}
\NormalTok{  toc: true,}
\NormalTok{  lang: "fr",}
\NormalTok{  font: "Satoshi",}
\NormalTok{  date: none,}
\NormalTok{  years: (2024, 2025),}
\NormalTok{  years{-}label: "Année universitaire",}
\NormalTok{  title,}
\NormalTok{  doc,}
\NormalTok{)}
\end{Highlighting}
\end{Shaded}

\subsubsection{Question}\label{question}

A question can be added like this:

\begin{Shaded}
\begin{Highlighting}[]
\NormalTok{\#question("Une question avec numéro ?")}
\NormalTok{\#question("Une question sans numéro ?", counter: false)}
\end{Highlighting}
\end{Shaded}

The first argument is the question content, and the second (OPTIONAL) is
the counter. If \texttt{\ counter\ } is set to \texttt{\ false\ } , the
question will not be numbered.

\subsubsection{Code-block}\label{code-block}

To use a \texttt{\ code-block\ } , you can do as follows :

\begin{Shaded}
\begin{Highlighting}[]
\NormalTok{\#code{-}block(read("code/main.py"), "py")}
\NormalTok{\#code{-}block(read("code/example.sql"), "sql", title: "Classic SQL")}
\end{Highlighting}
\end{Shaded}

The first argument is the code to display, the second is the language of
the code, and the third is the title of the code block.

\subsubsection{Remarque}\label{remarque}

To use a \texttt{\ remarque\ } , you can do as follows :

\begin{Shaded}
\begin{Highlighting}[]
\NormalTok{\#remarque("Ceci est une remarque")}
\NormalTok{\#remarque("Remarque personnalisée", bg{-}color: olive, text{-}color: white)}
\end{Highlighting}
\end{Shaded}

You can change the bg-color and text-color of the remark block to match
your needs.

\subsection{ðŸ``? License}\label{uxf0uxff-license}

This is MIT licensed.

\begin{quote}
Rapide means fast in French. tp is the abbreviation of “travaux
pratiques� which means practical work. MIAGE is a French degree in
computer science applied to management.
\end{quote}

\href{/app?template=miage-rapide-tp&version=0.1.2}{Create project in
app}

\subsubsection{How to use}\label{how-to-use}

Click the button above to create a new project using this template in
the Typst app.

You can also use the Typst CLI to start a new project on your computer
using this command:

\begin{verbatim}
typst init @preview/miage-rapide-tp:0.1.2
\end{verbatim}

\includesvg[width=0.16667in,height=0.16667in]{/assets/icons/16-copy.svg}

\subsubsection{About}\label{about}

\begin{description}
\tightlist
\item[Author :]
Rémi Saurel
\item[License:]
MIT-0
\item[Current version:]
0.1.2
\item[Last updated:]
September 25, 2024
\item[First released:]
September 11, 2024
\item[Archive size:]
299 kB
\href{https://packages.typst.org/preview/miage-rapide-tp-0.1.2.tar.gz}{\pandocbounded{\includesvg[keepaspectratio]{/assets/icons/16-download.svg}}}
\item[Discipline s :]
\begin{itemize}
\tightlist
\item[]
\item
  \href{https://typst.app/universe/search/?discipline=education}{Education}
\item
  \href{https://typst.app/universe/search/?discipline=engineering}{Engineering}
\item
  \href{https://typst.app/universe/search/?discipline=computer-science}{Computer
  Science}
\end{itemize}
\item[Categor y :]
\begin{itemize}
\tightlist
\item[]
\item
  \pandocbounded{\includesvg[keepaspectratio]{/assets/icons/16-speak.svg}}
  \href{https://typst.app/universe/search/?category=report}{Report}
\end{itemize}
\end{description}

\subsubsection{Where to report issues?}\label{where-to-report-issues}

This template is a project of Rémi Saurel . You can also try to ask for
help with this template on the \href{https://forum.typst.app}{Forum} .

Please report this template to the Typst team using the
\href{https://typst.app/contact}{contact form} if you believe it is a
safety hazard or infringes upon your rights.

\phantomsection\label{versions}
\subsubsection{Version history}\label{version-history}

\begin{longtable}[]{@{}ll@{}}
\toprule\noalign{}
Version & Release Date \\
\midrule\noalign{}
\endhead
\bottomrule\noalign{}
\endlastfoot
0.1.2 & September 25, 2024 \\
\href{https://typst.app/universe/package/miage-rapide-tp/0.1.1/}{0.1.1}
& September 14, 2024 \\
\href{https://typst.app/universe/package/miage-rapide-tp/0.1.0/}{0.1.0}
& September 11, 2024 \\
\end{longtable}

Typst GmbH did not create this template and cannot guarantee correct
functionality of this template or compatibility with any version of the
Typst compiler or app.


\section{Package List LaTeX/algorithmic.tex}
\title{typst.app/universe/package/algorithmic}

\phantomsection\label{banner}
\section{algorithmic}\label{algorithmic}

{ 0.1.0 }

Algorithm pseudocode typesetting for Typst, inspired by algorithmicx in
LaTeX

\phantomsection\label{readme}
This is a package inspired by the LaTeX
\href{https://ctan.org/pkg/algorithmicx}{\texttt{\ algorithmicx\ }}
package for Typst. It’s useful for writing pseudocode and typesetting
it all nicely.

\pandocbounded{\includegraphics[keepaspectratio]{https://github.com/typst/packages/raw/main/packages/preview/algorithmic/0.1.0/docs/assets/demo-rendered.png}}

Example:

\begin{Shaded}
\begin{Highlighting}[]
\NormalTok{\#import "@preview/algorithmic:0.1.0"}
\NormalTok{\#import algorithmic: algorithm}

\NormalTok{\#algorithm(\{}
\NormalTok{  import algorithmic: *}
\NormalTok{  Function("Binary{-}Search", args: ("A", "n", "v"), \{}
\NormalTok{    Cmt[Initialize the search range]}
\NormalTok{    Assign[$l$][$1$]}
\NormalTok{    Assign[$r$][$n$]}
\NormalTok{    State[]}
\NormalTok{    While(cond: $l \textless{}= r$, \{}
\NormalTok{      Assign([mid], FnI[floor][$(l + r)/2$])}
\NormalTok{      If(cond: $A ["mid"] \textless{} v$, \{}
\NormalTok{        Assign[$l$][$m + 1$]}
\NormalTok{      \})}
\NormalTok{      ElsIf(cond: [$A ["mid"] \textgreater{} v$], \{}
\NormalTok{        Assign[$r$][$m {-} 1$]}
\NormalTok{      \})}
\NormalTok{      Else(\{}
\NormalTok{        Return[$m$]}
\NormalTok{      \})}
\NormalTok{    \})}
\NormalTok{    Return[*null*]}
\NormalTok{  \})}
\NormalTok{\})}
\end{Highlighting}
\end{Shaded}

This DSL is implemented using the same trick as
\href{https://github.com/johannes-wolf/typst-canvas}{CeTZ} uses: a code
block of arrays gets those arrays joined together.

Currently this library is not really customizable. Please vendor it and
hack it up as needed then file an issue for the customization option
you’re missing.

\subsection{Reference}\label{reference}

\paragraph{stmt}\label{stmt}

Statement-level contexts in \texttt{\ algorithmic\ } generally accept
the type \texttt{\ body\ } in the following:

\begin{verbatim}
body = (ast|content)[] | ast | content
ast = (change_indent: int, body: body)
\end{verbatim}

\paragraph{inline}\label{inline}

Inline functions will generate plain content.

\paragraph{\texorpdfstring{\texttt{\ algorithmic(..bits)\ }}{ algorithmic(..bits) }}\label{algorithmic..bits}

Takes one or more lists of \texttt{\ ast\ } and creates an algorithmic
block with line numbers.

\subsubsection{Control flow}\label{control-flow}

\paragraph{\texorpdfstring{\texttt{\ Function\ } /
\texttt{\ Procedure\ }
(stmt)}{ Function  /  Procedure  (stmt)}}\label{function-procedure-stmt}

Defined as
\texttt{\ f(name:\ string\textbar{}content,\ args:\ content{[}{]}?,\ ..body)\ }
. Body can be one or more \texttt{\ body\ } values.

\paragraph{\texorpdfstring{\texttt{\ If\ } / \texttt{\ ElseIf\ } /
\texttt{\ Else\ } / \texttt{\ For\ } / \texttt{\ While\ }
(stmt)}{ If  /  ElseIf  /  Else  /  For  /  While  (stmt)}}\label{if-elseif-else-for-while-stmt}

Defined as \texttt{\ f(cond:\ string\textbar{}content,\ ..body)\ } .
Body can be one or more \texttt{\ body\ } values.

Generates an indented block with the body, and the specified
\texttt{\ cond\ } between the two keywords as condition.

\subsubsection{Statements}\label{statements}

\paragraph{\texorpdfstring{\texttt{\ Assign\ }
(stmt)}{ Assign  (stmt)}}\label{assign-stmt}

Defined as \texttt{\ Assign(var:\ content,\ val:\ content)\ } .

Generates \texttt{\ \#var\ \textless{}-\ \#val\ } .

\paragraph{\texorpdfstring{\texttt{\ CallI\ } (inline),
\texttt{\ Call\ }
(stmt)}{ CallI  (inline),  Call  (stmt)}}\label{calli-inline-call-stmt}

Defined as \texttt{\ f(name,\ args:\ content\textbar{}content{[}{]})\ }
.

Calls a function with the function name styled in smallcaps and the args
joined by commas.

\paragraph{\texorpdfstring{\texttt{\ Cmt\ }
(stmt)}{ Cmt  (stmt)}}\label{cmt-stmt}

Defined as \texttt{\ Cmt(body:\ content)\ } .

Makes a line comment.

\paragraph{\texorpdfstring{\texttt{\ FnI\ } (inline), \texttt{\ Fn\ }
(stmt)}{ FnI  (inline),  Fn  (stmt)}}\label{fni-inline-fn-stmt}

Defined as \texttt{\ f(name,\ args:\ content\textbar{}content{[}{]})\ }
.

Calls a function with the function name styled in bold and the args
joined by commas.

\paragraph{\texorpdfstring{\texttt{\ Ic\ }
(inline)}{ Ic  (inline)}}\label{ic-inline}

Defined as \texttt{\ Ic(body:\ content)\ -\textgreater{}\ content\ } .

Makes an inline comment.

\paragraph{\texorpdfstring{\texttt{\ Return\ }
(stmt)}{ Return  (stmt)}}\label{return-stmt}

Defined as \texttt{\ Return(arg:\ content)\ } .

Generates \texttt{\ return\ \#arg\ } .

\paragraph{\texorpdfstring{\texttt{\ State\ }
(stmt)}{ State  (stmt)}}\label{state-stmt}

Defined as \texttt{\ State(body:\ content)\ } .

Turns any content into a line.

\subsubsection{How to add}\label{how-to-add}

Copy this into your project and use the import as
\texttt{\ algorithmic\ }

\begin{verbatim}
#import "@preview/algorithmic:0.1.0"
\end{verbatim}

\includesvg[width=0.16667in,height=0.16667in]{/assets/icons/16-copy.svg}

Check the docs for
\href{https://typst.app/docs/reference/scripting/\#packages}{more
information on how to import packages} .

\subsubsection{About}\label{about}

\begin{description}
\tightlist
\item[Author :]
Jade Lovelace
\item[License:]
MIT
\item[Current version:]
0.1.0
\item[Last updated:]
August 19, 2023
\item[First released:]
August 19, 2023
\item[Archive size:]
3.29 kB
\href{https://packages.typst.org/preview/algorithmic-0.1.0.tar.gz}{\pandocbounded{\includesvg[keepaspectratio]{/assets/icons/16-download.svg}}}
\item[Repository:]
\href{https://github.com/lf-/typst-algorithmic}{GitHub}
\end{description}

\subsubsection{Where to report issues?}\label{where-to-report-issues}

This package is a project of Jade Lovelace . Report issues on
\href{https://github.com/lf-/typst-algorithmic}{their repository} . You
can also try to ask for help with this package on the
\href{https://forum.typst.app}{Forum} .

Please report this package to the Typst team using the
\href{https://typst.app/contact}{contact form} if you believe it is a
safety hazard or infringes upon your rights.

\phantomsection\label{versions}
\subsubsection{Version history}\label{version-history}

\begin{longtable}[]{@{}ll@{}}
\toprule\noalign{}
Version & Release Date \\
\midrule\noalign{}
\endhead
\bottomrule\noalign{}
\endlastfoot
0.1.0 & August 19, 2023 \\
\end{longtable}

Typst GmbH did not create this package and cannot guarantee correct
functionality of this package or compatibility with any version of the
Typst compiler or app.


\section{Package List LaTeX/lemmify.tex}
\title{typst.app/universe/package/lemmify}

\phantomsection\label{banner}
\section{lemmify}\label{lemmify}

{ 0.1.6 }

Theorem typesetting library.

\phantomsection\label{readme}
Lemmify is a library for typesetting mathematical theorems in typst. It
aims to be easy to use while trying to be as flexible and idiomatic as
possible. This means that the interface might change with updates to
typst (for example if user-defined element functions are introduced).
But no functionality should be lost.

\subsection{Basic Usage}\label{basic-usage}

To get started with Lemmify, follow these steps:

\begin{enumerate}
\tightlist
\item
  Import the Lemmify library:
\end{enumerate}

\begin{Shaded}
\begin{Highlighting}[]
\NormalTok{\#import "@preview/lemmify:0.1.6": *}
\end{Highlighting}
\end{Shaded}

\begin{enumerate}
\setcounter{enumi}{1}
\tightlist
\item
  Define the default styling for a few default theorem types:
\end{enumerate}

\begin{Shaded}
\begin{Highlighting}[]
\NormalTok{\#let (}
\NormalTok{  theorem, lemma, corollary,}
\NormalTok{  remark, proposition, example,}
\NormalTok{  proof, rules: thm{-}rules}
\NormalTok{) = default{-}theorems("thm{-}group", lang: "en")}
\end{Highlighting}
\end{Shaded}

\begin{enumerate}
\setcounter{enumi}{2}
\tightlist
\item
  Apply the generated styling:
\end{enumerate}

\begin{Shaded}
\begin{Highlighting}[]
\NormalTok{\#show: thm{-}rules}
\end{Highlighting}
\end{Shaded}

\begin{enumerate}
\setcounter{enumi}{3}
\tightlist
\item
  Create theorems, lemmas, and proofs using the defined styling:
\end{enumerate}

\begin{Shaded}
\begin{Highlighting}[]
\NormalTok{\#theorem(name: "Some theorem")[}
\NormalTok{  Theorem content goes here.}
\NormalTok{]\textless{}thm\textgreater{}}

\NormalTok{\#proof[}
\NormalTok{  Complicated proof.}
\NormalTok{]\textless{}proof\textgreater{}}

\NormalTok{@proof and @thm[theorem]}
\end{Highlighting}
\end{Shaded}

\begin{enumerate}
\setcounter{enumi}{4}
\tightlist
\item
  Customize the styling further using show rules. For example, to add a
  red box around proofs:
\end{enumerate}

\begin{Shaded}
\begin{Highlighting}[]
\NormalTok{\#show thm{-}selector("thm{-}group", subgroup: "proof"): it =\textgreater{} box(}
\NormalTok{  it,}
\NormalTok{  stroke: red + 1pt,}
\NormalTok{  inset: 1em}
\NormalTok{)}
\end{Highlighting}
\end{Shaded}

The result should now look something like this:

\pandocbounded{\includegraphics[keepaspectratio]{https://github.com/Marmare314/lemmify/assets/49279081/ba5e7ed4-336d-4b1a-8470-99fa23bf5119}}

\subsection{Useful examples}\label{useful-examples}

If you do not want to reset the theorem counter on headings you can use
the \texttt{\ max-reset-level\ } parameter:

\begin{Shaded}
\begin{Highlighting}[]
\NormalTok{default{-}theorems("thm{-}group", max{-}reset{-}level: 0)}
\end{Highlighting}
\end{Shaded}

It specifies the highest level at which the counter is reset. To
manually reset the counter you can use the
\texttt{\ thm-reset-counter\ } function.

\begin{center}\rule{0.5\linewidth}{0.5pt}\end{center}

By specifying \texttt{\ numbering:\ none\ } you can create unnumbered
theorems.

\begin{Shaded}
\begin{Highlighting}[]
\NormalTok{\#example(numbering: none)[}
\NormalTok{  Some example.}
\NormalTok{]}
\end{Highlighting}
\end{Shaded}

To make all examples unnumbered you could use the following code:

\begin{Shaded}
\begin{Highlighting}[]
\NormalTok{\#let example = example.with(numbering: none)}
\end{Highlighting}
\end{Shaded}

\begin{center}\rule{0.5\linewidth}{0.5pt}\end{center}

To create other types (or subgroups) of theorems you can use the
\texttt{\ new-theorems\ } function.

\begin{Shaded}
\begin{Highlighting}[]
\NormalTok{\#let (note, rules) = new{-}theorems("thm{-}group", ("note": text(red)[Note]))}
\NormalTok{\#show: rules}
\end{Highlighting}
\end{Shaded}

If you have already defined custom styling you will notice that the
newly created theorem does not use it. You can create a dictionary to
make applying it again easier.

\begin{Shaded}
\begin{Highlighting}[]
\NormalTok{\#let my{-}styling = (}
\NormalTok{  thm{-}styling: thm{-}styling{-}simple,}
\NormalTok{  thm{-}numbering: ..,}
\NormalTok{  ref{-}styling: ..}
\NormalTok{)}

\NormalTok{\#let (note, rules) = new{-}theorems("thm{-}group", ("note": "Note), ..my{-}styling)}
\end{Highlighting}
\end{Shaded}

\begin{center}\rule{0.5\linewidth}{0.5pt}\end{center}

By varying the \texttt{\ group\ } parameter you can create independently
numbered theorems:

\begin{Shaded}
\begin{Highlighting}[]
\NormalTok{\#let (}
\NormalTok{  theorem, proof,}
\NormalTok{  rules: thm{-}rules{-}a}
\NormalTok{) = default{-}theorems("thm{-}group{-}a")}
\NormalTok{\#let (}
\NormalTok{  definition,}
\NormalTok{  rules: thm{-}rules{-}b}
\NormalTok{) = default{-}theorems("thm{-}group{-}b")}

\NormalTok{\#show: thm{-}rules{-}a}
\NormalTok{\#show: thm{-}rules{-}b}
\end{Highlighting}
\end{Shaded}

\begin{center}\rule{0.5\linewidth}{0.5pt}\end{center}

To specify parameters of the
\href{https://github.com/typst/packages/raw/main/packages/preview/lemmify/0.1.6/\#styling-parameters}{styling}
functions the \texttt{\ .with\ } function is used.

\begin{Shaded}
\begin{Highlighting}[]
\NormalTok{\#let (}
\NormalTok{  theorem,}
\NormalTok{  rules: thm{-}rules}
\NormalTok{) = default{-}theorems(}
\NormalTok{  "thm{-}group",}
\NormalTok{  thm{-}numbering: thm{-}numbering{-}heading.with(max{-}heading{-}level: 2)}
\NormalTok{)}
\end{Highlighting}
\end{Shaded}

\subsection{Example}\label{example}

\begin{Shaded}
\begin{Highlighting}[]
\NormalTok{\#import "@preview/lemmify:0.1.6": *}

\NormalTok{\#let my{-}thm{-}style(}
\NormalTok{  thm{-}type, name, number, body}
\NormalTok{) = grid(}
\NormalTok{  columns: (1fr, 3fr),}
\NormalTok{  column{-}gutter: 1em,}
\NormalTok{  stack(spacing: .5em, strong(thm{-}type), number, emph(name)),}
\NormalTok{  body}
\NormalTok{)}

\NormalTok{\#let my{-}styling = (}
\NormalTok{  thm{-}styling: my{-}thm{-}style}
\NormalTok{)}

\NormalTok{\#let (}
\NormalTok{  theorem, rules}
\NormalTok{) = default{-}theorems("thm{-}group", lang: "en", ..my{-}styling)}
\NormalTok{\#show: rules}
\NormalTok{\#show thm{-}selector("thm{-}group"): box.with(inset: 1em)}

\NormalTok{\#lorem(20)}
\NormalTok{\#theorem[}
\NormalTok{  \#lorem(40)}
\NormalTok{]}
\NormalTok{\#lorem(20)}
\NormalTok{\#theorem(name: "Some theorem")[}
\NormalTok{  \#lorem(30)}
\NormalTok{]}
\end{Highlighting}
\end{Shaded}

\pandocbounded{\includegraphics[keepaspectratio]{https://github.com/Marmare314/lemmify/assets/49279081/b3c72b3e-7e21-4acd-82bb-3d63f87ec84b}}

\subsection{Documentation}\label{documentation}

The two most important functions are:

\texttt{\ default-theorems\ } : Create a default set of theorems based
on the given language and styling.

\begin{itemize}
\tightlist
\item
  \texttt{\ group\ } : The group id.
\item
  \texttt{\ lang\ } : The language to which the theorems are adapted.
\item
  \texttt{\ thm-styling\ } , \texttt{\ thm-numbering\ } ,
  \texttt{\ ref-styling\ } : Styling parameters are explained in further
  detail in the
  \href{https://github.com/typst/packages/raw/main/packages/preview/lemmify/0.1.6/\#styling-parameters}{Styling}
  section.
\item
  \texttt{\ proof-styling\ } : Styling which is only applied to proofs.
\item
  \texttt{\ max-reset-level\ } : The highest heading level on which
  theorems are still reset.
\end{itemize}

\texttt{\ new-theorems\ } : Create custom sets of theorems with the
given styling.

\begin{itemize}
\tightlist
\item
  \texttt{\ group\ } : The group id.
\item
  \texttt{\ subgroup-map\ } : Mapping from group id to some argument.
  The simple styles use \texttt{\ thm-type\ } as the argument (ie
  “Beispiel� or “Example� for group id “example�)
\item
  \texttt{\ thm-styling\ } , \texttt{\ thm-numbering\ } ,
  \texttt{\ ref-styling\ } , \texttt{\ ref-numbering\ } : Styling which
  to apply to all subgroups.
\end{itemize}

\begin{center}\rule{0.5\linewidth}{0.5pt}\end{center}

\texttt{\ use-proof-numbering\ } : Decreases the numbering of a theorem
function by one. See
\href{https://github.com/typst/packages/raw/main/packages/preview/lemmify/0.1.6/\#styling}{Styling}
for more information.

\begin{center}\rule{0.5\linewidth}{0.5pt}\end{center}

\texttt{\ thm-selector\ } : Returns a selector for all theorems of the
specified group. If subgroup is specified, only the theorems belonging
to it will be selected.

\begin{center}\rule{0.5\linewidth}{0.5pt}\end{center}

There are also a few functions to help with resetting counters.

\texttt{\ thm-reset-counter\ } : Reset theorem group counter manually.
Returned content needs to added to the document.

\texttt{\ thm-reset-counter-heading-at\ } : Reset theorem group counter
at headings of the specified level. Returns a rule that needs to be
shown.

\texttt{\ thm-reset-counter-heading\ } : Reset theorem group counter at
headings of at most the specified level. Returns a rule that needs to be
shown.

\subsubsection{Styling parameters}\label{styling-parameters}

If possible the best way to adapt the look of theorems is to use show
rules as shown
\href{https://github.com/typst/packages/raw/main/packages/preview/lemmify/0.1.6/\#basic-usage}{above}
, but this is not always possible. For example if we wanted theorems to
start with \texttt{\ 1.1\ Theorem\ } instead of
\texttt{\ Theorem\ 1.1\ } . You can provide the following functions to
adapt the look of the theorems.

\begin{center}\rule{0.5\linewidth}{0.5pt}\end{center}

\texttt{\ thm-styling\ } : A function:
\texttt{\ (arg,\ name,\ number,\ body)\ -\textgreater{}\ content\ } ,
that allows you to define the styling for different types of theorems.
Below only the \texttt{\ arg\ } will be specified.

Pre-defined functions

\begin{itemize}
\tightlist
\item
  \texttt{\ thm-style-simple(thm-type)\ } : \textbf{thm-type num}
  \emph{(name)} body
\item
  \texttt{\ thm-style-proof(thm-type)\ } : \textbf{thm-type num}
  \emph{(name)} body â--¡
\item
  \texttt{\ thm-style-reversed(thm-type)\ } : \textbf{num thm-type}
  \emph{(name)} body
\end{itemize}

\begin{center}\rule{0.5\linewidth}{0.5pt}\end{center}

\texttt{\ thm-numbering\ } : A function:
\texttt{\ figure\ -\textgreater{}\ content\ } , that determines how
theorems are numbered.

Pre-defined functions: (Assume heading is 1.1 and theorem count is 2)

\begin{itemize}
\tightlist
\item
  \texttt{\ thm-numbering-heading\ } : 1.1.2

  \begin{itemize}
  \tightlist
  \item
    \texttt{\ max-heading-level\ } : only use the a limited number of
    subheadings. In this case if it is set to \texttt{\ 1\ } the result
    would be \texttt{\ 1.2\ } instead.
  \end{itemize}
\item
  \texttt{\ thm-numbering-linear\ } : 2
\item
  \texttt{\ thm-numbering-proof\ } : No visible content is returned, but
  the counter is reduced by 1 (so that the proof keeps the same count as
  the theorem). Useful in combination with
  \texttt{\ use-proof-numbering\ } to create theorems that reference the
  previous theorem (like proofs).
\end{itemize}

\begin{center}\rule{0.5\linewidth}{0.5pt}\end{center}

\texttt{\ ref-styling\ } : A function:
\texttt{\ (arg,\ thm-numbering,\ ref)\ -\textgreater{}\ content\ } , to
style theorem references.

Pre-defined functions:

\begin{itemize}
\tightlist
\item
  \texttt{\ thm-ref-style-simple(thm-type)\ }

  \begin{itemize}
  \tightlist
  \item
    \texttt{\ @thm\ -\textgreater{}\ thm-type\ 1.1\ }
  \item
    \texttt{\ @thm{[}custom{]}\ -\textgreater{}\ custom\ 1.1\ }
  \end{itemize}
\end{itemize}

\begin{center}\rule{0.5\linewidth}{0.5pt}\end{center}

\texttt{\ ref-numbering\ } : Same as \texttt{\ thm-numbering\ } but only
applies to the references.

\subsection{Roadmap}\label{roadmap}

\begin{itemize}
\tightlist
\item
  More pre-defined styles.

  \begin{itemize}
  \tightlist
  \item
    Referencing theorems by name.
  \end{itemize}
\item
  Support more languages.
\item
  Better documentation.
\item
  Outlining theorems.
\end{itemize}

If you are encountering any bugs, have questions or are missing
features, feel free to open an issue on
\href{https://github.com/Marmare314/lemmify}{Github} .

\subsection{Changelog}\label{changelog}

\begin{itemize}
\item
  Version 0.1.6

  \begin{itemize}
  \tightlist
  \item
    Add Portuguese translation (
    \href{https://github.com/PgBiel}{@PgBiel} )
  \item
    Add Catalan translation (
    \href{https://github.com/Eloitor}{@Eloitor} )
  \item
    Add Spanish translation (
    \href{https://github.com/mismorgano}{@mismorgano} )
  \item
    Remove extra space before empty supplements (
    \href{https://github.com/PgBiel}{@PgBiel} )
  \item
    Use ref-styling parameter of default-theorems
  \end{itemize}
\item
  Version 0.1.5

  \begin{itemize}
  \tightlist
  \item
    Add Russian translation (
    \href{https://github.com/WeetHet}{@WeetHet} )
  \end{itemize}
\item
  Version 0.1.4

  \begin{itemize}
  \tightlist
  \item
    Fix error on unnamed theorems
  \end{itemize}
\item
  Version 0.1.3

  \begin{itemize}
  \tightlist
  \item
    Allow “1.1.� numbering style by default
  \item
    Ignore unnumbered subheadings
  \item
    Add max-heading-level parameter to thm-numbering-heading
  \item
    Adapt lemmify to typst version 0.8.0
  \end{itemize}
\item
  Version 0.1.2

  \begin{itemize}
  \tightlist
  \item
    Better error message on unnumbered headings
  \item
    Add Italian translations (
    \href{https://github.com/porcaror}{@porcaror} )
  \end{itemize}
\item
  Version 0.1.1

  \begin{itemize}
  \tightlist
  \item
    Add Dutch translations (
    \href{https://github.com/BroodjeKroepoek}{@BroodjeKroepoek} )
  \item
    Add French translations ( \href{https://github.com/MDLC01}{@MDLC01}
    )
  \item
    Fix size of default styles and make them breakable
  \end{itemize}
\end{itemize}

\subsubsection{How to add}\label{how-to-add}

Copy this into your project and use the import as \texttt{\ lemmify\ }

\begin{verbatim}
#import "@preview/lemmify:0.1.6"
\end{verbatim}

\includesvg[width=0.16667in,height=0.16667in]{/assets/icons/16-copy.svg}

Check the docs for
\href{https://typst.app/docs/reference/scripting/\#packages}{more
information on how to import packages} .

\subsubsection{About}\label{about}

\begin{description}
\tightlist
\item[Author :]
Marmare314
\item[License:]
GPL-3.0-only
\item[Current version:]
0.1.6
\item[Last updated:]
August 29, 2024
\item[First released:]
July 2, 2023
\item[Archive size:]
18.2 kB
\href{https://packages.typst.org/preview/lemmify-0.1.6.tar.gz}{\pandocbounded{\includesvg[keepaspectratio]{/assets/icons/16-download.svg}}}
\item[Repository:]
\href{https://github.com/Marmare314/lemmify}{GitHub}
\end{description}

\subsubsection{Where to report issues?}\label{where-to-report-issues}

This package is a project of Marmare314 . Report issues on
\href{https://github.com/Marmare314/lemmify}{their repository} . You can
also try to ask for help with this package on the
\href{https://forum.typst.app}{Forum} .

Please report this package to the Typst team using the
\href{https://typst.app/contact}{contact form} if you believe it is a
safety hazard or infringes upon your rights.

\phantomsection\label{versions}
\subsubsection{Version history}\label{version-history}

\begin{longtable}[]{@{}ll@{}}
\toprule\noalign{}
Version & Release Date \\
\midrule\noalign{}
\endhead
\bottomrule\noalign{}
\endlastfoot
0.1.6 & August 29, 2024 \\
\href{https://typst.app/universe/package/lemmify/0.1.5/}{0.1.5} &
December 3, 2023 \\
\href{https://typst.app/universe/package/lemmify/0.1.4/}{0.1.4} &
September 26, 2023 \\
\href{https://typst.app/universe/package/lemmify/0.1.3/}{0.1.3} &
September 22, 2023 \\
\href{https://typst.app/universe/package/lemmify/0.1.2/}{0.1.2} & July
24, 2023 \\
\href{https://typst.app/universe/package/lemmify/0.1.1/}{0.1.1} & July
8, 2023 \\
\href{https://typst.app/universe/package/lemmify/0.1.0/}{0.1.0} & July
2, 2023 \\
\end{longtable}

Typst GmbH did not create this package and cannot guarantee correct
functionality of this package or compatibility with any version of the
Typst compiler or app.


\section{Package List LaTeX/nulite.tex}
\title{typst.app/universe/package/nulite}

\phantomsection\label{banner}
\section{nulite}\label{nulite}

{ 0.1.0 }

Generate charts with vegalite.

\phantomsection\label{readme}
A typst plugin to generate charts using
\href{https://vega.github.io/vega-lite/}{vegalite}

\subsection{Usage}\label{usage}

\begin{Shaded}
\begin{Highlighting}[]
\NormalTok{\#import "@preview/nulite:0.1.0" as vegalite}

\NormalTok{\#vegalite.render(}
\NormalTok{  width: 100\%,}
\NormalTok{  height: 100\%,}
\NormalTok{  zoom: 1,}
\NormalTok{  json("spec.json")}
\NormalTok{  )}
\end{Highlighting}
\end{Shaded}

\pandocbounded{\includegraphics[keepaspectratio]{https://github.com/typst/packages/raw/main/packages/preview/nulite/0.1.0/examples/image.png}}

The module exports a single function, \texttt{\ render\ } with four
arguments

\begin{itemize}
\tightlist
\item
  \texttt{\ width\ } : Width of the chart in percent of the
  container’s width
\item
  \texttt{\ height\ } : Height of the chart in percent of the
  container’s height
\item
  \texttt{\ zoom\ } : Zoom factor applied to the SVG. This mainly
  affects the sizing of text in relation to the graphical elements.
\item
  \texttt{\ spec\ } :
  \href{https://vega.github.io/vega-lite/docs/spec.html}{Vegalite
  specification}
\end{itemize}

\subsection{Compatibility}\label{compatibility}

This plugin uses vegalite v5.21 and vega v5.30.

The following features of vegalite are \textbf{not supported} :

\begin{itemize}
\tightlist
\item
  Setting \texttt{\ width\ } and \texttt{\ height\ } in the spec. These
  values should be provided as arguments to \texttt{\ render\ } . If
  \texttt{\ width\ } or \texttt{\ height\ } are included in the spec
  then they will be ignored.
\item
  Loading data with the \texttt{\ url\ } property. Attempting to do this
  will result in an error while trying to compile the \texttt{\ typst\ }
  document. All data should be provided as part of the spec itself
  (inline).
\item
  Interactive charts and tooltips.
\end{itemize}

\subsubsection{How to add}\label{how-to-add}

Copy this into your project and use the import as \texttt{\ nulite\ }

\begin{verbatim}
#import "@preview/nulite:0.1.0"
\end{verbatim}

\includesvg[width=0.16667in,height=0.16667in]{/assets/icons/16-copy.svg}

Check the docs for
\href{https://typst.app/docs/reference/scripting/\#packages}{more
information on how to import packages} .

\subsubsection{About}\label{about}

\begin{description}
\tightlist
\item[Author :]
j-mueller
\item[License:]
MIT
\item[Current version:]
0.1.0
\item[Last updated:]
September 30, 2024
\item[First released:]
September 30, 2024
\item[Minimum Typst version:]
0.11.1
\item[Archive size:]
686 kB
\href{https://packages.typst.org/preview/nulite-0.1.0.tar.gz}{\pandocbounded{\includesvg[keepaspectratio]{/assets/icons/16-download.svg}}}
\item[Repository:]
\href{https://github.com/j-mueller/typst-vegalite}{GitHub}
\item[Discipline :]
\begin{itemize}
\tightlist
\item[]
\item
  \href{https://typst.app/universe/search/?discipline=mathematics}{Mathematics}
\end{itemize}
\item[Categor ies :]
\begin{itemize}
\tightlist
\item[]
\item
  \pandocbounded{\includesvg[keepaspectratio]{/assets/icons/16-chart.svg}}
  \href{https://typst.app/universe/search/?category=visualization}{Visualization}
\item
  \pandocbounded{\includesvg[keepaspectratio]{/assets/icons/16-integration.svg}}
  \href{https://typst.app/universe/search/?category=integration}{Integration}
\end{itemize}
\end{description}

\subsubsection{Where to report issues?}\label{where-to-report-issues}

This package is a project of j-mueller . Report issues on
\href{https://github.com/j-mueller/typst-vegalite}{their repository} .
You can also try to ask for help with this package on the
\href{https://forum.typst.app}{Forum} .

Please report this package to the Typst team using the
\href{https://typst.app/contact}{contact form} if you believe it is a
safety hazard or infringes upon your rights.

\phantomsection\label{versions}
\subsubsection{Version history}\label{version-history}

\begin{longtable}[]{@{}ll@{}}
\toprule\noalign{}
Version & Release Date \\
\midrule\noalign{}
\endhead
\bottomrule\noalign{}
\endlastfoot
0.1.0 & September 30, 2024 \\
\end{longtable}

Typst GmbH did not create this package and cannot guarantee correct
functionality of this package or compatibility with any version of the
Typst compiler or app.


\section{Package List LaTeX/acrostiche.tex}
\title{typst.app/universe/package/acrostiche}

\phantomsection\label{banner}
\section{acrostiche}\label{acrostiche}

{ 0.4.1 }

Manage acronyms and their definitions in Typst.

\phantomsection\label{readme}
Manages acronyms so you don’t have to.

\subsection{Quick Start}\label{quick-start}

\begin{verbatim}
#import "@preview/acrostiche:0.4.0": *

#init-acronyms((
  "WTP": ("Wonderful Typst Package","Wonderful Typst Packages"),
))

Acrostiche is a #acr("WTP")! This #acr("WTP") enables easy acronyms manipulation.

Its main features are auto-expansion of the first occurence, global or selective expansion reset #reset-all-acronyms(), implicit or manual plural form support (there may be multiple #acrpl("WTP")), and customizable index printing. Have Fun!
\end{verbatim}

\subsection{Usage}\label{usage}

The main goal of Acrostiche is to keep track of which acronyms to
define.

\subsubsection{Define acronyms}\label{define-acronyms}

All acronyms used with Acrostiche must be defined in a dictionary passed
to the \texttt{\ init-acronyms\ } function. There are two possible forms
for the definition, depending on the required features.

\paragraph{Simple Definitions}\label{simple-definitions}

For a quick and easy definion, you can use the acronym to display as the
key and an array of one or more strings as the singular and plural
versions of the expanded meaning of the acronym.

\begin{verbatim}
#init-acronyms((
  "SDA": ("Simply Defined Acronym","Simply Defined Acronyms"),
  "ASDA": ("Another Simply Defined Acronym","Another Simply Defined Acronyms"),
))
\end{verbatim}

If there is only a singular version of the definition, the array
contains only one value. If there are both singular and plural versions,
define the definition as an array where the first item is the singular
definition and the second item is the plural.

\paragraph{Advanced Definitions}\label{advanced-definitions}

If you find yourself needing more flexibility when defining the
acronyms, you can pass a dictionary for each acronym. The expected keys
are: \texttt{\ short\ } the singular short form to display,
\texttt{\ short-pl\ } the plural short form, \texttt{\ long\ } singular
long (expanded) form to display, and \texttt{\ long-pl\ } the plural
long form. The main benefit of this definition is to use a separate key
for calling the acronym, useful when acronyms are long and tedious to
write.

\begin{verbatim}
#init-acronyms((
  "la": (
short: "LATATW",
long: "Long And Tedious Acronym To Write",
short-pl: "LATAsTW",
long-pl: "Long And Tedious Acronyms To Write"),
))
\end{verbatim}

Any other keys than these will be discarded.

\subsubsection{Call Acrostiche
functions}\label{call-acrostiche-functions}

Once the acronyms are defined, you can use them in the text with the
\texttt{\ \#acr(...)\ } function. The argument is the acronym as a
string (for example, “BIOS�). On the first call of the function, it
prints the acronym with its definition (for example, “Basic
Input/Output System (BIOS)�). On the next calls, it prints only the
acronym.

To get the plural version of the acronym, you can use the
\texttt{\ \#acrpl(...)\ } function that adds an ‘s’ after the
acronym. If a plural version of the definition is provided, it will be
used if the first use of the acronym is plural. Otherwise, the singular
version is used, and a trailing ‘s’ is added.

To intentionally print the full version of the acronym (definition +
acronym, as for the first instance), without affecting the state, you
can use the \texttt{\ \#acrfull(...)\ } function. For the plural
version, use the \texttt{\ \#acrfullpl(...)\ } function. Both functions
have shortcuts with \texttt{\ \#acrf(...)\ } and
\texttt{\ \#acrfpl(...)\ } .

At any point in the document, you can reset acronyms with the functions
\texttt{\ \#reset-acronym(...)\ } (for a single acronym) or
\texttt{\ reset-all-acronyms()\ } (to reset all acronyms). After a
reset, the next use of the acronym is expanded. Both functions have
shortcuts with \texttt{\ \#racr(...)\ } and \texttt{\ \#raacr(...)\ } .

You can also print an index of all acronyms used in the document with
the \texttt{\ \#print-index()\ } function. The index is printed as a
section for which you can choose the heading level, the numbering, and
the outline parameters (with respectively the \texttt{\ level:\ int\ } ,
\texttt{\ numbering:\ none\ \textbar{}\ string\ \textbar{}\ function\ }
, and \texttt{\ outlined:\ bool\ } parameters). You can also choose
their order with the \texttt{\ sorted:\ string\ } parameter that accepts
either an empty string (print in the order they are defined), “up�
(print in ascending alphabetical order), or “down� (print in
descending alphabetical order). By default, the index contains all the
acronyms you defined. You can choose to only display acronyms that are
actually used in the document by passing \texttt{\ used-only:\ true\ }
to the function. Warning, the detection of used acronym uses the states
at the end of the document. Thus, if you reset an acronym and do not use
it again until the end, it will not appear in the index. You can use the
\texttt{\ title:\ string\ } parameter to change the name of the heading
for the index section. The default value is “Acronyms Index�.
Passing an empty string for \texttt{\ title\ } results in the index
having no heading (i.e., no section for the index). You can customize
the string displayed after the acronym in the list with the
\texttt{\ delimiter:\ ":"\ } parameter. To adjust the spacing between
the acronyms adjust the
\texttt{\ row-gutter:\ auto\ \textbar{}\ int\ \textbar{}\ relative\ \textbar{}\ fraction\ \textbar{}\ array\ }
parameter, the default is \texttt{\ 2pt\ } .

Finally, you can call the \texttt{\ \#display-def(...)\ } function to
display the definition of an acronym. Set the \texttt{\ plural\ }
parameter to true to get the plural version.

\subsubsection{Functions Summary:}\label{functions-summary}

\begin{longtable}[]{@{}ll@{}}
\toprule\noalign{}
\textbf{Function} & \textbf{Description} \\
\midrule\noalign{}
\endhead
\bottomrule\noalign{}
\endlastfoot
\textbf{\#init-acronyms(…)} & Initializes the acronyms by defining
them in a dictionary where the keys are acronyms and the values are
definitions. \\
\textbf{\#acr(…)} & Prints the acronym with its definition on the
first call, then just the acronym in subsequent calls. \\
\textbf{\#acrpl(…)} & Prints the plural version of the acronym. Uses
plural definition if available, otherwise adds an ‘s’ to the
acronym. \\
\textbf{\#acrfull(…)} & Displays the full (long) version of the
acronym without affecting the state or tracking its usage. \\
\textbf{\#acrfullpl(…)} & Displays the full plural version of the
acronym without affecting the state or tracking its usage. \\
\textbf{\#reset-acronym(…)} & Resets a single acronym so the next
usage will include its definition again. \\
\textbf{\#reset-all-acronyms()} & Resets all acronyms so the next usage
will include their definitions again. \\
\textbf{\#print-index(…)} & Prints an index of all acronyms used, with
customizable heading level, order, and display parameters. \\
\textbf{\#display-def(…)} & Displays the definition of an acronym. Use
\texttt{\ plural:\ true\ } to display the plural version of the
definition. \\
\textbf{racr, raacr, acrf, acrfpl} & Shortcuts names for respectively
\texttt{\ reset-acronym\ } , \texttt{\ reset-all-acronyms\ } ,
\texttt{\ acrfull\ } , and \texttt{\ acrfullpl\ } . \\
\end{longtable}

\subsection{Advanced Definitions}\label{advanced-definitions-1}

This is a bit of a hacky feature coming from pure serendipity. There is
no enforcement of the type of the definitions. Most users would
naturally use strings as definitions, but any other content is
acceptable. For example, you set your definition to a content block with
rainbow-fille text, or even an image. The rainbow text is kinda cool
because the gradient depend on the position in the page so depending on
the position of first use the acronym will have a pseudo-random color.

If you use anything else than string for the definition, do not forget
the trailing comma to force the definition to be an array (an array of a
single element is not an array in Typst at the time of writing this). I
cannot guarantee that arbitrary content will remain available in future
versions but I will do my best to keep it as it is kinda cool. If you
find cool uses, please reach out to show me!

\begin{quote}
Yes it is posible to build nest/recursive acronyms definitions. If you
point to another acronym, it all works fine. If you point to the same
acronym, you obviously create a recursive situation, and it fails. It
will not converge, and the compiler will warn you and will panic. Be
nice to the compiler, don\textquotesingle t throw recursive traps.
\end{quote}

Here is a minimal working example of funky acronyms:

\begin{verbatim}
#import "@preview/acrostiche:0.4.0": *                                                           
#init-acronyms((
  "RFA": ([#text(fill: gradient.linear(..color.map.rainbow))[Rainbow Filled Acronym]],),                                                             
  "NA": ([Nested #acr("RFA") Acronym],)
))
#acr("NA")
\end{verbatim}

\subsection{Possible Errors:}\label{possible-errors}

\begin{itemize}
\tightlist
\item
  If an acronym is not defined, an error will tell you which one is
  causing the error. Simply add it to the dictionary or check the
  spelling.
\item
  \texttt{\ display-def\ } leverages the state \texttt{\ display\ }
  function and only works if the return value is actually printed in the
  document. For more information on states, see the Typst documentation
  on states.
\item
  Acrostiche uses a state named \texttt{\ acronyms\ } to keep track of
  the definitions and usage. If you redefined this state or use it
  manually in your document, unexpacted behaviour might happen.
\end{itemize}

Thank you to the contributors: \textbf{caemor} , \textbf{AurelWeinhold}
, \textbf{daniel-eder} , \textbf{iostapyshyn} .

If you notice any bug or want to contribute a new feature, please open
an issue or a merge request on the fork
\href{https://github.com/Grisely/packages}{Grisely/packages}

\subsubsection{How to add}\label{how-to-add}

Copy this into your project and use the import as
\texttt{\ acrostiche\ }

\begin{verbatim}
#import "@preview/acrostiche:0.4.1"
\end{verbatim}

\includesvg[width=0.16667in,height=0.16667in]{/assets/icons/16-copy.svg}

Check the docs for
\href{https://typst.app/docs/reference/scripting/\#packages}{more
information on how to import packages} .

\subsubsection{About}\label{about}

\begin{description}
\tightlist
\item[Author :]
Grizzly
\item[License:]
MIT
\item[Current version:]
0.4.1
\item[Last updated:]
November 21, 2024
\item[First released:]
July 6, 2023
\item[Archive size:]
6.52 kB
\href{https://packages.typst.org/preview/acrostiche-0.4.1.tar.gz}{\pandocbounded{\includesvg[keepaspectratio]{/assets/icons/16-download.svg}}}
\item[Repository:]
\href{https://github.com/Grisely/packages}{GitHub}
\item[Categor ies :]
\begin{itemize}
\tightlist
\item[]
\item
  \pandocbounded{\includesvg[keepaspectratio]{/assets/icons/16-hammer.svg}}
  \href{https://typst.app/universe/search/?category=utility}{Utility}
\item
  \pandocbounded{\includesvg[keepaspectratio]{/assets/icons/16-list-unordered.svg}}
  \href{https://typst.app/universe/search/?category=model}{Model}
\end{itemize}
\end{description}

\subsubsection{Where to report issues?}\label{where-to-report-issues}

This package is a project of Grizzly . Report issues on
\href{https://github.com/Grisely/packages}{their repository} . You can
also try to ask for help with this package on the
\href{https://forum.typst.app}{Forum} .

Please report this package to the Typst team using the
\href{https://typst.app/contact}{contact form} if you believe it is a
safety hazard or infringes upon your rights.

\phantomsection\label{versions}
\subsubsection{Version history}\label{version-history}

\begin{longtable}[]{@{}ll@{}}
\toprule\noalign{}
Version & Release Date \\
\midrule\noalign{}
\endhead
\bottomrule\noalign{}
\endlastfoot
0.4.1 & November 21, 2024 \\
\href{https://typst.app/universe/package/acrostiche/0.4.0/}{0.4.0} &
October 31, 2024 \\
\href{https://typst.app/universe/package/acrostiche/0.3.5/}{0.3.5} &
October 28, 2024 \\
\href{https://typst.app/universe/package/acrostiche/0.3.4/}{0.3.4} &
October 22, 2024 \\
\href{https://typst.app/universe/package/acrostiche/0.3.3/}{0.3.3} &
September 24, 2024 \\
\href{https://typst.app/universe/package/acrostiche/0.3.2/}{0.3.2} &
July 10, 2024 \\
\href{https://typst.app/universe/package/acrostiche/0.3.1/}{0.3.1} &
January 6, 2024 \\
\href{https://typst.app/universe/package/acrostiche/0.3.0/}{0.3.0} &
August 22, 2023 \\
\href{https://typst.app/universe/package/acrostiche/0.2.0/}{0.2.0} &
July 8, 2023 \\
\href{https://typst.app/universe/package/acrostiche/0.1.0/}{0.1.0} &
July 6, 2023 \\
\end{longtable}

Typst GmbH did not create this package and cannot guarantee correct
functionality of this package or compatibility with any version of the
Typst compiler or app.


\section{Package List LaTeX/haw-hamburg.tex}
\title{typst.app/universe/package/haw-hamburg}

\phantomsection\label{banner}
\section{haw-hamburg}\label{haw-hamburg}

{ 0.3.1 }

Unofficial template for writing a report or thesis in the HAW Hamburg
department of Computer Science design.

\phantomsection\label{readme}
This is an \textbf{\texttt{\ unofficial\ }} template for writing a
report or thesis in the \texttt{\ HAW\ Hamburg\ } department of
\texttt{\ Computer\ Science\ } design using
\href{https://github.com/typst/typst}{Typst} .

\subsection{Required Fonts}\label{required-fonts}

To correctly render this template please make sure that the
\texttt{\ New\ Computer\ Modern\ } font is installed on your system.

\subsection{Usage}\label{usage}

To use this package just add the following code to your
\href{https://github.com/typst/typst}{Typst} document:

\subsubsection{Report}\label{report}

\begin{Shaded}
\begin{Highlighting}[]
\NormalTok{\#import "@preview/haw{-}hamburg:0.3.0": report}

\NormalTok{\#show: report.with(}
\NormalTok{  language: "en",}
\NormalTok{  title: "Example title",}
\NormalTok{  author:"Example author",}
\NormalTok{  faculty: "Engineering and Computer Science",}
\NormalTok{  department: "Computer Science",}
\NormalTok{  include{-}declaration{-}of{-}independent{-}processing: true,}
\NormalTok{)}
\end{Highlighting}
\end{Shaded}

\subsubsection{Bachelor Thesis}\label{bachelor-thesis}

\begin{Shaded}
\begin{Highlighting}[]
\NormalTok{\#import "@preview/haw{-}hamburg:0.3.0": bachelor{-}thesis}

\NormalTok{\#show: bachelor{-}thesis.with(}
\NormalTok{  language: "en",}

\NormalTok{  title{-}de: "Beispiel Titel",}
\NormalTok{  keywords{-}de: ("Stichwort", "Wichtig", "Super"),}
\NormalTok{  abstract{-}de: "Beispiel Zusammenfassung",}

\NormalTok{  title{-}en: "Example title",}
\NormalTok{  keywords{-}en:  ("Keyword", "Important", "Super"),}
\NormalTok{  abstract{-}en: "Example abstract",}

\NormalTok{  author: "Example author",}
\NormalTok{  faculty: "Engineering and Computer Science",}
\NormalTok{  department: "Computer Science",}
\NormalTok{  study{-}course: "Bachelor of Science Informatik Technischer Systeme",}
\NormalTok{  supervisors: ("Prof. Dr. Example", "Prof. Dr. Example"),}
\NormalTok{  submission{-}date: datetime(year: 1948, month: 12, day: 10),}
\NormalTok{  include{-}declaration{-}of{-}independent{-}processing: true,}
\NormalTok{)}
\end{Highlighting}
\end{Shaded}

\subsubsection{Master Thesis}\label{master-thesis}

\begin{Shaded}
\begin{Highlighting}[]
\NormalTok{\#import "@preview/haw{-}hamburg:0.3.0": master{-}thesis}

\NormalTok{\#show: master{-}thesis.with(}
\NormalTok{  language: "en",}

\NormalTok{  title{-}de: "Beispiel Titel",}
\NormalTok{  keywords{-}de: ("Stichwort", "Wichtig", "Super"),}
\NormalTok{  abstract{-}de: "Beispiel Zusammenfassung",}

\NormalTok{  title{-}en: "Example title",}
\NormalTok{  keywords{-}en:  ("Keyword", "Important", "Super"),}
\NormalTok{  abstract{-}en: "Example abstract",}

\NormalTok{  author: "The Computer",}
\NormalTok{  faculty: "Engineering and Computer Science",}
\NormalTok{  department: "Computer Science",}
\NormalTok{  study{-}course: "Master of Science Computer Science",}
\NormalTok{  supervisors: ("Prof. Dr. Example", "Prof. Dr. Example"),}
\NormalTok{  submission{-}date: datetime(year: 1948, month: 12, day: 10),}
\NormalTok{  include{-}declaration{-}of{-}independent{-}processing: true,}
\NormalTok{)}
\end{Highlighting}
\end{Shaded}

\subsection{How to Compile}\label{how-to-compile}

This project contains an example setup that splits individual chapters
into different files.\\
This can cause problems when using references etc.\\
These problems can be avoided by following these steps:

\begin{itemize}
\tightlist
\item
  Make sure to always compile your \texttt{\ main.typ\ } file which
  includes all of your chapters for references to work correctly.
\item
  VSCode:

  \begin{itemize}
  \tightlist
  \item
    Install the
    \href{https://marketplace.visualstudio.com/items?itemName=myriad-dreamin.tinymist}{Tinymist
    Typst} extension.
  \item
    Make sure to start the \texttt{\ PDF\ } or
    \texttt{\ Live\ Preview\ } only from within your
    \texttt{\ main.typ\ } file.
  \item
    If problems occur it usually helps to close the preview and restart
    it from your \texttt{\ main.typ\ } file.
  \end{itemize}
\end{itemize}

\subsection{Examples}\label{examples}

Examples can be found inside of the
\href{https://github.com/LasseRosenow/HAW-Hamburg-Typst-Template/tree/main/examples}{examples}
directory

\begin{itemize}
\tightlist
\item
  For Bachelor theses see:
  \href{https://github.com/LasseRosenow/HAW-Hamburg-Typst-Template/tree/main/examples/bachelor-thesis}{Bachelor
  thesis example}
\item
  For Master theses see:
  \href{https://github.com/LasseRosenow/HAW-Hamburg-Typst-Template/tree/main/examples/master-thesis}{Master
  thesis example}
\item
  For reports see:
  \href{https://github.com/LasseRosenow/HAW-Hamburg-Typst-Template/tree/main/examples/report}{Report
  example}
\end{itemize}

\subsubsection{How to add}\label{how-to-add}

Copy this into your project and use the import as
\texttt{\ haw-hamburg\ }

\begin{verbatim}
#import "@preview/haw-hamburg:0.3.1"
\end{verbatim}

\includesvg[width=0.16667in,height=0.16667in]{/assets/icons/16-copy.svg}

Check the docs for
\href{https://typst.app/docs/reference/scripting/\#packages}{more
information on how to import packages} .

\subsubsection{About}\label{about}

\begin{description}
\tightlist
\item[Author :]
Lasse Rosenow
\item[License:]
MIT
\item[Current version:]
0.3.1
\item[Last updated:]
November 13, 2024
\item[First released:]
September 26, 2024
\item[Archive size:]
12.2 kB
\href{https://packages.typst.org/preview/haw-hamburg-0.3.1.tar.gz}{\pandocbounded{\includesvg[keepaspectratio]{/assets/icons/16-download.svg}}}
\item[Repository:]
\href{https://github.com/LasseRosenow/HAW-Hamburg-Typst-Template}{GitHub}
\item[Categor ies :]
\begin{itemize}
\tightlist
\item[]
\item
  \pandocbounded{\includesvg[keepaspectratio]{/assets/icons/16-speak.svg}}
  \href{https://typst.app/universe/search/?category=report}{Report}
\item
  \pandocbounded{\includesvg[keepaspectratio]{/assets/icons/16-mortarboard.svg}}
  \href{https://typst.app/universe/search/?category=thesis}{Thesis}
\end{itemize}
\end{description}

\subsubsection{Where to report issues?}\label{where-to-report-issues}

This package is a project of Lasse Rosenow . Report issues on
\href{https://github.com/LasseRosenow/HAW-Hamburg-Typst-Template}{their
repository} . You can also try to ask for help with this package on the
\href{https://forum.typst.app}{Forum} .

Please report this package to the Typst team using the
\href{https://typst.app/contact}{contact form} if you believe it is a
safety hazard or infringes upon your rights.

\phantomsection\label{versions}
\subsubsection{Version history}\label{version-history}

\begin{longtable}[]{@{}ll@{}}
\toprule\noalign{}
Version & Release Date \\
\midrule\noalign{}
\endhead
\bottomrule\noalign{}
\endlastfoot
0.3.1 & November 13, 2024 \\
\href{https://typst.app/universe/package/haw-hamburg/0.3.0/}{0.3.0} &
October 14, 2024 \\
\href{https://typst.app/universe/package/haw-hamburg/0.2.0/}{0.2.0} &
October 9, 2024 \\
\href{https://typst.app/universe/package/haw-hamburg/0.1.0/}{0.1.0} &
September 26, 2024 \\
\end{longtable}

Typst GmbH did not create this package and cannot guarantee correct
functionality of this package or compatibility with any version of the
Typst compiler or app.


\section{Package List LaTeX/light-cv.tex}
\title{typst.app/universe/package/light-cv}

\phantomsection\label{banner}
\phantomsection\label{template-thumbnail}
\pandocbounded{\includegraphics[keepaspectratio]{https://packages.typst.org/preview/thumbnails/light-cv-0.1.1-small.webp}}

\section{light-cv}\label{light-cv}

{ 0.1.1 }

Minimalistic CV template for your own CV. Please install the font
awesome fonts on your system before using the template.

\href{/app?template=light-cv&version=0.1.1}{Create project in app}

\phantomsection\label{readme}
This is my CV template written in Typst. You can find a two example CVs
in this repository as PDFs:

\begin{itemize}
\tightlist
\item
  \href{https://github.com/AnsgarLichter/light-cv/blob/main/cv-de.pdf}{German
  CV}
\item
  \href{https://github.com/AnsgarLichter/light-cv/blob/main/cv-en.pdf}{English
  CV}
\end{itemize}

\subsection{Setup}\label{setup}

To use the CV you have to install the font awesome fonts for the icons
to work. Please refer to the intstructons of the font awesome package
itself. You can find these on: -
\href{https://typst.app/universe/package/fontawesome}{Typst Universe} -
\href{https://github.com/duskmoon314/typst-fontawesome}{GitHub} .

\subsection{Functions}\label{functions}

\begin{enumerate}
\item
  \texttt{\ header\ } : Creates a page haeder including your name,
  current job title or any other sub title, socials and profile picture

  \begin{itemize}
  \tightlist
  \item
    \texttt{\ full-name\ } : your full name
  \item
    \texttt{\ job-title\ } : your current job title rendered below your
    name
  \item
    \texttt{\ socials\ } : array containing all socials. Every social
    must have the following properties: \texttt{\ icon\ } ,
    \texttt{\ link\ } and \texttt{\ text\ }
  \item
    \texttt{\ profile-picture\ } : path to your profile picture
  \end{itemize}
\item
  \texttt{\ section\ } : Creates a new section, e. g.
  \texttt{\ Professional\ Experience\ } or \texttt{\ Education\ }

  \begin{itemize}
  \tightlist
  \item
    \texttt{\ title\ } : section’s title
  \end{itemize}
\item
  \texttt{\ entry\ } : Adds an entry / item to the current section

  \begin{itemize}
  \tightlist
  \item
    \texttt{\ title\ } : the entry’s title, e. g. your job title
  \item
    \texttt{\ company-or-university\ } : the name of the institution
    which you were at, e. g. company or university
  \item
    \texttt{\ date\ } : start and end date of this entry, e. g. 2020 -
    2022
  \item
    \texttt{\ location\ } : describes where you worked, e. g. London, UK
  \item
    \texttt{\ logo\ } : path to the logo of this entry
  \item
    ``description`: description what you have done - normally supplied
    as a list
  \end{itemize}
\end{enumerate}

\subsection{Customization}\label{customization}

In the folder \texttt{\ settings\ } you will a file
\texttt{\ styles.typ\ } which includes all used styles. You can
customize them as you want to.

\subsection{Multi Language Support}\label{multi-language-support}

If you want to add a new language, copy the \texttt{\ cv-en.typ\ } and
rename it. Afterwards you can adapt the text correspondingly. Maybe I
will introduce i18n in the future.

\subsection{Insipration}\label{insipration}

A big thanks to
\href{https://github.com/mintyfrankie/brilliant-CV}{brilliant-CV} as
this project was an inspiraton for my CV and for the scripting.

\subsection{Questions \& Issues}\label{questions-issues}

If you have questions, plase create a
\href{https://github.com/AnsgarLichter/light-cv/discussions}{discussion}
. If you have an issue, please create an
\href{https://github.com/AnsgarLichter/light-cv/issues}{issue} .

\href{/app?template=light-cv&version=0.1.1}{Create project in app}

\subsubsection{How to use}\label{how-to-use}

Click the button above to create a new project using this template in
the Typst app.

You can also use the Typst CLI to start a new project on your computer
using this command:

\begin{verbatim}
typst init @preview/light-cv:0.1.1
\end{verbatim}

\includesvg[width=0.16667in,height=0.16667in]{/assets/icons/16-copy.svg}

\subsubsection{About}\label{about}

\begin{description}
\tightlist
\item[Author :]
Ansgar Lichter
\item[License:]
MIT
\item[Current version:]
0.1.1
\item[Last updated:]
May 6, 2024
\item[First released:]
April 17, 2024
\item[Archive size:]
414 kB
\href{https://packages.typst.org/preview/light-cv-0.1.1.tar.gz}{\pandocbounded{\includesvg[keepaspectratio]{/assets/icons/16-download.svg}}}
\item[Repository:]
\href{https://github.com/AnsgarLichter/cv-typst-template}{GitHub}
\item[Categor y :]
\begin{itemize}
\tightlist
\item[]
\item
  \pandocbounded{\includesvg[keepaspectratio]{/assets/icons/16-user.svg}}
  \href{https://typst.app/universe/search/?category=cv}{CV}
\end{itemize}
\end{description}

\subsubsection{Where to report issues?}\label{where-to-report-issues}

This template is a project of Ansgar Lichter . Report issues on
\href{https://github.com/AnsgarLichter/cv-typst-template}{their
repository} . You can also try to ask for help with this template on the
\href{https://forum.typst.app}{Forum} .

Please report this template to the Typst team using the
\href{https://typst.app/contact}{contact form} if you believe it is a
safety hazard or infringes upon your rights.

\phantomsection\label{versions}
\subsubsection{Version history}\label{version-history}

\begin{longtable}[]{@{}ll@{}}
\toprule\noalign{}
Version & Release Date \\
\midrule\noalign{}
\endhead
\bottomrule\noalign{}
\endlastfoot
0.1.1 & May 6, 2024 \\
\href{https://typst.app/universe/package/light-cv/0.1.0/}{0.1.0} & April
17, 2024 \\
\end{longtable}

Typst GmbH did not create this template and cannot guarantee correct
functionality of this template or compatibility with any version of the
Typst compiler or app.


\section{Package List LaTeX/glossy.tex}
\title{typst.app/universe/package/glossy}

\phantomsection\label{banner}
\section{glossy}\label{glossy}

{ 0.2.0 }

A very simple glossary system with easily customizable output.

\phantomsection\label{readme}
This package provides utilities to manage and render glossaries within
documents. It includes functions to define and use glossary terms, track
their usage, and generate a glossary list with references to where terms
are used in the document.

\pandocbounded{\includegraphics[keepaspectratio]{https://github.com/typst/packages/raw/main/packages/preview/glossy/0.2.0/thumbnail.png}}

\subsection{Motivation}\label{motivation}

Glossy is heavily inspired by
\href{https://typst.app/universe/package/glossarium}{glossarium} , with
a few key different goals:

\begin{enumerate}
\tightlist
\item
  Provide a simple interface which allows for complete control over
  glossary display. To do this, \texttt{\ glossy\ } ’s
  \texttt{\ \#glossary()\ } function accepts a theme parameter. The goal
  here is to separate presentation and logic.
\item
  Simplify the user interface as much as possible. Glossy has exactly
  two exports, \texttt{\ init-glossary\ } and \texttt{\ glossary\ } .
\item
  Double-down on \texttt{\ glossy\ } ’s excellent \texttt{\ @term\ }
  reference approach, completely eliminating the need to make any calls
  to \texttt{\ gls()\ } and friends.
\item
  Mimic established patterns and best practices. For example,
  \texttt{\ glossy\ } ’s \texttt{\ \#glossary()\ } function is
  intentionally similar (in naming and parameters) to \texttt{\ typst\ }
  ’s built-in \texttt{\ \#bibliography\ } , to the degree possible.
\item
  Simplify the implementation. The \texttt{\ glossy\ } code is
  significantly shorter and easier to understand.
\end{enumerate}

\subsection{Features}\label{features}

\begin{itemize}
\tightlist
\item
  Define glossary terms with short and long forms, descriptions, and
  grouping
\item
  Automatically tracks term usage in the document through
  \texttt{\ @labels\ }
\item
  Supports modifiers to adjust how terms are displayed (capitalize,
  pluralize, etc.)
\item
  Generates a formatted glossary section with backlinks to term
  occurrences
\item
  Customizable themes for rendering glossary sections, groups, and
  entries
\item
  Automatic pluralization of terms with custom override options
\item
  Page number references back to term usage locations
\end{itemize}

\subsection{Usage}\label{usage}

\subsubsection{Import the package}\label{import-the-package}

\begin{Shaded}
\begin{Highlighting}[]
\NormalTok{\#import "@preview/glossy:0.2.0": *}
\end{Highlighting}
\end{Shaded}

\subsubsection{Defining Glossary Terms}\label{defining-glossary-terms}

Use the \texttt{\ init-glossary\ } function to initialize glossary
entries:

\begin{Shaded}
\begin{Highlighting}[]
\NormalTok{\#let myGlossary = (}
\NormalTok{    html: (}
\NormalTok{      short: "HTML",}
\NormalTok{      long: "Hypertext Markup Language",}
\NormalTok{      description: "A standard language for creating web pages",}
\NormalTok{      group: "Web"}
\NormalTok{    ),}
\NormalTok{    css: (}
\NormalTok{      short: "CSS",}
\NormalTok{      long: "Cascading Style Sheets",}
\NormalTok{      description: "A stylesheet language used for describing the presentation of a document",}
\NormalTok{      group: "Web"}
\NormalTok{    ),}
\NormalTok{    tps: (}
\NormalTok{      short: "TPS",}
\NormalTok{      long: "test procedure specification",}
\NormalTok{      description: "A formal document describing test steps and expected results",}
\NormalTok{      // Optional: Override automatic pluralization}
\NormalTok{      plural: "TPSes",}
\NormalTok{      longplural: "test procedure specifications"}
\NormalTok{    )}
\NormalTok{)}

\NormalTok{\#show: init{-}glossary.with(myGlossary)}
\end{Highlighting}
\end{Shaded}

Each glossary entry supports the following fields:

\begin{itemize}
\tightlist
\item
  \texttt{\ short\ } (required): Short form of the term
\item
  \texttt{\ long\ } (optional): Long form of the term
\item
  \texttt{\ description\ } (optional): Term description (often a
  definition)
\item
  \texttt{\ group\ } (optional): Category grouping
\item
  \texttt{\ plural\ } (optional): Override automatic pluralization of
  short form
\item
  \texttt{\ longplural\ } (optional): Override automatic pluralization
  of long form
\end{itemize}

You can also load glossary entries from a data file using \#yaml(),
\#json(), or similar.

For example, the above glossary could be in this YAML file:

\begin{Shaded}
\begin{Highlighting}[]
\FunctionTok{html}\KeywordTok{:}
\AttributeTok{  }\FunctionTok{short}\KeywordTok{:}\AttributeTok{ HTML}
\AttributeTok{  }\FunctionTok{long}\KeywordTok{:}\AttributeTok{ Hypertext Markup Language}
\AttributeTok{  }\FunctionTok{description}\KeywordTok{:}\AttributeTok{ A standard language for creating web pages}
\AttributeTok{  }\FunctionTok{group}\KeywordTok{:}\AttributeTok{ Web}

\FunctionTok{css}\KeywordTok{:}
\AttributeTok{  }\FunctionTok{short}\KeywordTok{:}\AttributeTok{ CSS}
\AttributeTok{  }\FunctionTok{long}\KeywordTok{:}\AttributeTok{ Cascading Style Sheets}
\AttributeTok{  }\FunctionTok{description}\KeywordTok{:}\AttributeTok{ A stylesheet language used for describing the presentation of a document}
\AttributeTok{  }\FunctionTok{group}\KeywordTok{:}\AttributeTok{ Web}

\FunctionTok{tps}\KeywordTok{:}
\AttributeTok{  }\FunctionTok{short}\KeywordTok{:}\AttributeTok{ TPS}
\AttributeTok{  }\FunctionTok{long}\KeywordTok{:}\AttributeTok{ test procedure specification}
\AttributeTok{  }\FunctionTok{description}\KeywordTok{:}\AttributeTok{ A formal document describing test steps and expected results}
\AttributeTok{  }\FunctionTok{plural}\KeywordTok{:}\AttributeTok{ TPSes}
\AttributeTok{  }\FunctionTok{longplural}\KeywordTok{:}\AttributeTok{ test procedure specifications}
\end{Highlighting}
\end{Shaded}

And then loaded during initialization as follows:

\begin{Shaded}
\begin{Highlighting}[]
\NormalTok{\#show: init{-}glossary.with(yaml("glossary.yaml"))}
\end{Highlighting}
\end{Shaded}

\subsubsection{Using Glossary Terms}\label{using-glossary-terms}

Reference glossary terms using Typst’s \texttt{\ @reference\ } syntax:

\begin{Shaded}
\begin{Highlighting}[]
\NormalTok{In modern web development, languages like @html and @css are essential.}
\NormalTok{The @tps:pl need to be submitted by Friday.}
\end{Highlighting}
\end{Shaded}

Available modifiers:

\begin{itemize}
\tightlist
\item
  \textbf{cap} : Capitalizes the term
\item
  \textbf{pl} : Uses the plural form
\item
  \textbf{both} : Shows “Long Form (Short Form)�
\item
  \textbf{short} : Shows only short form
\item
  \textbf{long} : Shows only long form
\item
  \textbf{def} or \textbf{desc} : Shows the description
\end{itemize}

Modifiers can be combined with colons:

\begin{longtable}[]{@{}ll@{}}
\toprule\noalign{}
\textbf{Input} & \textbf{Output} \\
\midrule\noalign{}
\endhead
\bottomrule\noalign{}
\endlastfoot
\texttt{\ @tps\ } (first use) & “test procedure specification
(TPS)� \\
\texttt{\ @tps\ } (subsequent) & “TPS� \\
\texttt{\ @tps:short\ } & “TPS� \\
\texttt{\ @tps:long\ } & “test procedure specification� \\
\texttt{\ @tps:both\ } & “test procedure specification (TPS)� \\
\texttt{\ @tps:long:cap\ } & “Test procedure specification� \\
\texttt{\ @tps:long:pl\ } & “test procedure specifications� \\
\texttt{\ @tps:short:pl\ } & “TPSes� \\
\texttt{\ @tps:both:pl:cap\ } & “Technical procedure specifications
(TPSes)� \\
\texttt{\ @tps:def\ } & “A formal document describing test steps and
expected results� \\
\end{longtable}

\subsubsection{Generating the Glossary}\label{generating-the-glossary}

Display the glossary using the \texttt{\ glossary()\ } function:

\begin{Shaded}
\begin{Highlighting}[]
\NormalTok{\#glossary(}
\NormalTok{  title: "Web Development Glossary",}
\NormalTok{  theme: my{-}theme,}
\NormalTok{  groups: ("Web")  // Optional: Filter to specific groups}
\NormalTok{)}
\end{Highlighting}
\end{Shaded}

\subsubsection{Customizing Term Display}\label{customizing-term-display}

Control how terms appear in the document by providing a custom
\texttt{\ show-term\ } function:

\begin{Shaded}
\begin{Highlighting}[]
\NormalTok{\#let emph{-}term(term{-}body) = \{ emph(term{-}body) \}}

\NormalTok{\#show: init{-}glossary.with(}
\NormalTok{  myGlossary,}
\NormalTok{  show{-}term: emph{-}term}
\NormalTok{)}
\end{Highlighting}
\end{Shaded}

\subsubsection{Glossary Themes}\label{glossary-themes}

\paragraph{Included Themes}\label{included-themes}

Glossy comes with several built-in themes that can be used directly or
serve as examples for custom themes:

\subparagraph{theme-twocol}\label{theme-twocol}

A professional two-column layout ideal for technical documentation.
Features:

\begin{itemize}
\tightlist
\item
  Two-column layout for efficient space usage
\item
  Dotted leaders to page numbers
\item
  Clear hierarchy with optional group headings
\item
  Compact but readable formatting
\item
  Terms in regular weight with long forms and descriptions inline
\end{itemize}

\begin{Shaded}
\begin{Highlighting}[]
\NormalTok{\#glossary(theme: theme{-}twocol)}
\end{Highlighting}
\end{Shaded}

\subparagraph{theme-basic}\label{theme-basic}

A traditional single-column layout similar to book glossaries. Features:

\begin{itemize}
\tightlist
\item
  Bold terms with indented content
\item
  Clear separation between entries
\item
  Hanging indentation for wrapped lines
\item
  Parenthetical long forms
\item
  Page numbers with “pp.� prefix
\item
  Simple, clean design
\end{itemize}

\begin{Shaded}
\begin{Highlighting}[]
\NormalTok{\#glossary(theme: theme{-}basic)}
\end{Highlighting}
\end{Shaded}

\subparagraph{theme-compact}\label{theme-compact}

A space-efficient layout perfect for technical documents or appendices.
Features:

\begin{itemize}
\tightlist
\item
  Reduced vertical spacing
\item
  Smaller font sizes for secondary information
\item
  Color-coded term components
\item
  Grid-based alignment
\item
  Minimal decorative elements
\item
  Gray text for supplementary information
\item
  Bullet separators between components
\end{itemize}

\begin{Shaded}
\begin{Highlighting}[]
\NormalTok{\#glossary(theme: theme{-}compact)}
\end{Highlighting}
\end{Shaded}

\paragraph{Custom Themes}\label{custom-themes}

Customize glossary appearance by defining a theme with three functions:

\begin{Shaded}
\begin{Highlighting}[]
\NormalTok{\#let my{-}theme = (}
\NormalTok{  // Main glossary section}
\NormalTok{  section: (title, body) =\textgreater{} \{}
\NormalTok{    heading(level: 1, title)}
\NormalTok{    body}
\NormalTok{  \},}

\NormalTok{  // Group of related terms}
\NormalTok{  group: (name, index, total, body) =\textgreater{} \{}
\NormalTok{    // index = group index, total = total groups}
\NormalTok{    if name != "" and total \textgreater{} 1 \{}
\NormalTok{      heading(level: 2, name)}
\NormalTok{    \}}
\NormalTok{    body}
\NormalTok{  \},}

\NormalTok{  // Individual glossary entry}
\NormalTok{  entry: (entry, index, total) =\textgreater{} \{}
\NormalTok{    // index = entry index, total = total entries in group}
\NormalTok{    let output = [\#entry.short]}
\NormalTok{    if entry.long != none \{}
\NormalTok{      output = [\#output {-}{-} \#entry.long]}
\NormalTok{    \}}
\NormalTok{    if entry.description != none \{}
\NormalTok{      output = [\#output: \#entry.description]}
\NormalTok{    \}}
\NormalTok{    block(}
\NormalTok{      grid(}
\NormalTok{        columns: (auto, 1fr, auto),}
\NormalTok{        output,}
\NormalTok{        repeat([\#h(0.25em) . \#h(0.25em)]),}
\NormalTok{        entry.pages,}
\NormalTok{      )}
\NormalTok{    )}
\NormalTok{  \}}
\NormalTok{)}
\end{Highlighting}
\end{Shaded}

Entry fields available to themes:

\begin{itemize}
\tightlist
\item
  \texttt{\ short\ } : Short form (always present)
\item
  \texttt{\ long\ } : Long form (can be \texttt{\ none\ } )
\item
  \texttt{\ description\ } : Term description (can be \texttt{\ none\ }
  )
\item
  \texttt{\ label\ } : Term’s dictionary label
\item
  \texttt{\ pages\ } : Linked page numbers where term appears
\end{itemize}

\subsection{License}\label{license}

This project is licensed under the MIT License.

\subsubsection{How to add}\label{how-to-add}

Copy this into your project and use the import as \texttt{\ glossy\ }

\begin{verbatim}
#import "@preview/glossy:0.2.0"
\end{verbatim}

\includesvg[width=0.16667in,height=0.16667in]{/assets/icons/16-copy.svg}

Check the docs for
\href{https://typst.app/docs/reference/scripting/\#packages}{more
information on how to import packages} .

\subsubsection{About}\label{about}

\begin{description}
\tightlist
\item[Author :]
\href{mailto:steve@waits.net}{Stephen Waits}
\item[License:]
MIT
\item[Current version:]
0.2.0
\item[Last updated:]
November 26, 2024
\item[First released:]
October 23, 2024
\item[Archive size:]
10.2 kB
\href{https://packages.typst.org/preview/glossy-0.2.0.tar.gz}{\pandocbounded{\includesvg[keepaspectratio]{/assets/icons/16-download.svg}}}
\item[Repository:]
\href{https://github.com/swaits/typst-collection}{GitHub}
\item[Categor y :]
\begin{itemize}
\tightlist
\item[]
\item
  \pandocbounded{\includesvg[keepaspectratio]{/assets/icons/16-list-unordered.svg}}
  \href{https://typst.app/universe/search/?category=model}{Model}
\end{itemize}
\end{description}

\subsubsection{Where to report issues?}\label{where-to-report-issues}

This package is a project of Stephen Waits . Report issues on
\href{https://github.com/swaits/typst-collection}{their repository} .
You can also try to ask for help with this package on the
\href{https://forum.typst.app}{Forum} .

Please report this package to the Typst team using the
\href{https://typst.app/contact}{contact form} if you believe it is a
safety hazard or infringes upon your rights.

\phantomsection\label{versions}
\subsubsection{Version history}\label{version-history}

\begin{longtable}[]{@{}ll@{}}
\toprule\noalign{}
Version & Release Date \\
\midrule\noalign{}
\endhead
\bottomrule\noalign{}
\endlastfoot
0.2.0 & November 26, 2024 \\
\href{https://typst.app/universe/package/glossy/0.1.2/}{0.1.2} & October
31, 2024 \\
\href{https://typst.app/universe/package/glossy/0.1.1/}{0.1.1} & October
24, 2024 \\
\href{https://typst.app/universe/package/glossy/0.1.0/}{0.1.0} & October
23, 2024 \\
\end{longtable}

Typst GmbH did not create this package and cannot guarantee correct
functionality of this package or compatibility with any version of the
Typst compiler or app.


\section{Package List LaTeX/minimal-cv.tex}
\title{typst.app/universe/package/minimal-cv}

\phantomsection\label{banner}
\phantomsection\label{template-thumbnail}
\pandocbounded{\includegraphics[keepaspectratio]{https://packages.typst.org/preview/thumbnails/minimal-cv-0.1.0-small.webp}}

\section{minimal-cv}\label{minimal-cv}

{ 0.1.0 }

A clean and customizable CV template

\href{/app?template=minimal-cv&version=0.1.0}{Create project in app}

\phantomsection\label{readme}
Yet another John Doe CV.

\href{https://github.com/typst/packages/raw/main/packages/preview/minimal-cv/0.1.0/thumbnail.png}{\includegraphics[width=3.125in,height=\textheight,keepaspectratio]{https://github.com/typst/packages/raw/main/packages/preview/minimal-cv/0.1.0/thumbnail.png}}

A Typst CV template that aims for :

\begin{itemize}
\tightlist
\item
  Clean aesthetics
\item
  Easy customizability
\end{itemize}

\subsection{Usage}\label{usage}

\subsubsection{From Typst app}\label{from-typst-app}

Create a new project based on the template
\href{https://typst.app/universe/package/minimal-cv}{minimal-cv} .

\subsubsection{Locally}\label{locally}

The default font is
\href{https://fonts.google.com/specimen/Inria+Sans}{“Inria Sans�} .
Make sure it is installed on your system, or change it in
\href{https://github.com/typst/packages/raw/main/packages/preview/minimal-cv/0.1.0/\#theme}{\#
Theme} .

Copy the
\href{https://raw.githubusercontent.com/lelimacon/typst-minimal-cv/main/template/cv.typ}{template}
to your Typst project.

\subsubsection{From a blank project}\label{from-a-blank-project}

Import the library :

\begin{Shaded}
\begin{Highlighting}[]
\NormalTok{\#import "@preview/minimal{-}cv:0.1.0": *}
\end{Highlighting}
\end{Shaded}

Show the root \texttt{\ cv\ } function :

\begin{Shaded}
\begin{Highlighting}[]
\NormalTok{\#show: cv.with(}
\NormalTok{  theme: (),}
\NormalTok{  title: "YOUR NAME",}
\NormalTok{  subtitle: "YOUR POSITION",}
\NormalTok{  aside: [}
\NormalTok{    ASIDE CONTENT}
\NormalTok{  ]}

\NormalTok{MAIN CONTENT}
\end{Highlighting}
\end{Shaded}

Several content functions are available.

\textbf{Section}

\begin{Shaded}
\begin{Highlighting}[]
\NormalTok{\#section(}
\NormalTok{  theme: (),}
\NormalTok{  "TITLE\_CONTENT",}
\NormalTok{  "BODY\_CONTENT",}
\NormalTok{)}
\end{Highlighting}
\end{Shaded}

\textbf{Entry}

\begin{Shaded}
\begin{Highlighting}[]
\NormalTok{\#entry(}
\NormalTok{  theme: (),}
\NormalTok{  right: "FLOATING\_CONTENT",}

\NormalTok{  "GUTTER\_CONTENT",}
\NormalTok{  "TITLE\_CONTENT",}
\NormalTok{  "BODY\_CONTENT",}
\NormalTok{)}
\end{Highlighting}
\end{Shaded}

\textbf{Progress bar}

\begin{Shaded}
\begin{Highlighting}[]
\NormalTok{\#progress{-}bar(50\%)}
\end{Highlighting}
\end{Shaded}

\subsection{Theme}\label{theme}

Customize the theme by specifying the \texttt{\ theme\ } parameter and
overriding 1 or more keys.

\subsubsection{\texorpdfstring{Function
\texttt{\ cv\ }}{Function  cv }}\label{function-cv}

\begin{longtable}[]{@{}lll@{}}
\toprule\noalign{}
Key & Type & Default \\
\midrule\noalign{}
\endhead
\bottomrule\noalign{}
\endlastfoot
\texttt{\ margin\ } & relative & \texttt{\ 22pt\ } \\
\texttt{\ font\ } & relative & \texttt{\ "Inria\ Sans"\ } \\
\texttt{\ font-size\ } & relative & \texttt{\ 11pt\ } \\
\texttt{\ accent-color\ } & color & \texttt{\ blue\ } \\
\texttt{\ body-color\ } & color & \texttt{\ rgb("222")\ } \\
\texttt{\ header-accent-color\ } & color & inherit \\
\texttt{\ header-body-color\ } & color & inherit \\
\texttt{\ main-accent-color\ } & color & inherit \\
\texttt{\ main-body-color\ } & color & inherit \\
\texttt{\ main-width\ } & relative & \texttt{\ 5fr\ } \\
\texttt{\ main-gutter-width\ } & relative & \texttt{\ 64pt\ } \\
\texttt{\ aside-accent-color\ } & color & inherit \\
\texttt{\ aside-body-color\ } & color & inherit \\
\texttt{\ aside-width\ } & relative & \texttt{\ 3fr\ } \\
\texttt{\ aside-gutter-width\ } & relative & \texttt{\ 48pt\ } \\
\end{longtable}

\subsubsection{\texorpdfstring{Function
\texttt{\ section\ }}{Function  section }}\label{function-section}

\begin{longtable}[]{@{}lll@{}}
\toprule\noalign{}
Key & Type & Default \\
\midrule\noalign{}
\endhead
\bottomrule\noalign{}
\endlastfoot
\texttt{\ gutter-size\ } & color & inherit \\
\texttt{\ accent-color\ } & color & inherit \\
\texttt{\ body-color\ } & color & inherit \\
\end{longtable}

\subsubsection{\texorpdfstring{Function
\texttt{\ entry\ }}{Function  entry }}\label{function-entry}

\begin{longtable}[]{@{}lll@{}}
\toprule\noalign{}
Key & Type & Default \\
\midrule\noalign{}
\endhead
\bottomrule\noalign{}
\endlastfoot
\texttt{\ gutter-size\ } & color & inherit \\
\texttt{\ accent-color\ } & color & inherit \\
\texttt{\ body-color\ } & color & inherit \\
\end{longtable}

\href{/app?template=minimal-cv&version=0.1.0}{Create project in app}

\subsubsection{How to use}\label{how-to-use}

Click the button above to create a new project using this template in
the Typst app.

You can also use the Typst CLI to start a new project on your computer
using this command:

\begin{verbatim}
typst init @preview/minimal-cv:0.1.0
\end{verbatim}

\includesvg[width=0.16667in,height=0.16667in]{/assets/icons/16-copy.svg}

\subsubsection{About}\label{about}

\begin{description}
\tightlist
\item[Author :]
\href{https://github.com/lelimacon}{lelimacon}
\item[License:]
MIT
\item[Current version:]
0.1.0
\item[Last updated:]
June 12, 2024
\item[First released:]
June 12, 2024
\item[Archive size:]
4.43 kB
\href{https://packages.typst.org/preview/minimal-cv-0.1.0.tar.gz}{\pandocbounded{\includesvg[keepaspectratio]{/assets/icons/16-download.svg}}}
\item[Repository:]
\href{https://github.com/lelimacon/typst-minimal-cv}{GitHub}
\item[Categor y :]
\begin{itemize}
\tightlist
\item[]
\item
  \pandocbounded{\includesvg[keepaspectratio]{/assets/icons/16-user.svg}}
  \href{https://typst.app/universe/search/?category=cv}{CV}
\end{itemize}
\end{description}

\subsubsection{Where to report issues?}\label{where-to-report-issues}

This template is a project of lelimacon . Report issues on
\href{https://github.com/lelimacon/typst-minimal-cv}{their repository} .
You can also try to ask for help with this template on the
\href{https://forum.typst.app}{Forum} .

Please report this template to the Typst team using the
\href{https://typst.app/contact}{contact form} if you believe it is a
safety hazard or infringes upon your rights.

\phantomsection\label{versions}
\subsubsection{Version history}\label{version-history}

\begin{longtable}[]{@{}ll@{}}
\toprule\noalign{}
Version & Release Date \\
\midrule\noalign{}
\endhead
\bottomrule\noalign{}
\endlastfoot
0.1.0 & June 12, 2024 \\
\end{longtable}

Typst GmbH did not create this template and cannot guarantee correct
functionality of this template or compatibility with any version of the
Typst compiler or app.


\section{Package List LaTeX/mitex.tex}
\title{typst.app/universe/package/mitex}

\phantomsection\label{banner}
\section{mitex}\label{mitex}

{ 0.2.4 }

LaTeX support for Typst, powered by Rust and WASM.

\phantomsection\label{readme}
\textbf{\href{https://www.latex-project.org/}{LaTeX} support for
\href{https://typst.app/}{Typst} , powered by
\href{https://www.rust-lang.org/}{Rust} and
\href{https://webassembly.org/}{WASM} .}

\href{https://github.com/mitex-rs/mitex}{MiTeX} processes LaTeX code
into an abstract syntax tree (AST). Then it transforms the AST into
Typst code and evaluates code into Typst content by \texttt{\ eval\ }
function.

MiTeX has been proved to be practical on a large project. It has already
correctly converted 32.5k equations from
\href{https://github.com/OI-wiki/OI-wiki}{OI Wiki} . Compared to
\href{https://github.com/jgm/texmath}{texmath} , MiTeX has a better
display effect and performance in that wiki project. It is also more
easy to use, since importing MiTeX to Typst is just one line of code,
while texmath is an external program.

In addition, MiTeX is not only \textbf{SMALL} but also \textbf{FAST} !
MiTeX has a size of just about 185 KB, comparing that texmath has a size
of 17 MB. A not strict but intuitive comparison is shown below. To
convert 32.5k equations from OI Wiki, texmath takes about 109s, while
MiTeX WASM takes only 2.28s and MiTeX x86 takes merely 0.085s.

Thanks to \href{https://github.com/Myriad-Dreamin}{@Myriad-Dreamin} , he
completed the most complex development work: developing the parser for
generating AST.

\subsection{MiTeX as a Typst Package}\label{mitex-as-a-typst-package}

\begin{itemize}
\tightlist
\item
  Use \texttt{\ mitex-convert\ } to convert LaTeX code into Typst code
  in string.
\item
  Use \texttt{\ mi\ } to render an inline LaTeX equation in Typst.
\item
  Use \texttt{\ mitex(numbering:\ none,\ supplement:\ auto,\ ..)\ } or
  \texttt{\ mimath\ } to render a block LaTeX equation in Typst.
\item
  Use \texttt{\ mitext\ } to render a LaTeX text in Typst.
\end{itemize}

PS: \texttt{\ \#set\ math.equation(numbering:\ "(1)")\ } is also valid
for MiTeX.

Following is
\href{https://github.com/mitex-rs/mitex/blob/main/packages/mitex/examples/example.typ}{a
simple example} of using MiTeX in Typst:

\begin{Shaded}
\begin{Highlighting}[]
\NormalTok{\#import "@preview/mitex:0.2.4": *}

\NormalTok{\#assert.eq(mitex{-}convert("\textbackslash{}alpha x"), "alpha  x ")}

\NormalTok{Write inline equations like \#mi("x") or \#mi[y].}

\NormalTok{Also block equations (this case is from \#text(blue.lighten(20\%), link("https://katex.org/")[katex.org])):}

\NormalTok{\#mitex(\textasciigrave{}}
\NormalTok{  \textbackslash{}newcommand\{\textbackslash{}f\}[2]\{\#1f(\#2)\}}
\NormalTok{  \textbackslash{}f\textbackslash{}relax\{x\} = \textbackslash{}int\_\{{-}\textbackslash{}infty\}\^{}\textbackslash{}infty}
\NormalTok{    \textbackslash{}f\textbackslash{}hat\textbackslash{}xi\textbackslash{},e\^{}\{2 \textbackslash{}pi i \textbackslash{}xi x\}}
\NormalTok{    \textbackslash{},d\textbackslash{}xi}
\NormalTok{\textasciigrave{})}

\NormalTok{We also support text mode (in development):}

\NormalTok{\#mitext(\textasciigrave{}}
\NormalTok{  \textbackslash{}iftypst}
\NormalTok{    \#set math.equation(numbering: "(1)", supplement: "equation")}
\NormalTok{  \textbackslash{}fi}

\NormalTok{  \textbackslash{}section\{Title\}}

\NormalTok{  A \textbackslash{}textbf\{strong\} text, a \textbackslash{}emph\{emph\} text and inline equation $x + y$.}

\NormalTok{  Also block \textbackslash{}eqref\{eq:pythagoras\}.}

\NormalTok{  \textbackslash{}begin\{equation\}}
\NormalTok{    a\^{}2 + b\^{}2 = c\^{}2 \textbackslash{}label\{eq:pythagoras\}}
\NormalTok{  \textbackslash{}end\{equation\}}
\NormalTok{\textasciigrave{})}
\end{Highlighting}
\end{Shaded}

\pandocbounded{\includegraphics[keepaspectratio]{https://github.com/typst/packages/raw/main/packages/preview/mitex/0.2.4/examples/example.png}}

\subsection{MiTeX as a CLI Tool}\label{mitex-as-a-cli-tool}

\subsubsection{Installation}\label{installation}

Install latest nightly version by
\texttt{\ cargo\ install\ -\/-git\ https://github.com/mitex-rs/mitex\ mitex-cli\ }
.

\subsubsection{Usage}\label{usage}

\begin{Shaded}
\begin{Highlighting}[]
\ExtensionTok{mitex}\NormalTok{ compile main.tex}
\CommentTok{\# or (same as above)}
\ExtensionTok{mitex}\NormalTok{ compile main.tex mitex.typ}
\end{Highlighting}
\end{Shaded}

\subsection{MiTeX as a Web App}\label{mitex-as-a-web-app}

\subsubsection{MiTeX Online Math
Converter}\label{mitex-online-math-converter}

We can convert LaTeX equations to Typst equations in web by wasm.
\url{https://mitex-rs.github.io/mitex/}

\subsubsection{Underleaf}\label{underleaf}

We made a proof of concept online tex editor to show our conversion
speed in text mode. The PoC loads files from a git repository and then
runs typst compile in browser. As you see, each keystroking get response
in preview quickly.

\url{https://mitex-rs.github.io/mitex/tools/underleaf.html}

\url{https://github.com/mitex-rs/mitex/assets/34951714/0ce77a2c-0a7d-445f-b26d-e139f3038f83}

\subsection{Implemented Features}\label{implemented-features}

\begin{itemize}
\tightlist
\item
  {[}x{]} User-defined TeX (macro) commands, such as
  \texttt{\ \textbackslash{}newcommand\{\textbackslash{}mysym\}\{\textbackslash{}alpha\}\ }
  .
\item
  {[}x{]} LaTeX equations support.

  \begin{itemize}
  \tightlist
  \item
    {[}x{]} Coloring commands (
    \texttt{\ \textbackslash{}color\{red\}\ text\ } ,
    \texttt{\ \textbackslash{}textcolor\{red\}\{text\}\ } ).
  \item
    {[}x{]} Support for various environments, such as aligned, matrix,
    cases.
  \end{itemize}
\item
  {[}x{]} Basic text mode support, you can use it to write LaTeX drafts.

  \begin{itemize}
  \tightlist
  \item
    {[}x{]} \texttt{\ \textbackslash{}section\ } ,
    \texttt{\ \textbackslash{}textbf\ } ,
    \texttt{\ \textbackslash{}emph\ } .
  \item
    {[}x{]} Inline and block math equations.
  \item
    {[}x{]} \texttt{\ \textbackslash{}ref\ } ,
    \texttt{\ \textbackslash{}eqref\ } and
    \texttt{\ \textbackslash{}label\ } .
  \item
    {[}x{]} \texttt{\ itemize\ } and \texttt{\ enumerate\ }
    environments.
  \end{itemize}
\end{itemize}

\subsection{Features to Implement}\label{features-to-implement}

\begin{itemize}
\tightlist
\item
  {[} {]} Pass command specification to MiTeX plugin dynamically. With
  that you can define a typst function
  \texttt{\ let\ myop(it)\ =\ op(upright(it))\ } and then use it by
  \texttt{\ \textbackslash{}myop\{it\}\ } .
\item
  {[} {]} Package support, which means that you can change set of
  commands by telling MiTeX to use a list of packages.
\item
  {[} {]} Better text mode support, such as figure, algorithm and
  description environments.
\end{itemize}

To achieve the latter two goals, we need a well-structured architecture
for the text mode, along with intricate work. Currently, we don’t have
very clear ideas yet. If you are willing to contribute by discussing in
the issues or even submitting pull requests, your contribution is highly
welcome.

\subsection{Differences between MiTeX and other
solutions}\label{differences-between-mitex-and-other-solutions}

MiTeX has different objectives compared to
\href{https://github.com/jgm/texmath}{texmath} (a.k.a.
\href{https://pandoc.org/}{pandoc} ):

\begin{itemize}
\tightlist
\item
  MiTeX focuses on rendering LaTeX content correctly within Typst,
  leveraging the powerful programming capabilities of WASM and typst to
  achieve results that are essentially consistent with LaTeX display.
\item
  texmath aims to be general-purpose converters and generate strings
  that are more human-readable.
\end{itemize}

For example, MiTeX transforms
\texttt{\ \textbackslash{}frac\{1\}\{2\}\_3\ } into
\texttt{\ frac(1,\ 2)\_3\ } , while texmath converts it into
\texttt{\ 1\ /\ 2\_3\ } . The latter’s display is not entirely
correct, whereas the former ensures consistency in display.

Another example is that MiTeX transforms
\texttt{\ (\textbackslash{}frac\{1\}\{2\})\ } into
\texttt{\ \textbackslash{}(frac(1,\ 2)\textbackslash{})\ } instead of
\texttt{\ (frac(1,\ 2))\ } , avoiding the use of automatic Left/Right to
achieve consistency with LaTeX rendering.

\textbf{Certainly, the greatest advantage is that you can directly write
LaTeX content in Typst without the need for manual conversion!}

\subsection{Submitting Issues}\label{submitting-issues}

If you find missing commands or bugs of MiTeX, please feel free to
submit an issue \href{https://github.com/mitex-rs/mitex/issues}{here} .

\subsection{Contributing to MiTeX}\label{contributing-to-mitex}

Currently, MiTeX maintains following three parts of code:

\begin{itemize}
\tightlist
\item
  A TeX parser library written in \textbf{Rust} , see
  \href{https://github.com/mitex-rs/mitex/tree/main/crates/mitex-lexer}{mitex-lexer}
  and
  \href{https://github.com/mitex-rs/mitex/tree/main/crates/mitex-parser}{mitex-parser}
  .
\item
  A TeX to Typst converter library written in \textbf{Rust} , see
  \href{https://github.com/mitex-rs/mitex/tree/main/crates/mitex}{mitex}
  .
\item
  A list of TeX packages and comamnds written in \textbf{Typst} , which
  then used by the typst package, see
  \href{https://github.com/mitex-rs/mitex/tree/main/packages/mitex/specs}{MiTeX
  Command Specification} .
\end{itemize}

For a translation process, for example, we have:

\begin{verbatim}
\frac{1}{2}

===[parser]===> AST ===[converter]===>

#eval("$frac(1, 2)$", scope: (frac: (num, den) => $(num)/(den)$))
\end{verbatim}

You can use the \texttt{\ \#mitex-convert()\ } function to get the Typst
Code generated from LaTeX Code.

\subsubsection{Add missing TeX commands}\label{add-missing-tex-commands}

Even if you don’t know Rust at all, you can still add missing TeX
commands to MiTeX by modifying
\href{https://github.com/mitex-rs/mitex/tree/main/packages/mitex/specs}{specification
files} , since they are written in typst! You can open an issue to
acquire the commands you want to add, or you can edit the files and
submit a pull request.

In the future, we will provide the ability to customize TeX commands,
which will make it easier for you to use the commands you create for
yourself.

\subsubsection{Develop the parser and the
converter}\label{develop-the-parser-and-the-converter}

See
\href{https://github.com/mitex-rs/mitex/blob/main/CONTRIBUTING.md}{CONTRIBUTING.md}
.

\subsubsection{How to add}\label{how-to-add}

Copy this into your project and use the import as \texttt{\ mitex\ }

\begin{verbatim}
#import "@preview/mitex:0.2.4"
\end{verbatim}

\includesvg[width=0.16667in,height=0.16667in]{/assets/icons/16-copy.svg}

Check the docs for
\href{https://typst.app/docs/reference/scripting/\#packages}{more
information on how to import packages} .

\subsubsection{About}\label{about}

\begin{description}
\tightlist
\item[Author s :]
Myriad-Dreamin , OrangeX4 , \& Enter-tainer
\item[License:]
Apache-2.0
\item[Current version:]
0.2.4
\item[Last updated:]
May 13, 2024
\item[First released:]
December 23, 2023
\item[Archive size:]
109 kB
\href{https://packages.typst.org/preview/mitex-0.2.4.tar.gz}{\pandocbounded{\includesvg[keepaspectratio]{/assets/icons/16-download.svg}}}
\item[Repository:]
\href{https://github.com/mitex-rs/mitex}{GitHub}
\item[Categor y :]
\begin{itemize}
\tightlist
\item[]
\item
  \pandocbounded{\includesvg[keepaspectratio]{/assets/icons/16-hammer.svg}}
  \href{https://typst.app/universe/search/?category=utility}{Utility}
\end{itemize}
\end{description}

\subsubsection{Where to report issues?}\label{where-to-report-issues}

This package is a project of Myriad-Dreamin, OrangeX4, and Enter-tainer
. Report issues on \href{https://github.com/mitex-rs/mitex}{their
repository} . You can also try to ask for help with this package on the
\href{https://forum.typst.app}{Forum} .

Please report this package to the Typst team using the
\href{https://typst.app/contact}{contact form} if you believe it is a
safety hazard or infringes upon your rights.

\phantomsection\label{versions}
\subsubsection{Version history}\label{version-history}

\begin{longtable}[]{@{}ll@{}}
\toprule\noalign{}
Version & Release Date \\
\midrule\noalign{}
\endhead
\bottomrule\noalign{}
\endlastfoot
0.2.4 & May 13, 2024 \\
\href{https://typst.app/universe/package/mitex/0.2.3/}{0.2.3} & April 1,
2024 \\
\href{https://typst.app/universe/package/mitex/0.2.2/}{0.2.2} & March
10, 2024 \\
\href{https://typst.app/universe/package/mitex/0.2.1/}{0.2.1} & January
15, 2024 \\
\href{https://typst.app/universe/package/mitex/0.2.0/}{0.2.0} & January
1, 2024 \\
\href{https://typst.app/universe/package/mitex/0.1.0/}{0.1.0} & December
23, 2023 \\
\end{longtable}

Typst GmbH did not create this package and cannot guarantee correct
functionality of this package or compatibility with any version of the
Typst compiler or app.


\section{Package List LaTeX/cetz.tex}
\title{typst.app/universe/package/cetz}

\phantomsection\label{banner}
\section{cetz}\label{cetz}

{ 0.3.1 }

Drawing with Typst made easy, providing an API inspired by TikZ and
Processing. Includes modules for plotting, charts and tree layout.

{ } Featured Package

\phantomsection\label{readme}
CeTZ (CeTZ, ein Typst Zeichenpaket) is a library for drawing with
\href{https://typst.app/}{Typst} with an API inspired by TikZ and
\href{https://processing.org/}{Processing} .

\subsection{Examples}\label{examples}

\begin{longtable}[]{@{}lll@{}}
\toprule\noalign{}
\endhead
\bottomrule\noalign{}
\endlastfoot
\href{https://github.com/typst/packages/raw/main/packages/preview/cetz/0.3.1/gallery/karls-picture.typ}{\includegraphics[width=2.60417in,height=\textheight,keepaspectratio]{https://github.com/typst/packages/raw/main/packages/preview/cetz/0.3.1/gallery/karls-picture.png}}
&
\href{https://github.com/typst/packages/raw/main/packages/preview/cetz/0.3.1/gallery/tree.typ}{\includegraphics[width=2.60417in,height=\textheight,keepaspectratio]{https://github.com/typst/packages/raw/main/packages/preview/cetz/0.3.1/gallery/tree.png}}
&
\href{https://github.com/typst/packages/raw/main/packages/preview/cetz/0.3.1/gallery/waves.typ}{\includegraphics[width=2.60417in,height=\textheight,keepaspectratio]{https://github.com/typst/packages/raw/main/packages/preview/cetz/0.3.1/gallery/waves.png}} \\
Karl\textquotesingle s Picture & Tree Layout & Waves \\
\end{longtable}

\emph{Click on the example image to jump to the code.}

\subsection{Usage}\label{usage}

For information, see the
\href{https://cetz-package.github.io/docs}{online manual} .

To use this package, simply add the following code to your document:

\begin{verbatim}
#import "@preview/cetz:0.3.1"

#cetz.canvas({
  import cetz.draw: *
  // Your drawing code goes here
})
\end{verbatim}

\subsection{CeTZ Libraries}\label{cetz-libraries}

\begin{itemize}
\tightlist
\item
  \href{https://github.com/cetz-package/cetz-plot}{cetz-plot - Plotting
  and Charts Library}
\item
  \href{https://github.com/cetz-package/cetz-venn}{cetz-venn - Simple
  two- or three-set Venn diagrams}
\end{itemize}

\subsection{Installing}\label{installing}

To install the CeTZ package under
\href{https://github.com/typst/packages?tab=readme-ov-file\#local-packages}{your
local typst package dir} you can use the \texttt{\ install\ } script
from the repository.

\begin{Shaded}
\begin{Highlighting}[]
\ExtensionTok{just}\NormalTok{ install}
\end{Highlighting}
\end{Shaded}

The installed version can be imported by prefixing the package name with
\texttt{\ @local\ } .

\begin{Shaded}
\begin{Highlighting}[]
\NormalTok{\#import "@local/cetz:0.3.1"}

\NormalTok{\#cetz.canvas(\{}
\NormalTok{  import cetz.draw: *}
\NormalTok{  // Your drawing code goes here}
\NormalTok{\})}
\end{Highlighting}
\end{Shaded}

\subsubsection{Just}\label{just}

This project uses \href{https://github.com/casey/just}{just} , a handy
command runner.

You can run all commands without having \texttt{\ just\ } installed,
just have a look into the \texttt{\ justfile\ } . To install
\texttt{\ just\ } on your system, use your systems package manager. On
Windows, \href{https://doc.rust-lang.org/cargo/}{Cargo} (
\texttt{\ cargo\ install\ just\ } ),
\href{https://chocolatey.org/}{Chocolatey} (
\texttt{\ choco\ install\ just\ } ) and
\href{https://just.systems/man/en/chapter_4.html}{some other sources}
can be used. You need to run it from a \texttt{\ sh\ } compatible shell
on Windows (e.g git-bash).

\subsection{Testing}\label{testing}

This package comes with some unit tests under the \texttt{\ tests\ }
directory. To run all tests you can run the \texttt{\ just\ test\ }
target. You need to have
\href{https://github.com/tingerrr/typst-test/}{\texttt{\ typst-test\ }}
in your \texttt{\ PATH\ } :
\texttt{\ cargo\ install\ typst-test\ -\/-git\ https://github.com/tingerrr/typst-test\ }
.

\subsection{Projects using CeTZ}\label{projects-using-cetz}

\begin{itemize}
\tightlist
\item
  \href{https://github.com/fenjalien/cirCeTZ}{cirCeTZ} A port of
  \href{https://github.com/circuitikz/circuitikz}{circuitikz} to Typst.
\item
  \href{https://github.com/sitandr/conchord}{conchord} Package for
  writing lyrics with chords that generates fretboard diagrams using
  CeTZ.
\item
  \href{https://github.com/jneug/typst-finite}{finite} Finite is a Typst
  package for rendering finite automata.
\item
  \href{https://github.com/Jollywatt/typst-fletcher}{fletcher} Package
  for drawing commutative diagrams and figures with arrows.
\item
  \href{https://github.com/ThatOneCalculator/riesketcher}{riesketcher}
  Package for drawing Riemann sums.
\end{itemize}

\subsubsection{How to add}\label{how-to-add}

Copy this into your project and use the import as \texttt{\ cetz\ }

\begin{verbatim}
#import "@preview/cetz:0.3.1"
\end{verbatim}

\includesvg[width=0.16667in,height=0.16667in]{/assets/icons/16-copy.svg}

Check the docs for
\href{https://typst.app/docs/reference/scripting/\#packages}{more
information on how to import packages} .

\subsubsection{About}\label{about}

\begin{description}
\tightlist
\item[Author s :]
\href{https://github.com/johannes-wolf}{Johannes Wolf} \&
\href{https://github.com/fenjalien}{fenjalien}
\item[License:]
LGPL-3.0-or-later
\item[Current version:]
0.3.1
\item[Last updated:]
October 21, 2024
\item[First released:]
July 8, 2023
\item[Minimum Typst version:]
0.12.0
\item[Archive size:]
74.3 kB
\href{https://packages.typst.org/preview/cetz-0.3.1.tar.gz}{\pandocbounded{\includesvg[keepaspectratio]{/assets/icons/16-download.svg}}}
\item[Repository:]
\href{https://github.com/cetz-package/cetz}{GitHub}
\item[Categor y :]
\begin{itemize}
\tightlist
\item[]
\item
  \pandocbounded{\includesvg[keepaspectratio]{/assets/icons/16-chart.svg}}
  \href{https://typst.app/universe/search/?category=visualization}{Visualization}
\end{itemize}
\end{description}

\subsubsection{Where to report issues?}\label{where-to-report-issues}

This package is a project of Johannes Wolf and fenjalien . Report issues
on \href{https://github.com/cetz-package/cetz}{their repository} . You
can also try to ask for help with this package on the
\href{https://forum.typst.app}{Forum} .

Please report this package to the Typst team using the
\href{https://typst.app/contact}{contact form} if you believe it is a
safety hazard or infringes upon your rights.

\phantomsection\label{versions}
\subsubsection{Version history}\label{version-history}

\begin{longtable}[]{@{}ll@{}}
\toprule\noalign{}
Version & Release Date \\
\midrule\noalign{}
\endhead
\bottomrule\noalign{}
\endlastfoot
0.3.1 & October 21, 2024 \\
\href{https://typst.app/universe/package/cetz/0.3.0/}{0.3.0} & October
15, 2024 \\
\href{https://typst.app/universe/package/cetz/0.2.2/}{0.2.2} & March 18,
2024 \\
\href{https://typst.app/universe/package/cetz/0.2.1/}{0.2.1} & February
23, 2024 \\
\href{https://typst.app/universe/package/cetz/0.2.0/}{0.2.0} & January
16, 2024 \\
\href{https://typst.app/universe/package/cetz/0.1.2/}{0.1.2} & October
1, 2023 \\
\href{https://typst.app/universe/package/cetz/0.1.1/}{0.1.1} & September
11, 2023 \\
\href{https://typst.app/universe/package/cetz/0.1.0/}{0.1.0} & August
19, 2023 \\
\href{https://typst.app/universe/package/cetz/0.0.2/}{0.0.2} & July 31,
2023 \\
\href{https://typst.app/universe/package/cetz/0.0.1/}{0.0.1} & July 8,
2023 \\
\end{longtable}

Typst GmbH did not create this package and cannot guarantee correct
functionality of this package or compatibility with any version of the
Typst compiler or app.


\section{Package List LaTeX/touying-simpl-hkustgz.tex}
\title{typst.app/universe/package/touying-simpl-hkustgz}

\phantomsection\label{banner}
\phantomsection\label{template-thumbnail}
\pandocbounded{\includegraphics[keepaspectratio]{https://packages.typst.org/preview/thumbnails/touying-simpl-hkustgz-0.1.1-small.webp}}

\section{touying-simpl-hkustgz}\label{touying-simpl-hkustgz}

{ 0.1.1 }

Touying Slide Theme for HKUST(GZ)

\href{/app?template=touying-simpl-hkustgz&version=0.1.1}{Create project
in app}

\phantomsection\label{readme}
Inspired by \href{https://github.com/Coekjan/touying-buaa}{Touying Slide
Theme for Beihang University}

\subsection{Use as Typst Template
Package}\label{use-as-typst-template-package}

Use \texttt{\ typst\ init\ @preview/touying-simpl-hkustgz\ } to create a
new project with this theme.

\begin{Shaded}
\begin{Highlighting}[]
\NormalTok{$ typst init @preview/touying{-}simpl{-}hkustgz}
\NormalTok{Successfully created new project from @preview/touying{-}simpl{-}hkustgz:}
\NormalTok{To start writing, run:}
\NormalTok{\textgreater{} cd touying{-}simpl{-}hkustgz}
\NormalTok{\textgreater{} typst watch main.typ}
\end{Highlighting}
\end{Shaded}

\subsection{Examples}\label{examples}

See
\href{https://github.com/typst/packages/raw/main/packages/preview/touying-simpl-hkustgz/0.1.1/examples}{examples}
and \href{https://exaclior.github.io/touying-simpl-hkustgz}{Github
Pages} for more details.

You can compile the examples by yourself.

\begin{Shaded}
\begin{Highlighting}[]
\NormalTok{$ typst compile ./examples/main.typ {-}{-}root .}
\end{Highlighting}
\end{Shaded}

And the PDF file \texttt{\ ./examples/main.pdf\ } will be generated.

\subsection{License}\label{license}

Licensed under the
\href{https://github.com/typst/packages/raw/main/packages/preview/touying-simpl-hkustgz/0.1.1/LICENSE}{MIT
License} .

\href{/app?template=touying-simpl-hkustgz&version=0.1.1}{Create project
in app}

\subsubsection{How to use}\label{how-to-use}

Click the button above to create a new project using this template in
the Typst app.

You can also use the Typst CLI to start a new project on your computer
using this command:

\begin{verbatim}
typst init @preview/touying-simpl-hkustgz:0.1.1
\end{verbatim}

\includesvg[width=0.16667in,height=0.16667in]{/assets/icons/16-copy.svg}

\subsubsection{About}\label{about}

\begin{description}
\tightlist
\item[Author :]
\href{mailto:yushengzhao2020@outlook.com}{Yusheng Zhao}
\item[License:]
MIT
\item[Current version:]
0.1.1
\item[Last updated:]
November 12, 2024
\item[First released:]
August 28, 2024
\item[Archive size:]
9.83 kB
\href{https://packages.typst.org/preview/touying-simpl-hkustgz-0.1.1.tar.gz}{\pandocbounded{\includesvg[keepaspectratio]{/assets/icons/16-download.svg}}}
\item[Repository:]
\href{https://github.com/exAClior/touying-simpl-hkustgz}{GitHub}
\item[Categor y :]
\begin{itemize}
\tightlist
\item[]
\item
  \pandocbounded{\includesvg[keepaspectratio]{/assets/icons/16-presentation.svg}}
  \href{https://typst.app/universe/search/?category=presentation}{Presentation}
\end{itemize}
\end{description}

\subsubsection{Where to report issues?}\label{where-to-report-issues}

This template is a project of Yusheng Zhao . Report issues on
\href{https://github.com/exAClior/touying-simpl-hkustgz}{their
repository} . You can also try to ask for help with this template on the
\href{https://forum.typst.app}{Forum} .

Please report this template to the Typst team using the
\href{https://typst.app/contact}{contact form} if you believe it is a
safety hazard or infringes upon your rights.

\phantomsection\label{versions}
\subsubsection{Version history}\label{version-history}

\begin{longtable}[]{@{}ll@{}}
\toprule\noalign{}
Version & Release Date \\
\midrule\noalign{}
\endhead
\bottomrule\noalign{}
\endlastfoot
0.1.1 & November 12, 2024 \\
\href{https://typst.app/universe/package/touying-simpl-hkustgz/0.1.0/}{0.1.0}
& August 28, 2024 \\
\end{longtable}

Typst GmbH did not create this template and cannot guarantee correct
functionality of this template or compatibility with any version of the
Typst compiler or app.


\section{Package List LaTeX/orange-book.tex}
\title{typst.app/universe/package/orange-book}

\phantomsection\label{banner}
\phantomsection\label{template-thumbnail}
\pandocbounded{\includegraphics[keepaspectratio]{https://packages.typst.org/preview/thumbnails/orange-book-0.4.0-small.webp}}

\section{orange-book}\label{orange-book}

{ 0.4.0 }

A book template inspired by The Legrand Orange Book of Mathias Legrand
and Vel

\href{/app?template=orange-book&version=0.4.0}{Create project in app}

\phantomsection\label{readme}
A book template inspired by The Legrand Orange Book of Mathias Legrand
and Vel
\url{https://www.latextemplates.com/template/legrand-orange-book} .

\subsection{Usage}\label{usage}

You can use this template in the Typst web app by clicking “Start from
template� on the dashboard and searching for \texttt{\ orange-book\ }
.

Alternatively, you can use the CLI to kick this project off using the
command

\begin{verbatim}
typst init @preview/orange-book
\end{verbatim}

Typst will create a new directory with all the files needed to get you
started.

\subsection{Configuration}\label{configuration}

This template exports the \texttt{\ book\ } function with the following
named arguments:

\begin{itemize}
\tightlist
\item
  \texttt{\ title\ } : The book’s title as content.
\item
  \texttt{\ subtitle\ } : The book’s subtitle as content.
\item
  \texttt{\ author\ } : Content or an array of content to specify the
  author.
\item
  \texttt{\ paper-size\ } : Defaults to \texttt{\ a4\ } . Specify a
  \href{https://typst.app/docs/reference/layout/page/\#parameters-paper}{paper
  size string} to change the page format.
\item
  \texttt{\ copyright\ } : Details about the copyright or
  \texttt{\ none\ } .
\item
  \texttt{\ lowercase-references\ } : True to have references in
  lowercase (Eg. table 1.1)
\end{itemize}

The function also accepts a single, positional argument for the body of
the book.

The template will initialize your package with a sample call to the
\texttt{\ book\ } function in a show rule. If you, however, want to
change an existing project to use this template, you can add a show rule
like this at the top of your file:

\begin{Shaded}
\begin{Highlighting}[]
\NormalTok{\#import "@preview/orange{-}book:0.1.0": book}

\NormalTok{\#show: book.with(}
\NormalTok{  title: "Exploring the Physical Manifestation of Humanity’s Subconscious Desires",}
\NormalTok{  subtitle: "A Practical Guide",}
\NormalTok{  date: "Anno scolastico 2023{-}2024",}
\NormalTok{  author: "Goro Akechi",}
\NormalTok{  mainColor: rgb("\#F36619"),}
\NormalTok{  lang: "en",}
\NormalTok{  cover: image("./background.svg"),}
\NormalTok{  imageIndex: image("./orange1.jpg"),}
\NormalTok{  listOfFigureTitle: "List of Figures",}
\NormalTok{  listOfTableTitle: "List of Tables",}
\NormalTok{  supplementChapter: "Chapter",}
\NormalTok{  supplementPart: "Part",}
\NormalTok{  part\_style: 0,}
\NormalTok{  copyright: [}
\NormalTok{    Copyright © 2023 Flavio Barisi}

\NormalTok{    PUBLISHED BY PUBLISHER}

\NormalTok{    \#link("https://github.com/flavio20002/typst{-}orange{-}template", "TEMPLATE{-}WEBSITE")}

\NormalTok{    Licensed under the Apache 2.0 License (the “License”).}
\NormalTok{    You may not use this file except in compliance with the License. You may obtain a copy of}
\NormalTok{    the License at https://www.apache.org/licenses/LICENSE{-}2.0. Unless required by}
\NormalTok{    applicable law or agreed to in writing, software distributed under the License is distributed on an}
\NormalTok{    “AS IS” BASIS, WITHOUT WARRANTIES OR CONDITIONS OF ANY KIND, either express or implied.}
\NormalTok{    See the License for the specific language governing permissions and limitations under the License.}

\NormalTok{    \_First printing, July 2023\_}
\NormalTok{  ],}
\NormalTok{  lowercase{-}references: false}
\NormalTok{)}

\NormalTok{// Your content goes below.}
\end{Highlighting}
\end{Shaded}

\href{/app?template=orange-book&version=0.4.0}{Create project in app}

\subsubsection{How to use}\label{how-to-use}

Click the button above to create a new project using this template in
the Typst app.

You can also use the Typst CLI to start a new project on your computer
using this command:

\begin{verbatim}
typst init @preview/orange-book:0.4.0
\end{verbatim}

\includesvg[width=0.16667in,height=0.16667in]{/assets/icons/16-copy.svg}

\subsubsection{About}\label{about}

\begin{description}
\tightlist
\item[Author :]
Flavio Barisi
\item[License:]
MIT-0
\item[Current version:]
0.4.0
\item[Last updated:]
November 4, 2024
\item[First released:]
August 26, 2024
\item[Minimum Typst version:]
0.12.0
\item[Archive size:]
661 kB
\href{https://packages.typst.org/preview/orange-book-0.4.0.tar.gz}{\pandocbounded{\includesvg[keepaspectratio]{/assets/icons/16-download.svg}}}
\item[Repository:]
\href{https://github.com/flavio20002/typst-orange-template}{GitHub}
\item[Categor y :]
\begin{itemize}
\tightlist
\item[]
\item
  \pandocbounded{\includesvg[keepaspectratio]{/assets/icons/16-docs.svg}}
  \href{https://typst.app/universe/search/?category=book}{Book}
\end{itemize}
\end{description}

\subsubsection{Where to report issues?}\label{where-to-report-issues}

This template is a project of Flavio Barisi . Report issues on
\href{https://github.com/flavio20002/typst-orange-template}{their
repository} . You can also try to ask for help with this template on the
\href{https://forum.typst.app}{Forum} .

Please report this template to the Typst team using the
\href{https://typst.app/contact}{contact form} if you believe it is a
safety hazard or infringes upon your rights.

\phantomsection\label{versions}
\subsubsection{Version history}\label{version-history}

\begin{longtable}[]{@{}ll@{}}
\toprule\noalign{}
Version & Release Date \\
\midrule\noalign{}
\endhead
\bottomrule\noalign{}
\endlastfoot
0.4.0 & November 4, 2024 \\
\href{https://typst.app/universe/package/orange-book/0.3.0/}{0.3.0} &
October 22, 2024 \\
\href{https://typst.app/universe/package/orange-book/0.2.0/}{0.2.0} &
October 3, 2024 \\
\href{https://typst.app/universe/package/orange-book/0.1.0/}{0.1.0} &
August 26, 2024 \\
\end{longtable}

Typst GmbH did not create this template and cannot guarantee correct
functionality of this template or compatibility with any version of the
Typst compiler or app.


\section{Package List LaTeX/riesketcher.tex}
\title{typst.app/universe/package/riesketcher}

\phantomsection\label{banner}
\section{riesketcher}\label{riesketcher}

{ 0.2.1 }

A package to draw Riemann sums (and their plots) of a function with
CeTZ.

\phantomsection\label{readme}
A package to draw Riemann sums (and their plots) of a function with
CeTZ.

Usage example and docs:
\href{https://github.com/ThatOneCalculator/riesketcher/blob/main/manual.pdf}{manual.pdf}

\begin{Shaded}
\begin{Highlighting}[]
\NormalTok{\#import "@preview/riesketcher:0.2.1": riesketcher}
\end{Highlighting}
\end{Shaded}

\pandocbounded{\includegraphics[keepaspectratio]{https://github.com/ThatOneCalculator/riesketcher/assets/44733677/4f87b750-e4be-4698-b650-74f4fe56789d}}

\subsubsection{How to add}\label{how-to-add}

Copy this into your project and use the import as
\texttt{\ riesketcher\ }

\begin{verbatim}
#import "@preview/riesketcher:0.2.1"
\end{verbatim}

\includesvg[width=0.16667in,height=0.16667in]{/assets/icons/16-copy.svg}

Check the docs for
\href{https://typst.app/docs/reference/scripting/\#packages}{more
information on how to import packages} .

\subsubsection{About}\label{about}

\begin{description}
\tightlist
\item[Author :]
Kainoa Kanter
\item[License:]
MIT
\item[Current version:]
0.2.1
\item[Last updated:]
May 22, 2024
\item[First released:]
December 19, 2023
\item[Archive size:]
2.40 kB
\href{https://packages.typst.org/preview/riesketcher-0.2.1.tar.gz}{\pandocbounded{\includesvg[keepaspectratio]{/assets/icons/16-download.svg}}}
\item[Repository:]
\href{https://github.com/ThatOneCalculator/riesketcher}{GitHub}
\end{description}

\subsubsection{Where to report issues?}\label{where-to-report-issues}

This package is a project of Kainoa Kanter . Report issues on
\href{https://github.com/ThatOneCalculator/riesketcher}{their
repository} . You can also try to ask for help with this package on the
\href{https://forum.typst.app}{Forum} .

Please report this package to the Typst team using the
\href{https://typst.app/contact}{contact form} if you believe it is a
safety hazard or infringes upon your rights.

\phantomsection\label{versions}
\subsubsection{Version history}\label{version-history}

\begin{longtable}[]{@{}ll@{}}
\toprule\noalign{}
Version & Release Date \\
\midrule\noalign{}
\endhead
\bottomrule\noalign{}
\endlastfoot
0.2.1 & May 22, 2024 \\
\href{https://typst.app/universe/package/riesketcher/0.2.0/}{0.2.0} &
January 17, 2024 \\
\href{https://typst.app/universe/package/riesketcher/0.1.0/}{0.1.0} &
December 19, 2023 \\
\end{longtable}

Typst GmbH did not create this package and cannot guarantee correct
functionality of this package or compatibility with any version of the
Typst compiler or app.


\section{Package List LaTeX/outrageous.tex}
\title{typst.app/universe/package/outrageous}

\phantomsection\label{banner}
\section{outrageous}\label{outrageous}

{ 0.3.0 }

Easier customization of outline entries.

\phantomsection\label{readme}
Easier customization of outline entries.

\subsection{Examples}\label{examples}

For the full source see
\href{https://github.com/typst/packages/raw/main/packages/preview/outrageous/0.3.0/examples/basic.typ}{\texttt{\ examples/basic.typ\ }}
and for more examples see the
\href{https://github.com/typst/packages/raw/main/packages/preview/outrageous/0.3.0/examples}{\texttt{\ examples\ }
directory} .

\subsubsection{Default Style}\label{default-style}

\pandocbounded{\includegraphics[keepaspectratio]{https://github.com/typst/packages/raw/main/packages/preview/outrageous/0.3.0/example-default.png}}

\begin{Shaded}
\begin{Highlighting}[]
\NormalTok{\#import "@preview/outrageous:0.1.0"}
\NormalTok{\#show outline.entry: outrageous.show{-}entry}
\end{Highlighting}
\end{Shaded}

\subsubsection{Custom Settings}\label{custom-settings}

\pandocbounded{\includegraphics[keepaspectratio]{https://github.com/typst/packages/raw/main/packages/preview/outrageous/0.3.0/example-custom.png}}

\begin{Shaded}
\begin{Highlighting}[]
\NormalTok{\#import "@preview/outrageous:0.1.0"}
\NormalTok{\#show outline.entry: outrageous.show{-}entry.with(}
\NormalTok{  // the typst preset retains the normal Typst appearance}
\NormalTok{  ..outrageous.presets.typst,}
\NormalTok{  // we only override a few things:}
\NormalTok{  // level{-}1 entries are italic, all others keep their font style}
\NormalTok{  font{-}style: ("italic", auto),}
\NormalTok{  // no fill for level{-}1 entries, a thin gray line for all deeper levels}
\NormalTok{  fill: (none, line(length: 100\%, stroke: gray + .5pt)),}
\NormalTok{)}
\end{Highlighting}
\end{Shaded}

\subsection{Usage}\label{usage}

\subsubsection{\texorpdfstring{\texttt{\ show-entry\ }}{ show-entry }}\label{show-entry}

Show the given outline entry with the provided styling. Should be used
in a show rule like
\texttt{\ \#show\ outline.entry:\ outrageous.show-entry\ } .

\begin{Shaded}
\begin{Highlighting}[]
\NormalTok{\#let show{-}entry(}
\NormalTok{  entry,}
\NormalTok{  font{-}weight: presets.outrageous{-}toc.font{-}weight,}
\NormalTok{  font{-}style: presets.outrageous{-}toc.font{-}style,}
\NormalTok{  vspace: presets.outrageous{-}toc.vspace,}
\NormalTok{  font: presets.outrageous{-}toc.font,}
\NormalTok{  fill: presets.outrageous{-}toc.fill,}
\NormalTok{  fill{-}right{-}pad: presets.outrageous{-}toc.fill{-}right{-}pad,}
\NormalTok{  fill{-}align: presets.outrageous{-}toc.fill{-}align,}
\NormalTok{  body{-}transform: presets.outrageous{-}toc.body{-}transform,}
\NormalTok{  label: \textless{}outrageous{-}modified{-}entry\textgreater{},}
\NormalTok{  state{-}key: "outline{-}page{-}number{-}max{-}width",}
\NormalTok{) = \{ .. \}}
\end{Highlighting}
\end{Shaded}

\textbf{Arguments:}

For all the arguments that take arrays, the array’s first item
specifies the value for all level-one entries, the second item for
level-two, and so on. The array’s last item will be used for all
deeper/following levels as well.

\begin{itemize}
\tightlist
\item
  \texttt{\ entry\ } :
  \href{https://typst.app/docs/reference/foundations/content/}{\texttt{\ content\ }}
  â€'' The
  \href{https://typst.app/docs/reference/model/outline/\#definitions-entry}{\texttt{\ outline.entry\ }}
  element from the show rule.
\item
  \texttt{\ font-weight\ } :
  \href{https://typst.app/docs/reference/foundations/array/}{\texttt{\ array\ }}
  of (
  \href{https://typst.app/docs/reference/foundations/str/}{\texttt{\ str\ }}
  or
  \href{https://typst.app/docs/reference/foundations/int/}{\texttt{\ int\ }}
  or \texttt{\ auto\ } or \texttt{\ none\ } ) â€'' The entry’s font
  weight. Setting to \texttt{\ auto\ } or \texttt{\ none\ } keeps the
  context’s current style.
\item
  \texttt{\ font-style\ } :
  \href{https://typst.app/docs/reference/foundations/array/}{\texttt{\ array\ }}
  of (
  \href{https://typst.app/docs/reference/foundations/str/}{\texttt{\ str\ }}
  or \texttt{\ auto\ } or \texttt{\ none\ } ) â€'' The entry’s font
  style. Setting to \texttt{\ auto\ } or \texttt{\ none\ } keeps the
  context’s current style.
\item
  \texttt{\ vspace\ } :
  \href{https://typst.app/docs/reference/foundations/array/}{\texttt{\ array\ }}
  of (
  \href{https://typst.app/docs/reference/layout/relative/}{\texttt{\ relative\ }}
  or
  \href{https://typst.app/docs/reference/layout/fraction/}{\texttt{\ fraction\ }}
  or \texttt{\ none\ } ) â€'' Vertical spacing to add above the entry.
  Setting to \texttt{\ none\ } adds no space.
\item
  \texttt{\ font\ } :
  \href{https://typst.app/docs/reference/foundations/array/}{\texttt{\ array\ }}
  of (
  \href{https://typst.app/docs/reference/foundations/str/}{\texttt{\ str\ }}
  or
  \href{https://typst.app/docs/reference/foundations/array/}{\texttt{\ array\ }}
  or \texttt{\ auto\ } or \texttt{\ none\ } ) â€'' The entry’s font.
  Setting to \texttt{\ auto\ } or \texttt{\ none\ } keeps the
  context’s current font.
\item
  \texttt{\ fill\ } :
  \href{https://typst.app/docs/reference/foundations/array/}{\texttt{\ array\ }}
  of (
  \href{https://typst.app/docs/reference/foundations/content/}{\texttt{\ content\ }}
  or \texttt{\ auto\ } or \texttt{\ none\ } ) â€'' The entry’s fill.
  Setting to \texttt{\ auto\ } keeps the context’s current setting.
\item
  \texttt{\ fill-right-pad\ } :
  \href{https://typst.app/docs/reference/layout/relative/}{\texttt{\ relative\ }}
  or \texttt{\ none\ } â€'' Horizontal space to put between the fill and
  page number.
\item
  \texttt{\ fill-align\ } :
  \href{https://typst.app/docs/reference/foundations/bool/}{\texttt{\ bool\ }}
  â€'' Whether \texttt{\ fill-right-pad\ } should be relative to the
  current page number or the widest page number. Setting this to
  \texttt{\ true\ } has the effect of all fills ending on the same
  vertical line.
\item
  \texttt{\ body-transform\ } :
  \href{https://typst.app/docs/reference/foundations/function/}{\texttt{\ function\ }}
  or \texttt{\ none\ } â€'' Callback for custom edits to the entry’s
  body. It gets passed the entry’s level (
  \href{https://typst.app/docs/reference/foundations/int/}{\texttt{\ int\ }}
  ) and body (
  \href{https://typst.app/docs/reference/foundations/content/}{\texttt{\ content\ }}
  ) and should return
  \href{https://typst.app/docs/reference/foundations/content/}{\texttt{\ content\ }}
  or \texttt{\ none\ } . If \texttt{\ none\ } is returned, no
  modifications are made.
\item
  \texttt{\ page-transform\ } :
  \href{https://typst.app/docs/reference/foundations/function/}{\texttt{\ function\ }}
  or \texttt{\ none\ } â€'' Callback for custom edits to the entry’s
  page number. It gets passed the entry’s level (
  \href{https://typst.app/docs/reference/foundations/int/}{\texttt{\ int\ }}
  ) and page number (
  \href{https://typst.app/docs/reference/foundations/content/}{\texttt{\ content\ }}
  ) and should return
  \href{https://typst.app/docs/reference/foundations/content/}{\texttt{\ content\ }}
  or \texttt{\ none\ } . If \texttt{\ none\ } is returned, no
  modifications are made.
\item
  \texttt{\ label\ } :
  \href{https://typst.app/docs/reference/foundations/label/}{\texttt{\ label\ }}
  â€'' The label to internally use for tracking recursion. This should
  not need to be modified.
\item
  \texttt{\ state-key\ } :
  \href{https://typst.app/docs/reference/foundations/str/}{\texttt{\ str\ }}
  â€'' The key to use for the internal state which tracks the maximum
  page number width. The state is global for the entire document and
  thus applies to all outlines. If you wish to re-calculate the max page
  number width for \texttt{\ fill-align\ } , then you must provide a
  different key for each outline.
\end{itemize}

\textbf{Returns:}
\href{https://typst.app/docs/reference/foundations/content/}{\texttt{\ content\ }}

\subsubsection{\texorpdfstring{\texttt{\ presets\ }}{ presets }}\label{presets}

Presets for the arguments for
\href{https://github.com/typst/packages/raw/main/packages/preview/outrageous/0.3.0/\#show-entry}{\texttt{\ show-entry()\ }}
. You can use them in your show rule with
\texttt{\ \#show\ outline.entry:\ outrageous.show-entry.with(..outrageous.presets.outrageous-figures)\ }
.

\begin{Shaded}
\begin{Highlighting}[]
\NormalTok{\#let presets = (}
\NormalTok{  // outrageous preset for a Table of Contents}
\NormalTok{  outrageous{-}toc: (}
\NormalTok{    // ...}
\NormalTok{  ),}
\NormalTok{  // outrageous preset for List of Figures/Tables/Listings}
\NormalTok{  outrageous{-}figures: (}
\NormalTok{    // ...}
\NormalTok{  ),}
\NormalTok{  // preset without any style changes}
\NormalTok{  typst: (}
\NormalTok{    // ...}
\NormalTok{  ),}
\NormalTok{)}
\end{Highlighting}
\end{Shaded}

\subsubsection{\texorpdfstring{\texttt{\ repeat\ }}{ repeat }}\label{repeat}

Utility function to repeat content to fill space with a fixed size gap.

\begin{Shaded}
\begin{Highlighting}[]
\NormalTok{\#let repeat(gap: none, justify: false, body) = \{ .. \}}
\end{Highlighting}
\end{Shaded}

\textbf{Arguments:}

\begin{itemize}
\tightlist
\item
  \texttt{\ gap\ } :
  \href{https://typst.app/docs/reference/layout/length/}{\texttt{\ length\ }}
  or \texttt{\ none\ } â€'' The gap between repeated items.
\item
  \texttt{\ justify\ } :
  \href{https://typst.app/docs/reference/foundations/bool/}{\texttt{\ bool\ }}
  â€'' Whether to increase the gap to justify the items.
\item
  \texttt{\ body\ } :
  \href{https://typst.app/docs/reference/foundations/content/}{\texttt{\ content\ }}
  â€'' The content to repeat.
\end{itemize}

\textbf{Returns:}
\href{https://typst.app/docs/reference/foundations/content/}{\texttt{\ content\ }}

\subsubsection{\texorpdfstring{\texttt{\ align-helper\ }}{ align-helper }}\label{align-helper}

Utility function to help with aligning multiple items.

\begin{Shaded}
\begin{Highlighting}[]
\NormalTok{\#let align{-}helper(state{-}key, what{-}to{-}measure, display) = \{ .. \}}
\end{Highlighting}
\end{Shaded}

\textbf{Arguments:}

\begin{itemize}
\tightlist
\item
  \texttt{\ state-key\ } :
  \href{https://typst.app/docs/reference/foundations/str/}{\texttt{\ str\ }}
  â€'' The key to use for the
  \href{https://typst.app/docs/reference/introspection/state/}{\texttt{\ state\ }}
  that keeps track of the maximum encountered width.
\item
  \texttt{\ what-to-measure\ } :
  \href{https://typst.app/docs/reference/foundations/content/}{\texttt{\ content\ }}
  â€'' The content to measure at this call.
\item
  \texttt{\ display\ } :
  \href{https://typst.app/docs/reference/foundations/function/}{\texttt{\ function\ }}
  â€'' A callback which gets passed the maximum encountered width and
  the width of the current item (what was given to
  \texttt{\ what-to-measure\ } ), both as
  \href{https://typst.app/docs/reference/layout/length/}{\texttt{\ length\ }}
  , and should return
  \href{https://typst.app/docs/reference/foundations/content/}{\texttt{\ content\ }}
  which can make use of these widths for alignment.
\end{itemize}

\textbf{Returns:}
\href{https://typst.app/docs/reference/foundations/content/}{\texttt{\ content\ }}

\subsubsection{How to add}\label{how-to-add}

Copy this into your project and use the import as
\texttt{\ outrageous\ }

\begin{verbatim}
#import "@preview/outrageous:0.3.0"
\end{verbatim}

\includesvg[width=0.16667in,height=0.16667in]{/assets/icons/16-copy.svg}

Check the docs for
\href{https://typst.app/docs/reference/scripting/\#packages}{more
information on how to import packages} .

\subsubsection{About}\label{about}

\begin{description}
\tightlist
\item[Author :]
RubixDev
\item[License:]
GPL-3.0-only
\item[Current version:]
0.3.0
\item[Last updated:]
October 21, 2024
\item[First released:]
October 9, 2023
\item[Minimum Typst version:]
0.11.0
\item[Archive size:]
15.8 kB
\href{https://packages.typst.org/preview/outrageous-0.3.0.tar.gz}{\pandocbounded{\includesvg[keepaspectratio]{/assets/icons/16-download.svg}}}
\item[Repository:]
\href{https://github.com/RubixDev/typst-outrageous}{GitHub}
\end{description}

\subsubsection{Where to report issues?}\label{where-to-report-issues}

This package is a project of RubixDev . Report issues on
\href{https://github.com/RubixDev/typst-outrageous}{their repository} .
You can also try to ask for help with this package on the
\href{https://forum.typst.app}{Forum} .

Please report this package to the Typst team using the
\href{https://typst.app/contact}{contact form} if you believe it is a
safety hazard or infringes upon your rights.

\phantomsection\label{versions}
\subsubsection{Version history}\label{version-history}

\begin{longtable}[]{@{}ll@{}}
\toprule\noalign{}
Version & Release Date \\
\midrule\noalign{}
\endhead
\bottomrule\noalign{}
\endlastfoot
0.3.0 & October 21, 2024 \\
\href{https://typst.app/universe/package/outrageous/0.2.0/}{0.2.0} &
September 14, 2024 \\
\href{https://typst.app/universe/package/outrageous/0.1.0/}{0.1.0} &
October 9, 2023 \\
\end{longtable}

Typst GmbH did not create this package and cannot guarantee correct
functionality of this package or compatibility with any version of the
Typst compiler or app.


\section{Package List LaTeX/glossarium.tex}
\title{typst.app/universe/package/glossarium}

\phantomsection\label{banner}
\section{glossarium}\label{glossarium}

{ 0.5.1 }

Glossarium is a simple, easily customizable typst glossary.

{ } Featured Package

\phantomsection\label{readme}
\begin{quote}
{[}!TIP{]} Glossarium is based in great part of the work of
\href{https://github.com/Dherse}{Sébastien d’Herbais de Thun} from
his master thesis available at:
\url{https://github.com/Dherse/masterproef} . His glossary is available
under the MIT license
\href{https://github.com/Dherse/masterproef/blob/main/elems/acronyms.typ}{here}
.
\end{quote}

Glossarium is a simple, easily customizable typst glossary inspired by
\href{https://www.ctan.org/pkg/glossaries}{LaTeX glossaries package} .
You can see various examples showcasing the different features in the
\texttt{\ examples\ } folder.

\pandocbounded{\includegraphics[keepaspectratio]{https://github.com/typst/packages/raw/main/packages/preview/glossarium/0.5.1/.github/example.png}}

\subsection{Manual}\label{manual}

\subsection{Fast start}\label{fast-start}

\begin{Shaded}
\begin{Highlighting}[]
\NormalTok{\#import "@preview/glossarium:0.5.1": make{-}glossary, register{-}glossary, print{-}glossary, gls, glspl}
\NormalTok{\#show: make{-}glossary}
\NormalTok{\#let entry{-}list = (}
\NormalTok{  (}
\NormalTok{    key: "kuleuven",}
\NormalTok{    short: "KU Leuven",}
\NormalTok{    long: "Katholieke Universiteit Leuven",}
\NormalTok{    description: "A university in Belgium.",}
\NormalTok{  ),}
\NormalTok{  // Add more terms}
\NormalTok{)}
\NormalTok{\#register{-}glossary(entry{-}list)}
\NormalTok{// Your document body}
\NormalTok{\#print{-}glossary(}
\NormalTok{ entry{-}list}
\NormalTok{)}
\end{Highlighting}
\end{Shaded}

\subsubsection{Import and setup}\label{import-and-setup}

This manual assume you have a good enough understanding of typst markup
and scripting.

For Typst 0.6.0 or later import the package from the typst preview
repository:

\begin{Shaded}
\begin{Highlighting}[]
\NormalTok{\#import "@preview/glossarium:0.5.1": make{-}glossary, register{-}glossary, print{-}glossary, gls, glspl}
\end{Highlighting}
\end{Shaded}

For Typst before 0.6.0 or to use \textbf{glossarium} as a local module,
download the package files into your project folder and import
\texttt{\ glossarium.typ\ } :

\begin{Shaded}
\begin{Highlighting}[]
\NormalTok{\#import "glossarium.typ": make{-}glossary, register{-}glossary, print{-}glossary, gls, glspl}
\end{Highlighting}
\end{Shaded}

After importing the package and before making any calls to
\texttt{\ gls\ } , \texttt{\ print-glossary\ } or \texttt{\ glspl\ } ,
please \emph{\textbf{MAKE SURE}} you add this line

\begin{Shaded}
\begin{Highlighting}[]
\NormalTok{\#show: make{-}glossary}
\end{Highlighting}
\end{Shaded}

\begin{quote}
\emph{WHY DO WE NEED THAT ?} : In order to be able to create references
to the terms in your glossary using typst ref syntax \texttt{\ @key\ }
glossarium needs to setup some
\href{https://typst.app/docs/tutorial/advanced-styling/}{show rules}
before any references exist. This is due to the way typst works, there
is no workaround.

Therefore I recommend that you always put the \texttt{\ \#show:\ ...\ }
statement on the line just below the \texttt{\ \#import\ } statement.
\end{quote}

\subsubsection{Registering the glossary}\label{registering-the-glossary}

First we have to define the terms. A term is a
\href{https://typst.app/docs/reference/types/dictionary/}{dictionary} as
follows:

\begin{longtable}[]{@{}llll@{}}
\toprule\noalign{}
Key & Type & Required/Optional & Description \\
\midrule\noalign{}
\endhead
\bottomrule\noalign{}
\endlastfoot
\texttt{\ key\ } & string & required & Case-sensitive, unique identifier
used to reference the term. \\
\texttt{\ short\ } & string & semi-optional & The short form of the term
replacing the term citation. \\
\texttt{\ long\ } & string or content & semi-optional & The long form of
the term, displayed in the glossary and on the first citation of the
term. \\
\texttt{\ description\ } & string or content & optional & The
description of the term. \\
\texttt{\ plural\ } & string or content & optional & The pluralized
short form of the term. \\
\texttt{\ longplural\ } & string or content & optional & The pluralized
long form of the term. \\
\texttt{\ group\ } & string & optional & Case-sensitive group the term
belongs to. The terms are displayed by groups in the glossary. \\
\end{longtable}

\begin{Shaded}
\begin{Highlighting}[]
\NormalTok{\#let entry{-}list = (}
\NormalTok{  // minimal term}
\NormalTok{  (}
\NormalTok{    key: "kuleuven",}
\NormalTok{    short: "KU Leuven"}
\NormalTok{  ),}
\NormalTok{  // a term with a long form and a group}
\NormalTok{  (}
\NormalTok{    key: "unamur",}
\NormalTok{    short: "UNamur",}
\NormalTok{    long: "Namur University",}
\NormalTok{    group: "Universities"}
\NormalTok{  ),}
\NormalTok{  // a term with a markup description}
\NormalTok{  (}
\NormalTok{    key: "oidc",}
\NormalTok{    short: "OIDC",}
\NormalTok{    long: "OpenID Connect",}
\NormalTok{    description: [}
\NormalTok{      OpenID is an open standard and decentralized authentication protocol promoted by the non{-}profit}
\NormalTok{      \#link("https://en.wikipedia.org/wiki/OpenID\#OpenID\_Foundation")[OpenID Foundation].}
\NormalTok{    ],}
\NormalTok{    group: "Acronyms",}
\NormalTok{  ),}
\NormalTok{  // a term with a short plural}
\NormalTok{  (}
\NormalTok{    key: "potato",}
\NormalTok{    short: "potato",}
\NormalTok{    // "plural" will be used when "short" should be pluralized}
\NormalTok{    plural: "potatoes",}
\NormalTok{    description: [\#lorem(10)],}
\NormalTok{  ),}
\NormalTok{  // a term with a long plural}
\NormalTok{  (}
\NormalTok{    key: "dm",}
\NormalTok{    short: "DM",}
\NormalTok{    long: "diagonal matrix",}
\NormalTok{    // "longplural" will be used when "long" should be pluralized}
\NormalTok{    longplural: "diagonal matrices",}
\NormalTok{    description: "Probably some math stuff idk",}
\NormalTok{  ),}
\NormalTok{)}
\end{Highlighting}
\end{Shaded}

Then the terms are passed as a list to \texttt{\ register-glossary\ }

\begin{Shaded}
\begin{Highlighting}[]
\NormalTok{\#register{-}glossary(entry{-}list)}
\end{Highlighting}
\end{Shaded}

\subsubsection{Printing the glossary}\label{printing-the-glossary}

Now, you can display the glossary using the \texttt{\ print-glossary\ }
function.

\begin{Shaded}
\begin{Highlighting}[]
\NormalTok{\#print{-}glossary(entry{-}list)}
\end{Highlighting}
\end{Shaded}

By default, the terms that are not referenced in the document are not
shown in the glossary, you can force their appearance by setting the
\texttt{\ show-all\ } argument to true.

You can also disable the back-references by setting the parameter
\texttt{\ disable-back-references\ } to \texttt{\ true\ } .

By default, group breaks use \texttt{\ linebreaks\ } . This behaviour
can be changed by setting the \texttt{\ user-group-break\ } parameter to
\texttt{\ pagebreak()\ } , or \texttt{\ colbreak()\ } , or any other
function that returns the \texttt{\ content\ } you want.

You can call this function from anywhere in your document.

\subsubsection{Referencing terms.}\label{referencing-terms.}

Referencing terms is done using the key of the terms using the
\texttt{\ gls\ } function or the reference syntax.

\begin{Shaded}
\begin{Highlighting}[]
\NormalTok{// referencing the OIDC term using gls}
\NormalTok{\#gls("oidc")}
\NormalTok{// displaying the long form forcibly}
\NormalTok{\#gls("oidc", long: true)}

\NormalTok{// referencing the OIDC term using the reference syntax}
\NormalTok{@oidc}
\end{Highlighting}
\end{Shaded}

\paragraph{Handling plurals}\label{handling-plurals}

You can use the \texttt{\ glspl\ } function and the references
supplements to pluralize terms. The \texttt{\ plural\ } key will be used
when \texttt{\ short\ } should be pluralized and \texttt{\ longplural\ }
will be used when \texttt{\ long\ } should be pluralized. If the
\texttt{\ plural\ } key is missing then glossarium will add an ‘s’
at the end of the short form as a fallback.

\begin{Shaded}
\begin{Highlighting}[]
\NormalTok{\#glspl("potato")}
\end{Highlighting}
\end{Shaded}

Please look at the examples regarding plurals.

\paragraph{Overriding the text shown}\label{overriding-the-text-shown}

You can also override the text displayed by setting the
\texttt{\ display\ } argument.

\begin{Shaded}
\begin{Highlighting}[]
\NormalTok{\#gls("oidc", display: "whatever you want")}
\end{Highlighting}
\end{Shaded}

\subsection{Final tips}\label{final-tips}

I recommend setting a show rule for the links to that your readers
understand that they can click on the references to go to the term in
the glossary.

\begin{Shaded}
\begin{Highlighting}[]
\NormalTok{\#show link: set text(fill: blue.darken(60\%))}
\NormalTok{// links are now blue !}
\end{Highlighting}
\end{Shaded}

\subsubsection{How to add}\label{how-to-add}

Copy this into your project and use the import as
\texttt{\ glossarium\ }

\begin{verbatim}
#import "@preview/glossarium:0.5.1"
\end{verbatim}

\includesvg[width=0.16667in,height=0.16667in]{/assets/icons/16-copy.svg}

Check the docs for
\href{https://typst.app/docs/reference/scripting/\#packages}{more
information on how to import packages} .

\subsubsection{About}\label{about}

\begin{description}
\tightlist
\item[Author s :]
slashformotion \& Dherse
\item[License:]
MIT
\item[Current version:]
0.5.1
\item[Last updated:]
October 28, 2024
\item[First released:]
July 31, 2023
\item[Archive size:]
10.5 kB
\href{https://packages.typst.org/preview/glossarium-0.5.1.tar.gz}{\pandocbounded{\includesvg[keepaspectratio]{/assets/icons/16-download.svg}}}
\item[Repository:]
\href{https://github.com/typst-community/glossarium}{GitHub}
\end{description}

\subsubsection{Where to report issues?}\label{where-to-report-issues}

This package is a project of slashformotion and Dherse . Report issues
on \href{https://github.com/typst-community/glossarium}{their
repository} . You can also try to ask for help with this package on the
\href{https://forum.typst.app}{Forum} .

Please report this package to the Typst team using the
\href{https://typst.app/contact}{contact form} if you believe it is a
safety hazard or infringes upon your rights.

\phantomsection\label{versions}
\subsubsection{Version history}\label{version-history}

\begin{longtable}[]{@{}ll@{}}
\toprule\noalign{}
Version & Release Date \\
\midrule\noalign{}
\endhead
\bottomrule\noalign{}
\endlastfoot
0.5.1 & October 28, 2024 \\
\href{https://typst.app/universe/package/glossarium/0.5.0/}{0.5.0} &
October 14, 2024 \\
\href{https://typst.app/universe/package/glossarium/0.4.2/}{0.4.2} &
October 7, 2024 \\
\href{https://typst.app/universe/package/glossarium/0.4.1/}{0.4.1} & May
29, 2024 \\
\href{https://typst.app/universe/package/glossarium/0.4.0/}{0.4.0} &
April 29, 2024 \\
\href{https://typst.app/universe/package/glossarium/0.3.0/}{0.3.0} &
April 8, 2024 \\
\href{https://typst.app/universe/package/glossarium/0.2.6/}{0.2.6} &
January 29, 2024 \\
\href{https://typst.app/universe/package/glossarium/0.2.5/}{0.2.5} &
December 3, 2023 \\
\href{https://typst.app/universe/package/glossarium/0.2.4/}{0.2.4} &
November 16, 2023 \\
\href{https://typst.app/universe/package/glossarium/0.2.3/}{0.2.3} &
October 30, 2023 \\
\href{https://typst.app/universe/package/glossarium/0.2.2/}{0.2.2} &
September 16, 2023 \\
\href{https://typst.app/universe/package/glossarium/0.2.1/}{0.2.1} &
September 3, 2023 \\
\href{https://typst.app/universe/package/glossarium/0.2.0/}{0.2.0} &
August 19, 2023 \\
\href{https://typst.app/universe/package/glossarium/0.1.0/}{0.1.0} &
July 31, 2023 \\
\end{longtable}

Typst GmbH did not create this package and cannot guarantee correct
functionality of this package or compatibility with any version of the
Typst compiler or app.


\section{Package List LaTeX/stv-vub-huisstijl.tex}
\title{typst.app/universe/package/stv-vub-huisstijl}

\phantomsection\label{banner}
\phantomsection\label{template-thumbnail}
\pandocbounded{\includegraphics[keepaspectratio]{https://packages.typst.org/preview/thumbnails/stv-vub-huisstijl-0.1.0-small.webp}}

\section{stv-vub-huisstijl}\label{stv-vub-huisstijl}

{ 0.1.0 }

An unofficial template to get the look of the Vrije Universiteit Brussel
(VUB) huisstijl in Typst.

\href{/app?template=stv-vub-huisstijl&version=0.1.0}{Create project in
app}

\phantomsection\label{readme}
An unofficial template to get the look of the
\href{https://www.vub.be/}{Vrije Universiteit Brussel (VUB)} huisstijl
in Typst based on \href{https://gitlab.com/rubdos/texlive-vub}{this
LaTeX template}

\subsection{Getting Started}\label{getting-started}

You can choose “Start from template� in the web app, and search for
\texttt{\ vub-huisstijl\ } .

If you are running Typst locally, you can use the following command to
initialize the template:

\begin{Shaded}
\begin{Highlighting}[]
\NormalTok{typst init @preview/stv{-}vub{-}huisstijl:0.1.0}
\end{Highlighting}
\end{Shaded}

\subsubsection{Fonts}\label{fonts}

The package makes use of the “TeX Gyre Adventor� font, with
“Roboto� as a fallback. These should be installed for the title page
to look right. They are available for free, and also come bundled with
texlive.

\subsection{Note}\label{note}

This only provides a template for a thesis title page, not for slides.
That can be added in the future.

\subsection{About the name}\label{about-the-name}

St V ( \href{https://en.wikipedia.org/wiki/Saint_Verhaegen}{Saint
Verhaegen} ) is an important part of the folklore of the VUB and the
ULB.

\href{/app?template=stv-vub-huisstijl&version=0.1.0}{Create project in
app}

\subsubsection{How to use}\label{how-to-use}

Click the button above to create a new project using this template in
the Typst app.

You can also use the Typst CLI to start a new project on your computer
using this command:

\begin{verbatim}
typst init @preview/stv-vub-huisstijl:0.1.0
\end{verbatim}

\includesvg[width=0.16667in,height=0.16667in]{/assets/icons/16-copy.svg}

\subsubsection{About}\label{about}

\begin{description}
\tightlist
\item[Author :]
\href{https://github.com/WannesMalfait}{Wannes Malfait}
\item[License:]
MIT
\item[Current version:]
0.1.0
\item[Last updated:]
October 21, 2024
\item[First released:]
October 21, 2024
\item[Archive size:]
8.02 kB
\href{https://packages.typst.org/preview/stv-vub-huisstijl-0.1.0.tar.gz}{\pandocbounded{\includesvg[keepaspectratio]{/assets/icons/16-download.svg}}}
\item[Repository:]
\href{https://github.com/WannesMalfait/vub-huisstijl-typst/}{GitHub}
\item[Categor y :]
\begin{itemize}
\tightlist
\item[]
\item
  \pandocbounded{\includesvg[keepaspectratio]{/assets/icons/16-mortarboard.svg}}
  \href{https://typst.app/universe/search/?category=thesis}{Thesis}
\end{itemize}
\end{description}

\subsubsection{Where to report issues?}\label{where-to-report-issues}

This template is a project of Wannes Malfait . Report issues on
\href{https://github.com/WannesMalfait/vub-huisstijl-typst/}{their
repository} . You can also try to ask for help with this template on the
\href{https://forum.typst.app}{Forum} .

Please report this template to the Typst team using the
\href{https://typst.app/contact}{contact form} if you believe it is a
safety hazard or infringes upon your rights.

\phantomsection\label{versions}
\subsubsection{Version history}\label{version-history}

\begin{longtable}[]{@{}ll@{}}
\toprule\noalign{}
Version & Release Date \\
\midrule\noalign{}
\endhead
\bottomrule\noalign{}
\endlastfoot
0.1.0 & October 21, 2024 \\
\end{longtable}

Typst GmbH did not create this template and cannot guarantee correct
functionality of this template or compatibility with any version of the
Typst compiler or app.


\section{Package List LaTeX/aero-check.tex}
\title{typst.app/universe/package/aero-check}

\phantomsection\label{banner}
\phantomsection\label{template-thumbnail}
\pandocbounded{\includegraphics[keepaspectratio]{https://packages.typst.org/preview/thumbnails/aero-check-0.1.1-small.webp}}

\section{aero-check}\label{aero-check}

{ 0.1.1 }

A simple template to create checklists with an aviation inspired style.

{ } Featured Template

\href{/app?template=aero-check&version=0.1.1}{Create project in app}

\phantomsection\label{readme}
\pandocbounded{\includegraphics[keepaspectratio]{https://img.shields.io/github/v/release/TomVer99/Typst-checklist-template?style=flat-square}}
\pandocbounded{\includegraphics[keepaspectratio]{https://img.shields.io/github/stars/TomVer99/Typst-checklist-template?style=flat-square}}

\pandocbounded{\includegraphics[keepaspectratio]{https://img.shields.io/maintenance/Yes/2024?style=flat-square}}

This template is meant to create checklists with a style inspired by
aviation checklists.

It includes 2 different styles!

\subsection{Usage}\label{usage}

Start your checklist with the following code:

\begin{Shaded}
\begin{Highlighting}[]
\NormalTok{\#import "@preview/aero{-}check:0.1.1": *}

\NormalTok{\#show: checklist.with(}
\NormalTok{  title: "Title",}
\NormalTok{  // disclaimer: "",}
\NormalTok{  // 0 or 1}
\NormalTok{  // style: 0,}
\NormalTok{)}
\end{Highlighting}
\end{Shaded}

You can then add items to your checklist with the following code:

\begin{Shaded}
\begin{Highlighting}[]
\NormalTok{\#topic("Topic")[}
\NormalTok{  \#section("Section")[}
\NormalTok{    \#step("Step", "Check")}
\NormalTok{    \#step("Step", "Check")}
\NormalTok{    \#step("Step", "Check")}
\NormalTok{    \#step("Step", "Check")}
\NormalTok{  ]}
\NormalTok{\#section("Section")[}
\NormalTok{    \#step("Step", "Check")}
\NormalTok{    \#step("Step", "Check")}
\NormalTok{    \#step("Step", "Check")}
\NormalTok{    \#step("Step", "Check")}
\NormalTok{  ]}
\NormalTok{]}
\end{Highlighting}
\end{Shaded}

And you can use \texttt{\ \#colbreak()\ } to add a column break.

\href{/app?template=aero-check&version=0.1.1}{Create project in app}

\subsubsection{How to use}\label{how-to-use}

Click the button above to create a new project using this template in
the Typst app.

You can also use the Typst CLI to start a new project on your computer
using this command:

\begin{verbatim}
typst init @preview/aero-check:0.1.1
\end{verbatim}

\includesvg[width=0.16667in,height=0.16667in]{/assets/icons/16-copy.svg}

\subsubsection{About}\label{about}

\begin{description}
\tightlist
\item[Author :]
TomVer99
\item[License:]
MIT
\item[Current version:]
0.1.1
\item[Last updated:]
September 14, 2024
\item[First released:]
May 13, 2024
\item[Minimum Typst version:]
0.11.0
\item[Archive size:]
2.59 kB
\href{https://packages.typst.org/preview/aero-check-0.1.1.tar.gz}{\pandocbounded{\includesvg[keepaspectratio]{/assets/icons/16-download.svg}}}
\item[Repository:]
\href{https://github.com/TomVer99/Typst-checklist-template}{GitHub}
\item[Categor y :]
\begin{itemize}
\tightlist
\item[]
\item
  \pandocbounded{\includesvg[keepaspectratio]{/assets/icons/16-hammer.svg}}
  \href{https://typst.app/universe/search/?category=utility}{Utility}
\end{itemize}
\end{description}

\subsubsection{Where to report issues?}\label{where-to-report-issues}

This template is a project of TomVer99 . Report issues on
\href{https://github.com/TomVer99/Typst-checklist-template}{their
repository} . You can also try to ask for help with this template on the
\href{https://forum.typst.app}{Forum} .

Please report this template to the Typst team using the
\href{https://typst.app/contact}{contact form} if you believe it is a
safety hazard or infringes upon your rights.

\phantomsection\label{versions}
\subsubsection{Version history}\label{version-history}

\begin{longtable}[]{@{}ll@{}}
\toprule\noalign{}
Version & Release Date \\
\midrule\noalign{}
\endhead
\bottomrule\noalign{}
\endlastfoot
0.1.1 & September 14, 2024 \\
\href{https://typst.app/universe/package/aero-check/0.1.0/}{0.1.0} & May
13, 2024 \\
\end{longtable}

Typst GmbH did not create this template and cannot guarantee correct
functionality of this template or compatibility with any version of the
Typst compiler or app.


\section{Package List LaTeX/sweet-graduate-resume.tex}
\title{typst.app/universe/package/sweet-graduate-resume}

\phantomsection\label{banner}
\phantomsection\label{template-thumbnail}
\pandocbounded{\includegraphics[keepaspectratio]{https://packages.typst.org/preview/thumbnails/sweet-graduate-resume-0.1.0-small.webp}}

\section{sweet-graduate-resume}\label{sweet-graduate-resume}

{ 0.1.0 }

A simple graduate student resume template

\href{/app?template=sweet-graduate-resume&version=0.1.0}{Create project
in app}

\phantomsection\label{readme}
A basic resume template in typst.

To compile/watch, make sure you pass the argument
\texttt{\ -\/-font-path\ ./fonts/\ } to typst-cli.

If you use the helix editor, a configuration has been given in the
repository. This auto-compiles your document on saving and you can
preview the results in real time using a pdf viewer. Also provides
autocomplete, code renaming, and other cool (LSP) features.

Make sure you have tinymist and typstyle installed before using the
helix config.

\subsection{Preview}\label{preview}

\pandocbounded{\includegraphics[keepaspectratio]{https://github.com/typst/packages/raw/main/packages/preview/sweet-graduate-resume/0.1.0/screenshot.png}}

\subsection{LICENSE}\label{license}

The FontAwesome Free/Brand fonts are licensed under
\href{https://github.com/FortAwesome/Font-Awesome?tab=License-1-ov-file}{Font
Awesome Free License}

The Codeberg SVG in the svg directory are licensed under
\href{https://codeberg.org/Codeberg/Design/src/commit/ac514aa9aaa2457d4af3c3e13df3ab136d22a49a/LICENSE}{Creative
Commons CC0}

For the rest, see
\href{https://github.com/typst/packages/raw/main/packages/preview/sweet-graduate-resume/0.1.0/LICENSE}{LICENSE}
.

\subsection{Fonts}\label{fonts}

For fonts, please install the fonts from the repository of the project
in codeberg.

\href{/app?template=sweet-graduate-resume&version=0.1.0}{Create project
in app}

\subsubsection{How to use}\label{how-to-use}

Click the button above to create a new project using this template in
the Typst app.

You can also use the Typst CLI to start a new project on your computer
using this command:

\begin{verbatim}
typst init @preview/sweet-graduate-resume:0.1.0
\end{verbatim}

\includesvg[width=0.16667in,height=0.16667in]{/assets/icons/16-copy.svg}

\subsubsection{About}\label{about}

\begin{description}
\tightlist
\item[Author :]
\href{https://innocent_zero.codeberg.page}{innocentzero (Md Isfarul
Haque)}
\item[License:]
MIT
\item[Current version:]
0.1.0
\item[Last updated:]
July 4, 2024
\item[First released:]
July 4, 2024
\item[Archive size:]
47.6 kB
\href{https://packages.typst.org/preview/sweet-graduate-resume-0.1.0.tar.gz}{\pandocbounded{\includesvg[keepaspectratio]{/assets/icons/16-download.svg}}}
\item[Repository:]
\href{https://codeberg.org/innocent_zero/typst-resume}{Codeberg}
\item[Discipline :]
\begin{itemize}
\tightlist
\item[]
\item
  \href{https://typst.app/universe/search/?discipline=engineering}{Engineering}
\end{itemize}
\item[Categor y :]
\begin{itemize}
\tightlist
\item[]
\item
  \pandocbounded{\includesvg[keepaspectratio]{/assets/icons/16-user.svg}}
  \href{https://typst.app/universe/search/?category=cv}{CV}
\end{itemize}
\end{description}

\subsubsection{Where to report issues?}\label{where-to-report-issues}

This template is a project of innocentzero (Md Isfarul Haque) . Report
issues on \href{https://codeberg.org/innocent_zero/typst-resume}{their
repository} . You can also try to ask for help with this template on the
\href{https://forum.typst.app}{Forum} .

Please report this template to the Typst team using the
\href{https://typst.app/contact}{contact form} if you believe it is a
safety hazard or infringes upon your rights.

\phantomsection\label{versions}
\subsubsection{Version history}\label{version-history}

\begin{longtable}[]{@{}ll@{}}
\toprule\noalign{}
Version & Release Date \\
\midrule\noalign{}
\endhead
\bottomrule\noalign{}
\endlastfoot
0.1.0 & July 4, 2024 \\
\end{longtable}

Typst GmbH did not create this template and cannot guarantee correct
functionality of this template or compatibility with any version of the
Typst compiler or app.


\section{Package List LaTeX/t4t.tex}
\title{typst.app/universe/package/t4t}

\phantomsection\label{banner}
\section{t4t}\label{t4t}

{ 0.3.2 }

A utility package for typst package authors

\phantomsection\label{readme}
\begin{quote}
A utility package for typst package authors.
\end{quote}

\textbf{Tools for Typst} ( \texttt{\ t4t\ } in short) is a utility
package for
\href{https://github.com/typst/packages/raw/main/packages/preview/t4t/0.3.2/typst/typst}{Typst}
package and template authors. It provides solutions to some recurring
tasks in package development.

The package can be imported or any useful parts of it copied into a
project. It is perfectly fine to treat \texttt{\ t4t\ } as a snippet
collection and to pick and choose only some useful functions. For this
reason, most functions are implemented without further dependencies.

Hopefully, this collection will grow over time with \emph{Typst} to
provide solutions for common problems.

\subsection{Usage}\label{usage}

Either import the package from the Typst preview repository:

\begin{Shaded}
\begin{Highlighting}[]
\NormalTok{\#import }\StringTok{"@preview/t4t:0.3.2"}\OperatorTok{:} \OperatorTok{*}
\end{Highlighting}
\end{Shaded}

If only a few functions from \texttt{\ t4t\ } are needed, simply copy
the necessary code to the beginning of the document.

\subsection{Reference}\label{reference}

\begin{center}\rule{0.5\linewidth}{0.5pt}\end{center}

\textbf{Note:} This reference might be out of date. Please refer to the
manual for a complete overview of all functions.

\begin{center}\rule{0.5\linewidth}{0.5pt}\end{center}

The functions are categorized into different submodules that can be
imported separately.

The modules are:

\begin{itemize}
\tightlist
\item
  \texttt{\ is\ }
\item
  \texttt{\ def\ }
\item
  \texttt{\ assert\ }
\item
  \texttt{\ alias\ }
\item
  \texttt{\ math\ }
\item
  \texttt{\ get\ }
\end{itemize}

Any or all modules can be imported the usual way:

\begin{Shaded}
\begin{Highlighting}[]
\CommentTok{// Import as "t4t"}
\NormalTok{\#import }\StringTok{"@preview/t4t:0.3.2"}
\CommentTok{// Import all modules}
\NormalTok{\#import }\StringTok{"@preview/t4t:0.3.2"}\OperatorTok{:} \OperatorTok{*}
\CommentTok{// Import specific modules}
\NormalTok{\#import }\StringTok{"@preview/t4t:0.3.2"}\OperatorTok{:}\NormalTok{ is}\OperatorTok{,}\NormalTok{ def}
\end{Highlighting}
\end{Shaded}

In general, the main value is passed last to the utility functions.
\texttt{\ \#def.if-none()\ } , for example, takes the default value
first and the value to test second. This is somewhat counterintuitive at
first, but allows the use of \texttt{\ .with()\ } to generate derivative
functions:

\begin{Shaded}
\begin{Highlighting}[]
\NormalTok{\#let is}\OperatorTok{{-}}\NormalTok{foo }\OperatorTok{=}\NormalTok{ eq}\OperatorTok{.}\FunctionTok{with}\NormalTok{(}\StringTok{"foo"}\NormalTok{)}
\end{Highlighting}
\end{Shaded}

\subsubsection{Test functions}\label{test-functions}

\begin{Shaded}
\begin{Highlighting}[]
\NormalTok{\#import }\StringTok{"@preview/t4t:0.3.2"}\OperatorTok{:}\NormalTok{ is}
\end{Highlighting}
\end{Shaded}

These functions provide shortcuts to common tests like
\texttt{\ \#is.eq()\ } . Some of these are not shorter as writing pure
Typst code (e.g. \texttt{\ a\ ==\ b\ } ), but can easily be used in
\texttt{\ .any()\ } or \texttt{\ .find()\ } calls:

\begin{Shaded}
\begin{Highlighting}[]
\CommentTok{// check all values for none}
\ControlFlowTok{if}\NormalTok{ some}\OperatorTok{{-}}\NormalTok{array}\OperatorTok{.}\FunctionTok{any}\NormalTok{(is}\OperatorTok{{-}}\NormalTok{none) \{}
    \OperatorTok{...}
\NormalTok{\}}

\CommentTok{// find first not none value}
\KeywordTok{let}\NormalTok{ x }\OperatorTok{=}\NormalTok{ (none}\OperatorTok{,}\NormalTok{ none}\OperatorTok{,} \DecValTok{5}\OperatorTok{,}\NormalTok{ none)}\OperatorTok{.}\FunctionTok{find}\NormalTok{(is}\OperatorTok{.}\AttributeTok{not}\OperatorTok{{-}}\NormalTok{none)}

\CommentTok{// find position of a value}
\KeywordTok{let}\NormalTok{ pos}\OperatorTok{{-}}\NormalTok{bar }\OperatorTok{=}\NormalTok{ args}\OperatorTok{.}\FunctionTok{pos}\NormalTok{()}\OperatorTok{.}\FunctionTok{position}\NormalTok{(is}\OperatorTok{.}\AttributeTok{eq}\OperatorTok{.}\FunctionTok{with}\NormalTok{(}\StringTok{"|"}\NormalTok{))}
\end{Highlighting}
\end{Shaded}

There are two exceptions: \texttt{\ is-none\ } and \texttt{\ is-auto\ }
. Since keywords can’t be used as function names, the \texttt{\ is\ }
module can’t define a function to do \texttt{\ is.none()\ } .
Therefore the methods \texttt{\ is-none\ } and \texttt{\ is-auto\ } are
provided in the base module of \texttt{\ t4t\ } :

\begin{Shaded}
\begin{Highlighting}[]
\NormalTok{\#import }\StringTok{"@preview/t4t:0.3.2"}\OperatorTok{:}\NormalTok{ is}\OperatorTok{{-}}\NormalTok{none}\OperatorTok{,}\NormalTok{ is}\OperatorTok{{-}}\NormalTok{auto}
\end{Highlighting}
\end{Shaded}

The \texttt{\ is\ } submodule still has these tests, but under different
names ( \texttt{\ is.n\ } and \texttt{\ is.non\ } for \texttt{\ none\ }
and \texttt{\ is.a\ } and \texttt{\ is.aut\ } for \texttt{\ auto\ } ).

\begin{itemize}
\item
  \texttt{\ \#is.neq(\ test\ )\ } : Creates a new test function that is
  \texttt{\ true\ } when \texttt{\ test\ } is \texttt{\ false\ } . Can
  be used to create negations of tests like
  \texttt{\ \#let\ not-raw\ =\ is.neg(is.raw)\ } .
\item
  \texttt{\ \#is.eq(\ a,\ b\ )\ } : Tests if values \texttt{\ a\ } and
  \texttt{\ b\ } are equal.
\item
  \texttt{\ \#is.neq(\ a,\ b\ )\ } : Tests if values \texttt{\ a\ } and
  \texttt{\ b\ } are not equal.
\item
  \texttt{\ \#is.n(\ ..values\ )\ } : Tests if any of the passed
  \texttt{\ values\ } is \texttt{\ none\ } .
\item
  \texttt{\ \#is.non(\ ..values\ )\ } : Alias for \texttt{\ is.n\ } .
\item
  \texttt{\ \#is.not-none(\ ..values\ )\ } : Tests if all of the passed
  \texttt{\ values\ } are not \texttt{\ none\ } .
\item
  \texttt{\ \#is.not-n(\ ..values\ )\ } : Alias for
  \texttt{\ is.not-none\ } .
\item
  \texttt{\ \#is.a(\ ..values\ )\ } : Tests if any of the passed
  \texttt{\ values\ } is \texttt{\ auto\ } .
\item
  \texttt{\ \#is.aut(\ ..values\ )\ } : Alias for \texttt{\ is-a\ } .
\item
  \texttt{\ \#is.not-auto(\ ..values\ )\ } : Tests if all of the passed
  \texttt{\ values\ } are not \texttt{\ auto\ } .
\item
  \texttt{\ \#is.not-a(\ ..values\ )\ } : Alias for
  \texttt{\ is.not-auto\ } .
\item
  \texttt{\ \#is.empty(\ value\ )\ } : Tests if \texttt{\ value\ } is
  \emph{empty} . A value is considered \emph{empty} if it is an empty
  array, dictionary or string or \texttt{\ none\ } otherwise.
\item
  \texttt{\ \#is.not-empty(\ value\ )\ } : Tests if \texttt{\ value\ }
  is not empty.
\item
  \texttt{\ \#is.any(\ ..compare,\ value\ )\ } : Tests if
  \texttt{\ value\ } is equal to any one of the other passed-in values.
\item
  \texttt{\ \#is.not-any(\ ..compare,\ value)\ } : Tests if
  \texttt{\ value\ } is not equal to any one of the other passed-in
  values.
\item
  \texttt{\ \#is.has(\ ..keys,\ value\ )\ } : Tests if
  \texttt{\ value\ } contains all the passed \texttt{\ keys\ } . Either
  as keys in a dictionary or elements in an array. If \texttt{\ value\ }
  is neither of those types, \texttt{\ false\ } is returned.
\item
  \texttt{\ \#is.type(\ t,\ value\ )\ } : Tests if \texttt{\ value\ } is
  of type \texttt{\ t\ } .
\item
  \texttt{\ \#is.dict(\ value\ )\ } : Tests if \texttt{\ value\ } is a
  dictionary.
\item
  \texttt{\ \#is.arr(\ value\ )\ } : Tests if \texttt{\ value\ } is an
  array.
\item
  \texttt{\ \#is.content(\ value\ )\ } : Tests if \texttt{\ value\ } is
  of type content.
\item
  \texttt{\ \#is.color(\ value\ )\ } : Tests if \texttt{\ value\ } is a
  color.
\item
  \texttt{\ \#is.stroke(\ value\ )\ } : Tests if \texttt{\ value\ } is a
  stroke.
\item
  \texttt{\ \#is.loc(\ value\ )\ } : Tests if \texttt{\ value\ } is a
  location.
\item
  \texttt{\ \#is.bool(\ value\ )\ } : Tests if \texttt{\ value\ } is a
  boolean.
\item
  \texttt{\ \#is.str(\ value\ )\ } : Tests if \texttt{\ value\ } is a
  string.
\item
  \texttt{\ \#is.int(\ value\ )\ } : Tests if \texttt{\ value\ } is an
  integer.
\item
  \texttt{\ \#is.float(\ value\ )\ } : Tests if \texttt{\ value\ } is a
  float.
\item
  \texttt{\ \#is.num(\ value\ )\ } : Tests if \texttt{\ value\ } is a
  numeric value ( \texttt{\ integer\ } or \texttt{\ float\ } ).
\item
  \texttt{\ \#is.frac(\ value\ )\ } : Tests if \texttt{\ value\ } is a
  fraction.
\item
  \texttt{\ \#is.length(\ value\ )\ } : Tests if \texttt{\ value\ } is a
  length.
\item
  \texttt{\ \#is.rlength(\ value\ )\ } : Tests if \texttt{\ value\ } is
  a relative length.
\item
  \texttt{\ \#is.ratio(\ value\ )\ } : Tests if \texttt{\ value\ } is a
  ratio.
\item
  \texttt{\ \#is.align(\ value\ )\ } : Tests if \texttt{\ value\ } is an
  alignment.
\item
  \texttt{\ \#is.align2d(\ value\ )\ } : Tests if \texttt{\ value\ } is
  a 2d alignment.
\item
  \texttt{\ \#is.func(\ value\ )\ } : Tests if \texttt{\ value\ } is a
  function.
\item
  \texttt{\ \#is.any-type(\ ..types,\ value\ )\ } : Tests if
  \texttt{\ value\ } has any of the passed-in types.
\item
  \texttt{\ \#is.same-type(\ ..values\ )\ } : Tests if all passed-in
  values have the same type.
\item
  \texttt{\ \#is.all-of-type(\ t,\ ..values\ )\ } : Tests if all of the
  passed-in values have the type \texttt{\ t\ } .
\item
  \texttt{\ \#is.none-of-type(\ t,\ ..values\ )\ } : Tests if none of
  the passed-in values has the type \texttt{\ t\ } .
\item
  \texttt{\ \#is.one-not-none(\ ..values\ )\ } : Checks, if at least one
  value in \texttt{\ values\ } is not equal to \texttt{\ none\ } .
  Useful for checking multiple optional arguments for a valid value:
  \texttt{\ \#if\ is.one-not-none(..args.pos())\ {[}\ \#args.pos().find(is.not-none)\ {]}\ }
\item
  \texttt{\ \#is.elem(\ func,\ value\ )\ } : Tests if \texttt{\ value\ }
  is a content element with \texttt{\ value.func()\ ==\ func\ } . If
  \texttt{\ func\ } is a string, \texttt{\ value\ } will be compared to
  \texttt{\ repr(value.func())\ } instead.

  Both of these effectively do the same:

\begin{Shaded}
\begin{Highlighting}[]
\NormalTok{\#is}\OperatorTok{.}\FunctionTok{elem}\NormalTok{(raw}\OperatorTok{,}\NormalTok{ some\_content)}
\NormalTok{\#is}\OperatorTok{.}\FunctionTok{elem}\NormalTok{(}\StringTok{"raw"}\OperatorTok{,}\NormalTok{ some\_content)}
\end{Highlighting}
\end{Shaded}
\item
  \texttt{\ \#is.sequence(\ value\ )\ } : Tests if \texttt{\ value\ } is
  a sequence of content.
\item
  \texttt{\ \#is.raw(\ value\ )\ } : Tests if \texttt{\ value\ } is a
  raw element.
\item
  \texttt{\ \#is.table(\ value\ )\ } : Tests if \texttt{\ value\ } is a
  table element.
\item
  \texttt{\ \#is.list(\ value\ )\ } : Tests if \texttt{\ value\ } is a
  list element.
\item
  \texttt{\ \#is.enum(\ value\ )\ } : Tests if \texttt{\ value\ } is an
  enum element.
\item
  \texttt{\ \#is.terms(\ value\ )\ } : Tests if \texttt{\ value\ } is a
  terms element.
\item
  \texttt{\ \#is.cols(\ value\ )\ } : Tests if \texttt{\ value\ } is a
  columns element.
\item
  \texttt{\ \#is.grid(\ value\ )\ } : Tests if \texttt{\ value\ } is a
  grid element.
\item
  \texttt{\ \#is.stack(\ value\ )\ } : Tests if \texttt{\ value\ } is a
  stack element.
\item
  \texttt{\ \#is.label(\ value\ )\ } : Tests if \texttt{\ value\ } is of
  type \texttt{\ label\ } .
\end{itemize}

\subsubsection{Default values}\label{default-values}

\begin{Shaded}
\begin{Highlighting}[]
\NormalTok{\#import }\StringTok{"@preview/t4t:0.3.2"}\OperatorTok{:}\NormalTok{ def}
\end{Highlighting}
\end{Shaded}

These functions perform a test to decide if a given \texttt{\ value\ }
is \emph{invalid} . If the test \emph{passes} , the \texttt{\ default\ }
is returned, the \texttt{\ value\ } otherwise.

Almost all functions support an optional \texttt{\ do\ } argument, to be
set to a function of one argument, that will be applied to the value if
the test fails. For example:

\begin{Shaded}
\begin{Highlighting}[]
\CommentTok{// Sets date to a datetime from an optional}
\CommentTok{// string argument in the format "YYYY{-}MM{-}DD"}
\NormalTok{\#let date }\OperatorTok{=}\NormalTok{ def}\OperatorTok{.}\AttributeTok{if}\OperatorTok{{-}}\FunctionTok{none}\NormalTok{(}
\NormalTok{    datetime}\OperatorTok{.}\FunctionTok{today}\NormalTok{()}\OperatorTok{,}   \CommentTok{// default}
\NormalTok{    passed\_date}\OperatorTok{,}        \CommentTok{// passed{-}in argument}
    \ControlFlowTok{do}\OperatorTok{:}\NormalTok{ (d) }\OperatorTok{\textgreater{}=}\NormalTok{ \{     }\CommentTok{// post{-}processor}
\NormalTok{        d }\OperatorTok{=}\NormalTok{ d}\OperatorTok{.}\FunctionTok{split}\NormalTok{(}\StringTok{"{-}"}\NormalTok{)}
        \FunctionTok{datetime}\NormalTok{(year}\OperatorTok{=}\NormalTok{d[}\DecValTok{0}\NormalTok{]}\OperatorTok{,}\NormalTok{ month}\OperatorTok{=}\NormalTok{d[}\DecValTok{1}\NormalTok{]}\OperatorTok{,}\NormalTok{ day}\OperatorTok{=}\NormalTok{d[}\DecValTok{2}\NormalTok{])}
\NormalTok{    \}}
\NormalTok{)}
\end{Highlighting}
\end{Shaded}

\begin{itemize}
\tightlist
\item
  \texttt{\ \#def.if-true(\ test,\ default,\ do:none,\ value\ )\ } :
  Returns \texttt{\ default\ } if \texttt{\ test\ } is \texttt{\ true\ }
  , \texttt{\ value\ } otherwise.
\item
  \texttt{\ \#def.if-false(\ test,\ default,\ do:none,\ value\ )\ } :
  Returns \texttt{\ default\ } if \texttt{\ test\ } is
  \texttt{\ false\ } , \texttt{\ value\ } otherwise.
\item
  \texttt{\ \#def.if-none(\ default,\ do:none,\ value\ )\ } : Returns
  \texttt{\ default\ } if \texttt{\ value\ } is \texttt{\ none\ } ,
  \texttt{\ value\ } otherwise.
\item
  \texttt{\ \#def.if-auto(\ default,\ do:none,\ value\ )\ } : Returns
  \texttt{\ default\ } if \texttt{\ value\ } is \texttt{\ auto\ } ,
  \texttt{\ value\ } otherwise.
\item
  \texttt{\ \#def.if-any(\ ..compare,\ default,\ do:none,\ value\ )\ } :
  Returns \texttt{\ default\ } if \texttt{\ value\ } is equal to any of
  the passed-in values, \texttt{\ value\ } otherwise. (
  \texttt{\ \#def.if-any(none,\ auto,\ 1pt,\ width)\ } )
\item
  \texttt{\ \#def.if-not-any(\ ..compare,\ default,\ do:none,\ value\ )\ }
  : Returns \texttt{\ default\ } if \texttt{\ value\ } is not equal to
  any of the passed-in values, \texttt{\ value\ } otherwise. (
  \texttt{\ \#def.if-not-any(left,\ right,\ top,\ bottom,\ position)\ }
  )
\item
  \texttt{\ \#def.if-empty(\ default,\ do:none,\ value\ )\ } : Returns
  \texttt{\ default\ } if \texttt{\ value\ } is \emph{empty} ,
  \texttt{\ value\ } otherwise.
\item
  \texttt{\ \#def.as-arr(\ ..values\ )\ } : Always returns an array
  containing all \texttt{\ values\ } . Any arrays in \texttt{\ values\ }
  will be flattened into the result. This is useful for arguments, that
  can have one element or an array of elements:
  \texttt{\ \#def.as-arr(author).join(",\ ")\ } .
\end{itemize}

\subsubsection{Assertions}\label{assertions}

\begin{Shaded}
\begin{Highlighting}[]
\NormalTok{\#import }\StringTok{"@preview/t4t:0.3.2"}\OperatorTok{:}\NormalTok{ assert}
\end{Highlighting}
\end{Shaded}

This submodule overloads the default \texttt{\ assert\ } function and
provides more asserts to quickly check if given values are valid. All
functions use \texttt{\ assert\ } in the background.

Since a module in Typst is not callable, the \texttt{\ assert\ }
function is now available as \texttt{\ assert.that()\ } .
\texttt{\ assert.eq\ } and \texttt{\ assert.ne\ } work as expected.

All assert functions take an optional argument \texttt{\ message\ } to
set the error message shown if the assert fails.

\begin{itemize}
\item
  \texttt{\ \#assert.that(\ test\ )\ } : Asserts that the passed
  \texttt{\ test\ } is \texttt{\ true\ } .
\item
  \texttt{\ \#assert.that-not(\ test\ )\ } : Asserts that the passed
  \texttt{\ test\ } is \texttt{\ false\ } .
\item
  \texttt{\ \#assert.eq(\ a,\ b\ )\ } : Asserts that \texttt{\ a\ } is
  equal to \texttt{\ b\ } .
\item
  \texttt{\ \#assert.ne(\ a,\ b\ )\ } : Asserts that \texttt{\ a\ } is
  not equal to \texttt{\ b\ } .
\item
  \texttt{\ \#assert.neq(\ a,\ b\ )\ } : Alias for
  \texttt{\ assert.ne\ } .
\item
  \texttt{\ \#assert.not-none(\ value\ )\ } : Asserts that
  \texttt{\ value\ } is not equal to \texttt{\ none\ } .
\item
  \texttt{\ \#assert.any(\ ..values,\ value\ )\ } : Asserts that
  \texttt{\ value\ } is one of the passed \texttt{\ values\ } .
\item
  \texttt{\ \#assert.not-any(\ ..values,\ value\ )\ } : Asserts that
  \texttt{\ value\ } is not one of the passed \texttt{\ values\ } .
\item
  \texttt{\ \#assert.any-type(\ ..types,\ value\ )\ } : Asserts that the
  type of \texttt{\ value\ } is one of the passed \texttt{\ types\ } .
\item
  \texttt{\ \#assert.not-any-type(\ ..types,\ value\ )\ } : Asserts that
  the type of \texttt{\ value\ } is not one of the passed
  \texttt{\ types\ } .
\item
  \texttt{\ \#assert.all-of-type(\ t,\ ..values\ )\ } : Asserts that the
  type of all passed \texttt{\ values\ } is equal to \texttt{\ t\ } .
\item
  \texttt{\ \#assert.none-of-type(\ t,\ ..values\ )\ } : Asserts that
  the type of all passed \texttt{\ values\ } is not equal to
  \texttt{\ t\ } .
\item
  \texttt{\ \#assert.not-empty(\ value\ )\ } : Asserts that
  \texttt{\ value\ } is not \emph{empty} .
\item
  \texttt{\ \#assert.new(\ test\ )\ } : Creates a new assert function
  that uses the passed \texttt{\ test\ } . \texttt{\ test\ } is a
  function with signature \texttt{\ (any)\ =\textgreater{}\ boolean\ } .
  This is a quick way to create an assertion from any of the
  \texttt{\ is\ } functions:

\begin{Shaded}
\begin{Highlighting}[]
\NormalTok{\#let assert}\OperatorTok{{-}}\NormalTok{foo }\OperatorTok{=}\NormalTok{ assert}\OperatorTok{.}\FunctionTok{new}\NormalTok{(is}\OperatorTok{.}\AttributeTok{eq}\OperatorTok{.}\FunctionTok{with}\NormalTok{(}\StringTok{"foo"}\NormalTok{))}

\NormalTok{\#let assert}\OperatorTok{{-}}\NormalTok{length }\OperatorTok{=}\NormalTok{ assert}\OperatorTok{.}\FunctionTok{new}\NormalTok{(is}\OperatorTok{.}\AttributeTok{length}\NormalTok{)}
\end{Highlighting}
\end{Shaded}
\end{itemize}

\subsection{Element helpers}\label{element-helpers}

\begin{Shaded}
\begin{Highlighting}[]
\NormalTok{\#import }\StringTok{"@preview/t4t:0.3.2"}\OperatorTok{:} \KeywordTok{get}
\end{Highlighting}
\end{Shaded}

This submodule is a collection of functions, that mostly deal with
content elements and \emph{get} some information from them. Though some
handle other types like dictionaries.

\begin{itemize}
\item
  \texttt{\ \#get.dict(\ ..values\ )\ } : Create a new dictionary from
  the passed \texttt{\ values\ } . All named arguments are stored in the
  new dictionary as is. All positional arguments are grouped in
  key-value pairs and inserted into the dictionary:

\begin{Shaded}
\begin{Highlighting}[]
\NormalTok{\#get}\OperatorTok{.}\FunctionTok{dict}\NormalTok{(}\StringTok{"a"}\OperatorTok{,} \DecValTok{1}\OperatorTok{,} \StringTok{"b"}\OperatorTok{,} \DecValTok{2}\OperatorTok{,} \StringTok{"c"}\OperatorTok{,}\NormalTok{ d}\OperatorTok{:}\DecValTok{4}\OperatorTok{,}\NormalTok{ e}\OperatorTok{:}\DecValTok{5}\NormalTok{)}

\CommentTok{// (a:1, b:2, c:none, d:4, e:5)}
\end{Highlighting}
\end{Shaded}
\item
  \texttt{\ \#get.dict-merge(\ ..dicts\ )\ } : Recursively merges the
  passed-in dictionaries:

\begin{Shaded}
\begin{Highlighting}[]
\NormalTok{\#get}\OperatorTok{.}\AttributeTok{dict}\OperatorTok{{-}}\FunctionTok{merge}\NormalTok{(}
\NormalTok{    (a}\OperatorTok{:} \DecValTok{1}\NormalTok{)}\OperatorTok{,}
\NormalTok{    (a}\OperatorTok{:}\NormalTok{ (one}\OperatorTok{:} \DecValTok{1}\OperatorTok{,}\NormalTok{ two}\OperatorTok{:}\DecValTok{2}\NormalTok{))}\OperatorTok{,}
\NormalTok{    (a}\OperatorTok{:}\NormalTok{ (two}\OperatorTok{:} \DecValTok{4}\OperatorTok{,}\NormalTok{ three}\OperatorTok{:}\DecValTok{3}\NormalTok{))}
\NormalTok{)}

\CommentTok{// (a:(one:1, two:4, three:3))}
\end{Highlighting}
\end{Shaded}
\item
  \texttt{\ \#get.args(\ args,\ prefix:\ ""\ )\ } : Creates a function
  to extract values from an argument sink \texttt{\ args\ } .

  The resulting function takes any number of positional and named
  arguments and creates a dictionary with values from
  \texttt{\ args.named()\ } . Positional arguments are present in the
  result if they are present in \texttt{\ args.named()\ } . Named
  arguments are always present, either with their value from
  \texttt{\ args.named()\ } or with the provided value.

  A \texttt{\ prefix\ } can be specified, to extract only specific
  arguments. The resulting dictionary will have all keys with the prefix
  removed, though.

\begin{Shaded}
\begin{Highlighting}[]
\NormalTok{\#let my}\OperatorTok{{-}}\FunctionTok{func}\NormalTok{( }\OperatorTok{..}\AttributeTok{options}\OperatorTok{,}\NormalTok{ title ) }\OperatorTok{=} \FunctionTok{block}\NormalTok{(}
    \OperatorTok{..}\AttributeTok{get}\OperatorTok{.}\FunctionTok{args}\NormalTok{(options)(}
        \StringTok{"spacing"}\OperatorTok{,} \StringTok{"above"}\OperatorTok{,} \StringTok{"below"}\OperatorTok{,}
\NormalTok{        width}\OperatorTok{:}\DecValTok{100}\OperatorTok{\%}
\NormalTok{    )}
\NormalTok{)[}
\NormalTok{    \#}\FunctionTok{text}\NormalTok{(}\OperatorTok{..}\AttributeTok{get}\OperatorTok{.}\FunctionTok{args}\NormalTok{(options}\OperatorTok{,}\NormalTok{ prefix}\OperatorTok{:}\StringTok{"text{-}"}\NormalTok{)(}
\NormalTok{        fill}\OperatorTok{:}\NormalTok{black}\OperatorTok{,}\NormalTok{ size}\OperatorTok{:}\FloatTok{0.8}\NormalTok{em}
\NormalTok{    )}\OperatorTok{,}\NormalTok{ title)}
\NormalTok{]}

\NormalTok{\#my}\OperatorTok{{-}}\FunctionTok{func}\NormalTok{(}
\NormalTok{    width}\OperatorTok{:} \DecValTok{50}\OperatorTok{\%,}
\NormalTok{    text}\OperatorTok{{-}}\NormalTok{fill}\OperatorTok{:}\NormalTok{ red}\OperatorTok{,}\NormalTok{ text}\OperatorTok{{-}}\NormalTok{size}\OperatorTok{:} \FloatTok{1.2}\NormalTok{em}
\NormalTok{)[\#}\FunctionTok{lorem}\NormalTok{(}\DecValTok{5}\NormalTok{)]}
\end{Highlighting}
\end{Shaded}
\item
  \texttt{\ \#get.text(\ element,\ sep:\ ""\ )\ } : Recursively extracts
  the text content of a content element. If present, all child elements
  are converted to text and joined with \texttt{\ sep\ } .
\item
  \texttt{\ \#get.stroke-paint(\ stroke,\ default:\ black\ )\ } :
  Returns the color of \texttt{\ stroke\ } . If no color information is
  available, \texttt{\ default\ } is used. (Deprecated, use
  \texttt{\ stroke.paint\ } instead.)
\item
  \texttt{\ \#get.stroke-thickness(\ stroke,\ default:\ 1pt\ )\ } :
  Returns the thickness of \texttt{\ stroke\ } . If no thickness
  information is available, \texttt{\ default\ } is used. (Deprecated,
  use \texttt{\ stroke.thickness\ } instead.)
\item
  \texttt{\ \#get.stroke-dict(\ stroke,\ ..overrides\ )\ } : Creates a
  dictionary with the keys necessary for a stroke. The returned
  dictionary is guaranteed to have the keys \texttt{\ paint\ } ,
  \texttt{\ thickness\ } , \texttt{\ dash\ } , \texttt{\ cap\ } and
  \texttt{\ join\ } .

  If \texttt{\ stroke\ } is a dictionary itself, all key-value pairs are
  copied to the resulting stroke. Any named arguments in
  \texttt{\ overrides\ } will override the previous value.
\item
  \texttt{\ \#get.inset-at(\ direction,\ inset,\ default:\ 0pt\ )\ } :
  Returns the inset (or outset) in a given \texttt{\ direction\ } ,
  ascertained from \texttt{\ inset\ } .
\item
  \texttt{\ \#get.inset-dict(\ inset,\ ..overrides\ )\ } : Creates a
  dictionary usable as an inset (or outset) argument.

  The resulting dictionary is guaranteed to have the keys
  \texttt{\ top\ } , \texttt{\ left\ } , \texttt{\ bottom\ } and
  \texttt{\ right\ } .

  If \texttt{\ inset\ } is a dictionary itself, all key-value pairs are
  copied to the resulting stroke. Any named arguments in
  \texttt{\ overrides\ } will override the previous value.
\item
  \texttt{\ \#get.x-align(\ align,\ default:left\ )\ } : Returns the
  alignment along the x-axis from the passed-in \texttt{\ align\ }
  value. If none is present, \texttt{\ default\ } is returned.
  (Deprecated, use \texttt{\ align.x\ } instead.)

\begin{Shaded}
\begin{Highlighting}[]
\NormalTok{\#get}\OperatorTok{.}\AttributeTok{x}\OperatorTok{{-}}\FunctionTok{align}\NormalTok{(top }\OperatorTok{+}\NormalTok{ center) }\CommentTok{// center}
\end{Highlighting}
\end{Shaded}
\item
  \texttt{\ \#get.y-align(\ align,\ default:top\ )\ } : Returns the
  alignment along the y-axis from the passed-in \texttt{\ align\ }
  value. If none is present, \texttt{\ default\ } is returned.
  (Deprecated, use \texttt{\ align.y\ } instead.)
\end{itemize}

\subsection{Math functions}\label{math-functions}

\begin{Shaded}
\begin{Highlighting}[]
\NormalTok{\#import }\StringTok{"@preview/t4t:0.3.2"}\OperatorTok{:}\NormalTok{ math}
\end{Highlighting}
\end{Shaded}

Some functions to complement the native \texttt{\ calc\ } module.

\begin{itemize}
\item
  \texttt{\ \#math.minmax(\ a,\ b\ )\ } : Returns an array with the
  minimum of \texttt{\ a\ } and \texttt{\ b\ } as the first element and
  the maximum as the second:

\begin{Shaded}
\begin{Highlighting}[]
\NormalTok{\#}\FunctionTok{let}\NormalTok{ (min}\OperatorTok{,}\NormalTok{ max) }\OperatorTok{=}\NormalTok{ math}\OperatorTok{.}\FunctionTok{minmax}\NormalTok{(a}\OperatorTok{,}\NormalTok{ b)}
\end{Highlighting}
\end{Shaded}
\item
  \texttt{\ \#math.clamp(\ min,\ max,\ value\ )\ } : Clamps a value
  between \texttt{\ min\ } and \texttt{\ max\ } . In contrast to
  \texttt{\ calc.clamp()\ } this function works for other values than
  numbers, as long as they are comparable.

\begin{Shaded}
\begin{Highlighting}[]
\NormalTok{text}\OperatorTok{{-}}\NormalTok{size }\OperatorTok{=}\NormalTok{ math}\OperatorTok{.}\FunctionTok{clamp}\NormalTok{(}\FloatTok{0.8}\NormalTok{em}\OperatorTok{,} \FloatTok{1.2}\NormalTok{em}\OperatorTok{,}\NormalTok{ text}\OperatorTok{{-}}\NormalTok{size)}
\end{Highlighting}
\end{Shaded}
\item
  \texttt{\ \#lerp(\ min,\ max,\ t\ )\ } : Calculates the linear
  interpolation of \texttt{\ t\ } between \texttt{\ min\ } and
  \texttt{\ max\ } .

\begin{Shaded}
\begin{Highlighting}[]
\NormalTok{\#let width }\OperatorTok{=}\NormalTok{ math}\OperatorTok{.}\FunctionTok{lerp}\NormalTok{(}\DecValTok{0}\OperatorTok{\%,} \DecValTok{100}\OperatorTok{\%,}\NormalTok{ x)}
\end{Highlighting}
\end{Shaded}

  \texttt{\ t\ } should be a value between 0 and 1, but the
  interpolation works with other values, too. To constrain the result
  into the given interval, use \texttt{\ math.clamp\ } :

\begin{Shaded}
\begin{Highlighting}[]
\NormalTok{\#let width }\OperatorTok{=}\NormalTok{ math}\OperatorTok{.}\FunctionTok{lerp}\NormalTok{(}\DecValTok{0}\OperatorTok{\%,} \DecValTok{100}\OperatorTok{\%,}\NormalTok{ math}\OperatorTok{.}\FunctionTok{clamp}\NormalTok{(}\DecValTok{0}\OperatorTok{,} \DecValTok{1}\OperatorTok{,}\NormalTok{ x))}
\end{Highlighting}
\end{Shaded}
\item
  \texttt{\ \#map(\ min,\ max,\ range-min,\ range-max,\ value\ )\ } :
  Maps a \texttt{\ value\ } from the interval
  \texttt{\ {[}min,\ max{]}\ } into the interval
  \texttt{\ {[}range-min,\ range-max{]}\ } :

\begin{Shaded}
\begin{Highlighting}[]
\NormalTok{\#let text}\OperatorTok{{-}}\NormalTok{weight }\OperatorTok{=} \FunctionTok{int}\NormalTok{(math}\OperatorTok{.}\FunctionTok{map}\NormalTok{(}\DecValTok{8}\ErrorTok{pt}\OperatorTok{,} \DecValTok{16}\ErrorTok{pt}\OperatorTok{,} \DecValTok{400}\OperatorTok{,} \DecValTok{800}\OperatorTok{,}\NormalTok{ text}\OperatorTok{{-}}\NormalTok{size))}
\end{Highlighting}
\end{Shaded}
\end{itemize}

\subsection{Alias functions}\label{alias-functions}

\begin{Shaded}
\begin{Highlighting}[]
\NormalTok{\#import }\StringTok{"@preview/t4t:0.3.2"}\OperatorTok{:}\NormalTok{ alias}
\end{Highlighting}
\end{Shaded}

Some of the native Typst function as aliases, to prevent collisions with
some common argument names.

For example using \texttt{\ numbering\ } as an argument is not possible
if the value is supposed to be passed to the \texttt{\ numbering()\ }
function. To still allow argument names that are in line with the common
Typst names (like \texttt{\ numbering\ } , \texttt{\ align\ } …),
these alias functions can be used:

\begin{Shaded}
\begin{Highlighting}[]
\NormalTok{\#let }\FunctionTok{excercise}\NormalTok{( no}\OperatorTok{,}\NormalTok{ numbering}\OperatorTok{:} \StringTok{"1)"}\NormalTok{ ) }\OperatorTok{=}\NormalTok{ [}
\NormalTok{    Exercise \#alias}\OperatorTok{.}\FunctionTok{numbering}\NormalTok{(numbering}\OperatorTok{,}\NormalTok{ no)}
\NormalTok{]}
\end{Highlighting}
\end{Shaded}

The following functions have aliases right now:

\begin{itemize}
\tightlist
\item
  \texttt{\ numbering\ }
\item
  \texttt{\ align\ }
\item
  \texttt{\ type\ }
\item
  \texttt{\ label\ }
\item
  \texttt{\ text\ }
\item
  \texttt{\ raw\ }
\item
  \texttt{\ table\ }
\item
  \texttt{\ list\ }
\item
  \texttt{\ enum\ }
\item
  \texttt{\ terms\ }
\item
  \texttt{\ grid\ }
\item
  \texttt{\ stack\ }
\item
  \texttt{\ columns\ }
\end{itemize}

\subsection{Changelog}\label{changelog}

\subsubsection{Version 0.3.2}\label{version-0.3.2}

\begin{itemize}
\tightlist
\item
  Fixed issue with the new \texttt{\ \#type\ } function in Typst 0.8.0.
\end{itemize}

\subsubsection{Version 0.3.1}\label{version-0.3.1}

\begin{itemize}
\tightlist
\item
  Some fixes for message evaluation in \texttt{\ assert\ } module.
\end{itemize}

\subsubsection{Version 0.3.0}\label{version-0.3.0}

\begin{itemize}
\tightlist
\item
  Added a manual (build with \href{https://github.com/Mc-Zen/tidy}{tidy}
  and \href{https://github.com/jneug/typst-mantys}{Mantys} ).
\item
  Added simple tests for all functions.
\item
  Fixed bug in \texttt{\ is.elem\ } (see \#2).
\item
  Added \texttt{\ assert.has-pos\ } , \texttt{\ assert.no-pos\ } ,
  \texttt{\ assert.has-named\ } and \texttt{\ assert.no-named\ } .
\item
  Added meaningful messages to asserts.

  \begin{itemize}
  \tightlist
  \item
    Asserts now support functions as message arguments that can
    dynamically build an error message from the arguments.
  \end{itemize}
\item
  Improved spelling. (Thanks to @cherryblossom000 for proofreading.)
\end{itemize}

\subsubsection{Version 0.2.0}\label{version-0.2.0}

\begin{itemize}
\tightlist
\item
  Added \texttt{\ is.neg\ } function to negate a test function.
\item
  Added \texttt{\ alias.label\ } .
\item
  Added \texttt{\ is.label\ } test.
\item
  Added \texttt{\ def.as-arr\ } to create an array of the supplied
  values. Useful if an argument can be both a single value or an array.
\item
  Added \texttt{\ assert.that-not\ } for negated assertions.
\item
  Added \texttt{\ is.one-not-none\ } test to check multiple values, if
  at least one is not none.
\item
  Added \texttt{\ do\ } argument to all functions in \texttt{\ def\ } .
\item
  Allowed strings in \texttt{\ is.elem\ } (see \#1).

  \begin{itemize}
  \tightlist
  \item
    Added \texttt{\ is.sequence\ } test.
  \end{itemize}
\item
  Deprecated \texttt{\ get.stroke-paint\ } ,
  \texttt{\ get.stroke-thickness\ } , \texttt{\ get.x-align\ } and
  \texttt{\ get.y-align\ } in favor of new Typst 0.7.0 features.
\end{itemize}

\subsubsection{Version 0.1.0}\label{version-0.1.0}

\begin{itemize}
\tightlist
\item
  Initial release
\end{itemize}

\subsubsection{How to add}\label{how-to-add}

Copy this into your project and use the import as \texttt{\ t4t\ }

\begin{verbatim}
#import "@preview/t4t:0.3.2"
\end{verbatim}

\includesvg[width=0.16667in,height=0.16667in]{/assets/icons/16-copy.svg}

Check the docs for
\href{https://typst.app/docs/reference/scripting/\#packages}{more
information on how to import packages} .

\subsubsection{About}\label{about}

\begin{description}
\tightlist
\item[Author :]
Jonas Neugebauer
\item[License:]
MIT
\item[Current version:]
0.3.2
\item[Last updated:]
September 15, 2023
\item[First released:]
July 31, 2023
\item[Archive size:]
15.5 kB
\href{https://packages.typst.org/preview/t4t-0.3.2.tar.gz}{\pandocbounded{\includesvg[keepaspectratio]{/assets/icons/16-download.svg}}}
\item[Repository:]
\href{https://github.com/jneug/typst-tools4typst}{GitHub}
\end{description}

\subsubsection{Where to report issues?}\label{where-to-report-issues}

This package is a project of Jonas Neugebauer . Report issues on
\href{https://github.com/jneug/typst-tools4typst}{their repository} .
You can also try to ask for help with this package on the
\href{https://forum.typst.app}{Forum} .

Please report this package to the Typst team using the
\href{https://typst.app/contact}{contact form} if you believe it is a
safety hazard or infringes upon your rights.

\phantomsection\label{versions}
\subsubsection{Version history}\label{version-history}

\begin{longtable}[]{@{}ll@{}}
\toprule\noalign{}
Version & Release Date \\
\midrule\noalign{}
\endhead
\bottomrule\noalign{}
\endlastfoot
0.3.2 & September 15, 2023 \\
\href{https://typst.app/universe/package/t4t/0.3.1/}{0.3.1} & September
7, 2023 \\
\href{https://typst.app/universe/package/t4t/0.3.0/}{0.3.0} & August 24,
2023 \\
\href{https://typst.app/universe/package/t4t/0.2.0/}{0.2.0} & August 8,
2023 \\
\href{https://typst.app/universe/package/t4t/0.1.0/}{0.1.0} & July 31,
2023 \\
\end{longtable}

Typst GmbH did not create this package and cannot guarantee correct
functionality of this package or compatibility with any version of the
Typst compiler or app.


\section{Package List LaTeX/xarrow.tex}
\title{typst.app/universe/package/xarrow}

\phantomsection\label{banner}
\section{xarrow}\label{xarrow}

{ 0.3.1 }

Variable-length arrows in Typst.

\phantomsection\label{readme}
Variable-length arrows in Typst, fitting the width of a given content.

\subsection{Usage}\label{usage}

This library mainly provides the \texttt{\ xarrow\ } function. This
function takes one positional argument, which is the content to display
on top of the arrow. Additionally, the library provides the following
arrow styles:

\begin{itemize}
\tightlist
\item
  \texttt{\ xarrowDashed\ } using arrow \texttt{\ sym.arrow.dashed\ } .
\item
  \texttt{\ xarrowDouble\ } using arrow
  \texttt{\ sym.arrow.double.long\ } ;
\item
  \texttt{\ xarrowHook\ } using arrow \texttt{\ sym.arrow.hook\ } ;
\item
  \texttt{\ xarrowSquiggly\ } using arrow
  \texttt{\ sym.arrow.long.squiggly\ } ;
\item
  \texttt{\ xarrowTwoHead\ } using arrow \texttt{\ sym.arrow.twohead\ }
  ;
\item
  …
\end{itemize}

These names use camlCase in order to be simply called from math mode.
This may change in the future, if it becomes possible to have the
function names mirror the dot-separated name of the symbols themselves.

\subsubsection{Arguments}\label{arguments}

Users can provide the following arguments to any of the
previously-mentioned functions:

\begin{itemize}
\tightlist
\item
  \texttt{\ width\ } defines the width of the arrow. It defaults to
  \texttt{\ auto\ } , which makes the arrow adapt to the size of the
  body.
\item
  \texttt{\ margins\ } defines the spacing on each side of the
  \texttt{\ body\ } argument. Ignored when \texttt{\ width\ } is not
  \texttt{\ auto\ } .
\item
  \texttt{\ position\ } defines whether the main \texttt{\ body\ }
  argument will be set above or below the arrow. Default is
  \texttt{\ top\ } ; the only other accepted value is
  \texttt{\ bottom\ } .
\item
  \texttt{\ opposite\ } sets the content that is displayed on the other,
  non-default side of the arrow. Default is \texttt{\ none\ } .
\end{itemize}

\subsubsection{Example}\label{example}

\begin{verbatim}
#import "@preview/xarrow:0.3.0": xarrow, xarrowSquiggly, xarrowTwoHead

$
  a xarrow(sym: <--, QQ\, 1 + 1^4) b \
  c xarrowSquiggly("very long boi") d \
  c / ( a xarrowTwoHead("NP" limits(sum)^*) b times 4)
$
\end{verbatim}

\subsection{Customisation}\label{customisation}

The \texttt{\ xarrow\ } function has several named arguments which serve
to create new arrow designs:

\begin{itemize}
\tightlist
\item
  \texttt{\ sym\ } is the base symbol.
\item
  \texttt{\ sections\ } defines the way the symbol is divided. Drawing
  an arrow consists of drawing its tail, then repeating a central part
  that is defined by \texttt{\ sections\ } , then drawing the head. This
  is the parameter that has to be tweaked if observing artefacts.
  \texttt{\ sections\ } are given as two ratios, delimiting respectively
  the beginning and the end of the central, repeated part of the symbol.
\item
  \texttt{\ partial\_repeats\ } indicates whether the central part of
  the symbol can be partially repeated at the end in order to match the
  exact desired width. This has to be disabled when the repeated part
  has a clear period (like the squiggly arrow).
\end{itemize}

\subsubsection{Example}\label{example-1}

\begin{verbatim}
#let xarrowSquigglyBottom = xarrow.with(
  sym: sym.arrow.long.squiggly,
  sections: (20%, 45%),
  partial_repeats: false,
  position: bottom,
)
\end{verbatim}

\subsection{Limitations}\label{limitations}

\begin{itemize}
\tightlist
\item
  The predefined arrows are tweaked with the Computer Modern Math font
  in mind. With different glyphs, more sophisticated arrows will require
  manual modifications (of the \texttt{\ sections\ } argument) to be
  rendered correctly.
\item
  The \texttt{\ width\ } argument cannot be given ratio/fractions like
  other shapes. This would be a nice feature to have, in order to be
  able to create an arrow that takes 50\% of the available line width
  for instance.
\item
  I would like to make a proper manual for this library in the future,
  using something cool like
  \href{https://github.com/jneug/typst-mantys}{mantys} .
\end{itemize}

\subsubsection{How to add}\label{how-to-add}

Copy this into your project and use the import as \texttt{\ xarrow\ }

\begin{verbatim}
#import "@preview/xarrow:0.3.1"
\end{verbatim}

\includesvg[width=0.16667in,height=0.16667in]{/assets/icons/16-copy.svg}

Check the docs for
\href{https://typst.app/docs/reference/scripting/\#packages}{more
information on how to import packages} .

\subsubsection{About}\label{about}

\begin{description}
\tightlist
\item[Author :]
loutr
\item[License:]
GPL-3.0-only
\item[Current version:]
0.3.1
\item[Last updated:]
March 20, 2024
\item[First released:]
July 10, 2023
\item[Minimum Typst version:]
0.11.0
\item[Archive size:]
3.50 kB
\href{https://packages.typst.org/preview/xarrow-0.3.1.tar.gz}{\pandocbounded{\includesvg[keepaspectratio]{/assets/icons/16-download.svg}}}
\item[Repository:]
\href{https://codeberg.org/loutr/typst-xarrow/}{Codeberg}
\end{description}

\subsubsection{Where to report issues?}\label{where-to-report-issues}

This package is a project of loutr . Report issues on
\href{https://codeberg.org/loutr/typst-xarrow/}{their repository} . You
can also try to ask for help with this package on the
\href{https://forum.typst.app}{Forum} .

Please report this package to the Typst team using the
\href{https://typst.app/contact}{contact form} if you believe it is a
safety hazard or infringes upon your rights.

\phantomsection\label{versions}
\subsubsection{Version history}\label{version-history}

\begin{longtable}[]{@{}ll@{}}
\toprule\noalign{}
Version & Release Date \\
\midrule\noalign{}
\endhead
\bottomrule\noalign{}
\endlastfoot
0.3.1 & March 20, 2024 \\
\href{https://typst.app/universe/package/xarrow/0.3.0/}{0.3.0} & January
10, 2024 \\
\href{https://typst.app/universe/package/xarrow/0.2.0/}{0.2.0} &
September 26, 2023 \\
\href{https://typst.app/universe/package/xarrow/0.1.1/}{0.1.1} & July
11, 2023 \\
\href{https://typst.app/universe/package/xarrow/0.1.0/}{0.1.0} & July
10, 2023 \\
\end{longtable}

Typst GmbH did not create this package and cannot guarantee correct
functionality of this package or compatibility with any version of the
Typst compiler or app.


\section{Package List LaTeX/tutor.tex}
\title{typst.app/universe/package/tutor}

\phantomsection\label{banner}
\section{tutor}\label{tutor}

{ 0.7.0 }

Utilities to create exams.

\phantomsection\label{readme}
Utilities to write exams and exercises with integrated solutions. Set
the variable \texttt{\ \#(cfg.sol\ =\ true)\ } to display the solutions
of a document.

Currently the following features are supported:

\begin{itemize}
\tightlist
\item
  Automatic total point calculation through the \texttt{\ \#points()\ }
  and \texttt{\ \#totalpoints()\ } functions.
\item
  Checkboxes that are either blank or show the solution state using eg.
  \texttt{\ \#checkbox(cfg,\ true)\ } .
\item
  Display blank lines allowing students to write their answer using eg.
  \texttt{\ \#lines(cfg,\ 3)\ } .
\item
  A proposition for a project structure allowing self-contained
  exercises and a mechanism to show or hide the solutions of an
  exercise.
\end{itemize}

\subsection{Usage}\label{usage}

\subsubsection{Minimal Example}\label{minimal-example}

\begin{Shaded}
\begin{Highlighting}[]
\NormalTok{\#import "@local/tutor:0.6.1": points, totalpoints, lines, checkbox, default{-}config}

\NormalTok{\#let cfg = default{-}config()}
\NormalTok{// enable solution mode}
\NormalTok{\#(cfg.sol = true)}

\NormalTok{// display 3 lines (for hand written answer)}
\NormalTok{\#lines(cfg, 3)}
\NormalTok{// checkbox for multiple choice (indicates correct state)}
\NormalTok{\#checkbox(cfg, true)}

\NormalTok{// show achievable points}
\NormalTok{Max. points: \#points(2)}
\NormalTok{Max. points: \#points(3)}
\NormalTok{// show sum of all total achievable points (will show 5)}
\NormalTok{Total points: \#totalpoints(cfg)}
\end{Highlighting}
\end{Shaded}

\subsubsection{Practical Example}\label{practical-example}

Check
\href{https://github.com/rangerjo/tutor/tree/main/example}{example} for
a more practical example.

\texttt{\ tutor\ } is best used with the following directory and file
structure:

\begin{verbatim}
├── main.typ
├── src
│   ├── ex1
│   │   └── ex.typ
│   └── ex2
│       └── ex.typ
└── tutor.toml
\end{verbatim}

Every directory in \texttt{\ src\ } holds one self-contained exercise.
The exercises can be imported into \texttt{\ main.typ\ } :

\begin{Shaded}
\begin{Highlighting}[]
\NormalTok{\#import "@local/tutor:0.6.1": totalpoints, lines, default{-}config}

\NormalTok{\#import "src/ex1/ex.typ" as ex1}
\NormalTok{\#import "src/ex2/ex.typ" as ex2}


\NormalTok{\#let cfg = default{-}config()}
\NormalTok{\#ex1.exercise(cfg)}
\NormalTok{\#ex2.exercise(cfg)}
\end{Highlighting}
\end{Shaded}

Supporting self-contained exercises is one of \texttt{\ tutor\ } s
primary design goals. Each exercise lives within a folder and can easily
be copied or referenced in a new document.

An exercise is a folder that contains an \texttt{\ ex.typ\ } file along
with any other assets (images, source code aso). The following exercise
shows a practical usage of the \texttt{\ \#checkbox()\ } and
\texttt{\ \#points()\ } functions.

\texttt{\ src/ex1/ex.typ\ }

\begin{Shaded}
\begin{Highlighting}[]
\NormalTok{\#import "@local/tutor:0.6.1": points, checkbox}

\NormalTok{\#let exercise(cfg) = [}
\NormalTok{\#heading(level:cfg.lvl, [Abbreviation FHIR (\#points(1) point)])}

\NormalTok{What does FHIR stand for?}

\NormalTok{\#set list(marker: none)}
\NormalTok{{-} \#checkbox(cfg, false)  Finally He Is Real}
\NormalTok{{-} \#checkbox(cfg, true)   Fast Health Interoperability Resources}
\NormalTok{{-} \#checkbox(cfg, false)   First Health Inactivation Restriction}

\NormalTok{\#if cfg.sol \{}
\NormalTok{  [ Further explanation: FHIR is the new standard developed by HL7. ]}
\NormalTok{\}}
\NormalTok{]}
\end{Highlighting}
\end{Shaded}

Finally this second example shows the \texttt{\ \#lines()\ } function.
\texttt{\ src/ex2/ex.typ\ }

\begin{Shaded}
\begin{Highlighting}[]
\NormalTok{\#import "@local/tutor:0.6.1": points, lines }

\NormalTok{\#let exercise(cfg) = [}
\NormalTok{\#heading(level:cfg.lvl, [FHIR vs HL7v2 (\#points(4.5) points)])}

\NormalTok{List two differences between HL7v2 and FHIR:}

\NormalTok{+ \#if cfg.sol \{ [ HL7v2 uses a non{-}standard line format, where as FHIR uses XML or JSON] \} else \{ [ \#lines(cfg, 3) ] \}}
\NormalTok{+ \#if cfg.sol \{ [ FHIR specifies various resources that can be queried, where as HL7v2 has a number of fixed fields that are either filled in or not]\} else \{ [ \#lines(cfg, 3) ] \}}
\NormalTok{]}
\end{Highlighting}
\end{Shaded}

This would then give the following output in question mode (
\texttt{\ \#(cfg.sol=false)\ } ) and in solution mode (
\texttt{\ \#(cfg.sol=true)\ } ):
\pandocbounded{\includegraphics[keepaspectratio]{https://raw.githubusercontent.com/rangerjo/tutor/main/imgs/example_mod.png}}

\subsection{Utilities}\label{utilities}

\subsubsection{lines}\label{lines}

\texttt{\ \#lines(cfg,\ count)\ } prints \texttt{\ count\ } lines for
students to write their answer.

Configuration:

\begin{Shaded}
\begin{Highlighting}[]
\NormalTok{// Vertical line spacing between rows. }
\NormalTok{\#(cfg.utils.lines.spacing = 8mm)}
\end{Highlighting}
\end{Shaded}

\subsubsection{grid}\label{grid}

\texttt{\ \#grid(cfg,\ width,\ height)\ } prints a grid for students to
write their answer.

Configuration:

\begin{Shaded}
\begin{Highlighting}[]
\NormalTok{// Grid spacing. }
\NormalTok{\#(cfg.utils.grid.spacing = 4mm)}
\end{Highlighting}
\end{Shaded}

\subsubsection{checkbox}\label{checkbox}

\texttt{\ \#checkbox(cfg,\ answer)\ } shows a checkbox. In solution
mode, the checkbox is shown filled out.

Configuration:

\begin{Shaded}
\begin{Highlighting}[]
\NormalTok{// Symbol to show if answer is true }
\NormalTok{\#(cfg.utils.checkbox.sym\_true = "☒")}
\NormalTok{// Symbol to show if answer is false}
\NormalTok{\#(cfg.utils.checkbox.sym\_false = "☐")}
\NormalTok{// Symbol to show in question mode}
\NormalTok{\#(cfg.utils.checkbox.sym\_question = "☐")}
\end{Highlighting}
\end{Shaded}

\subsubsection{points}\label{points}

\texttt{\ \#points(cfg,\ num)\ } displays the given \texttt{\ num\ }
while adding its value to the total points counter.

Configuration: none

\subsubsection{totalpoints}\label{totalpoints}

\texttt{\ \#totalpoints(cfg)\ } shows the final value of the total
points counter.

Configuration:

\begin{Shaded}
\begin{Highlighting}[]
\NormalTok{// If points() is used in the outline, totalpoints value becomes doubled.}
\NormalTok{// By setting outline to true, totalpoints gets divided by half.}
\NormalTok{\#(cfg.utils.totalpoints.outline = false)}
\end{Highlighting}
\end{Shaded}

\subsection{Modes}\label{modes}

\texttt{\ tutor\ } comes with a solution and a test mode.

\subsubsection{solution mode}\label{solution-mode}

Solution mode controls wheter solutions are shown or not. This mode
controls eg. the utility \texttt{\ \#checkbox(cfg,\ answer)\ } .

\begin{enumerate}
\tightlist
\item
  \texttt{\ (cfg.sol\ =\ false)\ } : Solutions are hidden. This is used
  for the actual exam handed out to students.
\item
  \texttt{\ (cfg.sol\ =\ true)\ } : Solutions are shown. This is used to
  create the exam solutions.
\end{enumerate}

You can also use the following helper functions:

\begin{itemize}
\tightlist
\item
  \texttt{\ if-sol(cfg,{[}Content\ only\ shown\ in\ solution\ mode.{]})\ }
\item
  \texttt{\ if-sol-else(cfg,{[}Content\ only\ shown\ in\ solution\ mode.{]},\ {[}Content\ only\ shown\ in\ exam\ mode.{]})\ }
\end{itemize}

\subsubsection{test mode}\label{test-mode}

Test mode can be used to show or hide additional information. In test
mode, one might want

\begin{enumerate}
\item
  \texttt{\ (cfg.test\ =\ true)\ } : Test information are shown. Use
  this eg. to display \texttt{\ \#points(4)\ } . This is used in case
  the document is used as an exam/test.
\item
  \texttt{\ (cfg.test\ =\ false)\ } : Test information are hidden. This
  is used in case the document is used as an excerise.
\end{enumerate}

The following would show the points only in test mode.

\begin{Shaded}
\begin{Highlighting}[]
\NormalTok{\#if cfg.test \{}
\NormalTok{  \#points(4)}
\NormalTok{\}}
\end{Highlighting}
\end{Shaded}

Or you can use the following helper functions:

\begin{itemize}
\tightlist
\item
  \texttt{\ if-test(cfg,{[}Content\ only\ shown\ in\ test\ mode.{]})\ }
\item
  \texttt{\ if-test-else(cfg,{[}Content\ only\ shown\ in\ test\ mode.{]},\ {[}Content\ only\ shown\ in\ exercise\ mode.{]})\ }
\end{itemize}

\subsection{Configuration}\label{configuration}

\texttt{\ tutor\ } is designed to create exams and solutions with one
single document source. Furthermore, the individual utilities provided
by \texttt{\ tutor\ } can be configured. This can be done in one of
three ways:

\begin{enumerate}
\tightlist
\item
  Use the \texttt{\ \#default-config()\ } function and patch your
  configuration. The following example would configure the solution mode
  and basic line spacings to 8 millimeters:
\end{enumerate}

\begin{Shaded}
\begin{Highlighting}[]
\NormalTok{\#let cfg = default{-}config()}
\NormalTok{\#(cfg.sol = false)}
\NormalTok{\#(cfg.utils.lines.spacing = 8mm)}
\end{Highlighting}
\end{Shaded}

\begin{enumerate}
\setcounter{enumi}{1}
\tightlist
\item
  Use an external file to hold the configurations in your prefered
  format. See
  \href{https://github.com/rangerjo/tutor/blob/main/example/tutor.toml}{tutor.toml}
  for a configuration in TOML. Load the configuration into your main
  document using
\end{enumerate}

\begin{Shaded}
\begin{Highlighting}[]
\NormalTok{\#let cfg = toml("tutor.toml")}
\end{Highlighting}
\end{Shaded}

\begin{enumerate}
\setcounter{enumi}{2}
\tightlist
\item
  Use typst’s input feature added with compiler version 0.11.0. Add
  the following snippet to load the configuration, then overwrite it
  from the CLI like this:
  \texttt{\ typst\ compile\ -\/-input\ tutor\_sol=true\ main.typ\ }
\end{enumerate}

\begin{Shaded}
\begin{Highlighting}[]
\NormalTok{\#let cfg = toml("tutor.toml")}

\NormalTok{\#if sys.inputs.tutor\_sol == "true" \{}
\NormalTok{  (cfg.sol = true)}
\NormalTok{\} else if sys.inputs.tutor\_sol == "false" \{}
\NormalTok{  (cfg.sol = false)}
\NormalTok{\}}
\end{Highlighting}
\end{Shaded}

\subsubsection{How to add}\label{how-to-add}

Copy this into your project and use the import as \texttt{\ tutor\ }

\begin{verbatim}
#import "@preview/tutor:0.7.0"
\end{verbatim}

\includesvg[width=0.16667in,height=0.16667in]{/assets/icons/16-copy.svg}

Check the docs for
\href{https://typst.app/docs/reference/scripting/\#packages}{more
information on how to import packages} .

\subsubsection{About}\label{about}

\begin{description}
\tightlist
\item[Author :]
Jonas Amstutz
\item[License:]
MIT
\item[Current version:]
0.7.0
\item[Last updated:]
October 9, 2024
\item[First released:]
October 17, 2023
\item[Minimum Typst version:]
0.11.0
\item[Archive size:]
4.82 kB
\href{https://packages.typst.org/preview/tutor-0.7.0.tar.gz}{\pandocbounded{\includesvg[keepaspectratio]{/assets/icons/16-download.svg}}}
\item[Repository:]
\href{https://github.com/rangerjo/tutor}{GitHub}
\item[Discipline :]
\begin{itemize}
\tightlist
\item[]
\item
  \href{https://typst.app/universe/search/?discipline=education}{Education}
\end{itemize}
\item[Categor y :]
\begin{itemize}
\tightlist
\item[]
\item
  \pandocbounded{\includesvg[keepaspectratio]{/assets/icons/16-package.svg}}
  \href{https://typst.app/universe/search/?category=components}{Components}
\end{itemize}
\end{description}

\subsubsection{Where to report issues?}\label{where-to-report-issues}

This package is a project of Jonas Amstutz . Report issues on
\href{https://github.com/rangerjo/tutor}{their repository} . You can
also try to ask for help with this package on the
\href{https://forum.typst.app}{Forum} .

Please report this package to the Typst team using the
\href{https://typst.app/contact}{contact form} if you believe it is a
safety hazard or infringes upon your rights.

\phantomsection\label{versions}
\subsubsection{Version history}\label{version-history}

\begin{longtable}[]{@{}ll@{}}
\toprule\noalign{}
Version & Release Date \\
\midrule\noalign{}
\endhead
\bottomrule\noalign{}
\endlastfoot
0.7.0 & October 9, 2024 \\
\href{https://typst.app/universe/package/tutor/0.6.1/}{0.6.1} & October
9, 2024 \\
\href{https://typst.app/universe/package/tutor/0.4.0/}{0.4.0} & March
19, 2024 \\
\href{https://typst.app/universe/package/tutor/0.3.0/}{0.3.0} & October
17, 2023 \\
\end{longtable}

Typst GmbH did not create this package and cannot guarantee correct
functionality of this package or compatibility with any version of the
Typst compiler or app.


\section{Package List LaTeX/modern-uit-thesis.tex}
\title{typst.app/universe/package/modern-uit-thesis}

\phantomsection\label{banner}
\phantomsection\label{template-thumbnail}
\pandocbounded{\includegraphics[keepaspectratio]{https://packages.typst.org/preview/thumbnails/modern-uit-thesis-0.1.2-small.webp}}

\section{modern-uit-thesis}\label{modern-uit-thesis}

{ 0.1.2 }

A Modern Thesis Template in Typst.

\href{/app?template=modern-uit-thesis&version=0.1.2}{Create project in
app}

\phantomsection\label{readme}
Port of the \href{https://github.com/egraff/uit-thesis}{uit-thesis}
-latex template to Typst.

\texttt{\ thesis.typ\ } contains a full usage example, see
\texttt{\ thesis.pdf\ } for a rendered pdf.

\subsection{Usage}\label{usage}

Using the Typst Universe package/template:

\begin{Shaded}
\begin{Highlighting}[]
\NormalTok{typst init @preview/modern{-}uit{-}thesis:0.1.2}
\end{Highlighting}
\end{Shaded}

\subsubsection{Fonts}\label{fonts}

This template uses a number of different fonts:

\begin{itemize}
\tightlist
\item
  Open Sans (Noto Sans)
\item
  JetBrains Mono (Fira Code)
\item
  Charis SIL (Charter)
\end{itemize}

The above parenthesized fonts are fallback typefaces available by
default in \href{https://typst.app/}{the web app} . If you’d like to
use the main fonts instead, simply upload the \texttt{\ .ttf\ } s to the
web app and it will detect and apply them automatically.

If you’re running typst locally, install the fonts in a directory of
your choosing and specify it with \texttt{\ -\/-font-path\ } .

\subsection{License}\label{license}

This project is licensed under the MIT License - see the
\href{https://github.com/typst/packages/raw/main/packages/preview/modern-uit-thesis/0.1.2/LICENSE}{LICENSE}
file for details.

\href{/app?template=modern-uit-thesis&version=0.1.2}{Create project in
app}

\subsubsection{How to use}\label{how-to-use}

Click the button above to create a new project using this template in
the Typst app.

You can also use the Typst CLI to start a new project on your computer
using this command:

\begin{verbatim}
typst init @preview/modern-uit-thesis:0.1.2
\end{verbatim}

\includesvg[width=0.16667in,height=0.16667in]{/assets/icons/16-copy.svg}

\subsubsection{About}\label{about}

\begin{description}
\tightlist
\item[Author s :]
\href{https://github.com/mrtz-j}{Moritz Jörg} \&
\href{https://github.com/otytlandsvik}{Ole Tytlandsvik}
\item[License:]
MIT
\item[Current version:]
0.1.2
\item[Last updated:]
October 29, 2024
\item[First released:]
September 18, 2024
\item[Minimum Typst version:]
0.12.0
\item[Archive size:]
543 kB
\href{https://packages.typst.org/preview/modern-uit-thesis-0.1.2.tar.gz}{\pandocbounded{\includesvg[keepaspectratio]{/assets/icons/16-download.svg}}}
\item[Repository:]
\href{https://github.com/mrtz-j/typst-thesis-template}{GitHub}
\item[Categor y :]
\begin{itemize}
\tightlist
\item[]
\item
  \pandocbounded{\includesvg[keepaspectratio]{/assets/icons/16-mortarboard.svg}}
  \href{https://typst.app/universe/search/?category=thesis}{Thesis}
\end{itemize}
\end{description}

\subsubsection{Where to report issues?}\label{where-to-report-issues}

This template is a project of Moritz Jörg and Ole Tytlandsvik . Report
issues on \href{https://github.com/mrtz-j/typst-thesis-template}{their
repository} . You can also try to ask for help with this template on the
\href{https://forum.typst.app}{Forum} .

Please report this template to the Typst team using the
\href{https://typst.app/contact}{contact form} if you believe it is a
safety hazard or infringes upon your rights.

\phantomsection\label{versions}
\subsubsection{Version history}\label{version-history}

\begin{longtable}[]{@{}ll@{}}
\toprule\noalign{}
Version & Release Date \\
\midrule\noalign{}
\endhead
\bottomrule\noalign{}
\endlastfoot
0.1.2 & October 29, 2024 \\
\href{https://typst.app/universe/package/modern-uit-thesis/0.1.1/}{0.1.1}
& September 19, 2024 \\
\href{https://typst.app/universe/package/modern-uit-thesis/0.1.0/}{0.1.0}
& September 18, 2024 \\
\end{longtable}

Typst GmbH did not create this template and cannot guarantee correct
functionality of this template or compatibility with any version of the
Typst compiler or app.


\section{Package List LaTeX/drafting.tex}
\title{typst.app/universe/package/drafting}

\phantomsection\label{banner}
\section{drafting}\label{drafting}

{ 0.2.1 }

Helpful functions for content positioning and margin comments/notes

{ } Featured Package

\phantomsection\label{readme}
\subsection{Setup}\label{setup}

\texttt{\ drafting\ } exists in the official typst package repository,
so the recommended approach is to import it from the
\texttt{\ preview\ } namespace:

\begin{Shaded}
\begin{Highlighting}[]
\NormalTok{\#import "@preview/drafting:0.2.1"}
\end{Highlighting}
\end{Shaded}

Margin notes cannot lay themselves out correctly until they know your
page size and margins. By default, they occupy nearly the entirety of
the left or right margin, but you can provide explicit left/right bounds
if desired:

\begin{Shaded}
\begin{Highlighting}[]
\NormalTok{// Example:}
\NormalTok{// Default margin in typst is 2.5cm, but we want to use 2cm}
\NormalTok{// On the left}
\NormalTok{\#set{-}page{-}properties(margin{-}left: 2cm)}
\end{Highlighting}
\end{Shaded}

\subsection{The basics}\label{the-basics}

\begin{Shaded}
\begin{Highlighting}[]
\NormalTok{\#lorem(20)}
\NormalTok{\#margin{-}note(side: left)[Hello, world!]}
\NormalTok{\#lorem(10)}
\NormalTok{\#margin{-}note[Hello from the other side]}
\NormalTok{\#margin{-}note[When notes are about to overlap, they\textquotesingle{}re automatically shifted]}
\NormalTok{\#margin{-}note(stroke: aqua + 3pt)[To avoid collision]}
\NormalTok{\#lorem(25)}
\NormalTok{\#margin{-}note(stroke: green, side: left)[You can provide two positional arguments if you want to highlight a phrase associated with your note.][The first is text which should be inline{-}noted, and the second is the standard margin note.]}

\NormalTok{\#let caution{-}rect = rect.with(inset: 1em, radius: 0.5em, fill: orange.lighten(80\%))}
\NormalTok{\#inline{-}note(rect: caution{-}rect)[}
\NormalTok{  Be aware that \textasciigrave{}typst\textasciigrave{} will complain when 4 notes overlap, and stop automatically avoiding collisions when 5 or more notes}
\NormalTok{  overlap. This is because the compiler stops attempting to reposition notes after a few attempts}
\NormalTok{  (initial layout + adjustment for each note).}

\NormalTok{  You can manually adjust the position of notes with \textasciigrave{}dy\textasciigrave{} to silence the warning.}
\NormalTok{]}
\end{Highlighting}
\end{Shaded}

\pandocbounded{\includegraphics[keepaspectratio]{https://www.github.com/ntjess/typst-drafting/raw/v0.2.1/assets/example-1.png}}

\subsection{Adjusting the default
style}\label{adjusting-the-default-style}

All function defaults are customizable through updating the module
state:

\begin{Shaded}
\begin{Highlighting}[]
\NormalTok{\#lorem(14) \#margin{-}note[Default style]}
\NormalTok{\#lorem(10)}
\NormalTok{\#set{-}margin{-}note{-}defaults(stroke: orange, side: left)}
\NormalTok{\#margin{-}note[Updated style]}
\NormalTok{\#lorem(10)}
\end{Highlighting}
\end{Shaded}

\pandocbounded{\includegraphics[keepaspectratio]{https://www.github.com/ntjess/typst-drafting/raw/v0.2.1/assets/example-2.png}}

Even deeper customization is possible by overriding the default
\texttt{\ rect\ } :

\begin{Shaded}
\begin{Highlighting}[]
\NormalTok{\#import "@preview/colorful{-}boxes:1.1.0": stickybox}

\NormalTok{\#let default{-}rect(stroke: none, fill: none, width: 0pt, content) = \{}
\NormalTok{  set text(0.9em)}
\NormalTok{  stickybox(rotation: 30deg, width: width/1.5, content)}
\NormalTok{\}}
\NormalTok{\#set{-}margin{-}note{-}defaults(rect: default{-}rect, stroke: none, side: right)}

\NormalTok{\#lorem(20)}
\NormalTok{\#margin{-}note(dy: {-}5em)[Why not use sticky notes in the margin?]}

\NormalTok{// Undo changes from this example}
\NormalTok{\#set{-}margin{-}note{-}defaults(rect: rect, stroke: red)}
\end{Highlighting}
\end{Shaded}

\pandocbounded{\includegraphics[keepaspectratio]{https://www.github.com/ntjess/typst-drafting/raw/v0.2.1/assets/example-3.png}}

\subsection{Multiple document
reviewers}\label{multiple-document-reviewers}

\begin{Shaded}
\begin{Highlighting}[]
\NormalTok{\#let reviewer{-}a = margin{-}note.with(stroke: blue)}
\NormalTok{\#let reviewer{-}b = margin{-}note.with(stroke: purple)}
\NormalTok{\#lorem(10)}
\NormalTok{\#reviewer{-}a[Comment from reviewer A]}
\NormalTok{\#lorem(5)}
\NormalTok{\#reviewer{-}b(side: left)[Reviewer B comment]}
\NormalTok{\#lorem(10)}
\end{Highlighting}
\end{Shaded}

\pandocbounded{\includegraphics[keepaspectratio]{https://www.github.com/ntjess/typst-drafting/raw/v0.2.1/assets/example-4.png}}

\subsection{Inline Notes}\label{inline-notes}

\begin{Shaded}
\begin{Highlighting}[]
\NormalTok{\#lorem(10)}
\NormalTok{\#inline{-}note[The default inline note will split the paragraph at its location]}
\NormalTok{\#lorem(10)}
\NormalTok{\#inline{-}note(par{-}break: false, stroke: (paint: orange, dash: "dashed"))[}
\NormalTok{  But you can specify \textasciigrave{}par{-}break: false\textasciigrave{} to prevent this}
\NormalTok{]}
\NormalTok{\#lorem(10)}
\end{Highlighting}
\end{Shaded}

\pandocbounded{\includegraphics[keepaspectratio]{https://www.github.com/ntjess/typst-drafting/raw/v0.2.1/assets/example-5.png}}

\subsection{Hiding notes for print
preview}\label{hiding-notes-for-print-preview}

\begin{Shaded}
\begin{Highlighting}[]
\NormalTok{\#set{-}margin{-}note{-}defaults(hidden: true)}

\NormalTok{\#lorem(20)}
\NormalTok{\#margin{-}note[This will respect the global "hidden" state]}
\NormalTok{\#margin{-}note(hidden: false, dy: {-}2.5em)[This note will never be hidden]}
\NormalTok{// Undo these changes}
\NormalTok{\#set{-}margin{-}note{-}defaults(hidden: false)}
\end{Highlighting}
\end{Shaded}

\pandocbounded{\includegraphics[keepaspectratio]{https://www.github.com/ntjess/typst-drafting/raw/v0.2.1/assets/example-6.png}}

\subsection{Precise placement: rule
grid}\label{precise-placement-rule-grid}

Need to measure space for fine-tuned positioning? You can use
\texttt{\ rule-grid\ } to cross-hatch the page with rule lines:

\begin{Shaded}
\begin{Highlighting}[]
\NormalTok{\#rule{-}grid(width: 10cm, height: 3cm, spacing: 20pt)}
\NormalTok{\#place(}
\NormalTok{  dx: 180pt,}
\NormalTok{  dy: 40pt,}
\NormalTok{  rect(fill: white, stroke: red, width: 1in, "This will originate at (180pt, 40pt)")}
\NormalTok{)}

\NormalTok{// Optionally specify divisions of the smallest dimension to automatically calculate}
\NormalTok{// spacing}
\NormalTok{\#rule{-}grid(dx: 10cm + 3em, width: 3cm, height: 1.2cm, divisions: 5, square: true,  stroke: green)}

\NormalTok{// The rule grid doesn\textquotesingle{}t take up space, so add it explicitly}
\NormalTok{\#v(3cm + 1em)}
\end{Highlighting}
\end{Shaded}

\pandocbounded{\includegraphics[keepaspectratio]{https://www.github.com/ntjess/typst-drafting/raw/v0.2.1/assets/example-7.png}}

\subsection{Absolute positioning}\label{absolute-positioning}

What about absolutely positioning something regardless of margin and
relative location? \texttt{\ absolute-place\ } is your friend. You can
put content anywhere:

\begin{Shaded}
\begin{Highlighting}[]
\NormalTok{\#context \{}
\NormalTok{  let (dx, dy) = (here().position().x, here().position().y)}
\NormalTok{  let content{-}str = (}
\NormalTok{    "This absolutely{-}placed box will originate at (" + repr(dx) + ", " + repr(dy) + ")"}
\NormalTok{    + " in page coordinates"}
\NormalTok{  )}
\NormalTok{  absolute{-}place(}
\NormalTok{    dx: dx, dy: dy,}
\NormalTok{    rect(}
\NormalTok{      fill: green.lighten(60\%),}
\NormalTok{      radius: 0.5em,}
\NormalTok{      width: 2.5in,}
\NormalTok{      height: 0.5in,}
\NormalTok{      [\#align(center + horizon, content{-}str)]}
\NormalTok{    )}
\NormalTok{  )}
\NormalTok{\}}
\NormalTok{\#v(0.5in)}
\end{Highlighting}
\end{Shaded}

\pandocbounded{\includegraphics[keepaspectratio]{https://www.github.com/ntjess/typst-drafting/raw/v0.2.1/assets/example-8.png}}

The “rule-grid� also supports absolute placement at the top-left of
the page by passing \texttt{\ relative:\ false\ } . This is helpful for
“rule“-ing the whole page.

\subsubsection{How to add}\label{how-to-add}

Copy this into your project and use the import as \texttt{\ drafting\ }

\begin{verbatim}
#import "@preview/drafting:0.2.1"
\end{verbatim}

\includesvg[width=0.16667in,height=0.16667in]{/assets/icons/16-copy.svg}

Check the docs for
\href{https://typst.app/docs/reference/scripting/\#packages}{more
information on how to import packages} .

\subsubsection{About}\label{about}

\begin{description}
\tightlist
\item[Author :]
Nathan Jessurun
\item[License:]
Unlicense
\item[Current version:]
0.2.1
\item[Last updated:]
November 25, 2024
\item[First released:]
September 3, 2023
\item[Minimum Typst version:]
0.12.0
\item[Archive size:]
7.98 kB
\href{https://packages.typst.org/preview/drafting-0.2.1.tar.gz}{\pandocbounded{\includesvg[keepaspectratio]{/assets/icons/16-download.svg}}}
\item[Repository:]
\href{https://github.com/ntjess/typst-drafting}{GitHub}
\item[Categor ies :]
\begin{itemize}
\tightlist
\item[]
\item
  \pandocbounded{\includesvg[keepaspectratio]{/assets/icons/16-layout.svg}}
  \href{https://typst.app/universe/search/?category=layout}{Layout}
\item
  \pandocbounded{\includesvg[keepaspectratio]{/assets/icons/16-hammer.svg}}
  \href{https://typst.app/universe/search/?category=utility}{Utility}
\end{itemize}
\end{description}

\subsubsection{Where to report issues?}\label{where-to-report-issues}

This package is a project of Nathan Jessurun . Report issues on
\href{https://github.com/ntjess/typst-drafting}{their repository} . You
can also try to ask for help with this package on the
\href{https://forum.typst.app}{Forum} .

Please report this package to the Typst team using the
\href{https://typst.app/contact}{contact form} if you believe it is a
safety hazard or infringes upon your rights.

\phantomsection\label{versions}
\subsubsection{Version history}\label{version-history}

\begin{longtable}[]{@{}ll@{}}
\toprule\noalign{}
Version & Release Date \\
\midrule\noalign{}
\endhead
\bottomrule\noalign{}
\endlastfoot
0.2.1 & November 25, 2024 \\
\href{https://typst.app/universe/package/drafting/0.2.0/}{0.2.0} & March
16, 2024 \\
\href{https://typst.app/universe/package/drafting/0.1.2/}{0.1.2} &
December 11, 2023 \\
\href{https://typst.app/universe/package/drafting/0.1.1/}{0.1.1} &
September 11, 2023 \\
\href{https://typst.app/universe/package/drafting/0.1.0/}{0.1.0} &
September 3, 2023 \\
\end{longtable}

Typst GmbH did not create this package and cannot guarantee correct
functionality of this package or compatibility with any version of the
Typst compiler or app.


\section{Package List LaTeX/modern-russian-dissertation.tex}
\title{typst.app/universe/package/modern-russian-dissertation}

\phantomsection\label{banner}
\phantomsection\label{template-thumbnail}
\pandocbounded{\includegraphics[keepaspectratio]{https://packages.typst.org/preview/thumbnails/modern-russian-dissertation-0.0.1-small.webp}}

\section{modern-russian-dissertation}\label{modern-russian-dissertation}

{ 0.0.1 }

A russian phd thesis template

\href{/app?template=modern-russian-dissertation&version=0.0.1}{Create
project in app}

\phantomsection\label{readme}
Шаблон руÑ?Ñ?кой кандидатÑ?кой
диÑ?Ñ?ертации на Ñ?зыке разметки
\href{https://typst.app/}{Typst} - Ñ?овременной
альтернативы LaTeX.

\subsection{ИÑ?пользование}\label{uxf0uxf1uxf0uxf0uxbeuxf0uxf1ux153uxf0uxf0uxbeuxf0uxb2uxf0uxf0uxbduxf0uxf0uxb5}

Ð' веб-приложении нажмите “Start from templateâ€?
и на панели найдите
\texttt{\ modern-russian-dissertation\ } .

Ð'Ñ‹ также можете инициализировать
проект командой:

\begin{Shaded}
\begin{Highlighting}[]
\ExtensionTok{typst}\NormalTok{ init @preview/modern{-}russian{-}dissertation}
\end{Highlighting}
\end{Shaded}

Ð`удет Ñ?оздана новаÑ? директориÑ? Ñ?о
вÑ?еми файлами, необÑ\ldots одимыми длÑ?
начала работы.

\subsection{КонфигурациÑ?}\label{uxf0ux161uxf0uxbeuxf0uxbduxf1uxf0uxf0uxb3uxf1ux192uxf1uxf0uxf1uxf0uxf1}

СпиÑ?ок литературы формируетÑ?Ñ? из
файлов \texttt{\ common/external.bib\ } и
\texttt{\ common/author.bib\ } .

СпиÑ?ок Ñ?окращений и уÑ?ловныÑ\ldots{}
обозначений формируетÑ?Ñ? из данныÑ\ldots,
запиÑ?анныÑ\ldots{} в файле
\texttt{\ common/acronyms.typ\ } \texttt{\ common/symbols.typ\ } .
СпиÑ?ок определений формируетÑ?Ñ? из
данныÑ\ldots{} в файле \texttt{\ common/glossary.typ\ } .

\subsection{КомпилÑ?циÑ?}\label{uxf0ux161uxf0uxbeuxf0uxbcuxf0uxf0uxf0uxf1uxf1uxf0uxf1}

Ð''лÑ? компилÑ?ции проекта из CLI
иÑ?пользуйте:

\begin{Shaded}
\begin{Highlighting}[]
\ExtensionTok{typst}\NormalTok{ compile thesis.typ}
\end{Highlighting}
\end{Shaded}

Или еÑ?ли вы Ñ\ldots отите Ñ?ледить за
изменениÑ?ми:

\begin{Shaded}
\begin{Highlighting}[]
\ExtensionTok{typst}\NormalTok{ watch thesis.typ}
\end{Highlighting}
\end{Shaded}

\subsection{ОÑ?обенноÑ?ти}\label{uxf0ux17euxf1uxf0uxbeuxf0uxf0uxb5uxf0uxbduxf0uxbduxf0uxbeuxf1uxf1uxf0}

\begin{itemize}
\tightlist
\item
  Стандарт Ð``ОСТ Ð~ 7.0.11-2011.
\end{itemize}

\subsection{Ð`лагодарноÑ?ти}\label{uxf0uxf0uxf0uxf0uxb3uxf0uxbeuxf0uxf0uxf1uxf0uxbduxf0uxbeuxf1uxf1uxf0}

\begin{itemize}
\item
  Ð`лагодарноÑ?Ñ‚ÑŒ авторам шаблона
  диÑ?Ñ?ертации на
  \href{https://github.com/AndreyAkinshin/Russian-Phd-LaTeX-Dissertation-Template}{LaTeX}
\item
  \href{https://github.com/AndreyAkinshin/Russian-Phd-LaTeX-Dissertation-Template/wiki/Links\#\%D0\%BF\%D1\%80\%D0\%BE\%D1\%87\%D0\%B8\%D0\%B5-\%D1\%80\%D0\%B5\%D0\%BF\%D0\%BE\%D0\%B7\%D0\%B8\%D1\%82\%D0\%BE\%D1\%80\%D0\%B8\%D0\%B8-\%D1\%81-\%D0\%BF\%D0\%BE\%D0\%BB\%D0\%B5\%D0\%B7\%D0\%BD\%D1\%8B\%D0\%BC\%D0\%B8-\%D0\%BF\%D1\%80\%D0\%B8\%D0\%BC\%D0\%B5\%D1\%80\%D0\%B0\%D0\%BC\%D0\%B8}{Полезные
  Ñ?Ñ?ылки}
\end{itemize}

\href{/app?template=modern-russian-dissertation&version=0.0.1}{Create
project in app}

\subsubsection{How to use}\label{how-to-use}

Click the button above to create a new project using this template in
the Typst app.

You can also use the Typst CLI to start a new project on your computer
using this command:

\begin{verbatim}
typst init @preview/modern-russian-dissertation:0.0.1
\end{verbatim}

\includesvg[width=0.16667in,height=0.16667in]{/assets/icons/16-copy.svg}

\subsubsection{About}\label{about}

\begin{description}
\tightlist
\item[Author :]
Sergei Gorchakov
\item[License:]
MIT
\item[Current version:]
0.0.1
\item[Last updated:]
April 12, 2024
\item[First released:]
April 12, 2024
\item[Minimum Typst version:]
0.11.0
\item[Archive size:]
32.7 kB
\href{https://packages.typst.org/preview/modern-russian-dissertation-0.0.1.tar.gz}{\pandocbounded{\includesvg[keepaspectratio]{/assets/icons/16-download.svg}}}
\item[Repository:]
\href{https://github.com/SergeyGorchakov/russian-phd-thesis-template-typst}{GitHub}
\item[Categor y :]
\begin{itemize}
\tightlist
\item[]
\item
  \pandocbounded{\includesvg[keepaspectratio]{/assets/icons/16-mortarboard.svg}}
  \href{https://typst.app/universe/search/?category=thesis}{Thesis}
\end{itemize}
\end{description}

\subsubsection{Where to report issues?}\label{where-to-report-issues}

This template is a project of Sergei Gorchakov . Report issues on
\href{https://github.com/SergeyGorchakov/russian-phd-thesis-template-typst}{their
repository} . You can also try to ask for help with this template on the
\href{https://forum.typst.app}{Forum} .

Please report this template to the Typst team using the
\href{https://typst.app/contact}{contact form} if you believe it is a
safety hazard or infringes upon your rights.

\phantomsection\label{versions}
\subsubsection{Version history}\label{version-history}

\begin{longtable}[]{@{}ll@{}}
\toprule\noalign{}
Version & Release Date \\
\midrule\noalign{}
\endhead
\bottomrule\noalign{}
\endlastfoot
0.0.1 & April 12, 2024 \\
\end{longtable}

Typst GmbH did not create this template and cannot guarantee correct
functionality of this template or compatibility with any version of the
Typst compiler or app.


\section{Package List LaTeX/modern-nju-thesis.tex}
\title{typst.app/universe/package/modern-nju-thesis}

\phantomsection\label{banner}
\phantomsection\label{template-thumbnail}
\pandocbounded{\includegraphics[keepaspectratio]{https://packages.typst.org/preview/thumbnails/modern-nju-thesis-0.3.4-small.webp}}

\section{modern-nju-thesis}\label{modern-nju-thesis}

{ 0.3.4 }

å?---京大学学ä½?论æ--‡æ¨¡æ?¿ã€‚Modern Nanjing University Thesis.

\href{/app?template=modern-nju-thesis&version=0.3.4}{Create project in
app}

\phantomsection\label{readme}
å?---京大学毕业论æ--‡ï¼ˆè®¾è®¡ï¼‰çš„ Typst
模æ?¿ï¼Œèƒ½å¤Ÿç®€æ´?ã€?快速ã€?æŒ?ç»­ç''Ÿæˆ? PDF
æ~¼å¼?的毕业论æ--‡ã€‚
\href{https://typst.app/universe/package/modern-nju-thesis}{Typst
Universe}

Typst é?žå®˜æ--¹ä¸­æ--‡äº¤æµ?群:793548390

å?---京大学 Typst 交æµ?群:943622984

\pandocbounded{\includegraphics[keepaspectratio]{https://github.com/typst/packages/raw/main/packages/preview/modern-nju-thesis/0.3.4/imgs/editor.png}}

\subsection{劣势}\label{uxe5ux161uxe5ux161}

\begin{itemize}
\tightlist
\item
  Typst 是一é---¨æ--°ç''Ÿçš„æŽ'版æ~‡è®°è¯­è¨€ï¼Œè¿˜å?šä¸?到åƒ? Word
  æˆ-- LaTeX 一æ~·æˆ?熟稳定。
\item
  该模æ?¿å¹¶é?žå®˜æ--¹æ¨¡æ?¿ï¼Œè€Œæ˜¯æ°`é---´æ¨¡æ?¿ï¼Œ
  \textbf{存在�被认�的风险} 。
\end{itemize}

\subsection{优势}\label{uxe4uxbcuxe5ux161}

Typst
是å?¯ç''¨äºŽå‡ºç‰ˆçš„å?¯ç¼--程æ~‡è®°è¯­è¨€ï¼Œæ‹¥æœ‰å?˜é‡?ã€?函数与åŒ\ldots 管ç?†ç­‰çŽ°ä»£ç¼--程语言的特性,注é‡?于ç§`学写作
(science writing),定ä½?与 LaTeX 相似。å?¯ä»¥é˜\ldots 读æˆ`çš„
\href{https://zhuanlan.zhihu.com/p/669097092}{一篇知乎æ--‡ç«}
进一步了解 Typst 的优势。

\begin{itemize}
\tightlist
\item
  \textbf{语法简�} :上手难度跟 Markdown
  相å½``,æ--‡æœ¬æº?ç~?é˜\ldots 读性高,ä¸?会åƒ? LaTeX
  一æ~·å\ldots\ldots æ--¥ç?€å??æ--œæ?~与花括å?·ã€‚
\item
  \textbf{ç¼--è¯`速度快} :Typst 使ç''¨ Rust 语言ç¼--写,å?³
  typ(esetting+ru)st,目æ~‡è¿?行平å?°æ˜¯WASM,å?³æµ?览器本地离线è¿?行;也å?¯ä»¥ç¼--è¯`æˆ?å`½ä»¤è¡Œå·¥å\ldots·ï¼Œé‡‡ç''¨ä¸€ç§?
  \textbf{增é‡?ç¼--è¯`}
  ç®---法å'Œä¸€ç§?有约æ?Ÿçš„版é?¢ç¼``å­˜æ--¹æ¡ˆï¼Œ
  \textbf{æ--‡æ¡£é•¿åº¦åŸºæœ¬ä¸?会影å``?ç¼--è¯`速度,ä¸''ç¼--è¯`速度与常è§?
  Markdown 渲æŸ``引æ``Žæ¸²æŸ``速度相å½``} 。
\item
  \textbf{环境�建简�} :�需�� LaTeX
  一æ~·æŠ˜è\ldots¾å‡~个 G
  çš„å¼€å?{}`环境,原ç''Ÿæ''¯æŒ?中æ---¥éŸ©ç­‰é?žæ‹‰ä¸?语言,æ---~论是官æ--¹
  Web App 在线ç¼--è¾`,还是使ç''¨ VS Code
  安è£\ldots æ?'件本地开å?{}`,都是 \textbf{å?³å¼€å?³ç''¨} 。
\item
  \textbf{现代ç¼--程语言} :Typst
  是å?¯ç''¨äºŽå‡ºç‰ˆçš„å?¯ç¼--程æ~‡è®°è¯­è¨€ï¼Œæ‹¥æœ‰
  \textbf{å?˜é‡?ã€?函数ã€?åŒ\ldots 管ç?†ä¸Žé''™è¯¯æ£€æŸ¥}
  等现代ç¼--程语言的特性,å?Œæ---¶ä¹Ÿæ??供了 \textbf{é---­åŒ}
  等特性,便于进行 \textbf{函数å¼?ç¼--程}
  。以å?ŠåŒ\ldots 括了 \texttt{\ {[}æ~‡è®°æ¨¡å¼?{]}\ } ã€?
  \texttt{\ \{脚本模�\}\ } 与 \texttt{\ \$数学模�\$\ }
  等多ç§?模å¼?的作ç''¨åŸŸï¼Œå¹¶ä¸''它们å?¯ä»¥ä¸?é™?深度地ã€?交äº'地嵌å¥---。并ä¸''通过
  \textbf{åŒ\ldots 管ç?†} ,ä½~ä¸?å†?需è¦?åƒ? TexLive
  一æ~·åœ¨æœ¬åœ°å®‰è£\ldots 一大å~†å¹¶ä¸?å¿\ldots è¦?çš„å®?åŒ\ldots ,而是
  \textbf{按需自动从äº`端下载} 。
\end{itemize}

å?¯ä»¥å?‚考æˆ`å?‚与æ?­å»ºå'Œç¿»è¯`çš„
\href{https://typst-doc-cn.github.io/docs/}{Typst 中æ--‡æ--‡æ¡£ç½`ç«™}
è¿\ldots 速å\ldots¥é---¨ã€‚

\subsection{使ç''¨}\label{uxe4uxbduxe7}

快速�览效果: 查看
\href{https://github.com/nju-lug/modern-nju-thesis/releases/latest/download/thesis.pdf}{thesis.pdf}
,æ~·ä¾‹è®ºæ--‡æº?ç~?:查看
\href{https://github.com/nju-lug/modern-nju-thesis/blob/main/template/thesis.typ}{thesis.typ}

\textbf{ä½~å?ªéœ€è¦?ä¿®æ''¹ \texttt{\ thesis.typ\ }
æ--‡ä»¶å?³å?¯ï¼ŒåŸºæœ¬å?¯ä»¥æ»¡è¶³ä½~的所有需求。}

如果ä½~认为ä¸?能满足ä½~的需求,å?¯ä»¥å\ldots ˆæŸ¥é˜\ldots å?Žé?¢çš„
\href{https://github.com/typst/packages/raw/main/packages/preview/modern-nju-thesis/0.3.4/\#Q\%26A}{Q\&A}
部分。

模æ?¿å·²ç»?上ä¼~到了 Typst
Universe,使ç''¨èµ·æ?¥å??分简å?•ï¼Œç?†è®ºä¸Šå?ªéœ€è¦?通过

\begin{Shaded}
\begin{Highlighting}[]
\NormalTok{\#import "@preview/modern{-}nju{-}thesis:0.3.4": documentclass}
\end{Highlighting}
\end{Shaded}

导å\ldots¥å?³å?¯ã€‚

\subsubsection{在线ç¼--è¾`}\label{uxe5ux153uxe7uxbauxe7uxbcuxe8uxbe}

Typst æ??供了官æ--¹çš„ Web App,æ''¯æŒ?åƒ? Overleaf
一æ~·åœ¨çº¿ç¼--è¾`,这是一个
\href{https://typst.app/project/rgiwHIjdPOnXr9HJb8H0oa}{例�} 。

实é™\ldots 上,æˆ`们å?ªéœ€è¦?在
\href{https://typst.app/?template=modern-nju-thesis&version=0.3.4}{Web
App} 中的 \texttt{\ Start\ from\ template\ } 里选择
\texttt{\ modern-nju-thesis\ } ,å?³å?¯åœ¨çº¿åˆ›å»ºæ¨¡æ?¿å¹¶ä½¿ç''¨ã€‚

\pandocbounded{\includegraphics[keepaspectratio]{https://github.com/typst/packages/raw/main/packages/preview/modern-nju-thesis/0.3.4/imgs/template.png}}

\pandocbounded{\includegraphics[keepaspectratio]{https://github.com/typst/packages/raw/main/packages/preview/modern-nju-thesis/0.3.4/imgs/webapp.png}}

\textbf{但是 Web App 并没有安è£\ldots 本地 Windows æˆ-- MacOS
所拥有的å­---ä½``,所以å­---ä½``上å?¯èƒ½å­˜åœ¨å·®å¼‚,所以推è??本地ç¼--è¾`ï¼?}

\textbf{ä½~需è¦?手动上ä¼~ fonts
目录下的å­---ä½``æ--‡ä»¶åˆ°é¡¹ç›®ä¸­ï¼Œå?¦åˆ™ä¼šå¯¼è‡´å­---ä½``显示é''™è¯¯ï¼?}

PS: 虽然与 Overleaf
看起�相似,但是它们底层原�并�相�。Overleaf
是在���务器�行了一个 LaTeX
ç¼--è¯`器,本质上是计ç®---密集型的æœ?务;而 Typst
å?ªéœ€è¦?在æµ?览器端使ç''¨ WASM 技术执行,本质上是 IO
密集型的æœ?务,所以对æœ?务器压力很å°?(å?ªéœ€è¦?è´Ÿè´£æ--‡ä»¶çš„äº`存储与å??作å?Œæ­¥åŠŸèƒ½ï¼‰ã€‚

\subsubsection{VS Code
本地ç¼--è¾`(推è??)}\label{vs-code-uxe6ux153uxe5ux153uxe7uxbcuxe8uxbeuxefuxbcux2c6uxe6ux17euxe8uxefuxbc}

\begin{enumerate}
\tightlist
\item
  在 VS Code 中安è£
  \href{https://marketplace.visualstudio.com/items?itemName=myriad-dreamin.tinymist}{Tinymist
  Typst} å'Œ
  \href{https://marketplace.visualstudio.com/items?itemName=mgt19937.typst-preview}{Typst
  Preview}
  æ?'件。å‰?è€\ldots 负责语法高亮å'Œé''™è¯¯æ£€æŸ¥ç­‰åŠŸèƒ½ï¼Œå?Žè€\ldots 负责预览。

  \begin{itemize}
  \tightlist
  \item
    也推è??下载
    \href{https://marketplace.visualstudio.com/items?itemName=CalebFiggers.typst-companion}{Typst
    Companion} æ?'件,å\ldots¶æ??供了例如 \texttt{\ Ctrl\ +\ B\ }
    进行åŠ~ç²---等便æ?·çš„å¿«æ?·é''®ã€‚
  \item
    ä½~还å?¯ä»¥ä¸‹è½½æˆ`å¼€å?{}`çš„
    \href{https://marketplace.visualstudio.com/items?itemName=OrangeX4.vscode-typst-sync}{Typst
    Sync} å'Œ
    \href{https://marketplace.visualstudio.com/items?itemName=OrangeX4.vscode-typst-sympy-calculator}{Typst
    Sympy Calculator}
    æ?'件,å‰?è€\ldots æ??供了本地åŒ\ldots çš„äº`å?Œæ­¥åŠŸèƒ½ï¼Œå?Žè€\ldots æ??供了基于
    Typst 语法的ç§`学计ç®---器功能。
  \end{itemize}
\item
  按下 \texttt{\ Ctrl\ +\ Shift\ +\ P\ }
  æ‰``å¼€å`½ä»¤ç•Œé?¢ï¼Œè¾``å\ldots¥
  \texttt{\ Typst:\ Show\ available\ Typst\ templates\ (gallery)\ for\ picking\ up\ a\ template\ }
  æ‰``å¼€ Tinymist æ??供的 Template Gallery,然å?Žä»Žé‡Œé?¢æ‰¾åˆ°
  \texttt{\ modern-nju-thesis\ } ,点击 \texttt{\ �\ }
  按é'®è¿›è¡Œæ''¶è---?,以å?Šç‚¹å‡» \texttt{\ +\ }
  å?·ï¼Œå°±å?¯ä»¥åˆ›å»ºå¯¹åº''的论æ--‡æ¨¡æ?¿äº†ã€‚
\item
  最å?Žç''¨ VS Code æ‰``å¼€ç''Ÿæˆ?的目录,æ‰``å¼€
  \texttt{\ thesis.typ\ } æ--‡ä»¶ï¼Œå¹¶æŒ‰ä¸‹ \texttt{\ Ctrl\ +\ K\ V\ }
  进行实æ---¶ç¼--è¾`å'Œé¢„览。
\end{enumerate}

\pandocbounded{\includegraphics[keepaspectratio]{https://github.com/typst/packages/raw/main/packages/preview/modern-nju-thesis/0.3.4/imgs/gallery.png}}

\subsubsection{特性 /
路线图}\label{uxe7uxb9uxe6-uxe8uxe7uxbauxe5uxbe}

\begin{itemize}
\tightlist
\item
  \textbf{说明æ--‡æ¡£}

  \begin{itemize}
  \tightlist
  \item
    {[} {]} ç¼--写更详细的说明æ--‡æ¡£ï¼Œå?Žç»­è€ƒè™`使ç''¨
    \href{https://github.com/typst/packages/tree/main/packages/preview/tidy/0.1.0}{tidy}
    ç¼--写,ä½~现在å?¯ä»¥å\ldots ˆå?‚考
    \href{https://mirror-hk.koddos.net/CTAN/macros/unicodetex/latex/njuthesis/njuthesis.pdf}{NJUThesis}
    çš„æ--‡æ¡£ï¼Œå?‚数大ä½``ä¿?æŒ?一致,æˆ--è€\ldots 直接查é˜\ldots 对åº''æº?ç~?函数的å?‚æ•°
  \end{itemize}
\item
  \textbf{类型检查}

  \begin{itemize}
  \tightlist
  \item
    {[} {]}
    åº''该对所有函数å\ldots¥å?‚进行类型检查,å?Šæ---¶æŠ¥é''™
  \end{itemize}
\item
  \textbf{å\ldots¨å±€é\ldots?ç½®}

  \begin{itemize}
  \tightlist
  \item
    {[}x{]} 类似 LaTeX 中的 \texttt{\ documentclass\ }
    çš„å\ldots¨å±€ä¿¡æ?¯é\ldots?ç½®
  \item
    {[}x{]} \textbf{盲审模�}
    ,将个人信æ?¯æ›¿æ?¢æˆ?å°?é»`æ?¡ï¼Œå¹¶ä¸''éš?è---?致谢页é?¢ï¼Œè®ºæ--‡æ??交阶段使ç''¨
  \item
    {[}x{]} \textbf{��模�}
    ,会åŠ~å\ldots¥ç©ºç™½é¡µï¼Œä¾¿äºŽæ‰``å?°
  \item
    {[}x{]} \textbf{自定义å­---ä½``é\ldots?ç½®}
    ,å?¯ä»¥é\ldots?置「宋ä½``ã€?ã€?「é»`ä½``ã€?与「楷ä½``ã€?ç­‰å­---ä½``对åº''çš„å\ldots·ä½``å­---ä½``
  \item
    {[}x{]} \textbf{æ•°å­¦å­---ä½``é\ldots?ç½®}
    :模æ?¿ä¸?æ??ä¾›é\ldots?置,ç''¨æˆ·å?¯ä»¥è‡ªå·±ä½¿ç''¨
    \texttt{\ \#show\ math.equation:\ set\ text(font:\ "Fira\ Math")\ }
    æ›´æ''¹
  \end{itemize}
\item
  \textbf{模�}

  \begin{itemize}
  \tightlist
  \item
    {[}x{]} 本ç§`ç''Ÿæ¨¡æ?¿

    \begin{itemize}
    \tightlist
    \item
      {[}x{]} å­---ä½``测试页
    \item
      {[}x{]} å°?é?¢
    \item
      {[}x{]} 声明页
    \item
      {[}x{]} 中æ--‡æ`˜è¦?
    \item
      {[}x{]} 英æ--‡æ`˜è¦?
    \item
      {[}x{]} 目录页
    \item
      {[}x{]} æ?'图目录
    \item
      {[}x{]} 表æ~¼ç›®å½•
    \item
      {[}x{]} 符�表
    \item
      {[}x{]} 致谢
    \end{itemize}
  \item
    {[}x{]} ç~''究ç''Ÿæ¨¡æ?¿

    \begin{itemize}
    \tightlist
    \item
      {[}x{]} å°?é?¢
    \item
      {[}x{]} 声明页
    \item
      {[}x{]} æ`˜è¦?
    \item
      {[}x{]} 页眉
    \item
      {[} {]} 国家图书馆��
    \item
      {[} {]} 出版授�书
    \end{itemize}
  \item
    {[} {]} �士�模�
  \end{itemize}
\item
  \textbf{ç¼--å?·}

  \begin{itemize}
  \tightlist
  \item
    {[}x{]} å‰?言使ç''¨ç½---马数å­---ç¼--å?·
  \item
    {[}x{]} 附录使ç''¨ç½---马数å­---ç¼--å?·
  \item
    {[}x{]} 表æ~¼ä½¿ç''¨ \texttt{\ 1.1\ } æ~¼å¼?进行ç¼--å?·
  \item
    {[}x{]} æ•°å­¦å\ldots¬å¼?使ç''¨ \texttt{\ (1.1)\ }
    æ~¼å¼?进行ç¼--å?·
  \end{itemize}
\item
  \textbf{环境}

  \begin{itemize}
  \tightlist
  \item
    {[} {]}
    定ç?†çŽ¯å¢ƒï¼ˆè¿™ä¸ªä¹Ÿå?¯ä»¥è‡ªå·±ä½¿ç''¨ç¬¬ä¸‰æ--¹åŒ\ldots é\ldots?置)
  \end{itemize}
\item
  \textbf{å\ldots¶ä»--æ--‡ä»¶}

  \begin{itemize}
  \tightlist
  \item
    {[}x{]} 本ç§`ç''Ÿå¼€é¢˜æŠ¥å`Š
  \item
    {[}x{]} ç~''究ç''Ÿå¼€é¢˜æŠ¥å`Š
  \end{itemize}
\end{itemize}

\subsection{å\ldots¶ä»--æ--‡ä»¶}\label{uxe5uxe4uxe6uxe4}

还实现了本ç§`ç''Ÿå'Œç~''究ç''Ÿçš„开题报å`Šï¼Œå?ªéœ€è¦?预览å'Œç¼--è¾`
\texttt{\ others\ } 目录下的æ--‡ä»¶å?³å?¯ã€‚

\pandocbounded{\includegraphics[keepaspectratio]{https://github.com/typst/packages/raw/main/packages/preview/modern-nju-thesis/0.3.4/imgs/proposal.png}}

\subsection{Q\&A}\label{qa}

\subsubsection{æˆ`ä¸?会
LaTeX,å?¯ä»¥ç''¨è¿™ä¸ªæ¨¡æ?¿å†™è®ºæ--‡å?---?}\label{uxe6ux2c6uxe4uxe4uxbcux161-latexuxefuxbcux153uxe5uxe4uxe7uxe8uxe4uxaauxe6uxe6uxe5uxe8uxbauxe6uxe5uxefuxbcuxff}

�以。

如果ä½~ä¸?å\ldots³æ³¨æ¨¡æ?¿çš„å\ldots·ä½``实现原ç?†ï¼Œä½~å?¯ä»¥ç''¨
Markdown Like
的语法进行ç¼--写,å?ªéœ€è¦?按ç\ldots§æ¨¡æ?¿çš„ç»``æž„ç¼--写å?³å?¯ã€‚

\subsubsection{æˆ`ä¸?会ç¼--程,å?¯ä»¥ç''¨è¿™ä¸ªæ¨¡æ?¿å†™è®ºæ--‡å?---?}\label{uxe6ux2c6uxe4uxe4uxbcux161uxe7uxbcuxe7uxefuxbcux153uxe5uxe4uxe7uxe8uxe4uxaauxe6uxe6uxe5uxe8uxbauxe6uxe5uxefuxbcuxff}

å?Œæ~·å?¯ä»¥ã€‚

如果ä»\ldots ä»\ldots 是å½``æˆ?是å\ldots¥é---¨ä¸€æ¬¾ç±»ä¼¼äºŽ
Markdown 的语言,相信使ç''¨è¯¥æ¨¡æ?¿çš„ä½``验会æ¯''使ç''¨ Word
ç¼--写更好。

\subsubsection{为什么æˆ`çš„å­---ä½``没有显示出æ?¥ï¼Œè€Œæ˜¯ä¸€ä¸ªä¸ªã€Œè±†è\ldots?å?---ã€??}\label{uxe4uxbauxe4uxe4uxb9ux2c6uxe6ux2c6uxe7ux161uxe5uxe4uxbduxe6uxb2uxe6ux153uxe6uxbeuxe7uxbauxe5uxbauxe6uxefuxbcux153uxe8ux153uxe6uxe4uxe4uxaauxe4uxaauxe3ux153uxe8uxe8uxe5uxe3uxefuxbcuxff}

这是å›~为本地没有对åº''çš„å­---ä½``,
\textbf{è¿™ç§?æƒ\ldots 况ç»?常å?{}`ç''Ÿåœ¨ MacOS
的「楷ä½``ã€?显示上} 。

ä½~åº''该安è£\ldots 本目录下的 \texttt{\ fonts\ }
里的所有å­---ä½``,里é?¢åŒ\ldots å?«äº†å?¯ä»¥å\ldots?费商ç''¨çš„「æ--¹æ­£æ¥·ä½``ã€?å'Œã€Œæ--¹æ­£ä»¿å®‹ã€?,然å?Žå†?é‡?æ--°æ¸²æŸ``测试å?³å?¯ã€‚

ä½~å?¯ä»¥ä½¿ç''¨ \texttt{\ \#fonts-display-page()\ }
显示一个å­---ä½``渲æŸ``测试页é?¢ï¼ŒæŸ¥çœ‹å¯¹åº''çš„å­---ä½``是å?¦æ˜¾ç¤ºæˆ?功。

如果还是ä¸?能æˆ?功,ä½~å?¯ä»¥æŒ‰ç\ldots§æ¨¡æ?¿é‡Œçš„说明自行é\ldots?ç½®å­---ä½``,例如

\begin{Shaded}
\begin{Highlighting}[]
\NormalTok{\#let (...) = documentclass(}
\NormalTok{  fonts: (楷体: ("Times New Roman", "FZKai{-}Z03S")),}
\NormalTok{)}
\end{Highlighting}
\end{Shaded}

å\ldots ˆæ˜¯å¡«å†™è‹±æ--‡å­---ä½``,然å?Žå†?填写ä½~需è¦?的「楷ä½``ã€?中æ--‡å­---ä½``。

\textbf{å­---ä½``å??称å?¯ä»¥é€šè¿‡ \texttt{\ typst\ fonts\ }
å`½ä»¤æŸ¥è¯¢ã€‚}

如果找ä¸?到ä½~所需è¦?çš„å­---ä½``,å?¯èƒ½æ˜¯å›~为
\textbf{该å­---ä½``å?˜ä½``(Variants)数é‡?过å°`} ,导致 Typst
æ---~法识别到该中æ--‡å­---ä½``。

\subsubsection{å­¦ä¹~ Typst
需è¦?多ä¹\ldots ?}\label{uxe5uxe4uxb9-typst-uxe9ux153uxe8uxe5ux161uxe4uxb9uxefuxbcuxff}

一般而言,ä»\ldots ä»\ldots 进行简å?•çš„ç¼--写,ä¸?å\ldots³æ³¨å¸ƒå±€çš„è¯?,ä½~å?¯ä»¥æ‰``开模æ?¿å°±å¼€å§‹å†™äº†ã€‚

如果ä½~想进一步学ä¹~ Typst
的语法,例如如何æŽ'篇布局,如何设置页脚页眉等,一般å?ªéœ€è¦?å‡~个å°?æ---¶å°±èƒ½å­¦ä¼šã€‚

如果ä½~还想学ä¹~ Typst 的「
\href{https://typst-doc-cn.github.io/docs/reference/meta/}{å\ldots ƒä¿¡æ?¯}
ã€?部分,进而能够ç¼--写自己的模æ?¿ï¼Œä¸€èˆ¬è€Œè¨€éœ€è¦?å‡~天的æ---¶é---´é˜\ldots 读æ--‡æ¡£ï¼Œä»¥å?Šä»--人ç¼--写的模æ?¿ä»£ç~?。

如果ä½~有 Python æˆ-- JavaScript
等脚本语言的ç¼--写ç»?验,了解过函数å¼?ç¼--程ã€?å®?ã€?æ~·å¼?ã€?组件åŒ--å¼€å?{}`等概念,å\ldots¥é---¨é€Ÿåº¦ä¼šå¿«å¾ˆå¤šã€‚

\subsubsection{æˆ`有ç¼--写 LaTeX
çš„ç»?验,如何快速å\ldots¥é---¨ï¼Ÿ}\label{uxe6ux2c6uxe6ux153uxe7uxbcuxe5-latex-uxe7ux161uxe7uxe9uxaaux153uxefuxbcux153uxe5uxe4uxbduxe5uxe9uxffuxe5uxe9uxefuxbcuxff}

�以�考
\href{https://typst-doc-cn.github.io/docs/guides/guide-for-latex-users/}{é?¢å?{}`
LaTeX ç''¨æˆ·çš„ Typst å\ldots¥é---¨æŒ‡å?---} 。

\subsubsection{目� Typst
有å``ªäº›ç¬¬ä¸‰æ--¹åŒ\ldots å'Œæ¨¡æ?¿ï¼Ÿ}\label{uxe7uxe5-typst-uxe6ux153uxe5uxaauxe4uxbauxe7uxe4uxe6uxb9uxe5ux153uxe5ux153uxe6uxe6uxefuxbcuxff}

�以查看 \href{https://typst.app/universe}{Typst Universe} 。

\subsubsection{为什么�有一个 thesis.typ
æ--‡ä»¶ï¼Œæ²¡æœ‰æŒ‰ç«~节分多个æ--‡ä»¶ï¼Ÿ}\label{uxe4uxbauxe4uxe4uxb9ux2c6uxe5uxaauxe6ux153uxe4uxe4uxaa-thesis.typ-uxe6uxe4uxefuxbcux153uxe6uxb2uxe6ux153uxe6ux153uxe7-uxe8ux161uxe5ux2c6uxe5ux161uxe4uxaauxe6uxe4uxefuxbcuxff}

å›~为 Typst \textbf{语法足够简æ´?} ã€?
\textbf{ç¼--è¯`速度足够快} ã€?并ä¸''
\textbf{拥有å\ldots‰æ~‡ç‚¹å‡»å¤„å?Œå?{}`é``¾æŽ¥åŠŸèƒ½} 。

语法简æ´?的好处是,å?³ä½¿æŠŠæ‰€æœ‰å†\ldots 容都写在å?Œä¸€ä¸ªæ--‡ä»¶ï¼Œä½~也å?¯ä»¥å¾ˆç®€å?•åœ°åˆ†è¾¨å‡ºå?„个部分的å†\ldots 容。

ç¼--è¯`速度足够快的好处是,ä½~ä¸?å†?需è¦?åƒ? LaTeX
一æ~·ï¼Œå°†å†\ldots 容分散在å‡~个æ--‡ä»¶ï¼Œå¹¶é€šè¿‡æ³¨é‡Šçš„æ--¹å¼?æ??高ç¼--è¯`速度。

å\ldots‰æ~‡ç‚¹å‡»å¤„å?Œå?{}`é``¾æŽ¥åŠŸèƒ½ï¼Œä½¿å¾---ä½~å?¯ä»¥ç›´æŽ¥æ‹--动预览çª---å?£åˆ°ä½~想è¦?çš„ä½?置,然å?Žç''¨é¼~æ~‡ç‚¹å‡»å?³å?¯åˆ°è¾¾å¯¹åº''æº?ç~?所在ä½?置。

还有一个好处是,å?•ä¸ªæº?æ--‡ä»¶ä¾¿äºŽå?Œæ­¥å'Œåˆ†äº«ã€‚

å?³ä½¿ä½~还是想è¦?分æˆ?å‡~个ç«~节,也是å?¯ä»¥çš„,Typst
æ''¯æŒ?ä½~使ç''¨ \texttt{\ \#import\ } å'Œ \texttt{\ \#include\ }
语法将å\ldots¶ä»--æ--‡ä»¶çš„å†\ldots 容导å\ldots¥æˆ--ç½®å\ldots¥ã€‚ä½~å?¯ä»¥æ--°å»ºæ--‡ä»¶å¤¹
\texttt{\ chapters\ }
,然å?Žå°†å?„个ç«~节的æº?æ--‡ä»¶æ''¾è¿›åŽ»ï¼Œç„¶å?Žé€šè¿‡
\texttt{\ \#include\ } ç½®å\ldots¥ \texttt{\ thesis.typ\ } 里。

\subsubsection{æˆ`如何更æ''¹é¡µé?¢ä¸Šçš„æ~·å¼??å\ldots·ä½``的语法是怎么æ~·çš„?}\label{uxe6ux2c6uxe5uxe4uxbduxe6uxe6uxb9uxe9uxb5uxe9uxe4ux161uxe7ux161uxe6-uxe5uxbcuxefuxbcuxffuxe5uxe4uxbduxe7ux161uxe8uxe6uxb3uxe6uxe6ux17euxe4uxb9ux2c6uxe6-uxe7ux161uxefuxbcuxff}

ç?†è®ºä¸Šä½~并ä¸?需è¦?æ›´æ''¹ \texttt{\ nju-thesis\ }
目录下的任何æ--‡ä»¶ï¼Œæ---~论是æ~·å¼?还是å\ldots¶ä»--çš„é\ldots?置,ä½~都å?¯ä»¥åœ¨
\texttt{\ thesis.typ\ }
æ--‡ä»¶å†\ldots ä¿®æ''¹å‡½æ•°å?‚数实现更æ''¹ã€‚å\ldots·ä½``çš„æ›´æ''¹æ--¹å¼?å?¯ä»¥é˜\ldots 读
\texttt{\ nju-thesis\ } 目录下的æ--‡ä»¶çš„函数å?‚数。

例如,想è¦?æ›´æ''¹é¡µé?¢è¾¹è·?为 \texttt{\ 50pt\ } ,å?ªéœ€è¦?å°†

\begin{Shaded}
\begin{Highlighting}[]
\NormalTok{\#show: doc}
\end{Highlighting}
\end{Shaded}

æ''¹ä¸º

\begin{Shaded}
\begin{Highlighting}[]
\NormalTok{\#show: doc.with(margin: (x: 50pt))}
\end{Highlighting}
\end{Shaded}

��。

å?Žç»­æˆ`也会ç¼--写一个更详细的æ--‡æ¡£ï¼Œå?¯èƒ½ä¼šè€ƒè™`使ç''¨
\href{https://github.com/typst/packages/tree/main/packages/preview/tidy/0.1.0}{tidy}
æ?¥ç¼--写。

如果ä½~é˜\ldots 读了那些函数的å?‚数,ä»?然ä¸?知é?{}``如何修æ''¹å¾---到ä½~需è¦?çš„æ~·å¼?,欢迎æ??出
Issue,å?ªè¦?æ??è¿°æ¸\ldots 楚é---®é¢˜å?³å?¯ã€‚

æˆ--è€\ldots 也欢迎åŠ~群讨论:943622984

\subsubsection{该模æ?¿å'Œå\ldots¶ä»--现存 Typst
中æ--‡è®ºæ--‡æ¨¡æ?¿çš„区别?}\label{uxe8uxe6uxe6uxe5ux153uxe5uxe4uxe7ux17euxe5-typst-uxe4uxe6uxe8uxbauxe6uxe6uxe6uxe7ux161uxe5ux153uxbauxe5ux2c6uxefuxbcuxff}

å\ldots¶ä»--现存的 Typst 中æ--‡è®ºæ--‡æ¨¡æ?¿å¤§å¤šéƒ½æ˜¯åœ¨ 2023 å¹´ 7
月份之å‰?(Typst Verison 0.6 之å‰?)开å?{}`的,å½``æ---¶ Typst
还ä¸?ä¸?够æˆ?熟,ç''šè‡³è¿ž \textbf{åŒ\ldots 管ç?†}
功能都还没有,å›~æ­¤å½``æ---¶çš„ Typst
中æ--‡è®ºæ--‡æ¨¡æ?¿çš„å¼€å?{}`è€\ldots 基本都是自己从头写了一é??需è¦?的功能/函数,å›~æ­¤é€~æˆ?了
\textbf{代ç~?耦å?ˆåº¦é«˜} ã€? \textbf{æ„?大利é?¢æ?¡å¼?代ç~?} ã€?
\textbf{é‡?å¤?é€~è½®å­?} 与 \textbf{难以自定义æ~·å¼?} ç­‰é---®é¢˜ã€‚

该模�是在 2023 年 10 ~ 11 月份(Typst Verison 0.9
æ---¶ï¼‰å¼€å?{}`的,此æ---¶ Typst 语法基本稳定,并ä¸''æ??供了
\textbf{åŒ\ldots 管ç?†}
功能,å›~此能够å‡?å°`很多ä¸?å¿\ldots è¦?的代ç~?。

并ä¸''æˆ`对模æ?¿çš„æ--‡ä»¶æž¶æž„进行了解耦,主è¦?分为了
\texttt{\ utils\ } ã€? \texttt{\ pages\ } å'Œ \texttt{\ layouts\ }
三个目录,这三个目录å?¯ä»¥çœ‹å?Žæ--‡çš„å¼€å?{}`è€\ldots 指å?---,并ä¸''使ç''¨
\textbf{é---­åŒ}
特性实现了类似ä¸?å?¯å?˜å\ldots¨å±€å?˜é‡?çš„å\ldots¨å±€é\ldots?置能力,å?³æ¨¡æ?¿ä¸­çš„
\texttt{\ documentclass\ } 函数类。

\subsubsection{æˆ`ä¸?是å?---京大学本ç§`ç''Ÿï¼Œå¦‚何è¿?移该模æ?¿ï¼Ÿ}\label{uxe6ux2c6uxe4uxe6uxe5uxe4uxbauxe5uxe5uxe6ux153uxe7uxe7uxffuxefuxbcux153uxe5uxe4uxbduxe8uxe7uxe8uxe6uxe6uxefuxbcuxff}

æˆ`在开å?{}`的过程中已ç»?对模æ?¿çš„å?„个模æ?¿è¿›è¡Œäº†è§£è€¦ï¼Œç?†è®ºä¸Šä½~å?ªéœ€è¦?在
\texttt{\ pages\ }
目录中åŠ~å\ldots¥ä½~需è¦?的页é?¢ï¼Œç„¶å?Žæ›´æ''¹å°`许ã€?æˆ--è€\ldots ä¸?需è¦?æ›´æ''¹å\ldots¶ä»--目录的代ç~?。

å\ldots·ä½``目录è?Œè´£åˆ'分å?¯ä»¥çœ‹ä¸‹é?¢çš„å¼€å?{}`è€\ldots 指å?---。

\subsection{å¼€å?{}`è€\ldots 指å?---}\label{uxe5uxbcuxe5uxe8uxe6ux153uxe5}

\subsubsection{template 目录}\label{template-uxe7uxe5uxbd}

\begin{itemize}
\tightlist
\item
  \texttt{\ thesis.typ\ } æ--‡ä»¶:
  ä½~的论æ--‡æº?æ--‡ä»¶ï¼Œå?¯ä»¥éš?æ„?æ›´æ''¹è¿™ä¸ªæ--‡ä»¶çš„å??å­---,ç''šè‡³ä½~å?¯ä»¥å°†è¿™ä¸ªæ--‡ä»¶åœ¨å?Œçº§ç›®å½•ä¸‹å¤?制多份,维æŒ?多个版本。
\item
  \texttt{\ ref.bib\ } æ--‡ä»¶: ç''¨äºŽæ''¾ç½®å?‚考æ--‡çŒ®ã€‚
\item
  \texttt{\ images\ } 目录: ç''¨äºŽæ''¾ç½®å›¾ç‰‡ã€‚
\end{itemize}

\subsubsection{å†\ldots 部目录}\label{uxe5uxe9ux192uxe7uxe5uxbd}

\begin{itemize}
\tightlist
\item
  \texttt{\ utils\ } 目录:
  åŒ\ldots å?«äº†æ¨¡æ?¿ä½¿ç''¨åˆ°çš„å?„ç§?自定义è¾\ldots 助函数,存æ''¾æ²¡æœ‰å¤--部ä¾?èµ--,ä¸''
  \textbf{ä¸?会渲æŸ``出页é?¢çš„函数} 。
\item
  \texttt{\ pages\ } 目录: åŒ\ldots å?«äº†æ¨¡æ?¿ç''¨åˆ°çš„å?„个
  \textbf{独立页é?¢} ,例如å°?é?¢é¡µã€?声明页ã€?æ`˜è¦?等,å?³
  \textbf{会渲æŸ``出ä¸?å½±å``?å\ldots¶ä»--页é?¢çš„独立页é?¢çš„函数}
  。
\item
  \texttt{\ layouts\ } 目录:
  布局目录,存æ''¾ç?€ç''¨äºŽæŽ'篇布局的ã€?åº''ç''¨äºŽ
  \texttt{\ show\ } 指令的� \textbf{横跨多个页�的函数}
  ,例如为了给页脚进行ç½---马数å­---ç¼--ç~?çš„å‰?言
  \texttt{\ preface\ } 函数。

  \begin{itemize}
  \tightlist
  \item
    主è¦?分æˆ?了 \texttt{\ doc\ } æ--‡ç¨¿ã€? \texttt{\ preface\ }
    å‰?言ã€? \texttt{\ mainmatter\ } æ­£æ--‡ä¸Ž \texttt{\ appendix\ }
    附录/�记。
  \end{itemize}
\item
  \texttt{\ lib.typ\ } :

  \begin{itemize}
  \tightlist
  \item
    \textbf{�责一} :
    作为一个统一的对å¤--接å?£ï¼Œæš´éœ²å‡ºå†\ldots 部的 utils
    函数。
  \item
    \textbf{è?Œè´£äºŒ} : 使ç''¨ \textbf{函数é---­åŒ} 特性,通过
    \texttt{\ documentclass\ }
    函数类进行å\ldots¨å±€ä¿¡æ?¯é\ldots?置,然å?Žæš´éœ²å‡ºæ‹¥æœ‰äº†å\ldots¨å±€é\ldots?置的ã€?å\ldots·ä½``çš„
    \texttt{\ layouts\ } å'Œ \texttt{\ pages\ } å†\ldots 部函数。
  \end{itemize}
\end{itemize}

\subsection{�与贡献}\label{uxe5uxe4ux17euxe8uxe7ux153}

\begin{itemize}
\tightlist
\item
  在 Issues
  中æ??出ä½~的想法,如果是æ--°ç‰¹æ€§ï¼Œå?¯ä»¥åŠ~å\ldots¥è·¯çº¿å›¾ï¼?
\item
  实现路线图中ä»?未实现的部分,然å?Žæ¬¢è¿Žæ??出ä½~çš„
  PR。
\item
  å?Œæ~·æ¬¢è¿Ž \textbf{将这个模æ?¿è¿?移至ä½~çš„å­¦æ~¡è®ºæ--‡æ¨¡æ?¿}
  ,大家一起æ?­å»ºæ›´å¥½çš„ Typst 社区å'Œç''Ÿæ€?å?§ã€‚
\end{itemize}

\subsection{致谢}\label{uxe8uxe8}

\begin{itemize}
\tightlist
\item
  æ„Ÿè°¢ \href{https://github.com/atxy-blip}{@atxy-blip} å¼€å?{}`çš„
  \href{https://github.com/nju-lug/NJUThesis}{NJUThesis} LaTeX
  模æ?¿ï¼Œæ--‡æ¡£å??分详细,本模æ?¿å¤§ä½``ç»``构都是å?‚考
  NJUThesis çš„æ--‡æ¡£å¼€å?{}`的。
\item
  æ„Ÿè°¢ \href{https://github.com/csimide}{@csimide}
  帮忙补å\ldots\ldots çš„
  \href{https://github.com/nju-lug/modern-nju-thesis/issues/3}{bilingual-bibliography}
  。
\item
  æ„Ÿè°¢
  \href{https://github.com/werifu/HUST-typst-template}{HUST-typst-template}
  与
  \href{https://github.com/howardlau1999/sysu-thesis-typst}{sysu-thesis-typst}
  ç­‰ Typst 中æ--‡è®ºæ--‡æ¨¡æ?¿ã€‚
\end{itemize}

\subsection{License}\label{license}

This project is licensed under the MIT License.

\href{/app?template=modern-nju-thesis&version=0.3.4}{Create project in
app}

\subsubsection{How to use}\label{how-to-use}

Click the button above to create a new project using this template in
the Typst app.

You can also use the Typst CLI to start a new project on your computer
using this command:

\begin{verbatim}
typst init @preview/modern-nju-thesis:0.3.4
\end{verbatim}

\includesvg[width=0.16667in,height=0.16667in]{/assets/icons/16-copy.svg}

\subsubsection{About}\label{about}

\begin{description}
\tightlist
\item[Author :]
OrangeX4
\item[License:]
MIT
\item[Current version:]
0.3.4
\item[Last updated:]
May 13, 2024
\item[First released:]
April 8, 2024
\item[Archive size:]
124 kB
\href{https://packages.typst.org/preview/modern-nju-thesis-0.3.4.tar.gz}{\pandocbounded{\includesvg[keepaspectratio]{/assets/icons/16-download.svg}}}
\item[Repository:]
\href{https://github.com/nju-lug/modern-nju-thesis}{GitHub}
\item[Categor y :]
\begin{itemize}
\tightlist
\item[]
\item
  \pandocbounded{\includesvg[keepaspectratio]{/assets/icons/16-mortarboard.svg}}
  \href{https://typst.app/universe/search/?category=thesis}{Thesis}
\end{itemize}
\end{description}

\subsubsection{Where to report issues?}\label{where-to-report-issues}

This template is a project of OrangeX4 . Report issues on
\href{https://github.com/nju-lug/modern-nju-thesis}{their repository} .
You can also try to ask for help with this template on the
\href{https://forum.typst.app}{Forum} .

Please report this template to the Typst team using the
\href{https://typst.app/contact}{contact form} if you believe it is a
safety hazard or infringes upon your rights.

\phantomsection\label{versions}
\subsubsection{Version history}\label{version-history}

\begin{longtable}[]{@{}ll@{}}
\toprule\noalign{}
Version & Release Date \\
\midrule\noalign{}
\endhead
\bottomrule\noalign{}
\endlastfoot
0.3.4 & May 13, 2024 \\
\href{https://typst.app/universe/package/modern-nju-thesis/0.3.3/}{0.3.3}
& April 15, 2024 \\
\href{https://typst.app/universe/package/modern-nju-thesis/0.3.2/}{0.3.2}
& April 9, 2024 \\
\href{https://typst.app/universe/package/modern-nju-thesis/0.3.1/}{0.3.1}
& April 8, 2024 \\
\href{https://typst.app/universe/package/modern-nju-thesis/0.3.0/}{0.3.0}
& April 8, 2024 \\
\end{longtable}

Typst GmbH did not create this template and cannot guarantee correct
functionality of this template or compatibility with any version of the
Typst compiler or app.


\section{Package List LaTeX/imprecv.tex}
\title{typst.app/universe/package/imprecv}

\phantomsection\label{banner}
\phantomsection\label{template-thumbnail}
\pandocbounded{\includegraphics[keepaspectratio]{https://packages.typst.org/preview/thumbnails/imprecv-1.0.1-small.webp}}

\section{imprecv}\label{imprecv}

{ 1.0.1 }

A no-frills curriculum vitae (CV) template using Typst and YAML to
version control CV data.

\href{/app?template=imprecv&version=1.0.1}{Create project in app}

\phantomsection\label{readme}
\href{https://github.com/jskherman/imprecv/stargazers}{\pandocbounded{\includesvg[keepaspectratio]{https://img.shields.io/badge/Star\%20Repo-\%E2\%AD\%90-1081c2.svg}}}
\href{https://github.com/typst/packages/raw/main/packages/preview/imprecv/1.0.1/LICENSE}{\pandocbounded{\includegraphics[keepaspectratio]{https://img.shields.io/badge/license-Apache\%202-brightgreen}}}
\href{https://github.com/jskherman/imprecv/releases}{\pandocbounded{\includegraphics[keepaspectratio]{https://img.shields.io/github/v/release/jskherman/imprecv}}}

\texttt{\ imprecv\ } is a no-frills curriculum vitae (CV) template for
\href{https://github.com/typst/typst}{Typst} that uses a YAML file for
data input in order to version control CV data easily.

This is based on the
\href{https://web.archive.org/https://old.reddit.com/r/jobs/comments/7y8k6p/im_an_exrecruiter_for_some_of_the_top_companies/}{popular
template on Reddit} by
\href{https://web.archive.org/https://old.reddit.com/user/SheetsGiggles}{u/SheetsGiggles}
and the recommendations of the
\href{https://web.archive.org/https://old.reddit.com/r/EngineeringResumes/comments/m2cc65/new_and_improved_wiki}{r/EngineeringResumes
wiki} .

\subsection{Demo}\label{demo}

See
\href{https://github.com/jskherman/imprecv/releases/latest/download/example.pdf}{\textbf{example
CV}} and \href{https://go.jskherman.com/cv}{@jskherman’s CV} :

\pandocbounded{\includegraphics[keepaspectratio]{https://raw.githubusercontent.com/jskherman/imprecv/main/assets/thumbnail.1.png}}
\pandocbounded{\includegraphics[keepaspectratio]{https://raw.githubusercontent.com/jskherman/imprecv/main/assets/thumbnail.2.png}}

\subsection{Usage}\label{usage}

This \texttt{\ imprecv\ } is intended to be used by importing the
template’s
\href{https://github.com/typst/packages/raw/main/packages/preview/imprecv/1.0.1/cv.typ}{package
entrypoint} from a “content� file (see
\href{https://github.com/typst/packages/raw/main/packages/preview/imprecv/1.0.1/template/template.typ}{\texttt{\ template.typ\ }}
as an example). In this content file, call the functions which apply
document styles, show CV components, and load CV data from a YAML file
(see
\href{https://github.com/typst/packages/raw/main/packages/preview/imprecv/1.0.1/template/template.yml}{\texttt{\ template.yml\ }}
as an example). Inside the content file you can modify several style
variables and even override existing function implementations to your
own needs and preferences.

\subsubsection{\texorpdfstring{With the
\href{https://github.com/typst/typst}{Typst
CLI}}{With the Typst CLI}}\label{with-the-typst-cli}

The recommended usage with the Typst CLI is by running the command
\texttt{\ typst\ init\ @preview/imprecv:1.0.1\ } in your project
directory. This will create a new Typst project with the
\texttt{\ imprecv\ } template and the necessary files to get started.
You can then run \texttt{\ typst\ compile\ template.typ\ } to compile
your file to PDF.

Take a look at the
\href{https://github.com/jskherman/cv.typ-example-repo}{example setup}
for ideas on how to get started. It includes a GitHub action workflow to
compile the Typst files to PDF and upload it to Cloudflare R2.

\subsubsection{\texorpdfstring{With
\href{https://typst.app/}{typst.app}}{With typst.app}}\label{with-typst.app}

From the Dashboard, select “Start from template�, search and choose
the \texttt{\ imprecv\ } template. From there, decide on a name for your
project and click “Create�. You can now edit the template files and
preview the result on the right.

You can also click the \texttt{\ Create\ project\ in\ app\ } button in
\href{https://typst.app/universe/package/imprecv}{Typst Universe} to
create a new project with the \texttt{\ imprecv\ } template.

\subsection{Contributing}\label{contributing}

\href{https://github.com/jskherman}{I’m} only doing programming as a
hobby so it might take me a while to respond to issues and pull
requests. If you would like to contribute to this project, I would be
happy to review your pull requests when I can. Thank you for your
understanding.

\href{/app?template=imprecv&version=1.0.1}{Create project in app}

\subsubsection{How to use}\label{how-to-use}

Click the button above to create a new project using this template in
the Typst app.

You can also use the Typst CLI to start a new project on your computer
using this command:

\begin{verbatim}
typst init @preview/imprecv:1.0.1
\end{verbatim}

\includesvg[width=0.16667in,height=0.16667in]{/assets/icons/16-copy.svg}

\subsubsection{About}\label{about}

\begin{description}
\tightlist
\item[Author :]
\href{https://jskherman.com}{Je Sian Keith Herman}
\item[License:]
Apache-2.0
\item[Current version:]
1.0.1
\item[Last updated:]
June 17, 2024
\item[First released:]
June 3, 2024
\item[Minimum Typst version:]
0.11.0
\item[Archive size:]
51.6 kB
\href{https://packages.typst.org/preview/imprecv-1.0.1.tar.gz}{\pandocbounded{\includesvg[keepaspectratio]{/assets/icons/16-download.svg}}}
\item[Repository:]
\href{https://github.com/jskherman/imprecv}{GitHub}
\item[Categor y :]
\begin{itemize}
\tightlist
\item[]
\item
  \pandocbounded{\includesvg[keepaspectratio]{/assets/icons/16-user.svg}}
  \href{https://typst.app/universe/search/?category=cv}{CV}
\end{itemize}
\end{description}

\subsubsection{Where to report issues?}\label{where-to-report-issues}

This template is a project of Je Sian Keith Herman . Report issues on
\href{https://github.com/jskherman/imprecv}{their repository} . You can
also try to ask for help with this template on the
\href{https://forum.typst.app}{Forum} .

Please report this template to the Typst team using the
\href{https://typst.app/contact}{contact form} if you believe it is a
safety hazard or infringes upon your rights.

\phantomsection\label{versions}
\subsubsection{Version history}\label{version-history}

\begin{longtable}[]{@{}ll@{}}
\toprule\noalign{}
Version & Release Date \\
\midrule\noalign{}
\endhead
\bottomrule\noalign{}
\endlastfoot
1.0.1 & June 17, 2024 \\
\href{https://typst.app/universe/package/imprecv/1.0.0/}{1.0.0} & June
3, 2024 \\
\end{longtable}

Typst GmbH did not create this template and cannot guarantee correct
functionality of this template or compatibility with any version of the
Typst compiler or app.


\section{Package List LaTeX/modern-acad-cv.tex}
\title{typst.app/universe/package/modern-acad-cv}

\phantomsection\label{banner}
\phantomsection\label{template-thumbnail}
\pandocbounded{\includegraphics[keepaspectratio]{https://packages.typst.org/preview/thumbnails/modern-acad-cv-0.1.1-small.webp}}

\section{modern-acad-cv}\label{modern-acad-cv}

{ 0.1.1 }

A CV template for academics based on moderncv LaTeX package.

\href{/app?template=modern-acad-cv&version=0.1.1}{Create project in app}

\phantomsection\label{readme}
This template for an academic CV serves the peculiarities of academic
CVs. If you are not an academic, this template is not useful. Most of
the times in academics, applicants need to show everything they have
done. This makes it a bit cumbersome doing it by single entries. In
addition, academics might apply to institutions around the globe, making
it necessary to send translated CVs or at least translations of some
parts (i.e., title of papers in different languages).

This template serves these special needs in introducting automated
sections based on indicated \texttt{\ yaml\ } -files. Furthermore, it
has a simplified multilingual support by setting different headers,
title etc. for different languages (by the user in the \texttt{\ yaml\ }
-fields). With this template, it might be more handy to keep your CV
easier on track, especially when you need in different languages, since
managing a \texttt{\ yaml\ } -file is easier than checking typesetting
files against each other.

This template is influenced by LaTeX’s
\href{https://github.com/moderncv/moderncv}{moderncv} and its typst
translation
\href{https://github.com/DeveloperPaul123/modern-cv}{moderner-cv} .

\subsection{Fonts}\label{fonts}

In this template, the use of FontAwesome icons via the
\href{https://typst.app/universe/package/fontawesome}{fontawesome typst
package} is possible, as well as the icons from Academicons
\href{https://typst.app/universe/package/use-academicons}{use-academicons
typst package} . To use these icons properly, you need to install each
fonts on your system. You can download
\href{https://fontawesome.com/download}{fontawesome here} and
\href{https://jpswalsh.github.io/academicons/}{academicons here} . Both
typst packages will be load by the template itself.

Furthermore, I included my favorite font
\href{https://fonts.google.com/specimen/Fira+Sans}{Fira Sans} . You can
download it here
\href{https://fonts.google.com/specimen/Fira+Sans}{here} , or just
change the font argument in \texttt{\ modern-acad-cv()\ } .

\subsection{Usage}\label{usage}

The main function to load the construct of the academic CV is
\texttt{\ modern-acad-cv()\ } . After importing the template, you can
call it right away. If you don’t have
\href{https://fonts.google.com/specimen/Fira+Sans}{Fira Sans} installed,
choose a different font. Examples are given below.

\begin{Shaded}
\begin{Highlighting}[]
\NormalTok{\#import "@preview/modern{-}acad{-}cv:0.1.1": *}

\NormalTok{\#show: modern{-}acad{-}cv.with(}
\NormalTok{  metadata,}
\NormalTok{  multilingual,}
\NormalTok{  lang: "en",}
\NormalTok{  font: ("Fira Sans", "Andale Mono", "Roboto"),}
\NormalTok{  show{-}date: true,}
\NormalTok{  body}
\NormalTok{)    }

\NormalTok{// ...}
\end{Highlighting}
\end{Shaded}

In the remainder, I show basic settings and how to use the automated
functions with the corresponding \texttt{\ yaml\ } -file.

\subsubsection{\texorpdfstring{Setting up the main file and the access
to the \texttt{\ yaml\ }
-files}{Setting up the main file and the access to the  yaml  -files}}\label{setting-up-the-main-file-and-the-access-to-the-yaml--files}

A first step in your document is to invoke the template. Second, since
this template works with \texttt{\ yaml\ } -files in the background you
need to specify paths to each \texttt{\ yaml\ } -file you want to use
throughout the document.

The template comes along with the \texttt{\ metadata.yaml\ } . In the
beginning of this yaml-file you set colors. Feel free to change it to
your preferred color scheme.

\begin{Shaded}
\begin{Highlighting}[]
\FunctionTok{colors}\KeywordTok{:}
\AttributeTok{  }\FunctionTok{main\_color}\KeywordTok{:}\AttributeTok{ }\StringTok{"\#579D90"}
\AttributeTok{  }\FunctionTok{lightgray\_color}\KeywordTok{:}\AttributeTok{ }\StringTok{"\#d5d5d5"}
\AttributeTok{  }\FunctionTok{gray\_color}\KeywordTok{:}\AttributeTok{ }\StringTok{"\#737373"}
\AttributeTok{  ...}
\end{Highlighting}
\end{Shaded}

At the beginning of your document, you just set then set the
metadata-object:

\begin{Shaded}
\begin{Highlighting}[]
\NormalTok{\#import "@preview/modern{-}acad{-}cv:0.1.0": *}

\NormalTok{\#let metadata = yaml("metadata.yaml")}
\end{Highlighting}
\end{Shaded}

Initially, the \texttt{\ metadata.yaml\ } is located on the same level
as the \texttt{\ example.typ\ } . All other \texttt{\ yaml\ } -files are
saved in the folder \texttt{\ dbs\ } . Since \texttt{\ typ\ } -documents
search for paths from the root of the document in that the function is
called, you have to give the databases for the entry along the
\texttt{\ metadata.yaml\ } within each function call.

\subsubsection{socials}\label{socials}

Contact details are important. In this CV template, you have the
possibility to use fontawesome icons and academicons. To use socials,
you just need to specify in \texttt{\ metadata.yaml\ } , the wanted
entries.

As you can see below, you set a category, i.e. email or lattes and then
you have to define four arguments: \texttt{\ username\ } ,
\texttt{\ prefix\ } , \texttt{\ icon\ } , and \texttt{\ set\ } . The
\texttt{\ username\ } will be used for constructing the link and will be
shown next to the logo. The \texttt{\ prefix\ } is needed to build the
valid link. The \texttt{\ icon\ } is the name of the icon in the
respective set, which is chosen in \texttt{\ set\ } .

\begin{Shaded}
\begin{Highlighting}[]
\FunctionTok{personal}\KeywordTok{:}
\AttributeTok{  }\FunctionTok{name}\KeywordTok{:}\AttributeTok{ }\KeywordTok{[}\StringTok{"Mustermensch, Momo"}\KeywordTok{]}
\AttributeTok{  }\FunctionTok{socials}\KeywordTok{:}
\AttributeTok{    }\FunctionTok{email}\KeywordTok{:}
\AttributeTok{      }\FunctionTok{username}\KeywordTok{:}\AttributeTok{ momo@mustermensch.com}
\AttributeTok{      }\FunctionTok{prefix}\KeywordTok{:}\AttributeTok{ }\StringTok{"mailto:"}
\AttributeTok{      }\FunctionTok{icon}\KeywordTok{:}\AttributeTok{ paper{-}plane}
\AttributeTok{      }\FunctionTok{set}\KeywordTok{:}\AttributeTok{ fa}
\AttributeTok{    }\FunctionTok{homepage}\KeywordTok{:}
\AttributeTok{      }\FunctionTok{username}\KeywordTok{:}\AttributeTok{ momo.github.io}
\AttributeTok{      }\FunctionTok{prefix}\KeywordTok{:}\AttributeTok{ https://}
\AttributeTok{      }\FunctionTok{icon}\KeywordTok{:}\AttributeTok{ globe}
\AttributeTok{      }\FunctionTok{set}\KeywordTok{:}\AttributeTok{ fa}
\AttributeTok{    }\FunctionTok{orcid}\KeywordTok{:}
\AttributeTok{      }\FunctionTok{username}\KeywordTok{:}\AttributeTok{ 0000{-}0000{-}0000{-}0000}
\AttributeTok{      }\FunctionTok{prefix}\KeywordTok{:}\AttributeTok{ https://orcid.org}
\AttributeTok{      }\FunctionTok{icon}\KeywordTok{:}\AttributeTok{ orcid}
\AttributeTok{      }\FunctionTok{set}\KeywordTok{:}\AttributeTok{ ai}
\AttributeTok{    }\FunctionTok{lattes}\KeywordTok{:}
\AttributeTok{      }\FunctionTok{username}\KeywordTok{:}\AttributeTok{ }\StringTok{"1234567891234567"}
\AttributeTok{      }\FunctionTok{prefix}\KeywordTok{:}\AttributeTok{ http://lattes.cnpq.br/}
\AttributeTok{      }\FunctionTok{icon}\KeywordTok{:}\AttributeTok{ lattes}
\AttributeTok{      }\FunctionTok{set}\KeywordTok{:}\AttributeTok{ ai}
\AttributeTok{    ...}
\end{Highlighting}
\end{Shaded}

\subsubsection{Language setting \&
headers}\label{language-setting-headers}

In order to support changing headers, you need to specify the language
and the different content for each header in each language in the
\texttt{\ i18n.yaml\ } in the folder \texttt{\ dbs\ } .

The structure of the yaml is simple:

\begin{Shaded}
\begin{Highlighting}[]
\FunctionTok{lang}\KeywordTok{:}
\AttributeTok{  }\FunctionTok{de}\KeywordTok{:}
\AttributeTok{    }\FunctionTok{subtitle}\KeywordTok{:}\AttributeTok{ Short CV}
\AttributeTok{    }\FunctionTok{education}\KeywordTok{:}\AttributeTok{ Hochschulbildung}
\AttributeTok{    }\FunctionTok{work}\KeywordTok{:}\AttributeTok{ Akademische Berufserfahrung (Auswahl)}
\AttributeTok{    }\FunctionTok{grants}\KeywordTok{:}\AttributeTok{ Fördermittel, Stipendien \& Preise}
\AttributeTok{    ...}
\AttributeTok{  }\FunctionTok{en}\KeywordTok{:}
\AttributeTok{    }\FunctionTok{subtitle}\KeywordTok{:}\AttributeTok{ Short CV}
\AttributeTok{    }\FunctionTok{education}\KeywordTok{:}\AttributeTok{ Higher education}
\AttributeTok{    }\FunctionTok{work}\KeywordTok{:}\AttributeTok{ Academic work experience (selection)}
\AttributeTok{    }\FunctionTok{grants}\KeywordTok{:}\AttributeTok{ Scholarships \& awards}
\AttributeTok{    ...}
\AttributeTok{  }\FunctionTok{pt}\KeywordTok{:}
\AttributeTok{    }\FunctionTok{subtitle}\KeywordTok{:}\AttributeTok{ Currículo}
\AttributeTok{    }\FunctionTok{education}\KeywordTok{:}\AttributeTok{ Formação acadêmica}
\AttributeTok{    }\FunctionTok{work}\KeywordTok{:}\AttributeTok{ Atuação profissional (seleção)}
\AttributeTok{    }\FunctionTok{grants}\KeywordTok{:}\AttributeTok{ Bolsas de estudo e prémios}
\AttributeTok{    ...}
\end{Highlighting}
\end{Shaded}

For each language, you want to use later, you have to define all the
entries. Reminder, don’t change the entry names, since the functions
won’t find it under different names without changing the functions.

First you have to set up a variable that inherits the ISO-language code,
save the database into an object (here \texttt{\ multilingual\ } ) and
then give the object \texttt{\ multilingual\ } and \texttt{\ language\ }
to the function \texttt{\ create-headers\ } :

\begin{Shaded}
\begin{Highlighting}[]
\NormalTok{// set the language of the document}
\NormalTok{\#let language = "pt"      }

\NormalTok{// loading multilingual database}
\NormalTok{\#let multilingual = yaml("dbs/i18n.yaml")}

\NormalTok{// defining variables}
\NormalTok{\#let headerLabs = create{-}headers(multilingual, lang: language)}
\end{Highlighting}
\end{Shaded}

You create an object \texttt{\ headerLabs\ } that uses the function
\texttt{\ create-headers()\ } which will define the headers as you
provided in the \texttt{\ yaml\ } . Then by switching the language
object, all headers (if used accordingly to the naming in the
\texttt{\ yaml\ } ) will change directly.

Throughout the document you then reference the created
\texttt{\ headerLabs\ } object. If you change language, and values are
provided, these automatically change.

\begin{Shaded}
\begin{Highlighting}[]
\NormalTok{= \#headerLabs.at("work")}

\NormalTok{...}

\NormalTok{= \#headerLabs.at("education")}

\NormalTok{...}

\NormalTok{= \#headerLabs.at("grants")}
\end{Highlighting}
\end{Shaded}

\subsubsection{Automated functions}\label{automated-functions}

All of the following functions share common arguments: \texttt{\ what\ }
, \texttt{\ multilingual\ } , and \texttt{\ lang\ } . In
\texttt{\ what\ } , you always declare the database you want to use with
the function.

For example, to get work entries, you choose \texttt{\ work\ } , which
you defined beforehand as input from \texttt{\ work.yaml\ } . In the
\texttt{\ multilingual\ } argument, you just pass the
\texttt{\ multilingual\ } object. In \texttt{\ lang\ } you pass your
\texttt{\ language\ } object.

\begin{Shaded}
\begin{Highlighting}[]
\NormalTok{\#let multilingual = yaml("dbs/multilingual.yaml")}
\NormalTok{\#let work = yaml("dbs/work.yaml")}
\NormalTok{\#let language = "pt"}

\NormalTok{// Function call with objects}
\NormalTok{\#cv{-}auto{-}stc(work, multilingual, lang: language)}
\end{Highlighting}
\end{Shaded}

\subsubsection{Sorting publications and referencing your own name or
correpsonding}\label{sorting-publications-and-referencing-your-own-name-or-correpsonding}

Since \texttt{\ typst\ } so far does not support multiple bibliographies
or subsetting these, this function let you choose specific entries via
the \texttt{\ entries\ } argument or group of entries by the
\texttt{\ tag\ } argument. Furthermore, you can indicate a string in
\texttt{\ me\ } that can be highlighted in every output entry (i.e.,
your formatted name). So far, this function leads to another function
that create APA-style format, if you want to use any other citation
style, you need to download the template on
\href{https://github.com/bpkleer/modern-acad-cv}{github} , introduce
your own styling and then add it in the \texttt{\ cv-refs()\ } function.

\begin{Shaded}
\begin{Highlighting}[]
\NormalTok{\#let multilingual = yaml("dbs/multilingual.yaml")}
\NormalTok{\#let refs = yaml("dbs/refs/yaml")}

\NormalTok{// function call of group of peer{-}reviewed with tag \textasciigrave{}peer\textasciigrave{}}
\NormalTok{\#cv{-}refs(refs, multilingual, tag: "peer", me: [Mustermensch, M.], lang: language)}
\end{Highlighting}
\end{Shaded}

You see in the example pictures that I used this function to built five
different subheaders, i.e. for peer reviewed articles (
\texttt{\ tag:\ "peer"\ } ) and chapters in edited books (
\texttt{\ tag:\ "edited"\ } ). You can define the tags how you want,
however, they need to put them into
\texttt{\ tag:\ \textless{}str\textgreater{}\ } .

Sometimes, it is not only necessary to highlight your own name, you
might also want to indicate yourself as corresponding author. This can
be done through the \texttt{\ refs.yaml\ } which adhere to
\href{https://github.com/typst/hayagriva}{Hayagriva} . By adding an
argument \texttt{\ corresponding\ } in the yaml and setting the value to
\texttt{\ true\ } , a small \texttt{\ C\ } will appear next to your
name.

\begin{Shaded}
\begin{Highlighting}[]
\FunctionTok{Mustermensch2023}\KeywordTok{:}
\AttributeTok{  }\FunctionTok{type}\KeywordTok{:}\AttributeTok{ }\StringTok{"article"}
\AttributeTok{  }\FunctionTok{date}\KeywordTok{:}\AttributeTok{ }\DecValTok{2023}
\AttributeTok{  }\FunctionTok{page{-}range}\KeywordTok{:}\AttributeTok{ 55{-}78}
\AttributeTok{  }\FunctionTok{title}\KeywordTok{:}\AttributeTok{ }\StringTok{"Populism and Social Media: A Comparative Study of Political Mobilization"}
\AttributeTok{  }\FunctionTok{tags}\KeywordTok{:}\AttributeTok{ }\StringTok{"peer"}
\AttributeTok{  }\FunctionTok{author}\KeywordTok{:}\AttributeTok{ }\KeywordTok{[}\AttributeTok{ }\StringTok{"Mustermensch, Momo"}\KeywordTok{,}\AttributeTok{ }\StringTok{"Rivera, Casey"}\AttributeTok{ }\KeywordTok{]}
\AttributeTok{  }\FunctionTok{corresponding}\KeywordTok{:}\AttributeTok{ }\CharTok{true}
\AttributeTok{  }\FunctionTok{parent}\KeywordTok{:}
\AttributeTok{    }\FunctionTok{title}\KeywordTok{:}\AttributeTok{ }\StringTok{"Journal of Political Communication"}
\AttributeTok{    }\FunctionTok{volume}\KeywordTok{:}\AttributeTok{ }\DecValTok{41}
\AttributeTok{    }\FunctionTok{issue}\KeywordTok{:}\AttributeTok{ }\DecValTok{3}
\AttributeTok{  }\FunctionTok{serial{-}number}\KeywordTok{:}
\AttributeTok{    }\FunctionTok{doi}\KeywordTok{:}\AttributeTok{ }\StringTok{"10.1016/j.jpolcom.2023.102865"}
\end{Highlighting}
\end{Shaded}

For applications abroad, it might be worth to translate at least title
of the publications so that other persons easily can see what the paper
is about. In every \texttt{\ title\ } argument, you can therefore
provide a dictionary with the language codes and the titles. Keep the
original title in \texttt{\ main\ } and the translations with the
corresponding language shortcut (i.e., \texttt{\ "en"\ } or
\texttt{\ "pt"\ } ). The function prints the main and translated title,
depending on the provided translation in the \texttt{\ refs.yaml\ } . Be
aware, here you find not \texttt{\ de\ } in the dictionary, instead you
find \texttt{\ main\ } . The original title needs to be wrapped in
\texttt{\ main\ } .

\begin{Shaded}
\begin{Highlighting}[]
\FunctionTok{Mustermensch2023}\KeywordTok{:}
\AttributeTok{  }\FunctionTok{type}\KeywordTok{:}\AttributeTok{ }\StringTok{"article"}
\AttributeTok{  }\FunctionTok{date}\KeywordTok{:}\AttributeTok{ }\DecValTok{2023}
\AttributeTok{  }\FunctionTok{page{-}range}\KeywordTok{:}\AttributeTok{ 55{-}78}
\AttributeTok{  }\FunctionTok{title}\KeywordTok{:}\AttributeTok{ }
\AttributeTok{    }\FunctionTok{main}\KeywordTok{:}\AttributeTok{ }\StringTok{"Populismus und soziale Medien: Eine vergleichende Studie zur politischen Mobilisierung"}
\AttributeTok{    }\FunctionTok{en}\KeywordTok{:}\AttributeTok{ }\StringTok{"Populism and Social Media: A Comparative Study of Political Mobilization"}
\AttributeTok{    }\FunctionTok{pt}\KeywordTok{:}\AttributeTok{ }\StringTok{"Populismo e redes sociais: Um Estudo Comparativo de Mobilização Política"}
\AttributeTok{  }\FunctionTok{tags}\KeywordTok{:}\AttributeTok{ }\StringTok{"peer"}
\AttributeTok{  }\FunctionTok{author}\KeywordTok{:}\AttributeTok{ }\KeywordTok{[}\AttributeTok{ }\StringTok{"Mustermensch, Momo"}\KeywordTok{,}\AttributeTok{ }\StringTok{"Rivera, Casey"}\AttributeTok{ }\KeywordTok{]}
\AttributeTok{  }\FunctionTok{corresponding}\KeywordTok{:}\AttributeTok{ }\CharTok{true}
\AttributeTok{  }\FunctionTok{parent}\KeywordTok{:}
\AttributeTok{    }\FunctionTok{title}\KeywordTok{:}\AttributeTok{ }\StringTok{"Journal of Political Communication"}
\AttributeTok{    }\FunctionTok{volume}\KeywordTok{:}\AttributeTok{ }\DecValTok{41}
\AttributeTok{    }\FunctionTok{issue}\KeywordTok{:}\AttributeTok{ }\DecValTok{3}
\AttributeTok{  }\FunctionTok{serial{-}number}\KeywordTok{:}
\AttributeTok{    }\FunctionTok{doi}\KeywordTok{:}\AttributeTok{ }\StringTok{"10.1016/j.jpolcom.2023.102865"}
\end{Highlighting}
\end{Shaded}

\subsubsection{cv-auto-skills()}\label{cv-auto-skills}

Instead of just enumerating your skills or your knowledge of specific
software, you can build a skill-matrix with this function. In this
skill-matrix, you can have sections, i.e. \emph{Computer Languages} ,
\emph{Programs} and \emph{Languages} . These sections are the highest
level in the corresponding \texttt{\ skills.yaml\ } :

\begin{Shaded}
\begin{Highlighting}[]
\FunctionTok{computer}\KeywordTok{:}
\AttributeTok{  ...}
\FunctionTok{programs}\KeywordTok{:}
\AttributeTok{  ...}
\FunctionTok{languages}\KeywordTok{:}
\AttributeTok{  ...}
\end{Highlighting}
\end{Shaded}

You can then define in each categories specific skills, i.e. German and
Portuguese in \texttt{\ languages\ } :

\begin{Shaded}
\begin{Highlighting}[]
\FunctionTok{computer}\KeywordTok{:}
\AttributeTok{  ...}
\FunctionTok{programs}\KeywordTok{:}
\AttributeTok{  ...}
\FunctionTok{languages}\KeywordTok{:}
\AttributeTok{  }\FunctionTok{german}\KeywordTok{:}
\AttributeTok{    ...}
\AttributeTok{  }\FunctionTok{portugues}\KeywordTok{:}
\AttributeTok{   ...}
\end{Highlighting}
\end{Shaded}

For each entry, you have to define \texttt{\ name\ } ,
\texttt{\ level\ } and \texttt{\ description\ } .

\begin{Shaded}
\begin{Highlighting}[]
\FunctionTok{languages}\KeywordTok{:}\AttributeTok{ }
\AttributeTok{  ...}
\AttributeTok{  }\FunctionTok{pt}\KeywordTok{:}\AttributeTok{ }
\AttributeTok{    }\FunctionTok{name}\KeywordTok{:}
\AttributeTok{      }\FunctionTok{de}\KeywordTok{:}\AttributeTok{ Portugiesisch}
\AttributeTok{      }\FunctionTok{en}\KeywordTok{:}\AttributeTok{ Portuguese}
\AttributeTok{      }\FunctionTok{pt}\KeywordTok{:}\AttributeTok{ Português}
\AttributeTok{    }\FunctionTok{level}\KeywordTok{:}\AttributeTok{ }\DecValTok{3}
\AttributeTok{    }\FunctionTok{description}\KeywordTok{:}
\AttributeTok{      }\FunctionTok{de}\KeywordTok{:}\AttributeTok{ fortgeschritten}
\AttributeTok{      }\FunctionTok{en}\KeywordTok{:}\AttributeTok{ advanced}
\AttributeTok{      }\FunctionTok{pt}\KeywordTok{:}\AttributeTok{ avançado}
\end{Highlighting}
\end{Shaded}

As you can see, you can again define language-dependent names in
\texttt{\ name\ } and descriptions in \texttt{\ description\ } .
\texttt{\ level\ } is a numeric value and indicates how many of the four
boxes are filled to indicate you level of proficiency. If you don’t
have the need for a CV of different languages, you can directly define
\texttt{\ name\ } or \texttt{\ description\ } .

You have to call the function with three objects \texttt{\ skills\ } ,
\texttt{\ multilingual\ } , and \texttt{\ metadata\ } and the
corresponding \texttt{\ language\ } of the document:

\begin{Shaded}
\begin{Highlighting}[]
\NormalTok{\#let skills = yaml("dbs/skills.yaml")}
\NormalTok{\#let multilingual = yaml("dbs/multilingual.yaml")}
\NormalTok{\#let metadata = yaml("dbs/metadata.yaml")}
\NormalTok{\#let language = "pt"}

\NormalTok{\#cv{-}auto{-}skills(skills, multilingual, metadata, lang: language)}
\end{Highlighting}
\end{Shaded}

\subsubsection{Print your info without any
formatting}\label{print-your-info-without-any-formatting}

The function \texttt{\ cv-auto\ } is the base function for printing the
provided infos in the specified \texttt{\ yaml\ } file with no further
formatting. The functions \texttt{\ cv-auto-stc\ } and
\texttt{\ cv-auto-stp\ } do only differ in the point that
\texttt{\ cv-auto-stc\ } both give the title in bold,
\texttt{\ cv-auto-stp\ } puts the subtitle in parentheses and
\texttt{\ cv-auto-stc\ } puts the subtitle after a comma.

The structure of the corresponding \texttt{\ yaml\ } files is simple: in
each entry you can have the following entries: \texttt{\ title\ } ,
\texttt{\ subtitle\ } , \texttt{\ location\ } , \texttt{\ description\ }
and \texttt{\ left\ } . \texttt{\ title\ } is mandatory,
\texttt{\ subtitle\ } , \texttt{\ location\ } , and
\texttt{\ description\ } are voluntary. In all functions you need to
specify \texttt{\ left\ } , which indicates period of time, or year. For
\texttt{\ title\ } , \texttt{\ subtitle\ } , \texttt{\ location\ } , and
\texttt{\ description\ } , you can provide a dictionary for different
languages (see below).

\begin{Shaded}
\begin{Highlighting}[]
\FunctionTok{master}\KeywordTok{:}
\AttributeTok{  }\FunctionTok{title}\KeywordTok{:}
\AttributeTok{    }\FunctionTok{de}\KeywordTok{:}\AttributeTok{ Master of Arts}
\AttributeTok{    }\FunctionTok{en}\KeywordTok{:}\AttributeTok{ Master of Arts}
\AttributeTok{    }\FunctionTok{pt}\KeywordTok{:}\AttributeTok{ Pós{-}Graduação}
\AttributeTok{  }\FunctionTok{subtitle}\KeywordTok{:}
\AttributeTok{    }\FunctionTok{de}\KeywordTok{:}\AttributeTok{ Sozialwissenschaften}
\AttributeTok{    }\FunctionTok{en}\KeywordTok{:}\AttributeTok{ Social Sciences}
\AttributeTok{    }\FunctionTok{pt}\KeywordTok{:}\AttributeTok{ Ciências Sociais}
\AttributeTok{  }\FunctionTok{location}\KeywordTok{:}
\AttributeTok{    }\FunctionTok{de}\KeywordTok{:}\AttributeTok{ Exzellenz{-}Universität}
\AttributeTok{    }\FunctionTok{en}\KeywordTok{:}\AttributeTok{ University of Excellence}
\AttributeTok{    }\FunctionTok{pt}\KeywordTok{:}\AttributeTok{ Universidade de Excelência}
\AttributeTok{  }\FunctionTok{description}\KeywordTok{:}
\AttributeTok{    }\FunctionTok{de}\KeywordTok{:}\AttributeTok{ mit Auszeichnung}
\AttributeTok{    }\FunctionTok{en}\KeywordTok{:}\AttributeTok{ with distinction}
\AttributeTok{    }\FunctionTok{pt}\KeywordTok{:}\AttributeTok{ com distinção}
\AttributeTok{  }\FunctionTok{left}\KeywordTok{:}\AttributeTok{ }\StringTok{"2014"}
\end{Highlighting}
\end{Shaded}

In your main document, you then easily call the function and transfer
the standard arguments \texttt{\ what\ } , \texttt{\ metadata\ } , and
\texttt{\ lang\ } .

\begin{Shaded}
\begin{Highlighting}[]
\NormalTok{// section of education }
\NormalTok{\#let education = yaml("dbs/education.yaml")}
\NormalTok{\#let multilingual = yaml("dbs/multilingual.yaml")}
\NormalTok{\#let language = "pt"}

\NormalTok{\#cv{-}auto{-}stp(education, multilingual, lang: language) }

\NormalTok{// section of work positions}
\NormalTok{\#let work = yaml("dbs/work.yaml")}
\NormalTok{\#let multilingual = yaml("dbs/multilingual.yaml")}
\NormalTok{\#let language = "pt"}

\NormalTok{\#cv{-}auto{-}stc(work, multilingual, lang: language)}

\NormalTok{// section of given talks}
\NormalTok{\#let talks = yaml("dbs/talks.yaml")}
\NormalTok{\#let multilingual = yaml("dbs/multilingual.yaml")}
\NormalTok{\#let language = "pt"}

\NormalTok{\#cv{-}auto(talks, multilingual, lang: language)}
\end{Highlighting}
\end{Shaded}

\subsubsection{Creating a list instead of single
entrie}\label{creating-a-list-instead-of-single-entrie}

Sometimes, instead of giving every entry, you want to group by year.
Another example for this case could be that you want to summarize your
memberships or reviewer duties.

The function \texttt{\ cv-auto-list\ } uses just the standard input:

\begin{Shaded}
\begin{Highlighting}[]
\NormalTok{\#let conferences = yaml("dbs/conferences.yaml")}
\NormalTok{\#let multilingual = yaml("dbs/multilingual.yaml")}
\NormalTok{\#let language = "pt"}

\NormalTok{\#cv{-}auto{-}list(conferences, multilingual, lang: language)}
\end{Highlighting}
\end{Shaded}

The corresponding \texttt{\ yaml\ } file is differently organized: The
entry point in the file is the corresponding year. In every year, you
organize your entries (i.e. conference participations). In each entry in
a year, you have the \texttt{\ name\ } and \texttt{\ action\ } entry.
You can provide a dictionary for the \texttt{\ name\ } . For
\texttt{\ action\ } , I used \texttt{\ P\ } and \texttt{\ C\ } , for
\emph{paper/presentation} and \emph{chair} . You can then manually
define this upfront the function call for the reader, or you use the
\texttt{\ i18n.yaml\ } , indicate the explanations for each language in
\texttt{\ exp-confs\ } and then it automatically changes with the
specific language code.

\begin{Shaded}
\begin{Highlighting}[]
\FunctionTok{"2024"}\KeywordTok{:}
\AttributeTok{  }\FunctionTok{conference2}\KeywordTok{:}
\AttributeTok{    }\FunctionTok{name}\KeywordTok{:}\AttributeTok{ European Conference on Gender and Politics}
\AttributeTok{    }\FunctionTok{action}\KeywordTok{:}\AttributeTok{ P}
\AttributeTok{  }\FunctionTok{conference1}\KeywordTok{:}
\AttributeTok{    }\FunctionTok{name}\KeywordTok{:}\AttributeTok{ ECPR General Conference}
\AttributeTok{    }\FunctionTok{action}\KeywordTok{:}\AttributeTok{ P, C}
\end{Highlighting}
\end{Shaded}

The action will be added after each conference name in superscripts.

\subsubsection{Creating a table}\label{creating-a-table}

This case is mostly used for listing your prior teaching experience. The
corresponding \texttt{\ teaching.yaml\ } for this description, is
organized as followed:

\begin{Shaded}
\begin{Highlighting}[]
\FunctionTok{"2024"}\KeywordTok{:}
\AttributeTok{  }\FunctionTok{course1}\KeywordTok{:}
\AttributeTok{    }\FunctionTok{summer}\KeywordTok{:}\AttributeTok{ T}
\AttributeTok{    }\FunctionTok{name}\KeywordTok{:}
\AttributeTok{      }\FunctionTok{de}\KeywordTok{:}\AttributeTok{ }\StringTok{"Statistik+: Einstieg in R leicht gemacht"}
\AttributeTok{      }\FunctionTok{en}\KeywordTok{:}\AttributeTok{ }\StringTok{"Statistics+: Starting with R (de)"}
\AttributeTok{      }\FunctionTok{pt}\KeywordTok{:}\AttributeTok{ }\StringTok{"Estatística+: Começando com R (de)"}
\AttributeTok{    }\FunctionTok{study}\KeywordTok{:}
\AttributeTok{      }\FunctionTok{de}\KeywordTok{:}\AttributeTok{ Bachelor}
\AttributeTok{      }\FunctionTok{en}\KeywordTok{:}\AttributeTok{ Bachelor}
\AttributeTok{      }\FunctionTok{pt}\KeywordTok{:}\AttributeTok{ Graduação}
\AttributeTok{  ...}
\end{Highlighting}
\end{Shaded}

First you indicate the year \texttt{\ "2024"\ } and then you organize
all courses you gave within that year (i.e. here \texttt{\ course1\ } ).
Mandatory are \texttt{\ name\ } and \texttt{\ study\ } . For both you
can indicate a single value or a dictionary corresponding to your chosen
languages. You can provide \texttt{\ summer\ } if you want to indicate
differences for terms. This is \texttt{\ boolean\ } , the specific word
is then given in the \texttt{\ i18n.yaml\ } under
\texttt{\ table-winter\ } resp. \texttt{\ table-summer\ } .

The function then uses again just the standard arguments and plots a
table with the indicated year, name, and study area.

\begin{Shaded}
\begin{Highlighting}[]
\NormalTok{\#let teaching = yaml("dbs/teaching.yaml")}
\NormalTok{\#let multilingual = yaml("dbs/multilingual.yaml")}
\NormalTok{\#let language = "pt"}

\NormalTok{\#cv{-}table{-}teaching(teaching, multilingual, lang: language)}
\end{Highlighting}
\end{Shaded}

\subsubsection{cv-auto-cats()}\label{cv-auto-cats}

In case you want to directly print entries from categories that belong
to one \texttt{\ yaml\ } -file, you can use \texttt{\ cv-auto-cats\ } .
This will print the header for each subcategory and then the belonging
entries.

An example is given in \texttt{\ training.yaml\ } . In this file,
further training is given by categories (i.e., methods and didactics).
Within the categories you have here courses and then \texttt{\ title\ }
, \texttt{\ location\ } , and \texttt{\ left\ } . \texttt{\ location\ }
and \texttt{\ title\ } can be dictionaries if you want to translate
between different languages.

\begin{Shaded}
\begin{Highlighting}[]
\FunctionTok{methods}\KeywordTok{:}
\AttributeTok{  }\FunctionTok{course2}\KeywordTok{:}
\AttributeTok{    }\FunctionTok{title}\KeywordTok{:}\AttributeTok{ Bayesian modelling in the Social Sciences}
\AttributeTok{    }\FunctionTok{location}\KeywordTok{:}\AttributeTok{ An expensive Spring Seminar}
\AttributeTok{    }\FunctionTok{left}\KeywordTok{:}\AttributeTok{ }\StringTok{"2024"}
\AttributeTok{  ...}
\FunctionTok{didactics}\KeywordTok{:}
\AttributeTok{  }\FunctionTok{course2}\KeywordTok{:}
\AttributeTok{    }\FunctionTok{title}\KeywordTok{:}
\AttributeTok{      }\FunctionTok{de}\KeywordTok{:}\AttributeTok{ Konfliktkompetenz I + II}
\AttributeTok{      }\FunctionTok{en}\KeywordTok{:}\AttributeTok{ Conflict competence I + II}
\AttributeTok{      }\FunctionTok{pt}\KeywordTok{:}\AttributeTok{ Competência de conflitos I + II}
\AttributeTok{    }\FunctionTok{location}\KeywordTok{:}\AttributeTok{ }
\AttributeTok{      }\FunctionTok{de}\KeywordTok{:}\AttributeTok{ Universitätsallianz}
\AttributeTok{      }\FunctionTok{en}\KeywordTok{:}\AttributeTok{ University Alliance}
\AttributeTok{      }\FunctionTok{pt}\KeywordTok{:}\AttributeTok{ Aliança Universitária}
\AttributeTok{    }\FunctionTok{left}\KeywordTok{:}\AttributeTok{ }\StringTok{"2019"}
\end{Highlighting}
\end{Shaded}

Call the function as usal:

\begin{Shaded}
\begin{Highlighting}[]
\NormalTok{\#let training = yaml("dbs/training.yaml")}
\NormalTok{\#let multilingual = yaml("dbs/multilingual.yaml")}
\NormalTok{\#let language = "pt"}
\NormalTok{\#let headerLabs = create{-}headers(multilingual, lang: language)}

\NormalTok{\#cv{-}auto{-}cats(training, multilingual, headerLabs, lang: language)}
\end{Highlighting}
\end{Shaded}

\subsubsection{Special cases: long
names}\label{special-cases-long-names}

If you have a long name that crosses the social media side, just set the
argument \texttt{\ split\ } to \texttt{\ true\ } within
\texttt{\ metadata.yaml\ } :

\begin{Shaded}
\begin{Highlighting}[]
\CommentTok{...}
\CommentTok{  personal:}
\CommentTok{    name: ["Mustermensch, Momo"]}
\CommentTok{    split: true}
\CommentTok{  ...    }
\end{Highlighting}
\end{Shaded}

\subsection{Examples}\label{examples}

\pandocbounded{\includegraphics[keepaspectratio]{https://github.com/typst/packages/raw/main/packages/preview/modern-acad-cv/0.1.1/assets/example1.png}}
\pandocbounded{\includegraphics[keepaspectratio]{https://github.com/typst/packages/raw/main/packages/preview/modern-acad-cv/0.1.1/assets/example2.png}}
\pandocbounded{\includegraphics[keepaspectratio]{https://github.com/typst/packages/raw/main/packages/preview/modern-acad-cv/0.1.1/assets/example3.png}}

\href{/app?template=modern-acad-cv&version=0.1.1}{Create project in app}

\subsubsection{How to use}\label{how-to-use}

Click the button above to create a new project using this template in
the Typst app.

You can also use the Typst CLI to start a new project on your computer
using this command:

\begin{verbatim}
typst init @preview/modern-acad-cv:0.1.1
\end{verbatim}

\includesvg[width=0.16667in,height=0.16667in]{/assets/icons/16-copy.svg}

\subsubsection{About}\label{about}

\begin{description}
\tightlist
\item[Author :]
\href{mailto:philipp.kleer@posteo.com}{bpkleer (Philipp Kleer)}
\item[License:]
MIT
\item[Current version:]
0.1.1
\item[Last updated:]
November 28, 2024
\item[First released:]
August 23, 2024
\item[Minimum Typst version:]
0.12.0
\item[Archive size:]
25.1 kB
\href{https://packages.typst.org/preview/modern-acad-cv-0.1.1.tar.gz}{\pandocbounded{\includesvg[keepaspectratio]{/assets/icons/16-download.svg}}}
\item[Repository:]
\href{https://github.com/bpkleer/typst-modern-acad-cv}{GitHub}
\item[Categor y :]
\begin{itemize}
\tightlist
\item[]
\item
  \pandocbounded{\includesvg[keepaspectratio]{/assets/icons/16-user.svg}}
  \href{https://typst.app/universe/search/?category=cv}{CV}
\end{itemize}
\end{description}

\subsubsection{Where to report issues?}\label{where-to-report-issues}

This template is a project of bpkleer (Philipp Kleer) . Report issues on
\href{https://github.com/bpkleer/typst-modern-acad-cv}{their repository}
. You can also try to ask for help with this template on the
\href{https://forum.typst.app}{Forum} .

Please report this template to the Typst team using the
\href{https://typst.app/contact}{contact form} if you believe it is a
safety hazard or infringes upon your rights.

\phantomsection\label{versions}
\subsubsection{Version history}\label{version-history}

\begin{longtable}[]{@{}ll@{}}
\toprule\noalign{}
Version & Release Date \\
\midrule\noalign{}
\endhead
\bottomrule\noalign{}
\endlastfoot
0.1.1 & November 28, 2024 \\
\href{https://typst.app/universe/package/modern-acad-cv/0.1.0/}{0.1.0} &
August 23, 2024 \\
\end{longtable}

Typst GmbH did not create this template and cannot guarantee correct
functionality of this template or compatibility with any version of the
Typst compiler or app.


\section{Package List LaTeX/syntree.tex}
\title{typst.app/universe/package/syntree}

\phantomsection\label{banner}
\section{syntree}\label{syntree}

{ 0.2.0 }

Linguistics syntax/parse tree rendering

\phantomsection\label{readme}
\textbf{syntree} is a typst package for rendering syntax trees / parse
trees (the kind linguists use).

The name and syntax are inspired by Miles Shang’s
\href{https://github.com/mshang/syntree}{syntree} . Here’s an example
to get started:

\begin{longtable}[]{@{}
  >{\raggedright\arraybackslash}p{(\linewidth - 2\tabcolsep) * \real{0.5000}}
  >{\raggedright\arraybackslash}p{(\linewidth - 2\tabcolsep) * \real{0.5000}}@{}}
\toprule\noalign{}
\endhead
\bottomrule\noalign{}
\endlastfoot
\begin{minipage}[t]{\linewidth}\raggedright
\begin{Shaded}
\begin{Highlighting}[]
\NormalTok{\#import "@preview/syntree:0.2.0": syntree}

\NormalTok{\#syntree(}
\NormalTok{  nonterminal: (font: "Linux Biolinum"),}
\NormalTok{  terminal: (fill: blue),}
\NormalTok{  child{-}spacing: 3em, // default 1em}
\NormalTok{  layer{-}spacing: 2em, // default 2.3em}
\NormalTok{  "[S [NP This] [VP [V is] [\^{}NP a wug]]]"}
\NormalTok{)}
\end{Highlighting}
\end{Shaded}
\end{minipage} &
\pandocbounded{\includegraphics[keepaspectratio]{https://github.com/lynn/typst-syntree/assets/16232127/d0c680b2-4fd0-420f-b350-9e9c96ac37f3}} \\
\end{longtable}

There’s limited support for formulas inside nodes; try
\texttt{\ \#syntree("{[}DP\$zws\_i\$\ this{]}")\ } or
\texttt{\ \#syntree("{[}C\ \$diameter\${]}")\ } .

For more flexible tree-drawing, use \texttt{\ tree\ } :

\begin{longtable}[]{@{}
  >{\raggedright\arraybackslash}p{(\linewidth - 2\tabcolsep) * \real{0.5000}}
  >{\raggedright\arraybackslash}p{(\linewidth - 2\tabcolsep) * \real{0.5000}}@{}}
\toprule\noalign{}
\endhead
\bottomrule\noalign{}
\endlastfoot
\begin{minipage}[t]{\linewidth}\raggedright
\begin{Shaded}
\begin{Highlighting}[]
\NormalTok{\#import "@preview/syntree:0.2.0": tree}

\NormalTok{\#let bx(col) = box(fill: col, width: 1em, height: 1em)}
\NormalTok{\#tree("colors",}
\NormalTok{  tree("warm", bx(red), bx(orange)),}
\NormalTok{  tree("cool", bx(blue), bx(teal)))}
\end{Highlighting}
\end{Shaded}
\end{minipage} &
\pandocbounded{\includegraphics[keepaspectratio]{https://github.com/lynn/typst-syntree/assets/16232127/bc979614-e2ce-4616-97d1-1584788fc71f}} \\
\end{longtable}

\subsubsection{How to add}\label{how-to-add}

Copy this into your project and use the import as \texttt{\ syntree\ }

\begin{verbatim}
#import "@preview/syntree:0.2.0"
\end{verbatim}

\includesvg[width=0.16667in,height=0.16667in]{/assets/icons/16-copy.svg}

Check the docs for
\href{https://typst.app/docs/reference/scripting/\#packages}{more
information on how to import packages} .

\subsubsection{About}\label{about}

\begin{description}
\tightlist
\item[Author :]
\href{https://github.com/lynn}{Lynn}
\item[License:]
MIT
\item[Current version:]
0.2.0
\item[Last updated:]
January 12, 2024
\item[First released:]
July 8, 2023
\item[Archive size:]
2.50 kB
\href{https://packages.typst.org/preview/syntree-0.2.0.tar.gz}{\pandocbounded{\includesvg[keepaspectratio]{/assets/icons/16-download.svg}}}
\item[Repository:]
\href{https://github.com/lynn/typst-syntree}{GitHub}
\end{description}

\subsubsection{Where to report issues?}\label{where-to-report-issues}

This package is a project of Lynn . Report issues on
\href{https://github.com/lynn/typst-syntree}{their repository} . You can
also try to ask for help with this package on the
\href{https://forum.typst.app}{Forum} .

Please report this package to the Typst team using the
\href{https://typst.app/contact}{contact form} if you believe it is a
safety hazard or infringes upon your rights.

\phantomsection\label{versions}
\subsubsection{Version history}\label{version-history}

\begin{longtable}[]{@{}ll@{}}
\toprule\noalign{}
Version & Release Date \\
\midrule\noalign{}
\endhead
\bottomrule\noalign{}
\endlastfoot
0.2.0 & January 12, 2024 \\
\href{https://typst.app/universe/package/syntree/0.1.0/}{0.1.0} & July
8, 2023 \\
\end{longtable}

Typst GmbH did not create this package and cannot guarantee correct
functionality of this package or compatibility with any version of the
Typst compiler or app.


\section{Package List LaTeX/minimalistic-latex-cv.tex}
\title{typst.app/universe/package/minimalistic-latex-cv}

\phantomsection\label{banner}
\phantomsection\label{template-thumbnail}
\pandocbounded{\includegraphics[keepaspectratio]{https://packages.typst.org/preview/thumbnails/minimalistic-latex-cv-0.1.1-small.webp}}

\section{minimalistic-latex-cv}\label{minimalistic-latex-cv}

{ 0.1.1 }

A minimalistic LaTeX-style CV template for professionals.

\href{/app?template=minimalistic-latex-cv&version=0.1.1}{Create project
in app}

\phantomsection\label{readme}
This is a Typst template for a minimalistic LaTeX-style CV. It provides
a simple structure for a CV with a header, a section for professional
experience, a section for education, and a section for skills and
languages.

\subsection{Usage}\label{usage}

You can use this template in the Typst web app by clicking “Start from
template� on the dashboard and searching for
\texttt{\ minimalistic-latex-cv\ } .

Alternatively, you can use the CLI to kick this project off using the
command

\begin{verbatim}
typst init @preview/minimalistic-latex-cv
\end{verbatim}

Typst will create a new directory with all the files needed to get you
started.

\subsection{Configuration}\label{configuration}

This template exports the \texttt{\ cv\ } function with the following
named arguments:

\begin{itemize}
\tightlist
\item
  \texttt{\ name\ } : The name of the person.
\item
  \texttt{\ metadata\ } : A dictionary of metadata of the person to be
  displayed in the header.
\item
  \texttt{\ photo\ } : The path to the photo of the person.
\item
  \texttt{\ lang\ } : The language of the document.
\end{itemize}

The function also accepts a single, positional argument for the body of
the paper.

The template will initialize your package with a sample call to the
\texttt{\ cv\ } function in a show rule. If you want to change an
existing project to use this template, you can add a show rule like this
at the top of your file:

\begin{Shaded}
\begin{Highlighting}[]
\NormalTok{\#import "@preview/minimalistic{-}latex{-}cv:0.1.1": cv}

\NormalTok{\#show: cv.with(}
\NormalTok{  name: "Your Name",}
\NormalTok{  metadata: (}
\NormalTok{    email: "your@email.com",}
\NormalTok{    telephone: "+123456789",}
\NormalTok{  ),}
\NormalTok{  photo: image("photo.jpeg"),}
\NormalTok{  lang: "en",}
\NormalTok{)}

\NormalTok{// Your content goes below.}
\end{Highlighting}
\end{Shaded}

\href{/app?template=minimalistic-latex-cv&version=0.1.1}{Create project
in app}

\subsubsection{How to use}\label{how-to-use}

Click the button above to create a new project using this template in
the Typst app.

You can also use the Typst CLI to start a new project on your computer
using this command:

\begin{verbatim}
typst init @preview/minimalistic-latex-cv:0.1.1
\end{verbatim}

\includesvg[width=0.16667in,height=0.16667in]{/assets/icons/16-copy.svg}

\subsubsection{About}\label{about}

\begin{description}
\tightlist
\item[Author :]
\href{https://github.com/itsyoboieltr}{Norbert Elter}
\item[License:]
MIT-0
\item[Current version:]
0.1.1
\item[Last updated:]
May 23, 2024
\item[First released:]
March 19, 2024
\item[Minimum Typst version:]
0.11.1
\item[Archive size:]
7.18 kB
\href{https://packages.typst.org/preview/minimalistic-latex-cv-0.1.1.tar.gz}{\pandocbounded{\includesvg[keepaspectratio]{/assets/icons/16-download.svg}}}
\item[Categor y :]
\begin{itemize}
\tightlist
\item[]
\item
  \pandocbounded{\includesvg[keepaspectratio]{/assets/icons/16-user.svg}}
  \href{https://typst.app/universe/search/?category=cv}{CV}
\end{itemize}
\end{description}

\subsubsection{Where to report issues?}\label{where-to-report-issues}

This template is a project of Norbert Elter . You can also try to ask
for help with this template on the \href{https://forum.typst.app}{Forum}
.

Please report this template to the Typst team using the
\href{https://typst.app/contact}{contact form} if you believe it is a
safety hazard or infringes upon your rights.

\phantomsection\label{versions}
\subsubsection{Version history}\label{version-history}

\begin{longtable}[]{@{}ll@{}}
\toprule\noalign{}
Version & Release Date \\
\midrule\noalign{}
\endhead
\bottomrule\noalign{}
\endlastfoot
0.1.1 & May 23, 2024 \\
\href{https://typst.app/universe/package/minimalistic-latex-cv/0.1.0/}{0.1.0}
& March 19, 2024 \\
\end{longtable}

Typst GmbH did not create this template and cannot guarantee correct
functionality of this template or compatibility with any version of the
Typst compiler or app.


\section{Package List LaTeX/soviet-matrix.tex}
\title{typst.app/universe/package/soviet-matrix}

\phantomsection\label{banner}
\phantomsection\label{template-thumbnail}
\pandocbounded{\includegraphics[keepaspectratio]{https://packages.typst.org/preview/thumbnails/soviet-matrix-0.1.1-small.webp}}

\section{soviet-matrix}\label{soviet-matrix}

{ 0.1.1 }

Tetris game in Typst

\href{/app?template=soviet-matrix&version=0.1.1}{Create project in app}

\phantomsection\label{readme}
This is a classic Tetris game implemented using Typst. The goal is to
manipulate falling blocks to create and clear horizontal lines without
letting the blocks stack up to the top of the playing field.

\pandocbounded{\includegraphics[keepaspectratio]{https://github.com/typst/packages/raw/main/packages/preview/soviet-matrix/0.1.1/demo.gif}}

\subsection{How to Play}\label{how-to-play}

You can play the game in two ways:

\begin{enumerate}
\item
  \textbf{Online:}

  \begin{itemize}
  \tightlist
  \item
    Visit
    \url{https://typst.app/app?template=soviet-matrix&version=0.1.0} .
  \item
    Enter any title of your choice and click \textbf{Create} .
  \end{itemize}
\item
  \textbf{Locally:}

  \begin{itemize}
  \item
    Open your command line interface.
  \item
    Run the following command:

\begin{Shaded}
\begin{Highlighting}[]
\ExtensionTok{typst}\NormalTok{ init @preview/soviet{-}matrix}
\end{Highlighting}
\end{Shaded}
  \item
    Typst will create a new directory.
  \item
    Open \texttt{\ main.typ\ } in the created directory.
  \item
    Use the
    \href{https://marketplace.visualstudio.com/items?itemName=mgt19937.typst-preview}{Typst
    Preview VS Code extension} for live preview and gameplay.
  \end{itemize}
\end{enumerate}

Enjoy the game!

\subsection{Controls}\label{controls}

\begin{itemize}
\tightlist
\item
  Move Left: a
\item
  Move Right: d
\item
  Soft Drop: s
\item
  Hard Drop: f
\item
  Rotate Left: q
\item
  Rotate Right: e
\item
  180-degree Rotate: w
\end{itemize}

\subsection{Changing the Game Seed}\label{changing-the-game-seed}

If you want to play different game scenarios, you can change the game
seed using the following method:

\begin{Shaded}
\begin{Highlighting}[]
\NormalTok{\#import "@preview/soviet{-}matrix:0.1.0": game}
\NormalTok{\#show: game.with(seed: 123) // Change the game seed}
\end{Highlighting}
\end{Shaded}

Replace \texttt{\ 123\ } with any number of your choice.

\subsection{Changing Key Bindings}\label{changing-key-bindings}

Modify the \texttt{\ actions\ } parameter in the \texttt{\ game.with\ }
method to change the key bindings. The default key bindings are as
follows:

\begin{Shaded}
\begin{Highlighting}[]
\NormalTok{\#show: game.with(seed: 0, actions: (}
\NormalTok{  left: ("a", ),}
\NormalTok{  right: ("d", ),}
\NormalTok{  down: ("s", ),}
\NormalTok{  left{-}rotate: ("q", ),}
\NormalTok{  right{-}rotate: ("e", ),}
\NormalTok{  half{-}turn: ("w", ),}
\NormalTok{  fast{-}drop: ("f", ),}
\NormalTok{))}
\end{Highlighting}
\end{Shaded}

\href{/app?template=soviet-matrix&version=0.1.1}{Create project in app}

\subsubsection{How to use}\label{how-to-use}

Click the button above to create a new project using this template in
the Typst app.

You can also use the Typst CLI to start a new project on your computer
using this command:

\begin{verbatim}
typst init @preview/soviet-matrix:0.1.1
\end{verbatim}

\includesvg[width=0.16667in,height=0.16667in]{/assets/icons/16-copy.svg}

\subsubsection{About}\label{about}

\begin{description}
\tightlist
\item[Author :]
\href{https://github.com/YouXam}{YouXam}
\item[License:]
MIT
\item[Current version:]
0.1.1
\item[Last updated:]
July 3, 2024
\item[First released:]
June 10, 2024
\item[Minimum Typst version:]
0.11.0
\item[Archive size:]
5.17 kB
\href{https://packages.typst.org/preview/soviet-matrix-0.1.1.tar.gz}{\pandocbounded{\includesvg[keepaspectratio]{/assets/icons/16-download.svg}}}
\item[Repository:]
\href{https://github.com/YouXam/soviet-matrix}{GitHub}
\item[Categor y :]
\begin{itemize}
\tightlist
\item[]
\item
  \pandocbounded{\includesvg[keepaspectratio]{/assets/icons/16-smile.svg}}
  \href{https://typst.app/universe/search/?category=fun}{Fun}
\end{itemize}
\end{description}

\subsubsection{Where to report issues?}\label{where-to-report-issues}

This template is a project of YouXam . Report issues on
\href{https://github.com/YouXam/soviet-matrix}{their repository} . You
can also try to ask for help with this template on the
\href{https://forum.typst.app}{Forum} .

Please report this template to the Typst team using the
\href{https://typst.app/contact}{contact form} if you believe it is a
safety hazard or infringes upon your rights.

\phantomsection\label{versions}
\subsubsection{Version history}\label{version-history}

\begin{longtable}[]{@{}ll@{}}
\toprule\noalign{}
Version & Release Date \\
\midrule\noalign{}
\endhead
\bottomrule\noalign{}
\endlastfoot
0.1.1 & July 3, 2024 \\
\href{https://typst.app/universe/package/soviet-matrix/0.1.0/}{0.1.0} &
June 10, 2024 \\
\end{longtable}

Typst GmbH did not create this template and cannot guarantee correct
functionality of this template or compatibility with any version of the
Typst compiler or app.


\section{Package List LaTeX/plotst.tex}
\title{typst.app/universe/package/plotst}

\phantomsection\label{banner}
\section{plotst}\label{plotst}

{ 0.2.0 }

A library to draw a variety of graphs and plots to use in your papers

\phantomsection\label{readme}
A Typst library for drawing graphs and plots. Made by Gewi413 and
Pegacraffft

\subsection{Currently supported
graphs}\label{currently-supported-graphs}

\begin{itemize}
\item
  Scatter plots
\item
  Graph charts
\item
  Histograms
\item
  Bar charts
\item
  Pie charts
\item
  Overlaying plots/charts

  (more to come)
\end{itemize}

\subsection{How to use}\label{how-to-use}

To use the package you can import it through this command
\texttt{\ import\ "@preview/plotst:0.2.0":\ *\ } . The documentation is
found in the
\href{https://github.com/Pegacraft/typst-plotting/blob/8d834689359b708ce75fe51be05eed45570e463e/docs/Docs.pdf}{Docs.pdf}
file. It contains all functions necessary to use this library. It also
includes a tutorial to create every available plot under their
respective render methods.

If you need some example code, check out
\href{https://github.com/Pegacraft/typst-plotting/blob/8d834689359b708ce75fe51be05eed45570e463e/example/main.typ}{main.typ}
. It also includes a
\href{https://github.com/Pegacraft/typst-plotting/blob/8d834689359b708ce75fe51be05eed45570e463e/example/Plotting.pdf}{compiled
version} .

\subsection{Examples:}\label{examples}

All these images were created using the
\href{https://github.com/Pegacraft/typst-plotting/blob/8d834689359b708ce75fe51be05eed45570e463e/example/main.typ}{main.typ}
.

\subsubsection{Scatter plots}\label{scatter-plots}

\begin{Shaded}
\begin{Highlighting}[]
\CommentTok{// Plot 1:}
\CommentTok{// The data to be displayed  }
\KeywordTok{let}\NormalTok{ gender\_data }\OperatorTok{=}\NormalTok{ (}
\NormalTok{  (}\StringTok{"w"}\OperatorTok{,} \DecValTok{1}\NormalTok{)}\OperatorTok{,}\NormalTok{ (}\StringTok{"w"}\OperatorTok{,} \DecValTok{3}\NormalTok{)}\OperatorTok{,}\NormalTok{ (}\StringTok{"w"}\OperatorTok{,} \DecValTok{5}\NormalTok{)}\OperatorTok{,}\NormalTok{ (}\StringTok{"w"}\OperatorTok{,} \DecValTok{4}\NormalTok{)}\OperatorTok{,}\NormalTok{ (}\StringTok{"m"}\OperatorTok{,} \DecValTok{2}\NormalTok{)}\OperatorTok{,}\NormalTok{ (}\StringTok{"m"}\OperatorTok{,} \DecValTok{2}\NormalTok{)}\OperatorTok{,}
\NormalTok{  (}\StringTok{"m"}\OperatorTok{,} \DecValTok{4}\NormalTok{)}\OperatorTok{,}\NormalTok{ (}\StringTok{"m"}\OperatorTok{,} \DecValTok{6}\NormalTok{)}\OperatorTok{,}\NormalTok{ (}\StringTok{"d"}\OperatorTok{,} \DecValTok{1}\NormalTok{)}\OperatorTok{,}\NormalTok{ (}\StringTok{"d"}\OperatorTok{,} \DecValTok{9}\NormalTok{)}\OperatorTok{,}\NormalTok{ (}\StringTok{"d"}\OperatorTok{,} \DecValTok{5}\NormalTok{)}\OperatorTok{,}\NormalTok{ (}\StringTok{"d"}\OperatorTok{,} \DecValTok{8}\NormalTok{)}\OperatorTok{,}
\NormalTok{  (}\StringTok{"d"}\OperatorTok{,} \DecValTok{3}\NormalTok{)}\OperatorTok{,}\NormalTok{ (}\StringTok{"d"}\OperatorTok{,} \DecValTok{1}\NormalTok{)}\OperatorTok{,}\NormalTok{ (}\DecValTok{0}\OperatorTok{,} \DecValTok{11}\NormalTok{)}
\NormalTok{)}

\CommentTok{// Create the axes used for the chart}
\KeywordTok{let}\NormalTok{ y\_axis }\OperatorTok{=} \FunctionTok{axis}\NormalTok{(min}\OperatorTok{:} \DecValTok{0}\OperatorTok{,}\NormalTok{ max}\OperatorTok{:} \DecValTok{11}\OperatorTok{,}\NormalTok{ step}\OperatorTok{:} \DecValTok{1}\OperatorTok{,}\NormalTok{ location}\OperatorTok{:} \StringTok{"left"}\OperatorTok{,}\NormalTok{ helper\_lines}\OperatorTok{:} \KeywordTok{true}\OperatorTok{,}\NormalTok{ invert\_markings}\OperatorTok{:} \KeywordTok{false}\OperatorTok{,}\NormalTok{ title}\OperatorTok{:} \StringTok{"foo"}\NormalTok{)}
\KeywordTok{let}\NormalTok{ x\_axis }\OperatorTok{=} \FunctionTok{axis}\NormalTok{(values}\OperatorTok{:}\NormalTok{ (}\StringTok{""}\OperatorTok{,} \StringTok{"m"}\OperatorTok{,} \StringTok{"w"}\OperatorTok{,} \StringTok{"d"}\NormalTok{)}\OperatorTok{,}\NormalTok{ location}\OperatorTok{:} \StringTok{"bottom"}\OperatorTok{,}\NormalTok{ helper\_lines}\OperatorTok{:} \KeywordTok{true}\OperatorTok{,}\NormalTok{ invert\_markings}\OperatorTok{:} \KeywordTok{false}\OperatorTok{,}\NormalTok{ title}\OperatorTok{:} \StringTok{"Gender"}\NormalTok{)}

\CommentTok{// Combine the axes and the data and feed it to the plot render function.}
\KeywordTok{let}\NormalTok{ pl }\OperatorTok{=} \FunctionTok{plot}\NormalTok{(data}\OperatorTok{:}\NormalTok{ gender\_data}\OperatorTok{,}\NormalTok{ axes}\OperatorTok{:}\NormalTok{ (x\_axis}\OperatorTok{,}\NormalTok{ y\_axis))}
\FunctionTok{scatter\_plot}\NormalTok{(pl}\OperatorTok{,}\NormalTok{ (}\DecValTok{100}\OperatorTok{\%,}\DecValTok{50}\OperatorTok{\%}\NormalTok{))}

\CommentTok{// Plot 2:}
\CommentTok{// Same as above}
\KeywordTok{let}\NormalTok{ data }\OperatorTok{=}\NormalTok{ (}
\NormalTok{  (}\DecValTok{0}\OperatorTok{,} \DecValTok{0}\NormalTok{)}\OperatorTok{,}\NormalTok{ (}\DecValTok{2}\OperatorTok{,} \DecValTok{2}\NormalTok{)}\OperatorTok{,}\NormalTok{ (}\DecValTok{3}\OperatorTok{,} \DecValTok{0}\NormalTok{)}\OperatorTok{,}\NormalTok{ (}\DecValTok{4}\OperatorTok{,} \DecValTok{4}\NormalTok{)}\OperatorTok{,}\NormalTok{ (}\DecValTok{5}\OperatorTok{,} \DecValTok{7}\NormalTok{)}\OperatorTok{,}\NormalTok{ (}\DecValTok{6}\OperatorTok{,} \DecValTok{6}\NormalTok{)}\OperatorTok{,}\NormalTok{ (}\DecValTok{7}\OperatorTok{,} \DecValTok{9}\NormalTok{)}\OperatorTok{,}\NormalTok{ (}\DecValTok{8}\OperatorTok{,} \DecValTok{5}\NormalTok{)}\OperatorTok{,}\NormalTok{ (}\DecValTok{9}\OperatorTok{,} \DecValTok{9}\NormalTok{)}\OperatorTok{,}\NormalTok{ (}\DecValTok{10}\OperatorTok{,} \DecValTok{1}\NormalTok{)}
\NormalTok{)}
\KeywordTok{let}\NormalTok{ x\_axis }\OperatorTok{=} \FunctionTok{axis}\NormalTok{(min}\OperatorTok{:} \DecValTok{0}\OperatorTok{,}\NormalTok{ max}\OperatorTok{:} \DecValTok{11}\OperatorTok{,}\NormalTok{ step}\OperatorTok{:} \DecValTok{2}\OperatorTok{,}\NormalTok{ location}\OperatorTok{:} \StringTok{"bottom"}\NormalTok{)}
\KeywordTok{let}\NormalTok{ y\_axis }\OperatorTok{=} \FunctionTok{axis}\NormalTok{(min}\OperatorTok{:} \DecValTok{0}\OperatorTok{,}\NormalTok{ max}\OperatorTok{:} \DecValTok{11}\OperatorTok{,}\NormalTok{ step}\OperatorTok{:} \DecValTok{2}\OperatorTok{,}\NormalTok{ location}\OperatorTok{:} \StringTok{"left"}\OperatorTok{,}\NormalTok{ helper\_lines}\OperatorTok{:} \KeywordTok{false}\NormalTok{)}
\KeywordTok{let}\NormalTok{ pl }\OperatorTok{=} \FunctionTok{plot}\NormalTok{(data}\OperatorTok{:}\NormalTok{ data}\OperatorTok{,}\NormalTok{ axes}\OperatorTok{:}\NormalTok{ (x\_axis}\OperatorTok{,}\NormalTok{ y\_axis))}
\FunctionTok{scatter\_plot}\NormalTok{(pl}\OperatorTok{,}\NormalTok{ (}\DecValTok{100}\OperatorTok{\%,} \DecValTok{25}\OperatorTok{\%}\NormalTok{))}
\end{Highlighting}
\end{Shaded}

\pandocbounded{\includegraphics[keepaspectratio]{https://raw.githubusercontent.com/Pegacraft/typst-plotting/8d834689359b708ce75fe51be05eed45570e463e/images/scatter.png}}

\subsubsection{Graph charts}\label{graph-charts}

\begin{Shaded}
\begin{Highlighting}[]
\CommentTok{// The data to be displayed}
\KeywordTok{let}\NormalTok{ data }\OperatorTok{=}\NormalTok{ (}
\NormalTok{  (}\DecValTok{0}\OperatorTok{,} \DecValTok{0}\NormalTok{)}\OperatorTok{,}\NormalTok{ (}\DecValTok{2}\OperatorTok{,} \DecValTok{2}\NormalTok{)}\OperatorTok{,}\NormalTok{ (}\DecValTok{3}\OperatorTok{,} \DecValTok{0}\NormalTok{)}\OperatorTok{,}\NormalTok{ (}\DecValTok{4}\OperatorTok{,} \DecValTok{4}\NormalTok{)}\OperatorTok{,}\NormalTok{ (}\DecValTok{5}\OperatorTok{,} \DecValTok{7}\NormalTok{)}\OperatorTok{,}\NormalTok{ (}\DecValTok{6}\OperatorTok{,} \DecValTok{6}\NormalTok{)}\OperatorTok{,}\NormalTok{ (}\DecValTok{7}\OperatorTok{,} \DecValTok{9}\NormalTok{)}\OperatorTok{,}\NormalTok{ (}\DecValTok{8}\OperatorTok{,} \DecValTok{5}\NormalTok{)}\OperatorTok{,}\NormalTok{ (}\DecValTok{9}\OperatorTok{,} \DecValTok{9}\NormalTok{)}\OperatorTok{,}\NormalTok{ (}\DecValTok{10}\OperatorTok{,} \DecValTok{1}\NormalTok{)}
\NormalTok{)}

\CommentTok{// Create the axes used for the chart }
\KeywordTok{let}\NormalTok{ x\_axis }\OperatorTok{=} \FunctionTok{axis}\NormalTok{(min}\OperatorTok{:} \DecValTok{0}\OperatorTok{,}\NormalTok{ max}\OperatorTok{:} \DecValTok{11}\OperatorTok{,}\NormalTok{ step}\OperatorTok{:} \DecValTok{2}\OperatorTok{,}\NormalTok{ location}\OperatorTok{:} \StringTok{"bottom"}\NormalTok{)}
\KeywordTok{let}\NormalTok{ y\_axis }\OperatorTok{=} \FunctionTok{axis}\NormalTok{(min}\OperatorTok{:} \DecValTok{0}\OperatorTok{,}\NormalTok{ max}\OperatorTok{:} \DecValTok{11}\OperatorTok{,}\NormalTok{ step}\OperatorTok{:} \DecValTok{2}\OperatorTok{,}\NormalTok{ location}\OperatorTok{:} \StringTok{"left"}\OperatorTok{,}\NormalTok{ helper\_lines}\OperatorTok{:} \KeywordTok{false}\NormalTok{)}

\CommentTok{// Combine the axes and the data and feed it to the plot render function.}
\KeywordTok{let}\NormalTok{ pl }\OperatorTok{=} \FunctionTok{plot}\NormalTok{(data}\OperatorTok{:}\NormalTok{ data}\OperatorTok{,}\NormalTok{ axes}\OperatorTok{:}\NormalTok{ (x\_axis}\OperatorTok{,}\NormalTok{ y\_axis))}
\FunctionTok{graph\_plot}\NormalTok{(pl}\OperatorTok{,}\NormalTok{ (}\DecValTok{100}\OperatorTok{\%,} \DecValTok{25}\OperatorTok{\%}\NormalTok{))}
\FunctionTok{graph\_plot}\NormalTok{(pl}\OperatorTok{,}\NormalTok{ (}\DecValTok{100}\OperatorTok{\%,} \DecValTok{25}\OperatorTok{\%}\NormalTok{)}\OperatorTok{,}\NormalTok{ rounding}\OperatorTok{:} \DecValTok{30}\OperatorTok{\%,}\NormalTok{ caption}\OperatorTok{:} \StringTok{"Graph Plot with caption and rounding"}\NormalTok{)}
\end{Highlighting}
\end{Shaded}

\pandocbounded{\includegraphics[keepaspectratio]{https://raw.githubusercontent.com/Pegacraft/typst-plotting/8d834689359b708ce75fe51be05eed45570e463e/images/graph.png}}

\subsubsection{Histograms}\label{histograms}

\begin{Shaded}
\begin{Highlighting}[]
\CommentTok{// Plot 1:}
\CommentTok{// The data to be displayed}
\KeywordTok{let}\NormalTok{ data }\OperatorTok{=}\NormalTok{ (}
  \DecValTok{18000}\OperatorTok{,} \DecValTok{18000}\OperatorTok{,} \DecValTok{18000}\OperatorTok{,} \DecValTok{18000}\OperatorTok{,} \DecValTok{18000}\OperatorTok{,} \DecValTok{18000}\OperatorTok{,} \DecValTok{18000}\OperatorTok{,} \DecValTok{18000}\OperatorTok{,}
  \DecValTok{18000}\OperatorTok{,} \DecValTok{18000}\OperatorTok{,} \DecValTok{28000}\OperatorTok{,} \DecValTok{28000}\OperatorTok{,} \DecValTok{28000}\OperatorTok{,} \DecValTok{28000}\OperatorTok{,} \DecValTok{28000}\OperatorTok{,} \DecValTok{28000}\OperatorTok{,}
  \DecValTok{28000}\OperatorTok{,} \DecValTok{28000}\OperatorTok{,} \DecValTok{28000}\OperatorTok{,} \DecValTok{28000}\OperatorTok{,} \DecValTok{28000}\OperatorTok{,} \DecValTok{28000}\OperatorTok{,} \DecValTok{28000}\OperatorTok{,} \DecValTok{28000}\OperatorTok{,}
  \DecValTok{28000}\OperatorTok{,} \DecValTok{28000}\OperatorTok{,} \DecValTok{28000}\OperatorTok{,} \DecValTok{28000}\OperatorTok{,} \DecValTok{28000}\OperatorTok{,} \DecValTok{28000}\OperatorTok{,} \DecValTok{28000}\OperatorTok{,} \DecValTok{28000}\OperatorTok{,}
  \DecValTok{35000}\OperatorTok{,} \DecValTok{46000}\OperatorTok{,} \DecValTok{75000}\OperatorTok{,} \DecValTok{95000}
\NormalTok{)}

\CommentTok{// Classify the data}
\KeywordTok{let}\NormalTok{ classes }\OperatorTok{=} \FunctionTok{class\_generator}\NormalTok{(}\DecValTok{10000}\OperatorTok{,} \DecValTok{50000}\OperatorTok{,} \DecValTok{4}\NormalTok{)}
\NormalTok{classes}\OperatorTok{.}\FunctionTok{push}\NormalTok{(}\KeywordTok{class}\NormalTok{(}\DecValTok{50000}\OperatorTok{,} \DecValTok{100000}\NormalTok{))}
\NormalTok{classes }\OperatorTok{=} \FunctionTok{classify}\NormalTok{(data}\OperatorTok{,}\NormalTok{ classes)}

\CommentTok{// Create the axes used for the chart }
\KeywordTok{let}\NormalTok{ x\_axis }\OperatorTok{=} \FunctionTok{axis}\NormalTok{(min}\OperatorTok{:} \DecValTok{0}\OperatorTok{,}\NormalTok{ max}\OperatorTok{:} \DecValTok{100000}\OperatorTok{,}\NormalTok{ step}\OperatorTok{:} \DecValTok{10000}\OperatorTok{,}\NormalTok{ location}\OperatorTok{:} \StringTok{"bottom"}\NormalTok{)}
\KeywordTok{let}\NormalTok{ y\_axis }\OperatorTok{=} \FunctionTok{axis}\NormalTok{(min}\OperatorTok{:} \DecValTok{0}\OperatorTok{,}\NormalTok{ max}\OperatorTok{:} \DecValTok{31}\OperatorTok{,}\NormalTok{ step}\OperatorTok{:} \DecValTok{5}\OperatorTok{,}\NormalTok{ location}\OperatorTok{:} \StringTok{"left"}\OperatorTok{,}\NormalTok{ helper\_lines}\OperatorTok{:} \KeywordTok{true}\NormalTok{)}

\CommentTok{// Combine the axes and the data and feed it to the plot render function.}
\KeywordTok{let}\NormalTok{ pl }\OperatorTok{=} \FunctionTok{plot}\NormalTok{(data}\OperatorTok{:}\NormalTok{ classes}\OperatorTok{,}\NormalTok{ axes}\OperatorTok{:}\NormalTok{ (x\_axis}\OperatorTok{,}\NormalTok{ y\_axis))}
\FunctionTok{histogram}\NormalTok{(pl}\OperatorTok{,}\NormalTok{ (}\DecValTok{100}\OperatorTok{\%,} \DecValTok{40}\OperatorTok{\%}\NormalTok{)}\OperatorTok{,}\NormalTok{ stroke}\OperatorTok{:}\NormalTok{ black}\OperatorTok{,}\NormalTok{ fill}\OperatorTok{:}\NormalTok{ (purple}\OperatorTok{,}\NormalTok{ blue}\OperatorTok{,}\NormalTok{ red}\OperatorTok{,}\NormalTok{ green}\OperatorTok{,}\NormalTok{ yellow))}

\CommentTok{// Plot 2:}
\CommentTok{// Create the different classes}
\KeywordTok{let}\NormalTok{ classes }\OperatorTok{=}\NormalTok{ ()}
\NormalTok{classes}\OperatorTok{.}\FunctionTok{push}\NormalTok{(}\KeywordTok{class}\NormalTok{(}\DecValTok{11}\OperatorTok{,} \DecValTok{13}\NormalTok{))}
\NormalTok{classes}\OperatorTok{.}\FunctionTok{push}\NormalTok{(}\KeywordTok{class}\NormalTok{(}\DecValTok{13}\OperatorTok{,} \DecValTok{15}\NormalTok{))}
\NormalTok{classes}\OperatorTok{.}\FunctionTok{push}\NormalTok{(}\KeywordTok{class}\NormalTok{(}\DecValTok{1}\OperatorTok{,} \DecValTok{6}\NormalTok{))}
\NormalTok{classes}\OperatorTok{.}\FunctionTok{push}\NormalTok{(}\KeywordTok{class}\NormalTok{(}\DecValTok{6}\OperatorTok{,} \DecValTok{11}\NormalTok{))}
\NormalTok{classes}\OperatorTok{.}\FunctionTok{push}\NormalTok{(}\KeywordTok{class}\NormalTok{(}\DecValTok{15}\OperatorTok{,} \DecValTok{30}\NormalTok{))}

\CommentTok{// Define the data to map}
\KeywordTok{let}\NormalTok{ data }\OperatorTok{=}\NormalTok{ ((}\DecValTok{20}\OperatorTok{,} \DecValTok{2}\NormalTok{)}\OperatorTok{,}\NormalTok{ (}\DecValTok{30}\OperatorTok{,} \DecValTok{7}\NormalTok{)}\OperatorTok{,}\NormalTok{ (}\DecValTok{16}\OperatorTok{,} \DecValTok{12}\NormalTok{)}\OperatorTok{,}\NormalTok{ (}\DecValTok{40}\OperatorTok{,} \DecValTok{13}\NormalTok{)}\OperatorTok{,}\NormalTok{ (}\DecValTok{5}\OperatorTok{,} \DecValTok{17}\NormalTok{))}

\CommentTok{// Create the axes}
\KeywordTok{let}\NormalTok{ x\_axis }\OperatorTok{=} \FunctionTok{axis}\NormalTok{(min}\OperatorTok{:} \DecValTok{0}\OperatorTok{,}\NormalTok{ max}\OperatorTok{:} \DecValTok{31}\OperatorTok{,}\NormalTok{ step}\OperatorTok{:} \DecValTok{1}\OperatorTok{,}\NormalTok{ location}\OperatorTok{:} \StringTok{"bottom"}\OperatorTok{,}\NormalTok{ show\_markings}\OperatorTok{:} \KeywordTok{false}\NormalTok{)}
\KeywordTok{let}\NormalTok{ y\_axis }\OperatorTok{=} \FunctionTok{axis}\NormalTok{(min}\OperatorTok{:} \DecValTok{0}\OperatorTok{,}\NormalTok{ max}\OperatorTok{:} \DecValTok{41}\OperatorTok{,}\NormalTok{ step}\OperatorTok{:} \DecValTok{5}\OperatorTok{,}\NormalTok{ location}\OperatorTok{:} \StringTok{"left"}\OperatorTok{,}\NormalTok{ helper\_lines}\OperatorTok{:} \KeywordTok{true}\NormalTok{)}

\CommentTok{// Classify the data}
\NormalTok{classes }\OperatorTok{=} \FunctionTok{classify}\NormalTok{(data}\OperatorTok{,}\NormalTok{ classes)}

\CommentTok{// Combine the axes and the data and feed it to the plot render function.}
\KeywordTok{let}\NormalTok{ pl }\OperatorTok{=} \FunctionTok{plot}\NormalTok{(axes}\OperatorTok{:}\NormalTok{ (x\_axis}\OperatorTok{,}\NormalTok{ y\_axis)}\OperatorTok{,}\NormalTok{ data}\OperatorTok{:}\NormalTok{ classes)}
\FunctionTok{histogram}\NormalTok{(pl}\OperatorTok{,}\NormalTok{ (}\DecValTok{100}\OperatorTok{\%,} \DecValTok{40}\OperatorTok{\%}\NormalTok{))}
\end{Highlighting}
\end{Shaded}

\pandocbounded{\includegraphics[keepaspectratio]{https://raw.githubusercontent.com/Pegacraft/typst-plotting/8d834689359b708ce75fe51be05eed45570e463e/images/histogram.png}}

\subsubsection{Bar charts}\label{bar-charts}

\begin{Shaded}
\begin{Highlighting}[]
\CommentTok{// Plot 1:}
\CommentTok{// The data to be displayed}
\KeywordTok{let}\NormalTok{ data }\OperatorTok{=}\NormalTok{ ((}\DecValTok{10}\OperatorTok{,} \StringTok{"Monday"}\NormalTok{)}\OperatorTok{,}\NormalTok{ (}\DecValTok{5}\OperatorTok{,} \StringTok{"Tuesday"}\NormalTok{)}\OperatorTok{,}\NormalTok{ (}\DecValTok{15}\OperatorTok{,} \StringTok{"Wednesday"}\NormalTok{)}\OperatorTok{,}\NormalTok{ (}\DecValTok{9}\OperatorTok{,} \StringTok{"Thursday"}\NormalTok{)}\OperatorTok{,}\NormalTok{ (}\DecValTok{11}\OperatorTok{,} \StringTok{"Friday"}\NormalTok{))}

\CommentTok{// Create the necessary axes}
\KeywordTok{let}\NormalTok{ y\_axis }\OperatorTok{=} \FunctionTok{axis}\NormalTok{(values}\OperatorTok{:}\NormalTok{ (}\StringTok{""}\OperatorTok{,} \StringTok{"Monday"}\OperatorTok{,} \StringTok{"Tuesday"}\OperatorTok{,} \StringTok{"Wednesday"}\OperatorTok{,} \StringTok{"Thursday"}\OperatorTok{,} \StringTok{"Friday"}\NormalTok{)}\OperatorTok{,}\NormalTok{ location}\OperatorTok{:} \StringTok{"left"}\OperatorTok{,}\NormalTok{ show\_markings}\OperatorTok{:} \KeywordTok{true}\NormalTok{)}
\KeywordTok{let}\NormalTok{ x\_axis }\OperatorTok{=} \FunctionTok{axis}\NormalTok{(min}\OperatorTok{:} \DecValTok{0}\OperatorTok{,}\NormalTok{ max}\OperatorTok{:} \DecValTok{20}\OperatorTok{,}\NormalTok{ step}\OperatorTok{:} \DecValTok{2}\OperatorTok{,}\NormalTok{ location}\OperatorTok{:} \StringTok{"bottom"}\OperatorTok{,}\NormalTok{ helper\_lines}\OperatorTok{:} \KeywordTok{true}\NormalTok{)}

\CommentTok{// Combine the axes and the data and feed it to the plot render function.}
\KeywordTok{let}\NormalTok{ pl }\OperatorTok{=} \FunctionTok{plot}\NormalTok{(axes}\OperatorTok{:}\NormalTok{ (x\_axis}\OperatorTok{,}\NormalTok{ y\_axis)}\OperatorTok{,}\NormalTok{ data}\OperatorTok{:}\NormalTok{ data)}
\FunctionTok{bar\_chart}\NormalTok{(pl}\OperatorTok{,}\NormalTok{ (}\DecValTok{100}\OperatorTok{\%,} \DecValTok{33}\OperatorTok{\%}\NormalTok{)}\OperatorTok{,}\NormalTok{ fill}\OperatorTok{:}\NormalTok{ (purple}\OperatorTok{,}\NormalTok{ blue}\OperatorTok{,}\NormalTok{ red}\OperatorTok{,}\NormalTok{ green}\OperatorTok{,}\NormalTok{ yellow)}\OperatorTok{,}\NormalTok{ bar\_width}\OperatorTok{:} \DecValTok{70}\OperatorTok{\%,}\NormalTok{ rotated}\OperatorTok{:} \KeywordTok{true}\NormalTok{)}

\CommentTok{// Plot 2:}
\CommentTok{// Same as above, but with numbers as data}
\KeywordTok{let}\NormalTok{ data\_2 }\OperatorTok{=}\NormalTok{ ((}\DecValTok{20}\OperatorTok{,} \DecValTok{2}\NormalTok{)}\OperatorTok{,}\NormalTok{ (}\DecValTok{30}\OperatorTok{,} \DecValTok{7}\NormalTok{)}\OperatorTok{,}\NormalTok{ (}\DecValTok{16}\OperatorTok{,} \DecValTok{12}\NormalTok{)}\OperatorTok{,}\NormalTok{ (}\DecValTok{40}\OperatorTok{,} \DecValTok{13}\NormalTok{)}\OperatorTok{,}\NormalTok{ (}\DecValTok{5}\OperatorTok{,} \DecValTok{17}\NormalTok{))}
\KeywordTok{let}\NormalTok{ y\_axis\_2 }\OperatorTok{=} \FunctionTok{axis}\NormalTok{(min}\OperatorTok{:} \DecValTok{0}\OperatorTok{,}\NormalTok{ max}\OperatorTok{:} \DecValTok{41}\OperatorTok{,}\NormalTok{ step}\OperatorTok{:} \DecValTok{5}\OperatorTok{,}\NormalTok{ location}\OperatorTok{:} \StringTok{"left"}\OperatorTok{,}\NormalTok{ show\_markings}\OperatorTok{:} \KeywordTok{true}\OperatorTok{,}\NormalTok{ helper\_lines}\OperatorTok{:} \KeywordTok{true}\NormalTok{)}
\KeywordTok{let}\NormalTok{ x\_axis\_2 }\OperatorTok{=} \FunctionTok{axis}\NormalTok{(min}\OperatorTok{:} \DecValTok{0}\OperatorTok{,}\NormalTok{ max}\OperatorTok{:} \DecValTok{21}\OperatorTok{,}\NormalTok{ step}\OperatorTok{:} \DecValTok{1}\OperatorTok{,}\NormalTok{ location}\OperatorTok{:} \StringTok{"bottom"}\NormalTok{)}
\KeywordTok{let}\NormalTok{ pl\_2 }\OperatorTok{=} \FunctionTok{plot}\NormalTok{(axes}\OperatorTok{:}\NormalTok{ (x\_axis\_2}\OperatorTok{,}\NormalTok{ y\_axis\_2)}\OperatorTok{,}\NormalTok{ data}\OperatorTok{:}\NormalTok{ data\_2)}
\FunctionTok{bar\_chart}\NormalTok{(pl\_2}\OperatorTok{,}\NormalTok{ (}\DecValTok{100}\OperatorTok{\%,} \DecValTok{60}\OperatorTok{\%}\NormalTok{)}\OperatorTok{,}\NormalTok{ bar\_width}\OperatorTok{:} \DecValTok{100}\OperatorTok{\%}\NormalTok{)}
\end{Highlighting}
\end{Shaded}

\pandocbounded{\includegraphics[keepaspectratio]{https://raw.githubusercontent.com/Pegacraft/typst-plotting/8d834689359b708ce75fe51be05eed45570e463e/images/bar.png}}

\subsubsection{Pie charts}\label{pie-charts}

\begin{Shaded}
\begin{Highlighting}[]
\NormalTok{show}\OperatorTok{:}\NormalTok{ r }\KeywordTok{=\textgreater{}} \FunctionTok{columns}\NormalTok{(}\DecValTok{2}\OperatorTok{,}\NormalTok{ r)}

\CommentTok{// create the sample data}
\KeywordTok{let}\NormalTok{ data }\OperatorTok{=}\NormalTok{ ((}\DecValTok{10}\OperatorTok{,} \StringTok{"Male"}\NormalTok{)}\OperatorTok{,}\NormalTok{ (}\DecValTok{20}\OperatorTok{,} \StringTok{"Female"}\NormalTok{)}\OperatorTok{,}\NormalTok{ (}\DecValTok{15}\OperatorTok{,} \StringTok{"Divers"}\NormalTok{)}\OperatorTok{,}\NormalTok{ (}\DecValTok{2}\OperatorTok{,} \StringTok{"Other"}\NormalTok{)}

\CommentTok{// Skip the axis step, as no axes are needed}

\CommentTok{// Put the data into a plot }
\KeywordTok{let}\NormalTok{ p }\OperatorTok{=} \FunctionTok{plot}\NormalTok{(data}\OperatorTok{:}\NormalTok{ data)}

\CommentTok{// Display the pie\_charts in all different display ways}
\FunctionTok{pie\_chart}\NormalTok{(p}\OperatorTok{,}\NormalTok{ (}\DecValTok{100}\OperatorTok{\%,} \DecValTok{20}\OperatorTok{\%}\NormalTok{)}\OperatorTok{,}\NormalTok{ display\_style}\OperatorTok{:} \StringTok{"legend{-}inside{-}chart"}\NormalTok{)}
\FunctionTok{pie\_chart}\NormalTok{(p}\OperatorTok{,}\NormalTok{ (}\DecValTok{100}\OperatorTok{\%,} \DecValTok{20}\OperatorTok{\%}\NormalTok{)}\OperatorTok{,}\NormalTok{ display\_style}\OperatorTok{:} \StringTok{"hor{-}chart{-}legend"}\NormalTok{)}
\FunctionTok{pie\_chart}\NormalTok{(p}\OperatorTok{,}\NormalTok{ (}\DecValTok{100}\OperatorTok{\%,} \DecValTok{20}\OperatorTok{\%}\NormalTok{)}\OperatorTok{,}\NormalTok{ display\_style}\OperatorTok{:} \StringTok{"hor{-}legend{-}chart"}\NormalTok{)}
\FunctionTok{pie\_chart}\NormalTok{(p}\OperatorTok{,}\NormalTok{ (}\DecValTok{100}\OperatorTok{\%,} \DecValTok{20}\OperatorTok{\%}\NormalTok{)}\OperatorTok{,}\NormalTok{ display\_style}\OperatorTok{:} \StringTok{"vert{-}chart{-}legend"}\NormalTok{)}
\FunctionTok{pie\_chart}\NormalTok{(p}\OperatorTok{,}\NormalTok{ (}\DecValTok{100}\OperatorTok{\%,} \DecValTok{20}\OperatorTok{\%}\NormalTok{)}\OperatorTok{,}\NormalTok{ display\_style}\OperatorTok{:} \StringTok{"vert{-}legend{-}chart"}\NormalTok{)}
\end{Highlighting}
\end{Shaded}

\pandocbounded{\includegraphics[keepaspectratio]{https://raw.githubusercontent.com/Pegacraft/typst-plotting/8d834689359b708ce75fe51be05eed45570e463e/images/pie.png}}

\textbf{Overlayed Graphs}

\begin{Shaded}
\begin{Highlighting}[]
\CommentTok{// Create the data for the two plots to overlay}
\KeywordTok{let}\NormalTok{ data\_scatter }\OperatorTok{=}\NormalTok{ (}
\NormalTok{  (}\DecValTok{0}\OperatorTok{,} \DecValTok{0}\NormalTok{)}\OperatorTok{,}\NormalTok{ (}\DecValTok{2}\OperatorTok{,} \DecValTok{2}\NormalTok{)}\OperatorTok{,}\NormalTok{ (}\DecValTok{3}\OperatorTok{,} \DecValTok{0}\NormalTok{)}\OperatorTok{,}\NormalTok{ (}\DecValTok{4}\OperatorTok{,} \DecValTok{4}\NormalTok{)}\OperatorTok{,}\NormalTok{ (}\DecValTok{5}\OperatorTok{,} \DecValTok{7}\NormalTok{)}\OperatorTok{,}\NormalTok{ (}\DecValTok{6}\OperatorTok{,} \DecValTok{6}\NormalTok{)}\OperatorTok{,}\NormalTok{ (}\DecValTok{7}\OperatorTok{,} \DecValTok{9}\NormalTok{)}\OperatorTok{,}\NormalTok{ (}\DecValTok{8}\OperatorTok{,} \DecValTok{5}\NormalTok{)}\OperatorTok{,}\NormalTok{ (}\DecValTok{9}\OperatorTok{,} \DecValTok{9}\NormalTok{)}\OperatorTok{,}\NormalTok{ (}\DecValTok{10}\OperatorTok{,} \DecValTok{1}\NormalTok{)}
\NormalTok{)}
\KeywordTok{let}\NormalTok{ data\_graph }\OperatorTok{=}\NormalTok{ (}
\NormalTok{  (}\DecValTok{0}\OperatorTok{,} \DecValTok{3}\NormalTok{)}\OperatorTok{,}\NormalTok{ (}\DecValTok{1}\OperatorTok{,} \DecValTok{5}\NormalTok{)}\OperatorTok{,}\NormalTok{ (}\DecValTok{2}\OperatorTok{,} \DecValTok{1}\NormalTok{)}\OperatorTok{,}\NormalTok{ (}\DecValTok{3}\OperatorTok{,} \DecValTok{7}\NormalTok{)}\OperatorTok{,}\NormalTok{ (}\DecValTok{4}\OperatorTok{,} \DecValTok{3}\NormalTok{)}\OperatorTok{,}\NormalTok{ (}\DecValTok{5}\OperatorTok{,} \DecValTok{5}\NormalTok{)}\OperatorTok{,}\NormalTok{ (}\DecValTok{6}\OperatorTok{,} \DecValTok{7}\NormalTok{)}\OperatorTok{,}\NormalTok{(}\DecValTok{7}\OperatorTok{,} \DecValTok{4}\NormalTok{)}\OperatorTok{,}\NormalTok{(}\DecValTok{11}\OperatorTok{,} \DecValTok{6}\NormalTok{)}
\NormalTok{)}

\CommentTok{// Create the axes for the overlay plot}
\KeywordTok{let}\NormalTok{ x\_axis }\OperatorTok{=} \FunctionTok{axis}\NormalTok{(min}\OperatorTok{:} \DecValTok{0}\OperatorTok{,}\NormalTok{ max}\OperatorTok{:} \DecValTok{11}\OperatorTok{,}\NormalTok{ step}\OperatorTok{:} \DecValTok{2}\OperatorTok{,}\NormalTok{ location}\OperatorTok{:} \StringTok{"bottom"}\NormalTok{)}
\KeywordTok{let}\NormalTok{ y\_axis }\OperatorTok{=} \FunctionTok{axis}\NormalTok{(min}\OperatorTok{:} \DecValTok{0}\OperatorTok{,}\NormalTok{ max}\OperatorTok{:} \DecValTok{11}\OperatorTok{,}\NormalTok{ step}\OperatorTok{:} \DecValTok{2}\OperatorTok{,}\NormalTok{ location}\OperatorTok{:} \StringTok{"left"}\OperatorTok{,}\NormalTok{ helper\_lines}\OperatorTok{:} \KeywordTok{false}\NormalTok{)}

\CommentTok{// create a plot for each individual plot type and save the render call}
\KeywordTok{let}\NormalTok{ pl\_scatter }\OperatorTok{=} \FunctionTok{plot}\NormalTok{(data}\OperatorTok{:}\NormalTok{ data\_scatter}\OperatorTok{,}\NormalTok{ axes}\OperatorTok{:}\NormalTok{ (x\_axis}\OperatorTok{,}\NormalTok{ y\_axis))}
\KeywordTok{let}\NormalTok{ scatter\_display }\OperatorTok{=} \FunctionTok{scatter\_plot}\NormalTok{(pl\_scatter}\OperatorTok{,}\NormalTok{ (}\DecValTok{100}\OperatorTok{\%,} \DecValTok{25}\OperatorTok{\%}\NormalTok{)}\OperatorTok{,}\NormalTok{ stroke}\OperatorTok{:}\NormalTok{ red)}
\KeywordTok{let}\NormalTok{ pl\_graph }\OperatorTok{=} \FunctionTok{plot}\NormalTok{(data}\OperatorTok{:}\NormalTok{ data\_graph}\OperatorTok{,}\NormalTok{ axes}\OperatorTok{:}\NormalTok{ (x\_axis}\OperatorTok{,}\NormalTok{ y\_axis))}
\KeywordTok{let}\NormalTok{ graph\_display }\OperatorTok{=} \FunctionTok{graph\_plot}\NormalTok{(pl\_graph}\OperatorTok{,}\NormalTok{ (}\DecValTok{100}\OperatorTok{\%,} \DecValTok{25}\OperatorTok{\%}\NormalTok{)}\OperatorTok{,}\NormalTok{ stroke}\OperatorTok{:}\NormalTok{ blue)}

\CommentTok{// overlay the plots using the overlay function}
\FunctionTok{overlay}\NormalTok{((scatter\_display}\OperatorTok{,}\NormalTok{ graph\_display)}\OperatorTok{,}\NormalTok{ (}\DecValTok{100}\OperatorTok{\%,} \DecValTok{25}\OperatorTok{\%}\NormalTok{))}
\end{Highlighting}
\end{Shaded}

\pandocbounded{\includegraphics[keepaspectratio]{https://raw.githubusercontent.com/Pegacraft/typst-plotting/8d834689359b708ce75fe51be05eed45570e463e/images/overlay.png}}

\subsubsection{How to add}\label{how-to-add}

Copy this into your project and use the import as \texttt{\ plotst\ }

\begin{verbatim}
#import "@preview/plotst:0.2.0"
\end{verbatim}

\includesvg[width=0.16667in,height=0.16667in]{/assets/icons/16-copy.svg}

Check the docs for
\href{https://typst.app/docs/reference/scripting/\#packages}{more
information on how to import packages} .

\subsubsection{About}\label{about}

\begin{description}
\tightlist
\item[Author s :]
Pegacraft \& Gewi413
\item[License:]
MIT
\item[Current version:]
0.2.0
\item[Last updated:]
October 28, 2023
\item[First released:]
July 2, 2023
\item[Archive size:]
15.2 kB
\href{https://packages.typst.org/preview/plotst-0.2.0.tar.gz}{\pandocbounded{\includesvg[keepaspectratio]{/assets/icons/16-download.svg}}}
\item[Repository:]
\href{https://github.com/Pegacraft/typst-plotting}{GitHub}
\end{description}

\subsubsection{Where to report issues?}\label{where-to-report-issues}

This package is a project of Pegacraft and Gewi413 . Report issues on
\href{https://github.com/Pegacraft/typst-plotting}{their repository} .
You can also try to ask for help with this package on the
\href{https://forum.typst.app}{Forum} .

Please report this package to the Typst team using the
\href{https://typst.app/contact}{contact form} if you believe it is a
safety hazard or infringes upon your rights.

\phantomsection\label{versions}
\subsubsection{Version history}\label{version-history}

\begin{longtable}[]{@{}ll@{}}
\toprule\noalign{}
Version & Release Date \\
\midrule\noalign{}
\endhead
\bottomrule\noalign{}
\endlastfoot
0.2.0 & October 28, 2023 \\
\href{https://typst.app/universe/package/plotst/0.1.0/}{0.1.0} & July 2,
2023 \\
\end{longtable}

Typst GmbH did not create this package and cannot guarantee correct
functionality of this package or compatibility with any version of the
Typst compiler or app.


\section{Package List LaTeX/modernpro-cv.tex}
\title{typst.app/universe/package/modernpro-cv}

\phantomsection\label{banner}
\phantomsection\label{template-thumbnail}
\pandocbounded{\includegraphics[keepaspectratio]{https://packages.typst.org/preview/thumbnails/modernpro-cv-1.0.2-small.webp}}

\section{modernpro-cv}\label{modernpro-cv}

{ 1.0.2 }

A CV template inspired by Deedy-Resume.

\href{/app?template=modernpro-cv&version=1.0.2}{Create project in app}

\phantomsection\label{readme}
This Typst CV template is inspired by the Latex template
\href{https://github.com/deedy/Deedy-Resume}{Deedy-Resume} . You can use
it for both industry and academia.

If you want to find a cover letter template, you can check out
\href{https://github.com/jxpeng98/typst-coverletter}{modernpro-coverletter}
.

\subsection{How to start}\label{how-to-start}

\subsubsection{Use Typst CLI}\label{use-typst-cli}

If you use Typst CLI, you can use the following command to create a new
project:

\begin{Shaded}
\begin{Highlighting}[]
\ExtensionTok{typst}\NormalTok{ init modernpro{-}cv}
\end{Highlighting}
\end{Shaded}

It will create a folder named \texttt{\ modernpro-cv\ } with the
following structure:

\begin{Shaded}
\begin{Highlighting}[]
\NormalTok{modernpro{-}cv}
\NormalTok{├── bib.bib}
\NormalTok{├── cv\_double.typ}
\NormalTok{└── cv\_single.typ}
\end{Highlighting}
\end{Shaded}

If you want to use the single-column version, you can modify the
template \texttt{\ cv-single.typ\ } . If you prefer the two-column
version, you can use the \texttt{\ cv-double.typ\ } .

\textbf{Note:} The \texttt{\ bib.bib\ } is the bibliography file. You
can modify it to add your publications.

\subsubsection{Manual Download}\label{manual-download}

If you want to manually download the template, you can download
\texttt{\ modernpro-cv-\{version\}.zip\ } from the
\href{https://github.com/jxpeng98/Typst-CV-Resume/releases}{release
page}

\subsubsection{Typst website}\label{typst-website}

If you want to use the template via \href{https://typst.app/}{Typst} ,
You can \texttt{\ start\ from\ template\ } and search for
\texttt{\ modernpro-cv\ } .

\subsection{How to use the template}\label{how-to-use-the-template}

\subsubsection{The arguments}\label{the-arguments}

The template has the following arguments:

\begin{longtable}[]{@{}lll@{}}
\toprule\noalign{}
Argument & Description & Default \\
\midrule\noalign{}
\endhead
\bottomrule\noalign{}
\endlastfoot
\texttt{\ font-type\ } & The font type. You can choose any supported
font in your system. & \texttt{\ Times\ New\ Roman\ } \\
\texttt{\ continue-header\ } & Whether to continue the header on the
follwing pages. & \texttt{\ false\ } \\
\texttt{\ name\ } & Your name. & \texttt{\ ""\ } \\
\texttt{\ address\ } & Your address. & \texttt{\ ""\ } \\
\texttt{\ lastupdated\ } & Whether to show the last updated date. &
\texttt{\ true\ } \\
\texttt{\ pagecount\ } & Whether to show the page count. &
\texttt{\ true\ } \\
\texttt{\ date\ } & The date of the CV. & \texttt{\ today\ } \\
\texttt{\ contacts\ } & contact details, e.g phone number, email, etc. &
\texttt{\ (text:\ "",\ link:\ "")\ } \\
\end{longtable}

\subsubsection{Start single column
version}\label{start-single-column-version}

If you want to use the single column version, you create a new
\texttt{\ .typ\ } file and copy the following code:

\begin{Shaded}
\begin{Highlighting}[]
\NormalTok{\#import "@preview/modernpro{-}cv:1.0.2": *}
\NormalTok{\#import "@preview/fontawesome:0.5.0": *}

\NormalTok{\#show: cv{-}single.with(}
\NormalTok{  font{-}type: "PT Serif",}
\NormalTok{  continue{-}header: "false",}
\NormalTok{  name: [],}
\NormalTok{  address: [],}
\NormalTok{  lastupdated: "true",}
\NormalTok{  pagecount: "true",}
\NormalTok{  date: "2024{-}07{-}03",}
\NormalTok{  contacts: (}
\NormalTok{    (text: [\#fa{-}icon("location{-}dot") UK]),}
\NormalTok{    (text: [\#fa{-}icon("mobile") 123{-}456{-}789], link: "tel:123{-}456{-}789"),}
\NormalTok{    (text: [\#fa{-}icon("link") example.com], link: "https://www.example.com"),}
\NormalTok{  )}
\NormalTok{)}
\end{Highlighting}
\end{Shaded}

\subsubsection{Start double column
version}\label{start-double-column-version}

The double column version is similar to the single column version.
However, you need to add contents to the specific \texttt{\ left\ } and
\texttt{\ right\ } sections.

\begin{Shaded}
\begin{Highlighting}[]
\NormalTok{\#import "@preview/modernpro{-}cv:1.0.2": *}
\NormalTok{\#import "@preview/fontawesome:0.5.0": *}

\NormalTok{\#show: cv{-}double(}
\NormalTok{  font{-}type: "PT Sans",}
\NormalTok{  continue{-}header: "true",}
\NormalTok{  name: [\#lorem(2)],}
\NormalTok{  address: [\#lorem(4)],}
\NormalTok{  lastupdated: "true",}
\NormalTok{  pagecount: "true",}
\NormalTok{  date: "2024{-}07{-}03",}
\NormalTok{  contacts: (}
\NormalTok{    (text: [\#fa{-}icon("location{-}dot") UK]),}
\NormalTok{    (text: [\#fa{-}icon("mobile") 123{-}456{-}789], link: "tel:123{-}456{-}789"),}
\NormalTok{    (text: [\#fa{-}icon("link") example.com], link: "https://www.example.com"),}
\NormalTok{  ),}
\NormalTok{  left: [}
\NormalTok{    // contents for the left column}
\NormalTok{  ],}
\NormalTok{  right:[}
\NormalTok{    // contents for the right column}
\NormalTok{  ]}
\NormalTok{)}
\end{Highlighting}
\end{Shaded}

\subsubsection{Start the CV}\label{start-the-cv}

Once you set up the arguments, you can start to add details to your CV /
Resume.

I preset the following functions for you to create different parts:

\begin{longtable}[]{@{}ll@{}}
\toprule\noalign{}
Function & Description \\
\midrule\noalign{}
\endhead
\bottomrule\noalign{}
\endlastfoot
\texttt{\ \#section("Section\ Name")\ } & Start a new section \\
\texttt{\ \#sectionsep\ } & End the section \\
\texttt{\ \#oneline-title-item(title:\ "",\ content:\ "")\ } & Add a
one-line item ( \textbf{Title:} content) \\
\texttt{\ \#oneline-two(entry1:\ "",\ entry2:\ "")\ } & Add a one-line
item with two entries, aligned left and right \\
\texttt{\ \#descript("descriptions")\ } & Add a description for
self-introduction \\
\texttt{\ \#award(award:\ "",\ date:\ "",\ institution:\ "")\ } & Add an
award ( \textbf{award} , \emph{institution} \emph{date} ) \\
\texttt{\ \#education(institution:\ "",\ major:\ "",\ date:\ "",\ institution:\ "",\ core-modules:\ "")\ }
& Add an education experience \\
\texttt{\ \#job(position:\ "",\ institution:\ "",\ location:\ "",\ date:\ "",\ description:\ {[}{]})\ }
& Add a job experience (description is optional) \\
\texttt{\ \#twoline-item(entry1:\ "",\ entry2:\ "",\ entry3:\ "",\ entry4:\ "")\ }
& Two line items, similar to education and job experiences \\
\texttt{\ \#references(references:())\ } & Add a reference list. In the
\texttt{\ ()\ } , you can add multi reference entries with the following
format
\texttt{\ (name:\ "",\ position:\ "",\ department:\ "",\ institution:\ "",\ address:\ "",\ email:\ "",),\ } \\
\texttt{\ \#show\ bibliography:\ none\ \#bibliography("bib.bib")\ } &
Add a bibliography. You can modify the \texttt{\ bib.bib\ } file to add
your publications. \textbf{Note:} Keep this at the end of your CV \\
\end{longtable}

\textbf{Note:} Use \texttt{\ +\ @ref\ } to display your publications.
For example,

\begin{Shaded}
\begin{Highlighting}[]
\NormalTok{\#section("Publications")}

\NormalTok{// numbering list }
\NormalTok{+ @quenouille1949approximate}
\NormalTok{+ @quenouille1949approximate}

\NormalTok{// Keep this at the end}
\NormalTok{\#show bibliography: none}
\NormalTok{\#bibliography("bib.bib")}
\end{Highlighting}
\end{Shaded}

\subsection{Preview}\label{preview}

\subsubsection{Single Column}\label{single-column}

\pandocbounded{\includegraphics[keepaspectratio]{https://minioapi.pjx.ac.cn/img1/2024/07/a81ac7ec96be0625eefccb81ead160d3.png}}

\subsubsection{Double Column}\label{double-column}

\pandocbounded{\includegraphics[keepaspectratio]{https://minioapi.pjx.ac.cn/img1/2024/07/12e9b31e306055f615edf49f9b8ffe55.png}}

\subsection{Legacy Version}\label{legacy-version}

I redesigned the template and submitted the new version to Typst
Universe. However, you can find the legacy version in the
\texttt{\ legacy\ } folder if you prefer to use the multi-font setting.
You can also download the \texttt{\ modernpro-cv-legacy.zip\ } from the
\href{https://github.com/jxpeng98/Typst-CV-Resume/releases}{release
page} .

\textbf{Note:} The legacy version also has a cover letter template. You
can use it with the CV template.

\subsection{Cover Letter}\label{cover-letter}

If you used the previous version of this template, you might know that I
also provided a cover letter template.

If you want to use a consistent cover letter with the new version of the
CV template, you can find it from another repository
\href{https://github.com/jxpeng98/typst-coverletter}{typst-coverletter}
.

you can also use the following code in the command line:

\begin{Shaded}
\begin{Highlighting}[]
\ExtensionTok{typst}\NormalTok{ init modernpro{-}coverletter}
\end{Highlighting}
\end{Shaded}

\subsection{License}\label{license}

The template is released under the MIT License. For more information,
please refer to the
\href{https://github.com/jxpeng98/Typst-CV-Resume/blob/main/LICENSE}{LICENSE}
file.

\href{/app?template=modernpro-cv&version=1.0.2}{Create project in app}

\subsubsection{How to use}\label{how-to-use}

Click the button above to create a new project using this template in
the Typst app.

You can also use the Typst CLI to start a new project on your computer
using this command:

\begin{verbatim}
typst init @preview/modernpro-cv:1.0.2
\end{verbatim}

\includesvg[width=0.16667in,height=0.16667in]{/assets/icons/16-copy.svg}

\subsubsection{About}\label{about}

\begin{description}
\tightlist
\item[Author :]
jxpeng98
\item[License:]
MIT
\item[Current version:]
1.0.2
\item[Last updated:]
October 22, 2024
\item[First released:]
August 7, 2024
\item[Archive size:]
5.70 kB
\href{https://packages.typst.org/preview/modernpro-cv-1.0.2.tar.gz}{\pandocbounded{\includesvg[keepaspectratio]{/assets/icons/16-download.svg}}}
\item[Repository:]
\href{https://github.com/jxpeng98/Typst-CV-Resume}{GitHub}
\item[Categor y :]
\begin{itemize}
\tightlist
\item[]
\item
  \pandocbounded{\includesvg[keepaspectratio]{/assets/icons/16-user.svg}}
  \href{https://typst.app/universe/search/?category=cv}{CV}
\end{itemize}
\end{description}

\subsubsection{Where to report issues?}\label{where-to-report-issues}

This template is a project of jxpeng98 . Report issues on
\href{https://github.com/jxpeng98/Typst-CV-Resume}{their repository} .
You can also try to ask for help with this template on the
\href{https://forum.typst.app}{Forum} .

Please report this template to the Typst team using the
\href{https://typst.app/contact}{contact form} if you believe it is a
safety hazard or infringes upon your rights.

\phantomsection\label{versions}
\subsubsection{Version history}\label{version-history}

\begin{longtable}[]{@{}ll@{}}
\toprule\noalign{}
Version & Release Date \\
\midrule\noalign{}
\endhead
\bottomrule\noalign{}
\endlastfoot
1.0.2 & October 22, 2024 \\
\href{https://typst.app/universe/package/modernpro-cv/1.0.1/}{1.0.1} &
August 30, 2024 \\
\href{https://typst.app/universe/package/modernpro-cv/1.0.0/}{1.0.0} &
August 7, 2024 \\
\end{longtable}

Typst GmbH did not create this template and cannot guarantee correct
functionality of this template or compatibility with any version of the
Typst compiler or app.


\section{Package List LaTeX/statastic.tex}
\title{typst.app/universe/package/statastic}

\phantomsection\label{banner}
\section{statastic}\label{statastic}

{ 1.0.0 }

A library to calculate statistics for numerical data

\phantomsection\label{readme}
A library to calculate statistics for numerical data in typst.

\subsection{Description}\label{description}

\texttt{\ Statastic\ } is a Typst library designed to provide various
statistical functions for numerical data. It offers functionalities like
extracting specific columns from datasets, converting array elements to
different data types, and computing various statistical measures such as
average, median, mode, variance, standard deviation, and percentiles.

\subsection{Features}\label{features}

\begin{itemize}
\tightlist
\item
  \textbf{Extract Column} : Extracts a specific column from a given
  dataset.
\item
  \textbf{Type Conversion} : Convert array elements to floating point
  numbers or integers.
\item
  \textbf{Statistical Measures} : Calculate average, median, mode,
  variance, standard deviation, and specific percentiles for an array or
  a specific column in a dataset.
\end{itemize}

\subsection{Usage}\label{usage}

To use the package you can import it through this command
\texttt{\ import\ "@preview/statastical:1.0.0":\ *\ } (as soon as the
pull request ist accepted). The documentation is found in the
\texttt{\ docs.pdf\ } in the development
\href{https://github.com/dikkadev/typst-statastic}{repo}

\subsection{License}\label{license}

This project is licensed under the Unlicense.

\subsubsection{How to add}\label{how-to-add}

Copy this into your project and use the import as \texttt{\ statastic\ }

\begin{verbatim}
#import "@preview/statastic:1.0.0"
\end{verbatim}

\includesvg[width=0.16667in,height=0.16667in]{/assets/icons/16-copy.svg}

Check the docs for
\href{https://typst.app/docs/reference/scripting/\#packages}{more
information on how to import packages} .

\subsubsection{About}\label{about}

\begin{description}
\tightlist
\item[Author :]
dikkadev
\item[License:]
Unlicense
\item[Current version:]
1.0.0
\item[Last updated:]
October 24, 2024
\item[First released:]
September 3, 2023
\item[Archive size:]
4.27 kB
\href{https://packages.typst.org/preview/statastic-1.0.0.tar.gz}{\pandocbounded{\includesvg[keepaspectratio]{/assets/icons/16-download.svg}}}
\item[Repository:]
\href{https://github.com/dikkadev/typst-statastic}{GitHub}
\end{description}

\subsubsection{Where to report issues?}\label{where-to-report-issues}

This package is a project of dikkadev . Report issues on
\href{https://github.com/dikkadev/typst-statastic}{their repository} .
You can also try to ask for help with this package on the
\href{https://forum.typst.app}{Forum} .

Please report this package to the Typst team using the
\href{https://typst.app/contact}{contact form} if you believe it is a
safety hazard or infringes upon your rights.

\phantomsection\label{versions}
\subsubsection{Version history}\label{version-history}

\begin{longtable}[]{@{}ll@{}}
\toprule\noalign{}
Version & Release Date \\
\midrule\noalign{}
\endhead
\bottomrule\noalign{}
\endlastfoot
1.0.0 & October 24, 2024 \\
\href{https://typst.app/universe/package/statastic/0.1.0/}{0.1.0} &
September 3, 2023 \\
\end{longtable}

Typst GmbH did not create this package and cannot guarantee correct
functionality of this package or compatibility with any version of the
Typst compiler or app.


\section{Package List LaTeX/slydst.tex}
\title{typst.app/universe/package/slydst}

\phantomsection\label{banner}
\phantomsection\label{template-thumbnail}
\pandocbounded{\includegraphics[keepaspectratio]{https://packages.typst.org/preview/thumbnails/slydst-0.1.3-small.webp}}

\section{slydst}\label{slydst}

{ 0.1.3 }

Create simple static slides using standard headings

\href{/app?template=slydst&version=0.1.3}{Create project in app}

\phantomsection\label{readme}
Create simple static slides with Typst.

Slydst allows the creation of slides using Typst headings. This
simplicity comes at the expense of dynamic content such as subslide
animations. For more complete and complex slides functionalities, see
other tools such as Polylux.

See the
\href{https://github.com/typst/packages/raw/main/packages/preview/slydst/0.1.3/\#example}{preview}
below.

\subsection{Usage}\label{usage}

To start, just use the following preamble (only the title is required).

\begin{Shaded}
\begin{Highlighting}[]
\NormalTok{\#import "@preview/slydst:0.1.3": *}

\NormalTok{\#show: slides.with(}
\NormalTok{  title: "Slydst: Slides with Typst",}
\NormalTok{  subtitle: none,}
\NormalTok{  date: none,}
\NormalTok{  authors: ("Gaspard Lambrechts",),}
\NormalTok{  layout: "medium",}
\NormalTok{  ratio: 4/3,}
\NormalTok{  title{-}color: none,}
\NormalTok{)}

\NormalTok{Insert your content here.}
\end{Highlighting}
\end{Shaded}

Then, insert your content.

\begin{itemize}
\tightlist
\item
  \textbf{Level-one headings} corresponds to new sections.
\item
  \textbf{Level-two headings} corresponds to new slides.
\item
  Blank space can be filled with \textbf{vertical spaces} like
  \texttt{\ \#v(1fr)\ } .
\end{itemize}

\begin{Shaded}
\begin{Highlighting}[]
\NormalTok{== Outline}

\NormalTok{\#outline()}

\NormalTok{= First section}

\NormalTok{== First slide}

\NormalTok{\#figure(image("figure.png", width: 60\%), caption: "Caption")}

\NormalTok{\#v(1fr)}

\NormalTok{\#lorem(20)}
\end{Highlighting}
\end{Shaded}

\subsection{Title page}\label{title-page}

Alternatively, you can omit the title argument and write your own title
page. Note that the subtitle, date and authors arguments be ignored in
that case.

\begin{Shaded}
\begin{Highlighting}[]
\NormalTok{\#show: slides.with(}
\NormalTok{  layout: "medium",}
\NormalTok{)}

\NormalTok{\#align(center + horizon)[}
\NormalTok{  \#text(2em, default{-}color)[*Slydst: Slides in Typst*]}
\NormalTok{]}

\NormalTok{Insert your content here.}
\end{Highlighting}
\end{Shaded}

We advise the use of the \texttt{\ title-slide\ } function that ensures
a proper centering and no page numbering.

\begin{Shaded}
\begin{Highlighting}[]
\NormalTok{\#show: slides}

\NormalTok{\#title{-}slide(layout: "medium")[}
\NormalTok{  \#text(2em, default{-}color)[*Slydst: Slides in Typst*]}
\NormalTok{]}

\NormalTok{Insert your content here.}
\end{Highlighting}
\end{Shaded}

\subsection{Components}\label{components}

Definitions, theorems, lemmas, corollaries and algorithms boxes are also
available.

\begin{Shaded}
\begin{Highlighting}[]
\NormalTok{\#definition(title: "An interesting definition")[}
\NormalTok{  \#lorem(20)}
\NormalTok{]}
\end{Highlighting}
\end{Shaded}

\subsection{Documentation}\label{documentation}

\subsubsection{\texorpdfstring{\texttt{\ slides\ }}{ slides }}\label{slides}

\begin{itemize}
\tightlist
\item
  \texttt{\ content\ } : \texttt{\ content\ } - content of the
  presentation
\item
  \texttt{\ title\ } : \texttt{\ str\ } - title (required)
\item
  \texttt{\ subtitle\ } : \texttt{\ str\ } - subtitle
\item
  \texttt{\ date\ } : \texttt{\ str\ } - date
\item
  \texttt{\ authors\ } : \texttt{\ array\ } of \texttt{\ content\ } or
  \texttt{\ content\ } - list of authors or author content
\item
  \texttt{\ layout\ } :
  \texttt{\ str\ in\ ("small",\ "medium",\ "large")\ } - layout
  selection
\item
  \texttt{\ ratio\ } : \texttt{\ float\ } or \texttt{\ ratio\ } or
  \texttt{\ int\ } - width to height ratio
\item
  \texttt{\ title-color\ } : \texttt{\ color\ } or \texttt{\ gradient\ }
  - color of title and headings
\end{itemize}

\subsubsection{\texorpdfstring{\texttt{\ title-slide\ }}{ title-slide }}\label{title-slide}

\begin{itemize}
\tightlist
\item
  \texttt{\ content\ } : \texttt{\ content\ } - content of the slide
\end{itemize}

\subsubsection{\texorpdfstring{\texttt{\ definition\ } ,
\texttt{\ theorem\ } , \texttt{\ lemma\ } , \texttt{\ corollary\ } ,
\texttt{\ algorithm\ }}{ definition  ,  theorem  ,  lemma  ,  corollary  ,  algorithm }}\label{definition-theorem-lemma-corollary-algorithm}

\begin{itemize}
\tightlist
\item
  \texttt{\ content\ } : \texttt{\ content\ } - content of the block
\item
  \texttt{\ title\ } : \texttt{\ str\ } - title of the block
\item
  \texttt{\ fill-header\ } : \texttt{\ color\ } - color of the header
  (inferred if only \texttt{\ fill-body\ } is specified)
\item
  \texttt{\ fill-body\ } : \texttt{\ color\ } - color of the body
  (inferred if only \texttt{\ fill-header\ } is specified)
\item
  \texttt{\ radius\ } : \texttt{\ length\ } - radius of the corners of
  the block
\end{itemize}

\subsection{Example}\label{example}

{
\includesvg[width=3.125in,height=\textheight,keepaspectratio]{https://github.com/typst/packages/raw/main/packages/preview/slydst/0.1.3/svg/example-01.svg}
} {
\includesvg[width=3.125in,height=\textheight,keepaspectratio]{https://github.com/typst/packages/raw/main/packages/preview/slydst/0.1.3/svg/example-02.svg}
} {
\includesvg[width=3.125in,height=\textheight,keepaspectratio]{https://github.com/typst/packages/raw/main/packages/preview/slydst/0.1.3/svg/example-03.svg}
} {
\includesvg[width=3.125in,height=\textheight,keepaspectratio]{https://github.com/typst/packages/raw/main/packages/preview/slydst/0.1.3/svg/example-04.svg}
} {
\includesvg[width=3.125in,height=\textheight,keepaspectratio]{https://github.com/typst/packages/raw/main/packages/preview/slydst/0.1.3/svg/example-05.svg}
} {
\includesvg[width=3.125in,height=\textheight,keepaspectratio]{https://github.com/typst/packages/raw/main/packages/preview/slydst/0.1.3/svg/example-06.svg}
} {
\includesvg[width=3.125in,height=\textheight,keepaspectratio]{https://github.com/typst/packages/raw/main/packages/preview/slydst/0.1.3/svg/example-07.svg}
} {
\includesvg[width=3.125in,height=\textheight,keepaspectratio]{https://github.com/typst/packages/raw/main/packages/preview/slydst/0.1.3/svg/example-08.svg}
} {
\includesvg[width=3.125in,height=\textheight,keepaspectratio]{https://github.com/typst/packages/raw/main/packages/preview/slydst/0.1.3/svg/example-09.svg}
} {
\includesvg[width=3.125in,height=\textheight,keepaspectratio]{https://github.com/typst/packages/raw/main/packages/preview/slydst/0.1.3/svg/example-10.svg}
}

\href{/app?template=slydst&version=0.1.3}{Create project in app}

\subsubsection{How to use}\label{how-to-use}

Click the button above to create a new project using this template in
the Typst app.

You can also use the Typst CLI to start a new project on your computer
using this command:

\begin{verbatim}
typst init @preview/slydst:0.1.3
\end{verbatim}

\includesvg[width=0.16667in,height=0.16667in]{/assets/icons/16-copy.svg}

\subsubsection{About}\label{about}

\begin{description}
\tightlist
\item[Author :]
\href{https://github.com/glambrechts}{Gaspard Lambrechts}
\item[License:]
MIT
\item[Current version:]
0.1.3
\item[Last updated:]
November 12, 2024
\item[First released:]
November 18, 2023
\item[Minimum Typst version:]
0.12.0
\item[Archive size:]
4.12 kB
\href{https://packages.typst.org/preview/slydst-0.1.3.tar.gz}{\pandocbounded{\includesvg[keepaspectratio]{/assets/icons/16-download.svg}}}
\item[Repository:]
\href{https://github.com/glambrechts/slydst}{GitHub}
\item[Categor ies :]
\begin{itemize}
\tightlist
\item[]
\item
  \pandocbounded{\includesvg[keepaspectratio]{/assets/icons/16-presentation.svg}}
  \href{https://typst.app/universe/search/?category=presentation}{Presentation}
\item
  \pandocbounded{\includesvg[keepaspectratio]{/assets/icons/16-layout.svg}}
  \href{https://typst.app/universe/search/?category=layout}{Layout}
\item
  \pandocbounded{\includesvg[keepaspectratio]{/assets/icons/16-package.svg}}
  \href{https://typst.app/universe/search/?category=components}{Components}
\end{itemize}
\end{description}

\subsubsection{Where to report issues?}\label{where-to-report-issues}

This template is a project of Gaspard Lambrechts . Report issues on
\href{https://github.com/glambrechts/slydst}{their repository} . You can
also try to ask for help with this template on the
\href{https://forum.typst.app}{Forum} .

Please report this template to the Typst team using the
\href{https://typst.app/contact}{contact form} if you believe it is a
safety hazard or infringes upon your rights.

\phantomsection\label{versions}
\subsubsection{Version history}\label{version-history}

\begin{longtable}[]{@{}ll@{}}
\toprule\noalign{}
Version & Release Date \\
\midrule\noalign{}
\endhead
\bottomrule\noalign{}
\endlastfoot
0.1.3 & November 12, 2024 \\
\href{https://typst.app/universe/package/slydst/0.1.2/}{0.1.2} &
November 12, 2024 \\
\href{https://typst.app/universe/package/slydst/0.1.1/}{0.1.1} & March
18, 2024 \\
\href{https://typst.app/universe/package/slydst/0.1.0/}{0.1.0} &
November 18, 2023 \\
\end{longtable}

Typst GmbH did not create this template and cannot guarantee correct
functionality of this template or compatibility with any version of the
Typst compiler or app.


\section{Package List LaTeX/funarray.tex}
\title{typst.app/universe/package/funarray}

\phantomsection\label{banner}
\section{funarray}\label{funarray}

{ 0.4.0 }

Package providing convenient functional functions to use on arrays.

\phantomsection\label{readme}
This package provides some convinient functional functions for
\href{https://typst.app/}{typst} to use on arrays.

\subsection{Usage}\label{usage}

To use this package simply
\texttt{\ \#import\ "@preview/funarray:0.3.0"\ } . To import all
functions use \texttt{\ :\ *\ } and for specific ones, use either the
module or as described in the
\href{https://typst.app/docs/reference/scripting\#modules}{typst docs} .

\subsection{Important note}\label{important-note}

Almost all functions are one-liners, which could, instead of being
loaded via a package import, also be just copied directly into your
source files.

\subsection{Dokumentation}\label{dokumentation}

A prettier und easier to read version of the documentation exists in the
example folder, which is done in typst and exported to pdf. Otherwise,
bellow is the markdown version.

\subsection{Functions}\label{functions}

Let us define
\texttt{\ a\ =\ (1,\ "not\ prime",\ 2,\ "prime",\ 3,\ "prime",\ 4,\ "not\ prime",\ 5,\ "prime")\ }

\subsubsection{chunks}\label{chunks}

The chunks function translates the array to an array of array. It groups
the elements to chunks of a given size and collects them in an bigger
array.

\texttt{\ chunks(a,\ 2)\ =\ (\ (1,\ "not\ prime"),\ (2,\ "prime"),\ (3,\ "prime"),\ (4,\ "not\ prime"),\ (5,\ "prime")\ )\ }

\subsubsection{unzip}\label{unzip}

The unzip function is the inverse of the zip method, it transforms an
array of pairs to a pair of vectors. You can also give input an array of
\texttt{\ n\ } -tuples resulting in in \texttt{\ n\ } arrays.

\texttt{\ unzip(b)\ =\ (\ (1,\ 2,\ 3,\ 4,\ 5),\ (\ "not\ prime",\ "prime",\ "prime",\ "not\ prime",\ "prime"\ )\ )\ }

\subsubsection{cycle}\label{cycle}

The cycle function concatenates the array to itself until it has a given
size.

\begin{Shaded}
\begin{Highlighting}[]
\NormalTok{let c = cycle(range(5), 8)}
\NormalTok{c = (0, 1, 2, 3, 4, 0, 1, 2)}
\end{Highlighting}
\end{Shaded}

Note that there is also the functionality to concatenate with
\texttt{\ +\ } and \texttt{\ *\ } in typst.

\subsubsection{windows and
circular-windows}\label{windows-and-circular-windows}

This function provides a running window

\texttt{\ windows(c,\ 5)\ =\ (\ (0,\ 1,\ 2,\ 3,\ 4),\ (1,\ 2,\ 3,\ 4,\ 0),\ (2,\ 3,\ 4,\ 0,\ 1),\ (3,\ 4,\ 0,\ 1,\ 2)\ )\ }

whereas the circular version wraps over.

\texttt{\ circular-windows(c,\ 5)\ =\ (\ (0,\ 1,\ 2,\ 3,\ 4),\ (1,\ 2,\ 3,\ 4,\ 0),\ (2,\ 3,\ 4,\ 0,\ 1),\ (3,\ 4,\ 0,\ 1,\ 2),\ (4,\ 0,\ 1,\ 2,\ 4),\ (0,\ 1,\ 2,\ 4,\ 0),\ (1,\ 2,\ 4,\ 0,\ 1),\ (2,\ 4,\ 0,\ 1,\ 2)\ )\ }

\subsubsection{partition and
partition-map}\label{partition-and-partition-map}

The partition function seperates the array in two according to a
predicate function. The result is an array with all elements, where the
predicate returned true followed by an array with all elements, where
the predicate returned false.

\begin{Shaded}
\begin{Highlighting}[]
\NormalTok{let (primesp, nonprimesp) = partition(b, x =\textgreater{} x.at(1) == "prime")}
\NormalTok{primesp = ((2, "prime"), (3, "prime"), (5, "prime"))}
\NormalTok{nonprimesp = ((1, "not prime"), (4, "not prime"))}
\end{Highlighting}
\end{Shaded}

There is also a partition-map function, which after partition also
applies a second function on both collections.

\begin{Shaded}
\begin{Highlighting}[]
\NormalTok{let (primes, nonprimes) = partition{-}map(b, x =\textgreater{} x.at(1) == "prime", x =\textgreater{} x.at(0))}
\NormalTok{primes = (2, 3, 5)}
\NormalTok{nonprimes = (1, 4)}
\end{Highlighting}
\end{Shaded}

\subsubsection{group-by}\label{group-by}

This functions groups according to a predicate into maximally sized
chunks, where all elements have the same predicate value.

\begin{Shaded}
\begin{Highlighting}[]
\NormalTok{let f = (0,0,1,1,1,0,0,1)}
\NormalTok{let g = group{-}by(f, x =\textgreater{} x == 0)}
\NormalTok{g = ((0, 0), (1, 1, 1), (0, 0), (1,))}
\end{Highlighting}
\end{Shaded}

\subsubsection{flatten}\label{flatten}

Typst has a \texttt{\ flatten\ } method for arrays, however that method
acts recursively. For instance

\texttt{\ (((1,2,3),\ (2,3)),\ ((1,2,3),\ (1,2))).flatten()\ =\ (1,\ 2,\ 3,\ 2,\ 3,\ 1,\ 2,\ 3,\ 1,\ 2)\ }

Normally, one would only have flattened one level. To do this, we can
use the typst array concatenation method +, or by folding, the sum
method for arrays:

\texttt{\ (((1,2,3),\ (2,3)),\ ((1,2,3),\ (1,2))).sum()\ =\ ((1,\ 2,\ 3),\ (2,\ 3),\ (1,\ 2,\ 3),\ (1,\ 2))\ }

To handle further depth, one can use flatten again, so that in our
example:

\texttt{\ (((1,2,3),\ (2,3)),\ ((1,2,3),\ (1,2))).sum().sum()\ =\ (((1,2,3),\ (2,3)),\ ((1,2,3),\ (1,2))).flatten()\ }

\subsubsection{intersperse}\label{intersperse}

This function has been removed in version 0.3, as typst 0.8 provides
such functionality by default.

\subsubsection{take-while and
skip-while}\label{take-while-and-skip-while}

These functions do exactly as they say.

\begin{Shaded}
\begin{Highlighting}[]
\NormalTok{take{-}while(h, x =\textgreater{} x \textless{} 1) = (0, 0, 0.25, 0.5, 0.75)}
\NormalTok{skip{-}while(h, x =\textgreater{} x \textless{} 1) = (1, 1, 1, 0.25, 0.5, 0.75, 0, 0, 0.25, 0.5, 0.75, 1)}
\end{Highlighting}
\end{Shaded}

\subsection{Unsafe Functions}\label{unsafe-functions}

The core functions are defined in \texttt{\ funarray-unsafe.typ\ } .
However, assertions (error checking) are not there and it is generally
not being advised to use these directly. Still, if being cautious, one
can use the imported \texttt{\ funarray-unsafe\ } module in
\texttt{\ funarray(.typ)\ } . All function names are the same.

To do this from the package, do as follows:

\begin{verbatim}
#import @preview/funarray:0.3.0

#funarray.funarray-unsafe.chunks(range(10), 3)
\end{verbatim}

\subsubsection{How to add}\label{how-to-add}

Copy this into your project and use the import as \texttt{\ funarray\ }

\begin{verbatim}
#import "@preview/funarray:0.4.0"
\end{verbatim}

\includesvg[width=0.16667in,height=0.16667in]{/assets/icons/16-copy.svg}

Check the docs for
\href{https://typst.app/docs/reference/scripting/\#packages}{more
information on how to import packages} .

\subsubsection{About}\label{about}

\begin{description}
\tightlist
\item[Author :]
Ludwig Austermann
\item[License:]
MIT
\item[Current version:]
0.4.0
\item[Last updated:]
October 24, 2023
\item[First released:]
August 1, 2023
\item[Minimum Typst version:]
0.8.0
\item[Archive size:]
4.19 kB
\href{https://packages.typst.org/preview/funarray-0.4.0.tar.gz}{\pandocbounded{\includesvg[keepaspectratio]{/assets/icons/16-download.svg}}}
\item[Repository:]
\href{https://github.com/ludwig-austermann/typst-funarray}{GitHub}
\end{description}

\subsubsection{Where to report issues?}\label{where-to-report-issues}

This package is a project of Ludwig Austermann . Report issues on
\href{https://github.com/ludwig-austermann/typst-funarray}{their
repository} . You can also try to ask for help with this package on the
\href{https://forum.typst.app}{Forum} .

Please report this package to the Typst team using the
\href{https://typst.app/contact}{contact form} if you believe it is a
safety hazard or infringes upon your rights.

\phantomsection\label{versions}
\subsubsection{Version history}\label{version-history}

\begin{longtable}[]{@{}ll@{}}
\toprule\noalign{}
Version & Release Date \\
\midrule\noalign{}
\endhead
\bottomrule\noalign{}
\endlastfoot
0.4.0 & October 24, 2023 \\
\href{https://typst.app/universe/package/funarray/0.3.0/}{0.3.0} &
September 25, 2023 \\
\href{https://typst.app/universe/package/funarray/0.2.0/}{0.2.0} &
August 1, 2023 \\
\end{longtable}

Typst GmbH did not create this package and cannot guarantee correct
functionality of this package or compatibility with any version of the
Typst compiler or app.


\section{Package List LaTeX/gqe-lemoulon-presentation.tex}
\title{typst.app/universe/package/gqe-lemoulon-presentation}

\phantomsection\label{banner}
\phantomsection\label{template-thumbnail}
\pandocbounded{\includegraphics[keepaspectratio]{https://packages.typst.org/preview/thumbnails/gqe-lemoulon-presentation-0.0.4-small.webp}}

\section{gqe-lemoulon-presentation}\label{gqe-lemoulon-presentation}

{ 0.0.4 }

Quickly generate slides for a GQE-Le moulon presentation.

\href{/app?template=gqe-lemoulon-presentation&version=0.0.4}{Create
project in app}

\phantomsection\label{readme}
template \href{https://typst.app/}{Typst web app} to generate GQE slides

\subsection{ðŸ§`â€?ðŸ'» Usage}\label{uxf0uxffuxe2uxf0uxff-usage}

\begin{itemize}
\item
  Directly from \href{https://typst.app/}{Typst web app} by clicking
  “Start from template� on the dashboard and searching for
  \texttt{\ gqe-lemoulon-presentation\ } .
\item
  With CLI:
\end{itemize}

\begin{verbatim}
typst init @preview/gqe-lemoulon-presentation:{version}
\end{verbatim}

\subsection{Documentation}\label{documentation}

gqe-presentation is based on
\href{https://touying-typ.github.io/}{touying} package. The
documentation is available \href{https://touying-typ.github.io/}{here} .

\subsection{Local installation}\label{local-installation}

\subsubsection{Install Rust and Typst}\label{install-rust-and-typst}

\begin{verbatim}
curl --proto '=https' --tlsv1.2 -sSf https://sh.rustup.rs | sh
\end{verbatim}

and then install
\href{https://github.com/typst/typst\#installation}{Typst}

\begin{verbatim}
cargo install typst-cli
\end{verbatim}

\subsubsection{Install the “gqe-presentation� theme on
linux}\label{install-the-uxe2ux153gqe-presentationuxe2-theme-on-linux}

clone the repository in your file system and install the theme
“gqe-lemoulon-presentation� :

\begin{verbatim}
git clone https://forgemia.inra.fr/gqe-moulon/gqe-presentation.git
mkdir -p ~/.local/share/typst/packages/local/gqe-lemoulon-presentation/0.0.4/
cp -r gqe-presentation/* ~/.local/share/typst/packages/local/gqe-lemoulon-presentation/0.0.4/
\end{verbatim}

\subsubsection{Start a new document}\label{start-a-new-document}

\begin{verbatim}
#import "@local/gqe-lemoulon-presentation:0.0.4":*



#show: gqe-theme.with(
  aspect-ratio: "4-3",
  gqe-font: "PT Sans"
  // config-common(handout: true),
  config-info(
    title: [Full native timsTOF data parser implementation in the i2MassChroq software package],
    subtitle: [sous titre],
    author: [Olivier Langella],
    gqe-equipe: [Base],
  ),
)




#title-slide()


#slide()[
= Bioinformatics challenges
#pave("Scientific projects and hardware")[
- High throughput
- Metaproteomics
- Instrument improvements
]
#pause
#pave("Means")[
- Free software (as a speech)
- Finding new algorithms
- Upgrade existing ones
- Controlling infrastructure
- Controlling costs
]

]


#slide()[
= Yes but...
bla bla
]
\end{verbatim}

\subsection{ðŸ``? License}\label{uxf0uxff-license}

This is GPLv3 licensed.

\href{/app?template=gqe-lemoulon-presentation&version=0.0.4}{Create
project in app}

\subsubsection{How to use}\label{how-to-use}

Click the button above to create a new project using this template in
the Typst app.

You can also use the Typst CLI to start a new project on your computer
using this command:

\begin{verbatim}
typst init @preview/gqe-lemoulon-presentation:0.0.4
\end{verbatim}

\includesvg[width=0.16667in,height=0.16667in]{/assets/icons/16-copy.svg}

\subsubsection{About}\label{about}

\begin{description}
\tightlist
\item[Author :]
Olivier Langella
\item[License:]
GPL-3.0
\item[Current version:]
0.0.4
\item[Last updated:]
November 5, 2024
\item[First released:]
November 5, 2024
\item[Archive size:]
336 kB
\href{https://packages.typst.org/preview/gqe-lemoulon-presentation-0.0.4.tar.gz}{\pandocbounded{\includesvg[keepaspectratio]{/assets/icons/16-download.svg}}}
\item[Discipline s :]
\begin{itemize}
\tightlist
\item[]
\item
  \href{https://typst.app/universe/search/?discipline=biology}{Biology}
\item
  \href{https://typst.app/universe/search/?discipline=education}{Education}
\item
  \href{https://typst.app/universe/search/?discipline=agriculture}{Agriculture}
\end{itemize}
\item[Categor y :]
\begin{itemize}
\tightlist
\item[]
\item
  \pandocbounded{\includesvg[keepaspectratio]{/assets/icons/16-presentation.svg}}
  \href{https://typst.app/universe/search/?category=presentation}{Presentation}
\end{itemize}
\end{description}

\subsubsection{Where to report issues?}\label{where-to-report-issues}

This template is a project of Olivier Langella . You can also try to ask
for help with this template on the \href{https://forum.typst.app}{Forum}
.

Please report this template to the Typst team using the
\href{https://typst.app/contact}{contact form} if you believe it is a
safety hazard or infringes upon your rights.

\phantomsection\label{versions}
\subsubsection{Version history}\label{version-history}

\begin{longtable}[]{@{}ll@{}}
\toprule\noalign{}
Version & Release Date \\
\midrule\noalign{}
\endhead
\bottomrule\noalign{}
\endlastfoot
0.0.4 & November 5, 2024 \\
\end{longtable}

Typst GmbH did not create this template and cannot guarantee correct
functionality of this template or compatibility with any version of the
Typst compiler or app.


\section{Package List LaTeX/name-it.tex}
\title{typst.app/universe/package/name-it}

\phantomsection\label{banner}
\section{name-it}\label{name-it}

{ 0.1.2 }

Get the English names of integers.

\phantomsection\label{readme}
Get the English names of integers.

\subsection{Example}\label{example}

\pandocbounded{\includegraphics[keepaspectratio]{https://github.com/typst/packages/raw/main/packages/preview/name-it/0.1.2/example.png}}

\begin{Shaded}
\begin{Highlighting}[]
\NormalTok{\#import "@preview/name{-}it:0.1.0": name{-}it}

\NormalTok{\#set page(width: auto, height: auto, margin: 1cm)}

\NormalTok{{-} \#name{-}it({-}5)}
\NormalTok{{-} \#name{-}it({-}5, negative{-}prefix: "minus")}
\NormalTok{{-} \#name{-}it(0)}
\NormalTok{{-} \#name{-}it(1)}
\NormalTok{{-} \#name{-}it(10)}
\NormalTok{{-} \#name{-}it(11)}
\NormalTok{{-} \#name{-}it(42)}
\NormalTok{{-} \#name{-}it(100)}
\NormalTok{{-} \#name{-}it(110)}
\NormalTok{{-} \#name{-}it(1104)}
\NormalTok{{-} \#name{-}it(11040)}
\NormalTok{{-} \#name{-}it(11000)}
\NormalTok{{-} \#name{-}it(110000)}
\NormalTok{{-} \#name{-}it(1100004)}
\NormalTok{{-} \#name{-}it(10000000000006)}
\NormalTok{{-} \#name{-}it(10000000000006, show{-}and: false)}
\NormalTok{{-} \#name{-}it("200000000000000000000000007")}
\end{Highlighting}
\end{Shaded}

\subsection{Usage}\label{usage}

\subsubsection{\texorpdfstring{\texttt{\ name-it\ }}{ name-it }}\label{name-it-1}

Convert the given number into its English word representation.

\begin{Shaded}
\begin{Highlighting}[]
\NormalTok{\#let name{-}it(num, show{-}and: true, negative{-}prefix: "negative") = \{ .. \}}
\end{Highlighting}
\end{Shaded}

\textbf{Arguments:}

\begin{itemize}
\tightlist
\item
  \texttt{\ num\ } :
  \href{https://typst.app/docs/reference/foundations/int/}{\texttt{\ int\ }}
  ,
  \href{https://typst.app/docs/reference/foundations/str/}{\texttt{\ str\ }}
  â€'' The number to name.
\item
  \texttt{\ show-and\ } :
  \href{https://typst.app/docs/reference/foundations/bool/}{\texttt{\ bool\ }}
  â€'' Whether an “andâ€? should be used in certain places. For
  example, “one hundred ten� vs “one hundred and ten�.
\item
  \texttt{\ negative-prefix\ } :
  \href{https://typst.app/docs/reference/foundations/str/}{\texttt{\ str\ }}
  â€'' The prefix to use for negative numbers.
\end{itemize}

\subsubsection{How to add}\label{how-to-add}

Copy this into your project and use the import as \texttt{\ name-it\ }

\begin{verbatim}
#import "@preview/name-it:0.1.2"
\end{verbatim}

\includesvg[width=0.16667in,height=0.16667in]{/assets/icons/16-copy.svg}

Check the docs for
\href{https://typst.app/docs/reference/scripting/\#packages}{more
information on how to import packages} .

\subsubsection{About}\label{about}

\begin{description}
\tightlist
\item[Author :]
RubixDev
\item[License:]
GPL-3.0-only
\item[Current version:]
0.1.2
\item[Last updated:]
November 12, 2024
\item[First released:]
October 4, 2023
\item[Archive size:]
74.2 kB
\href{https://packages.typst.org/preview/name-it-0.1.2.tar.gz}{\pandocbounded{\includesvg[keepaspectratio]{/assets/icons/16-download.svg}}}
\item[Repository:]
\href{https://github.com/RubixDev/typst-name-it}{GitHub}
\end{description}

\subsubsection{Where to report issues?}\label{where-to-report-issues}

This package is a project of RubixDev . Report issues on
\href{https://github.com/RubixDev/typst-name-it}{their repository} . You
can also try to ask for help with this package on the
\href{https://forum.typst.app}{Forum} .

Please report this package to the Typst team using the
\href{https://typst.app/contact}{contact form} if you believe it is a
safety hazard or infringes upon your rights.

\phantomsection\label{versions}
\subsubsection{Version history}\label{version-history}

\begin{longtable}[]{@{}ll@{}}
\toprule\noalign{}
Version & Release Date \\
\midrule\noalign{}
\endhead
\bottomrule\noalign{}
\endlastfoot
0.1.2 & November 12, 2024 \\
\href{https://typst.app/universe/package/name-it/0.1.1/}{0.1.1} &
October 5, 2023 \\
\href{https://typst.app/universe/package/name-it/0.1.0/}{0.1.0} &
October 4, 2023 \\
\end{longtable}

Typst GmbH did not create this package and cannot guarantee correct
functionality of this package or compatibility with any version of the
Typst compiler or app.


\section{Package List LaTeX/vercanard.tex}
\title{typst.app/universe/package/vercanard}

\phantomsection\label{banner}
\phantomsection\label{template-thumbnail}
\pandocbounded{\includegraphics[keepaspectratio]{https://packages.typst.org/preview/thumbnails/vercanard-1.0.2-small.webp}}

\section{vercanard}\label{vercanard}

{ 1.0.2 }

A colorful CV template

\href{/app?template=vercanard&version=1.0.2}{Create project in app}

\phantomsection\label{readme}
A colorful resume template for Typst.

The
\href{https://github.com/typst/packages/raw/main/packages/preview/vercanard/1.0.2/template/main.typ}{demo}
file showcases what it is possible to do. You can see the result in
\href{https://github.com/typst/packages/raw/main/packages/preview/vercanard/1.0.2/demo.pdf}{the
corresponding PDF} .

First of all, copy the template to your Typst project, and import it.

\begin{Shaded}
\begin{Highlighting}[]
\NormalTok{\#import "@preview/vercanard:1.0.2": *}
\end{Highlighting}
\end{Shaded}

Then, call the \texttt{\ resume\ } in a global \texttt{\ show\ } rule
function to use it. This function takes a few arguments that we explain
in comments below:

\begin{Shaded}
\begin{Highlighting}[]
\NormalTok{\#show: resume.with(}
\NormalTok{  // The title of your resume, generally your name}
\NormalTok{  name: "Your name",}
\NormalTok{  // The subtitle, which is the position you are looking for most of the time}
\NormalTok{  title: "What you are looking for",}
\NormalTok{  // The accent color to use (here a vibrant yellow)}
\NormalTok{  accent{-}color: rgb("f3bc54"),}
\NormalTok{  // the margins (only used for top and left page margins actually,}
\NormalTok{  // but the other ones are proportional)}
\NormalTok{  margin: 2.6cm,}
\NormalTok{  // The content to put in the right aside block}
\NormalTok{  aside: [}
\NormalTok{    = Contact}

\NormalTok{    // lists in the aside are right aligned}
\NormalTok{    {-} \#link("mailto:example@example.org")}
\NormalTok{    {-} +33 6 66 66 66 66}
\NormalTok{  ]}
\NormalTok{)}

\NormalTok{// And finally the main body of your resume can come here}
\end{Highlighting}
\end{Shaded}

When writing the body, you can use level-1 headings as section titles,
and format an entry with the \texttt{\ entry\ } function (that takes
three content blocks as arguments, for title, description and details).

\begin{Shaded}
\begin{Highlighting}[]
\NormalTok{= Personal projects}

\NormalTok{\#entry[Vercanard][A resume template for Typst][2023 — Typst]}
\end{Highlighting}
\end{Shaded}

This template is under the GPLv3 licence, but resume built using it are
not considered binary derivatives, only output from another program, so
you can keep full copyright on them and chose not to licence them under
a free licence.

\href{/app?template=vercanard&version=1.0.2}{Create project in app}

\subsubsection{How to use}\label{how-to-use}

Click the button above to create a new project using this template in
the Typst app.

You can also use the Typst CLI to start a new project on your computer
using this command:

\begin{verbatim}
typst init @preview/vercanard:1.0.2
\end{verbatim}

\includesvg[width=0.16667in,height=0.16667in]{/assets/icons/16-copy.svg}

\subsubsection{About}\label{about}

\begin{description}
\tightlist
\item[Author :]
\href{https://ana.gelez.xyz}{Ana Gelez}
\item[License:]
GPL-3.0
\item[Current version:]
1.0.2
\item[Last updated:]
October 21, 2024
\item[First released:]
April 2, 2024
\item[Archive size:]
14.3 kB
\href{https://packages.typst.org/preview/vercanard-1.0.2.tar.gz}{\pandocbounded{\includesvg[keepaspectratio]{/assets/icons/16-download.svg}}}
\item[Repository:]
\href{https://github.com/elegaanz/vercanard}{GitHub}
\item[Categor y :]
\begin{itemize}
\tightlist
\item[]
\item
  \pandocbounded{\includesvg[keepaspectratio]{/assets/icons/16-user.svg}}
  \href{https://typst.app/universe/search/?category=cv}{CV}
\end{itemize}
\end{description}

\subsubsection{Where to report issues?}\label{where-to-report-issues}

This template is a project of Ana Gelez . Report issues on
\href{https://github.com/elegaanz/vercanard}{their repository} . You can
also try to ask for help with this template on the
\href{https://forum.typst.app}{Forum} .

Please report this template to the Typst team using the
\href{https://typst.app/contact}{contact form} if you believe it is a
safety hazard or infringes upon your rights.

\phantomsection\label{versions}
\subsubsection{Version history}\label{version-history}

\begin{longtable}[]{@{}ll@{}}
\toprule\noalign{}
Version & Release Date \\
\midrule\noalign{}
\endhead
\bottomrule\noalign{}
\endlastfoot
1.0.2 & October 21, 2024 \\
\href{https://typst.app/universe/package/vercanard/1.0.1/}{1.0.1} & May
23, 2024 \\
\href{https://typst.app/universe/package/vercanard/1.0.0/}{1.0.0} &
April 2, 2024 \\
\end{longtable}

Typst GmbH did not create this template and cannot guarantee correct
functionality of this template or compatibility with any version of the
Typst compiler or app.


\section{Package List LaTeX/sunny-famnit.tex}
\title{typst.app/universe/package/sunny-famnit}

\phantomsection\label{banner}
\phantomsection\label{template-thumbnail}
\pandocbounded{\includegraphics[keepaspectratio]{https://packages.typst.org/preview/thumbnails/sunny-famnit-0.2.0-small.webp}}

\section{sunny-famnit}\label{sunny-famnit}

{ 0.2.0 }

Thesis template for University of Primorska, FAMNIT

\href{/app?template=sunny-famnit&version=0.2.0}{Create project in app}

\phantomsection\label{readme}
\pandocbounded{\includegraphics[keepaspectratio]{https://img.shields.io/github/v/release/Tiggax/famnit_typst_template}}
\pandocbounded{\includegraphics[keepaspectratio]{https://img.shields.io/github/stars/Tiggax/famnit_typst_template}}

\pandocbounded{\includegraphics[keepaspectratio]{https://www.famnit.upr.si/img/UP_FAMNIT.png}}

\emph{University of Primorska,}

\emph{Faculty of Mathematics, Natural Sciences and Information
Technologies}

\begin{center}\rule{0.5\linewidth}{0.5pt}\end{center}

This is a Typst template for FAMNIT final work.

\begin{center}\rule{0.5\linewidth}{0.5pt}\end{center}

\subsection{configuration example}\label{configuration-example}

\begin{Shaded}
\begin{Highlighting}[]
\NormalTok{\#import "@preview/sunny{-}famnit:0.2.0": project}

\NormalTok{\#show project.with(}
\NormalTok{    date: datetime(day: 1, month: 1, year: 2024), // you could also do \textasciigrave{}datetime.today()\textasciigrave{}}
\NormalTok{    text\_lang: "en" // the language that the thesis is gonna be written in.}
    
\NormalTok{    author: "your name"}
\NormalTok{    studij: "your course",}
\NormalTok{    mentor: (}
\NormalTok{    name: "his name", }
\NormalTok{    en: ("prepends","postpends"), // you can prepend or postpend any titles}
\NormalTok{    sl: ("predstavki","postavki"),// you can prepend or postpend any titles}
\NormalTok{    ),}
\NormalTok{    somentor: none, // if you have a co{-}mentor write him here the same way as mentor, else you can just remove the line.}
\NormalTok{    work\_mentor: none, // if you have a work mentor, the same as above}

\NormalTok{    naslov: "your title in slovene",}
\NormalTok{    title: "your title",}

\NormalTok{    izvleček: [}
\NormalTok{        your abstract in slovene.}
\NormalTok{    ],}
\NormalTok{    abstract: [}
\NormalTok{        your abstract}
\NormalTok{    ],}

\NormalTok{    ključne\_besede: ("Typst", "je", "super!"),}
\NormalTok{    key\_words: ("Typst", "is", "Awesome!"),}

\NormalTok{    kratice: (}
\NormalTok{        "Famnit": "Fakulteta za matematiko naravoslovje in informacijske tehnologije",}
\NormalTok{        "PDF": "Portable document format",}
\NormalTok{    ),}

\NormalTok{    priloge: (), // you can add attachments as a dict of a title and content like \textasciigrave{}"name": [content],\textasciigrave{}}

\NormalTok{    zahvala: [}
\NormalTok{        you can add an acknowlegment.}
\NormalTok{    ],}

\NormalTok{  bib\_file: bibliography(}
\NormalTok{    "my\_references.bib",}
\NormalTok{    style: "ieee",}
\NormalTok{    title: [Bibliography],}
\NormalTok{  ),}

\NormalTok{    /* Additional content and their defaults}
\NormalTok{    kraj: "Koper",}
\NormalTok{    */}
\NormalTok{)}

\NormalTok{// Your content goes below.}
\end{Highlighting}
\end{Shaded}

\subsection{Abbreviations (kratice)}\label{abbreviations-kratice}

You can specify Abbreviations at the start as an attribute
\texttt{\ kratice\ } and pass it a dictionary of the abbriviation and
it’s explanation. Then you can reference them in text using
\texttt{\ @\textless{}short\ name\textgreater{}\ } to create a link to
it.

\subsection{Attachments}\label{attachments}

Some thesis need Attachments that are shown at the end of the file. To
add these attachments add them in your project under
\texttt{\ priloge\ } as a dictionary of the attachment name and its
content. I suggest having a seperate \texttt{\ attachments.typ\ } file,
from where you can reference them in the main project.

\subsection{Language}\label{language}

The writing of the thesis can be achieved in two languages; Slovene and
English. They have some differences between them in the way the template
is generated, as the thesis needs to be different for each one. you can
specify the language with the \texttt{\ text\_lang\ } attribute.

\begin{center}\rule{0.5\linewidth}{0.5pt}\end{center}

If you have any questions, suggestion or improvements open an issue or a
pull request
\href{https://github.com/Tiggax/famnit_typst_template}{here}

\href{/app?template=sunny-famnit&version=0.2.0}{Create project in app}

\subsubsection{How to use}\label{how-to-use}

Click the button above to create a new project using this template in
the Typst app.

You can also use the Typst CLI to start a new project on your computer
using this command:

\begin{verbatim}
typst init @preview/sunny-famnit:0.2.0
\end{verbatim}

\includesvg[width=0.16667in,height=0.16667in]{/assets/icons/16-copy.svg}

\subsubsection{About}\label{about}

\begin{description}
\tightlist
\item[Author :]
Tilen Gimpelj {[}@Tiggax{]}
\item[License:]
MIT
\item[Current version:]
0.2.0
\item[Last updated:]
July 19, 2024
\item[First released:]
March 18, 2024
\item[Minimum Typst version:]
0.11.0
\item[Archive size:]
5.38 kB
\href{https://packages.typst.org/preview/sunny-famnit-0.2.0.tar.gz}{\pandocbounded{\includesvg[keepaspectratio]{/assets/icons/16-download.svg}}}
\item[Repository:]
\href{https://github.com/Tiggax/famnit_typst_template}{GitHub}
\item[Discipline s :]
\begin{itemize}
\tightlist
\item[]
\item
  \href{https://typst.app/universe/search/?discipline=computer-science}{Computer
  Science}
\item
  \href{https://typst.app/universe/search/?discipline=biology}{Biology}
\item
  \href{https://typst.app/universe/search/?discipline=mathematics}{Mathematics}
\end{itemize}
\item[Categor y :]
\begin{itemize}
\tightlist
\item[]
\item
  \pandocbounded{\includesvg[keepaspectratio]{/assets/icons/16-mortarboard.svg}}
  \href{https://typst.app/universe/search/?category=thesis}{Thesis}
\end{itemize}
\end{description}

\subsubsection{Where to report issues?}\label{where-to-report-issues}

This template is a project of Tilen Gimpelj {[}@Tiggax{]} . Report
issues on \href{https://github.com/Tiggax/famnit_typst_template}{their
repository} . You can also try to ask for help with this template on the
\href{https://forum.typst.app}{Forum} .

Please report this template to the Typst team using the
\href{https://typst.app/contact}{contact form} if you believe it is a
safety hazard or infringes upon your rights.

\phantomsection\label{versions}
\subsubsection{Version history}\label{version-history}

\begin{longtable}[]{@{}ll@{}}
\toprule\noalign{}
Version & Release Date \\
\midrule\noalign{}
\endhead
\bottomrule\noalign{}
\endlastfoot
0.2.0 & July 19, 2024 \\
\href{https://typst.app/universe/package/sunny-famnit/0.1.0/}{0.1.0} &
March 18, 2024 \\
\end{longtable}

Typst GmbH did not create this template and cannot guarantee correct
functionality of this template or compatibility with any version of the
Typst compiler or app.


\section{Package List LaTeX/jurz.tex}
\title{typst.app/universe/package/jurz}

\phantomsection\label{banner}
\section{jurz}\label{jurz}

{ 0.1.0 }

Randziffern in Typst

\phantomsection\label{readme}
\href{https://de.wikipedia.org/w/index.php?title=Randnummer&oldid=231943223}{\emph{Randziffern}}
(also called \emph{Randnummern} ) are a way to reference text passages
in a document, independent of the page number or the section number.
They are used in many German legal texts, for example. This package
provides a way to create \emph{Randziffern} in Typst.

\subsection{Demo}\label{demo}

\begin{longtable}[]{@{}ll@{}}
\toprule\noalign{}
\endhead
\bottomrule\noalign{}
\endlastfoot
\pandocbounded{\includesvg[keepaspectratio]{https://github.com/typst/packages/raw/main/packages/preview/jurz/0.1.0/demo-2.svg}}
&
\pandocbounded{\includesvg[keepaspectratio]{https://github.com/typst/packages/raw/main/packages/preview/jurz/0.1.0/demo-3.svg}} \\
\end{longtable}

View source

\begin{Shaded}
\begin{Highlighting}[]
\NormalTok{\#show: init{-}jurz.with(}
\NormalTok{  gap: 1em,}
\NormalTok{  two{-}sided: true}
\NormalTok{)}

\NormalTok{\#rz \#lorem(50)}

\NormalTok{\#lorem(20)}

\NormalTok{\#rz\textless{}abc\textgreater{} \#lorem(30)}

\NormalTok{\#rz \#lorem(40)}

\NormalTok{\#rz \#lorem(50)}

\NormalTok{\#lorem(20)}

\NormalTok{\#rz \#lorem(24)}

\NormalTok{Fur further information, look at @abc.}
\end{Highlighting}
\end{Shaded}

\subsection{Reference}\label{reference}

\subsubsection{\texorpdfstring{\texttt{\ init-jurz\ }}{ init-jurz }}\label{init-jurz}

A show rule that initializes the \emph{Randziffern} for the document.
This rule should be placed at the beginning of the document. It also
allows customizing the behavior of the \emph{Randziffern} .

\paragraph{Usage}\label{usage}

\begin{Shaded}
\begin{Highlighting}[]
\NormalTok{\#show: init{-}jurz.with(}
\NormalTok{ // parameters}
\NormalTok{ // two{-}sided: true,}
\NormalTok{ // gap: 1em,}
\NormalTok{ // supplement: "Rz.",}
\NormalTok{ // reset{-}level: 0,}
\NormalTok{)}
\end{Highlighting}
\end{Shaded}

\paragraph{Parameters}\label{parameters}

\begin{itemize}
\tightlist
\item
  \texttt{\ two-sided\ } (optional): If \texttt{\ true\ } , the
  \emph{Randziffern} are placed on the outer margin of the page. If
  \texttt{\ false\ } , they are placed on the left margin. Default is
  \texttt{\ true\ } .
\item
  \texttt{\ gap\ } (optional): The distance between the
  \emph{Randziffer} and the text. Default is \texttt{\ 1em\ } .
\item
  \texttt{\ supplement\ } (optional): The text that is placed before the
  \emph{Randziffer} when referencing it. Default is \texttt{\ "Rz."\ } .
\item
  \texttt{\ reset-level\ } (optional): The heading level at which the
  \emph{Randziffern} are reset. If set to \texttt{\ 3\ } , for example,
  the numbering of the \emph{Randziffern} restarts after every heading
  of levels \texttt{\ 1\ } , \texttt{\ 2\ } , or \texttt{\ 3\ } .
  Default is \texttt{\ 0\ } .
\end{itemize}

\subsubsection{\texorpdfstring{\texttt{\ rz\ }}{ rz }}\label{rz}

Adds a \emph{Randziffer} to the text. The \emph{Randziffer} is a unique
identifier that can be referenced in the text.

You can add references the same way you can with headings. In fact, the
\emph{Randziffer} is treated as a heading of level \texttt{\ 99\ } under
the hood.

\paragraph{Usage}\label{usage-1}

\begin{Shaded}
\begin{Highlighting}[]
\NormalTok{\#rz \#lorem(100)}
\NormalTok{\#rz\textless{}abc\textgreater{} \#lorem(100)}

\NormalTok{See also @abc.}
\end{Highlighting}
\end{Shaded}

\subsection{License}\label{license}

This package is licensed under the MIT License.

\subsubsection{How to add}\label{how-to-add}

Copy this into your project and use the import as \texttt{\ jurz\ }

\begin{verbatim}
#import "@preview/jurz:0.1.0"
\end{verbatim}

\includesvg[width=0.16667in,height=0.16667in]{/assets/icons/16-copy.svg}

Check the docs for
\href{https://typst.app/docs/reference/scripting/\#packages}{more
information on how to import packages} .

\subsubsection{About}\label{about}

\begin{description}
\tightlist
\item[Author :]
\href{https://github.com/pklaschka}{Zuri Klaschka}
\item[License:]
MIT
\item[Current version:]
0.1.0
\item[Last updated:]
April 4, 2024
\item[First released:]
April 4, 2024
\item[Archive size:]
2.46 kB
\href{https://packages.typst.org/preview/jurz-0.1.0.tar.gz}{\pandocbounded{\includesvg[keepaspectratio]{/assets/icons/16-download.svg}}}
\item[Discipline :]
\begin{itemize}
\tightlist
\item[]
\item
  \href{https://typst.app/universe/search/?discipline=law}{Law}
\end{itemize}
\item[Categor ies :]
\begin{itemize}
\tightlist
\item[]
\item
  \pandocbounded{\includesvg[keepaspectratio]{/assets/icons/16-envelope.svg}}
  \href{https://typst.app/universe/search/?category=office}{Office}
\item
  \pandocbounded{\includesvg[keepaspectratio]{/assets/icons/16-package.svg}}
  \href{https://typst.app/universe/search/?category=components}{Components}
\item
  \pandocbounded{\includesvg[keepaspectratio]{/assets/icons/16-layout.svg}}
  \href{https://typst.app/universe/search/?category=layout}{Layout}
\end{itemize}
\end{description}

\subsubsection{Where to report issues?}\label{where-to-report-issues}

This package is a project of Zuri Klaschka . You can also try to ask for
help with this package on the \href{https://forum.typst.app}{Forum} .

Please report this package to the Typst team using the
\href{https://typst.app/contact}{contact form} if you believe it is a
safety hazard or infringes upon your rights.

\phantomsection\label{versions}
\subsubsection{Version history}\label{version-history}

\begin{longtable}[]{@{}ll@{}}
\toprule\noalign{}
Version & Release Date \\
\midrule\noalign{}
\endhead
\bottomrule\noalign{}
\endlastfoot
0.1.0 & April 4, 2024 \\
\end{longtable}

Typst GmbH did not create this package and cannot guarantee correct
functionality of this package or compatibility with any version of the
Typst compiler or app.


\section{Package List LaTeX/easy-pinyin.tex}
\title{typst.app/universe/package/easy-pinyin}

\phantomsection\label{banner}
\section{easy-pinyin}\label{easy-pinyin}

{ 0.1.0 }

Write Chinese pinyin easily.

\phantomsection\label{readme}
Write Chinese pinyin easily.

\subsection{Usage}\label{usage}

Import the package:

\begin{Shaded}
\begin{Highlighting}[]
\NormalTok{\#import "@preview/easy{-}pinyin:0.1.0": pinyin, zhuyin}
\end{Highlighting}
\end{Shaded}

With the \texttt{\ pinyin\ } function, you can use \texttt{\ a2\ } to
write an \texttt{\ É‘Ì?\ } , \texttt{\ o3\ } to write an \texttt{\ Ç’\ }
, \texttt{\ v4\ } to represent \texttt{\ ǜ\ } , etc.

With \texttt{\ zhuyin\ } function,you can put pinyin above the text
easily, with parameters:

\begin{itemize}
\tightlist
\item
  positional parameters:

  \begin{itemize}
  \tightlist
  \item
    \texttt{\ doc:\ content\textbar{}string\ } : main characters
  \item
    \texttt{\ ruby:\ content\textbar{}string\ } : zhuyin characters
  \end{itemize}
\item
  named parameters:

  \begin{itemize}
  \tightlist
  \item
    \texttt{\ scale:\ number\ =\ 0.7\ } : font size scale of
    \texttt{\ ruby\ } , default \texttt{\ 0.7\ }
  \item
    \texttt{\ gutter:\ length\ =\ 0.3em\ } : spacing between
    \texttt{\ doc\ } and \texttt{\ ruby\ } , default \texttt{\ 0.3em\ }
  \item
    \texttt{\ delimiter:\ string\textbar{}none\ =\ none\ } : if not
    none, use this character to split \texttt{\ doc\ } and
    \texttt{\ ruby\ } into parts
  \item
    \texttt{\ spacing:\ length\textbar{}none\ =\ none\ } : spacing
    between each parts
  \end{itemize}
\end{itemize}

See example bellow.

\subsection{Example}\label{example}

\begin{Shaded}
\begin{Highlighting}[]
\NormalTok{汉(\#pinyin[ha4n])语(\#pinyin[yu3])拼(\#pinyin[pi1n])音(\#pinyin[yi1n])。}

\NormalTok{\#let per{-}char(f) = [\#f(delimiter: "|")[汉|语|拼|音][ha4n|yu3|pi1n|yi1n]]}
\NormalTok{\#let per{-}word(f) = [\#f(delimiter: "|")[汉语|拼音][ha4nyu3|pi1nyi1n]]}
\NormalTok{\#let all{-}in{-}one(f) = [\#f[汉语拼音][ha4nyu3pi1nyi1n]]}
\NormalTok{\#let example(f) = (per{-}char(f), per{-}word(f), all{-}in{-}one(f))}

\NormalTok{// argument of scale and spacing}
\NormalTok{\#let arguments = ((0.5, none), (0.7, none), (0.7, 0.1em), (1.0, none), (1.0, 0.2em))}

\NormalTok{\#table(}
\NormalTok{  columns: (auto, auto, auto, auto),}
\NormalTok{  align: (center + horizon, center, center, center),}
\NormalTok{  [arguments], [per char], [per word], [all in one],}
\NormalTok{  ..arguments.map(((scale, spacing)) =\textgreater{} (}
\NormalTok{    text(size: 0.7em)[\#scale,\#repr(spacing)], }
\NormalTok{    ..example(zhuyin.with(scale: scale, spacing: spacing))}
\NormalTok{  )).flatten(),}
\NormalTok{)}
\end{Highlighting}
\end{Shaded}

\pandocbounded{\includegraphics[keepaspectratio]{https://raw.githubusercontent.com/7sDream/typst-easy-pinyin/master/example.png?raw=true}}

\subsection{LICENSE}\label{license}

MIT, see License file.

\subsubsection{How to add}\label{how-to-add}

Copy this into your project and use the import as
\texttt{\ easy-pinyin\ }

\begin{verbatim}
#import "@preview/easy-pinyin:0.1.0"
\end{verbatim}

\includesvg[width=0.16667in,height=0.16667in]{/assets/icons/16-copy.svg}

Check the docs for
\href{https://typst.app/docs/reference/scripting/\#packages}{more
information on how to import packages} .

\subsubsection{About}\label{about}

\begin{description}
\tightlist
\item[Author s :]
7sDream \& Other open-source contributors
\item[License:]
MIT
\item[Current version:]
0.1.0
\item[Last updated:]
July 6, 2023
\item[First released:]
July 6, 2023
\item[Archive size:]
2.43 kB
\href{https://packages.typst.org/preview/easy-pinyin-0.1.0.tar.gz}{\pandocbounded{\includesvg[keepaspectratio]{/assets/icons/16-download.svg}}}
\item[Repository:]
\href{https://github.com/7sDream/typst-easy-pinyin}{GitHub}
\end{description}

\subsubsection{Where to report issues?}\label{where-to-report-issues}

This package is a project of 7sDream and Other open-source contributors
. Report issues on
\href{https://github.com/7sDream/typst-easy-pinyin}{their repository} .
You can also try to ask for help with this package on the
\href{https://forum.typst.app}{Forum} .

Please report this package to the Typst team using the
\href{https://typst.app/contact}{contact form} if you believe it is a
safety hazard or infringes upon your rights.

\phantomsection\label{versions}
\subsubsection{Version history}\label{version-history}

\begin{longtable}[]{@{}ll@{}}
\toprule\noalign{}
Version & Release Date \\
\midrule\noalign{}
\endhead
\bottomrule\noalign{}
\endlastfoot
0.1.0 & July 6, 2023 \\
\end{longtable}

Typst GmbH did not create this package and cannot guarantee correct
functionality of this package or compatibility with any version of the
Typst compiler or app.


\section{Package List LaTeX/mcm-scaffold.tex}
\title{typst.app/universe/package/mcm-scaffold}

\phantomsection\label{banner}
\phantomsection\label{template-thumbnail}
\pandocbounded{\includegraphics[keepaspectratio]{https://packages.typst.org/preview/thumbnails/mcm-scaffold-0.1.0-small.webp}}

\section{mcm-scaffold}\label{mcm-scaffold}

{ 0.1.0 }

A Typst template for COMAP\textquotesingle s Mathematical Contest in
MCM/ICM

\href{/app?template=mcm-scaffold&version=0.1.0}{Create project in app}

\phantomsection\label{readme}
This is a Typst template for COMAP’s Mathematical Contest in MCM/ICM.

\subsection{Usage}\label{usage}

You can use this template in the Typst web app by clicking “Start from
template� on the dashboard and searching for \texttt{\ mcm-scaffold\ }
.

Alternatively, you can use the CLI to kick this project off using the
command

\begin{verbatim}
typst init @preview/mcm-scaffold
\end{verbatim}

Typst will create a new directory with all the files needed to get you
started.

\subsection{Configuration}\label{configuration}

This template exports the \texttt{\ mcm\ } function with the following
named arguments:

\begin{itemize}
\tightlist
\item
  \texttt{\ title\ } : The paper’s title as content.
\item
  \texttt{\ problem-chosen\ } : The problem your team have chosen.
\item
  \texttt{\ team-control-number\ } : Your team control number.
\item
  \texttt{\ year\ } : When did the competition took place.
\item
  \texttt{\ summary\ } : The content of a brief summary of the paper.
  Appears at the top of the first column in boldface.
\item
  \texttt{\ keywords\ } : Keywords of the paper.
\item
  \texttt{\ magic-leading\ } : adjust the leading of the summary.
\end{itemize}

The function also accepts a single, positional argument for the body of
the paper.

The template will initialize your package with a sample call to the
\texttt{\ mcm\ } function in a show rule. If you want to change an
existing project to use this template, you can add a show rule like this
at the top of your file:

\begin{Shaded}
\begin{Highlighting}[]
\NormalTok{\#import "@preview/mcm{-}scaffold:0.1.0": *}

\NormalTok{\#show: mcm.with(}
\NormalTok{  title: "A Simple Example for MCM/ICM Typst Template",}
\NormalTok{  problem{-}chosen: "ABCDEF",}
\NormalTok{  team{-}control{-}number: "1111111",}
\NormalTok{  year: "2025",}
\NormalTok{  summary: [}
\NormalTok{    \#lorem(100)}
    
\NormalTok{    \#lorem(100)}
    
\NormalTok{    \#lorem(100)}

\NormalTok{    \#lorem(100)}
\NormalTok{  ],}
\NormalTok{  keywords: [MCM; ICM; Mathemetical; template],}
\NormalTok{  magic{-}leading: 0.65em,}
\NormalTok{)}

\NormalTok{// Your content goes below.}
\end{Highlighting}
\end{Shaded}

\href{/app?template=mcm-scaffold&version=0.1.0}{Create project in app}

\subsubsection{How to use}\label{how-to-use}

Click the button above to create a new project using this template in
the Typst app.

You can also use the Typst CLI to start a new project on your computer
using this command:

\begin{verbatim}
typst init @preview/mcm-scaffold:0.1.0
\end{verbatim}

\includesvg[width=0.16667in,height=0.16667in]{/assets/icons/16-copy.svg}

\subsubsection{About}\label{about}

\begin{description}
\tightlist
\item[Author :]
\href{https://github.com/sxdl}{LuoQiu}
\item[License:]
Apache-2.0
\item[Current version:]
0.1.0
\item[Last updated:]
April 2, 2024
\item[First released:]
April 2, 2024
\item[Archive size:]
492 kB
\href{https://packages.typst.org/preview/mcm-scaffold-0.1.0.tar.gz}{\pandocbounded{\includesvg[keepaspectratio]{/assets/icons/16-download.svg}}}
\item[Repository:]
\href{https://github.com/sxdl/MCM-Typst-template}{GitHub}
\item[Discipline s :]
\begin{itemize}
\tightlist
\item[]
\item
  \href{https://typst.app/universe/search/?discipline=mathematics}{Mathematics}
\item
  \href{https://typst.app/universe/search/?discipline=computer-science}{Computer
  Science}
\end{itemize}
\item[Categor y :]
\begin{itemize}
\tightlist
\item[]
\item
  \pandocbounded{\includesvg[keepaspectratio]{/assets/icons/16-mortarboard.svg}}
  \href{https://typst.app/universe/search/?category=thesis}{Thesis}
\end{itemize}
\end{description}

\subsubsection{Where to report issues?}\label{where-to-report-issues}

This template is a project of LuoQiu . Report issues on
\href{https://github.com/sxdl/MCM-Typst-template}{their repository} .
You can also try to ask for help with this template on the
\href{https://forum.typst.app}{Forum} .

Please report this template to the Typst team using the
\href{https://typst.app/contact}{contact form} if you believe it is a
safety hazard or infringes upon your rights.

\phantomsection\label{versions}
\subsubsection{Version history}\label{version-history}

\begin{longtable}[]{@{}ll@{}}
\toprule\noalign{}
Version & Release Date \\
\midrule\noalign{}
\endhead
\bottomrule\noalign{}
\endlastfoot
0.1.0 & April 2, 2024 \\
\end{longtable}

Typst GmbH did not create this template and cannot guarantee correct
functionality of this template or compatibility with any version of the
Typst compiler or app.


\section{Package List LaTeX/classic-aau-report.tex}
\title{typst.app/universe/package/classic-aau-report}

\phantomsection\label{banner}
\phantomsection\label{template-thumbnail}
\pandocbounded{\includegraphics[keepaspectratio]{https://packages.typst.org/preview/thumbnails/classic-aau-report-0.1.0-small.webp}}

\section{classic-aau-report}\label{classic-aau-report}

{ 0.1.0 }

An example package.

\href{/app?template=classic-aau-report&version=0.1.0}{Create project in
app}

\phantomsection\label{readme}
Unofficial Typst template for project reports at Aalborg University
(AAU). This is based on the LaTeX template
\url{https://github.com/jkjaer/aauLatexTemplates} .

The template is generic to any field of study, but defaults to Computer
Science.

\subsection{Usage}\label{usage}

Click “Create project in app�.

Or via the CLI

\begin{Shaded}
\begin{Highlighting}[]
\ExtensionTok{typst}\NormalTok{ init @preview/classic{-}aau{-}report}
\end{Highlighting}
\end{Shaded}

\textbf{NOTE:} The template tries to use the
\texttt{\ Palatino\ Linotype\ } font, which is \emph{not} available in
Typst. It is available
\href{https://github.com/Tinggaard/classic-aau-report/tree/main/fonts}{here}

To use it in the \emph{web-app} , put the \texttt{\ .ttf\ } files
anywhere in the project tree.

To use it \emph{locally} specify the \texttt{\ -\/-font-path\ } flag (or
see the
\href{https://typst.app/docs/reference/text/text/\#parameters-font}{docs}
).

\subsection{Confugiration}\label{confugiration}

The \texttt{\ project\ } function takes the following (optional)
arguments:

\begin{itemize}
\item
  \texttt{\ meta\ } : Metadata about the project

  \begin{itemize}
  \tightlist
  \item
    \texttt{\ project-group\ } : The project group name
  \item
    \texttt{\ participants\ } : A list of participants
  \item
    \texttt{\ supervisors\ } : A list of supervisors
  \item
    \texttt{\ field-of-study\ } : The field of study
  \item
    \texttt{\ project-type\ } : The type of project
  \end{itemize}
\item
  \texttt{\ en\ } : English project info

  \begin{itemize}
  \tightlist
  \item
    \texttt{\ title\ } : The title of the project
  \item
    \texttt{\ theme\ } : The theme of the project
  \item
    \texttt{\ abstract\ } : The English abstract of the project
  \item
    \texttt{\ department\ } : The department name
  \item
    \texttt{\ department-url\ } : The department URL
  \end{itemize}
\item
  \texttt{\ dk\ } : Danish project info

  \begin{itemize}
  \tightlist
  \item
    \texttt{\ title\ } : The Danish title of the project
  \item
    \texttt{\ theme\ } : The theme of the project in Danish
  \item
    \texttt{\ abstract\ } : The Danish abstract of the project
  \item
    \texttt{\ department\ } : The department name in Danish
  \item
    \texttt{\ department-url\ } : The Danish department URL
  \end{itemize}
\end{itemize}

The defaults are as follows:

\begin{Shaded}
\begin{Highlighting}[]
\NormalTok{\#let defaults = (}
\NormalTok{  meta: (}
\NormalTok{    project{-}group: "No group name provided",}
\NormalTok{    participants: (),}
\NormalTok{    supervisors: (),}
\NormalTok{    field{-}of{-}study: "Computer Science",}
\NormalTok{    project{-}type: "Semester Project"}
\NormalTok{  ),}
\NormalTok{  en: (}
\NormalTok{    title: "Untitled",}
\NormalTok{    theme: "",}
\NormalTok{    abstract: [],}
\NormalTok{    department: "Department of Computer Science",}
\NormalTok{    department{-}url: "https://www.cs.aau.dk",}
\NormalTok{  ),}
\NormalTok{  dk: (}
\NormalTok{    title: "Uden titel",}
\NormalTok{    theme: "",}
\NormalTok{    abstract: [],}
\NormalTok{    department: "Institut for Datalogi",}
\NormalTok{    department{-}url: "https://www.dat.aau.dk",}
\NormalTok{  ),}
\NormalTok{)}
\end{Highlighting}
\end{Shaded}

Furthermore, the template exports the shawrules

\begin{itemize}
\tightlist
\item
  \texttt{\ frontmatter\ } : Sets the page numbering to arabic and
  chapter numbering to none
\item
  \texttt{\ mainmatter\ } : Sets the chapter numbering
  \texttt{\ Chapter\ } followed by a number.
\item
  \texttt{\ backmatter\ } : Sets the chapter numbering back to none
\item
  \texttt{\ appendix\ } : Sets the chapter numbering to
  \texttt{\ Appeendix\ } followed by a letter.
\end{itemize}

To use it in an existing project, add the following show rule to the top
of your file.

\begin{Shaded}
\begin{Highlighting}[]
\NormalTok{\#include "@preview/classic{-}aau{-}report:0.1.0": project, frontmatter, mainmatter, backmatter, appendix}

\NormalTok{// Any of the below can be omitted, the defaults are either empty values or CS specific}
\NormalTok{\#show: project.with(}
\NormalTok{  meta: (}
\NormalTok{    project{-}group: "CS{-}xx{-}DAT{-}y{-}zz",}
\NormalTok{    participants: (}
\NormalTok{      "Alice",}
\NormalTok{      "Bob",}
\NormalTok{      "Chad",}
\NormalTok{    ),}
\NormalTok{    supervisors: "John McClane"}
\NormalTok{  ),}
\NormalTok{  en: (}
\NormalTok{    title: "An awesome project",}
\NormalTok{    theme: "Writing a project in Typst",}
\NormalTok{    abstract: [],}
\NormalTok{  ),}
\NormalTok{  dk: (}
\NormalTok{    title: "Et fantastisk projekt",}
\NormalTok{    theme: "Et projekt i Typst",}
\NormalTok{    abstract: [],}
\NormalTok{  ),}
\NormalTok{)}

\NormalTok{// \#show{-}todos()}

\NormalTok{\#show: frontmatter}
\NormalTok{\#include "chapters/introduction.typ"}

\NormalTok{\#show: mainmatter}
\NormalTok{\#include "chapters/problem{-}analysis.typ"}
\NormalTok{\#include "chapters/conclusion.typ"}

\NormalTok{\#show: backmatter}
\NormalTok{\#bibliography("references.bib", title: "References")}

\NormalTok{\#show: appendix}
\NormalTok{\#include "appendices/code{-}snippets.typ"}
\end{Highlighting}
\end{Shaded}

\href{/app?template=classic-aau-report&version=0.1.0}{Create project in
app}

\subsubsection{How to use}\label{how-to-use}

Click the button above to create a new project using this template in
the Typst app.

You can also use the Typst CLI to start a new project on your computer
using this command:

\begin{verbatim}
typst init @preview/classic-aau-report:0.1.0
\end{verbatim}

\includesvg[width=0.16667in,height=0.16667in]{/assets/icons/16-copy.svg}

\subsubsection{About}\label{about}

\begin{description}
\tightlist
\item[Author :]
\href{https://github.com/Tinggaard}{Jens Tinggaard}
\item[License:]
MIT
\item[Current version:]
0.1.0
\item[Last updated:]
November 22, 2024
\item[First released:]
November 22, 2024
\item[Minimum Typst version:]
0.12.0
\item[Archive size:]
149 kB
\href{https://packages.typst.org/preview/classic-aau-report-0.1.0.tar.gz}{\pandocbounded{\includesvg[keepaspectratio]{/assets/icons/16-download.svg}}}
\item[Repository:]
\href{https://github.com/Tinggaard/classic-aau-report}{GitHub}
\item[Categor ies :]
\begin{itemize}
\tightlist
\item[]
\item
  \pandocbounded{\includesvg[keepaspectratio]{/assets/icons/16-speak.svg}}
  \href{https://typst.app/universe/search/?category=report}{Report}
\item
  \pandocbounded{\includesvg[keepaspectratio]{/assets/icons/16-mortarboard.svg}}
  \href{https://typst.app/universe/search/?category=thesis}{Thesis}
\end{itemize}
\end{description}

\subsubsection{Where to report issues?}\label{where-to-report-issues}

This template is a project of Jens Tinggaard . Report issues on
\href{https://github.com/Tinggaard/classic-aau-report}{their repository}
. You can also try to ask for help with this template on the
\href{https://forum.typst.app}{Forum} .

Please report this template to the Typst team using the
\href{https://typst.app/contact}{contact form} if you believe it is a
safety hazard or infringes upon your rights.

\phantomsection\label{versions}
\subsubsection{Version history}\label{version-history}

\begin{longtable}[]{@{}ll@{}}
\toprule\noalign{}
Version & Release Date \\
\midrule\noalign{}
\endhead
\bottomrule\noalign{}
\endlastfoot
0.1.0 & November 22, 2024 \\
\end{longtable}

Typst GmbH did not create this template and cannot guarantee correct
functionality of this template or compatibility with any version of the
Typst compiler or app.


\section{Package List LaTeX/codetastic.tex}
\title{typst.app/universe/package/codetastic}

\phantomsection\label{banner}
\section{codetastic}\label{codetastic}

{ 0.2.2 }

Generate all sorts of codes in Typst.

\phantomsection\label{readme}
\textbf{Codetastic} is a \href{https://github.com/typst/typst}{Typst}
package for drawing barcodes and 2d codes.

\subsection{Usage}\label{usage}

For Typst 0.6.0 or later, import the package from the Typst preview
repository:

\begin{Shaded}
\begin{Highlighting}[]
\NormalTok{\#import "@preview/codetastic:0.2.2"}
\end{Highlighting}
\end{Shaded}

After importing the package call any of the code generation functions:

\begin{Shaded}
\begin{Highlighting}[]
\NormalTok{\#import "@preview/codetastic:0.2.2": ean13, qrcode}

\NormalTok{\#ean13(4012345678901)}

\NormalTok{\#qrcode("https://github.com/typst/typst")}
\end{Highlighting}
\end{Shaded}

The output should look like this:
\pandocbounded{\includegraphics[keepaspectratio]{https://github.com/typst/packages/raw/main/packages/preview/codetastic/0.2.2/assets/example.png}}

\subsection{Further documentation}\label{further-documentation}

See \texttt{\ manual.pdf\ } for a full manual of the package.

\subsection{Development}\label{development}

The documentation is created using
\href{https://github.com/jneug/typst-mantys}{Mantys} , a Typst template
for creating package documentation.

To compile the manual, Mantys needs to be available as a local package.
Refer to Mantys’ manual for instructions on how to do so.

\subsection{Changelog}\label{changelog}

\subsubsection{Version 0.2.2}\label{version-0.2.2}

\begin{itemize}
\tightlist
\item
  qrcodes:

  \begin{itemize}
  \tightlist
  \item
    Fixed issue with alignment pattern placement.
  \item
    Removed minimal borders around modules for sharper edges with small
    module sizes.
  \end{itemize}
\end{itemize}

\subsubsection{Version 0.2.1}\label{version-0.2.1}

\begin{itemize}
\tightlist
\item
  qrcodes:

  \begin{itemize}
  \tightlist
  \item
    Fixed wrong sizing for \texttt{\ width\ } key.

    \begin{itemize}
    \tightlist
    \item
      The code didn’t take the quiet zone into account.
    \end{itemize}
  \item
    Moved debug information into quiet zone.
  \end{itemize}
\end{itemize}

\subsubsection{Version 0.2.0}\label{version-0.2.0}

\begin{itemize}
\tightlist
\item
  Removed CeTZ as a dependecy.

  \begin{itemize}
  \tightlist
  \item
    Now using native Typst drawing functions.
  \end{itemize}
\item
  Hugh speed improvements for large QR-Codes.
\item
  Fixed issue with checksum calculation for gtin/ean codes.
\end{itemize}

\subsubsection{Version 0.1.0}\label{version-0.1.0}

\begin{itemize}
\tightlist
\item
  Initial release submitted to
  \href{https://github.com/typst/packages}{typst/packages} .
\end{itemize}

\subsubsection{How to add}\label{how-to-add}

Copy this into your project and use the import as
\texttt{\ codetastic\ }

\begin{verbatim}
#import "@preview/codetastic:0.2.2"
\end{verbatim}

\includesvg[width=0.16667in,height=0.16667in]{/assets/icons/16-copy.svg}

Check the docs for
\href{https://typst.app/docs/reference/scripting/\#packages}{more
information on how to import packages} .

\subsubsection{About}\label{about}

\begin{description}
\tightlist
\item[Author :]
J. Neugebauer
\item[License:]
MIT
\item[Current version:]
0.2.2
\item[Last updated:]
September 23, 2023
\item[First released:]
September 12, 2023
\item[Archive size:]
22.9 kB
\href{https://packages.typst.org/preview/codetastic-0.2.2.tar.gz}{\pandocbounded{\includesvg[keepaspectratio]{/assets/icons/16-download.svg}}}
\item[Repository:]
\href{https://github.com/jneug/typst-codetastic}{GitHub}
\end{description}

\subsubsection{Where to report issues?}\label{where-to-report-issues}

This package is a project of J. Neugebauer . Report issues on
\href{https://github.com/jneug/typst-codetastic}{their repository} . You
can also try to ask for help with this package on the
\href{https://forum.typst.app}{Forum} .

Please report this package to the Typst team using the
\href{https://typst.app/contact}{contact form} if you believe it is a
safety hazard or infringes upon your rights.

\phantomsection\label{versions}
\subsubsection{Version history}\label{version-history}

\begin{longtable}[]{@{}ll@{}}
\toprule\noalign{}
Version & Release Date \\
\midrule\noalign{}
\endhead
\bottomrule\noalign{}
\endlastfoot
0.2.2 & September 23, 2023 \\
\href{https://typst.app/universe/package/codetastic/0.2.0/}{0.2.0} &
September 19, 2023 \\
\href{https://typst.app/universe/package/codetastic/0.1.0/}{0.1.0} &
September 12, 2023 \\
\end{longtable}

Typst GmbH did not create this package and cannot guarantee correct
functionality of this package or compatibility with any version of the
Typst compiler or app.


\section{Package List LaTeX/iridis.tex}
\title{typst.app/universe/package/iridis}

\phantomsection\label{banner}
\section{iridis}\label{iridis}

{ 0.1.0 }

A package to colors matching parenthesis

\phantomsection\label{readme}
Iridis is a package to colorize parenthesis in your typst document.
Iridis is a latin word that means “rainbow�. This package is
inspired by the many rainbow parenthesis plugins available for code
editors.

\subsection{Usage}\label{usage}

The package provides a single show-rule \texttt{\ iridis-show\ } that
can be used to colorize parenthesis in your document and a palette
\texttt{\ iridis-palette\ } that can be used to define the colors used.

The rule takes 3 arguments:

\begin{itemize}
\tightlist
\item
  \texttt{\ opening-parenthesis\ } : The opening parenthesis character.
  Default is \texttt{\ ("(",\ "{[}",\ "\{")\ } .
\item
  \texttt{\ closing-parenthesis\ } : The closing parenthesis character.
  Default is \texttt{\ (")",\ "{]}",\ "\}")\ } .
\item
  \texttt{\ palette\ } : The color palette to use. Default is
  \texttt{\ iridis-palette\ } .
\end{itemize}

If the symbols are single characters, they are interpreted as normal
strings but if you use multi-character strings, then they are
interpreted as regular expressions.

\subsection{Exemples}\label{exemples}

\begin{Shaded}
\begin{Highlighting}[]
\NormalTok{\#show: iridis.iridis{-}show}

\NormalTok{\textasciigrave{}\textasciigrave{}\textasciigrave{}rs}
\NormalTok{fn main() \{}
\NormalTok{    let n = false;}
\NormalTok{    if n \{}
\NormalTok{        println!("Hello, world!");}
\NormalTok{    \} else \{}
\NormalTok{        println!("Goodbye, world!");}
\NormalTok{    \}}
\NormalTok{\}}
\NormalTok{\textasciigrave{}\textasciigrave{}\textasciigrave{}}

\NormalTok{\textasciigrave{}\textasciigrave{}\textasciigrave{}cpp}
\NormalTok{\#include \textless{}iostream\textgreater{}}

\NormalTok{int main() \{}
\NormalTok{    bool n = false;}
\NormalTok{    if (n) \{}
\NormalTok{        std::cout \textless{}\textless{} "Hello, world!" \textless{}\textless{} std::endl;}
\NormalTok{    \} else \{}
\NormalTok{        std::cout \textless{}\textless{} "Goodbye, world!" \textless{}\textless{} std::endl;}
\NormalTok{    \}}
\NormalTok{\}}
\NormalTok{\textasciigrave{}\textasciigrave{}\textasciigrave{}}

\NormalTok{\textasciigrave{}\textasciigrave{}\textasciigrave{}py}
\NormalTok{if \_\_name\_\_ == "\_\_main\_\_":}
\NormalTok{    n = False}
\NormalTok{    if n:}
\NormalTok{        print("Hello, world!")}
\NormalTok{    else:}
\NormalTok{        print("Goodbye, world!")}
\NormalTok{\textasciigrave{}\textasciigrave{}\textasciigrave{}}
\end{Highlighting}
\end{Shaded}

\pandocbounded{\includegraphics[keepaspectratio]{https://raw.githubusercontent.com/Robotechnic/iridis/master/images/code1.png}}

\begin{Shaded}
\begin{Highlighting}[]
\NormalTok{\#show: iridis.iridis{-}show}

\NormalTok{$}
\NormalTok{    "plus" equiv lambda m. f lambda n. lambda f. lambda x. m f (n f x) \textbackslash{}}
\NormalTok{    "succ" equiv lambda n. lambda f. lambda x. f (n f x) \textbackslash{}}
\NormalTok{    "mult" equiv lambda m. lambda n. lambda f. lambda x. m (n f) x \textbackslash{}}
\NormalTok{    "pred" equiv lambda n. lambda f. lambda x. n (lambda g. lambda h. h (g f)) (lambda u. x) (lambda u. u) \textbackslash{}}
\NormalTok{    "zero" equiv lambda f. lambda x. x \textbackslash{}}
\NormalTok{    "one" equiv lambda f. lambda x. f x \textbackslash{}}
\NormalTok{    "two" equiv lambda f. lambda x. f (f x) \textbackslash{}}
\NormalTok{    "three" equiv lambda f. lambda x. f (f (f x)) \textbackslash{}}
\NormalTok{    "four" equiv lambda f. lambda x. f (f (f (f x))) \textbackslash{}}
\NormalTok{$}

\NormalTok{$}
\NormalTok{    (((1 + 5) * 7) / (5 {-} 8 * 7) + 3) * 2 approx 4.352941176}
\NormalTok{$}

\NormalTok{$ mat(}
\NormalTok{  1, 2, ..., (10 / 2);}
\NormalTok{  2, 2, ..., 10;}
\NormalTok{  dots.v, dots.v, dots.down, dots.v;}
\NormalTok{  10, (10 {-} (5 * 8)) / 2, ..., 10;}
\NormalTok{) $}
\end{Highlighting}
\end{Shaded}

\pandocbounded{\includegraphics[keepaspectratio]{https://raw.githubusercontent.com/Robotechnic/iridis/master/images/math1.png}}

\subsection{Changelog}\label{changelog}

\subsubsection{0.1.0}\label{section}

\begin{itemize}
\tightlist
\item
  Initial release
\end{itemize}

\subsubsection{How to add}\label{how-to-add}

Copy this into your project and use the import as \texttt{\ iridis\ }

\begin{verbatim}
#import "@preview/iridis:0.1.0"
\end{verbatim}

\includesvg[width=0.16667in,height=0.16667in]{/assets/icons/16-copy.svg}

Check the docs for
\href{https://typst.app/docs/reference/scripting/\#packages}{more
information on how to import packages} .

\subsubsection{About}\label{about}

\begin{description}
\tightlist
\item[Author :]
\href{https://github.com/Robotechnic}{Robotechnic}
\item[License:]
MIT
\item[Current version:]
0.1.0
\item[Last updated:]
June 24, 2024
\item[First released:]
June 24, 2024
\item[Minimum Typst version:]
0.11.0
\item[Archive size:]
3.17 kB
\href{https://packages.typst.org/preview/iridis-0.1.0.tar.gz}{\pandocbounded{\includesvg[keepaspectratio]{/assets/icons/16-download.svg}}}
\end{description}

\subsubsection{Where to report issues?}\label{where-to-report-issues}

This package is a project of Robotechnic . You can also try to ask for
help with this package on the \href{https://forum.typst.app}{Forum} .

Please report this package to the Typst team using the
\href{https://typst.app/contact}{contact form} if you believe it is a
safety hazard or infringes upon your rights.

\phantomsection\label{versions}
\subsubsection{Version history}\label{version-history}

\begin{longtable}[]{@{}ll@{}}
\toprule\noalign{}
Version & Release Date \\
\midrule\noalign{}
\endhead
\bottomrule\noalign{}
\endlastfoot
0.1.0 & June 24, 2024 \\
\end{longtable}

Typst GmbH did not create this package and cannot guarantee correct
functionality of this package or compatibility with any version of the
Typst compiler or app.


\section{Package List LaTeX/use-academicons.tex}
\title{typst.app/universe/package/use-academicons}

\phantomsection\label{banner}
\section{use-academicons}\label{use-academicons}

{ 0.1.0 }

A Typst library for Academicons the desktop fonts.

\phantomsection\label{readme}
A Typst library for Academicons through the desktop fonts.

This is based on the code from \texttt{\ duskmoon314\ } and the package
for
\href{https://github.com/duskmoon314/typst-fontawesome}{\textbf{typst-fontawesome}}
.

p.s. The library is based on the Academicons desktop fonts (v1.9.4)

\subsection{Usage}\label{usage}

\subsubsection{Install the fonts}\label{install-the-fonts}

You can download the fonts from the
\href{https://jpswalsh.github.io/academicons/}{official website}

After downloading the zip file, you can install the fonts depending on
your OS.

\paragraph{Typst web app}\label{typst-web-app}

You can simply upload the \texttt{\ ttf\ } files to the web app and use
them with this package.

\paragraph{Mac}\label{mac}

You can double click the \texttt{\ ttf\ } files to install them.

\paragraph{Windows}\label{windows}

You can right-click the \texttt{\ ttf\ } files and select
\texttt{\ Install\ } .

\subsubsection{Import the library}\label{import-the-library}

\paragraph{Using the typst packages}\label{using-the-typst-packages}

You can install the library using the typst packages:

\texttt{\ \#import\ "@preview/use-academicons:0.1.0":\ *\ }

\paragraph{Manually install}\label{manually-install}

Copy all files start with \texttt{\ lib\ } to your project and import
the library:

\texttt{\ \#import\ "lib.typ":\ *\ }

There are three files:

\begin{itemize}
\tightlist
\item
  \texttt{\ lib.typ\ } : The main entrypoint of the library.
\item
  \texttt{\ lib-impl.typ\ } : The implementation of \texttt{\ ai-icon\ }
  .
\item
  \texttt{\ lib-gen.typ\ } : The generated icon map and functions.
\end{itemize}

I recommend renaming these files to avoid conflicts with other
libraries.

\subsubsection{Use the icons}\label{use-the-icons}

You can use the \texttt{\ ai-icon\ } function to create an icon with its
name:

\texttt{\ \#ai-icon("lattes")\ }

Or you can use the \texttt{\ ai-\ } prefix to create an icon with its
name:

\texttt{\ \#ai-lattes()\ } (This is equivalent to
\texttt{\ \#ai-icon().with("lattes")\ } )

\paragraph{Full list of icons}\label{full-list-of-icons}

You can find all icons on the
\href{https://jpswalsh.github.io/academicons/}{official website}

\paragraph{Customization}\label{customization}

The \texttt{\ ai-icon\ } function passes args to \texttt{\ text\ } , so
you can customize the icon by passing parameters to it:

\texttt{\ \#ai-icon("lattes",\ fill:\ blue)\ }

\paragraph{Stacking icons}\label{stacking-icons}

The \texttt{\ ai-stack\ } function can be used to create stacked icons:

\texttt{\ \#ai-stack(ai-icon-args:\ (fill:\ black),\ "doi",\ ("cv",\ (fill:\ blue,\ size:\ 20pt)))\ }

Declaration is
\texttt{\ ai-stack(box-args:\ (:),\ grid-args:\ (:),\ ai-icon-args:\ (:),\ ..icons)\ }

\begin{itemize}
\tightlist
\item
  The order of the icons is from the bottom to the top.
\item
  \texttt{\ ai-icon-args\ } is used to set the default args for all
  icons.
\item
  You can also control the internal \texttt{\ box\ } and
  \texttt{\ grid\ } by passing the \texttt{\ box-args\ } and
  \texttt{\ grid-args\ } to the \texttt{\ ai-stack\ } function.
\item
  Currently, four types of icons are supported. The first three types
  leverage the \texttt{\ ai-icon\ } function, and the last type is just
  a content you want to put in the stack.

  \begin{itemize}
  \tightlist
  \item
    \texttt{\ str\ } , e.g., \texttt{\ "lattes"\ }
  \item
    \texttt{\ array\ } , e.g.,
    \texttt{\ ("lattes",\ (fill:\ white,\ size:\ 5.5pt))\ }
  \item
    \texttt{\ arguments\ } , e.g.
    \texttt{\ arguments("lattes",\ fill:\ white)\ }
  \item
    \texttt{\ content\ } , e.g. \texttt{\ ai-lattes(fill:\ white)\ }
  \end{itemize}
\end{itemize}

\subsection{Example}\label{example}

See the
\href{https://typst.app/project/rsgOFC4YkwpN7OqtRyiXP3}{\texttt{\ use-academicons.typ\ }}
file for a complete example.

\subsection{Contribution}\label{contribution}

Feel free to open an issue or a pull request if you find any problems or
have any suggestions.

\subsubsection{R helper}\label{r-helper}

The \texttt{\ helper.R\ } script is used to get unicodes for icons and
generate typst code.

\subsubsection{Repo structure}\label{repo-structure}

\begin{itemize}
\tightlist
\item
  \texttt{\ helper.R\ } : The helper script to get unicodes and generate
  typst code.
\item
  \texttt{\ lib.typ\ } : The main entrypoint of the library.
\item
  \texttt{\ lib-impl.typ\ } : The implementation of \texttt{\ ai-icon\ }
  .
\item
  \texttt{\ lib-gen.typ\ } : The generated functions of icons.
\item
  \texttt{\ example.typ\ } : An example file to show how to use the
  library.
\item
  \texttt{\ gallery.typ\ } : The generated gallery of icons. It is used
  in the example file.
\end{itemize}

\subsection{License}\label{license}

This library is licensed under the MIT license. Feel free to use it in
your project.

\subsubsection{How to add}\label{how-to-add}

Copy this into your project and use the import as
\texttt{\ use-academicons\ }

\begin{verbatim}
#import "@preview/use-academicons:0.1.0"
\end{verbatim}

\includesvg[width=0.16667in,height=0.16667in]{/assets/icons/16-copy.svg}

Check the docs for
\href{https://typst.app/docs/reference/scripting/\#packages}{more
information on how to import packages} .

\subsubsection{About}\label{about}

\begin{description}
\tightlist
\item[Author s :]
\href{mailto:kp.campbell.he@duskmoon314.com}{duskmoon (Campbell He)} \&
\href{mailto:philipp.kleer@posteo.com}{bpkleer (Philipp Kleer)}
\item[License:]
MIT
\item[Current version:]
0.1.0
\item[Last updated:]
August 8, 2024
\item[First released:]
August 8, 2024
\item[Archive size:]
5.41 kB
\href{https://packages.typst.org/preview/use-academicons-0.1.0.tar.gz}{\pandocbounded{\includesvg[keepaspectratio]{/assets/icons/16-download.svg}}}
\item[Repository:]
\href{https://github.com/bpkleer/typst-academicons}{GitHub}
\end{description}

\subsubsection{Where to report issues?}\label{where-to-report-issues}

This package is a project of duskmoon (Campbell He) and bpkleer (Philipp
Kleer) . Report issues on
\href{https://github.com/bpkleer/typst-academicons}{their repository} .
You can also try to ask for help with this package on the
\href{https://forum.typst.app}{Forum} .

Please report this package to the Typst team using the
\href{https://typst.app/contact}{contact form} if you believe it is a
safety hazard or infringes upon your rights.

\phantomsection\label{versions}
\subsubsection{Version history}\label{version-history}

\begin{longtable}[]{@{}ll@{}}
\toprule\noalign{}
Version & Release Date \\
\midrule\noalign{}
\endhead
\bottomrule\noalign{}
\endlastfoot
0.1.0 & August 8, 2024 \\
\end{longtable}

Typst GmbH did not create this package and cannot guarantee correct
functionality of this package or compatibility with any version of the
Typst compiler or app.


\section{Package List LaTeX/gviz.tex}
\title{typst.app/universe/package/gviz}

\phantomsection\label{banner}
\section{gviz}\label{gviz}

{ 0.1.0 }

Generate graphs using the graphviz dot language.

\phantomsection\label{readme}
GViz is a typst plugin that can render graphviz graphs.

It uses \url{https://codeberg.org/Sekoia/layout} as a backend, which
means it can currently only render to SVG, and mostly supports basic
features.

Import it like any other plugin:
\texttt{\ \#import\ "@preview/gviz:0.1.0":\ *\ } .

\subsection{Usage}\label{usage}

\begin{Shaded}
\begin{Highlighting}[]
\NormalTok{\#import "@preview/gviz:0.1.0": *}

\NormalTok{\#show raw.where(lang: "dot{-}render"): it =\textgreater{} render{-}image(it.text)}

\NormalTok{\textasciigrave{}\textasciigrave{}\textasciigrave{}dot{-}render}
\NormalTok{digraph mygraph \{}
\NormalTok{  node [shape=box];}
\NormalTok{  A {-}\textgreater{} B;}
\NormalTok{  B {-}\textgreater{} C;}
\NormalTok{  B {-}\textgreater{} D;}
\NormalTok{  C {-}\textgreater{} E;}
\NormalTok{  D {-}\textgreater{} E;}
\NormalTok{  E {-}\textgreater{} F;}
\NormalTok{  A {-}\textgreater{} F [label="one"];}
\NormalTok{  A {-}\textgreater{} F [label="two"];}
\NormalTok{  A {-}\textgreater{} F [label="three"];}
\NormalTok{  A {-}\textgreater{} F [label="four"];}
\NormalTok{  A {-}\textgreater{} F [label="five"];}
\NormalTok{\}\textasciigrave{}\textasciigrave{}\textasciigrave{}}

\NormalTok{\#let my{-}graph = "digraph \{A {-}\textgreater{} B\}"}
\NormalTok{\#render{-}image(my{-}graph)}

\NormalTok{SVG:}
\NormalTok{\#raw(render(my{-}graph), block: true, lang: "svg")}
\end{Highlighting}
\end{Shaded}

\subsection{API}\label{api}

\subsubsection{render}\label{render}

Renders a graph in dot language and returns SVG code for it.

Parameters:

\begin{itemize}
\tightlist
\item
  code (string, bytes): Dot language code to be rendered.
\end{itemize}

Returns: string

\subsubsection{render-image}\label{render-image}

Renders a graph in dot language and returns an SVG image of it. Uses the
same parameters as image.decode.

Parameters:

\begin{itemize}
\tightlist
\item
  code (string, bytes): Dot language code to be rendered.
\item
  width (auto, relative): The width of the image.
\item
  height (auto, relative): The height of the image.
\item
  alt (none, string): A text describing the image.
\item
  fit (string): How the image should adjust itself to a given area. See
  image.decode.
\end{itemize}

Returns: content

\subsubsection{How to add}\label{how-to-add}

Copy this into your project and use the import as \texttt{\ gviz\ }

\begin{verbatim}
#import "@preview/gviz:0.1.0"
\end{verbatim}

\includesvg[width=0.16667in,height=0.16667in]{/assets/icons/16-copy.svg}

Check the docs for
\href{https://typst.app/docs/reference/scripting/\#packages}{more
information on how to import packages} .

\subsubsection{About}\label{about}

\begin{description}
\tightlist
\item[Author :]
\href{https://codeberg.org/Sekoia\%3E\%20\%3Chttps://github.com/SekoiaTree}{Sekoia}
\item[License:]
Unlicense
\item[Current version:]
0.1.0
\item[Last updated:]
September 15, 2023
\item[First released:]
September 15, 2023
\item[Minimum Typst version:]
0.8.0
\item[Archive size:]
85.7 kB
\href{https://packages.typst.org/preview/gviz-0.1.0.tar.gz}{\pandocbounded{\includesvg[keepaspectratio]{/assets/icons/16-download.svg}}}
\item[Repository:]
\href{https://codeberg.org/Sekoia/gviz-typst}{Codeberg}
\end{description}

\subsubsection{Where to report issues?}\label{where-to-report-issues}

This package is a project of Sekoia . Report issues on
\href{https://codeberg.org/Sekoia/gviz-typst}{their repository} . You
can also try to ask for help with this package on the
\href{https://forum.typst.app}{Forum} .

Please report this package to the Typst team using the
\href{https://typst.app/contact}{contact form} if you believe it is a
safety hazard or infringes upon your rights.

\phantomsection\label{versions}
\subsubsection{Version history}\label{version-history}

\begin{longtable}[]{@{}ll@{}}
\toprule\noalign{}
Version & Release Date \\
\midrule\noalign{}
\endhead
\bottomrule\noalign{}
\endlastfoot
0.1.0 & September 15, 2023 \\
\end{longtable}

Typst GmbH did not create this package and cannot guarantee correct
functionality of this package or compatibility with any version of the
Typst compiler or app.


\section{Package List LaTeX/cmarker.tex}
\title{typst.app/universe/package/cmarker}

\phantomsection\label{banner}
\section{cmarker}\label{cmarker}

{ 0.1.1 }

Transpile CommonMark Markdown to Typst, from within Typst!

\phantomsection\label{readme}
\#set document(title: "cmarker.typ") \#set page(numbering: "1",
number-align: center) \#set text(lang: "en") \#align(center,
text(weight: 700, 1.75em){[}cmarker.typ{]}) \#set heading(numbering:
"1.") \#show link: c =\textgreater{} text(underline(c), fill: blue)
\#set image(height: 2em) \#show par: set block(above: 1.2em, below:
1.2em) \#align(center){[}https://github.com/SabrinaJewson/cmarker.typ{]}
\#"

This package enables you to write CommonMark Markdown, and import it
directly into Typst.

\begin{longtable}[]{@{}
  >{\raggedright\arraybackslash}p{(\linewidth - 2\tabcolsep) * \real{0.5000}}
  >{\raggedright\arraybackslash}p{(\linewidth - 2\tabcolsep) * \real{0.5000}}@{}}
\toprule\noalign{}
\endhead
\bottomrule\noalign{}
\endlastfoot
\texttt{\ simple.typ\ } & \texttt{\ simple.md\ } \\
\begin{minipage}[t]{\linewidth}\raggedright
\begin{Shaded}
\begin{Highlighting}[]
\NormalTok{\#import "@preview/cmarker:0.1.1"}

\NormalTok{\#cmarker.render(read("simple.md"))}
\end{Highlighting}
\end{Shaded}
\end{minipage} & \begin{minipage}[t]{\linewidth}\raggedright
\begin{Shaded}
\begin{Highlighting}[]
\FunctionTok{\# We can write Markdown!}

\NormalTok{*Using* \_\_lots\_\_ \textasciitilde{}of\textasciitilde{} }\InformationTok{\textasciigrave{}fancy\textasciigrave{}} \CommentTok{[}\OtherTok{features}\CommentTok{](https://example.org/)}\NormalTok{.}
\end{Highlighting}
\end{Shaded}
\end{minipage} \\
\end{longtable}

\begin{longtable}[]{@{}l@{}}
\toprule\noalign{}
\endhead
\bottomrule\noalign{}
\endlastfoot
\texttt{\ simple.pdf\ } \\
\pandocbounded{\includegraphics[keepaspectratio]{https://github.com/typst/packages/raw/main/packages/preview/cmarker/0.1.1/examples/simple.png}} \\
\end{longtable}

This document is available in
\href{https://github.com/SabrinaJewson/cmarker.typ/tree/main\#cmarker}{Markdown}
and
\href{https://github.com/SabrinaJewson/cmarker.typ/blob/main/README.pdf}{rendered
PDF} formats.

\subsection{API}\label{api}

We offer a single function:

\begin{Shaded}
\begin{Highlighting}[]
\NormalTok{render(}
\NormalTok{  markdown,}
\NormalTok{  smart{-}punctuation: true,}
\NormalTok{  blockquote: none,}
\NormalTok{  math: none,}
\NormalTok{  h1{-}level: 1,}
\NormalTok{  raw{-}typst: true,}
\NormalTok{  scope: (:),}
\NormalTok{  show{-}source: false,}
\NormalTok{) {-}\textgreater{} content}
\end{Highlighting}
\end{Shaded}

The parameters are as follows:

\begin{itemize}
\item
  \texttt{\ markdown\ } : The
  \href{https://spec.commonmark.org/0.30/}{CommonMark} Markdown string
  to be processed. Parsed with the
  \href{https://docs.rs/pulldown-cmark}{pulldown-cmark} Rust library.
  You can set this to \texttt{\ read("somefile.md")\ } to import an
  external markdown file; see the
  \href{https://typst.app/docs/reference/data-loading/read/}{documentation
  for the read function} .

  \begin{itemize}
  \tightlist
  \item
    Accepted values: Strings and raw text code blocks.
  \item
    Required.
  \end{itemize}
\item
  \texttt{\ smart-punctuation\ } : Automatically convert ASCII
  punctuation to Unicode equivalents:

  \begin{itemize}
  \tightlist
  \item
    nondirectional quotations (" and \textquotesingle) become
    directional (“� and ‘’);
  \item
    three consecutive full stops (…) become ellipses (…);
  \item
    two and three consecutive hypen-minus signs (-\/- and â€'') become
    en and em dashes (â€`` and â€'').
  \end{itemize}

  Note that although Typst also offers this functionality, this
  conversion is done through the Markdown parser rather than Typst.

  \begin{itemize}
  \tightlist
  \item
    Accepted values: Booleans.
  \item
    Default value: \texttt{\ true\ } .
  \end{itemize}
\item
  \texttt{\ blockquote\ } : A callback to be used when a blockquote is
  encountered in the Markdown, or \texttt{\ none\ } if blockquotes
  should be treated as normal text. Because Typst does not support
  blockquotes natively, the user must configure this.

  \begin{itemize}
  \tightlist
  \item
    Accepted values: Functions accepting content and returning content,
    or \texttt{\ none\ } .
  \item
    Default value: \texttt{\ none\ } .
  \end{itemize}

  For example, to display a black border to the left of the text one can
  use:

\begin{Shaded}
\begin{Highlighting}[]
\NormalTok{box.with(stroke: (left: 1pt + black), inset: (left: 5pt, y: 6pt))}
\end{Highlighting}
\end{Shaded}
\item
  \texttt{\ math\ } : A callback to be used when equations are
  encountered in the Markdown, or \texttt{\ none\ } if it should be
  treated as normal text. Because Typst does not support LaTeX equations
  natively, the user must configure this.

  \begin{itemize}
  \tightlist
  \item
    Accepted values: Functions that take a boolean argument named
    \texttt{\ block\ } and a positional string argument (often, the
    \texttt{\ mitex\ } function from
    \href{https://typst.app/universe/package/mitex}{the mitex package}
    ), or \texttt{\ none\ } .
  \item
    Default value: \texttt{\ none\ } .
  \end{itemize}

  For example, to render math equation as a Typst math block, one can
  use:

\begin{Shaded}
\begin{Highlighting}[]
\NormalTok{\#import "@preview/mitex:0.2.4": mitex}
\NormalTok{\#cmarker.render(\textasciigrave{}$\textbackslash{}int\_1\^{}2 x \textbackslash{}mathrm\{d\} x$\textasciigrave{}, math: mitex)}
\end{Highlighting}
\end{Shaded}
\item
  \texttt{\ h1-level\ } : The level that top-level headings in Markdown
  should get in Typst. When set to zero, top-level headings are treated
  as text, \texttt{\ \#\#\ } headings become \texttt{\ =\ } headings,
  \texttt{\ \#\#\#\ } headings become \texttt{\ ==\ } headings, et
  cetera; when set to \texttt{\ 2\ } , \texttt{\ \#\ } headings become
  \texttt{\ ==\ } headings, \texttt{\ \#\#\ } headings become
  \texttt{\ ===\ } headings, et cetera.

  \begin{itemize}
  \tightlist
  \item
    Accepted values: Integers in the range {[}0, 255{]}.
  \item
    Default value: 1.
  \end{itemize}
\item
  \texttt{\ raw-typst\ } : Whether to allow raw Typst code to be
  injected into the document via HTML comments. If disabled, the
  comments will act as regular HTML comments.

  \begin{itemize}
  \tightlist
  \item
    Accepted values: Booleans.
  \item
    Default value: \texttt{\ true\ } .
  \end{itemize}

  For example, when this is enabled,
  \texttt{\ \textless{}!-\/-raw-typst\ \#circle(radius:\ 10pt)\ -\/-\textgreater{}\ }
  will result in a circle in the document (but only when rendered
  through Typst). See also
  \texttt{\ \textless{}!-\/-typst-begin-exclude-\/-\textgreater{}\ } and
  \texttt{\ \textless{}!-\/-typst-end-exclude-\/-\textgreater{}\ } ,
  which is the inverse of this.
\item
  \texttt{\ scope\ } : When \texttt{\ raw-typst\ } is enabled, this is a
  dictionary providing the context in which the evaluated Typst code
  runs. It is useful to pass values in to code inside
  \texttt{\ \textless{}!-\/-raw-typst-\/-\textgreater{}\ } blocks.

  \begin{itemize}
  \tightlist
  \item
    Accepted values: Any dictionary.
  \item
    Default value: \texttt{\ (:)\ } .
  \end{itemize}
\item
  \texttt{\ show-source\ } : A debugging tool. When set to
  \texttt{\ true\ } , the Typst code that would otherwise have been
  displayed will be instead rendered in a code block.

  \begin{itemize}
  \tightlist
  \item
    Accepted values: Booleans.
  \item
    Default value: \texttt{\ false\ } .
  \end{itemize}
\end{itemize}

This function returns the rendered \texttt{\ content\ } .

\subsection{Supported Markdown Syntax}\label{supported-markdown-syntax}

We support CommonMark with a couple extensions.

\begin{itemize}
\item
  Paragraph breaks: Two newlines, i.e. one blank line.
\item
  Hard line breaks (used more in poetry than prose): Put two spaces at
  the end of the line.
\item
  \texttt{\ *emphasis*\ } or \texttt{\ \_emphasis\_\ } : \emph{emphasis}
\item
  \texttt{\ **strong**\ } or \texttt{\ \_\_strong\_\_\ } :
  \textbf{strong}
\item
  \texttt{\ \textasciitilde{}strikethrough\textasciitilde{}\ } :
  \textasciitilde strikethrough\textasciitilde{}
\item
  \texttt{\ {[}links{]}(https://example.org)\ } :
  \href{https://example.org/}{links}
\item
  \texttt{\ \#\#\#\ Headings\ } , where \texttt{\ \#\ } is a top-level
  heading, \texttt{\ \#\#\ } a subheading, \texttt{\ \#\#\#\ } a
  sub-subheading, etc
\item
  \texttt{\ \textasciigrave{}inline\ code\ blocks\textasciigrave{}\ } :
  \texttt{\ inline\ code\ blocks\ }
\item
\begin{verbatim}
```
out of line code blocks
```
\end{verbatim}

  Syntax highlighting can be achieved by specifying a language after the
  opening backticks:

\begin{verbatim}
```rust
let x = 5;
```
\end{verbatim}

  giving:

\begin{Shaded}
\begin{Highlighting}[]
\KeywordTok{let}\NormalTok{ x }\OperatorTok{=} \DecValTok{5}\OperatorTok{;}
\end{Highlighting}
\end{Shaded}
\item
  \texttt{\ -\/-\/-\ } , making a horizontal rule:
\end{itemize}

\begin{center}\rule{0.5\linewidth}{0.5pt}\end{center}

\begin{itemize}
\item
\begin{Shaded}
\begin{Highlighting}[]
\SpecialStringTok{{-} }\NormalTok{Unordered}
\SpecialStringTok{{-} }\NormalTok{lists}
\end{Highlighting}
\end{Shaded}

  \begin{itemize}
  \tightlist
  \item
    Unordered
  \item
    Lists
  \end{itemize}
\item
\begin{Shaded}
\begin{Highlighting}[]
\SpecialStringTok{1. }\NormalTok{Ordered}
\SpecialStringTok{1. }\NormalTok{Lists}
\end{Highlighting}
\end{Shaded}

  \begin{enumerate}
  \tightlist
  \item
    Ordered
  \item
    Lists
  \end{enumerate}
\item
  \texttt{\ \$x\ +\ y\$\ } or \texttt{\ \$\$x\ +\ y\$\$\ } : math
  equations, if the \texttt{\ math\ } parameter is set.
\item
  \texttt{\ \textgreater{}\ blockquotes\ } , if the
  \texttt{\ blockquote\ } parameter is set.
\item
  Images:
  \texttt{\ !{[}Some\ tiled\ hexagons{]}(examples/hexagons.png)\ } ,
  giving
  \pandocbounded{\includegraphics[keepaspectratio]{https://github.com/typst/packages/raw/main/packages/preview/cmarker/0.1.1/examples/hexagons.png}}
\end{itemize}

\subsection{Interleaving Markdown and
Typst}\label{interleaving-markdown-and-typst}

Sometimes, you might want to render a certain section of the document
only when viewed as Markdown, or only when viewed through Typst. To
achieve the former, you can simply wrap the section in
\texttt{\ \textless{}!-\/-typst-begin-exclude-\/-\textgreater{}\ } and
\texttt{\ \textless{}!-\/-typst-end-exclude-\/-\textgreater{}\ } :

\begin{Shaded}
\begin{Highlighting}[]
\NormalTok{Hello from not Typst!}
\end{Highlighting}
\end{Shaded}

Most Markdown parsers support HTML comments, so from their perspective
this is no different to just writing out the Markdown directly; but
\texttt{\ cmarker.typ\ } knows to search for those comments and avoid
rendering the content in between.

Note that when the opening comment is followed by the end of an element,
\texttt{\ cmarker.typ\ } will close the block for you. For example:

\begin{Shaded}
\begin{Highlighting}[]
\AttributeTok{\textgreater{} }
\AttributeTok{\textgreater{} One}

\NormalTok{Two}
\end{Highlighting}
\end{Shaded}

In this code, “Two� will be given no matter where the document is
rendered. This is done to prevent us from generating invalid Typst code.

Conversely, one can put Typst code inside a HTML comment of the form
\texttt{\ \textless{}!-\/-raw-typst\ {[}…{]}-\/-\textgreater{}\ } to
have it evaluated directly as Typst code (but only if the
\texttt{\ raw-typst\ } option to \texttt{\ render\ } is set to
\texttt{\ true\ } , otherwise it will just be seen as a regular comment
and removed):

\begin{Shaded}
\begin{Highlighting}[]

\end{Highlighting}
\end{Shaded}

\subsection{Markdownâ€``Typst
Polyglots}\label{markdownuxe2typst-polyglots}

This project has a manual as a PDF and a README as a Markdown document,
but by the power of this library they are in fact the same thing!
Furthermore, one can read the \texttt{\ README.md\ } in a markdown
viewer and it will display correctly, but one can \emph{also} run
\texttt{\ typst\ compile\ README.md\ } to generate the Typst-typeset
\texttt{\ README.pdf\ } .

How does this work? We just have to be clever about how we write the
README:

\begin{Shaded}
\begin{Highlighting}[]
\NormalTok{(Typst preamble content)}
\NormalTok{\#"}


\NormalTok{Regular Markdown goes here…}

\end{Highlighting}
\end{Shaded}

The same code but syntax-highlighted as Typst code helps to illuminate
it:

\begin{Shaded}
\begin{Highlighting}[]
\NormalTok{\textless{}picture\textgreater{}}
\NormalTok{(Typst preamble content)}
\NormalTok{\#"\textless{}/picture\textgreater{}}
\NormalTok{\textless{}!{-}{-}".slice(0,0)}
\NormalTok{\#import "@preview/cmarker:0.1.1"}
\NormalTok{\#let markdown = read("README.md")}
\NormalTok{\#markdown = markdown.slice(markdown.position("\textless{}/picture\textgreater{}") + "\textless{}/picture\textgreater{}".len())}
\NormalTok{\#cmarker.render(markdown, h1{-}level: 0)}
\NormalTok{/*{-}{-}\textgreater{}}

\NormalTok{Regular Markdown goes here…}

\NormalTok{\textless{}!{-}{-}*///{-}{-}\textgreater{}}
\end{Highlighting}
\end{Shaded}

\subsection{Limitations}\label{limitations}

\begin{itemize}
\tightlist
\item
  We do not currently support HTML tags, and they will be stripped from
  the output; for example, GitHub supports writing
  \texttt{\ \textless{}sub\textgreater{}text\textless{}/sub\textgreater{}\ }
  to get subscript text, but we will render that as simply “text�.
  In future it would be nice to support a subset of HTML tags.
\item
  We do not currently support Markdown tables and footnotes. In future
  it would be good to.
\item
  Although I tried my best to escape everything correctly, I won’t
  provide a hard guarantee that everything is fully sandboxed even if
  you set \texttt{\ raw-typst:\ false\ } . That said, Typst itself is
  well-sandboxed anyway.
\end{itemize}

\subsection{Development}\label{development}

\begin{itemize}
\tightlist
\item
  Build the plugin with \texttt{\ ./build.sh\ } , which produces the
  \texttt{\ plugin.wasm\ } necessary to use this.
\item
  Compile examples with
  \texttt{\ typst\ compile\ examples/\{name\}.typ\ -\/-root\ .\ } .
\item
  Compile this README to PDF with \texttt{\ typst\ compile\ README.md\ }
  .
\end{itemize}

\subsubsection{How to add}\label{how-to-add}

Copy this into your project and use the import as \texttt{\ cmarker\ }

\begin{verbatim}
#import "@preview/cmarker:0.1.1"
\end{verbatim}

\includesvg[width=0.16667in,height=0.16667in]{/assets/icons/16-copy.svg}

Check the docs for
\href{https://typst.app/docs/reference/scripting/\#packages}{more
information on how to import packages} .

\subsubsection{About}\label{about}

\begin{description}
\tightlist
\item[Author :]
Sabrina Jewson
\item[License:]
MIT
\item[Current version:]
0.1.1
\item[Last updated:]
September 11, 2024
\item[First released:]
October 23, 2023
\item[Minimum Typst version:]
0.8.0
\item[Archive size:]
94.8 kB
\href{https://packages.typst.org/preview/cmarker-0.1.1.tar.gz}{\pandocbounded{\includesvg[keepaspectratio]{/assets/icons/16-download.svg}}}
\item[Repository:]
\href{https://github.com/SabrinaJewson/cmarker.typ}{GitHub}
\end{description}

\subsubsection{Where to report issues?}\label{where-to-report-issues}

This package is a project of Sabrina Jewson . Report issues on
\href{https://github.com/SabrinaJewson/cmarker.typ}{their repository} .
You can also try to ask for help with this package on the
\href{https://forum.typst.app}{Forum} .

Please report this package to the Typst team using the
\href{https://typst.app/contact}{contact form} if you believe it is a
safety hazard or infringes upon your rights.

\phantomsection\label{versions}
\subsubsection{Version history}\label{version-history}

\begin{longtable}[]{@{}ll@{}}
\toprule\noalign{}
Version & Release Date \\
\midrule\noalign{}
\endhead
\bottomrule\noalign{}
\endlastfoot
0.1.1 & September 11, 2024 \\
\href{https://typst.app/universe/package/cmarker/0.1.0/}{0.1.0} &
October 23, 2023 \\
\end{longtable}

Typst GmbH did not create this package and cannot guarantee correct
functionality of this package or compatibility with any version of the
Typst compiler or app.


\section{Package List LaTeX/quick-maths.tex}
\title{typst.app/universe/package/quick-maths}

\phantomsection\label{banner}
\section{quick-maths}\label{quick-maths}

{ 0.2.0 }

Custom shorthands for math equations.

{ } Featured Package

\phantomsection\label{readme}
A package for creating custom shorthands for math equations.

\subsection{Usage}\label{usage}

The package comes with a single template function
\texttt{\ shorthands\ } that takes one or more tuples of the form
\texttt{\ (shorthand,\ replacement)\ } , where \texttt{\ shorthand\ }
can be a string or content.

There are some small quality of life features for interaction of
shorthands with fractions and attachments:

\begin{itemize}
\tightlist
\item
  If the right-most symbol of a shorthand has any attachments, they are
  moved to the shorthand’s replacement.
\item
  If a shorthand ends in the numerator of a fraction, the whole
  replacement is placed in the numerator.
\item
  If a shorthand starts in the denominator of a fraction, the whole
  replacement is placed in the denominator.
\end{itemize}

As the implementation of these features is quite hacky, you may
encounter some edge cases, where the use of explicit parentheses
hopefully saves you.

\subsection{Notes}\label{notes}

\begin{itemize}
\item
  Shorthands are parsed in the order they are given, so if you have a
  shorthand that is a prefix of another shorthand, you should put the
  longer shorthand first.
\item
  The content of an equation is traversed from left to right, so the
  left-most matching shorthand will be replaced first.
\item
  Shorthands consisting of only a single character or element may be
  applied using show rules, so that they can affect non-sequence
  elements. This may lead to different behavior than multi-character
  shorthands.
\item
  If the replacement of a shorthand contains a shorthand itself, there
  are no protections against infinite recursion or overflows.
\end{itemize}

\subsection{Example}\label{example}

\begin{Shaded}
\begin{Highlighting}[]
\NormalTok{\#import "@preview/quick{-}maths:0.2.0": shorthands}

\NormalTok{\#show: shorthands.with(}
\NormalTok{  ($+{-}$, $plus.minus$),}
\NormalTok{  ($|{-}$, math.tack),}
\NormalTok{  ($\textless{}=$, math.arrow.l.double) // Replaces \textquotesingle{}≤\textquotesingle{}}
\NormalTok{)}

\NormalTok{$ x\^{}2 = 9 quad \textless{}==\textgreater{} quad x = +{-}3 $}
\NormalTok{$ A or B |{-} A $}
\NormalTok{$ x \textless{}= y $}
\end{Highlighting}
\end{Shaded}

\pandocbounded{\includesvg[keepaspectratio]{https://github.com/typst/packages/raw/main/packages/preview/quick-maths/0.2.0/assets/example.svg}}

\subsubsection{How to add}\label{how-to-add}

Copy this into your project and use the import as
\texttt{\ quick-maths\ }

\begin{verbatim}
#import "@preview/quick-maths:0.2.0"
\end{verbatim}

\includesvg[width=0.16667in,height=0.16667in]{/assets/icons/16-copy.svg}

Check the docs for
\href{https://typst.app/docs/reference/scripting/\#packages}{more
information on how to import packages} .

\subsubsection{About}\label{about}

\begin{description}
\tightlist
\item[Author :]
Eric Biedert
\item[License:]
MIT
\item[Current version:]
0.2.0
\item[Last updated:]
November 18, 2024
\item[First released:]
July 5, 2024
\item[Archive size:]
3.58 kB
\href{https://packages.typst.org/preview/quick-maths-0.2.0.tar.gz}{\pandocbounded{\includesvg[keepaspectratio]{/assets/icons/16-download.svg}}}
\item[Repository:]
\href{https://github.com/EpicEricEE/typst-quick-maths}{GitHub}
\item[Categor y :]
\begin{itemize}
\tightlist
\item[]
\item
  \pandocbounded{\includesvg[keepaspectratio]{/assets/icons/16-hammer.svg}}
  \href{https://typst.app/universe/search/?category=utility}{Utility}
\end{itemize}
\end{description}

\subsubsection{Where to report issues?}\label{where-to-report-issues}

This package is a project of Eric Biedert . Report issues on
\href{https://github.com/EpicEricEE/typst-quick-maths}{their repository}
. You can also try to ask for help with this package on the
\href{https://forum.typst.app}{Forum} .

Please report this package to the Typst team using the
\href{https://typst.app/contact}{contact form} if you believe it is a
safety hazard or infringes upon your rights.

\phantomsection\label{versions}
\subsubsection{Version history}\label{version-history}

\begin{longtable}[]{@{}ll@{}}
\toprule\noalign{}
Version & Release Date \\
\midrule\noalign{}
\endhead
\bottomrule\noalign{}
\endlastfoot
0.2.0 & November 18, 2024 \\
\href{https://typst.app/universe/package/quick-maths/0.1.0/}{0.1.0} &
July 5, 2024 \\
\end{longtable}

Typst GmbH did not create this package and cannot guarantee correct
functionality of this package or compatibility with any version of the
Typst compiler or app.


\section{Package List LaTeX/stellar-iac.tex}
\title{typst.app/universe/package/stellar-iac}

\phantomsection\label{banner}
\phantomsection\label{template-thumbnail}
\pandocbounded{\includegraphics[keepaspectratio]{https://packages.typst.org/preview/thumbnails/stellar-iac-0.4.1-small.webp}}

\section{stellar-iac}\label{stellar-iac}

{ 0.4.1 }

Template for the International Astronautical Congress (IAC) manuscript

\href{/app?template=stellar-iac&version=0.4.1}{Create project in app}

\phantomsection\label{readme}
This is an unofficial Typst template for the International Astronautical
Congress (IAC) manuscript, which is based on
\href{https://www.iafastro.org/assets/files/IAC\%202023\%20Manuscript\%20Guidelines.pdf}{the
74th IAC Manuscript Guidelines (PDF file)} and
\href{https://www.iafastro.org/assets/files/IAC\%202023_Manuscript-Template.doc}{the
Manuscript Template and Style Guide (MS Word file)} .

\subsection{Usage}\label{usage}

To initialize a project with this template, run the following command:

\begin{Shaded}
\begin{Highlighting}[]
\ExtensionTok{typst}\NormalTok{ init @preview/stellar{-}iac}
\end{Highlighting}
\end{Shaded}

Or, you can manually add the following line at the beginning of your
Typst file:

\begin{Shaded}
\begin{Highlighting}[]
\NormalTok{\#import "@preview/stellar{-}iac:0.4.1": project}
\end{Highlighting}
\end{Shaded}

The \texttt{\ project\ } function exported by this template has the
following named arguments:

\begin{itemize}
\tightlist
\item
  \texttt{\ paper-code\ } (default: \texttt{\ ""\ } ): The paper code of
  the manuscript.
\item
  \texttt{\ title\ } (default: \texttt{\ ""\ } ): The title of the
  manuscript.
\item
  \texttt{\ authors\ } (default: \texttt{\ ()\ } ): The authors of the
  manuscript. Each item in the array should be a dictionary with the
  following keys:

  \begin{itemize}
  \tightlist
  \item
    \texttt{\ name\ } (required): The name of the author.
  \item
    \texttt{\ email\ } (required): The email address of the author.
  \item
    \texttt{\ affiliation\ } (required): The affiliation of the author.
    The value must match one of the affiliation names defined in the
    \texttt{\ organizations\ } argument.
  \item
    \texttt{\ corresponding\ } (default: \texttt{\ false\ } ): Whether
    the author is the corresponding author.
  \end{itemize}
\item
  \texttt{\ organizations\ } (default: \texttt{\ ()\ } ): The
  organizations of the authors. Each item in the array should be a
  dictionary with the following keys:

  \begin{itemize}
  \tightlist
  \item
    \texttt{\ name\ } (required): The name of the organization. This
    name should be used in the \texttt{\ affiliation\ } field of the
    \texttt{\ authors\ } argument.
  \item
    \texttt{\ display\ } (required): The display name of the
    organization, including its address.
  \end{itemize}
\item
  \texttt{\ keywords\ } (default: \texttt{\ ()\ } ): The keywords of the
  manuscript.
\item
  \texttt{\ header\ } (default: \texttt{\ {[}{]}\ } ): The header of the
  manuscript. For IAC 2024, it should be
  \texttt{\ {[}75\#super{[}th{]}\ International\ Astronautical\ Congress\ (IAC),\ Milan,\ Italy,\ 14-18\ October\ 2024.\textbackslash{}\ Copyright\ \#\{sym.copyright\}2024\ by\ the\ International\ Astronautical\ Federation\ (IAF).\ All\ rights\ reserved.{]}\ }
  .
\item
  \texttt{\ abstract\ } (default: \texttt{\ ""\ } ): The abstract of the
  manuscript.
\end{itemize}

See
\href{https://github.com/typst/packages/raw/main/packages/preview/stellar-iac/0.4.1/template/main.typ}{\texttt{\ main.typ\ }}
for more details.

\subsection{Notable differences from the official
template}\label{notable-differences-from-the-official-template}

\begin{itemize}
\tightlist
\item
  The citation style is not exactly the same as the official template.
  This could be fixed by manually preparing a CSL file, but it is
  “good enough.�
\end{itemize}

\subsection{Directory Structure and
Licensing}\label{directory-structure-and-licensing}

\begin{itemize}
\item
  The \texttt{\ lib.typ\ } file and all other files in this package,
  except for the content in the \texttt{\ template\ } directory, are
  licensed under the MIT License. See
  \href{https://github.com/typst/packages/raw/main/packages/preview/stellar-iac/0.4.1/LICENSE-MIT.txt}{LICENSE-MIT.txt}
  for details.
\item
  The content in the \texttt{\ template\ } directory is licensed under
  the MIT-0 License, which allows for unlimited reuse without any
  restrictions. See
  \href{https://github.com/typst/packages/raw/main/packages/preview/stellar-iac/0.4.1/LICENSE-MIT-0.txt}{LICENSE-MIT-0.txt}
  for more information.
\item
  The \texttt{\ reproduction\ } directory is included in this repository
  to demonstrate how the Typst template in \texttt{\ lib.typ\ } can be
  used to reproduce a layout similar to
  \href{https://www.iafastro.org/assets/files/IAC\%202023_Manuscript-Template.doc}{the
  original MS Word template} copyrighted by the International
  Astronautical Federation (IAF). \textbf{This content is not part of
  the distributed package and is provided here solely for demonstration
  purposes.} It is not licensed for use, modification, or distribution
  without permission from the copyright holder.
\end{itemize}

\href{/app?template=stellar-iac&version=0.4.1}{Create project in app}

\subsubsection{How to use}\label{how-to-use}

Click the button above to create a new project using this template in
the Typst app.

You can also use the Typst CLI to start a new project on your computer
using this command:

\begin{verbatim}
typst init @preview/stellar-iac:0.4.1
\end{verbatim}

\includesvg[width=0.16667in,height=0.16667in]{/assets/icons/16-copy.svg}

\subsubsection{About}\label{about}

\begin{description}
\tightlist
\item[Author s :]
\href{https://github.com/shunichironomura}{Shunichiro Nomura} \&
\href{https://github.com/conjikidow}{Riki Nakamura}
\item[License:]
MIT AND MIT-0
\item[Current version:]
0.4.1
\item[Last updated:]
September 24, 2024
\item[First released:]
September 24, 2024
\item[Minimum Typst version:]
0.11.1
\item[Archive size:]
182 kB
\href{https://packages.typst.org/preview/stellar-iac-0.4.1.tar.gz}{\pandocbounded{\includesvg[keepaspectratio]{/assets/icons/16-download.svg}}}
\item[Repository:]
\href{https://github.com/shunichironomura/iac-typst-template}{GitHub}
\item[Discipline :]
\begin{itemize}
\tightlist
\item[]
\item
  \href{https://typst.app/universe/search/?discipline=engineering}{Engineering}
\end{itemize}
\item[Categor y :]
\begin{itemize}
\tightlist
\item[]
\item
  \pandocbounded{\includesvg[keepaspectratio]{/assets/icons/16-atom.svg}}
  \href{https://typst.app/universe/search/?category=paper}{Paper}
\end{itemize}
\end{description}

\subsubsection{Where to report issues?}\label{where-to-report-issues}

This template is a project of Shunichiro Nomura and Riki Nakamura .
Report issues on
\href{https://github.com/shunichironomura/iac-typst-template}{their
repository} . You can also try to ask for help with this template on the
\href{https://forum.typst.app}{Forum} .

Please report this template to the Typst team using the
\href{https://typst.app/contact}{contact form} if you believe it is a
safety hazard or infringes upon your rights.

\phantomsection\label{versions}
\subsubsection{Version history}\label{version-history}

\begin{longtable}[]{@{}ll@{}}
\toprule\noalign{}
Version & Release Date \\
\midrule\noalign{}
\endhead
\bottomrule\noalign{}
\endlastfoot
0.4.1 & September 24, 2024 \\
\end{longtable}

Typst GmbH did not create this template and cannot guarantee correct
functionality of this template or compatibility with any version of the
Typst compiler or app.


\section{Package List LaTeX/apa7-ish.tex}
\title{typst.app/universe/package/apa7-ish}

\phantomsection\label{banner}
\phantomsection\label{template-thumbnail}
\pandocbounded{\includegraphics[keepaspectratio]{https://packages.typst.org/preview/thumbnails/apa7-ish-0.2.0-small.webp}}

\section{apa7-ish}\label{apa7-ish}

{ 0.2.0 }

Typst Template that (mostly) complies with APA7 Style (Work in
Progress).

\href{/app?template=apa7-ish&version=0.2.0}{Create project in app}

\phantomsection\label{readme}
\href{https://typst.app/}{Typst} Template that (mostly) complies with
APA7 Style (Work in Progress).

The template does not follow all recommendations by the APA Manual,
especially when the suggestions break with typographic conventions (such
as double line spacing :vomiting\_face:). Instead, the goal of this
template is that it generates you a high-quality manuscript that has all
the important components of the APA7 format, but is aesthetically
pleasing.

The following works already quite well:

\begin{itemize}
\tightlist
\item
  consistent and simple typesetting
\item
  correct display of author information / author note
\item
  citations anfalsed references
\item
  Page headers and footers (show short title in header)
\item
  Option to anonymize the paper
\item
  Tables: consisting of 3 parts (caption, content, and table notes)
\end{itemize}

This is still not finished:

\begin{itemize}
\tightlist
\item
  figures
\item
  complete pandoc integration (template for pandoc to replace
  Latex-based workflows)
\item
  automatic calculation of page margins (like memoir-class for Latex)
\end{itemize}

The easiest way to get started is to edit the example file, which has
sensible default values. Most fields in the configuration are optional
and will safely be ignored (not rendered) when you set them to
\texttt{\ none\ } .

\subsection{Authors}\label{authors}

The \texttt{\ authors\ } setting expects an array of dictionaries with
the following fields:

\begin{Shaded}
\begin{Highlighting}[]
\NormalTok{(}
\NormalTok{  name: "First Name Last Name", // Name of author as it should appear on the paper title page}
\NormalTok{  affiliation: "University, Department", // affiliation(s) of author as it should appear on the title page}
\NormalTok{  orcid: "0000{-}0000{-}0000{-}0000", // optional for author note}
\NormalTok{  corresponding: true, // optional to mark an author as corresponding author}
\NormalTok{  email: "email@upenn.edu", // optional email address, required if author is corresponding}
\NormalTok{  postal: "Longer string", // optional postal address for corresponding author}
\NormalTok{)}
\end{Highlighting}
\end{Shaded}

Note that the \texttt{\ affiliation\ } field also accepts an array, in
case an author has several affiliations:

\begin{Shaded}
\begin{Highlighting}[]
\NormalTok{(}
\NormalTok{  name: "First Name Last Name",}
\NormalTok{  affiliation: ("University A", "University B")}
\NormalTok{  ...}
\NormalTok{)}
\end{Highlighting}
\end{Shaded}

\subsection{Anonymization}\label{anonymization}

Sometimes you need to submit a paper without any author information. In
such cases you can set \texttt{\ anonymous\ } to \texttt{\ true\ } .

\subsection{Tables}\label{tables}

The template has basic support for tables with a handful of utilities.
Analogous to the \href{https://ctan.org/pkg/booktabs}{Latex booktabs
package} , there are pre-defined horizontal lines (“rules�):

\begin{itemize}
\tightlist
\item
  \texttt{\ \#toprule\ } : used at the top of the table, before the
  first row
\item
  \texttt{\ \#midrule\ } : used to separate the header row, or to
  separate a totals row at the bottom
\item
  \texttt{\ \#bottomrule\ } : used after the last row (technically the
  same as toprule, but may be useful later to define custom behaviour)
\end{itemize}

Addtionally, there is a \texttt{\ \#tablenote\ } function to be used to
place a table note below the table.

A minimal usage example (taken from the typst documentation):

\begin{Shaded}
\begin{Highlighting}[]
\NormalTok{// wrap everything in a \#figure}
\NormalTok{\#figure(}
\NormalTok{  [}
\NormalTok{    \#table(}
\NormalTok{      columns: 2,}
\NormalTok{      align: left,}
\NormalTok{      toprule, // separate table from other content}
\NormalTok{      table.header([Amount], [Ingredient]),}
\NormalTok{      midrule, // separation between table header and body}
\NormalTok{      [360g], [Baking flour],}
\NormalTok{      [250g], [Butter (room temp.)],}
\NormalTok{      [150g], [Brown sugar],}
\NormalTok{      [100g], [Cane sugar],}
\NormalTok{      [100g], [70\% cocoa chocolate],}
\NormalTok{      [100g], [35{-}40\% cocoa chocolate],}
\NormalTok{      [2], [Eggs],}
\NormalTok{      [Pinch], [Salt],}
\NormalTok{      [Drizzle], [Vanilla extract],}
\NormalTok{      bottomrule // separation after the last row}
\NormalTok{    )}
\NormalTok{    // tablenote goes after the \#table}
\NormalTok{    \#tablenote([Here are some additional notes.])}
\NormalTok{  ],}
\NormalTok{  // caption is part of the \#figure}
\NormalTok{  caption: [Here is the table caption]}
\NormalTok{)}
\end{Highlighting}
\end{Shaded}

\href{/app?template=apa7-ish&version=0.2.0}{Create project in app}

\subsubsection{How to use}\label{how-to-use}

Click the button above to create a new project using this template in
the Typst app.

You can also use the Typst CLI to start a new project on your computer
using this command:

\begin{verbatim}
typst init @preview/apa7-ish:0.2.0
\end{verbatim}

\includesvg[width=0.16667in,height=0.16667in]{/assets/icons/16-copy.svg}

\subsubsection{About}\label{about}

\begin{description}
\tightlist
\item[Author :]
MrWunderbar
\item[License:]
MIT
\item[Current version:]
0.2.0
\item[Last updated:]
October 30, 2024
\item[First released:]
October 21, 2024
\item[Minimum Typst version:]
0.12.0
\item[Archive size:]
8.21 kB
\href{https://packages.typst.org/preview/apa7-ish-0.2.0.tar.gz}{\pandocbounded{\includesvg[keepaspectratio]{/assets/icons/16-download.svg}}}
\item[Repository:]
\href{https://github.com/mrwunderbar666/typst-apa7ish}{GitHub}
\item[Categor y :]
\begin{itemize}
\tightlist
\item[]
\item
  \pandocbounded{\includesvg[keepaspectratio]{/assets/icons/16-atom.svg}}
  \href{https://typst.app/universe/search/?category=paper}{Paper}
\end{itemize}
\end{description}

\subsubsection{Where to report issues?}\label{where-to-report-issues}

This template is a project of MrWunderbar . Report issues on
\href{https://github.com/mrwunderbar666/typst-apa7ish}{their repository}
. You can also try to ask for help with this template on the
\href{https://forum.typst.app}{Forum} .

Please report this template to the Typst team using the
\href{https://typst.app/contact}{contact form} if you believe it is a
safety hazard or infringes upon your rights.

\phantomsection\label{versions}
\subsubsection{Version history}\label{version-history}

\begin{longtable}[]{@{}ll@{}}
\toprule\noalign{}
Version & Release Date \\
\midrule\noalign{}
\endhead
\bottomrule\noalign{}
\endlastfoot
0.2.0 & October 30, 2024 \\
\href{https://typst.app/universe/package/apa7-ish/0.1.0/}{0.1.0} &
October 21, 2024 \\
\end{longtable}

Typst GmbH did not create this template and cannot guarantee correct
functionality of this template or compatibility with any version of the
Typst compiler or app.


\section{Package List LaTeX/esotefy.tex}
\title{typst.app/universe/package/esotefy}

\phantomsection\label{banner}
\section{esotefy}\label{esotefy}

{ 1.0.0 }

A brainfuck implementation in pure Typst

\phantomsection\label{readme}
\begin{quote}
A compilation of esoteric languages including for now brainfuck
\end{quote}

\subsection{In pure brainfuck}\label{in-pure-brainfuck}

\begin{Shaded}
\begin{Highlighting}[]
\NormalTok{\#import "@preview/esotefy:1.0.0": brainf;}

\NormalTok{\#brainf("++++++++[\textgreater{}++++[\textgreater{}++\textgreater{}+++\textgreater{}+++\textgreater{}+\textless{}\textless{}\textless{}\textless{}{-}]\textgreater{}+\textgreater{}+\textgreater{}{-}\textgreater{}\textgreater{}+[\textless{}]\textless{}{-}]\textgreater{}\textgreater{}.\textgreater{}{-}{-}{-}.+++++++..+++.\textgreater{}\textgreater{}.\textless{}{-}.\textless{}.+++.{-}{-}{-}{-}{-}{-}.{-}{-}{-}{-}{-}{-}{-}{-}.\textgreater{}\textgreater{}+.\textgreater{}++.");}
\end{Highlighting}
\end{Shaded}

Into

\pandocbounded{\includegraphics[keepaspectratio]{https://media.discordapp.net/attachments/751591144919662752/1176988035309633647/image.png?ex=6570de86&is=655e6986&hm=60e18ac7187c117ab08a95c323f5059424342dbb9d8da49600c82502b5d14e7f&=&format=webp&width=328&height=102}}

\subsection{With inputs}\label{with-inputs}

\begin{Shaded}
\begin{Highlighting}[]
\NormalTok{\#import "@preview/esotefy:1.0.0": brainf;}

\NormalTok{\#brainf("++++++++++++++[{-}\textgreater{},.\textless{}]", inp: "Goodbye World!");}
\end{Highlighting}
\end{Shaded}

Into

\pandocbounded{\includegraphics[keepaspectratio]{https://media.discordapp.net/attachments/751591144919662752/1176988280613515366/image.png?ex=6570dec1&is=655e69c1&hm=f9285649f3e5ab72749af5820972c52827c727f6c52351b63d0bbd2ba9afce87&=&format=webp&width=808&height=181}}

I’ve based my implementation from theses documents:

\begin{itemize}
\tightlist
\item
  \href{https://en.wikipedia.org/wiki/Brainfuck}{Wikipedia}
\item
  \href{https://github.com/sunjay/brainfuck}{Github}
\item
  \href{https://onestepcode.com/brainfuck-compiler-c/}{A compiler of
  Brainfuck in c}
\end{itemize}

\subsubsection{How to add}\label{how-to-add}

Copy this into your project and use the import as \texttt{\ esotefy\ }

\begin{verbatim}
#import "@preview/esotefy:1.0.0"
\end{verbatim}

\includesvg[width=0.16667in,height=0.16667in]{/assets/icons/16-copy.svg}

Check the docs for
\href{https://typst.app/docs/reference/scripting/\#packages}{more
information on how to import packages} .

\subsubsection{About}\label{about}

\begin{description}
\tightlist
\item[Author :]
Thumus
\item[License:]
MIT
\item[Current version:]
1.0.0
\item[Last updated:]
November 29, 2023
\item[First released:]
November 29, 2023
\item[Minimum Typst version:]
0.7.0
\item[Archive size:]
2.03 kB
\href{https://packages.typst.org/preview/esotefy-1.0.0.tar.gz}{\pandocbounded{\includesvg[keepaspectratio]{/assets/icons/16-download.svg}}}
\item[Repository:]
\href{git@github.com:Thumuss/brainfuck.git}{GitHub}
\end{description}

\subsubsection{Where to report issues?}\label{where-to-report-issues}

This package is a project of Thumus . Report issues on
\href{git@github.com:Thumuss/brainfuck.git}{their repository} . You can
also try to ask for help with this package on the
\href{https://forum.typst.app}{Forum} .

Please report this package to the Typst team using the
\href{https://typst.app/contact}{contact form} if you believe it is a
safety hazard or infringes upon your rights.

\phantomsection\label{versions}
\subsubsection{Version history}\label{version-history}

\begin{longtable}[]{@{}ll@{}}
\toprule\noalign{}
Version & Release Date \\
\midrule\noalign{}
\endhead
\bottomrule\noalign{}
\endlastfoot
1.0.0 & November 29, 2023 \\
\end{longtable}

Typst GmbH did not create this package and cannot guarantee correct
functionality of this package or compatibility with any version of the
Typst compiler or app.


\section{Package List LaTeX/ilm.tex}
\title{typst.app/universe/package/ilm}

\phantomsection\label{banner}
\phantomsection\label{template-thumbnail}
\pandocbounded{\includegraphics[keepaspectratio]{https://packages.typst.org/preview/thumbnails/ilm-1.4.0-small.webp}}

\section{ilm}\label{ilm}

{ 1.4.0 }

Versatile and minimal template for non-fiction writing. Ideal for class
notes, reports, and books

{ } Featured Template

\href{/app?template=ilm&version=1.4.0}{Create project in app}

\phantomsection\label{readme}
\begin{quote}
‘Ilm (Urdu: عÙ?Ù„Ù'Ù\ldots) is the Urdu term for knowledge. It is
pronounced as
\href{https://en.wiktionary.org/wiki/\%D8\%B9\%D9\%84\%D9\%85\#Urdu}{/ə.ləm/}
.
\end{quote}

A versatile, clean and minimal template for non-fiction writing. The
template is ideal for class notes, reports, and books.

It contains a title page, a table of contents, and indices for different
types of figures; images, tables, code blocks.

Dynamic running footer contains the title of the chapter (top-level
heading).

See the
\href{https://github.com/talal/ilm/blob/main/example.pdf}{example.pdf}
file to see how it looks.

\subsection{Usage}\label{usage}

You can use this template in the Typst web app by clicking “Start from
template� on the dashboard and searching for \texttt{\ ilm\ } .

Alternatively, you can use the CLI to kick this project off using the
command:

\begin{Shaded}
\begin{Highlighting}[]
\ExtensionTok{typst}\NormalTok{ init @preview/ilm}
\end{Highlighting}
\end{Shaded}

Typst will create a new directory with all the files needed to get you
started.

The template will initialize your package with a sample call to the
\texttt{\ ilm\ } function in a show rule. If you, however, want to
change an existing project to use this template, you can add a show rule
like this at the top of your file:

\begin{Shaded}
\begin{Highlighting}[]
\NormalTok{\#import "@preview/ilm:1.4.0": *}

\NormalTok{\#set text(lang: "en")}

\NormalTok{\#show: ilm.with(}
\NormalTok{  title: [Your Title],}
\NormalTok{  author: "Max Mustermann",}
\NormalTok{  date: datetime(year: 2024, month: 03, day: 19),}
\NormalTok{  abstract: [\#lorem(30)],}
\NormalTok{  bibliography: bibliography("refs.bib"),}
\NormalTok{  figure{-}index: (enabled: true),}
\NormalTok{  table{-}index: (enabled: true),}
\NormalTok{  listing{-}index: (enabled: true)}
\NormalTok{)}

\NormalTok{// Your content goes below.}
\end{Highlighting}
\end{Shaded}

\begin{quote}
{[}!NOTE{]} This template uses the
\href{https://typeof.net/Iosevka/}{Iosevka} font for raw text. In order
to use Iosevka, the font must be installed on your computer. In case
Iosevka is not installed, as will be the case for Typst Web App, then
the template will fall back to the default “Fira Mono� font.
\end{quote}

\subsection{Configuration}\label{configuration}

This template exports the \texttt{\ ilm\ } function with the following
named arguments:

\begin{longtable}[]{@{}llll@{}}
\toprule\noalign{}
Argument & Default Value & Type & Description \\
\midrule\noalign{}
\endhead
\bottomrule\noalign{}
\endlastfoot
\texttt{\ title\ } & \texttt{\ Your\ Title\ } &
\href{https://typst.app/docs/reference/foundations/content/}{content} &
The title for your work. \\
\texttt{\ author\ } & \texttt{\ Author\ } &
\href{https://typst.app/docs/reference/foundations/content/}{content} &
A string to specify the author’s name \\
\texttt{\ paper-size\ } & \texttt{\ a4\ } &
\href{https://typst.app/docs/reference/foundations/str/}{string} &
Specify a
\href{https://typst.app/docs/reference/layout/page\#parameters-paper}{paper
size string} to change the page size. \\
\texttt{\ date\ } & \texttt{\ none\ } &
\href{https://typst.app/docs/reference/foundations/datetime/}{datetime}
& The date that will be displayed on the cover page. \\
\texttt{\ date-format\ } &
\texttt{\ {[}month\ repr:long{]}\ {[}day\ padding:zero{]},\ {[}year\ repr:full{]}\ }
& \href{https://typst.app/docs/reference/foundations/str/}{string} & The
format for the date that will be displayed on the cover page. By
default, the date will be displayed as \texttt{\ MMMM\ DD,\ YYYY\ } . \\
\texttt{\ abstract\ } & \texttt{\ none\ } &
\href{https://typst.app/docs/reference/foundations/content/}{content} &
A brief summary/description of your work. This is shown on the cover
page. \\
\texttt{\ preface\ } & \texttt{\ none\ } &
\href{https://typst.app/docs/reference/foundations/content/}{content} &
The preface for your work. The preface content is shown on its own
separate page after the cover. \\
\texttt{\ chapter-pagebreak\ } & \texttt{\ true\ } &
\href{https://typst.app/docs/reference/foundations/bool/}{bool} &
Setting this to \texttt{\ false\ } will prevent chapters from starting
on a new page. \\
\texttt{\ external-link-circle\ } & \texttt{\ true\ } &
\href{https://typst.app/docs/reference/foundations/bool/}{bool} &
Setting this to \texttt{\ false\ } will disable the maroon circle that
is shown next to external links. \\
\texttt{\ table-of-contents\ } & \texttt{\ outline()\ } &
\href{https://typst.app/docs/reference/foundations/content/}{content} &
The result of a call to the
\href{https://typst.app/docs/reference/model/outline/}{outline function}
or none. Setting this to \texttt{\ none\ } will disable the table of
contents. \\
\texttt{\ appendix\ } &
\texttt{\ (enabled:\ false,\ title:\ "Appendix",\ heading-numbering-format:\ "A.1.1.",\ body:\ none)\ }
&
\href{https://typst.app/docs/reference/foundations/dictionary/}{dictionary}
& Setting \texttt{\ enabled\ } to \texttt{\ true\ } and defining your
content in \texttt{\ body\ } will display the appendix after the main
body of your document and before the bibliography. \\
\texttt{\ bibliography\ } & \texttt{\ none\ } &
\href{https://typst.app/docs/reference/foundations/content/}{content} &
The result of a call to the
\href{https://typst.app/docs/reference/model/bibliography/}{bibliography
function} or none. Specifying this will configure numeric, IEEE-style
citations. \\
\texttt{\ figure-index\ } &
\texttt{\ (enabled:\ false,\ title:\ "Index\ of\ Figures")\ } &
\href{https://typst.app/docs/reference/foundations/dictionary/}{dictionary}
& Setting this to \texttt{\ true\ } will display an index of image
figures at the end of the document. \\
\texttt{\ table-index\ } &
\texttt{\ (enabled:\ false,\ title:\ "Index\ of\ Tables")\ } &
\href{https://typst.app/docs/reference/foundations/dictionary/}{dictionary}
& Setting this to \texttt{\ true\ } will display an index of table
figures at the end of the document. \\
\texttt{\ listing-index\ } &
\texttt{\ (enabled:\ false,\ title:\ "Index\ of\ Listings")\ } &
\href{https://typst.app/docs/reference/foundations/dictionary/}{dictionary}
& Setting this to \texttt{\ true\ } will display an index of listing
(code block) figures at the end of the document. \\
\end{longtable}

The above table gives you a \emph{brief description} of the different
options that you can choose to customize the template. For a detailed
explanation of these options, see the
\href{https://github.com/talal/ilm/blob/main/example.pdf}{example.pdf}
file.

The function also accepts a single, positional argument for the body.

\begin{quote}
{[}!NOTE{]} The language setting for text ( \texttt{\ lang\ } parameter
of \texttt{\ text\ } function) should be defined before the
\texttt{\ ilm\ } function so that headings such as table of contents and
bibliography will be defined as per the text language.
\end{quote}

\href{/app?template=ilm&version=1.4.0}{Create project in app}

\subsubsection{How to use}\label{how-to-use}

Click the button above to create a new project using this template in
the Typst app.

You can also use the Typst CLI to start a new project on your computer
using this command:

\begin{verbatim}
typst init @preview/ilm:1.4.0
\end{verbatim}

\includesvg[width=0.16667in,height=0.16667in]{/assets/icons/16-copy.svg}

\subsubsection{About}\label{about}

\begin{description}
\tightlist
\item[Author :]
\href{https://github.com/talal}{Muhammad Talal Anwar}
\item[License:]
MIT-0
\item[Current version:]
1.4.0
\item[Last updated:]
November 21, 2024
\item[First released:]
March 22, 2024
\item[Minimum Typst version:]
0.12.0
\item[Archive size:]
9.08 kB
\href{https://packages.typst.org/preview/ilm-1.4.0.tar.gz}{\pandocbounded{\includesvg[keepaspectratio]{/assets/icons/16-download.svg}}}
\item[Repository:]
\href{https://github.com/talal/ilm}{GitHub}
\item[Categor ies :]
\begin{itemize}
\tightlist
\item[]
\item
  \pandocbounded{\includesvg[keepaspectratio]{/assets/icons/16-docs.svg}}
  \href{https://typst.app/universe/search/?category=book}{Book}
\item
  \pandocbounded{\includesvg[keepaspectratio]{/assets/icons/16-speak.svg}}
  \href{https://typst.app/universe/search/?category=report}{Report}
\end{itemize}
\end{description}

\subsubsection{Where to report issues?}\label{where-to-report-issues}

This template is a project of Muhammad Talal Anwar . Report issues on
\href{https://github.com/talal/ilm}{their repository} . You can also try
to ask for help with this template on the
\href{https://forum.typst.app}{Forum} .

Please report this template to the Typst team using the
\href{https://typst.app/contact}{contact form} if you believe it is a
safety hazard or infringes upon your rights.

\phantomsection\label{versions}
\subsubsection{Version history}\label{version-history}

\begin{longtable}[]{@{}ll@{}}
\toprule\noalign{}
Version & Release Date \\
\midrule\noalign{}
\endhead
\bottomrule\noalign{}
\endlastfoot
1.4.0 & November 21, 2024 \\
\href{https://typst.app/universe/package/ilm/1.3.1/}{1.3.1} & November
13, 2024 \\
\href{https://typst.app/universe/package/ilm/1.3.0/}{1.3.0} & November
4, 2024 \\
\href{https://typst.app/universe/package/ilm/1.2.1/}{1.2.1} & August 7,
2024 \\
\href{https://typst.app/universe/package/ilm/1.2.0/}{1.2.0} & August 2,
2024 \\
\href{https://typst.app/universe/package/ilm/1.1.3/}{1.1.3} & July 24,
2024 \\
\href{https://typst.app/universe/package/ilm/1.1.2/}{1.1.2} & June 19,
2024 \\
\href{https://typst.app/universe/package/ilm/1.1.1/}{1.1.1} & May 6,
2024 \\
\href{https://typst.app/universe/package/ilm/1.1.0/}{1.1.0} & April 12,
2024 \\
\href{https://typst.app/universe/package/ilm/1.0.0/}{1.0.0} & April 9,
2024 \\
\href{https://typst.app/universe/package/ilm/0.1.3/}{0.1.3} & April 8,
2024 \\
\href{https://typst.app/universe/package/ilm/0.1.2/}{0.1.2} & March 26,
2024 \\
\href{https://typst.app/universe/package/ilm/0.1.1/}{0.1.1} & March 23,
2024 \\
\href{https://typst.app/universe/package/ilm/0.1.0/}{0.1.0} & March 22,
2024 \\
\end{longtable}

Typst GmbH did not create this template and cannot guarantee correct
functionality of this template or compatibility with any version of the
Typst compiler or app.


\section{Package List LaTeX/tuhi-course-poster-vuw.tex}
\title{typst.app/universe/package/tuhi-course-poster-vuw}

\phantomsection\label{banner}
\phantomsection\label{template-thumbnail}
\pandocbounded{\includegraphics[keepaspectratio]{https://packages.typst.org/preview/thumbnails/tuhi-course-poster-vuw-0.1.0-small.webp}}

\section{tuhi-course-poster-vuw}\label{tuhi-course-poster-vuw}

{ 0.1.0 }

A poster template for VUW course descriptions.

\href{/app?template=tuhi-course-poster-vuw&version=0.1.0}{Create project
in app}

\phantomsection\label{readme}
A Typst template for VUW course posters. To get started:

\begin{Shaded}
\begin{Highlighting}[]
\NormalTok{typst init @preview/tuhi{-}course{-}poster{-}vuw:0.1.0}
\end{Highlighting}
\end{Shaded}

And edit the \texttt{\ main.typ\ } example.

\pandocbounded{\includegraphics[keepaspectratio]{https://github.com/typst/packages/raw/main/packages/preview/tuhi-course-poster-vuw/0.1.0/thumbnail.png}}

\subsection{Contributing}\label{contributing}

PRs are welcome! And if you encounter any bugs or have any
requests/ideas, feel free to open an issue.

\href{/app?template=tuhi-course-poster-vuw&version=0.1.0}{Create project
in app}

\subsubsection{How to use}\label{how-to-use}

Click the button above to create a new project using this template in
the Typst app.

You can also use the Typst CLI to start a new project on your computer
using this command:

\begin{verbatim}
typst init @preview/tuhi-course-poster-vuw:0.1.0
\end{verbatim}

\includesvg[width=0.16667in,height=0.16667in]{/assets/icons/16-copy.svg}

\subsubsection{About}\label{about}

\begin{description}
\tightlist
\item[Author :]
\href{https://github.com/baptiste}{baptiste}
\item[License:]
MPL-2.0
\item[Current version:]
0.1.0
\item[Last updated:]
April 30, 2024
\item[First released:]
April 30, 2024
\item[Archive size:]
421 kB
\href{https://packages.typst.org/preview/tuhi-course-poster-vuw-0.1.0.tar.gz}{\pandocbounded{\includesvg[keepaspectratio]{/assets/icons/16-download.svg}}}
\item[Categor y :]
\begin{itemize}
\tightlist
\item[]
\item
  \pandocbounded{\includesvg[keepaspectratio]{/assets/icons/16-pin.svg}}
  \href{https://typst.app/universe/search/?category=poster}{Poster}
\end{itemize}
\end{description}

\subsubsection{Where to report issues?}\label{where-to-report-issues}

This template is a project of baptiste . You can also try to ask for
help with this template on the \href{https://forum.typst.app}{Forum} .

Please report this template to the Typst team using the
\href{https://typst.app/contact}{contact form} if you believe it is a
safety hazard or infringes upon your rights.

\phantomsection\label{versions}
\subsubsection{Version history}\label{version-history}

\begin{longtable}[]{@{}ll@{}}
\toprule\noalign{}
Version & Release Date \\
\midrule\noalign{}
\endhead
\bottomrule\noalign{}
\endlastfoot
0.1.0 & April 30, 2024 \\
\end{longtable}

Typst GmbH did not create this template and cannot guarantee correct
functionality of this template or compatibility with any version of the
Typst compiler or app.


\section{Package List LaTeX/dashy-todo.tex}
\title{typst.app/universe/package/dashy-todo}

\phantomsection\label{banner}
\section{dashy-todo}\label{dashy-todo}

{ 0.0.1 }

A method to display TODOs at the side of the page.

\phantomsection\label{readme}
Create TODO comments, which are displayed at the sides of the page.

\pandocbounded{\includesvg[keepaspectratio]{https://github.com/typst/packages/raw/main/packages/preview/dashy-todo/0.0.1/example.svg}}

\subsection{Usage}\label{usage}

The package provides a
\texttt{\ todo(message,\ position:\ auto\ \textbar{}\ left\ \textbar{}\ right)\ }
method. Call it anywhere you need a todo message.

\begin{Shaded}
\begin{Highlighting}[]
\NormalTok{\#import "@preview/dashy{-}todo:0.0.1": todo}

\NormalTok{// It automatically goes to the closer side (left or right)}
\NormalTok{A todo on the left \#todo[On the left].}

\NormalTok{// You can specify a side if you want to}
\NormalTok{\#todo(position: right)[Also right]}

\NormalTok{// You can add arbitrary content}
\NormalTok{\#todo[We need to fix the $lim\_(x {-}\textgreater{} oo)$ equation. See \#link("https://example.com")[example.com]]}

\NormalTok{// And you can create an outline for the TODOs}
\NormalTok{\#outline(title: "TODOs", target: figure.where(kind: "todo"))}
\end{Highlighting}
\end{Shaded}

\subsection{Styling}\label{styling}

You can modify the text by wrapping it, e.g.:

\begin{verbatim}
#let small-todo = (..args) => text(size: 0.6em)[#todo(..args)]

#small-todo[This will be in fine print]
\end{verbatim}

\subsubsection{How to add}\label{how-to-add}

Copy this into your project and use the import as
\texttt{\ dashy-todo\ }

\begin{verbatim}
#import "@preview/dashy-todo:0.0.1"
\end{verbatim}

\includesvg[width=0.16667in,height=0.16667in]{/assets/icons/16-copy.svg}

Check the docs for
\href{https://typst.app/docs/reference/scripting/\#packages}{more
information on how to import packages} .

\subsubsection{About}\label{about}

\begin{description}
\tightlist
\item[Author :]
Otto-AA
\item[License:]
MIT-0
\item[Current version:]
0.0.1
\item[Last updated:]
July 23, 2024
\item[First released:]
July 23, 2024
\item[Archive size:]
2.93 kB
\href{https://packages.typst.org/preview/dashy-todo-0.0.1.tar.gz}{\pandocbounded{\includesvg[keepaspectratio]{/assets/icons/16-download.svg}}}
\item[Repository:]
\href{https://github.com/Otto-AA/dashy-todo}{GitHub}
\item[Categor y :]
\begin{itemize}
\tightlist
\item[]
\item
  \pandocbounded{\includesvg[keepaspectratio]{/assets/icons/16-hammer.svg}}
  \href{https://typst.app/universe/search/?category=utility}{Utility}
\end{itemize}
\end{description}

\subsubsection{Where to report issues?}\label{where-to-report-issues}

This package is a project of Otto-AA . Report issues on
\href{https://github.com/Otto-AA/dashy-todo}{their repository} . You can
also try to ask for help with this package on the
\href{https://forum.typst.app}{Forum} .

Please report this package to the Typst team using the
\href{https://typst.app/contact}{contact form} if you believe it is a
safety hazard or infringes upon your rights.

\phantomsection\label{versions}
\subsubsection{Version history}\label{version-history}

\begin{longtable}[]{@{}ll@{}}
\toprule\noalign{}
Version & Release Date \\
\midrule\noalign{}
\endhead
\bottomrule\noalign{}
\endlastfoot
0.0.1 & July 23, 2024 \\
\end{longtable}

Typst GmbH did not create this package and cannot guarantee correct
functionality of this package or compatibility with any version of the
Typst compiler or app.


\section{Package List LaTeX/prooftrees.tex}
\title{typst.app/universe/package/prooftrees}

\phantomsection\label{banner}
\section{prooftrees}\label{prooftrees}

{ 0.1.0 }

Proof trees for natural deduction and type theories

\phantomsection\label{readme}
This package is for constructing proof trees in the style of natural
deduction or the sequent calculus, for \texttt{\ typst\ }
\texttt{\ 0.7.0\ } . Please do open issues for bugs etc :)

Features:

\begin{itemize}
\tightlist
\item
  Inferences can have arbitrarily many premises.
\item
  Inference lines can have left and/or right labels¹
\item
  Configurable² per tree and per line: premise spacing, the line
  stroke, etc… .
\item
  They’re proof trees.
\end{itemize}

¹ The placement of labels is currently very primitive, and requires
much user intervention.

² Options are quite limited.

\subsection{Usage}\label{usage}

The user interface is inspired by
\href{https://ctan.org/pkg/bussproofs}{bussproof} ’s; a tree is
constructed by a sequence of ‘lines’ that state their number of
premises.
\href{https://github.com/typst/packages/raw/main/packages/preview/prooftrees/0.1.0/src/prooftrees.typ}{\texttt{\ src/prooftrees.typ\ }}
contains the documentation and the main functions needed.

The code for some example trees can be seen in
\texttt{\ examples/prooftree\_test.typ\ } .

\subsubsection{Examples}\label{examples}

A single inference would be:

\begin{Shaded}
\begin{Highlighting}[]
\NormalTok{\#import "@preview/prooftrees:0.1.0"}

\NormalTok{\#prooftree.tree(}
\NormalTok{    prooftree.axi[$A =\textgreater{} A$],}
\NormalTok{    prooftree.uni[$A =\textgreater{} A, B$]}
\NormalTok{)}
\end{Highlighting}
\end{Shaded}

\includegraphics[width=0.3\linewidth,height=\textheight,keepaspectratio]{https://raw.githubusercontent.com/david-davies/typst-prooftree/main/examples/Example1.png}

A more interesting example:

\begin{Shaded}
\begin{Highlighting}[]
\NormalTok{\#import "@preview/prooftrees:0.1.0"}

\NormalTok{\#prooftree.tree(}
\NormalTok{    prooftree.axi[$B =\textgreater{} B$],}
\NormalTok{    prooftree.uni[$B =\textgreater{} B, A$],}
\NormalTok{    prooftree.uni[$B =\textgreater{} A, B$],}
\NormalTok{        prooftree.axi[$A =\textgreater{} A$],}
\NormalTok{        prooftree.uni[$A =\textgreater{} A, B$],}
\NormalTok{    prooftree.bin[$B =\textgreater{} A, B$]}
\NormalTok{)}
\end{Highlighting}
\end{Shaded}

\includegraphics[width=0.4\linewidth,height=\textheight,keepaspectratio]{https://raw.githubusercontent.com/david-davies/typst-prooftree/main/examples/Example2.png}

An n-ary inference can be made:

\begin{Shaded}
\begin{Highlighting}[]
\NormalTok{\#import "@preview/prooftrees:0.1.0"}

\NormalTok{\#prooftrees.tree(}
\NormalTok{    prooftrees.axi(pad(bottom: 2pt, [$P\_1$])),}
\NormalTok{    prooftrees.axi(pad(bottom: 2pt, [$P\_2$])),}
\NormalTok{    prooftrees.axi(pad(bottom: 2pt, [$P\_3$])),}
\NormalTok{    prooftrees.axi(pad(bottom: 2pt, [$P\_4$])),}
\NormalTok{    prooftrees.axi(pad(bottom: 2pt, [$P\_5$])),}
\NormalTok{    prooftrees.axi(pad(bottom: 2pt, [$P\_6$])),}
\NormalTok{    prooftrees.nary(6)[$C$],}
\NormalTok{)}
\end{Highlighting}
\end{Shaded}

\includegraphics[width=0.3\linewidth,height=\textheight,keepaspectratio]{https://raw.githubusercontent.com/david-davies/typst-prooftree/main/examples/Example3.png}

\subsection{Known Issues:}\label{known-issues}

\subsubsection{Superscripts and subscripts clip with the
line}\label{superscripts-and-subscripts-clip-with-the-line}

The boundaries of blocks containing math do not expand enough for
sub/pscripts; I think this is a typst issue. Short-term fix: add manual
vspace or padding in the cell.

\subsection{Implementation}\label{implementation}

The placement of the line and conclusion is calculated using
\texttt{\ measure\ } on the premises and labels, and doing geometric
arithmetic with these values.

\subsubsection{How to add}\label{how-to-add}

Copy this into your project and use the import as
\texttt{\ prooftrees\ }

\begin{verbatim}
#import "@preview/prooftrees:0.1.0"
\end{verbatim}

\includesvg[width=0.16667in,height=0.16667in]{/assets/icons/16-copy.svg}

Check the docs for
\href{https://typst.app/docs/reference/scripting/\#packages}{more
information on how to import packages} .

\subsubsection{About}\label{about}

\begin{description}
\tightlist
\item[Author :]
\href{https://github.com/david-davies}{david-davies}
\item[License:]
MIT
\item[Current version:]
0.1.0
\item[Last updated:]
September 3, 2023
\item[First released:]
September 3, 2023
\item[Archive size:]
8.43 kB
\href{https://packages.typst.org/preview/prooftrees-0.1.0.tar.gz}{\pandocbounded{\includesvg[keepaspectratio]{/assets/icons/16-download.svg}}}
\item[Repository:]
\href{https://github.com/david-davies/typst-prooftree}{GitHub}
\end{description}

\subsubsection{Where to report issues?}\label{where-to-report-issues}

This package is a project of david-davies . Report issues on
\href{https://github.com/david-davies/typst-prooftree}{their repository}
. You can also try to ask for help with this package on the
\href{https://forum.typst.app}{Forum} .

Please report this package to the Typst team using the
\href{https://typst.app/contact}{contact form} if you believe it is a
safety hazard or infringes upon your rights.

\phantomsection\label{versions}
\subsubsection{Version history}\label{version-history}

\begin{longtable}[]{@{}ll@{}}
\toprule\noalign{}
Version & Release Date \\
\midrule\noalign{}
\endhead
\bottomrule\noalign{}
\endlastfoot
0.1.0 & September 3, 2023 \\
\end{longtable}

Typst GmbH did not create this package and cannot guarantee correct
functionality of this package or compatibility with any version of the
Typst compiler or app.


\section{Package List LaTeX/structured-uib.tex}
\title{typst.app/universe/package/structured-uib}

\phantomsection\label{banner}
\phantomsection\label{template-thumbnail}
\pandocbounded{\includegraphics[keepaspectratio]{https://packages.typst.org/preview/thumbnails/structured-uib-0.1.0-small.webp}}

\section{structured-uib}\label{structured-uib}

{ 0.1.0 }

Lab report template for the course PHYS114 at the University of Bergen.

\href{/app?template=structured-uib&version=0.1.0}{Create project in app}

\phantomsection\label{readme}
Report template to be used for laboratory reports in the course PHYS114
- Basic Measurement Science and Experimental Physics, at the University
of Bergen ( \url{https://www.uib.no/en/courses/PHYS114} ). The template
is in Norwegian only as of now. English support may be added in the
future.

The first part of the packages name is arbitrary, such that it follows
the naming guidelines of Typst.

\textbf{Note:} The template uses the fonts \textbf{STIX Two Text} and
\textbf{STIX Two Math} (
\url{https://github.com/stipub/stixfonts/tree/master/fonts} ). If
running Typst locally on your computer, make sure you have these fonts
installed. There should be no font problems if you are using Typst via
\href{https://typst.app/}{https://typst.app} however.

Usage:

\begin{Shaded}
\begin{Highlighting}[]
\NormalTok{// IMPORTS}
\NormalTok{\#import "@preview/structured{-}uib:0.1.0": *}

\NormalTok{// TEMPLATE SETTINGS}
\NormalTok{\#show: report.with(}
\NormalTok{  task{-}no: "1",}
\NormalTok{  task{-}name: "Måling og behandling av måledata",}
\NormalTok{  authors: (}
\NormalTok{    "Student Enersen",}
\NormalTok{    "Student Toersen", }
\NormalTok{    "Student Treersen"}
\NormalTok{  ),}
\NormalTok{  mails: (}
\NormalTok{    "student.enersen@student.uib.no", }
\NormalTok{    "student.toersen@student.uib.no", }
\NormalTok{    "student.treersen@student.uib.no"}
\NormalTok{  ),}
\NormalTok{  group: "1{-}1",}
\NormalTok{  date: "29. Apr. 2024",}
\NormalTok{  supervisor: "Professor Professorsen",}
\NormalTok{)}

\NormalTok{// CONTENT HERE...}
\end{Highlighting}
\end{Shaded}

Front page:
\pandocbounded{\includegraphics[keepaspectratio]{https://github.com/AugustinWinther/structured-uib/assets/30674646/a93718d8-362d-453b-8047-3c3c4388d442}}

\subsection{Licenses}\label{licenses}

\texttt{\ lib.typ\ } is licensed under MIT. The contents of the
\texttt{\ template/\ } directory are licensed under MIT-0. The
logo/emblem of the University of Bergen (located at
\texttt{\ media/uib-emblem.svg\ } ) is owned by their respective owners.

\href{/app?template=structured-uib&version=0.1.0}{Create project in app}

\subsubsection{How to use}\label{how-to-use}

Click the button above to create a new project using this template in
the Typst app.

You can also use the Typst CLI to start a new project on your computer
using this command:

\begin{verbatim}
typst init @preview/structured-uib:0.1.0
\end{verbatim}

\includesvg[width=0.16667in,height=0.16667in]{/assets/icons/16-copy.svg}

\subsubsection{About}\label{about}

\begin{description}
\tightlist
\item[Author :]
\href{https://winther.io}{Augustin Winther}
\item[License:]
MIT AND MIT-0
\item[Current version:]
0.1.0
\item[Last updated:]
April 29, 2024
\item[First released:]
April 29, 2024
\item[Archive size:]
24.2 kB
\href{https://packages.typst.org/preview/structured-uib-0.1.0.tar.gz}{\pandocbounded{\includesvg[keepaspectratio]{/assets/icons/16-download.svg}}}
\item[Repository:]
\href{https://github.com/AugustinWinther/structured-uib}{GitHub}
\item[Categor y :]
\begin{itemize}
\tightlist
\item[]
\item
  \pandocbounded{\includesvg[keepaspectratio]{/assets/icons/16-speak.svg}}
  \href{https://typst.app/universe/search/?category=report}{Report}
\end{itemize}
\end{description}

\subsubsection{Where to report issues?}\label{where-to-report-issues}

This template is a project of Augustin Winther . Report issues on
\href{https://github.com/AugustinWinther/structured-uib}{their
repository} . You can also try to ask for help with this template on the
\href{https://forum.typst.app}{Forum} .

Please report this template to the Typst team using the
\href{https://typst.app/contact}{contact form} if you believe it is a
safety hazard or infringes upon your rights.

\phantomsection\label{versions}
\subsubsection{Version history}\label{version-history}

\begin{longtable}[]{@{}ll@{}}
\toprule\noalign{}
Version & Release Date \\
\midrule\noalign{}
\endhead
\bottomrule\noalign{}
\endlastfoot
0.1.0 & April 29, 2024 \\
\end{longtable}

Typst GmbH did not create this template and cannot guarantee correct
functionality of this template or compatibility with any version of the
Typst compiler or app.


\section{Package List LaTeX/kunskap.tex}
\title{typst.app/universe/package/kunskap}

\phantomsection\label{banner}
\phantomsection\label{template-thumbnail}
\pandocbounded{\includegraphics[keepaspectratio]{https://packages.typst.org/preview/thumbnails/kunskap-0.1.0-small.webp}}

\section{kunskap}\label{kunskap}

{ 0.1.0 }

A template with generous spacing for reports, assignments, course
documents, and similar (shorter) documents.

\href{/app?template=kunskap&version=0.1.0}{Create project in app}

\phantomsection\label{readme}
A \href{https://typst.app/}{Typst} template mainly intended for shorter
academic documents such as reports, assignments, course documents, and
so on. Its name, \emph{“kunskap�} , means \emph{knowledge} in
Swedish.

See
\href{https://github.com/mbollmann/typst-kunskap/blob/main/example.pdf}{this
example PDF} for a longer demonstration of how it looks.

\subsection{Usage}\label{usage}

You can use this template in the Typst web app by clicking “Start from
template� on the dashboard and searching for \texttt{\ kunskap\ } .

Alternatively, you can use the CLI to kick this project off using the
command

\begin{Shaded}
\begin{Highlighting}[]
\ExtensionTok{typst}\NormalTok{ init @preview/kunskap}
\end{Highlighting}
\end{Shaded}

Typst will create a new directory with all the files needed to get you
started.

\subsection{Configuration}\label{configuration}

This template exports the \texttt{\ kunskap\ } function with several
arguments. You will want to set at least the following, describing the
metadata of your document:

\begin{longtable}[]{@{}ll@{}}
\toprule\noalign{}
Argument & Description \\
\midrule\noalign{}
\endhead
\bottomrule\noalign{}
\endlastfoot
\texttt{\ title\ } & Title of your document \\
\texttt{\ author\ } & Author(s) of your document; can be any content, or
an array of strings \\
\texttt{\ date\ } & Date string to display below the author(s); defaults
to a string of today’s date, but can take any content. Set to
\texttt{\ none\ } if you don’t use it at all. \\
\texttt{\ header\ } & Content that appears in the left-hand side of the
header on every page; this is intended for e.g. the name of a course or
some other contextual information for the document, but can of course
also be left empty. \\
\end{longtable}

Additionally, you can configure some aspects of the template’s layout
with the following arguments:

\begin{longtable}[]{@{}lll@{}}
\toprule\noalign{}
Argument & Description & Default \\
\midrule\noalign{}
\endhead
\bottomrule\noalign{}
\endlastfoot
\texttt{\ paper-size\ } & Paper size of the document &
\texttt{\ "a4"\ } \\
\texttt{\ body-font\ } & Font for the body text &
\texttt{\ "Noto\ Serif"\ } \\
\texttt{\ body-font-size\ } & Font size for the body text &
\texttt{\ 10pt\ } \\
\texttt{\ headings-font\ } & Font for the headings &
\texttt{\ ("Source\ Sans\ Pro",\ "Source\ Sans\ 3")\ } \\
\texttt{\ raw-font\ } & Font for raw (i.e. monospaced) text &
\texttt{\ ("Hack",\ "Source\ Code\ Pro")\ } {[}\^{}1{]} \\
\texttt{\ raw-font-size\ } & Font size for raw text &
\texttt{\ 9pt\ } \\
\texttt{\ link-color\ } & Color for highlighting
\href{https://typst.app/docs/reference/model/link/}{links} &
\texttt{\ rgb("\#3282b8")\ }
\pandocbounded{\includegraphics[keepaspectratio]{https://img.shields.io/badge/steel_blue-3282b8}} \\
\texttt{\ muted-color\ } & Color for muted text, such as page numbers
and headers after the first page & \texttt{\ luma(160)\ } \\
\texttt{\ block-bg-color\ } & Color for the background of raw text &
\texttt{\ luma(240)\ } \\
\end{longtable}

The template will initialize your document with a sample call to the
\texttt{\ kunskap\ } function. Alternatively, here’s a minimal example
of how you could use this template in your document:

\begin{Shaded}
\begin{Highlighting}[]
\NormalTok{\#import "@preview/kunskap:0.1.0": *}

\NormalTok{\#show: kunskap.with(}
\NormalTok{    title: "Your report title",}
\NormalTok{    author: "Your name",}
\NormalTok{    date: datetime.today().display(),}
\NormalTok{    header: "Your course name",}
\NormalTok{)}

\NormalTok{\#lorem(120)}
\end{Highlighting}
\end{Shaded}

\subsection{Missing features}\label{missing-features}

As of now, this template has not yet been particularly optimized for
styling related to:

\begin{itemize}
\tightlist
\item
  Bibliographies
\item
  Outlines (e.g. table of contents)
\item
  Tables
\end{itemize}

\subsection{Credits}\label{credits}

This template started out by emulating the layout of course documents in
\href{https://liu.se/en/employee/marku61}{Marco Kuhlmann} ’s courses
at Linköping University.{[}\^{}2{]} On the technical side, this
template took a lot of inspiration from
\href{https://github.com/talal/ilm/}{the \texttt{\ ilm\ } template} ,
even if the design decisions may be radically different.

{[}\^{}1{]}: The \href{https://github.com/source-foundry/Hack}{Hack
font} is currently not available on the Typst web app, so the fallback
is Source Code Pro. {[}\^{}2{]}: If you work at Linköping University,
you can set \texttt{\ headings-font:\ "KorolevLiU"\ } to get a
LiU-branded version of this template.

\href{/app?template=kunskap&version=0.1.0}{Create project in app}

\subsubsection{How to use}\label{how-to-use}

Click the button above to create a new project using this template in
the Typst app.

You can also use the Typst CLI to start a new project on your computer
using this command:

\begin{verbatim}
typst init @preview/kunskap:0.1.0
\end{verbatim}

\includesvg[width=0.16667in,height=0.16667in]{/assets/icons/16-copy.svg}

\subsubsection{About}\label{about}

\begin{description}
\tightlist
\item[Author :]
\href{mailto:marcel@bollmann.me}{Marcel Bollmann}
\item[License:]
MIT-0
\item[Current version:]
0.1.0
\item[Last updated:]
October 30, 2024
\item[First released:]
October 30, 2024
\item[Minimum Typst version:]
0.12.0
\item[Archive size:]
4.16 kB
\href{https://packages.typst.org/preview/kunskap-0.1.0.tar.gz}{\pandocbounded{\includesvg[keepaspectratio]{/assets/icons/16-download.svg}}}
\item[Repository:]
\href{https://github.com/mbollmann/typst-kunskap}{GitHub}
\item[Categor y :]
\begin{itemize}
\tightlist
\item[]
\item
  \pandocbounded{\includesvg[keepaspectratio]{/assets/icons/16-speak.svg}}
  \href{https://typst.app/universe/search/?category=report}{Report}
\end{itemize}
\end{description}

\subsubsection{Where to report issues?}\label{where-to-report-issues}

This template is a project of Marcel Bollmann . Report issues on
\href{https://github.com/mbollmann/typst-kunskap}{their repository} .
You can also try to ask for help with this template on the
\href{https://forum.typst.app}{Forum} .

Please report this template to the Typst team using the
\href{https://typst.app/contact}{contact form} if you believe it is a
safety hazard or infringes upon your rights.

\phantomsection\label{versions}
\subsubsection{Version history}\label{version-history}

\begin{longtable}[]{@{}ll@{}}
\toprule\noalign{}
Version & Release Date \\
\midrule\noalign{}
\endhead
\bottomrule\noalign{}
\endlastfoot
0.1.0 & October 30, 2024 \\
\end{longtable}

Typst GmbH did not create this template and cannot guarantee correct
functionality of this template or compatibility with any version of the
Typst compiler or app.


\section{Package List LaTeX/exzellenz-tum-thesis.tex}
\title{typst.app/universe/package/exzellenz-tum-thesis}

\phantomsection\label{banner}
\phantomsection\label{template-thumbnail}
\pandocbounded{\includegraphics[keepaspectratio]{https://packages.typst.org/preview/thumbnails/exzellenz-tum-thesis-0.1.0-small.webp}}

\section{exzellenz-tum-thesis}\label{exzellenz-tum-thesis}

{ 0.1.0 }

Customizable template for a thesis at the TU Munich

\href{/app?template=exzellenz-tum-thesis&version=0.1.0}{Create project
in app}

\phantomsection\label{readme}
This is a Typst template for a thesis at TU Munich. I made it for my
thesis in the School CIT, but I think it can be adapted to other schools
as well.

\subsection{Usage}\label{usage}

You can use this template in the Typst web app by clicking “Start from
template� on the dashboard and searching for
\texttt{\ exzellenz-tum-thesis\ } .

Alternatively, you can use the CLI to kick this project off using the
command

\begin{verbatim}
typst init @preview/exzellenz-tum-thesis
\end{verbatim}

Typst will create a new directory with all the files needed to get you
started.

\subsection{Configuration}\label{configuration}

This template exports the \texttt{\ exzellenz-tum-thesis\ } function
with the following named arguments:

\begin{itemize}
\tightlist
\item
  \texttt{\ degree\ } : String
\item
  \texttt{\ program\ } : String
\item
  \texttt{\ school\ } : String
\item
  \texttt{\ supervisor\ } : String
\item
  \texttt{\ advisor\ } : Array of Strings
\item
  \texttt{\ author\ } : String
\item
  \texttt{\ startDate\ } : String
\item
  \texttt{\ titleEn\ } : String
\item
  \texttt{\ titleDe\ } : String
\item
  \texttt{\ abstractEn\ } : Content block
\item
  \texttt{\ abstractDe\ } : Content block
\item
  \texttt{\ acknowledgements\ } : Content block
\item
  \texttt{\ submissionDate\ } : String
\item
  \texttt{\ showTitleInHeader\ } : Boolean
\item
  \texttt{\ draft\ } : Boolean
\end{itemize}

The template will initialize your package with a sample call to the
\texttt{\ exzellenz-tum-thesis\ } function.

\href{/app?template=exzellenz-tum-thesis&version=0.1.0}{Create project
in app}

\subsubsection{How to use}\label{how-to-use}

Click the button above to create a new project using this template in
the Typst app.

You can also use the Typst CLI to start a new project on your computer
using this command:

\begin{verbatim}
typst init @preview/exzellenz-tum-thesis:0.1.0
\end{verbatim}

\includesvg[width=0.16667in,height=0.16667in]{/assets/icons/16-copy.svg}

\subsubsection{About}\label{about}

\begin{description}
\tightlist
\item[Author :]
\href{https://www.linkedin.com/in/fabian-scherer-de/}{Fabian Scherer}
\item[License:]
MIT-0
\item[Current version:]
0.1.0
\item[Last updated:]
April 8, 2024
\item[First released:]
April 8, 2024
\item[Minimum Typst version:]
0.11.0
\item[Archive size:]
8.18 kB
\href{https://packages.typst.org/preview/exzellenz-tum-thesis-0.1.0.tar.gz}{\pandocbounded{\includesvg[keepaspectratio]{/assets/icons/16-download.svg}}}
\item[Categor y :]
\begin{itemize}
\tightlist
\item[]
\item
  \pandocbounded{\includesvg[keepaspectratio]{/assets/icons/16-mortarboard.svg}}
  \href{https://typst.app/universe/search/?category=thesis}{Thesis}
\end{itemize}
\end{description}

\subsubsection{Where to report issues?}\label{where-to-report-issues}

This template is a project of Fabian Scherer . You can also try to ask
for help with this template on the \href{https://forum.typst.app}{Forum}
.

Please report this template to the Typst team using the
\href{https://typst.app/contact}{contact form} if you believe it is a
safety hazard or infringes upon your rights.

\phantomsection\label{versions}
\subsubsection{Version history}\label{version-history}

\begin{longtable}[]{@{}ll@{}}
\toprule\noalign{}
Version & Release Date \\
\midrule\noalign{}
\endhead
\bottomrule\noalign{}
\endlastfoot
0.1.0 & April 8, 2024 \\
\end{longtable}

Typst GmbH did not create this template and cannot guarantee correct
functionality of this template or compatibility with any version of the
Typst compiler or app.


\section{Package List LaTeX/whalogen.tex}
\title{typst.app/universe/package/whalogen}

\phantomsection\label{banner}
\section{whalogen}\label{whalogen}

{ 0.2.0 }

Typesetting chemical formulae, a port of mhchem

\phantomsection\label{readme}
whalogen is a library for typsetting chemical formulae with Typst,
inspired by mhchem.

GitHub repository: \url{https://github.com/schang412/typst-whalogen}

\subsection{Examples}\label{examples}

\pandocbounded{\includegraphics[keepaspectratio]{https://github.com/typst/packages/raw/main/packages/preview/whalogen/0.2.0/gallery/example.png}}

\begin{Shaded}
\begin{Highlighting}[]
\NormalTok{\#import "@preview/whalogen:0.2.0": ce}

\NormalTok{$ \#ce("HCl + H2O {-}\textgreater{} H3O+ + Cl{-}") $}
\end{Highlighting}
\end{Shaded}

See the
\href{https://github.com/typst/packages/raw/main/packages/preview/whalogen/0.2.0/manual.pdf}{manual}
for more details and examples.

\subsubsection{How to add}\label{how-to-add}

Copy this into your project and use the import as \texttt{\ whalogen\ }

\begin{verbatim}
#import "@preview/whalogen:0.2.0"
\end{verbatim}

\includesvg[width=0.16667in,height=0.16667in]{/assets/icons/16-copy.svg}

Check the docs for
\href{https://typst.app/docs/reference/scripting/\#packages}{more
information on how to import packages} .

\subsubsection{About}\label{about}

\begin{description}
\tightlist
\item[Author :]
\href{mailto:spencer@sycee.xyz}{Spencer Chang}
\item[License:]
Apache-2.0
\item[Current version:]
0.2.0
\item[Last updated:]
April 30, 2024
\item[First released:]
July 18, 2023
\item[Archive size:]
6.45 kB
\href{https://packages.typst.org/preview/whalogen-0.2.0.tar.gz}{\pandocbounded{\includesvg[keepaspectratio]{/assets/icons/16-download.svg}}}
\item[Repository:]
\href{https://github.com/schang412/typst-whalogen}{GitHub}
\end{description}

\subsubsection{Where to report issues?}\label{where-to-report-issues}

This package is a project of Spencer Chang . Report issues on
\href{https://github.com/schang412/typst-whalogen}{their repository} .
You can also try to ask for help with this package on the
\href{https://forum.typst.app}{Forum} .

Please report this package to the Typst team using the
\href{https://typst.app/contact}{contact form} if you believe it is a
safety hazard or infringes upon your rights.

\phantomsection\label{versions}
\subsubsection{Version history}\label{version-history}

\begin{longtable}[]{@{}ll@{}}
\toprule\noalign{}
Version & Release Date \\
\midrule\noalign{}
\endhead
\bottomrule\noalign{}
\endlastfoot
0.2.0 & April 30, 2024 \\
\href{https://typst.app/universe/package/whalogen/0.1.0/}{0.1.0} & July
18, 2023 \\
\end{longtable}

Typst GmbH did not create this package and cannot guarantee correct
functionality of this package or compatibility with any version of the
Typst compiler or app.


\section{Package List LaTeX/pavemat.tex}
\title{typst.app/universe/package/pavemat}

\phantomsection\label{banner}
\section{pavemat}\label{pavemat}

{ 0.1.0 }

Style matrices with custom paths, strokes and fills for appealing
visualizations.

{ } Featured Package

\phantomsection\label{readme}
\pandocbounded{\includesvg[keepaspectratio]{https://github.com/typst/packages/raw/main/packages/preview/pavemat/0.1.0/examples/logo.svg}}

repo: \url{https://github.com/QuadnucYard/pavemat}

\subsection{Introduction}\label{introduction}

The \emph{pavemat} is a tool for creating styled matrices with custom
paths, strokes, and fills. It allows users to define how paths should be
drawn through the matrix, apply different strokes to these paths, and
fill specific cells with various colors. This function is particularly
useful for visualizing complex data structures, mathematical matrices,
and creating custom grid layouts.

\subsection{Examples}\label{examples}

The logo example:

\begin{Shaded}
\begin{Highlighting}[]
\NormalTok{\#\{}
\NormalTok{  set math.mat(row{-}gap: 0.25em, column{-}gap: 0.1em)}
\NormalTok{  set text(size: 2em)}

\NormalTok{  pavemat(}
\NormalTok{    pave: (}
\NormalTok{      "SDS(dash: \textquotesingle{}solid\textquotesingle{})DDD]WW",}
\NormalTok{      (path: "sdDDD", stroke: aqua.darken(30\%))}
\NormalTok{    ),}
\NormalTok{    stroke: (dash: "dashed", thickness: 1pt, paint: yellow),}
\NormalTok{    fills: (}
\NormalTok{      "0{-}0": green.transparentize(80\%),}
\NormalTok{      "1{-}1": blue.transparentize(80\%),}
\NormalTok{      "[0{-}0]": green.transparentize(60\%),}
\NormalTok{      "[1{-}1]": blue.transparentize(60\%),}
\NormalTok{    ),}
\NormalTok{    delim: "[",}
\NormalTok{  )[$mat(P, a, v, e; "", m, a, t)$]}
\NormalTok{\}}
\end{Highlighting}
\end{Shaded}

Code of examples can be found in
\href{https://github.com/QuadnucYard/pavemat/tree/main/examples}{\texttt{\ examples/examples.typ\ }}
.

\pandocbounded{\includesvg[keepaspectratio]{https://github.com/typst/packages/raw/main/packages/preview/pavemat/0.1.0/examples/example1.svg}}
\pandocbounded{\includesvg[keepaspectratio]{https://github.com/typst/packages/raw/main/packages/preview/pavemat/0.1.0/examples/example2.svg}}
\pandocbounded{\includesvg[keepaspectratio]{https://github.com/typst/packages/raw/main/packages/preview/pavemat/0.1.0/examples/example4.svg}}
\pandocbounded{\includesvg[keepaspectratio]{https://github.com/typst/packages/raw/main/packages/preview/pavemat/0.1.0/examples/example5.svg}}

\subsection{Manual}\label{manual}

See
\href{https://github.com/QuadnucYard/pavemat/tree/main/docs}{\texttt{\ docs/manual.typ\ }}
.

\subsubsection{How to add}\label{how-to-add}

Copy this into your project and use the import as \texttt{\ pavemat\ }

\begin{verbatim}
#import "@preview/pavemat:0.1.0"
\end{verbatim}

\includesvg[width=0.16667in,height=0.16667in]{/assets/icons/16-copy.svg}

Check the docs for
\href{https://typst.app/docs/reference/scripting/\#packages}{more
information on how to import packages} .

\subsubsection{About}\label{about}

\begin{description}
\tightlist
\item[Author :]
\href{https://github.com/QuadnucYard}{QuadnucYard}
\item[License:]
MIT
\item[Current version:]
0.1.0
\item[Last updated:]
July 29, 2024
\item[First released:]
July 29, 2024
\item[Archive size:]
3.60 kB
\href{https://packages.typst.org/preview/pavemat-0.1.0.tar.gz}{\pandocbounded{\includesvg[keepaspectratio]{/assets/icons/16-download.svg}}}
\item[Repository:]
\href{https://github.com/QuadnucYard/pavemat}{GitHub}
\item[Categor y :]
\begin{itemize}
\tightlist
\item[]
\item
  \pandocbounded{\includesvg[keepaspectratio]{/assets/icons/16-chart.svg}}
  \href{https://typst.app/universe/search/?category=visualization}{Visualization}
\end{itemize}
\end{description}

\subsubsection{Where to report issues?}\label{where-to-report-issues}

This package is a project of QuadnucYard . Report issues on
\href{https://github.com/QuadnucYard/pavemat}{their repository} . You
can also try to ask for help with this package on the
\href{https://forum.typst.app}{Forum} .

Please report this package to the Typst team using the
\href{https://typst.app/contact}{contact form} if you believe it is a
safety hazard or infringes upon your rights.

\phantomsection\label{versions}
\subsubsection{Version history}\label{version-history}

\begin{longtable}[]{@{}ll@{}}
\toprule\noalign{}
Version & Release Date \\
\midrule\noalign{}
\endhead
\bottomrule\noalign{}
\endlastfoot
0.1.0 & July 29, 2024 \\
\end{longtable}

Typst GmbH did not create this package and cannot guarantee correct
functionality of this package or compatibility with any version of the
Typst compiler or app.


\section{Package List LaTeX/stonewall.tex}
\title{typst.app/universe/package/stonewall}

\phantomsection\label{banner}
\section{stonewall}\label{stonewall}

{ 0.1.0 }

Stonewall provides beautiful pride flag colours for gradients.

\phantomsection\label{readme}
You can use the colour palette with \emph{gradients} for maximum
results! For example the code in \texttt{\ example/example.typ\ } which
is

\begin{Shaded}
\begin{Highlighting}[]
\NormalTok{\#import "@preview/stonewall:0.1.0": flags}

\NormalTok{\#set page(width: 200pt, height: auto, margin: 0pt)}
\NormalTok{\#set text(fill: black, size: 12pt)}
\NormalTok{\#set text(top{-}edge: "bounds", bottom{-}edge: "bounds")}


\NormalTok{\#stack(}
\NormalTok{  spacing: 3pt,}
\NormalTok{  ..flags.map(((name, preset)) =\textgreater{} block(}
\NormalTok{    width: 100\%,}
\NormalTok{    height: 20pt,}
\NormalTok{    fill: gradient.linear(..preset),}
\NormalTok{    align(center + horizon, smallcaps(name)),}
\NormalTok{  ))}
\NormalTok{)}
\end{Highlighting}
\end{Shaded}

gives the following stack of flags as of v0.1.0
\pandocbounded{\includegraphics[keepaspectratio]{https://github.com/typst/packages/raw/main/packages/preview/stonewall/0.1.0/flags.png}}

To use only one flag you only import the one you want

\subsubsection{How to add}\label{how-to-add}

Copy this into your project and use the import as \texttt{\ stonewall\ }

\begin{verbatim}
#import "@preview/stonewall:0.1.0"
\end{verbatim}

\includesvg[width=0.16667in,height=0.16667in]{/assets/icons/16-copy.svg}

Check the docs for
\href{https://typst.app/docs/reference/scripting/\#packages}{more
information on how to import packages} .

\subsubsection{About}\label{about}

\begin{description}
\tightlist
\item[Author :]
Charlotte Thomas
\item[License:]
GPL-3.0-or-later
\item[Current version:]
0.1.0
\item[Last updated:]
November 7, 2023
\item[First released:]
November 7, 2023
\item[Minimum Typst version:]
0.9.0
\item[Archive size:]
14.3 kB
\href{https://packages.typst.org/preview/stonewall-0.1.0.tar.gz}{\pandocbounded{\includesvg[keepaspectratio]{/assets/icons/16-download.svg}}}
\item[Repository:]
\href{https://github.com/coco33920/stonewall}{GitHub}
\end{description}

\subsubsection{Where to report issues?}\label{where-to-report-issues}

This package is a project of Charlotte Thomas . Report issues on
\href{https://github.com/coco33920/stonewall}{their repository} . You
can also try to ask for help with this package on the
\href{https://forum.typst.app}{Forum} .

Please report this package to the Typst team using the
\href{https://typst.app/contact}{contact form} if you believe it is a
safety hazard or infringes upon your rights.

\phantomsection\label{versions}
\subsubsection{Version history}\label{version-history}

\begin{longtable}[]{@{}ll@{}}
\toprule\noalign{}
Version & Release Date \\
\midrule\noalign{}
\endhead
\bottomrule\noalign{}
\endlastfoot
0.1.0 & November 7, 2023 \\
\end{longtable}

Typst GmbH did not create this package and cannot guarantee correct
functionality of this package or compatibility with any version of the
Typst compiler or app.


\section{Package List LaTeX/splash.tex}
\title{typst.app/universe/package/splash}

\phantomsection\label{banner}
\section{splash}\label{splash}

{ 0.4.0 }

A library of color palettes for Typst.

\phantomsection\label{readme}
Add a splash of color to your project with these palettes for
\href{https://github.com/typst/typst}{Typst} .

This library provides different color palettes with human-readable names
in Typst dictionaries. Currently there are just a few different palettes
to choose from. Any contributions or suggestions are welcome!

\emph{Note} : \texttt{\ splash\ } is in the
\href{https://github.com/typst/packages}{Typst Package Repository} . See
how to use it in the example below.

\subsection{Usage}\label{usage}

\begin{Shaded}
\begin{Highlighting}[]
\NormalTok{\#import "@preview/splash:0.4.0": xcolor}

\NormalTok{\#box(width: 3em, height: 1em, fill: xcolor.dandelion)}
\end{Highlighting}
\end{Shaded}

\subsection{Documentation}\label{documentation}

See the different colors in the
\href{https://github.com/kaarmu/splash/blob/v0.4.0/doc/main.pdf}{documentation}
.

\subsubsection{How to add}\label{how-to-add}

Copy this into your project and use the import as \texttt{\ splash\ }

\begin{verbatim}
#import "@preview/splash:0.4.0"
\end{verbatim}

\includesvg[width=0.16667in,height=0.16667in]{/assets/icons/16-copy.svg}

Check the docs for
\href{https://typst.app/docs/reference/scripting/\#packages}{more
information on how to import packages} .

\subsubsection{About}\label{about}

\begin{description}
\tightlist
\item[Author :]
kaarmu
\item[License:]
MIT
\item[Current version:]
0.4.0
\item[Last updated:]
July 3, 2024
\item[First released:]
July 2, 2023
\item[Archive size:]
16.3 kB
\href{https://packages.typst.org/preview/splash-0.4.0.tar.gz}{\pandocbounded{\includesvg[keepaspectratio]{/assets/icons/16-download.svg}}}
\item[Repository:]
\href{https://github.com/kaarmu/typst-palettes}{GitHub}
\end{description}

\subsubsection{Where to report issues?}\label{where-to-report-issues}

This package is a project of kaarmu . Report issues on
\href{https://github.com/kaarmu/typst-palettes}{their repository} . You
can also try to ask for help with this package on the
\href{https://forum.typst.app}{Forum} .

Please report this package to the Typst team using the
\href{https://typst.app/contact}{contact form} if you believe it is a
safety hazard or infringes upon your rights.

\phantomsection\label{versions}
\subsubsection{Version history}\label{version-history}

\begin{longtable}[]{@{}ll@{}}
\toprule\noalign{}
Version & Release Date \\
\midrule\noalign{}
\endhead
\bottomrule\noalign{}
\endlastfoot
0.4.0 & July 3, 2024 \\
\href{https://typst.app/universe/package/splash/0.3.0/}{0.3.0} & July 2,
2023 \\
\end{longtable}

Typst GmbH did not create this package and cannot guarantee correct
functionality of this package or compatibility with any version of the
Typst compiler or app.


\section{Package List LaTeX/aloecius-aip.tex}
\title{typst.app/universe/package/aloecius-aip}

\phantomsection\label{banner}
\phantomsection\label{template-thumbnail}
\pandocbounded{\includegraphics[keepaspectratio]{https://packages.typst.org/preview/thumbnails/aloecius-aip-0.0.1-small.webp}}

\section{aloecius-aip}\label{aloecius-aip}

{ 0.0.1 }

Typst template for reproducing AIP - Journal of Chemical Physics paper
(draft)

\href{/app?template=aloecius-aip&version=0.0.1}{Create project in app}

\phantomsection\label{readme}
This is a typst template for reproducing papers of American Institute of
Physics (AIP) publishing house, mainly draft version of Journal of
Chemical Physics. This is inspired by the overleaf
\$\textbackslash LaTeX\$ template of AIP journals.

\subsection{Usage}\label{usage}

You can use this template with typst web app by simply clicking on
“Start from template� on the dashboard and searching for
\texttt{\ aloecius-aip\ } .

For local usage, you can use the typst CLI by invoking the following
command

\begin{verbatim}
typst init @preview/aloecius-aip
\end{verbatim}

typst will automatically create a new directory with all the necessary
files needed to compile the project.

\subsection{Configuration}\label{configuration}

The preamble or the header of the document should be written in the
following way with your own necessary input variables to recreate the
same formatting as seen in the
\href{https://github.com/typst/packages/raw/main/packages/preview/aloecius-aip/0.0.1/sample.pdf}{\texttt{\ sample.pdf\ }}

\begin{verbatim}
#import "@preview/aloecius-aip:0.0.1": *

#show: article.with(
  title: "Typst Template for Journal of Chemical Physics (Draft)",
  authors: (
    "Author 1": author-meta(
      "GU",
      email: "user1@domain.com",
    ),
    "Author 2": author-meta(
      "GU",
      cofirst: false
    ),
    "Author 3": author-meta(
      "UG"
    )
  ),
  affiliations: (
    "UG": "University of Global Sciences",
    "GU": "Institute for Chemistry, Global University of Sciences"
  ),
  abstract: [
  Here goes the abstract. 
  ],
  bib: bibliography("./reference.bib")
)
\end{verbatim}

\subsection{Important Variables}\label{important-variables}

\begin{itemize}
\tightlist
\item
  \texttt{\ title\ } : Title of the paper
\item
  \texttt{\ authors\ } : A dictionary connecting the key as name of the
  author(s) and the value to be the affiliation of them including
  university, email, mail address, authorship and an alias, an example
  usage is shown below
\end{itemize}

\begin{verbatim}
Example:
(
  "Author Name": (
    "affiliation": "affiliation-label",
    "email": "author.name@example.com", // Optional
    "address": "Mail address",  // Optional
    "name": "Alias Name", // Optional
    "cofirst": false // Optional, identify whether this author is the co-first author
    )
)
\end{verbatim}

\begin{itemize}
\tightlist
\item
  \texttt{\ affiliations\ } : Dictionary of affiliations where keys are
  affiliations labels and values are affiliations addresses, and example
  usage is as follows
\end{itemize}

\begin{verbatim}
Example:
 (
    "affiliation-label": "Institution Name, University Name, Road, Post Code, Country"
 )
\end{verbatim}

\begin{itemize}
\tightlist
\item
  \texttt{\ abstract\ } : Abstract of the paper
\item
  \texttt{\ bib\ } : passing the bibliography file wrapped into the
  typst \texttt{\ bibliography()\ } function, both
  \texttt{\ Hayagriva\ } and \texttt{\ .bib\ } format is supported.
\end{itemize}

\href{/app?template=aloecius-aip&version=0.0.1}{Create project in app}

\subsubsection{How to use}\label{how-to-use}

Click the button above to create a new project using this template in
the Typst app.

You can also use the Typst CLI to start a new project on your computer
using this command:

\begin{verbatim}
typst init @preview/aloecius-aip:0.0.1
\end{verbatim}

\includesvg[width=0.16667in,height=0.16667in]{/assets/icons/16-copy.svg}

\subsubsection{About}\label{about}

\begin{description}
\tightlist
\item[Author :]
Raunak Farhaz
\item[License:]
MIT
\item[Current version:]
0.0.1
\item[Last updated:]
July 3, 2024
\item[First released:]
July 3, 2024
\item[Minimum Typst version:]
0.11.1
\item[Archive size:]
13.7 kB
\href{https://packages.typst.org/preview/aloecius-aip-0.0.1.tar.gz}{\pandocbounded{\includesvg[keepaspectratio]{/assets/icons/16-download.svg}}}
\item[Repository:]
\href{https://github.com/Raunak12775/aloecius-aip}{GitHub}
\item[Discipline s :]
\begin{itemize}
\tightlist
\item[]
\item
  \href{https://typst.app/universe/search/?discipline=chemistry}{Chemistry}
\item
  \href{https://typst.app/universe/search/?discipline=physics}{Physics}
\item
  \href{https://typst.app/universe/search/?discipline=mathematics}{Mathematics}
\end{itemize}
\item[Categor y :]
\begin{itemize}
\tightlist
\item[]
\item
  \pandocbounded{\includesvg[keepaspectratio]{/assets/icons/16-atom.svg}}
  \href{https://typst.app/universe/search/?category=paper}{Paper}
\end{itemize}
\end{description}

\subsubsection{Where to report issues?}\label{where-to-report-issues}

This template is a project of Raunak Farhaz . Report issues on
\href{https://github.com/Raunak12775/aloecius-aip}{their repository} .
You can also try to ask for help with this template on the
\href{https://forum.typst.app}{Forum} .

Please report this template to the Typst team using the
\href{https://typst.app/contact}{contact form} if you believe it is a
safety hazard or infringes upon your rights.

\phantomsection\label{versions}
\subsubsection{Version history}\label{version-history}

\begin{longtable}[]{@{}ll@{}}
\toprule\noalign{}
Version & Release Date \\
\midrule\noalign{}
\endhead
\bottomrule\noalign{}
\endlastfoot
0.0.1 & July 3, 2024 \\
\end{longtable}

Typst GmbH did not create this template and cannot guarantee correct
functionality of this template or compatibility with any version of the
Typst compiler or app.


\section{Package List LaTeX/quick-minutes.tex}
\title{typst.app/universe/package/quick-minutes}

\phantomsection\label{banner}
\phantomsection\label{template-thumbnail}
\pandocbounded{\includegraphics[keepaspectratio]{https://packages.typst.org/preview/thumbnails/quick-minutes-1.2.0-small.webp}}

\section{quick-minutes}\label{quick-minutes}

{ 1.2.0 }

A typst template for the keeping of minutes.

\href{/app?template=quick-minutes&version=1.2.0}{Create project in app}

\phantomsection\label{readme}
With Quick Minutes you can record any meeting event by just typing it
out. No function calls needed!

\subsection{Usage}\label{usage}

\subsubsection{Import \& Initialisation}\label{import-initialisation}

\begin{Shaded}
\begin{Highlighting}[]
\NormalTok{\#import "@preview/quick{-}minutes:1.2.0": *}

\NormalTok{\#show: minutes.with(}
\NormalTok{  chairperson: "Name 1",}
\NormalTok{  secretary: "Name 2",}
\NormalTok{  date: auto,}
\NormalTok{  body{-}name: "Organisation",}
\NormalTok{  event{-}name: "Event",}
\NormalTok{  present: (}
\NormalTok{    "Name 1",}
\NormalTok{    "Name 2",}
\NormalTok{    "Name 3",}
\NormalTok{    "Name 4",}
\NormalTok{  )}
\NormalTok{)}

\NormalTok{...}
\end{Highlighting}
\end{Shaded}

\href{https://github.com/typst/packages/raw/main/packages/preview/quick-minutes/1.2.0/\#commands}{Commands}\\
\href{https://github.com/typst/packages/raw/main/packages/preview/quick-minutes/1.2.0/\#formats}{Formats}

\subsubsection{Parameters}\label{parameters}

\begin{longtable}[]{@{}
  >{\raggedright\arraybackslash}p{(\linewidth - 8\tabcolsep) * \real{0.2000}}
  >{\raggedright\arraybackslash}p{(\linewidth - 8\tabcolsep) * \real{0.2000}}
  >{\raggedright\arraybackslash}p{(\linewidth - 8\tabcolsep) * \real{0.2000}}
  >{\raggedright\arraybackslash}p{(\linewidth - 8\tabcolsep) * \real{0.2000}}
  >{\raggedright\arraybackslash}p{(\linewidth - 8\tabcolsep) * \real{0.2000}}@{}}
\toprule\noalign{}
\begin{minipage}[b]{\linewidth}\raggedright
\end{minipage} & \begin{minipage}[b]{\linewidth}\raggedright
name
\end{minipage} & \begin{minipage}[b]{\linewidth}\raggedright
explaination
\end{minipage} & \begin{minipage}[b]{\linewidth}\raggedright
default
\end{minipage} & \begin{minipage}[b]{\linewidth}\raggedright
type
\end{minipage} \\
\midrule\noalign{}
\endhead
\bottomrule\noalign{}
\endlastfoot
required & & & & \\
& & & & \\
& body-name & Name of the body holding the recorded meeting &
\texttt{\ none\ } & \texttt{\ string\ } \\
& event-name & Name of the meeting & \texttt{\ none\ } &
\texttt{\ string\ } \\
& date & Date of the meeting ( \texttt{\ auto\ } for current date,
datetime for formatted date) & \texttt{\ none\ } &
\texttt{\ string,\ auto,\ datetime\ } \\
& present & List with names of people present at the meeting &
\texttt{\ ()\ } & list \\
& chairperson & \begin{minipage}[t]{\linewidth}\raggedright
Name of the person chairing the meeting\\
Can be a \texttt{\ list\ } of people\strut
\end{minipage} & \texttt{\ none\ } & \texttt{\ string\ } ,
\texttt{\ list(string)\ } \\
& secretary & \begin{minipage}[t]{\linewidth}\raggedright
Name of the person taking minutes\\
Can be a \texttt{\ list\ } of people\strut
\end{minipage} & \texttt{\ none\ } & \texttt{\ string\ } ,
\texttt{\ list(string)\ } \\
optional & & & & \\
& & & & \\
& awareness & \begin{minipage}[t]{\linewidth}\raggedright
Name of the person responsible for awareness\\
Can be a \texttt{\ list\ } of people\strut
\end{minipage} & \texttt{\ none\ } & \texttt{\ string\ } ,
\texttt{\ list(string)\ } \\
& translation & \begin{minipage}[t]{\linewidth}\raggedright
Name of the person responsible for translating\\
Can be a \texttt{\ list\ } of people\strut
\end{minipage} & \texttt{\ none\ } & \texttt{\ string\ } ,
\texttt{\ list(string)\ } \\
& cosigner & Position of the Person signing the protocol, should they
differ from the chairperson & \texttt{\ none\ } & \texttt{\ string\ } \\
& cosigner-name & Name of the person signing the protocol, should they
differ from the chairperson & \texttt{\ none\ } & \texttt{\ string\ } \\
customisation & & & & \\
& & & & \\
& logo & Logo of the body holding the meeting & \texttt{\ none\ } &
\texttt{\ image\ } \\
& custom-header & \begin{minipage}[t]{\linewidth}\raggedright
Custom Header\\
\strut \\
Arguments:\\
(date, body-name, event-name, translate)\\
\strut \\
Set to \texttt{\ none\ } for empty header\strut
\end{minipage} & \texttt{\ auto\ } &
\texttt{\ function(content,\ content,\ content,\ function(string,\ ..string))\ }
, \texttt{\ auto\ } \\
& custom-footer & \begin{minipage}[t]{\linewidth}\raggedright
Custom Footer\\
\strut \\
Arguments:\\
(current-page, page-count, translate)\\
\strut \\
Set to \texttt{\ none\ } for empty footer\\
\strut \\
Is called inside of \texttt{\ context\ }\strut
\end{minipage} & \texttt{\ auto\ } &
\texttt{\ function(int,\ int,\ function(string,\ ..string))\ } ,
\texttt{\ auto\ } \\
& custom-background & \begin{minipage}[t]{\linewidth}\raggedright
Custom Background\\
\strut \\
Arguments:\\
(hole-mark)\\
\strut \\
Set to \texttt{\ none\ } for empty background\strut
\end{minipage} & \texttt{\ auto\ } & \texttt{\ function(bool)\ } ,
\texttt{\ auto\ } \\
& custom-head-section & \begin{minipage}[t]{\linewidth}\raggedright
Custom Head Section\\
\strut \\
Arguments:\\
(chairperson, secretary, awareness, translation, present, present-count,
start-time, end-time, translate, four-digits-to-time)\\
\strut \\
Set to \texttt{\ none\ } for empty head section\\
\strut \\
Handle start-time \& end-time like this:\\

\begin{verbatim}
let start-time = start-time.final()
if (start-time != none) [*#translate(“START”)*: #four-digits-to-time(start-time)\ ]
\end{verbatim}
\strut
\end{minipage} & \texttt{\ auto\ } &
\texttt{\ function(content,\ content,\ content/none,\ content/none,\ content,\ int/none,\ state,\ state,\ function(string,\ ..string),\ function(string))\ }
, \texttt{\ auto\ } \\
& custom-name-format & Formatting of names in the document &
\begin{minipage}[t]{\linewidth}\raggedright
\begin{verbatim}
(first-name, last-name, numbered) => [
 #if (numbered) [#first-name #last-name] else [
  #if (last-name != none) [#last-name, ]#first-name]
]
\end{verbatim}
\end{minipage} & \texttt{\ function(string,\ string,\ bool)\ } \\
& custom-name-style & Style of names in the document &
\begin{minipage}[t]{\linewidth}\raggedright
\begin{verbatim}
(name) => [name]
\end{verbatim}
\end{minipage} & \texttt{\ function(string)\ } \\
& item-numbering & Numbering of items. Reverts to
\texttt{\ DEFAULT\_ITEM\_NUMBERING\ } if \texttt{\ none\ } . &
\texttt{\ none\ } & \texttt{\ function(..int)\ } \\
& time-format & Datetime format \texttt{\ string\ } for times taken.
Reverts to \texttt{\ DEFAULT\_TIME\_FORMAT\ } if \texttt{\ none\ } . &
\texttt{\ none\ } & \texttt{\ string\ } \\
& date-format & Datetime format \texttt{\ string\ } for the date of the
event. Reverts to \texttt{\ DEFAULT\_DATE\_FORMAT\ } if
\texttt{\ none\ } . & \texttt{\ none\ } & \texttt{\ string\ } \\
& timestamp-margin & Size of gutter between timestamps and text &
\texttt{\ 10pt\ } & \begin{minipage}[t]{\linewidth}\raggedright
\texttt{\ length\ }\strut \\
(static (pt, cm …) recommended)\strut
\end{minipage} \\
& line-numbering & \texttt{\ none\ } for no line numbering,
\texttt{\ int\ } for every xth line numbered & \texttt{\ 5\ } &
\texttt{\ int\ } \\
& fancy-decisions & Draws a diagram underneath decisions &
\texttt{\ false\ } & \texttt{\ bool\ } \\
& fancy-dialogue & Splits dialogue up into paragraphs &
\texttt{\ false\ } & \texttt{\ bool\ } \\
& hole-mark & Draws a mark for the alignment of a hole punch &
\texttt{\ true\ } & \texttt{\ bool\ } \\
& separator-lines & Draws lines next to the titles & \texttt{\ true\ } &
\texttt{\ bool\ } \\
& signing & Do people have to sign this document? & \texttt{\ true\ } &
\texttt{\ bool\ } \\
& title-page & Should the actual protocol start after a
\texttt{\ pagebreak\ } ? & \texttt{\ false\ } & \texttt{\ bool\ } \\
& number-present & Should the number of people present be shown? &
\texttt{\ false\ } & \texttt{\ bool\ } \\
& show-arrival-time & Should the time of arrival be schown in the list
of people present? & \texttt{\ true\ } & \texttt{\ bool\ } \\
& locale & language of the document & \texttt{\ "en"\ } &
\texttt{\ string\ } \\
& translation-overrides &
\href{https://github.com/typst/packages/raw/main/packages/preview/quick-minutes/1.2.0/lang.json}{Translation}
Overrides & \texttt{\ (:)\ } & \texttt{\ dictionary\ } \\
& custom-royalty-connectors &
\begin{minipage}[t]{\linewidth}\raggedright
Additional \texttt{\ list\ } of surname beginnings like “von�.\\
Already recognises “von�, “von der� \& “de�\strut
\end{minipage} & \texttt{\ ()\ } & \texttt{\ list\ } \\
debug & & & & \\
& & & & \\
& display-all-warnings & Shows all warnings directly beneath their
occurence & \texttt{\ false\ } & \texttt{\ bool\ } \\
& hide-warnings & Hides all warnings & \texttt{\ false\ } &
\texttt{\ bool\ } \\
& warning-color & Color warnings are displayed in & \texttt{\ red\ } &
\texttt{\ color\ } \\
& enable-help-text & Should a help/debug text with state info be shown?
& \texttt{\ false\ } & \texttt{\ bool\ } \\
\end{longtable}

\subsubsection{Commands}\label{commands}

\begin{longtable}[]{@{}
  >{\raggedright\arraybackslash}p{(\linewidth - 4\tabcolsep) * \real{0.3333}}
  >{\raggedright\arraybackslash}p{(\linewidth - 4\tabcolsep) * \real{0.3333}}
  >{\raggedright\arraybackslash}p{(\linewidth - 4\tabcolsep) * \real{0.3333}}@{}}
\toprule\noalign{}
\begin{minipage}[b]{\linewidth}\raggedright
name
\end{minipage} & \begin{minipage}[b]{\linewidth}\raggedright
format
\end{minipage} & \begin{minipage}[b]{\linewidth}\raggedright
description
\end{minipage} \\
\midrule\noalign{}
\endhead
\bottomrule\noalign{}
\endlastfoot
Join & \begin{minipage}[t]{\linewidth}\raggedright
\begin{verbatim}
+(<time>/)?<name>
++(<time>/)?<name>
\end{verbatim}
\end{minipage} & \begin{minipage}[t]{\linewidth}\raggedright
Marks the arrival of someone\\
+: Come back from pause etc.\\
++: Arrive at event\\
\strut \\
\emph{Time is optional}\strut
\end{minipage} \\
Leave & \begin{minipage}[t]{\linewidth}\raggedright
\begin{verbatim}
-(<time>/)?<name>
–(<time>/)?<name>
\end{verbatim}
\end{minipage} & \begin{minipage}[t]{\linewidth}\raggedright
Marks the departure of someone.\\
-: Leave into pause etc.\\
â€``: Leave event\\
\strut \\
\emph{Time is optional}\strut
\end{minipage} \\
Time & \begin{minipage}[t]{\linewidth}\raggedright
\begin{verbatim}
<time>/<text>
\end{verbatim}
\end{minipage} & Time the following text \\
Mark Name & \begin{minipage}[t]{\linewidth}\raggedright
\begin{verbatim}
/<name>
\end{verbatim}
\end{minipage} & Marks following name \\
Vote & \begin{minipage}[t]{\linewidth}\raggedright
\begin{verbatim}
!(<time>/)?<text>/<vote>/<vote>…
\end{verbatim}
\end{minipage} & \begin{minipage}[t]{\linewidth}\raggedright
Vote on something (described in text)\\
\strut \\
/\textless vote\textgreater{} can be repeated as many times as needed
(min. 2)\\
3 unnamed \& uncolored votes will result in a “For� (green),
“Against� (red), “Abstain� (blue) vote\\
\strut \\
If you want to use \texttt{\ /\ } inside of a label or the text, you can
use \texttt{\ -/\ } to escape into a normal \texttt{\ /\ }\\
\strut \\
\emph{Time is optional}\strut
\end{minipage} \\
Dialogue & \begin{minipage}[t]{\linewidth}\raggedright
\begin{verbatim}
<name>: <text>
\end{verbatim}
\end{minipage} & \begin{minipage}[t]{\linewidth}\raggedright
Marks that someone is speaking\\
\strut \\
Can be escaped with a \texttt{\ -\ } (
\texttt{\ \textless{}name\textgreater{}-:\ } ) to avoid
restructuring\strut
\end{minipage} \\
End & \begin{minipage}[t]{\linewidth}\raggedright
\begin{verbatim}
/<time>
\end{verbatim}
\end{minipage} & \begin{minipage}[t]{\linewidth}\raggedright
End of the meeting.\\
\strut \\
Replace \textless time\textgreater{} with \texttt{\ "end"\ } if you dont
want to set a time\strut
\end{minipage} \\
\end{longtable}

\subsubsection{Formats}\label{formats}

\begin{longtable}[]{@{}
  >{\raggedright\arraybackslash}p{(\linewidth - 4\tabcolsep) * \real{0.3333}}
  >{\raggedright\arraybackslash}p{(\linewidth - 4\tabcolsep) * \real{0.3333}}
  >{\raggedright\arraybackslash}p{(\linewidth - 4\tabcolsep) * \real{0.3333}}@{}}
\toprule\noalign{}
\begin{minipage}[b]{\linewidth}\raggedright
name
\end{minipage} & \begin{minipage}[b]{\linewidth}\raggedright
format
\end{minipage} & \begin{minipage}[b]{\linewidth}\raggedright
example
\end{minipage} \\
\midrule\noalign{}
\endhead
\bottomrule\noalign{}
\endlastfoot
\textless vote\textgreater{} &
\begin{minipage}[t]{\linewidth}\raggedright
\begin{verbatim}
<label>(|<color>)?<count>
\end{verbatim}
\end{minipage} & \begin{minipage}[t]{\linewidth}\raggedright
\texttt{\ First\ Party\textbar{}green42\ }\strut \\
\texttt{\ Third\ choice22\ }\strut
\end{minipage} \\
\textless time\textgreater{} & 1-4 numbers &
\begin{minipage}[t]{\linewidth}\raggedright
\texttt{\ 1312\ } -\textgreater{} 01:12 pm\\
\texttt{\ 650\ } -\textgreater{} 06:50 pm\\
\texttt{\ 21\ } -\textgreater{} last timed hour:21 (pm/am)\\
\texttt{\ 4\ } -\textgreater{} last timed hour:04 (pm/am)\strut
\end{minipage} \\
\textless name\textgreater{} & Name in various formats &
\begin{minipage}[t]{\linewidth}\raggedright
\texttt{\ Last\ Name,\ First\ Name\ }\strut \\
\texttt{\ First\ Name\ Last\ Name\ }\strut \\
\texttt{\ First\ Name\ }\strut \\
\texttt{\ Last\ Name\ }\strut \\
\texttt{\ First\ Name\ L\ }\strut \\
\texttt{\ F\ Last\ Name\ }\strut \\
\texttt{\ FL\ }\strut \\
\strut \\
Last name can also start with a royalty connector like “de� or
“von�\\
\strut \\
\texttt{\ Name\ 1\ } , \texttt{\ Name\ 2\ } will render with the number
after the name, but the number is handled as a last name.\\
\strut \\
If you just want your name formatted by \texttt{\ custom-name-style\ }
you can escape the restructuring process with a \texttt{\ -\ } (
\texttt{\ /-\textless{}name\textgreater{}\ } ).\strut
\end{minipage} \\
\end{longtable}

\href{/app?template=quick-minutes&version=1.2.0}{Create project in app}

\subsubsection{How to use}\label{how-to-use}

Click the button above to create a new project using this template in
the Typst app.

You can also use the Typst CLI to start a new project on your computer
using this command:

\begin{verbatim}
typst init @preview/quick-minutes:1.2.0
\end{verbatim}

\includesvg[width=0.16667in,height=0.16667in]{/assets/icons/16-copy.svg}

\subsubsection{About}\label{about}

\begin{description}
\tightlist
\item[Author s :]
\href{https://github.com/katharinathoele}{Katharina Thöle} \&
\href{https://github.com/Lypsilonx}{Lyx Rothböck}
\item[License:]
MIT
\item[Current version:]
1.2.0
\item[Last updated:]
November 12, 2024
\item[First released:]
October 28, 2024
\item[Minimum Typst version:]
0.12.0
\item[Archive size:]
13.0 kB
\href{https://packages.typst.org/preview/quick-minutes-1.2.0.tar.gz}{\pandocbounded{\includesvg[keepaspectratio]{/assets/icons/16-download.svg}}}
\item[Repository:]
\href{https://github.com/Lypsilonx/quick-minutes}{GitHub}
\item[Discipline s :]
\begin{itemize}
\tightlist
\item[]
\item
  \href{https://typst.app/universe/search/?discipline=business}{Business}
\item
  \href{https://typst.app/universe/search/?discipline=communication}{Communication}
\item
  \href{https://typst.app/universe/search/?discipline=education}{Education}
\item
  \href{https://typst.app/universe/search/?discipline=journalism}{Journalism}
\item
  \href{https://typst.app/universe/search/?discipline=law}{Law}
\end{itemize}
\item[Categor y :]
\begin{itemize}
\tightlist
\item[]
\item
  \pandocbounded{\includesvg[keepaspectratio]{/assets/icons/16-layout.svg}}
  \href{https://typst.app/universe/search/?category=layout}{Layout}
\end{itemize}
\end{description}

\subsubsection{Where to report issues?}\label{where-to-report-issues}

This template is a project of Katharina Thöle and Lyx Rothböck .
Report issues on \href{https://github.com/Lypsilonx/quick-minutes}{their
repository} . You can also try to ask for help with this template on the
\href{https://forum.typst.app}{Forum} .

Please report this template to the Typst team using the
\href{https://typst.app/contact}{contact form} if you believe it is a
safety hazard or infringes upon your rights.

\phantomsection\label{versions}
\subsubsection{Version history}\label{version-history}

\begin{longtable}[]{@{}ll@{}}
\toprule\noalign{}
Version & Release Date \\
\midrule\noalign{}
\endhead
\bottomrule\noalign{}
\endlastfoot
1.2.0 & November 12, 2024 \\
\href{https://typst.app/universe/package/quick-minutes/1.1.2/}{1.1.2} &
October 31, 2024 \\
\href{https://typst.app/universe/package/quick-minutes/1.1.1/}{1.1.1} &
October 30, 2024 \\
\href{https://typst.app/universe/package/quick-minutes/1.1.0/}{1.1.0} &
October 29, 2024 \\
\href{https://typst.app/universe/package/quick-minutes/1.0.1/}{1.0.1} &
October 28, 2024 \\
\href{https://typst.app/universe/package/quick-minutes/1.0.0/}{1.0.0} &
October 28, 2024 \\
\end{longtable}

Typst GmbH did not create this template and cannot guarantee correct
functionality of this template or compatibility with any version of the
Typst compiler or app.


\section{Package List LaTeX/wrap-indent.tex}
\title{typst.app/universe/package/wrap-indent}

\phantomsection\label{banner}
\section{wrap-indent}\label{wrap-indent}

{ 0.1.0 }

Wrap content in custom functions with just indentation

\phantomsection\label{readme}
\texttt{\ wrap-indent\ } is a package for wrapping content in custom
functions with just indentation. This lets you avoid using trailing
square brackets to wrap content, instead you just indent it!

This system works by re-purposing Typst’s existing
\href{https://typst.app/docs/reference/model/terms/}{term-list} syntax
via a custom show rule on \texttt{\ terms.item\ } . We give it our
custom function within
\href{https://typst.app/docs/reference/introspection/state/}{state} via
a new \texttt{\ wrap-in()\ } function.

\subsection{Here’s a minimal
example!}\label{hereuxe2s-a-minimal-example}

\pandocbounded{\includegraphics[keepaspectratio]{https://github.com/typst/packages/raw/main/packages/preview/wrap-indent/0.1.0/example_page1.png}}

\begin{Shaded}
\begin{Highlighting}[]
\NormalTok{\#set page(height: auto, width: 3.5in, margin: 0.25in)}

\NormalTok{\#import "@preview/wrap{-}indent:0.1.0": wrap{-}in, allow{-}wrapping}

\NormalTok{\#show terms.item: allow{-}wrapping}

\NormalTok{/ First {-}{-}:}
\NormalTok{  A normal term list}

\NormalTok{  with multiple paragraphs}

\NormalTok{But this text is separated}


\NormalTok{\#line(length: 100\%)}


\NormalTok{\#let custom{-}block(content) = rect(content,}
\NormalTok{  fill: orange.lighten(90\%),}
\NormalTok{  stroke: 1.5pt + gradient.linear(..color.map.flare)}
\NormalTok{)}

\NormalTok{/ \#wrap{-}in(custom{-}block):}
\NormalTok{  A *custom block* using the \textasciigrave{}wrap{-}in\textasciigrave{} function}

\NormalTok{  with indented text \textbackslash{}}
\NormalTok{  over multiple lines}

\NormalTok{And this text is \_still\_ separated!}
\end{Highlighting}
\end{Shaded}

And in its own code block, here’s the required initialization:

\begin{Shaded}
\begin{Highlighting}[]
\NormalTok{\#import "@preview/wrap{-}indent:0.1.0": wrap{-}in, allow{-}wrapping}

\NormalTok{\#show terms.item: allow{-}wrapping}
\end{Highlighting}
\end{Shaded}

\subsection{And here’s a more complicated
example!}\label{and-hereuxe2s-a-more-complicated-example}

\pandocbounded{\includegraphics[keepaspectratio]{https://github.com/typst/packages/raw/main/packages/preview/wrap-indent/0.1.0/example_page2.png}}

\begin{Shaded}
\begin{Highlighting}[]
\NormalTok{\#set page(height: auto, width: 4.1in, margin: 0.25in)}

\NormalTok{\#show heading: set text(size: 0.75em)}
\NormalTok{\#show heading: set block(below: 1em)}
\NormalTok{\#set heading(numbering: "1) ")}

\NormalTok{= Normal function call:}

\NormalTok{// A function for wrapping some text:}
\NormalTok{\#let custom{-}quote(body) = rect(}
\NormalTok{  body,}
\NormalTok{  width: 100\%,}
\NormalTok{  fill: luma(95\%),}
\NormalTok{  stroke: (left: 2pt + luma(30\%))}
\NormalTok{)}

\NormalTok{\#custom{-}quote[}
\NormalTok{  Some text in a \_custom quote\_ spread over multiple lines}
\NormalTok{  so it actually looks like it was typed in a document.}
\NormalTok{]}
\NormalTok{This text is outside the quote box}


\NormalTok{= Wrappped function call!}

\NormalTok{/ \#wrap{-}in(custom{-}quote):}
\NormalTok{  Some text in a \_custom quote\_ spread over multiple lines}
\NormalTok{  so it actually looks like it was typed in a document.}

\NormalTok{This text is \_still\_ outside the quote box!}


\NormalTok{= Arbitrary functions should \_just work\#emoji.tm;\_}

\NormalTok{/ \#wrap{-}in(x =\textgreater{} ellipse(align(center, x),}
\NormalTok{    stroke: 3pt + gradient.conic(..color.map.rainbow)}
\NormalTok{  )):}
\NormalTok{  Some text in a \_rainbow ellipse\_ spread}
\NormalTok{  over multiple lines so it actually looks}
\NormalTok{  like it was typed in a document.}


\NormalTok{= One{-}liners look great!}

\NormalTok{/ \#wrap{-}in(underline): Here\textquotesingle{}s one line underlined}


\NormalTok{= Let\textquotesingle{}s do some math:}

\NormalTok{\#let named{-}thm(name) = (content) =\textgreater{} \{}
\NormalTok{  pad(left: 2em, par(hanging{-}indent: {-}2em)[}
\NormalTok{    *Theorem* (\#name) \#emph(content)}
\NormalTok{  ])}
\NormalTok{\}}

\NormalTok{/ \#wrap{-}in(named{-}thm("Operational Soundness")):}
\NormalTok{  If $med tack e : tau$ and $e$ reduces to $e\textquotesingle{}$}
\NormalTok{  by zero or more steps and $"Irred"(e\textquotesingle{})$,}
\NormalTok{  then $e\textquotesingle{} in "Val"$ and $med tack e\textquotesingle{} : tau$.}


\NormalTok{= In{-}line styling doesn\textquotesingle{}t create blocks:}

\NormalTok{/ \#wrap{-}in(highlight):}
\NormalTok{  This text is highlighted.}
\NormalTok{This text isn\textquotesingle{}t.}

\NormalTok{Notice how there was *no* paragraph break between the}
\NormalTok{two sentences? This is a useful result that makes}
\NormalTok{\textasciigrave{}wrap{-}indent\textasciigrave{} really flexible!}

\NormalTok{(if you want separate blocks, use \textasciigrave{}block\textasciigrave{} in your function)}


\NormalTok{= Does it work with nesting?}

\NormalTok{/ \#wrap{-}in(custom{-}quote):}
\NormalTok{  Testing...}
\NormalTok{  / \#wrap{-}in(align.with(center)):}
\NormalTok{    / \#wrap{-}in(rect):}
\NormalTok{      / \#wrap{-}in(emph):}
\NormalTok{        Signs point to yes!}


\NormalTok{= Final thoughts}

\NormalTok{/ Note {-}{-}:}
\NormalTok{  Regular term lists still work!}

\NormalTok{/ Disclaimer {-}{-}:}
\NormalTok{  You may run into issues with other term list}
\NormalTok{  show rules conflicting with this rule. \textbackslash{}}
\NormalTok{  (although set rules should be unaffected)}

\NormalTok{  If you run into issues, \_let me know!\_ I\textquotesingle{}d love to hear}
\NormalTok{  about it to make this package as robust as possible.}


\NormalTok{= And}

\NormalTok{\#let big{-}statement(content) = \{}
\NormalTok{  align(center, text(}
\NormalTok{    underline(stroke: 1.5pt, content),}
\NormalTok{    size: 32pt,}
\NormalTok{    weight: "bold",}
\NormalTok{    style: "italic",}
\NormalTok{    fill: eastern,}
\NormalTok{  ))}
\NormalTok{\}}

\NormalTok{/ \#wrap{-}in(big{-}statement):}
\NormalTok{  That\textquotesingle{}s a wrap!}
\end{Highlighting}
\end{Shaded}

\subsection{References}\label{references}

You can find my original writeup here for more context:\\
\url{https://typst.app/project/r5ogFas7lj7E48iHw_M4yh}

And also see the GitHub issue that prompted me to make this:\\
\url{https://github.com/typst/typst/issues/1921}

\subsubsection{How to add}\label{how-to-add}

Copy this into your project and use the import as
\texttt{\ wrap-indent\ }

\begin{verbatim}
#import "@preview/wrap-indent:0.1.0"
\end{verbatim}

\includesvg[width=0.16667in,height=0.16667in]{/assets/icons/16-copy.svg}

Check the docs for
\href{https://typst.app/docs/reference/scripting/\#packages}{more
information on how to import packages} .

\subsubsection{About}\label{about}

\begin{description}
\tightlist
\item[Author :]
Ian Wrzesinski (LectronPusher)
\item[License:]
MIT
\item[Current version:]
0.1.0
\item[Last updated:]
May 3, 2024
\item[First released:]
May 3, 2024
\item[Archive size:]
3.69 kB
\href{https://packages.typst.org/preview/wrap-indent-0.1.0.tar.gz}{\pandocbounded{\includesvg[keepaspectratio]{/assets/icons/16-download.svg}}}
\item[Categor ies :]
\begin{itemize}
\tightlist
\item[]
\item
  \pandocbounded{\includesvg[keepaspectratio]{/assets/icons/16-code.svg}}
  \href{https://typst.app/universe/search/?category=scripting}{Scripting}
\item
  \pandocbounded{\includesvg[keepaspectratio]{/assets/icons/16-hammer.svg}}
  \href{https://typst.app/universe/search/?category=utility}{Utility}
\end{itemize}
\end{description}

\subsubsection{Where to report issues?}\label{where-to-report-issues}

This package is a project of Ian Wrzesinski (LectronPusher) . You can
also try to ask for help with this package on the
\href{https://forum.typst.app}{Forum} .

Please report this package to the Typst team using the
\href{https://typst.app/contact}{contact form} if you believe it is a
safety hazard or infringes upon your rights.

\phantomsection\label{versions}
\subsubsection{Version history}\label{version-history}

\begin{longtable}[]{@{}ll@{}}
\toprule\noalign{}
Version & Release Date \\
\midrule\noalign{}
\endhead
\bottomrule\noalign{}
\endlastfoot
0.1.0 & May 3, 2024 \\
\end{longtable}

Typst GmbH did not create this package and cannot guarantee correct
functionality of this package or compatibility with any version of the
Typst compiler or app.


\section{Package List LaTeX/anatomy.tex}
\title{typst.app/universe/package/anatomy}

\phantomsection\label{banner}
\section{anatomy}\label{anatomy}

{ 0.1.1 }

Anatomy of a Font. Visualise metrics.

\phantomsection\label{readme}
\emph{Anatomy of a Font} . Visualise metrics.

Import the \texttt{\ anatomy\ } package:

\begin{Shaded}
\begin{Highlighting}[]
\NormalTok{\#import "@preview/anatomy:0.1.1": metrics}
\end{Highlighting}
\end{Shaded}

\subsection{Display Metrics}\label{display-metrics}

\texttt{\ metrics(72pt,\ "EB\ Garamond",\ display:\ "Typewriter")\ }
will be rendered as follows:

\pandocbounded{\includesvg[keepaspectratio]{https://raw.githubusercontent.com/E8D08F/packages/main/packages/preview/anatomy/0.1.1/img/export-1.svg}}

Additionally, a closure using \texttt{\ metrics\ } dictionary as
parameter can be used to layout additional elements below:

\begin{Shaded}
\begin{Highlighting}[]
\NormalTok{\#metrics(54pt, "一點明體",}
\NormalTok{  display: "電傳打字機",}
\NormalTok{  use: metrics =\textgreater{} table(}
\NormalTok{    columns: 2,}
\NormalTok{    ..metrics.pairs().flatten().map(x =\textgreater{} [ \#x ])}
\NormalTok{  )}
\NormalTok{)}
\end{Highlighting}
\end{Shaded}

It will generate:

\pandocbounded{\includesvg[keepaspectratio]{https://raw.githubusercontent.com/E8D08F/packages/main/packages/preview/anatomy/0.1.1/img/export-2.svg}}

\subsection{Remark on Hybrid
Typesetting}\label{remark-on-hybrid-typesetting}

To typeset CJK text, adopting font’s ascender/descender as
\texttt{\ top-edge\ } / \texttt{\ bottom-edge\ } makes more sense in
some cases. As for most CJK fonts, the difference between ascender and
descender heights will be exact 1em.

Tested with
\texttt{\ metrics(54pt,\ "Hiragino\ Mincho\ ProN",\ "テレタイプ端末")\ }
:

\pandocbounded{\includesvg[keepaspectratio]{https://raw.githubusercontent.com/E8D08F/packages/main/packages/preview/anatomy/0.1.1/img/export-3.svg}}

Since Typst will use metrics of the font which has the largest advance
height amongst the list provided in
\texttt{\ set\ text(font:\ (\ ...\ ))\ } to set up top and bottom edges
of a line, leading might not work as expected in hybrid typesetting.
This issue can be solved by passing the document to
\texttt{\ metrics(use:\ metrics\ =\textgreater{}\ \{\ ...\ \})\ } like
this:

\begin{Shaded}
\begin{Highlighting}[]
\NormalTok{\#show: doc =\textgreater{} metrics(font.size, font.main,}
\NormalTok{  // Retrieve the metrics of the CJK font}
\NormalTok{  use: metrics =\textgreater{} \{}
\NormalTok{    set text(}
\NormalTok{      font.size,}
\NormalTok{      font: ( font.latin, font.main ),}
\NormalTok{      features: ( "pkna", ),}
\NormalTok{      // Use CJK font’s ascender/descender as top/bottom edges}
\NormalTok{      top{-}edge: metrics.ascender,}
\NormalTok{      bottom{-}edge: metrics.descender,}
\NormalTok{      // ...}
\NormalTok{    )}

\NormalTok{    doc}
\NormalTok{  \}}
\NormalTok{)}
\end{Highlighting}
\end{Shaded}

\subsubsection{How to add}\label{how-to-add}

Copy this into your project and use the import as \texttt{\ anatomy\ }

\begin{verbatim}
#import "@preview/anatomy:0.1.1"
\end{verbatim}

\includesvg[width=0.16667in,height=0.16667in]{/assets/icons/16-copy.svg}

Check the docs for
\href{https://typst.app/docs/reference/scripting/\#packages}{more
information on how to import packages} .

\subsubsection{About}\label{about}

\begin{description}
\tightlist
\item[Author :]
Toto
\item[License:]
MIT
\item[Current version:]
0.1.1
\item[Last updated:]
February 19, 2024
\item[First released:]
February 17, 2024
\item[Archive size:]
2.59 kB
\href{https://packages.typst.org/preview/anatomy-0.1.1.tar.gz}{\pandocbounded{\includesvg[keepaspectratio]{/assets/icons/16-download.svg}}}
\end{description}

\subsubsection{Where to report issues?}\label{where-to-report-issues}

This package is a project of Toto . You can also try to ask for help
with this package on the \href{https://forum.typst.app}{Forum} .

Please report this package to the Typst team using the
\href{https://typst.app/contact}{contact form} if you believe it is a
safety hazard or infringes upon your rights.

\phantomsection\label{versions}
\subsubsection{Version history}\label{version-history}

\begin{longtable}[]{@{}ll@{}}
\toprule\noalign{}
Version & Release Date \\
\midrule\noalign{}
\endhead
\bottomrule\noalign{}
\endlastfoot
0.1.1 & February 19, 2024 \\
\href{https://typst.app/universe/package/anatomy/0.1.0/}{0.1.0} &
February 17, 2024 \\
\end{longtable}

Typst GmbH did not create this package and cannot guarantee correct
functionality of this package or compatibility with any version of the
Typst compiler or app.


\section{Package List LaTeX/mino.tex}
\title{typst.app/universe/package/mino}

\phantomsection\label{banner}
\section{mino}\label{mino}

{ 0.1.2 }

Render tetris fumen in typst.

\phantomsection\label{readme}
Render tetris \href{https://harddrop.com/fumen/}{fumen} in typst!

\pandocbounded{\includesvg[keepaspectratio]{https://github.com/typst/packages/raw/main/packages/preview/mino/0.1.2/mino.svg}}

\begin{Shaded}
\begin{Highlighting}[]
\NormalTok{\#import "typst{-}package/lib.typ": decode{-}fumen, render{-}field}
\NormalTok{// Uncomment the following line to use the mino from the official package registry}
\NormalTok{// \#import "@preview/mino:0.1.1": decode{-}fumen, render{-}field}
\NormalTok{\#set page(margin: 1.5cm)}

\NormalTok{\#align(center)[}
\NormalTok{  \#text(size: 25pt)[}
\NormalTok{    DT Cannon}
\NormalTok{  ]}
\NormalTok{]}

\NormalTok{\#let fumen = decode{-}fumen("v115@vhPJHYaAkeEfEXoC+BlvlzByEEfE03k2AlP5ABwfAA?A+rQAAqsBsqBvtBTpBVhQeAlvlzByEEfE03k2AlP5ABwvDf?E33ZBBlfbOBV5AAAOfQeAlvlzByEEfE03+2BlP5ABwvDfEV?5k2AlPJVBjzAAA6WQAAzdBpeB0XBXUBFlQnAlvlzByEEfE0?3+2BlP5ABwvDfEXhWEBUYPNBkkuRA1GCLBUupAAdqQnAlvl?zByEEfE038UBlP5ABwvDfEXhWEBUYPNBkkuRA1GCLBU+rAA?AAPAA")}

\NormalTok{\#for i in range(fumen.len()) \{}
\NormalTok{  let field = fumen.at(i).at("field")}
\NormalTok{  [}
\NormalTok{    \#box[}
\NormalTok{      \#render{-}field(field, rows: 8, cell{-}size: 13pt) }
\NormalTok{      (\#(i+1))}
\NormalTok{      \#fumen.at(i).comment}
\NormalTok{    ]}
\NormalTok{    \#h(0.5pt)}
\NormalTok{  ]}
\NormalTok{\}}
\end{Highlighting}
\end{Shaded}

\subsection{Documentation}\label{documentation}

\subsubsection{\texorpdfstring{\texttt{\ decode-fumen\ }}{ decode-fumen }}\label{decode-fumen}

Decode a fumen string into a series of pages.

\paragraph{Arguments}\label{arguments}

\begin{itemize}
\tightlist
\item
  \texttt{\ data\ } : \texttt{\ str\ } - The fumen string to decode
\end{itemize}

\paragraph{Returns}\label{returns}

The pages, of type
\texttt{\ Array\textless{}\{\ field:\ Array\textless{}string,\ 20\textgreater{},\ comment:\ string\ \}\textgreater{}\ }
.

\begin{verbatim}
(
  (
    field: (
      "....",
      "....",
      ...
    ),
    comment: "..."
  ),
  ...
)
\end{verbatim}

\subsubsection{\texorpdfstring{\texttt{\ render-field\ }}{ render-field }}\label{render-field}

\paragraph{Arguments}\label{arguments-1}

\begin{itemize}
\tightlist
\item
  \texttt{\ field\ } : \texttt{\ array\ } of \texttt{\ str\ } - The
  field to render
\item
  \texttt{\ rows\ } : \texttt{\ number\ } - The number of rows to
  render, default to \texttt{\ 20\ }
\item
  \texttt{\ cell-size\ } : \texttt{\ length\ } - The size of each cell,
  default to \texttt{\ 10pt\ }
\item
  \texttt{\ bg-color\ } : \texttt{\ color\ } - The background color,
  default to \texttt{\ \#f3f3ed\ }
\item
  \texttt{\ stroke\ } : The stroke for the field, default to
  \texttt{\ none\ }
\item
  \texttt{\ radius\ } : The border radius for the field, default to
  \texttt{\ 0.25\ *\ cell-size\ }
\item
  \texttt{\ shadow\ } : Whether to show shadow for cells, default to
  \texttt{\ true\ }
\item
  \texttt{\ highlight\ } : Whether to highlight cells, default to
  \texttt{\ true\ }
\item
  \texttt{\ color-data\ } : The color data for the field, default to
  \texttt{\ default-color-data\ } :
\item
  \texttt{\ overdraw\ } : (default, 5) Draw each cell multiple times to
  avoid thin lines between cells. See
  \url{https://github.com/linebender/vello/issues/49}
\end{itemize}

\begin{Shaded}
\begin{Highlighting}[]
\NormalTok{\#let default{-}color = (}
\NormalTok{  "Z": rgb("\#ef624d"),}
\NormalTok{  "S": rgb("\#66c65c"),}
\NormalTok{  "L": rgb("\#ef9535"),}
\NormalTok{  "J": rgb("\#1983bf"),}
\NormalTok{  "T": rgb("\#9c27b0"),}
\NormalTok{  "O": rgb("\#f7d33e"),}
\NormalTok{  "I": rgb("\#41afde"),}
\NormalTok{  "X": rgb("\#686868")}
\NormalTok{)}
\end{Highlighting}
\end{Shaded}

\begin{itemize}
\tightlist
\item
  \texttt{\ highlight-color-data\ } : The highlight color data for the
  field, default to \texttt{\ default-highlight-color\ } :
\end{itemize}

\begin{Shaded}
\begin{Highlighting}[]
\NormalTok{\#let default{-}highlight{-}color = (}
\NormalTok{  "Z": rgb("\#ff9484"),}
\NormalTok{  "S": rgb("\#88ee86"),}
\NormalTok{  "L": rgb("\#ffbf60"),}
\NormalTok{  "J": rgb("\#1ba6f9"),}
\NormalTok{  "T": rgb("\#e56add"),}
\NormalTok{  "O": rgb("\#fff952"),}
\NormalTok{  "I": rgb("\#43d3ff"),}
\NormalTok{  "X": rgb("\#949494")}
\NormalTok{)}
\end{Highlighting}
\end{Shaded}

\begin{itemize}
\tightlist
\item
  \texttt{\ shadow-color\ } : The shadow color for the field, default to
  \texttt{\ \#6f6f6f17\ }
\end{itemize}

\subsubsection{Credit}\label{credit}

The styles and color scheme are inspired by four.lol

\subsubsection{How to add}\label{how-to-add}

Copy this into your project and use the import as \texttt{\ mino\ }

\begin{verbatim}
#import "@preview/mino:0.1.2"
\end{verbatim}

\includesvg[width=0.16667in,height=0.16667in]{/assets/icons/16-copy.svg}

Check the docs for
\href{https://typst.app/docs/reference/scripting/\#packages}{more
information on how to import packages} .

\subsubsection{About}\label{about}

\begin{description}
\tightlist
\item[Author :]
Wenzhuo Liu
\item[License:]
MIT
\item[Current version:]
0.1.2
\item[Last updated:]
May 27, 2024
\item[First released:]
January 8, 2024
\item[Archive size:]
11.3 kB
\href{https://packages.typst.org/preview/mino-0.1.2.tar.gz}{\pandocbounded{\includesvg[keepaspectratio]{/assets/icons/16-download.svg}}}
\item[Repository:]
\href{https://github.com/Enter-tainer/mino}{GitHub}
\end{description}

\subsubsection{Where to report issues?}\label{where-to-report-issues}

This package is a project of Wenzhuo Liu . Report issues on
\href{https://github.com/Enter-tainer/mino}{their repository} . You can
also try to ask for help with this package on the
\href{https://forum.typst.app}{Forum} .

Please report this package to the Typst team using the
\href{https://typst.app/contact}{contact form} if you believe it is a
safety hazard or infringes upon your rights.

\phantomsection\label{versions}
\subsubsection{Version history}\label{version-history}

\begin{longtable}[]{@{}ll@{}}
\toprule\noalign{}
Version & Release Date \\
\midrule\noalign{}
\endhead
\bottomrule\noalign{}
\endlastfoot
0.1.2 & May 27, 2024 \\
\href{https://typst.app/universe/package/mino/0.1.1/}{0.1.1} & January
15, 2024 \\
\href{https://typst.app/universe/package/mino/0.1.0/}{0.1.0} & January
8, 2024 \\
\end{longtable}

Typst GmbH did not create this package and cannot guarantee correct
functionality of this package or compatibility with any version of the
Typst compiler or app.


\section{Package List LaTeX/circuiteria.tex}
\title{typst.app/universe/package/circuiteria}

\phantomsection\label{banner}
\section{circuiteria}\label{circuiteria}

{ 0.1.0 }

Drawing block circuits with Typst made easy, using CeTZ

\phantomsection\label{readme}
Circuiteria is a \href{https://typst.app/}{Typst} package for drawing
block circuit diagrams using the
\href{https://typst.app/universe/package/cetz}{CeTZ} package.

\pandocbounded{\includegraphics[keepaspectratio]{https://github.com/typst/packages/raw/main/packages/preview/circuiteria/0.1.0/gallery/platypus.png}}

\subsection{Examples}\label{examples}

\begin{longtable}[]{@{}ll@{}}
\toprule\noalign{}
\endhead
\bottomrule\noalign{}
\endlastfoot
\multicolumn{2}{@{}l@{}}{%
\href{https://github.com/typst/packages/raw/main/packages/preview/circuiteria/0.1.0/gallery/test.typ}{\includegraphics[width=5.20833in,height=\textheight,keepaspectratio]{https://github.com/typst/packages/raw/main/packages/preview/circuiteria/0.1.0/gallery/test.png}}} \\
\multicolumn{2}{@{}l@{}}{%
A bit of eveything} \\
\multicolumn{2}{@{}l@{}}{%
\href{https://github.com/typst/packages/raw/main/packages/preview/circuiteria/0.1.0/gallery/test5.typ}{\includegraphics[width=5.20833in,height=\textheight,keepaspectratio]{https://github.com/typst/packages/raw/main/packages/preview/circuiteria/0.1.0/gallery/test5.png}}} \\
\multicolumn{2}{@{}l@{}}{%
Wires everywhere} \\
\href{https://github.com/typst/packages/raw/main/packages/preview/circuiteria/0.1.0/gallery/test4.typ}{\includegraphics[width=2.60417in,height=\textheight,keepaspectratio]{https://github.com/typst/packages/raw/main/packages/preview/circuiteria/0.1.0/gallery/test4.png}}
&
\href{https://github.com/typst/packages/raw/main/packages/preview/circuiteria/0.1.0/gallery/test6.typ}{\includegraphics[width=2.60417in,height=\textheight,keepaspectratio]{https://github.com/typst/packages/raw/main/packages/preview/circuiteria/0.1.0/gallery/test6.png}} \\
Groups & Rotated \\
\end{longtable}

\begin{quote}
\textbf{Note}\\
These circuit layouts were copied from a digital design course given by
prof. S. Zahno and recreated using this package
\end{quote}

\emph{Click on the example image to jump to the code.}

\subsection{Usage}\label{usage}

For more information, see the
\href{https://github.com/typst/packages/raw/main/packages/preview/circuiteria/0.1.0/manual.pdf}{manual}

To use this package, simply import
\href{https://typst.app/universe/package/circuiteria}{circuiteria} and
call the \texttt{\ circuit\ } function:

\begin{Shaded}
\begin{Highlighting}[]
\NormalTok{\#import "@preview/circuiteria:0.1.0"}
\NormalTok{\#circuiteria.circuit(\{}
\NormalTok{  import circuiteria: *}
\NormalTok{  ...}
\NormalTok{\})}
\end{Highlighting}
\end{Shaded}

\subsubsection{How to add}\label{how-to-add}

Copy this into your project and use the import as
\texttt{\ circuiteria\ }

\begin{verbatim}
#import "@preview/circuiteria:0.1.0"
\end{verbatim}

\includesvg[width=0.16667in,height=0.16667in]{/assets/icons/16-copy.svg}

Check the docs for
\href{https://typst.app/docs/reference/scripting/\#packages}{more
information on how to import packages} .

\subsubsection{About}\label{about}

\begin{description}
\tightlist
\item[Author :]
\href{https://git.kb28.ch/HEL}{Louis Heredero}
\item[License:]
Apache-2.0
\item[Current version:]
0.1.0
\item[Last updated:]
October 3, 2024
\item[First released:]
October 3, 2024
\item[Minimum Typst version:]
0.11.0
\item[Archive size:]
193 kB
\href{https://packages.typst.org/preview/circuiteria-0.1.0.tar.gz}{\pandocbounded{\includesvg[keepaspectratio]{/assets/icons/16-download.svg}}}
\item[Repository:]
\href{https://git.kb28.ch/HEL/circuiteria}{git.kb28.ch}
\item[Categor y :]
\begin{itemize}
\tightlist
\item[]
\item
  \pandocbounded{\includesvg[keepaspectratio]{/assets/icons/16-chart.svg}}
  \href{https://typst.app/universe/search/?category=visualization}{Visualization}
\end{itemize}
\end{description}

\subsubsection{Where to report issues?}\label{where-to-report-issues}

This package is a project of Louis Heredero . Report issues on
\href{https://git.kb28.ch/HEL/circuiteria}{their repository} . You can
also try to ask for help with this package on the
\href{https://forum.typst.app}{Forum} .

Please report this package to the Typst team using the
\href{https://typst.app/contact}{contact form} if you believe it is a
safety hazard or infringes upon your rights.

\phantomsection\label{versions}
\subsubsection{Version history}\label{version-history}

\begin{longtable}[]{@{}ll@{}}
\toprule\noalign{}
Version & Release Date \\
\midrule\noalign{}
\endhead
\bottomrule\noalign{}
\endlastfoot
0.1.0 & October 3, 2024 \\
\end{longtable}

Typst GmbH did not create this package and cannot guarantee correct
functionality of this package or compatibility with any version of the
Typst compiler or app.


\section{Package List LaTeX/ucpc-solutions.tex}
\title{typst.app/universe/package/ucpc-solutions}

\phantomsection\label{banner}
\phantomsection\label{template-thumbnail}
\pandocbounded{\includegraphics[keepaspectratio]{https://packages.typst.org/preview/thumbnails/ucpc-solutions-0.1.0-small.webp}}

\section{ucpc-solutions}\label{ucpc-solutions}

{ 0.1.0 }

The port of UCPC solutions theme.

\href{/app?template=ucpc-solutions&version=0.1.0}{Create project in app}

\phantomsection\label{readme}
\href{https://github.com/ShapeLayer/ucpc-solutions__typst}{ucpc-solutions}
is the template for solutions editorial of algorithm contests, used
widely in the \href{https://acmicpc.net/}{“Baekjoon Online Judge�}
users community in Korea.

The original version of ucpc-solution is written in LaTeX(
\href{https://github.com/ucpcc/2020-solutions-theme}{ucpcc/2020-solutions-theme}
), and this is the port of LaTeX ver. This contains content-generating
utils for making solutions editorial and
\href{https://solved.ac/}{“solved.ac�} difficulty expression
presets, a rating system for Baekjoon Online Judge’s problems.

\subsection{Getting Started}\label{getting-started}

\begin{Shaded}
\begin{Highlighting}[]
\NormalTok{\#import "@preview/ucpc{-}solutions:0.1.0" as ucpc}

\NormalTok{\#show: ucpc.ucpc.with(}
\NormalTok{  title: "Contest Name",}
\NormalTok{  authors: ("Contest Authors", ),}
\NormalTok{)}
\end{Highlighting}
\end{Shaded}

\subsubsection{Requirements}\label{requirements}

\textbf{Fonts}

\begin{itemize}
\tightlist
\item
  \href{https://fonts.google.com/specimen/Inter}{Inter}
\item
  (optional) \href{https://fonts.google.com/specimen/Gothic+A1}{Gothic
  A1}
\item
  (optional)
  \href{https://github.com/orioncactus/pretendard/blob/main/packages/pretendard/docs/en/README.md}{Pretendard}
\end{itemize}

\subsection{Examples}\label{examples}

See
\href{https://github.com/typst/packages/raw/main/packages/preview/ucpc-solutions/0.1.0/examples/}{\texttt{\ /examples\ }}
.

You can also see other usecase using the original LaTeX theme. See the
\href{https://github.com/ucpcc/2020-solutions-theme\#\%ED\%85\%8C\%EB\%A7\%88-\%EC\%82\%AC\%EC\%9A\%A9-\%EC\%98\%88}{(KR)
“Theme Usage Examples(í\ldots Œë§ˆ 사용 예)â€? section} in the
origin repository’s README.

\subsection{For Contributing}\label{for-contributing}

Requirements: \href{https://github.com/casey/just}{just} ,
\href{https://github.com/tingerrr/typst-test}{typst-test}

\textbf{Recompile Refs for Testing}

\begin{Shaded}
\begin{Highlighting}[]
\ExtensionTok{just}\NormalTok{ update{-}test}
\end{Highlighting}
\end{Shaded}

\textbf{Run Test}

\begin{Shaded}
\begin{Highlighting}[]
\ExtensionTok{just}\NormalTok{ test}
\end{Highlighting}
\end{Shaded}

\begin{center}\rule{0.5\linewidth}{0.5pt}\end{center}

\begin{itemize}
\item
  Special Thanks: \href{https://github.com/kiwiyou}{@kiwiyou} - about
  technical issue
\item
  Since this ported version has been re-implemented only for appearance,
  this repository does not include the source code of any distribution
  or variant of ucpc-solutions.
\end{itemize}

\href{/app?template=ucpc-solutions&version=0.1.0}{Create project in app}

\subsubsection{How to use}\label{how-to-use}

Click the button above to create a new project using this template in
the Typst app.

You can also use the Typst CLI to start a new project on your computer
using this command:

\begin{verbatim}
typst init @preview/ucpc-solutions:0.1.0
\end{verbatim}

\includesvg[width=0.16667in,height=0.16667in]{/assets/icons/16-copy.svg}

\subsubsection{About}\label{about}

\begin{description}
\tightlist
\item[Author s :]
\href{https://github.com/ShapeLayer}{Jonghyeon Park} \&
\href{https://github.com/ucpcc}{The Union of Collegiate Programming
Contest Clubs}
\item[License:]
MIT
\item[Current version:]
0.1.0
\item[Last updated:]
August 14, 2024
\item[First released:]
August 14, 2024
\item[Minimum Typst version:]
0.1.0
\item[Archive size:]
22.2 kB
\href{https://packages.typst.org/preview/ucpc-solutions-0.1.0.tar.gz}{\pandocbounded{\includesvg[keepaspectratio]{/assets/icons/16-download.svg}}}
\item[Repository:]
\href{https://github.com/ShapeLayer/ucpc-solutions__typst}{GitHub}
\item[Categor y :]
\begin{itemize}
\tightlist
\item[]
\item
  \pandocbounded{\includesvg[keepaspectratio]{/assets/icons/16-presentation.svg}}
  \href{https://typst.app/universe/search/?category=presentation}{Presentation}
\end{itemize}
\end{description}

\subsubsection{Where to report issues?}\label{where-to-report-issues}

This template is a project of Jonghyeon Park and The Union of Collegiate
Programming Contest Clubs . Report issues on
\href{https://github.com/ShapeLayer/ucpc-solutions__typst}{their
repository} . You can also try to ask for help with this template on the
\href{https://forum.typst.app}{Forum} .

Please report this template to the Typst team using the
\href{https://typst.app/contact}{contact form} if you believe it is a
safety hazard or infringes upon your rights.

\phantomsection\label{versions}
\subsubsection{Version history}\label{version-history}

\begin{longtable}[]{@{}ll@{}}
\toprule\noalign{}
Version & Release Date \\
\midrule\noalign{}
\endhead
\bottomrule\noalign{}
\endlastfoot
0.1.0 & August 14, 2024 \\
\end{longtable}

Typst GmbH did not create this template and cannot guarantee correct
functionality of this template or compatibility with any version of the
Typst compiler or app.


\section{Package List LaTeX/dining-table.tex}
\title{typst.app/universe/package/dining-table}

\phantomsection\label{banner}
\section{dining-table}\label{dining-table}

{ 0.1.0 }

Column-wise table definitions for big data

\phantomsection\label{readme}
Version 0.1.0

Implements a layer on top of table to allow the user to define a table
by column rather than by row, to automatically handle headers and
footers, to implement table footnotes, to handle nested column quirks
for you, to handle rendering nested data structures.

Basically, if you are tabulating data where each row is an observation,
and some features (columns) are to be grouped (a common case for
scientific data) then this package might be worth checking out. Another
use case is where you have multiple tables with identical layouts, and
you wish to keep them all consistent with one another.

\pandocbounded{\includegraphics[keepaspectratio]{https://github.com/typst/packages/raw/main/packages/preview/dining-table/0.1.0/examples/ledger.png}}

\subsection{Usage}\label{usage}

See the manual for in-depth usage, but for a quick reference, here is
the ledger example (which is fully featured)

\begin{Shaded}
\begin{Highlighting}[]
\NormalTok{\#import "@preview/dining{-}table:0.1.0"}

\NormalTok{\#let data = (}
\NormalTok{  (}
\NormalTok{    date: datetime.today(),}
\NormalTok{    particulars: lorem(05),}
\NormalTok{    ledger: [JRS123] + dining{-}table.note.make[Hello World],}
\NormalTok{    amount: (unit: $100$, decimal: $00$),}
\NormalTok{    total: (unit: $99$, decimal: $00$),}
\NormalTok{  ),}
\NormalTok{)*7 }

\NormalTok{\#import "@preview/typpuccino:0.1.0"}
\NormalTok{\#let bg{-}fill{-}1 = typpuccino.latte.base}
\NormalTok{\#let bg{-}fill{-}2 = typpuccino.latte.mantle}

\NormalTok{\#let example = (}
\NormalTok{  (}
\NormalTok{    key: "date",}
\NormalTok{    header: align(left)[Date],}
\NormalTok{    display: (it)=\textgreater{}it.display(auto),}
\NormalTok{    fill: bg{-}fill{-}1,}
\NormalTok{    align: start,}
\NormalTok{    gutter: 0.5em,}
\NormalTok{  ),}
\NormalTok{  (}
\NormalTok{    key: "particulars",}
\NormalTok{    header: text(tracking: 5pt)[Particulars],}
\NormalTok{    width: 1fr,}
\NormalTok{    gutter: 0.5em,}
\NormalTok{  ),}
\NormalTok{  (}
\NormalTok{    key: "ledger",}
\NormalTok{    header: [Ledger],}
\NormalTok{    fill: bg{-}fill{-}2,}
\NormalTok{    width: 2cm,}
\NormalTok{    gutter: 0.5em,}
\NormalTok{  ),}
\NormalTok{  (}
\NormalTok{    header: align(center)[Amount],}
\NormalTok{    fill: bg{-}fill{-}1,}
\NormalTok{    gutter: 0.5em,}
\NormalTok{    hline: arguments(stroke: dining{-}table.lightrule),}
\NormalTok{    children: (}
\NormalTok{      (}
\NormalTok{        key: "amount.unit", }
\NormalTok{        header: align(left)[£], }
\NormalTok{        width: 5em, }
\NormalTok{        align: right,}
\NormalTok{        vline: arguments(stroke: dining{-}table.lightrule),}
\NormalTok{        gutter: 0em,}
\NormalTok{      ),}
\NormalTok{      (}
\NormalTok{        key: "amount.decimal",}
\NormalTok{        header: align(right, text(number{-}type: "old{-}style")[.00]), }
\NormalTok{        align: left}
\NormalTok{      ),}
\NormalTok{    )}
\NormalTok{  ),}
\NormalTok{  (}
\NormalTok{    header: align(center)[Total],}
\NormalTok{    gutter: 0.5em,}
\NormalTok{    hline: arguments(stroke: dining{-}table.lightrule),}
\NormalTok{    children: (}
\NormalTok{      (}
\NormalTok{        key: "total.unit", }
\NormalTok{        header: align(left)[£], }
\NormalTok{        width: 5em, }
\NormalTok{        align: right,}
\NormalTok{        vline: arguments(stroke: dining{-}table.lightrule),}
\NormalTok{        gutter: 0em,}
\NormalTok{      ),}
\NormalTok{      (}
\NormalTok{        key: "total.decimal",}
\NormalTok{        header: align(right, text(number{-}type: "old{-}style")[.00]), }
\NormalTok{        align: left}
\NormalTok{      ),}
\NormalTok{    )}
\NormalTok{  ),}
\NormalTok{)}

\NormalTok{\#set text(size: 11pt)}
\NormalTok{\#set page(height: auto, margin: 1em)}
\NormalTok{\#dining{-}table.make(columns: example, }
\NormalTok{  data: data, }
\NormalTok{  notes: dining{-}table.note.display{-}list}
\NormalTok{)}
\end{Highlighting}
\end{Shaded}

\subsubsection{How to add}\label{how-to-add}

Copy this into your project and use the import as
\texttt{\ dining-table\ }

\begin{verbatim}
#import "@preview/dining-table:0.1.0"
\end{verbatim}

\includesvg[width=0.16667in,height=0.16667in]{/assets/icons/16-copy.svg}

Check the docs for
\href{https://typst.app/docs/reference/scripting/\#packages}{more
information on how to import packages} .

\subsubsection{About}\label{about}

\begin{description}
\tightlist
\item[Author :]
James R. Swift
\item[License:]
Unlicense
\item[Current version:]
0.1.0
\item[Last updated:]
July 10, 2024
\item[First released:]
July 10, 2024
\item[Archive size:]
598 kB
\href{https://packages.typst.org/preview/dining-table-0.1.0.tar.gz}{\pandocbounded{\includesvg[keepaspectratio]{/assets/icons/16-download.svg}}}
\item[Repository:]
\href{https://github.com/JamesxX/dining-table}{GitHub}
\item[Discipline s :]
\begin{itemize}
\tightlist
\item[]
\item
  \href{https://typst.app/universe/search/?discipline=agriculture}{Agriculture}
\item
  \href{https://typst.app/universe/search/?discipline=biology}{Biology}
\item
  \href{https://typst.app/universe/search/?discipline=chemistry}{Chemistry}
\item
  \href{https://typst.app/universe/search/?discipline=communication}{Communication}
\item
  \href{https://typst.app/universe/search/?discipline=computer-science}{Computer
  Science}
\item
  \href{https://typst.app/universe/search/?discipline=economics}{Economics}
\item
  \href{https://typst.app/universe/search/?discipline=physics}{Physics}
\end{itemize}
\item[Categor ies :]
\begin{itemize}
\tightlist
\item[]
\item
  \pandocbounded{\includesvg[keepaspectratio]{/assets/icons/16-package.svg}}
  \href{https://typst.app/universe/search/?category=components}{Components}
\item
  \pandocbounded{\includesvg[keepaspectratio]{/assets/icons/16-chart.svg}}
  \href{https://typst.app/universe/search/?category=visualization}{Visualization}
\item
  \pandocbounded{\includesvg[keepaspectratio]{/assets/icons/16-list-unordered.svg}}
  \href{https://typst.app/universe/search/?category=model}{Model}
\end{itemize}
\end{description}

\subsubsection{Where to report issues?}\label{where-to-report-issues}

This package is a project of James R. Swift . Report issues on
\href{https://github.com/JamesxX/dining-table}{their repository} . You
can also try to ask for help with this package on the
\href{https://forum.typst.app}{Forum} .

Please report this package to the Typst team using the
\href{https://typst.app/contact}{contact form} if you believe it is a
safety hazard or infringes upon your rights.

\phantomsection\label{versions}
\subsubsection{Version history}\label{version-history}

\begin{longtable}[]{@{}ll@{}}
\toprule\noalign{}
Version & Release Date \\
\midrule\noalign{}
\endhead
\bottomrule\noalign{}
\endlastfoot
0.1.0 & July 10, 2024 \\
\end{longtable}

Typst GmbH did not create this package and cannot guarantee correct
functionality of this package or compatibility with any version of the
Typst compiler or app.


\section{Package List LaTeX/hydra.tex}
\title{typst.app/universe/package/hydra}

\phantomsection\label{banner}
\section{hydra}\label{hydra}

{ 0.5.1 }

Query and display headings in your documents and templates.

{ } Featured Package

\phantomsection\label{readme}
Hydra is a Typst package allowing you to easily display the heading like
elements anywhere in your document. It’s primary focus is to provide
the reader with a reminder of where they currently are in your document
only when it is needed.

\subsection{Example}\label{example}

\begin{Shaded}
\begin{Highlighting}[]
\NormalTok{\#import "@preview/hydra:0.5.1": hydra}

\NormalTok{\#set page(paper: "a7", margin: (y: 4em), numbering: "1", header: context \{}
\NormalTok{  if calc.odd(here().page()) \{}
\NormalTok{    align(right, emph(hydra(1)))}
\NormalTok{  \} else \{}
\NormalTok{    align(left, emph(hydra(2)))}
\NormalTok{  \}}
\NormalTok{  line(length: 100\%)}
\NormalTok{\})}
\NormalTok{\#set heading(numbering: "1.1")}
\NormalTok{\#show heading.where(level: 1): it =\textgreater{} pagebreak(weak: true) + it}

\NormalTok{= Introduction}
\NormalTok{\#lorem(50)}

\NormalTok{= Content}
\NormalTok{== First Section}
\NormalTok{\#lorem(50)}
\NormalTok{== Second Section}
\NormalTok{\#lorem(100)}
\end{Highlighting}
\end{Shaded}

\pandocbounded{\includegraphics[keepaspectratio]{https://github.com/typst/packages/raw/main/packages/preview/hydra/0.5.1/examples/example.png}}

\subsection{Documentation}\label{documentation}

For a more in-depth description of hydra’s functionality and the
reference read its
\href{https://github.com/typst/packages/raw/main/packages/preview/hydra/0.5.1/doc/manual.pdf}{manual}
.

\subsection{Contribution}\label{contribution}

For contributing, please take a look
\href{https://github.com/typst/packages/raw/main/packages/preview/hydra/0.5.1/CONTRIBUTING.md}{CONTRIBUTING}
.

\subsection{Etymology}\label{etymology}

The package name hydra /ˈhaɪdrə/ is a word play headings and headers,
inspired by the monster in greek and roman mythology resembling a
serpent with many heads.

\subsubsection{How to add}\label{how-to-add}

Copy this into your project and use the import as \texttt{\ hydra\ }

\begin{verbatim}
#import "@preview/hydra:0.5.1"
\end{verbatim}

\includesvg[width=0.16667in,height=0.16667in]{/assets/icons/16-copy.svg}

Check the docs for
\href{https://typst.app/docs/reference/scripting/\#packages}{more
information on how to import packages} .

\subsubsection{About}\label{about}

\begin{description}
\tightlist
\item[Author :]
\href{mailto:me@tinger.dev}{tinger}
\item[License:]
MIT
\item[Current version:]
0.5.1
\item[Last updated:]
July 25, 2024
\item[First released:]
November 19, 2023
\item[Minimum Typst version:]
0.11.0
\item[Archive size:]
238 kB
\href{https://packages.typst.org/preview/hydra-0.5.1.tar.gz}{\pandocbounded{\includesvg[keepaspectratio]{/assets/icons/16-download.svg}}}
\item[Repository:]
\href{https://github.com/tingerrr/hydra}{GitHub}
\item[Categor ies :]
\begin{itemize}
\tightlist
\item[]
\item
  \pandocbounded{\includesvg[keepaspectratio]{/assets/icons/16-package.svg}}
  \href{https://typst.app/universe/search/?category=components}{Components}
\item
  \pandocbounded{\includesvg[keepaspectratio]{/assets/icons/16-code.svg}}
  \href{https://typst.app/universe/search/?category=scripting}{Scripting}
\end{itemize}
\end{description}

\subsubsection{Where to report issues?}\label{where-to-report-issues}

This package is a project of tinger . Report issues on
\href{https://github.com/tingerrr/hydra}{their repository} . You can
also try to ask for help with this package on the
\href{https://forum.typst.app}{Forum} .

Please report this package to the Typst team using the
\href{https://typst.app/contact}{contact form} if you believe it is a
safety hazard or infringes upon your rights.

\phantomsection\label{versions}
\subsubsection{Version history}\label{version-history}

\begin{longtable}[]{@{}ll@{}}
\toprule\noalign{}
Version & Release Date \\
\midrule\noalign{}
\endhead
\bottomrule\noalign{}
\endlastfoot
0.5.1 & July 25, 2024 \\
\href{https://typst.app/universe/package/hydra/0.5.0/}{0.5.0} & July 3,
2024 \\
\href{https://typst.app/universe/package/hydra/0.4.0/}{0.4.0} & March
21, 2024 \\
\href{https://typst.app/universe/package/hydra/0.3.0/}{0.3.0} & January
8, 2024 \\
\href{https://typst.app/universe/package/hydra/0.2.0/}{0.2.0} & November
25, 2023 \\
\href{https://typst.app/universe/package/hydra/0.1.0/}{0.1.0} & November
19, 2023 \\
\end{longtable}

Typst GmbH did not create this package and cannot guarantee correct
functionality of this package or compatibility with any version of the
Typst compiler or app.


\section{Package List LaTeX/crossregex.tex}
\title{typst.app/universe/package/crossregex}

\phantomsection\label{banner}
\section{crossregex}\label{crossregex}

{ 0.2.0 }

A crossword-like regex game written in Typst.

\phantomsection\label{readme}
A crossword-like game written in Typst. You should fill in letters to
satisfy regular expression constraints. Currently, \emph{squares} and
\emph{regular hexagons} are supported.

\begin{quote}
{[}!note{]} This is not a puzzle solver, but a puzzle layout builder.
\end{quote}

It takes inspiration from a web image, which derives our standard
example.

\pandocbounded{\includesvg[keepaspectratio]{https://github.com/typst/packages/raw/main/packages/preview/crossregex/0.2.0/examples/standard.svg}}

\pandocbounded{\includesvg[keepaspectratio]{https://github.com/typst/packages/raw/main/packages/preview/crossregex/0.2.0/examples/sudoku-main.svg}}

More examples and source code:
\url{https://github.com/QuadnucYard/crossregex-typ}

\subsection{Basic Usage}\label{basic-usage}

We use \texttt{\ crossregex-square\ } and \texttt{\ crossregex-hex\ } to
build square and hex layouts respectively. They have the same argument
formats. A \texttt{\ crossregex\ } dispatcher function can be used for
dynamic grid kind, which is compatible with version 0.1.0.

\begin{Shaded}
\begin{Highlighting}[]
\NormalTok{\#import "@preview/crossregex:0.2.0": crossregex}
\NormalTok{// or import and use \textasciigrave{}crossregex{-}hex\textasciigrave{}}

\NormalTok{\#crossregex(}
\NormalTok{  3,}
\NormalTok{  constraints: (}
\NormalTok{    \textasciigrave{}A.*\textasciigrave{}, \textasciigrave{}B.*\textasciigrave{}, \textasciigrave{}C.*\textasciigrave{}, \textasciigrave{}D.*\textasciigrave{}, \textasciigrave{}E.*\textasciigrave{},}
\NormalTok{    \textasciigrave{}F.*\textasciigrave{}, \textasciigrave{}G.*\textasciigrave{}, \textasciigrave{}H.*\textasciigrave{}, \textasciigrave{}I.*\textasciigrave{}, \textasciigrave{}J.*\textasciigrave{},}
\NormalTok{    \textasciigrave{}K.*\textasciigrave{}, \textasciigrave{}L.*\textasciigrave{}, \textasciigrave{}M.*\textasciigrave{}, \textasciigrave{}N.*\textasciigrave{}, \textasciigrave{}O.*\textasciigrave{},}
\NormalTok{  ),}
\NormalTok{  answer: (}
\NormalTok{    "ABC",}
\NormalTok{    "DEFG",}
\NormalTok{    "HIJKL",}
\NormalTok{    "MNOP",}
\NormalTok{    "QRS",}
\NormalTok{  ),}
\NormalTok{)}
\end{Highlighting}
\end{Shaded}

\begin{Shaded}
\begin{Highlighting}[]
\NormalTok{\#import "@preview/crossregex:0.2.0": crossregex{-}square}

\NormalTok{\#crossregex{-}square(}
\NormalTok{  9,}
\NormalTok{  alphabet: regex("[0{-}9]"),}
\NormalTok{  constraints: (}
\NormalTok{    \textasciigrave{}.*\textasciigrave{},}
\NormalTok{    \textasciigrave{}.*\textasciigrave{},}
\NormalTok{    \textasciigrave{}.*\textasciigrave{},}
\NormalTok{    \textasciigrave{}.*\textasciigrave{},}
\NormalTok{    \textasciigrave{}.*[12]\{2\}8\textasciigrave{},}
\NormalTok{    \textasciigrave{}[1{-}9]9.*\textasciigrave{},}
\NormalTok{    \textasciigrave{}.*\textasciigrave{},}
\NormalTok{    \textasciigrave{}.*\textasciigrave{},}
\NormalTok{    \textasciigrave{}.*\textasciigrave{},}
\NormalTok{    \textasciigrave{}[1{-}9]7[29]\{2\}8.6.*\textasciigrave{},}
\NormalTok{    \textasciigrave{}.*2[\^{}3]\{2\}1.\textasciigrave{},}
\NormalTok{    \textasciigrave{}.9.315[\^{}6]+\textasciigrave{},}
\NormalTok{    \textasciigrave{}.+4[15]\{2\}79.\textasciigrave{},}
\NormalTok{    \textasciigrave{}[75]\{2\}18.63[1{-}9]+\textasciigrave{},}
\NormalTok{    \textasciigrave{}8.*[\^{}2][\^{}3][\^{}1]+56[\^{}6]\textasciigrave{},}
\NormalTok{    \textasciigrave{}[\^{}5{-}6][0{-}9][56]\{2\}.*9\textasciigrave{},}
\NormalTok{    \textasciigrave{}.*\textasciigrave{},}
\NormalTok{    \textasciigrave{}[98]\{2\}5.*[27]\{2\}6\textasciigrave{},}
\NormalTok{  ),}
\NormalTok{  answer: (}
\NormalTok{    "934872651",}
\NormalTok{    "812456937",}
\NormalTok{    "576913482",}
\NormalTok{    "125784369",}
\NormalTok{    "467395128",}
\NormalTok{    "398261574",}
\NormalTok{    "241537896",}
\NormalTok{    "783649215",}
\NormalTok{    "659128743",}
\NormalTok{  ),}
\NormalTok{  cell: rect(width: 1.4em, height: 1.4em, radius: 0.1em, stroke: 1pt + orange, fill: orange.transparentize(80\%)),}
\NormalTok{  cell{-}config: (size: 1.6em, text{-}style: (size: 1.4em)),}
\NormalTok{)}
\end{Highlighting}
\end{Shaded}

\subsection{Document}\label{document}

Details are shown in the doc comments above the \texttt{\ crossregex\ }
function in \texttt{\ lib.typ\ } . You can choose to turn off some
views.

Feel free to open issues if any problems.

\subsection{Changelog}\label{changelog}

\subsubsection{0.2.0}\label{section}

\begin{itemize}
\tightlist
\item
  Feature: Supports square shapes.
\item
  Feature: Supports customization the appearance of everything, even the
  cells.
\item
  Feature: Supports custom alphabets.
\item
  Fix: An mistake related to import in the README example.
\end{itemize}

\subsubsection{0.1.0}\label{section-1}

First release with basic hex features.

\subsubsection{How to add}\label{how-to-add}

Copy this into your project and use the import as
\texttt{\ crossregex\ }

\begin{verbatim}
#import "@preview/crossregex:0.2.0"
\end{verbatim}

\includesvg[width=0.16667in,height=0.16667in]{/assets/icons/16-copy.svg}

Check the docs for
\href{https://typst.app/docs/reference/scripting/\#packages}{more
information on how to import packages} .

\subsubsection{About}\label{about}

\begin{description}
\tightlist
\item[Author :]
QuadnucYard
\item[License:]
MIT
\item[Current version:]
0.2.0
\item[Last updated:]
September 22, 2024
\item[First released:]
September 3, 2024
\item[Minimum Typst version:]
0.11.0
\item[Archive size:]
197 kB
\href{https://packages.typst.org/preview/crossregex-0.2.0.tar.gz}{\pandocbounded{\includesvg[keepaspectratio]{/assets/icons/16-download.svg}}}
\item[Repository:]
\href{https://github.com/QuadnucYard/crossregex-typ}{GitHub}
\item[Categor y :]
\begin{itemize}
\tightlist
\item[]
\item
  \pandocbounded{\includesvg[keepaspectratio]{/assets/icons/16-smile.svg}}
  \href{https://typst.app/universe/search/?category=fun}{Fun}
\end{itemize}
\end{description}

\subsubsection{Where to report issues?}\label{where-to-report-issues}

This package is a project of QuadnucYard . Report issues on
\href{https://github.com/QuadnucYard/crossregex-typ}{their repository} .
You can also try to ask for help with this package on the
\href{https://forum.typst.app}{Forum} .

Please report this package to the Typst team using the
\href{https://typst.app/contact}{contact form} if you believe it is a
safety hazard or infringes upon your rights.

\phantomsection\label{versions}
\subsubsection{Version history}\label{version-history}

\begin{longtable}[]{@{}ll@{}}
\toprule\noalign{}
Version & Release Date \\
\midrule\noalign{}
\endhead
\bottomrule\noalign{}
\endlastfoot
0.2.0 & September 22, 2024 \\
\href{https://typst.app/universe/package/crossregex/0.1.0/}{0.1.0} &
September 3, 2024 \\
\end{longtable}

Typst GmbH did not create this package and cannot guarantee correct
functionality of this package or compatibility with any version of the
Typst compiler or app.


\section{Package List LaTeX/resume-ng.tex}
\title{typst.app/universe/package/resume-ng}

\phantomsection\label{banner}
\phantomsection\label{template-thumbnail}
\pandocbounded{\includegraphics[keepaspectratio]{https://packages.typst.org/preview/thumbnails/resume-ng-1.0.0-small.webp}}

\section{resume-ng}\label{resume-ng}

{ 1.0.0 }

A Typst resume designed for optimal information density and aesthetic
appeal.

\href{/app?template=resume-ng&version=1.0.0}{Create project in app}

\phantomsection\label{readme}
A typst resume designed for optimal information density and aesthetic
appeal.

A LaTeX version

\texttt{\ main.typ\ } will be a good start.

A minimal exmaple would be:

\begin{Shaded}
\begin{Highlighting}[]
\NormalTok{\#show: project.with(}
\NormalTok{  title: "Resume{-}ng",}
\NormalTok{  author: (name: "FengKaiyu"),}
\NormalTok{  contacts: }
\NormalTok{    (}
\NormalTok{      "+86 188{-}888{-}8888",}
\NormalTok{       link("https://github.com", "github.com/fky2015"),  }
\NormalTok{      // More items...}
\NormalTok{    )}
\NormalTok{)}

\NormalTok{\#resume{-}section("Educations")}
\NormalTok{\#resume{-}education(}
\NormalTok{  university: "BIT",}
\NormalTok{  degree: "Your degree",}
\NormalTok{  school: "Your Major and school",}
\NormalTok{  start: "2021{-}09",}
\NormalTok{  end: "2024{-}06"}
\NormalTok{)[}
\NormalTok{*GPA: 3.62/4.0*. My main research interest }
\NormalTok{is in \#strong("Byzantine Consensus Algorithm"), }
\NormalTok{and I have some research and engineering experience in the field of distributed systems.}
\NormalTok{]}

\NormalTok{\#resume{-}section[Work Experience]}
\NormalTok{\#resume{-}work(}
\NormalTok{  company: "A company",}
\NormalTok{  duty: "Your duty",}
\NormalTok{  start: "2020.10",}
\NormalTok{  end: "2021.03",}
\NormalTok{)[}
\NormalTok{  {-} *Independently responsible for the design, development, testing and deployment of XXX business backend.* Implemented station letter template rendering service through FaaS, Kafka and other platforms. Provided SDK code to upstream, added or upgraded various offline and online logic.}
\NormalTok{  {-} *Participate in XXX\textquotesingle{}s requirement analysis, system technical solution design; complete requirement development, grey scale testing, go{-}live and monitoring.*}
\NormalTok{]}

\NormalTok{\#resume{-}section[Projects]}

\NormalTok{\#resume{-}project(}
\NormalTok{  title: "Project name",}
\NormalTok{  duty: "Your duty",}
\NormalTok{  start: "2021.11",}
\NormalTok{  end: "2022.07",}
\NormalTok{)[}
\NormalTok{  {-} Implemented a memory pool manager based on an extensible hash table and LRU{-}K, and developed a concurrent B+ tree supporting optimistic locking for read and write operations.}
\NormalTok{  {-} Utilized the volcano model to implement executors for queries, updates, joins, and aggregations, and performed query rewriting and pushing down optimizations.}
\NormalTok{  {-} Implemented concurrency control using 2PL (two{-}phase locking), supporting deadlock handling, multiple isolation levels, table locks, and row locks.}
\NormalTok{]}
\end{Highlighting}
\end{Shaded}

\href{/app?template=resume-ng&version=1.0.0}{Create project in app}

\subsubsection{How to use}\label{how-to-use}

Click the button above to create a new project using this template in
the Typst app.

You can also use the Typst CLI to start a new project on your computer
using this command:

\begin{verbatim}
typst init @preview/resume-ng:1.0.0
\end{verbatim}

\includesvg[width=0.16667in,height=0.16667in]{/assets/icons/16-copy.svg}

\subsubsection{About}\label{about}

\begin{description}
\tightlist
\item[Author :]
\href{https://github.com/fky2015}{FengKaiyu}
\item[License:]
MIT
\item[Current version:]
1.0.0
\item[Last updated:]
October 8, 2024
\item[First released:]
October 8, 2024
\item[Archive size:]
5.27 kB
\href{https://packages.typst.org/preview/resume-ng-1.0.0.tar.gz}{\pandocbounded{\includesvg[keepaspectratio]{/assets/icons/16-download.svg}}}
\item[Repository:]
\href{https://github.com/fky2015/resume-ng-typst}{GitHub}
\item[Categor y :]
\begin{itemize}
\tightlist
\item[]
\item
  \pandocbounded{\includesvg[keepaspectratio]{/assets/icons/16-user.svg}}
  \href{https://typst.app/universe/search/?category=cv}{CV}
\end{itemize}
\end{description}

\subsubsection{Where to report issues?}\label{where-to-report-issues}

This template is a project of FengKaiyu . Report issues on
\href{https://github.com/fky2015/resume-ng-typst}{their repository} .
You can also try to ask for help with this template on the
\href{https://forum.typst.app}{Forum} .

Please report this template to the Typst team using the
\href{https://typst.app/contact}{contact form} if you believe it is a
safety hazard or infringes upon your rights.

\phantomsection\label{versions}
\subsubsection{Version history}\label{version-history}

\begin{longtable}[]{@{}ll@{}}
\toprule\noalign{}
Version & Release Date \\
\midrule\noalign{}
\endhead
\bottomrule\noalign{}
\endlastfoot
1.0.0 & October 8, 2024 \\
\end{longtable}

Typst GmbH did not create this template and cannot guarantee correct
functionality of this template or compatibility with any version of the
Typst compiler or app.


\section{Package List LaTeX/dvdtyp.tex}
\title{typst.app/universe/package/dvdtyp}

\phantomsection\label{banner}
\phantomsection\label{template-thumbnail}
\pandocbounded{\includegraphics[keepaspectratio]{https://packages.typst.org/preview/thumbnails/dvdtyp-1.0.0-small.webp}}

\section{dvdtyp}\label{dvdtyp}

{ 1.0.0 }

a colorful template for writting handouts or notes

\href{/app?template=dvdtyp&version=1.0.0}{Create project in app}

\phantomsection\label{readme}
A colorful template for writting handouts or notes

\pandocbounded{\includegraphics[keepaspectratio]{https://github.com/typst/packages/raw/main/packages/preview/dvdtyp/1.0.0/thumbnail.png}}

\href{/app?template=dvdtyp&version=1.0.0}{Create project in app}

\subsubsection{How to use}\label{how-to-use}

Click the button above to create a new project using this template in
the Typst app.

You can also use the Typst CLI to start a new project on your computer
using this command:

\begin{verbatim}
typst init @preview/dvdtyp:1.0.0
\end{verbatim}

\includesvg[width=0.16667in,height=0.16667in]{/assets/icons/16-copy.svg}

\subsubsection{About}\label{about}

\begin{description}
\tightlist
\item[Author :]
DVDTSB
\item[License:]
MIT-0
\item[Current version:]
1.0.0
\item[Last updated:]
July 10, 2024
\item[First released:]
July 10, 2024
\item[Archive size:]
3.21 kB
\href{https://packages.typst.org/preview/dvdtyp-1.0.0.tar.gz}{\pandocbounded{\includesvg[keepaspectratio]{/assets/icons/16-download.svg}}}
\item[Repository:]
\href{https://github.com/DVDTSB/dvdtyp}{GitHub}
\item[Categor ies :]
\begin{itemize}
\tightlist
\item[]
\item
  \pandocbounded{\includesvg[keepaspectratio]{/assets/icons/16-smile.svg}}
  \href{https://typst.app/universe/search/?category=fun}{Fun}
\item
  \pandocbounded{\includesvg[keepaspectratio]{/assets/icons/16-layout.svg}}
  \href{https://typst.app/universe/search/?category=layout}{Layout}
\item
  \pandocbounded{\includesvg[keepaspectratio]{/assets/icons/16-text.svg}}
  \href{https://typst.app/universe/search/?category=text}{Text}
\end{itemize}
\end{description}

\subsubsection{Where to report issues?}\label{where-to-report-issues}

This template is a project of DVDTSB . Report issues on
\href{https://github.com/DVDTSB/dvdtyp}{their repository} . You can also
try to ask for help with this template on the
\href{https://forum.typst.app}{Forum} .

Please report this template to the Typst team using the
\href{https://typst.app/contact}{contact form} if you believe it is a
safety hazard or infringes upon your rights.

\phantomsection\label{versions}
\subsubsection{Version history}\label{version-history}

\begin{longtable}[]{@{}ll@{}}
\toprule\noalign{}
Version & Release Date \\
\midrule\noalign{}
\endhead
\bottomrule\noalign{}
\endlastfoot
1.0.0 & July 10, 2024 \\
\end{longtable}

Typst GmbH did not create this template and cannot guarantee correct
functionality of this template or compatibility with any version of the
Typst compiler or app.


\section{Package List LaTeX/icicle.tex}
\title{typst.app/universe/package/icicle}

\phantomsection\label{banner}
\phantomsection\label{template-thumbnail}
\pandocbounded{\includegraphics[keepaspectratio]{https://packages.typst.org/preview/thumbnails/icicle-0.1.0-small.webp}}

\section{icicle}\label{icicle}

{ 0.1.0 }

Help the Typst Guys reach the helicopter pad and save Christmas!

\href{/app?template=icicle&version=0.1.0}{Create project in app}

\phantomsection\label{readme}
Help the Typst Guys reach the helicopter pad and save Christmas!
Navigate them with the WASD keys and solve puzzles with snowballs to
make way for the Typst Guys.

This small Christmas-themed game is playable in the Typst editor and
best enjoyed with the web app or \texttt{\ typst\ watch\ } . It was
first released for the 24 Days to Christmas campaign in winter of 2023.

\subsection{Usage}\label{usage}

You can use this template in the Typst web app by clicking “Start from
template� on the dashboard and searching for \texttt{\ icicle\ } .

Alternatively, you can use the CLI to kick this project off using the
command

\begin{verbatim}
typst init @preview/icicle
\end{verbatim}

Typst will create a new directory with all the files needed to get you
started.

\subsection{Configuration}\label{configuration}

This template exports the \texttt{\ game\ } function, which accepts a
positional argument for the game input.

The template will initialize your package with a sample call to the
\texttt{\ game\ } function in a show rule. If you want to change an
existing project to use this template, you can add a show rule like this
at the top of your file:

\begin{Shaded}
\begin{Highlighting}[]
\NormalTok{\#import "@preview/icicle:0.1.0": game}
\NormalTok{\#show: game}

\NormalTok{// Move with WASD.}
\end{Highlighting}
\end{Shaded}

You can also add your own levels by adding an array of level definition
strings in the \texttt{\ game\ } function’s named \texttt{\ levels\ }
argument. Each level file must conform to the following format:

\begin{itemize}
\tightlist
\item
  First, a line with two comma separated integers indicating the
  player’s starting position.
\item
  Then, a matrix with the characters f (floor), x (wall), w (water), or
  g (goal).
\item
  Finally, a matrix with the characters b (snowball) or \_ (nothing).
\end{itemize}

The three arguments must be separated by double newlines. Additionally,
each row in the matrices space-separates its values. Newlines terminate
the rows. Comments can be added with a double slash. Find an example for
a valid level string below:

\begin{verbatim}
// The starting position
0, 0

// The back layer
f f f w f f f
f f f w f f f
f f x w f f f
f f f w f f f
f f f w f x x
x x x g x x x

// The front layer.
_ _ b _ _ _ _
_ _ b _ _ _ _
_ _ _ _ b _ _
_ _ _ _ b _ _
_ _ _ _ _ _ _
_ _ _ _ _ _ _
\end{verbatim}

It’s best to put levels into separate files and load them with the
\texttt{\ read\ } function.

\href{/app?template=icicle&version=0.1.0}{Create project in app}

\subsubsection{How to use}\label{how-to-use}

Click the button above to create a new project using this template in
the Typst app.

You can also use the Typst CLI to start a new project on your computer
using this command:

\begin{verbatim}
typst init @preview/icicle:0.1.0
\end{verbatim}

\includesvg[width=0.16667in,height=0.16667in]{/assets/icons/16-copy.svg}

\subsubsection{About}\label{about}

\begin{description}
\tightlist
\item[Author :]
\href{https://typst.app}{Typst GmbH}
\item[License:]
MIT-0
\item[Current version:]
0.1.0
\item[Last updated:]
March 6, 2024
\item[First released:]
March 6, 2024
\item[Minimum Typst version:]
0.8.0
\item[Archive size:]
143 kB
\href{https://packages.typst.org/preview/icicle-0.1.0.tar.gz}{\pandocbounded{\includesvg[keepaspectratio]{/assets/icons/16-download.svg}}}
\item[Repository:]
\href{https://github.com/typst/templates}{GitHub}
\item[Categor y :]
\begin{itemize}
\tightlist
\item[]
\item
  \pandocbounded{\includesvg[keepaspectratio]{/assets/icons/16-smile.svg}}
  \href{https://typst.app/universe/search/?category=fun}{Fun}
\end{itemize}
\end{description}

\subsubsection{Where to report issues?}\label{where-to-report-issues}

This template is a project of Typst GmbH . Report issues on
\href{https://github.com/typst/templates}{their repository} . You can
also try to ask for help with this template on the
\href{https://forum.typst.app}{Forum} .

\phantomsection\label{versions}
\subsubsection{Version history}\label{version-history}

\begin{longtable}[]{@{}ll@{}}
\toprule\noalign{}
Version & Release Date \\
\midrule\noalign{}
\endhead
\bottomrule\noalign{}
\endlastfoot
0.1.0 & March 6, 2024 \\
\end{longtable}


\section{Package List LaTeX/georges-yetyp.tex}
\title{typst.app/universe/package/georges-yetyp}

\phantomsection\label{banner}
\phantomsection\label{template-thumbnail}
\pandocbounded{\includegraphics[keepaspectratio]{https://packages.typst.org/preview/thumbnails/georges-yetyp-0.2.0-small.webp}}

\section{georges-yetyp}\label{georges-yetyp}

{ 0.2.0 }

Unofficial template for Polytech Grenoble internship reports

\href{/app?template=georges-yetyp&version=0.2.0}{Create project in app}

\phantomsection\label{readme}
\href{https://github.com/typst/packages/raw/main/packages/preview/georges-yetyp/0.2.0/README.fr.md}{French
version}

Typst template for Polytech (Grenoble) internship reports.

\href{https://github.com/typst/packages/raw/main/packages/preview/georges-yetyp/0.2.0/thumbnail.png}{\pandocbounded{\includegraphics[keepaspectratio]{https://github.com/typst/packages/raw/main/packages/preview/georges-yetyp/0.2.0/thumbnail.png}}}

\subsection{Usage}\label{usage}

Either use this template
\href{https://typst.app/?template=georges-yetyp&version=0.1.0}{in the
Typst web app} , or use the command line to initialize a new project
based on this template:

\begin{Shaded}
\begin{Highlighting}[]
\ExtensionTok{typst}\NormalTok{ init @preview/georges{-}yetyp}
\end{Highlighting}
\end{Shaded}

Then, replace \texttt{\ logo.png\ } with the logo of the company you
worked for, fill in all the details in the \texttt{\ rapport\ }
parameters, and start writing below.

\subsection{Other schools}\label{other-schools}

Adding support for other schools of the Polytech network would be fairly
easy if you want to re-use this template. All that is needed is a copy
of their logo (with the authorization to use it). Submissions are
welcome.

\href{/app?template=georges-yetyp&version=0.2.0}{Create project in app}

\subsubsection{How to use}\label{how-to-use}

Click the button above to create a new project using this template in
the Typst app.

You can also use the Typst CLI to start a new project on your computer
using this command:

\begin{verbatim}
typst init @preview/georges-yetyp:0.2.0
\end{verbatim}

\includesvg[width=0.16667in,height=0.16667in]{/assets/icons/16-copy.svg}

\subsubsection{About}\label{about}

\begin{description}
\tightlist
\item[Author :]
\href{https://ana.gelez.xyz/}{Ana Gelez}
\item[License:]
GPL-3.0
\item[Current version:]
0.2.0
\item[Last updated:]
September 26, 2024
\item[First released:]
July 5, 2024
\item[Minimum Typst version:]
0.11.0
\item[Archive size:]
116 kB
\href{https://packages.typst.org/preview/georges-yetyp-0.2.0.tar.gz}{\pandocbounded{\includesvg[keepaspectratio]{/assets/icons/16-download.svg}}}
\item[Repository:]
\href{https://github.com/elegaanz/georges-yetyp}{GitHub}
\item[Discipline s :]
\begin{itemize}
\tightlist
\item[]
\item
  \href{https://typst.app/universe/search/?discipline=engineering}{Engineering}
\item
  \href{https://typst.app/universe/search/?discipline=computer-science}{Computer
  Science}
\item
  \href{https://typst.app/universe/search/?discipline=geology}{Geology}
\item
  \href{https://typst.app/universe/search/?discipline=medicine}{Medicine}
\end{itemize}
\item[Categor y :]
\begin{itemize}
\tightlist
\item[]
\item
  \pandocbounded{\includesvg[keepaspectratio]{/assets/icons/16-speak.svg}}
  \href{https://typst.app/universe/search/?category=report}{Report}
\end{itemize}
\end{description}

\subsubsection{Where to report issues?}\label{where-to-report-issues}

This template is a project of Ana Gelez . Report issues on
\href{https://github.com/elegaanz/georges-yetyp}{their repository} . You
can also try to ask for help with this template on the
\href{https://forum.typst.app}{Forum} .

Please report this template to the Typst team using the
\href{https://typst.app/contact}{contact form} if you believe it is a
safety hazard or infringes upon your rights.

\phantomsection\label{versions}
\subsubsection{Version history}\label{version-history}

\begin{longtable}[]{@{}ll@{}}
\toprule\noalign{}
Version & Release Date \\
\midrule\noalign{}
\endhead
\bottomrule\noalign{}
\endlastfoot
0.2.0 & September 26, 2024 \\
\href{https://typst.app/universe/package/georges-yetyp/0.1.0/}{0.1.0} &
July 5, 2024 \\
\end{longtable}

Typst GmbH did not create this template and cannot guarantee correct
functionality of this template or compatibility with any version of the
Typst compiler or app.


\section{Package List LaTeX/knowledge-key.tex}
\title{typst.app/universe/package/knowledge-key}

\phantomsection\label{banner}
\phantomsection\label{template-thumbnail}
\pandocbounded{\includegraphics[keepaspectratio]{https://packages.typst.org/preview/thumbnails/knowledge-key-1.0.1-small.webp}}

\section{knowledge-key}\label{knowledge-key}

{ 1.0.1 }

A compact cheat-sheet

\href{/app?template=knowledge-key&version=1.0.1}{Create project in app}

\phantomsection\label{readme}
This is a typst template for a compact cheat-sheet.

\subsection{Usage}\label{usage}

You can use this template in the Typst web app by clicking “Start from
template� on the dashboard and searching for
\texttt{\ knowledge-key\ } . Alternatively, you can use the CLI to kick
this project off using the command

\begin{verbatim}
typst init @preview/knowledge-key
\end{verbatim}

Typst will create a new directory with all the files needed to get you
started.

\subsection{Configuration}\label{configuration}

This template exports the \texttt{\ ieee\ } function with the following
named arguments:

\begin{itemize}
\tightlist
\item
  \texttt{\ title\ } : The title of the cheat-sheet
\item
  \texttt{\ authors\ } : A string of authors
\end{itemize}

The function also accepts a single, positional argument for the body of
the paper.

The template will initialize your package with a sample call to the
\texttt{\ knowledge-key\ } function in a show rule. If you want to
change an existing project to use this template, you can add a show rule
like this at the top of your file:

\begin{Shaded}
\begin{Highlighting}[]
\NormalTok{\#import "@preview/knowledge{-}key:1.0.1": *}

\NormalTok{\#show: knowledge{-}key.with(}
\NormalTok{  title: [Title],}
\NormalTok{  authors: "Author1, Author2"}
\NormalTok{)}

\NormalTok{// Your content goes below.}
\end{Highlighting}
\end{Shaded}

\href{/app?template=knowledge-key&version=1.0.1}{Create project in app}

\subsubsection{How to use}\label{how-to-use}

Click the button above to create a new project using this template in
the Typst app.

You can also use the Typst CLI to start a new project on your computer
using this command:

\begin{verbatim}
typst init @preview/knowledge-key:1.0.1
\end{verbatim}

\includesvg[width=0.16667in,height=0.16667in]{/assets/icons/16-copy.svg}

\subsubsection{About}\label{about}

\begin{description}
\tightlist
\item[Author :]
Nick Goetti
\item[License:]
MIT-0
\item[Current version:]
1.0.1
\item[Last updated:]
October 25, 2024
\item[First released:]
October 1, 2024
\item[Minimum Typst version:]
0.11.0
\item[Archive size:]
166 kB
\href{https://packages.typst.org/preview/knowledge-key-1.0.1.tar.gz}{\pandocbounded{\includesvg[keepaspectratio]{/assets/icons/16-download.svg}}}
\item[Repository:]
\href{https://github.com/ngoetti/knowledge-key}{GitHub}
\item[Discipline s :]
\begin{itemize}
\tightlist
\item[]
\item
  \href{https://typst.app/universe/search/?discipline=computer-science}{Computer
  Science}
\item
  \href{https://typst.app/universe/search/?discipline=engineering}{Engineering}
\end{itemize}
\item[Categor y :]
\begin{itemize}
\tightlist
\item[]
\item
  \pandocbounded{\includesvg[keepaspectratio]{/assets/icons/16-map.svg}}
  \href{https://typst.app/universe/search/?category=flyer}{Flyer}
\end{itemize}
\end{description}

\subsubsection{Where to report issues?}\label{where-to-report-issues}

This template is a project of Nick Goetti . Report issues on
\href{https://github.com/ngoetti/knowledge-key}{their repository} . You
can also try to ask for help with this template on the
\href{https://forum.typst.app}{Forum} .

Please report this template to the Typst team using the
\href{https://typst.app/contact}{contact form} if you believe it is a
safety hazard or infringes upon your rights.

\phantomsection\label{versions}
\subsubsection{Version history}\label{version-history}

\begin{longtable}[]{@{}ll@{}}
\toprule\noalign{}
Version & Release Date \\
\midrule\noalign{}
\endhead
\bottomrule\noalign{}
\endlastfoot
1.0.1 & October 25, 2024 \\
\href{https://typst.app/universe/package/knowledge-key/1.0.0/}{1.0.0} &
October 1, 2024 \\
\end{longtable}

Typst GmbH did not create this template and cannot guarantee correct
functionality of this template or compatibility with any version of the
Typst compiler or app.


\section{Package List LaTeX/cram-snap.tex}
\title{typst.app/universe/package/cram-snap}

\phantomsection\label{banner}
\phantomsection\label{template-thumbnail}
\pandocbounded{\includegraphics[keepaspectratio]{https://packages.typst.org/preview/thumbnails/cram-snap-0.2.1-small.webp}}

\section{cram-snap}\label{cram-snap}

{ 0.2.1 }

Compact and legible cheat sheets

{ } Featured Template

\href{/app?template=cram-snap&version=0.2.1}{Create project in app}

\phantomsection\label{readme}
Simple cheatsheet template for \href{https://typst.app/}{Typst} that
allows you to snap a quick picture of essential information and cram it
into a useful cheatsheet format.

\subsection{Usage}\label{usage}

You can use this template in the Typst web app by clicking “Start from
template� on the dashboard and searching for \texttt{\ cram-snap\ } .

Alternatively, you can use the CLI to kick this project off using the
command

\begin{verbatim}
typst init @preview/cram-snap
\end{verbatim}

Typst will create a new directory with all the files needed to get you
started.

\subsection{Configuration}\label{configuration}

This template exports the \texttt{\ cram-snap\ } function with the
following named arguments:

\begin{itemize}
\tightlist
\item
  \texttt{\ title\ } : Title of the document
\item
  \texttt{\ subtitle\ } : Subtitle of the document
\item
  \texttt{\ icon\ } : Image that appears next to the title
\item
  \texttt{\ column-number\ } : Number of columns
\end{itemize}

The \texttt{\ theader\ } function is a wrapper around the
\texttt{\ table.header\ } function that creates a header and takes
\texttt{\ colspan\ } as argument to span the header across multiple
table columns (by default it spans across two)

If you want to change an existing project to use this template, you can
add a show rule like this at the top of your file:

\begin{Shaded}
\begin{Highlighting}[]
\NormalTok{\#import "@preview/cram{-}snap:0.2.1": *}

\NormalTok{\#set page(paper: "a4", flipped: true, margin: 1cm)}
\NormalTok{\#set text(font: "Arial", size: 11pt)}

\NormalTok{\#show: cram{-}snap.with(}
\NormalTok{  title: [Cheatsheet],}
\NormalTok{  subtitle: [Cheatsheet for an amazing program],}
\NormalTok{  icon: image("icon.png"),}
\NormalTok{  column{-}number: 3,}
\NormalTok{)}

\NormalTok{// Use it if you want different table columns (the default are: (2fr, 3fr))}
\NormalTok{\#set table(columns: (2fr, 3fr, 3fr))}

\NormalTok{\#table(}
\NormalTok{  theader(colspan: 3)[Great heading that is really looooong],}
\NormalTok{  [Closing the program], [Type \textasciigrave{}:q\textasciigrave{}], [You can also type \textasciigrave{}QQ\textasciigrave{}]}
\NormalTok{)}
\end{Highlighting}
\end{Shaded}

\href{/app?template=cram-snap&version=0.2.1}{Create project in app}

\subsubsection{How to use}\label{how-to-use}

Click the button above to create a new project using this template in
the Typst app.

You can also use the Typst CLI to start a new project on your computer
using this command:

\begin{verbatim}
typst init @preview/cram-snap:0.2.1
\end{verbatim}

\includesvg[width=0.16667in,height=0.16667in]{/assets/icons/16-copy.svg}

\subsubsection{About}\label{about}

\begin{description}
\tightlist
\item[Author :]
kamack38
\item[License:]
MIT
\item[Current version:]
0.2.1
\item[Last updated:]
October 25, 2024
\item[First released:]
May 13, 2024
\item[Archive size:]
3.79 kB
\href{https://packages.typst.org/preview/cram-snap-0.2.1.tar.gz}{\pandocbounded{\includesvg[keepaspectratio]{/assets/icons/16-download.svg}}}
\item[Repository:]
\href{https://github.com/kamack38/cram-snap}{GitHub}
\item[Categor y :]
\begin{itemize}
\tightlist
\item[]
\item
  \pandocbounded{\includesvg[keepaspectratio]{/assets/icons/16-map.svg}}
  \href{https://typst.app/universe/search/?category=flyer}{Flyer}
\end{itemize}
\end{description}

\subsubsection{Where to report issues?}\label{where-to-report-issues}

This template is a project of kamack38 . Report issues on
\href{https://github.com/kamack38/cram-snap}{their repository} . You can
also try to ask for help with this template on the
\href{https://forum.typst.app}{Forum} .

Please report this template to the Typst team using the
\href{https://typst.app/contact}{contact form} if you believe it is a
safety hazard or infringes upon your rights.

\phantomsection\label{versions}
\subsubsection{Version history}\label{version-history}

\begin{longtable}[]{@{}ll@{}}
\toprule\noalign{}
Version & Release Date \\
\midrule\noalign{}
\endhead
\bottomrule\noalign{}
\endlastfoot
0.2.1 & October 25, 2024 \\
\href{https://typst.app/universe/package/cram-snap/0.2.0/}{0.2.0} &
October 15, 2024 \\
\href{https://typst.app/universe/package/cram-snap/0.1.0/}{0.1.0} & May
13, 2024 \\
\end{longtable}

Typst GmbH did not create this template and cannot guarantee correct
functionality of this template or compatibility with any version of the
Typst compiler or app.


\section{Package List LaTeX/cvssc.tex}
\title{typst.app/universe/package/cvssc}

\phantomsection\label{banner}
\section{cvssc}\label{cvssc}

{ 0.1.0 }

Common Vulnerability Scoring System Calculator

\phantomsection\label{readme}
\phantomsection\label{readme-top}{}

\href{https://github.com/DrakeAxelrod/cvssc/graphs/contributors}{\pandocbounded{\includegraphics[keepaspectratio]{https://img.shields.io/github/contributors/DrakeAxelrod/cvssc.svg?style=for-the-badge}}}
\href{https://github.com/DrakeAxelrod/cvssc/network/members}{\pandocbounded{\includegraphics[keepaspectratio]{https://img.shields.io/github/forks/DrakeAxelrod/cvssc.svg?style=for-the-badge}}}
\href{https://github.com/DrakeAxelrod/cvssc/stargazers}{\pandocbounded{\includegraphics[keepaspectratio]{https://img.shields.io/github/stars/DrakeAxelrod/cvssc.svg?style=for-the-badge}}}
\href{https://github.com/DrakeAxelrod/cvssc/issues}{\pandocbounded{\includegraphics[keepaspectratio]{https://img.shields.io/github/issues/DrakeAxelrod/cvssc.svg?style=for-the-badge}}}
\href{https://github.com/DrakeAxelrod/cvssc/blob/master/LICENSE.txt}{\pandocbounded{\includegraphics[keepaspectratio]{https://img.shields.io/github/license/DrakeAxelrod/cvssc.svg?style=for-the-badge}}}

\hfill\break

\subsubsection{cvssc}\label{cvssc-1}

\paragraph{Common Vulnerability Scoring System
Calculator}\label{common-vulnerability-scoring-system-calculator}

The CVSS Typst Library is a \href{https://github.com/typst/}{Typst}
package designed to facilitate the calculation of Common Vulnerability
Scoring System (CVSS) scores for vulnerabilities across multiple
versions, including CVSS 2.0, 3.0, 3.1, and 4.0. This library provides
developers, security analysts, and researchers with a reliable and
efficient toolset for assessing the severity of security vulnerabilities
based on the CVSS standards.\\
\href{https://github.com/DrakeAxelrod/cvssc/tree/main/cvssc/0.1.0/src/docs.pdf}{\textbf{Explore
the docs »}}\\
\strut \\
\href{https://github.com/DrakeAxelrod/cvssc/issues}{Report Bug} ·
\href{https://github.com/DrakeAxelrod/cvssc/issues}{Request Feature}

Table of Contents

\begin{enumerate}
\tightlist
\item
  \href{https://github.com/typst/packages/raw/main/packages/preview/cvssc/0.1.0/\#about-the-project}{About
  The Project}

  \begin{itemize}
  \tightlist
  \item
    \href{https://github.com/typst/packages/raw/main/packages/preview/cvssc/0.1.0/\#built-with}{Built
    With}
  \end{itemize}
\item
  \href{https://github.com/typst/packages/raw/main/packages/preview/cvssc/0.1.0/\#getting-started}{Getting
  Started}

  \begin{itemize}
  \tightlist
  \item
    \href{https://github.com/typst/packages/raw/main/packages/preview/cvssc/0.1.0/\#prerequisites}{Prerequisites}
  \item
    \href{https://github.com/typst/packages/raw/main/packages/preview/cvssc/0.1.0/\#installation}{Installation}
  \end{itemize}
\item
  \href{https://github.com/typst/packages/raw/main/packages/preview/cvssc/0.1.0/\#usage}{Usage}
\item
  \href{https://github.com/typst/packages/raw/main/packages/preview/cvssc/0.1.0/\#roadmap}{Roadmap}
\item
  \href{https://github.com/typst/packages/raw/main/packages/preview/cvssc/0.1.0/\#contributing}{Contributing}
\item
  \href{https://github.com/typst/packages/raw/main/packages/preview/cvssc/0.1.0/\#license}{License}
\item
  \href{https://github.com/typst/packages/raw/main/packages/preview/cvssc/0.1.0/\#contact}{Contact}
\item
  \href{https://github.com/typst/packages/raw/main/packages/preview/cvssc/0.1.0/\#acknowledgments}{Acknowledgments}
\end{enumerate}

(
\href{https://github.com/typst/packages/raw/main/packages/preview/cvssc/0.1.0/\#readme-top}{back
to top} )

\subsubsection{Built With}\label{built-with}

\begin{itemize}
\tightlist
\item
  \href{https://typst.app/}{\pandocbounded{\includegraphics[keepaspectratio]{https://img.shields.io/badge/Typst-239dad?style=for-the-badge&logo=typst&logoColor=white}}}
\item
  \href{https://www.rust-lang.org/}{\pandocbounded{\includegraphics[keepaspectratio]{https://img.shields.io/badge/Rust-b7410e?style=for-the-badge&logo=rust&logoColor=white}}}
\item
  \href{https://webassembly.org/}{\pandocbounded{\includegraphics[keepaspectratio]{https://img.shields.io/badge/WebAssembly-654FF0?style=for-the-badge&logo=webassembly&logoColor=white}}}
\end{itemize}

(
\href{https://github.com/typst/packages/raw/main/packages/preview/cvssc/0.1.0/\#readme-top}{back
to top} )

\subsection{Getting Started}\label{getting-started}

Ensure you have the Typst CLI installed.

\begin{enumerate}
\tightlist
\item
  Import the package
\end{enumerate}

\begin{Shaded}
\begin{Highlighting}[]
\NormalTok{\#import "@preview/cvssc:0.1.0";}
\end{Highlighting}
\end{Shaded}

\begin{enumerate}
\setcounter{enumi}{1}
\tightlist
\item
  Use the various library functions to calculate CVSS scores and
  severities.
\end{enumerate}

\begin{Shaded}
\begin{Highlighting}[]
\NormalTok{\#import "@preview/cvssc:0.1.0";}

\NormalTok{\#cvssc.v2("CVSS:2.0/AV:L/AC:H/Au:M/C:P/I:C/A:C")}

\NormalTok{\#cvssc.v3("CVSS:3.0/AV:N/AC:L/PR:N/UI:N/S:U/C:H/I:H/A:H")}

\NormalTok{\#cvssc.v3("CVSS:3.1/AV:N/AC:L/PR:N/UI:N/S:U/C:L/I:L/A:H")}

\NormalTok{\#cvssc.v4("CVSS:4.0/AV:A/AC:H/AT:P/PR:L/UI:P/VC:H/VI:H/VA:L/SC:L/SI:L/SA:L")}
\end{Highlighting}
\end{Shaded}

\subsubsection{Prerequisites}\label{prerequisites}

\begin{itemize}
\tightlist
\item
  typst (see \href{https://typst.app/}{Typst} )
\end{itemize}

\subsection{Usage}\label{usage}

\emph{Please refer to the
\href{https://github.com/typst/packages/raw/main/packages/preview/cvssc/0.1.0/cvssc/0.1.0/src/docs.pdf}{Docs}}

(
\href{https://github.com/typst/packages/raw/main/packages/preview/cvssc/0.1.0/\#readme-top}{back
to top} )

\subsection{Roadmap}\label{roadmap}

See the \href{https://github.com/DrakeAxelrod/cvssc/issues}{open issues}
for a full list of proposed features (and known issues).

(
\href{https://github.com/typst/packages/raw/main/packages/preview/cvssc/0.1.0/\#readme-top}{back
to top} )

\subsection{Contributing}\label{contributing}

Contributions are what make the open source community such an amazing
place to learn, inspire, and create. Any contributions you make are
\textbf{greatly appreciated} .

If you have a suggestion that would make this better, please fork the
repo and create a pull request. You can also simply open an issue with
the tag “enhancement�. Don’t forget to give the project a star!
Thanks again!

\begin{enumerate}
\tightlist
\item
  Fork the Project
\item
  Create your Feature Branch (
  \texttt{\ git\ checkout\ -b\ feature/AmazingFeature\ } )
\item
  Commit your Changes (
  \texttt{\ git\ commit\ -m\ \textquotesingle{}Add\ some\ AmazingFeature\textquotesingle{}\ }
  )
\item
  Push to the Branch (
  \texttt{\ git\ push\ origin\ feature/AmazingFeature\ } )
\item
  Open a Pull Request
\end{enumerate}

(
\href{https://github.com/typst/packages/raw/main/packages/preview/cvssc/0.1.0/\#readme-top}{back
to top} )

\subsection{License}\label{license}

Distributed under the MIT License. See \texttt{\ LICENSE.txt\ } for more
information.

(
\href{https://github.com/typst/packages/raw/main/packages/preview/cvssc/0.1.0/\#readme-top}{back
to top} )

\subsection{Contact}\label{contact}

Drake Axelrod -
\href{https://github.com/typst/packages/raw/main/packages/preview/cvssc/0.1.0/\%5Bhttps://github/\%5D(https://github.com/DrakeAxelrod/)}{Github
Profile} -
\href{mailto:drakeaxelrod@gmail.com}{\nolinkurl{drakeaxelrod@gmail.com}}

Project Link: \url{https://github.com/DrakeAxelrod/cvssc}

\subsection{Contributors}\label{contributors}

\begin{longtable}[]{@{}
  >{\centering\arraybackslash}p{(\linewidth - 0\tabcolsep) * \real{1.0000}}@{}}
\toprule\noalign{}
\endhead
\bottomrule\noalign{}
\endlastfoot
\begin{minipage}[t]{\linewidth}\centering
\href{https://github.com/DrakeAxelrod}{\pandocbounded{\includegraphics[keepaspectratio]{https://avatars.githubusercontent.com/u/51012876?v=4?s=64}}\\
\textsubscript{\textbf{Drake Axelrod}}}\\
\strut
\end{minipage} \\
\end{longtable}

(
\href{https://github.com/typst/packages/raw/main/packages/preview/cvssc/0.1.0/\#readme-top}{back
to top} )

\subsection{Acknowledgments}\label{acknowledgments}

\begin{itemize}
\tightlist
\item
  \href{https://typst.app/}{Typst}
\item
  \href{https://www.first.org/}{First.org}
\item
  \href{https://docs.rs/nvd-cvss}{Rust Library - nvd-cvss}
\end{itemize}

(
\href{https://github.com/typst/packages/raw/main/packages/preview/cvssc/0.1.0/\#readme-top}{back
to top} )

\subsubsection{How to add}\label{how-to-add}

Copy this into your project and use the import as \texttt{\ cvssc\ }

\begin{verbatim}
#import "@preview/cvssc:0.1.0"
\end{verbatim}

\includesvg[width=0.16667in,height=0.16667in]{/assets/icons/16-copy.svg}

Check the docs for
\href{https://typst.app/docs/reference/scripting/\#packages}{more
information on how to import packages} .

\subsubsection{About}\label{about}

\begin{description}
\tightlist
\item[Author :]
\href{mailto:drakeaxelrod@gmail.com}{Drake Axelrod}
\item[License:]
MIT
\item[Current version:]
0.1.0
\item[Last updated:]
October 28, 2024
\item[First released:]
October 28, 2024
\item[Minimum Typst version:]
0.12.0
\item[Archive size:]
193 kB
\href{https://packages.typst.org/preview/cvssc-0.1.0.tar.gz}{\pandocbounded{\includesvg[keepaspectratio]{/assets/icons/16-download.svg}}}
\end{description}

\subsubsection{Where to report issues?}\label{where-to-report-issues}

This package is a project of Drake Axelrod . You can also try to ask for
help with this package on the \href{https://forum.typst.app}{Forum} .

Please report this package to the Typst team using the
\href{https://typst.app/contact}{contact form} if you believe it is a
safety hazard or infringes upon your rights.

\phantomsection\label{versions}
\subsubsection{Version history}\label{version-history}

\begin{longtable}[]{@{}ll@{}}
\toprule\noalign{}
Version & Release Date \\
\midrule\noalign{}
\endhead
\bottomrule\noalign{}
\endlastfoot
0.1.0 & October 28, 2024 \\
\end{longtable}

Typst GmbH did not create this package and cannot guarantee correct
functionality of this package or compatibility with any version of the
Typst compiler or app.


\section{Package List LaTeX/slashion.tex}
\title{typst.app/universe/package/slashion}

\phantomsection\label{banner}
\section{slashion}\label{slashion}

{ 0.1.1 }

Fractions with slash.

\phantomsection\label{readme}
You might not like the inline fraction displayed in a vertical layout.
Just use \textbf{Slashion} to convert it to a slash fraction.

\begin{Shaded}
\begin{Highlighting}[]
\NormalTok{\#import "@preview/slashion:0.1.1": slash{-}frac}
\NormalTok{\#show math.equation.where(block: false): slash{-}frac}
\end{Highlighting}
\end{Shaded}

You may also use it solely

\begin{Shaded}
\begin{Highlighting}[]
\NormalTok{\#import "@preview/slashion:0.1.1": slash{-}frac as sfrac}
\NormalTok{$sfrac(1/2)$, $sfrac(3, 4)$ or even $sfrac((5 + 6) / 7 + 8)$ are acceptable.}
\end{Highlighting}
\end{Shaded}

\subsection{Notice}\label{notice}

\begin{enumerate}
\tightlist
\item
  This function converts only the outermoest fraction.
\item
  This function has an option to turn off the auto parenthesizing
  feature: \texttt{\ slash-frac.with(parens:\ false)\ }
\end{enumerate}

\subsubsection{How to add}\label{how-to-add}

Copy this into your project and use the import as \texttt{\ slashion\ }

\begin{verbatim}
#import "@preview/slashion:0.1.1"
\end{verbatim}

\includesvg[width=0.16667in,height=0.16667in]{/assets/icons/16-copy.svg}

Check the docs for
\href{https://typst.app/docs/reference/scripting/\#packages}{more
information on how to import packages} .

\subsubsection{About}\label{about}

\begin{description}
\tightlist
\item[Author :]
sjfhsjfh
\item[License:]
MIT
\item[Current version:]
0.1.1
\item[Last updated:]
November 13, 2024
\item[First released:]
November 12, 2024
\item[Archive size:]
2.23 kB
\href{https://packages.typst.org/preview/slashion-0.1.1.tar.gz}{\pandocbounded{\includesvg[keepaspectratio]{/assets/icons/16-download.svg}}}
\item[Repository:]
\href{https://github.com/sjfhsjfh/slashion}{GitHub}
\item[Categor y :]
\begin{itemize}
\tightlist
\item[]
\item
  \pandocbounded{\includesvg[keepaspectratio]{/assets/icons/16-layout.svg}}
  \href{https://typst.app/universe/search/?category=layout}{Layout}
\end{itemize}
\end{description}

\subsubsection{Where to report issues?}\label{where-to-report-issues}

This package is a project of sjfhsjfh . Report issues on
\href{https://github.com/sjfhsjfh/slashion}{their repository} . You can
also try to ask for help with this package on the
\href{https://forum.typst.app}{Forum} .

Please report this package to the Typst team using the
\href{https://typst.app/contact}{contact form} if you believe it is a
safety hazard or infringes upon your rights.

\phantomsection\label{versions}
\subsubsection{Version history}\label{version-history}

\begin{longtable}[]{@{}ll@{}}
\toprule\noalign{}
Version & Release Date \\
\midrule\noalign{}
\endhead
\bottomrule\noalign{}
\endlastfoot
0.1.1 & November 13, 2024 \\
\href{https://typst.app/universe/package/slashion/0.1.0/}{0.1.0} &
November 12, 2024 \\
\end{longtable}

Typst GmbH did not create this package and cannot guarantee correct
functionality of this package or compatibility with any version of the
Typst compiler or app.


\section{Package List LaTeX/derive-it.tex}
\title{typst.app/universe/package/derive-it}

\phantomsection\label{banner}
\section{derive-it}\label{derive-it}

{ 0.1.1 }

Simple functions for creating fitch-style natural deduction proofs and
derivations.

\phantomsection\label{readme}
A Typst package to create Fitch-style natural deductions.

\pandocbounded{\includegraphics[keepaspectratio]{https://github.com/typst/packages/raw/main/packages/preview/derive-it/0.1.1/examples/example.png}}

\subsection{Usage}\label{usage}

This package provides two functions:

\texttt{\ ded-nat\ } is a function that expects 2 parameters:

\begin{itemize}
\tightlist
\item
  \texttt{\ stcolor\ } : the stroke color of the indentation guides. The
  default is \texttt{\ black\ } .
\item
  \texttt{\ arr\ } : an array with the shape, it can be provided in two
  shapes.

  \begin{itemize}
  \tightlist
  \item
    4 items: (dependency: text content, indentation: integer starting
    from 0, formula: text content, rule: text content).
  \item
    3 items: the same as above, but without the dependency.
  \end{itemize}
\end{itemize}

\texttt{\ ded-nat-boxed\ } is a function that expects 4 parameters, and
returns the deduction in a \texttt{\ box\ } :

\begin{itemize}
\tightlist
\item
  \texttt{\ stcolor\ } : the stroke color of the indentation guides. The
  default is \texttt{\ black\ } .
\item
  \texttt{\ premises-and-conclusion\ } : bool, whether to automatically
  insert or not the premises and conclusion of the derivation above the
  lines. The default is \texttt{\ true\ } .
\item
  \texttt{\ premise-rule-text\ } : text content, used for finding the
  premises’ formulas when \texttt{\ premises-and-conclusion\ } is set
  to \texttt{\ true\ } . The default is \texttt{\ "PR"\ } .
\item
  \texttt{\ arr\ } : an array with the shape, it can be provided in two
  shapes.

  \begin{itemize}
  \tightlist
  \item
    4 items: (dependency: text content, indentation: integer starting
    from 0, formula: text content, rule: text content).
  \item
    3 items: the same as above, but without the dependency.
  \end{itemize}
\end{itemize}

\subsubsection{Example}\label{example}

\begin{Shaded}
\begin{Highlighting}[]
\NormalTok{\#import "@preview/derive{-}it:0.1.1": *}

\NormalTok{\#ded{-}nat(stcolor: black, arr:(}
\NormalTok{    ("1", 0, $forall x (P x) and forall x (Q x)$, "PR"),}
\NormalTok{    ("2", 0, $forall x (P x {-}\textgreater{} R x)$, "PR"),}
  
\NormalTok{    ("1", 0, $forall x (P x)$, "S 1"),}
\NormalTok{    ("1", 0, $P a$, "IU 3"),}
\NormalTok{    ("2", 0, $P a {-}\textgreater{} R a$, "IU 2"),}
\NormalTok{    ("1,2", 0, $R a$, "MP 4, 5"),}
  
\NormalTok{    ("1,2", 0, $forall x (R x)$, "GU 6"),}
\NormalTok{))}

\NormalTok{\#ded{-}nat{-}boxed(stcolor: black, premises{-}and{-}conclusion: false, arr: (}
\NormalTok{  ("1", 0, $forall x (S x b) and not forall y (P y {-}\textgreater{} Q b y)$, "PR"),}
\NormalTok{  ("2", 0, $forall x forall y (Q x y {-}\textgreater{} not Q y x)$, "PR"),}
\NormalTok{    ("3", 1, $not forall x (not P x) {-}\textgreater{} forall y (S y b {-}\textgreater{} Q b y)$, "Sup. RAA"),}
\NormalTok{    ("1", 1, $not forall y (P y {-}\textgreater{} Q b y)$, "S 1"),}
\NormalTok{    ("1", 1, $exists y not (P y {-}\textgreater{} Q b y)$, "EMC 4"),}
\NormalTok{      ("6", 2, $not (P a {-}\textgreater{} Q b a)$, "Sup. IE 5"),}
\NormalTok{        ("7", 3, $not (P a and not Q b a)$, "Sup. RAA"),}
\NormalTok{        ("7", 3, $not P a or not not Q  b a$, "DM 7"),}
\NormalTok{          ("9", 4, $not P a$, "Sup. PC"),}
\NormalTok{          ("9", 4, $not P a or Q b a$, "Disy. 9"),}
\NormalTok{        ("", 3, $not P a {-}\textgreater{} (not P a or Q b a)$, "PC 9{-}10"),}
\NormalTok{          ("12", 4, $not  not Q b a$, "Sup. PC"),}
\NormalTok{          ("12", 4, $Q b a$, "DN 12"),}
\NormalTok{          ("12", 4, $not P a or Q b a$, "Disy. 13"),}
\NormalTok{        ("", 3, $not not Q b a {-}\textgreater{} (not P a or Q b a)$, "PC 12{-}14"),}
\NormalTok{        ("7", 3, $not P a or Q b a$, "Dil. 8,11,15"),}
\NormalTok{        ("7", 3, $P a {-}\textgreater{} Q b a$, "IM 16"),}
\NormalTok{        ("6,7", 3, $(P a {-}\textgreater{} Q b a) and not (P a {-}\textgreater{} Q b a)$, "Conj. 6, 17"),}
\NormalTok{      ("6", 2, $P a and not Q b a$, "RAA 7{-}18"),}
\NormalTok{      ("6", 2, $P a$, "S 19"),}
\NormalTok{      ("6", 2, $exists x (P x)$, "GE 20"),}
\NormalTok{      ("6", 2, $not forall x (not P x)$, "EMC 21"),}
\NormalTok{      ("3,6", 2, $forall y (S y b {-}\textgreater{} Q b y)$, "MP 3, 22"),}
\NormalTok{      ("3,6", 2, $S a b {-}\textgreater{} Q b a$, "IU 23"),}
\NormalTok{      ("1", 2, $forall x (S x b)$, "S 1"),}
\NormalTok{      ("1", 2, $S a b$, "IU 25"),}
\NormalTok{      ("1,3,6", 2, $Q b a$, "MP 24, 25"),}
\NormalTok{      ("6", 2, $not Q b a$, "S 19"),}
\NormalTok{      ("1,3,6", 2, $Q b a or not exists y not (P y {-}\textgreater{} Q b y)$, "Disy. 27"),}
\NormalTok{      ("1,3,6", 2, $not exists y not (P y {-}\textgreater{} Q b y)$, "MTP 28, 29"),}
\NormalTok{    ("1,3", 1, $not exists y not (P y {-}\textgreater{} Q b y)$, "IE 5, 6, 30"),}
\NormalTok{    ("1,3", 1, $not exists y not (P y {-}\textgreater{} Q b y) and exists y not (P y {-}\textgreater{} Q b y)$, "Conj. 5, 31"),}

\NormalTok{  ("1", 0, $not (not forall x (not P x) {-}\textgreater{} forall y ( S y b {-}\textgreater{} Q b y))$, "RAA 3{-}32"),}
\NormalTok{))}
\end{Highlighting}
\end{Shaded}

In order to compile locally \texttt{\ examples/example.typ\ } the
command is:

\begin{Shaded}
\begin{Highlighting}[]
\ExtensionTok{typst}\NormalTok{ compile examples/example.typ }\AttributeTok{{-}root}\NormalTok{ .}
\end{Highlighting}
\end{Shaded}

\subsubsection{How to add}\label{how-to-add}

Copy this into your project and use the import as \texttt{\ derive-it\ }

\begin{verbatim}
#import "@preview/derive-it:0.1.1"
\end{verbatim}

\includesvg[width=0.16667in,height=0.16667in]{/assets/icons/16-copy.svg}

Check the docs for
\href{https://typst.app/docs/reference/scripting/\#packages}{more
information on how to import packages} .

\subsubsection{About}\label{about}

\begin{description}
\tightlist
\item[Author :]
\href{https://github.com/0rphee}{0rphee}
\item[License:]
MIT
\item[Current version:]
0.1.1
\item[Last updated:]
November 14, 2024
\item[First released:]
November 12, 2024
\item[Archive size:]
3.31 kB
\href{https://packages.typst.org/preview/derive-it-0.1.1.tar.gz}{\pandocbounded{\includesvg[keepaspectratio]{/assets/icons/16-download.svg}}}
\item[Repository:]
\href{https://github.com/0rphee/derive-it}{GitHub}
\item[Discipline s :]
\begin{itemize}
\tightlist
\item[]
\item
  \href{https://typst.app/universe/search/?discipline=mathematics}{Mathematics}
\item
  \href{https://typst.app/universe/search/?discipline=philosophy}{Philosophy}
\end{itemize}
\item[Categor ies :]
\begin{itemize}
\tightlist
\item[]
\item
  \pandocbounded{\includesvg[keepaspectratio]{/assets/icons/16-layout.svg}}
  \href{https://typst.app/universe/search/?category=layout}{Layout}
\item
  \pandocbounded{\includesvg[keepaspectratio]{/assets/icons/16-chart.svg}}
  \href{https://typst.app/universe/search/?category=visualization}{Visualization}
\end{itemize}
\end{description}

\subsubsection{Where to report issues?}\label{where-to-report-issues}

This package is a project of 0rphee . Report issues on
\href{https://github.com/0rphee/derive-it}{their repository} . You can
also try to ask for help with this package on the
\href{https://forum.typst.app}{Forum} .

Please report this package to the Typst team using the
\href{https://typst.app/contact}{contact form} if you believe it is a
safety hazard or infringes upon your rights.

\phantomsection\label{versions}
\subsubsection{Version history}\label{version-history}

\begin{longtable}[]{@{}ll@{}}
\toprule\noalign{}
Version & Release Date \\
\midrule\noalign{}
\endhead
\bottomrule\noalign{}
\endlastfoot
0.1.1 & November 14, 2024 \\
\href{https://typst.app/universe/package/derive-it/0.1.0/}{0.1.0} &
November 12, 2024 \\
\end{longtable}

Typst GmbH did not create this package and cannot guarantee correct
functionality of this package or compatibility with any version of the
Typst compiler or app.


\section{Package List LaTeX/mannot.tex}
\title{typst.app/universe/package/mannot}

\phantomsection\label{banner}
\section{mannot}\label{mannot}

{ 0.1.0 }

A package for highlighting and annotating in math blocks.

\phantomsection\label{readme}
A package for highlighting and annotating in math blocks in
\href{https://typst.app/}{Typst} .

A full documentation is
\href{https://github.com/typst/packages/raw/main/packages/preview/mannot/0.1.0/docs/doc.pdf}{here}
.

\subsection{Example}\label{example}

\begin{Shaded}
\begin{Highlighting}[]
\NormalTok{$}
\NormalTok{mark(1, tag: \#\textless{}num\textgreater{}) / mark(x + 1, tag: \#\textless{}den\textgreater{}, color: \#blue)}
\NormalTok{+ mark(2, tag: \#\textless{}quo\textgreater{}, color: \#red)}

\NormalTok{\#annot(\textless{}num\textgreater{}, pos: top)[Numerator]}
\NormalTok{\#annot(\textless{}den\textgreater{})[Denominator]}
\NormalTok{\#annot(\textless{}quo\textgreater{}, pos: right, yshift: 1em)[Quotient]}
\NormalTok{$}
\end{Highlighting}
\end{Shaded}

\pandocbounded{\includesvg[keepaspectratio]{https://github.com/typst/packages/raw/main/packages/preview/mannot/0.1.0/examples/showcase.svg}}

\subsection{Usage}\label{usage}

Import and initialize the package \texttt{\ mannot\ } on the top of your
document.

\begin{Shaded}
\begin{Highlighting}[]
\NormalTok{\#import "@preview/mannot:0.1.0": *}
\NormalTok{\#show: mannot{-}init}
\end{Highlighting}
\end{Shaded}

To highlight a part of a math block, use the \texttt{\ mark\ } function:

\begin{Shaded}
\begin{Highlighting}[]
\NormalTok{$}
\NormalTok{mark(x)}
\NormalTok{$}
\end{Highlighting}
\end{Shaded}

\pandocbounded{\includesvg[keepaspectratio]{https://github.com/typst/packages/raw/main/packages/preview/mannot/0.1.0/examples/usage1.svg}}

You can also specify a color for the highlighted part:

\begin{Shaded}
\begin{Highlighting}[]
\NormalTok{$ // Need \# before color names.}
\NormalTok{mark(3, color: \#red) mark(x, color: \#blue)}
\NormalTok{+ mark(integral x dif x, color: \#green)}
\NormalTok{$}
\end{Highlighting}
\end{Shaded}

\pandocbounded{\includesvg[keepaspectratio]{https://github.com/typst/packages/raw/main/packages/preview/mannot/0.1.0/examples/usage2.svg}}

To add an annotation to a highlighted part, use the \texttt{\ annot\ }
function. You need to specify the tag of the marked content:

\begin{Shaded}
\begin{Highlighting}[]
\NormalTok{$}
\NormalTok{mark(x, tag: \#\textless{}x\textgreater{})  // Need \# before tags.}
\NormalTok{\#annot(\textless{}x\textgreater{})[Annotation]}
\NormalTok{$}
\end{Highlighting}
\end{Shaded}

\pandocbounded{\includesvg[keepaspectratio]{https://github.com/typst/packages/raw/main/packages/preview/mannot/0.1.0/examples/usage3.svg}}

You can customize the position of the annotation and the vertical
distance from the marked content:

\begin{Shaded}
\begin{Highlighting}[]
\NormalTok{$}
\NormalTok{mark(integral x dif x, tag: \#\textless{}i\textgreater{}, color: \#green)}
\NormalTok{+ mark(3, tag: \#\textless{}3\textgreater{}, color: \#red) mark(x, tag: \#\textless{}x\textgreater{}, color: \#blue)}

\NormalTok{\#annot(\textless{}i\textgreater{}, pos: left)[Set pos to left.]}
\NormalTok{\#annot(\textless{}i\textgreater{}, pos: top + left)[Top left.]}
\NormalTok{\#annot(\textless{}3\textgreater{}, pos: top, yshift: 1.2em)[Use yshift.]}
\NormalTok{\#annot(\textless{}x\textgreater{}, pos: right, yshift: 1.2em)[Auto arrow.]}
\NormalTok{$}
\end{Highlighting}
\end{Shaded}

\pandocbounded{\includesvg[keepaspectratio]{https://github.com/typst/packages/raw/main/packages/preview/mannot/0.1.0/examples/usage4.svg}}

For convenience, you can define custom mark functions:

\begin{Shaded}
\begin{Highlighting}[]
\NormalTok{\#let rmark = mark.with(color: red)}
\NormalTok{\#let gmark = mark.with(color: green)}
\NormalTok{\#let bmark = mark.with(color: blue)}

\NormalTok{$}
\NormalTok{integral\_rmark(0, tag: \#\textless{}i0\textgreater{})\^{}bmark(1, tag: \#\textless{}i1\textgreater{})}
\NormalTok{mark(x\^{}2 + 1, tag: \#\textless{}i2\textgreater{}) dif gmark(x, tag: \#\textless{}i3\textgreater{})}

\NormalTok{\#annot(\textless{}i0\textgreater{})[Begin]}
\NormalTok{\#annot(\textless{}i1\textgreater{}, pos: top)[End]}
\NormalTok{\#annot(\textless{}i2\textgreater{}, pos: top + right)[Integrand]}
\NormalTok{\#annot(\textless{}i3\textgreater{}, pos: right, yshift: .6em)[Variable]}
\NormalTok{$}
\end{Highlighting}
\end{Shaded}

\pandocbounded{\includesvg[keepaspectratio]{https://github.com/typst/packages/raw/main/packages/preview/mannot/0.1.0/examples/usage5.svg}}

\subsubsection{How to add}\label{how-to-add}

Copy this into your project and use the import as \texttt{\ mannot\ }

\begin{verbatim}
#import "@preview/mannot:0.1.0"
\end{verbatim}

\includesvg[width=0.16667in,height=0.16667in]{/assets/icons/16-copy.svg}

Check the docs for
\href{https://typst.app/docs/reference/scripting/\#packages}{more
information on how to import packages} .

\subsubsection{About}\label{about}

\begin{description}
\tightlist
\item[Author :]
ryuryu-ymj
\item[License:]
MIT
\item[Current version:]
0.1.0
\item[Last updated:]
October 21, 2024
\item[First released:]
October 21, 2024
\item[Minimum Typst version:]
0.12.0
\item[Archive size:]
6.84 kB
\href{https://packages.typst.org/preview/mannot-0.1.0.tar.gz}{\pandocbounded{\includesvg[keepaspectratio]{/assets/icons/16-download.svg}}}
\item[Repository:]
\href{https://github.com/ryuryu-ymj/mannot}{GitHub}
\item[Categor ies :]
\begin{itemize}
\tightlist
\item[]
\item
  \pandocbounded{\includesvg[keepaspectratio]{/assets/icons/16-chart.svg}}
  \href{https://typst.app/universe/search/?category=visualization}{Visualization}
\item
  \pandocbounded{\includesvg[keepaspectratio]{/assets/icons/16-layout.svg}}
  \href{https://typst.app/universe/search/?category=layout}{Layout}
\end{itemize}
\end{description}

\subsubsection{Where to report issues?}\label{where-to-report-issues}

This package is a project of ryuryu-ymj . Report issues on
\href{https://github.com/ryuryu-ymj/mannot}{their repository} . You can
also try to ask for help with this package on the
\href{https://forum.typst.app}{Forum} .

Please report this package to the Typst team using the
\href{https://typst.app/contact}{contact form} if you believe it is a
safety hazard or infringes upon your rights.

\phantomsection\label{versions}
\subsubsection{Version history}\label{version-history}

\begin{longtable}[]{@{}ll@{}}
\toprule\noalign{}
Version & Release Date \\
\midrule\noalign{}
\endhead
\bottomrule\noalign{}
\endlastfoot
0.1.0 & October 21, 2024 \\
\end{longtable}

Typst GmbH did not create this package and cannot guarantee correct
functionality of this package or compatibility with any version of the
Typst compiler or app.


\section{Package List LaTeX/tud-corporate-design-slides.tex}
\title{typst.app/universe/package/tud-corporate-design-slides}

\phantomsection\label{banner}
\phantomsection\label{template-thumbnail}
\pandocbounded{\includegraphics[keepaspectratio]{https://packages.typst.org/preview/thumbnails/tud-corporate-design-slides-0.1.0-small.webp}}

\section{tud-corporate-design-slides}\label{tud-corporate-design-slides}

{ 0.1.0 }

Presentation template for TU Dresden (Technische Universität Dresden).

\href{/app?template=tud-corporate-design-slides&version=0.1.0}{Create
project in app}

\phantomsection\label{readme}
This template can be used to create presentations in
\href{https://github.com/typst/typst}{Typst} with the corporate design
of \href{https://www.tu-dresden.de/}{TU Dresden} .

\subsection{Usage}\label{usage}

Create a new typst project based on this template locally.

\begin{Shaded}
\begin{Highlighting}[]
\ExtensionTok{typst}\NormalTok{ init @preview/tud{-}corporate{-}design{-}slides}
\BuiltInTok{cd}\NormalTok{ tud{-}corporate{-}design{-}slides}
\end{Highlighting}
\end{Shaded}

Or create a project on the typst web app based on this template.

\subsubsection{Font setup}\label{font-setup}

The fonts \texttt{\ Open\ Sans\ } needs to be installed on your system:

You can download the fonts from the
\href{https://tu-dresden.de/intern/services-und-hilfe/ressourcen/dateien/kommunizieren_und_publizieren/corporate-design/cd-elemente/schrift-tud-open-sans}{TU
Dresden website} .

Once you download the fonts, make sure to install and activate them on
your system.

\subsubsection{Compile (and watch) your typst
file}\label{compile-and-watch-your-typst-file}

\begin{Shaded}
\begin{Highlighting}[]
\ExtensionTok{typst}\NormalTok{ w main.typ}
\end{Highlighting}
\end{Shaded}

This will watch your file and recompile it to a pdf when the file is
saved. For writing, you can use
\href{https://code.visualstudio.com/}{Vscode} with these extensions:
\href{https://marketplace.visualstudio.com/items?itemName=nvarner.typst-lsp}{Typst
LSP} and
\href{https://marketplace.visualstudio.com/items?itemName=mgt19937.typst-preview}{Typst
Preview} . Or use the \href{https://typst.app/}{typst web app} (here you
need to upload the fonts).

\subsection{Todos}\label{todos}

\begin{itemize}
\tightlist
\item
  {[} {]} Add more slide layouts (e.g. 2-column layout)
\item
  {[} {]} Port to \href{https://github.com/touying-typ/touying}{touying}
\end{itemize}

\href{/app?template=tud-corporate-design-slides&version=0.1.0}{Create
project in app}

\subsubsection{How to use}\label{how-to-use}

Click the button above to create a new project using this template in
the Typst app.

You can also use the Typst CLI to start a new project on your computer
using this command:

\begin{verbatim}
typst init @preview/tud-corporate-design-slides:0.1.0
\end{verbatim}

\includesvg[width=0.16667in,height=0.16667in]{/assets/icons/16-copy.svg}

\subsubsection{About}\label{about}

\begin{description}
\tightlist
\item[Author :]
\href{https://github.com/jakoblistabarth}{Jakob Listabarth}
\item[License:]
MIT
\item[Current version:]
0.1.0
\item[Last updated:]
October 10, 2024
\item[First released:]
October 10, 2024
\item[Archive size:]
9.54 kB
\href{https://packages.typst.org/preview/tud-corporate-design-slides-0.1.0.tar.gz}{\pandocbounded{\includesvg[keepaspectratio]{/assets/icons/16-download.svg}}}
\item[Repository:]
\href{https://github.com/jakoblistabarth/tud-corporate-design-slides-typst}{GitHub}
\item[Categor y :]
\begin{itemize}
\tightlist
\item[]
\item
  \pandocbounded{\includesvg[keepaspectratio]{/assets/icons/16-presentation.svg}}
  \href{https://typst.app/universe/search/?category=presentation}{Presentation}
\end{itemize}
\end{description}

\subsubsection{Where to report issues?}\label{where-to-report-issues}

This template is a project of Jakob Listabarth . Report issues on
\href{https://github.com/jakoblistabarth/tud-corporate-design-slides-typst}{their
repository} . You can also try to ask for help with this template on the
\href{https://forum.typst.app}{Forum} .

Please report this template to the Typst team using the
\href{https://typst.app/contact}{contact form} if you believe it is a
safety hazard or infringes upon your rights.

\phantomsection\label{versions}
\subsubsection{Version history}\label{version-history}

\begin{longtable}[]{@{}ll@{}}
\toprule\noalign{}
Version & Release Date \\
\midrule\noalign{}
\endhead
\bottomrule\noalign{}
\endlastfoot
0.1.0 & October 10, 2024 \\
\end{longtable}

Typst GmbH did not create this template and cannot guarantee correct
functionality of this template or compatibility with any version of the
Typst compiler or app.


\section{Package List LaTeX/m-jaxon.tex}
\title{typst.app/universe/package/m-jaxon}

\phantomsection\label{banner}
\section{m-jaxon}\label{m-jaxon}

{ 0.1.1 }

Render LaTeX equation in typst using MathJax.

\phantomsection\label{readme}
Render LaTeX equation in typst using MathJax.

\textbf{Note:} This package is made for fun and to demonstrate the
capability of typst plugins. And it is \textbf{slow} . To actually
convert LaTeX equations to typst ones, you should use \textbf{pandoc} or
\textbf{texmath} .

\pandocbounded{\includesvg[keepaspectratio]{https://github.com/typst/packages/raw/main/packages/preview/m-jaxon/0.1.1/mj.svg}}

\begin{Shaded}
\begin{Highlighting}[]
\NormalTok{\#import "./typst{-}package/lib.typ" as m{-}jaxon}
\NormalTok{// Uncomment the following line to use the m{-}jaxon from the official package registry}
\NormalTok{// \#import "@preview/m{-}jaxon:0.1.1"}

\NormalTok{= M{-}Jaxon}

\NormalTok{Typst, now with *MathJax*.}

\NormalTok{The equation of mass{-}energy equivalence is often written as $E=m c\^{}2$ in modern physics.}

\NormalTok{But we can also write it using M{-}Jaxon as: \#m{-}jaxon.render("E = mc\^{}2", inline: true)}
\end{Highlighting}
\end{Shaded}

\subsection{Limitations}\label{limitations}

\begin{itemize}
\tightlist
\item
  The baseline of the inline equation still looks a bit off.
\end{itemize}

\subsection{Documentation}\label{documentation}

\subsubsection{\texorpdfstring{\texttt{\ render\ }}{ render }}\label{render}

Render a LaTeX equation string to an svg image. Depending on the
\texttt{\ inline\ } argument, the image will be rendered as an inline
image or a block image.

\paragraph{Arguments}\label{arguments}

\begin{itemize}
\tightlist
\item
  \texttt{\ src\ } : \texttt{\ str\ } or \texttt{\ raw\ } block - The
  LaTeX equation string
\item
  \texttt{\ inline\ } : \texttt{\ bool\ } - Whether to render the image
  as an inline image or a block image
\end{itemize}

\paragraph{Returns}\label{returns}

The image, of type \texttt{\ content\ }

\subsubsection{How to add}\label{how-to-add}

Copy this into your project and use the import as \texttt{\ m-jaxon\ }

\begin{verbatim}
#import "@preview/m-jaxon:0.1.1"
\end{verbatim}

\includesvg[width=0.16667in,height=0.16667in]{/assets/icons/16-copy.svg}

Check the docs for
\href{https://typst.app/docs/reference/scripting/\#packages}{more
information on how to import packages} .

\subsubsection{About}\label{about}

\begin{description}
\tightlist
\item[Author :]
Wenzhuo Liu
\item[License:]
MIT
\item[Current version:]
0.1.1
\item[Last updated:]
January 17, 2024
\item[First released:]
December 14, 2023
\item[Archive size:]
633 kB
\href{https://packages.typst.org/preview/m-jaxon-0.1.1.tar.gz}{\pandocbounded{\includesvg[keepaspectratio]{/assets/icons/16-download.svg}}}
\item[Repository:]
\href{https://github.com/Enter-tainer/m-jaxon}{GitHub}
\end{description}

\subsubsection{Where to report issues?}\label{where-to-report-issues}

This package is a project of Wenzhuo Liu . Report issues on
\href{https://github.com/Enter-tainer/m-jaxon}{their repository} . You
can also try to ask for help with this package on the
\href{https://forum.typst.app}{Forum} .

Please report this package to the Typst team using the
\href{https://typst.app/contact}{contact form} if you believe it is a
safety hazard or infringes upon your rights.

\phantomsection\label{versions}
\subsubsection{Version history}\label{version-history}

\begin{longtable}[]{@{}ll@{}}
\toprule\noalign{}
Version & Release Date \\
\midrule\noalign{}
\endhead
\bottomrule\noalign{}
\endlastfoot
0.1.1 & January 17, 2024 \\
\href{https://typst.app/universe/package/m-jaxon/0.1.0/}{0.1.0} &
December 14, 2023 \\
\end{longtable}

Typst GmbH did not create this package and cannot guarantee correct
functionality of this package or compatibility with any version of the
Typst compiler or app.


\section{Package List LaTeX/umbra.tex}
\title{typst.app/universe/package/umbra}

\phantomsection\label{banner}
\section{umbra}\label{umbra}

{ 0.1.0 }

Basic shadows for Typst

{ } Featured Package

\phantomsection\label{readme}
Umbra is a library for drawing basic gradient shadows in
\href{https://typst.app/}{typst} . It currently provides only one
function for drawing a shadow along one edge of a path.

\subsection{Examples}\label{examples}

\subsubsection{Basic Shadow}\label{basic-shadow}

\href{https://github.com/typst/packages/raw/main/packages/preview/umbra/0.1.0/gallery/basic.typ}{\includegraphics[width=\linewidth,height=2.60417in,keepaspectratio]{https://github.com/typst/packages/raw/main/packages/preview/umbra/0.1.0/gallery/basic.png}}

\subsubsection{Neumorphism}\label{neumorphism}

\href{https://github.com/typst/packages/raw/main/packages/preview/umbra/0.1.0/gallery/neumorphism.typ}{\includegraphics[width=\linewidth,height=2.60417in,keepaspectratio]{https://github.com/typst/packages/raw/main/packages/preview/umbra/0.1.0/gallery/neumorphism.png}}

\subsubsection{Torn Paper}\label{torn-paper}

\href{https://github.com/typst/packages/raw/main/packages/preview/umbra/0.1.0/gallery/torn-paper.typ}{\includegraphics[width=\linewidth,height=2.60417in,keepaspectratio]{https://github.com/typst/packages/raw/main/packages/preview/umbra/0.1.0/gallery/torn-paper.png}}

\emph{Click on the example image to jump to the code.}

\subsection{Usage}\label{usage}

The following code creates a very basic square shadow:

\begin{verbatim}
#import "@preview/umbra:0.1.0": shadow-path

#shadow-path((10%, 10%), (10%, 90%), (90%, 90%), (90%, 10%), closed: true)
\end{verbatim}

The function syntax is similar to the normal path syntax. The following
arguments were added:

\begin{itemize}
\tightlist
\item
  \texttt{\ shadow-radius\ } (default \texttt{\ 0.5cm\ } ): The shadow
  size in the direction normal to the edge
\item
  \texttt{\ shadow-stops\ } (default \texttt{\ (gray,\ white)\ } ): The
  colours to be used in the shadow, passed directly to
  \texttt{\ gradient\ }
\item
  \texttt{\ correction\ } (default \texttt{\ 5deg\ } ): A small
  correction factor to be added to round shadows at corners. Otherwise,
  there will be a small gap between the two shadows
\end{itemize}

\subsubsection{Vertex Order}\label{vertex-order}

The order of the vertices defines the direction of the shadow. If the
shadow is the wrong way around, just reverse the vertices.

\subsubsection{Transparency}\label{transparency}

This package is designed in such a way that it should support
transparency in the gradients (i.e. corners define shadows using a path
which approximates the arc, instead of an entire circle). However, typst
doesn’t currently support transparency in gradients. (
\href{https://github.com/typst/typst/issues/2546}{issue} ).

In addition, the aforementioned correction factor would likely cause
issues with transparent gradients.

\subsubsection{How to add}\label{how-to-add}

Copy this into your project and use the import as \texttt{\ umbra\ }

\begin{verbatim}
#import "@preview/umbra:0.1.0"
\end{verbatim}

\includesvg[width=0.16667in,height=0.16667in]{/assets/icons/16-copy.svg}

Check the docs for
\href{https://typst.app/docs/reference/scripting/\#packages}{more
information on how to import packages} .

\subsubsection{About}\label{about}

\begin{description}
\tightlist
\item[Author :]
\href{https://github.com/davystrong}{David Armstrong}
\item[License:]
MIT
\item[Current version:]
0.1.0
\item[Last updated:]
August 30, 2024
\item[First released:]
August 30, 2024
\item[Minimum Typst version:]
0.10.0
\item[Archive size:]
3.50 kB
\href{https://packages.typst.org/preview/umbra-0.1.0.tar.gz}{\pandocbounded{\includesvg[keepaspectratio]{/assets/icons/16-download.svg}}}
\item[Repository:]
\href{https://github.com/davystrong/umbra}{GitHub}
\item[Categor y :]
\begin{itemize}
\tightlist
\item[]
\item
  \pandocbounded{\includesvg[keepaspectratio]{/assets/icons/16-chart.svg}}
  \href{https://typst.app/universe/search/?category=visualization}{Visualization}
\end{itemize}
\end{description}

\subsubsection{Where to report issues?}\label{where-to-report-issues}

This package is a project of David Armstrong . Report issues on
\href{https://github.com/davystrong/umbra}{their repository} . You can
also try to ask for help with this package on the
\href{https://forum.typst.app}{Forum} .

Please report this package to the Typst team using the
\href{https://typst.app/contact}{contact form} if you believe it is a
safety hazard or infringes upon your rights.

\phantomsection\label{versions}
\subsubsection{Version history}\label{version-history}

\begin{longtable}[]{@{}ll@{}}
\toprule\noalign{}
Version & Release Date \\
\midrule\noalign{}
\endhead
\bottomrule\noalign{}
\endlastfoot
0.1.0 & August 30, 2024 \\
\end{longtable}

Typst GmbH did not create this package and cannot guarantee correct
functionality of this package or compatibility with any version of the
Typst compiler or app.


\section{Package List LaTeX/neoplot.tex}
\title{typst.app/universe/package/neoplot}

\phantomsection\label{banner}
\section{neoplot}\label{neoplot}

{ 0.0.2 }

Gnuplot in Typst

\phantomsection\label{readme}
A Typst package to use \href{http://www.gnuplot.info/}{gnuplot} in
Typst.

\begin{Shaded}
\begin{Highlighting}[]
\NormalTok{\#import "@preview/neoplot:0.0.2" as gp}
\end{Highlighting}
\end{Shaded}

Execute gnuplot commands:

\begin{Shaded}
\begin{Highlighting}[]
\NormalTok{\#gp.exec(}
\NormalTok{    kind: "command",}
\NormalTok{    \textasciigrave{}\textasciigrave{}\textasciigrave{}gnuplot}
\NormalTok{    reset;}
\NormalTok{    set samples 1000;}
\NormalTok{    plot sin(x),}
\NormalTok{         cos(x)}
\NormalTok{    \textasciigrave{}\textasciigrave{}\textasciigrave{}}
\NormalTok{)}
\end{Highlighting}
\end{Shaded}

Execute a gnuplot script:

\begin{Shaded}
\begin{Highlighting}[]
\NormalTok{\#gp.exec(}
\NormalTok{    \textasciigrave{}\textasciigrave{}\textasciigrave{}gnuplot}
\NormalTok{    reset}
\NormalTok{    \# Can add comments since it is a script}
\NormalTok{    set samples 1000}
\NormalTok{    \# Use a backslash to extend commands}
\NormalTok{    plot sin(x), \textbackslash{}}
\NormalTok{         cos(x)}
\NormalTok{    \textasciigrave{}\textasciigrave{}\textasciigrave{}}
\NormalTok{)}
\end{Highlighting}
\end{Shaded}

To read a data file:

\begin{verbatim}
# datafile.dat
# x  y
  0  0
  2  4
  4  0
\end{verbatim}

\begin{Shaded}
\begin{Highlighting}[]
\NormalTok{\#gp.exec(}
\NormalTok{    \textasciigrave{}\textasciigrave{}\textasciigrave{}gnuplot}
\NormalTok{    $data \textless{}\textless{}EOD}
\NormalTok{    0  0}
\NormalTok{    2  4}
\NormalTok{    4  0}
\NormalTok{    EOD}
\NormalTok{    plot $data with linespoints}
\NormalTok{    \textasciigrave{}\textasciigrave{}\textasciigrave{}}
\NormalTok{)}
\end{Highlighting}
\end{Shaded}

or

\begin{Shaded}
\begin{Highlighting}[]
\NormalTok{\#gp.exec(}
\NormalTok{    // Use a datablock since Typst doesn\textquotesingle{}t support WASI}
\NormalTok{    "$data \textless{}\textless{}EOD\textbackslash{}n" +}
\NormalTok{    // Load "datafile.dat" using Typst}
\NormalTok{    read("datafile.dat") +}
\NormalTok{    "EOD\textbackslash{}n" +}
\NormalTok{    "plot $data with linespoints"}
\NormalTok{)}
\end{Highlighting}
\end{Shaded}

To print \texttt{\ \$data\ } :

\begin{Shaded}
\begin{Highlighting}[]
\NormalTok{\#gp.exec("print $data")}
\end{Highlighting}
\end{Shaded}

\subsubsection{How to add}\label{how-to-add}

Copy this into your project and use the import as \texttt{\ neoplot\ }

\begin{verbatim}
#import "@preview/neoplot:0.0.2"
\end{verbatim}

\includesvg[width=0.16667in,height=0.16667in]{/assets/icons/16-copy.svg}

Check the docs for
\href{https://typst.app/docs/reference/scripting/\#packages}{more
information on how to import packages} .

\subsubsection{About}\label{about}

\begin{description}
\tightlist
\item[Author :]
\href{https://github.com/KNnut}{KNnut}
\item[License:]
BSD-3-Clause
\item[Current version:]
0.0.2
\item[Last updated:]
July 16, 2024
\item[First released:]
June 17, 2024
\item[Minimum Typst version:]
0.11.1
\item[Archive size:]
512 kB
\href{https://packages.typst.org/preview/neoplot-0.0.2.tar.gz}{\pandocbounded{\includesvg[keepaspectratio]{/assets/icons/16-download.svg}}}
\item[Repository:]
\href{https://github.com/KNnut/neoplot}{GitHub}
\item[Discipline :]
\begin{itemize}
\tightlist
\item[]
\item
  \href{https://typst.app/universe/search/?discipline=mathematics}{Mathematics}
\end{itemize}
\item[Categor ies :]
\begin{itemize}
\tightlist
\item[]
\item
  \pandocbounded{\includesvg[keepaspectratio]{/assets/icons/16-chart.svg}}
  \href{https://typst.app/universe/search/?category=visualization}{Visualization}
\item
  \pandocbounded{\includesvg[keepaspectratio]{/assets/icons/16-integration.svg}}
  \href{https://typst.app/universe/search/?category=integration}{Integration}
\end{itemize}
\end{description}

\subsubsection{Where to report issues?}\label{where-to-report-issues}

This package is a project of KNnut . Report issues on
\href{https://github.com/KNnut/neoplot}{their repository} . You can also
try to ask for help with this package on the
\href{https://forum.typst.app}{Forum} .

Please report this package to the Typst team using the
\href{https://typst.app/contact}{contact form} if you believe it is a
safety hazard or infringes upon your rights.

\phantomsection\label{versions}
\subsubsection{Version history}\label{version-history}

\begin{longtable}[]{@{}ll@{}}
\toprule\noalign{}
Version & Release Date \\
\midrule\noalign{}
\endhead
\bottomrule\noalign{}
\endlastfoot
0.0.2 & July 16, 2024 \\
\href{https://typst.app/universe/package/neoplot/0.0.1/}{0.0.1} & June
17, 2024 \\
\end{longtable}

Typst GmbH did not create this package and cannot guarantee correct
functionality of this package or compatibility with any version of the
Typst compiler or app.


\section{Package List LaTeX/km.tex}
\title{typst.app/universe/package/km}

\phantomsection\label{banner}
\section{km}\label{km}

{ 0.1.0 }

Draw simple Karnaugh maps

\phantomsection\label{readme}
Draw simple Karnaugh maps. Use with:

\begin{Shaded}
\begin{Highlighting}[]
\NormalTok{\#import "@preview/km:0.1.0": karnaugh}

\NormalTok{\#karnaugh(("C", "AB"),}
\NormalTok{  implicants: (}
\NormalTok{    (0, 1, 1, 2),}
\NormalTok{    (1, 2, 2, 1),}
\NormalTok{  ),}
\NormalTok{  (}
\NormalTok{    (0, 1, 0, 0),}
\NormalTok{    (0, 1, 1, 1),}
\NormalTok{  )}
\NormalTok{)}
\end{Highlighting}
\end{Shaded}

Samples are available in
\href{https://github.com/typst/packages/blob/main/packages/preview/km/0.1.0/sample.pdf}{\texttt{\ sample.pdf\ }}
.

\subsubsection{How to add}\label{how-to-add}

Copy this into your project and use the import as \texttt{\ km\ }

\begin{verbatim}
#import "@preview/km:0.1.0"
\end{verbatim}

\includesvg[width=0.16667in,height=0.16667in]{/assets/icons/16-copy.svg}

Check the docs for
\href{https://typst.app/docs/reference/scripting/\#packages}{more
information on how to import packages} .

\subsubsection{About}\label{about}

\begin{description}
\tightlist
\item[Author :]
\href{mailto:the@unpopular.me}{Toto}
\item[License:]
MIT
\item[Current version:]
0.1.0
\item[Last updated:]
June 18, 2024
\item[First released:]
June 18, 2024
\item[Minimum Typst version:]
0.11.0
\item[Archive size:]
2.56 kB
\href{https://packages.typst.org/preview/km-0.1.0.tar.gz}{\pandocbounded{\includesvg[keepaspectratio]{/assets/icons/16-download.svg}}}
\item[Repository:]
\href{https://git.sr.ht/~toto/karnaugh}{git.sr.ht}
\item[Discipline :]
\begin{itemize}
\tightlist
\item[]
\item
  \href{https://typst.app/universe/search/?discipline=mathematics}{Mathematics}
\end{itemize}
\item[Categor y :]
\begin{itemize}
\tightlist
\item[]
\item
  \pandocbounded{\includesvg[keepaspectratio]{/assets/icons/16-hammer.svg}}
  \href{https://typst.app/universe/search/?category=utility}{Utility}
\end{itemize}
\end{description}

\subsubsection{Where to report issues?}\label{where-to-report-issues}

This package is a project of Toto . Report issues on
\href{https://git.sr.ht/~toto/karnaugh}{their repository} . You can also
try to ask for help with this package on the
\href{https://forum.typst.app}{Forum} .

Please report this package to the Typst team using the
\href{https://typst.app/contact}{contact form} if you believe it is a
safety hazard or infringes upon your rights.

\phantomsection\label{versions}
\subsubsection{Version history}\label{version-history}

\begin{longtable}[]{@{}ll@{}}
\toprule\noalign{}
Version & Release Date \\
\midrule\noalign{}
\endhead
\bottomrule\noalign{}
\endlastfoot
0.1.0 & June 18, 2024 \\
\end{longtable}

Typst GmbH did not create this package and cannot guarantee correct
functionality of this package or compatibility with any version of the
Typst compiler or app.


\section{Package List LaTeX/bone-resume.tex}
\title{typst.app/universe/package/bone-resume}

\phantomsection\label{banner}
\phantomsection\label{template-thumbnail}
\pandocbounded{\includegraphics[keepaspectratio]{https://packages.typst.org/preview/thumbnails/bone-resume-0.3.0-small.webp}}

\section{bone-resume}\label{bone-resume}

{ 0.3.0 }

A colorful resume template for chinese.

\href{/app?template=bone-resume&version=0.3.0}{Create project in app}

\phantomsection\label{readme}
This is a Typst template.

\subsection{Usage}\label{usage}

You can use this template in the Typst web app by clicking “Start from
template� on the dashboard and searching for \texttt{\ bone-resumee\ }
.

Alternatively, you can use the CLI to kick this project off using the
command

\begin{verbatim}
typst init @preview/bone-resume
\end{verbatim}

Typst will create a new directory with all the files needed to get you
started.

\subsubsection{Fonts}\label{fonts}

\begin{itemize}
\tightlist
\item
  \href{https://github.com/adobe-fonts/source-han-sans}{Source Han Sans}
\item
  \href{https://github.com/lxgw/LxgwWenkaiGB}{LXGW WenKai GB}
\item
  \href{https://www.nerdfonts.com/}{Hack Nerd Font}
\end{itemize}

\subsection{Configuration}\label{configuration}

This template exports the \texttt{\ resume-init\ } function with the
following named arguments:

\begin{itemize}
\tightlist
\item
  \texttt{\ authors\ } : Your name.
\item
  \texttt{\ title(optional)\ } : The resume’s title as content.
\item
  \texttt{\ footer(optional)\ } : The resume’s footer as content.
\end{itemize}

The function also accepts a single, positional argument for the body of
the paper.

The template will initialize your package with a sample call to the
\texttt{\ bone-resume\ } function in a show rule. If you want to change
an existing project to use this template, you can add a show rule like
this at the top of your file:

\begin{Shaded}
\begin{Highlighting}[]
\NormalTok{\#import "@preview/bone{-}resume:0.1.0": bone{-}resume}

\NormalTok{\#show: bone{-}resume.with(}
\NormalTok{  author: "六个骨头"}
\NormalTok{)}

\NormalTok{= 个人介绍}
\NormalTok{A Student.}

\NormalTok{// Your content goes below.}
\end{Highlighting}
\end{Shaded}

\href{/app?template=bone-resume&version=0.3.0}{Create project in app}

\subsubsection{How to use}\label{how-to-use}

Click the button above to create a new project using this template in
the Typst app.

You can also use the Typst CLI to start a new project on your computer
using this command:

\begin{verbatim}
typst init @preview/bone-resume:0.3.0
\end{verbatim}

\includesvg[width=0.16667in,height=0.16667in]{/assets/icons/16-copy.svg}

\subsubsection{About}\label{about}

\begin{description}
\tightlist
\item[Author :]
zrr1999
\item[License:]
Apache-2.0
\item[Current version:]
0.3.0
\item[Last updated:]
September 2, 2024
\item[First released:]
June 3, 2024
\item[Archive size:]
7.42 kB
\href{https://packages.typst.org/preview/bone-resume-0.3.0.tar.gz}{\pandocbounded{\includesvg[keepaspectratio]{/assets/icons/16-download.svg}}}
\item[Categor y :]
\begin{itemize}
\tightlist
\item[]
\item
  \pandocbounded{\includesvg[keepaspectratio]{/assets/icons/16-user.svg}}
  \href{https://typst.app/universe/search/?category=cv}{CV}
\end{itemize}
\end{description}

\subsubsection{Where to report issues?}\label{where-to-report-issues}

This template is a project of zrr1999 . You can also try to ask for help
with this template on the \href{https://forum.typst.app}{Forum} .

Please report this template to the Typst team using the
\href{https://typst.app/contact}{contact form} if you believe it is a
safety hazard or infringes upon your rights.

\phantomsection\label{versions}
\subsubsection{Version history}\label{version-history}

\begin{longtable}[]{@{}ll@{}}
\toprule\noalign{}
Version & Release Date \\
\midrule\noalign{}
\endhead
\bottomrule\noalign{}
\endlastfoot
0.3.0 & September 2, 2024 \\
\href{https://typst.app/universe/package/bone-resume/0.2.0/}{0.2.0} &
July 4, 2024 \\
\href{https://typst.app/universe/package/bone-resume/0.1.0/}{0.1.0} &
June 3, 2024 \\
\end{longtable}

Typst GmbH did not create this template and cannot guarantee correct
functionality of this template or compatibility with any version of the
Typst compiler or app.


