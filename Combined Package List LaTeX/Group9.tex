\title{typst.app/universe/package/pesha}

\phantomsection\label{banner}
\phantomsection\label{template-thumbnail}
\pandocbounded{\includegraphics[keepaspectratio]{https://packages.typst.org/preview/thumbnails/pesha-0.4.0-small.webp}}

\section{pesha}\label{pesha}

{ 0.4.0 }

A clean and minimal template for your résumé or CV

\href{/app?template=pesha&version=0.4.0}{Create project in app}

\phantomsection\label{readme}
\begin{quote}
Pesha (Urdu: پیشÛ?) is the Urdu term for occupation/profession. It is
pronounced as pay-sha.
\end{quote}

A clean and minimal template for your CV or résumé.

This template is inspired by Matthew Butterick’s excellent
\href{https://practicaltypography.com/}{\emph{Practical Typography}}
book.

See
\href{https://github.com/talal/pesha/blob/main/example.pdf}{example.pdf}
or
\href{https://github.com/talal/pesha/blob/main/example-profile-picture.pdf}{example-profile-picture.pdf}
file to see how it looks.

\subsection{Usage}\label{usage}

You can use this template in the Typst web app by clicking “Start from
template� on the dashboard and searching for \texttt{\ pesha\ } .

Alternatively, you can use the CLI to kick this project off using the
command

\begin{Shaded}
\begin{Highlighting}[]
\ExtensionTok{typst}\NormalTok{ init @preview/pesha}
\end{Highlighting}
\end{Shaded}

Typst will create a new directory with all the files needed to get you
started.

\subsection{Configuration}\label{configuration}

This template exports the \texttt{\ pesha\ } function with the following
named arguments:

\begin{longtable}[]{@{}lll@{}}
\toprule\noalign{}
Argument & Type & Description \\
\midrule\noalign{}
\endhead
\bottomrule\noalign{}
\endlastfoot
\texttt{\ name\ } &
\href{https://typst.app/docs/reference/foundations/str/}{string} & A
string to specify the author’s name. \\
\texttt{\ address\ } &
\href{https://typst.app/docs/reference/foundations/str/}{string} & A
string to specify the author’s address. \\
\texttt{\ contacts\ } &
\href{https://typst.app/docs/reference/foundations/array/}{array} & An
array of content to specify your contact information. E.g., phone
number, email, LinkedIn, etc. \\
\texttt{\ profile-picture\ } &
\href{https://typst.app/docs/reference/foundations/content/}{content} &
The result of a call to the
\href{https://typst.app/docs/reference/visualize/image/}{image function}
or \texttt{\ none\ } . For best result, make sure that your image has an
1:1 aspect ratio. \\
\texttt{\ paper-size\ } &
\href{https://typst.app/docs/reference/foundations/str/}{string} &
Specify a
\href{https://typst.app/docs/reference/layout/page\#parameters-paper}{paper
size string} to change the page size (default is \texttt{\ a4\ } ). \\
\texttt{\ footer-text\ } &
\href{https://typst.app/docs/reference/foundations/content/}{content} &
Content that will be prepended to the page numbering in the footer. \\
\texttt{\ page-numbering-format\ } &
\href{https://typst.app/docs/reference/foundations/str/}{string} &
\href{https://typst.app/docs/reference/model/numbering/\#parameters-numbering}{Pattern}
that will be used for displaying page numbering in the footer (default
is \texttt{\ 1\ of\ 1\ } ). \\
\end{longtable}

The function also accepts a single, positional argument for the body.

The template will initialize your package with a sample call to the
\texttt{\ pesha\ } function in a show rule. If you, however, want to
change an existing project to use this template, you can add a show rule
like this at the top of your file:

\begin{Shaded}
\begin{Highlighting}[]
\NormalTok{\#import "@preview/pesha:0.4.0": *}

\NormalTok{\#show: pesha.with(}
\NormalTok{  name: "Max Mustermann",}
\NormalTok{  address: "5419 Hollywood Blvd Ste c731, Los Angeles, CA 90027",}
\NormalTok{  contacts: (}
\NormalTok{    [(323) 555 1435],}
\NormalTok{    [\#link("mailto:max@mustermann.com")],}
\NormalTok{  ),}
\NormalTok{  paper{-}size: "us{-}letter",}
\NormalTok{  footer{-}text: [Mustermann Résumé {-}{-}{-}]}
\NormalTok{)}

\NormalTok{// Your content goes below.}
\end{Highlighting}
\end{Shaded}

\href{/app?template=pesha&version=0.4.0}{Create project in app}

\subsubsection{How to use}\label{how-to-use}

Click the button above to create a new project using this template in
the Typst app.

You can also use the Typst CLI to start a new project on your computer
using this command:

\begin{verbatim}
typst init @preview/pesha:0.4.0
\end{verbatim}

\includesvg[width=0.16667in,height=0.16667in]{/assets/icons/16-copy.svg}

\subsubsection{About}\label{about}

\begin{description}
\tightlist
\item[Author :]
\href{https://github.com/talal}{Muhammad Talal Anwar}
\item[License:]
MIT-0
\item[Current version:]
0.4.0
\item[Last updated:]
October 24, 2024
\item[First released:]
March 23, 2024
\item[Minimum Typst version:]
0.12.0
\item[Archive size:]
4.24 kB
\href{https://packages.typst.org/preview/pesha-0.4.0.tar.gz}{\pandocbounded{\includesvg[keepaspectratio]{/assets/icons/16-download.svg}}}
\item[Repository:]
\href{https://github.com/talal/pesha}{GitHub}
\item[Categor y :]
\begin{itemize}
\tightlist
\item[]
\item
  \pandocbounded{\includesvg[keepaspectratio]{/assets/icons/16-user.svg}}
  \href{https://typst.app/universe/search/?category=cv}{CV}
\end{itemize}
\end{description}

\subsubsection{Where to report issues?}\label{where-to-report-issues}

This template is a project of Muhammad Talal Anwar . Report issues on
\href{https://github.com/talal/pesha}{their repository} . You can also
try to ask for help with this template on the
\href{https://forum.typst.app}{Forum} .

Please report this template to the Typst team using the
\href{https://typst.app/contact}{contact form} if you believe it is a
safety hazard or infringes upon your rights.

\phantomsection\label{versions}
\subsubsection{Version history}\label{version-history}

\begin{longtable}[]{@{}ll@{}}
\toprule\noalign{}
Version & Release Date \\
\midrule\noalign{}
\endhead
\bottomrule\noalign{}
\endlastfoot
0.4.0 & October 24, 2024 \\
\href{https://typst.app/universe/package/pesha/0.3.1/}{0.3.1} & April
19, 2024 \\
\href{https://typst.app/universe/package/pesha/0.3.0/}{0.3.0} & April
15, 2024 \\
\href{https://typst.app/universe/package/pesha/0.2.0/}{0.2.0} & April
12, 2024 \\
\href{https://typst.app/universe/package/pesha/0.1.0/}{0.1.0} & March
23, 2024 \\
\end{longtable}

Typst GmbH did not create this template and cannot guarantee correct
functionality of this template or compatibility with any version of the
Typst compiler or app.


\title{typst.app/universe/package/postercise}

\phantomsection\label{banner}
\section{postercise}\label{postercise}

{ 0.1.0 }

Postercise allows users to easily create academic research posters with
different themes using Typst.

\phantomsection\label{readme}
\emph{Postercise} allows users to easily create academic research
posters with different themes using \href{https://typst.app/}{Typst} .

\pandocbounded{\includegraphics[keepaspectratio]{https://img.shields.io/github/license/dangh3014/postercise}}
\pandocbounded{\includegraphics[keepaspectratio]{https://img.shields.io/github/v/release/dangh3014/postercise}}
\pandocbounded{\includegraphics[keepaspectratio]{https://img.shields.io/github/stars/dangh3014/postercise}}

\subsection{Getting started}\label{getting-started}

You can get \textbf{Postercise} from the official package repository by
entering the following.

\begin{Shaded}
\begin{Highlighting}[]
\NormalTok{\#import "@preview/postercise:0.1.0": *}
\end{Highlighting}
\end{Shaded}

Another option is to use \textbf{Postercise} as a local module by
downloading the package files into your project folder.

Next you will need to import a theme, set up the page and font, and call
the \texttt{\ show\ } command.

\begin{Shaded}
\begin{Highlighting}[]
\NormalTok{\#import themes.basic: *}

\NormalTok{\#set page(width: 24in, height: 18in)}
\NormalTok{\#set text(size: 24pt)}

\NormalTok{\#show: theme}
\end{Highlighting}
\end{Shaded}

To add content to the poster, use the \texttt{\ poster-content\ }
command.

\begin{Shaded}
\begin{Highlighting}[]
\NormalTok{\#poster{-}content()[}
\NormalTok{  // Content goes here}
\NormalTok{]}
\end{Highlighting}
\end{Shaded}

There are a few options for types of content that should be added inside
the \texttt{\ poster-content\ } . The body of the poster can be typed as
normal, or two box styles are provided to headings and/or highlight
content in special ways.

\begin{Shaded}
\begin{Highlighting}[]
\NormalTok{\#normal{-}box[]}
\NormalTok{\#focus{-}box[]}
\end{Highlighting}
\end{Shaded}

Basic information like title and authors is placed as options using the
\texttt{\ poster-header\ } script.

\begin{Shaded}
\begin{Highlighting}[]
\NormalTok{\#poster{-}header(}
\NormalTok{  title: [Title],}
\NormalTok{  authors: [Author],}
\NormalTok{)}
\end{Highlighting}
\end{Shaded}

Finally, additional content can be added to the footer with the
\texttt{\ poster-footer\ } script.

\begin{Shaded}
\begin{Highlighting}[]
\NormalTok{\#poster{-}footer[]}
\end{Highlighting}
\end{Shaded}

Again, as a reminder, all of these scripts should be called from inside
of the \texttt{\ poster-content\ } block.

Using these commands, it is easy to produce posters like the following:
\pandocbounded{\includegraphics[keepaspectratio]{https://raw.githubusercontent.com/dangh3014/postercise/main/examples/postercise-examples.png}}

\subsection{More details}\label{more-details}

\subsubsection{\texorpdfstring{\texttt{\ themes\ }}{ themes }}\label{themes}

Currently, 3 themes are available. Use one of these \texttt{\ import\ }
commands to load that theme.

\begin{Shaded}
\begin{Highlighting}[]
\NormalTok{\#import themes.basic: *}
\NormalTok{\#import themes.better: *}
\NormalTok{\#import themes.boxes: *}
\end{Highlighting}
\end{Shaded}

\subsubsection{\texorpdfstring{\texttt{\ show:\ theme.with()\ }}{ show: theme.with() }}\label{show-theme.with}

Theme options allow you to adjust the color scheme, as well as the color
and size of the content in the header. The defaults are shown below.
(The ‘better.typ’ theme defaults to different titletext color and
size.)

\begin{Shaded}
\begin{Highlighting}[]
\NormalTok{\#show: theme.with(}
\NormalTok{  primary{-}color: rgb(28,55,103), // Dark blue}
\NormalTok{  background{-}color: white,}
\NormalTok{  accent{-}color: rgb(243,163,30), // Yellow}
\NormalTok{  titletext{-}color: white,}
\NormalTok{  titletext{-}size: 2em,}
\NormalTok{)}
\end{Highlighting}
\end{Shaded}

\subsubsection{\texorpdfstring{\texttt{\ poster-content(){[}{]}\ }}{ poster-content(){[}{]} }}\label{poster-content}

The only option for the main content is the number of columns. This
defaults to 3 for most themes. For the “better.typ� theme, there is
1 column and content is placed in the leftmost column below
\texttt{\ poster-header\ } .

\begin{Shaded}
\begin{Highlighting}[]
\NormalTok{\#poster{-}content(col: 3)[}
\NormalTok{  // Content goes here}
\NormalTok{]}
\end{Highlighting}
\end{Shaded}

\subsubsection{\texorpdfstring{\texttt{\ normal-box(){[}{]}\ } and
\texttt{\ focus-box(){[}{]}\ }}{ normal-box(){[}{]}  and  focus-box(){[}{]} }}\label{normal-box-and-focus-box}

By default, these boxes use the no fill and the accent-color fill,
respectively. However, they do accept color as an option, and will add a
primary-color stroke around the box if a color is given. For the
“better.typ� theme, use \texttt{\ focus-box\ } to place content in
the center column.

\begin{Shaded}
\begin{Highlighting}[]
\NormalTok{\#normal{-}box(color: none)[}
\NormalTok{  // Content}
\NormalTok{]}

\NormalTok{\#focus{-}box(color: none)[}
\NormalTok{  // Content}
\NormalTok{]}
\end{Highlighting}
\end{Shaded}

\subsubsection{\texorpdfstring{\texttt{\ poster-header()\ }}{ poster-header() }}\label{poster-header}

Available options for the poster header for most themes are shown below.
Note that logos should be explicitly labeled as images. Logos are not
currently displayed in the header in the “better.typ� theme.

\begin{Shaded}
\begin{Highlighting}[]
\NormalTok{\#poster{-}header(}
\NormalTok{  title: [Title],}
\NormalTok{  subtitle: [Subtitle],}
\NormalTok{  author: [Author],}
\NormalTok{  affiliation: [Affiliation],}
\NormalTok{  logo{-}1: image("placeholder.png")}
\NormalTok{  logo{-}2: image("placeholder.png") }
\NormalTok{)}
\end{Highlighting}
\end{Shaded}

\subsubsection{\texorpdfstring{\texttt{\ poster-footer{[}{]}\ }}{ poster-footer{[}{]} }}\label{poster-footer}

This command does not currently have any extra options. The content is
typically placed at the bottom of the poster, but it is placed in the
rightmost column for the “better.typ� theme.

\begin{Shaded}
\begin{Highlighting}[]
\NormalTok{\#poster{-}footer[}
\NormalTok{  // Content}
\NormalTok{]}
\end{Highlighting}
\end{Shaded}

\subsection{Known Issues}\label{known-issues}

\begin{itemize}
\tightlist
\item
  The bibliography does not work properly and must be done manually
\item
  Figure captions do not number correctly and must be done manually
\end{itemize}

\subsection{Planned Features/Themes}\label{planned-featuresthemes}

\begin{itemize}
\tightlist
\item
  Themes that use color gradients and background images
\item
  Add QR code generation
\end{itemize}

\subsubsection{How to add}\label{how-to-add}

Copy this into your project and use the import as
\texttt{\ postercise\ }

\begin{verbatim}
#import "@preview/postercise:0.1.0"
\end{verbatim}

\includesvg[width=0.16667in,height=0.16667in]{/assets/icons/16-copy.svg}

Check the docs for
\href{https://typst.app/docs/reference/scripting/\#packages}{more
information on how to import packages} .

\subsubsection{About}\label{about}

\begin{description}
\tightlist
\item[Author :]
Daniel King
\item[License:]
MIT
\item[Current version:]
0.1.0
\item[Last updated:]
May 27, 2024
\item[First released:]
May 27, 2024
\item[Archive size:]
5.11 kB
\href{https://packages.typst.org/preview/postercise-0.1.0.tar.gz}{\pandocbounded{\includesvg[keepaspectratio]{/assets/icons/16-download.svg}}}
\item[Repository:]
\href{https://github.com/dangh3014/postercise/}{GitHub}
\item[Categor y :]
\begin{itemize}
\tightlist
\item[]
\item
  \pandocbounded{\includesvg[keepaspectratio]{/assets/icons/16-pin.svg}}
  \href{https://typst.app/universe/search/?category=poster}{Poster}
\end{itemize}
\end{description}

\subsubsection{Where to report issues?}\label{where-to-report-issues}

This package is a project of Daniel King . Report issues on
\href{https://github.com/dangh3014/postercise/}{their repository} . You
can also try to ask for help with this package on the
\href{https://forum.typst.app}{Forum} .

Please report this package to the Typst team using the
\href{https://typst.app/contact}{contact form} if you believe it is a
safety hazard or infringes upon your rights.

\phantomsection\label{versions}
\subsubsection{Version history}\label{version-history}

\begin{longtable}[]{@{}ll@{}}
\toprule\noalign{}
Version & Release Date \\
\midrule\noalign{}
\endhead
\bottomrule\noalign{}
\endlastfoot
0.1.0 & May 27, 2024 \\
\end{longtable}

Typst GmbH did not create this package and cannot guarantee correct
functionality of this package or compatibility with any version of the
Typst compiler or app.


\title{typst.app/universe/package/datify}

\phantomsection\label{banner}
\section{datify}\label{datify}

{ 0.1.3 }

Datify is a simple date package that allows users to format dates in new
ways and addresses the issue of lacking date formats in different
languages.

\phantomsection\label{readme}
Datify is a simple date package that allows users to format dates in new
ways and addresses the issue of lacking date formats in different
languages.

\subsection{Installation}\label{installation}

To include this package in your Typst project, add the following to your
project file:

\begin{Shaded}
\begin{Highlighting}[]
\NormalTok{\#import "@preview/datify:0.1.3": day{-}name, month{-}name, custom{-}date{-}format}
\end{Highlighting}
\end{Shaded}

\subsection{Reference}\label{reference}

\subsubsection{\texorpdfstring{\texttt{\ day-name\ }}{ day-name }}\label{day-name}

Returns the name of the weekday.

\paragraph{Example}\label{example}

\begin{Shaded}
\begin{Highlighting}[]
\NormalTok{\#import "@preview/datify:0.1.3": day{-}name}

\NormalTok{\#day{-}name(2)}
\NormalTok{\#day{-}name(1,"fr",true)}
\end{Highlighting}
\end{Shaded}

Output:

\begin{verbatim}
tuesday
Lundi
\end{verbatim}

\paragraph{Parameters}\label{parameters}

\begin{Shaded}
\begin{Highlighting}[]
\NormalTok{day{-}name(weekday: int or str, lang: str, upper: boolean) {-}{-}\textgreater{} str}
\end{Highlighting}
\end{Shaded}

\begin{longtable}[]{@{}lll@{}}
\toprule\noalign{}
Parameter & Description & Default \\
\midrule\noalign{}
\endhead
\bottomrule\noalign{}
\endlastfoot
weekday* & The weekday as an integer (1-7) or a string
(“1�-“7�). & none \\
lang & An ISO 639-1 code representing the language. & en \\
upper & A boolean that sets the first letter to be uppercase. & false \\
\end{longtable}

* required

\subsubsection{\texorpdfstring{\texttt{\ month-name\ }}{ month-name }}\label{month-name}

Returns the name of the month.

\paragraph{Example}\label{example-1}

\begin{Shaded}
\begin{Highlighting}[]
\NormalTok{\#import "@preview/datify:0.1.3": month{-}name}

\NormalTok{\#month{-}name(2)}
\NormalTok{\#month{-}name(1, "fr", true)}
\end{Highlighting}
\end{Shaded}

Output:

\begin{verbatim}
february
Janvier
\end{verbatim}

\paragraph{Parameters}\label{parameters-1}

\begin{Shaded}
\begin{Highlighting}[]
\NormalTok{month{-}name(month: int or str, lang: str = \textquotesingle{}en\textquotesingle{}, upper: bool = false) {-}\textgreater{} str}
\end{Highlighting}
\end{Shaded}

\begin{longtable}[]{@{}lll@{}}
\toprule\noalign{}
Parameter & Description & Default \\
\midrule\noalign{}
\endhead
\bottomrule\noalign{}
\endlastfoot
month* & The month as an integer (1-12) or a string (“1�-“12�).
& none \\
lang & An ISO 639-1 code representing the language. & en \\
upper & A boolean that sets the first letter to be uppercase. & false \\
\end{longtable}

* required

\subsubsection{\texorpdfstring{\texttt{\ custom-date-format\ }}{ custom-date-format }}\label{custom-date-format}

Formats a given date according to a specified format and language.

\paragraph{Example}\label{example-2}

\begin{Shaded}
\begin{Highlighting}[]
\NormalTok{\#import "@preview/datify:0.1.3": custom{-}date{-}format}

\NormalTok{\#let my{-}date = datetime(year: 2024, month: 8, day: 4)}
\NormalTok{\#custom{-}date{-}format(my{-}date, "MMMM DDth, YYYY")}
\end{Highlighting}
\end{Shaded}

Output:

\begin{verbatim}
August 04th, 2024
\end{verbatim}

\paragraph{Parameters}\label{parameters-2}

\begin{Shaded}
\begin{Highlighting}[]
\NormalTok{custom{-}date{-}format(date: datetime, format: str, lang: str = \textquotesingle{}en\textquotesingle{}) {-}\textgreater{} str}
\end{Highlighting}
\end{Shaded}

\begin{longtable}[]{@{}lll@{}}
\toprule\noalign{}
Parameter & Description & Default \\
\midrule\noalign{}
\endhead
\bottomrule\noalign{}
\endlastfoot
date* & A datetime object representing the date. & none \\
format* & A string representing the desired date format. & none \\
lang & An ISO 639-1 code representing the language. & en \\
\end{longtable}

* required

\paragraph{Format Types}\label{format-types}

Below is a table of all possible format types that can be used in the
format string:

\begin{longtable}[]{@{}lll@{}}
\toprule\noalign{}
Format & Description & Example \\
\midrule\noalign{}
\endhead
\bottomrule\noalign{}
\endlastfoot
\texttt{\ DD\ } & Day of the month, 2 digits & 05 \\
\texttt{\ day\ } & Full name of the day & tuesday \\
\texttt{\ Day\ } & Capitalized full name of the day & Tuesday \\
\texttt{\ DAY\ } & Uppercase full name of the day & TUESDAY \\
\texttt{\ MMMM\ } & Capitalized full name of the month & May \\
\texttt{\ MMM\ } & Short name of the month (first 3 chars) & May \\
\texttt{\ MM\ } & Month number, 2 digits & 05 \\
\texttt{\ month\ } & Full name of the month & may \\
\texttt{\ Month\ } & Capitalized full name of the month & May \\
\texttt{\ MONTH\ } & Uppercase full name of the month & MAY \\
\texttt{\ YYYY\ } & 4-digit year & 2023 \\
\texttt{\ YY\ } & Last 2 digits of the year & 23 \\
\end{longtable}

\subsection{Examples}\label{examples}

Here are some examples demonstrating the usage of the functions provided
by the Datify package:

\begin{Shaded}
\begin{Highlighting}[]
\NormalTok{\#let my{-}date = datetime(year: 2024, month: 12, day: 25)}

\NormalTok{\#custom{-}date{-}format(my{-}date, "DD{-}MM{-}YYYY")  // Output: 25{-}12{-}2024}
\NormalTok{\#custom{-}date{-}format(my{-}date, "Day, DD Month YYYY", "fr")  // Output: Mercredi, 25 Décembre 2024}
\NormalTok{\#custom{-}date{-}format(my{-}date, "day, DD de month de YYYY", "es") // Output: miércoles, 25 de diciembre de 2024}
\NormalTok{\#custom{-}date{-}format(my{-}date, "day, DD de month de YYYY", "pt") // Output: terça{-}feira, 25 de dezembro de 2024}

\NormalTok{\#day{-}name(4)  // Output: thursday}

\NormalTok{\#month{-}name(12)  // Output: december}
\end{Highlighting}
\end{Shaded}

\subsection{Supported language}\label{supported-language}

\begin{longtable}[]{@{}ll@{}}
\toprule\noalign{}
ISO 639-1 code & Status \\
\midrule\noalign{}
\endhead
\bottomrule\noalign{}
\endlastfoot
aa & � \\
ab & � \\
ae & � \\
af & � \\
ak & � \\
am & � \\
an & � \\
ar & � \\
as & � \\
av & � \\
ay & � \\
az & � \\
ba & � \\
be & � \\
bg & � \\
bh & � \\
bi & � \\
bm & � \\
bn & � \\
bo & � \\
br & � \\
bs & � \\
ca & � \\
ce & � \\
ch & � \\
co & � \\
cr & � \\
cs & � \\
cu & � \\
cv & � \\
cy & � \\
da & � \\
de & ✠\\
dv & � \\
dz & � \\
ee & � \\
el & � \\
en & ✠\\
eo & � \\
es & ✠\\
et & � \\
eu & � \\
fa & � \\
ff & � \\
fi & � \\
fj & � \\
fo & � \\
fr & ✠\\
fy & � \\
ga & � \\
gd & � \\
gl & � \\
gn & � \\
gu & � \\
gv & � \\
ha & � \\
he & � \\
hi & � \\
ho & � \\
hr & � \\
ht & � \\
hu & � \\
hy & � \\
hz & � \\
ia & � \\
id & � \\
ie & � \\
ig & � \\
ii & � \\
ik & � \\
io & � \\
is & � \\
it & � \\
iu & � \\
ja & � \\
jv & � \\
ka & � \\
kg & � \\
ki & � \\
kj & � \\
kk & � \\
kl & � \\
km & � \\
kn & � \\
ko & � \\
kr & � \\
ks & � \\
ku & � \\
kv & � \\
kw & � \\
ky & � \\
la & � \\
lb & � \\
lg & � \\
li & � \\
ln & � \\
lo & � \\
lt & � \\
lu & � \\
lv & � \\
mg & � \\
mh & � \\
mi & � \\
mk & � \\
ml & � \\
mn & � \\
mr & � \\
ms & � \\
mt & � \\
my & � \\
na & � \\
nb & � \\
nd & � \\
ne & � \\
ng & � \\
nl & � \\
nn & � \\
no & � \\
nr & � \\
nv & � \\
ny & � \\
oc & � \\
oj & � \\
om & � \\
or & � \\
os & � \\
pa & � \\
pi & � \\
pl & � \\
ps & � \\
pt & ✠\\
qu & � \\
rm & � \\
rn & � \\
ro & � \\
ru & � \\
rw & � \\
sa & � \\
sc & � \\
sd & � \\
se & � \\
sg & � \\
si & � \\
sk & � \\
sl & � \\
sm & � \\
sn & � \\
so & � \\
sq & � \\
sr & � \\
ss & � \\
st & � \\
su & � \\
sv & � \\
sw & � \\
ta & � \\
te & � \\
tg & � \\
th & � \\
ti & � \\
tk & � \\
tl & � \\
tn & � \\
to & � \\
tr & � \\
ts & � \\
tt & � \\
tw & � \\
ty & � \\
ug & � \\
uk & � \\
ur & � \\
uz & � \\
ve & � \\
vi & � \\
vo & � \\
wa & � \\
wo & � \\
xh & � \\
yi & � \\
yo & � \\
za & � \\
zh & � \\
zu & � \\
\end{longtable}

\subsection{Contributing}\label{contributing}

Contributions are welcome! Please feel free to submit a pull request or
open an issue if you encounter any problems.

\subsection{License}\label{license}

This project is licensed under the MIT License.

\subsection{Planned}\label{planned}

\begin{itemize}
\tightlist
\item
  Adding support for more language
\item
  Adding set and get method to set default language for a whole document
\item
  Adding new methods
\end{itemize}

\subsubsection{How to add}\label{how-to-add}

Copy this into your project and use the import as \texttt{\ datify\ }

\begin{verbatim}
#import "@preview/datify:0.1.3"
\end{verbatim}

\includesvg[width=0.16667in,height=0.16667in]{/assets/icons/16-copy.svg}

Check the docs for
\href{https://typst.app/docs/reference/scripting/\#packages}{more
information on how to import packages} .

\subsubsection{About}\label{about}

\begin{description}
\tightlist
\item[Author :]
Jeomhps
\item[License:]
MIT
\item[Current version:]
0.1.3
\item[Last updated:]
November 4, 2024
\item[First released:]
May 27, 2024
\item[Minimum Typst version:]
0.11.1
\item[Archive size:]
4.81 kB
\href{https://packages.typst.org/preview/datify-0.1.3.tar.gz}{\pandocbounded{\includesvg[keepaspectratio]{/assets/icons/16-download.svg}}}
\item[Repository:]
\href{https://github.com/Jeomhps/datify}{GitHub}
\end{description}

\subsubsection{Where to report issues?}\label{where-to-report-issues}

This package is a project of Jeomhps . Report issues on
\href{https://github.com/Jeomhps/datify}{their repository} . You can
also try to ask for help with this package on the
\href{https://forum.typst.app}{Forum} .

Please report this package to the Typst team using the
\href{https://typst.app/contact}{contact form} if you believe it is a
safety hazard or infringes upon your rights.

\phantomsection\label{versions}
\subsubsection{Version history}\label{version-history}

\begin{longtable}[]{@{}ll@{}}
\toprule\noalign{}
Version & Release Date \\
\midrule\noalign{}
\endhead
\bottomrule\noalign{}
\endlastfoot
0.1.3 & November 4, 2024 \\
\href{https://typst.app/universe/package/datify/0.1.2/}{0.1.2} & May 31,
2024 \\
\href{https://typst.app/universe/package/datify/0.1.1/}{0.1.1} & May 27,
2024 \\
\end{longtable}

Typst GmbH did not create this package and cannot guarantee correct
functionality of this package or compatibility with any version of the
Typst compiler or app.


\title{typst.app/universe/package/supercharged-dhbw}

\phantomsection\label{banner}
\phantomsection\label{template-thumbnail}
\pandocbounded{\includegraphics[keepaspectratio]{https://packages.typst.org/preview/thumbnails/supercharged-dhbw-3.3.2-small.webp}}

\section{supercharged-dhbw}\label{supercharged-dhbw}

{ 3.3.2 }

Unofficial Template for DHBW

\href{/app?template=supercharged-dhbw&version=3.3.2}{Create project in
app}

\phantomsection\label{readme}
Unofficial \href{https://typst.app/}{Typst} template for DHBW students.

You can see an example PDF of how the template looks
\href{https://github.com/DannySeidel/typst-dhbw-template/blob/main/examples/example.pdf}{here}
.

To see an example of how you can use this template, check out the
\texttt{\ main.typ\ } file. More examples can be found in the
\href{https://github.com/DannySeidel/typst-dhbw-template/blob/main/examples}{examples
directory} of the GitHub repository.

\subsection{Usage}\label{usage}

You can use this template in the Typst web app by clicking “Start from
template� on the dashboard and searching for
\texttt{\ supercharged-dhbw\ } .

Alternatively, you can use the CLI to kick this project off using the
command

\begin{Shaded}
\begin{Highlighting}[]
\NormalTok{typst init @preview/supercharged{-}dhbw}
\end{Highlighting}
\end{Shaded}

Typst will create a new directory with all the files needed to get you
started.

\subsection{Fonts}\label{fonts}

This template uses the following fonts:

\begin{itemize}
\tightlist
\item
  \href{https://fonts.google.com/specimen/Montserrat}{Montserrat}
\item
  \href{https://fonts.google.com/specimen/Open+Sans}{Open Sans}
\end{itemize}

If you want to use typst locally, you can download the fonts from the
links above and install them on your system. Otherwise, when using the
web version add the fonts to your project.

For further information on how to add fonts to your project, please
refer to the
\href{https://typst.app/docs/reference/text/text/\#parameters-font}{Typst
documentation} .

\subsection{Used Packages}\label{used-packages}

This template uses the following packages:

\begin{itemize}
\tightlist
\item
  \href{https://typst.app/universe/package/codelst}{codelst} : To create
  code snippets
\end{itemize}

Insert code snippets using the following syntax:

\begin{Shaded}
\begin{Highlighting}[]
\NormalTok{\#figure(caption: "Codeblock Example", sourcecode[\textasciigrave{}\textasciigrave{}\textasciigrave{}ts}
\NormalTok{const ReactComponent = () =\textgreater{} \{}
\NormalTok{  return (}
\NormalTok{    \textless{}div\textgreater{}}
\NormalTok{      \textless{}h1\textgreater{}Hello World\textless{}/h1\textgreater{}}
\NormalTok{    \textless{}/div\textgreater{}}
\NormalTok{  );}
\NormalTok{\};}

\NormalTok{export default ReactComponent;}
\NormalTok{\textasciigrave{}\textasciigrave{}\textasciigrave{}])}
\end{Highlighting}
\end{Shaded}

\subsection{Configuration}\label{configuration}

This template exports the \texttt{\ supercharged-dhbw\ } function with
the following named arguments:

\texttt{\ title\ (str*)\ } : Title of the document

\texttt{\ authors\ (dictionary*)\ } : List of authors with the following
named arguments (max. 6 authors when in the company or 8 authors when at
DHBW):

\begin{itemize}
\tightlist
\item
  name (str*): Name of the author
\item
  student-id (str*): Student ID of the author
\item
  course (str*): Course of the author
\item
  course-of-studies (str*): Course of studies of the author
\item
  company (dictionary): Company of the author (only needed when
  \texttt{\ at-university\ } is \texttt{\ false\ } ) with the following
  named arguments:

  \begin{itemize}
  \tightlist
  \item
    name (str*): Name of the company
  \item
    post-code (str): Post code of the company
  \item
    city (str*): City of the company
  \item
    country (str): Country of the company
  \end{itemize}
\end{itemize}

\texttt{\ abstract\ (content)\ } : Content of the abstract, it is
recommended that you pass a variable containing the content or a
function that returns the content

\texttt{\ acronym-spacing\ (length)\ } : Spacing between the acronym and
its long form (check the
\href{https://typst.app/docs/reference/layout/length/}{Typst
documentation} for examples on how to provide parameters of type
length), default is \texttt{\ 5em\ }

\texttt{\ acronyms\ (dictionary)\ } : Pass a dictionary containing the
acronyms and their long forms (See the example in the
\texttt{\ acronyms.typ\ } file)

\texttt{\ appendix\ (content)\ } : Content of the appendix, it is
recommended that you pass a variable containing the content or a
function that returns the content

\texttt{\ at-university\ (bool*)\ } : Whether the document is written at
university or not, default is \texttt{\ false\ }

\texttt{\ bibliography\ (content)\ } : Path to the bibliography file

\texttt{\ bib-style\ (str)\ } : Style of the bibliography, default is
\texttt{\ ieee\ }

\texttt{\ city\ (str)\ } : City of the author (only needed when
\texttt{\ at-university\ } is \texttt{\ true\ } )

\texttt{\ confidentiality-marker:\ (dictionary)\ } : Configure the
confidentially marker (red or green circle) on the title page (using
this option reduces the maximum number of authors by 2 to 4 authors when
in the company or 6 authors when at DHBW)

\begin{itemize}
\tightlist
\item
  display (bool*): Whether the confidentiality marker should be shown,
  default is \texttt{\ false\ }
\item
  offset-x (length): Horizontal offset of the confidentiality marker,
  default is \texttt{\ 0pt\ }
\item
  offset-y (length): Vertical offset of the confidentiality marker,
  default is \texttt{\ 0pt\ }
\item
  size (length): Size of the confidentiality marker, default is
  \texttt{\ 7em\ }
\item
  title-spacing (length): Adds space below the title to make room for
  the confidentiality marker, default is \texttt{\ 2em\ }
\end{itemize}

\texttt{\ confidentiality-statement-content\ (content)\ } : Provide a
custom confidentiality statement

\texttt{\ date\ (datetime*\ \textbar{}\ array*)\ } : Provide a datetime
object to display one date (e.g. submission date) or a array containing
two datetime objects to display a date range (e.g. start and end date of
the project), default is \texttt{\ datetime.today()\ }

\texttt{\ date-format\ (str)\ } : Format of the displayed dates, default
is \texttt{\ "{[}day{]}.{[}month{]}.{[}year{]}"\ } (for more information
on possible formats check the
\href{https://typst.app/docs/reference/foundations/datetime/\#format}{Typst
documentation} )

\texttt{\ declaration-of-authorship-content\ (content)\ } : Provide a
custom declaration of authorship

\texttt{\ glossary\ (dictionary)\ } : Pass a dictionary containing the
glossary terms and their definitions (See the example in the
\texttt{\ glossary.typ\ } file)

\texttt{\ glossary-spacing\ (length)\ } : Spacing between the glossary
term and its definition (check the
\href{https://typst.app/docs/reference/layout/length/}{Typst
documentation} for examples on how to provide parameters of type
length), default is \texttt{\ 1.5em\ }

\texttt{\ header\ (dictionary)\ } : Configure the header of the document

\begin{itemize}
\tightlist
\item
  display (bool): Whether the header should be shown, default is
  \texttt{\ true\ }
\item
  show-chapter (bool): Whether the current chapter should be shown in
  the header, default is \texttt{\ true\ }
\item
  show-left-logo (bool): Whether the left logo should be shown in the
  header, default is \texttt{\ true\ }
\item
  show-right-logo (bool): Whether the right logo should be shown in the
  header, default is \texttt{\ true\ }
\item
  show-divider (bool): Whether the header divider should be shown,
  default is \texttt{\ true\ }
\item
  content (content): Content for a custom header, it is recommended that
  you pass a variable containing the content or a function that returns
  the content
\end{itemize}

\texttt{\ heading-numering\ (str)\ } : Numbering style of the headings,
default is \texttt{\ "1.1"\ } (for more information on possible
numbering formats check the
\href{https://typst.app/docs/reference/model/numbering}{Typst
documentation} )

\texttt{\ ignored-link-label-keys-for-highlighting\ (array)\ } : List of
keys of labels that should be ignored when highlighting links in the
document, default is \texttt{\ ()\ }

\texttt{\ language\ (str*)\ } : Language of the document which is either
\texttt{\ en\ } or \texttt{\ de\ } , default is \texttt{\ en\ }

\texttt{\ logo-left\ (content)\ } : Path to the logo on the left side of
the title page (usage: image(“path/to/image.png�)), default is the
\texttt{\ DHBW\ logo\ }

\texttt{\ logo-right\ (content)\ } : Path to the logo on the right side
of the title page (usage: image(“path/to/image.png�)), default is
\texttt{\ no\ logo\ }

\texttt{\ logo-size-ratio\ (str)\ } : Ratio between the right logo and
the left logo height (left-logo:right-logo), default is
\texttt{\ "1:1"\ }

\texttt{\ math-numbering\ (str)\ } : Numbering style of the math
equations, set to \texttt{\ none\ } to turn off equation numbering,
default is \texttt{\ "(1)"\ } (for more information on possible
numbering formats check the
\href{https://typst.app/docs/reference/model/numbering}{Typst
documentation} )

\texttt{\ numbering-alignment\ (alignment)\ } : Alignment of the page
numbering (for possible options check the
\href{https://typst.app/docs/reference/layout/alignment/}{Typst
documentation} ), default is \texttt{\ center\ }

\texttt{\ show-abstract\ (bool)\ } : Whether the abstract should be
shown, default is \texttt{\ true\ }

\texttt{\ show-acronyms\ (bool)\ } : Whether the list of acronyms should
be shown, default is \texttt{\ true\ }

\texttt{\ show-code-snippets\ (bool)\ } : Whether the code snippets
should be shown, default is \texttt{\ true\ }

\texttt{\ show-confidentiality-statement\ (bool)\ } : Whether the
confidentiality statement should be shown, default is \texttt{\ true\ }

\texttt{\ show-declaration-of-authorship\ (bool)\ } : Whether the
declaration of authorship should be shown, default is \texttt{\ true\ }

\texttt{\ show-list-of-figures\ (bool)\ } : Whether the list of figures
should be shown, default is \texttt{\ true\ }

\texttt{\ show-list-of-tables\ (bool)\ } : Whether the list of tables
should be shown, default is \texttt{\ true\ }

\texttt{\ show-table-of-contents\ (bool)\ } : Whether the table of
contents should be shown, default is \texttt{\ true\ }

\texttt{\ supervisor\ (dictionary*)\ } : Name of the supervisor at the
university and/or company (e.g. supervisor: (company: “John Doe�,
university: “Jane Doe�))

\begin{itemize}
\tightlist
\item
  company (str): Name of the supervisor at the company (note while the
  argument is optional at least one of the two arguments must be
  provided)
\item
  university (str): Name of the supervisor at the university (note while
  the argument is optional at least one of the two arguments must be
  provided)
\end{itemize}

\texttt{\ titlepage-content\ (content)\ } : Provide a custom title page

\texttt{\ toc-depth\ (int)\ } : Depth of the table of contents, default
is \texttt{\ 3\ }

\texttt{\ type-of-thesis\ (str)\ } : Type of the thesis, default is
\texttt{\ none\ } (using this option reduces the maximum number of
authors by 2 to 4 authors when in the company or 6 authors when at DHBW)

\texttt{\ type-of-degree\ (str)\ } : Type of the degree, default is
\texttt{\ none\ } (using this option reduces the maximum number of
authors by 2 to 4 authors when in the company or 6 authors when at DHBW)

\texttt{\ university\ (str*)\ } : Name of the university

\texttt{\ university-location\ (str*)\ } : Campus or city of the
university

\texttt{\ university-short\ (str*)\ } : Short name of the university
(e.g. DHBW), displayed for the university supervisor

Behind the arguments the type of the value is given in parentheses. All
arguments marked with \texttt{\ *\ } are required.

\subsection{Acronyms}\label{acronyms}

This template provides functions to reference acronyms in the text. To
use these functions, you need to define the acronyms in the
\texttt{\ acronyms\ } attribute of the template. The acronyms referenced
with the functions below will be linked to their definition in the list
of acronyms.

\subsubsection{Functions}\label{functions}

This template provides the following functions to reference acronyms:

\texttt{\ acr\ } : Reference an acronym in the text (e.g.
\texttt{\ acr("API")\ } -\textgreater{}
\texttt{\ Application\ Programming\ Interface\ (API)\ } or
\texttt{\ API\ } )

\texttt{\ acrpl\ } : Reference an acronym in the text in plural form
(e.g. \texttt{\ acrpl("API")\ } -\textgreater{}
\texttt{\ Application\ Programming\ Interfaces\ (API)\ } or
\texttt{\ APIs\ } )

\texttt{\ acrs\ } : Reference an acronym in the text in short form (e.g.
\texttt{\ acrs("API")\ } -\textgreater{} \texttt{\ API\ } )

\texttt{\ acrspl\ } : Reference an acronym in the text in short form in
plural form (e.g. \texttt{\ acrpl("API")\ } -\textgreater{}
\texttt{\ APIs\ } )

\texttt{\ acrl\ } : Reference an acronym in the text in long form (e.g.
\texttt{\ acrl("API")\ } -\textgreater{}
\texttt{\ Application\ Programming\ Interface\ } )

\texttt{\ acrlpl\ } : Reference an acronym in the text in long form in
plural form (e.g. \texttt{\ acrlpl("API")\ } -\textgreater{}
\texttt{\ Application\ Programming\ Interfaces\ } )

\texttt{\ acrf\ } : Reference an acronym in the text in full form (e.g.
\texttt{\ acrf("API")\ } -\textgreater{}
\texttt{\ Application\ Programming\ Interface\ (API)\ } )

\texttt{\ acrfpl\ } : Reference an acronym in the text in full form in
plural form (e.g. \texttt{\ acrfpl("API")\ } -\textgreater{}
\texttt{\ Application\ Programming\ Interfaces\ (API)\ } )

\subsubsection{Definition}\label{definition}

To define acronyms use a dictionary and pass it to the acronyms
attribute of the template. The dictionary should contain the acronyms as
keys and their long forms as values.

\begin{Shaded}
\begin{Highlighting}[]
\NormalTok{\#let acronyms = (}
\NormalTok{  API: "Application Programming Interface",}
\NormalTok{  HTTP: "Hypertext Transfer Protocol",}
\NormalTok{  REST: "Representational State Transfer",}
\NormalTok{)}
\end{Highlighting}
\end{Shaded}

To define the plural form of an acronym use a array as value with the
first element being the singular form and the second element being the
plural form. If you don’t define the plural form, the template will
automatically add an “s� to the singular form.

\begin{Shaded}
\begin{Highlighting}[]
\NormalTok{\#let acronyms = (}
\NormalTok{  API: ("Application Programming Interface", "Application Programming Interfaces"),}
\NormalTok{  HTTP: ("Hypertext Transfer Protocol", "Hypertext Transfer Protocols"),}
\NormalTok{  REST: ("Representational State Transfer", "Representational State Transfers"),}
\NormalTok{)}
\end{Highlighting}
\end{Shaded}

\subsection{Glossary}\label{glossary}

Similar to the acronyms, this template provides a function to reference
glossary terms in the text. To use the function, you need to define the
glossary terms in the \texttt{\ glossary\ } attribute of the template.
The glossary terms referenced with the function below will be linked to
their definition in the list of glossary terms.

\subsubsection{Reference}\label{reference}

\texttt{\ gls\ } : Reference a glossary term in the text (e.g.
\texttt{\ gls("Vulnerability")\ } -\textgreater{} link to the definition
of “Vulnerability� in the glossary)

\subsubsection{Definition}\label{definition-1}

The definition works analogously to the acronyms. Define the glossary
terms in a dictionary and pass it to the glossary attribute of the
template. The dictionary should contain the glossary terms as keys and
their definitions as values.

\begin{Shaded}
\begin{Highlighting}[]
\NormalTok{\#let glossary = (}
\NormalTok{  Vulnerability: "A Vulnerability is a flaw in a computer system that weakens the overall security of the system.",}
\NormalTok{  Patch: "A patch is data that is intended to be used to modify an existing software resource such as a program or a file, often to fix bugs and security vulnerabilities.",}
\NormalTok{  Exploit: "An exploit is a method or piece of code that takes advantage of vulnerabilities in software, applications, networks, operating systems, or hardware, typically for malicious purposes.",}
\NormalTok{)}
\end{Highlighting}
\end{Shaded}

\href{/app?template=supercharged-dhbw&version=3.3.2}{Create project in
app}

\subsubsection{How to use}\label{how-to-use}

Click the button above to create a new project using this template in
the Typst app.

You can also use the Typst CLI to start a new project on your computer
using this command:

\begin{verbatim}
typst init @preview/supercharged-dhbw:3.3.2
\end{verbatim}

\includesvg[width=0.16667in,height=0.16667in]{/assets/icons/16-copy.svg}

\subsubsection{About}\label{about}

\begin{description}
\tightlist
\item[Author :]
\href{https://github.com/DannySeidel}{Danny Seidel}
\item[License:]
MIT
\item[Current version:]
3.3.2
\item[Last updated:]
November 4, 2024
\item[First released:]
May 14, 2024
\item[Archive size:]
26.9 kB
\href{https://packages.typst.org/preview/supercharged-dhbw-3.3.2.tar.gz}{\pandocbounded{\includesvg[keepaspectratio]{/assets/icons/16-download.svg}}}
\item[Repository:]
\href{https://github.com/DannySeidel/typst-dhbw-template}{GitHub}
\item[Categor ies :]
\begin{itemize}
\tightlist
\item[]
\item
  \pandocbounded{\includesvg[keepaspectratio]{/assets/icons/16-atom.svg}}
  \href{https://typst.app/universe/search/?category=paper}{Paper}
\item
  \pandocbounded{\includesvg[keepaspectratio]{/assets/icons/16-mortarboard.svg}}
  \href{https://typst.app/universe/search/?category=thesis}{Thesis}
\item
  \pandocbounded{\includesvg[keepaspectratio]{/assets/icons/16-speak.svg}}
  \href{https://typst.app/universe/search/?category=report}{Report}
\end{itemize}
\end{description}

\subsubsection{Where to report issues?}\label{where-to-report-issues}

This template is a project of Danny Seidel . Report issues on
\href{https://github.com/DannySeidel/typst-dhbw-template}{their
repository} . You can also try to ask for help with this template on the
\href{https://forum.typst.app}{Forum} .

Please report this template to the Typst team using the
\href{https://typst.app/contact}{contact form} if you believe it is a
safety hazard or infringes upon your rights.

\phantomsection\label{versions}
\subsubsection{Version history}\label{version-history}

\begin{longtable}[]{@{}ll@{}}
\toprule\noalign{}
Version & Release Date \\
\midrule\noalign{}
\endhead
\bottomrule\noalign{}
\endlastfoot
3.3.2 & November 4, 2024 \\
\href{https://typst.app/universe/package/supercharged-dhbw/3.3.1/}{3.3.1}
& October 3, 2024 \\
\href{https://typst.app/universe/package/supercharged-dhbw/3.3.0/}{3.3.0}
& September 22, 2024 \\
\href{https://typst.app/universe/package/supercharged-dhbw/3.2.0/}{3.2.0}
& September 17, 2024 \\
\href{https://typst.app/universe/package/supercharged-dhbw/3.1.1/}{3.1.1}
& August 26, 2024 \\
\href{https://typst.app/universe/package/supercharged-dhbw/3.1.0/}{3.1.0}
& August 21, 2024 \\
\href{https://typst.app/universe/package/supercharged-dhbw/3.0.0/}{3.0.0}
& August 8, 2024 \\
\href{https://typst.app/universe/package/supercharged-dhbw/2.2.0/}{2.2.0}
& July 29, 2024 \\
\href{https://typst.app/universe/package/supercharged-dhbw/2.1.0/}{2.1.0}
& July 19, 2024 \\
\href{https://typst.app/universe/package/supercharged-dhbw/2.0.2/}{2.0.2}
& July 4, 2024 \\
\href{https://typst.app/universe/package/supercharged-dhbw/2.0.1/}{2.0.1}
& July 4, 2024 \\
\href{https://typst.app/universe/package/supercharged-dhbw/2.0.0/}{2.0.0}
& July 2, 2024 \\
\href{https://typst.app/universe/package/supercharged-dhbw/1.5.0/}{1.5.0}
& June 24, 2024 \\
\href{https://typst.app/universe/package/supercharged-dhbw/1.4.0/}{1.4.0}
& June 10, 2024 \\
\href{https://typst.app/universe/package/supercharged-dhbw/1.3.1/}{1.3.1}
& May 27, 2024 \\
\href{https://typst.app/universe/package/supercharged-dhbw/1.3.0/}{1.3.0}
& May 23, 2024 \\
\href{https://typst.app/universe/package/supercharged-dhbw/1.2.0/}{1.2.0}
& May 16, 2024 \\
\href{https://typst.app/universe/package/supercharged-dhbw/1.1.0/}{1.1.0}
& May 16, 2024 \\
\href{https://typst.app/universe/package/supercharged-dhbw/1.0.0/}{1.0.0}
& May 14, 2024 \\
\end{longtable}

Typst GmbH did not create this template and cannot guarantee correct
functionality of this template or compatibility with any version of the
Typst compiler or app.


\title{typst.app/universe/package/numberingx}

\phantomsection\label{banner}
\section{numberingx}\label{numberingx}

{ 0.0.1 }

Extended numbering patterns using the CSS Counter Styles spec

\phantomsection\label{readme}
\emph{Extended numbering patterns using the
\href{https://www.w3.org/TR/css-counter-styles-3/}{CSS Counter Styles}
specification, along with a number of
\href{https://www.w3.org/TR/predefined-counter-styles/}{Ready-made
Counter Styles} .}

\subsection{Usage}\label{usage}

\begin{Shaded}
\begin{Highlighting}[]
\CommentTok{// numberingx is expected to be imported with the syntax creating a named module}
\NormalTok{\#}\ImportTok{import} \StringTok{"@preview/numberingx:0.0.1"}

\CommentTok{// Use full{-}width roman numerals for titles, and lowercase ukrainian letters}
\NormalTok{\#set }\FunctionTok{heading}\NormalTok{(numbering}\OperatorTok{:}\NormalTok{ numberingx}\OperatorTok{.}\FunctionTok{formatter}\NormalTok{(}
  \StringTok{"\{fullwidth{-}upper{-}roman\}.\{fullwidth{-}lower{-}roman\}.\{lower{-}ukrainian\}"}
\NormalTok{))}
\end{Highlighting}
\end{Shaded}

\subsubsection{Patterns}\label{patterns}

numberingx’s patterns are similiar to typst’s
\href{https://typst.app/docs/reference/meta/numbering/}{numbering
patterns} and use the same notion of fragments with a prefix and a final
suffix. The main difference is that it doesn’t use special characters
and all numbering styles must be written within braces. To insert a
literal brace, you can double it.

A list of patterns can be found in the
\href{https://www.w3.org/TR/predefined-counter-styles/}{Ready-made
Counter Styles} document. Additionally, numberingx allows typst’s
numbering characters to be used in patterns. This way,
\texttt{\ "\{upper-roman\}.\{decimal\})"\ } can be shortened to
\texttt{\ "\{I\}.\{1\})"\ } .

\subsubsection{API}\label{api}

numberingx exposes two functions, \texttt{\ format\ } and
\texttt{\ formatter\ } .

\paragraph{\texorpdfstring{\texttt{\ format(fmt,\ styles:\ (:),\ ..nums)\ }}{ format(fmt, styles: (:), ..nums) }}\label{formatfmt-styles-..nums}

This function uses the same api as typst’s \texttt{\ numbering()\ }
and takes the pattern string as its first positional argument, and
numbers as trailing arguments. An optional \texttt{\ styles\ } argument
allows for
\href{https://github.com/typst/packages/raw/main/packages/preview/numberingx/0.0.1/\#user-defined-styles}{user-defined
styles} .

\paragraph{\texorpdfstring{\texttt{\ formatter(fmt,\ styles:\ (:))\ }}{ formatter(fmt, styles: (:)) }}\label{formatterfmt-styles}

This function is little more than a shorter version of
\texttt{\ format.with(..)\ } . It takes a pattern string and an optional
\texttt{\ styles\ } argument, and return the matching numbering
functions. This is mainly intended to be used for \texttt{\ \#set\ }
rules.

\subsection{User-defined styles}\label{user-defined-styles}

Custom styles can be defined according to the
\href{https://www.w3.org/TR/css-counter-styles-3/}{CSS Counter Styles}
spec and passed through a \texttt{\ styles\ } named argument to
\texttt{\ format\ } and \texttt{\ formatter\ } . It must be a dictionary
mapping style names to style descriptions.

Note that the \texttt{\ prefix\ } , \texttt{\ suffix\ } ,
\texttt{\ pad\ } , and \texttt{\ speak-as\ } descriptors are not
supported, nor is the \texttt{\ extends\ } system.

\subsection{License}\label{license}

This repository is licensed under
\href{https://spdx.org/licenses/MIT-0.html}{MIT-0} , which is the
closest I’m legally allowed to public domain while being OSI approved.

\subsubsection{How to add}\label{how-to-add}

Copy this into your project and use the import as
\texttt{\ numberingx\ }

\begin{verbatim}
#import "@preview/numberingx:0.0.1"
\end{verbatim}

\includesvg[width=0.16667in,height=0.16667in]{/assets/icons/16-copy.svg}

Check the docs for
\href{https://typst.app/docs/reference/scripting/\#packages}{more
information on how to import packages} .

\subsubsection{About}\label{about}

\begin{description}
\tightlist
\item[Author :]
Edhebi
\item[License:]
MIT-0
\item[Current version:]
0.0.1
\item[Last updated:]
July 21, 2023
\item[First released:]
July 21, 2023
\item[Archive size:]
13.9 kB
\href{https://packages.typst.org/preview/numberingx-0.0.1.tar.gz}{\pandocbounded{\includesvg[keepaspectratio]{/assets/icons/16-download.svg}}}
\item[Repository:]
\href{https://github.com/edhebi/numberingx}{GitHub}
\end{description}

\subsubsection{Where to report issues?}\label{where-to-report-issues}

This package is a project of Edhebi . Report issues on
\href{https://github.com/edhebi/numberingx}{their repository} . You can
also try to ask for help with this package on the
\href{https://forum.typst.app}{Forum} .

Please report this package to the Typst team using the
\href{https://typst.app/contact}{contact form} if you believe it is a
safety hazard or infringes upon your rights.

\phantomsection\label{versions}
\subsubsection{Version history}\label{version-history}

\begin{longtable}[]{@{}ll@{}}
\toprule\noalign{}
Version & Release Date \\
\midrule\noalign{}
\endhead
\bottomrule\noalign{}
\endlastfoot
0.0.1 & July 21, 2023 \\
\end{longtable}

Typst GmbH did not create this package and cannot guarantee correct
functionality of this package or compatibility with any version of the
Typst compiler or app.


\title{typst.app/universe/package/gloss-awe}

\phantomsection\label{banner}
\section{gloss-awe}\label{gloss-awe}

{ 0.0.5 }

Awesome Glossary for Typst.

\phantomsection\label{readme}
Automatically create a glossary in \href{https://typst.app/}{typst} .

This typst component creates a glossary page from a given pool of
potential glossary entries using only those entries, that are marked
with the \texttt{\ gls\ } or \texttt{\ gls-add\ } functions in the
document.

âš~ï¸? Typst is in beta and evolving, and this package evolves with it.
As a result, no backward compatibility is guaranteed yet. Also, the
package itself is under development and fine-tuning.

\subsection{Table of Contents}\label{table-of-contents}

\begin{itemize}
\tightlist
\item
  \href{https://github.com/typst/packages/raw/main/packages/preview/gloss-awe/0.0.5/\#usage}{Usage}

  \begin{itemize}
  \tightlist
  \item
    \href{https://github.com/typst/packages/raw/main/packages/preview/gloss-awe/0.0.5/\#marking-the-entries}{Marking
    the Entries}
  \item
    \href{https://github.com/typst/packages/raw/main/packages/preview/gloss-awe/0.0.5/\#controlling-the-show}{Controlling
    the Show}
  \item
    \href{https://github.com/typst/packages/raw/main/packages/preview/gloss-awe/0.0.5/\#hiding-entries-from-the-glossary-page}{Hiding
    Entries from the Glossary Page}
  \item
    \href{https://github.com/typst/packages/raw/main/packages/preview/gloss-awe/0.0.5/\#pool-of-entries}{Pool
    of Entries}
  \item
    \href{https://github.com/typst/packages/raw/main/packages/preview/gloss-awe/0.0.5/\#unknown-entries}{Unknown
    Entries}
  \item
    \href{https://github.com/typst/packages/raw/main/packages/preview/gloss-awe/0.0.5/\#creating-the-glossary-page}{Creating
    the glossary page}
  \end{itemize}
\item
  \href{https://github.com/typst/packages/raw/main/packages/preview/gloss-awe/0.0.5/\#changelog}{Changelog}

  \begin{itemize}
  \tightlist
  \item
    \href{https://github.com/typst/packages/raw/main/packages/preview/gloss-awe/0.0.5/\#v005}{v0.0.5}
  \item
    \href{https://github.com/typst/packages/raw/main/packages/preview/gloss-awe/0.0.5/\#v004}{v0.0.4}
  \item
    \href{https://github.com/typst/packages/raw/main/packages/preview/gloss-awe/0.0.5/\#v003}{v0.0.3}
  \item
    \href{https://github.com/typst/packages/raw/main/packages/preview/gloss-awe/0.0.5/\#v002}{v0.0.2}
  \end{itemize}
\end{itemize}

\subsection{Usage}\label{usage}

\subsubsection{Marking the Entries}\label{marking-the-entries}

To include a term into the glossary, it can be marked with the
\texttt{\ gls\ } function. The simplest form is like this:

\begin{Shaded}
\begin{Highlighting}[]
\NormalTok{This is how to mark something for the glossary in \#gls[Typst].}
\end{Highlighting}
\end{Shaded}

The function will render as defined with the specified show rule (see
below!).

\subsubsection{Controlling the Show}\label{controlling-the-show}

To control, how the markers are rendered in the document, a show rule
has to be defined. It usually looks like the following:

\begin{Shaded}
\begin{Highlighting}[]
\NormalTok{// marker display : this rule makes the glossary marker in the document visible.}
\NormalTok{\#show figure.where(kind: "jkrb\_glossary"): it =\textgreater{} \{it.body\}}
\end{Highlighting}
\end{Shaded}

\subsubsection{Hiding Entries from the Glossary
Page}\label{hiding-entries-from-the-glossary-page}

It is also possible to hide entries (temporarily) from the generated
glossary page without removing any markers for them from the document.

The following sample will hide the entries for “Amaranth� and
“Butterscotch� from the glossary, even if it is marked with
\texttt{\ gls{[}...{]}\ } or \texttt{\ gls-add{[}...{]}\ } somewhere in
the document.

\begin{Shaded}
\begin{Highlighting}[]
\NormalTok{    \#let hidden{-}entries = (}
\NormalTok{        "Amaranth",}
\NormalTok{        "Butterscotch"}
\NormalTok{    )}

\NormalTok{    \#make{-}glossary(glossary{-}pool, excluded: hidden{-}entries)}
\end{Highlighting}
\end{Shaded}

\subsubsection{Pool of Entries}\label{pool-of-entries}

A “pool of entries� is a typst dictionary. It can be read from a
file or may be the result of other operations.

One or more pool(s) are then given to the \texttt{\ make-glossary()\ }
function. This will create a glossary page of all referenced entries. If
given more than one pool, the order pools are searched in the order they
are given to the method. The first match wins. This can be used to have
a global pool to be used in different documents, and another one for
local usage only.

The pool consists of a dictionary of definition entries. The key of an
entry is the term. Note, that it is case-sensitive. Each entry itself is
also a dictionary, and its main key is \texttt{\ description\ } . This
is the content for the term. There may be other keys in an entry in the
future. For now, it supports:

\begin{itemize}
\tightlist
\item
  description
\item
  link
\end{itemize}

An entry in the pool for the term “Engine� file may look like this:

\begin{Shaded}
\begin{Highlighting}[]
\NormalTok{    Engine: (}
\NormalTok{        description: [}

\NormalTok{            In the context of software, an engine...}

\NormalTok{        ],}
\NormalTok{        link: [}

\NormalTok{            (1) Software engine {-} Wikipedia.}
\NormalTok{            https://en.wikipedia.org/wiki/Software\_engine}
\NormalTok{            (13.5.2023).}

\NormalTok{        ]}
\NormalTok{    ),}
\end{Highlighting}
\end{Shaded}

\subsubsection{Unknown Entries}\label{unknown-entries}

If the document marks a term that is not contained in the pool, an entry
will be generated anyway, but it will be visually marked as missing.
This is convenient for the workflow, where one can mark the desired
entries while writing the document, and provide missing entries later.

There is a parameter for the \texttt{\ make-glossary()\ } function named
\texttt{\ missing\ } , where a function can be provided to format or
even suppress the missing entries.

\subsubsection{Creating the Glossary
Page}\label{creating-the-glossary-page}

To create the glossary page, provide the pool of potential entries to
the make-glossary function. The following listing shows a complete
sample document with a glossary. The sample glossary pool is contained
in the main document as well:

\begin{Shaded}
\begin{Highlighting}[]
\NormalTok{    \#import "@preview/gloss{-}awe:0.0.5": *}

\NormalTok{    // Text settings}
\NormalTok{    \#set text(font: ("Arial", "Trebuchet MS"), size: 12pt)}

\NormalTok{    // Defining the Glossary Pool with definitions.}
\NormalTok{    \#let glossary{-}pool = (}
\NormalTok{        Cloud: (}
\NormalTok{            description: [}

\NormalTok{                Cloud computing is a model where computer resources are made available}
\NormalTok{                over the internet. Such resources can be assigned on demand in a very short}
\NormalTok{                time, and only as long as they are required by the user.}

\NormalTok{            ]}
\NormalTok{        ),}

\NormalTok{        Marker: (}
\NormalTok{            description: [}

\NormalTok{                A Marker in \textasciigrave{}gloss{-}awe\textasciigrave{} is a typst function to mark a word or phrase to appear}
\NormalTok{                in the documents glossary. The marker is also linked to the glossary section}
\NormalTok{                by referencing the label \textasciigrave{}\textless{}Glossary\textgreater{}\textasciigrave{}.}

\NormalTok{            ]}
\NormalTok{        ),}

\NormalTok{        Glossary: (}
\NormalTok{            description: [}

\NormalTok{                A glossary is a list of terms and their definitions that are specific to a}
\NormalTok{                particular subject or field. It is used to define the intended meaning of}
\NormalTok{                terms used in a document and to agree on a common definition of those terms. A}
\NormalTok{                well{-}defined glossary can be very helpful in documents where very specific}
\NormalTok{                meanings of certain terms are used.}

\NormalTok{            ]}
\NormalTok{        ),}

\NormalTok{        "Glossary Pool": (}
\NormalTok{            description: [}

\NormalTok{                A glossary pool is a collection of glossary entries. An automated tool can}
\NormalTok{                pull needed definitions from this pool to create the glossary pages for a}
\NormalTok{                specific context.}

\NormalTok{            ]}
\NormalTok{        ),}

\NormalTok{        REST: (}
\NormalTok{            description: [}

\NormalTok{                Representational State Transfer (abgekürzt REST) ist ein Paradigma für die}
\NormalTok{                Softwarearchitektur von verteilten Systemen, insbesondere für Webservices.}
\NormalTok{                REST ist eine Abstraktion der Struktur und des Verhaltens des World Wide}
\NormalTok{                Web. REST hat das Ziel, einen Architekturstil zu schaffen, der den}
\NormalTok{                Anforderungen des modernen Web besser genügt.}

\NormalTok{            ]}
\NormalTok{        ),}

\NormalTok{        XML: (}
\NormalTok{            description: [}

\NormalTok{                XML stands for \textasciigrave{}\textquotesingle{}eXtensible Markup Language\textquotesingle{}\textasciigrave{}.}

\NormalTok{            ],}
\NormalTok{            link: [https://www.w3.org/XML]}
\NormalTok{        ),}
\NormalTok{    )}

\NormalTok{    // Defining, how marked glossary entries in the document appear}
\NormalTok{    \#show figure.where(kind: "jkrb\_glossary"): it =\textgreater{} \{it.body\}}

\NormalTok{    // This alternate rule, creates links to the glossary for marked entries.}
\NormalTok{    // \#show figure.where(kind: "jkrb\_glossary"): it =\textgreater{} [\#link(\textless{}Glossar\textgreater{})[\#it.body]]}

\NormalTok{    = My Sample Document with \textasciigrave{}gloss{-}awe\textasciigrave{}}

\NormalTok{    In this document the usage of the \textasciigrave{}gloss{-}awe\textasciigrave{} package is demonstrated to create a glossary}
\NormalTok{    with the help of a simple and small sample glossary pool. We have defined the pool in a}
\NormalTok{    dictionary named \#gls[Glossary Pool] above. It contains the definitions the \textasciigrave{}gloss{-}awe\textasciigrave{}}
\NormalTok{    can use to build the glossary in the \#gls[Glossary] section of this document. The pool}
\NormalTok{    could also come from external files, like \#gls[JSON] or \#gls[XML] or other sources. Only}
\NormalTok{    those definitions are shown in the glossary, that are marked in this document with one of}
\NormalTok{    the \#gls(entry: "Marker")[marker] functions \textasciigrave{}gloss{-}awe\textasciigrave{} provides.}

\NormalTok{    If a Word is marked, that is not in the Glossary Pool, it gets a special markup in the}
\NormalTok{    resulting glossary, making it easy for the Author to spot the missing information an}
\NormalTok{    providing a definition. In this sample, there is no definition for "JSON" provided,}
\NormalTok{    resulting in an Entry with a red remark "\#text(fill: red)[No\textasciitilde{}glossary\textasciitilde{}entry]".}

\NormalTok{    There is also a way to include Entries in the glossary for Words that are not contained in}
\NormalTok{    the documents text:}

\NormalTok{    \#gls{-}add("Cloud")}
\NormalTok{    \#gls{-}add("REST")}


\NormalTok{    = Glossary \textless{}Glossary\textgreater{}}

\NormalTok{    This section contains the generated Glossary, in a nice two{-}column{-}layout. It also bears a}
\NormalTok{    label, to enable the linking from marked words to the glossary.}

\NormalTok{    \#line(length: 100\%)}
\NormalTok{    \#set text(font: ("Arial", "Trebuchet MS"), size: 10pt)}
\NormalTok{    \#columns(2)[}
\NormalTok{        \#make{-}glossary(glossary{-}pool)}
\NormalTok{    ]}
\end{Highlighting}
\end{Shaded}

More usage samples are shown in the document
\texttt{\ sample-usage.typ\ } on
\href{https://github.com/typst/packages/raw/main/packages/preview/gloss-awe/0.0.5/\%5BTitle\%5D(https://github.com/RolfBremer/typst-glossary)}{gloss-awe´s
GitHub} .

A more complex sample PDF is available there as well.

\subsection{Changelog}\label{changelog}

\subsubsection{v0.0.5}\label{v0.0.5}

\begin{itemize}
\tightlist
\item
  Address change in \texttt{\ figure.caption\ } in typst (commit:
  976abdf ).
\end{itemize}

\subsubsection{v0.0.4}\label{v0.0.4}

\begin{itemize}
\tightlist
\item
  Breaking: Renamed the main file from \texttt{\ glossary.typ\ } to
  \texttt{\ gloss-awe.typ\ } to match package.
\item
  Added support for hidden glossary entries.
\item
  Added a Changelog to this readme.
\end{itemize}

\subsubsection{v0.0.3}\label{v0.0.3}

\begin{itemize}
\tightlist
\item
  Added support for package manager in Typst.
\item
  Add \texttt{\ gls-add{[}...{]}\ } function for entries that are not in
  the document.
\end{itemize}

\subsubsection{v.0.0.2}\label{v.0.0.2}

\begin{itemize}
\tightlist
\item
  Moved version to Github.
\end{itemize}

\subsubsection{How to add}\label{how-to-add}

Copy this into your project and use the import as \texttt{\ gloss-awe\ }

\begin{verbatim}
#import "@preview/gloss-awe:0.0.5"
\end{verbatim}

\includesvg[width=0.16667in,height=0.16667in]{/assets/icons/16-copy.svg}

Check the docs for
\href{https://typst.app/docs/reference/scripting/\#packages}{more
information on how to import packages} .

\subsubsection{About}\label{about}

\begin{description}
\tightlist
\item[Author :]
\href{https://github.com/RolfBremer}{JKRB}
\item[License:]
Apache-2.0
\item[Current version:]
0.0.5
\item[Last updated:]
September 13, 2023
\item[First released:]
July 3, 2023
\item[Archive size:]
8.39 kB
\href{https://packages.typst.org/preview/gloss-awe-0.0.5.tar.gz}{\pandocbounded{\includesvg[keepaspectratio]{/assets/icons/16-download.svg}}}
\item[Repository:]
\href{https://github.com/RolfBremer/gloss-awe}{GitHub}
\end{description}

\subsubsection{Where to report issues?}\label{where-to-report-issues}

This package is a project of JKRB . Report issues on
\href{https://github.com/RolfBremer/gloss-awe}{their repository} . You
can also try to ask for help with this package on the
\href{https://forum.typst.app}{Forum} .

Please report this package to the Typst team using the
\href{https://typst.app/contact}{contact form} if you believe it is a
safety hazard or infringes upon your rights.

\phantomsection\label{versions}
\subsubsection{Version history}\label{version-history}

\begin{longtable}[]{@{}ll@{}}
\toprule\noalign{}
Version & Release Date \\
\midrule\noalign{}
\endhead
\bottomrule\noalign{}
\endlastfoot
0.0.5 & September 13, 2023 \\
\href{https://typst.app/universe/package/gloss-awe/0.0.4/}{0.0.4} & July
6, 2023 \\
\href{https://typst.app/universe/package/gloss-awe/0.0.3/}{0.0.3} & July
3, 2023 \\
\end{longtable}

Typst GmbH did not create this package and cannot guarantee correct
functionality of this package or compatibility with any version of the
Typst compiler or app.


\title{typst.app/universe/package/rich-counters}

\phantomsection\label{banner}
\section{rich-counters}\label{rich-counters}

{ 0.2.2 }

Counters which can inherit from other counters.

\phantomsection\label{readme}
This package allows you to have \textbf{counters which can inherit from
other counters} .

Concretely, it implements \texttt{\ rich-counter\ } , which is a counter
that can \emph{inherit} one or more levels from another counter.

The interface is pretty much the same as the
\href{https://typst.app/docs/reference/introspection/counter/}{usual
counter} . It provides a \texttt{\ display()\ } , \texttt{\ get()\ } ,
\texttt{\ final()\ } , \texttt{\ at()\ } , and a \texttt{\ step()\ }
method. An \texttt{\ update()\ } method will be implemented soon.

\subsection{Simple typical Showcase}\label{simple-typical-showcase}

In the following example, \texttt{\ mycounter\ } inherits the first
level from \texttt{\ heading\ } (but not deeper levels).

\begin{Shaded}
\begin{Highlighting}[]
\NormalTok{\#import "@preview/rich{-}counters:0.2.2": *}

\NormalTok{\#set heading(numbering: "1.1")}
\NormalTok{\#let mycounter = rich{-}counter(identifier: "mycounter", inherited\_levels: 1)}

\NormalTok{// DOCUMENT}

\NormalTok{Displaying \textasciigrave{}mycounter\textasciigrave{} here: \#context (mycounter.display)()}

\NormalTok{= First level heading}

\NormalTok{Displaying \textasciigrave{}mycounter\textasciigrave{} here: \#context (mycounter.display)()}

\NormalTok{Stepping \textasciigrave{}mycounter\textasciigrave{} here. \#(mycounter.step)()}

\NormalTok{Displaying \textasciigrave{}mycounter\textasciigrave{} here: \#context (mycounter.display)()}

\NormalTok{= Another first level heading}

\NormalTok{Displaying \textasciigrave{}mycounter\textasciigrave{} here: \#context (mycounter.display)()}

\NormalTok{Stepping \textasciigrave{}mycounter\textasciigrave{} here. \#(mycounter.step)()}

\NormalTok{Displaying \textasciigrave{}mycounter\textasciigrave{} here: \#context (mycounter.display)()}

\NormalTok{== Second level heading}

\NormalTok{Displaying \textasciigrave{}mycounter\textasciigrave{} here: \#context (mycounter.display)()}

\NormalTok{Stepping \textasciigrave{}mycounter\textasciigrave{} here. \#(mycounter.step)()}

\NormalTok{Displaying \textasciigrave{}mycounter\textasciigrave{} here: \#context (mycounter.display)()}

\NormalTok{= Aaand another first level heading}

\NormalTok{Displaying \textasciigrave{}mycounter\textasciigrave{} here: \#context (mycounter.display)()}

\NormalTok{Stepping \textasciigrave{}mycounter\textasciigrave{} here. \#(mycounter.step)()}

\NormalTok{Displaying \textasciigrave{}mycounter\textasciigrave{} here: \#context (mycounter.display)()}
\end{Highlighting}
\end{Shaded}

\pandocbounded{\includegraphics[keepaspectratio]{https://github.com/typst/packages/raw/main/packages/preview/rich-counters/0.2.2/example.png}}

\subsection{\texorpdfstring{Construction of a
\texttt{\ rich-counter\ }}{Construction of a  rich-counter }}\label{construction-of-a-rich-counter}

To create a \texttt{\ rich-counter\ } , you have to call the
\texttt{\ rich-counter(...)\ } function. It accepts three arguments:

\begin{itemize}
\item
  \texttt{\ identifier\ } (required)

  Must be a unique \texttt{\ string\ } which identifies the counter.
\item
  \texttt{\ inherited\_levels\ }

  Specifies how many levels should be inherited from the parent counter.
\item
  \texttt{\ inherited\_from\ } (Default: \texttt{\ heading\ } )

  Specifies the parent counter. Can be a \texttt{\ rich-counter\ } or
  any key that is accepted by the
  \href{https://typst.app/docs/reference/introspection/counter\#constructor}{\texttt{\ counter(...)\ }
  constructor} , such as a \texttt{\ label\ } , a \texttt{\ selector\ }
  , a \texttt{\ location\ } , or a \texttt{\ function\ } like
  \texttt{\ heading\ } . If not specified, defaults to
  \texttt{\ heading\ } (and hence it will inherit from the counter of
  the headings).

  If it’s a \texttt{\ rich-counter\ } and
  \texttt{\ inherited\_levels\ } is \emph{not} specified, then
  \texttt{\ inherited\_levels\ } will default to one level higher than
  the given \texttt{\ rich-counter\ } .
\end{itemize}

For example, the following creates a \texttt{\ rich-counter\ }
\texttt{\ foo\ } which inherits one level from the headings, and then
another \texttt{\ rich-counter\ } \texttt{\ bar\ } which inherits two
levels (implicitly) from \texttt{\ foo\ } .

\begin{Shaded}
\begin{Highlighting}[]
\NormalTok{\#import "@preview/rich{-}counters:0.2.2": *}

\NormalTok{\#let foo = rich{-}counter(identifier: "foo", inherited\_levels: 1)}
\NormalTok{\#let bar = rich{-}counter(identifier: "bar", inherited\_from: foo)}
\end{Highlighting}
\end{Shaded}

\subsection{\texorpdfstring{Usage of a
\texttt{\ rich-counter\ }}{Usage of a  rich-counter }}\label{usage-of-a-rich-counter}

\begin{itemize}
\item
  \texttt{\ display(numbering)\ } (needs \texttt{\ context\ } )

  Displays the current value of the counter with the given numbering
  style. Giving the numbering style is optional, with default value
  \texttt{\ "1.1"\ } .
\item
  \texttt{\ get()\ } (needs \texttt{\ context\ } )

  Returns the current value of the counter (as an \texttt{\ array\ } ).
\item
  \texttt{\ final()\ } (needs \texttt{\ context\ } )

  Returns the value of the counter at the end of the document.
\item
  \texttt{\ at(loc)\ } (needs \texttt{\ context\ } )

  Returns the value of the counter at \texttt{\ loc\ } , where
  \texttt{\ loc\ } can be a \texttt{\ label\ } , \texttt{\ selector\ } ,
  \texttt{\ location\ } , or \texttt{\ function\ } .
\item
  \texttt{\ step(depth:\ 1)\ }

  Steps the counter at the specified \texttt{\ depth\ } (default:
  \texttt{\ 1\ } ). That is, it steps the \texttt{\ rich-counter\ } at
  level \texttt{\ inherited\_levels\ +\ depth\ } .
\end{itemize}

\textbf{Due to a Typst limitation, you have to put parentheses before
you put the arguments. (See below.)}

For example, the following steps \texttt{\ mycounter\ } (at depth 1) and
then displays it.

\begin{Shaded}
\begin{Highlighting}[]
\NormalTok{\#import "@preview/rich{-}counters:0.2.2": *}
\NormalTok{\#let mycounter = rich{-}counter(...)}

\NormalTok{\#(mycounter.step)()}
\NormalTok{\#context (mycounter.display)("1.1")}
\end{Highlighting}
\end{Shaded}

\subsection{Limitations}\label{limitations}

Due to current Typst limitations, there is no way to detect manual
updates or steps of Typst-native counters, like
\texttt{\ counter(heading).update(...)\ } or
\texttt{\ counter(heading).step(...)\ } . Only occurrences of actual
\texttt{\ heading\ } s can be detected. So make sure that after you call
e.g. \texttt{\ counter(heading).update(...)\ } , you place a heading
directly after it, before you call any \texttt{\ rich-counter\ } s.

\subsection{Roadmap}\label{roadmap}

\begin{itemize}
\tightlist
\item
  implement \texttt{\ update()\ }
\item
  use Typst custom types as soon as they become available
\item
  adopt native Typst implementation of dependent counters as soon it
  becomes available
\end{itemize}

\subsubsection{How to add}\label{how-to-add}

Copy this into your project and use the import as
\texttt{\ rich-counters\ }

\begin{verbatim}
#import "@preview/rich-counters:0.2.2"
\end{verbatim}

\includesvg[width=0.16667in,height=0.16667in]{/assets/icons/16-copy.svg}

Check the docs for
\href{https://typst.app/docs/reference/scripting/\#packages}{more
information on how to import packages} .

\subsubsection{About}\label{about}

\begin{description}
\tightlist
\item[Author :]
\href{https://jbirnick.net}{Johann Birnick}
\item[License:]
MIT
\item[Current version:]
0.2.2
\item[Last updated:]
November 21, 2024
\item[First released:]
August 14, 2024
\item[Archive size:]
3.60 kB
\href{https://packages.typst.org/preview/rich-counters-0.2.2.tar.gz}{\pandocbounded{\includesvg[keepaspectratio]{/assets/icons/16-download.svg}}}
\item[Repository:]
\href{https://github.com/jbirnick/typst-rich-counters}{GitHub}
\item[Categor ies :]
\begin{itemize}
\tightlist
\item[]
\item
  \pandocbounded{\includesvg[keepaspectratio]{/assets/icons/16-list-unordered.svg}}
  \href{https://typst.app/universe/search/?category=model}{Model}
\item
  \pandocbounded{\includesvg[keepaspectratio]{/assets/icons/16-code.svg}}
  \href{https://typst.app/universe/search/?category=scripting}{Scripting}
\item
  \pandocbounded{\includesvg[keepaspectratio]{/assets/icons/16-hammer.svg}}
  \href{https://typst.app/universe/search/?category=utility}{Utility}
\end{itemize}
\end{description}

\subsubsection{Where to report issues?}\label{where-to-report-issues}

This package is a project of Johann Birnick . Report issues on
\href{https://github.com/jbirnick/typst-rich-counters}{their repository}
. You can also try to ask for help with this package on the
\href{https://forum.typst.app}{Forum} .

Please report this package to the Typst team using the
\href{https://typst.app/contact}{contact form} if you believe it is a
safety hazard or infringes upon your rights.

\phantomsection\label{versions}
\subsubsection{Version history}\label{version-history}

\begin{longtable}[]{@{}ll@{}}
\toprule\noalign{}
Version & Release Date \\
\midrule\noalign{}
\endhead
\bottomrule\noalign{}
\endlastfoot
0.2.2 & November 21, 2024 \\
\href{https://typst.app/universe/package/rich-counters/0.2.1/}{0.2.1} &
October 16, 2024 \\
\href{https://typst.app/universe/package/rich-counters/0.2.0/}{0.2.0} &
October 14, 2024 \\
\href{https://typst.app/universe/package/rich-counters/0.1.0/}{0.1.0} &
August 14, 2024 \\
\end{longtable}

Typst GmbH did not create this package and cannot guarantee correct
functionality of this package or compatibility with any version of the
Typst compiler or app.


\title{typst.app/universe/package/nassi}

\phantomsection\label{banner}
\section{nassi}\label{nassi}

{ 0.1.2 }

Draw Nassi-Shneiderman diagrams (Struktogramme) with Typst.

\phantomsection\label{readme}
\textbf{nassi} is a package for \href{https://typst.app/}{Typst} to draw
\href{https://en.wikipedia.org/wiki/Nassi\%E2\%80\%93Shneiderman_diagram}{Nassi-Shneiderman
diagrams} (Struktogramme).

\pandocbounded{\includegraphics[keepaspectratio]{https://github.com/typst/packages/raw/main/packages/preview/nassi/0.1.2/assets/example-1.png}}

\subsection{Usage}\label{usage}

Import \textbf{nassi} in your document:

\begin{Shaded}
\begin{Highlighting}[]
\NormalTok{\#import "@preview/nassi:0.1.2"}
\end{Highlighting}
\end{Shaded}

There are several options to draw diagrams. One is to parse all
code-blocks with the language “nassi�. Simply add a show-rule like
this:

\begin{Shaded}
\begin{Highlighting}[]
\NormalTok{\#import "@preview/nassi:0.1.2"}
\NormalTok{\#show: nassi.shneiderman()}

\NormalTok{\textasciigrave{}\textasciigrave{}\textasciigrave{}nassi}
\NormalTok{function ggt(a, b)}
\NormalTok{  while a \textgreater{} 0 and b \textgreater{} 0}
\NormalTok{    if a \textgreater{} b}
\NormalTok{      a \textless{}{-} a {-} b}
\NormalTok{    else}
\NormalTok{      b \textless{}{-} b {-} a}
\NormalTok{    endif}
\NormalTok{  endwhile}
\NormalTok{  if b == 0}
\NormalTok{    return a}
\NormalTok{  else}
\NormalTok{    return b}
\NormalTok{  endif}
\NormalTok{endfunction}
\NormalTok{\textasciigrave{}\textasciigrave{}\textasciigrave{}}
\end{Highlighting}
\end{Shaded}

In this case, the diagram is created from a simple pseudocode. To have
more control over the output, you can add blocks manually using the
element functions provided in \texttt{\ nassi.elements\ } :

\begin{Shaded}
\begin{Highlighting}[]
\NormalTok{\#import "@preview/nassi:0.1.2"}

\NormalTok{\#nassi.diagram(\{}
\NormalTok{    import nassi.elements: *}

\NormalTok{    function("ggt(a, b)", \{}
\NormalTok{        loop("a \textgreater{} b and b \textgreater{} 0", \{}
\NormalTok{            branch("a \textgreater{} b", \{}
\NormalTok{                assign("a", "a {-} b")}
\NormalTok{            \}, \{}
\NormalTok{                assign("b", "b {-} a",}
\NormalTok{                    fill: gradient.linear(..color.map.rainbow),}
\NormalTok{                    stroke:red + 2pt}
\NormalTok{                )}
\NormalTok{            \})}
\NormalTok{        \})}
\NormalTok{        branch("b == 0", \{ process("return a") \}, \{ process("return b") \})}
\NormalTok{    \})}
\NormalTok{\})}
\end{Highlighting}
\end{Shaded}

\pandocbounded{\includegraphics[keepaspectratio]{https://github.com/typst/packages/raw/main/packages/preview/nassi/0.1.2/assets/example-3.png}}

Since \textbf{nassi} uses \textbf{cetz} for drawing, you can add
diagrams directly to a canvas. Each block gets a name within the diagram
group to reference it in the drawing:

\begin{Shaded}
\begin{Highlighting}[]
\NormalTok{\#import "@preview/cetz:0.2.2"}
\NormalTok{\#import "@preview/nassi:0.1.2"}

\NormalTok{\#cetz.canvas(\{}
\NormalTok{  import nassi.draw: diagram}
\NormalTok{  import nassi.elements: *}
\NormalTok{  import cetz.draw: *}

\NormalTok{  diagram((4,4), \{}
\NormalTok{    function("ggt(a, b)", \{}
\NormalTok{      loop("a \textgreater{} b and b \textgreater{} 0", \{}
\NormalTok{        branch("a \textgreater{} b", \{}
\NormalTok{          assign("a", "a {-} b")}
\NormalTok{        \}, \{}
\NormalTok{          assign("b", "b {-} a")}
\NormalTok{        \})}
\NormalTok{      \})}
\NormalTok{      branch("b == 0", \{ process("return a") \}, \{ process("return b") \})}
\NormalTok{    \})}
\NormalTok{  \})}

\NormalTok{  for i in range(8) \{}
\NormalTok{    content(}
\NormalTok{      "nassi.e" + str(i+1) + ".north{-}west",}
\NormalTok{      stroke:red,}
\NormalTok{      fill:red.transparentize(50\%),}
\NormalTok{      frame:"circle",}
\NormalTok{      padding:.05,}
\NormalTok{      anchor:"north{-}west",}
\NormalTok{      text(white, weight:"bold", "e"+str(i)),}
\NormalTok{    )}
\NormalTok{  \}}
\NormalTok{\})}
\end{Highlighting}
\end{Shaded}

\pandocbounded{\includegraphics[keepaspectratio]{https://github.com/typst/packages/raw/main/packages/preview/nassi/0.1.2/assets/example-cetz-2.png}}

This can be useful to annotate a diagram:

\pandocbounded{\includegraphics[keepaspectratio]{https://github.com/typst/packages/raw/main/packages/preview/nassi/0.1.2/assets/example-cetz.png}}

See \texttt{\ assets/\ } for usage examples.

\subsection{Changelog}\label{changelog}

\subsubsection{Version 0.1.2}\label{version-0.1.2}

\begin{itemize}
\tightlist
\item
  Fix for deprecation warnings in Typst 0.12.
\end{itemize}

\subsubsection{Version 0.1.1}\label{version-0.1.1}

\begin{itemize}
\tightlist
\item
  Fixed labels option not working for branches in other elements.
\item
  Added \texttt{\ switch\ } statements (thanks to @Geronymos).
\end{itemize}

\subsubsection{Version 0.1.0}\label{version-0.1.0}

Initial release of \textbf{nassi} .

\subsubsection{How to add}\label{how-to-add}

Copy this into your project and use the import as \texttt{\ nassi\ }

\begin{verbatim}
#import "@preview/nassi:0.1.2"
\end{verbatim}

\includesvg[width=0.16667in,height=0.16667in]{/assets/icons/16-copy.svg}

Check the docs for
\href{https://typst.app/docs/reference/scripting/\#packages}{more
information on how to import packages} .

\subsubsection{About}\label{about}

\begin{description}
\tightlist
\item[Author :]
Jonas Neugebauer
\item[License:]
MIT
\item[Current version:]
0.1.2
\item[Last updated:]
October 23, 2024
\item[First released:]
June 3, 2024
\item[Minimum Typst version:]
0.11.0
\item[Archive size:]
5.93 kB
\href{https://packages.typst.org/preview/nassi-0.1.2.tar.gz}{\pandocbounded{\includesvg[keepaspectratio]{/assets/icons/16-download.svg}}}
\item[Repository:]
\href{https://github.com/jneug/typst-nassi}{GitHub}
\item[Discipline :]
\begin{itemize}
\tightlist
\item[]
\item
  \href{https://typst.app/universe/search/?discipline=computer-science}{Computer
  Science}
\end{itemize}
\item[Categor y :]
\begin{itemize}
\tightlist
\item[]
\item
  \pandocbounded{\includesvg[keepaspectratio]{/assets/icons/16-chart.svg}}
  \href{https://typst.app/universe/search/?category=visualization}{Visualization}
\end{itemize}
\end{description}

\subsubsection{Where to report issues?}\label{where-to-report-issues}

This package is a project of Jonas Neugebauer . Report issues on
\href{https://github.com/jneug/typst-nassi}{their repository} . You can
also try to ask for help with this package on the
\href{https://forum.typst.app}{Forum} .

Please report this package to the Typst team using the
\href{https://typst.app/contact}{contact form} if you believe it is a
safety hazard or infringes upon your rights.

\phantomsection\label{versions}
\subsubsection{Version history}\label{version-history}

\begin{longtable}[]{@{}ll@{}}
\toprule\noalign{}
Version & Release Date \\
\midrule\noalign{}
\endhead
\bottomrule\noalign{}
\endlastfoot
0.1.2 & October 23, 2024 \\
\href{https://typst.app/universe/package/nassi/0.1.1/}{0.1.1} & October
2, 2024 \\
\href{https://typst.app/universe/package/nassi/0.1.0/}{0.1.0} & June 3,
2024 \\
\end{longtable}

Typst GmbH did not create this package and cannot guarantee correct
functionality of this package or compatibility with any version of the
Typst compiler or app.


\title{typst.app/universe/package/in-dexter}

\phantomsection\label{banner}
\section{in-dexter}\label{in-dexter}

{ 0.5.3 }

Hand Picked Index for Typst.

\phantomsection\label{readme}
Automatically create a handcrafted index in
\href{https://typst.app/}{typst} . This typst component allows the
automatic creation of an Index page with entries that have been manually
marked in the document by its authors. This, in times of advanced search
functionality, seems somewhat outdated, but a handcrafted index like
this allows the authors to point the reader to just the right location
in the document.

âš~ï¸? Typst is in beta and evolving, and this package evolves with it.
As a result, no backward compatibility is guaranteed yet. Also, the
package itself is under development and fine-tuning.

\subsection{Table of Contents}\label{table-of-contents}

\begin{itemize}
\tightlist
\item
  \href{https://github.com/typst/packages/raw/main/packages/preview/in-dexter/0.5.3/\#usage}{Usage}

  \begin{itemize}
  \tightlist
  \item
    \href{https://github.com/typst/packages/raw/main/packages/preview/in-dexter/0.5.3/\#importing-the-component}{Importing
    the Component}
  \item
    \href{https://github.com/typst/packages/raw/main/packages/preview/in-dexter/0.5.3/\#remarks-for-new-version}{Remarks
    for new version}
  \item
    \href{https://github.com/typst/packages/raw/main/packages/preview/in-dexter/0.5.3/\#marking-entries}{Marking
    Entries}

    \begin{itemize}
    \tightlist
    \item
      \href{https://github.com/typst/packages/raw/main/packages/preview/in-dexter/0.5.3/\#generating-the-index-page}{Generating
      the index page}
    \item
      \href{https://github.com/typst/packages/raw/main/packages/preview/in-dexter/0.5.3/\#brief-sample-document}{Brief
      Sample Document}
    \item
      \href{https://github.com/typst/packages/raw/main/packages/preview/in-dexter/0.5.3/\#full-sample-document}{Full
      Sample Document}
    \end{itemize}
  \end{itemize}
\item
  \href{https://github.com/typst/packages/raw/main/packages/preview/in-dexter/0.5.3/\#changelog}{Changelog}

  \begin{itemize}
  \tightlist
  \item
    \href{https://github.com/typst/packages/raw/main/packages/preview/in-dexter/0.5.3/\#v053}{v0.5.3}
  \item
    \href{https://github.com/typst/packages/raw/main/packages/preview/in-dexter/0.5.3/\#v052}{v0.5.2}
  \item
    \href{https://github.com/typst/packages/raw/main/packages/preview/in-dexter/0.5.3/\#v051}{v0.5.1}
  \item
    \href{https://github.com/typst/packages/raw/main/packages/preview/in-dexter/0.5.3/\#v050}{v0.5.0}
  \item
    \href{https://github.com/typst/packages/raw/main/packages/preview/in-dexter/0.5.3/\#v043}{v0.4.3}
  \item
    \href{https://github.com/typst/packages/raw/main/packages/preview/in-dexter/0.5.3/\#v042}{v0.4.2}
  \item
    \href{https://github.com/typst/packages/raw/main/packages/preview/in-dexter/0.5.3/\#v041}{v0.4.1}
  \item
    \href{https://github.com/typst/packages/raw/main/packages/preview/in-dexter/0.5.3/\#v040}{v0.4.0}
  \item
    \href{https://github.com/typst/packages/raw/main/packages/preview/in-dexter/0.5.3/\#v032}{v0.3.2}
  \item
    \href{https://github.com/typst/packages/raw/main/packages/preview/in-dexter/0.5.3/\#v031}{v0.3.1}
  \item
    \href{https://github.com/typst/packages/raw/main/packages/preview/in-dexter/0.5.3/\#v030}{v0.3.0}
  \item
    \href{https://github.com/typst/packages/raw/main/packages/preview/in-dexter/0.5.3/\#v020}{v0.2.0}
  \item
    \href{https://github.com/typst/packages/raw/main/packages/preview/in-dexter/0.5.3/\#v010}{v0.1.0}
  \item
    \href{https://github.com/typst/packages/raw/main/packages/preview/in-dexter/0.5.3/\#v006}{v0.0.6}
  \item
    \href{https://github.com/typst/packages/raw/main/packages/preview/in-dexter/0.5.3/\#v005}{v0.0.5}
  \item
    \href{https://github.com/typst/packages/raw/main/packages/preview/in-dexter/0.5.3/\#v004}{v0.0.4}
  \item
    \href{https://github.com/typst/packages/raw/main/packages/preview/in-dexter/0.5.3/\#v003}{v0.0.3}
  \item
    \href{https://github.com/typst/packages/raw/main/packages/preview/in-dexter/0.5.3/\#v002}{v0.0.2}
  \end{itemize}
\end{itemize}

\subsection{Usage}\label{usage}

\subsection{Importing the Component}\label{importing-the-component}

To use the index functionality, the component must be available. This
can be achieved by importing the package \texttt{\ in-dexter\ } into the
project:

Add the following code to the head of the document file(s) that want to
use the index:

\begin{Shaded}
\begin{Highlighting}[]
\NormalTok{  \#import "@preview/in{-}dexter:0.5.3": *}
\end{Highlighting}
\end{Shaded}

Alternatively it can be loaded from the file, if you have it copied into
your project.

\begin{Shaded}
\begin{Highlighting}[]
\NormalTok{  \#import "in{-}dexter.typ": *}
\end{Highlighting}
\end{Shaded}

\subsection{Remarks for new version}\label{remarks-for-new-version}

In previous versions (before 0.0.6) of in-dexter, it was required to
hide the index entries with a show rule. This is not required anymore.

\subsection{Marking Entries}\label{marking-entries}

To mark a word to be included in the index, a simple function can be
used. In the following sample code, the word “elit� is marked to be
included into the index.

\begin{Shaded}
\begin{Highlighting}[]
\NormalTok{= Sample Text}
\NormalTok{Lorem ipsum dolor sit amet, consectetur adipiscing \#index[elit], sed do eiusmod tempor}
\NormalTok{incididunt ut labore et dolore.}
\end{Highlighting}
\end{Shaded}

Nested entries can be created - the following would create an entry
\texttt{\ adipiscing\ } with sub entry \texttt{\ elit\ } .

\begin{Shaded}
\begin{Highlighting}[]
\NormalTok{= Sample Text}
\NormalTok{Lorem ipsum dolor sit amet, consectetur adipiscing elit\#index("adipiscing", "elit"), sed do eiusmod}
\NormalTok{tempor incididunt ut labore et dolore.}
\end{Highlighting}
\end{Shaded}

The marking, by default, is invisible in the resulting text, while the
marked word will still be visible. With the marking in place, the index
component knows about the word, as well as its location in the document.

\subsection{Generating the Index Page}\label{generating-the-index-page}

The index page can be generated by the following function:

\begin{Shaded}
\begin{Highlighting}[]
\NormalTok{= Index}
\NormalTok{\#columns(3)[}
\NormalTok{  \#make{-}index(title: none)}
\NormalTok{]}
\end{Highlighting}
\end{Shaded}

This sample uses the optional title, outline, and use-page-counter
parameters:

\begin{Shaded}
\begin{Highlighting}[]
\NormalTok{\#make{-}index(title: [Index], outlined: true, use{-}page{-}counter: true)}
\end{Highlighting}
\end{Shaded}

The \texttt{\ make-index()\ } function takes three optional arguments:
\texttt{\ title\ } , \texttt{\ outlined\ } , and
\texttt{\ use-page-counter\ } .

\begin{itemize}
\tightlist
\item
  \texttt{\ title\ } adds a title (with \texttt{\ heading\ } ) and
\item
  \texttt{\ outlined\ } is \texttt{\ false\ } by default and is passed
  to the heading function
\item
  \texttt{\ use-page-counter\ } is \texttt{\ false\ } by default. If set
  to \texttt{\ true\ } it will use \texttt{\ counter(page).display()\ }
  for the page number text in the index instead of the absolute page
  position (the absolute position is still used for the actual link
  target)
\end{itemize}

If no title is given the heading should never appear in the layout.
Note: The heading is (currently) not numbered.

The first sample emits the index in three columns. Note: The actual
appearance depends on your template or other settings of your document.

You can find a preview image of the resulting page on
\href{https://github.com/RolfBremer/in-dexter}{in-dexter´s GitHub
repository} .

You may have noticed that some page numbers are displayed as bold. These
are index entries which are marked as “main� entries. Such entries
are meant to be the most important for the given entry. They can be
marked as follows:

\begin{Shaded}
\begin{Highlighting}[]
\NormalTok{\#index(fmt: strong, [Willkommen])}
\end{Highlighting}
\end{Shaded}

or you can use the predefined semantically helper function

\begin{Shaded}
\begin{Highlighting}[]
\NormalTok{\#index{-}main[Willkommen]}
\end{Highlighting}
\end{Shaded}

\subsubsection{Brief Sample Document}\label{brief-sample-document}

This is a very brief sample to demonstrate how in-dexter can be used.
The next chapter contains a more fleshed out sample.

\begin{Shaded}
\begin{Highlighting}[]
\NormalTok{\#import "@preview/in{-}dexter:0.5.3": *}


\NormalTok{= My Sample Document with \textasciigrave{}in{-}dexter\textasciigrave{}}

\NormalTok{In this document the usage of the \textasciigrave{}in{-}dexter\textasciigrave{} package is demonstrated to create}
\NormalTok{a hand picked \#index[Hand Picked] index. This sample \#index{-}main[Sample]}
\NormalTok{document \#index[Document] is quite short, and so is its index.}


\NormalTok{= Index}

\NormalTok{This section contains the generated Index.}

\NormalTok{\#make{-}index()}
\end{Highlighting}
\end{Shaded}

\subsubsection{Full Sample Document}\label{full-sample-document}

\begin{Shaded}
\begin{Highlighting}[]
\NormalTok{\#import "@preview/in{-}dexter:0.5.3": *}

\NormalTok{\#let index{-}main(..args) = index(fmt: strong, ..args)}

\NormalTok{// Document settings}
\NormalTok{\#set page("a5")}
\NormalTok{\#set text(font: ("Arial", "Trebuchet MS"), size: 12pt)}


\NormalTok{= My Sample Document with \textasciigrave{}in{-}dexter\textasciigrave{}}

\NormalTok{In this document \#index[Document] the usage of the \textasciigrave{}in{-}dexter\textasciigrave{} package \#index[Package]}
\NormalTok{is demonstrated to create a hand picked index \#index{-}main[Index]. This sample document}
\NormalTok{is quite short, and so is its index. So to fill this sample with some real text,}
\NormalTok{let´s elaborate on some aspects of a hand picked \#index[Hand Picked] index. So, "hand}
\NormalTok{picked" means, the entries \#index[Entries] in the index are carefully chosen by the}
\NormalTok{author(s) of the document to point the reader, who is interested in a specific topic}
\NormalTok{within the documents domain \#index[Domain], to the right spot \#index[Spot]. Thats, how}
\NormalTok{it should be; and it is quite different to what is done in this sample text, where the}
\NormalTok{objective \#index{-}main[Objective] was to put many different index markers}
\NormalTok{\#index[Markers] into a small text, because a sample should be as brief as possible,}
\NormalTok{while providing enough substance \#index[Substance] to demo the subject}
\NormalTok{\#index[Subject]. The resulting index in this demo is somewhat pointless}
\NormalTok{\#index[Pointless], because all entries are pointing to few different pages}
\NormalTok{\#index[Pages], due to the fact that the demo text only has few pages \#index[Page].}
\NormalTok{That is also the reason for what we chose the DIN A5 \#index[DIN A5] format, and we}
\NormalTok{also continue with some remarks \#index[Remarks] on the next page.}


\NormalTok{== Some more demo content without deeper meaning}

\NormalTok{\#lorem(50) \#index[Lorem]}

\NormalTok{\#pagebreak()}

\NormalTok{== Remarks}

\NormalTok{Here are some more remarks \#index{-}main[Remarks] to have some content on a second page, what}
\NormalTok{is a precondition \#index[Precondition] to demo that Index \#index[Index] entries}
\NormalTok{\#index[Entries] may point to multiple pages.}


\NormalTok{= Index}

\NormalTok{This section \#index[Section] contains the generated Index \#index[Index], in a nice}
\NormalTok{two{-}column{-}layout.}

\NormalTok{\#set text(size: 10pt)}
\NormalTok{\#columns(2)[}
\NormalTok{    \#make{-}index()}
\NormalTok{]}
\end{Highlighting}
\end{Shaded}

The following image shows a generated index page of another document,
with additional formatting for headers applied.

\pandocbounded{\includegraphics[keepaspectratio]{https://github.com/typst/packages/raw/main/packages/preview/in-dexter/0.5.3/gallery/SampleIndex.png}}

More usage samples are shown in the document
\texttt{\ sample-usage.typ\ } on
\href{https://github.com/RolfBremer/in-dexter}{in-dexter´s GitHub} .

A more complex sample PDF is available there as well.

\subsection{Changelog}\label{changelog}

\subsubsection{v0.5.3}\label{v0.5.3}

\begin{itemize}
\tightlist
\item
  fix error in typst.toml file.
\item
  Add a sample for raw display.
\end{itemize}

\subsubsection{v0.5.2}\label{v0.5.2}

\begin{itemize}
\tightlist
\item
  Fix a bug with bang notation.
\item
  Add compiler to toml file.
\end{itemize}

\subsubsection{v0.5.1}\label{v0.5.1}

\begin{itemize}
\tightlist
\item
  Migrate deprecated locate to context.
\end{itemize}

\subsubsection{v0.5.0}\label{v0.5.0}

\begin{itemize}
\tightlist
\item
  Support page numbering formats (i.e. roman), when
  \texttt{\ use-page-counter\ } is set to true. Thanks to
  @ThePuzzlemaker!
\end{itemize}

\subsubsection{v0.4.3}\label{v0.4.3}

\begin{itemize}
\tightlist
\item
  Suppress extra space character emitted by the \texttt{\ index()\ }
  function.
\item
  Fix a bug where math formulas are not displayed.
\item
  Introduce \texttt{\ apply-casing\ } parameter to \texttt{\ index()\ }
  to suppress entry-casing for individual entries.
\end{itemize}

\subsubsection{v0.4.2}\label{v0.4.2}

\begin{itemize}
\tightlist
\item
  Improve internal method \texttt{\ as-text\ } to be more robust.
\item
  tidy up sample-usage.typ.
\end{itemize}

\subsubsection{v0.4.1}\label{v0.4.1}

\begin{itemize}
\tightlist
\item
  Bug fixed: Fix a bug where an index entry with same name as a group
  hides the group.
\item
  Fixed typos in the sample-usage document.
\end{itemize}

\subsubsection{v0.4.0}\label{v0.4.0}

\begin{itemize}
\tightlist
\item
  Support for a \texttt{\ display\ } parameter for entries. This allows
  the usage of complex content, like math expressions in the index.
  (based on suggestions by @lukasjuhrich)
\item
  Also support a tuple value for display and key parameters of the
  entry.
\item
  Improve internal robustness and fix some errors in the sample
  document.
\end{itemize}

\subsubsection{v0.3.2}\label{v0.3.2}

\begin{itemize}
\tightlist
\item
  Fix initial parsing and returning fist letter (thanks to
  @lukasjuhrich, \#14)
\end{itemize}

\subsubsection{v0.3.1}\label{v0.3.1}

\begin{itemize}
\tightlist
\item
  Fix handling of trailing or multiple spaces and crlf in index entries.
\end{itemize}

\subsubsection{v0.3.0}\label{v0.3.0}

\begin{itemize}
\tightlist
\item
  Support multiple named indexes. Also allow the generation of combined
  index pages.
\item
  Support for LaTeX index group syntax (
  \texttt{\ \#index("Group1!Group2@Entry"\ } ).
\item
  Support for advanced case handling for the entries in the index. Note:
  The new default ist to ignore the casing for the sorting of the
  entries. The behavior can be changed by providing a
  \texttt{\ sort-order()\ } function to the \texttt{\ make-index\ }
  function.
\item
  The casing for the index entry can also be altered by providing a
  \texttt{\ entry-casing()\ } function to the \texttt{\ make-index\ }
  function. So it is possible that all entries have an uppercase first
  letter (which is also the new default!).
\end{itemize}

\subsubsection{v0.2.0}\label{v0.2.0}

\begin{itemize}
\tightlist
\item
  Allow index to respect unnumbered physical pages at the start of the
  document (Thanks to @jewelpit). See “Skipping physical pages� in
  the sample-usage document.
\end{itemize}

\subsubsection{v0.1.0}\label{v0.1.0}

\begin{itemize}
\tightlist
\item
  big refactor (by @epsilonhalbe).
\item
  changing “marker classes� to support direct format function
  \texttt{\ fmt:\ content\ -\textgreater{}\ content\ } e.g.
  \texttt{\ index(fmt:\ strong,\ {[}entry{]})\ } .
\item
  Implemented:

  \begin{itemize}
  \tightlist
  \item
    nested entries.
  \item
    custom initials + custom sorting.
  \end{itemize}
\end{itemize}

\subsubsection{v0.0.6}\label{v0.0.6}

\begin{itemize}
\tightlist
\item
  Change internal index marker to use metadata instead of figures. This
  allows a cleaner implementation and does not require a show rule to
  hide the marker-figure anymore.
\item
  This version requires Typst 0.8.0 due to the use of metadata().
\item
  Consolidated the \texttt{\ PackageReadme.md\ } into a single
  \texttt{\ README.md\ } .
\end{itemize}

\subsubsection{v0.0.5}\label{v0.0.5}

\begin{itemize}
\tightlist
\item
  Address change in \texttt{\ figure.caption\ } in typst (commit:
  976abdf ).
\end{itemize}

\subsubsection{v0.0.4}\label{v0.0.4}

\begin{itemize}
\tightlist
\item
  Add title and outline arguments to \#make-index() by @sbatial in \#4
\end{itemize}

\subsubsection{v0.0.3}\label{v0.0.3}

\begin{itemize}
\tightlist
\item
  Breaking: Renamed the main file from \texttt{\ index.typ\ } to
  \texttt{\ in-dexter.typ\ } to match package.
\item
  Added a Changelog to this README.
\item
  Introduced a brief and a full sample code to this README.
\item
  Added support for package manager in Typst.
\end{itemize}

\subsubsection{v.0.0.2}\label{v.0.0.2}

\begin{itemize}
\tightlist
\item
  Moved version to GitHub.
\end{itemize}

\subsubsection{How to add}\label{how-to-add}

Copy this into your project and use the import as \texttt{\ in-dexter\ }

\begin{verbatim}
#import "@preview/in-dexter:0.5.3"
\end{verbatim}

\includesvg[width=0.16667in,height=0.16667in]{/assets/icons/16-copy.svg}

Check the docs for
\href{https://typst.app/docs/reference/scripting/\#packages}{more
information on how to import packages} .

\subsubsection{About}\label{about}

\begin{description}
\tightlist
\item[Author s :]
\href{https://github.com/RolfBremer}{JKRB} \& in-dexter Contributors
\item[License:]
Apache-2.0
\item[Current version:]
0.5.3
\item[Last updated:]
August 14, 2024
\item[First released:]
July 10, 2023
\item[Minimum Typst version:]
0.11.0
\item[Archive size:]
11.3 kB
\href{https://packages.typst.org/preview/in-dexter-0.5.3.tar.gz}{\pandocbounded{\includesvg[keepaspectratio]{/assets/icons/16-download.svg}}}
\item[Repository:]
\href{https://github.com/RolfBremer/in-dexter}{GitHub}
\item[Categor y :]
\begin{itemize}
\tightlist
\item[]
\item
  \pandocbounded{\includesvg[keepaspectratio]{/assets/icons/16-package.svg}}
  \href{https://typst.app/universe/search/?category=components}{Components}
\end{itemize}
\end{description}

\subsubsection{Where to report issues?}\label{where-to-report-issues}

This package is a project of JKRB and in-dexter Contributors . Report
issues on \href{https://github.com/RolfBremer/in-dexter}{their
repository} . You can also try to ask for help with this package on the
\href{https://forum.typst.app}{Forum} .

Please report this package to the Typst team using the
\href{https://typst.app/contact}{contact form} if you believe it is a
safety hazard or infringes upon your rights.

\phantomsection\label{versions}
\subsubsection{Version history}\label{version-history}

\begin{longtable}[]{@{}ll@{}}
\toprule\noalign{}
Version & Release Date \\
\midrule\noalign{}
\endhead
\bottomrule\noalign{}
\endlastfoot
0.5.3 & August 14, 2024 \\
\href{https://typst.app/universe/package/in-dexter/0.5.2/}{0.5.2} &
August 23, 2024 \\
\href{https://typst.app/universe/package/in-dexter/0.4.2/}{0.4.2} & June
14, 2024 \\
\href{https://typst.app/universe/package/in-dexter/0.3.0/}{0.3.0} & May
13, 2024 \\
\href{https://typst.app/universe/package/in-dexter/0.2.0/}{0.2.0} &
April 30, 2024 \\
\href{https://typst.app/universe/package/in-dexter/0.1.0/}{0.1.0} &
January 8, 2024 \\
\href{https://typst.app/universe/package/in-dexter/0.0.6/}{0.0.6} &
October 1, 2023 \\
\href{https://typst.app/universe/package/in-dexter/0.0.5/}{0.0.5} &
September 13, 2023 \\
\href{https://typst.app/universe/package/in-dexter/0.0.4/}{0.0.4} &
August 6, 2023 \\
\href{https://typst.app/universe/package/in-dexter/0.0.3/}{0.0.3} & July
10, 2023 \\
\end{longtable}

Typst GmbH did not create this package and cannot guarantee correct
functionality of this package or compatibility with any version of the
Typst compiler or app.


\title{typst.app/universe/package/droplet}

\phantomsection\label{banner}
\section{droplet}\label{droplet}

{ 0.3.1 }

Customizable dropped capitals.

\phantomsection\label{readme}
A package for creating dropped capitals.

\subsection{Usage}\label{usage}

This package exports a single \texttt{\ dropcap\ } function that is used
to create dropped capitals. The function takes one or two positional
arguments, and several optional keyword arguments for customization:

\begin{longtable}[]{@{}llll@{}}
\toprule\noalign{}
Parameter & Type & Description & Default \\
\midrule\noalign{}
\endhead
\bottomrule\noalign{}
\endlastfoot
\texttt{\ height\ } & \texttt{\ integer\ } , \texttt{\ length\ } ,
\texttt{\ auto\ } & The height of the dropped capital. &
\texttt{\ 2\ } \\
\texttt{\ justify\ } & \texttt{\ boolean\ } , \texttt{\ auto\ } &
Whether the text should be justified. & \texttt{\ auto\ } \\
\texttt{\ gap\ } & \texttt{\ length\ } & The space between the dropped
capital and the text. & \texttt{\ 0pt\ } \\
\texttt{\ hanging-indent\ } & \texttt{\ length\ } & The indent of lines
after the first. & \texttt{\ 0pt\ } \\
\texttt{\ overhang\ } & \texttt{\ length\ } , \texttt{\ relative\ } ,
\texttt{\ ratio\ } & How much the dropped capital should hang into the
margin. & \texttt{\ 0pt\ } \\
\texttt{\ depth\ } & \texttt{\ integer\ } , \texttt{\ length\ } & The
space below the dropped capital. & \texttt{\ 0pt\ } \\
\texttt{\ transform\ } & \texttt{\ function\ } , \texttt{\ none\ } & A
function to be applied to the dropped capital. & \texttt{\ none\ } \\
\texttt{\ ..text-args\ } & & How to style the \texttt{\ text\ } of the
dropped capital. & \\
\end{longtable}

Some parameters allow values of different types for maximum flexibility:

\begin{itemize}
\tightlist
\item
  If \texttt{\ height\ } is given as an integer, it is interpreted as a
  number of lines. If given as \texttt{\ auto\ } , the dropped capital
  will not be scaled and remain at its original size.
\item
  If \texttt{\ overhang\ } has a relative part, it is resolved relative
  to the width of the dropped capital.
\item
  If \texttt{\ depth\ } is given as an integer, it is interpreted as a
  number of lines.
\item
  The \texttt{\ transform\ } function takes the extracted or passed
  dropped capital and returns the content to be shown.
\end{itemize}

If two positional arguments are given, the first is used as the dropped
capital, and the second as the text. If only one argument is given, the
dropped capital is automatically extracted from the text.

\subsubsection{Extraction}\label{extraction}

If no explicit dropped capital is passed, it is extracted automatically.
For this to work, the package looks into the content making up the first
paragraph and extracts the first letter of the first word. This letter
is then split off from the rest of the text and used as the dropped
capital. There are some special cases to consider:

\begin{itemize}
\tightlist
\item
  If the first element of the paragraph is a \texttt{\ box\ } , the
  whole box is used as the dropped capital.
\item
  If the first element is a list or enum item, it is assumed that the
  literal meaning of the list or enum syntax was intended, and the
  number or bullet is used as the dropped capital.
\end{itemize}

Affixes, such as punctuation, super- and subscripts, quotes, and spaces
will also be detected and stay with the dropped capital.

\subsubsection{Paragraph Splitting}\label{paragraph-splitting}

To wrap the text around the dropped capital, the paragraph is split into
two parts: the part next to the dropped capital and the part after it.
As Typst doesn’t natively support wrapping text around an element,
this package splits the paragraph at word boundaries and tries to fit as
much in the first part as possible. This approach comes with some
limitations:

\begin{itemize}
\tightlist
\item
  The paragraph is split at word boundaries, which makes hyphenation
  across the split impossible.
\item
  Some elements cannot be properly split, such as containers, lists, and
  context expressions.
\item
  The approach uses a greedy algorithm, which might not always find the
  optimal split.
\item
  If the split happens at a block element, the spacing between the two
  parts might be off.
\end{itemize}

To determine whether an elements fits into the first part, the position
of top edge of the element is crucial. If the top edge is above the
baseline of the dropped capital, the element is considered to be part of
the first part. This means that elements with a large height will be
part of the first part. This is done to avoid gaps between the two parts
of the paragraph.

\subsubsection{Styling}\label{styling}

In case you wish to style the dropped capital more than what is possible
with the arguments of the \texttt{\ text\ } function, you can use a
\texttt{\ transform\ } function. This function takes the extracted or
passed dropped capital and returns the content to be shown. The function
is provided with the context of the dropped capital.

Note that when using \texttt{\ em\ } units, they are resolved relative
to the font size of the dropped capital. When the dropped capital is
scaled to fit the given \texttt{\ height\ } parameter, the font size is
adjusted so that the \emph{bounds} of the transformed content match the
given height. For that, the \texttt{\ top-edge\ } and
\texttt{\ bottom-edge\ } parameters of \texttt{\ text-args\ } are set to
\texttt{\ bounds\ } by default.

\subsection{Example}\label{example}

\begin{Shaded}
\begin{Highlighting}[]
\NormalTok{\#import "@preview/droplet:0.3.1": dropcap}

\NormalTok{\#set par(justify: true)}

\NormalTok{\#dropcap(}
\NormalTok{  height: 3,}
\NormalTok{  gap: 4pt,}
\NormalTok{  hanging{-}indent: 1em,}
\NormalTok{  overhang: 8pt,}
\NormalTok{  font: "Curlz MT",}
\NormalTok{)[}
\NormalTok{  *Typst* is a new markup{-}based typesetting system that is designed to be as}
\NormalTok{  \_powerful\_ as LaTeX while being \_much easier\_ to learn and use. Typst has:}

\NormalTok{  {-} Built{-}in markup for the most common formatting tasks}
\NormalTok{  {-} Flexible functions for everything else}
\NormalTok{  {-} A tightly integrated scripting system}
\NormalTok{  {-} Math typesetting, bibliography management, and more}
\NormalTok{  {-} Fast compile times thanks to incremental compilation}
\NormalTok{  {-} Friendly error messages in case something goes wrong}
\NormalTok{]}
\end{Highlighting}
\end{Shaded}

\pandocbounded{\includesvg[keepaspectratio]{https://github.com/typst/packages/raw/main/packages/preview/droplet/0.3.1/assets/example.svg}}

\subsubsection{How to add}\label{how-to-add}

Copy this into your project and use the import as \texttt{\ droplet\ }

\begin{verbatim}
#import "@preview/droplet:0.3.1"
\end{verbatim}

\includesvg[width=0.16667in,height=0.16667in]{/assets/icons/16-copy.svg}

Check the docs for
\href{https://typst.app/docs/reference/scripting/\#packages}{more
information on how to import packages} .

\subsubsection{About}\label{about}

\begin{description}
\tightlist
\item[Author :]
Eric Biedert
\item[License:]
MIT
\item[Current version:]
0.3.1
\item[Last updated:]
November 21, 2024
\item[First released:]
July 5, 2024
\item[Minimum Typst version:]
0.11.0
\item[Archive size:]
7.82 kB
\href{https://packages.typst.org/preview/droplet-0.3.1.tar.gz}{\pandocbounded{\includesvg[keepaspectratio]{/assets/icons/16-download.svg}}}
\item[Repository:]
\href{https://github.com/EpicEricEE/typst-droplet}{GitHub}
\item[Categor y :]
\begin{itemize}
\tightlist
\item[]
\item
  \pandocbounded{\includesvg[keepaspectratio]{/assets/icons/16-text.svg}}
  \href{https://typst.app/universe/search/?category=text}{Text}
\end{itemize}
\end{description}

\subsubsection{Where to report issues?}\label{where-to-report-issues}

This package is a project of Eric Biedert . Report issues on
\href{https://github.com/EpicEricEE/typst-droplet}{their repository} .
You can also try to ask for help with this package on the
\href{https://forum.typst.app}{Forum} .

Please report this package to the Typst team using the
\href{https://typst.app/contact}{contact form} if you believe it is a
safety hazard or infringes upon your rights.

\phantomsection\label{versions}
\subsubsection{Version history}\label{version-history}

\begin{longtable}[]{@{}ll@{}}
\toprule\noalign{}
Version & Release Date \\
\midrule\noalign{}
\endhead
\bottomrule\noalign{}
\endlastfoot
0.3.1 & November 21, 2024 \\
\href{https://typst.app/universe/package/droplet/0.3.0/}{0.3.0} &
October 24, 2024 \\
\href{https://typst.app/universe/package/droplet/0.2.0/}{0.2.0} & July
5, 2024 \\
\href{https://typst.app/universe/package/droplet/0.1.0/}{0.1.0} & July
5, 2024 \\
\end{longtable}

Typst GmbH did not create this package and cannot guarantee correct
functionality of this package or compatibility with any version of the
Typst compiler or app.


\title{typst.app/universe/package/vartable}

\phantomsection\label{banner}
\section{vartable}\label{vartable}

{ 0.1.2 }

A simple package to make variation table

\phantomsection\label{readme}
An easy way to render variation table on typst, built on
\href{https://github.com/Jollywatt/typst-fletcher}{fletcher}\\
The
\href{https://github.com/Le-foucheur/Typst-VarTable/blob/main/documentation.pdf}{documention}

\begin{Shaded}
\begin{Highlighting}[]
\NormalTok{\#import "@preview/Tabvar:0.1.0": tabvar}
\end{Highlighting}
\end{Shaded}

\subsubsection{Trigonometric functions}\label{trigonometric-functions}

Turn this :

\begin{Shaded}
\begin{Highlighting}[]
\NormalTok{\#import }\StringTok{"@preview/Tabvar:0.1.0"}\OperatorTok{:}\NormalTok{ tabvar}

\NormalTok{\#}\FunctionTok{tabvar}\NormalTok{(}
\NormalTok{  init}\OperatorTok{:}\NormalTok{ (}
\NormalTok{    variable}\OperatorTok{:}\NormalTok{ $x$}\OperatorTok{,}
\NormalTok{    label}\OperatorTok{:}\NormalTok{ (}
\NormalTok{      ([sign }\KeywordTok{of}\NormalTok{ cos]}\OperatorTok{,} \StringTok{"Sign"}\NormalTok{)}\OperatorTok{,}
\NormalTok{      ([variation }\KeywordTok{of}\NormalTok{ cos]}\OperatorTok{,} \StringTok{"Variation"}\NormalTok{)}\OperatorTok{,}
\NormalTok{      ([sign }\KeywordTok{of}\NormalTok{ sin]}\OperatorTok{,} \StringTok{"Sign"}\NormalTok{)}\OperatorTok{,}
\NormalTok{      ([variation }\KeywordTok{of}\NormalTok{ sin]}\OperatorTok{,} \StringTok{"Variation"}\NormalTok{)}\OperatorTok{,}
\NormalTok{    )}\OperatorTok{,}
\NormalTok{  )}\OperatorTok{,}
\NormalTok{  domain}\OperatorTok{:}\NormalTok{ ($0$}\OperatorTok{,}\NormalTok{ $ pi }\OperatorTok{/} \DecValTok{2}\NormalTok{ $}\OperatorTok{,}\NormalTok{ $ pi $}\OperatorTok{,} \FunctionTok{$}\NormalTok{ (}\DecValTok{2}\ErrorTok{pi}\NormalTok{) }\OperatorTok{/} \DecValTok{3}\NormalTok{ $}\OperatorTok{,}\NormalTok{ $ }\DecValTok{2}\NormalTok{ pi $)}\OperatorTok{,}
\NormalTok{  content}\OperatorTok{:}\NormalTok{ (}
\NormalTok{    ($}\OperatorTok{{-}}\NormalTok{$}\OperatorTok{,}\NormalTok{ ()}\OperatorTok{,}\NormalTok{ $}\OperatorTok{+}\NormalTok{$}\OperatorTok{,}\NormalTok{ ())}\OperatorTok{,}
\NormalTok{    (}
\NormalTok{      (top}\OperatorTok{,}\NormalTok{ $1$)}\OperatorTok{,}
\NormalTok{      ()}\OperatorTok{,}
\NormalTok{      (bottom}\OperatorTok{,}\NormalTok{ $}\OperatorTok{{-}}\DecValTok{1}\ErrorTok{$}\NormalTok{)}\OperatorTok{,}
\NormalTok{      ()}\OperatorTok{,}
\NormalTok{      (top}\OperatorTok{,}\NormalTok{ $1$)}\OperatorTok{,}
\NormalTok{    )}\OperatorTok{,}
\NormalTok{    ($}\OperatorTok{+}\NormalTok{$}\OperatorTok{,}\NormalTok{ $}\OperatorTok{{-}}\NormalTok{$}\OperatorTok{,}\NormalTok{ ()}\OperatorTok{,}\NormalTok{ $}\OperatorTok{+}\NormalTok{$)}\OperatorTok{,}
\NormalTok{    (}
\NormalTok{      (center}\OperatorTok{,}\NormalTok{ $0$)}\OperatorTok{,}
\NormalTok{      (top}\OperatorTok{,}\NormalTok{ $1$)}\OperatorTok{,}
\NormalTok{      ()}\OperatorTok{,}
\NormalTok{      (bottom}\OperatorTok{,}\NormalTok{ $}\OperatorTok{{-}}\DecValTok{1}\ErrorTok{$}\NormalTok{)}\OperatorTok{,}
\NormalTok{      (top}\OperatorTok{,}\NormalTok{ $1$)}\OperatorTok{,}
\NormalTok{    )}\OperatorTok{,}
\NormalTok{  )}\OperatorTok{,}
\NormalTok{)}
\end{Highlighting}
\end{Shaded}

Into this

\pandocbounded{\includegraphics[keepaspectratio]{https://github.com/typst/packages/raw/main/packages/preview/vartable/0.1.2/examples/trigonometricFunction.png}}

\subsubsection{hyperbolic function \$f(x) = 1/x
\$}\label{hyperbolic-function-fx-1x}

\begin{Shaded}
\begin{Highlighting}[]
\NormalTok{\#import }\StringTok{"@preview/Tabvar:0.1.0"}\OperatorTok{:}\NormalTok{ tabvar}

\NormalTok{\#}\FunctionTok{tabvar}\NormalTok{(}
\NormalTok{    init}\OperatorTok{:}\NormalTok{ (}
\NormalTok{        variable}\OperatorTok{:}\NormalTok{ $x$}\OperatorTok{,}
\NormalTok{    label}\OperatorTok{:}\NormalTok{ (}
\NormalTok{        ([sign }\KeywordTok{of}\NormalTok{ $f$]}\OperatorTok{,} \StringTok{"Sign"}\NormalTok{)}\OperatorTok{,}
\NormalTok{      ([variation }\KeywordTok{of}\NormalTok{ $f$]}\OperatorTok{,} \StringTok{"Variation"}\NormalTok{)}\OperatorTok{,}
\NormalTok{    )}\OperatorTok{,}
\NormalTok{  )}\OperatorTok{,}
\NormalTok{  domain}\OperatorTok{:}\NormalTok{ ($ }\OperatorTok{{-}}\NormalTok{oo $}\OperatorTok{,}\NormalTok{ $ }\DecValTok{0}\NormalTok{ $}\OperatorTok{,}\NormalTok{ $ }\OperatorTok{+}\NormalTok{oo $)}\OperatorTok{,}
\NormalTok{  content}\OperatorTok{:}\NormalTok{ (}
\NormalTok{      ($}\OperatorTok{+}\NormalTok{$}\OperatorTok{,}\NormalTok{ (}\StringTok{"||"}\OperatorTok{,}\NormalTok{ $}\OperatorTok{+}\NormalTok{$))}\OperatorTok{,}
\NormalTok{    (}
\NormalTok{        (center}\OperatorTok{,}\NormalTok{ $0$)}\OperatorTok{,}
\NormalTok{      (bottom}\OperatorTok{,}\NormalTok{ top}\OperatorTok{,} \StringTok{"||"}\OperatorTok{,}\NormalTok{ $}\OperatorTok{{-}}\NormalTok{oo$}\OperatorTok{,}\NormalTok{ $}\OperatorTok{+}\NormalTok{oo$)}\OperatorTok{,}
\NormalTok{      (center}\OperatorTok{,}\NormalTok{ $0$)}\OperatorTok{,}
\NormalTok{    )}\OperatorTok{,}
\NormalTok{  )}\OperatorTok{,}
\NormalTok{)}
\end{Highlighting}
\end{Shaded}

\pandocbounded{\includegraphics[keepaspectratio]{https://github.com/typst/packages/raw/main/packages/preview/vartable/0.1.2/examples/hyperbolicFuntion.png}}

\begin{itemize}
\tightlist
\item
  if you put too wide an element for the last value of a variation
  table, this can create a space between the edge of the table and the
  lines separating the lines of the table
\end{itemize}

\pandocbounded{\includegraphics[keepaspectratio]{https://github.com/typst/packages/raw/main/packages/preview/vartable/0.1.2/examples/bug1.png}}

\subsection{·change log·}\label{uxe2change-loguxe2}

\paragraph{0.1.2 :}\label{uxe2}

\begin{itemize}
\tightlist
\item
  Support \texttt{\ fletcher\ 0.5.2\ }
\end{itemize}

\paragraph{0.1.1 :}\label{uxe2-1}

\begin{itemize}
\tightlist
\item
  added customisation of separator bars between signs
\end{itemize}

\subparagraph{0.1.0 :}\label{uxe2-2}

\begin{itemize}
\tightlist
\item
  publishing the package
\end{itemize}

\subsubsection{How to add}\label{how-to-add}

Copy this into your project and use the import as \texttt{\ vartable\ }

\begin{verbatim}
#import "@preview/vartable:0.1.2"
\end{verbatim}

\includesvg[width=0.16667in,height=0.16667in]{/assets/icons/16-copy.svg}

Check the docs for
\href{https://typst.app/docs/reference/scripting/\#packages}{more
information on how to import packages} .

\subsubsection{About}\label{about}

\begin{description}
\tightlist
\item[Author :]
Le\_Foucheur
\item[License:]
MIT
\item[Current version:]
0.1.2
\item[Last updated:]
October 29, 2024
\item[First released:]
July 2, 2024
\item[Archive size:]
114 kB
\href{https://packages.typst.org/preview/vartable-0.1.2.tar.gz}{\pandocbounded{\includesvg[keepaspectratio]{/assets/icons/16-download.svg}}}
\item[Repository:]
\href{https://github.com/Le-foucheur/Typst-VarTable}{GitHub}
\item[Categor y :]
\begin{itemize}
\tightlist
\item[]
\item
  \pandocbounded{\includesvg[keepaspectratio]{/assets/icons/16-chart.svg}}
  \href{https://typst.app/universe/search/?category=visualization}{Visualization}
\end{itemize}
\end{description}

\subsubsection{Where to report issues?}\label{where-to-report-issues}

This package is a project of Le\_Foucheur . Report issues on
\href{https://github.com/Le-foucheur/Typst-VarTable}{their repository} .
You can also try to ask for help with this package on the
\href{https://forum.typst.app}{Forum} .

Please report this package to the Typst team using the
\href{https://typst.app/contact}{contact form} if you believe it is a
safety hazard or infringes upon your rights.

\phantomsection\label{versions}
\subsubsection{Version history}\label{version-history}

\begin{longtable}[]{@{}ll@{}}
\toprule\noalign{}
Version & Release Date \\
\midrule\noalign{}
\endhead
\bottomrule\noalign{}
\endlastfoot
0.1.2 & October 29, 2024 \\
\href{https://typst.app/universe/package/vartable/0.1.1/}{0.1.1} &
October 14, 2024 \\
\href{https://typst.app/universe/package/vartable/0.1.0/}{0.1.0} & July
2, 2024 \\
\end{longtable}

Typst GmbH did not create this package and cannot guarantee correct
functionality of this package or compatibility with any version of the
Typst compiler or app.


\title{typst.app/universe/package/pubmatter}

\phantomsection\label{banner}
\section{pubmatter}\label{pubmatter}

{ 0.1.0 }

Parse, normalize and show publication frontmatter, including authors and
affiliations

\phantomsection\label{readme}
\emph{Beautiful scientific documents with structured metadata for
publishers}

\href{https://github.com/curvenote/pubmatter/blob/main/docs.pdf}{\pandocbounded{\includesvg[keepaspectratio]{https://img.shields.io/badge/typst-docs-orange.svg}}}
\href{https://github.com/curvenote/pubmatter/blob/main/LICENSE}{\pandocbounded{\includesvg[keepaspectratio]{https://img.shields.io/badge/license-MIT-blue.svg}}}

Pubmatter is a typst library for parsing, normalizing and showing
scientific publication frontmatter.

Utilities for loading, normalizing and working with authors,
affiliations, abstracts, keywords and other frontmatter information
common in scientific publications. Our goal is to introduce standardized
ways of working with this content to expose metadata to scientific
publishers who are interested in using typst in a standardized way. The
specification for this \texttt{\ pubmatter\ } is based on
\href{https://mystmd.org/}{MyST Markdown} and
\href{https://quarto.org/}{Quarto} , and can load their YAML files
directly.

\subsection{Examples}\label{examples}

Pubmatter was used to create these documents, for loading the authors in
a standardized way and creting the common elements (authors,
affiliations, ORCIDs, DOIs, Open Access Links, copyright statements,
etc.)

\pandocbounded{\includegraphics[keepaspectratio]{https://raw.githubusercontent.com/curvenote/pubmatter/main/images/lapreprint.png?raw=true}}

\pandocbounded{\includegraphics[keepaspectratio]{https://raw.githubusercontent.com/curvenote/pubmatter/main/images/scipy.png?raw=true}}

\pandocbounded{\includegraphics[keepaspectratio]{https://raw.githubusercontent.com/curvenote/pubmatter/main/images/agrogeo.png?raw=true}}

\subsection{Documentation}\label{documentation}

The full documentation can be found in
\href{https://github.com/curvenote/pubmatter/blob/main/docs.pdf}{docs.pdf}
. To use \texttt{\ pubmatter\ } import it:

\begin{Shaded}
\begin{Highlighting}[]
\NormalTok{\#import "@preview/pubmatter:0.1.0"}
\end{Highlighting}
\end{Shaded}

The docs also use \texttt{\ pubmatter\ } , in a simplified way, you can
see the
\href{https://github.com/curvenote/pubmatter/blob/main/docs.typ}{docs.typ}
to see a simple example of using various components to create a new
document. Here is a preview of the docs:

\href{https://github.com/curvenote/pubmatter/blob/main/docs.pdf}{\pandocbounded{\includegraphics[keepaspectratio]{https://raw.githubusercontent.com/curvenote/pubmatter/main/images/pubmatter.png?raw=true}}}

\subsubsection{Loading Frontmatter}\label{loading-frontmatter}

The frontmatter can contain all information for an article, including
title, authors, affiliations, abstracts and keywords. These are then
normalized into a standardized format that can be used with a number of
\texttt{\ show\ } functions like \texttt{\ show-authors\ } . For
example, we might have a YAML file that looks like this:

\begin{Shaded}
\begin{Highlighting}[]
\FunctionTok{author}\KeywordTok{:}\AttributeTok{ Rowan Cockett}
\FunctionTok{date}\KeywordTok{:}\AttributeTok{ 2024/01/26}
\end{Highlighting}
\end{Shaded}

You can load that file with \texttt{\ yaml\ } , and pass it to the
\texttt{\ load\ } function:

\begin{Shaded}
\begin{Highlighting}[]
\NormalTok{\#let fm = pubmatter.load(yaml("pubmatter.yml"))}
\end{Highlighting}
\end{Shaded}

This will give you a normalized data-structure that can be used with the
\texttt{\ show\ } functions for showing various parts of a document.

You can also use a \texttt{\ dictionary\ } directly:

\begin{Shaded}
\begin{Highlighting}[]
\NormalTok{\#let fm = pubmatter.load((}
\NormalTok{  author: (}
\NormalTok{    (}
\NormalTok{      name: "Rowan Cockett",}
\NormalTok{      email: "rowan@curvenote.com",}
\NormalTok{      orcid: "0000{-}0002{-}7859{-}8394",}
\NormalTok{      affiliations: "Curvenote Inc.",}
\NormalTok{    ),}
\NormalTok{  ),}
\NormalTok{  date: datetime(year: 2024, month: 01, day: 26),}
\NormalTok{  doi: "10.1190/tle35080703.1",}
\NormalTok{))}
\NormalTok{\#pubmatter.show{-}author{-}block(fm)}
\end{Highlighting}
\end{Shaded}

\pandocbounded{\includegraphics[keepaspectratio]{https://raw.githubusercontent.com/curvenote/pubmatter/main/images/author-block.png?raw=true}}

\subsubsection{Theming}\label{theming}

The theme including color and font choice can be set using the
\texttt{\ THEME\ } state. For example, this document has the following
theme set:

\begin{Shaded}
\begin{Highlighting}[]
\NormalTok{\#let theme = (color: red.darken(20\%), font: "Noto Sans")}
\NormalTok{\#set page(header: pubmatter.show{-}page{-}header(theme: theme, fm), footer: pubmatter.show{-}page{-}footer(fm))}
\NormalTok{\#state("THEME").update(theme)}
\end{Highlighting}
\end{Shaded}

Note that for the \texttt{\ header\ } the theme must be passed in
directly. This will hopefully become easier in the future, however,
there is a current bug that removes the page header/footer if you set
this above the \texttt{\ set\ page\ } . See
\href{https://github.com/typst/packages/raw/main/packages/preview/pubmatter/0.1.0/\#2987}{https://github.com/typst/typst/issues/2987}
.

The \texttt{\ font\ } option only corresponds to the frontmatter content
(abstracts, title, header/footer etc.) allowing the body of your
document to have a different font choice.

\subsubsection{Normalized Frontmatter
Object}\label{normalized-frontmatter-object}

The frontmatter object has the following normalized structure:

\begin{Shaded}
\begin{Highlighting}[]
\FunctionTok{title}\KeywordTok{:}\AttributeTok{ content}
\FunctionTok{subtitle}\KeywordTok{:}\AttributeTok{ content}
\FunctionTok{short{-}title}\KeywordTok{:}\AttributeTok{ string}\CommentTok{ \# alias: running{-}title, running{-}head}
\CommentTok{\# Authors Array}
\CommentTok{\# simple string provided for author is turned into ((name: string),)}
\FunctionTok{authors}\KeywordTok{:}\CommentTok{ \# alias: author}
\AttributeTok{  }\KeywordTok{{-}}\AttributeTok{ }\FunctionTok{name}\KeywordTok{:}\AttributeTok{ string}
\AttributeTok{    }\FunctionTok{url}\KeywordTok{:}\AttributeTok{ string}\CommentTok{ \# alias: website, homepage}
\AttributeTok{    }\FunctionTok{email}\KeywordTok{:}\AttributeTok{ string}
\AttributeTok{    }\FunctionTok{phone}\KeywordTok{:}\AttributeTok{ string}
\AttributeTok{    }\FunctionTok{fax}\KeywordTok{:}\AttributeTok{ string}
\AttributeTok{    }\FunctionTok{orcid}\KeywordTok{:}\AttributeTok{ string}\CommentTok{ \# alias: ORCID}
\AttributeTok{    }\FunctionTok{note}\KeywordTok{:}\AttributeTok{ string}
\AttributeTok{    }\FunctionTok{corresponding}\KeywordTok{:}\AttributeTok{ boolean}\CommentTok{ \# default: \textasciigrave{}true\textasciigrave{} when email set}
\AttributeTok{    }\FunctionTok{equal{-}contributor}\KeywordTok{:}\AttributeTok{ boolean}\CommentTok{ \# alias: equalContributor, equal\_contributor}
\AttributeTok{    }\FunctionTok{deceased}\KeywordTok{:}\AttributeTok{ boolean}
\AttributeTok{    }\FunctionTok{roles}\KeywordTok{:}\AttributeTok{ string[]}\CommentTok{ \# must be a contributor role}
\AttributeTok{    }\FunctionTok{affiliations}\KeywordTok{:}\CommentTok{ \# alias: affiliation}
\AttributeTok{      }\KeywordTok{{-}}\AttributeTok{ }\FunctionTok{id}\KeywordTok{:}\AttributeTok{ string}
\AttributeTok{        }\FunctionTok{index}\KeywordTok{:}\AttributeTok{ number}
\CommentTok{\# Affiliations Array}
\FunctionTok{affiliations}\KeywordTok{:}\CommentTok{ \# alias: affiliation}
\AttributeTok{  }\KeywordTok{{-}}\AttributeTok{ string}\CommentTok{ \# simple string is turned into (name: string)}
\AttributeTok{  }\KeywordTok{{-}}\AttributeTok{ }\FunctionTok{id}\KeywordTok{:}\AttributeTok{ string}
\AttributeTok{    }\FunctionTok{index}\KeywordTok{:}\AttributeTok{ number}
\AttributeTok{    }\FunctionTok{name}\KeywordTok{:}\AttributeTok{ string}
\AttributeTok{    }\FunctionTok{institution}\KeywordTok{:}\AttributeTok{ string}\CommentTok{ \# use either name or institution}
\CommentTok{\# Other publication metadata}
\FunctionTok{open{-}access}\KeywordTok{:}\AttributeTok{ boolean}
\FunctionTok{license}\KeywordTok{:}\CommentTok{ \# Can be set with a SPDX ID for creative commons}
\AttributeTok{  }\FunctionTok{id}\KeywordTok{:}\AttributeTok{ string}
\AttributeTok{  }\FunctionTok{url}\KeywordTok{:}\AttributeTok{ string}
\AttributeTok{  }\FunctionTok{name}\KeywordTok{:}\AttributeTok{ string}
\FunctionTok{doi}\KeywordTok{:}\AttributeTok{ string}\CommentTok{ \# must be only the ID, not the full URL}
\FunctionTok{date}\KeywordTok{:}\AttributeTok{ datetime}\CommentTok{ \# validates from \textquotesingle{}YYYY{-}MM{-}DD\textquotesingle{} if a string}
\FunctionTok{citation}\KeywordTok{:}\AttributeTok{ content}
\CommentTok{\# Abstracts Array}
\CommentTok{\# content is turned into ((title: "Abstract", content: string),)}
\FunctionTok{abstracts}\KeywordTok{:}\CommentTok{ \# alias: abstract}
\AttributeTok{  }\KeywordTok{{-}}\AttributeTok{ }\FunctionTok{title}\KeywordTok{:}\AttributeTok{ content}
\AttributeTok{    }\FunctionTok{content}\KeywordTok{:}\AttributeTok{ content}
\end{Highlighting}
\end{Shaded}

Note that you will usually write the affiliations directly in line, in
the following example, we can see that the output is a normalized
affiliation object that is linked by the \texttt{\ id\ } of the
affiliation (just the name if it is a string!).

\begin{Shaded}
\begin{Highlighting}[]
\NormalTok{\#let fm = pubmatter.load((}
\NormalTok{  authors: (}
\NormalTok{    (}
\NormalTok{      name: "Rowan Cockett",}
\NormalTok{      affiliations: "Curvenote Inc.",}
\NormalTok{    ),}
\NormalTok{    (}
\NormalTok{      name: "Steve Purves",}
\NormalTok{      affiliations: ("Executable Books", "Curvenote Inc."),}
\NormalTok{    ),}
\NormalTok{  ),}
\NormalTok{))}
\NormalTok{\#raw(lang:"yaml", yaml.encode(fm))}
\end{Highlighting}
\end{Shaded}

\pandocbounded{\includegraphics[keepaspectratio]{https://raw.githubusercontent.com/curvenote/pubmatter/main/images/normalized.png?raw=true}}

\subsubsection{Full List of Functions}\label{full-list-of-functions}

\begin{itemize}
\tightlist
\item
  \texttt{\ load()\ } - Load a raw frontmatter object
\item
  \texttt{\ doi-link()\ } - Create a DOI link
\item
  \texttt{\ email-link()\ } - Create a mailto link with an email icon
\item
  \texttt{\ github-link()\ } - Create a link to a GitHub profile with
  the GitHub icon
\item
  \texttt{\ open-access-link()\ } - Create a link to Wikipedia with an
  OpenAccess icon
\item
  \texttt{\ orcid-link()\ } - Create a ORCID link with an ORCID logo
\item
  \texttt{\ show-abstract-block()\ } - Show abstract-block including all
  abstracts and keywords
\item
  \texttt{\ show-abstracts()\ } - Show all abstracts (e.g. abstract,
  plain language summary)
\item
  \texttt{\ show-affiliations()\ } - Show affiliations
\item
  \texttt{\ show-author-block()\ } - Show author block, including
  author, icon links (e.g. ORCID, email, etc.) and affiliations
\item
  \texttt{\ show-authors()\ } - Show authors
\item
  \texttt{\ show-citation()\ } - Create a short citation in APA format,
  e.g. Cockett \emph{et al.} , 2023
\item
  \texttt{\ show-copyright()\ } - Show copyright statement based on
  license
\item
  \texttt{\ show-keywords()\ } - Show keywords as a list
\item
  \texttt{\ show-license-badge()\ } - Show the license badges
\item
  \texttt{\ show-page-footer()\ } - Show the venue, date and page
  numbers
\item
  \texttt{\ show-page-header()\ } - Show an open-access badge and the
  DOI and then the running-title and citation
\item
  \texttt{\ show-spaced-content()\ }
\item
  \texttt{\ show-title()\ } - Show title and subtitle
\item
  \texttt{\ show-title-block()\ } - Show title, authors and affiliations
\end{itemize}

\subsection{Contributing}\label{contributing}

To help with standardization of metadata or improve the show-functions
please contribute to this package:\\
\url{https://github.com/curvenote/pubmatter}

\subsubsection{How to add}\label{how-to-add}

Copy this into your project and use the import as \texttt{\ pubmatter\ }

\begin{verbatim}
#import "@preview/pubmatter:0.1.0"
\end{verbatim}

\includesvg[width=0.16667in,height=0.16667in]{/assets/icons/16-copy.svg}

Check the docs for
\href{https://typst.app/docs/reference/scripting/\#packages}{more
information on how to import packages} .

\subsubsection{About}\label{about}

\begin{description}
\tightlist
\item[Author :]
rowanc1
\item[License:]
MIT
\item[Current version:]
0.1.0
\item[Last updated:]
February 10, 2024
\item[First released:]
February 10, 2024
\item[Archive size:]
9.84 kB
\href{https://packages.typst.org/preview/pubmatter-0.1.0.tar.gz}{\pandocbounded{\includesvg[keepaspectratio]{/assets/icons/16-download.svg}}}
\item[Repository:]
\href{https://github.com/curvenote/pubmatter}{GitHub}
\end{description}

\subsubsection{Where to report issues?}\label{where-to-report-issues}

This package is a project of rowanc1 . Report issues on
\href{https://github.com/curvenote/pubmatter}{their repository} . You
can also try to ask for help with this package on the
\href{https://forum.typst.app}{Forum} .

Please report this package to the Typst team using the
\href{https://typst.app/contact}{contact form} if you believe it is a
safety hazard or infringes upon your rights.

\phantomsection\label{versions}
\subsubsection{Version history}\label{version-history}

\begin{longtable}[]{@{}ll@{}}
\toprule\noalign{}
Version & Release Date \\
\midrule\noalign{}
\endhead
\bottomrule\noalign{}
\endlastfoot
0.1.0 & February 10, 2024 \\
\end{longtable}

Typst GmbH did not create this package and cannot guarantee correct
functionality of this package or compatibility with any version of the
Typst compiler or app.


\title{typst.app/universe/package/subpar}

\phantomsection\label{banner}
\section{subpar}\label{subpar}

{ 0.2.0 }

Create sub figures easily.

\phantomsection\label{readme}
Subpar is a \href{https://typst.app/}{Typst} package for creating sub
figures.

\begin{Shaded}
\begin{Highlighting}[]
\NormalTok{\#import "@preview/subpar:0.2.0"}

\NormalTok{\#set page(height: auto)}
\NormalTok{\#set par(justify: true)}

\NormalTok{\#subpar.grid(}
\NormalTok{  figure(image("/assets/andromeda.jpg"), caption: [}
\NormalTok{    An image of the andromeda galaxy.}
\NormalTok{  ]), \textless{}a\textgreater{},}
\NormalTok{  figure(image("/assets/mountains.jpg"), caption: [}
\NormalTok{    A sunset illuminating the sky above a mountain range.}
\NormalTok{  ]), \textless{}b\textgreater{},}
\NormalTok{  columns: (1fr, 1fr),}
\NormalTok{  caption: [A figure composed of two sub figures.],}
\NormalTok{  label: \textless{}full\textgreater{},}
\NormalTok{)}

\NormalTok{Above in @full, we see a figure which is composed of two other figures, namely @a and @b.}
\end{Highlighting}
\end{Shaded}

\pandocbounded{\includegraphics[keepaspectratio]{https://github.com/typst/packages/raw/main/packages/preview/subpar/0.2.0/examples/example.png}}

\subsection{Contributing}\label{contributing}

Contributions are most welcome, make sure to let others know you’re
working on something beforehand so no two people waste their time
working on the same issue. It’s recommended to have
\href{https://github.com/tingerrr/typst-test}{\texttt{\ typst-test\ }}
installed to run tests locally.

\subsection{Documentation}\label{documentation}

A guide and API-reference for subpar can be found in it’s
\href{https://github.com/typst/packages/raw/main/packages/preview/subpar/0.2.0/doc/manual.pdf}{manual}
.

\subsubsection{How to add}\label{how-to-add}

Copy this into your project and use the import as \texttt{\ subpar\ }

\begin{verbatim}
#import "@preview/subpar:0.2.0"
\end{verbatim}

\includesvg[width=0.16667in,height=0.16667in]{/assets/icons/16-copy.svg}

Check the docs for
\href{https://typst.app/docs/reference/scripting/\#packages}{more
information on how to import packages} .

\subsubsection{About}\label{about}

\begin{description}
\tightlist
\item[Author :]
\href{mailto:me@tinger.dev}{tinger}
\item[License:]
MIT
\item[Current version:]
0.2.0
\item[Last updated:]
November 18, 2024
\item[First released:]
May 3, 2024
\item[Minimum Typst version:]
0.12.0
\item[Archive size:]
1.15 MB
\href{https://packages.typst.org/preview/subpar-0.2.0.tar.gz}{\pandocbounded{\includesvg[keepaspectratio]{/assets/icons/16-download.svg}}}
\item[Repository:]
\href{https://github.com/tingerrr/subpar}{GitHub}
\item[Categor ies :]
\begin{itemize}
\tightlist
\item[]
\item
  \pandocbounded{\includesvg[keepaspectratio]{/assets/icons/16-package.svg}}
  \href{https://typst.app/universe/search/?category=components}{Components}
\item
  \pandocbounded{\includesvg[keepaspectratio]{/assets/icons/16-list-unordered.svg}}
  \href{https://typst.app/universe/search/?category=model}{Model}
\end{itemize}
\end{description}

\subsubsection{Where to report issues?}\label{where-to-report-issues}

This package is a project of tinger . Report issues on
\href{https://github.com/tingerrr/subpar}{their repository} . You can
also try to ask for help with this package on the
\href{https://forum.typst.app}{Forum} .

Please report this package to the Typst team using the
\href{https://typst.app/contact}{contact form} if you believe it is a
safety hazard or infringes upon your rights.

\phantomsection\label{versions}
\subsubsection{Version history}\label{version-history}

\begin{longtable}[]{@{}ll@{}}
\toprule\noalign{}
Version & Release Date \\
\midrule\noalign{}
\endhead
\bottomrule\noalign{}
\endlastfoot
0.2.0 & November 18, 2024 \\
\href{https://typst.app/universe/package/subpar/0.1.1/}{0.1.1} & July 3,
2024 \\
\href{https://typst.app/universe/package/subpar/0.1.0/}{0.1.0} & May 3,
2024 \\
\end{longtable}

Typst GmbH did not create this package and cannot guarantee correct
functionality of this package or compatibility with any version of the
Typst compiler or app.


\title{typst.app/universe/package/badformer}

\phantomsection\label{banner}
\phantomsection\label{template-thumbnail}
\pandocbounded{\includegraphics[keepaspectratio]{https://packages.typst.org/preview/thumbnails/badformer-0.1.0-small.webp}}

\section{badformer}\label{badformer}

{ 0.1.0 }

Retro-gaming in Typst. Reach the goal and complete the mission.

\href{/app?template=badformer&version=0.1.0}{Create project in app}

\phantomsection\label{readme}
Reach the goal in this retro-inspired wireframing platformer. Play in 3
dimensions and compete for the lowest number of steps to win!

This small game is playable in the Typst editor and best enjoyed with
the web app or \texttt{\ typst\ watch\ } . It was first released for the
24 Days to Christmas campaign in winter of 2023.

\subsection{Usage}\label{usage}

You can use this template in the Typst web app by clicking “Start from
template� on the dashboard and searching for \texttt{\ badformer\ } .

Alternatively, you can use the CLI to kick this project off using the
command

\begin{verbatim}
typst init @preview/badformer
\end{verbatim}

Typst will create a new directory with all the files needed to get you
started.

Move with WASD and jump with space. You can also display a minimap by
pressing E.

\subsection{Configuration}\label{configuration}

This template exports the \texttt{\ game\ } function, which accepts a
positional argument for the game input.

The template will initialize your package with a sample call to the
\texttt{\ game\ } function in a show rule. If you want to change an
existing project to use this template, you can add a show rule like this
at the top of your file:

\begin{Shaded}
\begin{Highlighting}[]
\NormalTok{\#import "@preview/badformer:0.1.0": game}
\NormalTok{\#show: game(read("main.typ"))}

\NormalTok{// Move with WASD and jump with space.}
\end{Highlighting}
\end{Shaded}

\href{/app?template=badformer&version=0.1.0}{Create project in app}

\subsubsection{How to use}\label{how-to-use}

Click the button above to create a new project using this template in
the Typst app.

You can also use the Typst CLI to start a new project on your computer
using this command:

\begin{verbatim}
typst init @preview/badformer:0.1.0
\end{verbatim}

\includesvg[width=0.16667in,height=0.16667in]{/assets/icons/16-copy.svg}

\subsubsection{About}\label{about}

\begin{description}
\tightlist
\item[Author :]
\href{https://typst.app}{Typst GmbH}
\item[License:]
MIT-0
\item[Current version:]
0.1.0
\item[Last updated:]
March 6, 2024
\item[First released:]
March 6, 2024
\item[Minimum Typst version:]
0.10.0
\item[Archive size:]
5.43 kB
\href{https://packages.typst.org/preview/badformer-0.1.0.tar.gz}{\pandocbounded{\includesvg[keepaspectratio]{/assets/icons/16-download.svg}}}
\item[Repository:]
\href{https://github.com/typst/templates}{GitHub}
\item[Categor y :]
\begin{itemize}
\tightlist
\item[]
\item
  \pandocbounded{\includesvg[keepaspectratio]{/assets/icons/16-smile.svg}}
  \href{https://typst.app/universe/search/?category=fun}{Fun}
\end{itemize}
\end{description}

\subsubsection{Where to report issues?}\label{where-to-report-issues}

This template is a project of Typst GmbH . Report issues on
\href{https://github.com/typst/templates}{their repository} . You can
also try to ask for help with this template on the
\href{https://forum.typst.app}{Forum} .

\phantomsection\label{versions}
\subsubsection{Version history}\label{version-history}

\begin{longtable}[]{@{}ll@{}}
\toprule\noalign{}
Version & Release Date \\
\midrule\noalign{}
\endhead
\bottomrule\noalign{}
\endlastfoot
0.1.0 & March 6, 2024 \\
\end{longtable}


\title{typst.app/universe/package/ttt-lists}

\phantomsection\label{banner}
\phantomsection\label{template-thumbnail}
\pandocbounded{\includegraphics[keepaspectratio]{https://packages.typst.org/preview/thumbnails/ttt-lists-0.1.0-small.webp}}

\section{ttt-lists}\label{ttt-lists}

{ 0.1.0 }

Template to create student lists. Part of the ttt-collection to make a
teachers life easier.

\href{/app?template=ttt-lists&version=0.1.0}{Create project in app}

\phantomsection\label{readme}
\texttt{\ ttt-lists\ } is a \emph{template} to create class lists and
belongs to the
\href{https://github.com/jomaway/typst-teacher-templates}{typst-teacher-tools-collection}
.

\subsection{Usage}\label{usage}

Run this command inside your terminal to init a new list.

\begin{Shaded}
\begin{Highlighting}[]
\ExtensionTok{typst}\NormalTok{ init @preview/ttt{-}lists my{-}student{-}list}
\end{Highlighting}
\end{Shaded}

This will scaffold the following folder structure.

\begin{Shaded}
\begin{Highlighting}[]
\NormalTok{my{-}student{-}list/}
\NormalTok{├─ students.csv}
\NormalTok{└─ students.typ}
\end{Highlighting}
\end{Shaded}

Edit the \texttt{\ students.csv\ } file or replace it with your own.
Modify the \texttt{\ students.typ\ } to your liking or leave as is and
then run \texttt{\ typst\ compile\ students.typ\ } to create a beautiful
list.

\href{/app?template=ttt-lists&version=0.1.0}{Create project in app}

\subsubsection{How to use}\label{how-to-use}

Click the button above to create a new project using this template in
the Typst app.

You can also use the Typst CLI to start a new project on your computer
using this command:

\begin{verbatim}
typst init @preview/ttt-lists:0.1.0
\end{verbatim}

\includesvg[width=0.16667in,height=0.16667in]{/assets/icons/16-copy.svg}

\subsubsection{About}\label{about}

\begin{description}
\tightlist
\item[Author :]
\href{https://github.com/jomaway}{Jomaway}
\item[License:]
MIT
\item[Current version:]
0.1.0
\item[Last updated:]
April 2, 2024
\item[First released:]
April 2, 2024
\item[Minimum Typst version:]
0.11.0
\item[Archive size:]
3.03 kB
\href{https://packages.typst.org/preview/ttt-lists-0.1.0.tar.gz}{\pandocbounded{\includesvg[keepaspectratio]{/assets/icons/16-download.svg}}}
\item[Repository:]
\href{https://github.com/jomaway/typst-teacher-templates}{GitHub}
\item[Discipline :]
\begin{itemize}
\tightlist
\item[]
\item
  \href{https://typst.app/universe/search/?discipline=education}{Education}
\end{itemize}
\item[Categor ies :]
\begin{itemize}
\tightlist
\item[]
\item
  \pandocbounded{\includesvg[keepaspectratio]{/assets/icons/16-package.svg}}
  \href{https://typst.app/universe/search/?category=components}{Components}
\item
  \pandocbounded{\includesvg[keepaspectratio]{/assets/icons/16-hammer.svg}}
  \href{https://typst.app/universe/search/?category=utility}{Utility}
\item
  \pandocbounded{\includesvg[keepaspectratio]{/assets/icons/16-envelope.svg}}
  \href{https://typst.app/universe/search/?category=office}{Office}
\end{itemize}
\end{description}

\subsubsection{Where to report issues?}\label{where-to-report-issues}

This template is a project of Jomaway . Report issues on
\href{https://github.com/jomaway/typst-teacher-templates}{their
repository} . You can also try to ask for help with this template on the
\href{https://forum.typst.app}{Forum} .

Please report this template to the Typst team using the
\href{https://typst.app/contact}{contact form} if you believe it is a
safety hazard or infringes upon your rights.

\phantomsection\label{versions}
\subsubsection{Version history}\label{version-history}

\begin{longtable}[]{@{}ll@{}}
\toprule\noalign{}
Version & Release Date \\
\midrule\noalign{}
\endhead
\bottomrule\noalign{}
\endlastfoot
0.1.0 & April 2, 2024 \\
\end{longtable}

Typst GmbH did not create this template and cannot guarantee correct
functionality of this template or compatibility with any version of the
Typst compiler or app.


\title{typst.app/universe/package/haw-hamburg-report}

\phantomsection\label{banner}
\phantomsection\label{template-thumbnail}
\pandocbounded{\includegraphics[keepaspectratio]{https://packages.typst.org/preview/thumbnails/haw-hamburg-report-0.3.1-small.webp}}

\section{haw-hamburg-report}\label{haw-hamburg-report}

{ 0.3.1 }

Unofficial template for writing a report in the HAW Hamburg department
of Computer Science design.

\href{/app?template=haw-hamburg-report&version=0.3.1}{Create project in
app}

\phantomsection\label{readme}
This is an \textbf{\texttt{\ unofficial\ }} template for writing a
report in the \texttt{\ HAW\ Hamburg\ } department of
\texttt{\ Computer\ Science\ } design using
\href{https://github.com/typst/typst}{Typst} .

\subsection{Required Fonts}\label{required-fonts}

To correctly render this template please make sure that the
\texttt{\ New\ Computer\ Modern\ } font is installed on your system.

\subsection{Usage}\label{usage}

To use this package just add the following code to your
\href{https://github.com/typst/typst}{Typst} document:

\begin{Shaded}
\begin{Highlighting}[]
\NormalTok{\#import "@preview/haw{-}hamburg:0.3.1": report}

\NormalTok{\#show: report.with(}
\NormalTok{  language: "en",}
\NormalTok{  title: "Example title",}
\NormalTok{  author:"Example author",}
\NormalTok{  faculty: "Engineering and Computer Science",}
\NormalTok{  department: "Computer Science",}
\NormalTok{  include{-}declaration{-}of{-}independent{-}processing: true,}
\NormalTok{)}
\end{Highlighting}
\end{Shaded}

\subsection{How to Compile}\label{how-to-compile}

This project contains an example setup that splits individual chapters
into different files.\\
This can cause problems when using references etc.\\
These problems can be avoided by following these steps:

\begin{itemize}
\tightlist
\item
  Make sure to always compile your \texttt{\ main.typ\ } file which
  includes all of your chapters for references to work correctly.
\item
  VSCode:

  \begin{itemize}
  \tightlist
  \item
    Install the
    \href{https://marketplace.visualstudio.com/items?itemName=myriad-dreamin.tinymist}{Tinymist
    Typst} extension.
  \item
    Make sure to start the \texttt{\ PDF\ } or
    \texttt{\ Live\ Preview\ } only from within your
    \texttt{\ main.typ\ } file.
  \item
    If problems occur it usually helps to close the preview and restart
    it from your \texttt{\ main.typ\ } file.
  \end{itemize}
\end{itemize}

\href{/app?template=haw-hamburg-report&version=0.3.1}{Create project in
app}

\subsubsection{How to use}\label{how-to-use}

Click the button above to create a new project using this template in
the Typst app.

You can also use the Typst CLI to start a new project on your computer
using this command:

\begin{verbatim}
typst init @preview/haw-hamburg-report:0.3.1
\end{verbatim}

\includesvg[width=0.16667in,height=0.16667in]{/assets/icons/16-copy.svg}

\subsubsection{About}\label{about}

\begin{description}
\tightlist
\item[Author :]
Lasse Rosenow
\item[License:]
MIT
\item[Current version:]
0.3.1
\item[Last updated:]
November 13, 2024
\item[First released:]
October 14, 2024
\item[Archive size:]
6.11 kB
\href{https://packages.typst.org/preview/haw-hamburg-report-0.3.1.tar.gz}{\pandocbounded{\includesvg[keepaspectratio]{/assets/icons/16-download.svg}}}
\item[Repository:]
\href{https://github.com/LasseRosenow/HAW-Hamburg-Typst-Template}{GitHub}
\item[Categor y :]
\begin{itemize}
\tightlist
\item[]
\item
  \pandocbounded{\includesvg[keepaspectratio]{/assets/icons/16-speak.svg}}
  \href{https://typst.app/universe/search/?category=report}{Report}
\end{itemize}
\end{description}

\subsubsection{Where to report issues?}\label{where-to-report-issues}

This template is a project of Lasse Rosenow . Report issues on
\href{https://github.com/LasseRosenow/HAW-Hamburg-Typst-Template}{their
repository} . You can also try to ask for help with this template on the
\href{https://forum.typst.app}{Forum} .

Please report this template to the Typst team using the
\href{https://typst.app/contact}{contact form} if you believe it is a
safety hazard or infringes upon your rights.

\phantomsection\label{versions}
\subsubsection{Version history}\label{version-history}

\begin{longtable}[]{@{}ll@{}}
\toprule\noalign{}
Version & Release Date \\
\midrule\noalign{}
\endhead
\bottomrule\noalign{}
\endlastfoot
0.3.1 & November 13, 2024 \\
\href{https://typst.app/universe/package/haw-hamburg-report/0.3.0/}{0.3.0}
& October 14, 2024 \\
\end{longtable}

Typst GmbH did not create this template and cannot guarantee correct
functionality of this template or compatibility with any version of the
Typst compiler or app.


\title{typst.app/universe/package/modern-cv}

\phantomsection\label{banner}
\phantomsection\label{template-thumbnail}
\pandocbounded{\includegraphics[keepaspectratio]{https://packages.typst.org/preview/thumbnails/modern-cv-0.7.0-small.webp}}

\section{modern-cv}\label{modern-cv}

{ 0.7.0 }

A modern resume template based on the Awesome-CV Latex template.

\href{/app?template=modern-cv&version=0.7.0}{Create project in app}

\phantomsection\label{readme}
\href{https://github.com/DeveloperPaul123/modern-cv/stargazers}{\pandocbounded{\includesvg[keepaspectratio]{https://img.shields.io/badge/Say\%20Thanks-\%F0\%9F\%91\%8D-1EAEDB.svg}}}
\href{https://discord.gg/CX2ybByRnt}{\pandocbounded{\includegraphics[keepaspectratio]{https://img.shields.io/discord/652515194572111872?logo=Discord}}}
\pandocbounded{\includegraphics[keepaspectratio]{https://img.shields.io/github/v/release/DeveloperPaul123/modern-cv}}
\href{https://github.com/DeveloperPaul123/modern-cv/actions/workflows/tests.yml}{\pandocbounded{\includesvg[keepaspectratio]{https://github.com/DeveloperPaul123/modern-cv/actions/workflows/tests.yml/badge.svg}}}

A port of the \href{https://github.com/posquit0/Awesome-CV}{Awesome-CV}
Latex resume template in \href{https://github.com/typst/typst}{typst} .

\subsection{Requirements}\label{requirements}

\subsubsection{Tools}\label{tools}

The following tools are used for the development of this template:

\begin{itemize}
\tightlist
\item
  \href{https://github.com/typst/typst}{typst}
\item
  \href{https://github.com/tingerrr/typst-test}{typst-test} for running
  tests
\item
  \href{https://github.com/casey/just}{just} for simplifying command
  running
\item
  \href{https://github.com/shssoichiro/oxipng}{oxipng} for compressing
  thumbnails used in the README
\end{itemize}

\subsubsection{Fonts}\label{fonts}

You will need the \texttt{\ Roboto\ } and \texttt{\ Source\ Sans\ Pro\ }
fonts installed on your system or available somewhere. If you are using
the \texttt{\ typst\ } web app, no further action is necessary. You can
download them from the following links:

\begin{itemize}
\tightlist
\item
  \href{https://fonts.google.com/specimen/Roboto}{Roboto}
\item
  \href{https://github.com/adobe-fonts/source-sans-pro}{Source Sans Pro}
\end{itemize}

This template also uses FontAwesome icons via the
\href{https://typst.app/universe/package/fontawesome}{fontawesome}
package. You will need to install the fontawesome fonts on your system
or configure the \texttt{\ typst\ } web app to use them. You can
download fontawesome \href{https://fontawesome.com/download}{here} .

To use the fontawesome icons in the web app, add a \texttt{\ fonts\ }
folder to your project and upload the \texttt{\ otf\ } files from the
fontawesome download to this folder like so:

\pandocbounded{\includegraphics[keepaspectratio]{https://github.com/typst/packages/raw/main/packages/preview/modern-cv/0.7.0/assets/images/typst_web_editor.png}}

See \texttt{\ typst\ fonts\ -\/-help\ } for more information on
configuring fonts for \texttt{\ typst\ } that are not installed on your
system.

\subsubsection{Usage}\label{usage}

Below is a basic example for a simple resume:

\begin{Shaded}
\begin{Highlighting}[]
\NormalTok{\#import "@preview/modern{-}cv:0.7.0": *}

\NormalTok{\#show: resume.with(}
\NormalTok{  author: (}
\NormalTok{      firstname: "John", }
\NormalTok{      lastname: "Smith",}
\NormalTok{      email: "js@example.com", }
\NormalTok{      phone: "(+1) 111{-}111{-}1111",}
\NormalTok{      github: "DeveloperPaul123",}
\NormalTok{      linkedin: "Example",}
\NormalTok{      address: "111 Example St. Example City, EX 11111",}
\NormalTok{      positions: (}
\NormalTok{        "Software Engineer",}
\NormalTok{        "Software Architect"}
\NormalTok{      )}
\NormalTok{  ),}
\NormalTok{  date: datetime.today().display()}
\NormalTok{)}

\NormalTok{= Education}

\NormalTok{\#resume{-}entry(}
\NormalTok{  title: "Example University",}
\NormalTok{  location: "B.S. in Computer Science",}
\NormalTok{  date: "August 2014 {-} May 2019",}
\NormalTok{  description: "Example"}
\NormalTok{)}

\NormalTok{\#resume{-}item[}
\NormalTok{  {-} \#lorem(20)}
\NormalTok{  {-} \#lorem(15)}
\NormalTok{  {-} \#lorem(25)  }
\NormalTok{]}
\end{Highlighting}
\end{Shaded}

After saving to a \texttt{\ *.typ\ } file, compile your resume using the
following command:

\begin{Shaded}
\begin{Highlighting}[]
\ExtensionTok{typst}\NormalTok{ compile resume.typ}
\end{Highlighting}
\end{Shaded}

For more information on how to use and compile \texttt{\ typst\ } files,
see the \href{https://typst.app/docs}{official documentation} .

Documentation for this template is published with each commit. See the
attached PDF on each Github Action run
\href{https://github.com/DeveloperPaul123/modern-cv/actions}{here} .

\subsubsection{Output Examples}\label{output-examples}

\begin{longtable}[]{@{}ll@{}}
\toprule\noalign{}
Resumes & Cover letters \\
\midrule\noalign{}
\endhead
\bottomrule\noalign{}
\endlastfoot
\pandocbounded{\includegraphics[keepaspectratio]{https://github.com/typst/packages/raw/main/packages/preview/modern-cv/0.7.0/assets/images/resume.png}}
&
\pandocbounded{\includegraphics[keepaspectratio]{https://github.com/typst/packages/raw/main/packages/preview/modern-cv/0.7.0/assets/images/coverletter.png}} \\
\pandocbounded{\includegraphics[keepaspectratio]{https://github.com/typst/packages/raw/main/packages/preview/modern-cv/0.7.0/assets/images/resume2.png}}
&
\pandocbounded{\includegraphics[keepaspectratio]{https://github.com/typst/packages/raw/main/packages/preview/modern-cv/0.7.0/assets/images/coverletter2.png}} \\
\end{longtable}

\subsection{Building and Testing
Locally}\label{building-and-testing-locally}

To build and test the project locally, you will need to install the
\texttt{\ typst\ } CLI. You can find instructions on how to do this
\href{https://github.com/typst/typst\#installation}{here} .

With typst installed you can make changes to \texttt{\ lib.typ\ } and
then \texttt{\ just\ install\ } or \texttt{\ just\ install-preview\ } to
install the package locally. Change the import statements in the
template files to point to the local package (if needed):

\begin{Shaded}
\begin{Highlighting}[]
\NormalTok{\#import "@local/modern{-}cv:0.6.0": *}
\end{Highlighting}
\end{Shaded}

If you use \texttt{\ just\ install-preview\ } you will only need to
update the version number to match \texttt{\ typst.toml\ } .

Note that the script parses the \texttt{\ typst.toml\ } to determine the
version number and the folder the local files are installed to.

\subsubsection{Formatting}\label{formatting}

This project uses
\href{https://github.com/Enter-tainer/typstyle}{typstyle} to format the
code. Run \texttt{\ just\ format\ } to format all the \texttt{\ *.typ\ }
files in the project. Be sure to install \texttt{\ typstyle\ } before
running the script.

\subsection{License}\label{license}

The project is licensed under the MIT license. See
\href{https://github.com/typst/packages/raw/main/packages/preview/modern-cv/0.7.0/LICENSE}{LICENSE}
for more details.

\subsection{Author}\label{author}

\begin{longtable}[]{@{}
  >{\raggedright\arraybackslash}p{(\linewidth - 0\tabcolsep) * \real{1.0000}}@{}}
\toprule\noalign{}
\begin{minipage}[b]{\linewidth}\centering
\href{https://github.com/DeveloperPaul123}{\includegraphics[width=1.04167in,height=\textheight,keepaspectratio]{https://avatars0.githubusercontent.com/u/6591180?s=460&v=4}\\
\textsubscript{@DeveloperPaul123}}\strut
\end{minipage} \\
\midrule\noalign{}
\endhead
\bottomrule\noalign{}
\endlastfoot
\end{longtable}

\href{/app?template=modern-cv&version=0.7.0}{Create project in app}

\subsubsection{How to use}\label{how-to-use}

Click the button above to create a new project using this template in
the Typst app.

You can also use the Typst CLI to start a new project on your computer
using this command:

\begin{verbatim}
typst init @preview/modern-cv:0.7.0
\end{verbatim}

\includesvg[width=0.16667in,height=0.16667in]{/assets/icons/16-copy.svg}

\subsubsection{About}\label{about}

\begin{description}
\tightlist
\item[Author :]
\href{https://github.com/DeveloperPaul123}{Paul Tsouchlos}
\item[License:]
MIT
\item[Current version:]
0.7.0
\item[Last updated:]
November 4, 2024
\item[First released:]
March 26, 2024
\item[Minimum Typst version:]
0.12.0
\item[Archive size:]
20.3 kB
\href{https://packages.typst.org/preview/modern-cv-0.7.0.tar.gz}{\pandocbounded{\includesvg[keepaspectratio]{/assets/icons/16-download.svg}}}
\item[Repository:]
\href{https://github.com/DeveloperPaul123/modern-cv}{GitHub}
\item[Categor y :]
\begin{itemize}
\tightlist
\item[]
\item
  \pandocbounded{\includesvg[keepaspectratio]{/assets/icons/16-user.svg}}
  \href{https://typst.app/universe/search/?category=cv}{CV}
\end{itemize}
\end{description}

\subsubsection{Where to report issues?}\label{where-to-report-issues}

This template is a project of Paul Tsouchlos . Report issues on
\href{https://github.com/DeveloperPaul123/modern-cv}{their repository} .
You can also try to ask for help with this template on the
\href{https://forum.typst.app}{Forum} .

Please report this template to the Typst team using the
\href{https://typst.app/contact}{contact form} if you believe it is a
safety hazard or infringes upon your rights.

\phantomsection\label{versions}
\subsubsection{Version history}\label{version-history}

\begin{longtable}[]{@{}ll@{}}
\toprule\noalign{}
Version & Release Date \\
\midrule\noalign{}
\endhead
\bottomrule\noalign{}
\endlastfoot
0.7.0 & November 4, 2024 \\
\href{https://typst.app/universe/package/modern-cv/0.6.0/}{0.6.0} &
September 3, 2024 \\
\href{https://typst.app/universe/package/modern-cv/0.5.0/}{0.5.0} & July
23, 2024 \\
\href{https://typst.app/universe/package/modern-cv/0.4.0/}{0.4.0} & July
10, 2024 \\
\href{https://typst.app/universe/package/modern-cv/0.3.1/}{0.3.1} & May
16, 2024 \\
\href{https://typst.app/universe/package/modern-cv/0.3.0/}{0.3.0} &
April 17, 2024 \\
\href{https://typst.app/universe/package/modern-cv/0.2.0/}{0.2.0} &
April 4, 2024 \\
\href{https://typst.app/universe/package/modern-cv/0.1.0/}{0.1.0} &
March 26, 2024 \\
\end{longtable}

Typst GmbH did not create this template and cannot guarantee correct
functionality of this template or compatibility with any version of the
Typst compiler or app.


