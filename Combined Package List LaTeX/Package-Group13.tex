\title{typst.app/universe/package/modern-sysu-thesis}

\phantomsection\label{banner}
\phantomsection\label{template-thumbnail}
\pandocbounded{\includegraphics[keepaspectratio]{https://packages.typst.org/preview/thumbnails/modern-sysu-thesis-0.3.0-small.webp}}

\section{modern-sysu-thesis}\label{modern-sysu-thesis}

{ 0.3.0 }

中山大学学ä½?论æ--‡ Typst 模æ?¿ï¼ŒA Typst template for SYSU thesis

\href{/app?template=modern-sysu-thesis&version=0.3.0}{Create project in
app}

\phantomsection\label{readme}
\href{https://gitlab.com/sysu-gitlab/thesis-template/better-thesis/-/releases}{\pandocbounded{\includegraphics[keepaspectratio]{https://gitlab.com/sysu-gitlab/thesis-template/better-thesis/-/badges/release.svg?style=flat-square&value_width=100}}}
\href{https://github.com/sysu/better-thesis}{\pandocbounded{\includegraphics[keepaspectratio]{https://img.shields.io/github/stars/sysu/better-thesis.svg?style=social&label=Star&maxAge=2592000}}}

\textbf{\href{https://typst.app/app?template=modern-sysu-thesis&version=0.1.1}{点击此处注册
typst.app 并创建ä½~的论æ--‡å·¥ç¨‹}}

本ç§`ç''Ÿæ¨¡æ?¿å·²ç»?符å?ˆå­¦ä½?论æ--‡æ~¼å¼?è¦?求(
\href{https://gitlab.com/sysu-gitlab/thesis-template/better-thesis/-/issues/6}{\#6}
),欢迎å?Œå­¦/æ~¡å?‹ä»¬
\href{https://gitlab.com/sysu-gitlab/thesis-template/better-thesis/-/merge_requests}{贡献代ç~?}
/å??馈é---®é¢˜ï¼ˆ
\href{https://gitlab.com/sysu-gitlab/thesis-template/better-thesis/-/issues}{GitLab
issue} /
\href{mailto:contact-project+sysu-gitlab-thesis-template-better-thesis-57823416-issue-@incoming.gitlab.com}{邮件}
)�

模�交� QQ 群:
\href{https://jq.qq.com/?_wv=1027&k=m58va1kd}{797942860}

\subsection{�考规范}\label{uxe5uxe8ux192uxe8uxe8ux153ux192}

\begin{itemize}
\tightlist
\item
  本ç§`ç''Ÿè®ºæ--‡æ¨¡æ?¿å?‚考
  \href{https://spa.sysu.edu.cn/zh-hans/article/1744}{中山大学本ç§`ç''Ÿæ¯•ä¸šè®ºæ--‡ï¼ˆè®¾è®¡ï¼‰å†™ä½œä¸Žå?°åˆ¶è§„范
  2020å¹´å?{}`}
\item
  ç~''究ç''Ÿè®ºæ--‡æ¨¡æ?¿å?‚考
  \href{https://graduate.sysu.edu.cn/sites/graduate.prod.dpcms4.sysu.edu.cn/files/2019-04/\%E4\%B8\%AD\%E5\%B1\%B1\%E5\%A4\%A7\%E5\%AD\%A6\%E7\%A0\%94\%E7\%A9\%B6\%E7\%94\%9F\%E5\%AD\%A6\%E4\%BD\%8D\%E8\%AE\%BA\%E6\%96\%87\%E6\%A0\%BC\%E5\%BC\%8F\%E8\%A6\%81\%E6\%B1\%82.pdf}{中山大学ç~''究ç''Ÿå­¦ä½?论æ--‡æ~¼å¼?è¦?求}
\end{itemize}

\subsection{使ç''¨æ--¹æ³•}\label{uxe4uxbduxe7uxe6uxb9uxe6uxb3}

\subsubsection{typst.app}\label{typst.app}

ç»?过è¿`一月紧å¼~的迭代é‡?构,本模æ?¿å·²ç»?
\href{https://typst.app/universe/package/modern-sysu-thesis}{å?{}`布在typst-app.universe}
上,
\href{https://typst.app/app?template=modern-sysu-thesis&version=0.2.0}{点击此处直接创建ä½~的论æ--‡å·¥ç¨‹}
,并直接开始ç¼--写ä½~的论æ--‡ï¼?

\subsubsection{Windows ç''¨æˆ·}\label{windows-uxe7uxe6ux2c6}

\begin{enumerate}
\tightlist
\item
  \href{https://gitlab.com/sysu-gitlab/thesis-template/better-thesis/-/archive/main/better-thesis-main.zip}{下载本ä»``åº``}
  ,æˆ--è€\ldots 使ç''¨
  \texttt{\ git\ clone\ https://gitlab.com/sysu-gitlab/thesis-template/better-thesis\ }
  å`½ä»¤å\ldots‹éš†æœ¬ä»``åº``。
\item
  å?³é''® \texttt{\ install\_typst.ps1\ } æ--‡ä»¶ï¼Œé€‰æ‹©â€œç''¨
  Powershell è¿?è¡Œâ€?,等å¾\ldots{} Typst 安è£\ldots 完æˆ?。
\item
  æ~¹æ?® \href{https://typst.app/docs/}{Typst æ--‡æ¡£} ,å?‚考
  \href{https://github.com/typst/packages/raw/main/packages/preview/modern-sysu-thesis/0.3.0/\#\%E9\%A1\%B9\%E7\%9B\%AE\%E7\%BB\%93\%E6\%9E\%84}{项目ç»``æž„}
  中的说明,按ç\ldots§ä½~的需è¦?ä¿®æ''¹è®ºæ--‡çš„å?„个部分。
\item
  å?Œå‡»è¿?è¡Œ \texttt{\ compile.bat\ } ,å?³å?¯ç''Ÿæˆ?
  \texttt{\ thesis.pdf\ } æ--‡ä»¶ã€‚
\end{enumerate}

\subsubsection{Linux/macOS ç''¨æˆ·}\label{linuxmacos-uxe7uxe6ux2c6}

\begin{enumerate}
\tightlist
\item
  \href{https://gitlab.com/sysu-gitlab/thesis-template/better-thesis/-/archive/main/better-thesis-main.zip}{下载本ä»``åº``}
  ,æˆ--è€\ldots 使ç''¨
  \texttt{\ git\ clone\ https://gitlab.com/sysu-gitlab/thesis-template/better-thesis\ }
  å`½ä»¤å\ldots‹éš†æœ¬ä»``åº``。
\item
  使ç''¨å`½ä»¤è¡Œå®‰è£\ldots{} Rust å·¥å\ldots·é``¾ä»¥å?Š Typst:
\end{enumerate}

\begin{Shaded}
\begin{Highlighting}[]
\CommentTok{\# 安装 Rust 环境并激活,之前安装过则不需要执行下面这两行}
\ExtensionTok{curl} \AttributeTok{{-}{-}proto} \StringTok{\textquotesingle{}=https\textquotesingle{}} \AttributeTok{{-}{-}tlsv1.2} \AttributeTok{{-}sSf}\NormalTok{ https://sh.rustup.rs }\KeywordTok{|} \FunctionTok{sh} \AttributeTok{{-}s} \AttributeTok{{-}{-}} \AttributeTok{{-}y}
\BuiltInTok{source} \VariableTok{$HOME}\NormalTok{/.cargo/env}

\CommentTok{\# 安装 Typst CLI}
\ExtensionTok{cargo}\NormalTok{ install typst{-}cli}

\CommentTok{\# 访问缓慢的话,执行以下命令设置清华镜像源,并从镜像站安装}
\FunctionTok{cat} \OperatorTok{\textless{}\textless{} EOF} \OperatorTok{\textgreater{}} \VariableTok{$HOME}\NormalTok{/.cargo/config}
\StringTok{[source.crates{-}io]}
\StringTok{replace{-}with = "tuna"}

\StringTok{[source.tuna]}
\StringTok{registry = "https://mirrors.tuna.tsinghua.edu.cn/git/crates.io{-}index.git"}
\OperatorTok{EOF}
\ExtensionTok{cargo}\NormalTok{ install typst{-}cli}
\end{Highlighting}
\end{Shaded}

\begin{enumerate}
\setcounter{enumi}{2}
\tightlist
\item
  æ~¹æ?® \href{https://typst.app/docs/}{Typst æ--‡æ¡£} ,å?‚考
  \href{https://github.com/typst/packages/raw/main/packages/preview/modern-sysu-thesis/0.3.0/\#\%E9\%A1\%B9\%E7\%9B\%AE\%E7\%BB\%93\%E6\%9E\%84}{项目ç»``æž„}
  中的说明,按ç\ldots§ä½~的需è¦?ä¿®æ''¹è®ºæ--‡çš„å?„个部分。
\item
  执行 \texttt{\ make\ } å`½ä»¤ï¼Œå?³å?¯ç''Ÿæˆ?
  \texttt{\ thesis.pdf\ } æ--‡ä»¶ã€‚
\end{enumerate}

\subsection{项目ç»``æž„}\label{uxe9uxb9uxe7uxe7uxe6ux17e}

详� \texttt{\ template\textbackslash{}thesis.typ\ }

\subsection{FAQ}\label{faq}

\subsubsection{为什么 XXX
的功能ä¸?能ç''¨/ä¸?符å?ˆé¢„期?}\label{uxe4uxbauxe4uxe4uxb9ux2c6-xxx-uxe7ux161uxe5ux161uxffuxe8ux192uxbduxe4uxe8ux192uxbduxe7uxe4uxe7uxe5ux2c6uxe9uxe6ux153uxffuxefuxbcuxff}

\begin{enumerate}
\tightlist
\item
  å\ldots ˆå?‚考
  \href{https://typst-doc-cn.github.io/docs/chinese/}{Typst
  中æ--‡æ''¯æŒ?相å\ldots³é---®é¢˜} ,以å?Š
  \href{https://typst.app/docs/}{Typst 官æ--¹æ--‡æ¡£} 与
  \href{https://typst.app/universe}{tpyst.app/universe ä»``åº``}
  ,了解相å\ldots³é---®é¢˜è¿›å±•æˆ--解决æ--¹æ¡ˆ
\item
  如果在以上资æ--™ä¸­æ‰¾ä¸?到å\ldots³è?''资æ--™ï¼Œå?¯ä»¥å?‚考是å?¦åœ¨çš„
  \href{https://gitlab.com/sysu-gitlab/thesis-template/better-thesis/-/issues}{issue
  åˆ---表} 中能找到相å\ldots³é---®é¢˜ä¸Žè¿›å±•ã€‚
\item
  如果ä¾?然没有线索,欢迎å??馈é---®é¢˜ï¼ˆ
  \href{https://gitlab.com/sysu-gitlab/thesis-template/better-thesis/-/issues}{GitLab
  issue} /
  \href{mailto:contact-project+sysu-gitlab-thesis-template-better-thesis-57823416-issue-@incoming.gitlab.com}{邮件}
  )
\end{enumerate}

\subsubsection{\texorpdfstring{为什么学æ~¡å­¦ä½?论æ--‡å·²ç»?有了
\href{https://github.com/SYSU-SCC/sysu-thesis}{LaTeX 模�} ,还有
Typst
模æ?¿ï¼Ÿ}{为什么学æ~¡å­¦ä½?论æ--‡å·²ç»?有了 LaTeX 模æ?¿ ,还有 Typst 模æ?¿ï¼Ÿ}}\label{uxe4uxbauxe4uxe4uxb9ux2c6uxe5uxe6-uxe5uxe4uxbduxe8uxbauxe6uxe5uxb2uxe7uxe6ux153uxe4uxba-latex-uxe6uxe6-uxefuxbcux153uxe8uxe6ux153-typst-uxe6uxe6uxefuxbcuxff}

\begin{itemize}
\tightlist
\item
  �述 LaTeX
  模æ?¿ç›®å‰?ä»\ldots 有计ç®---机学院官æ--¹æŒ‡å®šä½¿ç''¨ï¼Œå\ldots¶ä»--学院并没有统一指定
\item
  考è™`到 LaTeX
  对于大部分é?žè®¡ç®---机/ç?†å·¥ç§`çš„å­¦ç''Ÿå\ldots¥é---¨æˆ?本æ¯''较高,å›~此有å¿\ldots è¦?æ??供一ç§?æ›´åŠ~简æ´?æ¸\ldots 晰并ä¸''æ--¹ä¾¿çš„论æ--‡æ¨¡æ?¿ï¼ŒåŒ\ldots 括:

  \begin{itemize}
  \tightlist
  \item
    开箱å?³ç''¨ï¼š

    \begin{itemize}
    \tightlist
    \item
      如
      \href{https://github.com/typst/packages/raw/main/packages/preview/modern-sysu-thesis/0.3.0/\#typstapp}{å‰?æ--‡æ‰€è¿°}
      ,本模æ?¿æ??供了在线直接ç¼--è¾`/ä¿?å­˜/备份æ--¹æ¡ˆ
    \item
      本地使ç''¨æ¨¡æ?¿æ---¶ï¼Œæ¨¡æ?¿ç»„件å?¯ä»¥ç®€å?•åœ°é€šè¿‡
      \texttt{\ typst\ } å`½ä»¤è‡ªåŠ¨ç®¡ç?†å®‰è£
    \end{itemize}
  \item
    语法简�:typst 是与 markdown
    类似的æ~‡è®°æ€§è¯­è¨€ï¼Œå?¯ä»¥é€šè¿‡æ~‡è®°çš„æ--¹å¼?æ?¥è½»æ?¾æŽ§åˆ¶è¯­æ³•ï¼ˆå¦‚
    \texttt{\ =\ æ~‡é¢˜\ } ã€? \texttt{\ *粗体*\ } ã€?
    \texttt{\ \_斜体\_\ } \texttt{\ @引用\ } ã€? æ•°å­¦å\ldots¬å¼?
    \texttt{\ \$E\ =\ m\ c\^{}2\$\ } )
  \end{itemize}
\end{itemize}

\subsubsection{\texorpdfstring{为什么有两份 Typst 模�(
\href{https://github.com/howardlau1999/sysu-thesis-typst}{sysu-thesis-typst}
å'Œ
modern-sysu-thesis)?}{为什么有两份 Typst 模æ?¿ï¼ˆ sysu-thesis-typst å'Œ modern-sysu-thesis)?}}\label{uxe4uxbauxe4uxe4uxb9ux2c6uxe6ux153uxe4uxe4uxbd-typst-uxe6uxe6uxefuxbcux2c6-sysu-thesis-typst-uxe5ux153-modern-sysu-thesisuxefuxbcuxefuxbcuxff}

å?Žè€\ldots 是在å‰?è€\ldots 的基础上,å?Œæ---¶å?‚考
\href{https://typst.app/universe/package/modern-nju-thesis}{modern-nju-thesis}
,æ''¹é€~å?Žé€‚é\ldots?了
\href{https://typst.app/universe}{typst.app/universe}
。以å?Šï¼Œæ''¾åˆ° \href{https://github.com/sysu}{@sysu}
组织下æ??高了æ›?å\ldots‰åº¦ã€‚

\subsection{致谢}\label{uxe8uxe8}

\begin{itemize}
\tightlist
\item
  æ„Ÿè°¢
  \href{https://github.com/howardlau1999/sysu-thesis-typst}{sysu-thesis-typst}
  æ??供了中山大学的页é?¢æ~·å¼?与åˆ?版æº?ç~?
\item
  æ„Ÿè°¢
  \href{https://typst.app/universe/package/modern-nju-thesis}{modern-nju-thesis}
  æ??供了一个更好的代ç~?组织架构
\item
  感谢中山大学 Typst 模�交�群(
  \href{https://jq.qq.com/?_wv=1027&k=m58va1kd}{797942860} )�Typst
  中æ--‡äº¤æµ?群(793548390)群å?‹çš„帮助交æµ?。
\end{itemize}

\href{/app?template=modern-sysu-thesis&version=0.3.0}{Create project in
app}

\subsubsection{How to use}\label{how-to-use}

Click the button above to create a new project using this template in
the Typst app.

You can also use the Typst CLI to start a new project on your computer
using this command:

\begin{verbatim}
typst init @preview/modern-sysu-thesis:0.3.0
\end{verbatim}

\includesvg[width=0.16667in,height=0.16667in]{/assets/icons/16-copy.svg}

\subsubsection{About}\label{about}

\begin{description}
\tightlist
\item[Author s :]
\href{https://github.com/howardlau1999}{howardlau1999} \&
\href{https://github.com/huangjj27}{Sunny Huang}
\item[License:]
MIT
\item[Current version:]
0.3.0
\item[Last updated:]
June 17, 2024
\item[First released:]
May 17, 2024
\item[Archive size:]
43.5 kB
\href{https://packages.typst.org/preview/modern-sysu-thesis-0.3.0.tar.gz}{\pandocbounded{\includesvg[keepaspectratio]{/assets/icons/16-download.svg}}}
\item[Repository:]
\href{https://gitlab.com/sysu-gitlab/thesis-template/better-thesis}{GitLab}
\item[Categor y :]
\begin{itemize}
\tightlist
\item[]
\item
  \pandocbounded{\includesvg[keepaspectratio]{/assets/icons/16-mortarboard.svg}}
  \href{https://typst.app/universe/search/?category=thesis}{Thesis}
\end{itemize}
\end{description}

\subsubsection{Where to report issues?}\label{where-to-report-issues}

This template is a project of howardlau1999 and Sunny Huang . Report
issues on
\href{https://gitlab.com/sysu-gitlab/thesis-template/better-thesis}{their
repository} . You can also try to ask for help with this template on the
\href{https://forum.typst.app}{Forum} .

Please report this template to the Typst team using the
\href{https://typst.app/contact}{contact form} if you believe it is a
safety hazard or infringes upon your rights.

\phantomsection\label{versions}
\subsubsection{Version history}\label{version-history}

\begin{longtable}[]{@{}ll@{}}
\toprule\noalign{}
Version & Release Date \\
\midrule\noalign{}
\endhead
\bottomrule\noalign{}
\endlastfoot
0.3.0 & June 17, 2024 \\
\href{https://typst.app/universe/package/modern-sysu-thesis/0.2.0/}{0.2.0}
& June 10, 2024 \\
\href{https://typst.app/universe/package/modern-sysu-thesis/0.1.1/}{0.1.1}
& May 23, 2024 \\
\href{https://typst.app/universe/package/modern-sysu-thesis/0.1.0/}{0.1.0}
& May 17, 2024 \\
\end{longtable}

Typst GmbH did not create this template and cannot guarantee correct
functionality of this template or compatibility with any version of the
Typst compiler or app.


\title{typst.app/universe/package/athena-tu-darmstadt-thesis}

\phantomsection\label{banner}
\phantomsection\label{template-thumbnail}
\pandocbounded{\includegraphics[keepaspectratio]{https://packages.typst.org/preview/thumbnails/athena-tu-darmstadt-thesis-0.1.0-small.webp}}

\section{athena-tu-darmstadt-thesis}\label{athena-tu-darmstadt-thesis}

{ 0.1.0 }

Thesis template for TU Darmstadt (Technische Universität Darmstadt).

\href{/app?template=athena-tu-darmstadt-thesis&version=0.1.0}{Create
project in app}

\phantomsection\label{readme}
This \textbf{unofficial} template can be used to write in
\href{https://github.com/typst/typst}{Typst} with the corporate design
of \href{https://www.tu-darmstadt.de/}{TU Darmstadt} .

\paragraph{Disclaimer}\label{disclaimer}

Please ask your supervisor if you are allowed to use typst and this
template for your thesis or other documents. Note that this template is
not checked by TU Darmstadt for correctness. Thus, this template does
not guarantee completeness or correctness. Also, note that submission in
TUbama requires PDF/A which typst currently can’t export to (
\url{https://github.com/typst/typst/issues/2942} ). You can use a
converter to convert from the typst output to PDF/A, but check that
there are no losses during the conversion. CMYK color space support may
be required for printing which is also currently not supported by typst
( \url{https://github.com/typst/typst/issues/2942} ), but this is not
relevant when you just submit online.

\subsection{Implemented Templates}\label{implemented-templates}

The templates imitate the style of the corresponding latex templates in
\href{https://github.com/tudace/tuda_latex_templates}{tuda\_latex\_templates}
. Note that there can be visual differences between the original latex
template and the typst template (you may open an issue when you find
one).

For missing features, ideas or other problems you can just open an issue
:wink:. Contributions are also welcome.

\begin{longtable}[]{@{}
  >{\raggedright\arraybackslash}p{(\linewidth - 6\tabcolsep) * \real{0.2500}}
  >{\raggedright\arraybackslash}p{(\linewidth - 6\tabcolsep) * \real{0.2500}}
  >{\raggedright\arraybackslash}p{(\linewidth - 6\tabcolsep) * \real{0.2500}}
  >{\raggedright\arraybackslash}p{(\linewidth - 6\tabcolsep) * \real{0.2500}}@{}}
\toprule\noalign{}
\begin{minipage}[b]{\linewidth}\raggedright
Template
\end{minipage} & \begin{minipage}[b]{\linewidth}\raggedright
Preview
\end{minipage} & \begin{minipage}[b]{\linewidth}\raggedright
Example
\end{minipage} & \begin{minipage}[b]{\linewidth}\raggedright
Scope
\end{minipage} \\
\midrule\noalign{}
\endhead
\bottomrule\noalign{}
\endlastfoot
\href{https://github.com/JeyRunner/tuda-typst-templates/blob/main/templates/tudapub/tudapub.typ}{tudapub}
&
\includegraphics[width=\linewidth,height=3.125in,keepaspectratio]{https://github.com/typst/packages/raw/main/packages/preview/athena-tu-darmstadt-thesis/0.1.0/img/tudapub_prev-01.png}
& \begin{minipage}[t]{\linewidth}\raggedright
\href{https://github.com/JeyRunner/tuda-typst-templates/blob/main/example_tudapub.pdf}{example\_tudapub.pdf}\\
\href{https://github.com/JeyRunner/tuda-typst-templates/blob/main/example_tudapub.typ}{example\_tudapub.typ}\strut
\end{minipage} & Master and Bachelor thesis \\
\end{longtable}

\subsection{Usage}\label{usage}

Create a new typst project based on this template locally.

\begin{Shaded}
\begin{Highlighting}[]
\ExtensionTok{typst}\NormalTok{ init @preview/athena{-}tu{-}darmstadt{-}thesis}
\BuiltInTok{cd}\NormalTok{ athena{-}tu{-}darmstadt{-}thesis}
\end{Highlighting}
\end{Shaded}

Or create a project on the typst web app based on this template.

Or do a manual installation of this template.

For a manual setup create a folder for your writing project and download
this template into the `templates` folder:

\begin{Shaded}
\begin{Highlighting}[]
\FunctionTok{mkdir}\NormalTok{ my\_thesis }\KeywordTok{\&\&} \BuiltInTok{cd}\NormalTok{ my\_thesis}
\FunctionTok{mkdir}\NormalTok{ templates }\KeywordTok{\&\&} \BuiltInTok{cd}\NormalTok{ templates}
\FunctionTok{git}\NormalTok{ clone https://github.com/JeyRunner/tuda{-}typst{-}templates templates/}
\end{Highlighting}
\end{Shaded}

\subsubsection{Logo and Font Setup}\label{logo-and-font-setup}

Download the tud logo from
\href{https://download.hrz.tu-darmstadt.de/protected/ULB/tuda_logo.pdf}{download.hrz.tu-darmstadt.de/protected/ULB/tuda\_logo.pdf}
and put it into the \texttt{\ logos\ } folder. Now execute the following
script in the \texttt{\ logos\ } folder to convert it into an svg:

\begin{Shaded}
\begin{Highlighting}[]
\BuiltInTok{cd}\NormalTok{ logos}
\ExtensionTok{./convert\_logo.sh}
\end{Highlighting}
\end{Shaded}

Also download the required fonts \texttt{\ Roboto\ } and
\texttt{\ XCharter\ } :

\begin{Shaded}
\begin{Highlighting}[]
\BuiltInTok{cd}\NormalTok{ fonts}
\ExtensionTok{./download\_fonts.sh}
\end{Highlighting}
\end{Shaded}

Now you can install all fonts in the folders in \texttt{\ fonts\ } on
your system.

Create a main.typ file for the manual template installation.

Create a simple `main.typ` in the root folder (`my\_thesis`) of your new
project:

\begin{Shaded}
\begin{Highlighting}[]
\NormalTok{\#import }\StringTok{" templates/tuda{-}typst{-}templates/templates/tudapub/tudapub.typ"}\OperatorTok{:}\NormalTok{ tudapub}

\NormalTok{\#show}\OperatorTok{:}\NormalTok{ tudapub}\OperatorTok{.}\FunctionTok{with}\NormalTok{(}
\NormalTok{  title}\OperatorTok{:}\NormalTok{ [}
\NormalTok{    My Thesis}
\NormalTok{  ]}\OperatorTok{,}
\NormalTok{  author}\OperatorTok{:} \StringTok{"My Name"}\OperatorTok{,}
\NormalTok{  accentcolor}\OperatorTok{:} \StringTok{"3d"}
\NormalTok{)}

\OperatorTok{=}\NormalTok{ My First Chapter}
\NormalTok{Some }\BuiltInTok{Text}
\end{Highlighting}
\end{Shaded}

\subsubsection{Compile you typst file}\label{compile-you-typst-file}

\begin{Shaded}
\begin{Highlighting}[]
\ExtensionTok{typst} \AttributeTok{{-}{-}watch}\NormalTok{ main.typ }\AttributeTok{{-}{-}font{-}path}\NormalTok{ fonts/}
\end{Highlighting}
\end{Shaded}

This will watch your file and recompile it to a pdf when the file is
saved. For writing, you can use
\href{https://code.visualstudio.com/}{Vscode} with these extensions:
\href{https://marketplace.visualstudio.com/items?itemName=nvarner.typst-lsp}{Typst
LSP} and
\href{https://marketplace.visualstudio.com/items?itemName=mgt19937.typst-preview}{Typst
Preview} . Or use the \href{https://typst.app/}{typst web app} (here you
need to upload the logo and the fonts).

Note that we add \texttt{\ -\/-font-path\ } to ensure that the correct
fonts are used. Due to a bug (typst/typst\#2917 typst/typst\#2098) typst
sometimes uses the font \texttt{\ Roboto\ condensed\ } instead of
\texttt{\ Roboto\ } . To be on the safe side, double-check the embedded
fonts in the pdf (there should be no \texttt{\ Roboto\ condensed\ } ).
What also works is to uninstall/deactivate all
\texttt{\ Roboto\ condensed\ } fonts from your system.

\subsection{Todos}\label{todos}

\begin{itemize}
\tightlist
\item
  {[} {]} some bug seems to insert an empty page at the end of the
  document when content (title page) appears before this second ‘set
  page’
\item
  {[} {]} numbering/labeling of sub-equations (that are aligned with the
  other sub-equations)
\item
  {[}x{]} remove page numbers in footer before \st{and at table of
  contents}
\item
  {[}x{]} fix first-level heading page is wrong

  \begin{itemize}
  \tightlist
  \item
    in the outline, the page of the first-level heading is sometimes the
    previous page of the heading. Just appears in combination with
    \texttt{\ figure\_numbering\_per\_chapter\ } .
  \end{itemize}
\item
  {[} {]} fix referencing figures respect figure numbering when using
  \texttt{\ figure\_numbering\_per\_chapter\ }
\item
  {[} {]} first-level headings should be referenced as ‘Chapter’ not
  as ‘Sections’
\item
  {[} {]} add pages for:

  \begin{itemize}
  \tightlist
  \item
    {[}x{]} abstract
  \item
    {[} {]} list of figures, tables, … other
  \item
    {[} {]} list of abbreviations (glossary)
  \item
    {[}x{]} references
  \end{itemize}
\item
  {[} {]} references list: use same citation style is ‘numeric’ in
  latex
\item
  {[} {]} reduce vertical spacing between adjacent headings when there
  is no text in between (looks better, latex template also does this)
\item
  {[}x{]} add arguments for optional pages:

  \begin{itemize}
  \tightlist
  \item
    after title page
  \item
    before outline table of contents
  \item
    after outline table of contents
  \end{itemize}
\item
  {[} {]} fix equation numbering per chapter (somehow increases in steps
  of 2)
\item
  {[}x{]} provide some default page margins (small, medium, large)
\item
  {[} {]} \st{make all font sizes relative to the main text font size
  (e.g. headings)}
\item
  {[} {]} switch to kebab case for template, function args
\end{itemize}

\href{/app?template=athena-tu-darmstadt-thesis&version=0.1.0}{Create
project in app}

\subsubsection{How to use}\label{how-to-use}

Click the button above to create a new project using this template in
the Typst app.

You can also use the Typst CLI to start a new project on your computer
using this command:

\begin{verbatim}
typst init @preview/athena-tu-darmstadt-thesis:0.1.0
\end{verbatim}

\includesvg[width=0.16667in,height=0.16667in]{/assets/icons/16-copy.svg}

\subsubsection{About}\label{about}

\begin{description}
\tightlist
\item[Author :]
\href{https://github.com/JeyRunner}{JeyRunner}
\item[License:]
MIT
\item[Current version:]
0.1.0
\item[Last updated:]
May 22, 2024
\item[First released:]
May 22, 2024
\item[Archive size:]
15.1 kB
\href{https://packages.typst.org/preview/athena-tu-darmstadt-thesis-0.1.0.tar.gz}{\pandocbounded{\includesvg[keepaspectratio]{/assets/icons/16-download.svg}}}
\item[Repository:]
\href{https://github.com/JeyRunner/tuda-typst-templates}{GitHub}
\item[Categor y :]
\begin{itemize}
\tightlist
\item[]
\item
  \pandocbounded{\includesvg[keepaspectratio]{/assets/icons/16-mortarboard.svg}}
  \href{https://typst.app/universe/search/?category=thesis}{Thesis}
\end{itemize}
\end{description}

\subsubsection{Where to report issues?}\label{where-to-report-issues}

This template is a project of JeyRunner . Report issues on
\href{https://github.com/JeyRunner/tuda-typst-templates}{their
repository} . You can also try to ask for help with this template on the
\href{https://forum.typst.app}{Forum} .

Please report this template to the Typst team using the
\href{https://typst.app/contact}{contact form} if you believe it is a
safety hazard or infringes upon your rights.

\phantomsection\label{versions}
\subsubsection{Version history}\label{version-history}

\begin{longtable}[]{@{}ll@{}}
\toprule\noalign{}
Version & Release Date \\
\midrule\noalign{}
\endhead
\bottomrule\noalign{}
\endlastfoot
0.1.0 & May 22, 2024 \\
\end{longtable}

Typst GmbH did not create this template and cannot guarantee correct
functionality of this template or compatibility with any version of the
Typst compiler or app.


\title{typst.app/universe/package/grotesk-cv}

\phantomsection\label{banner}
\phantomsection\label{template-thumbnail}
\pandocbounded{\includegraphics[keepaspectratio]{https://packages.typst.org/preview/thumbnails/grotesk-cv-1.0.1-small.webp}}

\section{grotesk-cv}\label{grotesk-cv}

{ 1.0.1 }

A clean CV and cover letter template based on Brilliant-cv and fireside
templates.

\href{/app?template=grotesk-cv&version=1.0.1}{Create project in app}

\phantomsection\label{readme}
Version 1.0.1

{ }

grotesk-cv provides a pair of elegant and simple, one-page CV and cover
letter templates, inspired by the
\href{https://typst.app/universe/package/brilliant-cv/}{Brilliant-cv}
and \href{https://typst.app/universe/package/fireside/1.0.0/}{fireside}
templates.

\subsubsection{Features}\label{features}

\begin{itemize}
\tightlist
\item
  Templates for multilingual CV and cover letter, enabled by flag
\item
  Separation of styling and content
\item
  Customizable fonts, colors and icons
\end{itemize}

\subsection{Preview}\label{preview}

\begin{longtable}[]{@{}cc@{}}
\toprule\noalign{}
CV & Cover Letter \\
\midrule\noalign{}
\endhead
\bottomrule\noalign{}
\endlastfoot
\pandocbounded{\includegraphics[keepaspectratio]{https://raw.githubusercontent.com/AsiSkarp/grotesk-cv/main/examples/cv_example.png?raw=true}}
&
\pandocbounded{\includegraphics[keepaspectratio]{https://raw.githubusercontent.com/AsiSkarp/grotesk-cv/main/examples/cover_letter_example.png?raw=true}} \\
\end{longtable}

\subsection{Getting Started}\label{getting-started}

To edit this template, changes are mostly made in either of two places.
Changes to contact information or layout settings are made in
\texttt{\ info.toml\ } . To change the section contents, navigate to the
corresponding section file e.g. \texttt{\ content/profile.typ\ } to edit
the \textbf{Profile} section.

\subsubsection{Adding or Removing
Sections}\label{adding-or-removing-sections}

To add a new section, create a new \texttt{\ .typ\ } file in the
\texttt{\ content\ } directory with the desired section name. To include
the section in the CV, add the section at the desired position in either
left or right panes in the \texttt{\ cv.typ\ } file. To remove sections,
simply remove or comment-out the section name in the same list of
section names in the \texttt{\ cv.typ\ } file. Sections are rendered in
the order they appear in the list. The section column width can be
adjusted in the \texttt{\ info.toml\ } file under the
\texttt{\ left\_pane\_width\ } value. In the following example, the
\texttt{\ projects.typ\ } section file has been created and is included
in the left pane of the CV, and the \texttt{\ education.typ\ } section
has been removed.

\begin{Shaded}
\begin{Highlighting}[]
\NormalTok{\#let left}\OperatorTok{{-}}\NormalTok{pane }\OperatorTok{=}\NormalTok{ (}
  \StringTok{"profile"}\OperatorTok{,}
  \StringTok{"experience"}\OperatorTok{,}
  \CommentTok{//"education",}
  \StringTok{"projects"}\OperatorTok{,}
\NormalTok{)}
\end{Highlighting}
\end{Shaded}

\subsubsection{Changing Profile Photo}\label{changing-profile-photo}

To change the profile photo, upload your image to the
\texttt{\ content/img\ } folder. To enable the new image, update the
\texttt{\ profile\_image\ } value in \texttt{\ info.toml\ } with the
name of your uploaded image.

\subsubsection{Changing FontAwesome
Icons}\label{changing-fontawesome-icons}

The template uses \href{https://fontawesome.com/}{FontAwesome} for all
icons. To change an icon, change the desired icon string in the
\texttt{\ info.toml\ } file with the corresponding FontAwesome icon
name. Icon strings can be found in the
\href{https://fontawesome.com/v4/cheatsheet/}{cheat sheet} . Note that
the icon strings must be written without the \texttt{\ fa-\ } prefix. To
disable the use of icons, set the \texttt{\ include\_icons\ } value to
\texttt{\ false\ } .

\subsubsection{Customizing Contact
Information}\label{customizing-contact-information}

To change or add contact information, update the corresponding value
under \texttt{\ {[}personal.info{]}\ } in the \texttt{\ info.toml\ }
file. Information is rendered in the order it appears in the file. To
add a new contact information field, add a new variable under
\texttt{\ {[}personal.info{]}\ } with the desired string value. Next,
assign a valid FontAwesome icon string to a variable of the same name
under \texttt{\ {[}personal.icon{]}\ } . In the following example, a
homepage field has been added to the contact information.

\begin{Shaded}
\begin{Highlighting}[]
\KeywordTok{[personal.info]}
\DataTypeTok{homepage} \OperatorTok{=} \StringTok{"www.myawesomehomepage.com"}

\KeywordTok{[personal.icon]}
\DataTypeTok{homepage} \OperatorTok{=} \StringTok{"globe"}
\end{Highlighting}
\end{Shaded}

\subsubsection{Changing language}\label{changing-language}

The template provides the option to instantly change the language of the
CV and cover letter by using a variable in the \texttt{\ info.toml\ }
file. The template demonstrates the use of the \texttt{\ language\ }
variable to switch between English and Spanish, but any language can be
used, provided that the information is entered manually inside the
corresponding section files. For instance, to change the alternate
language to German, changes would have to be made in the section files
to include the German text. In the following example, the language of
the \textbf{Profile} section has been changed from Spanish to German,
and the required changes have been made in the
\texttt{\ content/profile.typ\ } file.

\begin{verbatim}
// = Summary
= #if include-icon [#fa-icon(icon) #h(5pt)] #if language == "en" [Summary] else if language == "ger" [Zusammenfassung]

#v(5pt)

#if language == "en" [

  Experienced Software Engineer specializing in artificial intelligence, machine learning, and robotics. Proficient in C++, Python, and Java, with a knack for developing sentient AI systems capable of complex decision-making. Passionate about ethical AI development and eager to contribute to groundbreaking projects in dynamic environments.

] else if language == "ger" [

  Erfahrener Software-Ingenieur, der sich auf künstliche Intelligenz, maschinelles Lernen und Robotik spezialisiert hat. Er beherrscht C++, Python und Java und hat ein Händchen für die Entwicklung empfindungsfähiger KI-Systeme, die in der Lage sind, komplexe Entscheidungen zu treffen. Leidenschaft für ethische KI-Entwicklung und bestrebt, zu bahnbrechenden Projekten in dynamischen Umgebungen beizutragen.

]
\end{verbatim}

\subsubsection{Changing Fonts}\label{changing-fonts}

If using the template online with Typst Universe, multiple font types
are available to use, a list of which can be found by pressing the
\texttt{\ Ag\ } button. To use a different font, upload a
\texttt{\ ttf\ } or \texttt{\ otf\ } file to the
\texttt{\ content/fonts\ } folder and update the \texttt{\ font\ } value
in the \texttt{\ info.toml\ } file with the name of the uploaded font.

\subsubsection{Installation}\label{installation}

To use the template offline, clone this repository to your local
machine. Typst can be used and rendered offline by installing the Typst
CLI. My preferred workflow has been to use VSCode with the
\href{https://github.com/Myriad-Dreamin/tinymist/releases}{Tinymist}
extension, which provides LSP support, syntax highlighting, and error
checking, live rendered previews and automatic exports to PDF.

Please feel free to fork this repository and create PRs for any changes
or improvements.

\href{/app?template=grotesk-cv&version=1.0.1}{Create project in app}

\subsubsection{How to use}\label{how-to-use}

Click the button above to create a new project using this template in
the Typst app.

You can also use the Typst CLI to start a new project on your computer
using this command:

\begin{verbatim}
typst init @preview/grotesk-cv:1.0.1
\end{verbatim}

\includesvg[width=0.16667in,height=0.16667in]{/assets/icons/16-copy.svg}

\subsubsection{About}\label{about}

\begin{description}
\tightlist
\item[Author :]
\href{https://github.com/AsiSkarp}{�sbjörn Skarphéðinsson}
\item[License:]
Unlicense
\item[Current version:]
1.0.1
\item[Last updated:]
October 21, 2024
\item[First released:]
September 30, 2024
\item[Archive size:]
1.38 MB
\href{https://packages.typst.org/preview/grotesk-cv-1.0.1.tar.gz}{\pandocbounded{\includesvg[keepaspectratio]{/assets/icons/16-download.svg}}}
\item[Repository:]
\href{https://github.com/AsiSkarp/grotesk-cv}{GitHub}
\item[Categor ies :]
\begin{itemize}
\tightlist
\item[]
\item
  \pandocbounded{\includesvg[keepaspectratio]{/assets/icons/16-user.svg}}
  \href{https://typst.app/universe/search/?category=cv}{CV}
\item
  \pandocbounded{\includesvg[keepaspectratio]{/assets/icons/16-layout.svg}}
  \href{https://typst.app/universe/search/?category=layout}{Layout}
\end{itemize}
\end{description}

\subsubsection{Where to report issues?}\label{where-to-report-issues}

This template is a project of �sbjörn Skarphéðinsson . Report issues
on \href{https://github.com/AsiSkarp/grotesk-cv}{their repository} . You
can also try to ask for help with this template on the
\href{https://forum.typst.app}{Forum} .

Please report this template to the Typst team using the
\href{https://typst.app/contact}{contact form} if you believe it is a
safety hazard or infringes upon your rights.

\phantomsection\label{versions}
\subsubsection{Version history}\label{version-history}

\begin{longtable}[]{@{}ll@{}}
\toprule\noalign{}
Version & Release Date \\
\midrule\noalign{}
\endhead
\bottomrule\noalign{}
\endlastfoot
1.0.1 & October 21, 2024 \\
\href{https://typst.app/universe/package/grotesk-cv/1.0.0/}{1.0.0} &
October 17, 2024 \\
\href{https://typst.app/universe/package/grotesk-cv/0.1.6/}{0.1.6} &
October 11, 2024 \\
\href{https://typst.app/universe/package/grotesk-cv/0.1.5/}{0.1.5} &
October 10, 2024 \\
\href{https://typst.app/universe/package/grotesk-cv/0.1.4/}{0.1.4} &
October 9, 2024 \\
\href{https://typst.app/universe/package/grotesk-cv/0.1.3/}{0.1.3} &
October 8, 2024 \\
\href{https://typst.app/universe/package/grotesk-cv/0.1.2/}{0.1.2} &
October 7, 2024 \\
\href{https://typst.app/universe/package/grotesk-cv/0.1.1/}{0.1.1} &
October 2, 2024 \\
\href{https://typst.app/universe/package/grotesk-cv/0.1.0/}{0.1.0} &
September 30, 2024 \\
\end{longtable}

Typst GmbH did not create this template and cannot guarantee correct
functionality of this template or compatibility with any version of the
Typst compiler or app.


\title{typst.app/universe/package/icu-datetime}

\phantomsection\label{banner}
\section{icu-datetime}\label{icu-datetime}

{ 0.1.2 }

Date and time formatting using ICU4X via WASM

\phantomsection\label{readme}
This library is a wrapper around
\href{https://github.com/unicode-org/icu4x}{ICU4X} ’
\texttt{\ datetime\ } formatting for Typst which provides
internationalized formatting for dates, times, and timezones.

As the WASM bundle includes all localization data, it’s quite large
(about 8 MiB).

See \href{https://nerixyz.github.io/icu-typ}{nerixyz.github.io/icu-typ}
for a full API reference with more examples.

\subsection{Example}\label{example}

\begin{Shaded}
\begin{Highlighting}[]
\NormalTok{\#import "@preview/icu{-}datetime:0.1.2": fmt{-}date, fmt{-}time, fmt{-}datetime, experimental}
\NormalTok{// These functions may change at any time}
\NormalTok{\#import experimental: fmt{-}timezone, fmt{-}zoned{-}datetime}

\NormalTok{\#let day = datetime(}
\NormalTok{  year: 2024,}
\NormalTok{  month: 5,}
\NormalTok{  day: 31,}
\NormalTok{)}
\NormalTok{\#let time = datetime(}
\NormalTok{  hour: 18,}
\NormalTok{  minute: 2,}
\NormalTok{  second: 23,}
\NormalTok{)}
\NormalTok{\#let dt = datetime(}
\NormalTok{  year: 2024,}
\NormalTok{  month: 5,}
\NormalTok{  day: 31,}
\NormalTok{  hour: 18,}
\NormalTok{  minute: 2,}
\NormalTok{  second: 23,}
\NormalTok{)}
\NormalTok{\#let tz = (}
\NormalTok{  offset: "{-}07",}
\NormalTok{  iana: "America/Los\_Angeles",}
\NormalTok{  zone{-}variant: "st", // standard}
\NormalTok{)}

\NormalTok{= Dates}
\NormalTok{\#fmt{-}date(day, locale: "km", length: "full") \textbackslash{}}
\NormalTok{\#fmt{-}date(day, locale: "af", length: "full") \textbackslash{}}
\NormalTok{\#fmt{-}date(day, locale: "za", length: "full") \textbackslash{}}

\NormalTok{= Time}
\NormalTok{\#fmt{-}time(time, locale: "id", length: "medium") \textbackslash{}}
\NormalTok{\#fmt{-}time(time, locale: "en", length: "medium") \textbackslash{}}
\NormalTok{\#fmt{-}time(time, locale: "ga", length: "medium") \textbackslash{}}

\NormalTok{= Date and Time}
\NormalTok{\#fmt{-}datetime(dt, locale: "ru", date{-}length: "full") \textbackslash{}}
\NormalTok{\#fmt{-}datetime(dt, locale: "en{-}US", date{-}length: "full") \textbackslash{}}
\NormalTok{\#fmt{-}datetime(dt, locale: "zh{-}Hans{-}CN", date{-}length: "full") \textbackslash{}}
\NormalTok{\#fmt{-}datetime(dt, locale: "ar", date{-}length: "full") \textbackslash{}}
\NormalTok{\#fmt{-}datetime(dt, locale: "fi", date{-}length: "full")}

\NormalTok{= Timezones (experimental)}
\NormalTok{\#fmt{-}timezone(}
\NormalTok{  ..tz,}
\NormalTok{  local{-}date: datetime.today(),}
\NormalTok{  format: "specific{-}non{-}location{-}long"}
\NormalTok{) \textbackslash{}}
\NormalTok{\#fmt{-}timezone(}
\NormalTok{  ..tz,}
\NormalTok{  format: (}
\NormalTok{    iso8601: (}
\NormalTok{      format: "utc{-}extended",}
\NormalTok{      minutes: "required",}
\NormalTok{      seconds: "optional",}
\NormalTok{    )}
\NormalTok{  )}
\NormalTok{)}

\NormalTok{= Zoned Datetimes (experimental)}
\NormalTok{\#fmt{-}zoned{-}datetime(dt, tz) \textbackslash{}}
\NormalTok{\#fmt{-}zoned{-}datetime(dt, tz, locale: "lv") \textbackslash{}}
\NormalTok{\#fmt{-}zoned{-}datetime(dt, tz, locale: "en{-}CA{-}u{-}hc{-}h24{-}ca{-}buddhist")}
\end{Highlighting}
\end{Shaded}

\pandocbounded{\includegraphics[keepaspectratio]{https://github.com/typst/packages/raw/main/packages/preview/icu-datetime/0.1.2/res/example.png}}

Locales must be
\href{https://unicode.org/reports/tr35/tr35.html\#Unicode_locale_identifier}{Unicode
Locale Identifier} s. Use
\href{https://nerixyz.github.io/icu-typ/locale-info/}{\texttt{\ locale-info(locale)\ }}
to get information on how a locale is parsed. Unicode extensions are
supported, so you can, for example, set the hour cycle with
\texttt{\ hc-h12\ } or set the calendar with \texttt{\ ca-buddhist\ }
(e.g. \texttt{\ en-CA-u-hc-h24-ca-buddhist\ } ).

\subsection{Documentation}\label{documentation}

Documentation can be found on
\href{https://nerixyz.github.io/icu-typ}{nerixyz.github.io/icu-typ} .

\subsection{Using Locally}\label{using-locally}

Download the \href{https://github.com/Nerixyz/icu-typ/releases}{latest
release} , unzip it to your
\href{https://github.com/typst/packages\#local-packages}{local Typst
packages} , and use \texttt{\ \#import\ "@local/icu-datetime:0.1.2"\ } .

\subsection{Building}\label{building}

To build the library, you need to have
\href{https://www.rust-lang.org/}{Rust} ,
\href{https://just.systems/}{just} , and
\href{https://github.com/WebAssembly/binaryen}{\texttt{\ wasm-opt\ }}
installed.

\begin{Shaded}
\begin{Highlighting}[]
\ExtensionTok{just}\NormalTok{ build}
\CommentTok{\# to deploy the package locally, use \textasciigrave{}just deploy\textasciigrave{}}
\end{Highlighting}
\end{Shaded}

While developing, you can symlink the WASM file into the root of the
repository (it’s in the \texttt{\ .gitignore\ } ):

\begin{Shaded}
\begin{Highlighting}[]
\CommentTok{\# Windows (PowerShell)}
\ExtensionTok{New{-}Item}\NormalTok{ icu{-}datetime.wasm }\AttributeTok{{-}ItemType}\NormalTok{ SymbolicLink }\AttributeTok{{-}Value}\NormalTok{ ./target/wasm32{-}unknown{-}unknown/debug/icu\_typ.wasm}
\CommentTok{\# Unix}
\FunctionTok{ln} \AttributeTok{{-}s}\NormalTok{ ./target/wasm32{-}unknown{-}unknown/debug/icu\_typ.wasm icu{-}datetime.wasm}
\end{Highlighting}
\end{Shaded}

Use \texttt{\ cargo\ b\ -\/-target\ wasm32-unknown-unknown\ } to build
in debug mode.

\subsubsection{How to add}\label{how-to-add}

Copy this into your project and use the import as
\texttt{\ icu-datetime\ }

\begin{verbatim}
#import "@preview/icu-datetime:0.1.2"
\end{verbatim}

\includesvg[width=0.16667in,height=0.16667in]{/assets/icons/16-copy.svg}

Check the docs for
\href{https://typst.app/docs/reference/scripting/\#packages}{more
information on how to import packages} .

\subsubsection{About}\label{about}

\begin{description}
\tightlist
\item[Author :]
\href{https://github.com/Nerixyz}{Nerixyz}
\item[License:]
MIT
\item[Current version:]
0.1.2
\item[Last updated:]
June 14, 2024
\item[First released:]
June 3, 2024
\item[Minimum Typst version:]
0.11.0
\item[Archive size:]
1.55 MB
\href{https://packages.typst.org/preview/icu-datetime-0.1.2.tar.gz}{\pandocbounded{\includesvg[keepaspectratio]{/assets/icons/16-download.svg}}}
\item[Repository:]
\href{https://github.com/Nerixyz/icu-typ}{GitHub}
\item[Categor y :]
\begin{itemize}
\tightlist
\item[]
\item
  \pandocbounded{\includesvg[keepaspectratio]{/assets/icons/16-world.svg}}
  \href{https://typst.app/universe/search/?category=languages}{Languages}
\end{itemize}
\end{description}

\subsubsection{Where to report issues?}\label{where-to-report-issues}

This package is a project of Nerixyz . Report issues on
\href{https://github.com/Nerixyz/icu-typ}{their repository} . You can
also try to ask for help with this package on the
\href{https://forum.typst.app}{Forum} .

Please report this package to the Typst team using the
\href{https://typst.app/contact}{contact form} if you believe it is a
safety hazard or infringes upon your rights.

\phantomsection\label{versions}
\subsubsection{Version history}\label{version-history}

\begin{longtable}[]{@{}ll@{}}
\toprule\noalign{}
Version & Release Date \\
\midrule\noalign{}
\endhead
\bottomrule\noalign{}
\endlastfoot
0.1.2 & June 14, 2024 \\
\href{https://typst.app/universe/package/icu-datetime/0.1.1/}{0.1.1} &
June 10, 2024 \\
\href{https://typst.app/universe/package/icu-datetime/0.1.0/}{0.1.0} &
June 3, 2024 \\
\end{longtable}

Typst GmbH did not create this package and cannot guarantee correct
functionality of this package or compatibility with any version of the
Typst compiler or app.


\title{typst.app/universe/package/typographix-polytechnique-reports}

\phantomsection\label{banner}
\phantomsection\label{template-thumbnail}
\pandocbounded{\includegraphics[keepaspectratio]{https://packages.typst.org/preview/thumbnails/typographix-polytechnique-reports-0.1.4-small.webp}}

\section{typographix-polytechnique-reports}\label{typographix-polytechnique-reports}

{ 0.1.4 }

A report template for Polytechnique students (from TypographiX).

\href{/app?template=typographix-polytechnique-reports&version=0.1.4}{Create
project in app}

\phantomsection\label{readme}
A Typst package for Polytechnique student reports.

For a short introduction to Typst features and detailled information
about the package, check the
\href{https://github.com/remigerme/typst-polytechnique/blob/main/guide.pdf}{guide}
(available from the repo only).

\subsection{Usage}\label{usage}

If you want to use it on local, make sure you have the font “New
Computer Modern Sans� installed.

Define variables at the top of the template :

\begin{Shaded}
\begin{Highlighting}[]
\NormalTok{\#let title = "Rapport de stage en entreprise sur plusieurs lignes automatiquement"}
\NormalTok{\#let subtitle = "Un sous{-}titre pour expliquer ce titre"}
\NormalTok{\#let logo = image("path/to/my{-}logo.png")}
\NormalTok{\#let logo{-}horizontal = true}
\NormalTok{\#let short{-}title = "Rapport de stage"}
\NormalTok{\#let authors = ("Rémi Germe")}
\NormalTok{\#let date{-}start = datetime(year: 2024, month: 06, day: 05)}
\NormalTok{\#let date{-}end = datetime(year: 2024, month: 09, day: 05)}
\NormalTok{\#let despair{-}mode = false}

\NormalTok{\#set text(lang: "fr")}
\end{Highlighting}
\end{Shaded}

These variables will be used for PDF metadata, default cover page and
default header.

\subsection{Contributing}\label{contributing}

Contributions are welcomed ! See
\href{https://github.com/typst/packages/raw/main/packages/preview/typographix-polytechnique-reports/0.1.4/CONTRIBUTING.md}{contribution
guidelines} .

\subsection{Todo}\label{todo}

\begin{itemize}
\tightlist
\item
  {[} {]} heading not at the end of a page : might be tricky
\item
  {[}x{]} first line indent
\item
  {[} {]} better spacing between elements
\item
  {[}x{]} handle logos on cover page
\item
  {[}x{]} \st{handle logos on header} : feature canceled
\end{itemize}

\href{/app?template=typographix-polytechnique-reports&version=0.1.4}{Create
project in app}

\subsubsection{How to use}\label{how-to-use}

Click the button above to create a new project using this template in
the Typst app.

You can also use the Typst CLI to start a new project on your computer
using this command:

\begin{verbatim}
typst init @preview/typographix-polytechnique-reports:0.1.4
\end{verbatim}

\includesvg[width=0.16667in,height=0.16667in]{/assets/icons/16-copy.svg}

\subsubsection{About}\label{about}

\begin{description}
\tightlist
\item[Author :]
\href{https://github.com/remigerme}{Rémi Germe}
\item[License:]
MIT
\item[Current version:]
0.1.4
\item[Last updated:]
September 17, 2024
\item[First released:]
August 1, 2024
\item[Archive size:]
166 kB
\href{https://packages.typst.org/preview/typographix-polytechnique-reports-0.1.4.tar.gz}{\pandocbounded{\includesvg[keepaspectratio]{/assets/icons/16-download.svg}}}
\item[Repository:]
\href{https://github.com/remigerme/typst-polytechnique}{GitHub}
\item[Categor y :]
\begin{itemize}
\tightlist
\item[]
\item
  \pandocbounded{\includesvg[keepaspectratio]{/assets/icons/16-speak.svg}}
  \href{https://typst.app/universe/search/?category=report}{Report}
\end{itemize}
\end{description}

\subsubsection{Where to report issues?}\label{where-to-report-issues}

This template is a project of Rémi Germe . Report issues on
\href{https://github.com/remigerme/typst-polytechnique}{their
repository} . You can also try to ask for help with this template on the
\href{https://forum.typst.app}{Forum} .

Please report this template to the Typst team using the
\href{https://typst.app/contact}{contact form} if you believe it is a
safety hazard or infringes upon your rights.

\phantomsection\label{versions}
\subsubsection{Version history}\label{version-history}

\begin{longtable}[]{@{}ll@{}}
\toprule\noalign{}
Version & Release Date \\
\midrule\noalign{}
\endhead
\bottomrule\noalign{}
\endlastfoot
0.1.4 & September 17, 2024 \\
\href{https://typst.app/universe/package/typographix-polytechnique-reports/0.1.3/}{0.1.3}
& August 8, 2024 \\
\href{https://typst.app/universe/package/typographix-polytechnique-reports/0.1.2/}{0.1.2}
& August 1, 2024 \\
\end{longtable}

Typst GmbH did not create this template and cannot guarantee correct
functionality of this template or compatibility with any version of the
Typst compiler or app.


\title{typst.app/universe/package/embiggen}

\phantomsection\label{banner}
\section{embiggen}\label{embiggen}

{ 0.0.1 }

LaTeX-like delimeter sizing in Typst

\phantomsection\label{readme}
Get LaTeX-like delimeter sizing in Typst!

\subsection{Usage}\label{usage}

\begin{Shaded}
\begin{Highlighting}[]
\NormalTok{\#import "@preview/embiggen:0.0.1": *}

\NormalTok{= embiggen}

\NormalTok{Here\textquotesingle{}s an equation of sorts:}

\NormalTok{$ \{lr(1/2x\^{}2|)\^{}(x=n)\_(x=0) + (2x+3)\} $}

\NormalTok{And here are some bigger versions of it:}

\NormalTok{$ \{big(1/2x\^{}2|)\^{}(x=n)\_(x=0) + big((2x+3))\} $}
\NormalTok{$ \{Big(1/2x\^{}2|)\^{}(x=n)\_(x=0) + Big((2x+3))\} $}
\NormalTok{$ \{bigg(1/2x\^{}2|)\^{}(x=n)\_(x=0) + bigg((2x+3))\} $}
\NormalTok{$ \{Bigg(1/2x\^{}2|)\^{}(x=n)\_(x=0) + Bigg((2x+3))\} $}

\NormalTok{And now, some smaller versions (\#text([\#link("https://x.com/tsoding/status/1756517251497255167", "cAn YoUr LaTeX dO tHaT?")], fill: rgb(50, 20, 200), font: "Noto Mono")):}

\NormalTok{$ small(1/2x\^{}2|)\^{}(x=n)\_(x=0) $}
\NormalTok{$ Small(1/2x\^{}2|)\^{}(x=n)\_(x=0) $}
\NormalTok{$ smalll(1/2x\^{}2|)\^{}(x=n)\_(x=0) $}
\NormalTok{$ Smalll(1/2x\^{}2|)\^{}(x=n)\_(x=0) $}
\end{Highlighting}
\end{Shaded}

\subsection{Functions}\label{functions}

\subsubsection{big(…)}\label{biguxe2}

Applies a scale factor of \texttt{\ 125\%\ } to \texttt{\ \#lr\ }
pre-determined scale. Delimeters are enlarged by this amount compared to
what \texttt{\ \#lr\ } would normally do.

\subsubsection{Big(…)}\label{biguxe2-1}

Like \texttt{\ big(...)\ } , but applies a scale factor of
\texttt{\ 156.25\%\ } .

\subsubsection{bigg(…)}\label{bigguxe2}

Like \texttt{\ big(...)\ } , but applies a scale factor of
\texttt{\ 195.313\%\ } .

\subsubsection{Bigg(…)}\label{bigguxe2-1}

Like \texttt{\ big(...)\ } , but applies a scale factor of
\texttt{\ 244.141\%\ } .

\subsubsection{small(…)}\label{smalluxe2}

Applies a scale factor of \texttt{\ 80\%\ } to \texttt{\ \#lr\ }
pre-determined scale. Delimeters are shrunk by this amount compared to
what \texttt{\ \#lr\ } would normally do. This does \emph{not} exist in
standard LaTeX, but is necessary in this package because these functions
scale the output of \texttt{\ \#lr\ } , so delimeter sizes will get
larger depending on the content.

\subsubsection{Small(…)}\label{smalluxe2-1}

Like \texttt{\ small(...)\ } , but applies a scale factor of
\texttt{\ 64\%\ } .

\subsubsection{smalll(…)}\label{smallluxe2}

Like \texttt{\ small(...)\ } , but applies a scale factor of
\texttt{\ 51.2\%\ } .

\subsubsection{Smalll(…)}\label{smallluxe2-1}

Like \texttt{\ small(...)\ } , but applies a scale factor of
\texttt{\ 40.96\%\ } .

\subsubsection{How to add}\label{how-to-add}

Copy this into your project and use the import as \texttt{\ embiggen\ }

\begin{verbatim}
#import "@preview/embiggen:0.0.1"
\end{verbatim}

\includesvg[width=0.16667in,height=0.16667in]{/assets/icons/16-copy.svg}

Check the docs for
\href{https://typst.app/docs/reference/scripting/\#packages}{more
information on how to import packages} .

\subsubsection{About}\label{about}

\begin{description}
\tightlist
\item[Author :]
\href{mailto:dev.quantum9innovation@gmail.com}{Ananth Venkatesh}
\item[License:]
GPL-3.0-or-later
\item[Current version:]
0.0.1
\item[Last updated:]
June 18, 2024
\item[First released:]
June 18, 2024
\item[Archive size:]
13.6 kB
\href{https://packages.typst.org/preview/embiggen-0.0.1.tar.gz}{\pandocbounded{\includesvg[keepaspectratio]{/assets/icons/16-download.svg}}}
\item[Categor ies :]
\begin{itemize}
\tightlist
\item[]
\item
  \pandocbounded{\includesvg[keepaspectratio]{/assets/icons/16-text.svg}}
  \href{https://typst.app/universe/search/?category=text}{Text}
\item
  \pandocbounded{\includesvg[keepaspectratio]{/assets/icons/16-speak.svg}}
  \href{https://typst.app/universe/search/?category=report}{Report}
\item
  \pandocbounded{\includesvg[keepaspectratio]{/assets/icons/16-atom.svg}}
  \href{https://typst.app/universe/search/?category=paper}{Paper}
\end{itemize}
\end{description}

\subsubsection{Where to report issues?}\label{where-to-report-issues}

This package is a project of Ananth Venkatesh . You can also try to ask
for help with this package on the \href{https://forum.typst.app}{Forum}
.

Please report this package to the Typst team using the
\href{https://typst.app/contact}{contact form} if you believe it is a
safety hazard or infringes upon your rights.

\phantomsection\label{versions}
\subsubsection{Version history}\label{version-history}

\begin{longtable}[]{@{}ll@{}}
\toprule\noalign{}
Version & Release Date \\
\midrule\noalign{}
\endhead
\bottomrule\noalign{}
\endlastfoot
0.0.1 & June 18, 2024 \\
\end{longtable}

Typst GmbH did not create this package and cannot guarantee correct
functionality of this package or compatibility with any version of the
Typst compiler or app.


\title{typst.app/universe/package/kinase}

\phantomsection\label{banner}
\section{kinase}\label{kinase}

{ 0.1.0 }

Easy styling for different link types like mails and urls.

\phantomsection\label{readme}
Package for easy styling of links. See
\href{https://github.com/typst/packages/raw/main/packages/preview/kinase/0.1.0/docs/manual.pdf}{Docs}
for a detailed guide. Below is an example of the functionality that is
added. The problem the package solves is that different link types
cannot be styled seperatly, but are recognized as such. This package
allows for easy styling of phone numbers, urls and mail addresses. It
provides helper functions that return regex patterns for the most common
use cases.

\begin{Shaded}
\begin{Highlighting}[]
\NormalTok{\#import "@previes/kinase:0.0.1"}

\NormalTok{\#show: make{-}link}

\NormalTok{// Insert some rules}
\NormalTok{\#update{-}link{-}style(key: l{-}mailto(), value: it =\textgreater{} strong(it), )}
\NormalTok{\#update{-}link{-}style(key: l{-}url(base: "typst\textbackslash{}.app"), value: it =\textgreater{} emph(it))}
\NormalTok{\#update{-}link{-}style(key: l{-}url(base: "google\textbackslash{}.com"), before: l{-}url(base: "typst\textbackslash{}.app"), value: it =\textgreater{} highlight(it))}
\NormalTok{\#update{-}link{-}style(key: l{-}url(base: "typst\textbackslash{}.app/docs"), value: it =\textgreater{} strong(it), before: l{-}url(base: "typst\textbackslash{}.app"))}

\NormalTok{\#link("mailto:john.smith@typst.org") \textbackslash{}}

\NormalTok{\#link("https://www.typst.app/docs")}

\NormalTok{\#link("typst.app")}

\NormalTok{\#link("+49 2422424422")}
\end{Highlighting}
\end{Shaded}

\pandocbounded{\includegraphics[keepaspectratio]{https://github.com/typst/packages/raw/main/packages/preview/kinase/0.1.0/ressources/example.png}}

\subsubsection{How to add}\label{how-to-add}

Copy this into your project and use the import as \texttt{\ kinase\ }

\begin{verbatim}
#import "@preview/kinase:0.1.0"
\end{verbatim}

\includesvg[width=0.16667in,height=0.16667in]{/assets/icons/16-copy.svg}

Check the docs for
\href{https://typst.app/docs/reference/scripting/\#packages}{more
information on how to import packages} .

\subsubsection{About}\label{about}

\begin{description}
\tightlist
\item[Author :]
Lennart Schuster
\item[License:]
MIT
\item[Current version:]
0.1.0
\item[Last updated:]
May 16, 2024
\item[First released:]
May 16, 2024
\item[Archive size:]
3.10 kB
\href{https://packages.typst.org/preview/kinase-0.1.0.tar.gz}{\pandocbounded{\includesvg[keepaspectratio]{/assets/icons/16-download.svg}}}
\end{description}

\subsubsection{Where to report issues?}\label{where-to-report-issues}

This package is a project of Lennart Schuster . You can also try to ask
for help with this package on the \href{https://forum.typst.app}{Forum}
.

Please report this package to the Typst team using the
\href{https://typst.app/contact}{contact form} if you believe it is a
safety hazard or infringes upon your rights.

\phantomsection\label{versions}
\subsubsection{Version history}\label{version-history}

\begin{longtable}[]{@{}ll@{}}
\toprule\noalign{}
Version & Release Date \\
\midrule\noalign{}
\endhead
\bottomrule\noalign{}
\endlastfoot
0.1.0 & May 16, 2024 \\
\end{longtable}

Typst GmbH did not create this package and cannot guarantee correct
functionality of this package or compatibility with any version of the
Typst compiler or app.


\title{typst.app/universe/package/isc-hei-report}

\phantomsection\label{banner}
\phantomsection\label{template-thumbnail}
\pandocbounded{\includegraphics[keepaspectratio]{https://packages.typst.org/preview/thumbnails/isc-hei-report-0.1.5-small.webp}}

\section{isc-hei-report}\label{isc-hei-report}

{ 0.1.5 }

An official report template for the \textquotesingle Informatique et
systèmes de communication\textquotesingle{} (ISC) bachelor degree
programme at the School of Engineering (HEI) in Sion, Switzerland.

{ } Officially affiliated

\href{/app?template=isc-hei-report&version=0.1.5}{Create project in app}

\phantomsection\label{readme}
\pandocbounded{\includegraphics[keepaspectratio]{https://img.shields.io/github/stars/ISC-HEI/isc-hei-report}}
\pandocbounded{\includegraphics[keepaspectratio]{https://img.shields.io/github/v/release/ISC-HEI/isc-hei-report?include_prereleases}}

\href{https://hevs.ch/isc}{\includegraphics[width=0.5\linewidth,height=\textheight,keepaspectratio]{https://raw.githubusercontent.com/ISC-HEI/isc_logos/4f8d335f7f4b99d3d83ee579ef334c201a15166a/ISC\%20Logo\%20inline\%20v1.png?raw=true}}

This is an official template for students reports for the
\href{https://isc.hevs.ch/}{ISC degree programme} at the School of
engineering in Sion.

\subsection{Using the template}\label{using-the-template}

In the \texttt{\ Typst\ } web application, start with the
\texttt{\ isc-hei-report\ } and voilÃ~ ! Using the CLI, you can
initialize the project with the command :

\begin{Shaded}
\begin{Highlighting}[]
\ExtensionTok{typst}\NormalTok{ init @preview/isc{-}hei{-}report:0.1.5}
\end{Highlighting}
\end{Shaded}

This template will initialize an sample report with sensible default
values.

\subsection{Installing fonts locally}\label{installing-fonts-locally}

If you are running \texttt{\ typst\ } locally, you might miss some of
the required fonts. For your convenience, a font download script is
included in this repos. As all the fonts are released under the
\href{https://openfontlicense.org/}{SIL Open Font License} , there are
no file inclusion issues here.

To the install the fonts locally in a Linux environment, simply type

\begin{Shaded}
\begin{Highlighting}[]
\BuiltInTok{source}\NormalTok{ install\_fonts.sh}
\end{Highlighting}
\end{Shaded}

from within the \texttt{\ fonts\ } directory.

\subsection{PDF images inclusion}\label{pdf-images-inclusion}

Unfortunately, \texttt{\ typst\ } does not support PDF file types
inclusion at the time of writing this documentation. As a temporary
workaround, PDF files can be converted to SVG via \texttt{\ pdf2svg\ } .

When used locally, creating a PDF is straightforward with the command

\begin{Shaded}
\begin{Highlighting}[]
\ExtensionTok{typst}\NormalTok{ compile report.typ}
\end{Highlighting}
\end{Shaded}

Even nicer, the following command compiles the report every time the
file is modified.

\begin{Shaded}
\begin{Highlighting}[]
\ExtensionTok{typst}\NormalTok{ watch report.typ}
\end{Highlighting}
\end{Shaded}

Another nice possibility is of course to use a VScod{[}e \textbar{}
ium{]} via the \texttt{\ Typst\ LSP\ } plugin which enables direct
compilation.

In the future, several things \emph{might} be updated, such as :

\begin{itemize}
\tightlist
\item
  {[} {]} State diagrams and UML diagrams examples
\item
  {[} {]} Glossary inclusion
\item
  {[} {]} Master thesis version of this template
\item
  {[} {]} Themes for code
\item
  {[} {]} Other nice things
\item
  {[}x{]} Appendix
\item
  {[}x{]} Acronyms inclusion
\item
  {[}x{]} Basic support for including code files
\end{itemize}

If you need any help for installing or running those tools, do not
hesitate to get in touch with its maintainer
\href{https://github.com/pmudry}{pmudry} .

You can of course also propose changes using PR or raise issues if
something is not clear. Have fun writing reports!

\href{/app?template=isc-hei-report&version=0.1.5}{Create project in app}

\subsubsection{How to use}\label{how-to-use}

Click the button above to create a new project using this template in
the Typst app.

You can also use the Typst CLI to start a new project on your computer
using this command:

\begin{verbatim}
typst init @preview/isc-hei-report:0.1.5
\end{verbatim}

\includesvg[width=0.16667in,height=0.16667in]{/assets/icons/16-copy.svg}

\subsubsection{About}\label{about}

\begin{description}
\tightlist
\item[Author :]
Pierre-André Mudry
\item[License:]
MIT
\item[Current version:]
0.1.5
\item[Last updated:]
June 17, 2024
\item[First released:]
May 1, 2024
\item[Minimum Typst version:]
0.11.1
\item[Archive size:]
680 kB
\href{https://packages.typst.org/preview/isc-hei-report-0.1.5.tar.gz}{\pandocbounded{\includesvg[keepaspectratio]{/assets/icons/16-download.svg}}}
\item[Verification:]
We verified that the author is affiliated with their institution
\pandocbounded{\includesvg[keepaspectratio]{/assets/icons/16-verified.svg}}
\item[Repository:]
\href{https://github.com/ISC-HEI/ISC-report}{GitHub}
\item[Discipline s :]
\begin{itemize}
\tightlist
\item[]
\item
  \href{https://typst.app/universe/search/?discipline=computer-science}{Computer
  Science}
\item
  \href{https://typst.app/universe/search/?discipline=engineering}{Engineering}
\end{itemize}
\item[Categor y :]
\begin{itemize}
\tightlist
\item[]
\item
  \pandocbounded{\includesvg[keepaspectratio]{/assets/icons/16-speak.svg}}
  \href{https://typst.app/universe/search/?category=report}{Report}
\end{itemize}
\end{description}

\subsubsection{Where to report issues?}\label{where-to-report-issues}

This template is a project of Pierre-André Mudry . Report issues on
\href{https://github.com/ISC-HEI/ISC-report}{their repository} . You can
also try to ask for help with this template on the
\href{https://forum.typst.app}{Forum} .

Please report this template to the Typst team using the
\href{https://typst.app/contact}{contact form} if you believe it is a
safety hazard or infringes upon your rights.

\phantomsection\label{versions}
\subsubsection{Version history}\label{version-history}

\begin{longtable}[]{@{}ll@{}}
\toprule\noalign{}
Version & Release Date \\
\midrule\noalign{}
\endhead
\bottomrule\noalign{}
\endlastfoot
0.1.5 & June 17, 2024 \\
\href{https://typst.app/universe/package/isc-hei-report/0.1.3/}{0.1.3} &
June 13, 2024 \\
\href{https://typst.app/universe/package/isc-hei-report/0.1.0/}{0.1.0} &
May 1, 2024 \\
\end{longtable}

Typst GmbH did not create this template and cannot guarantee correct
functionality of this template or compatibility with any version of the
Typst compiler or app.


\title{typst.app/universe/package/zhconv}

\phantomsection\label{banner}
\section{zhconv}\label{zhconv}

{ 0.3.1 }

Convert Chinese text between Traditional/Simplified and regional
variants. 中æ--‡ç®€ç¹?å?Šåœ°å?€è©žè½‰æ?›

\phantomsection\label{readme}
zhconv-typst converts Chinese text between Traditional, Simplified and
regional variants in typst, utilizing
\href{https://github.com/Gowee/zhconv-rs}{zhconv-rs} .

\subsection{Usage}\label{usage}

To use the \texttt{\ zhconv\ } plugin in your Typst project, import it
as follows:

\begin{Shaded}
\begin{Highlighting}[]
\NormalTok{\#import "@preview/zhconv:0.3.1": zhconv}
\end{Highlighting}
\end{Shaded}

\subsubsection{Text Conversion}\label{text-conversion}

The primary function provided by this package is \texttt{\ zhconv\ } ,
which converts strings or nested contents to a target Chinese variant.

\begin{Shaded}
\begin{Highlighting}[]
\NormalTok{\#zhconv(content, "target{-}variant", wikitext: false)}
\end{Highlighting}
\end{Shaded}

\begin{itemize}
\tightlist
\item
  \texttt{\ content\ } : The text or content to be converted.
\item
  \texttt{\ target-variant\ } : The target Chinese variant (e.g.,
  \texttt{\ "zh-hant"\ } , \texttt{\ "zh-hans"\ } , \texttt{\ "zh-cn"\ }
  , \texttt{\ "zh-tw"\ } , \texttt{\ "zh-hk"\ } ).
\item
  \texttt{\ wikitext\ } : An optional boolean flag to specify if the
  text should be processed as wikitext (default is \texttt{\ false\ } ).
\end{itemize}

\paragraph{Example}\label{example}

\subparagraph{Convert a string}\label{convert-a-string}

\begin{Shaded}
\begin{Highlighting}[]
\NormalTok{\#let text = "互联网"}
\NormalTok{Original: \#text}
\NormalTok{{-} \#emph([zh{-}HK]): \#zhconv(text, "zh{-}hk")}
\NormalTok{{-} \#emph([zh{-}TW]): \#zhconv(text, "zh{-}tw")}
\end{Highlighting}
\end{Shaded}

\subparagraph{Convert nested contents}\label{convert-nested-contents}

\begin{Shaded}
\begin{Highlighting}[]
\NormalTok{\#zhconv([}
\NormalTok{柳外輕雷池上雨 \textbackslash{}}
\NormalTok{雨聲滴碎荷聲 \textbackslash{}}

\NormalTok{小樓西角斷虹明 \textbackslash{}}
\NormalTok{闌干倚處 \textbackslash{}}
\NormalTok{待得月華生 \textbackslash{}}
\NormalTok{], "zh{-}hans")}
\end{Highlighting}
\end{Shaded}

\subsubsection{How to add}\label{how-to-add}

Copy this into your project and use the import as \texttt{\ zhconv\ }

\begin{verbatim}
#import "@preview/zhconv:0.3.1"
\end{verbatim}

\includesvg[width=0.16667in,height=0.16667in]{/assets/icons/16-copy.svg}

Check the docs for
\href{https://typst.app/docs/reference/scripting/\#packages}{more
information on how to import packages} .

\subsubsection{About}\label{about}

\begin{description}
\tightlist
\item[Author :]
\href{mailto:whygowe@gmail.com}{Hung-I Wang}
\item[License:]
GPL-2.0
\item[Current version:]
0.3.1
\item[Last updated:]
August 14, 2024
\item[First released:]
August 14, 2024
\item[Archive size:]
804 kB
\href{https://packages.typst.org/preview/zhconv-0.3.1.tar.gz}{\pandocbounded{\includesvg[keepaspectratio]{/assets/icons/16-download.svg}}}
\item[Repository:]
\href{https://github.com/Gowee/zhconv-rs}{GitHub}
\end{description}

\subsubsection{Where to report issues?}\label{where-to-report-issues}

This package is a project of Hung-I Wang . Report issues on
\href{https://github.com/Gowee/zhconv-rs}{their repository} . You can
also try to ask for help with this package on the
\href{https://forum.typst.app}{Forum} .

Please report this package to the Typst team using the
\href{https://typst.app/contact}{contact form} if you believe it is a
safety hazard or infringes upon your rights.

\phantomsection\label{versions}
\subsubsection{Version history}\label{version-history}

\begin{longtable}[]{@{}ll@{}}
\toprule\noalign{}
Version & Release Date \\
\midrule\noalign{}
\endhead
\bottomrule\noalign{}
\endlastfoot
0.3.1 & August 14, 2024 \\
\end{longtable}

Typst GmbH did not create this package and cannot guarantee correct
functionality of this package or compatibility with any version of the
Typst compiler or app.


\title{typst.app/universe/package/cineca}

\phantomsection\label{banner}
\section{cineca}\label{cineca}

{ 0.4.0 }

A package to create calendar with events.

\phantomsection\label{readme}
CINECA Is Not an Electric Calendar App, but a Typst package to create
calendars with events.

\subsection{Usage}\label{usage}

The package now support creating events from ICS files (thanks
@Geronymos). To do so, read an ICS file and parse with
\texttt{\ ics-parser()\ } .

\begin{Shaded}
\begin{Highlighting}[]
\NormalTok{\#let events2 = ics{-}parser(read("sample.ics")).map(event =\textgreater{} (}
\NormalTok{  event.dtstart, }
\NormalTok{  event.dtstart,}
\NormalTok{  event.dtend,}
\NormalTok{  event.summary}
\NormalTok{))}

\NormalTok{\#calendar(events2, hour{-}range: (10, 14))}
\end{Highlighting}
\end{Shaded}

\subsubsection{Day view}\label{day-view}

\texttt{\ calendar(events,\ hour-range,\ minute-height,\ template,\ stroke)\ }

Parameters:

\begin{itemize}
\tightlist
\item
  \texttt{\ events\ } : An array of events. Each item is a 4-element
  array:

  \begin{itemize}
  \tightlist
  \item
    Date. Can be \texttt{\ datetime()\ } or valid args of
    \texttt{\ day()\ } .
  \item
    Start time. Can be valid args of \texttt{\ time()\ } .
  \item
    End time. Can be valid args of \texttt{\ time()\ } .
  \item
    Event body. Can be anything. Passed to the template.body to show
    more details.
  \end{itemize}
\item
  \texttt{\ hour-range\ } : Then range of hours, affacting the range of
  the calendar. Default: \texttt{\ (8,\ 20)\ } .
\item
  \texttt{\ minute-height\ } : Height of per minute. Each minute occupys
  a row. This number is to control the height of each row. Default:
  \texttt{\ 0.8pt\ } .
\item
  \texttt{\ template\ } : Templates for headers, times, or events. It
  takes a dictionary of the following entries: \texttt{\ header\ } ,
  \texttt{\ time\ } , and \texttt{\ event\ } . Default: \texttt{\ (:)\ }
  .
\item
  \texttt{\ stroke\ } : A stroke style to control the style of the
  default stroke, or a function taking two parameters
  \texttt{\ (x,\ y)\ } to control the stroke. The first row is the
  dates, and the first column is the times. Default: \texttt{\ none\ } .
\end{itemize}

\begin{quote}
{[}!NOTE{]} See below for more details about the format of start time
and end time.
\end{quote}

Example:

\pandocbounded{\includegraphics[keepaspectratio]{https://github.com/typst/packages/raw/main/packages/preview/cineca/0.4.0/test/day-view.png}}

\subsubsection{Month view}\label{month-view}

\texttt{\ calendar-month(events,\ template,\ sunday-first,\ ..args)\ }

\begin{itemize}
\tightlist
\item
  \texttt{\ events\ } : Event list. Each element is a two-element array.

  \begin{itemize}
  \tightlist
  \item
    Day. A datetime object.
  \item
    Additional information for showing a day. It actually depends on the
    template \texttt{\ day-body\ } . For the deafult template, it
    requires a content.
  \end{itemize}
\item
  \texttt{\ template\ } : Templates for headers, times, or events. It
  takes a dictionary of the following entries: \texttt{\ day-body\ } ,
  \texttt{\ day-head\ } , \texttt{\ month-head\ } , and
  \texttt{\ layout\ } .
\item
  \texttt{\ sunday-first\ } : Whether to put sunday as the first day of
  a week.
\item
  \texttt{\ ..args\ } : Additional arguments for the calendar’s grid.
\end{itemize}

Example:

\begin{Shaded}
\begin{Highlighting}[]
\NormalTok{\#let events = (}
\NormalTok{  (daytime("2024{-}2{-}1", "9:0:0"), [Lecture]),}
\NormalTok{  (daytime("2024{-}2{-}1", "10:0:0"), [Tutorial]),}
\NormalTok{  (daytime("2024{-}2{-}2", "10:0:0"), [Meeting]),}
\NormalTok{  (daytime("2024{-}2{-}10", "12:0:0"), [Lunch]),}
\NormalTok{  (daytime("2024{-}2{-}25", "8:0:0"), [Train]),}
\NormalTok{)}

\NormalTok{\#calendar{-}month(}
\NormalTok{  events,}
\NormalTok{  sunday{-}first: false,}
\NormalTok{  template: (}
\NormalTok{    month{-}head: (content) =\textgreater{} strong(content)}
\NormalTok{  )}
\NormalTok{)}
\end{Highlighting}
\end{Shaded}

\begin{Shaded}
\begin{Highlighting}[]
\NormalTok{\#let events2 = (}
\NormalTok{  (datetime(year: 2024, month: 5, day: 1, hour: 9, minute: 0, second: 0), ([Lecture], blue)),}
\NormalTok{  (datetime(year: 2024, month: 5, day: 1, hour: 10, minute: 0, second: 0), ([Tutorial], red)),}
\NormalTok{  (datetime(year: 2024, month: 5, day: 1, hour: 11, minute: 0, second: 0), [Lab]),}
\NormalTok{)}

\NormalTok{\#calendar{-}month(}
\NormalTok{  events2,}
\NormalTok{  sunday{-}first: true,}
\NormalTok{  rows: (2em,) * 2 + (8em,),}
\NormalTok{  template: (}
\NormalTok{    day{-}body: (day, events) =\textgreater{} \{}
\NormalTok{      show: block.with(width: 100\%, height: 100\%, inset: 2pt)}
\NormalTok{      set align(left + top)}
\NormalTok{      stack(}
\NormalTok{        spacing: 2pt,}
\NormalTok{        pad(bottom: 4pt, text(weight: "bold", day.display("[day]"))),}
\NormalTok{        ..events.map(((d, e)) =\textgreater{} \{}
\NormalTok{          let col = if type(e) == array and e.len() \textgreater{} 1 \{ e.at(1) \} else \{ yellow \}}
\NormalTok{          show: block.with(}
\NormalTok{            fill: col.lighten(40\%),}
\NormalTok{            stroke: col,}
\NormalTok{            width: 100\%,}
\NormalTok{            inset: 2pt,}
\NormalTok{            radius: 2pt}
\NormalTok{          )}
\NormalTok{          d.display("[hour]")}
\NormalTok{          h(4pt)}
\NormalTok{          if type(e) == array \{ e.at(0) \} else \{ e \}}
\NormalTok{        \})}
\NormalTok{      )}
\NormalTok{    \}}
\NormalTok{  )}
\NormalTok{)}
\end{Highlighting}
\end{Shaded}

\pandocbounded{\includegraphics[keepaspectratio]{https://github.com/typst/packages/raw/main/packages/preview/cineca/0.4.0/test/month-view.png}}

\subsubsection{Month-summary view}\label{month-summary-view}

\texttt{\ calendar-month-summary(events,\ template,\ sunday-first,\ ..args)\ }

\begin{itemize}
\tightlist
\item
  \texttt{\ events\ } : Event list. Each element is a two-element array.

  \begin{itemize}
  \tightlist
  \item
    Day. A datetime object.
  \item
    Additional information for showing a day. It actually depends on the
    template \texttt{\ day-body\ } . For the deafult template, it
    requires an array of two elements.

    \begin{itemize}
    \tightlist
    \item
      Shape. A function specify how to darw the shape, such as
      \texttt{\ circle\ } .
    \item
      Arguments. Further arguments for render a shape.
    \end{itemize}
  \end{itemize}
\item
  \texttt{\ template\ } : Templates for headers, times, or events. It
  takes a dictionary of the following entries: \texttt{\ day-body\ } ,
  \texttt{\ day-head\ } , \texttt{\ month-head\ } , and
  \texttt{\ layout\ } .
\item
  \texttt{\ sunday-first\ } : Whether to put sunday as the first day of
  a week.
\item
  \texttt{\ ..args\ } : Additional arguments for the calendar’s grid.
\end{itemize}

Example:

\begin{Shaded}
\begin{Highlighting}[]
\NormalTok{\#let events = (}
\NormalTok{  (day("2024{-}02{-}21"), (circle, (stroke: color.green, inset: 2pt))),}
\NormalTok{  (day("2024{-}02{-}22"), (circle, (stroke: color.green, inset: 2pt))),}
\NormalTok{  (day("2024{-}05{-}27"), (circle, (stroke: color.green, inset: 2pt))),}
\NormalTok{  (day("2024{-}05{-}28"), (circle, (stroke: color.blue, inset: 2pt))),}
\NormalTok{  (day("2024{-}05{-}29"), (circle, (stroke: color.blue, inset: 2pt))),}
\NormalTok{  (day("2024{-}06{-}03"), (circle, (stroke: color.blue, inset: 2pt))),}
\NormalTok{  (day("2024{-}06{-}04"), (circle, (stroke: color.yellow, inset: 2pt))),}
\NormalTok{  (day("2024{-}06{-}05"), (circle, (stroke: color.yellow, inset: 2pt))),}
\NormalTok{  (day("2024{-}06{-}10"), (circle, (stroke: color.red, inset: 2pt))),}
\NormalTok{)}

\NormalTok{\#calendar{-}month{-}summary(}
\NormalTok{  events: events}
\NormalTok{)}

\NormalTok{\#calendar{-}month{-}summary(}
\NormalTok{  events: events,}
\NormalTok{  sunday{-}first: true}
\NormalTok{)}

\NormalTok{// An empty calendar}
\NormalTok{\#calendar{-}month{-}summary(}
\NormalTok{  events: (}
\NormalTok{    (datetime(year: 2024, month: 05, day: 21), (none,)),}
\NormalTok{  ),}
\NormalTok{  stroke: 1pt,}
\NormalTok{)}
\end{Highlighting}
\end{Shaded}

\pandocbounded{\includegraphics[keepaspectratio]{https://github.com/typst/packages/raw/main/packages/preview/cineca/0.4.0/test/month-summary.png}}

\subsection{Day/Time/Daytime Format}\label{daytimedaytime-format}

In addition to using \texttt{\ datetime()\ } to set up time, the package
provided some other ways, supported by functions \texttt{\ day()\ } ,
\texttt{\ time()\ } , and \texttt{\ daytime()\ } .

\begin{Shaded}
\begin{Highlighting}[]
\NormalTok{{-} \#time(8)}
\NormalTok{{-} \#time(8, 10)}
\NormalTok{{-} \#time(8, 10, 30)}
\NormalTok{{-} \#time("8.30")}
\NormalTok{{-} \#time(decimal("12.10"))}
\NormalTok{{-} \#time(14.10)            // 24{-}hour format}
\NormalTok{{-} \#time("8:10:08")}

\NormalTok{{-} \#day(2024)}
\NormalTok{{-} \#day(2024, 2)}
\NormalTok{{-} \#day(2024, 2, 5)    // year, month, day}
\NormalTok{{-} \#day("2024{-}3{-}7")    // ISO format (year{-}month{-}day)}
\NormalTok{{-} \#day("26/12/2024")  // British format (day/month/year)}

\NormalTok{{-} \#daytime(2024)}
\NormalTok{{-} \#daytime(2024, 2)}
\NormalTok{{-} \#daytime(2024, 2, 5)}
\NormalTok{{-} \#daytime(2024, 2, 5, 8)}
\NormalTok{{-} \#daytime(2024, 2, 5, 8, 10)}
\NormalTok{{-} \#daytime("2024{-}6{-}1", 8)}
\NormalTok{{-} \#daytime("2024{-}6{-}1", 8, 10)}
\NormalTok{{-} \#daytime("2024{-}6{-}1", 8, 10, 30)}
\NormalTok{{-} \#daytime(2024, "12:00")}
\NormalTok{{-} \#daytime(2024, 2, "12:00")}
\NormalTok{{-} \#daytime(2024, 2, 5, "12:00")}
\NormalTok{{-} \#daytime("2024{-}3{-}7", "11:30:45")}
\NormalTok{{-} \#daytime("2024{-}12{-}26 8:30")}
\end{Highlighting}
\end{Shaded}

\subsection{Limitations}\label{limitations}

\begin{itemize}
\tightlist
\item
  Page breaking may be incorrect.
\item
  Items will overlap when they happens at the same time.
\end{itemize}

\subsubsection{How to add}\label{how-to-add}

Copy this into your project and use the import as \texttt{\ cineca\ }

\begin{verbatim}
#import "@preview/cineca:0.4.0"
\end{verbatim}

\includesvg[width=0.16667in,height=0.16667in]{/assets/icons/16-copy.svg}

Check the docs for
\href{https://typst.app/docs/reference/scripting/\#packages}{more
information on how to import packages} .

\subsubsection{About}\label{about}

\begin{description}
\tightlist
\item[Author :]
HPDell
\item[License:]
MIT
\item[Current version:]
0.4.0
\item[Last updated:]
November 25, 2024
\item[First released:]
April 1, 2024
\item[Archive size:]
7.04 kB
\href{https://packages.typst.org/preview/cineca-0.4.0.tar.gz}{\pandocbounded{\includesvg[keepaspectratio]{/assets/icons/16-download.svg}}}
\item[Repository:]
\href{https://github.com/HPdell/typst-cineca}{GitHub}
\item[Categor ies :]
\begin{itemize}
\tightlist
\item[]
\item
  \pandocbounded{\includesvg[keepaspectratio]{/assets/icons/16-layout.svg}}
  \href{https://typst.app/universe/search/?category=layout}{Layout}
\item
  \pandocbounded{\includesvg[keepaspectratio]{/assets/icons/16-chart.svg}}
  \href{https://typst.app/universe/search/?category=visualization}{Visualization}
\end{itemize}
\end{description}

\subsubsection{Where to report issues?}\label{where-to-report-issues}

This package is a project of HPDell . Report issues on
\href{https://github.com/HPdell/typst-cineca}{their repository} . You
can also try to ask for help with this package on the
\href{https://forum.typst.app}{Forum} .

Please report this package to the Typst team using the
\href{https://typst.app/contact}{contact form} if you believe it is a
safety hazard or infringes upon your rights.

\phantomsection\label{versions}
\subsubsection{Version history}\label{version-history}

\begin{longtable}[]{@{}ll@{}}
\toprule\noalign{}
Version & Release Date \\
\midrule\noalign{}
\endhead
\bottomrule\noalign{}
\endlastfoot
0.4.0 & November 25, 2024 \\
\href{https://typst.app/universe/package/cineca/0.3.0/}{0.3.0} &
November 18, 2024 \\
\href{https://typst.app/universe/package/cineca/0.2.1/}{0.2.1} & July 1,
2024 \\
\href{https://typst.app/universe/package/cineca/0.2.0/}{0.2.0} & May 22,
2024 \\
\href{https://typst.app/universe/package/cineca/0.1.0/}{0.1.0} & April
1, 2024 \\
\end{longtable}

Typst GmbH did not create this package and cannot guarantee correct
functionality of this package or compatibility with any version of the
Typst compiler or app.


\title{typst.app/universe/package/jlyfish}

\phantomsection\label{banner}
\section{jlyfish}\label{jlyfish}

{ 0.1.0 }

Julia code evaluation inside your Typst document

\phantomsection\label{readme}
\pandocbounded{\includesvg[keepaspectratio]{https://github.com/typst/packages/raw/main/packages/preview/jlyfish/0.1.0/assets/logo.svg}}

Jlyfish is a package for Julia and Typst that allows you to integrate
Julia computations in your Typst document.

\href{https://github.com/andreasKroepelin/TypstJlyfish.jl/wiki}{\pandocbounded{\includegraphics[keepaspectratio]{https://img.shields.io/badge/docs-wiki-blue}}}
\pandocbounded{\includegraphics[keepaspectratio]{https://img.shields.io/github/license/andreasKroepelin/TypstJlyfish.jl}}
\pandocbounded{\includegraphics[keepaspectratio]{https://img.shields.io/github/v/release/andreasKroepelin/TypstJlyfish.jl}}
\href{https://github.com/andreasKroepelin/TypstJlyfish.jl}{\pandocbounded{\includegraphics[keepaspectratio]{https://img.shields.io/github/stars/andreasKroepelin/TypstJlyfish.jl}}}

You should use Jlyfish if you want to write a Typst document and have
some of the content automatically produced by Julia code but want the
source code for that within your document source. It fills a similar
role as \href{https://github.com/gpoore/pythontex}{PythonTeX} does for
Python and LaTeX. Note that this is different from tools like
\href{https://quarto.org/}{Quarto} where you write documents in
Markdown, also integrate some Julia code, but then might use Typst only
as a backend to produce the final document.

See below for a quick introduction or read the
\href{https://github.com/andreasKroepelin/TypstJlyfish.jl/wiki}{wiki}
for an in depth explanation.

Since Jlyfish builds a bridge between Julia and Typst, we also have to
get two things running. First, install the Julia package
\texttt{\ TypstJlyfish\ } from the general registry by executing

\begin{Shaded}
\begin{Highlighting}[]
\NormalTok{julia\textgreater{} ]}

\NormalTok{(@v1.10) pkg\textgreater{} add TypstJlyfish}
\end{Highlighting}
\end{Shaded}

You only have to do this once. (It is like installing and using the
Pluto notebook system, if you are familiar with that.)

When you want to use Jlyfish in a Typst document (say,
\texttt{\ your-document.typ\ } ), add the following line at the top:

\begin{Shaded}
\begin{Highlighting}[]
\NormalTok{\#import "@preview/jlyfish:0.1.0": *}
\end{Highlighting}
\end{Shaded}

Then, open a Julia REPL and run

\begin{Shaded}
\begin{Highlighting}[]
\NormalTok{julia\textgreater{} import TypstJlyfish}

\NormalTok{julia\textgreater{} TypstJlyfish.watch("your{-}document.typ")}
\end{Highlighting}
\end{Shaded}

Jlyfish facilitates the communication between Julia and Typst via a JSON
file. By default, Jlyfish uses the name of your document and adds a
\texttt{\ -jlyfish.json\ } , so \texttt{\ your-document.typ\ } would
become \texttt{\ your-document-jlyfish.json\ } . This can be configured,
of course.

To let Typst know of the computed data in the JSON file, add the
following line to your document:

\begin{Shaded}
\begin{Highlighting}[]
\NormalTok{\#read{-}julia{-}output(json("your{-}document{-}jlyfish.json"))}
\end{Highlighting}
\end{Shaded}

You can then place some Julia code in your Typst source using the
\texttt{\ \#jl\ } function:

\begin{Shaded}
\begin{Highlighting}[]
\NormalTok{What is the sum of the whole numbers from one to a hundred? \#jl(\textasciigrave{}sum(1:100)\textasciigrave{})}
\end{Highlighting}
\end{Shaded}

Head over to the
\href{https://github.com/andreasKroepelin/TypstJlyfish.jl/wiki}{wiki} to
learn more!

Just to show what is possible with Jlyfish:

\pandocbounded{\includesvg[keepaspectratio]{https://github.com/typst/packages/raw/main/packages/preview/jlyfish/0.1.0/examples/demo.svg}}

\begin{Shaded}
\begin{Highlighting}[]
\NormalTok{\#import "@preview/jlyfish:0.1.0": *}

\NormalTok{\#set page(width: auto, height: auto, margin: 1em)}
\NormalTok{\#set text(font: "Alegreya Sans")}
\NormalTok{\#let note = text.with(size: .7em, fill: luma(100), style: "italic")}

\NormalTok{\#read{-}julia{-}output(json("demo{-}jlyfish.json"))}
\NormalTok{\#jl{-}pkg("Colors", "Typstry", "Makie", "CairoMakie")}

\NormalTok{\#grid(}
\NormalTok{  columns: 2,}
\NormalTok{  gutter: 1em,}
\NormalTok{  align: top,}
\NormalTok{  [}
\NormalTok{    \#note[Generate Typst code in Julia:]}

\NormalTok{    \#set text(size: 4em)}
\NormalTok{    \#jl(\textasciigrave{}\textasciigrave{}\textasciigrave{}julia}
\NormalTok{      using Typstry, Colors}

\NormalTok{      parts = map([:red, :green, :purple], ["Ju", "li", "a"]) do name, text}
\NormalTok{        color = hex(Colors.JULIA\_LOGO\_COLORS[name])}
\NormalTok{        "\#text(fill: rgb(\textbackslash{}"$color\textbackslash{}"))[$text]"}
\NormalTok{      end}
\NormalTok{      TypstText(join(parts))}
\NormalTok{    \textasciigrave{}\textasciigrave{}\textasciigrave{})}
\NormalTok{  ],}
\NormalTok{  [}
\NormalTok{    \#note[Produce images in Julia:]}

\NormalTok{    \#set image(width: 10em)}
\NormalTok{    \#jl(recompute: false, \textasciigrave{}\textasciigrave{}\textasciigrave{}}
\NormalTok{      using Makie, CairoMakie}

\NormalTok{      as = {-}2.2:.01:.7}
\NormalTok{      bs = {-}1.5:.01:1.5}
\NormalTok{      C = [a + b * im for a in as, b in bs]}
\NormalTok{      function mandelbrot(c)}
\NormalTok{        z = c}
\NormalTok{        i = 1}
\NormalTok{        while i \textless{} 100 \&\& abs2(z) \textless{} 4}
\NormalTok{          z = z\^{}2 + c}
\NormalTok{          i += 1}
\NormalTok{        end}
\NormalTok{        i}
\NormalTok{      end}

\NormalTok{      contour(as, bs, mandelbrot.(C), axis = (;aspect = DataAspect()))}
\NormalTok{    \textasciigrave{}\textasciigrave{}\textasciigrave{})}
\NormalTok{  ],}
\NormalTok{  [}
\NormalTok{    \#note[Hand over raw data from Julia to Typst:]}
\NormalTok{    \#let barchart(counts) = \{}
\NormalTok{      set align(bottom)}
\NormalTok{      let bars = counts.map(count =\textgreater{} rect(}
\NormalTok{        width: .3em,}
\NormalTok{        height: count * 9em,}
\NormalTok{        stroke: white,}
\NormalTok{        fill: blue,}
\NormalTok{      ))}
\NormalTok{      stack(dir: ltr, ..bars)}
\NormalTok{    \}}

\NormalTok{    \#jl{-}raw(fn: it =\textgreater{} barchart(it.result.data), \textasciigrave{}\textasciigrave{}\textasciigrave{}julia}
\NormalTok{      p = .5}
\NormalTok{      n = 40}
\NormalTok{      counts = zeros(n + 1)}
\NormalTok{      for \_ in 1:10\_000}
\NormalTok{        count = 0}
\NormalTok{        for \_ in 1:n}
\NormalTok{          if rand() \textless{} p}
\NormalTok{            count += 1}
\NormalTok{          end}
\NormalTok{        end}
\NormalTok{        counts[count + 1] += 1}
\NormalTok{      end}

\NormalTok{      counts ./= maximum(counts)}
\NormalTok{      lo, hi = findfirst(\textgreater{}(1e{-}3), counts), findlast(\textgreater{}(1e{-}3), counts)}
\NormalTok{      counts[lo:hi]}
\NormalTok{    \textasciigrave{}\textasciigrave{}\textasciigrave{})}
\NormalTok{  ],}
\NormalTok{  [}
\NormalTok{    \#note[See errors, stdout, and logs:]}

\NormalTok{    \#jl(\textasciigrave{}\textasciigrave{}\textasciigrave{}julia}
\NormalTok{      println("Hello from stdout!")}
\NormalTok{      @info "Something to note" n p}
\NormalTok{      @warn "You should read this!"}
\NormalTok{      this\_does\_not\_exist}
\NormalTok{    \textasciigrave{}\textasciigrave{}\textasciigrave{})}
\NormalTok{  ]}
\NormalTok{)}
\end{Highlighting}
\end{Shaded}

\subsubsection{How to add}\label{how-to-add}

Copy this into your project and use the import as \texttt{\ jlyfish\ }

\begin{verbatim}
#import "@preview/jlyfish:0.1.0"
\end{verbatim}

\includesvg[width=0.16667in,height=0.16667in]{/assets/icons/16-copy.svg}

Check the docs for
\href{https://typst.app/docs/reference/scripting/\#packages}{more
information on how to import packages} .

\subsubsection{About}\label{about}

\begin{description}
\tightlist
\item[Author :]
Andreas Kröpelin
\item[License:]
MIT
\item[Current version:]
0.1.0
\item[Last updated:]
July 8, 2024
\item[First released:]
July 8, 2024
\item[Archive size:]
2.75 kB
\href{https://packages.typst.org/preview/jlyfish-0.1.0.tar.gz}{\pandocbounded{\includesvg[keepaspectratio]{/assets/icons/16-download.svg}}}
\item[Repository:]
\href{https://github.com/andreasKroepelin/TypstJlyfish.jl}{GitHub}
\item[Categor ies :]
\begin{itemize}
\tightlist
\item[]
\item
  \pandocbounded{\includesvg[keepaspectratio]{/assets/icons/16-code.svg}}
  \href{https://typst.app/universe/search/?category=scripting}{Scripting}
\item
  \pandocbounded{\includesvg[keepaspectratio]{/assets/icons/16-hammer.svg}}
  \href{https://typst.app/universe/search/?category=utility}{Utility}
\item
  \pandocbounded{\includesvg[keepaspectratio]{/assets/icons/16-integration.svg}}
  \href{https://typst.app/universe/search/?category=integration}{Integration}
\end{itemize}
\end{description}

\subsubsection{Where to report issues?}\label{where-to-report-issues}

This package is a project of Andreas Kröpelin . Report issues on
\href{https://github.com/andreasKroepelin/TypstJlyfish.jl}{their
repository} . You can also try to ask for help with this package on the
\href{https://forum.typst.app}{Forum} .

Please report this package to the Typst team using the
\href{https://typst.app/contact}{contact form} if you believe it is a
safety hazard or infringes upon your rights.

\phantomsection\label{versions}
\subsubsection{Version history}\label{version-history}

\begin{longtable}[]{@{}ll@{}}
\toprule\noalign{}
Version & Release Date \\
\midrule\noalign{}
\endhead
\bottomrule\noalign{}
\endlastfoot
0.1.0 & July 8, 2024 \\
\end{longtable}

Typst GmbH did not create this package and cannot guarantee correct
functionality of this package or compatibility with any version of the
Typst compiler or app.


\title{typst.app/universe/package/marge}

\phantomsection\label{banner}
\section{marge}\label{marge}

{ 0.1.0 }

Easy-to-use but powerful and smart margin notes.

\phantomsection\label{readme}
A package for easy-to-use but powerful and smart margin notes.

\subsection{Usage}\label{usage}

The main function provided by this package is \texttt{\ sidenote\ } ,
which allows you to create margin notes. The function takes a single
positional argument (the text of the note) and several optional keyword
arguments for customization:

\begin{longtable}[]{@{}lll@{}}
\toprule\noalign{}
Parameter & Description & Default \\
\midrule\noalign{}
\endhead
\bottomrule\noalign{}
\endlastfoot
\texttt{\ side\ } & The margin where the note should be placed. &
\texttt{\ auto\ } \\
\texttt{\ dy\ } & The custom offset by which the note should be moved
along the y-axis. & \texttt{\ 0pt\ } \\
\texttt{\ padding\ } & The space between the note and the page or
content border. & \texttt{\ 2em\ } \\
\texttt{\ gap\ } & The minimum gap between this and neighboring notes. &
\texttt{\ 0.4em\ } \\
\texttt{\ numbering\ } & How the note should be numbered. &
\texttt{\ none\ } \\
\texttt{\ counter\ } & The counter to use for numbering. &
\texttt{\ counter("sidenote")\ } \\
\texttt{\ format\ } & The “show rule� for the note. &
\texttt{\ it\ =\textgreater{}\ it.default\ } \\
\end{longtable}

The parameters allow maximum flexibility and often allow values of
different types:

\begin{itemize}
\tightlist
\item
  The \texttt{\ side\ } parameter can be set to \texttt{\ auto\ } ,
  \texttt{\ "inside"\ } , \texttt{\ "outside"\ } or any horizontal
  \texttt{\ alignment\ } value. If set to \texttt{\ auto\ } , the note
  is placed on the larger of the two margins. If they are equally large,
  it is placed on the \texttt{\ "outside"\ } margin.
\item
  If the \texttt{\ dy\ } parameter has a relative part, it is resolved
  relative to the height of the note.
\item
  The \texttt{\ padding\ } parameter can be set either to a single
  length value or a dictionary. If a dictionary is used, the keys can be
  any horizontal alignment value, as well as \texttt{\ inside\ } and
  \texttt{\ outside\ } .
\item
  With the \texttt{\ counter\ } parameter, you can for example combine
  the numbering of footnotes and sidenotes.
\end{itemize}

An especially useful feature is the \texttt{\ format\ } parameter, as it
emulates the behavior of a show rule via a function. That function is
called with the context of the note and receives a dictionary with the
following keys:

\begin{longtable}[]{@{}lll@{}}
\toprule\noalign{}
Key & Description & Value or type \\
\midrule\noalign{}
\endhead
\bottomrule\noalign{}
\endlastfoot
\texttt{\ side\ } & The side of the page the note is placed on. &
\texttt{\ left\ } or \texttt{\ right\ } \\
\texttt{\ numbering\ } & The numbering of the note. & \texttt{\ str\ } ,
\texttt{\ function\ } or \texttt{\ none\ } \\
\texttt{\ counter\ } & The counter used for numbering the note. &
\texttt{\ counter\ } \\
\texttt{\ padding\ } & The padding of the note, resolved to
\texttt{\ left\ } and \texttt{\ right\ } . & \texttt{\ dictionary\ } \\
\texttt{\ margin\ } & The size of the margin, which the note is placed
on. & \texttt{\ length\ } \\
\texttt{\ source\ } & The location in the document where the note is
defined. & \texttt{\ location\ } \\
\texttt{\ body\ } & The content of the note. & \texttt{\ str\ } \\
\texttt{\ default\ } & The default look of the note. &
\texttt{\ content\ } \\
\end{longtable}

As the dictionary itself is not an element, you cannot directly use it
within the \texttt{\ format\ } function as you would be able to in a
normal show rule. To still be able to build upon the default look of the
note without having to reconstruct it, the \texttt{\ default\ } key is
provided. The default style sets the font size to \texttt{\ 0.85em\ }
and the paragraph’s leading to \texttt{\ 0.5em\ } , matching the
default style of footnotes. This can of course be overridden.

Aside from the customizability, the package also provides automatic
overlap and overflow protection. If a note would overlap with another
note, it is moved further down the page, so that the \texttt{\ gap\ }
parameters of both notes are respected. If a note would overflow the
page, it is moved upwards, so that the bottom of the note is aligned
with the bottom of the page content. Any previous notes, which would
then overlap with the moved note, are also moved accordingly.

\subsubsection{Note about pages with automatic
width}\label{note-about-pages-with-automatic-width}

If a note is placed in the right margin of a page with width set to
\texttt{\ auto\ } , additional configuration is necessary. As the final
width of the page is not known when the note is placed, the note’s
position cannot be calculated. To place notes on the right margin of
such pages, the package provides a \texttt{\ container\ } , which is
supposed to be included in the page’s \texttt{\ background\ } or
\texttt{\ foreground\ } :

\begin{Shaded}
\begin{Highlighting}[]
\NormalTok{\#import "@preview/marge:0.1.0": sidenote, container}

\NormalTok{\#set page(width: auto, background: container)}

\NormalTok{...}
\end{Highlighting}
\end{Shaded}

The use of the \texttt{\ container\ } variable is detected automatically
by the package, so that an error can be raised when it is required but
not set.

\subsubsection{Note about layout convergence and
performance}\label{note-about-layout-convergence-and-performance}

This package makes heavy use of states and contextual blocks, causing
Typst to require multiple layout passes to fully resolve the final
layout. Usually, the limit imposed by Typst is sufficient, but I cannot
guarantee that this will remain true for large documents with a lot of
notes. If you happen to run into this limit, you can try using the
\texttt{\ container\ } variable as mentioned above, as it can reduce the
number of layout passes required.

As each layout iteration adds to the total compile time, the use of the
\texttt{\ container\ } can also be beneficial for performance reasons.
Another performance tip is to keep the size of paragraphs containing
margin notes small, as the line breaking algorithm cannot be memoized
when the paragraph contains a note.

\subsubsection{Note about how lengths are
resolved}\label{note-about-how-lengths-are-resolved}

When a length is given in a context-dependent way (i.e. in
\texttt{\ em\ } units), it is resolved relative to the font size of the
\emph{content} , not the font size of the note (which is smaller by
default). This has the unfortunate side effect that a gap set to
\texttt{\ 0pt\ } will still have some space due to the content
paragraph’s leading (which is also larger than default leading of the
note). Similarly, if notes are defined in a context with a larger font
size, the padding and gap values may unexpectedly be larger than of
neighboring notes.

\subsection{Example}\label{example}

\begin{Shaded}
\begin{Highlighting}[]
\NormalTok{\#import "@preview/marge:0.1.0": sidenote}

\NormalTok{\#set page(margin: (right: 5cm))}
\NormalTok{\#set par(justify: true)}

\NormalTok{\#let sidenote = sidenote.with(numbering: "1", padding: 1em)}

\NormalTok{The Simpsons is an iconic animated series that began in 1989}
\NormalTok{\#sidenote[The show holds the record for the most episodes of any}
\NormalTok{American sitcom.]. The show features the Simpson family: Homer,}
\NormalTok{Marge, Bart, Lisa, and Maggie. }

\NormalTok{Bart is the rebellious son who often gets into trouble, and Lisa}
\NormalTok{is the intelligent and talented daughter \#sidenote[Lisa is known}
\NormalTok{for her saxophone playing and academic achievements.]. Baby}
\NormalTok{Maggie, though silent, has had moments of surprising brilliance}
\NormalTok{\#sidenote[Maggie once shot Mr. Burns in a dramatic plot twist.].}
\end{Highlighting}
\end{Shaded}

\pandocbounded{\includesvg[keepaspectratio]{https://github.com/typst/packages/raw/main/packages/preview/marge/0.1.0/assets/example.svg}}

\subsubsection{How to add}\label{how-to-add}

Copy this into your project and use the import as \texttt{\ marge\ }

\begin{verbatim}
#import "@preview/marge:0.1.0"
\end{verbatim}

\includesvg[width=0.16667in,height=0.16667in]{/assets/icons/16-copy.svg}

Check the docs for
\href{https://typst.app/docs/reference/scripting/\#packages}{more
information on how to import packages} .

\subsubsection{About}\label{about}

\begin{description}
\tightlist
\item[Author :]
Eric Biedert
\item[License:]
MIT
\item[Current version:]
0.1.0
\item[Last updated:]
November 19, 2024
\item[First released:]
November 19, 2024
\item[Minimum Typst version:]
0.11.0
\item[Archive size:]
7.98 kB
\href{https://packages.typst.org/preview/marge-0.1.0.tar.gz}{\pandocbounded{\includesvg[keepaspectratio]{/assets/icons/16-download.svg}}}
\item[Repository:]
\href{https://github.com/EpicEricEE/typst-marge}{GitHub}
\item[Categor ies :]
\begin{itemize}
\tightlist
\item[]
\item
  \pandocbounded{\includesvg[keepaspectratio]{/assets/icons/16-package.svg}}
  \href{https://typst.app/universe/search/?category=components}{Components}
\item
  \pandocbounded{\includesvg[keepaspectratio]{/assets/icons/16-list-unordered.svg}}
  \href{https://typst.app/universe/search/?category=model}{Model}
\item
  \pandocbounded{\includesvg[keepaspectratio]{/assets/icons/16-layout.svg}}
  \href{https://typst.app/universe/search/?category=layout}{Layout}
\end{itemize}
\end{description}

\subsubsection{Where to report issues?}\label{where-to-report-issues}

This package is a project of Eric Biedert . Report issues on
\href{https://github.com/EpicEricEE/typst-marge}{their repository} . You
can also try to ask for help with this package on the
\href{https://forum.typst.app}{Forum} .

Please report this package to the Typst team using the
\href{https://typst.app/contact}{contact form} if you believe it is a
safety hazard or infringes upon your rights.

\phantomsection\label{versions}
\subsubsection{Version history}\label{version-history}

\begin{longtable}[]{@{}ll@{}}
\toprule\noalign{}
Version & Release Date \\
\midrule\noalign{}
\endhead
\bottomrule\noalign{}
\endlastfoot
0.1.0 & November 19, 2024 \\
\end{longtable}

Typst GmbH did not create this package and cannot guarantee correct
functionality of this package or compatibility with any version of the
Typst compiler or app.


\title{typst.app/universe/package/example}

\phantomsection\label{banner}
\section{example}\label{example}

{ 0.1.0 }

An example package.

\phantomsection\label{readme}
An example package demonstrating the structure of a Typst package.

Displays the text “This is an example!� when included and exports
four functions \texttt{\ add\ } , \texttt{\ sub\ } , \texttt{\ mul\ } ,
and \texttt{\ div\ } that perform the respective mathematical operations
on two operands.

\subsubsection{How to add}\label{how-to-add}

Copy this into your project and use the import as \texttt{\ example\ }

\begin{verbatim}
#import "@preview/example:0.1.0"
\end{verbatim}

\includesvg[width=0.16667in,height=0.16667in]{/assets/icons/16-copy.svg}

Check the docs for
\href{https://typst.app/docs/reference/scripting/\#packages}{more
information on how to import packages} .

\subsubsection{About}\label{about}

\begin{description}
\tightlist
\item[Author :]
The Typst Project Developers
\item[License:]
Unlicense
\item[Current version:]
0.1.0
\item[Last updated:]
June 28, 2023
\item[First released:]
June 28, 2023
\item[Archive size:]
1.37 kB
\href{https://packages.typst.org/preview/example-0.1.0.tar.gz}{\pandocbounded{\includesvg[keepaspectratio]{/assets/icons/16-download.svg}}}
\end{description}

\subsubsection{Where to report issues?}\label{where-to-report-issues}

This package is a project of The Typst Project Developers . You can also
try to ask for help with this package on the
\href{https://forum.typst.app}{Forum} .

Please report this package to the Typst team using the
\href{https://typst.app/contact}{contact form} if you believe it is a
safety hazard or infringes upon your rights.

\phantomsection\label{versions}
\subsubsection{Version history}\label{version-history}

\begin{longtable}[]{@{}ll@{}}
\toprule\noalign{}
Version & Release Date \\
\midrule\noalign{}
\endhead
\bottomrule\noalign{}
\endlastfoot
0.1.0 & June 28, 2023 \\
\end{longtable}

Typst GmbH did not create this package and cannot guarantee correct
functionality of this package or compatibility with any version of the
Typst compiler or app.


\title{typst.app/universe/package/ionio-illustrate}

\phantomsection\label{banner}
\section{ionio-illustrate}\label{ionio-illustrate}

{ 0.2.0 }

Mass spectra with annotations for typst.

\phantomsection\label{readme}
\phantomsection\label{readme-top}{}

\href{https://github.com/jamesxx/ionio-illustrate/blob/master/LICENSE}{\pandocbounded{\includegraphics[keepaspectratio]{https://img.shields.io/github/license/jamesxx/ionio-illustrate}}}
\href{https://github.com/typst/packages/tree/main/packages/preview/ionio-illustrate}{\pandocbounded{\includegraphics[keepaspectratio]{https://img.shields.io/badge/typst-package-239dad}}}
\href{https://github.com/JamesxX/ionio-illustrate/tags}{\pandocbounded{\includegraphics[keepaspectratio]{https://img.shields.io/github/v/tag/jamesxx/ionio-illustrate}}}

This package implements a Cetz chart-like object for displying mass
spectrometric data in Typst documents. It allows for individually styled
mass peaks, callouts, titles, and mass callipers.\\

\href{https://github.com/jamesxx/ionio-illustrate/blob/main/manual.pdf}{\textbf{Explore
the docs »}}\\
\strut \\
\href{https://github.com/jamesxx/ionio-illustrate/issues}{Report Bug} ·
\href{https://github.com/jamesxx/ionio-illustrate/issues}{Request
Feature}

\subsection{Getting Started}\label{getting-started}

To make use of the \texttt{\ ionio-illustrate\ } package, you’ll need
to add it to your project like shown below. Make sure you are importing
a version that supports your end goal.

\begin{Shaded}
\begin{Highlighting}[]
\NormalTok{\#import "@preview/ionio{-}illustrate:0.2.0": *}
\end{Highlighting}
\end{Shaded}

Then, load in your mass spectrum data and pass it through to the package
like so. Data should be 2D array, and by default the mass-charge ratio
is in the first column, and the relative intensities are in the second
column.

\begin{Shaded}
\begin{Highlighting}[]
\NormalTok{\#let data = csv("isobutelene\_epoxide.csv")}

\NormalTok{\#let ms = mass{-}spectrum(massspec, args: (}
\NormalTok{  size: (12,6),}
\NormalTok{  range: (0,100),}
\NormalTok{)) }

\NormalTok{\#figure((ms.display)())}
\end{Highlighting}
\end{Shaded}

\pandocbounded{\includegraphics[keepaspectratio]{https://github.com/typst/packages/raw/main/packages/preview/ionio-illustrate/0.2.0/gallery/isobulelene_epoxide.typ.png}}

There are many ways to further enhance your spectrum, please check out
the manual to find out how.

(
\href{https://github.com/typst/packages/raw/main/packages/preview/ionio-illustrate/0.2.0/\#readme-top}{back
to top} )

\subsection{Roadmap}\label{roadmap}

\begin{itemize}
\tightlist
\item
  {[}x{]} Pass style options through to the plot (tracker: \#1)
\item
  {[} {]} Better placement of text depending on plot size
\item
  {[} {]} Improve default step on axes
\item
  {[} {]} Add support for callouts that are not immediately above their
  assigned peak

  \begin{itemize}
  \tightlist
  \item
    {[} {]} Automatically detect when two annotations are too close, and
    display accordingly
  \end{itemize}
\item
  {[} {]} Move to new Typst type system (waiting on upstream)
\item
  {[} {]} Add in function for displaying skeletal structure of chemical
\item
  {[} {]} Optional second axis for absolute intensity
\item
  {[} {]} Add additional display functions

  \begin{itemize}
  \tightlist
  \item
    {[} {]} Figure out function signature for multiple data sets
  \item
    {[} {]} Overlayed and shifted
  \item
    {[} {]} Horizontal reflection

    \begin{itemize}
    \tightlist
    \item
      {[} {]} How to update existing extras?
    \end{itemize}
  \end{itemize}
\end{itemize}

See the \href{https://github.com/jamesxx/ionio-illustrate/issues}{open
issues} for a full list of proposed features (and known issues).

(
\href{https://github.com/typst/packages/raw/main/packages/preview/ionio-illustrate/0.2.0/\#readme-top}{back
to top} )

\subsection{Contributing}\label{contributing}

Contributions are what make the open source community such an amazing
place to learn, inspire, and create. Any contributions you make are
\textbf{greatly appreciated} .

If you have a suggestion that would make this better, please fork the
repo and create a pull request. You can also simply open an issue with
the tag “enhancement�. Don’t forget to give the project a star!
Thanks again!

\begin{enumerate}
\tightlist
\item
  Fork the Project
\item
  Create your Feature Branch (
  \texttt{\ git\ checkout\ -b\ feature/AmazingFeature\ } )
\item
  Commit your Changes (
  \texttt{\ git\ commit\ -m\ \textquotesingle{}Add\ some\ AmazingFeature\textquotesingle{}\ }
  )
\item
  Push to the Branch (
  \texttt{\ git\ push\ origin\ feature/AmazingFeature\ } )
\item
  Open a Pull Request
\end{enumerate}

(
\href{https://github.com/typst/packages/raw/main/packages/preview/ionio-illustrate/0.2.0/\#readme-top}{back
to top} )

\subsection{License}\label{license}

Distributed under the MIT License. See
\href{https://github.com/jamesxx/ionio-illustrate/blob/master/LICENSE}{\texttt{\ LICENSE\ }}
for more information.

(
\href{https://github.com/typst/packages/raw/main/packages/preview/ionio-illustrate/0.2.0/\#readme-top}{back
to top} )

\subsection{Gallery}\label{gallery}

\pandocbounded{\includegraphics[keepaspectratio]{https://github.com/typst/packages/raw/main/packages/preview/ionio-illustrate/0.2.0/gallery/linalool.typ.png}}

(
\href{https://github.com/typst/packages/raw/main/packages/preview/ionio-illustrate/0.2.0/\#readme-top}{back
to top} )

\subsubsection{How to add}\label{how-to-add}

Copy this into your project and use the import as
\texttt{\ ionio-illustrate\ }

\begin{verbatim}
#import "@preview/ionio-illustrate:0.2.0"
\end{verbatim}

\includesvg[width=0.16667in,height=0.16667in]{/assets/icons/16-copy.svg}

Check the docs for
\href{https://typst.app/docs/reference/scripting/\#packages}{more
information on how to import packages} .

\subsubsection{About}\label{about}

\begin{description}
\tightlist
\item[Author :]
James (Fuzzy) Swift
\item[License:]
MIT
\item[Current version:]
0.2.0
\item[Last updated:]
October 22, 2023
\item[First released:]
October 21, 2023
\item[Archive size:]
5.76 kB
\href{https://packages.typst.org/preview/ionio-illustrate-0.2.0.tar.gz}{\pandocbounded{\includesvg[keepaspectratio]{/assets/icons/16-download.svg}}}
\item[Repository:]
\href{https://github.com/JamesxX/ionio-illustrate}{GitHub}
\end{description}

\subsubsection{Where to report issues?}\label{where-to-report-issues}

This package is a project of James (Fuzzy) Swift . Report issues on
\href{https://github.com/JamesxX/ionio-illustrate}{their repository} .
You can also try to ask for help with this package on the
\href{https://forum.typst.app}{Forum} .

Please report this package to the Typst team using the
\href{https://typst.app/contact}{contact form} if you believe it is a
safety hazard or infringes upon your rights.

\phantomsection\label{versions}
\subsubsection{Version history}\label{version-history}

\begin{longtable}[]{@{}ll@{}}
\toprule\noalign{}
Version & Release Date \\
\midrule\noalign{}
\endhead
\bottomrule\noalign{}
\endlastfoot
0.2.0 & October 22, 2023 \\
\href{https://typst.app/universe/package/ionio-illustrate/0.1.0/}{0.1.0}
& October 21, 2023 \\
\end{longtable}

Typst GmbH did not create this package and cannot guarantee correct
functionality of this package or compatibility with any version of the
Typst compiler or app.


\title{typst.app/universe/package/big-todo}

\phantomsection\label{banner}
\section{big-todo}\label{big-todo}

{ 0.2.0 }

Package to insert clear TODOs, optionally with an outline.

\phantomsection\label{readme}
Create clearly visible TODOs in your document, and add an outline to
keep track of them.

\subsection{Usage}\label{usage}

Import the package

\begin{Shaded}
\begin{Highlighting}[]
\NormalTok{import "@preview/big{-}todo:0.2.0": *}
\end{Highlighting}
\end{Shaded}

And use the \texttt{\ todo\ } function to create a TODO, and the put the
\texttt{\ todo\_outline\ } somewhere to keep track of them.

\begin{Shaded}
\begin{Highlighting}[]
\NormalTok{= Pirates}

\NormalTok{Pirates, a term often associated with seafaring outlaws, have left an indelible mark on world history. The term conjures images of Jolly Roger flags, eye patches, and treasure chests, but the reality of piracy is more complex and varied than its romanticized image suggests. Historically, pirates were motivated by wealth, adventure, or desperation and were not confined to the seas of the Caribbean but roamed the waters of the Mediterranean, the South China Sea, and the Atlantic Ocean. Pirate societies were notorious for their flouting of societal norms, and many pirate ships operated under democratic principles, offering crew members an equal share in the spoils and voting rights on important decisions. To get a better picture, it\textquotesingle{}s worth looking into how the social structures onboard these pirate vessels contrasted with those on merchant or navy vessels of the same era. \#todo([Research and provide more \_detail\_ on \#underline[pirate ship] governance and societal norms ])}

\NormalTok{Pirates\textquotesingle{} influence on history extends beyond their shipboard societies, however. Many pirates played important roles in global trade, war, and politics, often acting as privateers for countries at war. At times, they acted as de facto naval forces, protecting their patron countries\textquotesingle{} interests or disrupting those of their enemies. During the Golden Age of Piracy, roughly from 1650 to 1720, pirates were a major force in the Atlantic and the Caribbean \#todo("other seas?", inline: true), attacking the heavily laden ships of the Spanish Empire and others. They have also impacted popular culture, inspiring countless books, movies, and games. But their story is not finished. Modern{-}day piracy, especially off the coast of Somalia, has become a significant issue in international shipping.}

\NormalTok{\#todo\_outline}
\end{Highlighting}
\end{Shaded}

\pandocbounded{\includegraphics[keepaspectratio]{https://user-images.githubusercontent.com/64754924/250580952-e427a139-1c6e-4eb6-9eee-c07d98875c88.png}}

The \texttt{\ todo\ } function has the follwin parameters and defaults:

\begin{longtable}[]{@{}llll@{}}
\toprule\noalign{}
Parameter & Default & Type & Description \\
\midrule\noalign{}
\endhead
\bottomrule\noalign{}
\endlastfoot
\texttt{\ body\ } & / & Content & Content of the todo \\
\texttt{\ inline\ } & false & Boolean & If true, the todo will be
inline, otherwise it will be block \\
\texttt{\ big\_text\ } & 40pt & Length & Size of the
\texttt{\ !\ TODO\ !\ } text \\
\texttt{\ small\_text\ } & 15pt & Length & Size of the content \\
\texttt{\ gap\ } & 2mm & Length & Gap between the
\texttt{\ !\ TODO\ !\ } text and the content \\
\end{longtable}

\subsubsection{How to add}\label{how-to-add}

Copy this into your project and use the import as \texttt{\ big-todo\ }

\begin{verbatim}
#import "@preview/big-todo:0.2.0"
\end{verbatim}

\includesvg[width=0.16667in,height=0.16667in]{/assets/icons/16-copy.svg}

Check the docs for
\href{https://typst.app/docs/reference/scripting/\#packages}{more
information on how to import packages} .

\subsubsection{About}\label{about}

\begin{description}
\tightlist
\item[Author :]
Raik Rohde
\item[License:]
Unlicense
\item[Current version:]
0.2.0
\item[Last updated:]
July 4, 2023
\item[First released:]
July 4, 2023
\item[Archive size:]
2.81 kB
\href{https://packages.typst.org/preview/big-todo-0.2.0.tar.gz}{\pandocbounded{\includesvg[keepaspectratio]{/assets/icons/16-download.svg}}}
\end{description}

\subsubsection{Where to report issues?}\label{where-to-report-issues}

This package is a project of Raik Rohde . You can also try to ask for
help with this package on the \href{https://forum.typst.app}{Forum} .

Please report this package to the Typst team using the
\href{https://typst.app/contact}{contact form} if you believe it is a
safety hazard or infringes upon your rights.

\phantomsection\label{versions}
\subsubsection{Version history}\label{version-history}

\begin{longtable}[]{@{}ll@{}}
\toprule\noalign{}
Version & Release Date \\
\midrule\noalign{}
\endhead
\bottomrule\noalign{}
\endlastfoot
0.2.0 & July 4, 2023 \\
\href{https://typst.app/universe/package/big-todo/0.1.0/}{0.1.0} & July
7, 2023 \\
\end{longtable}

Typst GmbH did not create this package and cannot guarantee correct
functionality of this package or compatibility with any version of the
Typst compiler or app.


\title{typst.app/universe/package/fh-joanneum-iit-thesis}

\phantomsection\label{banner}
\phantomsection\label{template-thumbnail}
\pandocbounded{\includegraphics[keepaspectratio]{https://packages.typst.org/preview/thumbnails/fh-joanneum-iit-thesis-2.0.5-small.webp}}

\section{fh-joanneum-iit-thesis}\label{fh-joanneum-iit-thesis}

{ 2.0.5 }

BA or MA thesis at FH JOANNEUM

{ } Officially affiliated

\href{/app?template=fh-joanneum-iit-thesis&version=2.0.5}{Create project
in app}

\phantomsection\label{readme}
Template for Your Bachelor’s or Master’s Thesis at
\href{http://www.fh-joanneum.at/iit}{FH JOANNEUM, IIT} .

\subsubsection{TL;DR}\label{tldr}

Using the typst universe preview package/template

\begin{verbatim}
typst init @preview/fh-joanneum-iit-thesis
\end{verbatim}

\pandocbounded{\includegraphics[keepaspectratio]{https://github.com/typst/packages/raw/main/packages/preview/fh-joanneum-iit-thesis/2.0.5/thumbnail.png}}

\href{/app?template=fh-joanneum-iit-thesis&version=2.0.5}{Create project
in app}

\subsubsection{How to use}\label{how-to-use}

Click the button above to create a new project using this template in
the Typst app.

You can also use the Typst CLI to start a new project on your computer
using this command:

\begin{verbatim}
typst init @preview/fh-joanneum-iit-thesis:2.0.5
\end{verbatim}

\includesvg[width=0.16667in,height=0.16667in]{/assets/icons/16-copy.svg}

\subsubsection{About}\label{about}

\begin{description}
\tightlist
\item[Author :]
\href{https://fh-joanneum.at/iit}{IIT, FH JOANNEUM}
\item[License:]
MIT
\item[Current version:]
2.0.5
\item[Last updated:]
November 26, 2024
\item[First released:]
August 27, 2024
\item[Archive size:]
26.1 kB
\href{https://packages.typst.org/preview/fh-joanneum-iit-thesis-2.0.5.tar.gz}{\pandocbounded{\includesvg[keepaspectratio]{/assets/icons/16-download.svg}}}
\item[Verification:]
We verified that the author is affiliated with their institution
\pandocbounded{\includesvg[keepaspectratio]{/assets/icons/16-verified.svg}}
\item[Repository:]
\href{https://git-iit.fh-joanneum.at/oss/thesis-template}{git-iit.fh-joanneum.at}
\item[Categor y :]
\begin{itemize}
\tightlist
\item[]
\item
  \pandocbounded{\includesvg[keepaspectratio]{/assets/icons/16-mortarboard.svg}}
  \href{https://typst.app/universe/search/?category=thesis}{Thesis}
\end{itemize}
\end{description}

\subsubsection{Where to report issues?}\label{where-to-report-issues}

This template is a project of IIT, FH JOANNEUM . Report issues on
\href{https://git-iit.fh-joanneum.at/oss/thesis-template}{their
repository} . You can also try to ask for help with this template on the
\href{https://forum.typst.app}{Forum} .

Please report this template to the Typst team using the
\href{https://typst.app/contact}{contact form} if you believe it is a
safety hazard or infringes upon your rights.

\phantomsection\label{versions}
\subsubsection{Version history}\label{version-history}

\begin{longtable}[]{@{}ll@{}}
\toprule\noalign{}
Version & Release Date \\
\midrule\noalign{}
\endhead
\bottomrule\noalign{}
\endlastfoot
2.0.5 & November 26, 2024 \\
\href{https://typst.app/universe/package/fh-joanneum-iit-thesis/2.0.2/}{2.0.2}
& October 29, 2024 \\
\href{https://typst.app/universe/package/fh-joanneum-iit-thesis/1.2.3/}{1.2.3}
& September 19, 2024 \\
\href{https://typst.app/universe/package/fh-joanneum-iit-thesis/1.2.2/}{1.2.2}
& August 30, 2024 \\
\href{https://typst.app/universe/package/fh-joanneum-iit-thesis/1.2.0/}{1.2.0}
& August 28, 2024 \\
\href{https://typst.app/universe/package/fh-joanneum-iit-thesis/1.1.0/}{1.1.0}
& August 27, 2024 \\
\end{longtable}

Typst GmbH did not create this template and cannot guarantee correct
functionality of this template or compatibility with any version of the
Typst compiler or app.


\title{typst.app/universe/package/unichar}

\phantomsection\label{banner}
\section{unichar}\label{unichar}

{ 0.3.0 }

A partial port of the Unicode Character Database.

\phantomsection\label{readme}
This package ports part of the
\href{https://www.unicode.org/reports/tr44/}{Unicode Character Database}
to Typst. Notably, it includes information from
\href{https://unicode.org/reports/tr44/\#UnicodeData.txt}{UnicodeData.txt}
and \href{https://unicode.org/reports/tr44/\#Blocks.txt}{Blocks.txt} .

\subsection{Usage}\label{usage}

This package defines a single function: \texttt{\ codepoint\ } . It lets
you get the information related to a specific codepoint. The codepoint
can be specified as a string containing a single character, or with its
value.

\begin{Shaded}
\begin{Highlighting}[]
\NormalTok{\#codepoint("√").name \textbackslash{}}
\NormalTok{\#codepoint(sym.times).block.name \textbackslash{}}
\NormalTok{\#codepoint(0x00C9).general{-}category \textbackslash{}}
\NormalTok{\#codepoint(sym.eq).math{-}class}
\end{Highlighting}
\end{Shaded}

\pandocbounded{\includesvg[keepaspectratio]{https://github.com/typst/packages/raw/main/packages/preview/unichar/0.3.0/examples/example-1.svg}}

You can display a codepoint in the style of
\href{https://en.wikipedia.org/wiki/Template:Unichar}{Template:Unichar}
using the \texttt{\ show\ } entry:

\begin{Shaded}
\begin{Highlighting}[]
\NormalTok{\#codepoint("¤").show \textbackslash{}}
\NormalTok{\#codepoint(sym.copyright).show \textbackslash{}}
\NormalTok{\#codepoint(0x1249).show \textbackslash{}}
\NormalTok{\#codepoint(0x100000).show}
\end{Highlighting}
\end{Shaded}

\pandocbounded{\includesvg[keepaspectratio]{https://github.com/typst/packages/raw/main/packages/preview/unichar/0.3.0/examples/example-2.svg}}

\subsection{Changelog}\label{changelog}

\subsubsection{Version 0.3.0}\label{version-0.3.0}

\begin{itemize}
\item
  Add \texttt{\ math-class\ } attribute to codepoints.

  \begin{itemize}
  \tightlist
  \item
    Some codepoints have their math class overridden by Typst. This is
    the Unicode math class, not the one used by Typst.
  \end{itemize}
\item
  The \texttt{\ id\ } of codepoints now returns a string without the
  \texttt{\ "U+"\ } prefix.
\end{itemize}

\subsubsection{Version 0.2.0}\label{version-0.2.0}

\begin{itemize}
\item
  Codepoints now have an \texttt{\ id\ } attribute which is its
  corresponding “U+xxxx� string.
\item
  The \texttt{\ block\ } attribute of a codepoint now contains a
  \texttt{\ name\ } , a \texttt{\ start\ } , and a \texttt{\ size\ } .
\item
  Fix an issue that made some codepoints cause a panic.
\item
  Include data from NameAlias.txt.
\end{itemize}

\subsubsection{Version 0.1.0}\label{version-0.1.0}

\begin{itemize}
\tightlist
\item
  Add the \texttt{\ codepoint\ } function.
\end{itemize}

\subsubsection{How to add}\label{how-to-add}

Copy this into your project and use the import as \texttt{\ unichar\ }

\begin{verbatim}
#import "@preview/unichar:0.3.0"
\end{verbatim}

\includesvg[width=0.16667in,height=0.16667in]{/assets/icons/16-copy.svg}

Check the docs for
\href{https://typst.app/docs/reference/scripting/\#packages}{more
information on how to import packages} .

\subsubsection{About}\label{about}

\begin{description}
\tightlist
\item[Author :]
\href{https://github.com/MDLC01}{Malo}
\item[License:]
MIT AND Unicode-3.0
\item[Current version:]
0.3.0
\item[Last updated:]
September 19, 2024
\item[First released:]
September 14, 2024
\item[Minimum Typst version:]
0.11.0
\item[Archive size:]
202 kB
\href{https://packages.typst.org/preview/unichar-0.3.0.tar.gz}{\pandocbounded{\includesvg[keepaspectratio]{/assets/icons/16-download.svg}}}
\item[Repository:]
\href{https://github.com/MDLC01/unichar}{GitHub}
\item[Categor ies :]
\begin{itemize}
\tightlist
\item[]
\item
  \pandocbounded{\includesvg[keepaspectratio]{/assets/icons/16-code.svg}}
  \href{https://typst.app/universe/search/?category=scripting}{Scripting}
\item
  \pandocbounded{\includesvg[keepaspectratio]{/assets/icons/16-integration.svg}}
  \href{https://typst.app/universe/search/?category=integration}{Integration}
\end{itemize}
\end{description}

\subsubsection{Where to report issues?}\label{where-to-report-issues}

This package is a project of Malo . Report issues on
\href{https://github.com/MDLC01/unichar}{their repository} . You can
also try to ask for help with this package on the
\href{https://forum.typst.app}{Forum} .

Please report this package to the Typst team using the
\href{https://typst.app/contact}{contact form} if you believe it is a
safety hazard or infringes upon your rights.

\phantomsection\label{versions}
\subsubsection{Version history}\label{version-history}

\begin{longtable}[]{@{}ll@{}}
\toprule\noalign{}
Version & Release Date \\
\midrule\noalign{}
\endhead
\bottomrule\noalign{}
\endlastfoot
0.3.0 & September 19, 2024 \\
\href{https://typst.app/universe/package/unichar/0.2.0/}{0.2.0} &
September 15, 2024 \\
\href{https://typst.app/universe/package/unichar/0.1.0/}{0.1.0} &
September 14, 2024 \\
\end{longtable}

Typst GmbH did not create this package and cannot guarantee correct
functionality of this package or compatibility with any version of the
Typst compiler or app.


\title{typst.app/universe/package/tuhi-booklet-vuw}

\phantomsection\label{banner}
\phantomsection\label{template-thumbnail}
\pandocbounded{\includegraphics[keepaspectratio]{https://packages.typst.org/preview/thumbnails/tuhi-booklet-vuw-0.1.0-small.webp}}

\section{tuhi-booklet-vuw}\label{tuhi-booklet-vuw}

{ 0.1.0 }

A course description booklet template for VUW courses.

\href{/app?template=tuhi-booklet-vuw&version=0.1.0}{Create project in
app}

\phantomsection\label{readme}
A Typst template for VUW programme descriptions. To get started:

\begin{Shaded}
\begin{Highlighting}[]
\NormalTok{typst init @preview/tuhi{-}booklet{-}vuw:0.1.0}
\end{Highlighting}
\end{Shaded}

And edit the \texttt{\ main.typ\ } example.

\pandocbounded{\includegraphics[keepaspectratio]{https://github.com/typst/packages/raw/main/packages/preview/tuhi-booklet-vuw/0.1.0/thumbnail.png}}

\subsection{Contributing}\label{contributing}

PRs are welcome! And if you encounter any bugs or have any
requests/ideas, feel free to open an issue.

\href{/app?template=tuhi-booklet-vuw&version=0.1.0}{Create project in
app}

\subsubsection{How to use}\label{how-to-use}

Click the button above to create a new project using this template in
the Typst app.

You can also use the Typst CLI to start a new project on your computer
using this command:

\begin{verbatim}
typst init @preview/tuhi-booklet-vuw:0.1.0
\end{verbatim}

\includesvg[width=0.16667in,height=0.16667in]{/assets/icons/16-copy.svg}

\subsubsection{About}\label{about}

\begin{description}
\tightlist
\item[Author :]
\href{https://github.com/baptiste}{baptiste}
\item[License:]
MPL-2.0
\item[Current version:]
0.1.0
\item[Last updated:]
July 1, 2024
\item[First released:]
July 1, 2024
\item[Archive size:]
170 kB
\href{https://packages.typst.org/preview/tuhi-booklet-vuw-0.1.0.tar.gz}{\pandocbounded{\includesvg[keepaspectratio]{/assets/icons/16-download.svg}}}
\item[Categor y :]
\begin{itemize}
\tightlist
\item[]
\item
  \pandocbounded{\includesvg[keepaspectratio]{/assets/icons/16-envelope.svg}}
  \href{https://typst.app/universe/search/?category=office}{Office}
\end{itemize}
\end{description}

\subsubsection{Where to report issues?}\label{where-to-report-issues}

This template is a project of baptiste . You can also try to ask for
help with this template on the \href{https://forum.typst.app}{Forum} .

Please report this template to the Typst team using the
\href{https://typst.app/contact}{contact form} if you believe it is a
safety hazard or infringes upon your rights.

\phantomsection\label{versions}
\subsubsection{Version history}\label{version-history}

\begin{longtable}[]{@{}ll@{}}
\toprule\noalign{}
Version & Release Date \\
\midrule\noalign{}
\endhead
\bottomrule\noalign{}
\endlastfoot
0.1.0 & July 1, 2024 \\
\end{longtable}

Typst GmbH did not create this template and cannot guarantee correct
functionality of this template or compatibility with any version of the
Typst compiler or app.


\title{typst.app/universe/package/natrix}

\phantomsection\label{banner}
\section{natrix}\label{natrix}

{ 0.1.0 }

Natural and consistent matrix for typst.

\phantomsection\label{readme}
\pandocbounded{\includesvg[keepaspectratio]{https://github.com/typst/packages/raw/main/packages/preview/natrix/0.1.0/natrix.svg}}

\texttt{\ natrix.nat\ } is a drop-in replacement for \texttt{\ mat\ }
with some additional features. \texttt{\ nat\ } ensures that each row in
your matrix should have the similar height, unless one of them becomes
too tall.

At this moment, it is recommended to use \texttt{\ nat\ } only in
display equations, but not in inline equations. This is because
\texttt{\ nat\ } looks a little bit off when used in inline equations.

\subsection{Documentation}\label{documentation}

\subsubsection{\texorpdfstring{\texttt{\ nat\ }}{ nat }}\label{nat}

Every thing is the same as \texttt{\ mat\ } in typst.

This package also provides \texttt{\ bnat\ } , \texttt{\ Bnat\ } ,
\texttt{\ vnat\ } , \texttt{\ Vnat\ } ,

\subsubsection{How to add}\label{how-to-add}

Copy this into your project and use the import as \texttt{\ natrix\ }

\begin{verbatim}
#import "@preview/natrix:0.1.0"
\end{verbatim}

\includesvg[width=0.16667in,height=0.16667in]{/assets/icons/16-copy.svg}

Check the docs for
\href{https://typst.app/docs/reference/scripting/\#packages}{more
information on how to import packages} .

\subsubsection{About}\label{about}

\begin{description}
\tightlist
\item[Author :]
Wenzhuo Liu
\item[License:]
Apache-2.0
\item[Current version:]
0.1.0
\item[Last updated:]
May 16, 2024
\item[First released:]
May 16, 2024
\item[Archive size:]
5.26 kB
\href{https://packages.typst.org/preview/natrix-0.1.0.tar.gz}{\pandocbounded{\includesvg[keepaspectratio]{/assets/icons/16-download.svg}}}
\item[Repository:]
\href{https://github.com/Enter-tainer/natrix}{GitHub}
\end{description}

\subsubsection{Where to report issues?}\label{where-to-report-issues}

This package is a project of Wenzhuo Liu . Report issues on
\href{https://github.com/Enter-tainer/natrix}{their repository} . You
can also try to ask for help with this package on the
\href{https://forum.typst.app}{Forum} .

Please report this package to the Typst team using the
\href{https://typst.app/contact}{contact form} if you believe it is a
safety hazard or infringes upon your rights.

\phantomsection\label{versions}
\subsubsection{Version history}\label{version-history}

\begin{longtable}[]{@{}ll@{}}
\toprule\noalign{}
Version & Release Date \\
\midrule\noalign{}
\endhead
\bottomrule\noalign{}
\endlastfoot
0.1.0 & May 16, 2024 \\
\end{longtable}

Typst GmbH did not create this package and cannot guarantee correct
functionality of this package or compatibility with any version of the
Typst compiler or app.


\title{typst.app/universe/package/typpuccino}

\phantomsection\label{banner}
\section{typpuccino}\label{typpuccino}

{ 0.1.0 }

Use catppuccin palette with Typst.

\phantomsection\label{readme}
User everyone’s favorite
\href{https://github.com/catppuccin/catppuccin}{Catppuccin color
palettes} in your Typst projects.

\subsection{Usage}\label{usage}

To use the Catppuccin color palette in your Typst project, add the
following import statement to your Typst file, and then you can use all
the colors from the Catppuccin color palette.

\begin{Shaded}
\begin{Highlighting}[]
\NormalTok{\#import "@preview/typpuccino:0.1.0": latte, frappe, macchiato, mocha}

\NormalTok{\#square(fill: mocha.red)}
\end{Highlighting}
\end{Shaded}

For more information on available colors, see the
\href{https://github.com/typst/packages/raw/main/packages/preview/typpuccino/0.1.0/example.pdf}{this
example} .

\subsubsection{How to add}\label{how-to-add}

Copy this into your project and use the import as
\texttt{\ typpuccino\ }

\begin{verbatim}
#import "@preview/typpuccino:0.1.0"
\end{verbatim}

\includesvg[width=0.16667in,height=0.16667in]{/assets/icons/16-copy.svg}

Check the docs for
\href{https://typst.app/docs/reference/scripting/\#packages}{more
information on how to import packages} .

\subsubsection{About}\label{about}

\begin{description}
\tightlist
\item[Author :]
\href{https://github.com/TeddyHuang-00}{Nan Huang}
\item[License:]
MIT
\item[Current version:]
0.1.0
\item[Last updated:]
April 13, 2024
\item[First released:]
April 13, 2024
\item[Archive size:]
42.4 kB
\href{https://packages.typst.org/preview/typpuccino-0.1.0.tar.gz}{\pandocbounded{\includesvg[keepaspectratio]{/assets/icons/16-download.svg}}}
\item[Repository:]
\href{https://github.com/TeddyHuang-00/typpuccino}{GitHub}
\item[Categor y :]
\begin{itemize}
\tightlist
\item[]
\item
  \pandocbounded{\includesvg[keepaspectratio]{/assets/icons/16-package.svg}}
  \href{https://typst.app/universe/search/?category=components}{Components}
\end{itemize}
\end{description}

\subsubsection{Where to report issues?}\label{where-to-report-issues}

This package is a project of Nan Huang . Report issues on
\href{https://github.com/TeddyHuang-00/typpuccino}{their repository} .
You can also try to ask for help with this package on the
\href{https://forum.typst.app}{Forum} .

Please report this package to the Typst team using the
\href{https://typst.app/contact}{contact form} if you believe it is a
safety hazard or infringes upon your rights.

\phantomsection\label{versions}
\subsubsection{Version history}\label{version-history}

\begin{longtable}[]{@{}ll@{}}
\toprule\noalign{}
Version & Release Date \\
\midrule\noalign{}
\endhead
\bottomrule\noalign{}
\endlastfoot
0.1.0 & April 13, 2024 \\
\end{longtable}

Typst GmbH did not create this package and cannot guarantee correct
functionality of this package or compatibility with any version of the
Typst compiler or app.


\title{typst.app/universe/package/klaro-ifsc-sj}

\phantomsection\label{banner}
\phantomsection\label{template-thumbnail}
\pandocbounded{\includegraphics[keepaspectratio]{https://packages.typst.org/preview/thumbnails/klaro-ifsc-sj-0.1.0-small.webp}}

\section{klaro-ifsc-sj}\label{klaro-ifsc-sj}

{ 0.1.0 }

Report Typst template for IFSC.

\href{/app?template=klaro-ifsc-sj&version=0.1.0}{Create project in app}

\phantomsection\label{readme}
A report Typst template for \href{https://sj.ifsc.edu.br/}{IFSC-SJ} .

\subsection{Usage}\label{usage}

You can use this template in the Typst web app by clicking “Start from
template� on the dashboard and searching for
\texttt{\ klaro-ifsc-sj\ } .

Alternatively, you can use the CLI to kick this project off using the
command

\begin{verbatim}
typst init @preview/klaro-ifsc-sj
\end{verbatim}

Typst will create a new directory with all the files needed to get you
started.

\subsection{Configuration}\label{configuration}

This template exports the \texttt{\ report\ } function with the
following named arguments:

\begin{itemize}
\tightlist
\item
  \texttt{\ title\ } : The reoirt’s title as string. This is displayed
  at the center of the cover page.
\item
  \texttt{\ subtitle\ } : The report’s subtitle as string. This is
  displayed below the title at the cover page.
\item
  \texttt{\ authors\ } : The array of authors as strings. Each author is
  displayed on a separate line at the cover page.
\item
  \texttt{\ date\ } : The date of the last revision of the report. This
  is displayed at the bottom of the cover page.
\end{itemize}

\href{/app?template=klaro-ifsc-sj&version=0.1.0}{Create project in app}

\subsubsection{How to use}\label{how-to-use}

Click the button above to create a new project using this template in
the Typst app.

You can also use the Typst CLI to start a new project on your computer
using this command:

\begin{verbatim}
typst init @preview/klaro-ifsc-sj:0.1.0
\end{verbatim}

\includesvg[width=0.16667in,height=0.16667in]{/assets/icons/16-copy.svg}

\subsubsection{About}\label{about}

\begin{description}
\tightlist
\item[Author :]
\href{https://gabrielluizep.dev}{Gabriel Luiz Espindola Pedro}
\item[License:]
MIT-0
\item[Current version:]
0.1.0
\item[Last updated:]
March 27, 2024
\item[First released:]
March 27, 2024
\item[Minimum Typst version:]
0.10.0
\item[Archive size:]
41.7 kB
\href{https://packages.typst.org/preview/klaro-ifsc-sj-0.1.0.tar.gz}{\pandocbounded{\includesvg[keepaspectratio]{/assets/icons/16-download.svg}}}
\item[Repository:]
\href{https://github.com/gabrielluizep/klaro-ifsc-sj}{GitHub}
\item[Categor y :]
\begin{itemize}
\tightlist
\item[]
\item
  \pandocbounded{\includesvg[keepaspectratio]{/assets/icons/16-speak.svg}}
  \href{https://typst.app/universe/search/?category=report}{Report}
\end{itemize}
\end{description}

\subsubsection{Where to report issues?}\label{where-to-report-issues}

This template is a project of Gabriel Luiz Espindola Pedro . Report
issues on \href{https://github.com/gabrielluizep/klaro-ifsc-sj}{their
repository} . You can also try to ask for help with this template on the
\href{https://forum.typst.app}{Forum} .

Please report this template to the Typst team using the
\href{https://typst.app/contact}{contact form} if you believe it is a
safety hazard or infringes upon your rights.

\phantomsection\label{versions}
\subsubsection{Version history}\label{version-history}

\begin{longtable}[]{@{}ll@{}}
\toprule\noalign{}
Version & Release Date \\
\midrule\noalign{}
\endhead
\bottomrule\noalign{}
\endlastfoot
0.1.0 & March 27, 2024 \\
\end{longtable}

Typst GmbH did not create this template and cannot guarantee correct
functionality of this template or compatibility with any version of the
Typst compiler or app.


\title{typst.app/universe/package/codedis}

\phantomsection\label{banner}
\section{codedis}\label{codedis}

{ 0.1.0 }

A simple package for displaying code.

\phantomsection\label{readme}
Used to display code files in Typst. Main feature is that it displays
code blocks over multiple pages in a way that implies the code block
continues onto the next page. Also a simple and intuitive syntax for
displaying code blocks.

Usage:

\begin{Shaded}
\begin{Highlighting}[]
\NormalTok{// IMPORT PACKAGE}
\NormalTok{\#import "@preview/codedis:0.1.0": code}

\NormalTok{// READ IN CODE}
\NormalTok{\#let codeblock{-}1 = read("some\_code.py")}
\NormalTok{\#let codeblock{-}2 = read("some\_code.cpp")}

\NormalTok{\#set page(numbering: "1")}
\NormalTok{\#v(80\%)}

\NormalTok{// DEFAULT LANGUAGE IS Python ("py")}
\NormalTok{\#code(codeblock{-}1)}
\NormalTok{\#code(codeblock{-}2, lang: "cpp")}
\end{Highlighting}
\end{Shaded}

Renders to:
\pandocbounded{\includegraphics[keepaspectratio]{https://github.com/AugustinWinther/codedis/assets/30674646/76bb13d5-adc8-457f-bd55-53e3fd5c5df7}}

It is very basic and limited, but it does what I need it too, and hope
that it may be of help to others. I’m most likely not going to develop
it further than this.

\subsubsection{How to add}\label{how-to-add}

Copy this into your project and use the import as \texttt{\ codedis\ }

\begin{verbatim}
#import "@preview/codedis:0.1.0"
\end{verbatim}

\includesvg[width=0.16667in,height=0.16667in]{/assets/icons/16-copy.svg}

Check the docs for
\href{https://typst.app/docs/reference/scripting/\#packages}{more
information on how to import packages} .

\subsubsection{About}\label{about}

\begin{description}
\tightlist
\item[Author :]
\href{https://winther.io}{Augustin Winther}
\item[License:]
MIT
\item[Current version:]
0.1.0
\item[Last updated:]
April 29, 2024
\item[First released:]
April 29, 2024
\item[Archive size:]
2.08 kB
\href{https://packages.typst.org/preview/codedis-0.1.0.tar.gz}{\pandocbounded{\includesvg[keepaspectratio]{/assets/icons/16-download.svg}}}
\item[Repository:]
\href{https://github.com/AugustinWinther/codedis}{GitHub}
\item[Categor y :]
\begin{itemize}
\tightlist
\item[]
\item
  \pandocbounded{\includesvg[keepaspectratio]{/assets/icons/16-package.svg}}
  \href{https://typst.app/universe/search/?category=components}{Components}
\end{itemize}
\end{description}

\subsubsection{Where to report issues?}\label{where-to-report-issues}

This package is a project of Augustin Winther . Report issues on
\href{https://github.com/AugustinWinther/codedis}{their repository} .
You can also try to ask for help with this package on the
\href{https://forum.typst.app}{Forum} .

Please report this package to the Typst team using the
\href{https://typst.app/contact}{contact form} if you believe it is a
safety hazard or infringes upon your rights.

\phantomsection\label{versions}
\subsubsection{Version history}\label{version-history}

\begin{longtable}[]{@{}ll@{}}
\toprule\noalign{}
Version & Release Date \\
\midrule\noalign{}
\endhead
\bottomrule\noalign{}
\endlastfoot
0.1.0 & April 29, 2024 \\
\end{longtable}

Typst GmbH did not create this package and cannot guarantee correct
functionality of this package or compatibility with any version of the
Typst compiler or app.


\title{typst.app/universe/package/tinyset}

\phantomsection\label{banner}
\section{tinyset}\label{tinyset}

{ 0.1.0 }

Simple, consistent, and appealing math homework template

\phantomsection\label{readme}
Extremely simple \href{https://github.com/typst/typst}{typst} package
for writing math problem sets quickly and consistently. Under the hood
it is just typst fundamentals that could be defined by hand, however the
aim of this package is to make you faster and the code easier to read.

\subsection{Usage}\label{usage}

Import styles and create a new header. I like to copy this from the top
of the previous week’s homework (don’t forget to increment the
number!).

Example using proof, question, and part environments. Indentation in
source code is largely ignored and left to personal preference. By
default questions are numbered and each part is lettered, you can change
this based on course / instructor preference.

\begin{Shaded}
\begin{Highlighting}[]
\NormalTok{\#import "@preview/tinyset:0.1.0": *}
\NormalTok{\#header(number: 7, name: "Sylvan Franklin", class: "Math 3551 {-} Fall 2024")}

\NormalTok{+ \#qs[}
\NormalTok{Let $G\_1$ and $G\_2$ be groups, $phi : G\_1 {-}\textgreater{} G\_2$ be a homomorphism, and $H$ be}
\NormalTok{any subgroup of $G\_2$. Define}

\NormalTok{$ phi\^{}({-}1)(H) = \{g in G\_1 : phi(g) in H\}. $}

\NormalTok{+ \#pt[ }
\NormalTok{    Prove that $phi\^{}({-}1)(H)$ is a subgroup of $G\_1$.}
\NormalTok{    \#prf[ Non empty: Since $H$ is a subgroup it contains the indentity, and}
\NormalTok{    since $phi$ is a homomorphism and ... ]}
\NormalTok{]}

\NormalTok{+ \#pt[ }
\NormalTok{    What about a question that you don\textquotesingle{}t need a proof for?}
\NormalTok{    \#ans[Use the ans environment]}
\NormalTok{]}

\NormalTok{]}
\end{Highlighting}
\end{Shaded}

\subsection{Custom shorthand}\label{custom-shorthand}

Sometimes when thinking about math I find it easier to phonetically
write out these symbols instead of using the built in typst classes. For
certain others I find the original symbols annoying to type quickly.

\begin{longtable}[]{@{}ll@{}}
\toprule\noalign{}
shorthand & expansion \\
\midrule\noalign{}
\endhead
\bottomrule\noalign{}
\endlastfoot
implies / impl & ==\textgreater{} \\
iff & \textless==\textgreater{} \\
wlog & without loss of generality \\
inv() & ()\^{}(-1) \\
\end{longtable}

\subsubsection{How to add}\label{how-to-add}

Copy this into your project and use the import as \texttt{\ tinyset\ }

\begin{verbatim}
#import "@preview/tinyset:0.1.0"
\end{verbatim}

\includesvg[width=0.16667in,height=0.16667in]{/assets/icons/16-copy.svg}

Check the docs for
\href{https://typst.app/docs/reference/scripting/\#packages}{more
information on how to import packages} .

\subsubsection{About}\label{about}

\begin{description}
\tightlist
\item[Author :]
Sylvan Franklin
\item[License:]
MIT
\item[Current version:]
0.1.0
\item[Last updated:]
November 6, 2024
\item[First released:]
November 6, 2024
\item[Archive size:]
2.28 kB
\href{https://packages.typst.org/preview/tinyset-0.1.0.tar.gz}{\pandocbounded{\includesvg[keepaspectratio]{/assets/icons/16-download.svg}}}
\item[Repository:]
\href{https://github.com/sylvanfranklin/tinyset}{GitHub}
\item[Discipline :]
\begin{itemize}
\tightlist
\item[]
\item
  \href{https://typst.app/universe/search/?discipline=mathematics}{Mathematics}
\end{itemize}
\item[Categor y :]
\begin{itemize}
\tightlist
\item[]
\item
  \pandocbounded{\includesvg[keepaspectratio]{/assets/icons/16-layout.svg}}
  \href{https://typst.app/universe/search/?category=layout}{Layout}
\end{itemize}
\end{description}

\subsubsection{Where to report issues?}\label{where-to-report-issues}

This package is a project of Sylvan Franklin . Report issues on
\href{https://github.com/sylvanfranklin/tinyset}{their repository} . You
can also try to ask for help with this package on the
\href{https://forum.typst.app}{Forum} .

Please report this package to the Typst team using the
\href{https://typst.app/contact}{contact form} if you believe it is a
safety hazard or infringes upon your rights.

\phantomsection\label{versions}
\subsubsection{Version history}\label{version-history}

\begin{longtable}[]{@{}ll@{}}
\toprule\noalign{}
Version & Release Date \\
\midrule\noalign{}
\endhead
\bottomrule\noalign{}
\endlastfoot
0.1.0 & November 6, 2024 \\
\end{longtable}

Typst GmbH did not create this package and cannot guarantee correct
functionality of this package or compatibility with any version of the
Typst compiler or app.


\title{typst.app/universe/package/gentle-clues}

\phantomsection\label{banner}
\section{gentle-clues}\label{gentle-clues}

{ 1.0.0 }

A package to simply create and add some admonitions to your documents.

{ } Featured Package

\phantomsection\label{readme}
Simple admonitions for typst. Add predefined or define your own.

Inspired from
\href{https://tommilligan.github.io/mdbook-admonish/}{mdbook-admonish} .

\subsection{Overview of all predefined
clues:}\label{overview-of-all-predefined-clues}

\pandocbounded{\includesvg[keepaspectratio]{https://github.com/typst/packages/raw/main/packages/preview/gentle-clues/1.0.0/gc-overview.svg}}

\subsection{Usage}\label{usage}

For full information, see the
\href{https://github.com/jomaway/typst-gentle-clues/blob/main/docs.pdf}{docs.pdf}

To use this package, simply add the following code to your document:

\begin{Shaded}
\begin{Highlighting}[]
\NormalTok{\#import "@preview/gentle{-}clues:1.0.0": *}

\NormalTok{// add an info clue}
\NormalTok{\#info[ This is the info clue ... ]}

\NormalTok{// or a tip with custom title}
\NormalTok{\#tip(title: "Best tip ever")[Check out this cool package]}
\end{Highlighting}
\end{Shaded}

\_This will create an info clue and tip clue inside your document. See
the overview for all available clues.

\subsubsection{Features}\label{features}

This package provides some features which helps to customize the clues
to your liking.

\begin{itemize}
\tightlist
\item
  Set global default for all clues
\item
  Overwrite each parameter on a single clue for changing title, color,
  etc.
\item
  Show or hide a counter value on tasks.
\item
  Define your own clues very easily.
\item
  …
\end{itemize}

For a full list see the
\href{https://github.com/jomaway/typst-gentle-clues/blob/main/docs.pdf}{documentation}
.

\subsection{Language support}\label{language-support}

This package does use
\href{https://github.com/jomaway/typst-linguify}{linguify} to support
multiple languages.

\textbf{Header titles:} The language of the header titles is detected
automatically from the \texttt{\ context\ text.lang\ } . See the file
\href{https://github.com/jomaway/typst-gentle-clues/blob/main/lib/lang.toml}{lang.toml}
for currently supported languages.

If an unsupported language is set it will fallback to english as
default. Feel free to open a PR with your language added to the
\texttt{\ lang.toml\ } file.

\subsection{License}\label{license}

\href{https://github.com/typst/packages/raw/main/packages/preview/gentle-clues/1.0.0/LICENSE}{MIT
License}

\subsection{Changelog}\label{changelog}

\href{https://github.com/typst/packages/raw/main/packages/preview/gentle-clues/1.0.0/CHANGELOG.md}{See
CHANGELOG.md}

\subsubsection{How to add}\label{how-to-add}

Copy this into your project and use the import as
\texttt{\ gentle-clues\ }

\begin{verbatim}
#import "@preview/gentle-clues:1.0.0"
\end{verbatim}

\includesvg[width=0.16667in,height=0.16667in]{/assets/icons/16-copy.svg}

Check the docs for
\href{https://typst.app/docs/reference/scripting/\#packages}{more
information on how to import packages} .

\subsubsection{About}\label{about}

\begin{description}
\tightlist
\item[Author :]
\href{https://github.com/jomaway}{Jomaway}
\item[License:]
MIT
\item[Current version:]
1.0.0
\item[Last updated:]
September 8, 2024
\item[First released:]
September 15, 2023
\item[Minimum Typst version:]
0.11.0
\item[Archive size:]
70.3 kB
\href{https://packages.typst.org/preview/gentle-clues-1.0.0.tar.gz}{\pandocbounded{\includesvg[keepaspectratio]{/assets/icons/16-download.svg}}}
\item[Repository:]
\href{https://github.com/jomaway/typst-gentle-clues}{GitHub}
\item[Categor ies :]
\begin{itemize}
\tightlist
\item[]
\item
  \pandocbounded{\includesvg[keepaspectratio]{/assets/icons/16-package.svg}}
  \href{https://typst.app/universe/search/?category=components}{Components}
\item
  \pandocbounded{\includesvg[keepaspectratio]{/assets/icons/16-chart.svg}}
  \href{https://typst.app/universe/search/?category=visualization}{Visualization}
\end{itemize}
\end{description}

\subsubsection{Where to report issues?}\label{where-to-report-issues}

This package is a project of Jomaway . Report issues on
\href{https://github.com/jomaway/typst-gentle-clues}{their repository} .
You can also try to ask for help with this package on the
\href{https://forum.typst.app}{Forum} .

Please report this package to the Typst team using the
\href{https://typst.app/contact}{contact form} if you believe it is a
safety hazard or infringes upon your rights.

\phantomsection\label{versions}
\subsubsection{Version history}\label{version-history}

\begin{longtable}[]{@{}ll@{}}
\toprule\noalign{}
Version & Release Date \\
\midrule\noalign{}
\endhead
\bottomrule\noalign{}
\endlastfoot
1.0.0 & September 8, 2024 \\
\href{https://typst.app/universe/package/gentle-clues/0.9.0/}{0.9.0} &
July 1, 2024 \\
\href{https://typst.app/universe/package/gentle-clues/0.8.0/}{0.8.0} &
April 29, 2024 \\
\href{https://typst.app/universe/package/gentle-clues/0.7.1/}{0.7.1} &
March 26, 2024 \\
\href{https://typst.app/universe/package/gentle-clues/0.7.0/}{0.7.0} &
March 18, 2024 \\
\href{https://typst.app/universe/package/gentle-clues/0.6.0/}{0.6.0} &
January 11, 2024 \\
\href{https://typst.app/universe/package/gentle-clues/0.5.0/}{0.5.0} &
January 8, 2024 \\
\href{https://typst.app/universe/package/gentle-clues/0.4.0/}{0.4.0} &
November 17, 2023 \\
\href{https://typst.app/universe/package/gentle-clues/0.3.0/}{0.3.0} &
October 20, 2023 \\
\href{https://typst.app/universe/package/gentle-clues/0.2.0/}{0.2.0} &
September 26, 2023 \\
\href{https://typst.app/universe/package/gentle-clues/0.1.0/}{0.1.0} &
September 15, 2023 \\
\end{longtable}

Typst GmbH did not create this package and cannot guarantee correct
functionality of this package or compatibility with any version of the
Typst compiler or app.


\title{typst.app/universe/package/cumcm-muban}

\phantomsection\label{banner}
\phantomsection\label{template-thumbnail}
\pandocbounded{\includegraphics[keepaspectratio]{https://packages.typst.org/preview/thumbnails/cumcm-muban-0.3.0-small.webp}}

\section{cumcm-muban}\label{cumcm-muban}

{ 0.3.0 }

为高教社æ?¯å\ldots¨å›½å¤§å­¦ç''Ÿæ•°å­¦å»ºæ¨¡ç«žèµ›è®¾è®¡çš„ Typst
模�

\href{/app?template=cumcm-muban&version=0.3.0}{Create project in app}

\phantomsection\label{readme}
cumcm-muban
是一个为高教社æ?¯å\ldots¨å›½å¤§å­¦ç''Ÿæ•°å­¦å»ºæ¨¡ç«žèµ›è®¾è®¡çš„
Typst 模�。

\subsection{使ç''¨æ--¹æ³•}\label{uxe4uxbduxe7uxe6uxb9uxe6uxb3}

ä½~å?¯ä»¥åœ¨ Typst
ç½`页åº''ç''¨ä¸­ä½¿ç''¨æ­¤æ¨¡æ?¿ï¼Œå?ªéœ€åœ¨ä»ªè¡¨æ?¿ä¸Šç‚¹å‡» “Start
from template�,然��索 cumcm-muban。

å?¦å¤--,ä½~也å?¯ä»¥ä½¿ç''¨ CLI å`½ä»¤æ?¥å?¯åŠ¨è¿™ä¸ªé¡¹ç›®ã€‚

\begin{verbatim}
typst init @preview/cumcm-muban
\end{verbatim}

Typst
将会创建一个æ--°çš„目录,å\ldots¶ä¸­åŒ\ldots å?«äº†æ‰€æœ‰ä½~开始所需è¦?çš„æ--‡ä»¶ã€‚

\subsection{é\ldots?ç½®}\label{uxe9uxe7uxbd}

此模æ?¿å¯¼å‡ºäº† cumcm 函数,åŒ\ldots å?«ä»¥ä¸‹å`½å??å?‚数:

\begin{itemize}
\tightlist
\item
  title: 论æ--‡çš„æ~‡é¢˜
\item
  problem-chosen: 选择的题目
\item
  team-number: 团队的ç¼--å?·
\item
  college-name: 高æ~¡çš„å??称
\item
  member: 团队æˆ?å`˜çš„å§``å??
\item
  advisor: 指导教师的å§``å??
\item
  date: 竞赛开始的æ---¶é---´
\item
  cover-display: 是å?¦æ˜¾ç¤ºå°?é?¢ä»¥å?Šç¼--å?·é¡µ
\item
  abstract: æ`˜è¦?å†\ldots 容åŒ\ldots 裹在 \texttt{\ {[}{]}\ } 中
\item
  keywords: å\ldots³é''®å­---å†\ldots 容åŒ\ldots 裹在 \texttt{\ ()\ }
  中,使ç''¨é€---å?·åˆ†éš''
\end{itemize}

该函数还接å?---一个å?‚æ•° \texttt{\ body\ }
,ç''¨äºŽä¼~å\ldots¥è®ºæ--‡çš„æ­£æ--‡å†\ldots 容。

该模æ?¿å°†åœ¨æ˜¾ç¤ºè§„则中使ç''¨ \texttt{\ cumcm\ }
函数进行示例调ç''¨æ?¥åˆ?始åŒ--您的项目。如果您想è¦?将现有项目更æ''¹ä¸ºä½¿ç''¨æ­¤æ¨¡æ?¿ï¼Œæ‚¨å?¯ä»¥åœ¨æ--‡ä»¶é¡¶éƒ¨æ·»åŠ~一个类似于以下的显示规则:

\begin{Shaded}
\begin{Highlighting}[]
\NormalTok{\#import "@preview/cumcm{-}muban:0.3.0": *}
\NormalTok{\#show: thmrules}

\NormalTok{\#show: cumcm.with(}
\NormalTok{  title: "全国大学生数学建模竞赛 Typst 模板",}
\NormalTok{  problem{-}chosen: "A",}
\NormalTok{  team{-}number: "1234",}
\NormalTok{  college{-}name: " ",}
\NormalTok{  member: (}
\NormalTok{    A: " ",}
\NormalTok{    B: " ",}
\NormalTok{    C: " ",}
\NormalTok{  ),}
\NormalTok{  advisor: " ",}
\NormalTok{  date: datetime(year: 2023, month: 9, day: 8),}

\NormalTok{  cover{-}display: true,}

\NormalTok{  abstract: [],}
\NormalTok{  keywords: ("Typst", "模板", "数学建模"),}
\NormalTok{)}

\NormalTok{// 正文内容}

\NormalTok{// 参考文献}
\NormalTok{\#bib(bibliography("refs.bib"))}

\NormalTok{// 附录}
\NormalTok{\#appendix("附录标题", "附录内容")}
\end{Highlighting}
\end{Shaded}

\subsection{模�预览}\label{uxe6uxe6uxe9uxe8ux2c6}

\begin{longtable}[]{@{}ccc@{}}
\toprule\noalign{}
æ­£æ--‡1 & æ­£æ--‡2 & 附录 \\
\midrule\noalign{}
\endhead
\bottomrule\noalign{}
\endlastfoot
\pandocbounded{\includegraphics[keepaspectratio]{https://raw.githubusercontent.com/a-kkiri/CUMCM-typst-template/main/template/figures/p4.jpg?raw=true}}
&
\pandocbounded{\includegraphics[keepaspectratio]{https://raw.githubusercontent.com/a-kkiri/CUMCM-typst-template/main/template/figures/p6.jpg?raw=true}}
&
\pandocbounded{\includegraphics[keepaspectratio]{https://raw.githubusercontent.com/a-kkiri/CUMCM-typst-template/main/template/figures/p10.jpg?raw=true}} \\
\end{longtable}

\subsection{âš~ï¸?注æ„?}\label{uxe2ux161-uxefuxe6uxb3uxe6}

\begin{quote}
本模æ?¿ä½¿ç''¨åˆ°çš„å­---ä½``有
中æ˜``宋ä½``(SimSun),中æ˜``é»`ä½``(SimHei),中æ˜``楷ä½``(SimKai),Times
New Romans。这些å­---ä½``为 Windows 系统å†\ldots 置,ä¸?过对于
WebAPP/Linux/MacOS 使ç''¨è€\ldots 请到ä»``åº``自行获å?--
\end{quote}

\subsection{æ›´æ''¹è®°å½•}\label{uxe6uxe6uxb9uxe8uxe5uxbd}

2024-08-20

\begin{itemize}
\tightlist
\item
  æ›´æ''¹é™„录页代ç~?框æ~·å¼?
\item
  ä¿®å¤?æ~‡é¢˜æ---~法修æ''¹çš„é---®é¢˜
\item
  增åŠ~函数 \texttt{\ appendix\ } ç''¨äºŽæ˜¾ç¤ºé™„录å†\ldots 容
\item
  å°†ç²---ä½``çš„ \texttt{\ stroke\ } å?‚数设置为 0.02857em
\end{itemize}

\href{/app?template=cumcm-muban&version=0.3.0}{Create project in app}

\subsubsection{How to use}\label{how-to-use}

Click the button above to create a new project using this template in
the Typst app.

You can also use the Typst CLI to start a new project on your computer
using this command:

\begin{verbatim}
typst init @preview/cumcm-muban:0.3.0
\end{verbatim}

\includesvg[width=0.16667in,height=0.16667in]{/assets/icons/16-copy.svg}

\subsubsection{About}\label{about}

\begin{description}
\tightlist
\item[Author :]
\href{https://github.com/a-kkiri}{Akkiri}
\item[License:]
Apache-2.0
\item[Current version:]
0.3.0
\item[Last updated:]
August 22, 2024
\item[First released:]
March 18, 2024
\item[Archive size:]
235 kB
\href{https://packages.typst.org/preview/cumcm-muban-0.3.0.tar.gz}{\pandocbounded{\includesvg[keepaspectratio]{/assets/icons/16-download.svg}}}
\item[Repository:]
\href{https://github.com/a-kkiri/CUMCM-typst-template}{GitHub}
\item[Discipline s :]
\begin{itemize}
\tightlist
\item[]
\item
  \href{https://typst.app/universe/search/?discipline=mathematics}{Mathematics}
\item
  \href{https://typst.app/universe/search/?discipline=computer-science}{Computer
  Science}
\end{itemize}
\item[Categor y :]
\begin{itemize}
\tightlist
\item[]
\item
  \pandocbounded{\includesvg[keepaspectratio]{/assets/icons/16-mortarboard.svg}}
  \href{https://typst.app/universe/search/?category=thesis}{Thesis}
\end{itemize}
\end{description}

\subsubsection{Where to report issues?}\label{where-to-report-issues}

This template is a project of Akkiri . Report issues on
\href{https://github.com/a-kkiri/CUMCM-typst-template}{their repository}
. You can also try to ask for help with this template on the
\href{https://forum.typst.app}{Forum} .

Please report this template to the Typst team using the
\href{https://typst.app/contact}{contact form} if you believe it is a
safety hazard or infringes upon your rights.

\phantomsection\label{versions}
\subsubsection{Version history}\label{version-history}

\begin{longtable}[]{@{}ll@{}}
\toprule\noalign{}
Version & Release Date \\
\midrule\noalign{}
\endhead
\bottomrule\noalign{}
\endlastfoot
0.3.0 & August 22, 2024 \\
\href{https://typst.app/universe/package/cumcm-muban/0.2.0/}{0.2.0} &
April 3, 2024 \\
\href{https://typst.app/universe/package/cumcm-muban/0.1.0/}{0.1.0} &
March 18, 2024 \\
\end{longtable}

Typst GmbH did not create this template and cannot guarantee correct
functionality of this template or compatibility with any version of the
Typst compiler or app.


\title{typst.app/universe/package/solving-physics}

\phantomsection\label{banner}
\section{solving-physics}\label{solving-physics}

{ 0.1.0 }

A package to formulate the solution to a physical problem

\phantomsection\label{readme}
The easiest method is to import \texttt{\ solving-physics:\ task\ } from
the \texttt{\ @preview\ } package:

\begin{Shaded}
\begin{Highlighting}[]
\NormalTok{\#import "@preview/solving{-}physics:0.1.0": *}
\end{Highlighting}
\end{Shaded}

\begin{Shaded}
\begin{Highlighting}[]
\NormalTok{\#task(}
\NormalTok{  given: [}
\NormalTok{    $mu = 0.4$ \textbackslash{}}
\NormalTok{    $g = 10$ \textbackslash{}}
\NormalTok{    $m = 20$}
\NormalTok{  ],}
\NormalTok{  find: [}
\NormalTok{    $F$ {-}{-}{-} ?}
\NormalTok{  ],}
\NormalTok{  fig: image("./example.png", width: 5cm)}
\NormalTok{)[}
\NormalTok{  \#lorem(100)}
\NormalTok{]}
\end{Highlighting}
\end{Shaded}

\pandocbounded{\includegraphics[keepaspectratio]{https://raw.githubusercontent.com/yegorweb/solving-physics/master/examples/example1.png}}

\begin{Shaded}
\begin{Highlighting}[]
\NormalTok{\#task(}
\NormalTok{  given: [}
\NormalTok{    $mu = 0.4$ \textbackslash{}}
\NormalTok{    $g = 10$ \textbackslash{}}
\NormalTok{    $m = 20$}
\NormalTok{  ],}
\NormalTok{  find: [}
\NormalTok{    $F$ {-}{-}{-} ?}
\NormalTok{  ],}
\NormalTok{  stroke: "full"}
\NormalTok{)[]}
\end{Highlighting}
\end{Shaded}

\pandocbounded{\includesvg[keepaspectratio]{https://raw.githubusercontent.com/yegorweb/solving-physics/master/examples/example2.svg}}

\begin{Shaded}
\begin{Highlighting}[]
\NormalTok{\#task(}
\NormalTok{  given: [}
\NormalTok{    $mu = 0.4$ \textbackslash{}}
\NormalTok{    $g = 10$ \textbackslash{}}
\NormalTok{    $m = 20$}
\NormalTok{  ],}
\NormalTok{  find: [}
\NormalTok{    $F$ {-}{-}{-} ?}
\NormalTok{  ],}
\NormalTok{  stroke: "find"}
\NormalTok{)[]}
\end{Highlighting}
\end{Shaded}

\pandocbounded{\includesvg[keepaspectratio]{https://raw.githubusercontent.com/yegorweb/solving-physics/master/examples/example3.svg}}

\begin{Shaded}
\begin{Highlighting}[]
\NormalTok{\#task(}
\NormalTok{  given: [}
\NormalTok{    $mu = 0.4$ \textbackslash{}}
\NormalTok{    $g = 10$ \textbackslash{}}
\NormalTok{    $m = 20$}
\NormalTok{  ],}
\NormalTok{  find: [}
\NormalTok{    $F$ {-}{-}{-} ?}
\NormalTok{  ],}
\NormalTok{  stroke: none}
\NormalTok{)[]}
\end{Highlighting}
\end{Shaded}

\pandocbounded{\includesvg[keepaspectratio]{https://raw.githubusercontent.com/yegorweb/solving-physics/master/examples/example4.svg}}

If you have so large given you may use \texttt{\ given-width\ } :

\begin{Shaded}
\begin{Highlighting}[]
\NormalTok{\#task(}
\NormalTok{  given: [}
\NormalTok{    $mu = 0.4$ \textbackslash{}}
\NormalTok{    $g = 10$ \textbackslash{}}
\NormalTok{    $m = 20$ \textbackslash{}}
\NormalTok{    \#lorem(10)}
\NormalTok{  ],}
\NormalTok{  given{-}width: 6em,}
\NormalTok{  find: [}
\NormalTok{    $F$ {-}{-}{-} ?}
\NormalTok{  ],}
\NormalTok{)[]}
\end{Highlighting}
\end{Shaded}

\pandocbounded{\includesvg[keepaspectratio]{https://raw.githubusercontent.com/yegorweb/solving-physics/master/examples/example5.svg}}

You may locate you figure on the center of body by
\texttt{\ fig-align:\ top\ +\ center\ }

\begin{Shaded}
\begin{Highlighting}[]
\NormalTok{\#task(}
\NormalTok{  given: [}
\NormalTok{    $mu = 0.4$ \textbackslash{}}
\NormalTok{    $g = 10$ \textbackslash{}}
\NormalTok{    $m = 20$}
\NormalTok{  ],}
\NormalTok{  find: [}
\NormalTok{    $F$ {-}{-}{-} ?}
\NormalTok{  ],}
\NormalTok{  fig: image("./example.png", width: 60\%),}
\NormalTok{  fig{-}align: top + center}
\NormalTok{)[}
\NormalTok{  \#lorem(100)}
\NormalTok{]}
\end{Highlighting}
\end{Shaded}

\pandocbounded{\includegraphics[keepaspectratio]{https://raw.githubusercontent.com/yegorweb/solving-physics/master/examples/example6.png}}

\subsubsection{How to add}\label{how-to-add}

Copy this into your project and use the import as
\texttt{\ solving-physics\ }

\begin{verbatim}
#import "@preview/solving-physics:0.1.0"
\end{verbatim}

\includesvg[width=0.16667in,height=0.16667in]{/assets/icons/16-copy.svg}

Check the docs for
\href{https://typst.app/docs/reference/scripting/\#packages}{more
information on how to import packages} .

\subsubsection{About}\label{about}

\begin{description}
\tightlist
\item[Author :]
Yegor Knyazev
\item[License:]
MIT
\item[Current version:]
0.1.0
\item[Last updated:]
May 13, 2024
\item[First released:]
May 13, 2024
\item[Archive size:]
1.86 kB
\href{https://packages.typst.org/preview/solving-physics-0.1.0.tar.gz}{\pandocbounded{\includesvg[keepaspectratio]{/assets/icons/16-download.svg}}}
\item[Repository:]
\href{https://github.com/yegorweb/solving-physics}{GitHub}
\item[Discipline s :]
\begin{itemize}
\tightlist
\item[]
\item
  \href{https://typst.app/universe/search/?discipline=chemistry}{Chemistry}
\item
  \href{https://typst.app/universe/search/?discipline=education}{Education}
\item
  \href{https://typst.app/universe/search/?discipline=physics}{Physics}
\end{itemize}
\item[Categor y :]
\begin{itemize}
\tightlist
\item[]
\item
  \pandocbounded{\includesvg[keepaspectratio]{/assets/icons/16-package.svg}}
  \href{https://typst.app/universe/search/?category=components}{Components}
\end{itemize}
\end{description}

\subsubsection{Where to report issues?}\label{where-to-report-issues}

This package is a project of Yegor Knyazev . Report issues on
\href{https://github.com/yegorweb/solving-physics}{their repository} .
You can also try to ask for help with this package on the
\href{https://forum.typst.app}{Forum} .

Please report this package to the Typst team using the
\href{https://typst.app/contact}{contact form} if you believe it is a
safety hazard or infringes upon your rights.

\phantomsection\label{versions}
\subsubsection{Version history}\label{version-history}

\begin{longtable}[]{@{}ll@{}}
\toprule\noalign{}
Version & Release Date \\
\midrule\noalign{}
\endhead
\bottomrule\noalign{}
\endlastfoot
0.1.0 & May 13, 2024 \\
\end{longtable}

Typst GmbH did not create this package and cannot guarantee correct
functionality of this package or compatibility with any version of the
Typst compiler or app.


