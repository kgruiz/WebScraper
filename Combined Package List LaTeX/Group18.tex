\title{typst.app/universe/package/xarrow}

\phantomsection\label{banner}
\section{xarrow}\label{xarrow}

{ 0.3.1 }

Variable-length arrows in Typst.

\phantomsection\label{readme}
Variable-length arrows in Typst, fitting the width of a given content.

\subsection{Usage}\label{usage}

This library mainly provides the \texttt{\ xarrow\ } function. This
function takes one positional argument, which is the content to display
on top of the arrow. Additionally, the library provides the following
arrow styles:

\begin{itemize}
\tightlist
\item
  \texttt{\ xarrowDashed\ } using arrow \texttt{\ sym.arrow.dashed\ } .
\item
  \texttt{\ xarrowDouble\ } using arrow
  \texttt{\ sym.arrow.double.long\ } ;
\item
  \texttt{\ xarrowHook\ } using arrow \texttt{\ sym.arrow.hook\ } ;
\item
  \texttt{\ xarrowSquiggly\ } using arrow
  \texttt{\ sym.arrow.long.squiggly\ } ;
\item
  \texttt{\ xarrowTwoHead\ } using arrow \texttt{\ sym.arrow.twohead\ }
  ;
\item
  …
\end{itemize}

These names use camlCase in order to be simply called from math mode.
This may change in the future, if it becomes possible to have the
function names mirror the dot-separated name of the symbols themselves.

\subsubsection{Arguments}\label{arguments}

Users can provide the following arguments to any of the
previously-mentioned functions:

\begin{itemize}
\tightlist
\item
  \texttt{\ width\ } defines the width of the arrow. It defaults to
  \texttt{\ auto\ } , which makes the arrow adapt to the size of the
  body.
\item
  \texttt{\ margins\ } defines the spacing on each side of the
  \texttt{\ body\ } argument. Ignored when \texttt{\ width\ } is not
  \texttt{\ auto\ } .
\item
  \texttt{\ position\ } defines whether the main \texttt{\ body\ }
  argument will be set above or below the arrow. Default is
  \texttt{\ top\ } ; the only other accepted value is
  \texttt{\ bottom\ } .
\item
  \texttt{\ opposite\ } sets the content that is displayed on the other,
  non-default side of the arrow. Default is \texttt{\ none\ } .
\end{itemize}

\subsubsection{Example}\label{example}

\begin{verbatim}
#import "@preview/xarrow:0.3.0": xarrow, xarrowSquiggly, xarrowTwoHead

$
  a xarrow(sym: <--, QQ\, 1 + 1^4) b \
  c xarrowSquiggly("very long boi") d \
  c / ( a xarrowTwoHead("NP" limits(sum)^*) b times 4)
$
\end{verbatim}

\subsection{Customisation}\label{customisation}

The \texttt{\ xarrow\ } function has several named arguments which serve
to create new arrow designs:

\begin{itemize}
\tightlist
\item
  \texttt{\ sym\ } is the base symbol.
\item
  \texttt{\ sections\ } defines the way the symbol is divided. Drawing
  an arrow consists of drawing its tail, then repeating a central part
  that is defined by \texttt{\ sections\ } , then drawing the head. This
  is the parameter that has to be tweaked if observing artefacts.
  \texttt{\ sections\ } are given as two ratios, delimiting respectively
  the beginning and the end of the central, repeated part of the symbol.
\item
  \texttt{\ partial\_repeats\ } indicates whether the central part of
  the symbol can be partially repeated at the end in order to match the
  exact desired width. This has to be disabled when the repeated part
  has a clear period (like the squiggly arrow).
\end{itemize}

\subsubsection{Example}\label{example-1}

\begin{verbatim}
#let xarrowSquigglyBottom = xarrow.with(
  sym: sym.arrow.long.squiggly,
  sections: (20%, 45%),
  partial_repeats: false,
  position: bottom,
)
\end{verbatim}

\subsection{Limitations}\label{limitations}

\begin{itemize}
\tightlist
\item
  The predefined arrows are tweaked with the Computer Modern Math font
  in mind. With different glyphs, more sophisticated arrows will require
  manual modifications (of the \texttt{\ sections\ } argument) to be
  rendered correctly.
\item
  The \texttt{\ width\ } argument cannot be given ratio/fractions like
  other shapes. This would be a nice feature to have, in order to be
  able to create an arrow that takes 50\% of the available line width
  for instance.
\item
  I would like to make a proper manual for this library in the future,
  using something cool like
  \href{https://github.com/jneug/typst-mantys}{mantys} .
\end{itemize}

\subsubsection{How to add}\label{how-to-add}

Copy this into your project and use the import as \texttt{\ xarrow\ }

\begin{verbatim}
#import "@preview/xarrow:0.3.1"
\end{verbatim}

\includesvg[width=0.16667in,height=0.16667in]{/assets/icons/16-copy.svg}

Check the docs for
\href{https://typst.app/docs/reference/scripting/\#packages}{more
information on how to import packages} .

\subsubsection{About}\label{about}

\begin{description}
\tightlist
\item[Author :]
loutr
\item[License:]
GPL-3.0-only
\item[Current version:]
0.3.1
\item[Last updated:]
March 20, 2024
\item[First released:]
July 10, 2023
\item[Minimum Typst version:]
0.11.0
\item[Archive size:]
3.50 kB
\href{https://packages.typst.org/preview/xarrow-0.3.1.tar.gz}{\pandocbounded{\includesvg[keepaspectratio]{/assets/icons/16-download.svg}}}
\item[Repository:]
\href{https://codeberg.org/loutr/typst-xarrow/}{Codeberg}
\end{description}

\subsubsection{Where to report issues?}\label{where-to-report-issues}

This package is a project of loutr . Report issues on
\href{https://codeberg.org/loutr/typst-xarrow/}{their repository} . You
can also try to ask for help with this package on the
\href{https://forum.typst.app}{Forum} .

Please report this package to the Typst team using the
\href{https://typst.app/contact}{contact form} if you believe it is a
safety hazard or infringes upon your rights.

\phantomsection\label{versions}
\subsubsection{Version history}\label{version-history}

\begin{longtable}[]{@{}ll@{}}
\toprule\noalign{}
Version & Release Date \\
\midrule\noalign{}
\endhead
\bottomrule\noalign{}
\endlastfoot
0.3.1 & March 20, 2024 \\
\href{https://typst.app/universe/package/xarrow/0.3.0/}{0.3.0} & January
10, 2024 \\
\href{https://typst.app/universe/package/xarrow/0.2.0/}{0.2.0} &
September 26, 2023 \\
\href{https://typst.app/universe/package/xarrow/0.1.1/}{0.1.1} & July
11, 2023 \\
\href{https://typst.app/universe/package/xarrow/0.1.0/}{0.1.0} & July
10, 2023 \\
\end{longtable}

Typst GmbH did not create this package and cannot guarantee correct
functionality of this package or compatibility with any version of the
Typst compiler or app.


\title{typst.app/universe/package/tutor}

\phantomsection\label{banner}
\section{tutor}\label{tutor}

{ 0.7.0 }

Utilities to create exams.

\phantomsection\label{readme}
Utilities to write exams and exercises with integrated solutions. Set
the variable \texttt{\ \#(cfg.sol\ =\ true)\ } to display the solutions
of a document.

Currently the following features are supported:

\begin{itemize}
\tightlist
\item
  Automatic total point calculation through the \texttt{\ \#points()\ }
  and \texttt{\ \#totalpoints()\ } functions.
\item
  Checkboxes that are either blank or show the solution state using eg.
  \texttt{\ \#checkbox(cfg,\ true)\ } .
\item
  Display blank lines allowing students to write their answer using eg.
  \texttt{\ \#lines(cfg,\ 3)\ } .
\item
  A proposition for a project structure allowing self-contained
  exercises and a mechanism to show or hide the solutions of an
  exercise.
\end{itemize}

\subsection{Usage}\label{usage}

\subsubsection{Minimal Example}\label{minimal-example}

\begin{Shaded}
\begin{Highlighting}[]
\NormalTok{\#import "@local/tutor:0.6.1": points, totalpoints, lines, checkbox, default{-}config}

\NormalTok{\#let cfg = default{-}config()}
\NormalTok{// enable solution mode}
\NormalTok{\#(cfg.sol = true)}

\NormalTok{// display 3 lines (for hand written answer)}
\NormalTok{\#lines(cfg, 3)}
\NormalTok{// checkbox for multiple choice (indicates correct state)}
\NormalTok{\#checkbox(cfg, true)}

\NormalTok{// show achievable points}
\NormalTok{Max. points: \#points(2)}
\NormalTok{Max. points: \#points(3)}
\NormalTok{// show sum of all total achievable points (will show 5)}
\NormalTok{Total points: \#totalpoints(cfg)}
\end{Highlighting}
\end{Shaded}

\subsubsection{Practical Example}\label{practical-example}

Check
\href{https://github.com/rangerjo/tutor/tree/main/example}{example} for
a more practical example.

\texttt{\ tutor\ } is best used with the following directory and file
structure:

\begin{verbatim}
├── main.typ
├── src
│   ├── ex1
│   │   └── ex.typ
│   └── ex2
│       └── ex.typ
└── tutor.toml
\end{verbatim}

Every directory in \texttt{\ src\ } holds one self-contained exercise.
The exercises can be imported into \texttt{\ main.typ\ } :

\begin{Shaded}
\begin{Highlighting}[]
\NormalTok{\#import "@local/tutor:0.6.1": totalpoints, lines, default{-}config}

\NormalTok{\#import "src/ex1/ex.typ" as ex1}
\NormalTok{\#import "src/ex2/ex.typ" as ex2}


\NormalTok{\#let cfg = default{-}config()}
\NormalTok{\#ex1.exercise(cfg)}
\NormalTok{\#ex2.exercise(cfg)}
\end{Highlighting}
\end{Shaded}

Supporting self-contained exercises is one of \texttt{\ tutor\ } s
primary design goals. Each exercise lives within a folder and can easily
be copied or referenced in a new document.

An exercise is a folder that contains an \texttt{\ ex.typ\ } file along
with any other assets (images, source code aso). The following exercise
shows a practical usage of the \texttt{\ \#checkbox()\ } and
\texttt{\ \#points()\ } functions.

\texttt{\ src/ex1/ex.typ\ }

\begin{Shaded}
\begin{Highlighting}[]
\NormalTok{\#import "@local/tutor:0.6.1": points, checkbox}

\NormalTok{\#let exercise(cfg) = [}
\NormalTok{\#heading(level:cfg.lvl, [Abbreviation FHIR (\#points(1) point)])}

\NormalTok{What does FHIR stand for?}

\NormalTok{\#set list(marker: none)}
\NormalTok{{-} \#checkbox(cfg, false)  Finally He Is Real}
\NormalTok{{-} \#checkbox(cfg, true)   Fast Health Interoperability Resources}
\NormalTok{{-} \#checkbox(cfg, false)   First Health Inactivation Restriction}

\NormalTok{\#if cfg.sol \{}
\NormalTok{  [ Further explanation: FHIR is the new standard developed by HL7. ]}
\NormalTok{\}}
\NormalTok{]}
\end{Highlighting}
\end{Shaded}

Finally this second example shows the \texttt{\ \#lines()\ } function.
\texttt{\ src/ex2/ex.typ\ }

\begin{Shaded}
\begin{Highlighting}[]
\NormalTok{\#import "@local/tutor:0.6.1": points, lines }

\NormalTok{\#let exercise(cfg) = [}
\NormalTok{\#heading(level:cfg.lvl, [FHIR vs HL7v2 (\#points(4.5) points)])}

\NormalTok{List two differences between HL7v2 and FHIR:}

\NormalTok{+ \#if cfg.sol \{ [ HL7v2 uses a non{-}standard line format, where as FHIR uses XML or JSON] \} else \{ [ \#lines(cfg, 3) ] \}}
\NormalTok{+ \#if cfg.sol \{ [ FHIR specifies various resources that can be queried, where as HL7v2 has a number of fixed fields that are either filled in or not]\} else \{ [ \#lines(cfg, 3) ] \}}
\NormalTok{]}
\end{Highlighting}
\end{Shaded}

This would then give the following output in question mode (
\texttt{\ \#(cfg.sol=false)\ } ) and in solution mode (
\texttt{\ \#(cfg.sol=true)\ } ):
\pandocbounded{\includegraphics[keepaspectratio]{https://raw.githubusercontent.com/rangerjo/tutor/main/imgs/example_mod.png}}

\subsection{Utilities}\label{utilities}

\subsubsection{lines}\label{lines}

\texttt{\ \#lines(cfg,\ count)\ } prints \texttt{\ count\ } lines for
students to write their answer.

Configuration:

\begin{Shaded}
\begin{Highlighting}[]
\NormalTok{// Vertical line spacing between rows. }
\NormalTok{\#(cfg.utils.lines.spacing = 8mm)}
\end{Highlighting}
\end{Shaded}

\subsubsection{grid}\label{grid}

\texttt{\ \#grid(cfg,\ width,\ height)\ } prints a grid for students to
write their answer.

Configuration:

\begin{Shaded}
\begin{Highlighting}[]
\NormalTok{// Grid spacing. }
\NormalTok{\#(cfg.utils.grid.spacing = 4mm)}
\end{Highlighting}
\end{Shaded}

\subsubsection{checkbox}\label{checkbox}

\texttt{\ \#checkbox(cfg,\ answer)\ } shows a checkbox. In solution
mode, the checkbox is shown filled out.

Configuration:

\begin{Shaded}
\begin{Highlighting}[]
\NormalTok{// Symbol to show if answer is true }
\NormalTok{\#(cfg.utils.checkbox.sym\_true = "☒")}
\NormalTok{// Symbol to show if answer is false}
\NormalTok{\#(cfg.utils.checkbox.sym\_false = "☐")}
\NormalTok{// Symbol to show in question mode}
\NormalTok{\#(cfg.utils.checkbox.sym\_question = "☐")}
\end{Highlighting}
\end{Shaded}

\subsubsection{points}\label{points}

\texttt{\ \#points(cfg,\ num)\ } displays the given \texttt{\ num\ }
while adding its value to the total points counter.

Configuration: none

\subsubsection{totalpoints}\label{totalpoints}

\texttt{\ \#totalpoints(cfg)\ } shows the final value of the total
points counter.

Configuration:

\begin{Shaded}
\begin{Highlighting}[]
\NormalTok{// If points() is used in the outline, totalpoints value becomes doubled.}
\NormalTok{// By setting outline to true, totalpoints gets divided by half.}
\NormalTok{\#(cfg.utils.totalpoints.outline = false)}
\end{Highlighting}
\end{Shaded}

\subsection{Modes}\label{modes}

\texttt{\ tutor\ } comes with a solution and a test mode.

\subsubsection{solution mode}\label{solution-mode}

Solution mode controls wheter solutions are shown or not. This mode
controls eg. the utility \texttt{\ \#checkbox(cfg,\ answer)\ } .

\begin{enumerate}
\tightlist
\item
  \texttt{\ (cfg.sol\ =\ false)\ } : Solutions are hidden. This is used
  for the actual exam handed out to students.
\item
  \texttt{\ (cfg.sol\ =\ true)\ } : Solutions are shown. This is used to
  create the exam solutions.
\end{enumerate}

You can also use the following helper functions:

\begin{itemize}
\tightlist
\item
  \texttt{\ if-sol(cfg,{[}Content\ only\ shown\ in\ solution\ mode.{]})\ }
\item
  \texttt{\ if-sol-else(cfg,{[}Content\ only\ shown\ in\ solution\ mode.{]},\ {[}Content\ only\ shown\ in\ exam\ mode.{]})\ }
\end{itemize}

\subsubsection{test mode}\label{test-mode}

Test mode can be used to show or hide additional information. In test
mode, one might want

\begin{enumerate}
\item
  \texttt{\ (cfg.test\ =\ true)\ } : Test information are shown. Use
  this eg. to display \texttt{\ \#points(4)\ } . This is used in case
  the document is used as an exam/test.
\item
  \texttt{\ (cfg.test\ =\ false)\ } : Test information are hidden. This
  is used in case the document is used as an excerise.
\end{enumerate}

The following would show the points only in test mode.

\begin{Shaded}
\begin{Highlighting}[]
\NormalTok{\#if cfg.test \{}
\NormalTok{  \#points(4)}
\NormalTok{\}}
\end{Highlighting}
\end{Shaded}

Or you can use the following helper functions:

\begin{itemize}
\tightlist
\item
  \texttt{\ if-test(cfg,{[}Content\ only\ shown\ in\ test\ mode.{]})\ }
\item
  \texttt{\ if-test-else(cfg,{[}Content\ only\ shown\ in\ test\ mode.{]},\ {[}Content\ only\ shown\ in\ exercise\ mode.{]})\ }
\end{itemize}

\subsection{Configuration}\label{configuration}

\texttt{\ tutor\ } is designed to create exams and solutions with one
single document source. Furthermore, the individual utilities provided
by \texttt{\ tutor\ } can be configured. This can be done in one of
three ways:

\begin{enumerate}
\tightlist
\item
  Use the \texttt{\ \#default-config()\ } function and patch your
  configuration. The following example would configure the solution mode
  and basic line spacings to 8 millimeters:
\end{enumerate}

\begin{Shaded}
\begin{Highlighting}[]
\NormalTok{\#let cfg = default{-}config()}
\NormalTok{\#(cfg.sol = false)}
\NormalTok{\#(cfg.utils.lines.spacing = 8mm)}
\end{Highlighting}
\end{Shaded}

\begin{enumerate}
\setcounter{enumi}{1}
\tightlist
\item
  Use an external file to hold the configurations in your prefered
  format. See
  \href{https://github.com/rangerjo/tutor/blob/main/example/tutor.toml}{tutor.toml}
  for a configuration in TOML. Load the configuration into your main
  document using
\end{enumerate}

\begin{Shaded}
\begin{Highlighting}[]
\NormalTok{\#let cfg = toml("tutor.toml")}
\end{Highlighting}
\end{Shaded}

\begin{enumerate}
\setcounter{enumi}{2}
\tightlist
\item
  Use typst’s input feature added with compiler version 0.11.0. Add
  the following snippet to load the configuration, then overwrite it
  from the CLI like this:
  \texttt{\ typst\ compile\ -\/-input\ tutor\_sol=true\ main.typ\ }
\end{enumerate}

\begin{Shaded}
\begin{Highlighting}[]
\NormalTok{\#let cfg = toml("tutor.toml")}

\NormalTok{\#if sys.inputs.tutor\_sol == "true" \{}
\NormalTok{  (cfg.sol = true)}
\NormalTok{\} else if sys.inputs.tutor\_sol == "false" \{}
\NormalTok{  (cfg.sol = false)}
\NormalTok{\}}
\end{Highlighting}
\end{Shaded}

\subsubsection{How to add}\label{how-to-add}

Copy this into your project and use the import as \texttt{\ tutor\ }

\begin{verbatim}
#import "@preview/tutor:0.7.0"
\end{verbatim}

\includesvg[width=0.16667in,height=0.16667in]{/assets/icons/16-copy.svg}

Check the docs for
\href{https://typst.app/docs/reference/scripting/\#packages}{more
information on how to import packages} .

\subsubsection{About}\label{about}

\begin{description}
\tightlist
\item[Author :]
Jonas Amstutz
\item[License:]
MIT
\item[Current version:]
0.7.0
\item[Last updated:]
October 9, 2024
\item[First released:]
October 17, 2023
\item[Minimum Typst version:]
0.11.0
\item[Archive size:]
4.82 kB
\href{https://packages.typst.org/preview/tutor-0.7.0.tar.gz}{\pandocbounded{\includesvg[keepaspectratio]{/assets/icons/16-download.svg}}}
\item[Repository:]
\href{https://github.com/rangerjo/tutor}{GitHub}
\item[Discipline :]
\begin{itemize}
\tightlist
\item[]
\item
  \href{https://typst.app/universe/search/?discipline=education}{Education}
\end{itemize}
\item[Categor y :]
\begin{itemize}
\tightlist
\item[]
\item
  \pandocbounded{\includesvg[keepaspectratio]{/assets/icons/16-package.svg}}
  \href{https://typst.app/universe/search/?category=components}{Components}
\end{itemize}
\end{description}

\subsubsection{Where to report issues?}\label{where-to-report-issues}

This package is a project of Jonas Amstutz . Report issues on
\href{https://github.com/rangerjo/tutor}{their repository} . You can
also try to ask for help with this package on the
\href{https://forum.typst.app}{Forum} .

Please report this package to the Typst team using the
\href{https://typst.app/contact}{contact form} if you believe it is a
safety hazard or infringes upon your rights.

\phantomsection\label{versions}
\subsubsection{Version history}\label{version-history}

\begin{longtable}[]{@{}ll@{}}
\toprule\noalign{}
Version & Release Date \\
\midrule\noalign{}
\endhead
\bottomrule\noalign{}
\endlastfoot
0.7.0 & October 9, 2024 \\
\href{https://typst.app/universe/package/tutor/0.6.1/}{0.6.1} & October
9, 2024 \\
\href{https://typst.app/universe/package/tutor/0.4.0/}{0.4.0} & March
19, 2024 \\
\href{https://typst.app/universe/package/tutor/0.3.0/}{0.3.0} & October
17, 2023 \\
\end{longtable}

Typst GmbH did not create this package and cannot guarantee correct
functionality of this package or compatibility with any version of the
Typst compiler or app.


\title{typst.app/universe/package/modern-uit-thesis}

\phantomsection\label{banner}
\phantomsection\label{template-thumbnail}
\pandocbounded{\includegraphics[keepaspectratio]{https://packages.typst.org/preview/thumbnails/modern-uit-thesis-0.1.2-small.webp}}

\section{modern-uit-thesis}\label{modern-uit-thesis}

{ 0.1.2 }

A Modern Thesis Template in Typst.

\href{/app?template=modern-uit-thesis&version=0.1.2}{Create project in
app}

\phantomsection\label{readme}
Port of the \href{https://github.com/egraff/uit-thesis}{uit-thesis}
-latex template to Typst.

\texttt{\ thesis.typ\ } contains a full usage example, see
\texttt{\ thesis.pdf\ } for a rendered pdf.

\subsection{Usage}\label{usage}

Using the Typst Universe package/template:

\begin{Shaded}
\begin{Highlighting}[]
\NormalTok{typst init @preview/modern{-}uit{-}thesis:0.1.2}
\end{Highlighting}
\end{Shaded}

\subsubsection{Fonts}\label{fonts}

This template uses a number of different fonts:

\begin{itemize}
\tightlist
\item
  Open Sans (Noto Sans)
\item
  JetBrains Mono (Fira Code)
\item
  Charis SIL (Charter)
\end{itemize}

The above parenthesized fonts are fallback typefaces available by
default in \href{https://typst.app/}{the web app} . If you’d like to
use the main fonts instead, simply upload the \texttt{\ .ttf\ } s to the
web app and it will detect and apply them automatically.

If you’re running typst locally, install the fonts in a directory of
your choosing and specify it with \texttt{\ -\/-font-path\ } .

\subsection{License}\label{license}

This project is licensed under the MIT License - see the
\href{https://github.com/typst/packages/raw/main/packages/preview/modern-uit-thesis/0.1.2/LICENSE}{LICENSE}
file for details.

\href{/app?template=modern-uit-thesis&version=0.1.2}{Create project in
app}

\subsubsection{How to use}\label{how-to-use}

Click the button above to create a new project using this template in
the Typst app.

You can also use the Typst CLI to start a new project on your computer
using this command:

\begin{verbatim}
typst init @preview/modern-uit-thesis:0.1.2
\end{verbatim}

\includesvg[width=0.16667in,height=0.16667in]{/assets/icons/16-copy.svg}

\subsubsection{About}\label{about}

\begin{description}
\tightlist
\item[Author s :]
\href{https://github.com/mrtz-j}{Moritz Jörg} \&
\href{https://github.com/otytlandsvik}{Ole Tytlandsvik}
\item[License:]
MIT
\item[Current version:]
0.1.2
\item[Last updated:]
October 29, 2024
\item[First released:]
September 18, 2024
\item[Minimum Typst version:]
0.12.0
\item[Archive size:]
543 kB
\href{https://packages.typst.org/preview/modern-uit-thesis-0.1.2.tar.gz}{\pandocbounded{\includesvg[keepaspectratio]{/assets/icons/16-download.svg}}}
\item[Repository:]
\href{https://github.com/mrtz-j/typst-thesis-template}{GitHub}
\item[Categor y :]
\begin{itemize}
\tightlist
\item[]
\item
  \pandocbounded{\includesvg[keepaspectratio]{/assets/icons/16-mortarboard.svg}}
  \href{https://typst.app/universe/search/?category=thesis}{Thesis}
\end{itemize}
\end{description}

\subsubsection{Where to report issues?}\label{where-to-report-issues}

This template is a project of Moritz Jörg and Ole Tytlandsvik . Report
issues on \href{https://github.com/mrtz-j/typst-thesis-template}{their
repository} . You can also try to ask for help with this template on the
\href{https://forum.typst.app}{Forum} .

Please report this template to the Typst team using the
\href{https://typst.app/contact}{contact form} if you believe it is a
safety hazard or infringes upon your rights.

\phantomsection\label{versions}
\subsubsection{Version history}\label{version-history}

\begin{longtable}[]{@{}ll@{}}
\toprule\noalign{}
Version & Release Date \\
\midrule\noalign{}
\endhead
\bottomrule\noalign{}
\endlastfoot
0.1.2 & October 29, 2024 \\
\href{https://typst.app/universe/package/modern-uit-thesis/0.1.1/}{0.1.1}
& September 19, 2024 \\
\href{https://typst.app/universe/package/modern-uit-thesis/0.1.0/}{0.1.0}
& September 18, 2024 \\
\end{longtable}

Typst GmbH did not create this template and cannot guarantee correct
functionality of this template or compatibility with any version of the
Typst compiler or app.


\title{typst.app/universe/package/drafting}

\phantomsection\label{banner}
\section{drafting}\label{drafting}

{ 0.2.1 }

Helpful functions for content positioning and margin comments/notes

{ } Featured Package

\phantomsection\label{readme}
\subsection{Setup}\label{setup}

\texttt{\ drafting\ } exists in the official typst package repository,
so the recommended approach is to import it from the
\texttt{\ preview\ } namespace:

\begin{Shaded}
\begin{Highlighting}[]
\NormalTok{\#import "@preview/drafting:0.2.1"}
\end{Highlighting}
\end{Shaded}

Margin notes cannot lay themselves out correctly until they know your
page size and margins. By default, they occupy nearly the entirety of
the left or right margin, but you can provide explicit left/right bounds
if desired:

\begin{Shaded}
\begin{Highlighting}[]
\NormalTok{// Example:}
\NormalTok{// Default margin in typst is 2.5cm, but we want to use 2cm}
\NormalTok{// On the left}
\NormalTok{\#set{-}page{-}properties(margin{-}left: 2cm)}
\end{Highlighting}
\end{Shaded}

\subsection{The basics}\label{the-basics}

\begin{Shaded}
\begin{Highlighting}[]
\NormalTok{\#lorem(20)}
\NormalTok{\#margin{-}note(side: left)[Hello, world!]}
\NormalTok{\#lorem(10)}
\NormalTok{\#margin{-}note[Hello from the other side]}
\NormalTok{\#margin{-}note[When notes are about to overlap, they\textquotesingle{}re automatically shifted]}
\NormalTok{\#margin{-}note(stroke: aqua + 3pt)[To avoid collision]}
\NormalTok{\#lorem(25)}
\NormalTok{\#margin{-}note(stroke: green, side: left)[You can provide two positional arguments if you want to highlight a phrase associated with your note.][The first is text which should be inline{-}noted, and the second is the standard margin note.]}

\NormalTok{\#let caution{-}rect = rect.with(inset: 1em, radius: 0.5em, fill: orange.lighten(80\%))}
\NormalTok{\#inline{-}note(rect: caution{-}rect)[}
\NormalTok{  Be aware that \textasciigrave{}typst\textasciigrave{} will complain when 4 notes overlap, and stop automatically avoiding collisions when 5 or more notes}
\NormalTok{  overlap. This is because the compiler stops attempting to reposition notes after a few attempts}
\NormalTok{  (initial layout + adjustment for each note).}

\NormalTok{  You can manually adjust the position of notes with \textasciigrave{}dy\textasciigrave{} to silence the warning.}
\NormalTok{]}
\end{Highlighting}
\end{Shaded}

\pandocbounded{\includegraphics[keepaspectratio]{https://www.github.com/ntjess/typst-drafting/raw/v0.2.1/assets/example-1.png}}

\subsection{Adjusting the default
style}\label{adjusting-the-default-style}

All function defaults are customizable through updating the module
state:

\begin{Shaded}
\begin{Highlighting}[]
\NormalTok{\#lorem(14) \#margin{-}note[Default style]}
\NormalTok{\#lorem(10)}
\NormalTok{\#set{-}margin{-}note{-}defaults(stroke: orange, side: left)}
\NormalTok{\#margin{-}note[Updated style]}
\NormalTok{\#lorem(10)}
\end{Highlighting}
\end{Shaded}

\pandocbounded{\includegraphics[keepaspectratio]{https://www.github.com/ntjess/typst-drafting/raw/v0.2.1/assets/example-2.png}}

Even deeper customization is possible by overriding the default
\texttt{\ rect\ } :

\begin{Shaded}
\begin{Highlighting}[]
\NormalTok{\#import "@preview/colorful{-}boxes:1.1.0": stickybox}

\NormalTok{\#let default{-}rect(stroke: none, fill: none, width: 0pt, content) = \{}
\NormalTok{  set text(0.9em)}
\NormalTok{  stickybox(rotation: 30deg, width: width/1.5, content)}
\NormalTok{\}}
\NormalTok{\#set{-}margin{-}note{-}defaults(rect: default{-}rect, stroke: none, side: right)}

\NormalTok{\#lorem(20)}
\NormalTok{\#margin{-}note(dy: {-}5em)[Why not use sticky notes in the margin?]}

\NormalTok{// Undo changes from this example}
\NormalTok{\#set{-}margin{-}note{-}defaults(rect: rect, stroke: red)}
\end{Highlighting}
\end{Shaded}

\pandocbounded{\includegraphics[keepaspectratio]{https://www.github.com/ntjess/typst-drafting/raw/v0.2.1/assets/example-3.png}}

\subsection{Multiple document
reviewers}\label{multiple-document-reviewers}

\begin{Shaded}
\begin{Highlighting}[]
\NormalTok{\#let reviewer{-}a = margin{-}note.with(stroke: blue)}
\NormalTok{\#let reviewer{-}b = margin{-}note.with(stroke: purple)}
\NormalTok{\#lorem(10)}
\NormalTok{\#reviewer{-}a[Comment from reviewer A]}
\NormalTok{\#lorem(5)}
\NormalTok{\#reviewer{-}b(side: left)[Reviewer B comment]}
\NormalTok{\#lorem(10)}
\end{Highlighting}
\end{Shaded}

\pandocbounded{\includegraphics[keepaspectratio]{https://www.github.com/ntjess/typst-drafting/raw/v0.2.1/assets/example-4.png}}

\subsection{Inline Notes}\label{inline-notes}

\begin{Shaded}
\begin{Highlighting}[]
\NormalTok{\#lorem(10)}
\NormalTok{\#inline{-}note[The default inline note will split the paragraph at its location]}
\NormalTok{\#lorem(10)}
\NormalTok{\#inline{-}note(par{-}break: false, stroke: (paint: orange, dash: "dashed"))[}
\NormalTok{  But you can specify \textasciigrave{}par{-}break: false\textasciigrave{} to prevent this}
\NormalTok{]}
\NormalTok{\#lorem(10)}
\end{Highlighting}
\end{Shaded}

\pandocbounded{\includegraphics[keepaspectratio]{https://www.github.com/ntjess/typst-drafting/raw/v0.2.1/assets/example-5.png}}

\subsection{Hiding notes for print
preview}\label{hiding-notes-for-print-preview}

\begin{Shaded}
\begin{Highlighting}[]
\NormalTok{\#set{-}margin{-}note{-}defaults(hidden: true)}

\NormalTok{\#lorem(20)}
\NormalTok{\#margin{-}note[This will respect the global "hidden" state]}
\NormalTok{\#margin{-}note(hidden: false, dy: {-}2.5em)[This note will never be hidden]}
\NormalTok{// Undo these changes}
\NormalTok{\#set{-}margin{-}note{-}defaults(hidden: false)}
\end{Highlighting}
\end{Shaded}

\pandocbounded{\includegraphics[keepaspectratio]{https://www.github.com/ntjess/typst-drafting/raw/v0.2.1/assets/example-6.png}}

\subsection{Precise placement: rule
grid}\label{precise-placement-rule-grid}

Need to measure space for fine-tuned positioning? You can use
\texttt{\ rule-grid\ } to cross-hatch the page with rule lines:

\begin{Shaded}
\begin{Highlighting}[]
\NormalTok{\#rule{-}grid(width: 10cm, height: 3cm, spacing: 20pt)}
\NormalTok{\#place(}
\NormalTok{  dx: 180pt,}
\NormalTok{  dy: 40pt,}
\NormalTok{  rect(fill: white, stroke: red, width: 1in, "This will originate at (180pt, 40pt)")}
\NormalTok{)}

\NormalTok{// Optionally specify divisions of the smallest dimension to automatically calculate}
\NormalTok{// spacing}
\NormalTok{\#rule{-}grid(dx: 10cm + 3em, width: 3cm, height: 1.2cm, divisions: 5, square: true,  stroke: green)}

\NormalTok{// The rule grid doesn\textquotesingle{}t take up space, so add it explicitly}
\NormalTok{\#v(3cm + 1em)}
\end{Highlighting}
\end{Shaded}

\pandocbounded{\includegraphics[keepaspectratio]{https://www.github.com/ntjess/typst-drafting/raw/v0.2.1/assets/example-7.png}}

\subsection{Absolute positioning}\label{absolute-positioning}

What about absolutely positioning something regardless of margin and
relative location? \texttt{\ absolute-place\ } is your friend. You can
put content anywhere:

\begin{Shaded}
\begin{Highlighting}[]
\NormalTok{\#context \{}
\NormalTok{  let (dx, dy) = (here().position().x, here().position().y)}
\NormalTok{  let content{-}str = (}
\NormalTok{    "This absolutely{-}placed box will originate at (" + repr(dx) + ", " + repr(dy) + ")"}
\NormalTok{    + " in page coordinates"}
\NormalTok{  )}
\NormalTok{  absolute{-}place(}
\NormalTok{    dx: dx, dy: dy,}
\NormalTok{    rect(}
\NormalTok{      fill: green.lighten(60\%),}
\NormalTok{      radius: 0.5em,}
\NormalTok{      width: 2.5in,}
\NormalTok{      height: 0.5in,}
\NormalTok{      [\#align(center + horizon, content{-}str)]}
\NormalTok{    )}
\NormalTok{  )}
\NormalTok{\}}
\NormalTok{\#v(0.5in)}
\end{Highlighting}
\end{Shaded}

\pandocbounded{\includegraphics[keepaspectratio]{https://www.github.com/ntjess/typst-drafting/raw/v0.2.1/assets/example-8.png}}

The “rule-grid� also supports absolute placement at the top-left of
the page by passing \texttt{\ relative:\ false\ } . This is helpful for
“rule“-ing the whole page.

\subsubsection{How to add}\label{how-to-add}

Copy this into your project and use the import as \texttt{\ drafting\ }

\begin{verbatim}
#import "@preview/drafting:0.2.1"
\end{verbatim}

\includesvg[width=0.16667in,height=0.16667in]{/assets/icons/16-copy.svg}

Check the docs for
\href{https://typst.app/docs/reference/scripting/\#packages}{more
information on how to import packages} .

\subsubsection{About}\label{about}

\begin{description}
\tightlist
\item[Author :]
Nathan Jessurun
\item[License:]
Unlicense
\item[Current version:]
0.2.1
\item[Last updated:]
November 25, 2024
\item[First released:]
September 3, 2023
\item[Minimum Typst version:]
0.12.0
\item[Archive size:]
7.98 kB
\href{https://packages.typst.org/preview/drafting-0.2.1.tar.gz}{\pandocbounded{\includesvg[keepaspectratio]{/assets/icons/16-download.svg}}}
\item[Repository:]
\href{https://github.com/ntjess/typst-drafting}{GitHub}
\item[Categor ies :]
\begin{itemize}
\tightlist
\item[]
\item
  \pandocbounded{\includesvg[keepaspectratio]{/assets/icons/16-layout.svg}}
  \href{https://typst.app/universe/search/?category=layout}{Layout}
\item
  \pandocbounded{\includesvg[keepaspectratio]{/assets/icons/16-hammer.svg}}
  \href{https://typst.app/universe/search/?category=utility}{Utility}
\end{itemize}
\end{description}

\subsubsection{Where to report issues?}\label{where-to-report-issues}

This package is a project of Nathan Jessurun . Report issues on
\href{https://github.com/ntjess/typst-drafting}{their repository} . You
can also try to ask for help with this package on the
\href{https://forum.typst.app}{Forum} .

Please report this package to the Typst team using the
\href{https://typst.app/contact}{contact form} if you believe it is a
safety hazard or infringes upon your rights.

\phantomsection\label{versions}
\subsubsection{Version history}\label{version-history}

\begin{longtable}[]{@{}ll@{}}
\toprule\noalign{}
Version & Release Date \\
\midrule\noalign{}
\endhead
\bottomrule\noalign{}
\endlastfoot
0.2.1 & November 25, 2024 \\
\href{https://typst.app/universe/package/drafting/0.2.0/}{0.2.0} & March
16, 2024 \\
\href{https://typst.app/universe/package/drafting/0.1.2/}{0.1.2} &
December 11, 2023 \\
\href{https://typst.app/universe/package/drafting/0.1.1/}{0.1.1} &
September 11, 2023 \\
\href{https://typst.app/universe/package/drafting/0.1.0/}{0.1.0} &
September 3, 2023 \\
\end{longtable}

Typst GmbH did not create this package and cannot guarantee correct
functionality of this package or compatibility with any version of the
Typst compiler or app.


\title{typst.app/universe/package/modern-russian-dissertation}

\phantomsection\label{banner}
\phantomsection\label{template-thumbnail}
\pandocbounded{\includegraphics[keepaspectratio]{https://packages.typst.org/preview/thumbnails/modern-russian-dissertation-0.0.1-small.webp}}

\section{modern-russian-dissertation}\label{modern-russian-dissertation}

{ 0.0.1 }

A russian phd thesis template

\href{/app?template=modern-russian-dissertation&version=0.0.1}{Create
project in app}

\phantomsection\label{readme}
Шаблон руÑ?Ñ?кой кандидатÑ?кой
диÑ?Ñ?ертации на Ñ?зыке разметки
\href{https://typst.app/}{Typst} - Ñ?овременной
альтернативы LaTeX.

\subsection{ИÑ?пользование}\label{uxf0uxf1uxf0uxf0uxbeuxf0uxf1ux153uxf0uxf0uxbeuxf0uxb2uxf0uxf0uxbduxf0uxf0uxb5}

Ð' веб-приложении нажмите “Start from templateâ€?
и на панели найдите
\texttt{\ modern-russian-dissertation\ } .

Ð'Ñ‹ также можете инициализировать
проект командой:

\begin{Shaded}
\begin{Highlighting}[]
\ExtensionTok{typst}\NormalTok{ init @preview/modern{-}russian{-}dissertation}
\end{Highlighting}
\end{Shaded}

Ð`удет Ñ?оздана новаÑ? директориÑ? Ñ?о
вÑ?еми файлами, необÑ\ldots одимыми длÑ?
начала работы.

\subsection{КонфигурациÑ?}\label{uxf0ux161uxf0uxbeuxf0uxbduxf1uxf0uxf0uxb3uxf1ux192uxf1uxf0uxf1uxf0uxf1}

СпиÑ?ок литературы формируетÑ?Ñ? из
файлов \texttt{\ common/external.bib\ } и
\texttt{\ common/author.bib\ } .

СпиÑ?ок Ñ?окращений и уÑ?ловныÑ\ldots{}
обозначений формируетÑ?Ñ? из данныÑ\ldots,
запиÑ?анныÑ\ldots{} в файле
\texttt{\ common/acronyms.typ\ } \texttt{\ common/symbols.typ\ } .
СпиÑ?ок определений формируетÑ?Ñ? из
данныÑ\ldots{} в файле \texttt{\ common/glossary.typ\ } .

\subsection{КомпилÑ?циÑ?}\label{uxf0ux161uxf0uxbeuxf0uxbcuxf0uxf0uxf0uxf1uxf1uxf0uxf1}

Ð''лÑ? компилÑ?ции проекта из CLI
иÑ?пользуйте:

\begin{Shaded}
\begin{Highlighting}[]
\ExtensionTok{typst}\NormalTok{ compile thesis.typ}
\end{Highlighting}
\end{Shaded}

Или еÑ?ли вы Ñ\ldots отите Ñ?ледить за
изменениÑ?ми:

\begin{Shaded}
\begin{Highlighting}[]
\ExtensionTok{typst}\NormalTok{ watch thesis.typ}
\end{Highlighting}
\end{Shaded}

\subsection{ОÑ?обенноÑ?ти}\label{uxf0ux17euxf1uxf0uxbeuxf0uxf0uxb5uxf0uxbduxf0uxbduxf0uxbeuxf1uxf1uxf0}

\begin{itemize}
\tightlist
\item
  Стандарт Ð``ОСТ Ð~ 7.0.11-2011.
\end{itemize}

\subsection{Ð`лагодарноÑ?ти}\label{uxf0uxf0uxf0uxf0uxb3uxf0uxbeuxf0uxf0uxf1uxf0uxbduxf0uxbeuxf1uxf1uxf0}

\begin{itemize}
\item
  Ð`лагодарноÑ?Ñ‚ÑŒ авторам шаблона
  диÑ?Ñ?ертации на
  \href{https://github.com/AndreyAkinshin/Russian-Phd-LaTeX-Dissertation-Template}{LaTeX}
\item
  \href{https://github.com/AndreyAkinshin/Russian-Phd-LaTeX-Dissertation-Template/wiki/Links\#\%D0\%BF\%D1\%80\%D0\%BE\%D1\%87\%D0\%B8\%D0\%B5-\%D1\%80\%D0\%B5\%D0\%BF\%D0\%BE\%D0\%B7\%D0\%B8\%D1\%82\%D0\%BE\%D1\%80\%D0\%B8\%D0\%B8-\%D1\%81-\%D0\%BF\%D0\%BE\%D0\%BB\%D0\%B5\%D0\%B7\%D0\%BD\%D1\%8B\%D0\%BC\%D0\%B8-\%D0\%BF\%D1\%80\%D0\%B8\%D0\%BC\%D0\%B5\%D1\%80\%D0\%B0\%D0\%BC\%D0\%B8}{Полезные
  Ñ?Ñ?ылки}
\end{itemize}

\href{/app?template=modern-russian-dissertation&version=0.0.1}{Create
project in app}

\subsubsection{How to use}\label{how-to-use}

Click the button above to create a new project using this template in
the Typst app.

You can also use the Typst CLI to start a new project on your computer
using this command:

\begin{verbatim}
typst init @preview/modern-russian-dissertation:0.0.1
\end{verbatim}

\includesvg[width=0.16667in,height=0.16667in]{/assets/icons/16-copy.svg}

\subsubsection{About}\label{about}

\begin{description}
\tightlist
\item[Author :]
Sergei Gorchakov
\item[License:]
MIT
\item[Current version:]
0.0.1
\item[Last updated:]
April 12, 2024
\item[First released:]
April 12, 2024
\item[Minimum Typst version:]
0.11.0
\item[Archive size:]
32.7 kB
\href{https://packages.typst.org/preview/modern-russian-dissertation-0.0.1.tar.gz}{\pandocbounded{\includesvg[keepaspectratio]{/assets/icons/16-download.svg}}}
\item[Repository:]
\href{https://github.com/SergeyGorchakov/russian-phd-thesis-template-typst}{GitHub}
\item[Categor y :]
\begin{itemize}
\tightlist
\item[]
\item
  \pandocbounded{\includesvg[keepaspectratio]{/assets/icons/16-mortarboard.svg}}
  \href{https://typst.app/universe/search/?category=thesis}{Thesis}
\end{itemize}
\end{description}

\subsubsection{Where to report issues?}\label{where-to-report-issues}

This template is a project of Sergei Gorchakov . Report issues on
\href{https://github.com/SergeyGorchakov/russian-phd-thesis-template-typst}{their
repository} . You can also try to ask for help with this template on the
\href{https://forum.typst.app}{Forum} .

Please report this template to the Typst team using the
\href{https://typst.app/contact}{contact form} if you believe it is a
safety hazard or infringes upon your rights.

\phantomsection\label{versions}
\subsubsection{Version history}\label{version-history}

\begin{longtable}[]{@{}ll@{}}
\toprule\noalign{}
Version & Release Date \\
\midrule\noalign{}
\endhead
\bottomrule\noalign{}
\endlastfoot
0.0.1 & April 12, 2024 \\
\end{longtable}

Typst GmbH did not create this template and cannot guarantee correct
functionality of this template or compatibility with any version of the
Typst compiler or app.


\title{typst.app/universe/package/modern-nju-thesis}

\phantomsection\label{banner}
\phantomsection\label{template-thumbnail}
\pandocbounded{\includegraphics[keepaspectratio]{https://packages.typst.org/preview/thumbnails/modern-nju-thesis-0.3.4-small.webp}}

\section{modern-nju-thesis}\label{modern-nju-thesis}

{ 0.3.4 }

å?---京大学学ä½?论æ--‡æ¨¡æ?¿ã€‚Modern Nanjing University Thesis.

\href{/app?template=modern-nju-thesis&version=0.3.4}{Create project in
app}

\phantomsection\label{readme}
å?---京大学毕业论æ--‡ï¼ˆè®¾è®¡ï¼‰çš„ Typst
模æ?¿ï¼Œèƒ½å¤Ÿç®€æ´?ã€?快速ã€?æŒ?ç»­ç''Ÿæˆ? PDF
æ~¼å¼?的毕业论æ--‡ã€‚
\href{https://typst.app/universe/package/modern-nju-thesis}{Typst
Universe}

Typst é?žå®˜æ--¹ä¸­æ--‡äº¤æµ?群:793548390

å?---京大学 Typst 交æµ?群:943622984

\pandocbounded{\includegraphics[keepaspectratio]{https://github.com/typst/packages/raw/main/packages/preview/modern-nju-thesis/0.3.4/imgs/editor.png}}

\subsection{劣势}\label{uxe5ux161uxe5ux161}

\begin{itemize}
\tightlist
\item
  Typst 是一é---¨æ--°ç''Ÿçš„æŽ'版æ~‡è®°è¯­è¨€ï¼Œè¿˜å?šä¸?到åƒ? Word
  æˆ-- LaTeX 一æ~·æˆ?熟稳定。
\item
  该模æ?¿å¹¶é?žå®˜æ--¹æ¨¡æ?¿ï¼Œè€Œæ˜¯æ°`é---´æ¨¡æ?¿ï¼Œ
  \textbf{存在�被认�的风险} 。
\end{itemize}

\subsection{优势}\label{uxe4uxbcuxe5ux161}

Typst
是å?¯ç''¨äºŽå‡ºç‰ˆçš„å?¯ç¼--程æ~‡è®°è¯­è¨€ï¼Œæ‹¥æœ‰å?˜é‡?ã€?函数与åŒ\ldots 管ç?†ç­‰çŽ°ä»£ç¼--程语言的特性,注é‡?于ç§`学写作
(science writing),定ä½?与 LaTeX 相似。å?¯ä»¥é˜\ldots 读æˆ`çš„
\href{https://zhuanlan.zhihu.com/p/669097092}{一篇知乎æ--‡ç«}
进一步了解 Typst 的优势。

\begin{itemize}
\tightlist
\item
  \textbf{语法简�} :上手难度跟 Markdown
  相å½``,æ--‡æœ¬æº?ç~?é˜\ldots 读性高,ä¸?会åƒ? LaTeX
  一æ~·å\ldots\ldots æ--¥ç?€å??æ--œæ?~与花括å?·ã€‚
\item
  \textbf{ç¼--è¯`速度快} :Typst 使ç''¨ Rust 语言ç¼--写,å?³
  typ(esetting+ru)st,目æ~‡è¿?行平å?°æ˜¯WASM,å?³æµ?览器本地离线è¿?行;也å?¯ä»¥ç¼--è¯`æˆ?å`½ä»¤è¡Œå·¥å\ldots·ï¼Œé‡‡ç''¨ä¸€ç§?
  \textbf{增é‡?ç¼--è¯`}
  ç®---法å'Œä¸€ç§?有约æ?Ÿçš„版é?¢ç¼``å­˜æ--¹æ¡ˆï¼Œ
  \textbf{æ--‡æ¡£é•¿åº¦åŸºæœ¬ä¸?会影å``?ç¼--è¯`速度,ä¸''ç¼--è¯`速度与常è§?
  Markdown 渲æŸ``引æ``Žæ¸²æŸ``速度相å½``} 。
\item
  \textbf{环境�建简�} :�需�� LaTeX
  一æ~·æŠ˜è\ldots¾å‡~个 G
  çš„å¼€å?{}`环境,原ç''Ÿæ''¯æŒ?中æ---¥éŸ©ç­‰é?žæ‹‰ä¸?语言,æ---~论是官æ--¹
  Web App 在线ç¼--è¾`,还是使ç''¨ VS Code
  安è£\ldots æ?'件本地开å?{}`,都是 \textbf{å?³å¼€å?³ç''¨} 。
\item
  \textbf{现代ç¼--程语言} :Typst
  是å?¯ç''¨äºŽå‡ºç‰ˆçš„å?¯ç¼--程æ~‡è®°è¯­è¨€ï¼Œæ‹¥æœ‰
  \textbf{å?˜é‡?ã€?函数ã€?åŒ\ldots 管ç?†ä¸Žé''™è¯¯æ£€æŸ¥}
  等现代ç¼--程语言的特性,å?Œæ---¶ä¹Ÿæ??供了 \textbf{é---­åŒ}
  等特性,便于进行 \textbf{函数å¼?ç¼--程}
  。以å?ŠåŒ\ldots 括了 \texttt{\ {[}æ~‡è®°æ¨¡å¼?{]}\ } ã€?
  \texttt{\ \{脚本模�\}\ } 与 \texttt{\ \$数学模�\$\ }
  等多ç§?模å¼?的作ç''¨åŸŸï¼Œå¹¶ä¸''它们å?¯ä»¥ä¸?é™?深度地ã€?交äº'地嵌å¥---。并ä¸''通过
  \textbf{åŒ\ldots 管ç?†} ,ä½~ä¸?å†?需è¦?åƒ? TexLive
  一æ~·åœ¨æœ¬åœ°å®‰è£\ldots 一大å~†å¹¶ä¸?å¿\ldots è¦?çš„å®?åŒ\ldots ,而是
  \textbf{按需自动从äº`端下载} 。
\end{itemize}

å?¯ä»¥å?‚考æˆ`å?‚与æ?­å»ºå'Œç¿»è¯`çš„
\href{https://typst-doc-cn.github.io/docs/}{Typst 中æ--‡æ--‡æ¡£ç½`ç«™}
è¿\ldots 速å\ldots¥é---¨ã€‚

\subsection{使ç''¨}\label{uxe4uxbduxe7}

快速�览效果: 查看
\href{https://github.com/nju-lug/modern-nju-thesis/releases/latest/download/thesis.pdf}{thesis.pdf}
,æ~·ä¾‹è®ºæ--‡æº?ç~?:查看
\href{https://github.com/nju-lug/modern-nju-thesis/blob/main/template/thesis.typ}{thesis.typ}

\textbf{ä½~å?ªéœ€è¦?ä¿®æ''¹ \texttt{\ thesis.typ\ }
æ--‡ä»¶å?³å?¯ï¼ŒåŸºæœ¬å?¯ä»¥æ»¡è¶³ä½~的所有需求。}

如果ä½~认为ä¸?能满足ä½~的需求,å?¯ä»¥å\ldots ˆæŸ¥é˜\ldots å?Žé?¢çš„
\href{https://github.com/typst/packages/raw/main/packages/preview/modern-nju-thesis/0.3.4/\#Q\%26A}{Q\&A}
部分。

模æ?¿å·²ç»?上ä¼~到了 Typst
Universe,使ç''¨èµ·æ?¥å??分简å?•ï¼Œç?†è®ºä¸Šå?ªéœ€è¦?通过

\begin{Shaded}
\begin{Highlighting}[]
\NormalTok{\#import "@preview/modern{-}nju{-}thesis:0.3.4": documentclass}
\end{Highlighting}
\end{Shaded}

导å\ldots¥å?³å?¯ã€‚

\subsubsection{在线ç¼--è¾`}\label{uxe5ux153uxe7uxbauxe7uxbcuxe8uxbe}

Typst æ??供了官æ--¹çš„ Web App,æ''¯æŒ?åƒ? Overleaf
一æ~·åœ¨çº¿ç¼--è¾`,这是一个
\href{https://typst.app/project/rgiwHIjdPOnXr9HJb8H0oa}{例�} 。

实é™\ldots 上,æˆ`们å?ªéœ€è¦?在
\href{https://typst.app/?template=modern-nju-thesis&version=0.3.4}{Web
App} 中的 \texttt{\ Start\ from\ template\ } 里选择
\texttt{\ modern-nju-thesis\ } ,å?³å?¯åœ¨çº¿åˆ›å»ºæ¨¡æ?¿å¹¶ä½¿ç''¨ã€‚

\pandocbounded{\includegraphics[keepaspectratio]{https://github.com/typst/packages/raw/main/packages/preview/modern-nju-thesis/0.3.4/imgs/template.png}}

\pandocbounded{\includegraphics[keepaspectratio]{https://github.com/typst/packages/raw/main/packages/preview/modern-nju-thesis/0.3.4/imgs/webapp.png}}

\textbf{但是 Web App 并没有安è£\ldots 本地 Windows æˆ-- MacOS
所拥有的å­---ä½``,所以å­---ä½``上å?¯èƒ½å­˜åœ¨å·®å¼‚,所以推è??本地ç¼--è¾`ï¼?}

\textbf{ä½~需è¦?手动上ä¼~ fonts
目录下的å­---ä½``æ--‡ä»¶åˆ°é¡¹ç›®ä¸­ï¼Œå?¦åˆ™ä¼šå¯¼è‡´å­---ä½``显示é''™è¯¯ï¼?}

PS: 虽然与 Overleaf
看起�相似,但是它们底层原�并�相�。Overleaf
是在���务器�行了一个 LaTeX
ç¼--è¯`器,本质上是计ç®---密集型的æœ?务;而 Typst
å?ªéœ€è¦?在æµ?览器端使ç''¨ WASM 技术执行,本质上是 IO
密集型的æœ?务,所以对æœ?务器压力很å°?(å?ªéœ€è¦?è´Ÿè´£æ--‡ä»¶çš„äº`存储与å??作å?Œæ­¥åŠŸèƒ½ï¼‰ã€‚

\subsubsection{VS Code
本地ç¼--è¾`(推è??)}\label{vs-code-uxe6ux153uxe5ux153uxe7uxbcuxe8uxbeuxefuxbcux2c6uxe6ux17euxe8uxefuxbc}

\begin{enumerate}
\tightlist
\item
  在 VS Code 中安è£
  \href{https://marketplace.visualstudio.com/items?itemName=myriad-dreamin.tinymist}{Tinymist
  Typst} å'Œ
  \href{https://marketplace.visualstudio.com/items?itemName=mgt19937.typst-preview}{Typst
  Preview}
  æ?'件。å‰?è€\ldots 负责语法高亮å'Œé''™è¯¯æ£€æŸ¥ç­‰åŠŸèƒ½ï¼Œå?Žè€\ldots 负责预览。

  \begin{itemize}
  \tightlist
  \item
    也推è??下载
    \href{https://marketplace.visualstudio.com/items?itemName=CalebFiggers.typst-companion}{Typst
    Companion} æ?'件,å\ldots¶æ??供了例如 \texttt{\ Ctrl\ +\ B\ }
    进行åŠ~ç²---等便æ?·çš„å¿«æ?·é''®ã€‚
  \item
    ä½~还å?¯ä»¥ä¸‹è½½æˆ`å¼€å?{}`çš„
    \href{https://marketplace.visualstudio.com/items?itemName=OrangeX4.vscode-typst-sync}{Typst
    Sync} å'Œ
    \href{https://marketplace.visualstudio.com/items?itemName=OrangeX4.vscode-typst-sympy-calculator}{Typst
    Sympy Calculator}
    æ?'件,å‰?è€\ldots æ??供了本地åŒ\ldots çš„äº`å?Œæ­¥åŠŸèƒ½ï¼Œå?Žè€\ldots æ??供了基于
    Typst 语法的ç§`学计ç®---器功能。
  \end{itemize}
\item
  按下 \texttt{\ Ctrl\ +\ Shift\ +\ P\ }
  æ‰``å¼€å`½ä»¤ç•Œé?¢ï¼Œè¾``å\ldots¥
  \texttt{\ Typst:\ Show\ available\ Typst\ templates\ (gallery)\ for\ picking\ up\ a\ template\ }
  æ‰``å¼€ Tinymist æ??供的 Template Gallery,然å?Žä»Žé‡Œé?¢æ‰¾åˆ°
  \texttt{\ modern-nju-thesis\ } ,点击 \texttt{\ �\ }
  按é'®è¿›è¡Œæ''¶è---?,以å?Šç‚¹å‡» \texttt{\ +\ }
  å?·ï¼Œå°±å?¯ä»¥åˆ›å»ºå¯¹åº''的论æ--‡æ¨¡æ?¿äº†ã€‚
\item
  最å?Žç''¨ VS Code æ‰``å¼€ç''Ÿæˆ?的目录,æ‰``å¼€
  \texttt{\ thesis.typ\ } æ--‡ä»¶ï¼Œå¹¶æŒ‰ä¸‹ \texttt{\ Ctrl\ +\ K\ V\ }
  进行实æ---¶ç¼--è¾`å'Œé¢„览。
\end{enumerate}

\pandocbounded{\includegraphics[keepaspectratio]{https://github.com/typst/packages/raw/main/packages/preview/modern-nju-thesis/0.3.4/imgs/gallery.png}}

\subsubsection{特性 /
路线图}\label{uxe7uxb9uxe6-uxe8uxe7uxbauxe5uxbe}

\begin{itemize}
\tightlist
\item
  \textbf{说明æ--‡æ¡£}

  \begin{itemize}
  \tightlist
  \item
    {[} {]} ç¼--写更详细的说明æ--‡æ¡£ï¼Œå?Žç»­è€ƒè™`使ç''¨
    \href{https://github.com/typst/packages/tree/main/packages/preview/tidy/0.1.0}{tidy}
    ç¼--写,ä½~现在å?¯ä»¥å\ldots ˆå?‚考
    \href{https://mirror-hk.koddos.net/CTAN/macros/unicodetex/latex/njuthesis/njuthesis.pdf}{NJUThesis}
    çš„æ--‡æ¡£ï¼Œå?‚数大ä½``ä¿?æŒ?一致,æˆ--è€\ldots 直接查é˜\ldots 对åº''æº?ç~?函数的å?‚æ•°
  \end{itemize}
\item
  \textbf{类型检查}

  \begin{itemize}
  \tightlist
  \item
    {[} {]}
    åº''该对所有函数å\ldots¥å?‚进行类型检查,å?Šæ---¶æŠ¥é''™
  \end{itemize}
\item
  \textbf{å\ldots¨å±€é\ldots?ç½®}

  \begin{itemize}
  \tightlist
  \item
    {[}x{]} 类似 LaTeX 中的 \texttt{\ documentclass\ }
    çš„å\ldots¨å±€ä¿¡æ?¯é\ldots?ç½®
  \item
    {[}x{]} \textbf{盲审模�}
    ,将个人信æ?¯æ›¿æ?¢æˆ?å°?é»`æ?¡ï¼Œå¹¶ä¸''éš?è---?致谢页é?¢ï¼Œè®ºæ--‡æ??交阶段使ç''¨
  \item
    {[}x{]} \textbf{��模�}
    ,会åŠ~å\ldots¥ç©ºç™½é¡µï¼Œä¾¿äºŽæ‰``å?°
  \item
    {[}x{]} \textbf{自定义å­---ä½``é\ldots?ç½®}
    ,å?¯ä»¥é\ldots?置「宋ä½``ã€?ã€?「é»`ä½``ã€?与「楷ä½``ã€?ç­‰å­---ä½``对åº''çš„å\ldots·ä½``å­---ä½``
  \item
    {[}x{]} \textbf{æ•°å­¦å­---ä½``é\ldots?ç½®}
    :模æ?¿ä¸?æ??ä¾›é\ldots?置,ç''¨æˆ·å?¯ä»¥è‡ªå·±ä½¿ç''¨
    \texttt{\ \#show\ math.equation:\ set\ text(font:\ "Fira\ Math")\ }
    æ›´æ''¹
  \end{itemize}
\item
  \textbf{模�}

  \begin{itemize}
  \tightlist
  \item
    {[}x{]} 本ç§`ç''Ÿæ¨¡æ?¿

    \begin{itemize}
    \tightlist
    \item
      {[}x{]} å­---ä½``测试页
    \item
      {[}x{]} å°?é?¢
    \item
      {[}x{]} 声明页
    \item
      {[}x{]} 中æ--‡æ`˜è¦?
    \item
      {[}x{]} 英æ--‡æ`˜è¦?
    \item
      {[}x{]} 目录页
    \item
      {[}x{]} æ?'图目录
    \item
      {[}x{]} 表æ~¼ç›®å½•
    \item
      {[}x{]} 符�表
    \item
      {[}x{]} 致谢
    \end{itemize}
  \item
    {[}x{]} ç~''究ç''Ÿæ¨¡æ?¿

    \begin{itemize}
    \tightlist
    \item
      {[}x{]} å°?é?¢
    \item
      {[}x{]} 声明页
    \item
      {[}x{]} æ`˜è¦?
    \item
      {[}x{]} 页眉
    \item
      {[} {]} 国家图书馆��
    \item
      {[} {]} 出版授�书
    \end{itemize}
  \item
    {[} {]} �士�模�
  \end{itemize}
\item
  \textbf{ç¼--å?·}

  \begin{itemize}
  \tightlist
  \item
    {[}x{]} å‰?言使ç''¨ç½---马数å­---ç¼--å?·
  \item
    {[}x{]} 附录使ç''¨ç½---马数å­---ç¼--å?·
  \item
    {[}x{]} 表æ~¼ä½¿ç''¨ \texttt{\ 1.1\ } æ~¼å¼?进行ç¼--å?·
  \item
    {[}x{]} æ•°å­¦å\ldots¬å¼?使ç''¨ \texttt{\ (1.1)\ }
    æ~¼å¼?进行ç¼--å?·
  \end{itemize}
\item
  \textbf{环境}

  \begin{itemize}
  \tightlist
  \item
    {[} {]}
    定ç?†çŽ¯å¢ƒï¼ˆè¿™ä¸ªä¹Ÿå?¯ä»¥è‡ªå·±ä½¿ç''¨ç¬¬ä¸‰æ--¹åŒ\ldots é\ldots?置)
  \end{itemize}
\item
  \textbf{å\ldots¶ä»--æ--‡ä»¶}

  \begin{itemize}
  \tightlist
  \item
    {[}x{]} 本ç§`ç''Ÿå¼€é¢˜æŠ¥å`Š
  \item
    {[}x{]} ç~''究ç''Ÿå¼€é¢˜æŠ¥å`Š
  \end{itemize}
\end{itemize}

\subsection{å\ldots¶ä»--æ--‡ä»¶}\label{uxe5uxe4uxe6uxe4}

还实现了本ç§`ç''Ÿå'Œç~''究ç''Ÿçš„开题报å`Šï¼Œå?ªéœ€è¦?预览å'Œç¼--è¾`
\texttt{\ others\ } 目录下的æ--‡ä»¶å?³å?¯ã€‚

\pandocbounded{\includegraphics[keepaspectratio]{https://github.com/typst/packages/raw/main/packages/preview/modern-nju-thesis/0.3.4/imgs/proposal.png}}

\subsection{Q\&A}\label{qa}

\subsubsection{æˆ`ä¸?会
LaTeX,å?¯ä»¥ç''¨è¿™ä¸ªæ¨¡æ?¿å†™è®ºæ--‡å?---?}\label{uxe6ux2c6uxe4uxe4uxbcux161-latexuxefuxbcux153uxe5uxe4uxe7uxe8uxe4uxaauxe6uxe6uxe5uxe8uxbauxe6uxe5uxefuxbcuxff}

�以。

如果ä½~ä¸?å\ldots³æ³¨æ¨¡æ?¿çš„å\ldots·ä½``实现原ç?†ï¼Œä½~å?¯ä»¥ç''¨
Markdown Like
的语法进行ç¼--写,å?ªéœ€è¦?按ç\ldots§æ¨¡æ?¿çš„ç»``æž„ç¼--写å?³å?¯ã€‚

\subsubsection{æˆ`ä¸?会ç¼--程,å?¯ä»¥ç''¨è¿™ä¸ªæ¨¡æ?¿å†™è®ºæ--‡å?---?}\label{uxe6ux2c6uxe4uxe4uxbcux161uxe7uxbcuxe7uxefuxbcux153uxe5uxe4uxe7uxe8uxe4uxaauxe6uxe6uxe5uxe8uxbauxe6uxe5uxefuxbcuxff}

å?Œæ~·å?¯ä»¥ã€‚

如果ä»\ldots ä»\ldots 是å½``æˆ?是å\ldots¥é---¨ä¸€æ¬¾ç±»ä¼¼äºŽ
Markdown 的语言,相信使ç''¨è¯¥æ¨¡æ?¿çš„ä½``验会æ¯''使ç''¨ Word
ç¼--写更好。

\subsubsection{为什么æˆ`çš„å­---ä½``没有显示出æ?¥ï¼Œè€Œæ˜¯ä¸€ä¸ªä¸ªã€Œè±†è\ldots?å?---ã€??}\label{uxe4uxbauxe4uxe4uxb9ux2c6uxe6ux2c6uxe7ux161uxe5uxe4uxbduxe6uxb2uxe6ux153uxe6uxbeuxe7uxbauxe5uxbauxe6uxefuxbcux153uxe8ux153uxe6uxe4uxe4uxaauxe4uxaauxe3ux153uxe8uxe8uxe5uxe3uxefuxbcuxff}

这是å›~为本地没有对åº''çš„å­---ä½``,
\textbf{è¿™ç§?æƒ\ldots 况ç»?常å?{}`ç''Ÿåœ¨ MacOS
的「楷ä½``ã€?显示上} 。

ä½~åº''该安è£\ldots 本目录下的 \texttt{\ fonts\ }
里的所有å­---ä½``,里é?¢åŒ\ldots å?«äº†å?¯ä»¥å\ldots?费商ç''¨çš„「æ--¹æ­£æ¥·ä½``ã€?å'Œã€Œæ--¹æ­£ä»¿å®‹ã€?,然å?Žå†?é‡?æ--°æ¸²æŸ``测试å?³å?¯ã€‚

ä½~å?¯ä»¥ä½¿ç''¨ \texttt{\ \#fonts-display-page()\ }
显示一个å­---ä½``渲æŸ``测试页é?¢ï¼ŒæŸ¥çœ‹å¯¹åº''çš„å­---ä½``是å?¦æ˜¾ç¤ºæˆ?功。

如果还是ä¸?能æˆ?功,ä½~å?¯ä»¥æŒ‰ç\ldots§æ¨¡æ?¿é‡Œçš„说明自行é\ldots?ç½®å­---ä½``,例如

\begin{Shaded}
\begin{Highlighting}[]
\NormalTok{\#let (...) = documentclass(}
\NormalTok{  fonts: (楷体: ("Times New Roman", "FZKai{-}Z03S")),}
\NormalTok{)}
\end{Highlighting}
\end{Shaded}

å\ldots ˆæ˜¯å¡«å†™è‹±æ--‡å­---ä½``,然å?Žå†?填写ä½~需è¦?的「楷ä½``ã€?中æ--‡å­---ä½``。

\textbf{å­---ä½``å??称å?¯ä»¥é€šè¿‡ \texttt{\ typst\ fonts\ }
å`½ä»¤æŸ¥è¯¢ã€‚}

如果找ä¸?到ä½~所需è¦?çš„å­---ä½``,å?¯èƒ½æ˜¯å›~为
\textbf{该å­---ä½``å?˜ä½``(Variants)数é‡?过å°`} ,导致 Typst
æ---~法识别到该中æ--‡å­---ä½``。

\subsubsection{å­¦ä¹~ Typst
需è¦?多ä¹\ldots ?}\label{uxe5uxe4uxb9-typst-uxe9ux153uxe8uxe5ux161uxe4uxb9uxefuxbcuxff}

一般而言,ä»\ldots ä»\ldots 进行简å?•çš„ç¼--写,ä¸?å\ldots³æ³¨å¸ƒå±€çš„è¯?,ä½~å?¯ä»¥æ‰``开模æ?¿å°±å¼€å§‹å†™äº†ã€‚

如果ä½~想进一步学ä¹~ Typst
的语法,例如如何æŽ'篇布局,如何设置页脚页眉等,一般å?ªéœ€è¦?å‡~个å°?æ---¶å°±èƒ½å­¦ä¼šã€‚

如果ä½~还想学ä¹~ Typst 的「
\href{https://typst-doc-cn.github.io/docs/reference/meta/}{å\ldots ƒä¿¡æ?¯}
ã€?部分,进而能够ç¼--写自己的模æ?¿ï¼Œä¸€èˆ¬è€Œè¨€éœ€è¦?å‡~天的æ---¶é---´é˜\ldots 读æ--‡æ¡£ï¼Œä»¥å?Šä»--人ç¼--写的模æ?¿ä»£ç~?。

如果ä½~有 Python æˆ-- JavaScript
等脚本语言的ç¼--写ç»?验,了解过函数å¼?ç¼--程ã€?å®?ã€?æ~·å¼?ã€?组件åŒ--å¼€å?{}`等概念,å\ldots¥é---¨é€Ÿåº¦ä¼šå¿«å¾ˆå¤šã€‚

\subsubsection{æˆ`有ç¼--写 LaTeX
çš„ç»?验,如何快速å\ldots¥é---¨ï¼Ÿ}\label{uxe6ux2c6uxe6ux153uxe7uxbcuxe5-latex-uxe7ux161uxe7uxe9uxaaux153uxefuxbcux153uxe5uxe4uxbduxe5uxe9uxffuxe5uxe9uxefuxbcuxff}

�以�考
\href{https://typst-doc-cn.github.io/docs/guides/guide-for-latex-users/}{é?¢å?{}`
LaTeX ç''¨æˆ·çš„ Typst å\ldots¥é---¨æŒ‡å?---} 。

\subsubsection{目� Typst
有å``ªäº›ç¬¬ä¸‰æ--¹åŒ\ldots å'Œæ¨¡æ?¿ï¼Ÿ}\label{uxe7uxe5-typst-uxe6ux153uxe5uxaauxe4uxbauxe7uxe4uxe6uxb9uxe5ux153uxe5ux153uxe6uxe6uxefuxbcuxff}

�以查看 \href{https://typst.app/universe}{Typst Universe} 。

\subsubsection{为什么�有一个 thesis.typ
æ--‡ä»¶ï¼Œæ²¡æœ‰æŒ‰ç«~节分多个æ--‡ä»¶ï¼Ÿ}\label{uxe4uxbauxe4uxe4uxb9ux2c6uxe5uxaauxe6ux153uxe4uxe4uxaa-thesis.typ-uxe6uxe4uxefuxbcux153uxe6uxb2uxe6ux153uxe6ux153uxe7-uxe8ux161uxe5ux2c6uxe5ux161uxe4uxaauxe6uxe4uxefuxbcuxff}

å›~为 Typst \textbf{语法足够简æ´?} ã€?
\textbf{ç¼--è¯`速度足够快} ã€?并ä¸''
\textbf{拥有å\ldots‰æ~‡ç‚¹å‡»å¤„å?Œå?{}`é``¾æŽ¥åŠŸèƒ½} 。

语法简æ´?的好处是,å?³ä½¿æŠŠæ‰€æœ‰å†\ldots 容都写在å?Œä¸€ä¸ªæ--‡ä»¶ï¼Œä½~也å?¯ä»¥å¾ˆç®€å?•åœ°åˆ†è¾¨å‡ºå?„个部分的å†\ldots 容。

ç¼--è¯`速度足够快的好处是,ä½~ä¸?å†?需è¦?åƒ? LaTeX
一æ~·ï¼Œå°†å†\ldots 容分散在å‡~个æ--‡ä»¶ï¼Œå¹¶é€šè¿‡æ³¨é‡Šçš„æ--¹å¼?æ??高ç¼--è¯`速度。

å\ldots‰æ~‡ç‚¹å‡»å¤„å?Œå?{}`é``¾æŽ¥åŠŸèƒ½ï¼Œä½¿å¾---ä½~å?¯ä»¥ç›´æŽ¥æ‹--动预览çª---å?£åˆ°ä½~想è¦?çš„ä½?置,然å?Žç''¨é¼~æ~‡ç‚¹å‡»å?³å?¯åˆ°è¾¾å¯¹åº''æº?ç~?所在ä½?置。

还有一个好处是,å?•ä¸ªæº?æ--‡ä»¶ä¾¿äºŽå?Œæ­¥å'Œåˆ†äº«ã€‚

å?³ä½¿ä½~还是想è¦?分æˆ?å‡~个ç«~节,也是å?¯ä»¥çš„,Typst
æ''¯æŒ?ä½~使ç''¨ \texttt{\ \#import\ } å'Œ \texttt{\ \#include\ }
语法将å\ldots¶ä»--æ--‡ä»¶çš„å†\ldots 容导å\ldots¥æˆ--ç½®å\ldots¥ã€‚ä½~å?¯ä»¥æ--°å»ºæ--‡ä»¶å¤¹
\texttt{\ chapters\ }
,然å?Žå°†å?„个ç«~节的æº?æ--‡ä»¶æ''¾è¿›åŽ»ï¼Œç„¶å?Žé€šè¿‡
\texttt{\ \#include\ } ç½®å\ldots¥ \texttt{\ thesis.typ\ } 里。

\subsubsection{æˆ`如何更æ''¹é¡µé?¢ä¸Šçš„æ~·å¼??å\ldots·ä½``的语法是怎么æ~·çš„?}\label{uxe6ux2c6uxe5uxe4uxbduxe6uxe6uxb9uxe9uxb5uxe9uxe4ux161uxe7ux161uxe6-uxe5uxbcuxefuxbcuxffuxe5uxe4uxbduxe7ux161uxe8uxe6uxb3uxe6uxe6ux17euxe4uxb9ux2c6uxe6-uxe7ux161uxefuxbcuxff}

ç?†è®ºä¸Šä½~并ä¸?需è¦?æ›´æ''¹ \texttt{\ nju-thesis\ }
目录下的任何æ--‡ä»¶ï¼Œæ---~论是æ~·å¼?还是å\ldots¶ä»--çš„é\ldots?置,ä½~都å?¯ä»¥åœ¨
\texttt{\ thesis.typ\ }
æ--‡ä»¶å†\ldots ä¿®æ''¹å‡½æ•°å?‚数实现更æ''¹ã€‚å\ldots·ä½``çš„æ›´æ''¹æ--¹å¼?å?¯ä»¥é˜\ldots 读
\texttt{\ nju-thesis\ } 目录下的æ--‡ä»¶çš„函数å?‚数。

例如,想è¦?æ›´æ''¹é¡µé?¢è¾¹è·?为 \texttt{\ 50pt\ } ,å?ªéœ€è¦?å°†

\begin{Shaded}
\begin{Highlighting}[]
\NormalTok{\#show: doc}
\end{Highlighting}
\end{Shaded}

æ''¹ä¸º

\begin{Shaded}
\begin{Highlighting}[]
\NormalTok{\#show: doc.with(margin: (x: 50pt))}
\end{Highlighting}
\end{Shaded}

��。

å?Žç»­æˆ`也会ç¼--写一个更详细的æ--‡æ¡£ï¼Œå?¯èƒ½ä¼šè€ƒè™`使ç''¨
\href{https://github.com/typst/packages/tree/main/packages/preview/tidy/0.1.0}{tidy}
æ?¥ç¼--写。

如果ä½~é˜\ldots 读了那些函数的å?‚数,ä»?然ä¸?知é?{}``如何修æ''¹å¾---到ä½~需è¦?çš„æ~·å¼?,欢迎æ??出
Issue,å?ªè¦?æ??è¿°æ¸\ldots 楚é---®é¢˜å?³å?¯ã€‚

æˆ--è€\ldots 也欢迎åŠ~群讨论:943622984

\subsubsection{该模æ?¿å'Œå\ldots¶ä»--现存 Typst
中æ--‡è®ºæ--‡æ¨¡æ?¿çš„区别?}\label{uxe8uxe6uxe6uxe5ux153uxe5uxe4uxe7ux17euxe5-typst-uxe4uxe6uxe8uxbauxe6uxe6uxe6uxe7ux161uxe5ux153uxbauxe5ux2c6uxefuxbcuxff}

å\ldots¶ä»--现存的 Typst 中æ--‡è®ºæ--‡æ¨¡æ?¿å¤§å¤šéƒ½æ˜¯åœ¨ 2023 å¹´ 7
月份之å‰?(Typst Verison 0.6 之å‰?)开å?{}`的,å½``æ---¶ Typst
还ä¸?ä¸?够æˆ?熟,ç''šè‡³è¿ž \textbf{åŒ\ldots 管ç?†}
功能都还没有,å›~æ­¤å½``æ---¶çš„ Typst
中æ--‡è®ºæ--‡æ¨¡æ?¿çš„å¼€å?{}`è€\ldots 基本都是自己从头写了一é??需è¦?的功能/函数,å›~æ­¤é€~æˆ?了
\textbf{代ç~?耦å?ˆåº¦é«˜} ã€? \textbf{æ„?大利é?¢æ?¡å¼?代ç~?} ã€?
\textbf{é‡?å¤?é€~è½®å­?} 与 \textbf{难以自定义æ~·å¼?} ç­‰é---®é¢˜ã€‚

该模�是在 2023 年 10 ~ 11 月份(Typst Verison 0.9
æ---¶ï¼‰å¼€å?{}`的,此æ---¶ Typst 语法基本稳定,并ä¸''æ??供了
\textbf{åŒ\ldots 管ç?†}
功能,å›~此能够å‡?å°`很多ä¸?å¿\ldots è¦?的代ç~?。

并ä¸''æˆ`对模æ?¿çš„æ--‡ä»¶æž¶æž„进行了解耦,主è¦?分为了
\texttt{\ utils\ } ã€? \texttt{\ pages\ } å'Œ \texttt{\ layouts\ }
三个目录,这三个目录å?¯ä»¥çœ‹å?Žæ--‡çš„å¼€å?{}`è€\ldots 指å?---,并ä¸''使ç''¨
\textbf{é---­åŒ}
特性实现了类似ä¸?å?¯å?˜å\ldots¨å±€å?˜é‡?çš„å\ldots¨å±€é\ldots?置能力,å?³æ¨¡æ?¿ä¸­çš„
\texttt{\ documentclass\ } 函数类。

\subsubsection{æˆ`ä¸?是å?---京大学本ç§`ç''Ÿï¼Œå¦‚何è¿?移该模æ?¿ï¼Ÿ}\label{uxe6ux2c6uxe4uxe6uxe5uxe4uxbauxe5uxe5uxe6ux153uxe7uxe7uxffuxefuxbcux153uxe5uxe4uxbduxe8uxe7uxe8uxe6uxe6uxefuxbcuxff}

æˆ`在开å?{}`的过程中已ç»?对模æ?¿çš„å?„个模æ?¿è¿›è¡Œäº†è§£è€¦ï¼Œç?†è®ºä¸Šä½~å?ªéœ€è¦?在
\texttt{\ pages\ }
目录中åŠ~å\ldots¥ä½~需è¦?的页é?¢ï¼Œç„¶å?Žæ›´æ''¹å°`许ã€?æˆ--è€\ldots ä¸?需è¦?æ›´æ''¹å\ldots¶ä»--目录的代ç~?。

å\ldots·ä½``目录è?Œè´£åˆ'分å?¯ä»¥çœ‹ä¸‹é?¢çš„å¼€å?{}`è€\ldots 指å?---。

\subsection{å¼€å?{}`è€\ldots 指å?---}\label{uxe5uxbcuxe5uxe8uxe6ux153uxe5}

\subsubsection{template 目录}\label{template-uxe7uxe5uxbd}

\begin{itemize}
\tightlist
\item
  \texttt{\ thesis.typ\ } æ--‡ä»¶:
  ä½~的论æ--‡æº?æ--‡ä»¶ï¼Œå?¯ä»¥éš?æ„?æ›´æ''¹è¿™ä¸ªæ--‡ä»¶çš„å??å­---,ç''šè‡³ä½~å?¯ä»¥å°†è¿™ä¸ªæ--‡ä»¶åœ¨å?Œçº§ç›®å½•ä¸‹å¤?制多份,维æŒ?多个版本。
\item
  \texttt{\ ref.bib\ } æ--‡ä»¶: ç''¨äºŽæ''¾ç½®å?‚考æ--‡çŒ®ã€‚
\item
  \texttt{\ images\ } 目录: ç''¨äºŽæ''¾ç½®å›¾ç‰‡ã€‚
\end{itemize}

\subsubsection{å†\ldots 部目录}\label{uxe5uxe9ux192uxe7uxe5uxbd}

\begin{itemize}
\tightlist
\item
  \texttt{\ utils\ } 目录:
  åŒ\ldots å?«äº†æ¨¡æ?¿ä½¿ç''¨åˆ°çš„å?„ç§?自定义è¾\ldots 助函数,存æ''¾æ²¡æœ‰å¤--部ä¾?èµ--,ä¸''
  \textbf{ä¸?会渲æŸ``出页é?¢çš„函数} 。
\item
  \texttt{\ pages\ } 目录: åŒ\ldots å?«äº†æ¨¡æ?¿ç''¨åˆ°çš„å?„个
  \textbf{独立页é?¢} ,例如å°?é?¢é¡µã€?声明页ã€?æ`˜è¦?等,å?³
  \textbf{会渲æŸ``出ä¸?å½±å``?å\ldots¶ä»--页é?¢çš„独立页é?¢çš„函数}
  。
\item
  \texttt{\ layouts\ } 目录:
  布局目录,存æ''¾ç?€ç''¨äºŽæŽ'篇布局的ã€?åº''ç''¨äºŽ
  \texttt{\ show\ } 指令的� \textbf{横跨多个页�的函数}
  ,例如为了给页脚进行ç½---马数å­---ç¼--ç~?çš„å‰?言
  \texttt{\ preface\ } 函数。

  \begin{itemize}
  \tightlist
  \item
    主è¦?分æˆ?了 \texttt{\ doc\ } æ--‡ç¨¿ã€? \texttt{\ preface\ }
    å‰?言ã€? \texttt{\ mainmatter\ } æ­£æ--‡ä¸Ž \texttt{\ appendix\ }
    附录/�记。
  \end{itemize}
\item
  \texttt{\ lib.typ\ } :

  \begin{itemize}
  \tightlist
  \item
    \textbf{�责一} :
    作为一个统一的对å¤--接å?£ï¼Œæš´éœ²å‡ºå†\ldots 部的 utils
    函数。
  \item
    \textbf{è?Œè´£äºŒ} : 使ç''¨ \textbf{函数é---­åŒ} 特性,通过
    \texttt{\ documentclass\ }
    函数类进行å\ldots¨å±€ä¿¡æ?¯é\ldots?置,然å?Žæš´éœ²å‡ºæ‹¥æœ‰äº†å\ldots¨å±€é\ldots?置的ã€?å\ldots·ä½``çš„
    \texttt{\ layouts\ } å'Œ \texttt{\ pages\ } å†\ldots 部函数。
  \end{itemize}
\end{itemize}

\subsection{�与贡献}\label{uxe5uxe4ux17euxe8uxe7ux153}

\begin{itemize}
\tightlist
\item
  在 Issues
  中æ??出ä½~的想法,如果是æ--°ç‰¹æ€§ï¼Œå?¯ä»¥åŠ~å\ldots¥è·¯çº¿å›¾ï¼?
\item
  实现路线图中ä»?未实现的部分,然å?Žæ¬¢è¿Žæ??出ä½~çš„
  PR。
\item
  å?Œæ~·æ¬¢è¿Ž \textbf{将这个模æ?¿è¿?移至ä½~çš„å­¦æ~¡è®ºæ--‡æ¨¡æ?¿}
  ,大家一起æ?­å»ºæ›´å¥½çš„ Typst 社区å'Œç''Ÿæ€?å?§ã€‚
\end{itemize}

\subsection{致谢}\label{uxe8uxe8}

\begin{itemize}
\tightlist
\item
  æ„Ÿè°¢ \href{https://github.com/atxy-blip}{@atxy-blip} å¼€å?{}`çš„
  \href{https://github.com/nju-lug/NJUThesis}{NJUThesis} LaTeX
  模æ?¿ï¼Œæ--‡æ¡£å??分详细,本模æ?¿å¤§ä½``ç»``构都是å?‚考
  NJUThesis çš„æ--‡æ¡£å¼€å?{}`的。
\item
  æ„Ÿè°¢ \href{https://github.com/csimide}{@csimide}
  帮忙补å\ldots\ldots çš„
  \href{https://github.com/nju-lug/modern-nju-thesis/issues/3}{bilingual-bibliography}
  。
\item
  æ„Ÿè°¢
  \href{https://github.com/werifu/HUST-typst-template}{HUST-typst-template}
  与
  \href{https://github.com/howardlau1999/sysu-thesis-typst}{sysu-thesis-typst}
  ç­‰ Typst 中æ--‡è®ºæ--‡æ¨¡æ?¿ã€‚
\end{itemize}

\subsection{License}\label{license}

This project is licensed under the MIT License.

\href{/app?template=modern-nju-thesis&version=0.3.4}{Create project in
app}

\subsubsection{How to use}\label{how-to-use}

Click the button above to create a new project using this template in
the Typst app.

You can also use the Typst CLI to start a new project on your computer
using this command:

\begin{verbatim}
typst init @preview/modern-nju-thesis:0.3.4
\end{verbatim}

\includesvg[width=0.16667in,height=0.16667in]{/assets/icons/16-copy.svg}

\subsubsection{About}\label{about}

\begin{description}
\tightlist
\item[Author :]
OrangeX4
\item[License:]
MIT
\item[Current version:]
0.3.4
\item[Last updated:]
May 13, 2024
\item[First released:]
April 8, 2024
\item[Archive size:]
124 kB
\href{https://packages.typst.org/preview/modern-nju-thesis-0.3.4.tar.gz}{\pandocbounded{\includesvg[keepaspectratio]{/assets/icons/16-download.svg}}}
\item[Repository:]
\href{https://github.com/nju-lug/modern-nju-thesis}{GitHub}
\item[Categor y :]
\begin{itemize}
\tightlist
\item[]
\item
  \pandocbounded{\includesvg[keepaspectratio]{/assets/icons/16-mortarboard.svg}}
  \href{https://typst.app/universe/search/?category=thesis}{Thesis}
\end{itemize}
\end{description}

\subsubsection{Where to report issues?}\label{where-to-report-issues}

This template is a project of OrangeX4 . Report issues on
\href{https://github.com/nju-lug/modern-nju-thesis}{their repository} .
You can also try to ask for help with this template on the
\href{https://forum.typst.app}{Forum} .

Please report this template to the Typst team using the
\href{https://typst.app/contact}{contact form} if you believe it is a
safety hazard or infringes upon your rights.

\phantomsection\label{versions}
\subsubsection{Version history}\label{version-history}

\begin{longtable}[]{@{}ll@{}}
\toprule\noalign{}
Version & Release Date \\
\midrule\noalign{}
\endhead
\bottomrule\noalign{}
\endlastfoot
0.3.4 & May 13, 2024 \\
\href{https://typst.app/universe/package/modern-nju-thesis/0.3.3/}{0.3.3}
& April 15, 2024 \\
\href{https://typst.app/universe/package/modern-nju-thesis/0.3.2/}{0.3.2}
& April 9, 2024 \\
\href{https://typst.app/universe/package/modern-nju-thesis/0.3.1/}{0.3.1}
& April 8, 2024 \\
\href{https://typst.app/universe/package/modern-nju-thesis/0.3.0/}{0.3.0}
& April 8, 2024 \\
\end{longtable}

Typst GmbH did not create this template and cannot guarantee correct
functionality of this template or compatibility with any version of the
Typst compiler or app.


\title{typst.app/universe/package/imprecv}

\phantomsection\label{banner}
\phantomsection\label{template-thumbnail}
\pandocbounded{\includegraphics[keepaspectratio]{https://packages.typst.org/preview/thumbnails/imprecv-1.0.1-small.webp}}

\section{imprecv}\label{imprecv}

{ 1.0.1 }

A no-frills curriculum vitae (CV) template using Typst and YAML to
version control CV data.

\href{/app?template=imprecv&version=1.0.1}{Create project in app}

\phantomsection\label{readme}
\href{https://github.com/jskherman/imprecv/stargazers}{\pandocbounded{\includesvg[keepaspectratio]{https://img.shields.io/badge/Star\%20Repo-\%E2\%AD\%90-1081c2.svg}}}
\href{https://github.com/typst/packages/raw/main/packages/preview/imprecv/1.0.1/LICENSE}{\pandocbounded{\includegraphics[keepaspectratio]{https://img.shields.io/badge/license-Apache\%202-brightgreen}}}
\href{https://github.com/jskherman/imprecv/releases}{\pandocbounded{\includegraphics[keepaspectratio]{https://img.shields.io/github/v/release/jskherman/imprecv}}}

\texttt{\ imprecv\ } is a no-frills curriculum vitae (CV) template for
\href{https://github.com/typst/typst}{Typst} that uses a YAML file for
data input in order to version control CV data easily.

This is based on the
\href{https://web.archive.org/https://old.reddit.com/r/jobs/comments/7y8k6p/im_an_exrecruiter_for_some_of_the_top_companies/}{popular
template on Reddit} by
\href{https://web.archive.org/https://old.reddit.com/user/SheetsGiggles}{u/SheetsGiggles}
and the recommendations of the
\href{https://web.archive.org/https://old.reddit.com/r/EngineeringResumes/comments/m2cc65/new_and_improved_wiki}{r/EngineeringResumes
wiki} .

\subsection{Demo}\label{demo}

See
\href{https://github.com/jskherman/imprecv/releases/latest/download/example.pdf}{\textbf{example
CV}} and \href{https://go.jskherman.com/cv}{@jskherman’s CV} :

\pandocbounded{\includegraphics[keepaspectratio]{https://raw.githubusercontent.com/jskherman/imprecv/main/assets/thumbnail.1.png}}
\pandocbounded{\includegraphics[keepaspectratio]{https://raw.githubusercontent.com/jskherman/imprecv/main/assets/thumbnail.2.png}}

\subsection{Usage}\label{usage}

This \texttt{\ imprecv\ } is intended to be used by importing the
template’s
\href{https://github.com/typst/packages/raw/main/packages/preview/imprecv/1.0.1/cv.typ}{package
entrypoint} from a “content� file (see
\href{https://github.com/typst/packages/raw/main/packages/preview/imprecv/1.0.1/template/template.typ}{\texttt{\ template.typ\ }}
as an example). In this content file, call the functions which apply
document styles, show CV components, and load CV data from a YAML file
(see
\href{https://github.com/typst/packages/raw/main/packages/preview/imprecv/1.0.1/template/template.yml}{\texttt{\ template.yml\ }}
as an example). Inside the content file you can modify several style
variables and even override existing function implementations to your
own needs and preferences.

\subsubsection{\texorpdfstring{With the
\href{https://github.com/typst/typst}{Typst
CLI}}{With the Typst CLI}}\label{with-the-typst-cli}

The recommended usage with the Typst CLI is by running the command
\texttt{\ typst\ init\ @preview/imprecv:1.0.1\ } in your project
directory. This will create a new Typst project with the
\texttt{\ imprecv\ } template and the necessary files to get started.
You can then run \texttt{\ typst\ compile\ template.typ\ } to compile
your file to PDF.

Take a look at the
\href{https://github.com/jskherman/cv.typ-example-repo}{example setup}
for ideas on how to get started. It includes a GitHub action workflow to
compile the Typst files to PDF and upload it to Cloudflare R2.

\subsubsection{\texorpdfstring{With
\href{https://typst.app/}{typst.app}}{With typst.app}}\label{with-typst.app}

From the Dashboard, select “Start from template�, search and choose
the \texttt{\ imprecv\ } template. From there, decide on a name for your
project and click “Create�. You can now edit the template files and
preview the result on the right.

You can also click the \texttt{\ Create\ project\ in\ app\ } button in
\href{https://typst.app/universe/package/imprecv}{Typst Universe} to
create a new project with the \texttt{\ imprecv\ } template.

\subsection{Contributing}\label{contributing}

\href{https://github.com/jskherman}{I’m} only doing programming as a
hobby so it might take me a while to respond to issues and pull
requests. If you would like to contribute to this project, I would be
happy to review your pull requests when I can. Thank you for your
understanding.

\href{/app?template=imprecv&version=1.0.1}{Create project in app}

\subsubsection{How to use}\label{how-to-use}

Click the button above to create a new project using this template in
the Typst app.

You can also use the Typst CLI to start a new project on your computer
using this command:

\begin{verbatim}
typst init @preview/imprecv:1.0.1
\end{verbatim}

\includesvg[width=0.16667in,height=0.16667in]{/assets/icons/16-copy.svg}

\subsubsection{About}\label{about}

\begin{description}
\tightlist
\item[Author :]
\href{https://jskherman.com}{Je Sian Keith Herman}
\item[License:]
Apache-2.0
\item[Current version:]
1.0.1
\item[Last updated:]
June 17, 2024
\item[First released:]
June 3, 2024
\item[Minimum Typst version:]
0.11.0
\item[Archive size:]
51.6 kB
\href{https://packages.typst.org/preview/imprecv-1.0.1.tar.gz}{\pandocbounded{\includesvg[keepaspectratio]{/assets/icons/16-download.svg}}}
\item[Repository:]
\href{https://github.com/jskherman/imprecv}{GitHub}
\item[Categor y :]
\begin{itemize}
\tightlist
\item[]
\item
  \pandocbounded{\includesvg[keepaspectratio]{/assets/icons/16-user.svg}}
  \href{https://typst.app/universe/search/?category=cv}{CV}
\end{itemize}
\end{description}

\subsubsection{Where to report issues?}\label{where-to-report-issues}

This template is a project of Je Sian Keith Herman . Report issues on
\href{https://github.com/jskherman/imprecv}{their repository} . You can
also try to ask for help with this template on the
\href{https://forum.typst.app}{Forum} .

Please report this template to the Typst team using the
\href{https://typst.app/contact}{contact form} if you believe it is a
safety hazard or infringes upon your rights.

\phantomsection\label{versions}
\subsubsection{Version history}\label{version-history}

\begin{longtable}[]{@{}ll@{}}
\toprule\noalign{}
Version & Release Date \\
\midrule\noalign{}
\endhead
\bottomrule\noalign{}
\endlastfoot
1.0.1 & June 17, 2024 \\
\href{https://typst.app/universe/package/imprecv/1.0.0/}{1.0.0} & June
3, 2024 \\
\end{longtable}

Typst GmbH did not create this template and cannot guarantee correct
functionality of this template or compatibility with any version of the
Typst compiler or app.


\title{typst.app/universe/package/modern-acad-cv}

\phantomsection\label{banner}
\phantomsection\label{template-thumbnail}
\pandocbounded{\includegraphics[keepaspectratio]{https://packages.typst.org/preview/thumbnails/modern-acad-cv-0.1.1-small.webp}}

\section{modern-acad-cv}\label{modern-acad-cv}

{ 0.1.1 }

A CV template for academics based on moderncv LaTeX package.

\href{/app?template=modern-acad-cv&version=0.1.1}{Create project in app}

\phantomsection\label{readme}
This template for an academic CV serves the peculiarities of academic
CVs. If you are not an academic, this template is not useful. Most of
the times in academics, applicants need to show everything they have
done. This makes it a bit cumbersome doing it by single entries. In
addition, academics might apply to institutions around the globe, making
it necessary to send translated CVs or at least translations of some
parts (i.e., title of papers in different languages).

This template serves these special needs in introducting automated
sections based on indicated \texttt{\ yaml\ } -files. Furthermore, it
has a simplified multilingual support by setting different headers,
title etc. for different languages (by the user in the \texttt{\ yaml\ }
-fields). With this template, it might be more handy to keep your CV
easier on track, especially when you need in different languages, since
managing a \texttt{\ yaml\ } -file is easier than checking typesetting
files against each other.

This template is influenced by LaTeX’s
\href{https://github.com/moderncv/moderncv}{moderncv} and its typst
translation
\href{https://github.com/DeveloperPaul123/modern-cv}{moderner-cv} .

\subsection{Fonts}\label{fonts}

In this template, the use of FontAwesome icons via the
\href{https://typst.app/universe/package/fontawesome}{fontawesome typst
package} is possible, as well as the icons from Academicons
\href{https://typst.app/universe/package/use-academicons}{use-academicons
typst package} . To use these icons properly, you need to install each
fonts on your system. You can download
\href{https://fontawesome.com/download}{fontawesome here} and
\href{https://jpswalsh.github.io/academicons/}{academicons here} . Both
typst packages will be load by the template itself.

Furthermore, I included my favorite font
\href{https://fonts.google.com/specimen/Fira+Sans}{Fira Sans} . You can
download it here
\href{https://fonts.google.com/specimen/Fira+Sans}{here} , or just
change the font argument in \texttt{\ modern-acad-cv()\ } .

\subsection{Usage}\label{usage}

The main function to load the construct of the academic CV is
\texttt{\ modern-acad-cv()\ } . After importing the template, you can
call it right away. If you don’t have
\href{https://fonts.google.com/specimen/Fira+Sans}{Fira Sans} installed,
choose a different font. Examples are given below.

\begin{Shaded}
\begin{Highlighting}[]
\NormalTok{\#import "@preview/modern{-}acad{-}cv:0.1.1": *}

\NormalTok{\#show: modern{-}acad{-}cv.with(}
\NormalTok{  metadata,}
\NormalTok{  multilingual,}
\NormalTok{  lang: "en",}
\NormalTok{  font: ("Fira Sans", "Andale Mono", "Roboto"),}
\NormalTok{  show{-}date: true,}
\NormalTok{  body}
\NormalTok{)    }

\NormalTok{// ...}
\end{Highlighting}
\end{Shaded}

In the remainder, I show basic settings and how to use the automated
functions with the corresponding \texttt{\ yaml\ } -file.

\subsubsection{\texorpdfstring{Setting up the main file and the access
to the \texttt{\ yaml\ }
-files}{Setting up the main file and the access to the  yaml  -files}}\label{setting-up-the-main-file-and-the-access-to-the-yaml--files}

A first step in your document is to invoke the template. Second, since
this template works with \texttt{\ yaml\ } -files in the background you
need to specify paths to each \texttt{\ yaml\ } -file you want to use
throughout the document.

The template comes along with the \texttt{\ metadata.yaml\ } . In the
beginning of this yaml-file you set colors. Feel free to change it to
your preferred color scheme.

\begin{Shaded}
\begin{Highlighting}[]
\FunctionTok{colors}\KeywordTok{:}
\AttributeTok{  }\FunctionTok{main\_color}\KeywordTok{:}\AttributeTok{ }\StringTok{"\#579D90"}
\AttributeTok{  }\FunctionTok{lightgray\_color}\KeywordTok{:}\AttributeTok{ }\StringTok{"\#d5d5d5"}
\AttributeTok{  }\FunctionTok{gray\_color}\KeywordTok{:}\AttributeTok{ }\StringTok{"\#737373"}
\AttributeTok{  ...}
\end{Highlighting}
\end{Shaded}

At the beginning of your document, you just set then set the
metadata-object:

\begin{Shaded}
\begin{Highlighting}[]
\NormalTok{\#import "@preview/modern{-}acad{-}cv:0.1.0": *}

\NormalTok{\#let metadata = yaml("metadata.yaml")}
\end{Highlighting}
\end{Shaded}

Initially, the \texttt{\ metadata.yaml\ } is located on the same level
as the \texttt{\ example.typ\ } . All other \texttt{\ yaml\ } -files are
saved in the folder \texttt{\ dbs\ } . Since \texttt{\ typ\ } -documents
search for paths from the root of the document in that the function is
called, you have to give the databases for the entry along the
\texttt{\ metadata.yaml\ } within each function call.

\subsubsection{socials}\label{socials}

Contact details are important. In this CV template, you have the
possibility to use fontawesome icons and academicons. To use socials,
you just need to specify in \texttt{\ metadata.yaml\ } , the wanted
entries.

As you can see below, you set a category, i.e. email or lattes and then
you have to define four arguments: \texttt{\ username\ } ,
\texttt{\ prefix\ } , \texttt{\ icon\ } , and \texttt{\ set\ } . The
\texttt{\ username\ } will be used for constructing the link and will be
shown next to the logo. The \texttt{\ prefix\ } is needed to build the
valid link. The \texttt{\ icon\ } is the name of the icon in the
respective set, which is chosen in \texttt{\ set\ } .

\begin{Shaded}
\begin{Highlighting}[]
\FunctionTok{personal}\KeywordTok{:}
\AttributeTok{  }\FunctionTok{name}\KeywordTok{:}\AttributeTok{ }\KeywordTok{[}\StringTok{"Mustermensch, Momo"}\KeywordTok{]}
\AttributeTok{  }\FunctionTok{socials}\KeywordTok{:}
\AttributeTok{    }\FunctionTok{email}\KeywordTok{:}
\AttributeTok{      }\FunctionTok{username}\KeywordTok{:}\AttributeTok{ momo@mustermensch.com}
\AttributeTok{      }\FunctionTok{prefix}\KeywordTok{:}\AttributeTok{ }\StringTok{"mailto:"}
\AttributeTok{      }\FunctionTok{icon}\KeywordTok{:}\AttributeTok{ paper{-}plane}
\AttributeTok{      }\FunctionTok{set}\KeywordTok{:}\AttributeTok{ fa}
\AttributeTok{    }\FunctionTok{homepage}\KeywordTok{:}
\AttributeTok{      }\FunctionTok{username}\KeywordTok{:}\AttributeTok{ momo.github.io}
\AttributeTok{      }\FunctionTok{prefix}\KeywordTok{:}\AttributeTok{ https://}
\AttributeTok{      }\FunctionTok{icon}\KeywordTok{:}\AttributeTok{ globe}
\AttributeTok{      }\FunctionTok{set}\KeywordTok{:}\AttributeTok{ fa}
\AttributeTok{    }\FunctionTok{orcid}\KeywordTok{:}
\AttributeTok{      }\FunctionTok{username}\KeywordTok{:}\AttributeTok{ 0000{-}0000{-}0000{-}0000}
\AttributeTok{      }\FunctionTok{prefix}\KeywordTok{:}\AttributeTok{ https://orcid.org}
\AttributeTok{      }\FunctionTok{icon}\KeywordTok{:}\AttributeTok{ orcid}
\AttributeTok{      }\FunctionTok{set}\KeywordTok{:}\AttributeTok{ ai}
\AttributeTok{    }\FunctionTok{lattes}\KeywordTok{:}
\AttributeTok{      }\FunctionTok{username}\KeywordTok{:}\AttributeTok{ }\StringTok{"1234567891234567"}
\AttributeTok{      }\FunctionTok{prefix}\KeywordTok{:}\AttributeTok{ http://lattes.cnpq.br/}
\AttributeTok{      }\FunctionTok{icon}\KeywordTok{:}\AttributeTok{ lattes}
\AttributeTok{      }\FunctionTok{set}\KeywordTok{:}\AttributeTok{ ai}
\AttributeTok{    ...}
\end{Highlighting}
\end{Shaded}

\subsubsection{Language setting \&
headers}\label{language-setting-headers}

In order to support changing headers, you need to specify the language
and the different content for each header in each language in the
\texttt{\ i18n.yaml\ } in the folder \texttt{\ dbs\ } .

The structure of the yaml is simple:

\begin{Shaded}
\begin{Highlighting}[]
\FunctionTok{lang}\KeywordTok{:}
\AttributeTok{  }\FunctionTok{de}\KeywordTok{:}
\AttributeTok{    }\FunctionTok{subtitle}\KeywordTok{:}\AttributeTok{ Short CV}
\AttributeTok{    }\FunctionTok{education}\KeywordTok{:}\AttributeTok{ Hochschulbildung}
\AttributeTok{    }\FunctionTok{work}\KeywordTok{:}\AttributeTok{ Akademische Berufserfahrung (Auswahl)}
\AttributeTok{    }\FunctionTok{grants}\KeywordTok{:}\AttributeTok{ Fördermittel, Stipendien \& Preise}
\AttributeTok{    ...}
\AttributeTok{  }\FunctionTok{en}\KeywordTok{:}
\AttributeTok{    }\FunctionTok{subtitle}\KeywordTok{:}\AttributeTok{ Short CV}
\AttributeTok{    }\FunctionTok{education}\KeywordTok{:}\AttributeTok{ Higher education}
\AttributeTok{    }\FunctionTok{work}\KeywordTok{:}\AttributeTok{ Academic work experience (selection)}
\AttributeTok{    }\FunctionTok{grants}\KeywordTok{:}\AttributeTok{ Scholarships \& awards}
\AttributeTok{    ...}
\AttributeTok{  }\FunctionTok{pt}\KeywordTok{:}
\AttributeTok{    }\FunctionTok{subtitle}\KeywordTok{:}\AttributeTok{ Currículo}
\AttributeTok{    }\FunctionTok{education}\KeywordTok{:}\AttributeTok{ Formação acadêmica}
\AttributeTok{    }\FunctionTok{work}\KeywordTok{:}\AttributeTok{ Atuação profissional (seleção)}
\AttributeTok{    }\FunctionTok{grants}\KeywordTok{:}\AttributeTok{ Bolsas de estudo e prémios}
\AttributeTok{    ...}
\end{Highlighting}
\end{Shaded}

For each language, you want to use later, you have to define all the
entries. Reminder, don’t change the entry names, since the functions
won’t find it under different names without changing the functions.

First you have to set up a variable that inherits the ISO-language code,
save the database into an object (here \texttt{\ multilingual\ } ) and
then give the object \texttt{\ multilingual\ } and \texttt{\ language\ }
to the function \texttt{\ create-headers\ } :

\begin{Shaded}
\begin{Highlighting}[]
\NormalTok{// set the language of the document}
\NormalTok{\#let language = "pt"      }

\NormalTok{// loading multilingual database}
\NormalTok{\#let multilingual = yaml("dbs/i18n.yaml")}

\NormalTok{// defining variables}
\NormalTok{\#let headerLabs = create{-}headers(multilingual, lang: language)}
\end{Highlighting}
\end{Shaded}

You create an object \texttt{\ headerLabs\ } that uses the function
\texttt{\ create-headers()\ } which will define the headers as you
provided in the \texttt{\ yaml\ } . Then by switching the language
object, all headers (if used accordingly to the naming in the
\texttt{\ yaml\ } ) will change directly.

Throughout the document you then reference the created
\texttt{\ headerLabs\ } object. If you change language, and values are
provided, these automatically change.

\begin{Shaded}
\begin{Highlighting}[]
\NormalTok{= \#headerLabs.at("work")}

\NormalTok{...}

\NormalTok{= \#headerLabs.at("education")}

\NormalTok{...}

\NormalTok{= \#headerLabs.at("grants")}
\end{Highlighting}
\end{Shaded}

\subsubsection{Automated functions}\label{automated-functions}

All of the following functions share common arguments: \texttt{\ what\ }
, \texttt{\ multilingual\ } , and \texttt{\ lang\ } . In
\texttt{\ what\ } , you always declare the database you want to use with
the function.

For example, to get work entries, you choose \texttt{\ work\ } , which
you defined beforehand as input from \texttt{\ work.yaml\ } . In the
\texttt{\ multilingual\ } argument, you just pass the
\texttt{\ multilingual\ } object. In \texttt{\ lang\ } you pass your
\texttt{\ language\ } object.

\begin{Shaded}
\begin{Highlighting}[]
\NormalTok{\#let multilingual = yaml("dbs/multilingual.yaml")}
\NormalTok{\#let work = yaml("dbs/work.yaml")}
\NormalTok{\#let language = "pt"}

\NormalTok{// Function call with objects}
\NormalTok{\#cv{-}auto{-}stc(work, multilingual, lang: language)}
\end{Highlighting}
\end{Shaded}

\subsubsection{Sorting publications and referencing your own name or
correpsonding}\label{sorting-publications-and-referencing-your-own-name-or-correpsonding}

Since \texttt{\ typst\ } so far does not support multiple bibliographies
or subsetting these, this function let you choose specific entries via
the \texttt{\ entries\ } argument or group of entries by the
\texttt{\ tag\ } argument. Furthermore, you can indicate a string in
\texttt{\ me\ } that can be highlighted in every output entry (i.e.,
your formatted name). So far, this function leads to another function
that create APA-style format, if you want to use any other citation
style, you need to download the template on
\href{https://github.com/bpkleer/modern-acad-cv}{github} , introduce
your own styling and then add it in the \texttt{\ cv-refs()\ } function.

\begin{Shaded}
\begin{Highlighting}[]
\NormalTok{\#let multilingual = yaml("dbs/multilingual.yaml")}
\NormalTok{\#let refs = yaml("dbs/refs/yaml")}

\NormalTok{// function call of group of peer{-}reviewed with tag \textasciigrave{}peer\textasciigrave{}}
\NormalTok{\#cv{-}refs(refs, multilingual, tag: "peer", me: [Mustermensch, M.], lang: language)}
\end{Highlighting}
\end{Shaded}

You see in the example pictures that I used this function to built five
different subheaders, i.e. for peer reviewed articles (
\texttt{\ tag:\ "peer"\ } ) and chapters in edited books (
\texttt{\ tag:\ "edited"\ } ). You can define the tags how you want,
however, they need to put them into
\texttt{\ tag:\ \textless{}str\textgreater{}\ } .

Sometimes, it is not only necessary to highlight your own name, you
might also want to indicate yourself as corresponding author. This can
be done through the \texttt{\ refs.yaml\ } which adhere to
\href{https://github.com/typst/hayagriva}{Hayagriva} . By adding an
argument \texttt{\ corresponding\ } in the yaml and setting the value to
\texttt{\ true\ } , a small \texttt{\ C\ } will appear next to your
name.

\begin{Shaded}
\begin{Highlighting}[]
\FunctionTok{Mustermensch2023}\KeywordTok{:}
\AttributeTok{  }\FunctionTok{type}\KeywordTok{:}\AttributeTok{ }\StringTok{"article"}
\AttributeTok{  }\FunctionTok{date}\KeywordTok{:}\AttributeTok{ }\DecValTok{2023}
\AttributeTok{  }\FunctionTok{page{-}range}\KeywordTok{:}\AttributeTok{ 55{-}78}
\AttributeTok{  }\FunctionTok{title}\KeywordTok{:}\AttributeTok{ }\StringTok{"Populism and Social Media: A Comparative Study of Political Mobilization"}
\AttributeTok{  }\FunctionTok{tags}\KeywordTok{:}\AttributeTok{ }\StringTok{"peer"}
\AttributeTok{  }\FunctionTok{author}\KeywordTok{:}\AttributeTok{ }\KeywordTok{[}\AttributeTok{ }\StringTok{"Mustermensch, Momo"}\KeywordTok{,}\AttributeTok{ }\StringTok{"Rivera, Casey"}\AttributeTok{ }\KeywordTok{]}
\AttributeTok{  }\FunctionTok{corresponding}\KeywordTok{:}\AttributeTok{ }\CharTok{true}
\AttributeTok{  }\FunctionTok{parent}\KeywordTok{:}
\AttributeTok{    }\FunctionTok{title}\KeywordTok{:}\AttributeTok{ }\StringTok{"Journal of Political Communication"}
\AttributeTok{    }\FunctionTok{volume}\KeywordTok{:}\AttributeTok{ }\DecValTok{41}
\AttributeTok{    }\FunctionTok{issue}\KeywordTok{:}\AttributeTok{ }\DecValTok{3}
\AttributeTok{  }\FunctionTok{serial{-}number}\KeywordTok{:}
\AttributeTok{    }\FunctionTok{doi}\KeywordTok{:}\AttributeTok{ }\StringTok{"10.1016/j.jpolcom.2023.102865"}
\end{Highlighting}
\end{Shaded}

For applications abroad, it might be worth to translate at least title
of the publications so that other persons easily can see what the paper
is about. In every \texttt{\ title\ } argument, you can therefore
provide a dictionary with the language codes and the titles. Keep the
original title in \texttt{\ main\ } and the translations with the
corresponding language shortcut (i.e., \texttt{\ "en"\ } or
\texttt{\ "pt"\ } ). The function prints the main and translated title,
depending on the provided translation in the \texttt{\ refs.yaml\ } . Be
aware, here you find not \texttt{\ de\ } in the dictionary, instead you
find \texttt{\ main\ } . The original title needs to be wrapped in
\texttt{\ main\ } .

\begin{Shaded}
\begin{Highlighting}[]
\FunctionTok{Mustermensch2023}\KeywordTok{:}
\AttributeTok{  }\FunctionTok{type}\KeywordTok{:}\AttributeTok{ }\StringTok{"article"}
\AttributeTok{  }\FunctionTok{date}\KeywordTok{:}\AttributeTok{ }\DecValTok{2023}
\AttributeTok{  }\FunctionTok{page{-}range}\KeywordTok{:}\AttributeTok{ 55{-}78}
\AttributeTok{  }\FunctionTok{title}\KeywordTok{:}\AttributeTok{ }
\AttributeTok{    }\FunctionTok{main}\KeywordTok{:}\AttributeTok{ }\StringTok{"Populismus und soziale Medien: Eine vergleichende Studie zur politischen Mobilisierung"}
\AttributeTok{    }\FunctionTok{en}\KeywordTok{:}\AttributeTok{ }\StringTok{"Populism and Social Media: A Comparative Study of Political Mobilization"}
\AttributeTok{    }\FunctionTok{pt}\KeywordTok{:}\AttributeTok{ }\StringTok{"Populismo e redes sociais: Um Estudo Comparativo de Mobilização Política"}
\AttributeTok{  }\FunctionTok{tags}\KeywordTok{:}\AttributeTok{ }\StringTok{"peer"}
\AttributeTok{  }\FunctionTok{author}\KeywordTok{:}\AttributeTok{ }\KeywordTok{[}\AttributeTok{ }\StringTok{"Mustermensch, Momo"}\KeywordTok{,}\AttributeTok{ }\StringTok{"Rivera, Casey"}\AttributeTok{ }\KeywordTok{]}
\AttributeTok{  }\FunctionTok{corresponding}\KeywordTok{:}\AttributeTok{ }\CharTok{true}
\AttributeTok{  }\FunctionTok{parent}\KeywordTok{:}
\AttributeTok{    }\FunctionTok{title}\KeywordTok{:}\AttributeTok{ }\StringTok{"Journal of Political Communication"}
\AttributeTok{    }\FunctionTok{volume}\KeywordTok{:}\AttributeTok{ }\DecValTok{41}
\AttributeTok{    }\FunctionTok{issue}\KeywordTok{:}\AttributeTok{ }\DecValTok{3}
\AttributeTok{  }\FunctionTok{serial{-}number}\KeywordTok{:}
\AttributeTok{    }\FunctionTok{doi}\KeywordTok{:}\AttributeTok{ }\StringTok{"10.1016/j.jpolcom.2023.102865"}
\end{Highlighting}
\end{Shaded}

\subsubsection{cv-auto-skills()}\label{cv-auto-skills}

Instead of just enumerating your skills or your knowledge of specific
software, you can build a skill-matrix with this function. In this
skill-matrix, you can have sections, i.e. \emph{Computer Languages} ,
\emph{Programs} and \emph{Languages} . These sections are the highest
level in the corresponding \texttt{\ skills.yaml\ } :

\begin{Shaded}
\begin{Highlighting}[]
\FunctionTok{computer}\KeywordTok{:}
\AttributeTok{  ...}
\FunctionTok{programs}\KeywordTok{:}
\AttributeTok{  ...}
\FunctionTok{languages}\KeywordTok{:}
\AttributeTok{  ...}
\end{Highlighting}
\end{Shaded}

You can then define in each categories specific skills, i.e. German and
Portuguese in \texttt{\ languages\ } :

\begin{Shaded}
\begin{Highlighting}[]
\FunctionTok{computer}\KeywordTok{:}
\AttributeTok{  ...}
\FunctionTok{programs}\KeywordTok{:}
\AttributeTok{  ...}
\FunctionTok{languages}\KeywordTok{:}
\AttributeTok{  }\FunctionTok{german}\KeywordTok{:}
\AttributeTok{    ...}
\AttributeTok{  }\FunctionTok{portugues}\KeywordTok{:}
\AttributeTok{   ...}
\end{Highlighting}
\end{Shaded}

For each entry, you have to define \texttt{\ name\ } ,
\texttt{\ level\ } and \texttt{\ description\ } .

\begin{Shaded}
\begin{Highlighting}[]
\FunctionTok{languages}\KeywordTok{:}\AttributeTok{ }
\AttributeTok{  ...}
\AttributeTok{  }\FunctionTok{pt}\KeywordTok{:}\AttributeTok{ }
\AttributeTok{    }\FunctionTok{name}\KeywordTok{:}
\AttributeTok{      }\FunctionTok{de}\KeywordTok{:}\AttributeTok{ Portugiesisch}
\AttributeTok{      }\FunctionTok{en}\KeywordTok{:}\AttributeTok{ Portuguese}
\AttributeTok{      }\FunctionTok{pt}\KeywordTok{:}\AttributeTok{ Português}
\AttributeTok{    }\FunctionTok{level}\KeywordTok{:}\AttributeTok{ }\DecValTok{3}
\AttributeTok{    }\FunctionTok{description}\KeywordTok{:}
\AttributeTok{      }\FunctionTok{de}\KeywordTok{:}\AttributeTok{ fortgeschritten}
\AttributeTok{      }\FunctionTok{en}\KeywordTok{:}\AttributeTok{ advanced}
\AttributeTok{      }\FunctionTok{pt}\KeywordTok{:}\AttributeTok{ avançado}
\end{Highlighting}
\end{Shaded}

As you can see, you can again define language-dependent names in
\texttt{\ name\ } and descriptions in \texttt{\ description\ } .
\texttt{\ level\ } is a numeric value and indicates how many of the four
boxes are filled to indicate you level of proficiency. If you don’t
have the need for a CV of different languages, you can directly define
\texttt{\ name\ } or \texttt{\ description\ } .

You have to call the function with three objects \texttt{\ skills\ } ,
\texttt{\ multilingual\ } , and \texttt{\ metadata\ } and the
corresponding \texttt{\ language\ } of the document:

\begin{Shaded}
\begin{Highlighting}[]
\NormalTok{\#let skills = yaml("dbs/skills.yaml")}
\NormalTok{\#let multilingual = yaml("dbs/multilingual.yaml")}
\NormalTok{\#let metadata = yaml("dbs/metadata.yaml")}
\NormalTok{\#let language = "pt"}

\NormalTok{\#cv{-}auto{-}skills(skills, multilingual, metadata, lang: language)}
\end{Highlighting}
\end{Shaded}

\subsubsection{Print your info without any
formatting}\label{print-your-info-without-any-formatting}

The function \texttt{\ cv-auto\ } is the base function for printing the
provided infos in the specified \texttt{\ yaml\ } file with no further
formatting. The functions \texttt{\ cv-auto-stc\ } and
\texttt{\ cv-auto-stp\ } do only differ in the point that
\texttt{\ cv-auto-stc\ } both give the title in bold,
\texttt{\ cv-auto-stp\ } puts the subtitle in parentheses and
\texttt{\ cv-auto-stc\ } puts the subtitle after a comma.

The structure of the corresponding \texttt{\ yaml\ } files is simple: in
each entry you can have the following entries: \texttt{\ title\ } ,
\texttt{\ subtitle\ } , \texttt{\ location\ } , \texttt{\ description\ }
and \texttt{\ left\ } . \texttt{\ title\ } is mandatory,
\texttt{\ subtitle\ } , \texttt{\ location\ } , and
\texttt{\ description\ } are voluntary. In all functions you need to
specify \texttt{\ left\ } , which indicates period of time, or year. For
\texttt{\ title\ } , \texttt{\ subtitle\ } , \texttt{\ location\ } , and
\texttt{\ description\ } , you can provide a dictionary for different
languages (see below).

\begin{Shaded}
\begin{Highlighting}[]
\FunctionTok{master}\KeywordTok{:}
\AttributeTok{  }\FunctionTok{title}\KeywordTok{:}
\AttributeTok{    }\FunctionTok{de}\KeywordTok{:}\AttributeTok{ Master of Arts}
\AttributeTok{    }\FunctionTok{en}\KeywordTok{:}\AttributeTok{ Master of Arts}
\AttributeTok{    }\FunctionTok{pt}\KeywordTok{:}\AttributeTok{ Pós{-}Graduação}
\AttributeTok{  }\FunctionTok{subtitle}\KeywordTok{:}
\AttributeTok{    }\FunctionTok{de}\KeywordTok{:}\AttributeTok{ Sozialwissenschaften}
\AttributeTok{    }\FunctionTok{en}\KeywordTok{:}\AttributeTok{ Social Sciences}
\AttributeTok{    }\FunctionTok{pt}\KeywordTok{:}\AttributeTok{ Ciências Sociais}
\AttributeTok{  }\FunctionTok{location}\KeywordTok{:}
\AttributeTok{    }\FunctionTok{de}\KeywordTok{:}\AttributeTok{ Exzellenz{-}Universität}
\AttributeTok{    }\FunctionTok{en}\KeywordTok{:}\AttributeTok{ University of Excellence}
\AttributeTok{    }\FunctionTok{pt}\KeywordTok{:}\AttributeTok{ Universidade de Excelência}
\AttributeTok{  }\FunctionTok{description}\KeywordTok{:}
\AttributeTok{    }\FunctionTok{de}\KeywordTok{:}\AttributeTok{ mit Auszeichnung}
\AttributeTok{    }\FunctionTok{en}\KeywordTok{:}\AttributeTok{ with distinction}
\AttributeTok{    }\FunctionTok{pt}\KeywordTok{:}\AttributeTok{ com distinção}
\AttributeTok{  }\FunctionTok{left}\KeywordTok{:}\AttributeTok{ }\StringTok{"2014"}
\end{Highlighting}
\end{Shaded}

In your main document, you then easily call the function and transfer
the standard arguments \texttt{\ what\ } , \texttt{\ metadata\ } , and
\texttt{\ lang\ } .

\begin{Shaded}
\begin{Highlighting}[]
\NormalTok{// section of education }
\NormalTok{\#let education = yaml("dbs/education.yaml")}
\NormalTok{\#let multilingual = yaml("dbs/multilingual.yaml")}
\NormalTok{\#let language = "pt"}

\NormalTok{\#cv{-}auto{-}stp(education, multilingual, lang: language) }

\NormalTok{// section of work positions}
\NormalTok{\#let work = yaml("dbs/work.yaml")}
\NormalTok{\#let multilingual = yaml("dbs/multilingual.yaml")}
\NormalTok{\#let language = "pt"}

\NormalTok{\#cv{-}auto{-}stc(work, multilingual, lang: language)}

\NormalTok{// section of given talks}
\NormalTok{\#let talks = yaml("dbs/talks.yaml")}
\NormalTok{\#let multilingual = yaml("dbs/multilingual.yaml")}
\NormalTok{\#let language = "pt"}

\NormalTok{\#cv{-}auto(talks, multilingual, lang: language)}
\end{Highlighting}
\end{Shaded}

\subsubsection{Creating a list instead of single
entrie}\label{creating-a-list-instead-of-single-entrie}

Sometimes, instead of giving every entry, you want to group by year.
Another example for this case could be that you want to summarize your
memberships or reviewer duties.

The function \texttt{\ cv-auto-list\ } uses just the standard input:

\begin{Shaded}
\begin{Highlighting}[]
\NormalTok{\#let conferences = yaml("dbs/conferences.yaml")}
\NormalTok{\#let multilingual = yaml("dbs/multilingual.yaml")}
\NormalTok{\#let language = "pt"}

\NormalTok{\#cv{-}auto{-}list(conferences, multilingual, lang: language)}
\end{Highlighting}
\end{Shaded}

The corresponding \texttt{\ yaml\ } file is differently organized: The
entry point in the file is the corresponding year. In every year, you
organize your entries (i.e. conference participations). In each entry in
a year, you have the \texttt{\ name\ } and \texttt{\ action\ } entry.
You can provide a dictionary for the \texttt{\ name\ } . For
\texttt{\ action\ } , I used \texttt{\ P\ } and \texttt{\ C\ } , for
\emph{paper/presentation} and \emph{chair} . You can then manually
define this upfront the function call for the reader, or you use the
\texttt{\ i18n.yaml\ } , indicate the explanations for each language in
\texttt{\ exp-confs\ } and then it automatically changes with the
specific language code.

\begin{Shaded}
\begin{Highlighting}[]
\FunctionTok{"2024"}\KeywordTok{:}
\AttributeTok{  }\FunctionTok{conference2}\KeywordTok{:}
\AttributeTok{    }\FunctionTok{name}\KeywordTok{:}\AttributeTok{ European Conference on Gender and Politics}
\AttributeTok{    }\FunctionTok{action}\KeywordTok{:}\AttributeTok{ P}
\AttributeTok{  }\FunctionTok{conference1}\KeywordTok{:}
\AttributeTok{    }\FunctionTok{name}\KeywordTok{:}\AttributeTok{ ECPR General Conference}
\AttributeTok{    }\FunctionTok{action}\KeywordTok{:}\AttributeTok{ P, C}
\end{Highlighting}
\end{Shaded}

The action will be added after each conference name in superscripts.

\subsubsection{Creating a table}\label{creating-a-table}

This case is mostly used for listing your prior teaching experience. The
corresponding \texttt{\ teaching.yaml\ } for this description, is
organized as followed:

\begin{Shaded}
\begin{Highlighting}[]
\FunctionTok{"2024"}\KeywordTok{:}
\AttributeTok{  }\FunctionTok{course1}\KeywordTok{:}
\AttributeTok{    }\FunctionTok{summer}\KeywordTok{:}\AttributeTok{ T}
\AttributeTok{    }\FunctionTok{name}\KeywordTok{:}
\AttributeTok{      }\FunctionTok{de}\KeywordTok{:}\AttributeTok{ }\StringTok{"Statistik+: Einstieg in R leicht gemacht"}
\AttributeTok{      }\FunctionTok{en}\KeywordTok{:}\AttributeTok{ }\StringTok{"Statistics+: Starting with R (de)"}
\AttributeTok{      }\FunctionTok{pt}\KeywordTok{:}\AttributeTok{ }\StringTok{"Estatística+: Começando com R (de)"}
\AttributeTok{    }\FunctionTok{study}\KeywordTok{:}
\AttributeTok{      }\FunctionTok{de}\KeywordTok{:}\AttributeTok{ Bachelor}
\AttributeTok{      }\FunctionTok{en}\KeywordTok{:}\AttributeTok{ Bachelor}
\AttributeTok{      }\FunctionTok{pt}\KeywordTok{:}\AttributeTok{ Graduação}
\AttributeTok{  ...}
\end{Highlighting}
\end{Shaded}

First you indicate the year \texttt{\ "2024"\ } and then you organize
all courses you gave within that year (i.e. here \texttt{\ course1\ } ).
Mandatory are \texttt{\ name\ } and \texttt{\ study\ } . For both you
can indicate a single value or a dictionary corresponding to your chosen
languages. You can provide \texttt{\ summer\ } if you want to indicate
differences for terms. This is \texttt{\ boolean\ } , the specific word
is then given in the \texttt{\ i18n.yaml\ } under
\texttt{\ table-winter\ } resp. \texttt{\ table-summer\ } .

The function then uses again just the standard arguments and plots a
table with the indicated year, name, and study area.

\begin{Shaded}
\begin{Highlighting}[]
\NormalTok{\#let teaching = yaml("dbs/teaching.yaml")}
\NormalTok{\#let multilingual = yaml("dbs/multilingual.yaml")}
\NormalTok{\#let language = "pt"}

\NormalTok{\#cv{-}table{-}teaching(teaching, multilingual, lang: language)}
\end{Highlighting}
\end{Shaded}

\subsubsection{cv-auto-cats()}\label{cv-auto-cats}

In case you want to directly print entries from categories that belong
to one \texttt{\ yaml\ } -file, you can use \texttt{\ cv-auto-cats\ } .
This will print the header for each subcategory and then the belonging
entries.

An example is given in \texttt{\ training.yaml\ } . In this file,
further training is given by categories (i.e., methods and didactics).
Within the categories you have here courses and then \texttt{\ title\ }
, \texttt{\ location\ } , and \texttt{\ left\ } . \texttt{\ location\ }
and \texttt{\ title\ } can be dictionaries if you want to translate
between different languages.

\begin{Shaded}
\begin{Highlighting}[]
\FunctionTok{methods}\KeywordTok{:}
\AttributeTok{  }\FunctionTok{course2}\KeywordTok{:}
\AttributeTok{    }\FunctionTok{title}\KeywordTok{:}\AttributeTok{ Bayesian modelling in the Social Sciences}
\AttributeTok{    }\FunctionTok{location}\KeywordTok{:}\AttributeTok{ An expensive Spring Seminar}
\AttributeTok{    }\FunctionTok{left}\KeywordTok{:}\AttributeTok{ }\StringTok{"2024"}
\AttributeTok{  ...}
\FunctionTok{didactics}\KeywordTok{:}
\AttributeTok{  }\FunctionTok{course2}\KeywordTok{:}
\AttributeTok{    }\FunctionTok{title}\KeywordTok{:}
\AttributeTok{      }\FunctionTok{de}\KeywordTok{:}\AttributeTok{ Konfliktkompetenz I + II}
\AttributeTok{      }\FunctionTok{en}\KeywordTok{:}\AttributeTok{ Conflict competence I + II}
\AttributeTok{      }\FunctionTok{pt}\KeywordTok{:}\AttributeTok{ Competência de conflitos I + II}
\AttributeTok{    }\FunctionTok{location}\KeywordTok{:}\AttributeTok{ }
\AttributeTok{      }\FunctionTok{de}\KeywordTok{:}\AttributeTok{ Universitätsallianz}
\AttributeTok{      }\FunctionTok{en}\KeywordTok{:}\AttributeTok{ University Alliance}
\AttributeTok{      }\FunctionTok{pt}\KeywordTok{:}\AttributeTok{ Aliança Universitária}
\AttributeTok{    }\FunctionTok{left}\KeywordTok{:}\AttributeTok{ }\StringTok{"2019"}
\end{Highlighting}
\end{Shaded}

Call the function as usal:

\begin{Shaded}
\begin{Highlighting}[]
\NormalTok{\#let training = yaml("dbs/training.yaml")}
\NormalTok{\#let multilingual = yaml("dbs/multilingual.yaml")}
\NormalTok{\#let language = "pt"}
\NormalTok{\#let headerLabs = create{-}headers(multilingual, lang: language)}

\NormalTok{\#cv{-}auto{-}cats(training, multilingual, headerLabs, lang: language)}
\end{Highlighting}
\end{Shaded}

\subsubsection{Special cases: long
names}\label{special-cases-long-names}

If you have a long name that crosses the social media side, just set the
argument \texttt{\ split\ } to \texttt{\ true\ } within
\texttt{\ metadata.yaml\ } :

\begin{Shaded}
\begin{Highlighting}[]
\CommentTok{...}
\CommentTok{  personal:}
\CommentTok{    name: ["Mustermensch, Momo"]}
\CommentTok{    split: true}
\CommentTok{  ...    }
\end{Highlighting}
\end{Shaded}

\subsection{Examples}\label{examples}

\pandocbounded{\includegraphics[keepaspectratio]{https://github.com/typst/packages/raw/main/packages/preview/modern-acad-cv/0.1.1/assets/example1.png}}
\pandocbounded{\includegraphics[keepaspectratio]{https://github.com/typst/packages/raw/main/packages/preview/modern-acad-cv/0.1.1/assets/example2.png}}
\pandocbounded{\includegraphics[keepaspectratio]{https://github.com/typst/packages/raw/main/packages/preview/modern-acad-cv/0.1.1/assets/example3.png}}

\href{/app?template=modern-acad-cv&version=0.1.1}{Create project in app}

\subsubsection{How to use}\label{how-to-use}

Click the button above to create a new project using this template in
the Typst app.

You can also use the Typst CLI to start a new project on your computer
using this command:

\begin{verbatim}
typst init @preview/modern-acad-cv:0.1.1
\end{verbatim}

\includesvg[width=0.16667in,height=0.16667in]{/assets/icons/16-copy.svg}

\subsubsection{About}\label{about}

\begin{description}
\tightlist
\item[Author :]
\href{mailto:philipp.kleer@posteo.com}{bpkleer (Philipp Kleer)}
\item[License:]
MIT
\item[Current version:]
0.1.1
\item[Last updated:]
November 28, 2024
\item[First released:]
August 23, 2024
\item[Minimum Typst version:]
0.12.0
\item[Archive size:]
25.1 kB
\href{https://packages.typst.org/preview/modern-acad-cv-0.1.1.tar.gz}{\pandocbounded{\includesvg[keepaspectratio]{/assets/icons/16-download.svg}}}
\item[Repository:]
\href{https://github.com/bpkleer/typst-modern-acad-cv}{GitHub}
\item[Categor y :]
\begin{itemize}
\tightlist
\item[]
\item
  \pandocbounded{\includesvg[keepaspectratio]{/assets/icons/16-user.svg}}
  \href{https://typst.app/universe/search/?category=cv}{CV}
\end{itemize}
\end{description}

\subsubsection{Where to report issues?}\label{where-to-report-issues}

This template is a project of bpkleer (Philipp Kleer) . Report issues on
\href{https://github.com/bpkleer/typst-modern-acad-cv}{their repository}
. You can also try to ask for help with this template on the
\href{https://forum.typst.app}{Forum} .

Please report this template to the Typst team using the
\href{https://typst.app/contact}{contact form} if you believe it is a
safety hazard or infringes upon your rights.

\phantomsection\label{versions}
\subsubsection{Version history}\label{version-history}

\begin{longtable}[]{@{}ll@{}}
\toprule\noalign{}
Version & Release Date \\
\midrule\noalign{}
\endhead
\bottomrule\noalign{}
\endlastfoot
0.1.1 & November 28, 2024 \\
\href{https://typst.app/universe/package/modern-acad-cv/0.1.0/}{0.1.0} &
August 23, 2024 \\
\end{longtable}

Typst GmbH did not create this template and cannot guarantee correct
functionality of this template or compatibility with any version of the
Typst compiler or app.


\title{typst.app/universe/package/syntree}

\phantomsection\label{banner}
\section{syntree}\label{syntree}

{ 0.2.0 }

Linguistics syntax/parse tree rendering

\phantomsection\label{readme}
\textbf{syntree} is a typst package for rendering syntax trees / parse
trees (the kind linguists use).

The name and syntax are inspired by Miles Shang’s
\href{https://github.com/mshang/syntree}{syntree} . Here’s an example
to get started:

\begin{longtable}[]{@{}
  >{\raggedright\arraybackslash}p{(\linewidth - 2\tabcolsep) * \real{0.5000}}
  >{\raggedright\arraybackslash}p{(\linewidth - 2\tabcolsep) * \real{0.5000}}@{}}
\toprule\noalign{}
\endhead
\bottomrule\noalign{}
\endlastfoot
\begin{minipage}[t]{\linewidth}\raggedright
\begin{Shaded}
\begin{Highlighting}[]
\NormalTok{\#import "@preview/syntree:0.2.0": syntree}

\NormalTok{\#syntree(}
\NormalTok{  nonterminal: (font: "Linux Biolinum"),}
\NormalTok{  terminal: (fill: blue),}
\NormalTok{  child{-}spacing: 3em, // default 1em}
\NormalTok{  layer{-}spacing: 2em, // default 2.3em}
\NormalTok{  "[S [NP This] [VP [V is] [\^{}NP a wug]]]"}
\NormalTok{)}
\end{Highlighting}
\end{Shaded}
\end{minipage} &
\pandocbounded{\includegraphics[keepaspectratio]{https://github.com/lynn/typst-syntree/assets/16232127/d0c680b2-4fd0-420f-b350-9e9c96ac37f3}} \\
\end{longtable}

There’s limited support for formulas inside nodes; try
\texttt{\ \#syntree("{[}DP\$zws\_i\$\ this{]}")\ } or
\texttt{\ \#syntree("{[}C\ \$diameter\${]}")\ } .

For more flexible tree-drawing, use \texttt{\ tree\ } :

\begin{longtable}[]{@{}
  >{\raggedright\arraybackslash}p{(\linewidth - 2\tabcolsep) * \real{0.5000}}
  >{\raggedright\arraybackslash}p{(\linewidth - 2\tabcolsep) * \real{0.5000}}@{}}
\toprule\noalign{}
\endhead
\bottomrule\noalign{}
\endlastfoot
\begin{minipage}[t]{\linewidth}\raggedright
\begin{Shaded}
\begin{Highlighting}[]
\NormalTok{\#import "@preview/syntree:0.2.0": tree}

\NormalTok{\#let bx(col) = box(fill: col, width: 1em, height: 1em)}
\NormalTok{\#tree("colors",}
\NormalTok{  tree("warm", bx(red), bx(orange)),}
\NormalTok{  tree("cool", bx(blue), bx(teal)))}
\end{Highlighting}
\end{Shaded}
\end{minipage} &
\pandocbounded{\includegraphics[keepaspectratio]{https://github.com/lynn/typst-syntree/assets/16232127/bc979614-e2ce-4616-97d1-1584788fc71f}} \\
\end{longtable}

\subsubsection{How to add}\label{how-to-add}

Copy this into your project and use the import as \texttt{\ syntree\ }

\begin{verbatim}
#import "@preview/syntree:0.2.0"
\end{verbatim}

\includesvg[width=0.16667in,height=0.16667in]{/assets/icons/16-copy.svg}

Check the docs for
\href{https://typst.app/docs/reference/scripting/\#packages}{more
information on how to import packages} .

\subsubsection{About}\label{about}

\begin{description}
\tightlist
\item[Author :]
\href{https://github.com/lynn}{Lynn}
\item[License:]
MIT
\item[Current version:]
0.2.0
\item[Last updated:]
January 12, 2024
\item[First released:]
July 8, 2023
\item[Archive size:]
2.50 kB
\href{https://packages.typst.org/preview/syntree-0.2.0.tar.gz}{\pandocbounded{\includesvg[keepaspectratio]{/assets/icons/16-download.svg}}}
\item[Repository:]
\href{https://github.com/lynn/typst-syntree}{GitHub}
\end{description}

\subsubsection{Where to report issues?}\label{where-to-report-issues}

This package is a project of Lynn . Report issues on
\href{https://github.com/lynn/typst-syntree}{their repository} . You can
also try to ask for help with this package on the
\href{https://forum.typst.app}{Forum} .

Please report this package to the Typst team using the
\href{https://typst.app/contact}{contact form} if you believe it is a
safety hazard or infringes upon your rights.

\phantomsection\label{versions}
\subsubsection{Version history}\label{version-history}

\begin{longtable}[]{@{}ll@{}}
\toprule\noalign{}
Version & Release Date \\
\midrule\noalign{}
\endhead
\bottomrule\noalign{}
\endlastfoot
0.2.0 & January 12, 2024 \\
\href{https://typst.app/universe/package/syntree/0.1.0/}{0.1.0} & July
8, 2023 \\
\end{longtable}

Typst GmbH did not create this package and cannot guarantee correct
functionality of this package or compatibility with any version of the
Typst compiler or app.


\title{typst.app/universe/package/minimalistic-latex-cv}

\phantomsection\label{banner}
\phantomsection\label{template-thumbnail}
\pandocbounded{\includegraphics[keepaspectratio]{https://packages.typst.org/preview/thumbnails/minimalistic-latex-cv-0.1.1-small.webp}}

\section{minimalistic-latex-cv}\label{minimalistic-latex-cv}

{ 0.1.1 }

A minimalistic LaTeX-style CV template for professionals.

\href{/app?template=minimalistic-latex-cv&version=0.1.1}{Create project
in app}

\phantomsection\label{readme}
This is a Typst template for a minimalistic LaTeX-style CV. It provides
a simple structure for a CV with a header, a section for professional
experience, a section for education, and a section for skills and
languages.

\subsection{Usage}\label{usage}

You can use this template in the Typst web app by clicking “Start from
template� on the dashboard and searching for
\texttt{\ minimalistic-latex-cv\ } .

Alternatively, you can use the CLI to kick this project off using the
command

\begin{verbatim}
typst init @preview/minimalistic-latex-cv
\end{verbatim}

Typst will create a new directory with all the files needed to get you
started.

\subsection{Configuration}\label{configuration}

This template exports the \texttt{\ cv\ } function with the following
named arguments:

\begin{itemize}
\tightlist
\item
  \texttt{\ name\ } : The name of the person.
\item
  \texttt{\ metadata\ } : A dictionary of metadata of the person to be
  displayed in the header.
\item
  \texttt{\ photo\ } : The path to the photo of the person.
\item
  \texttt{\ lang\ } : The language of the document.
\end{itemize}

The function also accepts a single, positional argument for the body of
the paper.

The template will initialize your package with a sample call to the
\texttt{\ cv\ } function in a show rule. If you want to change an
existing project to use this template, you can add a show rule like this
at the top of your file:

\begin{Shaded}
\begin{Highlighting}[]
\NormalTok{\#import "@preview/minimalistic{-}latex{-}cv:0.1.1": cv}

\NormalTok{\#show: cv.with(}
\NormalTok{  name: "Your Name",}
\NormalTok{  metadata: (}
\NormalTok{    email: "your@email.com",}
\NormalTok{    telephone: "+123456789",}
\NormalTok{  ),}
\NormalTok{  photo: image("photo.jpeg"),}
\NormalTok{  lang: "en",}
\NormalTok{)}

\NormalTok{// Your content goes below.}
\end{Highlighting}
\end{Shaded}

\href{/app?template=minimalistic-latex-cv&version=0.1.1}{Create project
in app}

\subsubsection{How to use}\label{how-to-use}

Click the button above to create a new project using this template in
the Typst app.

You can also use the Typst CLI to start a new project on your computer
using this command:

\begin{verbatim}
typst init @preview/minimalistic-latex-cv:0.1.1
\end{verbatim}

\includesvg[width=0.16667in,height=0.16667in]{/assets/icons/16-copy.svg}

\subsubsection{About}\label{about}

\begin{description}
\tightlist
\item[Author :]
\href{https://github.com/itsyoboieltr}{Norbert Elter}
\item[License:]
MIT-0
\item[Current version:]
0.1.1
\item[Last updated:]
May 23, 2024
\item[First released:]
March 19, 2024
\item[Minimum Typst version:]
0.11.1
\item[Archive size:]
7.18 kB
\href{https://packages.typst.org/preview/minimalistic-latex-cv-0.1.1.tar.gz}{\pandocbounded{\includesvg[keepaspectratio]{/assets/icons/16-download.svg}}}
\item[Categor y :]
\begin{itemize}
\tightlist
\item[]
\item
  \pandocbounded{\includesvg[keepaspectratio]{/assets/icons/16-user.svg}}
  \href{https://typst.app/universe/search/?category=cv}{CV}
\end{itemize}
\end{description}

\subsubsection{Where to report issues?}\label{where-to-report-issues}

This template is a project of Norbert Elter . You can also try to ask
for help with this template on the \href{https://forum.typst.app}{Forum}
.

Please report this template to the Typst team using the
\href{https://typst.app/contact}{contact form} if you believe it is a
safety hazard or infringes upon your rights.

\phantomsection\label{versions}
\subsubsection{Version history}\label{version-history}

\begin{longtable}[]{@{}ll@{}}
\toprule\noalign{}
Version & Release Date \\
\midrule\noalign{}
\endhead
\bottomrule\noalign{}
\endlastfoot
0.1.1 & May 23, 2024 \\
\href{https://typst.app/universe/package/minimalistic-latex-cv/0.1.0/}{0.1.0}
& March 19, 2024 \\
\end{longtable}

Typst GmbH did not create this template and cannot guarantee correct
functionality of this template or compatibility with any version of the
Typst compiler or app.


\title{typst.app/universe/package/soviet-matrix}

\phantomsection\label{banner}
\phantomsection\label{template-thumbnail}
\pandocbounded{\includegraphics[keepaspectratio]{https://packages.typst.org/preview/thumbnails/soviet-matrix-0.1.1-small.webp}}

\section{soviet-matrix}\label{soviet-matrix}

{ 0.1.1 }

Tetris game in Typst

\href{/app?template=soviet-matrix&version=0.1.1}{Create project in app}

\phantomsection\label{readme}
This is a classic Tetris game implemented using Typst. The goal is to
manipulate falling blocks to create and clear horizontal lines without
letting the blocks stack up to the top of the playing field.

\pandocbounded{\includegraphics[keepaspectratio]{https://github.com/typst/packages/raw/main/packages/preview/soviet-matrix/0.1.1/demo.gif}}

\subsection{How to Play}\label{how-to-play}

You can play the game in two ways:

\begin{enumerate}
\item
  \textbf{Online:}

  \begin{itemize}
  \tightlist
  \item
    Visit
    \url{https://typst.app/app?template=soviet-matrix&version=0.1.0} .
  \item
    Enter any title of your choice and click \textbf{Create} .
  \end{itemize}
\item
  \textbf{Locally:}

  \begin{itemize}
  \item
    Open your command line interface.
  \item
    Run the following command:

\begin{Shaded}
\begin{Highlighting}[]
\ExtensionTok{typst}\NormalTok{ init @preview/soviet{-}matrix}
\end{Highlighting}
\end{Shaded}
  \item
    Typst will create a new directory.
  \item
    Open \texttt{\ main.typ\ } in the created directory.
  \item
    Use the
    \href{https://marketplace.visualstudio.com/items?itemName=mgt19937.typst-preview}{Typst
    Preview VS Code extension} for live preview and gameplay.
  \end{itemize}
\end{enumerate}

Enjoy the game!

\subsection{Controls}\label{controls}

\begin{itemize}
\tightlist
\item
  Move Left: a
\item
  Move Right: d
\item
  Soft Drop: s
\item
  Hard Drop: f
\item
  Rotate Left: q
\item
  Rotate Right: e
\item
  180-degree Rotate: w
\end{itemize}

\subsection{Changing the Game Seed}\label{changing-the-game-seed}

If you want to play different game scenarios, you can change the game
seed using the following method:

\begin{Shaded}
\begin{Highlighting}[]
\NormalTok{\#import "@preview/soviet{-}matrix:0.1.0": game}
\NormalTok{\#show: game.with(seed: 123) // Change the game seed}
\end{Highlighting}
\end{Shaded}

Replace \texttt{\ 123\ } with any number of your choice.

\subsection{Changing Key Bindings}\label{changing-key-bindings}

Modify the \texttt{\ actions\ } parameter in the \texttt{\ game.with\ }
method to change the key bindings. The default key bindings are as
follows:

\begin{Shaded}
\begin{Highlighting}[]
\NormalTok{\#show: game.with(seed: 0, actions: (}
\NormalTok{  left: ("a", ),}
\NormalTok{  right: ("d", ),}
\NormalTok{  down: ("s", ),}
\NormalTok{  left{-}rotate: ("q", ),}
\NormalTok{  right{-}rotate: ("e", ),}
\NormalTok{  half{-}turn: ("w", ),}
\NormalTok{  fast{-}drop: ("f", ),}
\NormalTok{))}
\end{Highlighting}
\end{Shaded}

\href{/app?template=soviet-matrix&version=0.1.1}{Create project in app}

\subsubsection{How to use}\label{how-to-use}

Click the button above to create a new project using this template in
the Typst app.

You can also use the Typst CLI to start a new project on your computer
using this command:

\begin{verbatim}
typst init @preview/soviet-matrix:0.1.1
\end{verbatim}

\includesvg[width=0.16667in,height=0.16667in]{/assets/icons/16-copy.svg}

\subsubsection{About}\label{about}

\begin{description}
\tightlist
\item[Author :]
\href{https://github.com/YouXam}{YouXam}
\item[License:]
MIT
\item[Current version:]
0.1.1
\item[Last updated:]
July 3, 2024
\item[First released:]
June 10, 2024
\item[Minimum Typst version:]
0.11.0
\item[Archive size:]
5.17 kB
\href{https://packages.typst.org/preview/soviet-matrix-0.1.1.tar.gz}{\pandocbounded{\includesvg[keepaspectratio]{/assets/icons/16-download.svg}}}
\item[Repository:]
\href{https://github.com/YouXam/soviet-matrix}{GitHub}
\item[Categor y :]
\begin{itemize}
\tightlist
\item[]
\item
  \pandocbounded{\includesvg[keepaspectratio]{/assets/icons/16-smile.svg}}
  \href{https://typst.app/universe/search/?category=fun}{Fun}
\end{itemize}
\end{description}

\subsubsection{Where to report issues?}\label{where-to-report-issues}

This template is a project of YouXam . Report issues on
\href{https://github.com/YouXam/soviet-matrix}{their repository} . You
can also try to ask for help with this template on the
\href{https://forum.typst.app}{Forum} .

Please report this template to the Typst team using the
\href{https://typst.app/contact}{contact form} if you believe it is a
safety hazard or infringes upon your rights.

\phantomsection\label{versions}
\subsubsection{Version history}\label{version-history}

\begin{longtable}[]{@{}ll@{}}
\toprule\noalign{}
Version & Release Date \\
\midrule\noalign{}
\endhead
\bottomrule\noalign{}
\endlastfoot
0.1.1 & July 3, 2024 \\
\href{https://typst.app/universe/package/soviet-matrix/0.1.0/}{0.1.0} &
June 10, 2024 \\
\end{longtable}

Typst GmbH did not create this template and cannot guarantee correct
functionality of this template or compatibility with any version of the
Typst compiler or app.


\title{typst.app/universe/package/plotst}

\phantomsection\label{banner}
\section{plotst}\label{plotst}

{ 0.2.0 }

A library to draw a variety of graphs and plots to use in your papers

\phantomsection\label{readme}
A Typst library for drawing graphs and plots. Made by Gewi413 and
Pegacraffft

\subsection{Currently supported
graphs}\label{currently-supported-graphs}

\begin{itemize}
\item
  Scatter plots
\item
  Graph charts
\item
  Histograms
\item
  Bar charts
\item
  Pie charts
\item
  Overlaying plots/charts

  (more to come)
\end{itemize}

\subsection{How to use}\label{how-to-use}

To use the package you can import it through this command
\texttt{\ import\ "@preview/plotst:0.2.0":\ *\ } . The documentation is
found in the
\href{https://github.com/Pegacraft/typst-plotting/blob/8d834689359b708ce75fe51be05eed45570e463e/docs/Docs.pdf}{Docs.pdf}
file. It contains all functions necessary to use this library. It also
includes a tutorial to create every available plot under their
respective render methods.

If you need some example code, check out
\href{https://github.com/Pegacraft/typst-plotting/blob/8d834689359b708ce75fe51be05eed45570e463e/example/main.typ}{main.typ}
. It also includes a
\href{https://github.com/Pegacraft/typst-plotting/blob/8d834689359b708ce75fe51be05eed45570e463e/example/Plotting.pdf}{compiled
version} .

\subsection{Examples:}\label{examples}

All these images were created using the
\href{https://github.com/Pegacraft/typst-plotting/blob/8d834689359b708ce75fe51be05eed45570e463e/example/main.typ}{main.typ}
.

\subsubsection{Scatter plots}\label{scatter-plots}

\begin{Shaded}
\begin{Highlighting}[]
\CommentTok{// Plot 1:}
\CommentTok{// The data to be displayed  }
\KeywordTok{let}\NormalTok{ gender\_data }\OperatorTok{=}\NormalTok{ (}
\NormalTok{  (}\StringTok{"w"}\OperatorTok{,} \DecValTok{1}\NormalTok{)}\OperatorTok{,}\NormalTok{ (}\StringTok{"w"}\OperatorTok{,} \DecValTok{3}\NormalTok{)}\OperatorTok{,}\NormalTok{ (}\StringTok{"w"}\OperatorTok{,} \DecValTok{5}\NormalTok{)}\OperatorTok{,}\NormalTok{ (}\StringTok{"w"}\OperatorTok{,} \DecValTok{4}\NormalTok{)}\OperatorTok{,}\NormalTok{ (}\StringTok{"m"}\OperatorTok{,} \DecValTok{2}\NormalTok{)}\OperatorTok{,}\NormalTok{ (}\StringTok{"m"}\OperatorTok{,} \DecValTok{2}\NormalTok{)}\OperatorTok{,}
\NormalTok{  (}\StringTok{"m"}\OperatorTok{,} \DecValTok{4}\NormalTok{)}\OperatorTok{,}\NormalTok{ (}\StringTok{"m"}\OperatorTok{,} \DecValTok{6}\NormalTok{)}\OperatorTok{,}\NormalTok{ (}\StringTok{"d"}\OperatorTok{,} \DecValTok{1}\NormalTok{)}\OperatorTok{,}\NormalTok{ (}\StringTok{"d"}\OperatorTok{,} \DecValTok{9}\NormalTok{)}\OperatorTok{,}\NormalTok{ (}\StringTok{"d"}\OperatorTok{,} \DecValTok{5}\NormalTok{)}\OperatorTok{,}\NormalTok{ (}\StringTok{"d"}\OperatorTok{,} \DecValTok{8}\NormalTok{)}\OperatorTok{,}
\NormalTok{  (}\StringTok{"d"}\OperatorTok{,} \DecValTok{3}\NormalTok{)}\OperatorTok{,}\NormalTok{ (}\StringTok{"d"}\OperatorTok{,} \DecValTok{1}\NormalTok{)}\OperatorTok{,}\NormalTok{ (}\DecValTok{0}\OperatorTok{,} \DecValTok{11}\NormalTok{)}
\NormalTok{)}

\CommentTok{// Create the axes used for the chart}
\KeywordTok{let}\NormalTok{ y\_axis }\OperatorTok{=} \FunctionTok{axis}\NormalTok{(min}\OperatorTok{:} \DecValTok{0}\OperatorTok{,}\NormalTok{ max}\OperatorTok{:} \DecValTok{11}\OperatorTok{,}\NormalTok{ step}\OperatorTok{:} \DecValTok{1}\OperatorTok{,}\NormalTok{ location}\OperatorTok{:} \StringTok{"left"}\OperatorTok{,}\NormalTok{ helper\_lines}\OperatorTok{:} \KeywordTok{true}\OperatorTok{,}\NormalTok{ invert\_markings}\OperatorTok{:} \KeywordTok{false}\OperatorTok{,}\NormalTok{ title}\OperatorTok{:} \StringTok{"foo"}\NormalTok{)}
\KeywordTok{let}\NormalTok{ x\_axis }\OperatorTok{=} \FunctionTok{axis}\NormalTok{(values}\OperatorTok{:}\NormalTok{ (}\StringTok{""}\OperatorTok{,} \StringTok{"m"}\OperatorTok{,} \StringTok{"w"}\OperatorTok{,} \StringTok{"d"}\NormalTok{)}\OperatorTok{,}\NormalTok{ location}\OperatorTok{:} \StringTok{"bottom"}\OperatorTok{,}\NormalTok{ helper\_lines}\OperatorTok{:} \KeywordTok{true}\OperatorTok{,}\NormalTok{ invert\_markings}\OperatorTok{:} \KeywordTok{false}\OperatorTok{,}\NormalTok{ title}\OperatorTok{:} \StringTok{"Gender"}\NormalTok{)}

\CommentTok{// Combine the axes and the data and feed it to the plot render function.}
\KeywordTok{let}\NormalTok{ pl }\OperatorTok{=} \FunctionTok{plot}\NormalTok{(data}\OperatorTok{:}\NormalTok{ gender\_data}\OperatorTok{,}\NormalTok{ axes}\OperatorTok{:}\NormalTok{ (x\_axis}\OperatorTok{,}\NormalTok{ y\_axis))}
\FunctionTok{scatter\_plot}\NormalTok{(pl}\OperatorTok{,}\NormalTok{ (}\DecValTok{100}\OperatorTok{\%,}\DecValTok{50}\OperatorTok{\%}\NormalTok{))}

\CommentTok{// Plot 2:}
\CommentTok{// Same as above}
\KeywordTok{let}\NormalTok{ data }\OperatorTok{=}\NormalTok{ (}
\NormalTok{  (}\DecValTok{0}\OperatorTok{,} \DecValTok{0}\NormalTok{)}\OperatorTok{,}\NormalTok{ (}\DecValTok{2}\OperatorTok{,} \DecValTok{2}\NormalTok{)}\OperatorTok{,}\NormalTok{ (}\DecValTok{3}\OperatorTok{,} \DecValTok{0}\NormalTok{)}\OperatorTok{,}\NormalTok{ (}\DecValTok{4}\OperatorTok{,} \DecValTok{4}\NormalTok{)}\OperatorTok{,}\NormalTok{ (}\DecValTok{5}\OperatorTok{,} \DecValTok{7}\NormalTok{)}\OperatorTok{,}\NormalTok{ (}\DecValTok{6}\OperatorTok{,} \DecValTok{6}\NormalTok{)}\OperatorTok{,}\NormalTok{ (}\DecValTok{7}\OperatorTok{,} \DecValTok{9}\NormalTok{)}\OperatorTok{,}\NormalTok{ (}\DecValTok{8}\OperatorTok{,} \DecValTok{5}\NormalTok{)}\OperatorTok{,}\NormalTok{ (}\DecValTok{9}\OperatorTok{,} \DecValTok{9}\NormalTok{)}\OperatorTok{,}\NormalTok{ (}\DecValTok{10}\OperatorTok{,} \DecValTok{1}\NormalTok{)}
\NormalTok{)}
\KeywordTok{let}\NormalTok{ x\_axis }\OperatorTok{=} \FunctionTok{axis}\NormalTok{(min}\OperatorTok{:} \DecValTok{0}\OperatorTok{,}\NormalTok{ max}\OperatorTok{:} \DecValTok{11}\OperatorTok{,}\NormalTok{ step}\OperatorTok{:} \DecValTok{2}\OperatorTok{,}\NormalTok{ location}\OperatorTok{:} \StringTok{"bottom"}\NormalTok{)}
\KeywordTok{let}\NormalTok{ y\_axis }\OperatorTok{=} \FunctionTok{axis}\NormalTok{(min}\OperatorTok{:} \DecValTok{0}\OperatorTok{,}\NormalTok{ max}\OperatorTok{:} \DecValTok{11}\OperatorTok{,}\NormalTok{ step}\OperatorTok{:} \DecValTok{2}\OperatorTok{,}\NormalTok{ location}\OperatorTok{:} \StringTok{"left"}\OperatorTok{,}\NormalTok{ helper\_lines}\OperatorTok{:} \KeywordTok{false}\NormalTok{)}
\KeywordTok{let}\NormalTok{ pl }\OperatorTok{=} \FunctionTok{plot}\NormalTok{(data}\OperatorTok{:}\NormalTok{ data}\OperatorTok{,}\NormalTok{ axes}\OperatorTok{:}\NormalTok{ (x\_axis}\OperatorTok{,}\NormalTok{ y\_axis))}
\FunctionTok{scatter\_plot}\NormalTok{(pl}\OperatorTok{,}\NormalTok{ (}\DecValTok{100}\OperatorTok{\%,} \DecValTok{25}\OperatorTok{\%}\NormalTok{))}
\end{Highlighting}
\end{Shaded}

\pandocbounded{\includegraphics[keepaspectratio]{https://raw.githubusercontent.com/Pegacraft/typst-plotting/8d834689359b708ce75fe51be05eed45570e463e/images/scatter.png}}

\subsubsection{Graph charts}\label{graph-charts}

\begin{Shaded}
\begin{Highlighting}[]
\CommentTok{// The data to be displayed}
\KeywordTok{let}\NormalTok{ data }\OperatorTok{=}\NormalTok{ (}
\NormalTok{  (}\DecValTok{0}\OperatorTok{,} \DecValTok{0}\NormalTok{)}\OperatorTok{,}\NormalTok{ (}\DecValTok{2}\OperatorTok{,} \DecValTok{2}\NormalTok{)}\OperatorTok{,}\NormalTok{ (}\DecValTok{3}\OperatorTok{,} \DecValTok{0}\NormalTok{)}\OperatorTok{,}\NormalTok{ (}\DecValTok{4}\OperatorTok{,} \DecValTok{4}\NormalTok{)}\OperatorTok{,}\NormalTok{ (}\DecValTok{5}\OperatorTok{,} \DecValTok{7}\NormalTok{)}\OperatorTok{,}\NormalTok{ (}\DecValTok{6}\OperatorTok{,} \DecValTok{6}\NormalTok{)}\OperatorTok{,}\NormalTok{ (}\DecValTok{7}\OperatorTok{,} \DecValTok{9}\NormalTok{)}\OperatorTok{,}\NormalTok{ (}\DecValTok{8}\OperatorTok{,} \DecValTok{5}\NormalTok{)}\OperatorTok{,}\NormalTok{ (}\DecValTok{9}\OperatorTok{,} \DecValTok{9}\NormalTok{)}\OperatorTok{,}\NormalTok{ (}\DecValTok{10}\OperatorTok{,} \DecValTok{1}\NormalTok{)}
\NormalTok{)}

\CommentTok{// Create the axes used for the chart }
\KeywordTok{let}\NormalTok{ x\_axis }\OperatorTok{=} \FunctionTok{axis}\NormalTok{(min}\OperatorTok{:} \DecValTok{0}\OperatorTok{,}\NormalTok{ max}\OperatorTok{:} \DecValTok{11}\OperatorTok{,}\NormalTok{ step}\OperatorTok{:} \DecValTok{2}\OperatorTok{,}\NormalTok{ location}\OperatorTok{:} \StringTok{"bottom"}\NormalTok{)}
\KeywordTok{let}\NormalTok{ y\_axis }\OperatorTok{=} \FunctionTok{axis}\NormalTok{(min}\OperatorTok{:} \DecValTok{0}\OperatorTok{,}\NormalTok{ max}\OperatorTok{:} \DecValTok{11}\OperatorTok{,}\NormalTok{ step}\OperatorTok{:} \DecValTok{2}\OperatorTok{,}\NormalTok{ location}\OperatorTok{:} \StringTok{"left"}\OperatorTok{,}\NormalTok{ helper\_lines}\OperatorTok{:} \KeywordTok{false}\NormalTok{)}

\CommentTok{// Combine the axes and the data and feed it to the plot render function.}
\KeywordTok{let}\NormalTok{ pl }\OperatorTok{=} \FunctionTok{plot}\NormalTok{(data}\OperatorTok{:}\NormalTok{ data}\OperatorTok{,}\NormalTok{ axes}\OperatorTok{:}\NormalTok{ (x\_axis}\OperatorTok{,}\NormalTok{ y\_axis))}
\FunctionTok{graph\_plot}\NormalTok{(pl}\OperatorTok{,}\NormalTok{ (}\DecValTok{100}\OperatorTok{\%,} \DecValTok{25}\OperatorTok{\%}\NormalTok{))}
\FunctionTok{graph\_plot}\NormalTok{(pl}\OperatorTok{,}\NormalTok{ (}\DecValTok{100}\OperatorTok{\%,} \DecValTok{25}\OperatorTok{\%}\NormalTok{)}\OperatorTok{,}\NormalTok{ rounding}\OperatorTok{:} \DecValTok{30}\OperatorTok{\%,}\NormalTok{ caption}\OperatorTok{:} \StringTok{"Graph Plot with caption and rounding"}\NormalTok{)}
\end{Highlighting}
\end{Shaded}

\pandocbounded{\includegraphics[keepaspectratio]{https://raw.githubusercontent.com/Pegacraft/typst-plotting/8d834689359b708ce75fe51be05eed45570e463e/images/graph.png}}

\subsubsection{Histograms}\label{histograms}

\begin{Shaded}
\begin{Highlighting}[]
\CommentTok{// Plot 1:}
\CommentTok{// The data to be displayed}
\KeywordTok{let}\NormalTok{ data }\OperatorTok{=}\NormalTok{ (}
  \DecValTok{18000}\OperatorTok{,} \DecValTok{18000}\OperatorTok{,} \DecValTok{18000}\OperatorTok{,} \DecValTok{18000}\OperatorTok{,} \DecValTok{18000}\OperatorTok{,} \DecValTok{18000}\OperatorTok{,} \DecValTok{18000}\OperatorTok{,} \DecValTok{18000}\OperatorTok{,}
  \DecValTok{18000}\OperatorTok{,} \DecValTok{18000}\OperatorTok{,} \DecValTok{28000}\OperatorTok{,} \DecValTok{28000}\OperatorTok{,} \DecValTok{28000}\OperatorTok{,} \DecValTok{28000}\OperatorTok{,} \DecValTok{28000}\OperatorTok{,} \DecValTok{28000}\OperatorTok{,}
  \DecValTok{28000}\OperatorTok{,} \DecValTok{28000}\OperatorTok{,} \DecValTok{28000}\OperatorTok{,} \DecValTok{28000}\OperatorTok{,} \DecValTok{28000}\OperatorTok{,} \DecValTok{28000}\OperatorTok{,} \DecValTok{28000}\OperatorTok{,} \DecValTok{28000}\OperatorTok{,}
  \DecValTok{28000}\OperatorTok{,} \DecValTok{28000}\OperatorTok{,} \DecValTok{28000}\OperatorTok{,} \DecValTok{28000}\OperatorTok{,} \DecValTok{28000}\OperatorTok{,} \DecValTok{28000}\OperatorTok{,} \DecValTok{28000}\OperatorTok{,} \DecValTok{28000}\OperatorTok{,}
  \DecValTok{35000}\OperatorTok{,} \DecValTok{46000}\OperatorTok{,} \DecValTok{75000}\OperatorTok{,} \DecValTok{95000}
\NormalTok{)}

\CommentTok{// Classify the data}
\KeywordTok{let}\NormalTok{ classes }\OperatorTok{=} \FunctionTok{class\_generator}\NormalTok{(}\DecValTok{10000}\OperatorTok{,} \DecValTok{50000}\OperatorTok{,} \DecValTok{4}\NormalTok{)}
\NormalTok{classes}\OperatorTok{.}\FunctionTok{push}\NormalTok{(}\KeywordTok{class}\NormalTok{(}\DecValTok{50000}\OperatorTok{,} \DecValTok{100000}\NormalTok{))}
\NormalTok{classes }\OperatorTok{=} \FunctionTok{classify}\NormalTok{(data}\OperatorTok{,}\NormalTok{ classes)}

\CommentTok{// Create the axes used for the chart }
\KeywordTok{let}\NormalTok{ x\_axis }\OperatorTok{=} \FunctionTok{axis}\NormalTok{(min}\OperatorTok{:} \DecValTok{0}\OperatorTok{,}\NormalTok{ max}\OperatorTok{:} \DecValTok{100000}\OperatorTok{,}\NormalTok{ step}\OperatorTok{:} \DecValTok{10000}\OperatorTok{,}\NormalTok{ location}\OperatorTok{:} \StringTok{"bottom"}\NormalTok{)}
\KeywordTok{let}\NormalTok{ y\_axis }\OperatorTok{=} \FunctionTok{axis}\NormalTok{(min}\OperatorTok{:} \DecValTok{0}\OperatorTok{,}\NormalTok{ max}\OperatorTok{:} \DecValTok{31}\OperatorTok{,}\NormalTok{ step}\OperatorTok{:} \DecValTok{5}\OperatorTok{,}\NormalTok{ location}\OperatorTok{:} \StringTok{"left"}\OperatorTok{,}\NormalTok{ helper\_lines}\OperatorTok{:} \KeywordTok{true}\NormalTok{)}

\CommentTok{// Combine the axes and the data and feed it to the plot render function.}
\KeywordTok{let}\NormalTok{ pl }\OperatorTok{=} \FunctionTok{plot}\NormalTok{(data}\OperatorTok{:}\NormalTok{ classes}\OperatorTok{,}\NormalTok{ axes}\OperatorTok{:}\NormalTok{ (x\_axis}\OperatorTok{,}\NormalTok{ y\_axis))}
\FunctionTok{histogram}\NormalTok{(pl}\OperatorTok{,}\NormalTok{ (}\DecValTok{100}\OperatorTok{\%,} \DecValTok{40}\OperatorTok{\%}\NormalTok{)}\OperatorTok{,}\NormalTok{ stroke}\OperatorTok{:}\NormalTok{ black}\OperatorTok{,}\NormalTok{ fill}\OperatorTok{:}\NormalTok{ (purple}\OperatorTok{,}\NormalTok{ blue}\OperatorTok{,}\NormalTok{ red}\OperatorTok{,}\NormalTok{ green}\OperatorTok{,}\NormalTok{ yellow))}

\CommentTok{// Plot 2:}
\CommentTok{// Create the different classes}
\KeywordTok{let}\NormalTok{ classes }\OperatorTok{=}\NormalTok{ ()}
\NormalTok{classes}\OperatorTok{.}\FunctionTok{push}\NormalTok{(}\KeywordTok{class}\NormalTok{(}\DecValTok{11}\OperatorTok{,} \DecValTok{13}\NormalTok{))}
\NormalTok{classes}\OperatorTok{.}\FunctionTok{push}\NormalTok{(}\KeywordTok{class}\NormalTok{(}\DecValTok{13}\OperatorTok{,} \DecValTok{15}\NormalTok{))}
\NormalTok{classes}\OperatorTok{.}\FunctionTok{push}\NormalTok{(}\KeywordTok{class}\NormalTok{(}\DecValTok{1}\OperatorTok{,} \DecValTok{6}\NormalTok{))}
\NormalTok{classes}\OperatorTok{.}\FunctionTok{push}\NormalTok{(}\KeywordTok{class}\NormalTok{(}\DecValTok{6}\OperatorTok{,} \DecValTok{11}\NormalTok{))}
\NormalTok{classes}\OperatorTok{.}\FunctionTok{push}\NormalTok{(}\KeywordTok{class}\NormalTok{(}\DecValTok{15}\OperatorTok{,} \DecValTok{30}\NormalTok{))}

\CommentTok{// Define the data to map}
\KeywordTok{let}\NormalTok{ data }\OperatorTok{=}\NormalTok{ ((}\DecValTok{20}\OperatorTok{,} \DecValTok{2}\NormalTok{)}\OperatorTok{,}\NormalTok{ (}\DecValTok{30}\OperatorTok{,} \DecValTok{7}\NormalTok{)}\OperatorTok{,}\NormalTok{ (}\DecValTok{16}\OperatorTok{,} \DecValTok{12}\NormalTok{)}\OperatorTok{,}\NormalTok{ (}\DecValTok{40}\OperatorTok{,} \DecValTok{13}\NormalTok{)}\OperatorTok{,}\NormalTok{ (}\DecValTok{5}\OperatorTok{,} \DecValTok{17}\NormalTok{))}

\CommentTok{// Create the axes}
\KeywordTok{let}\NormalTok{ x\_axis }\OperatorTok{=} \FunctionTok{axis}\NormalTok{(min}\OperatorTok{:} \DecValTok{0}\OperatorTok{,}\NormalTok{ max}\OperatorTok{:} \DecValTok{31}\OperatorTok{,}\NormalTok{ step}\OperatorTok{:} \DecValTok{1}\OperatorTok{,}\NormalTok{ location}\OperatorTok{:} \StringTok{"bottom"}\OperatorTok{,}\NormalTok{ show\_markings}\OperatorTok{:} \KeywordTok{false}\NormalTok{)}
\KeywordTok{let}\NormalTok{ y\_axis }\OperatorTok{=} \FunctionTok{axis}\NormalTok{(min}\OperatorTok{:} \DecValTok{0}\OperatorTok{,}\NormalTok{ max}\OperatorTok{:} \DecValTok{41}\OperatorTok{,}\NormalTok{ step}\OperatorTok{:} \DecValTok{5}\OperatorTok{,}\NormalTok{ location}\OperatorTok{:} \StringTok{"left"}\OperatorTok{,}\NormalTok{ helper\_lines}\OperatorTok{:} \KeywordTok{true}\NormalTok{)}

\CommentTok{// Classify the data}
\NormalTok{classes }\OperatorTok{=} \FunctionTok{classify}\NormalTok{(data}\OperatorTok{,}\NormalTok{ classes)}

\CommentTok{// Combine the axes and the data and feed it to the plot render function.}
\KeywordTok{let}\NormalTok{ pl }\OperatorTok{=} \FunctionTok{plot}\NormalTok{(axes}\OperatorTok{:}\NormalTok{ (x\_axis}\OperatorTok{,}\NormalTok{ y\_axis)}\OperatorTok{,}\NormalTok{ data}\OperatorTok{:}\NormalTok{ classes)}
\FunctionTok{histogram}\NormalTok{(pl}\OperatorTok{,}\NormalTok{ (}\DecValTok{100}\OperatorTok{\%,} \DecValTok{40}\OperatorTok{\%}\NormalTok{))}
\end{Highlighting}
\end{Shaded}

\pandocbounded{\includegraphics[keepaspectratio]{https://raw.githubusercontent.com/Pegacraft/typst-plotting/8d834689359b708ce75fe51be05eed45570e463e/images/histogram.png}}

\subsubsection{Bar charts}\label{bar-charts}

\begin{Shaded}
\begin{Highlighting}[]
\CommentTok{// Plot 1:}
\CommentTok{// The data to be displayed}
\KeywordTok{let}\NormalTok{ data }\OperatorTok{=}\NormalTok{ ((}\DecValTok{10}\OperatorTok{,} \StringTok{"Monday"}\NormalTok{)}\OperatorTok{,}\NormalTok{ (}\DecValTok{5}\OperatorTok{,} \StringTok{"Tuesday"}\NormalTok{)}\OperatorTok{,}\NormalTok{ (}\DecValTok{15}\OperatorTok{,} \StringTok{"Wednesday"}\NormalTok{)}\OperatorTok{,}\NormalTok{ (}\DecValTok{9}\OperatorTok{,} \StringTok{"Thursday"}\NormalTok{)}\OperatorTok{,}\NormalTok{ (}\DecValTok{11}\OperatorTok{,} \StringTok{"Friday"}\NormalTok{))}

\CommentTok{// Create the necessary axes}
\KeywordTok{let}\NormalTok{ y\_axis }\OperatorTok{=} \FunctionTok{axis}\NormalTok{(values}\OperatorTok{:}\NormalTok{ (}\StringTok{""}\OperatorTok{,} \StringTok{"Monday"}\OperatorTok{,} \StringTok{"Tuesday"}\OperatorTok{,} \StringTok{"Wednesday"}\OperatorTok{,} \StringTok{"Thursday"}\OperatorTok{,} \StringTok{"Friday"}\NormalTok{)}\OperatorTok{,}\NormalTok{ location}\OperatorTok{:} \StringTok{"left"}\OperatorTok{,}\NormalTok{ show\_markings}\OperatorTok{:} \KeywordTok{true}\NormalTok{)}
\KeywordTok{let}\NormalTok{ x\_axis }\OperatorTok{=} \FunctionTok{axis}\NormalTok{(min}\OperatorTok{:} \DecValTok{0}\OperatorTok{,}\NormalTok{ max}\OperatorTok{:} \DecValTok{20}\OperatorTok{,}\NormalTok{ step}\OperatorTok{:} \DecValTok{2}\OperatorTok{,}\NormalTok{ location}\OperatorTok{:} \StringTok{"bottom"}\OperatorTok{,}\NormalTok{ helper\_lines}\OperatorTok{:} \KeywordTok{true}\NormalTok{)}

\CommentTok{// Combine the axes and the data and feed it to the plot render function.}
\KeywordTok{let}\NormalTok{ pl }\OperatorTok{=} \FunctionTok{plot}\NormalTok{(axes}\OperatorTok{:}\NormalTok{ (x\_axis}\OperatorTok{,}\NormalTok{ y\_axis)}\OperatorTok{,}\NormalTok{ data}\OperatorTok{:}\NormalTok{ data)}
\FunctionTok{bar\_chart}\NormalTok{(pl}\OperatorTok{,}\NormalTok{ (}\DecValTok{100}\OperatorTok{\%,} \DecValTok{33}\OperatorTok{\%}\NormalTok{)}\OperatorTok{,}\NormalTok{ fill}\OperatorTok{:}\NormalTok{ (purple}\OperatorTok{,}\NormalTok{ blue}\OperatorTok{,}\NormalTok{ red}\OperatorTok{,}\NormalTok{ green}\OperatorTok{,}\NormalTok{ yellow)}\OperatorTok{,}\NormalTok{ bar\_width}\OperatorTok{:} \DecValTok{70}\OperatorTok{\%,}\NormalTok{ rotated}\OperatorTok{:} \KeywordTok{true}\NormalTok{)}

\CommentTok{// Plot 2:}
\CommentTok{// Same as above, but with numbers as data}
\KeywordTok{let}\NormalTok{ data\_2 }\OperatorTok{=}\NormalTok{ ((}\DecValTok{20}\OperatorTok{,} \DecValTok{2}\NormalTok{)}\OperatorTok{,}\NormalTok{ (}\DecValTok{30}\OperatorTok{,} \DecValTok{7}\NormalTok{)}\OperatorTok{,}\NormalTok{ (}\DecValTok{16}\OperatorTok{,} \DecValTok{12}\NormalTok{)}\OperatorTok{,}\NormalTok{ (}\DecValTok{40}\OperatorTok{,} \DecValTok{13}\NormalTok{)}\OperatorTok{,}\NormalTok{ (}\DecValTok{5}\OperatorTok{,} \DecValTok{17}\NormalTok{))}
\KeywordTok{let}\NormalTok{ y\_axis\_2 }\OperatorTok{=} \FunctionTok{axis}\NormalTok{(min}\OperatorTok{:} \DecValTok{0}\OperatorTok{,}\NormalTok{ max}\OperatorTok{:} \DecValTok{41}\OperatorTok{,}\NormalTok{ step}\OperatorTok{:} \DecValTok{5}\OperatorTok{,}\NormalTok{ location}\OperatorTok{:} \StringTok{"left"}\OperatorTok{,}\NormalTok{ show\_markings}\OperatorTok{:} \KeywordTok{true}\OperatorTok{,}\NormalTok{ helper\_lines}\OperatorTok{:} \KeywordTok{true}\NormalTok{)}
\KeywordTok{let}\NormalTok{ x\_axis\_2 }\OperatorTok{=} \FunctionTok{axis}\NormalTok{(min}\OperatorTok{:} \DecValTok{0}\OperatorTok{,}\NormalTok{ max}\OperatorTok{:} \DecValTok{21}\OperatorTok{,}\NormalTok{ step}\OperatorTok{:} \DecValTok{1}\OperatorTok{,}\NormalTok{ location}\OperatorTok{:} \StringTok{"bottom"}\NormalTok{)}
\KeywordTok{let}\NormalTok{ pl\_2 }\OperatorTok{=} \FunctionTok{plot}\NormalTok{(axes}\OperatorTok{:}\NormalTok{ (x\_axis\_2}\OperatorTok{,}\NormalTok{ y\_axis\_2)}\OperatorTok{,}\NormalTok{ data}\OperatorTok{:}\NormalTok{ data\_2)}
\FunctionTok{bar\_chart}\NormalTok{(pl\_2}\OperatorTok{,}\NormalTok{ (}\DecValTok{100}\OperatorTok{\%,} \DecValTok{60}\OperatorTok{\%}\NormalTok{)}\OperatorTok{,}\NormalTok{ bar\_width}\OperatorTok{:} \DecValTok{100}\OperatorTok{\%}\NormalTok{)}
\end{Highlighting}
\end{Shaded}

\pandocbounded{\includegraphics[keepaspectratio]{https://raw.githubusercontent.com/Pegacraft/typst-plotting/8d834689359b708ce75fe51be05eed45570e463e/images/bar.png}}

\subsubsection{Pie charts}\label{pie-charts}

\begin{Shaded}
\begin{Highlighting}[]
\NormalTok{show}\OperatorTok{:}\NormalTok{ r }\KeywordTok{=\textgreater{}} \FunctionTok{columns}\NormalTok{(}\DecValTok{2}\OperatorTok{,}\NormalTok{ r)}

\CommentTok{// create the sample data}
\KeywordTok{let}\NormalTok{ data }\OperatorTok{=}\NormalTok{ ((}\DecValTok{10}\OperatorTok{,} \StringTok{"Male"}\NormalTok{)}\OperatorTok{,}\NormalTok{ (}\DecValTok{20}\OperatorTok{,} \StringTok{"Female"}\NormalTok{)}\OperatorTok{,}\NormalTok{ (}\DecValTok{15}\OperatorTok{,} \StringTok{"Divers"}\NormalTok{)}\OperatorTok{,}\NormalTok{ (}\DecValTok{2}\OperatorTok{,} \StringTok{"Other"}\NormalTok{)}

\CommentTok{// Skip the axis step, as no axes are needed}

\CommentTok{// Put the data into a plot }
\KeywordTok{let}\NormalTok{ p }\OperatorTok{=} \FunctionTok{plot}\NormalTok{(data}\OperatorTok{:}\NormalTok{ data)}

\CommentTok{// Display the pie\_charts in all different display ways}
\FunctionTok{pie\_chart}\NormalTok{(p}\OperatorTok{,}\NormalTok{ (}\DecValTok{100}\OperatorTok{\%,} \DecValTok{20}\OperatorTok{\%}\NormalTok{)}\OperatorTok{,}\NormalTok{ display\_style}\OperatorTok{:} \StringTok{"legend{-}inside{-}chart"}\NormalTok{)}
\FunctionTok{pie\_chart}\NormalTok{(p}\OperatorTok{,}\NormalTok{ (}\DecValTok{100}\OperatorTok{\%,} \DecValTok{20}\OperatorTok{\%}\NormalTok{)}\OperatorTok{,}\NormalTok{ display\_style}\OperatorTok{:} \StringTok{"hor{-}chart{-}legend"}\NormalTok{)}
\FunctionTok{pie\_chart}\NormalTok{(p}\OperatorTok{,}\NormalTok{ (}\DecValTok{100}\OperatorTok{\%,} \DecValTok{20}\OperatorTok{\%}\NormalTok{)}\OperatorTok{,}\NormalTok{ display\_style}\OperatorTok{:} \StringTok{"hor{-}legend{-}chart"}\NormalTok{)}
\FunctionTok{pie\_chart}\NormalTok{(p}\OperatorTok{,}\NormalTok{ (}\DecValTok{100}\OperatorTok{\%,} \DecValTok{20}\OperatorTok{\%}\NormalTok{)}\OperatorTok{,}\NormalTok{ display\_style}\OperatorTok{:} \StringTok{"vert{-}chart{-}legend"}\NormalTok{)}
\FunctionTok{pie\_chart}\NormalTok{(p}\OperatorTok{,}\NormalTok{ (}\DecValTok{100}\OperatorTok{\%,} \DecValTok{20}\OperatorTok{\%}\NormalTok{)}\OperatorTok{,}\NormalTok{ display\_style}\OperatorTok{:} \StringTok{"vert{-}legend{-}chart"}\NormalTok{)}
\end{Highlighting}
\end{Shaded}

\pandocbounded{\includegraphics[keepaspectratio]{https://raw.githubusercontent.com/Pegacraft/typst-plotting/8d834689359b708ce75fe51be05eed45570e463e/images/pie.png}}

\textbf{Overlayed Graphs}

\begin{Shaded}
\begin{Highlighting}[]
\CommentTok{// Create the data for the two plots to overlay}
\KeywordTok{let}\NormalTok{ data\_scatter }\OperatorTok{=}\NormalTok{ (}
\NormalTok{  (}\DecValTok{0}\OperatorTok{,} \DecValTok{0}\NormalTok{)}\OperatorTok{,}\NormalTok{ (}\DecValTok{2}\OperatorTok{,} \DecValTok{2}\NormalTok{)}\OperatorTok{,}\NormalTok{ (}\DecValTok{3}\OperatorTok{,} \DecValTok{0}\NormalTok{)}\OperatorTok{,}\NormalTok{ (}\DecValTok{4}\OperatorTok{,} \DecValTok{4}\NormalTok{)}\OperatorTok{,}\NormalTok{ (}\DecValTok{5}\OperatorTok{,} \DecValTok{7}\NormalTok{)}\OperatorTok{,}\NormalTok{ (}\DecValTok{6}\OperatorTok{,} \DecValTok{6}\NormalTok{)}\OperatorTok{,}\NormalTok{ (}\DecValTok{7}\OperatorTok{,} \DecValTok{9}\NormalTok{)}\OperatorTok{,}\NormalTok{ (}\DecValTok{8}\OperatorTok{,} \DecValTok{5}\NormalTok{)}\OperatorTok{,}\NormalTok{ (}\DecValTok{9}\OperatorTok{,} \DecValTok{9}\NormalTok{)}\OperatorTok{,}\NormalTok{ (}\DecValTok{10}\OperatorTok{,} \DecValTok{1}\NormalTok{)}
\NormalTok{)}
\KeywordTok{let}\NormalTok{ data\_graph }\OperatorTok{=}\NormalTok{ (}
\NormalTok{  (}\DecValTok{0}\OperatorTok{,} \DecValTok{3}\NormalTok{)}\OperatorTok{,}\NormalTok{ (}\DecValTok{1}\OperatorTok{,} \DecValTok{5}\NormalTok{)}\OperatorTok{,}\NormalTok{ (}\DecValTok{2}\OperatorTok{,} \DecValTok{1}\NormalTok{)}\OperatorTok{,}\NormalTok{ (}\DecValTok{3}\OperatorTok{,} \DecValTok{7}\NormalTok{)}\OperatorTok{,}\NormalTok{ (}\DecValTok{4}\OperatorTok{,} \DecValTok{3}\NormalTok{)}\OperatorTok{,}\NormalTok{ (}\DecValTok{5}\OperatorTok{,} \DecValTok{5}\NormalTok{)}\OperatorTok{,}\NormalTok{ (}\DecValTok{6}\OperatorTok{,} \DecValTok{7}\NormalTok{)}\OperatorTok{,}\NormalTok{(}\DecValTok{7}\OperatorTok{,} \DecValTok{4}\NormalTok{)}\OperatorTok{,}\NormalTok{(}\DecValTok{11}\OperatorTok{,} \DecValTok{6}\NormalTok{)}
\NormalTok{)}

\CommentTok{// Create the axes for the overlay plot}
\KeywordTok{let}\NormalTok{ x\_axis }\OperatorTok{=} \FunctionTok{axis}\NormalTok{(min}\OperatorTok{:} \DecValTok{0}\OperatorTok{,}\NormalTok{ max}\OperatorTok{:} \DecValTok{11}\OperatorTok{,}\NormalTok{ step}\OperatorTok{:} \DecValTok{2}\OperatorTok{,}\NormalTok{ location}\OperatorTok{:} \StringTok{"bottom"}\NormalTok{)}
\KeywordTok{let}\NormalTok{ y\_axis }\OperatorTok{=} \FunctionTok{axis}\NormalTok{(min}\OperatorTok{:} \DecValTok{0}\OperatorTok{,}\NormalTok{ max}\OperatorTok{:} \DecValTok{11}\OperatorTok{,}\NormalTok{ step}\OperatorTok{:} \DecValTok{2}\OperatorTok{,}\NormalTok{ location}\OperatorTok{:} \StringTok{"left"}\OperatorTok{,}\NormalTok{ helper\_lines}\OperatorTok{:} \KeywordTok{false}\NormalTok{)}

\CommentTok{// create a plot for each individual plot type and save the render call}
\KeywordTok{let}\NormalTok{ pl\_scatter }\OperatorTok{=} \FunctionTok{plot}\NormalTok{(data}\OperatorTok{:}\NormalTok{ data\_scatter}\OperatorTok{,}\NormalTok{ axes}\OperatorTok{:}\NormalTok{ (x\_axis}\OperatorTok{,}\NormalTok{ y\_axis))}
\KeywordTok{let}\NormalTok{ scatter\_display }\OperatorTok{=} \FunctionTok{scatter\_plot}\NormalTok{(pl\_scatter}\OperatorTok{,}\NormalTok{ (}\DecValTok{100}\OperatorTok{\%,} \DecValTok{25}\OperatorTok{\%}\NormalTok{)}\OperatorTok{,}\NormalTok{ stroke}\OperatorTok{:}\NormalTok{ red)}
\KeywordTok{let}\NormalTok{ pl\_graph }\OperatorTok{=} \FunctionTok{plot}\NormalTok{(data}\OperatorTok{:}\NormalTok{ data\_graph}\OperatorTok{,}\NormalTok{ axes}\OperatorTok{:}\NormalTok{ (x\_axis}\OperatorTok{,}\NormalTok{ y\_axis))}
\KeywordTok{let}\NormalTok{ graph\_display }\OperatorTok{=} \FunctionTok{graph\_plot}\NormalTok{(pl\_graph}\OperatorTok{,}\NormalTok{ (}\DecValTok{100}\OperatorTok{\%,} \DecValTok{25}\OperatorTok{\%}\NormalTok{)}\OperatorTok{,}\NormalTok{ stroke}\OperatorTok{:}\NormalTok{ blue)}

\CommentTok{// overlay the plots using the overlay function}
\FunctionTok{overlay}\NormalTok{((scatter\_display}\OperatorTok{,}\NormalTok{ graph\_display)}\OperatorTok{,}\NormalTok{ (}\DecValTok{100}\OperatorTok{\%,} \DecValTok{25}\OperatorTok{\%}\NormalTok{))}
\end{Highlighting}
\end{Shaded}

\pandocbounded{\includegraphics[keepaspectratio]{https://raw.githubusercontent.com/Pegacraft/typst-plotting/8d834689359b708ce75fe51be05eed45570e463e/images/overlay.png}}

\subsubsection{How to add}\label{how-to-add}

Copy this into your project and use the import as \texttt{\ plotst\ }

\begin{verbatim}
#import "@preview/plotst:0.2.0"
\end{verbatim}

\includesvg[width=0.16667in,height=0.16667in]{/assets/icons/16-copy.svg}

Check the docs for
\href{https://typst.app/docs/reference/scripting/\#packages}{more
information on how to import packages} .

\subsubsection{About}\label{about}

\begin{description}
\tightlist
\item[Author s :]
Pegacraft \& Gewi413
\item[License:]
MIT
\item[Current version:]
0.2.0
\item[Last updated:]
October 28, 2023
\item[First released:]
July 2, 2023
\item[Archive size:]
15.2 kB
\href{https://packages.typst.org/preview/plotst-0.2.0.tar.gz}{\pandocbounded{\includesvg[keepaspectratio]{/assets/icons/16-download.svg}}}
\item[Repository:]
\href{https://github.com/Pegacraft/typst-plotting}{GitHub}
\end{description}

\subsubsection{Where to report issues?}\label{where-to-report-issues}

This package is a project of Pegacraft and Gewi413 . Report issues on
\href{https://github.com/Pegacraft/typst-plotting}{their repository} .
You can also try to ask for help with this package on the
\href{https://forum.typst.app}{Forum} .

Please report this package to the Typst team using the
\href{https://typst.app/contact}{contact form} if you believe it is a
safety hazard or infringes upon your rights.

\phantomsection\label{versions}
\subsubsection{Version history}\label{version-history}

\begin{longtable}[]{@{}ll@{}}
\toprule\noalign{}
Version & Release Date \\
\midrule\noalign{}
\endhead
\bottomrule\noalign{}
\endlastfoot
0.2.0 & October 28, 2023 \\
\href{https://typst.app/universe/package/plotst/0.1.0/}{0.1.0} & July 2,
2023 \\
\end{longtable}

Typst GmbH did not create this package and cannot guarantee correct
functionality of this package or compatibility with any version of the
Typst compiler or app.


\title{typst.app/universe/package/modernpro-cv}

\phantomsection\label{banner}
\phantomsection\label{template-thumbnail}
\pandocbounded{\includegraphics[keepaspectratio]{https://packages.typst.org/preview/thumbnails/modernpro-cv-1.0.2-small.webp}}

\section{modernpro-cv}\label{modernpro-cv}

{ 1.0.2 }

A CV template inspired by Deedy-Resume.

\href{/app?template=modernpro-cv&version=1.0.2}{Create project in app}

\phantomsection\label{readme}
This Typst CV template is inspired by the Latex template
\href{https://github.com/deedy/Deedy-Resume}{Deedy-Resume} . You can use
it for both industry and academia.

If you want to find a cover letter template, you can check out
\href{https://github.com/jxpeng98/typst-coverletter}{modernpro-coverletter}
.

\subsection{How to start}\label{how-to-start}

\subsubsection{Use Typst CLI}\label{use-typst-cli}

If you use Typst CLI, you can use the following command to create a new
project:

\begin{Shaded}
\begin{Highlighting}[]
\ExtensionTok{typst}\NormalTok{ init modernpro{-}cv}
\end{Highlighting}
\end{Shaded}

It will create a folder named \texttt{\ modernpro-cv\ } with the
following structure:

\begin{Shaded}
\begin{Highlighting}[]
\NormalTok{modernpro{-}cv}
\NormalTok{├── bib.bib}
\NormalTok{├── cv\_double.typ}
\NormalTok{└── cv\_single.typ}
\end{Highlighting}
\end{Shaded}

If you want to use the single-column version, you can modify the
template \texttt{\ cv-single.typ\ } . If you prefer the two-column
version, you can use the \texttt{\ cv-double.typ\ } .

\textbf{Note:} The \texttt{\ bib.bib\ } is the bibliography file. You
can modify it to add your publications.

\subsubsection{Manual Download}\label{manual-download}

If you want to manually download the template, you can download
\texttt{\ modernpro-cv-\{version\}.zip\ } from the
\href{https://github.com/jxpeng98/Typst-CV-Resume/releases}{release
page}

\subsubsection{Typst website}\label{typst-website}

If you want to use the template via \href{https://typst.app/}{Typst} ,
You can \texttt{\ start\ from\ template\ } and search for
\texttt{\ modernpro-cv\ } .

\subsection{How to use the template}\label{how-to-use-the-template}

\subsubsection{The arguments}\label{the-arguments}

The template has the following arguments:

\begin{longtable}[]{@{}lll@{}}
\toprule\noalign{}
Argument & Description & Default \\
\midrule\noalign{}
\endhead
\bottomrule\noalign{}
\endlastfoot
\texttt{\ font-type\ } & The font type. You can choose any supported
font in your system. & \texttt{\ Times\ New\ Roman\ } \\
\texttt{\ continue-header\ } & Whether to continue the header on the
follwing pages. & \texttt{\ false\ } \\
\texttt{\ name\ } & Your name. & \texttt{\ ""\ } \\
\texttt{\ address\ } & Your address. & \texttt{\ ""\ } \\
\texttt{\ lastupdated\ } & Whether to show the last updated date. &
\texttt{\ true\ } \\
\texttt{\ pagecount\ } & Whether to show the page count. &
\texttt{\ true\ } \\
\texttt{\ date\ } & The date of the CV. & \texttt{\ today\ } \\
\texttt{\ contacts\ } & contact details, e.g phone number, email, etc. &
\texttt{\ (text:\ "",\ link:\ "")\ } \\
\end{longtable}

\subsubsection{Start single column
version}\label{start-single-column-version}

If you want to use the single column version, you create a new
\texttt{\ .typ\ } file and copy the following code:

\begin{Shaded}
\begin{Highlighting}[]
\NormalTok{\#import "@preview/modernpro{-}cv:1.0.2": *}
\NormalTok{\#import "@preview/fontawesome:0.5.0": *}

\NormalTok{\#show: cv{-}single.with(}
\NormalTok{  font{-}type: "PT Serif",}
\NormalTok{  continue{-}header: "false",}
\NormalTok{  name: [],}
\NormalTok{  address: [],}
\NormalTok{  lastupdated: "true",}
\NormalTok{  pagecount: "true",}
\NormalTok{  date: "2024{-}07{-}03",}
\NormalTok{  contacts: (}
\NormalTok{    (text: [\#fa{-}icon("location{-}dot") UK]),}
\NormalTok{    (text: [\#fa{-}icon("mobile") 123{-}456{-}789], link: "tel:123{-}456{-}789"),}
\NormalTok{    (text: [\#fa{-}icon("link") example.com], link: "https://www.example.com"),}
\NormalTok{  )}
\NormalTok{)}
\end{Highlighting}
\end{Shaded}

\subsubsection{Start double column
version}\label{start-double-column-version}

The double column version is similar to the single column version.
However, you need to add contents to the specific \texttt{\ left\ } and
\texttt{\ right\ } sections.

\begin{Shaded}
\begin{Highlighting}[]
\NormalTok{\#import "@preview/modernpro{-}cv:1.0.2": *}
\NormalTok{\#import "@preview/fontawesome:0.5.0": *}

\NormalTok{\#show: cv{-}double(}
\NormalTok{  font{-}type: "PT Sans",}
\NormalTok{  continue{-}header: "true",}
\NormalTok{  name: [\#lorem(2)],}
\NormalTok{  address: [\#lorem(4)],}
\NormalTok{  lastupdated: "true",}
\NormalTok{  pagecount: "true",}
\NormalTok{  date: "2024{-}07{-}03",}
\NormalTok{  contacts: (}
\NormalTok{    (text: [\#fa{-}icon("location{-}dot") UK]),}
\NormalTok{    (text: [\#fa{-}icon("mobile") 123{-}456{-}789], link: "tel:123{-}456{-}789"),}
\NormalTok{    (text: [\#fa{-}icon("link") example.com], link: "https://www.example.com"),}
\NormalTok{  ),}
\NormalTok{  left: [}
\NormalTok{    // contents for the left column}
\NormalTok{  ],}
\NormalTok{  right:[}
\NormalTok{    // contents for the right column}
\NormalTok{  ]}
\NormalTok{)}
\end{Highlighting}
\end{Shaded}

\subsubsection{Start the CV}\label{start-the-cv}

Once you set up the arguments, you can start to add details to your CV /
Resume.

I preset the following functions for you to create different parts:

\begin{longtable}[]{@{}ll@{}}
\toprule\noalign{}
Function & Description \\
\midrule\noalign{}
\endhead
\bottomrule\noalign{}
\endlastfoot
\texttt{\ \#section("Section\ Name")\ } & Start a new section \\
\texttt{\ \#sectionsep\ } & End the section \\
\texttt{\ \#oneline-title-item(title:\ "",\ content:\ "")\ } & Add a
one-line item ( \textbf{Title:} content) \\
\texttt{\ \#oneline-two(entry1:\ "",\ entry2:\ "")\ } & Add a one-line
item with two entries, aligned left and right \\
\texttt{\ \#descript("descriptions")\ } & Add a description for
self-introduction \\
\texttt{\ \#award(award:\ "",\ date:\ "",\ institution:\ "")\ } & Add an
award ( \textbf{award} , \emph{institution} \emph{date} ) \\
\texttt{\ \#education(institution:\ "",\ major:\ "",\ date:\ "",\ institution:\ "",\ core-modules:\ "")\ }
& Add an education experience \\
\texttt{\ \#job(position:\ "",\ institution:\ "",\ location:\ "",\ date:\ "",\ description:\ {[}{]})\ }
& Add a job experience (description is optional) \\
\texttt{\ \#twoline-item(entry1:\ "",\ entry2:\ "",\ entry3:\ "",\ entry4:\ "")\ }
& Two line items, similar to education and job experiences \\
\texttt{\ \#references(references:())\ } & Add a reference list. In the
\texttt{\ ()\ } , you can add multi reference entries with the following
format
\texttt{\ (name:\ "",\ position:\ "",\ department:\ "",\ institution:\ "",\ address:\ "",\ email:\ "",),\ } \\
\texttt{\ \#show\ bibliography:\ none\ \#bibliography("bib.bib")\ } &
Add a bibliography. You can modify the \texttt{\ bib.bib\ } file to add
your publications. \textbf{Note:} Keep this at the end of your CV \\
\end{longtable}

\textbf{Note:} Use \texttt{\ +\ @ref\ } to display your publications.
For example,

\begin{Shaded}
\begin{Highlighting}[]
\NormalTok{\#section("Publications")}

\NormalTok{// numbering list }
\NormalTok{+ @quenouille1949approximate}
\NormalTok{+ @quenouille1949approximate}

\NormalTok{// Keep this at the end}
\NormalTok{\#show bibliography: none}
\NormalTok{\#bibliography("bib.bib")}
\end{Highlighting}
\end{Shaded}

\subsection{Preview}\label{preview}

\subsubsection{Single Column}\label{single-column}

\pandocbounded{\includegraphics[keepaspectratio]{https://minioapi.pjx.ac.cn/img1/2024/07/a81ac7ec96be0625eefccb81ead160d3.png}}

\subsubsection{Double Column}\label{double-column}

\pandocbounded{\includegraphics[keepaspectratio]{https://minioapi.pjx.ac.cn/img1/2024/07/12e9b31e306055f615edf49f9b8ffe55.png}}

\subsection{Legacy Version}\label{legacy-version}

I redesigned the template and submitted the new version to Typst
Universe. However, you can find the legacy version in the
\texttt{\ legacy\ } folder if you prefer to use the multi-font setting.
You can also download the \texttt{\ modernpro-cv-legacy.zip\ } from the
\href{https://github.com/jxpeng98/Typst-CV-Resume/releases}{release
page} .

\textbf{Note:} The legacy version also has a cover letter template. You
can use it with the CV template.

\subsection{Cover Letter}\label{cover-letter}

If you used the previous version of this template, you might know that I
also provided a cover letter template.

If you want to use a consistent cover letter with the new version of the
CV template, you can find it from another repository
\href{https://github.com/jxpeng98/typst-coverletter}{typst-coverletter}
.

you can also use the following code in the command line:

\begin{Shaded}
\begin{Highlighting}[]
\ExtensionTok{typst}\NormalTok{ init modernpro{-}coverletter}
\end{Highlighting}
\end{Shaded}

\subsection{License}\label{license}

The template is released under the MIT License. For more information,
please refer to the
\href{https://github.com/jxpeng98/Typst-CV-Resume/blob/main/LICENSE}{LICENSE}
file.

\href{/app?template=modernpro-cv&version=1.0.2}{Create project in app}

\subsubsection{How to use}\label{how-to-use}

Click the button above to create a new project using this template in
the Typst app.

You can also use the Typst CLI to start a new project on your computer
using this command:

\begin{verbatim}
typst init @preview/modernpro-cv:1.0.2
\end{verbatim}

\includesvg[width=0.16667in,height=0.16667in]{/assets/icons/16-copy.svg}

\subsubsection{About}\label{about}

\begin{description}
\tightlist
\item[Author :]
jxpeng98
\item[License:]
MIT
\item[Current version:]
1.0.2
\item[Last updated:]
October 22, 2024
\item[First released:]
August 7, 2024
\item[Archive size:]
5.70 kB
\href{https://packages.typst.org/preview/modernpro-cv-1.0.2.tar.gz}{\pandocbounded{\includesvg[keepaspectratio]{/assets/icons/16-download.svg}}}
\item[Repository:]
\href{https://github.com/jxpeng98/Typst-CV-Resume}{GitHub}
\item[Categor y :]
\begin{itemize}
\tightlist
\item[]
\item
  \pandocbounded{\includesvg[keepaspectratio]{/assets/icons/16-user.svg}}
  \href{https://typst.app/universe/search/?category=cv}{CV}
\end{itemize}
\end{description}

\subsubsection{Where to report issues?}\label{where-to-report-issues}

This template is a project of jxpeng98 . Report issues on
\href{https://github.com/jxpeng98/Typst-CV-Resume}{their repository} .
You can also try to ask for help with this template on the
\href{https://forum.typst.app}{Forum} .

Please report this template to the Typst team using the
\href{https://typst.app/contact}{contact form} if you believe it is a
safety hazard or infringes upon your rights.

\phantomsection\label{versions}
\subsubsection{Version history}\label{version-history}

\begin{longtable}[]{@{}ll@{}}
\toprule\noalign{}
Version & Release Date \\
\midrule\noalign{}
\endhead
\bottomrule\noalign{}
\endlastfoot
1.0.2 & October 22, 2024 \\
\href{https://typst.app/universe/package/modernpro-cv/1.0.1/}{1.0.1} &
August 30, 2024 \\
\href{https://typst.app/universe/package/modernpro-cv/1.0.0/}{1.0.0} &
August 7, 2024 \\
\end{longtable}

Typst GmbH did not create this template and cannot guarantee correct
functionality of this template or compatibility with any version of the
Typst compiler or app.


\title{typst.app/universe/package/statastic}

\phantomsection\label{banner}
\section{statastic}\label{statastic}

{ 1.0.0 }

A library to calculate statistics for numerical data

\phantomsection\label{readme}
A library to calculate statistics for numerical data in typst.

\subsection{Description}\label{description}

\texttt{\ Statastic\ } is a Typst library designed to provide various
statistical functions for numerical data. It offers functionalities like
extracting specific columns from datasets, converting array elements to
different data types, and computing various statistical measures such as
average, median, mode, variance, standard deviation, and percentiles.

\subsection{Features}\label{features}

\begin{itemize}
\tightlist
\item
  \textbf{Extract Column} : Extracts a specific column from a given
  dataset.
\item
  \textbf{Type Conversion} : Convert array elements to floating point
  numbers or integers.
\item
  \textbf{Statistical Measures} : Calculate average, median, mode,
  variance, standard deviation, and specific percentiles for an array or
  a specific column in a dataset.
\end{itemize}

\subsection{Usage}\label{usage}

To use the package you can import it through this command
\texttt{\ import\ "@preview/statastical:1.0.0":\ *\ } (as soon as the
pull request ist accepted). The documentation is found in the
\texttt{\ docs.pdf\ } in the development
\href{https://github.com/dikkadev/typst-statastic}{repo}

\subsection{License}\label{license}

This project is licensed under the Unlicense.

\subsubsection{How to add}\label{how-to-add}

Copy this into your project and use the import as \texttt{\ statastic\ }

\begin{verbatim}
#import "@preview/statastic:1.0.0"
\end{verbatim}

\includesvg[width=0.16667in,height=0.16667in]{/assets/icons/16-copy.svg}

Check the docs for
\href{https://typst.app/docs/reference/scripting/\#packages}{more
information on how to import packages} .

\subsubsection{About}\label{about}

\begin{description}
\tightlist
\item[Author :]
dikkadev
\item[License:]
Unlicense
\item[Current version:]
1.0.0
\item[Last updated:]
October 24, 2024
\item[First released:]
September 3, 2023
\item[Archive size:]
4.27 kB
\href{https://packages.typst.org/preview/statastic-1.0.0.tar.gz}{\pandocbounded{\includesvg[keepaspectratio]{/assets/icons/16-download.svg}}}
\item[Repository:]
\href{https://github.com/dikkadev/typst-statastic}{GitHub}
\end{description}

\subsubsection{Where to report issues?}\label{where-to-report-issues}

This package is a project of dikkadev . Report issues on
\href{https://github.com/dikkadev/typst-statastic}{their repository} .
You can also try to ask for help with this package on the
\href{https://forum.typst.app}{Forum} .

Please report this package to the Typst team using the
\href{https://typst.app/contact}{contact form} if you believe it is a
safety hazard or infringes upon your rights.

\phantomsection\label{versions}
\subsubsection{Version history}\label{version-history}

\begin{longtable}[]{@{}ll@{}}
\toprule\noalign{}
Version & Release Date \\
\midrule\noalign{}
\endhead
\bottomrule\noalign{}
\endlastfoot
1.0.0 & October 24, 2024 \\
\href{https://typst.app/universe/package/statastic/0.1.0/}{0.1.0} &
September 3, 2023 \\
\end{longtable}

Typst GmbH did not create this package and cannot guarantee correct
functionality of this package or compatibility with any version of the
Typst compiler or app.


\title{typst.app/universe/package/slydst}

\phantomsection\label{banner}
\phantomsection\label{template-thumbnail}
\pandocbounded{\includegraphics[keepaspectratio]{https://packages.typst.org/preview/thumbnails/slydst-0.1.3-small.webp}}

\section{slydst}\label{slydst}

{ 0.1.3 }

Create simple static slides using standard headings

\href{/app?template=slydst&version=0.1.3}{Create project in app}

\phantomsection\label{readme}
Create simple static slides with Typst.

Slydst allows the creation of slides using Typst headings. This
simplicity comes at the expense of dynamic content such as subslide
animations. For more complete and complex slides functionalities, see
other tools such as Polylux.

See the
\href{https://github.com/typst/packages/raw/main/packages/preview/slydst/0.1.3/\#example}{preview}
below.

\subsection{Usage}\label{usage}

To start, just use the following preamble (only the title is required).

\begin{Shaded}
\begin{Highlighting}[]
\NormalTok{\#import "@preview/slydst:0.1.3": *}

\NormalTok{\#show: slides.with(}
\NormalTok{  title: "Slydst: Slides with Typst",}
\NormalTok{  subtitle: none,}
\NormalTok{  date: none,}
\NormalTok{  authors: ("Gaspard Lambrechts",),}
\NormalTok{  layout: "medium",}
\NormalTok{  ratio: 4/3,}
\NormalTok{  title{-}color: none,}
\NormalTok{)}

\NormalTok{Insert your content here.}
\end{Highlighting}
\end{Shaded}

Then, insert your content.

\begin{itemize}
\tightlist
\item
  \textbf{Level-one headings} corresponds to new sections.
\item
  \textbf{Level-two headings} corresponds to new slides.
\item
  Blank space can be filled with \textbf{vertical spaces} like
  \texttt{\ \#v(1fr)\ } .
\end{itemize}

\begin{Shaded}
\begin{Highlighting}[]
\NormalTok{== Outline}

\NormalTok{\#outline()}

\NormalTok{= First section}

\NormalTok{== First slide}

\NormalTok{\#figure(image("figure.png", width: 60\%), caption: "Caption")}

\NormalTok{\#v(1fr)}

\NormalTok{\#lorem(20)}
\end{Highlighting}
\end{Shaded}

\subsection{Title page}\label{title-page}

Alternatively, you can omit the title argument and write your own title
page. Note that the subtitle, date and authors arguments be ignored in
that case.

\begin{Shaded}
\begin{Highlighting}[]
\NormalTok{\#show: slides.with(}
\NormalTok{  layout: "medium",}
\NormalTok{)}

\NormalTok{\#align(center + horizon)[}
\NormalTok{  \#text(2em, default{-}color)[*Slydst: Slides in Typst*]}
\NormalTok{]}

\NormalTok{Insert your content here.}
\end{Highlighting}
\end{Shaded}

We advise the use of the \texttt{\ title-slide\ } function that ensures
a proper centering and no page numbering.

\begin{Shaded}
\begin{Highlighting}[]
\NormalTok{\#show: slides}

\NormalTok{\#title{-}slide(layout: "medium")[}
\NormalTok{  \#text(2em, default{-}color)[*Slydst: Slides in Typst*]}
\NormalTok{]}

\NormalTok{Insert your content here.}
\end{Highlighting}
\end{Shaded}

\subsection{Components}\label{components}

Definitions, theorems, lemmas, corollaries and algorithms boxes are also
available.

\begin{Shaded}
\begin{Highlighting}[]
\NormalTok{\#definition(title: "An interesting definition")[}
\NormalTok{  \#lorem(20)}
\NormalTok{]}
\end{Highlighting}
\end{Shaded}

\subsection{Documentation}\label{documentation}

\subsubsection{\texorpdfstring{\texttt{\ slides\ }}{ slides }}\label{slides}

\begin{itemize}
\tightlist
\item
  \texttt{\ content\ } : \texttt{\ content\ } - content of the
  presentation
\item
  \texttt{\ title\ } : \texttt{\ str\ } - title (required)
\item
  \texttt{\ subtitle\ } : \texttt{\ str\ } - subtitle
\item
  \texttt{\ date\ } : \texttt{\ str\ } - date
\item
  \texttt{\ authors\ } : \texttt{\ array\ } of \texttt{\ content\ } or
  \texttt{\ content\ } - list of authors or author content
\item
  \texttt{\ layout\ } :
  \texttt{\ str\ in\ ("small",\ "medium",\ "large")\ } - layout
  selection
\item
  \texttt{\ ratio\ } : \texttt{\ float\ } or \texttt{\ ratio\ } or
  \texttt{\ int\ } - width to height ratio
\item
  \texttt{\ title-color\ } : \texttt{\ color\ } or \texttt{\ gradient\ }
  - color of title and headings
\end{itemize}

\subsubsection{\texorpdfstring{\texttt{\ title-slide\ }}{ title-slide }}\label{title-slide}

\begin{itemize}
\tightlist
\item
  \texttt{\ content\ } : \texttt{\ content\ } - content of the slide
\end{itemize}

\subsubsection{\texorpdfstring{\texttt{\ definition\ } ,
\texttt{\ theorem\ } , \texttt{\ lemma\ } , \texttt{\ corollary\ } ,
\texttt{\ algorithm\ }}{ definition  ,  theorem  ,  lemma  ,  corollary  ,  algorithm }}\label{definition-theorem-lemma-corollary-algorithm}

\begin{itemize}
\tightlist
\item
  \texttt{\ content\ } : \texttt{\ content\ } - content of the block
\item
  \texttt{\ title\ } : \texttt{\ str\ } - title of the block
\item
  \texttt{\ fill-header\ } : \texttt{\ color\ } - color of the header
  (inferred if only \texttt{\ fill-body\ } is specified)
\item
  \texttt{\ fill-body\ } : \texttt{\ color\ } - color of the body
  (inferred if only \texttt{\ fill-header\ } is specified)
\item
  \texttt{\ radius\ } : \texttt{\ length\ } - radius of the corners of
  the block
\end{itemize}

\subsection{Example}\label{example}

{
\includesvg[width=3.125in,height=\textheight,keepaspectratio]{https://github.com/typst/packages/raw/main/packages/preview/slydst/0.1.3/svg/example-01.svg}
} {
\includesvg[width=3.125in,height=\textheight,keepaspectratio]{https://github.com/typst/packages/raw/main/packages/preview/slydst/0.1.3/svg/example-02.svg}
} {
\includesvg[width=3.125in,height=\textheight,keepaspectratio]{https://github.com/typst/packages/raw/main/packages/preview/slydst/0.1.3/svg/example-03.svg}
} {
\includesvg[width=3.125in,height=\textheight,keepaspectratio]{https://github.com/typst/packages/raw/main/packages/preview/slydst/0.1.3/svg/example-04.svg}
} {
\includesvg[width=3.125in,height=\textheight,keepaspectratio]{https://github.com/typst/packages/raw/main/packages/preview/slydst/0.1.3/svg/example-05.svg}
} {
\includesvg[width=3.125in,height=\textheight,keepaspectratio]{https://github.com/typst/packages/raw/main/packages/preview/slydst/0.1.3/svg/example-06.svg}
} {
\includesvg[width=3.125in,height=\textheight,keepaspectratio]{https://github.com/typst/packages/raw/main/packages/preview/slydst/0.1.3/svg/example-07.svg}
} {
\includesvg[width=3.125in,height=\textheight,keepaspectratio]{https://github.com/typst/packages/raw/main/packages/preview/slydst/0.1.3/svg/example-08.svg}
} {
\includesvg[width=3.125in,height=\textheight,keepaspectratio]{https://github.com/typst/packages/raw/main/packages/preview/slydst/0.1.3/svg/example-09.svg}
} {
\includesvg[width=3.125in,height=\textheight,keepaspectratio]{https://github.com/typst/packages/raw/main/packages/preview/slydst/0.1.3/svg/example-10.svg}
}

\href{/app?template=slydst&version=0.1.3}{Create project in app}

\subsubsection{How to use}\label{how-to-use}

Click the button above to create a new project using this template in
the Typst app.

You can also use the Typst CLI to start a new project on your computer
using this command:

\begin{verbatim}
typst init @preview/slydst:0.1.3
\end{verbatim}

\includesvg[width=0.16667in,height=0.16667in]{/assets/icons/16-copy.svg}

\subsubsection{About}\label{about}

\begin{description}
\tightlist
\item[Author :]
\href{https://github.com/glambrechts}{Gaspard Lambrechts}
\item[License:]
MIT
\item[Current version:]
0.1.3
\item[Last updated:]
November 12, 2024
\item[First released:]
November 18, 2023
\item[Minimum Typst version:]
0.12.0
\item[Archive size:]
4.12 kB
\href{https://packages.typst.org/preview/slydst-0.1.3.tar.gz}{\pandocbounded{\includesvg[keepaspectratio]{/assets/icons/16-download.svg}}}
\item[Repository:]
\href{https://github.com/glambrechts/slydst}{GitHub}
\item[Categor ies :]
\begin{itemize}
\tightlist
\item[]
\item
  \pandocbounded{\includesvg[keepaspectratio]{/assets/icons/16-presentation.svg}}
  \href{https://typst.app/universe/search/?category=presentation}{Presentation}
\item
  \pandocbounded{\includesvg[keepaspectratio]{/assets/icons/16-layout.svg}}
  \href{https://typst.app/universe/search/?category=layout}{Layout}
\item
  \pandocbounded{\includesvg[keepaspectratio]{/assets/icons/16-package.svg}}
  \href{https://typst.app/universe/search/?category=components}{Components}
\end{itemize}
\end{description}

\subsubsection{Where to report issues?}\label{where-to-report-issues}

This template is a project of Gaspard Lambrechts . Report issues on
\href{https://github.com/glambrechts/slydst}{their repository} . You can
also try to ask for help with this template on the
\href{https://forum.typst.app}{Forum} .

Please report this template to the Typst team using the
\href{https://typst.app/contact}{contact form} if you believe it is a
safety hazard or infringes upon your rights.

\phantomsection\label{versions}
\subsubsection{Version history}\label{version-history}

\begin{longtable}[]{@{}ll@{}}
\toprule\noalign{}
Version & Release Date \\
\midrule\noalign{}
\endhead
\bottomrule\noalign{}
\endlastfoot
0.1.3 & November 12, 2024 \\
\href{https://typst.app/universe/package/slydst/0.1.2/}{0.1.2} &
November 12, 2024 \\
\href{https://typst.app/universe/package/slydst/0.1.1/}{0.1.1} & March
18, 2024 \\
\href{https://typst.app/universe/package/slydst/0.1.0/}{0.1.0} &
November 18, 2023 \\
\end{longtable}

Typst GmbH did not create this template and cannot guarantee correct
functionality of this template or compatibility with any version of the
Typst compiler or app.


\title{typst.app/universe/package/funarray}

\phantomsection\label{banner}
\section{funarray}\label{funarray}

{ 0.4.0 }

Package providing convenient functional functions to use on arrays.

\phantomsection\label{readme}
This package provides some convinient functional functions for
\href{https://typst.app/}{typst} to use on arrays.

\subsection{Usage}\label{usage}

To use this package simply
\texttt{\ \#import\ "@preview/funarray:0.3.0"\ } . To import all
functions use \texttt{\ :\ *\ } and for specific ones, use either the
module or as described in the
\href{https://typst.app/docs/reference/scripting\#modules}{typst docs} .

\subsection{Important note}\label{important-note}

Almost all functions are one-liners, which could, instead of being
loaded via a package import, also be just copied directly into your
source files.

\subsection{Dokumentation}\label{dokumentation}

A prettier und easier to read version of the documentation exists in the
example folder, which is done in typst and exported to pdf. Otherwise,
bellow is the markdown version.

\subsection{Functions}\label{functions}

Let us define
\texttt{\ a\ =\ (1,\ "not\ prime",\ 2,\ "prime",\ 3,\ "prime",\ 4,\ "not\ prime",\ 5,\ "prime")\ }

\subsubsection{chunks}\label{chunks}

The chunks function translates the array to an array of array. It groups
the elements to chunks of a given size and collects them in an bigger
array.

\texttt{\ chunks(a,\ 2)\ =\ (\ (1,\ "not\ prime"),\ (2,\ "prime"),\ (3,\ "prime"),\ (4,\ "not\ prime"),\ (5,\ "prime")\ )\ }

\subsubsection{unzip}\label{unzip}

The unzip function is the inverse of the zip method, it transforms an
array of pairs to a pair of vectors. You can also give input an array of
\texttt{\ n\ } -tuples resulting in in \texttt{\ n\ } arrays.

\texttt{\ unzip(b)\ =\ (\ (1,\ 2,\ 3,\ 4,\ 5),\ (\ "not\ prime",\ "prime",\ "prime",\ "not\ prime",\ "prime"\ )\ )\ }

\subsubsection{cycle}\label{cycle}

The cycle function concatenates the array to itself until it has a given
size.

\begin{Shaded}
\begin{Highlighting}[]
\NormalTok{let c = cycle(range(5), 8)}
\NormalTok{c = (0, 1, 2, 3, 4, 0, 1, 2)}
\end{Highlighting}
\end{Shaded}

Note that there is also the functionality to concatenate with
\texttt{\ +\ } and \texttt{\ *\ } in typst.

\subsubsection{windows and
circular-windows}\label{windows-and-circular-windows}

This function provides a running window

\texttt{\ windows(c,\ 5)\ =\ (\ (0,\ 1,\ 2,\ 3,\ 4),\ (1,\ 2,\ 3,\ 4,\ 0),\ (2,\ 3,\ 4,\ 0,\ 1),\ (3,\ 4,\ 0,\ 1,\ 2)\ )\ }

whereas the circular version wraps over.

\texttt{\ circular-windows(c,\ 5)\ =\ (\ (0,\ 1,\ 2,\ 3,\ 4),\ (1,\ 2,\ 3,\ 4,\ 0),\ (2,\ 3,\ 4,\ 0,\ 1),\ (3,\ 4,\ 0,\ 1,\ 2),\ (4,\ 0,\ 1,\ 2,\ 4),\ (0,\ 1,\ 2,\ 4,\ 0),\ (1,\ 2,\ 4,\ 0,\ 1),\ (2,\ 4,\ 0,\ 1,\ 2)\ )\ }

\subsubsection{partition and
partition-map}\label{partition-and-partition-map}

The partition function seperates the array in two according to a
predicate function. The result is an array with all elements, where the
predicate returned true followed by an array with all elements, where
the predicate returned false.

\begin{Shaded}
\begin{Highlighting}[]
\NormalTok{let (primesp, nonprimesp) = partition(b, x =\textgreater{} x.at(1) == "prime")}
\NormalTok{primesp = ((2, "prime"), (3, "prime"), (5, "prime"))}
\NormalTok{nonprimesp = ((1, "not prime"), (4, "not prime"))}
\end{Highlighting}
\end{Shaded}

There is also a partition-map function, which after partition also
applies a second function on both collections.

\begin{Shaded}
\begin{Highlighting}[]
\NormalTok{let (primes, nonprimes) = partition{-}map(b, x =\textgreater{} x.at(1) == "prime", x =\textgreater{} x.at(0))}
\NormalTok{primes = (2, 3, 5)}
\NormalTok{nonprimes = (1, 4)}
\end{Highlighting}
\end{Shaded}

\subsubsection{group-by}\label{group-by}

This functions groups according to a predicate into maximally sized
chunks, where all elements have the same predicate value.

\begin{Shaded}
\begin{Highlighting}[]
\NormalTok{let f = (0,0,1,1,1,0,0,1)}
\NormalTok{let g = group{-}by(f, x =\textgreater{} x == 0)}
\NormalTok{g = ((0, 0), (1, 1, 1), (0, 0), (1,))}
\end{Highlighting}
\end{Shaded}

\subsubsection{flatten}\label{flatten}

Typst has a \texttt{\ flatten\ } method for arrays, however that method
acts recursively. For instance

\texttt{\ (((1,2,3),\ (2,3)),\ ((1,2,3),\ (1,2))).flatten()\ =\ (1,\ 2,\ 3,\ 2,\ 3,\ 1,\ 2,\ 3,\ 1,\ 2)\ }

Normally, one would only have flattened one level. To do this, we can
use the typst array concatenation method +, or by folding, the sum
method for arrays:

\texttt{\ (((1,2,3),\ (2,3)),\ ((1,2,3),\ (1,2))).sum()\ =\ ((1,\ 2,\ 3),\ (2,\ 3),\ (1,\ 2,\ 3),\ (1,\ 2))\ }

To handle further depth, one can use flatten again, so that in our
example:

\texttt{\ (((1,2,3),\ (2,3)),\ ((1,2,3),\ (1,2))).sum().sum()\ =\ (((1,2,3),\ (2,3)),\ ((1,2,3),\ (1,2))).flatten()\ }

\subsubsection{intersperse}\label{intersperse}

This function has been removed in version 0.3, as typst 0.8 provides
such functionality by default.

\subsubsection{take-while and
skip-while}\label{take-while-and-skip-while}

These functions do exactly as they say.

\begin{Shaded}
\begin{Highlighting}[]
\NormalTok{take{-}while(h, x =\textgreater{} x \textless{} 1) = (0, 0, 0.25, 0.5, 0.75)}
\NormalTok{skip{-}while(h, x =\textgreater{} x \textless{} 1) = (1, 1, 1, 0.25, 0.5, 0.75, 0, 0, 0.25, 0.5, 0.75, 1)}
\end{Highlighting}
\end{Shaded}

\subsection{Unsafe Functions}\label{unsafe-functions}

The core functions are defined in \texttt{\ funarray-unsafe.typ\ } .
However, assertions (error checking) are not there and it is generally
not being advised to use these directly. Still, if being cautious, one
can use the imported \texttt{\ funarray-unsafe\ } module in
\texttt{\ funarray(.typ)\ } . All function names are the same.

To do this from the package, do as follows:

\begin{verbatim}
#import @preview/funarray:0.3.0

#funarray.funarray-unsafe.chunks(range(10), 3)
\end{verbatim}

\subsubsection{How to add}\label{how-to-add}

Copy this into your project and use the import as \texttt{\ funarray\ }

\begin{verbatim}
#import "@preview/funarray:0.4.0"
\end{verbatim}

\includesvg[width=0.16667in,height=0.16667in]{/assets/icons/16-copy.svg}

Check the docs for
\href{https://typst.app/docs/reference/scripting/\#packages}{more
information on how to import packages} .

\subsubsection{About}\label{about}

\begin{description}
\tightlist
\item[Author :]
Ludwig Austermann
\item[License:]
MIT
\item[Current version:]
0.4.0
\item[Last updated:]
October 24, 2023
\item[First released:]
August 1, 2023
\item[Minimum Typst version:]
0.8.0
\item[Archive size:]
4.19 kB
\href{https://packages.typst.org/preview/funarray-0.4.0.tar.gz}{\pandocbounded{\includesvg[keepaspectratio]{/assets/icons/16-download.svg}}}
\item[Repository:]
\href{https://github.com/ludwig-austermann/typst-funarray}{GitHub}
\end{description}

\subsubsection{Where to report issues?}\label{where-to-report-issues}

This package is a project of Ludwig Austermann . Report issues on
\href{https://github.com/ludwig-austermann/typst-funarray}{their
repository} . You can also try to ask for help with this package on the
\href{https://forum.typst.app}{Forum} .

Please report this package to the Typst team using the
\href{https://typst.app/contact}{contact form} if you believe it is a
safety hazard or infringes upon your rights.

\phantomsection\label{versions}
\subsubsection{Version history}\label{version-history}

\begin{longtable}[]{@{}ll@{}}
\toprule\noalign{}
Version & Release Date \\
\midrule\noalign{}
\endhead
\bottomrule\noalign{}
\endlastfoot
0.4.0 & October 24, 2023 \\
\href{https://typst.app/universe/package/funarray/0.3.0/}{0.3.0} &
September 25, 2023 \\
\href{https://typst.app/universe/package/funarray/0.2.0/}{0.2.0} &
August 1, 2023 \\
\end{longtable}

Typst GmbH did not create this package and cannot guarantee correct
functionality of this package or compatibility with any version of the
Typst compiler or app.


\title{typst.app/universe/package/gqe-lemoulon-presentation}

\phantomsection\label{banner}
\phantomsection\label{template-thumbnail}
\pandocbounded{\includegraphics[keepaspectratio]{https://packages.typst.org/preview/thumbnails/gqe-lemoulon-presentation-0.0.4-small.webp}}

\section{gqe-lemoulon-presentation}\label{gqe-lemoulon-presentation}

{ 0.0.4 }

Quickly generate slides for a GQE-Le moulon presentation.

\href{/app?template=gqe-lemoulon-presentation&version=0.0.4}{Create
project in app}

\phantomsection\label{readme}
template \href{https://typst.app/}{Typst web app} to generate GQE slides

\subsection{ðŸ§`â€?ðŸ'» Usage}\label{uxf0uxffuxe2uxf0uxff-usage}

\begin{itemize}
\item
  Directly from \href{https://typst.app/}{Typst web app} by clicking
  “Start from template� on the dashboard and searching for
  \texttt{\ gqe-lemoulon-presentation\ } .
\item
  With CLI:
\end{itemize}

\begin{verbatim}
typst init @preview/gqe-lemoulon-presentation:{version}
\end{verbatim}

\subsection{Documentation}\label{documentation}

gqe-presentation is based on
\href{https://touying-typ.github.io/}{touying} package. The
documentation is available \href{https://touying-typ.github.io/}{here} .

\subsection{Local installation}\label{local-installation}

\subsubsection{Install Rust and Typst}\label{install-rust-and-typst}

\begin{verbatim}
curl --proto '=https' --tlsv1.2 -sSf https://sh.rustup.rs | sh
\end{verbatim}

and then install
\href{https://github.com/typst/typst\#installation}{Typst}

\begin{verbatim}
cargo install typst-cli
\end{verbatim}

\subsubsection{Install the “gqe-presentation� theme on
linux}\label{install-the-uxe2ux153gqe-presentationuxe2-theme-on-linux}

clone the repository in your file system and install the theme
“gqe-lemoulon-presentation� :

\begin{verbatim}
git clone https://forgemia.inra.fr/gqe-moulon/gqe-presentation.git
mkdir -p ~/.local/share/typst/packages/local/gqe-lemoulon-presentation/0.0.4/
cp -r gqe-presentation/* ~/.local/share/typst/packages/local/gqe-lemoulon-presentation/0.0.4/
\end{verbatim}

\subsubsection{Start a new document}\label{start-a-new-document}

\begin{verbatim}
#import "@local/gqe-lemoulon-presentation:0.0.4":*



#show: gqe-theme.with(
  aspect-ratio: "4-3",
  gqe-font: "PT Sans"
  // config-common(handout: true),
  config-info(
    title: [Full native timsTOF data parser implementation in the i2MassChroq software package],
    subtitle: [sous titre],
    author: [Olivier Langella],
    gqe-equipe: [Base],
  ),
)




#title-slide()


#slide()[
= Bioinformatics challenges
#pave("Scientific projects and hardware")[
- High throughput
- Metaproteomics
- Instrument improvements
]
#pause
#pave("Means")[
- Free software (as a speech)
- Finding new algorithms
- Upgrade existing ones
- Controlling infrastructure
- Controlling costs
]

]


#slide()[
= Yes but...
bla bla
]
\end{verbatim}

\subsection{ðŸ``? License}\label{uxf0uxff-license}

This is GPLv3 licensed.

\href{/app?template=gqe-lemoulon-presentation&version=0.0.4}{Create
project in app}

\subsubsection{How to use}\label{how-to-use}

Click the button above to create a new project using this template in
the Typst app.

You can also use the Typst CLI to start a new project on your computer
using this command:

\begin{verbatim}
typst init @preview/gqe-lemoulon-presentation:0.0.4
\end{verbatim}

\includesvg[width=0.16667in,height=0.16667in]{/assets/icons/16-copy.svg}

\subsubsection{About}\label{about}

\begin{description}
\tightlist
\item[Author :]
Olivier Langella
\item[License:]
GPL-3.0
\item[Current version:]
0.0.4
\item[Last updated:]
November 5, 2024
\item[First released:]
November 5, 2024
\item[Archive size:]
336 kB
\href{https://packages.typst.org/preview/gqe-lemoulon-presentation-0.0.4.tar.gz}{\pandocbounded{\includesvg[keepaspectratio]{/assets/icons/16-download.svg}}}
\item[Discipline s :]
\begin{itemize}
\tightlist
\item[]
\item
  \href{https://typst.app/universe/search/?discipline=biology}{Biology}
\item
  \href{https://typst.app/universe/search/?discipline=education}{Education}
\item
  \href{https://typst.app/universe/search/?discipline=agriculture}{Agriculture}
\end{itemize}
\item[Categor y :]
\begin{itemize}
\tightlist
\item[]
\item
  \pandocbounded{\includesvg[keepaspectratio]{/assets/icons/16-presentation.svg}}
  \href{https://typst.app/universe/search/?category=presentation}{Presentation}
\end{itemize}
\end{description}

\subsubsection{Where to report issues?}\label{where-to-report-issues}

This template is a project of Olivier Langella . You can also try to ask
for help with this template on the \href{https://forum.typst.app}{Forum}
.

Please report this template to the Typst team using the
\href{https://typst.app/contact}{contact form} if you believe it is a
safety hazard or infringes upon your rights.

\phantomsection\label{versions}
\subsubsection{Version history}\label{version-history}

\begin{longtable}[]{@{}ll@{}}
\toprule\noalign{}
Version & Release Date \\
\midrule\noalign{}
\endhead
\bottomrule\noalign{}
\endlastfoot
0.0.4 & November 5, 2024 \\
\end{longtable}

Typst GmbH did not create this template and cannot guarantee correct
functionality of this template or compatibility with any version of the
Typst compiler or app.


\title{typst.app/universe/package/name-it}

\phantomsection\label{banner}
\section{name-it}\label{name-it}

{ 0.1.2 }

Get the English names of integers.

\phantomsection\label{readme}
Get the English names of integers.

\subsection{Example}\label{example}

\pandocbounded{\includegraphics[keepaspectratio]{https://github.com/typst/packages/raw/main/packages/preview/name-it/0.1.2/example.png}}

\begin{Shaded}
\begin{Highlighting}[]
\NormalTok{\#import "@preview/name{-}it:0.1.0": name{-}it}

\NormalTok{\#set page(width: auto, height: auto, margin: 1cm)}

\NormalTok{{-} \#name{-}it({-}5)}
\NormalTok{{-} \#name{-}it({-}5, negative{-}prefix: "minus")}
\NormalTok{{-} \#name{-}it(0)}
\NormalTok{{-} \#name{-}it(1)}
\NormalTok{{-} \#name{-}it(10)}
\NormalTok{{-} \#name{-}it(11)}
\NormalTok{{-} \#name{-}it(42)}
\NormalTok{{-} \#name{-}it(100)}
\NormalTok{{-} \#name{-}it(110)}
\NormalTok{{-} \#name{-}it(1104)}
\NormalTok{{-} \#name{-}it(11040)}
\NormalTok{{-} \#name{-}it(11000)}
\NormalTok{{-} \#name{-}it(110000)}
\NormalTok{{-} \#name{-}it(1100004)}
\NormalTok{{-} \#name{-}it(10000000000006)}
\NormalTok{{-} \#name{-}it(10000000000006, show{-}and: false)}
\NormalTok{{-} \#name{-}it("200000000000000000000000007")}
\end{Highlighting}
\end{Shaded}

\subsection{Usage}\label{usage}

\subsubsection{\texorpdfstring{\texttt{\ name-it\ }}{ name-it }}\label{name-it-1}

Convert the given number into its English word representation.

\begin{Shaded}
\begin{Highlighting}[]
\NormalTok{\#let name{-}it(num, show{-}and: true, negative{-}prefix: "negative") = \{ .. \}}
\end{Highlighting}
\end{Shaded}

\textbf{Arguments:}

\begin{itemize}
\tightlist
\item
  \texttt{\ num\ } :
  \href{https://typst.app/docs/reference/foundations/int/}{\texttt{\ int\ }}
  ,
  \href{https://typst.app/docs/reference/foundations/str/}{\texttt{\ str\ }}
  â€'' The number to name.
\item
  \texttt{\ show-and\ } :
  \href{https://typst.app/docs/reference/foundations/bool/}{\texttt{\ bool\ }}
  â€'' Whether an “andâ€? should be used in certain places. For
  example, “one hundred ten� vs “one hundred and ten�.
\item
  \texttt{\ negative-prefix\ } :
  \href{https://typst.app/docs/reference/foundations/str/}{\texttt{\ str\ }}
  â€'' The prefix to use for negative numbers.
\end{itemize}

\subsubsection{How to add}\label{how-to-add}

Copy this into your project and use the import as \texttt{\ name-it\ }

\begin{verbatim}
#import "@preview/name-it:0.1.2"
\end{verbatim}

\includesvg[width=0.16667in,height=0.16667in]{/assets/icons/16-copy.svg}

Check the docs for
\href{https://typst.app/docs/reference/scripting/\#packages}{more
information on how to import packages} .

\subsubsection{About}\label{about}

\begin{description}
\tightlist
\item[Author :]
RubixDev
\item[License:]
GPL-3.0-only
\item[Current version:]
0.1.2
\item[Last updated:]
November 12, 2024
\item[First released:]
October 4, 2023
\item[Archive size:]
74.2 kB
\href{https://packages.typst.org/preview/name-it-0.1.2.tar.gz}{\pandocbounded{\includesvg[keepaspectratio]{/assets/icons/16-download.svg}}}
\item[Repository:]
\href{https://github.com/RubixDev/typst-name-it}{GitHub}
\end{description}

\subsubsection{Where to report issues?}\label{where-to-report-issues}

This package is a project of RubixDev . Report issues on
\href{https://github.com/RubixDev/typst-name-it}{their repository} . You
can also try to ask for help with this package on the
\href{https://forum.typst.app}{Forum} .

Please report this package to the Typst team using the
\href{https://typst.app/contact}{contact form} if you believe it is a
safety hazard or infringes upon your rights.

\phantomsection\label{versions}
\subsubsection{Version history}\label{version-history}

\begin{longtable}[]{@{}ll@{}}
\toprule\noalign{}
Version & Release Date \\
\midrule\noalign{}
\endhead
\bottomrule\noalign{}
\endlastfoot
0.1.2 & November 12, 2024 \\
\href{https://typst.app/universe/package/name-it/0.1.1/}{0.1.1} &
October 5, 2023 \\
\href{https://typst.app/universe/package/name-it/0.1.0/}{0.1.0} &
October 4, 2023 \\
\end{longtable}

Typst GmbH did not create this package and cannot guarantee correct
functionality of this package or compatibility with any version of the
Typst compiler or app.


