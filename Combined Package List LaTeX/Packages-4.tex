\title{typst.app/universe/package/physica}

\phantomsection\label{banner}
\section{physica}\label{physica}

{ 0.9.3 }

Math constructs for science and engineering: derivative, differential,
vector field, matrix, tensor, Dirac braket, hbar, transpose, conjugate,
many operators, and more.

{ } Featured Package

\phantomsection\label{readme}
:green\_book: The
\href{https://github.com/Leedehai/typst-physics/blob/v0.9.3/physica-manual.pdf}{manual}
.

\includegraphics[width=5.67708in,height=\textheight,keepaspectratio]{https://github.com/Leedehai/typst-physics/assets/18319900/ed86198a-8ddb-4473-aed3-8111d5ecde60}

\href{https://github.com/Leedehai/typst-physics/actions/workflows/ci.yml}{\pandocbounded{\includesvg[keepaspectratio]{https://github.com/Leedehai/typst-physics/actions/workflows/ci.yml/badge.svg}}}
\href{https://github.com/Leedehai/typst-physics/releases/latest}{\pandocbounded{\includegraphics[keepaspectratio]{https://img.shields.io/github/v/release/Leedehai/typst-physics.svg?color=gold}}}

Available in the collection of
\href{https://typst.app/docs/packages/}{Typst packages} :
\texttt{\ \#import\ "@preview/physica:0.9.3":\ *\ }

\begin{quote}
physica \emph{noun} .

\begin{itemize}
\tightlist
\item
  Latin, study of nature
\end{itemize}
\end{quote}

This \href{https://typst.app/}{Typst} package provides handy typesetting
utilities for natural sciences, including:

\begin{itemize}
\tightlist
\item
  Braces,
\item
  Vectors and vector fields,
\item
  Matrices, including Jacobian and Hessian,
\item
  Smartly render \texttt{\ ..\^{}T\ } as transpose and
  \texttt{\ ..\^{}+\ } as dagger (conjugate transpose),
\item
  Dirac braket notations,
\item
  Common math functions,
\item
  Differentials and derivatives, including partial derivatives of mixed
  orders with automatic order summation,
\item
  Familiar “h-bar�, tensor abstract index notations, isotopes,
  Taylor series term,
\item
  Signal sequences i.e. digital timing diagrams.
\end{itemize}

\subsection{A quick look}\label{a-quick-look}

See the
\href{https://github.com/Leedehai/typst-physics/blob/v0.9.3/physica-manual.pdf}{manual}
for more details and examples.

\pandocbounded{\includegraphics[keepaspectratio]{https://github.com/Leedehai/typst-physics/assets/18319900/4a9f40df-f753-4324-8114-c682d270e9c7}}

A larger
\href{https://github.com/Leedehai/typst-physics/blob/master/demo.typ}{demo.typ}
:

\pandocbounded{\includegraphics[keepaspectratio]{https://github.com/Leedehai/typst-physics/assets/18319900/75b94ef8-cc98-434f-be5f-bfac1ef6aef9}}

\subsection{Using physica in your Typst
document}\label{using-physica-in-your-typst-document}

\subsubsection{\texorpdfstring{With \texttt{\ typst\ } package
management
(recommended)}{With  typst  package management (recommended)}}\label{with-typst-package-management-recommended}

See \url{https://github.com/typst/packages} . If you are using the
Typst’s web app, packages listed there are readily available; if you
are using the Typst compiler locally, it downloads packages on-demand
and caches them on-disk, see
\href{https://github.com/typst/packages\#downloads}{here} for details.

\includegraphics[width=1.80208in,height=\textheight,keepaspectratio]{https://github.com/Leedehai/typst-physics/assets/18319900/f2a3a2bd-3ef7-4383-ab92-9a71affb4e12}

\begin{Shaded}
\begin{Highlighting}[]
\NormalTok{// Style 1}
\NormalTok{\#import "@preview/physica:0.9.3": *}

\NormalTok{$ curl (grad f), tensor(T, {-}mu, +nu), pdv(f,x,y,[1,2]) $}
\end{Highlighting}
\end{Shaded}

\begin{Shaded}
\begin{Highlighting}[]
\NormalTok{// Style 2}
\NormalTok{\#import "@preview/physica:0.9.3": curl, grad, tensor, pdv}

\NormalTok{$ curl (grad f), tensor(T, {-}mu, +nu), pdv(f,x,y,[1,2]) $}
\end{Highlighting}
\end{Shaded}

\begin{Shaded}
\begin{Highlighting}[]
\NormalTok{// Style 3}
\NormalTok{\#import "@preview/physica:0.9.3"}

\NormalTok{$ physica.curl (physica.grad f), physica.tensor(T, {-}mu, +nu), physica.pdv(f,x,y,[1,2]) $}
\end{Highlighting}
\end{Shaded}

\subsubsection{\texorpdfstring{Without \texttt{\ typst\ } package
management}{Without  typst  package management}}\label{without-typst-package-management}

Similar to examples above, but import with the undecorated file path
like \texttt{\ "physica.typ"\ } .

\subsection{Typst version}\label{typst-version}

The version requirement for the compiler is in
\href{https://github.com/typst/packages/raw/main/packages/preview/physica/0.9.3/typst.toml}{typst.toml}
’s \texttt{\ compiler\ } field. If you are using an unsupported Typst
version, the compiler will throw an error. You may want to update your
compiler with \texttt{\ typst\ update\ } , or choose an earlier version
of the \texttt{\ physica\ } package.

Developed with compiler version:

\begin{Shaded}
\begin{Highlighting}[]
\ExtensionTok{$}\NormalTok{ typst }\AttributeTok{{-}{-}version}
\ExtensionTok{typst}\NormalTok{ 0.10.0 }\ErrorTok{(}\ExtensionTok{70ca0d25}\KeywordTok{)}
\end{Highlighting}
\end{Shaded}

\subsection{Manual}\label{manual}

See the
\href{https://github.com/Leedehai/typst-physics/blob/v0.9.3/physica-manual.pdf}{manual}
for a more comprehensive coverage, a PDF file generated directly with
the \href{https://typst.app/}{Typst} binary.

To regenerate the manual, use command

\begin{Shaded}
\begin{Highlighting}[]
\ExtensionTok{typst}\NormalTok{ watch physica{-}manual.typ}
\end{Highlighting}
\end{Shaded}

\subsection{Contribution}\label{contribution}

\begin{itemize}
\item
  Bug fixes are welcome!
\item
  New features: welcome as well. If it is small, feel free to create a
  pull request. If it is large, the best first step is creating an issue
  and let us explore the design together. Some features might warrant a
  package on its own.
\item
  Testing: currently testing is done by closely inspecting the generated
  \href{https://github.com/Leedehai/typst-physics/blob/v0.9.3/physica-manual.pdf}{manual}
  . This does not scale well. I plan to add programmatic testing by
  comparing rendered pictures with golden images.
\end{itemize}

\subsection{Change log}\label{change-log}

\href{https://github.com/Leedehai/typst-physics/blob/v0.9.3/changelog.md}{changelog.md}
.

\subsection{License}\label{license}

\begin{itemize}
\tightlist
\item
  Code: the
  \href{https://github.com/typst/packages/raw/main/packages/preview/physica/0.9.3/LICENSE.txt}{MIT
  License} .
\item
  Docs: the
  \href{https://creativecommons.org/licenses/by-nd/4.0/}{Creative
  Commons BY-ND 4.0 license} .
\end{itemize}

\subsubsection{How to add}\label{how-to-add}

Copy this into your project and use the import as \texttt{\ physica\ }

\begin{verbatim}
#import "@preview/physica:0.9.3"
\end{verbatim}

\includesvg[width=0.16667in,height=0.16667in]{/assets/icons/16-copy.svg}

Check the docs for
\href{https://typst.app/docs/reference/scripting/\#packages}{more
information on how to import packages} .

\subsubsection{About}\label{about}

\begin{description}
\tightlist
\item[Author :]
Leedehai
\item[License:]
MIT
\item[Current version:]
0.9.3
\item[Last updated:]
April 2, 2024
\item[First released:]
September 8, 2023
\item[Minimum Typst version:]
0.10.0
\item[Archive size:]
11.1 kB
\href{https://packages.typst.org/preview/physica-0.9.3.tar.gz}{\pandocbounded{\includesvg[keepaspectratio]{/assets/icons/16-download.svg}}}
\item[Repository:]
\href{https://github.com/Leedehai/typst-physics}{GitHub}
\item[Discipline s :]
\begin{itemize}
\tightlist
\item[]
\item
  \href{https://typst.app/universe/search/?discipline=chemistry}{Chemistry}
\item
  \href{https://typst.app/universe/search/?discipline=communication}{Communication}
\item
  \href{https://typst.app/universe/search/?discipline=economics}{Economics}
\item
  \href{https://typst.app/universe/search/?discipline=education}{Education}
\item
  \href{https://typst.app/universe/search/?discipline=engineering}{Engineering}
\item
  \href{https://typst.app/universe/search/?discipline=geology}{Geology}
\item
  \href{https://typst.app/universe/search/?discipline=mathematics}{Mathematics}
\item
  \href{https://typst.app/universe/search/?discipline=physics}{Physics}
\end{itemize}
\item[Categor ies :]
\begin{itemize}
\tightlist
\item[]
\item
  \pandocbounded{\includesvg[keepaspectratio]{/assets/icons/16-package.svg}}
  \href{https://typst.app/universe/search/?category=components}{Components}
\item
  \pandocbounded{\includesvg[keepaspectratio]{/assets/icons/16-hammer.svg}}
  \href{https://typst.app/universe/search/?category=utility}{Utility}
\end{itemize}
\end{description}

\subsubsection{Where to report issues?}\label{where-to-report-issues}

This package is a project of Leedehai . Report issues on
\href{https://github.com/Leedehai/typst-physics}{their repository} . You
can also try to ask for help with this package on the
\href{https://forum.typst.app}{Forum} .

Please report this package to the Typst team using the
\href{https://typst.app/contact}{contact form} if you believe it is a
safety hazard or infringes upon your rights.

\phantomsection\label{versions}
\subsubsection{Version history}\label{version-history}

\begin{longtable}[]{@{}ll@{}}
\toprule\noalign{}
Version & Release Date \\
\midrule\noalign{}
\endhead
\bottomrule\noalign{}
\endlastfoot
0.9.3 & April 2, 2024 \\
\href{https://typst.app/universe/package/physica/0.9.2/}{0.9.2} &
January 15, 2024 \\
\href{https://typst.app/universe/package/physica/0.9.1/}{0.9.1} &
December 23, 2023 \\
\href{https://typst.app/universe/package/physica/0.9.0/}{0.9.0} &
December 7, 2023 \\
\href{https://typst.app/universe/package/physica/0.8.1/}{0.8.1} &
November 1, 2023 \\
\href{https://typst.app/universe/package/physica/0.8.0/}{0.8.0} &
September 13, 2023 \\
\href{https://typst.app/universe/package/physica/0.7.5/}{0.7.5} &
September 8, 2023 \\
\end{longtable}

Typst GmbH did not create this package and cannot guarantee correct
functionality of this package or compatibility with any version of the
Typst compiler or app.


\title{typst.app/universe/package/thesist}

\phantomsection\label{banner}
\phantomsection\label{template-thumbnail}
\pandocbounded{\includegraphics[keepaspectratio]{https://packages.typst.org/preview/thumbnails/thesist-0.2.0-small.webp}}

\section{thesist}\label{thesist}

{ 0.2.0 }

A Master\textquotesingle s thesis template for Instituto Superior
Técnico (IST)

\href{/app?template=thesist&version=0.2.0}{Create project in app}

\phantomsection\label{readme}
ThesIST (pronounced “desist�) is an unofficial Master’s thesis
template for Instituto Superior Técnico written in Typst.

This template fully meets the official formatting requirements as
outlined
\href{https://tecnico.ulisboa.pt/files/2021/09/guia-disserta-o-mestrado.pdf}{here}
, and also attempts to follow most unwritten conventions. Regardless,
you can be on the lookout for things you may want to see added.

PIC2 reports are also supported. However, some conventions for these may
vary with the supervisors, so please check with them if anything needs
to be changed.

\subsection{Changelogs}\label{changelogs}

The changelogs of new versions are available on
\href{https://github.com/tfachada/thesist/releases}{the Releases page} .
Make sure to check the latest one(s) whenever you update the imported
\texttt{\ thesist\ } version.

\subsection{Usage}\label{usage}

If you are in the Typst web app, simply click on “Start from
template� and pick this template.

If you want to develop locally:

\begin{enumerate}
\tightlist
\item
  Make sure you have the \textbf{TeX Gyre Heros} font family installed.
\item
  Install the package with \texttt{\ typst\ init\ @preview/thesist\ } .
\end{enumerate}

\subsection{Overview}\label{overview}

\textbf{Please read the “Quick guide� chapter included in this
template to get set up. You can keep it as a reference if you want.}

This template’s source files, hidden from the user view, are the
following:

\begin{itemize}
\item
  \texttt{\ layout.typ\ } : The main configuration file, which
  initializes the thesis and contains its general formatting rules.
\item
  \texttt{\ figure-numbering.typ\ } : This file contains a function to
  set a chapter-relative numbering for the various types of figures. The
  function is called once or twice depending on whether the user decides
  to include appendices.
\item
  \texttt{\ utils.typ\ } : General functions that you may want to import
  and use for QoL improvements.
\end{itemize}

\subsubsection{A sidenote about
subfigures}\label{a-sidenote-about-subfigures}

Since subfigures are not yet native to Typst, the current
implementation, present in \texttt{\ utils.typ\ } , needs the user to
manually input whether each called subfigure figure (aka subfigure grid)
is in an appendix or not. This is because the numbering is different in
appendices, and because the functionality of
\texttt{\ figure-numbering.typ\ } can’t be applied to subfigure grids,
since they are imported with their default numbering once in every
chapter. \texttt{\ context\ } expressions also don’t work across
imports, so location within the document couldn’t be used as a
parameter (unless the user called \texttt{\ context\ } themselves, which
would be unintuitive). \textbf{Regardless, the workaround that was
found, which is explained in the quick guide, doesn’t need much
thinking from the user, so you can see this as a more technical note
that shouldn’t matter when you’re writing the thesis.}

\subsection{Final remarks}\label{final-remarks}

This template is not necessarily (or hopefully) a finished product. Feel
free to open issues or pull requests!

Also thanks to the Typst community members for the help in some of the
functionalities, and for the extensions used here.

\href{/app?template=thesist&version=0.2.0}{Create project in app}

\subsubsection{How to use}\label{how-to-use}

Click the button above to create a new project using this template in
the Typst app.

You can also use the Typst CLI to start a new project on your computer
using this command:

\begin{verbatim}
typst init @preview/thesist:0.2.0
\end{verbatim}

\includesvg[width=0.16667in,height=0.16667in]{/assets/icons/16-copy.svg}

\subsubsection{About}\label{about}

\begin{description}
\tightlist
\item[Author :]
\href{https://github.com/tfachada}{Tomás Fachada}
\item[License:]
MIT
\item[Current version:]
0.2.0
\item[Last updated:]
October 21, 2024
\item[First released:]
August 28, 2024
\item[Minimum Typst version:]
0.12.0
\item[Archive size:]
373 kB
\href{https://packages.typst.org/preview/thesist-0.2.0.tar.gz}{\pandocbounded{\includesvg[keepaspectratio]{/assets/icons/16-download.svg}}}
\item[Repository:]
\href{https://github.com/tfachada/thesist}{GitHub}
\item[Categor y :]
\begin{itemize}
\tightlist
\item[]
\item
  \pandocbounded{\includesvg[keepaspectratio]{/assets/icons/16-mortarboard.svg}}
  \href{https://typst.app/universe/search/?category=thesis}{Thesis}
\end{itemize}
\end{description}

\subsubsection{Where to report issues?}\label{where-to-report-issues}

This template is a project of Tomás Fachada . Report issues on
\href{https://github.com/tfachada/thesist}{their repository} . You can
also try to ask for help with this template on the
\href{https://forum.typst.app}{Forum} .

Please report this template to the Typst team using the
\href{https://typst.app/contact}{contact form} if you believe it is a
safety hazard or infringes upon your rights.

\phantomsection\label{versions}
\subsubsection{Version history}\label{version-history}

\begin{longtable}[]{@{}ll@{}}
\toprule\noalign{}
Version & Release Date \\
\midrule\noalign{}
\endhead
\bottomrule\noalign{}
\endlastfoot
0.2.0 & October 21, 2024 \\
\href{https://typst.app/universe/package/thesist/0.1.0/}{0.1.0} & August
28, 2024 \\
\end{longtable}

Typst GmbH did not create this template and cannot guarantee correct
functionality of this template or compatibility with any version of the
Typst compiler or app.


\title{typst.app/universe/package/grayness}

\phantomsection\label{banner}
\section{grayness}\label{grayness}

{ 0.2.0 }

Simple image editing capabilities like converting to grayscale and
cropping via a WASM plugin.

\phantomsection\label{readme}
A package providing simple image editing capabilities via a WASM plugin.

Available functionality includes converting images to grayscale,
cropping and flipping the images. Furthermore, this package supports
adding transparency and bluring (very slow) as well as handling
additional raster image formats.

The package name is inspired by the blurry, gray images of Nessie, the
\href{https://en.wikipedia.org/wiki/Loch_Ness_Monster}{Loch Ness
Monster}

\subsection{Usage}\label{usage}

Due to the way typst currently interprets given paths, you have to read
the images yourself in the calling typst file. This raw imagedata can
then be passed to the grayness-package functions, like grayscale-image.
These functions also optionally accept all additional parameters of the
original typst image function like \texttt{\ width\ } or
\texttt{\ height\ } :

\begin{Shaded}
\begin{Highlighting}[]
\NormalTok{\#import "@preview/grayness:0.2.0": grayscale{-}image}

\NormalTok{\#let data = read("Art.webp", encoding: none)}
\NormalTok{\#grayscale{-}image(data, width: 50\%)}
\end{Highlighting}
\end{Shaded}

A detailed descriptions of all available functions is provided in the
\href{https://github.com/typst/packages/raw/main/packages/preview/grayness/0.2.0/manual.pdf}{manual}
.

You can also use the built-in help functions provided by tidy:

\begin{Shaded}
\begin{Highlighting}[]
\NormalTok{\#import "@preview/grayness:0.2.0": *}
\NormalTok{\#help("flip{-}image{-}vertical")}
\end{Highlighting}
\end{Shaded}

The \texttt{\ grayscale-image\ } function also works with SVG images. To
do so you must specify the format as \texttt{\ "svg"\ } :

\begin{Shaded}
\begin{Highlighting}[]
\NormalTok{\#let data = read("example.svg", encoding: none)}
\NormalTok{\#grayscale{-}image(data, format: "svg")}
\end{Highlighting}
\end{Shaded}

\subsection{Examples}\label{examples}

Here are several functions applied to a WEBP image of
\href{https://commons.wikimedia.org/wiki/File:Arturo_Nieto-Dorantes.webp}{Arturo
Nieto Dorantes} (CC-By-SA 4.0):
\pandocbounded{\includegraphics[keepaspectratio]{https://github.com/typst/packages/raw/main/packages/preview/grayness/0.2.0/example.png}}

\subsubsection{How to add}\label{how-to-add}

Copy this into your project and use the import as \texttt{\ grayness\ }

\begin{verbatim}
#import "@preview/grayness:0.2.0"
\end{verbatim}

\includesvg[width=0.16667in,height=0.16667in]{/assets/icons/16-copy.svg}

Check the docs for
\href{https://typst.app/docs/reference/scripting/\#packages}{more
information on how to import packages} .

\subsubsection{About}\label{about}

\begin{description}
\tightlist
\item[Author :]
Nikolai Neff-Sarnow
\item[License:]
Apache-2.0
\item[Current version:]
0.2.0
\item[Last updated:]
October 10, 2024
\item[First released:]
April 13, 2024
\item[Minimum Typst version:]
0.11.0
\item[Archive size:]
682 kB
\href{https://packages.typst.org/preview/grayness-0.2.0.tar.gz}{\pandocbounded{\includesvg[keepaspectratio]{/assets/icons/16-download.svg}}}
\item[Repository:]
\href{https://github.com/nineff/grayness}{GitHub}
\item[Categor ies :]
\begin{itemize}
\tightlist
\item[]
\item
  \pandocbounded{\includesvg[keepaspectratio]{/assets/icons/16-chart.svg}}
  \href{https://typst.app/universe/search/?category=visualization}{Visualization}
\item
  \pandocbounded{\includesvg[keepaspectratio]{/assets/icons/16-integration.svg}}
  \href{https://typst.app/universe/search/?category=integration}{Integration}
\item
  \pandocbounded{\includesvg[keepaspectratio]{/assets/icons/16-hammer.svg}}
  \href{https://typst.app/universe/search/?category=utility}{Utility}
\end{itemize}
\end{description}

\subsubsection{Where to report issues?}\label{where-to-report-issues}

This package is a project of Nikolai Neff-Sarnow . Report issues on
\href{https://github.com/nineff/grayness}{their repository} . You can
also try to ask for help with this package on the
\href{https://forum.typst.app}{Forum} .

Please report this package to the Typst team using the
\href{https://typst.app/contact}{contact form} if you believe it is a
safety hazard or infringes upon your rights.

\phantomsection\label{versions}
\subsubsection{Version history}\label{version-history}

\begin{longtable}[]{@{}ll@{}}
\toprule\noalign{}
Version & Release Date \\
\midrule\noalign{}
\endhead
\bottomrule\noalign{}
\endlastfoot
0.2.0 & October 10, 2024 \\
\href{https://typst.app/universe/package/grayness/0.1.0/}{0.1.0} & April
13, 2024 \\
\end{longtable}

Typst GmbH did not create this package and cannot guarantee correct
functionality of this package or compatibility with any version of the
Typst compiler or app.


\title{typst.app/universe/package/modern-sjtu-thesis}

\phantomsection\label{banner}
\phantomsection\label{template-thumbnail}
\pandocbounded{\includegraphics[keepaspectratio]{https://packages.typst.org/preview/thumbnails/modern-sjtu-thesis-0.1.0-small.webp}}

\section{modern-sjtu-thesis}\label{modern-sjtu-thesis}

{ 0.1.0 }

上海交通大学硕士学ä½?论æ--‡ Typst 模æ?¿ã€‚Shanghai Jiao Tong
University Master Thesis Typst Template.

\href{/app?template=modern-sjtu-thesis&version=0.1.0}{Create project in
app}

\phantomsection\label{readme}
这是上海交通大学硕士学ä½?论æ--‡çš„ Typst
模æ?¿ï¼Œå®ƒèƒ½å¤Ÿç®€æ´?ã€?快速ã€?æŒ?ç»­ç''Ÿæˆ? PDF
æ~¼å¼?的毕业论æ--‡ï¼Œå®ƒåŸºäºŽç~''究ç''Ÿé™¢å®˜æ--¹æ??供的模æ?¿è¿›è¡Œå¼€å?{}`。基于ç~''究ç''Ÿé™¢æ??供的
\href{https://www.gs.sjtu.edu.cn/post/detail/Z3M2MjU=}{word 模�}
进行开å?{}`。

\subsection{使ç''¨}\label{uxe4uxbduxe7}

快速�览效果: 查看
\href{https://github.com/tzhTaylor/typst-sjtu-thesis-master/releases/download/v0.1.0/thesis.pdf}{thesis.pdf}
,æ~·ä¾‹è®ºæ--‡æº?ç~?:查看
\href{https://github.com/tzhTaylor/typst-sjtu-thesis-master/blob/main/template/thesis.typ}{thesis.typ}

\subsubsection{VS Code
本地ç¼--è¾`(推è??)}\label{vs-code-uxe6ux153uxe5ux153uxe7uxbcuxe8uxbeuxefuxbcux2c6uxe6ux17euxe8uxefuxbc}

\begin{enumerate}
\item
  在 VS Code 中安è£
  \href{https://marketplace.visualstudio.com/items?itemName=myriad-dreamin.tinymist}{Tinymist
  Typst} æ?'件,负责语法高亮, é''™è¯¯æ£€æŸ¥å'Œ PDF 预览。
\item
  按下 \texttt{\ Ctrl\ +\ Shift\ +\ P\ }
  æ‰``å¼€å`½ä»¤ç•Œé?¢ï¼Œè¾``å\ldots¥
  \texttt{\ Typst:\ Show\ available\ Typst\ templates\ (gallery)\ for\ picking\ up\ a\ template\ }
  æ‰``å¼€ Tinymist æ??供的 Template Gallery,然å?Žä»Žé‡Œé?¢æ‰¾åˆ°
  \texttt{\ modern-sjtu-thesis\ } ,点击 \texttt{\ �\ }
  按é'®è¿›è¡Œæ''¶è---?,以å?Šç‚¹å‡» \texttt{\ +\ }
  å?·ï¼Œå°±å?¯ä»¥åˆ›å»ºå¯¹åº''的论æ--‡æ¨¡æ?¿äº†ã€‚
\item
  最å?Žç''¨ VS Code æ‰``å¼€ç''Ÿæˆ?的目录,æ‰``å¼€
  \texttt{\ thesis.typ\ } æ--‡ä»¶ï¼ŒæŒ‰ä¸‹ \texttt{\ Ctrl\ +\ K\ V\ }
  (Windows) / \texttt{\ Command\ +\ K\ V\ } (MacOS)
  æˆ--è€\ldots 是点击å?³ä¸Šè§'的按é'®è¿›è¡Œå®žæ---¶ç¼--è¾`å'Œé¢„览。
\end{enumerate}

\subsection{致谢}\label{uxe8uxe8}

\begin{itemize}
\item
  æ„Ÿè°¢ \href{https://github.com/OrangeX4}{@OrangeX4} å¼€å?{}`çš„
  \href{https://github.com/nju-lug/modern-nju-thesis}{modern-nju-thesis}
  模æ?¿ï¼Œæœ¬æ¨¡æ?¿å¤§ä½``ç»``构都是å?‚考å\ldots¶å¼€å?{}`的。
\item
  æ„Ÿè°¢ \href{https://typst-doc-cn.github.io/guide/FAQ.html}{Typst
  中æ--‡ç¤¾åŒºå¯¼èˆª FAQ} ,帮忙解决了å?„ç§?ç--`éš¾æ?‚ç---‡ã€‚
\end{itemize}

\subsection{License}\label{license}

This project is licensed under the MIT License.

\href{/app?template=modern-sjtu-thesis&version=0.1.0}{Create project in
app}

\subsubsection{How to use}\label{how-to-use}

Click the button above to create a new project using this template in
the Typst app.

You can also use the Typst CLI to start a new project on your computer
using this command:

\begin{verbatim}
typst init @preview/modern-sjtu-thesis:0.1.0
\end{verbatim}

\includesvg[width=0.16667in,height=0.16667in]{/assets/icons/16-copy.svg}

\subsubsection{About}\label{about}

\begin{description}
\tightlist
\item[Author :]
tzhTaylor
\item[License:]
MIT
\item[Current version:]
0.1.0
\item[Last updated:]
November 19, 2024
\item[First released:]
November 19, 2024
\item[Archive size:]
84.9 kB
\href{https://packages.typst.org/preview/modern-sjtu-thesis-0.1.0.tar.gz}{\pandocbounded{\includesvg[keepaspectratio]{/assets/icons/16-download.svg}}}
\item[Repository:]
\href{https://github.com/tzhTaylor/typst-sjtu-thesis-master}{GitHub}
\item[Categor y :]
\begin{itemize}
\tightlist
\item[]
\item
  \pandocbounded{\includesvg[keepaspectratio]{/assets/icons/16-mortarboard.svg}}
  \href{https://typst.app/universe/search/?category=thesis}{Thesis}
\end{itemize}
\end{description}

\subsubsection{Where to report issues?}\label{where-to-report-issues}

This template is a project of tzhTaylor . Report issues on
\href{https://github.com/tzhTaylor/typst-sjtu-thesis-master}{their
repository} . You can also try to ask for help with this template on the
\href{https://forum.typst.app}{Forum} .

Please report this template to the Typst team using the
\href{https://typst.app/contact}{contact form} if you believe it is a
safety hazard or infringes upon your rights.

\phantomsection\label{versions}
\subsubsection{Version history}\label{version-history}

\begin{longtable}[]{@{}ll@{}}
\toprule\noalign{}
Version & Release Date \\
\midrule\noalign{}
\endhead
\bottomrule\noalign{}
\endlastfoot
0.1.0 & November 19, 2024 \\
\end{longtable}

Typst GmbH did not create this template and cannot guarantee correct
functionality of this template or compatibility with any version of the
Typst compiler or app.


\title{typst.app/universe/package/babble-bubbles}

\phantomsection\label{banner}
\section{babble-bubbles}\label{babble-bubbles}

{ 0.1.0 }

A package to create callouts.

\phantomsection\label{readme}
A package to create callouts in typst, inspired by the
\href{https://obsidian.md/}{Obsidan} callouts.

Use preset callouts, or create your own!

\pandocbounded{\includegraphics[keepaspectratio]{https://github.com/typst/packages/raw/main/packages/preview/babble-bubbles/0.1.0/examples/callouts.png}}

\subsection{Usage}\label{usage}

Import the package

\begin{Shaded}
\begin{Highlighting}[]
\NormalTok{\#import "@preview/babble{-}bubbles:0.1.0": *}
\end{Highlighting}
\end{Shaded}

Or grab it locally and use:

\begin{Shaded}
\begin{Highlighting}[]
\NormalTok{\#import "@local/babble{-}bubbles:0.1.0": *}
\end{Highlighting}
\end{Shaded}

\subsection{Presets}\label{presets}

Here you can find a list of presets and an example usage of each. You
can customise them with the same parameters as the \texttt{\ callout\ }
function. See the \texttt{\ Custom\ callouts\ } for more details.

\begin{Shaded}
\begin{Highlighting}[]
\NormalTok{\#info[This is information]}

\NormalTok{\#success[I\textquotesingle{}m making a note here: huge success]}

\NormalTok{\#check[This is checked!]}

\NormalTok{\#warning[First warning...]}

\NormalTok{\#note[My incredibly useful note]}

\NormalTok{\#question[Question?]}

\NormalTok{\#example[An example make things interesting]}

\NormalTok{\#quote[To be or not to be]}
\end{Highlighting}
\end{Shaded}

\subsection{Custom callouts}\label{custom-callouts}

\subsubsection{\texorpdfstring{\texttt{\ callout\ }}{ callout }}\label{callout}

Create a default callout. Tweak the parameters to create your own!

\begin{Shaded}
\begin{Highlighting}[]
\NormalTok{callout(}
\NormalTok{  body,}
\NormalTok{  title: "Callout",}
\NormalTok{  fill: blue,}
\NormalTok{  title{-}color: white,}
\NormalTok{  body{-}color: black,}
\NormalTok{  icon: none)}
\end{Highlighting}
\end{Shaded}

\subsubsection{Tips}\label{tips}

You can create aliases to more easily handle your newly create callouts
or customise presets by using
\href{https://typst.app/docs/reference/types/function/\#methods-with}{with}
.

\begin{verbatim}
#let mycallout = callout.with(title: "My callout")

#mycallout[Hey this is my custom callout!]
\end{verbatim}

\subsubsection{How to add}\label{how-to-add}

Copy this into your project and use the import as
\texttt{\ babble-bubbles\ }

\begin{verbatim}
#import "@preview/babble-bubbles:0.1.0"
\end{verbatim}

\includesvg[width=0.16667in,height=0.16667in]{/assets/icons/16-copy.svg}

Check the docs for
\href{https://typst.app/docs/reference/scripting/\#packages}{more
information on how to import packages} .

\subsubsection{About}\label{about}

\begin{description}
\tightlist
\item[Author :]
Dimitri Belopopsky
\item[License:]
MIT
\item[Current version:]
0.1.0
\item[Last updated:]
September 11, 2023
\item[First released:]
September 11, 2023
\item[Archive size:]
2.15 kB
\href{https://packages.typst.org/preview/babble-bubbles-0.1.0.tar.gz}{\pandocbounded{\includesvg[keepaspectratio]{/assets/icons/16-download.svg}}}
\item[Repository:]
\href{https://github.com/ShadowMitia/typst-babble-bubbles}{GitHub}
\end{description}

\subsubsection{Where to report issues?}\label{where-to-report-issues}

This package is a project of Dimitri Belopopsky . Report issues on
\href{https://github.com/ShadowMitia/typst-babble-bubbles}{their
repository} . You can also try to ask for help with this package on the
\href{https://forum.typst.app}{Forum} .

Please report this package to the Typst team using the
\href{https://typst.app/contact}{contact form} if you believe it is a
safety hazard or infringes upon your rights.

\phantomsection\label{versions}
\subsubsection{Version history}\label{version-history}

\begin{longtable}[]{@{}ll@{}}
\toprule\noalign{}
Version & Release Date \\
\midrule\noalign{}
\endhead
\bottomrule\noalign{}
\endlastfoot
0.1.0 & September 11, 2023 \\
\end{longtable}

Typst GmbH did not create this package and cannot guarantee correct
functionality of this package or compatibility with any version of the
Typst compiler or app.


\title{typst.app/universe/package/problemst}

\phantomsection\label{banner}
\phantomsection\label{template-thumbnail}
\pandocbounded{\includegraphics[keepaspectratio]{https://packages.typst.org/preview/thumbnails/problemst-0.1.0-small.webp}}

\section{problemst}\label{problemst}

{ 0.1.0 }

Simple and easy-to-use template for problem sets/homeworks/assignments.

\href{/app?template=problemst&version=0.1.0}{Create project in app}

\phantomsection\label{readme}
Simple and easy-to-use template for problem sets/homeworks/assignments.

\pandocbounded{\includegraphics[keepaspectratio]{https://github.com/typst/packages/raw/main/packages/preview/problemst/0.1.0/template/thumbnail.png}}

\subsection{Usage}\label{usage}

Click “Start from template� in the Typst web app and search for
\texttt{\ problemst\ } .

Alternatively, run the following command to create a directory
initialized with all necessary files:

\begin{verbatim}
typst init @preview/problemst:0.1.0
\end{verbatim}

\subsection{Configuration}\label{configuration}

The \texttt{\ pset\ } function takes the following named arguments:

\begin{itemize}
\tightlist
\item
  \texttt{\ class\ } (string): Class the assignment is for.
\item
  \texttt{\ student\ } (string): Student completing the assignment.
\item
  \texttt{\ title\ } (string): Title of the assignment.
\item
  \texttt{\ date\ } (datetime): Date to be displayed on the assignment.
\item
  \texttt{\ collaborators\ } (array of strings): Collaborators that
  worked on the assignment with the student. Can be \texttt{\ ()\ } .
\item
  \texttt{\ subproblems\ } (string): Numbering scheme for the
  subproblems.
\end{itemize}

\href{/app?template=problemst&version=0.1.0}{Create project in app}

\subsubsection{How to use}\label{how-to-use}

Click the button above to create a new project using this template in
the Typst app.

You can also use the Typst CLI to start a new project on your computer
using this command:

\begin{verbatim}
typst init @preview/problemst:0.1.0
\end{verbatim}

\includesvg[width=0.16667in,height=0.16667in]{/assets/icons/16-copy.svg}

\subsubsection{About}\label{about}

\begin{description}
\tightlist
\item[Author :]
\href{https://github.com/carreter}{Willow Carretero Chavez}
\item[License:]
MIT
\item[Current version:]
0.1.0
\item[Last updated:]
April 17, 2024
\item[First released:]
April 17, 2024
\item[Minimum Typst version:]
0.11.0
\item[Archive size:]
2.63 kB
\href{https://packages.typst.org/preview/problemst-0.1.0.tar.gz}{\pandocbounded{\includesvg[keepaspectratio]{/assets/icons/16-download.svg}}}
\item[Categor y :]
\begin{itemize}
\tightlist
\item[]
\item
  \pandocbounded{\includesvg[keepaspectratio]{/assets/icons/16-speak.svg}}
  \href{https://typst.app/universe/search/?category=report}{Report}
\end{itemize}
\end{description}

\subsubsection{Where to report issues?}\label{where-to-report-issues}

This template is a project of Willow Carretero Chavez . You can also try
to ask for help with this template on the
\href{https://forum.typst.app}{Forum} .

Please report this template to the Typst team using the
\href{https://typst.app/contact}{contact form} if you believe it is a
safety hazard or infringes upon your rights.

\phantomsection\label{versions}
\subsubsection{Version history}\label{version-history}

\begin{longtable}[]{@{}ll@{}}
\toprule\noalign{}
Version & Release Date \\
\midrule\noalign{}
\endhead
\bottomrule\noalign{}
\endlastfoot
0.1.0 & April 17, 2024 \\
\end{longtable}

Typst GmbH did not create this template and cannot guarantee correct
functionality of this template or compatibility with any version of the
Typst compiler or app.


\title{typst.app/universe/package/cheq}

\phantomsection\label{banner}
\section{cheq}\label{cheq}

{ 0.2.2 }

Write markdown-like checklist easily.

\phantomsection\label{readme}
Write markdown-like checklist easily.

\subsection{Usage}\label{usage}

Checklists are incredibly useful for keeping track of important items.
We can use the cheq package to achieve checklist syntax similar to
\href{https://github.github.com/gfm/\#task-list-items-extension-}{GitHub
Flavored Markdown} and \href{https://minimal.guide/checklists}{Minimal}
.

\begin{Shaded}
\begin{Highlighting}[]
\NormalTok{\#import "@preview/cheq:0.2.2": checklist}

\NormalTok{\#show: checklist}

\NormalTok{= Solar System Exploration, 1950s – 1960s}

\NormalTok{{-} [ ] Mercury}
\NormalTok{{-} [x] Venus}
\NormalTok{{-} [x] Earth (Orbit/Moon)}
\NormalTok{{-} [x] Mars}
\NormalTok{{-} [{-}] Jupiter}
\NormalTok{{-} [/] Saturn}
\NormalTok{{-} [ ] Uranus}
\NormalTok{{-} [ ] Neptune}
\NormalTok{{-} [ ] Comet Haley}

\NormalTok{= Extras}

\NormalTok{{-} [\textgreater{}] Forwarded}
\NormalTok{{-} [\textless{}] Scheduling}
\NormalTok{{-} [?] question}
\NormalTok{{-} [!] important}
\NormalTok{{-} [\textbackslash{}*] star}
\NormalTok{{-} ["] quote}
\NormalTok{{-} [l] location}
\NormalTok{{-} [b] bookmark}
\NormalTok{{-} [i] information}
\NormalTok{{-} [S] savings}
\NormalTok{{-} [I] idea}
\NormalTok{{-} [p] pros}
\NormalTok{{-} [c] cons}
\NormalTok{{-} [f] fire}
\NormalTok{{-} [k] key}
\NormalTok{{-} [w] win}
\NormalTok{{-} [u] up}
\NormalTok{{-} [d] down}
\end{Highlighting}
\end{Shaded}

\pandocbounded{\includegraphics[keepaspectratio]{https://github.com/typst/packages/raw/main/packages/preview/cheq/0.2.2/examples/example.png}}

\subsection{Custom Styles}\label{custom-styles}

\begin{Shaded}
\begin{Highlighting}[]
\NormalTok{\#import "@preview/cheq:0.2.2": checklist}

\NormalTok{\#show: checklist.with(fill: luma(95\%), stroke: blue, radius: .2em)}

\NormalTok{= Solar System Exploration, 1950s – 1960s}

\NormalTok{{-} [ ] Mercury}
\NormalTok{{-} [x] Venus}
\NormalTok{{-} [x] Earth (Orbit/Moon)}
\NormalTok{{-} [x] Mars}
\NormalTok{{-} [{-}] Jupiter}
\NormalTok{{-} [/] Saturn}
\NormalTok{{-} [ ] Uranus}
\NormalTok{{-} [ ] Neptune}
\NormalTok{{-} [ ] Comet Haley}

\NormalTok{\#show: checklist.with(marker{-}map: (" ": sym.ballot, "x": sym.ballot.x, "{-}": sym.bar.h, "/": sym.slash.double))}

\NormalTok{= Solar System Exploration, 1950s – 1960s}

\NormalTok{{-} [ ] Mercury}
\NormalTok{{-} [x] Venus}
\NormalTok{{-} [x] Earth (Orbit/Moon)}
\NormalTok{{-} [x] Mars}
\NormalTok{{-} [{-}] Jupiter}
\NormalTok{{-} [/] Saturn}
\NormalTok{{-} [ ] Uranus}
\NormalTok{{-} [ ] Neptune}
\NormalTok{{-} [ ] Comet Haley}
\end{Highlighting}
\end{Shaded}

\pandocbounded{\includegraphics[keepaspectratio]{https://github.com/typst/packages/raw/main/packages/preview/cheq/0.2.2/examples/custom-styles.png}}

\subsection{\texorpdfstring{\texttt{\ checklist\ }
function}{ checklist  function}}\label{checklist-function}

\begin{Shaded}
\begin{Highlighting}[]
\NormalTok{\#let checklist(}
\NormalTok{  fill: white,}
\NormalTok{  stroke: rgb("\#616161"),}
\NormalTok{  radius: .1em,}
\NormalTok{  marker{-}map: (:),}
\NormalTok{  body,}
\NormalTok{) = \{ .. \}}
\end{Highlighting}
\end{Shaded}

\textbf{Arguments:}

\begin{itemize}
\tightlist
\item
  \texttt{\ fill\ } : {[} \texttt{\ string\ } {]} â€'' The fill color
  for the checklist marker.
\item
  \texttt{\ stroke\ } : {[} \texttt{\ string\ } {]} â€'' The stroke
  color for the checklist marker.
\item
  \texttt{\ radius\ } : {[} \texttt{\ string\ } {]} â€'' The radius of
  the checklist marker.
\item
  \texttt{\ marker-map\ } : {[} \texttt{\ map\ } {]} â€'' The map of the
  checklist marker. It should be a map of character to symbol function,
  such as
  \texttt{\ ("\ ":\ sym.ballot,\ "x":\ sym.ballot.x,\ "-":\ sym.bar.h,\ "/":\ sym.slash.double)\ }
  .
\item
  \texttt{\ show-list-set-block\ } : {[} \texttt{\ dictionary\ } {]} -
  The configuration of the block in list. It should be a dictionary of
  \texttt{\ above\ } and \texttt{\ below\ } keys, such as
  \texttt{\ (above:\ .5em)\ } .
\item
  \texttt{\ body\ } : {[} \texttt{\ content\ } {]} â€'' The main body
  from \texttt{\ \#show:\ checklist\ } rule.
\end{itemize}

The default map is:

\begin{Shaded}
\begin{Highlighting}[]
\NormalTok{\#let default{-}map = (}
\NormalTok{  "x": checked{-}sym(fill: fill, stroke: stroke, radius: radius),}
\NormalTok{  " ": unchecked{-}sym(fill: fill, stroke: stroke, radius: radius),}
\NormalTok{  "/": incomplete{-}sym(fill: fill, stroke: stroke, radius: radius),}
\NormalTok{  "{-}": canceled{-}sym(fill: fill, stroke: stroke, radius: radius),}
\NormalTok{  "\textgreater{}": "➡",}
\NormalTok{  "\textless{}": "📆",}
\NormalTok{  "?": "❓",}
\NormalTok{  "!": "❗",}
\NormalTok{  "*": "⭐",}
\NormalTok{  "\textbackslash{}"": "❝",}
\NormalTok{  "l": "📍",}
\NormalTok{  "b": "🔖",}
\NormalTok{  "i": "ℹ️",}
\NormalTok{  "S": "💰",}
\NormalTok{  "I": "💡",}
\NormalTok{  "p": "👍",}
\NormalTok{  "c": "👎",}
\NormalTok{  "f": "🔥",}
\NormalTok{  "k": "🔑",}
\NormalTok{  "w": "🏆",}
\NormalTok{  "u": "🔼",}
\NormalTok{  "d": "🔽",}
\NormalTok{)}
\end{Highlighting}
\end{Shaded}

\subsection{\texorpdfstring{\texttt{\ unchecked-sym\ }
function}{ unchecked-sym  function}}\label{unchecked-sym-function}

\begin{Shaded}
\begin{Highlighting}[]
\NormalTok{\#let unchecked{-}sym(fill: white, stroke: rgb("\#616161"), radius: .1em) = \{ .. \}}
\end{Highlighting}
\end{Shaded}

\textbf{Arguments:}

\begin{itemize}
\tightlist
\item
  \texttt{\ fill\ } : {[} \texttt{\ string\ } {]} â€'' The fill color
  for the unchecked symbol.
\item
  \texttt{\ stroke\ } : {[} \texttt{\ string\ } {]} â€'' The stroke
  color for the unchecked symbol.
\item
  \texttt{\ radius\ } : {[} \texttt{\ string\ } {]} â€'' The radius of
  the unchecked symbol.
\end{itemize}

\subsection{\texorpdfstring{\texttt{\ checked-sym\ }
function}{ checked-sym  function}}\label{checked-sym-function}

\begin{Shaded}
\begin{Highlighting}[]
\NormalTok{\#let checked{-}sym(fill: white, stroke: rgb("\#616161"), radius: .1em) = \{ .. \}}
\end{Highlighting}
\end{Shaded}

\textbf{Arguments:}

\begin{itemize}
\tightlist
\item
  \texttt{\ fill\ } : {[} \texttt{\ string\ } {]} â€'' The fill color
  for the checked symbol.
\item
  \texttt{\ stroke\ } : {[} \texttt{\ string\ } {]} â€'' The stroke
  color for the checked symbol.
\item
  \texttt{\ radius\ } : {[} \texttt{\ string\ } {]} â€'' The radius of
  the checked symbol.
\end{itemize}

\subsection{\texorpdfstring{\texttt{\ incomplete-sym\ }
function}{ incomplete-sym  function}}\label{incomplete-sym-function}

\begin{Shaded}
\begin{Highlighting}[]
\NormalTok{\#let incomplete{-}sym(fill: white, stroke: rgb("\#616161"), radius: .1em) = \{ .. \}}
\end{Highlighting}
\end{Shaded}

\textbf{Arguments:}

\begin{itemize}
\tightlist
\item
  \texttt{\ fill\ } : {[} \texttt{\ string\ } {]} â€'' The fill color
  for the incomplete symbol.
\item
  \texttt{\ stroke\ } : {[} \texttt{\ string\ } {]} â€'' The stroke
  color for the incomplete symbol.
\item
  \texttt{\ radius\ } : {[} \texttt{\ string\ } {]} â€'' The radius of
  the incomplete symbol.
\end{itemize}

\subsection{\texorpdfstring{\texttt{\ canceled-sym\ }
function}{ canceled-sym  function}}\label{canceled-sym-function}

\begin{Shaded}
\begin{Highlighting}[]
\NormalTok{\#let canceled{-}sym(fill: white, stroke: rgb("\#616161"), radius: .1em) = \{ .. \}}
\end{Highlighting}
\end{Shaded}

\textbf{Arguments:}

\begin{itemize}
\tightlist
\item
  \texttt{\ fill\ } : {[} \texttt{\ string\ } {]} â€'' The fill color
  for the canceled symbol.
\item
  \texttt{\ stroke\ } : {[} \texttt{\ string\ } {]} â€'' The stroke
  color for the canceled symbol.
\item
  \texttt{\ radius\ } : {[} \texttt{\ string\ } {]} â€'' The radius of
  the canceled symbol.
\end{itemize}

\subsection{License}\label{license}

This project is licensed under the MIT License.

\subsubsection{How to add}\label{how-to-add}

Copy this into your project and use the import as \texttt{\ cheq\ }

\begin{verbatim}
#import "@preview/cheq:0.2.2"
\end{verbatim}

\includesvg[width=0.16667in,height=0.16667in]{/assets/icons/16-copy.svg}

Check the docs for
\href{https://typst.app/docs/reference/scripting/\#packages}{more
information on how to import packages} .

\subsubsection{About}\label{about}

\begin{description}
\tightlist
\item[Author s :]
OrangeX4 , Myriad-Dreamin , \& duskmoon314
\item[License:]
MIT
\item[Current version:]
0.2.2
\item[Last updated:]
October 17, 2024
\item[First released:]
April 12, 2024
\item[Archive size:]
3.33 kB
\href{https://packages.typst.org/preview/cheq-0.2.2.tar.gz}{\pandocbounded{\includesvg[keepaspectratio]{/assets/icons/16-download.svg}}}
\item[Repository:]
\href{https://github.com/OrangeX4/typst-cheq}{GitHub}
\item[Categor ies :]
\begin{itemize}
\tightlist
\item[]
\item
  \pandocbounded{\includesvg[keepaspectratio]{/assets/icons/16-package.svg}}
  \href{https://typst.app/universe/search/?category=components}{Components}
\item
  \pandocbounded{\includesvg[keepaspectratio]{/assets/icons/16-hammer.svg}}
  \href{https://typst.app/universe/search/?category=utility}{Utility}
\end{itemize}
\end{description}

\subsubsection{Where to report issues?}\label{where-to-report-issues}

This package is a project of OrangeX4, Myriad-Dreamin, and duskmoon314 .
Report issues on \href{https://github.com/OrangeX4/typst-cheq}{their
repository} . You can also try to ask for help with this package on the
\href{https://forum.typst.app}{Forum} .

Please report this package to the Typst team using the
\href{https://typst.app/contact}{contact form} if you believe it is a
safety hazard or infringes upon your rights.

\phantomsection\label{versions}
\subsubsection{Version history}\label{version-history}

\begin{longtable}[]{@{}ll@{}}
\toprule\noalign{}
Version & Release Date \\
\midrule\noalign{}
\endhead
\bottomrule\noalign{}
\endlastfoot
0.2.2 & October 17, 2024 \\
\href{https://typst.app/universe/package/cheq/0.2.1/}{0.2.1} & October
14, 2024 \\
\href{https://typst.app/universe/package/cheq/0.2.0/}{0.2.0} & September
8, 2024 \\
\href{https://typst.app/universe/package/cheq/0.1.0/}{0.1.0} & April 12,
2024 \\
\end{longtable}

Typst GmbH did not create this package and cannot guarantee correct
functionality of this package or compatibility with any version of the
Typst compiler or app.


\title{typst.app/universe/package/ascii-ipa}

\phantomsection\label{banner}
\section{ascii-ipa}\label{ascii-ipa}

{ 2.0.0 }

Converter for ASCII representations of the International Phonetic
Alphabet (IPA)

\phantomsection\label{readme}
ðŸ''„ ASCII / IPA conversion for Typst

This package allows you to easily convert different ASCII
representations of the International Phonetic Alphabet (IPA) to and from
the IPA. It also offers some minor utilities to make phonetic
transcriptions easier to use overall. The package is being maintained
\href{https://github.com/imatpot/typst-ascii-ipa}{here} .

Note: This is an extended port of the
\href{https://github.com/tirimid/ipa-translate}{\texttt{\ ipa-translate\ }}
Rust crate by \href{https://github.com/tirimid}{tirimid} ’s conversion
features into native Typst. Most conversions are implemented according
to
\href{https://en.wikipedia.org/wiki/Comparison_of_ASCII_encodings_of_the_International_Phonetic_Alphabet}{this
Wikipedia article} which in turn relies of the official specifications
of the respective ASCII representations.

\subsection{Conversion}\label{conversion}

The package supports multiple ASCII representations for the IPA with one
function each:

\begin{longtable}[]{@{}ll@{}}
\toprule\noalign{}
Notation & Function name \\
\midrule\noalign{}
\endhead
\bottomrule\noalign{}
\endlastfoot
Branner & \texttt{\ \#branner(...)\ } \\
Praat & \texttt{\ \#praat(...)\ } \\
SIL & \texttt{\ \#sil(...)\ } \\
X-SAMPA & \texttt{\ \#xsampa(...)\ } \\
\end{longtable}

They all return the converted value as a
\href{https://typst.app/docs/reference/foundations/str/}{\texttt{\ string\ }}
and accept the set of same parameters:

\begin{longtable}[]{@{}lllll@{}}
\toprule\noalign{}
Parameter & Type & Positional / Named & Default & Description \\
\midrule\noalign{}
\endhead
\bottomrule\noalign{}
\endlastfoot
\texttt{\ value\ } &
\href{https://typst.app/docs/reference/foundations/str/}{\texttt{\ string\ }}
& positional & & Main input to the function. Usually the transcription
in the corresponsing ASCII-based notation. \\
\texttt{\ reverse\ } &
\href{https://typst.app/docs/reference/foundations/bool/}{\texttt{\ bool\ }}
& named & \texttt{\ false\ } & Reverses the conversion. Pass Unicode IPA
into \texttt{\ value\ } to get the corresponsing ASCII-based notation
back. \\
\end{longtable}

\subsubsection{Examples}\label{examples}

All examples use the Swiss German word
\href{https://als.wikipedia.org/wiki/Chuchich\%C3\%A4schtli}{⟨Chuchichäschtli⟩
{[}ˈχʊχË?iˌχæʃË?tlɪ{]}} for the conversion.

\begin{Shaded}
\begin{Highlighting}[]
\NormalTok{\#import "@preview/ascii{-}ipa:2.0.0": *}

\NormalTok{// returns "ˈχʊχːiˌχæʃːtlɪ"}
\NormalTok{\#branner("\textquotesingle{}XUX:i,Xae)S:tlI")}

\NormalTok{// returns "\textquotesingle{}XUX:i,Xae)S:tlI"}
\NormalTok{\#branner("ˈχʊχːiˌχæʃːtlɪ", reverse: true)}

\NormalTok{// returns "ˈχʊχːiˌχæʃːtlɪ"}
\NormalTok{\#praat("\textbackslash{}\textbackslash{}\textquotesingle{}1\textbackslash{}\textbackslash{}cf\textbackslash{}\textbackslash{}hs\textbackslash{}\textbackslash{}cf\textbackslash{}\textbackslash{}:f\textbackslash{}\textbackslash{}\textquotesingle{}2\textbackslash{}\textbackslash{}ae\textbackslash{}\textbackslash{}sh\textbackslash{}\textbackslash{}:ftl\textbackslash{}\textbackslash{}ic")}

\NormalTok{// returns "\textbackslash{}\textbackslash{}\textquotesingle{}1\textbackslash{}\textbackslash{}cf\textbackslash{}\textbackslash{}hs\textbackslash{}\textbackslash{}cf\textbackslash{}\textbackslash{}:f\textbackslash{}\textbackslash{}\textquotesingle{}2\textbackslash{}\textbackslash{}ae\textbackslash{}\textbackslash{}sh\textbackslash{}\textbackslash{}:ftl\textbackslash{}\textbackslash{}ic"}
\NormalTok{\#praat("ˈχʊχːiˌχæʃːtlɪ", reverse: true)}

\NormalTok{// returns "ˈχʊχːiˌχæʃːtlɪ"}
\NormalTok{\#sil("\}x=u\textless{}x=:i\}\}x=a\textless{}s=:tli=")}

\NormalTok{// returns "\}x=u\textless{}x=:i\}\}x=a\textless{}s=:tli="}
\NormalTok{\#sil("ˈχʊχːiˌχæʃːtlɪ", reverse: true)}

\NormalTok{// returns "ˈχʊχːiˌχæʃːtlɪ"}
\NormalTok{\#xsampa("\textbackslash{}"XUX:i\%X\{S:tlI")}

\NormalTok{// returns "\textbackslash{}"XUX:i\%X\{S:tlI"}
\NormalTok{\#xsampa("ˈχʊχːiˌχæʃːtlɪ", reverse: true)}
\end{Highlighting}
\end{Shaded}

\subsubsection{\texorpdfstring{With
\texttt{\ raw\ }}{With  raw }}\label{with-raw}

You can also use
\href{https://typst.app/docs/reference/text/raw/}{\texttt{\ raw\ }} for
the conversion. This is useful if the notation uses a lot of
backslashes.

\begin{Shaded}
\begin{Highlighting}[]
\NormalTok{\#import "@preview/ascii{-}ipa:2.0.0": praat}

\NormalTok{// regular string}
\NormalTok{\#praat("\textbackslash{}\textbackslash{}\textquotesingle{}1\textbackslash{}\textbackslash{}cf\textbackslash{}\textbackslash{}hs\textbackslash{}\textbackslash{}cf\textbackslash{}\textbackslash{}:f\textbackslash{}\textbackslash{}\textquotesingle{}2\textbackslash{}\textbackslash{}ae\textbackslash{}\textbackslash{}sh\textbackslash{}\textbackslash{}:ftl\textbackslash{}\textbackslash{}ic")}

\NormalTok{// raw}
\NormalTok{\#praat(\textasciigrave{}\textbackslash{}\textquotesingle{}1\textbackslash{}cf\textbackslash{}hs\textbackslash{}cf\textbackslash{}:f\textbackslash{}\textquotesingle{}2\textbackslash{}ae\textbackslash{}sh\textbackslash{}:ftl\textbackslash{}ic\textasciigrave{})}
\end{Highlighting}
\end{Shaded}

Note: \texttt{\ raw\ } will not play nicely with notations that use
\texttt{\ \textasciigrave{}\ } a lot.

\subsection{Brackets \& Braces}\label{brackets-braces}

You can easily mark your notation text as different types of brackets or
braces.

\begin{Shaded}
\begin{Highlighting}[]
\NormalTok{\#import "@preview/ascii{-}ipa:2.0.0": *}

\NormalTok{\#phonetic("prʲɪˈvʲet") // [prʲɪˈvʲet]}
\NormalTok{\#phnt("prʲɪˈvʲet")     // [prʲɪˈvʲet]}

\NormalTok{\#precise("prʲɪˈvʲet") // ⟦prʲɪˈvʲet⟧}
\NormalTok{\#prec("prʲɪˈvʲet")    // ⟦prʲɪˈvʲet⟧}

\NormalTok{\#phonemic("prɪvet") // /prɪvet/}
\NormalTok{\#phnm("prɪvet")     // /prɪvet/}

\NormalTok{\#morphophonemic("prɪvet") // ⫽prɪvet⫽}
\NormalTok{\#mphnm("prɪvet")          // ⫽prɪvet⫽}

\NormalTok{\#indistinguishable("prʲɪˈvʲet") // (prʲɪˈvʲet)}
\NormalTok{\#idst("prʲɪˈvʲet")              // (prʲɪˈvʲet)}

\NormalTok{\#obscured("prʲɪˈvʲet") // ⸨prʲɪˈvʲet⸩}
\NormalTok{\#obsc("prʲɪˈvʲet")     // ⸨prʲɪˈvʲet⸩}

\NormalTok{\#orthographic("привет") // ⟨привет⟩}
\NormalTok{\#orth("привет")         // ⟨привет⟩}

\NormalTok{\#transliterated("privyet") // ⟪privyet⟫}
\NormalTok{\#trlt("privyet")           // ⟪privyet⟫}

\NormalTok{\#prosodic("prʲɪˈvʲet") // \{prʲɪˈvʲet\}}
\NormalTok{\#prsd("prʲɪˈvʲet")     // \{prʲɪˈvʲet\}}
\end{Highlighting}
\end{Shaded}

\subsubsection{How to add}\label{how-to-add}

Copy this into your project and use the import as \texttt{\ ascii-ipa\ }

\begin{verbatim}
#import "@preview/ascii-ipa:2.0.0"
\end{verbatim}

\includesvg[width=0.16667in,height=0.16667in]{/assets/icons/16-copy.svg}

Check the docs for
\href{https://typst.app/docs/reference/scripting/\#packages}{more
information on how to import packages} .

\subsubsection{About}\label{about}

\begin{description}
\tightlist
\item[Author :]
imatpot
\item[License:]
MIT
\item[Current version:]
2.0.0
\item[Last updated:]
May 14, 2024
\item[First released:]
March 26, 2024
\item[Minimum Typst version:]
0.10.0
\item[Archive size:]
9.84 kB
\href{https://packages.typst.org/preview/ascii-ipa-2.0.0.tar.gz}{\pandocbounded{\includesvg[keepaspectratio]{/assets/icons/16-download.svg}}}
\item[Repository:]
\href{https://github.com/imatpot/typst-ascii-ipa}{GitHub}
\item[Discipline :]
\begin{itemize}
\tightlist
\item[]
\item
  \href{https://typst.app/universe/search/?discipline=linguistics}{Linguistics}
\end{itemize}
\item[Categor y :]
\begin{itemize}
\tightlist
\item[]
\item
  \pandocbounded{\includesvg[keepaspectratio]{/assets/icons/16-text.svg}}
  \href{https://typst.app/universe/search/?category=text}{Text}
\end{itemize}
\end{description}

\subsubsection{Where to report issues?}\label{where-to-report-issues}

This package is a project of imatpot . Report issues on
\href{https://github.com/imatpot/typst-ascii-ipa}{their repository} .
You can also try to ask for help with this package on the
\href{https://forum.typst.app}{Forum} .

Please report this package to the Typst team using the
\href{https://typst.app/contact}{contact form} if you believe it is a
safety hazard or infringes upon your rights.

\phantomsection\label{versions}
\subsubsection{Version history}\label{version-history}

\begin{longtable}[]{@{}ll@{}}
\toprule\noalign{}
Version & Release Date \\
\midrule\noalign{}
\endhead
\bottomrule\noalign{}
\endlastfoot
2.0.0 & May 14, 2024 \\
\href{https://typst.app/universe/package/ascii-ipa/1.1.1/}{1.1.1} &
March 26, 2024 \\
\href{https://typst.app/universe/package/ascii-ipa/1.1.0/}{1.1.0} &
March 26, 2024 \\
\href{https://typst.app/universe/package/ascii-ipa/1.0.0/}{1.0.0} &
March 26, 2024 \\
\end{longtable}

Typst GmbH did not create this package and cannot guarantee correct
functionality of this package or compatibility with any version of the
Typst compiler or app.


\title{typst.app/universe/package/tidy}

\phantomsection\label{banner}
\section{tidy}\label{tidy}

{ 0.3.0 }

Documentation generator for Typst code in Typst.

{ } Featured Package

\phantomsection\label{readme}
\emph{Keep it tidy.}

\href{https://typst.app/universe/package/tidy}{\pandocbounded{\includegraphics[keepaspectratio]{https://img.shields.io/badge/dynamic/toml?url=https\%3A\%2F\%2Fraw.githubusercontent.com\%2FMc-Zen\%2Ftidy\%2Fmain\%2Ftypst.toml&query=\%24.package.version&prefix=v&logo=typst&label=package&color=239DAD}}}
\href{https://github.com/Mc-Zen/tidy/blob/main/LICENSE}{\pandocbounded{\includegraphics[keepaspectratio]{https://img.shields.io/badge/license-MIT-blue}}}
\href{https://github.com/Mc-Zen/tidy/releases/download/v0.3.0/tidy-guide.pdf}{\pandocbounded{\includegraphics[keepaspectratio]{https://img.shields.io/badge/manual-.pdf-purple}}}

\textbf{tidy} is a package that generates documentation directly in
\href{https://typst.app/}{Typst} for your Typst modules. It parses
docstring comments similar to javadoc and co. and can be used to easily
build a beautiful reference section for the parsed module. Within the
docstring you may use (almost) any Typst syntax âˆ' so markup, equations
and even figures are no problem!

Features:

\begin{itemize}
\tightlist
\item
  \textbf{Customizable} output styles.
\item
  Automatically
  \href{https://github.com/typst/packages/raw/main/packages/preview/tidy/0.3.0/\#example}{\textbf{render
  code examples}} .
\item
  \textbf{Annotate types} of parameters and return values.
\item
  Automatically read off default values for named parameters.
\item
  \href{https://github.com/typst/packages/raw/main/packages/preview/tidy/0.3.0/\#generate-a-help-command-for-you-package}{\textbf{Help}
  feature} for your package.
\item
  \href{https://github.com/typst/packages/raw/main/packages/preview/tidy/0.3.0/\#docstring-tests}{Docstring
  tests} .
\end{itemize}

The
\href{https://github.com/Mc-Zen/tidy/releases/download/v0.3.0/tidy-guide.pdf}{guide}
fully describes the usage of this module and defines the format for the
docstrings.

\subsection{Usage}\label{usage}

Using \texttt{\ tidy\ } is as simple as writing some docstrings and
calling:

\begin{Shaded}
\begin{Highlighting}[]
\NormalTok{\#import "@preview/tidy:0.3.0"}

\NormalTok{\#let docs = tidy.parse{-}module(read("my{-}module.typ"))}
\NormalTok{\#tidy.show{-}module(docs, style: tidy.styles.default)}
\end{Highlighting}
\end{Shaded}

The available predefined styles are currenty
\texttt{\ tidy.styles.default\ } and \texttt{\ tidy.styles.minimal\ } .
Custom styles can be added by hand (take a look at the
\href{https://github.com/Mc-Zen/tidy/releases/download/v0.3.0/tidy-guide.pdf}{guide}
).

\subsection{Example}\label{example}

A full example on how to use this module for your own package (maybe
even consisting of multiple files) can be found at
\href{https://github.com/Mc-Zen/tidy/tree/main/examples}{examples} .

\begin{Shaded}
\begin{Highlighting}[]
\NormalTok{/// This function computes the cardinal sine, $sinc(x)=sin(x)/x$. }
\NormalTok{///}
\NormalTok{/// \#example(\textasciigrave{}\#sinc(0)\textasciigrave{}, mode: "markup")}
\NormalTok{///}
\NormalTok{/// {-} x (int, float): The argument for the cardinal sine function. }
\NormalTok{/// {-}\textgreater{} float}
\NormalTok{\#let sinc(x) = if x == 0 \{1\} else \{calc.sin(x) / x\}}
\end{Highlighting}
\end{Shaded}

\textbf{tidy} turns this into:

\subsubsection{\texorpdfstring{\protect\pandocbounded{\includesvg[keepaspectratio]{https://github.com/typst/packages/raw/main/packages/preview/tidy/0.3.0/docs/images/sincx-docs.svg}}}{Tidy example output}}\label{tidy-example-output}

\subsection{Access user-defined functions and
images}\label{access-user-defined-functions-and-images}

The code in the docstrings is evaluated via \texttt{\ eval()\ } . In
order to access user-defined functions and images, you can make use of
the \texttt{\ scope\ } argument of \texttt{\ tidy.parse-module()\ } :

\begin{Shaded}
\begin{Highlighting}[]
\NormalTok{\#\{}
\NormalTok{    import "my{-}module.typ"}
\NormalTok{    let module = tidy.parse{-}module(read("my{-}module.typ"))}
\NormalTok{    let an{-}image = image("img.png")}
\NormalTok{    tidy.show{-}module(}
\NormalTok{        module,}
\NormalTok{        style: tidy.styles.default,}
\NormalTok{        scope: (my{-}module: my{-}module, img: an{-}image)}
\NormalTok{    )}
\NormalTok{\}}
\end{Highlighting}
\end{Shaded}

The docstrings in \texttt{\ my-module.typ\ } may now access the image
with \texttt{\ \#img\ } and can call any function or variable from
\texttt{\ my-module\ } in the style of
\texttt{\ \#my-module.my-function()\ } . This makes rendering examples
right in the docstrings as easy as a breeze!

\subsection{Generate a help command for you
package}\label{generate-a-help-command-for-you-package}

With \textbf{tidy} , you can add a help command to you package that
allows users to obtain the documentation of a specific definition or
parameter right in the document. This is similar to CLI-style help
commands. If you have already written docstrings for your package, it is
quite low-effort to add this feature. Once set up, the end-user can use
it like this:

\begin{Shaded}
\begin{Highlighting}[]
\NormalTok{// happily coding, but how do I use this one complex function again?}

\NormalTok{\#mypackage.help("func")}
\NormalTok{\#mypackage.help("func(param1)") // print only parameter description of param1}
\end{Highlighting}
\end{Shaded}

This will print the documentation of \texttt{\ func\ } directly into the
document â€'' no need to look it up in a manual. Read up in the
\href{https://github.com/Mc-Zen/tidy/releases/download/v0.3.0/tidy-guide.pdf}{guide}
for setup instructions.

\subsection{Docstring tests}\label{docstring-tests}

It is possible to add simple docstring tests â€'' assertions that will
be run when the documentation is generated. This is useful if you want
to keep small tests and documentation in one place.

\begin{Shaded}
\begin{Highlighting}[]
\NormalTok{/// \#test(}
\NormalTok{///   \textasciigrave{}num.my{-}square(2) == 4\textasciigrave{},}
\NormalTok{///   \textasciigrave{}num.my{-}square(4) == 16\textasciigrave{},}
\NormalTok{/// )}
\NormalTok{\#let my{-}square(n) = n * n}
\end{Highlighting}
\end{Shaded}

With the short-hand syntax, a unfulfilled assertion will even print the
line number of the failed test:

\begin{Shaded}
\begin{Highlighting}[]
\NormalTok{/// \textgreater{}\textgreater{}\textgreater{} my{-}square(2) == 4}
\NormalTok{/// \textgreater{}\textgreater{}\textgreater{} my{-}square(4) == 16}
\NormalTok{\#let my{-}square(n) = n * n}
\end{Highlighting}
\end{Shaded}

A few test assertion functions are available to improve readability,
simplicity, and error messages. Currently, these are
\texttt{\ eq(a,\ b)\ } for equality tests, \texttt{\ ne(a,\ b)\ } for
inequality tests and \texttt{\ approx(a,\ b,\ eps:\ 1e-10)\ } for
floating point comparisons. These assertion helper functions are always
available within docstring tests (with both \texttt{\ test()\ } and
\texttt{\ \textgreater{}\textgreater{}\textgreater{}\ } syntax).

\subsection{Changelog}\label{changelog}

\subsubsection{v0.3.0}\label{v0.3.0}

\begin{itemize}
\tightlist
\item
  New features:

  \begin{itemize}
  \tightlist
  \item
    Help feature.
  \item
    \texttt{\ preamble\ } option for examples (e.g., to add
    \texttt{\ import\ } statements).
  \item
    more options for \texttt{\ show-module\ } :
    \texttt{\ omit-private-definitions\ } ,
    \texttt{\ omit-private-parameters\ } ,
    \texttt{\ enable-cross-references\ } , \texttt{\ local-names\ } (for
    configuring language-specific strings).
  \end{itemize}
\item
  Improvements:

  \begin{itemize}
  \tightlist
  \item
    Allow using \texttt{\ show-example()\ } as standalone.
  \item
    Updated type names that changed with Typst 0.8.0, e.g., integer
    -\textgreater{} int.
  \end{itemize}
\item
  Fixes:

  \begin{itemize}
  \tightlist
  \item
    allow examples with ratio widths if \texttt{\ scale-preview\ } is
    not \texttt{\ auto\ } .
  \item
    \texttt{\ show-outline\ }
  \item
    explicitly use \texttt{\ raw(lang:\ none)\ } for types and function
    names.
  \end{itemize}
\end{itemize}

\subsubsection{v0.2.0}\label{v0.2.0}

\begin{itemize}
\tightlist
\item
  New features:

  \begin{itemize}
  \tightlist
  \item
    Add executable examples to docstrings.
  \item
    Documentation for variables (as well as functions).
  \item
    Docstring tests.
  \item
    Rainbow-colored types \texttt{\ color\ } and \texttt{\ gradient\ } .
  \end{itemize}
\item
  Improvements:

  \begin{itemize}
  \tightlist
  \item
    Allow customization of cross-references through
    \texttt{\ show-reference()\ } .
  \item
    Allow customization of spacing between functions through styles.
  \item
    Allow color customization (especially for the \texttt{\ default\ }
    theme).
  \end{itemize}
\item
  Fixes:

  \begin{itemize}
  \tightlist
  \item
    Empty parameter descriptions are omitted (if the corresponding
    option is set).
  \item
    Trim newline characters from parameter descriptions.
  \end{itemize}
\item
  âš~ï¸? Breaking changes:

  \begin{itemize}
  \tightlist
  \item
    Before, cross-references for functions using the \texttt{\ @@\ }
    syntax could omit the function parentheses. Now this is not possible
    anymore, since such references refer to variables now.
  \item
    (only concerning custom styles) The style functions
    \texttt{\ show-outline()\ } , \texttt{\ show-parameter-list\ } , and
    \texttt{\ show-type()\ } now take \texttt{\ style-args\ } arguments
    as well.
  \end{itemize}
\end{itemize}

\subsubsection{v0.1.0}\label{v0.1.0}

Initial Release.

\subsubsection{How to add}\label{how-to-add}

Copy this into your project and use the import as \texttt{\ tidy\ }

\begin{verbatim}
#import "@preview/tidy:0.3.0"
\end{verbatim}

\includesvg[width=0.16667in,height=0.16667in]{/assets/icons/16-copy.svg}

Check the docs for
\href{https://typst.app/docs/reference/scripting/\#packages}{more
information on how to import packages} .

\subsubsection{About}\label{about}

\begin{description}
\tightlist
\item[Author :]
\href{https://github.com/Mc-Zen}{Mc-Zen}
\item[License:]
MIT
\item[Current version:]
0.3.0
\item[Last updated:]
May 14, 2024
\item[First released:]
August 8, 2023
\item[Minimum Typst version:]
0.6.0
\item[Archive size:]
17.6 kB
\href{https://packages.typst.org/preview/tidy-0.3.0.tar.gz}{\pandocbounded{\includesvg[keepaspectratio]{/assets/icons/16-download.svg}}}
\item[Repository:]
\href{https://github.com/Mc-Zen/tidy}{GitHub}
\item[Categor ies :]
\begin{itemize}
\tightlist
\item[]
\item
  \pandocbounded{\includesvg[keepaspectratio]{/assets/icons/16-hammer.svg}}
  \href{https://typst.app/universe/search/?category=utility}{Utility}
\item
  \pandocbounded{\includesvg[keepaspectratio]{/assets/icons/16-code.svg}}
  \href{https://typst.app/universe/search/?category=scripting}{Scripting}
\item
  \pandocbounded{\includesvg[keepaspectratio]{/assets/icons/16-list-unordered.svg}}
  \href{https://typst.app/universe/search/?category=model}{Model}
\end{itemize}
\end{description}

\subsubsection{Where to report issues?}\label{where-to-report-issues}

This package is a project of Mc-Zen . Report issues on
\href{https://github.com/Mc-Zen/tidy}{their repository} . You can also
try to ask for help with this package on the
\href{https://forum.typst.app}{Forum} .

Please report this package to the Typst team using the
\href{https://typst.app/contact}{contact form} if you believe it is a
safety hazard or infringes upon your rights.

\phantomsection\label{versions}
\subsubsection{Version history}\label{version-history}

\begin{longtable}[]{@{}ll@{}}
\toprule\noalign{}
Version & Release Date \\
\midrule\noalign{}
\endhead
\bottomrule\noalign{}
\endlastfoot
0.3.0 & May 14, 2024 \\
\href{https://typst.app/universe/package/tidy/0.2.0/}{0.2.0} & January
3, 2024 \\
\href{https://typst.app/universe/package/tidy/0.1.0/}{0.1.0} & August 8,
2023 \\
\end{longtable}

Typst GmbH did not create this package and cannot guarantee correct
functionality of this package or compatibility with any version of the
Typst compiler or app.


\title{typst.app/universe/package/charged-ieee}

\phantomsection\label{banner}
\phantomsection\label{template-thumbnail}
\pandocbounded{\includegraphics[keepaspectratio]{https://packages.typst.org/preview/thumbnails/charged-ieee-0.1.3-small.webp}}

\section{charged-ieee}\label{charged-ieee}

{ 0.1.3 }

An IEEE-style paper template to publish at conferences and journals for
Electrical Engineering, Computer Science, and Computer Engineering

{ } Featured Template

\href{/app?template=charged-ieee&version=0.1.3}{Create project in app}

\phantomsection\label{readme}
This is a Typst template for a two-column paper from the proceedings of
the Institute of Electrical and Electronics Engineers. The paper is
tightly spaced, fits a lot of content and comes preconfigured for
numeric citations from BibLaTeX or Hayagriva files.

\subsection{Usage}\label{usage}

You can use this template in the Typst web app by clicking “Start from
template� on the dashboard and searching for \texttt{\ charged-ieee\ }
.

Alternatively, you can use the CLI to kick this project off using the
command

\begin{verbatim}
typst init @preview/charged-ieee
\end{verbatim}

Typst will create a new directory with all the files needed to get you
started.

\subsection{Configuration}\label{configuration}

This template exports the \texttt{\ ieee\ } function with the following
named arguments:

\begin{itemize}
\tightlist
\item
  \texttt{\ title\ } : The paper’s title as content.
\item
  \texttt{\ authors\ } : An array of author dictionaries. Each of the
  author dictionaries must have a \texttt{\ name\ } key and can have the
  keys \texttt{\ department\ } , \texttt{\ organization\ } ,
  \texttt{\ location\ } , and \texttt{\ email\ } . All keys accept
  content.
\item
  \texttt{\ abstract\ } : The content of a brief summary of the paper or
  \texttt{\ none\ } . Appears at the top of the first column in
  boldface.
\item
  \texttt{\ index-terms\ } : Array of index terms to display after the
  abstract. Shall be \texttt{\ content\ } .
\item
  \texttt{\ paper-size\ } : Defaults to \texttt{\ us-letter\ } . Specify
  a
  \href{https://typst.app/docs/reference/layout/page/\#parameters-paper}{paper
  size string} to change the page format.
\item
  \texttt{\ bibliography\ } : The result of a call to the
  \texttt{\ bibliography\ } function or \texttt{\ none\ } . Specifying
  this will configure numeric, IEEE-style citations.
\end{itemize}

The function also accepts a single, positional argument for the body of
the paper.

The template will initialize your package with a sample call to the
\texttt{\ ieee\ } function in a show rule. If you want to change an
existing project to use this template, you can add a show rule like this
at the top of your file:

\begin{Shaded}
\begin{Highlighting}[]
\NormalTok{\#import "@preview/charged{-}ieee:0.1.3": ieee}

\NormalTok{\#show: ieee.with(}
\NormalTok{  title: [A typesetting system to untangle the scientific writing process],}
\NormalTok{  abstract: [}
\NormalTok{    The process of scientific writing is often tangled up with the intricacies of typesetting, leading to frustration and wasted time for researchers. In this paper, we introduce Typst, a new typesetting system designed specifically for scientific writing. Typst untangles the typesetting process, allowing researchers to compose papers faster. In a series of experiments we demonstrate that Typst offers several advantages, including faster document creation, simplified syntax, and increased ease{-}of{-}use.}
\NormalTok{  ],}
\NormalTok{  authors: (}
\NormalTok{    (}
\NormalTok{      name: "Martin Haug",}
\NormalTok{      department: [Co{-}Founder],}
\NormalTok{      organization: [Typst GmbH],}
\NormalTok{      location: [Berlin, Germany],}
\NormalTok{      email: "haug@typst.app"}
\NormalTok{    ),}
\NormalTok{    (}
\NormalTok{      name: "Laurenz Mädje",}
\NormalTok{      department: [Co{-}Founder],}
\NormalTok{      organization: [Typst GmbH],}
\NormalTok{      location: [Berlin, Germany],}
\NormalTok{      email: "maedje@typst.app"}
\NormalTok{    ),}
\NormalTok{  ),}
\NormalTok{  index{-}terms: ("Scientific writing", "Typesetting", "Document creation", "Syntax"),}
\NormalTok{  bibliography: bibliography("refs.bib"),}
\NormalTok{)}

\NormalTok{// Your content goes below.}
\end{Highlighting}
\end{Shaded}

\href{/app?template=charged-ieee&version=0.1.3}{Create project in app}

\subsubsection{How to use}\label{how-to-use}

Click the button above to create a new project using this template in
the Typst app.

You can also use the Typst CLI to start a new project on your computer
using this command:

\begin{verbatim}
typst init @preview/charged-ieee:0.1.3
\end{verbatim}

\includesvg[width=0.16667in,height=0.16667in]{/assets/icons/16-copy.svg}

\subsubsection{About}\label{about}

\begin{description}
\tightlist
\item[Author :]
\href{https://typst.app}{Typst GmbH}
\item[License:]
MIT-0
\item[Current version:]
0.1.3
\item[Last updated:]
October 29, 2024
\item[First released:]
March 6, 2024
\item[Minimum Typst version:]
0.12.0
\item[Archive size:]
6.39 kB
\href{https://packages.typst.org/preview/charged-ieee-0.1.3.tar.gz}{\pandocbounded{\includesvg[keepaspectratio]{/assets/icons/16-download.svg}}}
\item[Repository:]
\href{https://github.com/typst/templates}{GitHub}
\item[Discipline s :]
\begin{itemize}
\tightlist
\item[]
\item
  \href{https://typst.app/universe/search/?discipline=computer-science}{Computer
  Science}
\item
  \href{https://typst.app/universe/search/?discipline=engineering}{Engineering}
\end{itemize}
\item[Categor y :]
\begin{itemize}
\tightlist
\item[]
\item
  \pandocbounded{\includesvg[keepaspectratio]{/assets/icons/16-atom.svg}}
  \href{https://typst.app/universe/search/?category=paper}{Paper}
\end{itemize}
\end{description}

\subsubsection{Where to report issues?}\label{where-to-report-issues}

This template is a project of Typst GmbH . Report issues on
\href{https://github.com/typst/templates}{their repository} . You can
also try to ask for help with this template on the
\href{https://forum.typst.app}{Forum} .

\phantomsection\label{versions}
\subsubsection{Version history}\label{version-history}

\begin{longtable}[]{@{}ll@{}}
\toprule\noalign{}
Version & Release Date \\
\midrule\noalign{}
\endhead
\bottomrule\noalign{}
\endlastfoot
0.1.3 & October 29, 2024 \\
\href{https://typst.app/universe/package/charged-ieee/0.1.2/}{0.1.2} &
August 15, 2024 \\
\href{https://typst.app/universe/package/charged-ieee/0.1.1/}{0.1.1} &
August 8, 2024 \\
\href{https://typst.app/universe/package/charged-ieee/0.1.0/}{0.1.0} &
March 6, 2024 \\
\end{longtable}


\title{typst.app/universe/package/algo}

\phantomsection\label{banner}
\section{algo}\label{algo}

{ 0.3.4 }

Beautifully typeset algorithms.

\phantomsection\label{readme}
A Typst library for writing algorithms. On Typst v0.6.0+ you can import
the \texttt{\ algo\ } package:

\begin{Shaded}
\begin{Highlighting}[]
\NormalTok{\#import "@preview/algo:0.3.4": algo, i, d, comment, code}
\end{Highlighting}
\end{Shaded}

Otherwise, add the \texttt{\ algo.typ\ } file to your project and import
it as normal:

\begin{Shaded}
\begin{Highlighting}[]
\NormalTok{\#import "algo.typ": algo, i, d, comment, code}
\end{Highlighting}
\end{Shaded}

Use the \texttt{\ algo\ } function for writing pseudocode and the
\texttt{\ code\ } function for writing code blocks with line numbers.
Check out the
\href{https://github.com/typst/packages/raw/main/packages/preview/algo/0.3.4/\#examples}{examples}
below for a quick overview. See the
\href{https://github.com/typst/packages/raw/main/packages/preview/algo/0.3.4/\#usage}{usage}
section to read about all the options each function has.

\subsection{Examples}\label{examples}

Here’s a basic use of \texttt{\ algo\ } :

\begin{Shaded}
\begin{Highlighting}[]
\NormalTok{\#algo(}
\NormalTok{  title: "Fib",}
\NormalTok{  parameters: ("n",)}
\NormalTok{)[}
\NormalTok{  if $n \textless{} 0$:\#i\textbackslash{}        // use \#i to indent the following lines}
\NormalTok{    return null\#d\textbackslash{}      // use \#d to dedent the following lines}
\NormalTok{  if $n = 0$ or $n = 1$:\#i \#comment[you can also]\textbackslash{}}
\NormalTok{    return $n$\#d \#comment[add comments!]\textbackslash{}}
\NormalTok{  return \#smallcaps("Fib")$(n{-}1) +$ \#smallcaps("Fib")$(n{-}2)$}
\NormalTok{]}
\end{Highlighting}
\end{Shaded}

\includegraphics[width=4.16667in,height=\textheight,keepaspectratio]{https://user-images.githubusercontent.com/40146328/235323240-e59ed7e2-ebb6-4b80-8742-eb171dd3721e.png}\\

Here’s a use of \texttt{\ algo\ } without a title, parameters, line
numbers, or syntax highlighting:

\begin{Shaded}
\begin{Highlighting}[]
\NormalTok{\#algo(}
\NormalTok{  line{-}numbers: false,}
\NormalTok{  strong{-}keywords: false}
\NormalTok{)[}
\NormalTok{  if $n \textless{} 0$:\#i\textbackslash{}}
\NormalTok{    return null\#d\textbackslash{}}
\NormalTok{  if $n = 0$ or $n = 1$:\#i\textbackslash{}}
\NormalTok{    return $n$\#d\textbackslash{}}
\NormalTok{  \textbackslash{}}
\NormalTok{  let $x \textless{}{-} 0$\textbackslash{}}
\NormalTok{  let $y \textless{}{-} 1$\textbackslash{}}
\NormalTok{  for $i \textless{}{-} 2$ to $n{-}1$:\#i \#comment[so dynamic!]\textbackslash{}}
\NormalTok{    let $z \textless{}{-} x+y$\textbackslash{}}
\NormalTok{    $x \textless{}{-} y$\textbackslash{}}
\NormalTok{    $y \textless{}{-} z$\#d\textbackslash{}}
\NormalTok{    \textbackslash{}}
\NormalTok{  return $x+y$}
\NormalTok{]}
\end{Highlighting}
\end{Shaded}

\includegraphics[width=3.125in,height=\textheight,keepaspectratio]{https://user-images.githubusercontent.com/40146328/235323261-d6e7a42c-ffb7-4c3a-bd2a-4c8fc2df5f36.png}\\

And here’s \texttt{\ algo\ } with more styling options:

\begin{Shaded}
\begin{Highlighting}[]
\NormalTok{\#algo(}
\NormalTok{  title: [                    // note that title and parameters}
\NormalTok{    \#set text(size: 15pt)     // can be content}
\NormalTok{    \#emph(smallcaps("Fib"))}
\NormalTok{  ],}
\NormalTok{  parameters: ([\#math.italic("n")],),}
\NormalTok{  comment{-}prefix: [\#sym.triangle.stroked.r ],}
\NormalTok{  comment{-}styles: (fill: rgb(100\%, 0\%, 0\%)),}
\NormalTok{  indent{-}size: 15pt,}
\NormalTok{  indent{-}guides: 1pt + gray,}
\NormalTok{  row{-}gutter: 5pt,}
\NormalTok{  column{-}gutter: 5pt,}
\NormalTok{  inset: 5pt,}
\NormalTok{  stroke: 2pt + black,}
\NormalTok{  fill: none,}
\NormalTok{)[}
\NormalTok{  if $n \textless{} 0$:\#i\textbackslash{}}
\NormalTok{    return null\#d\textbackslash{}}
\NormalTok{  if $n = 0$ or $n = 1$:\#i\textbackslash{}}
\NormalTok{    return $n$\#d\textbackslash{}}
\NormalTok{  \textbackslash{}}
\NormalTok{  let $x \textless{}{-} 0$\textbackslash{}}
\NormalTok{  let $y \textless{}{-} 1$\textbackslash{}}
\NormalTok{  for $i \textless{}{-} 2$ to $n{-}1$:\#i \#comment[so dynamic!]\textbackslash{}}
\NormalTok{    let $z \textless{}{-} x+y$\textbackslash{}}
\NormalTok{    $x \textless{}{-} y$\textbackslash{}}
\NormalTok{    $y \textless{}{-} z$\#d\textbackslash{}}
\NormalTok{    \textbackslash{}}
\NormalTok{  return $x+y$}
\NormalTok{]}
\end{Highlighting}
\end{Shaded}

\includegraphics[width=3.125in,height=\textheight,keepaspectratio]{https://github.com/platformer/typst-algorithms/assets/40146328/89f80b5d-bdb2-420a-935d-24f43ca597d8}

Here’s a basic use of \texttt{\ code\ } :

\begin{Shaded}
\begin{Highlighting}[]
\NormalTok{\#code()[}
\NormalTok{  \textasciigrave{}\textasciigrave{}\textasciigrave{}py}
\NormalTok{  def fib(n):}
\NormalTok{    if n \textless{} 0:}
\NormalTok{      return None}
\NormalTok{    if n == 0 or n == 1:        \# this comment is}
\NormalTok{      return n                  \# normal raw text}
\NormalTok{    return fib(n{-}1) + fib(n{-}2)}
\NormalTok{  \textasciigrave{}\textasciigrave{}\textasciigrave{}}
\NormalTok{]}
\end{Highlighting}
\end{Shaded}

\includegraphics[width=4.16667in,height=\textheight,keepaspectratio]{https://user-images.githubusercontent.com/40146328/235324088-a3596e0b-af90-4da3-b326-2de11158baac.png}\\

And here’s \texttt{\ code\ } with some styling options:

\begin{Shaded}
\begin{Highlighting}[]
\NormalTok{\#code(}
\NormalTok{  indent{-}guides: 1pt + gray,}
\NormalTok{  row{-}gutter: 5pt,}
\NormalTok{  column{-}gutter: 5pt,}
\NormalTok{  inset: 5pt,}
\NormalTok{  stroke: 2pt + black,}
\NormalTok{  fill: none,}
\NormalTok{)[}
\NormalTok{  \textasciigrave{}\textasciigrave{}\textasciigrave{}py}
\NormalTok{  def fib(n):}
\NormalTok{      if n \textless{} 0:}
\NormalTok{          return None}
\NormalTok{      if n == 0 or n == 1:        \# this comment is}
\NormalTok{          return n                \# normal raw text}
\NormalTok{      return fib(n{-}1) + fib(n{-}2)}
\NormalTok{  \textasciigrave{}\textasciigrave{}\textasciigrave{}}
\NormalTok{]}
\end{Highlighting}
\end{Shaded}

\includegraphics[width=4.16667in,height=\textheight,keepaspectratio]{https://github.com/platformer/typst-algorithms/assets/40146328/c091ac43-6861-40bc-8046-03ea285712c3}

\subsection{Usage}\label{usage}

\subsubsection{\texorpdfstring{\texttt{\ algo\ }}{ algo }}\label{algo-1}

Makes a pseudocode element.

\begin{Shaded}
\begin{Highlighting}[]
\NormalTok{algo(}
\NormalTok{  body,}
\NormalTok{  header: none,}
\NormalTok{  title: none,}
\NormalTok{  parameters: (),}
\NormalTok{  line{-}numbers: true,}
\NormalTok{  strong{-}keywords: true,}
\NormalTok{  keywords: \_algo{-}default{-}keywords, // see below}
\NormalTok{  comment{-}prefix: "// ",}
\NormalTok{  indent{-}size: 20pt,}
\NormalTok{  indent{-}guides: none,}
\NormalTok{  indent{-}guides{-}offset: 0pt,}
\NormalTok{  row{-}gutter: 10pt,}
\NormalTok{  column{-}gutter: 10pt,}
\NormalTok{  inset: 10pt,}
\NormalTok{  fill: rgb(98\%, 98\%, 98\%),}
\NormalTok{  stroke: 1pt + rgb(50\%, 50\%, 50\%),}
\NormalTok{  radius: 0pt,}
\NormalTok{  breakable: false,}
\NormalTok{  block{-}align: center,}
\NormalTok{  main{-}text{-}styles: (:),}
\NormalTok{  comment{-}styles: (fill: rgb(45\%, 45\%, 45\%)),}
\NormalTok{  line{-}number{-}styles: (:)}
\NormalTok{)}
\end{Highlighting}
\end{Shaded}

\textbf{Parameters:}

\begin{itemize}
\item
  \texttt{\ body\ } : \texttt{\ content\ } â€'' Main algorithm content.
\item
  \texttt{\ header\ } : \texttt{\ content\ } â€'' Algorithm header. If
  specified, \texttt{\ title\ } and \texttt{\ parameters\ } are ignored.
\item
  \texttt{\ title\ } : \texttt{\ string\ } or \texttt{\ content\ } â€''
  Algorithm title. Ignored if \texttt{\ header\ } is specified.
\item
  \texttt{\ Parameters\ } : \texttt{\ array\ } â€'' List of algorithm
  parameters. Elements can be \texttt{\ string\ } or
  \texttt{\ content\ } values. \texttt{\ string\ } values will
  automatically be displayed in math mode. Ignored if
  \texttt{\ header\ } is specified.
\item
  \texttt{\ line-numbers\ } : \texttt{\ boolean\ } â€'' Whether to
  display line numbers.
\item
  \texttt{\ strong-keywords\ } : \texttt{\ boolean\ } â€'' Whether to
  strongly emphasize keywords.
\item
  \texttt{\ keywords\ } : \texttt{\ array\ } â€'' List of terms to
  receive strong emphasis. Elements must be \texttt{\ string\ } values.
  Ignored if \texttt{\ strong-keywords\ } is \texttt{\ false\ } .

  The default list of keywords is stored in
  \texttt{\ \_algo-default-keywords\ } . This list contains the
  following terms:

\begin{verbatim}
("if", "else", "then", "while", "for",
"repeat", "do", "until", ":", "end",
"and", "or", "not", "in", "to",
"down", "let", "return", "goto")
\end{verbatim}

  Note that for each of the above terms,
  \texttt{\ \_algo-default-keywords\ } also contains the uppercase form
  of the term (e.g. “for� and “For�).
\item
  \texttt{\ comment-prefix\ } : \texttt{\ content\ } â€'' What to
  prepend comments with.
\item
  \texttt{\ indent-size\ } : \texttt{\ length\ } â€'' Size of line
  indentations.
\item
  \texttt{\ indent-guides\ } : \texttt{\ stroke\ } â€'' Stroke for
  indent guides.
\item
  \texttt{\ indent-guides-offset\ } : \texttt{\ length\ } â€''
  Horizontal offset of indent guides.
\item
  \texttt{\ row-gutter\ } : \texttt{\ length\ } â€'' Space between
  lines.
\item
  \texttt{\ column-gutter\ } : \texttt{\ length\ } â€'' Space between
  line numbers, text, and comments.
\item
  \texttt{\ inset\ } : \texttt{\ length\ } â€'' Size of inner padding.
\item
  \texttt{\ fill\ } : \texttt{\ color\ } â€'' Fill color.
\item
  \texttt{\ stroke\ } : \texttt{\ stroke\ } â€'' Stroke for the
  element’s border.
\item
  \texttt{\ radius\ } : \texttt{\ length\ } â€'' Corner radius.
\item
  \texttt{\ breakable\ } : \texttt{\ boolean\ } â€'' Whether the element
  can break across pages. WARNING: indent guides may look off when
  broken across pages.
\item
  \texttt{\ block-align\ } : \texttt{\ none\ } or \texttt{\ alignment\ }
  or \texttt{\ 2d\ alignment\ } â€'' Alignment of the \texttt{\ algo\ }
  on the page. Using \texttt{\ none\ } will cause the internal
  \texttt{\ block\ } element to be returned as-is.
\item
  \texttt{\ main-text-styles\ } : \texttt{\ dictionary\ } â€'' Styling
  options for the main algorithm text. Supports all parameters in
  Typst’s native \texttt{\ text\ } function.
\item
  \texttt{\ comment-styles\ } : \texttt{\ dictionary\ } â€'' Styling
  options for comment text. Supports all parameters in Typst’s native
  \texttt{\ text\ } function.
\item
  \texttt{\ line-number-styles\ } : \texttt{\ dictionary\ } â€'' Styling
  options for line numbers. Supports all parameters in Typst’s native
  \texttt{\ text\ } function.
\end{itemize}

\subsubsection{\texorpdfstring{\texttt{\ i\ } and
\texttt{\ d\ }}{ i  and  d }}\label{i-and-d}

For use in an \texttt{\ algo\ } body. \texttt{\ \#i\ } indents all
following lines and \texttt{\ \#d\ } dedents all following lines.

\subsubsection{\texorpdfstring{\texttt{\ comment\ }}{ comment }}\label{comment}

For use in an \texttt{\ algo\ } body. Adds a comment to the line in
which it’s placed.

\begin{Shaded}
\begin{Highlighting}[]
\NormalTok{comment(}
\NormalTok{  body,}
\NormalTok{  inline: false}
\NormalTok{)}
\end{Highlighting}
\end{Shaded}

\textbf{Parameters:}

\begin{itemize}
\item
  \texttt{\ body\ } : \texttt{\ content\ } â€'' Comment content.
\item
  \texttt{\ inline\ } : \texttt{\ boolean\ } â€'' If true, the comment
  is displayed in place rather than on the right side.

  NOTE: inline comments will respect both \texttt{\ main-text-styles\ }
  and \texttt{\ comment-styles\ } , preferring
  \texttt{\ comment-styles\ } when the two conflict.

  NOTE: to make inline comments insensitive to
  \texttt{\ strong-keywords\ } , strong emphasis is disabled within
  them. This can be circumvented via the \texttt{\ text\ } function:

\begin{Shaded}
\begin{Highlighting}[]
\NormalTok{\#comment(inline: true)[\#text(weight: 700)[...]]}
\end{Highlighting}
\end{Shaded}
\end{itemize}

\subsubsection{\texorpdfstring{\texttt{\ no-emph\ }}{ no-emph }}\label{no-emph}

For use in an \texttt{\ algo\ } body. Prevents the passed content from
being strongly emphasized. If a word appears in your algorithm both as a
keyword and as normal text, you may escape the non-keyword usages via
this function.

\begin{Shaded}
\begin{Highlighting}[]
\NormalTok{no{-}emph(}
\NormalTok{  body}
\NormalTok{)}
\end{Highlighting}
\end{Shaded}

\textbf{Parameters:}

\begin{itemize}
\tightlist
\item
  \texttt{\ body\ } : \texttt{\ content\ } â€'' Content to display
  without emphasis.
\end{itemize}

\subsubsection{\texorpdfstring{\texttt{\ code\ }}{ code }}\label{code}

Makes a code block element.

\begin{Shaded}
\begin{Highlighting}[]
\NormalTok{code(}
\NormalTok{  body,}
\NormalTok{  line{-}numbers: true,}
\NormalTok{  indent{-}guides: none,}
\NormalTok{  indent{-}guides{-}offset: 0pt,}
\NormalTok{  tab{-}size: auto,}
\NormalTok{  row{-}gutter: 10pt,}
\NormalTok{  column{-}gutter: 10pt,}
\NormalTok{  inset: 10pt,}
\NormalTok{  fill: rgb(98\%, 98\%, 98\%),}
\NormalTok{  stroke: 1pt + rgb(50\%, 50\%, 50\%),}
\NormalTok{  radius: 0pt,}
\NormalTok{  breakable: false,}
\NormalTok{  block{-}align: center,}
\NormalTok{  main{-}text{-}styles: (:),}
\NormalTok{  line{-}number{-}styles: (:)}
\NormalTok{)}
\end{Highlighting}
\end{Shaded}

\textbf{Parameters:}

\begin{itemize}
\item
  \texttt{\ body\ } : \texttt{\ content\ } â€'' Main content. Expects
  \texttt{\ raw\ } text.
\item
  \texttt{\ line-numbers\ } : \texttt{\ boolean\ } â€'' Whether to
  display line numbers.
\item
  \texttt{\ indent-guides\ } : \texttt{\ stroke\ } â€'' Stroke for
  indent guides.
\item
  \texttt{\ indent-guides-offset\ } : \texttt{\ length\ } â€''
  Horizontal offset of indent guides.
\item
  \texttt{\ tab-size\ } : \texttt{\ integer\ } â€'' Amount of spaces
  that should be considered an indent. If unspecified, the tab size is
  determined automatically from the first instance of starting
  whitespace.
\item
  \texttt{\ row-gutter\ } : \texttt{\ length\ } â€'' Space between
  lines.
\item
  \texttt{\ column-gutter\ } : \texttt{\ length\ } â€'' Space between
  line numbers and text.
\item
  \texttt{\ inset\ } : \texttt{\ length\ } â€'' Size of inner padding.
\item
  \texttt{\ fill\ } : \texttt{\ color\ } â€'' Fill color.
\item
  \texttt{\ stroke\ } : \texttt{\ stroke\ } â€'' Stroke for the
  element’s border.
\item
  \texttt{\ radius\ } : \texttt{\ length\ } â€'' Corner radius.
\item
  \texttt{\ breakable\ } : \texttt{\ boolean\ } â€'' Whether the element
  can break across pages. WARNING: indent guides may look off when
  broken across pages.
\item
  \texttt{\ block-align\ } : \texttt{\ none\ } or \texttt{\ alignment\ }
  or \texttt{\ 2d\ alignment\ } â€'' Alignment of the \texttt{\ code\ }
  on the page. Using \texttt{\ none\ } will cause the internal
  \texttt{\ block\ } element to be returned as-is.
\item
  \texttt{\ main-text-styles\ } : \texttt{\ dictionary\ } â€'' Styling
  options for the main raw text. Supports all parameters in Typst’s
  native \texttt{\ text\ } function.
\item
  \texttt{\ line-number-styles\ } : \texttt{\ dictionary\ } â€'' Styling
  options for line numbers. Supports all parameters in Typst’s native
  \texttt{\ text\ } function.
\end{itemize}

\subsection{Contributing}\label{contributing}

PRs are welcome! And if you encounter any bugs or have any
requests/ideas, feel free to open an issue.

\subsubsection{How to add}\label{how-to-add}

Copy this into your project and use the import as \texttt{\ algo\ }

\begin{verbatim}
#import "@preview/algo:0.3.4"
\end{verbatim}

\includesvg[width=0.16667in,height=0.16667in]{/assets/icons/16-copy.svg}

Check the docs for
\href{https://typst.app/docs/reference/scripting/\#packages}{more
information on how to import packages} .

\subsubsection{About}\label{about}

\begin{description}
\tightlist
\item[Author :]
\href{https://github.com/platformer}{platformer}
\item[License:]
MIT
\item[Current version:]
0.3.4
\item[Last updated:]
November 12, 2024
\item[First released:]
August 8, 2023
\item[Archive size:]
10.5 kB
\href{https://packages.typst.org/preview/algo-0.3.4.tar.gz}{\pandocbounded{\includesvg[keepaspectratio]{/assets/icons/16-download.svg}}}
\item[Repository:]
\href{https://github.com/platformer/typst-algorithms}{GitHub}
\item[Discipline :]
\begin{itemize}
\tightlist
\item[]
\item
  \href{https://typst.app/universe/search/?discipline=computer-science}{Computer
  Science}
\end{itemize}
\item[Categor y :]
\begin{itemize}
\tightlist
\item[]
\item
  \pandocbounded{\includesvg[keepaspectratio]{/assets/icons/16-package.svg}}
  \href{https://typst.app/universe/search/?category=components}{Components}
\end{itemize}
\end{description}

\subsubsection{Where to report issues?}\label{where-to-report-issues}

This package is a project of platformer . Report issues on
\href{https://github.com/platformer/typst-algorithms}{their repository}
. You can also try to ask for help with this package on the
\href{https://forum.typst.app}{Forum} .

Please report this package to the Typst team using the
\href{https://typst.app/contact}{contact form} if you believe it is a
safety hazard or infringes upon your rights.

\phantomsection\label{versions}
\subsubsection{Version history}\label{version-history}

\begin{longtable}[]{@{}ll@{}}
\toprule\noalign{}
Version & Release Date \\
\midrule\noalign{}
\endhead
\bottomrule\noalign{}
\endlastfoot
0.3.4 & November 12, 2024 \\
\href{https://typst.app/universe/package/algo/0.3.3/}{0.3.3} & September
21, 2023 \\
\href{https://typst.app/universe/package/algo/0.3.2/}{0.3.2} & September
3, 2023 \\
\href{https://typst.app/universe/package/algo/0.3.1/}{0.3.1} & August
19, 2023 \\
\href{https://typst.app/universe/package/algo/0.3.0/}{0.3.0} & August 8,
2023 \\
\end{longtable}

Typst GmbH did not create this package and cannot guarantee correct
functionality of this package or compatibility with any version of the
Typst compiler or app.


\title{typst.app/universe/package/showybox}

\phantomsection\label{banner}
\section{showybox}\label{showybox}

{ 2.0.3 }

Colorful and customizable boxes for Typst

{ } Featured Package

\phantomsection\label{readme}
\textbf{Showybox} is a Typst package for creating colorful and
customizable boxes.

\subsection{Usage}\label{usage}

To use this library through the Typst package manager (for Typst 0.6.0
or greater), write
\texttt{\ \#import\ "@preview/showybox:2.0.2":\ showybox\ } at the
beginning of your Typst file.

Once imported, you can create an empty showybox by using the function
\texttt{\ showybox()\ } and giving a default body content inside the
parenthesis or outside them using squared brackets \texttt{\ {[}{]}\ } .

By default a \texttt{\ showybox\ } with these properties will be
created:

\begin{itemize}
\tightlist
\item
  No title
\item
  No shadow
\item
  Not breakable
\item
  Black borders
\item
  White background
\item
  \texttt{\ 5pt\ } of border radius
\item
  \texttt{\ 1pt\ } of border thickness
\end{itemize}

\begin{Shaded}
\begin{Highlighting}[]
\NormalTok{\#import "@preview/showybox:2.0.3": showybox}

\NormalTok{\#showybox(}
\NormalTok{  [Hello world!]}
\NormalTok{)}
\end{Highlighting}
\end{Shaded}

\subsubsection{\texorpdfstring{\protect\pandocbounded{\includegraphics[keepaspectratio]{https://github.com/typst/packages/raw/main/packages/preview/showybox/2.0.3/assets/hello-world.png}}}{Hello world! example}}\label{hello-world-example}

Looks quite simple, but the “magic� starts when adding a title,
color and shadows. The following code creates two “unique� boxes
with defined colors and custom borders:

\begin{Shaded}
\begin{Highlighting}[]
\NormalTok{// First showybox}
\NormalTok{\#showybox(}
\NormalTok{  frame: (}
\NormalTok{    border{-}color: red.darken(50\%),}
\NormalTok{    title{-}color: red.lighten(60\%),}
\NormalTok{    body{-}color: red.lighten(80\%)}
\NormalTok{  ),}
\NormalTok{  title{-}style: (}
\NormalTok{    color: black,}
\NormalTok{    weight: "regular",}
\NormalTok{    align: center}
\NormalTok{  ),}
\NormalTok{  shadow: (}
\NormalTok{    offset: 3pt,}
\NormalTok{  ),}
\NormalTok{  title: "Red{-}ish showybox with separated sections!",}
\NormalTok{  lorem(20),}
\NormalTok{  lorem(12)}
\NormalTok{)}

\NormalTok{// Second showybox}
\NormalTok{\#showybox(}
\NormalTok{  frame: (}
\NormalTok{    dash: "dashed",}
\NormalTok{    border{-}color: red.darken(40\%)}
\NormalTok{  ),}
\NormalTok{  body{-}style: (}
\NormalTok{    align: center}
\NormalTok{  ),}
\NormalTok{  sep: (}
\NormalTok{    dash: "dashed"}
\NormalTok{  ),}
\NormalTok{  shadow: (}
\NormalTok{      offset: (x: 2pt, y: 3pt),}
\NormalTok{    color: yellow.lighten(70\%)}
\NormalTok{  ),}
\NormalTok{  [This is an important message!],}
\NormalTok{  [Be careful outside. There are dangerous bananas!]}
\NormalTok{)}
\end{Highlighting}
\end{Shaded}

\subsubsection{\texorpdfstring{\protect\pandocbounded{\includegraphics[keepaspectratio]{https://github.com/typst/packages/raw/main/packages/preview/showybox/2.0.3/assets/two-easy-examples.png}}}{Further examples}}\label{further-examples}

\subsection{Reference}\label{reference}

The \texttt{\ showybox()\ } function can receive the following
parameters:

\begin{itemize}
\tightlist
\item
  \texttt{\ title\ } : A string used as the title of the showybox
\item
  \texttt{\ footer\ } : A string used as the footer of the showybox
\item
  \texttt{\ frame\ } : A dictionary containing the frame’s properties
\item
  \texttt{\ title-style\ } : A dictionary containing the title’s
  styles
\item
  \texttt{\ body-style\ } : A dictionary containing the body’s styles
\item
  \texttt{\ footer-style\ } : A dictionary containing the footer’s
  styles
\item
  \texttt{\ sep\ } : A dictionary containing the separator’s
  properties
\item
  \texttt{\ shadow\ } : A dictionary containing the shadow’s
  properties
\item
  \texttt{\ width\ } : A relative length indicating the showybox’s
  width
\item
  \texttt{\ align\ } : An unidimensional alignement for the showybox in
  the page
\item
  \texttt{\ breakable\ } : A boolean indicating whether a showybox can
  break if it reached an end of page
\item
  \texttt{\ spacing\ } : Space above and below the showybox
\item
  \texttt{\ above\ } : Space above the showybox
\item
  \texttt{\ below\ } : Space below the showybox
\item
  \texttt{\ body\ } : The content of the showybox
\end{itemize}

\subsubsection{Frame properties}\label{frame-properties}

\begin{itemize}
\tightlist
\item
  \texttt{\ title-color\ } : Color used as background color where the
  title goes (default is \texttt{\ black\ } )
\item
  \texttt{\ body-color\ } : Color used as background color where the
  body goes (default is \texttt{\ white\ } )
\item
  \texttt{\ footer-color\ } : Color used as background color where the
  footer goes (default is \texttt{\ luma(85)\ } )
\item
  \texttt{\ border-color\ } : Color used for the showybox’s border
  (default is \texttt{\ black\ } )
\item
  \texttt{\ inset\ } : Inset used for title, body and footer elements
  (default is \texttt{\ (x:\ 1em,\ y:\ 0.65em)\ } ) if none of the
  followings are given:

  \begin{itemize}
  \tightlist
  \item
    \texttt{\ title-inset\ } : Inset used for the title
  \item
    \texttt{\ body-inset\ } : Inset used for the body
  \item
    \texttt{\ footer-inset\ } : Inset used for the body
  \end{itemize}
\item
  \texttt{\ radius\ } : Showybox’s radius (default is \texttt{\ 5pt\ }
  )
\item
  \texttt{\ thickness\ } : Border thickness of the showybox (default is
  \texttt{\ 1pt\ } )
\item
  \texttt{\ dash\ } : Showybox’s border style (default is
  \texttt{\ solid\ } )
\end{itemize}

\subsubsection{Title styles}\label{title-styles}

\begin{itemize}
\tightlist
\item
  \texttt{\ color\ } : Text color (default is \texttt{\ white\ } )
\item
  \texttt{\ weight\ } : Text weight (default is \texttt{\ bold\ } )
\item
  \texttt{\ align\ } : Text align (default is \texttt{\ left\ } )
\item
  \texttt{\ sep-thickness\ } : Thickness of the separator between title
  and body (default is \texttt{\ 1pt\ } )
\item
  \texttt{\ boxed-style\ } : If it’s a dictionary of properties,
  indicates that the title must appear like a “floating box� above
  the showybox. If it’s \texttt{\ none\ } , the title appears normally
  (default is \texttt{\ none\ } )
\end{itemize}

\paragraph{Boxed styles}\label{boxed-styles}

\begin{itemize}
\tightlist
\item
  \texttt{\ anchor\ } : Anchor of the boxed title

  \begin{itemize}
  \tightlist
  \item
    \texttt{\ y\ } : Vertical anchor ( \texttt{\ top\ } ,
    \texttt{\ horizon\ } or \texttt{\ bottom\ } â€`` default is
    \texttt{\ horizon\ } )
  \item
    \texttt{\ x\ } : Horizontal anchor ( \texttt{\ left\ } ,
    \texttt{\ start\ } , \texttt{\ center\ } , \texttt{\ right\ } ,
    \texttt{\ end\ } â€`` default is \texttt{\ left\ } )
  \end{itemize}
\item
  \texttt{\ offset\ } : How much to offset the boxed title in x and y
  direction as a dictionary with keys \texttt{\ x\ } and \texttt{\ y\ }
  (default is \texttt{\ 0pt\ } )
\item
  \texttt{\ radius\ } : Boxed title radius as a dictionary or relative
  length (default is \texttt{\ 5pt\ } )
\end{itemize}

\subsubsection{Body styles}\label{body-styles}

\begin{itemize}
\tightlist
\item
  \texttt{\ color\ } : Text color (default is \texttt{\ black\ } )
\item
  \texttt{\ align\ } : Text align (default is \texttt{\ left\ } )
\end{itemize}

\subsubsection{Footer styles}\label{footer-styles}

\begin{itemize}
\tightlist
\item
  \texttt{\ color\ } : Text color (default is \texttt{\ luma(85)\ } )
\item
  \texttt{\ weight\ } : Text weight (default is \texttt{\ regular\ } )
\item
  \texttt{\ align\ } : Text align (default is \texttt{\ left\ } )
\item
  \texttt{\ sep-thickness\ } : Thickness of the separator between body
  and footer (default is \texttt{\ 1pt\ } )
\end{itemize}

\subsubsection{Separator properties}\label{separator-properties}

\begin{itemize}
\tightlist
\item
  \texttt{\ thickness\ } : Separator’s thickness (default is
  \texttt{\ 1pt\ } )
\item
  \texttt{\ dash\ } : Separator’s style (as a \texttt{\ line\ } dash
  style, default is \texttt{\ "solid"\ } )
\item
  \texttt{\ gutter\ } : Separator’s space above and below (defalut is
  \texttt{\ 0.65em\ } )
\end{itemize}

\subsubsection{Shadow properties}\label{shadow-properties}

\begin{itemize}
\tightlist
\item
  \texttt{\ color\ } : Shadow color (default is \texttt{\ black\ } )
\item
  \texttt{\ offset\ } : How much to offset the shadow in x and y
  direction either as a length or a dictionary with keys \texttt{\ x\ }
  and \texttt{\ y\ } (default is \texttt{\ 4pt\ } )
\end{itemize}

\subsection{Gallery}\label{gallery}

\subsubsection{Colors for title, body and footer example (Stokes’
theorem)}\label{colors-for-title-body-and-footer-example-stokesuxe2-theorem}

\subsubsection{\texorpdfstring{\protect\pandocbounded{\includegraphics[keepaspectratio]{https://github.com/typst/packages/raw/main/packages/preview/showybox/2.0.3/assets/stokes-example.png}}}{Encapsulation}}\label{encapsulation}

\subsubsection{Boxed-title example (Clairaut’s
theorem)}\label{boxed-title-example-clairautuxe2s-theorem}

\subsubsection{\texorpdfstring{\protect\pandocbounded{\includegraphics[keepaspectratio]{https://github.com/typst/packages/raw/main/packages/preview/showybox/2.0.3/assets/clairaut-example.png}}}{Encapsulation}}\label{encapsulation-1}

\subsubsection{Encapsulation example}\label{encapsulation-example}

\subsubsection{\texorpdfstring{\protect\pandocbounded{\includegraphics[keepaspectratio]{https://github.com/typst/packages/raw/main/packages/preview/showybox/2.0.3/assets/encapsulation-example.png}}}{Encapsulation}}\label{encapsulation-2}

\subsubsection{Breakable showybox example (Newton’s second
law)}\label{breakable-showybox-example-newtonuxe2s-second-law}

\subsubsection{\texorpdfstring{\protect\pandocbounded{\includegraphics[keepaspectratio]{https://github.com/typst/packages/raw/main/packages/preview/showybox/2.0.3/assets/newton-example.png}}}{Enabling breakable}}\label{enabling-breakable}

\subsubsection{Custom radius and title’s separator thickness example
(Carnot’s cycle
efficency)}\label{custom-radius-and-titleuxe2s-separator-thickness-example-carnotuxe2s-cycle-efficency}

\subsubsection{\texorpdfstring{\protect\pandocbounded{\includegraphics[keepaspectratio]{https://github.com/typst/packages/raw/main/packages/preview/showybox/2.0.3/assets/carnot-example.png}}}{Custom radius}}\label{custom-radius}

\subsubsection{Custom border dash and inset example (Gauss’s
law)}\label{custom-border-dash-and-inset-example-gaussuxe2s-law}

\subsubsection{\texorpdfstring{\protect\pandocbounded{\includegraphics[keepaspectratio]{https://github.com/typst/packages/raw/main/packages/preview/showybox/2.0.3/assets/gauss-example.png}}}{Custom radius}}\label{custom-radius-1}

\subsubsection{Custom footer’s separator thickness example
(Divergence’s
theorem)}\label{custom-footeruxe2s-separator-thickness-example-divergenceuxe2s-theorem}

\subsubsection{\texorpdfstring{\protect\pandocbounded{\includegraphics[keepaspectratio]{https://github.com/typst/packages/raw/main/packages/preview/showybox/2.0.3/assets/divergence-example.png}}}{Custom radius}}\label{custom-radius-2}

\subsubsection{Colorful shadow example (Coulomb’s
law)}\label{colorful-shadow-example-coulombuxe2s-law}

\subsubsection{\texorpdfstring{\protect\pandocbounded{\includegraphics[keepaspectratio]{https://github.com/typst/packages/raw/main/packages/preview/showybox/2.0.3/assets/coulomb-example.png}}}{Custom radius}}\label{custom-radius-3}

\subsection{Changelog}\label{changelog}

\subsubsection{Version 2.0.3}\label{version-2.0.3}

\begin{itemize}
\tightlist
\item
  Revert fix breakable box empty before new page. Layout didn’t
  converge
\end{itemize}

\subsubsection{Version 2.0.2}\label{version-2.0.2}

\begin{itemize}
\tightlist
\item
  Remove deprecated functions in Typst 0.12.0
\item
  Fix breakable box empty before new page
\end{itemize}

\subsubsection{Version 2.0.1}\label{version-2.0.1}

\begin{itemize}
\tightlist
\item
  Fix bad behaviours of boxed-titles \texttt{\ anchor\ } inside a
  \texttt{\ figure\ }
\item
  Fix wrong \texttt{\ breakable\ } behaviour of showyboxes inside a
  \texttt{\ figure\ }
\item
  Fix Manual and README’s Stokes theorem example
\end{itemize}

\subsubsection{Version 2.0.0}\label{version-2.0.0}

\emph{Special thanks to Andrew Voynov (
\url{https://github.com/Andrew15-5} ) for the feedback while creating
the new behaviours for boxed-titles}

\begin{itemize}
\tightlist
\item
  Update \texttt{\ type()\ } conditionals to Typst 0.8.0 standards
\item
  Add \texttt{\ boxed-style\ } property, with \texttt{\ anchor\ } ,
  \texttt{\ offset\ } and \texttt{\ radius\ } properties.
\item
  Refactor \texttt{\ showy-inset()\ } for being general-purpose. Now
  it’s called \texttt{\ showy-value-in-direction()\ } and has a
  default value for handling properties defaults
\item
  Now sharp corners can be set by giving a dictionary to frame
  \texttt{\ radius\ } (e.g.
  \texttt{\ radius:\ (top:\ 5pt,\ bottom:\ 0pt)\ } ). Before this only
  was possible for untitled showyboxes.
\item
  Refactor shadow functions to be in a separated file.
\item
  Fix bug of bad behaviour while writing too long titles.
\item
  Fix bug while rendering separators with custom thickness. Now the
  thickness is gotten properly.
\item
  Fix bad shadow drawing in showyboxes with a boxed-title that has a
  “extreme� \texttt{\ offset\ } value.
\item
  Fix bad sizing while creating showyboxes with a \texttt{\ width\ } of
  less than \texttt{\ 100\%\ } , and a shadow.
\end{itemize}

\subsubsection{Version 1.1.0}\label{version-1.1.0}

\begin{itemize}
\tightlist
\item
  Added \texttt{\ boxed\ } option in title styles
\item
  Added \texttt{\ boxed-align\ } in title styles
\item
  Added \texttt{\ sep-thickness\ } for title and footer
\item
  Refactored repository’s files layout
\end{itemize}

\subsubsection{Version 1.0.0}\label{version-1.0.0}

\begin{itemize}
\tightlist
\item
  Fixed shadow displacement

  \begin{itemize}
  \tightlist
  \item
    \textbf{Details:} Instead of displacing the showybox’s body from
    the shadow, now the shadow is displaced from the body.
  \end{itemize}
\end{itemize}

\emph{Changes below were performed by Jonas Neugebauer (
\url{https://github.com/jneug} )}

\begin{itemize}
\tightlist
\item
  Added \texttt{\ title-inset\ } , \texttt{\ body-inset\ } ,
  \texttt{\ footer-inset\ } and \texttt{\ inset\ } options

  \begin{itemize}
  \tightlist
  \item
    \textbf{Details:} \texttt{\ title-inset\ } , \texttt{\ body-inset\ }
    and \texttt{\ footer-inset\ } will set the inset of the title, body
    and footer area respectively. \texttt{\ inset\ } is a fallback for
    those areas.
  \end{itemize}
\item
  Added a \texttt{\ sep.gutter\ } option to set the spacing around
  separator lines
\item
  Added option \texttt{\ width\ } to set the width of a showybox
\item
  Added option \texttt{\ align\ } to move a showybox with
  \texttt{\ width\ } \textless{} 100\% along the x-axis

  \begin{itemize}
  \tightlist
  \item
    \textbf{Details:} A showybox is now wrapped in another block to
    allow alignment. This also makes it possible to pass the spacing
    options \texttt{\ spacing\ } , \texttt{\ above\ } and
    \texttt{\ below\ } to \texttt{\ \#showybox()\ } .
  \end{itemize}
\item
  Added \texttt{\ footer\ } and \texttt{\ footer-style\ } options

  \begin{itemize}
  \tightlist
  \item
    \textbf{Details:} The optional footer is added at the bottom of the
    box.
  \end{itemize}
\end{itemize}

\subsubsection{Version 0.2.1}\label{version-0.2.1}

\emph{All changes listed here were performed by Jonas Neugebauer (
\url{https://github.com/jneug} )}

\begin{itemize}
\tightlist
\item
  Added the \texttt{\ shadow\ } option
\item
  Enabled auto-break ( \texttt{\ breakable\ } ) functionality for titled
  showyboxes
\item
  Removed a thin line that appears in showyboxes with no borders or
  dashed borders
\end{itemize}

\subsubsection{Version 0.2.0}\label{version-0.2.0}

\begin{itemize}
\tightlist
\item
  Improved code documentation
\item
  Enabled an auto-break functionality for non-titled showyboxes
\item
  Created a separator functionality to separate content inside a
  showybox with a horizontal line
\end{itemize}

\subsubsection{Version 0.1.1}\label{version-0.1.1}

\begin{itemize}
\tightlist
\item
  Changed package name from colorbox to showybox
\item
  Fixed a spacing bug in encapsulated showyboxes

  \begin{itemize}
  \tightlist
  \item
    \textbf{Details:} When a showybox was encapsulated inside another,
    the spacing after that showybox was \texttt{\ 0pt\ } , probably due
    to some “fixes� improved to manage default spacing between
    \texttt{\ rect\ } elements. The issue was solved by avoiding
    \texttt{\ \#set\ } statements and adding a \texttt{\ \#v(-1.1em)\ }
    to correct extra spacing between the title \texttt{\ rect\ } and the
    body \texttt{\ rect\ } .
  \end{itemize}
\end{itemize}

\subsubsection{Version 0.1.0}\label{version-0.1.0}

\begin{itemize}
\tightlist
\item
  Initial release
\end{itemize}

\subsubsection{How to add}\label{how-to-add}

Copy this into your project and use the import as \texttt{\ showybox\ }

\begin{verbatim}
#import "@preview/showybox:2.0.3"
\end{verbatim}

\includesvg[width=0.16667in,height=0.16667in]{/assets/icons/16-copy.svg}

Check the docs for
\href{https://typst.app/docs/reference/scripting/\#packages}{more
information on how to import packages} .

\subsubsection{About}\label{about}

\begin{description}
\tightlist
\item[Author s :]
Pablo González Calderón \& Showybox Contributors
\item[License:]
MIT
\item[Current version:]
2.0.3
\item[Last updated:]
October 24, 2024
\item[First released:]
July 3, 2023
\item[Minimum Typst version:]
0.12.0
\item[Archive size:]
9.41 kB
\href{https://packages.typst.org/preview/showybox-2.0.3.tar.gz}{\pandocbounded{\includesvg[keepaspectratio]{/assets/icons/16-download.svg}}}
\item[Repository:]
\href{https://github.com/Pablo-Gonzalez-Calderon/showybox-package}{GitHub}
\item[Categor y :]
\begin{itemize}
\tightlist
\item[]
\item
  \pandocbounded{\includesvg[keepaspectratio]{/assets/icons/16-package.svg}}
  \href{https://typst.app/universe/search/?category=components}{Components}
\end{itemize}
\end{description}

\subsubsection{Where to report issues?}\label{where-to-report-issues}

This package is a project of Pablo González Calderón and Showybox
Contributors . Report issues on
\href{https://github.com/Pablo-Gonzalez-Calderon/showybox-package}{their
repository} . You can also try to ask for help with this package on the
\href{https://forum.typst.app}{Forum} .

Please report this package to the Typst team using the
\href{https://typst.app/contact}{contact form} if you believe it is a
safety hazard or infringes upon your rights.

\phantomsection\label{versions}
\subsubsection{Version history}\label{version-history}

\begin{longtable}[]{@{}ll@{}}
\toprule\noalign{}
Version & Release Date \\
\midrule\noalign{}
\endhead
\bottomrule\noalign{}
\endlastfoot
2.0.3 & October 24, 2024 \\
\href{https://typst.app/universe/package/showybox/2.0.2/}{2.0.2} &
October 21, 2024 \\
\href{https://typst.app/universe/package/showybox/2.0.1/}{2.0.1} &
October 4, 2023 \\
\href{https://typst.app/universe/package/showybox/2.0.0/}{2.0.0} &
October 1, 2023 \\
\href{https://typst.app/universe/package/showybox/1.1.0/}{1.1.0} &
August 3, 2023 \\
\href{https://typst.app/universe/package/showybox/1.0.0/}{1.0.0} & July
31, 2023 \\
\href{https://typst.app/universe/package/showybox/0.2.1/}{0.2.1} & July
31, 2023 \\
\href{https://typst.app/universe/package/showybox/0.2.0/}{0.2.0} & July
10, 2023 \\
\href{https://typst.app/universe/package/showybox/0.1.1/}{0.1.1} & July
3, 2023 \\
\end{longtable}

Typst GmbH did not create this package and cannot guarantee correct
functionality of this package or compatibility with any version of the
Typst compiler or app.


\title{typst.app/universe/package/universal-cau-thesis}

\phantomsection\label{banner}
\phantomsection\label{template-thumbnail}
\pandocbounded{\includegraphics[keepaspectratio]{https://packages.typst.org/preview/thumbnails/universal-cau-thesis-0.1.0-small.webp}}

\section{universal-cau-thesis}\label{universal-cau-thesis}

{ 0.1.0 }

中国农业大学毕业论æ--‡çš„Typst模æ?¿

\href{/app?template=universal-cau-thesis&version=0.1.0}{Create project
in app}

\phantomsection\label{readme}
\subsubsection{为什么使ç''¨Typst}\label{uxe4uxbauxe4uxe4uxb9ux2c6uxe4uxbduxe7typst}

å›~为语法足够简å?•ï¼ˆç®€å?•æ˜``学)ã€?ç¼--è¯`器ä½``积å°?(éš?å?--éš?ç''¨ï¼‰ã€?ä¸''速度足够快(实æ---¶é¢„览)ï¼?

\pandocbounded{\includegraphics[keepaspectratio]{https://github.com/typst/packages/raw/main/packages/preview/universal-cau-thesis/0.1.0/images/PixPin_2024-03-13_17-19-53.png}}

\subsubsection{å\ldots³äºŽæœ¬è®ºæ--‡æ¨¡æ?¿}\label{uxe5uxb3uxe4uxbaux17euxe6ux153uxe8uxbauxe6uxe6uxe6}

本模��考
\href{https://jwc.cau.edu.cn/art/2020/2/25/art_14181_663910.html}{中国农业大学毕业论æ--‡æ¨¡æ?¿è¦?求}
ç¼--写,符å?ˆå­¦æ~¡è¦?求。对于å\ldots¶ä¸­ä¸€äº›å?¯ä»¥ç?µæ´»ä¿®æ''¹çš„æ~¼å¼?,本模æ?¿ä¹Ÿæ??供了é\ldots?置项,å\ldots·ä½``详è§?模æ?¿ä½¿ç''¨æ--¹æ³•ã€‚模æ?¿çš„效果è§?
\texttt{\ sample.pdf\ } æ--‡ä»¶ï¼š
\href{https://github.com/JWangL5/CAU-ThesisTemplate-Typst/blob/master/sample.pdf}{点击直达}

âš~ï¸?
但是,本模æ?¿ä¸ºä¸ªäººç¼--写使ç''¨ï¼Œæ²¡æœ‰å®Œå\ldots¨é€‚é\ldots?自动åŒ--逻è¾`å'Œæ¨¡å?---åŒ--代ç~?,å?¯èƒ½ä»?æ---§å­˜åœ¨éƒ¨åˆ†æƒ\ldots 况æŽ'版ä¸?å?ˆç?†ã€‚如果您在使ç''¨è¿‡ç¨‹ä¸­é?‡åˆ°é---®é¢˜ï¼Œå?¯ä»¥æ??交issue说明,也欢迎pull
request贡献。

如果该模æ?¿å¯¹æ‚¨æœ‰å¸®åŠ©å¹¶æ„¿æ„?æ''¯æŒ?æˆ`的工作,å?¯ä»¥åœ¨è¿™é‡Œ
\href{https://www.buymeacoffee.com/jwangl5}{buy me a coffee}
,å??分感谢😊ï¼?

\subsubsection{使ç''¨æ--¹æ³•}\label{uxe4uxbduxe7uxe6uxb9uxe6uxb3}

\begin{enumerate}
\item
  Typstå?¯ä»¥ä½¿ç''¨ \href{https://typst.app/}{线上WebApp}
  æˆ--本地下载ç¼--è¯`器å?Žè¿›è¡Œç¼--写,本地ç¼--写需è¦?
  \href{https://github.com/typst/typst/releases}{下载安è£\ldots ç¼--è¯`器}
  到本地,并将 \texttt{\ exe\ }
  æ~¼å¼?çš„ç¼--è¯`器添åŠ~到环境å?˜é‡?,以æ--¹ä¾¿è°ƒç''¨
\item
  下载该ä»``åº``到本地目录æˆ--WebApp的工作目录中,å?¯ä»¥ä½¿ç''¨gitå`½ä»¤æˆ--该页é?¢ä¸Šæ--¹çš„Code按é'®ç›´æŽ¥ä¸‹è½½

\begin{Shaded}
\begin{Highlighting}[]
\NormalTok{git clone https://github.com/JWangL5/CAU{-}ThesisTemplate{-}Typst.git}
\end{Highlighting}
\end{Shaded}
\item
  安è£\ldots 本模æ?¿æ‰€ä½¿ç''¨çš„相å\ldots³å­---ä½``(
  \texttt{\ /fonts\ } æ--‡ä»¶å¤¹å†\ldots å­---ä½``)

  PS:å?---é™?于中æ--‡å­---ä½``的衬线é---®é¢˜ï¼Œç›®å‰?版本的Typstæš‚ä¸?æ''¯æŒ?多数中æ--‡å­---ä½``çš„åŠ~ç²---,这里使ç''¨äº†å\ldots¶ä»--åŠ~ç²---å­---ä½``作为替代,å?¯ä»¥ç›´æŽ¥å?Œå‡»å­---ä½``æ--‡ä»¶æ‰``å¼€å?Žå®‰è£
\item
  本地ç¼--写æ---¶ï¼Œå»ºè®®ä½¿ç''¨
  \href{https://code.visualstudio.com/}{vscode}
  å?ŠTypsté\ldots?å¥---æ?'件(Typst-LSPã€?Typst Preview)
\item
  通过ç¼--写 \texttt{\ main.typ\ }
  æ--‡ä»¶å®Œæˆ?论æ--‡çš„æ'°å†™ï¼Œåœ¨è¯¥æ--‡ä»¶ä¸­ï¼Œä½¿ç''¨
  \texttt{\ import\ } å`½ä»¤å¼•å\ldots¥æ¨¡æ?¿ï¼Œå¹¶ä¿®æ''¹é\ldots?置项

  \begin{itemize}
  \tightlist
  \item
    kind:填写 \texttt{\ "本科"\ } , \texttt{\ "硕士"\ } ,
    \texttt{\ "�士"\ }
    ,å\ldots¶ä¼šå¯¹åº''ä¿®æ''¹å°?é?¢å'Œé¡µçœ‰å¤„çš„ä¿¡æ?¯
  \item
    title:论æ--‡æ~‡é¢˜ï¼Œå¡«å†™åœ¨æ‹¬å?· \texttt{\ {[}text{]}\ }
    å†\ldots ,使ç''¨ \texttt{\ \textbackslash{}\ } æ?¢è¡Œ
  \item
    abstract:论æ--‡æ`˜è¦?,需è¦?手动写å\ldots³é''®è¯?
  \item
    authors:论æ--‡ä½œè€\ldots å§``å??,å?¯ä»¥
    \texttt{\ {[}text{]}\ } ,也�以 \texttt{\ "text"\ }
  \item
    teacher:指导教师的å§``å??
  \item
    degree:ç''³è¯·å­¦ä½?é---¨ç±»çº§åˆ«ï¼Œæ¯''如
    \texttt{\ {[}�学硕士{]}\ }
  \item
    college, major,
    field:å°?é?¢ä¸Šçš„å†\ldots 容,学院ã€?ä¸``业å'Œç~''究æ--¹å?{}`
  \item
    signature:ä½~çš„ç''µå­?ç­¾å??æ--‡ä»¶è·¯å¾„,是论æ--‡ç‹¬åˆ›æ€§å£°æ˜Žå¤„çš„ç­¾å­---
  \item
    classification \& security:论æ--‡åœ¨å›¾ä¹¦é¦†æ''¶å½•æ---¶çš„
    \href{https://www.clcindex.com/}{中图法分类} å'Œä¿?密级别
  \item
    student\_ID:学�
  \item
    year, month, day:论æ--‡å°?é?¢å'Œè¯šä¿¡å£°æ˜Žé¡µä¸Šçš„æ---¥æœŸ
  \item
    draft:填写 \texttt{\ true\ }
    æ---¶æ·»åŠ~è?‰ç¨¿æ°´å?°ï¼Œç''¨ä»¥åŒºåˆ†æ˜¯å?¦ä¸ºæœ€ç»ˆç‰ˆæœ¬ï¼Œå¡«å†™
    \texttt{\ false\ } æ---¶åŽ»é™¤æ°´å?°å¹¶æ·»åŠ~论æ--‡ç«
  \item
    blindReview:填写为 \texttt{\ true\ }
    æ---¶éš?è---?å°?é?¢ä¸Šçš„相å\ldots³ä¿¡æ?¯ï¼Œä»¥å?Šè‡´è°¢å'Œä½œè€\ldots 介ç»?
  \end{itemize}
\item
  使ç''¨ \texttt{\ typst\ }
  å`½ä»¤ç''Ÿæˆ?pdfæ~¼å¼?æ--‡ä»¶ï¼Œæˆ--直接使ç''¨vscode的实æ---¶é¢„览æ?'件(默认快æ?·é''®
  \texttt{\ ctrl+k\ v\ } )

\begin{Shaded}
\begin{Highlighting}[]
\NormalTok{typst compile .}\AttributeTok{/sample}\NormalTok{.typ}
\end{Highlighting}
\end{Shaded}
\end{enumerate}

\subsubsection{Typstç¼--写简æ˜``指å?---}\label{typstuxe7uxbcuxe5uxe7uxe6uxe6ux153uxe5}

\begin{quote}
如果在使ç''¨Typstæ---¶é?‡åˆ°ä»»ä½•é---®é¢˜ï¼Œéƒ½å?¯ä»¥å?‚考
\href{https://typst.app/docs/}{官æ--¹å¸®åŠ©æ--‡æ¡£}
,下é?¢æ˜¯ç®€è¦?的使ç''¨æ--¹æ³•å?Šä¸Žæœ¬æ¨¡æ?¿ç›¸å\ldots³çš„é\ldots?å¥---设置,å?¯ä»¥å?‚考的示例æ--‡æ¡£
\texttt{\ sample.typ\ }
\end{quote}

\begin{itemize}
\item
  å\ldots³äºŽæ~‡é¢˜ï¼šTypst使ç''¨ \texttt{\ =\ }
  作为æ~‡é¢˜çš„指示符。本模æ?¿ä¸­ï¼Œä¸€çº§æ~‡é¢˜éœ€è¦?手动ç¼--å?·ï¼ŒäºŒã€?三级æ~‡é¢˜åˆ™ä¸?需è¦?

\begin{Shaded}
\begin{Highlighting}[]
\NormalTok{= 第一章 一级标题}
\NormalTok{== 二级标题}
\NormalTok{=== 三级标题}
\end{Highlighting}
\end{Shaded}
\item
  段è?½çš„ç¼--写:直接è¾``å\ldots¥æ--‡æœ¬å?³å?¯ç¼--写å†\ldots 容,使ç''¨ä¸¤ä¸ªå›žè½¦å?¦èµ·ä¸€æ®µ

\begin{Shaded}
\begin{Highlighting}[]
\NormalTok{这是内容的第一段,}
\NormalTok{这仍旧是第一段的内容}

\NormalTok{多一个换行符号后另起为第二段}
\end{Highlighting}
\end{Shaded}
\item
  æ--‡å­---å†\ldots 容的基础æ~¼å¼?:使ç''¨ \texttt{\ *text*\ }
  åŒ\ldots 括ä½?çš„æ--‡å­---å?¯ä»¥åŠ~ç²---,使ç''¨
  \texttt{\ \_text\_\ }
  åŒ\ldots 括ä½?çš„æ--‡å­---å?¯ä»¥æ--œä½``æ--‡å­---,使ç''¨
  \texttt{\ \#u{[}text{]}\ }
  åŒ\ldots 括ä½?çš„æ--‡å­---å?¯ä»¥å®žçŽ°ä¸‹åˆ'线,使ç''¨
  \texttt{\ \#sub{[}text{]}\ }
  åŒ\ldots 括ä½?çš„æ--‡å­---ä¿®æ''¹ä¸ºä¸‹æ~‡ï¼Œä½¿ç''¨
  \texttt{\ \#super{[}text{]}\ }
  åŒ\ldots 括ä½?çš„æ--‡å­---ä¿®æ''¹ä¸ºä¸Šæ~‡

\begin{Shaded}
\begin{Highlighting}[]
\NormalTok{这里是*加粗文字*内容,}
\NormalTok{这里是\_斜体文字\_内容,}
\NormalTok{这里是\#u[下划线文字]内容,}
\NormalTok{这里是\#sub[下标文字]内容,}
\NormalTok{这里是\#super[上标文字]内容}
\end{Highlighting}
\end{Shaded}
\item
  使ç''¨ \texttt{\ \#highlight(fill:\ red){[}text{]}\ }
  高亮æ--‡å­---æ~‡è®°ï¼Œä½¿ç''¨ \texttt{\ fill\ }
  å?‚æ•°ä¿®æ''¹é«˜äº®é¢œè‰²

\begin{Shaded}
\begin{Highlighting}[]
\NormalTok{这一段文字\#highlight[还需要修改]}
\end{Highlighting}
\end{Shaded}
\item
  使ç''¨ \texttt{\ \#link("your\ link\ here"){[}text{]}\ }
  æ~‡è®°è¶\ldots é``¾æŽ¥

\begin{Shaded}
\begin{Highlighting}[]
\NormalTok{这里是\#link("https://typst.app/home")[Typst官方网站]}
\end{Highlighting}
\end{Shaded}
\item
  使ç''¨ \texttt{\ -\ } æˆ-- \texttt{\ +\ }
  使ç''¨æ---~åº?åˆ---表æˆ--有åº?åˆ---表,使ç''¨ \texttt{\ Tab\ }
  缩进为二级åˆ---表

\begin{Shaded}
\begin{Highlighting}[]
\NormalTok{{-} 无序列表1}
\NormalTok{    + 有序列表1}
\NormalTok{    + 有序列表2}
\NormalTok{{-} 无序列表2}
\end{Highlighting}
\end{Shaded}
\item
  使ç''¨ \texttt{\ \#h(2em)\ } æˆ-- \texttt{\ \#v(1em)\ }
  æ?¥æ·»åŠ~æ°´å¹³æˆ--垂直的空白缩进,括å?·ä¸­çš„å?‚数为需è¦?空出的è·?离,å\ldots¶å?•ä½?å?¯ä»¥ä½¿ç''¨
  \texttt{\ \%\ } (页é?¢ç™¾åˆ†æ¯''), \texttt{\ pt\ }
  (点,å?°åˆ·è¡Œä¸šçš„ç»?对长度å?•ä½?,相å½``于1/72英寸),
  \texttt{\ em\ } (å­---符长度,相对于å½``å‰?å­---符大å°?)等

\begin{Shaded}
\begin{Highlighting}[]
\NormalTok{\#h(2em)默认格式会首行缩进两字符,额外添加会再次缩进}
\end{Highlighting}
\end{Shaded}
\item
  模æ?¿æ''¯æŒ?自动汇总图表目录,å?¯ä»¥ä½¿ç''¨
  \texttt{\ \#figure(image())\ } å`½ä»¤æ?¥æ·»åŠ~图片æˆ--使ç''¨
  \texttt{\ \#booktab()\ }
  æ·»åŠ~表æ~¼ï¼Œåœ¨å¡«å†™æ---¢å®šçš„é\ldots?置项å?Žå?¯æˆ?功渲æŸ``并自动汇总目录页,å?¯ä»¥ä½¿ç''¨
  \texttt{\ @\ } 引ç''¨å›¾è¡¨

\begin{Shaded}
\begin{Highlighting}[]
\NormalTok{\#figure(}
\NormalTok{    image(\textquotesingle{}./image/path.jpg\textquotesingle{}, width: 90\%),}
\NormalTok{    kind: image, }
\NormalTok{    supplement: [图],}
\NormalTok{    caption: [图片的标题],}
\NormalTok{)\textless{}img1\textgreater{}}

\NormalTok{\#booktab(}
\NormalTok{    width:60\%,}
\NormalTok{    columns: (20\%, 1fr, 2fr, 3fr),}
\NormalTok{    caption: [这里填写表格名称],}
\NormalTok{    kind: table, }
\NormalTok{    [1], [2], [3], [4],}
\NormalTok{    [a], [b], [c], [d],}
\NormalTok{    [e], [f], [g], [h],}
\NormalTok{    [i], [j], [k], [l]}
\NormalTok{)\textless{}tab1\textgreater{}}
\end{Highlighting}
\end{Shaded}
\item
  使ç''¨ \texttt{\ \$\ } ç¼--写数学å\ldots¬å¼?, \texttt{\ \$\ }
  符紧跟å†\ldots 容æ---¶ä¸ºè¡Œå†\ldots å\ldots¬å¼?,添åŠ~空æ~¼å?Žä¸ºè¡Œé---´å\ldots¬å¼?,å\ldots¬å¼?çš„å\ldots·ä½``规则å'Œ
  \href{https://typst.app/docs/reference/symbols/sym/}{符�} �以查
  \href{https://typst.app/docs/reference/math/}{帮助æ--‡æ¡£}

\begin{Shaded}
\begin{Highlighting}[]
\NormalTok{泰勒展开式(行内):}
\NormalTok{$f(x)= sum\_(n=0)\^{}(infinity) (f\^{}(n)(x\_0))/(n!) (x{-}x\_0)\^{}n$}

\NormalTok{欧拉公式(行间):}
\NormalTok{$ e\^{}(i theta) = cos theta + i sin theta \textbackslash{} e\^{}i pi + 1 = 0 $}
\end{Highlighting}
\end{Shaded}
\item
  使ç''¨
  \texttt{\ \textasciigrave{}\textasciigrave{}\textasciigrave{}code\textasciigrave{}\textasciigrave{}\textasciigrave{}\ }
  æ~‡è¯†ç¬¦è¾``å\ldots¥ä»£ç~?,Typstå?¯ä»¥æ¸²æŸ``ã€?显示代ç~?框,如果指定了语言类型,å?¯ä»¥æ~¹æ?®å\ldots¶è¯­æ³•æ~¼å¼?进行风æ~¼æ¸²æŸ``;使ç''¨å?•ä¸ªç¬¦å?·ä½¿ç''¨è¡Œå†\ldots 代ç~?

\begin{Shaded}
\begin{Highlighting}[]
\NormalTok{  \textasciigrave{}\textasciigrave{}\textasciigrave{}python}
\NormalTok{    print("hello world");}
\NormalTok{  \textasciigrave{}\textasciigrave{}\textasciigrave{}}
\NormalTok{同样支持行内代码\textasciigrave{}hello world\textasciigrave{}}
\end{Highlighting}
\end{Shaded}
\item
  ä¿®æ''¹ \texttt{\ ref\textbackslash{}acronyms.json\ }
  æ--‡ä»¶æ·»åŠ~缩略è¯?表,并使ç''¨ \texttt{\ \#acro("keyword1")\ }
  å`½ä»¤åœ¨æ--‡ä¸­å¼•å\ldots¥ç¼©ç•¥è¯?å\ldots¨ç§°ï¼Œåœ¨å¼•å\ldots¥å?Žä¼šè‡ªåŠ¨æ~¹æ?®jsonæ--‡ä»¶ä¸­ä¿¡æ?¯ï¼ŒæŽ'åº?å?Žæ·»åŠ~到缩略è¯?表中

\begin{Shaded}
\begin{Highlighting}[]
\FunctionTok{\{}
    \DataTypeTok{"keyword1"}\FunctionTok{:}\OtherTok{[}\StringTok{"英文缩写1"}\OtherTok{,} \StringTok{"英文全称1"}\OtherTok{,} \StringTok{"中文翻译1"}\OtherTok{]}\FunctionTok{,}
    \DataTypeTok{"keyword2"}\FunctionTok{:}\OtherTok{[}\StringTok{"英文缩写2"}\OtherTok{,} \StringTok{"英文全称2"}\OtherTok{,} \StringTok{"中文翻译2"}\OtherTok{]}
\FunctionTok{\}}
\end{Highlighting}
\end{Shaded}

\begin{Shaded}
\begin{Highlighting}[]
\NormalTok{在正文中可以使用\textasciigrave{}\#acro\textasciigrave{}命令引入缩略词(\#acro("ac"))。}
\end{Highlighting}
\end{Shaded}
\item
  使ç''¨ \texttt{\ \#{[}bibliography{]}()\ }
  æ·»åŠ~å?‚考æ--‡çŒ®ï¼Œæ‹¬å?·ä¸­éœ€è¦?填写 \texttt{\ .bib\ }
  æ~¼å¼?çš„å?‚考æ--‡çŒ®åˆ---表,在æ--‡ä¸­ä½¿ç''¨
  \texttt{\ @citationKey\ } 引ç''¨ï¼Œå\ldots·ä½``çš„ä¿¡æ?¯è§?
  \href{https://typst.app/docs/reference/model/bibliography/}{帮助æ--‡æ¡£}

  PS:å?¯ä»¥ä½¿ç''¨ \texttt{\ zotero+Better\ BibTex\ }
  自动导出/æ›´æ--° \texttt{\ .bib\ } æ~¼å¼?çš„å?‚考æ--‡çŒ®åˆ---表

  PS:在添åŠ~bib的代ç~?å?Žé?¢ï¼Œéš?è---?了一个heading,请ä¸?è¦?åˆ~除这一行,å?¦åˆ™å?‚考æ--‡çŒ®çš„页眉会出é''™

  PS:æ~¹æ?®å­¦é™¢è¦?求默认使ç''¨EmboJçš„æ~¼å¼?,如果需è¦?å\ldots¶ä»--æ~¼å¼?,å?ªè¦?下载到æ~¼å¼?说明
  \texttt{\ .csl\ } æ--‡ä»¶ä¿®æ''¹å?‚æ•°å?³å?¯
\item
  å½``æ--‡æœ¬å†\ldots 容ä»\ldots 有1页æ---¶ï¼Œæœ‰æ---¶é¡µçœ‰æ~‡é¢˜ä¼šå‡ºé''™ï¼Œå?¯ä»¥æ·»åŠ~一个空白æ~‡é¢˜è¿›è¡Œä¿®æ­£

\begin{Shaded}
\begin{Highlighting}[]
\NormalTok{\#heading(level: 6, numbering: none, outlined: false)[]}
\end{Highlighting}
\end{Shaded}
\end{itemize}

\subsubsection{致谢}\label{uxe8uxe8}

本模æ?¿åœ¨ç¼--写过程中å?‚考并学ä¹~了Typst模æ?¿çš„部分代ç~?,在这里统一致谢。

\href{/app?template=universal-cau-thesis&version=0.1.0}{Create project
in app}

\subsubsection{How to use}\label{how-to-use}

Click the button above to create a new project using this template in
the Typst app.

You can also use the Typst CLI to start a new project on your computer
using this command:

\begin{verbatim}
typst init @preview/universal-cau-thesis:0.1.0
\end{verbatim}

\includesvg[width=0.16667in,height=0.16667in]{/assets/icons/16-copy.svg}

\subsubsection{About}\label{about}

\begin{description}
\tightlist
\item[Author :]
JWangL5
\item[License:]
MIT
\item[Current version:]
0.1.0
\item[Last updated:]
May 27, 2024
\item[First released:]
May 27, 2024
\item[Archive size:]
107 kB
\href{https://packages.typst.org/preview/universal-cau-thesis-0.1.0.tar.gz}{\pandocbounded{\includesvg[keepaspectratio]{/assets/icons/16-download.svg}}}
\item[Repository:]
\href{https://github.com/JWangL5/CAU-ThesisTemplate-Typst}{GitHub}
\item[Categor y :]
\begin{itemize}
\tightlist
\item[]
\item
  \pandocbounded{\includesvg[keepaspectratio]{/assets/icons/16-mortarboard.svg}}
  \href{https://typst.app/universe/search/?category=thesis}{Thesis}
\end{itemize}
\end{description}

\subsubsection{Where to report issues?}\label{where-to-report-issues}

This template is a project of JWangL5 . Report issues on
\href{https://github.com/JWangL5/CAU-ThesisTemplate-Typst}{their
repository} . You can also try to ask for help with this template on the
\href{https://forum.typst.app}{Forum} .

Please report this template to the Typst team using the
\href{https://typst.app/contact}{contact form} if you believe it is a
safety hazard or infringes upon your rights.

\phantomsection\label{versions}
\subsubsection{Version history}\label{version-history}

\begin{longtable}[]{@{}ll@{}}
\toprule\noalign{}
Version & Release Date \\
\midrule\noalign{}
\endhead
\bottomrule\noalign{}
\endlastfoot
0.1.0 & May 27, 2024 \\
\end{longtable}

Typst GmbH did not create this template and cannot guarantee correct
functionality of this template or compatibility with any version of the
Typst compiler or app.


\title{typst.app/universe/package/octique}

\phantomsection\label{banner}
\section{octique}\label{octique}

{ 0.1.0 }

GitHub Octicons for Typst.

\phantomsection\label{readme}
GitHub \href{https://primer.style/foundations/icons/}{Octicons} for
Typst.

\subsection{Installation}\label{installation}

\begin{Shaded}
\begin{Highlighting}[]
\NormalTok{\#import "@preview/octique:0.1.0": *}
\end{Highlighting}
\end{Shaded}

\subsection{Usage}\label{usage}

\begin{Shaded}
\begin{Highlighting}[]
\NormalTok{// Returns an image for the given name.}
\NormalTok{octique(name, color: rgb("\#000000"), width: 1em, height: 1em)}

\NormalTok{// Returns a boxed image for the given name.}
\NormalTok{octique{-}inline(name, color: rgb("\#000000"), width: 1em, height: 1em, baseline: 25\%)}

\NormalTok{// Returns an SVG text for the given name.}
\NormalTok{octique{-}svg(name)}
\end{Highlighting}
\end{Shaded}

\subsection{List of Available Icons}\label{list-of-available-icons}

See also
\href{https://github.com/typst/packages/raw/main/packages/preview/octique/0.1.0/sample/sample.pdf}{\texttt{\ sample/sample.pdf\ }}
.

\begin{longtable}[]{@{}lc@{}}
\toprule\noalign{}
Code & Icon \\
\midrule\noalign{}
\endhead
\bottomrule\noalign{}
\endlastfoot
\texttt{\ \#octique("accessibility-inset")\ } &
\pandocbounded{\includesvg[keepaspectratio]{https://github.com/0x6b/typst-octique/wiki/assets/accessibility-inset.svg}} \\
\texttt{\ \#octique("accessibility")\ } &
\pandocbounded{\includesvg[keepaspectratio]{https://github.com/0x6b/typst-octique/wiki/assets/accessibility.svg}} \\
\texttt{\ \#octique("alert-fill")\ } &
\pandocbounded{\includesvg[keepaspectratio]{https://github.com/0x6b/typst-octique/wiki/assets/alert-fill.svg}} \\
\texttt{\ \#octique("alert")\ } &
\pandocbounded{\includesvg[keepaspectratio]{https://github.com/0x6b/typst-octique/wiki/assets/alert.svg}} \\
\texttt{\ \#octique("apps")\ } &
\pandocbounded{\includesvg[keepaspectratio]{https://github.com/0x6b/typst-octique/wiki/assets/apps.svg}} \\
\texttt{\ \#octique("archive")\ } &
\pandocbounded{\includesvg[keepaspectratio]{https://github.com/0x6b/typst-octique/wiki/assets/archive.svg}} \\
\texttt{\ \#octique("arrow-both")\ } &
\pandocbounded{\includesvg[keepaspectratio]{https://github.com/0x6b/typst-octique/wiki/assets/arrow-both.svg}} \\
\texttt{\ \#octique("arrow-down-left")\ } &
\pandocbounded{\includesvg[keepaspectratio]{https://github.com/0x6b/typst-octique/wiki/assets/arrow-down-left.svg}} \\
\texttt{\ \#octique("arrow-down-right")\ } &
\pandocbounded{\includesvg[keepaspectratio]{https://github.com/0x6b/typst-octique/wiki/assets/arrow-down-right.svg}} \\
\texttt{\ \#octique("arrow-down")\ } &
\pandocbounded{\includesvg[keepaspectratio]{https://github.com/0x6b/typst-octique/wiki/assets/arrow-down.svg}} \\
\texttt{\ \#octique("arrow-left")\ } &
\pandocbounded{\includesvg[keepaspectratio]{https://github.com/0x6b/typst-octique/wiki/assets/arrow-left.svg}} \\
\texttt{\ \#octique("arrow-right")\ } &
\pandocbounded{\includesvg[keepaspectratio]{https://github.com/0x6b/typst-octique/wiki/assets/arrow-right.svg}} \\
\texttt{\ \#octique("arrow-switch")\ } &
\pandocbounded{\includesvg[keepaspectratio]{https://github.com/0x6b/typst-octique/wiki/assets/arrow-switch.svg}} \\
\texttt{\ \#octique("arrow-up-left")\ } &
\pandocbounded{\includesvg[keepaspectratio]{https://github.com/0x6b/typst-octique/wiki/assets/arrow-up-left.svg}} \\
\texttt{\ \#octique("arrow-up-right")\ } &
\pandocbounded{\includesvg[keepaspectratio]{https://github.com/0x6b/typst-octique/wiki/assets/arrow-up-right.svg}} \\
\texttt{\ \#octique("arrow-up")\ } &
\pandocbounded{\includesvg[keepaspectratio]{https://github.com/0x6b/typst-octique/wiki/assets/arrow-up.svg}} \\
\texttt{\ \#octique("beaker")\ } &
\pandocbounded{\includesvg[keepaspectratio]{https://github.com/0x6b/typst-octique/wiki/assets/beaker.svg}} \\
\texttt{\ \#octique("bell-fill")\ } &
\pandocbounded{\includesvg[keepaspectratio]{https://github.com/0x6b/typst-octique/wiki/assets/bell-fill.svg}} \\
\texttt{\ \#octique("bell-slash")\ } &
\pandocbounded{\includesvg[keepaspectratio]{https://github.com/0x6b/typst-octique/wiki/assets/bell-slash.svg}} \\
\texttt{\ \#octique("bell")\ } &
\pandocbounded{\includesvg[keepaspectratio]{https://github.com/0x6b/typst-octique/wiki/assets/bell.svg}} \\
\texttt{\ \#octique("blocked")\ } &
\pandocbounded{\includesvg[keepaspectratio]{https://github.com/0x6b/typst-octique/wiki/assets/blocked.svg}} \\
\texttt{\ \#octique("bold")\ } &
\pandocbounded{\includesvg[keepaspectratio]{https://github.com/0x6b/typst-octique/wiki/assets/bold.svg}} \\
\texttt{\ \#octique("book")\ } &
\pandocbounded{\includesvg[keepaspectratio]{https://github.com/0x6b/typst-octique/wiki/assets/book.svg}} \\
\texttt{\ \#octique("bookmark-slash")\ } &
\pandocbounded{\includesvg[keepaspectratio]{https://github.com/0x6b/typst-octique/wiki/assets/bookmark-slash.svg}} \\
\texttt{\ \#octique("bookmark")\ } &
\pandocbounded{\includesvg[keepaspectratio]{https://github.com/0x6b/typst-octique/wiki/assets/bookmark.svg}} \\
\texttt{\ \#octique("briefcase")\ } &
\pandocbounded{\includesvg[keepaspectratio]{https://github.com/0x6b/typst-octique/wiki/assets/briefcase.svg}} \\
\texttt{\ \#octique("broadcast")\ } &
\pandocbounded{\includesvg[keepaspectratio]{https://github.com/0x6b/typst-octique/wiki/assets/broadcast.svg}} \\
\texttt{\ \#octique("browser")\ } &
\pandocbounded{\includesvg[keepaspectratio]{https://github.com/0x6b/typst-octique/wiki/assets/browser.svg}} \\
\texttt{\ \#octique("bug")\ } &
\pandocbounded{\includesvg[keepaspectratio]{https://github.com/0x6b/typst-octique/wiki/assets/bug.svg}} \\
\texttt{\ \#octique("cache")\ } &
\pandocbounded{\includesvg[keepaspectratio]{https://github.com/0x6b/typst-octique/wiki/assets/cache.svg}} \\
\texttt{\ \#octique("calendar")\ } &
\pandocbounded{\includesvg[keepaspectratio]{https://github.com/0x6b/typst-octique/wiki/assets/calendar.svg}} \\
\texttt{\ \#octique("check-circle-fill")\ } &
\pandocbounded{\includesvg[keepaspectratio]{https://github.com/0x6b/typst-octique/wiki/assets/check-circle-fill.svg}} \\
\texttt{\ \#octique("check-circle")\ } &
\pandocbounded{\includesvg[keepaspectratio]{https://github.com/0x6b/typst-octique/wiki/assets/check-circle.svg}} \\
\texttt{\ \#octique("check")\ } &
\pandocbounded{\includesvg[keepaspectratio]{https://github.com/0x6b/typst-octique/wiki/assets/check.svg}} \\
\texttt{\ \#octique("checkbox")\ } &
\pandocbounded{\includesvg[keepaspectratio]{https://github.com/0x6b/typst-octique/wiki/assets/checkbox.svg}} \\
\texttt{\ \#octique("checklist")\ } &
\pandocbounded{\includesvg[keepaspectratio]{https://github.com/0x6b/typst-octique/wiki/assets/checklist.svg}} \\
\texttt{\ \#octique("chevron-down")\ } &
\pandocbounded{\includesvg[keepaspectratio]{https://github.com/0x6b/typst-octique/wiki/assets/chevron-down.svg}} \\
\texttt{\ \#octique("chevron-left")\ } &
\pandocbounded{\includesvg[keepaspectratio]{https://github.com/0x6b/typst-octique/wiki/assets/chevron-left.svg}} \\
\texttt{\ \#octique("chevron-right")\ } &
\pandocbounded{\includesvg[keepaspectratio]{https://github.com/0x6b/typst-octique/wiki/assets/chevron-right.svg}} \\
\texttt{\ \#octique("chevron-up")\ } &
\pandocbounded{\includesvg[keepaspectratio]{https://github.com/0x6b/typst-octique/wiki/assets/chevron-up.svg}} \\
\texttt{\ \#octique("circle-slash")\ } &
\pandocbounded{\includesvg[keepaspectratio]{https://github.com/0x6b/typst-octique/wiki/assets/circle-slash.svg}} \\
\texttt{\ \#octique("circle")\ } &
\pandocbounded{\includesvg[keepaspectratio]{https://github.com/0x6b/typst-octique/wiki/assets/circle.svg}} \\
\texttt{\ \#octique("clock-fill")\ } &
\pandocbounded{\includesvg[keepaspectratio]{https://github.com/0x6b/typst-octique/wiki/assets/clock-fill.svg}} \\
\texttt{\ \#octique("clock")\ } &
\pandocbounded{\includesvg[keepaspectratio]{https://github.com/0x6b/typst-octique/wiki/assets/clock.svg}} \\
\texttt{\ \#octique("cloud-offline")\ } &
\pandocbounded{\includesvg[keepaspectratio]{https://github.com/0x6b/typst-octique/wiki/assets/cloud-offline.svg}} \\
\texttt{\ \#octique("cloud")\ } &
\pandocbounded{\includesvg[keepaspectratio]{https://github.com/0x6b/typst-octique/wiki/assets/cloud.svg}} \\
\texttt{\ \#octique("code-of-conduct")\ } &
\pandocbounded{\includesvg[keepaspectratio]{https://github.com/0x6b/typst-octique/wiki/assets/code-of-conduct.svg}} \\
\texttt{\ \#octique("code-review")\ } &
\pandocbounded{\includesvg[keepaspectratio]{https://github.com/0x6b/typst-octique/wiki/assets/code-review.svg}} \\
\texttt{\ \#octique("code")\ } &
\pandocbounded{\includesvg[keepaspectratio]{https://github.com/0x6b/typst-octique/wiki/assets/code.svg}} \\
\texttt{\ \#octique("code-square")\ } &
\pandocbounded{\includesvg[keepaspectratio]{https://github.com/0x6b/typst-octique/wiki/assets/code-square.svg}} \\
\texttt{\ \#octique("codescan-checkmark")\ } &
\pandocbounded{\includesvg[keepaspectratio]{https://github.com/0x6b/typst-octique/wiki/assets/codescan-checkmark.svg}} \\
\texttt{\ \#octique("codescan")\ } &
\pandocbounded{\includesvg[keepaspectratio]{https://github.com/0x6b/typst-octique/wiki/assets/codescan.svg}} \\
\texttt{\ \#octique("codespaces")\ } &
\pandocbounded{\includesvg[keepaspectratio]{https://github.com/0x6b/typst-octique/wiki/assets/codespaces.svg}} \\
\texttt{\ \#octique("columns")\ } &
\pandocbounded{\includesvg[keepaspectratio]{https://github.com/0x6b/typst-octique/wiki/assets/columns.svg}} \\
\texttt{\ \#octique("command-palette")\ } &
\pandocbounded{\includesvg[keepaspectratio]{https://github.com/0x6b/typst-octique/wiki/assets/command-palette.svg}} \\
\texttt{\ \#octique("comment-discussion")\ } &
\pandocbounded{\includesvg[keepaspectratio]{https://github.com/0x6b/typst-octique/wiki/assets/comment-discussion.svg}} \\
\texttt{\ \#octique("comment")\ } &
\pandocbounded{\includesvg[keepaspectratio]{https://github.com/0x6b/typst-octique/wiki/assets/comment.svg}} \\
\texttt{\ \#octique("container")\ } &
\pandocbounded{\includesvg[keepaspectratio]{https://github.com/0x6b/typst-octique/wiki/assets/container.svg}} \\
\texttt{\ \#octique("copilot-error")\ } &
\pandocbounded{\includesvg[keepaspectratio]{https://github.com/0x6b/typst-octique/wiki/assets/copilot-error.svg}} \\
\texttt{\ \#octique("copilot")\ } &
\pandocbounded{\includesvg[keepaspectratio]{https://github.com/0x6b/typst-octique/wiki/assets/copilot.svg}} \\
\texttt{\ \#octique("copilot-warning")\ } &
\pandocbounded{\includesvg[keepaspectratio]{https://github.com/0x6b/typst-octique/wiki/assets/copilot-warning.svg}} \\
\texttt{\ \#octique("copy")\ } &
\pandocbounded{\includesvg[keepaspectratio]{https://github.com/0x6b/typst-octique/wiki/assets/copy.svg}} \\
\texttt{\ \#octique("cpu")\ } &
\pandocbounded{\includesvg[keepaspectratio]{https://github.com/0x6b/typst-octique/wiki/assets/cpu.svg}} \\
\texttt{\ \#octique("credit-card")\ } &
\pandocbounded{\includesvg[keepaspectratio]{https://github.com/0x6b/typst-octique/wiki/assets/credit-card.svg}} \\
\texttt{\ \#octique("cross-reference")\ } &
\pandocbounded{\includesvg[keepaspectratio]{https://github.com/0x6b/typst-octique/wiki/assets/cross-reference.svg}} \\
\texttt{\ \#octique("dash")\ } &
\pandocbounded{\includesvg[keepaspectratio]{https://github.com/0x6b/typst-octique/wiki/assets/dash.svg}} \\
\texttt{\ \#octique("database")\ } &
\pandocbounded{\includesvg[keepaspectratio]{https://github.com/0x6b/typst-octique/wiki/assets/database.svg}} \\
\texttt{\ \#octique("dependabot")\ } &
\pandocbounded{\includesvg[keepaspectratio]{https://github.com/0x6b/typst-octique/wiki/assets/dependabot.svg}} \\
\texttt{\ \#octique("desktop-download")\ } &
\pandocbounded{\includesvg[keepaspectratio]{https://github.com/0x6b/typst-octique/wiki/assets/desktop-download.svg}} \\
\texttt{\ \#octique("device-camera")\ } &
\pandocbounded{\includesvg[keepaspectratio]{https://github.com/0x6b/typst-octique/wiki/assets/device-camera.svg}} \\
\texttt{\ \#octique("device-camera-video")\ } &
\pandocbounded{\includesvg[keepaspectratio]{https://github.com/0x6b/typst-octique/wiki/assets/device-camera-video.svg}} \\
\texttt{\ \#octique("device-desktop")\ } &
\pandocbounded{\includesvg[keepaspectratio]{https://github.com/0x6b/typst-octique/wiki/assets/device-desktop.svg}} \\
\texttt{\ \#octique("device-mobile")\ } &
\pandocbounded{\includesvg[keepaspectratio]{https://github.com/0x6b/typst-octique/wiki/assets/device-mobile.svg}} \\
\texttt{\ \#octique("devices")\ } &
\pandocbounded{\includesvg[keepaspectratio]{https://github.com/0x6b/typst-octique/wiki/assets/devices.svg}} \\
\texttt{\ \#octique("diamond")\ } &
\pandocbounded{\includesvg[keepaspectratio]{https://github.com/0x6b/typst-octique/wiki/assets/diamond.svg}} \\
\texttt{\ \#octique("diff-added")\ } &
\pandocbounded{\includesvg[keepaspectratio]{https://github.com/0x6b/typst-octique/wiki/assets/diff-added.svg}} \\
\texttt{\ \#octique("diff-ignored")\ } &
\pandocbounded{\includesvg[keepaspectratio]{https://github.com/0x6b/typst-octique/wiki/assets/diff-ignored.svg}} \\
\texttt{\ \#octique("diff-modified")\ } &
\pandocbounded{\includesvg[keepaspectratio]{https://github.com/0x6b/typst-octique/wiki/assets/diff-modified.svg}} \\
\texttt{\ \#octique("diff-removed")\ } &
\pandocbounded{\includesvg[keepaspectratio]{https://github.com/0x6b/typst-octique/wiki/assets/diff-removed.svg}} \\
\texttt{\ \#octique("diff-renamed")\ } &
\pandocbounded{\includesvg[keepaspectratio]{https://github.com/0x6b/typst-octique/wiki/assets/diff-renamed.svg}} \\
\texttt{\ \#octique("diff")\ } &
\pandocbounded{\includesvg[keepaspectratio]{https://github.com/0x6b/typst-octique/wiki/assets/diff.svg}} \\
\texttt{\ \#octique("discussion-closed")\ } &
\pandocbounded{\includesvg[keepaspectratio]{https://github.com/0x6b/typst-octique/wiki/assets/discussion-closed.svg}} \\
\texttt{\ \#octique("discussion-duplicate")\ } &
\pandocbounded{\includesvg[keepaspectratio]{https://github.com/0x6b/typst-octique/wiki/assets/discussion-duplicate.svg}} \\
\texttt{\ \#octique("discussion-outdated")\ } &
\pandocbounded{\includesvg[keepaspectratio]{https://github.com/0x6b/typst-octique/wiki/assets/discussion-outdated.svg}} \\
\texttt{\ \#octique("dot-fill")\ } &
\pandocbounded{\includesvg[keepaspectratio]{https://github.com/0x6b/typst-octique/wiki/assets/dot-fill.svg}} \\
\texttt{\ \#octique("dot")\ } &
\pandocbounded{\includesvg[keepaspectratio]{https://github.com/0x6b/typst-octique/wiki/assets/dot.svg}} \\
\texttt{\ \#octique("download")\ } &
\pandocbounded{\includesvg[keepaspectratio]{https://github.com/0x6b/typst-octique/wiki/assets/download.svg}} \\
\texttt{\ \#octique("duplicate")\ } &
\pandocbounded{\includesvg[keepaspectratio]{https://github.com/0x6b/typst-octique/wiki/assets/duplicate.svg}} \\
\texttt{\ \#octique("ellipsis")\ } &
\pandocbounded{\includesvg[keepaspectratio]{https://github.com/0x6b/typst-octique/wiki/assets/ellipsis.svg}} \\
\texttt{\ \#octique("eye-closed")\ } &
\pandocbounded{\includesvg[keepaspectratio]{https://github.com/0x6b/typst-octique/wiki/assets/eye-closed.svg}} \\
\texttt{\ \#octique("eye")\ } &
\pandocbounded{\includesvg[keepaspectratio]{https://github.com/0x6b/typst-octique/wiki/assets/eye.svg}} \\
\texttt{\ \#octique("feed-discussion")\ } &
\pandocbounded{\includesvg[keepaspectratio]{https://github.com/0x6b/typst-octique/wiki/assets/feed-discussion.svg}} \\
\texttt{\ \#octique("feed-forked")\ } &
\pandocbounded{\includesvg[keepaspectratio]{https://github.com/0x6b/typst-octique/wiki/assets/feed-forked.svg}} \\
\texttt{\ \#octique("feed-heart")\ } &
\pandocbounded{\includesvg[keepaspectratio]{https://github.com/0x6b/typst-octique/wiki/assets/feed-heart.svg}} \\
\texttt{\ \#octique("feed-issue-closed")\ } &
\pandocbounded{\includesvg[keepaspectratio]{https://github.com/0x6b/typst-octique/wiki/assets/feed-issue-closed.svg}} \\
\texttt{\ \#octique("feed-issue-draft")\ } &
\pandocbounded{\includesvg[keepaspectratio]{https://github.com/0x6b/typst-octique/wiki/assets/feed-issue-draft.svg}} \\
\texttt{\ \#octique("feed-issue-open")\ } &
\pandocbounded{\includesvg[keepaspectratio]{https://github.com/0x6b/typst-octique/wiki/assets/feed-issue-open.svg}} \\
\texttt{\ \#octique("feed-issue-reopen")\ } &
\pandocbounded{\includesvg[keepaspectratio]{https://github.com/0x6b/typst-octique/wiki/assets/feed-issue-reopen.svg}} \\
\texttt{\ \#octique("feed-merged")\ } &
\pandocbounded{\includesvg[keepaspectratio]{https://github.com/0x6b/typst-octique/wiki/assets/feed-merged.svg}} \\
\texttt{\ \#octique("feed-person")\ } &
\pandocbounded{\includesvg[keepaspectratio]{https://github.com/0x6b/typst-octique/wiki/assets/feed-person.svg}} \\
\texttt{\ \#octique("feed-plus")\ } &
\pandocbounded{\includesvg[keepaspectratio]{https://github.com/0x6b/typst-octique/wiki/assets/feed-plus.svg}} \\
\texttt{\ \#octique("feed-public")\ } &
\pandocbounded{\includesvg[keepaspectratio]{https://github.com/0x6b/typst-octique/wiki/assets/feed-public.svg}} \\
\texttt{\ \#octique("feed-pull-request-closed")\ } &
\pandocbounded{\includesvg[keepaspectratio]{https://github.com/0x6b/typst-octique/wiki/assets/feed-pull-request-closed.svg}} \\
\texttt{\ \#octique("feed-pull-request-draft")\ } &
\pandocbounded{\includesvg[keepaspectratio]{https://github.com/0x6b/typst-octique/wiki/assets/feed-pull-request-draft.svg}} \\
\texttt{\ \#octique("feed-pull-request-open")\ } &
\pandocbounded{\includesvg[keepaspectratio]{https://github.com/0x6b/typst-octique/wiki/assets/feed-pull-request-open.svg}} \\
\texttt{\ \#octique("feed-repo")\ } &
\pandocbounded{\includesvg[keepaspectratio]{https://github.com/0x6b/typst-octique/wiki/assets/feed-repo.svg}} \\
\texttt{\ \#octique("feed-rocket")\ } &
\pandocbounded{\includesvg[keepaspectratio]{https://github.com/0x6b/typst-octique/wiki/assets/feed-rocket.svg}} \\
\texttt{\ \#octique("feed-star")\ } &
\pandocbounded{\includesvg[keepaspectratio]{https://github.com/0x6b/typst-octique/wiki/assets/feed-star.svg}} \\
\texttt{\ \#octique("feed-tag")\ } &
\pandocbounded{\includesvg[keepaspectratio]{https://github.com/0x6b/typst-octique/wiki/assets/feed-tag.svg}} \\
\texttt{\ \#octique("feed-trophy")\ } &
\pandocbounded{\includesvg[keepaspectratio]{https://github.com/0x6b/typst-octique/wiki/assets/feed-trophy.svg}} \\
\texttt{\ \#octique("file-added")\ } &
\pandocbounded{\includesvg[keepaspectratio]{https://github.com/0x6b/typst-octique/wiki/assets/file-added.svg}} \\
\texttt{\ \#octique("file-badge")\ } &
\pandocbounded{\includesvg[keepaspectratio]{https://github.com/0x6b/typst-octique/wiki/assets/file-badge.svg}} \\
\texttt{\ \#octique("file-binary")\ } &
\pandocbounded{\includesvg[keepaspectratio]{https://github.com/0x6b/typst-octique/wiki/assets/file-binary.svg}} \\
\texttt{\ \#octique("file-code")\ } &
\pandocbounded{\includesvg[keepaspectratio]{https://github.com/0x6b/typst-octique/wiki/assets/file-code.svg}} \\
\texttt{\ \#octique("file-diff")\ } &
\pandocbounded{\includesvg[keepaspectratio]{https://github.com/0x6b/typst-octique/wiki/assets/file-diff.svg}} \\
\texttt{\ \#octique("file-directory-fill")\ } &
\pandocbounded{\includesvg[keepaspectratio]{https://github.com/0x6b/typst-octique/wiki/assets/file-directory-fill.svg}} \\
\texttt{\ \#octique("file-directory-open-fill")\ } &
\pandocbounded{\includesvg[keepaspectratio]{https://github.com/0x6b/typst-octique/wiki/assets/file-directory-open-fill.svg}} \\
\texttt{\ \#octique("file-directory")\ } &
\pandocbounded{\includesvg[keepaspectratio]{https://github.com/0x6b/typst-octique/wiki/assets/file-directory.svg}} \\
\texttt{\ \#octique("file-directory-symlink")\ } &
\pandocbounded{\includesvg[keepaspectratio]{https://github.com/0x6b/typst-octique/wiki/assets/file-directory-symlink.svg}} \\
\texttt{\ \#octique("file-moved")\ } &
\pandocbounded{\includesvg[keepaspectratio]{https://github.com/0x6b/typst-octique/wiki/assets/file-moved.svg}} \\
\texttt{\ \#octique("file-removed")\ } &
\pandocbounded{\includesvg[keepaspectratio]{https://github.com/0x6b/typst-octique/wiki/assets/file-removed.svg}} \\
\texttt{\ \#octique("file")\ } &
\pandocbounded{\includesvg[keepaspectratio]{https://github.com/0x6b/typst-octique/wiki/assets/file.svg}} \\
\texttt{\ \#octique("file-submodule")\ } &
\pandocbounded{\includesvg[keepaspectratio]{https://github.com/0x6b/typst-octique/wiki/assets/file-submodule.svg}} \\
\texttt{\ \#octique("file-symlink-file")\ } &
\pandocbounded{\includesvg[keepaspectratio]{https://github.com/0x6b/typst-octique/wiki/assets/file-symlink-file.svg}} \\
\texttt{\ \#octique("file-zip")\ } &
\pandocbounded{\includesvg[keepaspectratio]{https://github.com/0x6b/typst-octique/wiki/assets/file-zip.svg}} \\
\texttt{\ \#octique("filter")\ } &
\pandocbounded{\includesvg[keepaspectratio]{https://github.com/0x6b/typst-octique/wiki/assets/filter.svg}} \\
\texttt{\ \#octique("fiscal-host")\ } &
\pandocbounded{\includesvg[keepaspectratio]{https://github.com/0x6b/typst-octique/wiki/assets/fiscal-host.svg}} \\
\texttt{\ \#octique("flame")\ } &
\pandocbounded{\includesvg[keepaspectratio]{https://github.com/0x6b/typst-octique/wiki/assets/flame.svg}} \\
\texttt{\ \#octique("fold-down")\ } &
\pandocbounded{\includesvg[keepaspectratio]{https://github.com/0x6b/typst-octique/wiki/assets/fold-down.svg}} \\
\texttt{\ \#octique("fold")\ } &
\pandocbounded{\includesvg[keepaspectratio]{https://github.com/0x6b/typst-octique/wiki/assets/fold.svg}} \\
\texttt{\ \#octique("fold-up")\ } &
\pandocbounded{\includesvg[keepaspectratio]{https://github.com/0x6b/typst-octique/wiki/assets/fold-up.svg}} \\
\texttt{\ \#octique("gear")\ } &
\pandocbounded{\includesvg[keepaspectratio]{https://github.com/0x6b/typst-octique/wiki/assets/gear.svg}} \\
\texttt{\ \#octique("gift")\ } &
\pandocbounded{\includesvg[keepaspectratio]{https://github.com/0x6b/typst-octique/wiki/assets/gift.svg}} \\
\texttt{\ \#octique("git-branch")\ } &
\pandocbounded{\includesvg[keepaspectratio]{https://github.com/0x6b/typst-octique/wiki/assets/git-branch.svg}} \\
\texttt{\ \#octique("git-commit")\ } &
\pandocbounded{\includesvg[keepaspectratio]{https://github.com/0x6b/typst-octique/wiki/assets/git-commit.svg}} \\
\texttt{\ \#octique("git-compare")\ } &
\pandocbounded{\includesvg[keepaspectratio]{https://github.com/0x6b/typst-octique/wiki/assets/git-compare.svg}} \\
\texttt{\ \#octique("git-merge-queue")\ } &
\pandocbounded{\includesvg[keepaspectratio]{https://github.com/0x6b/typst-octique/wiki/assets/git-merge-queue.svg}} \\
\texttt{\ \#octique("git-merge")\ } &
\pandocbounded{\includesvg[keepaspectratio]{https://github.com/0x6b/typst-octique/wiki/assets/git-merge.svg}} \\
\texttt{\ \#octique("git-pull-request-closed")\ } &
\pandocbounded{\includesvg[keepaspectratio]{https://github.com/0x6b/typst-octique/wiki/assets/git-pull-request-closed.svg}} \\
\texttt{\ \#octique("git-pull-request-draft")\ } &
\pandocbounded{\includesvg[keepaspectratio]{https://github.com/0x6b/typst-octique/wiki/assets/git-pull-request-draft.svg}} \\
\texttt{\ \#octique("git-pull-request")\ } &
\pandocbounded{\includesvg[keepaspectratio]{https://github.com/0x6b/typst-octique/wiki/assets/git-pull-request.svg}} \\
\texttt{\ \#octique("globe")\ } &
\pandocbounded{\includesvg[keepaspectratio]{https://github.com/0x6b/typst-octique/wiki/assets/globe.svg}} \\
\texttt{\ \#octique("goal")\ } &
\pandocbounded{\includesvg[keepaspectratio]{https://github.com/0x6b/typst-octique/wiki/assets/goal.svg}} \\
\texttt{\ \#octique("grabber")\ } &
\pandocbounded{\includesvg[keepaspectratio]{https://github.com/0x6b/typst-octique/wiki/assets/grabber.svg}} \\
\texttt{\ \#octique("graph")\ } &
\pandocbounded{\includesvg[keepaspectratio]{https://github.com/0x6b/typst-octique/wiki/assets/graph.svg}} \\
\texttt{\ \#octique("hash")\ } &
\pandocbounded{\includesvg[keepaspectratio]{https://github.com/0x6b/typst-octique/wiki/assets/hash.svg}} \\
\texttt{\ \#octique("heading")\ } &
\pandocbounded{\includesvg[keepaspectratio]{https://github.com/0x6b/typst-octique/wiki/assets/heading.svg}} \\
\texttt{\ \#octique("heart-fill")\ } &
\pandocbounded{\includesvg[keepaspectratio]{https://github.com/0x6b/typst-octique/wiki/assets/heart-fill.svg}} \\
\texttt{\ \#octique("heart")\ } &
\pandocbounded{\includesvg[keepaspectratio]{https://github.com/0x6b/typst-octique/wiki/assets/heart.svg}} \\
\texttt{\ \#octique("history")\ } &
\pandocbounded{\includesvg[keepaspectratio]{https://github.com/0x6b/typst-octique/wiki/assets/history.svg}} \\
\texttt{\ \#octique("home")\ } &
\pandocbounded{\includesvg[keepaspectratio]{https://github.com/0x6b/typst-octique/wiki/assets/home.svg}} \\
\texttt{\ \#octique("horizontal-rule")\ } &
\pandocbounded{\includesvg[keepaspectratio]{https://github.com/0x6b/typst-octique/wiki/assets/horizontal-rule.svg}} \\
\texttt{\ \#octique("hourglass")\ } &
\pandocbounded{\includesvg[keepaspectratio]{https://github.com/0x6b/typst-octique/wiki/assets/hourglass.svg}} \\
\texttt{\ \#octique("hubot")\ } &
\pandocbounded{\includesvg[keepaspectratio]{https://github.com/0x6b/typst-octique/wiki/assets/hubot.svg}} \\
\texttt{\ \#octique("id-badge")\ } &
\pandocbounded{\includesvg[keepaspectratio]{https://github.com/0x6b/typst-octique/wiki/assets/id-badge.svg}} \\
\texttt{\ \#octique("image")\ } &
\pandocbounded{\includesvg[keepaspectratio]{https://github.com/0x6b/typst-octique/wiki/assets/image.svg}} \\
\texttt{\ \#octique("inbox")\ } &
\pandocbounded{\includesvg[keepaspectratio]{https://github.com/0x6b/typst-octique/wiki/assets/inbox.svg}} \\
\texttt{\ \#octique("infinity")\ } &
\pandocbounded{\includesvg[keepaspectratio]{https://github.com/0x6b/typst-octique/wiki/assets/infinity.svg}} \\
\texttt{\ \#octique("info")\ } &
\pandocbounded{\includesvg[keepaspectratio]{https://github.com/0x6b/typst-octique/wiki/assets/info.svg}} \\
\texttt{\ \#octique("issue-closed")\ } &
\pandocbounded{\includesvg[keepaspectratio]{https://github.com/0x6b/typst-octique/wiki/assets/issue-closed.svg}} \\
\texttt{\ \#octique("issue-draft")\ } &
\pandocbounded{\includesvg[keepaspectratio]{https://github.com/0x6b/typst-octique/wiki/assets/issue-draft.svg}} \\
\texttt{\ \#octique("issue-opened")\ } &
\pandocbounded{\includesvg[keepaspectratio]{https://github.com/0x6b/typst-octique/wiki/assets/issue-opened.svg}} \\
\texttt{\ \#octique("issue-reopened")\ } &
\pandocbounded{\includesvg[keepaspectratio]{https://github.com/0x6b/typst-octique/wiki/assets/issue-reopened.svg}} \\
\texttt{\ \#octique("issue-tracked-by")\ } &
\pandocbounded{\includesvg[keepaspectratio]{https://github.com/0x6b/typst-octique/wiki/assets/issue-tracked-by.svg}} \\
\texttt{\ \#octique("issue-tracks")\ } &
\pandocbounded{\includesvg[keepaspectratio]{https://github.com/0x6b/typst-octique/wiki/assets/issue-tracks.svg}} \\
\texttt{\ \#octique("italic")\ } &
\pandocbounded{\includesvg[keepaspectratio]{https://github.com/0x6b/typst-octique/wiki/assets/italic.svg}} \\
\texttt{\ \#octique("iterations")\ } &
\pandocbounded{\includesvg[keepaspectratio]{https://github.com/0x6b/typst-octique/wiki/assets/iterations.svg}} \\
\texttt{\ \#octique("kebab-horizontal")\ } &
\pandocbounded{\includesvg[keepaspectratio]{https://github.com/0x6b/typst-octique/wiki/assets/kebab-horizontal.svg}} \\
\texttt{\ \#octique("key-asterisk")\ } &
\pandocbounded{\includesvg[keepaspectratio]{https://github.com/0x6b/typst-octique/wiki/assets/key-asterisk.svg}} \\
\texttt{\ \#octique("key")\ } &
\pandocbounded{\includesvg[keepaspectratio]{https://github.com/0x6b/typst-octique/wiki/assets/key.svg}} \\
\texttt{\ \#octique("law")\ } &
\pandocbounded{\includesvg[keepaspectratio]{https://github.com/0x6b/typst-octique/wiki/assets/law.svg}} \\
\texttt{\ \#octique("light-bulb")\ } &
\pandocbounded{\includesvg[keepaspectratio]{https://github.com/0x6b/typst-octique/wiki/assets/light-bulb.svg}} \\
\texttt{\ \#octique("link-external")\ } &
\pandocbounded{\includesvg[keepaspectratio]{https://github.com/0x6b/typst-octique/wiki/assets/link-external.svg}} \\
\texttt{\ \#octique("link")\ } &
\pandocbounded{\includesvg[keepaspectratio]{https://github.com/0x6b/typst-octique/wiki/assets/link.svg}} \\
\texttt{\ \#octique("list-ordered")\ } &
\pandocbounded{\includesvg[keepaspectratio]{https://github.com/0x6b/typst-octique/wiki/assets/list-ordered.svg}} \\
\texttt{\ \#octique("list-unordered")\ } &
\pandocbounded{\includesvg[keepaspectratio]{https://github.com/0x6b/typst-octique/wiki/assets/list-unordered.svg}} \\
\texttt{\ \#octique("location")\ } &
\pandocbounded{\includesvg[keepaspectratio]{https://github.com/0x6b/typst-octique/wiki/assets/location.svg}} \\
\texttt{\ \#octique("lock")\ } &
\pandocbounded{\includesvg[keepaspectratio]{https://github.com/0x6b/typst-octique/wiki/assets/lock.svg}} \\
\texttt{\ \#octique("log")\ } &
\pandocbounded{\includesvg[keepaspectratio]{https://github.com/0x6b/typst-octique/wiki/assets/log.svg}} \\
\texttt{\ \#octique("logo-gist")\ } &
\pandocbounded{\includesvg[keepaspectratio]{https://github.com/0x6b/typst-octique/wiki/assets/logo-gist.svg}} \\
\texttt{\ \#octique("logo-github")\ } &
\pandocbounded{\includesvg[keepaspectratio]{https://github.com/0x6b/typst-octique/wiki/assets/logo-github.svg}} \\
\texttt{\ \#octique("mail")\ } &
\pandocbounded{\includesvg[keepaspectratio]{https://github.com/0x6b/typst-octique/wiki/assets/mail.svg}} \\
\texttt{\ \#octique("mark-github")\ } &
\pandocbounded{\includesvg[keepaspectratio]{https://github.com/0x6b/typst-octique/wiki/assets/mark-github.svg}} \\
\texttt{\ \#octique("markdown")\ } &
\pandocbounded{\includesvg[keepaspectratio]{https://github.com/0x6b/typst-octique/wiki/assets/markdown.svg}} \\
\texttt{\ \#octique("megaphone")\ } &
\pandocbounded{\includesvg[keepaspectratio]{https://github.com/0x6b/typst-octique/wiki/assets/megaphone.svg}} \\
\texttt{\ \#octique("mention")\ } &
\pandocbounded{\includesvg[keepaspectratio]{https://github.com/0x6b/typst-octique/wiki/assets/mention.svg}} \\
\texttt{\ \#octique("meter")\ } &
\pandocbounded{\includesvg[keepaspectratio]{https://github.com/0x6b/typst-octique/wiki/assets/meter.svg}} \\
\texttt{\ \#octique("milestone")\ } &
\pandocbounded{\includesvg[keepaspectratio]{https://github.com/0x6b/typst-octique/wiki/assets/milestone.svg}} \\
\texttt{\ \#octique("mirror")\ } &
\pandocbounded{\includesvg[keepaspectratio]{https://github.com/0x6b/typst-octique/wiki/assets/mirror.svg}} \\
\texttt{\ \#octique("moon")\ } &
\pandocbounded{\includesvg[keepaspectratio]{https://github.com/0x6b/typst-octique/wiki/assets/moon.svg}} \\
\texttt{\ \#octique("mortar-board")\ } &
\pandocbounded{\includesvg[keepaspectratio]{https://github.com/0x6b/typst-octique/wiki/assets/mortar-board.svg}} \\
\texttt{\ \#octique("move-to-bottom")\ } &
\pandocbounded{\includesvg[keepaspectratio]{https://github.com/0x6b/typst-octique/wiki/assets/move-to-bottom.svg}} \\
\texttt{\ \#octique("move-to-end")\ } &
\pandocbounded{\includesvg[keepaspectratio]{https://github.com/0x6b/typst-octique/wiki/assets/move-to-end.svg}} \\
\texttt{\ \#octique("move-to-start")\ } &
\pandocbounded{\includesvg[keepaspectratio]{https://github.com/0x6b/typst-octique/wiki/assets/move-to-start.svg}} \\
\texttt{\ \#octique("move-to-top")\ } &
\pandocbounded{\includesvg[keepaspectratio]{https://github.com/0x6b/typst-octique/wiki/assets/move-to-top.svg}} \\
\texttt{\ \#octique("multi-select")\ } &
\pandocbounded{\includesvg[keepaspectratio]{https://github.com/0x6b/typst-octique/wiki/assets/multi-select.svg}} \\
\texttt{\ \#octique("mute")\ } &
\pandocbounded{\includesvg[keepaspectratio]{https://github.com/0x6b/typst-octique/wiki/assets/mute.svg}} \\
\texttt{\ \#octique("no-entry")\ } &
\pandocbounded{\includesvg[keepaspectratio]{https://github.com/0x6b/typst-octique/wiki/assets/no-entry.svg}} \\
\texttt{\ \#octique("north-star")\ } &
\pandocbounded{\includesvg[keepaspectratio]{https://github.com/0x6b/typst-octique/wiki/assets/north-star.svg}} \\
\texttt{\ \#octique("note")\ } &
\pandocbounded{\includesvg[keepaspectratio]{https://github.com/0x6b/typst-octique/wiki/assets/note.svg}} \\
\texttt{\ \#octique("number")\ } &
\pandocbounded{\includesvg[keepaspectratio]{https://github.com/0x6b/typst-octique/wiki/assets/number.svg}} \\
\texttt{\ \#octique("organization")\ } &
\pandocbounded{\includesvg[keepaspectratio]{https://github.com/0x6b/typst-octique/wiki/assets/organization.svg}} \\
\texttt{\ \#octique("package-dependencies")\ } &
\pandocbounded{\includesvg[keepaspectratio]{https://github.com/0x6b/typst-octique/wiki/assets/package-dependencies.svg}} \\
\texttt{\ \#octique("package-dependents")\ } &
\pandocbounded{\includesvg[keepaspectratio]{https://github.com/0x6b/typst-octique/wiki/assets/package-dependents.svg}} \\
\texttt{\ \#octique("package")\ } &
\pandocbounded{\includesvg[keepaspectratio]{https://github.com/0x6b/typst-octique/wiki/assets/package.svg}} \\
\texttt{\ \#octique("paintbrush")\ } &
\pandocbounded{\includesvg[keepaspectratio]{https://github.com/0x6b/typst-octique/wiki/assets/paintbrush.svg}} \\
\texttt{\ \#octique("paper-airplane")\ } &
\pandocbounded{\includesvg[keepaspectratio]{https://github.com/0x6b/typst-octique/wiki/assets/paper-airplane.svg}} \\
\texttt{\ \#octique("paperclip")\ } &
\pandocbounded{\includesvg[keepaspectratio]{https://github.com/0x6b/typst-octique/wiki/assets/paperclip.svg}} \\
\texttt{\ \#octique("passkey-fill")\ } &
\pandocbounded{\includesvg[keepaspectratio]{https://github.com/0x6b/typst-octique/wiki/assets/passkey-fill.svg}} \\
\texttt{\ \#octique("paste")\ } &
\pandocbounded{\includesvg[keepaspectratio]{https://github.com/0x6b/typst-octique/wiki/assets/paste.svg}} \\
\texttt{\ \#octique("pencil")\ } &
\pandocbounded{\includesvg[keepaspectratio]{https://github.com/0x6b/typst-octique/wiki/assets/pencil.svg}} \\
\texttt{\ \#octique("people")\ } &
\pandocbounded{\includesvg[keepaspectratio]{https://github.com/0x6b/typst-octique/wiki/assets/people.svg}} \\
\texttt{\ \#octique("person-add")\ } &
\pandocbounded{\includesvg[keepaspectratio]{https://github.com/0x6b/typst-octique/wiki/assets/person-add.svg}} \\
\texttt{\ \#octique("person-fill")\ } &
\pandocbounded{\includesvg[keepaspectratio]{https://github.com/0x6b/typst-octique/wiki/assets/person-fill.svg}} \\
\texttt{\ \#octique("person")\ } &
\pandocbounded{\includesvg[keepaspectratio]{https://github.com/0x6b/typst-octique/wiki/assets/person.svg}} \\
\texttt{\ \#octique("pin-slash")\ } &
\pandocbounded{\includesvg[keepaspectratio]{https://github.com/0x6b/typst-octique/wiki/assets/pin-slash.svg}} \\
\texttt{\ \#octique("pin")\ } &
\pandocbounded{\includesvg[keepaspectratio]{https://github.com/0x6b/typst-octique/wiki/assets/pin.svg}} \\
\texttt{\ \#octique("pivot-column")\ } &
\pandocbounded{\includesvg[keepaspectratio]{https://github.com/0x6b/typst-octique/wiki/assets/pivot-column.svg}} \\
\texttt{\ \#octique("play")\ } &
\pandocbounded{\includesvg[keepaspectratio]{https://github.com/0x6b/typst-octique/wiki/assets/play.svg}} \\
\texttt{\ \#octique("plug")\ } &
\pandocbounded{\includesvg[keepaspectratio]{https://github.com/0x6b/typst-octique/wiki/assets/plug.svg}} \\
\texttt{\ \#octique("plus-circle")\ } &
\pandocbounded{\includesvg[keepaspectratio]{https://github.com/0x6b/typst-octique/wiki/assets/plus-circle.svg}} \\
\texttt{\ \#octique("plus")\ } &
\pandocbounded{\includesvg[keepaspectratio]{https://github.com/0x6b/typst-octique/wiki/assets/plus.svg}} \\
\texttt{\ \#octique("project-roadmap")\ } &
\pandocbounded{\includesvg[keepaspectratio]{https://github.com/0x6b/typst-octique/wiki/assets/project-roadmap.svg}} \\
\texttt{\ \#octique("project")\ } &
\pandocbounded{\includesvg[keepaspectratio]{https://github.com/0x6b/typst-octique/wiki/assets/project.svg}} \\
\texttt{\ \#octique("project-symlink")\ } &
\pandocbounded{\includesvg[keepaspectratio]{https://github.com/0x6b/typst-octique/wiki/assets/project-symlink.svg}} \\
\texttt{\ \#octique("project-template")\ } &
\pandocbounded{\includesvg[keepaspectratio]{https://github.com/0x6b/typst-octique/wiki/assets/project-template.svg}} \\
\texttt{\ \#octique("pulse")\ } &
\pandocbounded{\includesvg[keepaspectratio]{https://github.com/0x6b/typst-octique/wiki/assets/pulse.svg}} \\
\texttt{\ \#octique("question")\ } &
\pandocbounded{\includesvg[keepaspectratio]{https://github.com/0x6b/typst-octique/wiki/assets/question.svg}} \\
\texttt{\ \#octique("quote")\ } &
\pandocbounded{\includesvg[keepaspectratio]{https://github.com/0x6b/typst-octique/wiki/assets/quote.svg}} \\
\texttt{\ \#octique("read")\ } &
\pandocbounded{\includesvg[keepaspectratio]{https://github.com/0x6b/typst-octique/wiki/assets/read.svg}} \\
\texttt{\ \#octique("redo")\ } &
\pandocbounded{\includesvg[keepaspectratio]{https://github.com/0x6b/typst-octique/wiki/assets/redo.svg}} \\
\texttt{\ \#octique("rel-file-path")\ } &
\pandocbounded{\includesvg[keepaspectratio]{https://github.com/0x6b/typst-octique/wiki/assets/rel-file-path.svg}} \\
\texttt{\ \#octique("reply")\ } &
\pandocbounded{\includesvg[keepaspectratio]{https://github.com/0x6b/typst-octique/wiki/assets/reply.svg}} \\
\texttt{\ \#octique("repo-clone")\ } &
\pandocbounded{\includesvg[keepaspectratio]{https://github.com/0x6b/typst-octique/wiki/assets/repo-clone.svg}} \\
\texttt{\ \#octique("repo-deleted")\ } &
\pandocbounded{\includesvg[keepaspectratio]{https://github.com/0x6b/typst-octique/wiki/assets/repo-deleted.svg}} \\
\texttt{\ \#octique("repo-forked")\ } &
\pandocbounded{\includesvg[keepaspectratio]{https://github.com/0x6b/typst-octique/wiki/assets/repo-forked.svg}} \\
\texttt{\ \#octique("repo-locked")\ } &
\pandocbounded{\includesvg[keepaspectratio]{https://github.com/0x6b/typst-octique/wiki/assets/repo-locked.svg}} \\
\texttt{\ \#octique("repo-pull")\ } &
\pandocbounded{\includesvg[keepaspectratio]{https://github.com/0x6b/typst-octique/wiki/assets/repo-pull.svg}} \\
\texttt{\ \#octique("repo-push")\ } &
\pandocbounded{\includesvg[keepaspectratio]{https://github.com/0x6b/typst-octique/wiki/assets/repo-push.svg}} \\
\texttt{\ \#octique("repo")\ } &
\pandocbounded{\includesvg[keepaspectratio]{https://github.com/0x6b/typst-octique/wiki/assets/repo.svg}} \\
\texttt{\ \#octique("repo-template")\ } &
\pandocbounded{\includesvg[keepaspectratio]{https://github.com/0x6b/typst-octique/wiki/assets/repo-template.svg}} \\
\texttt{\ \#octique("report")\ } &
\pandocbounded{\includesvg[keepaspectratio]{https://github.com/0x6b/typst-octique/wiki/assets/report.svg}} \\
\texttt{\ \#octique("rocket")\ } &
\pandocbounded{\includesvg[keepaspectratio]{https://github.com/0x6b/typst-octique/wiki/assets/rocket.svg}} \\
\texttt{\ \#octique("rows")\ } &
\pandocbounded{\includesvg[keepaspectratio]{https://github.com/0x6b/typst-octique/wiki/assets/rows.svg}} \\
\texttt{\ \#octique("rss")\ } &
\pandocbounded{\includesvg[keepaspectratio]{https://github.com/0x6b/typst-octique/wiki/assets/rss.svg}} \\
\texttt{\ \#octique("ruby")\ } &
\pandocbounded{\includesvg[keepaspectratio]{https://github.com/0x6b/typst-octique/wiki/assets/ruby.svg}} \\
\texttt{\ \#octique("screen-full")\ } &
\pandocbounded{\includesvg[keepaspectratio]{https://github.com/0x6b/typst-octique/wiki/assets/screen-full.svg}} \\
\texttt{\ \#octique("screen-normal")\ } &
\pandocbounded{\includesvg[keepaspectratio]{https://github.com/0x6b/typst-octique/wiki/assets/screen-normal.svg}} \\
\texttt{\ \#octique("search")\ } &
\pandocbounded{\includesvg[keepaspectratio]{https://github.com/0x6b/typst-octique/wiki/assets/search.svg}} \\
\texttt{\ \#octique("server")\ } &
\pandocbounded{\includesvg[keepaspectratio]{https://github.com/0x6b/typst-octique/wiki/assets/server.svg}} \\
\texttt{\ \#octique("share-android")\ } &
\pandocbounded{\includesvg[keepaspectratio]{https://github.com/0x6b/typst-octique/wiki/assets/share-android.svg}} \\
\texttt{\ \#octique("share")\ } &
\pandocbounded{\includesvg[keepaspectratio]{https://github.com/0x6b/typst-octique/wiki/assets/share.svg}} \\
\texttt{\ \#octique("shield-check")\ } &
\pandocbounded{\includesvg[keepaspectratio]{https://github.com/0x6b/typst-octique/wiki/assets/shield-check.svg}} \\
\texttt{\ \#octique("shield-lock")\ } &
\pandocbounded{\includesvg[keepaspectratio]{https://github.com/0x6b/typst-octique/wiki/assets/shield-lock.svg}} \\
\texttt{\ \#octique("shield-slash")\ } &
\pandocbounded{\includesvg[keepaspectratio]{https://github.com/0x6b/typst-octique/wiki/assets/shield-slash.svg}} \\
\texttt{\ \#octique("shield")\ } &
\pandocbounded{\includesvg[keepaspectratio]{https://github.com/0x6b/typst-octique/wiki/assets/shield.svg}} \\
\texttt{\ \#octique("shield-x")\ } &
\pandocbounded{\includesvg[keepaspectratio]{https://github.com/0x6b/typst-octique/wiki/assets/shield-x.svg}} \\
\texttt{\ \#octique("sidebar-collapse")\ } &
\pandocbounded{\includesvg[keepaspectratio]{https://github.com/0x6b/typst-octique/wiki/assets/sidebar-collapse.svg}} \\
\texttt{\ \#octique("sidebar-expand")\ } &
\pandocbounded{\includesvg[keepaspectratio]{https://github.com/0x6b/typst-octique/wiki/assets/sidebar-expand.svg}} \\
\texttt{\ \#octique("sign-in")\ } &
\pandocbounded{\includesvg[keepaspectratio]{https://github.com/0x6b/typst-octique/wiki/assets/sign-in.svg}} \\
\texttt{\ \#octique("sign-out")\ } &
\pandocbounded{\includesvg[keepaspectratio]{https://github.com/0x6b/typst-octique/wiki/assets/sign-out.svg}} \\
\texttt{\ \#octique("single-select")\ } &
\pandocbounded{\includesvg[keepaspectratio]{https://github.com/0x6b/typst-octique/wiki/assets/single-select.svg}} \\
\texttt{\ \#octique("skip-fill")\ } &
\pandocbounded{\includesvg[keepaspectratio]{https://github.com/0x6b/typst-octique/wiki/assets/skip-fill.svg}} \\
\texttt{\ \#octique("skip")\ } &
\pandocbounded{\includesvg[keepaspectratio]{https://github.com/0x6b/typst-octique/wiki/assets/skip.svg}} \\
\texttt{\ \#octique("sliders")\ } &
\pandocbounded{\includesvg[keepaspectratio]{https://github.com/0x6b/typst-octique/wiki/assets/sliders.svg}} \\
\texttt{\ \#octique("smiley")\ } &
\pandocbounded{\includesvg[keepaspectratio]{https://github.com/0x6b/typst-octique/wiki/assets/smiley.svg}} \\
\texttt{\ \#octique("sort-asc")\ } &
\pandocbounded{\includesvg[keepaspectratio]{https://github.com/0x6b/typst-octique/wiki/assets/sort-asc.svg}} \\
\texttt{\ \#octique("sort-desc")\ } &
\pandocbounded{\includesvg[keepaspectratio]{https://github.com/0x6b/typst-octique/wiki/assets/sort-desc.svg}} \\
\texttt{\ \#octique("sparkle-fill")\ } &
\pandocbounded{\includesvg[keepaspectratio]{https://github.com/0x6b/typst-octique/wiki/assets/sparkle-fill.svg}} \\
\texttt{\ \#octique("sponsor-tiers")\ } &
\pandocbounded{\includesvg[keepaspectratio]{https://github.com/0x6b/typst-octique/wiki/assets/sponsor-tiers.svg}} \\
\texttt{\ \#octique("square-fill")\ } &
\pandocbounded{\includesvg[keepaspectratio]{https://github.com/0x6b/typst-octique/wiki/assets/square-fill.svg}} \\
\texttt{\ \#octique("square")\ } &
\pandocbounded{\includesvg[keepaspectratio]{https://github.com/0x6b/typst-octique/wiki/assets/square.svg}} \\
\texttt{\ \#octique("squirrel")\ } &
\pandocbounded{\includesvg[keepaspectratio]{https://github.com/0x6b/typst-octique/wiki/assets/squirrel.svg}} \\
\texttt{\ \#octique("stack")\ } &
\pandocbounded{\includesvg[keepaspectratio]{https://github.com/0x6b/typst-octique/wiki/assets/stack.svg}} \\
\texttt{\ \#octique("star-fill")\ } &
\pandocbounded{\includesvg[keepaspectratio]{https://github.com/0x6b/typst-octique/wiki/assets/star-fill.svg}} \\
\texttt{\ \#octique("star")\ } &
\pandocbounded{\includesvg[keepaspectratio]{https://github.com/0x6b/typst-octique/wiki/assets/star.svg}} \\
\texttt{\ \#octique("stop")\ } &
\pandocbounded{\includesvg[keepaspectratio]{https://github.com/0x6b/typst-octique/wiki/assets/stop.svg}} \\
\texttt{\ \#octique("stopwatch")\ } &
\pandocbounded{\includesvg[keepaspectratio]{https://github.com/0x6b/typst-octique/wiki/assets/stopwatch.svg}} \\
\texttt{\ \#octique("strikethrough")\ } &
\pandocbounded{\includesvg[keepaspectratio]{https://github.com/0x6b/typst-octique/wiki/assets/strikethrough.svg}} \\
\texttt{\ \#octique("sun")\ } &
\pandocbounded{\includesvg[keepaspectratio]{https://github.com/0x6b/typst-octique/wiki/assets/sun.svg}} \\
\texttt{\ \#octique("sync")\ } &
\pandocbounded{\includesvg[keepaspectratio]{https://github.com/0x6b/typst-octique/wiki/assets/sync.svg}} \\
\texttt{\ \#octique("tab-external")\ } &
\pandocbounded{\includesvg[keepaspectratio]{https://github.com/0x6b/typst-octique/wiki/assets/tab-external.svg}} \\
\texttt{\ \#octique("table")\ } &
\pandocbounded{\includesvg[keepaspectratio]{https://github.com/0x6b/typst-octique/wiki/assets/table.svg}} \\
\texttt{\ \#octique("tag")\ } &
\pandocbounded{\includesvg[keepaspectratio]{https://github.com/0x6b/typst-octique/wiki/assets/tag.svg}} \\
\texttt{\ \#octique("tasklist")\ } &
\pandocbounded{\includesvg[keepaspectratio]{https://github.com/0x6b/typst-octique/wiki/assets/tasklist.svg}} \\
\texttt{\ \#octique("telescope-fill")\ } &
\pandocbounded{\includesvg[keepaspectratio]{https://github.com/0x6b/typst-octique/wiki/assets/telescope-fill.svg}} \\
\texttt{\ \#octique("telescope")\ } &
\pandocbounded{\includesvg[keepaspectratio]{https://github.com/0x6b/typst-octique/wiki/assets/telescope.svg}} \\
\texttt{\ \#octique("terminal")\ } &
\pandocbounded{\includesvg[keepaspectratio]{https://github.com/0x6b/typst-octique/wiki/assets/terminal.svg}} \\
\texttt{\ \#octique("three-bars")\ } &
\pandocbounded{\includesvg[keepaspectratio]{https://github.com/0x6b/typst-octique/wiki/assets/three-bars.svg}} \\
\texttt{\ \#octique("thumbsdown")\ } &
\pandocbounded{\includesvg[keepaspectratio]{https://github.com/0x6b/typst-octique/wiki/assets/thumbsdown.svg}} \\
\texttt{\ \#octique("thumbsup")\ } &
\pandocbounded{\includesvg[keepaspectratio]{https://github.com/0x6b/typst-octique/wiki/assets/thumbsup.svg}} \\
\texttt{\ \#octique("tools")\ } &
\pandocbounded{\includesvg[keepaspectratio]{https://github.com/0x6b/typst-octique/wiki/assets/tools.svg}} \\
\texttt{\ \#octique("tracked-by-closed-completed")\ } &
\pandocbounded{\includesvg[keepaspectratio]{https://github.com/0x6b/typst-octique/wiki/assets/tracked-by-closed-completed.svg}} \\
\texttt{\ \#octique("tracked-by-closed-not-planned")\ } &
\pandocbounded{\includesvg[keepaspectratio]{https://github.com/0x6b/typst-octique/wiki/assets/tracked-by-closed-not-planned.svg}} \\
\texttt{\ \#octique("trash")\ } &
\pandocbounded{\includesvg[keepaspectratio]{https://github.com/0x6b/typst-octique/wiki/assets/trash.svg}} \\
\texttt{\ \#octique("triangle-down")\ } &
\pandocbounded{\includesvg[keepaspectratio]{https://github.com/0x6b/typst-octique/wiki/assets/triangle-down.svg}} \\
\texttt{\ \#octique("triangle-left")\ } &
\pandocbounded{\includesvg[keepaspectratio]{https://github.com/0x6b/typst-octique/wiki/assets/triangle-left.svg}} \\
\texttt{\ \#octique("triangle-right")\ } &
\pandocbounded{\includesvg[keepaspectratio]{https://github.com/0x6b/typst-octique/wiki/assets/triangle-right.svg}} \\
\texttt{\ \#octique("triangle-up")\ } &
\pandocbounded{\includesvg[keepaspectratio]{https://github.com/0x6b/typst-octique/wiki/assets/triangle-up.svg}} \\
\texttt{\ \#octique("trophy")\ } &
\pandocbounded{\includesvg[keepaspectratio]{https://github.com/0x6b/typst-octique/wiki/assets/trophy.svg}} \\
\texttt{\ \#octique("typography")\ } &
\pandocbounded{\includesvg[keepaspectratio]{https://github.com/0x6b/typst-octique/wiki/assets/typography.svg}} \\
\texttt{\ \#octique("undo")\ } &
\pandocbounded{\includesvg[keepaspectratio]{https://github.com/0x6b/typst-octique/wiki/assets/undo.svg}} \\
\texttt{\ \#octique("unfold")\ } &
\pandocbounded{\includesvg[keepaspectratio]{https://github.com/0x6b/typst-octique/wiki/assets/unfold.svg}} \\
\texttt{\ \#octique("unlink")\ } &
\pandocbounded{\includesvg[keepaspectratio]{https://github.com/0x6b/typst-octique/wiki/assets/unlink.svg}} \\
\texttt{\ \#octique("unlock")\ } &
\pandocbounded{\includesvg[keepaspectratio]{https://github.com/0x6b/typst-octique/wiki/assets/unlock.svg}} \\
\texttt{\ \#octique("unmute")\ } &
\pandocbounded{\includesvg[keepaspectratio]{https://github.com/0x6b/typst-octique/wiki/assets/unmute.svg}} \\
\texttt{\ \#octique("unread")\ } &
\pandocbounded{\includesvg[keepaspectratio]{https://github.com/0x6b/typst-octique/wiki/assets/unread.svg}} \\
\texttt{\ \#octique("unverified")\ } &
\pandocbounded{\includesvg[keepaspectratio]{https://github.com/0x6b/typst-octique/wiki/assets/unverified.svg}} \\
\texttt{\ \#octique("upload")\ } &
\pandocbounded{\includesvg[keepaspectratio]{https://github.com/0x6b/typst-octique/wiki/assets/upload.svg}} \\
\texttt{\ \#octique("verified")\ } &
\pandocbounded{\includesvg[keepaspectratio]{https://github.com/0x6b/typst-octique/wiki/assets/verified.svg}} \\
\texttt{\ \#octique("versions")\ } &
\pandocbounded{\includesvg[keepaspectratio]{https://github.com/0x6b/typst-octique/wiki/assets/versions.svg}} \\
\texttt{\ \#octique("video")\ } &
\pandocbounded{\includesvg[keepaspectratio]{https://github.com/0x6b/typst-octique/wiki/assets/video.svg}} \\
\texttt{\ \#octique("webhook")\ } &
\pandocbounded{\includesvg[keepaspectratio]{https://github.com/0x6b/typst-octique/wiki/assets/webhook.svg}} \\
\texttt{\ \#octique("workflow")\ } &
\pandocbounded{\includesvg[keepaspectratio]{https://github.com/0x6b/typst-octique/wiki/assets/workflow.svg}} \\
\texttt{\ \#octique("x-circle-fill")\ } &
\pandocbounded{\includesvg[keepaspectratio]{https://github.com/0x6b/typst-octique/wiki/assets/x-circle-fill.svg}} \\
\texttt{\ \#octique("x-circle")\ } &
\pandocbounded{\includesvg[keepaspectratio]{https://github.com/0x6b/typst-octique/wiki/assets/x-circle.svg}} \\
\texttt{\ \#octique("x")\ } &
\pandocbounded{\includesvg[keepaspectratio]{https://github.com/0x6b/typst-octique/wiki/assets/x.svg}} \\
\texttt{\ \#octique("zap")\ } &
\pandocbounded{\includesvg[keepaspectratio]{https://github.com/0x6b/typst-octique/wiki/assets/zap.svg}} \\
\texttt{\ \#octique("zoom-in")\ } &
\pandocbounded{\includesvg[keepaspectratio]{https://github.com/0x6b/typst-octique/wiki/assets/zoom-in.svg}} \\
\texttt{\ \#octique("zoom-out")\ } &
\pandocbounded{\includesvg[keepaspectratio]{https://github.com/0x6b/typst-octique/wiki/assets/zoom-out.svg}} \\
\end{longtable}

\subsection{License}\label{license}

MIT. See
\href{https://github.com/typst/packages/raw/main/packages/preview/octique/0.1.0/LICENSE}{LICENSE}
for detail.

Octicons are © GitHub, Inc. When using the GitHub logos, you should
follow the \href{https://github.com/logos}{GitHub logo guidelines} .

\subsubsection{How to add}\label{how-to-add}

Copy this into your project and use the import as \texttt{\ octique\ }

\begin{verbatim}
#import "@preview/octique:0.1.0"
\end{verbatim}

\includesvg[width=0.16667in,height=0.16667in]{/assets/icons/16-copy.svg}

Check the docs for
\href{https://typst.app/docs/reference/scripting/\#packages}{more
information on how to import packages} .

\subsubsection{About}\label{about}

\begin{description}
\tightlist
\item[Author :]
0x6b
\item[License:]
MIT
\item[Current version:]
0.1.0
\item[Last updated:]
November 18, 2023
\item[First released:]
November 18, 2023
\item[Archive size:]
51.1 kB
\href{https://packages.typst.org/preview/octique-0.1.0.tar.gz}{\pandocbounded{\includesvg[keepaspectratio]{/assets/icons/16-download.svg}}}
\item[Repository:]
\href{https://github.com/0x6b/typst-octique}{GitHub}
\end{description}

\subsubsection{Where to report issues?}\label{where-to-report-issues}

This package is a project of 0x6b . Report issues on
\href{https://github.com/0x6b/typst-octique}{their repository} . You can
also try to ask for help with this package on the
\href{https://forum.typst.app}{Forum} .

Please report this package to the Typst team using the
\href{https://typst.app/contact}{contact form} if you believe it is a
safety hazard or infringes upon your rights.

\phantomsection\label{versions}
\subsubsection{Version history}\label{version-history}

\begin{longtable}[]{@{}ll@{}}
\toprule\noalign{}
Version & Release Date \\
\midrule\noalign{}
\endhead
\bottomrule\noalign{}
\endlastfoot
0.1.0 & November 18, 2023 \\
\end{longtable}

Typst GmbH did not create this package and cannot guarantee correct
functionality of this package or compatibility with any version of the
Typst compiler or app.


\title{typst.app/universe/package/ssrn-scribe}

\phantomsection\label{banner}
\phantomsection\label{template-thumbnail}
\pandocbounded{\includegraphics[keepaspectratio]{https://packages.typst.org/preview/thumbnails/ssrn-scribe-0.6.0-small.webp}}

\section{ssrn-scribe}\label{ssrn-scribe}

{ 0.6.0 }

Personal working paper template for general doc and SSRN paper.

\href{/app?template=ssrn-scribe&version=0.6.0}{Create project in app}

\phantomsection\label{readme}
Following the official tutorial, I create a single-column paper template
for general use. You can use it for papers published on SSRN etc.

\subsection{How to use}\label{how-to-use}

\subsubsection{Use as a template
package}\label{use-as-a-template-package}

Typst integrated the template with their official package manager. You
can use it as the other third-party packages.

You only need to enter the following command in the terminal to
initialize the template.

\begin{Shaded}
\begin{Highlighting}[]
\ExtensionTok{typst}\NormalTok{ init @preview/ssrn{-}scribe}
\end{Highlighting}
\end{Shaded}

If will generate a subfolder \texttt{\ ssrn-scribe\ } including the
\texttt{\ main.typ\ } file in the current directory with the latest
version of the template.

\subsubsection{Mannully use}\label{mannully-use}

\begin{enumerate}
\tightlist
\item
  Download the template or clone the repository.
\item
  generate your bibliography file using \texttt{\ .biblatex\ } and store
  the file in the same directory of the template.
\item
  modify the \texttt{\ main.typ\ } file in the subfolder
  \texttt{\ /template\ } and compile it. \textbf{\emph{Note:} You should
  have \texttt{\ paper\_template.typ\ } and \texttt{\ main.typ\ } in the
  same directory.}
\end{enumerate}

In the template, you can modify the following parameters:

\texttt{\ maketitle\ } is a boolean ( \textbf{compulsory} ). If
\texttt{\ maketitle=true\ } , the template will generate a new page for
the title. Otherwise, the title will be shown on the first page.

\begin{itemize}
\tightlist
\item
  \texttt{\ maketitle=true\ } :
\end{itemize}

\begin{longtable}[]{@{}llll@{}}
\toprule\noalign{}
Parameter & Default & Optional & Description \\
\midrule\noalign{}
\endhead
\bottomrule\noalign{}
\endlastfoot
\texttt{\ font\ } & “PT Serif� & Yes & The font of the paper. You
can choose “Times New Roman� or “Palatino� \\
\texttt{\ fontsize\ } & 11pt & Yes & The font size of the paper. You can
choose 10pt or 12pt \\
\texttt{\ title\ } & “Title� & No & The title of the paper \\
\texttt{\ subtitle\ } & none & Yes & The subtitle of the paper, use
“� or {[}{]} \\
\texttt{\ authors\ } & none & No & The authors of the paper \\
\texttt{\ date\ } & none & Yes & The date of the paper \\
\texttt{\ abstract\ } & none & Yes & The abstract of the paper \\
\texttt{\ keywords\ } & none & Yes & The keywords of the paper \\
\texttt{\ JEL\ } & none & Yes & The JEL codes of the paper \\
\texttt{\ acknowledgments\ } & none & Yes & The acknowledgment of the
paper \\
\texttt{\ bibliography\ } & none & Yes & The bibliography of the paper
\texttt{\ bibliography:\ bibliography("bib.bib",\ title:\ "References",\ style:\ "apa")\ } \\
\end{longtable}

\begin{itemize}
\tightlist
\item
  \texttt{\ maketitle=false\ } :
\end{itemize}

\begin{longtable}[]{@{}llll@{}}
\toprule\noalign{}
Parameter & Default & Optional & Description \\
\midrule\noalign{}
\endhead
\bottomrule\noalign{}
\endlastfoot
\texttt{\ font\ } & “PT Serif� & Yes & The font of the paper. You
can choose “Times New Roman� or “Palatino� \\
\texttt{\ fontsize\ } & 11pt & Yes & The font size of the paper. You can
choose 10pt or 12pt \\
\texttt{\ title\ } & “Title� & No & The title of the paper \\
\texttt{\ subtitle\ } & none & Yes & The subtitle of the paper, use
“� or {[}{]} \\
\texttt{\ authors\ } & none & No & The authors of the paper \\
\texttt{\ date\ } & none & Yes & The date of the paper \\
\texttt{\ bibliography\ } & none & Yes & The bibliography of the paper
\texttt{\ bibliography:\ bibliography("bib.bib",\ title:\ "References",\ style:\ "apa")\ } \\
\end{longtable}

\textbf{Note: You need to keep the comma at the end of the first bracket
of the author’s list, even if you have only one author.}

\begin{Shaded}
\begin{Highlighting}[]
\NormalTok{    (}
\NormalTok{    name: "",}
\NormalTok{    affiliation: "", // optional}
\NormalTok{    email: "", // optional}
\NormalTok{    note: "", // optional}
\NormalTok{    ),}
\end{Highlighting}
\end{Shaded}

\begin{Shaded}
\begin{Highlighting}[]
\NormalTok{\#import "@preview/ssrn{-}scribe:0.6.0": *}

\NormalTok{\#show: paper.with(}
\NormalTok{  font: "PT Serif", // "Times New Roman"}
\NormalTok{  fontsize: 12pt, // 12pt}
\NormalTok{  maketitle: true, // whether to add new page for title}
\NormalTok{  title: [\#lorem(5)], // title }
\NormalTok{  subtitle: "A work in progress", // subtitle}
\NormalTok{  authors: (}
\NormalTok{    (}
\NormalTok{      name: "Theresa Tungsten",}
\NormalTok{      affiliation: "Artos Institute",}
\NormalTok{      email: "tung@artos.edu",}
\NormalTok{      note: "123",}
\NormalTok{    ),}
\NormalTok{  ),}
\NormalTok{  date: "July 2023",}
\NormalTok{  abstract: lorem(80), // replace lorem(80) with [ Your abstract here. ]}
\NormalTok{  keywords: [}
\NormalTok{    Imputation,}
\NormalTok{    Multiple Imputation,}
\NormalTok{    Bayesian,],}
\NormalTok{  JEL: [G11, G12],}
\NormalTok{  acknowledgments: "This paper is a work in progress. Please do not cite without permission.", }
\NormalTok{  // bibliography: bibliography("bib.bib", title: "References", style: "apa"),}
\NormalTok{)}
\NormalTok{= Introduction}
\NormalTok{\#lorem(50)}
\end{Highlighting}
\end{Shaded}

\subsection{Preview}\label{preview}

\subsubsection{Example}\label{example}

Here is a screenshot of the template:
\pandocbounded{\includegraphics[keepaspectratio]{https://minioapi.pjx.ac.cn/img1/2024/03/63ce084e2a43bc2e7e31bd79315a0fb5.png}}

\subsubsection{\texorpdfstring{Example-brief with
\texttt{\ maketitle=true\ }}{Example-brief with  maketitle=true }}\label{example-brief-with-maketitletrue}

\pandocbounded{\includegraphics[keepaspectratio]{https://minioapi.pjx.ac.cn/img1/2024/06/8d203bd7f2fbf20b39b33334f0ee4a36.png}}

\subsubsection{\texorpdfstring{Example-brief with
\texttt{\ maketitle=false\ }}{Example-brief with  maketitle=false }}\label{example-brief-with-maketitlefalse}

\pandocbounded{\includegraphics[keepaspectratio]{https://minioapi.pjx.ac.cn/img1/2024/06/83dd5821409031ce0a2c2a15e014cc60.png}}

\href{/app?template=ssrn-scribe&version=0.6.0}{Create project in app}

\subsubsection{How to use}\label{how-to-use-1}

Click the button above to create a new project using this template in
the Typst app.

You can also use the Typst CLI to start a new project on your computer
using this command:

\begin{verbatim}
typst init @preview/ssrn-scribe:0.6.0
\end{verbatim}

\includesvg[width=0.16667in,height=0.16667in]{/assets/icons/16-copy.svg}

\subsubsection{About}\label{about}

\begin{description}
\tightlist
\item[Author :]
jxpeng98
\item[License:]
MIT
\item[Current version:]
0.6.0
\item[Last updated:]
June 11, 2024
\item[First released:]
March 20, 2024
\item[Archive size:]
4.07 kB
\href{https://packages.typst.org/preview/ssrn-scribe-0.6.0.tar.gz}{\pandocbounded{\includesvg[keepaspectratio]{/assets/icons/16-download.svg}}}
\item[Repository:]
\href{https://github.com/jxpeng98/Typst-Paper-Template}{GitHub}
\item[Categor y :]
\begin{itemize}
\tightlist
\item[]
\item
  \pandocbounded{\includesvg[keepaspectratio]{/assets/icons/16-atom.svg}}
  \href{https://typst.app/universe/search/?category=paper}{Paper}
\end{itemize}
\end{description}

\subsubsection{Where to report issues?}\label{where-to-report-issues}

This template is a project of jxpeng98 . Report issues on
\href{https://github.com/jxpeng98/Typst-Paper-Template}{their
repository} . You can also try to ask for help with this template on the
\href{https://forum.typst.app}{Forum} .

Please report this template to the Typst team using the
\href{https://typst.app/contact}{contact form} if you believe it is a
safety hazard or infringes upon your rights.

\phantomsection\label{versions}
\subsubsection{Version history}\label{version-history}

\begin{longtable}[]{@{}ll@{}}
\toprule\noalign{}
Version & Release Date \\
\midrule\noalign{}
\endhead
\bottomrule\noalign{}
\endlastfoot
0.6.0 & June 11, 2024 \\
\href{https://typst.app/universe/package/ssrn-scribe/0.5.0/}{0.5.0} &
April 5, 2024 \\
\href{https://typst.app/universe/package/ssrn-scribe/0.4.9/}{0.4.9} &
March 20, 2024 \\
\end{longtable}

Typst GmbH did not create this template and cannot guarantee correct
functionality of this template or compatibility with any version of the
Typst compiler or app.


