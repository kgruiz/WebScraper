\title{typst.app/universe/package/tuhi-exam-vuw}

\phantomsection\label{banner}
\phantomsection\label{template-thumbnail}
\pandocbounded{\includegraphics[keepaspectratio]{https://packages.typst.org/preview/thumbnails/tuhi-exam-vuw-0.1.0-small.webp}}

\section{tuhi-exam-vuw}\label{tuhi-exam-vuw}

{ 0.1.0 }

A poster template for VUW exams.

\href{/app?template=tuhi-exam-vuw&version=0.1.0}{Create project in app}

\phantomsection\label{readme}
A Typst template for VUW exams. To get started:

\begin{Shaded}
\begin{Highlighting}[]
\NormalTok{typst init @preview/tuhi{-}exam{-}vuw:0.1.0}
\end{Highlighting}
\end{Shaded}

And edit the \texttt{\ main.typ\ } example.

\pandocbounded{\includegraphics[keepaspectratio]{https://github.com/typst/packages/raw/main/packages/preview/tuhi-exam-vuw/0.1.0/thumbnail.png}}

\subsection{Contributing}\label{contributing}

PRs are welcome! And if you encounter any bugs or have any
requests/ideas, feel free to open an issue.

\href{/app?template=tuhi-exam-vuw&version=0.1.0}{Create project in app}

\subsubsection{How to use}\label{how-to-use}

Click the button above to create a new project using this template in
the Typst app.

You can also use the Typst CLI to start a new project on your computer
using this command:

\begin{verbatim}
typst init @preview/tuhi-exam-vuw:0.1.0
\end{verbatim}

\includesvg[width=0.16667in,height=0.16667in]{/assets/icons/16-copy.svg}

\subsubsection{About}\label{about}

\begin{description}
\tightlist
\item[Author :]
\href{https://github.com/baptiste}{baptiste}
\item[License:]
MPL-2.0
\item[Current version:]
0.1.0
\item[Last updated:]
May 24, 2024
\item[First released:]
May 24, 2024
\item[Archive size:]
136 kB
\href{https://packages.typst.org/preview/tuhi-exam-vuw-0.1.0.tar.gz}{\pandocbounded{\includesvg[keepaspectratio]{/assets/icons/16-download.svg}}}
\item[Categor y :]
\begin{itemize}
\tightlist
\item[]
\item
  \pandocbounded{\includesvg[keepaspectratio]{/assets/icons/16-envelope.svg}}
  \href{https://typst.app/universe/search/?category=office}{Office}
\end{itemize}
\end{description}

\subsubsection{Where to report issues?}\label{where-to-report-issues}

This template is a project of baptiste . You can also try to ask for
help with this template on the \href{https://forum.typst.app}{Forum} .

Please report this template to the Typst team using the
\href{https://typst.app/contact}{contact form} if you believe it is a
safety hazard or infringes upon your rights.

\phantomsection\label{versions}
\subsubsection{Version history}\label{version-history}

\begin{longtable}[]{@{}ll@{}}
\toprule\noalign{}
Version & Release Date \\
\midrule\noalign{}
\endhead
\bottomrule\noalign{}
\endlastfoot
0.1.0 & May 24, 2024 \\
\end{longtable}

Typst GmbH did not create this template and cannot guarantee correct
functionality of this template or compatibility with any version of the
Typst compiler or app.


\title{typst.app/universe/package/fractusist}

\phantomsection\label{banner}
\section{fractusist}\label{fractusist}

{ 0.1.0 }

Create a variety of wonderful fractals in Typst.

\phantomsection\label{readme}
Create a variety of wonderful fractals in Typst.

\subsection{Examples}\label{examples}

The example below creates a dragon curve of the 12th iteration with the
\texttt{\ dragon-curve\ } function.

\pandocbounded{\includegraphics[keepaspectratio]{https://github.com/typst/packages/raw/main/packages/preview/fractusist/0.1.0/examples/dragon-curve-n12.png}}

Show code

\begin{Shaded}
\begin{Highlighting}[]
\NormalTok{\#set page(width: auto, height: auto, margin: 0pt)}

\NormalTok{\#dragon{-}curve(}
\NormalTok{  12,}
\NormalTok{  step{-}size: 6,}
\NormalTok{  stroke{-}style: stroke(}
\NormalTok{    paint: gradient.linear(..color.map.crest, angle: 45deg),}
\NormalTok{    thickness: 3pt,}
\NormalTok{    cap: "square"}
\NormalTok{  )}
\NormalTok{)}
\end{Highlighting}
\end{Shaded}

\subsection{Features}\label{features}

\begin{itemize}
\tightlist
\item
  Use SVG backend for image rendering.
\item
  Generate fractals using
  \href{https://en.wikipedia.org/wiki/L-system}{L-system} .
\item
  The number of iterations, step size, fill and stroke styles, etc. of
  generated fractals could be customized.
\end{itemize}

\subsection{Usage}\label{usage}

Import the latest version of this package with:

\begin{Shaded}
\begin{Highlighting}[]
\NormalTok{\#import "@preview/fractusist:0.1.0": *}
\end{Highlighting}
\end{Shaded}

Each function generates a specific fractal. The input and output
arguments of all functions have a similar style. Typical input arguments
are as follows:

\begin{itemize}
\tightlist
\item
  \texttt{\ n\ } : the number of iterations ( \textbf{the valid range of
  values depends on the specific function} ).
\item
  \emph{\texttt{\ step-size\ }} : step size (in pt).
\item
  \emph{\texttt{\ fill-style\ }} : fill style, can be \texttt{\ none\ }
  or color or gradient ( \textbf{exists only when the curve is closed}
  ).
\item
  \emph{\texttt{\ stroke-style\ }} : stroke style, can be
  \texttt{\ none\ } or color or gradient or stroke object.
\item
  \emph{\texttt{\ width\ }} : the width of the image.
\item
  \emph{\texttt{\ height\ }} : the height of the image.
\item
  \emph{\texttt{\ fit\ }} : how the image should adjust itself to a
  given area, “cover� / “contain� / “stretch�.
\end{itemize}

The content returned is the \texttt{\ image\ } element.

For more codes with these functions see
\href{https://github.com/typst/packages/raw/main/packages/preview/fractusist/0.1.0/tests}{tests}
.

\subsection{Reference}\label{reference}

\subsubsection{Dragon}\label{dragon}

\begin{itemize}
\tightlist
\item
  \texttt{\ dragon-curve\ } : Generate dragon curve (n: range
  \textbf{{[}0, 16{]}} ).
\end{itemize}

\begin{Shaded}
\begin{Highlighting}[]
\NormalTok{\#let dragon{-}curve(n, step{-}size: 10, stroke{-}style: black + 1pt, width: auto, height: auto, fit: "cover") = \{...\}}
\end{Highlighting}
\end{Shaded}

\subsubsection{Hilbert}\label{hilbert}

\begin{itemize}
\tightlist
\item
  \texttt{\ hilbert-curve\ } : Generate 2D Hilbert curve. (n: range
  \textbf{{[}1, 8{]}} ).
\end{itemize}

\begin{Shaded}
\begin{Highlighting}[]
\NormalTok{\#let hilbert{-}curve(n, step{-}size: 10, stroke{-}style: black + 1pt, width: auto, height: auto, fit: "cover") = \{...\}}
\end{Highlighting}
\end{Shaded}

\begin{itemize}
\tightlist
\item
  \texttt{\ peano-curve\ } : Generate 2D Peano curve (n: range
  \textbf{{[}1, 5{]}} ).
\end{itemize}

\begin{Shaded}
\begin{Highlighting}[]
\NormalTok{\#let peano{-}curve(n, step{-}size: 10, stroke{-}style: black + 1pt, width: auto, height: auto, fit: "cover") = \{...\}}
\end{Highlighting}
\end{Shaded}

\subsubsection{Koch}\label{koch}

\begin{itemize}
\tightlist
\item
  \texttt{\ koch-curve\ } : Generate Koch curve (n: range \textbf{{[}0,
  6{]}} ).
\end{itemize}

\begin{Shaded}
\begin{Highlighting}[]
\NormalTok{\#let koch{-}curve(n, step{-}size: 10, stroke{-}style: black + 1pt, width: auto, height: auto, fit: "cover") = \{...\}}
\end{Highlighting}
\end{Shaded}

\begin{itemize}
\tightlist
\item
  \texttt{\ koch-snowflake\ } : Generate Koch snowflake (n: range
  \textbf{{[}0, 6{]}} ).
\end{itemize}

\begin{Shaded}
\begin{Highlighting}[]
\NormalTok{\#let koch{-}snowflake(n, step{-}size: 10, fill{-}style: none, stroke{-}style: black + 1pt, width: auto, height: auto, fit: "cover") = \{...\}}
\end{Highlighting}
\end{Shaded}

\subsubsection{Sierpiński}\label{sierpiuxe5ski}

\begin{itemize}
\tightlist
\item
  \texttt{\ sierpinski-curve\ } : Generate classic Sierpiński curve (n:
  range \textbf{{[}0, 7{]}} ).
\end{itemize}

\begin{Shaded}
\begin{Highlighting}[]
\NormalTok{\#let sierpinski{-}curve(n, step{-}size: 10, fill{-}style: none, stroke{-}style: black + 1pt, width: auto, height: auto, fit: "cover") = \{...\}}
\end{Highlighting}
\end{Shaded}

\begin{itemize}
\tightlist
\item
  \texttt{\ sierpinski-square-curve\ } : Generate Sierpiński square
  curve (n: range \textbf{{[}0, 7{]}} ).
\end{itemize}

\begin{Shaded}
\begin{Highlighting}[]
\NormalTok{\#let sierpinski{-}square{-}curve(n, step{-}size: 10, fill{-}style: none, stroke{-}style: black + 1pt, width: auto, height: auto, fit: "cover") = \{...\}}
\end{Highlighting}
\end{Shaded}

\begin{itemize}
\tightlist
\item
  \texttt{\ sierpinski-arrowhead-curve\ } : Generate Sierpiński
  arrowhead curve (n: range \textbf{{[}0, 8{]}} ).
\end{itemize}

\begin{Shaded}
\begin{Highlighting}[]
\NormalTok{\#let sierpinski{-}arrowhead{-}curve(n, step{-}size: 10, stroke{-}style: black + 1pt, width: auto, height: auto, fit: "cover") = \{...\}}
\end{Highlighting}
\end{Shaded}

\begin{itemize}
\tightlist
\item
  \texttt{\ sierpinski-triangle\ } : Generate 2D Sierpiński triangle
  (n: range \textbf{{[}0, 6{]}} ).
\end{itemize}

\begin{Shaded}
\begin{Highlighting}[]
\NormalTok{\#let sierpinski{-}triangle(n, step{-}size: 10, fill{-}style: none, stroke{-}style: black + 1pt, width: auto, height: auto, fit: "cover") = \{...\}}
\end{Highlighting}
\end{Shaded}

\subsubsection{How to add}\label{how-to-add}

Copy this into your project and use the import as
\texttt{\ fractusist\ }

\begin{verbatim}
#import "@preview/fractusist:0.1.0"
\end{verbatim}

\includesvg[width=0.16667in,height=0.16667in]{/assets/icons/16-copy.svg}

Check the docs for
\href{https://typst.app/docs/reference/scripting/\#packages}{more
information on how to import packages} .

\subsubsection{About}\label{about}

\begin{description}
\tightlist
\item[Author :]
\href{https://github.com/liuguangxi}{Guangxi Liu}
\item[License:]
MIT
\item[Current version:]
0.1.0
\item[Last updated:]
May 6, 2024
\item[First released:]
May 6, 2024
\item[Minimum Typst version:]
0.11.0
\item[Archive size:]
5.75 kB
\href{https://packages.typst.org/preview/fractusist-0.1.0.tar.gz}{\pandocbounded{\includesvg[keepaspectratio]{/assets/icons/16-download.svg}}}
\item[Repository:]
\href{https://github.com/liuguangxi/fractusist}{GitHub}
\item[Discipline s :]
\begin{itemize}
\tightlist
\item[]
\item
  \href{https://typst.app/universe/search/?discipline=computer-science}{Computer
  Science}
\item
  \href{https://typst.app/universe/search/?discipline=mathematics}{Mathematics}
\end{itemize}
\item[Categor ies :]
\begin{itemize}
\tightlist
\item[]
\item
  \pandocbounded{\includesvg[keepaspectratio]{/assets/icons/16-package.svg}}
  \href{https://typst.app/universe/search/?category=components}{Components}
\item
  \pandocbounded{\includesvg[keepaspectratio]{/assets/icons/16-chart.svg}}
  \href{https://typst.app/universe/search/?category=visualization}{Visualization}
\end{itemize}
\end{description}

\subsubsection{Where to report issues?}\label{where-to-report-issues}

This package is a project of Guangxi Liu . Report issues on
\href{https://github.com/liuguangxi/fractusist}{their repository} . You
can also try to ask for help with this package on the
\href{https://forum.typst.app}{Forum} .

Please report this package to the Typst team using the
\href{https://typst.app/contact}{contact form} if you believe it is a
safety hazard or infringes upon your rights.

\phantomsection\label{versions}
\subsubsection{Version history}\label{version-history}

\begin{longtable}[]{@{}ll@{}}
\toprule\noalign{}
Version & Release Date \\
\midrule\noalign{}
\endhead
\bottomrule\noalign{}
\endlastfoot
0.1.0 & May 6, 2024 \\
\end{longtable}

Typst GmbH did not create this package and cannot guarantee correct
functionality of this package or compatibility with any version of the
Typst compiler or app.


\title{typst.app/universe/package/htlwienwest-da}

\phantomsection\label{banner}
\phantomsection\label{template-thumbnail}
\pandocbounded{\includegraphics[keepaspectratio]{https://packages.typst.org/preview/thumbnails/htlwienwest-da-0.1.0-small.webp}}

\section{htlwienwest-da}\label{htlwienwest-da}

{ 0.1.0 }

The diploma thesis template for students of the HTL Wien West.

{ } Officially affiliated

\href{/app?template=htlwienwest-da&version=0.1.0}{Create project in app}

\phantomsection\label{readme}
This is a Typst diploma thesis template for students of the HTL Wien
West. It fulfils all the necessary requirements for the diploma thesis.

\subsection{Usage}\label{usage}

You can use this template in the Typst web app by clicking “Start from
template� on the dashboard and searching for
\texttt{\ htlwienwest-da\ } .

Alternatively, you can use the CLI to kick this project off using the
command

\begin{verbatim}
typst init @preview/htlwienwest-da
\end{verbatim}

Typst will create a new directory with all the files needed to get you
started.

\subsection{Configuration}\label{configuration}

This template exports the \texttt{\ diplomarbeit\ } function with the
following named arguments:

\begin{itemize}
\tightlist
\item
  \texttt{\ titel\ } : \texttt{\ string\ } - The title of the thesis
\item
  \texttt{\ schuljahr\ } : \texttt{\ string\ } - The current school year
\item
  \texttt{\ abteilung\ } : \texttt{\ string\ } - The student’s
  department
\item
  \texttt{\ unterschrifts-datum\ } : \texttt{\ string\ } - The
  submission date
\item
  \texttt{\ autoren\ } : \texttt{\ array(dict)\ } - An array of all
  authors, represented as dictionaries. Each of them has the following
  properties

  \begin{itemize}
  \tightlist
  \item
    \texttt{\ vorname\ } : \texttt{\ string\ } - Firstname of the
    student
  \item
    \texttt{\ nachname\ } : \texttt{\ string\ } - Lastname of the
    student
  \item
    \texttt{\ klasse\ } : \texttt{\ string\ } - School class of the
    student
  \item
    \texttt{\ betreuer\ } : \texttt{\ dict\ } - The student’s advisor
    as dictionary

    \begin{itemize}
    \tightlist
    \item
      \texttt{\ name\ } : \texttt{\ string\ \textbar{}\ content\ } - The
      advisor’s name
    \item
      \texttt{\ geschlecht\ } :
      \texttt{\ "male"\ \textbar{}\ "female"\ } - Gender of advisor for
      correct gendering
    \end{itemize}
  \item
    \texttt{\ aufgaben\ } : \texttt{\ content\ } - The student’s
    responsibilities
  \end{itemize}
\item
  \texttt{\ kurzfassung\ } : \texttt{\ content\ } - Abstract in german
  as content block
\item
  \texttt{\ abstract\ } : \texttt{\ content\ } - Abstract in english as
  content block
\item
  \texttt{\ vorwort\ } : \texttt{\ content\ } - The thesis’ preface
\item
  \texttt{\ danksagung\ } : \texttt{\ content\ } - Acknowledgement
\item
  \texttt{\ anhang\ } : \texttt{\ content\ \textbar{}\ none\ } -
  Appendix
\item
  \texttt{\ literaturverzeichnis\ } : \texttt{\ function\ } - The
  bibliography prefilled with the BibTex file path
\end{itemize}

The function also accepts a single, positional argument for the body of
the paper.

The template will initialize your package with a sample call to the
\texttt{\ diplomarbeit\ } function in a show rule. If you want to change
an existing project to use thistemplate, you can add a show rule like
this at the top of your file:

\begin{Shaded}
\begin{Highlighting}[]
\NormalTok{\#import "@preview/htlwienwest{-}da:0.1.0": *}

\NormalTok{\#show: diplomarbeit.with(}
\NormalTok{  titel: "Titel der Diplomarbeit",}
\NormalTok{  abteilung: "Informationstechnologie",}
\NormalTok{  schuljahr: "2023/24",}
\NormalTok{  unterschrifts{-}datum: "20.04.2024",}
\NormalTok{  autoren: (}
\NormalTok{   (}
\NormalTok{     vorname: "Hans", nachname: "Mustermann",}
\NormalTok{     klasse: "5AHITN",}
\NormalTok{     betreuer: (name: "Dr. Walter Turbo", geschlecht: "male"),}
\NormalTok{     aufgaben: [}
\NormalTok{       \#lorem(100)}
\NormalTok{     ]}
\NormalTok{   ),}
\NormalTok{   (}
\NormalTok{     vorname: "Herta", nachname: "Musterfrau",}
\NormalTok{     klasse: "5AHITN",}
\NormalTok{     betreuer: (name: "Dipl.{-}Ing Maria Kreisel", geschlecht: "female"),}
\NormalTok{     aufgaben: [}
\NormalTok{       \#lorem(100)}
\NormalTok{     ]}
\NormalTok{   ),}
\NormalTok{  kurzfassung: [}
\NormalTok{    Die Kurzfassung muss die folgenden Inhalte darlegen (§8, Absatz 5 Prüfungsordnung): Thema, Fragestellung, Problemformulierung, wesentliche Ergebnisse. Sie soll einen prägnanten Überblick über die Arbeit geben.}
\NormalTok{  ],}
\NormalTok{  abstract: [}
\NormalTok{    Englische Version der Kurzfassung (siehe \#link(\textless{}Kurzfassung\textgreater{})[\_Kurzfassung\_])}
\NormalTok{  ],}
\NormalTok{  vorwort: [}
\NormalTok{    Perönlicher Zugang zum Thema. Gründe für die Themenwahl.}
\NormalTok{  ],}
\NormalTok{  danksagung: [}
\NormalTok{    Dank an Personen, die bei der Erstellung der Arbeit unterstützt haben.}
\NormalTok{  ],}
\NormalTok{  anhang: include "anhang.typ", // entfernen falls nicht benötigt}
\NormalTok{  literaturverzeichnis: bibliography.with("literaturverzeichnis.bib")}
\NormalTok{)}

\NormalTok{// Your content goes below.}
\end{Highlighting}
\end{Shaded}

\subsection{Provided Functions}\label{provided-functions}

Beside the \texttt{\ diplomarbeit\ } function, the template also
provides the \texttt{\ autor\ } function that is used after a heading to
indicate the specific author of the current section.

\begin{verbatim}
== Some Heading
#autor[Your Name]
\end{verbatim}

This will render additional information to the section’s heading.

To install the template locally, you can use

\begin{Shaded}
\begin{Highlighting}[]
\ExtensionTok{just}\NormalTok{ install}
\end{Highlighting}
\end{Shaded}

which uses the \href{https://github.com/casey/just}{just} command
runner.

If you don’t want to install \texttt{\ just\ } , you can run

\begin{Shaded}
\begin{Highlighting}[]
\FunctionTok{bash}\NormalTok{ ./scripts/package @local}
\end{Highlighting}
\end{Shaded}

The installed version can be used via \texttt{\ @local\ } instead of
\texttt{\ @preview\ } . To create a new typst project from the template,
run

\begin{Shaded}
\begin{Highlighting}[]
\ExtensionTok{typst}\NormalTok{ init @local/htlwienwest{-}da:}
\end{Highlighting}
\end{Shaded}

\href{/app?template=htlwienwest-da&version=0.1.0}{Create project in app}

\subsubsection{How to use}\label{how-to-use}

Click the button above to create a new project using this template in
the Typst app.

You can also use the Typst CLI to start a new project on your computer
using this command:

\begin{verbatim}
typst init @preview/htlwienwest-da:0.1.0
\end{verbatim}

\includesvg[width=0.16667in,height=0.16667in]{/assets/icons/16-copy.svg}

\subsubsection{About}\label{about}

\begin{description}
\tightlist
\item[Author s :]
\href{https://github.com/jozott00}{Johannes Zottele} \&
\href{https://github.com/peterw16}{peterw16}
\item[License:]
MIT
\item[Current version:]
0.1.0
\item[Last updated:]
May 3, 2024
\item[First released:]
May 3, 2024
\item[Archive size:]
61.9 kB
\href{https://packages.typst.org/preview/htlwienwest-da-0.1.0.tar.gz}{\pandocbounded{\includesvg[keepaspectratio]{/assets/icons/16-download.svg}}}
\item[Verification:]
We verified that the author is affiliated with their institution
\pandocbounded{\includesvg[keepaspectratio]{/assets/icons/16-verified.svg}}
\item[Repository:]
\href{https://github.com/htlwienwest/da-vorlage-typst}{GitHub}
\item[Categor y :]
\begin{itemize}
\tightlist
\item[]
\item
  \pandocbounded{\includesvg[keepaspectratio]{/assets/icons/16-mortarboard.svg}}
  \href{https://typst.app/universe/search/?category=thesis}{Thesis}
\end{itemize}
\end{description}

\subsubsection{Where to report issues?}\label{where-to-report-issues}

This template is a project of Johannes Zottele and peterw16 . Report
issues on \href{https://github.com/htlwienwest/da-vorlage-typst}{their
repository} . You can also try to ask for help with this template on the
\href{https://forum.typst.app}{Forum} .

Please report this template to the Typst team using the
\href{https://typst.app/contact}{contact form} if you believe it is a
safety hazard or infringes upon your rights.

\phantomsection\label{versions}
\subsubsection{Version history}\label{version-history}

\begin{longtable}[]{@{}ll@{}}
\toprule\noalign{}
Version & Release Date \\
\midrule\noalign{}
\endhead
\bottomrule\noalign{}
\endlastfoot
0.1.0 & May 3, 2024 \\
\end{longtable}

Typst GmbH did not create this template and cannot guarantee correct
functionality of this template or compatibility with any version of the
Typst compiler or app.


\title{typst.app/universe/package/athena-tu-darmstadt-exercise}

\phantomsection\label{banner}
\phantomsection\label{template-thumbnail}
\pandocbounded{\includegraphics[keepaspectratio]{https://packages.typst.org/preview/thumbnails/athena-tu-darmstadt-exercise-0.1.0-small.webp}}

\section{athena-tu-darmstadt-exercise}\label{athena-tu-darmstadt-exercise}

{ 0.1.0 }

Exercise template for TU Darmstadt (Technische Universität Darmstadt).

\href{/app?template=athena-tu-darmstadt-exercise&version=0.1.0}{Create
project in app}

\phantomsection\label{readme}
These \textbf{unofficial} templates can be used to write in
\href{https://github.com/typst/typst}{Typst} with the corporate design
of \href{https://www.tu-darmstadt.de/}{TU Darmstadt} .

\paragraph{Disclaimer}\label{disclaimer}

Please ask your supervisor if you are allowed to use typst and this
template for your thesis or other documents. Note that this template is
not checked by TU Darmstadt for correctness. Thus, this template does
not guarantee completeness or correctness. Also, note that submission in
TUbama requires PDF/A which typst currently can’t export to (
\url{https://github.com/typst/typst/issues/2942} ). You can use a
converter to convert from the typst output to PDF/A, but check that
there are no losses during the conversion. CMYK color space support may
be required for printing which is also currently not supported by typst
( \url{https://github.com/typst/typst/issues/2942} ), but this is not
relevant when you just submit online.

\subsection{Implemented Templates}\label{implemented-templates}

The templates imitate the style of the corresponding latex templates in
\href{https://github.com/tudace/tuda_latex_templates}{tuda\_latex\_templates}
. Note that there can be visual differences between the original latex
template and the typst template (you may open an issue when you find
one).

For missing features, ideas or other problems you can just open an issue
:wink:. Contributions are also welcome.

\begin{longtable}[]{@{}
  >{\raggedright\arraybackslash}p{(\linewidth - 6\tabcolsep) * \real{0.2500}}
  >{\raggedright\arraybackslash}p{(\linewidth - 6\tabcolsep) * \real{0.2500}}
  >{\raggedright\arraybackslash}p{(\linewidth - 6\tabcolsep) * \real{0.2500}}
  >{\raggedright\arraybackslash}p{(\linewidth - 6\tabcolsep) * \real{0.2500}}@{}}
\toprule\noalign{}
\begin{minipage}[b]{\linewidth}\raggedright
Template
\end{minipage} & \begin{minipage}[b]{\linewidth}\raggedright
Preview
\end{minipage} & \begin{minipage}[b]{\linewidth}\raggedright
Example
\end{minipage} & \begin{minipage}[b]{\linewidth}\raggedright
Scope
\end{minipage} \\
\midrule\noalign{}
\endhead
\bottomrule\noalign{}
\endlastfoot
\href{https://github.com/JeyRunner/tuda-typst-templates/blob/main/templates/tudapub/template/tudapub.typ}{tudapub}
&
\includegraphics[width=\linewidth,height=3.125in,keepaspectratio]{https://raw.githubusercontent.com/JeyRunner/tuda-typst-templates/refs/heads/main/templates/tudapub/preview/tudapub_prev-01.png}
& \begin{minipage}[t]{\linewidth}\raggedright
\href{https://github.com/JeyRunner/tuda-typst-templates/blob/main/example_tudapub.pdf}{example\_tudapub.pdf}\\
\href{https://github.com/JeyRunner/tuda-typst-templates/blob/main/example_tudapub.typ}{example\_tudapub.typ}\strut
\end{minipage} & Master and Bachelor thesis \\
\href{https://github.com/JeyRunner/tuda-typst-templates/blob/main/templates/tudaexercise/template/tudaexercise.typ}{tudaexercise}
&
\includegraphics[width=\linewidth,height=3.125in,keepaspectratio]{https://raw.githubusercontent.com/JeyRunner/tuda-typst-templates/refs/heads/main/templates/tudaexercise/preview/tudaexercise_prev-1.png}
&
\href{https://github.com/JeyRunner/tuda-typst-templates/blob/main/templates_examples/tudaexercise/main.typ}{Example
File} & Exercises \\
\end{longtable}

\subsection{Usage}\label{usage}

Create a new typst project based on this template locally.

\begin{Shaded}
\begin{Highlighting}[]
\CommentTok{\# for tudapub}
\ExtensionTok{typst}\NormalTok{ init @preview/athena{-}tu{-}darmstadt{-}thesis}
\BuiltInTok{cd}\NormalTok{ athena{-}tu{-}darmstadt{-}thesis}

\CommentTok{\# for tudaexercise}
\ExtensionTok{typst}\NormalTok{ init @preview/athena{-}tu{-}darmstadt{-}exercise}
\BuiltInTok{cd}\NormalTok{ athena{-}tu{-}darmstadt{-}exercise}
\end{Highlighting}
\end{Shaded}

Or create a project on the typst web app based on this template.

Or do a manual installation of this template.

For a manual setup create a folder for your writing project and download
this template into the `templates` folder:

\begin{Shaded}
\begin{Highlighting}[]
\FunctionTok{mkdir}\NormalTok{ my\_exercise }\KeywordTok{\&\&} \BuiltInTok{cd}\NormalTok{ my\_exercise}
\FunctionTok{git}\NormalTok{ clone https://github.com/JeyRunner/tuda{-}typst{-}templates}
\end{Highlighting}
\end{Shaded}

\subsubsection{Logo and Font Setup}\label{logo-and-font-setup}

Download the tud logo from
\href{https://download.hrz.tu-darmstadt.de/protected/ULB/tuda_logo.pdf}{download.hrz.tu-darmstadt.de/protected/ULB/tuda\_logo.pdf}
and put it into the \texttt{\ asssets/logos\ } folder. Now execute the
following script in the \texttt{\ asssets/logos\ } folder to convert it
into an svg:

\begin{Shaded}
\begin{Highlighting}[]
\BuiltInTok{cd}\NormalTok{ asssets/logos}
\ExtensionTok{./convert\_logo.sh}
\end{Highlighting}
\end{Shaded}

Note: The here used \texttt{\ pdf2svg\ } command might not be available.
In this case we recommend a online converter like
\href{https://tools.pdf24.org/en/pdf-to-svg}{PDF24 Tools} . There also
is a \href{https://github.com/FussballAndy/typst-img-to-local}{tool} to
install images as local typst packages.

Also download the required fonts \texttt{\ Roboto\ } and
\texttt{\ XCharter\ } :

\begin{Shaded}
\begin{Highlighting}[]
\BuiltInTok{cd}\NormalTok{ asssets/fonts}
\ExtensionTok{./download\_fonts.sh}
\end{Highlighting}
\end{Shaded}

Optionally you can install all fonts in the folders in
\texttt{\ fonts\ } on your system. But you can also use Typst’s
\texttt{\ -\/-font-path\ } option. Or install them in a folder and add
the folder to \texttt{\ TYPST\_FONT\_PATHS\ } for a single font folder.

Note: wget might not be available. In this case either download it or
replace the command with something like
\texttt{\ curl\ \textless{}url\textgreater{}\ -o\ \textless{}filename\textgreater{}\ -L\ }

Create a main.typ file for the manual template installation.

Create a simple `main.typ` in the root folder (`my\_exercise`) of your
new project:

\begin{Shaded}
\begin{Highlighting}[]
\NormalTok{\#import "tuda{-}typst{-}templates/templates/tudaexercise/template/lib.typ": *}

\NormalTok{\#show: tudaexercise.with(}
\NormalTok{  info: (}
\NormalTok{    title: "My Exercise",}
\NormalTok{    auhtor: "Your name",}
\NormalTok{    sheetnumber: 1    }
\NormalTok{  ),}
\NormalTok{  logo: image("tuda{-}typst{-}templates/assets/logos/tuda\_logo.svg")}
\NormalTok{)}

\NormalTok{= My First Task}
\NormalTok{Some Text}
\end{Highlighting}
\end{Shaded}

\subsubsection{Compile you typst file}\label{compile-you-typst-file}

\begin{Shaded}
\begin{Highlighting}[]
\ExtensionTok{typst} \AttributeTok{{-}{-}watch}\NormalTok{ main.typ }\AttributeTok{{-}{-}font{-}path}\NormalTok{ asssets/fonts/}
\end{Highlighting}
\end{Shaded}

This will watch your file and recompile it to a pdf when the file is
saved. For writing, you can use
\href{https://code.visualstudio.com/}{Vscode} with these extensions:
\href{https://marketplace.visualstudio.com/items?itemName=myriad-dreamin.tinymist}{Tinymist
Typst} . Or use the \href{https://typst.app/}{typst web app} (here you
need to upload the logo and the fonts).

Note that we add \texttt{\ -\/-font-path\ } to ensure that the correct
fonts are used. Due to a bug (typst/typst\#2917 typst/typst\#2098) typst
sometimes uses the font \texttt{\ Roboto\ condensed\ } instead of
\texttt{\ Roboto\ } . To be on the safe side, double-check the embedded
fonts in the pdf (there should be no \texttt{\ Roboto\ condensed\ } ).
What also works is to uninstall/deactivate all
\texttt{\ Roboto\ condensed\ } fonts from your system.

\subsection{Todos}\label{todos}

\begin{itemize}
\tightlist
\item
  \href{https://github.com/JeyRunner/tuda-typst-templates/blob/main/templates/tudapub/TODO.md}{todos
  of thesis template}
\end{itemize}

\href{/app?template=athena-tu-darmstadt-exercise&version=0.1.0}{Create
project in app}

\subsubsection{How to use}\label{how-to-use}

Click the button above to create a new project using this template in
the Typst app.

You can also use the Typst CLI to start a new project on your computer
using this command:

\begin{verbatim}
typst init @preview/athena-tu-darmstadt-exercise:0.1.0
\end{verbatim}

\includesvg[width=0.16667in,height=0.16667in]{/assets/icons/16-copy.svg}

\subsubsection{About}\label{about}

\begin{description}
\tightlist
\item[Author s :]
\href{https://github.com/JeyRunner}{JeyRunner} \&
\href{https://github.com/FussballAndy}{FussballAndy}
\item[License:]
MIT
\item[Current version:]
0.1.0
\item[Last updated:]
November 25, 2024
\item[First released:]
November 25, 2024
\item[Minimum Typst version:]
0.12.0
\item[Archive size:]
10.9 kB
\href{https://packages.typst.org/preview/athena-tu-darmstadt-exercise-0.1.0.tar.gz}{\pandocbounded{\includesvg[keepaspectratio]{/assets/icons/16-download.svg}}}
\item[Repository:]
\href{https://github.com/JeyRunner/tuda-typst-templates}{GitHub}
\item[Categor y :]
\begin{itemize}
\tightlist
\item[]
\item
  \pandocbounded{\includesvg[keepaspectratio]{/assets/icons/16-layout.svg}}
  \href{https://typst.app/universe/search/?category=layout}{Layout}
\end{itemize}
\end{description}

\subsubsection{Where to report issues?}\label{where-to-report-issues}

This template is a project of JeyRunner and FussballAndy . Report issues
on \href{https://github.com/JeyRunner/tuda-typst-templates}{their
repository} . You can also try to ask for help with this template on the
\href{https://forum.typst.app}{Forum} .

Please report this template to the Typst team using the
\href{https://typst.app/contact}{contact form} if you believe it is a
safety hazard or infringes upon your rights.

\phantomsection\label{versions}
\subsubsection{Version history}\label{version-history}

\begin{longtable}[]{@{}ll@{}}
\toprule\noalign{}
Version & Release Date \\
\midrule\noalign{}
\endhead
\bottomrule\noalign{}
\endlastfoot
0.1.0 & November 25, 2024 \\
\end{longtable}

Typst GmbH did not create this template and cannot guarantee correct
functionality of this template or compatibility with any version of the
Typst compiler or app.


\title{typst.app/universe/package/meppp}

\phantomsection\label{banner}
\phantomsection\label{template-thumbnail}
\pandocbounded{\includegraphics[keepaspectratio]{https://packages.typst.org/preview/thumbnails/meppp-0.2.1-small.webp}}

\section{meppp}\label{meppp}

{ 0.2.1 }

Template for modern physics experiment reports at the Physics School of
PKU.

\href{/app?template=meppp&version=0.2.1}{Create project in app}

\phantomsection\label{readme}
A simple template for modern physics experiments (MPE) courses at the
Physics School of PKU.

\subsection{meppp-lab-report}\label{meppp-lab-report}

The recommended report format of MPE course. Default arguments are shown
as below:

\begin{Shaded}
\begin{Highlighting}[]
\NormalTok{\#import "@preview/meppp:0.2.1": *}

\NormalTok{\#let meppp{-}lab{-}report(}
\NormalTok{  title: "",}
\NormalTok{  author: "",}
\NormalTok{  info: [],}
\NormalTok{  abstract: [],}
\NormalTok{  keywords: (),}
\NormalTok{  author{-}footnote: [],}
\NormalTok{  heading{-}numbering{-}array: ("I" ,"A", "1", "a"),}
\NormalTok{  heading{-}suffix: ". ",}
\NormalTok{  doc,}
\NormalTok{) = ...}
\end{Highlighting}
\end{Shaded}

\begin{itemize}
\tightlist
\item
  \texttt{\ title\ } is the title of the report.
\item
  \texttt{\ author\ } is the name of the author.
\item
  \texttt{\ info\ } is a line (or lines) of brief information of author
  and the report (e.g. student ID, school, experiment date…)
\item
  \texttt{\ abstract\ } is the abstract of the report, not shown when it
  is empty.
\item
  \texttt{\ keywords\ } are keywords of the report, only shown when the
  abstract is shown.
\item
  \texttt{\ author-footnote\ } is the phone number or the e-mail of the
  author, shown in the footnote.
\item
  \texttt{\ heading-numbering-array\ } is the heading numbering of each
  level. Only shows the numbering of the deepest level.
\item
  \texttt{\ heading-suffix\ } is the suffix of headings
\end{itemize}

It is recommended to use \texttt{\ \#show\ } to use the template:

\begin{Shaded}
\begin{Highlighting}[]
\NormalTok{\#show: meppp{-}lab{-}report.with(}
\NormalTok{    title: [Test title],}
\NormalTok{    ..args}
\NormalTok{)}
\NormalTok{...your report below.}
\end{Highlighting}
\end{Shaded}

\subsection{meppp-tl-table}\label{meppp-tl-table}

Modify your input \texttt{\ table\ } to a three-lined table (AIP style),
returned as a \texttt{\ figure\ } . Double-lines above and below the
table, and a single line below the header.

\begin{Shaded}
\begin{Highlighting}[]
\NormalTok{\#let meppp{-}tl{-}table(}
\NormalTok{  caption: none,}
\NormalTok{  supplement: auto,}
\NormalTok{  stroke: 0.5pt,}
\NormalTok{  tbl}
\NormalTok{) = ...}
\end{Highlighting}
\end{Shaded}

\begin{itemize}
\tightlist
\item
  \texttt{\ caption\ } is the caption above the table, center-aligned
\item
  \texttt{\ supplement\ } is same as the supplement in the figure.
\item
  \texttt{\ stroke\ } is the stroke used in the three lines (maybe five
  lines).
\item
  \texttt{\ tbl\ } is the input table, which must contains a
  \texttt{\ table.header\ }
\end{itemize}

Example:

\begin{Shaded}
\begin{Highlighting}[]
\NormalTok{\#meppp{-}tl{-}table(}
\NormalTok{  table(}
\NormalTok{    columns: 4,}
\NormalTok{    rows: 2,}
\NormalTok{    table.header([Item1], [Item2], [Item3], [Item4]),}
\NormalTok{    [Data1], [Data2], [Data3], [Data4],}
\NormalTok{  )}
\NormalTok{)}
\end{Highlighting}
\end{Shaded}

\subsection{subfigure}\label{subfigure}

Counts subfigures and displays in the figure, mostly used when inserting
multiple images.

\begin{Shaded}
\begin{Highlighting}[]
\NormalTok{\#let subfigure(}
\NormalTok{  body,}
\NormalTok{  caption: none,}
\NormalTok{  numbering: "(a)",}
\NormalTok{  inside: true,}
\NormalTok{  dx: 10pt,}
\NormalTok{  dy: 10pt,}
\NormalTok{  boxargs: (fill: white, inset: 5pt),}
\NormalTok{  alignment: top + left,}
\NormalTok{) = ...}
\end{Highlighting}
\end{Shaded}

\subsection{pku-logo}\label{pku-logo}

The logo of PKU, returned as a \texttt{\ image\ }

\begin{Shaded}
\begin{Highlighting}[]
\NormalTok{\#let pku{-}logo(..args) = image("pkulogo.png", ..args)}
\end{Highlighting}
\end{Shaded}

Example:

\begin{Shaded}
\begin{Highlighting}[]
\NormalTok{\#pku{-}logo(width: 50\%)}
\NormalTok{\#pku{-}logo()}
\end{Highlighting}
\end{Shaded}

\href{/app?template=meppp&version=0.2.1}{Create project in app}

\subsubsection{How to use}\label{how-to-use}

Click the button above to create a new project using this template in
the Typst app.

You can also use the Typst CLI to start a new project on your computer
using this command:

\begin{verbatim}
typst init @preview/meppp:0.2.1
\end{verbatim}

\includesvg[width=0.16667in,height=0.16667in]{/assets/icons/16-copy.svg}

\subsubsection{About}\label{about}

\begin{description}
\tightlist
\item[Author :]
\href{https://github.com/CL4R3T}{CL4R3T}
\item[License:]
MIT
\item[Current version:]
0.2.1
\item[Last updated:]
September 22, 2024
\item[First released:]
May 8, 2024
\item[Archive size:]
103 kB
\href{https://packages.typst.org/preview/meppp-0.2.1.tar.gz}{\pandocbounded{\includesvg[keepaspectratio]{/assets/icons/16-download.svg}}}
\item[Repository:]
\href{https://github.com/pku-typst/meppp}{GitHub}
\item[Categor y :]
\begin{itemize}
\tightlist
\item[]
\item
  \pandocbounded{\includesvg[keepaspectratio]{/assets/icons/16-speak.svg}}
  \href{https://typst.app/universe/search/?category=report}{Report}
\end{itemize}
\end{description}

\subsubsection{Where to report issues?}\label{where-to-report-issues}

This template is a project of CL4R3T . Report issues on
\href{https://github.com/pku-typst/meppp}{their repository} . You can
also try to ask for help with this template on the
\href{https://forum.typst.app}{Forum} .

Please report this template to the Typst team using the
\href{https://typst.app/contact}{contact form} if you believe it is a
safety hazard or infringes upon your rights.

\phantomsection\label{versions}
\subsubsection{Version history}\label{version-history}

\begin{longtable}[]{@{}ll@{}}
\toprule\noalign{}
Version & Release Date \\
\midrule\noalign{}
\endhead
\bottomrule\noalign{}
\endlastfoot
0.2.1 & September 22, 2024 \\
\href{https://typst.app/universe/package/meppp/0.2.0/}{0.2.0} &
September 14, 2024 \\
\href{https://typst.app/universe/package/meppp/0.1.0/}{0.1.0} & May 8,
2024 \\
\end{longtable}

Typst GmbH did not create this template and cannot guarantee correct
functionality of this template or compatibility with any version of the
Typst compiler or app.


\title{typst.app/universe/package/tgm-hit-thesis}

\phantomsection\label{banner}
\phantomsection\label{template-thumbnail}
\pandocbounded{\includegraphics[keepaspectratio]{https://packages.typst.org/preview/thumbnails/tgm-hit-thesis-0.2.0-small.webp}}

\section{tgm-hit-thesis}\label{tgm-hit-thesis}

{ 0.2.0 }

Diploma thesis template for students of the HIT department at TGM Wien

{ } Officially affiliated

\href{/app?template=tgm-hit-thesis&version=0.2.0}{Create project in app}

\phantomsection\label{readme}
This is a port of the
\href{https://github.com/TGM-HIT/diploma-thesis}{LaTeX diploma thesis
template} available for students of the information technology
department at the TGM technical secondary school in Vienna.

\subsection{Getting Started}\label{getting-started}

Using the Typst web app, you can create a project by e.g. using this
link: \url{https://typst.app/?template=tgm-hit-thesis&version=latest} .

To work locally, use the following command:

\begin{Shaded}
\begin{Highlighting}[]
\ExtensionTok{typst}\NormalTok{ init @preview/tgm{-}hit{-}thesis}
\end{Highlighting}
\end{Shaded}

\subsection{Usage}\label{usage}

The template (
\href{https://github.com/typst/packages/raw/main/packages/preview/tgm-hit-thesis/0.2.0/example.pdf}{rendered
PDF} ) contains thesis writing advice (in German) as example content. If
you are looking for the details of this template package’s function,
take a look at the
\href{https://github.com/typst/packages/raw/main/packages/preview/tgm-hit-thesis/0.2.0/docs/manual.pdf}{manual}
.

\href{/app?template=tgm-hit-thesis&version=0.2.0}{Create project in app}

\subsubsection{How to use}\label{how-to-use}

Click the button above to create a new project using this template in
the Typst app.

You can also use the Typst CLI to start a new project on your computer
using this command:

\begin{verbatim}
typst init @preview/tgm-hit-thesis:0.2.0
\end{verbatim}

\includesvg[width=0.16667in,height=0.16667in]{/assets/icons/16-copy.svg}

\subsubsection{About}\label{about}

\begin{description}
\tightlist
\item[Author :]
\href{https://github.com/SillyFreak/}{Clemens Koza}
\item[License:]
MIT
\item[Current version:]
0.2.0
\item[Last updated:]
October 24, 2024
\item[First released:]
July 15, 2024
\item[Minimum Typst version:]
0.11.0
\item[Archive size:]
86.8 kB
\href{https://packages.typst.org/preview/tgm-hit-thesis-0.2.0.tar.gz}{\pandocbounded{\includesvg[keepaspectratio]{/assets/icons/16-download.svg}}}
\item[Verification:]
We verified that the author is affiliated with their institution
\pandocbounded{\includesvg[keepaspectratio]{/assets/icons/16-verified.svg}}
\item[Repository:]
\href{https://github.com/TGM-HIT/typst-diploma-thesis}{GitHub}
\item[Discipline :]
\begin{itemize}
\tightlist
\item[]
\item
  \href{https://typst.app/universe/search/?discipline=computer-science}{Computer
  Science}
\end{itemize}
\item[Categor y :]
\begin{itemize}
\tightlist
\item[]
\item
  \pandocbounded{\includesvg[keepaspectratio]{/assets/icons/16-mortarboard.svg}}
  \href{https://typst.app/universe/search/?category=thesis}{Thesis}
\end{itemize}
\end{description}

\subsubsection{Where to report issues?}\label{where-to-report-issues}

This template is a project of Clemens Koza . Report issues on
\href{https://github.com/TGM-HIT/typst-diploma-thesis}{their repository}
. You can also try to ask for help with this template on the
\href{https://forum.typst.app}{Forum} .

Please report this template to the Typst team using the
\href{https://typst.app/contact}{contact form} if you believe it is a
safety hazard or infringes upon your rights.

\phantomsection\label{versions}
\subsubsection{Version history}\label{version-history}

\begin{longtable}[]{@{}ll@{}}
\toprule\noalign{}
Version & Release Date \\
\midrule\noalign{}
\endhead
\bottomrule\noalign{}
\endlastfoot
0.2.0 & October 24, 2024 \\
\href{https://typst.app/universe/package/tgm-hit-thesis/0.1.3/}{0.1.3} &
September 15, 2024 \\
\href{https://typst.app/universe/package/tgm-hit-thesis/0.1.2/}{0.1.2} &
September 14, 2024 \\
\href{https://typst.app/universe/package/tgm-hit-thesis/0.1.1/}{0.1.1} &
September 11, 2024 \\
\href{https://typst.app/universe/package/tgm-hit-thesis/0.1.0/}{0.1.0} &
July 15, 2024 \\
\end{longtable}

Typst GmbH did not create this template and cannot guarantee correct
functionality of this template or compatibility with any version of the
Typst compiler or app.


\title{typst.app/universe/package/diagraph}

\phantomsection\label{banner}
\section{diagraph}\label{diagraph}

{ 0.3.0 }

Draw graphs with Graphviz. Use mathematical formulas as labels.

\phantomsection\label{readme}
A simple Graphviz binding for Typst using the WebAssembly plugin system.

\subsection{Usage}\label{usage}

\subsubsection{Basic usage}\label{basic-usage}

You can render a Graphviz Dot string to a SVG image using the
\texttt{\ render\ } function:

\begin{Shaded}
\begin{Highlighting}[]
\NormalTok{\#render("digraph \{ a {-}\textgreater{} b \}")}
\end{Highlighting}
\end{Shaded}

Alternatively, you can use \texttt{\ raw-render\ } to pass a
\texttt{\ raw\ } instead of a string:

\begin{Shaded}
\begin{Highlighting}[]
\NormalTok{\#raw{-}render(}
\NormalTok{  \textasciigrave{}\textasciigrave{}\textasciigrave{}dot}
\NormalTok{  digraph \{}
\NormalTok{    a {-}\textgreater{} b}
\NormalTok{  \}}
\NormalTok{  \textasciigrave{}\textasciigrave{}\textasciigrave{}}
\NormalTok{)}
\end{Highlighting}
\end{Shaded}

\pandocbounded{\includegraphics[keepaspectratio]{https://raw.githubusercontent.com/Robotechnic/diagraph/main/images/raw-render1.png}}

For more information about the Graphviz Dot language, you can check the
\href{https://graphviz.org/documentation/}{official documentation} .

\subsubsection{Advanced usage}\label{advanced-usage}

Check the
\href{https://raw.githubusercontent.com/Robotechnic/diagraph/main/doc/manual.pdf}{manual}
for more information about the plugin.

\subsection{License}\label{license}

This project is licensed under the MIT License - see the
\href{https://github.com/typst/packages/raw/main/packages/preview/diagraph/0.3.0/LICENSE}{LICENSE}
file for details

\subsection{Changelog}\label{changelog}

\subsubsection{0.3.0}\label{section}

\begin{itemize}
\tightlist
\item
  Added support for edge labels
\item
  Added a manual generated with Typst
\item
  Updated graphviz version
\item
  Fix an error in math mode detection
\end{itemize}

\subsubsection{0.2.5}\label{section-1}

\begin{itemize}
\tightlist
\item
  If the shape is point, the label isn’t displayed
\item
  Now a minimum size is not enforced if the node label is empty
\item
  Added support for font alternatives
\end{itemize}

\subsubsection{0.2.4}\label{section-2}

\begin{itemize}
\tightlist
\item
  Added support for xlabels which are now rendered by Typst
\item
  Added support for cluster labels which are now rendered by Typst
\item
  Fix a margin problem with the clusters
\end{itemize}

\subsubsection{0.2.3}\label{section-3}

\begin{itemize}
\tightlist
\item
  Updated to typst 0.11.0
\item
  Added support for \texttt{\ fontcolor\ } , \texttt{\ fontsize\ } and
  \texttt{\ fontname\ } nodes attributes
\item
  Diagraph now uses a protocol generator to generate the wasm interface
\end{itemize}

\subsubsection{0.2.2}\label{section-4}

\begin{itemize}
\tightlist
\item
  Fix an alignment issue
\item
  Added a better mathematic formula recognition for node labels
\end{itemize}

\subsubsection{0.2.1}\label{section-5}

\begin{itemize}
\tightlist
\item
  Added support for relative lenghts in the \texttt{\ width\ } and
  \texttt{\ height\ } arguments
\item
  Fix various bugs
\end{itemize}

\subsubsection{0.2.0}\label{section-6}

\begin{itemize}
\tightlist
\item
  Node labels are now handled by Typst
\end{itemize}

\subsubsection{0.1.2}\label{section-7}

\begin{itemize}
\tightlist
\item
  Graphs are now scaled to make the graph text size match the document
  text size
\end{itemize}

\subsubsection{0.1.1}\label{section-8}

\begin{itemize}
\tightlist
\item
  Remove the \texttt{\ raw-render-rule\ } show rule because it doesn’t
  allow use of custom font and the \texttt{\ render\ } /
  \texttt{\ raw-render\ } functions are more flexible
\item
  Add the \texttt{\ background\ } parameter to the \texttt{\ render\ }
  and \texttt{\ raw-render\ } typst functions and default it to
  \texttt{\ transparent\ } instead of \texttt{\ white\ }
\item
  Add center attribute to draw graph in the center of the svg in the
  \texttt{\ render\ } c function
\end{itemize}

\subsubsection{0.1.0}\label{section-9}

Initial working version

\subsubsection{How to add}\label{how-to-add}

Copy this into your project and use the import as \texttt{\ diagraph\ }

\begin{verbatim}
#import "@preview/diagraph:0.3.0"
\end{verbatim}

\includesvg[width=0.16667in,height=0.16667in]{/assets/icons/16-copy.svg}

Check the docs for
\href{https://typst.app/docs/reference/scripting/\#packages}{more
information on how to import packages} .

\subsubsection{About}\label{about}

\begin{description}
\tightlist
\item[Author s :]
\href{https://github.com/Robotechnic}{Robotechnic} \&
\href{https://github.com/MDLC01}{Malo}
\item[License:]
MIT
\item[Current version:]
0.3.0
\item[Last updated:]
September 3, 2024
\item[First released:]
September 23, 2023
\item[Minimum Typst version:]
0.11.0
\item[Archive size:]
450 kB
\href{https://packages.typst.org/preview/diagraph-0.3.0.tar.gz}{\pandocbounded{\includesvg[keepaspectratio]{/assets/icons/16-download.svg}}}
\item[Repository:]
\href{https://github.com/Robotechnic/diagraph.git}{GitHub}
\item[Categor ies :]
\begin{itemize}
\tightlist
\item[]
\item
  \pandocbounded{\includesvg[keepaspectratio]{/assets/icons/16-package.svg}}
  \href{https://typst.app/universe/search/?category=components}{Components}
\item
  \pandocbounded{\includesvg[keepaspectratio]{/assets/icons/16-chart.svg}}
  \href{https://typst.app/universe/search/?category=visualization}{Visualization}
\item
  \pandocbounded{\includesvg[keepaspectratio]{/assets/icons/16-integration.svg}}
  \href{https://typst.app/universe/search/?category=integration}{Integration}
\end{itemize}
\end{description}

\subsubsection{Where to report issues?}\label{where-to-report-issues}

This package is a project of Robotechnic and Malo . Report issues on
\href{https://github.com/Robotechnic/diagraph.git}{their repository} .
You can also try to ask for help with this package on the
\href{https://forum.typst.app}{Forum} .

Please report this package to the Typst team using the
\href{https://typst.app/contact}{contact form} if you believe it is a
safety hazard or infringes upon your rights.

\phantomsection\label{versions}
\subsubsection{Version history}\label{version-history}

\begin{longtable}[]{@{}ll@{}}
\toprule\noalign{}
Version & Release Date \\
\midrule\noalign{}
\endhead
\bottomrule\noalign{}
\endlastfoot
0.3.0 & September 3, 2024 \\
\href{https://typst.app/universe/package/diagraph/0.2.5/}{0.2.5} & June
11, 2024 \\
\href{https://typst.app/universe/package/diagraph/0.2.4/}{0.2.4} & May
23, 2024 \\
\href{https://typst.app/universe/package/diagraph/0.2.3/}{0.2.3} & May
13, 2024 \\
\href{https://typst.app/universe/package/diagraph/0.2.2/}{0.2.2} & March
15, 2024 \\
\href{https://typst.app/universe/package/diagraph/0.2.1/}{0.2.1} &
January 16, 2024 \\
\href{https://typst.app/universe/package/diagraph/0.2.0/}{0.2.0} &
November 18, 2023 \\
\href{https://typst.app/universe/package/diagraph/0.1.2/}{0.1.2} &
November 6, 2023 \\
\href{https://typst.app/universe/package/diagraph/0.1.1/}{0.1.1} &
September 28, 2023 \\
\href{https://typst.app/universe/package/diagraph/0.1.0/}{0.1.0} &
September 23, 2023 \\
\end{longtable}

Typst GmbH did not create this package and cannot guarantee correct
functionality of this package or compatibility with any version of the
Typst compiler or app.


\title{typst.app/universe/package/board-n-pieces}

\phantomsection\label{banner}
\section{board-n-pieces}\label{board-n-pieces}

{ 0.5.0 }

Display chessboards.

\phantomsection\label{readme}
Display chessboards in Typst.

\subsection{Displaying chessboards}\label{displaying-chessboards}

The main function of this package is \texttt{\ board\ } . It lets you
display a specific position on a board.

\begin{Shaded}
\begin{Highlighting}[]
\NormalTok{\#board(starting{-}position)}
\end{Highlighting}
\end{Shaded}

\pandocbounded{\includesvg[keepaspectratio]{https://github.com/typst/packages/raw/main/packages/preview/board-n-pieces/0.5.0/examples/example-1.svg}}

\texttt{\ starting-position\ } is a position that is provided by the
package. It represents the initial position of a chess game.

You can create a different position using the \texttt{\ position\ }
function. It accepts strings representing each rank. Use upper-case
letters for white pieces, and lower-case letters for black pieces. Dots
and spaces correspond to empty squares.

\begin{Shaded}
\begin{Highlighting}[]
\NormalTok{\#board(position(}
\NormalTok{  "....r...",}
\NormalTok{  "........",}
\NormalTok{  "..p..PPk",}
\NormalTok{  ".p.r....",}
\NormalTok{  "pP..p.R.",}
\NormalTok{  "P.B.....",}
\NormalTok{  "..P..K..",}
\NormalTok{  "........",}
\NormalTok{))}
\end{Highlighting}
\end{Shaded}

\pandocbounded{\includesvg[keepaspectratio]{https://github.com/typst/packages/raw/main/packages/preview/board-n-pieces/0.5.0/examples/example-2.svg}}

Alternatively, you can use the \texttt{\ fen\ } function to create a
position using
\href{https://en.wikipedia.org/wiki/Forsyth\%E2\%80\%93Edwards_Notation}{Forsythâ€``Edwards
notation} :

\begin{Shaded}
\begin{Highlighting}[]
\NormalTok{\#board(fen("r1bk3r/p2pBpNp/n4n2/1p1NP2P/6P1/3P4/P1P1K3/q5b1 b {-} {-} 1 23"))}
\end{Highlighting}
\end{Shaded}

\pandocbounded{\includesvg[keepaspectratio]{https://github.com/typst/packages/raw/main/packages/preview/board-n-pieces/0.5.0/examples/example-3.svg}}

Note that you can specify only the first part of the FEN string:

\begin{Shaded}
\begin{Highlighting}[]
\NormalTok{\#board(fen("r4rk1/pp2Bpbp/1qp3p1/8/2BP2b1/Q1n2N2/P4PPP/3RK2R"))}
\end{Highlighting}
\end{Shaded}

\pandocbounded{\includesvg[keepaspectratio]{https://github.com/typst/packages/raw/main/packages/preview/board-n-pieces/0.5.0/examples/example-4.svg}}

Also note that positions do not need to be on a standard 8Ã---8 board:

\begin{Shaded}
\begin{Highlighting}[]
\NormalTok{\#board(position(}
\NormalTok{  "....Q....",}
\NormalTok{  "......Q..",}
\NormalTok{  "........Q",}
\NormalTok{  "...Q.....",}
\NormalTok{  ".Q.......",}
\NormalTok{  ".......Q.",}
\NormalTok{  ".....Q...",}
\NormalTok{  "..Q......",}
\NormalTok{  "Q........",}
\NormalTok{))}
\end{Highlighting}
\end{Shaded}

\pandocbounded{\includesvg[keepaspectratio]{https://github.com/typst/packages/raw/main/packages/preview/board-n-pieces/0.5.0/examples/example-5.svg}}

\subsection{\texorpdfstring{Using the \texttt{\ game\ }
function}{Using the  game  function}}\label{using-the-game-function}

The \texttt{\ game\ } function creates an array of positions from a full
chess game. A game is described by a series of turns written in
\href{https://en.wikipedia.org/wiki/Algebraic_notation_(chess)}{standard
algebraic notation} . Those turns can be specified as an array of
strings, or as a single string containing whitespace-separated moves.

\begin{Shaded}
\begin{Highlighting}[]
\NormalTok{The scholar\textquotesingle{}s mate:}
\NormalTok{\#let positions = game("e4 e5 Qh5 Nc6 Bc4 Nf6 Qxf7")}
\NormalTok{\#grid(}
\NormalTok{  columns: 4,}
\NormalTok{  gutter: 0.2cm,}
\NormalTok{  ..positions.map(board.with(square{-}size: 0.5cm)),}
\NormalTok{)}
\end{Highlighting}
\end{Shaded}

\pandocbounded{\includesvg[keepaspectratio]{https://github.com/typst/packages/raw/main/packages/preview/board-n-pieces/0.5.0/examples/example-6.svg}}

You can specify an alternative starting position to the
\texttt{\ game\ } function with the \texttt{\ starting-position\ } named
argument.

\subsection{\texorpdfstring{Using the \texttt{\ pgn\ } function to
import PGN
files}{Using the  pgn  function to import PGN files}}\label{using-the-pgn-function-to-import-pgn-files}

Similarly to the \texttt{\ game\ } function, the \texttt{\ pgn\ }
function creates an array of positions. It accepts a single argument,
which is a string containing
\href{https://en.wikipedia.org/wiki/Portable_Game_Notation}{portable
game notation} . To read a game from a PGN file, you can use this
function in combination with Typst’s native
\href{https://typst.app/docs/reference/data-loading/read/}{\texttt{\ read\ }}
function.

\begin{Shaded}
\begin{Highlighting}[]
\NormalTok{\#let positions = pgn(read("game.pgn"))}
\end{Highlighting}
\end{Shaded}

Note that the argument to \texttt{\ pgn\ } must describe a single game.
If you have a PGN file containing multiple games, you will need to split
them using other means.

\subsection{Using non-standard chess
pieces}\label{using-non-standard-chess-pieces}

The \texttt{\ board\ } function’s \texttt{\ pieces\ } argument lets
you specify how to display each piece by mapping each piece character to
some content. You can use this feature to display non-standard chess
pieces:

\begin{Shaded}
\begin{Highlighting}[]
\NormalTok{\#board(}
\NormalTok{  fen("g7/5g2/8/8/8/8/p6g/k1K4G"),}
\NormalTok{  pieces: (}
\NormalTok{    // We use symbols for the example.}
\NormalTok{    // In practice, you should import your own images.}
\NormalTok{    g: chess{-}sym.queen.black.b,}
\NormalTok{    p: chess{-}sym.pawn.black,}
\NormalTok{    k: chess{-}sym.king.black,}
\NormalTok{    K: chess{-}sym.king.white,}
\NormalTok{    G: chess{-}sym.queen.white.b,}
\NormalTok{  ),}
\NormalTok{)}
\end{Highlighting}
\end{Shaded}

\pandocbounded{\includesvg[keepaspectratio]{https://github.com/typst/packages/raw/main/packages/preview/board-n-pieces/0.5.0/examples/example-7.svg}}

\subsection{Customizing a chessboard}\label{customizing-a-chessboard}

The \texttt{\ board\ } function lets you customize the appearance of the
board in various ways, as illustrated in the example below.

\begin{Shaded}
\begin{Highlighting}[]
\NormalTok{// From https://lichess.org/study/Xf1PGrM0.}
\NormalTok{\#board(}
\NormalTok{  fen("3k4/7R/8/2PK4/8/8/8/6r1 b {-} {-} 0 1"),}

\NormalTok{  marked{-}squares: "c7 c6 h6",}
\NormalTok{  arrows: ("d8 c8", "d8 c7", "g1 g6", "h7 h6"),}
\NormalTok{  display{-}numbers: true,}

\NormalTok{  white{-}square{-}fill: rgb("\#d2eeea"),}
\NormalTok{  black{-}square{-}fill: rgb("\#567f96"),}
\NormalTok{  marking{-}color: rgb("\#2bcbC6"),}
\NormalTok{  arrow{-}stroke: 0.2cm + rgb("\#38f442df"),}

\NormalTok{  stroke: 0.8pt + black,}
\NormalTok{)}
\end{Highlighting}
\end{Shaded}

\pandocbounded{\includesvg[keepaspectratio]{https://github.com/typst/packages/raw/main/packages/preview/board-n-pieces/0.5.0/examples/example-8.svg}}

Here is a list of all the available arguments:

\begin{itemize}
\item
  \texttt{\ marked-squares\ } is a list of squares to mark (e.g.,
  \texttt{\ ("d3",\ "d2",\ "e3")\ } ). It can also be specified as a
  single string containing whitespace-separated squares (e.g.,
  \texttt{\ "d3\ d2\ e3"\ } ).
\item
  \texttt{\ arrows\ } is a list of arrows to draw (e.g.,
  \texttt{\ ("e2\ e4",\ "e7\ e5")\ } ).
\item
  \texttt{\ reverse\ } is a boolean indicating whether to reverse the
  board, displaying it from black’s point of view. This is
  \texttt{\ false\ } by default, meaning the board is displayed from
  white’s point of view.
\item
  \texttt{\ display-numbers\ } is a boolean indicating whether ranks and
  files should be numbered. This is \texttt{\ false\ } by default.
\item
  \texttt{\ rank-numbering\ } and \texttt{\ file-numbering\ } are
  functions describing how ranks and files should be numbered. By
  default they are respectively \texttt{\ numbering.with("1")\ } and
  \texttt{\ numbering.with("a")\ } .
\item
  \texttt{\ square-size\ } is a length describing the size of each
  square. By default, this is \texttt{\ 1cm\ } .
\item
  \texttt{\ white-square-fill\ } and \texttt{\ black-square-fill\ }
  indicate how squares should be filled. They can be colors, gradient or
  patterns.
\item
  \texttt{\ marking-color\ } is the color to use for markings (marked
  squares and arrows).
\item
  \texttt{\ marked-white-square-background\ } and
  \texttt{\ marked-black-square-background\ } define the content to
  display in the background of marked squares. By default, this is a
  circle using the \texttt{\ marking-color\ } .
\item
  \texttt{\ arrow-stroke\ } is the stroke to draw the arrows with. If
  only a length is given, \texttt{\ marking-color\ } is used.
  Alternatively, a stroke can be passed to specify a different color.
\item
  \texttt{\ pieces\ } is a dictionary containing images representing
  each piece. If specified, the dictionary must contain an entry for
  every piece kind in the displayed position. Keys are single upper-case
  letters for white pieces and single lower-case letters for black
  pieces. The default images are taken from
  \href{https://commons.wikimedia.org/wiki/Category:SVG_chess_pieces}{Wikimedia
  Commons} . Please refer to
  \href{https://github.com/typst/packages/raw/main/packages/preview/board-n-pieces/0.5.0/\#licensing}{the
  section on licensing} for information on how you can use them in your
  documents.
\item
  \texttt{\ stroke\ } has the same structure as
  \href{https://typst.app/docs/reference/visualize/rect/\#parameters-stroke}{\texttt{\ rect\ }
  ’s \texttt{\ stroke\ } parameter} and corresponds to the stroke to
  use around the board. If \texttt{\ display-numbers\ } is
  \texttt{\ true\ } , the numbers are displayed outside the stroke. The
  default value is \texttt{\ none\ } .
\end{itemize}

\subsection{Chess symbols}\label{chess-symbols}

This package also exports chess symbols for all Unicode chess-related
codepoints under the \texttt{\ chess-sym\ } submodule. Standard chess
pieces are available as
\texttt{\ chess-sym.\{pawn,knight,bishop,rook,queen,king\}.\{white,black,neutral\}\ }
. Alternatively, you can use \texttt{\ stroked\ } and
\texttt{\ filled\ } instead of, respectively, \texttt{\ white\ } and
\texttt{\ black\ } . They can be rotated rightward, downward, and
leftward respectively with with \texttt{\ .r\ } , \texttt{\ .b\ } , and
\texttt{\ .l\ } . Chinese chess pieces are also available as
\texttt{\ chess-sym.\{soldier,cannon,chariot,horse,elephant,mandarin,general\}.\{red,black\}\ }
. Similarly, you can use \texttt{\ stroked\ } and \texttt{\ filled\ } as
alternatives to, respectively, \texttt{\ red\ } and \texttt{\ black\ } .
Note that most fonts only support black and white versions of standard
pieces. To use the other symbols, you may have to use a font such as
Noto Sans Symbols 2.

\begin{Shaded}
\begin{Highlighting}[]
\NormalTok{The best move in this position is \#chess{-}sym.knight.white;c6.}
\end{Highlighting}
\end{Shaded}

\pandocbounded{\includesvg[keepaspectratio]{https://github.com/typst/packages/raw/main/packages/preview/board-n-pieces/0.5.0/examples/example-9.svg}}

\subsection{Licensing}\label{licensing}

The default images for chess pieces used by the \texttt{\ board\ }
function come from
\href{https://commons.wikimedia.org/wiki/Category:SVG_chess_pieces}{Wikimedia
Commons} . They are all licensed the
\href{https://www.gnu.org/licenses/old-licenses/gpl-2.0.html}{GNU
General Public License, version 2} by their original author:
\href{https://en.wikipedia.org/wiki/User:Cburnett}{Cburnett} .

\subsection{Changelog}\label{changelog}

\subsubsection{Version 0.5.0}\label{version-0.5.0}

\begin{itemize}
\item
  Add symbols for all Unicode chess-related codepoints.
\item
  Change the signature of the \texttt{\ board\ } function.

  \begin{itemize}
  \tightlist
  \item
    Rename argument \texttt{\ highlighted-squares\ } to
    \texttt{\ marked-squares\ } .
  \item
    Remove arguments \texttt{\ highlighted-white-square-fill\ } and
    \texttt{\ highlighted-black-square-fill\ } .
  \item
    Add argument \texttt{\ marking-color\ } , together with
    \texttt{\ marked-white-square-background\ } and
    \texttt{\ marked-black-square-background\ } .
  \item
    Support passing a length as \texttt{\ arrow-stroke\ } .
  \end{itemize}
\item
  Fix arrows not being displayed properly on reversed boards.
\end{itemize}

\subsubsection{Version 0.4.0}\label{version-0.4.0}

\begin{itemize}
\tightlist
\item
  Add the ability to draw arrows in \texttt{\ board\ } .
\end{itemize}

\subsubsection{Version 0.3.0}\label{version-0.3.0}

\begin{itemize}
\item
  Detect moves that put the king in check as illegal, improving SAN
  support.
\item
  Add \texttt{\ stroke\ } argument to the \texttt{\ board\ } function.
\item
  Rename \texttt{\ \{highlighted-,\}\{white,black\}-square-color\ }
  arguments to the \texttt{\ board\ } function to
  \texttt{\ \{highlighted-,\}\{white,black\}-square-fill\ } .
\end{itemize}

\subsubsection{Version 0.2.0}\label{version-0.2.0}

\begin{itemize}
\item
  Allow using dashes for empty squares in \texttt{\ position\ }
  function.
\item
  Allow passing highlighted squares as a single string of
  whitespace-separated squares.
\item
  Describe entire games using algebraic notation with the
  \texttt{\ game\ } function.
\item
  Initial PGN support through the \texttt{\ pgn\ } function.
\end{itemize}

\subsubsection{Version 0.1.0}\label{version-0.1.0}

\begin{itemize}
\item
  Display a chess position on a chessboard with the \texttt{\ board\ }
  function.
\item
  Get the starting position with \texttt{\ starting-position\ } .
\item
  Use chess-related symbols with the \texttt{\ chess-sym\ } module.
\end{itemize}

\subsubsection{How to add}\label{how-to-add}

Copy this into your project and use the import as
\texttt{\ board-n-pieces\ }

\begin{verbatim}
#import "@preview/board-n-pieces:0.5.0"
\end{verbatim}

\includesvg[width=0.16667in,height=0.16667in]{/assets/icons/16-copy.svg}

Check the docs for
\href{https://typst.app/docs/reference/scripting/\#packages}{more
information on how to import packages} .

\subsubsection{About}\label{about}

\begin{description}
\tightlist
\item[Author :]
\href{https://github.com/MDLC01}{Malo}
\item[License:]
MIT AND GPL-2.0-only
\item[Current version:]
0.5.0
\item[Last updated:]
July 22, 2024
\item[First released:]
March 20, 2024
\item[Minimum Typst version:]
0.11.0
\item[Archive size:]
52.2 kB
\href{https://packages.typst.org/preview/board-n-pieces-0.5.0.tar.gz}{\pandocbounded{\includesvg[keepaspectratio]{/assets/icons/16-download.svg}}}
\item[Repository:]
\href{https://github.com/MDLC01/board-n-pieces}{GitHub}
\item[Discipline s :]
\begin{itemize}
\tightlist
\item[]
\item
  \href{https://typst.app/universe/search/?discipline=computer-science}{Computer
  Science}
\item
  \href{https://typst.app/universe/search/?discipline=mathematics}{Mathematics}
\end{itemize}
\item[Categor y :]
\begin{itemize}
\tightlist
\item[]
\item
  \pandocbounded{\includesvg[keepaspectratio]{/assets/icons/16-chart.svg}}
  \href{https://typst.app/universe/search/?category=visualization}{Visualization}
\end{itemize}
\end{description}

\subsubsection{Where to report issues?}\label{where-to-report-issues}

This package is a project of Malo . Report issues on
\href{https://github.com/MDLC01/board-n-pieces}{their repository} . You
can also try to ask for help with this package on the
\href{https://forum.typst.app}{Forum} .

Please report this package to the Typst team using the
\href{https://typst.app/contact}{contact form} if you believe it is a
safety hazard or infringes upon your rights.

\phantomsection\label{versions}
\subsubsection{Version history}\label{version-history}

\begin{longtable}[]{@{}ll@{}}
\toprule\noalign{}
Version & Release Date \\
\midrule\noalign{}
\endhead
\bottomrule\noalign{}
\endlastfoot
0.5.0 & July 22, 2024 \\
\href{https://typst.app/universe/package/board-n-pieces/0.4.0/}{0.4.0} &
July 8, 2024 \\
\href{https://typst.app/universe/package/board-n-pieces/0.3.0/}{0.3.0} &
May 23, 2024 \\
\href{https://typst.app/universe/package/board-n-pieces/0.2.0/}{0.2.0} &
April 29, 2024 \\
\href{https://typst.app/universe/package/board-n-pieces/0.1.0/}{0.1.0} &
March 20, 2024 \\
\end{longtable}

Typst GmbH did not create this package and cannot guarantee correct
functionality of this package or compatibility with any version of the
Typst compiler or app.


\title{typst.app/universe/package/light-report-uia}

\phantomsection\label{banner}
\phantomsection\label{template-thumbnail}
\pandocbounded{\includegraphics[keepaspectratio]{https://packages.typst.org/preview/thumbnails/light-report-uia-0.1.0-small.webp}}

\section{light-report-uia}\label{light-report-uia}

{ 0.1.0 }

Template for reports at the University of Agder

\href{/app?template=light-report-uia&version=0.1.0}{Create project in
app}

\phantomsection\label{readme}
Unofficial report template for reports at the University of Agder.

Supports both norwegian and english.

Usage:

\begin{verbatim}
#import "@preview/light-report-uia:0.1.0": *

// CHANGE THESE
#show: report.with(
  title: "New project",
  authors: (
    "Lars Larsen",
    "Lise Lisesen",
    "Knut Knutsen",
  ),
  group_name: "Group 14",
  course_code: "IKT123-G",
  course_name: "Course name",
  date: "august 2024",
  lang: "en", // use "no" for norwegian
)
// then do anything
\end{verbatim}

\href{/app?template=light-report-uia&version=0.1.0}{Create project in
app}

\subsubsection{How to use}\label{how-to-use}

Click the button above to create a new project using this template in
the Typst app.

You can also use the Typst CLI to start a new project on your computer
using this command:

\begin{verbatim}
typst init @preview/light-report-uia:0.1.0
\end{verbatim}

\includesvg[width=0.16667in,height=0.16667in]{/assets/icons/16-copy.svg}

\subsubsection{About}\label{about}

\begin{description}
\tightlist
\item[Author :]
Sebastian Dalheim Knudsen
\item[License:]
MIT
\item[Current version:]
0.1.0
\item[Last updated:]
September 14, 2024
\item[First released:]
September 14, 2024
\item[Archive size:]
7.18 kB
\href{https://packages.typst.org/preview/light-report-uia-0.1.0.tar.gz}{\pandocbounded{\includesvg[keepaspectratio]{/assets/icons/16-download.svg}}}
\item[Repository:]
\href{https://github.com/sebastos1/light-report-uia}{GitHub}
\item[Categor y :]
\begin{itemize}
\tightlist
\item[]
\item
  \pandocbounded{\includesvg[keepaspectratio]{/assets/icons/16-speak.svg}}
  \href{https://typst.app/universe/search/?category=report}{Report}
\end{itemize}
\end{description}

\subsubsection{Where to report issues?}\label{where-to-report-issues}

This template is a project of Sebastian Dalheim Knudsen . Report issues
on \href{https://github.com/sebastos1/light-report-uia}{their
repository} . You can also try to ask for help with this template on the
\href{https://forum.typst.app}{Forum} .

Please report this template to the Typst team using the
\href{https://typst.app/contact}{contact form} if you believe it is a
safety hazard or infringes upon your rights.

\phantomsection\label{versions}
\subsubsection{Version history}\label{version-history}

\begin{longtable}[]{@{}ll@{}}
\toprule\noalign{}
Version & Release Date \\
\midrule\noalign{}
\endhead
\bottomrule\noalign{}
\endlastfoot
0.1.0 & September 14, 2024 \\
\end{longtable}

Typst GmbH did not create this template and cannot guarantee correct
functionality of this template or compatibility with any version of the
Typst compiler or app.


\title{typst.app/universe/package/down}

\phantomsection\label{banner}
\section{down}\label{down}

{ 0.1.0 }

Pass down arguments of `sum`, `integral`, etc. to the next line, which
can generate shorthand to present reusable segments.

\phantomsection\label{readme}
Pass down arguments of \texttt{\ sum\ } , \texttt{\ integral\ } , etc.
to the next line, which can generate shorthand to present reusable
segments. While writing long step-by-step equations, only certain parts
of a line change. \texttt{\ down\ } leverages Typst’s
\texttt{\ context\ } (from version 0.11.0) to help relieve the pressure
of writing long and repetitive formulae.

Import the package:

\begin{Shaded}
\begin{Highlighting}[]
\NormalTok{\#import "@preview/down:0.1.0": *}
\end{Highlighting}
\end{Shaded}

\subsection{Usage}\label{usage}

Create new contexts by using camel-case commands, such as
\texttt{\ Limit(x,\ +0)\ } . Retrieve the contextual with
\texttt{\ cLimit\ } .

\begin{itemize}
\tightlist
\item
  \texttt{\ Limit(x,\ c)\ } and \texttt{\ cLimit\ } :
\end{itemize}

\begin{Shaded}
\begin{Highlighting}[]
\NormalTok{$}
\NormalTok{Lim(x, +0) x ln(sin x)}
\NormalTok{  = cLim ln(sin x) / x\^{}({-}1)}
\NormalTok{  = cLim x / (sin x) cos x}
\NormalTok{  = 0}
\NormalTok{$}
\end{Highlighting}
\end{Shaded}

\begin{itemize}
\tightlist
\item
  \texttt{\ Sum(index,\ lower,\ upper)\ } and \texttt{\ cSum\ } :
\end{itemize}

\begin{Shaded}
\begin{Highlighting}[]
\NormalTok{$}
\NormalTok{Sum(n, 0, oo) 1 / sqrt(n + 1)}
\NormalTok{  = Sum(\#none, 0, \#none) 1 / sqrt(n)}
\NormalTok{  = cSum 1 / n\^{}(1 / 2)}
\NormalTok{$}
\end{Highlighting}
\end{Shaded}

\begin{itemize}
\tightlist
\item
  \texttt{\ Integral(lower,\ upper,\ f,\ dif:\ {[}x{]})\ } ,
  \texttt{\ cIntegral(f)\ } and \texttt{\ cIntegrated(f)\ } :
\end{itemize}

\begin{Shaded}
\begin{Highlighting}[]
\NormalTok{$}
\NormalTok{Integral(0, pi / 3, sqrt(1 + tan\^{}2 x))}
\NormalTok{  = cIntegral(1 / (cos x))}
\NormalTok{  = cIntegrated(ln (cos x / 2 + sin x / 2) / (cos x / 2 {-} sin x / 2))}
\NormalTok{  = ln (2 + sqrt(3))}
\NormalTok{$}
\end{Highlighting}
\end{Shaded}

Refer to \texttt{\ ./sample.pdf\ } for more complex application.

\subsubsection{How to add}\label{how-to-add}

Copy this into your project and use the import as \texttt{\ down\ }

\begin{verbatim}
#import "@preview/down:0.1.0"
\end{verbatim}

\includesvg[width=0.16667in,height=0.16667in]{/assets/icons/16-copy.svg}

Check the docs for
\href{https://typst.app/docs/reference/scripting/\#packages}{more
information on how to import packages} .

\subsubsection{About}\label{about}

\begin{description}
\tightlist
\item[Author :]
\href{mailto:the@unpopular.me}{Toto}
\item[License:]
MIT
\item[Current version:]
0.1.0
\item[Last updated:]
April 1, 2024
\item[First released:]
April 1, 2024
\item[Minimum Typst version:]
0.11.0
\item[Archive size:]
2.15 kB
\href{https://packages.typst.org/preview/down-0.1.0.tar.gz}{\pandocbounded{\includesvg[keepaspectratio]{/assets/icons/16-download.svg}}}
\item[Repository:]
\href{https://git.sr.ht/~toto/down}{git.sr.ht}
\item[Discipline :]
\begin{itemize}
\tightlist
\item[]
\item
  \href{https://typst.app/universe/search/?discipline=mathematics}{Mathematics}
\end{itemize}
\item[Categor y :]
\begin{itemize}
\tightlist
\item[]
\item
  \pandocbounded{\includesvg[keepaspectratio]{/assets/icons/16-hammer.svg}}
  \href{https://typst.app/universe/search/?category=utility}{Utility}
\end{itemize}
\end{description}

\subsubsection{Where to report issues?}\label{where-to-report-issues}

This package is a project of Toto . Report issues on
\href{https://git.sr.ht/~toto/down}{their repository} . You can also try
to ask for help with this package on the
\href{https://forum.typst.app}{Forum} .

Please report this package to the Typst team using the
\href{https://typst.app/contact}{contact form} if you believe it is a
safety hazard or infringes upon your rights.

\phantomsection\label{versions}
\subsubsection{Version history}\label{version-history}

\begin{longtable}[]{@{}ll@{}}
\toprule\noalign{}
Version & Release Date \\
\midrule\noalign{}
\endhead
\bottomrule\noalign{}
\endlastfoot
0.1.0 & April 1, 2024 \\
\end{longtable}

Typst GmbH did not create this package and cannot guarantee correct
functionality of this package or compatibility with any version of the
Typst compiler or app.


\title{typst.app/universe/package/lucky-icml}

\phantomsection\label{banner}
\phantomsection\label{template-thumbnail}
\pandocbounded{\includegraphics[keepaspectratio]{https://packages.typst.org/preview/thumbnails/lucky-icml-0.2.1-small.webp}}

\section{lucky-icml}\label{lucky-icml}

{ 0.2.1 }

ICML-style paper template to publish at conferences for International
Conference on Machine Learning

\href{/app?template=lucky-icml&version=0.2.1}{Create project in app}

\phantomsection\label{readme}
\subsection{Usage}\label{usage}

You can use this template in the Typst web app by clicking \emph{Start
from template} on the dashboard and searching for
\texttt{\ lucky-icml\ } .

Alternatively, you can use the CLI to kick this project off using the
command

\begin{Shaded}
\begin{Highlighting}[]
\NormalTok{typst init @preview/lucky{-}icml}
\end{Highlighting}
\end{Shaded}

Typst will create a new directory with all the files needed to get you
started.

\subsection{Configuration}\label{configuration}

This template exports the \texttt{\ icml2024\ } function with the
following named arguments.

\begin{itemize}
\tightlist
\item
  \texttt{\ title\ } : The paper’s title as content.
\item
  \texttt{\ authors\ } : An array of author dictionaries. Each of the
  author dictionaries must have a name key and can have the keys
  department, organization, location, and email.
\item
  \texttt{\ abstract\ } : The content of a brief summary of the paper or
  none. Appears at the top under the title.
\item
  \texttt{\ bibliography\ } : The result of a call to the bibliography
  function or none. The function also accepts a single, positional
  argument for the body of the paper.
\item
  \texttt{\ accepted\ } : If this is set to \texttt{\ false\ } then
  anonymized ready for submission document is produced;
  \texttt{\ accepted:\ true\ } produces camera-redy version. If the
  argument is set to \texttt{\ none\ } then preprint version is produced
  (can be uploaded to arXiv).
\end{itemize}

The template will initialize your package with a sample call to the
\texttt{\ icml2024\ } function in a show rule. If you want to change an
existing project to use this template, you can add a show rule at the
top of your file.

\subsection{Issues}\label{issues}

This template is developed at
\href{https://github.com/daskol/typst-templates}{daskol/typst-templates}
repo. Please report all issues there.

\subsubsection{Running Title}\label{running-title}

\begin{enumerate}
\tightlist
\item
  Runing title should be 10pt above the main text. With top margin 1in
  it gives that a solid line should be located at 62pt. Actual, position
  is 57pt in the original template.
\item
  Default value between header ruler and header text baseline is 4pt in
  \texttt{\ fancyhdr\ } . But actual value is 3pt due to thickness of a
  ruler in 1pt.
\end{enumerate}

\subsubsection{Page Numbering}\label{page-numbering}

\begin{enumerate}
\tightlist
\item
  Basis line of page number should be located 25pt below of main text.
  There is a discrepancy in about \textasciitilde1pt.
\end{enumerate}

\subsubsection{Heading}\label{heading}

\begin{enumerate}
\tightlist
\item
  Required space after level 3 headers is 0.1in or 7.2pt. Actual space
  size is large (e.g. distance between Section 2.3.1 header and text
  after it about 12pt).
\end{enumerate}

\subsubsection{Figures and Tables}\label{figures-and-tables}

\begin{enumerate}
\tightlist
\item
  At the moment Typst has limited support for multi-column documents. It
  allows define multi-column blocks and documents but there is no
  ability to typeset complex layout (e.g. page width figures or tables
  in two-column documents).
\end{enumerate}

\subsubsection{Citations and References}\label{citations-and-references}

\begin{enumerate}
\item
  There is a small bug in CSL processor which fails to recognize
  bibliography entries with \texttt{\ chapter\ } field. It is already
  report and will be fixed in the future.
\item
  There is no suitable bibliography style so we use default APA while
  ICML requires APA-like style but not exact APA. The difference is that
  ICML APA-like style places entry year at the end of reference entry.
  In order to fix the issue, we need to describe ICML bibliography style
  in CSL-format.
\item
  In the original template links are colored with dark blue. We can
  reproduce appearance with some additional effort. First of all,
  \texttt{\ icml2024.csl\ } shoule be updated as follows.

\begin{verbatim}
diff --git a/icml/icml2024.csl b/icml/icml2024.csl
index 3b9e9a2..3fe9f74 100644
--- a/icml/icml2024.csl
+++ b/icml/icml2024.csl
@@ -1648,7 +1648,8 @@
       
       
     
-    
+    
+    
       
         
         
\end{verbatim}

  Then instead of convenient citation shortcut
  \texttt{\ @cite-key1\ @cite-key2\ } , one should use special procedure
  \texttt{\ refer\ } with variadic arguments.

\begin{Shaded}
\begin{Highlighting}[]
\NormalTok{\#refer(\textless{}cite{-}key1\textgreater{}, \textless{}cite{-}key2\textgreater{})}
\end{Highlighting}
\end{Shaded}
\end{enumerate}

\href{/app?template=lucky-icml&version=0.2.1}{Create project in app}

\subsubsection{How to use}\label{how-to-use}

Click the button above to create a new project using this template in
the Typst app.

You can also use the Typst CLI to start a new project on your computer
using this command:

\begin{verbatim}
typst init @preview/lucky-icml:0.2.1
\end{verbatim}

\includesvg[width=0.16667in,height=0.16667in]{/assets/icons/16-copy.svg}

\subsubsection{About}\label{about}

\begin{description}
\tightlist
\item[Author :]
\href{mailto:d.bershatsky2@skoltech.ru}{Daniel Bershatsky}
\item[License:]
MIT
\item[Current version:]
0.2.1
\item[Last updated:]
March 19, 2024
\item[First released:]
March 19, 2024
\item[Minimum Typst version:]
0.10.0
\item[Archive size:]
51.2 kB
\href{https://packages.typst.org/preview/lucky-icml-0.2.1.tar.gz}{\pandocbounded{\includesvg[keepaspectratio]{/assets/icons/16-download.svg}}}
\item[Repository:]
\href{https://github.com/daskol/typst-templates}{GitHub}
\item[Discipline s :]
\begin{itemize}
\tightlist
\item[]
\item
  \href{https://typst.app/universe/search/?discipline=computer-science}{Computer
  Science}
\item
  \href{https://typst.app/universe/search/?discipline=mathematics}{Mathematics}
\end{itemize}
\item[Categor y :]
\begin{itemize}
\tightlist
\item[]
\item
  \pandocbounded{\includesvg[keepaspectratio]{/assets/icons/16-atom.svg}}
  \href{https://typst.app/universe/search/?category=paper}{Paper}
\end{itemize}
\end{description}

\subsubsection{Where to report issues?}\label{where-to-report-issues}

This template is a project of Daniel Bershatsky . Report issues on
\href{https://github.com/daskol/typst-templates}{their repository} . You
can also try to ask for help with this template on the
\href{https://forum.typst.app}{Forum} .

Please report this template to the Typst team using the
\href{https://typst.app/contact}{contact form} if you believe it is a
safety hazard or infringes upon your rights.

\phantomsection\label{versions}
\subsubsection{Version history}\label{version-history}

\begin{longtable}[]{@{}ll@{}}
\toprule\noalign{}
Version & Release Date \\
\midrule\noalign{}
\endhead
\bottomrule\noalign{}
\endlastfoot
0.2.1 & March 19, 2024 \\
\end{longtable}

Typst GmbH did not create this template and cannot guarantee correct
functionality of this template or compatibility with any version of the
Typst compiler or app.


\title{typst.app/universe/package/lovelace}

\phantomsection\label{banner}
\section{lovelace}\label{lovelace}

{ 0.3.0 }

Algorithms in pseudocode, unopinionated and flexible

{ } Featured Package

\phantomsection\label{readme}
This is a package for writing pseudocode in
\href{https://typst.app/}{Typst} . It is named after the computer
science pioneer \href{https://en.wikipedia.org/wiki/Ada_Lovelace}{Ada
Lovelace} and inspired by the \href{https://ctan.org/pkg/pseudo}{pseudo
package} for LaTeX.

\pandocbounded{\includegraphics[keepaspectratio]{https://img.shields.io/github/license/andreasKroepelin/lovelace}}
\pandocbounded{\includegraphics[keepaspectratio]{https://img.shields.io/github/v/release/andreasKroepelin/lovelace}}
\pandocbounded{\includegraphics[keepaspectratio]{https://img.shields.io/github/stars/andreasKroepelin/lovelace}}

Pseudocode is not a programming language, it doesn’t have strict
syntax, so you should be able to write it however you need to in your
specific situation. Lovelace lets you do exactly that.

Main features include:

\begin{itemize}
\tightlist
\item
  arbitrary keywords and syntax structures
\item
  optional line numbering
\item
  line labels
\item
  lots of customisation with sensible defaults
\end{itemize}

\subsection{Usage}\label{usage}

\begin{itemize}
\tightlist
\item
  \href{https://github.com/typst/packages/raw/main/packages/preview/lovelace/0.3.0/\#getting-started}{Getting
  started}
\item
  \href{https://github.com/typst/packages/raw/main/packages/preview/lovelace/0.3.0/\#lower-level-interface}{Lower
  level interface}
\item
  \href{https://github.com/typst/packages/raw/main/packages/preview/lovelace/0.3.0/\#line-numbers}{Line
  numbers}
\item
  \href{https://github.com/typst/packages/raw/main/packages/preview/lovelace/0.3.0/\#referencing-lines}{Referencing
  lines}
\item
  \href{https://github.com/typst/packages/raw/main/packages/preview/lovelace/0.3.0/\#indentation-guides}{Indentation
  guides}
\item
  \href{https://github.com/typst/packages/raw/main/packages/preview/lovelace/0.3.0/\#spacing}{Spacing}
\item
  \href{https://github.com/typst/packages/raw/main/packages/preview/lovelace/0.3.0/\#decorations}{Decorations}
\item
  \href{https://github.com/typst/packages/raw/main/packages/preview/lovelace/0.3.0/\#algorithm-as-figure}{Algorithm
  as figure}
\item
  \href{https://github.com/typst/packages/raw/main/packages/preview/lovelace/0.3.0/\#customisation-overview}{Customisation
  overview}
\item
  \href{https://github.com/typst/packages/raw/main/packages/preview/lovelace/0.3.0/\#exported-functions}{Exported
  functions}
\end{itemize}

\subsubsection{Getting started}\label{getting-started}

Import the package using

\begin{Shaded}
\begin{Highlighting}[]
\NormalTok{\#import "@preview/lovelace:0.3.0": *}
\end{Highlighting}
\end{Shaded}

The simplest usage is via \texttt{\ pseudocode-list\ } which transforms
a nested list into pseudocode:

\begin{Shaded}
\begin{Highlighting}[]
\NormalTok{\#pseudocode{-}list[}
\NormalTok{  + do something}
\NormalTok{  + do something else}
\NormalTok{  + *while* still something to do}
\NormalTok{    + do even more}
\NormalTok{    + *if* not done yet *then*}
\NormalTok{      + wait a bit}
\NormalTok{      + resume working}
\NormalTok{    + *else*}
\NormalTok{      + go home}
\NormalTok{    + *end*}
\NormalTok{  + *end*}
\NormalTok{]}
\end{Highlighting}
\end{Shaded}

resulting in:

\pandocbounded{\includesvg[keepaspectratio]{https://github.com/typst/packages/raw/main/packages/preview/lovelace/0.3.0/examples/simple.svg}}

As you can see, every list item becomes one line of code and nested
lists become indented blocks. There are no special commands for common
keywords and control structures, you just use whatever you like.

Maybe in your domain very uncommon structures make more sense? No
problem!

\begin{Shaded}
\begin{Highlighting}[]
\NormalTok{\#pseudocode{-}list[}
\NormalTok{  + *in parallel for each* $i = 1, ..., n$ *do*}
\NormalTok{    + fetch chunk of data $i$}
\NormalTok{    + *with probability* $exp({-}epsilon\_i slash k T)$ *do*}
\NormalTok{      + perform update}
\NormalTok{    + *end*}
\NormalTok{  + *end*}
\NormalTok{]}
\end{Highlighting}
\end{Shaded}

\pandocbounded{\includesvg[keepaspectratio]{https://github.com/typst/packages/raw/main/packages/preview/lovelace/0.3.0/examples/custom.svg}}

\subsubsection{Lower level interface}\label{lower-level-interface}

If you feel uncomfortable with abusing Typst’s lists like we do here,
you can also use the \texttt{\ pseudocode\ } function directly:

\begin{Shaded}
\begin{Highlighting}[]
\NormalTok{\#pseudocode(}
\NormalTok{  [do something],}
\NormalTok{  [do something else],}
\NormalTok{  [*while* still something to do],}
\NormalTok{  indent(}
\NormalTok{    [do even more],}
\NormalTok{    [*if* not done yet *then*],}
\NormalTok{    indent(}
\NormalTok{      [wait a bit],}
\NormalTok{      [resume working],}
\NormalTok{    ),}
\NormalTok{    [*else*],}
\NormalTok{    indent(}
\NormalTok{      [go home],}
\NormalTok{    ),}
\NormalTok{    [*end*],}
\NormalTok{  ),}
\NormalTok{  [*end*],}
\NormalTok{)}
\end{Highlighting}
\end{Shaded}

This is equivalent to the first example. Note that each line is given as
one content argument and you indent a block by using the
\texttt{\ indent\ } function.

This approach has the advantage that you do not rely on significant
whitespace and code formatters can automatically correctly indent your
Typst code.

\subsubsection{Line numbers}\label{line-numbers}

Lovelace puts a number in front of each line by default. If you want no
numbers at all, you can set the \texttt{\ line-numbering\ } option to
\texttt{\ none\ } . The initial example then looks like this:

\begin{Shaded}
\begin{Highlighting}[]
\NormalTok{\#pseudocode{-}list(line{-}numbering: none)[}
\NormalTok{  + do something}
\NormalTok{  + do something else}
\NormalTok{  + *while* still something to do}
\NormalTok{    + do even more}
\NormalTok{    + *if* not done yet *then*}
\NormalTok{      + wait a bit}
\NormalTok{      + resume working}
\NormalTok{    + *else*}
\NormalTok{      + go home}
\NormalTok{    + *end*}
\NormalTok{  + *end*}
\NormalTok{]}
\end{Highlighting}
\end{Shaded}

\pandocbounded{\includesvg[keepaspectratio]{https://github.com/typst/packages/raw/main/packages/preview/lovelace/0.3.0/examples/simple-no-numbers.svg}}

(You can also pass this keyword argument to \texttt{\ pseudocode\ } .)

If you do want line numbers in general but need to turn them off for
specific lines, you can use \texttt{\ -\ } items instead of
\texttt{\ +\ } items in \texttt{\ pseudocode-list\ } :

\begin{Shaded}
\begin{Highlighting}[]
\NormalTok{\#pseudocode{-}list[}
\NormalTok{  + normal line with a number}
\NormalTok{  {-} this line has no number}
\NormalTok{  + this one has a number again}
\NormalTok{]}
\end{Highlighting}
\end{Shaded}

\pandocbounded{\includesvg[keepaspectratio]{https://github.com/typst/packages/raw/main/packages/preview/lovelace/0.3.0/examples/number-no-number.svg}}

It’s easy to remember: \texttt{\ -\ } items usually produce unnumbered
lists and \texttt{\ +\ } items produce numbered lists!

When using the \texttt{\ pseudocode\ } function, you can achieve the
same using \texttt{\ no-number\ } :

\begin{Shaded}
\begin{Highlighting}[]
\NormalTok{\#pseudocode(}
\NormalTok{  [normal line with a number],}
\NormalTok{  no{-}number[this line has no number],}
\NormalTok{  [this one has a number again],}
\NormalTok{)}
\end{Highlighting}
\end{Shaded}

\paragraph{More line number
customisation}\label{more-line-number-customisation}

Other than \texttt{\ none\ } , you can assign anything listed
\href{https://typst.app/docs/reference/model/numbering/\#parameters-numbering}{here}
to \texttt{\ line-numbering\ } .

So maybe you happen to think about the Roman Empire a lot and want to
reflect that in your pseudocode?

\begin{Shaded}
\begin{Highlighting}[]
\NormalTok{\#set text(font: "Cinzel")}

\NormalTok{\#pseudocode{-}list(line{-}numbering: "I:")[}
\NormalTok{  + explore European tribes}
\NormalTok{  + *while* not all tribes conquered}
\NormalTok{    + *for each* tribe *in* unconquered tribes}
\NormalTok{      + try to conquer tribe}
\NormalTok{    + *end*}
\NormalTok{  + *end*}
\NormalTok{]}
\end{Highlighting}
\end{Shaded}

\pandocbounded{\includesvg[keepaspectratio]{https://github.com/typst/packages/raw/main/packages/preview/lovelace/0.3.0/examples/roman.svg}}

\subsubsection{Referencing lines}\label{referencing-lines}

You can reference an inividual line of a pseudocode by giving it a
label. Inside \texttt{\ pseudocode-list\ } , you can use
\texttt{\ line-label\ } :

\begin{Shaded}
\begin{Highlighting}[]
\NormalTok{\#pseudocode{-}list[}
\NormalTok{  + \#line{-}label(\textless{}start\textgreater{}) do something}
\NormalTok{  + \#line{-}label(\textless{}important\textgreater{}) do something important}
\NormalTok{  + go back to @start}
\NormalTok{]}

\NormalTok{The relevance of the step in @important cannot be overstated.}
\end{Highlighting}
\end{Shaded}

\pandocbounded{\includesvg[keepaspectratio]{https://github.com/typst/packages/raw/main/packages/preview/lovelace/0.3.0/examples/label.svg}}

When using \texttt{\ pseudocode\ } , you can use
\texttt{\ with-line-label\ } :

\begin{Shaded}
\begin{Highlighting}[]
\NormalTok{\#pseudocode(}
\NormalTok{  with{-}line{-}label(\textless{}start\textgreater{})[do something],}
\NormalTok{  with{-}line{-}label(\textless{}important\textgreater{})[do something important],}
\NormalTok{  [go back to @start],}
\NormalTok{)}

\NormalTok{The relevance of the step in @important cannot be overstated.}
\end{Highlighting}
\end{Shaded}

This has the same effect as the previous example.

The number shown in the reference uses the numbering scheme defined in
the \texttt{\ line-numbering\ } option (see previous section).

By default, \texttt{\ "Line"\ } is used as the supplement for
referencing lines. You can change that using the
\texttt{\ line-number-supplement\ } option to \texttt{\ pseudocode\ } or
\texttt{\ pseudocode-list\ } .

\subsubsection{Indentation guides}\label{indentation-guides}

By default, Lovelace puts a thin gray ( \texttt{\ gray\ +\ 1pt\ } ) line
to the left of each indented block, which guides the reader in
understanding the indentations, just like a code editor would. You can
customise this using the \texttt{\ stroke\ } option which takes any
value that is a valid
\href{https://typst.app/docs/reference/visualize/stroke/}{Typst stroke}
. You can especially set it to \texttt{\ none\ } to have no indentation
guides.

The example from the beginning becomes:

\begin{Shaded}
\begin{Highlighting}[]
\NormalTok{\#pseudocode{-}list(stroke: none)[}
\NormalTok{  + do something}
\NormalTok{  + do something else}
\NormalTok{  + *while* still something to do}
\NormalTok{    + do even more}
\NormalTok{    + *if* not done yet *then*}
\NormalTok{      + wait a bit}
\NormalTok{      + resume working}
\NormalTok{    + *else*}
\NormalTok{      + go home}
\NormalTok{    + *end*}
\NormalTok{  + *end*}
\NormalTok{]}
\end{Highlighting}
\end{Shaded}

\pandocbounded{\includesvg[keepaspectratio]{https://github.com/typst/packages/raw/main/packages/preview/lovelace/0.3.0/examples/simple-no-stroke.svg}}

\paragraph{End blocks with hooks}\label{end-blocks-with-hooks}

Some people prefer using the indentation guide to signal the end of a
block instead of writing something like “ \textbf{end} � by having a
small “hook� at the end. To achieve that in Lovelace, you can make
use of the \texttt{\ hook\ } option and specify how far a line should
extend to the right from the indentation guide:

\begin{Shaded}
\begin{Highlighting}[]
\NormalTok{\#pseudocode{-}list(hooks: .5em)[}
\NormalTok{  + do something}
\NormalTok{  + do something else}
\NormalTok{  + *while* still something to do}
\NormalTok{    + do even more}
\NormalTok{    + *if* not done yet *then*}
\NormalTok{      + wait a bit}
\NormalTok{      + resume working}
\NormalTok{    + *else*}
\NormalTok{      + go home}
\NormalTok{]}
\end{Highlighting}
\end{Shaded}

\pandocbounded{\includesvg[keepaspectratio]{https://github.com/typst/packages/raw/main/packages/preview/lovelace/0.3.0/examples/hooks.svg}}

\subsubsection{Spacing}\label{spacing}

You can control how far indented lines are shifted right by the
\texttt{\ indentation\ } option. To change the space between lines, use
the \texttt{\ line-gap\ } option.

\begin{Shaded}
\begin{Highlighting}[]
\NormalTok{\#pseudocode{-}list(indentation: 3em, line{-}gap: 1.5em)[}
\NormalTok{  + do something}
\NormalTok{  + do something else}
\NormalTok{  + *while* still something to do}
\NormalTok{    + do even more}
\NormalTok{    + *if* not done yet *then*}
\NormalTok{      + wait a bit}
\NormalTok{      + resume working}
\NormalTok{    + *else*}
\NormalTok{      + go home}
\NormalTok{    + *end*}
\NormalTok{  + *end*}
\NormalTok{]}
\end{Highlighting}
\end{Shaded}

\pandocbounded{\includesvg[keepaspectratio]{https://github.com/typst/packages/raw/main/packages/preview/lovelace/0.3.0/examples/spacing.svg}}

\subsubsection{Decorations}\label{decorations}

You can also add a title and/or a frame around your algorithm if you
like:

\paragraph{Title}\label{title}

Using the \texttt{\ title\ } option, you can give your pseudocode a
title (surprise!). For example, to achieve
\href{https://en.wikipedia.org/wiki/Introduction_to_Algorithms}{CLRS
style} , you can do something like

\begin{Shaded}
\begin{Highlighting}[]
\NormalTok{\#pseudocode{-}list(stroke: none, title: smallcaps[Fancy{-}Algorithm])[}
\NormalTok{  + do something}
\NormalTok{  + do something else}
\NormalTok{  + *while* still something to do}
\NormalTok{    + do even more}
\NormalTok{    + *if* not done yet *then*}
\NormalTok{      + wait a bit}
\NormalTok{      + resume working}
\NormalTok{    + *else*}
\NormalTok{      + go home}
\NormalTok{    + *end*}
\NormalTok{  + *end*}
\NormalTok{]}
\end{Highlighting}
\end{Shaded}

\pandocbounded{\includesvg[keepaspectratio]{https://github.com/typst/packages/raw/main/packages/preview/lovelace/0.3.0/examples/title.svg}}

\paragraph{Booktabs}\label{booktabs}

If you like wrapping your algorithm in elegant horizontal lines, you can
do so by setting the \texttt{\ booktabs\ } option to \texttt{\ true\ } .

\begin{Shaded}
\begin{Highlighting}[]
\NormalTok{\#pseudocode{-}list(booktabs: true)[}
\NormalTok{  + do something}
\NormalTok{  + do something else}
\NormalTok{  + *while* still something to do}
\NormalTok{    + do even more}
\NormalTok{    + *if* not done yet *then*}
\NormalTok{      + wait a bit}
\NormalTok{      + resume working}
\NormalTok{    + *else*}
\NormalTok{      + go home}
\NormalTok{    + *end*}
\NormalTok{  + *end*}
\NormalTok{]}
\end{Highlighting}
\end{Shaded}

\pandocbounded{\includesvg[keepaspectratio]{https://github.com/typst/packages/raw/main/packages/preview/lovelace/0.3.0/examples/booktabs.svg}}

Together with the \texttt{\ title\ } option, you can produce

\begin{Shaded}
\begin{Highlighting}[]
\NormalTok{\#pseudocode{-}list(booktabs: true, title: [My cool title])[}
\NormalTok{  + do something}
\NormalTok{  + do something else}
\NormalTok{  + *while* still something to do}
\NormalTok{    + do even more}
\NormalTok{    + *if* not done yet *then*}
\NormalTok{      + wait a bit}
\NormalTok{      + resume working}
\NormalTok{    + *else*}
\NormalTok{      + go home}
\NormalTok{    + *end*}
\NormalTok{  + *end*}
\NormalTok{]}
\end{Highlighting}
\end{Shaded}

\pandocbounded{\includesvg[keepaspectratio]{https://github.com/typst/packages/raw/main/packages/preview/lovelace/0.3.0/examples/booktabs-title.svg}}

By default, the outer booktab strokes are \texttt{\ black\ +\ 2pt\ } .
You can change that with the option \texttt{\ booktabs-stroke\ } to any
valid \href{https://typst.app/docs/reference/visualize/stroke/}{Typst
stroke} . The inner line will always have the same stroke as the outer
ones, just with half the thickness.

\subsubsection{Algorithm as figure}\label{algorithm-as-figure}

To make algorithms referencable and being able to float in the document,
you can use Typst’s \texttt{\ figure\ } function with a custom
\texttt{\ kind\ } .

\begin{Shaded}
\begin{Highlighting}[]
\NormalTok{\#figure(}
\NormalTok{  kind: "algorithm",}
\NormalTok{  supplement: [Algorithm],}
\NormalTok{  caption: [My cool algorithm],}

\NormalTok{  pseudocode{-}list[}
\NormalTok{    + do something}
\NormalTok{    + do something else}
\NormalTok{    + *while* still something to do}
\NormalTok{      + do even more}
\NormalTok{      + *if* not done yet *then*}
\NormalTok{        + wait a bit}
\NormalTok{        + resume working}
\NormalTok{      + *else*}
\NormalTok{        + go home}
\NormalTok{      + *end*}
\NormalTok{    + *end*}
\NormalTok{  ]}
\NormalTok{)}
\end{Highlighting}
\end{Shaded}

\pandocbounded{\includesvg[keepaspectratio]{https://github.com/typst/packages/raw/main/packages/preview/lovelace/0.3.0/examples/figure.svg}}

If you want to have the algorithm counter inside the title instead (see
previous section), there is the option \texttt{\ numbered-title\ } :

\begin{Shaded}
\begin{Highlighting}[]
\NormalTok{\#figure(}
\NormalTok{  kind: "algorithm",}
\NormalTok{  supplement: [Algorithm],}

\NormalTok{  pseudocode{-}list(booktabs: true, numbered{-}title: [My cool algorithm])[}
\NormalTok{    + do something}
\NormalTok{    + do something else}
\NormalTok{    + *while* still something to do}
\NormalTok{      + do even more}
\NormalTok{      + *if* not done yet *then*}
\NormalTok{        + wait a bit}
\NormalTok{        + resume working}
\NormalTok{      + *else*}
\NormalTok{        + go home}
\NormalTok{      + *end*}
\NormalTok{    + *end*}
\NormalTok{  ]}
\NormalTok{) \textless{}cool\textgreater{}}

\NormalTok{See @cool for details on how to do something cool.}
\end{Highlighting}
\end{Shaded}

\pandocbounded{\includesvg[keepaspectratio]{https://github.com/typst/packages/raw/main/packages/preview/lovelace/0.3.0/examples/figure-title.svg}}

Note that the \texttt{\ numbered-title\ } option only makes sense when
nesting your pseudocode inside a figure with
\texttt{\ kind:\ "algorithm"\ } , otherwise it produces undefined
results.

\subsubsection{Customisation overview}\label{customisation-overview}

Both \texttt{\ pseudocode\ } and \texttt{\ pseudocode-list\ } accept the
following configuration arguments:

\begin{longtable}[]{@{}lll@{}}
\toprule\noalign{}
\textbf{option} & \textbf{type} & \textbf{default} \\
\midrule\noalign{}
\endhead
\bottomrule\noalign{}
\endlastfoot
\href{https://github.com/typst/packages/raw/main/packages/preview/lovelace/0.3.0/\#line-numbers}{\texttt{\ line-numbering\ }}
& \texttt{\ none\ } or a
\href{https://typst.app/docs/reference/model/numbering/\#parameters-numbering}{numbering}
& \texttt{\ "1"\ } \\
\href{https://github.com/typst/packages/raw/main/packages/preview/lovelace/0.3.0/\#more-line-number-customisation}{\texttt{\ line-number-supplement\ }}
& content & \texttt{\ "Line"\ } \\
\href{https://github.com/typst/packages/raw/main/packages/preview/lovelace/0.3.0/\#indentation-guides}{\texttt{\ stroke\ }}
& stroke & \texttt{\ 1pt\ +\ gray\ } \\
\href{https://github.com/typst/packages/raw/main/packages/preview/lovelace/0.3.0/\#end-blocks-with-hooks}{\texttt{\ hooks\ }}
& length & \texttt{\ 0pt\ } \\
\href{https://github.com/typst/packages/raw/main/packages/preview/lovelace/0.3.0/\#spacing}{\texttt{\ indentation\ }}
& length & \texttt{\ 1em\ } \\
\href{https://github.com/typst/packages/raw/main/packages/preview/lovelace/0.3.0/\#spacing}{\texttt{\ line-gap\ }}
& length & \texttt{\ .8em\ } \\
\href{https://github.com/typst/packages/raw/main/packages/preview/lovelace/0.3.0/\#booktabs}{\texttt{\ booktabs\ }}
& bool & \texttt{\ false\ } \\
\href{https://github.com/typst/packages/raw/main/packages/preview/lovelace/0.3.0/\#booktabs}{\texttt{\ booktabs-stroke\ }}
& stroke & \texttt{\ 2pt\ +\ black\ } \\
\href{https://github.com/typst/packages/raw/main/packages/preview/lovelace/0.3.0/\#title}{\texttt{\ title\ }}
& content or \texttt{\ none\ } & \texttt{\ none\ } \\
\href{https://github.com/typst/packages/raw/main/packages/preview/lovelace/0.3.0/\#algorithm-as-figure}{\texttt{\ numbered-title\ }}
& content or \texttt{\ none\ } & \texttt{\ none\ } \\
\end{longtable}

Until Typst supports user defined types, we can use the following trick
when wanting to set own default values for these options. Say, you
always want your algorithms to have colons after the line numbers, no
indentation guides and, if present, blue booktabs. In this case, you
would put the following at the top of your document:

\begin{Shaded}
\begin{Highlighting}[]
\NormalTok{\#let my{-}lovelace{-}defaults = (}
\NormalTok{  line{-}numbering: "1:",}
\NormalTok{  stroke: none,}
\NormalTok{  booktabs{-}stroke: 2pt + blue,}
\NormalTok{)}

\NormalTok{\#let pseudocode = pseudocode.with(..my{-}lovelace{-}defaults)}
\NormalTok{\#let pseudocode{-}list = pseudocode{-}list.with(..my{-}lovelace{-}defaults)}
\end{Highlighting}
\end{Shaded}

\subsubsection{Exported functions}\label{exported-functions}

Lovelace exports the following functions:

\begin{itemize}
\tightlist
\item
  \texttt{\ pseudocode\ } : Typeset pseudocode with each line as an
  individual content argument, see
  \href{https://github.com/typst/packages/raw/main/packages/preview/lovelace/0.3.0/\#lower-level-interface}{here}
  for details. Has
  \href{https://github.com/typst/packages/raw/main/packages/preview/lovelace/0.3.0/\#customisation-overview}{these}
  optional arguments.
\item
  \texttt{\ pseudocode-list\ } : Takes a standard Typst list and
  transforms it into a pseudocode. Has
  \href{https://github.com/typst/packages/raw/main/packages/preview/lovelace/0.3.0/\#customisation-overview}{these}
  optional arguments.
\item
  \texttt{\ indent\ } : Inside the argument list of
  \texttt{\ pseudocode\ } , use \texttt{\ indent\ } to specify an
  indented block, see
  \href{https://github.com/typst/packages/raw/main/packages/preview/lovelace/0.3.0/\#lower-level-interface}{here}
  for details.
\item
  \texttt{\ no-number\ } : Wrap an argument to \texttt{\ pseudocode\ }
  in this function to have the corresponding line be unnumbered, see
  \href{https://github.com/typst/packages/raw/main/packages/preview/lovelace/0.3.0/\#line-numbers}{here}
  for details.
\item
  \texttt{\ with-line-label\ } : Use this function in the
  \texttt{\ pseudocode\ } arguments to add a label to a specific line,
  see
  \href{https://github.com/typst/packages/raw/main/packages/preview/lovelace/0.3.0/\#referencing-lines}{here}
  for details.
\item
  \texttt{\ line-label\ } : When using \texttt{\ pseudocode-list\ } ,
  you do \emph{not} use \texttt{\ with-line-label\ } but insert a call
  to \texttt{\ line-label\ } somewhere in a line to add a label, see
  \href{https://github.com/typst/packages/raw/main/packages/preview/lovelace/0.3.0/\#referencing-lines}{here}
  for details.
\end{itemize}

\subsubsection{How to add}\label{how-to-add}

Copy this into your project and use the import as \texttt{\ lovelace\ }

\begin{verbatim}
#import "@preview/lovelace:0.3.0"
\end{verbatim}

\includesvg[width=0.16667in,height=0.16667in]{/assets/icons/16-copy.svg}

Check the docs for
\href{https://typst.app/docs/reference/scripting/\#packages}{more
information on how to import packages} .

\subsubsection{About}\label{about}

\begin{description}
\tightlist
\item[Author s :]
Andreas Kröpelin \& Lovelace contributors
\item[License:]
MIT
\item[Current version:]
0.3.0
\item[Last updated:]
July 1, 2024
\item[First released:]
September 6, 2023
\item[Archive size:]
3.44 kB
\href{https://packages.typst.org/preview/lovelace-0.3.0.tar.gz}{\pandocbounded{\includesvg[keepaspectratio]{/assets/icons/16-download.svg}}}
\item[Repository:]
\href{https://github.com/andreasKroepelin/lovelace}{GitHub}
\end{description}

\subsubsection{Where to report issues?}\label{where-to-report-issues}

This package is a project of Andreas Kröpelin and Lovelace contributors
. Report issues on
\href{https://github.com/andreasKroepelin/lovelace}{their repository} .
You can also try to ask for help with this package on the
\href{https://forum.typst.app}{Forum} .

Please report this package to the Typst team using the
\href{https://typst.app/contact}{contact form} if you believe it is a
safety hazard or infringes upon your rights.

\phantomsection\label{versions}
\subsubsection{Version history}\label{version-history}

\begin{longtable}[]{@{}ll@{}}
\toprule\noalign{}
Version & Release Date \\
\midrule\noalign{}
\endhead
\bottomrule\noalign{}
\endlastfoot
0.3.0 & July 1, 2024 \\
\href{https://typst.app/universe/package/lovelace/0.2.0/}{0.2.0} &
January 9, 2024 \\
\href{https://typst.app/universe/package/lovelace/0.1.0/}{0.1.0} &
September 6, 2023 \\
\end{longtable}

Typst GmbH did not create this package and cannot guarantee correct
functionality of this package or compatibility with any version of the
Typst compiler or app.


\title{typst.app/universe/package/ansi-render}

\phantomsection\label{banner}
\section{ansi-render}\label{ansi-render}

{ 0.6.1 }

provides a simple way to render text with ANSI escape sequences.

\phantomsection\label{readme}
\href{https://github.com/8LWXpg/typst-ansi-render/tags}{\pandocbounded{\includegraphics[keepaspectratio]{https://img.shields.io/github/v/tag/8LWXpg/typst-ansi-render}}}
\href{https://github.com/8LWXpg/typst-ansi-render}{\pandocbounded{\includegraphics[keepaspectratio]{https://img.shields.io/github/stars/8LWXpg/typst-ansi-render}}}
\href{https://github.com/8LWXpg/typst-ansi-render/blob/master/LICENSE}{\pandocbounded{\includegraphics[keepaspectratio]{https://img.shields.io/github/license/8LWXpg/typst-ansi-render}}}
\href{https://github.com/typst/packages/tree/main/packages/preview/ansi-render}{\pandocbounded{\includegraphics[keepaspectratio]{https://img.shields.io/badge/typst-package-239dad}}}

This script provides a simple way to render text with ANSI escape
sequences. Package \texttt{\ ansi-render\ } provides a function
\texttt{\ ansi-render\ } , and a dictionary of themes
\texttt{\ terminal-themes\ } .

contribution is welcomed!

\subsection{Usage}\label{usage}

\begin{Shaded}
\begin{Highlighting}[]
\NormalTok{\#import "@preview/ansi{-}render:0.6.1": *}

\NormalTok{\#ansi{-}render(}
\NormalTok{  string,}
\NormalTok{  font:           string or none,}
\NormalTok{  size:           length,}
\NormalTok{  width:          auto or relative length,}
\NormalTok{  height:         auto or relative length,}
\NormalTok{  breakable:      boolean,}
\NormalTok{  radius:         relative length or dictionary,}
\NormalTok{  inset:          relative length or dictionary,}
\NormalTok{  outset:         relative length or dictionary,}
\NormalTok{  spacing:        relative length or fraction,}
\NormalTok{  above:          relative length or fraction,}
\NormalTok{  below:          relative length or fraction,}
\NormalTok{  clip:           boolean,}
\NormalTok{  bold{-}is{-}bright: boolean,}
\NormalTok{  theme:          terminal{-}themes.theme,}
\NormalTok{)}
\end{Highlighting}
\end{Shaded}

\subsubsection{Parameters}\label{parameters}

\begin{itemize}
\tightlist
\item
  \texttt{\ string\ } - string with ANSI escape sequences
\item
  \texttt{\ font\ } - font name or none, default is
  \texttt{\ Cascadia\ Code\ } , set to \texttt{\ none\ } to use the same
  font as \texttt{\ raw\ }
\item
  \texttt{\ size\ } - font size, default is \texttt{\ 1em\ }
\item
  \texttt{\ bold-is-bright\ } - boolean, whether bold text is rendered
  with bright colors, default is \texttt{\ false\ }
\item
  \texttt{\ theme\ } - theme, default is \texttt{\ vscode-light\ }
\item
  parameters from
  \href{https://typst.app/docs/reference/layout/block/}{\texttt{\ block\ }}
  function with the same default value, only affects outmost block
  layout:

  \begin{itemize}
  \tightlist
  \item
    \texttt{\ width\ }
  \item
    \texttt{\ height\ }
  \item
    \texttt{\ breakable\ }
  \item
    \texttt{\ radius\ }
  \item
    \texttt{\ inset\ }
  \item
    \texttt{\ outset\ }
  \item
    \texttt{\ spacing\ }
  \item
    \texttt{\ above\ }
  \item
    \texttt{\ below\ }
  \item
    \texttt{\ clip\ }
  \end{itemize}
\end{itemize}

\subsection{Themes}\label{themes}

see
\href{https://github.com/8LWXpg/typst-ansi-render/blob/master/test/themes.pdf}{themes}

\subsection{Demo}\label{demo}

see
\href{https://github.com/8LWXpg/typst-ansi-render/blob/master/test/demo.typ}{demo.typ}
\href{https://github.com/8LWXpg/typst-ansi-render/blob/master/test/demo.pdf}{demo.pdf}

\begin{Shaded}
\begin{Highlighting}[]
\NormalTok{\#ansi{-}render(}
\NormalTok{"\textbackslash{}u\{1b\}[38;2;255;0;0mThis text is red.\textbackslash{}u\{1b\}[0m}
\NormalTok{\textbackslash{}u\{1b\}[48;2;0;255;0mThis background is green.\textbackslash{}u\{1b\}[0m}
\NormalTok{\textbackslash{}u\{1b\}[38;2;255;255;255m\textbackslash{}u\{1b\}[48;2;0;0;255mThis text is white on a blue background.\textbackslash{}u\{1b\}[0m}
\NormalTok{\textbackslash{}u\{1b\}[1mThis text is bold.\textbackslash{}u\{1b\}[0m}
\NormalTok{\textbackslash{}u\{1b\}[4mThis text is underlined.\textbackslash{}u\{1b\}[0m}
\NormalTok{\textbackslash{}u\{1b\}[38;2;255;165;0m\textbackslash{}u\{1b\}[48;2;255;255;0mThis text is orange on a yellow background.\textbackslash{}u\{1b\}[0m",}
\NormalTok{inset: 5pt, radius: 3pt,}
\NormalTok{theme: terminal{-}themes.vscode}
\NormalTok{)}
\end{Highlighting}
\end{Shaded}

\pandocbounded{\includegraphics[keepaspectratio]{https://raw.githubusercontent.com/8LWXpg/typst-ansi-render/master/img/1.png}}

\begin{Shaded}
\begin{Highlighting}[]
\NormalTok{\#ansi{-}render(}
\NormalTok{"\textbackslash{}u\{1b\}[38;5;196mRed text\textbackslash{}u\{1b\}[0m}
\NormalTok{\textbackslash{}u\{1b\}[48;5;27mBlue background\textbackslash{}u\{1b\}[0m}
\NormalTok{\textbackslash{}u\{1b\}[38;5;226;48;5;18mYellow text on blue background\textbackslash{}u\{1b\}[0m}
\NormalTok{\textbackslash{}u\{1b\}[7mInverted text\textbackslash{}u\{1b\}[0m}
\NormalTok{\textbackslash{}u\{1b\}[38;5;208;48;5;237mOrange text on gray background\textbackslash{}u\{1b\}[0m}
\NormalTok{\textbackslash{}u\{1b\}[38;5;39;48;5;208mBlue text on orange background\textbackslash{}u\{1b\}[0m}
\NormalTok{\textbackslash{}u\{1b\}[38;5;255;48;5;0mWhite text on black background\textbackslash{}u\{1b\}[0m",}
\NormalTok{inset: 5pt, radius: 3pt,}
\NormalTok{theme: terminal{-}themes.vscode}
\NormalTok{)}
\end{Highlighting}
\end{Shaded}

\pandocbounded{\includegraphics[keepaspectratio]{https://raw.githubusercontent.com/8LWXpg/typst-ansi-render/master/img/2.png}}

\begin{Shaded}
\begin{Highlighting}[]
\NormalTok{\#ansi{-}render(}
\NormalTok{"\textbackslash{}u\{1b\}[31;1mHello \textbackslash{}u\{1b\}[7mWorld\textbackslash{}u\{1b\}[0m}

\NormalTok{\textbackslash{}u\{1b\}[53;4;36mOver  and \textbackslash{}u\{1b\}[35m Under!}
\NormalTok{\textbackslash{}u\{1b\}[7;90mreverse\textbackslash{}u\{1b\}[101m and \textbackslash{}u\{1b\}[94;27mreverse",}
\NormalTok{inset: 5pt, radius: 3pt,}
\NormalTok{theme: terminal{-}themes.vscode}
\NormalTok{)}
\end{Highlighting}
\end{Shaded}

\pandocbounded{\includegraphics[keepaspectratio]{https://raw.githubusercontent.com/8LWXpg/typst-ansi-render/master/img/3.png}}

\begin{Shaded}
\begin{Highlighting}[]
\NormalTok{// uses the font that supports ligatures}
\NormalTok{\#ansi{-}render(read("test.txt"), inset: 5pt, radius: 3pt, font: "Cascadia Code", theme: terminal{-}themes.putty)}
\end{Highlighting}
\end{Shaded}

\pandocbounded{\includegraphics[keepaspectratio]{https://raw.githubusercontent.com/8LWXpg/typst-ansi-render/master/img/4.png}}

\subsubsection{How to add}\label{how-to-add}

Copy this into your project and use the import as
\texttt{\ ansi-render\ }

\begin{verbatim}
#import "@preview/ansi-render:0.6.1"
\end{verbatim}

\includesvg[width=0.16667in,height=0.16667in]{/assets/icons/16-copy.svg}

Check the docs for
\href{https://typst.app/docs/reference/scripting/\#packages}{more
information on how to import packages} .

\subsubsection{About}\label{about}

\begin{description}
\tightlist
\item[Author :]
8LWXpg
\item[License:]
MIT
\item[Current version:]
0.6.1
\item[Last updated:]
December 28, 2023
\item[First released:]
July 3, 2023
\item[Minimum Typst version:]
0.10.0
\item[Archive size:]
6.23 kB
\href{https://packages.typst.org/preview/ansi-render-0.6.1.tar.gz}{\pandocbounded{\includesvg[keepaspectratio]{/assets/icons/16-download.svg}}}
\item[Repository:]
\href{https://github.com/8LWXpg/typst-ansi-render}{GitHub}
\end{description}

\subsubsection{Where to report issues?}\label{where-to-report-issues}

This package is a project of 8LWXpg . Report issues on
\href{https://github.com/8LWXpg/typst-ansi-render}{their repository} .
You can also try to ask for help with this package on the
\href{https://forum.typst.app}{Forum} .

Please report this package to the Typst team using the
\href{https://typst.app/contact}{contact form} if you believe it is a
safety hazard or infringes upon your rights.

\phantomsection\label{versions}
\subsubsection{Version history}\label{version-history}

\begin{longtable}[]{@{}ll@{}}
\toprule\noalign{}
Version & Release Date \\
\midrule\noalign{}
\endhead
\bottomrule\noalign{}
\endlastfoot
0.6.1 & December 28, 2023 \\
\href{https://typst.app/universe/package/ansi-render/0.6.0/}{0.6.0} &
December 10, 2023 \\
\href{https://typst.app/universe/package/ansi-render/0.5.1/}{0.5.1} &
October 21, 2023 \\
\href{https://typst.app/universe/package/ansi-render/0.5.0/}{0.5.0} &
September 29, 2023 \\
\href{https://typst.app/universe/package/ansi-render/0.4.2/}{0.4.2} &
September 25, 2023 \\
\href{https://typst.app/universe/package/ansi-render/0.4.1/}{0.4.1} &
September 22, 2023 \\
\href{https://typst.app/universe/package/ansi-render/0.4.0/}{0.4.0} &
September 13, 2023 \\
\href{https://typst.app/universe/package/ansi-render/0.3.0/}{0.3.0} &
September 9, 2023 \\
\href{https://typst.app/universe/package/ansi-render/0.2.0/}{0.2.0} &
August 5, 2023 \\
\href{https://typst.app/universe/package/ansi-render/0.1.0/}{0.1.0} &
July 3, 2023 \\
\end{longtable}

Typst GmbH did not create this package and cannot guarantee correct
functionality of this package or compatibility with any version of the
Typst compiler or app.


\title{typst.app/universe/package/splendid-mdpi}

\phantomsection\label{banner}
\phantomsection\label{template-thumbnail}
\pandocbounded{\includegraphics[keepaspectratio]{https://packages.typst.org/preview/thumbnails/splendid-mdpi-0.1.0-small.webp}}

\section{splendid-mdpi}\label{splendid-mdpi}

{ 0.1.0 }

An MDPI-style paper template to publish at conferences and journals

{ } Featured Template

\href{/app?template=splendid-mdpi&version=0.1.0}{Create project in app}

\phantomsection\label{readme}
Version 0.1.0

A recreation of the MDPI template shown on the typst.app homepage.

\subsection{Media}\label{media}

\includegraphics[width=0.45\linewidth,height=\textheight,keepaspectratio]{https://github.com/typst/packages/raw/main/packages/preview/splendid-mdpi/0.1.0/thumbnails/1.png}
\includegraphics[width=0.45\linewidth,height=\textheight,keepaspectratio]{https://github.com/typst/packages/raw/main/packages/preview/splendid-mdpi/0.1.0/thumbnails/2.png}

\subsection{Getting Started}\label{getting-started}

To use this template, simply import it as shown below:

\begin{Shaded}
\begin{Highlighting}[]
\NormalTok{\#import "@preview/splendid{-}mdpi:0.1.0"}

\NormalTok{\#show: splendid{-}mdpi.template.with(}
\NormalTok{  title: [Towards Swifter Interstellar Mail Delivery],}
\NormalTok{  authors: (}
\NormalTok{    (}
\NormalTok{      name: "Johanna Swift",}
\NormalTok{      department: "Primary Logistics Department",}
\NormalTok{      institution: "Delivery Institute",}
\NormalTok{      city: "Berlin",}
\NormalTok{      country: "Germany",}
\NormalTok{      mail: "swift@delivery.de",}
\NormalTok{    ),}
\NormalTok{    (}
\NormalTok{      name: "Egon Stellaris",}
\NormalTok{      department: "Communications Group",}
\NormalTok{      institution: "Space Institute",}
\NormalTok{      city: "Florence",}
\NormalTok{      country: "Italy",}
\NormalTok{      mail: "stegonaris@space.it",}
\NormalTok{    ),}
\NormalTok{    (}
\NormalTok{      name: "Oliver Liam",}
\NormalTok{      department: "Missing Letters Task Force",}
\NormalTok{      institution: "Mail Institute",}
\NormalTok{      city: "Budapest",}
\NormalTok{      country: "Hungary",}
\NormalTok{      mail: "oliver.liam@mail.hu",}
\NormalTok{    ),}
\NormalTok{  ),}
\NormalTok{  date: (}
\NormalTok{    year: 2022,}
\NormalTok{    month: "May",}
\NormalTok{    day: 17,}
\NormalTok{  ),}
\NormalTok{  keywords: (}
\NormalTok{    "Space",}
\NormalTok{    "Mail",}
\NormalTok{    "Astromail",}
\NormalTok{    "Faster{-}than{-}Light",}
\NormalTok{    "Mars",}
\NormalTok{  ),}
\NormalTok{  doi: "10:7891/120948510",}
\NormalTok{  abstract: [}
\NormalTok{    Recent advances in space{-}based document processing have enabled faster mail delivery between different planets of a solar system. Given the time it takes for a message to be transmitted from one planet to the next, its estimated that even a one{-}way trip to a distant destination could take up to one year. During these periods of interplanetary mail delivery there is a slight possibility of mail being lost in transit. This issue is considered so serious that space management employs P.I. agents to track down and retrieve lost mail. We propose A{-}Mail, a new anti{-}matter based approach that can ensure that mail loss occurring during interplanetary transit is unobservable and therefore potentially undetectable. Going even further, we extend A{-}Mail to predict problems and apply existing and new best practices to ensure the mail is delivered without any issues. We call this extension AI{-}Mail.}
\NormalTok{  ]}
\NormalTok{)}
\end{Highlighting}
\end{Shaded}

\href{/app?template=splendid-mdpi&version=0.1.0}{Create project in app}

\subsubsection{How to use}\label{how-to-use}

Click the button above to create a new project using this template in
the Typst app.

You can also use the Typst CLI to start a new project on your computer
using this command:

\begin{verbatim}
typst init @preview/splendid-mdpi:0.1.0
\end{verbatim}

\includesvg[width=0.16667in,height=0.16667in]{/assets/icons/16-copy.svg}

\subsubsection{About}\label{about}

\begin{description}
\tightlist
\item[Author :]
James R. Swift
\item[License:]
Unlicense
\item[Current version:]
0.1.0
\item[Last updated:]
July 16, 2024
\item[First released:]
July 16, 2024
\item[Archive size:]
34.2 kB
\href{https://packages.typst.org/preview/splendid-mdpi-0.1.0.tar.gz}{\pandocbounded{\includesvg[keepaspectratio]{/assets/icons/16-download.svg}}}
\item[Repository:]
\href{https://github.com/JamesxX/splendid-mdpi}{GitHub}
\item[Categor y :]
\begin{itemize}
\tightlist
\item[]
\item
  \pandocbounded{\includesvg[keepaspectratio]{/assets/icons/16-atom.svg}}
  \href{https://typst.app/universe/search/?category=paper}{Paper}
\end{itemize}
\end{description}

\subsubsection{Where to report issues?}\label{where-to-report-issues}

This template is a project of James R. Swift . Report issues on
\href{https://github.com/JamesxX/splendid-mdpi}{their repository} . You
can also try to ask for help with this template on the
\href{https://forum.typst.app}{Forum} .

Please report this template to the Typst team using the
\href{https://typst.app/contact}{contact form} if you believe it is a
safety hazard or infringes upon your rights.

\phantomsection\label{versions}
\subsubsection{Version history}\label{version-history}

\begin{longtable}[]{@{}ll@{}}
\toprule\noalign{}
Version & Release Date \\
\midrule\noalign{}
\endhead
\bottomrule\noalign{}
\endlastfoot
0.1.0 & July 16, 2024 \\
\end{longtable}

Typst GmbH did not create this template and cannot guarantee correct
functionality of this template or compatibility with any version of the
Typst compiler or app.


\title{typst.app/universe/package/great-theorems}

\phantomsection\label{banner}
\section{great-theorems}\label{great-theorems}

{ 0.1.1 }

Straightforward and functional theorem/proof environments.

\phantomsection\label{readme}
This package allows you to make \textbf{theorem/proof/remark/…}
blocks.

Features:

\begin{itemize}
\tightlist
\item
  supports advanced counters through both
  \href{https://typst.app/universe/package/headcount/}{headcount} and
  \href{https://typst.app/universe/package/rich-counters/}{rich-counters}
\item
  easy adjustment of style:

  \begin{itemize}
  \tightlist
  \item
    change prefix
  \item
    change how title is displayed
  \item
    change formatting of body
  \item
    change suffix
  \item
    change numbering style
  \item
    configure \emph{all} parameters of the
    \href{https://typst.app/docs/reference/layout/block/}{\texttt{\ block\ }}
    , including background color, stroke color, rounded corners, inset,
    …
  \end{itemize}
\item
  can adjust style also on individual basis (e.g. to highlight main
  theorem)
\item
  works with labels/references
\item
  sane and smart defaults
\end{itemize}

\subsection{Showcase}\label{showcase}

In the following example we use
\href{https://typst.app/universe/package/rich-counters/}{rich-counters}
to configure section-based counters. You can also use
\href{https://typst.app/universe/package/headcount/}{headcount} .

\begin{Shaded}
\begin{Highlighting}[]
\NormalTok{\#import "@preview/great{-}theorems:0.1.1": *}
\NormalTok{\#import "@preview/rich{-}counters:0.2.1": *}

\NormalTok{\#set heading(numbering: "1.1")}
\NormalTok{\#show: great{-}theorems{-}init}

\NormalTok{\#show link: text.with(fill: blue)}

\NormalTok{\#let mathcounter = rich{-}counter(}
\NormalTok{  identifier: "mathblocks",}
\NormalTok{  inherited\_levels: 1}
\NormalTok{)}

\NormalTok{\#let theorem = mathblock(}
\NormalTok{  blocktitle: "Theorem",}
\NormalTok{  counter: mathcounter,}
\NormalTok{)}

\NormalTok{\#let lemma = mathblock(}
\NormalTok{  blocktitle: "Lemma",}
\NormalTok{  counter: mathcounter,}
\NormalTok{)}

\NormalTok{\#let remark = mathblock(}
\NormalTok{  blocktitle: "Remark",}
\NormalTok{  prefix: [\_Remark.\_],}
\NormalTok{  inset: 5pt,}
\NormalTok{  fill: lime,}
\NormalTok{  radius: 5pt,}
\NormalTok{)}

\NormalTok{\#let proof = proofblock()}

\NormalTok{= Some Heading}

\NormalTok{\#theorem[}
\NormalTok{  This is some theorem.}
\NormalTok{] \textless{}mythm\textgreater{}}

\NormalTok{\#lemma[}
\NormalTok{  This is a lemma. Maybe it\textquotesingle{}s used to prove @mythm.}
\NormalTok{]}

\NormalTok{\#proof[}
\NormalTok{  This is a proof.}
\NormalTok{]}

\NormalTok{= Another Heading}

\NormalTok{\#theorem(title: "some title")[}
\NormalTok{  This is a theorem with a title.}
\NormalTok{] \textless{}thm2\textgreater{}}

\NormalTok{\#proof(of: \textless{}thm2\textgreater{})[}
\NormalTok{  This is a proof of the theorem which has a title.}
\NormalTok{]}

\NormalTok{\#remark[}
\NormalTok{  This is a remark.}
\NormalTok{  The remark box has some custom styling applied.}
\NormalTok{]}
\end{Highlighting}
\end{Shaded}

\pandocbounded{\includegraphics[keepaspectratio]{https://github.com/typst/packages/raw/main/packages/preview/great-theorems/0.1.1/example.png}}

\subsection{Usage}\label{usage}

\subsubsection{\texorpdfstring{\texttt{\ great-theorems-init\ }}{ great-theorems-init }}\label{great-theorems-init}

First, make sure to apply the following inital \texttt{\ show\ } rule to
your document:

\begin{Shaded}
\begin{Highlighting}[]
\NormalTok{\#show: great{-}theorems{-}init}
\end{Highlighting}
\end{Shaded}

This is important to make the blocks have the correct alignment and to
display references correctly.

\subsubsection{\texorpdfstring{\texttt{\ mathblock\ }}{ mathblock }}\label{mathblock}

The main constructor you will use is \texttt{\ mathblock\ } , which
allows you to construct a theorem/proof/remark/… environment in
exactly the way you like it.

Please see the showcase above for on example on how to use it. We now
list and explain all possible arguments.

\begin{itemize}
\item
  \texttt{\ blocktitle\ } (required)

  Usually something like \texttt{\ "Theorem"\ } or \texttt{\ "Lemma"\ }
  . Determines how references are displayed, and also determines the
  default \texttt{\ prefix\ } .
\item
  \texttt{\ counter\ } (default: \texttt{\ none\ } )

  If you want your \texttt{\ mathblock\ } to be counted, pass the
  counter here. Accepts either a Typst-native
  \href{https://typst.app/docs/reference/introspection/counter/}{\texttt{\ counter\ }}
  (which can be made to depend on the section with the
  \href{https://typst.app/universe/package/headcount/}{headcount}
  package) or a \texttt{\ rich-counter\ } from the
  \href{https://typst.app/universe/package/rich-counters/}{rich-counters}
  package. If you want multiple \texttt{\ mathblock\ } environments to
  share the same counter, just pass the same counter to all of them.
\item
  \texttt{\ numbering\ } (default: \texttt{\ "1.1"\ } )

  The numbering style that should be used to display the counters.

  \textbf{Note:} If you use the
  \href{https://typst.app/universe/package/headcount/}{headcount}
  package for your counters, you have to pass the
  \texttt{\ dependent-numbering\ } here.
\item
  \texttt{\ prefix\ } (default: contructed from \texttt{\ blocktitle\ }
  , bold style)

  What should be displayed before the body. If you didn’t pass a
  counter, it should just be a piece of content like
  \texttt{\ {[}*Theorem.*{]}\ } . \emph{If you passed a counter} , it
  should a function/closure, which takes the current counter value as an
  argument and returns the corresponding prefix; for example
  \texttt{\ (count)\ =\textgreater{}\ {[}*Theorem\ \#count.*{]}\ }
\item
  \texttt{\ titlix\ } (default:
  \texttt{\ title\ =\textgreater{}\ {[}(\#title){]}\ } )

  How a title should be displayed. Will be placed after the prefix if a
  title is present. Must be function which takes the title and returns
  the corresponding content that should be displayed.
\item
  \texttt{\ suffix\ } (default: \texttt{\ none\ } )

  A suffix that will be displayed after the body.
\item
  \texttt{\ bodyfmt\ } (default:
  \texttt{\ body\ =\textgreater{}\ body\ } i.e. no special formatting)

  A function that will style/transform the body. For example, if you
  want your theorem contents to be displayed in oblique style, you could
  pass \texttt{\ text.with(style:\ "oblique")\ } .
\item
  arguments for the surrounding
  \href{https://typst.app/docs/reference/layout/block/}{\texttt{\ block\ }}

  The \texttt{\ mathblock\ } , as the name suggests, is surrounded by a
  \href{https://typst.app/docs/reference/layout/block/}{\texttt{\ block\ }}
  , which can be styled to have a background color, stroke color,
  rounded corners, etc. . You can just pass all arguments that you could
  pass to a \texttt{\ block\ } also to \texttt{\ mathblock\ } , and it
  will be “passed through� the surrounding \texttt{\ block\ } . For
  example, you could write
  \texttt{\ \#let\ theorem\ =\ mathblock(...,\ fill:\ yellow,\ inset:\ 5pt)\ }
  .
\end{itemize}

So far we have discussed how you \emph{setup} your environment with
\texttt{\ \#let\ theorem\ =\ mathblock(...)\ } . Now let’s discuss how
to use the resulting \texttt{\ theorem\ } command. Again, please see the
showcase above for some examples on how to use it. We now list and
explain all possible arguments (apart from the body).

\begin{itemize}
\item
  \texttt{\ title\ } (default: \texttt{\ none\ } )

  This allows you to set a title for your theorem/lemma/…, which will
  be displayed according to \texttt{\ titlix\ } .
\item
  all the arguments from \texttt{\ mathblock\ } , except
  \texttt{\ blocktitle\ } and \texttt{\ counter\ }

  You can change all the parameters of your \texttt{\ mathblock\ } also
  on an individual basis, i.e. for each occurrence separately, by just
  passing the respective arguments, including \texttt{\ numbering\ } ,
  \texttt{\ prefix\ } , \texttt{\ titlix\ } , \texttt{\ suffix\ } ,
  \texttt{\ bodyfmt\ } , and arguments for \texttt{\ block\ } . These
  will take precedence over the global configuration.
\end{itemize}

\subsubsection{\texorpdfstring{\texttt{\ proofblock\ }}{ proofblock }}\label{proofblock}

Also a proof environment can be constructed with \texttt{\ mathblock\ }
, for example:

\begin{verbatim}
#let proof = mathblock(
  blocktitle: "Proof",
  prefix: [_Proof._],
  suffix: [#h(1fr) $square$],
)
\end{verbatim}

However, for convenience, we have made another \texttt{\ proofblock\ }
constructor. It works exactly the same as \texttt{\ mathblock\ } , the
only differences being:

\begin{itemize}
\tightlist
\item
  it has different default values for \texttt{\ blocktitle\ } ,
  \texttt{\ prefix\ } , and \texttt{\ suffix\ }
\item
  it has no \texttt{\ counter\ } and \texttt{\ numbering\ } argument
\item
  the \texttt{\ titlix\ } argument is replaced with a
  \texttt{\ prefix\_with\_of\ } argument (also consisting of a
  function), which will be used as a prefix when the constructed
  environment is used with \texttt{\ of\ } parameter
\end{itemize}

The constructed environment will have the following changes compared to
an environment constructed with \texttt{\ mathblock\ }

\begin{itemize}
\item
  the \texttt{\ title\ } argument is replaced with an \texttt{\ of\ }
  argument, which is used to denote to which theorem/lemma/… the proof
  belongs

  This can be either just content, or a label, in which case a reference
  to the label is displayed.
\end{itemize}

\subsection{FAQ}\label{faq}

\begin{itemize}
\item
  \emph{What is the difference to the ctheorems package?}

  You can achieve pretty much the same results with both packages. One
  goal of \texttt{\ great-theorems\ } was to have a cleaner
  implementation, for example by separating the counter functionality
  from the theorem block functionality. \texttt{\ ctheorems\ } also uses
  deprecated Typst functionality that will soon be removed. In the end,
  however, in comes down to personal preference, and
  \texttt{\ ctheorems\ } was certainly a big inspiration for this
  package!
\item
  \emph{How to set up the counters the way I want?}

  Please consult the documentation of
  \href{https://typst.app/universe/package/headcount/}{headcount} and
  \href{https://typst.app/universe/package/rich-counters/}{rich-counters}
  respectively, we support both packages as well as native
  \href{https://typst.app/docs/reference/introspection/counter/}{\texttt{\ counter\ }}
  s.
\item
  \emph{My theorems are all center aligned?!}

  You forgot to put the initial show rule at the start of your document:

\begin{Shaded}
\begin{Highlighting}[]
\NormalTok{\#show: great{-}theorems{-}init}
\end{Highlighting}
\end{Shaded}
\item
  \emph{My theorems break across pages, how do I stop that behavior?}

  You can pass \texttt{\ breakable:\ false\ } to \texttt{\ mathblock\ }
  to construct a non-breakable environment.
\item
  \emph{I have a default style for all my theorems/lemmas/remarks/…,
  and I’m writing boilerplate when I construct theorem/lemma/remark
  environments.}

  You can essentially define your own defaults like this:

\begin{Shaded}
\begin{Highlighting}[]
\NormalTok{\#let my\_mathblock = mathblock.with(fill: yellow, radius: 5pt, inset: 5pt)}

\NormalTok{\#let theorem = my\_mathblock(...)}
\NormalTok{\#let lemma = my\_mathblock(...)}
\NormalTok{\#let remark = my\_mathblock(...)}
\NormalTok{...}
\end{Highlighting}
\end{Shaded}
\item
  \emph{The documentation is too short or unclear… how do I do X?}

  Please just open an
  \href{https://github.com/jbirnick/typst-great-theorems/issues}{issue
  on GitHub} , and I will happily answer your question and extend the
  documentation!
\end{itemize}

\subsubsection{How to add}\label{how-to-add}

Copy this into your project and use the import as
\texttt{\ great-theorems\ }

\begin{verbatim}
#import "@preview/great-theorems:0.1.1"
\end{verbatim}

\includesvg[width=0.16667in,height=0.16667in]{/assets/icons/16-copy.svg}

Check the docs for
\href{https://typst.app/docs/reference/scripting/\#packages}{more
information on how to import packages} .

\subsubsection{About}\label{about}

\begin{description}
\tightlist
\item[Author :]
\href{https://jbirnick.net}{Johann Birnick}
\item[License:]
MIT
\item[Current version:]
0.1.1
\item[Last updated:]
October 22, 2024
\item[First released:]
October 16, 2024
\item[Archive size:]
5.32 kB
\href{https://packages.typst.org/preview/great-theorems-0.1.1.tar.gz}{\pandocbounded{\includesvg[keepaspectratio]{/assets/icons/16-download.svg}}}
\item[Repository:]
\href{https://github.com/jbirnick/typst-great-theorems}{GitHub}
\item[Discipline s :]
\begin{itemize}
\tightlist
\item[]
\item
  \href{https://typst.app/universe/search/?discipline=mathematics}{Mathematics}
\item
  \href{https://typst.app/universe/search/?discipline=computer-science}{Computer
  Science}
\item
  \href{https://typst.app/universe/search/?discipline=physics}{Physics}
\item
  \href{https://typst.app/universe/search/?discipline=engineering}{Engineering}
\item
  \href{https://typst.app/universe/search/?discipline=philosophy}{Philosophy}
\item
  \href{https://typst.app/universe/search/?discipline=education}{Education}
\end{itemize}
\item[Categor ies :]
\begin{itemize}
\tightlist
\item[]
\item
  \pandocbounded{\includesvg[keepaspectratio]{/assets/icons/16-package.svg}}
  \href{https://typst.app/universe/search/?category=components}{Components}
\item
  \pandocbounded{\includesvg[keepaspectratio]{/assets/icons/16-list-unordered.svg}}
  \href{https://typst.app/universe/search/?category=model}{Model}
\end{itemize}
\end{description}

\subsubsection{Where to report issues?}\label{where-to-report-issues}

This package is a project of Johann Birnick . Report issues on
\href{https://github.com/jbirnick/typst-great-theorems}{their
repository} . You can also try to ask for help with this package on the
\href{https://forum.typst.app}{Forum} .

Please report this package to the Typst team using the
\href{https://typst.app/contact}{contact form} if you believe it is a
safety hazard or infringes upon your rights.

\phantomsection\label{versions}
\subsubsection{Version history}\label{version-history}

\begin{longtable}[]{@{}ll@{}}
\toprule\noalign{}
Version & Release Date \\
\midrule\noalign{}
\endhead
\bottomrule\noalign{}
\endlastfoot
0.1.1 & October 22, 2024 \\
\href{https://typst.app/universe/package/great-theorems/0.1.0/}{0.1.0} &
October 16, 2024 \\
\end{longtable}

Typst GmbH did not create this package and cannot guarantee correct
functionality of this package or compatibility with any version of the
Typst compiler or app.


\title{typst.app/universe/package/timeliney}

\phantomsection\label{banner}
\section{timeliney}\label{timeliney}

{ 0.1.0 }

Create Gantt charts in Typst.

{ } Featured Package

\phantomsection\label{readme}
Create Gantt charts automatically with Typst!

Here’s a fully-featured example:

\begin{Shaded}
\begin{Highlighting}[]
\NormalTok{\#import "@preview/timeliney:0.1.0"}

\NormalTok{\#timeliney.timeline(}
\NormalTok{  show{-}grid: true,}
\NormalTok{  \{}
\NormalTok{    import timeliney: *}
      
\NormalTok{    headerline(group(([*2023*], 4)), group(([*2024*], 4)))}
\NormalTok{    headerline(}
\NormalTok{      group(..range(4).map(n =\textgreater{} strong("Q" + str(n + 1)))),}
\NormalTok{      group(..range(4).map(n =\textgreater{} strong("Q" + str(n + 1)))),}
\NormalTok{    )}
  
\NormalTok{    taskgroup(title: [*Research*], \{}
\NormalTok{      task("Research the market", (0, 2), style: (stroke: 2pt + gray))}
\NormalTok{      task("Conduct user surveys", (1, 3), style: (stroke: 2pt + gray))}
\NormalTok{    \})}

\NormalTok{    taskgroup(title: [*Development*], \{}
\NormalTok{      task("Create mock{-}ups", (2, 3), style: (stroke: 2pt + gray))}
\NormalTok{      task("Develop application", (3, 5), style: (stroke: 2pt + gray))}
\NormalTok{      task("QA", (3.5, 6), style: (stroke: 2pt + gray))}
\NormalTok{    \})}

\NormalTok{    taskgroup(title: [*Marketing*], \{}
\NormalTok{      task("Press demos", (3.5, 7), style: (stroke: 2pt + gray))}
\NormalTok{      task("Social media advertising", (6, 7.5), style: (stroke: 2pt + gray))}
\NormalTok{    \})}

\NormalTok{    milestone(}
\NormalTok{      at: 3.75,}
\NormalTok{      style: (stroke: (dash: "dashed")),}
\NormalTok{      align(center, [}
\NormalTok{        *Conference demo*\textbackslash{}}
\NormalTok{        Dec 2023}
\NormalTok{      ])}
\NormalTok{    )}

\NormalTok{    milestone(}
\NormalTok{      at: 6.5,}
\NormalTok{      style: (stroke: (dash: "dashed")),}
\NormalTok{      align(center, [}
\NormalTok{        *App store launch*\textbackslash{}}
\NormalTok{        Aug 2024}
\NormalTok{      ])}
\NormalTok{    )}
\NormalTok{  \}}
\NormalTok{)}
\end{Highlighting}
\end{Shaded}

\pandocbounded{\includegraphics[keepaspectratio]{https://github.com/typst/packages/raw/main/packages/preview/timeliney/0.1.0/sample.png}}

\subsection{Installation}\label{installation}

Import with \texttt{\ \#import\ "@preview/timeliney:0.1.0"\ } . Then,
call the \texttt{\ timeliney.timeline\ } function.

\subsection{Documentation}\label{documentation}

See
\href{https://github.com/typst/packages/raw/main/packages/preview/timeliney/0.1.0/manual.pdf}{the
manual} !

\subsection{Changelog}\label{changelog}

\subsubsection{0.1.0}\label{section}

\begin{itemize}
\tightlist
\item
  Update CeTZ to 0.2.2 (@LordBaryhobal)
\item
  Add offset parameter
\end{itemize}

\subsubsection{How to add}\label{how-to-add}

Copy this into your project and use the import as \texttt{\ timeliney\ }

\begin{verbatim}
#import "@preview/timeliney:0.1.0"
\end{verbatim}

\includesvg[width=0.16667in,height=0.16667in]{/assets/icons/16-copy.svg}

Check the docs for
\href{https://typst.app/docs/reference/scripting/\#packages}{more
information on how to import packages} .

\subsubsection{About}\label{about}

\begin{description}
\tightlist
\item[Author :]
Pedro Alves
\item[License:]
MIT
\item[Current version:]
0.1.0
\item[Last updated:]
October 17, 2024
\item[First released:]
October 12, 2023
\item[Archive size:]
6.16 kB
\href{https://packages.typst.org/preview/timeliney-0.1.0.tar.gz}{\pandocbounded{\includesvg[keepaspectratio]{/assets/icons/16-download.svg}}}
\item[Repository:]
\href{https://github.com/pta2002/typst-timeliney}{GitHub}
\end{description}

\subsubsection{Where to report issues?}\label{where-to-report-issues}

This package is a project of Pedro Alves . Report issues on
\href{https://github.com/pta2002/typst-timeliney}{their repository} .
You can also try to ask for help with this package on the
\href{https://forum.typst.app}{Forum} .

Please report this package to the Typst team using the
\href{https://typst.app/contact}{contact form} if you believe it is a
safety hazard or infringes upon your rights.

\phantomsection\label{versions}
\subsubsection{Version history}\label{version-history}

\begin{longtable}[]{@{}ll@{}}
\toprule\noalign{}
Version & Release Date \\
\midrule\noalign{}
\endhead
\bottomrule\noalign{}
\endlastfoot
0.1.0 & October 17, 2024 \\
\href{https://typst.app/universe/package/timeliney/0.0.1/}{0.0.1} &
October 12, 2023 \\
\end{longtable}

Typst GmbH did not create this package and cannot guarantee correct
functionality of this package or compatibility with any version of the
Typst compiler or app.


\title{typst.app/universe/package/koma-labeling}

\phantomsection\label{banner}
\section{koma-labeling}\label{koma-labeling}

{ 0.1.0 }

This package introduces a labeling feature to Typst, inspired by the
KOMA-Script\textquotesingle s labeling environment.

\phantomsection\label{readme}
Version 0.1.0

The koma-labeling package for Typst is inspired by the labeling
environment from the KOMA-Script bundle in LaTeX. It provides a
convenient way to create labeled lists with customizable label widths
and optional delimiters, making it perfect for creating structured
descriptions and lists in your Typst documents.

\subsection{Getting Started}\label{getting-started}

To get started with koma-labeling, simply import the package in your
Typst document and use the labeling environment to create your labeled
lists.

\begin{Shaded}
\begin{Highlighting}[]
\NormalTok{\#import "@preview/koma{-}labeling:0.1.0": labeling}

\NormalTok{\#labeling(}
\NormalTok{  (}
\NormalTok{    (lorem(1), lorem(10)),}
\NormalTok{    (lorem(2), lorem(20)),}
\NormalTok{    (lorem(3), lorem(30)),}
\NormalTok{  )}
\NormalTok{)}

\NormalTok{// or}

\NormalTok{\#labeling(}
\NormalTok{  (}
\NormalTok{    ([\#lorem(1)], [\#lorem(10)]),}
\NormalTok{    ([\#lorem(2)], [\#lorem(20)]),}
\NormalTok{    ([\#lorem(3)], [\#lorem(30)]),}
\NormalTok{  )}
\NormalTok{)}
\end{Highlighting}
\end{Shaded}

Output:

\pandocbounded{\includegraphics[keepaspectratio]{https://github.com/user-attachments/assets/bf382afe-f66d-4032-9055-f46c72a2e7dd}}

\textbf{Note:} Remember to terminate the list with a comma, even if only
one pair of items is passed.

\begin{Shaded}
\begin{Highlighting}[]
\NormalTok{\#import "@preview/koma{-}labeling:0.1.0": labeling}

\NormalTok{\#labeling(}
\NormalTok{  (}
\NormalTok{    (lorem(1), lorem(10)),  // Terminating the list with a comma is REQUIRED}
\NormalTok{  )}
\NormalTok{)}
\end{Highlighting}
\end{Shaded}

\subsection{Parameters}\label{parameters}

Although labeling is implemented using \texttt{\ tables\ } , its usage
is similar to \texttt{\ terms\ } , except that it lacks the
\texttt{\ tight\ } and \texttt{\ hanging-indent\ } parameters. If you
have any questions about the parameters for \texttt{\ labeling\ } , you
can refer to
\href{https://typst.app/docs/reference/model/terms/}{\texttt{\ terms\ }}
.

\begin{Shaded}
\begin{Highlighting}[]
\NormalTok{labeling(}
\NormalTok{  separator: content,}
\NormalTok{  indent: length,}
\NormalTok{  spacing: auto length}
\NormalTok{  pairs: ((content, content))}
\NormalTok{)}
\end{Highlighting}
\end{Shaded}

\subsubsection{separator}\label{separator}

The separator between the item and the description.

Default: \texttt{\ {[}:\#h(0.6em){]}\ }

\subsubsection{indent}\label{indent}

The indentation of each item.

Default: \texttt{\ 0pt\ }

\subsubsection{spacing}\label{spacing}

The spacing between the items of the term list.

Default: \texttt{\ auto\ }

\subsubsection{pairs}\label{pairs}

An array of \texttt{\ (item,\ description)\ } pairs.

Example:

\begin{Shaded}
\begin{Highlighting}[]
\NormalTok{\#labeling(}
\NormalTok{  (}
\NormalTok{    ([key 1],[description 1]),}
\NormalTok{    ([keyword 2],[description 2]),}
\NormalTok{  )}
\NormalTok{)}
\end{Highlighting}
\end{Shaded}

\subsection{Additional Documentation and
Acknowledgments}\label{additional-documentation-and-acknowledgments}

For more information on the koma-labeling package and its features, you
can refer to the following resources:

\begin{itemize}
\tightlist
\item
  Typst Documentation: \href{https://typst.app/docs}{Typst
  Documentation}
\item
  KOMA-Script Documentation:
  \href{https://ctan.org/pkg/koma-script}{KOMA-Script Documentation}
\end{itemize}

\subsubsection{How to add}\label{how-to-add}

Copy this into your project and use the import as
\texttt{\ koma-labeling\ }

\begin{verbatim}
#import "@preview/koma-labeling:0.1.0"
\end{verbatim}

\includesvg[width=0.16667in,height=0.16667in]{/assets/icons/16-copy.svg}

Check the docs for
\href{https://typst.app/docs/reference/scripting/\#packages}{more
information on how to import packages} .

\subsubsection{About}\label{about}

\begin{description}
\tightlist
\item[Author :]
Laniakea Kamasylvia
\item[License:]
MIT
\item[Current version:]
0.1.0
\item[Last updated:]
October 28, 2024
\item[First released:]
October 28, 2024
\item[Minimum Typst version:]
0.11.0
\item[Archive size:]
2.54 kB
\href{https://packages.typst.org/preview/koma-labeling-0.1.0.tar.gz}{\pandocbounded{\includesvg[keepaspectratio]{/assets/icons/16-download.svg}}}
\item[Categor y :]
\begin{itemize}
\tightlist
\item[]
\item
  \pandocbounded{\includesvg[keepaspectratio]{/assets/icons/16-code.svg}}
  \href{https://typst.app/universe/search/?category=scripting}{Scripting}
\end{itemize}
\end{description}

\subsubsection{Where to report issues?}\label{where-to-report-issues}

This package is a project of Laniakea Kamasylvia . You can also try to
ask for help with this package on the
\href{https://forum.typst.app}{Forum} .

Please report this package to the Typst team using the
\href{https://typst.app/contact}{contact form} if you believe it is a
safety hazard or infringes upon your rights.

\phantomsection\label{versions}
\subsubsection{Version history}\label{version-history}

\begin{longtable}[]{@{}ll@{}}
\toprule\noalign{}
Version & Release Date \\
\midrule\noalign{}
\endhead
\bottomrule\noalign{}
\endlastfoot
0.1.0 & October 28, 2024 \\
\end{longtable}

Typst GmbH did not create this package and cannot guarantee correct
functionality of this package or compatibility with any version of the
Typst compiler or app.


\title{typst.app/universe/package/teig}

\phantomsection\label{banner}
\section{teig}\label{teig}

{ 0.1.0 }

Calculate eigenvalues of matrices

\phantomsection\label{readme}
This package provides an \texttt{\ eigenvalue\ } function that
calculates the eigenvalues of a matrix.

\begin{Shaded}
\begin{Highlighting}[]
\NormalTok{\#import "@preview/teig:0.1.0": eigenvalues}

\NormalTok{\#let data = (}
\NormalTok{  (1, 2, 3),}
\NormalTok{  (4, 5, 6),}
\NormalTok{  (7, 8, 9),}
\NormalTok{)}

\NormalTok{\#let evals = eigenvalues(data)}

\NormalTok{The eigenvalues of}
\NormalTok{$}
\NormalTok{  \#math.mat(..data)}
\NormalTok{$}
\NormalTok{are approximately}

\NormalTok{$}
\NormalTok{  \#math.vec(..evals.map(x =\textgreater{} str(calc.round(x, digits: 3))))}
\NormalTok{$}
\end{Highlighting}
\end{Shaded}

\pandocbounded{\includegraphics[keepaspectratio]{https://github.com/typst/packages/raw/main/packages/preview/teig/0.1.0/example.png}}

\subsubsection{How to add}\label{how-to-add}

Copy this into your project and use the import as \texttt{\ teig\ }

\begin{verbatim}
#import "@preview/teig:0.1.0"
\end{verbatim}

\includesvg[width=0.16667in,height=0.16667in]{/assets/icons/16-copy.svg}

Check the docs for
\href{https://typst.app/docs/reference/scripting/\#packages}{more
information on how to import packages} .

\subsubsection{About}\label{about}

\begin{description}
\tightlist
\item[Author :]
SolidTux
\item[License:]
MIT
\item[Current version:]
0.1.0
\item[Last updated:]
October 2, 2024
\item[First released:]
October 2, 2024
\item[Archive size:]
62.2 kB
\href{https://packages.typst.org/preview/teig-0.1.0.tar.gz}{\pandocbounded{\includesvg[keepaspectratio]{/assets/icons/16-download.svg}}}
\item[Repository:]
\href{https://gitlab.com/SolidTux/teig}{GitLab}
\item[Discipline :]
\begin{itemize}
\tightlist
\item[]
\item
  \href{https://typst.app/universe/search/?discipline=mathematics}{Mathematics}
\end{itemize}
\item[Categor ies :]
\begin{itemize}
\tightlist
\item[]
\item
  \pandocbounded{\includesvg[keepaspectratio]{/assets/icons/16-code.svg}}
  \href{https://typst.app/universe/search/?category=scripting}{Scripting}
\item
  \pandocbounded{\includesvg[keepaspectratio]{/assets/icons/16-hammer.svg}}
  \href{https://typst.app/universe/search/?category=utility}{Utility}
\end{itemize}
\end{description}

\subsubsection{Where to report issues?}\label{where-to-report-issues}

This package is a project of SolidTux . Report issues on
\href{https://gitlab.com/SolidTux/teig}{their repository} . You can also
try to ask for help with this package on the
\href{https://forum.typst.app}{Forum} .

Please report this package to the Typst team using the
\href{https://typst.app/contact}{contact form} if you believe it is a
safety hazard or infringes upon your rights.

\phantomsection\label{versions}
\subsubsection{Version history}\label{version-history}

\begin{longtable}[]{@{}ll@{}}
\toprule\noalign{}
Version & Release Date \\
\midrule\noalign{}
\endhead
\bottomrule\noalign{}
\endlastfoot
0.1.0 & October 2, 2024 \\
\end{longtable}

Typst GmbH did not create this package and cannot guarantee correct
functionality of this package or compatibility with any version of the
Typst compiler or app.


