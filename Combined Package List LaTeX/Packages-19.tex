\title{typst.app/universe/package/sunny-famnit}

\phantomsection\label{banner}
\phantomsection\label{template-thumbnail}
\pandocbounded{\includegraphics[keepaspectratio]{https://packages.typst.org/preview/thumbnails/sunny-famnit-0.2.0-small.webp}}

\section{sunny-famnit}\label{sunny-famnit}

{ 0.2.0 }

Thesis template for University of Primorska, FAMNIT

\href{/app?template=sunny-famnit&version=0.2.0}{Create project in app}

\phantomsection\label{readme}
\pandocbounded{\includegraphics[keepaspectratio]{https://img.shields.io/github/v/release/Tiggax/famnit_typst_template}}
\pandocbounded{\includegraphics[keepaspectratio]{https://img.shields.io/github/stars/Tiggax/famnit_typst_template}}

\pandocbounded{\includegraphics[keepaspectratio]{https://www.famnit.upr.si/img/UP_FAMNIT.png}}

\emph{University of Primorska,}

\emph{Faculty of Mathematics, Natural Sciences and Information
Technologies}

\begin{center}\rule{0.5\linewidth}{0.5pt}\end{center}

This is a Typst template for FAMNIT final work.

\begin{center}\rule{0.5\linewidth}{0.5pt}\end{center}

\subsection{configuration example}\label{configuration-example}

\begin{Shaded}
\begin{Highlighting}[]
\NormalTok{\#import "@preview/sunny{-}famnit:0.2.0": project}

\NormalTok{\#show project.with(}
\NormalTok{    date: datetime(day: 1, month: 1, year: 2024), // you could also do \textasciigrave{}datetime.today()\textasciigrave{}}
\NormalTok{    text\_lang: "en" // the language that the thesis is gonna be written in.}
    
\NormalTok{    author: "your name"}
\NormalTok{    studij: "your course",}
\NormalTok{    mentor: (}
\NormalTok{    name: "his name", }
\NormalTok{    en: ("prepends","postpends"), // you can prepend or postpend any titles}
\NormalTok{    sl: ("predstavki","postavki"),// you can prepend or postpend any titles}
\NormalTok{    ),}
\NormalTok{    somentor: none, // if you have a co{-}mentor write him here the same way as mentor, else you can just remove the line.}
\NormalTok{    work\_mentor: none, // if you have a work mentor, the same as above}

\NormalTok{    naslov: "your title in slovene",}
\NormalTok{    title: "your title",}

\NormalTok{    izvleček: [}
\NormalTok{        your abstract in slovene.}
\NormalTok{    ],}
\NormalTok{    abstract: [}
\NormalTok{        your abstract}
\NormalTok{    ],}

\NormalTok{    ključne\_besede: ("Typst", "je", "super!"),}
\NormalTok{    key\_words: ("Typst", "is", "Awesome!"),}

\NormalTok{    kratice: (}
\NormalTok{        "Famnit": "Fakulteta za matematiko naravoslovje in informacijske tehnologije",}
\NormalTok{        "PDF": "Portable document format",}
\NormalTok{    ),}

\NormalTok{    priloge: (), // you can add attachments as a dict of a title and content like \textasciigrave{}"name": [content],\textasciigrave{}}

\NormalTok{    zahvala: [}
\NormalTok{        you can add an acknowlegment.}
\NormalTok{    ],}

\NormalTok{  bib\_file: bibliography(}
\NormalTok{    "my\_references.bib",}
\NormalTok{    style: "ieee",}
\NormalTok{    title: [Bibliography],}
\NormalTok{  ),}

\NormalTok{    /* Additional content and their defaults}
\NormalTok{    kraj: "Koper",}
\NormalTok{    */}
\NormalTok{)}

\NormalTok{// Your content goes below.}
\end{Highlighting}
\end{Shaded}

\subsection{Abbreviations (kratice)}\label{abbreviations-kratice}

You can specify Abbreviations at the start as an attribute
\texttt{\ kratice\ } and pass it a dictionary of the abbriviation and
it’s explanation. Then you can reference them in text using
\texttt{\ @\textless{}short\ name\textgreater{}\ } to create a link to
it.

\subsection{Attachments}\label{attachments}

Some thesis need Attachments that are shown at the end of the file. To
add these attachments add them in your project under
\texttt{\ priloge\ } as a dictionary of the attachment name and its
content. I suggest having a seperate \texttt{\ attachments.typ\ } file,
from where you can reference them in the main project.

\subsection{Language}\label{language}

The writing of the thesis can be achieved in two languages; Slovene and
English. They have some differences between them in the way the template
is generated, as the thesis needs to be different for each one. you can
specify the language with the \texttt{\ text\_lang\ } attribute.

\begin{center}\rule{0.5\linewidth}{0.5pt}\end{center}

If you have any questions, suggestion or improvements open an issue or a
pull request
\href{https://github.com/Tiggax/famnit_typst_template}{here}

\href{/app?template=sunny-famnit&version=0.2.0}{Create project in app}

\subsubsection{How to use}\label{how-to-use}

Click the button above to create a new project using this template in
the Typst app.

You can also use the Typst CLI to start a new project on your computer
using this command:

\begin{verbatim}
typst init @preview/sunny-famnit:0.2.0
\end{verbatim}

\includesvg[width=0.16667in,height=0.16667in]{/assets/icons/16-copy.svg}

\subsubsection{About}\label{about}

\begin{description}
\tightlist
\item[Author :]
Tilen Gimpelj {[}@Tiggax{]}
\item[License:]
MIT
\item[Current version:]
0.2.0
\item[Last updated:]
July 19, 2024
\item[First released:]
March 18, 2024
\item[Minimum Typst version:]
0.11.0
\item[Archive size:]
5.38 kB
\href{https://packages.typst.org/preview/sunny-famnit-0.2.0.tar.gz}{\pandocbounded{\includesvg[keepaspectratio]{/assets/icons/16-download.svg}}}
\item[Repository:]
\href{https://github.com/Tiggax/famnit_typst_template}{GitHub}
\item[Discipline s :]
\begin{itemize}
\tightlist
\item[]
\item
  \href{https://typst.app/universe/search/?discipline=computer-science}{Computer
  Science}
\item
  \href{https://typst.app/universe/search/?discipline=biology}{Biology}
\item
  \href{https://typst.app/universe/search/?discipline=mathematics}{Mathematics}
\end{itemize}
\item[Categor y :]
\begin{itemize}
\tightlist
\item[]
\item
  \pandocbounded{\includesvg[keepaspectratio]{/assets/icons/16-mortarboard.svg}}
  \href{https://typst.app/universe/search/?category=thesis}{Thesis}
\end{itemize}
\end{description}

\subsubsection{Where to report issues?}\label{where-to-report-issues}

This template is a project of Tilen Gimpelj {[}@Tiggax{]} . Report
issues on \href{https://github.com/Tiggax/famnit_typst_template}{their
repository} . You can also try to ask for help with this template on the
\href{https://forum.typst.app}{Forum} .

Please report this template to the Typst team using the
\href{https://typst.app/contact}{contact form} if you believe it is a
safety hazard or infringes upon your rights.

\phantomsection\label{versions}
\subsubsection{Version history}\label{version-history}

\begin{longtable}[]{@{}ll@{}}
\toprule\noalign{}
Version & Release Date \\
\midrule\noalign{}
\endhead
\bottomrule\noalign{}
\endlastfoot
0.2.0 & July 19, 2024 \\
\href{https://typst.app/universe/package/sunny-famnit/0.1.0/}{0.1.0} &
March 18, 2024 \\
\end{longtable}

Typst GmbH did not create this template and cannot guarantee correct
functionality of this template or compatibility with any version of the
Typst compiler or app.


\title{typst.app/universe/package/jurz}

\phantomsection\label{banner}
\section{jurz}\label{jurz}

{ 0.1.0 }

Randziffern in Typst

\phantomsection\label{readme}
\href{https://de.wikipedia.org/w/index.php?title=Randnummer&oldid=231943223}{\emph{Randziffern}}
(also called \emph{Randnummern} ) are a way to reference text passages
in a document, independent of the page number or the section number.
They are used in many German legal texts, for example. This package
provides a way to create \emph{Randziffern} in Typst.

\subsection{Demo}\label{demo}

\begin{longtable}[]{@{}ll@{}}
\toprule\noalign{}
\endhead
\bottomrule\noalign{}
\endlastfoot
\pandocbounded{\includesvg[keepaspectratio]{https://github.com/typst/packages/raw/main/packages/preview/jurz/0.1.0/demo-2.svg}}
&
\pandocbounded{\includesvg[keepaspectratio]{https://github.com/typst/packages/raw/main/packages/preview/jurz/0.1.0/demo-3.svg}} \\
\end{longtable}

View source

\begin{Shaded}
\begin{Highlighting}[]
\NormalTok{\#show: init{-}jurz.with(}
\NormalTok{  gap: 1em,}
\NormalTok{  two{-}sided: true}
\NormalTok{)}

\NormalTok{\#rz \#lorem(50)}

\NormalTok{\#lorem(20)}

\NormalTok{\#rz\textless{}abc\textgreater{} \#lorem(30)}

\NormalTok{\#rz \#lorem(40)}

\NormalTok{\#rz \#lorem(50)}

\NormalTok{\#lorem(20)}

\NormalTok{\#rz \#lorem(24)}

\NormalTok{Fur further information, look at @abc.}
\end{Highlighting}
\end{Shaded}

\subsection{Reference}\label{reference}

\subsubsection{\texorpdfstring{\texttt{\ init-jurz\ }}{ init-jurz }}\label{init-jurz}

A show rule that initializes the \emph{Randziffern} for the document.
This rule should be placed at the beginning of the document. It also
allows customizing the behavior of the \emph{Randziffern} .

\paragraph{Usage}\label{usage}

\begin{Shaded}
\begin{Highlighting}[]
\NormalTok{\#show: init{-}jurz.with(}
\NormalTok{ // parameters}
\NormalTok{ // two{-}sided: true,}
\NormalTok{ // gap: 1em,}
\NormalTok{ // supplement: "Rz.",}
\NormalTok{ // reset{-}level: 0,}
\NormalTok{)}
\end{Highlighting}
\end{Shaded}

\paragraph{Parameters}\label{parameters}

\begin{itemize}
\tightlist
\item
  \texttt{\ two-sided\ } (optional): If \texttt{\ true\ } , the
  \emph{Randziffern} are placed on the outer margin of the page. If
  \texttt{\ false\ } , they are placed on the left margin. Default is
  \texttt{\ true\ } .
\item
  \texttt{\ gap\ } (optional): The distance between the
  \emph{Randziffer} and the text. Default is \texttt{\ 1em\ } .
\item
  \texttt{\ supplement\ } (optional): The text that is placed before the
  \emph{Randziffer} when referencing it. Default is \texttt{\ "Rz."\ } .
\item
  \texttt{\ reset-level\ } (optional): The heading level at which the
  \emph{Randziffern} are reset. If set to \texttt{\ 3\ } , for example,
  the numbering of the \emph{Randziffern} restarts after every heading
  of levels \texttt{\ 1\ } , \texttt{\ 2\ } , or \texttt{\ 3\ } .
  Default is \texttt{\ 0\ } .
\end{itemize}

\subsubsection{\texorpdfstring{\texttt{\ rz\ }}{ rz }}\label{rz}

Adds a \emph{Randziffer} to the text. The \emph{Randziffer} is a unique
identifier that can be referenced in the text.

You can add references the same way you can with headings. In fact, the
\emph{Randziffer} is treated as a heading of level \texttt{\ 99\ } under
the hood.

\paragraph{Usage}\label{usage-1}

\begin{Shaded}
\begin{Highlighting}[]
\NormalTok{\#rz \#lorem(100)}
\NormalTok{\#rz\textless{}abc\textgreater{} \#lorem(100)}

\NormalTok{See also @abc.}
\end{Highlighting}
\end{Shaded}

\subsection{License}\label{license}

This package is licensed under the MIT License.

\subsubsection{How to add}\label{how-to-add}

Copy this into your project and use the import as \texttt{\ jurz\ }

\begin{verbatim}
#import "@preview/jurz:0.1.0"
\end{verbatim}

\includesvg[width=0.16667in,height=0.16667in]{/assets/icons/16-copy.svg}

Check the docs for
\href{https://typst.app/docs/reference/scripting/\#packages}{more
information on how to import packages} .

\subsubsection{About}\label{about}

\begin{description}
\tightlist
\item[Author :]
\href{https://github.com/pklaschka}{Zuri Klaschka}
\item[License:]
MIT
\item[Current version:]
0.1.0
\item[Last updated:]
April 4, 2024
\item[First released:]
April 4, 2024
\item[Archive size:]
2.46 kB
\href{https://packages.typst.org/preview/jurz-0.1.0.tar.gz}{\pandocbounded{\includesvg[keepaspectratio]{/assets/icons/16-download.svg}}}
\item[Discipline :]
\begin{itemize}
\tightlist
\item[]
\item
  \href{https://typst.app/universe/search/?discipline=law}{Law}
\end{itemize}
\item[Categor ies :]
\begin{itemize}
\tightlist
\item[]
\item
  \pandocbounded{\includesvg[keepaspectratio]{/assets/icons/16-envelope.svg}}
  \href{https://typst.app/universe/search/?category=office}{Office}
\item
  \pandocbounded{\includesvg[keepaspectratio]{/assets/icons/16-package.svg}}
  \href{https://typst.app/universe/search/?category=components}{Components}
\item
  \pandocbounded{\includesvg[keepaspectratio]{/assets/icons/16-layout.svg}}
  \href{https://typst.app/universe/search/?category=layout}{Layout}
\end{itemize}
\end{description}

\subsubsection{Where to report issues?}\label{where-to-report-issues}

This package is a project of Zuri Klaschka . You can also try to ask for
help with this package on the \href{https://forum.typst.app}{Forum} .

Please report this package to the Typst team using the
\href{https://typst.app/contact}{contact form} if you believe it is a
safety hazard or infringes upon your rights.

\phantomsection\label{versions}
\subsubsection{Version history}\label{version-history}

\begin{longtable}[]{@{}ll@{}}
\toprule\noalign{}
Version & Release Date \\
\midrule\noalign{}
\endhead
\bottomrule\noalign{}
\endlastfoot
0.1.0 & April 4, 2024 \\
\end{longtable}

Typst GmbH did not create this package and cannot guarantee correct
functionality of this package or compatibility with any version of the
Typst compiler or app.


\title{typst.app/universe/package/easy-pinyin}

\phantomsection\label{banner}
\section{easy-pinyin}\label{easy-pinyin}

{ 0.1.0 }

Write Chinese pinyin easily.

\phantomsection\label{readme}
Write Chinese pinyin easily.

\subsection{Usage}\label{usage}

Import the package:

\begin{Shaded}
\begin{Highlighting}[]
\NormalTok{\#import "@preview/easy{-}pinyin:0.1.0": pinyin, zhuyin}
\end{Highlighting}
\end{Shaded}

With the \texttt{\ pinyin\ } function, you can use \texttt{\ a2\ } to
write an \texttt{\ É‘Ì?\ } , \texttt{\ o3\ } to write an \texttt{\ Ç’\ }
, \texttt{\ v4\ } to represent \texttt{\ ǜ\ } , etc.

With \texttt{\ zhuyin\ } function,you can put pinyin above the text
easily, with parameters:

\begin{itemize}
\tightlist
\item
  positional parameters:

  \begin{itemize}
  \tightlist
  \item
    \texttt{\ doc:\ content\textbar{}string\ } : main characters
  \item
    \texttt{\ ruby:\ content\textbar{}string\ } : zhuyin characters
  \end{itemize}
\item
  named parameters:

  \begin{itemize}
  \tightlist
  \item
    \texttt{\ scale:\ number\ =\ 0.7\ } : font size scale of
    \texttt{\ ruby\ } , default \texttt{\ 0.7\ }
  \item
    \texttt{\ gutter:\ length\ =\ 0.3em\ } : spacing between
    \texttt{\ doc\ } and \texttt{\ ruby\ } , default \texttt{\ 0.3em\ }
  \item
    \texttt{\ delimiter:\ string\textbar{}none\ =\ none\ } : if not
    none, use this character to split \texttt{\ doc\ } and
    \texttt{\ ruby\ } into parts
  \item
    \texttt{\ spacing:\ length\textbar{}none\ =\ none\ } : spacing
    between each parts
  \end{itemize}
\end{itemize}

See example bellow.

\subsection{Example}\label{example}

\begin{Shaded}
\begin{Highlighting}[]
\NormalTok{汉(\#pinyin[ha4n])语(\#pinyin[yu3])拼(\#pinyin[pi1n])音(\#pinyin[yi1n])。}

\NormalTok{\#let per{-}char(f) = [\#f(delimiter: "|")[汉|语|拼|音][ha4n|yu3|pi1n|yi1n]]}
\NormalTok{\#let per{-}word(f) = [\#f(delimiter: "|")[汉语|拼音][ha4nyu3|pi1nyi1n]]}
\NormalTok{\#let all{-}in{-}one(f) = [\#f[汉语拼音][ha4nyu3pi1nyi1n]]}
\NormalTok{\#let example(f) = (per{-}char(f), per{-}word(f), all{-}in{-}one(f))}

\NormalTok{// argument of scale and spacing}
\NormalTok{\#let arguments = ((0.5, none), (0.7, none), (0.7, 0.1em), (1.0, none), (1.0, 0.2em))}

\NormalTok{\#table(}
\NormalTok{  columns: (auto, auto, auto, auto),}
\NormalTok{  align: (center + horizon, center, center, center),}
\NormalTok{  [arguments], [per char], [per word], [all in one],}
\NormalTok{  ..arguments.map(((scale, spacing)) =\textgreater{} (}
\NormalTok{    text(size: 0.7em)[\#scale,\#repr(spacing)], }
\NormalTok{    ..example(zhuyin.with(scale: scale, spacing: spacing))}
\NormalTok{  )).flatten(),}
\NormalTok{)}
\end{Highlighting}
\end{Shaded}

\pandocbounded{\includegraphics[keepaspectratio]{https://raw.githubusercontent.com/7sDream/typst-easy-pinyin/master/example.png?raw=true}}

\subsection{LICENSE}\label{license}

MIT, see License file.

\subsubsection{How to add}\label{how-to-add}

Copy this into your project and use the import as
\texttt{\ easy-pinyin\ }

\begin{verbatim}
#import "@preview/easy-pinyin:0.1.0"
\end{verbatim}

\includesvg[width=0.16667in,height=0.16667in]{/assets/icons/16-copy.svg}

Check the docs for
\href{https://typst.app/docs/reference/scripting/\#packages}{more
information on how to import packages} .

\subsubsection{About}\label{about}

\begin{description}
\tightlist
\item[Author s :]
7sDream \& Other open-source contributors
\item[License:]
MIT
\item[Current version:]
0.1.0
\item[Last updated:]
July 6, 2023
\item[First released:]
July 6, 2023
\item[Archive size:]
2.43 kB
\href{https://packages.typst.org/preview/easy-pinyin-0.1.0.tar.gz}{\pandocbounded{\includesvg[keepaspectratio]{/assets/icons/16-download.svg}}}
\item[Repository:]
\href{https://github.com/7sDream/typst-easy-pinyin}{GitHub}
\end{description}

\subsubsection{Where to report issues?}\label{where-to-report-issues}

This package is a project of 7sDream and Other open-source contributors
. Report issues on
\href{https://github.com/7sDream/typst-easy-pinyin}{their repository} .
You can also try to ask for help with this package on the
\href{https://forum.typst.app}{Forum} .

Please report this package to the Typst team using the
\href{https://typst.app/contact}{contact form} if you believe it is a
safety hazard or infringes upon your rights.

\phantomsection\label{versions}
\subsubsection{Version history}\label{version-history}

\begin{longtable}[]{@{}ll@{}}
\toprule\noalign{}
Version & Release Date \\
\midrule\noalign{}
\endhead
\bottomrule\noalign{}
\endlastfoot
0.1.0 & July 6, 2023 \\
\end{longtable}

Typst GmbH did not create this package and cannot guarantee correct
functionality of this package or compatibility with any version of the
Typst compiler or app.


\title{typst.app/universe/package/mcm-scaffold}

\phantomsection\label{banner}
\phantomsection\label{template-thumbnail}
\pandocbounded{\includegraphics[keepaspectratio]{https://packages.typst.org/preview/thumbnails/mcm-scaffold-0.1.0-small.webp}}

\section{mcm-scaffold}\label{mcm-scaffold}

{ 0.1.0 }

A Typst template for COMAP\textquotesingle s Mathematical Contest in
MCM/ICM

\href{/app?template=mcm-scaffold&version=0.1.0}{Create project in app}

\phantomsection\label{readme}
This is a Typst template for COMAP’s Mathematical Contest in MCM/ICM.

\subsection{Usage}\label{usage}

You can use this template in the Typst web app by clicking “Start from
template� on the dashboard and searching for \texttt{\ mcm-scaffold\ }
.

Alternatively, you can use the CLI to kick this project off using the
command

\begin{verbatim}
typst init @preview/mcm-scaffold
\end{verbatim}

Typst will create a new directory with all the files needed to get you
started.

\subsection{Configuration}\label{configuration}

This template exports the \texttt{\ mcm\ } function with the following
named arguments:

\begin{itemize}
\tightlist
\item
  \texttt{\ title\ } : The paper’s title as content.
\item
  \texttt{\ problem-chosen\ } : The problem your team have chosen.
\item
  \texttt{\ team-control-number\ } : Your team control number.
\item
  \texttt{\ year\ } : When did the competition took place.
\item
  \texttt{\ summary\ } : The content of a brief summary of the paper.
  Appears at the top of the first column in boldface.
\item
  \texttt{\ keywords\ } : Keywords of the paper.
\item
  \texttt{\ magic-leading\ } : adjust the leading of the summary.
\end{itemize}

The function also accepts a single, positional argument for the body of
the paper.

The template will initialize your package with a sample call to the
\texttt{\ mcm\ } function in a show rule. If you want to change an
existing project to use this template, you can add a show rule like this
at the top of your file:

\begin{Shaded}
\begin{Highlighting}[]
\NormalTok{\#import "@preview/mcm{-}scaffold:0.1.0": *}

\NormalTok{\#show: mcm.with(}
\NormalTok{  title: "A Simple Example for MCM/ICM Typst Template",}
\NormalTok{  problem{-}chosen: "ABCDEF",}
\NormalTok{  team{-}control{-}number: "1111111",}
\NormalTok{  year: "2025",}
\NormalTok{  summary: [}
\NormalTok{    \#lorem(100)}
    
\NormalTok{    \#lorem(100)}
    
\NormalTok{    \#lorem(100)}

\NormalTok{    \#lorem(100)}
\NormalTok{  ],}
\NormalTok{  keywords: [MCM; ICM; Mathemetical; template],}
\NormalTok{  magic{-}leading: 0.65em,}
\NormalTok{)}

\NormalTok{// Your content goes below.}
\end{Highlighting}
\end{Shaded}

\href{/app?template=mcm-scaffold&version=0.1.0}{Create project in app}

\subsubsection{How to use}\label{how-to-use}

Click the button above to create a new project using this template in
the Typst app.

You can also use the Typst CLI to start a new project on your computer
using this command:

\begin{verbatim}
typst init @preview/mcm-scaffold:0.1.0
\end{verbatim}

\includesvg[width=0.16667in,height=0.16667in]{/assets/icons/16-copy.svg}

\subsubsection{About}\label{about}

\begin{description}
\tightlist
\item[Author :]
\href{https://github.com/sxdl}{LuoQiu}
\item[License:]
Apache-2.0
\item[Current version:]
0.1.0
\item[Last updated:]
April 2, 2024
\item[First released:]
April 2, 2024
\item[Archive size:]
492 kB
\href{https://packages.typst.org/preview/mcm-scaffold-0.1.0.tar.gz}{\pandocbounded{\includesvg[keepaspectratio]{/assets/icons/16-download.svg}}}
\item[Repository:]
\href{https://github.com/sxdl/MCM-Typst-template}{GitHub}
\item[Discipline s :]
\begin{itemize}
\tightlist
\item[]
\item
  \href{https://typst.app/universe/search/?discipline=mathematics}{Mathematics}
\item
  \href{https://typst.app/universe/search/?discipline=computer-science}{Computer
  Science}
\end{itemize}
\item[Categor y :]
\begin{itemize}
\tightlist
\item[]
\item
  \pandocbounded{\includesvg[keepaspectratio]{/assets/icons/16-mortarboard.svg}}
  \href{https://typst.app/universe/search/?category=thesis}{Thesis}
\end{itemize}
\end{description}

\subsubsection{Where to report issues?}\label{where-to-report-issues}

This template is a project of LuoQiu . Report issues on
\href{https://github.com/sxdl/MCM-Typst-template}{their repository} .
You can also try to ask for help with this template on the
\href{https://forum.typst.app}{Forum} .

Please report this template to the Typst team using the
\href{https://typst.app/contact}{contact form} if you believe it is a
safety hazard or infringes upon your rights.

\phantomsection\label{versions}
\subsubsection{Version history}\label{version-history}

\begin{longtable}[]{@{}ll@{}}
\toprule\noalign{}
Version & Release Date \\
\midrule\noalign{}
\endhead
\bottomrule\noalign{}
\endlastfoot
0.1.0 & April 2, 2024 \\
\end{longtable}

Typst GmbH did not create this template and cannot guarantee correct
functionality of this template or compatibility with any version of the
Typst compiler or app.


\title{typst.app/universe/package/classic-aau-report}

\phantomsection\label{banner}
\phantomsection\label{template-thumbnail}
\pandocbounded{\includegraphics[keepaspectratio]{https://packages.typst.org/preview/thumbnails/classic-aau-report-0.1.0-small.webp}}

\section{classic-aau-report}\label{classic-aau-report}

{ 0.1.0 }

An example package.

\href{/app?template=classic-aau-report&version=0.1.0}{Create project in
app}

\phantomsection\label{readme}
Unofficial Typst template for project reports at Aalborg University
(AAU). This is based on the LaTeX template
\url{https://github.com/jkjaer/aauLatexTemplates} .

The template is generic to any field of study, but defaults to Computer
Science.

\subsection{Usage}\label{usage}

Click “Create project in app�.

Or via the CLI

\begin{Shaded}
\begin{Highlighting}[]
\ExtensionTok{typst}\NormalTok{ init @preview/classic{-}aau{-}report}
\end{Highlighting}
\end{Shaded}

\textbf{NOTE:} The template tries to use the
\texttt{\ Palatino\ Linotype\ } font, which is \emph{not} available in
Typst. It is available
\href{https://github.com/Tinggaard/classic-aau-report/tree/main/fonts}{here}

To use it in the \emph{web-app} , put the \texttt{\ .ttf\ } files
anywhere in the project tree.

To use it \emph{locally} specify the \texttt{\ -\/-font-path\ } flag (or
see the
\href{https://typst.app/docs/reference/text/text/\#parameters-font}{docs}
).

\subsection{Confugiration}\label{confugiration}

The \texttt{\ project\ } function takes the following (optional)
arguments:

\begin{itemize}
\item
  \texttt{\ meta\ } : Metadata about the project

  \begin{itemize}
  \tightlist
  \item
    \texttt{\ project-group\ } : The project group name
  \item
    \texttt{\ participants\ } : A list of participants
  \item
    \texttt{\ supervisors\ } : A list of supervisors
  \item
    \texttt{\ field-of-study\ } : The field of study
  \item
    \texttt{\ project-type\ } : The type of project
  \end{itemize}
\item
  \texttt{\ en\ } : English project info

  \begin{itemize}
  \tightlist
  \item
    \texttt{\ title\ } : The title of the project
  \item
    \texttt{\ theme\ } : The theme of the project
  \item
    \texttt{\ abstract\ } : The English abstract of the project
  \item
    \texttt{\ department\ } : The department name
  \item
    \texttt{\ department-url\ } : The department URL
  \end{itemize}
\item
  \texttt{\ dk\ } : Danish project info

  \begin{itemize}
  \tightlist
  \item
    \texttt{\ title\ } : The Danish title of the project
  \item
    \texttt{\ theme\ } : The theme of the project in Danish
  \item
    \texttt{\ abstract\ } : The Danish abstract of the project
  \item
    \texttt{\ department\ } : The department name in Danish
  \item
    \texttt{\ department-url\ } : The Danish department URL
  \end{itemize}
\end{itemize}

The defaults are as follows:

\begin{Shaded}
\begin{Highlighting}[]
\NormalTok{\#let defaults = (}
\NormalTok{  meta: (}
\NormalTok{    project{-}group: "No group name provided",}
\NormalTok{    participants: (),}
\NormalTok{    supervisors: (),}
\NormalTok{    field{-}of{-}study: "Computer Science",}
\NormalTok{    project{-}type: "Semester Project"}
\NormalTok{  ),}
\NormalTok{  en: (}
\NormalTok{    title: "Untitled",}
\NormalTok{    theme: "",}
\NormalTok{    abstract: [],}
\NormalTok{    department: "Department of Computer Science",}
\NormalTok{    department{-}url: "https://www.cs.aau.dk",}
\NormalTok{  ),}
\NormalTok{  dk: (}
\NormalTok{    title: "Uden titel",}
\NormalTok{    theme: "",}
\NormalTok{    abstract: [],}
\NormalTok{    department: "Institut for Datalogi",}
\NormalTok{    department{-}url: "https://www.dat.aau.dk",}
\NormalTok{  ),}
\NormalTok{)}
\end{Highlighting}
\end{Shaded}

Furthermore, the template exports the shawrules

\begin{itemize}
\tightlist
\item
  \texttt{\ frontmatter\ } : Sets the page numbering to arabic and
  chapter numbering to none
\item
  \texttt{\ mainmatter\ } : Sets the chapter numbering
  \texttt{\ Chapter\ } followed by a number.
\item
  \texttt{\ backmatter\ } : Sets the chapter numbering back to none
\item
  \texttt{\ appendix\ } : Sets the chapter numbering to
  \texttt{\ Appeendix\ } followed by a letter.
\end{itemize}

To use it in an existing project, add the following show rule to the top
of your file.

\begin{Shaded}
\begin{Highlighting}[]
\NormalTok{\#include "@preview/classic{-}aau{-}report:0.1.0": project, frontmatter, mainmatter, backmatter, appendix}

\NormalTok{// Any of the below can be omitted, the defaults are either empty values or CS specific}
\NormalTok{\#show: project.with(}
\NormalTok{  meta: (}
\NormalTok{    project{-}group: "CS{-}xx{-}DAT{-}y{-}zz",}
\NormalTok{    participants: (}
\NormalTok{      "Alice",}
\NormalTok{      "Bob",}
\NormalTok{      "Chad",}
\NormalTok{    ),}
\NormalTok{    supervisors: "John McClane"}
\NormalTok{  ),}
\NormalTok{  en: (}
\NormalTok{    title: "An awesome project",}
\NormalTok{    theme: "Writing a project in Typst",}
\NormalTok{    abstract: [],}
\NormalTok{  ),}
\NormalTok{  dk: (}
\NormalTok{    title: "Et fantastisk projekt",}
\NormalTok{    theme: "Et projekt i Typst",}
\NormalTok{    abstract: [],}
\NormalTok{  ),}
\NormalTok{)}

\NormalTok{// \#show{-}todos()}

\NormalTok{\#show: frontmatter}
\NormalTok{\#include "chapters/introduction.typ"}

\NormalTok{\#show: mainmatter}
\NormalTok{\#include "chapters/problem{-}analysis.typ"}
\NormalTok{\#include "chapters/conclusion.typ"}

\NormalTok{\#show: backmatter}
\NormalTok{\#bibliography("references.bib", title: "References")}

\NormalTok{\#show: appendix}
\NormalTok{\#include "appendices/code{-}snippets.typ"}
\end{Highlighting}
\end{Shaded}

\href{/app?template=classic-aau-report&version=0.1.0}{Create project in
app}

\subsubsection{How to use}\label{how-to-use}

Click the button above to create a new project using this template in
the Typst app.

You can also use the Typst CLI to start a new project on your computer
using this command:

\begin{verbatim}
typst init @preview/classic-aau-report:0.1.0
\end{verbatim}

\includesvg[width=0.16667in,height=0.16667in]{/assets/icons/16-copy.svg}

\subsubsection{About}\label{about}

\begin{description}
\tightlist
\item[Author :]
\href{https://github.com/Tinggaard}{Jens Tinggaard}
\item[License:]
MIT
\item[Current version:]
0.1.0
\item[Last updated:]
November 22, 2024
\item[First released:]
November 22, 2024
\item[Minimum Typst version:]
0.12.0
\item[Archive size:]
149 kB
\href{https://packages.typst.org/preview/classic-aau-report-0.1.0.tar.gz}{\pandocbounded{\includesvg[keepaspectratio]{/assets/icons/16-download.svg}}}
\item[Repository:]
\href{https://github.com/Tinggaard/classic-aau-report}{GitHub}
\item[Categor ies :]
\begin{itemize}
\tightlist
\item[]
\item
  \pandocbounded{\includesvg[keepaspectratio]{/assets/icons/16-speak.svg}}
  \href{https://typst.app/universe/search/?category=report}{Report}
\item
  \pandocbounded{\includesvg[keepaspectratio]{/assets/icons/16-mortarboard.svg}}
  \href{https://typst.app/universe/search/?category=thesis}{Thesis}
\end{itemize}
\end{description}

\subsubsection{Where to report issues?}\label{where-to-report-issues}

This template is a project of Jens Tinggaard . Report issues on
\href{https://github.com/Tinggaard/classic-aau-report}{their repository}
. You can also try to ask for help with this template on the
\href{https://forum.typst.app}{Forum} .

Please report this template to the Typst team using the
\href{https://typst.app/contact}{contact form} if you believe it is a
safety hazard or infringes upon your rights.

\phantomsection\label{versions}
\subsubsection{Version history}\label{version-history}

\begin{longtable}[]{@{}ll@{}}
\toprule\noalign{}
Version & Release Date \\
\midrule\noalign{}
\endhead
\bottomrule\noalign{}
\endlastfoot
0.1.0 & November 22, 2024 \\
\end{longtable}

Typst GmbH did not create this template and cannot guarantee correct
functionality of this template or compatibility with any version of the
Typst compiler or app.


\title{typst.app/universe/package/codetastic}

\phantomsection\label{banner}
\section{codetastic}\label{codetastic}

{ 0.2.2 }

Generate all sorts of codes in Typst.

\phantomsection\label{readme}
\textbf{Codetastic} is a \href{https://github.com/typst/typst}{Typst}
package for drawing barcodes and 2d codes.

\subsection{Usage}\label{usage}

For Typst 0.6.0 or later, import the package from the Typst preview
repository:

\begin{Shaded}
\begin{Highlighting}[]
\NormalTok{\#import "@preview/codetastic:0.2.2"}
\end{Highlighting}
\end{Shaded}

After importing the package call any of the code generation functions:

\begin{Shaded}
\begin{Highlighting}[]
\NormalTok{\#import "@preview/codetastic:0.2.2": ean13, qrcode}

\NormalTok{\#ean13(4012345678901)}

\NormalTok{\#qrcode("https://github.com/typst/typst")}
\end{Highlighting}
\end{Shaded}

The output should look like this:
\pandocbounded{\includegraphics[keepaspectratio]{https://github.com/typst/packages/raw/main/packages/preview/codetastic/0.2.2/assets/example.png}}

\subsection{Further documentation}\label{further-documentation}

See \texttt{\ manual.pdf\ } for a full manual of the package.

\subsection{Development}\label{development}

The documentation is created using
\href{https://github.com/jneug/typst-mantys}{Mantys} , a Typst template
for creating package documentation.

To compile the manual, Mantys needs to be available as a local package.
Refer to Mantys’ manual for instructions on how to do so.

\subsection{Changelog}\label{changelog}

\subsubsection{Version 0.2.2}\label{version-0.2.2}

\begin{itemize}
\tightlist
\item
  qrcodes:

  \begin{itemize}
  \tightlist
  \item
    Fixed issue with alignment pattern placement.
  \item
    Removed minimal borders around modules for sharper edges with small
    module sizes.
  \end{itemize}
\end{itemize}

\subsubsection{Version 0.2.1}\label{version-0.2.1}

\begin{itemize}
\tightlist
\item
  qrcodes:

  \begin{itemize}
  \tightlist
  \item
    Fixed wrong sizing for \texttt{\ width\ } key.

    \begin{itemize}
    \tightlist
    \item
      The code didn’t take the quiet zone into account.
    \end{itemize}
  \item
    Moved debug information into quiet zone.
  \end{itemize}
\end{itemize}

\subsubsection{Version 0.2.0}\label{version-0.2.0}

\begin{itemize}
\tightlist
\item
  Removed CeTZ as a dependecy.

  \begin{itemize}
  \tightlist
  \item
    Now using native Typst drawing functions.
  \end{itemize}
\item
  Hugh speed improvements for large QR-Codes.
\item
  Fixed issue with checksum calculation for gtin/ean codes.
\end{itemize}

\subsubsection{Version 0.1.0}\label{version-0.1.0}

\begin{itemize}
\tightlist
\item
  Initial release submitted to
  \href{https://github.com/typst/packages}{typst/packages} .
\end{itemize}

\subsubsection{How to add}\label{how-to-add}

Copy this into your project and use the import as
\texttt{\ codetastic\ }

\begin{verbatim}
#import "@preview/codetastic:0.2.2"
\end{verbatim}

\includesvg[width=0.16667in,height=0.16667in]{/assets/icons/16-copy.svg}

Check the docs for
\href{https://typst.app/docs/reference/scripting/\#packages}{more
information on how to import packages} .

\subsubsection{About}\label{about}

\begin{description}
\tightlist
\item[Author :]
J. Neugebauer
\item[License:]
MIT
\item[Current version:]
0.2.2
\item[Last updated:]
September 23, 2023
\item[First released:]
September 12, 2023
\item[Archive size:]
22.9 kB
\href{https://packages.typst.org/preview/codetastic-0.2.2.tar.gz}{\pandocbounded{\includesvg[keepaspectratio]{/assets/icons/16-download.svg}}}
\item[Repository:]
\href{https://github.com/jneug/typst-codetastic}{GitHub}
\end{description}

\subsubsection{Where to report issues?}\label{where-to-report-issues}

This package is a project of J. Neugebauer . Report issues on
\href{https://github.com/jneug/typst-codetastic}{their repository} . You
can also try to ask for help with this package on the
\href{https://forum.typst.app}{Forum} .

Please report this package to the Typst team using the
\href{https://typst.app/contact}{contact form} if you believe it is a
safety hazard or infringes upon your rights.

\phantomsection\label{versions}
\subsubsection{Version history}\label{version-history}

\begin{longtable}[]{@{}ll@{}}
\toprule\noalign{}
Version & Release Date \\
\midrule\noalign{}
\endhead
\bottomrule\noalign{}
\endlastfoot
0.2.2 & September 23, 2023 \\
\href{https://typst.app/universe/package/codetastic/0.2.0/}{0.2.0} &
September 19, 2023 \\
\href{https://typst.app/universe/package/codetastic/0.1.0/}{0.1.0} &
September 12, 2023 \\
\end{longtable}

Typst GmbH did not create this package and cannot guarantee correct
functionality of this package or compatibility with any version of the
Typst compiler or app.


\title{typst.app/universe/package/iridis}

\phantomsection\label{banner}
\section{iridis}\label{iridis}

{ 0.1.0 }

A package to colors matching parenthesis

\phantomsection\label{readme}
Iridis is a package to colorize parenthesis in your typst document.
Iridis is a latin word that means “rainbow�. This package is
inspired by the many rainbow parenthesis plugins available for code
editors.

\subsection{Usage}\label{usage}

The package provides a single show-rule \texttt{\ iridis-show\ } that
can be used to colorize parenthesis in your document and a palette
\texttt{\ iridis-palette\ } that can be used to define the colors used.

The rule takes 3 arguments:

\begin{itemize}
\tightlist
\item
  \texttt{\ opening-parenthesis\ } : The opening parenthesis character.
  Default is \texttt{\ ("(",\ "{[}",\ "\{")\ } .
\item
  \texttt{\ closing-parenthesis\ } : The closing parenthesis character.
  Default is \texttt{\ (")",\ "{]}",\ "\}")\ } .
\item
  \texttt{\ palette\ } : The color palette to use. Default is
  \texttt{\ iridis-palette\ } .
\end{itemize}

If the symbols are single characters, they are interpreted as normal
strings but if you use multi-character strings, then they are
interpreted as regular expressions.

\subsection{Exemples}\label{exemples}

\begin{Shaded}
\begin{Highlighting}[]
\NormalTok{\#show: iridis.iridis{-}show}

\NormalTok{\textasciigrave{}\textasciigrave{}\textasciigrave{}rs}
\NormalTok{fn main() \{}
\NormalTok{    let n = false;}
\NormalTok{    if n \{}
\NormalTok{        println!("Hello, world!");}
\NormalTok{    \} else \{}
\NormalTok{        println!("Goodbye, world!");}
\NormalTok{    \}}
\NormalTok{\}}
\NormalTok{\textasciigrave{}\textasciigrave{}\textasciigrave{}}

\NormalTok{\textasciigrave{}\textasciigrave{}\textasciigrave{}cpp}
\NormalTok{\#include \textless{}iostream\textgreater{}}

\NormalTok{int main() \{}
\NormalTok{    bool n = false;}
\NormalTok{    if (n) \{}
\NormalTok{        std::cout \textless{}\textless{} "Hello, world!" \textless{}\textless{} std::endl;}
\NormalTok{    \} else \{}
\NormalTok{        std::cout \textless{}\textless{} "Goodbye, world!" \textless{}\textless{} std::endl;}
\NormalTok{    \}}
\NormalTok{\}}
\NormalTok{\textasciigrave{}\textasciigrave{}\textasciigrave{}}

\NormalTok{\textasciigrave{}\textasciigrave{}\textasciigrave{}py}
\NormalTok{if \_\_name\_\_ == "\_\_main\_\_":}
\NormalTok{    n = False}
\NormalTok{    if n:}
\NormalTok{        print("Hello, world!")}
\NormalTok{    else:}
\NormalTok{        print("Goodbye, world!")}
\NormalTok{\textasciigrave{}\textasciigrave{}\textasciigrave{}}
\end{Highlighting}
\end{Shaded}

\pandocbounded{\includegraphics[keepaspectratio]{https://raw.githubusercontent.com/Robotechnic/iridis/master/images/code1.png}}

\begin{Shaded}
\begin{Highlighting}[]
\NormalTok{\#show: iridis.iridis{-}show}

\NormalTok{$}
\NormalTok{    "plus" equiv lambda m. f lambda n. lambda f. lambda x. m f (n f x) \textbackslash{}}
\NormalTok{    "succ" equiv lambda n. lambda f. lambda x. f (n f x) \textbackslash{}}
\NormalTok{    "mult" equiv lambda m. lambda n. lambda f. lambda x. m (n f) x \textbackslash{}}
\NormalTok{    "pred" equiv lambda n. lambda f. lambda x. n (lambda g. lambda h. h (g f)) (lambda u. x) (lambda u. u) \textbackslash{}}
\NormalTok{    "zero" equiv lambda f. lambda x. x \textbackslash{}}
\NormalTok{    "one" equiv lambda f. lambda x. f x \textbackslash{}}
\NormalTok{    "two" equiv lambda f. lambda x. f (f x) \textbackslash{}}
\NormalTok{    "three" equiv lambda f. lambda x. f (f (f x)) \textbackslash{}}
\NormalTok{    "four" equiv lambda f. lambda x. f (f (f (f x))) \textbackslash{}}
\NormalTok{$}

\NormalTok{$}
\NormalTok{    (((1 + 5) * 7) / (5 {-} 8 * 7) + 3) * 2 approx 4.352941176}
\NormalTok{$}

\NormalTok{$ mat(}
\NormalTok{  1, 2, ..., (10 / 2);}
\NormalTok{  2, 2, ..., 10;}
\NormalTok{  dots.v, dots.v, dots.down, dots.v;}
\NormalTok{  10, (10 {-} (5 * 8)) / 2, ..., 10;}
\NormalTok{) $}
\end{Highlighting}
\end{Shaded}

\pandocbounded{\includegraphics[keepaspectratio]{https://raw.githubusercontent.com/Robotechnic/iridis/master/images/math1.png}}

\subsection{Changelog}\label{changelog}

\subsubsection{0.1.0}\label{section}

\begin{itemize}
\tightlist
\item
  Initial release
\end{itemize}

\subsubsection{How to add}\label{how-to-add}

Copy this into your project and use the import as \texttt{\ iridis\ }

\begin{verbatim}
#import "@preview/iridis:0.1.0"
\end{verbatim}

\includesvg[width=0.16667in,height=0.16667in]{/assets/icons/16-copy.svg}

Check the docs for
\href{https://typst.app/docs/reference/scripting/\#packages}{more
information on how to import packages} .

\subsubsection{About}\label{about}

\begin{description}
\tightlist
\item[Author :]
\href{https://github.com/Robotechnic}{Robotechnic}
\item[License:]
MIT
\item[Current version:]
0.1.0
\item[Last updated:]
June 24, 2024
\item[First released:]
June 24, 2024
\item[Minimum Typst version:]
0.11.0
\item[Archive size:]
3.17 kB
\href{https://packages.typst.org/preview/iridis-0.1.0.tar.gz}{\pandocbounded{\includesvg[keepaspectratio]{/assets/icons/16-download.svg}}}
\end{description}

\subsubsection{Where to report issues?}\label{where-to-report-issues}

This package is a project of Robotechnic . You can also try to ask for
help with this package on the \href{https://forum.typst.app}{Forum} .

Please report this package to the Typst team using the
\href{https://typst.app/contact}{contact form} if you believe it is a
safety hazard or infringes upon your rights.

\phantomsection\label{versions}
\subsubsection{Version history}\label{version-history}

\begin{longtable}[]{@{}ll@{}}
\toprule\noalign{}
Version & Release Date \\
\midrule\noalign{}
\endhead
\bottomrule\noalign{}
\endlastfoot
0.1.0 & June 24, 2024 \\
\end{longtable}

Typst GmbH did not create this package and cannot guarantee correct
functionality of this package or compatibility with any version of the
Typst compiler or app.


\title{typst.app/universe/package/use-academicons}

\phantomsection\label{banner}
\section{use-academicons}\label{use-academicons}

{ 0.1.0 }

A Typst library for Academicons the desktop fonts.

\phantomsection\label{readme}
A Typst library for Academicons through the desktop fonts.

This is based on the code from \texttt{\ duskmoon314\ } and the package
for
\href{https://github.com/duskmoon314/typst-fontawesome}{\textbf{typst-fontawesome}}
.

p.s. The library is based on the Academicons desktop fonts (v1.9.4)

\subsection{Usage}\label{usage}

\subsubsection{Install the fonts}\label{install-the-fonts}

You can download the fonts from the
\href{https://jpswalsh.github.io/academicons/}{official website}

After downloading the zip file, you can install the fonts depending on
your OS.

\paragraph{Typst web app}\label{typst-web-app}

You can simply upload the \texttt{\ ttf\ } files to the web app and use
them with this package.

\paragraph{Mac}\label{mac}

You can double click the \texttt{\ ttf\ } files to install them.

\paragraph{Windows}\label{windows}

You can right-click the \texttt{\ ttf\ } files and select
\texttt{\ Install\ } .

\subsubsection{Import the library}\label{import-the-library}

\paragraph{Using the typst packages}\label{using-the-typst-packages}

You can install the library using the typst packages:

\texttt{\ \#import\ "@preview/use-academicons:0.1.0":\ *\ }

\paragraph{Manually install}\label{manually-install}

Copy all files start with \texttt{\ lib\ } to your project and import
the library:

\texttt{\ \#import\ "lib.typ":\ *\ }

There are three files:

\begin{itemize}
\tightlist
\item
  \texttt{\ lib.typ\ } : The main entrypoint of the library.
\item
  \texttt{\ lib-impl.typ\ } : The implementation of \texttt{\ ai-icon\ }
  .
\item
  \texttt{\ lib-gen.typ\ } : The generated icon map and functions.
\end{itemize}

I recommend renaming these files to avoid conflicts with other
libraries.

\subsubsection{Use the icons}\label{use-the-icons}

You can use the \texttt{\ ai-icon\ } function to create an icon with its
name:

\texttt{\ \#ai-icon("lattes")\ }

Or you can use the \texttt{\ ai-\ } prefix to create an icon with its
name:

\texttt{\ \#ai-lattes()\ } (This is equivalent to
\texttt{\ \#ai-icon().with("lattes")\ } )

\paragraph{Full list of icons}\label{full-list-of-icons}

You can find all icons on the
\href{https://jpswalsh.github.io/academicons/}{official website}

\paragraph{Customization}\label{customization}

The \texttt{\ ai-icon\ } function passes args to \texttt{\ text\ } , so
you can customize the icon by passing parameters to it:

\texttt{\ \#ai-icon("lattes",\ fill:\ blue)\ }

\paragraph{Stacking icons}\label{stacking-icons}

The \texttt{\ ai-stack\ } function can be used to create stacked icons:

\texttt{\ \#ai-stack(ai-icon-args:\ (fill:\ black),\ "doi",\ ("cv",\ (fill:\ blue,\ size:\ 20pt)))\ }

Declaration is
\texttt{\ ai-stack(box-args:\ (:),\ grid-args:\ (:),\ ai-icon-args:\ (:),\ ..icons)\ }

\begin{itemize}
\tightlist
\item
  The order of the icons is from the bottom to the top.
\item
  \texttt{\ ai-icon-args\ } is used to set the default args for all
  icons.
\item
  You can also control the internal \texttt{\ box\ } and
  \texttt{\ grid\ } by passing the \texttt{\ box-args\ } and
  \texttt{\ grid-args\ } to the \texttt{\ ai-stack\ } function.
\item
  Currently, four types of icons are supported. The first three types
  leverage the \texttt{\ ai-icon\ } function, and the last type is just
  a content you want to put in the stack.

  \begin{itemize}
  \tightlist
  \item
    \texttt{\ str\ } , e.g., \texttt{\ "lattes"\ }
  \item
    \texttt{\ array\ } , e.g.,
    \texttt{\ ("lattes",\ (fill:\ white,\ size:\ 5.5pt))\ }
  \item
    \texttt{\ arguments\ } , e.g.
    \texttt{\ arguments("lattes",\ fill:\ white)\ }
  \item
    \texttt{\ content\ } , e.g. \texttt{\ ai-lattes(fill:\ white)\ }
  \end{itemize}
\end{itemize}

\subsection{Example}\label{example}

See the
\href{https://typst.app/project/rsgOFC4YkwpN7OqtRyiXP3}{\texttt{\ use-academicons.typ\ }}
file for a complete example.

\subsection{Contribution}\label{contribution}

Feel free to open an issue or a pull request if you find any problems or
have any suggestions.

\subsubsection{R helper}\label{r-helper}

The \texttt{\ helper.R\ } script is used to get unicodes for icons and
generate typst code.

\subsubsection{Repo structure}\label{repo-structure}

\begin{itemize}
\tightlist
\item
  \texttt{\ helper.R\ } : The helper script to get unicodes and generate
  typst code.
\item
  \texttt{\ lib.typ\ } : The main entrypoint of the library.
\item
  \texttt{\ lib-impl.typ\ } : The implementation of \texttt{\ ai-icon\ }
  .
\item
  \texttt{\ lib-gen.typ\ } : The generated functions of icons.
\item
  \texttt{\ example.typ\ } : An example file to show how to use the
  library.
\item
  \texttt{\ gallery.typ\ } : The generated gallery of icons. It is used
  in the example file.
\end{itemize}

\subsection{License}\label{license}

This library is licensed under the MIT license. Feel free to use it in
your project.

\subsubsection{How to add}\label{how-to-add}

Copy this into your project and use the import as
\texttt{\ use-academicons\ }

\begin{verbatim}
#import "@preview/use-academicons:0.1.0"
\end{verbatim}

\includesvg[width=0.16667in,height=0.16667in]{/assets/icons/16-copy.svg}

Check the docs for
\href{https://typst.app/docs/reference/scripting/\#packages}{more
information on how to import packages} .

\subsubsection{About}\label{about}

\begin{description}
\tightlist
\item[Author s :]
\href{mailto:kp.campbell.he@duskmoon314.com}{duskmoon (Campbell He)} \&
\href{mailto:philipp.kleer@posteo.com}{bpkleer (Philipp Kleer)}
\item[License:]
MIT
\item[Current version:]
0.1.0
\item[Last updated:]
August 8, 2024
\item[First released:]
August 8, 2024
\item[Archive size:]
5.41 kB
\href{https://packages.typst.org/preview/use-academicons-0.1.0.tar.gz}{\pandocbounded{\includesvg[keepaspectratio]{/assets/icons/16-download.svg}}}
\item[Repository:]
\href{https://github.com/bpkleer/typst-academicons}{GitHub}
\end{description}

\subsubsection{Where to report issues?}\label{where-to-report-issues}

This package is a project of duskmoon (Campbell He) and bpkleer (Philipp
Kleer) . Report issues on
\href{https://github.com/bpkleer/typst-academicons}{their repository} .
You can also try to ask for help with this package on the
\href{https://forum.typst.app}{Forum} .

Please report this package to the Typst team using the
\href{https://typst.app/contact}{contact form} if you believe it is a
safety hazard or infringes upon your rights.

\phantomsection\label{versions}
\subsubsection{Version history}\label{version-history}

\begin{longtable}[]{@{}ll@{}}
\toprule\noalign{}
Version & Release Date \\
\midrule\noalign{}
\endhead
\bottomrule\noalign{}
\endlastfoot
0.1.0 & August 8, 2024 \\
\end{longtable}

Typst GmbH did not create this package and cannot guarantee correct
functionality of this package or compatibility with any version of the
Typst compiler or app.


\title{typst.app/universe/package/gviz}

\phantomsection\label{banner}
\section{gviz}\label{gviz}

{ 0.1.0 }

Generate graphs using the graphviz dot language.

\phantomsection\label{readme}
GViz is a typst plugin that can render graphviz graphs.

It uses \url{https://codeberg.org/Sekoia/layout} as a backend, which
means it can currently only render to SVG, and mostly supports basic
features.

Import it like any other plugin:
\texttt{\ \#import\ "@preview/gviz:0.1.0":\ *\ } .

\subsection{Usage}\label{usage}

\begin{Shaded}
\begin{Highlighting}[]
\NormalTok{\#import "@preview/gviz:0.1.0": *}

\NormalTok{\#show raw.where(lang: "dot{-}render"): it =\textgreater{} render{-}image(it.text)}

\NormalTok{\textasciigrave{}\textasciigrave{}\textasciigrave{}dot{-}render}
\NormalTok{digraph mygraph \{}
\NormalTok{  node [shape=box];}
\NormalTok{  A {-}\textgreater{} B;}
\NormalTok{  B {-}\textgreater{} C;}
\NormalTok{  B {-}\textgreater{} D;}
\NormalTok{  C {-}\textgreater{} E;}
\NormalTok{  D {-}\textgreater{} E;}
\NormalTok{  E {-}\textgreater{} F;}
\NormalTok{  A {-}\textgreater{} F [label="one"];}
\NormalTok{  A {-}\textgreater{} F [label="two"];}
\NormalTok{  A {-}\textgreater{} F [label="three"];}
\NormalTok{  A {-}\textgreater{} F [label="four"];}
\NormalTok{  A {-}\textgreater{} F [label="five"];}
\NormalTok{\}\textasciigrave{}\textasciigrave{}\textasciigrave{}}

\NormalTok{\#let my{-}graph = "digraph \{A {-}\textgreater{} B\}"}
\NormalTok{\#render{-}image(my{-}graph)}

\NormalTok{SVG:}
\NormalTok{\#raw(render(my{-}graph), block: true, lang: "svg")}
\end{Highlighting}
\end{Shaded}

\subsection{API}\label{api}

\subsubsection{render}\label{render}

Renders a graph in dot language and returns SVG code for it.

Parameters:

\begin{itemize}
\tightlist
\item
  code (string, bytes): Dot language code to be rendered.
\end{itemize}

Returns: string

\subsubsection{render-image}\label{render-image}

Renders a graph in dot language and returns an SVG image of it. Uses the
same parameters as image.decode.

Parameters:

\begin{itemize}
\tightlist
\item
  code (string, bytes): Dot language code to be rendered.
\item
  width (auto, relative): The width of the image.
\item
  height (auto, relative): The height of the image.
\item
  alt (none, string): A text describing the image.
\item
  fit (string): How the image should adjust itself to a given area. See
  image.decode.
\end{itemize}

Returns: content

\subsubsection{How to add}\label{how-to-add}

Copy this into your project and use the import as \texttt{\ gviz\ }

\begin{verbatim}
#import "@preview/gviz:0.1.0"
\end{verbatim}

\includesvg[width=0.16667in,height=0.16667in]{/assets/icons/16-copy.svg}

Check the docs for
\href{https://typst.app/docs/reference/scripting/\#packages}{more
information on how to import packages} .

\subsubsection{About}\label{about}

\begin{description}
\tightlist
\item[Author :]
\href{https://codeberg.org/Sekoia\%3E\%20\%3Chttps://github.com/SekoiaTree}{Sekoia}
\item[License:]
Unlicense
\item[Current version:]
0.1.0
\item[Last updated:]
September 15, 2023
\item[First released:]
September 15, 2023
\item[Minimum Typst version:]
0.8.0
\item[Archive size:]
85.7 kB
\href{https://packages.typst.org/preview/gviz-0.1.0.tar.gz}{\pandocbounded{\includesvg[keepaspectratio]{/assets/icons/16-download.svg}}}
\item[Repository:]
\href{https://codeberg.org/Sekoia/gviz-typst}{Codeberg}
\end{description}

\subsubsection{Where to report issues?}\label{where-to-report-issues}

This package is a project of Sekoia . Report issues on
\href{https://codeberg.org/Sekoia/gviz-typst}{their repository} . You
can also try to ask for help with this package on the
\href{https://forum.typst.app}{Forum} .

Please report this package to the Typst team using the
\href{https://typst.app/contact}{contact form} if you believe it is a
safety hazard or infringes upon your rights.

\phantomsection\label{versions}
\subsubsection{Version history}\label{version-history}

\begin{longtable}[]{@{}ll@{}}
\toprule\noalign{}
Version & Release Date \\
\midrule\noalign{}
\endhead
\bottomrule\noalign{}
\endlastfoot
0.1.0 & September 15, 2023 \\
\end{longtable}

Typst GmbH did not create this package and cannot guarantee correct
functionality of this package or compatibility with any version of the
Typst compiler or app.


\title{typst.app/universe/package/cmarker}

\phantomsection\label{banner}
\section{cmarker}\label{cmarker}

{ 0.1.1 }

Transpile CommonMark Markdown to Typst, from within Typst!

\phantomsection\label{readme}
\#set document(title: "cmarker.typ") \#set page(numbering: "1",
number-align: center) \#set text(lang: "en") \#align(center,
text(weight: 700, 1.75em){[}cmarker.typ{]}) \#set heading(numbering:
"1.") \#show link: c =\textgreater{} text(underline(c), fill: blue)
\#set image(height: 2em) \#show par: set block(above: 1.2em, below:
1.2em) \#align(center){[}https://github.com/SabrinaJewson/cmarker.typ{]}
\#"

This package enables you to write CommonMark Markdown, and import it
directly into Typst.

\begin{longtable}[]{@{}
  >{\raggedright\arraybackslash}p{(\linewidth - 2\tabcolsep) * \real{0.5000}}
  >{\raggedright\arraybackslash}p{(\linewidth - 2\tabcolsep) * \real{0.5000}}@{}}
\toprule\noalign{}
\endhead
\bottomrule\noalign{}
\endlastfoot
\texttt{\ simple.typ\ } & \texttt{\ simple.md\ } \\
\begin{minipage}[t]{\linewidth}\raggedright
\begin{Shaded}
\begin{Highlighting}[]
\NormalTok{\#import "@preview/cmarker:0.1.1"}

\NormalTok{\#cmarker.render(read("simple.md"))}
\end{Highlighting}
\end{Shaded}
\end{minipage} & \begin{minipage}[t]{\linewidth}\raggedright
\begin{Shaded}
\begin{Highlighting}[]
\FunctionTok{\# We can write Markdown!}

\NormalTok{*Using* \_\_lots\_\_ \textasciitilde{}of\textasciitilde{} }\InformationTok{\textasciigrave{}fancy\textasciigrave{}} \CommentTok{[}\OtherTok{features}\CommentTok{](https://example.org/)}\NormalTok{.}
\end{Highlighting}
\end{Shaded}
\end{minipage} \\
\end{longtable}

\begin{longtable}[]{@{}l@{}}
\toprule\noalign{}
\endhead
\bottomrule\noalign{}
\endlastfoot
\texttt{\ simple.pdf\ } \\
\pandocbounded{\includegraphics[keepaspectratio]{https://github.com/typst/packages/raw/main/packages/preview/cmarker/0.1.1/examples/simple.png}} \\
\end{longtable}

This document is available in
\href{https://github.com/SabrinaJewson/cmarker.typ/tree/main\#cmarker}{Markdown}
and
\href{https://github.com/SabrinaJewson/cmarker.typ/blob/main/README.pdf}{rendered
PDF} formats.

\subsection{API}\label{api}

We offer a single function:

\begin{Shaded}
\begin{Highlighting}[]
\NormalTok{render(}
\NormalTok{  markdown,}
\NormalTok{  smart{-}punctuation: true,}
\NormalTok{  blockquote: none,}
\NormalTok{  math: none,}
\NormalTok{  h1{-}level: 1,}
\NormalTok{  raw{-}typst: true,}
\NormalTok{  scope: (:),}
\NormalTok{  show{-}source: false,}
\NormalTok{) {-}\textgreater{} content}
\end{Highlighting}
\end{Shaded}

The parameters are as follows:

\begin{itemize}
\item
  \texttt{\ markdown\ } : The
  \href{https://spec.commonmark.org/0.30/}{CommonMark} Markdown string
  to be processed. Parsed with the
  \href{https://docs.rs/pulldown-cmark}{pulldown-cmark} Rust library.
  You can set this to \texttt{\ read("somefile.md")\ } to import an
  external markdown file; see the
  \href{https://typst.app/docs/reference/data-loading/read/}{documentation
  for the read function} .

  \begin{itemize}
  \tightlist
  \item
    Accepted values: Strings and raw text code blocks.
  \item
    Required.
  \end{itemize}
\item
  \texttt{\ smart-punctuation\ } : Automatically convert ASCII
  punctuation to Unicode equivalents:

  \begin{itemize}
  \tightlist
  \item
    nondirectional quotations (" and \textquotesingle) become
    directional (“� and ‘’);
  \item
    three consecutive full stops (…) become ellipses (…);
  \item
    two and three consecutive hypen-minus signs (-\/- and â€'') become
    en and em dashes (â€`` and â€'').
  \end{itemize}

  Note that although Typst also offers this functionality, this
  conversion is done through the Markdown parser rather than Typst.

  \begin{itemize}
  \tightlist
  \item
    Accepted values: Booleans.
  \item
    Default value: \texttt{\ true\ } .
  \end{itemize}
\item
  \texttt{\ blockquote\ } : A callback to be used when a blockquote is
  encountered in the Markdown, or \texttt{\ none\ } if blockquotes
  should be treated as normal text. Because Typst does not support
  blockquotes natively, the user must configure this.

  \begin{itemize}
  \tightlist
  \item
    Accepted values: Functions accepting content and returning content,
    or \texttt{\ none\ } .
  \item
    Default value: \texttt{\ none\ } .
  \end{itemize}

  For example, to display a black border to the left of the text one can
  use:

\begin{Shaded}
\begin{Highlighting}[]
\NormalTok{box.with(stroke: (left: 1pt + black), inset: (left: 5pt, y: 6pt))}
\end{Highlighting}
\end{Shaded}
\item
  \texttt{\ math\ } : A callback to be used when equations are
  encountered in the Markdown, or \texttt{\ none\ } if it should be
  treated as normal text. Because Typst does not support LaTeX equations
  natively, the user must configure this.

  \begin{itemize}
  \tightlist
  \item
    Accepted values: Functions that take a boolean argument named
    \texttt{\ block\ } and a positional string argument (often, the
    \texttt{\ mitex\ } function from
    \href{https://typst.app/universe/package/mitex}{the mitex package}
    ), or \texttt{\ none\ } .
  \item
    Default value: \texttt{\ none\ } .
  \end{itemize}

  For example, to render math equation as a Typst math block, one can
  use:

\begin{Shaded}
\begin{Highlighting}[]
\NormalTok{\#import "@preview/mitex:0.2.4": mitex}
\NormalTok{\#cmarker.render(\textasciigrave{}$\textbackslash{}int\_1\^{}2 x \textbackslash{}mathrm\{d\} x$\textasciigrave{}, math: mitex)}
\end{Highlighting}
\end{Shaded}
\item
  \texttt{\ h1-level\ } : The level that top-level headings in Markdown
  should get in Typst. When set to zero, top-level headings are treated
  as text, \texttt{\ \#\#\ } headings become \texttt{\ =\ } headings,
  \texttt{\ \#\#\#\ } headings become \texttt{\ ==\ } headings, et
  cetera; when set to \texttt{\ 2\ } , \texttt{\ \#\ } headings become
  \texttt{\ ==\ } headings, \texttt{\ \#\#\ } headings become
  \texttt{\ ===\ } headings, et cetera.

  \begin{itemize}
  \tightlist
  \item
    Accepted values: Integers in the range {[}0, 255{]}.
  \item
    Default value: 1.
  \end{itemize}
\item
  \texttt{\ raw-typst\ } : Whether to allow raw Typst code to be
  injected into the document via HTML comments. If disabled, the
  comments will act as regular HTML comments.

  \begin{itemize}
  \tightlist
  \item
    Accepted values: Booleans.
  \item
    Default value: \texttt{\ true\ } .
  \end{itemize}

  For example, when this is enabled,
  \texttt{\ \textless{}!-\/-raw-typst\ \#circle(radius:\ 10pt)\ -\/-\textgreater{}\ }
  will result in a circle in the document (but only when rendered
  through Typst). See also
  \texttt{\ \textless{}!-\/-typst-begin-exclude-\/-\textgreater{}\ } and
  \texttt{\ \textless{}!-\/-typst-end-exclude-\/-\textgreater{}\ } ,
  which is the inverse of this.
\item
  \texttt{\ scope\ } : When \texttt{\ raw-typst\ } is enabled, this is a
  dictionary providing the context in which the evaluated Typst code
  runs. It is useful to pass values in to code inside
  \texttt{\ \textless{}!-\/-raw-typst-\/-\textgreater{}\ } blocks.

  \begin{itemize}
  \tightlist
  \item
    Accepted values: Any dictionary.
  \item
    Default value: \texttt{\ (:)\ } .
  \end{itemize}
\item
  \texttt{\ show-source\ } : A debugging tool. When set to
  \texttt{\ true\ } , the Typst code that would otherwise have been
  displayed will be instead rendered in a code block.

  \begin{itemize}
  \tightlist
  \item
    Accepted values: Booleans.
  \item
    Default value: \texttt{\ false\ } .
  \end{itemize}
\end{itemize}

This function returns the rendered \texttt{\ content\ } .

\subsection{Supported Markdown Syntax}\label{supported-markdown-syntax}

We support CommonMark with a couple extensions.

\begin{itemize}
\item
  Paragraph breaks: Two newlines, i.e. one blank line.
\item
  Hard line breaks (used more in poetry than prose): Put two spaces at
  the end of the line.
\item
  \texttt{\ *emphasis*\ } or \texttt{\ \_emphasis\_\ } : \emph{emphasis}
\item
  \texttt{\ **strong**\ } or \texttt{\ \_\_strong\_\_\ } :
  \textbf{strong}
\item
  \texttt{\ \textasciitilde{}strikethrough\textasciitilde{}\ } :
  \textasciitilde strikethrough\textasciitilde{}
\item
  \texttt{\ {[}links{]}(https://example.org)\ } :
  \href{https://example.org/}{links}
\item
  \texttt{\ \#\#\#\ Headings\ } , where \texttt{\ \#\ } is a top-level
  heading, \texttt{\ \#\#\ } a subheading, \texttt{\ \#\#\#\ } a
  sub-subheading, etc
\item
  \texttt{\ \textasciigrave{}inline\ code\ blocks\textasciigrave{}\ } :
  \texttt{\ inline\ code\ blocks\ }
\item
\begin{verbatim}
```
out of line code blocks
```
\end{verbatim}

  Syntax highlighting can be achieved by specifying a language after the
  opening backticks:

\begin{verbatim}
```rust
let x = 5;
```
\end{verbatim}

  giving:

\begin{Shaded}
\begin{Highlighting}[]
\KeywordTok{let}\NormalTok{ x }\OperatorTok{=} \DecValTok{5}\OperatorTok{;}
\end{Highlighting}
\end{Shaded}
\item
  \texttt{\ -\/-\/-\ } , making a horizontal rule:
\end{itemize}

\begin{center}\rule{0.5\linewidth}{0.5pt}\end{center}

\begin{itemize}
\item
\begin{Shaded}
\begin{Highlighting}[]
\SpecialStringTok{{-} }\NormalTok{Unordered}
\SpecialStringTok{{-} }\NormalTok{lists}
\end{Highlighting}
\end{Shaded}

  \begin{itemize}
  \tightlist
  \item
    Unordered
  \item
    Lists
  \end{itemize}
\item
\begin{Shaded}
\begin{Highlighting}[]
\SpecialStringTok{1. }\NormalTok{Ordered}
\SpecialStringTok{1. }\NormalTok{Lists}
\end{Highlighting}
\end{Shaded}

  \begin{enumerate}
  \tightlist
  \item
    Ordered
  \item
    Lists
  \end{enumerate}
\item
  \texttt{\ \$x\ +\ y\$\ } or \texttt{\ \$\$x\ +\ y\$\$\ } : math
  equations, if the \texttt{\ math\ } parameter is set.
\item
  \texttt{\ \textgreater{}\ blockquotes\ } , if the
  \texttt{\ blockquote\ } parameter is set.
\item
  Images:
  \texttt{\ !{[}Some\ tiled\ hexagons{]}(examples/hexagons.png)\ } ,
  giving
  \pandocbounded{\includegraphics[keepaspectratio]{https://github.com/typst/packages/raw/main/packages/preview/cmarker/0.1.1/examples/hexagons.png}}
\end{itemize}

\subsection{Interleaving Markdown and
Typst}\label{interleaving-markdown-and-typst}

Sometimes, you might want to render a certain section of the document
only when viewed as Markdown, or only when viewed through Typst. To
achieve the former, you can simply wrap the section in
\texttt{\ \textless{}!-\/-typst-begin-exclude-\/-\textgreater{}\ } and
\texttt{\ \textless{}!-\/-typst-end-exclude-\/-\textgreater{}\ } :

\begin{Shaded}
\begin{Highlighting}[]
\NormalTok{Hello from not Typst!}
\end{Highlighting}
\end{Shaded}

Most Markdown parsers support HTML comments, so from their perspective
this is no different to just writing out the Markdown directly; but
\texttt{\ cmarker.typ\ } knows to search for those comments and avoid
rendering the content in between.

Note that when the opening comment is followed by the end of an element,
\texttt{\ cmarker.typ\ } will close the block for you. For example:

\begin{Shaded}
\begin{Highlighting}[]
\AttributeTok{\textgreater{} }
\AttributeTok{\textgreater{} One}

\NormalTok{Two}
\end{Highlighting}
\end{Shaded}

In this code, “Two� will be given no matter where the document is
rendered. This is done to prevent us from generating invalid Typst code.

Conversely, one can put Typst code inside a HTML comment of the form
\texttt{\ \textless{}!-\/-raw-typst\ {[}…{]}-\/-\textgreater{}\ } to
have it evaluated directly as Typst code (but only if the
\texttt{\ raw-typst\ } option to \texttt{\ render\ } is set to
\texttt{\ true\ } , otherwise it will just be seen as a regular comment
and removed):

\begin{Shaded}
\begin{Highlighting}[]

\end{Highlighting}
\end{Shaded}

\subsection{Markdownâ€``Typst
Polyglots}\label{markdownuxe2typst-polyglots}

This project has a manual as a PDF and a README as a Markdown document,
but by the power of this library they are in fact the same thing!
Furthermore, one can read the \texttt{\ README.md\ } in a markdown
viewer and it will display correctly, but one can \emph{also} run
\texttt{\ typst\ compile\ README.md\ } to generate the Typst-typeset
\texttt{\ README.pdf\ } .

How does this work? We just have to be clever about how we write the
README:

\begin{Shaded}
\begin{Highlighting}[]
\NormalTok{(Typst preamble content)}
\NormalTok{\#"}


\NormalTok{Regular Markdown goes here…}

\end{Highlighting}
\end{Shaded}

The same code but syntax-highlighted as Typst code helps to illuminate
it:

\begin{Shaded}
\begin{Highlighting}[]
\NormalTok{\textless{}picture\textgreater{}}
\NormalTok{(Typst preamble content)}
\NormalTok{\#"\textless{}/picture\textgreater{}}
\NormalTok{\textless{}!{-}{-}".slice(0,0)}
\NormalTok{\#import "@preview/cmarker:0.1.1"}
\NormalTok{\#let markdown = read("README.md")}
\NormalTok{\#markdown = markdown.slice(markdown.position("\textless{}/picture\textgreater{}") + "\textless{}/picture\textgreater{}".len())}
\NormalTok{\#cmarker.render(markdown, h1{-}level: 0)}
\NormalTok{/*{-}{-}\textgreater{}}

\NormalTok{Regular Markdown goes here…}

\NormalTok{\textless{}!{-}{-}*///{-}{-}\textgreater{}}
\end{Highlighting}
\end{Shaded}

\subsection{Limitations}\label{limitations}

\begin{itemize}
\tightlist
\item
  We do not currently support HTML tags, and they will be stripped from
  the output; for example, GitHub supports writing
  \texttt{\ \textless{}sub\textgreater{}text\textless{}/sub\textgreater{}\ }
  to get subscript text, but we will render that as simply “text�.
  In future it would be nice to support a subset of HTML tags.
\item
  We do not currently support Markdown tables and footnotes. In future
  it would be good to.
\item
  Although I tried my best to escape everything correctly, I won’t
  provide a hard guarantee that everything is fully sandboxed even if
  you set \texttt{\ raw-typst:\ false\ } . That said, Typst itself is
  well-sandboxed anyway.
\end{itemize}

\subsection{Development}\label{development}

\begin{itemize}
\tightlist
\item
  Build the plugin with \texttt{\ ./build.sh\ } , which produces the
  \texttt{\ plugin.wasm\ } necessary to use this.
\item
  Compile examples with
  \texttt{\ typst\ compile\ examples/\{name\}.typ\ -\/-root\ .\ } .
\item
  Compile this README to PDF with \texttt{\ typst\ compile\ README.md\ }
  .
\end{itemize}

\subsubsection{How to add}\label{how-to-add}

Copy this into your project and use the import as \texttt{\ cmarker\ }

\begin{verbatim}
#import "@preview/cmarker:0.1.1"
\end{verbatim}

\includesvg[width=0.16667in,height=0.16667in]{/assets/icons/16-copy.svg}

Check the docs for
\href{https://typst.app/docs/reference/scripting/\#packages}{more
information on how to import packages} .

\subsubsection{About}\label{about}

\begin{description}
\tightlist
\item[Author :]
Sabrina Jewson
\item[License:]
MIT
\item[Current version:]
0.1.1
\item[Last updated:]
September 11, 2024
\item[First released:]
October 23, 2023
\item[Minimum Typst version:]
0.8.0
\item[Archive size:]
94.8 kB
\href{https://packages.typst.org/preview/cmarker-0.1.1.tar.gz}{\pandocbounded{\includesvg[keepaspectratio]{/assets/icons/16-download.svg}}}
\item[Repository:]
\href{https://github.com/SabrinaJewson/cmarker.typ}{GitHub}
\end{description}

\subsubsection{Where to report issues?}\label{where-to-report-issues}

This package is a project of Sabrina Jewson . Report issues on
\href{https://github.com/SabrinaJewson/cmarker.typ}{their repository} .
You can also try to ask for help with this package on the
\href{https://forum.typst.app}{Forum} .

Please report this package to the Typst team using the
\href{https://typst.app/contact}{contact form} if you believe it is a
safety hazard or infringes upon your rights.

\phantomsection\label{versions}
\subsubsection{Version history}\label{version-history}

\begin{longtable}[]{@{}ll@{}}
\toprule\noalign{}
Version & Release Date \\
\midrule\noalign{}
\endhead
\bottomrule\noalign{}
\endlastfoot
0.1.1 & September 11, 2024 \\
\href{https://typst.app/universe/package/cmarker/0.1.0/}{0.1.0} &
October 23, 2023 \\
\end{longtable}

Typst GmbH did not create this package and cannot guarantee correct
functionality of this package or compatibility with any version of the
Typst compiler or app.


\title{typst.app/universe/package/quick-maths}

\phantomsection\label{banner}
\section{quick-maths}\label{quick-maths}

{ 0.2.0 }

Custom shorthands for math equations.

{ } Featured Package

\phantomsection\label{readme}
A package for creating custom shorthands for math equations.

\subsection{Usage}\label{usage}

The package comes with a single template function
\texttt{\ shorthands\ } that takes one or more tuples of the form
\texttt{\ (shorthand,\ replacement)\ } , where \texttt{\ shorthand\ }
can be a string or content.

There are some small quality of life features for interaction of
shorthands with fractions and attachments:

\begin{itemize}
\tightlist
\item
  If the right-most symbol of a shorthand has any attachments, they are
  moved to the shorthand’s replacement.
\item
  If a shorthand ends in the numerator of a fraction, the whole
  replacement is placed in the numerator.
\item
  If a shorthand starts in the denominator of a fraction, the whole
  replacement is placed in the denominator.
\end{itemize}

As the implementation of these features is quite hacky, you may
encounter some edge cases, where the use of explicit parentheses
hopefully saves you.

\subsection{Notes}\label{notes}

\begin{itemize}
\item
  Shorthands are parsed in the order they are given, so if you have a
  shorthand that is a prefix of another shorthand, you should put the
  longer shorthand first.
\item
  The content of an equation is traversed from left to right, so the
  left-most matching shorthand will be replaced first.
\item
  Shorthands consisting of only a single character or element may be
  applied using show rules, so that they can affect non-sequence
  elements. This may lead to different behavior than multi-character
  shorthands.
\item
  If the replacement of a shorthand contains a shorthand itself, there
  are no protections against infinite recursion or overflows.
\end{itemize}

\subsection{Example}\label{example}

\begin{Shaded}
\begin{Highlighting}[]
\NormalTok{\#import "@preview/quick{-}maths:0.2.0": shorthands}

\NormalTok{\#show: shorthands.with(}
\NormalTok{  ($+{-}$, $plus.minus$),}
\NormalTok{  ($|{-}$, math.tack),}
\NormalTok{  ($\textless{}=$, math.arrow.l.double) // Replaces \textquotesingle{}≤\textquotesingle{}}
\NormalTok{)}

\NormalTok{$ x\^{}2 = 9 quad \textless{}==\textgreater{} quad x = +{-}3 $}
\NormalTok{$ A or B |{-} A $}
\NormalTok{$ x \textless{}= y $}
\end{Highlighting}
\end{Shaded}

\pandocbounded{\includesvg[keepaspectratio]{https://github.com/typst/packages/raw/main/packages/preview/quick-maths/0.2.0/assets/example.svg}}

\subsubsection{How to add}\label{how-to-add}

Copy this into your project and use the import as
\texttt{\ quick-maths\ }

\begin{verbatim}
#import "@preview/quick-maths:0.2.0"
\end{verbatim}

\includesvg[width=0.16667in,height=0.16667in]{/assets/icons/16-copy.svg}

Check the docs for
\href{https://typst.app/docs/reference/scripting/\#packages}{more
information on how to import packages} .

\subsubsection{About}\label{about}

\begin{description}
\tightlist
\item[Author :]
Eric Biedert
\item[License:]
MIT
\item[Current version:]
0.2.0
\item[Last updated:]
November 18, 2024
\item[First released:]
July 5, 2024
\item[Archive size:]
3.58 kB
\href{https://packages.typst.org/preview/quick-maths-0.2.0.tar.gz}{\pandocbounded{\includesvg[keepaspectratio]{/assets/icons/16-download.svg}}}
\item[Repository:]
\href{https://github.com/EpicEricEE/typst-quick-maths}{GitHub}
\item[Categor y :]
\begin{itemize}
\tightlist
\item[]
\item
  \pandocbounded{\includesvg[keepaspectratio]{/assets/icons/16-hammer.svg}}
  \href{https://typst.app/universe/search/?category=utility}{Utility}
\end{itemize}
\end{description}

\subsubsection{Where to report issues?}\label{where-to-report-issues}

This package is a project of Eric Biedert . Report issues on
\href{https://github.com/EpicEricEE/typst-quick-maths}{their repository}
. You can also try to ask for help with this package on the
\href{https://forum.typst.app}{Forum} .

Please report this package to the Typst team using the
\href{https://typst.app/contact}{contact form} if you believe it is a
safety hazard or infringes upon your rights.

\phantomsection\label{versions}
\subsubsection{Version history}\label{version-history}

\begin{longtable}[]{@{}ll@{}}
\toprule\noalign{}
Version & Release Date \\
\midrule\noalign{}
\endhead
\bottomrule\noalign{}
\endlastfoot
0.2.0 & November 18, 2024 \\
\href{https://typst.app/universe/package/quick-maths/0.1.0/}{0.1.0} &
July 5, 2024 \\
\end{longtable}

Typst GmbH did not create this package and cannot guarantee correct
functionality of this package or compatibility with any version of the
Typst compiler or app.


\title{typst.app/universe/package/stellar-iac}

\phantomsection\label{banner}
\phantomsection\label{template-thumbnail}
\pandocbounded{\includegraphics[keepaspectratio]{https://packages.typst.org/preview/thumbnails/stellar-iac-0.4.1-small.webp}}

\section{stellar-iac}\label{stellar-iac}

{ 0.4.1 }

Template for the International Astronautical Congress (IAC) manuscript

\href{/app?template=stellar-iac&version=0.4.1}{Create project in app}

\phantomsection\label{readme}
This is an unofficial Typst template for the International Astronautical
Congress (IAC) manuscript, which is based on
\href{https://www.iafastro.org/assets/files/IAC\%202023\%20Manuscript\%20Guidelines.pdf}{the
74th IAC Manuscript Guidelines (PDF file)} and
\href{https://www.iafastro.org/assets/files/IAC\%202023_Manuscript-Template.doc}{the
Manuscript Template and Style Guide (MS Word file)} .

\subsection{Usage}\label{usage}

To initialize a project with this template, run the following command:

\begin{Shaded}
\begin{Highlighting}[]
\ExtensionTok{typst}\NormalTok{ init @preview/stellar{-}iac}
\end{Highlighting}
\end{Shaded}

Or, you can manually add the following line at the beginning of your
Typst file:

\begin{Shaded}
\begin{Highlighting}[]
\NormalTok{\#import "@preview/stellar{-}iac:0.4.1": project}
\end{Highlighting}
\end{Shaded}

The \texttt{\ project\ } function exported by this template has the
following named arguments:

\begin{itemize}
\tightlist
\item
  \texttt{\ paper-code\ } (default: \texttt{\ ""\ } ): The paper code of
  the manuscript.
\item
  \texttt{\ title\ } (default: \texttt{\ ""\ } ): The title of the
  manuscript.
\item
  \texttt{\ authors\ } (default: \texttt{\ ()\ } ): The authors of the
  manuscript. Each item in the array should be a dictionary with the
  following keys:

  \begin{itemize}
  \tightlist
  \item
    \texttt{\ name\ } (required): The name of the author.
  \item
    \texttt{\ email\ } (required): The email address of the author.
  \item
    \texttt{\ affiliation\ } (required): The affiliation of the author.
    The value must match one of the affiliation names defined in the
    \texttt{\ organizations\ } argument.
  \item
    \texttt{\ corresponding\ } (default: \texttt{\ false\ } ): Whether
    the author is the corresponding author.
  \end{itemize}
\item
  \texttt{\ organizations\ } (default: \texttt{\ ()\ } ): The
  organizations of the authors. Each item in the array should be a
  dictionary with the following keys:

  \begin{itemize}
  \tightlist
  \item
    \texttt{\ name\ } (required): The name of the organization. This
    name should be used in the \texttt{\ affiliation\ } field of the
    \texttt{\ authors\ } argument.
  \item
    \texttt{\ display\ } (required): The display name of the
    organization, including its address.
  \end{itemize}
\item
  \texttt{\ keywords\ } (default: \texttt{\ ()\ } ): The keywords of the
  manuscript.
\item
  \texttt{\ header\ } (default: \texttt{\ {[}{]}\ } ): The header of the
  manuscript. For IAC 2024, it should be
  \texttt{\ {[}75\#super{[}th{]}\ International\ Astronautical\ Congress\ (IAC),\ Milan,\ Italy,\ 14-18\ October\ 2024.\textbackslash{}\ Copyright\ \#\{sym.copyright\}2024\ by\ the\ International\ Astronautical\ Federation\ (IAF).\ All\ rights\ reserved.{]}\ }
  .
\item
  \texttt{\ abstract\ } (default: \texttt{\ ""\ } ): The abstract of the
  manuscript.
\end{itemize}

See
\href{https://github.com/typst/packages/raw/main/packages/preview/stellar-iac/0.4.1/template/main.typ}{\texttt{\ main.typ\ }}
for more details.

\subsection{Notable differences from the official
template}\label{notable-differences-from-the-official-template}

\begin{itemize}
\tightlist
\item
  The citation style is not exactly the same as the official template.
  This could be fixed by manually preparing a CSL file, but it is
  “good enough.�
\end{itemize}

\subsection{Directory Structure and
Licensing}\label{directory-structure-and-licensing}

\begin{itemize}
\item
  The \texttt{\ lib.typ\ } file and all other files in this package,
  except for the content in the \texttt{\ template\ } directory, are
  licensed under the MIT License. See
  \href{https://github.com/typst/packages/raw/main/packages/preview/stellar-iac/0.4.1/LICENSE-MIT.txt}{LICENSE-MIT.txt}
  for details.
\item
  The content in the \texttt{\ template\ } directory is licensed under
  the MIT-0 License, which allows for unlimited reuse without any
  restrictions. See
  \href{https://github.com/typst/packages/raw/main/packages/preview/stellar-iac/0.4.1/LICENSE-MIT-0.txt}{LICENSE-MIT-0.txt}
  for more information.
\item
  The \texttt{\ reproduction\ } directory is included in this repository
  to demonstrate how the Typst template in \texttt{\ lib.typ\ } can be
  used to reproduce a layout similar to
  \href{https://www.iafastro.org/assets/files/IAC\%202023_Manuscript-Template.doc}{the
  original MS Word template} copyrighted by the International
  Astronautical Federation (IAF). \textbf{This content is not part of
  the distributed package and is provided here solely for demonstration
  purposes.} It is not licensed for use, modification, or distribution
  without permission from the copyright holder.
\end{itemize}

\href{/app?template=stellar-iac&version=0.4.1}{Create project in app}

\subsubsection{How to use}\label{how-to-use}

Click the button above to create a new project using this template in
the Typst app.

You can also use the Typst CLI to start a new project on your computer
using this command:

\begin{verbatim}
typst init @preview/stellar-iac:0.4.1
\end{verbatim}

\includesvg[width=0.16667in,height=0.16667in]{/assets/icons/16-copy.svg}

\subsubsection{About}\label{about}

\begin{description}
\tightlist
\item[Author s :]
\href{https://github.com/shunichironomura}{Shunichiro Nomura} \&
\href{https://github.com/conjikidow}{Riki Nakamura}
\item[License:]
MIT AND MIT-0
\item[Current version:]
0.4.1
\item[Last updated:]
September 24, 2024
\item[First released:]
September 24, 2024
\item[Minimum Typst version:]
0.11.1
\item[Archive size:]
182 kB
\href{https://packages.typst.org/preview/stellar-iac-0.4.1.tar.gz}{\pandocbounded{\includesvg[keepaspectratio]{/assets/icons/16-download.svg}}}
\item[Repository:]
\href{https://github.com/shunichironomura/iac-typst-template}{GitHub}
\item[Discipline :]
\begin{itemize}
\tightlist
\item[]
\item
  \href{https://typst.app/universe/search/?discipline=engineering}{Engineering}
\end{itemize}
\item[Categor y :]
\begin{itemize}
\tightlist
\item[]
\item
  \pandocbounded{\includesvg[keepaspectratio]{/assets/icons/16-atom.svg}}
  \href{https://typst.app/universe/search/?category=paper}{Paper}
\end{itemize}
\end{description}

\subsubsection{Where to report issues?}\label{where-to-report-issues}

This template is a project of Shunichiro Nomura and Riki Nakamura .
Report issues on
\href{https://github.com/shunichironomura/iac-typst-template}{their
repository} . You can also try to ask for help with this template on the
\href{https://forum.typst.app}{Forum} .

Please report this template to the Typst team using the
\href{https://typst.app/contact}{contact form} if you believe it is a
safety hazard or infringes upon your rights.

\phantomsection\label{versions}
\subsubsection{Version history}\label{version-history}

\begin{longtable}[]{@{}ll@{}}
\toprule\noalign{}
Version & Release Date \\
\midrule\noalign{}
\endhead
\bottomrule\noalign{}
\endlastfoot
0.4.1 & September 24, 2024 \\
\end{longtable}

Typst GmbH did not create this template and cannot guarantee correct
functionality of this template or compatibility with any version of the
Typst compiler or app.


\title{typst.app/universe/package/apa7-ish}

\phantomsection\label{banner}
\phantomsection\label{template-thumbnail}
\pandocbounded{\includegraphics[keepaspectratio]{https://packages.typst.org/preview/thumbnails/apa7-ish-0.2.0-small.webp}}

\section{apa7-ish}\label{apa7-ish}

{ 0.2.0 }

Typst Template that (mostly) complies with APA7 Style (Work in
Progress).

\href{/app?template=apa7-ish&version=0.2.0}{Create project in app}

\phantomsection\label{readme}
\href{https://typst.app/}{Typst} Template that (mostly) complies with
APA7 Style (Work in Progress).

The template does not follow all recommendations by the APA Manual,
especially when the suggestions break with typographic conventions (such
as double line spacing :vomiting\_face:). Instead, the goal of this
template is that it generates you a high-quality manuscript that has all
the important components of the APA7 format, but is aesthetically
pleasing.

The following works already quite well:

\begin{itemize}
\tightlist
\item
  consistent and simple typesetting
\item
  correct display of author information / author note
\item
  citations anfalsed references
\item
  Page headers and footers (show short title in header)
\item
  Option to anonymize the paper
\item
  Tables: consisting of 3 parts (caption, content, and table notes)
\end{itemize}

This is still not finished:

\begin{itemize}
\tightlist
\item
  figures
\item
  complete pandoc integration (template for pandoc to replace
  Latex-based workflows)
\item
  automatic calculation of page margins (like memoir-class for Latex)
\end{itemize}

The easiest way to get started is to edit the example file, which has
sensible default values. Most fields in the configuration are optional
and will safely be ignored (not rendered) when you set them to
\texttt{\ none\ } .

\subsection{Authors}\label{authors}

The \texttt{\ authors\ } setting expects an array of dictionaries with
the following fields:

\begin{Shaded}
\begin{Highlighting}[]
\NormalTok{(}
\NormalTok{  name: "First Name Last Name", // Name of author as it should appear on the paper title page}
\NormalTok{  affiliation: "University, Department", // affiliation(s) of author as it should appear on the title page}
\NormalTok{  orcid: "0000{-}0000{-}0000{-}0000", // optional for author note}
\NormalTok{  corresponding: true, // optional to mark an author as corresponding author}
\NormalTok{  email: "email@upenn.edu", // optional email address, required if author is corresponding}
\NormalTok{  postal: "Longer string", // optional postal address for corresponding author}
\NormalTok{)}
\end{Highlighting}
\end{Shaded}

Note that the \texttt{\ affiliation\ } field also accepts an array, in
case an author has several affiliations:

\begin{Shaded}
\begin{Highlighting}[]
\NormalTok{(}
\NormalTok{  name: "First Name Last Name",}
\NormalTok{  affiliation: ("University A", "University B")}
\NormalTok{  ...}
\NormalTok{)}
\end{Highlighting}
\end{Shaded}

\subsection{Anonymization}\label{anonymization}

Sometimes you need to submit a paper without any author information. In
such cases you can set \texttt{\ anonymous\ } to \texttt{\ true\ } .

\subsection{Tables}\label{tables}

The template has basic support for tables with a handful of utilities.
Analogous to the \href{https://ctan.org/pkg/booktabs}{Latex booktabs
package} , there are pre-defined horizontal lines (“rules�):

\begin{itemize}
\tightlist
\item
  \texttt{\ \#toprule\ } : used at the top of the table, before the
  first row
\item
  \texttt{\ \#midrule\ } : used to separate the header row, or to
  separate a totals row at the bottom
\item
  \texttt{\ \#bottomrule\ } : used after the last row (technically the
  same as toprule, but may be useful later to define custom behaviour)
\end{itemize}

Addtionally, there is a \texttt{\ \#tablenote\ } function to be used to
place a table note below the table.

A minimal usage example (taken from the typst documentation):

\begin{Shaded}
\begin{Highlighting}[]
\NormalTok{// wrap everything in a \#figure}
\NormalTok{\#figure(}
\NormalTok{  [}
\NormalTok{    \#table(}
\NormalTok{      columns: 2,}
\NormalTok{      align: left,}
\NormalTok{      toprule, // separate table from other content}
\NormalTok{      table.header([Amount], [Ingredient]),}
\NormalTok{      midrule, // separation between table header and body}
\NormalTok{      [360g], [Baking flour],}
\NormalTok{      [250g], [Butter (room temp.)],}
\NormalTok{      [150g], [Brown sugar],}
\NormalTok{      [100g], [Cane sugar],}
\NormalTok{      [100g], [70\% cocoa chocolate],}
\NormalTok{      [100g], [35{-}40\% cocoa chocolate],}
\NormalTok{      [2], [Eggs],}
\NormalTok{      [Pinch], [Salt],}
\NormalTok{      [Drizzle], [Vanilla extract],}
\NormalTok{      bottomrule // separation after the last row}
\NormalTok{    )}
\NormalTok{    // tablenote goes after the \#table}
\NormalTok{    \#tablenote([Here are some additional notes.])}
\NormalTok{  ],}
\NormalTok{  // caption is part of the \#figure}
\NormalTok{  caption: [Here is the table caption]}
\NormalTok{)}
\end{Highlighting}
\end{Shaded}

\href{/app?template=apa7-ish&version=0.2.0}{Create project in app}

\subsubsection{How to use}\label{how-to-use}

Click the button above to create a new project using this template in
the Typst app.

You can also use the Typst CLI to start a new project on your computer
using this command:

\begin{verbatim}
typst init @preview/apa7-ish:0.2.0
\end{verbatim}

\includesvg[width=0.16667in,height=0.16667in]{/assets/icons/16-copy.svg}

\subsubsection{About}\label{about}

\begin{description}
\tightlist
\item[Author :]
MrWunderbar
\item[License:]
MIT
\item[Current version:]
0.2.0
\item[Last updated:]
October 30, 2024
\item[First released:]
October 21, 2024
\item[Minimum Typst version:]
0.12.0
\item[Archive size:]
8.21 kB
\href{https://packages.typst.org/preview/apa7-ish-0.2.0.tar.gz}{\pandocbounded{\includesvg[keepaspectratio]{/assets/icons/16-download.svg}}}
\item[Repository:]
\href{https://github.com/mrwunderbar666/typst-apa7ish}{GitHub}
\item[Categor y :]
\begin{itemize}
\tightlist
\item[]
\item
  \pandocbounded{\includesvg[keepaspectratio]{/assets/icons/16-atom.svg}}
  \href{https://typst.app/universe/search/?category=paper}{Paper}
\end{itemize}
\end{description}

\subsubsection{Where to report issues?}\label{where-to-report-issues}

This template is a project of MrWunderbar . Report issues on
\href{https://github.com/mrwunderbar666/typst-apa7ish}{their repository}
. You can also try to ask for help with this template on the
\href{https://forum.typst.app}{Forum} .

Please report this template to the Typst team using the
\href{https://typst.app/contact}{contact form} if you believe it is a
safety hazard or infringes upon your rights.

\phantomsection\label{versions}
\subsubsection{Version history}\label{version-history}

\begin{longtable}[]{@{}ll@{}}
\toprule\noalign{}
Version & Release Date \\
\midrule\noalign{}
\endhead
\bottomrule\noalign{}
\endlastfoot
0.2.0 & October 30, 2024 \\
\href{https://typst.app/universe/package/apa7-ish/0.1.0/}{0.1.0} &
October 21, 2024 \\
\end{longtable}

Typst GmbH did not create this template and cannot guarantee correct
functionality of this template or compatibility with any version of the
Typst compiler or app.


\title{typst.app/universe/package/esotefy}

\phantomsection\label{banner}
\section{esotefy}\label{esotefy}

{ 1.0.0 }

A brainfuck implementation in pure Typst

\phantomsection\label{readme}
\begin{quote}
A compilation of esoteric languages including for now brainfuck
\end{quote}

\subsection{In pure brainfuck}\label{in-pure-brainfuck}

\begin{Shaded}
\begin{Highlighting}[]
\NormalTok{\#import "@preview/esotefy:1.0.0": brainf;}

\NormalTok{\#brainf("++++++++[\textgreater{}++++[\textgreater{}++\textgreater{}+++\textgreater{}+++\textgreater{}+\textless{}\textless{}\textless{}\textless{}{-}]\textgreater{}+\textgreater{}+\textgreater{}{-}\textgreater{}\textgreater{}+[\textless{}]\textless{}{-}]\textgreater{}\textgreater{}.\textgreater{}{-}{-}{-}.+++++++..+++.\textgreater{}\textgreater{}.\textless{}{-}.\textless{}.+++.{-}{-}{-}{-}{-}{-}.{-}{-}{-}{-}{-}{-}{-}{-}.\textgreater{}\textgreater{}+.\textgreater{}++.");}
\end{Highlighting}
\end{Shaded}

Into

\pandocbounded{\includegraphics[keepaspectratio]{https://media.discordapp.net/attachments/751591144919662752/1176988035309633647/image.png?ex=6570de86&is=655e6986&hm=60e18ac7187c117ab08a95c323f5059424342dbb9d8da49600c82502b5d14e7f&=&format=webp&width=328&height=102}}

\subsection{With inputs}\label{with-inputs}

\begin{Shaded}
\begin{Highlighting}[]
\NormalTok{\#import "@preview/esotefy:1.0.0": brainf;}

\NormalTok{\#brainf("++++++++++++++[{-}\textgreater{},.\textless{}]", inp: "Goodbye World!");}
\end{Highlighting}
\end{Shaded}

Into

\pandocbounded{\includegraphics[keepaspectratio]{https://media.discordapp.net/attachments/751591144919662752/1176988280613515366/image.png?ex=6570dec1&is=655e69c1&hm=f9285649f3e5ab72749af5820972c52827c727f6c52351b63d0bbd2ba9afce87&=&format=webp&width=808&height=181}}

I’ve based my implementation from theses documents:

\begin{itemize}
\tightlist
\item
  \href{https://en.wikipedia.org/wiki/Brainfuck}{Wikipedia}
\item
  \href{https://github.com/sunjay/brainfuck}{Github}
\item
  \href{https://onestepcode.com/brainfuck-compiler-c/}{A compiler of
  Brainfuck in c}
\end{itemize}

\subsubsection{How to add}\label{how-to-add}

Copy this into your project and use the import as \texttt{\ esotefy\ }

\begin{verbatim}
#import "@preview/esotefy:1.0.0"
\end{verbatim}

\includesvg[width=0.16667in,height=0.16667in]{/assets/icons/16-copy.svg}

Check the docs for
\href{https://typst.app/docs/reference/scripting/\#packages}{more
information on how to import packages} .

\subsubsection{About}\label{about}

\begin{description}
\tightlist
\item[Author :]
Thumus
\item[License:]
MIT
\item[Current version:]
1.0.0
\item[Last updated:]
November 29, 2023
\item[First released:]
November 29, 2023
\item[Minimum Typst version:]
0.7.0
\item[Archive size:]
2.03 kB
\href{https://packages.typst.org/preview/esotefy-1.0.0.tar.gz}{\pandocbounded{\includesvg[keepaspectratio]{/assets/icons/16-download.svg}}}
\item[Repository:]
\href{git@github.com:Thumuss/brainfuck.git}{GitHub}
\end{description}

\subsubsection{Where to report issues?}\label{where-to-report-issues}

This package is a project of Thumus . Report issues on
\href{git@github.com:Thumuss/brainfuck.git}{their repository} . You can
also try to ask for help with this package on the
\href{https://forum.typst.app}{Forum} .

Please report this package to the Typst team using the
\href{https://typst.app/contact}{contact form} if you believe it is a
safety hazard or infringes upon your rights.

\phantomsection\label{versions}
\subsubsection{Version history}\label{version-history}

\begin{longtable}[]{@{}ll@{}}
\toprule\noalign{}
Version & Release Date \\
\midrule\noalign{}
\endhead
\bottomrule\noalign{}
\endlastfoot
1.0.0 & November 29, 2023 \\
\end{longtable}

Typst GmbH did not create this package and cannot guarantee correct
functionality of this package or compatibility with any version of the
Typst compiler or app.


\title{typst.app/universe/package/ilm}

\phantomsection\label{banner}
\phantomsection\label{template-thumbnail}
\pandocbounded{\includegraphics[keepaspectratio]{https://packages.typst.org/preview/thumbnails/ilm-1.4.0-small.webp}}

\section{ilm}\label{ilm}

{ 1.4.0 }

Versatile and minimal template for non-fiction writing. Ideal for class
notes, reports, and books

{ } Featured Template

\href{/app?template=ilm&version=1.4.0}{Create project in app}

\phantomsection\label{readme}
\begin{quote}
‘Ilm (Urdu: عÙ?Ù„Ù'Ù\ldots) is the Urdu term for knowledge. It is
pronounced as
\href{https://en.wiktionary.org/wiki/\%D8\%B9\%D9\%84\%D9\%85\#Urdu}{/ə.ləm/}
.
\end{quote}

A versatile, clean and minimal template for non-fiction writing. The
template is ideal for class notes, reports, and books.

It contains a title page, a table of contents, and indices for different
types of figures; images, tables, code blocks.

Dynamic running footer contains the title of the chapter (top-level
heading).

See the
\href{https://github.com/talal/ilm/blob/main/example.pdf}{example.pdf}
file to see how it looks.

\subsection{Usage}\label{usage}

You can use this template in the Typst web app by clicking “Start from
template� on the dashboard and searching for \texttt{\ ilm\ } .

Alternatively, you can use the CLI to kick this project off using the
command:

\begin{Shaded}
\begin{Highlighting}[]
\ExtensionTok{typst}\NormalTok{ init @preview/ilm}
\end{Highlighting}
\end{Shaded}

Typst will create a new directory with all the files needed to get you
started.

The template will initialize your package with a sample call to the
\texttt{\ ilm\ } function in a show rule. If you, however, want to
change an existing project to use this template, you can add a show rule
like this at the top of your file:

\begin{Shaded}
\begin{Highlighting}[]
\NormalTok{\#import "@preview/ilm:1.4.0": *}

\NormalTok{\#set text(lang: "en")}

\NormalTok{\#show: ilm.with(}
\NormalTok{  title: [Your Title],}
\NormalTok{  author: "Max Mustermann",}
\NormalTok{  date: datetime(year: 2024, month: 03, day: 19),}
\NormalTok{  abstract: [\#lorem(30)],}
\NormalTok{  bibliography: bibliography("refs.bib"),}
\NormalTok{  figure{-}index: (enabled: true),}
\NormalTok{  table{-}index: (enabled: true),}
\NormalTok{  listing{-}index: (enabled: true)}
\NormalTok{)}

\NormalTok{// Your content goes below.}
\end{Highlighting}
\end{Shaded}

\begin{quote}
{[}!NOTE{]} This template uses the
\href{https://typeof.net/Iosevka/}{Iosevka} font for raw text. In order
to use Iosevka, the font must be installed on your computer. In case
Iosevka is not installed, as will be the case for Typst Web App, then
the template will fall back to the default “Fira Mono� font.
\end{quote}

\subsection{Configuration}\label{configuration}

This template exports the \texttt{\ ilm\ } function with the following
named arguments:

\begin{longtable}[]{@{}llll@{}}
\toprule\noalign{}
Argument & Default Value & Type & Description \\
\midrule\noalign{}
\endhead
\bottomrule\noalign{}
\endlastfoot
\texttt{\ title\ } & \texttt{\ Your\ Title\ } &
\href{https://typst.app/docs/reference/foundations/content/}{content} &
The title for your work. \\
\texttt{\ author\ } & \texttt{\ Author\ } &
\href{https://typst.app/docs/reference/foundations/content/}{content} &
A string to specify the author’s name \\
\texttt{\ paper-size\ } & \texttt{\ a4\ } &
\href{https://typst.app/docs/reference/foundations/str/}{string} &
Specify a
\href{https://typst.app/docs/reference/layout/page\#parameters-paper}{paper
size string} to change the page size. \\
\texttt{\ date\ } & \texttt{\ none\ } &
\href{https://typst.app/docs/reference/foundations/datetime/}{datetime}
& The date that will be displayed on the cover page. \\
\texttt{\ date-format\ } &
\texttt{\ {[}month\ repr:long{]}\ {[}day\ padding:zero{]},\ {[}year\ repr:full{]}\ }
& \href{https://typst.app/docs/reference/foundations/str/}{string} & The
format for the date that will be displayed on the cover page. By
default, the date will be displayed as \texttt{\ MMMM\ DD,\ YYYY\ } . \\
\texttt{\ abstract\ } & \texttt{\ none\ } &
\href{https://typst.app/docs/reference/foundations/content/}{content} &
A brief summary/description of your work. This is shown on the cover
page. \\
\texttt{\ preface\ } & \texttt{\ none\ } &
\href{https://typst.app/docs/reference/foundations/content/}{content} &
The preface for your work. The preface content is shown on its own
separate page after the cover. \\
\texttt{\ chapter-pagebreak\ } & \texttt{\ true\ } &
\href{https://typst.app/docs/reference/foundations/bool/}{bool} &
Setting this to \texttt{\ false\ } will prevent chapters from starting
on a new page. \\
\texttt{\ external-link-circle\ } & \texttt{\ true\ } &
\href{https://typst.app/docs/reference/foundations/bool/}{bool} &
Setting this to \texttt{\ false\ } will disable the maroon circle that
is shown next to external links. \\
\texttt{\ table-of-contents\ } & \texttt{\ outline()\ } &
\href{https://typst.app/docs/reference/foundations/content/}{content} &
The result of a call to the
\href{https://typst.app/docs/reference/model/outline/}{outline function}
or none. Setting this to \texttt{\ none\ } will disable the table of
contents. \\
\texttt{\ appendix\ } &
\texttt{\ (enabled:\ false,\ title:\ "Appendix",\ heading-numbering-format:\ "A.1.1.",\ body:\ none)\ }
&
\href{https://typst.app/docs/reference/foundations/dictionary/}{dictionary}
& Setting \texttt{\ enabled\ } to \texttt{\ true\ } and defining your
content in \texttt{\ body\ } will display the appendix after the main
body of your document and before the bibliography. \\
\texttt{\ bibliography\ } & \texttt{\ none\ } &
\href{https://typst.app/docs/reference/foundations/content/}{content} &
The result of a call to the
\href{https://typst.app/docs/reference/model/bibliography/}{bibliography
function} or none. Specifying this will configure numeric, IEEE-style
citations. \\
\texttt{\ figure-index\ } &
\texttt{\ (enabled:\ false,\ title:\ "Index\ of\ Figures")\ } &
\href{https://typst.app/docs/reference/foundations/dictionary/}{dictionary}
& Setting this to \texttt{\ true\ } will display an index of image
figures at the end of the document. \\
\texttt{\ table-index\ } &
\texttt{\ (enabled:\ false,\ title:\ "Index\ of\ Tables")\ } &
\href{https://typst.app/docs/reference/foundations/dictionary/}{dictionary}
& Setting this to \texttt{\ true\ } will display an index of table
figures at the end of the document. \\
\texttt{\ listing-index\ } &
\texttt{\ (enabled:\ false,\ title:\ "Index\ of\ Listings")\ } &
\href{https://typst.app/docs/reference/foundations/dictionary/}{dictionary}
& Setting this to \texttt{\ true\ } will display an index of listing
(code block) figures at the end of the document. \\
\end{longtable}

The above table gives you a \emph{brief description} of the different
options that you can choose to customize the template. For a detailed
explanation of these options, see the
\href{https://github.com/talal/ilm/blob/main/example.pdf}{example.pdf}
file.

The function also accepts a single, positional argument for the body.

\begin{quote}
{[}!NOTE{]} The language setting for text ( \texttt{\ lang\ } parameter
of \texttt{\ text\ } function) should be defined before the
\texttt{\ ilm\ } function so that headings such as table of contents and
bibliography will be defined as per the text language.
\end{quote}

\href{/app?template=ilm&version=1.4.0}{Create project in app}

\subsubsection{How to use}\label{how-to-use}

Click the button above to create a new project using this template in
the Typst app.

You can also use the Typst CLI to start a new project on your computer
using this command:

\begin{verbatim}
typst init @preview/ilm:1.4.0
\end{verbatim}

\includesvg[width=0.16667in,height=0.16667in]{/assets/icons/16-copy.svg}

\subsubsection{About}\label{about}

\begin{description}
\tightlist
\item[Author :]
\href{https://github.com/talal}{Muhammad Talal Anwar}
\item[License:]
MIT-0
\item[Current version:]
1.4.0
\item[Last updated:]
November 21, 2024
\item[First released:]
March 22, 2024
\item[Minimum Typst version:]
0.12.0
\item[Archive size:]
9.08 kB
\href{https://packages.typst.org/preview/ilm-1.4.0.tar.gz}{\pandocbounded{\includesvg[keepaspectratio]{/assets/icons/16-download.svg}}}
\item[Repository:]
\href{https://github.com/talal/ilm}{GitHub}
\item[Categor ies :]
\begin{itemize}
\tightlist
\item[]
\item
  \pandocbounded{\includesvg[keepaspectratio]{/assets/icons/16-docs.svg}}
  \href{https://typst.app/universe/search/?category=book}{Book}
\item
  \pandocbounded{\includesvg[keepaspectratio]{/assets/icons/16-speak.svg}}
  \href{https://typst.app/universe/search/?category=report}{Report}
\end{itemize}
\end{description}

\subsubsection{Where to report issues?}\label{where-to-report-issues}

This template is a project of Muhammad Talal Anwar . Report issues on
\href{https://github.com/talal/ilm}{their repository} . You can also try
to ask for help with this template on the
\href{https://forum.typst.app}{Forum} .

Please report this template to the Typst team using the
\href{https://typst.app/contact}{contact form} if you believe it is a
safety hazard or infringes upon your rights.

\phantomsection\label{versions}
\subsubsection{Version history}\label{version-history}

\begin{longtable}[]{@{}ll@{}}
\toprule\noalign{}
Version & Release Date \\
\midrule\noalign{}
\endhead
\bottomrule\noalign{}
\endlastfoot
1.4.0 & November 21, 2024 \\
\href{https://typst.app/universe/package/ilm/1.3.1/}{1.3.1} & November
13, 2024 \\
\href{https://typst.app/universe/package/ilm/1.3.0/}{1.3.0} & November
4, 2024 \\
\href{https://typst.app/universe/package/ilm/1.2.1/}{1.2.1} & August 7,
2024 \\
\href{https://typst.app/universe/package/ilm/1.2.0/}{1.2.0} & August 2,
2024 \\
\href{https://typst.app/universe/package/ilm/1.1.3/}{1.1.3} & July 24,
2024 \\
\href{https://typst.app/universe/package/ilm/1.1.2/}{1.1.2} & June 19,
2024 \\
\href{https://typst.app/universe/package/ilm/1.1.1/}{1.1.1} & May 6,
2024 \\
\href{https://typst.app/universe/package/ilm/1.1.0/}{1.1.0} & April 12,
2024 \\
\href{https://typst.app/universe/package/ilm/1.0.0/}{1.0.0} & April 9,
2024 \\
\href{https://typst.app/universe/package/ilm/0.1.3/}{0.1.3} & April 8,
2024 \\
\href{https://typst.app/universe/package/ilm/0.1.2/}{0.1.2} & March 26,
2024 \\
\href{https://typst.app/universe/package/ilm/0.1.1/}{0.1.1} & March 23,
2024 \\
\href{https://typst.app/universe/package/ilm/0.1.0/}{0.1.0} & March 22,
2024 \\
\end{longtable}

Typst GmbH did not create this template and cannot guarantee correct
functionality of this template or compatibility with any version of the
Typst compiler or app.


\title{typst.app/universe/package/tuhi-course-poster-vuw}

\phantomsection\label{banner}
\phantomsection\label{template-thumbnail}
\pandocbounded{\includegraphics[keepaspectratio]{https://packages.typst.org/preview/thumbnails/tuhi-course-poster-vuw-0.1.0-small.webp}}

\section{tuhi-course-poster-vuw}\label{tuhi-course-poster-vuw}

{ 0.1.0 }

A poster template for VUW course descriptions.

\href{/app?template=tuhi-course-poster-vuw&version=0.1.0}{Create project
in app}

\phantomsection\label{readme}
A Typst template for VUW course posters. To get started:

\begin{Shaded}
\begin{Highlighting}[]
\NormalTok{typst init @preview/tuhi{-}course{-}poster{-}vuw:0.1.0}
\end{Highlighting}
\end{Shaded}

And edit the \texttt{\ main.typ\ } example.

\pandocbounded{\includegraphics[keepaspectratio]{https://github.com/typst/packages/raw/main/packages/preview/tuhi-course-poster-vuw/0.1.0/thumbnail.png}}

\subsection{Contributing}\label{contributing}

PRs are welcome! And if you encounter any bugs or have any
requests/ideas, feel free to open an issue.

\href{/app?template=tuhi-course-poster-vuw&version=0.1.0}{Create project
in app}

\subsubsection{How to use}\label{how-to-use}

Click the button above to create a new project using this template in
the Typst app.

You can also use the Typst CLI to start a new project on your computer
using this command:

\begin{verbatim}
typst init @preview/tuhi-course-poster-vuw:0.1.0
\end{verbatim}

\includesvg[width=0.16667in,height=0.16667in]{/assets/icons/16-copy.svg}

\subsubsection{About}\label{about}

\begin{description}
\tightlist
\item[Author :]
\href{https://github.com/baptiste}{baptiste}
\item[License:]
MPL-2.0
\item[Current version:]
0.1.0
\item[Last updated:]
April 30, 2024
\item[First released:]
April 30, 2024
\item[Archive size:]
421 kB
\href{https://packages.typst.org/preview/tuhi-course-poster-vuw-0.1.0.tar.gz}{\pandocbounded{\includesvg[keepaspectratio]{/assets/icons/16-download.svg}}}
\item[Categor y :]
\begin{itemize}
\tightlist
\item[]
\item
  \pandocbounded{\includesvg[keepaspectratio]{/assets/icons/16-pin.svg}}
  \href{https://typst.app/universe/search/?category=poster}{Poster}
\end{itemize}
\end{description}

\subsubsection{Where to report issues?}\label{where-to-report-issues}

This template is a project of baptiste . You can also try to ask for
help with this template on the \href{https://forum.typst.app}{Forum} .

Please report this template to the Typst team using the
\href{https://typst.app/contact}{contact form} if you believe it is a
safety hazard or infringes upon your rights.

\phantomsection\label{versions}
\subsubsection{Version history}\label{version-history}

\begin{longtable}[]{@{}ll@{}}
\toprule\noalign{}
Version & Release Date \\
\midrule\noalign{}
\endhead
\bottomrule\noalign{}
\endlastfoot
0.1.0 & April 30, 2024 \\
\end{longtable}

Typst GmbH did not create this template and cannot guarantee correct
functionality of this template or compatibility with any version of the
Typst compiler or app.


\title{typst.app/universe/package/dashy-todo}

\phantomsection\label{banner}
\section{dashy-todo}\label{dashy-todo}

{ 0.0.1 }

A method to display TODOs at the side of the page.

\phantomsection\label{readme}
Create TODO comments, which are displayed at the sides of the page.

\pandocbounded{\includesvg[keepaspectratio]{https://github.com/typst/packages/raw/main/packages/preview/dashy-todo/0.0.1/example.svg}}

\subsection{Usage}\label{usage}

The package provides a
\texttt{\ todo(message,\ position:\ auto\ \textbar{}\ left\ \textbar{}\ right)\ }
method. Call it anywhere you need a todo message.

\begin{Shaded}
\begin{Highlighting}[]
\NormalTok{\#import "@preview/dashy{-}todo:0.0.1": todo}

\NormalTok{// It automatically goes to the closer side (left or right)}
\NormalTok{A todo on the left \#todo[On the left].}

\NormalTok{// You can specify a side if you want to}
\NormalTok{\#todo(position: right)[Also right]}

\NormalTok{// You can add arbitrary content}
\NormalTok{\#todo[We need to fix the $lim\_(x {-}\textgreater{} oo)$ equation. See \#link("https://example.com")[example.com]]}

\NormalTok{// And you can create an outline for the TODOs}
\NormalTok{\#outline(title: "TODOs", target: figure.where(kind: "todo"))}
\end{Highlighting}
\end{Shaded}

\subsection{Styling}\label{styling}

You can modify the text by wrapping it, e.g.:

\begin{verbatim}
#let small-todo = (..args) => text(size: 0.6em)[#todo(..args)]

#small-todo[This will be in fine print]
\end{verbatim}

\subsubsection{How to add}\label{how-to-add}

Copy this into your project and use the import as
\texttt{\ dashy-todo\ }

\begin{verbatim}
#import "@preview/dashy-todo:0.0.1"
\end{verbatim}

\includesvg[width=0.16667in,height=0.16667in]{/assets/icons/16-copy.svg}

Check the docs for
\href{https://typst.app/docs/reference/scripting/\#packages}{more
information on how to import packages} .

\subsubsection{About}\label{about}

\begin{description}
\tightlist
\item[Author :]
Otto-AA
\item[License:]
MIT-0
\item[Current version:]
0.0.1
\item[Last updated:]
July 23, 2024
\item[First released:]
July 23, 2024
\item[Archive size:]
2.93 kB
\href{https://packages.typst.org/preview/dashy-todo-0.0.1.tar.gz}{\pandocbounded{\includesvg[keepaspectratio]{/assets/icons/16-download.svg}}}
\item[Repository:]
\href{https://github.com/Otto-AA/dashy-todo}{GitHub}
\item[Categor y :]
\begin{itemize}
\tightlist
\item[]
\item
  \pandocbounded{\includesvg[keepaspectratio]{/assets/icons/16-hammer.svg}}
  \href{https://typst.app/universe/search/?category=utility}{Utility}
\end{itemize}
\end{description}

\subsubsection{Where to report issues?}\label{where-to-report-issues}

This package is a project of Otto-AA . Report issues on
\href{https://github.com/Otto-AA/dashy-todo}{their repository} . You can
also try to ask for help with this package on the
\href{https://forum.typst.app}{Forum} .

Please report this package to the Typst team using the
\href{https://typst.app/contact}{contact form} if you believe it is a
safety hazard or infringes upon your rights.

\phantomsection\label{versions}
\subsubsection{Version history}\label{version-history}

\begin{longtable}[]{@{}ll@{}}
\toprule\noalign{}
Version & Release Date \\
\midrule\noalign{}
\endhead
\bottomrule\noalign{}
\endlastfoot
0.0.1 & July 23, 2024 \\
\end{longtable}

Typst GmbH did not create this package and cannot guarantee correct
functionality of this package or compatibility with any version of the
Typst compiler or app.


\title{typst.app/universe/package/prooftrees}

\phantomsection\label{banner}
\section{prooftrees}\label{prooftrees}

{ 0.1.0 }

Proof trees for natural deduction and type theories

\phantomsection\label{readme}
This package is for constructing proof trees in the style of natural
deduction or the sequent calculus, for \texttt{\ typst\ }
\texttt{\ 0.7.0\ } . Please do open issues for bugs etc :)

Features:

\begin{itemize}
\tightlist
\item
  Inferences can have arbitrarily many premises.
\item
  Inference lines can have left and/or right labels¹
\item
  Configurable² per tree and per line: premise spacing, the line
  stroke, etc… .
\item
  They’re proof trees.
\end{itemize}

¹ The placement of labels is currently very primitive, and requires
much user intervention.

² Options are quite limited.

\subsection{Usage}\label{usage}

The user interface is inspired by
\href{https://ctan.org/pkg/bussproofs}{bussproof} ’s; a tree is
constructed by a sequence of ‘lines’ that state their number of
premises.
\href{https://github.com/typst/packages/raw/main/packages/preview/prooftrees/0.1.0/src/prooftrees.typ}{\texttt{\ src/prooftrees.typ\ }}
contains the documentation and the main functions needed.

The code for some example trees can be seen in
\texttt{\ examples/prooftree\_test.typ\ } .

\subsubsection{Examples}\label{examples}

A single inference would be:

\begin{Shaded}
\begin{Highlighting}[]
\NormalTok{\#import "@preview/prooftrees:0.1.0"}

\NormalTok{\#prooftree.tree(}
\NormalTok{    prooftree.axi[$A =\textgreater{} A$],}
\NormalTok{    prooftree.uni[$A =\textgreater{} A, B$]}
\NormalTok{)}
\end{Highlighting}
\end{Shaded}

\includegraphics[width=0.3\linewidth,height=\textheight,keepaspectratio]{https://raw.githubusercontent.com/david-davies/typst-prooftree/main/examples/Example1.png}

A more interesting example:

\begin{Shaded}
\begin{Highlighting}[]
\NormalTok{\#import "@preview/prooftrees:0.1.0"}

\NormalTok{\#prooftree.tree(}
\NormalTok{    prooftree.axi[$B =\textgreater{} B$],}
\NormalTok{    prooftree.uni[$B =\textgreater{} B, A$],}
\NormalTok{    prooftree.uni[$B =\textgreater{} A, B$],}
\NormalTok{        prooftree.axi[$A =\textgreater{} A$],}
\NormalTok{        prooftree.uni[$A =\textgreater{} A, B$],}
\NormalTok{    prooftree.bin[$B =\textgreater{} A, B$]}
\NormalTok{)}
\end{Highlighting}
\end{Shaded}

\includegraphics[width=0.4\linewidth,height=\textheight,keepaspectratio]{https://raw.githubusercontent.com/david-davies/typst-prooftree/main/examples/Example2.png}

An n-ary inference can be made:

\begin{Shaded}
\begin{Highlighting}[]
\NormalTok{\#import "@preview/prooftrees:0.1.0"}

\NormalTok{\#prooftrees.tree(}
\NormalTok{    prooftrees.axi(pad(bottom: 2pt, [$P\_1$])),}
\NormalTok{    prooftrees.axi(pad(bottom: 2pt, [$P\_2$])),}
\NormalTok{    prooftrees.axi(pad(bottom: 2pt, [$P\_3$])),}
\NormalTok{    prooftrees.axi(pad(bottom: 2pt, [$P\_4$])),}
\NormalTok{    prooftrees.axi(pad(bottom: 2pt, [$P\_5$])),}
\NormalTok{    prooftrees.axi(pad(bottom: 2pt, [$P\_6$])),}
\NormalTok{    prooftrees.nary(6)[$C$],}
\NormalTok{)}
\end{Highlighting}
\end{Shaded}

\includegraphics[width=0.3\linewidth,height=\textheight,keepaspectratio]{https://raw.githubusercontent.com/david-davies/typst-prooftree/main/examples/Example3.png}

\subsection{Known Issues:}\label{known-issues}

\subsubsection{Superscripts and subscripts clip with the
line}\label{superscripts-and-subscripts-clip-with-the-line}

The boundaries of blocks containing math do not expand enough for
sub/pscripts; I think this is a typst issue. Short-term fix: add manual
vspace or padding in the cell.

\subsection{Implementation}\label{implementation}

The placement of the line and conclusion is calculated using
\texttt{\ measure\ } on the premises and labels, and doing geometric
arithmetic with these values.

\subsubsection{How to add}\label{how-to-add}

Copy this into your project and use the import as
\texttt{\ prooftrees\ }

\begin{verbatim}
#import "@preview/prooftrees:0.1.0"
\end{verbatim}

\includesvg[width=0.16667in,height=0.16667in]{/assets/icons/16-copy.svg}

Check the docs for
\href{https://typst.app/docs/reference/scripting/\#packages}{more
information on how to import packages} .

\subsubsection{About}\label{about}

\begin{description}
\tightlist
\item[Author :]
\href{https://github.com/david-davies}{david-davies}
\item[License:]
MIT
\item[Current version:]
0.1.0
\item[Last updated:]
September 3, 2023
\item[First released:]
September 3, 2023
\item[Archive size:]
8.43 kB
\href{https://packages.typst.org/preview/prooftrees-0.1.0.tar.gz}{\pandocbounded{\includesvg[keepaspectratio]{/assets/icons/16-download.svg}}}
\item[Repository:]
\href{https://github.com/david-davies/typst-prooftree}{GitHub}
\end{description}

\subsubsection{Where to report issues?}\label{where-to-report-issues}

This package is a project of david-davies . Report issues on
\href{https://github.com/david-davies/typst-prooftree}{their repository}
. You can also try to ask for help with this package on the
\href{https://forum.typst.app}{Forum} .

Please report this package to the Typst team using the
\href{https://typst.app/contact}{contact form} if you believe it is a
safety hazard or infringes upon your rights.

\phantomsection\label{versions}
\subsubsection{Version history}\label{version-history}

\begin{longtable}[]{@{}ll@{}}
\toprule\noalign{}
Version & Release Date \\
\midrule\noalign{}
\endhead
\bottomrule\noalign{}
\endlastfoot
0.1.0 & September 3, 2023 \\
\end{longtable}

Typst GmbH did not create this package and cannot guarantee correct
functionality of this package or compatibility with any version of the
Typst compiler or app.


\title{typst.app/universe/package/structured-uib}

\phantomsection\label{banner}
\phantomsection\label{template-thumbnail}
\pandocbounded{\includegraphics[keepaspectratio]{https://packages.typst.org/preview/thumbnails/structured-uib-0.1.0-small.webp}}

\section{structured-uib}\label{structured-uib}

{ 0.1.0 }

Lab report template for the course PHYS114 at the University of Bergen.

\href{/app?template=structured-uib&version=0.1.0}{Create project in app}

\phantomsection\label{readme}
Report template to be used for laboratory reports in the course PHYS114
- Basic Measurement Science and Experimental Physics, at the University
of Bergen ( \url{https://www.uib.no/en/courses/PHYS114} ). The template
is in Norwegian only as of now. English support may be added in the
future.

The first part of the packages name is arbitrary, such that it follows
the naming guidelines of Typst.

\textbf{Note:} The template uses the fonts \textbf{STIX Two Text} and
\textbf{STIX Two Math} (
\url{https://github.com/stipub/stixfonts/tree/master/fonts} ). If
running Typst locally on your computer, make sure you have these fonts
installed. There should be no font problems if you are using Typst via
\href{https://typst.app/}{https://typst.app} however.

Usage:

\begin{Shaded}
\begin{Highlighting}[]
\NormalTok{// IMPORTS}
\NormalTok{\#import "@preview/structured{-}uib:0.1.0": *}

\NormalTok{// TEMPLATE SETTINGS}
\NormalTok{\#show: report.with(}
\NormalTok{  task{-}no: "1",}
\NormalTok{  task{-}name: "Måling og behandling av måledata",}
\NormalTok{  authors: (}
\NormalTok{    "Student Enersen",}
\NormalTok{    "Student Toersen", }
\NormalTok{    "Student Treersen"}
\NormalTok{  ),}
\NormalTok{  mails: (}
\NormalTok{    "student.enersen@student.uib.no", }
\NormalTok{    "student.toersen@student.uib.no", }
\NormalTok{    "student.treersen@student.uib.no"}
\NormalTok{  ),}
\NormalTok{  group: "1{-}1",}
\NormalTok{  date: "29. Apr. 2024",}
\NormalTok{  supervisor: "Professor Professorsen",}
\NormalTok{)}

\NormalTok{// CONTENT HERE...}
\end{Highlighting}
\end{Shaded}

Front page:
\pandocbounded{\includegraphics[keepaspectratio]{https://github.com/AugustinWinther/structured-uib/assets/30674646/a93718d8-362d-453b-8047-3c3c4388d442}}

\subsection{Licenses}\label{licenses}

\texttt{\ lib.typ\ } is licensed under MIT. The contents of the
\texttt{\ template/\ } directory are licensed under MIT-0. The
logo/emblem of the University of Bergen (located at
\texttt{\ media/uib-emblem.svg\ } ) is owned by their respective owners.

\href{/app?template=structured-uib&version=0.1.0}{Create project in app}

\subsubsection{How to use}\label{how-to-use}

Click the button above to create a new project using this template in
the Typst app.

You can also use the Typst CLI to start a new project on your computer
using this command:

\begin{verbatim}
typst init @preview/structured-uib:0.1.0
\end{verbatim}

\includesvg[width=0.16667in,height=0.16667in]{/assets/icons/16-copy.svg}

\subsubsection{About}\label{about}

\begin{description}
\tightlist
\item[Author :]
\href{https://winther.io}{Augustin Winther}
\item[License:]
MIT AND MIT-0
\item[Current version:]
0.1.0
\item[Last updated:]
April 29, 2024
\item[First released:]
April 29, 2024
\item[Archive size:]
24.2 kB
\href{https://packages.typst.org/preview/structured-uib-0.1.0.tar.gz}{\pandocbounded{\includesvg[keepaspectratio]{/assets/icons/16-download.svg}}}
\item[Repository:]
\href{https://github.com/AugustinWinther/structured-uib}{GitHub}
\item[Categor y :]
\begin{itemize}
\tightlist
\item[]
\item
  \pandocbounded{\includesvg[keepaspectratio]{/assets/icons/16-speak.svg}}
  \href{https://typst.app/universe/search/?category=report}{Report}
\end{itemize}
\end{description}

\subsubsection{Where to report issues?}\label{where-to-report-issues}

This template is a project of Augustin Winther . Report issues on
\href{https://github.com/AugustinWinther/structured-uib}{their
repository} . You can also try to ask for help with this template on the
\href{https://forum.typst.app}{Forum} .

Please report this template to the Typst team using the
\href{https://typst.app/contact}{contact form} if you believe it is a
safety hazard or infringes upon your rights.

\phantomsection\label{versions}
\subsubsection{Version history}\label{version-history}

\begin{longtable}[]{@{}ll@{}}
\toprule\noalign{}
Version & Release Date \\
\midrule\noalign{}
\endhead
\bottomrule\noalign{}
\endlastfoot
0.1.0 & April 29, 2024 \\
\end{longtable}

Typst GmbH did not create this template and cannot guarantee correct
functionality of this template or compatibility with any version of the
Typst compiler or app.


\title{typst.app/universe/package/kunskap}

\phantomsection\label{banner}
\phantomsection\label{template-thumbnail}
\pandocbounded{\includegraphics[keepaspectratio]{https://packages.typst.org/preview/thumbnails/kunskap-0.1.0-small.webp}}

\section{kunskap}\label{kunskap}

{ 0.1.0 }

A template with generous spacing for reports, assignments, course
documents, and similar (shorter) documents.

\href{/app?template=kunskap&version=0.1.0}{Create project in app}

\phantomsection\label{readme}
A \href{https://typst.app/}{Typst} template mainly intended for shorter
academic documents such as reports, assignments, course documents, and
so on. Its name, \emph{“kunskap�} , means \emph{knowledge} in
Swedish.

See
\href{https://github.com/mbollmann/typst-kunskap/blob/main/example.pdf}{this
example PDF} for a longer demonstration of how it looks.

\subsection{Usage}\label{usage}

You can use this template in the Typst web app by clicking “Start from
template� on the dashboard and searching for \texttt{\ kunskap\ } .

Alternatively, you can use the CLI to kick this project off using the
command

\begin{Shaded}
\begin{Highlighting}[]
\ExtensionTok{typst}\NormalTok{ init @preview/kunskap}
\end{Highlighting}
\end{Shaded}

Typst will create a new directory with all the files needed to get you
started.

\subsection{Configuration}\label{configuration}

This template exports the \texttt{\ kunskap\ } function with several
arguments. You will want to set at least the following, describing the
metadata of your document:

\begin{longtable}[]{@{}ll@{}}
\toprule\noalign{}
Argument & Description \\
\midrule\noalign{}
\endhead
\bottomrule\noalign{}
\endlastfoot
\texttt{\ title\ } & Title of your document \\
\texttt{\ author\ } & Author(s) of your document; can be any content, or
an array of strings \\
\texttt{\ date\ } & Date string to display below the author(s); defaults
to a string of today’s date, but can take any content. Set to
\texttt{\ none\ } if you don’t use it at all. \\
\texttt{\ header\ } & Content that appears in the left-hand side of the
header on every page; this is intended for e.g. the name of a course or
some other contextual information for the document, but can of course
also be left empty. \\
\end{longtable}

Additionally, you can configure some aspects of the template’s layout
with the following arguments:

\begin{longtable}[]{@{}lll@{}}
\toprule\noalign{}
Argument & Description & Default \\
\midrule\noalign{}
\endhead
\bottomrule\noalign{}
\endlastfoot
\texttt{\ paper-size\ } & Paper size of the document &
\texttt{\ "a4"\ } \\
\texttt{\ body-font\ } & Font for the body text &
\texttt{\ "Noto\ Serif"\ } \\
\texttt{\ body-font-size\ } & Font size for the body text &
\texttt{\ 10pt\ } \\
\texttt{\ headings-font\ } & Font for the headings &
\texttt{\ ("Source\ Sans\ Pro",\ "Source\ Sans\ 3")\ } \\
\texttt{\ raw-font\ } & Font for raw (i.e. monospaced) text &
\texttt{\ ("Hack",\ "Source\ Code\ Pro")\ } {[}\^{}1{]} \\
\texttt{\ raw-font-size\ } & Font size for raw text &
\texttt{\ 9pt\ } \\
\texttt{\ link-color\ } & Color for highlighting
\href{https://typst.app/docs/reference/model/link/}{links} &
\texttt{\ rgb("\#3282b8")\ }
\pandocbounded{\includegraphics[keepaspectratio]{https://img.shields.io/badge/steel_blue-3282b8}} \\
\texttt{\ muted-color\ } & Color for muted text, such as page numbers
and headers after the first page & \texttt{\ luma(160)\ } \\
\texttt{\ block-bg-color\ } & Color for the background of raw text &
\texttt{\ luma(240)\ } \\
\end{longtable}

The template will initialize your document with a sample call to the
\texttt{\ kunskap\ } function. Alternatively, here’s a minimal example
of how you could use this template in your document:

\begin{Shaded}
\begin{Highlighting}[]
\NormalTok{\#import "@preview/kunskap:0.1.0": *}

\NormalTok{\#show: kunskap.with(}
\NormalTok{    title: "Your report title",}
\NormalTok{    author: "Your name",}
\NormalTok{    date: datetime.today().display(),}
\NormalTok{    header: "Your course name",}
\NormalTok{)}

\NormalTok{\#lorem(120)}
\end{Highlighting}
\end{Shaded}

\subsection{Missing features}\label{missing-features}

As of now, this template has not yet been particularly optimized for
styling related to:

\begin{itemize}
\tightlist
\item
  Bibliographies
\item
  Outlines (e.g. table of contents)
\item
  Tables
\end{itemize}

\subsection{Credits}\label{credits}

This template started out by emulating the layout of course documents in
\href{https://liu.se/en/employee/marku61}{Marco Kuhlmann} ’s courses
at Linköping University.{[}\^{}2{]} On the technical side, this
template took a lot of inspiration from
\href{https://github.com/talal/ilm/}{the \texttt{\ ilm\ } template} ,
even if the design decisions may be radically different.

{[}\^{}1{]}: The \href{https://github.com/source-foundry/Hack}{Hack
font} is currently not available on the Typst web app, so the fallback
is Source Code Pro. {[}\^{}2{]}: If you work at Linköping University,
you can set \texttt{\ headings-font:\ "KorolevLiU"\ } to get a
LiU-branded version of this template.

\href{/app?template=kunskap&version=0.1.0}{Create project in app}

\subsubsection{How to use}\label{how-to-use}

Click the button above to create a new project using this template in
the Typst app.

You can also use the Typst CLI to start a new project on your computer
using this command:

\begin{verbatim}
typst init @preview/kunskap:0.1.0
\end{verbatim}

\includesvg[width=0.16667in,height=0.16667in]{/assets/icons/16-copy.svg}

\subsubsection{About}\label{about}

\begin{description}
\tightlist
\item[Author :]
\href{mailto:marcel@bollmann.me}{Marcel Bollmann}
\item[License:]
MIT-0
\item[Current version:]
0.1.0
\item[Last updated:]
October 30, 2024
\item[First released:]
October 30, 2024
\item[Minimum Typst version:]
0.12.0
\item[Archive size:]
4.16 kB
\href{https://packages.typst.org/preview/kunskap-0.1.0.tar.gz}{\pandocbounded{\includesvg[keepaspectratio]{/assets/icons/16-download.svg}}}
\item[Repository:]
\href{https://github.com/mbollmann/typst-kunskap}{GitHub}
\item[Categor y :]
\begin{itemize}
\tightlist
\item[]
\item
  \pandocbounded{\includesvg[keepaspectratio]{/assets/icons/16-speak.svg}}
  \href{https://typst.app/universe/search/?category=report}{Report}
\end{itemize}
\end{description}

\subsubsection{Where to report issues?}\label{where-to-report-issues}

This template is a project of Marcel Bollmann . Report issues on
\href{https://github.com/mbollmann/typst-kunskap}{their repository} .
You can also try to ask for help with this template on the
\href{https://forum.typst.app}{Forum} .

Please report this template to the Typst team using the
\href{https://typst.app/contact}{contact form} if you believe it is a
safety hazard or infringes upon your rights.

\phantomsection\label{versions}
\subsubsection{Version history}\label{version-history}

\begin{longtable}[]{@{}ll@{}}
\toprule\noalign{}
Version & Release Date \\
\midrule\noalign{}
\endhead
\bottomrule\noalign{}
\endlastfoot
0.1.0 & October 30, 2024 \\
\end{longtable}

Typst GmbH did not create this template and cannot guarantee correct
functionality of this template or compatibility with any version of the
Typst compiler or app.


\title{typst.app/universe/package/exzellenz-tum-thesis}

\phantomsection\label{banner}
\phantomsection\label{template-thumbnail}
\pandocbounded{\includegraphics[keepaspectratio]{https://packages.typst.org/preview/thumbnails/exzellenz-tum-thesis-0.1.0-small.webp}}

\section{exzellenz-tum-thesis}\label{exzellenz-tum-thesis}

{ 0.1.0 }

Customizable template for a thesis at the TU Munich

\href{/app?template=exzellenz-tum-thesis&version=0.1.0}{Create project
in app}

\phantomsection\label{readme}
This is a Typst template for a thesis at TU Munich. I made it for my
thesis in the School CIT, but I think it can be adapted to other schools
as well.

\subsection{Usage}\label{usage}

You can use this template in the Typst web app by clicking “Start from
template� on the dashboard and searching for
\texttt{\ exzellenz-tum-thesis\ } .

Alternatively, you can use the CLI to kick this project off using the
command

\begin{verbatim}
typst init @preview/exzellenz-tum-thesis
\end{verbatim}

Typst will create a new directory with all the files needed to get you
started.

\subsection{Configuration}\label{configuration}

This template exports the \texttt{\ exzellenz-tum-thesis\ } function
with the following named arguments:

\begin{itemize}
\tightlist
\item
  \texttt{\ degree\ } : String
\item
  \texttt{\ program\ } : String
\item
  \texttt{\ school\ } : String
\item
  \texttt{\ supervisor\ } : String
\item
  \texttt{\ advisor\ } : Array of Strings
\item
  \texttt{\ author\ } : String
\item
  \texttt{\ startDate\ } : String
\item
  \texttt{\ titleEn\ } : String
\item
  \texttt{\ titleDe\ } : String
\item
  \texttt{\ abstractEn\ } : Content block
\item
  \texttt{\ abstractDe\ } : Content block
\item
  \texttt{\ acknowledgements\ } : Content block
\item
  \texttt{\ submissionDate\ } : String
\item
  \texttt{\ showTitleInHeader\ } : Boolean
\item
  \texttt{\ draft\ } : Boolean
\end{itemize}

The template will initialize your package with a sample call to the
\texttt{\ exzellenz-tum-thesis\ } function.

\href{/app?template=exzellenz-tum-thesis&version=0.1.0}{Create project
in app}

\subsubsection{How to use}\label{how-to-use}

Click the button above to create a new project using this template in
the Typst app.

You can also use the Typst CLI to start a new project on your computer
using this command:

\begin{verbatim}
typst init @preview/exzellenz-tum-thesis:0.1.0
\end{verbatim}

\includesvg[width=0.16667in,height=0.16667in]{/assets/icons/16-copy.svg}

\subsubsection{About}\label{about}

\begin{description}
\tightlist
\item[Author :]
\href{https://www.linkedin.com/in/fabian-scherer-de/}{Fabian Scherer}
\item[License:]
MIT-0
\item[Current version:]
0.1.0
\item[Last updated:]
April 8, 2024
\item[First released:]
April 8, 2024
\item[Minimum Typst version:]
0.11.0
\item[Archive size:]
8.18 kB
\href{https://packages.typst.org/preview/exzellenz-tum-thesis-0.1.0.tar.gz}{\pandocbounded{\includesvg[keepaspectratio]{/assets/icons/16-download.svg}}}
\item[Categor y :]
\begin{itemize}
\tightlist
\item[]
\item
  \pandocbounded{\includesvg[keepaspectratio]{/assets/icons/16-mortarboard.svg}}
  \href{https://typst.app/universe/search/?category=thesis}{Thesis}
\end{itemize}
\end{description}

\subsubsection{Where to report issues?}\label{where-to-report-issues}

This template is a project of Fabian Scherer . You can also try to ask
for help with this template on the \href{https://forum.typst.app}{Forum}
.

Please report this template to the Typst team using the
\href{https://typst.app/contact}{contact form} if you believe it is a
safety hazard or infringes upon your rights.

\phantomsection\label{versions}
\subsubsection{Version history}\label{version-history}

\begin{longtable}[]{@{}ll@{}}
\toprule\noalign{}
Version & Release Date \\
\midrule\noalign{}
\endhead
\bottomrule\noalign{}
\endlastfoot
0.1.0 & April 8, 2024 \\
\end{longtable}

Typst GmbH did not create this template and cannot guarantee correct
functionality of this template or compatibility with any version of the
Typst compiler or app.


\title{typst.app/universe/package/whalogen}

\phantomsection\label{banner}
\section{whalogen}\label{whalogen}

{ 0.2.0 }

Typesetting chemical formulae, a port of mhchem

\phantomsection\label{readme}
whalogen is a library for typsetting chemical formulae with Typst,
inspired by mhchem.

GitHub repository: \url{https://github.com/schang412/typst-whalogen}

\subsection{Examples}\label{examples}

\pandocbounded{\includegraphics[keepaspectratio]{https://github.com/typst/packages/raw/main/packages/preview/whalogen/0.2.0/gallery/example.png}}

\begin{Shaded}
\begin{Highlighting}[]
\NormalTok{\#import "@preview/whalogen:0.2.0": ce}

\NormalTok{$ \#ce("HCl + H2O {-}\textgreater{} H3O+ + Cl{-}") $}
\end{Highlighting}
\end{Shaded}

See the
\href{https://github.com/typst/packages/raw/main/packages/preview/whalogen/0.2.0/manual.pdf}{manual}
for more details and examples.

\subsubsection{How to add}\label{how-to-add}

Copy this into your project and use the import as \texttt{\ whalogen\ }

\begin{verbatim}
#import "@preview/whalogen:0.2.0"
\end{verbatim}

\includesvg[width=0.16667in,height=0.16667in]{/assets/icons/16-copy.svg}

Check the docs for
\href{https://typst.app/docs/reference/scripting/\#packages}{more
information on how to import packages} .

\subsubsection{About}\label{about}

\begin{description}
\tightlist
\item[Author :]
\href{mailto:spencer@sycee.xyz}{Spencer Chang}
\item[License:]
Apache-2.0
\item[Current version:]
0.2.0
\item[Last updated:]
April 30, 2024
\item[First released:]
July 18, 2023
\item[Archive size:]
6.45 kB
\href{https://packages.typst.org/preview/whalogen-0.2.0.tar.gz}{\pandocbounded{\includesvg[keepaspectratio]{/assets/icons/16-download.svg}}}
\item[Repository:]
\href{https://github.com/schang412/typst-whalogen}{GitHub}
\end{description}

\subsubsection{Where to report issues?}\label{where-to-report-issues}

This package is a project of Spencer Chang . Report issues on
\href{https://github.com/schang412/typst-whalogen}{their repository} .
You can also try to ask for help with this package on the
\href{https://forum.typst.app}{Forum} .

Please report this package to the Typst team using the
\href{https://typst.app/contact}{contact form} if you believe it is a
safety hazard or infringes upon your rights.

\phantomsection\label{versions}
\subsubsection{Version history}\label{version-history}

\begin{longtable}[]{@{}ll@{}}
\toprule\noalign{}
Version & Release Date \\
\midrule\noalign{}
\endhead
\bottomrule\noalign{}
\endlastfoot
0.2.0 & April 30, 2024 \\
\href{https://typst.app/universe/package/whalogen/0.1.0/}{0.1.0} & July
18, 2023 \\
\end{longtable}

Typst GmbH did not create this package and cannot guarantee correct
functionality of this package or compatibility with any version of the
Typst compiler or app.


\title{typst.app/universe/package/pavemat}

\phantomsection\label{banner}
\section{pavemat}\label{pavemat}

{ 0.1.0 }

Style matrices with custom paths, strokes and fills for appealing
visualizations.

{ } Featured Package

\phantomsection\label{readme}
\pandocbounded{\includesvg[keepaspectratio]{https://github.com/typst/packages/raw/main/packages/preview/pavemat/0.1.0/examples/logo.svg}}

repo: \url{https://github.com/QuadnucYard/pavemat}

\subsection{Introduction}\label{introduction}

The \emph{pavemat} is a tool for creating styled matrices with custom
paths, strokes, and fills. It allows users to define how paths should be
drawn through the matrix, apply different strokes to these paths, and
fill specific cells with various colors. This function is particularly
useful for visualizing complex data structures, mathematical matrices,
and creating custom grid layouts.

\subsection{Examples}\label{examples}

The logo example:

\begin{Shaded}
\begin{Highlighting}[]
\NormalTok{\#\{}
\NormalTok{  set math.mat(row{-}gap: 0.25em, column{-}gap: 0.1em)}
\NormalTok{  set text(size: 2em)}

\NormalTok{  pavemat(}
\NormalTok{    pave: (}
\NormalTok{      "SDS(dash: \textquotesingle{}solid\textquotesingle{})DDD]WW",}
\NormalTok{      (path: "sdDDD", stroke: aqua.darken(30\%))}
\NormalTok{    ),}
\NormalTok{    stroke: (dash: "dashed", thickness: 1pt, paint: yellow),}
\NormalTok{    fills: (}
\NormalTok{      "0{-}0": green.transparentize(80\%),}
\NormalTok{      "1{-}1": blue.transparentize(80\%),}
\NormalTok{      "[0{-}0]": green.transparentize(60\%),}
\NormalTok{      "[1{-}1]": blue.transparentize(60\%),}
\NormalTok{    ),}
\NormalTok{    delim: "[",}
\NormalTok{  )[$mat(P, a, v, e; "", m, a, t)$]}
\NormalTok{\}}
\end{Highlighting}
\end{Shaded}

Code of examples can be found in
\href{https://github.com/QuadnucYard/pavemat/tree/main/examples}{\texttt{\ examples/examples.typ\ }}
.

\pandocbounded{\includesvg[keepaspectratio]{https://github.com/typst/packages/raw/main/packages/preview/pavemat/0.1.0/examples/example1.svg}}
\pandocbounded{\includesvg[keepaspectratio]{https://github.com/typst/packages/raw/main/packages/preview/pavemat/0.1.0/examples/example2.svg}}
\pandocbounded{\includesvg[keepaspectratio]{https://github.com/typst/packages/raw/main/packages/preview/pavemat/0.1.0/examples/example4.svg}}
\pandocbounded{\includesvg[keepaspectratio]{https://github.com/typst/packages/raw/main/packages/preview/pavemat/0.1.0/examples/example5.svg}}

\subsection{Manual}\label{manual}

See
\href{https://github.com/QuadnucYard/pavemat/tree/main/docs}{\texttt{\ docs/manual.typ\ }}
.

\subsubsection{How to add}\label{how-to-add}

Copy this into your project and use the import as \texttt{\ pavemat\ }

\begin{verbatim}
#import "@preview/pavemat:0.1.0"
\end{verbatim}

\includesvg[width=0.16667in,height=0.16667in]{/assets/icons/16-copy.svg}

Check the docs for
\href{https://typst.app/docs/reference/scripting/\#packages}{more
information on how to import packages} .

\subsubsection{About}\label{about}

\begin{description}
\tightlist
\item[Author :]
\href{https://github.com/QuadnucYard}{QuadnucYard}
\item[License:]
MIT
\item[Current version:]
0.1.0
\item[Last updated:]
July 29, 2024
\item[First released:]
July 29, 2024
\item[Archive size:]
3.60 kB
\href{https://packages.typst.org/preview/pavemat-0.1.0.tar.gz}{\pandocbounded{\includesvg[keepaspectratio]{/assets/icons/16-download.svg}}}
\item[Repository:]
\href{https://github.com/QuadnucYard/pavemat}{GitHub}
\item[Categor y :]
\begin{itemize}
\tightlist
\item[]
\item
  \pandocbounded{\includesvg[keepaspectratio]{/assets/icons/16-chart.svg}}
  \href{https://typst.app/universe/search/?category=visualization}{Visualization}
\end{itemize}
\end{description}

\subsubsection{Where to report issues?}\label{where-to-report-issues}

This package is a project of QuadnucYard . Report issues on
\href{https://github.com/QuadnucYard/pavemat}{their repository} . You
can also try to ask for help with this package on the
\href{https://forum.typst.app}{Forum} .

Please report this package to the Typst team using the
\href{https://typst.app/contact}{contact form} if you believe it is a
safety hazard or infringes upon your rights.

\phantomsection\label{versions}
\subsubsection{Version history}\label{version-history}

\begin{longtable}[]{@{}ll@{}}
\toprule\noalign{}
Version & Release Date \\
\midrule\noalign{}
\endhead
\bottomrule\noalign{}
\endlastfoot
0.1.0 & July 29, 2024 \\
\end{longtable}

Typst GmbH did not create this package and cannot guarantee correct
functionality of this package or compatibility with any version of the
Typst compiler or app.


\title{typst.app/universe/package/stonewall}

\phantomsection\label{banner}
\section{stonewall}\label{stonewall}

{ 0.1.0 }

Stonewall provides beautiful pride flag colours for gradients.

\phantomsection\label{readme}
You can use the colour palette with \emph{gradients} for maximum
results! For example the code in \texttt{\ example/example.typ\ } which
is

\begin{Shaded}
\begin{Highlighting}[]
\NormalTok{\#import "@preview/stonewall:0.1.0": flags}

\NormalTok{\#set page(width: 200pt, height: auto, margin: 0pt)}
\NormalTok{\#set text(fill: black, size: 12pt)}
\NormalTok{\#set text(top{-}edge: "bounds", bottom{-}edge: "bounds")}


\NormalTok{\#stack(}
\NormalTok{  spacing: 3pt,}
\NormalTok{  ..flags.map(((name, preset)) =\textgreater{} block(}
\NormalTok{    width: 100\%,}
\NormalTok{    height: 20pt,}
\NormalTok{    fill: gradient.linear(..preset),}
\NormalTok{    align(center + horizon, smallcaps(name)),}
\NormalTok{  ))}
\NormalTok{)}
\end{Highlighting}
\end{Shaded}

gives the following stack of flags as of v0.1.0
\pandocbounded{\includegraphics[keepaspectratio]{https://github.com/typst/packages/raw/main/packages/preview/stonewall/0.1.0/flags.png}}

To use only one flag you only import the one you want

\subsubsection{How to add}\label{how-to-add}

Copy this into your project and use the import as \texttt{\ stonewall\ }

\begin{verbatim}
#import "@preview/stonewall:0.1.0"
\end{verbatim}

\includesvg[width=0.16667in,height=0.16667in]{/assets/icons/16-copy.svg}

Check the docs for
\href{https://typst.app/docs/reference/scripting/\#packages}{more
information on how to import packages} .

\subsubsection{About}\label{about}

\begin{description}
\tightlist
\item[Author :]
Charlotte Thomas
\item[License:]
GPL-3.0-or-later
\item[Current version:]
0.1.0
\item[Last updated:]
November 7, 2023
\item[First released:]
November 7, 2023
\item[Minimum Typst version:]
0.9.0
\item[Archive size:]
14.3 kB
\href{https://packages.typst.org/preview/stonewall-0.1.0.tar.gz}{\pandocbounded{\includesvg[keepaspectratio]{/assets/icons/16-download.svg}}}
\item[Repository:]
\href{https://github.com/coco33920/stonewall}{GitHub}
\end{description}

\subsubsection{Where to report issues?}\label{where-to-report-issues}

This package is a project of Charlotte Thomas . Report issues on
\href{https://github.com/coco33920/stonewall}{their repository} . You
can also try to ask for help with this package on the
\href{https://forum.typst.app}{Forum} .

Please report this package to the Typst team using the
\href{https://typst.app/contact}{contact form} if you believe it is a
safety hazard or infringes upon your rights.

\phantomsection\label{versions}
\subsubsection{Version history}\label{version-history}

\begin{longtable}[]{@{}ll@{}}
\toprule\noalign{}
Version & Release Date \\
\midrule\noalign{}
\endhead
\bottomrule\noalign{}
\endlastfoot
0.1.0 & November 7, 2023 \\
\end{longtable}

Typst GmbH did not create this package and cannot guarantee correct
functionality of this package or compatibility with any version of the
Typst compiler or app.


\title{typst.app/universe/package/splash}

\phantomsection\label{banner}
\section{splash}\label{splash}

{ 0.4.0 }

A library of color palettes for Typst.

\phantomsection\label{readme}
Add a splash of color to your project with these palettes for
\href{https://github.com/typst/typst}{Typst} .

This library provides different color palettes with human-readable names
in Typst dictionaries. Currently there are just a few different palettes
to choose from. Any contributions or suggestions are welcome!

\emph{Note} : \texttt{\ splash\ } is in the
\href{https://github.com/typst/packages}{Typst Package Repository} . See
how to use it in the example below.

\subsection{Usage}\label{usage}

\begin{Shaded}
\begin{Highlighting}[]
\NormalTok{\#import "@preview/splash:0.4.0": xcolor}

\NormalTok{\#box(width: 3em, height: 1em, fill: xcolor.dandelion)}
\end{Highlighting}
\end{Shaded}

\subsection{Documentation}\label{documentation}

See the different colors in the
\href{https://github.com/kaarmu/splash/blob/v0.4.0/doc/main.pdf}{documentation}
.

\subsubsection{How to add}\label{how-to-add}

Copy this into your project and use the import as \texttt{\ splash\ }

\begin{verbatim}
#import "@preview/splash:0.4.0"
\end{verbatim}

\includesvg[width=0.16667in,height=0.16667in]{/assets/icons/16-copy.svg}

Check the docs for
\href{https://typst.app/docs/reference/scripting/\#packages}{more
information on how to import packages} .

\subsubsection{About}\label{about}

\begin{description}
\tightlist
\item[Author :]
kaarmu
\item[License:]
MIT
\item[Current version:]
0.4.0
\item[Last updated:]
July 3, 2024
\item[First released:]
July 2, 2023
\item[Archive size:]
16.3 kB
\href{https://packages.typst.org/preview/splash-0.4.0.tar.gz}{\pandocbounded{\includesvg[keepaspectratio]{/assets/icons/16-download.svg}}}
\item[Repository:]
\href{https://github.com/kaarmu/typst-palettes}{GitHub}
\end{description}

\subsubsection{Where to report issues?}\label{where-to-report-issues}

This package is a project of kaarmu . Report issues on
\href{https://github.com/kaarmu/typst-palettes}{their repository} . You
can also try to ask for help with this package on the
\href{https://forum.typst.app}{Forum} .

Please report this package to the Typst team using the
\href{https://typst.app/contact}{contact form} if you believe it is a
safety hazard or infringes upon your rights.

\phantomsection\label{versions}
\subsubsection{Version history}\label{version-history}

\begin{longtable}[]{@{}ll@{}}
\toprule\noalign{}
Version & Release Date \\
\midrule\noalign{}
\endhead
\bottomrule\noalign{}
\endlastfoot
0.4.0 & July 3, 2024 \\
\href{https://typst.app/universe/package/splash/0.3.0/}{0.3.0} & July 2,
2023 \\
\end{longtable}

Typst GmbH did not create this package and cannot guarantee correct
functionality of this package or compatibility with any version of the
Typst compiler or app.


\title{typst.app/universe/package/aloecius-aip}

\phantomsection\label{banner}
\phantomsection\label{template-thumbnail}
\pandocbounded{\includegraphics[keepaspectratio]{https://packages.typst.org/preview/thumbnails/aloecius-aip-0.0.1-small.webp}}

\section{aloecius-aip}\label{aloecius-aip}

{ 0.0.1 }

Typst template for reproducing AIP - Journal of Chemical Physics paper
(draft)

\href{/app?template=aloecius-aip&version=0.0.1}{Create project in app}

\phantomsection\label{readme}
This is a typst template for reproducing papers of American Institute of
Physics (AIP) publishing house, mainly draft version of Journal of
Chemical Physics. This is inspired by the overleaf
\$\textbackslash LaTeX\$ template of AIP journals.

\subsection{Usage}\label{usage}

You can use this template with typst web app by simply clicking on
“Start from template� on the dashboard and searching for
\texttt{\ aloecius-aip\ } .

For local usage, you can use the typst CLI by invoking the following
command

\begin{verbatim}
typst init @preview/aloecius-aip
\end{verbatim}

typst will automatically create a new directory with all the necessary
files needed to compile the project.

\subsection{Configuration}\label{configuration}

The preamble or the header of the document should be written in the
following way with your own necessary input variables to recreate the
same formatting as seen in the
\href{https://github.com/typst/packages/raw/main/packages/preview/aloecius-aip/0.0.1/sample.pdf}{\texttt{\ sample.pdf\ }}

\begin{verbatim}
#import "@preview/aloecius-aip:0.0.1": *

#show: article.with(
  title: "Typst Template for Journal of Chemical Physics (Draft)",
  authors: (
    "Author 1": author-meta(
      "GU",
      email: "user1@domain.com",
    ),
    "Author 2": author-meta(
      "GU",
      cofirst: false
    ),
    "Author 3": author-meta(
      "UG"
    )
  ),
  affiliations: (
    "UG": "University of Global Sciences",
    "GU": "Institute for Chemistry, Global University of Sciences"
  ),
  abstract: [
  Here goes the abstract. 
  ],
  bib: bibliography("./reference.bib")
)
\end{verbatim}

\subsection{Important Variables}\label{important-variables}

\begin{itemize}
\tightlist
\item
  \texttt{\ title\ } : Title of the paper
\item
  \texttt{\ authors\ } : A dictionary connecting the key as name of the
  author(s) and the value to be the affiliation of them including
  university, email, mail address, authorship and an alias, an example
  usage is shown below
\end{itemize}

\begin{verbatim}
Example:
(
  "Author Name": (
    "affiliation": "affiliation-label",
    "email": "author.name@example.com", // Optional
    "address": "Mail address",  // Optional
    "name": "Alias Name", // Optional
    "cofirst": false // Optional, identify whether this author is the co-first author
    )
)
\end{verbatim}

\begin{itemize}
\tightlist
\item
  \texttt{\ affiliations\ } : Dictionary of affiliations where keys are
  affiliations labels and values are affiliations addresses, and example
  usage is as follows
\end{itemize}

\begin{verbatim}
Example:
 (
    "affiliation-label": "Institution Name, University Name, Road, Post Code, Country"
 )
\end{verbatim}

\begin{itemize}
\tightlist
\item
  \texttt{\ abstract\ } : Abstract of the paper
\item
  \texttt{\ bib\ } : passing the bibliography file wrapped into the
  typst \texttt{\ bibliography()\ } function, both
  \texttt{\ Hayagriva\ } and \texttt{\ .bib\ } format is supported.
\end{itemize}

\href{/app?template=aloecius-aip&version=0.0.1}{Create project in app}

\subsubsection{How to use}\label{how-to-use}

Click the button above to create a new project using this template in
the Typst app.

You can also use the Typst CLI to start a new project on your computer
using this command:

\begin{verbatim}
typst init @preview/aloecius-aip:0.0.1
\end{verbatim}

\includesvg[width=0.16667in,height=0.16667in]{/assets/icons/16-copy.svg}

\subsubsection{About}\label{about}

\begin{description}
\tightlist
\item[Author :]
Raunak Farhaz
\item[License:]
MIT
\item[Current version:]
0.0.1
\item[Last updated:]
July 3, 2024
\item[First released:]
July 3, 2024
\item[Minimum Typst version:]
0.11.1
\item[Archive size:]
13.7 kB
\href{https://packages.typst.org/preview/aloecius-aip-0.0.1.tar.gz}{\pandocbounded{\includesvg[keepaspectratio]{/assets/icons/16-download.svg}}}
\item[Repository:]
\href{https://github.com/Raunak12775/aloecius-aip}{GitHub}
\item[Discipline s :]
\begin{itemize}
\tightlist
\item[]
\item
  \href{https://typst.app/universe/search/?discipline=chemistry}{Chemistry}
\item
  \href{https://typst.app/universe/search/?discipline=physics}{Physics}
\item
  \href{https://typst.app/universe/search/?discipline=mathematics}{Mathematics}
\end{itemize}
\item[Categor y :]
\begin{itemize}
\tightlist
\item[]
\item
  \pandocbounded{\includesvg[keepaspectratio]{/assets/icons/16-atom.svg}}
  \href{https://typst.app/universe/search/?category=paper}{Paper}
\end{itemize}
\end{description}

\subsubsection{Where to report issues?}\label{where-to-report-issues}

This template is a project of Raunak Farhaz . Report issues on
\href{https://github.com/Raunak12775/aloecius-aip}{their repository} .
You can also try to ask for help with this template on the
\href{https://forum.typst.app}{Forum} .

Please report this template to the Typst team using the
\href{https://typst.app/contact}{contact form} if you believe it is a
safety hazard or infringes upon your rights.

\phantomsection\label{versions}
\subsubsection{Version history}\label{version-history}

\begin{longtable}[]{@{}ll@{}}
\toprule\noalign{}
Version & Release Date \\
\midrule\noalign{}
\endhead
\bottomrule\noalign{}
\endlastfoot
0.0.1 & July 3, 2024 \\
\end{longtable}

Typst GmbH did not create this template and cannot guarantee correct
functionality of this template or compatibility with any version of the
Typst compiler or app.


