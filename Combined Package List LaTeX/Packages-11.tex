\title{typst.app/universe/package/typearea}

\phantomsection\label{banner}
\section{typearea}\label{typearea}

{ 0.2.0 }

A KOMA-Script inspired package to better configure your typearea and
margins.

\phantomsection\label{readme}
A KOMA-Script inspired package to better configure your typearea and
margins.

\begin{Shaded}
\begin{Highlighting}[]
\NormalTok{\#import "@preview/typearea:0.2.0": typearea}

\NormalTok{\#show: typearea.with(}
\NormalTok{  paper: "a4",}
\NormalTok{  div: 9,}
\NormalTok{  binding{-}correction: 11mm,}
\NormalTok{)}

\NormalTok{= Hello World}
\end{Highlighting}
\end{Shaded}

\subsection{Reference}\label{reference}

\texttt{\ typearea\ } accepts the following options:

\subsubsection{two-sided}\label{two-sided}

Whether the document is two-sided. Defaults to \texttt{\ true\ } .

\subsubsection{binding-correction}\label{binding-correction}

Binding correction. Defaults to \texttt{\ 0pt\ } .

Additional margin on the inside of a page when two-sided is true. If
two-sided is false it will be on the left or right side, depending on
the value of \texttt{\ binding\ } . A \texttt{\ binding\ } value of
\texttt{\ auto\ } will currently default to \texttt{\ left\ } .

Note: Before version 0.2.0 this was called \texttt{\ bcor\ } .

\subsubsection{div}\label{div}

How many equal parts to split the page into. Controls the margins.
Defautls to \texttt{\ 9\ } .

The top and bottom margin will always be one and two parts respectively.
In two-sided mode the inside margin will be one part and the outside
margin two parts, so the combined margins between the text on the left
side and the text on the right side is the same as the margins from the
outer edge of the text to the outer edge of the page.

In one-sided mode the left and right margin will take 1.5 parts each.

\subsubsection{header-height /
footer-height}\label{header-height-footer-height}

The height of the page header/footer.

\subsubsection{header-sep / footer-sep}\label{header-sep-footer-sep}

The distance between the page header/footer and the text area.

\subsubsection{header-include /
footer-include}\label{header-include-footer-include}

Whether the header/footer should be counted as part of the text area
when calculating the margins. Defaults to \texttt{\ false\ } .

\subsubsection{…rest}\label{uxe2rest}

All other arguments are passed on to \texttt{\ page()\ } as is. You can
see which arguments \texttt{\ page()\ } accepts in the
\href{https://typst.app/docs/reference/layout/page/}{typst reference for
the page function} .

You should prefer this over calling or setting \texttt{\ page()\ }
directly as doing so could break some of \texttt{\ typearea\ } ’s
functionality.

\subsubsection{How to add}\label{how-to-add}

Copy this into your project and use the import as \texttt{\ typearea\ }

\begin{verbatim}
#import "@preview/typearea:0.2.0"
\end{verbatim}

\includesvg[width=0.16667in,height=0.16667in]{/assets/icons/16-copy.svg}

Check the docs for
\href{https://typst.app/docs/reference/scripting/\#packages}{more
information on how to import packages} .

\subsubsection{About}\label{about}

\begin{description}
\tightlist
\item[Author :]
Adrian Freund
\item[License:]
MIT
\item[Current version:]
0.2.0
\item[Last updated:]
June 13, 2024
\item[First released:]
October 29, 2023
\item[Archive size:]
2.39 kB
\href{https://packages.typst.org/preview/typearea-0.2.0.tar.gz}{\pandocbounded{\includesvg[keepaspectratio]{/assets/icons/16-download.svg}}}
\item[Repository:]
\href{https://github.com/freundTech/typst-typearea}{GitHub}
\end{description}

\subsubsection{Where to report issues?}\label{where-to-report-issues}

This package is a project of Adrian Freund . Report issues on
\href{https://github.com/freundTech/typst-typearea}{their repository} .
You can also try to ask for help with this package on the
\href{https://forum.typst.app}{Forum} .

Please report this package to the Typst team using the
\href{https://typst.app/contact}{contact form} if you believe it is a
safety hazard or infringes upon your rights.

\phantomsection\label{versions}
\subsubsection{Version history}\label{version-history}

\begin{longtable}[]{@{}ll@{}}
\toprule\noalign{}
Version & Release Date \\
\midrule\noalign{}
\endhead
\bottomrule\noalign{}
\endlastfoot
0.2.0 & June 13, 2024 \\
\href{https://typst.app/universe/package/typearea/0.1.0/}{0.1.0} &
October 29, 2023 \\
\end{longtable}

Typst GmbH did not create this package and cannot guarantee correct
functionality of this package or compatibility with any version of the
Typst compiler or app.


\title{typst.app/universe/package/super-suboptimal}

\phantomsection\label{banner}
\section{super-suboptimal}\label{super-suboptimal}

{ 0.1.0 }

Unicode super- and subscripts in equations.

\phantomsection\label{readme}
A Typst package enabling support for Unicode super- and subscript
characters in equations.

\subsection{Usage}\label{usage}

The package exposes the template-function \texttt{\ super-subscripts\ }
. It affects all \texttt{\ math.equation\ } s by attaching every
superscript- and subscript-character to the first non-space-element on
its left.

\begin{Shaded}
\begin{Highlighting}[]
\NormalTok{\#import "@preview/super{-}suboptimal:0.1.0": *}
\NormalTok{\#show: super{-}subscripts}

\NormalTok{For all $(x,y)∈ℝ²$:}
\NormalTok{$}
\NormalTok{  q := norm((x,y))₂ \textless{} 1}
\NormalTok{  ==\textgreater{} ∑ᵢ₌₁ⁿ q ⁱ \textless{} ∞}
\NormalTok{$}
\end{Highlighting}
\end{Shaded}

\pandocbounded{\includesvg[keepaspectratio]{https://github.com/typst/packages/raw/main/packages/preview/super-suboptimal/0.1.0/assets/example0.svg}}

Because code like \texttt{\ \$x+yᶻ\$\ } throws an “unknown
variable� error, the package also exposes the function \texttt{\ eq\ }
, which inserts spaces before every superscript- and subscript-character
and passing it on to \texttt{\ math.equation\ } . This comes at the cost
of missing syntax-highlighting and code-suggestions in your IDE.

\texttt{\ eq\ } accepts a \texttt{\ raw\ } string as a positional
parameter, and an argument-sink that’s passed onto
\texttt{\ math.equation\ } . Unless specified otherwise in the
argument-sink, the resulting equation is typeset with
\texttt{\ block:\ true\ } if and only if the \texttt{\ raw\ } also
satisfied \texttt{\ block:\ true\ } . \texttt{\ eq\ } is automatically
applied to every \texttt{\ raw\ } with \texttt{\ lang:\ "eq"\ } .

\begin{Shaded}
\begin{Highlighting}[]
\NormalTok{\#eq(\textasciigrave{}0 = aᵇ\textasciigrave{})}

\NormalTok{\#eq(\textasciigrave{}\textasciigrave{}\textasciigrave{}}
\NormalTok{  1 = x+yᶻ}
\NormalTok{\textasciigrave{}\textasciigrave{}\textasciigrave{})}

\NormalTok{\#eq(\textasciigrave{}2 = aᵇ\textasciigrave{}, block: true, numbering: "(1)")}

\NormalTok{\textasciigrave{}\textasciigrave{}\textasciigrave{}eq}
\NormalTok{  3 = aᵇᶜ⁺ᵈ₃ₑ⁽ᶠ⁻ᵍ⁾ₕᵢ}
\NormalTok{\textasciigrave{}\textasciigrave{}\textasciigrave{}}
\end{Highlighting}
\end{Shaded}

\pandocbounded{\includesvg[keepaspectratio]{https://github.com/typst/packages/raw/main/packages/preview/super-suboptimal/0.1.0/assets/example1.svg}}

Sometimes in mathematical writing, variables are decorated with an
asterisk, e.g. \texttt{\ \$x\^{}*\$\ } . The character \texttt{\ ꙳\ }
can now be used, as well: \texttt{\ \$x꙳\ =\ x\^{}*\$\ } .

\subsection{Known issues}\label{known-issues}

\begin{itemize}
\item
  As mentioned above, \texttt{\ \$aᵇ\$\ } leads to an “unknown
  variable� error. As a workaround, \texttt{\ \$a\ ᵇ\$\ } produces
  the same output, or you can use the \texttt{\ eq\ } function described
  above.

  \begin{itemize}
  \tightlist
  \item
    The first workaround also means I can’t reasonably implement
    top-left and bottom-left attachments. For example,
    \texttt{\ \$a\ ³b\$\ } is rendered like
    \texttt{\ \$attach(a,\ t:\ 3)\ b\$\ } , rather than
    \texttt{\ \$a\ attach(b,\ tl:\ 3)\$\ } .
  \end{itemize}
\item
  Multiple attachments are concatenated into one content without another
  pass of \texttt{\ equation\ } . For example,
  \texttt{\ \#eq(\textasciigrave{}0ˢ�����\textasciigrave{})\ }
  is equivalent to \texttt{\ \$0\^{}(s\ i\ n\ "("\ k\ ")")\$\ } , rather
  than \texttt{\ \$0\^{}sin(k)\$\ } . I won’t fix this, because:

  \begin{itemize}
  \tightlist
  \item
    Another pass of \texttt{\ equation\ } would cause performance issues
    at best, and infinite loops at worst.
  \item
    If this were fixed, code such as \texttt{\ \$e\ ˣ\ ʸ\$\ } would
    undesirably produce an “unknown variable \texttt{\ xy\ } �
    error.
  \end{itemize}
\item
  Let’s call a piece of content “small� if it consists of only a
  single non-separated sequence of characters in Typst (internally, this
  is the distinction between the content-functions \texttt{\ sequence\ }
  and \texttt{\ text\ } ). For instance, \texttt{\ \$1234\$\ } and
  \texttt{\ \$a\$\ } constitute “small� content, but
  \texttt{\ \$a\ b\$\ } and \texttt{\ \$3a\$\ } and
  \texttt{\ \$1+2+3+4+5\$\ } do not.

  This package only runs on non-“small� pieces of content. For
  example, \texttt{\ \$sqrt(35²)\$\ } still renders with the
  default-Unicode-character and will look different from
  \texttt{\ \$sqrt(35\^{}2)\$\ } . On the other hand,
  \texttt{\ \$sqrt(aâ?¶)\$\ } \emph{is} rendered correctly. This is
  because \texttt{\ 35²\ } constitutes “small� content, but
  \texttt{\ aâ?¶\ } does not.

  A workaround is implemented for “small� content immediately within
  an equation, i.e. not nested within another content-function. For
  example, \texttt{\ \$7²\$\ } renders the same as
  \texttt{\ \$7\^{}2\$\ } , even though it’s “small� content.
\item
  Equations within other content-elements might trigger multiple
  show-rule-passes, possibly causing performance-issues.
\end{itemize}

\subsubsection{How to add}\label{how-to-add}

Copy this into your project and use the import as
\texttt{\ super-suboptimal\ }

\begin{verbatim}
#import "@preview/super-suboptimal:0.1.0"
\end{verbatim}

\includesvg[width=0.16667in,height=0.16667in]{/assets/icons/16-copy.svg}

Check the docs for
\href{https://typst.app/docs/reference/scripting/\#packages}{more
information on how to import packages} .

\subsubsection{About}\label{about}

\begin{description}
\tightlist
\item[Author s :]
Eric Biedert \& Lumi
\item[License:]
MIT
\item[Current version:]
0.1.0
\item[Last updated:]
January 29, 2024
\item[First released:]
January 29, 2024
\item[Archive size:]
6.15 kB
\href{https://packages.typst.org/preview/super-suboptimal-0.1.0.tar.gz}{\pandocbounded{\includesvg[keepaspectratio]{/assets/icons/16-download.svg}}}
\end{description}

\subsubsection{Where to report issues?}\label{where-to-report-issues}

This package is a project of Eric Biedert and Lumi . You can also try to
ask for help with this package on the
\href{https://forum.typst.app}{Forum} .

Please report this package to the Typst team using the
\href{https://typst.app/contact}{contact form} if you believe it is a
safety hazard or infringes upon your rights.

\phantomsection\label{versions}
\subsubsection{Version history}\label{version-history}

\begin{longtable}[]{@{}ll@{}}
\toprule\noalign{}
Version & Release Date \\
\midrule\noalign{}
\endhead
\bottomrule\noalign{}
\endlastfoot
0.1.0 & January 29, 2024 \\
\end{longtable}

Typst GmbH did not create this package and cannot guarantee correct
functionality of this package or compatibility with any version of the
Typst compiler or app.


\title{typst.app/universe/package/note-me}

\phantomsection\label{banner}
\section{note-me}\label{note-me}

{ 0.3.0 }

Adds GitHub-style Admonitions (Alerts) to Typst.

\phantomsection\label{readme}
\begin{quote}
{[}!NOTE{]} Add GitHub style admonitions (also known as alerts) to
Typst.
\end{quote}

\subsection{Usage}\label{usage}

Import this package, and do

\begin{Shaded}
\begin{Highlighting}[]
\NormalTok{// Import from @preview namespace is suggested}
\NormalTok{// \#import "@preview/note{-}me:0.3.0": *}

\NormalTok{// Import from @local namespace is only for debugging purpose}
\NormalTok{// \#import "@local/note{-}me:0.3.0": *}

\NormalTok{// Import relatively is for development purpose}
\NormalTok{\#import "lib.typ": *}

\NormalTok{= Basic Examples}

\NormalTok{\#note[}
\NormalTok{  Highlights information that users should take into account, even when skimming.}
\NormalTok{]}

\NormalTok{\#tip[}
\NormalTok{  Optional information to help a user be more successful.}
\NormalTok{]}

\NormalTok{\#important[}
\NormalTok{  Crucial information necessary for users to succeed.}
\NormalTok{]}

\NormalTok{\#warning[}
\NormalTok{  Critical content demanding immediate user attention due to potential risks.}
\NormalTok{]}

\NormalTok{\#caution[}
\NormalTok{  Negative potential consequences of an action.}
\NormalTok{]}

\NormalTok{\#admonition(}
\NormalTok{  icon{-}path: "icons/stop.svg",}
\NormalTok{  color: color.fuchsia,}
\NormalTok{  title: "Customize",}
\NormalTok{  foreground{-}color: color.white,}
\NormalTok{  background{-}color: color.black,}
\NormalTok{)[}
\NormalTok{  The icon, (theme) color, title, foreground and background color are customizable.}
\NormalTok{]}

\NormalTok{\#admonition(}
\NormalTok{  icon{-}string: read("icons/light{-}bulb.svg"),}
\NormalTok{  color: color.fuchsia,}
\NormalTok{  title: "Customize",}
\NormalTok{)[}
\NormalTok{  The icon can be specified as a string of SVG. This is useful if the user want to use an SVG icon that is not available in this package.}
\NormalTok{]}

\NormalTok{\#admonition(}
\NormalTok{  icon: [🙈],}
\NormalTok{  color: color.fuchsia,}
\NormalTok{  title: "Customize",}
\NormalTok{)[}
\NormalTok{  Or, pass a content directly as the icon...}
\NormalTok{]}

\NormalTok{= More Examples}

\NormalTok{\#todo[}
\NormalTok{  Fix \textasciigrave{}note{-}me\textasciigrave{} package.}
\NormalTok{]}


\NormalTok{= Prevent Page Breaks from Breaking Admonitions}

\NormalTok{\#box(}
\NormalTok{  width: 1fr,}
\NormalTok{  height: 50pt,}
\NormalTok{  fill: gray,}
\NormalTok{)}

\NormalTok{\#note[}
\NormalTok{  \#lorem(100)}
\NormalTok{]}
\end{Highlighting}
\end{Shaded}

\pandocbounded{\includesvg[keepaspectratio]{https://github.com/typst/packages/raw/main/packages/preview/note-me/0.3.0/example.svg}}

Further Reading:

\begin{itemize}
\tightlist
\item
  \url{https://github.com/orgs/community/discussions/16925}
\item
  \url{https://docs.asciidoctor.org/asciidoc/latest/blocks/admonitions/}
\end{itemize}

\subsection{Style}\label{style}

It borrows the style of GitHub’s admonition.

\begin{quote}
{[}!NOTE{]}\\
Highlights information that users should take into account, even when
skimming.
\end{quote}

\begin{quote}
{[}!TIP{]} Optional information to help a user be more successful.
\end{quote}

\begin{quote}
{[}!IMPORTANT{]}\\
Crucial information necessary for users to succeed.
\end{quote}

\begin{quote}
{[}!WARNING{]}\\
Critical content demanding immediate user attention due to potential
risks.
\end{quote}

\begin{quote}
{[}!CAUTION{]} Negative potential consequences of an action.
\end{quote}

\subsection{Credits}\label{credits}

The admonition icons are from
\href{https://github.com/primer/octicons}{Octicons} .

\subsubsection{How to add}\label{how-to-add}

Copy this into your project and use the import as \texttt{\ note-me\ }

\begin{verbatim}
#import "@preview/note-me:0.3.0"
\end{verbatim}

\includesvg[width=0.16667in,height=0.16667in]{/assets/icons/16-copy.svg}

Check the docs for
\href{https://typst.app/docs/reference/scripting/\#packages}{more
information on how to import packages} .

\subsubsection{About}\label{about}

\begin{description}
\tightlist
\item[Author :]
Flandia Yingman
\item[License:]
MIT
\item[Current version:]
0.3.0
\item[Last updated:]
September 30, 2024
\item[First released:]
February 11, 2024
\item[Archive size:]
5.02 kB
\href{https://packages.typst.org/preview/note-me-0.3.0.tar.gz}{\pandocbounded{\includesvg[keepaspectratio]{/assets/icons/16-download.svg}}}
\item[Repository:]
\href{https://github.com/FlandiaYingman/note-me}{GitHub}
\end{description}

\subsubsection{Where to report issues?}\label{where-to-report-issues}

This package is a project of Flandia Yingman . Report issues on
\href{https://github.com/FlandiaYingman/note-me}{their repository} . You
can also try to ask for help with this package on the
\href{https://forum.typst.app}{Forum} .

Please report this package to the Typst team using the
\href{https://typst.app/contact}{contact form} if you believe it is a
safety hazard or infringes upon your rights.

\phantomsection\label{versions}
\subsubsection{Version history}\label{version-history}

\begin{longtable}[]{@{}ll@{}}
\toprule\noalign{}
Version & Release Date \\
\midrule\noalign{}
\endhead
\bottomrule\noalign{}
\endlastfoot
0.3.0 & September 30, 2024 \\
\href{https://typst.app/universe/package/note-me/0.2.1/}{0.2.1} & March
8, 2024 \\
\href{https://typst.app/universe/package/note-me/0.1.1/}{0.1.1} &
February 25, 2024 \\
\href{https://typst.app/universe/package/note-me/0.1.0/}{0.1.0} &
February 11, 2024 \\
\end{longtable}

Typst GmbH did not create this package and cannot guarantee correct
functionality of this package or compatibility with any version of the
Typst compiler or app.


\title{typst.app/universe/package/tenv}

\phantomsection\label{banner}
\section{tenv}\label{tenv}

{ 0.1.1 }

Parse a .env content.

\phantomsection\label{readme}
Parse a .env content.

\subsection{Usage}\label{usage}

\begin{Shaded}
\begin{Highlighting}[]
\NormalTok{\#import "@preview/tenv.typ:0.1.1": parse\_dotenv}

\NormalTok{\#let env = parse\_dotenv(read(".env"))}
\end{Highlighting}
\end{Shaded}

\subsection{Example}\label{example}

\pandocbounded{\includegraphics[keepaspectratio]{https://github.com/typst/packages/raw/main/packages/preview/tenv/0.1.1/example.png}}

\subsubsection{How to add}\label{how-to-add}

Copy this into your project and use the import as \texttt{\ tenv\ }

\begin{verbatim}
#import "@preview/tenv:0.1.1"
\end{verbatim}

\includesvg[width=0.16667in,height=0.16667in]{/assets/icons/16-copy.svg}

Check the docs for
\href{https://typst.app/docs/reference/scripting/\#packages}{more
information on how to import packages} .

\subsubsection{About}\label{about}

\begin{description}
\tightlist
\item[Author :]
chillcicada
\item[License:]
MIT
\item[Current version:]
0.1.1
\item[Last updated:]
May 16, 2024
\item[First released:]
May 16, 2024
\item[Archive size:]
1.42 kB
\href{https://packages.typst.org/preview/tenv-0.1.1.tar.gz}{\pandocbounded{\includesvg[keepaspectratio]{/assets/icons/16-download.svg}}}
\item[Repository:]
\href{https://github.com/chillcicada/typst-dotenv}{GitHub}
\end{description}

\subsubsection{Where to report issues?}\label{where-to-report-issues}

This package is a project of chillcicada . Report issues on
\href{https://github.com/chillcicada/typst-dotenv}{their repository} .
You can also try to ask for help with this package on the
\href{https://forum.typst.app}{Forum} .

Please report this package to the Typst team using the
\href{https://typst.app/contact}{contact form} if you believe it is a
safety hazard or infringes upon your rights.

\phantomsection\label{versions}
\subsubsection{Version history}\label{version-history}

\begin{longtable}[]{@{}ll@{}}
\toprule\noalign{}
Version & Release Date \\
\midrule\noalign{}
\endhead
\bottomrule\noalign{}
\endlastfoot
0.1.1 & May 16, 2024 \\
\end{longtable}

Typst GmbH did not create this package and cannot guarantee correct
functionality of this package or compatibility with any version of the
Typst compiler or app.


\title{typst.app/universe/package/scholarly-tauthesis}

\phantomsection\label{banner}
\phantomsection\label{template-thumbnail}
\pandocbounded{\includegraphics[keepaspectratio]{https://packages.typst.org/preview/thumbnails/scholarly-tauthesis-0.9.0-small.webp}}

\section{scholarly-tauthesis}\label{scholarly-tauthesis}

{ 0.9.0 }

A template for writing Tampere University theses.

\href{/app?template=scholarly-tauthesis&version=0.9.0}{Create project in
app}

\phantomsection\label{readme}
This is a TAU thesis template written in the
\href{https://github.com/typst/typst}{\texttt{\ typst\ }} typesetting
language, a potential successor to LaTeΧ. The version of typst used to
test this template is
\href{https://github.com/typst/typst/releases/tag/v0.12.0}{\texttt{\ 0.12.0\ }}
.

\subsection{Using the template on
typst.app}\label{using-the-template-on-typst.app}

This template is also available on
\href{https://typst.app/universe}{Typst Universe} as
\href{https://typst.app/universe/package/scholarly-tauthesis}{\texttt{\ scholarly-tauthesis\ }}
. Simply create an account on \url{https://typst.app/} and start a new
\texttt{\ scholarly-tauthesis\ } project by clicking on \textbf{Start
from template} and searching for \textbf{scholarly-tauthesis} .

If you have initialized your project with an older stable version of
this template and wish to upgrade to a newer release, the simplest way
to do it is to change the value of \texttt{\ \$VERSION\ } ≥
\texttt{\ 0.9.0\ } in the import statements

\begin{Shaded}
\begin{Highlighting}[]
\NormalTok{\#import "@preview/scholarly{-}tauthesis:$VERSION" as tauthesis}
\end{Highlighting}
\end{Shaded}

to correspond to a newer released version. Alternatively, you could
download the \texttt{\ tauthesis.typ\ } file from the
\href{https://gitlab.com/tuni-official/thesis-templates/tau-typst-thesis-template}{thesis
template repository} , and upload it into you project on
\url{https://typst.app/} . Then use

\begin{Shaded}
\begin{Highlighting}[]
\NormalTok{\#import "path/to/tauthesis.typ" as tauthesis}
\end{Highlighting}
\end{Shaded}

instead of

\begin{Shaded}
\begin{Highlighting}[]
\NormalTok{\#import "@preview/scholarly{-}tauthesis:$VERSION" as tauthesis}
\end{Highlighting}
\end{Shaded}

to import the \texttt{\ tauthesis\ } module.

\subsubsection{Note}\label{note}

Versions of this template before 0.9.0 do not actually work with
typst.app due to a packaging issue.

\subsection{Local installation}\label{local-installation}

If \href{https://typst.app/universe}{Typst Universe} is online, this
template will be downloaded automatically to

\begin{verbatim}
$CACHEDIR/typst/packages/preview/scholarly-tauthesis/$VERSION/
\end{verbatim}

when one runs the command

\begin{verbatim}
typst init @preview/scholarly-tauthesis:$VERSION mythesis
\end{verbatim}

Here \texttt{\ \$VERSION\ } should be ≥ 0.9.0. The value
\texttt{\ \$CACHEDIR\ } for your OS can be discovered from
\url{https://docs.rs/dirs/latest/dirs/fn.cache_dir.html} .

For a manual installation, download the contents of this repository via
Git or as a ZIP file from the template
\href{https://gitlab.com/tuni-official/thesis-templates/tau-typst-thesis-template/-/tags}{tags}
page. Then, make a symbolic link

\begin{verbatim}
$DATADIR/typst/packages/preview/scholarly-tauthesis/$VERSION/ → /path/to/root/of/tauthesis/
\end{verbatim}

so that a local installation of \texttt{\ typst\ } can discover the
\texttt{\ tauthesis.typ\ } file no matter where you are running it from.
To find out the value \texttt{\ \$DATADIR\ } for your operating system,
see \url{https://docs.rs/dirs/latest/dirs/fn.data_dir.html} . The value
\texttt{\ \$VERSION\ } is the version \texttt{\ A.B.C\ } ≥
\texttt{\ 0.9.0\ } of this template you wish to use.

Once the package has been installed, the command

\begin{verbatim}
typst init @preview/scholarly-tauthesis:$VERSION mythesis
\end{verbatim}

creates a folder \texttt{\ mythesis\ } with the template files in place.
Simply make the \texttt{\ mythesis\ } folder you current working
directory and run

\begin{Shaded}
\begin{Highlighting}[]
\ExtensionTok{typst}\NormalTok{ compile main.typ}
\end{Highlighting}
\end{Shaded}

in the shell of your choice to compile the document from scratch.
Alternatively, type

\begin{Shaded}
\begin{Highlighting}[]
\ExtensionTok{typst}\NormalTok{ watch main.typ }\OperatorTok{\&\textgreater{}}\NormalTok{ typst.log }\KeywordTok{\&}
\end{Highlighting}
\end{Shaded}

to have a \href{https://github.com/typst/typst}{\texttt{\ typst\ }}
process watch the file for changes and compile it when a file is
changed. Possible error messages can then be viewed by checking the
contents of the mentioned file \texttt{\ typst.log\ } .

This template can also be uploaded to the typst online editor at
\url{https://typst.app/} . However, the file paths related to the
\texttt{\ tauthesis\ } file will need to be changed if this is done
manually. See the tutorial at \url{https://typst.app/docs/tutorial/} to
learn the basics of the language. Some examples are also given in the
template itself.

\subsection{Archiving the final version of your
work}\label{archiving-the-final-version-of-your-work}

Before submitting your thesis to the university archives, it needs to be
converted to PDF/A format. Typst versions ≥ 0.12.0 should support the
creation of PDF/A-2b files, when run with the command

\begin{Shaded}
\begin{Highlighting}[]
\ExtensionTok{typst}\NormalTok{ compile }\AttributeTok{{-}{-}pdf{-}standard}\NormalTok{ a{-}2b template/main.typ}
\end{Highlighting}
\end{Shaded}

If a verification program such as
\href{https://docs.verapdf.org/install/}{veraPDF} still complains that
the file \texttt{\ template/main.pdf\ } does not conform to the
standard, the Muuntaja-service of Tampere University should be used to
do the final conversion. See the related instructions (
\href{https://libguides.tuni.fi/opinnaytteet/pdfa}{link} ) for how to do
it. Basically it boils down to feeding your compiled PDF document to the
converter at \href{https://muuntaja.tuni.fi/}{https://muuntaja.tuni.fi}
. \textbf{Remember to check that the output of the converter is not
corrupted, before submitting your thesis to the archives.}

\subsection{Usage}\label{usage}

You can either write your entire \emph{main matter} in the
\href{https://github.com/typst/packages/raw/main/packages/preview/scholarly-tauthesis/0.9.0/template/main.typ}{\texttt{\ main.typ\ }}
file, or more preferrably, split it into multiple chapter-specific files
and place those in the
\href{https://github.com/typst/packages/raw/main/packages/preview/scholarly-tauthesis/0.9.0/template/content}{\texttt{\ contents/\ }}
folder, which this template tries to demonstrate. If you choose to write
your own commands (functions) in the
\href{https://github.com/typst/packages/raw/main/packages/preview/scholarly-tauthesis/0.9.0/template/preamble.typ}{\texttt{\ preamble.typ\ }}
file, this needs to be imported at the start of each chapter you plan to
use the commands in. Sections that come before the main matter, like the
Finnish and English abstracts (
\href{https://github.com/typst/packages/raw/main/packages/preview/scholarly-tauthesis/0.9.0/template/content/tiivistelm\%C3\%A4.typ}{\texttt{\ tiivistelmä.typ\ }}
\textbar{}
\href{https://github.com/typst/packages/raw/main/packages/preview/scholarly-tauthesis/0.9.0/template/content/abstract.typ}{\texttt{\ abstract.typ\ }}
) and
\href{https://github.com/typst/packages/raw/main/packages/preview/scholarly-tauthesis/0.9.0/template/content/preface.typ}{\texttt{\ preface.typ\ }}
must \emph{not} be removed from the
\href{https://github.com/typst/packages/raw/main/packages/preview/scholarly-tauthesis/0.9.0/template/content}{\texttt{\ contents\ }}
folder, as the automation supposes that they are located there.

You should probably \emph{not} modify the file
\href{https://github.com/typst/packages/raw/main/packages/preview/scholarly-tauthesis/0.9.0/tauthesis.typ}{\texttt{\ tauthesis.typ\ }}
, unless there is a bug that needs fixing right now, and not when the
maintainer of this project manages to find the time to do it.

\subsection{Contributing}\label{contributing}

Issues may be created in the issue tracker on the
\href{https://gitlab.com/tuni-official/thesis-templates/tau-typst-thesis-template}{template
GitLab repository} , if one has a GitLab account. Merge requests may
also be performed after GitLab account creation, and forking the
project. See GitLab’s documentation on this to find out how to do it
\href{https://docs.gitlab.com/ee/user/project/repository/forking_workflow.html}{link}
.

\subsection{License}\label{license}

This project itself uses the MIT license. See the
\href{https://github.com/typst/packages/raw/main/packages/preview/scholarly-tauthesis/0.9.0/LICENSE}{LICENSE}
file for details.

\href{/app?template=scholarly-tauthesis&version=0.9.0}{Create project in
app}

\subsubsection{How to use}\label{how-to-use}

Click the button above to create a new project using this template in
the Typst app.

You can also use the Typst CLI to start a new project on your computer
using this command:

\begin{verbatim}
typst init @preview/scholarly-tauthesis:0.9.0
\end{verbatim}

\includesvg[width=0.16667in,height=0.16667in]{/assets/icons/16-copy.svg}

\subsubsection{About}\label{about}

\begin{description}
\tightlist
\item[Author :]
\href{mailto:santtu.soderholm@tuni.fi}{Santtu Söderholm}
\item[License:]
MIT
\item[Current version:]
0.9.0
\item[Last updated:]
November 12, 2024
\item[First released:]
April 9, 2024
\item[Minimum Typst version:]
0.12.0
\item[Archive size:]
36.4 kB
\href{https://packages.typst.org/preview/scholarly-tauthesis-0.9.0.tar.gz}{\pandocbounded{\includesvg[keepaspectratio]{/assets/icons/16-download.svg}}}
\item[Repository:]
\href{https://gitlab.com/tuni-official/thesis-templates/tau-typst-thesis-template}{GitLab}
\item[Discipline :]
\begin{itemize}
\tightlist
\item[]
\item
  \href{https://typst.app/universe/search/?discipline=education}{Education}
\end{itemize}
\item[Categor y :]
\begin{itemize}
\tightlist
\item[]
\item
  \pandocbounded{\includesvg[keepaspectratio]{/assets/icons/16-mortarboard.svg}}
  \href{https://typst.app/universe/search/?category=thesis}{Thesis}
\end{itemize}
\end{description}

\subsubsection{Where to report issues?}\label{where-to-report-issues}

This template is a project of Santtu Söderholm . Report issues on
\href{https://gitlab.com/tuni-official/thesis-templates/tau-typst-thesis-template}{their
repository} . You can also try to ask for help with this template on the
\href{https://forum.typst.app}{Forum} .

Please report this template to the Typst team using the
\href{https://typst.app/contact}{contact form} if you believe it is a
safety hazard or infringes upon your rights.

\phantomsection\label{versions}
\subsubsection{Version history}\label{version-history}

\begin{longtable}[]{@{}ll@{}}
\toprule\noalign{}
Version & Release Date \\
\midrule\noalign{}
\endhead
\bottomrule\noalign{}
\endlastfoot
0.9.0 & November 12, 2024 \\
\href{https://typst.app/universe/package/scholarly-tauthesis/0.8.0/}{0.8.0}
& October 21, 2024 \\
\href{https://typst.app/universe/package/scholarly-tauthesis/0.7.0/}{0.7.0}
& September 17, 2024 \\
\href{https://typst.app/universe/package/scholarly-tauthesis/0.6.2/}{0.6.2}
& April 29, 2024 \\
\href{https://typst.app/universe/package/scholarly-tauthesis/0.5.0/}{0.5.0}
& April 15, 2024 \\
\href{https://typst.app/universe/package/scholarly-tauthesis/0.4.1/}{0.4.1}
& April 13, 2024 \\
\href{https://typst.app/universe/package/scholarly-tauthesis/0.4.0/}{0.4.0}
& April 9, 2024 \\
\end{longtable}

Typst GmbH did not create this template and cannot guarantee correct
functionality of this template or compatibility with any version of the
Typst compiler or app.


\title{typst.app/universe/package/uo-pup-thesis-manuscript}

\phantomsection\label{banner}
\phantomsection\label{template-thumbnail}
\pandocbounded{\includegraphics[keepaspectratio]{https://packages.typst.org/preview/thumbnails/uo-pup-thesis-manuscript-0.1.0-small.webp}}

\section{uo-pup-thesis-manuscript}\label{uo-pup-thesis-manuscript}

{ 0.1.0 }

Unofficial Typst template for PUP (Polytechnic University of the
Philippines) undergraduate thesis manuscript

\href{/app?template=uo-pup-thesis-manuscript&version=0.1.0}{Create
project in app}

\phantomsection\label{readme}
Unofficial \href{https://typst.app/}{typst} template for undergraduate
thesis manuscript for PUP (Polytechnic University of the Philippines).
This template adheres to the University’s Thesis and Dissertation
Manual as of 2017 (ISBN: 978-971â€``95208-8-7 (Online)). An example
manuscript is also provided (see \texttt{\ ./thesis.pdf\ } ).

\subsection{Setup}\label{setup}

Using
\href{https://github.com/typst/typst?tab=readme-ov-file\#installation}{Typst
CLI} :

\begin{Shaded}
\begin{Highlighting}[]
\ExtensionTok{typst}\NormalTok{ init @preview/uo{-}pup{-}thesis{-}manuscript my{-}thesis}
\BuiltInTok{cd}\NormalTok{ my{-}thesis}
\ExtensionTok{typst}\NormalTok{ compile thesis.typ  }\CommentTok{\# to compile to PDF}
\end{Highlighting}
\end{Shaded}

or run

\begin{Shaded}
\begin{Highlighting}[]
\ExtensionTok{typst}\NormalTok{ watch thesis.typ  }\CommentTok{\# to automatically compiles PDF on save}
\end{Highlighting}
\end{Shaded}

\subsection{Usage}\label{usage}

The template already provided an example structure and some guides. But
to start from nothing, make an entrypoint file with a basic structure
like this:

\begin{Shaded}
\begin{Highlighting}[]
\NormalTok{// thesis.typ}
\NormalTok{\#import "@preview/uo{-}pup{-}thesis{-}manuscript:0.1.0": *}


\NormalTok{\#show: template.with(}
\NormalTok{  [\textless{}your thesis title here\textgreater{}],}
\NormalTok{  ("Author 1", "Author 2", ..., "Author N"),}
\NormalTok{  "name of your college here",}
\NormalTok{  "name of your deg. program here",}
\NormalTok{  "Month YYYY"}
\NormalTok{)}


\NormalTok{// Main content starts here}

\NormalTok{// This provides a customized heading for}
\NormalTok{// chapters that follows the manual}
\NormalTok{\#chapter(1, "Chapter 1 Title") }

\NormalTok{// Since \#chapter() provides a heading level 1,}
\NormalTok{// start each headings under chapters with level 2}
\NormalTok{// to avoid messing up the generated Table of Contents}
\NormalTok{== Introduction}

\NormalTok{...}

\NormalTok{\#chapter(2, "Chapter 2 Title")}

\NormalTok{== Topic A}

\NormalTok{...}

\NormalTok{// End of main content}


\NormalTok{// Bibliography formatting setup}
\NormalTok{\#set par(first{-}line{-}indent: 0pt, hanging{-}indent: 0.5in)}
\NormalTok{\#set page(header: context [\#h(1fr) \#counter(page).get().first()])}
\NormalTok{\#align(center)[ \#heading("REFERENCES") ]}
\NormalTok{\#set par(spacing: 1.5em)}

\NormalTok{// Get the apa.csl file from \textasciigrave{}template/\textasciigrave{} folder}
\NormalTok{\#bibliography(title: none, style: "./apa.csl", "path/to/your/bibtex/file.bib")}


\NormalTok{// Appendices}
\NormalTok{\#show: appendices{-}section}

\NormalTok{\#appendix(1, "Appendix Title")}

\NormalTok{...}

\NormalTok{\#pagebreak()}

\NormalTok{\#appendix(2, "Appendix Title")}

\NormalTok{...}
\end{Highlighting}
\end{Shaded}

There are also provided utilities for some parts that have a specific
way of formatting.

For example, in \texttt{\ Definition\ of\ Terms\ } and
\texttt{\ Significance\ of\ the\ Study\ } sections, use
\texttt{\ \#description\ } function:

\begin{Shaded}
\begin{Highlighting}[]
\NormalTok{== Significance of the Study}
\NormalTok{\#description(}
\NormalTok{  (}
\NormalTok{    (term: [Topic A], desc: [\#lorem(30)]),}
\NormalTok{    (term: [Topic B], desc: [\#lorem(30)]),}
\NormalTok{    (term: [Topic C], desc: [\#lorem(30)]),}
\NormalTok{  )}
\NormalTok{)}

\NormalTok{...}

\NormalTok{== Definition of Terms}
\NormalTok{\#description(}
\NormalTok{  (}
\NormalTok{    (term: [Topic A], desc: [\#lorem(30)]),}
\NormalTok{    (term: [Topic B], desc: [\#lorem(30)]),}
\NormalTok{  )}
\NormalTok{)}
\end{Highlighting}
\end{Shaded}

\begin{center}\rule{0.5\linewidth}{0.5pt}\end{center}

If there’s any mistakes, wrong formatting (e.g., not actually
following the manual), etc., file an issue or a pull request.

\begin{center}\rule{0.5\linewidth}{0.5pt}\end{center}

\subsection{TODO}\label{todo}

\begin{itemize}
\tightlist
\item
  {[} {]} Chapter 4
\item
  {[} {]} Chapter 5
\item
  {[} {]} Abstract
\item
  {[} {]} Acknowledgement
\item
  {[} {]} Copyright
\item
  If possible:

  \begin{itemize}
  \tightlist
  \item
    Approval Sheet
  \item
    Certificate of Originality
  \end{itemize}
\end{itemize}

\href{/app?template=uo-pup-thesis-manuscript&version=0.1.0}{Create
project in app}

\subsubsection{How to use}\label{how-to-use}

Click the button above to create a new project using this template in
the Typst app.

You can also use the Typst CLI to start a new project on your computer
using this command:

\begin{verbatim}
typst init @preview/uo-pup-thesis-manuscript:0.1.0
\end{verbatim}

\includesvg[width=0.16667in,height=0.16667in]{/assets/icons/16-copy.svg}

\subsubsection{About}\label{about}

\begin{description}
\tightlist
\item[Author :]
\href{https://github.com/datsudo}{Datsudo}
\item[License:]
MIT
\item[Current version:]
0.1.0
\item[Last updated:]
November 4, 2024
\item[First released:]
November 4, 2024
\item[Archive size:]
21.6 kB
\href{https://packages.typst.org/preview/uo-pup-thesis-manuscript-0.1.0.tar.gz}{\pandocbounded{\includesvg[keepaspectratio]{/assets/icons/16-download.svg}}}
\item[Repository:]
\href{https://gitlab.com/datsudo/uo-pup-thesis-manuscript}{GitLab}
\item[Categor y :]
\begin{itemize}
\tightlist
\item[]
\item
  \pandocbounded{\includesvg[keepaspectratio]{/assets/icons/16-mortarboard.svg}}
  \href{https://typst.app/universe/search/?category=thesis}{Thesis}
\end{itemize}
\end{description}

\subsubsection{Where to report issues?}\label{where-to-report-issues}

This template is a project of Datsudo . Report issues on
\href{https://gitlab.com/datsudo/uo-pup-thesis-manuscript}{their
repository} . You can also try to ask for help with this template on the
\href{https://forum.typst.app}{Forum} .

Please report this template to the Typst team using the
\href{https://typst.app/contact}{contact form} if you believe it is a
safety hazard or infringes upon your rights.

\phantomsection\label{versions}
\subsubsection{Version history}\label{version-history}

\begin{longtable}[]{@{}ll@{}}
\toprule\noalign{}
Version & Release Date \\
\midrule\noalign{}
\endhead
\bottomrule\noalign{}
\endlastfoot
0.1.0 & November 4, 2024 \\
\end{longtable}

Typst GmbH did not create this template and cannot guarantee correct
functionality of this template or compatibility with any version of the
Typst compiler or app.


\title{typst.app/universe/package/acrotastic}

\phantomsection\label{banner}
\section{acrotastic}\label{acrotastic}

{ 0.1.1 }

Manage acronyms and their definitions in Typst.

\phantomsection\label{readme}
Manages all your acronyms for you.

Acrotastics main features are clickable abbreviations that auto-expand
on the first occurence, manual short and long forms, implicit or manual
plural form support, and customizable index printing.

\subsection{Quick Start}\label{quick-start}

\begin{verbatim}
#import "@preview/acrotastic:0.1.1": *

#init-acronyms((
  "WTP": ("Wonderful Typst Package","Wonderful Typst Packages"),
))

Acrotastic is a #acr("WTP")! This #acr("WTP") enables easy acronym manipulation.
\end{verbatim}

\subsection{Usage}\label{usage}

\subsubsection{Define acronyms}\label{define-acronyms}

First, define the acronyms in a dictionary, with the keys being the
acronyms and the values being arrays of their definitions. If there is
only a singular version of the definition, the array contains only one
value. If there are both singular and plural versions, define the
definition as an array where the first item is the singular definition
and the second item is the plural. Then, initialize Acrotastic by
passing the dictionay you just defined to the
\texttt{\ \#init-acronyms(...)\ } function.

Here is a example of the \texttt{\ acronyms.typ\ } file:

\begin{verbatim}
#import "@preview/acrotastic:0.1.1": *

#init-acronyms((
  "NN": ("Neural Network"),
  "OS": ("Operating System",),
  "BIOS": ("Basic Input/Output System", "Basic Input/Output Systems"),
))
\end{verbatim}

\subsubsection{Call Acrotastic
functions}\label{call-acrotastic-functions}

There is a large number of different functions to fit every use case.
You will find an overview of all functions and their descriptions in the
table below.

\begin{longtable}[]{@{}ll@{}}
\toprule\noalign{}
Function & Description \\
\midrule\noalign{}
\endhead
\bottomrule\noalign{}
\endlastfoot
\texttt{\ \#acr(...)\ } & On the first occurrence the long version of
the abbreviation and the abbreviation itself are displayed in brackets.
The next time only the abbreviation is displayed. \\
\texttt{\ \#acrpl(...)\ } & Same as \texttt{\ \#acr(...)\ } but the
plural will be diplayed. If no plural is defined, an ‘s’ is added to
the singular form. \\
\texttt{\ \#acrf(...)\ } & The acronym will be displayed as if it is the
first time. This means that it is again shown in the long form and the
abbreviation in brackets. \\
\texttt{\ \#acrfpl(...)\ } & Same as \texttt{\ \#acrf(...)\ } but the
plural will be displayed. If no plural is defined, an ‘s’ is added
to the singular form. \\
\texttt{\ \#acrs(...)\ } & Always displays the short form of the
acronym. \\
\texttt{\ \#acrspl(...)\ } & Same as \texttt{\ \#acrs(...)\ } but adds
an ‘s’ to the acronym for the plural form. \\
\texttt{\ \#acrl(...)\ } & Always displays the long form of the
acronym. \\
\texttt{\ \#acrlpl(...)\ } & Same as \texttt{\ \#acrl(...)\ } but the
plural will be displayed. If no plural is defined, an ‘s’ is added
to the singular form. \\
\texttt{\ \#reset-acronym(...)\ } & Resets a specific acronym. The
acronym will be expanded on the next use. \\
\texttt{\ reset-all-acronyms()\ } & Resets all acronyms. The acronyms
will be expanded on their next use. \\
\end{longtable}

You can alternatively use \texttt{\ \#acr(...)\ } ,
\texttt{\ \#acrf(...)\ } , \texttt{\ \#acrs(...)\ } and
\texttt{\ \#acrl(...)\ } with \texttt{\ plural:\ true\ } to display the
plural form.

\begin{verbatim}
#acr("BIOS", plural: true)
\end{verbatim}

To deactivate the link to the abbreviations directory (for whatever
reason), you can set \texttt{\ link:\ false\ } .

\begin{verbatim}
#acr("BIOS", link: false)
\end{verbatim}

\subsubsection{Print Abbreviations
directory}\label{print-abbreviations-directory}

You can also print an index of all acronyms used in the document with
the \texttt{\ \#print-index()\ } function. There are some parameters for
customization.

\begin{longtable}[]{@{}llll@{}}
\toprule\noalign{}
parameter & values & default & description \\
\midrule\noalign{}
\endhead
\bottomrule\noalign{}
\endlastfoot
title & string & “List of Abbreviations� & Heading of the acronym
index \\
level & number & 1 & Level of the heading \\
sorted & “up�, “down�, “keep� & “up� & “Up� sorts
alphabetically, “Down� sorts reversed alphabetically and “keep�
uses the order from initialization \\
delimiter & string & “:� & String to place after the acronym in the
list \\
acr-col-size & percentage & 20\% & Size of the acronym column in
percent \\
outlined & bool & false & Make the index section outlined \\
\end{longtable}

\subsection{Possible Errors}\label{possible-errors}

\begin{longtable}[]{@{}ll@{}}
\toprule\noalign{}
Error & Solution \\
\midrule\noalign{}
\endhead
\bottomrule\noalign{}
\endlastfoot
Acronym is not a key in the acronyms dictionary. & Make sure that the
acronym is defined in the dictionary passed to
\texttt{\ \#init-acronyms(dict)\ } \\
No definitions found for acronym. Make sure it is defined in the
dictionary passed to \#init-acronyms(dict) & The acronym is in the
dictionary, but has no correct definition. \\
Definitions should be arrays of one or two strings. Definition of
acronym is: & The acronym has a definition, but the definition doesn’t
have the right type. Make sure it’s an array of one or two strings. \\
\end{longtable}

Moreover you have to be careful when using states.

\begin{itemize}
\tightlist
\item
  For every acronym “ABC� that you define, the state named
  “acronym-state-ABC� is initialized and used. To avoid errors, do
  not try to use this state manually for other purposes. Similarly, the
  state named “acronyms� is reserved to Acrotastic. Please avoid
  using it.
\item
  The functions above are leveraging the state \texttt{\ display\ }
  function and only works if the return value is actually printed in the
  document. For more information on states, see the
  \href{https://typst.app/docs/reference/introspection/state/}{Typst
  documentation on states} .
\end{itemize}

\subsection{Contributing}\label{contributing}

If you notice any bug or want to contribute a new feature, please open
an issue or a pull request on the fork
\href{https://github.com/Julian702/typst-packages?tab=readme-ov-file}{Julian702/typst-packages}

\subsection{Acknowledgement}\label{acknowledgement}

Thanks to @Grisely who developed the
\href{https://typst.app/universe/package/acrostiche/}{acrostiche
package} which was the basis for acrotastic.

\subsubsection{How to add}\label{how-to-add}

Copy this into your project and use the import as
\texttt{\ acrotastic\ }

\begin{verbatim}
#import "@preview/acrotastic:0.1.1"
\end{verbatim}

\includesvg[width=0.16667in,height=0.16667in]{/assets/icons/16-copy.svg}

Check the docs for
\href{https://typst.app/docs/reference/scripting/\#packages}{more
information on how to import packages} .

\subsubsection{About}\label{about}

\begin{description}
\tightlist
\item[Author s :]
@Julian702 \& Gaetan Lepage @GaetanLepage
\item[License:]
MIT
\item[Current version:]
0.1.1
\item[Last updated:]
September 3, 2024
\item[First released:]
April 29, 2024
\item[Archive size:]
4.20 kB
\href{https://packages.typst.org/preview/acrotastic-0.1.1.tar.gz}{\pandocbounded{\includesvg[keepaspectratio]{/assets/icons/16-download.svg}}}
\item[Repository:]
\href{https://github.com/Julian702/typst-packages}{GitHub}
\item[Categor y :]
\begin{itemize}
\tightlist
\item[]
\item
  \pandocbounded{\includesvg[keepaspectratio]{/assets/icons/16-list-unordered.svg}}
  \href{https://typst.app/universe/search/?category=model}{Model}
\end{itemize}
\end{description}

\subsubsection{Where to report issues?}\label{where-to-report-issues}

This package is a project of @Julian702 and Gaetan Lepage @GaetanLepage
. Report issues on
\href{https://github.com/Julian702/typst-packages}{their repository} .
You can also try to ask for help with this package on the
\href{https://forum.typst.app}{Forum} .

Please report this package to the Typst team using the
\href{https://typst.app/contact}{contact form} if you believe it is a
safety hazard or infringes upon your rights.

\phantomsection\label{versions}
\subsubsection{Version history}\label{version-history}

\begin{longtable}[]{@{}ll@{}}
\toprule\noalign{}
Version & Release Date \\
\midrule\noalign{}
\endhead
\bottomrule\noalign{}
\endlastfoot
0.1.1 & September 3, 2024 \\
\href{https://typst.app/universe/package/acrotastic/0.1.0/}{0.1.0} &
April 29, 2024 \\
\end{longtable}

Typst GmbH did not create this package and cannot guarantee correct
functionality of this package or compatibility with any version of the
Typst compiler or app.


\title{typst.app/universe/package/universal-hit-thesis}

\phantomsection\label{banner}
\phantomsection\label{template-thumbnail}
\pandocbounded{\includegraphics[keepaspectratio]{https://packages.typst.org/preview/thumbnails/universal-hit-thesis-0.2.1-small.webp}}

\section{universal-hit-thesis}\label{universal-hit-thesis}

{ 0.2.1 }

å``ˆå°''滨工业大学学ä½?论æ--‡æ¨¡æ?¿ \textbar{} Universal Harbin
Institute of Technology Thesis

\href{/app?template=universal-hit-thesis&version=0.2.1}{Create project
in app}

\phantomsection\label{readme}
适ç''¨äºŽå``ˆå°''滨工业大学学ä½?论æ--‡çš„ Typst 模æ?¿

\pandocbounded{\includegraphics[keepaspectratio]{https://vonbrank-images.oss-cn-hangzhou.aliyuncs.com/20240426-HIT-Thesis-Typst/hit-thesis-typst-development-cover-01.jpg}}

\begin{quote}
{[}!WARNING{]}
本模æ?¿æ­£å¤„于积æž?å¼€å?{}`阶段,存在一些æ~¼å¼?é---®é¢˜ï¼Œé€‚å?ˆå°?鲜
Typst 特性

本模æ?¿æ˜¯æ°`é---´æ¨¡æ?¿ï¼Œ \textbf{å?¯èƒ½ä¸?被学æ~¡è®¤å?¯}
,正å¼?使ç''¨è¿‡ç¨‹ä¸­è¯·å?šå¥½éš?æ---¶å°†å†\ldots 容è¿?移至 Word
æˆ-- LaTeX 的准备
\end{quote}

\subsection{å\ldots³äºŽæœ¬é¡¹ç›®}\label{uxe5uxb3uxe4uxbaux17euxe6ux153uxe9uxb9uxe7}

\href{https://typst.app/}{Typst} 是使ç''¨ Rust
语言开å?{}`çš„å\ldots¨æ--°æ--‡æ¡£æŽ'版系统,有望以 Markdown
级别的简æ´?语法å'Œç¼--è¯`速度实现 LaTeX
级别的æŽ'版能力,å?³é€šè¿‡ç¼--写é?µå¾ª Typst
语法规则的æ--‡æœ¬æ--‡æ¡£ã€?执行ç¼--è¯`å`½ä»¤ï¼Œæ?¥å?¯ç''Ÿæˆ?ç›®æ~‡æ~¼å¼?çš„
PDF æ--‡æ¡£ã€‚

\textbf{HIT Thesis Typst}
是一å¥---简å?•æ˜``ç''¨çš„å``ˆå°''滨工业大学学ä½?论æ--‡ Typst
模æ?¿ï¼Œå?--- \href{https://github.com/hithesis/hithesis}{hithesis}
å?¯å?{}`,计åˆ'囊括一æ~¡ä¸‰åŒºæœ¬ç§`ã€?硕士ã€?å?šå£«çš„å­¦ä½?论æ--‡æ~¼å¼?。

\textbf{预览效果}

\begin{itemize}
\tightlist
\item
  本ç§`通ç''¨ï¼š
  \href{https://github.com/chosertech/HIT-Thesis-Typst/blob/build/universal-bachelor.pdf}{universal-bachelor.pdf}
\end{itemize}

\subsection{使ç''¨}\label{uxe4uxbduxe7}

\subsubsection{本地ç¼--è¾` â\ldots~
(推è??)}\label{uxe6ux153uxe5ux153uxe7uxbcuxe8uxbe-uxe2-uxefuxbcux2c6uxe6ux17euxe8uxefuxbc}

è¿™ç§?æ--¹å¼?适å?ˆå¤§å¤šæ•°ç''¨æˆ·ã€‚

é¦--å\ldots ˆå®‰è£\ldots{} Typst,您å?¯ä»¥åœ¨ Typst Github ä»``åº``çš„
\href{https://github.com/typst/typst/releases/}{Release 页�}
下载最æ--°çš„安è£\ldots åŒ\ldots 直接安è£\ldots ,并将
\texttt{\ typst\ } å?¯æ‰§è¡Œç¨‹åº?æ·»åŠ~到 \texttt{\ PATH\ }
环境å?˜é‡?;如果您使ç''¨ Scoop
åŒ\ldots 管ç?†å™¨ï¼Œåˆ™å?¯ä»¥ç›´æŽ¥é€šè¿‡
\texttt{\ scoop\ install\ typst\ } å`½ä»¤å®‰è£\ldots 。

安è£\ldots 好 Typst
之å?Žï¼Œæ‚¨å?ªéœ€è¦?选择一个您å--œæ¬¢çš„目录,并在此目录下执行以下å`½ä»¤ï¼š

\begin{Shaded}
\begin{Highlighting}[]
\ExtensionTok{typst}\NormalTok{ init @preview/universal{-}hit{-}thesis:0.2.1}
\end{Highlighting}
\end{Shaded}

Typst 将会创建一个å??为 \texttt{\ universal-hit-thesis\ }
çš„æ--‡ä»¶å¤¹ï¼Œè¿›å\ldots¥è¯¥ç›®å½•å?Žï¼Œæ‚¨å?¯ä»¥ç›´æŽ¥ä¿®æ''¹ç›®å½•ä¸‹çš„
\texttt{\ universal-bachelor.typ\ }
,然å?Žæ‰§è¡Œä»¥ä¸‹å`½ä»¤è¿›è¡Œç¼--è¯`ç''Ÿæˆ? \texttt{\ .pdf\ }
æ--‡æ¡£ï¼š

\begin{Shaded}
\begin{Highlighting}[]
\ExtensionTok{typst}\NormalTok{ compile universal{-}bachelor.typ}
\end{Highlighting}
\end{Shaded}

æˆ--è€\ldots 使ç''¨ä»¥ä¸‹å`½ä»¤è¿›è¡Œå®žæ---¶é¢„览:

\begin{Shaded}
\begin{Highlighting}[]
\ExtensionTok{typst}\NormalTok{ watch universal{-}bachelor.typ}
\end{Highlighting}
\end{Shaded}

å½``您è¦?实æ---¶é¢„览æ---¶ï¼Œæˆ`们更推è??使ç''¨ Visual Studio
Code 进行ç¼--è¾`,é\ldots?å?ˆ
\href{https://marketplace.visualstudio.com/items?itemName=nvarner.typst-lsp}{Tinymist
Typst} ,
\href{https://marketplace.visualstudio.com/items?itemName=mgt19937.typst-preview}{Typst
Preview} ç­‰æ?'件å?¯ä»¥å¤§å¹\ldots æ??å?‡æ‚¨çš„ç¼--è¾`ä½``验。

\subsubsection{本地ç¼--è¾`
â\ldots¡}\label{uxe6ux153uxe5ux153uxe7uxbcuxe8uxbe-uxe2}

è¿™ç§?æ--¹æ³•é€‚å?ˆ Typst å¼€å?{}`è€\ldots 。

é¦--å\ldots ˆä½¿ç''¨ \texttt{\ git\ clone\ } å`½ä»¤ clone
本项目,æˆ--è€\ldots 直接在 Release
页é?¢ä¸‹è½½ç‰¹å®šç‰ˆæœ¬çš„æº?ç~?。在 \texttt{\ templates/\ }
目录下选择您需è¦?的模æ?¿ï¼Œç›´æŽ¥ä¿®æ''¹æˆ--å¤?制一份,在æ~¹ç›®å½•è¿?行以下å`½ä»¤è¿›è¡Œç¼--è¯`:

\begin{Shaded}
\begin{Highlighting}[]
\ExtensionTok{typst}\NormalTok{ compile ./templates/.typ }\AttributeTok{{-}{-}root}\NormalTok{ ./}
\end{Highlighting}
\end{Shaded}

æˆ--è€\ldots 使ç''¨å¦‚下å`½ä»¤è¿›è¡Œå®žæ---¶é¢„览:

\begin{Shaded}
\begin{Highlighting}[]
\ExtensionTok{typst}\NormalTok{ watch ./templates/.typ }\AttributeTok{{-}{-}root}\NormalTok{ ./}
\end{Highlighting}
\end{Shaded}

\begin{quote}
{[}!TIP{]}
本模æ?¿æ­£å¤„于积æž?å¼€å?{}`阶段,更æ--°è¾ƒä¸ºé¢`ç¹?,虽然已ç»?上ä¼~至
Typst Universe,但是您�然�以借助 Typst local packages
�实现在 Typst Universe
å?Œæ­¥æœ¬æ¨¡æ?¿çš„最æ--°ç‰ˆæœ¬å‰?,在本地ä½``验本模æ?¿çš„最æ--°ç‰ˆæœ¬ï¼Œå\ldots·ä½``å?šæ³•ä¸ºï¼š

\begin{itemize}
\item
  在 Release
  页é?¢ä¸‹è½½å¯¹åº''版本的æº?ç~?压缩åŒ\ldots ,并将å\ldots¶è§£åŽ‹åˆ°
  \texttt{\ \{data-dir\}/typst/packages/local/universal-hit-thesis/\{version\}\ }
  , \texttt{\ \{data-dir\}\ } 在ä¸?å?Œæ``?作系统下的值为:

  \begin{itemize}
  \tightlist
  \item
    \texttt{\ \$XDG\_DATA\_HOME\ } or
    \texttt{\ \textasciitilde{}/.local/share\ } on Linux
  \item
    \texttt{\ \textasciitilde{}/Library/Application\ } Support on macOS
  \item
    \texttt{\ \%LOCALAPPDATA\%\ } on Windows
  \end{itemize}

  \texttt{\ \{version\}\ } 的值为 \texttt{\ typst.toml\ } 中
  \texttt{\ version\ } 项的值.

  解压完æˆ?å?Ž \texttt{\ typst.toml\ } æ--‡ä»¶åº''该出现在
  \texttt{\ \{data-dir\}/typst/packages/local/universal-hit-thesis/\{version\}\ }
  目录下.
\item
  接ç?€æ‚¨éœ€è¦?在您的论æ--‡ä¸­å°†
  \texttt{\ \#import\ "@preview/universal-hit-thesis:0.2.1"\ }
  ä¿®æ''¹ä¸º
  \texttt{\ \#import\ "@local/universal-hit-thesis:\{version\}"\ }
  ,å?³å?¯æ›´æ--°æ¨¡æ?¿.
\end{itemize}
\end{quote}

\subsubsection{在线ç¼--è¾`}\label{uxe5ux153uxe7uxbauxe7uxbcuxe8uxbe}

本模æ?¿å·²ä¸Šä¼~ Typst Universe,您å?¯ä»¥ä½¿ç''¨ Typst 的官æ--¹
Web App 进行ç¼--è¾`。

å\ldots·ä½``æ?¥è¯´ï¼Œåœ¨ Typst Web App 登录å?Žï¼Œç‚¹å‡»
\texttt{\ Start\ from\ template\ } ,在弹出的çª---å?£ä¸­é€‰æ‹©
\texttt{\ universal-hit-thesis\ } ,��从模�创建项目。

\pandocbounded{\includegraphics[keepaspectratio]{https://vonbrank-images.oss-cn-hangzhou.aliyuncs.com/20240426-HIT-Thesis-Typst/hit-thesis-web-app-create.jpg}}

\pandocbounded{\includegraphics[keepaspectratio]{https://vonbrank-images.oss-cn-hangzhou.aliyuncs.com/20240426-HIT-Thesis-Typst/hit-thesis-web-app-demo.jpg}}

\begin{quote}
{[}!NOTE{]}

Typst Web App
çš„æŽ'版渲æŸ``在æµ?览器本地执行,所以实æ---¶é¢„览ä½``验å‡~乎与在本地ç¼--è¾`æ---~异。

默认æƒ\ldots 况下,å½``您在 Web App
使ç''¨æ¨¡æ?¿åˆ›å»ºè®ºæ--‡é¡¹ç›®å?Žï¼Œå?¯èƒ½åœ¨é¡¹ç›®ä¸­çœ‹åˆ°å¤§é‡?é'ˆå¯¹ä¸­æ--‡æ--‡æœ¬çš„拼写é''™è¯¯è­¦å`Šï¼Œæ‚¨å?¯ä»¥é€šè¿‡åœ¨
\texttt{\ \#cover()\ } 函数调ç''¨ç‚¹å‰?æ?'å\ldots¥è¯­å?¥
\texttt{\ \#set\ text(lang:\ "zh")\ }
æ?¥æ¶ˆé™¤è¿™äº›è­¦å`Šï¼Œè¯¥é---®é¢˜å°†åœ¨æœªæ?¥çš„版本中å¾---到修å¤?.

æ­¤å¤--您å?¯èƒ½å·²ç»?注æ„?到,Web App
中的模æ?¿å­---ä½``显示与预期存在差è·?,这是å›~为 Web App
默认ä¸?æ??ä¾› \texttt{\ SimSun\ } , \texttt{\ Times\ New\ Roman\ }
等中æ--‡æŽ'版常ç''¨å­---ä½``。为了解决这个é---®é¢˜ï¼Œæ‚¨å?¯ä»¥åœ¨æ?œç´¢å¼•æ``Žæ?œç´¢ä»¥ä¸‹å­---ä½``æ--‡ä»¶ï¼š

\begin{itemize}
\tightlist
\item
  \texttt{\ TimesNewRoman.ttf\ } (åŒ\ldots 括 \texttt{\ Bold\ } ,
  \texttt{\ Italic\ } \texttt{\ Bold-Italic\ } 等版本)
\item
  \texttt{\ SimSun.ttf\ }
\item
  \texttt{\ SimHei.ttf\ }
\item
  \texttt{\ Kaiti.ttf\ }
\item
  \texttt{\ Consolas.ttf\ }
\item
  \texttt{\ Courier\ New.ttf\ }
\end{itemize}

并将这些æ--‡ä»¶æ‰‹åŠ¨ä¸Šä¼~至 Web App
项目æ~¹ç›®å½•ä¸­ï¼Œæˆ--为了目录整æ´?,å?¯ä»¥åˆ›å»ºä¸€ä¸ª
\texttt{\ fonts\ } æ--‡ä»¶å¤¹å¹¶å°†å­---ä½``置于å\ldots¶ä¸­ï¼ŒTypst Web
App 将自动åŠ~载这些å­---ä½``,并正确渲æŸ``到预览çª---å?£ä¸­.

ç''±äºŽæ¯?次在 Typst Web App
中æ‰``开项目æ---¶éƒ½éœ€è¦?é‡?æ--°ä¸‹è½½å­---ä½``,而中æ--‡å­---ä½``ä½``积普é??较大,åŠ~è½½æ---¶é---´è¾ƒé•¿ï¼Œå›~æ­¤æˆ`们更推è??
\textbf{本地ç¼--è¾`} 。
\end{quote}

\begin{center}\rule{0.5\linewidth}{0.5pt}\end{center}

\begin{quote}
{[}!NOTE{]} 注æ„?到,官æ--¹æ??供的本ç§`毕业设计 Microsoft
Word 论æ--‡æ¨¡æ?¿
\texttt{\ 本科毕业论文(设计)书写范例(�工类).doc\ }
在一æ~¡ä¸‰åŒºæ˜¯é€šç''¨çš„,æ„?å`³ç?€æœ¬ Typst
模æ?¿çš„本ç§`论æ--‡éƒ¨åˆ†ç?†è®ºä¸Šä¹Ÿæ˜¯åœ¨ä¸€æ~¡ä¸‰åŒºé€šç''¨çš„,å›~æ­¤æˆ`们æ??供适ç''¨äºŽå?„æ~¡åŒºçš„本ç§`毕业论æ--‡æ¨¡æ?¿æ¨¡å?---导出,å?³ä»¥ä¸‹å››ç§?导å\ldots¥æ¨¡å?---çš„æ--¹å¼?效果相å?Œï¼š

\begin{Shaded}
\begin{Highlighting}[]
\NormalTok{\#import "@preview/universal{-}hit{-}thesis:0.2.1": harbin{-}bachelor}
\NormalTok{\#import harbin{-}bachelor: * // 哈尔滨校区本科}
\end{Highlighting}
\end{Shaded}

\begin{Shaded}
\begin{Highlighting}[]
\NormalTok{\#import "@preview/universal{-}hit{-}thesis:0.2.1": weihai{-}bachelor}
\NormalTok{\#import weihai{-}bachelor: * // 威海校区本科}
\end{Highlighting}
\end{Shaded}

\begin{Shaded}
\begin{Highlighting}[]
\NormalTok{\#import "@preview/universal{-}hit{-}thesis:0.2.1": shenzhen{-}bachelor}
\NormalTok{\#import shenzhen{-}bachelor: * // 深圳校区本科}
\end{Highlighting}
\end{Shaded}

\begin{Shaded}
\begin{Highlighting}[]
\NormalTok{\#import "@preview/universal{-}hit{-}thesis:0.2.1": universal{-}bachelor}
\NormalTok{\#import universal{-}bachelor: * // 一校三区本科通用}
\end{Highlighting}
\end{Shaded}
\end{quote}

\subsection{ä¾?èµ--}\label{uxe4uxbeuxe8uxb5}

\subsubsection{å?¯é€‰ä¾?èµ--}\label{uxe5uxe9uxe4uxbeuxe8uxb5}

è‹¥è¦?书写å'Œå¼•ç''¨ä¼ªä»£ç~?,您å?¯ä»¥ä½¿ç''¨
\texttt{\ algorithm-figure\ } ,为此,您需è¦?导å\ldots¥
\texttt{\ algorithmic\ } æˆ-- \texttt{\ lovelace\ } åŒ\ldots 。

\begin{Shaded}
\begin{Highlighting}[]
\NormalTok{\#import "@preview/algorithmic:0.1.0"}
\NormalTok{\#import algorithmic: algorithm}

\NormalTok{\#import "@preview/lovelace:0.2.0": *}
\end{Highlighting}
\end{Shaded}

使ç''¨æ--¹å¼?详è§?
\href{https://github.com/chosertech/HIT-Thesis-Typst/blob/main/templates/universal-bachelor.typ}{模�}
中的 \texttt{\ 伪代ç~?\ } 节

\subsection{已知é---®é¢˜}\label{uxe5uxb2uxe7uxffuxe9uxe9}

\subsubsection{æŽ'版}\label{uxe6ux17euxe7ux2c6}

尽管本 Typst 模æ?¿å?„部分å­---ä½``ã€?å­---å?·ç­‰è®¾ç½®å?‡ä¸ŽåŽŸ Word
模æ?¿ä¸€è‡´ï¼Œä½†æ®µè?½æŽ'版视觉上ä»?与 Word
模æ?¿æœ‰ä¸€äº›å·®åˆ«ï¼Œè¿™ä¸Žå­---符é---´è·?ã€?è¡Œè·?ã€?段è?½é---´è·?有一定肉眼æŽ'版æˆ?分有å\ldots³.

\subsubsection{å?‚考æ--‡çŒ®}\label{uxe5uxe8ux192uxe6uxe7ux153}

\begin{itemize}
\tightlist
\item
  å­¦æ~¡å¯¹å?‚考æ--‡çŒ®æ~¼å¼?çš„è¦?求与æ~‡å‡†çš„
  \texttt{\ GB/T\ 7714-2015\ numeric\ }
  æ~¼å¼?存在差异,æˆ`们已修æ''¹ç›¸å\ldots³ CSL æ--‡ä»¶å¹¶å½¢æˆ?
  \texttt{\ gb-t-7714-2015-numeric-hit.csl\ }
  以修å¤?作è€\ldots å??å­---大å°?写等é---®é¢˜ï¼Œä½†ä»?有以下已知特性尚未æ''¯æŒ?:

  \begin{itemize}
  \tightlist
  \item
    ä»\ldots 纯ç''µå­?资æº?(如ç½`页ã€?软件)显示引ç''¨æ---¥æœŸå'Œ
    URL
  \item
    æ---~ DOI
  \end{itemize}
\item
  引ç''¨å\ldots¶ä»--å­¦æ~¡çš„å­¦ä½?论æ--‡æ---¶å?‚考æ--‡çŒ®é¡µå¯¹åº''æ?¡ç›®å­˜åœ¨æ~¼å¼?é---®é¢˜ï¼Œå›~为
  Typst å°šä¸?æ''¯æŒ? CSL æ--‡ä»¶ä¸­çš„ \texttt{\ school\ } ç­‰å­---段.
\item
  目�版本的 Typst 对 CSL
  æ''¯æŒ?程度æˆ?谜,更多é---®é¢˜å?¯å?‚考
  \href{https://github.com/csimide/SEU-Typst-Template/?tab=readme-ov-file\#\%E5\%8F\%82\%E8\%80\%83\%E6\%96\%87\%E7\%8C\%AE}{SEU-Typst-Template
  å?‚考æ--‡çŒ®å·²çŸ¥é---®é¢˜} .
\end{itemize}

\subsection{致谢}\label{uxe8uxe8}

\begin{itemize}
\tightlist
\item
  æ„Ÿè°¢
  \href{https://github.com/werifu/HUST-typst-template}{HUST-typst-template}
  为本模æ?¿æ---©æœŸç‰ˆæœ¬çš„框架æ??ä¾›æ€?è·¯.
\item
  æ„Ÿè°¢ \href{https://gist.github.com/csimide}{@csimide} å'Œ
  \href{https://github.com/OrangeX4}{@OrangeX4}
  æ??供的中英å?Œè¯­å?‚考æ--‡çŒ®å®žçŽ°.
\item
  æ„Ÿè°¢
  \href{https://github.com/nju-lug/modern-nju-thesis}{modern-nju-thesis}
  为本模æ?¿çš„一些特性æ??供实现æ€?è·¯.
\end{itemize}

\href{/app?template=universal-hit-thesis&version=0.2.1}{Create project
in app}

\subsubsection{How to use}\label{how-to-use}

Click the button above to create a new project using this template in
the Typst app.

You can also use the Typst CLI to start a new project on your computer
using this command:

\begin{verbatim}
typst init @preview/universal-hit-thesis:0.2.1
\end{verbatim}

\includesvg[width=0.16667in,height=0.16667in]{/assets/icons/16-copy.svg}

\subsubsection{About}\label{about}

\begin{description}
\tightlist
\item[Author :]
CHOSER Tech
\item[License:]
MIT
\item[Current version:]
0.2.1
\item[Last updated:]
June 19, 2024
\item[First released:]
May 17, 2024
\item[Archive size:]
23.7 kB
\href{https://packages.typst.org/preview/universal-hit-thesis-0.2.1.tar.gz}{\pandocbounded{\includesvg[keepaspectratio]{/assets/icons/16-download.svg}}}
\item[Repository:]
\href{https://github.com/chosertech/HIT-Thesis-Typst}{GitHub}
\item[Categor y :]
\begin{itemize}
\tightlist
\item[]
\item
  \pandocbounded{\includesvg[keepaspectratio]{/assets/icons/16-mortarboard.svg}}
  \href{https://typst.app/universe/search/?category=thesis}{Thesis}
\end{itemize}
\end{description}

\subsubsection{Where to report issues?}\label{where-to-report-issues}

This template is a project of CHOSER Tech . Report issues on
\href{https://github.com/chosertech/HIT-Thesis-Typst}{their repository}
. You can also try to ask for help with this template on the
\href{https://forum.typst.app}{Forum} .

Please report this template to the Typst team using the
\href{https://typst.app/contact}{contact form} if you believe it is a
safety hazard or infringes upon your rights.

\phantomsection\label{versions}
\subsubsection{Version history}\label{version-history}

\begin{longtable}[]{@{}ll@{}}
\toprule\noalign{}
Version & Release Date \\
\midrule\noalign{}
\endhead
\bottomrule\noalign{}
\endlastfoot
0.2.1 & June 19, 2024 \\
\href{https://typst.app/universe/package/universal-hit-thesis/0.2.0/}{0.2.0}
& May 17, 2024 \\
\end{longtable}

Typst GmbH did not create this template and cannot guarantee correct
functionality of this template or compatibility with any version of the
Typst compiler or app.


\title{typst.app/universe/package/jogs}

\phantomsection\label{banner}
\section{jogs}\label{jogs}

{ 0.2.3 }

QuickJS JavaScript runtime for Typst

\phantomsection\label{readme}
Quickjs javascript runtime for typst. This package provides a typst
plugin for evaluating javascript code.

\subsection{Example}\label{example}

\begin{Shaded}
\begin{Highlighting}[]
\NormalTok{\#import "@preview/jogs:0.2.3": *}

\NormalTok{\#set page(height: auto, width: auto, fill: black, margin: 1em)}
\NormalTok{\#set text(fill: white)}

\NormalTok{\#let code = \textasciigrave{}\textasciigrave{}\textasciigrave{}}
\NormalTok{function sum() \{}
\NormalTok{  const total = Array.prototype.slice.call(arguments).reduce(function(a, b) \{}
\NormalTok{      return a + b;}
\NormalTok{  \}, 0);}
\NormalTok{  return total;}
\NormalTok{\}}

\NormalTok{function string\_join(arr) \{}
\NormalTok{  return arr.join(" | ");}
\NormalTok{\}}

\NormalTok{function return\_complex\_obj() \{}
\NormalTok{  return \{}
\NormalTok{    a: 1,}
\NormalTok{    b: "2",}
\NormalTok{    c: [1, 2, 3],}
\NormalTok{    d: \{}
\NormalTok{      e: 1,}
\NormalTok{    \},}
\NormalTok{  \};}
\NormalTok{\}}

\NormalTok{\textasciigrave{}\textasciigrave{}\textasciigrave{}}
\NormalTok{\#let bytecode = compile{-}js(code)}

\NormalTok{\#list{-}global{-}property(bytecode)}

\NormalTok{\#call{-}js{-}function(bytecode, "sum", 6, 7, 8, 9, 10)}

\NormalTok{\#call{-}js{-}function(bytecode, "string\_join", ("a", "b", "c", "d", "e"),)}

\NormalTok{\#call{-}js{-}function(bytecode, "return\_complex\_obj",)}

\end{Highlighting}
\end{Shaded}

result:

\pandocbounded{\includesvg[keepaspectratio]{https://github.com/typst/packages/raw/main/packages/preview/jogs/0.2.3/typst-package/examples/fib.svg}}

\subsection{Documentation}\label{documentation}

This package provide following function(s):

\subsubsection{\texorpdfstring{\texttt{\ eval-js\ }}{ eval-js }}\label{eval-js}

Run a Javascript code snippet.

\paragraph{Arguments}\label{arguments}

\begin{itemize}
\tightlist
\item
  \texttt{\ code\ } - The Javascript code to run. It can be a string or
  a raw block.
\end{itemize}

\paragraph{Returns}\label{returns}

The result of the Javascript code. The type is the typst type which most
closely resembles the Javascript type.

\paragraph{Example}\label{example-1}

\begin{Shaded}
\begin{Highlighting}[]
\NormalTok{\#let result = eval{-}js("1 + 1")}
\end{Highlighting}
\end{Shaded}

\subsubsection{\texorpdfstring{\texttt{\ compile-js\ }}{ compile-js }}\label{compile-js}

Compile a Javascript code snippet into bytecode.

\paragraph{Arguments}\label{arguments-1}

\begin{itemize}
\tightlist
\item
  \texttt{\ code\ } - The Javascript code to compile. It can be a string
  or a raw block.
\end{itemize}

\paragraph{Returns}\label{returns-1}

The bytecode of the Javascript code. The type is \texttt{\ bytes\ } .

\paragraph{Example}\label{example-2}

\begin{Shaded}
\begin{Highlighting}[]
\NormalTok{\#let bytecode = compile{-}js("function sum(a, b) \{ return a + b; \}")}
\end{Highlighting}
\end{Shaded}

\subsubsection{\texorpdfstring{\texttt{\ call-js-function\ }}{ call-js-function }}\label{call-js-function}

Call a Javascript function with arguments.

\paragraph{Arguments}\label{arguments-2}

\begin{itemize}
\tightlist
\item
  \texttt{\ bytecode\ } - The bytecode of the Javascript function. It
  can be obtained by calling \texttt{\ compile-js\ } .
\item
  \texttt{\ name\ } - The name of the Javascript function.
\item
  \texttt{\ ..args\ } - The arguments to pass to the Javascript
  function. All other positional arguments are passed to the Javascript
  function as is.
\end{itemize}

\paragraph{Returns}\label{returns-2}

The result of the Javascript function. The type is the typst type which
most closely resembles the Javascript type.

\paragraph{Example}\label{example-3}

\begin{Shaded}
\begin{Highlighting}[]
\NormalTok{\#let bytecode = compile{-}js("function sum(a, b) \{ return a + b; \}")}
\NormalTok{\#let result = call{-}js{-}function(bytecode, "sum", 1, 2)}
\end{Highlighting}
\end{Shaded}

\subsubsection{\texorpdfstring{\texttt{\ list-global-property\ }}{ list-global-property }}\label{list-global-property}

List all global properties of a compiled Javascript bytecode. This is
useful for inspecting the compiled Javascript bytecode.

\paragraph{Arguments}\label{arguments-3}

\begin{itemize}
\tightlist
\item
  \texttt{\ bytecode\ } - The bytecode of the Javascript function. It
  can be obtained by calling \texttt{\ compile-js\ } .
\end{itemize}

\paragraph{Returns}\label{returns-3}

A list of all global properties of the Javascript bytecode. The type is
\texttt{\ array\ } .

\paragraph{Example}\label{example-4}

\begin{Shaded}
\begin{Highlighting}[]
\NormalTok{\#let bytecode = compile{-}js("function sum(a, b) \{ return a + b; \}")}
\NormalTok{\#let result = list{-}global{-}property(bytecode)}
\end{Highlighting}
\end{Shaded}

\subsubsection{How to add}\label{how-to-add}

Copy this into your project and use the import as \texttt{\ jogs\ }

\begin{verbatim}
#import "@preview/jogs:0.2.3"
\end{verbatim}

\includesvg[width=0.16667in,height=0.16667in]{/assets/icons/16-copy.svg}

Check the docs for
\href{https://typst.app/docs/reference/scripting/\#packages}{more
information on how to import packages} .

\subsubsection{About}\label{about}

\begin{description}
\tightlist
\item[Author :]
Wenzhuo Liu
\item[License:]
MIT
\item[Current version:]
0.2.3
\item[Last updated:]
February 1, 2024
\item[First released:]
November 6, 2023
\item[Archive size:]
354 kB
\href{https://packages.typst.org/preview/jogs-0.2.3.tar.gz}{\pandocbounded{\includesvg[keepaspectratio]{/assets/icons/16-download.svg}}}
\item[Repository:]
\href{https://github.com/Enter-tainer/jogs}{GitHub}
\end{description}

\subsubsection{Where to report issues?}\label{where-to-report-issues}

This package is a project of Wenzhuo Liu . Report issues on
\href{https://github.com/Enter-tainer/jogs}{their repository} . You can
also try to ask for help with this package on the
\href{https://forum.typst.app}{Forum} .

Please report this package to the Typst team using the
\href{https://typst.app/contact}{contact form} if you believe it is a
safety hazard or infringes upon your rights.

\phantomsection\label{versions}
\subsubsection{Version history}\label{version-history}

\begin{longtable}[]{@{}ll@{}}
\toprule\noalign{}
Version & Release Date \\
\midrule\noalign{}
\endhead
\bottomrule\noalign{}
\endlastfoot
0.2.3 & February 1, 2024 \\
\href{https://typst.app/universe/package/jogs/0.2.2/}{0.2.2} & January
15, 2024 \\
\href{https://typst.app/universe/package/jogs/0.2.1/}{0.2.1} & November
29, 2023 \\
\href{https://typst.app/universe/package/jogs/0.2.0/}{0.2.0} & November
7, 2023 \\
\href{https://typst.app/universe/package/jogs/0.1.0/}{0.1.0} & November
6, 2023 \\
\end{longtable}

Typst GmbH did not create this package and cannot guarantee correct
functionality of this package or compatibility with any version of the
Typst compiler or app.


\title{typst.app/universe/package/shiroa}

\phantomsection\label{banner}
\section{shiroa}\label{shiroa}

{ 0.1.2 }

A simple tool for creating modern online books in pure typst.

\phantomsection\label{readme}
\href{https://github.com/Myriad-Dreamin/shiroa}{\emph{shiroa}} (
\emph{Shiro A} , or \emph{The White} , or \emph{äº`笺} ) is a simple
tool for creating modern online (cloud) books in pure typst.

\subsection{Installation (shiroa CLI)}\label{installation-shiroa-cli}

There are multiple ways to install the
\href{https://github.com/Myriad-Dreamin/shiroa}{shiroa} CLI tool. Choose
any one of the methods below that best suit your needs.

\subsubsection{Pre-compiled binaries}\label{pre-compiled-binaries}

Executable binaries are available for download on the
\href{https://github.com/Myriad-Dreamin/shiroa/releases}{GitHub Releases
page} . Download the binary for your platform (Windows, macOS, or Linux)
and extract the archive. The archive contains an \texttt{\ shiroa\ }
executable which you can run to build your books.

To make it easier to run, put the path to the binary into your
\texttt{\ PATH\ } .

\subsubsection{Build from source using
Rust}\label{build-from-source-using-rust}

To build the \texttt{\ shiroa\ } executable from source, you will first
need to install Yarn, Rust, and Cargo. Follow the instructions on the
\href{https://classic.yarnpkg.com/en/docs/install}{Yarn installation
page} and \href{https://www.rust-lang.org/tools/install}{Rust
installation page} . shiroa currently requires at least Rust version
1.75.

To build with precompiled artifacts, run the following commands:

\begin{Shaded}
\begin{Highlighting}[]
\ExtensionTok{cargo}\NormalTok{ install }\AttributeTok{{-}{-}git}\NormalTok{ https://github.com/Myriad{-}Dreamin/shiroa }\AttributeTok{{-}{-}locked}\NormalTok{ shiroa{-}cli}
\end{Highlighting}
\end{Shaded}

To build from source, run the following commands:

\begin{Shaded}
\begin{Highlighting}[]
\FunctionTok{git}\NormalTok{ clone https://github.com/Myriad{-}Dreamin/shiroa.git}
\FunctionTok{git}\NormalTok{ submodule update }\AttributeTok{{-}{-}recursive} \AttributeTok{{-}{-}init}
\ExtensionTok{cargo}\NormalTok{ run }\AttributeTok{{-}{-}bin}\NormalTok{ shiroa{-}build}
\CommentTok{\# optional: install it globally}
\ExtensionTok{cargo}\NormalTok{ install }\AttributeTok{{-}{-}path}\NormalTok{ ./cli}
\end{Highlighting}
\end{Shaded}

With global installation, to uninstall, run the command
\texttt{\ cargo\ uninstall\ shiroa\ } .

Again, make sure to add the Cargo bin directory to your
\texttt{\ PATH\ } .

\subsubsection{Get started}\label{get-started}

See the
\href{https://myriad-dreamin.github.io/shiroa/guide/get-started.html}{Get-started}
online documentation.

\subsubsection{Setup for writing your
book}\label{setup-for-writing-your-book}

We don’t provide a watch command, but \texttt{\ shiroa\ } is
designated to embracing all of the approaches to writing typst
documents. It’s feasible to preview your documents by following
approaches (like previewing normal documents):

\begin{itemize}
\item
  via \href{https://typst.app/}{Official Web App} .
\item
  via VSCod(e,ium), see
  \href{https://marketplace.visualstudio.com/items?itemName=myriad-dreamin.tinymist}{Tinymist}
  and
  \href{https://marketplace.visualstudio.com/items?itemName=mgt19937.typst-preview}{Typst
  Preview} .
\item
  via other editors. For example of neovim, see
  \href{https://github.com/kaarmu/typst.vim}{typst.vim} and
  \href{https://github.com/Enter-tainer/typst-preview\#use-without-vscode}{Typst
  Preview} .
\item
  via \texttt{\ typst-cli\ watch\ } , See
  \href{https://github.com/typst/typst\#usage}{typst-cli watch} .
\end{itemize}

\subsubsection{Acknowledgement}\label{acknowledgement}

\begin{itemize}
\item
  The
  \href{https://github.com/typst/packages/raw/main/packages/preview/shiroa/0.1.2/themes/mdbook/}{mdbook
  theme} is borrowed from
  \href{https://github.com/rust-lang/mdBook/tree/master/src/theme}{mdBook}
  project.
\item
  Compile the document with awesome
  \href{https://github.com/typst/typst}{Typst} .
\end{itemize}

\subsubsection{How to add}\label{how-to-add}

Copy this into your project and use the import as \texttt{\ shiroa\ }

\begin{verbatim}
#import "@preview/shiroa:0.1.2"
\end{verbatim}

\includesvg[width=0.16667in,height=0.16667in]{/assets/icons/16-copy.svg}

Check the docs for
\href{https://typst.app/docs/reference/scripting/\#packages}{more
information on how to import packages} .

\subsubsection{About}\label{about}

\begin{description}
\tightlist
\item[Author :]
\href{https://github.com/Myriad-Dreamin}{Myriad-Dreamin}
\item[License:]
Apache-2.0
\item[Current version:]
0.1.2
\item[Last updated:]
October 22, 2024
\item[First released:]
June 17, 2024
\item[Archive size:]
11.2 kB
\href{https://packages.typst.org/preview/shiroa-0.1.2.tar.gz}{\pandocbounded{\includesvg[keepaspectratio]{/assets/icons/16-download.svg}}}
\item[Repository:]
\href{https://github.com/Myriad-Dreamin/shiroa}{GitHub}
\item[Categor y :]
\begin{itemize}
\tightlist
\item[]
\item
  \pandocbounded{\includesvg[keepaspectratio]{/assets/icons/16-docs.svg}}
  \href{https://typst.app/universe/search/?category=book}{Book}
\end{itemize}
\end{description}

\subsubsection{Where to report issues?}\label{where-to-report-issues}

This package is a project of Myriad-Dreamin . Report issues on
\href{https://github.com/Myriad-Dreamin/shiroa}{their repository} . You
can also try to ask for help with this package on the
\href{https://forum.typst.app}{Forum} .

Please report this package to the Typst team using the
\href{https://typst.app/contact}{contact form} if you believe it is a
safety hazard or infringes upon your rights.

\phantomsection\label{versions}
\subsubsection{Version history}\label{version-history}

\begin{longtable}[]{@{}ll@{}}
\toprule\noalign{}
Version & Release Date \\
\midrule\noalign{}
\endhead
\bottomrule\noalign{}
\endlastfoot
0.1.2 & October 22, 2024 \\
\href{https://typst.app/universe/package/shiroa/0.1.1/}{0.1.1} & August
26, 2024 \\
\href{https://typst.app/universe/package/shiroa/0.1.0/}{0.1.0} & June
17, 2024 \\
\end{longtable}

Typst GmbH did not create this package and cannot guarantee correct
functionality of this package or compatibility with any version of the
Typst compiler or app.


\title{typst.app/universe/package/fletcher}

\phantomsection\label{banner}
\section{fletcher}\label{fletcher}

{ 0.5.2 }

Draw diagrams with nodes and arrows.

{ } Featured Package

\phantomsection\label{readme}
\href{https://github.com/typst/packages/raw/main/packages/preview/fletcher/0.5.2/docs/manual.pdf?raw=true}{\pandocbounded{\includegraphics[keepaspectratio]{https://img.shields.io/badge/docs-manual.pdf-green}}}
\pandocbounded{\includegraphics[keepaspectratio]{https://img.shields.io/badge/version-0.5.2-green}}
\href{https://github.com/Jollywatt/typst-fletcher/tree/dev}{\pandocbounded{\includegraphics[keepaspectratio]{https://img.shields.io/badge/dynamic/toml?url=https\%3A\%2F\%2Fgithub.com\%2FJollywatt\%2Ftypst-fletcher\%2Fraw\%2Fdev\%2Ftypst.toml&query=package.version&label=dev&color=blue}}}
\href{https://github.com/Jollywatt/typst-fletcher}{\pandocbounded{\includegraphics[keepaspectratio]{https://img.shields.io/badge/GitHub-repo-blue}}}

\emph{\textbf{fletcher} (noun) a maker of arrows}

A \href{https://typst.app/}{Typst} package for drawing diagrams with
arrows, built on top of
\href{https://github.com/johannes-wolf/cetz}{CeTZ} . See the
\href{https://github.com/typst/packages/raw/main/packages/preview/fletcher/0.5.2/docs/manual.pdf?raw=true}{manual}
for documentation.

\begin{Shaded}
\begin{Highlighting}[]
\NormalTok{\#import "@preview/fletcher:0.5.2" as fletcher: diagram, node, edge}
\end{Highlighting}
\end{Shaded}

\pandocbounded{\includesvg[keepaspectratio]{https://github.com/typst/packages/raw/main/packages/preview/fletcher/0.5.2/docs/readme-examples/first-isomorphism-theorem-light.svg}}

\begin{Shaded}
\begin{Highlighting}[]
\NormalTok{\#diagram(cell{-}size: 15mm, $}
\NormalTok{  G edge(f, {-}\textgreater{}) edge("d", pi, {-}\textgreater{}\textgreater{}) \& im(f) \textbackslash{}}
\NormalTok{  G slash ker(f) edge("ur", tilde(f), "hook{-}{-}\textgreater{}")}
\NormalTok{$)}
\end{Highlighting}
\end{Shaded}

\pandocbounded{\includesvg[keepaspectratio]{https://github.com/typst/packages/raw/main/packages/preview/fletcher/0.5.2/docs/readme-examples/flowchart-trap-light.svg}}

\begin{Shaded}
\begin{Highlighting}[]
\NormalTok{// https://xkcd.com/1195/}
\NormalTok{\#import fletcher.shapes: diamond}
\NormalTok{\#set text(font: "Comic Neue", weight: 600)}

\NormalTok{\#diagram(}
\NormalTok{  node{-}stroke: 1pt,}
\NormalTok{  edge{-}stroke: 1pt,}
\NormalTok{  node((0,0), [Start], corner{-}radius: 2pt, extrude: (0, 3)),}
\NormalTok{  edge("{-}|\textgreater{}"),}
\NormalTok{  node((0,1), align(center)[}
\NormalTok{    Hey, wait,\textbackslash{} this flowchart\textbackslash{} is a trap!}
\NormalTok{  ], shape: diamond),}
\NormalTok{  edge("d,r,u,l", "{-}|\textgreater{}", [Yes], label{-}pos: 0.1)}
\NormalTok{)}
\end{Highlighting}
\end{Shaded}

\pandocbounded{\includesvg[keepaspectratio]{https://github.com/typst/packages/raw/main/packages/preview/fletcher/0.5.2/docs/readme-examples/state-machine-light.svg}}

\begin{Shaded}
\begin{Highlighting}[]
\NormalTok{\#set text(10pt)}
\NormalTok{\#diagram(}
\NormalTok{  node{-}stroke: .1em,}
\NormalTok{  node{-}fill: gradient.radial(blue.lighten(80\%), blue, center: (30\%, 20\%), radius: 80\%),}
\NormalTok{  spacing: 4em,}
\NormalTok{  edge(({-}1,0), "r", "{-}|\textgreater{}", \textasciigrave{}open(path)\textasciigrave{}, label{-}pos: 0, label{-}side: center),}
\NormalTok{  node((0,0), \textasciigrave{}reading\textasciigrave{}, radius: 2em),}
\NormalTok{  edge(\textasciigrave{}read()\textasciigrave{}, "{-}|\textgreater{}"),}
\NormalTok{  node((1,0), \textasciigrave{}eof\textasciigrave{}, radius: 2em),}
\NormalTok{  edge(\textasciigrave{}close()\textasciigrave{}, "{-}|\textgreater{}"),}
\NormalTok{  node((2,0), \textasciigrave{}closed\textasciigrave{}, radius: 2em, extrude: ({-}2.5, 0)),}
\NormalTok{  edge((0,0), (0,0), \textasciigrave{}read()\textasciigrave{}, "{-}{-}|\textgreater{}", bend: 130deg),}
\NormalTok{  edge((0,0), (2,0), \textasciigrave{}close()\textasciigrave{}, "{-}|\textgreater{}", bend: {-}40deg),}
\NormalTok{)}
\end{Highlighting}
\end{Shaded}

\pandocbounded{\includesvg[keepaspectratio]{https://github.com/typst/packages/raw/main/packages/preview/fletcher/0.5.2/docs/readme-examples/feynman-diagram-light.svg}}

\begin{Shaded}
\begin{Highlighting}[]
\NormalTok{\#diagram($}
\NormalTok{  e\^{}{-} edge("rd", "{-}\textless{}|{-}") \& \& \& edge("ld", "{-}|\textgreater{}{-}") e\^{}+ \textbackslash{}}
\NormalTok{  \& edge(gamma, "wave") \textbackslash{}}
\NormalTok{  e\^{}+ edge("ru", "{-}|\textgreater{}{-}") \& \& \& edge("lu", "{-}\textless{}|{-}") e\^{}{-} \textbackslash{}}
\NormalTok{$)}
\end{Highlighting}
\end{Shaded}

Pull requests are most welcome!

\begin{longtable}[]{@{}
  >{\raggedright\arraybackslash}p{(\linewidth - 2\tabcolsep) * \real{0.5000}}
  >{\raggedright\arraybackslash}p{(\linewidth - 2\tabcolsep) * \real{0.5000}}@{}}
\toprule\noalign{}
\endhead
\bottomrule\noalign{}
\endlastfoot
\href{https://github.com/typst/packages/raw/main/packages/preview/fletcher/0.5.2/docs/gallery/commutative.typ}{}

\includesvg[width=1\linewidth,height=\textheight,keepaspectratio]{https://github.com/typst/packages/raw/main/packages/preview/fletcher/0.5.2/docs/gallery/commutative.svg}
&
\href{https://github.com/typst/packages/raw/main/packages/preview/fletcher/0.5.2/docs/gallery/algebra-cube.typ}{}

\includesvg[width=1\linewidth,height=\textheight,keepaspectratio]{https://github.com/typst/packages/raw/main/packages/preview/fletcher/0.5.2/docs/gallery/algebra-cube.svg} \\
\href{https://github.com/typst/packages/raw/main/packages/preview/fletcher/0.5.2/docs/gallery/ml-architecture.typ}{}

\includesvg[width=1\linewidth,height=\textheight,keepaspectratio]{https://github.com/typst/packages/raw/main/packages/preview/fletcher/0.5.2/docs/gallery/ml-architecture.svg}
&
\href{https://github.com/typst/packages/raw/main/packages/preview/fletcher/0.5.2/docs/gallery/io-flowchart.typ}{}

\includesvg[width=1\linewidth,height=\textheight,keepaspectratio]{https://github.com/typst/packages/raw/main/packages/preview/fletcher/0.5.2/docs/gallery/io-flowchart.svg} \\
\href{https://github.com/typst/packages/raw/main/packages/preview/fletcher/0.5.2/docs/gallery/digraph.typ}{}

\includesvg[width=1\linewidth,height=\textheight,keepaspectratio]{https://github.com/typst/packages/raw/main/packages/preview/fletcher/0.5.2/docs/gallery/digraph.svg}
&
\href{https://github.com/typst/packages/raw/main/packages/preview/fletcher/0.5.2/docs/gallery/node-groups.typ}{}

\includesvg[width=1\linewidth,height=\textheight,keepaspectratio]{https://github.com/typst/packages/raw/main/packages/preview/fletcher/0.5.2/docs/gallery/node-groups.svg} \\
\href{https://github.com/typst/packages/raw/main/packages/preview/fletcher/0.5.2/docs/gallery/uml-diagram.typ}{}

\includesvg[width=1\linewidth,height=\textheight,keepaspectratio]{https://github.com/typst/packages/raw/main/packages/preview/fletcher/0.5.2/docs/gallery/uml-diagram.svg}
& \\
\end{longtable}

\subsection{Change log}\label{change-log}

\subsubsection{0.5.2}\label{section}

\begin{itemize}
\tightlist
\item
  \textbf{Require \texttt{\ typst\ } version
  \texttt{\ \textgreater{}=0.12.0\ } .}
\item
  Update \texttt{\ cetz\ } dependency to \texttt{\ 0.3.1\ } .
  \textbf{Note:} This may slightly change edge label positions.
\item
  Add \texttt{\ loop-angle\ } option to \texttt{\ edge()\ } (\#36).
\end{itemize}

\subsubsection{0.5.1}\label{section-1}

\begin{itemize}
\tightlist
\item
  Fix nodes which \texttt{\ enclose\ } absolute coordinates.
\item
  Allow CeTZ-style coordinate expressions in node \texttt{\ enclose\ }
  option.
\item
  Fix crash with polar coordinates.
\item
  Fix edges which bend at 0deg or 180deg (e.g., \texttt{\ edge("r,r")\ }
  or \texttt{\ edge("r,l")\ } ) and enhance the way the corner radius
  adapts to the bend angle. \textbf{Note:} This may change diagram
  layout from previous versions.
\item
  Improve error messages.
\item
  Add marks for crow’s foot notation: \texttt{\ n\ } (many),
  \texttt{\ n?\ } (zero or more), \texttt{\ n!\ } (one or more),
  \texttt{\ 1\ } (one), \texttt{\ 1?\ } (zero or one), \texttt{\ 1!\ }
  (exactly one).
\item
  Add \texttt{\ node-shape\ } option to \texttt{\ diagram()\ } .
\end{itemize}

\subsubsection{0.5.0}\label{section-2}

\begin{itemize}
\tightlist
\item
  Greatly enhance coordinate system.

  \begin{itemize}
  \tightlist
  \item
    Support CeTZ-style coordinate expressions (relative, polar,
    interpolating, named coordinates, etc).
  \item
    Absolute coordinates (physical lengths) can be used alongside
    “elastic� coordinates (row/column positions).
  \end{itemize}
\item
  Add \texttt{\ label-angle\ } option to \texttt{\ edge()\ } .
\item
  Add \texttt{\ label-wrapper\ } option to allow changing the label
  inset, outline stroke, and so on (\#26).
\item
  Add \texttt{\ label-size\ } option to control default edge label text
  size (\#35)
\item
  Add \texttt{\ trapezium\ } node shape.
\item
  Disallow string labels to be passed as positional arguments to
  \texttt{\ edge()\ } (to eliminate ambiguity). Used named argument or
  pass content instead.
\end{itemize}

\subsubsection{0.4.5}\label{section-3}

\begin{itemize}
\tightlist
\item
  Add isosceles triangle node shape (\#31).
\item
  Add \texttt{\ fit\ } and \texttt{\ dir\ } options to various node
  shapes to adjust sizing and orientation.
\item
  Allow more than one consecutive edge to have an implicit end vertex.
\item
  Allow \texttt{\ snap-to\ } to be \texttt{\ none\ } to disable edge
  snapping (\#32).
\end{itemize}

\subsubsection{0.4.4}\label{section-4}

\begin{itemize}
\tightlist
\item
  Support fully customisable marks/arrowheads!

  \begin{itemize}
  \tightlist
  \item
    Added new mark styles and tweaked some existing defaults.
    \textbf{Note.} This may change the micro-typography of diagrams from
    previous versions.
  \end{itemize}
\item
  Add node groups via the \texttt{\ enclose\ } option of
  \texttt{\ node()\ } .
\item
  Node labels can be aligned inside the node with \texttt{\ align()\ } .
\item
  Node labels wrap naturally when label text is wider than the node.
  \textbf{Note:} This may change diagram layout from previous versions.
\item
  Add a \texttt{\ layer\ } option to nodes and edges to control drawing
  order.
\item
  Add node shapes: \texttt{\ ellipse\ } , \texttt{\ octagon\ } .
\end{itemize}

\subsubsection{0.4.3}\label{section-5}

\begin{itemize}
\tightlist
\item
  Fixed edge crossing backgrounds being drawn above nodes (\#14).
\item
  Add \texttt{\ fletcher.hide()\ } to hide elements with/without
  affecting layout, useful for incremental diagrams in slides (\#15).
\item
  Support \texttt{\ shift\ } ing edges by coordinate deltas as well as
  absolute lengths (\#13).
\item
  Support node names (\#8).
\end{itemize}

\subsubsection{0.4.2}\label{section-6}

\begin{itemize}
\tightlist
\item
  Improve edge-to-node snapping. Edges can terminate anywhere near a
  node (not just at its center) and will automatically snap to the node
  outline. Added \texttt{\ snap-to\ } option to \texttt{\ edge()\ } .
\item
  Fix node \texttt{\ inset\ } being half the amount specified. If
  upgrading from previous version, you will need to divide node
  \texttt{\ inset\ } values by two to preserve diagram layout.
\item
  Add \texttt{\ decorations\ } option to \texttt{\ edge()\ } for CeTZ
  path decorations ( \texttt{\ "wave"\ } , \texttt{\ "zigzag"\ } , and
  \texttt{\ "coil"\ } , also accepted as positional string arguments).
\end{itemize}

\subsubsection{0.4.1}\label{section-7}

\begin{itemize}
\tightlist
\item
  Support custom node shapes! Edges connect to node outlines
  automatically.

  \begin{itemize}
  \tightlist
  \item
    New \texttt{\ shapes\ } submodule, containing \texttt{\ diamond\ } ,
    \texttt{\ pill\ } , \texttt{\ parallelogram\ } ,
    \texttt{\ hexagon\ } , and other node shapes.
  \end{itemize}
\item
  Allow edges to have multiple segments.

  \begin{itemize}
  \tightlist
  \item
    Add \texttt{\ vertices\ } an \texttt{\ corner-radius\ } options to
    \texttt{\ edge()\ } .
  \item
    Relative coordinate shorthands may be comma separated to signify
    multiple segments, e.g., \texttt{\ "r,u,ll"\ } .
  \end{itemize}
\item
  Add \texttt{\ dodge\ } option to \texttt{\ edge()\ } to adjust end
  points.
\item
  Support \texttt{\ cetz:0.2.0\ } .
\end{itemize}

\subsubsection{0.4.0}\label{section-8}

\begin{itemize}
\tightlist
\item
  Add ability to specify diagrams in math-mode, using \texttt{\ \&\ } to
  separate nodes.
\item
  Allow implicit and relative edge coordinates, e.g.,
  \texttt{\ edge("d")\ } becomes \texttt{\ edge(prev-node,\ (0,\ 1))\ }
  .
\item
  Add ability to place marks anywhere along an edge. Shorthands now
  accept an optional middle mark, for example
  \texttt{\ \textbar{}-\textgreater{}-\textbar{}\ } and
  \texttt{\ hook-/-\textgreater{}\textgreater{}\ } .
\item
  Add “hanging tail� correction to marks on curved edges. Marks now
  rotate a bit to fit more comfortably along tightly curving arcs.
\item
  Add more arrowheads for the sake of it: \texttt{\ \}\textgreater{}\ }
  , \texttt{\ \textless{}\{\ } , \texttt{\ /\ } ,
  \texttt{\ \textbackslash{}\ } , \texttt{\ x\ } , \texttt{\ X\ } ,
  \texttt{\ *\ } (solid dot), \texttt{\ @\ } (solid circle).
\item
  Add \texttt{\ axes\ } option to \texttt{\ diagram()\ } to control the
  direction of each axis in the diagram’s coordinate system.
\item
  Add \texttt{\ width\ } , \texttt{\ height\ } and \texttt{\ radius\ }
  options to \texttt{\ node()\ } for explicit control over size.
\item
  Add \texttt{\ corner-radius\ } option to \texttt{\ node()\ } .
\item
  Add \texttt{\ stroke\ } option to \texttt{\ edge()\ } replacing
  \texttt{\ thickness\ } and \texttt{\ paint\ } options.
\item
  Add \texttt{\ edge-stroke\ } option to \texttt{\ diagram()\ }
  replacing \texttt{\ edge-thickness\ } .
\end{itemize}

\subsubsection{0.3.0}\label{section-9}

\begin{itemize}
\tightlist
\item
  Make round-style arrow heads better approximate the default math font.
\item
  Add solid arrow heads with shorthand
  \texttt{\ \textless{}\textbar{}-\ } ,
  \texttt{\ -\textbar{}\textgreater{}\ } and double-bar
  \texttt{\ \textbar{}\textbar{}-\ } ,
  \texttt{\ -\textbar{}\textbar{}\ } .
\item
  Add an \texttt{\ extrude\ } option to \texttt{\ node()\ } which
  duplicates and extrudes the node’s stroke, enabling double stroke
  effects.
\end{itemize}

\subsubsection{0.2.0}\label{section-10}

\begin{itemize}
\tightlist
\item
  Experimental support for customising arrowheads.
\item
  Add right-angled edges with
  \texttt{\ edge(...,\ corner:\ left/right)\ } .
\end{itemize}

\subsection{Star History}\label{star-history}

\href{https://star-history.com/\#jollywatt/typst-fletcher&Date}{\pandocbounded{\includegraphics[keepaspectratio]{https://api.star-history.com/svg?repos=jollywatt/typst-fletcher&type=Date}}}

\subsubsection{How to add}\label{how-to-add}

Copy this into your project and use the import as \texttt{\ fletcher\ }

\begin{verbatim}
#import "@preview/fletcher:0.5.2"
\end{verbatim}

\includesvg[width=0.16667in,height=0.16667in]{/assets/icons/16-copy.svg}

Check the docs for
\href{https://typst.app/docs/reference/scripting/\#packages}{more
information on how to import packages} .

\subsubsection{About}\label{about}

\begin{description}
\tightlist
\item[Author :]
Joseph Wilson (Jollywatt)
\item[License:]
MIT
\item[Current version:]
0.5.2
\item[Last updated:]
October 25, 2024
\item[First released:]
November 23, 2023
\item[Minimum Typst version:]
0.12.0
\item[Archive size:]
51.1 kB
\href{https://packages.typst.org/preview/fletcher-0.5.2.tar.gz}{\pandocbounded{\includesvg[keepaspectratio]{/assets/icons/16-download.svg}}}
\item[Repository:]
\href{https://github.com/Jollywatt/typst-fletcher}{GitHub}
\end{description}

\subsubsection{Where to report issues?}\label{where-to-report-issues}

This package is a project of Joseph Wilson (Jollywatt) . Report issues
on \href{https://github.com/Jollywatt/typst-fletcher}{their repository}
. You can also try to ask for help with this package on the
\href{https://forum.typst.app}{Forum} .

Please report this package to the Typst team using the
\href{https://typst.app/contact}{contact form} if you believe it is a
safety hazard or infringes upon your rights.

\phantomsection\label{versions}
\subsubsection{Version history}\label{version-history}

\begin{longtable}[]{@{}ll@{}}
\toprule\noalign{}
Version & Release Date \\
\midrule\noalign{}
\endhead
\bottomrule\noalign{}
\endlastfoot
0.5.2 & October 25, 2024 \\
\href{https://typst.app/universe/package/fletcher/0.5.1/}{0.5.1} & July
10, 2024 \\
\href{https://typst.app/universe/package/fletcher/0.5.0/}{0.5.0} & June
11, 2024 \\
\href{https://typst.app/universe/package/fletcher/0.4.5/}{0.4.5} & May
17, 2024 \\
\href{https://typst.app/universe/package/fletcher/0.4.4/}{0.4.4} & May
3, 2024 \\
\href{https://typst.app/universe/package/fletcher/0.4.3/}{0.4.3} & April
2, 2024 \\
\href{https://typst.app/universe/package/fletcher/0.4.2/}{0.4.2} &
February 23, 2024 \\
\href{https://typst.app/universe/package/fletcher/0.4.1/}{0.4.1} &
February 8, 2024 \\
\href{https://typst.app/universe/package/fletcher/0.4.0/}{0.4.0} &
January 30, 2024 \\
\href{https://typst.app/universe/package/fletcher/0.3.0/}{0.3.0} &
January 1, 2024 \\
\href{https://typst.app/universe/package/fletcher/0.2.0/}{0.2.0} &
November 29, 2023 \\
\href{https://typst.app/universe/package/fletcher/0.1.1/}{0.1.1} &
November 23, 2023 \\
\end{longtable}

Typst GmbH did not create this package and cannot guarantee correct
functionality of this package or compatibility with any version of the
Typst compiler or app.


\title{typst.app/universe/package/numty}

\phantomsection\label{banner}
\section{numty}\label{numty}

{ 0.0.5 }

Numeric Typst

\phantomsection\label{readme}
\subsection{Numty}\label{numty-1}

\subsubsection{Numeric Typst}\label{numeric-typst}

A library for performing mathematical operations on n-dimensional
matrices, vectors/arrays, and numbers in Typst, with support for
broadcasting and handling NaN values. Numty’s broadcasting rules and
API are inspired by NumPy.

\begin{Shaded}
\begin{Highlighting}[]
\NormalTok{\#import "numty.typ" as nt}

\NormalTok{// Define vectors and matrices}
\NormalTok{\#let a = (1, 2, 3)}
\NormalTok{\#let b = 2}
\NormalTok{\#let m = ((1, 2), (1, 3))}

\NormalTok{// Element{-}wise operations with broadcasting}
\NormalTok{\#nt.mult(a, b)  // Multiply vector \textquotesingle{}a\textquotesingle{} by scalar \textquotesingle{}b\textquotesingle{}: (2, 4, 6)}
\NormalTok{\#nt.add(a, a)   // Add two vectors: (2, 4, 6)}
\NormalTok{\#nt.add(2, a)   // Add scalar \textquotesingle{}2\textquotesingle{} to each element of vector \textquotesingle{}a\textquotesingle{}: (3, 4, 5)}
\NormalTok{\#nt.add(m, 1)   // Add scalar \textquotesingle{}1\textquotesingle{} to each element of matrix \textquotesingle{}m\textquotesingle{}: ((2, 3), (2, 4))}

\NormalTok{// Dot product of vectors}
\NormalTok{\#nt.dot(a, a)   // Dot product of vector \textquotesingle{}a\textquotesingle{} with itself: 14}

\NormalTok{// Handling NaN cases in mathematical functions}
\NormalTok{\#calc.sin((3, 4)) // Fails, as Typst does not support vector operations directly}
\NormalTok{\#nt.sin((3.4))    // Sine of each element in vector: (0.14411, 0.90929)}

\NormalTok{// Generate equally spaced values and vectorized functions}
\NormalTok{\#let x = nt.linspace(0, 10, 3)  // Generate 3 equally spaced values between 0 and 10: (0, 5, 10)}
\NormalTok{\#let y = nt.sin(x)              // Apply sine function to each element: (0, {-}0.95, {-}0.54)}

\NormalTok{// Logical operations}
\NormalTok{\#nt.eq(a, b)      // Compare each element in \textquotesingle{}a\textquotesingle{} to \textquotesingle{}b\textquotesingle{}: (false, true, false)}
\NormalTok{\#nt.any(nt.eq(a, b)) // Check if any element in \textquotesingle{}a\textquotesingle{} equals \textquotesingle{}b\textquotesingle{}: true}
\NormalTok{\#nt.all(nt.eq(a, b)) // Check if all elements in \textquotesingle{}a\textquotesingle{} equal \textquotesingle{}b\textquotesingle{}: false}

\NormalTok{// Handling special cases like division by zero}
\NormalTok{\#nt.div((1, 3), (2, 0))  // Element{-}wise division, with NaN for division by zero: (0.5, float.nan)}

\NormalTok{// Matrix operations (element{-}wise)}
\NormalTok{\#nt.add(m, 1)  // Add scalar to matrix elements: ((2, 3), (2, 4))}

\NormalTok{// matrix}
\NormalTok{\#nt.transpose(m)  // transposition}
\NormalTok{\#nt.matmul(m,m) //  matrix multipliation}
\NormalTok{\#nt.matmul(c(1,2), r(2,3)) //  colum vector times row vector multiplication.}
\NormalTok{\#np.trace(m) // trace}
\NormalTok{\#np.det(m) /2x2 determinant}
 
\NormalTok{// printing}
\NormalTok{\#nt.print(m, " != " , (1,2))  // long dollar print, see in pdf }
\NormalTok{\#nt.p(m, " != " , (1,2))  //  short long print print, see in pdf }
\end{Highlighting}
\end{Shaded}

Since vesion 0.0.4 n-dim matrices are supported as well in most
operations.

\subsection{Supported Features:}\label{supported-features}

\subsubsection{Dimensions:}\label{dimensions}

Numty uses standard typst list as a base type, most 1d operations like
dot are suported directly for them.

For matrix specific operations we use 2d arrays / nested arrays, that
are also the standard typst list, but nested like in: ((1,2), (1,1)).

For convenience you can create column or row vectors with the \#nt.c and
\#nt.r functions.

\begin{Shaded}
\begin{Highlighting}[]
\NormalTok{\#import "numty.typ" as nt}
\NormalTok{\#import "numty.typ": c, r}

\NormalTok{\#let a = (1,2,3)}
\NormalTok{\#let b = (3,2,1)}
\NormalTok{\#c(..a) // ((1,),(2,),(3,)) }
\NormalTok{\#r(..b) // ((3,2,1),)}
\NormalTok{\#nt.matmul(c(..a), r(..b)) // column @ row}
\end{Highlighting}
\end{Shaded}

\subsubsection{Logic Operations:}\label{logic-operations}

\begin{Shaded}
\begin{Highlighting}[]
\NormalTok{\#let a = (1,2,3)}
\NormalTok{\#let b = 2}

\NormalTok{\#nt.eq(a,b)  // (false, true, false)}
\NormalTok{\#nt.all(nt.eq(a,b))  // false}
\NormalTok{\#nt.any(nt.eq(a,b))  // true}
\end{Highlighting}
\end{Shaded}

\subsubsection{Math operators:}\label{math-operators}

All operators are element-wise, traditional matrix multiplication is not
yet supported.

\begin{Shaded}
\begin{Highlighting}[]
\NormalTok{\#nt.add((0,1,3), 1)  // (1,2,4)}
\NormalTok{\#nt.mult((1,3),(2,2)) // (2,6)}
\NormalTok{\#nt.div((1,3), (2,0)) // (0.5,float.nan)}
\end{Highlighting}
\end{Shaded}

\subsubsection{Algebra with Nan
handling:}\label{algebra-with-nan-handling}

\begin{Shaded}
\begin{Highlighting}[]
\NormalTok{\#nt.log((0,1,3)) //  (float.nan, 0 , 0.47...)}
\NormalTok{\#nt.sin((1,3)) //  (0.84.. , 0.14...)}
\end{Highlighting}
\end{Shaded}

\subsubsection{Vector operations:}\label{vector-operations}

Basic vector operations

\begin{Shaded}
\begin{Highlighting}[]
\NormalTok{\#nt.dot((1,2),(2,4))  //  9}
\NormalTok{\#nt.normalize((1,4), l:1) // (1/5,4/5)}
\end{Highlighting}
\end{Shaded}

\subsubsection{Others:}\label{others}

Functions for creating equally spaced indexes in linear and logspace,
usefull for log plots, to sample in acordance to the logarithmic space.

\begin{Shaded}
\begin{Highlighting}[]
\NormalTok{\#nt.linspace(0,10,3) // (0,5,10)}
\NormalTok{\#nt.logspace(1,3,3)}
\NormalTok{\#nt.geomspace(1,3,3) }
\end{Highlighting}
\end{Shaded}

\subsubsection{Matrix}\label{matrix}

\begin{Shaded}
\begin{Highlighting}[]
\NormalTok{\#nt.matmul(m,m )              // matrix multiplication}
\NormalTok{\#nt.det(((1,3), (3,4)))       // only 2x2 supported for now}
\NormalTok{\#nt.trace(((1,3), (3,4)))     // trace of square matrix}
\NormalTok{\#nt.transpose(((1,3), (3,4))) // matrix transposition}
\end{Highlighting}
\end{Shaded}

\subsubsection{Printing}\label{printing}

Numty supports \$ printing to the pdf of numerical matrices, both long
and short format.

\begin{Shaded}
\begin{Highlighting}[]
\NormalTok{\#nt.print((1,2),(4,2)))  // to display in the pdf}
\NormalTok{\#nt.p((1,2),(4,2)), " random string ")     // to display in the pdf}
\end{Highlighting}
\end{Shaded}

\subsubsection{How to add}\label{how-to-add}

Copy this into your project and use the import as \texttt{\ numty\ }

\begin{verbatim}
#import "@preview/numty:0.0.5"
\end{verbatim}

\includesvg[width=0.16667in,height=0.16667in]{/assets/icons/16-copy.svg}

Check the docs for
\href{https://typst.app/docs/reference/scripting/\#packages}{more
information on how to import packages} .

\subsubsection{About}\label{about}

\begin{description}
\tightlist
\item[Author :]
Pablo Ruiz Cuevas
\item[License:]
MIT
\item[Current version:]
0.0.5
\item[Last updated:]
November 12, 2024
\item[First released:]
October 22, 2024
\item[Archive size:]
4.27 kB
\href{https://packages.typst.org/preview/numty-0.0.5.tar.gz}{\pandocbounded{\includesvg[keepaspectratio]{/assets/icons/16-download.svg}}}
\item[Repository:]
\href{https://github.com/PabloRuizCuevas/numty}{GitHub}
\item[Categor ies :]
\begin{itemize}
\tightlist
\item[]
\item
  \pandocbounded{\includesvg[keepaspectratio]{/assets/icons/16-hammer.svg}}
  \href{https://typst.app/universe/search/?category=utility}{Utility}
\item
  \pandocbounded{\includesvg[keepaspectratio]{/assets/icons/16-code.svg}}
  \href{https://typst.app/universe/search/?category=scripting}{Scripting}
\end{itemize}
\end{description}

\subsubsection{Where to report issues?}\label{where-to-report-issues}

This package is a project of Pablo Ruiz Cuevas . Report issues on
\href{https://github.com/PabloRuizCuevas/numty}{their repository} . You
can also try to ask for help with this package on the
\href{https://forum.typst.app}{Forum} .

Please report this package to the Typst team using the
\href{https://typst.app/contact}{contact form} if you believe it is a
safety hazard or infringes upon your rights.

\phantomsection\label{versions}
\subsubsection{Version history}\label{version-history}

\begin{longtable}[]{@{}ll@{}}
\toprule\noalign{}
Version & Release Date \\
\midrule\noalign{}
\endhead
\bottomrule\noalign{}
\endlastfoot
0.0.5 & November 12, 2024 \\
\href{https://typst.app/universe/package/numty/0.0.4/}{0.0.4} & October
31, 2024 \\
\href{https://typst.app/universe/package/numty/0.0.3/}{0.0.3} & October
23, 2024 \\
\href{https://typst.app/universe/package/numty/0.0.2/}{0.0.2} & October
22, 2024 \\
\href{https://typst.app/universe/package/numty/0.0.1/}{0.0.1} & October
22, 2024 \\
\end{longtable}

Typst GmbH did not create this package and cannot guarantee correct
functionality of this package or compatibility with any version of the
Typst compiler or app.


\title{typst.app/universe/package/modern-iu-thesis}

\phantomsection\label{banner}
\phantomsection\label{template-thumbnail}
\pandocbounded{\includegraphics[keepaspectratio]{https://packages.typst.org/preview/thumbnails/modern-iu-thesis-0.1.0-small.webp}}

\section{modern-iu-thesis}\label{modern-iu-thesis}

{ 0.1.0 }

Modern Typst thesis template for Indiana University

\href{/app?template=modern-iu-thesis&version=0.1.0}{Create project in
app}

\phantomsection\label{readme}
IU thesis template in Typst

Simple template which meets thesis and dissertation
\href{https://graduate.indiana.edu/academic-requirements/thesis-dissertation/index.html}{requirements}
of IU Bloomington’s Graduate School. This template provides a couple
functions:

\begin{Shaded}
\begin{Highlighting}[]
\NormalTok{\#show: thesis.with(}
\NormalTok{    title: [content],}
\NormalTok{    author: [content],}
\NormalTok{    dept: [content],}
\NormalTok{    year: [content],}
\NormalTok{    month: [content],}
\NormalTok{    day: [content],}
\NormalTok{    committee: (}
\NormalTok{        (}
\NormalTok{            name: "Person 1",}
\NormalTok{            title: "Ph.D.",}
\NormalTok{        ),}
\NormalTok{    ),}
\NormalTok{    dedication: [content],}
\NormalTok{    acknowledgement: [content],}
\NormalTok{    abstract: [content],}
\NormalTok{)}
\end{Highlighting}
\end{Shaded}

and

\begin{Shaded}
\begin{Highlighting}[]
\NormalTok{\#iuquote(content)}
\end{Highlighting}
\end{Shaded}

Everything else follows along with basic Typst syntax such as headings,
figures, tables, etc. See the provided template as an example.

\href{/app?template=modern-iu-thesis&version=0.1.0}{Create project in
app}

\subsubsection{How to use}\label{how-to-use}

Click the button above to create a new project using this template in
the Typst app.

You can also use the Typst CLI to start a new project on your computer
using this command:

\begin{verbatim}
typst init @preview/modern-iu-thesis:0.1.0
\end{verbatim}

\includesvg[width=0.16667in,height=0.16667in]{/assets/icons/16-copy.svg}

\subsubsection{About}\label{about}

\begin{description}
\tightlist
\item[Author :]
Bo Johnson
\item[License:]
MIT
\item[Current version:]
0.1.0
\item[Last updated:]
November 12, 2024
\item[First released:]
November 12, 2024
\item[Archive size:]
2.97 kB
\href{https://packages.typst.org/preview/modern-iu-thesis-0.1.0.tar.gz}{\pandocbounded{\includesvg[keepaspectratio]{/assets/icons/16-download.svg}}}
\item[Repository:]
\href{https://github.com/bojohnson5/modern-iu-thesis}{GitHub}
\item[Categor y :]
\begin{itemize}
\tightlist
\item[]
\item
  \pandocbounded{\includesvg[keepaspectratio]{/assets/icons/16-mortarboard.svg}}
  \href{https://typst.app/universe/search/?category=thesis}{Thesis}
\end{itemize}
\end{description}

\subsubsection{Where to report issues?}\label{where-to-report-issues}

This template is a project of Bo Johnson . Report issues on
\href{https://github.com/bojohnson5/modern-iu-thesis}{their repository}
. You can also try to ask for help with this template on the
\href{https://forum.typst.app}{Forum} .

Please report this template to the Typst team using the
\href{https://typst.app/contact}{contact form} if you believe it is a
safety hazard or infringes upon your rights.

\phantomsection\label{versions}
\subsubsection{Version history}\label{version-history}

\begin{longtable}[]{@{}ll@{}}
\toprule\noalign{}
Version & Release Date \\
\midrule\noalign{}
\endhead
\bottomrule\noalign{}
\endlastfoot
0.1.0 & November 12, 2024 \\
\end{longtable}

Typst GmbH did not create this template and cannot guarantee correct
functionality of this template or compatibility with any version of the
Typst compiler or app.


\title{typst.app/universe/package/cades}

\phantomsection\label{banner}
\section{cades}\label{cades}

{ 0.3.0 }

Generate QR codes in typst.

\phantomsection\label{readme}
Draw QR codes in typst.

\begin{Shaded}
\begin{Highlighting}[]
\NormalTok{\#import "@preview/cades:0.3.0": qr{-}code}

\NormalTok{= QR Code for \textasciigrave{}typst.app\textasciigrave{}:}
\NormalTok{\#qr{-}code("https://typst.app", width: 3cm)}
\end{Highlighting}
\end{Shaded}

\subsection{Documentation}\label{documentation}

\subsubsection{\texorpdfstring{\texttt{\ qr-code\ }}{ qr-code }}\label{qr-code}

Draw a qr code to an image.

\paragraph{Arguments}\label{arguments}

\begin{itemize}
\tightlist
\item
  \texttt{\ content\ } : \texttt{\ str\ } - the content of the qr code
\item
  \texttt{\ width\ } : \texttt{\ length\ } \textbar{} \texttt{\ auto\ }
  - the width of the qr code, default is \texttt{\ auto\ }
\item
  \texttt{\ height\ } : \texttt{\ length\ } \textbar{} \texttt{\ auto\ }
  - the height of the qr code, default is \texttt{\ auto\ }
\item
  \texttt{\ color\ } : \texttt{\ color\ } - the color of the qrcode,
  default is \texttt{\ black\ }
\item
  \texttt{\ background\ } : \texttt{\ color\ } - the background color
  behind the qrcode, default is \texttt{\ white\ }
\item
  \texttt{\ error-correction\ } : \texttt{\ "L"\ } \textbar{}
  \texttt{\ "M"\ } \textbar{} \texttt{\ "Q"\ } \textbar{}
  \texttt{\ "H"\ } - the error correction level for the qr code, default
  is \texttt{\ "M"\ }
\end{itemize}

\paragraph{Returns}\label{returns}

The image, of type \texttt{\ content\ } .

\subsection{Acknowledgements}\label{acknowledgements}

This package uses \href{https://github.com/Enter-tainer/jogs}{Jogs} by
\href{https://github.com/Enter-tainer}{Wenzhuo Liu} and the qr code
rendering code is based on
\href{https://github.com/papnkukn/qrcode-svg/}{qrcode-svg} by
\href{https://github.com/papnkukn}{papnkukn} .

\subsubsection{How to add}\label{how-to-add}

Copy this into your project and use the import as \texttt{\ cades\ }

\begin{verbatim}
#import "@preview/cades:0.3.0"
\end{verbatim}

\includesvg[width=0.16667in,height=0.16667in]{/assets/icons/16-copy.svg}

Check the docs for
\href{https://typst.app/docs/reference/scripting/\#packages}{more
information on how to import packages} .

\subsubsection{About}\label{about}

\begin{description}
\tightlist
\item[Author :]
Niklas Ausborn
\item[License:]
MIT
\item[Current version:]
0.3.0
\item[Last updated:]
November 25, 2023
\item[First released:]
November 10, 2023
\item[Archive size:]
8.61 kB
\href{https://packages.typst.org/preview/cades-0.3.0.tar.gz}{\pandocbounded{\includesvg[keepaspectratio]{/assets/icons/16-download.svg}}}
\item[Repository:]
\href{https://github.com/Midbin/cades}{GitHub}
\end{description}

\subsubsection{Where to report issues?}\label{where-to-report-issues}

This package is a project of Niklas Ausborn . Report issues on
\href{https://github.com/Midbin/cades}{their repository} . You can also
try to ask for help with this package on the
\href{https://forum.typst.app}{Forum} .

Please report this package to the Typst team using the
\href{https://typst.app/contact}{contact form} if you believe it is a
safety hazard or infringes upon your rights.

\phantomsection\label{versions}
\subsubsection{Version history}\label{version-history}

\begin{longtable}[]{@{}ll@{}}
\toprule\noalign{}
Version & Release Date \\
\midrule\noalign{}
\endhead
\bottomrule\noalign{}
\endlastfoot
0.3.0 & November 25, 2023 \\
\href{https://typst.app/universe/package/cades/0.2.0/}{0.2.0} & November
10, 2023 \\
\end{longtable}

Typst GmbH did not create this package and cannot guarantee correct
functionality of this package or compatibility with any version of the
Typst compiler or app.


\title{typst.app/universe/package/bamdone-rebuttal}

\phantomsection\label{banner}
\phantomsection\label{template-thumbnail}
\pandocbounded{\includegraphics[keepaspectratio]{https://packages.typst.org/preview/thumbnails/bamdone-rebuttal-0.1.1-small.webp}}

\section{bamdone-rebuttal}\label{bamdone-rebuttal}

{ 0.1.1 }

Rebuttal/response letter template that allows authors to respond to
feedback given by reviewers in a peer-review process on a point-by-point
basis.

{ } Featured Template

\href{/app?template=bamdone-rebuttal&version=0.1.1}{Create project in
app}

\phantomsection\label{readme}
This is a Typst template for a rebuttal/response letter. It allows
authors to respond to feedback given by reviewers in a peer-review
process on a point-by-point basis. This template is based heavily on the
LaTeX template from Zenke Lab (see
\href{https://zenkelab.org/resources/latex-rebuttal-response-to-reviewers-template/}{here}
).

\subsection{Usage}\label{usage}

You can use this template in the Typst web app by clicking “Start from
template� on the dashboard and searching for
\texttt{\ bamdone-rebuttal\ } .

Alternatively, you can use the CLI to kick this project off using the
command

\begin{verbatim}
typst init @preview/bamdone-rebuttal
\end{verbatim}

Typst will create a new directory with all the files needed to get you
started.

\subsection{Configuration}\label{configuration}

This template exports the \texttt{\ rebuttal\ } function with the
following named arguments:

\begin{itemize}
\tightlist
\item
  \texttt{\ title\ } : (content), something like “Response Letter�
  (the default) or “Rebuttal�.
\item
  \texttt{\ authors\ } : (content), list of author names the top of the
  first column in boldface.
\item
  \texttt{\ date\ } : (content), defaults to
  \texttt{\ datetime.today().display()\ }
\item
  \texttt{\ paper-size\ } : Defaults to \texttt{\ us-letter\ } . Specify
  a
  \href{https://typst.app/docs/reference/layout/page/\#parameters-paper}{paper
  size string} to change the page format. Specifying this will configure
  numeric, IEEE-style citations.
\end{itemize}

The function also accepts a single, positional argument for the body of
the letter.

In addition, the template exports the \texttt{\ configure\ } function
which accepts the following named arguments corresponding to the text
color of various pieces of the letter:

\begin{itemize}
\tightlist
\item
  \texttt{\ point-color\ } : defaults to \texttt{\ blue.darken(30\%)\ }
  , the text color for reviewers’ points
\item
  \texttt{\ response-color\ } : defaults to \texttt{\ black\ } , the
  text color for the authors’ responses
\item
  \texttt{\ new-color\ } : defaults to \texttt{\ green.darken(30\%)\ } ,
  the text color for changes/additions to the manuscript (i.e., within a
  \texttt{\ quote\ } block to show what’s changed from the initial
  submission)
\end{itemize}

The template will initialize your package with a sample call to the
\texttt{\ rebuttal\ } function in a show rule.

\begin{Shaded}
\begin{Highlighting}[]
\NormalTok{// Optional color configuration}
\NormalTok{\#let (point, response, new) = configure(}
\NormalTok{  point{-}color: blue.darken(30\%),}
\NormalTok{  response{-}color: black,}
\NormalTok{  new{-}color: green.darken(30\%)}
\NormalTok{)}

\NormalTok{// Setup the rebuttal}
\NormalTok{\#show: rebuttal.with(}
\NormalTok{  authors: [First A. Author and Second B. Author],}
\NormalTok{  // date: ,}
\NormalTok{  // paper{-}size: ,}
\NormalTok{)}

\NormalTok{// Your content goes below}
\NormalTok{We thank the reviewers...}
\end{Highlighting}
\end{Shaded}

\href{/app?template=bamdone-rebuttal&version=0.1.1}{Create project in
app}

\subsubsection{How to use}\label{how-to-use}

Click the button above to create a new project using this template in
the Typst app.

You can also use the Typst CLI to start a new project on your computer
using this command:

\begin{verbatim}
typst init @preview/bamdone-rebuttal:0.1.1
\end{verbatim}

\includesvg[width=0.16667in,height=0.16667in]{/assets/icons/16-copy.svg}

\subsubsection{About}\label{about}

\begin{description}
\tightlist
\item[Author s :]
\href{https://avonmoll.github.io}{Alexander Von Moll} \&
\href{https://wwww.isaacew.com}{Isaac Weintraub}
\item[License:]
MIT-0
\item[Current version:]
0.1.1
\item[Last updated:]
November 12, 2024
\item[First released:]
May 16, 2024
\item[Minimum Typst version:]
0.12.0
\item[Archive size:]
4.26 kB
\href{https://packages.typst.org/preview/bamdone-rebuttal-0.1.1.tar.gz}{\pandocbounded{\includesvg[keepaspectratio]{/assets/icons/16-download.svg}}}
\item[Repository:]
\href{https://github.com/avonmoll/bamdone-rebuttal}{GitHub}
\item[Categor ies :]
\begin{itemize}
\tightlist
\item[]
\item
  \pandocbounded{\includesvg[keepaspectratio]{/assets/icons/16-envelope.svg}}
  \href{https://typst.app/universe/search/?category=office}{Office}
\item
  \pandocbounded{\includesvg[keepaspectratio]{/assets/icons/16-speak.svg}}
  \href{https://typst.app/universe/search/?category=report}{Report}
\end{itemize}
\end{description}

\subsubsection{Where to report issues?}\label{where-to-report-issues}

This template is a project of Alexander Von Moll and Isaac Weintraub .
Report issues on
\href{https://github.com/avonmoll/bamdone-rebuttal}{their repository} .
You can also try to ask for help with this template on the
\href{https://forum.typst.app}{Forum} .

Please report this template to the Typst team using the
\href{https://typst.app/contact}{contact form} if you believe it is a
safety hazard or infringes upon your rights.

\phantomsection\label{versions}
\subsubsection{Version history}\label{version-history}

\begin{longtable}[]{@{}ll@{}}
\toprule\noalign{}
Version & Release Date \\
\midrule\noalign{}
\endhead
\bottomrule\noalign{}
\endlastfoot
0.1.1 & November 12, 2024 \\
\href{https://typst.app/universe/package/bamdone-rebuttal/0.1.0/}{0.1.0}
& May 16, 2024 \\
\end{longtable}

Typst GmbH did not create this template and cannot guarantee correct
functionality of this template or compatibility with any version of the
Typst compiler or app.


\title{typst.app/universe/package/iconic-salmon-svg}

\phantomsection\label{banner}
\section{iconic-salmon-svg}\label{iconic-salmon-svg}

{ 1.1.0 }

A Typst library for Social Media references with scalable vector
graphics icons.

\phantomsection\label{readme}
The \texttt{\ iconic-salmon-svg\ } package is designed to help you
create your curriculum vitae (CV). It allows you to easily reference
your social media profiles with the typical icon of the service plus a
link to your profile.\\
The package name is a combination of the acronym \emph{SociAL Media
icONs} and the word \emph{iconic} because all these icons have an iconic
design (and iconic also contains the word \emph{icon} ).

\subsection{Features}\label{features}

\begin{itemize}
\tightlist
\item
  Support for popular social media, developer and career platforms
\item
  Uniform design for all entries
\item
  Based on publicly available SVG symbols
\item
  Easy to use
\item
  Allows the customization of the look (extra args are passed to
  \href{https://typst.app/docs/reference/text/text/}{\texttt{\ text\ }}
  )
\end{itemize}

\subsection{Usage}\label{usage}

\subsubsection{Using Typst’s package
manager}\label{using-typstuxe2s-package-manager}

You can install the library using the
\href{https://github.com/typst/packages}{typst packages} :

\begin{Shaded}
\begin{Highlighting}[]
\NormalTok{\#import "@preview/iconic{-}salmon{-}svg:1.0.0": *}
\end{Highlighting}
\end{Shaded}

\subsubsection{Install manually}\label{install-manually}

Put the \texttt{\ iconic-salmon-svg.typ\ } file in your project
directory and import it:

\begin{Shaded}
\begin{Highlighting}[]
\NormalTok{\#import "iconic{-}salmon{-}svg.typ": *}
\end{Highlighting}
\end{Shaded}

\subsubsection{Minimal Example}\label{minimal-example}

\begin{Shaded}
\begin{Highlighting}[]
\NormalTok{// \#import "@preview/iconic{-}salmon{-}svg:1.0.0": github{-}info, gitlab{-}info}
\NormalTok{\#import "iconic{-}salmon{-}svg.typ": github{-}info, gitlab{-}info}

\NormalTok{This project was created by \#github{-}info("Bi0T1N"). You can also find me on \#gitlab{-}info("GitLab", rgb("\#811052"), url: "https://gitlab.com/Bi0T1N").}
\end{Highlighting}
\end{Shaded}

\subsubsection{Examples}\label{examples}

See the
\href{https://github.com/typst/packages/raw/main/packages/preview/iconic-salmon-svg/1.1.0/examples/examples.typ}{\texttt{\ examples.typ\ }}
file for a complete example. The
\href{https://github.com/typst/packages/raw/main/packages/preview/iconic-salmon-svg/1.1.0/examples/}{generated
PDF files} are also available for preview.

\subsection{Contribution}\label{contribution}

Feel free to open an issue or a pull request if you find any problems or
have any suggestions.

\subsection{License}\label{license}

This library is licensed under the MIT license. Feel free to use it in
your project.

\subsection{Trademark Disclaimer}\label{trademark-disclaimer}

Product names, logos, brands and other trademarks referred to in this
project are the property of their respective trademark holders.\\
These trademark holders are not affiliated with this Typst library, nor
are the authors officially endorsed by them, nor do the authors claim
ownership of these trademarks.

\subsubsection{How to add}\label{how-to-add}

Copy this into your project and use the import as
\texttt{\ iconic-salmon-svg\ }

\begin{verbatim}
#import "@preview/iconic-salmon-svg:1.1.0"
\end{verbatim}

\includesvg[width=0.16667in,height=0.16667in]{/assets/icons/16-copy.svg}

Check the docs for
\href{https://typst.app/docs/reference/scripting/\#packages}{more
information on how to import packages} .

\subsubsection{About}\label{about}

\begin{description}
\tightlist
\item[Author :]
Nico Neumann (Bi0T1N)
\item[License:]
MIT
\item[Current version:]
1.1.0
\item[Last updated:]
May 23, 2024
\item[First released:]
April 15, 2024
\item[Archive size:]
22.5 kB
\href{https://packages.typst.org/preview/iconic-salmon-svg-1.1.0.tar.gz}{\pandocbounded{\includesvg[keepaspectratio]{/assets/icons/16-download.svg}}}
\item[Repository:]
\href{https://github.com/Bi0T1N/typst-iconic-salmon-svg}{GitHub}
\item[Categor y :]
\begin{itemize}
\tightlist
\item[]
\item
  \pandocbounded{\includesvg[keepaspectratio]{/assets/icons/16-package.svg}}
  \href{https://typst.app/universe/search/?category=components}{Components}
\end{itemize}
\end{description}

\subsubsection{Where to report issues?}\label{where-to-report-issues}

This package is a project of Nico Neumann (Bi0T1N) . Report issues on
\href{https://github.com/Bi0T1N/typst-iconic-salmon-svg}{their
repository} . You can also try to ask for help with this package on the
\href{https://forum.typst.app}{Forum} .

Please report this package to the Typst team using the
\href{https://typst.app/contact}{contact form} if you believe it is a
safety hazard or infringes upon your rights.

\phantomsection\label{versions}
\subsubsection{Version history}\label{version-history}

\begin{longtable}[]{@{}ll@{}}
\toprule\noalign{}
Version & Release Date \\
\midrule\noalign{}
\endhead
\bottomrule\noalign{}
\endlastfoot
1.1.0 & May 23, 2024 \\
\href{https://typst.app/universe/package/iconic-salmon-svg/1.0.0/}{1.0.0}
& April 15, 2024 \\
\end{longtable}

Typst GmbH did not create this package and cannot guarantee correct
functionality of this package or compatibility with any version of the
Typst compiler or app.


\title{typst.app/universe/package/cuti}

\phantomsection\label{banner}
\section{cuti}\label{cuti}

{ 0.3.0 }

Easily simulate (fake) bold, italic and small capital characters.

\phantomsection\label{readme}
Cuti (/kjuË?ti/) is a package that simulates fake bold / fake italic /
fake small captials. This package is typically used on fonts that do not
have a \texttt{\ bold\ } weight, such as “SimSun�.

\subsection{Usage}\label{usage}

Please refer to the
\href{https://csimide.github.io/cuti-docs/en/}{Documentation} .

本 Package æ??供中æ--‡æ--‡æ¡£ï¼š
\href{https://csimide.github.io/cuti-docs/zh-CN/}{中æ--‡æ--‡æ¡£} 。

\subsubsection{Getting Started Quickly (For Chinese
User)}\label{getting-started-quickly-for-chinese-user}

Please add the following content at the beginning of the document:

\begin{Shaded}
\begin{Highlighting}[]
\NormalTok{\#import "@preview/cuti:0.3.0": show{-}cn{-}fakebold}
\NormalTok{\#show: show{-}cn{-}fakebold}
\end{Highlighting}
\end{Shaded}

Then, the bolding for SimHei, SimSun, and KaiTi fonts should work
correctly.

\subsection{Changelog}\label{changelog}

\subsubsection{\texorpdfstring{\texttt{\ 0.3.0\ }}{ 0.3.0 }}\label{section}

\begin{itemize}
\tightlist
\item
  feat: Add fake small caps feature by Tetragramm.
\item
  fix: \texttt{\ show-fakebold\ } may crash on Typst version 0.12.0.
\end{itemize}

\subsubsection{\texorpdfstring{\texttt{\ 0.2.1\ }}{ 0.2.1 }}\label{section-1}

\begin{itemize}
\tightlist
\item
  feat: The stroke of fake bold will use the same color as the text.
\item
  fix: Attempted to fix the issue with the spacing of punctuation in
  fake italic (\#2), but there are still problems.
\end{itemize}

\subsubsection{\texorpdfstring{\texttt{\ 0.2.0\ }}{ 0.2.0 }}\label{section-2}

\begin{itemize}
\tightlist
\item
  feat: Added fake italic functionality.
\end{itemize}

\subsubsection{\texorpdfstring{\texttt{\ 0.1.0\ }}{ 0.1.0 }}\label{section-3}

\begin{itemize}
\tightlist
\item
  Basic fake bold functionality.
\end{itemize}

\subsection{License}\label{license}

MIT License

This package refers to the following content:

\begin{itemize}
\tightlist
\item
  \href{https://zhuanlan.zhihu.com/p/19686102}{TeX and Chinese Character
  Processing: Fake Bold and Fake Italic}
\item
  Typst issue \href{https://github.com/typst/typst/issues/394}{\#394}
\item
  Typst issue \href{https://github.com/typst/typst/issues/2749}{\#2749}
  (The function \texttt{\ \_skew\ } comes from Enivex’s code.)
\end{itemize}

Thanks to Enter-tainer for the assistance.

\subsubsection{How to add}\label{how-to-add}

Copy this into your project and use the import as \texttt{\ cuti\ }

\begin{verbatim}
#import "@preview/cuti:0.3.0"
\end{verbatim}

\includesvg[width=0.16667in,height=0.16667in]{/assets/icons/16-copy.svg}

Check the docs for
\href{https://typst.app/docs/reference/scripting/\#packages}{more
information on how to import packages} .

\subsubsection{About}\label{about}

\begin{description}
\tightlist
\item[Author s :]
csimide , Enivex , \& Tetragramm
\item[License:]
MIT
\item[Current version:]
0.3.0
\item[Last updated:]
November 25, 2024
\item[First released:]
March 18, 2024
\item[Minimum Typst version:]
0.12.0
\item[Archive size:]
2.37 kB
\href{https://packages.typst.org/preview/cuti-0.3.0.tar.gz}{\pandocbounded{\includesvg[keepaspectratio]{/assets/icons/16-download.svg}}}
\item[Repository:]
\href{https://github.com/csimide/cuti}{GitHub}
\end{description}

\subsubsection{Where to report issues?}\label{where-to-report-issues}

This package is a project of csimide, Enivex, and Tetragramm . Report
issues on \href{https://github.com/csimide/cuti}{their repository} . You
can also try to ask for help with this package on the
\href{https://forum.typst.app}{Forum} .

Please report this package to the Typst team using the
\href{https://typst.app/contact}{contact form} if you believe it is a
safety hazard or infringes upon your rights.

\phantomsection\label{versions}
\subsubsection{Version history}\label{version-history}

\begin{longtable}[]{@{}ll@{}}
\toprule\noalign{}
Version & Release Date \\
\midrule\noalign{}
\endhead
\bottomrule\noalign{}
\endlastfoot
0.3.0 & November 25, 2024 \\
\href{https://typst.app/universe/package/cuti/0.2.1/}{0.2.1} & April 5,
2024 \\
\href{https://typst.app/universe/package/cuti/0.2.0/}{0.2.0} & March 20,
2024 \\
\href{https://typst.app/universe/package/cuti/0.1.0/}{0.1.0} & March 18,
2024 \\
\end{longtable}

Typst GmbH did not create this package and cannot guarantee correct
functionality of this package or compatibility with any version of the
Typst compiler or app.


\title{typst.app/universe/package/modern-technique-report}

\phantomsection\label{banner}
\phantomsection\label{template-thumbnail}
\pandocbounded{\includegraphics[keepaspectratio]{https://packages.typst.org/preview/thumbnails/modern-technique-report-0.1.0-small.webp}}

\section{modern-technique-report}\label{modern-technique-report}

{ 0.1.0 }

A template for creating modern-style technique report in Typst.

\href{/app?template=modern-technique-report&version=0.1.0}{Create
project in app}

\phantomsection\label{readme}
= Modern Technique Report

A template support modern technique report in Typst.

= Usage

\begin{Shaded}
\begin{Highlighting}[]
\NormalTok{\#import "@preview/modern{-}technique{-}report:0.1.0": *}

\NormalTok{\#show: modern{-}technique{-}report.with(}
\NormalTok{  title: [Project Name \textbackslash{} Multiline When too Long],}
\NormalTok{  subtitle: [}
\NormalTok{    *Fourth Edition*, \textbackslash{} by \_H.L. Royden\_ and \_P.M. Fitzpatrick\_}
\NormalTok{  ],}
\NormalTok{  series: [Mathematics Courses \textbackslash{} Framework Series],}
\NormalTok{  author: grid(}
\NormalTok{    align: left + horizon,}
\NormalTok{    columns: 3,}
\NormalTok{    inset: 7pt,}
\NormalTok{    [*Member*], [B. Alice], [qwertyuiop\textbackslash{}@youremail.com],}
\NormalTok{    [], [B. Alice], [qwertyuiop\textbackslash{}@youremail.com],}
\NormalTok{    [], [B. Alice], [qwertyuiop\textbackslash{}@youremail.com],}
\NormalTok{    [*Advisor*], [E. Eric], [qwertyuiop\textbackslash{}@youremail.com],}
\NormalTok{  ),}
\NormalTok{  date: datetime.today().display("[year] {-}{-} [month] {-}{-} [day]"),}
\NormalTok{  background: image("bg.jpg"),}
\NormalTok{  theme\_color: rgb(21, 74, 135),}
\NormalTok{  font: "New Computer Modern",}
\NormalTok{  title\_font: "Noto Sans",}
\NormalTok{)}
\end{Highlighting}
\end{Shaded}

Then a cover page and a content page will be automatically generated.
Template also manipulates the style of headings and some contents.

\href{/app?template=modern-technique-report&version=0.1.0}{Create
project in app}

\subsubsection{How to use}\label{how-to-use}

Click the button above to create a new project using this template in
the Typst app.

You can also use the Typst CLI to start a new project on your computer
using this command:

\begin{verbatim}
typst init @preview/modern-technique-report:0.1.0
\end{verbatim}

\includesvg[width=0.16667in,height=0.16667in]{/assets/icons/16-copy.svg}

\subsubsection{About}\label{about}

\begin{description}
\tightlist
\item[Author :]
aytony
\item[License:]
MIT
\item[Current version:]
0.1.0
\item[Last updated:]
April 16, 2024
\item[First released:]
April 16, 2024
\item[Minimum Typst version:]
0.11.0
\item[Archive size:]
132 kB
\href{https://packages.typst.org/preview/modern-technique-report-0.1.0.tar.gz}{\pandocbounded{\includesvg[keepaspectratio]{/assets/icons/16-download.svg}}}
\item[Categor ies :]
\begin{itemize}
\tightlist
\item[]
\item
  \pandocbounded{\includesvg[keepaspectratio]{/assets/icons/16-layout.svg}}
  \href{https://typst.app/universe/search/?category=layout}{Layout}
\item
  \pandocbounded{\includesvg[keepaspectratio]{/assets/icons/16-speak.svg}}
  \href{https://typst.app/universe/search/?category=report}{Report}
\end{itemize}
\end{description}

\subsubsection{Where to report issues?}\label{where-to-report-issues}

This template is a project of aytony . You can also try to ask for help
with this template on the \href{https://forum.typst.app}{Forum} .

Please report this template to the Typst team using the
\href{https://typst.app/contact}{contact form} if you believe it is a
safety hazard or infringes upon your rights.

\phantomsection\label{versions}
\subsubsection{Version history}\label{version-history}

\begin{longtable}[]{@{}ll@{}}
\toprule\noalign{}
Version & Release Date \\
\midrule\noalign{}
\endhead
\bottomrule\noalign{}
\endlastfoot
0.1.0 & April 16, 2024 \\
\end{longtable}

Typst GmbH did not create this template and cannot guarantee correct
functionality of this template or compatibility with any version of the
Typst compiler or app.


\title{typst.app/universe/package/ctxjs}

\phantomsection\label{banner}
\section{ctxjs}\label{ctxjs}

{ 0.2.0 }

Run javascript in contexts.

\phantomsection\label{readme}
A typst plugin to evaluate javascript code.

\begin{itemize}
\tightlist
\item
  multiple javascript contexts
\item
  load javascript modules as source or bytecode
\item
  simple evaluations
\item
  formated evaluations (execute your code with your typst data)
\item
  call functions
\item
  call functions in modules
\item
  create bytecode with an extra tool (ctxjs\_module\_bytecode\_builder)
\item
  allow later evaluation of javascript code
\end{itemize}

\subsection{Example}\label{example}

\begin{Shaded}
\begin{Highlighting}[]
\NormalTok{\#import "@preview/ctxjs:0.2.0"}

\NormalTok{\#\{}
\NormalTok{  \_ = ctxjs.create{-}context("context\_name")}
\NormalTok{  let test = ctxjs.eval("context\_name", "function test(data) \{return data + 2;\}")}
\NormalTok{  let returns{-}4 = ctxjs.call{-}function("context\_name", "test", (2,))}
\NormalTok{  let returns{-}6 = ctxjs.eval{-}format("context\_name", "test(\{test\})", (test: 4))}
\NormalTok{  let code = \textasciigrave{}\textasciigrave{}\textasciigrave{}}
\NormalTok{    export function another\_test\_function() \{ return \{data: \textquotesingle{}test\textquotesingle{}\}; \}}
\NormalTok{  \textasciigrave{}\textasciigrave{}\textasciigrave{};}
\NormalTok{  \_ = ctxjs.load{-}module{-}js("context\_name", "module\_name", code.text)}
\NormalTok{  let returns{-}array{-}with{-}another{-}test = ctxjs.get{-}module{-}properties("context\_name", "module\_name")}
\NormalTok{  let returns{-}data{-}with{-}test{-}string = ctxjs.call{-}module{-}function("context\_name", "module\_name", "another\_test\_function", ())}
\NormalTok{  let returns{-}8 = ctxjs.eval{-}format("context\_name", "test(\{test\})", (test: ctxjs.eval{-}later("4 + 4")))}
\NormalTok{\}}
\end{Highlighting}
\end{Shaded}

\subsubsection{How to add}\label{how-to-add}

Copy this into your project and use the import as \texttt{\ ctxjs\ }

\begin{verbatim}
#import "@preview/ctxjs:0.2.0"
\end{verbatim}

\includesvg[width=0.16667in,height=0.16667in]{/assets/icons/16-copy.svg}

Check the docs for
\href{https://typst.app/docs/reference/scripting/\#packages}{more
information on how to import packages} .

\subsubsection{About}\label{about}

\begin{description}
\tightlist
\item[Author :]
lublak
\item[License:]
MIT
\item[Current version:]
0.2.0
\item[Last updated:]
November 28, 2024
\item[First released:]
September 11, 2024
\item[Archive size:]
427 kB
\href{https://packages.typst.org/preview/ctxjs-0.2.0.tar.gz}{\pandocbounded{\includesvg[keepaspectratio]{/assets/icons/16-download.svg}}}
\item[Repository:]
\href{https://github.com/lublak/typst-ctxjs-package}{GitHub}
\end{description}

\subsubsection{Where to report issues?}\label{where-to-report-issues}

This package is a project of lublak . Report issues on
\href{https://github.com/lublak/typst-ctxjs-package}{their repository} .
You can also try to ask for help with this package on the
\href{https://forum.typst.app}{Forum} .

Please report this package to the Typst team using the
\href{https://typst.app/contact}{contact form} if you believe it is a
safety hazard or infringes upon your rights.

\phantomsection\label{versions}
\subsubsection{Version history}\label{version-history}

\begin{longtable}[]{@{}ll@{}}
\toprule\noalign{}
Version & Release Date \\
\midrule\noalign{}
\endhead
\bottomrule\noalign{}
\endlastfoot
0.2.0 & November 28, 2024 \\
\href{https://typst.app/universe/package/ctxjs/0.1.1/}{0.1.1} &
September 30, 2024 \\
\href{https://typst.app/universe/package/ctxjs/0.1.0/}{0.1.0} &
September 11, 2024 \\
\end{longtable}

Typst GmbH did not create this package and cannot guarantee correct
functionality of this package or compatibility with any version of the
Typst compiler or app.


\title{typst.app/universe/package/mantys}

\phantomsection\label{banner}
\phantomsection\label{template-thumbnail}
\pandocbounded{\includegraphics[keepaspectratio]{https://packages.typst.org/preview/thumbnails/mantys-0.1.4-small.webp}}

\section{mantys}\label{mantys}

{ 0.1.4 }

Helpers to build manuals for Typst packages.

\href{/app?template=mantys&version=0.1.4}{Create project in app}

\phantomsection\label{readme}
\begin{quote}
\textbf{MAN} uals for \textbf{TY} p \textbf{S} t
\end{quote}

Template for documenting \href{https://github.com/typst/typst}{typst}
packages and templates.

\subsection{Usage}\label{usage}

Just import the package at the beginning of your manual:

\begin{Shaded}
\begin{Highlighting}[]
\NormalTok{\#import "@preview/mantys:0.1.4": *}
\end{Highlighting}
\end{Shaded}

Mantys supports \textbf{Typst 0.11.0} and newer.

\subsection{Writing basics}\label{writing-basics}

A basic template for a manual could look like this:

\begin{Shaded}
\begin{Highlighting}[]
\NormalTok{\#import "@local/mantys:0.1.4": *}

\NormalTok{\#import "your{-}package.typ"}

\NormalTok{\#show: mantys.with(}
\NormalTok{    name:        "your{-}package{-}name",}
\NormalTok{    title:       [A title for the manual],}
\NormalTok{    subtitle:    [A subtitle for the manual],}
\NormalTok{    info:        [A short descriptive text for the package.],}
\NormalTok{    authors: "Your Name",}
\NormalTok{    url:     "https://github.com/repository/url",}
\NormalTok{    version: "0.0.1",}
\NormalTok{    date:        "date{-}of{-}release",}
\NormalTok{    abstract:    [}
\NormalTok{        A few paragraphs of text to describe the package.}
\NormalTok{    ],}

\NormalTok{    example{-}imports: (your{-}package: your{-}package)}
\NormalTok{)}

\NormalTok{// end of preamble}

\NormalTok{\# About}
\NormalTok{\#lorem(50)}

\NormalTok{\# Usage}
\NormalTok{\#lorem(50)}

\NormalTok{\# Available commands}
\NormalTok{\#lorem(50)}
\end{Highlighting}
\end{Shaded}

Use \texttt{\ \#command(name,\ ..args){[}description{]}\ } to describe
commands and \texttt{\ \#argument(name,\ ...){[}description{]}\ } for
arguments:

\begin{Shaded}
\begin{Highlighting}[]
\NormalTok{\#command("headline", arg[color], arg(size:1.8em), sarg[other{-}args], barg[body])[}
\NormalTok{    Renders a prominent headline using \#doc("meta/heading").}

\NormalTok{    \#argument("color", type:"color")[}
\NormalTok{    The color of the headline will be used as the background of a \#doc("layout/block") element containing the headline.}
\NormalTok{  ]}
\NormalTok{  \#argument("size", default:1.8em)[}
\NormalTok{    The text size for the headline.}
\NormalTok{  ]}
\NormalTok{  \#argument("sarg", is{-}sink:true)[}
\NormalTok{    Other options will get passed directly to \#doc("meta/heading").}
\NormalTok{  ]}
\NormalTok{  \#argument("body", type:"content")[}
\NormalTok{    The text for the headline.}
\NormalTok{  ]}

\NormalTok{  The headline is shown as a prominent colored block to highlight important news articles in the newsletter:}

\NormalTok{  \#example[\textasciigrave{}\textasciigrave{}\textasciigrave{}}
\NormalTok{  \#headline(blue, size: 2em, level: 3)[}
\NormalTok{    \#lorem(8)}
\NormalTok{  ]}
\NormalTok{  \textasciigrave{}\textasciigrave{}\textasciigrave{}]}
\NormalTok{]}
\end{Highlighting}
\end{Shaded}

The result might look something like this:

\pandocbounded{\includegraphics[keepaspectratio]{https://github.com/typst/packages/raw/main/packages/preview/mantys/0.1.4/docs/assets/headline-example.png}}

For a full reference of available commands read
\href{https://github.com/typst/packages/raw/main/packages/preview/mantys/0.1.4/docs/mantys-manual.pdf}{the
manual} .

\subsection{Changelog}\label{changelog}

\subsubsection{Version 0.1.4}\label{version-0.1.4}

\begin{itemize}
\tightlist
\item
  Fix missing links in outline (@tingerrr).
\item
  Fixed problem when evaluating default values with Tidy.
\end{itemize}

\subsubsection{Version 0.1.3}\label{version-0.1.3}

\begin{itemize}
\tightlist
\item
  Fix for some datatypes not being displayed properly (thanks to
  @tingerrr).
\item
  Fix for imbalanced outline columns (thanks again to @tingerrr).
\end{itemize}

\subsubsection{Version 0.1.2}\label{version-0.1.2}

\begin{itemize}
\tightlist
\item
  Added \href{https://typst.app/universe/package/hydra}{hydra} for
  better detection of headings in page headers (thanks to @tingerrr for
  the suggestion).
\item
  Fixed problem with multiple quotes around default string values in
  tidy docs.
\item
  Fixed datatypes linking to wrong documentation urls.
\end{itemize}

\subsubsection{Version 0.1.1}\label{version-0.1.1}

\begin{itemize}
\tightlist
\item
  Added template files for submission to \emph{Typst Universe} .
\end{itemize}

\subsubsection{Version 0.1.0}\label{version-0.1.0}

\begin{itemize}
\tightlist
\item
  Refactorings and some style changes
\item
  Updated manual.
\item
  Restructuring of package repository.
\end{itemize}

\subsubsection{Version 0.0.4}\label{version-0.0.4}

\begin{itemize}
\tightlist
\item
  Added integration with \href{https://github.com/Mc-Zen/tidy}{tidy} .
\item
  Fixed issue with types in argument boxes.
\item
  \texttt{\ \#lambda\ } now uses \texttt{\ \#dtype\ }
\end{itemize}

\paragraph{Breaking changes}\label{breaking-changes}

\begin{itemize}
\tightlist
\item
  Adapted \texttt{\ scope\ } argument for \texttt{\ eval\ } in examples.

  \begin{itemize}
  \tightlist
  \item
    \texttt{\ \#example()\ } , \texttt{\ \#side-by-side()\ } and
    \texttt{\ \#shortex()\ } now support the \texttt{\ scope\ } and
    \texttt{\ mode\ } argument.
  \item
    The option \texttt{\ example-imports\ } was replaced by
    \texttt{\ examples-scope\ } .
  \end{itemize}
\end{itemize}

\subsubsection{Version 0.0.3}\label{version-0.0.3}

\begin{itemize}
\item
  It is now possible to load a packages’ \texttt{\ typst.toml\ } file
  directly into \texttt{\ \#mantys\ } :

\begin{Shaded}
\begin{Highlighting}[]
\NormalTok{\#show: mantys.with( ..toml("typst.toml") )}
\end{Highlighting}
\end{Shaded}
\item
  Added some dependencies:

  \begin{itemize}
  \tightlist
  \item
    \href{https://github.com/jneug/typst-tools4typst}{jneug/typst-tools4typst}
    for some common utilities,
  \item
    \href{https://github.com/jneug/typst-codelst}{jneug/typst-codelst}
    for rendering examples and source code,
  \item
    \href{https://github.com/Pablo-Gonzalez-Calderon/showybox-package}{Pablo-Gonzalez-Calderon/showybox-package}
    for adding frames to different areas of a manual (like examples).
  \end{itemize}
\item
  Redesign of some elements:

  \begin{itemize}
  \tightlist
  \item
    Argument display in command descriptions,
  \item
    Alert boxes.
  \end{itemize}
\item
  Added \texttt{\ \#version(since:(),\ until:())\ } command to add
  version markers to commands.
\item
  Styles moved to a separate \texttt{\ theme.typ\ } file to allow easy
  customization of colors and styles.
\item
  Added \texttt{\ \#func()\ } , \texttt{\ \#lambda()\ } and
  \texttt{\ \#symbol()\ } commands, to handle special cases for values.
\item
  Fixes and code improvements.
\end{itemize}

\subsubsection{Version 0.0.2}\label{version-0.0.2}

\begin{itemize}
\tightlist
\item
  Some major updates to the core commands and styles.
\end{itemize}

\subsubsection{Version 0.0.1}\label{version-0.0.1}

\begin{itemize}
\tightlist
\item
  Initial release.
\end{itemize}

\href{/app?template=mantys&version=0.1.4}{Create project in app}

\subsubsection{How to use}\label{how-to-use}

Click the button above to create a new project using this template in
the Typst app.

You can also use the Typst CLI to start a new project on your computer
using this command:

\begin{verbatim}
typst init @preview/mantys:0.1.4
\end{verbatim}

\includesvg[width=0.16667in,height=0.16667in]{/assets/icons/16-copy.svg}

\subsubsection{About}\label{about}

\begin{description}
\tightlist
\item[Author :]
Jonas Neugebauer
\item[License:]
MIT
\item[Current version:]
0.1.4
\item[Last updated:]
May 23, 2024
\item[First released:]
March 21, 2024
\item[Minimum Typst version:]
0.11.0
\item[Archive size:]
19.7 kB
\href{https://packages.typst.org/preview/mantys-0.1.4.tar.gz}{\pandocbounded{\includesvg[keepaspectratio]{/assets/icons/16-download.svg}}}
\item[Repository:]
\href{https://github.com/jneug/typst-mantys}{GitHub}
\item[Categor ies :]
\begin{itemize}
\tightlist
\item[]
\item
  \pandocbounded{\includesvg[keepaspectratio]{/assets/icons/16-layout.svg}}
  \href{https://typst.app/universe/search/?category=layout}{Layout}
\item
  \pandocbounded{\includesvg[keepaspectratio]{/assets/icons/16-list-unordered.svg}}
  \href{https://typst.app/universe/search/?category=model}{Model}
\item
  \pandocbounded{\includesvg[keepaspectratio]{/assets/icons/16-hammer.svg}}
  \href{https://typst.app/universe/search/?category=utility}{Utility}
\end{itemize}
\end{description}

\subsubsection{Where to report issues?}\label{where-to-report-issues}

This template is a project of Jonas Neugebauer . Report issues on
\href{https://github.com/jneug/typst-mantys}{their repository} . You can
also try to ask for help with this template on the
\href{https://forum.typst.app}{Forum} .

Please report this template to the Typst team using the
\href{https://typst.app/contact}{contact form} if you believe it is a
safety hazard or infringes upon your rights.

\phantomsection\label{versions}
\subsubsection{Version history}\label{version-history}

\begin{longtable}[]{@{}ll@{}}
\toprule\noalign{}
Version & Release Date \\
\midrule\noalign{}
\endhead
\bottomrule\noalign{}
\endlastfoot
0.1.4 & May 23, 2024 \\
\href{https://typst.app/universe/package/mantys/0.1.3/}{0.1.3} & April
29, 2024 \\
\href{https://typst.app/universe/package/mantys/0.1.1/}{0.1.1} & March
21, 2024 \\
\end{longtable}

Typst GmbH did not create this template and cannot guarantee correct
functionality of this template or compatibility with any version of the
Typst compiler or app.


\title{typst.app/universe/package/modernpro-coverletter}

\phantomsection\label{banner}
\phantomsection\label{template-thumbnail}
\pandocbounded{\includegraphics[keepaspectratio]{https://packages.typst.org/preview/thumbnails/modernpro-coverletter-0.0.5-small.webp}}

\section{modernpro-coverletter}\label{modernpro-coverletter}

{ 0.0.5 }

A cover letter template with modern Sans font for job applications and
other formal letters.

\href{/app?template=modernpro-coverletter&version=0.0.5}{Create project
in app}

\phantomsection\label{readme}
This is a cover letter template for Typst with Sans font. It is a modern
and professional cover letter template. It is easy to use and customize.
This cover letter template is suitable for any job application or
general purpose.

If you want to find a CV template, you can check out
\href{https://github.com/jxpeng98/Typst-CV-Resume/blob/main/README.md}{modernpro-cv}
.

\subsection{How to use}\label{how-to-use}

\subsubsection{Use from the Typst
Universe}\label{use-from-the-typst-universe}

It is simple and easy to use this template from the Typst Universe. If
you prefer to use the local editor and \texttt{\ typst-cli\ } , you can
use the following command to create a new cover letter project with this
template.

\begin{Shaded}
\begin{Highlighting}[]
\ExtensionTok{typst}\NormalTok{ init @preview/modernpro{-}coverletter}
\end{Highlighting}
\end{Shaded}

It will create a new cover letter project with this template in the
current directory.

\subsubsection{Use from GitHub}\label{use-from-github}

You can also use this template from GitHub. You can clone this
repository and use it as a normal project.

\begin{Shaded}
\begin{Highlighting}[]
\FunctionTok{git}\NormalTok{ clone https://github.com/jxpeng98/typst{-}coverletter.git}
\end{Highlighting}
\end{Shaded}

\subsection{Features}\label{features}

This package provides one \textbf{cover letter} template and one
\textbf{statement} template.

\subsubsection{Cover Letter}\label{cover-letter}

\begin{Shaded}
\begin{Highlighting}[]
\NormalTok{\#import "@preview/fontawesome:0.5.0": *}
\NormalTok{\#import "@preview/modernpro{-}coverletter:0.0.5": *}

\NormalTok{\#show: coverletter.with(}
\NormalTok{  font{-}type: "PT Serif",}
\NormalTok{  name: [example],}
\NormalTok{  address: [],}
\NormalTok{  contacts: (}
\NormalTok{    (text: [\#fa{-}icon("location{-}dot") UK]),}
\NormalTok{    (text: [123{-}456{-}789], link: "tel:123{-}456{-}789"),}
\NormalTok{    (text: [example.com], link: "https://www.example.com"),}
\NormalTok{    (text: [github], link: "https://github.com/"),}
\NormalTok{    (text: [example\textbackslash{}@example.com], link: "mailto:example@example.com"),}
\NormalTok{  ),}
\NormalTok{  recipient: (}
\NormalTok{    start{-}title: [],}
\NormalTok{    cl{-}title: [],}
\NormalTok{    date: [],}
\NormalTok{    department: [],}
\NormalTok{    institution: [],}
\NormalTok{    address: [],}
\NormalTok{    postcode: [],}
\NormalTok{  ),}
\NormalTok{)}

\NormalTok{\#set par(justify: true, first{-}line{-}indent: 2em)}
\NormalTok{\#set text(weight: "regular", size: 12pt)}
\end{Highlighting}
\end{Shaded}

\begin{longtable}[]{@{}ll@{}}
\toprule\noalign{}
Parameter & Description \\
\midrule\noalign{}
\endhead
\bottomrule\noalign{}
\endlastfoot
\texttt{\ font-type\ } & The font type of the cover letter, e.g. “PT
Serif� \\
\texttt{\ name\ } & The name of the sender \\
\texttt{\ address\ } & The address of the sender \\
\texttt{\ contacts\ } & The contact information of the
sender(text:{[}{]}, link: {[}{]}) \\
\end{longtable}

\begin{longtable}[]{@{}ll@{}}
\toprule\noalign{}
Parameter in Recipient & Description \\
\midrule\noalign{}
\endhead
\bottomrule\noalign{}
\endlastfoot
\texttt{\ start-title\ } & The start title of the letter \\
\texttt{\ cl-title\ } & The title of the letter (i.g., Job Application
for Hiring Manager) \\
\texttt{\ date\ } & The date of the letter(If “� or {[}{]}, it will
generate the current date) \\
\texttt{\ department\ } & The department of the recipient, can be “�
or {[}{]} \\
\texttt{\ institution\ } & The institution of the recipient \\
\texttt{\ address\ } & The address of the recipient \\
\texttt{\ postcode\ } & The postcode of the recipient \\
\end{longtable}

\subsubsection{Statement}\label{statement}

\begin{Shaded}
\begin{Highlighting}[]
\NormalTok{\#import "@preview/fontawesome:0.5.0": *}
\NormalTok{\#import "@preview/modernpro{-}coverletter:0.0.5": *}

\NormalTok{\#show: statement.with(}
\NormalTok{  font{-}type: "PT Serif",}
\NormalTok{  name: [],}
\NormalTok{  address: [],}
\NormalTok{  contacts: (}
\NormalTok{    (text: [\#fa{-}icon("location{-}dot")]),}
\NormalTok{    (text: [\#fa{-}icon("mobile") 123{-}456{-}789], link: "tel:123{-}456{-}789"),}
\NormalTok{    (text: [\#fa{-}icon("link") example.com], link: "https://www.example.com"),}
\NormalTok{    (text: [\#fa{-}icon("github") github], link: "https://github.com/"),}
\NormalTok{    (text: [\#fa{-}icon("envelope") example\textbackslash{}@example.com], link: "mailto:example@example.com"),}
\NormalTok{  ),}
\NormalTok{)}

\NormalTok{\#v(1em)}
\NormalTok{\#align(center, text(13pt, weight: "semibold")[\#underline([Title])])}
\NormalTok{\#set par(first{-}line{-}indent: 2em, justify: true)}
\NormalTok{\#set text(11pt, weight: "regular")}

\NormalTok{// Main body of the statement}
\end{Highlighting}
\end{Shaded}

\begin{longtable}[]{@{}ll@{}}
\toprule\noalign{}
Parameter & Description \\
\midrule\noalign{}
\endhead
\bottomrule\noalign{}
\endlastfoot
\texttt{\ font-type\ } & The font type of the cover letter, e.g. “PT
Serif� \\
\texttt{\ name\ } & The name of the sender \\
\texttt{\ address\ } & The address of the sender \\
\texttt{\ contacts\ } & The contact information of the
sender(text:{[}{]}, link: {[}{]}) \\
\end{longtable}

\subsubsection{Icons}\label{icons}

The new version also integrates the FontAwesome icons. You can use the
\texttt{\ \#fa-icon("icon")\ } function to insert the icons in the cover
letter or statement template as shown above.

You just need to import the FontAwesome package at the beginning of the
document.

\begin{Shaded}
\begin{Highlighting}[]
\NormalTok{\#import "@preview/fontawesome:0.5.0": *}
\end{Highlighting}
\end{Shaded}

\subsection{Preview}\label{preview}

\subsubsection{Cover Letter}\label{cover-letter-1}

\pandocbounded{\includegraphics[keepaspectratio]{https://minioapi.pjx.ac.cn/img1/2024/08/79decf8975b899d31b9dc76c5466a01a.png}}

\subsubsection{Statement}\label{statement-1}

\pandocbounded{\includegraphics[keepaspectratio]{https://minioapi.pjx.ac.cn/img1/2024/08/0483a06862932e1e9a9f1589676ce862.png}}

\href{/app?template=modernpro-coverletter&version=0.0.5}{Create project
in app}

\subsubsection{How to use}\label{how-to-use-1}

Click the button above to create a new project using this template in
the Typst app.

You can also use the Typst CLI to start a new project on your computer
using this command:

\begin{verbatim}
typst init @preview/modernpro-coverletter:0.0.5
\end{verbatim}

\includesvg[width=0.16667in,height=0.16667in]{/assets/icons/16-copy.svg}

\subsubsection{About}\label{about}

\begin{description}
\tightlist
\item[Author :]
jxpeng98
\item[License:]
MIT
\item[Current version:]
0.0.5
\item[Last updated:]
October 22, 2024
\item[First released:]
April 29, 2024
\item[Archive size:]
2.97 kB
\href{https://packages.typst.org/preview/modernpro-coverletter-0.0.5.tar.gz}{\pandocbounded{\includesvg[keepaspectratio]{/assets/icons/16-download.svg}}}
\item[Repository:]
\href{https://github.com/jxpeng98/typst-coverletter}{GitHub}
\item[Categor ies :]
\begin{itemize}
\tightlist
\item[]
\item
  \pandocbounded{\includesvg[keepaspectratio]{/assets/icons/16-user.svg}}
  \href{https://typst.app/universe/search/?category=cv}{CV}
\item
  \pandocbounded{\includesvg[keepaspectratio]{/assets/icons/16-hammer.svg}}
  \href{https://typst.app/universe/search/?category=utility}{Utility}
\end{itemize}
\end{description}

\subsubsection{Where to report issues?}\label{where-to-report-issues}

This template is a project of jxpeng98 . Report issues on
\href{https://github.com/jxpeng98/typst-coverletter}{their repository} .
You can also try to ask for help with this template on the
\href{https://forum.typst.app}{Forum} .

Please report this template to the Typst team using the
\href{https://typst.app/contact}{contact form} if you believe it is a
safety hazard or infringes upon your rights.

\phantomsection\label{versions}
\subsubsection{Version history}\label{version-history}

\begin{longtable}[]{@{}ll@{}}
\toprule\noalign{}
Version & Release Date \\
\midrule\noalign{}
\endhead
\bottomrule\noalign{}
\endlastfoot
0.0.5 & October 22, 2024 \\
\href{https://typst.app/universe/package/modernpro-coverletter/0.0.4/}{0.0.4}
& September 2, 2024 \\
\href{https://typst.app/universe/package/modernpro-coverletter/0.0.3/}{0.0.3}
& August 14, 2024 \\
\href{https://typst.app/universe/package/modernpro-coverletter/0.0.2/}{0.0.2}
& July 29, 2024 \\
\href{https://typst.app/universe/package/modernpro-coverletter/0.0.1/}{0.0.1}
& April 29, 2024 \\
\end{longtable}

Typst GmbH did not create this template and cannot guarantee correct
functionality of this template or compatibility with any version of the
Typst compiler or app.


\title{typst.app/universe/package/wonderous-book}

\phantomsection\label{banner}
\phantomsection\label{template-thumbnail}
\pandocbounded{\includegraphics[keepaspectratio]{https://packages.typst.org/preview/thumbnails/wonderous-book-0.1.1-small.webp}}

\section{wonderous-book}\label{wonderous-book}

{ 0.1.1 }

A fiction book template with running headers and serif typography

\href{/app?template=wonderous-book&version=0.1.1}{Create project in app}

\phantomsection\label{readme}
A book template for fiction. The template contains a title page, a table
of contents, and a chapter template.

Dynamic running headers contain the title of the chapter and the book.

\subsection{Usage}\label{usage}

You can use this template in the Typst web app by clicking “Start from
template� on the dashboard and searching for
\texttt{\ wonderous-book\ } .

Alternatively, you can use the CLI to kick this project off using the
command

\begin{verbatim}
typst init @preview/wonderous-book
\end{verbatim}

Typst will create a new directory with all the files needed to get you
started.

\subsection{Configuration}\label{configuration}

This template exports the \texttt{\ book\ } function with the following
named arguments:

\begin{itemize}
\tightlist
\item
  \texttt{\ title\ } : The book’s title as content.
\item
  \texttt{\ author\ } : Content or an array of content to specify the
  author.
\item
  \texttt{\ paper-size\ } : Defaults to \texttt{\ iso-b5\ } . Specify a
  \href{https://typst.app/docs/reference/layout/page/\#parameters-paper}{paper
  size string} to change the page format.
\item
  \texttt{\ dedication\ } : Who or what this book is dedicated to as
  content or \texttt{\ none\ } . Will appear on its own page.
\item
  \texttt{\ publishing-info\ } : Details for the front matter of this
  book as content or \texttt{\ none\ } .
\end{itemize}

The function also accepts a single, positional argument for the body of
the book.

The template will initialize your package with a sample call to the
\texttt{\ book\ } function in a show rule. If you, however, want to
change an existing project to use this template, you can add a show rule
like this at the top of your file:

\begin{Shaded}
\begin{Highlighting}[]
\NormalTok{\#import "@preview/wonderous{-}book:0.1.1": book}

\NormalTok{\#show: book.with(}
\NormalTok{  title: [Liam\textquotesingle{}s Playlist],}
\NormalTok{  author: "Janet Doe",}
\NormalTok{  dedication: [for Rachel],}
\NormalTok{  publishing{-}info: [}
\NormalTok{    UK Publishing, Inc. \textbackslash{}}
\NormalTok{    6 Abbey Road \textbackslash{}}
\NormalTok{    Vaughnham, 1PX 8A3}

\NormalTok{    \#link("https://example.co.uk/")}

\NormalTok{    971{-}1{-}XXXXXX{-}XX{-}X}
\NormalTok{  ],}
\NormalTok{)}

\NormalTok{// Your content goes below.}
\end{Highlighting}
\end{Shaded}

\href{/app?template=wonderous-book&version=0.1.1}{Create project in app}

\subsubsection{How to use}\label{how-to-use}

Click the button above to create a new project using this template in
the Typst app.

You can also use the Typst CLI to start a new project on your computer
using this command:

\begin{verbatim}
typst init @preview/wonderous-book:0.1.1
\end{verbatim}

\includesvg[width=0.16667in,height=0.16667in]{/assets/icons/16-copy.svg}

\subsubsection{About}\label{about}

\begin{description}
\tightlist
\item[Author :]
\href{https://typst.app}{Typst GmbH}
\item[License:]
MIT-0
\item[Current version:]
0.1.1
\item[Last updated:]
October 29, 2024
\item[First released:]
March 6, 2024
\item[Minimum Typst version:]
0.12.0
\item[Archive size:]
4.06 kB
\href{https://packages.typst.org/preview/wonderous-book-0.1.1.tar.gz}{\pandocbounded{\includesvg[keepaspectratio]{/assets/icons/16-download.svg}}}
\item[Repository:]
\href{https://github.com/typst/templates}{GitHub}
\item[Categor y :]
\begin{itemize}
\tightlist
\item[]
\item
  \pandocbounded{\includesvg[keepaspectratio]{/assets/icons/16-docs.svg}}
  \href{https://typst.app/universe/search/?category=book}{Book}
\end{itemize}
\end{description}

\subsubsection{Where to report issues?}\label{where-to-report-issues}

This template is a project of Typst GmbH . Report issues on
\href{https://github.com/typst/templates}{their repository} . You can
also try to ask for help with this template on the
\href{https://forum.typst.app}{Forum} .

\phantomsection\label{versions}
\subsubsection{Version history}\label{version-history}

\begin{longtable}[]{@{}ll@{}}
\toprule\noalign{}
Version & Release Date \\
\midrule\noalign{}
\endhead
\bottomrule\noalign{}
\endlastfoot
0.1.1 & October 29, 2024 \\
\href{https://typst.app/universe/package/wonderous-book/0.1.0/}{0.1.0} &
March 6, 2024 \\
\end{longtable}


