\title{sitandr.github.io/typst-examples-book/book/basics/states/query}

\section{\texorpdfstring{\hyperref[query]{Query}}{Query}}\label{query}

This section is outdated. It may be still useful, but it is strongly
recommended to study new context system (using the reference).

\begin{quote}
Link \href{https://typst.app/docs/reference/meta/query/}{there}
\end{quote}

Query is a thing that allows you getting location by \emph{selector}
(this is the same thing we used in show rules).

That enables "time travel", getting information about document from its
parts and so on. \emph{That is a way to violate Typst\textquotesingle s
purity.}

It is currently one of the \emph{the darkest magics currently existing
in Typst} . It gives you great powers, but with great power comes great
responsibility.

\subsection{\texorpdfstring{\hyperref[time-travel]{Time
travel}}{Time travel}}\label{time-travel}

\begin{verbatim}
#let s = state("x", 0)
#let compute(expr) = [
  #s.update(x =>
    eval(expr.replace("x", str(x)))
  )
  New value is #s.display().
]

Value at `<here>` is
#context s.at(
  query(<here>)
    .first()
    .location()
)

#compute("10") \
#compute("x + 3") \
*Here.* <here> \
#compute("x * 2") \
#compute("x - 5")
\end{verbatim}

\pandocbounded{\includesvg[keepaspectratio]{typst-img/130940aa5ae2ceb3364ef655c84cf8e7d2178210851b8fb20e6c0c3345c3ace7-1.svg}}

\subsection{\texorpdfstring{\hyperref[getting-nearest-chapter]{Getting
nearest
chapter}}{Getting nearest chapter}}\label{getting-nearest-chapter}

\begin{verbatim}
#set page(header: context {
  let elems = query(
    selector(heading).before(here()),
    here(),
  )
  let academy = smallcaps[
    Typst Academy
  ]
  if elems == () {
    align(right, academy)
  } else {
    let body = elems.last().body
    academy + h(1fr) + emph(body)
  }
})

= Introduction
#lorem(23)

= Background
#lorem(30)

= Analysis
#lorem(15)
\end{verbatim}

\pandocbounded{\includesvg[keepaspectratio]{typst-img/b4d0562911dd308b0d9cbc36ad20ba6ed91fc3c3da5b6259eb6721f3a53a18e3-1.svg}}


\title{sitandr.github.io/typst-examples-book/book/basics/states/states}

\section{\texorpdfstring{\hyperref[states]{States}}{States}}\label{states}

This section is outdated. It may be still useful, but it is strongly
recommended to study new context system (using the reference).

Before we start something practical, it is important to understand
states in general.

Here is a good explanation of why do we \emph{need} them:
\href{https://typst.app/docs/reference/meta/state/}{Official Reference
about states} . It is highly recommended to read it first.

So instead of

\begin{verbatim}
#let x = 0
#let compute(expr) = {
  // eval evaluates string as Typst code
  // to calculate new x value
  x = eval(
    expr.replace("x", str(x))
  )
  [New value is #x.]
}

#compute("10") \
#compute("x + 3") \
#compute("x * 2") \
#compute("x - 5")
\end{verbatim}

\textbf{THIS DOES NOT COMPILE:} Variables from outside the function are
read-only and cannot be modified

Instead, you should write

\begin{verbatim}
#let s = state("x", 0)
#let compute(expr) = [
  // updates x current state with this function
  #s.update(x =>
    eval(expr.replace("x", str(x)))
  )
  // and displays it
  New value is #context s.get().
]

#compute("10") \
#compute("x + 3") \
#compute("x * 2") \
#compute("x - 5")

The computations will be made _in order_ they are _located_ in the document. So if you create computations first, but put them in the document later... See yourself:

#let more = [
  #compute("x * 2") \
  #compute("x - 5")
]

#compute("10") \
#compute("x + 3") \
#more
\end{verbatim}

\pandocbounded{\includesvg[keepaspectratio]{typst-img/9a88397d1a9b5a44b1a3a218894595121bd4c5ec875df2b960638f2925060334-1.svg}}

\subsection{\texorpdfstring{\hyperref[context-magic]{Context
magic}}{Context magic}}\label{context-magic}

So what does this magic
\texttt{\ }{\texttt{\ context\ s.get()\ }}\texttt{\ } mean?

\begin{quote}
\href{https://typst.app/docs/reference/context/}{Context in Reference}
\end{quote}

In short, it specifies what part of your code (or markup) can
\emph{depend on states outside} . This context-expression is packed then
as one object, and it is evaluated on layout stage.

That means it is impossible to look from "normal" code at whatever is
inside the \texttt{\ }{\texttt{\ context\ }}\texttt{\ } . This is a
black box that would be known \emph{only after putting it into the
document} .

We will discuss \texttt{\ }{\texttt{\ context\ }}\texttt{\ } features
later.

\subsection{\texorpdfstring{\hyperref[operations-with-states]{Operations
with states}}{Operations with states}}\label{operations-with-states}

\subsubsection{\texorpdfstring{\hyperref[creating-new-state]{Creating
new state}}{Creating new state}}\label{creating-new-state}

\begin{verbatim}
#let x = state("state-id")
#let y = state("state-id", 2)

#x, #y

State is #context x.get() \ // the same as
#context [State is #y.get()] \ // the same as
#context {"State is" + str(y.get())}
\end{verbatim}

\pandocbounded{\includesvg[keepaspectratio]{typst-img/4a52375bdeea2b7ca31dc51740563d01b3678f817dd6bc8c349d0714c2ac503f-1.svg}}

\subsubsection{\texorpdfstring{\hyperref[update]{Update}}{Update}}\label{update}

Updating is \emph{a content} that is an instruction. That instruction
tells compiler that in this place of document the state \emph{should be
updated} .

\begin{verbatim}
#let x = state("x", 0)
#context x.get() \
#let _ = x.update(3)
// nothing happens, we don't put `update` into the document flow
#context x.get()

#repr(x.update(3)) // this is how that content looks \

#context x.update(3)
#context x.get() // Finally!
\end{verbatim}

\pandocbounded{\includesvg[keepaspectratio]{typst-img/3732a9c7bca8c4faedf9b024e09e647a65222c8244e9f3235a6057dfebc0a511-1.svg}}

Here we can see one of \emph{important
\texttt{\ }{\texttt{\ context\ }}\texttt{\ } traits} : it "sees" states
from outside, but can\textquotesingle t see how they change inside it:

\begin{verbatim}
#let x = state("x", 0)

#context {
  x.update(3)
  str(x.get())
}
\end{verbatim}

\pandocbounded{\includesvg[keepaspectratio]{typst-img/78e500b80cb85e086a81302e2ce3dad88cb4304d4685b88e3f59111bc71f6748-1.svg}}

\subsubsection{\texorpdfstring{\hyperref[id-collision]{ID
collision}}{ID collision}}\label{id-collision}

\emph{TLDR; \textbf{Never allow colliding states.}}

States are described by their id-s, if they are the same, the code will
break.

So, if you write functions or loops that are used several times,
\emph{be careful} !

\begin{verbatim}
#let f(x) = {
  // return new state…
  // …but their id-s are the same!
  // so it will always be the same state!
  let y = state("x", 0)
  y.update(y => y + x)
  context y.get()
}

#let a = f(2)
#let b = f(3)

#a, #b \
#raw(repr(a) + "\n" + repr(b))
\end{verbatim}

\pandocbounded{\includesvg[keepaspectratio]{typst-img/31a3e88747ed09ae6078bd3caf986f0e6ba744e055d0889d92bfa23941e7e451-1.svg}}

However, this \emph{may seem} okay:

\begin{verbatim}
// locations in code are different!
#let x = state("state-id")
#let y = state("state-id", 2)

#x, #y
\end{verbatim}

\pandocbounded{\includesvg[keepaspectratio]{typst-img/1901e1449942d821c66f53bd6bc5fda10d63591aa45346fdf88bcbc3f2ab3425-1.svg}}

But in fact, it \emph{isn\textquotesingle t} :

\begin{verbatim}
#let x = state("state-id")
#let y = state("state-id", 2)

#context [#x.get(); #y.get()]

#x.update(3)

#context [#x.get(); #y.get()]
\end{verbatim}

\pandocbounded{\includesvg[keepaspectratio]{typst-img/9185a298f9bcf8c519fa85481b9272e6ef3a00c117a0904d0509920a6abef8b2-1.svg}}


\title{sitandr.github.io/typst-examples-book/book/basics/states/metadata}

\section{\texorpdfstring{\hyperref[metadata]{Metadata}}{Metadata}}\label{metadata}

Metadata is invisible content that can be extracted using query or other
content. This may be very useful with
\texttt{\ }{\texttt{\ typst\ query\ }}\texttt{\ } to pass values to
external tools.

\begin{verbatim}
// Put metadata somewhere.
#metadata("This is a note") <note>

// And find it from anywhere else.
#context {
  query(<note>).first().value
}
\end{verbatim}

\pandocbounded{\includesvg[keepaspectratio]{typst-img/ef1c7d9faf74901f6c5266d48ae006167003a22754408a70ae9f9d1088b5fe24-1.svg}}


\title{sitandr.github.io/typst-examples-book/book/basics/states/counters}

\section{\texorpdfstring{\hyperref[counters]{Counters}}{Counters}}\label{counters}

This section is outdated. It may be still useful, but it is strongly
recommended to study new context system (using the reference).

Counters are special states that \emph{count} elements of some type. As
with states, you can create your own with identifier strings.

\emph{Important:} to initiate counters of elements, you need to
\emph{set numbering for them} .

\subsection{\texorpdfstring{\hyperref[states-methods]{States
methods}}{States methods}}\label{states-methods}

Counters are states, so they can do all things states can do.

\begin{verbatim}
#set heading(numbering: "1.")

= Background
#counter(heading).update(3)
#counter(heading).update(n => n * 2)

== Analysis
Current heading number: #counter(heading).display().
\end{verbatim}

\pandocbounded{\includesvg[keepaspectratio]{typst-img/c57c9907a5f238f0b5eee74f8c23c57a5e2d5b0c9cbf7ebd1befdfcbd33289df-1.svg}}

\begin{verbatim}
#let mine = counter("mycounter")
#mine.display()

#mine.step()
#mine.display()

#mine.update(c => c * 3)
#mine.display()
\end{verbatim}

\pandocbounded{\includesvg[keepaspectratio]{typst-img/876103777c9564f0bb524f83a988a6d444c4e889baed31ee960548d90f3233e2-1.svg}}

\subsection{\texorpdfstring{\hyperref[displaying-counters]{Displaying
counters}}{Displaying counters}}\label{displaying-counters}

\begin{verbatim}
#set heading(numbering: "1.")

= Introduction
Some text here.

= Background
The current value is:
#counter(heading).display()

Or in roman numerals:
#counter(heading).display("I")
\end{verbatim}

\pandocbounded{\includesvg[keepaspectratio]{typst-img/1ac65f4be42131b3cca1d7c56c6c60c3932a703e5e499c1c5cb874458028abea-1.svg}}

Counters also support displaying \emph{both current and final values}
out-of-box:

\begin{verbatim}
#set heading(numbering: "1.")

= Introduction
Some text here.

#counter(heading).display(both: true) \
#counter(heading).display("1 of 1", both: true) \
#counter(heading).display(
  (num, max) => [#num of #max],
   both: true
)

= Background
The current value is: #counter(heading).display()
\end{verbatim}

\pandocbounded{\includesvg[keepaspectratio]{typst-img/af9d0da905bbb2215461b07b39653ef3890ff11a364afe018dae4ce4216f4961-1.svg}}

\subsection{\texorpdfstring{\hyperref[step]{Step}}{Step}}\label{step}

That\textquotesingle s quite easy, for counters you can increment value
using \texttt{\ }{\texttt{\ step\ }}\texttt{\ } . It works the same way
as \texttt{\ }{\texttt{\ update\ }}\texttt{\ } .

\begin{verbatim}
#set heading(numbering: "1.")

= Introduction
#counter(heading).step()

= Analysis
Let's skip 3.1.
#counter(heading).step(level: 2)

== Analysis
At #counter(heading).display().
\end{verbatim}

\pandocbounded{\includesvg[keepaspectratio]{typst-img/12446a2258e9862d8df8b6b250ff14efbb9c35da165a2a04e8c4aa12c9b68cdf-1.svg}}

\subsection{\texorpdfstring{\hyperref[you-can-use-counters-in-your-functions]{You
can use counters in your
functions:}}{You can use counters in your functions:}}\label{you-can-use-counters-in-your-functions}

\begin{verbatim}
#let c = counter("theorem")
#let theorem(it) = block[
  #c.step()
  *Theorem #c.display():*
  #it
]

#theorem[$1 = 1$]
#theorem[$2 < 3$]
\end{verbatim}

\pandocbounded{\includesvg[keepaspectratio]{typst-img/0f178f614e49a7400d646926705364a92ca3d4d888423b2693f071f83ce09e7d-1.svg}}


\title{sitandr.github.io/typst-examples-book/book/basics/states/index}

\section{\texorpdfstring{\hyperref[states--query]{States \&
Query}}{States \& Query}}\label{states--query}

This section is outdated. It may be still useful, but it is strongly
recommended to study new context system (using the reference).

Typst tries to be a \emph{pure language} as much as possible.

That means, a function can\textquotesingle t change anything outside of
it. That also means, if you call function, the result should be always
the same.

Unfortunately, our world (and therefore our documents)
isn\textquotesingle t pure. If you create a heading №2, you want the
next number to be three.

That section will guide you to using impure Typst. Don\textquotesingle t
overuse it, as this knowledge comes close to the Dark Arts of Typst!


