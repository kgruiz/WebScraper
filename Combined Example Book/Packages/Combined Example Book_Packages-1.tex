\title{sitandr.github.io/typst-examples-book/book/packages/drawing}

\section{\texorpdfstring{\hyperref[drawing]{Drawing}}{Drawing}}\label{drawing}

\subsection{\texorpdfstring{\hyperref[cetz]{\texttt{\ }{\texttt{\ cetz\ }}\texttt{\ }}}{  cetz  }}\label{cetz}

Cetz is an analogue of LaTeX\textquotesingle s
\texttt{\ }{\texttt{\ tikz\ }}\texttt{\ } . Maybe it is not as powerful
yet, but certainly easier to learn and use.

It is the best choice in most of cases you want to draw something in
Typst.

\begin{verbatim}
#import "@preview/cetz:0.1.2"

#cetz.canvas(length: 1cm, {
  import cetz.draw: *
  import cetz.angle: angle
  let (a, b, c) = ((0,0), (-1,1), (1.5,0))
  line(a, b)
  line(a, c)
  set-style(angle: (radius: 1, label-radius: .5), stroke: blue)
  angle(a, c, b, label: $alpha$, mark: (end: ">"), stroke: blue)
  set-style(stroke: red)
  angle(a, b, c, label: n => $#{n/1deg} degree$,
    mark: (end: ">"), stroke: red, inner: false)
})
\end{verbatim}

\pandocbounded{\includesvg[keepaspectratio]{typst-img/d3b5277dd18dffb6da9a8f41486cb85a5044597821e80867652f062724ed8dd4-1.svg}}

\begin{verbatim}
#import "@preview/cetz:0.1.2": canvas, draw

#canvas(length: 1cm, {
  import draw: *
  intersections(name: "demo", {
    circle((0, 0))
    bezier((0,0), (3,0), (1,-1), (2,1))
    line((0,-1), (0,1))
    rect((1.5,-1),(2.5,1))
  })
  for-each-anchor("demo", (name) => {
    circle("demo." + name, radius: .1, fill: black)
  })
})
\end{verbatim}

\pandocbounded{\includesvg[keepaspectratio]{typst-img/05a1dbe2a2d17e5e81991406bed640775db6ab4ce2d585bc5a0d1175def43ea1-1.svg}}

\begin{verbatim}
#import "@preview/cetz:0.1.2": canvas, draw

#canvas(length: 1cm, {
  import draw: *
  let (a, b, c) = ((0, 0), (1, 1), (2, -1))
  line(a, b, c, stroke: gray)
  bezier-through(a, b, c, name: "b")
  // Show calculated control points
  line(a, "b.ctrl-1", "b.ctrl-2", c, stroke: gray)
})
\end{verbatim}

\pandocbounded{\includesvg[keepaspectratio]{typst-img/8e7d39d73212ebf8f230a0bd54a7fb7e58607a99f327e29809c4021b9e797345-1.svg}}

\begin{verbatim}
#import "@preview/cetz:0.1.2": canvas, draw

#canvas(length: 1cm, {
  import draw: *
  group(name: "g", {
    rotate(45deg)
    rect((0,0), (1,1), name: "r")
    copy-anchors("r")
  })
  circle("g.top", radius: .1, fill: black)
})
\end{verbatim}

\pandocbounded{\includesvg[keepaspectratio]{typst-img/b3d0b37a84cddb77a1508333743f851509e2250930abdcbda7ec4675e00077c3-1.svg}}

\begin{verbatim}
// author: LDemetrios
#import "@preview/cetz:0.2.2"

#cetz.canvas({
  let left = (a:2, b:1, d:-1, e:-2)
  let right = (p:2.7, q: 1.8, r: 0.9, s: -.3, t: -1.5, u: -2.4)
  let edges = "as,bq,dq,et".split(",")

  let ell-width = 1.5
  let ell-height = 3
  let dist = 5
  let dot-radius = 0.1
  let dot-clr = blue

  import cetz.draw: *
  circle((-dist/2, 0), radius:(ell-width ,  ell-height))
  circle((+dist/2, 0), radius:(ell-width ,  ell-height))

  for (name, y) in left {
    circle((-dist/2, y), radius:dot-radius, fill:dot-clr, name:name)
    content(name, anchor:"east", pad(right:.7em, text(fill:dot-clr, name)))
  }

  for (name, y) in right {
    circle((dist/2, y), radius:dot-radius, fill:dot-clr, name:name)
    content(name, anchor:"west", pad(left:.7em, text(fill:dot-clr, name)))
  }

  for edge in edges {
    let from = edge.at(0)
    let to = edge.at(1)
    line(from, to)
    mark(from, to, symbol: ">",  fill: black)
  }

  content((0, - ell-height), text(fill:blue)[APPLICATION], anchor:"south")
})
\end{verbatim}

\pandocbounded{\includesvg[keepaspectratio]{typst-img/7a4a9224b76305ecd694fd4505b3fdee8c706ccea76ac0e59fd13d469c343dd4-1.svg}}


\title{sitandr.github.io/typst-examples-book/book/packages/physics}

\section{\texorpdfstring{\hyperref[physics]{Physics}}{Physics}}\label{physics}

\subsection{\texorpdfstring{\hyperref[physica]{\texttt{\ }{\texttt{\ physica\ }}\texttt{\ }}}{  physica  }}\label{physica}

\begin{quote}
Physica (Latin for \emph{natural sciences} ) provides utilities that
simplify otherwise complex and repetitive mathematical expressions in
natural sciences.
\end{quote}

\begin{quote}
Its
\href{https://github.com/Leedehai/typst-physics/blob/master/physica-manual.pdf}{manual}
provides a full set of demonstrations of how the package could be
helpful.
\end{quote}

\subsubsection{\texorpdfstring{\hyperref[mathematical-physics]{Mathematical
physics}}{Mathematical physics}}\label{mathematical-physics}

The \href{./math.html\#common-notations}{packages/math.md} page has more
examples on its math capabilities. Below is a preview that may be of
particular interest in the domain of physics:

\begin{itemize}
\tightlist
\item
  Calculus: differential, ordinary and partial derivatives

  \begin{itemize}
  \tightlist
  \item
    Optional function name,
  \item
    Optional order number or array of order numbers,
  \item
    Customizable "d" symbol and product joiner (say, exterior product),
  \item
    Overridable total order calculation,
  \end{itemize}
\item
  Vectors and vector fields: div, grad, curl,
\item
  Taylor expansion,
\item
  Dirac braket notations,
\item
  Tensors with abstract index notations,
\item
  Matrix transpose and dagger (conjugate transpose).
\item
  Special matrices: determinant, (anti-)diagonal, identity, zero,
  Jacobian, Hessian, etc.
\end{itemize}

A partial glimpse:

\begin{verbatim}
#import "@preview/physica:0.9.1": *
#show: super-T-as-transpose // put in a #[...] to limit its scope...
#show: super-plus-as-dagger // ...or use scripts() to manually override

$ dd(x,y,2) quad dv(f,x,d:Delta)      quad pdv(,x,y,[2i+1,2+i]) quad
  vb(a) va(a) vu(a_i)  quad mat(laplacian, div; grad, curl)     quad
  tensor(T,+a,-b,+c)   quad ket(phi)  quad A^+ e^scripts(+) A^T integral^T $
\end{verbatim}

\pandocbounded{\includesvg[keepaspectratio]{typst-img/fa8a12d2904a08958d4f83d69dda6bb38308b431055a25790d286a250e364c6c-1.svg}}

\subsubsection{\texorpdfstring{\hyperref[isotopes]{Isotopes}}{Isotopes}}\label{isotopes}

\begin{verbatim}
#import "@preview/physica:0.9.1": isotope

// a: mass number A
// z: the atomic number Z
$
isotope(I, a:127), quad isotope("Fe", z:26), quad
isotope("Tl",a:207,z:81) --> isotope("Pb",a:207,z:82) + isotope(e,a:0,z:-1)
$
\end{verbatim}

\pandocbounded{\includesvg[keepaspectratio]{typst-img/b290d801c6760a41e50520401d9e72cb63a8691aa136308cbad87349e7e436f0-1.svg}}

\subsubsection{\texorpdfstring{\hyperref[reduced-planck-constant-hbar]{Reduced
Planck constant
(hbar)}}{Reduced Planck constant (hbar)}}\label{reduced-planck-constant-hbar}

In the default font, the Typst built-in symbol
\texttt{\ }{\texttt{\ planck.reduce\ }}\texttt{\ } looks a bit off: on
letter "h" there is a slash instead of a horizontal bar, contrary to the
symbol\textquotesingle s colloquial name "h-bar". This package offers
\texttt{\ }{\texttt{\ hbar\ }}\texttt{\ } to render the symbol in the
familiar form⁠. Contrast:

\begin{verbatim}
#import "@preview/physica:0.9.1": hbar

$ E = planck.reduce omega => E = hbar omega, wide
  frac(planck.reduce^2, 2m) => frac(hbar^2, 2m), wide
  (pi G^2) / (planck.reduce c^4) => (pi G^2) / (hbar c^4), wide
  e^(frac(i(p x - E t), planck.reduce)) => e^(frac(i(p x - E t), hbar)) $
\end{verbatim}

\pandocbounded{\includesvg[keepaspectratio]{typst-img/efab3b0486d1cddc3388248c4100e1cc919088cdb93f3e072001547c40005f01-1.svg}}

\subsection{\texorpdfstring{\hyperref[quill-quantum-diagrams]{\texttt{\ }{\texttt{\ quill\ }}\texttt{\ }
: quantum
diagrams}}{  quill   : quantum diagrams}}\label{quill-quantum-diagrams}

\begin{quote}
See \href{https://github.com/Mc-Zen/quill/tree/main}{documentation} .
\end{quote}

\begin{verbatim}
#import "@preview/quill:0.2.0": *
#quantum-circuit(
  lstick($|0〉$), gate($H$), ctrl(1), rstick($(|00〉+|11〉)/√2$, n: 2), [\ ],
  lstick($|0〉$), 1, targ(), 1
)
\end{verbatim}

\pandocbounded{\includesvg[keepaspectratio]{typst-img/bd14c65cd60e1efc4d15ae7234e364c6d5740a168e2cb275743ed1fbcc9483eb-1.svg}}

\begin{verbatim}
#import "@preview/quill:0.2.0": *

#let ancillas = (setwire(0), 5, lstick($|0〉$), setwire(1), targ(), 2, [\ ],
setwire(0), 5, lstick($|0〉$), setwire(1), 1, targ(), 1)

#quantum-circuit(
  scale-factor: 80%,
  lstick($|ψ〉$), 1, 10pt, ctrl(3), ctrl(6), $H$, 1, 15pt, 
    ctrl(1), ctrl(2), 1, [\ ],
  ..ancillas, [\ ],
  lstick($|0〉$), 1, targ(), 1, $H$, 1, ctrl(1), ctrl(2), 
    1, [\ ],
  ..ancillas, [\ ],
  lstick($|0〉$), 2, targ(),  $H$, 1, ctrl(1), ctrl(2), 
    1, [\ ],
  ..ancillas
)
\end{verbatim}

\pandocbounded{\includesvg[keepaspectratio]{typst-img/597640923e31369199c6e7158de9094a2c94f2c5dae6ced72c6b83b1067fa8e4-1.svg}}

\begin{verbatim}
#import "@preview/quill:0.2.0": *

#quantum-circuit(
  lstick($|psi〉$),  ctrl(1), gate($H$), 1, ctrl(2), meter(), [\ ],
  lstick($|beta_00〉$, n: 2), targ(), 1, ctrl(1), 1, meter(), [\ ],
  3, gate($X$), gate($Z$),  midstick($|psi〉$)
)
\end{verbatim}

\pandocbounded{\includesvg[keepaspectratio]{typst-img/cc71bc052c7a80c702289f780ee42a168c1491076dd5934408373895ca95c35e-1.svg}}


\title{sitandr.github.io/typst-examples-book/book/packages/wrapping}

\section{\texorpdfstring{\hyperref[wrapping-figures]{Wrapping
figures}}{Wrapping figures}}\label{wrapping-figures}

The better native support for wrapping is planned, however, something is
already possible via package:

\begin{verbatim}
#import "@preview/wrap-it:0.1.0": wrap-content, wrap-top-bottom

#set par(justify: true)
#let fig = figure(
  rect(fill: teal, radius: 0.5em, width: 8em),
  caption: [A figure],
)
#let body = lorem(40)
#wrap-content(fig, body)

#wrap-content(
  fig,
  body,
  align: bottom + right,
  column-gutter: 2em
)

#let boxed = box(fig, inset: 0.5em)
#wrap-content(boxed)[
  #lorem(40)
]

#let fig2 = figure(
  rect(fill: lime, radius: 0.5em),
  caption: [Another figure],
)
#wrap-top-bottom(boxed, fig2, lorem(60))
\end{verbatim}

\pandocbounded{\includesvg[keepaspectratio]{typst-img/1d249d6947bbea7f94c4f5f111c873f278dcf473e0cf672d6c55800c0eb6822c-1.svg}}

Limitations: non-ideal spacing near warping, only top-bottom left/right
are supported.


\title{sitandr.github.io/typst-examples-book/book/packages/math}

\section{\texorpdfstring{\hyperref[math]{Math}}{Math}}\label{math}

\subsection{\texorpdfstring{\hyperref[general]{General}}{General}}\label{general}

\subsubsection{\texorpdfstring{\hyperref[physica]{\texttt{\ }{\texttt{\ physica\ }}\texttt{\ }}}{  physica  }}\label{physica}

\begin{quote}
Physica (Latin for \emph{natural sciences} ) provides utilities that
simplify otherwise complex and repetitive mathematical expressions in
natural sciences.
\end{quote}

\begin{quote}
Its
\href{https://github.com/Leedehai/typst-physics/blob/master/physica-manual.pdf}{manual}
provides a full set of demonstrations of how the package could be
helpful.
\end{quote}

\paragraph{\texorpdfstring{\hyperref[common-notations]{Common
notations}}{Common notations}}\label{common-notations}

\begin{itemize}
\tightlist
\item
  Calculus: differential, ordinary and partial derivatives

  \begin{itemize}
  \tightlist
  \item
    Optional function name,
  \item
    Optional order number or an array of thereof,
  \item
    Customizable "d" symbol and product joiner (say, exterior product),
  \item
    Overridable total order calculation,
  \end{itemize}
\item
  Vectors and vector fields: div, grad, curl,
\item
  Taylor expansion,
\item
  Dirac braket notations,
\item
  Tensors with abstract index notations,
\item
  Matrix transpose and dagger (conjugate transpose).
\item
  Special matrices: determinant, (anti-)diagonal, identity, zero,
  Jacobian, Hessian, etc.
\end{itemize}

Below is a preview of those notations.

\begin{verbatim}
#import "@preview/physica:0.9.1": * // Symbol names annotated below

#table(
  columns: 4, align: horizon, stroke: none, gutter: 1em,

  // vectors: bold, unit, arrow
  [$ vb(a), vb(e_i), vu(a), vu(e_i), va(a), va(e_i) $],
  // dprod (dot product), cprod (cross product), iprod (innerproduct)
  [$ a dprod b, a cprod b, iprod(a, b) $],
  // laplacian (different from built-in laplace)
  [$ dot.double(u) = laplacian u =: laplace u $],
  // grad, div, curl (vector fields)
  [$ grad phi, div vb(E), \ curl vb(B) $],
)
\end{verbatim}

\pandocbounded{\includesvg[keepaspectratio]{typst-img/3be7ef86d6c5f7044a42c69fcf93afccd936eb0fcbe987122702c7dda467480f-1.svg}}

\begin{verbatim}
#import "@preview/physica:0.9.1": * // Symbol names annotated below

#table(
  columns: 4, align: horizon, stroke: none, gutter: 1em,

  // Row 1.
  // dd (differential), var (variation), difference
  [$ dd(f), var(f), difference(f) $],
  // dd, with an order number or an array thereof
  [$ dd(x,y), dd(x,y,2), \ dd(x,y,[1,n]), dd(vb(x),t,[3,]) $],
  // dd, with custom "d" symbol and joiner
  [$ dd(x,y,p:and), dd(x,y,d:Delta), \ dd(x,y,z,[1,1,n+1],d:d,p:dot) $],
  // dv (ordinary derivative), with custom "d" symbol and joiner
  [$ dv(phi,t,d:Delta), dv(phi,t,d:upright(D)), dv(phi,t,d:delta) $],

  // Row 2.
  // dv, with optional function name and order
  [$ dv(,t) (dv(x,t)) = dv(x,t,2) $],
  // pdv (partial derivative)
  [$ pdv(f,x,y,2), pdv(,x,y,[k,]) $],
  // pdv, with auto-added overridable total
  [$ pdv(,x,y,[i+2,i+1]), pdv(,y,x,[i+1,i+2],total:3+2i) $],
  // In a flat form
  [$ dv(u,x,s:slash), \ pdv(u,x,y,2,s:slash) $],
)
\end{verbatim}

\pandocbounded{\includesvg[keepaspectratio]{typst-img/0835a840454f88ed2e3b98ddfe37d6f8026729812372a6298d86129611f348c3-1.svg}}

\begin{verbatim}
#import "@preview/physica:0.9.1": * // Symbol names annotated below

#table(
  columns: 3, align: horizon, stroke: none, gutter: 1em,

  // tensor
  [$ tensor(T,+a,-b,-c) != tensor(T,-b,-c,+a) != tensor(T,+a',-b,-c) $],
  // Set builder notation
  [$ Set(p, {q^*, p} = 1) $],
  // taylorterm (Taylor series term)
  [$ taylorterm(f,x,x_0,1) \ taylorterm(f,x,x_0,(n+1)) $],
)
\end{verbatim}

\pandocbounded{\includesvg[keepaspectratio]{typst-img/5c08b65761578f38762229692a33e2b05f096aa8fb7859238b1018240f054d10-1.svg}}

\begin{verbatim}
#import "@preview/physica:0.9.1": * // Symbol names annotated below

#table(
  columns: 3, align: horizon, stroke: none, gutter: 1em,

  // expval (mean/expectation value), eval (evaluation boundary)
  [$ expval(X) = eval(f(x)/g(x))^oo_1 $],
  // Dirac braket notations
  [$
    bra(u), braket(u), braket(u, v), \
    ket(u), ketbra(u), ketbra(u, v), \
    mel(phi, hat(p), psi) $],
  // Superscript show rules that need to be enabled explicitly.
  // If put in a content block, they only control that block's scope.
  [
    #show: super-T-as-transpose // "..^T" just like handwriting
    #show: super-plus-as-dagger // "..^+" just like handwriting
    $ op("conj")A^T =^"def" A^+ \
      e^scripts(T), e^scripts(+) $ ], // Override with scripts()
)
\end{verbatim}

\pandocbounded{\includesvg[keepaspectratio]{typst-img/8965812a9c349892988d61872ff06418581098d15169b775f67b30e3460dd854-1.svg}}

\paragraph{\texorpdfstring{\hyperref[matrices]{Matrices}}{Matrices}}\label{matrices}

In addition to Typst\textquotesingle s built-in
\texttt{\ }{\texttt{\ mat()\ }}\texttt{\ } to write a matrix, physica
provides a number of handy tools to make it even easier.

\begin{verbatim}
#import "@preview/physica:0.9.1": TT, mdet

$
// Matrix transpose with "TT", though it is recommended to
// use super-T-as-transpose so that "A^T" also works (more on that later).
A^TT,
// Determinant with "mdet(...)".
det mat(a, b; c, d) := mdet(a, b; c, d)
$
\end{verbatim}

\pandocbounded{\includesvg[keepaspectratio]{typst-img/7eccaa3a0cf838bca4daf9ebf573452506d3ea724086fcda0c9eb4264e66b5d9-1.svg}}

Diagonal matrix
\texttt{\ }{\texttt{\ dmat(\ }}\texttt{\ }{\texttt{\ ...\ }}\texttt{\ }{\texttt{\ )\ }}\texttt{\ }
, antidiagonal matrix
\texttt{\ }{\texttt{\ admat(\ }}\texttt{\ }{\texttt{\ ...\ }}\texttt{\ }{\texttt{\ )\ }}\texttt{\ }
, identity matrix \texttt{\ }{\texttt{\ imat(n)\ }}\texttt{\ } , and
zero matrix \texttt{\ }{\texttt{\ zmat(n)\ }}\texttt{\ } .

\begin{verbatim}
#import "@preview/physica:0.9.1": dmat, admat, imat, zmat

$ dmat(1, 2)  dmat(1, a_1, xi, fill:0)               quad
  admat(1, 2) admat(1, a_1, xi, fill:dot, delim:"[") quad
  imat(2)     imat(3, delim:"{",fill:*) quad
  zmat(2)     zmat(3, delim:"|") $
\end{verbatim}

\pandocbounded{\includesvg[keepaspectratio]{typst-img/66bbf5294be293cc58d98de6ca078eb17c58539169325e6b59a6aa78e7a49f62-1.svg}}

Jacobian matrix with
\texttt{\ }{\texttt{\ jmat(func;\ }}\texttt{\ }{\texttt{\ ...\ }}\texttt{\ }{\texttt{\ )\ }}\texttt{\ }
or the longer name \texttt{\ }{\texttt{\ jacobianmatrix\ }}\texttt{\ } ,
Hessian matrix with
\texttt{\ }{\texttt{\ hmat(func;\ }}\texttt{\ }{\texttt{\ ...\ }}\texttt{\ }{\texttt{\ )\ }}\texttt{\ }
or the longer name \texttt{\ }{\texttt{\ hessianmatrix\ }}\texttt{\ } ,
and finally \texttt{\ }{\texttt{\ xmat(row,\ col,\ func)\ }}\texttt{\ }
to build a matrix.

\begin{verbatim}
#import "@preview/physica:0.9.1": jmat, hmat, xmat

$
jmat(f_1,f_2; x,y) jmat(f,g; x,y,z; delim:"[") quad
hmat(f; x,y)       hmat(; x,y; big:#true)      quad

#let elem-ij = (i,j) => $g^(#(i - 1)#(j - 1)) = #calc.pow(i,j)$
xmat(2, 2, #elem-ij)
$
\end{verbatim}

\pandocbounded{\includesvg[keepaspectratio]{typst-img/7cd30cef52b187d17459d7806a94d5ae56118d0f969760bbbaabeb83007e6869-1.svg}}

\subsubsection{\texorpdfstring{\hyperref[mitex]{\texttt{\ }{\texttt{\ mitex\ }}\texttt{\ }}}{  mitex  }}\label{mitex}

\begin{quote}
MiTeX provides LaTeX support powered by WASM in Typst, including
real-time rendering of LaTeX math equations. You can also use LaTeX
syntax to write
\texttt{\ }{\texttt{\ \textbackslash{}r\ }}\texttt{\ }{\texttt{\ ef\ }}\texttt{\ }
and
\texttt{\ }{\texttt{\ \textbackslash{}l\ }}\texttt{\ }{\texttt{\ abel\ }}\texttt{\ }
.
\end{quote}

\begin{quote}
Please refer to the \href{https://github.com/mitex-rs/mitex}{manual} for
more details.
\end{quote}

\begin{verbatim}
#import "@preview/mitex:0.2.4": *

Write inline equations like #mi("x") or #mi[y].

Also block equations:

#mitex(`
  \newcommand{\f}[2]{#1f(#2)}
  \f\relax{x} = \int_{-\infty}^\infty
    \f\hat\xi\,e^{2 \pi i \xi x}
    \,d\xi
`)

Text mode:

#mitext(`
  \iftypst
    #set math.equation(numbering: "(1)", supplement: "equation")
  \fi

  An inline equation $x + y$ and block \eqref{eq:pythagoras}.

  \begin{equation}
    a^2 + b^2 = c^2 \label{eq:pythagoras}
  \end{equation}
`)
\end{verbatim}

\pandocbounded{\includesvg[keepaspectratio]{typst-img/a3ff500a39b6d93d85b223af0aa162a5bfbe93fad3436dba80ee022638ed727a-1.svg}}

\subsubsection{\texorpdfstring{\hyperref[i-figured]{\texttt{\ }{\texttt{\ i-figured\ }}\texttt{\ }}}{  i-figured  }}\label{i-figured}

Configurable equation numbering per section in Typst. There is also
figure numbering per section, see more examples in its
\href{https://github.com/RubixDev/typst-i-figured}{manual} .

\begin{verbatim}
#import "@preview/i-figured:0.2.3"

// make sure you have some heading numbering set
#set heading(numbering: "1.1")

// apply the show rules (these can be customized)
#show heading: i-figured.reset-counters
#show math.equation: i-figured.show-equation.with(
  level: 1,
  zero-fill: true,
  leading-zero: true,
  numbering: "(1.1)",
  prefix: "eqt:",
  only-labeled: false,  // numbering all block equations implicitly
  unnumbered-label: "-",
)


= Introduction

You can write inline equations such as $x + y$, and numbered block equations like:

$ phi.alt := (1 + sqrt(5)) / 2 $ <ratio>

To reference a math equation, please use the `eqt:` prefix. For example, with @eqt:ratio, we have:

$ F_n = floor(1 / sqrt(5) phi.alt^n) $


= Appdendix

Additionally, you can use the <-> tag to indicate that a block formula should not be numbered:

$ y = integral_1^2 x^2 dif x $ <->

Subsequent math equations will continue to be numbered as usual:

$ F_n = floor(1 / sqrt(5) phi.alt^n) $
\end{verbatim}

\pandocbounded{\includesvg[keepaspectratio]{typst-img/b338b679a09371841be9322ac7cee901b6a1415582c3495677833602e344cae0-1.svg}}

\subsection{\texorpdfstring{\hyperref[theorems]{Theorems}}{Theorems}}\label{theorems}

\subsubsection{\texorpdfstring{\hyperref[ctheorem]{\texttt{\ }{\texttt{\ ctheorem\ }}\texttt{\ }}}{  ctheorem  }}\label{ctheorem}

A numbered theorem environment in Typst. See more examples in its
\href{https://github.com/sahasatvik/typst-theorems/blob/main/manual.pdf}{manual}
.

\begin{verbatim}
#import "@preview/ctheorems:1.1.0": *
#show: thmrules

#set page(width: 16cm, height: auto, margin: 1.5cm)
#set heading(numbering: "1.1")

#let theorem = thmbox("theorem", "Theorem", fill: rgb("#eeffee"))
#let corollary = thmplain("corollary", "Corollary", base: "theorem", titlefmt: strong)
#let definition = thmbox("definition", "Definition", inset: (x: 1.2em, top: 1em))

#let example = thmplain("example", "Example").with(numbering: none)
#let proof = thmplain(
  "proof", "Proof", base: "theorem",
  bodyfmt: body => [#body #h(1fr) $square$]
).with(numbering: none)

= Prime Numbers
#lorem(7)
#definition[ A natural number is called a #highlight[_prime number_] if ... ]
#example[
  The numbers $2$, $3$, and $17$ are prime. See @cor_largest_prime shows that
  this list is not exhaustive!
]
#theorem("Euclid")[There are infinitely many primes.]
#proof[
  Suppose to the contrary that $p_1, p_2, dots, p_n$ is a finite enumeration
  of all primes. ... a contradiction.
]
#corollary[
  There is no largest prime number.
] <cor_largest_prime>
#corollary[There are infinitely many composite numbers.]
\end{verbatim}

\pandocbounded{\includesvg[keepaspectratio]{typst-img/54d7817ddc4a8481da09052aa51dc6e4dde19bd85f40173b92750c402b07ff73-1.svg}}

\subsubsection{\texorpdfstring{\hyperref[lemmify]{\texttt{\ }{\texttt{\ lemmify\ }}\texttt{\ }}}{  lemmify  }}\label{lemmify}

Lemmify is another theorem evironment generator with many selector and
numbering capabilities. See documentations in its
\href{https://github.com/Marmare314/lemmify}{readme} .

\begin{verbatim}
#import "@preview/lemmify:0.1.5": *

#let my-thm-style(
  thm-type, name, number, body
) = grid(
  columns: (1fr, 3fr),
  column-gutter: 1em,
  stack(spacing: .5em, [#strong(thm-type) #number], emph(name)),
  body
)
#let my-styling = ( thm-styling: my-thm-style )
#let (
  definition, theorem, proof, lemma, rules
) = default-theorems("thm-group", lang: "en", ..my-styling)
#show: rules
#show thm-selector("thm-group"): box.with(inset: 0.8em)
#show thm-selector("thm-group", subgroup: "theorem"): it => box(
  it, fill: rgb("#eeffee"))

#set heading(numbering: "1.1")

= Prime numbers
#lorem(7) @proof and @thm[theorem]
#definition[ A natural number is called a #highlight[_prime number_] if ... ]
#theorem(name: "Theorem name")[There are infinitely many primes.]<thm>
#proof[
  Suppose to the contrary that $p_1, p_2, dots, p_n$ is a finite enumeration
  of all primes. ... #highlight[_a contradiction_].]<proof>
#lemma[There are infinitely many composite numbers.]
\end{verbatim}

\pandocbounded{\includesvg[keepaspectratio]{typst-img/46b0a27243980ee99b20133dbba1f00d4d819adff6e645ca0749820f5caf3589-1.svg}}


\title{sitandr.github.io/typst-examples-book/book/packages/graphs}

\section{\texorpdfstring{\hyperref[graphs]{Graphs}}{Graphs}}\label{graphs}

\subsection{\texorpdfstring{\hyperref[cetz]{\texttt{\ }{\texttt{\ cetz\ }}\texttt{\ }}}{  cetz  }}\label{cetz}

Cetz comes with quite built-in support of drawing basic graphs. It is
much more customizable and extensible then packages like
\texttt{\ }{\texttt{\ plotst\ }}\texttt{\ } , so it is recommended to
skim through its possibilities.

\begin{quote}
See full manual
\href{https://github.com/johannes-wolf/cetz/blob/master/manual.pdf?raw=true}{there}
.
\end{quote}

\begin{verbatim}
#let data = (
  [A], ([B], [C], [D]), ([E], [F])
)

#import "@preview/cetz:0.1.2": canvas, draw, tree

#canvas(length: 1cm, {
  import draw: *

  set-style(content: (padding: .2),
    fill: gray.lighten(70%),
    stroke: gray.lighten(70%))

  tree.tree(data, spread: 2.5, grow: 1.5, draw-node: (node, _) => {
    circle((), radius: .45, stroke: none)
    content((), node.content)
  }, draw-edge: (from, to, _) => {
    line((a: from, number: .6, abs: true, b: to),
         (a: to, number: .6, abs: true, b: from), mark: (end: ">"))
  }, name: "tree")

  // Draw a "custom" connection between two nodes
  let (a, b) = ("tree.0-0-1", "tree.0-1-0",)
  line((a: a, number: .6, abs: true, b: b), (a: b, number: .6, abs: true, b: a), mark: (end: ">", start: ">"))
})
\end{verbatim}

\pandocbounded{\includesvg[keepaspectratio]{typst-img/18fc5bbebb79c44df6fd484d2cc0c763b6a64e4a6455535738e40932f5fa39b4-1.svg}}

\begin{verbatim}
#import "@preview/cetz:0.1.2": canvas, draw

#canvas({
    import draw: *
    circle((90deg, 3), radius: 0, name: "content")
    circle((210deg, 3), radius: 0, name: "structure")
    circle((-30deg, 3), radius: 0, name: "form")
    for (c, a) in (
    ("content", "bottom"),
    ("structure", "top-right"),
    ("form", "top-left")
    ) {
    content(c, box(c + " oriented", inset: 5pt), anchor:
    a)
    }
    stroke(gray + 1.2pt)
    line("content", "structure", "form", close: true)
    for (c, s, f, cont) in (
    (0.5, 0.1, 1, "PostScript"),
    (1, 0, 0.4, "DVI"),
    (0.5, 0.5, 1, "PDF"),
    (0, 0.25, 1, "CSS"),
    (0.5, 1, 0, "XML"),
    (0.5, 1, 0.4, "HTML"),
    (1, 0.2, 0.8, "LaTeX"),
    (1, 0.6, 0.8, "TeX"),
    (0.8, 0.8, 1, "Word"),
    (1, 0.05, 0.05, "ASCII")
    ) {
    content((bary: (content: c, structure: s, form:
    f)),cont)
    }
})
\end{verbatim}

\pandocbounded{\includesvg[keepaspectratio]{typst-img/e93f89ca321c612b1157fd81cea439ade85d17485d0111a08b94e54e59e356db-1.svg}}

\begin{verbatim}
#import "@preview/cetz:0.1.2": canvas, chart

#let data2 = (
  ([15-24], 18.0, 20.1, 23.0, 17.0),
  ([25-29], 16.3, 17.6, 19.4, 15.3),
  ([30-34], 14.0, 15.3, 13.9, 18.7),
  ([35-44], 35.5, 26.5, 29.4, 25.8),
  ([45-54], 25.0, 20.6, 22.4, 22.0),
  ([55+],   19.9, 18.2, 19.2, 16.4),
)

#canvas({
  chart.barchart(mode: "clustered",
                 size: (9, auto),
                 label-key: 0,
                 value-key: (..range(1, 5)),
                 bar-width: .8,
                 x-tick-step: 2.5,
                 data2)
})
\end{verbatim}

\pandocbounded{\includesvg[keepaspectratio]{typst-img/3d162509c91794a0814503ed304bea48b221b2f58559c9d832c3254580cd0d2b-1.svg}}

\subsubsection{\texorpdfstring{\hyperref[draw-a-graph-in-polar-coords]{Draw
a graph in polar
coords}}{Draw a graph in polar coords}}\label{draw-a-graph-in-polar-coords}

\begin{verbatim}
#import "@preview/cetz:0.1.2": canvas, plot

#figure(
canvas(length: 1cm, {
  plot.plot(size: (5, 5),
    x-tick-step: 5,
    y-tick-step: 5,
    x-max: 20,
    y-max: 20,
    x-min: -20,
    y-min: -20,
    x-grid: true,
    y-grid: true,
    {
      plot.add(
        domain: (0,2*calc.pi),
        samples: 100,
        t => (13*calc.cos(t)-5*calc.cos(2*t)-2*calc.cos(3*t)-calc.cos(4*t), 16*calc.sin(t)*calc.sin(t)*calc.sin(t))
        )
    })
}), caption: "Plot made with cetz",)
\end{verbatim}

\pandocbounded{\includesvg[keepaspectratio]{typst-img/d24c6270b5c074f9331b16cdde3b626129537c5b4760c17b4e447a7ef3f22388-1.svg}}

\subsection{\texorpdfstring{\hyperref[diagraph]{\texttt{\ }{\texttt{\ diagraph\ }}\texttt{\ }}}{  diagraph  }}\label{diagraph}

\subsubsection{\texorpdfstring{\hyperref[test]{Test}}{Test}}\label{test}

\begin{verbatim}
#import "@preview/diagraph:0.2.0": *
#let renderc(code) = render(code.text)

#renderc(
  ```
  digraph {
    rankdir=LR;
    f -> B
    B -> f
    C -> D
    D -> B
    E -> F
    f -> E
    B -> F
  }
  ```
)
\end{verbatim}

\pandocbounded{\includesvg[keepaspectratio]{typst-img/f47c3218e9b78fba4f38d6daeaff627ee6b210bda8dd26fcbc56f14a7bb984ee-1.svg}}

\subsubsection{\texorpdfstring{\hyperref[eating]{Eating}}{Eating}}\label{eating}

\begin{verbatim}
#import "@preview/diagraph:0.2.0": *
#let renderc(code) = render(code.text)

#renderc(
  ```
  digraph {
    orange -> fruit
    apple -> fruit
    fruit -> food
    carrot -> vegetable
    vegetable -> food
    food -> eat
    eat -> survive
  }
  ```
)
\end{verbatim}

\pandocbounded{\includesvg[keepaspectratio]{typst-img/0a7fcbfb15be7bac447381d10af9488a7353071c92d849d1e4b7800a360c7659-1.svg}}

\subsubsection{\texorpdfstring{\hyperref[fft]{FFT}}{FFT}}\label{fft}

Labels are overridden manually.

\begin{verbatim}
#import "@preview/diagraph:0.2.0": *
#let renderc(code) = render(code.text)

#renderc(
  ```
  digraph {
    node [shape=none]
    1
    2
    3
    r1
    r2
    r3
    1->2
    1->3
    2->r1 [color=red]
    3->r2 [color=red]
    r1->r3 [color=red]
    r2->r3 [color=red]
  }
  ```
)
\end{verbatim}

\pandocbounded{\includesvg[keepaspectratio]{typst-img/5d7074ff82c6786fa2fad07b25ff4c238dbb9333b0a806d3ea74474fbf8d005e-1.svg}}

\subsubsection{\texorpdfstring{\hyperref[state-machine]{State
Machine}}{State Machine}}\label{state-machine}

\begin{verbatim}
#import "@preview/diagraph:0.2.0": *
#set page(width: auto)
#let renderc(code) = render(code.text)

#renderc(
  ```
  digraph finite_state_machine {
    rankdir=LR
    size="8,5"

    node [shape=doublecircle]
    LR_0
    LR_3
    LR_4
    LR_8

    node [shape=circle]
    LR_0 -> LR_2 [label="SS(B)"]
    LR_0 -> LR_1 [label="SS(S)"]
    LR_1 -> LR_3 [label="S($end)"]
    LR_2 -> LR_6 [label="SS(b)"]
    LR_2 -> LR_5 [label="SS(a)"]
    LR_2 -> LR_4 [label="S(A)"]
    LR_5 -> LR_7 [label="S(b)"]
    LR_5 -> LR_5 [label="S(a)"]
    LR_6 -> LR_6 [label="S(b)"]
    LR_6 -> LR_5 [label="S(a)"]
    LR_7 -> LR_8 [label="S(b)"]
    LR_7 -> LR_5 [label="S(a)"]
    LR_8 -> LR_6 [label="S(b)"]
    LR_8 -> LR_5 [label="S(a)"]
  }
  ```
)
\end{verbatim}

\pandocbounded{\includesvg[keepaspectratio]{typst-img/ce09c93e743aceb45852a12c83839cafd73a5c68d370ff2f863c79ec5896ff10-1.svg}}

\subsubsection{\texorpdfstring{\hyperref[clustering]{Clustering}}{Clustering}}\label{clustering}

\begin{quote}
See \href{http://www.graphviz.org/content/cluster}{docs} .
\end{quote}

\begin{verbatim}
#import "@preview/diagraph:0.2.0": *
#let renderc(code) = render(code.text)

#renderc(
  ```
  digraph G {

    subgraph cluster_0 {
      style=filled;
      color=lightgrey;
      node [style=filled,color=white];
      a0 -> a1 -> a2 -> a3;
      label = "process #1";
    }

    subgraph cluster_1 {
      node [style=filled];
      b0 -> b1 -> b2 -> b3;
      label = "process #2";
      color=blue
    }

    start -> a0;
    start -> b0;
    a1 -> b3;
    b2 -> a3;
    a3 -> a0;
    a3 -> end;
    b3 -> end;

    start [shape=Mdiamond];
    end [shape=Msquare];
  }
  ```
)
\end{verbatim}

\pandocbounded{\includesvg[keepaspectratio]{typst-img/5b51a47ca589de6fdd481db4b61f96395ef246f12a54d77d6d9c443c3cd2fc72-1.svg}}

\subsubsection{\texorpdfstring{\hyperref[html]{HTML}}{HTML}}\label{html}

\begin{verbatim}
#import "@preview/diagraph:0.2.0": *
#let renderc(code) = render(code.text)

#renderc(
  ```
  digraph structs {
      node [shape=plaintext]
      struct1 [label=<
  <TABLE BORDER="0" CELLBORDER="1" CELLSPACING="0">
    <TR><TD>left</TD><TD PORT="f1">mid dle</TD><TD PORT="f2">right</TD></TR>
  </TABLE>>];
      struct2 [label=<
  <TABLE BORDER="0" CELLBORDER="1" CELLSPACING="0">
    <TR><TD PORT="f0">one</TD><TD>two</TD></TR>
  </TABLE>>];
      struct3 [label=<
  <TABLE BORDER="0" CELLBORDER="1" CELLSPACING="0" CELLPADDING="4">
    <TR>
      <TD ROWSPAN="3">hello<BR/>world</TD>
      <TD COLSPAN="3">b</TD>
      <TD ROWSPAN="3">g</TD>
      <TD ROWSPAN="3">h</TD>
    </TR>
    <TR>
      <TD>c</TD><TD PORT="here">d</TD><TD>e</TD>
    </TR>
    <TR>
      <TD COLSPAN="3">f</TD>
    </TR>
  </TABLE>>];
      struct1:f1 -> struct2:f0;
      struct1:f2 -> struct3:here;
  }
  ```
)
\end{verbatim}

\pandocbounded{\includesvg[keepaspectratio]{typst-img/104d9f0e05417c58dce29ff55b47019eadd8538eed11bf552b03c9803fb8ce5b-1.svg}}

\subsubsection{\texorpdfstring{\hyperref[overridden-labels]{Overridden
labels}}{Overridden labels}}\label{overridden-labels}

Labels for nodes \texttt{\ }{\texttt{\ big\ }}\texttt{\ } and
\texttt{\ }{\texttt{\ sum\ }}\texttt{\ } are overridden.

\begin{verbatim}
#import "@preview/diagraph:0.2.0": *
#set page(width: auto)

#raw-render(
  ```
  digraph {
    rankdir=LR
    node[shape=circle]
    Hmm -> a_0
    Hmm -> big
    a_0 -> "a'" -> big [style="dashed"]
    big -> sum
  }
  ```,
  labels: (:
    big: [_some_#text(2em)[ big ]*text*],
    sum: $ sum_(i=0)^n 1/i $,
  ),
)
\end{verbatim}

\pandocbounded{\includesvg[keepaspectratio]{typst-img/a89c13a3c9aad0509c224ede97b8f1ed14c899049f92e6f23a2effc0bd56de40-1.svg}}

\subsection{\texorpdfstring{\hyperref[bob-draw]{\texttt{\ }{\texttt{\ bob-draw\ }}\texttt{\ }}}{  bob-draw  }}\label{bob-draw}

WASM plugin for \href{https://github.com/ivanceras/svgbob}{svgbob} to
draw easily with ASCII,.

\begin{verbatim}
#import "@preview/bob-draw:0.1.0": *
#render(```
         /\_/\
bob ->  ( o.o )
         \ " /
  .------/  /
 (        | |
  `====== o o
```)
\end{verbatim}

\pandocbounded{\includesvg[keepaspectratio]{typst-img/6f2c3c039f98a852450fad73ef9ee68d6e4ddcef39edc2376903cf0aa72606a2-1.svg}}

\begin{verbatim}
#import "@preview/bob-draw:0.1.0": *
#show raw.where(lang: "bob"): it => render(it)

#render(
    ```
      0       3  
       *-------* 
    1 /|    2 /| 
     *-+-----* | 
     | |4    | |7
     | *-----|-*
     |/      |/
     *-------*
    5       6
    ```,
    width: 25%,
)

```bob
"cats:"
 /\_/\  /\_/\  /\_/\  /\_/\ 
( o.o )( o.o )( o.o )( o.o )
```

```bob
       +10-15V           ___0,047R
      *---------o-----o-|___|-o--o---------o----o-------.
    + |         |     |       |  |         |    |       |
    -===-      _|_    |       | .+.        |    |       |
    -===-      .-.    |       | | | 2k2    |    |       |
    -===-    470| +   |       | | |        |    |      _|_
    - |       uF|     '--.    | '+'       .+.   |      \ / LED
      +---------o        |6   |7 |8    1k | |   |      -+-
             ___|___   .-+----+--+--.     | |   |       |
              -═══-    |            |     '+'   |       |
                -      |            |1     |  |/  BC    |
               GND     |            +------o--+   547   |
                       |            |      |  |`>       |
                       |            |     ,+.   |       |
               .-------+            | 220R| |   o----||-+  IRF9Z34
               |       |            |     | |   |    |+->
               |       |  MC34063   |     `+'   |    ||-+
               |       |            |      |    |       |  BYV29     -12V6
               |       |            |      '----'       o--|<-o----o--X OUT
 6000 micro  - | +     |            |2                  |     |    |
 Farad, 40V ___|_____  |            |--o                C|    |    |
 Capacitor  ~ ~ ~ ~ ~  |            | GND         30uH  C|    |   --- 470
               |       |            |3      1nF         C|    |   ###  uF
               |       |            |-------||--.       |     |    | +
               |       '-----+----+-'           |      GND    |   GND
               |            5|   4|             |             |
               |             |    '-------------o-------------o
               |             |                           ___  |
               `-------------*------/\/\/------------o--|___|-'
                                     2k              |       1k0
                                                    .+.
                                                    | | 5k6 + 3k3
                                                    | | in Serie
                                                    '+'
                                                     |
                                                    GND
```
\end{verbatim}

\pandocbounded{\includesvg[keepaspectratio]{typst-img/850abc33fa97f8b80bbda399475b0e4436d275c03f1ca369187eea9e72948b01-1.svg}}

\subsection{\texorpdfstring{\hyperref[wavy]{\texttt{\ }{\texttt{\ wavy\ }}\texttt{\ }}}{  wavy  }}\label{wavy}

\subsection{\texorpdfstring{\hyperref[finite]{\texttt{\ }{\texttt{\ finite\ }}\texttt{\ }}}{  finite  }}\label{finite}

Finite automata. See the
\href{https://github.com/jneug/typst-finite/blob/main/manual.pdf}{manual}
for a full documentation.

\begin{verbatim}
#import "@preview/finite:0.3.0": automaton

#automaton((
  q0: (q1:0, q0:"0,1"),
  q1: (q0:(0,1), q2:"0"),
  q2: (),
))
\end{verbatim}

\pandocbounded{\includesvg[keepaspectratio]{typst-img/9eddd9b18a2df43372188dab692be9e2973fac63f3764683c431a2c0fb8ba873-1.svg}}


\title{sitandr.github.io/typst-examples-book/book/packages/misc}

\section{\texorpdfstring{\hyperref[misc]{Misc}}{Misc}}\label{misc}

\section{\texorpdfstring{\hyperref[formatting-strings]{Formatting
strings}}{Formatting strings}}\label{formatting-strings}

\subsection{\texorpdfstring{\hyperref[oxifmt-general-purpose-string-formatter]{\texttt{\ }{\texttt{\ oxifmt\ }}\texttt{\ }
, general purpose string
formatter}}{  oxifmt   , general purpose string formatter}}\label{oxifmt-general-purpose-string-formatter}

\begin{verbatim}
#import "@preview/oxifmt:0.2.0": strfmt
#strfmt("I'm {}. I have {num} cars. I'm {0}. {} is {{cool}}.", "John", "Carl", num: 10) \
#strfmt("{0:?}, {test:+012e}, {1:-<#8x}", "hi", -74, test: 569.4) \
#strfmt("{:_>+11.5}", 59.4) \
#strfmt("Dict: {:!<10?}", (a: 5))
\end{verbatim}

\pandocbounded{\includesvg[keepaspectratio]{typst-img/f4f305da3efacf420f5d2a5159a57cca479ebbfd9b7412246d483de520135087-1.svg}}

\begin{verbatim}
#import "@preview/oxifmt:0.2.0": strfmt
#strfmt("First: {}, Second: {}, Fourth: {3}, Banana: {banana} (brackets: {{escaped}})", 1, 2.1, 3, label("four"), banana: "Banana!!")\
#strfmt("The value is: {:?} | Also the label is {:?}", "something", label("label"))\
#strfmt("Values: {:?}, {1:?}, {stuff:?}", (test: 500), ("a", 5.1), stuff: [a])\
#strfmt("Left5 {:_<5}, Right6 {:*>6}, Center10 {centered: ^10?}, Left3 {tleft:_<3}", "xx", 539, tleft: "okay", centered: [a])\
\end{verbatim}

\pandocbounded{\includesvg[keepaspectratio]{typst-img/39d725a28a184c450c74f3f895d1d59d26271b86acbddd454da564df76b668c8-1.svg}}

\begin{verbatim}
#import "@preview/oxifmt:0.2.0": strfmt
#repr(strfmt("Left-padded7 numbers: {:07} {:07} {:07} {3:07}", 123, -344, 44224059, 45.32))\
#strfmt("Some numbers: {:+} {:+08}; With fill and align: {:_<+8}; Negative (no-op): {neg:+}", 123, 456, 4444, neg: -435)\
#strfmt("Bases (10, 2, 8, 16(l), 16(U):) {0} {0:b} {0:o} {0:x} {0:X} | W/ prefixes and modifiers: {0:#b} {0:+#09o} {0:_>+#9X}", 124)\
#strfmt("{0:.8} {0:.2$} {0:.potato$}", 1.234, 0, 2, potato: 5)\
#strfmt("{0:e} {0:E} {0:+.9e} | {1:e} | {2:.4E}", 124.2312, 50, -0.02)\
#strfmt("{0} {0:.6} {0:.5e}", 1.432, fmt-decimal-separator: ",")
\end{verbatim}

\pandocbounded{\includesvg[keepaspectratio]{typst-img/7b709cd252c147436c88822b60d49ede25a23040531eeac41fb2ba37ca46a2d8-1.svg}}

\subsection{\texorpdfstring{\hyperref[name-it-integer-to-text]{\texttt{\ }{\texttt{\ name-it\ }}\texttt{\ }
, integer to
text}}{  name-it   , integer to text}}\label{name-it-integer-to-text}

\begin{verbatim}
#import "@preview/name-it:0.1.0": name-it

- #name-it(2418345)
\end{verbatim}

\pandocbounded{\includesvg[keepaspectratio]{typst-img/825de955e9f7261cd08d725520caf813e797aa4891da32ed7b43bafbe8b19f28-1.svg}}

\subsection{\texorpdfstring{\hyperref[nth-nth-element]{\texttt{\ }{\texttt{\ nth\ }}\texttt{\ }
, Nth element}}{  nth   , Nth element}}\label{nth-nth-element}

\begin{verbatim}
#import "@preview/nth:0.2.0": nth
#nth(3), #nth(5), #nth(2421)
\end{verbatim}

\pandocbounded{\includesvg[keepaspectratio]{typst-img/f8389763af9ec32227285bdc25885f02b4ad74d6a5900852ccd0664989f1d3cb-1.svg}}


\title{sitandr.github.io/typst-examples-book/book/packages/tables}

\section{\texorpdfstring{\hyperref[tables]{Tables}}{Tables}}\label{tables}

\subsection{\texorpdfstring{\hyperref[tablex-general-purpose-tables-library]{Tablex:
general purpose tables
library}}{Tablex: general purpose tables library}}\label{tablex-general-purpose-tables-library}

\begin{verbatim}
#import "@preview/tablex:0.0.7": tablex, rowspanx, colspanx

#tablex(
  columns: 4,
  align: center + horizon,
  auto-vlines: false,

  // indicate the first two rows are the header
  // (in case we need to eventually
  // enable repeating the header across pages)
  header-rows: 2,

  // color the last column's cells
  // based on the written number
  map-cells: cell => {
    if cell.x == 3 and cell.y > 1 {
      cell.content = {
        let value = int(cell.content.text)
        let text-color = if value < 10 {
          red.lighten(30%)
        } else if value < 15 {
          yellow.darken(13%)
        } else {
          green
        }
        set text(text-color)
        strong(cell.content)
      }
    }
    cell
  },

  /* --- header --- */
  rowspanx(2)[*Username*], colspanx(2)[*Data*], (), rowspanx(2)[*Score*],
  (),                 [*Location*], [*Height*], (),
  /* -------------- */

  [John], [Second St.], [180 cm], [5],
  [Wally], [Third Av.], [160 cm], [10],
  [Jason], [Some St.], [150 cm], [15],
  [Robert], [123 Av.], [190 cm], [20],
  [Other], [Unknown St.], [170 cm], [25],
)
\end{verbatim}

\pandocbounded{\includesvg[keepaspectratio]{typst-img/9283c11489e3997fb302d12c4958a964543f3de172f3f8e21eb739f97ae78ae2-1.svg}}

\begin{verbatim}
#import "@preview/tablex:0.0.7": tablex, hlinex, vlinex, colspanx, rowspanx

#pagebreak()
#v(80%)

#tablex(
  columns: 4,
  align: center + horizon,
  auto-vlines: false,
  repeat-header: true,

  /* --- header --- */
  rowspanx(2)[*Names*], colspanx(2)[*Properties*], (), rowspanx(2)[*Creators*],
  (),                 [*Type*], [*Size*], (),
  /* -------------- */

  [Machine], [Steel], [5 $"cm"^3$], [John p& Kate],
  [Frog], [Animal], [6 $"cm"^3$], [Robert],
  [Frog], [Animal], [6 $"cm"^3$], [Robert],
  [Frog], [Animal], [6 $"cm"^3$], [Robert],
  [Frog], [Animal], [6 $"cm"^3$], [Robert],
  [Frog], [Animal], [6 $"cm"^3$], [Robert],
  [Frog], [Animal], [6 $"cm"^3$], [Robert],
  [Frog], [Animal], [6 $"cm"^3$], [Rodbert],
)
\end{verbatim}

\pandocbounded{\includesvg[keepaspectratio]{typst-img/03fd8d593886849d39370d731f423691b255e47da0a391649235f3f746c25e5c-1.svg}}

\pandocbounded{\includesvg[keepaspectratio]{typst-img/03fd8d593886849d39370d731f423691b255e47da0a391649235f3f746c25e5c-2.svg}}

\begin{verbatim}
#import "@preview/tablex:0.0.7": tablex, gridx, hlinex, vlinex, colspanx, rowspanx

#tablex(
  columns: 4,
  auto-lines: false,

  // skip a column here         vv
  vlinex(), vlinex(), vlinex(), (), vlinex(),
  colspanx(2)[a], (),  [b], [J],
  [c], rowspanx(2)[d], [e], [K],
  [f], (),             [g], [L],
  //   ^^ '()' after the first cell are 100% ignored
)

#tablex(
  columns: 4,
  auto-vlines: false,
  colspanx(2)[a], (),  [b], [J],
  [c], rowspanx(2)[d], [e], [K],
  [f], (),             [g], [L],
)

#gridx(
  columns: 4,
  (), (), vlinex(end: 2),
  hlinex(stroke: yellow + 2pt),
  colspanx(2)[a], (),  [b], [J],
  hlinex(start: 0, end: 1, stroke: yellow + 2pt),
  hlinex(start: 1, end: 2, stroke: green + 2pt),
  hlinex(start: 2, end: 3, stroke: red + 2pt),
  hlinex(start: 3, end: 4, stroke: blue.lighten(50%) + 2pt),
  [c], rowspanx(2)[d], [e], [K],
  hlinex(start: 2),
  [f], (),             [g], [L],
)
\end{verbatim}

\pandocbounded{\includesvg[keepaspectratio]{typst-img/4d25fc4ba39ee99bf9b8c043ab89bc74cf61cad3f4640b3384dad2e69f5f64c8-1.svg}}

\begin{verbatim}
#import "@preview/tablex:0.0.7": tablex, colspanx, rowspanx

#tablex(
  columns: 3,
  map-hlines: h => (..h, stroke: blue),
  map-vlines: v => (..v, stroke: green + 2pt),
  colspanx(2)[a], (),  [b],
  [c], rowspanx(2)[d], [ed],
  [f], (),             [g]
)
\end{verbatim}

\pandocbounded{\includesvg[keepaspectratio]{typst-img/9f721aa89d44247b880a2d34d64940cce12a782d4888a09b6a031a2918805128-1.svg}}

\begin{verbatim}
#import "@preview/tablex:0.0.7": tablex, colspanx, rowspanx

#tablex(
  columns: 4,
  auto-vlines: true,

  // make all cells italicized
  map-cells: cell => {
    (..cell, content: emph(cell.content))
  },

  // add some arbitrary content to entire rows
  map-rows: (row, cells) => cells.map(c =>
    if c == none {
      c  // keeping 'none' is important
    } else {
      (..c, content: [#c.content\ *R#row*])
    }
  ),

  // color cells based on their columns
  // (using 'fill: (column, row) => color' also works
  // for this particular purpose)
  map-cols: (col, cells) => cells.map(c =>
    if c == none {
      c
    } else {
      (..c, fill: if col < 2 { blue } else { yellow })
    }
  ),

  colspanx(2)[a], (),  [b], [J],
  [c], rowspanx(2)[dd], [e], [K],
  [f], (),             [g], [L],
)
\end{verbatim}

\pandocbounded{\includesvg[keepaspectratio]{typst-img/e4aeb7879544c21da12283922f4e3110d740059da77b65d94e34ed39229ffad1-1.svg}}

\begin{verbatim}
#import "@preview/tablex:0.0.7": gridx

#gridx(
  columns: 3,
  rows: 6,
  fill: (col, row) => (blue, red, green).at(calc.rem(row + col - 1, 3)),
  map-cols: (col, cells) => {
    let last = cells.last()
    last.content = [
      #cells.slice(0, cells.len() - 1).fold(0, (acc, c) => if c != none { acc + eval(c.content.text) } else { acc })
    ]
    last.fill = aqua
    cells.last() = last
    cells
  },
  [0], [5], [10],
  [1], [6], [11],
  [2], [7], [12],
  [3], [8], [13],
  [4], [9], [14],
  [s], [s], [s]
)
\end{verbatim}

\pandocbounded{\includesvg[keepaspectratio]{typst-img/c67cc9e428f9b21ae2e9c4ba792eacc7391fce70f06375f49d3a5f08234a5a77-1.svg}}

\subsection{\texorpdfstring{\hyperref[tada-data-manipulation]{Tada: data
manipulation}}{Tada: data manipulation}}\label{tada-data-manipulation}

\begin{verbatim}
#import "@preview/tada:0.1.0"

#let column-data = (
  name: ("Bread", "Milk", "Eggs"),
  price: (1.25, 2.50, 1.50),
  quantity: (2, 1, 3),
)
#let record-data = (
  (name: "Bread", price: 1.25, quantity: 2),
  (name: "Milk", price: 2.50, quantity: 1),
  (name: "Eggs", price: 1.50, quantity: 3),
)
#let row-data = (
  ("Bread", 1.25, 2),
  ("Milk", 2.50, 1),
  ("Eggs", 1.50, 3),
)

#import tada: TableData, to-tablex
#let td = TableData(data: column-data)
// Equivalent to:
#let td2 = tada.from-records(record-data)
// _Not_ equivalent to (since field names are unknown):
#let td3 = tada.from-rows(row-data)

#to-tablex(td)
#to-tablex(td2)
#to-tablex(td3)
\end{verbatim}

\pandocbounded{\includesvg[keepaspectratio]{typst-img/06c7045a0bb3aad12c70133b4aa55b1cadc17c944d28803e9418a376187afb2d-1.svg}}

\subsection{\texorpdfstring{\hyperref[tablem-markdown-tables]{Tablem:
markdown
tables}}{Tablem: markdown tables}}\label{tablem-markdown-tables}

\begin{quote}
See documentation \href{https://github.com/OrangeX4/typst-tablem}{there}
\end{quote}

Render markdown tables in Typst.

\begin{verbatim}
#import "@preview/tablem:0.1.0": tablem

#tablem[
  | *Name* | *Location* | *Height* | *Score* |
  | ------ | ---------- | -------- | ------- |
  | John   | Second St. | 180 cm   |  5      |
  | Wally  | Third Av.  | 160 cm   |  10     |
]
\end{verbatim}

\pandocbounded{\includesvg[keepaspectratio]{typst-img/6845ef64c7c12ce5f6616f130172c76974b184e97976e59a3a957c273c9084eb-1.svg}}

\subsubsection{\texorpdfstring{\hyperref[custom-render]{Custom
render}}{Custom render}}\label{custom-render}

\begin{verbatim}
#import "@preview/tablex:0.0.6": tablex, hlinex
#import "@preview/tablem:0.1.0": tablem

#let three-line-table = tablem.with(
  render: (columns: auto, ..args) => {
    tablex(
      columns: columns,
      auto-lines: false,
      align: center + horizon,
      hlinex(y: 0),
      hlinex(y: 1),
      ..args,
      hlinex(),
    )
  }
)

#three-line-table[
  | *Name* | *Location* | *Height* | *Score* |
  | ------ | ---------- | -------- | ------- |
  | John   | Second St. | 180 cm   |  5      |
  | Wally  | Third Av.  | 160 cm   |  10     |
]
\end{verbatim}

\pandocbounded{\includesvg[keepaspectratio]{typst-img/ebddbdf17a6518755d55af3900eabe9ffb8fa2c0d8b0326518dac03ca1856648-1.svg}}


\title{sitandr.github.io/typst-examples-book/book/packages/headers}

\section{\texorpdfstring{\hyperref[headers]{Headers}}{Headers}}\label{headers}

\subsection{\texorpdfstring{\hyperref[hydra-contextual-headers]{\texttt{\ }{\texttt{\ hydra\ }}\texttt{\ }
: Contextual
headers}}{  hydra   : Contextual headers}}\label{hydra-contextual-headers}

We have discussed in \texttt{\ }{\texttt{\ Typst\ Basics\ }}\texttt{\ }
how to get current heading with
\texttt{\ }{\texttt{\ query(selector(heading).before(here()))\ }}\texttt{\ }
for headers. However, this works badly for nested headings with
numbering and similar things. For these cases there is
\texttt{\ }{\texttt{\ hydra\ }}\texttt{\ } :

\begin{verbatim}
#import "@preview/hydra:0.5.1": hydra

#set page(height: 10 * 20pt, margin: (y: 4em), numbering: "1", header: context {
  if calc.odd(here().page()) {
    align(right, emph(hydra(1)))
  } else {
    align(left, emph(hydra(2)))
  }
  line(length: 100%)
})
#set heading(numbering: "1.1")
#show heading.where(level: 1): it => pagebreak(weak: true) + it

= Introduction
#lorem(50)

= Content
== First Section
#lorem(50)
== Second Section
#lorem(100)
\end{verbatim}

\pandocbounded{\includesvg[keepaspectratio]{typst-img/1a1e2d4655c80e3b0cd9cd7db25c191054aac7ff69aa9cf7cda6935041b614ae-1.svg}}

\pandocbounded{\includesvg[keepaspectratio]{typst-img/1a1e2d4655c80e3b0cd9cd7db25c191054aac7ff69aa9cf7cda6935041b614ae-2.svg}}

\pandocbounded{\includesvg[keepaspectratio]{typst-img/1a1e2d4655c80e3b0cd9cd7db25c191054aac7ff69aa9cf7cda6935041b614ae-3.svg}}

\pandocbounded{\includesvg[keepaspectratio]{typst-img/1a1e2d4655c80e3b0cd9cd7db25c191054aac7ff69aa9cf7cda6935041b614ae-4.svg}}


\title{sitandr.github.io/typst-examples-book/book/packages/layout}

\section{\texorpdfstring{\hyperref[layouting]{Layouting}}{Layouting}}\label{layouting}

General useful things.

\subsection{\texorpdfstring{\hyperref[pinit-relative-place-by-pins]{Pinit:
relative place by
pins}}{Pinit: relative place by pins}}\label{pinit-relative-place-by-pins}

The idea of \href{https://github.com/OrangeX4/typst-pinit}{pinit} is
pinning pins on the normal flow of the text, and then placing the
content relative to pins.

\begin{verbatim}
#import "@preview/pinit:0.1.3": *
#set page(height: 6em, width: 20em)

#set text(size: 24pt)

A simple #pin(1)highlighted text#pin(2).

#pinit-highlight(1, 2)

#pinit-point-from(2)[It is simple.]
\end{verbatim}

\pandocbounded{\includesvg[keepaspectratio]{typst-img/b0a3a289ec65a00a9b39e0689578c9c139a65d1d9f379fa1593ba8ea9268af25-1.svg}}

More complex example:

\begin{verbatim}
#import "@preview/pinit:0.1.3": *

// Pages
#set page(paper: "presentation-4-3")
#set text(size: 20pt)
#show heading: set text(weight: "regular")
#show heading: set block(above: 1.4em, below: 1em)
#show heading.where(level: 1): set text(size: 1.5em)

// Useful functions
#let crimson = rgb("#c00000")
#let greybox(..args, body) = rect(fill: luma(95%), stroke: 0.5pt, inset: 0pt, outset: 10pt, ..args, body)
#let redbold(body) = {
  set text(fill: crimson, weight: "bold")
  body
}
#let blueit(body) = {
  set text(fill: blue)
  body
}

// Main body
#block[
  = Asymptotic Notation: $O$

  Use #pin("h1")asymptotic notations#pin("h2") to describe asymptotic efficiency of algorithms.
  (Ignore constant coefficients and lower-order terms.)

  #greybox[
    Given a function $g(n)$, we denote by $O(g(n))$ the following *set of functions*:
    #redbold(${f(n): "exists" c > 0 "and" n_0 > 0, "such that" f(n) <= c dot g(n) "for all" n >= n_0}$)
  ]

  #pinit-highlight("h1", "h2")

  $f(n) = O(g(n))$: #pin(1)$f(n)$ is *asymptotically smaller* than $g(n)$.#pin(2)

  $f(n) redbold(in) O(g(n))$: $f(n)$ is *asymptotically* #redbold[at most] $g(n)$.

  #pinit-line(stroke: 3pt + crimson, start-dy: -0.25em, end-dy: -0.25em, 1, 2)

  #block[Insertion Sort as an #pin("r1")example#pin("r2"):]

  - Best Case: $T(n) approx c n + c' n - c''$ #pin(3)
  - Worst case: $T(n) approx c n + (c' \/ 2) n^2 - c''$ #pin(4)

  #pinit-rect("r1", "r2")

  #pinit-place(3, dx: 15pt, dy: -15pt)[#redbold[$T(n) = O(n)$]]
  #pinit-place(4, dx: 15pt, dy: -15pt)[#redbold[$T(n) = O(n)$]]

  #blueit[Q: Is $n^(3) = O(n^2)$#pin("que")? How to prove your answer#pin("ans")?]

  #pinit-point-to("que", fill: crimson, redbold[No.])
  #pinit-point-from("ans", body-dx: -150pt)[
    Show that the equation $(3/2)^n >= c$ \
    has infinitely many solutions for $n$.
  ]
]
\end{verbatim}

\pandocbounded{\includesvg[keepaspectratio]{typst-img/4cc4ac1de81450b49f618408d35cd551858a4fcee317859f7f2a5d84482a9612-1.svg}}

\subsection{\texorpdfstring{\hyperref[margin-notes]{Margin
notes}}{Margin notes}}\label{margin-notes}

\begin{verbatim}
#import "@preview/drafting:0.1.1": *

#let (l-margin, r-margin) = (1in, 2in)
#set page(
  margin: (left: l-margin, right: r-margin, rest: 0.1in),
)
#set-page-properties(margin-left: l-margin, margin-right: r-margin)

= Margin Notes
== Setup
Unfortunately `typst` doesn't expose margins to calling functions, so you'll need to set them explicitly. This is done using `set-page-properties` *before you place any content*:

// At the top of your source file
// Of course, you can substitute any margin numbers you prefer
// provided the page margins match what you pass to `set-page-properties`

== The basics
#lorem(20)
#margin-note(side: left)[Hello, world!]
#lorem(10)
#margin-note[Hello from the other side]

#lorem(25)
#margin-note[When notes are about to overlap, they're automatically shifted]
#margin-note(stroke: aqua + 3pt)[To avoid collision]
#lorem(25)

#let caution-rect = rect.with(inset: 1em, radius: 0.5em, fill: orange.lighten(80%))
#inline-note(rect: caution-rect)[
  Be aware that notes will stop automatically avoiding collisions if 4 or more notes
  overlap. This is because `typst` warns when the layout doesn't resolve after 5 attempts
  (initial layout + adjustment for each note)
]
\end{verbatim}

\pandocbounded{\includesvg[keepaspectratio]{typst-img/80c65cf70b8da549afe447ce97f6dc71087cc0654dd85cd4f5e95bea388e3179-1.svg}}

\begin{verbatim}
#import "@preview/drafting:0.1.1": *

#let (l-margin, r-margin) = (1in, 2in)
#set page(
  margin: (left: l-margin, right: r-margin, rest: 0.1in),
)
#set-page-properties(margin-left: l-margin, margin-right: r-margin)

== Adjusting the default style
All function defaults are customizable through updating the module state:

#lorem(4) #margin-note(dy: -2em)[Default style]
#set-margin-note-defaults(stroke: orange, side: left)
#lorem(4) #margin-note[Updated style]


Even deeper customization is possible by overriding the default `rect`:

#import "@preview/colorful-boxes:1.1.0": stickybox

#let default-rect(stroke: none, fill: none, width: 0pt, content) = {
  stickybox(rotation: 30deg, width: width/1.5, content)
}
#set-margin-note-defaults(rect: default-rect, stroke: none, side: right)

#lorem(20)
#margin-note(dy: -25pt)[Why not use sticky notes in the margin?]

// Undo changes from last example
#set-margin-note-defaults(rect: rect, stroke: red)

== Multiple document reviewers
#let reviewer-a = margin-note.with(stroke: blue)
#let reviewer-b = margin-note.with(stroke: purple)
#lorem(20)
#reviewer-a[Comment from reviewer A]
#lorem(15)
#reviewer-b(side: left)[Comment from reviewer B]

== Inline Notes
#lorem(10)
#inline-note[The default inline note will split the paragraph at its location]
#lorem(10)
/*
// Should work, but doesn't? Created an issue in repo.
#inline-note(par-break: false, stroke: (paint: orange, dash: "dashed"))[
  But you can specify `par-break: false` to prevent this
]
*/
#lorem(10)
\end{verbatim}

\pandocbounded{\includesvg[keepaspectratio]{typst-img/282de769e728a8bdb9c78c665664b382ecbf59fd7d3c915fab67aae7055e2acb-1.svg}}

\begin{verbatim}
#import "@preview/drafting:0.1.1": *

#let (l-margin, r-margin) = (1in, 2in)
#set page(
  margin: (left: l-margin, right: r-margin, rest: 0.1in),
)
#set-page-properties(margin-left: l-margin, margin-right: r-margin)

== Hiding notes for print preview
#set-margin-note-defaults(hidden: true)

#lorem(20)
#margin-note[This will respect the global "hidden" state]
#margin-note(hidden: false, dy: -2.5em)[This note will never be hidden]

= Positioning
== Precise placement: rule grid
Need to measure space for fine-tuned positioning? You can use `rule-grid` to cross-hatch
the page with rule lines:

#rule-grid(width: 10cm, height: 3cm, spacing: 20pt)
#place(
  dx: 180pt,
  dy: 40pt,
  rect(fill: white, stroke: red, width: 1in, "This will originate at (180pt, 40pt)")
)

// Optionally specify divisions of the smallest dimension to automatically calculate
// spacing
#rule-grid(dx: 10cm + 3em, width: 3cm, height: 1.2cm, divisions: 5, square: true,  stroke: green)

// The rule grid doesn't take up space, so add it explicitly
#v(3cm + 1em)

== Absolute positioning
What about absolutely positioning something regardless of margin and relative location? `absolute-place` is your friend. You can put content anywhere:

#context {
  let (dx, dy) = (25%, here().position().y)
  let content-str = (
    "This absolutely-placed box will originate at (" + repr(dx) + ", " + repr(dy) + ")"
    + " in page coordinates"
  )
  absolute-place(
    dx: dx, dy: dy,
    rect(
      fill: green.lighten(60%),
      radius: 0.5em,
      width: 2.5in,
      height: 0.5in,
      [#align(center + horizon, content-str)]
    )
  )
}
#v(1in)

The "rule-grid" also supports absolute placement at the top-left of the page by passing `relative: false`. This is helpful for "rule"-ing the whole page.
\end{verbatim}

\pandocbounded{\includesvg[keepaspectratio]{typst-img/212dfc0f37bc9749e459085bb305f46a1db7ab3fbb22760f62ec58e349837d9e-1.svg}}

\subsection{\texorpdfstring{\hyperref[dropped-capitals]{Dropped
capitals}}{Dropped capitals}}\label{dropped-capitals}

\begin{quote}
Get more info
\href{https://github.com/EpicEricEE/typst-plugins/tree/master/droplet}{here}
\end{quote}

\subsubsection{\texorpdfstring{\hyperref[basic-usage]{Basic
usage}}{Basic usage}}\label{basic-usage}

\begin{verbatim}
#import "@preview/droplet:0.1.0": dropcap

#dropcap(gap: -2pt, hanging-indent: 8pt)[
  #lorem(42)
]
\end{verbatim}

\pandocbounded{\includesvg[keepaspectratio]{typst-img/a9c411d628d90aa8313aa9f0829bfdf43122c4532ad0d9d323a64b989a049d64-1.svg}}

\subsubsection{\texorpdfstring{\hyperref[extended-styling]{Extended
styling}}{Extended styling}}\label{extended-styling}

\begin{verbatim}
#import "@preview/droplet:0.1.0": dropcap

#dropcap(
  height: 2,
  justify: true,
  gap: 6pt,
  transform: letter => style(styles => {
    let height = measure(letter, styles).height

    grid(columns: 2, gutter: 6pt,
      align(center + horizon, text(blue, letter)),
      // Use "place" to ignore the line's height when
      // the font size is calculated later on.
      place(horizon, line(
        angle: 90deg,
        length: height + 6pt,
        stroke: blue.lighten(40%) + 1pt
      )),
    )
  })
)[
  #lorem(42)
]
\end{verbatim}

\pandocbounded{\includesvg[keepaspectratio]{typst-img/50d7ee4ffb1e61856535409373b040d579ab05734f3f304a4dc15f23361fd710-1.svg}}

\subsection{\texorpdfstring{\hyperref[headings-for-actual-current-chapter]{Headings
for actual current
chapter}}{Headings for actual current chapter}}\label{headings-for-actual-current-chapter}

\begin{quote}
See \href{https://github.com/tingerrr/hydra}{hydra}
\end{quote}

\begin{verbatim}
#import "@preview/hydra:0.2.0": hydra

#set page(header: hydra() + line(length: 100%))
#set heading(numbering: "1.1")
#show heading.where(level: 1): it => pagebreak(weak: true) + it

= Introduction
#lorem(750)

= Content
== First Section
#lorem(500)
== Second Section
#lorem(250)
== Third Section
#lorem(500)

= Annex
#lorem(10)
\end{verbatim}

\pandocbounded{\includesvg[keepaspectratio]{typst-img/ab3a07e72c5e19f28936f6d2249c9c5bcd102a27f4af177db63d7715c5c64f33-1.svg}}

\pandocbounded{\includesvg[keepaspectratio]{typst-img/ab3a07e72c5e19f28936f6d2249c9c5bcd102a27f4af177db63d7715c5c64f33-2.svg}}

\pandocbounded{\includesvg[keepaspectratio]{typst-img/ab3a07e72c5e19f28936f6d2249c9c5bcd102a27f4af177db63d7715c5c64f33-3.svg}}

\pandocbounded{\includesvg[keepaspectratio]{typst-img/ab3a07e72c5e19f28936f6d2249c9c5bcd102a27f4af177db63d7715c5c64f33-4.svg}}

\pandocbounded{\includesvg[keepaspectratio]{typst-img/ab3a07e72c5e19f28936f6d2249c9c5bcd102a27f4af177db63d7715c5c64f33-5.svg}}


\title{sitandr.github.io/typst-examples-book/book/packages/glossary}

\section{\texorpdfstring{\hyperref[glossary]{Glossary}}{Glossary}}\label{glossary}

\subsection{\texorpdfstring{\hyperref[glossarium]{glossarium}}{glossarium}}\label{glossarium}

\begin{quote}
\href{https://typst.app/universe/package/glossarium}{Link to the
universe}
\end{quote}

Package to manage glossary and abbreviations.

One of the very first cool packages of Typst, made specially for
(probably) the first thesis written in Typst.

\begin{verbatim}
#import "@preview/glossarium:0.4.1": make-glossary, print-glossary, gls, glspl
#show: make-glossary

// for better link visibility
#show link: set text(fill: blue.darken(60%))

#print-glossary(
    (
    // minimal term
    (key: "kuleuven", short: "KU Leuven"),

    // a term with a long form and a group
    (key: "unamur", short: "UNamur", long: "Namur University", group: "Universities"),

    // a term with a markup description
    (
      key: "oidc",
      short: "OIDC",
      long: "OpenID Connect",
      desc: [OpenID is an open standard and decentralized authentication protocol promoted by the non-profit
      #link("https://en.wikipedia.org/wiki/OpenID#OpenID_Foundation")[OpenID Foundation].],
      group: "Accronyms",
    ),

    // a term with a short plural
    (
      key: "potato",
      short: "potato",
      // "plural" will be used when "short" should be pluralized
      plural: "potatoes",
      desc: [#lorem(10)],
    ),

    // a term with a long plural
    (
      key: "dm",
      short: "DM",
      long: "diagonal matrix",
      // "longplural" will be used when "long" should be pluralized
      longplural: "diagonal matrices",
      desc: "Probably some math stuff idk",
    ),
  )
)

// referencing the OIDC term using gls
#gls("oidc")
// displaying the long form forcibly
#gls("oidc", long: true)

// referencing the OIDC term using the reference syntax
@oidc

Plural: #glspl("potato")

#gls("oidc", display: "whatever you want")
\end{verbatim}

\pandocbounded{\includesvg[keepaspectratio]{typst-img/c17c1be6563520252dfc59ccc646a6c48fb29e467d03f2892fdbfbddb496c3f9-1.svg}}


\title{sitandr.github.io/typst-examples-book/book/packages/boxes}

\section{\texorpdfstring{\hyperref[custom-boxes]{Custom
boxes}}{Custom boxes}}\label{custom-boxes}

\subsection{\texorpdfstring{\hyperref[showbox]{Showbox}}{Showbox}}\label{showbox}

\begin{verbatim}
#import "@preview/showybox:2.0.1": showybox

#showybox(
  [Hello world!]
)
\end{verbatim}

\pandocbounded{\includesvg[keepaspectratio]{typst-img/5b1a31dde61cee643fe9c8550a396d2cad3d27bcaf56412fb1b1a1a2563c462e-1.svg}}

\begin{verbatim}
#import "@preview/showybox:2.0.1": showybox

// First showybox
#showybox(
  frame: (
    border-color: red.darken(50%),
    title-color: red.lighten(60%),
    body-color: red.lighten(80%)
  ),
  title-style: (
    color: black,
    weight: "regular",
    align: center
  ),
  shadow: (
    offset: 3pt,
  ),
  title: "Red-ish showybox with separated sections!",
  lorem(20),
  lorem(12)
)

// Second showybox
#showybox(
  frame: (
    dash: "dashed",
    border-color: red.darken(40%)
  ),
  body-style: (
    align: center
  ),
  sep: (
    dash: "dashed"
  ),
  shadow: (
    offset: (x: 2pt, y: 3pt),
    color: yellow.lighten(70%)
  ),
  [This is an important message!],
  [Be careful outside. There are dangerous bananas!]
)
\end{verbatim}

\pandocbounded{\includesvg[keepaspectratio]{typst-img/71353a03ef746508398e53dc16ea676041d953dadb029a8e186fd9c317085510-1.svg}}

\begin{verbatim}
#import "@preview/showybox:2.0.1": showybox

#showybox(
  title: "Stokes' theorem",
  frame: (
    border-color: blue,
    title-color: blue.lighten(30%),
    body-color: blue.lighten(95%),
    footer-color: blue.lighten(80%)
  ),
  footer: "Information extracted from a well-known public encyclopedia"
)[
  Let $Sigma$ be a smooth oriented surface in $RR^3$ with boundary $diff Sigma equiv Gamma$. If a vector field $bold(F)(x,y,z)=(F_x (x,y,z), F_y (x,y,z), F_z (x,y,z))$ is defined and has continuous first order partial derivatives in a region containing $Sigma$, then

  $ integral.double_Sigma (bold(nabla) times bold(F)) dot bold(Sigma) = integral.cont_(diff Sigma) bold(F) dot dif bold(Gamma) $
]
\end{verbatim}

\pandocbounded{\includesvg[keepaspectratio]{typst-img/9e5c363090d9b928ee6c998876dd9e15a388ab6f6ae793f8a86ad688d2a62f2c-1.svg}}

\begin{verbatim}
#import "@preview/showybox:2.0.1": showybox

#showybox(
  title-style: (
    weight: 900,
    color: red.darken(40%),
    sep-thickness: 0pt,
    align: center
  ),
  frame: (
    title-color: red.lighten(80%),
    border-color: red.darken(40%),
    thickness: (left: 1pt),
    radius: 0pt
  ),
  title: "Carnot cycle's efficiency"
)[
  Inside a Carnot cycle, the efficiency $eta$ is defined to be:

  $ eta = W/Q_H = frac(Q_H + Q_C, Q_H) = 1 - T_C/T_H $
]
\end{verbatim}

\pandocbounded{\includesvg[keepaspectratio]{typst-img/3ce2b6bf5cd66f8aaa6c960c8f18902b63518eb4c6ee3f41337c1857e31128e9-1.svg}}

\begin{verbatim}
#import "@preview/showybox:2.0.1": showybox

#showybox(
  footer-style: (
    sep-thickness: 0pt,
    align: right,
    color: black
  ),
  title: "Divergence theorem",
  footer: [
    In the case of $n=3$, $V$ represents a volume in three-dimensional space, and $diff V = S$ its surface
  ]
)[
  Suppose $V$ is a subset of $RR^n$ which is compact and has a piecewise smooth boundary $S$ (also indicated with $diff V = S$). If $bold(F)$ is a continuously differentiable vector field defined on a neighborhood of $V$, then:

  $ integral.triple_V (bold(nabla) dot bold(F)) dif V = integral.surf_S (bold(F) dot bold(hat(n))) dif S $
]
\end{verbatim}

\pandocbounded{\includesvg[keepaspectratio]{typst-img/9abf5c05795f94a0b36b0e0fe84bb13ae210e6c234ad306606ed9bf52bd5e481-1.svg}}

\begin{verbatim}
#import "@preview/showybox:2.0.1": showybox

#showybox(
  frame: (
    border-color: red.darken(30%),
    title-color: red.darken(30%),
    radius: 0pt,
    thickness: 2pt,
    body-inset: 2em,
    dash: "densely-dash-dotted"
  ),
  title: "Gauss's Law"
)[
  The net electric flux through any hypothetical closed surface is equal to $1/epsilon_0$ times the net electric charge enclosed within that closed surface. The closed surface is also referred to as Gaussian surface. In its integral form:

  $ Phi_E = integral.surf_S bold(E) dot dif bold(A) = Q/epsilon_0 $
]
\end{verbatim}

\pandocbounded{\includesvg[keepaspectratio]{typst-img/9ae97a9b51a35a54fab7e017b1f500b5062b7e644928fa132a4cd1b218e8aad8-1.svg}}

\subsection{\texorpdfstring{\hyperref[colorful-boxes]{Colorful
boxes}}{Colorful boxes}}\label{colorful-boxes}

\begin{verbatim}
#import "@preview/colorful-boxes:1.2.0": colorbox, slantedColorbox, outlinebox, stickybox

#colorbox(
  title: lorem(5),
  color: "blue",
  radius: 2pt,
  width: auto
)[
  #lorem(50)
]

#slantedColorbox(
  title: lorem(5),
  color: "red",
  radius: 0pt,
  width: auto
)[
  #lorem(50)
]

#outlinebox(
  title: lorem(5),
  color: none,
  width: auto,
  radius: 2pt,
  centering: false
)[
  #lorem(50)
]

#outlinebox(
  title: lorem(5),
  color: "green",
  width: auto,
  radius: 2pt,
  centering: true
)[
  #lorem(50)
]

#stickybox(
  rotation: 3deg,
  width: 7cm
)[
  #lorem(20)
]
\end{verbatim}

\pandocbounded{\includesvg[keepaspectratio]{typst-img/a8efee5212da42450ccb46cedda2280b5e876e22cc08ab656a73d379754c8661-1.svg}}

\subsection{\texorpdfstring{\hyperref[theorems]{Theorems}}{Theorems}}\label{theorems}

See \href{./math.html}{math}


\title{sitandr.github.io/typst-examples-book/book/packages/code}

\section{\texorpdfstring{\hyperref[code]{Code}}{Code}}\label{code}

\subsection{\texorpdfstring{\hyperref[codly]{\texttt{\ }{\texttt{\ codly\ }}\texttt{\ }}}{  codly  }}\label{codly}

\begin{quote}
See docs \href{https://github.com/Dherse/codly}{there}
\end{quote}

\begin{verbatim}
#import "@preview/codly:0.1.0": codly-init, codly, disable-codly
#show: codly-init.with()

#codly(languages: (
        typst: (name: "Typst", color: rgb("#41A241"), icon: none),
    ),
    breakable: false
)

```typst
#import "@preview/codly:0.1.0": codly-init
#show: codly-init.with()
```

// Still formatted!
```rust
pub fn main() {
    println!("Hello, world!");
}
```

#disable-codly()
\end{verbatim}

\pandocbounded{\includesvg[keepaspectratio]{typst-img/eaa07afd21b4783a4be0a9726e714a8a4644421e5a93383e7deaeffaf4765105-1.svg}}

\subsection{\texorpdfstring{\hyperref[codelst]{Codelst}}{Codelst}}\label{codelst}

\begin{verbatim}
#import "@preview/codelst:2.0.0": sourcecode

#sourcecode[```typ
#show "ArtosFlow": name => box[
  #box(image(
    "logo.svg",
    height: 0.7em,
  ))
  #name
]

This report is embedded in the
ArtosFlow project. ArtosFlow is a
project of the Artos Institute.
```]
\end{verbatim}

\pandocbounded{\includesvg[keepaspectratio]{typst-img/2b2bbf130111979e4bc4cbc33171a39842467b3ea5e67a7fa0fcbf26222e8f90-1.svg}}


\title{sitandr.github.io/typst-examples-book/book/packages/external}

\section{\texorpdfstring{\hyperref[external]{External}}{External}}\label{external}

These are not official packages. Maybe once they will become one.

However, they may be very useful.

\subsection{\texorpdfstring{\hyperref[treemap-display]{Treemap
display}}{Treemap display}}\label{treemap-display}

\href{https://gist.github.com/taylorh140/9e353fdf737f1ef51aacb332efdd9516}{Code
Link}

\pandocbounded{\includegraphics[keepaspectratio]{img/treemap.png}}


\title{sitandr.github.io/typst-examples-book/book/packages/presentation}

\section{\texorpdfstring{\hyperref[presentations]{Presentations}}{Presentations}}\label{presentations}

\subsection{\texorpdfstring{\hyperref[polylux]{Polylux}}{Polylux}}\label{polylux}

\begin{quote}
See \href{https://polylux.dev/book/}{polylux book}
\end{quote}

\begin{verbatim}
// Get Polylux from the official package repository
#import "@preview/polylux:0.3.1": *

// Make the paper dimensions fit for a presentation and the text larger
#set page(paper: "presentation-16-9")
#set text(size: 25pt)

// Use #polylux-slide to create a slide and style it using your favourite Typst functions
#polylux-slide[
  #align(horizon + center)[
    = Very minimalist slides

    A lazy author

    July 23, 2023
  ]
]

#polylux-slide[
  == First slide

  Some static text on this slide.
]

#polylux-slide[
  == This slide changes!

  You can always see this.
  // Make use of features like #uncover, #only, and others to create dynamic content
  #uncover(2)[But this appears later!]
]
\end{verbatim}

\pandocbounded{\includesvg[keepaspectratio]{typst-img/f46993d445b33c112929c1b2e3308a9a2b27297acc2eb470701fbe6b8707f710-1.svg}}

\pandocbounded{\includesvg[keepaspectratio]{typst-img/f46993d445b33c112929c1b2e3308a9a2b27297acc2eb470701fbe6b8707f710-2.svg}}

\pandocbounded{\includesvg[keepaspectratio]{typst-img/f46993d445b33c112929c1b2e3308a9a2b27297acc2eb470701fbe6b8707f710-3.svg}}

\pandocbounded{\includesvg[keepaspectratio]{typst-img/f46993d445b33c112929c1b2e3308a9a2b27297acc2eb470701fbe6b8707f710-4.svg}}

\subsection{\texorpdfstring{\hyperref[slydst]{Slydst}}{Slydst}}\label{slydst}

\begin{quote}
See the documentation
\href{https://github.com/glambrechts/slydst?ysclid=lr2gszrkck541184604}{there}
.
\end{quote}

Much more simpler and less powerful than polulyx:

\begin{verbatim}
#import "@preview/slydst:0.1.0": *

#show: slides.with(
  title: "Insert your title here", // Required
  subtitle: none,
  date: none,
  authors: (),
  layout: "medium",
  title-color: none,
)

== Outline

#outline()

= First section

== First slide

#figure(rect(width: 60%), caption: "Caption")

#v(1fr)

#lorem(20)

#definition(title: "An interesting definition")[
  #lorem(20)
]
\end{verbatim}

\pandocbounded{\includesvg[keepaspectratio]{typst-img/9d718fb02239fe71227dce959f0f468c0520df208e9b55e518dcf43f554bbd28-1.svg}}

\pandocbounded{\includesvg[keepaspectratio]{typst-img/9d718fb02239fe71227dce959f0f468c0520df208e9b55e518dcf43f554bbd28-2.svg}}

\pandocbounded{\includesvg[keepaspectratio]{typst-img/9d718fb02239fe71227dce959f0f468c0520df208e9b55e518dcf43f554bbd28-3.svg}}

\pandocbounded{\includesvg[keepaspectratio]{typst-img/9d718fb02239fe71227dce959f0f468c0520df208e9b55e518dcf43f554bbd28-4.svg}}


\title{sitandr.github.io/typst-examples-book/book/packages/index}

\section{\texorpdfstring{\hyperref[packages]{Packages}}{Packages}}\label{packages}

Once the \href{https://typst.app/universe}{Typst Universe} was launched,
this chapter has become almost redundant. The Universe is actually a
very cool place to look for packages.

However, there are still some cool examples of interesting package
usage.

\subsection{\texorpdfstring{\hyperref[general]{General}}{General}}\label{general}

Typst has packages, but, unlike LaTeX, you need to remember:

\begin{itemize}
\tightlist
\item
  You need them only for some specialized tasks, basic formatting
  \emph{can be totally done without them} .
\item
  Packages are much lighter and much easier "installed" than LaTeX ones.
\item
  Packages are just plain Typst files (and sometimes plugins), so you
  can easily write your own!
\end{itemize}

To use mighty package, just write, like this:

\begin{verbatim}
#import "@preview/cetz:0.1.2": canvas, plot

#canvas(length: 1cm, {
  plot.plot(size: (8, 6),
    x-tick-step: none,
    x-ticks: ((-calc.pi, $-pi$), (0, $0$), (calc.pi, $pi$)),
    y-tick-step: 1,
    {
      plot.add(
        style: plot.palette.blue,
        domain: (-calc.pi, calc.pi), x => calc.sin(x * 1rad))
      plot.add(
        hypograph: true,
        style: plot.palette.blue,
        domain: (-calc.pi, calc.pi), x => calc.cos(x * 1rad))
      plot.add(
        hypograph: true,
        style: plot.palette.blue,
        domain: (-calc.pi, calc.pi), x => calc.cos((x + calc.pi) * 1rad))
    })
})
\end{verbatim}

\pandocbounded{\includesvg[keepaspectratio]{typst-img/29d7015ed96122fa3fb663929c1ac58d25340995423c82456ab8815811373979-1.svg}}

\subsection{\texorpdfstring{\hyperref[contributing]{Contributing}}{Contributing}}\label{contributing}

If you are author of a package or just want to make a fair overview,
feel free to make issues/PR-s!


\title{sitandr.github.io/typst-examples-book/book/packages/word_count}

\section{\texorpdfstring{\hyperref[counting-words]{Counting
words}}{Counting words}}\label{counting-words}

\subsection{\texorpdfstring{\hyperref[wordometr]{Wordometr}}{Wordometr}}\label{wordometr}

\begin{verbatim}
#import "@preview/wordometer:0.1.0": word-count, total-words

#show: word-count

In this document, there are #total-words words all up.

#word-count(total => [
  The number of words in this block is #total.words
  and there are #total.characters letters.
])
\end{verbatim}

\pandocbounded{\includesvg[keepaspectratio]{typst-img/a36d12209002f93aeaf23d4b21fcd4dcb1f9326f6ad358ad01558f09dede39c2-1.svg}}

\subsubsection{\texorpdfstring{\hyperref[excluding-elements]{Excluding
elements}}{Excluding elements}}\label{excluding-elements}

You can exclude elements by name (e.g.,
\texttt{\ }{\texttt{\ "caption"\ }}\texttt{\ } ), function (e.g.,
\texttt{\ }{\texttt{\ figure.caption\ }}\texttt{\ } ), where-selector
(e.g., \texttt{\ }{\texttt{\ raw.where(block:\ true)\ }}\texttt{\ } ),
or \texttt{\ }{\texttt{\ label\ }}\texttt{\ } (e.g.,
\texttt{\ }{\texttt{\ \textless{}\ }}\texttt{\ }{\texttt{\ no-wc\ }}\texttt{\ }{\texttt{\ \textgreater{}\ }}\texttt{\ }
).

\begin{verbatim}
#import "@preview/wordometer:0.1.0": word-count, total-words

#show: word-count.with(exclude: (heading.where(level: 1), strike))

= This Heading Doesn't Count
== But I do!

In this document #strike[(excluding me)], there are #total-words words all up.

#word-count(total => [
  You can exclude elements by label, too.
  #[That was #total-words, excluding this sentence!] <no-wc>
], exclude: <no-wc>)
\end{verbatim}

\pandocbounded{\includesvg[keepaspectratio]{typst-img/0e46f8aa570972e4f8a92bfa4b8f7b86b6374d632fa27bd043c102b683d70f96-1.svg}}


