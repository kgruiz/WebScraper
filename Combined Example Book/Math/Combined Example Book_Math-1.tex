\title{sitandr.github.io/typst-examples-book/book/basics/math/limits}

\section{\texorpdfstring{\hyperref[setting-limits]{Setting
limits}}{Setting limits}}\label{setting-limits}

Sometimes we want to change how the default attaching should work.

\subsection{\texorpdfstring{\hyperref[limits]{Limits}}{Limits}}\label{limits}

For example, in many countries it is common to write definite integrals
with limits below and above. To set this, use
\texttt{\ }{\texttt{\ limits\ }}\texttt{\ } function:

\begin{verbatim}
$
integral_a^b\
limits(integral)_a^b
$
\end{verbatim}

\pandocbounded{\includesvg[keepaspectratio]{typst-img/ade8f85a6178d42d58769da477afa5349a3db9df3075a3d5f8e4a6b546c3d43e-1.svg}}

You can set this by default using
\texttt{\ }{\texttt{\ show\ }}\texttt{\ } rule:

\begin{verbatim}
#show math.integral: math.limits

$
integral_a^b
$

This is inline equation: $integral_a^b$
\end{verbatim}

\pandocbounded{\includesvg[keepaspectratio]{typst-img/e0011edccf76468c3d77a7502ce1dc001c82bfd9d590b258d8c8453d056bc966-1.svg}}

\subsection{\texorpdfstring{\hyperref[only-display-mode]{Only display
mode}}{Only display mode}}\label{only-display-mode}

Notice that this will also affect inline equations. To enable limits for
display math only, use
\texttt{\ }{\texttt{\ limits(inline:\ false)\ }}\texttt{\ } :

\begin{verbatim}
#show math.integral: math.limits.with(inline: false)

$
integral_a^b
$

This is inline equation: $integral_a^b$.
\end{verbatim}

\pandocbounded{\includesvg[keepaspectratio]{typst-img/d37f1132cdf338670e131079a57ae724a7dfcb102f3125dad712173fbf115bcd-1.svg}}

Of course, it is possible to move them back as bottom attachments:

\begin{verbatim}
$
sum_a^b, scripts(sum)_a^b
$
\end{verbatim}

\pandocbounded{\includesvg[keepaspectratio]{typst-img/7134a72120f7217b1f11438e166fa7e53f3a9287fa4c9079019181a6e16affb8-1.svg}}

\subsection{\texorpdfstring{\hyperref[operations]{Operations}}{Operations}}\label{operations}

The same scheme works for operations. By default, they are attached to
the bottom and top:

\begin{verbatim}
$a =_"By lemme 1" b, a scripts(=)_+ b$
\end{verbatim}

\pandocbounded{\includesvg[keepaspectratio]{typst-img/98d790005c43aa666b392f8a35f1e9564ff315aaf9881ceb309e53bd5db542b1-1.svg}}


\title{sitandr.github.io/typst-examples-book/book/basics/math/index}

\section{\texorpdfstring{\hyperref[math]{Math}}{Math}}\label{math}

Math is a special environment that has special features related to...
math.

\subsection{\texorpdfstring{\hyperref[syntax]{Syntax}}{Syntax}}\label{syntax}

To start math environment, \texttt{\ }{\texttt{\ \$\ }}\texttt{\ } . The
spacing around \texttt{\ }{\texttt{\ \$\ }}\texttt{\ } will make it
either \emph{inline} math (smaller, used in text) or \emph{display} math
(used on math equations on their own).

\begin{verbatim}
// This is inline math
Let $a$, $b$, and $c$ be the side
lengths of right-angled triangle.
Then, we know that:

// This is display math
$ a^2 + b^2 = c^2 $

Prove by induction:

// You can use new lines as spacing too!
$
sum_(k=1)^n k = (n(n+1)) / 2
$
\end{verbatim}

\pandocbounded{\includesvg[keepaspectratio]{typst-img/068db3a521a38c3acede771ebb6342807cca4fd98baf5b2b508184a6854ea8ff-1.svg}}

\subsection{\texorpdfstring{\hyperref[mathequation]{Math.equation}}{Math.equation}}\label{mathequation}

The element that math is displayed in is called
\texttt{\ }{\texttt{\ math.equation\ }}\texttt{\ } . You can use it for
set/show rules:

\begin{verbatim}
#show math.equation: set text(red)

$
integral_0^oo (f(t) + g(t))/2
$
\end{verbatim}

\pandocbounded{\includesvg[keepaspectratio]{typst-img/94e0532dd7224d08e966cb82834283efd8889d7f117b04116e721a788bfcc16c-1.svg}}

Any symbol/command that is available in math, \emph{is also available}
in code mode using \texttt{\ }{\texttt{\ math.command\ }}\texttt{\ } :

\begin{verbatim}
#math.integral, #math.underbrace([a + b], [c])
\end{verbatim}

\pandocbounded{\includesvg[keepaspectratio]{typst-img/b4ca12d7f34ed342f3cb3fba2ee1f5b58faa8fceecb74671baacd9166fcbb5aa-1.svg}}

\subsection{\texorpdfstring{\hyperref[letters-and-commands]{Letters and
commands}}{Letters and commands}}\label{letters-and-commands}

Typst aims to have as simple and effective syntax for math as possible.
That means no special symbols, just using commands.

To make it short, Typst uses several simple rules:

\begin{itemize}
\item
  All single-letter words \emph{turn into variables} . That includes any
  \emph{unicode symbols} too!
\item
  All multi-letter words \emph{turn into commands} . They may be
  built-in commands (available with math.something outside of math
  environment). Or they \textbf{may be user-defined variables/functions}
  . If the command \textbf{isn\textquotesingle t defined} , there will
  be \textbf{compilation error} .

  If you use kebab-case or snake\_case for variables you want to use in
  math, you will have to refer to them as \#snake-case-variable.
\item
  To write simple text, use quotes:

\begin{verbatim}
$a "equals to" 2$
\end{verbatim}

  \pandocbounded{\includesvg[keepaspectratio]{typst-img/811f30ede68d08bec254f184c1be319958c3e11f9f9d58c40b2f460bba037e3d-1.svg}}

  Spacing matters there!

\begin{verbatim}
$a "is" 2$, $a"is"2$
\end{verbatim}

  \pandocbounded{\includesvg[keepaspectratio]{typst-img/9cc2d263c76646c623e1e6b73756e1fe1e2c56d7fe0324ee945652107e6456ba-1.svg}}
\item
  You can turn it into multi-letter variables using
  \texttt{\ }{\texttt{\ italic\ }}\texttt{\ } :

\begin{verbatim}
$(italic("mass") v^2)/2$
\end{verbatim}

  \pandocbounded{\includesvg[keepaspectratio]{typst-img/141d8a3b9beb3559387411170f7378078c80cb2ff80d8d5f5345c3231f55df9c-1.svg}}
\end{itemize}

Commands see
\href{https://typst.app/docs/reference/math/\#definitions}{there} (go to
the links to see the commands).

All symbols see
\href{https://typst.app/docs/reference/symbols/sym/}{there} .

\subsection{\texorpdfstring{\hyperref[multiline-equations]{Multiline
equations}}{Multiline equations}}\label{multiline-equations}

To create multiline \emph{display equation} , use the same symbol as in
markup mode: \texttt{\ }{\texttt{\ \textbackslash{}\ }}\texttt{\ } :

\begin{verbatim}
$
a = b\
a = c
$
\end{verbatim}

\pandocbounded{\includesvg[keepaspectratio]{typst-img/2f16d9e64e38ff22ca27a09b0d8eaef1b020e4eccd7d2ce1380e10a0efcea163-1.svg}}

\subsection{\texorpdfstring{\hyperref[escaping]{Escaping}}{Escaping}}\label{escaping}

Any symbol that is used may be escaped with
\texttt{\ }{\texttt{\ \textbackslash{}\ }}\texttt{\ } , like in markup
mode. For example, you can disable fraction:

\begin{verbatim}
$
a  / b \
a \/ b
$
\end{verbatim}

\pandocbounded{\includesvg[keepaspectratio]{typst-img/e7931e55a2772ee992446af8506d8d25b96167e3bb71d5c63ed8ca156530f2d9-1.svg}}

The same way it works with any other syntax.

\subsection{\texorpdfstring{\hyperref[wrapping-inline-math]{Wrapping
inline math}}{Wrapping inline math}}\label{wrapping-inline-math}

Sometimes, when you write large math, it may be too close to text
(especially for some long letter tails).

\begin{verbatim}
#lorem(17) $display(1)/display(1+x^n)$ #lorem(20)
\end{verbatim}

\pandocbounded{\includesvg[keepaspectratio]{typst-img/a9cce2b851a01939a0abfc02e8cd994d20c465d2800cf64c5c6051ead5bc4e9a-1.svg}}

You may easily increase the distance it by wrapping into box:

\begin{verbatim}
#lorem(17) #box($display(1)/display(1+x^n)$, inset: 0.2em) #lorem(20)
\end{verbatim}

\pandocbounded{\includesvg[keepaspectratio]{typst-img/ee9fc5a3ec529a9f3e811a70724c1585c294d82454c22ee9343235556f572792-1.svg}}


\title{sitandr.github.io/typst-examples-book/book/basics/math/alignment}

\section{\texorpdfstring{\hyperref[alignment]{Alignment}}{Alignment}}\label{alignment}

\subsection{\texorpdfstring{\hyperref[general-alignment]{General
alignment}}{General alignment}}\label{general-alignment}

By default display math is center-aligned, but that can be set up with
\texttt{\ }{\texttt{\ show\ }}\texttt{\ } rule:

\begin{verbatim}
#show math.equation: set align(right)

$
(a + b)/2
$
\end{verbatim}

\pandocbounded{\includesvg[keepaspectratio]{typst-img/bcd19808066d4eee09c984bf17077653b1c1bf25115c10a155611056a30e2cb6-1.svg}}

Or using \texttt{\ }{\texttt{\ align\ }}\texttt{\ } element:

\begin{verbatim}
#align(left, block($ x = 5 $))
\end{verbatim}

\pandocbounded{\includesvg[keepaspectratio]{typst-img/4545bd54c4090d4c9599e639aa441b68eb214011861d9949652df140843af042-1.svg}}

\subsection{\texorpdfstring{\hyperref[alignment-points]{Alignment
points}}{Alignment points}}\label{alignment-points}

When equations include multiple alignment points (\&), this creates
blocks of alternatingly \emph{right-} and \emph{left-} aligned columns.

In the example below, the expression
\texttt{\ }{\texttt{\ (3x\ +\ y)\ /\ 7\ }}\texttt{\ } is
\emph{right-aligned} and
\texttt{\ }{\texttt{\ =\ }}\texttt{\ }{\texttt{\ 9\ }}\texttt{\ } is
\emph{left-aligned} .

\begin{verbatim}
$ (3x + y) / 7 &= 9 && "given" \
  3x + y &= 63 & "multiply by 7" \
  3x &= 63 - y && "subtract y" \
  x &= 21 - y/3 & "divide by 3" $
\end{verbatim}

\pandocbounded{\includesvg[keepaspectratio]{typst-img/bfb7a5df8873923079f45d12fa92204afeddecb15ec31d6b8624ac4610d29677-1.svg}}

The word "given" is also left-aligned because
\texttt{\ }{\texttt{\ \&\&\ }}\texttt{\ } creates two alignment points
in a row, \emph{alternating the alignment twice} .

\texttt{\ }{\texttt{\ \&\ \&\ }}\texttt{\ } and
\texttt{\ }{\texttt{\ \&\&\ }}\texttt{\ } behave exactly the same way.
Meanwhile, "multiply by 7" is left-aligned because just one
\texttt{\ }{\texttt{\ \&\ }}\texttt{\ } precedes it.

\textbf{Each alignment point simply alternates between
right-aligned/left-aligned.}


\title{sitandr.github.io/typst-examples-book/book/basics/math/grouping}

\section{\texorpdfstring{\hyperref[grouping]{Grouping}}{Grouping}}\label{grouping}

Every grouping can be (currently) done by parenthesis. So the
parenthesis may be both "real" parenthesis and grouping ones.

For example, these parentheses specify nominator of the fraction:

\begin{verbatim}
$ (a^2 + b^2)/2 $
\end{verbatim}

\pandocbounded{\includesvg[keepaspectratio]{typst-img/6f4767b2aee69b5c3a22df5f394105df9f19c9762678d02b297c4d4f8d1cf6ad-1.svg}}

\subsection{\texorpdfstring{\hyperref[left-right]{Left-right}}{Left-right}}\label{left-right}

\begin{quote}
See \href{https://typst.app/docs/reference/math/lr}{official
documentation} .
\end{quote}

If there are two matching braces of any kind, they will be wrapped as
\texttt{\ }{\texttt{\ lr\ }}\texttt{\ } (left-right) group.

\begin{verbatim}
$
{[((a + b)/2) + 1]_0}
$
\end{verbatim}

\pandocbounded{\includesvg[keepaspectratio]{typst-img/a4137ff5d1f577cc816776cb4279cce4cd964601c20eb244d12e170deecd5d6a-1.svg}}

You can disable it by escaping.

You can also match braces of any kind by using
\texttt{\ }{\texttt{\ lr\ }}\texttt{\ } directly:

\begin{verbatim}
$
lr([a/2, b)) \
lr([a/2, b), size: #150%)
$
\end{verbatim}

\pandocbounded{\includesvg[keepaspectratio]{typst-img/fb81420a901d8b570ef03d1f50c83f7b8c483c9366222156ea193ac2976b63ed-1.svg}}

\subsection{\texorpdfstring{\hyperref[fences]{Fences}}{Fences}}\label{fences}

Fences \emph{are not matched automatically} because of large amount of
false-positives.

You can use \texttt{\ }{\texttt{\ abs\ }}\texttt{\ } or
\texttt{\ }{\texttt{\ norm\ }}\texttt{\ } to match them:

\begin{verbatim}
$
abs(a + b), norm(a + b), floor(a + b), ceil(a + b), round(a + b)
$
\end{verbatim}

\pandocbounded{\includesvg[keepaspectratio]{typst-img/fd8454b2a97d649525827367f459f3163d830b5db9181178304d5fd2b44fcca1-1.svg}}


\title{sitandr.github.io/typst-examples-book/book/basics/math/symbols}

\section{\texorpdfstring{\hyperref[symbols]{Symbols}}{Symbols}}\label{symbols}

Multiletter words in math refer either to local variables, functions,
text operators, spacing or \emph{special symbols} . The latter are very
important for advanced math.

\begin{verbatim}
$
forall v, w in V, alpha in KK: alpha dot (v + w) = alpha v + alpha w
$
\end{verbatim}

\pandocbounded{\includesvg[keepaspectratio]{typst-img/60a6e3e08582c87ec082b6714a45a90a914dd1299f788e2bb21b0cc5adc80e6a-1.svg}}

You can write the same with unicode:

\begin{verbatim}
$
∀ v, w ∈ V, α ∈ 𝕂: α ⋅ (v + w) = α v + α w
$
\end{verbatim}

\pandocbounded{\includesvg[keepaspectratio]{typst-img/d37776c21d5c4d692e4ebbe7e5ce7e7cdf5e2c0777a88a47abe0c0c5992cf41a-1.svg}}

\subsection{\texorpdfstring{\hyperref[symbols-naming]{Symbols
naming}}{Symbols naming}}\label{symbols-naming}

\begin{quote}
See all available symbols list
\href{https://typst.app/docs/reference/symbols/sym/}{there} .
\end{quote}

\subsubsection{\texorpdfstring{\hyperref[general-idea]{General
idea}}{General idea}}\label{general-idea}

Typst wants to define some "basic" symbols with small easy-to-remember
words, and build complex ones using combinations. For example,

\begin{verbatim}
$
// cont — contour
integral, integral.cont, integral.double, integral.square, sum.integral\

// lt — less than, gt — greater than
lt, lt.circle, lt.eq, lt.not, lt.eq.not, lt.tri, lt.tri.eq, lt.tri.eq.not, gt, lt.gt.eq, lt.gt.not
$
\end{verbatim}

\pandocbounded{\includesvg[keepaspectratio]{typst-img/a0ee196d2bf305ca6c2d812008f9955e5ae526de0b0ac0b83ca016a66bdc00f1-1.svg}}

I highly recommend using WebApp/Typst LSP when writing math with lots of
complex symbols. That helps you to quickly choose the right symbol
within all combinations.

Sometimes the names are not obvious, for example, sometimes it is used
prefix \texttt{\ }{\texttt{\ n-\ }}\texttt{\ } instead of
\texttt{\ }{\texttt{\ not\ }}\texttt{\ } :

\begin{verbatim}
$
gt.nequiv, gt.napprox, gt.ntilde, gt.tilde.not
$
\end{verbatim}

\pandocbounded{\includesvg[keepaspectratio]{typst-img/e4d0ef024efaf9f4334ebf04a2ac4e015fc5ec76617be8b6d7aad2f4429e3317-1.svg}}

\subsubsection{\texorpdfstring{\hyperref[common-modifiers]{Common
modifiers}}{Common modifiers}}\label{common-modifiers}

\begin{itemize}
\item
  \texttt{\ }{\texttt{\ .b,\ .t,\ .l,\ .r\ }}\texttt{\ } : bottom, top,
  left, right. Change direction of symbol.

\begin{verbatim}
$arrow.b, triangle.r, angle.l$
\end{verbatim}

  \pandocbounded{\includesvg[keepaspectratio]{typst-img/8ab0fa590b7a39023b1467e7a376a4810f997f720dd5d221ad83d7e741943b55-1.svg}}
\end{itemize}


\title{sitandr.github.io/typst-examples-book/book/basics/math/vec}

\section{\texorpdfstring{\hyperref[vectors-matrices-semicolumn-syntax]{Vectors,
matrices, semicolumn
syntax}}{Vectors, matrices, semicolumn syntax}}\label{vectors-matrices-semicolumn-syntax}

\subsection{\texorpdfstring{\hyperref[vectors]{Vectors}}{Vectors}}\label{vectors}

\begin{quote}
By vector we mean a column there.\\
To write arrow notations for letters, use
\texttt{\ }{\texttt{\ \$\ }}\texttt{\ }{\texttt{\ arrow\ }}\texttt{\ }{\texttt{\ (\ }}\texttt{\ }{\texttt{\ v\ }}\texttt{\ }{\texttt{\ )\ }}\texttt{\ }{\texttt{\ \$\ }}\texttt{\ }\\
I recommend to create shortcut for this, like
\texttt{\ }{\texttt{\ \#let\ }}\texttt{\ }{\texttt{\ arr\ }}\texttt{\ }{\texttt{\ =\ }}\texttt{\ }{\texttt{\ math.arrow\ }}\texttt{\ }
\end{quote}

To write columns, use \texttt{\ }{\texttt{\ vec\ }}\texttt{\ } command:

\begin{verbatim}
$
vec(a, b, c) + vec(1, 2, 3) = vec(a + 1, b + 2, c + 3)
$
\end{verbatim}

\pandocbounded{\includesvg[keepaspectratio]{typst-img/92aa72b3d4f797123f550cc8630b34e09176956c4b116cc0a4fe48d457e1ee0a-1.svg}}

\subsubsection{\texorpdfstring{\hyperref[delimiter]{Delimiter}}{Delimiter}}\label{delimiter}

You can change parentheses around the column or even remove them:

\begin{verbatim}
$
vec(1, 2, 3, delim: "{") \
vec(1, 2, 3, delim: bar.double) \
vec(1, 2, 3, delim: #none)
$
\end{verbatim}

\pandocbounded{\includesvg[keepaspectratio]{typst-img/efd7cc6c6abb317c316b746f7a286ab2f8b2a023fe19bf77c15638db9c6bed8f-1.svg}}

\subsubsection{\texorpdfstring{\hyperref[gap]{Gap}}{Gap}}\label{gap}

You can change the size of gap between rows:

\begin{verbatim}
$
vec(a, b, c)
vec(a, b, c, gap:#0em)
vec(a, b, c, gap:#1em)
$
\end{verbatim}

\pandocbounded{\includesvg[keepaspectratio]{typst-img/8977ff36f1f7a4b78c2fdbaef8764fec4b2cb42092f63b07176cca13707c0407-1.svg}}

\subsubsection{\texorpdfstring{\hyperref[making-gap-even]{Making gap
even}}{Making gap even}}\label{making-gap-even}

You can easily note that the gap isn\textquotesingle t necessarily even
or the same in different vectors:

\begin{verbatim}
$
vec(a/b, a/b, a/b) = vec(1, 1, 1)
$
\end{verbatim}

\pandocbounded{\includesvg[keepaspectratio]{typst-img/c3141fb95a4280df589e5e9fc0d605d59b16a8da6b4a01be532fab0bf04f6a00-1.svg}}

That happens because \texttt{\ }{\texttt{\ gap\ }}\texttt{\ } refers to
\emph{spacing between} elements, not the distance between their centers.

To fix this, you can use \href{../../snippets/math/vecs.html}{this
snippet} .

\subsection{\texorpdfstring{\hyperref[matrix]{Matrix}}{Matrix}}\label{matrix}

\begin{quote}
See \href{https://typst.app/docs/reference/math/mat/}{official
reference}
\end{quote}

Matrix is very similar to \texttt{\ }{\texttt{\ vec\ }}\texttt{\ } , but
accepts rows, separated by \texttt{\ }{\texttt{\ ;\ }}\texttt{\ } :

\begin{verbatim}
$
mat(
    1, 2, ..., 10;
    2, 2, ..., 10;
    dots.v, dots.v, dots.down, dots.v;
    10, 10, ..., 10; // `;` in the end is optional
)
$
\end{verbatim}

\pandocbounded{\includesvg[keepaspectratio]{typst-img/ca1e7bdfe61f2ae541843aeff854d40882487bed8fd5b1e094852cf662a759f8-1.svg}}

\subsubsection{\texorpdfstring{\hyperref[delimiters-and-gaps]{Delimiters
and gaps}}{Delimiters and gaps}}\label{delimiters-and-gaps}

You can specify them the same way as for vectors.

Specify the arguments either before the content, or \textbf{after the
semicolon} . The code will panic if there is no semicolon!

\begin{verbatim}
$
mat(
    delim: "|",
    1, 2, ..., 10;
    2, 2, ..., 10;
    dots.v, dots.v, dots.down, dots.v;
    10, 10, ..., 10;
    gap: #0.3em
)
$
\end{verbatim}

\pandocbounded{\includesvg[keepaspectratio]{typst-img/8fd5effce0cef589ea8f7e7388cf221f1c8d7f0ac6c76d8d7d2fb14c4840bef7-1.svg}}

\subsection{\texorpdfstring{\hyperref[semicolon-syntax]{Semicolon
syntax}}{Semicolon syntax}}\label{semicolon-syntax}

When you use semicolons, the arguments \emph{between the semicolons} are
merged into arrays. See yourself:

\begin{verbatim}
#let fun(..args) = {
    args.pos()
}

$
fun(1, 2;3, 4; 6, ; 8)
$
\end{verbatim}

\pandocbounded{\includesvg[keepaspectratio]{typst-img/a589a9f51ffa925d9dce1da521c4d15373e236fd19db49317091d681c2fface0-1.svg}}

If you miss some of elements, they will be replaced by
\texttt{\ }{\texttt{\ none\ }}\texttt{\ } -s.

You can mix semicolon syntax and named arguments, but be careful!

\begin{verbatim}
#let fun(..args) = {
    repr(args.pos())
    repr(args.named())
}

$
fun(1, 2; gap: #3em, 4)
$
\end{verbatim}

\pandocbounded{\includesvg[keepaspectratio]{typst-img/7a3c90212650f7f7df0cb42177753236eddae675ac3220fbabd0f40e4af8b842-1.svg}}

For example, this will not work:

\begin{verbatim}
$
//         ↓ there is no `;`, so it tries to add (gap:) to array
mat(1, 2; 4, gap: #3em)
$
\end{verbatim}


\title{sitandr.github.io/typst-examples-book/book/basics/math/classes}

\section{\texorpdfstring{\hyperref[classes]{Classes}}{Classes}}\label{classes}

\begin{quote}
See \href{https://typst.app/docs/reference/math/class/}{official
documentation}
\end{quote}

Each math symbol has its own "class", the way it behaves.
That\textquotesingle s one of the main reasons why they are layouted
differently.

\subsection{\texorpdfstring{\hyperref[classes-1]{Classes}}{Classes}}\label{classes-1}

\begin{verbatim}
$
a b c\
a class("normal", b) c\
a class("punctuation", b) c\
a class("opening", b) c\
a lr(b c]) c\
a lr(class("opening", b) c ]) c // notice it is moved vertically \
a class("closing", b) c\
a class("fence", b) c\
a class("large", b) c\
a class("relation", b) c\
a class("unary", b) c\
a class("binary", b) c\
a class("vary", b) c\
$
\end{verbatim}

\pandocbounded{\includesvg[keepaspectratio]{typst-img/5d4604274229b2f53ee04b88ff0e73d9aa8365643c5e60052fcca1298d4f5a23-1.svg}}

\subsection{\texorpdfstring{\hyperref[setting-class-for-symbol]{Setting
class for
symbol}}{Setting class for symbol}}\label{setting-class-for-symbol}

\begin{verbatim}
Default:

$square circle square$

With `#h(0)`:

$square #h(0pt) circle #h(0pt) square$

With `math.class`:

#show math.circle: math.class.with("normal")
$square circle square$
\end{verbatim}

\pandocbounded{\includesvg[keepaspectratio]{typst-img/86a709c6189649b79005752253a842631eed4722b350e4197116e0be19094035-1.svg}}


\title{sitandr.github.io/typst-examples-book/book/basics/math/operators}

\section{\texorpdfstring{\hyperref[operators]{Operators}}{Operators}}\label{operators}

\begin{quote}
See \href{https://typst.app/docs/reference/math/op/}{reference} .
\end{quote}

There are lots of built-in "text operators" in Typst math. This is a
symbol that behaves very close to plain text. Nevertheless, it is
different:

\begin{verbatim}
$
lim x_n, "lim" x_n, "lim"x_n
$
\end{verbatim}

\pandocbounded{\includesvg[keepaspectratio]{typst-img/b195783135218e8117ac954790e7a108297d7a3e532136d851e2c397358509f0-1.svg}}

\subsection{\texorpdfstring{\hyperref[predefined-operators]{Predefined
operators}}{Predefined operators}}\label{predefined-operators}

Here are all text operators Typst has built-in:

\begin{verbatim}
$
arccos, arcsin, arctan, arg, cos, cosh, cot, coth, csc,\
csch, ctg, deg, det, dim, exp, gcd, hom, id, im, inf, ker,\
lg, lim, liminf, limsup, ln, log, max, min, mod, Pr, sec,\
sech, sin, sinc, sinh, sup, tan, tanh, tg "and" tr
$
\end{verbatim}

\pandocbounded{\includesvg[keepaspectratio]{typst-img/8a14bfdd8bd657d613ccbcd3f77d68f31e6d73e509ba85dd8e6f5207d5c8c7e4-1.svg}}

\subsection{\texorpdfstring{\hyperref[creating-custom-operator]{Creating
custom
operator}}{Creating custom operator}}\label{creating-custom-operator}

Of course, there always will be some text operators you will need that
are not in the list.

But don\textquotesingle t worry, it is very easy to add your own:

\begin{verbatim}
#let arcsinh = math.op("arcsinh")

$
arcsinh x
$
\end{verbatim}

\pandocbounded{\includesvg[keepaspectratio]{typst-img/e4f5a9aa5dfd03914d26ad85ed73eff426d21badca21ea5a6e8de5032b2f29bb-1.svg}}

\subsubsection{\texorpdfstring{\hyperref[limits-for-operators]{Limits
for operators}}{Limits for operators}}\label{limits-for-operators}

When creating operators (upright text with proper spacing), you can set
limits for \emph{display mode} at the same time:

\begin{verbatim}
$
op("liminf")_a, op("liminf", limits: #true)_a
$
\end{verbatim}

\pandocbounded{\includesvg[keepaspectratio]{typst-img/9c3593b91bf3810a593b622e4972c5a87d637696f35850422f9232c74802a394-1.svg}}

This is roughly equivalent to

\begin{verbatim}
$
limits(op("liminf"))_a
$
\end{verbatim}

\pandocbounded{\includesvg[keepaspectratio]{typst-img/7aaabb25d8e73d54504aa3e99b9c8b341759f165923439447f4990871ec3943f-1.svg}}

Everything can be combined to create new operators:

\begin{verbatim}
#let liminf = math.op(math.underline(math.lim), limits: true)
#let limsup = math.op(math.overline(math.lim), limits: true)
#let integrate = math.op($integral dif x$)

$
liminf_(x->oo)\
limsup_(x->oo)\
integrate x^2
$
\end{verbatim}

\pandocbounded{\includesvg[keepaspectratio]{typst-img/adf6ee9659a71ecefb64d09f5f27f01acdd193bc79c792abf95fc56821bca4cb-1.svg}}


\title{sitandr.github.io/typst-examples-book/book/basics/math/sizes}

\section{\texorpdfstring{\hyperref[location-and-sizes]{Location and
sizes}}{Location and sizes}}\label{location-and-sizes}

We talked already about display and inline math. They differ not only by
aligning and spacing, but also by size and style:

\begin{verbatim}
Inline: $a/(b + 1/c), sum_(n=0)^3 x_n$

$
a/(b + 1/c), sum_(n=0)^3 x_n
$
\end{verbatim}

\pandocbounded{\includesvg[keepaspectratio]{typst-img/7de20fcaee4fb6ea523b34bfe9b2be6b91cc6e6a5b46fab0eebe7f0155689f8e-1.svg}}

The size and style of current environment is described by Math Size, see
\href{https://typst.app/docs/reference/math/sizes}{reference} .

There are for sizes:

\begin{itemize}
\tightlist
\item
  Display math size ( \texttt{\ }{\texttt{\ display\ }}\texttt{\ } )
\item
  Inline math size ( \texttt{\ }{\texttt{\ inline\ }}\texttt{\ } )
\item
  Script math size ( \texttt{\ }{\texttt{\ script\ }}\texttt{\ } )
\item
  Sub/super script math size (
  \texttt{\ }{\texttt{\ sscript\ }}\texttt{\ } )
\end{itemize}

Each time thing is used in fraction, script or exponent, it is moved
several "levels lowers", becoming smaller and more "crapping".
\texttt{\ }{\texttt{\ sscript\ }}\texttt{\ } isn\textquotesingle t
reduced father:

\begin{verbatim}
$
"display:" 1/("inline:" a + 1/("script:" b + 1/("sscript:" c + 1/("sscript:" d + 1/("sscript:" e + 1/f)))))
$
\end{verbatim}

\pandocbounded{\includesvg[keepaspectratio]{typst-img/9c8cbc46da7dc8eb9436c561107cbb97a836aaa7b120a9bc3f044dd648d702e1-1.svg}}

\subsection{\texorpdfstring{\hyperref[setting-sizes-manually]{Setting
sizes manually}}{Setting sizes manually}}\label{setting-sizes-manually}

Just use the corresponding command:

\begin{verbatim}
Inine: $sum_0^oo e^x^a$\
Inline with limits: $limits(sum)_0^oo e^x^a$\
Inline, but like true display: $display(sum_0^oo e^x^a)$
\end{verbatim}

\pandocbounded{\includesvg[keepaspectratio]{typst-img/0d16a9d157c9689f4b3cce434ebf89d9a18d67b4916ac0ebfbce3daccb94e709-1.svg}}


