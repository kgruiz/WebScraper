\title{sitandr.github.io/typst-examples-book/book/snippets/numbering}

\section{\texorpdfstring{\hyperref[numbering]{Numbering}}{Numbering}}\label{numbering}

\subsection{\texorpdfstring{\hyperref[individual-heading-without-numbering]{Individual
heading without
numbering}}{Individual heading without numbering}}\label{individual-heading-without-numbering}

\begin{verbatim}
#let numless(it) = {set heading(numbering: none); it }

= Heading
#numless[=No numbering heading]
\end{verbatim}

\pandocbounded{\includesvg[keepaspectratio]{typst-img/e04f844b270049702ac72dff7bfadf5963cdb2bc8a541e81b685124fbb61c48e-1.svg}}

\subsection{\texorpdfstring{\hyperref[clean-numbering]{"Clean"
numbering}}{"Clean" numbering}}\label{clean-numbering}

\begin{verbatim}
// original author: tromboneher

// Number sections according to a number of schemes, omitting previous leading elements.
// For example, where the numbering pattern "A.I.1." would produce:
//
// A. A part of the story
//   A.I. A chapter
//   A.II. Another chapter
//     A.II.1. A section
//       A.II.1.a. A subsection
//       A.II.1.b. Another subsection
//     A.II.2. Another section
// B. Another part of the story
//   B.I. A chapter in the second part
//   B.II. Another chapter in the second part
//
// clean_numbering("A.", "I.", "1.a.") would produce:
//
// A. A part of the story
//   I. A chapter
//   II. Another chapter
//     1. A section
//       1.a. A subsection
//       1.b. Another subsection
//     2. Another section
// B. Another part of the story
//   I. A chapter in the second part
//   II. Another chapter in the second part
//
#let clean_numbering(..schemes) = {
  (..nums) => {
    let (section, ..subsections) = nums.pos()
    let (section_scheme, ..subschemes) = schemes.pos()

    if subsections.len() == 0 {
      numbering(section_scheme, section)
    } else if subschemes.len() == 0 {
      numbering(section_scheme, ..nums.pos())
    }
    else {
      clean_numbering(..subschemes)(..subsections)
    }
  }
}

#set heading(numbering: clean_numbering("A.", "I.", "1.a."))

= Part
== Chapter
== Another chapter
=== Section
==== Subsection
==== Another subsection
= Another part of the story
== A chapter in the second part
== Another chapter in the second part
\end{verbatim}

\pandocbounded{\includesvg[keepaspectratio]{typst-img/4e29319442704545bf58d12448745836598c12f59162d3199aaad21c752e4483-1.svg}}

\subsection{\texorpdfstring{\hyperref[math-numbering]{Math
numbering}}{Math numbering}}\label{math-numbering}

See \href{./math/numbering.html}{there} .

\subsection{\texorpdfstring{\hyperref[numbering-each-paragraph]{Numbering
each
paragraph}}{Numbering each paragraph}}\label{numbering-each-paragraph}

By the 0.12 version of Typst, this should be replaced with good native
solution.

\begin{verbatim}
// original author: roehlichA
// Legal formatting of enumeration
#show enum: it => context {
  // Retrieve the last heading so we know what level to step at
  let headings = query(selector(heading).before(here()))
  let last = headings.at(-1)

  // Combine the output items
  let output = ()
  for item in it.children {
    output.push([
      #context{
        counter(heading).step(level: last.level + 1)
      }
      #context {
        counter(heading).display() 
      }
    ])
    output.push([
      #text(item.body)
      #parbreak()
    ])
  }

  // Display in a grid
  grid(
    columns: (auto, 1fr),
    column-gutter: 1em,
    row-gutter: 1em,
    ..output
  )

}

#set heading(numbering: "1.")

= Some heading
+ Paragraph
= Other
+ Paragraphs here are preceded with a number so they can be referenced directly.
+ _#lorem(100)_
+ _#lorem(100)_

== A subheading
+ Paragraphs are also numbered correctly in subheadings.
+ _#lorem(50)_
+ _#lorem(50)_
\end{verbatim}

\pandocbounded{\includesvg[keepaspectratio]{typst-img/8d5603f93334c1d0fd7391811f90b161d4ff8c7eb81100dc152caac5c6d13daf-1.svg}}


\title{sitandr.github.io/typst-examples-book/book/snippets/index}

\section{\texorpdfstring{\hyperref[typst-snippets]{Typst
Snippets}}{Typst Snippets}}\label{typst-snippets}

Useful snippets for common (and not) tasks.


\title{sitandr.github.io/typst-examples-book/book/snippets/demos}

\section{\texorpdfstring{\hyperref[demos]{Demos}}{Demos}}\label{demos}

\subsection{\texorpdfstring{\hyperref[resume-using-template]{Resume
(using
template)}}{Resume (using template)}}\label{resume-using-template}

\begin{verbatim}
#import "@preview/modern-cv:0.1.0": *

#show: resume.with(
  author: (
      firstname: "John", 
      lastname: "Smith",
      email: "js@example.com", 
      phone: "(+1) 111-111-1111",
      github: "DeveloperPaul123",
      linkedin: "Example",
      address: "111 Example St. Example City, EX 11111",
      positions: (
        "Software Engineer",
        "Software Architect"
      )
  ),
  date: datetime.today().display()
)

= Education

#resume-entry(
  title: "Example University",
  location: "B.S. in Computer Science",
  date: "August 2014 - May 2019",
  description: "Example"
)

#resume-item[
  - #lorem(20)
  - #lorem(15)
  - #lorem(25)
]
\end{verbatim}

\pandocbounded{\includesvg[keepaspectratio]{typst-img/fc69693c49a6cf8021751980642ed7649c9d905056f510fb8e4a994937faeaa2-1.svg}}

\subsection{\texorpdfstring{\hyperref[book-cover]{Book
cover}}{Book cover}}\label{book-cover}

\begin{verbatim}
// author: bamdone
#let accent  = rgb("#00A98F")
#let accent1 = rgb("#98FFB3")
#let accent2 = rgb("#D1FF94")
#let accent3 = rgb("#D3D3D3")
#let accent4 = rgb("#ADD8E6")
#let accent5 = rgb("#FFFFCC")
#let accent6 = rgb("#F5F5DC")

#set page(paper: "a4",margin: 0.0in, fill: accent)

#set rect(stroke: 4pt)
#move(
  dx: -6cm, dy: 1.0cm,
  rotate(-45deg,
    rect(
      width: 100cm,
      height: 2cm,
      radius: 50%,
      stroke: 0pt,
      fill:accent1,
)))

#set rect(stroke: 4pt)
#move(
  dx: -2cm, dy: -1.0cm,
  rotate(-45deg,
    rect(
      width: 100cm,
      height: 2cm,
      radius: 50%,
      stroke: 0pt,
      fill:accent2,
)))

#set rect(stroke: 4pt)
#move(
  dx: 8cm, dy: -10cm,
  rotate(-45deg,
    rect(
      width: 100cm,
      height: 1cm,
      radius: 50%,
      stroke: 0pt,
      fill:accent3,
)))

#set rect(stroke: 4pt)
#move(
  dx: 7cm, dy: -8cm,
  rotate(-45deg,
    rect(
      width: 1000cm,
      height: 2cm,
      radius: 50%,
      stroke: 0pt,
      fill:accent4,
)))

#set rect(stroke: 4pt)
#move(
  dx: 0cm, dy: -0cm,
  rotate(-45deg,
    rect(
      width: 1000cm,
      height: 2cm,
      radius: 50%,
      stroke: 0pt,
      fill:accent1,
)))

#set rect(stroke: 4pt)
#move(
  dx: 9cm, dy: -7cm,
  rotate(-45deg,
    rect(
      width: 1000cm,
      height: 1.5cm,
      radius: 50%,
      stroke: 0pt,
      fill:accent6,
)))

#set rect(stroke: 4pt)
#move(
  dx: 16cm, dy: -13cm,
  rotate(-45deg,
    rect(
      width: 1000cm,
      height: 1cm,
      radius: 50%,
      stroke: 0pt,
      fill:accent2,
)))

#align(center)[
  #rect(width: 30%,
    fill: accent4,
    stroke:none,
    [#align(center)[
      #text(size: 60pt,[Title])
    ]
    ])
]

#align(center)[
  #rect(width: 30%,
    fill: accent4,
    stroke:none,
    [#align(center)[
      #text(size: 20pt,[author])
    ]
    ])
]
\end{verbatim}

\pandocbounded{\includesvg[keepaspectratio]{typst-img/7c2e798dacec8ac970ac2b328c60f8145441d059f16f7bd193f389d78d121981-1.svg}}


\title{sitandr.github.io/typst-examples-book/book/snippets/external}

\section{\texorpdfstring{\hyperref[use-with-external-tools]{Use with
external
tools}}{Use with external tools}}\label{use-with-external-tools}

Currently the best ways to communicate is using

\begin{enumerate}
\tightlist
\item
  Preprocessing. The tool should generate Typst file
\item
  Typst Query (CLI). See the docs
  \href{https://typst.app/docs/reference/meta/query\#command-line-queries}{there}
  .
\item
  WebAssembly plugins. See the docs
  \href{https://typst.app/docs/reference/foundations/plugin/}{there} .
\end{enumerate}

In some time there will be examples of successful usage of first two
methods. For the third one, see \href{../packages/index.html}{packages}
.


\title{sitandr.github.io/typst-examples-book/book/snippets/labels}

\section{\texorpdfstring{\hyperref[labels]{Labels}}{Labels}}\label{labels}

\subsection{\texorpdfstring{\hyperref[get-chapter-of-label]{Get chapter
of label}}{Get chapter of label}}\label{get-chapter-of-label}

\begin{verbatim}
#let ref-heading(label) = context {
  let elems = query(label)
  if elems.len() != 1 {
    panic("found multiple elements")
  }
  let element = elems.first()
  if element.func() != heading {
    panic("label must target heading")
  }
  link(label, element.body)
}

= Design <design>
#lorem(20)

= Implementation
In #ref-heading(<design>), we discussed...
\end{verbatim}

\pandocbounded{\includesvg[keepaspectratio]{typst-img/7a4a9436d9aa0cbf0d3212b45d54bd90a896181c30b036326d99dee9f58eb117-1.svg}}

\subsection{\texorpdfstring{\hyperref[allow-missing-references]{Allow
missing
references}}{Allow missing references}}\label{allow-missing-references}

\begin{verbatim}
// author: Enivex
#set heading(numbering: "1.")

#let myref(label) = context {
    if query(label).len() != 0 {
        ref(label)
    } else {
        // missing reference
        text(fill: red)[???]
    }
}

= Second <test2>

#myref(<test>)

#myref(<test2>)
\end{verbatim}

\pandocbounded{\includesvg[keepaspectratio]{typst-img/cd5f1f34e81c117063da3bb176c1dda726bbc18ac981121f75555d5834b08058-1.svg}}


\title{sitandr.github.io/typst-examples-book/book/snippets/gradients}

\section{\texorpdfstring{\hyperref[color--gradients]{Color \&
Gradients}}{Color \& Gradients}}\label{color--gradients}

\subsection{\texorpdfstring{\hyperref[gradients]{Gradients}}{Gradients}}\label{gradients}

Gradients may be very cool for presentations or just a pretty look.

\begin{verbatim}
/// author: frozolotl
#set page(paper: "presentation-16-9", margin: 0pt)
#set text(fill: white, font: "Inter")

#let grad = gradient.linear(rgb("#953afa"), rgb("#c61a22"), angle: 135deg)

#place(horizon + left, image(width: 60%, "../img/landscape.png"))

#place(top, polygon(
  (0%, 0%),
  (70%, 0%),
  (70%, 25%),
  (0%, 29%),
  fill: white,
))
#place(bottom, polygon(
  (0%, 100%),
  (100%, 100%),
  (100%, 30%),
  (60%, 30% + 60% * 4%),
  (60%, 60%),
  (0%, 64%),
  fill: grad,
))

#place(top + right, block(inset: 7pt, image(height: 19%, "../img/tub.png")))

#place(bottom, block(inset: 40pt)[
  #text(size: 30pt)[
    Presentation Title
  ]

  #text(size: 16pt)[#lorem(20) | #datetime.today().display()]
])
\end{verbatim}

\pandocbounded{\includesvg[keepaspectratio]{typst-img/19558fb55cdd5c8f8193724d92b54b1258ca8ee3d4d4c9e7077fc50d11e1a79d-1.svg}}


\title{sitandr.github.io/typst-examples-book/book/snippets/grids}

\subsection{\texorpdfstring{\hyperref[fractional-grids]{Fractional
grids}}{Fractional grids}}\label{fractional-grids}

For tables with lines of changing length, you can try using \emph{grids
in grids} .

Don\textquotesingle t use this where
\href{https://typst.app/docs/reference/model/table/\#definitions-cell-colspan}{cell.colspan
and rowspan} will do.

\begin{verbatim}
// author: jimpjorps

#grid(
  columns: (1fr,),
  grid(
    columns: (1fr,)*2, inset: 5pt, stroke: 1pt, [hello], [world]
  ),
  grid(
    columns: (1fr,)*3, inset: 5pt, stroke: 1pt, [foo], [bar], [baz]
  ),
  grid.cell(inset: 5pt, stroke: 1pt)[abcxyz]
)
\end{verbatim}

\pandocbounded{\includesvg[keepaspectratio]{typst-img/5b2869a2b2efca1af57cb7ed6fab90ad0c83a35b76c05258a1ae64096d5a8173-1.svg}}

\subsection{\texorpdfstring{\hyperref[automerge-adjacent-cells-with-same-values]{Automerge
adjacent cells with same
values}}{Automerge adjacent cells with same values}}\label{automerge-adjacent-cells-with-same-values}

This example works for adjacent cells horizontally, but
it\textquotesingle s not hard to upgrade it to columns too.

\begin{verbatim}
// author: tebine
#let merge(children, n-cols) = {
  let rows = children.chunks(n-cols)
  let new-children = ()
  for r in rows {
    // First group starts at index 0
    let i = 0 
    // Search next group
    while i < r.len() {
      // Group starts with one cell
      let c = r.at(i).body
      let n = 1
      for j in range(i+1, r.len()) {
        let c-next = r.at(j).body
        if c-next == c {
          // Add cell to group
          n += 1
        } else {
          break
        }
      }
      // Group is finished
      new-children.push(table.cell(colspan: n, c))
      i += n
    }
  }
  return new-children
}
#show table: it => {
  let merged = merge(it.children, it.columns.len())
  if it.children.len() == merged.len() { // trick to avoid recursion
    return it
  }
  table(columns: it.columns.len(), ..merged)
}
#table(columns: 2,
  [1], [2],
  [3], [3],
  [4], [5],
)
\end{verbatim}

\pandocbounded{\includesvg[keepaspectratio]{typst-img/5bf649017afba6f1af8a5ae7e6a1e8b614def90749a092f92e5886a58b351205-1.svg}}

\subsection{\texorpdfstring{\hyperref[slanted-column-headers-with-slanted-borders]{Slanted
column headers with slanted
borders}}{Slanted column headers with slanted borders}}\label{slanted-column-headers-with-slanted-borders}

\begin{verbatim}
// author: tebine
#let slanted(it, alpha: 45deg, len: 2.5cm) = layout(size => {
  let width = size.width
  let b = box(inset: 5pt, rotate(-alpha, reflow: true, it))
  let b-size = measure(b)
  let l = line(angle: -alpha, length: len)
  let l-width = len * calc.cos(alpha)
  let l-height = len * calc.sin(alpha)
  place(bottom+left, l)
  place(bottom+left, l, dx: width)
  place(bottom+left, line(length: width), dx: l-width, dy: -l-height)
  place(bottom+left, dx: width/2, b)
  box(height: l-height) // invisible box to set the height
})

#table(
  columns: 2,
  align: center,
  table.header(
    table.cell(stroke: none, inset: 0pt, slanted[*AAA*]),
    table.cell(stroke: none, inset: 0pt, slanted[*BBBBBB*]),
  ),
  [aaaaa], [bbbbbb], [c], [d],
)
\end{verbatim}

\pandocbounded{\includesvg[keepaspectratio]{typst-img/d5f49858e9acc4bad217904e87abb368aa5e38652bdcba27a971b3ddd10f0361-1.svg}}


\title{sitandr.github.io/typst-examples-book/book/snippets/logos}

\section{\texorpdfstring{\hyperref[logos--figures]{Logos \&
Figures}}{Logos \& Figures}}\label{logos--figures}

Using SVG-s images is totally fine. Totally. But if you are lazy and
don\textquotesingle t want to search for images, here are some logos you
can just copy-paste in your document.

\textbf{Important} : \emph{Typst in text doesn\textquotesingle t need a
special writing (unlike LaTeX)} . Just write "Typst", maybe "
\textbf{Typst} ", and it is okay.

\subsection{\texorpdfstring{\hyperref[tex-and-latex]{TeX and
LaTeX}}{TeX and LaTeX}}\label{tex-and-latex}

\begin{verbatim}
#let TeX = {
  set text(font: "New Computer Modern", weight: "regular")
  box(width: 1.7em, {
    [T]
    place(top, dx: 0.56em, dy: 0.22em)[E]
    place(top, dx: 1.1em)[X]
  })
}

#let LaTeX = {
  set text(font: "New Computer Modern", weight: "regular")
  box(width: 2.55em, {
    [L]
    place(top, dx: 0.3em, text(size: 0.7em)[A])
    place(top, dx: 0.7em)[#TeX]
  })
}

Typst is not that hard to learn when you know #TeX and #LaTeX.
\end{verbatim}

\pandocbounded{\includesvg[keepaspectratio]{typst-img/9432efecd4502f681e3582d8581d0c325e0a89729d57b6d4bea732c2b9f476ec-1.svg}}

\subsection{\texorpdfstring{\hyperref[typst-guy]{Typst
guy}}{Typst guy}}\label{typst-guy}

\begin{verbatim}
// author: fenjalien
#import "@preview/cetz:0.1.2": *

#set page(width: auto, height: auto)

#canvas(length: 1pt, {
  import draw: *
  let color = rgb("239DAD")
  scale((y: -1))
  set-style(fill: color, stroke: none,)

  // body
  merge-path({
    bezier(
      (112.847, 134.007),
      (114.835, 143.178),
      (112.847, 138.562),
      (113.509, 141.619),
      name: "b"
    )
    bezier(
      "b.end",
      (122.063, 145.515),
      (116.16, 144.736),
      (118.569, 145.515),
      name: "b"
    )
    bezier(
      "b.end",
      (135.977, 140.121),
      (125.677, 145.515),
      (130.315, 143.717)
    )
    bezier(
      (139.591, 146.055),
      (113.389, 159.182),
      (128.99, 154.806),
      (120.256, 159.182),
      name: "b"
    )
    bezier(
      "b.end",
      (97.1258, 154.327),
      (106.522, 159.182),
      (101.101, 157.563),
      name: "b"
    )
    bezier(
      "b.end",
      (91.1626, 136.704),
      (93.1503, 150.97),
      (91.1626, 145.096),
      name: "b"
    )
    line(
      (rel: (0, -47.1126), to: "b.end"),
      (rel: (-9.0352, 0)),
      (80.6818, 82.9381),
      (91.1626, 79.7013),
      (rel: (0, -8.8112)),
      (112.847, 61),
      (rel: (0, 19.7802)),
      (134.17, 79.1618),
      (132.182, 90.8501),
      (112.847, 90.1309)
    )
  })

  // left pupil
  merge-path({
    bezier(
      (70.4667, 65.6833),
      (71.9727, 70.5068),
      (71.4946, 66.9075),
      (71.899, 69.4091)
    )
    bezier(
      (71.9727, 70.5068),
      (75.9104, 64.5912),
      (72.9675, 69.6715),
      (75.1477, 67.319)
    )
    bezier(
      (75.9104, 64.5912),
      (72.0556, 60.0005),
      (76.8638, 61.1815),
      (74.4045, 59.7677)
    )
    bezier(
      (72.0556, 60.0005),
      (66.833, 64.3859),
      (70.1766, 60.1867),
      (67.7909, 63.0017)
    )
    bezier(
      (66.833, 64.3859),
      (70.4667, 65.6833),
      (67.6159, 64.3083),
      (69.4388, 64.4591)
    )
  })

  // right pupil
  merge-path({
    bezier(
      (132.37, 61.668),
      (133.948, 66.7212),
      (133.447, 62.9505),
      (133.87, 65.5712)
    )
    bezier(
      (133.948, 66.7212),
      (138.073, 60.5239),
      (134.99, 65.8461),
      (137.274, 63.3815)
    )
    bezier(
      (138.073, 60.5239),
      (134.034, 55.7145),
      (139.066, 56.9513),
      (136.495, 55.4706)
    )
    bezier(
      (134.034, 55.7145),
      (128.563, 60.3087),
      (132.066, 55.9066),
      (129.567, 58.8586),
    )
    bezier(
      (128.563, 60.3087),
      (132.37, 61.668),
      (129.383, 60.2274),
      (131.293, 60.3855),
    )
  })

  set-style(
    stroke: (paint: rgb("239DAD"), thickness: 6pt, cap: "round"),
    fill: none,
  )

  // left eye
  merge-path({
    bezier(
      (58.5, 64.7273),
      (73.6136, 52),
      (58.5, 58.3636),
      (64.0682, 52.7955),
      name: "b"
    )
    bezier(
      "b.end",
      (84.75, 64.7273),
      (81.5682, 52),
      (84.75, 57.5682),
      name: "b"
    )
    bezier(
      "b.end",
      (71.2273, 76.6591),
      (84.75, 71.8864),
      (79.1818, 76.6591),
      name: "b"
    )
    bezier(
      "b.end",
      (58.5, 64.7273),
      (63.2727, 76.6591),
      (58.5, 71.0909)
    )
  })
  // eye lash
  line(
    (62.5, 55),
    (59.5, 52),
  )

  merge-path({
    bezier(
      (146.5, 61.043),
      (136.234, 49),
      (146.5, 52.7634),
      (141.367, 49)
    )
    bezier(
      (136.234, 49),
      (121.569, 62.5484),
      (125.969, 49),
      (120.836, 54.2688)
    )
    bezier(
      (121.569, 62.5484),
      (134.034, 72.3333),
      (122.302, 70.8279),
      (128.168, 72.3333)
    )
    bezier(
      (134.034, 72.3333),
      (146.5, 61.043),
      (139.901, 72.3333),
      (146.5, 69.3225)
    )
  })

  set-style(stroke: (thickness: 4pt))

  // right arm
  merge-path({
    bezier(
      (109.523, 115.614),
      (127.679, 110.918),
      (115.413, 115.3675),
      (122.283, 113.112)
    )
    bezier(
      (127.679, 110.918),
      (137, 106.591),
      (130.378, 109.821),
      (132.708, 108.739)
    )
  })

  // right first finger
  bezier(
    (137, 106.591),
    (140.5, 98.0908),
    (137.385, 102.891),
    (138.562, 99.817)
  )

  // right second finger
  bezier(
    (137, 106.591),
    (146, 101.591),
    (139.21, 103.799),
    (142.425, 101.713)
  )

  // right third finger
  line(
    (137, 106.591),
    (148, 106.591)
  )

  //right forth finger
  bezier(
    (137, 106.591),
    (146, 111.091),
    (140.243, 109.552),
    (143.119, 110.812)
  )

  // left arm
  bezier(
    (95.365, 116.979),
    (73.5, 107.591),
    (88.691, 115.549),
    (80.587, 112.887)
  )

  // left first finger
  line(
    (73.5, 107.591),
    (rel: (0, -9.5))
  )
  // left second finger
  line(
    (73.5, 107.591),
    (65.396, 100.824)
  )
  // left third finger
  line(
    (73.5, 107.591),
    (63.012, 105.839)
  )
  // left fourth finger
  bezier(
    (73.5, 107.591),
    (63.012, 111.04),
    (70.783, 109.121),
    (67.214, 111.255)
  )
})
\end{verbatim}

\pandocbounded{\includesvg[keepaspectratio]{typst-img/4a142b60394d5730a373a7ee2229a3a42a8af8f31b314c70b5bd192210982b09-1.svg}}


\title{sitandr.github.io/typst-examples-book/book/snippets/code}

\section{\texorpdfstring{\hyperref[code-formatting]{Code
formatting}}{Code formatting}}\label{code-formatting}

\subsection{\texorpdfstring{\hyperref[inline-highlighting]{Inline
highlighting}}{Inline highlighting}}\label{inline-highlighting}

\begin{verbatim}
#let r = raw.with(lang: "r")

This can then be used like: #r("x <- c(10, 42)")
\end{verbatim}

\pandocbounded{\includesvg[keepaspectratio]{typst-img/dadb41acb1c458d9af5b909d657de5f46dca019f1a81cc17b75b9863d60fa9eb-1.svg}}

\subsection{\texorpdfstring{\hyperref[tab-size]{Tab
size}}{Tab size}}\label{tab-size}

\begin{verbatim}
#set raw(tab-size: 8)
```tsv
Year  Month   Day
2000  2   3
2001  2   1
2002  3   10
```
\end{verbatim}

\pandocbounded{\includesvg[keepaspectratio]{typst-img/1c3900a79521f6b9cd852a68a3dea627ddbd1b8fc6062d3ca344e4259a30d212-1.svg}}

\subsection{\texorpdfstring{\hyperref[theme]{Theme}}{Theme}}\label{theme}

See
\href{https://typst.app/docs/reference/text/raw/\#parameters-theme}{reference}

\subsection{\texorpdfstring{\hyperref[enable-ligatures-for-code]{Enable
ligatures for
code}}{Enable ligatures for code}}\label{enable-ligatures-for-code}

\begin{verbatim}
#show raw: set text(ligatures: true, font: "Cascadia Code")

Then the code becomes `x <- a`
\end{verbatim}

\pandocbounded{\includesvg[keepaspectratio]{typst-img/3513eee2f5ca33825d09149f4ad9169abf95d3b8ad02cc5a7bf91cc9b96517d0-1.svg}}

\subsection{\texorpdfstring{\hyperref[advanced-formatting]{Advanced
formatting}}{Advanced formatting}}\label{advanced-formatting}

See \href{../packages/code.html}{packages} section.


\title{sitandr.github.io/typst-examples-book/book/snippets/pretty}

\section{\texorpdfstring{\hyperref[pretty-things]{Pretty
things}}{Pretty things}}\label{pretty-things}

\subsection{\texorpdfstring{\hyperref[set-bar-to-the-texts-left]{Set bar
to the text\textquotesingle s
left}}{Set bar to the text\textquotesingle s left}}\label{set-bar-to-the-texts-left}

(also known as quote formatting)

\begin{verbatim}
#let line-block = rect.with(fill: luma(240), stroke: (left: 0.25em))

+ #lorem(10) \
  #line-block[
    *Solution:* #lorem(10)

    $ a_(n+1)x^n = 2... $
  ]
\end{verbatim}

\pandocbounded{\includesvg[keepaspectratio]{typst-img/fcddd92f117eeeb99d7b422dfc0c20a254e163e09fc5b80251a088771792ff5a-1.svg}}

\subsection{\texorpdfstring{\hyperref[text-on-box-top]{Text on box
top}}{Text on box top}}\label{text-on-box-top}

\begin{verbatim}
// author: gaiajack
#let todo(body) = block(
  above: 2em, stroke: 0.5pt + red,
  width: 100%, inset: 14pt
)[
  #set text(font: "Noto Sans", fill: red)
  #place(
    top + left,
    dy: -6pt - 14pt, // Account for inset of block
    dx: 6pt - 14pt,
    block(fill: white, inset: 2pt)[*DRAFT*]
  )
  #body
]

#todo(lorem(100))
\end{verbatim}

\pandocbounded{\includesvg[keepaspectratio]{typst-img/7a5d79c63f3a0b28ec6bdec78da80d81252ff1975b883162c84b813f938c94c0-1.svg}}

\subsection{\texorpdfstring{\hyperref[book-ornament]{Book
Ornament}}{Book Ornament}}\label{book-ornament}

\begin{verbatim}
// author: thevec

#let parSepOrnament = [\ \ #h(1fr) $#line(start:(0em,-.15em), end:(12em,-.15em), stroke: (cap: "round", paint:gradient.linear(white,black,white))) #move(dx:.5em,dy:0em,"🙠")#text(15pt)[🙣] #h(0.4em) #move(dy:-0.25em,text(12pt)[✢]) #h(0.4em) #text(15pt)[🙡]#move(dx:-.5em,dy:0em,"🙢") #line(start:(0em,-.15em), end:(12em,-.15em), stroke: (cap: "round", paint:gradient.linear(white,black,white)))$ #h(1fr)\ \ ];

#lorem(30)
#parSepOrnament
#lorem(30)
\end{verbatim}

\pandocbounded{\includesvg[keepaspectratio]{typst-img/ad56a859952fab3706dcb76434e492a9c14057bff1ee897ae2bfe3672fe17e18-1.svg}}


