\title{typst.app/universe/package/rich-counters}

\phantomsection\label{banner}
\section{rich-counters}\label{rich-counters}

{ 0.2.2 }

Counters which can inherit from other counters.

\phantomsection\label{readme}
This package allows you to have \textbf{counters which can inherit from
other counters} .

Concretely, it implements \texttt{\ rich-counter\ } , which is a counter
that can \emph{inherit} one or more levels from another counter.

The interface is pretty much the same as the
\href{https://typst.app/docs/reference/introspection/counter/}{usual
counter} . It provides a \texttt{\ display()\ } , \texttt{\ get()\ } ,
\texttt{\ final()\ } , \texttt{\ at()\ } , and a \texttt{\ step()\ }
method. An \texttt{\ update()\ } method will be implemented soon.

\subsection{Simple typical Showcase}\label{simple-typical-showcase}

In the following example, \texttt{\ mycounter\ } inherits the first
level from \texttt{\ heading\ } (but not deeper levels).

\begin{Shaded}
\begin{Highlighting}[]
\NormalTok{\#import "@preview/rich{-}counters:0.2.2": *}

\NormalTok{\#set heading(numbering: "1.1")}
\NormalTok{\#let mycounter = rich{-}counter(identifier: "mycounter", inherited\_levels: 1)}

\NormalTok{// DOCUMENT}

\NormalTok{Displaying \textasciigrave{}mycounter\textasciigrave{} here: \#context (mycounter.display)()}

\NormalTok{= First level heading}

\NormalTok{Displaying \textasciigrave{}mycounter\textasciigrave{} here: \#context (mycounter.display)()}

\NormalTok{Stepping \textasciigrave{}mycounter\textasciigrave{} here. \#(mycounter.step)()}

\NormalTok{Displaying \textasciigrave{}mycounter\textasciigrave{} here: \#context (mycounter.display)()}

\NormalTok{= Another first level heading}

\NormalTok{Displaying \textasciigrave{}mycounter\textasciigrave{} here: \#context (mycounter.display)()}

\NormalTok{Stepping \textasciigrave{}mycounter\textasciigrave{} here. \#(mycounter.step)()}

\NormalTok{Displaying \textasciigrave{}mycounter\textasciigrave{} here: \#context (mycounter.display)()}

\NormalTok{== Second level heading}

\NormalTok{Displaying \textasciigrave{}mycounter\textasciigrave{} here: \#context (mycounter.display)()}

\NormalTok{Stepping \textasciigrave{}mycounter\textasciigrave{} here. \#(mycounter.step)()}

\NormalTok{Displaying \textasciigrave{}mycounter\textasciigrave{} here: \#context (mycounter.display)()}

\NormalTok{= Aaand another first level heading}

\NormalTok{Displaying \textasciigrave{}mycounter\textasciigrave{} here: \#context (mycounter.display)()}

\NormalTok{Stepping \textasciigrave{}mycounter\textasciigrave{} here. \#(mycounter.step)()}

\NormalTok{Displaying \textasciigrave{}mycounter\textasciigrave{} here: \#context (mycounter.display)()}
\end{Highlighting}
\end{Shaded}

\pandocbounded{\includegraphics[keepaspectratio]{https://github.com/typst/packages/raw/main/packages/preview/rich-counters/0.2.2/example.png}}

\subsection{\texorpdfstring{Construction of a
\texttt{\ rich-counter\ }}{Construction of a  rich-counter }}\label{construction-of-a-rich-counter}

To create a \texttt{\ rich-counter\ } , you have to call the
\texttt{\ rich-counter(...)\ } function. It accepts three arguments:

\begin{itemize}
\item
  \texttt{\ identifier\ } (required)

  Must be a unique \texttt{\ string\ } which identifies the counter.
\item
  \texttt{\ inherited\_levels\ }

  Specifies how many levels should be inherited from the parent counter.
\item
  \texttt{\ inherited\_from\ } (Default: \texttt{\ heading\ } )

  Specifies the parent counter. Can be a \texttt{\ rich-counter\ } or
  any key that is accepted by the
  \href{https://typst.app/docs/reference/introspection/counter\#constructor}{\texttt{\ counter(...)\ }
  constructor} , such as a \texttt{\ label\ } , a \texttt{\ selector\ }
  , a \texttt{\ location\ } , or a \texttt{\ function\ } like
  \texttt{\ heading\ } . If not specified, defaults to
  \texttt{\ heading\ } (and hence it will inherit from the counter of
  the headings).

  If it’s a \texttt{\ rich-counter\ } and
  \texttt{\ inherited\_levels\ } is \emph{not} specified, then
  \texttt{\ inherited\_levels\ } will default to one level higher than
  the given \texttt{\ rich-counter\ } .
\end{itemize}

For example, the following creates a \texttt{\ rich-counter\ }
\texttt{\ foo\ } which inherits one level from the headings, and then
another \texttt{\ rich-counter\ } \texttt{\ bar\ } which inherits two
levels (implicitly) from \texttt{\ foo\ } .

\begin{Shaded}
\begin{Highlighting}[]
\NormalTok{\#import "@preview/rich{-}counters:0.2.2": *}

\NormalTok{\#let foo = rich{-}counter(identifier: "foo", inherited\_levels: 1)}
\NormalTok{\#let bar = rich{-}counter(identifier: "bar", inherited\_from: foo)}
\end{Highlighting}
\end{Shaded}

\subsection{\texorpdfstring{Usage of a
\texttt{\ rich-counter\ }}{Usage of a  rich-counter }}\label{usage-of-a-rich-counter}

\begin{itemize}
\item
  \texttt{\ display(numbering)\ } (needs \texttt{\ context\ } )

  Displays the current value of the counter with the given numbering
  style. Giving the numbering style is optional, with default value
  \texttt{\ "1.1"\ } .
\item
  \texttt{\ get()\ } (needs \texttt{\ context\ } )

  Returns the current value of the counter (as an \texttt{\ array\ } ).
\item
  \texttt{\ final()\ } (needs \texttt{\ context\ } )

  Returns the value of the counter at the end of the document.
\item
  \texttt{\ at(loc)\ } (needs \texttt{\ context\ } )

  Returns the value of the counter at \texttt{\ loc\ } , where
  \texttt{\ loc\ } can be a \texttt{\ label\ } , \texttt{\ selector\ } ,
  \texttt{\ location\ } , or \texttt{\ function\ } .
\item
  \texttt{\ step(depth:\ 1)\ }

  Steps the counter at the specified \texttt{\ depth\ } (default:
  \texttt{\ 1\ } ). That is, it steps the \texttt{\ rich-counter\ } at
  level \texttt{\ inherited\_levels\ +\ depth\ } .
\end{itemize}

\textbf{Due to a Typst limitation, you have to put parentheses before
you put the arguments. (See below.)}

For example, the following steps \texttt{\ mycounter\ } (at depth 1) and
then displays it.

\begin{Shaded}
\begin{Highlighting}[]
\NormalTok{\#import "@preview/rich{-}counters:0.2.2": *}
\NormalTok{\#let mycounter = rich{-}counter(...)}

\NormalTok{\#(mycounter.step)()}
\NormalTok{\#context (mycounter.display)("1.1")}
\end{Highlighting}
\end{Shaded}

\subsection{Limitations}\label{limitations}

Due to current Typst limitations, there is no way to detect manual
updates or steps of Typst-native counters, like
\texttt{\ counter(heading).update(...)\ } or
\texttt{\ counter(heading).step(...)\ } . Only occurrences of actual
\texttt{\ heading\ } s can be detected. So make sure that after you call
e.g. \texttt{\ counter(heading).update(...)\ } , you place a heading
directly after it, before you call any \texttt{\ rich-counter\ } s.

\subsection{Roadmap}\label{roadmap}

\begin{itemize}
\tightlist
\item
  implement \texttt{\ update()\ }
\item
  use Typst custom types as soon as they become available
\item
  adopt native Typst implementation of dependent counters as soon it
  becomes available
\end{itemize}

\subsubsection{How to add}\label{how-to-add}

Copy this into your project and use the import as
\texttt{\ rich-counters\ }

\begin{verbatim}
#import "@preview/rich-counters:0.2.2"
\end{verbatim}

\includesvg[width=0.16667in,height=0.16667in]{/assets/icons/16-copy.svg}

Check the docs for
\href{https://typst.app/docs/reference/scripting/\#packages}{more
information on how to import packages} .

\subsubsection{About}\label{about}

\begin{description}
\tightlist
\item[Author :]
\href{https://jbirnick.net}{Johann Birnick}
\item[License:]
MIT
\item[Current version:]
0.2.2
\item[Last updated:]
November 21, 2024
\item[First released:]
August 14, 2024
\item[Archive size:]
3.60 kB
\href{https://packages.typst.org/preview/rich-counters-0.2.2.tar.gz}{\pandocbounded{\includesvg[keepaspectratio]{/assets/icons/16-download.svg}}}
\item[Repository:]
\href{https://github.com/jbirnick/typst-rich-counters}{GitHub}
\item[Categor ies :]
\begin{itemize}
\tightlist
\item[]
\item
  \pandocbounded{\includesvg[keepaspectratio]{/assets/icons/16-list-unordered.svg}}
  \href{https://typst.app/universe/search/?category=model}{Model}
\item
  \pandocbounded{\includesvg[keepaspectratio]{/assets/icons/16-code.svg}}
  \href{https://typst.app/universe/search/?category=scripting}{Scripting}
\item
  \pandocbounded{\includesvg[keepaspectratio]{/assets/icons/16-hammer.svg}}
  \href{https://typst.app/universe/search/?category=utility}{Utility}
\end{itemize}
\end{description}

\subsubsection{Where to report issues?}\label{where-to-report-issues}

This package is a project of Johann Birnick . Report issues on
\href{https://github.com/jbirnick/typst-rich-counters}{their repository}
. You can also try to ask for help with this package on the
\href{https://forum.typst.app}{Forum} .

Please report this package to the Typst team using the
\href{https://typst.app/contact}{contact form} if you believe it is a
safety hazard or infringes upon your rights.

\phantomsection\label{versions}
\subsubsection{Version history}\label{version-history}

\begin{longtable}[]{@{}ll@{}}
\toprule\noalign{}
Version & Release Date \\
\midrule\noalign{}
\endhead
\bottomrule\noalign{}
\endlastfoot
0.2.2 & November 21, 2024 \\
\href{https://typst.app/universe/package/rich-counters/0.2.1/}{0.2.1} &
October 16, 2024 \\
\href{https://typst.app/universe/package/rich-counters/0.2.0/}{0.2.0} &
October 14, 2024 \\
\href{https://typst.app/universe/package/rich-counters/0.1.0/}{0.1.0} &
August 14, 2024 \\
\end{longtable}

Typst GmbH did not create this package and cannot guarantee correct
functionality of this package or compatibility with any version of the
Typst compiler or app.
