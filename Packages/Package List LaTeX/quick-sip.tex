\title{typst.app/universe/package/quick-sip}

\phantomsection\label{banner}
\phantomsection\label{template-thumbnail}
\pandocbounded{\includegraphics[keepaspectratio]{https://packages.typst.org/preview/thumbnails/quick-sip-0.1.0-small.webp}}

\section{quick-sip}\label{quick-sip}

{ 0.1.0 }

A template for creating quick reference handbook style checklists.

\href{/app?template=quick-sip&version=0.1.0}{Create project in app}

\phantomsection\label{readme}
Creates aviation style checklists like Quick Reference Handbooks.

\includegraphics[width=3.125in,height=\textheight,keepaspectratio]{https://github.com/typst/packages/raw/main/packages/preview/quick-sip/0.1.0/thumbnail.png}

\subsubsection{Features:}\label{features}

\begin{itemize}
\tightlist
\item
  Index
\item
  Section
\item
  Conditions
\item
  Objective
\item
  Step (When/If)
\item
  Sub Step
\item
  Caution
\item
  Note
\item
  Choose One
\item
  Go to step
\item
  End section now
\end{itemize}

\subsection{Start with}\label{start-with}

\begin{Shaded}
\begin{Highlighting}[]
\NormalTok{\#import "@preview/quick{-}sip:0.1.0": *}
\NormalTok{\#show: QRH.with(title: "Cup of Tea")}
\end{Highlighting}
\end{Shaded}

Then add a section:

\begin{Shaded}
\begin{Highlighting}[]
\NormalTok{\#section("Cup of Tea preparation")[}
\NormalTok{  \#step("KETTLE", "Filled to 1 CUP")}
\NormalTok{  \#step([*When* KETTLE boiled:], "")}
\NormalTok{  \#step([*If* sugar required], "")}
\NormalTok{    //.. Rest of section goes here}
\NormalTok{]}
\end{Highlighting}
\end{Shaded}

\paragraph{Index}\label{index}

An index with an entry for each section in the document.

\begin{Shaded}
\begin{Highlighting}[]
\NormalTok{\#index()}
\end{Highlighting}
\end{Shaded}

\paragraph{Section}\label{section}

A section title, forces capitalisation.

\begin{Shaded}
\begin{Highlighting}[]
\NormalTok{\#section("Cup of Tea preparation")[}
\NormalTok{    //.. Rest of section goes here}
\NormalTok{]}
\end{Highlighting}
\end{Shaded}

\paragraph{Conditions}\label{conditions}

Conditionals for this section.

\begin{Shaded}
\begin{Highlighting}[]
\NormalTok{\#condition[}
\NormalTok{    {-} Dehydration}
\NormalTok{    {-} Fatigue}
\NormalTok{    {-} Inability to Concentrate}
\NormalTok{]}
\end{Highlighting}
\end{Shaded}

\paragraph{Objective}\label{objective}

An objective for this section (optional).

\begin{Shaded}
\begin{Highlighting}[]
\NormalTok{\#objective[To replenish fluids.]}
\end{Highlighting}
\end{Shaded}

\paragraph{Step}\label{step}

A numbered step in the checklist. The first parameter is to the left of
the dotted line, the second is to the right. If the second parameter is
\texttt{\ ""\ } then there is no dotted line.

\begin{Shaded}
\begin{Highlighting}[]
\NormalTok{\#step("KETTLE", "Filled to 1 CUP")}
\NormalTok{\#step([*When* KETTLE boiled:], "")}
\NormalTok{\#step([*If* sugar required], "")}
\end{Highlighting}
\end{Shaded}

\paragraph{Tab}\label{tab}

Indents contents by one tab.

\begin{Shaded}
\begin{Highlighting}[]
\NormalTok{\#tab(goto("9"))}
\NormalTok{\#tab(tab("Large mugs may require more water."))}
\end{Highlighting}
\end{Shaded}

\paragraph{Caution}\label{caution}

Adds a caution element.

\begin{Shaded}
\begin{Highlighting}[]
\NormalTok{\#caution([HOT WATER \#linebreak()Adult supervision required.])}
\end{Highlighting}
\end{Shaded}

\paragraph{Note}\label{note}

Adds a note.

\begin{Shaded}
\begin{Highlighting}[]
\NormalTok{\#note("Stir after each step")}
\end{Highlighting}
\end{Shaded}

\paragraph{Choose One}\label{choose-one}

A numbered step with options.

\begin{Shaded}
\begin{Highlighting}[]
\NormalTok{ \#choose{-}one[}
\NormalTok{    \#option[Black tea *required:*]}
\NormalTok{    \#option[Tea with MILK *required:*]}
\NormalTok{  ]}
\end{Highlighting}
\end{Shaded}

\paragraph{Go to step}\label{go-to-step}

Two right facing arrow heads followed by Go to step
\texttt{\ step\ number\ } . Links to step in pdf.

\begin{Shaded}
\begin{Highlighting}[]
\NormalTok{\#goto("9")}
\end{Highlighting}
\end{Shaded}

\paragraph{End}\label{end}

Ends the section here with 4 dots.

\begin{Shaded}
\begin{Highlighting}[]
\NormalTok{\#end()}
\end{Highlighting}
\end{Shaded}

\paragraph{Wait}\label{wait}

Long small dotted line for waiting for a task to complete.

\begin{Shaded}
\begin{Highlighting}[]
\NormalTok{\#wait()}
\end{Highlighting}
\end{Shaded}

\href{/app?template=quick-sip&version=0.1.0}{Create project in app}

\subsubsection{How to use}\label{how-to-use}

Click the button above to create a new project using this template in
the Typst app.

You can also use the Typst CLI to start a new project on your computer
using this command:

\begin{verbatim}
typst init @preview/quick-sip:0.1.0
\end{verbatim}

\includesvg[width=0.16667in,height=0.16667in]{/assets/icons/16-copy.svg}

\subsubsection{About}\label{about}

\begin{description}
\tightlist
\item[Author :]
\href{https://github.com/artomweb}{Archie Webster}
\item[License:]
MIT
\item[Current version:]
0.1.0
\item[Last updated:]
October 16, 2024
\item[First released:]
October 16, 2024
\item[Archive size:]
4.28 kB
\href{https://packages.typst.org/preview/quick-sip-0.1.0.tar.gz}{\pandocbounded{\includesvg[keepaspectratio]{/assets/icons/16-download.svg}}}
\item[Repository:]
\href{https://github.com/artomweb/Quick-Sip-Typst-Template}{GitHub}
\item[Categor y :]
\begin{itemize}
\tightlist
\item[]
\item
  \pandocbounded{\includesvg[keepaspectratio]{/assets/icons/16-hammer.svg}}
  \href{https://typst.app/universe/search/?category=utility}{Utility}
\end{itemize}
\end{description}

\subsubsection{Where to report issues?}\label{where-to-report-issues}

This template is a project of Archie Webster . Report issues on
\href{https://github.com/artomweb/Quick-Sip-Typst-Template}{their
repository} . You can also try to ask for help with this template on the
\href{https://forum.typst.app}{Forum} .

Please report this template to the Typst team using the
\href{https://typst.app/contact}{contact form} if you believe it is a
safety hazard or infringes upon your rights.

\phantomsection\label{versions}
\subsubsection{Version history}\label{version-history}

\begin{longtable}[]{@{}ll@{}}
\toprule\noalign{}
Version & Release Date \\
\midrule\noalign{}
\endhead
\bottomrule\noalign{}
\endlastfoot
0.1.0 & October 16, 2024 \\
\end{longtable}

Typst GmbH did not create this template and cannot guarantee correct
functionality of this template or compatibility with any version of the
Typst compiler or app.
