\title{typst.app/universe/package/modernpro-cv}

\phantomsection\label{banner}
\phantomsection\label{template-thumbnail}
\pandocbounded{\includegraphics[keepaspectratio]{https://packages.typst.org/preview/thumbnails/modernpro-cv-1.0.2-small.webp}}

\section{modernpro-cv}\label{modernpro-cv}

{ 1.0.2 }

A CV template inspired by Deedy-Resume.

\href{/app?template=modernpro-cv&version=1.0.2}{Create project in app}

\phantomsection\label{readme}
This Typst CV template is inspired by the Latex template
\href{https://github.com/deedy/Deedy-Resume}{Deedy-Resume} . You can use
it for both industry and academia.

If you want to find a cover letter template, you can check out
\href{https://github.com/jxpeng98/typst-coverletter}{modernpro-coverletter}
.

\subsection{How to start}\label{how-to-start}

\subsubsection{Use Typst CLI}\label{use-typst-cli}

If you use Typst CLI, you can use the following command to create a new
project:

\begin{Shaded}
\begin{Highlighting}[]
\ExtensionTok{typst}\NormalTok{ init modernpro{-}cv}
\end{Highlighting}
\end{Shaded}

It will create a folder named \texttt{\ modernpro-cv\ } with the
following structure:

\begin{Shaded}
\begin{Highlighting}[]
\NormalTok{modernpro{-}cv}
\NormalTok{├── bib.bib}
\NormalTok{├── cv\_double.typ}
\NormalTok{└── cv\_single.typ}
\end{Highlighting}
\end{Shaded}

If you want to use the single-column version, you can modify the
template \texttt{\ cv-single.typ\ } . If you prefer the two-column
version, you can use the \texttt{\ cv-double.typ\ } .

\textbf{Note:} The \texttt{\ bib.bib\ } is the bibliography file. You
can modify it to add your publications.

\subsubsection{Manual Download}\label{manual-download}

If you want to manually download the template, you can download
\texttt{\ modernpro-cv-\{version\}.zip\ } from the
\href{https://github.com/jxpeng98/Typst-CV-Resume/releases}{release
page}

\subsubsection{Typst website}\label{typst-website}

If you want to use the template via \href{https://typst.app/}{Typst} ,
You can \texttt{\ start\ from\ template\ } and search for
\texttt{\ modernpro-cv\ } .

\subsection{How to use the template}\label{how-to-use-the-template}

\subsubsection{The arguments}\label{the-arguments}

The template has the following arguments:

\begin{longtable}[]{@{}lll@{}}
\toprule\noalign{}
Argument & Description & Default \\
\midrule\noalign{}
\endhead
\bottomrule\noalign{}
\endlastfoot
\texttt{\ font-type\ } & The font type. You can choose any supported
font in your system. & \texttt{\ Times\ New\ Roman\ } \\
\texttt{\ continue-header\ } & Whether to continue the header on the
follwing pages. & \texttt{\ false\ } \\
\texttt{\ name\ } & Your name. & \texttt{\ ""\ } \\
\texttt{\ address\ } & Your address. & \texttt{\ ""\ } \\
\texttt{\ lastupdated\ } & Whether to show the last updated date. &
\texttt{\ true\ } \\
\texttt{\ pagecount\ } & Whether to show the page count. &
\texttt{\ true\ } \\
\texttt{\ date\ } & The date of the CV. & \texttt{\ today\ } \\
\texttt{\ contacts\ } & contact details, e.g phone number, email, etc. &
\texttt{\ (text:\ "",\ link:\ "")\ } \\
\end{longtable}

\subsubsection{Start single column
version}\label{start-single-column-version}

If you want to use the single column version, you create a new
\texttt{\ .typ\ } file and copy the following code:

\begin{Shaded}
\begin{Highlighting}[]
\NormalTok{\#import "@preview/modernpro{-}cv:1.0.2": *}
\NormalTok{\#import "@preview/fontawesome:0.5.0": *}

\NormalTok{\#show: cv{-}single.with(}
\NormalTok{  font{-}type: "PT Serif",}
\NormalTok{  continue{-}header: "false",}
\NormalTok{  name: [],}
\NormalTok{  address: [],}
\NormalTok{  lastupdated: "true",}
\NormalTok{  pagecount: "true",}
\NormalTok{  date: "2024{-}07{-}03",}
\NormalTok{  contacts: (}
\NormalTok{    (text: [\#fa{-}icon("location{-}dot") UK]),}
\NormalTok{    (text: [\#fa{-}icon("mobile") 123{-}456{-}789], link: "tel:123{-}456{-}789"),}
\NormalTok{    (text: [\#fa{-}icon("link") example.com], link: "https://www.example.com"),}
\NormalTok{  )}
\NormalTok{)}
\end{Highlighting}
\end{Shaded}

\subsubsection{Start double column
version}\label{start-double-column-version}

The double column version is similar to the single column version.
However, you need to add contents to the specific \texttt{\ left\ } and
\texttt{\ right\ } sections.

\begin{Shaded}
\begin{Highlighting}[]
\NormalTok{\#import "@preview/modernpro{-}cv:1.0.2": *}
\NormalTok{\#import "@preview/fontawesome:0.5.0": *}

\NormalTok{\#show: cv{-}double(}
\NormalTok{  font{-}type: "PT Sans",}
\NormalTok{  continue{-}header: "true",}
\NormalTok{  name: [\#lorem(2)],}
\NormalTok{  address: [\#lorem(4)],}
\NormalTok{  lastupdated: "true",}
\NormalTok{  pagecount: "true",}
\NormalTok{  date: "2024{-}07{-}03",}
\NormalTok{  contacts: (}
\NormalTok{    (text: [\#fa{-}icon("location{-}dot") UK]),}
\NormalTok{    (text: [\#fa{-}icon("mobile") 123{-}456{-}789], link: "tel:123{-}456{-}789"),}
\NormalTok{    (text: [\#fa{-}icon("link") example.com], link: "https://www.example.com"),}
\NormalTok{  ),}
\NormalTok{  left: [}
\NormalTok{    // contents for the left column}
\NormalTok{  ],}
\NormalTok{  right:[}
\NormalTok{    // contents for the right column}
\NormalTok{  ]}
\NormalTok{)}
\end{Highlighting}
\end{Shaded}

\subsubsection{Start the CV}\label{start-the-cv}

Once you set up the arguments, you can start to add details to your CV /
Resume.

I preset the following functions for you to create different parts:

\begin{longtable}[]{@{}ll@{}}
\toprule\noalign{}
Function & Description \\
\midrule\noalign{}
\endhead
\bottomrule\noalign{}
\endlastfoot
\texttt{\ \#section("Section\ Name")\ } & Start a new section \\
\texttt{\ \#sectionsep\ } & End the section \\
\texttt{\ \#oneline-title-item(title:\ "",\ content:\ "")\ } & Add a
one-line item ( \textbf{Title:} content) \\
\texttt{\ \#oneline-two(entry1:\ "",\ entry2:\ "")\ } & Add a one-line
item with two entries, aligned left and right \\
\texttt{\ \#descript("descriptions")\ } & Add a description for
self-introduction \\
\texttt{\ \#award(award:\ "",\ date:\ "",\ institution:\ "")\ } & Add an
award ( \textbf{award} , \emph{institution} \emph{date} ) \\
\texttt{\ \#education(institution:\ "",\ major:\ "",\ date:\ "",\ institution:\ "",\ core-modules:\ "")\ }
& Add an education experience \\
\texttt{\ \#job(position:\ "",\ institution:\ "",\ location:\ "",\ date:\ "",\ description:\ {[}{]})\ }
& Add a job experience (description is optional) \\
\texttt{\ \#twoline-item(entry1:\ "",\ entry2:\ "",\ entry3:\ "",\ entry4:\ "")\ }
& Two line items, similar to education and job experiences \\
\texttt{\ \#references(references:())\ } & Add a reference list. In the
\texttt{\ ()\ } , you can add multi reference entries with the following
format
\texttt{\ (name:\ "",\ position:\ "",\ department:\ "",\ institution:\ "",\ address:\ "",\ email:\ "",),\ } \\
\texttt{\ \#show\ bibliography:\ none\ \#bibliography("bib.bib")\ } &
Add a bibliography. You can modify the \texttt{\ bib.bib\ } file to add
your publications. \textbf{Note:} Keep this at the end of your CV \\
\end{longtable}

\textbf{Note:} Use \texttt{\ +\ @ref\ } to display your publications.
For example,

\begin{Shaded}
\begin{Highlighting}[]
\NormalTok{\#section("Publications")}

\NormalTok{// numbering list }
\NormalTok{+ @quenouille1949approximate}
\NormalTok{+ @quenouille1949approximate}

\NormalTok{// Keep this at the end}
\NormalTok{\#show bibliography: none}
\NormalTok{\#bibliography("bib.bib")}
\end{Highlighting}
\end{Shaded}

\subsection{Preview}\label{preview}

\subsubsection{Single Column}\label{single-column}

\pandocbounded{\includegraphics[keepaspectratio]{https://minioapi.pjx.ac.cn/img1/2024/07/a81ac7ec96be0625eefccb81ead160d3.png}}

\subsubsection{Double Column}\label{double-column}

\pandocbounded{\includegraphics[keepaspectratio]{https://minioapi.pjx.ac.cn/img1/2024/07/12e9b31e306055f615edf49f9b8ffe55.png}}

\subsection{Legacy Version}\label{legacy-version}

I redesigned the template and submitted the new version to Typst
Universe. However, you can find the legacy version in the
\texttt{\ legacy\ } folder if you prefer to use the multi-font setting.
You can also download the \texttt{\ modernpro-cv-legacy.zip\ } from the
\href{https://github.com/jxpeng98/Typst-CV-Resume/releases}{release
page} .

\textbf{Note:} The legacy version also has a cover letter template. You
can use it with the CV template.

\subsection{Cover Letter}\label{cover-letter}

If you used the previous version of this template, you might know that I
also provided a cover letter template.

If you want to use a consistent cover letter with the new version of the
CV template, you can find it from another repository
\href{https://github.com/jxpeng98/typst-coverletter}{typst-coverletter}
.

you can also use the following code in the command line:

\begin{Shaded}
\begin{Highlighting}[]
\ExtensionTok{typst}\NormalTok{ init modernpro{-}coverletter}
\end{Highlighting}
\end{Shaded}

\subsection{License}\label{license}

The template is released under the MIT License. For more information,
please refer to the
\href{https://github.com/jxpeng98/Typst-CV-Resume/blob/main/LICENSE}{LICENSE}
file.

\href{/app?template=modernpro-cv&version=1.0.2}{Create project in app}

\subsubsection{How to use}\label{how-to-use}

Click the button above to create a new project using this template in
the Typst app.

You can also use the Typst CLI to start a new project on your computer
using this command:

\begin{verbatim}
typst init @preview/modernpro-cv:1.0.2
\end{verbatim}

\includesvg[width=0.16667in,height=0.16667in]{/assets/icons/16-copy.svg}

\subsubsection{About}\label{about}

\begin{description}
\tightlist
\item[Author :]
jxpeng98
\item[License:]
MIT
\item[Current version:]
1.0.2
\item[Last updated:]
October 22, 2024
\item[First released:]
August 7, 2024
\item[Archive size:]
5.70 kB
\href{https://packages.typst.org/preview/modernpro-cv-1.0.2.tar.gz}{\pandocbounded{\includesvg[keepaspectratio]{/assets/icons/16-download.svg}}}
\item[Repository:]
\href{https://github.com/jxpeng98/Typst-CV-Resume}{GitHub}
\item[Categor y :]
\begin{itemize}
\tightlist
\item[]
\item
  \pandocbounded{\includesvg[keepaspectratio]{/assets/icons/16-user.svg}}
  \href{https://typst.app/universe/search/?category=cv}{CV}
\end{itemize}
\end{description}

\subsubsection{Where to report issues?}\label{where-to-report-issues}

This template is a project of jxpeng98 . Report issues on
\href{https://github.com/jxpeng98/Typst-CV-Resume}{their repository} .
You can also try to ask for help with this template on the
\href{https://forum.typst.app}{Forum} .

Please report this template to the Typst team using the
\href{https://typst.app/contact}{contact form} if you believe it is a
safety hazard or infringes upon your rights.

\phantomsection\label{versions}
\subsubsection{Version history}\label{version-history}

\begin{longtable}[]{@{}ll@{}}
\toprule\noalign{}
Version & Release Date \\
\midrule\noalign{}
\endhead
\bottomrule\noalign{}
\endlastfoot
1.0.2 & October 22, 2024 \\
\href{https://typst.app/universe/package/modernpro-cv/1.0.1/}{1.0.1} &
August 30, 2024 \\
\href{https://typst.app/universe/package/modernpro-cv/1.0.0/}{1.0.0} &
August 7, 2024 \\
\end{longtable}

Typst GmbH did not create this template and cannot guarantee correct
functionality of this template or compatibility with any version of the
Typst compiler or app.
