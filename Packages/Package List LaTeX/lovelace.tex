\title{typst.app/universe/package/lovelace}

\phantomsection\label{banner}
\section{lovelace}\label{lovelace}

{ 0.3.0 }

Algorithms in pseudocode, unopinionated and flexible

{ } Featured Package

\phantomsection\label{readme}
This is a package for writing pseudocode in
\href{https://typst.app/}{Typst} . It is named after the computer
science pioneer \href{https://en.wikipedia.org/wiki/Ada_Lovelace}{Ada
Lovelace} and inspired by the \href{https://ctan.org/pkg/pseudo}{pseudo
package} for LaTeX.

\pandocbounded{\includegraphics[keepaspectratio]{https://img.shields.io/github/license/andreasKroepelin/lovelace}}
\pandocbounded{\includegraphics[keepaspectratio]{https://img.shields.io/github/v/release/andreasKroepelin/lovelace}}
\pandocbounded{\includegraphics[keepaspectratio]{https://img.shields.io/github/stars/andreasKroepelin/lovelace}}

Pseudocode is not a programming language, it doesn’t have strict
syntax, so you should be able to write it however you need to in your
specific situation. Lovelace lets you do exactly that.

Main features include:

\begin{itemize}
\tightlist
\item
  arbitrary keywords and syntax structures
\item
  optional line numbering
\item
  line labels
\item
  lots of customisation with sensible defaults
\end{itemize}

\subsection{Usage}\label{usage}

\begin{itemize}
\tightlist
\item
  \href{https://github.com/typst/packages/raw/main/packages/preview/lovelace/0.3.0/\#getting-started}{Getting
  started}
\item
  \href{https://github.com/typst/packages/raw/main/packages/preview/lovelace/0.3.0/\#lower-level-interface}{Lower
  level interface}
\item
  \href{https://github.com/typst/packages/raw/main/packages/preview/lovelace/0.3.0/\#line-numbers}{Line
  numbers}
\item
  \href{https://github.com/typst/packages/raw/main/packages/preview/lovelace/0.3.0/\#referencing-lines}{Referencing
  lines}
\item
  \href{https://github.com/typst/packages/raw/main/packages/preview/lovelace/0.3.0/\#indentation-guides}{Indentation
  guides}
\item
  \href{https://github.com/typst/packages/raw/main/packages/preview/lovelace/0.3.0/\#spacing}{Spacing}
\item
  \href{https://github.com/typst/packages/raw/main/packages/preview/lovelace/0.3.0/\#decorations}{Decorations}
\item
  \href{https://github.com/typst/packages/raw/main/packages/preview/lovelace/0.3.0/\#algorithm-as-figure}{Algorithm
  as figure}
\item
  \href{https://github.com/typst/packages/raw/main/packages/preview/lovelace/0.3.0/\#customisation-overview}{Customisation
  overview}
\item
  \href{https://github.com/typst/packages/raw/main/packages/preview/lovelace/0.3.0/\#exported-functions}{Exported
  functions}
\end{itemize}

\subsubsection{Getting started}\label{getting-started}

Import the package using

\begin{Shaded}
\begin{Highlighting}[]
\NormalTok{\#import "@preview/lovelace:0.3.0": *}
\end{Highlighting}
\end{Shaded}

The simplest usage is via \texttt{\ pseudocode-list\ } which transforms
a nested list into pseudocode:

\begin{Shaded}
\begin{Highlighting}[]
\NormalTok{\#pseudocode{-}list[}
\NormalTok{  + do something}
\NormalTok{  + do something else}
\NormalTok{  + *while* still something to do}
\NormalTok{    + do even more}
\NormalTok{    + *if* not done yet *then*}
\NormalTok{      + wait a bit}
\NormalTok{      + resume working}
\NormalTok{    + *else*}
\NormalTok{      + go home}
\NormalTok{    + *end*}
\NormalTok{  + *end*}
\NormalTok{]}
\end{Highlighting}
\end{Shaded}

resulting in:

\pandocbounded{\includesvg[keepaspectratio]{https://github.com/typst/packages/raw/main/packages/preview/lovelace/0.3.0/examples/simple.svg}}

As you can see, every list item becomes one line of code and nested
lists become indented blocks. There are no special commands for common
keywords and control structures, you just use whatever you like.

Maybe in your domain very uncommon structures make more sense? No
problem!

\begin{Shaded}
\begin{Highlighting}[]
\NormalTok{\#pseudocode{-}list[}
\NormalTok{  + *in parallel for each* $i = 1, ..., n$ *do*}
\NormalTok{    + fetch chunk of data $i$}
\NormalTok{    + *with probability* $exp({-}epsilon\_i slash k T)$ *do*}
\NormalTok{      + perform update}
\NormalTok{    + *end*}
\NormalTok{  + *end*}
\NormalTok{]}
\end{Highlighting}
\end{Shaded}

\pandocbounded{\includesvg[keepaspectratio]{https://github.com/typst/packages/raw/main/packages/preview/lovelace/0.3.0/examples/custom.svg}}

\subsubsection{Lower level interface}\label{lower-level-interface}

If you feel uncomfortable with abusing Typst’s lists like we do here,
you can also use the \texttt{\ pseudocode\ } function directly:

\begin{Shaded}
\begin{Highlighting}[]
\NormalTok{\#pseudocode(}
\NormalTok{  [do something],}
\NormalTok{  [do something else],}
\NormalTok{  [*while* still something to do],}
\NormalTok{  indent(}
\NormalTok{    [do even more],}
\NormalTok{    [*if* not done yet *then*],}
\NormalTok{    indent(}
\NormalTok{      [wait a bit],}
\NormalTok{      [resume working],}
\NormalTok{    ),}
\NormalTok{    [*else*],}
\NormalTok{    indent(}
\NormalTok{      [go home],}
\NormalTok{    ),}
\NormalTok{    [*end*],}
\NormalTok{  ),}
\NormalTok{  [*end*],}
\NormalTok{)}
\end{Highlighting}
\end{Shaded}

This is equivalent to the first example. Note that each line is given as
one content argument and you indent a block by using the
\texttt{\ indent\ } function.

This approach has the advantage that you do not rely on significant
whitespace and code formatters can automatically correctly indent your
Typst code.

\subsubsection{Line numbers}\label{line-numbers}

Lovelace puts a number in front of each line by default. If you want no
numbers at all, you can set the \texttt{\ line-numbering\ } option to
\texttt{\ none\ } . The initial example then looks like this:

\begin{Shaded}
\begin{Highlighting}[]
\NormalTok{\#pseudocode{-}list(line{-}numbering: none)[}
\NormalTok{  + do something}
\NormalTok{  + do something else}
\NormalTok{  + *while* still something to do}
\NormalTok{    + do even more}
\NormalTok{    + *if* not done yet *then*}
\NormalTok{      + wait a bit}
\NormalTok{      + resume working}
\NormalTok{    + *else*}
\NormalTok{      + go home}
\NormalTok{    + *end*}
\NormalTok{  + *end*}
\NormalTok{]}
\end{Highlighting}
\end{Shaded}

\pandocbounded{\includesvg[keepaspectratio]{https://github.com/typst/packages/raw/main/packages/preview/lovelace/0.3.0/examples/simple-no-numbers.svg}}

(You can also pass this keyword argument to \texttt{\ pseudocode\ } .)

If you do want line numbers in general but need to turn them off for
specific lines, you can use \texttt{\ -\ } items instead of
\texttt{\ +\ } items in \texttt{\ pseudocode-list\ } :

\begin{Shaded}
\begin{Highlighting}[]
\NormalTok{\#pseudocode{-}list[}
\NormalTok{  + normal line with a number}
\NormalTok{  {-} this line has no number}
\NormalTok{  + this one has a number again}
\NormalTok{]}
\end{Highlighting}
\end{Shaded}

\pandocbounded{\includesvg[keepaspectratio]{https://github.com/typst/packages/raw/main/packages/preview/lovelace/0.3.0/examples/number-no-number.svg}}

It’s easy to remember: \texttt{\ -\ } items usually produce unnumbered
lists and \texttt{\ +\ } items produce numbered lists!

When using the \texttt{\ pseudocode\ } function, you can achieve the
same using \texttt{\ no-number\ } :

\begin{Shaded}
\begin{Highlighting}[]
\NormalTok{\#pseudocode(}
\NormalTok{  [normal line with a number],}
\NormalTok{  no{-}number[this line has no number],}
\NormalTok{  [this one has a number again],}
\NormalTok{)}
\end{Highlighting}
\end{Shaded}

\paragraph{More line number
customisation}\label{more-line-number-customisation}

Other than \texttt{\ none\ } , you can assign anything listed
\href{https://typst.app/docs/reference/model/numbering/\#parameters-numbering}{here}
to \texttt{\ line-numbering\ } .

So maybe you happen to think about the Roman Empire a lot and want to
reflect that in your pseudocode?

\begin{Shaded}
\begin{Highlighting}[]
\NormalTok{\#set text(font: "Cinzel")}

\NormalTok{\#pseudocode{-}list(line{-}numbering: "I:")[}
\NormalTok{  + explore European tribes}
\NormalTok{  + *while* not all tribes conquered}
\NormalTok{    + *for each* tribe *in* unconquered tribes}
\NormalTok{      + try to conquer tribe}
\NormalTok{    + *end*}
\NormalTok{  + *end*}
\NormalTok{]}
\end{Highlighting}
\end{Shaded}

\pandocbounded{\includesvg[keepaspectratio]{https://github.com/typst/packages/raw/main/packages/preview/lovelace/0.3.0/examples/roman.svg}}

\subsubsection{Referencing lines}\label{referencing-lines}

You can reference an inividual line of a pseudocode by giving it a
label. Inside \texttt{\ pseudocode-list\ } , you can use
\texttt{\ line-label\ } :

\begin{Shaded}
\begin{Highlighting}[]
\NormalTok{\#pseudocode{-}list[}
\NormalTok{  + \#line{-}label(\textless{}start\textgreater{}) do something}
\NormalTok{  + \#line{-}label(\textless{}important\textgreater{}) do something important}
\NormalTok{  + go back to @start}
\NormalTok{]}

\NormalTok{The relevance of the step in @important cannot be overstated.}
\end{Highlighting}
\end{Shaded}

\pandocbounded{\includesvg[keepaspectratio]{https://github.com/typst/packages/raw/main/packages/preview/lovelace/0.3.0/examples/label.svg}}

When using \texttt{\ pseudocode\ } , you can use
\texttt{\ with-line-label\ } :

\begin{Shaded}
\begin{Highlighting}[]
\NormalTok{\#pseudocode(}
\NormalTok{  with{-}line{-}label(\textless{}start\textgreater{})[do something],}
\NormalTok{  with{-}line{-}label(\textless{}important\textgreater{})[do something important],}
\NormalTok{  [go back to @start],}
\NormalTok{)}

\NormalTok{The relevance of the step in @important cannot be overstated.}
\end{Highlighting}
\end{Shaded}

This has the same effect as the previous example.

The number shown in the reference uses the numbering scheme defined in
the \texttt{\ line-numbering\ } option (see previous section).

By default, \texttt{\ "Line"\ } is used as the supplement for
referencing lines. You can change that using the
\texttt{\ line-number-supplement\ } option to \texttt{\ pseudocode\ } or
\texttt{\ pseudocode-list\ } .

\subsubsection{Indentation guides}\label{indentation-guides}

By default, Lovelace puts a thin gray ( \texttt{\ gray\ +\ 1pt\ } ) line
to the left of each indented block, which guides the reader in
understanding the indentations, just like a code editor would. You can
customise this using the \texttt{\ stroke\ } option which takes any
value that is a valid
\href{https://typst.app/docs/reference/visualize/stroke/}{Typst stroke}
. You can especially set it to \texttt{\ none\ } to have no indentation
guides.

The example from the beginning becomes:

\begin{Shaded}
\begin{Highlighting}[]
\NormalTok{\#pseudocode{-}list(stroke: none)[}
\NormalTok{  + do something}
\NormalTok{  + do something else}
\NormalTok{  + *while* still something to do}
\NormalTok{    + do even more}
\NormalTok{    + *if* not done yet *then*}
\NormalTok{      + wait a bit}
\NormalTok{      + resume working}
\NormalTok{    + *else*}
\NormalTok{      + go home}
\NormalTok{    + *end*}
\NormalTok{  + *end*}
\NormalTok{]}
\end{Highlighting}
\end{Shaded}

\pandocbounded{\includesvg[keepaspectratio]{https://github.com/typst/packages/raw/main/packages/preview/lovelace/0.3.0/examples/simple-no-stroke.svg}}

\paragraph{End blocks with hooks}\label{end-blocks-with-hooks}

Some people prefer using the indentation guide to signal the end of a
block instead of writing something like “ \textbf{end} � by having a
small “hook� at the end. To achieve that in Lovelace, you can make
use of the \texttt{\ hook\ } option and specify how far a line should
extend to the right from the indentation guide:

\begin{Shaded}
\begin{Highlighting}[]
\NormalTok{\#pseudocode{-}list(hooks: .5em)[}
\NormalTok{  + do something}
\NormalTok{  + do something else}
\NormalTok{  + *while* still something to do}
\NormalTok{    + do even more}
\NormalTok{    + *if* not done yet *then*}
\NormalTok{      + wait a bit}
\NormalTok{      + resume working}
\NormalTok{    + *else*}
\NormalTok{      + go home}
\NormalTok{]}
\end{Highlighting}
\end{Shaded}

\pandocbounded{\includesvg[keepaspectratio]{https://github.com/typst/packages/raw/main/packages/preview/lovelace/0.3.0/examples/hooks.svg}}

\subsubsection{Spacing}\label{spacing}

You can control how far indented lines are shifted right by the
\texttt{\ indentation\ } option. To change the space between lines, use
the \texttt{\ line-gap\ } option.

\begin{Shaded}
\begin{Highlighting}[]
\NormalTok{\#pseudocode{-}list(indentation: 3em, line{-}gap: 1.5em)[}
\NormalTok{  + do something}
\NormalTok{  + do something else}
\NormalTok{  + *while* still something to do}
\NormalTok{    + do even more}
\NormalTok{    + *if* not done yet *then*}
\NormalTok{      + wait a bit}
\NormalTok{      + resume working}
\NormalTok{    + *else*}
\NormalTok{      + go home}
\NormalTok{    + *end*}
\NormalTok{  + *end*}
\NormalTok{]}
\end{Highlighting}
\end{Shaded}

\pandocbounded{\includesvg[keepaspectratio]{https://github.com/typst/packages/raw/main/packages/preview/lovelace/0.3.0/examples/spacing.svg}}

\subsubsection{Decorations}\label{decorations}

You can also add a title and/or a frame around your algorithm if you
like:

\paragraph{Title}\label{title}

Using the \texttt{\ title\ } option, you can give your pseudocode a
title (surprise!). For example, to achieve
\href{https://en.wikipedia.org/wiki/Introduction_to_Algorithms}{CLRS
style} , you can do something like

\begin{Shaded}
\begin{Highlighting}[]
\NormalTok{\#pseudocode{-}list(stroke: none, title: smallcaps[Fancy{-}Algorithm])[}
\NormalTok{  + do something}
\NormalTok{  + do something else}
\NormalTok{  + *while* still something to do}
\NormalTok{    + do even more}
\NormalTok{    + *if* not done yet *then*}
\NormalTok{      + wait a bit}
\NormalTok{      + resume working}
\NormalTok{    + *else*}
\NormalTok{      + go home}
\NormalTok{    + *end*}
\NormalTok{  + *end*}
\NormalTok{]}
\end{Highlighting}
\end{Shaded}

\pandocbounded{\includesvg[keepaspectratio]{https://github.com/typst/packages/raw/main/packages/preview/lovelace/0.3.0/examples/title.svg}}

\paragraph{Booktabs}\label{booktabs}

If you like wrapping your algorithm in elegant horizontal lines, you can
do so by setting the \texttt{\ booktabs\ } option to \texttt{\ true\ } .

\begin{Shaded}
\begin{Highlighting}[]
\NormalTok{\#pseudocode{-}list(booktabs: true)[}
\NormalTok{  + do something}
\NormalTok{  + do something else}
\NormalTok{  + *while* still something to do}
\NormalTok{    + do even more}
\NormalTok{    + *if* not done yet *then*}
\NormalTok{      + wait a bit}
\NormalTok{      + resume working}
\NormalTok{    + *else*}
\NormalTok{      + go home}
\NormalTok{    + *end*}
\NormalTok{  + *end*}
\NormalTok{]}
\end{Highlighting}
\end{Shaded}

\pandocbounded{\includesvg[keepaspectratio]{https://github.com/typst/packages/raw/main/packages/preview/lovelace/0.3.0/examples/booktabs.svg}}

Together with the \texttt{\ title\ } option, you can produce

\begin{Shaded}
\begin{Highlighting}[]
\NormalTok{\#pseudocode{-}list(booktabs: true, title: [My cool title])[}
\NormalTok{  + do something}
\NormalTok{  + do something else}
\NormalTok{  + *while* still something to do}
\NormalTok{    + do even more}
\NormalTok{    + *if* not done yet *then*}
\NormalTok{      + wait a bit}
\NormalTok{      + resume working}
\NormalTok{    + *else*}
\NormalTok{      + go home}
\NormalTok{    + *end*}
\NormalTok{  + *end*}
\NormalTok{]}
\end{Highlighting}
\end{Shaded}

\pandocbounded{\includesvg[keepaspectratio]{https://github.com/typst/packages/raw/main/packages/preview/lovelace/0.3.0/examples/booktabs-title.svg}}

By default, the outer booktab strokes are \texttt{\ black\ +\ 2pt\ } .
You can change that with the option \texttt{\ booktabs-stroke\ } to any
valid \href{https://typst.app/docs/reference/visualize/stroke/}{Typst
stroke} . The inner line will always have the same stroke as the outer
ones, just with half the thickness.

\subsubsection{Algorithm as figure}\label{algorithm-as-figure}

To make algorithms referencable and being able to float in the document,
you can use Typst’s \texttt{\ figure\ } function with a custom
\texttt{\ kind\ } .

\begin{Shaded}
\begin{Highlighting}[]
\NormalTok{\#figure(}
\NormalTok{  kind: "algorithm",}
\NormalTok{  supplement: [Algorithm],}
\NormalTok{  caption: [My cool algorithm],}

\NormalTok{  pseudocode{-}list[}
\NormalTok{    + do something}
\NormalTok{    + do something else}
\NormalTok{    + *while* still something to do}
\NormalTok{      + do even more}
\NormalTok{      + *if* not done yet *then*}
\NormalTok{        + wait a bit}
\NormalTok{        + resume working}
\NormalTok{      + *else*}
\NormalTok{        + go home}
\NormalTok{      + *end*}
\NormalTok{    + *end*}
\NormalTok{  ]}
\NormalTok{)}
\end{Highlighting}
\end{Shaded}

\pandocbounded{\includesvg[keepaspectratio]{https://github.com/typst/packages/raw/main/packages/preview/lovelace/0.3.0/examples/figure.svg}}

If you want to have the algorithm counter inside the title instead (see
previous section), there is the option \texttt{\ numbered-title\ } :

\begin{Shaded}
\begin{Highlighting}[]
\NormalTok{\#figure(}
\NormalTok{  kind: "algorithm",}
\NormalTok{  supplement: [Algorithm],}

\NormalTok{  pseudocode{-}list(booktabs: true, numbered{-}title: [My cool algorithm])[}
\NormalTok{    + do something}
\NormalTok{    + do something else}
\NormalTok{    + *while* still something to do}
\NormalTok{      + do even more}
\NormalTok{      + *if* not done yet *then*}
\NormalTok{        + wait a bit}
\NormalTok{        + resume working}
\NormalTok{      + *else*}
\NormalTok{        + go home}
\NormalTok{      + *end*}
\NormalTok{    + *end*}
\NormalTok{  ]}
\NormalTok{) \textless{}cool\textgreater{}}

\NormalTok{See @cool for details on how to do something cool.}
\end{Highlighting}
\end{Shaded}

\pandocbounded{\includesvg[keepaspectratio]{https://github.com/typst/packages/raw/main/packages/preview/lovelace/0.3.0/examples/figure-title.svg}}

Note that the \texttt{\ numbered-title\ } option only makes sense when
nesting your pseudocode inside a figure with
\texttt{\ kind:\ "algorithm"\ } , otherwise it produces undefined
results.

\subsubsection{Customisation overview}\label{customisation-overview}

Both \texttt{\ pseudocode\ } and \texttt{\ pseudocode-list\ } accept the
following configuration arguments:

\begin{longtable}[]{@{}lll@{}}
\toprule\noalign{}
\textbf{option} & \textbf{type} & \textbf{default} \\
\midrule\noalign{}
\endhead
\bottomrule\noalign{}
\endlastfoot
\href{https://github.com/typst/packages/raw/main/packages/preview/lovelace/0.3.0/\#line-numbers}{\texttt{\ line-numbering\ }}
& \texttt{\ none\ } or a
\href{https://typst.app/docs/reference/model/numbering/\#parameters-numbering}{numbering}
& \texttt{\ "1"\ } \\
\href{https://github.com/typst/packages/raw/main/packages/preview/lovelace/0.3.0/\#more-line-number-customisation}{\texttt{\ line-number-supplement\ }}
& content & \texttt{\ "Line"\ } \\
\href{https://github.com/typst/packages/raw/main/packages/preview/lovelace/0.3.0/\#indentation-guides}{\texttt{\ stroke\ }}
& stroke & \texttt{\ 1pt\ +\ gray\ } \\
\href{https://github.com/typst/packages/raw/main/packages/preview/lovelace/0.3.0/\#end-blocks-with-hooks}{\texttt{\ hooks\ }}
& length & \texttt{\ 0pt\ } \\
\href{https://github.com/typst/packages/raw/main/packages/preview/lovelace/0.3.0/\#spacing}{\texttt{\ indentation\ }}
& length & \texttt{\ 1em\ } \\
\href{https://github.com/typst/packages/raw/main/packages/preview/lovelace/0.3.0/\#spacing}{\texttt{\ line-gap\ }}
& length & \texttt{\ .8em\ } \\
\href{https://github.com/typst/packages/raw/main/packages/preview/lovelace/0.3.0/\#booktabs}{\texttt{\ booktabs\ }}
& bool & \texttt{\ false\ } \\
\href{https://github.com/typst/packages/raw/main/packages/preview/lovelace/0.3.0/\#booktabs}{\texttt{\ booktabs-stroke\ }}
& stroke & \texttt{\ 2pt\ +\ black\ } \\
\href{https://github.com/typst/packages/raw/main/packages/preview/lovelace/0.3.0/\#title}{\texttt{\ title\ }}
& content or \texttt{\ none\ } & \texttt{\ none\ } \\
\href{https://github.com/typst/packages/raw/main/packages/preview/lovelace/0.3.0/\#algorithm-as-figure}{\texttt{\ numbered-title\ }}
& content or \texttt{\ none\ } & \texttt{\ none\ } \\
\end{longtable}

Until Typst supports user defined types, we can use the following trick
when wanting to set own default values for these options. Say, you
always want your algorithms to have colons after the line numbers, no
indentation guides and, if present, blue booktabs. In this case, you
would put the following at the top of your document:

\begin{Shaded}
\begin{Highlighting}[]
\NormalTok{\#let my{-}lovelace{-}defaults = (}
\NormalTok{  line{-}numbering: "1:",}
\NormalTok{  stroke: none,}
\NormalTok{  booktabs{-}stroke: 2pt + blue,}
\NormalTok{)}

\NormalTok{\#let pseudocode = pseudocode.with(..my{-}lovelace{-}defaults)}
\NormalTok{\#let pseudocode{-}list = pseudocode{-}list.with(..my{-}lovelace{-}defaults)}
\end{Highlighting}
\end{Shaded}

\subsubsection{Exported functions}\label{exported-functions}

Lovelace exports the following functions:

\begin{itemize}
\tightlist
\item
  \texttt{\ pseudocode\ } : Typeset pseudocode with each line as an
  individual content argument, see
  \href{https://github.com/typst/packages/raw/main/packages/preview/lovelace/0.3.0/\#lower-level-interface}{here}
  for details. Has
  \href{https://github.com/typst/packages/raw/main/packages/preview/lovelace/0.3.0/\#customisation-overview}{these}
  optional arguments.
\item
  \texttt{\ pseudocode-list\ } : Takes a standard Typst list and
  transforms it into a pseudocode. Has
  \href{https://github.com/typst/packages/raw/main/packages/preview/lovelace/0.3.0/\#customisation-overview}{these}
  optional arguments.
\item
  \texttt{\ indent\ } : Inside the argument list of
  \texttt{\ pseudocode\ } , use \texttt{\ indent\ } to specify an
  indented block, see
  \href{https://github.com/typst/packages/raw/main/packages/preview/lovelace/0.3.0/\#lower-level-interface}{here}
  for details.
\item
  \texttt{\ no-number\ } : Wrap an argument to \texttt{\ pseudocode\ }
  in this function to have the corresponding line be unnumbered, see
  \href{https://github.com/typst/packages/raw/main/packages/preview/lovelace/0.3.0/\#line-numbers}{here}
  for details.
\item
  \texttt{\ with-line-label\ } : Use this function in the
  \texttt{\ pseudocode\ } arguments to add a label to a specific line,
  see
  \href{https://github.com/typst/packages/raw/main/packages/preview/lovelace/0.3.0/\#referencing-lines}{here}
  for details.
\item
  \texttt{\ line-label\ } : When using \texttt{\ pseudocode-list\ } ,
  you do \emph{not} use \texttt{\ with-line-label\ } but insert a call
  to \texttt{\ line-label\ } somewhere in a line to add a label, see
  \href{https://github.com/typst/packages/raw/main/packages/preview/lovelace/0.3.0/\#referencing-lines}{here}
  for details.
\end{itemize}

\subsubsection{How to add}\label{how-to-add}

Copy this into your project and use the import as \texttt{\ lovelace\ }

\begin{verbatim}
#import "@preview/lovelace:0.3.0"
\end{verbatim}

\includesvg[width=0.16667in,height=0.16667in]{/assets/icons/16-copy.svg}

Check the docs for
\href{https://typst.app/docs/reference/scripting/\#packages}{more
information on how to import packages} .

\subsubsection{About}\label{about}

\begin{description}
\tightlist
\item[Author s :]
Andreas Kröpelin \& Lovelace contributors
\item[License:]
MIT
\item[Current version:]
0.3.0
\item[Last updated:]
July 1, 2024
\item[First released:]
September 6, 2023
\item[Archive size:]
3.44 kB
\href{https://packages.typst.org/preview/lovelace-0.3.0.tar.gz}{\pandocbounded{\includesvg[keepaspectratio]{/assets/icons/16-download.svg}}}
\item[Repository:]
\href{https://github.com/andreasKroepelin/lovelace}{GitHub}
\end{description}

\subsubsection{Where to report issues?}\label{where-to-report-issues}

This package is a project of Andreas Kröpelin and Lovelace contributors
. Report issues on
\href{https://github.com/andreasKroepelin/lovelace}{their repository} .
You can also try to ask for help with this package on the
\href{https://forum.typst.app}{Forum} .

Please report this package to the Typst team using the
\href{https://typst.app/contact}{contact form} if you believe it is a
safety hazard or infringes upon your rights.

\phantomsection\label{versions}
\subsubsection{Version history}\label{version-history}

\begin{longtable}[]{@{}ll@{}}
\toprule\noalign{}
Version & Release Date \\
\midrule\noalign{}
\endhead
\bottomrule\noalign{}
\endlastfoot
0.3.0 & July 1, 2024 \\
\href{https://typst.app/universe/package/lovelace/0.2.0/}{0.2.0} &
January 9, 2024 \\
\href{https://typst.app/universe/package/lovelace/0.1.0/}{0.1.0} &
September 6, 2023 \\
\end{longtable}

Typst GmbH did not create this package and cannot guarantee correct
functionality of this package or compatibility with any version of the
Typst compiler or app.
