\title{typst.app/universe/package/babble-bubbles}

\phantomsection\label{banner}
\section{babble-bubbles}\label{babble-bubbles}

{ 0.1.0 }

A package to create callouts.

\phantomsection\label{readme}
A package to create callouts in typst, inspired by the
\href{https://obsidian.md/}{Obsidan} callouts.

Use preset callouts, or create your own!

\pandocbounded{\includegraphics[keepaspectratio]{https://github.com/typst/packages/raw/main/packages/preview/babble-bubbles/0.1.0/examples/callouts.png}}

\subsection{Usage}\label{usage}

Import the package

\begin{Shaded}
\begin{Highlighting}[]
\NormalTok{\#import "@preview/babble{-}bubbles:0.1.0": *}
\end{Highlighting}
\end{Shaded}

Or grab it locally and use:

\begin{Shaded}
\begin{Highlighting}[]
\NormalTok{\#import "@local/babble{-}bubbles:0.1.0": *}
\end{Highlighting}
\end{Shaded}

\subsection{Presets}\label{presets}

Here you can find a list of presets and an example usage of each. You
can customise them with the same parameters as the \texttt{\ callout\ }
function. See the \texttt{\ Custom\ callouts\ } for more details.

\begin{Shaded}
\begin{Highlighting}[]
\NormalTok{\#info[This is information]}

\NormalTok{\#success[I\textquotesingle{}m making a note here: huge success]}

\NormalTok{\#check[This is checked!]}

\NormalTok{\#warning[First warning...]}

\NormalTok{\#note[My incredibly useful note]}

\NormalTok{\#question[Question?]}

\NormalTok{\#example[An example make things interesting]}

\NormalTok{\#quote[To be or not to be]}
\end{Highlighting}
\end{Shaded}

\subsection{Custom callouts}\label{custom-callouts}

\subsubsection{\texorpdfstring{\texttt{\ callout\ }}{ callout }}\label{callout}

Create a default callout. Tweak the parameters to create your own!

\begin{Shaded}
\begin{Highlighting}[]
\NormalTok{callout(}
\NormalTok{  body,}
\NormalTok{  title: "Callout",}
\NormalTok{  fill: blue,}
\NormalTok{  title{-}color: white,}
\NormalTok{  body{-}color: black,}
\NormalTok{  icon: none)}
\end{Highlighting}
\end{Shaded}

\subsubsection{Tips}\label{tips}

You can create aliases to more easily handle your newly create callouts
or customise presets by using
\href{https://typst.app/docs/reference/types/function/\#methods-with}{with}
.

\begin{verbatim}
#let mycallout = callout.with(title: "My callout")

#mycallout[Hey this is my custom callout!]
\end{verbatim}

\subsubsection{How to add}\label{how-to-add}

Copy this into your project and use the import as
\texttt{\ babble-bubbles\ }

\begin{verbatim}
#import "@preview/babble-bubbles:0.1.0"
\end{verbatim}

\includesvg[width=0.16667in,height=0.16667in]{/assets/icons/16-copy.svg}

Check the docs for
\href{https://typst.app/docs/reference/scripting/\#packages}{more
information on how to import packages} .

\subsubsection{About}\label{about}

\begin{description}
\tightlist
\item[Author :]
Dimitri Belopopsky
\item[License:]
MIT
\item[Current version:]
0.1.0
\item[Last updated:]
September 11, 2023
\item[First released:]
September 11, 2023
\item[Archive size:]
2.15 kB
\href{https://packages.typst.org/preview/babble-bubbles-0.1.0.tar.gz}{\pandocbounded{\includesvg[keepaspectratio]{/assets/icons/16-download.svg}}}
\item[Repository:]
\href{https://github.com/ShadowMitia/typst-babble-bubbles}{GitHub}
\end{description}

\subsubsection{Where to report issues?}\label{where-to-report-issues}

This package is a project of Dimitri Belopopsky . Report issues on
\href{https://github.com/ShadowMitia/typst-babble-bubbles}{their
repository} . You can also try to ask for help with this package on the
\href{https://forum.typst.app}{Forum} .

Please report this package to the Typst team using the
\href{https://typst.app/contact}{contact form} if you believe it is a
safety hazard or infringes upon your rights.

\phantomsection\label{versions}
\subsubsection{Version history}\label{version-history}

\begin{longtable}[]{@{}ll@{}}
\toprule\noalign{}
Version & Release Date \\
\midrule\noalign{}
\endhead
\bottomrule\noalign{}
\endlastfoot
0.1.0 & September 11, 2023 \\
\end{longtable}

Typst GmbH did not create this package and cannot guarantee correct
functionality of this package or compatibility with any version of the
Typst compiler or app.
