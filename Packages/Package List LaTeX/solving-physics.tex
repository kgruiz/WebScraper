\title{typst.app/universe/package/solving-physics}

\phantomsection\label{banner}
\section{solving-physics}\label{solving-physics}

{ 0.1.0 }

A package to formulate the solution to a physical problem

\phantomsection\label{readme}
The easiest method is to import \texttt{\ solving-physics:\ task\ } from
the \texttt{\ @preview\ } package:

\begin{Shaded}
\begin{Highlighting}[]
\NormalTok{\#import "@preview/solving{-}physics:0.1.0": *}
\end{Highlighting}
\end{Shaded}

\begin{Shaded}
\begin{Highlighting}[]
\NormalTok{\#task(}
\NormalTok{  given: [}
\NormalTok{    $mu = 0.4$ \textbackslash{}}
\NormalTok{    $g = 10$ \textbackslash{}}
\NormalTok{    $m = 20$}
\NormalTok{  ],}
\NormalTok{  find: [}
\NormalTok{    $F$ {-}{-}{-} ?}
\NormalTok{  ],}
\NormalTok{  fig: image("./example.png", width: 5cm)}
\NormalTok{)[}
\NormalTok{  \#lorem(100)}
\NormalTok{]}
\end{Highlighting}
\end{Shaded}

\pandocbounded{\includegraphics[keepaspectratio]{https://raw.githubusercontent.com/yegorweb/solving-physics/master/examples/example1.png}}

\begin{Shaded}
\begin{Highlighting}[]
\NormalTok{\#task(}
\NormalTok{  given: [}
\NormalTok{    $mu = 0.4$ \textbackslash{}}
\NormalTok{    $g = 10$ \textbackslash{}}
\NormalTok{    $m = 20$}
\NormalTok{  ],}
\NormalTok{  find: [}
\NormalTok{    $F$ {-}{-}{-} ?}
\NormalTok{  ],}
\NormalTok{  stroke: "full"}
\NormalTok{)[]}
\end{Highlighting}
\end{Shaded}

\pandocbounded{\includesvg[keepaspectratio]{https://raw.githubusercontent.com/yegorweb/solving-physics/master/examples/example2.svg}}

\begin{Shaded}
\begin{Highlighting}[]
\NormalTok{\#task(}
\NormalTok{  given: [}
\NormalTok{    $mu = 0.4$ \textbackslash{}}
\NormalTok{    $g = 10$ \textbackslash{}}
\NormalTok{    $m = 20$}
\NormalTok{  ],}
\NormalTok{  find: [}
\NormalTok{    $F$ {-}{-}{-} ?}
\NormalTok{  ],}
\NormalTok{  stroke: "find"}
\NormalTok{)[]}
\end{Highlighting}
\end{Shaded}

\pandocbounded{\includesvg[keepaspectratio]{https://raw.githubusercontent.com/yegorweb/solving-physics/master/examples/example3.svg}}

\begin{Shaded}
\begin{Highlighting}[]
\NormalTok{\#task(}
\NormalTok{  given: [}
\NormalTok{    $mu = 0.4$ \textbackslash{}}
\NormalTok{    $g = 10$ \textbackslash{}}
\NormalTok{    $m = 20$}
\NormalTok{  ],}
\NormalTok{  find: [}
\NormalTok{    $F$ {-}{-}{-} ?}
\NormalTok{  ],}
\NormalTok{  stroke: none}
\NormalTok{)[]}
\end{Highlighting}
\end{Shaded}

\pandocbounded{\includesvg[keepaspectratio]{https://raw.githubusercontent.com/yegorweb/solving-physics/master/examples/example4.svg}}

If you have so large given you may use \texttt{\ given-width\ } :

\begin{Shaded}
\begin{Highlighting}[]
\NormalTok{\#task(}
\NormalTok{  given: [}
\NormalTok{    $mu = 0.4$ \textbackslash{}}
\NormalTok{    $g = 10$ \textbackslash{}}
\NormalTok{    $m = 20$ \textbackslash{}}
\NormalTok{    \#lorem(10)}
\NormalTok{  ],}
\NormalTok{  given{-}width: 6em,}
\NormalTok{  find: [}
\NormalTok{    $F$ {-}{-}{-} ?}
\NormalTok{  ],}
\NormalTok{)[]}
\end{Highlighting}
\end{Shaded}

\pandocbounded{\includesvg[keepaspectratio]{https://raw.githubusercontent.com/yegorweb/solving-physics/master/examples/example5.svg}}

You may locate you figure on the center of body by
\texttt{\ fig-align:\ top\ +\ center\ }

\begin{Shaded}
\begin{Highlighting}[]
\NormalTok{\#task(}
\NormalTok{  given: [}
\NormalTok{    $mu = 0.4$ \textbackslash{}}
\NormalTok{    $g = 10$ \textbackslash{}}
\NormalTok{    $m = 20$}
\NormalTok{  ],}
\NormalTok{  find: [}
\NormalTok{    $F$ {-}{-}{-} ?}
\NormalTok{  ],}
\NormalTok{  fig: image("./example.png", width: 60\%),}
\NormalTok{  fig{-}align: top + center}
\NormalTok{)[}
\NormalTok{  \#lorem(100)}
\NormalTok{]}
\end{Highlighting}
\end{Shaded}

\pandocbounded{\includegraphics[keepaspectratio]{https://raw.githubusercontent.com/yegorweb/solving-physics/master/examples/example6.png}}

\subsubsection{How to add}\label{how-to-add}

Copy this into your project and use the import as
\texttt{\ solving-physics\ }

\begin{verbatim}
#import "@preview/solving-physics:0.1.0"
\end{verbatim}

\includesvg[width=0.16667in,height=0.16667in]{/assets/icons/16-copy.svg}

Check the docs for
\href{https://typst.app/docs/reference/scripting/\#packages}{more
information on how to import packages} .

\subsubsection{About}\label{about}

\begin{description}
\tightlist
\item[Author :]
Yegor Knyazev
\item[License:]
MIT
\item[Current version:]
0.1.0
\item[Last updated:]
May 13, 2024
\item[First released:]
May 13, 2024
\item[Archive size:]
1.86 kB
\href{https://packages.typst.org/preview/solving-physics-0.1.0.tar.gz}{\pandocbounded{\includesvg[keepaspectratio]{/assets/icons/16-download.svg}}}
\item[Repository:]
\href{https://github.com/yegorweb/solving-physics}{GitHub}
\item[Discipline s :]
\begin{itemize}
\tightlist
\item[]
\item
  \href{https://typst.app/universe/search/?discipline=chemistry}{Chemistry}
\item
  \href{https://typst.app/universe/search/?discipline=education}{Education}
\item
  \href{https://typst.app/universe/search/?discipline=physics}{Physics}
\end{itemize}
\item[Categor y :]
\begin{itemize}
\tightlist
\item[]
\item
  \pandocbounded{\includesvg[keepaspectratio]{/assets/icons/16-package.svg}}
  \href{https://typst.app/universe/search/?category=components}{Components}
\end{itemize}
\end{description}

\subsubsection{Where to report issues?}\label{where-to-report-issues}

This package is a project of Yegor Knyazev . Report issues on
\href{https://github.com/yegorweb/solving-physics}{their repository} .
You can also try to ask for help with this package on the
\href{https://forum.typst.app}{Forum} .

Please report this package to the Typst team using the
\href{https://typst.app/contact}{contact form} if you believe it is a
safety hazard or infringes upon your rights.

\phantomsection\label{versions}
\subsubsection{Version history}\label{version-history}

\begin{longtable}[]{@{}ll@{}}
\toprule\noalign{}
Version & Release Date \\
\midrule\noalign{}
\endhead
\bottomrule\noalign{}
\endlastfoot
0.1.0 & May 13, 2024 \\
\end{longtable}

Typst GmbH did not create this package and cannot guarantee correct
functionality of this package or compatibility with any version of the
Typst compiler or app.
