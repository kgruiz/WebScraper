\title{typst.app/universe/package/super-suboptimal}

\phantomsection\label{banner}
\section{super-suboptimal}\label{super-suboptimal}

{ 0.1.0 }

Unicode super- and subscripts in equations.

\phantomsection\label{readme}
A Typst package enabling support for Unicode super- and subscript
characters in equations.

\subsection{Usage}\label{usage}

The package exposes the template-function \texttt{\ super-subscripts\ }
. It affects all \texttt{\ math.equation\ } s by attaching every
superscript- and subscript-character to the first non-space-element on
its left.

\begin{Shaded}
\begin{Highlighting}[]
\NormalTok{\#import "@preview/super{-}suboptimal:0.1.0": *}
\NormalTok{\#show: super{-}subscripts}

\NormalTok{For all $(x,y)∈ℝ²$:}
\NormalTok{$}
\NormalTok{  q := norm((x,y))₂ \textless{} 1}
\NormalTok{  ==\textgreater{} ∑ᵢ₌₁ⁿ q ⁱ \textless{} ∞}
\NormalTok{$}
\end{Highlighting}
\end{Shaded}

\pandocbounded{\includesvg[keepaspectratio]{https://github.com/typst/packages/raw/main/packages/preview/super-suboptimal/0.1.0/assets/example0.svg}}

Because code like \texttt{\ \$x+yᶻ\$\ } throws an “unknown
variable� error, the package also exposes the function \texttt{\ eq\ }
, which inserts spaces before every superscript- and subscript-character
and passing it on to \texttt{\ math.equation\ } . This comes at the cost
of missing syntax-highlighting and code-suggestions in your IDE.

\texttt{\ eq\ } accepts a \texttt{\ raw\ } string as a positional
parameter, and an argument-sink that’s passed onto
\texttt{\ math.equation\ } . Unless specified otherwise in the
argument-sink, the resulting equation is typeset with
\texttt{\ block:\ true\ } if and only if the \texttt{\ raw\ } also
satisfied \texttt{\ block:\ true\ } . \texttt{\ eq\ } is automatically
applied to every \texttt{\ raw\ } with \texttt{\ lang:\ "eq"\ } .

\begin{Shaded}
\begin{Highlighting}[]
\NormalTok{\#eq(\textasciigrave{}0 = aᵇ\textasciigrave{})}

\NormalTok{\#eq(\textasciigrave{}\textasciigrave{}\textasciigrave{}}
\NormalTok{  1 = x+yᶻ}
\NormalTok{\textasciigrave{}\textasciigrave{}\textasciigrave{})}

\NormalTok{\#eq(\textasciigrave{}2 = aᵇ\textasciigrave{}, block: true, numbering: "(1)")}

\NormalTok{\textasciigrave{}\textasciigrave{}\textasciigrave{}eq}
\NormalTok{  3 = aᵇᶜ⁺ᵈ₃ₑ⁽ᶠ⁻ᵍ⁾ₕᵢ}
\NormalTok{\textasciigrave{}\textasciigrave{}\textasciigrave{}}
\end{Highlighting}
\end{Shaded}

\pandocbounded{\includesvg[keepaspectratio]{https://github.com/typst/packages/raw/main/packages/preview/super-suboptimal/0.1.0/assets/example1.svg}}

Sometimes in mathematical writing, variables are decorated with an
asterisk, e.g. \texttt{\ \$x\^{}*\$\ } . The character \texttt{\ ꙳\ }
can now be used, as well: \texttt{\ \$x꙳\ =\ x\^{}*\$\ } .

\subsection{Known issues}\label{known-issues}

\begin{itemize}
\item
  As mentioned above, \texttt{\ \$aᵇ\$\ } leads to an “unknown
  variable� error. As a workaround, \texttt{\ \$a\ ᵇ\$\ } produces
  the same output, or you can use the \texttt{\ eq\ } function described
  above.

  \begin{itemize}
  \tightlist
  \item
    The first workaround also means I can’t reasonably implement
    top-left and bottom-left attachments. For example,
    \texttt{\ \$a\ ³b\$\ } is rendered like
    \texttt{\ \$attach(a,\ t:\ 3)\ b\$\ } , rather than
    \texttt{\ \$a\ attach(b,\ tl:\ 3)\$\ } .
  \end{itemize}
\item
  Multiple attachments are concatenated into one content without another
  pass of \texttt{\ equation\ } . For example,
  \texttt{\ \#eq(\textasciigrave{}0ˢ�����\textasciigrave{})\ }
  is equivalent to \texttt{\ \$0\^{}(s\ i\ n\ "("\ k\ ")")\$\ } , rather
  than \texttt{\ \$0\^{}sin(k)\$\ } . I won’t fix this, because:

  \begin{itemize}
  \tightlist
  \item
    Another pass of \texttt{\ equation\ } would cause performance issues
    at best, and infinite loops at worst.
  \item
    If this were fixed, code such as \texttt{\ \$e\ ˣ\ ʸ\$\ } would
    undesirably produce an “unknown variable \texttt{\ xy\ } �
    error.
  \end{itemize}
\item
  Let’s call a piece of content “small� if it consists of only a
  single non-separated sequence of characters in Typst (internally, this
  is the distinction between the content-functions \texttt{\ sequence\ }
  and \texttt{\ text\ } ). For instance, \texttt{\ \$1234\$\ } and
  \texttt{\ \$a\$\ } constitute “small� content, but
  \texttt{\ \$a\ b\$\ } and \texttt{\ \$3a\$\ } and
  \texttt{\ \$1+2+3+4+5\$\ } do not.

  This package only runs on non-“small� pieces of content. For
  example, \texttt{\ \$sqrt(35²)\$\ } still renders with the
  default-Unicode-character and will look different from
  \texttt{\ \$sqrt(35\^{}2)\$\ } . On the other hand,
  \texttt{\ \$sqrt(aâ?¶)\$\ } \emph{is} rendered correctly. This is
  because \texttt{\ 35²\ } constitutes “small� content, but
  \texttt{\ aâ?¶\ } does not.

  A workaround is implemented for “small� content immediately within
  an equation, i.e. not nested within another content-function. For
  example, \texttt{\ \$7²\$\ } renders the same as
  \texttt{\ \$7\^{}2\$\ } , even though it’s “small� content.
\item
  Equations within other content-elements might trigger multiple
  show-rule-passes, possibly causing performance-issues.
\end{itemize}

\subsubsection{How to add}\label{how-to-add}

Copy this into your project and use the import as
\texttt{\ super-suboptimal\ }

\begin{verbatim}
#import "@preview/super-suboptimal:0.1.0"
\end{verbatim}

\includesvg[width=0.16667in,height=0.16667in]{/assets/icons/16-copy.svg}

Check the docs for
\href{https://typst.app/docs/reference/scripting/\#packages}{more
information on how to import packages} .

\subsubsection{About}\label{about}

\begin{description}
\tightlist
\item[Author s :]
Eric Biedert \& Lumi
\item[License:]
MIT
\item[Current version:]
0.1.0
\item[Last updated:]
January 29, 2024
\item[First released:]
January 29, 2024
\item[Archive size:]
6.15 kB
\href{https://packages.typst.org/preview/super-suboptimal-0.1.0.tar.gz}{\pandocbounded{\includesvg[keepaspectratio]{/assets/icons/16-download.svg}}}
\end{description}

\subsubsection{Where to report issues?}\label{where-to-report-issues}

This package is a project of Eric Biedert and Lumi . You can also try to
ask for help with this package on the
\href{https://forum.typst.app}{Forum} .

Please report this package to the Typst team using the
\href{https://typst.app/contact}{contact form} if you believe it is a
safety hazard or infringes upon your rights.

\phantomsection\label{versions}
\subsubsection{Version history}\label{version-history}

\begin{longtable}[]{@{}ll@{}}
\toprule\noalign{}
Version & Release Date \\
\midrule\noalign{}
\endhead
\bottomrule\noalign{}
\endlastfoot
0.1.0 & January 29, 2024 \\
\end{longtable}

Typst GmbH did not create this package and cannot guarantee correct
functionality of this package or compatibility with any version of the
Typst compiler or app.
