\title{typst.app/universe/package/thesist}

\phantomsection\label{banner}
\phantomsection\label{template-thumbnail}
\pandocbounded{\includegraphics[keepaspectratio]{https://packages.typst.org/preview/thumbnails/thesist-0.2.0-small.webp}}

\section{thesist}\label{thesist}

{ 0.2.0 }

A Master\textquotesingle s thesis template for Instituto Superior
Técnico (IST)

\href{/app?template=thesist&version=0.2.0}{Create project in app}

\phantomsection\label{readme}
ThesIST (pronounced “desist�) is an unofficial Master’s thesis
template for Instituto Superior Técnico written in Typst.

This template fully meets the official formatting requirements as
outlined
\href{https://tecnico.ulisboa.pt/files/2021/09/guia-disserta-o-mestrado.pdf}{here}
, and also attempts to follow most unwritten conventions. Regardless,
you can be on the lookout for things you may want to see added.

PIC2 reports are also supported. However, some conventions for these may
vary with the supervisors, so please check with them if anything needs
to be changed.

\subsection{Changelogs}\label{changelogs}

The changelogs of new versions are available on
\href{https://github.com/tfachada/thesist/releases}{the Releases page} .
Make sure to check the latest one(s) whenever you update the imported
\texttt{\ thesist\ } version.

\subsection{Usage}\label{usage}

If you are in the Typst web app, simply click on “Start from
template� and pick this template.

If you want to develop locally:

\begin{enumerate}
\tightlist
\item
  Make sure you have the \textbf{TeX Gyre Heros} font family installed.
\item
  Install the package with \texttt{\ typst\ init\ @preview/thesist\ } .
\end{enumerate}

\subsection{Overview}\label{overview}

\textbf{Please read the “Quick guide� chapter included in this
template to get set up. You can keep it as a reference if you want.}

This template’s source files, hidden from the user view, are the
following:

\begin{itemize}
\item
  \texttt{\ layout.typ\ } : The main configuration file, which
  initializes the thesis and contains its general formatting rules.
\item
  \texttt{\ figure-numbering.typ\ } : This file contains a function to
  set a chapter-relative numbering for the various types of figures. The
  function is called once or twice depending on whether the user decides
  to include appendices.
\item
  \texttt{\ utils.typ\ } : General functions that you may want to import
  and use for QoL improvements.
\end{itemize}

\subsubsection{A sidenote about
subfigures}\label{a-sidenote-about-subfigures}

Since subfigures are not yet native to Typst, the current
implementation, present in \texttt{\ utils.typ\ } , needs the user to
manually input whether each called subfigure figure (aka subfigure grid)
is in an appendix or not. This is because the numbering is different in
appendices, and because the functionality of
\texttt{\ figure-numbering.typ\ } can’t be applied to subfigure grids,
since they are imported with their default numbering once in every
chapter. \texttt{\ context\ } expressions also don’t work across
imports, so location within the document couldn’t be used as a
parameter (unless the user called \texttt{\ context\ } themselves, which
would be unintuitive). \textbf{Regardless, the workaround that was
found, which is explained in the quick guide, doesn’t need much
thinking from the user, so you can see this as a more technical note
that shouldn’t matter when you’re writing the thesis.}

\subsection{Final remarks}\label{final-remarks}

This template is not necessarily (or hopefully) a finished product. Feel
free to open issues or pull requests!

Also thanks to the Typst community members for the help in some of the
functionalities, and for the extensions used here.

\href{/app?template=thesist&version=0.2.0}{Create project in app}

\subsubsection{How to use}\label{how-to-use}

Click the button above to create a new project using this template in
the Typst app.

You can also use the Typst CLI to start a new project on your computer
using this command:

\begin{verbatim}
typst init @preview/thesist:0.2.0
\end{verbatim}

\includesvg[width=0.16667in,height=0.16667in]{/assets/icons/16-copy.svg}

\subsubsection{About}\label{about}

\begin{description}
\tightlist
\item[Author :]
\href{https://github.com/tfachada}{Tomás Fachada}
\item[License:]
MIT
\item[Current version:]
0.2.0
\item[Last updated:]
October 21, 2024
\item[First released:]
August 28, 2024
\item[Minimum Typst version:]
0.12.0
\item[Archive size:]
373 kB
\href{https://packages.typst.org/preview/thesist-0.2.0.tar.gz}{\pandocbounded{\includesvg[keepaspectratio]{/assets/icons/16-download.svg}}}
\item[Repository:]
\href{https://github.com/tfachada/thesist}{GitHub}
\item[Categor y :]
\begin{itemize}
\tightlist
\item[]
\item
  \pandocbounded{\includesvg[keepaspectratio]{/assets/icons/16-mortarboard.svg}}
  \href{https://typst.app/universe/search/?category=thesis}{Thesis}
\end{itemize}
\end{description}

\subsubsection{Where to report issues?}\label{where-to-report-issues}

This template is a project of Tomás Fachada . Report issues on
\href{https://github.com/tfachada/thesist}{their repository} . You can
also try to ask for help with this template on the
\href{https://forum.typst.app}{Forum} .

Please report this template to the Typst team using the
\href{https://typst.app/contact}{contact form} if you believe it is a
safety hazard or infringes upon your rights.

\phantomsection\label{versions}
\subsubsection{Version history}\label{version-history}

\begin{longtable}[]{@{}ll@{}}
\toprule\noalign{}
Version & Release Date \\
\midrule\noalign{}
\endhead
\bottomrule\noalign{}
\endlastfoot
0.2.0 & October 21, 2024 \\
\href{https://typst.app/universe/package/thesist/0.1.0/}{0.1.0} & August
28, 2024 \\
\end{longtable}

Typst GmbH did not create this template and cannot guarantee correct
functionality of this template or compatibility with any version of the
Typst compiler or app.
