\title{typst.app/universe/package/vercanard}

\phantomsection\label{banner}
\phantomsection\label{template-thumbnail}
\pandocbounded{\includegraphics[keepaspectratio]{https://packages.typst.org/preview/thumbnails/vercanard-1.0.2-small.webp}}

\section{vercanard}\label{vercanard}

{ 1.0.2 }

A colorful CV template

\href{/app?template=vercanard&version=1.0.2}{Create project in app}

\phantomsection\label{readme}
A colorful resume template for Typst.

The
\href{https://github.com/typst/packages/raw/main/packages/preview/vercanard/1.0.2/template/main.typ}{demo}
file showcases what it is possible to do. You can see the result in
\href{https://github.com/typst/packages/raw/main/packages/preview/vercanard/1.0.2/demo.pdf}{the
corresponding PDF} .

First of all, copy the template to your Typst project, and import it.

\begin{Shaded}
\begin{Highlighting}[]
\NormalTok{\#import "@preview/vercanard:1.0.2": *}
\end{Highlighting}
\end{Shaded}

Then, call the \texttt{\ resume\ } in a global \texttt{\ show\ } rule
function to use it. This function takes a few arguments that we explain
in comments below:

\begin{Shaded}
\begin{Highlighting}[]
\NormalTok{\#show: resume.with(}
\NormalTok{  // The title of your resume, generally your name}
\NormalTok{  name: "Your name",}
\NormalTok{  // The subtitle, which is the position you are looking for most of the time}
\NormalTok{  title: "What you are looking for",}
\NormalTok{  // The accent color to use (here a vibrant yellow)}
\NormalTok{  accent{-}color: rgb("f3bc54"),}
\NormalTok{  // the margins (only used for top and left page margins actually,}
\NormalTok{  // but the other ones are proportional)}
\NormalTok{  margin: 2.6cm,}
\NormalTok{  // The content to put in the right aside block}
\NormalTok{  aside: [}
\NormalTok{    = Contact}

\NormalTok{    // lists in the aside are right aligned}
\NormalTok{    {-} \#link("mailto:example@example.org")}
\NormalTok{    {-} +33 6 66 66 66 66}
\NormalTok{  ]}
\NormalTok{)}

\NormalTok{// And finally the main body of your resume can come here}
\end{Highlighting}
\end{Shaded}

When writing the body, you can use level-1 headings as section titles,
and format an entry with the \texttt{\ entry\ } function (that takes
three content blocks as arguments, for title, description and details).

\begin{Shaded}
\begin{Highlighting}[]
\NormalTok{= Personal projects}

\NormalTok{\#entry[Vercanard][A resume template for Typst][2023 — Typst]}
\end{Highlighting}
\end{Shaded}

This template is under the GPLv3 licence, but resume built using it are
not considered binary derivatives, only output from another program, so
you can keep full copyright on them and chose not to licence them under
a free licence.

\href{/app?template=vercanard&version=1.0.2}{Create project in app}

\subsubsection{How to use}\label{how-to-use}

Click the button above to create a new project using this template in
the Typst app.

You can also use the Typst CLI to start a new project on your computer
using this command:

\begin{verbatim}
typst init @preview/vercanard:1.0.2
\end{verbatim}

\includesvg[width=0.16667in,height=0.16667in]{/assets/icons/16-copy.svg}

\subsubsection{About}\label{about}

\begin{description}
\tightlist
\item[Author :]
\href{https://ana.gelez.xyz}{Ana Gelez}
\item[License:]
GPL-3.0
\item[Current version:]
1.0.2
\item[Last updated:]
October 21, 2024
\item[First released:]
April 2, 2024
\item[Archive size:]
14.3 kB
\href{https://packages.typst.org/preview/vercanard-1.0.2.tar.gz}{\pandocbounded{\includesvg[keepaspectratio]{/assets/icons/16-download.svg}}}
\item[Repository:]
\href{https://github.com/elegaanz/vercanard}{GitHub}
\item[Categor y :]
\begin{itemize}
\tightlist
\item[]
\item
  \pandocbounded{\includesvg[keepaspectratio]{/assets/icons/16-user.svg}}
  \href{https://typst.app/universe/search/?category=cv}{CV}
\end{itemize}
\end{description}

\subsubsection{Where to report issues?}\label{where-to-report-issues}

This template is a project of Ana Gelez . Report issues on
\href{https://github.com/elegaanz/vercanard}{their repository} . You can
also try to ask for help with this template on the
\href{https://forum.typst.app}{Forum} .

Please report this template to the Typst team using the
\href{https://typst.app/contact}{contact form} if you believe it is a
safety hazard or infringes upon your rights.

\phantomsection\label{versions}
\subsubsection{Version history}\label{version-history}

\begin{longtable}[]{@{}ll@{}}
\toprule\noalign{}
Version & Release Date \\
\midrule\noalign{}
\endhead
\bottomrule\noalign{}
\endlastfoot
1.0.2 & October 21, 2024 \\
\href{https://typst.app/universe/package/vercanard/1.0.1/}{1.0.1} & May
23, 2024 \\
\href{https://typst.app/universe/package/vercanard/1.0.0/}{1.0.0} &
April 2, 2024 \\
\end{longtable}

Typst GmbH did not create this template and cannot guarantee correct
functionality of this template or compatibility with any version of the
Typst compiler or app.
