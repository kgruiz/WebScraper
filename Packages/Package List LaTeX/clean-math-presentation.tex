\title{typst.app/universe/package/clean-math-presentation}

\phantomsection\label{banner}
\phantomsection\label{template-thumbnail}
\pandocbounded{\includegraphics[keepaspectratio]{https://packages.typst.org/preview/thumbnails/clean-math-presentation-0.1.0-small.webp}}

\section{clean-math-presentation}\label{clean-math-presentation}

{ 0.1.0 }

A simple and good looking template for mathematical presentations

\href{/app?template=clean-math-presentation&version=0.1.0}{Create
project in app}

\phantomsection\label{readme}
\href{https://github.com/JoshuaLampert/clean-math-presentation/actions/workflows/build.yml}{\pandocbounded{\includesvg[keepaspectratio]{https://github.com/JoshuaLampert/clean-math-presentation/actions/workflows/build.yml/badge.svg}}}
\href{https://github.com/JoshuaLampert/clean-math-presentation}{\pandocbounded{\includegraphics[keepaspectratio]{https://img.shields.io/badge/GitHub-repo-blue}}}
\href{https://opensource.org/licenses/MIT}{\pandocbounded{\includesvg[keepaspectratio]{https://img.shields.io/badge/License-MIT-success.svg}}}

\href{https://typst.app/home/}{Typst} template for presentations built
for simple, efficient use and a clean look using
\href{https://touying-typ.github.io/}{touying} . The template provides a
custom title page, a footer, a header, and built-in support for theorem
blocks and proofs.

\subsection{Usage}\label{usage}

The template is already filled with dummy data, to give users an
impression how it looks like. The paper is obtained by compiling
\texttt{\ main.typ\ } .

\begin{itemize}
\tightlist
\item
  after
  \href{https://github.com/typst/typst?tab=readme-ov-file\#installation}{installing
  Typst} you can conveniently use the following to create a new folder
  containing this project.
\end{itemize}

\begin{Shaded}
\begin{Highlighting}[]
\ExtensionTok{typst}\NormalTok{ init @preview/clean{-}math{-}presentation:0.1.0}
\end{Highlighting}
\end{Shaded}

\begin{itemize}
\tightlist
\item
  edit the data in \texttt{\ main.typ\ } â†'
  \texttt{\ \#show\ template.with({[}your\ data{]})\ }
\end{itemize}

\subsubsection{Parameters of the
Template}\label{parameters-of-the-template}

\begin{itemize}
\tightlist
\item
  \texttt{\ title\ } : Title of the presentation.
\item
  \texttt{\ subtitle\ } : Subtitle of the presentation, optional.
\item
  \texttt{\ short-title\ } : Short version of the presentation to be
  shown in the footer, optional. If not short title is provided, the
  \texttt{\ title\ } will be shown in the footer.
\item
  \texttt{\ date\ } : Date of the presentation.
\item
  \texttt{\ authors\ } : List of names of the authors of the paper. Each
  entry of the list is a dictionary with the following keys:

  \begin{itemize}
  \tightlist
  \item
    \texttt{\ name\ } : Name of the author.
  \item
    \texttt{\ affiliation-id\ } : The ID of the affiliation in
    \texttt{\ affiliations\ } , see below.
  \end{itemize}
\item
  \texttt{\ affiliations\ } : List of affiliations of the authors. Each
  entry of the list is a dictionary with the following keys:

  \begin{itemize}
  \tightlist
  \item
    \texttt{\ id\ } : ID of the affiliation, which is used to link the
    authors to the affiliation, see above.
  \item
    \texttt{\ name\ } : Name of the affiliation.
  \end{itemize}
\item
  \texttt{\ author\ } : The name of the presenting author, which will be
  displayed in the footer of each slide. If the \texttt{\ author\ }
  matches one of the \texttt{\ authors\ } , this name will be underlined
  in the title slide.
\end{itemize}

Other arguments like \texttt{\ align\ } , \texttt{\ progess-bar\ } and
more are available and similar to other templates in touying, especially
the \href{https://touying-typ.github.io/docs/themes/stargazer}{stargazer
theme} . The colorscheme can be adjusted by passing
\texttt{\ config-colors\ } to the \texttt{\ template\ } , e.g.

\begin{Shaded}
\begin{Highlighting}[]
\NormalTok{config{-}colors(}
\NormalTok{  primary: rgb("\#6068d6"),}
\NormalTok{  secondary: rgb("\#2f1971"),}
\NormalTok{)}
\end{Highlighting}
\end{Shaded}

The title page can be created with \texttt{\ \#title-slide\ } . It
accepts optionally a \texttt{\ background\ } , which can be an image or
\texttt{\ none\ } (default) and up to two logos \texttt{\ logo1\ } and
\texttt{\ logo2\ } ( \texttt{\ none\ } by default).

The theme provides different types of slides like
\texttt{\ \#outline-slide\ } , \texttt{\ \#focus-slide\ } ,
\texttt{\ \#ending-slide\ } , and the usual \texttt{\ \#slide\ } .
Additionally,it already provides support for theorems and alike by the
functions \texttt{\ \#theorem\ } , \texttt{\ \#lemma\ } ,
\texttt{\ \#corollary\ } , \texttt{\ \#definition\ } ,
\texttt{\ \#example\ } , and \texttt{\ \#proof\ } .

\subsection{Acknowledgements}\label{acknowledgements}

Some parts of this template are based on the
\href{https://github.com/touying-typ/touying/blob/main/themes/stargazer.typ}{stargazer}
theme from touying.

\subsection{Feedback \& Improvements}\label{feedback-improvements}

If you encounter problems, please open issues. In case you found useful
extensions or improved anything We are also very happy to accept pull
requests.

\href{/app?template=clean-math-presentation&version=0.1.0}{Create
project in app}

\subsubsection{How to use}\label{how-to-use}

Click the button above to create a new project using this template in
the Typst app.

You can also use the Typst CLI to start a new project on your computer
using this command:

\begin{verbatim}
typst init @preview/clean-math-presentation:0.1.0
\end{verbatim}

\includesvg[width=0.16667in,height=0.16667in]{/assets/icons/16-copy.svg}

\subsubsection{About}\label{about}

\begin{description}
\tightlist
\item[Author :]
\href{https://github.com/JoshuaLampert}{Joshua Lampert}
\item[License:]
MIT
\item[Current version:]
0.1.0
\item[Last updated:]
November 21, 2024
\item[First released:]
November 21, 2024
\item[Minimum Typst version:]
0.12.0
\item[Archive size:]
10.3 kB
\href{https://packages.typst.org/preview/clean-math-presentation-0.1.0.tar.gz}{\pandocbounded{\includesvg[keepaspectratio]{/assets/icons/16-download.svg}}}
\item[Repository:]
\href{https://github.com/JoshuaLampert/clean-math-presentation}{GitHub}
\item[Discipline s :]
\begin{itemize}
\tightlist
\item[]
\item
  \href{https://typst.app/universe/search/?discipline=mathematics}{Mathematics}
\item
  \href{https://typst.app/universe/search/?discipline=engineering}{Engineering}
\item
  \href{https://typst.app/universe/search/?discipline=computer-science}{Computer
  Science}
\end{itemize}
\item[Categor y :]
\begin{itemize}
\tightlist
\item[]
\item
  \pandocbounded{\includesvg[keepaspectratio]{/assets/icons/16-presentation.svg}}
  \href{https://typst.app/universe/search/?category=presentation}{Presentation}
\end{itemize}
\end{description}

\subsubsection{Where to report issues?}\label{where-to-report-issues}

This template is a project of Joshua Lampert . Report issues on
\href{https://github.com/JoshuaLampert/clean-math-presentation}{their
repository} . You can also try to ask for help with this template on the
\href{https://forum.typst.app}{Forum} .

Please report this template to the Typst team using the
\href{https://typst.app/contact}{contact form} if you believe it is a
safety hazard or infringes upon your rights.

\phantomsection\label{versions}
\subsubsection{Version history}\label{version-history}

\begin{longtable}[]{@{}ll@{}}
\toprule\noalign{}
Version & Release Date \\
\midrule\noalign{}
\endhead
\bottomrule\noalign{}
\endlastfoot
0.1.0 & November 21, 2024 \\
\end{longtable}

Typst GmbH did not create this template and cannot guarantee correct
functionality of this template or compatibility with any version of the
Typst compiler or app.
