\title{typst.app/universe/package/ez-today}

\phantomsection\label{banner}
\section{ez-today}\label{ez-today}

{ 0.1.0 }

Simply displays the full current date.

\phantomsection\label{readme}
Simply displays the current date with easy to use customization.

\subsection{Included languages}\label{included-languages}

German, English, French and Italian months can be used out of the box.
If you want to use a language that is not included, you can easily add
it yourself. This is shown in the customization section below.

\subsection{Usage}\label{usage}

The usage is very simple, because there is only the \texttt{\ today()\ }
function.

\begin{Shaded}
\begin{Highlighting}[]
\NormalTok{\#import "@preview/ez{-}today:0.1.0"}

\NormalTok{// To get the current date use this}
\NormalTok{\#ez{-}today.today()}
\end{Highlighting}
\end{Shaded}

\subsection{Reference}\label{reference}

\subsubsection{\texorpdfstring{\texttt{\ today\ }}{ today }}\label{today}

Prints the current date with given arguments.

\begin{Shaded}
\begin{Highlighting}[]
\NormalTok{\#let today(}
\NormalTok{  lang: "de",}
\NormalTok{  format: "d. M Y",}
\NormalTok{  custom{-}months: ()}
\NormalTok{) = \{ .. \}}
\end{Highlighting}
\end{Shaded}

\textbf{Arguments:}

\begin{itemize}
\tightlist
\item
  \texttt{\ lang\ } : {[} \texttt{\ str\ } {]} â€'' Select one of the
  included languages (de, en, fr, it).
\item
  \texttt{\ format\ } : {[} \texttt{\ str\ } {]} â€'' Specify the output
  format.
\item
  \texttt{\ custom-months\ } : {[} \texttt{\ array\ } {]} of {[}
  \texttt{\ str\ } {]} â€'' Use custom names for each month. This array
  must have 12 entries. If this is used, the \texttt{\ lang\ } argument
  does nothing.
\end{itemize}

\subsection{Customization}\label{customization}

The default output prints the full current date with German months like
this:

\begin{Shaded}
\begin{Highlighting}[]
\NormalTok{\#ez{-}today.today()   // 11. Oktober 2024}
\end{Highlighting}
\end{Shaded}

You can choose one of the included languages with the \texttt{\ lang\ }
argument:

\begin{Shaded}
\begin{Highlighting}[]
\NormalTok{\#ez{-}today.today(lang: "en")   // 11. October 2024}
\NormalTok{\#ez{-}today.today(lang: "fr")   // 11. Octobre 2024}
\NormalTok{\#ez{-}today.today(lang: "it")   // 11. Ottobre 2024}
\end{Highlighting}
\end{Shaded}

You can also change the format of the output with the
\texttt{\ format\ } argument. Pass any string you want here, but know
that the following characters will be replaced with the following:

\begin{itemize}
\tightlist
\item
  \texttt{\ d\ } â€'' The current day as a decimal
\item
  \texttt{\ m\ } â€'' The current month as a decimal ( \texttt{\ lang\ }
  argument does nothing)
\item
  \texttt{\ M\ } â€'' The current month as a string with either the
  selected language or the custom array
\item
  \texttt{\ y\ } â€'' The current year as a decimal with the last two
  numbers
\item
  \texttt{\ Y\ } â€'' The current year as a decimal
\end{itemize}

Here are some examples:

\begin{Shaded}
\begin{Highlighting}[]
\NormalTok{\#ez{-}today.today(lang: "en", format: "M d Y")    // October 11 2024}
\NormalTok{\#ez{-}today.today(format: "m{-}d{-}y")                // 10{-}11{-}24}
\NormalTok{\#ez{-}today.today(format: "d/m")                  // 11/10}
\NormalTok{\#ez{-}today.today(format: "d.m.Y")                // 11.10.2024}
\end{Highlighting}
\end{Shaded}

Use the \texttt{\ custom-months\ } argument to give each month a custom
name. You can add a new language or use short terms for each month.

\begin{Shaded}
\begin{Highlighting}[]
\NormalTok{// Defining some custom names}
\NormalTok{\#let my{-}months = ("Jan", "Feb", "Mar", "Apr", "May", "Jun", "Jul", "Aug", "Sep", "Oct", "Nov", "Dec")}
\NormalTok{// Get current date with custom names}
\NormalTok{\#ez{-}today.today(custom{-}months: my{-}months, format: "M{-}y")    // Oct{-}24}
\end{Highlighting}
\end{Shaded}

\subsubsection{How to add}\label{how-to-add}

Copy this into your project and use the import as \texttt{\ ez-today\ }

\begin{verbatim}
#import "@preview/ez-today:0.1.0"
\end{verbatim}

\includesvg[width=0.16667in,height=0.16667in]{/assets/icons/16-copy.svg}

Check the docs for
\href{https://typst.app/docs/reference/scripting/\#packages}{more
information on how to import packages} .

\subsubsection{About}\label{about}

\begin{description}
\tightlist
\item[Author :]
Carlo Schafflik
\item[License:]
MIT
\item[Current version:]
0.1.0
\item[Last updated:]
October 11, 2024
\item[First released:]
October 11, 2024
\item[Archive size:]
2.42 kB
\href{https://packages.typst.org/preview/ez-today-0.1.0.tar.gz}{\pandocbounded{\includesvg[keepaspectratio]{/assets/icons/16-download.svg}}}
\item[Repository:]
\href{https://github.com/CarloSchafflik12/typst-ez-today}{GitHub}
\end{description}

\subsubsection{Where to report issues?}\label{where-to-report-issues}

This package is a project of Carlo Schafflik . Report issues on
\href{https://github.com/CarloSchafflik12/typst-ez-today}{their
repository} . You can also try to ask for help with this package on the
\href{https://forum.typst.app}{Forum} .

Please report this package to the Typst team using the
\href{https://typst.app/contact}{contact form} if you believe it is a
safety hazard or infringes upon your rights.

\phantomsection\label{versions}
\subsubsection{Version history}\label{version-history}

\begin{longtable}[]{@{}ll@{}}
\toprule\noalign{}
Version & Release Date \\
\midrule\noalign{}
\endhead
\bottomrule\noalign{}
\endlastfoot
0.1.0 & October 11, 2024 \\
\end{longtable}

Typst GmbH did not create this package and cannot guarantee correct
functionality of this package or compatibility with any version of the
Typst compiler or app.
