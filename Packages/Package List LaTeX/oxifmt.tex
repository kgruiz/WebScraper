\title{typst.app/universe/package/oxifmt}

\phantomsection\label{banner}
\section{oxifmt}\label{oxifmt}

{ 0.2.1 }

Convenient Rust-like string formatting in Typst

\phantomsection\label{readme}
A Typst library that brings convenient string formatting and
interpolation through the \texttt{\ strfmt\ } function. Its syntax is
taken directly from Rust’s \texttt{\ format!\ } syntax, so feel free
to read its page for more information (
\url{https://doc.rust-lang.org/std/fmt/} ); however, this README should
have enough information and examples for all expected uses of the
library. Only a few things aren’t supported from the Rust syntax, such
as the \texttt{\ p\ } (pointer) format type, or the \texttt{\ .*\ }
precision specifier.

A few extras (beyond the Rust-like syntax) will be added over time,
though (feel free to drop suggestions at the repository:
\url{https://github.com/PgBiel/typst-oxifmt} ). The first “extra� so
far is the \texttt{\ fmt-decimal-separator:\ "string"\ } parameter,
which lets you customize the decimal separator for decimal numbers
(floats) inserted into strings. E.g.
\texttt{\ strfmt("Result:\ \{\}",\ 5.8,\ fmt-decimal-separator:\ ",")\ }
will return the string \texttt{\ "Result:\ 5,8"\ } (comma instead of
dot). See more below.

\textbf{Compatible with:} \href{https://github.com/typst/typst}{Typst}
v0.4.0+

\subsection{Table of Contents}\label{table-of-contents}

\begin{itemize}
\tightlist
\item
  \href{https://github.com/typst/packages/raw/main/packages/preview/oxifmt/0.2.1/\#usage}{Usage}

  \begin{itemize}
  \tightlist
  \item
    \href{https://github.com/typst/packages/raw/main/packages/preview/oxifmt/0.2.1/\#formatting-options}{Formatting
    options}
  \item
    \href{https://github.com/typst/packages/raw/main/packages/preview/oxifmt/0.2.1/\#examples}{Examples}
  \item
    \href{https://github.com/typst/packages/raw/main/packages/preview/oxifmt/0.2.1/\#grammar}{Grammar}
  \end{itemize}
\item
  \href{https://github.com/typst/packages/raw/main/packages/preview/oxifmt/0.2.1/\#issues-and-contributing}{Issues
  and Contributing}
\item
  \href{https://github.com/typst/packages/raw/main/packages/preview/oxifmt/0.2.1/\#testing}{Testing}
\item
  \href{https://github.com/typst/packages/raw/main/packages/preview/oxifmt/0.2.1/\#changelog}{Changelog}
\item
  \href{https://github.com/typst/packages/raw/main/packages/preview/oxifmt/0.2.1/\#license}{License}
\end{itemize}

\subsection{Usage}\label{usage}

You can use this library through Typst’s package manager (for Typst
v0.6.0+):

\begin{Shaded}
\begin{Highlighting}[]
\NormalTok{\#import "@preview/oxifmt:0.2.1": strfmt}
\end{Highlighting}
\end{Shaded}

For older Typst versions, download the \texttt{\ oxifmt.typ\ } file
either from Releases or directly from the repository. Then, move it to
your project’s folder, and write at the top of your Typst file(s):

\begin{Shaded}
\begin{Highlighting}[]
\NormalTok{\#import "oxifmt.typ": strfmt}
\end{Highlighting}
\end{Shaded}

Doing the above will give you access to the main function provided by
this library ( \texttt{\ strfmt\ } ), which accepts a format string,
followed by zero or more replacements to insert in that string
(according to \texttt{\ \{...\}\ } formats inserted in that string), an
optional \texttt{\ fmt-decimal-separator\ } parameter, and returns the
formatted string, as described below.

Its syntax is almost identical to Rust’s \texttt{\ format!\ } (as
specified here: \url{https://doc.rust-lang.org/std/fmt/} ). You can
escape formats by duplicating braces ( \texttt{\ \{\{\ } and
\texttt{\ \}\}\ } become \texttt{\ \{\ } and \texttt{\ \}\ } ). Here’s
an example (see more examples in the file
\texttt{\ tests/strfmt-tests.typ\ } ):

\begin{Shaded}
\begin{Highlighting}[]
\NormalTok{\#import "@preview/oxifmt:0.2.1": strfmt}

\NormalTok{\#let s = strfmt("I\textquotesingle{}m \{\}. I have \{num\} cars. I\textquotesingle{}m \{0\}. \{\} is \{\{cool\}\}.", "John", "Carl", num: 10)}
\NormalTok{\#assert.eq(s, "I\textquotesingle{}m John. I have 10 cars. I\textquotesingle{}m John. Carl is \{cool\}.")}
\end{Highlighting}
\end{Shaded}

Note that \texttt{\ \{\}\ } extracts positional arguments after the
string sequentially (the first \texttt{\ \{\}\ } extracts the first one,
the second \texttt{\ \{\}\ } extracts the second one, and so on), while
\texttt{\ \{0\}\ } , \texttt{\ \{1\}\ } , etc. will always extract the
first, the second etc. positional arguments after the string.
Additionally, \texttt{\ \{bananas\}\ } will extract the named argument
“bananas�.

\subsubsection{Formatting options}\label{formatting-options}

You can use \texttt{\ \{:spec\}\ } to customize your output. See the
Rust docs linked above for more info, but a summary is below.

(You may also want to check out the examples at
\href{https://github.com/typst/packages/raw/main/packages/preview/oxifmt/0.2.1/\#examples}{Examples}
.)

\begin{itemize}
\tightlist
\item
  Adding a \texttt{\ ?\ } at the end of \texttt{\ spec\ } (that is,
  writing e.g. \texttt{\ \{0:?\}\ } ) will call \texttt{\ repr()\ } to
  stringify your argument, instead of \texttt{\ str()\ } . Note that
  this only has an effect if your argument is a string, an integer, a
  float or a \texttt{\ label()\ } /
  \texttt{\ \textless{}label\textgreater{}\ } - for all other types
  (such as booleans or elements), \texttt{\ repr()\ } is always called
  (as \texttt{\ str()\ } is unsupported for those).

  \begin{itemize}
  \tightlist
  \item
    For strings, \texttt{\ ?\ } (and thus \texttt{\ repr()\ } ) has the
    effect of printing them with double quotes. For floats, this ensures
    a \texttt{\ .0\ } appears after it, even if it doesn’t have
    decimal digits. For integers, this doesn’t change anything.
    Finally, for labels, the \texttt{\ \textless{}label\textgreater{}\ }
    (with \texttt{\ ?\ } ) is printed as
    \texttt{\ \textless{}label\textgreater{}\ } instead of
    \texttt{\ label\ } .
  \item
    \textbf{TIP:} Prefer to always use \texttt{\ ?\ } when you’re
    inserting something that isn’t a string, number or label, in order
    to ensure consistent results even if the library eventually changes
    the non- \texttt{\ ?\ } representation.
  \end{itemize}
\item
  After the \texttt{\ :\ } , add e.g. \texttt{\ \_\textless{}8\ } to
  align the string to the left, padding it with as many \texttt{\ \_\ }
  s as necessary for it to be at least \texttt{\ 8\ } characters long
  (for example). Replace \texttt{\ \textless{}\ } by
  \texttt{\ \textgreater{}\ } for right alignment, or \texttt{\ \^{}\ }
  for center alignment. (If the \texttt{\ \_\ } is omitted, it defaults
  to ’ ’ (aligns with spaces).)

  \begin{itemize}
  \tightlist
  \item
    If you prefer to specify the minimum width (the \texttt{\ 8\ }
    there) as a separate argument to \texttt{\ strfmt\ } instead, you
    can specify \texttt{\ argument\$\ } in place of the width, which
    will extract it from the integer at \texttt{\ argument\ } . For
    example, \texttt{\ \_\^{}3\$\ } will center align the output with
    \texttt{\ \_\ } s, where the minimum width desired is specified by
    the fourth positional argument (index \texttt{\ 3\ } ), as an
    integer. This means that a call such as
    \texttt{\ strfmt("\{:\_\^{}3\$\}",\ 1,\ 2,\ 3,\ 4)\ } would produce
    \texttt{\ "\_\_1\_\_"\ } , as \texttt{\ 3\$\ } would evaluate to
    \texttt{\ 4\ } (the value at the fourth positional argument/index
    \texttt{\ 3\ } ). Similarly, \texttt{\ named\$\ } would take the
    width from the argument with name \texttt{\ named\ } , if it is an
    integer (otherwise, error).
  \end{itemize}
\item
  \textbf{For numbers:}

  \begin{itemize}
  \tightlist
  \item
    Specify \texttt{\ +\ } after the \texttt{\ :\ } to ensure zero or
    positive numbers are prefixed with \texttt{\ +\ } before them
    (instead of having no sign). \texttt{\ -\ } is also accepted but
    ignored (negative numbers always specify their sign anyways).
  \item
    Use something like \texttt{\ :09\ } to add zeroes to the left of the
    number until it has at least 9 digits / characters.

    \begin{itemize}
    \tightlist
    \item
      The \texttt{\ 9\ } here is also a width, so the same comment from
      before applies (you can add \texttt{\ \$\ } to take it from an
      argument to the \texttt{\ strfmt\ } function).
    \end{itemize}
  \item
    Use \texttt{\ :.5\ } to ensure your float is represented with 5
    decimal digits of precision (zeroes are added to the right if
    needed; otherwise, it is rounded, \textbf{not truncated} ).

    \begin{itemize}
    \tightlist
    \item
      Note that floating point inaccuracies can be sometimes observed
      here, which is an unfortunate current limitation.
    \item
      Similarly to \texttt{\ width\ } , the precision can also be
      specified via an argument with the \texttt{\ \$\ } syntax:
      \texttt{\ .5\$\ } will take the precision from the integer at
      argument number 5 (the sixth one), while \texttt{\ .test\$\ } will
      take it from the argument named \texttt{\ test\ } .
    \end{itemize}
  \item
    \textbf{Integers only:} Add \texttt{\ x\ } (lowercase hex) or
    \texttt{\ X\ } (uppercase) at the end of the \texttt{\ spec\ } to
    convert the number to hexadecimal. Also, \texttt{\ b\ } will convert
    it to binary, while \texttt{\ o\ } will convert to octal.

    \begin{itemize}
    \tightlist
    \item
      Specify a hashtag, e.g. \texttt{\ \#x\ } or \texttt{\ \#b\ } , to
      prepend the corresponding base prefix to the base-converted
      number, e.g. \texttt{\ 0xABC\ } instead of \texttt{\ ABC\ } .
    \end{itemize}
  \item
    Add \texttt{\ e\ } or \texttt{\ E\ } at the end of the
    \texttt{\ spec\ } to ensure the number is represented in scientific
    notation (with \texttt{\ e\ } or \texttt{\ E\ } as the exponent
    separator, respectively).
  \item
    For decimal numbers (floats), you can specify
    \texttt{\ fmt-decimal-separator:\ ","\ } to \texttt{\ strfmt\ } to
    have the decimal separator be a comma instead of a dot, for example.

    \begin{itemize}
    \tightlist
    \item
      To have this be the default, you can alias \texttt{\ strfmt\ } ,
      such as using
      \texttt{\ \#let\ strfmt\ =\ strfmt.with(fmt-decimal-separator:\ ",")\ }
      .
    \end{itemize}
  \item
    Number spec arguments (such as \texttt{\ .5\ } ) are ignored when
    the argument is not a number, but e.g. a string, even if it looks
    like a number (such as \texttt{\ "5"\ } ).
  \end{itemize}
\item
  Note that all spec arguments above \textbf{have to be specified in
  order} - if you mix up the order, it won’t work properly!

  \begin{itemize}
  \tightlist
  \item
    Check the grammar below for the proper order, but, in summary: fill
    (character) with align ( \texttt{\ \textless{}\ } ,
    \texttt{\ \textgreater{}\ } or \texttt{\ \^{}\ } ) -\textgreater{}
    sign ( \texttt{\ +\ } or \texttt{\ -\ } ) -\textgreater{}
    \texttt{\ \#\ } -\textgreater{} \texttt{\ 0\ } (for 0 left-padding
    of numbers) -\textgreater{} width (e.g. \texttt{\ 8\ } from
    \texttt{\ 08\ } or \texttt{\ 9\ } from \texttt{\ -\textless{}9\ } )
    -\textgreater{} \texttt{\ .precision\ } -\textgreater{} spec type (
    \texttt{\ ?\ } , \texttt{\ x\ } , \texttt{\ X\ } , \texttt{\ b\ } ,
    \texttt{\ o\ } , \texttt{\ e\ } , \texttt{\ E\ } )).
  \end{itemize}
\end{itemize}

Some examples:

\begin{Shaded}
\begin{Highlighting}[]
\NormalTok{\#import "@preview/oxifmt:0.2.1": strfmt}

\NormalTok{\#let s1 = strfmt("\{0:?\}, \{test:+012e\}, \{1:{-}\textless{}\#8x\}", "hi", {-}74, test: 569.4)}
\NormalTok{\#assert.eq(s1, "\textbackslash{}"hi\textbackslash{}", +00005.694e2, {-}0x4a{-}{-}{-}")}

\NormalTok{\#let s2 = strfmt("\{:\_\textgreater{}+11.5\}", 59.4)}
\NormalTok{\#assert.eq(s2, "\_\_+59.40000")}

\NormalTok{\#let s3 = strfmt("Dict: \{:!\textless{}10?\}", (a: 5))}
\NormalTok{\#assert.eq(s3, "Dict: (a: 5)!!!!")}
\end{Highlighting}
\end{Shaded}

\subsubsection{Examples}\label{examples}

\begin{itemize}
\tightlist
\item
  \textbf{Inserting labels, text and numbers into strings:}
\end{itemize}

\begin{Shaded}
\begin{Highlighting}[]
\NormalTok{\#import "@preview/oxifmt:0.2.1": strfmt}

\NormalTok{\#let s = strfmt("First: \{\}, Second: \{\}, Fourth: \{3\}, Banana: \{banana\} (brackets: \{\{escaped\}\})", 1, 2.1, 3, label("four"), banana: "Banana!!")}
\NormalTok{\#assert.eq(s, "First: 1, Second: 2.1, Fourth: four, Banana: Banana!! (brackets: \{escaped\})")}
\end{Highlighting}
\end{Shaded}

\begin{itemize}
\tightlist
\item
  \textbf{Forcing \texttt{\ repr()\ } with \texttt{\ \{:?\}\ }} (which
  adds quotes around strings, and other things - basically represents a
  Typst value):
\end{itemize}

\begin{Shaded}
\begin{Highlighting}[]
\NormalTok{\#import "@preview/oxifmt:0.2.1": strfmt}

\NormalTok{\#let s = strfmt("The value is: \{:?\} | Also the label is \{:?\}", "something", label("label"))}
\NormalTok{\#assert.eq(s, "The value is: \textbackslash{}"something\textbackslash{}" | Also the label is \textless{}label\textgreater{}")}
\end{Highlighting}
\end{Shaded}

\begin{itemize}
\tightlist
\item
  \textbf{Inserting other types than numbers and strings} (for now, they
  will always use \texttt{\ repr()\ } , even without
  \texttt{\ \{...:?\}\ } , although that is more explicit):
\end{itemize}

\begin{Shaded}
\begin{Highlighting}[]
\NormalTok{\#import "@preview/oxifmt:0.2.1": strfmt}

\NormalTok{\#let s = strfmt("Values: \{:?\}, \{1:?\}, \{stuff:?\}", (test: 500), ("a", 5.1), stuff: [a])}
\NormalTok{\#assert.eq(s, "Values: (test: 500), (\textbackslash{}"a\textbackslash{}", 5.1), [a]")}
\end{Highlighting}
\end{Shaded}

\begin{itemize}
\tightlist
\item
  \textbf{Padding to a certain width with characters:} Use
  \texttt{\ \{:x\textless{}8\}\ } , where \texttt{\ x\ } is the
  \textbf{character to pad with} (e.g. space or \texttt{\ \_\ } , but
  can be anything), \texttt{\ \textless{}\ } is the \textbf{alignment of
  the original text} relative to the padding (can be
  \texttt{\ \textless{}\ } for left aligned (padding goes to the right),
  \texttt{\ \textgreater{}\ } for right aligned (padded to its left) and
  \texttt{\ \^{}\ } for center aligned (padded at both left and right)),
  and \texttt{\ 8\ } is the \textbf{desired total width} (padding will
  add enough characters to reach this width; if the replacement string
  already has this width, no padding will be added):
\end{itemize}

\begin{Shaded}
\begin{Highlighting}[]
\NormalTok{\#import "@preview/oxifmt:0.2.1": strfmt}

\NormalTok{\#let s = strfmt("Left5 \{:{-}\textless{}5\}, Right6 \{:=\textgreater{}6\}, Center10 \{centered: \^{}10?\}, Left3 \{tleft:\_\textless{}3\}", "xx", 539, tleft: "okay", centered: [a])}
\NormalTok{\#assert.eq(s, "Left5 xx{-}{-}{-}, Right6 ===539, Center10     [a]    , Left3 okay")}
\NormalTok{// note how \textquotesingle{}okay\textquotesingle{} didn\textquotesingle{}t suffer any padding at all (it already had at least the desired total width).}
\end{Highlighting}
\end{Shaded}

\begin{itemize}
\tightlist
\item
  \textbf{Padding numbers with zeroes to the left:} It’s a similar
  functionality to the above, however you write \texttt{\ \{:08\}\ } for
  8 characters (for instance) - note that any characters in the
  number’s representation matter for width (including sign, dot and
  decimal part):
\end{itemize}

\begin{Shaded}
\begin{Highlighting}[]
\NormalTok{\#import "@preview/oxifmt:0.2.1": strfmt}

\NormalTok{\#let s = strfmt("Left{-}padded7 numbers: \{:07\} \{:07\} \{:07\} \{3:07\}", 123, {-}344, 44224059, 45.32)}
\NormalTok{\#assert.eq(s, "Left{-}padded7 numbers: 0000123 {-}000344 44224059 0045.32")}
\end{Highlighting}
\end{Shaded}

\begin{itemize}
\tightlist
\item
  \textbf{Defining padding-to width using parameters, not literals:} If
  you want the desired replacement width (the \texttt{\ 8\ } in
  \texttt{\ \{:08\}\ } or \texttt{\ \{:\ \^{}8\}\ } ) to be passed via
  parameter (instead of being hardcoded into the format string), you can
  specify \texttt{\ parameter\$\ } in place of the width, e.g.
  \texttt{\ \{:02\$\}\ } to take it from the third positional parameter,
  or \texttt{\ \{:a\textgreater{}banana\$\}\ } to take it from the
  parameter named \texttt{\ banana\ } - note that the chosen parameter
  \textbf{must be an integer} (desired total width):
\end{itemize}

\begin{Shaded}
\begin{Highlighting}[]
\NormalTok{\#import "@preview/oxifmt:0.2.1": strfmt}

\NormalTok{\#let s = strfmt("Padding depending on parameter: \{0:02$\} and \{0:a\textgreater{}banana$\}", 432, 0, 5, banana: 9)}
\NormalTok{\#assert.eq(s, "Padding depending on parameter: 00432 aaaaaa432")  // widths 5 and 9}
\end{Highlighting}
\end{Shaded}

\begin{itemize}
\tightlist
\item
  \textbf{Displaying \texttt{\ +\ } on positive numbers:} Just add a
  \texttt{\ +\ } at the “beginning�, i.e., before the
  \texttt{\ \#0\ } (if either is there), or after the custom fill and
  align (if it’s there and not \texttt{\ 0\ } - see
  \href{https://github.com/typst/packages/raw/main/packages/preview/oxifmt/0.2.1/\#grammar}{Grammar}
  for the exact positioning), like so:
\end{itemize}

\begin{Shaded}
\begin{Highlighting}[]
\NormalTok{\#import "@preview/oxifmt:0.2.1": strfmt}

\NormalTok{\#let s = strfmt("Some numbers: \{:+\} \{:+08\}; With fill and align: \{:\_\textless{}+8\}; Negative (no{-}op): \{neg:+\}", 123, 456, 4444, neg: {-}435)}
\NormalTok{\#assert.eq(s, "Some numbers: +123 +0000456; With fill and align: +4444\_\_\_; Negative (no{-}op): {-}435")}
\end{Highlighting}
\end{Shaded}

\begin{itemize}
\tightlist
\item
  \textbf{Converting numbers to bases 2, 8 and 16:} Use one of the
  following specifier types (i.e., characters which always go at the
  very end of the format): \texttt{\ b\ } (binary), \texttt{\ o\ }
  (octal), \texttt{\ x\ } (lowercase hexadecimal) or \texttt{\ X\ }
  (uppercase hexadecimal). You can also add a \texttt{\ \#\ } between
  \texttt{\ +\ } and \texttt{\ 0\ } (see the exact position at the
  \href{https://github.com/typst/packages/raw/main/packages/preview/oxifmt/0.2.1/\#grammar}{Grammar}
  ) to display a \textbf{base prefix} before the number (i.e.
  \texttt{\ 0b\ } for binary, \texttt{\ 0o\ } for octal and
  \texttt{\ 0x\ } for hexadecimal):
\end{itemize}

\begin{Shaded}
\begin{Highlighting}[]
\NormalTok{\#import "@preview/oxifmt:0.2.1": strfmt}

\NormalTok{\#let s = strfmt("Bases (10, 2, 8, 16(l), 16(U):) \{0\} \{0:b\} \{0:o\} \{0:x\} \{0:X\} | W/ prefixes and modifiers: \{0:\#b\} \{0:+\#09o\} \{0:\_\textgreater{}+\#9X\}", 124)}
\NormalTok{\#assert.eq(s, "Bases (10, 2, 8, 16(l), 16(U):) 124 1111100 174 7c 7C | W/ prefixes and modifiers: 0b1111100 +0o000174 \_\_\_\_+0x7C")}
\end{Highlighting}
\end{Shaded}

\begin{itemize}
\tightlist
\item
  \textbf{Picking float precision (right-extending with zeroes):} Add,
  at the end of the format (just before the spec type (such as
  \texttt{\ ?\ } ), if there’s any), either \texttt{\ .precision\ }
  (hardcoded, e.g. \texttt{\ .8\ } for 8 decimal digits) or
  \texttt{\ .parameter\$\ } (taking the precision value from the
  specified parameter, like with \texttt{\ width\ } ):
\end{itemize}

\begin{Shaded}
\begin{Highlighting}[]
\NormalTok{\#import "@preview/oxifmt:0.2.1": strfmt}

\NormalTok{\#let s = strfmt("\{0:.8\} \{0:.2$\} \{0:.potato$\}", 1.234, 0, 2, potato: 5)}
\NormalTok{\#assert.eq(s, "1.23400000 1.23 1.23400")}
\end{Highlighting}
\end{Shaded}

\begin{itemize}
\tightlist
\item
  \textbf{Scientific notation:} Use \texttt{\ e\ } (lowercase) or
  \texttt{\ E\ } (uppercase) as specifier types (can be combined with
  precision):
\end{itemize}

\begin{Shaded}
\begin{Highlighting}[]
\NormalTok{\#import "@preview/oxifmt:0.2.1": strfmt}

\NormalTok{\#let s = strfmt("\{0:e\} \{0:E\} \{0:+.9e\} | \{1:e\} | \{2:.4E\}", 124.2312, 50, {-}0.02)}
\NormalTok{\#assert.eq(s, "1.242312e2 1.242312E2 +1.242312000e2 | 5e1 | {-}2.0000E{-}2")}
\end{Highlighting}
\end{Shaded}

\begin{itemize}
\tightlist
\item
  \textbf{Customizing the decimal separator on floats:} Just specify
  \texttt{\ fmt-decimal-separator:\ ","\ } (comma as an example):
\end{itemize}

\begin{Shaded}
\begin{Highlighting}[]
\NormalTok{\#import "@preview/oxifmt:0.2.1": strfmt}

\NormalTok{\#let s = strfmt("\{0\} \{0:.6\} \{0:.5e\}", 1.432, fmt{-}decimal{-}separator: ",")}
\NormalTok{\#assert.eq(s, "1,432 1,432000 1,43200e0")}
\end{Highlighting}
\end{Shaded}

\subsubsection{Grammar}\label{grammar}

Here’s the grammar specification for valid format \texttt{\ spec\ } s
(in \texttt{\ \{name:spec\}\ } ), which is basically Rust’s format:

\begin{verbatim}
format_spec := [[fill]align][sign]['#']['0'][width]['.' precision]type
fill := character
align := '<' | '^' | '>'
sign := '+' | '-'
width := count
precision := count | '*'
type := '' | '?' | 'x?' | 'X?' | identifier
count := parameter | integer
parameter := argument '$'
\end{verbatim}

Note, however, that precision of type \texttt{\ .*\ } is not supported
yet and will raise an error.

\subsection{Issues and Contributing}\label{issues-and-contributing}

Please report any issues or send any contributions (through pull
requests) to the repository at
\url{https://github.com/PgBiel/typst-oxifmt}

\subsection{Testing}\label{testing}

If you wish to contribute, you may clone the repository and test this
package with the following commands (from the project root folder):

\begin{Shaded}
\begin{Highlighting}[]
\FunctionTok{git}\NormalTok{ clone https://github.com/PgBiel/typst{-}oxifmt}
\BuiltInTok{cd}\NormalTok{ typst{-}oxifmt/tests}
\ExtensionTok{typst}\NormalTok{ c strfmt{-}tests.typ }\AttributeTok{{-}{-}root}\NormalTok{ ..}
\end{Highlighting}
\end{Shaded}

The tests succeeded if you received no error messages from the last
command (please ensure you’re using a supported Typst version).

\subsection{Changelog}\label{changelog}

\subsubsection{v0.2.1}\label{v0.2.1}

\begin{itemize}
\tightlist
\item
  Fixed formatting of UTF-8 strings. Before, strings with multi-byte
  UTF-8 codepoints would cause formatting inconsistencies or even
  crashes. (
  \href{https://github.com/PgBiel/typst-oxifmt/issues/6}{Issue \#6} )
\item
  Fixed an inconsistency in negative number formatting. Now, it will
  always print a regular hyphen (e.g. ‘-2’), which is consistent
  with Rust’s behavior; before, it would occasionally print a minus
  sign instead (as observed in a comment to
  \href{https://github.com/PgBiel/typst-oxifmt/issues/4}{Issue \#4} ).
\item
  Added compatibility with Typst 0.8.0’s new type system.
\end{itemize}

\subsubsection{v0.2.0}\label{v0.2.0}

\begin{itemize}
\tightlist
\item
  The package’s name is now \texttt{\ oxifmt\ } !
\item
  \texttt{\ oxifmt:0.2.0\ } is now available through Typst’s Package
  Manager! You can now write
  \texttt{\ \#import\ "@preview/oxifmt:0.2.0":\ strfmt\ } to use the
  library.
\item
  Greatly improved the README, adding a section for common examples.
\item
  Fixed negative numbers being formatted with two minus signs.
\item
  Fixed custom precision of floats not working when they are exact
  integers.
\end{itemize}

\subsubsection{v0.1.0}\label{v0.1.0}

\begin{itemize}
\tightlist
\item
  Initial release, added \texttt{\ strfmt\ } .
\end{itemize}

\subsection{License}\label{license}

MIT-0 license (see the \texttt{\ LICENSE\ } file).

\subsubsection{How to add}\label{how-to-add}

Copy this into your project and use the import as \texttt{\ oxifmt\ }

\begin{verbatim}
#import "@preview/oxifmt:0.2.1"
\end{verbatim}

\includesvg[width=0.16667in,height=0.16667in]{/assets/icons/16-copy.svg}

Check the docs for
\href{https://typst.app/docs/reference/scripting/\#packages}{more
information on how to import packages} .

\subsubsection{About}\label{about}

\begin{description}
\tightlist
\item[Author :]
\href{https://github.com/PgBiel}{PgBiel}
\item[License:]
MIT-0
\item[Current version:]
0.2.1
\item[Last updated:]
May 6, 2024
\item[First released:]
August 2, 2023
\item[Archive size:]
12.0 kB
\href{https://packages.typst.org/preview/oxifmt-0.2.1.tar.gz}{\pandocbounded{\includesvg[keepaspectratio]{/assets/icons/16-download.svg}}}
\item[Repository:]
\href{https://github.com/PgBiel/typst-oxifmt}{GitHub}
\end{description}

\subsubsection{Where to report issues?}\label{where-to-report-issues}

This package is a project of PgBiel . Report issues on
\href{https://github.com/PgBiel/typst-oxifmt}{their repository} . You
can also try to ask for help with this package on the
\href{https://forum.typst.app}{Forum} .

Please report this package to the Typst team using the
\href{https://typst.app/contact}{contact form} if you believe it is a
safety hazard or infringes upon your rights.

\phantomsection\label{versions}
\subsubsection{Version history}\label{version-history}

\begin{longtable}[]{@{}ll@{}}
\toprule\noalign{}
Version & Release Date \\
\midrule\noalign{}
\endhead
\bottomrule\noalign{}
\endlastfoot
0.2.1 & May 6, 2024 \\
\href{https://typst.app/universe/package/oxifmt/0.2.0/}{0.2.0} & August
2, 2023 \\
\end{longtable}

Typst GmbH did not create this package and cannot guarantee correct
functionality of this package or compatibility with any version of the
Typst compiler or app.
