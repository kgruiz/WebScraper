\title{typst.app/universe/package/keyle}

\phantomsection\label{banner}
\section{keyle}\label{keyle}

{ 0.2.0 }

This package provides a simple way to style keyboard shortcuts in your
documentation.

\phantomsection\label{readme}
\href{https://raw.githubusercontent.com/magicwenli/keyle/main/doc/keyle.pdf}{\pandocbounded{\includegraphics[keepaspectratio]{https://img.shields.io/website?down_message=offline&label=manual&up_color=007aff&up_message=online&url=https://raw.githubusercontent.com/magicwenli/keyle/main/doc/keyle.pdf}}}
\href{https://github.com/magicwenli/keyle/blob/main/LICENSE}{\pandocbounded{\includegraphics[keepaspectratio]{https://img.shields.io/badge/license-MIT-brightgreen}}}

A simple way to style keyboard shortcuts in your documentation.

This package was inspired by
\href{https://auth0.github.io/kbd/}{auth0/kbd} and
\href{https://github.com/dogezen/badgery}{dogezen/badgery} . Also thanks
to \href{https://github.com/tweh/menukeys}{tweh/menukeys} â€`` A LaTeX
package for menu keys generation.

Document generating using
\href{https://github.com/jneug/typst-mantys}{jneug/typst-mantys} .

Send them respect and love.

\subsection{Usage}\label{usage}

Please see the
\href{https://github.com/magicwenli/keyle/blob/main/doc/keyle.pdf}{keyle.pdf}
for more documentation.

\texttt{\ keyle\ } is imported using:

\begin{Shaded}
\begin{Highlighting}[]
\NormalTok{\#import "@preview/keyle:0.2.0"}
\end{Highlighting}
\end{Shaded}

\subsubsection{Example}\label{example}

\paragraph{Custom Delimiter}\label{custom-delimiter}

\begin{Shaded}
\begin{Highlighting}[]
\NormalTok{\#let kbd = keyle.config()}
\NormalTok{\#kbd("Ctrl", "Shift", "K", delim: "{-}")}
\end{Highlighting}
\end{Shaded}

\pandocbounded{\includegraphics[keepaspectratio]{https://github.com/typst/packages/raw/main/packages/preview/keyle/0.2.0/test/test-1.png}}

\paragraph{Compact Mode}\label{compact-mode}

\begin{Shaded}
\begin{Highlighting}[]
\NormalTok{\#let kbd = keyle.config()}
\NormalTok{\#kbd("Ctrl", "Shift", "K", compact: true)}
\end{Highlighting}
\end{Shaded}

\pandocbounded{\includegraphics[keepaspectratio]{https://github.com/typst/packages/raw/main/packages/preview/keyle/0.2.0/test/test-2.png}}

\paragraph{Standard Theme}\label{standard-theme}

\begin{Shaded}
\begin{Highlighting}[]
\NormalTok{\#let kbd = keyle.config(theme: keyle.themes.standard)}
\NormalTok{\#keyle.gen{-}examples(kbd)}
\end{Highlighting}
\end{Shaded}

\pandocbounded{\includegraphics[keepaspectratio]{https://github.com/typst/packages/raw/main/packages/preview/keyle/0.2.0/test/test-3.png}}

\paragraph{Deep Blue Theme}\label{deep-blue-theme}

\begin{Shaded}
\begin{Highlighting}[]
\NormalTok{\#let kbd = keyle.config(theme: keyle.themes.deep{-}blue)}
\NormalTok{\#keyle.gen{-}examples(kbd)}
\end{Highlighting}
\end{Shaded}

\pandocbounded{\includegraphics[keepaspectratio]{https://github.com/typst/packages/raw/main/packages/preview/keyle/0.2.0/test/test-4.png}}

\paragraph{Type Writer Theme}\label{type-writer-theme}

\begin{Shaded}
\begin{Highlighting}[]
\NormalTok{\#let kbd = keyle.config(theme: keyle.themes.type{-}writer)}
\NormalTok{\#keyle.gen{-}examples(kbd)}
\end{Highlighting}
\end{Shaded}

\pandocbounded{\includegraphics[keepaspectratio]{https://github.com/typst/packages/raw/main/packages/preview/keyle/0.2.0/test/test-5.png}}

\paragraph{Biolinum Theme}\label{biolinum-theme}

\begin{Shaded}
\begin{Highlighting}[]
\NormalTok{\#let kbd = keyle.config(theme: keyle.themes.biolinum, delim: keyle.biolinum{-}key.delim\_plus)}
\NormalTok{\#keyle.gen{-}examples(kbd)}
\end{Highlighting}
\end{Shaded}

\pandocbounded{\includegraphics[keepaspectratio]{https://github.com/typst/packages/raw/main/packages/preview/keyle/0.2.0/test/test-6.png}}

\paragraph{Custom Theme}\label{custom-theme}

\begin{Shaded}
\begin{Highlighting}[]
\NormalTok{// https://www.radix{-}ui.com/themes/playground\#kbd}
\NormalTok{\#let radix\_kdb(content) = box(}
\NormalTok{  rect(}
\NormalTok{    inset: (x: 0.5em),}
\NormalTok{    outset: (y:0.05em),}
\NormalTok{    stroke: rgb("\#1c2024") + 0.3pt,}
\NormalTok{    radius: 0.35em,}
\NormalTok{    fill: rgb("\#fcfcfd"),}
\NormalTok{    text(fill: black, font: (}
\NormalTok{      "Roboto",}
\NormalTok{      "Helvetica Neue",}
\NormalTok{    ), content),}
\NormalTok{  ),}
\NormalTok{)}
\NormalTok{\#let kbd = keyle.config(theme: radix\_kdb)}
\NormalTok{\#keyle.gen{-}examples(kbd)}
\end{Highlighting}
\end{Shaded}

\pandocbounded{\includegraphics[keepaspectratio]{https://github.com/typst/packages/raw/main/packages/preview/keyle/0.2.0/test/test-7.png}}

\subsection{License}\label{license}

MIT

\subsubsection{How to add}\label{how-to-add}

Copy this into your project and use the import as \texttt{\ keyle\ }

\begin{verbatim}
#import "@preview/keyle:0.2.0"
\end{verbatim}

\includesvg[width=0.16667in,height=0.16667in]{/assets/icons/16-copy.svg}

Check the docs for
\href{https://typst.app/docs/reference/scripting/\#packages}{more
information on how to import packages} .

\subsubsection{About}\label{about}

\begin{description}
\tightlist
\item[Author :]
\href{mailto:yxnian@outlook.com}{magicwenli}
\item[License:]
MIT
\item[Current version:]
0.2.0
\item[Last updated:]
August 27, 2024
\item[First released:]
July 24, 2024
\item[Minimum Typst version:]
0.11.1
\item[Archive size:]
5.97 kB
\href{https://packages.typst.org/preview/keyle-0.2.0.tar.gz}{\pandocbounded{\includesvg[keepaspectratio]{/assets/icons/16-download.svg}}}
\item[Repository:]
\href{https://github.com/magicwenli/keyle}{GitHub}
\item[Categor ies :]
\begin{itemize}
\tightlist
\item[]
\item
  \pandocbounded{\includesvg[keepaspectratio]{/assets/icons/16-hammer.svg}}
  \href{https://typst.app/universe/search/?category=utility}{Utility}
\item
  \pandocbounded{\includesvg[keepaspectratio]{/assets/icons/16-smile.svg}}
  \href{https://typst.app/universe/search/?category=fun}{Fun}
\end{itemize}
\end{description}

\subsubsection{Where to report issues?}\label{where-to-report-issues}

This package is a project of magicwenli . Report issues on
\href{https://github.com/magicwenli/keyle}{their repository} . You can
also try to ask for help with this package on the
\href{https://forum.typst.app}{Forum} .

Please report this package to the Typst team using the
\href{https://typst.app/contact}{contact form} if you believe it is a
safety hazard or infringes upon your rights.

\phantomsection\label{versions}
\subsubsection{Version history}\label{version-history}

\begin{longtable}[]{@{}ll@{}}
\toprule\noalign{}
Version & Release Date \\
\midrule\noalign{}
\endhead
\bottomrule\noalign{}
\endlastfoot
0.2.0 & August 27, 2024 \\
\href{https://typst.app/universe/package/keyle/0.1.1/}{0.1.1} & August
12, 2024 \\
\href{https://typst.app/universe/package/keyle/0.1.0/}{0.1.0} & July 24,
2024 \\
\end{longtable}

Typst GmbH did not create this package and cannot guarantee correct
functionality of this package or compatibility with any version of the
Typst compiler or app.
