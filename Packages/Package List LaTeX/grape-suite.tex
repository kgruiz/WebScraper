\title{typst.app/universe/package/grape-suite}

\phantomsection\label{banner}
\phantomsection\label{template-thumbnail}
\pandocbounded{\includegraphics[keepaspectratio]{https://packages.typst.org/preview/thumbnails/grape-suite-1.0.0-small.webp}}

\section{grape-suite}\label{grape-suite}

{ 1.0.0 }

Library of templates for exams, seminar papers, homeworks, etc.

{ } Featured Template

\href{/app?template=grape-suite&version=1.0.0}{Create project in app}

\phantomsection\label{readme}
The grape suite is a suite consisting of following templates:

\begin{itemize}
\item
  exercises (for exams, homework, etc.)
\item
  seminar papers
\item
  slides (using polylux)
\end{itemize}

\subsection{Exercises}\label{exercises}

\subsubsection{Setup}\label{setup}

\begin{Shaded}
\begin{Highlighting}[]
\NormalTok{\#import "@preview/grape{-}suite:1.0.0": exercise}
\NormalTok{\#import exercise: project, task, subtask}

\NormalTok{\#show: project.with(}
\NormalTok{    title: "Lorem ipsum dolor sit",}

\NormalTok{    university: [University],}
\NormalTok{    institute: [Institute],}
\NormalTok{    seminar: [Seminar],}

\NormalTok{    abstract: lorem(100),}
\NormalTok{    show{-}outline: true,}

\NormalTok{    author: "John Doe",}

\NormalTok{    show{-}solutions: false}
\NormalTok{)}
\end{Highlighting}
\end{Shaded}

\subsubsection{API-Documentation}\label{api-documentation}

\begin{longtable}[]{@{}ll@{}}
\toprule\noalign{}
\texttt{\ project\ } & \\
\midrule\noalign{}
\endhead
\bottomrule\noalign{}
\endlastfoot
\texttt{\ no\ } & optional, number, default: \texttt{\ none\ } , number
of the sheet in the series \\
\texttt{\ type\ } & optional, content, default: \texttt{\ {[}Exam{]}\ }
, type of the series, eg. exam, homework, protocol, … \\
\texttt{\ title\ } & optional, content, default: \texttt{\ none\ } ,
title of the document: if none, then generated from no, type and
suffix-title \\
\texttt{\ suffix-title\ } & optional, content, default:
\texttt{\ none\ } , used if title is none to generate the title of the
document \\
\texttt{\ show-outline\ } & optional, bool, default: \texttt{\ false\ }
, show outline after title iff true \\
\texttt{\ abstract\ } & optional, content, default: \texttt{\ none\ } ,
show abstract between outline and title \\
\texttt{\ document-title\ } & optional, content, default:
\texttt{\ none\ } , shown in the upper right corner of the page header:
if none, \texttt{\ title\ } is used \\
\texttt{\ show-hints\ } & optional, bool, default: \texttt{\ false\ } ,
generate hints from tasks iff true \\
\texttt{\ show-solutions\ } & optional, bool, default:
\texttt{\ false\ } , generate solutions from tasks iff true \\
\texttt{\ show-namefield\ } & optional, bool, default:
\texttt{\ false\ } , show namefield at the end of the left header iff
true \\
\texttt{\ namefield\ } & optional, content, default:
\texttt{\ {[}Name:{]}\ } , content shown iff
\texttt{\ show-namefield\ } \\
\texttt{\ show-timefield\ } & optional, bool, default:
\texttt{\ false\ } , show timefield at the end of right header iff
true \\
\texttt{\ timefield\ } & optional, function, default:
\texttt{\ (time)\ =\textgreater{}\ {[}Time:\ \#time\ min.{]}\ } , to
generate the content shown as the timefield iff
\texttt{\ show-timefield\ } is true \\
\texttt{\ max-time\ } & optional, number, default: \texttt{\ 0\ } , time
value used in the \texttt{\ timefield\ } function generateor \\
\texttt{\ show-lines\ } & optional, bool, default: \texttt{\ false\ } ,
draw automatic lines for each task, if \texttt{\ lines\ } parameter of
\texttt{\ task\ } is set \\
\texttt{\ show-point-distribution-in-tasks\ } & optional, bool, default:
\texttt{\ false\ } , show point distribution after tasks iff true \\
\texttt{\ show-point-distribution-in-solutions\ } & optional, bool,
default: \texttt{\ false\ } , show point distributions after solutions
iff true \\
\texttt{\ solutions-as-matrix\ } & optional, bool, default:
\texttt{\ false\ } , show solutions as a matrix iff true, \textbf{mind
that} : now the solution parameter of task expects a list of 2-tuples,
where the first element of the 2-tuple is the amount of points, a number
and the second element is content, how to achieve all points \\
\texttt{\ university\ } & optional, content, default:
\texttt{\ none\ } \\
\texttt{\ faculty\ } & optional, content, default: \texttt{\ none\ } \\
\texttt{\ institute\ } & optional, content, default:
\texttt{\ none\ } \\
\texttt{\ seminar\ } & optional, content, default: \texttt{\ none\ } \\
\texttt{\ semester\ } & optional, content, default: \texttt{\ none\ } \\
\texttt{\ docent\ } & optional, content, default: \texttt{\ none\ } \\
\texttt{\ author\ } & optional, content, default: \texttt{\ none\ } \\
\texttt{\ date\ } & optional, datetime, default:
\texttt{\ datetime.today()\ } \\
\texttt{\ header\ } & optional, content, default: \texttt{\ none\ } ,
overwrite page header \\
\texttt{\ header-right\ } & optional, content, default:
\texttt{\ none\ } , overwrite right header part \\
\texttt{\ header-middle\ } & optional, content, default:
\texttt{\ none\ } , overwrite middle header part \\
\texttt{\ header-left\ } & optional, content, default: \texttt{\ none\ }
, overwrite left header part \\
\texttt{\ footer\ } & optional, content, default: \texttt{\ none\ } ,
overwrite footer part \\
\texttt{\ footer-right\ } & optional, content, default:
\texttt{\ none\ } , overwrite right footer part \\
\texttt{\ footer-middle\ } & optional, content, default:
\texttt{\ none\ } , overwrite middle footer part \\
\texttt{\ footer-left\ } & optional, content, default: \texttt{\ none\ }
, overwrite left footer part \\
\texttt{\ task-type\ } & optional, content, default:
\texttt{\ {[}Task{]}\ } , content shown in task title box before
numbering \\
\texttt{\ extra-task-type\ } & optional, content, default:
\texttt{\ {[}Extra\ task{]}\ } , for tasks where the \texttt{\ extra\ }
parameter is true, content shown in title box before numbering \\
\texttt{\ box-task-title\ } & optional, content, default:
\texttt{\ {[}Task{]}\ } , shown as the title of a task box used by the
\texttt{\ slides\ } library \\
\texttt{\ box-hint-title\ } & optional, content, default:
\texttt{\ {[}Hint{]}\ } , shown as the title of a tasks colored hint
box \\
\texttt{\ box-solution-title\ } & optional, content, default:
\texttt{\ {[}Solution{]}\ } , shown as the title of a tasks colored
solution box \\
\texttt{\ box-definition-title\ } & optional, content, default:
\texttt{\ {[}Definition{]}\ } , shown as the title of a definition box
used by the \texttt{\ slides\ } library \\
\texttt{\ box-notice-title\ } & optional, content, default:
\texttt{\ {[}Notice{]}\ } , shown as the title of a notice box used by
the \texttt{\ slides\ } library \\
\texttt{\ box-example-title\ } & optional, content, default:
\texttt{\ {[}Example{]}\ } , shown as the title of a example box used by
the \texttt{\ slides\ } library \\
\texttt{\ hint-type\ } & optional, content, default:
\texttt{\ {[}Hint{]}\ } , title of a tasks hint version \\
\texttt{\ hints-title\ } & optional, content, default:
\texttt{\ {[}Hints{]}\ } , title of the hints section \\
\texttt{\ solution-type\ } & optional, content, default:
\texttt{\ {[}Suggested\ solution{]}\ } , title of a tasks solution
version \\
\texttt{\ solutions-title\ } & optional, content, default:
\texttt{\ {[}Suggested\ solutions{]}\ } , title of the solutions
section \\
\texttt{\ solution-matrix-task-header\ } & optional, content, default:
\texttt{\ {[}Tasks{]}\ } , first column header of solution matrix,
column contains the reasons on how to achieve the points \\
\texttt{\ solution-matrix-achieved-points-header\ } & optional, content,
default: \texttt{\ {[}Points\ achieved{]}\ } , second column header of
solution matrix, column contains the points the one achieved \\
\texttt{\ show-solution-matrix-comment-field\ } & optional, bool,
default: \texttt{\ false\ } , show comment field in solution matrix \\
\texttt{\ solution-matrix-comment-field-value\ } & optional, content,
default: \texttt{\ {[}*Note:*\ \#v(0.5cm){]}\ } , value of solution
matrix comment fields \\
\texttt{\ distribution-header-point-value\ } & optional, content,
default: \texttt{\ {[}Point{]}\ } , first row of point distribution,
used to indicate the points needed to get a specific grade \\
\texttt{\ distribution-header-point-grade\ } & optional, content,
default: \texttt{\ {[}Grade{]}\ } , second row of point distribution \\
\texttt{\ message\ } & optional, function, default:
\texttt{\ (points-sum,\ extrapoints-sum)\ =\textgreater{}\ {[}In\ sum\ \#points-sum\ +\ \#extrapoints-sum\ P.\ are\ achievable.\ You\ achieved\ \#box(line(stroke:\ purple,\ length:\ 1cm))\ out\ of\ \#points-sum\ points.{]}\ }
, used to generate the message part above the point distribution \\
\texttt{\ grade-scale\ } & optional, array, default:
\texttt{\ (({[}excellent{]},\ 0.9),\ ({[}very\ good{]},\ 0.8),\ ({[}good{]},\ 0.7),\ ({[}pass{]},\ 0.6),\ ({[}fail{]},\ 0.49))\ }
, list of grades and percentage of points to reach that grade \\
\texttt{\ page-margins\ } & optional, margins, default:
\texttt{\ none\ } , overwrite page margins \\
\texttt{\ fontsize\ } & optional, size, default: \texttt{\ 11pt\ } ,
overwrite font size \\
\texttt{\ show-todolist\ } & optional, bool, default: \texttt{\ true\ }
, show list of usages of the \texttt{\ todo\ } function after the
outline \\
\texttt{\ body\ } & content, document content \\
\end{longtable}

\texttt{\ task\ } creates a task element in an exercise project.

\begin{longtable}[]{@{}ll@{}}
\toprule\noalign{}
\texttt{\ task\ } & \\
\midrule\noalign{}
\endhead
\bottomrule\noalign{}
\endlastfoot
\texttt{\ lines\ } & optional, number, default: \texttt{\ 0\ } , number
of lines to draw if \texttt{\ show-lines\ } in exercise’s
\texttt{\ project\ } is set to \texttt{\ true\ } \\
\texttt{\ points\ } & optional, number, default: \texttt{\ 0\ } , number
of points achievable \\
\texttt{\ extra\ } & optional, bool, default: \texttt{\ false\ } ,
determines if the task is obligatory ( \texttt{\ false\ } ) or
additional ( \texttt{\ true\ } ) \\
\texttt{\ numbering-format\ } & optional, function, default:
\texttt{\ none\ } , \\
\texttt{\ title\ } & content, title of the task \\
\texttt{\ instruction\ } & content, instruction of the task,
highlighted \\
\texttt{\ ..args\ } & 1: content, task body; 2: content, task solution,
not highlighted (see \texttt{\ solution-as-matrix\ } of exercise’s
\texttt{\ project\ } ), 3: content, task hint \\
\end{longtable}

\texttt{\ subtask\ } creates a part of a task. Its points are added to
the parent task. \emph{\textbf{Subtasks are to be use inside of the
task’s body or inside of another subtask’s body.}}

\begin{longtable}[]{@{}ll@{}}
\toprule\noalign{}
\texttt{\ subtask\ } & \\
\midrule\noalign{}
\endhead
\bottomrule\noalign{}
\endlastfoot
\texttt{\ points\ } & optional, number, default: \texttt{\ 0\ } , points
achievable, adds to a tasks point \\
\texttt{\ tight\ } & optional, bool, default: \texttt{\ false\ } , enum
style \\
\texttt{\ markers\ } & optional, array, default:
\texttt{\ ("1.",\ "a)")\ } , numbering format for each level, fallback
is \texttt{\ i.\ } \\
\texttt{\ show-points\ } & optional, bool, default: \texttt{\ true\ } ,
show points next to subtask’s body iff \texttt{\ true\ } \\
\texttt{\ counter\ } & optional, counter, default: \texttt{\ none\ } ,
change number styled by the numbering format; if \texttt{\ none\ } ,
each level has an incrementel auto counter \\
\texttt{\ content\ } & content, subtask body \\
\end{longtable}

\subsection{Seminar paper}\label{seminar-paper}

\subsubsection{Setup}\label{setup-1}

\begin{Shaded}
\begin{Highlighting}[]
\NormalTok{\#import "@preview/grape{-}suite:1.0.0": seminar{-}paper}

\NormalTok{\#show: seminar{-}paper.project.with(}
\NormalTok{    title: "Die Intensionalität von dass{-}Sätzen",}
\NormalTok{    subtitle: "Intensionale Kontexte in philosophischen Argumenten",}

\NormalTok{    university: [Universität Musterstadt],}
\NormalTok{    faculty: [Exemplarische Fakultät],}
\NormalTok{    institute: [Institut für Philosophie],}
\NormalTok{    docent: [Dr. phil. Berta Beispielprüferin],}
\NormalTok{    seminar: [Beispielseminar],}

\NormalTok{    submit{-}to: [Eingereicht bei],}
\NormalTok{    submit{-}by: [Eingereicht durch],}

\NormalTok{    semester: german{-}dates.semester(datetime.today()),}

\NormalTok{    author: "Max Muster",}
\NormalTok{    email: "max.muster@uni{-}musterstadt.uni",}
\NormalTok{    address: [}
\NormalTok{        12345 Musterstadt \textbackslash{}}
\NormalTok{        Musterstraße 67}
\NormalTok{    ]}
\NormalTok{)}
\end{Highlighting}
\end{Shaded}

\subsubsection{Documentation}\label{documentation}

\begin{longtable}[]{@{}ll@{}}
\toprule\noalign{}
\texttt{\ project\ } & \\
\midrule\noalign{}
\endhead
\bottomrule\noalign{}
\endlastfoot
\texttt{\ title\ } & optional, content, default: \texttt{\ none\ } ,
title used on the title page \\
\texttt{\ subtitle\ } & optional, content, default: \texttt{\ none\ } ,
subtitle used on title page \\
\texttt{\ submit-to\ } & optional, content, default:
\texttt{\ "Submitted\ to"\ } , title for the assignees’s section \\
\texttt{\ submit-by\ } & optional, content, default:
\texttt{\ "Submitted\ by"\ } , title for the assigned’s section \\
\texttt{\ university\ } & optional, content, default:
\texttt{\ "UNIVERSITY"\ } \\
\texttt{\ faculty\ } & optional, content, default:
\texttt{\ "FACULTY"\ } \\
\texttt{\ institute\ } & optional, content, default:
\texttt{\ "INSTITUTE"\ } \\
\texttt{\ seminar\ } & optional, content, default:
\texttt{\ "SEMINAR"\ } \\
\texttt{\ semester\ } & optional, content, default:
\texttt{\ "SEMESTER"\ } \\
\texttt{\ docent\ } & optional, content, default:
\texttt{\ "DOCENT"\ } \\
\texttt{\ author\ } & optional, content, default:
\texttt{\ "AUTHOR"\ } \\
\texttt{\ email\ } & optional, content, default: \texttt{\ "EMAIL"\ } \\
\texttt{\ address\ } & optional, content, default:
\texttt{\ "ADDRESS"\ } \\
\texttt{\ title-page-part\ } & optional, content, default:
\texttt{\ none\ } , overwrite date, assignee and assigned section \\
\texttt{\ title-page-part-submit-date\ } & optional, content, default:
\texttt{\ none\ } , overwrite date section \\
\texttt{\ title-page-part-submit-to\ } & optional, content, default:
\texttt{\ none\ } , overwrite assignee section \\
\texttt{\ title-page-part-submit-by\ } & optional, content, default:
\texttt{\ none\ } , overwrite assigned section \\
\texttt{\ date\ } & optional, datetime, default:
\texttt{\ datetime.today()\ } \\
\texttt{\ date-format\ } & optional, function, default:
\texttt{\ (date)\ =\textgreater{}\ date.display("{[}day{]}.{[}month{]}.{[}year{]}")\ } \\
\texttt{\ header\ } & optional, content, default: \texttt{\ none\ } ,
overwrite page header \\
\texttt{\ header-right\ } & optional, content, default:
\texttt{\ none\ } , overwrite right header part \\
\texttt{\ header-middle\ } & optional, content, default:
\texttt{\ none\ } , overwrite middle header part \\
\texttt{\ header-left\ } & optional, content, default: \texttt{\ none\ }
, overwrite left header part \\
\texttt{\ footer\ } & optional, content, default: \texttt{\ none\ } ,
overwrite footer part \\
\texttt{\ footer-right\ } & optional, content, default:
\texttt{\ none\ } , overwrite right footer part \\
\texttt{\ footer-middle\ } & optional, content, default:
\texttt{\ none\ } , overwrite middle footer part \\
\texttt{\ footer-left\ } & optional, content, default: \texttt{\ none\ }
, overwrite left footer part \\
\texttt{\ show-outline\ } & optional, bool, default: \texttt{\ true\ } ,
show outline \\
\texttt{\ show-declaration-of-independent-work\ } & optional, bool,
default: \texttt{\ true\ } , show German declaration of independent
work \\
\texttt{\ page-margins\ } & optional, margins, default:
\texttt{\ none\ } , overwrite page margins \\
\texttt{\ fontsize\ } & optional, size, default: \texttt{\ 11pt\ } ,
overwrite fontsize \\
\texttt{\ show-todolist\ } & optional, bool, default: \texttt{\ true\ }
, show list of usages of the \texttt{\ todo\ } function after the
outline \\
\texttt{\ body\ } & content, document content \\
\end{longtable}

\begin{longtable}[]{@{}ll@{}}
\toprule\noalign{}
\texttt{\ sidenote\ } & \\
\midrule\noalign{}
\endhead
\bottomrule\noalign{}
\endlastfoot
\texttt{\ body\ } & sidenote content, which is a block with 3cm width
and will be displayed in the right margin of the page \\
\end{longtable}

\subsection{Slides}\label{slides}

\subsubsection{Setup}\label{setup-2}

\begin{Shaded}
\begin{Highlighting}[]
\NormalTok{\#import "@preview/grape{-}suite:1.0.0": slides}
\NormalTok{\#import slides: *}

\NormalTok{\#show: slides.with(}
\NormalTok{    no: 1,}
\NormalTok{    series: [Logik{-}Tutorium],}
\NormalTok{    title: [Organisatorisches und Einführung in die Logik],}

\NormalTok{    author: "Tristan Pieper",}
\NormalTok{    email: link("mailto:tristan.pieper@uni{-}rostock.de"),}
\NormalTok{)}
\end{Highlighting}
\end{Shaded}

\subsubsection{Documentation}\label{documentation-1}

\begin{longtable}[]{@{}ll@{}}
\toprule\noalign{}
\texttt{\ slides\ } & \\
\midrule\noalign{}
\endhead
\bottomrule\noalign{}
\endlastfoot
\texttt{\ no\ } & optional, number, default: \texttt{\ 0\ } , number in
the series \\
\texttt{\ series\ } & optional, content, default: \texttt{\ none\ } ,
name of the series \\
\texttt{\ title\ } & optional, content, default: \texttt{\ none\ } ,
title of the presentation \\
\texttt{\ topics\ } & optional, array, default: \texttt{\ ()\ } , topics
of the presentation \\
\texttt{\ author\ } & optional, content, default: \texttt{\ none\ } ,
author \\
\texttt{\ email\ } & optional, content, default: \texttt{\ none\ } ,
author’s email \\
\texttt{\ head-replacement\ } & optional, content, default:
\texttt{\ none\ } , replace head on title slide with given content \\
\texttt{\ title-replacement\ } & optional, content, default:
\texttt{\ none\ } , replace title below head on title slide with given
content \\
\texttt{\ footer\ } & optional, content, default: \texttt{\ none\ } ,
replace footer on slides with given content \\
\texttt{\ page-numbering\ } & optional, function, default:
\texttt{\ (n,\ total)\ =\textgreater{}\ \{...\}\ } , function that
creates the page numbering (where \texttt{\ n\ } is the current,
\texttt{\ total\ } is the last page) \\
\texttt{\ show-semester\ } & optional, bool, default: \texttt{\ true\ }
, show name of the semester (e.g. “SoSe 24�) \\
\texttt{\ show-date\ } & optional, bool, default: \texttt{\ true\ } ,
show date in german format \\
\texttt{\ show-outline\ } & optional, bool, default: \texttt{\ true\ } ,
show outline on the second slide \\
\texttt{\ box-task-title\ } & optional, content, default:
\texttt{\ {[}Task{]}\ } , shown as the title of a slide’s task box \\
\texttt{\ box-hint-title\ } & optional, content, default:
\texttt{\ {[}Hint{]}\ } , shown as the title of a slide’s tasks
colored \\
\texttt{\ box-solution-title\ } & optional, content, default:
\texttt{\ {[}Solution{]}\ } , shown as the title of a slide’s tasks
colored \\
\texttt{\ box-definition-title\ } & optional, content, default:
\texttt{\ {[}Definition{]}\ } , shown as the title of a slide’s
definition box \\
\texttt{\ box-notice-title\ } & optional, content, default:
\texttt{\ {[}Notice{]}\ } , shown as the title of a slide’s notice
box \\
\texttt{\ box-example-title\ } & optional, content, default:
\texttt{\ {[}Example{]}\ } , shown as the title of a slide’s example
box \\
\texttt{\ date\ } & optional, datetime, default:
\texttt{\ datetime.today()\ } \\
\texttt{\ show-todolist\ } & optional, bool, default: \texttt{\ true\ }
, show list of usages of the \texttt{\ todo\ } function after the
outline \\
\texttt{\ show-title-slide\ } & optional, bool, default:
\texttt{\ true\ } , show title slide \\
\texttt{\ show-author\ } & optional, bool, default: \texttt{\ true\ } ,
show author name on title slide \\
\texttt{\ show-footer\ } & optional, bool, default: \texttt{\ true\ } ,
show footer on slides \\
\texttt{\ show-page-numbers\ } & optional, bool, default:
\texttt{\ true\ } , show page numbering \\
\texttt{\ outline-title-text\ } & optional, content, default:
\texttt{\ "Outline"\ } , title for the outline \\
\texttt{\ body\ } & content, document content \\
\end{longtable}

\begin{itemize}
\tightlist
\item
  \texttt{\ slide\ } , \texttt{\ pause\ } , \texttt{\ only\ } ,
  \texttt{\ uncover\ } : imported from polylux
\end{itemize}

\subsubsection{Todos}\label{todos}

The following functions can be imported from \texttt{\ slides\ } ,
\texttt{\ exercise\ } and \texttt{\ seminar-paper\ } :

\begin{itemize}
\tightlist
\item
  \texttt{\ todo(content,\ ...)\ } - create a highlighted inline
  todo-note
\item
  \texttt{\ list-todos()\ } - create list of all todo-usages with page
  of usage and content
\item
  \texttt{\ hide-todos()\ } - hides all usages of \texttt{\ todo()\ } in
  the document
\end{itemize}

\subsubsection{Elements}\label{elements}

The following functions can be imported from \texttt{\ slides\ } ,
\texttt{\ exercise\ } and \texttt{\ seminar-paper\ } :
\texttt{\ definition\ }

\subsection{1.0.0}\label{section}

New:

\begin{itemize}
\tightlist
\item
  \texttt{\ todo\ } , \texttt{\ list-todos\ } , \texttt{\ hide-todos\ }
  in \texttt{\ todo.typ\ } , importable from \texttt{\ slides\ } ,
  \texttt{\ exercise.project\ } and \texttt{\ seminar-paper.project\ }
\item
  \texttt{\ show-todolist\ } attribute in above templates
\item
  \texttt{\ ignore-points\ } attribute in \texttt{\ task\ } and
  \texttt{\ subtask\ } of exercises, so that their points won’t be
  shown in the solution matrix or point distribution
\item
  comment field and a standard-value for solution matrix via
  \texttt{\ show-solution-matrix-comment-field\ } and
  \texttt{\ solution-matrix-comment-field-value\ } options in
  \texttt{\ exercise.project\ }
\item
  optional parameter \texttt{\ type\ } in \texttt{\ slides.task\ }
\item
  new parameters in \texttt{\ sllides.slides\ } :

  \begin{itemize}
  \tightlist
  \item
    \texttt{\ head-replacement\ }
  \item
    \texttt{\ title-replacement\ }
  \item
    \texttt{\ footer\ }
  \item
    \texttt{\ page-numbering\ }
  \item
    \texttt{\ show-title-slide\ }
  \item
    \texttt{\ show-author\ } (on title slide)
  \item
    \texttt{\ show-date\ }
  \item
    \texttt{\ show-footer\ }
  \item
    \texttt{\ show-page-numbers\ }
  \end{itemize}
\item
  optional parameter \texttt{\ show-outline\ } in
  \texttt{\ seminar-paper.project\ }
\end{itemize}

Changes:

\begin{itemize}
\tightlist
\item
  \texttt{\ dates.typ\ } becomes \texttt{\ german-dates.typ\ }
\end{itemize}

Fixes:

\begin{itemize}
\tightlist
\item
  remove forced German from the slides template
\item
  long headings are now properly aligned
\item
  subtask counter now resets for each part of task
\end{itemize}

\textbf{Breaking Changes:}

\begin{itemize}
\tightlist
\item
  \texttt{\ dates\ } becomes \texttt{\ german-dates\ }
\item
  changed all \texttt{\ with-outline\ } to \texttt{\ show-outline\ }
\end{itemize}

\href{/app?template=grape-suite&version=1.0.0}{Create project in app}

\subsubsection{How to use}\label{how-to-use}

Click the button above to create a new project using this template in
the Typst app.

You can also use the Typst CLI to start a new project on your computer
using this command:

\begin{verbatim}
typst init @preview/grape-suite:1.0.0
\end{verbatim}

\includesvg[width=0.16667in,height=0.16667in]{/assets/icons/16-copy.svg}

\subsubsection{About}\label{about}

\begin{description}
\tightlist
\item[Author :]
\href{mailto:tristanpieper080803@gmail.com}{Tristan Pieper}
\item[License:]
MIT
\item[Current version:]
1.0.0
\item[Last updated:]
July 22, 2024
\item[First released:]
May 3, 2024
\item[Minimum Typst version:]
0.11.0
\item[Archive size:]
15.7 kB
\href{https://packages.typst.org/preview/grape-suite-1.0.0.tar.gz}{\pandocbounded{\includesvg[keepaspectratio]{/assets/icons/16-download.svg}}}
\item[Repository:]
\href{https://github.com/piepert/grape-suite}{GitHub}
\item[Categor ies :]
\begin{itemize}
\tightlist
\item[]
\item
  \pandocbounded{\includesvg[keepaspectratio]{/assets/icons/16-layout.svg}}
  \href{https://typst.app/universe/search/?category=layout}{Layout}
\item
  \pandocbounded{\includesvg[keepaspectratio]{/assets/icons/16-atom.svg}}
  \href{https://typst.app/universe/search/?category=paper}{Paper}
\item
  \pandocbounded{\includesvg[keepaspectratio]{/assets/icons/16-presentation.svg}}
  \href{https://typst.app/universe/search/?category=presentation}{Presentation}
\end{itemize}
\end{description}

\subsubsection{Where to report issues?}\label{where-to-report-issues}

This template is a project of Tristan Pieper . Report issues on
\href{https://github.com/piepert/grape-suite}{their repository} . You
can also try to ask for help with this template on the
\href{https://forum.typst.app}{Forum} .

Please report this template to the Typst team using the
\href{https://typst.app/contact}{contact form} if you believe it is a
safety hazard or infringes upon your rights.

\phantomsection\label{versions}
\subsubsection{Version history}\label{version-history}

\begin{longtable}[]{@{}ll@{}}
\toprule\noalign{}
Version & Release Date \\
\midrule\noalign{}
\endhead
\bottomrule\noalign{}
\endlastfoot
1.0.0 & July 22, 2024 \\
\href{https://typst.app/universe/package/grape-suite/0.1.0/}{0.1.0} &
May 3, 2024 \\
\end{longtable}

Typst GmbH did not create this template and cannot guarantee correct
functionality of this template or compatibility with any version of the
Typst compiler or app.
