\title{typst.app/universe/package/tufte-memo}

\phantomsection\label{banner}
\phantomsection\label{template-thumbnail}
\pandocbounded{\includegraphics[keepaspectratio]{https://packages.typst.org/preview/thumbnails/tufte-memo-0.1.2-small.webp}}

\section{tufte-memo}\label{tufte-memo}

{ 0.1.2 }

A memo template inspired by the design of Edward Tufte\textquotesingle s
books

{ } Featured Template

\href{/app?template=tufte-memo&version=0.1.2}{Create project in app}

\phantomsection\label{readme}
A memo document template inspired by the design of Edward Tufte’s
books for the Typst typesetting program.

For usage, see the usage guide
\href{https://github.com/nogula/tufte-memo/blob/main/template/main.pdf}{here}
.

The template provides handy functions: \texttt{\ template\ } ,
\texttt{\ note\ } , and \texttt{\ wideblock\ } . To create a document
with this template, use:

\begin{Shaded}
\begin{Highlighting}[]
\NormalTok{\#import "@preview/tufte{-}memo:0.1.2": *}

\NormalTok{\#show: template.with(}
\NormalTok{    title: [Document Title],}
\NormalTok{    authors: (}
\NormalTok{        (}
\NormalTok{        name: "Author Name",}
\NormalTok{        role: "Optional Role Line",}
\NormalTok{        affiliation: "Optional Affiliation Line",}
\NormalTok{        email: "email@company.com"}
\NormalTok{        ),}
\NormalTok{    )}
\NormalTok{)}
\NormalTok{...}
\end{Highlighting}
\end{Shaded}

additional configuration information is available in the usage guide.

The \texttt{\ note()\ } function provides the ability to produce
sidenotes next to the main body content. It can be called simply with
\texttt{\ \#note{[}...{]}\ } . Additionally, \texttt{\ wideblock()\ }
expands the width of its content to fill the full 6.5-inch-wide space,
rather than be compressed in to a four-inch column. It is simply called
with \texttt{\ wideblock{[}...{]}\ } .

\href{/app?template=tufte-memo&version=0.1.2}{Create project in app}

\subsubsection{How to use}\label{how-to-use}

Click the button above to create a new project using this template in
the Typst app.

You can also use the Typst CLI to start a new project on your computer
using this command:

\begin{verbatim}
typst init @preview/tufte-memo:0.1.2
\end{verbatim}

\includesvg[width=0.16667in,height=0.16667in]{/assets/icons/16-copy.svg}

\subsubsection{About}\label{about}

\begin{description}
\tightlist
\item[Author :]
Noah Gula
\item[License:]
MIT
\item[Current version:]
0.1.2
\item[Last updated:]
August 12, 2024
\item[First released:]
June 3, 2024
\item[Archive size:]
9.31 kB
\href{https://packages.typst.org/preview/tufte-memo-0.1.2.tar.gz}{\pandocbounded{\includesvg[keepaspectratio]{/assets/icons/16-download.svg}}}
\item[Repository:]
\href{https://github.com/nogula/tufte-memo}{GitHub}
\item[Categor y :]
\begin{itemize}
\tightlist
\item[]
\item
  \pandocbounded{\includesvg[keepaspectratio]{/assets/icons/16-speak.svg}}
  \href{https://typst.app/universe/search/?category=report}{Report}
\end{itemize}
\end{description}

\subsubsection{Where to report issues?}\label{where-to-report-issues}

This template is a project of Noah Gula . Report issues on
\href{https://github.com/nogula/tufte-memo}{their repository} . You can
also try to ask for help with this template on the
\href{https://forum.typst.app}{Forum} .

Please report this template to the Typst team using the
\href{https://typst.app/contact}{contact form} if you believe it is a
safety hazard or infringes upon your rights.

\phantomsection\label{versions}
\subsubsection{Version history}\label{version-history}

\begin{longtable}[]{@{}ll@{}}
\toprule\noalign{}
Version & Release Date \\
\midrule\noalign{}
\endhead
\bottomrule\noalign{}
\endlastfoot
0.1.2 & August 12, 2024 \\
\href{https://typst.app/universe/package/tufte-memo/0.1.1/}{0.1.1} &
June 5, 2024 \\
\href{https://typst.app/universe/package/tufte-memo/0.1.0/}{0.1.0} &
June 3, 2024 \\
\end{longtable}

Typst GmbH did not create this template and cannot guarantee correct
functionality of this template or compatibility with any version of the
Typst compiler or app.
