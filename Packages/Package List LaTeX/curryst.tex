\title{typst.app/universe/package/curryst}

\phantomsection\label{banner}
\section{curryst}\label{curryst}

{ 0.3.0 }

Typeset trees of inference rules.

{ } Featured Package

\phantomsection\label{readme}
A Typst package for typesetting proof trees.

\subsection{Import}\label{import}

You can import the latest version of this package with:

\begin{Shaded}
\begin{Highlighting}[]
\NormalTok{\#import "@preview/curryst:0.3.0": rule, proof{-}tree}
\end{Highlighting}
\end{Shaded}

\subsection{Basic usage}\label{basic-usage}

To display a proof tree, you first need to create a tree, using the
\texttt{\ rule\ } function. Its first argument is the conclusion, and
the other positional arguments are the premises. It also accepts a
\texttt{\ name\ } for the rule name, displayed on the right of the bar,
as well as a \texttt{\ label\ } , displayed on the left of the bar.

\begin{Shaded}
\begin{Highlighting}[]
\NormalTok{\#let tree = rule(}
\NormalTok{  label: [Label],}
\NormalTok{  name: [Rule name],}
\NormalTok{  [Conclusion],}
\NormalTok{  [Premise 1],}
\NormalTok{  [Premise 2],}
\NormalTok{  [Premise 3]}
\NormalTok{)}
\end{Highlighting}
\end{Shaded}

Then, you can display the tree with the \texttt{\ proof-tree\ }
function:

\begin{Shaded}
\begin{Highlighting}[]
\NormalTok{\#proof{-}tree(tree)}
\end{Highlighting}
\end{Shaded}

In this case, we get the following result:

\pandocbounded{\includesvg[keepaspectratio]{https://github.com/typst/packages/raw/main/packages/preview/curryst/0.3.0/examples/usage.svg}}

Proof trees can be part of mathematical formulas:

\begin{Shaded}
\begin{Highlighting}[]
\NormalTok{Consider the following tree:}
\NormalTok{$}
\NormalTok{  Pi quad = quad \#proof{-}tree(}
\NormalTok{    rule(}
\NormalTok{      $phi$,}
\NormalTok{      $Pi\_1$,}
\NormalTok{      $Pi\_2$,}
\NormalTok{    )}
\NormalTok{  )}
\NormalTok{$}
\NormalTok{$Pi$ constitutes a derivation of $phi$.s}
\end{Highlighting}
\end{Shaded}

\pandocbounded{\includesvg[keepaspectratio]{https://github.com/typst/packages/raw/main/packages/preview/curryst/0.3.0/examples/math-formula.svg}}

You can specify a rule as the premises of a rule in order to create a
tree:

\begin{Shaded}
\begin{Highlighting}[]
\NormalTok{\#proof{-}tree(}
\NormalTok{  rule(}
\NormalTok{    name: $R$,}
\NormalTok{    $C\_1 or C\_2 or C\_3$,}
\NormalTok{    rule(}
\NormalTok{      name: $A$,}
\NormalTok{      $C\_1 or C\_2 or L$,}
\NormalTok{      rule(}
\NormalTok{        $C\_1 or L$,}
\NormalTok{        $Pi\_1$,}
\NormalTok{      ),}
\NormalTok{    ),}
\NormalTok{    rule(}
\NormalTok{      $C\_2 or overline(L)$,}
\NormalTok{      $Pi\_2$,}
\NormalTok{    ),}
\NormalTok{  )}
\NormalTok{)}
\end{Highlighting}
\end{Shaded}

\pandocbounded{\includesvg[keepaspectratio]{https://github.com/typst/packages/raw/main/packages/preview/curryst/0.3.0/examples/rule-as-premise.svg}}

As an example, here is a natural deduction proof tree generated with
Curryst:

\pandocbounded{\includesvg[keepaspectratio]{https://github.com/typst/packages/raw/main/packages/preview/curryst/0.3.0/examples/natural-deduction.svg}}

Show code

\begin{Shaded}
\begin{Highlighting}[]
\NormalTok{\#let ax = rule.with(name: [ax])}
\NormalTok{\#let and{-}el = rule.with(name: $and\_e\^{}ell$)}
\NormalTok{\#let and{-}er = rule.with(name: $and\_e\^{}r$)}
\NormalTok{\#let impl{-}i = rule.with(name: $scripts({-}\textgreater{})\_i$)}
\NormalTok{\#let impl{-}e = rule.with(name: $scripts({-}\textgreater{})\_e$)}
\NormalTok{\#let not{-}i = rule.with(name: $not\_i$)}
\NormalTok{\#let not{-}e = rule.with(name: $not\_e$)}

\NormalTok{\#proof{-}tree(}
\NormalTok{  impl{-}i(}
\NormalTok{    $tack (p {-}\textgreater{} q) {-}\textgreater{} not (p and not q)$,}
\NormalTok{    not{-}i(}
\NormalTok{      $p {-}\textgreater{} q tack  not (p and not q)$,}
\NormalTok{      not{-}e(}
\NormalTok{        $ underbrace(p {-}\textgreater{} q\textbackslash{}, p and not q, Gamma) tack bot $,}
\NormalTok{        impl{-}e(}
\NormalTok{          $Gamma tack q$,}
\NormalTok{          ax($Gamma tack p {-}\textgreater{} q$),}
\NormalTok{          and{-}el(}
\NormalTok{            $Gamma tack p$,}
\NormalTok{            ax($Gamma tack p and not q$),}
\NormalTok{          ),}
\NormalTok{        ),}
\NormalTok{        and{-}er(}
\NormalTok{          $Gamma tack not q$,}
\NormalTok{          ax($Gamma tack p and not q$),}
\NormalTok{        ),}
\NormalTok{      ),}
\NormalTok{    ),}
\NormalTok{  )}
\NormalTok{)}
\end{Highlighting}
\end{Shaded}

\subsection{Advanced usage}\label{advanced-usage}

The \texttt{\ proof-tree\ } function accepts multiple named arguments
that let you customize the tree:

\begin{description}
\tightlist
\item[\texttt{\ prem-min-spacing\ }]
The minimum amount of space between two premises.
\item[\texttt{\ title-inset\ }]
The amount width with which to extend the horizontal bar beyond the
content. Also determines how far from the bar labels and names are
displayed.
\item[\texttt{\ stroke\ }]
The stroke to use for the horizontal bars.
\item[\texttt{\ horizontal-spacing\ }]
The space between the bottom of the bar and the conclusion, and between
the top of the bar and the premises.
\item[\texttt{\ min-bar-height\ }]
The minimum height of the box containing the horizontal bar.
\end{description}

For more information, please refer to the documentation in
\href{https://github.com/typst/packages/raw/main/packages/preview/curryst/0.3.0/curryst.typ}{\texttt{\ curryst.typ\ }}
.

\subsubsection{How to add}\label{how-to-add}

Copy this into your project and use the import as \texttt{\ curryst\ }

\begin{verbatim}
#import "@preview/curryst:0.3.0"
\end{verbatim}

\includesvg[width=0.16667in,height=0.16667in]{/assets/icons/16-copy.svg}

Check the docs for
\href{https://typst.app/docs/reference/scripting/\#packages}{more
information on how to import packages} .

\subsubsection{About}\label{about}

\begin{description}
\tightlist
\item[Author s :]
\href{https://github.com/remih23}{Rémi Hutin} ,
\href{https://github.com/pauladam94}{Paul Adam} , \&
\href{https://github.com/MDLC01}{Malo}
\item[License:]
MIT
\item[Current version:]
0.3.0
\item[Last updated:]
April 16, 2024
\item[First released:]
December 7, 2023
\item[Minimum Typst version:]
0.11.0
\item[Archive size:]
4.71 kB
\href{https://packages.typst.org/preview/curryst-0.3.0.tar.gz}{\pandocbounded{\includesvg[keepaspectratio]{/assets/icons/16-download.svg}}}
\item[Repository:]
\href{https://github.com/pauladam94/curryst}{GitHub}
\item[Discipline s :]
\begin{itemize}
\tightlist
\item[]
\item
  \href{https://typst.app/universe/search/?discipline=computer-science}{Computer
  Science}
\item
  \href{https://typst.app/universe/search/?discipline=mathematics}{Mathematics}
\end{itemize}
\item[Categor ies :]
\begin{itemize}
\tightlist
\item[]
\item
  \pandocbounded{\includesvg[keepaspectratio]{/assets/icons/16-package.svg}}
  \href{https://typst.app/universe/search/?category=components}{Components}
\item
  \pandocbounded{\includesvg[keepaspectratio]{/assets/icons/16-chart.svg}}
  \href{https://typst.app/universe/search/?category=visualization}{Visualization}
\item
  \pandocbounded{\includesvg[keepaspectratio]{/assets/icons/16-integration.svg}}
  \href{https://typst.app/universe/search/?category=integration}{Integration}
\end{itemize}
\end{description}

\subsubsection{Where to report issues?}\label{where-to-report-issues}

This package is a project of Rémi Hutin, Paul Adam, and Malo . Report
issues on \href{https://github.com/pauladam94/curryst}{their repository}
. You can also try to ask for help with this package on the
\href{https://forum.typst.app}{Forum} .

Please report this package to the Typst team using the
\href{https://typst.app/contact}{contact form} if you believe it is a
safety hazard or infringes upon your rights.

\phantomsection\label{versions}
\subsubsection{Version history}\label{version-history}

\begin{longtable}[]{@{}ll@{}}
\toprule\noalign{}
Version & Release Date \\
\midrule\noalign{}
\endhead
\bottomrule\noalign{}
\endlastfoot
0.3.0 & April 16, 2024 \\
\href{https://typst.app/universe/package/curryst/0.2.0/}{0.2.0} & March
19, 2024 \\
\href{https://typst.app/universe/package/curryst/0.1.1/}{0.1.1} &
January 31, 2024 \\
\href{https://typst.app/universe/package/curryst/0.1.0/}{0.1.0} &
December 7, 2023 \\
\end{longtable}

Typst GmbH did not create this package and cannot guarantee correct
functionality of this package or compatibility with any version of the
Typst compiler or app.
