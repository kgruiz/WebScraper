\title{typst.app/universe/package/algo}

\phantomsection\label{banner}
\section{algo}\label{algo}

{ 0.3.4 }

Beautifully typeset algorithms.

\phantomsection\label{readme}
A Typst library for writing algorithms. On Typst v0.6.0+ you can import
the \texttt{\ algo\ } package:

\begin{Shaded}
\begin{Highlighting}[]
\NormalTok{\#import "@preview/algo:0.3.4": algo, i, d, comment, code}
\end{Highlighting}
\end{Shaded}

Otherwise, add the \texttt{\ algo.typ\ } file to your project and import
it as normal:

\begin{Shaded}
\begin{Highlighting}[]
\NormalTok{\#import "algo.typ": algo, i, d, comment, code}
\end{Highlighting}
\end{Shaded}

Use the \texttt{\ algo\ } function for writing pseudocode and the
\texttt{\ code\ } function for writing code blocks with line numbers.
Check out the
\href{https://github.com/typst/packages/raw/main/packages/preview/algo/0.3.4/\#examples}{examples}
below for a quick overview. See the
\href{https://github.com/typst/packages/raw/main/packages/preview/algo/0.3.4/\#usage}{usage}
section to read about all the options each function has.

\subsection{Examples}\label{examples}

Here’s a basic use of \texttt{\ algo\ } :

\begin{Shaded}
\begin{Highlighting}[]
\NormalTok{\#algo(}
\NormalTok{  title: "Fib",}
\NormalTok{  parameters: ("n",)}
\NormalTok{)[}
\NormalTok{  if $n \textless{} 0$:\#i\textbackslash{}        // use \#i to indent the following lines}
\NormalTok{    return null\#d\textbackslash{}      // use \#d to dedent the following lines}
\NormalTok{  if $n = 0$ or $n = 1$:\#i \#comment[you can also]\textbackslash{}}
\NormalTok{    return $n$\#d \#comment[add comments!]\textbackslash{}}
\NormalTok{  return \#smallcaps("Fib")$(n{-}1) +$ \#smallcaps("Fib")$(n{-}2)$}
\NormalTok{]}
\end{Highlighting}
\end{Shaded}

\includegraphics[width=4.16667in,height=\textheight,keepaspectratio]{https://user-images.githubusercontent.com/40146328/235323240-e59ed7e2-ebb6-4b80-8742-eb171dd3721e.png}\\

Here’s a use of \texttt{\ algo\ } without a title, parameters, line
numbers, or syntax highlighting:

\begin{Shaded}
\begin{Highlighting}[]
\NormalTok{\#algo(}
\NormalTok{  line{-}numbers: false,}
\NormalTok{  strong{-}keywords: false}
\NormalTok{)[}
\NormalTok{  if $n \textless{} 0$:\#i\textbackslash{}}
\NormalTok{    return null\#d\textbackslash{}}
\NormalTok{  if $n = 0$ or $n = 1$:\#i\textbackslash{}}
\NormalTok{    return $n$\#d\textbackslash{}}
\NormalTok{  \textbackslash{}}
\NormalTok{  let $x \textless{}{-} 0$\textbackslash{}}
\NormalTok{  let $y \textless{}{-} 1$\textbackslash{}}
\NormalTok{  for $i \textless{}{-} 2$ to $n{-}1$:\#i \#comment[so dynamic!]\textbackslash{}}
\NormalTok{    let $z \textless{}{-} x+y$\textbackslash{}}
\NormalTok{    $x \textless{}{-} y$\textbackslash{}}
\NormalTok{    $y \textless{}{-} z$\#d\textbackslash{}}
\NormalTok{    \textbackslash{}}
\NormalTok{  return $x+y$}
\NormalTok{]}
\end{Highlighting}
\end{Shaded}

\includegraphics[width=3.125in,height=\textheight,keepaspectratio]{https://user-images.githubusercontent.com/40146328/235323261-d6e7a42c-ffb7-4c3a-bd2a-4c8fc2df5f36.png}\\

And here’s \texttt{\ algo\ } with more styling options:

\begin{Shaded}
\begin{Highlighting}[]
\NormalTok{\#algo(}
\NormalTok{  title: [                    // note that title and parameters}
\NormalTok{    \#set text(size: 15pt)     // can be content}
\NormalTok{    \#emph(smallcaps("Fib"))}
\NormalTok{  ],}
\NormalTok{  parameters: ([\#math.italic("n")],),}
\NormalTok{  comment{-}prefix: [\#sym.triangle.stroked.r ],}
\NormalTok{  comment{-}styles: (fill: rgb(100\%, 0\%, 0\%)),}
\NormalTok{  indent{-}size: 15pt,}
\NormalTok{  indent{-}guides: 1pt + gray,}
\NormalTok{  row{-}gutter: 5pt,}
\NormalTok{  column{-}gutter: 5pt,}
\NormalTok{  inset: 5pt,}
\NormalTok{  stroke: 2pt + black,}
\NormalTok{  fill: none,}
\NormalTok{)[}
\NormalTok{  if $n \textless{} 0$:\#i\textbackslash{}}
\NormalTok{    return null\#d\textbackslash{}}
\NormalTok{  if $n = 0$ or $n = 1$:\#i\textbackslash{}}
\NormalTok{    return $n$\#d\textbackslash{}}
\NormalTok{  \textbackslash{}}
\NormalTok{  let $x \textless{}{-} 0$\textbackslash{}}
\NormalTok{  let $y \textless{}{-} 1$\textbackslash{}}
\NormalTok{  for $i \textless{}{-} 2$ to $n{-}1$:\#i \#comment[so dynamic!]\textbackslash{}}
\NormalTok{    let $z \textless{}{-} x+y$\textbackslash{}}
\NormalTok{    $x \textless{}{-} y$\textbackslash{}}
\NormalTok{    $y \textless{}{-} z$\#d\textbackslash{}}
\NormalTok{    \textbackslash{}}
\NormalTok{  return $x+y$}
\NormalTok{]}
\end{Highlighting}
\end{Shaded}

\includegraphics[width=3.125in,height=\textheight,keepaspectratio]{https://github.com/platformer/typst-algorithms/assets/40146328/89f80b5d-bdb2-420a-935d-24f43ca597d8}

Here’s a basic use of \texttt{\ code\ } :

\begin{Shaded}
\begin{Highlighting}[]
\NormalTok{\#code()[}
\NormalTok{  \textasciigrave{}\textasciigrave{}\textasciigrave{}py}
\NormalTok{  def fib(n):}
\NormalTok{    if n \textless{} 0:}
\NormalTok{      return None}
\NormalTok{    if n == 0 or n == 1:        \# this comment is}
\NormalTok{      return n                  \# normal raw text}
\NormalTok{    return fib(n{-}1) + fib(n{-}2)}
\NormalTok{  \textasciigrave{}\textasciigrave{}\textasciigrave{}}
\NormalTok{]}
\end{Highlighting}
\end{Shaded}

\includegraphics[width=4.16667in,height=\textheight,keepaspectratio]{https://user-images.githubusercontent.com/40146328/235324088-a3596e0b-af90-4da3-b326-2de11158baac.png}\\

And here’s \texttt{\ code\ } with some styling options:

\begin{Shaded}
\begin{Highlighting}[]
\NormalTok{\#code(}
\NormalTok{  indent{-}guides: 1pt + gray,}
\NormalTok{  row{-}gutter: 5pt,}
\NormalTok{  column{-}gutter: 5pt,}
\NormalTok{  inset: 5pt,}
\NormalTok{  stroke: 2pt + black,}
\NormalTok{  fill: none,}
\NormalTok{)[}
\NormalTok{  \textasciigrave{}\textasciigrave{}\textasciigrave{}py}
\NormalTok{  def fib(n):}
\NormalTok{      if n \textless{} 0:}
\NormalTok{          return None}
\NormalTok{      if n == 0 or n == 1:        \# this comment is}
\NormalTok{          return n                \# normal raw text}
\NormalTok{      return fib(n{-}1) + fib(n{-}2)}
\NormalTok{  \textasciigrave{}\textasciigrave{}\textasciigrave{}}
\NormalTok{]}
\end{Highlighting}
\end{Shaded}

\includegraphics[width=4.16667in,height=\textheight,keepaspectratio]{https://github.com/platformer/typst-algorithms/assets/40146328/c091ac43-6861-40bc-8046-03ea285712c3}

\subsection{Usage}\label{usage}

\subsubsection{\texorpdfstring{\texttt{\ algo\ }}{ algo }}\label{algo-1}

Makes a pseudocode element.

\begin{Shaded}
\begin{Highlighting}[]
\NormalTok{algo(}
\NormalTok{  body,}
\NormalTok{  header: none,}
\NormalTok{  title: none,}
\NormalTok{  parameters: (),}
\NormalTok{  line{-}numbers: true,}
\NormalTok{  strong{-}keywords: true,}
\NormalTok{  keywords: \_algo{-}default{-}keywords, // see below}
\NormalTok{  comment{-}prefix: "// ",}
\NormalTok{  indent{-}size: 20pt,}
\NormalTok{  indent{-}guides: none,}
\NormalTok{  indent{-}guides{-}offset: 0pt,}
\NormalTok{  row{-}gutter: 10pt,}
\NormalTok{  column{-}gutter: 10pt,}
\NormalTok{  inset: 10pt,}
\NormalTok{  fill: rgb(98\%, 98\%, 98\%),}
\NormalTok{  stroke: 1pt + rgb(50\%, 50\%, 50\%),}
\NormalTok{  radius: 0pt,}
\NormalTok{  breakable: false,}
\NormalTok{  block{-}align: center,}
\NormalTok{  main{-}text{-}styles: (:),}
\NormalTok{  comment{-}styles: (fill: rgb(45\%, 45\%, 45\%)),}
\NormalTok{  line{-}number{-}styles: (:)}
\NormalTok{)}
\end{Highlighting}
\end{Shaded}

\textbf{Parameters:}

\begin{itemize}
\item
  \texttt{\ body\ } : \texttt{\ content\ } â€'' Main algorithm content.
\item
  \texttt{\ header\ } : \texttt{\ content\ } â€'' Algorithm header. If
  specified, \texttt{\ title\ } and \texttt{\ parameters\ } are ignored.
\item
  \texttt{\ title\ } : \texttt{\ string\ } or \texttt{\ content\ } â€''
  Algorithm title. Ignored if \texttt{\ header\ } is specified.
\item
  \texttt{\ Parameters\ } : \texttt{\ array\ } â€'' List of algorithm
  parameters. Elements can be \texttt{\ string\ } or
  \texttt{\ content\ } values. \texttt{\ string\ } values will
  automatically be displayed in math mode. Ignored if
  \texttt{\ header\ } is specified.
\item
  \texttt{\ line-numbers\ } : \texttt{\ boolean\ } â€'' Whether to
  display line numbers.
\item
  \texttt{\ strong-keywords\ } : \texttt{\ boolean\ } â€'' Whether to
  strongly emphasize keywords.
\item
  \texttt{\ keywords\ } : \texttt{\ array\ } â€'' List of terms to
  receive strong emphasis. Elements must be \texttt{\ string\ } values.
  Ignored if \texttt{\ strong-keywords\ } is \texttt{\ false\ } .

  The default list of keywords is stored in
  \texttt{\ \_algo-default-keywords\ } . This list contains the
  following terms:

\begin{verbatim}
("if", "else", "then", "while", "for",
"repeat", "do", "until", ":", "end",
"and", "or", "not", "in", "to",
"down", "let", "return", "goto")
\end{verbatim}

  Note that for each of the above terms,
  \texttt{\ \_algo-default-keywords\ } also contains the uppercase form
  of the term (e.g. “for� and “For�).
\item
  \texttt{\ comment-prefix\ } : \texttt{\ content\ } â€'' What to
  prepend comments with.
\item
  \texttt{\ indent-size\ } : \texttt{\ length\ } â€'' Size of line
  indentations.
\item
  \texttt{\ indent-guides\ } : \texttt{\ stroke\ } â€'' Stroke for
  indent guides.
\item
  \texttt{\ indent-guides-offset\ } : \texttt{\ length\ } â€''
  Horizontal offset of indent guides.
\item
  \texttt{\ row-gutter\ } : \texttt{\ length\ } â€'' Space between
  lines.
\item
  \texttt{\ column-gutter\ } : \texttt{\ length\ } â€'' Space between
  line numbers, text, and comments.
\item
  \texttt{\ inset\ } : \texttt{\ length\ } â€'' Size of inner padding.
\item
  \texttt{\ fill\ } : \texttt{\ color\ } â€'' Fill color.
\item
  \texttt{\ stroke\ } : \texttt{\ stroke\ } â€'' Stroke for the
  element’s border.
\item
  \texttt{\ radius\ } : \texttt{\ length\ } â€'' Corner radius.
\item
  \texttt{\ breakable\ } : \texttt{\ boolean\ } â€'' Whether the element
  can break across pages. WARNING: indent guides may look off when
  broken across pages.
\item
  \texttt{\ block-align\ } : \texttt{\ none\ } or \texttt{\ alignment\ }
  or \texttt{\ 2d\ alignment\ } â€'' Alignment of the \texttt{\ algo\ }
  on the page. Using \texttt{\ none\ } will cause the internal
  \texttt{\ block\ } element to be returned as-is.
\item
  \texttt{\ main-text-styles\ } : \texttt{\ dictionary\ } â€'' Styling
  options for the main algorithm text. Supports all parameters in
  Typst’s native \texttt{\ text\ } function.
\item
  \texttt{\ comment-styles\ } : \texttt{\ dictionary\ } â€'' Styling
  options for comment text. Supports all parameters in Typst’s native
  \texttt{\ text\ } function.
\item
  \texttt{\ line-number-styles\ } : \texttt{\ dictionary\ } â€'' Styling
  options for line numbers. Supports all parameters in Typst’s native
  \texttt{\ text\ } function.
\end{itemize}

\subsubsection{\texorpdfstring{\texttt{\ i\ } and
\texttt{\ d\ }}{ i  and  d }}\label{i-and-d}

For use in an \texttt{\ algo\ } body. \texttt{\ \#i\ } indents all
following lines and \texttt{\ \#d\ } dedents all following lines.

\subsubsection{\texorpdfstring{\texttt{\ comment\ }}{ comment }}\label{comment}

For use in an \texttt{\ algo\ } body. Adds a comment to the line in
which it’s placed.

\begin{Shaded}
\begin{Highlighting}[]
\NormalTok{comment(}
\NormalTok{  body,}
\NormalTok{  inline: false}
\NormalTok{)}
\end{Highlighting}
\end{Shaded}

\textbf{Parameters:}

\begin{itemize}
\item
  \texttt{\ body\ } : \texttt{\ content\ } â€'' Comment content.
\item
  \texttt{\ inline\ } : \texttt{\ boolean\ } â€'' If true, the comment
  is displayed in place rather than on the right side.

  NOTE: inline comments will respect both \texttt{\ main-text-styles\ }
  and \texttt{\ comment-styles\ } , preferring
  \texttt{\ comment-styles\ } when the two conflict.

  NOTE: to make inline comments insensitive to
  \texttt{\ strong-keywords\ } , strong emphasis is disabled within
  them. This can be circumvented via the \texttt{\ text\ } function:

\begin{Shaded}
\begin{Highlighting}[]
\NormalTok{\#comment(inline: true)[\#text(weight: 700)[...]]}
\end{Highlighting}
\end{Shaded}
\end{itemize}

\subsubsection{\texorpdfstring{\texttt{\ no-emph\ }}{ no-emph }}\label{no-emph}

For use in an \texttt{\ algo\ } body. Prevents the passed content from
being strongly emphasized. If a word appears in your algorithm both as a
keyword and as normal text, you may escape the non-keyword usages via
this function.

\begin{Shaded}
\begin{Highlighting}[]
\NormalTok{no{-}emph(}
\NormalTok{  body}
\NormalTok{)}
\end{Highlighting}
\end{Shaded}

\textbf{Parameters:}

\begin{itemize}
\tightlist
\item
  \texttt{\ body\ } : \texttt{\ content\ } â€'' Content to display
  without emphasis.
\end{itemize}

\subsubsection{\texorpdfstring{\texttt{\ code\ }}{ code }}\label{code}

Makes a code block element.

\begin{Shaded}
\begin{Highlighting}[]
\NormalTok{code(}
\NormalTok{  body,}
\NormalTok{  line{-}numbers: true,}
\NormalTok{  indent{-}guides: none,}
\NormalTok{  indent{-}guides{-}offset: 0pt,}
\NormalTok{  tab{-}size: auto,}
\NormalTok{  row{-}gutter: 10pt,}
\NormalTok{  column{-}gutter: 10pt,}
\NormalTok{  inset: 10pt,}
\NormalTok{  fill: rgb(98\%, 98\%, 98\%),}
\NormalTok{  stroke: 1pt + rgb(50\%, 50\%, 50\%),}
\NormalTok{  radius: 0pt,}
\NormalTok{  breakable: false,}
\NormalTok{  block{-}align: center,}
\NormalTok{  main{-}text{-}styles: (:),}
\NormalTok{  line{-}number{-}styles: (:)}
\NormalTok{)}
\end{Highlighting}
\end{Shaded}

\textbf{Parameters:}

\begin{itemize}
\item
  \texttt{\ body\ } : \texttt{\ content\ } â€'' Main content. Expects
  \texttt{\ raw\ } text.
\item
  \texttt{\ line-numbers\ } : \texttt{\ boolean\ } â€'' Whether to
  display line numbers.
\item
  \texttt{\ indent-guides\ } : \texttt{\ stroke\ } â€'' Stroke for
  indent guides.
\item
  \texttt{\ indent-guides-offset\ } : \texttt{\ length\ } â€''
  Horizontal offset of indent guides.
\item
  \texttt{\ tab-size\ } : \texttt{\ integer\ } â€'' Amount of spaces
  that should be considered an indent. If unspecified, the tab size is
  determined automatically from the first instance of starting
  whitespace.
\item
  \texttt{\ row-gutter\ } : \texttt{\ length\ } â€'' Space between
  lines.
\item
  \texttt{\ column-gutter\ } : \texttt{\ length\ } â€'' Space between
  line numbers and text.
\item
  \texttt{\ inset\ } : \texttt{\ length\ } â€'' Size of inner padding.
\item
  \texttt{\ fill\ } : \texttt{\ color\ } â€'' Fill color.
\item
  \texttt{\ stroke\ } : \texttt{\ stroke\ } â€'' Stroke for the
  element’s border.
\item
  \texttt{\ radius\ } : \texttt{\ length\ } â€'' Corner radius.
\item
  \texttt{\ breakable\ } : \texttt{\ boolean\ } â€'' Whether the element
  can break across pages. WARNING: indent guides may look off when
  broken across pages.
\item
  \texttt{\ block-align\ } : \texttt{\ none\ } or \texttt{\ alignment\ }
  or \texttt{\ 2d\ alignment\ } â€'' Alignment of the \texttt{\ code\ }
  on the page. Using \texttt{\ none\ } will cause the internal
  \texttt{\ block\ } element to be returned as-is.
\item
  \texttt{\ main-text-styles\ } : \texttt{\ dictionary\ } â€'' Styling
  options for the main raw text. Supports all parameters in Typst’s
  native \texttt{\ text\ } function.
\item
  \texttt{\ line-number-styles\ } : \texttt{\ dictionary\ } â€'' Styling
  options for line numbers. Supports all parameters in Typst’s native
  \texttt{\ text\ } function.
\end{itemize}

\subsection{Contributing}\label{contributing}

PRs are welcome! And if you encounter any bugs or have any
requests/ideas, feel free to open an issue.

\subsubsection{How to add}\label{how-to-add}

Copy this into your project and use the import as \texttt{\ algo\ }

\begin{verbatim}
#import "@preview/algo:0.3.4"
\end{verbatim}

\includesvg[width=0.16667in,height=0.16667in]{/assets/icons/16-copy.svg}

Check the docs for
\href{https://typst.app/docs/reference/scripting/\#packages}{more
information on how to import packages} .

\subsubsection{About}\label{about}

\begin{description}
\tightlist
\item[Author :]
\href{https://github.com/platformer}{platformer}
\item[License:]
MIT
\item[Current version:]
0.3.4
\item[Last updated:]
November 12, 2024
\item[First released:]
August 8, 2023
\item[Archive size:]
10.5 kB
\href{https://packages.typst.org/preview/algo-0.3.4.tar.gz}{\pandocbounded{\includesvg[keepaspectratio]{/assets/icons/16-download.svg}}}
\item[Repository:]
\href{https://github.com/platformer/typst-algorithms}{GitHub}
\item[Discipline :]
\begin{itemize}
\tightlist
\item[]
\item
  \href{https://typst.app/universe/search/?discipline=computer-science}{Computer
  Science}
\end{itemize}
\item[Categor y :]
\begin{itemize}
\tightlist
\item[]
\item
  \pandocbounded{\includesvg[keepaspectratio]{/assets/icons/16-package.svg}}
  \href{https://typst.app/universe/search/?category=components}{Components}
\end{itemize}
\end{description}

\subsubsection{Where to report issues?}\label{where-to-report-issues}

This package is a project of platformer . Report issues on
\href{https://github.com/platformer/typst-algorithms}{their repository}
. You can also try to ask for help with this package on the
\href{https://forum.typst.app}{Forum} .

Please report this package to the Typst team using the
\href{https://typst.app/contact}{contact form} if you believe it is a
safety hazard or infringes upon your rights.

\phantomsection\label{versions}
\subsubsection{Version history}\label{version-history}

\begin{longtable}[]{@{}ll@{}}
\toprule\noalign{}
Version & Release Date \\
\midrule\noalign{}
\endhead
\bottomrule\noalign{}
\endlastfoot
0.3.4 & November 12, 2024 \\
\href{https://typst.app/universe/package/algo/0.3.3/}{0.3.3} & September
21, 2023 \\
\href{https://typst.app/universe/package/algo/0.3.2/}{0.3.2} & September
3, 2023 \\
\href{https://typst.app/universe/package/algo/0.3.1/}{0.3.1} & August
19, 2023 \\
\href{https://typst.app/universe/package/algo/0.3.0/}{0.3.0} & August 8,
2023 \\
\end{longtable}

Typst GmbH did not create this package and cannot guarantee correct
functionality of this package or compatibility with any version of the
Typst compiler or app.
