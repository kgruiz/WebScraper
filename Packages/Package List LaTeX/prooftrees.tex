\title{typst.app/universe/package/prooftrees}

\phantomsection\label{banner}
\section{prooftrees}\label{prooftrees}

{ 0.1.0 }

Proof trees for natural deduction and type theories

\phantomsection\label{readme}
This package is for constructing proof trees in the style of natural
deduction or the sequent calculus, for \texttt{\ typst\ }
\texttt{\ 0.7.0\ } . Please do open issues for bugs etc :)

Features:

\begin{itemize}
\tightlist
\item
  Inferences can have arbitrarily many premises.
\item
  Inference lines can have left and/or right labels¹
\item
  Configurable² per tree and per line: premise spacing, the line
  stroke, etc… .
\item
  They’re proof trees.
\end{itemize}

¹ The placement of labels is currently very primitive, and requires
much user intervention.

² Options are quite limited.

\subsection{Usage}\label{usage}

The user interface is inspired by
\href{https://ctan.org/pkg/bussproofs}{bussproof} ’s; a tree is
constructed by a sequence of ‘lines’ that state their number of
premises.
\href{https://github.com/typst/packages/raw/main/packages/preview/prooftrees/0.1.0/src/prooftrees.typ}{\texttt{\ src/prooftrees.typ\ }}
contains the documentation and the main functions needed.

The code for some example trees can be seen in
\texttt{\ examples/prooftree\_test.typ\ } .

\subsubsection{Examples}\label{examples}

A single inference would be:

\begin{Shaded}
\begin{Highlighting}[]
\NormalTok{\#import "@preview/prooftrees:0.1.0"}

\NormalTok{\#prooftree.tree(}
\NormalTok{    prooftree.axi[$A =\textgreater{} A$],}
\NormalTok{    prooftree.uni[$A =\textgreater{} A, B$]}
\NormalTok{)}
\end{Highlighting}
\end{Shaded}

\includegraphics[width=0.3\linewidth,height=\textheight,keepaspectratio]{https://raw.githubusercontent.com/david-davies/typst-prooftree/main/examples/Example1.png}

A more interesting example:

\begin{Shaded}
\begin{Highlighting}[]
\NormalTok{\#import "@preview/prooftrees:0.1.0"}

\NormalTok{\#prooftree.tree(}
\NormalTok{    prooftree.axi[$B =\textgreater{} B$],}
\NormalTok{    prooftree.uni[$B =\textgreater{} B, A$],}
\NormalTok{    prooftree.uni[$B =\textgreater{} A, B$],}
\NormalTok{        prooftree.axi[$A =\textgreater{} A$],}
\NormalTok{        prooftree.uni[$A =\textgreater{} A, B$],}
\NormalTok{    prooftree.bin[$B =\textgreater{} A, B$]}
\NormalTok{)}
\end{Highlighting}
\end{Shaded}

\includegraphics[width=0.4\linewidth,height=\textheight,keepaspectratio]{https://raw.githubusercontent.com/david-davies/typst-prooftree/main/examples/Example2.png}

An n-ary inference can be made:

\begin{Shaded}
\begin{Highlighting}[]
\NormalTok{\#import "@preview/prooftrees:0.1.0"}

\NormalTok{\#prooftrees.tree(}
\NormalTok{    prooftrees.axi(pad(bottom: 2pt, [$P\_1$])),}
\NormalTok{    prooftrees.axi(pad(bottom: 2pt, [$P\_2$])),}
\NormalTok{    prooftrees.axi(pad(bottom: 2pt, [$P\_3$])),}
\NormalTok{    prooftrees.axi(pad(bottom: 2pt, [$P\_4$])),}
\NormalTok{    prooftrees.axi(pad(bottom: 2pt, [$P\_5$])),}
\NormalTok{    prooftrees.axi(pad(bottom: 2pt, [$P\_6$])),}
\NormalTok{    prooftrees.nary(6)[$C$],}
\NormalTok{)}
\end{Highlighting}
\end{Shaded}

\includegraphics[width=0.3\linewidth,height=\textheight,keepaspectratio]{https://raw.githubusercontent.com/david-davies/typst-prooftree/main/examples/Example3.png}

\subsection{Known Issues:}\label{known-issues}

\subsubsection{Superscripts and subscripts clip with the
line}\label{superscripts-and-subscripts-clip-with-the-line}

The boundaries of blocks containing math do not expand enough for
sub/pscripts; I think this is a typst issue. Short-term fix: add manual
vspace or padding in the cell.

\subsection{Implementation}\label{implementation}

The placement of the line and conclusion is calculated using
\texttt{\ measure\ } on the premises and labels, and doing geometric
arithmetic with these values.

\subsubsection{How to add}\label{how-to-add}

Copy this into your project and use the import as
\texttt{\ prooftrees\ }

\begin{verbatim}
#import "@preview/prooftrees:0.1.0"
\end{verbatim}

\includesvg[width=0.16667in,height=0.16667in]{/assets/icons/16-copy.svg}

Check the docs for
\href{https://typst.app/docs/reference/scripting/\#packages}{more
information on how to import packages} .

\subsubsection{About}\label{about}

\begin{description}
\tightlist
\item[Author :]
\href{https://github.com/david-davies}{david-davies}
\item[License:]
MIT
\item[Current version:]
0.1.0
\item[Last updated:]
September 3, 2023
\item[First released:]
September 3, 2023
\item[Archive size:]
8.43 kB
\href{https://packages.typst.org/preview/prooftrees-0.1.0.tar.gz}{\pandocbounded{\includesvg[keepaspectratio]{/assets/icons/16-download.svg}}}
\item[Repository:]
\href{https://github.com/david-davies/typst-prooftree}{GitHub}
\end{description}

\subsubsection{Where to report issues?}\label{where-to-report-issues}

This package is a project of david-davies . Report issues on
\href{https://github.com/david-davies/typst-prooftree}{their repository}
. You can also try to ask for help with this package on the
\href{https://forum.typst.app}{Forum} .

Please report this package to the Typst team using the
\href{https://typst.app/contact}{contact form} if you believe it is a
safety hazard or infringes upon your rights.

\phantomsection\label{versions}
\subsubsection{Version history}\label{version-history}

\begin{longtable}[]{@{}ll@{}}
\toprule\noalign{}
Version & Release Date \\
\midrule\noalign{}
\endhead
\bottomrule\noalign{}
\endlastfoot
0.1.0 & September 3, 2023 \\
\end{longtable}

Typst GmbH did not create this package and cannot guarantee correct
functionality of this package or compatibility with any version of the
Typst compiler or app.
