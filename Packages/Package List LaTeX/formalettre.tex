\title{typst.app/universe/package/formalettre}

\phantomsection\label{banner}
\phantomsection\label{template-thumbnail}
\pandocbounded{\includegraphics[keepaspectratio]{https://packages.typst.org/preview/thumbnails/formalettre-0.1.1-small.webp}}

\section{formalettre}\label{formalettre}

{ 0.1.1 }

French formal letter template

\href{/app?template=formalettre&version=0.1.1}{Create project in app}

\phantomsection\label{readme}
Un template destiné Ã~ écrire des lettres selon une typographie
francophone, et inspiré du package LaTeX
\href{https://ctan.org/pkg/lettre}{lettre} .

Pour utiliser le template, il est possible de recopier le fichier
exemple.

\subsection{Documentation des
variables}\label{documentation-des-variables}

\subsubsection{Expéditeur}\label{expuxe3diteur}

\begin{itemize}
\tightlist
\item
  \texttt{\ expediteur.nom\ } : nom de famille de l’expéditeur·ice,
  \textbf{requis} .
\item
  \texttt{\ expediteur.prenom\ } : prénom de l’expéditeur·ice,
  \textbf{requis} .
\item
  \texttt{\ expediteur.voie\ } : numéro de voie et nom de la voie,
  \textbf{requis} .
\item
  \texttt{\ expediteur.complement\_adresse\ } : la seconde ligne parfois
  requise dans une adresse, \emph{facultatif} .
\item
  \texttt{\ expediteur.code\_postal\ } : code postal, \textbf{requis} .
\item
  \texttt{\ expediteur.commune\ } : commune de l’expéditeur·ice,
  \textbf{requis} .
\item
  \texttt{\ expediteur.telephone\ } : numéro de téléphone. Le format
  est libre et l’affichage en police mono. \emph{Facultatif} .
\item
  \texttt{\ expediteur.email\ } : l’email fourni sera affiché en
  police mono et cliquable. \emph{Facultatif}
\item
  \texttt{\ expediteur.signature\ } : peut être \texttt{\ true\ } ou
  \texttt{\ false\ } , par défaut \texttt{\ false\ } . Prévient le
  paquet qu’une image de signature sera ajoutée, de manière Ã~
  organiser la superposition de la signature et du nom apposé en fin de
  courrier.
\end{itemize}

\subsection{Destinataire}\label{destinataire}

\begin{itemize}
\tightlist
\item
  \texttt{\ destinataire.titre\ } : titre du ou de la destinataire,
  \textbf{requis} .
\item
  \texttt{\ destinataire.voie\ } : numéro de voie et nom de la voie,
  \textbf{requis} .
\item
  \texttt{\ destinataire.complement\_adresse\ } : la seconde ligne
  parfois requise dans une adresse, \emph{facultatif} .
\item
  \texttt{\ destinataire.code\_postal\ } : code postal, \textbf{requis}
  .
\item
  \texttt{\ destinataire.commune\ } : commune de l’expéditeur·ice,
  \textbf{requis} .
\item
  \texttt{\ destinataire.sc\ } : si le courrier est envoyé “sous
  couvert� d’une hiérarchie intermédiaire, spécifier cette
  autorité. \emph{Facultatif} .
\end{itemize}

\subsection{Lettre}\label{lettre}

\begin{itemize}
\tightlist
\item
  \texttt{\ objet\ } : l’objet du courrier, \textbf{requis} .
\item
  \texttt{\ date\ } : date Ã~ indiquer sous forme libre, \textbf{requis}
  .
\item
  \texttt{\ lieu\ } : lieu de rédaction, \textbf{requis} .
\item
  \texttt{\ pj\ } : permet d’indiquer la présence de pièces jointes.
  Il est possible d’en faire une liste, par exemple :
\end{itemize}

\begin{verbatim}
pj: [
    + Dossier n°1
    + Dossier n° 2
    + Attestation
    ]
\end{verbatim}

Le texte de la lettre proprement dite se situe après la configuration
de la lettre.

À la fin de la lettre, il est possible de décommenter les deux
dernières lignes pour ajouter une image en guise de signature. Veillez
dans ce cas Ã~ positionner la varibale \texttt{\ expediteur.signature\ }
à \texttt{\ true\ } .

\href{/app?template=formalettre&version=0.1.1}{Create project in app}

\subsubsection{How to use}\label{how-to-use}

Click the button above to create a new project using this template in
the Typst app.

You can also use the Typst CLI to start a new project on your computer
using this command:

\begin{verbatim}
typst init @preview/formalettre:0.1.1
\end{verbatim}

\includesvg[width=0.16667in,height=0.16667in]{/assets/icons/16-copy.svg}

\subsubsection{About}\label{about}

\begin{description}
\tightlist
\item[Author :]
@Brndan
\item[License:]
BSD-3-Clause
\item[Current version:]
0.1.1
\item[Last updated:]
October 23, 2024
\item[First released:]
October 22, 2024
\item[Archive size:]
3.20 kB
\href{https://packages.typst.org/preview/formalettre-0.1.1.tar.gz}{\pandocbounded{\includesvg[keepaspectratio]{/assets/icons/16-download.svg}}}
\item[Repository:]
\href{https://github.com/Brndan/lettre}{GitHub}
\item[Categor y :]
\begin{itemize}
\tightlist
\item[]
\item
  \pandocbounded{\includesvg[keepaspectratio]{/assets/icons/16-envelope.svg}}
  \href{https://typst.app/universe/search/?category=office}{Office}
\end{itemize}
\end{description}

\subsubsection{Where to report issues?}\label{where-to-report-issues}

This template is a project of @Brndan . Report issues on
\href{https://github.com/Brndan/lettre}{their repository} . You can also
try to ask for help with this template on the
\href{https://forum.typst.app}{Forum} .

Please report this template to the Typst team using the
\href{https://typst.app/contact}{contact form} if you believe it is a
safety hazard or infringes upon your rights.

\phantomsection\label{versions}
\subsubsection{Version history}\label{version-history}

\begin{longtable}[]{@{}ll@{}}
\toprule\noalign{}
Version & Release Date \\
\midrule\noalign{}
\endhead
\bottomrule\noalign{}
\endlastfoot
0.1.1 & October 23, 2024 \\
\href{https://typst.app/universe/package/formalettre/0.1.0/}{0.1.0} &
October 22, 2024 \\
\end{longtable}

Typst GmbH did not create this template and cannot guarantee correct
functionality of this template or compatibility with any version of the
Typst compiler or app.
