\title{typst.app/universe/package/outrageous}

\phantomsection\label{banner}
\section{outrageous}\label{outrageous}

{ 0.3.0 }

Easier customization of outline entries.

\phantomsection\label{readme}
Easier customization of outline entries.

\subsection{Examples}\label{examples}

For the full source see
\href{https://github.com/typst/packages/raw/main/packages/preview/outrageous/0.3.0/examples/basic.typ}{\texttt{\ examples/basic.typ\ }}
and for more examples see the
\href{https://github.com/typst/packages/raw/main/packages/preview/outrageous/0.3.0/examples}{\texttt{\ examples\ }
directory} .

\subsubsection{Default Style}\label{default-style}

\pandocbounded{\includegraphics[keepaspectratio]{https://github.com/typst/packages/raw/main/packages/preview/outrageous/0.3.0/example-default.png}}

\begin{Shaded}
\begin{Highlighting}[]
\NormalTok{\#import "@preview/outrageous:0.1.0"}
\NormalTok{\#show outline.entry: outrageous.show{-}entry}
\end{Highlighting}
\end{Shaded}

\subsubsection{Custom Settings}\label{custom-settings}

\pandocbounded{\includegraphics[keepaspectratio]{https://github.com/typst/packages/raw/main/packages/preview/outrageous/0.3.0/example-custom.png}}

\begin{Shaded}
\begin{Highlighting}[]
\NormalTok{\#import "@preview/outrageous:0.1.0"}
\NormalTok{\#show outline.entry: outrageous.show{-}entry.with(}
\NormalTok{  // the typst preset retains the normal Typst appearance}
\NormalTok{  ..outrageous.presets.typst,}
\NormalTok{  // we only override a few things:}
\NormalTok{  // level{-}1 entries are italic, all others keep their font style}
\NormalTok{  font{-}style: ("italic", auto),}
\NormalTok{  // no fill for level{-}1 entries, a thin gray line for all deeper levels}
\NormalTok{  fill: (none, line(length: 100\%, stroke: gray + .5pt)),}
\NormalTok{)}
\end{Highlighting}
\end{Shaded}

\subsection{Usage}\label{usage}

\subsubsection{\texorpdfstring{\texttt{\ show-entry\ }}{ show-entry }}\label{show-entry}

Show the given outline entry with the provided styling. Should be used
in a show rule like
\texttt{\ \#show\ outline.entry:\ outrageous.show-entry\ } .

\begin{Shaded}
\begin{Highlighting}[]
\NormalTok{\#let show{-}entry(}
\NormalTok{  entry,}
\NormalTok{  font{-}weight: presets.outrageous{-}toc.font{-}weight,}
\NormalTok{  font{-}style: presets.outrageous{-}toc.font{-}style,}
\NormalTok{  vspace: presets.outrageous{-}toc.vspace,}
\NormalTok{  font: presets.outrageous{-}toc.font,}
\NormalTok{  fill: presets.outrageous{-}toc.fill,}
\NormalTok{  fill{-}right{-}pad: presets.outrageous{-}toc.fill{-}right{-}pad,}
\NormalTok{  fill{-}align: presets.outrageous{-}toc.fill{-}align,}
\NormalTok{  body{-}transform: presets.outrageous{-}toc.body{-}transform,}
\NormalTok{  label: \textless{}outrageous{-}modified{-}entry\textgreater{},}
\NormalTok{  state{-}key: "outline{-}page{-}number{-}max{-}width",}
\NormalTok{) = \{ .. \}}
\end{Highlighting}
\end{Shaded}

\textbf{Arguments:}

For all the arguments that take arrays, the array’s first item
specifies the value for all level-one entries, the second item for
level-two, and so on. The array’s last item will be used for all
deeper/following levels as well.

\begin{itemize}
\tightlist
\item
  \texttt{\ entry\ } :
  \href{https://typst.app/docs/reference/foundations/content/}{\texttt{\ content\ }}
  â€'' The
  \href{https://typst.app/docs/reference/model/outline/\#definitions-entry}{\texttt{\ outline.entry\ }}
  element from the show rule.
\item
  \texttt{\ font-weight\ } :
  \href{https://typst.app/docs/reference/foundations/array/}{\texttt{\ array\ }}
  of (
  \href{https://typst.app/docs/reference/foundations/str/}{\texttt{\ str\ }}
  or
  \href{https://typst.app/docs/reference/foundations/int/}{\texttt{\ int\ }}
  or \texttt{\ auto\ } or \texttt{\ none\ } ) â€'' The entry’s font
  weight. Setting to \texttt{\ auto\ } or \texttt{\ none\ } keeps the
  context’s current style.
\item
  \texttt{\ font-style\ } :
  \href{https://typst.app/docs/reference/foundations/array/}{\texttt{\ array\ }}
  of (
  \href{https://typst.app/docs/reference/foundations/str/}{\texttt{\ str\ }}
  or \texttt{\ auto\ } or \texttt{\ none\ } ) â€'' The entry’s font
  style. Setting to \texttt{\ auto\ } or \texttt{\ none\ } keeps the
  context’s current style.
\item
  \texttt{\ vspace\ } :
  \href{https://typst.app/docs/reference/foundations/array/}{\texttt{\ array\ }}
  of (
  \href{https://typst.app/docs/reference/layout/relative/}{\texttt{\ relative\ }}
  or
  \href{https://typst.app/docs/reference/layout/fraction/}{\texttt{\ fraction\ }}
  or \texttt{\ none\ } ) â€'' Vertical spacing to add above the entry.
  Setting to \texttt{\ none\ } adds no space.
\item
  \texttt{\ font\ } :
  \href{https://typst.app/docs/reference/foundations/array/}{\texttt{\ array\ }}
  of (
  \href{https://typst.app/docs/reference/foundations/str/}{\texttt{\ str\ }}
  or
  \href{https://typst.app/docs/reference/foundations/array/}{\texttt{\ array\ }}
  or \texttt{\ auto\ } or \texttt{\ none\ } ) â€'' The entry’s font.
  Setting to \texttt{\ auto\ } or \texttt{\ none\ } keeps the
  context’s current font.
\item
  \texttt{\ fill\ } :
  \href{https://typst.app/docs/reference/foundations/array/}{\texttt{\ array\ }}
  of (
  \href{https://typst.app/docs/reference/foundations/content/}{\texttt{\ content\ }}
  or \texttt{\ auto\ } or \texttt{\ none\ } ) â€'' The entry’s fill.
  Setting to \texttt{\ auto\ } keeps the context’s current setting.
\item
  \texttt{\ fill-right-pad\ } :
  \href{https://typst.app/docs/reference/layout/relative/}{\texttt{\ relative\ }}
  or \texttt{\ none\ } â€'' Horizontal space to put between the fill and
  page number.
\item
  \texttt{\ fill-align\ } :
  \href{https://typst.app/docs/reference/foundations/bool/}{\texttt{\ bool\ }}
  â€'' Whether \texttt{\ fill-right-pad\ } should be relative to the
  current page number or the widest page number. Setting this to
  \texttt{\ true\ } has the effect of all fills ending on the same
  vertical line.
\item
  \texttt{\ body-transform\ } :
  \href{https://typst.app/docs/reference/foundations/function/}{\texttt{\ function\ }}
  or \texttt{\ none\ } â€'' Callback for custom edits to the entry’s
  body. It gets passed the entry’s level (
  \href{https://typst.app/docs/reference/foundations/int/}{\texttt{\ int\ }}
  ) and body (
  \href{https://typst.app/docs/reference/foundations/content/}{\texttt{\ content\ }}
  ) and should return
  \href{https://typst.app/docs/reference/foundations/content/}{\texttt{\ content\ }}
  or \texttt{\ none\ } . If \texttt{\ none\ } is returned, no
  modifications are made.
\item
  \texttt{\ page-transform\ } :
  \href{https://typst.app/docs/reference/foundations/function/}{\texttt{\ function\ }}
  or \texttt{\ none\ } â€'' Callback for custom edits to the entry’s
  page number. It gets passed the entry’s level (
  \href{https://typst.app/docs/reference/foundations/int/}{\texttt{\ int\ }}
  ) and page number (
  \href{https://typst.app/docs/reference/foundations/content/}{\texttt{\ content\ }}
  ) and should return
  \href{https://typst.app/docs/reference/foundations/content/}{\texttt{\ content\ }}
  or \texttt{\ none\ } . If \texttt{\ none\ } is returned, no
  modifications are made.
\item
  \texttt{\ label\ } :
  \href{https://typst.app/docs/reference/foundations/label/}{\texttt{\ label\ }}
  â€'' The label to internally use for tracking recursion. This should
  not need to be modified.
\item
  \texttt{\ state-key\ } :
  \href{https://typst.app/docs/reference/foundations/str/}{\texttt{\ str\ }}
  â€'' The key to use for the internal state which tracks the maximum
  page number width. The state is global for the entire document and
  thus applies to all outlines. If you wish to re-calculate the max page
  number width for \texttt{\ fill-align\ } , then you must provide a
  different key for each outline.
\end{itemize}

\textbf{Returns:}
\href{https://typst.app/docs/reference/foundations/content/}{\texttt{\ content\ }}

\subsubsection{\texorpdfstring{\texttt{\ presets\ }}{ presets }}\label{presets}

Presets for the arguments for
\href{https://github.com/typst/packages/raw/main/packages/preview/outrageous/0.3.0/\#show-entry}{\texttt{\ show-entry()\ }}
. You can use them in your show rule with
\texttt{\ \#show\ outline.entry:\ outrageous.show-entry.with(..outrageous.presets.outrageous-figures)\ }
.

\begin{Shaded}
\begin{Highlighting}[]
\NormalTok{\#let presets = (}
\NormalTok{  // outrageous preset for a Table of Contents}
\NormalTok{  outrageous{-}toc: (}
\NormalTok{    // ...}
\NormalTok{  ),}
\NormalTok{  // outrageous preset for List of Figures/Tables/Listings}
\NormalTok{  outrageous{-}figures: (}
\NormalTok{    // ...}
\NormalTok{  ),}
\NormalTok{  // preset without any style changes}
\NormalTok{  typst: (}
\NormalTok{    // ...}
\NormalTok{  ),}
\NormalTok{)}
\end{Highlighting}
\end{Shaded}

\subsubsection{\texorpdfstring{\texttt{\ repeat\ }}{ repeat }}\label{repeat}

Utility function to repeat content to fill space with a fixed size gap.

\begin{Shaded}
\begin{Highlighting}[]
\NormalTok{\#let repeat(gap: none, justify: false, body) = \{ .. \}}
\end{Highlighting}
\end{Shaded}

\textbf{Arguments:}

\begin{itemize}
\tightlist
\item
  \texttt{\ gap\ } :
  \href{https://typst.app/docs/reference/layout/length/}{\texttt{\ length\ }}
  or \texttt{\ none\ } â€'' The gap between repeated items.
\item
  \texttt{\ justify\ } :
  \href{https://typst.app/docs/reference/foundations/bool/}{\texttt{\ bool\ }}
  â€'' Whether to increase the gap to justify the items.
\item
  \texttt{\ body\ } :
  \href{https://typst.app/docs/reference/foundations/content/}{\texttt{\ content\ }}
  â€'' The content to repeat.
\end{itemize}

\textbf{Returns:}
\href{https://typst.app/docs/reference/foundations/content/}{\texttt{\ content\ }}

\subsubsection{\texorpdfstring{\texttt{\ align-helper\ }}{ align-helper }}\label{align-helper}

Utility function to help with aligning multiple items.

\begin{Shaded}
\begin{Highlighting}[]
\NormalTok{\#let align{-}helper(state{-}key, what{-}to{-}measure, display) = \{ .. \}}
\end{Highlighting}
\end{Shaded}

\textbf{Arguments:}

\begin{itemize}
\tightlist
\item
  \texttt{\ state-key\ } :
  \href{https://typst.app/docs/reference/foundations/str/}{\texttt{\ str\ }}
  â€'' The key to use for the
  \href{https://typst.app/docs/reference/introspection/state/}{\texttt{\ state\ }}
  that keeps track of the maximum encountered width.
\item
  \texttt{\ what-to-measure\ } :
  \href{https://typst.app/docs/reference/foundations/content/}{\texttt{\ content\ }}
  â€'' The content to measure at this call.
\item
  \texttt{\ display\ } :
  \href{https://typst.app/docs/reference/foundations/function/}{\texttt{\ function\ }}
  â€'' A callback which gets passed the maximum encountered width and
  the width of the current item (what was given to
  \texttt{\ what-to-measure\ } ), both as
  \href{https://typst.app/docs/reference/layout/length/}{\texttt{\ length\ }}
  , and should return
  \href{https://typst.app/docs/reference/foundations/content/}{\texttt{\ content\ }}
  which can make use of these widths for alignment.
\end{itemize}

\textbf{Returns:}
\href{https://typst.app/docs/reference/foundations/content/}{\texttt{\ content\ }}

\subsubsection{How to add}\label{how-to-add}

Copy this into your project and use the import as
\texttt{\ outrageous\ }

\begin{verbatim}
#import "@preview/outrageous:0.3.0"
\end{verbatim}

\includesvg[width=0.16667in,height=0.16667in]{/assets/icons/16-copy.svg}

Check the docs for
\href{https://typst.app/docs/reference/scripting/\#packages}{more
information on how to import packages} .

\subsubsection{About}\label{about}

\begin{description}
\tightlist
\item[Author :]
RubixDev
\item[License:]
GPL-3.0-only
\item[Current version:]
0.3.0
\item[Last updated:]
October 21, 2024
\item[First released:]
October 9, 2023
\item[Minimum Typst version:]
0.11.0
\item[Archive size:]
15.8 kB
\href{https://packages.typst.org/preview/outrageous-0.3.0.tar.gz}{\pandocbounded{\includesvg[keepaspectratio]{/assets/icons/16-download.svg}}}
\item[Repository:]
\href{https://github.com/RubixDev/typst-outrageous}{GitHub}
\end{description}

\subsubsection{Where to report issues?}\label{where-to-report-issues}

This package is a project of RubixDev . Report issues on
\href{https://github.com/RubixDev/typst-outrageous}{their repository} .
You can also try to ask for help with this package on the
\href{https://forum.typst.app}{Forum} .

Please report this package to the Typst team using the
\href{https://typst.app/contact}{contact form} if you believe it is a
safety hazard or infringes upon your rights.

\phantomsection\label{versions}
\subsubsection{Version history}\label{version-history}

\begin{longtable}[]{@{}ll@{}}
\toprule\noalign{}
Version & Release Date \\
\midrule\noalign{}
\endhead
\bottomrule\noalign{}
\endlastfoot
0.3.0 & October 21, 2024 \\
\href{https://typst.app/universe/package/outrageous/0.2.0/}{0.2.0} &
September 14, 2024 \\
\href{https://typst.app/universe/package/outrageous/0.1.0/}{0.1.0} &
October 9, 2023 \\
\end{longtable}

Typst GmbH did not create this package and cannot guarantee correct
functionality of this package or compatibility with any version of the
Typst compiler or app.
