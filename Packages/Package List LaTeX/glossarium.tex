\title{typst.app/universe/package/glossarium}

\phantomsection\label{banner}
\section{glossarium}\label{glossarium}

{ 0.5.1 }

Glossarium is a simple, easily customizable typst glossary.

{ } Featured Package

\phantomsection\label{readme}
\begin{quote}
{[}!TIP{]} Glossarium is based in great part of the work of
\href{https://github.com/Dherse}{Sébastien d’Herbais de Thun} from
his master thesis available at:
\url{https://github.com/Dherse/masterproef} . His glossary is available
under the MIT license
\href{https://github.com/Dherse/masterproef/blob/main/elems/acronyms.typ}{here}
.
\end{quote}

Glossarium is a simple, easily customizable typst glossary inspired by
\href{https://www.ctan.org/pkg/glossaries}{LaTeX glossaries package} .
You can see various examples showcasing the different features in the
\texttt{\ examples\ } folder.

\pandocbounded{\includegraphics[keepaspectratio]{https://github.com/typst/packages/raw/main/packages/preview/glossarium/0.5.1/.github/example.png}}

\subsection{Manual}\label{manual}

\subsection{Fast start}\label{fast-start}

\begin{Shaded}
\begin{Highlighting}[]
\NormalTok{\#import "@preview/glossarium:0.5.1": make{-}glossary, register{-}glossary, print{-}glossary, gls, glspl}
\NormalTok{\#show: make{-}glossary}
\NormalTok{\#let entry{-}list = (}
\NormalTok{  (}
\NormalTok{    key: "kuleuven",}
\NormalTok{    short: "KU Leuven",}
\NormalTok{    long: "Katholieke Universiteit Leuven",}
\NormalTok{    description: "A university in Belgium.",}
\NormalTok{  ),}
\NormalTok{  // Add more terms}
\NormalTok{)}
\NormalTok{\#register{-}glossary(entry{-}list)}
\NormalTok{// Your document body}
\NormalTok{\#print{-}glossary(}
\NormalTok{ entry{-}list}
\NormalTok{)}
\end{Highlighting}
\end{Shaded}

\subsubsection{Import and setup}\label{import-and-setup}

This manual assume you have a good enough understanding of typst markup
and scripting.

For Typst 0.6.0 or later import the package from the typst preview
repository:

\begin{Shaded}
\begin{Highlighting}[]
\NormalTok{\#import "@preview/glossarium:0.5.1": make{-}glossary, register{-}glossary, print{-}glossary, gls, glspl}
\end{Highlighting}
\end{Shaded}

For Typst before 0.6.0 or to use \textbf{glossarium} as a local module,
download the package files into your project folder and import
\texttt{\ glossarium.typ\ } :

\begin{Shaded}
\begin{Highlighting}[]
\NormalTok{\#import "glossarium.typ": make{-}glossary, register{-}glossary, print{-}glossary, gls, glspl}
\end{Highlighting}
\end{Shaded}

After importing the package and before making any calls to
\texttt{\ gls\ } , \texttt{\ print-glossary\ } or \texttt{\ glspl\ } ,
please \emph{\textbf{MAKE SURE}} you add this line

\begin{Shaded}
\begin{Highlighting}[]
\NormalTok{\#show: make{-}glossary}
\end{Highlighting}
\end{Shaded}

\begin{quote}
\emph{WHY DO WE NEED THAT ?} : In order to be able to create references
to the terms in your glossary using typst ref syntax \texttt{\ @key\ }
glossarium needs to setup some
\href{https://typst.app/docs/tutorial/advanced-styling/}{show rules}
before any references exist. This is due to the way typst works, there
is no workaround.

Therefore I recommend that you always put the \texttt{\ \#show:\ ...\ }
statement on the line just below the \texttt{\ \#import\ } statement.
\end{quote}

\subsubsection{Registering the glossary}\label{registering-the-glossary}

First we have to define the terms. A term is a
\href{https://typst.app/docs/reference/types/dictionary/}{dictionary} as
follows:

\begin{longtable}[]{@{}llll@{}}
\toprule\noalign{}
Key & Type & Required/Optional & Description \\
\midrule\noalign{}
\endhead
\bottomrule\noalign{}
\endlastfoot
\texttt{\ key\ } & string & required & Case-sensitive, unique identifier
used to reference the term. \\
\texttt{\ short\ } & string & semi-optional & The short form of the term
replacing the term citation. \\
\texttt{\ long\ } & string or content & semi-optional & The long form of
the term, displayed in the glossary and on the first citation of the
term. \\
\texttt{\ description\ } & string or content & optional & The
description of the term. \\
\texttt{\ plural\ } & string or content & optional & The pluralized
short form of the term. \\
\texttt{\ longplural\ } & string or content & optional & The pluralized
long form of the term. \\
\texttt{\ group\ } & string & optional & Case-sensitive group the term
belongs to. The terms are displayed by groups in the glossary. \\
\end{longtable}

\begin{Shaded}
\begin{Highlighting}[]
\NormalTok{\#let entry{-}list = (}
\NormalTok{  // minimal term}
\NormalTok{  (}
\NormalTok{    key: "kuleuven",}
\NormalTok{    short: "KU Leuven"}
\NormalTok{  ),}
\NormalTok{  // a term with a long form and a group}
\NormalTok{  (}
\NormalTok{    key: "unamur",}
\NormalTok{    short: "UNamur",}
\NormalTok{    long: "Namur University",}
\NormalTok{    group: "Universities"}
\NormalTok{  ),}
\NormalTok{  // a term with a markup description}
\NormalTok{  (}
\NormalTok{    key: "oidc",}
\NormalTok{    short: "OIDC",}
\NormalTok{    long: "OpenID Connect",}
\NormalTok{    description: [}
\NormalTok{      OpenID is an open standard and decentralized authentication protocol promoted by the non{-}profit}
\NormalTok{      \#link("https://en.wikipedia.org/wiki/OpenID\#OpenID\_Foundation")[OpenID Foundation].}
\NormalTok{    ],}
\NormalTok{    group: "Acronyms",}
\NormalTok{  ),}
\NormalTok{  // a term with a short plural}
\NormalTok{  (}
\NormalTok{    key: "potato",}
\NormalTok{    short: "potato",}
\NormalTok{    // "plural" will be used when "short" should be pluralized}
\NormalTok{    plural: "potatoes",}
\NormalTok{    description: [\#lorem(10)],}
\NormalTok{  ),}
\NormalTok{  // a term with a long plural}
\NormalTok{  (}
\NormalTok{    key: "dm",}
\NormalTok{    short: "DM",}
\NormalTok{    long: "diagonal matrix",}
\NormalTok{    // "longplural" will be used when "long" should be pluralized}
\NormalTok{    longplural: "diagonal matrices",}
\NormalTok{    description: "Probably some math stuff idk",}
\NormalTok{  ),}
\NormalTok{)}
\end{Highlighting}
\end{Shaded}

Then the terms are passed as a list to \texttt{\ register-glossary\ }

\begin{Shaded}
\begin{Highlighting}[]
\NormalTok{\#register{-}glossary(entry{-}list)}
\end{Highlighting}
\end{Shaded}

\subsubsection{Printing the glossary}\label{printing-the-glossary}

Now, you can display the glossary using the \texttt{\ print-glossary\ }
function.

\begin{Shaded}
\begin{Highlighting}[]
\NormalTok{\#print{-}glossary(entry{-}list)}
\end{Highlighting}
\end{Shaded}

By default, the terms that are not referenced in the document are not
shown in the glossary, you can force their appearance by setting the
\texttt{\ show-all\ } argument to true.

You can also disable the back-references by setting the parameter
\texttt{\ disable-back-references\ } to \texttt{\ true\ } .

By default, group breaks use \texttt{\ linebreaks\ } . This behaviour
can be changed by setting the \texttt{\ user-group-break\ } parameter to
\texttt{\ pagebreak()\ } , or \texttt{\ colbreak()\ } , or any other
function that returns the \texttt{\ content\ } you want.

You can call this function from anywhere in your document.

\subsubsection{Referencing terms.}\label{referencing-terms.}

Referencing terms is done using the key of the terms using the
\texttt{\ gls\ } function or the reference syntax.

\begin{Shaded}
\begin{Highlighting}[]
\NormalTok{// referencing the OIDC term using gls}
\NormalTok{\#gls("oidc")}
\NormalTok{// displaying the long form forcibly}
\NormalTok{\#gls("oidc", long: true)}

\NormalTok{// referencing the OIDC term using the reference syntax}
\NormalTok{@oidc}
\end{Highlighting}
\end{Shaded}

\paragraph{Handling plurals}\label{handling-plurals}

You can use the \texttt{\ glspl\ } function and the references
supplements to pluralize terms. The \texttt{\ plural\ } key will be used
when \texttt{\ short\ } should be pluralized and \texttt{\ longplural\ }
will be used when \texttt{\ long\ } should be pluralized. If the
\texttt{\ plural\ } key is missing then glossarium will add an ‘s’
at the end of the short form as a fallback.

\begin{Shaded}
\begin{Highlighting}[]
\NormalTok{\#glspl("potato")}
\end{Highlighting}
\end{Shaded}

Please look at the examples regarding plurals.

\paragraph{Overriding the text shown}\label{overriding-the-text-shown}

You can also override the text displayed by setting the
\texttt{\ display\ } argument.

\begin{Shaded}
\begin{Highlighting}[]
\NormalTok{\#gls("oidc", display: "whatever you want")}
\end{Highlighting}
\end{Shaded}

\subsection{Final tips}\label{final-tips}

I recommend setting a show rule for the links to that your readers
understand that they can click on the references to go to the term in
the glossary.

\begin{Shaded}
\begin{Highlighting}[]
\NormalTok{\#show link: set text(fill: blue.darken(60\%))}
\NormalTok{// links are now blue !}
\end{Highlighting}
\end{Shaded}

\subsubsection{How to add}\label{how-to-add}

Copy this into your project and use the import as
\texttt{\ glossarium\ }

\begin{verbatim}
#import "@preview/glossarium:0.5.1"
\end{verbatim}

\includesvg[width=0.16667in,height=0.16667in]{/assets/icons/16-copy.svg}

Check the docs for
\href{https://typst.app/docs/reference/scripting/\#packages}{more
information on how to import packages} .

\subsubsection{About}\label{about}

\begin{description}
\tightlist
\item[Author s :]
slashformotion \& Dherse
\item[License:]
MIT
\item[Current version:]
0.5.1
\item[Last updated:]
October 28, 2024
\item[First released:]
July 31, 2023
\item[Archive size:]
10.5 kB
\href{https://packages.typst.org/preview/glossarium-0.5.1.tar.gz}{\pandocbounded{\includesvg[keepaspectratio]{/assets/icons/16-download.svg}}}
\item[Repository:]
\href{https://github.com/typst-community/glossarium}{GitHub}
\end{description}

\subsubsection{Where to report issues?}\label{where-to-report-issues}

This package is a project of slashformotion and Dherse . Report issues
on \href{https://github.com/typst-community/glossarium}{their
repository} . You can also try to ask for help with this package on the
\href{https://forum.typst.app}{Forum} .

Please report this package to the Typst team using the
\href{https://typst.app/contact}{contact form} if you believe it is a
safety hazard or infringes upon your rights.

\phantomsection\label{versions}
\subsubsection{Version history}\label{version-history}

\begin{longtable}[]{@{}ll@{}}
\toprule\noalign{}
Version & Release Date \\
\midrule\noalign{}
\endhead
\bottomrule\noalign{}
\endlastfoot
0.5.1 & October 28, 2024 \\
\href{https://typst.app/universe/package/glossarium/0.5.0/}{0.5.0} &
October 14, 2024 \\
\href{https://typst.app/universe/package/glossarium/0.4.2/}{0.4.2} &
October 7, 2024 \\
\href{https://typst.app/universe/package/glossarium/0.4.1/}{0.4.1} & May
29, 2024 \\
\href{https://typst.app/universe/package/glossarium/0.4.0/}{0.4.0} &
April 29, 2024 \\
\href{https://typst.app/universe/package/glossarium/0.3.0/}{0.3.0} &
April 8, 2024 \\
\href{https://typst.app/universe/package/glossarium/0.2.6/}{0.2.6} &
January 29, 2024 \\
\href{https://typst.app/universe/package/glossarium/0.2.5/}{0.2.5} &
December 3, 2023 \\
\href{https://typst.app/universe/package/glossarium/0.2.4/}{0.2.4} &
November 16, 2023 \\
\href{https://typst.app/universe/package/glossarium/0.2.3/}{0.2.3} &
October 30, 2023 \\
\href{https://typst.app/universe/package/glossarium/0.2.2/}{0.2.2} &
September 16, 2023 \\
\href{https://typst.app/universe/package/glossarium/0.2.1/}{0.2.1} &
September 3, 2023 \\
\href{https://typst.app/universe/package/glossarium/0.2.0/}{0.2.0} &
August 19, 2023 \\
\href{https://typst.app/universe/package/glossarium/0.1.0/}{0.1.0} &
July 31, 2023 \\
\end{longtable}

Typst GmbH did not create this package and cannot guarantee correct
functionality of this package or compatibility with any version of the
Typst compiler or app.
