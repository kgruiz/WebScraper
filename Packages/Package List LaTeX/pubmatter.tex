\title{typst.app/universe/package/pubmatter}

\phantomsection\label{banner}
\section{pubmatter}\label{pubmatter}

{ 0.1.0 }

Parse, normalize and show publication frontmatter, including authors and
affiliations

\phantomsection\label{readme}
\emph{Beautiful scientific documents with structured metadata for
publishers}

\href{https://github.com/curvenote/pubmatter/blob/main/docs.pdf}{\pandocbounded{\includesvg[keepaspectratio]{https://img.shields.io/badge/typst-docs-orange.svg}}}
\href{https://github.com/curvenote/pubmatter/blob/main/LICENSE}{\pandocbounded{\includesvg[keepaspectratio]{https://img.shields.io/badge/license-MIT-blue.svg}}}

Pubmatter is a typst library for parsing, normalizing and showing
scientific publication frontmatter.

Utilities for loading, normalizing and working with authors,
affiliations, abstracts, keywords and other frontmatter information
common in scientific publications. Our goal is to introduce standardized
ways of working with this content to expose metadata to scientific
publishers who are interested in using typst in a standardized way. The
specification for this \texttt{\ pubmatter\ } is based on
\href{https://mystmd.org/}{MyST Markdown} and
\href{https://quarto.org/}{Quarto} , and can load their YAML files
directly.

\subsection{Examples}\label{examples}

Pubmatter was used to create these documents, for loading the authors in
a standardized way and creting the common elements (authors,
affiliations, ORCIDs, DOIs, Open Access Links, copyright statements,
etc.)

\pandocbounded{\includegraphics[keepaspectratio]{https://raw.githubusercontent.com/curvenote/pubmatter/main/images/lapreprint.png?raw=true}}

\pandocbounded{\includegraphics[keepaspectratio]{https://raw.githubusercontent.com/curvenote/pubmatter/main/images/scipy.png?raw=true}}

\pandocbounded{\includegraphics[keepaspectratio]{https://raw.githubusercontent.com/curvenote/pubmatter/main/images/agrogeo.png?raw=true}}

\subsection{Documentation}\label{documentation}

The full documentation can be found in
\href{https://github.com/curvenote/pubmatter/blob/main/docs.pdf}{docs.pdf}
. To use \texttt{\ pubmatter\ } import it:

\begin{Shaded}
\begin{Highlighting}[]
\NormalTok{\#import "@preview/pubmatter:0.1.0"}
\end{Highlighting}
\end{Shaded}

The docs also use \texttt{\ pubmatter\ } , in a simplified way, you can
see the
\href{https://github.com/curvenote/pubmatter/blob/main/docs.typ}{docs.typ}
to see a simple example of using various components to create a new
document. Here is a preview of the docs:

\href{https://github.com/curvenote/pubmatter/blob/main/docs.pdf}{\pandocbounded{\includegraphics[keepaspectratio]{https://raw.githubusercontent.com/curvenote/pubmatter/main/images/pubmatter.png?raw=true}}}

\subsubsection{Loading Frontmatter}\label{loading-frontmatter}

The frontmatter can contain all information for an article, including
title, authors, affiliations, abstracts and keywords. These are then
normalized into a standardized format that can be used with a number of
\texttt{\ show\ } functions like \texttt{\ show-authors\ } . For
example, we might have a YAML file that looks like this:

\begin{Shaded}
\begin{Highlighting}[]
\FunctionTok{author}\KeywordTok{:}\AttributeTok{ Rowan Cockett}
\FunctionTok{date}\KeywordTok{:}\AttributeTok{ 2024/01/26}
\end{Highlighting}
\end{Shaded}

You can load that file with \texttt{\ yaml\ } , and pass it to the
\texttt{\ load\ } function:

\begin{Shaded}
\begin{Highlighting}[]
\NormalTok{\#let fm = pubmatter.load(yaml("pubmatter.yml"))}
\end{Highlighting}
\end{Shaded}

This will give you a normalized data-structure that can be used with the
\texttt{\ show\ } functions for showing various parts of a document.

You can also use a \texttt{\ dictionary\ } directly:

\begin{Shaded}
\begin{Highlighting}[]
\NormalTok{\#let fm = pubmatter.load((}
\NormalTok{  author: (}
\NormalTok{    (}
\NormalTok{      name: "Rowan Cockett",}
\NormalTok{      email: "rowan@curvenote.com",}
\NormalTok{      orcid: "0000{-}0002{-}7859{-}8394",}
\NormalTok{      affiliations: "Curvenote Inc.",}
\NormalTok{    ),}
\NormalTok{  ),}
\NormalTok{  date: datetime(year: 2024, month: 01, day: 26),}
\NormalTok{  doi: "10.1190/tle35080703.1",}
\NormalTok{))}
\NormalTok{\#pubmatter.show{-}author{-}block(fm)}
\end{Highlighting}
\end{Shaded}

\pandocbounded{\includegraphics[keepaspectratio]{https://raw.githubusercontent.com/curvenote/pubmatter/main/images/author-block.png?raw=true}}

\subsubsection{Theming}\label{theming}

The theme including color and font choice can be set using the
\texttt{\ THEME\ } state. For example, this document has the following
theme set:

\begin{Shaded}
\begin{Highlighting}[]
\NormalTok{\#let theme = (color: red.darken(20\%), font: "Noto Sans")}
\NormalTok{\#set page(header: pubmatter.show{-}page{-}header(theme: theme, fm), footer: pubmatter.show{-}page{-}footer(fm))}
\NormalTok{\#state("THEME").update(theme)}
\end{Highlighting}
\end{Shaded}

Note that for the \texttt{\ header\ } the theme must be passed in
directly. This will hopefully become easier in the future, however,
there is a current bug that removes the page header/footer if you set
this above the \texttt{\ set\ page\ } . See
\href{https://github.com/typst/packages/raw/main/packages/preview/pubmatter/0.1.0/\#2987}{https://github.com/typst/typst/issues/2987}
.

The \texttt{\ font\ } option only corresponds to the frontmatter content
(abstracts, title, header/footer etc.) allowing the body of your
document to have a different font choice.

\subsubsection{Normalized Frontmatter
Object}\label{normalized-frontmatter-object}

The frontmatter object has the following normalized structure:

\begin{Shaded}
\begin{Highlighting}[]
\FunctionTok{title}\KeywordTok{:}\AttributeTok{ content}
\FunctionTok{subtitle}\KeywordTok{:}\AttributeTok{ content}
\FunctionTok{short{-}title}\KeywordTok{:}\AttributeTok{ string}\CommentTok{ \# alias: running{-}title, running{-}head}
\CommentTok{\# Authors Array}
\CommentTok{\# simple string provided for author is turned into ((name: string),)}
\FunctionTok{authors}\KeywordTok{:}\CommentTok{ \# alias: author}
\AttributeTok{  }\KeywordTok{{-}}\AttributeTok{ }\FunctionTok{name}\KeywordTok{:}\AttributeTok{ string}
\AttributeTok{    }\FunctionTok{url}\KeywordTok{:}\AttributeTok{ string}\CommentTok{ \# alias: website, homepage}
\AttributeTok{    }\FunctionTok{email}\KeywordTok{:}\AttributeTok{ string}
\AttributeTok{    }\FunctionTok{phone}\KeywordTok{:}\AttributeTok{ string}
\AttributeTok{    }\FunctionTok{fax}\KeywordTok{:}\AttributeTok{ string}
\AttributeTok{    }\FunctionTok{orcid}\KeywordTok{:}\AttributeTok{ string}\CommentTok{ \# alias: ORCID}
\AttributeTok{    }\FunctionTok{note}\KeywordTok{:}\AttributeTok{ string}
\AttributeTok{    }\FunctionTok{corresponding}\KeywordTok{:}\AttributeTok{ boolean}\CommentTok{ \# default: \textasciigrave{}true\textasciigrave{} when email set}
\AttributeTok{    }\FunctionTok{equal{-}contributor}\KeywordTok{:}\AttributeTok{ boolean}\CommentTok{ \# alias: equalContributor, equal\_contributor}
\AttributeTok{    }\FunctionTok{deceased}\KeywordTok{:}\AttributeTok{ boolean}
\AttributeTok{    }\FunctionTok{roles}\KeywordTok{:}\AttributeTok{ string[]}\CommentTok{ \# must be a contributor role}
\AttributeTok{    }\FunctionTok{affiliations}\KeywordTok{:}\CommentTok{ \# alias: affiliation}
\AttributeTok{      }\KeywordTok{{-}}\AttributeTok{ }\FunctionTok{id}\KeywordTok{:}\AttributeTok{ string}
\AttributeTok{        }\FunctionTok{index}\KeywordTok{:}\AttributeTok{ number}
\CommentTok{\# Affiliations Array}
\FunctionTok{affiliations}\KeywordTok{:}\CommentTok{ \# alias: affiliation}
\AttributeTok{  }\KeywordTok{{-}}\AttributeTok{ string}\CommentTok{ \# simple string is turned into (name: string)}
\AttributeTok{  }\KeywordTok{{-}}\AttributeTok{ }\FunctionTok{id}\KeywordTok{:}\AttributeTok{ string}
\AttributeTok{    }\FunctionTok{index}\KeywordTok{:}\AttributeTok{ number}
\AttributeTok{    }\FunctionTok{name}\KeywordTok{:}\AttributeTok{ string}
\AttributeTok{    }\FunctionTok{institution}\KeywordTok{:}\AttributeTok{ string}\CommentTok{ \# use either name or institution}
\CommentTok{\# Other publication metadata}
\FunctionTok{open{-}access}\KeywordTok{:}\AttributeTok{ boolean}
\FunctionTok{license}\KeywordTok{:}\CommentTok{ \# Can be set with a SPDX ID for creative commons}
\AttributeTok{  }\FunctionTok{id}\KeywordTok{:}\AttributeTok{ string}
\AttributeTok{  }\FunctionTok{url}\KeywordTok{:}\AttributeTok{ string}
\AttributeTok{  }\FunctionTok{name}\KeywordTok{:}\AttributeTok{ string}
\FunctionTok{doi}\KeywordTok{:}\AttributeTok{ string}\CommentTok{ \# must be only the ID, not the full URL}
\FunctionTok{date}\KeywordTok{:}\AttributeTok{ datetime}\CommentTok{ \# validates from \textquotesingle{}YYYY{-}MM{-}DD\textquotesingle{} if a string}
\FunctionTok{citation}\KeywordTok{:}\AttributeTok{ content}
\CommentTok{\# Abstracts Array}
\CommentTok{\# content is turned into ((title: "Abstract", content: string),)}
\FunctionTok{abstracts}\KeywordTok{:}\CommentTok{ \# alias: abstract}
\AttributeTok{  }\KeywordTok{{-}}\AttributeTok{ }\FunctionTok{title}\KeywordTok{:}\AttributeTok{ content}
\AttributeTok{    }\FunctionTok{content}\KeywordTok{:}\AttributeTok{ content}
\end{Highlighting}
\end{Shaded}

Note that you will usually write the affiliations directly in line, in
the following example, we can see that the output is a normalized
affiliation object that is linked by the \texttt{\ id\ } of the
affiliation (just the name if it is a string!).

\begin{Shaded}
\begin{Highlighting}[]
\NormalTok{\#let fm = pubmatter.load((}
\NormalTok{  authors: (}
\NormalTok{    (}
\NormalTok{      name: "Rowan Cockett",}
\NormalTok{      affiliations: "Curvenote Inc.",}
\NormalTok{    ),}
\NormalTok{    (}
\NormalTok{      name: "Steve Purves",}
\NormalTok{      affiliations: ("Executable Books", "Curvenote Inc."),}
\NormalTok{    ),}
\NormalTok{  ),}
\NormalTok{))}
\NormalTok{\#raw(lang:"yaml", yaml.encode(fm))}
\end{Highlighting}
\end{Shaded}

\pandocbounded{\includegraphics[keepaspectratio]{https://raw.githubusercontent.com/curvenote/pubmatter/main/images/normalized.png?raw=true}}

\subsubsection{Full List of Functions}\label{full-list-of-functions}

\begin{itemize}
\tightlist
\item
  \texttt{\ load()\ } - Load a raw frontmatter object
\item
  \texttt{\ doi-link()\ } - Create a DOI link
\item
  \texttt{\ email-link()\ } - Create a mailto link with an email icon
\item
  \texttt{\ github-link()\ } - Create a link to a GitHub profile with
  the GitHub icon
\item
  \texttt{\ open-access-link()\ } - Create a link to Wikipedia with an
  OpenAccess icon
\item
  \texttt{\ orcid-link()\ } - Create a ORCID link with an ORCID logo
\item
  \texttt{\ show-abstract-block()\ } - Show abstract-block including all
  abstracts and keywords
\item
  \texttt{\ show-abstracts()\ } - Show all abstracts (e.g. abstract,
  plain language summary)
\item
  \texttt{\ show-affiliations()\ } - Show affiliations
\item
  \texttt{\ show-author-block()\ } - Show author block, including
  author, icon links (e.g. ORCID, email, etc.) and affiliations
\item
  \texttt{\ show-authors()\ } - Show authors
\item
  \texttt{\ show-citation()\ } - Create a short citation in APA format,
  e.g. Cockett \emph{et al.} , 2023
\item
  \texttt{\ show-copyright()\ } - Show copyright statement based on
  license
\item
  \texttt{\ show-keywords()\ } - Show keywords as a list
\item
  \texttt{\ show-license-badge()\ } - Show the license badges
\item
  \texttt{\ show-page-footer()\ } - Show the venue, date and page
  numbers
\item
  \texttt{\ show-page-header()\ } - Show an open-access badge and the
  DOI and then the running-title and citation
\item
  \texttt{\ show-spaced-content()\ }
\item
  \texttt{\ show-title()\ } - Show title and subtitle
\item
  \texttt{\ show-title-block()\ } - Show title, authors and affiliations
\end{itemize}

\subsection{Contributing}\label{contributing}

To help with standardization of metadata or improve the show-functions
please contribute to this package:\\
\url{https://github.com/curvenote/pubmatter}

\subsubsection{How to add}\label{how-to-add}

Copy this into your project and use the import as \texttt{\ pubmatter\ }

\begin{verbatim}
#import "@preview/pubmatter:0.1.0"
\end{verbatim}

\includesvg[width=0.16667in,height=0.16667in]{/assets/icons/16-copy.svg}

Check the docs for
\href{https://typst.app/docs/reference/scripting/\#packages}{more
information on how to import packages} .

\subsubsection{About}\label{about}

\begin{description}
\tightlist
\item[Author :]
rowanc1
\item[License:]
MIT
\item[Current version:]
0.1.0
\item[Last updated:]
February 10, 2024
\item[First released:]
February 10, 2024
\item[Archive size:]
9.84 kB
\href{https://packages.typst.org/preview/pubmatter-0.1.0.tar.gz}{\pandocbounded{\includesvg[keepaspectratio]{/assets/icons/16-download.svg}}}
\item[Repository:]
\href{https://github.com/curvenote/pubmatter}{GitHub}
\end{description}

\subsubsection{Where to report issues?}\label{where-to-report-issues}

This package is a project of rowanc1 . Report issues on
\href{https://github.com/curvenote/pubmatter}{their repository} . You
can also try to ask for help with this package on the
\href{https://forum.typst.app}{Forum} .

Please report this package to the Typst team using the
\href{https://typst.app/contact}{contact form} if you believe it is a
safety hazard or infringes upon your rights.

\phantomsection\label{versions}
\subsubsection{Version history}\label{version-history}

\begin{longtable}[]{@{}ll@{}}
\toprule\noalign{}
Version & Release Date \\
\midrule\noalign{}
\endhead
\bottomrule\noalign{}
\endlastfoot
0.1.0 & February 10, 2024 \\
\end{longtable}

Typst GmbH did not create this package and cannot guarantee correct
functionality of this package or compatibility with any version of the
Typst compiler or app.
