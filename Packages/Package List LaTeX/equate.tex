\title{typst.app/universe/package/equate}

\phantomsection\label{banner}
\section{equate}\label{equate}

{ 0.2.1 }

Breakable equations with improved numbering.

{ } Featured Package

\phantomsection\label{readme}
A package for improved layout of equations and mathematical expressions.

When applied, this package will split up multi-line block equations into
multiple elements, so that each line can be assigned a separate number.
By default, the equation counter is incremented for each line, but this
behavior can be changed by setting the \texttt{\ sub-numbering\ }
argument to \texttt{\ true\ } . In this case, the equation counter will
only be incremented once for the entire block, and each line will be
assigned a sub-number like \texttt{\ 1a\ } , \texttt{\ 2.1\ } , or
similar, depending on the set equation numbering. You can also set the
\texttt{\ number-mode\ } argument to \texttt{\ "label"\ } to only number
labelled lines. If a label is only applied to the full equation, all
lines will be numbered.

This splitting also makes it possible to spread equations over page
boundaries while keeping alignment in place, which can be useful for
long derivations or proofs. This can be configured by the
\texttt{\ breakable\ } parameter of the \texttt{\ equate\ } function, or
by setting the \texttt{\ breakable\ } parameter of \texttt{\ block\ }
for equations via a show-set rule. Additionally, the alignment of the
equation number is improved, so that it always matches the baseline of
the equation.

If you want to create a “standard� equation with a single equation
number centered across all lines, you can attach the
\texttt{\ \textless{}equate:revoke\textgreater{}\ } label to the
equation. This will disable the effect of this package for the current
equation. This label can also be used in single lines of an equation to
disable numbering for that line only.

\subsection{Usage}\label{usage}

The package comes with a single \texttt{\ equate\ } function that is
supposed to be used as a template. It takes two optional arguments for
customization:

\begin{longtable}[]{@{}llll@{}}
\toprule\noalign{}
Argument & Type & Description & Default \\
\midrule\noalign{}
\endhead
\bottomrule\noalign{}
\endlastfoot
\texttt{\ breakable\ } & \texttt{\ boolean\ } , \texttt{\ auto\ } &
Whether to allow the equation to break across pages. &
\texttt{\ auto\ } \\
\texttt{\ sub-numbering\ } & \texttt{\ boolean\ } & Whether to assign
sub-numbers to each line of an equation. & \texttt{\ false\ } \\
\texttt{\ number-mode\ } & \texttt{\ "line"\ } , \texttt{\ "label"\ } &
Whether to number all lines or only those with a label. &
\texttt{\ "line"\ } \\
\end{longtable}

To reference a specific line of an equation, include the label at the
end of the line, like in the following example:

\begin{Shaded}
\begin{Highlighting}[]
\NormalTok{\#import "@preview/equate:0.2.1": equate}

\NormalTok{\#show: equate.with(breakable: true, sub{-}numbering: true)}
\NormalTok{\#set math.equation(numbering: "(1.1)")}

\NormalTok{The dot product of two vectors $arrow(a)$ and $arrow(b)$ can}
\NormalTok{be calculated as shown in @dot{-}product.}

\NormalTok{$}
\NormalTok{  angle.l a, b angle.r \&= arrow(a) dot arrow(b) \textbackslash{}}
\NormalTok{                       \&= a\_1 b\_1 + a\_2 b\_2 + ... a\_n b\_n \textbackslash{}}
\NormalTok{                       \&= sum\_(i=1)\^{}n a\_i b\_i. \#\textless{}sum\textgreater{}}
\NormalTok{$ \textless{}dot{-}product\textgreater{}}

\NormalTok{The sum notation in @sum is a useful way to express the dot}
\NormalTok{product of two vectors.}
\end{Highlighting}
\end{Shaded}

\pandocbounded{\includesvg[keepaspectratio]{https://github.com/typst/packages/raw/main/packages/preview/equate/0.2.1/assets/example-1.svg}}\\
\pandocbounded{\includesvg[keepaspectratio]{https://github.com/typst/packages/raw/main/packages/preview/equate/0.2.1/assets/example-2.svg}}

\subsubsection{Local Usage}\label{local-usage}

If you only want to use the package features on selected equations, you
can also apply the \texttt{\ equate\ } function directly to the
equation. This will override the default behavior for the current
equation only. Note, that this will require you to use the
\texttt{\ equate\ } function as a show rule for references, as shown in
the following example:

\begin{Shaded}
\begin{Highlighting}[]
\NormalTok{\#import "@preview/equate:0.2.1": equate}

\NormalTok{// Allow references to a line of the equation.}
\NormalTok{\#show ref: equate}

\NormalTok{\#set math.equation(numbering: "(1.1)", supplement: "Eq.")}

\NormalTok{\#equate($}
\NormalTok{  E \&= m c\^{}2 \#\textless{}short\textgreater{} \textbackslash{}}
\NormalTok{    \&= sqrt(p\^{}2 c\^{}2 + m\^{}2 c\^{}4) \#\textless{}long\textgreater{}}
\NormalTok{$)}

\NormalTok{While @short is the famous equation by Einstein, @long is a}
\NormalTok{more general form of the energy{-}momentum relation.}
\end{Highlighting}
\end{Shaded}

\pandocbounded{\includesvg[keepaspectratio]{https://github.com/typst/packages/raw/main/packages/preview/equate/0.2.1/assets/example-local.svg}}

As an alternative to the show rule, it is also possible to manually wrap
each reference in an \texttt{\ equate\ } function, though this is less
convenient and more prone to mistakes.

\subsubsection{How to add}\label{how-to-add}

Copy this into your project and use the import as \texttt{\ equate\ }

\begin{verbatim}
#import "@preview/equate:0.2.1"
\end{verbatim}

\includesvg[width=0.16667in,height=0.16667in]{/assets/icons/16-copy.svg}

Check the docs for
\href{https://typst.app/docs/reference/scripting/\#packages}{more
information on how to import packages} .

\subsubsection{About}\label{about}

\begin{description}
\tightlist
\item[Author :]
Eric Biedert
\item[License:]
MIT
\item[Current version:]
0.2.1
\item[Last updated:]
September 11, 2024
\item[First released:]
July 5, 2024
\item[Minimum Typst version:]
0.11.0
\item[Archive size:]
5.81 kB
\href{https://packages.typst.org/preview/equate-0.2.1.tar.gz}{\pandocbounded{\includesvg[keepaspectratio]{/assets/icons/16-download.svg}}}
\item[Repository:]
\href{https://github.com/EpicEricEE/typst-equate}{GitHub}
\item[Categor ies :]
\begin{itemize}
\tightlist
\item[]
\item
  \pandocbounded{\includesvg[keepaspectratio]{/assets/icons/16-layout.svg}}
  \href{https://typst.app/universe/search/?category=layout}{Layout}
\item
  \pandocbounded{\includesvg[keepaspectratio]{/assets/icons/16-list-unordered.svg}}
  \href{https://typst.app/universe/search/?category=model}{Model}
\end{itemize}
\end{description}

\subsubsection{Where to report issues?}\label{where-to-report-issues}

This package is a project of Eric Biedert . Report issues on
\href{https://github.com/EpicEricEE/typst-equate}{their repository} .
You can also try to ask for help with this package on the
\href{https://forum.typst.app}{Forum} .

Please report this package to the Typst team using the
\href{https://typst.app/contact}{contact form} if you believe it is a
safety hazard or infringes upon your rights.

\phantomsection\label{versions}
\subsubsection{Version history}\label{version-history}

\begin{longtable}[]{@{}ll@{}}
\toprule\noalign{}
Version & Release Date \\
\midrule\noalign{}
\endhead
\bottomrule\noalign{}
\endlastfoot
0.2.1 & September 11, 2024 \\
\href{https://typst.app/universe/package/equate/0.2.0/}{0.2.0} & July 5,
2024 \\
\href{https://typst.app/universe/package/equate/0.1.0/}{0.1.0} & July 5,
2024 \\
\end{longtable}

Typst GmbH did not create this package and cannot guarantee correct
functionality of this package or compatibility with any version of the
Typst compiler or app.
