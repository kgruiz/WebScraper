\title{typst.app/universe/package/acrotastic}

\phantomsection\label{banner}
\section{acrotastic}\label{acrotastic}

{ 0.1.1 }

Manage acronyms and their definitions in Typst.

\phantomsection\label{readme}
Manages all your acronyms for you.

Acrotastics main features are clickable abbreviations that auto-expand
on the first occurence, manual short and long forms, implicit or manual
plural form support, and customizable index printing.

\subsection{Quick Start}\label{quick-start}

\begin{verbatim}
#import "@preview/acrotastic:0.1.1": *

#init-acronyms((
  "WTP": ("Wonderful Typst Package","Wonderful Typst Packages"),
))

Acrotastic is a #acr("WTP")! This #acr("WTP") enables easy acronym manipulation.
\end{verbatim}

\subsection{Usage}\label{usage}

\subsubsection{Define acronyms}\label{define-acronyms}

First, define the acronyms in a dictionary, with the keys being the
acronyms and the values being arrays of their definitions. If there is
only a singular version of the definition, the array contains only one
value. If there are both singular and plural versions, define the
definition as an array where the first item is the singular definition
and the second item is the plural. Then, initialize Acrotastic by
passing the dictionay you just defined to the
\texttt{\ \#init-acronyms(...)\ } function.

Here is a example of the \texttt{\ acronyms.typ\ } file:

\begin{verbatim}
#import "@preview/acrotastic:0.1.1": *

#init-acronyms((
  "NN": ("Neural Network"),
  "OS": ("Operating System",),
  "BIOS": ("Basic Input/Output System", "Basic Input/Output Systems"),
))
\end{verbatim}

\subsubsection{Call Acrotastic
functions}\label{call-acrotastic-functions}

There is a large number of different functions to fit every use case.
You will find an overview of all functions and their descriptions in the
table below.

\begin{longtable}[]{@{}ll@{}}
\toprule\noalign{}
Function & Description \\
\midrule\noalign{}
\endhead
\bottomrule\noalign{}
\endlastfoot
\texttt{\ \#acr(...)\ } & On the first occurrence the long version of
the abbreviation and the abbreviation itself are displayed in brackets.
The next time only the abbreviation is displayed. \\
\texttt{\ \#acrpl(...)\ } & Same as \texttt{\ \#acr(...)\ } but the
plural will be diplayed. If no plural is defined, an ‘s’ is added to
the singular form. \\
\texttt{\ \#acrf(...)\ } & The acronym will be displayed as if it is the
first time. This means that it is again shown in the long form and the
abbreviation in brackets. \\
\texttt{\ \#acrfpl(...)\ } & Same as \texttt{\ \#acrf(...)\ } but the
plural will be displayed. If no plural is defined, an ‘s’ is added
to the singular form. \\
\texttt{\ \#acrs(...)\ } & Always displays the short form of the
acronym. \\
\texttt{\ \#acrspl(...)\ } & Same as \texttt{\ \#acrs(...)\ } but adds
an ‘s’ to the acronym for the plural form. \\
\texttt{\ \#acrl(...)\ } & Always displays the long form of the
acronym. \\
\texttt{\ \#acrlpl(...)\ } & Same as \texttt{\ \#acrl(...)\ } but the
plural will be displayed. If no plural is defined, an ‘s’ is added
to the singular form. \\
\texttt{\ \#reset-acronym(...)\ } & Resets a specific acronym. The
acronym will be expanded on the next use. \\
\texttt{\ reset-all-acronyms()\ } & Resets all acronyms. The acronyms
will be expanded on their next use. \\
\end{longtable}

You can alternatively use \texttt{\ \#acr(...)\ } ,
\texttt{\ \#acrf(...)\ } , \texttt{\ \#acrs(...)\ } and
\texttt{\ \#acrl(...)\ } with \texttt{\ plural:\ true\ } to display the
plural form.

\begin{verbatim}
#acr("BIOS", plural: true)
\end{verbatim}

To deactivate the link to the abbreviations directory (for whatever
reason), you can set \texttt{\ link:\ false\ } .

\begin{verbatim}
#acr("BIOS", link: false)
\end{verbatim}

\subsubsection{Print Abbreviations
directory}\label{print-abbreviations-directory}

You can also print an index of all acronyms used in the document with
the \texttt{\ \#print-index()\ } function. There are some parameters for
customization.

\begin{longtable}[]{@{}llll@{}}
\toprule\noalign{}
parameter & values & default & description \\
\midrule\noalign{}
\endhead
\bottomrule\noalign{}
\endlastfoot
title & string & “List of Abbreviations� & Heading of the acronym
index \\
level & number & 1 & Level of the heading \\
sorted & “up�, “down�, “keep� & “up� & “Up� sorts
alphabetically, “Down� sorts reversed alphabetically and “keep�
uses the order from initialization \\
delimiter & string & “:� & String to place after the acronym in the
list \\
acr-col-size & percentage & 20\% & Size of the acronym column in
percent \\
outlined & bool & false & Make the index section outlined \\
\end{longtable}

\subsection{Possible Errors}\label{possible-errors}

\begin{longtable}[]{@{}ll@{}}
\toprule\noalign{}
Error & Solution \\
\midrule\noalign{}
\endhead
\bottomrule\noalign{}
\endlastfoot
Acronym is not a key in the acronyms dictionary. & Make sure that the
acronym is defined in the dictionary passed to
\texttt{\ \#init-acronyms(dict)\ } \\
No definitions found for acronym. Make sure it is defined in the
dictionary passed to \#init-acronyms(dict) & The acronym is in the
dictionary, but has no correct definition. \\
Definitions should be arrays of one or two strings. Definition of
acronym is: & The acronym has a definition, but the definition doesn’t
have the right type. Make sure it’s an array of one or two strings. \\
\end{longtable}

Moreover you have to be careful when using states.

\begin{itemize}
\tightlist
\item
  For every acronym “ABC� that you define, the state named
  “acronym-state-ABC� is initialized and used. To avoid errors, do
  not try to use this state manually for other purposes. Similarly, the
  state named “acronyms� is reserved to Acrotastic. Please avoid
  using it.
\item
  The functions above are leveraging the state \texttt{\ display\ }
  function and only works if the return value is actually printed in the
  document. For more information on states, see the
  \href{https://typst.app/docs/reference/introspection/state/}{Typst
  documentation on states} .
\end{itemize}

\subsection{Contributing}\label{contributing}

If you notice any bug or want to contribute a new feature, please open
an issue or a pull request on the fork
\href{https://github.com/Julian702/typst-packages?tab=readme-ov-file}{Julian702/typst-packages}

\subsection{Acknowledgement}\label{acknowledgement}

Thanks to @Grisely who developed the
\href{https://typst.app/universe/package/acrostiche/}{acrostiche
package} which was the basis for acrotastic.

\subsubsection{How to add}\label{how-to-add}

Copy this into your project and use the import as
\texttt{\ acrotastic\ }

\begin{verbatim}
#import "@preview/acrotastic:0.1.1"
\end{verbatim}

\includesvg[width=0.16667in,height=0.16667in]{/assets/icons/16-copy.svg}

Check the docs for
\href{https://typst.app/docs/reference/scripting/\#packages}{more
information on how to import packages} .

\subsubsection{About}\label{about}

\begin{description}
\tightlist
\item[Author s :]
@Julian702 \& Gaetan Lepage @GaetanLepage
\item[License:]
MIT
\item[Current version:]
0.1.1
\item[Last updated:]
September 3, 2024
\item[First released:]
April 29, 2024
\item[Archive size:]
4.20 kB
\href{https://packages.typst.org/preview/acrotastic-0.1.1.tar.gz}{\pandocbounded{\includesvg[keepaspectratio]{/assets/icons/16-download.svg}}}
\item[Repository:]
\href{https://github.com/Julian702/typst-packages}{GitHub}
\item[Categor y :]
\begin{itemize}
\tightlist
\item[]
\item
  \pandocbounded{\includesvg[keepaspectratio]{/assets/icons/16-list-unordered.svg}}
  \href{https://typst.app/universe/search/?category=model}{Model}
\end{itemize}
\end{description}

\subsubsection{Where to report issues?}\label{where-to-report-issues}

This package is a project of @Julian702 and Gaetan Lepage @GaetanLepage
. Report issues on
\href{https://github.com/Julian702/typst-packages}{their repository} .
You can also try to ask for help with this package on the
\href{https://forum.typst.app}{Forum} .

Please report this package to the Typst team using the
\href{https://typst.app/contact}{contact form} if you believe it is a
safety hazard or infringes upon your rights.

\phantomsection\label{versions}
\subsubsection{Version history}\label{version-history}

\begin{longtable}[]{@{}ll@{}}
\toprule\noalign{}
Version & Release Date \\
\midrule\noalign{}
\endhead
\bottomrule\noalign{}
\endlastfoot
0.1.1 & September 3, 2024 \\
\href{https://typst.app/universe/package/acrotastic/0.1.0/}{0.1.0} &
April 29, 2024 \\
\end{longtable}

Typst GmbH did not create this package and cannot guarantee correct
functionality of this package or compatibility with any version of the
Typst compiler or app.
