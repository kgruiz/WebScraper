\title{typst.app/universe/package/songb}

\phantomsection\label{banner}
\section{songb}\label{songb}

{ 0.1.0 }

A songbook package, to display chords above the lyrics and show a
number-based index (similar to patacrep)

\phantomsection\label{readme}
Attempt at creating a songbook package to replace
\href{https://github.com/patacrep/patacrep}{patacrep} (which is based on
LaTeX + \href{https://songs.sourceforge.net/}{Songs} ).

\subsection{Quickstart}\label{quickstart}

First, create a \texttt{\ main.typ\ } file, like the following:

\begin{Shaded}
\begin{Highlighting}[]
\NormalTok{\#set page(paper: "a6",margin: (inside: 14mm, outside: 6mm, y: 10mm))}

\NormalTok{\#import "@preview/songb:0.1.0": autobreak, index{-}by{-}letter}

\NormalTok{// helper function, to include you own songs (feel free to customize)}
\NormalTok{\#let song(path) = \{}
\NormalTok{    // WARNING: autobreak is currently broken (does not converge)}
\NormalTok{    // see https://github.com/typst/typst/discussions/4530}
\NormalTok{    autobreak(include path)}
\NormalTok{    v({-}1.19em)}
\NormalTok{\}}

\NormalTok{// indexes (put them wherever you want, or comment them out)}
\NormalTok{= Song Index}
\NormalTok{\#index{-}by{-}letter(\textless{}song\textgreater{})}

\NormalTok{= Singer Index}
\NormalTok{\#index{-}by{-}letter(\textless{}singer\textgreater{})}

\NormalTok{\#pagebreak()}

\NormalTok{// include all you songs, in the right order}
\NormalTok{\#song("./songs/first\_song.typ")}

\NormalTok{\#song("./songs/other\_song.typ")}

\NormalTok{// ...}
\end{Highlighting}
\end{Shaded}

Then, create your song files, like \texttt{\ songs/first\_song.typ\ } :

\begin{Shaded}
\begin{Highlighting}[]
\NormalTok{\#import "@preview/songb:0.1.0": song, chorus, verse, chord}

\NormalTok{\#show: doc =\textgreater{} song(}
\NormalTok{  title: "First Song",}
\NormalTok{  singer: "Sing",}
\NormalTok{  doc,}
\NormalTok{)}

\NormalTok{\#chorus[}
\NormalTok{  \#chord[Am]First line,\#chord[G][ ]of the chorus\textbackslash{}}
\NormalTok{  \#chord[Am]Second line,\#chord[G][ ]of the chorus.}
\NormalTok{]}


\NormalTok{\#verse[}
\NormalTok{  \#chord[Em]First verse\textbackslash{}}
\NormalTok{  With multiple\textbackslash{}}
\NormalTok{  \#chord[C]Lines}
\NormalTok{]}

\NormalTok{If there is \#chord[D][a] bridge\textbackslash{}}
\NormalTok{you can write it directly}
\end{Highlighting}
\end{Shaded}

\subsection{Writing a song}\label{writing-a-song}

\subsubsection{song}\label{song}

\begin{Shaded}
\begin{Highlighting}[]
\NormalTok{\#let song(}
\NormalTok{  title: none,}
\NormalTok{  title{-}index: none,}
\NormalTok{  singer: none,}
\NormalTok{  singer{-}index: none,}
\NormalTok{  references: (),}
\NormalTok{  line{-}color: rgb(0xd0, 0xd0, 0xd0),}
\NormalTok{  header{-}display: (number, title, singer) =\textgreater{} (...),}
\NormalTok{  doc}
\NormalTok{)}
\end{Highlighting}
\end{Shaded}

\subsubsection{chord}\label{chord}

\begin{Shaded}
\begin{Highlighting}[]
\NormalTok{// first argument: chord name}
\NormalTok{// optional second argument: text below the chord (useful for whitespace for instance)}
\NormalTok{\#let chord(..content)}
\end{Highlighting}
\end{Shaded}

\subsubsection{verse}\label{verse}

\begin{Shaded}
\begin{Highlighting}[]
\NormalTok{\#let verse(body)}
\end{Highlighting}
\end{Shaded}

\subsubsection{chorus}\label{chorus}

\begin{Shaded}
\begin{Highlighting}[]
\NormalTok{\#let chorus(body)}
\end{Highlighting}
\end{Shaded}

\subsection{Organizing songs}\label{organizing-songs}

\subsubsection{autobreak}\label{autobreak}

\begin{quote}
{[}!WARNING{]} Currently broken (lack of convergence for bigger
documents) See \url{https://github.com/typst/typst/discussions/4530}
\end{quote}

This function aims at putting the content on a single page (or on facing
pages), by introducing pagebreaks when needed.

\begin{Shaded}
\begin{Highlighting}[]
\NormalTok{\#let autobreak(content)}
\end{Highlighting}
\end{Shaded}

\subsubsection{index-by-letter}\label{index-by-letter}

\begin{Shaded}
\begin{Highlighting}[]
\NormalTok{\#let index{-}by{-}letter(label, letter{-}highlight: (letter) =\textgreater{} (...))}
\end{Highlighting}
\end{Shaded}

label: \texttt{\ \textless{}song\textgreater{}\ } or
\texttt{\ \textless{}singer\textgreater{}\ } are provided by the
\texttt{\ song\ } function.

\subsubsection{How to add}\label{how-to-add}

Copy this into your project and use the import as \texttt{\ songb\ }

\begin{verbatim}
#import "@preview/songb:0.1.0"
\end{verbatim}

\includesvg[width=0.16667in,height=0.16667in]{/assets/icons/16-copy.svg}

Check the docs for
\href{https://typst.app/docs/reference/scripting/\#packages}{more
information on how to import packages} .

\subsubsection{About}\label{about}

\begin{description}
\tightlist
\item[Author :]
\href{mailto:git@olivier.pfad.fr}{Oliverpool}
\item[License:]
EUPL-1.2+
\item[Current version:]
0.1.0
\item[Last updated:]
July 25, 2024
\item[First released:]
July 25, 2024
\item[Archive size:]
12.7 kB
\href{https://packages.typst.org/preview/songb-0.1.0.tar.gz}{\pandocbounded{\includesvg[keepaspectratio]{/assets/icons/16-download.svg}}}
\item[Repository:]
\href{https://codeberg.org/pfad.fr/typst-songbook}{Codeberg}
\item[Discipline :]
\begin{itemize}
\tightlist
\item[]
\item
  \href{https://typst.app/universe/search/?discipline=music}{Music}
\end{itemize}
\end{description}

\subsubsection{Where to report issues?}\label{where-to-report-issues}

This package is a project of Oliverpool . Report issues on
\href{https://codeberg.org/pfad.fr/typst-songbook}{their repository} .
You can also try to ask for help with this package on the
\href{https://forum.typst.app}{Forum} .

Please report this package to the Typst team using the
\href{https://typst.app/contact}{contact form} if you believe it is a
safety hazard or infringes upon your rights.

\phantomsection\label{versions}
\subsubsection{Version history}\label{version-history}

\begin{longtable}[]{@{}ll@{}}
\toprule\noalign{}
Version & Release Date \\
\midrule\noalign{}
\endhead
\bottomrule\noalign{}
\endlastfoot
0.1.0 & July 25, 2024 \\
\end{longtable}

Typst GmbH did not create this package and cannot guarantee correct
functionality of this package or compatibility with any version of the
Typst compiler or app.
