\title{typst.app/universe/package/derive-it}

\phantomsection\label{banner}
\section{derive-it}\label{derive-it}

{ 0.1.1 }

Simple functions for creating fitch-style natural deduction proofs and
derivations.

\phantomsection\label{readme}
A Typst package to create Fitch-style natural deductions.

\pandocbounded{\includegraphics[keepaspectratio]{https://github.com/typst/packages/raw/main/packages/preview/derive-it/0.1.1/examples/example.png}}

\subsection{Usage}\label{usage}

This package provides two functions:

\texttt{\ ded-nat\ } is a function that expects 2 parameters:

\begin{itemize}
\tightlist
\item
  \texttt{\ stcolor\ } : the stroke color of the indentation guides. The
  default is \texttt{\ black\ } .
\item
  \texttt{\ arr\ } : an array with the shape, it can be provided in two
  shapes.

  \begin{itemize}
  \tightlist
  \item
    4 items: (dependency: text content, indentation: integer starting
    from 0, formula: text content, rule: text content).
  \item
    3 items: the same as above, but without the dependency.
  \end{itemize}
\end{itemize}

\texttt{\ ded-nat-boxed\ } is a function that expects 4 parameters, and
returns the deduction in a \texttt{\ box\ } :

\begin{itemize}
\tightlist
\item
  \texttt{\ stcolor\ } : the stroke color of the indentation guides. The
  default is \texttt{\ black\ } .
\item
  \texttt{\ premises-and-conclusion\ } : bool, whether to automatically
  insert or not the premises and conclusion of the derivation above the
  lines. The default is \texttt{\ true\ } .
\item
  \texttt{\ premise-rule-text\ } : text content, used for finding the
  premises’ formulas when \texttt{\ premises-and-conclusion\ } is set
  to \texttt{\ true\ } . The default is \texttt{\ "PR"\ } .
\item
  \texttt{\ arr\ } : an array with the shape, it can be provided in two
  shapes.

  \begin{itemize}
  \tightlist
  \item
    4 items: (dependency: text content, indentation: integer starting
    from 0, formula: text content, rule: text content).
  \item
    3 items: the same as above, but without the dependency.
  \end{itemize}
\end{itemize}

\subsubsection{Example}\label{example}

\begin{Shaded}
\begin{Highlighting}[]
\NormalTok{\#import "@preview/derive{-}it:0.1.1": *}

\NormalTok{\#ded{-}nat(stcolor: black, arr:(}
\NormalTok{    ("1", 0, $forall x (P x) and forall x (Q x)$, "PR"),}
\NormalTok{    ("2", 0, $forall x (P x {-}\textgreater{} R x)$, "PR"),}
  
\NormalTok{    ("1", 0, $forall x (P x)$, "S 1"),}
\NormalTok{    ("1", 0, $P a$, "IU 3"),}
\NormalTok{    ("2", 0, $P a {-}\textgreater{} R a$, "IU 2"),}
\NormalTok{    ("1,2", 0, $R a$, "MP 4, 5"),}
  
\NormalTok{    ("1,2", 0, $forall x (R x)$, "GU 6"),}
\NormalTok{))}

\NormalTok{\#ded{-}nat{-}boxed(stcolor: black, premises{-}and{-}conclusion: false, arr: (}
\NormalTok{  ("1", 0, $forall x (S x b) and not forall y (P y {-}\textgreater{} Q b y)$, "PR"),}
\NormalTok{  ("2", 0, $forall x forall y (Q x y {-}\textgreater{} not Q y x)$, "PR"),}
\NormalTok{    ("3", 1, $not forall x (not P x) {-}\textgreater{} forall y (S y b {-}\textgreater{} Q b y)$, "Sup. RAA"),}
\NormalTok{    ("1", 1, $not forall y (P y {-}\textgreater{} Q b y)$, "S 1"),}
\NormalTok{    ("1", 1, $exists y not (P y {-}\textgreater{} Q b y)$, "EMC 4"),}
\NormalTok{      ("6", 2, $not (P a {-}\textgreater{} Q b a)$, "Sup. IE 5"),}
\NormalTok{        ("7", 3, $not (P a and not Q b a)$, "Sup. RAA"),}
\NormalTok{        ("7", 3, $not P a or not not Q  b a$, "DM 7"),}
\NormalTok{          ("9", 4, $not P a$, "Sup. PC"),}
\NormalTok{          ("9", 4, $not P a or Q b a$, "Disy. 9"),}
\NormalTok{        ("", 3, $not P a {-}\textgreater{} (not P a or Q b a)$, "PC 9{-}10"),}
\NormalTok{          ("12", 4, $not  not Q b a$, "Sup. PC"),}
\NormalTok{          ("12", 4, $Q b a$, "DN 12"),}
\NormalTok{          ("12", 4, $not P a or Q b a$, "Disy. 13"),}
\NormalTok{        ("", 3, $not not Q b a {-}\textgreater{} (not P a or Q b a)$, "PC 12{-}14"),}
\NormalTok{        ("7", 3, $not P a or Q b a$, "Dil. 8,11,15"),}
\NormalTok{        ("7", 3, $P a {-}\textgreater{} Q b a$, "IM 16"),}
\NormalTok{        ("6,7", 3, $(P a {-}\textgreater{} Q b a) and not (P a {-}\textgreater{} Q b a)$, "Conj. 6, 17"),}
\NormalTok{      ("6", 2, $P a and not Q b a$, "RAA 7{-}18"),}
\NormalTok{      ("6", 2, $P a$, "S 19"),}
\NormalTok{      ("6", 2, $exists x (P x)$, "GE 20"),}
\NormalTok{      ("6", 2, $not forall x (not P x)$, "EMC 21"),}
\NormalTok{      ("3,6", 2, $forall y (S y b {-}\textgreater{} Q b y)$, "MP 3, 22"),}
\NormalTok{      ("3,6", 2, $S a b {-}\textgreater{} Q b a$, "IU 23"),}
\NormalTok{      ("1", 2, $forall x (S x b)$, "S 1"),}
\NormalTok{      ("1", 2, $S a b$, "IU 25"),}
\NormalTok{      ("1,3,6", 2, $Q b a$, "MP 24, 25"),}
\NormalTok{      ("6", 2, $not Q b a$, "S 19"),}
\NormalTok{      ("1,3,6", 2, $Q b a or not exists y not (P y {-}\textgreater{} Q b y)$, "Disy. 27"),}
\NormalTok{      ("1,3,6", 2, $not exists y not (P y {-}\textgreater{} Q b y)$, "MTP 28, 29"),}
\NormalTok{    ("1,3", 1, $not exists y not (P y {-}\textgreater{} Q b y)$, "IE 5, 6, 30"),}
\NormalTok{    ("1,3", 1, $not exists y not (P y {-}\textgreater{} Q b y) and exists y not (P y {-}\textgreater{} Q b y)$, "Conj. 5, 31"),}

\NormalTok{  ("1", 0, $not (not forall x (not P x) {-}\textgreater{} forall y ( S y b {-}\textgreater{} Q b y))$, "RAA 3{-}32"),}
\NormalTok{))}
\end{Highlighting}
\end{Shaded}

In order to compile locally \texttt{\ examples/example.typ\ } the
command is:

\begin{Shaded}
\begin{Highlighting}[]
\ExtensionTok{typst}\NormalTok{ compile examples/example.typ }\AttributeTok{{-}root}\NormalTok{ .}
\end{Highlighting}
\end{Shaded}

\subsubsection{How to add}\label{how-to-add}

Copy this into your project and use the import as \texttt{\ derive-it\ }

\begin{verbatim}
#import "@preview/derive-it:0.1.1"
\end{verbatim}

\includesvg[width=0.16667in,height=0.16667in]{/assets/icons/16-copy.svg}

Check the docs for
\href{https://typst.app/docs/reference/scripting/\#packages}{more
information on how to import packages} .

\subsubsection{About}\label{about}

\begin{description}
\tightlist
\item[Author :]
\href{https://github.com/0rphee}{0rphee}
\item[License:]
MIT
\item[Current version:]
0.1.1
\item[Last updated:]
November 14, 2024
\item[First released:]
November 12, 2024
\item[Archive size:]
3.31 kB
\href{https://packages.typst.org/preview/derive-it-0.1.1.tar.gz}{\pandocbounded{\includesvg[keepaspectratio]{/assets/icons/16-download.svg}}}
\item[Repository:]
\href{https://github.com/0rphee/derive-it}{GitHub}
\item[Discipline s :]
\begin{itemize}
\tightlist
\item[]
\item
  \href{https://typst.app/universe/search/?discipline=mathematics}{Mathematics}
\item
  \href{https://typst.app/universe/search/?discipline=philosophy}{Philosophy}
\end{itemize}
\item[Categor ies :]
\begin{itemize}
\tightlist
\item[]
\item
  \pandocbounded{\includesvg[keepaspectratio]{/assets/icons/16-layout.svg}}
  \href{https://typst.app/universe/search/?category=layout}{Layout}
\item
  \pandocbounded{\includesvg[keepaspectratio]{/assets/icons/16-chart.svg}}
  \href{https://typst.app/universe/search/?category=visualization}{Visualization}
\end{itemize}
\end{description}

\subsubsection{Where to report issues?}\label{where-to-report-issues}

This package is a project of 0rphee . Report issues on
\href{https://github.com/0rphee/derive-it}{their repository} . You can
also try to ask for help with this package on the
\href{https://forum.typst.app}{Forum} .

Please report this package to the Typst team using the
\href{https://typst.app/contact}{contact form} if you believe it is a
safety hazard or infringes upon your rights.

\phantomsection\label{versions}
\subsubsection{Version history}\label{version-history}

\begin{longtable}[]{@{}ll@{}}
\toprule\noalign{}
Version & Release Date \\
\midrule\noalign{}
\endhead
\bottomrule\noalign{}
\endlastfoot
0.1.1 & November 14, 2024 \\
\href{https://typst.app/universe/package/derive-it/0.1.0/}{0.1.0} &
November 12, 2024 \\
\end{longtable}

Typst GmbH did not create this package and cannot guarantee correct
functionality of this package or compatibility with any version of the
Typst compiler or app.
