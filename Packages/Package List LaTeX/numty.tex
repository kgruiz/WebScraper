\title{typst.app/universe/package/numty}

\phantomsection\label{banner}
\section{numty}\label{numty}

{ 0.0.5 }

Numeric Typst

\phantomsection\label{readme}
\subsection{Numty}\label{numty-1}

\subsubsection{Numeric Typst}\label{numeric-typst}

A library for performing mathematical operations on n-dimensional
matrices, vectors/arrays, and numbers in Typst, with support for
broadcasting and handling NaN values. Numty’s broadcasting rules and
API are inspired by NumPy.

\begin{Shaded}
\begin{Highlighting}[]
\NormalTok{\#import "numty.typ" as nt}

\NormalTok{// Define vectors and matrices}
\NormalTok{\#let a = (1, 2, 3)}
\NormalTok{\#let b = 2}
\NormalTok{\#let m = ((1, 2), (1, 3))}

\NormalTok{// Element{-}wise operations with broadcasting}
\NormalTok{\#nt.mult(a, b)  // Multiply vector \textquotesingle{}a\textquotesingle{} by scalar \textquotesingle{}b\textquotesingle{}: (2, 4, 6)}
\NormalTok{\#nt.add(a, a)   // Add two vectors: (2, 4, 6)}
\NormalTok{\#nt.add(2, a)   // Add scalar \textquotesingle{}2\textquotesingle{} to each element of vector \textquotesingle{}a\textquotesingle{}: (3, 4, 5)}
\NormalTok{\#nt.add(m, 1)   // Add scalar \textquotesingle{}1\textquotesingle{} to each element of matrix \textquotesingle{}m\textquotesingle{}: ((2, 3), (2, 4))}

\NormalTok{// Dot product of vectors}
\NormalTok{\#nt.dot(a, a)   // Dot product of vector \textquotesingle{}a\textquotesingle{} with itself: 14}

\NormalTok{// Handling NaN cases in mathematical functions}
\NormalTok{\#calc.sin((3, 4)) // Fails, as Typst does not support vector operations directly}
\NormalTok{\#nt.sin((3.4))    // Sine of each element in vector: (0.14411, 0.90929)}

\NormalTok{// Generate equally spaced values and vectorized functions}
\NormalTok{\#let x = nt.linspace(0, 10, 3)  // Generate 3 equally spaced values between 0 and 10: (0, 5, 10)}
\NormalTok{\#let y = nt.sin(x)              // Apply sine function to each element: (0, {-}0.95, {-}0.54)}

\NormalTok{// Logical operations}
\NormalTok{\#nt.eq(a, b)      // Compare each element in \textquotesingle{}a\textquotesingle{} to \textquotesingle{}b\textquotesingle{}: (false, true, false)}
\NormalTok{\#nt.any(nt.eq(a, b)) // Check if any element in \textquotesingle{}a\textquotesingle{} equals \textquotesingle{}b\textquotesingle{}: true}
\NormalTok{\#nt.all(nt.eq(a, b)) // Check if all elements in \textquotesingle{}a\textquotesingle{} equal \textquotesingle{}b\textquotesingle{}: false}

\NormalTok{// Handling special cases like division by zero}
\NormalTok{\#nt.div((1, 3), (2, 0))  // Element{-}wise division, with NaN for division by zero: (0.5, float.nan)}

\NormalTok{// Matrix operations (element{-}wise)}
\NormalTok{\#nt.add(m, 1)  // Add scalar to matrix elements: ((2, 3), (2, 4))}

\NormalTok{// matrix}
\NormalTok{\#nt.transpose(m)  // transposition}
\NormalTok{\#nt.matmul(m,m) //  matrix multipliation}
\NormalTok{\#nt.matmul(c(1,2), r(2,3)) //  colum vector times row vector multiplication.}
\NormalTok{\#np.trace(m) // trace}
\NormalTok{\#np.det(m) /2x2 determinant}
 
\NormalTok{// printing}
\NormalTok{\#nt.print(m, " != " , (1,2))  // long dollar print, see in pdf }
\NormalTok{\#nt.p(m, " != " , (1,2))  //  short long print print, see in pdf }
\end{Highlighting}
\end{Shaded}

Since vesion 0.0.4 n-dim matrices are supported as well in most
operations.

\subsection{Supported Features:}\label{supported-features}

\subsubsection{Dimensions:}\label{dimensions}

Numty uses standard typst list as a base type, most 1d operations like
dot are suported directly for them.

For matrix specific operations we use 2d arrays / nested arrays, that
are also the standard typst list, but nested like in: ((1,2), (1,1)).

For convenience you can create column or row vectors with the \#nt.c and
\#nt.r functions.

\begin{Shaded}
\begin{Highlighting}[]
\NormalTok{\#import "numty.typ" as nt}
\NormalTok{\#import "numty.typ": c, r}

\NormalTok{\#let a = (1,2,3)}
\NormalTok{\#let b = (3,2,1)}
\NormalTok{\#c(..a) // ((1,),(2,),(3,)) }
\NormalTok{\#r(..b) // ((3,2,1),)}
\NormalTok{\#nt.matmul(c(..a), r(..b)) // column @ row}
\end{Highlighting}
\end{Shaded}

\subsubsection{Logic Operations:}\label{logic-operations}

\begin{Shaded}
\begin{Highlighting}[]
\NormalTok{\#let a = (1,2,3)}
\NormalTok{\#let b = 2}

\NormalTok{\#nt.eq(a,b)  // (false, true, false)}
\NormalTok{\#nt.all(nt.eq(a,b))  // false}
\NormalTok{\#nt.any(nt.eq(a,b))  // true}
\end{Highlighting}
\end{Shaded}

\subsubsection{Math operators:}\label{math-operators}

All operators are element-wise, traditional matrix multiplication is not
yet supported.

\begin{Shaded}
\begin{Highlighting}[]
\NormalTok{\#nt.add((0,1,3), 1)  // (1,2,4)}
\NormalTok{\#nt.mult((1,3),(2,2)) // (2,6)}
\NormalTok{\#nt.div((1,3), (2,0)) // (0.5,float.nan)}
\end{Highlighting}
\end{Shaded}

\subsubsection{Algebra with Nan
handling:}\label{algebra-with-nan-handling}

\begin{Shaded}
\begin{Highlighting}[]
\NormalTok{\#nt.log((0,1,3)) //  (float.nan, 0 , 0.47...)}
\NormalTok{\#nt.sin((1,3)) //  (0.84.. , 0.14...)}
\end{Highlighting}
\end{Shaded}

\subsubsection{Vector operations:}\label{vector-operations}

Basic vector operations

\begin{Shaded}
\begin{Highlighting}[]
\NormalTok{\#nt.dot((1,2),(2,4))  //  9}
\NormalTok{\#nt.normalize((1,4), l:1) // (1/5,4/5)}
\end{Highlighting}
\end{Shaded}

\subsubsection{Others:}\label{others}

Functions for creating equally spaced indexes in linear and logspace,
usefull for log plots, to sample in acordance to the logarithmic space.

\begin{Shaded}
\begin{Highlighting}[]
\NormalTok{\#nt.linspace(0,10,3) // (0,5,10)}
\NormalTok{\#nt.logspace(1,3,3)}
\NormalTok{\#nt.geomspace(1,3,3) }
\end{Highlighting}
\end{Shaded}

\subsubsection{Matrix}\label{matrix}

\begin{Shaded}
\begin{Highlighting}[]
\NormalTok{\#nt.matmul(m,m )              // matrix multiplication}
\NormalTok{\#nt.det(((1,3), (3,4)))       // only 2x2 supported for now}
\NormalTok{\#nt.trace(((1,3), (3,4)))     // trace of square matrix}
\NormalTok{\#nt.transpose(((1,3), (3,4))) // matrix transposition}
\end{Highlighting}
\end{Shaded}

\subsubsection{Printing}\label{printing}

Numty supports \$ printing to the pdf of numerical matrices, both long
and short format.

\begin{Shaded}
\begin{Highlighting}[]
\NormalTok{\#nt.print((1,2),(4,2)))  // to display in the pdf}
\NormalTok{\#nt.p((1,2),(4,2)), " random string ")     // to display in the pdf}
\end{Highlighting}
\end{Shaded}

\subsubsection{How to add}\label{how-to-add}

Copy this into your project and use the import as \texttt{\ numty\ }

\begin{verbatim}
#import "@preview/numty:0.0.5"
\end{verbatim}

\includesvg[width=0.16667in,height=0.16667in]{/assets/icons/16-copy.svg}

Check the docs for
\href{https://typst.app/docs/reference/scripting/\#packages}{more
information on how to import packages} .

\subsubsection{About}\label{about}

\begin{description}
\tightlist
\item[Author :]
Pablo Ruiz Cuevas
\item[License:]
MIT
\item[Current version:]
0.0.5
\item[Last updated:]
November 12, 2024
\item[First released:]
October 22, 2024
\item[Archive size:]
4.27 kB
\href{https://packages.typst.org/preview/numty-0.0.5.tar.gz}{\pandocbounded{\includesvg[keepaspectratio]{/assets/icons/16-download.svg}}}
\item[Repository:]
\href{https://github.com/PabloRuizCuevas/numty}{GitHub}
\item[Categor ies :]
\begin{itemize}
\tightlist
\item[]
\item
  \pandocbounded{\includesvg[keepaspectratio]{/assets/icons/16-hammer.svg}}
  \href{https://typst.app/universe/search/?category=utility}{Utility}
\item
  \pandocbounded{\includesvg[keepaspectratio]{/assets/icons/16-code.svg}}
  \href{https://typst.app/universe/search/?category=scripting}{Scripting}
\end{itemize}
\end{description}

\subsubsection{Where to report issues?}\label{where-to-report-issues}

This package is a project of Pablo Ruiz Cuevas . Report issues on
\href{https://github.com/PabloRuizCuevas/numty}{their repository} . You
can also try to ask for help with this package on the
\href{https://forum.typst.app}{Forum} .

Please report this package to the Typst team using the
\href{https://typst.app/contact}{contact form} if you believe it is a
safety hazard or infringes upon your rights.

\phantomsection\label{versions}
\subsubsection{Version history}\label{version-history}

\begin{longtable}[]{@{}ll@{}}
\toprule\noalign{}
Version & Release Date \\
\midrule\noalign{}
\endhead
\bottomrule\noalign{}
\endlastfoot
0.0.5 & November 12, 2024 \\
\href{https://typst.app/universe/package/numty/0.0.4/}{0.0.4} & October
31, 2024 \\
\href{https://typst.app/universe/package/numty/0.0.3/}{0.0.3} & October
23, 2024 \\
\href{https://typst.app/universe/package/numty/0.0.2/}{0.0.2} & October
22, 2024 \\
\href{https://typst.app/universe/package/numty/0.0.1/}{0.0.1} & October
22, 2024 \\
\end{longtable}

Typst GmbH did not create this package and cannot guarantee correct
functionality of this package or compatibility with any version of the
Typst compiler or app.
