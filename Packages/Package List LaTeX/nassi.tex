\title{typst.app/universe/package/nassi}

\phantomsection\label{banner}
\section{nassi}\label{nassi}

{ 0.1.2 }

Draw Nassi-Shneiderman diagrams (Struktogramme) with Typst.

\phantomsection\label{readme}
\textbf{nassi} is a package for \href{https://typst.app/}{Typst} to draw
\href{https://en.wikipedia.org/wiki/Nassi\%E2\%80\%93Shneiderman_diagram}{Nassi-Shneiderman
diagrams} (Struktogramme).

\pandocbounded{\includegraphics[keepaspectratio]{https://github.com/typst/packages/raw/main/packages/preview/nassi/0.1.2/assets/example-1.png}}

\subsection{Usage}\label{usage}

Import \textbf{nassi} in your document:

\begin{Shaded}
\begin{Highlighting}[]
\NormalTok{\#import "@preview/nassi:0.1.2"}
\end{Highlighting}
\end{Shaded}

There are several options to draw diagrams. One is to parse all
code-blocks with the language “nassi�. Simply add a show-rule like
this:

\begin{Shaded}
\begin{Highlighting}[]
\NormalTok{\#import "@preview/nassi:0.1.2"}
\NormalTok{\#show: nassi.shneiderman()}

\NormalTok{\textasciigrave{}\textasciigrave{}\textasciigrave{}nassi}
\NormalTok{function ggt(a, b)}
\NormalTok{  while a \textgreater{} 0 and b \textgreater{} 0}
\NormalTok{    if a \textgreater{} b}
\NormalTok{      a \textless{}{-} a {-} b}
\NormalTok{    else}
\NormalTok{      b \textless{}{-} b {-} a}
\NormalTok{    endif}
\NormalTok{  endwhile}
\NormalTok{  if b == 0}
\NormalTok{    return a}
\NormalTok{  else}
\NormalTok{    return b}
\NormalTok{  endif}
\NormalTok{endfunction}
\NormalTok{\textasciigrave{}\textasciigrave{}\textasciigrave{}}
\end{Highlighting}
\end{Shaded}

In this case, the diagram is created from a simple pseudocode. To have
more control over the output, you can add blocks manually using the
element functions provided in \texttt{\ nassi.elements\ } :

\begin{Shaded}
\begin{Highlighting}[]
\NormalTok{\#import "@preview/nassi:0.1.2"}

\NormalTok{\#nassi.diagram(\{}
\NormalTok{    import nassi.elements: *}

\NormalTok{    function("ggt(a, b)", \{}
\NormalTok{        loop("a \textgreater{} b and b \textgreater{} 0", \{}
\NormalTok{            branch("a \textgreater{} b", \{}
\NormalTok{                assign("a", "a {-} b")}
\NormalTok{            \}, \{}
\NormalTok{                assign("b", "b {-} a",}
\NormalTok{                    fill: gradient.linear(..color.map.rainbow),}
\NormalTok{                    stroke:red + 2pt}
\NormalTok{                )}
\NormalTok{            \})}
\NormalTok{        \})}
\NormalTok{        branch("b == 0", \{ process("return a") \}, \{ process("return b") \})}
\NormalTok{    \})}
\NormalTok{\})}
\end{Highlighting}
\end{Shaded}

\pandocbounded{\includegraphics[keepaspectratio]{https://github.com/typst/packages/raw/main/packages/preview/nassi/0.1.2/assets/example-3.png}}

Since \textbf{nassi} uses \textbf{cetz} for drawing, you can add
diagrams directly to a canvas. Each block gets a name within the diagram
group to reference it in the drawing:

\begin{Shaded}
\begin{Highlighting}[]
\NormalTok{\#import "@preview/cetz:0.2.2"}
\NormalTok{\#import "@preview/nassi:0.1.2"}

\NormalTok{\#cetz.canvas(\{}
\NormalTok{  import nassi.draw: diagram}
\NormalTok{  import nassi.elements: *}
\NormalTok{  import cetz.draw: *}

\NormalTok{  diagram((4,4), \{}
\NormalTok{    function("ggt(a, b)", \{}
\NormalTok{      loop("a \textgreater{} b and b \textgreater{} 0", \{}
\NormalTok{        branch("a \textgreater{} b", \{}
\NormalTok{          assign("a", "a {-} b")}
\NormalTok{        \}, \{}
\NormalTok{          assign("b", "b {-} a")}
\NormalTok{        \})}
\NormalTok{      \})}
\NormalTok{      branch("b == 0", \{ process("return a") \}, \{ process("return b") \})}
\NormalTok{    \})}
\NormalTok{  \})}

\NormalTok{  for i in range(8) \{}
\NormalTok{    content(}
\NormalTok{      "nassi.e" + str(i+1) + ".north{-}west",}
\NormalTok{      stroke:red,}
\NormalTok{      fill:red.transparentize(50\%),}
\NormalTok{      frame:"circle",}
\NormalTok{      padding:.05,}
\NormalTok{      anchor:"north{-}west",}
\NormalTok{      text(white, weight:"bold", "e"+str(i)),}
\NormalTok{    )}
\NormalTok{  \}}
\NormalTok{\})}
\end{Highlighting}
\end{Shaded}

\pandocbounded{\includegraphics[keepaspectratio]{https://github.com/typst/packages/raw/main/packages/preview/nassi/0.1.2/assets/example-cetz-2.png}}

This can be useful to annotate a diagram:

\pandocbounded{\includegraphics[keepaspectratio]{https://github.com/typst/packages/raw/main/packages/preview/nassi/0.1.2/assets/example-cetz.png}}

See \texttt{\ assets/\ } for usage examples.

\subsection{Changelog}\label{changelog}

\subsubsection{Version 0.1.2}\label{version-0.1.2}

\begin{itemize}
\tightlist
\item
  Fix for deprecation warnings in Typst 0.12.
\end{itemize}

\subsubsection{Version 0.1.1}\label{version-0.1.1}

\begin{itemize}
\tightlist
\item
  Fixed labels option not working for branches in other elements.
\item
  Added \texttt{\ switch\ } statements (thanks to @Geronymos).
\end{itemize}

\subsubsection{Version 0.1.0}\label{version-0.1.0}

Initial release of \textbf{nassi} .

\subsubsection{How to add}\label{how-to-add}

Copy this into your project and use the import as \texttt{\ nassi\ }

\begin{verbatim}
#import "@preview/nassi:0.1.2"
\end{verbatim}

\includesvg[width=0.16667in,height=0.16667in]{/assets/icons/16-copy.svg}

Check the docs for
\href{https://typst.app/docs/reference/scripting/\#packages}{more
information on how to import packages} .

\subsubsection{About}\label{about}

\begin{description}
\tightlist
\item[Author :]
Jonas Neugebauer
\item[License:]
MIT
\item[Current version:]
0.1.2
\item[Last updated:]
October 23, 2024
\item[First released:]
June 3, 2024
\item[Minimum Typst version:]
0.11.0
\item[Archive size:]
5.93 kB
\href{https://packages.typst.org/preview/nassi-0.1.2.tar.gz}{\pandocbounded{\includesvg[keepaspectratio]{/assets/icons/16-download.svg}}}
\item[Repository:]
\href{https://github.com/jneug/typst-nassi}{GitHub}
\item[Discipline :]
\begin{itemize}
\tightlist
\item[]
\item
  \href{https://typst.app/universe/search/?discipline=computer-science}{Computer
  Science}
\end{itemize}
\item[Categor y :]
\begin{itemize}
\tightlist
\item[]
\item
  \pandocbounded{\includesvg[keepaspectratio]{/assets/icons/16-chart.svg}}
  \href{https://typst.app/universe/search/?category=visualization}{Visualization}
\end{itemize}
\end{description}

\subsubsection{Where to report issues?}\label{where-to-report-issues}

This package is a project of Jonas Neugebauer . Report issues on
\href{https://github.com/jneug/typst-nassi}{their repository} . You can
also try to ask for help with this package on the
\href{https://forum.typst.app}{Forum} .

Please report this package to the Typst team using the
\href{https://typst.app/contact}{contact form} if you believe it is a
safety hazard or infringes upon your rights.

\phantomsection\label{versions}
\subsubsection{Version history}\label{version-history}

\begin{longtable}[]{@{}ll@{}}
\toprule\noalign{}
Version & Release Date \\
\midrule\noalign{}
\endhead
\bottomrule\noalign{}
\endlastfoot
0.1.2 & October 23, 2024 \\
\href{https://typst.app/universe/package/nassi/0.1.1/}{0.1.1} & October
2, 2024 \\
\href{https://typst.app/universe/package/nassi/0.1.0/}{0.1.0} & June 3,
2024 \\
\end{longtable}

Typst GmbH did not create this package and cannot guarantee correct
functionality of this package or compatibility with any version of the
Typst compiler or app.
