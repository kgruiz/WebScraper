\title{typst.app/universe/package/tablem}

\phantomsection\label{banner}
\section{tablem}\label{tablem}

{ 0.1.0 }

Write markdown-like tables easily.

\phantomsection\label{readme}
Write markdown-like tables easily.

\subsection{Example}\label{example}

Have a look at the source
\href{https://github.com/typst/packages/raw/main/packages/preview/tablem/0.1.0/examples/example.typ}{here}
.

\pandocbounded{\includegraphics[keepaspectratio]{https://github.com/typst/packages/raw/main/packages/preview/tablem/0.1.0/examples/example.png}}

\subsection{Usage}\label{usage}

You can simply copy the markdown table and paste it in
\texttt{\ tablem\ } function.

\begin{Shaded}
\begin{Highlighting}[]
\NormalTok{\#import "@preview/tablem:0.1.0": tablem}

\NormalTok{\#tablem[}
\NormalTok{  | *Name* | *Location* | *Height* | *Score* |}
\NormalTok{  | {-}{-}{-}{-}{-}{-} | {-}{-}{-}{-}{-}{-}{-}{-}{-}{-} | {-}{-}{-}{-}{-}{-}{-}{-} | {-}{-}{-}{-}{-}{-}{-} |}
\NormalTok{  | John   | Second St. | 180 cm   |  5      |}
\NormalTok{  | Wally  | Third Av.  | 160 cm   |  10     |}
\NormalTok{]}
\end{Highlighting}
\end{Shaded}

And you can use custom render function.

\begin{Shaded}
\begin{Highlighting}[]
\NormalTok{\#import "@preview/tablex:0.0.6": tablex, hlinex}
\NormalTok{\#import "@preview/tablem:0.1.0": tablem}

\NormalTok{\#let three{-}line{-}table = tablem.with(}
\NormalTok{  render: (columns: auto, ..args) =\textgreater{} \{}
\NormalTok{    tablex(}
\NormalTok{      columns: columns,}
\NormalTok{      auto{-}lines: false,}
\NormalTok{      align: center + horizon,}
\NormalTok{      hlinex(y: 0),}
\NormalTok{      hlinex(y: 1),}
\NormalTok{      ..args,}
\NormalTok{      hlinex(),}
\NormalTok{    )}
\NormalTok{  \}}
\NormalTok{)}

\NormalTok{\#three{-}line{-}table[}
\NormalTok{  | *Name* | *Location* | *Height* | *Score* |}
\NormalTok{  | {-}{-}{-}{-}{-}{-} | {-}{-}{-}{-}{-}{-}{-}{-}{-}{-} | {-}{-}{-}{-}{-}{-}{-}{-} | {-}{-}{-}{-}{-}{-}{-} |}
\NormalTok{  | John   | Second St. | 180 cm   |  5      |}
\NormalTok{  | Wally  | Third Av.  | 160 cm   |  10     |}
\NormalTok{]}
\end{Highlighting}
\end{Shaded}

\pandocbounded{\includegraphics[keepaspectratio]{https://github.com/typst/packages/raw/main/packages/preview/tablem/0.1.0/examples/example.png}}

\subsection{\texorpdfstring{\texttt{\ tablem\ }
function}{ tablem  function}}\label{tablem-function}

\begin{Shaded}
\begin{Highlighting}[]
\NormalTok{\#let tablem(}
\NormalTok{  render: table,}
\NormalTok{  ignore{-}second{-}row: true,}
\NormalTok{  ..args,}
\NormalTok{  body}
\NormalTok{) = \{ .. \}}
\end{Highlighting}
\end{Shaded}

\textbf{Arguments:}

\begin{itemize}
\tightlist
\item
  \texttt{\ render\ } : {[}
  \texttt{\ (columns:\ int,\ ..args)\ =\textgreater{}\ \{\ ..\ \}\ } {]}
  â€'' Custom render function, default to be \texttt{\ table\ } ,
  receiving a integer-type columns, which is the count of first row.
  \texttt{\ ..args\ } is the combination of \texttt{\ args\ } of
  \texttt{\ tablem\ } function and children genenerated from
  \texttt{\ body\ } .
\item
  \texttt{\ ignore-second-row\ } : {[} \texttt{\ boolean\ } {]} â€''
  Whether to ignore the second row (something like
  \texttt{\ \textbar{}-\/-\/-\textbar{}\ } ).
\item
  \texttt{\ args\ } : {[} \texttt{\ any\ } {]} â€'' Some arguments you
  want to pass to \texttt{\ render\ } function.
\item
  \texttt{\ body\ } : {[} \texttt{\ content\ } {]} â€'' The
  markdown-like table. There should be no extra line breaks in it.
\end{itemize}

\subsection{Limitations}\label{limitations}

Cell merging has not yet been implemented.

\subsection{License}\label{license}

This project is licensed under the MIT License.

\subsubsection{How to add}\label{how-to-add}

Copy this into your project and use the import as \texttt{\ tablem\ }

\begin{verbatim}
#import "@preview/tablem:0.1.0"
\end{verbatim}

\includesvg[width=0.16667in,height=0.16667in]{/assets/icons/16-copy.svg}

Check the docs for
\href{https://typst.app/docs/reference/scripting/\#packages}{more
information on how to import packages} .

\subsubsection{About}\label{about}

\begin{description}
\tightlist
\item[Author :]
OrangeX4
\item[License:]
MIT
\item[Current version:]
0.1.0
\item[Last updated:]
November 18, 2023
\item[First released:]
November 18, 2023
\item[Archive size:]
2.37 kB
\href{https://packages.typst.org/preview/tablem-0.1.0.tar.gz}{\pandocbounded{\includesvg[keepaspectratio]{/assets/icons/16-download.svg}}}
\item[Repository:]
\href{https://github.com/OrangeX4/typst-tablem}{GitHub}
\end{description}

\subsubsection{Where to report issues?}\label{where-to-report-issues}

This package is a project of OrangeX4 . Report issues on
\href{https://github.com/OrangeX4/typst-tablem}{their repository} . You
can also try to ask for help with this package on the
\href{https://forum.typst.app}{Forum} .

Please report this package to the Typst team using the
\href{https://typst.app/contact}{contact form} if you believe it is a
safety hazard or infringes upon your rights.

\phantomsection\label{versions}
\subsubsection{Version history}\label{version-history}

\begin{longtable}[]{@{}ll@{}}
\toprule\noalign{}
Version & Release Date \\
\midrule\noalign{}
\endhead
\bottomrule\noalign{}
\endlastfoot
0.1.0 & November 18, 2023 \\
\end{longtable}

Typst GmbH did not create this package and cannot guarantee correct
functionality of this package or compatibility with any version of the
Typst compiler or app.
