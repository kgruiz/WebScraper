\title{typst.app/universe/package/jogs}

\phantomsection\label{banner}
\section{jogs}\label{jogs}

{ 0.2.3 }

QuickJS JavaScript runtime for Typst

\phantomsection\label{readme}
Quickjs javascript runtime for typst. This package provides a typst
plugin for evaluating javascript code.

\subsection{Example}\label{example}

\begin{Shaded}
\begin{Highlighting}[]
\NormalTok{\#import "@preview/jogs:0.2.3": *}

\NormalTok{\#set page(height: auto, width: auto, fill: black, margin: 1em)}
\NormalTok{\#set text(fill: white)}

\NormalTok{\#let code = \textasciigrave{}\textasciigrave{}\textasciigrave{}}
\NormalTok{function sum() \{}
\NormalTok{  const total = Array.prototype.slice.call(arguments).reduce(function(a, b) \{}
\NormalTok{      return a + b;}
\NormalTok{  \}, 0);}
\NormalTok{  return total;}
\NormalTok{\}}

\NormalTok{function string\_join(arr) \{}
\NormalTok{  return arr.join(" | ");}
\NormalTok{\}}

\NormalTok{function return\_complex\_obj() \{}
\NormalTok{  return \{}
\NormalTok{    a: 1,}
\NormalTok{    b: "2",}
\NormalTok{    c: [1, 2, 3],}
\NormalTok{    d: \{}
\NormalTok{      e: 1,}
\NormalTok{    \},}
\NormalTok{  \};}
\NormalTok{\}}

\NormalTok{\textasciigrave{}\textasciigrave{}\textasciigrave{}}
\NormalTok{\#let bytecode = compile{-}js(code)}

\NormalTok{\#list{-}global{-}property(bytecode)}

\NormalTok{\#call{-}js{-}function(bytecode, "sum", 6, 7, 8, 9, 10)}

\NormalTok{\#call{-}js{-}function(bytecode, "string\_join", ("a", "b", "c", "d", "e"),)}

\NormalTok{\#call{-}js{-}function(bytecode, "return\_complex\_obj",)}

\end{Highlighting}
\end{Shaded}

result:

\pandocbounded{\includesvg[keepaspectratio]{https://github.com/typst/packages/raw/main/packages/preview/jogs/0.2.3/typst-package/examples/fib.svg}}

\subsection{Documentation}\label{documentation}

This package provide following function(s):

\subsubsection{\texorpdfstring{\texttt{\ eval-js\ }}{ eval-js }}\label{eval-js}

Run a Javascript code snippet.

\paragraph{Arguments}\label{arguments}

\begin{itemize}
\tightlist
\item
  \texttt{\ code\ } - The Javascript code to run. It can be a string or
  a raw block.
\end{itemize}

\paragraph{Returns}\label{returns}

The result of the Javascript code. The type is the typst type which most
closely resembles the Javascript type.

\paragraph{Example}\label{example-1}

\begin{Shaded}
\begin{Highlighting}[]
\NormalTok{\#let result = eval{-}js("1 + 1")}
\end{Highlighting}
\end{Shaded}

\subsubsection{\texorpdfstring{\texttt{\ compile-js\ }}{ compile-js }}\label{compile-js}

Compile a Javascript code snippet into bytecode.

\paragraph{Arguments}\label{arguments-1}

\begin{itemize}
\tightlist
\item
  \texttt{\ code\ } - The Javascript code to compile. It can be a string
  or a raw block.
\end{itemize}

\paragraph{Returns}\label{returns-1}

The bytecode of the Javascript code. The type is \texttt{\ bytes\ } .

\paragraph{Example}\label{example-2}

\begin{Shaded}
\begin{Highlighting}[]
\NormalTok{\#let bytecode = compile{-}js("function sum(a, b) \{ return a + b; \}")}
\end{Highlighting}
\end{Shaded}

\subsubsection{\texorpdfstring{\texttt{\ call-js-function\ }}{ call-js-function }}\label{call-js-function}

Call a Javascript function with arguments.

\paragraph{Arguments}\label{arguments-2}

\begin{itemize}
\tightlist
\item
  \texttt{\ bytecode\ } - The bytecode of the Javascript function. It
  can be obtained by calling \texttt{\ compile-js\ } .
\item
  \texttt{\ name\ } - The name of the Javascript function.
\item
  \texttt{\ ..args\ } - The arguments to pass to the Javascript
  function. All other positional arguments are passed to the Javascript
  function as is.
\end{itemize}

\paragraph{Returns}\label{returns-2}

The result of the Javascript function. The type is the typst type which
most closely resembles the Javascript type.

\paragraph{Example}\label{example-3}

\begin{Shaded}
\begin{Highlighting}[]
\NormalTok{\#let bytecode = compile{-}js("function sum(a, b) \{ return a + b; \}")}
\NormalTok{\#let result = call{-}js{-}function(bytecode, "sum", 1, 2)}
\end{Highlighting}
\end{Shaded}

\subsubsection{\texorpdfstring{\texttt{\ list-global-property\ }}{ list-global-property }}\label{list-global-property}

List all global properties of a compiled Javascript bytecode. This is
useful for inspecting the compiled Javascript bytecode.

\paragraph{Arguments}\label{arguments-3}

\begin{itemize}
\tightlist
\item
  \texttt{\ bytecode\ } - The bytecode of the Javascript function. It
  can be obtained by calling \texttt{\ compile-js\ } .
\end{itemize}

\paragraph{Returns}\label{returns-3}

A list of all global properties of the Javascript bytecode. The type is
\texttt{\ array\ } .

\paragraph{Example}\label{example-4}

\begin{Shaded}
\begin{Highlighting}[]
\NormalTok{\#let bytecode = compile{-}js("function sum(a, b) \{ return a + b; \}")}
\NormalTok{\#let result = list{-}global{-}property(bytecode)}
\end{Highlighting}
\end{Shaded}

\subsubsection{How to add}\label{how-to-add}

Copy this into your project and use the import as \texttt{\ jogs\ }

\begin{verbatim}
#import "@preview/jogs:0.2.3"
\end{verbatim}

\includesvg[width=0.16667in,height=0.16667in]{/assets/icons/16-copy.svg}

Check the docs for
\href{https://typst.app/docs/reference/scripting/\#packages}{more
information on how to import packages} .

\subsubsection{About}\label{about}

\begin{description}
\tightlist
\item[Author :]
Wenzhuo Liu
\item[License:]
MIT
\item[Current version:]
0.2.3
\item[Last updated:]
February 1, 2024
\item[First released:]
November 6, 2023
\item[Archive size:]
354 kB
\href{https://packages.typst.org/preview/jogs-0.2.3.tar.gz}{\pandocbounded{\includesvg[keepaspectratio]{/assets/icons/16-download.svg}}}
\item[Repository:]
\href{https://github.com/Enter-tainer/jogs}{GitHub}
\end{description}

\subsubsection{Where to report issues?}\label{where-to-report-issues}

This package is a project of Wenzhuo Liu . Report issues on
\href{https://github.com/Enter-tainer/jogs}{their repository} . You can
also try to ask for help with this package on the
\href{https://forum.typst.app}{Forum} .

Please report this package to the Typst team using the
\href{https://typst.app/contact}{contact form} if you believe it is a
safety hazard or infringes upon your rights.

\phantomsection\label{versions}
\subsubsection{Version history}\label{version-history}

\begin{longtable}[]{@{}ll@{}}
\toprule\noalign{}
Version & Release Date \\
\midrule\noalign{}
\endhead
\bottomrule\noalign{}
\endlastfoot
0.2.3 & February 1, 2024 \\
\href{https://typst.app/universe/package/jogs/0.2.2/}{0.2.2} & January
15, 2024 \\
\href{https://typst.app/universe/package/jogs/0.2.1/}{0.2.1} & November
29, 2023 \\
\href{https://typst.app/universe/package/jogs/0.2.0/}{0.2.0} & November
7, 2023 \\
\href{https://typst.app/universe/package/jogs/0.1.0/}{0.1.0} & November
6, 2023 \\
\end{longtable}

Typst GmbH did not create this package and cannot guarantee correct
functionality of this package or compatibility with any version of the
Typst compiler or app.
