\title{typst.app/universe/package/pavemat}

\phantomsection\label{banner}
\section{pavemat}\label{pavemat}

{ 0.1.0 }

Style matrices with custom paths, strokes and fills for appealing
visualizations.

{ } Featured Package

\phantomsection\label{readme}
\pandocbounded{\includesvg[keepaspectratio]{https://github.com/typst/packages/raw/main/packages/preview/pavemat/0.1.0/examples/logo.svg}}

repo: \url{https://github.com/QuadnucYard/pavemat}

\subsection{Introduction}\label{introduction}

The \emph{pavemat} is a tool for creating styled matrices with custom
paths, strokes, and fills. It allows users to define how paths should be
drawn through the matrix, apply different strokes to these paths, and
fill specific cells with various colors. This function is particularly
useful for visualizing complex data structures, mathematical matrices,
and creating custom grid layouts.

\subsection{Examples}\label{examples}

The logo example:

\begin{Shaded}
\begin{Highlighting}[]
\NormalTok{\#\{}
\NormalTok{  set math.mat(row{-}gap: 0.25em, column{-}gap: 0.1em)}
\NormalTok{  set text(size: 2em)}

\NormalTok{  pavemat(}
\NormalTok{    pave: (}
\NormalTok{      "SDS(dash: \textquotesingle{}solid\textquotesingle{})DDD]WW",}
\NormalTok{      (path: "sdDDD", stroke: aqua.darken(30\%))}
\NormalTok{    ),}
\NormalTok{    stroke: (dash: "dashed", thickness: 1pt, paint: yellow),}
\NormalTok{    fills: (}
\NormalTok{      "0{-}0": green.transparentize(80\%),}
\NormalTok{      "1{-}1": blue.transparentize(80\%),}
\NormalTok{      "[0{-}0]": green.transparentize(60\%),}
\NormalTok{      "[1{-}1]": blue.transparentize(60\%),}
\NormalTok{    ),}
\NormalTok{    delim: "[",}
\NormalTok{  )[$mat(P, a, v, e; "", m, a, t)$]}
\NormalTok{\}}
\end{Highlighting}
\end{Shaded}

Code of examples can be found in
\href{https://github.com/QuadnucYard/pavemat/tree/main/examples}{\texttt{\ examples/examples.typ\ }}
.

\pandocbounded{\includesvg[keepaspectratio]{https://github.com/typst/packages/raw/main/packages/preview/pavemat/0.1.0/examples/example1.svg}}
\pandocbounded{\includesvg[keepaspectratio]{https://github.com/typst/packages/raw/main/packages/preview/pavemat/0.1.0/examples/example2.svg}}
\pandocbounded{\includesvg[keepaspectratio]{https://github.com/typst/packages/raw/main/packages/preview/pavemat/0.1.0/examples/example4.svg}}
\pandocbounded{\includesvg[keepaspectratio]{https://github.com/typst/packages/raw/main/packages/preview/pavemat/0.1.0/examples/example5.svg}}

\subsection{Manual}\label{manual}

See
\href{https://github.com/QuadnucYard/pavemat/tree/main/docs}{\texttt{\ docs/manual.typ\ }}
.

\subsubsection{How to add}\label{how-to-add}

Copy this into your project and use the import as \texttt{\ pavemat\ }

\begin{verbatim}
#import "@preview/pavemat:0.1.0"
\end{verbatim}

\includesvg[width=0.16667in,height=0.16667in]{/assets/icons/16-copy.svg}

Check the docs for
\href{https://typst.app/docs/reference/scripting/\#packages}{more
information on how to import packages} .

\subsubsection{About}\label{about}

\begin{description}
\tightlist
\item[Author :]
\href{https://github.com/QuadnucYard}{QuadnucYard}
\item[License:]
MIT
\item[Current version:]
0.1.0
\item[Last updated:]
July 29, 2024
\item[First released:]
July 29, 2024
\item[Archive size:]
3.60 kB
\href{https://packages.typst.org/preview/pavemat-0.1.0.tar.gz}{\pandocbounded{\includesvg[keepaspectratio]{/assets/icons/16-download.svg}}}
\item[Repository:]
\href{https://github.com/QuadnucYard/pavemat}{GitHub}
\item[Categor y :]
\begin{itemize}
\tightlist
\item[]
\item
  \pandocbounded{\includesvg[keepaspectratio]{/assets/icons/16-chart.svg}}
  \href{https://typst.app/universe/search/?category=visualization}{Visualization}
\end{itemize}
\end{description}

\subsubsection{Where to report issues?}\label{where-to-report-issues}

This package is a project of QuadnucYard . Report issues on
\href{https://github.com/QuadnucYard/pavemat}{their repository} . You
can also try to ask for help with this package on the
\href{https://forum.typst.app}{Forum} .

Please report this package to the Typst team using the
\href{https://typst.app/contact}{contact form} if you believe it is a
safety hazard or infringes upon your rights.

\phantomsection\label{versions}
\subsubsection{Version history}\label{version-history}

\begin{longtable}[]{@{}ll@{}}
\toprule\noalign{}
Version & Release Date \\
\midrule\noalign{}
\endhead
\bottomrule\noalign{}
\endlastfoot
0.1.0 & July 29, 2024 \\
\end{longtable}

Typst GmbH did not create this package and cannot guarantee correct
functionality of this package or compatibility with any version of the
Typst compiler or app.
