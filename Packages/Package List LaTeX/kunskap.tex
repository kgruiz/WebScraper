\title{typst.app/universe/package/kunskap}

\phantomsection\label{banner}
\phantomsection\label{template-thumbnail}
\pandocbounded{\includegraphics[keepaspectratio]{https://packages.typst.org/preview/thumbnails/kunskap-0.1.0-small.webp}}

\section{kunskap}\label{kunskap}

{ 0.1.0 }

A template with generous spacing for reports, assignments, course
documents, and similar (shorter) documents.

\href{/app?template=kunskap&version=0.1.0}{Create project in app}

\phantomsection\label{readme}
A \href{https://typst.app/}{Typst} template mainly intended for shorter
academic documents such as reports, assignments, course documents, and
so on. Its name, \emph{“kunskap�} , means \emph{knowledge} in
Swedish.

See
\href{https://github.com/mbollmann/typst-kunskap/blob/main/example.pdf}{this
example PDF} for a longer demonstration of how it looks.

\subsection{Usage}\label{usage}

You can use this template in the Typst web app by clicking “Start from
template� on the dashboard and searching for \texttt{\ kunskap\ } .

Alternatively, you can use the CLI to kick this project off using the
command

\begin{Shaded}
\begin{Highlighting}[]
\ExtensionTok{typst}\NormalTok{ init @preview/kunskap}
\end{Highlighting}
\end{Shaded}

Typst will create a new directory with all the files needed to get you
started.

\subsection{Configuration}\label{configuration}

This template exports the \texttt{\ kunskap\ } function with several
arguments. You will want to set at least the following, describing the
metadata of your document:

\begin{longtable}[]{@{}ll@{}}
\toprule\noalign{}
Argument & Description \\
\midrule\noalign{}
\endhead
\bottomrule\noalign{}
\endlastfoot
\texttt{\ title\ } & Title of your document \\
\texttt{\ author\ } & Author(s) of your document; can be any content, or
an array of strings \\
\texttt{\ date\ } & Date string to display below the author(s); defaults
to a string of today’s date, but can take any content. Set to
\texttt{\ none\ } if you don’t use it at all. \\
\texttt{\ header\ } & Content that appears in the left-hand side of the
header on every page; this is intended for e.g. the name of a course or
some other contextual information for the document, but can of course
also be left empty. \\
\end{longtable}

Additionally, you can configure some aspects of the template’s layout
with the following arguments:

\begin{longtable}[]{@{}lll@{}}
\toprule\noalign{}
Argument & Description & Default \\
\midrule\noalign{}
\endhead
\bottomrule\noalign{}
\endlastfoot
\texttt{\ paper-size\ } & Paper size of the document &
\texttt{\ "a4"\ } \\
\texttt{\ body-font\ } & Font for the body text &
\texttt{\ "Noto\ Serif"\ } \\
\texttt{\ body-font-size\ } & Font size for the body text &
\texttt{\ 10pt\ } \\
\texttt{\ headings-font\ } & Font for the headings &
\texttt{\ ("Source\ Sans\ Pro",\ "Source\ Sans\ 3")\ } \\
\texttt{\ raw-font\ } & Font for raw (i.e. monospaced) text &
\texttt{\ ("Hack",\ "Source\ Code\ Pro")\ } {[}\^{}1{]} \\
\texttt{\ raw-font-size\ } & Font size for raw text &
\texttt{\ 9pt\ } \\
\texttt{\ link-color\ } & Color for highlighting
\href{https://typst.app/docs/reference/model/link/}{links} &
\texttt{\ rgb("\#3282b8")\ }
\pandocbounded{\includegraphics[keepaspectratio]{https://img.shields.io/badge/steel_blue-3282b8}} \\
\texttt{\ muted-color\ } & Color for muted text, such as page numbers
and headers after the first page & \texttt{\ luma(160)\ } \\
\texttt{\ block-bg-color\ } & Color for the background of raw text &
\texttt{\ luma(240)\ } \\
\end{longtable}

The template will initialize your document with a sample call to the
\texttt{\ kunskap\ } function. Alternatively, here’s a minimal example
of how you could use this template in your document:

\begin{Shaded}
\begin{Highlighting}[]
\NormalTok{\#import "@preview/kunskap:0.1.0": *}

\NormalTok{\#show: kunskap.with(}
\NormalTok{    title: "Your report title",}
\NormalTok{    author: "Your name",}
\NormalTok{    date: datetime.today().display(),}
\NormalTok{    header: "Your course name",}
\NormalTok{)}

\NormalTok{\#lorem(120)}
\end{Highlighting}
\end{Shaded}

\subsection{Missing features}\label{missing-features}

As of now, this template has not yet been particularly optimized for
styling related to:

\begin{itemize}
\tightlist
\item
  Bibliographies
\item
  Outlines (e.g. table of contents)
\item
  Tables
\end{itemize}

\subsection{Credits}\label{credits}

This template started out by emulating the layout of course documents in
\href{https://liu.se/en/employee/marku61}{Marco Kuhlmann} ’s courses
at Linköping University.{[}\^{}2{]} On the technical side, this
template took a lot of inspiration from
\href{https://github.com/talal/ilm/}{the \texttt{\ ilm\ } template} ,
even if the design decisions may be radically different.

{[}\^{}1{]}: The \href{https://github.com/source-foundry/Hack}{Hack
font} is currently not available on the Typst web app, so the fallback
is Source Code Pro. {[}\^{}2{]}: If you work at Linköping University,
you can set \texttt{\ headings-font:\ "KorolevLiU"\ } to get a
LiU-branded version of this template.

\href{/app?template=kunskap&version=0.1.0}{Create project in app}

\subsubsection{How to use}\label{how-to-use}

Click the button above to create a new project using this template in
the Typst app.

You can also use the Typst CLI to start a new project on your computer
using this command:

\begin{verbatim}
typst init @preview/kunskap:0.1.0
\end{verbatim}

\includesvg[width=0.16667in,height=0.16667in]{/assets/icons/16-copy.svg}

\subsubsection{About}\label{about}

\begin{description}
\tightlist
\item[Author :]
\href{mailto:marcel@bollmann.me}{Marcel Bollmann}
\item[License:]
MIT-0
\item[Current version:]
0.1.0
\item[Last updated:]
October 30, 2024
\item[First released:]
October 30, 2024
\item[Minimum Typst version:]
0.12.0
\item[Archive size:]
4.16 kB
\href{https://packages.typst.org/preview/kunskap-0.1.0.tar.gz}{\pandocbounded{\includesvg[keepaspectratio]{/assets/icons/16-download.svg}}}
\item[Repository:]
\href{https://github.com/mbollmann/typst-kunskap}{GitHub}
\item[Categor y :]
\begin{itemize}
\tightlist
\item[]
\item
  \pandocbounded{\includesvg[keepaspectratio]{/assets/icons/16-speak.svg}}
  \href{https://typst.app/universe/search/?category=report}{Report}
\end{itemize}
\end{description}

\subsubsection{Where to report issues?}\label{where-to-report-issues}

This template is a project of Marcel Bollmann . Report issues on
\href{https://github.com/mbollmann/typst-kunskap}{their repository} .
You can also try to ask for help with this template on the
\href{https://forum.typst.app}{Forum} .

Please report this template to the Typst team using the
\href{https://typst.app/contact}{contact form} if you believe it is a
safety hazard or infringes upon your rights.

\phantomsection\label{versions}
\subsubsection{Version history}\label{version-history}

\begin{longtable}[]{@{}ll@{}}
\toprule\noalign{}
Version & Release Date \\
\midrule\noalign{}
\endhead
\bottomrule\noalign{}
\endlastfoot
0.1.0 & October 30, 2024 \\
\end{longtable}

Typst GmbH did not create this template and cannot guarantee correct
functionality of this template or compatibility with any version of the
Typst compiler or app.
