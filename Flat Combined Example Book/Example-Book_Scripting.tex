\title{sitandr.github.io/typst-examples-book/book/basics/scripting/arguments}

\section{\texorpdfstring{\hyperref[advanced-arguments]{Advanced
arguments}}{Advanced arguments}}\label{advanced-arguments}

\subsection{\texorpdfstring{\hyperref[spreading-arguments-from-list]{Spreading
arguments from
list}}{Spreading arguments from list}}\label{spreading-arguments-from-list}

Spreading operator allows you to "unpack" the list of values into
arguments of function:

\begin{verbatim}
#let func(a, b, c, d, e) = [#a #b #c #d #e]
#func(..(([hi],) * 5))
\end{verbatim}

\pandocbounded{\includesvg[keepaspectratio]{typst-img/0586f1f7eb73effd507824b57f7282f12fe2612119d64413f72e6518aba01513-1.svg}}

This may be super useful in tables:

\begin{verbatim}
#let a = ("hi", "b", "c")

#table(columns: 3,
  [test], [x], [hello],
  ..a
)
\end{verbatim}

\pandocbounded{\includesvg[keepaspectratio]{typst-img/eb669f70df63815adcbe764fdb8635eecab33651c7eef55ea4de6cd63c96d9de-1.svg}}

\subsection{\texorpdfstring{\hyperref[key-arguments]{Key
arguments}}{Key arguments}}\label{key-arguments}

The same idea works with key arguments:

\begin{verbatim}
#let text-params = (fill: blue, size: 0.8em)

Some #text(..text-params)[text].
\end{verbatim}

\pandocbounded{\includesvg[keepaspectratio]{typst-img/e56483e8f4285f8fed8cd6867e720b9a1c9f62ef0bffea28d124159f8a61648d-1.svg}}

\section{\texorpdfstring{\hyperref[managing-arbitrary-arguments]{Managing
arbitrary
arguments}}{Managing arbitrary arguments}}\label{managing-arbitrary-arguments}

Typst allows taking as many arbitrary positional and key arguments as
you want.

In that case function is given special
\texttt{\ }{\texttt{\ arguments\ }}\texttt{\ } object that stores in it
positional and named arguments.

\begin{quote}
Link to
\href{https://typst.app/docs/reference/foundations/arguments/}{reference}
\end{quote}

\begin{verbatim}
#let f(..args) = [
  #args.pos()\
  #args.named()
]

#f(1, "a", width: 50%, block: false)
\end{verbatim}

\pandocbounded{\includesvg[keepaspectratio]{typst-img/2fc64c8521734ea689368ec83fe54025eb94b016a8ed1f6d6a9880ac6c94edf5-1.svg}}

You can combine them with other arguments. Spreading operator will "eat"
all remaining arguments:

\begin{verbatim}
#let format(title, ..authors) = {
  let by = authors
    .pos()
    .join(", ", last: " and ")

  [*#title* \ _Written by #by;_]
}

#format("ArtosFlow", "Jane", "Joe")
\end{verbatim}

\pandocbounded{\includesvg[keepaspectratio]{typst-img/4ba76c5176e0b93c6c2b03c38d55f88702546a5183717ed8c3567865c0d1bf5d-1.svg}}

\subsection{\texorpdfstring{\hyperref[optional-argument]{Optional
argument}}{Optional argument}}\label{optional-argument}

\emph{Currently the only way in Typst to create optional positional
arguments is using \texttt{\ }{\texttt{\ arguments\ }}\texttt{\ }
object:}

TODO


\title{sitandr.github.io/typst-examples-book/book/basics/scripting/index}

\section{\texorpdfstring{\hyperref[scripting]{Scripting}}{Scripting}}\label{scripting}

\textbf{Typst} has a complete interpreted language inside. One of key
aspects of working with your document in a nicer way


\title{sitandr.github.io/typst-examples-book/book/basics/scripting/types_2}

\section{\texorpdfstring{\hyperref[types-part-ii]{Types, part
II}}{Types, part II}}\label{types-part-ii}

In Typst, most of things are \textbf{immutable} . You
can\textquotesingle t change content, you can just create new using this
one (for example, using addition).

Immutability is very important for Typst since it tries to be \emph{as
pure language as possible} . Functions do nothing outside of returning
some value.

However, purity is partly "broken" by these types. They are
\emph{super-useful} and not adding them would make Typst much pain.

However, using them adds complexity.

\subsection{\texorpdfstring{\hyperref[arrays-array]{Arrays (
\texttt{\ }{\texttt{\ array\ }}\texttt{\ }
)}}{Arrays (   array   )}}\label{arrays-array}

\begin{quote}
\href{https://typst.app/docs/reference/foundations/array/}{Link to
Reference} .
\end{quote}

Mutable object that stores data with their indices.

\subsubsection{\texorpdfstring{\hyperref[working-with-indices]{Working
with indices}}{Working with indices}}\label{working-with-indices}

\begin{verbatim}
#let values = (1, 7, 4, -3, 2)

// take value at index 0
#values.at(0) \
// set value at 0 to 3
#(values.at(0) = 3)
// negative index => start from the back
#values.at(-1) \
// add index of something that is even
#values.find(calc.even)
\end{verbatim}

\pandocbounded{\includesvg[keepaspectratio]{typst-img/0374c20b28fbf2b2d15bc32e5428f7f5121ea9d673d96de3274a0c6d988d5fb5-1.svg}}

\subsubsection{\texorpdfstring{\hyperref[iterating-methods]{Iterating
methods}}{Iterating methods}}\label{iterating-methods}

\begin{verbatim}
#let values = (1, 7, 4, -3, 2)

// leave only what is odd
#values.filter(calc.odd) \
// create new list of absolute values of list values
#values.map(calc.abs) \
// reverse
#values.rev() \
// convert array of arrays to flat array
#(1, (2, 3)).flatten() \
// join array of string to string
#(("A", "B", "C")
 .join(", ", last: " and "))
\end{verbatim}

\pandocbounded{\includesvg[keepaspectratio]{typst-img/684400186916f8f16a2d7edb151b7f5023c7e4c010b23a2c6566f0bd7a224061-1.svg}}

\subsubsection{\texorpdfstring{\hyperref[list-operations]{List
operations}}{List operations}}\label{list-operations}

\begin{verbatim}
// sum of lists:
#((1, 2, 3) + (4, 5, 6))

// list product:
#((1, 2, 3) * 4)
\end{verbatim}

\pandocbounded{\includesvg[keepaspectratio]{typst-img/abe2d311638b351e0938be0e432f10265ca81a69a9ed7d2e6f88f656c60dfc65-1.svg}}

\subsubsection{\texorpdfstring{\hyperref[empty-list]{Empty
list}}{Empty list}}\label{empty-list}

\begin{verbatim}
#() \ // this is an empty list
#(1,) \  // this is a list with one element
BAD: #(1) // this is just an element, not a list!
\end{verbatim}

\pandocbounded{\includesvg[keepaspectratio]{typst-img/da4f77f8784462ca5c4f73862e58420695916064d56921e4adef7a7e37d5a532-1.svg}}

\subsection{\texorpdfstring{\hyperref[dictionaries-dict]{Dictionaries (
\texttt{\ }{\texttt{\ dict\ }}\texttt{\ }
)}}{Dictionaries (   dict   )}}\label{dictionaries-dict}

\begin{quote}
\href{https://typst.app/docs/reference/foundations/dictionary/}{Link to
Reference} .
\end{quote}

Dictionaries are objects that store a string "key" and a value,
associated with that key.

\begin{verbatim}
#let dict = (
  name: "Typst",
  born: 2019,
)

#dict.name \
#(dict.launch = 20)
#dict.len() \
#dict.keys() \
#dict.values() \
#dict.at("born") \
#dict.insert("city", "Berlin ")
#("name" in dict)
\end{verbatim}

\pandocbounded{\includesvg[keepaspectratio]{typst-img/638ada64eb36af0b1891def1b2c0a2cc97a14d87987df8c16f5f3872244553d6-1.svg}}

\subsubsection{\texorpdfstring{\hyperref[empty-dictionary]{Empty
dictionary}}{Empty dictionary}}\label{empty-dictionary}

\begin{verbatim}
This is an empty list: #() \
This is an empty dict: #(:)
\end{verbatim}

\pandocbounded{\includesvg[keepaspectratio]{typst-img/6ef41801d46f0b7256bb6913482fde054c811a1850ecae3a446660eb6d1c8850-1.svg}}


\title{sitandr.github.io/typst-examples-book/book/basics/scripting/conditions}

\section{\texorpdfstring{\hyperref[conditions--loops]{Conditions \&
loops}}{Conditions \& loops}}\label{conditions--loops}

\subsection{\texorpdfstring{\hyperref[conditions]{Conditions}}{Conditions}}\label{conditions}

\begin{quote}
See
\href{https://typst.app/docs/reference/scripting/\#conditionals}{official
documentation} .
\end{quote}

In Typst, you can use \texttt{\ }{\texttt{\ if-else\ }}\texttt{\ }
statements. This is especially useful inside function bodies to vary
behavior depending on arguments types or many other things.

\begin{verbatim}
#if 1 < 2 [
  This is shown
] else [
  This is not.
]
\end{verbatim}

\pandocbounded{\includesvg[keepaspectratio]{typst-img/2e914defa3353d6fd42ed58c37a97aedcc2237cfe20228f0cc0d223dfff4619a-1.svg}}

Of course, \texttt{\ }{\texttt{\ else\ }}\texttt{\ } is unnecessary:

\begin{verbatim}
#let a = 3

#if a < 4 {
  a = 5
}

#a
\end{verbatim}

\pandocbounded{\includesvg[keepaspectratio]{typst-img/a7264774be154606a44d829d31edae18bf686262ccea66de9ed97fa20c720bd8-1.svg}}

You can also use \texttt{\ }{\texttt{\ else\ if\ }}\texttt{\ } statement
(known as \texttt{\ }{\texttt{\ elif\ }}\texttt{\ } in Python):

\begin{verbatim}
#let a = 5

#if a < 4 {
  a = 5
} else if a < 6 {
  a = -3
}

#a
\end{verbatim}

\pandocbounded{\includesvg[keepaspectratio]{typst-img/9f65678fc26af2d197d979e1b0a5295ed64037ee00c30fa28c9c417a2c7dc308-1.svg}}

\subsubsection{\texorpdfstring{\hyperref[booleans]{Booleans}}{Booleans}}\label{booleans}

\texttt{\ }{\texttt{\ if,\ else\ if,\ else\ }}\texttt{\ } accept
\emph{only boolean} values as a switch. You can combine booleans as
described in \href{./types.html\#boolean-bool}{types section} :

\begin{verbatim}
#let a = 5

#if (a > 1 and a <= 4) or a == 5 [
    `a` matches the condition
]
\end{verbatim}

\pandocbounded{\includesvg[keepaspectratio]{typst-img/21d3a48404d4e0c59bc0fccb114fdeac7384189db0020247796f44b0e9a7c362-1.svg}}

\subsection{\texorpdfstring{\hyperref[loops]{Loops}}{Loops}}\label{loops}

\begin{quote}
See \href{https://typst.app/docs/reference/scripting/\#loops}{official
documentation} .
\end{quote}

There are two kinds of loops: \texttt{\ }{\texttt{\ while\ }}\texttt{\ }
and \texttt{\ }{\texttt{\ for\ }}\texttt{\ } . While repeats body while
the condition is met:

\begin{verbatim}
#let a = 3

#while a < 100 {
    a *= 2
    str(a)
    " "
}
\end{verbatim}

\pandocbounded{\includesvg[keepaspectratio]{typst-img/ece06c012663616cac05b0f365bd02ea5607dcddfaa0249963088ceff797c100-1.svg}}

\texttt{\ }{\texttt{\ for\ }}\texttt{\ } iterates over all elements of
sequence. The sequence may be an
\texttt{\ }{\texttt{\ array\ }}\texttt{\ } ,
\texttt{\ }{\texttt{\ string\ }}\texttt{\ } or
\texttt{\ }{\texttt{\ dictionary\ }}\texttt{\ } (
\texttt{\ }{\texttt{\ for\ }}\texttt{\ } iterates over its
\emph{key-value pairs} ).

\begin{verbatim}
#for c in "ABC" [
  #c is a letter.
]
\end{verbatim}

\pandocbounded{\includesvg[keepaspectratio]{typst-img/9e70091e4c1f276d548f8200329298bf6b98946c331ca4630fec8313d5a91eff-1.svg}}

To iterate to all numbers from \texttt{\ }{\texttt{\ a\ }}\texttt{\ } to
\texttt{\ }{\texttt{\ b\ }}\texttt{\ } , use
\texttt{\ }{\texttt{\ range(a,\ b+1)\ }}\texttt{\ } :

\begin{verbatim}
#let s = 0

#for i in range(3, 6) {
    s += i
    [Number #i is added to sum. Now sum is #s.]
}
\end{verbatim}

\pandocbounded{\includesvg[keepaspectratio]{typst-img/1e3d95ee79d7bc6989e40ff1e27c0ef6e3b152a1e5f8a0df5b2819621e0e299f-1.svg}}

Because range is end-exclusive this is equal to

\begin{verbatim}
#let s = 0

#for i in (3, 4, 5) {
    s += i
    [Number #i is added to sum. Now sum is #s.]
}
\end{verbatim}

\pandocbounded{\includesvg[keepaspectratio]{typst-img/6158d29261339f8f285d592deff8992ca129ce32264abcdcf6734ac44cf558a4-1.svg}}

\begin{verbatim}
#let people = (Alice: 3, Bob: 5)

#for (name, value) in people [
    #name has #value apples.
]
\end{verbatim}

\pandocbounded{\includesvg[keepaspectratio]{typst-img/50ff0963afe8c9ec5a0562d518431b63d5dd3810525f55f084f812452b11eb21-1.svg}}

\subsubsection{\texorpdfstring{\hyperref[break-and-continue]{Break and
continue}}{Break and continue}}\label{break-and-continue}

Inside loops can be used \texttt{\ }{\texttt{\ break\ }}\texttt{\ } and
\texttt{\ }{\texttt{\ continue\ }}\texttt{\ } commands.
\texttt{\ }{\texttt{\ break\ }}\texttt{\ } breaks loop, jumping outside.
\texttt{\ }{\texttt{\ continue\ }}\texttt{\ } jumps to next loop
iteration.

See the difference on these examples:

\begin{verbatim}
#for letter in "abc nope" {
  if letter == " " {
    // stop when there is space
    break
  }

  letter
}
\end{verbatim}

\pandocbounded{\includesvg[keepaspectratio]{typst-img/a744551cab635d3ab70d9bf4258bb5fc26fe384f8e9f487ad0b8eee986ffe581-1.svg}}

\begin{verbatim}
#for letter in "abc nope" {
  if letter == " " {
    // skip the space
    continue
  }

  letter
}
\end{verbatim}

\pandocbounded{\includesvg[keepaspectratio]{typst-img/bbb719820f986e52fbf64306536766ecbfd7264d29429a5c62d1bd648a4754c5-1.svg}}


\title{sitandr.github.io/typst-examples-book/book/basics/scripting/basics}

\section{\texorpdfstring{\hyperref[basics]{Basics}}{Basics}}\label{basics}

\subsection{\texorpdfstring{\hyperref[variables-i]{Variables
I}}{Variables I}}\label{variables-i}

Let\textquotesingle s start with \emph{variables} .

The concept is very simple, just some value you can reuse:

\begin{verbatim}
#let author = "John Doe"

This is a book by #author. #author is a great guy.

#quote(block: true, attribution: author)[
  \<Some quote\>
]
\end{verbatim}

\pandocbounded{\includesvg[keepaspectratio]{typst-img/c311c1612cafa802f16f0d4ca2d6f1ecca59f545ed1f6ee99d3c4ae06ee2bff4-1.svg}}

\subsection{\texorpdfstring{\hyperref[variables-ii]{Variables
II}}{Variables II}}\label{variables-ii}

You can store \emph{any} Typst value in variable:

\begin{verbatim}
#let block_text = block(stroke: red, inset: 1em)[Text]

#block_text

#figure(caption: "The block", block_text)
\end{verbatim}

\pandocbounded{\includesvg[keepaspectratio]{typst-img/c6290389652d1771d5149c9393af8eb32bd37e4b2bfb2c11764f9f22c294f84b-1.svg}}

\subsection{\texorpdfstring{\hyperref[functions]{Functions}}{Functions}}\label{functions}

We have already seen some "custom" functions in
\href{../tutorial/advanced_styling.html}{Advanced Styling} chapter.

Functions are values that take some values and output some values:

\begin{verbatim}
// This is a syntax that we have seen earlier
#let f = (name) => "Hello, " + name

#f("world!")
\end{verbatim}

\pandocbounded{\includesvg[keepaspectratio]{typst-img/23fba8e9081a8b32b16d7deb54018bb73a8ac910adbfb1a0ca577eb3520a73b4-1.svg}}

\subsubsection{\texorpdfstring{\hyperref[alternative-syntax]{Alternative
syntax}}{Alternative syntax}}\label{alternative-syntax}

You can write the same shorter:

\begin{verbatim}
// The following syntaxes are equivalent
#let f = (name) => "Hello, " + name
#let f(name) = "Hello, " + name

#f("world!")
\end{verbatim}

\pandocbounded{\includesvg[keepaspectratio]{typst-img/e6e4bd179a38f1b3af96f3e7c6308be6f9494f41f43daa26ebabf7a77fc54780-1.svg}}


\title{sitandr.github.io/typst-examples-book/book/basics/scripting/braces}

\section{\texorpdfstring{\hyperref[braces-brackets-and-default]{Braces,
brackets and
default}}{Braces, brackets and default}}\label{braces-brackets-and-default}

\subsection{\texorpdfstring{\hyperref[square-brackets]{Square
brackets}}{Square brackets}}\label{square-brackets}

You may remember that square brackets convert everything inside to
\emph{content} .

\begin{verbatim}
#let v = [Some text, _markup_ and other #strong[functions]]
#v
\end{verbatim}

\pandocbounded{\includesvg[keepaspectratio]{typst-img/5ba617daa8d4c166d96a0abbba02d6502fe7fde1ded460afa78682993295142d-1.svg}}

We may use same for functions bodies:

\begin{verbatim}
#let f(name) = [Hello, #name]
#f[World] // also don't forget we can use it to pass content!
\end{verbatim}

\pandocbounded{\includesvg[keepaspectratio]{typst-img/4545deeee45655ee6666feb4773416cd075fe7522cbfd80d0847c615c6c5f30a-1.svg}}

\textbf{Important:} It is very hard to convert \emph{content} to
\emph{plain text} , as \emph{content} may contain \emph{anything} ! Sp
be careful when passing and storing content in variables.

\subsection{\texorpdfstring{\hyperref[braces]{Braces}}{Braces}}\label{braces}

However, we often want to use code inside functions.
That\textquotesingle s when we use
\texttt{\ }{\texttt{\ \{\}\ }}\texttt{\ } :

\begin{verbatim}
#let f(name) = {
  // this is code mode

  // First part of our output
  "Hello, "

  // we check if name is empty, and if it is,
  // insert placeholder
  if name == "" {
      "anonym"
  } else {
      name
  }

  // finish sentence
  "!"
}

#f("")
#f("Joe")
#f("world")
\end{verbatim}

\pandocbounded{\includesvg[keepaspectratio]{typst-img/f2bc6aebef06f213c9a8e740266a96e424318d953c09cffba6c5811375e91395-1.svg}}

\subsection{\texorpdfstring{\hyperref[scopes]{Scopes}}{Scopes}}\label{scopes}

\textbf{This is a very important thing to remember} .

\emph{You can\textquotesingle t use variables outside of scopes they are
defined (unless it is file root, then you can import them)} . \emph{Set
and show rules affect things in their scope only.}

\begin{verbatim}
#{
  let a = 3;
}
// can't use "a" there.

#[
  #show "true": "false"

  This is true.
]

This is true.
\end{verbatim}

\pandocbounded{\includesvg[keepaspectratio]{typst-img/c25d356831eeea19bb243b87c0f32d062c7086a55b4ee432e41b388d626f875b-1.svg}}

\subsection{\texorpdfstring{\hyperref[return]{Return}}{Return}}\label{return}

\textbf{Important} : by default braces return anything that "returns"
into them. For example,

\begin{verbatim}
#let change_world() = {
  // some code there changing everything in the world
  str(4e7)
  // another code changing the world
}

#let g() = {
  "Hahaha, I will change the world now! "
  change_world()
  " So here is my long evil monologue..."
}

#g()
\end{verbatim}

\pandocbounded{\includesvg[keepaspectratio]{typst-img/160d9672bd7abc64ea61943d1bfcbd1b06dc70f87be5e5cf9c411fe4ee6d2a44-1.svg}}

To avoid returning everything, return only what you want explicitly,
otherwise everything will be joined:

\begin{verbatim}
#let f() = {
  "Some long text"
  // Crazy numbers
  "2e7"
  return none
}

// Returns nothing
#f()
\end{verbatim}

\pandocbounded{\includesvg[keepaspectratio]{typst-img/14c935733a8c91165ee4ebe8246efb841207feeaa0309e36a1cde2888acffb10-1.svg}}

\subsection{\texorpdfstring{\hyperref[default-values]{Default
values}}{Default values}}\label{default-values}

What we made just now was inventing "default values".

They are very common in styling, so there is a special syntax for them:

\begin{verbatim}
#let f(name: "anonym") = [Hello, #name!]

#f()
#f(name: "Joe")
#f(name: "world")
\end{verbatim}

\pandocbounded{\includesvg[keepaspectratio]{typst-img/e9730d0d1f30ec9f2404179560ae4a4b19dd788b1afc2f6b956fb32268439cb6-1.svg}}

You may have noticed that the argument became \emph{named} now. In
Typst, named argument is an argument \emph{that has default value} .


\title{sitandr.github.io/typst-examples-book/book/basics/scripting/tips}

\section{\texorpdfstring{\hyperref[tips]{Tips}}{Tips}}\label{tips}

There are lots of elements in Typst scripting that are not obvious, but
important. All the book is designated to show them, but some of them

\subsection{\texorpdfstring{\hyperref[equality]{Equality}}{Equality}}\label{equality}

Equality doesn\textquotesingle t mean objects are really the same, like
in many other objects:

\begin{verbatim}
#let a = 7
#let b = 7.0
#(a == b)
#(type(a) == type(b))
\end{verbatim}

\pandocbounded{\includesvg[keepaspectratio]{typst-img/3632e0202f7aae6ed6e2958b7bc6360a6cba31aa3d1aaf169a133ef987c839de-1.svg}}

That may be less obvious for dictionaries. In dictionaries \textbf{the
order may matter} , so equality doesn\textquotesingle t mean they behave
exactly the same way:

\begin{verbatim}
#let a = (x: 1, y: 2)
#let b = (y: 2, x: 1)
#(a == b)
#(a.pairs() == b.pairs())
\end{verbatim}

\pandocbounded{\includesvg[keepaspectratio]{typst-img/f7277d7cc170d7cc2ae1de5436b534fb113cda82d8e7829a0fc92e950b78238f-1.svg}}

\subsection{\texorpdfstring{\hyperref[check-key-is-in-dictionary]{Check
key is in
dictionary}}{Check key is in dictionary}}\label{check-key-is-in-dictionary}

Use the keyword \texttt{\ }{\texttt{\ in\ }}\texttt{\ } , like in
\texttt{\ }{\texttt{\ Python\ }}\texttt{\ } :

\begin{verbatim}
#let dict = (a: 1, b: 2)

#("a" in dict)
// gives the same as
#(dict.keys().contains("a"))
\end{verbatim}

\pandocbounded{\includesvg[keepaspectratio]{typst-img/c4ae77418e54911af371f203d2bd3d5badb7269496bb8f07a2e3010e15f18922-1.svg}}

Note it works for lists too:

\begin{verbatim}
#("a" in ("b", "c", "a"))
#(("b", "c", "a").contains("a"))
\end{verbatim}

\pandocbounded{\includesvg[keepaspectratio]{typst-img/0fc3ff7d44bbb5bcacd38e921f199699d2ea43ce0a142e79f67314d4f24386a7-1.svg}}


\title{sitandr.github.io/typst-examples-book/book/basics/scripting/types}

\section{\texorpdfstring{\hyperref[types-part-i]{Types, part
I}}{Types, part I}}\label{types-part-i}

Each value in Typst has a type. You don\textquotesingle t have to
specify it, but it is important.

\subsection{\texorpdfstring{\hyperref[content-content]{Content (
\texttt{\ }{\texttt{\ content\ }}\texttt{\ }
)}}{Content (   content   )}}\label{content-content}

\begin{quote}
\href{https://typst.app/docs/reference/foundations/content/}{Link to
Reference} .
\end{quote}

We have already seen it. A type that represents what is displayed in
document.

\begin{verbatim}
#let c = [It is _content_!]

// Check type of c
#(type(c) == content)

#c

// repr gives an "inner representation" of value
#repr(c)
\end{verbatim}

\pandocbounded{\includesvg[keepaspectratio]{typst-img/21fd80460de8e8a377a9ef2046a27232ad88924070509ccf8647c9135c9c2fe3-1.svg}}

\textbf{Important:} It is very hard to convert \emph{content} to
\emph{plain text} , as \emph{content} may contain \emph{anything} ! So
be careful when passing and storing content in variables.

\subsection{\texorpdfstring{\hyperref[none-none]{None (
\texttt{\ }{\texttt{\ none\ }}\texttt{\ }
)}}{None (   none   )}}\label{none-none}

Nothing. Also known as \texttt{\ }{\texttt{\ null\ }}\texttt{\ } in
other languages. It isn\textquotesingle t displayed, converts to empty
content.

\begin{verbatim}
#none
#repr(none)
\end{verbatim}

\pandocbounded{\includesvg[keepaspectratio]{typst-img/c4100c1d1df8fc0a51bd99945d9bac3c5aa67de19b8f872fd33fd9068bb2507b-1.svg}}

\subsection{\texorpdfstring{\hyperref[string-str]{String (
\texttt{\ }{\texttt{\ str\ }}\texttt{\ }
)}}{String (   str   )}}\label{string-str}

\begin{quote}
\href{https://typst.app/docs/reference/foundations/str/}{Link to
Reference} .
\end{quote}

String contains only plain text and no formatting. Just some chars. That
allows us to work with chars:

\begin{verbatim}
#let s = "Some large string. There could be escape sentences: \n,
 line breaks, and even unicode codes: \u{1251}"
#s \
#type(s) \
`repr`: #repr(s)

#let s = "another small string"
#s.replace("a", sym.alpha) \
#s.split(" ") // split by space
\end{verbatim}

\pandocbounded{\includesvg[keepaspectratio]{typst-img/b797f9c4a540fcf1429bec801d0b334e7d88dc9ccd10e3b7b859f451e269f30f-1.svg}}

You can convert other types to their string representation using this
type\textquotesingle s constructor (e.g. convert number to string):

\begin{verbatim}
#str(5) // string, can be worked with as string
\end{verbatim}

\pandocbounded{\includesvg[keepaspectratio]{typst-img/ab4d4a5d93533525f7f9b2cc8378b79f1561904f3c5d5f6d2ec4bdc448669cb5-1.svg}}

\subsection{\texorpdfstring{\hyperref[boolean-bool]{Boolean (
\texttt{\ }{\texttt{\ bool\ }}\texttt{\ }
)}}{Boolean (   bool   )}}\label{boolean-bool}

\begin{quote}
\href{https://typst.app/docs/reference/foundations/bool/}{Link to
Reference} .
\end{quote}

true/false. Used in \texttt{\ }{\texttt{\ if\ }}\texttt{\ } and many
others

\begin{verbatim}
#let b = false
#b \
#repr(b) \
#(true and not true or true) = #((true and (not true)) or true) \
#if (4 > 3) {
  "4 is more than 3"
}
\end{verbatim}

\pandocbounded{\includesvg[keepaspectratio]{typst-img/e848d78e220ca8cf3b6c323a99d5d963e529aad36857f0e6753c56c02984a616-1.svg}}

\subsection{\texorpdfstring{\hyperref[integer-int]{Integer (
\texttt{\ }{\texttt{\ int\ }}\texttt{\ }
)}}{Integer (   int   )}}\label{integer-int}

\begin{quote}
\href{https://typst.app/docs/reference/foundations/int/}{Link to
Reference} .
\end{quote}

A whole number.

The number can also be specified as hexadecimal, octal, or binary by
starting it with a zero followed by either x, o, or b.

\begin{verbatim}
#let n = 5
#n \
#(n += 1) \
#n \
#calc.pow(2, n) \
#type(n) \
#repr(n)
\end{verbatim}

\pandocbounded{\includesvg[keepaspectratio]{typst-img/6f1c9e02393e14aa23add33d0e6dc2b596ee97a0d425cd3edb3e2b912c6ef6b0-1.svg}}

\begin{verbatim}
#(1 + 2) \
#(2 - 5) \
#(3 + 4 < 8)
\end{verbatim}

\pandocbounded{\includesvg[keepaspectratio]{typst-img/e610f15659cb6b64c3516be48740b54e6caf3d933919004157ba64b757389ba5-1.svg}}

\begin{verbatim}
#0xff \
#0o10 \
#0b1001
\end{verbatim}

\pandocbounded{\includesvg[keepaspectratio]{typst-img/1446dba05ee6f8006884c280ff32e31ede8425d4847445e97cae5dfcde1efe7f-1.svg}}

You can convert a value to an integer with this type\textquotesingle s
constructor (e.g. convert string to int).

\begin{verbatim}
#int(false) \
#int(true) \
#int(2.7) \
#(int("27") + int("4"))
\end{verbatim}

\pandocbounded{\includesvg[keepaspectratio]{typst-img/b44779a87fd984d317ec4d1aed732c0ebdc6220fd4764e407f77fedd139c0d8c-1.svg}}

\subsection{\texorpdfstring{\hyperref[float-float]{Float (
\texttt{\ }{\texttt{\ float\ }}\texttt{\ }
)}}{Float (   float   )}}\label{float-float}

\begin{quote}
\href{https://typst.app/docs/reference/foundations/float/}{Link to
Reference} .
\end{quote}

Works the same way as integer, but can store floating point numbers.
However, precision may be lost.

\begin{verbatim}
#let n = 5.0

// You can mix floats and integers, 
// they will be implicitly converted
#(n += 1) \
#calc.pow(2, n) \
#(0.2 + 0.1) \
#type(n) 
\end{verbatim}

\pandocbounded{\includesvg[keepaspectratio]{typst-img/21cafe751ec803dd9598c871b283a29bc3c6b2e302f0f9bd78edc17330b45616-1.svg}}

\begin{verbatim}
#3.14 \
#1e4 \
#(10 / 4)
\end{verbatim}

\pandocbounded{\includesvg[keepaspectratio]{typst-img/05bd400096c1df5a954fda0897f3c1756c9f99f73503d32d992b3222667a45cd-1.svg}}

You can convert a value to a float with this type\textquotesingle s
constructor (e.g. convert string to float).

\begin{verbatim}
#float(40%) \
#float("2.7") \
#float("1e5")
\end{verbatim}

\pandocbounded{\includesvg[keepaspectratio]{typst-img/f50a22cbea42fded97ab8340f0939e786e5c1cdb5ea531cd4b35b1f732947b7f-1.svg}}


