\title{sitandr.github.io/typst-examples-book/book/packages}

\section{\texorpdfstring{\hyperref[packages]{Packages}}{Packages}}\label{packages}

Once the \href{https://typst.app/universe}{Typst Universe} was launched,
this chapter has become almost redundant. The Universe is actually a
very cool place to look for packages.

However, there are still some cool examples of interesting package
usage.

\subsection{\texorpdfstring{\hyperref[general]{General}}{General}}\label{general}

Typst has packages, but, unlike LaTeX, you need to remember:

\begin{itemize}
\tightlist
\item
  You need them only for some specialized tasks, basic formatting
  \emph{can be totally done without them} .
\item
  Packages are much lighter and much easier "installed" than LaTeX ones.
\item
  Packages are just plain Typst files (and sometimes plugins), so you
  can easily write your own!
\end{itemize}

To use mighty package, just write, like this:

\begin{verbatim}
#import "@preview/cetz:0.1.2": canvas, plot

#canvas(length: 1cm, {
  plot.plot(size: (8, 6),
    x-tick-step: none,
    x-ticks: ((-calc.pi, $-pi$), (0, $0$), (calc.pi, $pi$)),
    y-tick-step: 1,
    {
      plot.add(
        style: plot.palette.blue,
        domain: (-calc.pi, calc.pi), x => calc.sin(x * 1rad))
      plot.add(
        hypograph: true,
        style: plot.palette.blue,
        domain: (-calc.pi, calc.pi), x => calc.cos(x * 1rad))
      plot.add(
        hypograph: true,
        style: plot.palette.blue,
        domain: (-calc.pi, calc.pi), x => calc.cos((x + calc.pi) * 1rad))
    })
})
\end{verbatim}

\pandocbounded{\includesvg[keepaspectratio]{typst-img/29d7015ed96122fa3fb663929c1ac58d25340995423c82456ab8815811373979-1.svg}}

\subsection{\texorpdfstring{\hyperref[contributing]{Contributing}}{Contributing}}\label{contributing}

If you are author of a package or just want to make a fair overview,
feel free to make issues/PR-s!


\title{sitandr.github.io/typst-examples-book/book/about}

\section{\texorpdfstring{\hyperref[typst-examples-book]{Typst Examples
Book}}{Typst Examples Book}}\label{typst-examples-book}

This book provides an extended \emph{tutorial} and lots of
\href{https://github.com/typst/typst}{Typst} snippets that can help you
to write better Typst code.

This is an unofficial book. Some snippets \& suggestions here may be
outdated or useless (please let me know if you find some).

However, \emph{all of them should compile on last version of Typst
\textsuperscript{\hyperref[1]{1}}} .

\textbf{CAUTION:} the book is (probably forever) a \textbf{WIP} , so
don\textquotesingle t rely on it.

If you like it, consider
\href{https://github.com/sitandr/typst-examples-book}{giving a star on
GitHub} !

This will help me to stay motivated and continue working on this book.

\subsection{\texorpdfstring{\hyperref[navigation]{Navigation}}{Navigation}}\label{navigation}

The book consists of several chapters, each with its own goal:

\begin{enumerate}
\tightlist
\item
  \href{./basics/index.html}{Typst Basics}
\item
  \href{./snippets/index.html}{Typst Snippets}
\item
  \href{./packages/index.html}{Typst Packages}
\item
  \href{./typstonomicon/index.html}{Typstonomicon}
\end{enumerate}

\subsection{\texorpdfstring{\hyperref[contributions]{Contributions}}{Contributions}}\label{contributions}

Any contributions are very welcome! If you have a good code snippet that
you want to share, feel free to submit an issue with snippet or make a
PR in the
\href{https://github.com/sitandr/typst-examples-book}{repository} .

I will especially appreciate submissions of active community members and
compiler contributors!

However, I will also really appreciate feedback from beginners to make
the book as comprehensible as possible!

\subsection{\texorpdfstring{\hyperref[acknowledgements]{Acknowledgements}}{Acknowledgements}}\label{acknowledgements}

Thanks to everyone in the community who published their code snippets!

If someone doesn\textquotesingle t like their code and/or name being
published, please contact me.

\phantomsection\label{1}
\textsuperscript{1}

When a new version launches, it may take some time to update the book,
feel free to tag me to speed up the process.


