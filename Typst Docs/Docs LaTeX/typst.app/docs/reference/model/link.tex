\title{typst.app/docs/reference/model/link}

\begin{itemize}
\tightlist
\item
  \href{/docs}{\includesvg[width=0.16667in,height=0.16667in]{/assets/icons/16-docs-dark.svg}}
\item
  \includesvg[width=0.16667in,height=0.16667in]{/assets/icons/16-arrow-right.svg}
\item
  \href{/docs/reference/}{Reference}
\item
  \includesvg[width=0.16667in,height=0.16667in]{/assets/icons/16-arrow-right.svg}
\item
  \href{/docs/reference/model/}{Model}
\item
  \includesvg[width=0.16667in,height=0.16667in]{/assets/icons/16-arrow-right.svg}
\item
  \href{/docs/reference/model/link/}{Link}
\end{itemize}

\section{\texorpdfstring{\texttt{\ link\ } {{ Element
}}}{ link   Element }}\label{summary}

\phantomsection\label{element-tooltip}
Element functions can be customized with \texttt{\ set\ } and
\texttt{\ show\ } rules.

Links to a URL or a location in the document.

By default, links are not styled any different from normal text.
However, you can easily apply a style of your choice with a show rule.

\subsection{Example}\label{example}

\begin{verbatim}
#show link: underline

https://example.com \

#link("https://example.com") \
#link("https://example.com")[
  See example.com
]
\end{verbatim}

\includegraphics[width=5in,height=\textheight,keepaspectratio]{/assets/docs/mBfQJYO4ObjIyuLi_FjKfgAAAAAAAAAA.png}

\subsection{Syntax}\label{syntax}

This function also has dedicated syntax: Text that starts with
\texttt{\ http://\ } or \texttt{\ https://\ } is automatically turned
into a link.

\subsection{\texorpdfstring{{ Parameters
}}{ Parameters }}\label{parameters}

\phantomsection\label{parameters-tooltip}
Parameters are the inputs to a function. They are specified in
parentheses after the function name.

{ link } (

{ \href{/docs/reference/foundations/str/}{str}
\href{/docs/reference/foundations/label/}{label}
\href{/docs/reference/introspection/location/}{location}
\href{/docs/reference/foundations/dictionary/}{dictionary} , } {
\href{/docs/reference/foundations/content/}{content} , }

) -\textgreater{} \href{/docs/reference/foundations/content/}{content}

\subsubsection{\texorpdfstring{\texttt{\ dest\ }}{ dest }}\label{parameters-dest}

\href{/docs/reference/foundations/str/}{str} {or}
\href{/docs/reference/foundations/label/}{label} {or}
\href{/docs/reference/introspection/location/}{location} {or}
\href{/docs/reference/foundations/dictionary/}{dictionary}

{Required} {{ Positional }}

\phantomsection\label{parameters-dest-positional-tooltip}
Positional parameters are specified in order, without names.

The destination the link points to.

\begin{itemize}
\item
  To link to web pages, \texttt{\ dest\ } should be a valid URL string.
  If the URL is in the \texttt{\ mailto:\ } or \texttt{\ tel:\ } scheme
  and the \texttt{\ body\ } parameter is omitted, the email address or
  phone number will be the link\textquotesingle s body, without the
  scheme.
\item
  To link to another part of the document, \texttt{\ dest\ } can take
  one of three forms:

  \begin{itemize}
  \item
    A \href{/docs/reference/foundations/label/}{label} attached to an
    element. If you also want automatic text for the link based on the
    element, consider using a
    \href{/docs/reference/model/ref/}{reference} instead.
  \item
    A
    \href{/docs/reference/introspection/location/}{\texttt{\ location\ }}
    (typically retrieved from
    \href{/docs/reference/introspection/here/}{\texttt{\ here\ }} ,
    \href{/docs/reference/introspection/locate/}{\texttt{\ locate\ }} or
    \href{/docs/reference/introspection/query/}{\texttt{\ query\ }} ).
  \item
    A dictionary with a \texttt{\ page\ } key of type
    \href{/docs/reference/foundations/int/}{integer} and \texttt{\ x\ }
    and \texttt{\ y\ } coordinates of type
    \href{/docs/reference/layout/length/}{length} . Pages are counted
    from one, and the coordinates are relative to the
    page\textquotesingle s top left corner.
  \end{itemize}
\end{itemize}

\includesvg[width=0.16667in,height=0.16667in]{/assets/icons/16-arrow-right.svg}
View example

\begin{verbatim}
= Introduction <intro>
#link("mailto:hello@typst.app") \
#link(<intro>)[Go to intro] \
#link((page: 1, x: 0pt, y: 0pt))[
  Go to top
]
\end{verbatim}

\includegraphics[width=5in,height=\textheight,keepaspectratio]{/assets/docs/r-LwcI2C1K4OtUWhtvg8QgAAAAAAAAAA.png}

\subsubsection{\texorpdfstring{\texttt{\ body\ }}{ body }}\label{parameters-body}

\href{/docs/reference/foundations/content/}{content}

{Required} {{ Positional }}

\phantomsection\label{parameters-body-positional-tooltip}
Positional parameters are specified in order, without names.

The content that should become a link.

If \texttt{\ dest\ } is an URL string, the parameter can be omitted. In
this case, the URL will be shown as the link.

\href{/docs/reference/model/heading/}{\pandocbounded{\includesvg[keepaspectratio]{/assets/icons/16-arrow-right.svg}}}

{ Heading } { Previous page }

\href{/docs/reference/model/enum/}{\pandocbounded{\includesvg[keepaspectratio]{/assets/icons/16-arrow-right.svg}}}

{ Numbered List } { Next page }
