\title{typst.app/docs/reference/model/numbering}

\begin{itemize}
\tightlist
\item
  \href{/docs}{\includesvg[width=0.16667in,height=0.16667in]{/assets/icons/16-docs-dark.svg}}
\item
  \includesvg[width=0.16667in,height=0.16667in]{/assets/icons/16-arrow-right.svg}
\item
  \href{/docs/reference/}{Reference}
\item
  \includesvg[width=0.16667in,height=0.16667in]{/assets/icons/16-arrow-right.svg}
\item
  \href{/docs/reference/model/}{Model}
\item
  \includesvg[width=0.16667in,height=0.16667in]{/assets/icons/16-arrow-right.svg}
\item
  \href{/docs/reference/model/numbering/}{Numbering}
\end{itemize}

\section{\texorpdfstring{\texttt{\ numbering\ }}{ numbering }}\label{summary}

Applies a numbering to a sequence of numbers.

A numbering defines how a sequence of numbers should be displayed as
content. It is defined either through a pattern string or an arbitrary
function.

A numbering pattern consists of counting symbols, for which the actual
number is substituted, their prefixes, and one suffix. The prefixes and
the suffix are repeated as-is.

\subsection{Example}\label{example}

\begin{verbatim}
#numbering("1.1)", 1, 2, 3) \
#numbering("1.a.i", 1, 2) \
#numbering("I – 1", 12, 2) \
#numbering(
  (..nums) => nums
    .pos()
    .map(str)
    .join(".") + ")",
  1, 2, 3,
)
\end{verbatim}

\includegraphics[width=5in,height=\textheight,keepaspectratio]{/assets/docs/ViM4jxlRNjTCcZLHAqTQsQAAAAAAAAAA.png}

\subsection{Numbering patterns and numbering
functions}\label{numbering-patterns-and-numbering-functions}

There are multiple instances where you can provide a numbering pattern
or function in Typst. For example, when defining how to number
\href{/docs/reference/model/heading/}{headings} or
\href{/docs/reference/model/figure/}{figures} . Every time, the expected
format is the same as the one described below for the
\href{/docs/reference/model/numbering/\#parameters-numbering}{\texttt{\ numbering\ }}
parameter.

The following example illustrates that a numbering function is just a
regular \href{/docs/reference/foundations/function/}{function} that
accepts numbers and returns
\href{/docs/reference/foundations/content/}{\texttt{\ content\ }} .

\begin{verbatim}
#let unary(.., last) = "|" * last
#set heading(numbering: unary)
= First heading
= Second heading
= Third heading
\end{verbatim}

\includegraphics[width=5in,height=\textheight,keepaspectratio]{/assets/docs/y3Y2xT6PKYJ3nJF6y9bcPwAAAAAAAAAA.png}

\subsection{\texorpdfstring{{ Parameters
}}{ Parameters }}\label{parameters}

\phantomsection\label{parameters-tooltip}
Parameters are the inputs to a function. They are specified in
parentheses after the function name.

{ numbering } (

{ \href{/docs/reference/foundations/str/}{str}
\href{/docs/reference/foundations/function/}{function} , } {
\hyperref[parameters-numbers]{..}
\href{/docs/reference/foundations/int/}{int} , }

) -\textgreater{} { any }

\subsubsection{\texorpdfstring{\texttt{\ numbering\ }}{ numbering }}\label{parameters-numbering}

\href{/docs/reference/foundations/str/}{str} {or}
\href{/docs/reference/foundations/function/}{function}

{Required} {{ Positional }}

\phantomsection\label{parameters-numbering-positional-tooltip}
Positional parameters are specified in order, without names.

Defines how the numbering works.

\textbf{Counting symbols} are \texttt{\ 1\ } , \texttt{\ a\ } ,
\texttt{\ A\ } , \texttt{\ i\ } , \texttt{\ I\ } , \texttt{\ 一\ } ,
\texttt{\ 壹\ } , \texttt{\ �\ } , \texttt{\ �\ } ,
\texttt{\ ã‚¢\ } , \texttt{\ イ\ } , \texttt{\ ×?\ } , \texttt{\ ê°€\ }
, \texttt{\ ㄱ\ } , \texttt{\ *\ } , \texttt{\ â‘\ } , and
\texttt{\ ⓵\ } . They are replaced by the number in the sequence,
preserving the original case.

The \texttt{\ *\ } character means that symbols should be used to count,
in the order of \texttt{\ *\ } , \texttt{\ â€\ } , \texttt{\ ‡\ } ,
\texttt{\ §\ } , \texttt{\ ¶\ } , \texttt{\ ‖\ } . If there are more
than six items, the number is represented using repeated symbols.

\textbf{Suffixes} are all characters after the last counting symbol.
They are repeated as-is at the end of any rendered number.

\textbf{Prefixes} are all characters that are neither counting symbols
nor suffixes. They are repeated as-is at in front of their rendered
equivalent of their counting symbol.

This parameter can also be an arbitrary function that gets each number
as an individual argument. When given a function, the
\texttt{\ numbering\ } function just forwards the arguments to that
function. While this is not particularly useful in itself, it means that
you can just give arbitrary numberings to the \texttt{\ numbering\ }
function without caring whether they are defined as a pattern or
function.

\subsubsection{\texorpdfstring{\texttt{\ numbers\ }}{ numbers }}\label{parameters-numbers}

\href{/docs/reference/foundations/int/}{int}

{Required} {{ Positional }}

\phantomsection\label{parameters-numbers-positional-tooltip}
Positional parameters are specified in order, without names.

{{ Variadic }}

\phantomsection\label{parameters-numbers-variadic-tooltip}
Variadic parameters can be specified multiple times.

The numbers to apply the numbering to. Must be positive.

If \texttt{\ numbering\ } is a pattern and more numbers than counting
symbols are given, the last counting symbol with its prefix is repeated.

\href{/docs/reference/model/enum/}{\pandocbounded{\includesvg[keepaspectratio]{/assets/icons/16-arrow-right.svg}}}

{ Numbered List } { Previous page }

\href{/docs/reference/model/outline/}{\pandocbounded{\includesvg[keepaspectratio]{/assets/icons/16-arrow-right.svg}}}

{ Outline } { Next page }
