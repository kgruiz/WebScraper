\title{typst.app/docs/reference/visualize/image}

\begin{itemize}
\tightlist
\item
  \href{/docs}{\includesvg[width=0.16667in,height=0.16667in]{/assets/icons/16-docs-dark.svg}}
\item
  \includesvg[width=0.16667in,height=0.16667in]{/assets/icons/16-arrow-right.svg}
\item
  \href{/docs/reference/}{Reference}
\item
  \includesvg[width=0.16667in,height=0.16667in]{/assets/icons/16-arrow-right.svg}
\item
  \href{/docs/reference/visualize/}{Visualize}
\item
  \includesvg[width=0.16667in,height=0.16667in]{/assets/icons/16-arrow-right.svg}
\item
  \href{/docs/reference/visualize/image/}{Image}
\end{itemize}

\section{\texorpdfstring{\texttt{\ image\ } {{ Element
}}}{ image   Element }}\label{summary}

\phantomsection\label{element-tooltip}
Element functions can be customized with \texttt{\ set\ } and
\texttt{\ show\ } rules.

A raster or vector graphic.

You can wrap the image in a
\href{/docs/reference/model/figure/}{\texttt{\ figure\ }} to give it a
number and caption.

Like most elements, images are \emph{block-level} by default and thus do
not integrate themselves into adjacent paragraphs. To force an image to
become inline, put it into a
\href{/docs/reference/layout/box/}{\texttt{\ box\ }} .

\subsection{Example}\label{example}

\begin{verbatim}
#figure(
  image("molecular.jpg", width: 80%),
  caption: [
    A step in the molecular testing
    pipeline of our lab.
  ],
)
\end{verbatim}

\includegraphics[width=5in,height=\textheight,keepaspectratio]{/assets/docs/znWnPh4HT5GrpkEcbnfOxAAAAAAAAAAA.png}

\subsection{\texorpdfstring{{ Parameters
}}{ Parameters }}\label{parameters}

\phantomsection\label{parameters-tooltip}
Parameters are the inputs to a function. They are specified in
parentheses after the function name.

{ image } (

{ \href{/docs/reference/foundations/str/}{str} , } {
\hyperref[parameters-format]{format :}
\href{/docs/reference/foundations/auto/}{auto}
\href{/docs/reference/foundations/str/}{str} , } {
\hyperref[parameters-width]{width :}
\href{/docs/reference/foundations/auto/}{auto}
\href{/docs/reference/layout/relative/}{relative} , } {
\hyperref[parameters-height]{height :}
\href{/docs/reference/foundations/auto/}{auto}
\href{/docs/reference/layout/relative/}{relative}
\href{/docs/reference/layout/fraction/}{fraction} , } {
\hyperref[parameters-alt]{alt :}
\href{/docs/reference/foundations/none/}{none}
\href{/docs/reference/foundations/str/}{str} , } {
\hyperref[parameters-fit]{fit :}
\href{/docs/reference/foundations/str/}{str} , }

) -\textgreater{} \href{/docs/reference/foundations/content/}{content}

\subsubsection{\texorpdfstring{\texttt{\ path\ }}{ path }}\label{parameters-path}

\href{/docs/reference/foundations/str/}{str}

{Required} {{ Positional }}

\phantomsection\label{parameters-path-positional-tooltip}
Positional parameters are specified in order, without names.

Path to an image file

For more details, see the \href{/docs/reference/syntax/\#paths}{Paths
section} .

\subsubsection{\texorpdfstring{\texttt{\ format\ }}{ format }}\label{parameters-format}

\href{/docs/reference/foundations/auto/}{auto} {or}
\href{/docs/reference/foundations/str/}{str}

{{ Settable }}

\phantomsection\label{parameters-format-settable-tooltip}
Settable parameters can be customized for all following uses of the
function with a \texttt{\ set\ } rule.

The image\textquotesingle s format. Detected automatically by default.

Supported formats are PNG, JPEG, GIF, and SVG. Using a PDF as an image
is \href{https://github.com/typst/typst/issues/145}{not currently
supported} .

\begin{longtable}[]{@{}ll@{}}
\toprule\noalign{}
Variant & Details \\
\midrule\noalign{}
\endhead
\bottomrule\noalign{}
\endlastfoot
\texttt{\ "\ png\ "\ } & Raster format for illustrations and transparent
graphics. \\
\texttt{\ "\ jpg\ "\ } & Lossy raster format suitable for photos. \\
\texttt{\ "\ gif\ "\ } & Raster format that is typically used for short
animated clips. \\
\texttt{\ "\ svg\ "\ } & The vector graphics format of the web. \\
\end{longtable}

Default: \texttt{\ }{\texttt{\ auto\ }}\texttt{\ }

\subsubsection{\texorpdfstring{\texttt{\ width\ }}{ width }}\label{parameters-width}

\href{/docs/reference/foundations/auto/}{auto} {or}
\href{/docs/reference/layout/relative/}{relative}

{{ Settable }}

\phantomsection\label{parameters-width-settable-tooltip}
Settable parameters can be customized for all following uses of the
function with a \texttt{\ set\ } rule.

The width of the image.

Default: \texttt{\ }{\texttt{\ auto\ }}\texttt{\ }

\subsubsection{\texorpdfstring{\texttt{\ height\ }}{ height }}\label{parameters-height}

\href{/docs/reference/foundations/auto/}{auto} {or}
\href{/docs/reference/layout/relative/}{relative} {or}
\href{/docs/reference/layout/fraction/}{fraction}

{{ Settable }}

\phantomsection\label{parameters-height-settable-tooltip}
Settable parameters can be customized for all following uses of the
function with a \texttt{\ set\ } rule.

The height of the image.

Default: \texttt{\ }{\texttt{\ auto\ }}\texttt{\ }

\subsubsection{\texorpdfstring{\texttt{\ alt\ }}{ alt }}\label{parameters-alt}

\href{/docs/reference/foundations/none/}{none} {or}
\href{/docs/reference/foundations/str/}{str}

{{ Settable }}

\phantomsection\label{parameters-alt-settable-tooltip}
Settable parameters can be customized for all following uses of the
function with a \texttt{\ set\ } rule.

A text describing the image.

Default: \texttt{\ }{\texttt{\ none\ }}\texttt{\ }

\subsubsection{\texorpdfstring{\texttt{\ fit\ }}{ fit }}\label{parameters-fit}

\href{/docs/reference/foundations/str/}{str}

{{ Settable }}

\phantomsection\label{parameters-fit-settable-tooltip}
Settable parameters can be customized for all following uses of the
function with a \texttt{\ set\ } rule.

How the image should adjust itself to a given area (the area is defined
by the \texttt{\ width\ } and \texttt{\ height\ } fields). Note that
\texttt{\ fit\ } doesn\textquotesingle t visually change anything if the
area\textquotesingle s aspect ratio is the same as the
image\textquotesingle s one.

\begin{longtable}[]{@{}ll@{}}
\toprule\noalign{}
Variant & Details \\
\midrule\noalign{}
\endhead
\bottomrule\noalign{}
\endlastfoot
\texttt{\ "\ cover\ "\ } & The image should completely cover the area
(preserves aspect ratio by cropping the image only horizontally or
vertically). This is the default. \\
\texttt{\ "\ contain\ "\ } & The image should be fully contained in the
area (preserves aspect ratio; doesn\textquotesingle t crop the image;
one dimension can be narrower than specified). \\
\texttt{\ "\ stretch\ "\ } & The image should be stretched so that it
exactly fills the area, even if this means that the image will be
distorted (doesn\textquotesingle t preserve aspect ratio and
doesn\textquotesingle t crop the image). \\
\end{longtable}

Default: \texttt{\ }{\texttt{\ "cover"\ }}\texttt{\ }

\includesvg[width=0.16667in,height=0.16667in]{/assets/icons/16-arrow-right.svg}
View example

\begin{verbatim}
#set page(width: 300pt, height: 50pt, margin: 10pt)
#image("tiger.jpg", width: 100%, fit: "cover")
#image("tiger.jpg", width: 100%, fit: "contain")
#image("tiger.jpg", width: 100%, fit: "stretch")
\end{verbatim}

\includegraphics[width=6.25in,height=\textheight,keepaspectratio]{/assets/docs/oZRwamqZZ0p_tV8oioYxxgAAAAAAAAAA.png}
\includegraphics[width=6.25in,height=\textheight,keepaspectratio]{/assets/docs/oZRwamqZZ0p_tV8oioYxxgAAAAAAAAAB.png}
\includegraphics[width=6.25in,height=\textheight,keepaspectratio]{/assets/docs/oZRwamqZZ0p_tV8oioYxxgAAAAAAAAAC.png}

\subsection{\texorpdfstring{{ Definitions
}}{ Definitions }}\label{definitions}

\phantomsection\label{definitions-tooltip}
Functions and types and can have associated definitions. These are
accessed by specifying the function or type, followed by a period, and
then the definition\textquotesingle s name.

\subsubsection{\texorpdfstring{\texttt{\ decode\ }}{ decode }}\label{definitions-decode}

Decode a raster or vector graphic from bytes or a string.

image { . } { decode } (

{ \href{/docs/reference/foundations/str/}{str}
\href{/docs/reference/foundations/bytes/}{bytes} , } {
\hyperref[definitions-decode-parameters-format]{format :}
\href{/docs/reference/foundations/auto/}{auto}
\href{/docs/reference/foundations/str/}{str} , } {
\hyperref[definitions-decode-parameters-width]{width :}
\href{/docs/reference/foundations/auto/}{auto}
\href{/docs/reference/layout/relative/}{relative} , } {
\hyperref[definitions-decode-parameters-height]{height :}
\href{/docs/reference/foundations/auto/}{auto}
\href{/docs/reference/layout/relative/}{relative}
\href{/docs/reference/layout/fraction/}{fraction} , } {
\hyperref[definitions-decode-parameters-alt]{alt :}
\href{/docs/reference/foundations/none/}{none}
\href{/docs/reference/foundations/str/}{str} , } {
\hyperref[definitions-decode-parameters-fit]{fit :}
\href{/docs/reference/foundations/str/}{str} , }

) -\textgreater{} \href{/docs/reference/foundations/content/}{content}

\begin{verbatim}
#let original = read("diagram.svg")
#let changed = original.replace(
  "#2B80FF", // blue
  green.to-hex(),
)

#image.decode(original)
#image.decode(changed)
\end{verbatim}

\includegraphics[width=5in,height=\textheight,keepaspectratio]{/assets/docs/yVFFVjYQ7xibSWu-658yNwAAAAAAAAAA.png}

\paragraph{\texorpdfstring{\texttt{\ data\ }}{ data }}\label{definitions-decode-data}

\href{/docs/reference/foundations/str/}{str} {or}
\href{/docs/reference/foundations/bytes/}{bytes}

{Required} {{ Positional }}

\phantomsection\label{definitions-decode-data-positional-tooltip}
Positional parameters are specified in order, without names.

The data to decode as an image. Can be a string for SVGs.

\paragraph{\texorpdfstring{\texttt{\ format\ }}{ format }}\label{definitions-decode-format}

\href{/docs/reference/foundations/auto/}{auto} {or}
\href{/docs/reference/foundations/str/}{str}

The image\textquotesingle s format. Detected automatically by default.

\begin{longtable}[]{@{}ll@{}}
\toprule\noalign{}
Variant & Details \\
\midrule\noalign{}
\endhead
\bottomrule\noalign{}
\endlastfoot
\texttt{\ "\ png\ "\ } & Raster format for illustrations and transparent
graphics. \\
\texttt{\ "\ jpg\ "\ } & Lossy raster format suitable for photos. \\
\texttt{\ "\ gif\ "\ } & Raster format that is typically used for short
animated clips. \\
\texttt{\ "\ svg\ "\ } & The vector graphics format of the web. \\
\end{longtable}

\paragraph{\texorpdfstring{\texttt{\ width\ }}{ width }}\label{definitions-decode-width}

\href{/docs/reference/foundations/auto/}{auto} {or}
\href{/docs/reference/layout/relative/}{relative}

The width of the image.

\paragraph{\texorpdfstring{\texttt{\ height\ }}{ height }}\label{definitions-decode-height}

\href{/docs/reference/foundations/auto/}{auto} {or}
\href{/docs/reference/layout/relative/}{relative} {or}
\href{/docs/reference/layout/fraction/}{fraction}

The height of the image.

\paragraph{\texorpdfstring{\texttt{\ alt\ }}{ alt }}\label{definitions-decode-alt}

\href{/docs/reference/foundations/none/}{none} {or}
\href{/docs/reference/foundations/str/}{str}

A text describing the image.

\paragraph{\texorpdfstring{\texttt{\ fit\ }}{ fit }}\label{definitions-decode-fit}

\href{/docs/reference/foundations/str/}{str}

How the image should adjust itself to a given area.

\begin{longtable}[]{@{}ll@{}}
\toprule\noalign{}
Variant & Details \\
\midrule\noalign{}
\endhead
\bottomrule\noalign{}
\endlastfoot
\texttt{\ "\ cover\ "\ } & The image should completely cover the area
(preserves aspect ratio by cropping the image only horizontally or
vertically). This is the default. \\
\texttt{\ "\ contain\ "\ } & The image should be fully contained in the
area (preserves aspect ratio; doesn\textquotesingle t crop the image;
one dimension can be narrower than specified). \\
\texttt{\ "\ stretch\ "\ } & The image should be stretched so that it
exactly fills the area, even if this means that the image will be
distorted (doesn\textquotesingle t preserve aspect ratio and
doesn\textquotesingle t crop the image). \\
\end{longtable}

\href{/docs/reference/visualize/gradient/}{\pandocbounded{\includesvg[keepaspectratio]{/assets/icons/16-arrow-right.svg}}}

{ Gradient } { Previous page }

\href{/docs/reference/visualize/line/}{\pandocbounded{\includesvg[keepaspectratio]{/assets/icons/16-arrow-right.svg}}}

{ Line } { Next page }
