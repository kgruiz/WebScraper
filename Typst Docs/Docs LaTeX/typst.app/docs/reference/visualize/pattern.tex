\title{typst.app/docs/reference/visualize/pattern}

\begin{itemize}
\tightlist
\item
  \href{/docs}{\includesvg[width=0.16667in,height=0.16667in]{/assets/icons/16-docs-dark.svg}}
\item
  \includesvg[width=0.16667in,height=0.16667in]{/assets/icons/16-arrow-right.svg}
\item
  \href{/docs/reference/}{Reference}
\item
  \includesvg[width=0.16667in,height=0.16667in]{/assets/icons/16-arrow-right.svg}
\item
  \href{/docs/reference/visualize/}{Visualize}
\item
  \includesvg[width=0.16667in,height=0.16667in]{/assets/icons/16-arrow-right.svg}
\item
  \href{/docs/reference/visualize/pattern/}{Pattern}
\end{itemize}

\section{\texorpdfstring{{ pattern }}{ pattern }}\label{summary}

A repeating pattern fill.

Typst supports the most common pattern type of tiled patterns, where a
pattern is repeated in a grid-like fashion, covering the entire area of
an element that is filled or stroked. The pattern is defined by a tile
size and a body defining the content of each cell. You can also add
horizontal or vertical spacing between the cells of the pattern.

\subsection{Examples}\label{examples}

\begin{verbatim}
#let pat = pattern(size: (30pt, 30pt))[
  #place(line(start: (0%, 0%), end: (100%, 100%)))
  #place(line(start: (0%, 100%), end: (100%, 0%)))
]

#rect(fill: pat, width: 100%, height: 60pt, stroke: 1pt)
\end{verbatim}

\includegraphics[width=5in,height=\textheight,keepaspectratio]{/assets/docs/coeD6IerbqenB1CPjs7dfAAAAAAAAAAA.png}

Patterns are also supported on text, but only when setting the
\href{/docs/reference/visualize/pattern/\#parameters-relative}{relativeness}
to either \texttt{\ }{\texttt{\ auto\ }}\texttt{\ } (the default value)
or \texttt{\ }{\texttt{\ "parent"\ }}\texttt{\ } . To create
word-by-word or glyph-by-glyph patterns, you can wrap the words or
characters of your text in \href{/docs/reference/layout/box/}{boxes}
manually or through a \href{/docs/reference/styling/\#show-rules}{show
rule} .

\begin{verbatim}
#let pat = pattern(
  size: (30pt, 30pt),
  relative: "parent",
  square(
    size: 30pt,
    fill: gradient
      .conic(..color.map.rainbow),
  )
)

#set text(fill: pat)
#lorem(10)
\end{verbatim}

\includegraphics[width=5in,height=\textheight,keepaspectratio]{/assets/docs/Vk9hYVErruhpSxeZVudFjQAAAAAAAAAA.png}

You can also space the elements further or closer apart using the
\href{/docs/reference/visualize/pattern/\#parameters-spacing}{\texttt{\ spacing\ }}
feature of the pattern. If the spacing is lower than the size of the
pattern, the pattern will overlap. If it is higher, the pattern will
have gaps of the same color as the background of the pattern.

\begin{verbatim}
#let pat = pattern(
  size: (30pt, 30pt),
  spacing: (10pt, 10pt),
  relative: "parent",
  square(
    size: 30pt,
    fill: gradient
     .conic(..color.map.rainbow),
  ),
)

#rect(
  width: 100%,
  height: 60pt,
  fill: pat,
)
\end{verbatim}

\includegraphics[width=5in,height=\textheight,keepaspectratio]{/assets/docs/yPTj9FTOvqrbv-4eK83U7gAAAAAAAAAA.png}

\subsection{Relativeness}\label{relativeness}

The location of the starting point of the pattern is dependent on the
dimensions of a container. This container can either be the shape that
it is being painted on, or the closest surrounding container. This is
controlled by the \texttt{\ relative\ } argument of a pattern
constructor. By default, patterns are relative to the shape they are
being painted on, unless the pattern is applied on text, in which case
they are relative to the closest ancestor container.

Typst determines the ancestor container as follows:

\begin{itemize}
\tightlist
\item
  For shapes that are placed at the root/top level of the document, the
  closest ancestor is the page itself.
\item
  For other shapes, the ancestor is the innermost
  \href{/docs/reference/layout/block/}{\texttt{\ block\ }} or
  \href{/docs/reference/layout/box/}{\texttt{\ box\ }} that contains the
  shape. This includes the boxes and blocks that are implicitly created
  by show rules and elements. For example, a
  \href{/docs/reference/layout/rotate/}{\texttt{\ rotate\ }} will not
  affect the parent of a gradient, but a
  \href{/docs/reference/layout/grid/}{\texttt{\ grid\ }} will.
\end{itemize}

\subsection{\texorpdfstring{Constructor
{}}{Constructor }}\label{constructor}

\phantomsection\label{constructor-constructor-tooltip}
If a type has a constructor, you can call it like a function to create a
new value of the type.

Construct a new pattern.

{ pattern } (

{ \hyperref[constructor-parameters-size]{size :}
\href{/docs/reference/foundations/auto/}{auto}
\href{/docs/reference/foundations/array/}{array} , } {
\hyperref[constructor-parameters-spacing]{spacing :}
\href{/docs/reference/foundations/array/}{array} , } {
\hyperref[constructor-parameters-relative]{relative :}
\href{/docs/reference/foundations/auto/}{auto}
\href{/docs/reference/foundations/str/}{str} , } {
\href{/docs/reference/foundations/content/}{content} , }

) -\textgreater{} \href{/docs/reference/visualize/pattern/}{pattern}

\begin{verbatim}
#let pat = pattern(
  size: (20pt, 20pt),
  relative: "parent",
  place(
    dx: 5pt,
    dy: 5pt,
    rotate(45deg, square(
      size: 5pt,
      fill: black,
    )),
  ),
)

#rect(width: 100%, height: 60pt, fill: pat)
\end{verbatim}

\includegraphics[width=5in,height=\textheight,keepaspectratio]{/assets/docs/s7EOLk1zJeZ_4afTw83qRwAAAAAAAAAA.png}

\paragraph{\texorpdfstring{\texttt{\ size\ }}{ size }}\label{constructor-size}

\href{/docs/reference/foundations/auto/}{auto} {or}
\href{/docs/reference/foundations/array/}{array}

The bounding box of each cell of the pattern.

Default: \texttt{\ }{\texttt{\ auto\ }}\texttt{\ }

\paragraph{\texorpdfstring{\texttt{\ spacing\ }}{ spacing }}\label{constructor-spacing}

\href{/docs/reference/foundations/array/}{array}

The spacing between cells of the pattern.

Default:
\texttt{\ }{\texttt{\ (\ }}\texttt{\ }{\texttt{\ 0pt\ }}\texttt{\ }{\texttt{\ ,\ }}\texttt{\ }{\texttt{\ 0pt\ }}\texttt{\ }{\texttt{\ )\ }}\texttt{\ }

\paragraph{\texorpdfstring{\texttt{\ relative\ }}{ relative }}\label{constructor-relative}

\href{/docs/reference/foundations/auto/}{auto} {or}
\href{/docs/reference/foundations/str/}{str}

The \hyperref[relativeness]{relative placement} of the pattern.

For an element placed at the root/top level of the document, the parent
is the page itself. For other elements, the parent is the innermost
block, box, column, grid, or stack that contains the element.

\begin{longtable}[]{@{}ll@{}}
\toprule\noalign{}
Variant & Details \\
\midrule\noalign{}
\endhead
\bottomrule\noalign{}
\endlastfoot
\texttt{\ "\ self\ "\ } & The gradient is relative to itself (its own
bounding box). \\
\texttt{\ "\ parent\ "\ } & The gradient is relative to its parent (the
parent\textquotesingle s bounding box). \\
\end{longtable}

Default: \texttt{\ }{\texttt{\ auto\ }}\texttt{\ }

\paragraph{\texorpdfstring{\texttt{\ body\ }}{ body }}\label{constructor-body}

\href{/docs/reference/foundations/content/}{content}

{Required} {{ Positional }}

\phantomsection\label{constructor-body-positional-tooltip}
Positional parameters are specified in order, without names.

The content of each cell of the pattern.

\href{/docs/reference/visualize/path/}{\pandocbounded{\includesvg[keepaspectratio]{/assets/icons/16-arrow-right.svg}}}

{ Path } { Previous page }

\href{/docs/reference/visualize/polygon/}{\pandocbounded{\includesvg[keepaspectratio]{/assets/icons/16-arrow-right.svg}}}

{ Polygon } { Next page }
