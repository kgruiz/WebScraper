\title{typst.app/docs/reference/visualize/line}

\begin{itemize}
\tightlist
\item
  \href{/docs}{\includesvg[width=0.16667in,height=0.16667in]{/assets/icons/16-docs-dark.svg}}
\item
  \includesvg[width=0.16667in,height=0.16667in]{/assets/icons/16-arrow-right.svg}
\item
  \href{/docs/reference/}{Reference}
\item
  \includesvg[width=0.16667in,height=0.16667in]{/assets/icons/16-arrow-right.svg}
\item
  \href{/docs/reference/visualize/}{Visualize}
\item
  \includesvg[width=0.16667in,height=0.16667in]{/assets/icons/16-arrow-right.svg}
\item
  \href{/docs/reference/visualize/line/}{Line}
\end{itemize}

\section{\texorpdfstring{\texttt{\ line\ } {{ Element
}}}{ line   Element }}\label{summary}

\phantomsection\label{element-tooltip}
Element functions can be customized with \texttt{\ set\ } and
\texttt{\ show\ } rules.

A line from one point to another.

\subsection{Example}\label{example}

\begin{verbatim}
#set page(height: 100pt)

#line(length: 100%)
#line(end: (50%, 50%))
#line(
  length: 4cm,
  stroke: 2pt + maroon,
)
\end{verbatim}

\includegraphics[width=5in,height=\textheight,keepaspectratio]{/assets/docs/IBdLCKW0h9kNWs6W_8DKAwAAAAAAAAAA.png}

\subsection{\texorpdfstring{{ Parameters
}}{ Parameters }}\label{parameters}

\phantomsection\label{parameters-tooltip}
Parameters are the inputs to a function. They are specified in
parentheses after the function name.

{ line } (

{ \hyperref[parameters-start]{start :}
\href{/docs/reference/foundations/array/}{array} , } {
\hyperref[parameters-end]{end :}
\href{/docs/reference/foundations/none/}{none}
\href{/docs/reference/foundations/array/}{array} , } {
\hyperref[parameters-length]{length :}
\href{/docs/reference/layout/relative/}{relative} , } {
\hyperref[parameters-angle]{angle :}
\href{/docs/reference/layout/angle/}{angle} , } {
\hyperref[parameters-stroke]{stroke :}
\href{/docs/reference/layout/length/}{length}
\href{/docs/reference/visualize/color/}{color}
\href{/docs/reference/visualize/gradient/}{gradient}
\href{/docs/reference/visualize/stroke/}{stroke}
\href{/docs/reference/visualize/pattern/}{pattern}
\href{/docs/reference/foundations/dictionary/}{dictionary} , }

) -\textgreater{} \href{/docs/reference/foundations/content/}{content}

\subsubsection{\texorpdfstring{\texttt{\ start\ }}{ start }}\label{parameters-start}

\href{/docs/reference/foundations/array/}{array}

{{ Settable }}

\phantomsection\label{parameters-start-settable-tooltip}
Settable parameters can be customized for all following uses of the
function with a \texttt{\ set\ } rule.

The start point of the line.

Must be an array of exactly two relative lengths.

Default:
\texttt{\ }{\texttt{\ (\ }}\texttt{\ }{\texttt{\ 0\%\ }}\texttt{\ }{\texttt{\ +\ }}\texttt{\ }{\texttt{\ 0pt\ }}\texttt{\ }{\texttt{\ ,\ }}\texttt{\ }{\texttt{\ 0\%\ }}\texttt{\ }{\texttt{\ +\ }}\texttt{\ }{\texttt{\ 0pt\ }}\texttt{\ }{\texttt{\ )\ }}\texttt{\ }

\subsubsection{\texorpdfstring{\texttt{\ end\ }}{ end }}\label{parameters-end}

\href{/docs/reference/foundations/none/}{none} {or}
\href{/docs/reference/foundations/array/}{array}

{{ Settable }}

\phantomsection\label{parameters-end-settable-tooltip}
Settable parameters can be customized for all following uses of the
function with a \texttt{\ set\ } rule.

The offset from \texttt{\ start\ } where the line ends.

Default: \texttt{\ }{\texttt{\ none\ }}\texttt{\ }

\subsubsection{\texorpdfstring{\texttt{\ length\ }}{ length }}\label{parameters-length}

\href{/docs/reference/layout/relative/}{relative}

{{ Settable }}

\phantomsection\label{parameters-length-settable-tooltip}
Settable parameters can be customized for all following uses of the
function with a \texttt{\ set\ } rule.

The line\textquotesingle s length. This is only respected if
\texttt{\ end\ } is \texttt{\ }{\texttt{\ none\ }}\texttt{\ } .

Default:
\texttt{\ }{\texttt{\ 0\%\ }}\texttt{\ }{\texttt{\ +\ }}\texttt{\ }{\texttt{\ 30pt\ }}\texttt{\ }

\subsubsection{\texorpdfstring{\texttt{\ angle\ }}{ angle }}\label{parameters-angle}

\href{/docs/reference/layout/angle/}{angle}

{{ Settable }}

\phantomsection\label{parameters-angle-settable-tooltip}
Settable parameters can be customized for all following uses of the
function with a \texttt{\ set\ } rule.

The angle at which the line points away from the origin. This is only
respected if \texttt{\ end\ } is
\texttt{\ }{\texttt{\ none\ }}\texttt{\ } .

Default: \texttt{\ }{\texttt{\ 0deg\ }}\texttt{\ }

\subsubsection{\texorpdfstring{\texttt{\ stroke\ }}{ stroke }}\label{parameters-stroke}

\href{/docs/reference/layout/length/}{length} {or}
\href{/docs/reference/visualize/color/}{color} {or}
\href{/docs/reference/visualize/gradient/}{gradient} {or}
\href{/docs/reference/visualize/stroke/}{stroke} {or}
\href{/docs/reference/visualize/pattern/}{pattern} {or}
\href{/docs/reference/foundations/dictionary/}{dictionary}

{{ Settable }}

\phantomsection\label{parameters-stroke-settable-tooltip}
Settable parameters can be customized for all following uses of the
function with a \texttt{\ set\ } rule.

How to \href{/docs/reference/visualize/stroke/}{stroke} the line.

Default:
\texttt{\ }{\texttt{\ 1pt\ }}\texttt{\ }{\texttt{\ +\ }}\texttt{\ black\ }

\includesvg[width=0.16667in,height=0.16667in]{/assets/icons/16-arrow-right.svg}
View example

\begin{verbatim}
#set line(length: 100%)
#stack(
  spacing: 1em,
  line(stroke: 2pt + red),
  line(stroke: (paint: blue, thickness: 4pt, cap: "round")),
  line(stroke: (paint: blue, thickness: 1pt, dash: "dashed")),
  line(stroke: (paint: blue, thickness: 1pt, dash: ("dot", 2pt, 4pt, 2pt))),
)
\end{verbatim}

\includegraphics[width=5in,height=\textheight,keepaspectratio]{/assets/docs/Shwqpl9XrWkg6A1XzBok6AAAAAAAAAAA.png}

\href{/docs/reference/visualize/image/}{\pandocbounded{\includesvg[keepaspectratio]{/assets/icons/16-arrow-right.svg}}}

{ Image } { Previous page }

\href{/docs/reference/visualize/path/}{\pandocbounded{\includesvg[keepaspectratio]{/assets/icons/16-arrow-right.svg}}}

{ Path } { Next page }
