\title{typst.app/docs/reference/foundations/regex}

\begin{itemize}
\tightlist
\item
  \href{/docs}{\includesvg[width=0.16667in,height=0.16667in]{/assets/icons/16-docs-dark.svg}}
\item
  \includesvg[width=0.16667in,height=0.16667in]{/assets/icons/16-arrow-right.svg}
\item
  \href{/docs/reference/}{Reference}
\item
  \includesvg[width=0.16667in,height=0.16667in]{/assets/icons/16-arrow-right.svg}
\item
  \href{/docs/reference/foundations/}{Foundations}
\item
  \includesvg[width=0.16667in,height=0.16667in]{/assets/icons/16-arrow-right.svg}
\item
  \href{/docs/reference/foundations/regex/}{Regex}
\end{itemize}

\section{\texorpdfstring{{ regex }}{ regex }}\label{summary}

A regular expression.

Can be used as a \href{/docs/reference/styling/\#show-rules}{show rule
selector} and with \href{/docs/reference/foundations/str/}{string
methods} like \texttt{\ find\ } , \texttt{\ split\ } , and
\texttt{\ replace\ } .

\href{https://docs.rs/regex/latest/regex/\#syntax}{See here} for a
specification of the supported syntax.

\subsection{Example}\label{example}

\begin{verbatim}
// Works with string methods.
#"a,b;c".split(regex("[,;]"))

// Works with show rules.
#show regex("\d+"): set text(red)

The numbers 1 to 10.
\end{verbatim}

\includegraphics[width=5in,height=\textheight,keepaspectratio]{/assets/docs/UtfXJAklKdjyBZ3HmRwY-AAAAAAAAAAA.png}

\subsection{\texorpdfstring{Constructor
{}}{Constructor }}\label{constructor}

\phantomsection\label{constructor-constructor-tooltip}
If a type has a constructor, you can call it like a function to create a
new value of the type.

Create a regular expression from a string.

{ regex } (

{ \href{/docs/reference/foundations/str/}{str} }

) -\textgreater{} \href{/docs/reference/foundations/regex/}{regex}

\paragraph{\texorpdfstring{\texttt{\ regex\ }}{ regex }}\label{constructor-regex}

\href{/docs/reference/foundations/str/}{str}

{Required} {{ Positional }}

\phantomsection\label{constructor-regex-positional-tooltip}
Positional parameters are specified in order, without names.

The regular expression as a string.

Most regex escape sequences just work because they are not valid Typst
escape sequences. To produce regex escape sequences that are also valid
in Typst (e.g.
\texttt{\ }{\texttt{\ \textbackslash{}\textbackslash{}\ }}\texttt{\ } ),
you need to escape twice. Thus, to match a verbatim backslash, you would
need to write
\texttt{\ }{\texttt{\ regex\ }}\texttt{\ }{\texttt{\ (\ }}\texttt{\ }{\texttt{\ "\textbackslash{}\textbackslash{}\textbackslash{}\textbackslash{}"\ }}\texttt{\ }{\texttt{\ )\ }}\texttt{\ }
.

If you need many escape sequences, you can also create a raw element and
extract its text to use it for your regular expressions:

\includesvg[width=0.16667in,height=0.16667in]{/assets/icons/16-arrow-right.svg}
View example

\texttt{\ }{\texttt{\ regex\ }}\texttt{\ }{\texttt{\ (\ }}\texttt{\ }{\texttt{\ \textasciigrave{}\textbackslash{}d+\textbackslash{}.\textbackslash{}d+\textbackslash{}.\textbackslash{}d+\textasciigrave{}\ }}\texttt{\ }{\texttt{\ .\ }}\texttt{\ text\ }{\texttt{\ )\ }}\texttt{\ }
.

\href{/docs/reference/foundations/plugin/}{\pandocbounded{\includesvg[keepaspectratio]{/assets/icons/16-arrow-right.svg}}}

{ Plugin } { Previous page }

\href{/docs/reference/foundations/repr/}{\pandocbounded{\includesvg[keepaspectratio]{/assets/icons/16-arrow-right.svg}}}

{ Representation } { Next page }
