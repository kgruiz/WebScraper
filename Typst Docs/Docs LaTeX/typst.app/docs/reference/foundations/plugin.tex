\title{typst.app/docs/reference/foundations/plugin}

\begin{itemize}
\tightlist
\item
  \href{/docs}{\includesvg[width=0.16667in,height=0.16667in]{/assets/icons/16-docs-dark.svg}}
\item
  \includesvg[width=0.16667in,height=0.16667in]{/assets/icons/16-arrow-right.svg}
\item
  \href{/docs/reference/}{Reference}
\item
  \includesvg[width=0.16667in,height=0.16667in]{/assets/icons/16-arrow-right.svg}
\item
  \href{/docs/reference/foundations/}{Foundations}
\item
  \includesvg[width=0.16667in,height=0.16667in]{/assets/icons/16-arrow-right.svg}
\item
  \href{/docs/reference/foundations/plugin/}{Plugin}
\end{itemize}

\section{\texorpdfstring{{ plugin }}{ plugin }}\label{summary}

A WebAssembly plugin.

Typst is capable of interfacing with plugins compiled to WebAssembly.
Plugin functions may accept multiple
\href{/docs/reference/foundations/bytes/}{byte buffers} as arguments and
return a single byte buffer. They should typically be wrapped in
idiomatic Typst functions that perform the necessary conversions between
native Typst types and bytes.

Plugins run in isolation from your system, which means that printing,
reading files, or anything like that will not be supported for security
reasons. To run as a plugin, a program needs to be compiled to a 32-bit
shared WebAssembly library. Many compilers will use the
\href{https://wasi.dev/}{WASI ABI} by default or as their only option
(e.g. emscripten), which allows printing, reading files, etc. This ABI
will not directly work with Typst. You will either need to compile to a
different target or
\href{https://github.com/astrale-sharp/wasm-minimal-protocol/blob/master/wasi-stub}{stub
all functions} .

\subsection{Plugins and Packages}\label{plugins-and-packages}

Plugins are distributed as packages. A package can make use of a plugin
simply by including a WebAssembly file and loading it. Because the
byte-based plugin interface is quite low-level, plugins are typically
exposed through wrapper functions, that also live in the same package.

\subsection{Purity}\label{purity}

Plugin functions must be pure: Given the same arguments, they must
always return the same value. The reason for this is that Typst
functions must be pure (which is quite fundamental to the language
design) and, since Typst function can call plugin functions, this
requirement is inherited. In particular, if a plugin function is called
twice with the same arguments, Typst might cache the results and call
your function only once.

\subsection{Example}\label{example}

\begin{verbatim}
#let myplugin = plugin("hello.wasm")
#let concat(a, b) = str(
  myplugin.concatenate(
    bytes(a),
    bytes(b),
  )
)

#concat("hello", "world")
\end{verbatim}

\includegraphics[width=5in,height=\textheight,keepaspectratio]{/assets/docs/Vj65tyYDxxD3OHZUaiQ94QAAAAAAAAAA.png}

\subsection{Protocol}\label{protocol}

To be used as a plugin, a WebAssembly module must conform to the
following protocol:

\subsubsection{Exports}\label{exports}

A plugin module can export functions to make them callable from Typst.
To conform to the protocol, an exported function should:

\begin{itemize}
\item
  Take \texttt{\ n\ } 32-bit integer arguments \texttt{\ a\_1\ } ,
  \texttt{\ a\_2\ } , ..., \texttt{\ a\_n\ } (interpreted as lengths, so
  \texttt{\ usize/size\_t\ } may be preferable), and return one 32-bit
  integer.
\item
  The function should first allocate a buffer \texttt{\ buf\ } of length
  \texttt{\ a\_1\ +\ a\_2\ +\ ...\ +\ a\_n\ } , and then call
  \texttt{\ wasm\_minimal\_protocol\_write\_args\_to\_buffer(buf.ptr)\ }
  .
\item
  The \texttt{\ a\_1\ } first bytes of the buffer now constitute the
  first argument, the \texttt{\ a\_2\ } next bytes the second argument,
  and so on.
\item
  The function can now do its job with the arguments and produce an
  output buffer. Before returning, it should call
  \texttt{\ wasm\_minimal\_protocol\_send\_result\_to\_host\ } to send
  its result back to the host.
\item
  To signal success, the function should return \texttt{\ 0\ } .
\item
  To signal an error, the function should return \texttt{\ 1\ } . The
  written buffer is then interpreted as an UTF-8 encoded error message.
\end{itemize}

\subsubsection{Imports}\label{imports}

Plugin modules need to import two functions that are provided by the
runtime. (Types and functions are described using WAT syntax.)

\begin{itemize}
\item
  \texttt{\ (import\ "typst\_env"\ "wasm\_minimal\_protocol\_write\_args\_to\_buffer"\ (func\ (param\ i32)))\ }

  Writes the arguments for the current function into a plugin-allocated
  buffer. When a plugin function is called, it
  \hyperref[exports]{receives the lengths} of its input buffers as
  arguments. It should then allocate a buffer whose capacity is at least
  the sum of these lengths. It should then call this function with a
  \texttt{\ ptr\ } to the buffer to fill it with the arguments, one
  after another.
\item
  \texttt{\ (import\ "typst\_env"\ "wasm\_minimal\_protocol\_send\_result\_to\_host"\ (func\ (param\ i32\ i32)))\ }

  Sends the output of the current function to the host (Typst). The
  first parameter shall be a pointer to a buffer ( \texttt{\ ptr\ } ),
  while the second is the length of that buffer ( \texttt{\ len\ } ).
  The memory pointed at by \texttt{\ ptr\ } can be freed immediately
  after this function returns. If the message should be interpreted as
  an error message, it should be encoded as UTF-8.
\end{itemize}

\subsection{Resources}\label{resources}

For more resources, check out the
\href{https://github.com/astrale-sharp/wasm-minimal-protocol}{wasm-minimal-protocol
repository} . It contains:

\begin{itemize}
\tightlist
\item
  A list of example plugin implementations and a test runner for these
  examples
\item
  Wrappers to help you write your plugin in Rust (Zig wrapper in
  development)
\item
  A stubber for WASI
\end{itemize}

\subsection{\texorpdfstring{Constructor
{}}{Constructor }}\label{constructor}

\phantomsection\label{constructor-constructor-tooltip}
If a type has a constructor, you can call it like a function to create a
new value of the type.

Creates a new plugin from a WebAssembly file.

{ plugin } (

{ \href{/docs/reference/foundations/str/}{str} }

) -\textgreater{} \href{/docs/reference/foundations/plugin/}{plugin}

\paragraph{\texorpdfstring{\texttt{\ path\ }}{ path }}\label{constructor-path}

\href{/docs/reference/foundations/str/}{str}

{Required} {{ Positional }}

\phantomsection\label{constructor-path-positional-tooltip}
Positional parameters are specified in order, without names.

Path to a WebAssembly file.

For more details, see the \href{/docs/reference/syntax/\#paths}{Paths
section} .

\href{/docs/reference/foundations/panic/}{\pandocbounded{\includesvg[keepaspectratio]{/assets/icons/16-arrow-right.svg}}}

{ Panic } { Previous page }

\href{/docs/reference/foundations/regex/}{\pandocbounded{\includesvg[keepaspectratio]{/assets/icons/16-arrow-right.svg}}}

{ Regex } { Next page }
