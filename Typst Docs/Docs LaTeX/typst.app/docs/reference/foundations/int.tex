\title{typst.app/docs/reference/foundations/int}

\begin{itemize}
\tightlist
\item
  \href{/docs}{\includesvg[width=0.16667in,height=0.16667in]{/assets/icons/16-docs-dark.svg}}
\item
  \includesvg[width=0.16667in,height=0.16667in]{/assets/icons/16-arrow-right.svg}
\item
  \href{/docs/reference/}{Reference}
\item
  \includesvg[width=0.16667in,height=0.16667in]{/assets/icons/16-arrow-right.svg}
\item
  \href{/docs/reference/foundations/}{Foundations}
\item
  \includesvg[width=0.16667in,height=0.16667in]{/assets/icons/16-arrow-right.svg}
\item
  \href{/docs/reference/foundations/int/}{Integer}
\end{itemize}

\section{\texorpdfstring{{ int }}{ int }}\label{summary}

A whole number.

The number can be negative, zero, or positive. As Typst uses 64 bits to
store integers, integers cannot be smaller than
\texttt{\ }{\texttt{\ -\ }}\texttt{\ }{\texttt{\ 9223372036854775808\ }}\texttt{\ }
or larger than \texttt{\ }{\texttt{\ 9223372036854775807\ }}\texttt{\ }
.

The number can also be specified as hexadecimal, octal, or binary by
starting it with a zero followed by either \texttt{\ x\ } ,
\texttt{\ o\ } , or \texttt{\ b\ } .

You can convert a value to an integer with this type\textquotesingle s
constructor.

\subsection{Example}\label{example}

\begin{verbatim}
#(1 + 2) \
#(2 - 5) \
#(3 + 4 < 8)

#0xff \
#0o10 \
#0b1001
\end{verbatim}

\includegraphics[width=5in,height=\textheight,keepaspectratio]{/assets/docs/wfpxRJDZrNeGDA3RjEgFJgAAAAAAAAAA.png}

\subsection{\texorpdfstring{Constructor
{}}{Constructor }}\label{constructor}

\phantomsection\label{constructor-constructor-tooltip}
If a type has a constructor, you can call it like a function to create a
new value of the type.

Converts a value to an integer. Raises an error if there is an attempt
to produce an integer larger than the maximum 64-bit signed integer or
smaller than the minimum 64-bit signed integer.

\begin{itemize}
\tightlist
\item
  Booleans are converted to \texttt{\ 0\ } or \texttt{\ 1\ } .
\item
  Floats and decimals are truncated to the next 64-bit integer.
\item
  Strings are parsed in base 10.
\end{itemize}

{ int } (

{ \href{/docs/reference/foundations/bool/}{bool}
\href{/docs/reference/foundations/int/}{int}
\href{/docs/reference/foundations/float/}{float}
\href{/docs/reference/foundations/str/}{str}
\href{/docs/reference/foundations/decimal/}{decimal} }

) -\textgreater{} \href{/docs/reference/foundations/int/}{int}

\begin{verbatim}
#int(false) \
#int(true) \
#int(2.7) \
#int(decimal("3.8")) \
#(int("27") + int("4"))
\end{verbatim}

\includegraphics[width=5in,height=\textheight,keepaspectratio]{/assets/docs/4vDM_wHvGAGqziHd9y2LQQAAAAAAAAAA.png}

\paragraph{\texorpdfstring{\texttt{\ value\ }}{ value }}\label{constructor-value}

\href{/docs/reference/foundations/bool/}{bool} {or}
\href{/docs/reference/foundations/int/}{int} {or}
\href{/docs/reference/foundations/float/}{float} {or}
\href{/docs/reference/foundations/str/}{str} {or}
\href{/docs/reference/foundations/decimal/}{decimal}

{Required} {{ Positional }}

\phantomsection\label{constructor-value-positional-tooltip}
Positional parameters are specified in order, without names.

The value that should be converted to an integer.

\subsection{\texorpdfstring{{ Definitions
}}{ Definitions }}\label{definitions}

\phantomsection\label{definitions-tooltip}
Functions and types and can have associated definitions. These are
accessed by specifying the function or type, followed by a period, and
then the definition\textquotesingle s name.

\subsubsection{\texorpdfstring{\texttt{\ signum\ }}{ signum }}\label{definitions-signum}

Calculates the sign of an integer.

\begin{itemize}
\tightlist
\item
  If the number is positive, returns
  \texttt{\ }{\texttt{\ 1\ }}\texttt{\ } .
\item
  If the number is negative, returns
  \texttt{\ }{\texttt{\ -\ }}\texttt{\ }{\texttt{\ 1\ }}\texttt{\ } .
\item
  If the number is zero, returns \texttt{\ }{\texttt{\ 0\ }}\texttt{\ }
  .
\end{itemize}

self { . } { signum } (

) -\textgreater{} \href{/docs/reference/foundations/int/}{int}

\begin{verbatim}
#(5).signum() \
#(-5).signum() \
#(0).signum()
\end{verbatim}

\includegraphics[width=5in,height=\textheight,keepaspectratio]{/assets/docs/Vicm2VF6Z98sgjNZQYlaBgAAAAAAAAAA.png}

\subsubsection{\texorpdfstring{\texttt{\ bit-not\ }}{ bit-not }}\label{definitions-bit-not}

Calculates the bitwise NOT of an integer.

For the purposes of this function, the operand is treated as a signed
integer of 64 bits.

self { . } { bit-not } (

) -\textgreater{} \href{/docs/reference/foundations/int/}{int}

\begin{verbatim}
#4.bit-not() \
#(-1).bit-not()
\end{verbatim}

\includegraphics[width=5in,height=\textheight,keepaspectratio]{/assets/docs/3AYO-p6E-z3VLH4vNyWEKgAAAAAAAAAA.png}

\subsubsection{\texorpdfstring{\texttt{\ bit-and\ }}{ bit-and }}\label{definitions-bit-and}

Calculates the bitwise AND between two integers.

For the purposes of this function, the operands are treated as signed
integers of 64 bits.

self { . } { bit-and } (

{ \href{/docs/reference/foundations/int/}{int} }

) -\textgreater{} \href{/docs/reference/foundations/int/}{int}

\begin{verbatim}
#128.bit-and(192)
\end{verbatim}

\includegraphics[width=5in,height=\textheight,keepaspectratio]{/assets/docs/knwTrW-Xbj5sqbcdza7ewgAAAAAAAAAA.png}

\paragraph{\texorpdfstring{\texttt{\ rhs\ }}{ rhs }}\label{definitions-bit-and-rhs}

\href{/docs/reference/foundations/int/}{int}

{Required} {{ Positional }}

\phantomsection\label{definitions-bit-and-rhs-positional-tooltip}
Positional parameters are specified in order, without names.

The right-hand operand of the bitwise AND.

\subsubsection{\texorpdfstring{\texttt{\ bit-or\ }}{ bit-or }}\label{definitions-bit-or}

Calculates the bitwise OR between two integers.

For the purposes of this function, the operands are treated as signed
integers of 64 bits.

self { . } { bit-or } (

{ \href{/docs/reference/foundations/int/}{int} }

) -\textgreater{} \href{/docs/reference/foundations/int/}{int}

\begin{verbatim}
#64.bit-or(32)
\end{verbatim}

\includegraphics[width=5in,height=\textheight,keepaspectratio]{/assets/docs/zaVKMztfj-8VIfbLXeJFUAAAAAAAAAAA.png}

\paragraph{\texorpdfstring{\texttt{\ rhs\ }}{ rhs }}\label{definitions-bit-or-rhs}

\href{/docs/reference/foundations/int/}{int}

{Required} {{ Positional }}

\phantomsection\label{definitions-bit-or-rhs-positional-tooltip}
Positional parameters are specified in order, without names.

The right-hand operand of the bitwise OR.

\subsubsection{\texorpdfstring{\texttt{\ bit-xor\ }}{ bit-xor }}\label{definitions-bit-xor}

Calculates the bitwise XOR between two integers.

For the purposes of this function, the operands are treated as signed
integers of 64 bits.

self { . } { bit-xor } (

{ \href{/docs/reference/foundations/int/}{int} }

) -\textgreater{} \href{/docs/reference/foundations/int/}{int}

\begin{verbatim}
#64.bit-xor(96)
\end{verbatim}

\includegraphics[width=5in,height=\textheight,keepaspectratio]{/assets/docs/KUPqsOL5IXWcGSfAhFpL6wAAAAAAAAAA.png}

\paragraph{\texorpdfstring{\texttt{\ rhs\ }}{ rhs }}\label{definitions-bit-xor-rhs}

\href{/docs/reference/foundations/int/}{int}

{Required} {{ Positional }}

\phantomsection\label{definitions-bit-xor-rhs-positional-tooltip}
Positional parameters are specified in order, without names.

The right-hand operand of the bitwise XOR.

\subsubsection{\texorpdfstring{\texttt{\ bit-lshift\ }}{ bit-lshift }}\label{definitions-bit-lshift}

Shifts the operand\textquotesingle s bits to the left by the specified
amount.

For the purposes of this function, the operand is treated as a signed
integer of 64 bits. An error will occur if the result is too large to
fit in a 64-bit integer.

self { . } { bit-lshift } (

{ \href{/docs/reference/foundations/int/}{int} }

) -\textgreater{} \href{/docs/reference/foundations/int/}{int}

\begin{verbatim}
#33.bit-lshift(2) \
#(-1).bit-lshift(3)
\end{verbatim}

\includegraphics[width=5in,height=\textheight,keepaspectratio]{/assets/docs/kIVISyJsbGpK3k_fu59O2AAAAAAAAAAA.png}

\paragraph{\texorpdfstring{\texttt{\ shift\ }}{ shift }}\label{definitions-bit-lshift-shift}

\href{/docs/reference/foundations/int/}{int}

{Required} {{ Positional }}

\phantomsection\label{definitions-bit-lshift-shift-positional-tooltip}
Positional parameters are specified in order, without names.

The amount of bits to shift. Must not be negative.

\subsubsection{\texorpdfstring{\texttt{\ bit-rshift\ }}{ bit-rshift }}\label{definitions-bit-rshift}

Shifts the operand\textquotesingle s bits to the right by the specified
amount. Performs an arithmetic shift by default (extends the sign bit to
the left, such that negative numbers stay negative), but that can be
changed by the \texttt{\ logical\ } parameter.

For the purposes of this function, the operand is treated as a signed
integer of 64 bits.

self { . } { bit-rshift } (

{ \href{/docs/reference/foundations/int/}{int} , } {
\hyperref[definitions-bit-rshift-parameters-logical]{logical :}
\href{/docs/reference/foundations/bool/}{bool} , }

) -\textgreater{} \href{/docs/reference/foundations/int/}{int}

\begin{verbatim}
#64.bit-rshift(2) \
#(-8).bit-rshift(2) \
#(-8).bit-rshift(2, logical: true)
\end{verbatim}

\includegraphics[width=5in,height=\textheight,keepaspectratio]{/assets/docs/gebaB-CZOzDtnvfjDfjtTgAAAAAAAAAA.png}

\paragraph{\texorpdfstring{\texttt{\ shift\ }}{ shift }}\label{definitions-bit-rshift-shift}

\href{/docs/reference/foundations/int/}{int}

{Required} {{ Positional }}

\phantomsection\label{definitions-bit-rshift-shift-positional-tooltip}
Positional parameters are specified in order, without names.

The amount of bits to shift. Must not be negative.

Shifts larger than 63 are allowed and will cause the return value to
saturate. For non-negative numbers, the return value saturates at
\texttt{\ }{\texttt{\ 0\ }}\texttt{\ } , while, for negative numbers, it
saturates at
\texttt{\ }{\texttt{\ -\ }}\texttt{\ }{\texttt{\ 1\ }}\texttt{\ } if
\texttt{\ logical\ } is set to
\texttt{\ }{\texttt{\ false\ }}\texttt{\ } , or
\texttt{\ }{\texttt{\ 0\ }}\texttt{\ } if it is
\texttt{\ }{\texttt{\ true\ }}\texttt{\ } . This behavior is consistent
with just applying this operation multiple times. Therefore, the shift
will always succeed.

\paragraph{\texorpdfstring{\texttt{\ logical\ }}{ logical }}\label{definitions-bit-rshift-logical}

\href{/docs/reference/foundations/bool/}{bool}

Toggles whether a logical (unsigned) right shift should be performed
instead of arithmetic right shift. If this is
\texttt{\ }{\texttt{\ true\ }}\texttt{\ } , negative operands will not
preserve their sign bit, and bits which appear to the left after the
shift will be \texttt{\ }{\texttt{\ 0\ }}\texttt{\ } . This parameter
has no effect on non-negative operands.

Default: \texttt{\ }{\texttt{\ false\ }}\texttt{\ }

\subsubsection{\texorpdfstring{\texttt{\ from-bytes\ }}{ from-bytes }}\label{definitions-from-bytes}

Converts bytes to an integer.

int { . } { from-bytes } (

{ \href{/docs/reference/foundations/bytes/}{bytes} , } {
\hyperref[definitions-from-bytes-parameters-endian]{endian :}
\href{/docs/reference/foundations/str/}{str} , } {
\hyperref[definitions-from-bytes-parameters-signed]{signed :}
\href{/docs/reference/foundations/bool/}{bool} , }

) -\textgreater{} \href{/docs/reference/foundations/int/}{int}

\begin{verbatim}
#int.from-bytes(bytes((0, 0, 0, 0, 0, 0, 0, 1))) \
#int.from-bytes(bytes((1, 0, 0, 0, 0, 0, 0, 0)), endian: "big")
\end{verbatim}

\includegraphics[width=5in,height=\textheight,keepaspectratio]{/assets/docs/I0LPQ0WUii0fthcD20cosAAAAAAAAAAA.png}

\paragraph{\texorpdfstring{\texttt{\ bytes\ }}{ bytes }}\label{definitions-from-bytes-bytes}

\href{/docs/reference/foundations/bytes/}{bytes}

{Required} {{ Positional }}

\phantomsection\label{definitions-from-bytes-bytes-positional-tooltip}
Positional parameters are specified in order, without names.

The bytes that should be converted to an integer.

Must be of length at most 8 so that the result fits into a 64-bit signed
integer.

\paragraph{\texorpdfstring{\texttt{\ endian\ }}{ endian }}\label{definitions-from-bytes-endian}

\href{/docs/reference/foundations/str/}{str}

The endianness of the conversion.

\begin{longtable}[]{@{}ll@{}}
\toprule\noalign{}
Variant & Details \\
\midrule\noalign{}
\endhead
\bottomrule\noalign{}
\endlastfoot
\texttt{\ "\ big\ "\ } & Big-endian byte order: The highest-value byte
is at the beginning of the bytes. \\
\texttt{\ "\ little\ "\ } & Little-endian byte order: The lowest-value
byte is at the beginning of the bytes. \\
\end{longtable}

Default: \texttt{\ }{\texttt{\ "little"\ }}\texttt{\ }

\paragraph{\texorpdfstring{\texttt{\ signed\ }}{ signed }}\label{definitions-from-bytes-signed}

\href{/docs/reference/foundations/bool/}{bool}

Whether the bytes should be treated as a signed integer. If this is
\texttt{\ }{\texttt{\ true\ }}\texttt{\ } and the most significant bit
is set, the resulting number will negative.

Default: \texttt{\ }{\texttt{\ true\ }}\texttt{\ }

\subsubsection{\texorpdfstring{\texttt{\ to-bytes\ }}{ to-bytes }}\label{definitions-to-bytes}

Converts an integer to bytes.

self { . } { to-bytes } (

{ \hyperref[definitions-to-bytes-parameters-endian]{endian :}
\href{/docs/reference/foundations/str/}{str} , } {
\hyperref[definitions-to-bytes-parameters-size]{size :}
\href{/docs/reference/foundations/int/}{int} , }

) -\textgreater{} \href{/docs/reference/foundations/bytes/}{bytes}

\begin{verbatim}
#array(10000.to-bytes(endian: "big")) \
#array(10000.to-bytes(size: 4))
\end{verbatim}

\includegraphics[width=5in,height=\textheight,keepaspectratio]{/assets/docs/FF7gGW4eVOEhYjIZXy8BIgAAAAAAAAAA.png}

\paragraph{\texorpdfstring{\texttt{\ endian\ }}{ endian }}\label{definitions-to-bytes-endian}

\href{/docs/reference/foundations/str/}{str}

The endianness of the conversion.

\begin{longtable}[]{@{}ll@{}}
\toprule\noalign{}
Variant & Details \\
\midrule\noalign{}
\endhead
\bottomrule\noalign{}
\endlastfoot
\texttt{\ "\ big\ "\ } & Big-endian byte order: The highest-value byte
is at the beginning of the bytes. \\
\texttt{\ "\ little\ "\ } & Little-endian byte order: The lowest-value
byte is at the beginning of the bytes. \\
\end{longtable}

Default: \texttt{\ }{\texttt{\ "little"\ }}\texttt{\ }

\paragraph{\texorpdfstring{\texttt{\ size\ }}{ size }}\label{definitions-to-bytes-size}

\href{/docs/reference/foundations/int/}{int}

The size in bytes of the resulting bytes (must be at least zero). If the
integer is too large to fit in the specified size, the conversion will
truncate the remaining bytes based on the endianness. To keep the same
resulting value, if the endianness is big-endian, the truncation will
happen at the rightmost bytes. Otherwise, if the endianness is
little-endian, the truncation will happen at the leftmost bytes.

Be aware that if the integer is negative and the size is not enough to
make the number fit, when passing the resulting bytes to
\texttt{\ int.from-bytes\ } , the resulting number might be positive, as
the most significant bit might not be set to 1.

Default: \texttt{\ }{\texttt{\ 8\ }}\texttt{\ }

\href{/docs/reference/foundations/function/}{\pandocbounded{\includesvg[keepaspectratio]{/assets/icons/16-arrow-right.svg}}}

{ Function } { Previous page }

\href{/docs/reference/foundations/label/}{\pandocbounded{\includesvg[keepaspectratio]{/assets/icons/16-arrow-right.svg}}}

{ Label } { Next page }
