\title{typst.app/docs/reference/foundations/label}

\begin{itemize}
\tightlist
\item
  \href{/docs}{\includesvg[width=0.16667in,height=0.16667in]{/assets/icons/16-docs-dark.svg}}
\item
  \includesvg[width=0.16667in,height=0.16667in]{/assets/icons/16-arrow-right.svg}
\item
  \href{/docs/reference/}{Reference}
\item
  \includesvg[width=0.16667in,height=0.16667in]{/assets/icons/16-arrow-right.svg}
\item
  \href{/docs/reference/foundations/}{Foundations}
\item
  \includesvg[width=0.16667in,height=0.16667in]{/assets/icons/16-arrow-right.svg}
\item
  \href{/docs/reference/foundations/label/}{Label}
\end{itemize}

\section{\texorpdfstring{{ label }}{ label }}\label{summary}

A label for an element.

Inserting a label into content attaches it to the closest preceding
element that is not a space. The preceding element must be in the same
scope as the label, which means that
\texttt{\ Hello\ }{\texttt{\ \#\ }}\texttt{\ }{\texttt{\ {[}\ }}\texttt{\ }{\texttt{\ \textless{}label\textgreater{}\ }}\texttt{\ }{\texttt{\ {]}\ }}\texttt{\ }
, for instance, wouldn\textquotesingle t work.

A labelled element can be \href{/docs/reference/model/ref/}{referenced}
, \href{/docs/reference/introspection/query/}{queried} for, and
\href{/docs/reference/styling/}{styled} through its label.

Once constructed, you can get the name of a label using
\href{/docs/reference/foundations/str/\#constructor}{\texttt{\ str\ }} .

\subsection{Example}\label{example}

\begin{verbatim}
#show <a>: set text(blue)
#show label("b"): set text(red)

= Heading <a>
*Strong* #label("b")
\end{verbatim}

\includegraphics[width=5in,height=\textheight,keepaspectratio]{/assets/docs/l3ZXI9iv-ZpcNuL82oagnwAAAAAAAAAA.png}

\subsection{Syntax}\label{syntax}

This function also has dedicated syntax: You can create a label by
enclosing its name in angle brackets. This works both in markup and
code. A label\textquotesingle s name can contain letters, numbers,
\texttt{\ \_\ } , \texttt{\ -\ } , \texttt{\ :\ } , and \texttt{\ .\ } .

Note that there is a syntactical difference when using the dedicated
syntax for this function. In the code below, the
\texttt{\ }{\texttt{\ \textless{}a\textgreater{}\ }}\texttt{\ }
terminates the heading and thus attaches to the heading itself, whereas
the
\texttt{\ }{\texttt{\ \#\ }}\texttt{\ }{\texttt{\ label\ }}\texttt{\ }{\texttt{\ (\ }}\texttt{\ }{\texttt{\ "b"\ }}\texttt{\ }{\texttt{\ )\ }}\texttt{\ }
is part of the heading and thus attaches to the
heading\textquotesingle s text.

\begin{verbatim}
// Equivalent to `#heading[Introduction] <a>`.
= Introduction <a>

// Equivalent to `#heading[Conclusion #label("b")]`.
= Conclusion #label("b")
\end{verbatim}

Currently, labels can only be attached to elements in markup mode, not
in code mode. This might change in the future.

\subsection{\texorpdfstring{Constructor
{}}{Constructor }}\label{constructor}

\phantomsection\label{constructor-constructor-tooltip}
If a type has a constructor, you can call it like a function to create a
new value of the type.

Creates a label from a string.

{ label } (

{ \href{/docs/reference/foundations/str/}{str} }

) -\textgreater{} \href{/docs/reference/foundations/label/}{label}

\paragraph{\texorpdfstring{\texttt{\ name\ }}{ name }}\label{constructor-name}

\href{/docs/reference/foundations/str/}{str}

{Required} {{ Positional }}

\phantomsection\label{constructor-name-positional-tooltip}
Positional parameters are specified in order, without names.

The name of the label.

\href{/docs/reference/foundations/int/}{\pandocbounded{\includesvg[keepaspectratio]{/assets/icons/16-arrow-right.svg}}}

{ Integer } { Previous page }

\href{/docs/reference/foundations/module/}{\pandocbounded{\includesvg[keepaspectratio]{/assets/icons/16-arrow-right.svg}}}

{ Module } { Next page }
