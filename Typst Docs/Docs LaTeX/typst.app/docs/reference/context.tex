\title{typst.app/docs/reference/context}

\begin{itemize}
\tightlist
\item
  \href{/docs}{\includesvg[width=0.16667in,height=0.16667in]{/assets/icons/16-docs-dark.svg}}
\item
  \includesvg[width=0.16667in,height=0.16667in]{/assets/icons/16-arrow-right.svg}
\item
  \href{/docs/reference/}{Reference}
\item
  \includesvg[width=0.16667in,height=0.16667in]{/assets/icons/16-arrow-right.svg}
\item
  \href{/docs/reference/context/}{Context}
\end{itemize}

\section{Context}\label{context}

Sometimes, we want to create content that reacts to its location in the
document. This could be a localized phrase that depends on the
configured text language or something as simple as a heading number
which prints the right value based on how many headings came before it.
However, Typst code isn\textquotesingle t directly aware of its location
in the document. Some code at the beginning of the source text could
yield content that ends up at the back of the document.

To produce content that is reactive to its surroundings, we must thus
specifically instruct Typst: We do this with the
\texttt{\ }{\texttt{\ context\ }}\texttt{\ } keyword, which precedes an
expression and ensures that it is computed with knowledge of its
environment. In return, the context expression itself ends up opaque. We
cannot directly access whatever results from it in our code, precisely
because it is contextual: There is no one correct result, there may be
multiple results in different places of the document. For this reason,
everything that depends on the contextual data must happen inside of the
context expression.

Aside from explicit context expressions, context is also established
implicitly in some places that are also aware of their location in the
document: \href{/docs/reference/styling/\#show-rules}{Show rules}
provide context \textsuperscript{\hyperref[1]{1}} and numberings in the
outline, for instance, also provide the proper context to resolve
counters.

\subsection{Style context}\label{style-context}

With set rules, we can adjust style properties for parts or the whole of
our document. We cannot access these without a known context, as they
may change throughout the course of the document. When context is
available, we can retrieve them simply by accessing them as fields on
the respective element function.

\begin{verbatim}
#set text(lang: "de")
#context text.lang
\end{verbatim}

\includegraphics[width=5in,height=\textheight,keepaspectratio]{/assets/docs/ETetUaSK2J1pHpdRRUWzagAAAAAAAAAA.png}

As explained above, a context expression is reactive to the different
environments it is placed into. In the example below, we create a single
context expression, store it in the \texttt{\ value\ } variable and use
it multiple times. Each use properly reacts to the current surroundings.

\begin{verbatim}
#let value = context text.lang
#value

#set text(lang: "de")
#value

#set text(lang: "fr")
#value
\end{verbatim}

\includegraphics[width=5in,height=\textheight,keepaspectratio]{/assets/docs/cUJma0l-7W2Pm0tXEKJmjAAAAAAAAAAA.png}

Crucially, upon creation, \texttt{\ value\ } becomes opaque
\href{/docs/reference/foundations/content/}{content} that we cannot peek
into. It can only be resolved when placed somewhere because only then
the context is known. The body of a context expression may be evaluated
zero, one, or multiple times, depending on how many different places it
is put into.

\subsection{Location context}\label{location-context}

We\textquotesingle ve already seen that context gives us access to set
rule values. But it can do more: It also lets us know \emph{where} in
the document we currently are, relative to other elements, and
absolutely on the pages. We can use this information to create very
flexible interactions between different document parts. This underpins
features like heading numbering, the table of contents, or page headers
dependent on section headings.

Some functions like
\href{/docs/reference/introspection/counter/\#definitions-get}{\texttt{\ counter.get\ }}
implicitly access the current location. In the example below, we want to
retrieve the value of the heading counter. Since it changes throughout
the document, we need to first enter a context expression. Then, we use
\texttt{\ get\ } to retrieve the counter\textquotesingle s current
value. This function accesses the current location from the context to
resolve the counter value. Counters have multiple levels and
\texttt{\ get\ } returns an array with the resolved numbers. Thus, we
get the following result:

\begin{verbatim}
#set heading(numbering: "1.")

= Introduction
#lorem(5)

#context counter(heading).get()

= Background
#lorem(5)

#context counter(heading).get()
\end{verbatim}

\includegraphics[width=5in,height=\textheight,keepaspectratio]{/assets/docs/bQONUXVpXWNuuUEOrLszpQAAAAAAAAAA.png}

For more flexibility, we can also use the
\href{/docs/reference/introspection/here/}{\texttt{\ here\ }} function
to directly extract the current
\href{/docs/reference/introspection/location/}{location} from the
context. The example below demonstrates this:

\begin{itemize}
\tightlist
\item
  We first have
  \texttt{\ }{\texttt{\ counter\ }}\texttt{\ }{\texttt{\ (\ }}\texttt{\ heading\ }{\texttt{\ )\ }}\texttt{\ }{\texttt{\ .\ }}\texttt{\ }{\texttt{\ get\ }}\texttt{\ }{\texttt{\ (\ }}\texttt{\ }{\texttt{\ )\ }}\texttt{\ }
  , which resolves to
  \texttt{\ }{\texttt{\ (\ }}\texttt{\ }{\texttt{\ 2\ }}\texttt{\ }{\texttt{\ ,\ }}\texttt{\ }{\texttt{\ )\ }}\texttt{\ }
  as before.
\item
  We then use the more powerful
  \href{/docs/reference/introspection/counter/\#definitions-at}{\texttt{\ counter.at\ }}
  with \href{/docs/reference/introspection/here/}{\texttt{\ here\ }} ,
  which in combination is equivalent to \texttt{\ get\ } , and thus get
  \texttt{\ }{\texttt{\ (\ }}\texttt{\ }{\texttt{\ 2\ }}\texttt{\ }{\texttt{\ ,\ }}\texttt{\ }{\texttt{\ )\ }}\texttt{\ }
  .
\item
  Finally, we use \texttt{\ at\ } with a
  \href{/docs/reference/foundations/label/}{label} to retrieve the value
  of the counter at a \emph{different} location in the document, in our
  case that of the introduction heading. This yields
  \texttt{\ }{\texttt{\ (\ }}\texttt{\ }{\texttt{\ 1\ }}\texttt{\ }{\texttt{\ ,\ }}\texttt{\ }{\texttt{\ )\ }}\texttt{\ }
  . Typst\textquotesingle s context system gives us time travel
  abilities and lets us retrieve the values of any counters and states
  at \emph{any} location in the document.
\end{itemize}

\begin{verbatim}
#set heading(numbering: "1.")

= Introduction <intro>
#lorem(5)

= Background <back>
#lorem(5)

#context [
  #counter(heading).get() \
  #counter(heading).at(here()) \
  #counter(heading).at(<intro>)
]
\end{verbatim}

\includegraphics[width=5in,height=\textheight,keepaspectratio]{/assets/docs/gip9ugheiaYydjAEj2_jbgAAAAAAAAAA.png}

As mentioned before, we can also use context to get the physical
position of elements on the pages. We do this with the
\href{/docs/reference/introspection/locate/}{\texttt{\ locate\ }}
function, which works similarly to \texttt{\ counter.at\ } : It takes a
location or other \href{/docs/reference/foundations/selector/}{selector}
that resolves to a unique element (could also be a label) and returns
the position on the pages for that element.

\begin{verbatim}
Background is at: \
#context locate(<back>).position()

= Introduction <intro>
#lorem(5)
#pagebreak()

= Background <back>
#lorem(5)
\end{verbatim}

\includegraphics[width=5in,height=\textheight,keepaspectratio]{/assets/docs/AV1GaGSFxqcGN8RTlxty3gAAAAAAAAAA.png}
\includegraphics[width=5in,height=\textheight,keepaspectratio]{/assets/docs/AV1GaGSFxqcGN8RTlxty3gAAAAAAAAAB.png}

There are other functions that make use of the location context, most
prominently
\href{/docs/reference/introspection/query/}{\texttt{\ query\ }} . Take a
look at the \href{/docs/reference/introspection/}{introspection}
category for more details on those.

\subsection{Nested contexts}\label{nested-contexts}

Context is also accessible from within function calls nested in context
blocks. In the example below, \texttt{\ foo\ } itself becomes a
contextual function, just like
\href{/docs/reference/layout/length/\#definitions-to-absolute}{\texttt{\ to-absolute\ }}
is.

\begin{verbatim}
#let foo() = 1em.to-absolute()
#context {
  foo() == text.size
}
\end{verbatim}

\includegraphics[width=5in,height=\textheight,keepaspectratio]{/assets/docs/tBYLufutRlRl2ZJ_PAm-owAAAAAAAAAA.png}

Context blocks can be nested. Contextual code will then always access
the innermost context. The example below demonstrates this: The first
\texttt{\ text.lang\ } will access the outer context
block\textquotesingle s styles and as such, it will \textbf{not} see the
effect of
\texttt{\ }{\texttt{\ set\ }}\texttt{\ }{\texttt{\ text\ }}\texttt{\ }{\texttt{\ (\ }}\texttt{\ lang\ }{\texttt{\ :\ }}\texttt{\ }{\texttt{\ "fr"\ }}\texttt{\ }{\texttt{\ )\ }}\texttt{\ }
. The nested context block around the second \texttt{\ text.lang\ } ,
however, starts after the set rule and will thus show its effect.

\begin{verbatim}
#set text(lang: "de")
#context [
  #set text(lang: "fr")
  #text.lang \
  #context text.lang
]
\end{verbatim}

\includegraphics[width=5in,height=\textheight,keepaspectratio]{/assets/docs/-8ZHuN0AkDNg1gXmAO7X2wAAAAAAAAAA.png}

You might wonder why Typst ignores the French set rule when computing
the first \texttt{\ text.lang\ } in the example above. The reason is
that, in the general case, Typst cannot know all the styles that will
apply as set rules can be applied to content after it has been
constructed. Below, \texttt{\ text.lang\ } is already computed when the
template function is applied. As such, it cannot possibly be aware of
the language change to French in the template.

\begin{verbatim}
#let template(body) = {
  set text(lang: "fr")
  upper(body)
}

#set text(lang: "de")
#context [
  #show: template
  #text.lang \
  #context text.lang
]
\end{verbatim}

\includegraphics[width=5in,height=\textheight,keepaspectratio]{/assets/docs/ptMaFdqycQGV8lm06g29-gAAAAAAAAAA.png}

The second \texttt{\ text.lang\ } , however, \emph{does} react to the
language change because evaluation of its surrounding context block is
deferred until the styles for it are known. This illustrates the
importance of picking the right insertion point for a context to get
access to precisely the right styles.

The same also holds true for the location context. Below, the first
\texttt{\ c\ }{\texttt{\ .\ }}\texttt{\ }{\texttt{\ display\ }}\texttt{\ }{\texttt{\ (\ }}\texttt{\ }{\texttt{\ )\ }}\texttt{\ }
call will access the outer context block and will thus not see the
effect of
\texttt{\ c\ }{\texttt{\ .\ }}\texttt{\ }{\texttt{\ update\ }}\texttt{\ }{\texttt{\ (\ }}\texttt{\ }{\texttt{\ 2\ }}\texttt{\ }{\texttt{\ )\ }}\texttt{\ }
while the second
\texttt{\ c\ }{\texttt{\ .\ }}\texttt{\ }{\texttt{\ display\ }}\texttt{\ }{\texttt{\ (\ }}\texttt{\ }{\texttt{\ )\ }}\texttt{\ }
accesses the inner context and will thus see it.

\begin{verbatim}
#let c = counter("mycounter")
#c.update(1)
#context [
  #c.update(2)
  #c.display() \
  #context c.display()
]
\end{verbatim}

\includegraphics[width=5in,height=\textheight,keepaspectratio]{/assets/docs/6mlAfSm7646viO4S8ua6gwAAAAAAAAAA.png}

\subsection{Compiler iterations}\label{compiler-iterations}

To resolve contextual interactions, the Typst compiler processes your
document multiple times. For instance, to resolve a \texttt{\ locate\ }
call, Typst first provides a placeholder position, layouts your document
and then recompiles with the known position from the finished layout.
The same approach is taken to resolve counters, states, and queries. In
certain cases, Typst may even need more than two iterations to resolve
everything. While that\textquotesingle s sometimes a necessity, it may
also be a sign of misuse of contextual functions (e.g. of
\href{/docs/reference/introspection/state/\#caution}{state} ). If Typst
cannot resolve everything within five attempts, it will stop and output
the warning "layout did not converge within 5 attempts."

A very careful reader might have noticed that not all of the functions
presented above actually make use of the current location. While
\texttt{\ }{\texttt{\ counter\ }}\texttt{\ }{\texttt{\ (\ }}\texttt{\ heading\ }{\texttt{\ )\ }}\texttt{\ }{\texttt{\ .\ }}\texttt{\ }{\texttt{\ get\ }}\texttt{\ }{\texttt{\ (\ }}\texttt{\ }{\texttt{\ )\ }}\texttt{\ }
definitely depends on it,
\texttt{\ }{\texttt{\ counter\ }}\texttt{\ }{\texttt{\ (\ }}\texttt{\ heading\ }{\texttt{\ )\ }}\texttt{\ }{\texttt{\ .\ }}\texttt{\ }{\texttt{\ at\ }}\texttt{\ }{\texttt{\ (\ }}\texttt{\ }{\texttt{\ \textless{}intro\textgreater{}\ }}\texttt{\ }{\texttt{\ )\ }}\texttt{\ }
, for instance, does not. However, it still requires context. While its
value is always the same \emph{within} one compilation iteration, it may
change over the course of multiple compiler iterations. If one could
call it directly at the top level of a module, the whole module and its
exports could change over the course of multiple compiler iterations,
which would not be desirable.

\phantomsection\label{1}
\textsuperscript{1}

Currently, all show rules provide styling context, but only show rules
on \href{/docs/reference/introspection/location/\#locatable}{locatable}
elements provide a location context.

\href{/docs/reference/scripting/}{\pandocbounded{\includesvg[keepaspectratio]{/assets/icons/16-arrow-right.svg}}}

{ Scripting } { Previous page }

\href{/docs/reference/foundations/}{\pandocbounded{\includesvg[keepaspectratio]{/assets/icons/16-arrow-right.svg}}}

{ Foundations } { Next page }
