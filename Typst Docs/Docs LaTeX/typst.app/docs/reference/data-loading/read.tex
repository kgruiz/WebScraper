\title{typst.app/docs/reference/data-loading/read}

\begin{itemize}
\tightlist
\item
  \href{/docs}{\includesvg[width=0.16667in,height=0.16667in]{/assets/icons/16-docs-dark.svg}}
\item
  \includesvg[width=0.16667in,height=0.16667in]{/assets/icons/16-arrow-right.svg}
\item
  \href{/docs/reference/}{Reference}
\item
  \includesvg[width=0.16667in,height=0.16667in]{/assets/icons/16-arrow-right.svg}
\item
  \href{/docs/reference/data-loading/}{Data Loading}
\item
  \includesvg[width=0.16667in,height=0.16667in]{/assets/icons/16-arrow-right.svg}
\item
  \href{/docs/reference/data-loading/read/}{Read}
\end{itemize}

\section{\texorpdfstring{\texttt{\ read\ }}{ read }}\label{summary}

Reads plain text or data from a file.

By default, the file will be read as UTF-8 and returned as a
\href{/docs/reference/foundations/str/}{string} .

If you specify \texttt{\ encoding:\ }{\texttt{\ none\ }}\texttt{\ } ,
this returns raw \href{/docs/reference/foundations/bytes/}{bytes}
instead.

\subsection{Example}\label{example}

\begin{verbatim}
An example for a HTML file: \
#let text = read("example.html")
#raw(text, lang: "html")

Raw bytes:
#read("tiger.jpg", encoding: none)
\end{verbatim}

\includegraphics[width=5in,height=\textheight,keepaspectratio]{/assets/docs/uS5DrZwzU2PIqO_vdJc7GQAAAAAAAAAA.png}

\subsection{\texorpdfstring{{ Parameters
}}{ Parameters }}\label{parameters}

\phantomsection\label{parameters-tooltip}
Parameters are the inputs to a function. They are specified in
parentheses after the function name.

{ read } (

{ \href{/docs/reference/foundations/str/}{str} , } {
\hyperref[parameters-encoding]{encoding :}
\href{/docs/reference/foundations/none/}{none}
\href{/docs/reference/foundations/str/}{str} , }

) -\textgreater{} \href{/docs/reference/foundations/str/}{str}
\href{/docs/reference/foundations/bytes/}{bytes}

\subsubsection{\texorpdfstring{\texttt{\ path\ }}{ path }}\label{parameters-path}

\href{/docs/reference/foundations/str/}{str}

{Required} {{ Positional }}

\phantomsection\label{parameters-path-positional-tooltip}
Positional parameters are specified in order, without names.

Path to a file.

For more details, see the \href{/docs/reference/syntax/\#paths}{Paths
section} .

\subsubsection{\texorpdfstring{\texttt{\ encoding\ }}{ encoding }}\label{parameters-encoding}

\href{/docs/reference/foundations/none/}{none} {or}
\href{/docs/reference/foundations/str/}{str}

The encoding to read the file with.

If set to \texttt{\ }{\texttt{\ none\ }}\texttt{\ } , this function
returns raw bytes.

\begin{longtable}[]{@{}ll@{}}
\toprule\noalign{}
Variant & Details \\
\midrule\noalign{}
\endhead
\bottomrule\noalign{}
\endlastfoot
\texttt{\ "\ utf8\ "\ } & The Unicode UTF-8 encoding. \\
\end{longtable}

Default: \texttt{\ }{\texttt{\ "utf8"\ }}\texttt{\ }

\href{/docs/reference/data-loading/json/}{\pandocbounded{\includesvg[keepaspectratio]{/assets/icons/16-arrow-right.svg}}}

{ JSON } { Previous page }

\href{/docs/reference/data-loading/toml/}{\pandocbounded{\includesvg[keepaspectratio]{/assets/icons/16-arrow-right.svg}}}

{ TOML } { Next page }
