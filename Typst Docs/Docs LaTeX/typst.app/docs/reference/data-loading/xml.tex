\title{typst.app/docs/reference/data-loading/xml}

\begin{itemize}
\tightlist
\item
  \href{/docs}{\includesvg[width=0.16667in,height=0.16667in]{/assets/icons/16-docs-dark.svg}}
\item
  \includesvg[width=0.16667in,height=0.16667in]{/assets/icons/16-arrow-right.svg}
\item
  \href{/docs/reference/}{Reference}
\item
  \includesvg[width=0.16667in,height=0.16667in]{/assets/icons/16-arrow-right.svg}
\item
  \href{/docs/reference/data-loading/}{Data Loading}
\item
  \includesvg[width=0.16667in,height=0.16667in]{/assets/icons/16-arrow-right.svg}
\item
  \href{/docs/reference/data-loading/xml/}{XML}
\end{itemize}

\section{\texorpdfstring{\texttt{\ xml\ }}{ xml }}\label{summary}

Reads structured data from an XML file.

The XML file is parsed into an array of dictionaries and strings. XML
nodes can be elements or strings. Elements are represented as
dictionaries with the following keys:

\begin{itemize}
\tightlist
\item
  \texttt{\ tag\ } : The name of the element as a string.
\item
  \texttt{\ attrs\ } : A dictionary of the element\textquotesingle s
  attributes as strings.
\item
  \texttt{\ children\ } : An array of the element\textquotesingle s
  child nodes.
\end{itemize}

The XML file in the example contains a root \texttt{\ news\ } tag with
multiple \texttt{\ article\ } tags. Each article has a
\texttt{\ title\ } , \texttt{\ author\ } , and \texttt{\ content\ } tag.
The \texttt{\ content\ } tag contains one or more paragraphs, which are
represented as \texttt{\ p\ } tags.

\subsection{Example}\label{example}

\begin{verbatim}
#let find-child(elem, tag) = {
  elem.children
    .find(e => "tag" in e and e.tag == tag)
}

#let article(elem) = {
  let title = find-child(elem, "title")
  let author = find-child(elem, "author")
  let pars = find-child(elem, "content")

  heading(title.children.first())
  text(10pt, weight: "medium")[
    Published by
    #author.children.first()
  ]

  for p in pars.children {
    if (type(p) == "dictionary") {
      parbreak()
      p.children.first()
    }
  }
}

#let data = xml("example.xml")
#for elem in data.first().children {
  if (type(elem) == "dictionary") {
    article(elem)
  }
}
\end{verbatim}

\includegraphics[width=5in,height=\textheight,keepaspectratio]{/assets/docs/ImsUm8fcO-Uh3s95k6HvEQAAAAAAAAAA.png}

\subsection{\texorpdfstring{{ Parameters
}}{ Parameters }}\label{parameters}

\phantomsection\label{parameters-tooltip}
Parameters are the inputs to a function. They are specified in
parentheses after the function name.

{ xml } (

{ \href{/docs/reference/foundations/str/}{str} }

) -\textgreater{} { any }

\subsubsection{\texorpdfstring{\texttt{\ path\ }}{ path }}\label{parameters-path}

\href{/docs/reference/foundations/str/}{str}

{Required} {{ Positional }}

\phantomsection\label{parameters-path-positional-tooltip}
Positional parameters are specified in order, without names.

Path to an XML file.

For more details, see the \href{/docs/reference/syntax/\#paths}{Paths
section} .

\subsection{\texorpdfstring{{ Definitions
}}{ Definitions }}\label{definitions}

\phantomsection\label{definitions-tooltip}
Functions and types and can have associated definitions. These are
accessed by specifying the function or type, followed by a period, and
then the definition\textquotesingle s name.

\subsubsection{\texorpdfstring{\texttt{\ decode\ }}{ decode }}\label{definitions-decode}

Reads structured data from an XML string/bytes.

xml { . } { decode } (

{ \href{/docs/reference/foundations/str/}{str}
\href{/docs/reference/foundations/bytes/}{bytes} }

) -\textgreater{} { any }

\paragraph{\texorpdfstring{\texttt{\ data\ }}{ data }}\label{definitions-decode-data}

\href{/docs/reference/foundations/str/}{str} {or}
\href{/docs/reference/foundations/bytes/}{bytes}

{Required} {{ Positional }}

\phantomsection\label{definitions-decode-data-positional-tooltip}
Positional parameters are specified in order, without names.

XML data.

\href{/docs/reference/data-loading/toml/}{\pandocbounded{\includesvg[keepaspectratio]{/assets/icons/16-arrow-right.svg}}}

{ TOML } { Previous page }

\href{/docs/reference/data-loading/yaml/}{\pandocbounded{\includesvg[keepaspectratio]{/assets/icons/16-arrow-right.svg}}}

{ YAML } { Next page }
