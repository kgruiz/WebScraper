\title{typst.app/docs/reference/math/cases}

\begin{itemize}
\tightlist
\item
  \href{/docs}{\includesvg[width=0.16667in,height=0.16667in]{/assets/icons/16-docs-dark.svg}}
\item
  \includesvg[width=0.16667in,height=0.16667in]{/assets/icons/16-arrow-right.svg}
\item
  \href{/docs/reference/}{Reference}
\item
  \includesvg[width=0.16667in,height=0.16667in]{/assets/icons/16-arrow-right.svg}
\item
  \href{/docs/reference/math/}{Math}
\item
  \includesvg[width=0.16667in,height=0.16667in]{/assets/icons/16-arrow-right.svg}
\item
  \href{/docs/reference/math/cases/}{Cases}
\end{itemize}

\section{\texorpdfstring{\texttt{\ cases\ } {{ Element
}}}{ cases   Element }}\label{summary}

\phantomsection\label{element-tooltip}
Element functions can be customized with \texttt{\ set\ } and
\texttt{\ show\ } rules.

A case distinction.

Content across different branches can be aligned with the
\texttt{\ \&\ } symbol.

\subsection{Example}\label{example}

\begin{verbatim}
$ f(x, y) := cases(
  1 "if" (x dot y)/2 <= 0,
  2 "if" x "is even",
  3 "if" x in NN,
  4 "else",
) $
\end{verbatim}

\includegraphics[width=5in,height=\textheight,keepaspectratio]{/assets/docs/0X1AFPDieBd9jiawKpc0-AAAAAAAAAAA.png}

\subsection{\texorpdfstring{{ Parameters
}}{ Parameters }}\label{parameters}

\phantomsection\label{parameters-tooltip}
Parameters are the inputs to a function. They are specified in
parentheses after the function name.

math { . } { cases } (

{ \hyperref[parameters-delim]{delim :}
\href{/docs/reference/foundations/none/}{none}
\href{/docs/reference/foundations/str/}{str}
\href{/docs/reference/foundations/array/}{array}
\href{/docs/reference/symbols/symbol/}{symbol} , } {
\hyperref[parameters-reverse]{reverse :}
\href{/docs/reference/foundations/bool/}{bool} , } {
\hyperref[parameters-gap]{gap :}
\href{/docs/reference/layout/relative/}{relative} , } {
\hyperref[parameters-children]{..}
\href{/docs/reference/foundations/content/}{content} , }

) -\textgreater{} \href{/docs/reference/foundations/content/}{content}

\subsubsection{\texorpdfstring{\texttt{\ delim\ }}{ delim }}\label{parameters-delim}

\href{/docs/reference/foundations/none/}{none} {or}
\href{/docs/reference/foundations/str/}{str} {or}
\href{/docs/reference/foundations/array/}{array} {or}
\href{/docs/reference/symbols/symbol/}{symbol}

{{ Settable }}

\phantomsection\label{parameters-delim-settable-tooltip}
Settable parameters can be customized for all following uses of the
function with a \texttt{\ set\ } rule.

The delimiter to use.

Can be a single character specifying the left delimiter, in which case
the right delimiter is inferred. Otherwise, can be an array containing a
left and a right delimiter.

Default:
\texttt{\ }{\texttt{\ (\ }}\texttt{\ }{\texttt{\ "\{"\ }}\texttt{\ }{\texttt{\ ,\ }}\texttt{\ }{\texttt{\ "\}"\ }}\texttt{\ }{\texttt{\ )\ }}\texttt{\ }

\includesvg[width=0.16667in,height=0.16667in]{/assets/icons/16-arrow-right.svg}
View example

\begin{verbatim}
#set math.cases(delim: "[")
$ x = cases(1, 2) $
\end{verbatim}

\includegraphics[width=5in,height=\textheight,keepaspectratio]{/assets/docs/bErdOHWWOQLSKtsxtJeY5QAAAAAAAAAA.png}

\subsubsection{\texorpdfstring{\texttt{\ reverse\ }}{ reverse }}\label{parameters-reverse}

\href{/docs/reference/foundations/bool/}{bool}

{{ Settable }}

\phantomsection\label{parameters-reverse-settable-tooltip}
Settable parameters can be customized for all following uses of the
function with a \texttt{\ set\ } rule.

Whether the direction of cases should be reversed.

Default: \texttt{\ }{\texttt{\ false\ }}\texttt{\ }

\includesvg[width=0.16667in,height=0.16667in]{/assets/icons/16-arrow-right.svg}
View example

\begin{verbatim}
#set math.cases(reverse: true)
$ cases(1, 2) = x $
\end{verbatim}

\includegraphics[width=5in,height=\textheight,keepaspectratio]{/assets/docs/z6AQZKJsH9nM95e6Aw0hGgAAAAAAAAAA.png}

\subsubsection{\texorpdfstring{\texttt{\ gap\ }}{ gap }}\label{parameters-gap}

\href{/docs/reference/layout/relative/}{relative}

{{ Settable }}

\phantomsection\label{parameters-gap-settable-tooltip}
Settable parameters can be customized for all following uses of the
function with a \texttt{\ set\ } rule.

The gap between branches.

Default:
\texttt{\ }{\texttt{\ 0\%\ }}\texttt{\ }{\texttt{\ +\ }}\texttt{\ }{\texttt{\ 0.2em\ }}\texttt{\ }

\includesvg[width=0.16667in,height=0.16667in]{/assets/icons/16-arrow-right.svg}
View example

\begin{verbatim}
#set math.cases(gap: 1em)
$ x = cases(1, 2) $
\end{verbatim}

\includegraphics[width=5in,height=\textheight,keepaspectratio]{/assets/docs/-xscfzRH4Dw6Yi5TCvpkVwAAAAAAAAAA.png}

\subsubsection{\texorpdfstring{\texttt{\ children\ }}{ children }}\label{parameters-children}

\href{/docs/reference/foundations/content/}{content}

{Required} {{ Positional }}

\phantomsection\label{parameters-children-positional-tooltip}
Positional parameters are specified in order, without names.

{{ Variadic }}

\phantomsection\label{parameters-children-variadic-tooltip}
Variadic parameters can be specified multiple times.

The branches of the case distinction.

\href{/docs/reference/math/cancel/}{\pandocbounded{\includesvg[keepaspectratio]{/assets/icons/16-arrow-right.svg}}}

{ Cancel } { Previous page }

\href{/docs/reference/math/class/}{\pandocbounded{\includesvg[keepaspectratio]{/assets/icons/16-arrow-right.svg}}}

{ Class } { Next page }
