\title{typst.app/docs/reference/layout/relative}

\begin{itemize}
\tightlist
\item
  \href{/docs}{\includesvg[width=0.16667in,height=0.16667in]{/assets/icons/16-docs-dark.svg}}
\item
  \includesvg[width=0.16667in,height=0.16667in]{/assets/icons/16-arrow-right.svg}
\item
  \href{/docs/reference/}{Reference}
\item
  \includesvg[width=0.16667in,height=0.16667in]{/assets/icons/16-arrow-right.svg}
\item
  \href{/docs/reference/layout/}{Layout}
\item
  \includesvg[width=0.16667in,height=0.16667in]{/assets/icons/16-arrow-right.svg}
\item
  \href{/docs/reference/layout/relative/}{Relative Length}
\end{itemize}

\section{\texorpdfstring{{ relative }}{ relative }}\label{summary}

A length in relation to some known length.

This type is a combination of a
\href{/docs/reference/layout/length/}{length} with a
\href{/docs/reference/layout/ratio/}{ratio} . It results from addition
and subtraction of a length and a ratio. Wherever a relative length is
expected, you can also use a bare length or ratio.

\subsection{Example}\label{example}

\begin{verbatim}
#rect(width: 100% - 50pt)

#(100% - 50pt).length \
#(100% - 50pt).ratio
\end{verbatim}

\includegraphics[width=5in,height=\textheight,keepaspectratio]{/assets/docs/eMTS_wIJ-8rLzP6A-A6wPAAAAAAAAAAA.png}

A relative length has the following fields:

\begin{itemize}
\tightlist
\item
  \texttt{\ length\ } : Its length component.
\item
  \texttt{\ ratio\ } : Its ratio component.
\end{itemize}

\href{/docs/reference/layout/ratio/}{\pandocbounded{\includesvg[keepaspectratio]{/assets/icons/16-arrow-right.svg}}}

{ Ratio } { Previous page }

\href{/docs/reference/layout/repeat/}{\pandocbounded{\includesvg[keepaspectratio]{/assets/icons/16-arrow-right.svg}}}

{ Repeat } { Next page }
