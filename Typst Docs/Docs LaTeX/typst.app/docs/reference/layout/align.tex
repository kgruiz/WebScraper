\title{typst.app/docs/reference/layout/align}

\begin{itemize}
\tightlist
\item
  \href{/docs}{\includesvg[width=0.16667in,height=0.16667in]{/assets/icons/16-docs-dark.svg}}
\item
  \includesvg[width=0.16667in,height=0.16667in]{/assets/icons/16-arrow-right.svg}
\item
  \href{/docs/reference/}{Reference}
\item
  \includesvg[width=0.16667in,height=0.16667in]{/assets/icons/16-arrow-right.svg}
\item
  \href{/docs/reference/layout/}{Layout}
\item
  \includesvg[width=0.16667in,height=0.16667in]{/assets/icons/16-arrow-right.svg}
\item
  \href{/docs/reference/layout/align/}{Align}
\end{itemize}

\section{\texorpdfstring{\texttt{\ align\ } {{ Element
}}}{ align   Element }}\label{summary}

\phantomsection\label{element-tooltip}
Element functions can be customized with \texttt{\ set\ } and
\texttt{\ show\ } rules.

Aligns content horizontally and vertically.

\subsection{Example}\label{example}

Let\textquotesingle s start with centering our content horizontally:

\begin{verbatim}
#set page(height: 120pt)
#set align(center)

Centered text, a sight to see \
In perfect balance, visually \
Not left nor right, it stands alone \
A work of art, a visual throne
\end{verbatim}

\includegraphics[width=5in,height=\textheight,keepaspectratio]{/assets/docs/kcNIG-bYA8T9BUDnjCUJGgAAAAAAAAAA.png}

To center something vertically, use \emph{horizon} alignment:

\begin{verbatim}
#set page(height: 120pt)
#set align(horizon)

Vertically centered, \
the stage had entered, \
a new paragraph.
\end{verbatim}

\includegraphics[width=5in,height=\textheight,keepaspectratio]{/assets/docs/y9OO-MSDQIHWsGPc_6pNnAAAAAAAAAAA.png}

\subsection{Combining alignments}\label{combining-alignments}

You can combine two alignments with the \texttt{\ +\ } operator.
Let\textquotesingle s also only apply this to one piece of content by
using the function form instead of a set rule:

\begin{verbatim}
#set page(height: 120pt)
Though left in the beginning ...

#align(right + bottom)[
  ... they were right in the end, \
  and with addition had gotten, \
  the paragraph to the bottom!
]
\end{verbatim}

\includegraphics[width=5in,height=\textheight,keepaspectratio]{/assets/docs/gXaqAMYC8Licj_UCK0JSFgAAAAAAAAAA.png}

\subsection{Nested alignment}\label{nested-alignment}

You can use varying alignments for layout containers and the elements
within them. This way, you can create intricate layouts:

\begin{verbatim}
#align(center, block[
  #set align(left)
  Though centered together \
  alone \
  we \
  are \
  left.
])
\end{verbatim}

\includegraphics[width=5in,height=\textheight,keepaspectratio]{/assets/docs/B6Y-WWFtiUjCHNJ9B8R8vQAAAAAAAAAA.png}

\subsection{Alignment within the same
line}\label{alignment-within-the-same-line}

The \texttt{\ align\ } function performs block-level alignment and thus
always interrupts the current paragraph. To have different alignment for
parts of the same line, you should use
\href{/docs/reference/layout/h/}{fractional spacing} instead:

\begin{verbatim}
Start #h(1fr) End
\end{verbatim}

\includegraphics[width=5in,height=\textheight,keepaspectratio]{/assets/docs/jlafwbE2ZuISwJPQNRzA3gAAAAAAAAAA.png}

\subsection{\texorpdfstring{{ Parameters
}}{ Parameters }}\label{parameters}

\phantomsection\label{parameters-tooltip}
Parameters are the inputs to a function. They are specified in
parentheses after the function name.

{ align } (

{ \hyperref[parameters-alignment]{}
\href{/docs/reference/layout/alignment/}{alignment} , } {
\href{/docs/reference/foundations/content/}{content} , }

) -\textgreater{} \href{/docs/reference/foundations/content/}{content}

\subsubsection{\texorpdfstring{\texttt{\ alignment\ }}{ alignment }}\label{parameters-alignment}

\href{/docs/reference/layout/alignment/}{alignment}

{{ Positional }}

\phantomsection\label{parameters-alignment-positional-tooltip}
Positional parameters are specified in order, without names.

{{ Settable }}

\phantomsection\label{parameters-alignment-settable-tooltip}
Settable parameters can be customized for all following uses of the
function with a \texttt{\ set\ } rule.

The \href{/docs/reference/layout/alignment/}{alignment} along both axes.

Default: \texttt{\ start\ }{\texttt{\ +\ }}\texttt{\ top\ }

\includesvg[width=0.16667in,height=0.16667in]{/assets/icons/16-arrow-right.svg}
View example

\begin{verbatim}
#set page(height: 6cm)
#set text(lang: "ar")

مثال
#align(
  end + horizon,
  rect(inset: 12pt)[ركن]
)
\end{verbatim}

\includegraphics[width=5in,height=\textheight,keepaspectratio]{/assets/docs/3176vm6IE_BNfZrVpc9_xAAAAAAAAAAA.png}

\subsubsection{\texorpdfstring{\texttt{\ body\ }}{ body }}\label{parameters-body}

\href{/docs/reference/foundations/content/}{content}

{Required} {{ Positional }}

\phantomsection\label{parameters-body-positional-tooltip}
Positional parameters are specified in order, without names.

The content to align.

\href{/docs/reference/layout/}{\pandocbounded{\includesvg[keepaspectratio]{/assets/icons/16-arrow-right.svg}}}

{ Layout } { Previous page }

\href{/docs/reference/layout/alignment/}{\pandocbounded{\includesvg[keepaspectratio]{/assets/icons/16-arrow-right.svg}}}

{ Alignment } { Next page }
