\title{typst.app/docs/reference/layout/place}

\begin{itemize}
\tightlist
\item
  \href{/docs}{\includesvg[width=0.16667in,height=0.16667in]{/assets/icons/16-docs-dark.svg}}
\item
  \includesvg[width=0.16667in,height=0.16667in]{/assets/icons/16-arrow-right.svg}
\item
  \href{/docs/reference/}{Reference}
\item
  \includesvg[width=0.16667in,height=0.16667in]{/assets/icons/16-arrow-right.svg}
\item
  \href{/docs/reference/layout/}{Layout}
\item
  \includesvg[width=0.16667in,height=0.16667in]{/assets/icons/16-arrow-right.svg}
\item
  \href{/docs/reference/layout/place/}{Place}
\end{itemize}

\section{\texorpdfstring{\texttt{\ place\ } {{ Element
}}}{ place   Element }}\label{summary}

\phantomsection\label{element-tooltip}
Element functions can be customized with \texttt{\ set\ } and
\texttt{\ show\ } rules.

Places content relatively to its parent container.

Placed content can be either overlaid (the default) or floating.
Overlaid content is aligned with the parent container according to the
given
\href{/docs/reference/layout/place/\#parameters-alignment}{\texttt{\ alignment\ }}
, and shown over any other content added so far in the container.
Floating content is placed at the top or bottom of the container,
displacing other content down or up respectively. In both cases, the
content position can be adjusted with
\href{/docs/reference/layout/place/\#parameters-dx}{\texttt{\ dx\ }} and
\href{/docs/reference/layout/place/\#parameters-dy}{\texttt{\ dy\ }}
offsets without affecting the layout.

The parent can be any container such as a
\href{/docs/reference/layout/block/}{\texttt{\ block\ }} ,
\href{/docs/reference/layout/box/}{\texttt{\ box\ }} ,
\href{/docs/reference/visualize/rect/}{\texttt{\ rect\ }} , etc. A top
level \texttt{\ place\ } call will place content directly in the text
area of the current page. This can be used for absolute positioning on
the page: with a \texttt{\ top\ +\ left\ }
\href{/docs/reference/layout/place/\#parameters-alignment}{\texttt{\ alignment\ }}
, the offsets \texttt{\ dx\ } and \texttt{\ dy\ } will set the position
of the element\textquotesingle s top left corner relatively to the top
left corner of the text area. For absolute positioning on the full page
including margins, you can use \texttt{\ place\ } in
\href{/docs/reference/layout/page/\#parameters-foreground}{\texttt{\ page.foreground\ }}
or
\href{/docs/reference/layout/page/\#parameters-background}{\texttt{\ page.background\ }}
.

\subsection{Examples}\label{examples}

\begin{verbatim}
#set page(height: 120pt)
Hello, world!

#rect(
  width: 100%,
  height: 2cm,
  place(horizon + right, square()),
)

#place(
  top + left,
  dx: -5pt,
  square(size: 5pt, fill: red),
)
\end{verbatim}

\includegraphics[width=5in,height=\textheight,keepaspectratio]{/assets/docs/b3Ue37sNl2HDpslyo5trfgAAAAAAAAAA.png}

\subsection{Effect on the position of other
elements}\label{effect-on-other-elements}

Overlaid elements don\textquotesingle t take space in the flow of
content, but a \texttt{\ place\ } call inserts an invisible block-level
element in the flow. This can affect the layout by breaking the current
paragraph. To avoid this, you can wrap the \texttt{\ place\ } call in a
\href{/docs/reference/layout/box/}{\texttt{\ box\ }} when the call is
made in the middle of a paragraph. The alignment and offsets will then
be relative to this zero-size box. To make sure it
doesn\textquotesingle t interfere with spacing, the box should be
attached to a word using a word joiner.

For example, the following defines a function for attaching an
annotation to the following word:

\begin{verbatim}
#let annotate(..args) = {
  box(place(..args))
  sym.wj
  h(0pt, weak: true)
}

A placed #annotate(square(), dy: 2pt)
square in my text.
\end{verbatim}

\includegraphics[width=5in,height=\textheight,keepaspectratio]{/assets/docs/QIJqPsAAp5jqe-EB4bZF1gAAAAAAAAAA.png}

The zero-width weak spacing serves to discard spaces between the
function call and the next word.

\subsection{\texorpdfstring{{ Parameters
}}{ Parameters }}\label{parameters}

\phantomsection\label{parameters-tooltip}
Parameters are the inputs to a function. They are specified in
parentheses after the function name.

{ place } (

{ \hyperref[parameters-alignment]{}
\href{/docs/reference/foundations/auto/}{auto}
\href{/docs/reference/layout/alignment/}{alignment} , } {
\hyperref[parameters-scope]{scope :}
\href{/docs/reference/foundations/str/}{str} , } {
\hyperref[parameters-float]{float :}
\href{/docs/reference/foundations/bool/}{bool} , } {
\hyperref[parameters-clearance]{clearance :}
\href{/docs/reference/layout/length/}{length} , } {
\hyperref[parameters-dx]{dx :}
\href{/docs/reference/layout/relative/}{relative} , } {
\hyperref[parameters-dy]{dy :}
\href{/docs/reference/layout/relative/}{relative} , } {
\href{/docs/reference/foundations/content/}{content} , }

) -\textgreater{} \href{/docs/reference/foundations/content/}{content}

\subsubsection{\texorpdfstring{\texttt{\ alignment\ }}{ alignment }}\label{parameters-alignment}

\href{/docs/reference/foundations/auto/}{auto} {or}
\href{/docs/reference/layout/alignment/}{alignment}

{{ Positional }}

\phantomsection\label{parameters-alignment-positional-tooltip}
Positional parameters are specified in order, without names.

{{ Settable }}

\phantomsection\label{parameters-alignment-settable-tooltip}
Settable parameters can be customized for all following uses of the
function with a \texttt{\ set\ } rule.

Relative to which position in the parent container to place the content.

\begin{itemize}
\tightlist
\item
  If \texttt{\ float\ } is \texttt{\ }{\texttt{\ false\ }}\texttt{\ } ,
  then this can be any alignment other than
  \texttt{\ }{\texttt{\ auto\ }}\texttt{\ } .
\item
  If \texttt{\ float\ } is \texttt{\ }{\texttt{\ true\ }}\texttt{\ } ,
  then this must be \texttt{\ }{\texttt{\ auto\ }}\texttt{\ } ,
  \texttt{\ top\ } , or \texttt{\ bottom\ } .
\end{itemize}

When \texttt{\ float\ } is \texttt{\ }{\texttt{\ false\ }}\texttt{\ }
and no vertical alignment is specified, the content is placed at the
current position on the vertical axis.

Default: \texttt{\ start\ }

\subsubsection{\texorpdfstring{\texttt{\ scope\ }}{ scope }}\label{parameters-scope}

\href{/docs/reference/foundations/str/}{str}

{{ Settable }}

\phantomsection\label{parameters-scope-settable-tooltip}
Settable parameters can be customized for all following uses of the
function with a \texttt{\ set\ } rule.

Relative to which containing scope something is placed.

The parent scope is primarily used with figures and, for this reason,
the figure function has a mirrored
\href{/docs/reference/model/figure/\#parameters-scope}{\texttt{\ scope\ }
parameter} . Nonetheless, it can also be more generally useful to break
out of the columns. A typical example would be to
\href{/docs/guides/page-setup-guide/\#columns}{create a single-column
title section} in a two-column document.

Note that parent-scoped placement is currently only supported if
\texttt{\ float\ } is \texttt{\ }{\texttt{\ true\ }}\texttt{\ } . This
may change in the future.

\begin{longtable}[]{@{}ll@{}}
\toprule\noalign{}
Variant & Details \\
\midrule\noalign{}
\endhead
\bottomrule\noalign{}
\endlastfoot
\texttt{\ "\ column\ "\ } & Place into the current column. \\
\texttt{\ "\ parent\ "\ } & Place relative to the parent, letting the
content span over all columns. \\
\end{longtable}

Default: \texttt{\ }{\texttt{\ "column"\ }}\texttt{\ }

\includesvg[width=0.16667in,height=0.16667in]{/assets/icons/16-arrow-right.svg}
View example

\begin{verbatim}
#set page(height: 150pt, columns: 2)
#place(
  top + center,
  scope: "parent",
  float: true,
  rect(width: 80%, fill: aqua),
)

#lorem(25)
\end{verbatim}

\includegraphics[width=5in,height=\textheight,keepaspectratio]{/assets/docs/9xhEXBaN2g3N9Vju7GUzFwAAAAAAAAAA.png}

\subsubsection{\texorpdfstring{\texttt{\ float\ }}{ float }}\label{parameters-float}

\href{/docs/reference/foundations/bool/}{bool}

{{ Settable }}

\phantomsection\label{parameters-float-settable-tooltip}
Settable parameters can be customized for all following uses of the
function with a \texttt{\ set\ } rule.

Whether the placed element has floating layout.

Floating elements are positioned at the top or bottom of the parent
container, displacing in-flow content. They are always placed in the
in-flow order relative to each other, as well as before any content
following a later
\href{/docs/reference/layout/place/\#definitions-flush}{\texttt{\ place.flush\ }}
element.

Default: \texttt{\ }{\texttt{\ false\ }}\texttt{\ }

\includesvg[width=0.16667in,height=0.16667in]{/assets/icons/16-arrow-right.svg}
View example

\begin{verbatim}
#set page(height: 150pt)
#let note(where, body) = place(
  center + where,
  float: true,
  clearance: 6pt,
  rect(body),
)

#lorem(10)
#note(bottom)[Bottom 1]
#note(bottom)[Bottom 2]
#lorem(40)
#note(top)[Top]
#lorem(10)
\end{verbatim}

\includegraphics[width=5in,height=\textheight,keepaspectratio]{/assets/docs/t5SJ49ulSlCH5SgTOH20JAAAAAAAAAAA.png}
\includegraphics[width=5in,height=\textheight,keepaspectratio]{/assets/docs/t5SJ49ulSlCH5SgTOH20JAAAAAAAAAAB.png}

\subsubsection{\texorpdfstring{\texttt{\ clearance\ }}{ clearance }}\label{parameters-clearance}

\href{/docs/reference/layout/length/}{length}

{{ Settable }}

\phantomsection\label{parameters-clearance-settable-tooltip}
Settable parameters can be customized for all following uses of the
function with a \texttt{\ set\ } rule.

The spacing between the placed element and other elements in a floating
layout.

Has no effect if \texttt{\ float\ } is
\texttt{\ }{\texttt{\ false\ }}\texttt{\ } .

Default: \texttt{\ }{\texttt{\ 1.5em\ }}\texttt{\ }

\subsubsection{\texorpdfstring{\texttt{\ dx\ }}{ dx }}\label{parameters-dx}

\href{/docs/reference/layout/relative/}{relative}

{{ Settable }}

\phantomsection\label{parameters-dx-settable-tooltip}
Settable parameters can be customized for all following uses of the
function with a \texttt{\ set\ } rule.

The horizontal displacement of the placed content.

Default:
\texttt{\ }{\texttt{\ 0\%\ }}\texttt{\ }{\texttt{\ +\ }}\texttt{\ }{\texttt{\ 0pt\ }}\texttt{\ }

\includesvg[width=0.16667in,height=0.16667in]{/assets/icons/16-arrow-right.svg}
View example

\begin{verbatim}
#set page(height: 100pt)
#for i in range(16) {
  let amount = i * 4pt
  place(center, dx: amount - 32pt, dy: amount)[A]
}
\end{verbatim}

\includegraphics[width=5in,height=\textheight,keepaspectratio]{/assets/docs/kAqGzNrSyPcytdYDwTZgaQAAAAAAAAAA.png}

This does not affect the layout of in-flow content. In other words, the
placed content is treated as if it were wrapped in a
\href{/docs/reference/layout/move/}{\texttt{\ move\ }} element.

\subsubsection{\texorpdfstring{\texttt{\ dy\ }}{ dy }}\label{parameters-dy}

\href{/docs/reference/layout/relative/}{relative}

{{ Settable }}

\phantomsection\label{parameters-dy-settable-tooltip}
Settable parameters can be customized for all following uses of the
function with a \texttt{\ set\ } rule.

The vertical displacement of the placed content.

This does not affect the layout of in-flow content. In other words, the
placed content is treated as if it were wrapped in a
\href{/docs/reference/layout/move/}{\texttt{\ move\ }} element.

Default:
\texttt{\ }{\texttt{\ 0\%\ }}\texttt{\ }{\texttt{\ +\ }}\texttt{\ }{\texttt{\ 0pt\ }}\texttt{\ }

\subsubsection{\texorpdfstring{\texttt{\ body\ }}{ body }}\label{parameters-body}

\href{/docs/reference/foundations/content/}{content}

{Required} {{ Positional }}

\phantomsection\label{parameters-body-positional-tooltip}
Positional parameters are specified in order, without names.

The content to place.

\subsection{\texorpdfstring{{ Definitions
}}{ Definitions }}\label{definitions}

\phantomsection\label{definitions-tooltip}
Functions and types and can have associated definitions. These are
accessed by specifying the function or type, followed by a period, and
then the definition\textquotesingle s name.

\subsubsection{\texorpdfstring{\texttt{\ flush\ } {{ Element
}}}{ flush   Element }}\label{definitions-flush}

\phantomsection\label{definitions-flush-element-tooltip}
Element functions can be customized with \texttt{\ set\ } and
\texttt{\ show\ } rules.

Asks the layout algorithm to place pending floating elements before
continuing with the content.

This is useful for preventing floating figures from spilling into the
next section.

place { . } { flush } (

) -\textgreater{} \href{/docs/reference/foundations/content/}{content}

\begin{verbatim}
#lorem(15)

#figure(
  rect(width: 100%, height: 50pt),
  placement: auto,
  caption: [A rectangle],
)

#place.flush()

This text appears after the figure.
\end{verbatim}

\includegraphics[width=3.125in,height=\textheight,keepaspectratio]{/assets/docs/8qp5vfUImMtnXndzjQCsNQAAAAAAAAAA.png}
\includegraphics[width=3.125in,height=\textheight,keepaspectratio]{/assets/docs/8qp5vfUImMtnXndzjQCsNQAAAAAAAAAB.png}

\href{/docs/reference/layout/pagebreak/}{\pandocbounded{\includesvg[keepaspectratio]{/assets/icons/16-arrow-right.svg}}}

{ Page Break } { Previous page }

\href{/docs/reference/layout/ratio/}{\pandocbounded{\includesvg[keepaspectratio]{/assets/icons/16-arrow-right.svg}}}

{ Ratio } { Next page }
