\title{typst.app/docs}

\begin{itemize}
\tightlist
\item
  \href{/docs}{\includesvg[width=0.16667in,height=0.16667in]{/assets/icons/16-docs-dark.svg}}
\item
  \includesvg[width=0.16667in,height=0.16667in]{/assets/icons/16-arrow-right.svg}
\item
  \href{/docs/}{Overview}
\end{itemize}

\section{Overview}\label{overview}

Welcome to Typst\textquotesingle s documentation! Typst is a new
markup-based typesetting system for the sciences. It is designed to be
an alternative both to advanced tools like LaTeX and simpler tools like
Word and Google Docs. Our goal with Typst is to build a typesetting tool
that is highly capable \emph{and} a pleasure to use.

This documentation is split into two parts: A beginner-friendly tutorial
that introduces Typst through a practical use case and a comprehensive
reference that explains all of Typst\textquotesingle s concepts and
features.

We also invite you to join the community we\textquotesingle re building
around Typst. Typst is still a very young project, so your feedback is
more than valuable.

\href{/docs/tutorial}{\includesvg[width=0.33333in,height=0.33333in]{/assets/icons/32-tutorial-c.svg}
\textbf{Tutorial}}

Step-by-step guide to help you get started.

\href{/docs/reference}{\includesvg[width=0.33333in,height=0.33333in]{/assets/icons/32-reference-c.svg}
\textbf{Reference}}

Details about all syntax, concepts, types, and functions.
