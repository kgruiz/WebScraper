\title{typst.app/docs/reference/layout/grid}

\begin{itemize}
\tightlist
\item
  \href{/docs}{\includesvg[width=0.16667in,height=0.16667in]{/assets/icons/16-docs-dark.svg}}
\item
  \includesvg[width=0.16667in,height=0.16667in]{/assets/icons/16-arrow-right.svg}
\item
  \href{/docs/reference/}{Reference}
\item
  \includesvg[width=0.16667in,height=0.16667in]{/assets/icons/16-arrow-right.svg}
\item
  \href{/docs/reference/layout/}{Layout}
\item
  \includesvg[width=0.16667in,height=0.16667in]{/assets/icons/16-arrow-right.svg}
\item
  \href{/docs/reference/layout/grid/}{Grid}
\end{itemize}

\section{\texorpdfstring{\texttt{\ grid\ } {{ Element
}}}{ grid   Element }}\label{summary}

\phantomsection\label{element-tooltip}
Element functions can be customized with \texttt{\ set\ } and
\texttt{\ show\ } rules.

Arranges content in a grid.

The grid element allows you to arrange content in a grid. You can define
the number of rows and columns, as well as the size of the gutters
between them. There are multiple sizing modes for columns and rows that
can be used to create complex layouts.

While the grid and table elements work very similarly, they are intended
for different use cases and carry different semantics. The grid element
is intended for presentational and layout purposes, while the
\href{/docs/reference/model/table/}{\texttt{\ table\ }} element is
intended for, in broad terms, presenting multiple related data points.
In the future, Typst will annotate its output such that screenreaders
will announce content in \texttt{\ table\ } as tabular while a
grid\textquotesingle s content will be announced no different than
multiple content blocks in the document flow. Set and show rules on one
of these elements do not affect the other.

A grid\textquotesingle s sizing is determined by the track sizes
specified in the arguments. Because each of the sizing parameters
accepts the same values, we will explain them just once, here. Each
sizing argument accepts an array of individual track sizes. A track size
is either:

\begin{itemize}
\item
  \texttt{\ }{\texttt{\ auto\ }}\texttt{\ } : The track will be sized to
  fit its contents. It will be at most as large as the remaining space.
  If there is more than one \texttt{\ }{\texttt{\ auto\ }}\texttt{\ }
  track width, and together they claim more than the available space,
  the \texttt{\ }{\texttt{\ auto\ }}\texttt{\ } tracks will fairly
  distribute the available space among themselves.
\item
  A fixed or relative length (e.g.
  \texttt{\ }{\texttt{\ 10pt\ }}\texttt{\ } or
  \texttt{\ }{\texttt{\ 20\%\ }}\texttt{\ }{\texttt{\ -\ }}\texttt{\ }{\texttt{\ 1cm\ }}\texttt{\ }
  ): The track will be exactly of this size.
\item
  A fractional length (e.g. \texttt{\ }{\texttt{\ 1fr\ }}\texttt{\ } ):
  Once all other tracks have been sized, the remaining space will be
  divided among the fractional tracks according to their fractions. For
  example, if there are two fractional tracks, each with a fraction of
  \texttt{\ }{\texttt{\ 1fr\ }}\texttt{\ } , they will each take up half
  of the remaining space.
\end{itemize}

To specify a single track, the array can be omitted in favor of a single
value. To specify multiple \texttt{\ }{\texttt{\ auto\ }}\texttt{\ }
tracks, enter the number of tracks instead of an array. For example,
\texttt{\ columns:\ } \texttt{\ }{\texttt{\ 3\ }}\texttt{\ } is
equivalent to \texttt{\ columns:\ }
\texttt{\ }{\texttt{\ (\ }}\texttt{\ }{\texttt{\ auto\ }}\texttt{\ }{\texttt{\ ,\ }}\texttt{\ }{\texttt{\ auto\ }}\texttt{\ }{\texttt{\ ,\ }}\texttt{\ }{\texttt{\ auto\ }}\texttt{\ }{\texttt{\ )\ }}\texttt{\ }
.

\subsection{Examples}\label{examples}

The example below demonstrates the different track sizing options. It
also shows how you can use
\href{/docs/reference/layout/grid/\#definitions-cell}{\texttt{\ grid.cell\ }}
to make an individual cell span two grid tracks.

\begin{verbatim}
// We use `rect` to emphasize the
// area of cells.
#set rect(
  inset: 8pt,
  fill: rgb("e4e5ea"),
  width: 100%,
)

#grid(
  columns: (60pt, 1fr, 2fr),
  rows: (auto, 60pt),
  gutter: 3pt,
  rect[Fixed width, auto height],
  rect[1/3 of the remains],
  rect[2/3 of the remains],
  rect(height: 100%)[Fixed height],
  grid.cell(
    colspan: 2,
    image("tiger.jpg", width: 100%),
  ),
)
\end{verbatim}

\includegraphics[width=5in,height=\textheight,keepaspectratio]{/assets/docs/nU6HFHUP8AJwyw_E8LwJrgAAAAAAAAAA.png}

You can also
\href{/docs/reference/foundations/arguments/\#spreading}{spread} an
array of strings or content into a grid to populate its cells.

\begin{verbatim}
#grid(
  columns: 5,
  gutter: 5pt,
  ..range(25).map(str)
)
\end{verbatim}

\includegraphics[width=5in,height=\textheight,keepaspectratio]{/assets/docs/qtEXI9WWslJNDT0wWvWAggAAAAAAAAAA.png}

\subsection{Styling the grid}\label{styling-the-grid}

The grid\textquotesingle s appearance can be customized through
different parameters. These are the most important ones:

\begin{itemize}
\tightlist
\item
  \href{/docs/reference/layout/grid/\#parameters-fill}{\texttt{\ fill\ }}
  to give all cells a background
\item
  \href{/docs/reference/layout/grid/\#parameters-align}{\texttt{\ align\ }}
  to change how cells are aligned
\item
  \href{/docs/reference/layout/grid/\#parameters-inset}{\texttt{\ inset\ }}
  to optionally add internal padding to each cell
\item
  \href{/docs/reference/layout/grid/\#parameters-stroke}{\texttt{\ stroke\ }}
  to optionally enable grid lines with a certain stroke
\end{itemize}

If you need to override one of the above options for a single cell, you
can use the
\href{/docs/reference/layout/grid/\#definitions-cell}{\texttt{\ grid.cell\ }}
element. Likewise, you can override individual grid lines with the
\href{/docs/reference/layout/grid/\#definitions-hline}{\texttt{\ grid.hline\ }}
and
\href{/docs/reference/layout/grid/\#definitions-vline}{\texttt{\ grid.vline\ }}
elements.

Alternatively, if you need the appearance options to depend on a
cell\textquotesingle s position (column and row), you may specify a
function to \texttt{\ fill\ } or \texttt{\ align\ } of the form
\texttt{\ (column,\ row)\ =\textgreater{}\ value\ } . You may also use a
show rule on
\href{/docs/reference/layout/grid/\#definitions-cell}{\texttt{\ grid.cell\ }}
- see that element\textquotesingle s examples or the examples below for
more information.

Locating most of your styling in set and show rules is recommended, as
it keeps the grid\textquotesingle s or table\textquotesingle s actual
usages clean and easy to read. It also allows you to easily change the
grid\textquotesingle s appearance in one place.

\subsubsection{Stroke styling
precedence}\label{stroke-styling-precedence}

There are three ways to set the stroke of a grid cell: through
\href{/docs/reference/layout/grid/\#definitions-cell-stroke}{\texttt{\ grid\ }{\texttt{\ .\ }}\texttt{\ cell\ }
\textquotesingle s \texttt{\ stroke\ } field} , by using
\href{/docs/reference/layout/grid/\#definitions-hline}{\texttt{\ grid\ }{\texttt{\ .\ }}\texttt{\ hline\ }}
and
\href{/docs/reference/layout/grid/\#definitions-vline}{\texttt{\ grid\ }{\texttt{\ .\ }}\texttt{\ vline\ }}
, or by setting the
\href{/docs/reference/layout/grid/\#parameters-stroke}{\texttt{\ grid\ }
\textquotesingle s \texttt{\ stroke\ } field} . When multiple of these
settings are present and conflict, the \texttt{\ hline\ } and
\texttt{\ vline\ } settings take the highest precedence, followed by the
\texttt{\ cell\ } settings, and finally the \texttt{\ grid\ } settings.

Furthermore, strokes of a repeated grid header or footer will take
precedence over regular cell strokes.

\subsection{\texorpdfstring{{ Parameters
}}{ Parameters }}\label{parameters}

\phantomsection\label{parameters-tooltip}
Parameters are the inputs to a function. They are specified in
parentheses after the function name.

{ grid } (

{ \hyperref[parameters-columns]{columns :}
\href{/docs/reference/foundations/auto/}{auto}
\href{/docs/reference/foundations/int/}{int}
\href{/docs/reference/layout/relative/}{relative}
\href{/docs/reference/layout/fraction/}{fraction}
\href{/docs/reference/foundations/array/}{array} , } {
\hyperref[parameters-rows]{rows :}
\href{/docs/reference/foundations/auto/}{auto}
\href{/docs/reference/foundations/int/}{int}
\href{/docs/reference/layout/relative/}{relative}
\href{/docs/reference/layout/fraction/}{fraction}
\href{/docs/reference/foundations/array/}{array} , } {
\hyperref[parameters-gutter]{gutter :}
\href{/docs/reference/foundations/auto/}{auto}
\href{/docs/reference/foundations/int/}{int}
\href{/docs/reference/layout/relative/}{relative}
\href{/docs/reference/layout/fraction/}{fraction}
\href{/docs/reference/foundations/array/}{array} , } {
\hyperref[parameters-column-gutter]{column-gutter :}
\href{/docs/reference/foundations/auto/}{auto}
\href{/docs/reference/foundations/int/}{int}
\href{/docs/reference/layout/relative/}{relative}
\href{/docs/reference/layout/fraction/}{fraction}
\href{/docs/reference/foundations/array/}{array} , } {
\hyperref[parameters-row-gutter]{row-gutter :}
\href{/docs/reference/foundations/auto/}{auto}
\href{/docs/reference/foundations/int/}{int}
\href{/docs/reference/layout/relative/}{relative}
\href{/docs/reference/layout/fraction/}{fraction}
\href{/docs/reference/foundations/array/}{array} , } {
\hyperref[parameters-fill]{fill :}
\href{/docs/reference/foundations/none/}{none}
\href{/docs/reference/visualize/color/}{color}
\href{/docs/reference/visualize/gradient/}{gradient}
\href{/docs/reference/foundations/array/}{array}
\href{/docs/reference/visualize/pattern/}{pattern}
\href{/docs/reference/foundations/function/}{function} , } {
\hyperref[parameters-align]{align :}
\href{/docs/reference/foundations/auto/}{auto}
\href{/docs/reference/foundations/array/}{array}
\href{/docs/reference/layout/alignment/}{alignment}
\href{/docs/reference/foundations/function/}{function} , } {
\hyperref[parameters-stroke]{stroke :}
\href{/docs/reference/foundations/none/}{none}
\href{/docs/reference/layout/length/}{length}
\href{/docs/reference/visualize/color/}{color}
\href{/docs/reference/visualize/gradient/}{gradient}
\href{/docs/reference/foundations/array/}{array}
\href{/docs/reference/visualize/stroke/}{stroke}
\href{/docs/reference/visualize/pattern/}{pattern}
\href{/docs/reference/foundations/dictionary/}{dictionary}
\href{/docs/reference/foundations/function/}{function} , } {
\hyperref[parameters-inset]{inset :}
\href{/docs/reference/layout/relative/}{relative}
\href{/docs/reference/foundations/array/}{array}
\href{/docs/reference/foundations/dictionary/}{dictionary}
\href{/docs/reference/foundations/function/}{function} , } {
\hyperref[parameters-children]{..}
\href{/docs/reference/foundations/content/}{content} , }

) -\textgreater{} \href{/docs/reference/foundations/content/}{content}

\subsubsection{\texorpdfstring{\texttt{\ columns\ }}{ columns }}\label{parameters-columns}

\href{/docs/reference/foundations/auto/}{auto} {or}
\href{/docs/reference/foundations/int/}{int} {or}
\href{/docs/reference/layout/relative/}{relative} {or}
\href{/docs/reference/layout/fraction/}{fraction} {or}
\href{/docs/reference/foundations/array/}{array}

{{ Settable }}

\phantomsection\label{parameters-columns-settable-tooltip}
Settable parameters can be customized for all following uses of the
function with a \texttt{\ set\ } rule.

The column sizes.

Either specify a track size array or provide an integer to create a grid
with that many \texttt{\ }{\texttt{\ auto\ }}\texttt{\ } -sized columns.
Note that opposed to rows and gutters, providing a single track size
will only ever create a single column.

Default:
\texttt{\ }{\texttt{\ (\ }}\texttt{\ }{\texttt{\ )\ }}\texttt{\ }

\subsubsection{\texorpdfstring{\texttt{\ rows\ }}{ rows }}\label{parameters-rows}

\href{/docs/reference/foundations/auto/}{auto} {or}
\href{/docs/reference/foundations/int/}{int} {or}
\href{/docs/reference/layout/relative/}{relative} {or}
\href{/docs/reference/layout/fraction/}{fraction} {or}
\href{/docs/reference/foundations/array/}{array}

{{ Settable }}

\phantomsection\label{parameters-rows-settable-tooltip}
Settable parameters can be customized for all following uses of the
function with a \texttt{\ set\ } rule.

The row sizes.

If there are more cells than fit the defined rows, the last row is
repeated until there are no more cells.

Default:
\texttt{\ }{\texttt{\ (\ }}\texttt{\ }{\texttt{\ )\ }}\texttt{\ }

\subsubsection{\texorpdfstring{\texttt{\ gutter\ }}{ gutter }}\label{parameters-gutter}

\href{/docs/reference/foundations/auto/}{auto} {or}
\href{/docs/reference/foundations/int/}{int} {or}
\href{/docs/reference/layout/relative/}{relative} {or}
\href{/docs/reference/layout/fraction/}{fraction} {or}
\href{/docs/reference/foundations/array/}{array}

{{ Settable }}

\phantomsection\label{parameters-gutter-settable-tooltip}
Settable parameters can be customized for all following uses of the
function with a \texttt{\ set\ } rule.

The gaps between rows and columns.

If there are more gutters than defined sizes, the last gutter is
repeated.

This is a shorthand to set \texttt{\ column-gutter\ } and
\texttt{\ row-gutter\ } to the same value.

Default:
\texttt{\ }{\texttt{\ (\ }}\texttt{\ }{\texttt{\ )\ }}\texttt{\ }

\subsubsection{\texorpdfstring{\texttt{\ column-gutter\ }}{ column-gutter }}\label{parameters-column-gutter}

\href{/docs/reference/foundations/auto/}{auto} {or}
\href{/docs/reference/foundations/int/}{int} {or}
\href{/docs/reference/layout/relative/}{relative} {or}
\href{/docs/reference/layout/fraction/}{fraction} {or}
\href{/docs/reference/foundations/array/}{array}

{{ Settable }}

\phantomsection\label{parameters-column-gutter-settable-tooltip}
Settable parameters can be customized for all following uses of the
function with a \texttt{\ set\ } rule.

The gaps between columns.

Default:
\texttt{\ }{\texttt{\ (\ }}\texttt{\ }{\texttt{\ )\ }}\texttt{\ }

\subsubsection{\texorpdfstring{\texttt{\ row-gutter\ }}{ row-gutter }}\label{parameters-row-gutter}

\href{/docs/reference/foundations/auto/}{auto} {or}
\href{/docs/reference/foundations/int/}{int} {or}
\href{/docs/reference/layout/relative/}{relative} {or}
\href{/docs/reference/layout/fraction/}{fraction} {or}
\href{/docs/reference/foundations/array/}{array}

{{ Settable }}

\phantomsection\label{parameters-row-gutter-settable-tooltip}
Settable parameters can be customized for all following uses of the
function with a \texttt{\ set\ } rule.

The gaps between rows.

Default:
\texttt{\ }{\texttt{\ (\ }}\texttt{\ }{\texttt{\ )\ }}\texttt{\ }

\subsubsection{\texorpdfstring{\texttt{\ fill\ }}{ fill }}\label{parameters-fill}

\href{/docs/reference/foundations/none/}{none} {or}
\href{/docs/reference/visualize/color/}{color} {or}
\href{/docs/reference/visualize/gradient/}{gradient} {or}
\href{/docs/reference/foundations/array/}{array} {or}
\href{/docs/reference/visualize/pattern/}{pattern} {or}
\href{/docs/reference/foundations/function/}{function}

{{ Settable }}

\phantomsection\label{parameters-fill-settable-tooltip}
Settable parameters can be customized for all following uses of the
function with a \texttt{\ set\ } rule.

How to fill the cells.

This can be a color or a function that returns a color. The function
receives the cells\textquotesingle{} column and row indices, starting
from zero. This can be used to implement striped grids.

Default: \texttt{\ }{\texttt{\ none\ }}\texttt{\ }

\includesvg[width=0.16667in,height=0.16667in]{/assets/icons/16-arrow-right.svg}
View example

\begin{verbatim}
#grid(
  fill: (x, y) =>
    if calc.even(x + y) { luma(230) }
    else { white },
  align: center + horizon,
  columns: 4,
  inset: 2pt,
  [X], [O], [X], [O],
  [O], [X], [O], [X],
  [X], [O], [X], [O],
  [O], [X], [O], [X],
)
\end{verbatim}

\includegraphics[width=5in,height=\textheight,keepaspectratio]{/assets/docs/YWpStHlSHlCZTmUmBJs9XQAAAAAAAAAA.png}

\subsubsection{\texorpdfstring{\texttt{\ align\ }}{ align }}\label{parameters-align}

\href{/docs/reference/foundations/auto/}{auto} {or}
\href{/docs/reference/foundations/array/}{array} {or}
\href{/docs/reference/layout/alignment/}{alignment} {or}
\href{/docs/reference/foundations/function/}{function}

{{ Settable }}

\phantomsection\label{parameters-align-settable-tooltip}
Settable parameters can be customized for all following uses of the
function with a \texttt{\ set\ } rule.

How to align the cells\textquotesingle{} content.

This can either be a single alignment, an array of alignments
(corresponding to each column) or a function that returns an alignment.
The function receives the cells\textquotesingle{} column and row
indices, starting from zero. If set to
\texttt{\ }{\texttt{\ auto\ }}\texttt{\ } , the outer alignment is used.

You can find an example for this argument at the
\href{/docs/reference/model/table/\#parameters-align}{\texttt{\ table.align\ }}
parameter.

Default: \texttt{\ }{\texttt{\ auto\ }}\texttt{\ }

\subsubsection{\texorpdfstring{\texttt{\ stroke\ }}{ stroke }}\label{parameters-stroke}

\href{/docs/reference/foundations/none/}{none} {or}
\href{/docs/reference/layout/length/}{length} {or}
\href{/docs/reference/visualize/color/}{color} {or}
\href{/docs/reference/visualize/gradient/}{gradient} {or}
\href{/docs/reference/foundations/array/}{array} {or}
\href{/docs/reference/visualize/stroke/}{stroke} {or}
\href{/docs/reference/visualize/pattern/}{pattern} {or}
\href{/docs/reference/foundations/dictionary/}{dictionary} {or}
\href{/docs/reference/foundations/function/}{function}

{{ Settable }}

\phantomsection\label{parameters-stroke-settable-tooltip}
Settable parameters can be customized for all following uses of the
function with a \texttt{\ set\ } rule.

How to \href{/docs/reference/visualize/stroke/}{stroke} the cells.

Grids have no strokes by default, which can be changed by setting this
option to the desired stroke.

If it is necessary to place lines which can cross spacing between cells
produced by the \texttt{\ gutter\ } option, or to override the stroke
between multiple specific cells, consider specifying one or more of
\href{/docs/reference/layout/grid/\#definitions-hline}{\texttt{\ grid.hline\ }}
and
\href{/docs/reference/layout/grid/\#definitions-vline}{\texttt{\ grid.vline\ }}
alongside your grid cells.

Default:
\texttt{\ }{\texttt{\ (\ }}\texttt{\ }{\texttt{\ :\ }}\texttt{\ }{\texttt{\ )\ }}\texttt{\ }

\includesvg[width=0.16667in,height=0.16667in]{/assets/icons/16-arrow-right.svg}
View example

\begin{verbatim}
#set page(height: 13em, width: 26em)

#let cv(..jobs) = grid(
    columns: 2,
    inset: 5pt,
    stroke: (x, y) => if x == 0 and y > 0 {
      (right: (
        paint: luma(180),
        thickness: 1.5pt,
        dash: "dotted"
      ))
    },
    grid.header(grid.cell(colspan: 2)[
      *Professional Experience*
      #box(width: 1fr, line(length: 100%, stroke: luma(180)))
    ]),
    ..{
      let last = none
      for job in jobs.pos() {
        (
          if job.year != last [*#job.year*],
          [
            *#job.company* - #job.role _(#job.timeframe)_ \
            #job.details
          ]
        )
        last = job.year
      }
    }
  )

  #cv(
    (
      year: 2012,
      company: [Pear Seed & Co.],
      role: [Lead Engineer],
      timeframe: [Jul - Dec],
      details: [
        - Raised engineers from 3x to 10x
        - Did a great job
      ],
    ),
    (
      year: 2012,
      company: [Mega Corp.],
      role: [VP of Sales],
      timeframe: [Mar - Jun],
      details: [- Closed tons of customers],
    ),
    (
      year: 2013,
      company: [Tiny Co.],
      role: [CEO],
      timeframe: [Jan - Dec],
      details: [- Delivered 4x more shareholder value],
    ),
    (
      year: 2014,
      company: [Glorbocorp Ltd],
      role: [CTO],
      timeframe: [Jan - Mar],
      details: [- Drove containerization forward],
    ),
  )
\end{verbatim}

\includegraphics[width=5.95833in,height=\textheight,keepaspectratio]{/assets/docs/5kfvlcbAPUFkWJtXr3FdMgAAAAAAAAAA.png}
\includegraphics[width=5.95833in,height=\textheight,keepaspectratio]{/assets/docs/5kfvlcbAPUFkWJtXr3FdMgAAAAAAAAAB.png}

\subsubsection{\texorpdfstring{\texttt{\ inset\ }}{ inset }}\label{parameters-inset}

\href{/docs/reference/layout/relative/}{relative} {or}
\href{/docs/reference/foundations/array/}{array} {or}
\href{/docs/reference/foundations/dictionary/}{dictionary} {or}
\href{/docs/reference/foundations/function/}{function}

{{ Settable }}

\phantomsection\label{parameters-inset-settable-tooltip}
Settable parameters can be customized for all following uses of the
function with a \texttt{\ set\ } rule.

How much to pad the cells\textquotesingle{} content.

You can find an example for this argument at the
\href{/docs/reference/model/table/\#parameters-inset}{\texttt{\ table.inset\ }}
parameter.

Default:
\texttt{\ }{\texttt{\ (\ }}\texttt{\ }{\texttt{\ :\ }}\texttt{\ }{\texttt{\ )\ }}\texttt{\ }

\subsubsection{\texorpdfstring{\texttt{\ children\ }}{ children }}\label{parameters-children}

\href{/docs/reference/foundations/content/}{content}

{Required} {{ Positional }}

\phantomsection\label{parameters-children-positional-tooltip}
Positional parameters are specified in order, without names.

{{ Variadic }}

\phantomsection\label{parameters-children-variadic-tooltip}
Variadic parameters can be specified multiple times.

The contents of the grid cells, plus any extra grid lines specified with
the
\href{/docs/reference/layout/grid/\#definitions-hline}{\texttt{\ grid.hline\ }}
and
\href{/docs/reference/layout/grid/\#definitions-vline}{\texttt{\ grid.vline\ }}
elements.

The cells are populated in row-major order.

\subsection{\texorpdfstring{{ Definitions
}}{ Definitions }}\label{definitions}

\phantomsection\label{definitions-tooltip}
Functions and types and can have associated definitions. These are
accessed by specifying the function or type, followed by a period, and
then the definition\textquotesingle s name.

\subsubsection{\texorpdfstring{\texttt{\ cell\ } {{ Element
}}}{ cell   Element }}\label{definitions-cell}

\phantomsection\label{definitions-cell-element-tooltip}
Element functions can be customized with \texttt{\ set\ } and
\texttt{\ show\ } rules.

A cell in the grid. You can use this function in the argument list of a
grid to override grid style properties for an individual cell or
manually positioning it within the grid. You can also use this function
in show rules to apply certain styles to multiple cells at once.

For example, you can override the position and stroke for a single cell:

grid { . } { cell } (

{ \href{/docs/reference/foundations/content/}{content} , } {
\hyperref[definitions-cell-parameters-x]{x :}
\href{/docs/reference/foundations/auto/}{auto}
\href{/docs/reference/foundations/int/}{int} , } {
\hyperref[definitions-cell-parameters-y]{y :}
\href{/docs/reference/foundations/auto/}{auto}
\href{/docs/reference/foundations/int/}{int} , } {
\hyperref[definitions-cell-parameters-colspan]{colspan :}
\href{/docs/reference/foundations/int/}{int} , } {
\hyperref[definitions-cell-parameters-rowspan]{rowspan :}
\href{/docs/reference/foundations/int/}{int} , } {
\hyperref[definitions-cell-parameters-fill]{fill :}
\href{/docs/reference/foundations/none/}{none}
\href{/docs/reference/foundations/auto/}{auto}
\href{/docs/reference/visualize/color/}{color}
\href{/docs/reference/visualize/gradient/}{gradient}
\href{/docs/reference/visualize/pattern/}{pattern} , } {
\hyperref[definitions-cell-parameters-align]{align :}
\href{/docs/reference/foundations/auto/}{auto}
\href{/docs/reference/layout/alignment/}{alignment} , } {
\hyperref[definitions-cell-parameters-inset]{inset :}
\href{/docs/reference/foundations/auto/}{auto}
\href{/docs/reference/layout/relative/}{relative}
\href{/docs/reference/foundations/dictionary/}{dictionary} , } {
\hyperref[definitions-cell-parameters-stroke]{stroke :}
\href{/docs/reference/foundations/none/}{none}
\href{/docs/reference/layout/length/}{length}
\href{/docs/reference/visualize/color/}{color}
\href{/docs/reference/visualize/gradient/}{gradient}
\href{/docs/reference/visualize/stroke/}{stroke}
\href{/docs/reference/visualize/pattern/}{pattern}
\href{/docs/reference/foundations/dictionary/}{dictionary} , } {
\hyperref[definitions-cell-parameters-breakable]{breakable :}
\href{/docs/reference/foundations/auto/}{auto}
\href{/docs/reference/foundations/bool/}{bool} , }

) -\textgreater{} \href{/docs/reference/foundations/content/}{content}

\begin{verbatim}
#set text(15pt, font: "Noto Sans Symbols 2")
#show regex("[♚-♟︎]"): set text(fill: rgb("21212A"))
#show regex("[♔-♙]"): set text(fill: rgb("111015"))

#grid(
  fill: (x, y) => rgb(
    if calc.odd(x + y) { "7F8396" }
    else { "EFF0F3" }
  ),
  columns: (1em,) * 8,
  rows: 1em,
  align: center + horizon,

  [♖], [♘], [♗], [♕], [♔], [♗], [♘], [♖],
  [♙], [♙], [♙], [♙], [],  [♙], [♙], [♙],
  grid.cell(
    x: 4, y: 3,
    stroke: blue.transparentize(60%)
  )[♙],

  ..(grid.cell(y: 6)[♟],) * 8,
  ..([♜], [♞], [♝], [♛], [♚], [♝], [♞], [♜])
    .map(grid.cell.with(y: 7)),
)
\end{verbatim}

\includegraphics[width=3.125in,height=\textheight,keepaspectratio]{/assets/docs/hagMogxzgYo1z-9CqYbmiQAAAAAAAAAA.png}

You may also apply a show rule on \texttt{\ grid.cell\ } to style all
cells at once, which allows you, for example, to apply styles based on a
cell\textquotesingle s position. Refer to the examples of the
\href{/docs/reference/model/table/\#definitions-cell}{\texttt{\ table.cell\ }}
element to learn more about this.

\paragraph{\texorpdfstring{\texttt{\ body\ }}{ body }}\label{definitions-cell-body}

\href{/docs/reference/foundations/content/}{content}

{Required} {{ Positional }}

\phantomsection\label{definitions-cell-body-positional-tooltip}
Positional parameters are specified in order, without names.

The cell\textquotesingle s body.

\paragraph{\texorpdfstring{\texttt{\ x\ }}{ x }}\label{definitions-cell-x}

\href{/docs/reference/foundations/auto/}{auto} {or}
\href{/docs/reference/foundations/int/}{int}

{{ Settable }}

\phantomsection\label{definitions-cell-x-settable-tooltip}
Settable parameters can be customized for all following uses of the
function with a \texttt{\ set\ } rule.

The cell\textquotesingle s column (zero-indexed). This field may be used
in show rules to style a cell depending on its column.

You may override this field to pick in which column the cell must be
placed. If no row ( \texttt{\ y\ } ) is chosen, the cell will be placed
in the first row (starting at row 0) with that column available (or a
new row if none). If both \texttt{\ x\ } and \texttt{\ y\ } are chosen,
however, the cell will be placed in that exact position. An error is
raised if that position is not available (thus, it is usually wise to
specify cells with a custom position before cells with automatic
positions).

Default: \texttt{\ }{\texttt{\ auto\ }}\texttt{\ }

\includesvg[width=0.16667in,height=0.16667in]{/assets/icons/16-arrow-right.svg}
View example

\begin{verbatim}
#let circ(c) = circle(
    fill: c, width: 5mm
)

#grid(
  columns: 4,
  rows: 7mm,
  stroke: .5pt + blue,
  align: center + horizon,
  inset: 1mm,

  grid.cell(x: 2, y: 2, circ(aqua)),
  circ(yellow),
  grid.cell(x: 3, circ(green)),
  circ(black),
)
\end{verbatim}

\includegraphics[width=5in,height=\textheight,keepaspectratio]{/assets/docs/1ClWJM7tWFhsIyNZJlD1owAAAAAAAAAA.png}

\paragraph{\texorpdfstring{\texttt{\ y\ }}{ y }}\label{definitions-cell-y}

\href{/docs/reference/foundations/auto/}{auto} {or}
\href{/docs/reference/foundations/int/}{int}

{{ Settable }}

\phantomsection\label{definitions-cell-y-settable-tooltip}
Settable parameters can be customized for all following uses of the
function with a \texttt{\ set\ } rule.

The cell\textquotesingle s row (zero-indexed). This field may be used in
show rules to style a cell depending on its row.

You may override this field to pick in which row the cell must be
placed. If no column ( \texttt{\ x\ } ) is chosen, the cell will be
placed in the first column (starting at column 0) available in the
chosen row. If all columns in the chosen row are already occupied, an
error is raised.

Default: \texttt{\ }{\texttt{\ auto\ }}\texttt{\ }

\includesvg[width=0.16667in,height=0.16667in]{/assets/icons/16-arrow-right.svg}
View example

\begin{verbatim}
#let tri(c) = polygon.regular(
  fill: c,
  size: 5mm,
  vertices: 3,
)

#grid(
  columns: 2,
  stroke: blue,
  inset: 1mm,

  tri(black),
  grid.cell(y: 1, tri(teal)),
  grid.cell(y: 1, tri(red)),
  grid.cell(y: 2, tri(orange))
)
\end{verbatim}

\includegraphics[width=5in,height=\textheight,keepaspectratio]{/assets/docs/KqESjHcjVY-CskMVImXGSAAAAAAAAAAA.png}

\paragraph{\texorpdfstring{\texttt{\ colspan\ }}{ colspan }}\label{definitions-cell-colspan}

\href{/docs/reference/foundations/int/}{int}

{{ Settable }}

\phantomsection\label{definitions-cell-colspan-settable-tooltip}
Settable parameters can be customized for all following uses of the
function with a \texttt{\ set\ } rule.

The amount of columns spanned by this cell.

Default: \texttt{\ }{\texttt{\ 1\ }}\texttt{\ }

\paragraph{\texorpdfstring{\texttt{\ rowspan\ }}{ rowspan }}\label{definitions-cell-rowspan}

\href{/docs/reference/foundations/int/}{int}

{{ Settable }}

\phantomsection\label{definitions-cell-rowspan-settable-tooltip}
Settable parameters can be customized for all following uses of the
function with a \texttt{\ set\ } rule.

The amount of rows spanned by this cell.

Default: \texttt{\ }{\texttt{\ 1\ }}\texttt{\ }

\paragraph{\texorpdfstring{\texttt{\ fill\ }}{ fill }}\label{definitions-cell-fill}

\href{/docs/reference/foundations/none/}{none} {or}
\href{/docs/reference/foundations/auto/}{auto} {or}
\href{/docs/reference/visualize/color/}{color} {or}
\href{/docs/reference/visualize/gradient/}{gradient} {or}
\href{/docs/reference/visualize/pattern/}{pattern}

{{ Settable }}

\phantomsection\label{definitions-cell-fill-settable-tooltip}
Settable parameters can be customized for all following uses of the
function with a \texttt{\ set\ } rule.

The cell\textquotesingle s
\href{/docs/reference/layout/grid/\#parameters-fill}{fill} override.

Default: \texttt{\ }{\texttt{\ auto\ }}\texttt{\ }

\paragraph{\texorpdfstring{\texttt{\ align\ }}{ align }}\label{definitions-cell-align}

\href{/docs/reference/foundations/auto/}{auto} {or}
\href{/docs/reference/layout/alignment/}{alignment}

{{ Settable }}

\phantomsection\label{definitions-cell-align-settable-tooltip}
Settable parameters can be customized for all following uses of the
function with a \texttt{\ set\ } rule.

The cell\textquotesingle s
\href{/docs/reference/layout/grid/\#parameters-align}{alignment}
override.

Default: \texttt{\ }{\texttt{\ auto\ }}\texttt{\ }

\paragraph{\texorpdfstring{\texttt{\ inset\ }}{ inset }}\label{definitions-cell-inset}

\href{/docs/reference/foundations/auto/}{auto} {or}
\href{/docs/reference/layout/relative/}{relative} {or}
\href{/docs/reference/foundations/dictionary/}{dictionary}

{{ Settable }}

\phantomsection\label{definitions-cell-inset-settable-tooltip}
Settable parameters can be customized for all following uses of the
function with a \texttt{\ set\ } rule.

The cell\textquotesingle s
\href{/docs/reference/layout/grid/\#parameters-inset}{inset} override.

Default: \texttt{\ }{\texttt{\ auto\ }}\texttt{\ }

\paragraph{\texorpdfstring{\texttt{\ stroke\ }}{ stroke }}\label{definitions-cell-stroke}

\href{/docs/reference/foundations/none/}{none} {or}
\href{/docs/reference/layout/length/}{length} {or}
\href{/docs/reference/visualize/color/}{color} {or}
\href{/docs/reference/visualize/gradient/}{gradient} {or}
\href{/docs/reference/visualize/stroke/}{stroke} {or}
\href{/docs/reference/visualize/pattern/}{pattern} {or}
\href{/docs/reference/foundations/dictionary/}{dictionary}

{{ Settable }}

\phantomsection\label{definitions-cell-stroke-settable-tooltip}
Settable parameters can be customized for all following uses of the
function with a \texttt{\ set\ } rule.

The cell\textquotesingle s
\href{/docs/reference/layout/grid/\#parameters-stroke}{stroke} override.

Default:
\texttt{\ }{\texttt{\ (\ }}\texttt{\ }{\texttt{\ :\ }}\texttt{\ }{\texttt{\ )\ }}\texttt{\ }

\paragraph{\texorpdfstring{\texttt{\ breakable\ }}{ breakable }}\label{definitions-cell-breakable}

\href{/docs/reference/foundations/auto/}{auto} {or}
\href{/docs/reference/foundations/bool/}{bool}

{{ Settable }}

\phantomsection\label{definitions-cell-breakable-settable-tooltip}
Settable parameters can be customized for all following uses of the
function with a \texttt{\ set\ } rule.

Whether rows spanned by this cell can be placed in different pages. When
equal to \texttt{\ }{\texttt{\ auto\ }}\texttt{\ } , a cell spanning
only fixed-size rows is unbreakable, while a cell spanning at least one
\texttt{\ }{\texttt{\ auto\ }}\texttt{\ } -sized row is breakable.

Default: \texttt{\ }{\texttt{\ auto\ }}\texttt{\ }

\subsubsection{\texorpdfstring{\texttt{\ hline\ } {{ Element
}}}{ hline   Element }}\label{definitions-hline}

\phantomsection\label{definitions-hline-element-tooltip}
Element functions can be customized with \texttt{\ set\ } and
\texttt{\ show\ } rules.

A horizontal line in the grid.

Overrides any per-cell stroke, including stroke specified through the
grid\textquotesingle s \texttt{\ stroke\ } field. Can cross spacing
between cells created through the grid\textquotesingle s
\texttt{\ column-gutter\ } option.

An example for this function can be found at the
\href{/docs/reference/model/table/\#definitions-hline}{\texttt{\ table.hline\ }}
element.

grid { . } { hline } (

{ \hyperref[definitions-hline-parameters-y]{y :}
\href{/docs/reference/foundations/auto/}{auto}
\href{/docs/reference/foundations/int/}{int} , } {
\hyperref[definitions-hline-parameters-start]{start :}
\href{/docs/reference/foundations/int/}{int} , } {
\hyperref[definitions-hline-parameters-end]{end :}
\href{/docs/reference/foundations/none/}{none}
\href{/docs/reference/foundations/int/}{int} , } {
\hyperref[definitions-hline-parameters-stroke]{stroke :}
\href{/docs/reference/foundations/none/}{none}
\href{/docs/reference/layout/length/}{length}
\href{/docs/reference/visualize/color/}{color}
\href{/docs/reference/visualize/gradient/}{gradient}
\href{/docs/reference/visualize/stroke/}{stroke}
\href{/docs/reference/visualize/pattern/}{pattern}
\href{/docs/reference/foundations/dictionary/}{dictionary} , } {
\hyperref[definitions-hline-parameters-position]{position :}
\href{/docs/reference/layout/alignment/}{alignment} , }

) -\textgreater{} \href{/docs/reference/foundations/content/}{content}

\paragraph{\texorpdfstring{\texttt{\ y\ }}{ y }}\label{definitions-hline-y}

\href{/docs/reference/foundations/auto/}{auto} {or}
\href{/docs/reference/foundations/int/}{int}

{{ Settable }}

\phantomsection\label{definitions-hline-y-settable-tooltip}
Settable parameters can be customized for all following uses of the
function with a \texttt{\ set\ } rule.

The row above which the horizontal line is placed (zero-indexed). If the
\texttt{\ position\ } field is set to \texttt{\ bottom\ } , the line is
placed below the row with the given index instead (see that
field\textquotesingle s docs for details).

Specifying \texttt{\ }{\texttt{\ auto\ }}\texttt{\ } causes the line to
be placed at the row below the last automatically positioned cell (that
is, cell without coordinate overrides) before the line among the
grid\textquotesingle s children. If there is no such cell before the
line, it is placed at the top of the grid (row 0). Note that specifying
for this option exactly the total amount of rows in the grid causes this
horizontal line to override the bottom border of the grid, while a value
of 0 overrides the top border.

Default: \texttt{\ }{\texttt{\ auto\ }}\texttt{\ }

\paragraph{\texorpdfstring{\texttt{\ start\ }}{ start }}\label{definitions-hline-start}

\href{/docs/reference/foundations/int/}{int}

{{ Settable }}

\phantomsection\label{definitions-hline-start-settable-tooltip}
Settable parameters can be customized for all following uses of the
function with a \texttt{\ set\ } rule.

The column at which the horizontal line starts (zero-indexed,
inclusive).

Default: \texttt{\ }{\texttt{\ 0\ }}\texttt{\ }

\paragraph{\texorpdfstring{\texttt{\ end\ }}{ end }}\label{definitions-hline-end}

\href{/docs/reference/foundations/none/}{none} {or}
\href{/docs/reference/foundations/int/}{int}

{{ Settable }}

\phantomsection\label{definitions-hline-end-settable-tooltip}
Settable parameters can be customized for all following uses of the
function with a \texttt{\ set\ } rule.

The column before which the horizontal line ends (zero-indexed,
exclusive). Therefore, the horizontal line will be drawn up to and
across column \texttt{\ end\ -\ 1\ } .

A value equal to \texttt{\ }{\texttt{\ none\ }}\texttt{\ } or to the
amount of columns causes it to extend all the way towards the end of the
grid.

Default: \texttt{\ }{\texttt{\ none\ }}\texttt{\ }

\paragraph{\texorpdfstring{\texttt{\ stroke\ }}{ stroke }}\label{definitions-hline-stroke}

\href{/docs/reference/foundations/none/}{none} {or}
\href{/docs/reference/layout/length/}{length} {or}
\href{/docs/reference/visualize/color/}{color} {or}
\href{/docs/reference/visualize/gradient/}{gradient} {or}
\href{/docs/reference/visualize/stroke/}{stroke} {or}
\href{/docs/reference/visualize/pattern/}{pattern} {or}
\href{/docs/reference/foundations/dictionary/}{dictionary}

{{ Settable }}

\phantomsection\label{definitions-hline-stroke-settable-tooltip}
Settable parameters can be customized for all following uses of the
function with a \texttt{\ set\ } rule.

The line\textquotesingle s stroke.

Specifying \texttt{\ }{\texttt{\ none\ }}\texttt{\ } removes any lines
previously placed across this line\textquotesingle s range, including
hlines or per-cell stroke below it.

Default:
\texttt{\ }{\texttt{\ 1pt\ }}\texttt{\ }{\texttt{\ +\ }}\texttt{\ black\ }

\paragraph{\texorpdfstring{\texttt{\ position\ }}{ position }}\label{definitions-hline-position}

\href{/docs/reference/layout/alignment/}{alignment}

{{ Settable }}

\phantomsection\label{definitions-hline-position-settable-tooltip}
Settable parameters can be customized for all following uses of the
function with a \texttt{\ set\ } rule.

The position at which the line is placed, given its row ( \texttt{\ y\ }
) - either \texttt{\ top\ } to draw above it or \texttt{\ bottom\ } to
draw below it.

This setting is only relevant when row gutter is enabled (and
shouldn\textquotesingle t be used otherwise - prefer just increasing the
\texttt{\ y\ } field by one instead), since then the position below a
row becomes different from the position above the next row due to the
spacing between both.

Default: \texttt{\ top\ }

\subsubsection{\texorpdfstring{\texttt{\ vline\ } {{ Element
}}}{ vline   Element }}\label{definitions-vline}

\phantomsection\label{definitions-vline-element-tooltip}
Element functions can be customized with \texttt{\ set\ } and
\texttt{\ show\ } rules.

A vertical line in the grid.

Overrides any per-cell stroke, including stroke specified through the
grid\textquotesingle s \texttt{\ stroke\ } field. Can cross spacing
between cells created through the grid\textquotesingle s
\texttt{\ row-gutter\ } option.

grid { . } { vline } (

{ \hyperref[definitions-vline-parameters-x]{x :}
\href{/docs/reference/foundations/auto/}{auto}
\href{/docs/reference/foundations/int/}{int} , } {
\hyperref[definitions-vline-parameters-start]{start :}
\href{/docs/reference/foundations/int/}{int} , } {
\hyperref[definitions-vline-parameters-end]{end :}
\href{/docs/reference/foundations/none/}{none}
\href{/docs/reference/foundations/int/}{int} , } {
\hyperref[definitions-vline-parameters-stroke]{stroke :}
\href{/docs/reference/foundations/none/}{none}
\href{/docs/reference/layout/length/}{length}
\href{/docs/reference/visualize/color/}{color}
\href{/docs/reference/visualize/gradient/}{gradient}
\href{/docs/reference/visualize/stroke/}{stroke}
\href{/docs/reference/visualize/pattern/}{pattern}
\href{/docs/reference/foundations/dictionary/}{dictionary} , } {
\hyperref[definitions-vline-parameters-position]{position :}
\href{/docs/reference/layout/alignment/}{alignment} , }

) -\textgreater{} \href{/docs/reference/foundations/content/}{content}

\paragraph{\texorpdfstring{\texttt{\ x\ }}{ x }}\label{definitions-vline-x}

\href{/docs/reference/foundations/auto/}{auto} {or}
\href{/docs/reference/foundations/int/}{int}

{{ Settable }}

\phantomsection\label{definitions-vline-x-settable-tooltip}
Settable parameters can be customized for all following uses of the
function with a \texttt{\ set\ } rule.

The column before which the horizontal line is placed (zero-indexed). If
the \texttt{\ position\ } field is set to \texttt{\ end\ } , the line is
placed after the column with the given index instead (see that
field\textquotesingle s docs for details).

Specifying \texttt{\ }{\texttt{\ auto\ }}\texttt{\ } causes the line to
be placed at the column after the last automatically positioned cell
(that is, cell without coordinate overrides) before the line among the
grid\textquotesingle s children. If there is no such cell before the
line, it is placed before the grid\textquotesingle s first column
(column 0). Note that specifying for this option exactly the total
amount of columns in the grid causes this vertical line to override the
end border of the grid (right in LTR, left in RTL), while a value of 0
overrides the start border (left in LTR, right in RTL).

Default: \texttt{\ }{\texttt{\ auto\ }}\texttt{\ }

\paragraph{\texorpdfstring{\texttt{\ start\ }}{ start }}\label{definitions-vline-start}

\href{/docs/reference/foundations/int/}{int}

{{ Settable }}

\phantomsection\label{definitions-vline-start-settable-tooltip}
Settable parameters can be customized for all following uses of the
function with a \texttt{\ set\ } rule.

The row at which the vertical line starts (zero-indexed, inclusive).

Default: \texttt{\ }{\texttt{\ 0\ }}\texttt{\ }

\paragraph{\texorpdfstring{\texttt{\ end\ }}{ end }}\label{definitions-vline-end}

\href{/docs/reference/foundations/none/}{none} {or}
\href{/docs/reference/foundations/int/}{int}

{{ Settable }}

\phantomsection\label{definitions-vline-end-settable-tooltip}
Settable parameters can be customized for all following uses of the
function with a \texttt{\ set\ } rule.

The row on top of which the vertical line ends (zero-indexed,
exclusive). Therefore, the vertical line will be drawn up to and across
row \texttt{\ end\ -\ 1\ } .

A value equal to \texttt{\ }{\texttt{\ none\ }}\texttt{\ } or to the
amount of rows causes it to extend all the way towards the bottom of the
grid.

Default: \texttt{\ }{\texttt{\ none\ }}\texttt{\ }

\paragraph{\texorpdfstring{\texttt{\ stroke\ }}{ stroke }}\label{definitions-vline-stroke}

\href{/docs/reference/foundations/none/}{none} {or}
\href{/docs/reference/layout/length/}{length} {or}
\href{/docs/reference/visualize/color/}{color} {or}
\href{/docs/reference/visualize/gradient/}{gradient} {or}
\href{/docs/reference/visualize/stroke/}{stroke} {or}
\href{/docs/reference/visualize/pattern/}{pattern} {or}
\href{/docs/reference/foundations/dictionary/}{dictionary}

{{ Settable }}

\phantomsection\label{definitions-vline-stroke-settable-tooltip}
Settable parameters can be customized for all following uses of the
function with a \texttt{\ set\ } rule.

The line\textquotesingle s stroke.

Specifying \texttt{\ }{\texttt{\ none\ }}\texttt{\ } removes any lines
previously placed across this line\textquotesingle s range, including
vlines or per-cell stroke below it.

Default:
\texttt{\ }{\texttt{\ 1pt\ }}\texttt{\ }{\texttt{\ +\ }}\texttt{\ black\ }

\paragraph{\texorpdfstring{\texttt{\ position\ }}{ position }}\label{definitions-vline-position}

\href{/docs/reference/layout/alignment/}{alignment}

{{ Settable }}

\phantomsection\label{definitions-vline-position-settable-tooltip}
Settable parameters can be customized for all following uses of the
function with a \texttt{\ set\ } rule.

The position at which the line is placed, given its column (
\texttt{\ x\ } ) - either \texttt{\ start\ } to draw before it or
\texttt{\ end\ } to draw after it.

The values \texttt{\ left\ } and \texttt{\ right\ } are also accepted,
but discouraged as they cause your grid to be inconsistent between
left-to-right and right-to-left documents.

This setting is only relevant when column gutter is enabled (and
shouldn\textquotesingle t be used otherwise - prefer just increasing the
\texttt{\ x\ } field by one instead), since then the position after a
column becomes different from the position before the next column due to
the spacing between both.

Default: \texttt{\ start\ }

\subsubsection{\texorpdfstring{\texttt{\ header\ } {{ Element
}}}{ header   Element }}\label{definitions-header}

\phantomsection\label{definitions-header-element-tooltip}
Element functions can be customized with \texttt{\ set\ } and
\texttt{\ show\ } rules.

A repeatable grid header.

If \texttt{\ repeat\ } is set to \texttt{\ true\ } , the header will be
repeated across pages. For an example, refer to the
\href{/docs/reference/model/table/\#definitions-header}{\texttt{\ table.header\ }}
element and the
\href{/docs/reference/layout/grid/\#parameters-stroke}{\texttt{\ grid.stroke\ }}
parameter.

grid { . } { header } (

{ \hyperref[definitions-header-parameters-repeat]{repeat :}
\href{/docs/reference/foundations/bool/}{bool} , } {
\hyperref[definitions-header-parameters-children]{..}
\href{/docs/reference/foundations/content/}{content} , }

) -\textgreater{} \href{/docs/reference/foundations/content/}{content}

\paragraph{\texorpdfstring{\texttt{\ repeat\ }}{ repeat }}\label{definitions-header-repeat}

\href{/docs/reference/foundations/bool/}{bool}

{{ Settable }}

\phantomsection\label{definitions-header-repeat-settable-tooltip}
Settable parameters can be customized for all following uses of the
function with a \texttt{\ set\ } rule.

Whether this header should be repeated across pages.

Default: \texttt{\ }{\texttt{\ true\ }}\texttt{\ }

\paragraph{\texorpdfstring{\texttt{\ children\ }}{ children }}\label{definitions-header-children}

\href{/docs/reference/foundations/content/}{content}

{Required} {{ Positional }}

\phantomsection\label{definitions-header-children-positional-tooltip}
Positional parameters are specified in order, without names.

{{ Variadic }}

\phantomsection\label{definitions-header-children-variadic-tooltip}
Variadic parameters can be specified multiple times.

The cells and lines within the header.

\subsubsection{\texorpdfstring{\texttt{\ footer\ } {{ Element
}}}{ footer   Element }}\label{definitions-footer}

\phantomsection\label{definitions-footer-element-tooltip}
Element functions can be customized with \texttt{\ set\ } and
\texttt{\ show\ } rules.

A repeatable grid footer.

Just like the
\href{/docs/reference/layout/grid/\#definitions-header}{\texttt{\ grid.header\ }}
element, the footer can repeat itself on every page of the table.

No other grid cells may be placed after the footer.

grid { . } { footer } (

{ \hyperref[definitions-footer-parameters-repeat]{repeat :}
\href{/docs/reference/foundations/bool/}{bool} , } {
\hyperref[definitions-footer-parameters-children]{..}
\href{/docs/reference/foundations/content/}{content} , }

) -\textgreater{} \href{/docs/reference/foundations/content/}{content}

\paragraph{\texorpdfstring{\texttt{\ repeat\ }}{ repeat }}\label{definitions-footer-repeat}

\href{/docs/reference/foundations/bool/}{bool}

{{ Settable }}

\phantomsection\label{definitions-footer-repeat-settable-tooltip}
Settable parameters can be customized for all following uses of the
function with a \texttt{\ set\ } rule.

Whether this footer should be repeated across pages.

Default: \texttt{\ }{\texttt{\ true\ }}\texttt{\ }

\paragraph{\texorpdfstring{\texttt{\ children\ }}{ children }}\label{definitions-footer-children}

\href{/docs/reference/foundations/content/}{content}

{Required} {{ Positional }}

\phantomsection\label{definitions-footer-children-positional-tooltip}
Positional parameters are specified in order, without names.

{{ Variadic }}

\phantomsection\label{definitions-footer-children-variadic-tooltip}
Variadic parameters can be specified multiple times.

The cells and lines within the footer.

\href{/docs/reference/layout/fraction/}{\pandocbounded{\includesvg[keepaspectratio]{/assets/icons/16-arrow-right.svg}}}

{ Fraction } { Previous page }

\href{/docs/reference/layout/hide/}{\pandocbounded{\includesvg[keepaspectratio]{/assets/icons/16-arrow-right.svg}}}

{ Hide } { Next page }
