\title{typst.app/docs/reference/foundations/version}

\begin{itemize}
\tightlist
\item
  \href{/docs}{\includesvg[width=0.16667in,height=0.16667in]{/assets/icons/16-docs-dark.svg}}
\item
  \includesvg[width=0.16667in,height=0.16667in]{/assets/icons/16-arrow-right.svg}
\item
  \href{/docs/reference/}{Reference}
\item
  \includesvg[width=0.16667in,height=0.16667in]{/assets/icons/16-arrow-right.svg}
\item
  \href{/docs/reference/foundations/}{Foundations}
\item
  \includesvg[width=0.16667in,height=0.16667in]{/assets/icons/16-arrow-right.svg}
\item
  \href{/docs/reference/foundations/version/}{Version}
\end{itemize}

\section{\texorpdfstring{{ version }}{ version }}\label{summary}

A version with an arbitrary number of components.

The first three components have names that can be used as fields:
\texttt{\ major\ } , \texttt{\ minor\ } , \texttt{\ patch\ } . All
following components do not have names.

The list of components is semantically extended by an infinite list of
zeros. This means that, for example, \texttt{\ 0.8\ } is the same as
\texttt{\ 0.8.0\ } . As a special case, the empty version (that has no
components at all) is the same as \texttt{\ 0\ } , \texttt{\ 0.0\ } ,
\texttt{\ 0.0.0\ } , and so on.

The current version of the Typst compiler is available as
\texttt{\ sys.version\ } .

You can convert a version to an array of explicitly given components
using the \href{/docs/reference/foundations/array/}{\texttt{\ array\ }}
constructor.

\subsection{\texorpdfstring{Constructor
{}}{Constructor }}\label{constructor}

\phantomsection\label{constructor-constructor-tooltip}
If a type has a constructor, you can call it like a function to create a
new value of the type.

Creates a new version.

It can have any number of components (even zero).

{ version } (

{ \hyperref[constructor-parameters-components]{..}
\href{/docs/reference/foundations/int/}{int}
\href{/docs/reference/foundations/array/}{array} }

) -\textgreater{} \href{/docs/reference/foundations/version/}{version}

\begin{verbatim}
#version() \
#version(1) \
#version(1, 2, 3, 4) \
#version((1, 2, 3, 4)) \
#version((1, 2), 3)
\end{verbatim}

\includegraphics[width=5in,height=\textheight,keepaspectratio]{/assets/docs/Fx1_6ds8kbJ35Werk0qIqQAAAAAAAAAA.png}

\paragraph{\texorpdfstring{\texttt{\ components\ }}{ components }}\label{constructor-components}

\href{/docs/reference/foundations/int/}{int} {or}
\href{/docs/reference/foundations/array/}{array}

{Required} {{ Positional }}

\phantomsection\label{constructor-components-positional-tooltip}
Positional parameters are specified in order, without names.

{{ Variadic }}

\phantomsection\label{constructor-components-variadic-tooltip}
Variadic parameters can be specified multiple times.

The components of the version (array arguments are flattened)

\subsection{\texorpdfstring{{ Definitions
}}{ Definitions }}\label{definitions}

\phantomsection\label{definitions-tooltip}
Functions and types and can have associated definitions. These are
accessed by specifying the function or type, followed by a period, and
then the definition\textquotesingle s name.

\subsubsection{\texorpdfstring{\texttt{\ at\ }}{ at }}\label{definitions-at}

Retrieves a component of a version.

The returned integer is always non-negative. Returns \texttt{\ 0\ } if
the version isn\textquotesingle t specified to the necessary length.

self { . } { at } (

{ \href{/docs/reference/foundations/int/}{int} }

) -\textgreater{} \href{/docs/reference/foundations/int/}{int}

\paragraph{\texorpdfstring{\texttt{\ index\ }}{ index }}\label{definitions-at-index}

\href{/docs/reference/foundations/int/}{int}

{Required} {{ Positional }}

\phantomsection\label{definitions-at-index-positional-tooltip}
Positional parameters are specified in order, without names.

The index at which to retrieve the component. If negative, indexes from
the back of the explicitly given components.

\href{/docs/reference/foundations/type/}{\pandocbounded{\includesvg[keepaspectratio]{/assets/icons/16-arrow-right.svg}}}

{ Type } { Previous page }

\href{/docs/reference/model/}{\pandocbounded{\includesvg[keepaspectratio]{/assets/icons/16-arrow-right.svg}}}

{ Model } { Next page }

\textbf{On this page}

\begin{itemize}
\tightlist
\item
  \hyperref[summary]{Summary}
\item
  \hyperref[constructor]{Constructor}

  \begin{itemize}
  \tightlist
  \item
    \hyperref[constructor-components]{components}
  \end{itemize}
\item
  \hyperref[definitions]{Definitions}

  \begin{itemize}
  \tightlist
  \item
    \hyperref[definitions-at]{At}

    \begin{itemize}
    \tightlist
    \item
      \hyperref[definitions-at-index]{index}
    \end{itemize}
  \end{itemize}
\end{itemize}

\begin{itemize}
\tightlist
\item
  \href{/}{Home}
\item
  \href{/pricing/}{Pricing}
\item
  \href{/docs/}{Documentation}
\item
  \href{/universe/}{Universe}
\item
  \href{/about/}{About Us}
\item
  \href{/contact/}{Contact Us}
\item
  \href{/privacy/}{Privacy}
\item
  \href{https://typst.app/terms}{Terms and Conditions}
\item
  \href{/legal/}{Legal (Impressum)}
\end{itemize}

\begin{itemize}
\tightlist
\item
  \href{https://forum.typst.app}{Forum}
\item
  \href{/tools/}{Tools}
\item
  \href{/blog/}{Blog}
\item
  \href{https://github.com/typst/}{GitHub}
\item
  \href{https://discord.gg/2uDybryKPe}{Discord}
\item
  \href{https://mastodon.social/@typst}{Mastodon}
\item
  \href{https://bsky.app/profile/typst.app}{Bluesky}
\item
  \href{https://www.linkedin.com/company/typst/}{LinkedIn}
\item
  \href{https://instagram.com/typstapp/}{Instagram}
\end{itemize}

Made in Berlin
