\title{typst.app/docs/reference/foundations/panic}

\begin{itemize}
\tightlist
\item
  \href{/docs}{\includesvg[width=0.16667in,height=0.16667in]{/assets/icons/16-docs-dark.svg}}
\item
  \includesvg[width=0.16667in,height=0.16667in]{/assets/icons/16-arrow-right.svg}
\item
  \href{/docs/reference/}{Reference}
\item
  \includesvg[width=0.16667in,height=0.16667in]{/assets/icons/16-arrow-right.svg}
\item
  \href{/docs/reference/foundations/}{Foundations}
\item
  \includesvg[width=0.16667in,height=0.16667in]{/assets/icons/16-arrow-right.svg}
\item
  \href{/docs/reference/foundations/panic/}{Panic}
\end{itemize}

\section{\texorpdfstring{\texttt{\ panic\ }}{ panic }}\label{summary}

Fails with an error.

Arguments are displayed to the user (not rendered in the document) as
strings, converting with \texttt{\ repr\ } if necessary.

\subsection{Example}\label{example}

The code below produces the error
\texttt{\ panicked\ with:\ "this\ is\ wrong"\ } .

\begin{verbatim}
#panic("this is wrong")
\end{verbatim}

\subsection{\texorpdfstring{{ Parameters
}}{ Parameters }}\label{parameters}

\phantomsection\label{parameters-tooltip}
Parameters are the inputs to a function. They are specified in
parentheses after the function name.

{ panic } (

{ \hyperref[parameters-values]{..} { any } }

)

\subsubsection{\texorpdfstring{\texttt{\ values\ }}{ values }}\label{parameters-values}

{ any }

{Required} {{ Positional }}

\phantomsection\label{parameters-values-positional-tooltip}
Positional parameters are specified in order, without names.

{{ Variadic }}

\phantomsection\label{parameters-values-variadic-tooltip}
Variadic parameters can be specified multiple times.

The values to panic with and display to the user.

\href{/docs/reference/foundations/none/}{\pandocbounded{\includesvg[keepaspectratio]{/assets/icons/16-arrow-right.svg}}}

{ None } { Previous page }

\href{/docs/reference/foundations/plugin/}{\pandocbounded{\includesvg[keepaspectratio]{/assets/icons/16-arrow-right.svg}}}

{ Plugin } { Next page }
