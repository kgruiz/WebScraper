\title{typst.app/docs/reference/foundations/type}

\begin{itemize}
\tightlist
\item
  \href{/docs}{\includesvg[width=0.16667in,height=0.16667in]{/assets/icons/16-docs-dark.svg}}
\item
  \includesvg[width=0.16667in,height=0.16667in]{/assets/icons/16-arrow-right.svg}
\item
  \href{/docs/reference/}{Reference}
\item
  \includesvg[width=0.16667in,height=0.16667in]{/assets/icons/16-arrow-right.svg}
\item
  \href{/docs/reference/foundations/}{Foundations}
\item
  \includesvg[width=0.16667in,height=0.16667in]{/assets/icons/16-arrow-right.svg}
\item
  \href{/docs/reference/foundations/type/}{Type}
\end{itemize}

\section{\texorpdfstring{{ type }}{ type }}\label{summary}

Describes a kind of value.

To style your document, you need to work with values of different kinds:
Lengths specifying the size of your elements, colors for your text and
shapes, and more. Typst categorizes these into clearly defined
\emph{types} and tells you where it expects which type of value.

Apart from basic types for numeric values and
\href{/docs/reference/foundations/int/}{typical}
\href{/docs/reference/foundations/float/}{types}
\href{/docs/reference/foundations/str/}{known}
\href{/docs/reference/foundations/array/}{from}
\href{/docs/reference/foundations/dictionary/}{programming} languages,
Typst provides a special type for
\href{/docs/reference/foundations/content/}{\emph{content.}} A value of
this type can hold anything that you can enter into your document: Text,
elements like headings and shapes, and style information.

\subsection{Example}\label{example}

\begin{verbatim}
#let x = 10
#if type(x) == int [
  #x is an integer!
] else [
  #x is another value...
]

An image is of type
#type(image("glacier.jpg")).
\end{verbatim}

\includegraphics[width=5in,height=\textheight,keepaspectratio]{/assets/docs/dTjHaEMO5150e0-XVg1OzwAAAAAAAAAA.png}

The type of \texttt{\ 10\ } is \texttt{\ int\ } . Now, what is the type
of \texttt{\ int\ } or even \texttt{\ type\ } ?

\begin{verbatim}
#type(int) \
#type(type)
\end{verbatim}

\includegraphics[width=5in,height=\textheight,keepaspectratio]{/assets/docs/HqIgZy_wqBbnboRlZ-Iv4AAAAAAAAAAA.png}

\subsection{Compatibility}\label{compatibility}

In Typst 0.7 and lower, the \texttt{\ type\ } function returned a string
instead of a type. Compatibility with the old way will remain for a
while to give package authors time to upgrade, but it will be removed at
some point.

\begin{itemize}
\tightlist
\item
  Checks like
  \texttt{\ int\ }{\texttt{\ ==\ }}\texttt{\ }{\texttt{\ "integer"\ }}\texttt{\ }
  evaluate to \texttt{\ }{\texttt{\ true\ }}\texttt{\ }
\item
  Adding/joining a type and string will yield a string
\item
  The \texttt{\ in\ } operator on a type and a dictionary will evaluate
  to \texttt{\ }{\texttt{\ true\ }}\texttt{\ } if the dictionary has a
  string key matching the type\textquotesingle s name
\end{itemize}

\subsection{\texorpdfstring{Constructor
{}}{Constructor }}\label{constructor}

\phantomsection\label{constructor-constructor-tooltip}
If a type has a constructor, you can call it like a function to create a
new value of the type.

Determines a value\textquotesingle s type.

{ type } (

{ { any } }

) -\textgreater{} \href{/docs/reference/foundations/type/}{type}

\begin{verbatim}
#type(12) \
#type(14.7) \
#type("hello") \
#type(<glacier>) \
#type([Hi]) \
#type(x => x + 1) \
#type(type)
\end{verbatim}

\includegraphics[width=5in,height=\textheight,keepaspectratio]{/assets/docs/A7_wGHgPK0Jhrp3CDC6IegAAAAAAAAAA.png}

\paragraph{\texorpdfstring{\texttt{\ value\ }}{ value }}\label{constructor-value}

{ any }

{Required} {{ Positional }}

\phantomsection\label{constructor-value-positional-tooltip}
Positional parameters are specified in order, without names.

The value whose type\textquotesingle s to determine.

\href{/docs/reference/foundations/sys/}{\pandocbounded{\includesvg[keepaspectratio]{/assets/icons/16-arrow-right.svg}}}

{ System } { Previous page }

\href{/docs/reference/foundations/version/}{\pandocbounded{\includesvg[keepaspectratio]{/assets/icons/16-arrow-right.svg}}}

{ Version } { Next page }
