\title{typst.app/docs/reference/foundations/assert}

\begin{itemize}
\tightlist
\item
  \href{/docs}{\includesvg[width=0.16667in,height=0.16667in]{/assets/icons/16-docs-dark.svg}}
\item
  \includesvg[width=0.16667in,height=0.16667in]{/assets/icons/16-arrow-right.svg}
\item
  \href{/docs/reference/}{Reference}
\item
  \includesvg[width=0.16667in,height=0.16667in]{/assets/icons/16-arrow-right.svg}
\item
  \href{/docs/reference/foundations/}{Foundations}
\item
  \includesvg[width=0.16667in,height=0.16667in]{/assets/icons/16-arrow-right.svg}
\item
  \href{/docs/reference/foundations/assert/}{Assert}
\end{itemize}

\section{\texorpdfstring{\texttt{\ assert\ }}{ assert }}\label{summary}

Ensures that a condition is fulfilled.

Fails with an error if the condition is not fulfilled. Does not produce
any output in the document.

If you wish to test equality between two values, see
\href{/docs/reference/foundations/assert/\#definitions-eq}{\texttt{\ assert.eq\ }}
and
\href{/docs/reference/foundations/assert/\#definitions-ne}{\texttt{\ assert.ne\ }}
.

\subsection{Example}\label{example}

\begin{verbatim}
#assert(1 < 2, message: "math broke")
\end{verbatim}

\subsection{\texorpdfstring{{ Parameters
}}{ Parameters }}\label{parameters}

\phantomsection\label{parameters-tooltip}
Parameters are the inputs to a function. They are specified in
parentheses after the function name.

{ assert } (

{ \href{/docs/reference/foundations/bool/}{bool} , } {
\hyperref[parameters-message]{message :}
\href{/docs/reference/foundations/str/}{str} , }

)

\subsubsection{\texorpdfstring{\texttt{\ condition\ }}{ condition }}\label{parameters-condition}

\href{/docs/reference/foundations/bool/}{bool}

{Required} {{ Positional }}

\phantomsection\label{parameters-condition-positional-tooltip}
Positional parameters are specified in order, without names.

The condition that must be true for the assertion to pass.

\subsubsection{\texorpdfstring{\texttt{\ message\ }}{ message }}\label{parameters-message}

\href{/docs/reference/foundations/str/}{str}

The error message when the assertion fails.

\subsection{\texorpdfstring{{ Definitions
}}{ Definitions }}\label{definitions}

\phantomsection\label{definitions-tooltip}
Functions and types and can have associated definitions. These are
accessed by specifying the function or type, followed by a period, and
then the definition\textquotesingle s name.

\subsubsection{\texorpdfstring{\texttt{\ eq\ }}{ eq }}\label{definitions-eq}

Ensures that two values are equal.

Fails with an error if the first value is not equal to the second. Does
not produce any output in the document.

assert { . } { eq } (

{ { any } , } { { any } , } {
\hyperref[definitions-eq-parameters-message]{message :}
\href{/docs/reference/foundations/str/}{str} , }

)

\includesvg[width=0.16667in,height=0.16667in]{/assets/icons/16-arrow-right.svg}
View example

\begin{verbatim}
#assert.eq(10, 10)
\end{verbatim}

\paragraph{\texorpdfstring{\texttt{\ left\ }}{ left }}\label{definitions-eq-left}

{ any }

{Required} {{ Positional }}

\phantomsection\label{definitions-eq-left-positional-tooltip}
Positional parameters are specified in order, without names.

The first value to compare.

\paragraph{\texorpdfstring{\texttt{\ right\ }}{ right }}\label{definitions-eq-right}

{ any }

{Required} {{ Positional }}

\phantomsection\label{definitions-eq-right-positional-tooltip}
Positional parameters are specified in order, without names.

The second value to compare.

\paragraph{\texorpdfstring{\texttt{\ message\ }}{ message }}\label{definitions-eq-message}

\href{/docs/reference/foundations/str/}{str}

An optional message to display on error instead of the representations
of the compared values.

\subsubsection{\texorpdfstring{\texttt{\ ne\ }}{ ne }}\label{definitions-ne}

Ensures that two values are not equal.

Fails with an error if the first value is equal to the second. Does not
produce any output in the document.

assert { . } { ne } (

{ { any } , } { { any } , } {
\hyperref[definitions-ne-parameters-message]{message :}
\href{/docs/reference/foundations/str/}{str} , }

)

\includesvg[width=0.16667in,height=0.16667in]{/assets/icons/16-arrow-right.svg}
View example

\begin{verbatim}
#assert.ne(3, 4)
\end{verbatim}

\paragraph{\texorpdfstring{\texttt{\ left\ }}{ left }}\label{definitions-ne-left}

{ any }

{Required} {{ Positional }}

\phantomsection\label{definitions-ne-left-positional-tooltip}
Positional parameters are specified in order, without names.

The first value to compare.

\paragraph{\texorpdfstring{\texttt{\ right\ }}{ right }}\label{definitions-ne-right}

{ any }

{Required} {{ Positional }}

\phantomsection\label{definitions-ne-right-positional-tooltip}
Positional parameters are specified in order, without names.

The second value to compare.

\paragraph{\texorpdfstring{\texttt{\ message\ }}{ message }}\label{definitions-ne-message}

\href{/docs/reference/foundations/str/}{str}

An optional message to display on error instead of the representations
of the compared values.

\href{/docs/reference/foundations/array/}{\pandocbounded{\includesvg[keepaspectratio]{/assets/icons/16-arrow-right.svg}}}

{ Array } { Previous page }

\href{/docs/reference/foundations/auto/}{\pandocbounded{\includesvg[keepaspectratio]{/assets/icons/16-arrow-right.svg}}}

{ Auto } { Next page }
