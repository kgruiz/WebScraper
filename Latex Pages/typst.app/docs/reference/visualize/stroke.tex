\title{typst.app/docs/reference/visualize/stroke}

\begin{itemize}
\tightlist
\item
  \href{/docs}{\includesvg[width=0.16667in,height=0.16667in]{/assets/icons/16-docs-dark.svg}}
\item
  \includesvg[width=0.16667in,height=0.16667in]{/assets/icons/16-arrow-right.svg}
\item
  \href{/docs/reference/}{Reference}
\item
  \includesvg[width=0.16667in,height=0.16667in]{/assets/icons/16-arrow-right.svg}
\item
  \href{/docs/reference/visualize/}{Visualize}
\item
  \includesvg[width=0.16667in,height=0.16667in]{/assets/icons/16-arrow-right.svg}
\item
  \href{/docs/reference/visualize/stroke/}{Stroke}
\end{itemize}

\section{\texorpdfstring{{ stroke }}{ stroke }}\label{summary}

Defines how to draw a line.

A stroke has a \emph{paint} (a solid color or gradient), a
\emph{thickness,} a line \emph{cap,} a line \emph{join,} a \emph{miter
limit,} and a \emph{dash} pattern. All of these values are optional and
have sensible defaults.

\subsection{Example}\label{example}

\begin{verbatim}
#set line(length: 100%)
#stack(
  spacing: 1em,
  line(stroke: 2pt + red),
  line(stroke: (paint: blue, thickness: 4pt, cap: "round")),
  line(stroke: (paint: blue, thickness: 1pt, dash: "dashed")),
  line(stroke: 2pt + gradient.linear(..color.map.rainbow)),
)
\end{verbatim}

\includegraphics[width=5in,height=\textheight,keepaspectratio]{/assets/docs/3NofubbwIllodsFawlNd8wAAAAAAAAAA.png}

\subsection{Simple strokes}\label{simple-strokes}

You can create a simple solid stroke from a color, a thickness, or a
combination of the two. Specifically, wherever a stroke is expected you
can pass any of the following values:

\begin{itemize}
\tightlist
\item
  A length specifying the stroke\textquotesingle s thickness. The color
  is inherited, defaulting to black.
\item
  A color to use for the stroke. The thickness is inherited, defaulting
  to \texttt{\ }{\texttt{\ 1pt\ }}\texttt{\ } .
\item
  A stroke combined from color and thickness using the \texttt{\ +\ }
  operator as in
  \texttt{\ }{\texttt{\ 2pt\ }}\texttt{\ }{\texttt{\ +\ }}\texttt{\ red\ }
  .
\end{itemize}

For full control, you can also provide a
\href{/docs/reference/foundations/dictionary/}{dictionary} or a
\texttt{\ stroke\ } object to any function that expects a stroke. The
dictionary\textquotesingle s keys may include any of the parameters for
the constructor function, shown below.

\subsection{Fields}\label{fields}

On a stroke object, you can access any of the fields listed in the
constructor function. For example,
\texttt{\ }{\texttt{\ (\ }}\texttt{\ }{\texttt{\ 2pt\ }}\texttt{\ }{\texttt{\ +\ }}\texttt{\ blue\ }{\texttt{\ )\ }}\texttt{\ }{\texttt{\ .\ }}\texttt{\ thickness\ }
is \texttt{\ }{\texttt{\ 2pt\ }}\texttt{\ } . Meanwhile,
\texttt{\ }{\texttt{\ stroke\ }}\texttt{\ }{\texttt{\ (\ }}\texttt{\ red\ }{\texttt{\ )\ }}\texttt{\ }{\texttt{\ .\ }}\texttt{\ cap\ }
is \texttt{\ }{\texttt{\ auto\ }}\texttt{\ } because
it\textquotesingle s unspecified. Fields set to
\texttt{\ }{\texttt{\ auto\ }}\texttt{\ } are inherited.

\subsection{\texorpdfstring{Constructor
{}}{Constructor }}\label{constructor}

\phantomsection\label{constructor-constructor-tooltip}
If a type has a constructor, you can call it like a function to create a
new value of the type.

Converts a value to a stroke or constructs a stroke with the given
parameters.

Note that in most cases you do not need to convert values to strokes in
order to use them, as they will be converted automatically. However,
this constructor can be useful to ensure a value has all the fields of a
stroke.

{ stroke } (

{ \href{/docs/reference/foundations/auto/}{auto}
\href{/docs/reference/visualize/color/}{color}
\href{/docs/reference/visualize/gradient/}{gradient}
\href{/docs/reference/visualize/pattern/}{pattern} , } {
\href{/docs/reference/foundations/auto/}{auto}
\href{/docs/reference/layout/length/}{length} , } {
\href{/docs/reference/foundations/auto/}{auto}
\href{/docs/reference/foundations/str/}{str} , } {
\href{/docs/reference/foundations/auto/}{auto}
\href{/docs/reference/foundations/str/}{str} , } {
\href{/docs/reference/foundations/none/}{none}
\href{/docs/reference/foundations/auto/}{auto}
\href{/docs/reference/foundations/str/}{str}
\href{/docs/reference/foundations/array/}{array}
\href{/docs/reference/foundations/dictionary/}{dictionary} , } {
\href{/docs/reference/foundations/auto/}{auto}
\href{/docs/reference/foundations/float/}{float} , }

) -\textgreater{} \href{/docs/reference/visualize/stroke/}{stroke}

\begin{verbatim}
#let my-func(x) = {
    x = stroke(x) // Convert to a stroke
    [Stroke has thickness #x.thickness.]
}
#my-func(3pt) \
#my-func(red) \
#my-func(stroke(cap: "round", thickness: 1pt))
\end{verbatim}

\includegraphics[width=5in,height=\textheight,keepaspectratio]{/assets/docs/oulcXDNcpunCSxVvCPXMJQAAAAAAAAAA.png}

\paragraph{\texorpdfstring{\texttt{\ paint\ }}{ paint }}\label{constructor-paint}

\href{/docs/reference/foundations/auto/}{auto} {or}
\href{/docs/reference/visualize/color/}{color} {or}
\href{/docs/reference/visualize/gradient/}{gradient} {or}
\href{/docs/reference/visualize/pattern/}{pattern}

{Required} {{ Positional }}

\phantomsection\label{constructor-paint-positional-tooltip}
Positional parameters are specified in order, without names.

The color or gradient to use for the stroke.

If set to \texttt{\ }{\texttt{\ auto\ }}\texttt{\ } , the value is
inherited, defaulting to \texttt{\ black\ } .

\paragraph{\texorpdfstring{\texttt{\ thickness\ }}{ thickness }}\label{constructor-thickness}

\href{/docs/reference/foundations/auto/}{auto} {or}
\href{/docs/reference/layout/length/}{length}

{Required} {{ Positional }}

\phantomsection\label{constructor-thickness-positional-tooltip}
Positional parameters are specified in order, without names.

The stroke\textquotesingle s thickness.

If set to \texttt{\ }{\texttt{\ auto\ }}\texttt{\ } , the value is
inherited, defaulting to \texttt{\ }{\texttt{\ 1pt\ }}\texttt{\ } .

\paragraph{\texorpdfstring{\texttt{\ cap\ }}{ cap }}\label{constructor-cap}

\href{/docs/reference/foundations/auto/}{auto} {or}
\href{/docs/reference/foundations/str/}{str}

{Required} {{ Positional }}

\phantomsection\label{constructor-cap-positional-tooltip}
Positional parameters are specified in order, without names.

How the ends of the stroke are rendered.

If set to \texttt{\ }{\texttt{\ auto\ }}\texttt{\ } , the value is
inherited, defaulting to \texttt{\ }{\texttt{\ "butt"\ }}\texttt{\ } .

\begin{longtable}[]{@{}ll@{}}
\toprule\noalign{}
Variant & Details \\
\midrule\noalign{}
\endhead
\bottomrule\noalign{}
\endlastfoot
\texttt{\ "\ butt\ "\ } & Square stroke cap with the edge at the
stroke\textquotesingle s end point. \\
\texttt{\ "\ round\ "\ } & Circular stroke cap centered at the
stroke\textquotesingle s end point. \\
\texttt{\ "\ square\ "\ } & Square stroke cap centered at the
stroke\textquotesingle s end point. \\
\end{longtable}

\paragraph{\texorpdfstring{\texttt{\ join\ }}{ join }}\label{constructor-join}

\href{/docs/reference/foundations/auto/}{auto} {or}
\href{/docs/reference/foundations/str/}{str}

{Required} {{ Positional }}

\phantomsection\label{constructor-join-positional-tooltip}
Positional parameters are specified in order, without names.

How sharp turns are rendered.

If set to \texttt{\ }{\texttt{\ auto\ }}\texttt{\ } , the value is
inherited, defaulting to \texttt{\ }{\texttt{\ "miter"\ }}\texttt{\ } .

\begin{longtable}[]{@{}ll@{}}
\toprule\noalign{}
Variant & Details \\
\midrule\noalign{}
\endhead
\bottomrule\noalign{}
\endlastfoot
\texttt{\ "\ miter\ "\ } & Segments are joined with sharp edges. Sharp
bends exceeding the miter limit are bevelled instead. \\
\texttt{\ "\ round\ "\ } & Segments are joined with circular corners. \\
\texttt{\ "\ bevel\ "\ } & Segments are joined with a bevel (a straight
edge connecting the butts of the joined segments). \\
\end{longtable}

\paragraph{\texorpdfstring{\texttt{\ dash\ }}{ dash }}\label{constructor-dash}

\href{/docs/reference/foundations/none/}{none} {or}
\href{/docs/reference/foundations/auto/}{auto} {or}
\href{/docs/reference/foundations/str/}{str} {or}
\href{/docs/reference/foundations/array/}{array} {or}
\href{/docs/reference/foundations/dictionary/}{dictionary}

{Required} {{ Positional }}

\phantomsection\label{constructor-dash-positional-tooltip}
Positional parameters are specified in order, without names.

The dash pattern to use. This can be:

\begin{itemize}
\tightlist
\item
  One of the predefined patterns:

  \begin{itemize}
  \tightlist
  \item
    \texttt{\ }{\texttt{\ "solid"\ }}\texttt{\ } or
    \texttt{\ }{\texttt{\ none\ }}\texttt{\ }
  \item
    \texttt{\ }{\texttt{\ "dotted"\ }}\texttt{\ }
  \item
    \texttt{\ }{\texttt{\ "densely-dotted"\ }}\texttt{\ }
  \item
    \texttt{\ }{\texttt{\ "loosely-dotted"\ }}\texttt{\ }
  \item
    \texttt{\ }{\texttt{\ "dashed"\ }}\texttt{\ }
  \item
    \texttt{\ }{\texttt{\ "densely-dashed"\ }}\texttt{\ }
  \item
    \texttt{\ }{\texttt{\ "loosely-dashed"\ }}\texttt{\ }
  \item
    \texttt{\ }{\texttt{\ "dash-dotted"\ }}\texttt{\ }
  \item
    \texttt{\ }{\texttt{\ "densely-dash-dotted"\ }}\texttt{\ }
  \item
    \texttt{\ }{\texttt{\ "loosely-dash-dotted"\ }}\texttt{\ }
  \end{itemize}
\item
  An \href{/docs/reference/foundations/array/}{array} with alternating
  lengths for dashes and gaps. You can also use the string
  \texttt{\ }{\texttt{\ "dot"\ }}\texttt{\ } for a length equal to the
  line thickness.
\item
  A \href{/docs/reference/foundations/dictionary/}{dictionary} with the
  keys \texttt{\ array\ } (same as the array above), and
  \texttt{\ phase\ } (of type
  \href{/docs/reference/layout/length/}{length} ), which defines where
  in the pattern to start drawing.
\end{itemize}

If set to \texttt{\ }{\texttt{\ auto\ }}\texttt{\ } , the value is
inherited, defaulting to \texttt{\ }{\texttt{\ none\ }}\texttt{\ } .

\includesvg[width=0.16667in,height=0.16667in]{/assets/icons/16-arrow-right.svg}
View options

\begin{longtable}[]{@{}ll@{}}
\toprule\noalign{}
Variant & Details \\
\midrule\noalign{}
\endhead
\bottomrule\noalign{}
\endlastfoot
\texttt{\ "\ solid\ "\ } & \\
\texttt{\ "\ dotted\ "\ } & \\
\texttt{\ "\ densely-dotted\ "\ } & \\
\texttt{\ "\ loosely-dotted\ "\ } & \\
\texttt{\ "\ dashed\ "\ } & \\
\texttt{\ "\ densely-dashed\ "\ } & \\
\texttt{\ "\ loosely-dashed\ "\ } & \\
\texttt{\ "\ dash-dotted\ "\ } & \\
\texttt{\ "\ densely-dash-dotted\ "\ } & \\
\texttt{\ "\ loosely-dash-dotted\ "\ } & \\
\end{longtable}

\includesvg[width=0.16667in,height=0.16667in]{/assets/icons/16-arrow-right.svg}
View example

\begin{verbatim}
#set line(length: 100%, stroke: 2pt)
#stack(
  spacing: 1em,
  line(stroke: (dash: "dashed")),
  line(stroke: (dash: (10pt, 5pt, "dot", 5pt))),
  line(stroke: (dash: (array: (10pt, 5pt, "dot", 5pt), phase: 10pt))),
)
\end{verbatim}

\includegraphics[width=5in,height=\textheight,keepaspectratio]{/assets/docs/P38gFluKZcw64WdZR85nHgAAAAAAAAAA.png}

\paragraph{\texorpdfstring{\texttt{\ miter-limit\ }}{ miter-limit }}\label{constructor-miter-limit}

\href{/docs/reference/foundations/auto/}{auto} {or}
\href{/docs/reference/foundations/float/}{float}

{Required} {{ Positional }}

\phantomsection\label{constructor-miter-limit-positional-tooltip}
Positional parameters are specified in order, without names.

Number at which protruding sharp bends are rendered with a bevel instead
or a miter join. The higher the number, the sharper an angle can be
before it is bevelled. Only applicable if \texttt{\ join\ } is
\texttt{\ }{\texttt{\ "miter"\ }}\texttt{\ } .

Specifically, the miter limit is the maximum ratio between the
corner\textquotesingle s protrusion length and the
stroke\textquotesingle s thickness.

If set to \texttt{\ }{\texttt{\ auto\ }}\texttt{\ } , the value is
inherited, defaulting to \texttt{\ }{\texttt{\ 4.0\ }}\texttt{\ } .

\includesvg[width=0.16667in,height=0.16667in]{/assets/icons/16-arrow-right.svg}
View example

\begin{verbatim}
#let points = ((15pt, 0pt), (0pt, 30pt), (30pt, 30pt), (10pt, 20pt))
#set path(stroke: 6pt + blue)
#stack(
    dir: ltr,
    spacing: 1cm,
    path(stroke: (miter-limit: 1), ..points),
    path(stroke: (miter-limit: 4), ..points),
    path(stroke: (miter-limit: 5), ..points),
)
\end{verbatim}

\includegraphics[width=5in,height=\textheight,keepaspectratio]{/assets/docs/3zeU1BuQq8_VfdTfAQbv5QAAAAAAAAAA.png}

\href{/docs/reference/visualize/square/}{\pandocbounded{\includesvg[keepaspectratio]{/assets/icons/16-arrow-right.svg}}}

{ Square } { Previous page }

\href{/docs/reference/introspection/}{\pandocbounded{\includesvg[keepaspectratio]{/assets/icons/16-arrow-right.svg}}}

{ Introspection } { Next page }

\textbf{On this page}

\begin{itemize}
\tightlist
\item
  \hyperref[summary]{Summary}
\item
  \hyperref[example]{Example}
\item
  \hyperref[simple-strokes]{Simple Strokes}
\item
  \hyperref[fields]{Fields}
\item
  \hyperref[constructor]{Constructor}

  \begin{itemize}
  \tightlist
  \item
    \hyperref[constructor-paint]{paint}
  \item
    \hyperref[constructor-thickness]{thickness}
  \item
    \hyperref[constructor-cap]{cap}
  \item
    \hyperref[constructor-join]{join}
  \item
    \hyperref[constructor-dash]{dash}
  \item
    \hyperref[constructor-miter-limit]{miter-limit}
  \end{itemize}
\end{itemize}

\begin{itemize}
\tightlist
\item
  \href{/}{Home}
\item
  \href{/pricing/}{Pricing}
\item
  \href{/docs/}{Documentation}
\item
  \href{/universe/}{Universe}
\item
  \href{/about/}{About Us}
\item
  \href{/contact/}{Contact Us}
\item
  \href{/privacy/}{Privacy}
\item
  \href{https://typst.app/terms}{Terms and Conditions}
\item
  \href{/legal/}{Legal (Impressum)}
\end{itemize}

\begin{itemize}
\tightlist
\item
  \href{https://forum.typst.app}{Forum}
\item
  \href{/tools/}{Tools}
\item
  \href{/blog/}{Blog}
\item
  \href{https://github.com/typst/}{GitHub}
\item
  \href{https://discord.gg/2uDybryKPe}{Discord}
\item
  \href{https://mastodon.social/@typst}{Mastodon}
\item
  \href{https://bsky.app/profile/typst.app}{Bluesky}
\item
  \href{https://www.linkedin.com/company/typst/}{LinkedIn}
\item
  \href{https://instagram.com/typstapp/}{Instagram}
\end{itemize}

Made in Berlin
