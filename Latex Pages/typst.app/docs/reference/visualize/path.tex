\title{typst.app/docs/reference/visualize/path}

\begin{itemize}
\tightlist
\item
  \href{/docs}{\includesvg[width=0.16667in,height=0.16667in]{/assets/icons/16-docs-dark.svg}}
\item
  \includesvg[width=0.16667in,height=0.16667in]{/assets/icons/16-arrow-right.svg}
\item
  \href{/docs/reference/}{Reference}
\item
  \includesvg[width=0.16667in,height=0.16667in]{/assets/icons/16-arrow-right.svg}
\item
  \href{/docs/reference/visualize/}{Visualize}
\item
  \includesvg[width=0.16667in,height=0.16667in]{/assets/icons/16-arrow-right.svg}
\item
  \href{/docs/reference/visualize/path/}{Path}
\end{itemize}

\section{\texorpdfstring{\texttt{\ path\ } {{ Element
}}}{ path   Element }}\label{summary}

\phantomsection\label{element-tooltip}
Element functions can be customized with \texttt{\ set\ } and
\texttt{\ show\ } rules.

A path through a list of points, connected by Bezier curves.

\subsection{Example}\label{example}

\begin{verbatim}
#path(
  fill: blue.lighten(80%),
  stroke: blue,
  closed: true,
  (0pt, 50pt),
  (100%, 50pt),
  ((50%, 0pt), (40pt, 0pt)),
)
\end{verbatim}

\includegraphics[width=5in,height=\textheight,keepaspectratio]{/assets/docs/fHH_90d6MEksjFQh_gCkDwAAAAAAAAAA.png}

\subsection{\texorpdfstring{{ Parameters
}}{ Parameters }}\label{parameters}

\phantomsection\label{parameters-tooltip}
Parameters are the inputs to a function. They are specified in
parentheses after the function name.

{ path } (

{ \hyperref[parameters-fill]{fill :}
\href{/docs/reference/foundations/none/}{none}
\href{/docs/reference/visualize/color/}{color}
\href{/docs/reference/visualize/gradient/}{gradient}
\href{/docs/reference/visualize/pattern/}{pattern} , } {
\hyperref[parameters-fill-rule]{fill-rule :}
\href{/docs/reference/foundations/str/}{str} , } {
\hyperref[parameters-stroke]{stroke :}
\href{/docs/reference/foundations/none/}{none}
\href{/docs/reference/foundations/auto/}{auto}
\href{/docs/reference/layout/length/}{length}
\href{/docs/reference/visualize/color/}{color}
\href{/docs/reference/visualize/gradient/}{gradient}
\href{/docs/reference/visualize/stroke/}{stroke}
\href{/docs/reference/visualize/pattern/}{pattern}
\href{/docs/reference/foundations/dictionary/}{dictionary} , } {
\hyperref[parameters-closed]{closed :}
\href{/docs/reference/foundations/bool/}{bool} , } {
\hyperref[parameters-vertices]{..}
\href{/docs/reference/foundations/array/}{array} , }

) -\textgreater{} \href{/docs/reference/foundations/content/}{content}

\subsubsection{\texorpdfstring{\texttt{\ fill\ }}{ fill }}\label{parameters-fill}

\href{/docs/reference/foundations/none/}{none} {or}
\href{/docs/reference/visualize/color/}{color} {or}
\href{/docs/reference/visualize/gradient/}{gradient} {or}
\href{/docs/reference/visualize/pattern/}{pattern}

{{ Settable }}

\phantomsection\label{parameters-fill-settable-tooltip}
Settable parameters can be customized for all following uses of the
function with a \texttt{\ set\ } rule.

How to fill the path.

When setting a fill, the default stroke disappears. To create a
rectangle with both fill and stroke, you have to configure both.

Default: \texttt{\ }{\texttt{\ none\ }}\texttt{\ }

\subsubsection{\texorpdfstring{\texttt{\ fill-rule\ }}{ fill-rule }}\label{parameters-fill-rule}

\href{/docs/reference/foundations/str/}{str}

{{ Settable }}

\phantomsection\label{parameters-fill-rule-settable-tooltip}
Settable parameters can be customized for all following uses of the
function with a \texttt{\ set\ } rule.

The drawing rule used to fill the path.

\begin{longtable}[]{@{}ll@{}}
\toprule\noalign{}
Variant & Details \\
\midrule\noalign{}
\endhead
\bottomrule\noalign{}
\endlastfoot
\texttt{\ "\ non-zero\ "\ } & Specifies that "inside" is computed by a
non-zero sum of signed edge crossings. \\
\texttt{\ "\ even-odd\ "\ } & Specifies that "inside" is computed by an
odd number of edge crossings. \\
\end{longtable}

Default: \texttt{\ }{\texttt{\ "non-zero"\ }}\texttt{\ }

\includesvg[width=0.16667in,height=0.16667in]{/assets/icons/16-arrow-right.svg}
View example

\begin{verbatim}
// We use `.with` to get a new
// function that has the common
// arguments pre-applied.
#let star = path.with(
  fill: red,
  closed: true,
  (25pt, 0pt),
  (10pt, 50pt),
  (50pt, 20pt),
  (0pt, 20pt),
  (40pt, 50pt),
)

#star(fill-rule: "non-zero")
#star(fill-rule: "even-odd")
\end{verbatim}

\includegraphics[width=5in,height=\textheight,keepaspectratio]{/assets/docs/MJEOUf62l7aK0PG-Hl3HKgAAAAAAAAAA.png}

\subsubsection{\texorpdfstring{\texttt{\ stroke\ }}{ stroke }}\label{parameters-stroke}

\href{/docs/reference/foundations/none/}{none} {or}
\href{/docs/reference/foundations/auto/}{auto} {or}
\href{/docs/reference/layout/length/}{length} {or}
\href{/docs/reference/visualize/color/}{color} {or}
\href{/docs/reference/visualize/gradient/}{gradient} {or}
\href{/docs/reference/visualize/stroke/}{stroke} {or}
\href{/docs/reference/visualize/pattern/}{pattern} {or}
\href{/docs/reference/foundations/dictionary/}{dictionary}

{{ Settable }}

\phantomsection\label{parameters-stroke-settable-tooltip}
Settable parameters can be customized for all following uses of the
function with a \texttt{\ set\ } rule.

How to \href{/docs/reference/visualize/stroke/}{stroke} the path. This
can be:

Can be set to \texttt{\ }{\texttt{\ none\ }}\texttt{\ } to disable the
stroke or to \texttt{\ }{\texttt{\ auto\ }}\texttt{\ } for a stroke of
\texttt{\ }{\texttt{\ 1pt\ }}\texttt{\ } black if and if only if no fill
is given.

Default: \texttt{\ }{\texttt{\ auto\ }}\texttt{\ }

\subsubsection{\texorpdfstring{\texttt{\ closed\ }}{ closed }}\label{parameters-closed}

\href{/docs/reference/foundations/bool/}{bool}

{{ Settable }}

\phantomsection\label{parameters-closed-settable-tooltip}
Settable parameters can be customized for all following uses of the
function with a \texttt{\ set\ } rule.

Whether to close this path with one last bezier curve. This curve will
takes into account the adjacent control points. If you want to close
with a straight line, simply add one last point that\textquotesingle s
the same as the start point.

Default: \texttt{\ }{\texttt{\ false\ }}\texttt{\ }

\subsubsection{\texorpdfstring{\texttt{\ vertices\ }}{ vertices }}\label{parameters-vertices}

\href{/docs/reference/foundations/array/}{array}

{Required} {{ Positional }}

\phantomsection\label{parameters-vertices-positional-tooltip}
Positional parameters are specified in order, without names.

{{ Variadic }}

\phantomsection\label{parameters-vertices-variadic-tooltip}
Variadic parameters can be specified multiple times.

The vertices of the path.

Each vertex can be defined in 3 ways:

\begin{itemize}
\tightlist
\item
  A regular point, as given to the
  \href{/docs/reference/visualize/line/}{\texttt{\ line\ }} or
  \href{/docs/reference/visualize/polygon/}{\texttt{\ polygon\ }}
  function.
\item
  An array of two points, the first being the vertex and the second
  being the control point. The control point is expressed relative to
  the vertex and is mirrored to get the second control point. The given
  control point is the one that affects the curve coming \emph{into}
  this vertex (even for the first point). The mirrored control point
  affects the curve going out of this vertex.
\item
  An array of three points, the first being the vertex and the next
  being the control points (control point for curves coming in and out,
  respectively).
\end{itemize}

\href{/docs/reference/visualize/line/}{\pandocbounded{\includesvg[keepaspectratio]{/assets/icons/16-arrow-right.svg}}}

{ Line } { Previous page }

\href{/docs/reference/visualize/pattern/}{\pandocbounded{\includesvg[keepaspectratio]{/assets/icons/16-arrow-right.svg}}}

{ Pattern } { Next page }
