\title{typst.app/docs/reference/introspection/counter}

\begin{itemize}
\tightlist
\item
  \href{/docs}{\includesvg[width=0.16667in,height=0.16667in]{/assets/icons/16-docs-dark.svg}}
\item
  \includesvg[width=0.16667in,height=0.16667in]{/assets/icons/16-arrow-right.svg}
\item
  \href{/docs/reference/}{Reference}
\item
  \includesvg[width=0.16667in,height=0.16667in]{/assets/icons/16-arrow-right.svg}
\item
  \href{/docs/reference/introspection/}{Introspection}
\item
  \includesvg[width=0.16667in,height=0.16667in]{/assets/icons/16-arrow-right.svg}
\item
  \href{/docs/reference/introspection/counter/}{Counter}
\end{itemize}

\section{\texorpdfstring{{ counter }}{ counter }}\label{summary}

Counts through pages, elements, and more.

With the counter function, you can access and modify counters for pages,
headings, figures, and more. Moreover, you can define custom counters
for other things you want to count.

Since counters change throughout the course of the document, their
current value is \emph{contextual.} It is recommended to read the
chapter on \href{/docs/reference/context/}{context} before continuing
here.

\subsection{Accessing a counter}\label{accessing}

To access the raw value of a counter, we can use the
\href{/docs/reference/introspection/counter/\#definitions-get}{\texttt{\ get\ }}
function. This function returns an
\href{/docs/reference/foundations/array/}{array} : Counters can have
multiple levels (in the case of headings for sections, subsections, and
so on), and each item in the array corresponds to one level.

\begin{verbatim}
#set heading(numbering: "1.")

= Introduction
Raw value of heading counter is
#context counter(heading).get()
\end{verbatim}

\includegraphics[width=5in,height=\textheight,keepaspectratio]{/assets/docs/jqVSznl_yGBcNN9ecF8OVAAAAAAAAAAA.png}

\subsection{Displaying a counter}\label{displaying}

Often, we want to display the value of a counter in a more
human-readable way. To do that, we can call the
\href{/docs/reference/introspection/counter/\#definitions-display}{\texttt{\ display\ }}
function on the counter. This function retrieves the current counter
value and formats it either with a provided or with an automatically
inferred \href{/docs/reference/model/numbering/}{numbering} .

\begin{verbatim}
#set heading(numbering: "1.")

= Introduction
Some text here.

= Background
The current value is: #context {
  counter(heading).display()
}

Or in roman numerals: #context {
  counter(heading).display("I")
}
\end{verbatim}

\includegraphics[width=5in,height=\textheight,keepaspectratio]{/assets/docs/7EUi61p1PXmzQyka_2NqiAAAAAAAAAAA.png}

\subsection{Modifying a counter}\label{modifying}

To modify a counter, you can use the \texttt{\ step\ } and
\texttt{\ update\ } methods:

\begin{itemize}
\item
  The \texttt{\ step\ } method increases the value of the counter by
  one. Because counters can have multiple levels , it optionally takes a
  \texttt{\ level\ } argument. If given, the counter steps at the given
  depth.
\item
  The \texttt{\ update\ } method allows you to arbitrarily modify the
  counter. In its basic form, you give it an integer (or an array for
  multiple levels). For more flexibility, you can instead also give it a
  function that receives the current value and returns a new value.
\end{itemize}

The heading counter is stepped before the heading is displayed, so
\texttt{\ Analysis\ } gets the number seven even though the counter is
at six after the second update.

\begin{verbatim}
#set heading(numbering: "1.")

= Introduction
#counter(heading).step()

= Background
#counter(heading).update(3)
#counter(heading).update(n => n * 2)

= Analysis
Let's skip 7.1.
#counter(heading).step(level: 2)

== Analysis
Still at #context {
  counter(heading).display()
}
\end{verbatim}

\includegraphics[width=5in,height=\textheight,keepaspectratio]{/assets/docs/EOYqv5YWVpmiQyBJoYpqQAAAAAAAAAAA.png}

\subsection{Page counter}\label{page-counter}

The page counter is special. It is automatically stepped at each
pagebreak. But like other counters, you can also step it manually. For
example, you could have Roman page numbers for your preface, then switch
to Arabic page numbers for your main content and reset the page counter
to one.

\begin{verbatim}
#set page(numbering: "(i)")

= Preface
The preface is numbered with
roman numerals.

#set page(numbering: "1 / 1")
#counter(page).update(1)

= Main text
Here, the counter is reset to one.
We also display both the current
page and total number of pages in
Arabic numbers.
\end{verbatim}

\includegraphics[width=5in,height=\textheight,keepaspectratio]{/assets/docs/PDCorO6nPZEoa3HjHUVgRwAAAAAAAAAA.png}
\includegraphics[width=5in,height=\textheight,keepaspectratio]{/assets/docs/PDCorO6nPZEoa3HjHUVgRwAAAAAAAAAB.png}

\subsection{Custom counters}\label{custom-counters}

To define your own counter, call the \texttt{\ counter\ } function with
a string as a key. This key identifies the counter globally.

\begin{verbatim}
#let mine = counter("mycounter")
#context mine.display() \
#mine.step()
#context mine.display() \
#mine.update(c => c * 3)
#context mine.display()
\end{verbatim}

\includegraphics[width=5in,height=\textheight,keepaspectratio]{/assets/docs/CxXLMyCvJp2FnmacPN3WUgAAAAAAAAAA.png}

\subsection{How to step}\label{how-to-step}

When you define and use a custom counter, in general, you should first
step the counter and then display it. This way, the stepping behaviour
of a counter can depend on the element it is stepped for. If you were
writing a counter for, let\textquotesingle s say, theorems, your
theorem\textquotesingle s definition would thus first include the
counter step and only then display the counter and the
theorem\textquotesingle s contents.

\begin{verbatim}
#let c = counter("theorem")
#let theorem(it) = block[
  #c.step()
  *Theorem #context c.display():*
  #it
]

#theorem[$1 = 1$]
#theorem[$2 < 3$]
\end{verbatim}

\includegraphics[width=5in,height=\textheight,keepaspectratio]{/assets/docs/af6Y7nOR_IldvYHIWDmkIQAAAAAAAAAA.png}

The rationale behind this is best explained on the example of the
heading counter: An update to the heading counter depends on the
heading\textquotesingle s level. By stepping directly before the
heading, we can correctly step from \texttt{\ 1\ } to \texttt{\ 1.1\ }
when encountering a level 2 heading. If we were to step after the
heading, we wouldn\textquotesingle t know what to step to.

Because counters should always be stepped before the elements they
count, they always start at zero. This way, they are at one for the
first display (which happens after the first step).

\subsection{Time travel}\label{time-travel}

Counters can travel through time! You can find out the final value of
the counter before it is reached and even determine what the value was
at any particular location in the document.

\begin{verbatim}
#let mine = counter("mycounter")

= Values
#context [
  Value here: #mine.get() \
  At intro: #mine.at(<intro>) \
  Final value: #mine.final()
]

#mine.update(n => n + 3)

= Introduction <intro>
#lorem(10)

#mine.step()
#mine.step()
\end{verbatim}

\includegraphics[width=5in,height=\textheight,keepaspectratio]{/assets/docs/wodRGpSsJgDZtfsMk_GNgwAAAAAAAAAA.png}

\subsection{Other kinds of state}\label{other-state}

The \texttt{\ counter\ } type is closely related to
\href{/docs/reference/introspection/state/}{state} type. Read its
documentation for more details on state management in Typst and why it
doesn\textquotesingle t just use normal variables for counters.

\subsection{\texorpdfstring{Constructor
{}}{Constructor }}\label{constructor}

\phantomsection\label{constructor-constructor-tooltip}
If a type has a constructor, you can call it like a function to create a
new value of the type.

Create a new counter identified by a key.

{ counter } (

{ \href{/docs/reference/foundations/str/}{str}
\href{/docs/reference/foundations/label/}{label}
\href{/docs/reference/foundations/selector/}{selector}
\href{/docs/reference/introspection/location/}{location}
\href{/docs/reference/foundations/function/}{function} }

) -\textgreater{} \href{/docs/reference/introspection/counter/}{counter}

\paragraph{\texorpdfstring{\texttt{\ key\ }}{ key }}\label{constructor-key}

\href{/docs/reference/foundations/str/}{str} {or}
\href{/docs/reference/foundations/label/}{label} {or}
\href{/docs/reference/foundations/selector/}{selector} {or}
\href{/docs/reference/introspection/location/}{location} {or}
\href{/docs/reference/foundations/function/}{function}

{Required} {{ Positional }}

\phantomsection\label{constructor-key-positional-tooltip}
Positional parameters are specified in order, without names.

The key that identifies this counter.

\begin{itemize}
\tightlist
\item
  If it is a string, creates a custom counter that is only affected by
  manual updates,
\item
  If it is the \href{/docs/reference/layout/page/}{\texttt{\ page\ }}
  function, counts through pages,
\item
  If it is a \href{/docs/reference/foundations/selector/}{selector} ,
  counts through elements that matches with the selector. For example,

  \begin{itemize}
  \tightlist
  \item
    provide an element function: counts elements of that type,
  \item
    provide a
    \href{/docs/reference/foundations/label/}{\texttt{\ }{\texttt{\ \textless{}label\textgreater{}\ }}\texttt{\ }}
    : counts elements with that label.
  \end{itemize}
\end{itemize}

\subsection{\texorpdfstring{{ Definitions
}}{ Definitions }}\label{definitions}

\phantomsection\label{definitions-tooltip}
Functions and types and can have associated definitions. These are
accessed by specifying the function or type, followed by a period, and
then the definition\textquotesingle s name.

\subsubsection{\texorpdfstring{\texttt{\ get\ } {{ Contextual
}}}{ get   Contextual }}\label{definitions-get}

\phantomsection\label{definitions-get-contextual-tooltip}
Contextual functions can only be used when the context is known

Retrieves the value of the counter at the current location. Always
returns an array of integers, even if the counter has just one number.

This is equivalent to
\texttt{\ counter\ }{\texttt{\ .\ }}\texttt{\ }{\texttt{\ at\ }}\texttt{\ }{\texttt{\ (\ }}\texttt{\ }{\texttt{\ here\ }}\texttt{\ }{\texttt{\ (\ }}\texttt{\ }{\texttt{\ )\ }}\texttt{\ }{\texttt{\ )\ }}\texttt{\ }
.

self { . } { get } (

) -\textgreater{} \href{/docs/reference/foundations/int/}{int}
\href{/docs/reference/foundations/array/}{array}

\subsubsection{\texorpdfstring{\texttt{\ display\ } {{ Contextual
}}}{ display   Contextual }}\label{definitions-display}

\phantomsection\label{definitions-display-contextual-tooltip}
Contextual functions can only be used when the context is known

Displays the current value of the counter with a numbering and returns
the formatted output.

\emph{Compatibility:} For compatibility with Typst 0.10 and lower, this
function also works without an established context. Then, it will create
opaque contextual content rather than directly returning the output of
the numbering. This behaviour will be removed in a future release.

self { . } { display } (

{ \href{/docs/reference/foundations/auto/}{auto}
\href{/docs/reference/foundations/str/}{str}
\href{/docs/reference/foundations/function/}{function} , } {
\hyperref[definitions-display-parameters-both]{both :}
\href{/docs/reference/foundations/bool/}{bool} , }

) -\textgreater{} { any }

\paragraph{\texorpdfstring{\texttt{\ numbering\ }}{ numbering }}\label{definitions-display-numbering}

\href{/docs/reference/foundations/auto/}{auto} {or}
\href{/docs/reference/foundations/str/}{str} {or}
\href{/docs/reference/foundations/function/}{function}

{{ Positional }}

\phantomsection\label{definitions-display-numbering-positional-tooltip}
Positional parameters are specified in order, without names.

A \href{/docs/reference/model/numbering/}{numbering pattern or a
function} , which specifies how to display the counter. If given a
function, that function receives each number of the counter as a
separate argument. If the amount of numbers varies, e.g. for the heading
argument, you can use an
\href{/docs/reference/foundations/arguments/}{argument sink} .

If this is omitted or set to \texttt{\ }{\texttt{\ auto\ }}\texttt{\ } ,
displays the counter with the numbering style for the counted element or
with the pattern \texttt{\ }{\texttt{\ "1.1"\ }}\texttt{\ } if no such
style exists.

Default: \texttt{\ }{\texttt{\ auto\ }}\texttt{\ }

\paragraph{\texorpdfstring{\texttt{\ both\ }}{ both }}\label{definitions-display-both}

\href{/docs/reference/foundations/bool/}{bool}

If enabled, displays the current and final top-level count together.
Both can be styled through a single numbering pattern. This is used by
the page numbering property to display the current and total number of
pages when a pattern like \texttt{\ }{\texttt{\ "1\ /\ 1"\ }}\texttt{\ }
is given.

Default: \texttt{\ }{\texttt{\ false\ }}\texttt{\ }

\subsubsection{\texorpdfstring{\texttt{\ at\ } {{ Contextual
}}}{ at   Contextual }}\label{definitions-at}

\phantomsection\label{definitions-at-contextual-tooltip}
Contextual functions can only be used when the context is known

Retrieves the value of the counter at the given location. Always returns
an array of integers, even if the counter has just one number.

The \texttt{\ selector\ } must match exactly one element in the
document. The most useful kinds of selectors for this are
\href{/docs/reference/foundations/label/}{labels} and
\href{/docs/reference/introspection/location/}{locations} .

\emph{Compatibility:} For compatibility with Typst 0.10 and lower, this
function also works without a known context if the \texttt{\ selector\ }
is a location. This behaviour will be removed in a future release.

self { . } { at } (

{ \href{/docs/reference/foundations/label/}{label}
\href{/docs/reference/foundations/selector/}{selector}
\href{/docs/reference/introspection/location/}{location}
\href{/docs/reference/foundations/function/}{function} }

) -\textgreater{} \href{/docs/reference/foundations/int/}{int}
\href{/docs/reference/foundations/array/}{array}

\paragraph{\texorpdfstring{\texttt{\ selector\ }}{ selector }}\label{definitions-at-selector}

\href{/docs/reference/foundations/label/}{label} {or}
\href{/docs/reference/foundations/selector/}{selector} {or}
\href{/docs/reference/introspection/location/}{location} {or}
\href{/docs/reference/foundations/function/}{function}

{Required} {{ Positional }}

\phantomsection\label{definitions-at-selector-positional-tooltip}
Positional parameters are specified in order, without names.

The place at which the counter\textquotesingle s value should be
retrieved.

\subsubsection{\texorpdfstring{\texttt{\ final\ } {{ Contextual
}}}{ final   Contextual }}\label{definitions-final}

\phantomsection\label{definitions-final-contextual-tooltip}
Contextual functions can only be used when the context is known

Retrieves the value of the counter at the end of the document. Always
returns an array of integers, even if the counter has just one number.

self { . } { final } (

{ \href{/docs/reference/foundations/none/}{none}
\href{/docs/reference/introspection/location/}{location} }

) -\textgreater{} \href{/docs/reference/foundations/int/}{int}
\href{/docs/reference/foundations/array/}{array}

\paragraph{\texorpdfstring{\texttt{\ location\ }}{ location }}\label{definitions-final-location}

\href{/docs/reference/foundations/none/}{none} {or}
\href{/docs/reference/introspection/location/}{location}

{{ Positional }}

\phantomsection\label{definitions-final-location-positional-tooltip}
Positional parameters are specified in order, without names.

\emph{Compatibility:} This argument is deprecated. It only exists for
compatibility with Typst 0.10 and lower and shouldn\textquotesingle t be
used anymore.

Default: \texttt{\ }{\texttt{\ none\ }}\texttt{\ }

\subsubsection{\texorpdfstring{\texttt{\ step\ }}{ step }}\label{definitions-step}

Increases the value of the counter by one.

The update will be in effect at the position where the returned content
is inserted into the document. If you don\textquotesingle t put the
output into the document, nothing happens! This would be the case, for
example, if you write
\texttt{\ }{\texttt{\ let\ }}\texttt{\ \_\ }{\texttt{\ =\ }}\texttt{\ }{\texttt{\ counter\ }}\texttt{\ }{\texttt{\ (\ }}\texttt{\ page\ }{\texttt{\ )\ }}\texttt{\ }{\texttt{\ .\ }}\texttt{\ }{\texttt{\ step\ }}\texttt{\ }{\texttt{\ (\ }}\texttt{\ }{\texttt{\ )\ }}\texttt{\ }
. Counter updates are always applied in layout order and in that case,
Typst wouldn\textquotesingle t know when to step the counter.

self { . } { step } (

{ \hyperref[definitions-step-parameters-level]{level :}
\href{/docs/reference/foundations/int/}{int} }

) -\textgreater{} \href{/docs/reference/foundations/content/}{content}

\paragraph{\texorpdfstring{\texttt{\ level\ }}{ level }}\label{definitions-step-level}

\href{/docs/reference/foundations/int/}{int}

The depth at which to step the counter. Defaults to
\texttt{\ }{\texttt{\ 1\ }}\texttt{\ } .

Default: \texttt{\ }{\texttt{\ 1\ }}\texttt{\ }

\subsubsection{\texorpdfstring{\texttt{\ update\ }}{ update }}\label{definitions-update}

Updates the value of the counter.

Just like with \texttt{\ step\ } , the update only occurs if you put the
resulting content into the document.

self { . } { update } (

{ \href{/docs/reference/foundations/int/}{int}
\href{/docs/reference/foundations/array/}{array}
\href{/docs/reference/foundations/function/}{function} }

) -\textgreater{} \href{/docs/reference/foundations/content/}{content}

\paragraph{\texorpdfstring{\texttt{\ update\ }}{ update }}\label{definitions-update-update}

\href{/docs/reference/foundations/int/}{int} {or}
\href{/docs/reference/foundations/array/}{array} {or}
\href{/docs/reference/foundations/function/}{function}

{Required} {{ Positional }}

\phantomsection\label{definitions-update-update-positional-tooltip}
Positional parameters are specified in order, without names.

If given an integer or array of integers, sets the counter to that
value. If given a function, that function receives the previous counter
value (with each number as a separate argument) and has to return the
new value (integer or array).

\href{/docs/reference/introspection/}{\pandocbounded{\includesvg[keepaspectratio]{/assets/icons/16-arrow-right.svg}}}

{ Introspection } { Previous page }

\href{/docs/reference/introspection/here/}{\pandocbounded{\includesvg[keepaspectratio]{/assets/icons/16-arrow-right.svg}}}

{ Here } { Next page }

\textbf{On this page}

\begin{itemize}
\tightlist
\item
  \hyperref[summary]{Summary}
\item
  \hyperref[accessing]{Accessing}
\item
  \hyperref[displaying]{Displaying}
\item
  \hyperref[modifying]{Modifying}
\item
  \hyperref[page-counter]{Page Counter}
\item
  \hyperref[custom-counters]{Custom Counters}
\item
  \hyperref[how-to-step]{How To Step}
\item
  \hyperref[time-travel]{Time Travel}
\item
  \hyperref[other-state]{Other State}
\item
  \hyperref[constructor]{Constructor}

  \begin{itemize}
  \tightlist
  \item
    \hyperref[constructor-key]{key}
  \end{itemize}
\item
  \hyperref[definitions]{Definitions}

  \begin{itemize}
  \tightlist
  \item
    \hyperref[definitions-get]{Get}
  \item
    \hyperref[definitions-display]{Display}

    \begin{itemize}
    \tightlist
    \item
      \hyperref[definitions-display-numbering]{numbering}
    \item
      \hyperref[definitions-display-both]{both}
    \end{itemize}
  \item
    \hyperref[definitions-at]{At}

    \begin{itemize}
    \tightlist
    \item
      \hyperref[definitions-at-selector]{selector}
    \end{itemize}
  \item
    \hyperref[definitions-final]{Final}

    \begin{itemize}
    \tightlist
    \item
      \hyperref[definitions-final-location]{location}
    \end{itemize}
  \item
    \hyperref[definitions-step]{Step}

    \begin{itemize}
    \tightlist
    \item
      \hyperref[definitions-step-level]{level}
    \end{itemize}
  \item
    \hyperref[definitions-update]{Update}

    \begin{itemize}
    \tightlist
    \item
      \hyperref[definitions-update-update]{update}
    \end{itemize}
  \end{itemize}
\end{itemize}

\begin{itemize}
\tightlist
\item
  \href{/}{Home}
\item
  \href{/pricing/}{Pricing}
\item
  \href{/docs/}{Documentation}
\item
  \href{/universe/}{Universe}
\item
  \href{/about/}{About Us}
\item
  \href{/contact/}{Contact Us}
\item
  \href{/privacy/}{Privacy}
\item
  \href{https://typst.app/terms}{Terms and Conditions}
\item
  \href{/legal/}{Legal (Impressum)}
\end{itemize}

\begin{itemize}
\tightlist
\item
  \href{https://forum.typst.app}{Forum}
\item
  \href{/tools/}{Tools}
\item
  \href{/blog/}{Blog}
\item
  \href{https://github.com/typst/}{GitHub}
\item
  \href{https://discord.gg/2uDybryKPe}{Discord}
\item
  \href{https://mastodon.social/@typst}{Mastodon}
\item
  \href{https://bsky.app/profile/typst.app}{Bluesky}
\item
  \href{https://www.linkedin.com/company/typst/}{LinkedIn}
\item
  \href{https://instagram.com/typstapp/}{Instagram}
\end{itemize}

Made in Berlin
