\title{typst.app/docs/reference/data-loading/json}

\begin{itemize}
\tightlist
\item
  \href{/docs}{\includesvg[width=0.16667in,height=0.16667in]{/assets/icons/16-docs-dark.svg}}
\item
  \includesvg[width=0.16667in,height=0.16667in]{/assets/icons/16-arrow-right.svg}
\item
  \href{/docs/reference/}{Reference}
\item
  \includesvg[width=0.16667in,height=0.16667in]{/assets/icons/16-arrow-right.svg}
\item
  \href{/docs/reference/data-loading/}{Data Loading}
\item
  \includesvg[width=0.16667in,height=0.16667in]{/assets/icons/16-arrow-right.svg}
\item
  \href{/docs/reference/data-loading/json/}{JSON}
\end{itemize}

\section{\texorpdfstring{\texttt{\ json\ }}{ json }}\label{summary}

Reads structured data from a JSON file.

The file must contain a valid JSON value, such as object or array. JSON
objects will be converted into Typst dictionaries, and JSON arrays will
be converted into Typst arrays. Strings and booleans will be converted
into the Typst equivalents, \texttt{\ null\ } will be converted into
\texttt{\ }{\texttt{\ none\ }}\texttt{\ } , and numbers will be
converted to floats or integers depending on whether they are whole
numbers.

Be aware that integers larger than 2 \textsuperscript{63} -1 will be
converted to floating point numbers, which may result in an
approximative value.

The function returns a dictionary, an array or, depending on the JSON
file, another JSON data type.

The JSON files in the example contain objects with the keys
\texttt{\ temperature\ } , \texttt{\ unit\ } , and \texttt{\ weather\ }
.

\subsection{Example}\label{example}

\begin{verbatim}
#let forecast(day) = block[
  #box(square(
    width: 2cm,
    inset: 8pt,
    fill: if day.weather == "sunny" {
      yellow
    } else {
      aqua
    },
    align(
      bottom + right,
      strong(day.weather),
    ),
  ))
  #h(6pt)
  #set text(22pt, baseline: -8pt)
  #day.temperature °#day.unit
]

#forecast(json("monday.json"))
#forecast(json("tuesday.json"))
\end{verbatim}

\includegraphics[width=5in,height=\textheight,keepaspectratio]{/assets/docs/9TGGThvdnznDbVRRo5-HsgAAAAAAAAAA.png}

\subsection{\texorpdfstring{{ Parameters
}}{ Parameters }}\label{parameters}

\phantomsection\label{parameters-tooltip}
Parameters are the inputs to a function. They are specified in
parentheses after the function name.

{ json } (

{ \href{/docs/reference/foundations/str/}{str} }

) -\textgreater{} { any }

\subsubsection{\texorpdfstring{\texttt{\ path\ }}{ path }}\label{parameters-path}

\href{/docs/reference/foundations/str/}{str}

{Required} {{ Positional }}

\phantomsection\label{parameters-path-positional-tooltip}
Positional parameters are specified in order, without names.

Path to a JSON file.

For more details, see the \href{/docs/reference/syntax/\#paths}{Paths
section} .

\subsection{\texorpdfstring{{ Definitions
}}{ Definitions }}\label{definitions}

\phantomsection\label{definitions-tooltip}
Functions and types and can have associated definitions. These are
accessed by specifying the function or type, followed by a period, and
then the definition\textquotesingle s name.

\subsubsection{\texorpdfstring{\texttt{\ decode\ }}{ decode }}\label{definitions-decode}

Reads structured data from a JSON string/bytes.

json { . } { decode } (

{ \href{/docs/reference/foundations/str/}{str}
\href{/docs/reference/foundations/bytes/}{bytes} }

) -\textgreater{} { any }

\paragraph{\texorpdfstring{\texttt{\ data\ }}{ data }}\label{definitions-decode-data}

\href{/docs/reference/foundations/str/}{str} {or}
\href{/docs/reference/foundations/bytes/}{bytes}

{Required} {{ Positional }}

\phantomsection\label{definitions-decode-data-positional-tooltip}
Positional parameters are specified in order, without names.

JSON data.

\subsubsection{\texorpdfstring{\texttt{\ encode\ }}{ encode }}\label{definitions-encode}

Encodes structured data into a JSON string.

json { . } { encode } (

{ { any } , } { \hyperref[definitions-encode-parameters-pretty]{pretty
:} \href{/docs/reference/foundations/bool/}{bool} , }

) -\textgreater{} \href{/docs/reference/foundations/str/}{str}

\paragraph{\texorpdfstring{\texttt{\ value\ }}{ value }}\label{definitions-encode-value}

{ any }

{Required} {{ Positional }}

\phantomsection\label{definitions-encode-value-positional-tooltip}
Positional parameters are specified in order, without names.

Value to be encoded.

\paragraph{\texorpdfstring{\texttt{\ pretty\ }}{ pretty }}\label{definitions-encode-pretty}

\href{/docs/reference/foundations/bool/}{bool}

Whether to pretty print the JSON with newlines and indentation.

Default: \texttt{\ }{\texttt{\ true\ }}\texttt{\ }

\href{/docs/reference/data-loading/csv/}{\pandocbounded{\includesvg[keepaspectratio]{/assets/icons/16-arrow-right.svg}}}

{ CSV } { Previous page }

\href{/docs/reference/data-loading/read/}{\pandocbounded{\includesvg[keepaspectratio]{/assets/icons/16-arrow-right.svg}}}

{ Read } { Next page }

\textbf{On this page}

\begin{itemize}
\tightlist
\item
  \hyperref[summary]{Summary}
\item
  \hyperref[example]{Example}
\item
  \hyperref[parameters]{Parameters}

  \begin{itemize}
  \tightlist
  \item
    \hyperref[parameters-path]{path}
  \end{itemize}
\item
  \hyperref[definitions]{Definitions}

  \begin{itemize}
  \tightlist
  \item
    \hyperref[definitions-decode]{Decode JSON}

    \begin{itemize}
    \tightlist
    \item
      \hyperref[definitions-decode-data]{data}
    \end{itemize}
  \item
    \hyperref[definitions-encode]{Encode JSON}

    \begin{itemize}
    \tightlist
    \item
      \hyperref[definitions-encode-value]{value}
    \item
      \hyperref[definitions-encode-pretty]{pretty}
    \end{itemize}
  \end{itemize}
\end{itemize}

\begin{itemize}
\tightlist
\item
  \href{/}{Home}
\item
  \href{/pricing/}{Pricing}
\item
  \href{/docs/}{Documentation}
\item
  \href{/universe/}{Universe}
\item
  \href{/about/}{About Us}
\item
  \href{/contact/}{Contact Us}
\item
  \href{/privacy/}{Privacy}
\item
  \href{https://typst.app/terms}{Terms and Conditions}
\item
  \href{/legal/}{Legal (Impressum)}
\end{itemize}

\begin{itemize}
\tightlist
\item
  \href{https://forum.typst.app}{Forum}
\item
  \href{/tools/}{Tools}
\item
  \href{/blog/}{Blog}
\item
  \href{https://github.com/typst/}{GitHub}
\item
  \href{https://discord.gg/2uDybryKPe}{Discord}
\item
  \href{https://mastodon.social/@typst}{Mastodon}
\item
  \href{https://bsky.app/profile/typst.app}{Bluesky}
\item
  \href{https://www.linkedin.com/company/typst/}{LinkedIn}
\item
  \href{https://instagram.com/typstapp/}{Instagram}
\end{itemize}

Made in Berlin
