\title{typst.app/docs/reference/scripting}

\begin{itemize}
\tightlist
\item
  \href{/docs}{\includesvg[width=0.16667in,height=0.16667in]{/assets/icons/16-docs-dark.svg}}
\item
  \includesvg[width=0.16667in,height=0.16667in]{/assets/icons/16-arrow-right.svg}
\item
  \href{/docs/reference/}{Reference}
\item
  \includesvg[width=0.16667in,height=0.16667in]{/assets/icons/16-arrow-right.svg}
\item
  \href{/docs/reference/scripting/}{Scripting}
\end{itemize}

\section{Scripting}\label{scripting}

Typst embeds a powerful scripting language. You can automate your
documents and create more sophisticated styles with code. Below is an
overview over the scripting concepts.

\subsection{Expressions}\label{expressions}

In Typst, markup and code are fused into one. All but the most common
elements are created with \emph{functions.} To make this as convenient
as possible, Typst provides compact syntax to embed a code expression
into markup: An expression is introduced with a hash ( \texttt{\ \#\ } )
and normal markup parsing resumes after the expression is finished. If a
character would continue the expression but should be interpreted as
text, the expression can forcibly be ended with a semicolon (
\texttt{\ ;\ } ).

\begin{verbatim}
#emph[Hello] \
#emoji.face \
#"hello".len()
\end{verbatim}

\includegraphics[width=5in,height=\textheight,keepaspectratio]{/assets/docs/Vvzr_VTofgbwk2d3ymxPMQAAAAAAAAAA.png}

The example above shows a few of the available expressions, including
\href{/docs/reference/foundations/function/}{function calls} ,
\href{/docs/reference/scripting/\#fields}{field accesses} , and
\href{/docs/reference/scripting/\#methods}{method calls} . More kinds of
expressions are discussed in the remainder of this chapter. A few kinds
of expressions are not compatible with the hash syntax (e.g. binary
operator expressions). To embed these into markup, you can use
parentheses, as in
\texttt{\ }{\texttt{\ \#\ }}\texttt{\ }{\texttt{\ (\ }}\texttt{\ }{\texttt{\ 1\ }}\texttt{\ }{\texttt{\ +\ }}\texttt{\ }{\texttt{\ 2\ }}\texttt{\ }{\texttt{\ )\ }}\texttt{\ }
.

\subsection{Blocks}\label{blocks}

To structure your code and embed markup into it, Typst provides two
kinds of \emph{blocks:}

\begin{itemize}
\item
  \textbf{Code block:}
  \texttt{\ }{\texttt{\ \{\ }}\texttt{\ }{\texttt{\ let\ }}\texttt{\ x\ }{\texttt{\ =\ }}\texttt{\ }{\texttt{\ 1\ }}\texttt{\ }{\texttt{\ ;\ }}\texttt{\ x\ }{\texttt{\ +\ }}\texttt{\ }{\texttt{\ 2\ }}\texttt{\ }{\texttt{\ \}\ }}\texttt{\ }\\
  When writing code, you\textquotesingle ll probably want to split up
  your computation into multiple statements, create some intermediate
  variables and so on. Code blocks let you write multiple expressions
  where one is expected. The individual expressions in a code block
  should be separated by line breaks or semicolons. The output values of
  the individual expressions in a code block are joined to determine the
  block\textquotesingle s value. Expressions without useful output, like
  \texttt{\ }{\texttt{\ let\ }}\texttt{\ } bindings yield
  \texttt{\ }{\texttt{\ none\ }}\texttt{\ } , which can be joined with
  any value without effect.
\item
  \textbf{Content block:}
  \texttt{\ }{\texttt{\ {[}\ }}\texttt{\ }{\texttt{\ *Hey*\ }}\texttt{\ there!\ }{\texttt{\ {]}\ }}\texttt{\ }\\
  With content blocks, you can handle markup/content as a programmatic
  value, store it in variables and pass it to
  \href{/docs/reference/foundations/function/}{functions} . Content
  blocks are delimited by square brackets and can contain arbitrary
  markup. A content block results in a value of type
  \href{/docs/reference/foundations/content/}{content} . An arbitrary
  number of content blocks can be passed as trailing arguments to
  functions. That is,
  \texttt{\ }{\texttt{\ list\ }}\texttt{\ }{\texttt{\ (\ }}\texttt{\ }{\texttt{\ {[}\ }}\texttt{\ A\ }{\texttt{\ {]}\ }}\texttt{\ }{\texttt{\ ,\ }}\texttt{\ }{\texttt{\ {[}\ }}\texttt{\ B\ }{\texttt{\ {]}\ }}\texttt{\ }{\texttt{\ )\ }}\texttt{\ }
  is equivalent to
  \texttt{\ }{\texttt{\ list\ }}\texttt{\ }{\texttt{\ {[}\ }}\texttt{\ A\ }{\texttt{\ {]}\ }}\texttt{\ }{\texttt{\ {[}\ }}\texttt{\ B\ }{\texttt{\ {]}\ }}\texttt{\ }
  .
\end{itemize}

Content and code blocks can be nested arbitrarily. In the example below,
\texttt{\ }{\texttt{\ {[}\ }}\texttt{\ hello\ }{\texttt{\ {]}\ }}\texttt{\ }
is joined with the output of
\texttt{\ a\ }{\texttt{\ +\ }}\texttt{\ }{\texttt{\ {[}\ }}\texttt{\ the\ }{\texttt{\ {]}\ }}\texttt{\ }{\texttt{\ +\ }}\texttt{\ b\ }
yielding
\texttt{\ }{\texttt{\ {[}\ }}\texttt{\ hello\ from\ the\ }{\texttt{\ *world*\ }}\texttt{\ }{\texttt{\ {]}\ }}\texttt{\ }
.

\begin{verbatim}
#{
  let a = [from]
  let b = [*world*]
  [hello ]
  a + [ the ] + b
}
\end{verbatim}

\includegraphics[width=5in,height=\textheight,keepaspectratio]{/assets/docs/9fGlmpI93XiZ44REV_aDoQAAAAAAAAAA.png}

\subsection{Bindings and Destructuring}\label{bindings}

As already demonstrated above, variables can be defined with
\texttt{\ }{\texttt{\ let\ }}\texttt{\ } bindings. The variable is
assigned the value of the expression that follows the \texttt{\ =\ }
sign. The assignment of a value is optional, if no value is assigned,
the variable will be initialized as
\texttt{\ }{\texttt{\ none\ }}\texttt{\ } . The
\texttt{\ }{\texttt{\ let\ }}\texttt{\ } keyword can also be used to
create a
\href{/docs/reference/foundations/function/\#defining-functions}{custom
named function} . Variables can be accessed for the rest of the
containing block (or the rest of the file if there is no containing
block).

\begin{verbatim}
#let name = "Typst"
This is #name's documentation.
It explains #name.

#let add(x, y) = x + y
Sum is #add(2, 3).
\end{verbatim}

\includegraphics[width=5in,height=\textheight,keepaspectratio]{/assets/docs/yrL9Iv6avU1LgnwbwwruLwAAAAAAAAAA.png}

Let bindings can also be used to destructure
\href{/docs/reference/foundations/array/}{arrays} and
\href{/docs/reference/foundations/dictionary/}{dictionaries} . In this
case, the left-hand side of the assignment should mirror an array or
dictionary. The \texttt{\ ..\ } operator can be used once in the pattern
to collect the remainder of the array\textquotesingle s or
dictionary\textquotesingle s items.

\begin{verbatim}
#let (x, y) = (1, 2)
The coordinates are #x, #y.

#let (a, .., b) = (1, 2, 3, 4)
The first element is #a.
The last element is #b.

#let books = (
  Shakespeare: "Hamlet",
  Homer: "The Odyssey",
  Austen: "Persuasion",
)

#let (Austen,) = books
Austen wrote #Austen.

#let (Homer: h) = books
Homer wrote #h.

#let (Homer, ..other) = books
#for (author, title) in other [
  #author wrote #title.
]
\end{verbatim}

\includegraphics[width=5in,height=\textheight,keepaspectratio]{/assets/docs/V0qKVNlCRuARFWpYu5iPEAAAAAAAAAAA.png}

You can use the underscore to discard elements in a destructuring
pattern:

\begin{verbatim}
#let (_, y, _) = (1, 2, 3)
The y coordinate is #y.
\end{verbatim}

\includegraphics[width=5in,height=\textheight,keepaspectratio]{/assets/docs/LTIPVXoxwTxHgnZJYm9hXgAAAAAAAAAA.png}

Destructuring also work in argument lists of functions ...

\begin{verbatim}
#let left = (2, 4, 5)
#let right = (3, 2, 6)
#left.zip(right).map(
  ((a,b)) => a + b
)
\end{verbatim}

\includegraphics[width=5in,height=\textheight,keepaspectratio]{/assets/docs/60UdChzzZGHopWziA6zwZwAAAAAAAAAA.png}

... and on the left-hand side of normal assignments. This can be useful
to swap variables among other things.

\begin{verbatim}
#{
  let a = 1
  let b = 2
  (a, b) = (b, a)
  [a = #a, b = #b]
}
\end{verbatim}

\includegraphics[width=5in,height=\textheight,keepaspectratio]{/assets/docs/LCvMaiUJqV2qC8Tphu5-bQAAAAAAAAAA.png}

\subsection{Conditionals}\label{conditionals}

With a conditional, you can display or compute different things
depending on whether some condition is fulfilled. Typst supports
\texttt{\ }{\texttt{\ if\ }}\texttt{\ } ,
\texttt{\ else\ }{\texttt{\ if\ }}\texttt{\ } and \texttt{\ else\ }
expression. When the condition evaluates to
\texttt{\ }{\texttt{\ true\ }}\texttt{\ } , the conditional yields the
value resulting from the if\textquotesingle s body, otherwise yields the
value resulting from the else\textquotesingle s body.

\begin{verbatim}
#if 1 < 2 [
  This is shown
] else [
  This is not.
]
\end{verbatim}

\includegraphics[width=5in,height=\textheight,keepaspectratio]{/assets/docs/nwPf6X84WrGm0BEnqRUsTQAAAAAAAAAA.png}

Each branch can have a code or content block as its body.

\begin{itemize}
\tightlist
\item
  \texttt{\ }{\texttt{\ if\ }}\texttt{\ condition\ }{\texttt{\ \{\ }}\texttt{\ ..\ }{\texttt{\ \}\ }}\texttt{\ }
\item
  \texttt{\ }{\texttt{\ if\ }}\texttt{\ condition\ }{\texttt{\ {[}\ }}\texttt{\ ..\ }{\texttt{\ {]}\ }}\texttt{\ }
\item
  \texttt{\ }{\texttt{\ if\ }}\texttt{\ condition\ }{\texttt{\ {[}\ }}\texttt{\ ..\ }{\texttt{\ {]}\ }}\texttt{\ }{\texttt{\ else\ }}\texttt{\ }{\texttt{\ \{\ }}\texttt{\ ..\ }{\texttt{\ \}\ }}\texttt{\ }
\item
  \texttt{\ }{\texttt{\ if\ }}\texttt{\ condition\ }{\texttt{\ {[}\ }}\texttt{\ ..\ }{\texttt{\ {]}\ }}\texttt{\ }{\texttt{\ else\ }}\texttt{\ }{\texttt{\ if\ }}\texttt{\ condition\ }{\texttt{\ \{\ }}\texttt{\ ..\ }{\texttt{\ \}\ }}\texttt{\ }{\texttt{\ else\ }}\texttt{\ }{\texttt{\ {[}\ }}\texttt{\ ..\ }{\texttt{\ {]}\ }}\texttt{\ }
\end{itemize}

\subsection{Loops}\label{loops}

With loops, you can repeat content or compute something iteratively.
Typst supports two types of loops:
\texttt{\ }{\texttt{\ for\ }}\texttt{\ } and
\texttt{\ }{\texttt{\ while\ }}\texttt{\ } loops. The former iterate
over a specified collection whereas the latter iterate as long as a
condition stays fulfilled. Just like blocks, loops \emph{join} the
results from each iteration into one value.

In the example below, the three sentences created by the for loop join
together into a single content value and the length-1 arrays in the
while loop join together into one larger array.

\begin{verbatim}
#for c in "ABC" [
  #c is a letter.
]

#let n = 2
#while n < 10 {
  n = (n * 2) - 1
  (n,)
}
\end{verbatim}

\includegraphics[width=5in,height=\textheight,keepaspectratio]{/assets/docs/74fsCbbxVkZaLeuPLKRiCwAAAAAAAAAA.png}

For loops can iterate over a variety of collections:

\begin{itemize}
\item
  \texttt{\ }{\texttt{\ for\ }}\texttt{\ value\ }{\texttt{\ in\ }}\texttt{\ array\ }{\texttt{\ \{\ }}\texttt{\ ..\ }{\texttt{\ \}\ }}\texttt{\ }\strut \\
  Iterates over the items in the
  \href{/docs/reference/foundations/array/}{array} . The destructuring
  syntax described in \href{/docs/reference/scripting/\#bindings}{Let
  binding} can also be used here.
\item
  \texttt{\ }{\texttt{\ for\ }}\texttt{\ pair\ }{\texttt{\ in\ }}\texttt{\ dict\ }{\texttt{\ \{\ }}\texttt{\ ..\ }{\texttt{\ \}\ }}\texttt{\ }\strut \\
  Iterates over the key-value pairs of the
  \href{/docs/reference/foundations/dictionary/}{dictionary} . The pairs
  can also be destructured by using
  \texttt{\ }{\texttt{\ for\ }}\texttt{\ }{\texttt{\ (\ }}\texttt{\ key\ }{\texttt{\ ,\ }}\texttt{\ value\ }{\texttt{\ )\ }}\texttt{\ }{\texttt{\ in\ }}\texttt{\ dict\ }{\texttt{\ \{\ }}\texttt{\ ..\ }{\texttt{\ \}\ }}\texttt{\ }
  . It is more efficient than
  \texttt{\ }{\texttt{\ for\ }}\texttt{\ pair\ }{\texttt{\ in\ }}\texttt{\ dict\ }{\texttt{\ .\ }}\texttt{\ }{\texttt{\ pairs\ }}\texttt{\ }{\texttt{\ (\ }}\texttt{\ }{\texttt{\ )\ }}\texttt{\ }{\texttt{\ \{\ }}\texttt{\ ..\ }{\texttt{\ \}\ }}\texttt{\ }
  because it doesn\textquotesingle t create a temporary array of all
  key-value pairs.
\item
  \texttt{\ }{\texttt{\ for\ }}\texttt{\ letter\ }{\texttt{\ in\ }}\texttt{\ }{\texttt{\ "abc"\ }}\texttt{\ }{\texttt{\ \{\ }}\texttt{\ ..\ }{\texttt{\ \}\ }}\texttt{\ }\strut \\
  Iterates over the characters of the
  \href{/docs/reference/foundations/str/}{string} . Technically, it
  iterates over the grapheme clusters of the string. Most of the time, a
  grapheme cluster is just a single codepoint. However, a grapheme
  cluster could contain multiple codepoints, like a flag emoji.
\item
  \texttt{\ }{\texttt{\ for\ }}\texttt{\ byte\ }{\texttt{\ in\ }}\texttt{\ }{\texttt{\ bytes\ }}\texttt{\ }{\texttt{\ (\ }}\texttt{\ }{\texttt{\ "😀"\ }}\texttt{\ }{\texttt{\ )\ }}\texttt{\ }{\texttt{\ \{\ }}\texttt{\ ..\ }{\texttt{\ \}\ }}\texttt{\ }\strut \\
  Iterates over the \href{/docs/reference/foundations/bytes/}{bytes} ,
  which can be converted from a
  \href{/docs/reference/foundations/str/}{string} or
  \href{/docs/reference/data-loading/read/}{read} from a file without
  encoding. Each byte value is an
  \href{/docs/reference/foundations/int/}{integer} between
  \texttt{\ }{\texttt{\ 0\ }}\texttt{\ } and
  \texttt{\ }{\texttt{\ 255\ }}\texttt{\ } .
\end{itemize}

To control the execution of the loop, Typst provides the
\texttt{\ }{\texttt{\ break\ }}\texttt{\ } and
\texttt{\ }{\texttt{\ continue\ }}\texttt{\ } statements. The former
performs an early exit from the loop while the latter skips ahead to the
next iteration of the loop.

\begin{verbatim}
#for letter in "abc nope" {
  if letter == " " {
    break
  }

  letter
}
\end{verbatim}

\includegraphics[width=5in,height=\textheight,keepaspectratio]{/assets/docs/i6FFy2h6Ocj5FGq9VflD-QAAAAAAAAAA.png}

The body of a loop can be a code or content block:

\begin{itemize}
\tightlist
\item
  \texttt{\ }{\texttt{\ for\ }}\texttt{\ ..\ in\ collection\ }{\texttt{\ \{\ }}\texttt{\ ..\ }{\texttt{\ \}\ }}\texttt{\ }
\item
  \texttt{\ }{\texttt{\ for\ }}\texttt{\ ..\ in\ collection\ }{\texttt{\ {[}\ }}\texttt{\ ..\ }{\texttt{\ {]}\ }}\texttt{\ }
\item
  \texttt{\ }{\texttt{\ while\ }}\texttt{\ condition\ }{\texttt{\ \{\ }}\texttt{\ ..\ }{\texttt{\ \}\ }}\texttt{\ }
\item
  \texttt{\ }{\texttt{\ while\ }}\texttt{\ condition\ }{\texttt{\ {[}\ }}\texttt{\ ..\ }{\texttt{\ {]}\ }}\texttt{\ }
\end{itemize}

\subsection{Fields}\label{fields}

You can use \emph{dot notation} to access fields on a value. For values
of type
\href{/docs/reference/foundations/content/}{\texttt{\ content\ }} , you
can also use the
\href{/docs/reference/foundations/content/\#definitions-fields}{\texttt{\ fields\ }}
function to list the fields.

The value in question can be either:

\begin{itemize}
\tightlist
\item
  a \href{/docs/reference/foundations/dictionary/}{dictionary} that has
  the specified key,
\item
  a \href{/docs/reference/symbols/symbol/}{symbol} that has the
  specified modifier,
\item
  a \href{/docs/reference/foundations/module/}{module} containing the
  specified definition,
\item
  \href{/docs/reference/foundations/content/}{content} consisting of an
  element that has the specified field. The available fields match the
  arguments of the
  \href{/docs/reference/foundations/function/\#element-functions}{element
  function} that were given when the element was constructed.
\end{itemize}

\begin{verbatim}
#let it = [= Heading]
#it.body \
#it.depth \
#it.fields()

#let dict = (greet: "Hello")
#dict.greet \
#emoji.face
\end{verbatim}

\includegraphics[width=5in,height=\textheight,keepaspectratio]{/assets/docs/5WDPdcADtV7mFakfulSQigAAAAAAAAAA.png}

\subsection{Methods}\label{methods}

A \emph{method call} is a convenient way to call a function that is
scoped to a value\textquotesingle s
\href{/docs/reference/foundations/type/}{type} . For example, we can
call the
\href{/docs/reference/foundations/str/\#definitions-len}{\texttt{\ str.len\ }}
function in the following two equivalent ways:

\begin{verbatim}
#str.len("abc") is the same as
#"abc".len()
\end{verbatim}

\includegraphics[width=5in,height=\textheight,keepaspectratio]{/assets/docs/j8kTSWLqLKb4876qGnfJQAAAAAAAAAAA.png}

The structure of a method call is
\texttt{\ value\ }{\texttt{\ .\ }}\texttt{\ }{\texttt{\ method\ }}\texttt{\ }{\texttt{\ (\ }}\texttt{\ }{\texttt{\ ..\ }}\texttt{\ args\ }{\texttt{\ )\ }}\texttt{\ }
and its equivalent full function call is
\texttt{\ }{\texttt{\ type\ }}\texttt{\ }{\texttt{\ (\ }}\texttt{\ value\ }{\texttt{\ )\ }}\texttt{\ }{\texttt{\ .\ }}\texttt{\ }{\texttt{\ method\ }}\texttt{\ }{\texttt{\ (\ }}\texttt{\ value\ }{\texttt{\ ,\ }}\texttt{\ }{\texttt{\ ..\ }}\texttt{\ args\ }{\texttt{\ )\ }}\texttt{\ }
. The documentation of each type lists it\textquotesingle s scoped
functions. You cannot currently define your own methods.

\begin{verbatim}
#let values = (1, 2, 3, 4)
#values.pop() \
#values.len() \

#("a, b, c"
    .split(", ")
    .join[ --- ])

#"abc".len() is the same as
#str.len("abc")
\end{verbatim}

\includegraphics[width=5in,height=\textheight,keepaspectratio]{/assets/docs/acgddNIdTiEri93pewLJSQAAAAAAAAAA.png}

There are a few special functions that modify the value they are called
on (e.g.
\href{/docs/reference/foundations/array/\#definitions-push}{\texttt{\ array.push\ }}
). These functions \emph{must} be called in method form. In some cases,
when the method is only called for its side effect, its return value
should be ignored (and not participate in joining). The canonical way to
discard a value is with a let binding:
\texttt{\ }{\texttt{\ let\ }}\texttt{\ \_\ }{\texttt{\ =\ }}\texttt{\ array\ }{\texttt{\ .\ }}\texttt{\ }{\texttt{\ remove\ }}\texttt{\ }{\texttt{\ (\ }}\texttt{\ }{\texttt{\ 1\ }}\texttt{\ }{\texttt{\ )\ }}\texttt{\ }
.

\subsection{Modules}\label{modules}

You can split up your Typst projects into multiple files called
\emph{modules.} A module can refer to the content and definitions of
another module in multiple ways:

\begin{itemize}
\item
  \textbf{Including:}
  \texttt{\ }{\texttt{\ include\ }}\texttt{\ }{\texttt{\ "bar.typ"\ }}\texttt{\ }\\
  Evaluates the file at the path \texttt{\ bar.typ\ } and returns the
  resulting \href{/docs/reference/foundations/content/}{content} .
\item
  \textbf{Import:}
  \texttt{\ }{\texttt{\ import\ }}\texttt{\ }{\texttt{\ "bar.typ"\ }}\texttt{\ }\\
  Evaluates the file at the path \texttt{\ bar.typ\ } and inserts the
  resulting \href{/docs/reference/foundations/module/}{module} into the
  current scope as \texttt{\ bar\ } (filename without extension). You
  can use the \texttt{\ as\ } keyword to rename the imported module:
  \texttt{\ }{\texttt{\ import\ }}\texttt{\ }{\texttt{\ "bar.typ"\ }}\texttt{\ }{\texttt{\ as\ }}\texttt{\ baz\ }
  . You can import nested items using dot notation:
  \texttt{\ }{\texttt{\ import\ }}\texttt{\ }{\texttt{\ "bar.typ"\ }}\texttt{\ }{\texttt{\ :\ }}\texttt{\ baz\ }{\texttt{\ .\ }}\texttt{\ a\ }
  .
\item
  \textbf{Import items:}
  \texttt{\ }{\texttt{\ import\ }}\texttt{\ }{\texttt{\ "bar.typ"\ }}\texttt{\ }{\texttt{\ :\ }}\texttt{\ a\ }{\texttt{\ ,\ }}\texttt{\ b\ }\\
  Evaluates the file at the path \texttt{\ bar.typ\ } , extracts the
  values of the variables \texttt{\ a\ } and \texttt{\ b\ } (that need
  to be defined in \texttt{\ bar.typ\ } , e.g. through
  \texttt{\ }{\texttt{\ let\ }}\texttt{\ } bindings) and defines them in
  the current file. Replacing \texttt{\ a,\ b\ } with \texttt{\ *\ }
  loads all variables defined in a module. You can use the
  \texttt{\ as\ } keyword to rename the individual items:
  \texttt{\ }{\texttt{\ import\ }}\texttt{\ }{\texttt{\ "bar.typ"\ }}\texttt{\ }{\texttt{\ :\ }}\texttt{\ a\ }{\texttt{\ as\ }}\texttt{\ one\ }{\texttt{\ ,\ }}\texttt{\ b\ }{\texttt{\ as\ }}\texttt{\ two\ }
\end{itemize}

Instead of a path, you can also use a
\href{/docs/reference/foundations/module/}{module value} , as shown in
the following example:

\begin{verbatim}
#import emoji: face
#face.grin
\end{verbatim}

\includegraphics[width=5in,height=\textheight,keepaspectratio]{/assets/docs/nDjZVIO8y_dHmoJCcIerKgAAAAAAAAAA.png}

\subsection{Packages}\label{packages}

To reuse building blocks across projects, you can also create and import
Typst \emph{packages.} A package import is specified as a triple of a
namespace, a name, and a version.

\begin{verbatim}
#import "@preview/example:0.1.0": add
#add(2, 7)
\end{verbatim}

\includegraphics[width=5in,height=\textheight,keepaspectratio]{/assets/docs/PZlFYONRfjgFeuwUfEQd5gAAAAAAAAAA.png}

The \texttt{\ preview\ } namespace contains packages shared by the
community. You can find all available community packages on
\href{https://typst.app/universe/}{Typst Universe} .

If you are using Typst locally, you can also create your own
system-local packages. For more details on this, see the
\href{https://github.com/typst/packages}{package repository} .

\subsection{Operators}\label{operators}

The following table lists all available unary and binary operators with
effect, arity (unary, binary) and precedence level (higher binds
stronger).

\begin{longtable}[]{@{}clcc@{}}
\toprule\noalign{}
Operator & Effect & Arity & Precedence \\
\midrule\noalign{}
\endhead
\bottomrule\noalign{}
\endlastfoot
\texttt{\ }{\texttt{\ -\ }}\texttt{\ } & Negation & Unary & 7 \\
\texttt{\ }{\texttt{\ +\ }}\texttt{\ } & No effect (exists for symmetry)
& Unary & 7 \\
\texttt{\ *\ } & Multiplication & Binary & 6 \\
\texttt{\ /\ } & Division & Binary & 6 \\
\texttt{\ }{\texttt{\ +\ }}\texttt{\ } & Addition & Binary & 5 \\
\texttt{\ }{\texttt{\ -\ }}\texttt{\ } & Subtraction & Binary & 5 \\
\texttt{\ ==\ } & Check equality & Binary & 4 \\
\texttt{\ !=\ } & Check inequality & Binary & 4 \\
\texttt{\ \textless{}\ } & Check less-than & Binary & 4 \\
\texttt{\ \textless{}=\ } & Check less-than or equal & Binary & 4 \\
\texttt{\ \textgreater{}\ } & Check greater-than & Binary & 4 \\
\texttt{\ \textgreater{}=\ } & Check greater-than or equal & Binary &
4 \\
\texttt{\ in\ } & Check if in collection & Binary & 4 \\
\texttt{\ }{\texttt{\ not\ }}\texttt{\ }{\texttt{\ in\ }}\texttt{\ } &
Check if not in collection & Binary & 4 \\
\texttt{\ }{\texttt{\ not\ }}\texttt{\ } & Logical "not" & Unary & 3 \\
\texttt{\ and\ } & Short-circuiting logical "and" & Binary & 3 \\
\texttt{\ or\ } & Short-circuiting logical "or & Binary & 2 \\
\texttt{\ =\ } & Assignment & Binary & 1 \\
\texttt{\ +=\ } & Add-Assignment & Binary & 1 \\
\texttt{\ -=\ } & Subtraction-Assignment & Binary & 1 \\
\texttt{\ *=\ } & Multiplication-Assignment & Binary & 1 \\
\texttt{\ /=\ } & Division-Assignment & Binary & 1 \\
\end{longtable}

\href{/docs/reference/styling/}{\pandocbounded{\includesvg[keepaspectratio]{/assets/icons/16-arrow-right.svg}}}

{ Styling } { Previous page }

\href{/docs/reference/context/}{\pandocbounded{\includesvg[keepaspectratio]{/assets/icons/16-arrow-right.svg}}}

{ Context } { Next page }
