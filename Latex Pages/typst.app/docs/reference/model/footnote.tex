\title{typst.app/docs/reference/model/footnote}

\begin{itemize}
\tightlist
\item
  \href{/docs}{\includesvg[width=0.16667in,height=0.16667in]{/assets/icons/16-docs-dark.svg}}
\item
  \includesvg[width=0.16667in,height=0.16667in]{/assets/icons/16-arrow-right.svg}
\item
  \href{/docs/reference/}{Reference}
\item
  \includesvg[width=0.16667in,height=0.16667in]{/assets/icons/16-arrow-right.svg}
\item
  \href{/docs/reference/model/}{Model}
\item
  \includesvg[width=0.16667in,height=0.16667in]{/assets/icons/16-arrow-right.svg}
\item
  \href{/docs/reference/model/footnote/}{Footnote}
\end{itemize}

\section{\texorpdfstring{\texttt{\ footnote\ } {{ Element
}}}{ footnote   Element }}\label{summary}

\phantomsection\label{element-tooltip}
Element functions can be customized with \texttt{\ set\ } and
\texttt{\ show\ } rules.

A footnote.

Includes additional remarks and references on the same page with
footnotes. A footnote will insert a superscript number that links to the
note at the bottom of the page. Notes are numbered sequentially
throughout your document and can break across multiple pages.

To customize the appearance of the entry in the footnote listing, see
\href{/docs/reference/model/footnote/\#definitions-entry}{\texttt{\ footnote.entry\ }}
. The footnote itself is realized as a normal superscript, so you can
use a set rule on the
\href{/docs/reference/text/super/}{\texttt{\ super\ }} function to
customize it. You can also apply a show rule to customize only the
footnote marker (superscript number) in the running text.

\subsection{Example}\label{example}

\begin{verbatim}
Check the docs for more details.
#footnote[https://typst.app/docs]
\end{verbatim}

\includegraphics[width=5in,height=\textheight,keepaspectratio]{/assets/docs/Rux1n4IPwOkOpn1s57WxpAAAAAAAAAAA.png}

The footnote automatically attaches itself to the preceding word, even
if there is a space before it in the markup. To force space, you can use
the string
\texttt{\ }{\texttt{\ \#\ }}\texttt{\ }{\texttt{\ "\ "\ }}\texttt{\ } or
explicit \href{/docs/reference/layout/h/}{horizontal spacing} .

By giving a label to a footnote, you can have multiple references to it.

\begin{verbatim}
You can edit Typst documents online.
#footnote[https://typst.app/app] <fn>
Checkout Typst's website. @fn
And the online app. #footnote(<fn>)
\end{verbatim}

\includegraphics[width=5in,height=\textheight,keepaspectratio]{/assets/docs/xECSHtr0VhzFh5onpw48GQAAAAAAAAAA.png}

\emph{Note:} Set and show rules in the scope where \texttt{\ footnote\ }
is called may not apply to the footnote\textquotesingle s content. See
\href{https://github.com/typst/typst/issues/1467\#issuecomment-1588799440}{here}
for more information.

\subsection{\texorpdfstring{{ Parameters
}}{ Parameters }}\label{parameters}

\phantomsection\label{parameters-tooltip}
Parameters are the inputs to a function. They are specified in
parentheses after the function name.

{ footnote } (

{ \hyperref[parameters-numbering]{numbering :}
\href{/docs/reference/foundations/str/}{str}
\href{/docs/reference/foundations/function/}{function} , } {
\href{/docs/reference/foundations/label/}{label}
\href{/docs/reference/foundations/content/}{content} , }

) -\textgreater{} \href{/docs/reference/foundations/content/}{content}

\subsubsection{\texorpdfstring{\texttt{\ numbering\ }}{ numbering }}\label{parameters-numbering}

\href{/docs/reference/foundations/str/}{str} {or}
\href{/docs/reference/foundations/function/}{function}

{{ Settable }}

\phantomsection\label{parameters-numbering-settable-tooltip}
Settable parameters can be customized for all following uses of the
function with a \texttt{\ set\ } rule.

How to number footnotes.

By default, the footnote numbering continues throughout your document.
If you prefer per-page footnote numbering, you can reset the footnote
\href{/docs/reference/introspection/counter/}{counter} in the page
\href{/docs/reference/layout/page/\#parameters-header}{header} . In the
future, there might be a simpler way to achieve this.

Default: \texttt{\ }{\texttt{\ "1"\ }}\texttt{\ }

\includesvg[width=0.16667in,height=0.16667in]{/assets/icons/16-arrow-right.svg}
View example

\begin{verbatim}
#set footnote(numbering: "*")

Footnotes:
#footnote[Star],
#footnote[Dagger]
\end{verbatim}

\includegraphics[width=5in,height=\textheight,keepaspectratio]{/assets/docs/CVlSBedIWhhzGwE8LefQmwAAAAAAAAAA.png}

\subsubsection{\texorpdfstring{\texttt{\ body\ }}{ body }}\label{parameters-body}

\href{/docs/reference/foundations/label/}{label} {or}
\href{/docs/reference/foundations/content/}{content}

{Required} {{ Positional }}

\phantomsection\label{parameters-body-positional-tooltip}
Positional parameters are specified in order, without names.

The content to put into the footnote. Can also be the label of another
footnote this one should point to.

\subsection{\texorpdfstring{{ Definitions
}}{ Definitions }}\label{definitions}

\phantomsection\label{definitions-tooltip}
Functions and types and can have associated definitions. These are
accessed by specifying the function or type, followed by a period, and
then the definition\textquotesingle s name.

\subsubsection{\texorpdfstring{\texttt{\ entry\ } {{ Element
}}}{ entry   Element }}\label{definitions-entry}

\phantomsection\label{definitions-entry-element-tooltip}
Element functions can be customized with \texttt{\ set\ } and
\texttt{\ show\ } rules.

An entry in a footnote list.

This function is not intended to be called directly. Instead, it is used
in set and show rules to customize footnote listings.

footnote { . } { entry } (

{ \href{/docs/reference/foundations/content/}{content} , } {
\hyperref[definitions-entry-parameters-separator]{separator :}
\href{/docs/reference/foundations/content/}{content} , } {
\hyperref[definitions-entry-parameters-clearance]{clearance :}
\href{/docs/reference/layout/length/}{length} , } {
\hyperref[definitions-entry-parameters-gap]{gap :}
\href{/docs/reference/layout/length/}{length} , } {
\hyperref[definitions-entry-parameters-indent]{indent :}
\href{/docs/reference/layout/length/}{length} , }

) -\textgreater{} \href{/docs/reference/foundations/content/}{content}

\begin{verbatim}
#show footnote.entry: set text(red)

My footnote listing
#footnote[It's down here]
has red text!
\end{verbatim}

\includegraphics[width=5in,height=\textheight,keepaspectratio]{/assets/docs/OQcOLIwIWFG81ucXxeuiVwAAAAAAAAAA.png}

\emph{Note:} Footnote entry properties must be uniform across each page
run (a page run is a sequence of pages without an explicit pagebreak in
between). For this reason, set and show rules for footnote entries
should be defined before any page content, typically at the very start
of the document.

\paragraph{\texorpdfstring{\texttt{\ note\ }}{ note }}\label{definitions-entry-note}

\href{/docs/reference/foundations/content/}{content}

{Required} {{ Positional }}

\phantomsection\label{definitions-entry-note-positional-tooltip}
Positional parameters are specified in order, without names.

The footnote for this entry. It\textquotesingle s location can be used
to determine the footnote counter state.

\includesvg[width=0.16667in,height=0.16667in]{/assets/icons/16-arrow-right.svg}
View example

\begin{verbatim}
#show footnote.entry: it => {
  let loc = it.note.location()
  numbering(
    "1: ",
    ..counter(footnote).at(loc),
  )
  it.note.body
}

Customized #footnote[Hello]
listing #footnote[World! 🌏]
\end{verbatim}

\includegraphics[width=5in,height=\textheight,keepaspectratio]{/assets/docs/pITXewKM6sSB5ed44fUp7wAAAAAAAAAA.png}

\paragraph{\texorpdfstring{\texttt{\ separator\ }}{ separator }}\label{definitions-entry-separator}

\href{/docs/reference/foundations/content/}{content}

{{ Settable }}

\phantomsection\label{definitions-entry-separator-settable-tooltip}
Settable parameters can be customized for all following uses of the
function with a \texttt{\ set\ } rule.

The separator between the document body and the footnote listing.

Default:
\texttt{\ }{\texttt{\ line\ }}\texttt{\ }{\texttt{\ (\ }}\texttt{\ length\ }{\texttt{\ :\ }}\texttt{\ }{\texttt{\ 30\%\ }}\texttt{\ }{\texttt{\ +\ }}\texttt{\ }{\texttt{\ 0pt\ }}\texttt{\ }{\texttt{\ ,\ }}\texttt{\ stroke\ }{\texttt{\ :\ }}\texttt{\ }{\texttt{\ 0.5pt\ }}\texttt{\ }{\texttt{\ )\ }}\texttt{\ }

\includesvg[width=0.16667in,height=0.16667in]{/assets/icons/16-arrow-right.svg}
View example

\begin{verbatim}
#set footnote.entry(
  separator: repeat[.]
)

Testing a different separator.
#footnote[
  Unconventional, but maybe
  not that bad?
]
\end{verbatim}

\includegraphics[width=5in,height=\textheight,keepaspectratio]{/assets/docs/2BZbfiOf16u6fje-JM2KhwAAAAAAAAAA.png}

\paragraph{\texorpdfstring{\texttt{\ clearance\ }}{ clearance }}\label{definitions-entry-clearance}

\href{/docs/reference/layout/length/}{length}

{{ Settable }}

\phantomsection\label{definitions-entry-clearance-settable-tooltip}
Settable parameters can be customized for all following uses of the
function with a \texttt{\ set\ } rule.

The amount of clearance between the document body and the separator.

Default: \texttt{\ }{\texttt{\ 1em\ }}\texttt{\ }

\includesvg[width=0.16667in,height=0.16667in]{/assets/icons/16-arrow-right.svg}
View example

\begin{verbatim}
#set footnote.entry(clearance: 3em)

Footnotes also need ...
#footnote[
  ... some space to breathe.
]
\end{verbatim}

\includegraphics[width=5in,height=\textheight,keepaspectratio]{/assets/docs/jGI_-Yxsz0NqX0MjmZS_qQAAAAAAAAAA.png}

\paragraph{\texorpdfstring{\texttt{\ gap\ }}{ gap }}\label{definitions-entry-gap}

\href{/docs/reference/layout/length/}{length}

{{ Settable }}

\phantomsection\label{definitions-entry-gap-settable-tooltip}
Settable parameters can be customized for all following uses of the
function with a \texttt{\ set\ } rule.

The gap between footnote entries.

Default: \texttt{\ }{\texttt{\ 0.5em\ }}\texttt{\ }

\includesvg[width=0.16667in,height=0.16667in]{/assets/icons/16-arrow-right.svg}
View example

\begin{verbatim}
#set footnote.entry(gap: 0.8em)

Footnotes:
#footnote[Spaced],
#footnote[Apart]
\end{verbatim}

\includegraphics[width=5in,height=\textheight,keepaspectratio]{/assets/docs/3sggupXU7L_bO6KYRBDWHQAAAAAAAAAA.png}

\paragraph{\texorpdfstring{\texttt{\ indent\ }}{ indent }}\label{definitions-entry-indent}

\href{/docs/reference/layout/length/}{length}

{{ Settable }}

\phantomsection\label{definitions-entry-indent-settable-tooltip}
Settable parameters can be customized for all following uses of the
function with a \texttt{\ set\ } rule.

The indent of each footnote entry.

Default: \texttt{\ }{\texttt{\ 1em\ }}\texttt{\ }

\includesvg[width=0.16667in,height=0.16667in]{/assets/icons/16-arrow-right.svg}
View example

\begin{verbatim}
#set footnote.entry(indent: 0em)

Footnotes:
#footnote[No],
#footnote[Indent]
\end{verbatim}

\includegraphics[width=5in,height=\textheight,keepaspectratio]{/assets/docs/-zkE_ejQDpF6KTPTlZZ3gwAAAAAAAAAA.png}

\href{/docs/reference/model/figure/}{\pandocbounded{\includesvg[keepaspectratio]{/assets/icons/16-arrow-right.svg}}}

{ Figure } { Previous page }

\href{/docs/reference/model/heading/}{\pandocbounded{\includesvg[keepaspectratio]{/assets/icons/16-arrow-right.svg}}}

{ Heading } { Next page }
