\title{typst.app/docs/reference/model/emph}

\begin{itemize}
\tightlist
\item
  \href{/docs}{\includesvg[width=0.16667in,height=0.16667in]{/assets/icons/16-docs-dark.svg}}
\item
  \includesvg[width=0.16667in,height=0.16667in]{/assets/icons/16-arrow-right.svg}
\item
  \href{/docs/reference/}{Reference}
\item
  \includesvg[width=0.16667in,height=0.16667in]{/assets/icons/16-arrow-right.svg}
\item
  \href{/docs/reference/model/}{Model}
\item
  \includesvg[width=0.16667in,height=0.16667in]{/assets/icons/16-arrow-right.svg}
\item
  \href{/docs/reference/model/emph/}{Emphasis}
\end{itemize}

\section{\texorpdfstring{\texttt{\ emph\ } {{ Element
}}}{ emph   Element }}\label{summary}

\phantomsection\label{element-tooltip}
Element functions can be customized with \texttt{\ set\ } and
\texttt{\ show\ } rules.

Emphasizes content by toggling italics.

\begin{itemize}
\tightlist
\item
  If the current
  \href{/docs/reference/text/text/\#parameters-style}{text style} is
  \texttt{\ }{\texttt{\ "normal"\ }}\texttt{\ } , this turns it into
  \texttt{\ }{\texttt{\ "italic"\ }}\texttt{\ } .
\item
  If it is already \texttt{\ }{\texttt{\ "italic"\ }}\texttt{\ } or
  \texttt{\ }{\texttt{\ "oblique"\ }}\texttt{\ } , it turns it back to
  \texttt{\ }{\texttt{\ "normal"\ }}\texttt{\ } .
\end{itemize}

\subsection{Example}\label{example}

\begin{verbatim}
This is _emphasized._ \
This is #emph[too.]

#show emph: it => {
  text(blue, it.body)
}

This is _emphasized_ differently.
\end{verbatim}

\includegraphics[width=5in,height=\textheight,keepaspectratio]{/assets/docs/p3cGCgaJdrkrScOita7QfgAAAAAAAAAA.png}

\subsection{Syntax}\label{syntax}

This function also has dedicated syntax: To emphasize content, simply
enclose it in underscores ( \texttt{\ \_\ } ). Note that this only works
at word boundaries. To emphasize part of a word, you have to use the
function.

\subsection{\texorpdfstring{{ Parameters
}}{ Parameters }}\label{parameters}

\phantomsection\label{parameters-tooltip}
Parameters are the inputs to a function. They are specified in
parentheses after the function name.

{ emph } (

{ \href{/docs/reference/foundations/content/}{content} }

) -\textgreater{} \href{/docs/reference/foundations/content/}{content}

\subsubsection{\texorpdfstring{\texttt{\ body\ }}{ body }}\label{parameters-body}

\href{/docs/reference/foundations/content/}{content}

{Required} {{ Positional }}

\phantomsection\label{parameters-body-positional-tooltip}
Positional parameters are specified in order, without names.

The content to emphasize.

\href{/docs/reference/model/document/}{\pandocbounded{\includesvg[keepaspectratio]{/assets/icons/16-arrow-right.svg}}}

{ Document } { Previous page }

\href{/docs/reference/model/figure/}{\pandocbounded{\includesvg[keepaspectratio]{/assets/icons/16-arrow-right.svg}}}

{ Figure } { Next page }
