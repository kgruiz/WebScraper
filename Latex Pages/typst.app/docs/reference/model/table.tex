\title{typst.app/docs/reference/model/table}

\begin{itemize}
\tightlist
\item
  \href{/docs}{\includesvg[width=0.16667in,height=0.16667in]{/assets/icons/16-docs-dark.svg}}
\item
  \includesvg[width=0.16667in,height=0.16667in]{/assets/icons/16-arrow-right.svg}
\item
  \href{/docs/reference/}{Reference}
\item
  \includesvg[width=0.16667in,height=0.16667in]{/assets/icons/16-arrow-right.svg}
\item
  \href{/docs/reference/model/}{Model}
\item
  \includesvg[width=0.16667in,height=0.16667in]{/assets/icons/16-arrow-right.svg}
\item
  \href{/docs/reference/model/table/}{Table}
\end{itemize}

\section{\texorpdfstring{\texttt{\ table\ } {{ Element
}}}{ table   Element }}\label{summary}

\phantomsection\label{element-tooltip}
Element functions can be customized with \texttt{\ set\ } and
\texttt{\ show\ } rules.

A table of items.

Tables are used to arrange content in cells. Cells can contain arbitrary
content, including multiple paragraphs and are specified in row-major
order. For a hands-on explanation of all the ways you can use and
customize tables in Typst, check out the
\href{/docs/guides/table-guide/}{table guide} .

Because tables are just grids with different defaults for some cell
properties (notably \texttt{\ stroke\ } and \texttt{\ inset\ } ), refer
to the \href{/docs/reference/layout/grid/}{grid documentation} for more
information on how to size the table tracks and specify the cell
appearance properties.

If you are unsure whether you should be using a table or a grid,
consider whether the content you are arranging semantically belongs
together as a set of related data points or similar or whether you are
just want to enhance your presentation by arranging unrelated content in
a grid. In the former case, a table is the right choice, while in the
latter case, a grid is more appropriate. Furthermore, Typst will
annotate its output in the future such that screenreaders will announce
content in \texttt{\ table\ } as tabular while a grid\textquotesingle s
content will be announced no different than multiple content blocks in
the document flow.

Note that, to override a particular cell\textquotesingle s properties or
apply show rules on table cells, you can use the
\href{/docs/reference/model/table/\#definitions-cell}{\texttt{\ table.cell\ }}
element. See its documentation for more information.

Although the \texttt{\ table\ } and the \texttt{\ grid\ } share most
properties, set and show rules on one of them do not affect the other.

To give a table a caption and make it
\href{/docs/reference/model/ref/}{referenceable} , put it into a
\href{/docs/reference/model/figure/}{figure} .

\subsection{Example}\label{example}

The example below demonstrates some of the most common table options.

\begin{verbatim}
#table(
  columns: (1fr, auto, auto),
  inset: 10pt,
  align: horizon,
  table.header(
    [], [*Volume*], [*Parameters*],
  ),
  image("cylinder.svg"),
  $ pi h (D^2 - d^2) / 4 $,
  [
    $h$: height \
    $D$: outer radius \
    $d$: inner radius
  ],
  image("tetrahedron.svg"),
  $ sqrt(2) / 12 a^3 $,
  [$a$: edge length]
)
\end{verbatim}

\includegraphics[width=5in,height=\textheight,keepaspectratio]{/assets/docs/KSzjBsOqtudzwvK6Zvp9uwAAAAAAAAAA.png}

Much like with grids, you can use
\href{/docs/reference/model/table/\#definitions-cell}{\texttt{\ table.cell\ }}
to customize the appearance and the position of each cell.

\begin{verbatim}
#set table(
  stroke: none,
  gutter: 0.2em,
  fill: (x, y) =>
    if x == 0 or y == 0 { gray },
  inset: (right: 1.5em),
)

#show table.cell: it => {
  if it.x == 0 or it.y == 0 {
    set text(white)
    strong(it)
  } else if it.body == [] {
    // Replace empty cells with 'N/A'
    pad(..it.inset)[_N/A_]
  } else {
    it
  }
}

#let a = table.cell(
  fill: green.lighten(60%),
)[A]
#let b = table.cell(
  fill: aqua.lighten(60%),
)[B]

#table(
  columns: 4,
  [], [Exam 1], [Exam 2], [Exam 3],

  [John], [], a, [],
  [Mary], [], a, a,
  [Robert], b, a, b,
)
\end{verbatim}

\includegraphics[width=5.66667in,height=\textheight,keepaspectratio]{/assets/docs/D_wYQ9Nqm8ZPq6ssgJwiZQAAAAAAAAAA.png}

\subsection{\texorpdfstring{{ Parameters
}}{ Parameters }}\label{parameters}

\phantomsection\label{parameters-tooltip}
Parameters are the inputs to a function. They are specified in
parentheses after the function name.

{ table } (

{ \hyperref[parameters-columns]{columns :}
\href{/docs/reference/foundations/auto/}{auto}
\href{/docs/reference/foundations/int/}{int}
\href{/docs/reference/layout/relative/}{relative}
\href{/docs/reference/layout/fraction/}{fraction}
\href{/docs/reference/foundations/array/}{array} , } {
\hyperref[parameters-rows]{rows :}
\href{/docs/reference/foundations/auto/}{auto}
\href{/docs/reference/foundations/int/}{int}
\href{/docs/reference/layout/relative/}{relative}
\href{/docs/reference/layout/fraction/}{fraction}
\href{/docs/reference/foundations/array/}{array} , } {
\hyperref[parameters-gutter]{gutter :}
\href{/docs/reference/foundations/auto/}{auto}
\href{/docs/reference/foundations/int/}{int}
\href{/docs/reference/layout/relative/}{relative}
\href{/docs/reference/layout/fraction/}{fraction}
\href{/docs/reference/foundations/array/}{array} , } {
\hyperref[parameters-column-gutter]{column-gutter :}
\href{/docs/reference/foundations/auto/}{auto}
\href{/docs/reference/foundations/int/}{int}
\href{/docs/reference/layout/relative/}{relative}
\href{/docs/reference/layout/fraction/}{fraction}
\href{/docs/reference/foundations/array/}{array} , } {
\hyperref[parameters-row-gutter]{row-gutter :}
\href{/docs/reference/foundations/auto/}{auto}
\href{/docs/reference/foundations/int/}{int}
\href{/docs/reference/layout/relative/}{relative}
\href{/docs/reference/layout/fraction/}{fraction}
\href{/docs/reference/foundations/array/}{array} , } {
\hyperref[parameters-fill]{fill :}
\href{/docs/reference/foundations/none/}{none}
\href{/docs/reference/visualize/color/}{color}
\href{/docs/reference/visualize/gradient/}{gradient}
\href{/docs/reference/foundations/array/}{array}
\href{/docs/reference/visualize/pattern/}{pattern}
\href{/docs/reference/foundations/function/}{function} , } {
\hyperref[parameters-align]{align :}
\href{/docs/reference/foundations/auto/}{auto}
\href{/docs/reference/foundations/array/}{array}
\href{/docs/reference/layout/alignment/}{alignment}
\href{/docs/reference/foundations/function/}{function} , } {
\hyperref[parameters-stroke]{stroke :}
\href{/docs/reference/foundations/none/}{none}
\href{/docs/reference/layout/length/}{length}
\href{/docs/reference/visualize/color/}{color}
\href{/docs/reference/visualize/gradient/}{gradient}
\href{/docs/reference/foundations/array/}{array}
\href{/docs/reference/visualize/stroke/}{stroke}
\href{/docs/reference/visualize/pattern/}{pattern}
\href{/docs/reference/foundations/dictionary/}{dictionary}
\href{/docs/reference/foundations/function/}{function} , } {
\hyperref[parameters-inset]{inset :}
\href{/docs/reference/layout/relative/}{relative}
\href{/docs/reference/foundations/array/}{array}
\href{/docs/reference/foundations/dictionary/}{dictionary}
\href{/docs/reference/foundations/function/}{function} , } {
\hyperref[parameters-children]{..}
\href{/docs/reference/foundations/content/}{content} , }

) -\textgreater{} \href{/docs/reference/foundations/content/}{content}

\subsubsection{\texorpdfstring{\texttt{\ columns\ }}{ columns }}\label{parameters-columns}

\href{/docs/reference/foundations/auto/}{auto} {or}
\href{/docs/reference/foundations/int/}{int} {or}
\href{/docs/reference/layout/relative/}{relative} {or}
\href{/docs/reference/layout/fraction/}{fraction} {or}
\href{/docs/reference/foundations/array/}{array}

{{ Settable }}

\phantomsection\label{parameters-columns-settable-tooltip}
Settable parameters can be customized for all following uses of the
function with a \texttt{\ set\ } rule.

The column sizes. See the \href{/docs/reference/layout/grid/}{grid
documentation} for more information on track sizing.

Default:
\texttt{\ }{\texttt{\ (\ }}\texttt{\ }{\texttt{\ )\ }}\texttt{\ }

\subsubsection{\texorpdfstring{\texttt{\ rows\ }}{ rows }}\label{parameters-rows}

\href{/docs/reference/foundations/auto/}{auto} {or}
\href{/docs/reference/foundations/int/}{int} {or}
\href{/docs/reference/layout/relative/}{relative} {or}
\href{/docs/reference/layout/fraction/}{fraction} {or}
\href{/docs/reference/foundations/array/}{array}

{{ Settable }}

\phantomsection\label{parameters-rows-settable-tooltip}
Settable parameters can be customized for all following uses of the
function with a \texttt{\ set\ } rule.

The row sizes. See the \href{/docs/reference/layout/grid/}{grid
documentation} for more information on track sizing.

Default:
\texttt{\ }{\texttt{\ (\ }}\texttt{\ }{\texttt{\ )\ }}\texttt{\ }

\subsubsection{\texorpdfstring{\texttt{\ gutter\ }}{ gutter }}\label{parameters-gutter}

\href{/docs/reference/foundations/auto/}{auto} {or}
\href{/docs/reference/foundations/int/}{int} {or}
\href{/docs/reference/layout/relative/}{relative} {or}
\href{/docs/reference/layout/fraction/}{fraction} {or}
\href{/docs/reference/foundations/array/}{array}

{{ Settable }}

\phantomsection\label{parameters-gutter-settable-tooltip}
Settable parameters can be customized for all following uses of the
function with a \texttt{\ set\ } rule.

The gaps between rows and columns. This is a shorthand for setting
\texttt{\ column-gutter\ } and \texttt{\ row-gutter\ } to the same
value. See the \href{/docs/reference/layout/grid/}{grid documentation}
for more information on gutters.

Default:
\texttt{\ }{\texttt{\ (\ }}\texttt{\ }{\texttt{\ )\ }}\texttt{\ }

\subsubsection{\texorpdfstring{\texttt{\ column-gutter\ }}{ column-gutter }}\label{parameters-column-gutter}

\href{/docs/reference/foundations/auto/}{auto} {or}
\href{/docs/reference/foundations/int/}{int} {or}
\href{/docs/reference/layout/relative/}{relative} {or}
\href{/docs/reference/layout/fraction/}{fraction} {or}
\href{/docs/reference/foundations/array/}{array}

{{ Settable }}

\phantomsection\label{parameters-column-gutter-settable-tooltip}
Settable parameters can be customized for all following uses of the
function with a \texttt{\ set\ } rule.

The gaps between columns. Takes precedence over \texttt{\ gutter\ } .
See the \href{/docs/reference/layout/grid/}{grid documentation} for more
information on gutters.

Default:
\texttt{\ }{\texttt{\ (\ }}\texttt{\ }{\texttt{\ )\ }}\texttt{\ }

\subsubsection{\texorpdfstring{\texttt{\ row-gutter\ }}{ row-gutter }}\label{parameters-row-gutter}

\href{/docs/reference/foundations/auto/}{auto} {or}
\href{/docs/reference/foundations/int/}{int} {or}
\href{/docs/reference/layout/relative/}{relative} {or}
\href{/docs/reference/layout/fraction/}{fraction} {or}
\href{/docs/reference/foundations/array/}{array}

{{ Settable }}

\phantomsection\label{parameters-row-gutter-settable-tooltip}
Settable parameters can be customized for all following uses of the
function with a \texttt{\ set\ } rule.

The gaps between rows. Takes precedence over \texttt{\ gutter\ } . See
the \href{/docs/reference/layout/grid/}{grid documentation} for more
information on gutters.

Default:
\texttt{\ }{\texttt{\ (\ }}\texttt{\ }{\texttt{\ )\ }}\texttt{\ }

\subsubsection{\texorpdfstring{\texttt{\ fill\ }}{ fill }}\label{parameters-fill}

\href{/docs/reference/foundations/none/}{none} {or}
\href{/docs/reference/visualize/color/}{color} {or}
\href{/docs/reference/visualize/gradient/}{gradient} {or}
\href{/docs/reference/foundations/array/}{array} {or}
\href{/docs/reference/visualize/pattern/}{pattern} {or}
\href{/docs/reference/foundations/function/}{function}

{{ Settable }}

\phantomsection\label{parameters-fill-settable-tooltip}
Settable parameters can be customized for all following uses of the
function with a \texttt{\ set\ } rule.

How to fill the cells.

This can be a color or a function that returns a color. The function
receives the cells\textquotesingle{} column and row indices, starting
from zero. This can be used to implement striped tables.

Default: \texttt{\ }{\texttt{\ none\ }}\texttt{\ }

\includesvg[width=0.16667in,height=0.16667in]{/assets/icons/16-arrow-right.svg}
View example

\begin{verbatim}
#table(
  fill: (x, _) =>
    if calc.odd(x) { luma(240) }
    else { white },
  align: (x, y) =>
    if y == 0 { center }
    else if x == 0 { left }
    else { right },
  columns: 4,
  [], [*Q1*], [*Q2*], [*Q3*],
  [Revenue:], [1000 €], [2000 €], [3000 €],
  [Expenses:], [500 €], [1000 €], [1500 €],
  [Profit:], [500 €], [1000 €], [1500 €],
)
\end{verbatim}

\includegraphics[width=5in,height=\textheight,keepaspectratio]{/assets/docs/HObhPJHvYkiYqHCjRK1JHwAAAAAAAAAA.png}

\subsubsection{\texorpdfstring{\texttt{\ align\ }}{ align }}\label{parameters-align}

\href{/docs/reference/foundations/auto/}{auto} {or}
\href{/docs/reference/foundations/array/}{array} {or}
\href{/docs/reference/layout/alignment/}{alignment} {or}
\href{/docs/reference/foundations/function/}{function}

{{ Settable }}

\phantomsection\label{parameters-align-settable-tooltip}
Settable parameters can be customized for all following uses of the
function with a \texttt{\ set\ } rule.

How to align the cells\textquotesingle{} content.

This can either be a single alignment, an array of alignments
(corresponding to each column) or a function that returns an alignment.
The function receives the cells\textquotesingle{} column and row
indices, starting from zero. If set to
\texttt{\ }{\texttt{\ auto\ }}\texttt{\ } , the outer alignment is used.

Default: \texttt{\ }{\texttt{\ auto\ }}\texttt{\ }

\includesvg[width=0.16667in,height=0.16667in]{/assets/icons/16-arrow-right.svg}
View example

\begin{verbatim}
#table(
  columns: 3,
  align: (left, center, right),
  [Hello], [Hello], [Hello],
  [A], [B], [C],
)
\end{verbatim}

\includegraphics[width=5in,height=\textheight,keepaspectratio]{/assets/docs/_fBgotCl-LtVjvGU4yJFLQAAAAAAAAAA.png}

\subsubsection{\texorpdfstring{\texttt{\ stroke\ }}{ stroke }}\label{parameters-stroke}

\href{/docs/reference/foundations/none/}{none} {or}
\href{/docs/reference/layout/length/}{length} {or}
\href{/docs/reference/visualize/color/}{color} {or}
\href{/docs/reference/visualize/gradient/}{gradient} {or}
\href{/docs/reference/foundations/array/}{array} {or}
\href{/docs/reference/visualize/stroke/}{stroke} {or}
\href{/docs/reference/visualize/pattern/}{pattern} {or}
\href{/docs/reference/foundations/dictionary/}{dictionary} {or}
\href{/docs/reference/foundations/function/}{function}

{{ Settable }}

\phantomsection\label{parameters-stroke-settable-tooltip}
Settable parameters can be customized for all following uses of the
function with a \texttt{\ set\ } rule.

How to \href{/docs/reference/visualize/stroke/}{stroke} the cells.

Strokes can be disabled by setting this to
\texttt{\ }{\texttt{\ none\ }}\texttt{\ } .

If it is necessary to place lines which can cross spacing between cells
produced by the \texttt{\ gutter\ } option, or to override the stroke
between multiple specific cells, consider specifying one or more of
\href{/docs/reference/model/table/\#definitions-hline}{\texttt{\ table.hline\ }}
and
\href{/docs/reference/model/table/\#definitions-vline}{\texttt{\ table.vline\ }}
alongside your table cells.

See the \href{/docs/reference/layout/grid/\#parameters-stroke}{grid
documentation} for more information on strokes.

Default:
\texttt{\ }{\texttt{\ 1pt\ }}\texttt{\ }{\texttt{\ +\ }}\texttt{\ black\ }

\subsubsection{\texorpdfstring{\texttt{\ inset\ }}{ inset }}\label{parameters-inset}

\href{/docs/reference/layout/relative/}{relative} {or}
\href{/docs/reference/foundations/array/}{array} {or}
\href{/docs/reference/foundations/dictionary/}{dictionary} {or}
\href{/docs/reference/foundations/function/}{function}

{{ Settable }}

\phantomsection\label{parameters-inset-settable-tooltip}
Settable parameters can be customized for all following uses of the
function with a \texttt{\ set\ } rule.

How much to pad the cells\textquotesingle{} content.

Default:
\texttt{\ }{\texttt{\ 0\%\ }}\texttt{\ }{\texttt{\ +\ }}\texttt{\ }{\texttt{\ 5pt\ }}\texttt{\ }

\includesvg[width=0.16667in,height=0.16667in]{/assets/icons/16-arrow-right.svg}
View example

\begin{verbatim}
#table(
  inset: 10pt,
  [Hello],
  [World],
)

#table(
  columns: 2,
  inset: (
    x: 20pt,
    y: 10pt,
  ),
  [Hello],
  [World],
)
\end{verbatim}

\includegraphics[width=5in,height=\textheight,keepaspectratio]{/assets/docs/f1kE1ENTTB02iZKKPoV_XwAAAAAAAAAA.png}

\subsubsection{\texorpdfstring{\texttt{\ children\ }}{ children }}\label{parameters-children}

\href{/docs/reference/foundations/content/}{content}

{Required} {{ Positional }}

\phantomsection\label{parameters-children-positional-tooltip}
Positional parameters are specified in order, without names.

{{ Variadic }}

\phantomsection\label{parameters-children-variadic-tooltip}
Variadic parameters can be specified multiple times.

The contents of the table cells, plus any extra table lines specified
with the
\href{/docs/reference/model/table/\#definitions-hline}{\texttt{\ table.hline\ }}
and
\href{/docs/reference/model/table/\#definitions-vline}{\texttt{\ table.vline\ }}
elements.

\subsection{\texorpdfstring{{ Definitions
}}{ Definitions }}\label{definitions}

\phantomsection\label{definitions-tooltip}
Functions and types and can have associated definitions. These are
accessed by specifying the function or type, followed by a period, and
then the definition\textquotesingle s name.

\subsubsection{\texorpdfstring{\texttt{\ cell\ } {{ Element
}}}{ cell   Element }}\label{definitions-cell}

\phantomsection\label{definitions-cell-element-tooltip}
Element functions can be customized with \texttt{\ set\ } and
\texttt{\ show\ } rules.

A cell in the table. Use this to position a cell manually or to apply
styling. To do the latter, you can either use the function to override
the properties for a particular cell, or use it in show rules to apply
certain styles to multiple cells at once.

Perhaps the most important use case of
\texttt{\ table\ }{\texttt{\ .\ }}\texttt{\ cell\ } is to make a cell
span multiple columns and/or rows with the \texttt{\ colspan\ } and
\texttt{\ rowspan\ } fields.

table { . } { cell } (

{ \href{/docs/reference/foundations/content/}{content} , } {
\hyperref[definitions-cell-parameters-x]{x :}
\href{/docs/reference/foundations/auto/}{auto}
\href{/docs/reference/foundations/int/}{int} , } {
\hyperref[definitions-cell-parameters-y]{y :}
\href{/docs/reference/foundations/auto/}{auto}
\href{/docs/reference/foundations/int/}{int} , } {
\hyperref[definitions-cell-parameters-colspan]{colspan :}
\href{/docs/reference/foundations/int/}{int} , } {
\hyperref[definitions-cell-parameters-rowspan]{rowspan :}
\href{/docs/reference/foundations/int/}{int} , } {
\hyperref[definitions-cell-parameters-fill]{fill :}
\href{/docs/reference/foundations/none/}{none}
\href{/docs/reference/foundations/auto/}{auto}
\href{/docs/reference/visualize/color/}{color}
\href{/docs/reference/visualize/gradient/}{gradient}
\href{/docs/reference/visualize/pattern/}{pattern} , } {
\hyperref[definitions-cell-parameters-align]{align :}
\href{/docs/reference/foundations/auto/}{auto}
\href{/docs/reference/layout/alignment/}{alignment} , } {
\hyperref[definitions-cell-parameters-inset]{inset :}
\href{/docs/reference/foundations/auto/}{auto}
\href{/docs/reference/layout/relative/}{relative}
\href{/docs/reference/foundations/dictionary/}{dictionary} , } {
\hyperref[definitions-cell-parameters-stroke]{stroke :}
\href{/docs/reference/foundations/none/}{none}
\href{/docs/reference/layout/length/}{length}
\href{/docs/reference/visualize/color/}{color}
\href{/docs/reference/visualize/gradient/}{gradient}
\href{/docs/reference/visualize/stroke/}{stroke}
\href{/docs/reference/visualize/pattern/}{pattern}
\href{/docs/reference/foundations/dictionary/}{dictionary} , } {
\hyperref[definitions-cell-parameters-breakable]{breakable :}
\href{/docs/reference/foundations/auto/}{auto}
\href{/docs/reference/foundations/bool/}{bool} , }

) -\textgreater{} \href{/docs/reference/foundations/content/}{content}

\begin{verbatim}
#show table.cell.where(y: 0): strong
#set table(
  stroke: (x, y) => if y == 0 {
    (bottom: 0.7pt + black)
  },
  align: (x, y) => (
    if x > 0 { center }
    else { left }
  )
)

#table(
  columns: 3,
  table.header(
    [Substance],
    [Subcritical °C],
    [Supercritical °C],
  ),
  [Hydrochloric Acid],
  [12.0], [92.1],
  [Sodium Myreth Sulfate],
  [16.6], [104],
  [Potassium Hydroxide],
  table.cell(colspan: 2)[24.7],
)
\end{verbatim}

\includegraphics[width=6.39583in,height=\textheight,keepaspectratio]{/assets/docs/2rQPm8gbRwbFqiITJlD6oAAAAAAAAAAA.png}

For example, you can override the fill, alignment or inset for a single
cell:

\begin{verbatim}
// You can also import those.
#import table: cell, header

#table(
  columns: 2,
  align: center,
  header(
    [*Trip progress*],
    [*Itinerary*],
  ),
  cell(
    align: right,
    fill: fuchsia.lighten(80%),
    [🚗],
  ),
  [Get in, folks!],
  [🚗], [Eat curbside hotdog],
  cell(align: left)[🌴🚗],
  cell(
    inset: 0.06em,
    text(1.62em)[🛖🌅🌊],
  ),
)
\end{verbatim}

\includegraphics[width=4.29167in,height=\textheight,keepaspectratio]{/assets/docs/VtayZlhMrUWzOmBAyEorDQAAAAAAAAAA.png}

You may also apply a show rule on \texttt{\ table.cell\ } to style all
cells at once. Combined with selectors, this allows you to apply styles
based on a cell\textquotesingle s position:

\begin{verbatim}
#show table.cell.where(x: 0): strong

#table(
  columns: 3,
  gutter: 3pt,
  [Name], [Age], [Strength],
  [Hannes], [36], [Grace],
  [Irma], [50], [Resourcefulness],
  [Vikram], [49], [Perseverance],
)
\end{verbatim}

\includegraphics[width=5in,height=\textheight,keepaspectratio]{/assets/docs/c2SP069qvMBzeFbrjVs8pwAAAAAAAAAA.png}

\paragraph{\texorpdfstring{\texttt{\ body\ }}{ body }}\label{definitions-cell-body}

\href{/docs/reference/foundations/content/}{content}

{Required} {{ Positional }}

\phantomsection\label{definitions-cell-body-positional-tooltip}
Positional parameters are specified in order, without names.

The cell\textquotesingle s body.

\paragraph{\texorpdfstring{\texttt{\ x\ }}{ x }}\label{definitions-cell-x}

\href{/docs/reference/foundations/auto/}{auto} {or}
\href{/docs/reference/foundations/int/}{int}

{{ Settable }}

\phantomsection\label{definitions-cell-x-settable-tooltip}
Settable parameters can be customized for all following uses of the
function with a \texttt{\ set\ } rule.

The cell\textquotesingle s column (zero-indexed). Functions identically
to the \texttt{\ x\ } field in
\href{/docs/reference/layout/grid/\#definitions-cell}{\texttt{\ grid.cell\ }}
.

Default: \texttt{\ }{\texttt{\ auto\ }}\texttt{\ }

\paragraph{\texorpdfstring{\texttt{\ y\ }}{ y }}\label{definitions-cell-y}

\href{/docs/reference/foundations/auto/}{auto} {or}
\href{/docs/reference/foundations/int/}{int}

{{ Settable }}

\phantomsection\label{definitions-cell-y-settable-tooltip}
Settable parameters can be customized for all following uses of the
function with a \texttt{\ set\ } rule.

The cell\textquotesingle s row (zero-indexed). Functions identically to
the \texttt{\ y\ } field in
\href{/docs/reference/layout/grid/\#definitions-cell}{\texttt{\ grid.cell\ }}
.

Default: \texttt{\ }{\texttt{\ auto\ }}\texttt{\ }

\paragraph{\texorpdfstring{\texttt{\ colspan\ }}{ colspan }}\label{definitions-cell-colspan}

\href{/docs/reference/foundations/int/}{int}

{{ Settable }}

\phantomsection\label{definitions-cell-colspan-settable-tooltip}
Settable parameters can be customized for all following uses of the
function with a \texttt{\ set\ } rule.

The amount of columns spanned by this cell.

Default: \texttt{\ }{\texttt{\ 1\ }}\texttt{\ }

\paragraph{\texorpdfstring{\texttt{\ rowspan\ }}{ rowspan }}\label{definitions-cell-rowspan}

\href{/docs/reference/foundations/int/}{int}

{{ Settable }}

\phantomsection\label{definitions-cell-rowspan-settable-tooltip}
Settable parameters can be customized for all following uses of the
function with a \texttt{\ set\ } rule.

The amount of rows spanned by this cell.

Default: \texttt{\ }{\texttt{\ 1\ }}\texttt{\ }

\paragraph{\texorpdfstring{\texttt{\ fill\ }}{ fill }}\label{definitions-cell-fill}

\href{/docs/reference/foundations/none/}{none} {or}
\href{/docs/reference/foundations/auto/}{auto} {or}
\href{/docs/reference/visualize/color/}{color} {or}
\href{/docs/reference/visualize/gradient/}{gradient} {or}
\href{/docs/reference/visualize/pattern/}{pattern}

{{ Settable }}

\phantomsection\label{definitions-cell-fill-settable-tooltip}
Settable parameters can be customized for all following uses of the
function with a \texttt{\ set\ } rule.

The cell\textquotesingle s
\href{/docs/reference/model/table/\#parameters-fill}{fill} override.

Default: \texttt{\ }{\texttt{\ auto\ }}\texttt{\ }

\paragraph{\texorpdfstring{\texttt{\ align\ }}{ align }}\label{definitions-cell-align}

\href{/docs/reference/foundations/auto/}{auto} {or}
\href{/docs/reference/layout/alignment/}{alignment}

{{ Settable }}

\phantomsection\label{definitions-cell-align-settable-tooltip}
Settable parameters can be customized for all following uses of the
function with a \texttt{\ set\ } rule.

The cell\textquotesingle s
\href{/docs/reference/model/table/\#parameters-align}{alignment}
override.

Default: \texttt{\ }{\texttt{\ auto\ }}\texttt{\ }

\paragraph{\texorpdfstring{\texttt{\ inset\ }}{ inset }}\label{definitions-cell-inset}

\href{/docs/reference/foundations/auto/}{auto} {or}
\href{/docs/reference/layout/relative/}{relative} {or}
\href{/docs/reference/foundations/dictionary/}{dictionary}

{{ Settable }}

\phantomsection\label{definitions-cell-inset-settable-tooltip}
Settable parameters can be customized for all following uses of the
function with a \texttt{\ set\ } rule.

The cell\textquotesingle s
\href{/docs/reference/model/table/\#parameters-inset}{inset} override.

Default: \texttt{\ }{\texttt{\ auto\ }}\texttt{\ }

\paragraph{\texorpdfstring{\texttt{\ stroke\ }}{ stroke }}\label{definitions-cell-stroke}

\href{/docs/reference/foundations/none/}{none} {or}
\href{/docs/reference/layout/length/}{length} {or}
\href{/docs/reference/visualize/color/}{color} {or}
\href{/docs/reference/visualize/gradient/}{gradient} {or}
\href{/docs/reference/visualize/stroke/}{stroke} {or}
\href{/docs/reference/visualize/pattern/}{pattern} {or}
\href{/docs/reference/foundations/dictionary/}{dictionary}

{{ Settable }}

\phantomsection\label{definitions-cell-stroke-settable-tooltip}
Settable parameters can be customized for all following uses of the
function with a \texttt{\ set\ } rule.

The cell\textquotesingle s
\href{/docs/reference/model/table/\#parameters-stroke}{stroke} override.

Default:
\texttt{\ }{\texttt{\ (\ }}\texttt{\ }{\texttt{\ :\ }}\texttt{\ }{\texttt{\ )\ }}\texttt{\ }

\paragraph{\texorpdfstring{\texttt{\ breakable\ }}{ breakable }}\label{definitions-cell-breakable}

\href{/docs/reference/foundations/auto/}{auto} {or}
\href{/docs/reference/foundations/bool/}{bool}

{{ Settable }}

\phantomsection\label{definitions-cell-breakable-settable-tooltip}
Settable parameters can be customized for all following uses of the
function with a \texttt{\ set\ } rule.

Whether rows spanned by this cell can be placed in different pages. When
equal to \texttt{\ }{\texttt{\ auto\ }}\texttt{\ } , a cell spanning
only fixed-size rows is unbreakable, while a cell spanning at least one
\texttt{\ }{\texttt{\ auto\ }}\texttt{\ } -sized row is breakable.

Default: \texttt{\ }{\texttt{\ auto\ }}\texttt{\ }

\subsubsection{\texorpdfstring{\texttt{\ hline\ } {{ Element
}}}{ hline   Element }}\label{definitions-hline}

\phantomsection\label{definitions-hline-element-tooltip}
Element functions can be customized with \texttt{\ set\ } and
\texttt{\ show\ } rules.

A horizontal line in the table.

Overrides any per-cell stroke, including stroke specified through the
table\textquotesingle s \texttt{\ stroke\ } field. Can cross spacing
between cells created through the table\textquotesingle s
\href{/docs/reference/model/table/\#parameters-column-gutter}{\texttt{\ column-gutter\ }}
option.

Use this function instead of the table\textquotesingle s
\texttt{\ stroke\ } field if you want to manually place a horizontal
line at a specific position in a single table. Consider using
\href{/docs/reference/model/table/\#parameters-stroke}{table\textquotesingle s
\texttt{\ stroke\ }} field or
\href{/docs/reference/model/table/\#definitions-cell-stroke}{\texttt{\ table.cell\ }
\textquotesingle s \texttt{\ stroke\ }} field instead if the line you
want to place is part of all your tables\textquotesingle{} designs.

table { . } { hline } (

{ \hyperref[definitions-hline-parameters-y]{y :}
\href{/docs/reference/foundations/auto/}{auto}
\href{/docs/reference/foundations/int/}{int} , } {
\hyperref[definitions-hline-parameters-start]{start :}
\href{/docs/reference/foundations/int/}{int} , } {
\hyperref[definitions-hline-parameters-end]{end :}
\href{/docs/reference/foundations/none/}{none}
\href{/docs/reference/foundations/int/}{int} , } {
\hyperref[definitions-hline-parameters-stroke]{stroke :}
\href{/docs/reference/foundations/none/}{none}
\href{/docs/reference/layout/length/}{length}
\href{/docs/reference/visualize/color/}{color}
\href{/docs/reference/visualize/gradient/}{gradient}
\href{/docs/reference/visualize/stroke/}{stroke}
\href{/docs/reference/visualize/pattern/}{pattern}
\href{/docs/reference/foundations/dictionary/}{dictionary} , } {
\hyperref[definitions-hline-parameters-position]{position :}
\href{/docs/reference/layout/alignment/}{alignment} , }

) -\textgreater{} \href{/docs/reference/foundations/content/}{content}

\begin{verbatim}
#set table.hline(stroke: .6pt)

#table(
  stroke: none,
  columns: (auto, 1fr),
  [09:00], [Badge pick up],
  [09:45], [Opening Keynote],
  [10:30], [Talk: Typst's Future],
  [11:15], [Session: Good PRs],
  table.hline(start: 1),
  [Noon], [_Lunch break_],
  table.hline(start: 1),
  [14:00], [Talk: Tracked Layout],
  [15:00], [Talk: Automations],
  [16:00], [Workshop: Tables],
  table.hline(),
  [19:00], [Day 1 Attendee Mixer],
)
\end{verbatim}

\includegraphics[width=5in,height=\textheight,keepaspectratio]{/assets/docs/Fl-W72wh8hlKb72YjlJ0PgAAAAAAAAAA.png}

\paragraph{\texorpdfstring{\texttt{\ y\ }}{ y }}\label{definitions-hline-y}

\href{/docs/reference/foundations/auto/}{auto} {or}
\href{/docs/reference/foundations/int/}{int}

{{ Settable }}

\phantomsection\label{definitions-hline-y-settable-tooltip}
Settable parameters can be customized for all following uses of the
function with a \texttt{\ set\ } rule.

The row above which the horizontal line is placed (zero-indexed).
Functions identically to the \texttt{\ y\ } field in
\href{/docs/reference/layout/grid/\#definitions-hline-y}{\texttt{\ grid.hline\ }}
.

Default: \texttt{\ }{\texttt{\ auto\ }}\texttt{\ }

\paragraph{\texorpdfstring{\texttt{\ start\ }}{ start }}\label{definitions-hline-start}

\href{/docs/reference/foundations/int/}{int}

{{ Settable }}

\phantomsection\label{definitions-hline-start-settable-tooltip}
Settable parameters can be customized for all following uses of the
function with a \texttt{\ set\ } rule.

The column at which the horizontal line starts (zero-indexed,
inclusive).

Default: \texttt{\ }{\texttt{\ 0\ }}\texttt{\ }

\paragraph{\texorpdfstring{\texttt{\ end\ }}{ end }}\label{definitions-hline-end}

\href{/docs/reference/foundations/none/}{none} {or}
\href{/docs/reference/foundations/int/}{int}

{{ Settable }}

\phantomsection\label{definitions-hline-end-settable-tooltip}
Settable parameters can be customized for all following uses of the
function with a \texttt{\ set\ } rule.

The column before which the horizontal line ends (zero-indexed,
exclusive).

Default: \texttt{\ }{\texttt{\ none\ }}\texttt{\ }

\paragraph{\texorpdfstring{\texttt{\ stroke\ }}{ stroke }}\label{definitions-hline-stroke}

\href{/docs/reference/foundations/none/}{none} {or}
\href{/docs/reference/layout/length/}{length} {or}
\href{/docs/reference/visualize/color/}{color} {or}
\href{/docs/reference/visualize/gradient/}{gradient} {or}
\href{/docs/reference/visualize/stroke/}{stroke} {or}
\href{/docs/reference/visualize/pattern/}{pattern} {or}
\href{/docs/reference/foundations/dictionary/}{dictionary}

{{ Settable }}

\phantomsection\label{definitions-hline-stroke-settable-tooltip}
Settable parameters can be customized for all following uses of the
function with a \texttt{\ set\ } rule.

The line\textquotesingle s stroke.

Specifying \texttt{\ }{\texttt{\ none\ }}\texttt{\ } removes any lines
previously placed across this line\textquotesingle s range, including
hlines or per-cell stroke below it.

Default:
\texttt{\ }{\texttt{\ 1pt\ }}\texttt{\ }{\texttt{\ +\ }}\texttt{\ black\ }

\paragraph{\texorpdfstring{\texttt{\ position\ }}{ position }}\label{definitions-hline-position}

\href{/docs/reference/layout/alignment/}{alignment}

{{ Settable }}

\phantomsection\label{definitions-hline-position-settable-tooltip}
Settable parameters can be customized for all following uses of the
function with a \texttt{\ set\ } rule.

The position at which the line is placed, given its row ( \texttt{\ y\ }
) - either \texttt{\ top\ } to draw above it or \texttt{\ bottom\ } to
draw below it.

This setting is only relevant when row gutter is enabled (and
shouldn\textquotesingle t be used otherwise - prefer just increasing the
\texttt{\ y\ } field by one instead), since then the position below a
row becomes different from the position above the next row due to the
spacing between both.

Default: \texttt{\ top\ }

\subsubsection{\texorpdfstring{\texttt{\ vline\ } {{ Element
}}}{ vline   Element }}\label{definitions-vline}

\phantomsection\label{definitions-vline-element-tooltip}
Element functions can be customized with \texttt{\ set\ } and
\texttt{\ show\ } rules.

A vertical line in the table. See the docs for
\href{/docs/reference/layout/grid/\#definitions-vline}{\texttt{\ grid.vline\ }}
for more information regarding how to use this element\textquotesingle s
fields.

Overrides any per-cell stroke, including stroke specified through the
table\textquotesingle s \texttt{\ stroke\ } field. Can cross spacing
between cells created through the table\textquotesingle s
\href{/docs/reference/model/table/\#parameters-row-gutter}{\texttt{\ row-gutter\ }}
option.

Similar to
\href{/docs/reference/model/table/\#definitions-hline}{\texttt{\ table.hline\ }}
, use this function if you want to manually place a vertical line at a
specific position in a single table and use the
\href{/docs/reference/model/table/\#parameters-stroke}{table\textquotesingle s
\texttt{\ stroke\ }} field or
\href{/docs/reference/model/table/\#definitions-cell-stroke}{\texttt{\ table.cell\ }
\textquotesingle s \texttt{\ stroke\ }} field instead if the line you
want to place is part of all your tables\textquotesingle{} designs.

table { . } { vline } (

{ \hyperref[definitions-vline-parameters-x]{x :}
\href{/docs/reference/foundations/auto/}{auto}
\href{/docs/reference/foundations/int/}{int} , } {
\hyperref[definitions-vline-parameters-start]{start :}
\href{/docs/reference/foundations/int/}{int} , } {
\hyperref[definitions-vline-parameters-end]{end :}
\href{/docs/reference/foundations/none/}{none}
\href{/docs/reference/foundations/int/}{int} , } {
\hyperref[definitions-vline-parameters-stroke]{stroke :}
\href{/docs/reference/foundations/none/}{none}
\href{/docs/reference/layout/length/}{length}
\href{/docs/reference/visualize/color/}{color}
\href{/docs/reference/visualize/gradient/}{gradient}
\href{/docs/reference/visualize/stroke/}{stroke}
\href{/docs/reference/visualize/pattern/}{pattern}
\href{/docs/reference/foundations/dictionary/}{dictionary} , } {
\hyperref[definitions-vline-parameters-position]{position :}
\href{/docs/reference/layout/alignment/}{alignment} , }

) -\textgreater{} \href{/docs/reference/foundations/content/}{content}

\paragraph{\texorpdfstring{\texttt{\ x\ }}{ x }}\label{definitions-vline-x}

\href{/docs/reference/foundations/auto/}{auto} {or}
\href{/docs/reference/foundations/int/}{int}

{{ Settable }}

\phantomsection\label{definitions-vline-x-settable-tooltip}
Settable parameters can be customized for all following uses of the
function with a \texttt{\ set\ } rule.

The column before which the horizontal line is placed (zero-indexed).
Functions identically to the \texttt{\ x\ } field in
\href{/docs/reference/layout/grid/\#definitions-vline}{\texttt{\ grid.vline\ }}
.

Default: \texttt{\ }{\texttt{\ auto\ }}\texttt{\ }

\paragraph{\texorpdfstring{\texttt{\ start\ }}{ start }}\label{definitions-vline-start}

\href{/docs/reference/foundations/int/}{int}

{{ Settable }}

\phantomsection\label{definitions-vline-start-settable-tooltip}
Settable parameters can be customized for all following uses of the
function with a \texttt{\ set\ } rule.

The row at which the vertical line starts (zero-indexed, inclusive).

Default: \texttt{\ }{\texttt{\ 0\ }}\texttt{\ }

\paragraph{\texorpdfstring{\texttt{\ end\ }}{ end }}\label{definitions-vline-end}

\href{/docs/reference/foundations/none/}{none} {or}
\href{/docs/reference/foundations/int/}{int}

{{ Settable }}

\phantomsection\label{definitions-vline-end-settable-tooltip}
Settable parameters can be customized for all following uses of the
function with a \texttt{\ set\ } rule.

The row on top of which the vertical line ends (zero-indexed,
exclusive).

Default: \texttt{\ }{\texttt{\ none\ }}\texttt{\ }

\paragraph{\texorpdfstring{\texttt{\ stroke\ }}{ stroke }}\label{definitions-vline-stroke}

\href{/docs/reference/foundations/none/}{none} {or}
\href{/docs/reference/layout/length/}{length} {or}
\href{/docs/reference/visualize/color/}{color} {or}
\href{/docs/reference/visualize/gradient/}{gradient} {or}
\href{/docs/reference/visualize/stroke/}{stroke} {or}
\href{/docs/reference/visualize/pattern/}{pattern} {or}
\href{/docs/reference/foundations/dictionary/}{dictionary}

{{ Settable }}

\phantomsection\label{definitions-vline-stroke-settable-tooltip}
Settable parameters can be customized for all following uses of the
function with a \texttt{\ set\ } rule.

The line\textquotesingle s stroke.

Specifying \texttt{\ }{\texttt{\ none\ }}\texttt{\ } removes any lines
previously placed across this line\textquotesingle s range, including
vlines or per-cell stroke below it.

Default:
\texttt{\ }{\texttt{\ 1pt\ }}\texttt{\ }{\texttt{\ +\ }}\texttt{\ black\ }

\paragraph{\texorpdfstring{\texttt{\ position\ }}{ position }}\label{definitions-vline-position}

\href{/docs/reference/layout/alignment/}{alignment}

{{ Settable }}

\phantomsection\label{definitions-vline-position-settable-tooltip}
Settable parameters can be customized for all following uses of the
function with a \texttt{\ set\ } rule.

The position at which the line is placed, given its column (
\texttt{\ x\ } ) - either \texttt{\ start\ } to draw before it or
\texttt{\ end\ } to draw after it.

The values \texttt{\ left\ } and \texttt{\ right\ } are also accepted,
but discouraged as they cause your table to be inconsistent between
left-to-right and right-to-left documents.

This setting is only relevant when column gutter is enabled (and
shouldn\textquotesingle t be used otherwise - prefer just increasing the
\texttt{\ x\ } field by one instead), since then the position after a
column becomes different from the position before the next column due to
the spacing between both.

Default: \texttt{\ start\ }

\subsubsection{\texorpdfstring{\texttt{\ header\ } {{ Element
}}}{ header   Element }}\label{definitions-header}

\phantomsection\label{definitions-header-element-tooltip}
Element functions can be customized with \texttt{\ set\ } and
\texttt{\ show\ } rules.

A repeatable table header.

You should wrap your tables\textquotesingle{} heading rows in this
function even if you do not plan to wrap your table across pages because
Typst will use this function to attach accessibility metadata to tables
in the future and ensure universal access to your document.

You can use the \texttt{\ repeat\ } parameter to control whether your
table\textquotesingle s header will be repeated across pages.

table { . } { header } (

{ \hyperref[definitions-header-parameters-repeat]{repeat :}
\href{/docs/reference/foundations/bool/}{bool} , } {
\hyperref[definitions-header-parameters-children]{..}
\href{/docs/reference/foundations/content/}{content} , }

) -\textgreater{} \href{/docs/reference/foundations/content/}{content}

\begin{verbatim}
#set page(height: 11.5em)
#set table(
  fill: (x, y) =>
    if x == 0 or y == 0 {
      gray.lighten(40%)
    },
  align: right,
)

#show table.cell.where(x: 0): strong
#show table.cell.where(y: 0): strong

#table(
  columns: 4,
  table.header(
    [], [Blue chip],
    [Fresh IPO], [Penny st'k],
  ),
  table.cell(
    rowspan: 6,
    align: horizon,
    rotate(-90deg, reflow: true)[
      *USD / day*
    ],
  ),
  [0.20], [104], [5],
  [3.17], [108], [4],
  [1.59], [84],  [1],
  [0.26], [98],  [15],
  [0.01], [195], [4],
  [7.34], [57],  [2],
)
\end{verbatim}

\includegraphics[width=5in,height=\textheight,keepaspectratio]{/assets/docs/IHpzp-b7mQ7ctAllSxEWfQAAAAAAAAAA.png}
\includegraphics[width=5in,height=\textheight,keepaspectratio]{/assets/docs/IHpzp-b7mQ7ctAllSxEWfQAAAAAAAAAB.png}

\paragraph{\texorpdfstring{\texttt{\ repeat\ }}{ repeat }}\label{definitions-header-repeat}

\href{/docs/reference/foundations/bool/}{bool}

{{ Settable }}

\phantomsection\label{definitions-header-repeat-settable-tooltip}
Settable parameters can be customized for all following uses of the
function with a \texttt{\ set\ } rule.

Whether this header should be repeated across pages.

Default: \texttt{\ }{\texttt{\ true\ }}\texttt{\ }

\paragraph{\texorpdfstring{\texttt{\ children\ }}{ children }}\label{definitions-header-children}

\href{/docs/reference/foundations/content/}{content}

{Required} {{ Positional }}

\phantomsection\label{definitions-header-children-positional-tooltip}
Positional parameters are specified in order, without names.

{{ Variadic }}

\phantomsection\label{definitions-header-children-variadic-tooltip}
Variadic parameters can be specified multiple times.

The cells and lines within the header.

\subsubsection{\texorpdfstring{\texttt{\ footer\ } {{ Element
}}}{ footer   Element }}\label{definitions-footer}

\phantomsection\label{definitions-footer-element-tooltip}
Element functions can be customized with \texttt{\ set\ } and
\texttt{\ show\ } rules.

A repeatable table footer.

Just like the
\href{/docs/reference/model/table/\#definitions-header}{\texttt{\ table.header\ }}
element, the footer can repeat itself on every page of the table. This
is useful for improving legibility by adding the column labels in both
the header and footer of a large table, totals, or other information
that should be visible on every page.

No other table cells may be placed after the footer.

table { . } { footer } (

{ \hyperref[definitions-footer-parameters-repeat]{repeat :}
\href{/docs/reference/foundations/bool/}{bool} , } {
\hyperref[definitions-footer-parameters-children]{..}
\href{/docs/reference/foundations/content/}{content} , }

) -\textgreater{} \href{/docs/reference/foundations/content/}{content}

\paragraph{\texorpdfstring{\texttt{\ repeat\ }}{ repeat }}\label{definitions-footer-repeat}

\href{/docs/reference/foundations/bool/}{bool}

{{ Settable }}

\phantomsection\label{definitions-footer-repeat-settable-tooltip}
Settable parameters can be customized for all following uses of the
function with a \texttt{\ set\ } rule.

Whether this footer should be repeated across pages.

Default: \texttt{\ }{\texttt{\ true\ }}\texttt{\ }

\paragraph{\texorpdfstring{\texttt{\ children\ }}{ children }}\label{definitions-footer-children}

\href{/docs/reference/foundations/content/}{content}

{Required} {{ Positional }}

\phantomsection\label{definitions-footer-children-positional-tooltip}
Positional parameters are specified in order, without names.

{{ Variadic }}

\phantomsection\label{definitions-footer-children-variadic-tooltip}
Variadic parameters can be specified multiple times.

The cells and lines within the footer.

\href{/docs/reference/model/strong/}{\pandocbounded{\includesvg[keepaspectratio]{/assets/icons/16-arrow-right.svg}}}

{ Strong Emphasis } { Previous page }

\href{/docs/reference/model/terms/}{\pandocbounded{\includesvg[keepaspectratio]{/assets/icons/16-arrow-right.svg}}}

{ Term List } { Next page }
