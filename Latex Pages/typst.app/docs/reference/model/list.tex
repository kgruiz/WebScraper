\title{typst.app/docs/reference/model/list}

\begin{itemize}
\tightlist
\item
  \href{/docs}{\includesvg[width=0.16667in,height=0.16667in]{/assets/icons/16-docs-dark.svg}}
\item
  \includesvg[width=0.16667in,height=0.16667in]{/assets/icons/16-arrow-right.svg}
\item
  \href{/docs/reference/}{Reference}
\item
  \includesvg[width=0.16667in,height=0.16667in]{/assets/icons/16-arrow-right.svg}
\item
  \href{/docs/reference/model/}{Model}
\item
  \includesvg[width=0.16667in,height=0.16667in]{/assets/icons/16-arrow-right.svg}
\item
  \href{/docs/reference/model/list/}{Bullet List}
\end{itemize}

\section{\texorpdfstring{\texttt{\ list\ } {{ Element
}}}{ list   Element }}\label{summary}

\phantomsection\label{element-tooltip}
Element functions can be customized with \texttt{\ set\ } and
\texttt{\ show\ } rules.

A bullet list.

Displays a sequence of items vertically, with each item introduced by a
marker.

\subsection{Example}\label{example}

\begin{verbatim}
Normal list.
- Text
- Math
- Layout
- ...

Multiple lines.
- This list item spans multiple
  lines because it is indented.

Function call.
#list(
  [Foundations],
  [Calculate],
  [Construct],
  [Data Loading],
)
\end{verbatim}

\includegraphics[width=5in,height=\textheight,keepaspectratio]{/assets/docs/dGd96M9aTTHo-jKJ9y73kwAAAAAAAAAA.png}

\subsection{Syntax}\label{syntax}

This functions also has dedicated syntax: Start a line with a hyphen,
followed by a space to create a list item. A list item can contain
multiple paragraphs and other block-level content. All content that is
indented more than an item\textquotesingle s marker becomes part of that
item.

\subsection{\texorpdfstring{{ Parameters
}}{ Parameters }}\label{parameters}

\phantomsection\label{parameters-tooltip}
Parameters are the inputs to a function. They are specified in
parentheses after the function name.

{ list } (

{ \hyperref[parameters-tight]{tight :}
\href{/docs/reference/foundations/bool/}{bool} , } {
\hyperref[parameters-marker]{marker :}
\href{/docs/reference/foundations/content/}{content}
\href{/docs/reference/foundations/array/}{array}
\href{/docs/reference/foundations/function/}{function} , } {
\hyperref[parameters-indent]{indent :}
\href{/docs/reference/layout/length/}{length} , } {
\hyperref[parameters-body-indent]{body-indent :}
\href{/docs/reference/layout/length/}{length} , } {
\hyperref[parameters-spacing]{spacing :}
\href{/docs/reference/foundations/auto/}{auto}
\href{/docs/reference/layout/length/}{length} , } {
\hyperref[parameters-children]{..}
\href{/docs/reference/foundations/content/}{content} , }

) -\textgreater{} \href{/docs/reference/foundations/content/}{content}

\subsubsection{\texorpdfstring{\texttt{\ tight\ }}{ tight }}\label{parameters-tight}

\href{/docs/reference/foundations/bool/}{bool}

{{ Settable }}

\phantomsection\label{parameters-tight-settable-tooltip}
Settable parameters can be customized for all following uses of the
function with a \texttt{\ set\ } rule.

Defines the default
\href{/docs/reference/model/list/\#parameters-spacing}{spacing} of the
list. If it is \texttt{\ }{\texttt{\ false\ }}\texttt{\ } , the items
are spaced apart with
\href{/docs/reference/model/par/\#parameters-spacing}{paragraph spacing}
. If it is \texttt{\ }{\texttt{\ true\ }}\texttt{\ } , they use
\href{/docs/reference/model/par/\#parameters-leading}{paragraph leading}
instead. This makes the list more compact, which can look better if the
items are short.

In markup mode, the value of this parameter is determined based on
whether items are separated with a blank line. If items directly follow
each other, this is set to \texttt{\ }{\texttt{\ true\ }}\texttt{\ } ;
if items are separated by a blank line, this is set to
\texttt{\ }{\texttt{\ false\ }}\texttt{\ } . The markup-defined
tightness cannot be overridden with set rules.

Default: \texttt{\ }{\texttt{\ true\ }}\texttt{\ }

\includesvg[width=0.16667in,height=0.16667in]{/assets/icons/16-arrow-right.svg}
View example

\begin{verbatim}
- If a list has a lot of text, and
  maybe other inline content, it
  should not be tight anymore.

- To make a list wide, simply insert
  a blank line between the items.
\end{verbatim}

\includegraphics[width=5in,height=\textheight,keepaspectratio]{/assets/docs/4FUPGE5Zxz4-Z1S-m_IFCQAAAAAAAAAA.png}

\subsubsection{\texorpdfstring{\texttt{\ marker\ }}{ marker }}\label{parameters-marker}

\href{/docs/reference/foundations/content/}{content} {or}
\href{/docs/reference/foundations/array/}{array} {or}
\href{/docs/reference/foundations/function/}{function}

{{ Settable }}

\phantomsection\label{parameters-marker-settable-tooltip}
Settable parameters can be customized for all following uses of the
function with a \texttt{\ set\ } rule.

The marker which introduces each item.

Instead of plain content, you can also pass an array with multiple
markers that should be used for nested lists. If the list nesting depth
exceeds the number of markers, the markers are cycled. For total
control, you may pass a function that maps the list\textquotesingle s
nesting depth (starting from \texttt{\ }{\texttt{\ 0\ }}\texttt{\ } ) to
a desired marker.

Default:
\texttt{\ }{\texttt{\ (\ }}\texttt{\ }{\texttt{\ {[}\ }}\texttt{\ •\ }{\texttt{\ {]}\ }}\texttt{\ }{\texttt{\ ,\ }}\texttt{\ }{\texttt{\ {[}\ }}\texttt{\ ‣\ }{\texttt{\ {]}\ }}\texttt{\ }{\texttt{\ ,\ }}\texttt{\ }{\texttt{\ {[}\ }}\texttt{\ –\ }{\texttt{\ {]}\ }}\texttt{\ }{\texttt{\ )\ }}\texttt{\ }

\includesvg[width=0.16667in,height=0.16667in]{/assets/icons/16-arrow-right.svg}
View example

\begin{verbatim}
#set list(marker: [--])
- A more classic list
- With en-dashes

#set list(marker: ([•], [--]))
- Top-level
  - Nested
  - Items
- Items
\end{verbatim}

\includegraphics[width=5in,height=\textheight,keepaspectratio]{/assets/docs/rGFZOVIfGIEORB3iENBotQAAAAAAAAAA.png}

\subsubsection{\texorpdfstring{\texttt{\ indent\ }}{ indent }}\label{parameters-indent}

\href{/docs/reference/layout/length/}{length}

{{ Settable }}

\phantomsection\label{parameters-indent-settable-tooltip}
Settable parameters can be customized for all following uses of the
function with a \texttt{\ set\ } rule.

The indent of each item.

Default: \texttt{\ }{\texttt{\ 0pt\ }}\texttt{\ }

\subsubsection{\texorpdfstring{\texttt{\ body-indent\ }}{ body-indent }}\label{parameters-body-indent}

\href{/docs/reference/layout/length/}{length}

{{ Settable }}

\phantomsection\label{parameters-body-indent-settable-tooltip}
Settable parameters can be customized for all following uses of the
function with a \texttt{\ set\ } rule.

The spacing between the marker and the body of each item.

Default: \texttt{\ }{\texttt{\ 0.5em\ }}\texttt{\ }

\subsubsection{\texorpdfstring{\texttt{\ spacing\ }}{ spacing }}\label{parameters-spacing}

\href{/docs/reference/foundations/auto/}{auto} {or}
\href{/docs/reference/layout/length/}{length}

{{ Settable }}

\phantomsection\label{parameters-spacing-settable-tooltip}
Settable parameters can be customized for all following uses of the
function with a \texttt{\ set\ } rule.

The spacing between the items of the list.

If set to \texttt{\ }{\texttt{\ auto\ }}\texttt{\ } , uses paragraph
\href{/docs/reference/model/par/\#parameters-leading}{\texttt{\ leading\ }}
for tight lists and paragraph
\href{/docs/reference/model/par/\#parameters-spacing}{\texttt{\ spacing\ }}
for wide (non-tight) lists.

Default: \texttt{\ }{\texttt{\ auto\ }}\texttt{\ }

\subsubsection{\texorpdfstring{\texttt{\ children\ }}{ children }}\label{parameters-children}

\href{/docs/reference/foundations/content/}{content}

{Required} {{ Positional }}

\phantomsection\label{parameters-children-positional-tooltip}
Positional parameters are specified in order, without names.

{{ Variadic }}

\phantomsection\label{parameters-children-variadic-tooltip}
Variadic parameters can be specified multiple times.

The bullet list\textquotesingle s children.

When using the list syntax, adjacent items are automatically collected
into lists, even through constructs like for loops.

\includesvg[width=0.16667in,height=0.16667in]{/assets/icons/16-arrow-right.svg}
View example

\begin{verbatim}
#for letter in "ABC" [
  - Letter #letter
]
\end{verbatim}

\includegraphics[width=5in,height=\textheight,keepaspectratio]{/assets/docs/scttBXkLjYOvlJchbuo00wAAAAAAAAAA.png}

\subsection{\texorpdfstring{{ Definitions
}}{ Definitions }}\label{definitions}

\phantomsection\label{definitions-tooltip}
Functions and types and can have associated definitions. These are
accessed by specifying the function or type, followed by a period, and
then the definition\textquotesingle s name.

\subsubsection{\texorpdfstring{\texttt{\ item\ } {{ Element
}}}{ item   Element }}\label{definitions-item}

\phantomsection\label{definitions-item-element-tooltip}
Element functions can be customized with \texttt{\ set\ } and
\texttt{\ show\ } rules.

A bullet list item.

list { . } { item } (

{ \href{/docs/reference/foundations/content/}{content} }

) -\textgreater{} \href{/docs/reference/foundations/content/}{content}

\paragraph{\texorpdfstring{\texttt{\ body\ }}{ body }}\label{definitions-item-body}

\href{/docs/reference/foundations/content/}{content}

{Required} {{ Positional }}

\phantomsection\label{definitions-item-body-positional-tooltip}
Positional parameters are specified in order, without names.

The item\textquotesingle s body.

\href{/docs/reference/model/bibliography/}{\pandocbounded{\includesvg[keepaspectratio]{/assets/icons/16-arrow-right.svg}}}

{ Bibliography } { Previous page }

\href{/docs/reference/model/cite/}{\pandocbounded{\includesvg[keepaspectratio]{/assets/icons/16-arrow-right.svg}}}

{ Cite } { Next page }

\textbf{On this page}

\begin{itemize}
\tightlist
\item
  \hyperref[summary]{Summary}
\item
  \hyperref[example]{Example}
\item
  \hyperref[syntax]{Syntax}
\item
  \hyperref[parameters]{Parameters}

  \begin{itemize}
  \tightlist
  \item
    \hyperref[parameters-tight]{tight}
  \item
    \hyperref[parameters-marker]{marker}
  \item
    \hyperref[parameters-indent]{indent}
  \item
    \hyperref[parameters-body-indent]{body-indent}
  \item
    \hyperref[parameters-spacing]{spacing}
  \item
    \hyperref[parameters-children]{children}
  \end{itemize}
\item
  \hyperref[definitions]{Definitions}

  \begin{itemize}
  \tightlist
  \item
    \hyperref[definitions-item]{Bullet List Item}

    \begin{itemize}
    \tightlist
    \item
      \hyperref[definitions-item-body]{body}
    \end{itemize}
  \end{itemize}
\end{itemize}

\begin{itemize}
\tightlist
\item
  \href{/}{Home}
\item
  \href{/pricing/}{Pricing}
\item
  \href{/docs/}{Documentation}
\item
  \href{/universe/}{Universe}
\item
  \href{/about/}{About Us}
\item
  \href{/contact/}{Contact Us}
\item
  \href{/privacy/}{Privacy}
\item
  \href{https://typst.app/terms}{Terms and Conditions}
\item
  \href{/legal/}{Legal (Impressum)}
\end{itemize}

\begin{itemize}
\tightlist
\item
  \href{https://forum.typst.app}{Forum}
\item
  \href{/tools/}{Tools}
\item
  \href{/blog/}{Blog}
\item
  \href{https://github.com/typst/}{GitHub}
\item
  \href{https://discord.gg/2uDybryKPe}{Discord}
\item
  \href{https://mastodon.social/@typst}{Mastodon}
\item
  \href{https://bsky.app/profile/typst.app}{Bluesky}
\item
  \href{https://www.linkedin.com/company/typst/}{LinkedIn}
\item
  \href{https://instagram.com/typstapp/}{Instagram}
\end{itemize}

Made in Berlin
