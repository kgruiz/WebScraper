\title{typst.app/docs/reference/math/cancel}

\begin{itemize}
\tightlist
\item
  \href{/docs}{\includesvg[width=0.16667in,height=0.16667in]{/assets/icons/16-docs-dark.svg}}
\item
  \includesvg[width=0.16667in,height=0.16667in]{/assets/icons/16-arrow-right.svg}
\item
  \href{/docs/reference/}{Reference}
\item
  \includesvg[width=0.16667in,height=0.16667in]{/assets/icons/16-arrow-right.svg}
\item
  \href{/docs/reference/math/}{Math}
\item
  \includesvg[width=0.16667in,height=0.16667in]{/assets/icons/16-arrow-right.svg}
\item
  \href{/docs/reference/math/cancel/}{Cancel}
\end{itemize}

\section{\texorpdfstring{\texttt{\ cancel\ } {{ Element
}}}{ cancel   Element }}\label{summary}

\phantomsection\label{element-tooltip}
Element functions can be customized with \texttt{\ set\ } and
\texttt{\ show\ } rules.

Displays a diagonal line over a part of an equation.

This is commonly used to show the elimination of a term.

\subsection{Example}\label{example}

\begin{verbatim}
Here, we can simplify:
$ (a dot b dot cancel(x)) /
    cancel(x) $
\end{verbatim}

\includegraphics[width=2.91667in,height=\textheight,keepaspectratio]{/assets/docs/fVEZvXjKTk2s3WO88t3K8AAAAAAAAAAA.png}

\subsection{\texorpdfstring{{ Parameters
}}{ Parameters }}\label{parameters}

\phantomsection\label{parameters-tooltip}
Parameters are the inputs to a function. They are specified in
parentheses after the function name.

math { . } { cancel } (

{ \href{/docs/reference/foundations/content/}{content} , } {
\hyperref[parameters-length]{length :}
\href{/docs/reference/layout/relative/}{relative} , } {
\hyperref[parameters-inverted]{inverted :}
\href{/docs/reference/foundations/bool/}{bool} , } {
\hyperref[parameters-cross]{cross :}
\href{/docs/reference/foundations/bool/}{bool} , } {
\hyperref[parameters-angle]{angle :}
\href{/docs/reference/foundations/auto/}{auto}
\href{/docs/reference/layout/angle/}{angle}
\href{/docs/reference/foundations/function/}{function} , } {
\hyperref[parameters-stroke]{stroke :}
\href{/docs/reference/layout/length/}{length}
\href{/docs/reference/visualize/color/}{color}
\href{/docs/reference/visualize/gradient/}{gradient}
\href{/docs/reference/visualize/stroke/}{stroke}
\href{/docs/reference/visualize/pattern/}{pattern}
\href{/docs/reference/foundations/dictionary/}{dictionary} , }

) -\textgreater{} \href{/docs/reference/foundations/content/}{content}

\subsubsection{\texorpdfstring{\texttt{\ body\ }}{ body }}\label{parameters-body}

\href{/docs/reference/foundations/content/}{content}

{Required} {{ Positional }}

\phantomsection\label{parameters-body-positional-tooltip}
Positional parameters are specified in order, without names.

The content over which the line should be placed.

\subsubsection{\texorpdfstring{\texttt{\ length\ }}{ length }}\label{parameters-length}

\href{/docs/reference/layout/relative/}{relative}

{{ Settable }}

\phantomsection\label{parameters-length-settable-tooltip}
Settable parameters can be customized for all following uses of the
function with a \texttt{\ set\ } rule.

The length of the line, relative to the length of the diagonal spanning
the whole element being "cancelled". A value of
\texttt{\ }{\texttt{\ 100\%\ }}\texttt{\ } would then have the line span
precisely the element\textquotesingle s diagonal.

Default:
\texttt{\ }{\texttt{\ 100\%\ }}\texttt{\ }{\texttt{\ +\ }}\texttt{\ }{\texttt{\ 3pt\ }}\texttt{\ }

\includesvg[width=0.16667in,height=0.16667in]{/assets/icons/16-arrow-right.svg}
View example

\begin{verbatim}
$ a + cancel(x, length: #200%)
    - cancel(x, length: #200%) $
\end{verbatim}

\includegraphics[width=2.91667in,height=\textheight,keepaspectratio]{/assets/docs/_RSKVrNDnF5_pAJyRMmcrAAAAAAAAAAA.png}

\subsubsection{\texorpdfstring{\texttt{\ inverted\ }}{ inverted }}\label{parameters-inverted}

\href{/docs/reference/foundations/bool/}{bool}

{{ Settable }}

\phantomsection\label{parameters-inverted-settable-tooltip}
Settable parameters can be customized for all following uses of the
function with a \texttt{\ set\ } rule.

Whether the cancel line should be inverted (flipped along the y-axis).
For the default angle setting, inverted means the cancel line points to
the top left instead of top right.

Default: \texttt{\ }{\texttt{\ false\ }}\texttt{\ }

\includesvg[width=0.16667in,height=0.16667in]{/assets/icons/16-arrow-right.svg}
View example

\begin{verbatim}
$ (a cancel((b + c), inverted: #true)) /
    cancel(b + c, inverted: #true) $
\end{verbatim}

\includegraphics[width=2.91667in,height=\textheight,keepaspectratio]{/assets/docs/GWluRapeZy8kHQiZ5c3XbQAAAAAAAAAA.png}

\subsubsection{\texorpdfstring{\texttt{\ cross\ }}{ cross }}\label{parameters-cross}

\href{/docs/reference/foundations/bool/}{bool}

{{ Settable }}

\phantomsection\label{parameters-cross-settable-tooltip}
Settable parameters can be customized for all following uses of the
function with a \texttt{\ set\ } rule.

Whether two opposing cancel lines should be drawn, forming a cross over
the element. Overrides \texttt{\ inverted\ } .

Default: \texttt{\ }{\texttt{\ false\ }}\texttt{\ }

\includesvg[width=0.16667in,height=0.16667in]{/assets/icons/16-arrow-right.svg}
View example

\begin{verbatim}
$ cancel(Pi, cross: #true) $
\end{verbatim}

\includegraphics[width=2.91667in,height=\textheight,keepaspectratio]{/assets/docs/biIi09LikcDnwaA0WaNwJQAAAAAAAAAA.png}

\subsubsection{\texorpdfstring{\texttt{\ angle\ }}{ angle }}\label{parameters-angle}

\href{/docs/reference/foundations/auto/}{auto} {or}
\href{/docs/reference/layout/angle/}{angle} {or}
\href{/docs/reference/foundations/function/}{function}

{{ Settable }}

\phantomsection\label{parameters-angle-settable-tooltip}
Settable parameters can be customized for all following uses of the
function with a \texttt{\ set\ } rule.

How much to rotate the cancel line.

\begin{itemize}
\tightlist
\item
  If given an angle, the line is rotated by that angle clockwise with
  respect to the y-axis.
\item
  If \texttt{\ }{\texttt{\ auto\ }}\texttt{\ } , the line assumes the
  default angle; that is, along the rising diagonal of the content box.
\item
  If given a function \texttt{\ angle\ =\textgreater{}\ angle\ } , the
  line is rotated, with respect to the y-axis, by the angle returned by
  that function. The function receives the default angle as its input.
\end{itemize}

Default: \texttt{\ }{\texttt{\ auto\ }}\texttt{\ }

\includesvg[width=0.16667in,height=0.16667in]{/assets/icons/16-arrow-right.svg}
View example

\begin{verbatim}
$ cancel(Pi)
  cancel(Pi, angle: #0deg)
  cancel(Pi, angle: #45deg)
  cancel(Pi, angle: #90deg)
  cancel(1/(1+x), angle: #(a => a + 45deg))
  cancel(1/(1+x), angle: #(a => a + 90deg)) $
\end{verbatim}

\includegraphics[width=2.91667in,height=\textheight,keepaspectratio]{/assets/docs/OCEmML9KQSY4Sru0zk3XGwAAAAAAAAAA.png}

\subsubsection{\texorpdfstring{\texttt{\ stroke\ }}{ stroke }}\label{parameters-stroke}

\href{/docs/reference/layout/length/}{length} {or}
\href{/docs/reference/visualize/color/}{color} {or}
\href{/docs/reference/visualize/gradient/}{gradient} {or}
\href{/docs/reference/visualize/stroke/}{stroke} {or}
\href{/docs/reference/visualize/pattern/}{pattern} {or}
\href{/docs/reference/foundations/dictionary/}{dictionary}

{{ Settable }}

\phantomsection\label{parameters-stroke-settable-tooltip}
Settable parameters can be customized for all following uses of the
function with a \texttt{\ set\ } rule.

How to \href{/docs/reference/visualize/stroke/}{stroke} the cancel line.

Default: \texttt{\ }{\texttt{\ 0.5pt\ }}\texttt{\ }

\includesvg[width=0.16667in,height=0.16667in]{/assets/icons/16-arrow-right.svg}
View example

\begin{verbatim}
$ cancel(
  sum x,
  stroke: #(
    paint: red,
    thickness: 1.5pt,
    dash: "dashed",
  ),
) $
\end{verbatim}

\includegraphics[width=2.91667in,height=\textheight,keepaspectratio]{/assets/docs/KCV7eimRh0Q3LxZudj8IDAAAAAAAAAAA.png}

\href{/docs/reference/math/binom/}{\pandocbounded{\includesvg[keepaspectratio]{/assets/icons/16-arrow-right.svg}}}

{ Binomial } { Previous page }

\href{/docs/reference/math/cases/}{\pandocbounded{\includesvg[keepaspectratio]{/assets/icons/16-arrow-right.svg}}}

{ Cases } { Next page }

\textbf{On this page}

\begin{itemize}
\tightlist
\item
  \hyperref[summary]{Summary}
\item
  \hyperref[example]{Example}
\item
  \hyperref[parameters]{Parameters}

  \begin{itemize}
  \tightlist
  \item
    \hyperref[parameters-body]{body}
  \item
    \hyperref[parameters-length]{length}
  \item
    \hyperref[parameters-inverted]{inverted}
  \item
    \hyperref[parameters-cross]{cross}
  \item
    \hyperref[parameters-angle]{angle}
  \item
    \hyperref[parameters-stroke]{stroke}
  \end{itemize}
\end{itemize}

\begin{itemize}
\tightlist
\item
  \href{/}{Home}
\item
  \href{/pricing/}{Pricing}
\item
  \href{/docs/}{Documentation}
\item
  \href{/universe/}{Universe}
\item
  \href{/about/}{About Us}
\item
  \href{/contact/}{Contact Us}
\item
  \href{/privacy/}{Privacy}
\item
  \href{https://typst.app/terms}{Terms and Conditions}
\item
  \href{/legal/}{Legal (Impressum)}
\end{itemize}

\begin{itemize}
\tightlist
\item
  \href{https://forum.typst.app}{Forum}
\item
  \href{/tools/}{Tools}
\item
  \href{/blog/}{Blog}
\item
  \href{https://github.com/typst/}{GitHub}
\item
  \href{https://discord.gg/2uDybryKPe}{Discord}
\item
  \href{https://mastodon.social/@typst}{Mastodon}
\item
  \href{https://bsky.app/profile/typst.app}{Bluesky}
\item
  \href{https://www.linkedin.com/company/typst/}{LinkedIn}
\item
  \href{https://instagram.com/typstapp/}{Instagram}
\end{itemize}

Made in Berlin
