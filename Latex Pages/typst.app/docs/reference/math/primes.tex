\title{typst.app/docs/reference/math/primes}

\begin{itemize}
\tightlist
\item
  \href{/docs}{\includesvg[width=0.16667in,height=0.16667in]{/assets/icons/16-docs-dark.svg}}
\item
  \includesvg[width=0.16667in,height=0.16667in]{/assets/icons/16-arrow-right.svg}
\item
  \href{/docs/reference/}{Reference}
\item
  \includesvg[width=0.16667in,height=0.16667in]{/assets/icons/16-arrow-right.svg}
\item
  \href{/docs/reference/math/}{Math}
\item
  \includesvg[width=0.16667in,height=0.16667in]{/assets/icons/16-arrow-right.svg}
\item
  \href{/docs/reference/math/primes/}{Primes}
\end{itemize}

\section{\texorpdfstring{\texttt{\ primes\ } {{ Element
}}}{ primes   Element }}\label{summary}

\phantomsection\label{element-tooltip}
Element functions can be customized with \texttt{\ set\ } and
\texttt{\ show\ } rules.

Grouped primes.

\begin{verbatim}
$ a'''_b = a^'''_b $
\end{verbatim}

\includegraphics[width=5in,height=\textheight,keepaspectratio]{/assets/docs/uHgNvego3SyqChIc3iZ9sQAAAAAAAAAA.png}

\subsection{Syntax}\label{syntax}

This function has dedicated syntax: use apostrophes instead of primes.
They will automatically attach to the previous element, moving
superscripts to the next level.

\subsection{\texorpdfstring{{ Parameters
}}{ Parameters }}\label{parameters}

\phantomsection\label{parameters-tooltip}
Parameters are the inputs to a function. They are specified in
parentheses after the function name.

math { . } { primes } (

{ \href{/docs/reference/foundations/int/}{int} }

) -\textgreater{} \href{/docs/reference/foundations/content/}{content}

\subsubsection{\texorpdfstring{\texttt{\ count\ }}{ count }}\label{parameters-count}

\href{/docs/reference/foundations/int/}{int}

{Required} {{ Positional }}

\phantomsection\label{parameters-count-positional-tooltip}
Positional parameters are specified in order, without names.

The number of grouped primes.

\href{/docs/reference/math/mat/}{\pandocbounded{\includesvg[keepaspectratio]{/assets/icons/16-arrow-right.svg}}}

{ Matrix } { Previous page }

\href{/docs/reference/math/roots/}{\pandocbounded{\includesvg[keepaspectratio]{/assets/icons/16-arrow-right.svg}}}

{ Roots } { Next page }
