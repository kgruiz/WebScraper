\title{typst.app/docs/reference/math/op}

\begin{itemize}
\tightlist
\item
  \href{/docs}{\includesvg[width=0.16667in,height=0.16667in]{/assets/icons/16-docs-dark.svg}}
\item
  \includesvg[width=0.16667in,height=0.16667in]{/assets/icons/16-arrow-right.svg}
\item
  \href{/docs/reference/}{Reference}
\item
  \includesvg[width=0.16667in,height=0.16667in]{/assets/icons/16-arrow-right.svg}
\item
  \href{/docs/reference/math/}{Math}
\item
  \includesvg[width=0.16667in,height=0.16667in]{/assets/icons/16-arrow-right.svg}
\item
  \href{/docs/reference/math/op/}{Text Operator}
\end{itemize}

\section{\texorpdfstring{\texttt{\ op\ } {{ Element
}}}{ op   Element }}\label{summary}

\phantomsection\label{element-tooltip}
Element functions can be customized with \texttt{\ set\ } and
\texttt{\ show\ } rules.

A text operator in an equation.

\subsection{Example}\label{example}

\begin{verbatim}
$ tan x = (sin x)/(cos x) $
$ op("custom",
     limits: #true)_(n->oo) n $
\end{verbatim}

\includegraphics[width=5in,height=\textheight,keepaspectratio]{/assets/docs/n9yefElmfwTi92ejfLzhZwAAAAAAAAAA.png}

\subsection{Predefined Operators}\label{predefined}

Typst predefines the operators \texttt{\ arccos\ } , \texttt{\ arcsin\ }
, \texttt{\ arctan\ } , \texttt{\ arg\ } , \texttt{\ cos\ } ,
\texttt{\ cosh\ } , \texttt{\ cot\ } , \texttt{\ coth\ } ,
\texttt{\ csc\ } , \texttt{\ csch\ } , \texttt{\ ctg\ } ,
\texttt{\ deg\ } , \texttt{\ det\ } , \texttt{\ dim\ } ,
\texttt{\ exp\ } , \texttt{\ gcd\ } , \texttt{\ hom\ } , \texttt{\ id\ }
, \texttt{\ im\ } , \texttt{\ inf\ } , \texttt{\ ker\ } ,
\texttt{\ lg\ } , \texttt{\ lim\ } , \texttt{\ liminf\ } ,
\texttt{\ limsup\ } , \texttt{\ ln\ } , \texttt{\ log\ } ,
\texttt{\ max\ } , \texttt{\ min\ } , \texttt{\ mod\ } , \texttt{\ Pr\ }
, \texttt{\ sec\ } , \texttt{\ sech\ } , \texttt{\ sin\ } ,
\texttt{\ sinc\ } , \texttt{\ sinh\ } , \texttt{\ sup\ } ,
\texttt{\ tan\ } , \texttt{\ tanh\ } , \texttt{\ tg\ } and
\texttt{\ tr\ } .

\subsection{\texorpdfstring{{ Parameters
}}{ Parameters }}\label{parameters}

\phantomsection\label{parameters-tooltip}
Parameters are the inputs to a function. They are specified in
parentheses after the function name.

math { . } { op } (

{ \href{/docs/reference/foundations/content/}{content} , } {
\hyperref[parameters-limits]{limits :}
\href{/docs/reference/foundations/bool/}{bool} , }

) -\textgreater{} \href{/docs/reference/foundations/content/}{content}

\subsubsection{\texorpdfstring{\texttt{\ text\ }}{ text }}\label{parameters-text}

\href{/docs/reference/foundations/content/}{content}

{Required} {{ Positional }}

\phantomsection\label{parameters-text-positional-tooltip}
Positional parameters are specified in order, without names.

The operator\textquotesingle s text.

\subsubsection{\texorpdfstring{\texttt{\ limits\ }}{ limits }}\label{parameters-limits}

\href{/docs/reference/foundations/bool/}{bool}

{{ Settable }}

\phantomsection\label{parameters-limits-settable-tooltip}
Settable parameters can be customized for all following uses of the
function with a \texttt{\ set\ } rule.

Whether the operator should show attachments as limits in display mode.

Default: \texttt{\ }{\texttt{\ false\ }}\texttt{\ }

\href{/docs/reference/math/styles/}{\pandocbounded{\includesvg[keepaspectratio]{/assets/icons/16-arrow-right.svg}}}

{ Styles } { Previous page }

\href{/docs/reference/math/underover/}{\pandocbounded{\includesvg[keepaspectratio]{/assets/icons/16-arrow-right.svg}}}

{ Under/Over } { Next page }
