\title{typst.app/docs/reference/math/mat}

\begin{itemize}
\tightlist
\item
  \href{/docs}{\includesvg[width=0.16667in,height=0.16667in]{/assets/icons/16-docs-dark.svg}}
\item
  \includesvg[width=0.16667in,height=0.16667in]{/assets/icons/16-arrow-right.svg}
\item
  \href{/docs/reference/}{Reference}
\item
  \includesvg[width=0.16667in,height=0.16667in]{/assets/icons/16-arrow-right.svg}
\item
  \href{/docs/reference/math/}{Math}
\item
  \includesvg[width=0.16667in,height=0.16667in]{/assets/icons/16-arrow-right.svg}
\item
  \href{/docs/reference/math/mat/}{Matrix}
\end{itemize}

\section{\texorpdfstring{\texttt{\ mat\ } {{ Element
}}}{ mat   Element }}\label{summary}

\phantomsection\label{element-tooltip}
Element functions can be customized with \texttt{\ set\ } and
\texttt{\ show\ } rules.

A matrix.

The elements of a row should be separated by commas, while the rows
themselves should be separated by semicolons. The semicolon syntax
merges preceding arguments separated by commas into an array. You can
also use this special syntax of math function calls to define custom
functions that take 2D data.

Content in cells can be aligned with the
\href{/docs/reference/math/mat/\#parameters-align}{\texttt{\ align\ }}
parameter, or content in cells that are in the same row can be aligned
with the \texttt{\ \&\ } symbol.

\subsection{Example}\label{example}

\begin{verbatim}
$ mat(
  1, 2, ..., 10;
  2, 2, ..., 10;
  dots.v, dots.v, dots.down, dots.v;
  10, 10, ..., 10;
) $
\end{verbatim}

\includegraphics[width=5in,height=\textheight,keepaspectratio]{/assets/docs/yiSilYGQ1wRBpIK3ON349AAAAAAAAAAA.png}

\subsection{\texorpdfstring{{ Parameters
}}{ Parameters }}\label{parameters}

\phantomsection\label{parameters-tooltip}
Parameters are the inputs to a function. They are specified in
parentheses after the function name.

math { . } { mat } (

{ \hyperref[parameters-delim]{delim :}
\href{/docs/reference/foundations/none/}{none}
\href{/docs/reference/foundations/str/}{str}
\href{/docs/reference/foundations/array/}{array}
\href{/docs/reference/symbols/symbol/}{symbol} , } {
\hyperref[parameters-align]{align :}
\href{/docs/reference/layout/alignment/}{alignment} , } {
\hyperref[parameters-augment]{augment :}
\href{/docs/reference/foundations/none/}{none}
\href{/docs/reference/foundations/int/}{int}
\href{/docs/reference/foundations/dictionary/}{dictionary} , } {
\hyperref[parameters-gap]{gap :}
\href{/docs/reference/layout/relative/}{relative} , } {
\hyperref[parameters-row-gap]{row-gap :}
\href{/docs/reference/layout/relative/}{relative} , } {
\hyperref[parameters-column-gap]{column-gap :}
\href{/docs/reference/layout/relative/}{relative} , } {
\hyperref[parameters-rows]{..}
\href{/docs/reference/foundations/array/}{array} , }

) -\textgreater{} \href{/docs/reference/foundations/content/}{content}

\subsubsection{\texorpdfstring{\texttt{\ delim\ }}{ delim }}\label{parameters-delim}

\href{/docs/reference/foundations/none/}{none} {or}
\href{/docs/reference/foundations/str/}{str} {or}
\href{/docs/reference/foundations/array/}{array} {or}
\href{/docs/reference/symbols/symbol/}{symbol}

{{ Settable }}

\phantomsection\label{parameters-delim-settable-tooltip}
Settable parameters can be customized for all following uses of the
function with a \texttt{\ set\ } rule.

The delimiter to use.

Can be a single character specifying the left delimiter, in which case
the right delimiter is inferred. Otherwise, can be an array containing a
left and a right delimiter.

Default:
\texttt{\ }{\texttt{\ (\ }}\texttt{\ }{\texttt{\ "("\ }}\texttt{\ }{\texttt{\ ,\ }}\texttt{\ }{\texttt{\ ")"\ }}\texttt{\ }{\texttt{\ )\ }}\texttt{\ }

\includesvg[width=0.16667in,height=0.16667in]{/assets/icons/16-arrow-right.svg}
View example

\begin{verbatim}
#set math.mat(delim: "[")
$ mat(1, 2; 3, 4) $
\end{verbatim}

\includegraphics[width=5in,height=\textheight,keepaspectratio]{/assets/docs/CpCAX34oIjWq-jvec_NKoQAAAAAAAAAA.png}

\subsubsection{\texorpdfstring{\texttt{\ align\ }}{ align }}\label{parameters-align}

\href{/docs/reference/layout/alignment/}{alignment}

{{ Settable }}

\phantomsection\label{parameters-align-settable-tooltip}
Settable parameters can be customized for all following uses of the
function with a \texttt{\ set\ } rule.

The horizontal alignment that each cell should have.

Default: \texttt{\ center\ }

\includesvg[width=0.16667in,height=0.16667in]{/assets/icons/16-arrow-right.svg}
View example

\begin{verbatim}
#set math.mat(align: right)
$ mat(-1, 1, 1; 1, -1, 1; 1, 1, -1) $
\end{verbatim}

\includegraphics[width=5in,height=\textheight,keepaspectratio]{/assets/docs/X3QXNtgXqEVUQfvJRQOPRwAAAAAAAAAA.png}

\subsubsection{\texorpdfstring{\texttt{\ augment\ }}{ augment }}\label{parameters-augment}

\href{/docs/reference/foundations/none/}{none} {or}
\href{/docs/reference/foundations/int/}{int} {or}
\href{/docs/reference/foundations/dictionary/}{dictionary}

{{ Settable }}

\phantomsection\label{parameters-augment-settable-tooltip}
Settable parameters can be customized for all following uses of the
function with a \texttt{\ set\ } rule.

Draws augmentation lines in a matrix.

\begin{itemize}
\tightlist
\item
  \texttt{\ }{\texttt{\ none\ }}\texttt{\ } : No lines are drawn.
\item
  A single number: A vertical augmentation line is drawn after the
  specified column number. Negative numbers start from the end.
\item
  A dictionary: With a dictionary, multiple augmentation lines can be
  drawn both horizontally and vertically. Additionally, the style of the
  lines can be set. The dictionary can contain the following keys:

  \begin{itemize}
  \tightlist
  \item
    \texttt{\ hline\ } : The offsets at which horizontal lines should be
    drawn. For example, an offset of \texttt{\ 2\ } would result in a
    horizontal line being drawn after the second row of the matrix.
    Accepts either an integer for a single line, or an array of integers
    for multiple lines. Like for a single number, negative numbers start
    from the end.
  \item
    \texttt{\ vline\ } : The offsets at which vertical lines should be
    drawn. For example, an offset of \texttt{\ 2\ } would result in a
    vertical line being drawn after the second column of the matrix.
    Accepts either an integer for a single line, or an array of integers
    for multiple lines. Like for a single number, negative numbers start
    from the end.
  \item
    \texttt{\ stroke\ } : How to
    \href{/docs/reference/visualize/stroke/}{stroke} the line. If set to
    \texttt{\ }{\texttt{\ auto\ }}\texttt{\ } , takes on a thickness of
    0.05em and square line caps.
  \end{itemize}
\end{itemize}

Default: \texttt{\ }{\texttt{\ none\ }}\texttt{\ }

\includesvg[width=0.16667in,height=0.16667in]{/assets/icons/16-arrow-right.svg}
View example

\begin{verbatim}
$ mat(1, 0, 1; 0, 1, 2; augment: #2) $
// Equivalent to:
$ mat(1, 0, 1; 0, 1, 2; augment: #(-1)) $
\end{verbatim}

\includegraphics[width=5in,height=\textheight,keepaspectratio]{/assets/docs/4iip0Z9ppDA0SxnJHJihkQAAAAAAAAAA.png}

\begin{verbatim}
$ mat(0, 0, 0; 1, 1, 1; augment: #(hline: 1, stroke: 2pt + green)) $
\end{verbatim}

\includegraphics[width=5in,height=\textheight,keepaspectratio]{/assets/docs/3PHAJpsviSZ-Rqtb3sBd4AAAAAAAAAAA.png}

\subsubsection{\texorpdfstring{\texttt{\ gap\ }}{ gap }}\label{parameters-gap}

\href{/docs/reference/layout/relative/}{relative}

{{ Settable }}

\phantomsection\label{parameters-gap-settable-tooltip}
Settable parameters can be customized for all following uses of the
function with a \texttt{\ set\ } rule.

The gap between rows and columns.

This is a shorthand to set \texttt{\ row-gap\ } and
\texttt{\ column-gap\ } to the same value.

Default:
\texttt{\ }{\texttt{\ 0\%\ }}\texttt{\ }{\texttt{\ +\ }}\texttt{\ }{\texttt{\ 0pt\ }}\texttt{\ }

\includesvg[width=0.16667in,height=0.16667in]{/assets/icons/16-arrow-right.svg}
View example

\begin{verbatim}
#set math.mat(gap: 1em)
$ mat(1, 2; 3, 4) $
\end{verbatim}

\includegraphics[width=5in,height=\textheight,keepaspectratio]{/assets/docs/kaypJSdE1P1lOWZ-cMMpyAAAAAAAAAAA.png}

\subsubsection{\texorpdfstring{\texttt{\ row-gap\ }}{ row-gap }}\label{parameters-row-gap}

\href{/docs/reference/layout/relative/}{relative}

{{ Settable }}

\phantomsection\label{parameters-row-gap-settable-tooltip}
Settable parameters can be customized for all following uses of the
function with a \texttt{\ set\ } rule.

The gap between rows.

Default:
\texttt{\ }{\texttt{\ 0\%\ }}\texttt{\ }{\texttt{\ +\ }}\texttt{\ }{\texttt{\ 0.2em\ }}\texttt{\ }

\includesvg[width=0.16667in,height=0.16667in]{/assets/icons/16-arrow-right.svg}
View example

\begin{verbatim}
#set math.mat(row-gap: 1em)
$ mat(1, 2; 3, 4) $
\end{verbatim}

\includegraphics[width=5in,height=\textheight,keepaspectratio]{/assets/docs/YNVJ8uCnPvrs8e0YkWIQFgAAAAAAAAAA.png}

\subsubsection{\texorpdfstring{\texttt{\ column-gap\ }}{ column-gap }}\label{parameters-column-gap}

\href{/docs/reference/layout/relative/}{relative}

{{ Settable }}

\phantomsection\label{parameters-column-gap-settable-tooltip}
Settable parameters can be customized for all following uses of the
function with a \texttt{\ set\ } rule.

The gap between columns.

Default:
\texttt{\ }{\texttt{\ 0\%\ }}\texttt{\ }{\texttt{\ +\ }}\texttt{\ }{\texttt{\ 0.5em\ }}\texttt{\ }

\includesvg[width=0.16667in,height=0.16667in]{/assets/icons/16-arrow-right.svg}
View example

\begin{verbatim}
#set math.mat(column-gap: 1em)
$ mat(1, 2; 3, 4) $
\end{verbatim}

\includegraphics[width=5in,height=\textheight,keepaspectratio]{/assets/docs/tKmrTRxYwVIL8x7N4tnRyQAAAAAAAAAA.png}

\subsubsection{\texorpdfstring{\texttt{\ rows\ }}{ rows }}\label{parameters-rows}

\href{/docs/reference/foundations/array/}{array}

{Required} {{ Positional }}

\phantomsection\label{parameters-rows-positional-tooltip}
Positional parameters are specified in order, without names.

{{ Variadic }}

\phantomsection\label{parameters-rows-variadic-tooltip}
Variadic parameters can be specified multiple times.

An array of arrays with the rows of the matrix.

\includesvg[width=0.16667in,height=0.16667in]{/assets/icons/16-arrow-right.svg}
View example

\begin{verbatim}
#let data = ((1, 2, 3), (4, 5, 6))
#let matrix = math.mat(..data)
$ v := matrix $
\end{verbatim}

\includegraphics[width=5in,height=\textheight,keepaspectratio]{/assets/docs/N-7caJ4FsPlOdlVrUrNk9gAAAAAAAAAA.png}

\href{/docs/reference/math/lr/}{\pandocbounded{\includesvg[keepaspectratio]{/assets/icons/16-arrow-right.svg}}}

{ Left/Right } { Previous page }

\href{/docs/reference/math/primes/}{\pandocbounded{\includesvg[keepaspectratio]{/assets/icons/16-arrow-right.svg}}}

{ Primes } { Next page }
