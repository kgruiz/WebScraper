\title{typst.app/docs/guides/page-setup-guide}

\begin{itemize}
\tightlist
\item
  \href{/docs}{\includesvg[width=0.16667in,height=0.16667in]{/assets/icons/16-docs-dark.svg}}
\item
  \includesvg[width=0.16667in,height=0.16667in]{/assets/icons/16-arrow-right.svg}
\item
  \href{/docs/guides/}{Guides}
\item
  \includesvg[width=0.16667in,height=0.16667in]{/assets/icons/16-arrow-right.svg}
\item
  \href{/docs/guides/page-setup-guide/}{Page setup guide}
\end{itemize}

\section{Page setup guide}\label{page-setup-guide}

Your page setup is a big part of the first impression your document
gives. Line lengths, margins, and columns influence
\href{https://practicaltypography.com/page-margins.html}{appearance} and
\href{https://designregression.com/article/line-length-revisited-following-the-research}{legibility}
while the right headers and footers will help your reader easily
navigate your document. This guide will help you to customize pages,
margins, headers, footers, and page numbers so that they are the right
fit for your content and you can get started with writing.

In Typst, each page has a width, a height, and margins on all four
sides. The top and bottom margins may contain a header and footer. The
set rule of the \href{/docs/reference/layout/page/}{\texttt{\ page\ }}
element is where you control all of the page setup. If you make changes
with this set rule, Typst will ensure that there is a new and conforming
empty page afterward, so it may insert a page break. Therefore, it is
best to specify your
\href{/docs/reference/layout/page/}{\texttt{\ page\ }} set rule at the
start of your document or in your template.

\begin{verbatim}
#set rect(
  width: 100%,
  height: 100%,
  inset: 4pt,
)

#set page(
  paper: "iso-b7",
  header: rect(fill: aqua)[Header],
  footer: rect(fill: aqua)[Footer],
  number-align: center,
)

#rect(fill: aqua.lighten(40%))
\end{verbatim}

\includegraphics[width=5.19792in,height=\textheight,keepaspectratio]{/assets/docs/f-lyLMO5vdXuAMoj28s3AgAAAAAAAAAA.png}

This example visualizes the dimensions for page content, headers, and
footers. The page content is the page size (ISO B7) minus each
side\textquotesingle s default margin. In the top and the bottom margin,
there are stroked rectangles visualizing the header and footer. They do
not touch the main content, instead, they are offset by 30\% of the
respective margin. You can control this offset by specifying the
\href{/docs/reference/layout/page/\#parameters-header-ascent}{\texttt{\ header-ascent\ }}
and
\href{/docs/reference/layout/page/\#parameters-footer-descent}{\texttt{\ footer-descent\ }}
arguments.

Below, the guide will go more into detail on how to accomplish common
page setup requirements with examples.

\subsection{Customize page size and margins}\label{customize-margins}

Typst\textquotesingle s default page size is A4 paper. Depending on your
region and your use case, you will want to change this. You can do this
by using the \href{/docs/reference/layout/page/}{\texttt{\ page\ }} set
rule and passing it a string argument to use a common page size. Options
include the complete ISO 216 series (e.g. \texttt{\ "iso-a4"\ } ,
\texttt{\ "iso-c2"\ } ), customary US formats like
\texttt{\ "us-legal"\ } or \texttt{\ "us-letter"\ } , and more. Check
out the reference for the
\href{/docs/reference/layout/page/\#parameters-paper}{page\textquotesingle s
paper argument} to learn about all available options.

\begin{verbatim}
#set page("us-letter")

This page likes freedom.
\end{verbatim}

\includegraphics[width=12.75in,height=\textheight,keepaspectratio]{/assets/docs/Qkb9fYb5SMfDaJ63aYycTwAAAAAAAAAA.png}

If you need to customize your page size to some dimensions, you can
specify the named arguments
\href{/docs/reference/layout/page/\#parameters-width}{\texttt{\ width\ }}
and
\href{/docs/reference/layout/page/\#parameters-height}{\texttt{\ height\ }}
instead.

\begin{verbatim}
#set page(width: 12cm, height: 12cm)

This page is a square.
\end{verbatim}

\includegraphics[width=7.08333in,height=\textheight,keepaspectratio]{/assets/docs/RltrP5jjKi_qENhy9B-HyQAAAAAAAAAA.png}

\subsubsection{Change the page\textquotesingle s
margins}\label{change-margins}

Margins are a vital ingredient for good typography:
\href{http://webtypography.net/2.1.2}{Typographers consider lines that
fit between 45 and 75 characters best length for legibility} and your
margins and \hyperref[columns]{columns} help define line widths. By
default, Typst will create margins proportional to the page size of your
document. To set custom margins, you will use the
\href{/docs/reference/layout/page/\#parameters-margin}{\texttt{\ margin\ }}
argument in the \href{/docs/reference/layout/page/}{\texttt{\ page\ }}
set rule.

The \texttt{\ margin\ } argument will accept a length if you want to set
all margins to the same width. However, you often want to set different
margins on each side. To do this, you can pass a dictionary:

\begin{verbatim}
#set page(margin: (
  top: 3cm,
  bottom: 2cm,
  x: 1.5cm,
))

#lorem(100)
\end{verbatim}

\includegraphics[width=5in,height=\textheight,keepaspectratio]{/assets/docs/yJb2DmVYA18DCHWDRA50QQAAAAAAAAAA.png}

The page margin dictionary can have keys for each side (
\texttt{\ top\ } , \texttt{\ bottom\ } , \texttt{\ left\ } ,
\texttt{\ right\ } ), but you can also control left and right together
by setting the \texttt{\ x\ } key of the margin dictionary, like in the
example. Likewise, the top and bottom margins can be adjusted together
by setting the \texttt{\ y\ } key.

If you do not specify margins for all sides in the margin dictionary,
the old margins will remain in effect for the unset sides. To prevent
this and set all remaining margins to a common size, you can use the
\texttt{\ rest\ } key. For example,
\texttt{\ }{\texttt{\ \#\ }}\texttt{\ }{\texttt{\ set\ }}\texttt{\ }{\texttt{\ page\ }}\texttt{\ }{\texttt{\ (\ }}\texttt{\ margin\ }{\texttt{\ :\ }}\texttt{\ }{\texttt{\ (\ }}\texttt{\ left\ }{\texttt{\ :\ }}\texttt{\ }{\texttt{\ 1.5in\ }}\texttt{\ }{\texttt{\ ,\ }}\texttt{\ rest\ }{\texttt{\ :\ }}\texttt{\ }{\texttt{\ 1in\ }}\texttt{\ }{\texttt{\ )\ }}\texttt{\ }{\texttt{\ )\ }}\texttt{\ }
will set the left margin to 1.5 inches and the remaining margins to one
inch.

\subsubsection{Different margins on alternating
pages}\label{alternating-margins}

Sometimes, you\textquotesingle ll need to alternate horizontal margins
for even and odd pages, for example, to have more room towards the spine
of a book than on the outsides of its pages. Typst keeps track of
whether a page is to the left or right of the binding. You can use this
information and set the \texttt{\ inside\ } or \texttt{\ outside\ } keys
of the margin dictionary. The \texttt{\ inside\ } margin points towards
the spine, and the \texttt{\ outside\ } margin points towards the edge
of the bound book.

\begin{verbatim}
#set page(margin: (inside: 2.5cm, outside: 2cm, y: 1.75cm))
\end{verbatim}

Typst will assume that documents written in Left-to-Right scripts are
bound on the left while books written in Right-to-Left scripts are bound
on the right. However, you will need to change this in some cases: If
your first page is output by a different app, the binding is reversed
from Typst\textquotesingle s perspective. Also, some books, like
English-language Mangas are customarily bound on the right, despite
English using Left-to-Right script. To change the binding side and
explicitly set where the \texttt{\ inside\ } and \texttt{\ outside\ }
are, set the
\href{/docs/reference/layout/page/\#parameters-binding}{\texttt{\ binding\ }}
argument in the \href{/docs/reference/layout/page/}{\texttt{\ page\ }}
set rule.

\begin{verbatim}
// Produce a book bound on the right,
// even though it is set in Spanish.
#set text(lang: "es")
#set page(binding: right)
\end{verbatim}

If \texttt{\ binding\ } is \texttt{\ left\ } , \texttt{\ inside\ }
margins will be on the left on odd pages, and vice versa.

\subsection{Add headers and footers}\label{headers-and-footers}

Headers and footers are inserted in the top and bottom margins of every
page. You can add custom headers and footers or just insert a page
number.

In case you need more than just a page number, the best way to insert a
header and a footer are the
\href{/docs/reference/layout/page/\#parameters-header}{\texttt{\ header\ }}
and
\href{/docs/reference/layout/page/\#parameters-footer}{\texttt{\ footer\ }}
arguments of the \href{/docs/reference/layout/page/}{\texttt{\ page\ }}
set rule. You can pass any content as their values:

\begin{verbatim}
#set page(header: [
  _Lisa Strassner's Thesis_
  #h(1fr)
  National Academy of Sciences
])

#lorem(150)
\end{verbatim}

\includegraphics[width=8.73958in,height=\textheight,keepaspectratio]{/assets/docs/8IFKrlz4CTSpYgWgKq02tAAAAAAAAAAA.png}

Headers are bottom-aligned by default so that they do not collide with
the top edge of the page. You can change this by wrapping your header in
the \href{/docs/reference/layout/align/}{\texttt{\ align\ }} function.

\subsubsection{Different header and footer on specific
pages}\label{specific-pages}

You\textquotesingle ll need different headers and footers on some pages.
For example, you may not want a header and footer on the title page. The
example below shows how to conditionally remove the header on the first
page:

\begin{verbatim}
#set page(header: context {
  if counter(page).get().first() > 1 [
    _Lisa Strassner's Thesis_
    #h(1fr)
    National Academy of Sciences
  ]
})

#lorem(150)
\end{verbatim}

This example may look intimidating, but let\textquotesingle s break it
down: By using the \texttt{\ }{\texttt{\ context\ }}\texttt{\ } keyword,
we are telling Typst that the header depends on where we are in the
document. We then ask Typst if the page
\href{/docs/reference/introspection/counter/}{counter} is larger than
one at our (context-dependent) current position. The page counter starts
at one, so we are skipping the header on a single page. Counters may
have multiple levels. This feature is used for items like headings, but
the page counter will always have a single level, so we can just look at
the first one.

You can, of course, add an \texttt{\ else\ } to this example to add a
different header to the first page instead.

\subsubsection{Adapt headers and footers on pages with specific
elements}\label{specific-elements}

The technique described in the previous section can be adapted to
perform more advanced tasks using Typst\textquotesingle s labels. For
example, pages with big tables could omit their headers to help keep
clutter down. We will mark our tables with a
\texttt{\ \textless{}big-table\textgreater{}\ }
\href{/docs/reference/foundations/label/}{label} and use the
\href{/docs/reference/introspection/query/}{query system} to find out if
such a label exists on the current page:

\begin{verbatim}
#set page(header: context {
  let page-counter =
  let matches = query(<big-table>)
  let current = counter(page).get()
  let has-table = matches.any(m =>
    counter(page).at(m.location()) == current
  )

  if not has-table [
    _Lisa Strassner's Thesis_
    #h(1fr)
    National Academy of Sciences
  ]
}))

#lorem(100)
#pagebreak()

#table(
  columns: 2 * (1fr,),
  [A], [B],
  [C], [D],
) <big-table>
\end{verbatim}

Here, we query for all instances of the
\texttt{\ \textless{}big-table\textgreater{}\ } label. We then check
that none of the tables are on the page at our current position. If so,
we print the header. This example also uses variables to be more
concise. Just as above, you could add an \texttt{\ else\ } to add
another header instead of deleting it.

\subsection{Add and customize page numbers}\label{page-numbers}

Page numbers help readers keep track of and reference your document more
easily. The simplest way to insert page numbers is the
\href{/docs/reference/layout/page/\#parameters-numbering}{\texttt{\ numbering\ }}
argument of the \href{/docs/reference/layout/page/}{\texttt{\ page\ }}
set rule. You can pass a
\href{/docs/reference/model/numbering/\#parameters-numbering}{\emph{numbering
pattern}} string that shows how you want your pages to be numbered.

\begin{verbatim}
#set page(numbering: "1")

This is a numbered page.
\end{verbatim}

\includegraphics[width=7.38542in,height=\textheight,keepaspectratio]{/assets/docs/tItTkMk79gxARzJosrxVsgAAAAAAAAAA.png}

Above, you can check out the simplest conceivable example. It adds a
single Arabic page number at the center of the footer. You can specify
other characters than \texttt{\ "1"\ } to get other numerals. For
example, \texttt{\ "i"\ } will yield lowercase Roman numerals. Any
character that is not interpreted as a number will be output as-is. For
example, put dashes around your page number by typing this:

\begin{verbatim}
#set page(numbering: "— 1 —")

This is a — numbered — page.
\end{verbatim}

\includegraphics[width=7.38542in,height=\textheight,keepaspectratio]{/assets/docs/1_R2fSOS46hX8DevWO3SHwAAAAAAAAAA.png}

You can add the total number of pages by entering a second number
character in the string.

\begin{verbatim}
#set page(numbering: "1 of 1")

This is one of many numbered pages.
\end{verbatim}

\includegraphics[width=7.38542in,height=\textheight,keepaspectratio]{/assets/docs/y7IZgFjQvtyaq__YsS98MAAAAAAAAAAA.png}

Go to the
\href{/docs/reference/model/numbering/\#parameters-numbering}{\texttt{\ numbering\ }
function reference} to learn more about the arguments you can pass here.

In case you need to right- or left-align the page number, use the
\href{/docs/reference/layout/page/\#parameters-number-align}{\texttt{\ number-align\ }}
argument of the \href{/docs/reference/layout/page/}{\texttt{\ page\ }}
set rule. Alternating alignment between even and odd pages is not
currently supported using this property. To do this,
you\textquotesingle ll need to specify a custom footer with your
footnote and query the page counter as described in the section on
conditionally omitting headers and footers.

\subsubsection{Custom footer with page
numbers}\label{custom-footer-with-page-numbers}

Sometimes, you need to add other content than a page number to your
footer. However, once a footer is specified, the
\href{/docs/reference/layout/page/\#parameters-numbering}{\texttt{\ numbering\ }}
argument of the \href{/docs/reference/layout/page/}{\texttt{\ page\ }}
set rule is ignored. This section shows you how to add a custom footer
with page numbers and more.

\begin{verbatim}
#set page(footer: context [
  *American Society of Proceedings*
  #h(1fr)
  #counter(page).display(
    "1/1",
    both: true,
  )
])

This page has a custom footer.
\end{verbatim}

\includegraphics[width=7.38542in,height=\textheight,keepaspectratio]{/assets/docs/pa4FfkSAmZ8SbHMJhFhITAAAAAAAAAAA.png}

First, we add some strongly emphasized text on the left and add free
space to fill the line. Then, we call \texttt{\ counter(page)\ } to
retrieve the page counter and use its \texttt{\ display\ } function to
show its current value. We also set \texttt{\ both\ } to
\texttt{\ }{\texttt{\ true\ }}\texttt{\ } so that our numbering pattern
applies to the current \emph{and} final page number.

We can also get more creative with the page number. For example,
let\textquotesingle s insert a circle for each page.

\begin{verbatim}
#set page(footer: context [
  *Fun Typography Club*
  #h(1fr)
  #let (num,) = counter(page).get()
  #let circles = num * (
    box(circle(
      radius: 2pt,
      fill: navy,
    )),
  )
  #box(
    inset: (bottom: 1pt),
    circles.join(h(1pt))
  )
])

This page has a custom footer.
\end{verbatim}

\includegraphics[width=7.38542in,height=\textheight,keepaspectratio]{/assets/docs/Px4MgFDGZh5TfTBwUSS_KAAAAAAAAAAA.png}

In this example, we use the number of pages to create an array of
\href{/docs/reference/visualize/circle/}{circles} . The circles are
wrapped in a \href{/docs/reference/layout/box/}{box} so they can all
appear on the same line because they are blocks and would otherwise
create paragraph breaks. The length of this
\href{/docs/reference/foundations/array/}{array} depends on the current
page number.

We then insert the circles at the right side of the footer, with 1pt of
space between them. The join method of an array will attempt to
\href{/docs/reference/scripting/\#blocks}{\emph{join}} the different
values of an array into a single value, interspersed with its argument.
In our case, we get a single content value with circles and spaces
between them that we can use with the align function. Finally, we use
another box to ensure that the text and the circles can share a line and
use the
\href{/docs/reference/layout/box/\#parameters-inset}{\texttt{\ inset\ }
argument} to raise the circles a bit so they line up nicely with the
text.

\subsubsection{Reset the page number and skip pages}\label{skip-pages}

Do you, at some point in your document, need to reset the page number?
Maybe you want to start with the first page only after the title page.
Or maybe you need to skip a few page numbers because you will insert
pages into the final printed product.

The right way to modify the page number is to manipulate the page
\href{/docs/reference/introspection/counter/}{counter} . The simplest
manipulation is to set the counter back to 1.

\begin{verbatim}
#counter(page).update(1)
\end{verbatim}

This line will reset the page counter back to one. It should be placed
at the start of a page because it will otherwise create a page break.
You can also update the counter given its previous value by passing a
function:

\begin{verbatim}
#counter(page).update(n => n + 5)
\end{verbatim}

In this example, we skip five pages. \texttt{\ n\ } is the current value
of the page counter and \texttt{\ n\ +\ 5\ } is the return value of our
function.

In case you need to retrieve the actual page number instead of the value
of the page counter, you can use the
\href{/docs/reference/introspection/location/\#definitions-page}{\texttt{\ page\ }}
method on the return value of the
\href{/docs/reference/introspection/here/}{\texttt{\ here\ }} function:

\begin{verbatim}
#counter(page).update(n => n + 5)

// This returns one even though the
// page counter was incremented by 5.
#context here().page()
\end{verbatim}

\includegraphics[width=5in,height=\textheight,keepaspectratio]{/assets/docs/09ytRFFbm_ZOLxjka15n_QAAAAAAAAAA.png}

You can also obtain the page numbering pattern from the location
returned by \texttt{\ here\ } with the
\href{/docs/reference/introspection/location/\#definitions-page-numbering}{\texttt{\ page-numbering\ }}
method.

\subsection{Add columns}\label{columns}

Add columns to your document to fit more on a page while maintaining
legible line lengths. Columns are vertical blocks of text which are
separated by some whitespace. This space is called the gutter.

To lay out your content in columns, just specify the desired number of
columns in a
\href{/docs/reference/layout/page/\#parameters-columns}{\texttt{\ page\ }}
set rule. To adjust the amount of space between the columns, add a set
rule on the \href{/docs/reference/layout/columns/}{\texttt{\ columns\ }
function} , specifying the \texttt{\ gutter\ } parameter.

\begin{verbatim}
#set page(columns: 2)
#set columns(gutter: 12pt)

#lorem(30)
\end{verbatim}

\includegraphics[width=5in,height=\textheight,keepaspectratio]{/assets/docs/FgQ5mR5BdbOwdaPvmQss-wAAAAAAAAAA.png}

Very commonly, scientific papers have a single-column title and
abstract, while the main body is set in two-columns. To achieve this
effect, Typst\textquotesingle s
\href{/docs/reference/layout/place/}{\texttt{\ place\ } function} can
temporarily escape the two-column layout by specifying
\texttt{\ float:\ }{\texttt{\ true\ }}\texttt{\ } and
\texttt{\ scope:\ }{\texttt{\ "parent"\ }}\texttt{\ } :

\begin{verbatim}
#set page(columns: 2)
#set par(justify: true)

#place(
  top + center,
  float: true,
  scope: "parent",
  text(1.4em, weight: "bold")[
    Impacts of Odobenidae
  ],
)

== About seals in the wild
#lorem(80)
\end{verbatim}

\includegraphics[width=5in,height=\textheight,keepaspectratio]{/assets/docs/rwgbgpUV52923GFV_9bCEQAAAAAAAAAA.png}

\emph{Floating placement} refers to elements being pushed to the top or
bottom of the column or page, with the remaining content flowing in
between. It is also frequently used for
\href{/docs/reference/model/figure/\#parameters-placement}{figures} .

\subsubsection{Use columns anywhere in your
document}\label{columns-anywhere}

To create columns within a nested layout, e.g. within a rectangle, you
can use the \href{/docs/reference/layout/columns/}{\texttt{\ columns\ }
function} directly. However, it really should only be used within nested
layouts. At the page-level, the page set rule is preferrable because it
has better interactions with things like page-level floats, footnotes,
and line numbers.

\begin{verbatim}
#rect(
  width: 6cm,
  height: 3.5cm,
  columns(2, gutter: 12pt)[
    In the dimly lit gas station,
    a solitary taxi stood silently,
    its yellow paint fading with
    time. Its windows were dark,
    its engine idle, and its tires
    rested on the cold concrete.
  ]
)
\end{verbatim}

\includegraphics[width=5in,height=\textheight,keepaspectratio]{/assets/docs/4ald2xJRDpaFMxVXu7aiUQAAAAAAAAAA.png}

\subsubsection{Balanced columns}\label{balanced-columns}

If the columns on the last page of a document differ greatly in length,
they may create a lopsided and unappealing layout.
That\textquotesingle s why typographers will often equalize the length
of columns on the last page. This effect is called balancing columns.
Typst cannot yet balance columns automatically. However, you can balance
columns manually by placing
\href{/docs/reference/layout/colbreak/}{\texttt{\ }{\texttt{\ \#\ }}\texttt{\ }{\texttt{\ colbreak\ }}\texttt{\ }{\texttt{\ (\ }}\texttt{\ }{\texttt{\ )\ }}\texttt{\ }}
at an appropriate spot in your markup, creating the desired column break
manually.

\subsection{One-off modifications}\label{one-off-modifications}

You do not need to override your page settings if you need to insert a
single page with a different setup. For example, you may want to insert
a page that\textquotesingle s flipped to landscape to insert a big table
or change the margin and columns for your title page. In this case, you
can call \href{/docs/reference/layout/page/}{\texttt{\ page\ }} as a
function with your content as an argument and the overrides as the other
arguments. This will insert enough new pages with your overridden
settings to place your content on them. Typst will revert to the page
settings from the set rule after the call.

\begin{verbatim}
#page(flipped: true)[
  = Multiplication table

  #table(
    columns: 5 * (1fr,),
    ..for x in range(1, 10) {
      for y in range(1, 6) {
        (str(x*y),)
      }
    }
  )
]
\end{verbatim}

\includegraphics[width=8.73958in,height=\textheight,keepaspectratio]{/assets/docs/U5FByA07ZCcGdizR_XbtEwAAAAAAAAAA.png}

\href{/docs/guides/guide-for-latex-users/}{\pandocbounded{\includesvg[keepaspectratio]{/assets/icons/16-arrow-right.svg}}}

{ Guide for LaTeX users } { Previous page }

\href{/docs/guides/table-guide/}{\pandocbounded{\includesvg[keepaspectratio]{/assets/icons/16-arrow-right.svg}}}

{ Table guide } { Next page }

\textbf{On this page}

\begin{itemize}
\tightlist
\item
  \hyperref[customize-margins]{Customize Margins}

  \begin{itemize}
  \tightlist
  \item
    \hyperref[change-margins]{Change Margins}
  \item
    \hyperref[alternating-margins]{Alternating Margins}
  \end{itemize}
\item
  \hyperref[headers-and-footers]{Headers And Footers}

  \begin{itemize}
  \tightlist
  \item
    \hyperref[specific-pages]{Specific Pages}
  \item
    \hyperref[specific-elements]{Specific Elements}
  \end{itemize}
\item
  \hyperref[page-numbers]{Page Numbers}

  \begin{itemize}
  \tightlist
  \item
    \hyperref[custom-footer-with-page-numbers]{Custom Footer With Page
    Numbers}
  \item
    \hyperref[skip-pages]{Skip Pages}
  \end{itemize}
\item
  \hyperref[columns]{Columns}

  \begin{itemize}
  \tightlist
  \item
    \hyperref[columns-anywhere]{Columns Anywhere}
  \item
    \hyperref[balanced-columns]{Balanced Columns}
  \end{itemize}
\item
  \hyperref[one-off-modifications]{One Off Modifications}
\end{itemize}

\begin{itemize}
\tightlist
\item
  \href{/}{Home}
\item
  \href{/pricing/}{Pricing}
\item
  \href{/docs/}{Documentation}
\item
  \href{/universe/}{Universe}
\item
  \href{/about/}{About Us}
\item
  \href{/contact/}{Contact Us}
\item
  \href{/privacy/}{Privacy}
\item
  \href{https://typst.app/terms}{Terms and Conditions}
\item
  \href{/legal/}{Legal (Impressum)}
\end{itemize}

\begin{itemize}
\tightlist
\item
  \href{https://forum.typst.app}{Forum}
\item
  \href{/tools/}{Tools}
\item
  \href{/blog/}{Blog}
\item
  \href{https://github.com/typst/}{GitHub}
\item
  \href{https://discord.gg/2uDybryKPe}{Discord}
\item
  \href{https://mastodon.social/@typst}{Mastodon}
\item
  \href{https://bsky.app/profile/typst.app}{Bluesky}
\item
  \href{https://www.linkedin.com/company/typst/}{LinkedIn}
\item
  \href{https://instagram.com/typstapp/}{Instagram}
\end{itemize}

Made in Berlin
