\title{typst.app/docs/reference/text/raw}

\begin{itemize}
\tightlist
\item
  \href{/docs}{\includesvg[width=0.16667in,height=0.16667in]{/assets/icons/16-docs-dark.svg}}
\item
  \includesvg[width=0.16667in,height=0.16667in]{/assets/icons/16-arrow-right.svg}
\item
  \href{/docs/reference/}{Reference}
\item
  \includesvg[width=0.16667in,height=0.16667in]{/assets/icons/16-arrow-right.svg}
\item
  \href{/docs/reference/text/}{Text}
\item
  \includesvg[width=0.16667in,height=0.16667in]{/assets/icons/16-arrow-right.svg}
\item
  \href{/docs/reference/text/raw/}{Raw Text / Code}
\end{itemize}

\section{\texorpdfstring{\texttt{\ raw\ } {{ Element
}}}{ raw   Element }}\label{summary}

\phantomsection\label{element-tooltip}
Element functions can be customized with \texttt{\ set\ } and
\texttt{\ show\ } rules.

Raw text with optional syntax highlighting.

Displays the text verbatim and in a monospace font. This is typically
used to embed computer code into your document.

\subsection{Example}\label{example}

\begin{verbatim}
Adding `rbx` to `rcx` gives
the desired result.

What is ```rust fn main()``` in Rust
would be ```c int main()``` in C.

```rust
fn main() {
    println!("Hello World!");
}
```

This has ``` `backticks` ``` in it
(but the spaces are trimmed). And
``` here``` the leading space is
also trimmed.
\end{verbatim}

\includegraphics[width=5in,height=\textheight,keepaspectratio]{/assets/docs/HG5qpETGRO7ndBI1Qrek9gAAAAAAAAAA.png}

You can also construct a
\href{/docs/reference/text/raw/}{\texttt{\ raw\ }} element
programmatically from a string (and provide the language tag via the
optional
\href{/docs/reference/text/raw/\#parameters-lang}{\texttt{\ lang\ }}
argument).

\begin{verbatim}
#raw("fn " + "main() {}", lang: "rust")
\end{verbatim}

\includegraphics[width=5in,height=\textheight,keepaspectratio]{/assets/docs/MNABiMKxTxPPaXzIwfuPPQAAAAAAAAAA.png}

\subsection{Syntax}\label{syntax}

This function also has dedicated syntax. You can enclose text in 1 or 3+
backticks ( \texttt{\ \textasciigrave{}\ } ) to make it raw. Two
backticks produce empty raw text. This works both in markup and code.

When you use three or more backticks, you can additionally specify a
language tag for syntax highlighting directly after the opening
backticks. Within raw blocks, everything (except for the language tag,
if applicable) is rendered as is, in particular, there are no escape
sequences.

The language tag is an identifier that directly follows the opening
backticks only if there are three or more backticks. If your text starts
with something that looks like an identifier, but no syntax highlighting
is needed, start the text with a single space (which will be trimmed) or
use the single backtick syntax. If your text should start or end with a
backtick, put a space before or after it (it will be trimmed).

\subsection{\texorpdfstring{{ Parameters
}}{ Parameters }}\label{parameters}

\phantomsection\label{parameters-tooltip}
Parameters are the inputs to a function. They are specified in
parentheses after the function name.

{ raw } (

{ \href{/docs/reference/foundations/str/}{str} , } {
\hyperref[parameters-block]{block :}
\href{/docs/reference/foundations/bool/}{bool} , } {
\hyperref[parameters-lang]{lang :}
\href{/docs/reference/foundations/none/}{none}
\href{/docs/reference/foundations/str/}{str} , } {
\hyperref[parameters-align]{align :}
\href{/docs/reference/layout/alignment/}{alignment} , } {
\hyperref[parameters-syntaxes]{syntaxes :}
\href{/docs/reference/foundations/str/}{str}
\href{/docs/reference/foundations/array/}{array} , } {
\hyperref[parameters-theme]{theme :}
\href{/docs/reference/foundations/none/}{none}
\href{/docs/reference/foundations/auto/}{auto}
\href{/docs/reference/foundations/str/}{str} , } {
\hyperref[parameters-tab-size]{tab-size :}
\href{/docs/reference/foundations/int/}{int} , }

) -\textgreater{} \href{/docs/reference/foundations/content/}{content}

\subsubsection{\texorpdfstring{\texttt{\ text\ }}{ text }}\label{parameters-text}

\href{/docs/reference/foundations/str/}{str}

{Required} {{ Positional }}

\phantomsection\label{parameters-text-positional-tooltip}
Positional parameters are specified in order, without names.

The raw text.

You can also use raw blocks creatively to create custom syntaxes for
your automations.

\includesvg[width=0.16667in,height=0.16667in]{/assets/icons/16-arrow-right.svg}
View example

\begin{verbatim}
// Parse numbers in raw blocks with the
// `mydsl` tag and sum them up.
#show raw.where(lang: "mydsl"): it => {
  let sum = 0
  for part in it.text.split("+") {
    sum += int(part.trim())
  }
  sum
}

```mydsl
1 + 2 + 3 + 4 + 5
```
\end{verbatim}

\includegraphics[width=5in,height=\textheight,keepaspectratio]{/assets/docs/6VperjQoP8Ey0LiUk5m0HQAAAAAAAAAA.png}

\subsubsection{\texorpdfstring{\texttt{\ block\ }}{ block }}\label{parameters-block}

\href{/docs/reference/foundations/bool/}{bool}

{{ Settable }}

\phantomsection\label{parameters-block-settable-tooltip}
Settable parameters can be customized for all following uses of the
function with a \texttt{\ set\ } rule.

Whether the raw text is displayed as a separate block.

In markup mode, using one-backtick notation makes this
\texttt{\ }{\texttt{\ false\ }}\texttt{\ } . Using three-backtick
notation makes it \texttt{\ }{\texttt{\ true\ }}\texttt{\ } if the
enclosed content contains at least one line break.

Default: \texttt{\ }{\texttt{\ false\ }}\texttt{\ }

\includesvg[width=0.16667in,height=0.16667in]{/assets/icons/16-arrow-right.svg}
View example

\begin{verbatim}
// Display inline code in a small box
// that retains the correct baseline.
#show raw.where(block: false): box.with(
  fill: luma(240),
  inset: (x: 3pt, y: 0pt),
  outset: (y: 3pt),
  radius: 2pt,
)

// Display block code in a larger block
// with more padding.
#show raw.where(block: true): block.with(
  fill: luma(240),
  inset: 10pt,
  radius: 4pt,
)

With `rg`, you can search through your files quickly.
This example searches the current directory recursively
for the text `Hello World`:

```bash
rg "Hello World"
```
\end{verbatim}

\includegraphics[width=5in,height=\textheight,keepaspectratio]{/assets/docs/PgXCmmr2Cn53ZnpWQOnMwwAAAAAAAAAA.png}

\subsubsection{\texorpdfstring{\texttt{\ lang\ }}{ lang }}\label{parameters-lang}

\href{/docs/reference/foundations/none/}{none} {or}
\href{/docs/reference/foundations/str/}{str}

{{ Settable }}

\phantomsection\label{parameters-lang-settable-tooltip}
Settable parameters can be customized for all following uses of the
function with a \texttt{\ set\ } rule.

The language to syntax-highlight in.

Apart from typical language tags known from Markdown, this supports the
\texttt{\ }{\texttt{\ "typ"\ }}\texttt{\ } ,
\texttt{\ }{\texttt{\ "typc"\ }}\texttt{\ } , and
\texttt{\ }{\texttt{\ "typm"\ }}\texttt{\ } tags for
\href{/docs/reference/syntax/\#markup}{Typst markup} ,
\href{/docs/reference/syntax/\#code}{Typst code} , and
\href{/docs/reference/syntax/\#math}{Typst math} , respectively.

Default: \texttt{\ }{\texttt{\ none\ }}\texttt{\ }

\includesvg[width=0.16667in,height=0.16667in]{/assets/icons/16-arrow-right.svg}
View example

\begin{verbatim}
```typ
This is *Typst!*
```

This is ```typ also *Typst*```, but inline!
\end{verbatim}

\includegraphics[width=5in,height=\textheight,keepaspectratio]{/assets/docs/bjU3PMlFs9msUi72QThHnAAAAAAAAAAA.png}

\subsubsection{\texorpdfstring{\texttt{\ align\ }}{ align }}\label{parameters-align}

\href{/docs/reference/layout/alignment/}{alignment}

{{ Settable }}

\phantomsection\label{parameters-align-settable-tooltip}
Settable parameters can be customized for all following uses of the
function with a \texttt{\ set\ } rule.

The horizontal alignment that each line in a raw block should have. This
option is ignored if this is not a raw block (if specified
\texttt{\ block:\ false\ } or single backticks were used in markup
mode).

By default, this is set to \texttt{\ start\ } , meaning that raw text is
aligned towards the start of the text direction inside the block by
default, regardless of the current context\textquotesingle s alignment
(allowing you to center the raw block itself without centering the text
inside it, for example).

Default: \texttt{\ start\ }

\includesvg[width=0.16667in,height=0.16667in]{/assets/icons/16-arrow-right.svg}
View example

\begin{verbatim}
#set raw(align: center)

```typc
let f(x) = x
code = "centered"
```
\end{verbatim}

\includegraphics[width=5in,height=\textheight,keepaspectratio]{/assets/docs/QoY61HWjc7MIUABTr8mvwwAAAAAAAAAA.png}

\subsubsection{\texorpdfstring{\texttt{\ syntaxes\ }}{ syntaxes }}\label{parameters-syntaxes}

\href{/docs/reference/foundations/str/}{str} {or}
\href{/docs/reference/foundations/array/}{array}

{{ Settable }}

\phantomsection\label{parameters-syntaxes-settable-tooltip}
Settable parameters can be customized for all following uses of the
function with a \texttt{\ set\ } rule.

One or multiple additional syntax definitions to load. The syntax
definitions should be in the
\href{https://www.sublimetext.com/docs/syntax.html}{\texttt{\ sublime-syntax\ }
file format} .

Default:
\texttt{\ }{\texttt{\ (\ }}\texttt{\ }{\texttt{\ )\ }}\texttt{\ }

\includesvg[width=0.16667in,height=0.16667in]{/assets/icons/16-arrow-right.svg}
View example

\begin{verbatim}
#set raw(syntaxes: "SExpressions.sublime-syntax")

```sexp
(defun factorial (x)
  (if (zerop x)
    ; with a comment
    1
    (* x (factorial (- x 1)))))
```
\end{verbatim}

\includegraphics[width=5in,height=\textheight,keepaspectratio]{/assets/docs/f1wEtKdjbuwy-LVNGIZ_igAAAAAAAAAA.png}

\subsubsection{\texorpdfstring{\texttt{\ theme\ }}{ theme }}\label{parameters-theme}

\href{/docs/reference/foundations/none/}{none} {or}
\href{/docs/reference/foundations/auto/}{auto} {or}
\href{/docs/reference/foundations/str/}{str}

{{ Settable }}

\phantomsection\label{parameters-theme-settable-tooltip}
Settable parameters can be customized for all following uses of the
function with a \texttt{\ set\ } rule.

The theme to use for syntax highlighting. Theme files should be in the
\href{https://www.sublimetext.com/docs/color_schemes_tmtheme.html}{\texttt{\ tmTheme\ }
file format} .

Applying a theme only affects the color of specifically highlighted
text. It does not consider the theme\textquotesingle s foreground and
background properties, so that you retain control over the color of raw
text. You can apply the foreground color yourself with the
\href{/docs/reference/text/text/}{\texttt{\ text\ }} function and the
background with a
\href{/docs/reference/layout/block/\#parameters-fill}{filled block} .
You could also use the
\href{/docs/reference/data-loading/xml/}{\texttt{\ xml\ }} function to
extract these properties from the theme.

Additionally, you can set the theme to
\texttt{\ }{\texttt{\ none\ }}\texttt{\ } to disable highlighting.

Default: \texttt{\ }{\texttt{\ auto\ }}\texttt{\ }

\includesvg[width=0.16667in,height=0.16667in]{/assets/icons/16-arrow-right.svg}
View example

\begin{verbatim}
#set raw(theme: "halcyon.tmTheme")
#show raw: it => block(
  fill: rgb("#1d2433"),
  inset: 8pt,
  radius: 5pt,
  text(fill: rgb("#a2aabc"), it)
)

```typ
= Chapter 1
#let hi = "Hello World"
```
\end{verbatim}

\includegraphics[width=5in,height=\textheight,keepaspectratio]{/assets/docs/_3ndU0y1KsOpDAMv999GwwAAAAAAAAAA.png}

\subsubsection{\texorpdfstring{\texttt{\ tab-size\ }}{ tab-size }}\label{parameters-tab-size}

\href{/docs/reference/foundations/int/}{int}

{{ Settable }}

\phantomsection\label{parameters-tab-size-settable-tooltip}
Settable parameters can be customized for all following uses of the
function with a \texttt{\ set\ } rule.

The size for a tab stop in spaces. A tab is replaced with enough spaces
to align with the next multiple of the size.

Default: \texttt{\ }{\texttt{\ 2\ }}\texttt{\ }

\includesvg[width=0.16667in,height=0.16667in]{/assets/icons/16-arrow-right.svg}
View example

\begin{verbatim}
#set raw(tab-size: 8)
```tsv
Year    Month   Day
2000    2   3
2001    2   1
2002    3   10
```
\end{verbatim}

\includegraphics[width=5in,height=\textheight,keepaspectratio]{/assets/docs/OAN98lLQ4wUhrTrjbVCTywAAAAAAAAAA.png}

\subsection{\texorpdfstring{{ Definitions
}}{ Definitions }}\label{definitions}

\phantomsection\label{definitions-tooltip}
Functions and types and can have associated definitions. These are
accessed by specifying the function or type, followed by a period, and
then the definition\textquotesingle s name.

\subsubsection{\texorpdfstring{\texttt{\ line\ } {{ Element
}}}{ line   Element }}\label{definitions-line}

\phantomsection\label{definitions-line-element-tooltip}
Element functions can be customized with \texttt{\ set\ } and
\texttt{\ show\ } rules.

A highlighted line of raw text.

This is a helper element that is synthesized by
\href{/docs/reference/text/raw/}{\texttt{\ raw\ }} elements.

It allows you to access various properties of the line, such as the line
number, the raw non-highlighted text, the highlighted text, and whether
it is the first or last line of the raw block.

raw { . } { line } (

{ \href{/docs/reference/foundations/int/}{int} , } {
\href{/docs/reference/foundations/int/}{int} , } {
\href{/docs/reference/foundations/str/}{str} , } {
\href{/docs/reference/foundations/content/}{content} , }

) -\textgreater{} \href{/docs/reference/foundations/content/}{content}

\paragraph{\texorpdfstring{\texttt{\ number\ }}{ number }}\label{definitions-line-number}

\href{/docs/reference/foundations/int/}{int}

{Required} {{ Positional }}

\phantomsection\label{definitions-line-number-positional-tooltip}
Positional parameters are specified in order, without names.

The line number of the raw line inside of the raw block, starts at 1.

\paragraph{\texorpdfstring{\texttt{\ count\ }}{ count }}\label{definitions-line-count}

\href{/docs/reference/foundations/int/}{int}

{Required} {{ Positional }}

\phantomsection\label{definitions-line-count-positional-tooltip}
Positional parameters are specified in order, without names.

The total number of lines in the raw block.

\paragraph{\texorpdfstring{\texttt{\ text\ }}{ text }}\label{definitions-line-text}

\href{/docs/reference/foundations/str/}{str}

{Required} {{ Positional }}

\phantomsection\label{definitions-line-text-positional-tooltip}
Positional parameters are specified in order, without names.

The line of raw text.

\paragraph{\texorpdfstring{\texttt{\ body\ }}{ body }}\label{definitions-line-body}

\href{/docs/reference/foundations/content/}{content}

{Required} {{ Positional }}

\phantomsection\label{definitions-line-body-positional-tooltip}
Positional parameters are specified in order, without names.

The highlighted raw text.

\href{/docs/reference/text/overline/}{\pandocbounded{\includesvg[keepaspectratio]{/assets/icons/16-arrow-right.svg}}}

{ Overline } { Previous page }

\href{/docs/reference/text/smallcaps/}{\pandocbounded{\includesvg[keepaspectratio]{/assets/icons/16-arrow-right.svg}}}

{ Small Capitals } { Next page }
