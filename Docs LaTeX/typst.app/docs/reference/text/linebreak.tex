\title{typst.app/docs/reference/text/linebreak}

\begin{itemize}
\tightlist
\item
  \href{/docs}{\includesvg[width=0.16667in,height=0.16667in]{/assets/icons/16-docs-dark.svg}}
\item
  \includesvg[width=0.16667in,height=0.16667in]{/assets/icons/16-arrow-right.svg}
\item
  \href{/docs/reference/}{Reference}
\item
  \includesvg[width=0.16667in,height=0.16667in]{/assets/icons/16-arrow-right.svg}
\item
  \href{/docs/reference/text/}{Text}
\item
  \includesvg[width=0.16667in,height=0.16667in]{/assets/icons/16-arrow-right.svg}
\item
  \href{/docs/reference/text/linebreak/}{Line Break}
\end{itemize}

\section{\texorpdfstring{\texttt{\ linebreak\ } {{ Element
}}}{ linebreak   Element }}\label{summary}

\phantomsection\label{element-tooltip}
Element functions can be customized with \texttt{\ set\ } and
\texttt{\ show\ } rules.

Inserts a line break.

Advances the paragraph to the next line. A single trailing line break at
the end of a paragraph is ignored, but more than one creates additional
empty lines.

\subsection{Example}\label{example}

\begin{verbatim}
*Date:* 26.12.2022 \
*Topic:* Infrastructure Test \
*Severity:* High \
\end{verbatim}

\includegraphics[width=5in,height=\textheight,keepaspectratio]{/assets/docs/OEyyibskK4bIsTh7Qcp7OAAAAAAAAAAA.png}

\subsection{Syntax}\label{syntax}

This function also has dedicated syntax: To insert a line break, simply
write a backslash followed by whitespace. This always creates an
unjustified break.

\subsection{\texorpdfstring{{ Parameters
}}{ Parameters }}\label{parameters}

\phantomsection\label{parameters-tooltip}
Parameters are the inputs to a function. They are specified in
parentheses after the function name.

{ linebreak } (

{ \hyperref[parameters-justify]{justify :}
\href{/docs/reference/foundations/bool/}{bool} }

) -\textgreater{} \href{/docs/reference/foundations/content/}{content}

\subsubsection{\texorpdfstring{\texttt{\ justify\ }}{ justify }}\label{parameters-justify}

\href{/docs/reference/foundations/bool/}{bool}

{{ Settable }}

\phantomsection\label{parameters-justify-settable-tooltip}
Settable parameters can be customized for all following uses of the
function with a \texttt{\ set\ } rule.

Whether to justify the line before the break.

This is useful if you found a better line break opportunity in your
justified text than Typst did.

Default: \texttt{\ }{\texttt{\ false\ }}\texttt{\ }

\includesvg[width=0.16667in,height=0.16667in]{/assets/icons/16-arrow-right.svg}
View example

\begin{verbatim}
#set par(justify: true)
#let jb = linebreak(justify: true)

I have manually tuned the #jb
line breaks in this paragraph #jb
for an _interesting_ result. #jb
\end{verbatim}

\includegraphics[width=5in,height=\textheight,keepaspectratio]{/assets/docs/RlJnAEDPiPVRCZ7poOHTOwAAAAAAAAAA.png}

\href{/docs/reference/text/highlight/}{\pandocbounded{\includesvg[keepaspectratio]{/assets/icons/16-arrow-right.svg}}}

{ Highlight } { Previous page }

\href{/docs/reference/text/lorem/}{\pandocbounded{\includesvg[keepaspectratio]{/assets/icons/16-arrow-right.svg}}}

{ Lorem } { Next page }
