\title{typst.app/docs/reference/syntax}

\begin{itemize}
\tightlist
\item
  \href{/docs}{\includesvg[width=0.16667in,height=0.16667in]{/assets/icons/16-docs-dark.svg}}
\item
  \includesvg[width=0.16667in,height=0.16667in]{/assets/icons/16-arrow-right.svg}
\item
  \href{/docs/reference/}{Reference}
\item
  \includesvg[width=0.16667in,height=0.16667in]{/assets/icons/16-arrow-right.svg}
\item
  \href{/docs/reference/syntax/}{Syntax}
\end{itemize}

\section{Syntax}\label{syntax}

Typst is a markup language. This means that you can use simple syntax to
accomplish common layout tasks. The lightweight markup syntax is
complemented by set and show rules, which let you style your document
easily and automatically. All this is backed by a tightly integrated
scripting language with built-in and user-defined functions.

\subsection{Modes}\label{modes}

Typst has three syntactical modes: Markup, math, and code. Markup mode
is the default in a Typst document, math mode lets you write
mathematical formulas, and code mode lets you use
Typst\textquotesingle s scripting features.

You can switch to a specific mode at any point by referring to the
following table:

\begin{longtable}[]{@{}lll@{}}
\toprule\noalign{}
New mode & Syntax & Example \\
\midrule\noalign{}
\endhead
\bottomrule\noalign{}
\endlastfoot
Code & Prefix the code with \texttt{\ \#\ } &
\texttt{\ Number:\ }{\texttt{\ \#\ }}\texttt{\ }{\texttt{\ (\ }}\texttt{\ }{\texttt{\ 1\ }}\texttt{\ }{\texttt{\ +\ }}\texttt{\ }{\texttt{\ 2\ }}\texttt{\ }{\texttt{\ )\ }}\texttt{\ } \\
Math & Surround equation with
\texttt{\ }{\texttt{\ \$\ }}\texttt{\ ..\ }{\texttt{\ \$\ }}\texttt{\ }
&
\texttt{\ }{\texttt{\ \$\ }}\texttt{\ }{\texttt{\ -\ }}\texttt{\ x\ }{\texttt{\ \$\ }}\texttt{\ is\ the\ opposite\ of\ }{\texttt{\ \$\ }}\texttt{\ x\ }{\texttt{\ \$\ }}\texttt{\ } \\
Markup & Surround markup with \texttt{\ {[}..{]}\ } &
\texttt{\ }{\texttt{\ let\ }}\texttt{\ name\ }{\texttt{\ =\ }}\texttt{\ }{\texttt{\ {[}\ }}\texttt{\ }{\texttt{\ *Typst!*\ }}\texttt{\ }{\texttt{\ {]}\ }}\texttt{\ } \\
\end{longtable}

Once you have entered code mode with \texttt{\ \#\ } , you
don\textquotesingle t need to use further hashes unless you switched
back to markup or math mode in between.

\subsection{Markup}\label{markup}

Typst provides built-in markup for the most common document elements.
Most of the syntax elements are just shortcuts for a corresponding
function. The table below lists all markup that is available and links
to the best place to learn more about their syntax and usage.

\begin{longtable}[]{@{}lll@{}}
\toprule\noalign{}
Name & Example & See \\
\midrule\noalign{}
\endhead
\bottomrule\noalign{}
\endlastfoot
Paragraph break & Blank line &
\href{/docs/reference/model/parbreak/}{\texttt{\ parbreak\ }} \\
Strong emphasis & \texttt{\ }{\texttt{\ *strong*\ }}\texttt{\ } &
\href{/docs/reference/model/strong/}{\texttt{\ strong\ }} \\
Emphasis & \texttt{\ }{\texttt{\ \_emphasis\_\ }}\texttt{\ } &
\href{/docs/reference/model/emph/}{\texttt{\ emph\ }} \\
Raw text &
\texttt{\ }{\texttt{\ \textasciigrave{}print(1)\textasciigrave{}\ }}\texttt{\ }
& \href{/docs/reference/text/raw/}{\texttt{\ raw\ }} \\
Link & \texttt{\ }{\texttt{\ https://typst.app/\ }}\texttt{\ } &
\href{/docs/reference/model/link/}{\texttt{\ link\ }} \\
Label &
\texttt{\ }{\texttt{\ \textless{}intro\textgreater{}\ }}\texttt{\ } &
\href{/docs/reference/foundations/label/}{\texttt{\ label\ }} \\
Reference & \texttt{\ }{\texttt{\ @intro\ }}\texttt{\ } &
\href{/docs/reference/model/ref/}{\texttt{\ ref\ }} \\
Heading & \texttt{\ }{\texttt{\ =\ Heading\ }}\texttt{\ } &
\href{/docs/reference/model/heading/}{\texttt{\ heading\ }} \\
Bullet list & \texttt{\ }{\texttt{\ -\ }}\texttt{\ item\ } &
\href{/docs/reference/model/list/}{\texttt{\ list\ }} \\
Numbered list & \texttt{\ }{\texttt{\ +\ }}\texttt{\ item\ } &
\href{/docs/reference/model/enum/}{\texttt{\ enum\ }} \\
Term list &
\texttt{\ }{\texttt{\ /\ }}\texttt{\ }{\texttt{\ Term\ }}\texttt{\ }{\texttt{\ :\ }}\texttt{\ description\ }
& \href{/docs/reference/model/terms/}{\texttt{\ terms\ }} \\
Math &
\texttt{\ }{\texttt{\ \$\ }}\texttt{\ x\ }{\texttt{\ \^{}\ }}\texttt{\ 2\ }{\texttt{\ \$\ }}\texttt{\ }
& \href{/docs/reference/math/}{Math} \\
Line break & \texttt{\ }{\texttt{\ \textbackslash{}\ }}\texttt{\ } &
\href{/docs/reference/text/linebreak/}{\texttt{\ linebreak\ }} \\
Smart quote &
\texttt{\ \textquotesingle{}single\textquotesingle{}\ or\ "double"\ } &
\href{/docs/reference/text/smartquote/}{\texttt{\ smartquote\ }} \\
Symbol shorthand &
\texttt{\ }{\texttt{\ \textasciitilde{}\ }}\texttt{\ } ,
\texttt{\ }{\texttt{\ -\/-\/-\ }}\texttt{\ } &
\href{/docs/reference/symbols/sym/}{Symbols} \\
Code expression &
\texttt{\ }{\texttt{\ \#\ }}\texttt{\ }{\texttt{\ rect\ }}\texttt{\ }{\texttt{\ (\ }}\texttt{\ width\ }{\texttt{\ :\ }}\texttt{\ }{\texttt{\ 1cm\ }}\texttt{\ }{\texttt{\ )\ }}\texttt{\ }
& \href{/docs/reference/scripting/\#expressions}{Scripting} \\
Character escape &
\texttt{\ Tweet\ at\ us\ }{\texttt{\ \textbackslash{}\#\ }}\texttt{\ ad\ }
& \hyperref[escapes]{Below} \\
Comment & \texttt{\ }{\texttt{\ /*\ block\ */\ }}\texttt{\ } ,
\texttt{\ }{\texttt{\ //\ line\ }}\texttt{\ } &
\hyperref[comments]{Below} \\
\end{longtable}

\subsection{Math mode}\label{math}

Math mode is a special markup mode that is used to typeset mathematical
formulas. It is entered by wrapping an equation in \texttt{\ \$\ }
characters. This works both in markup and code. The equation will be
typeset into its own block if it starts and ends with at least one space
(e.g.
\texttt{\ }{\texttt{\ \$\ }}\texttt{\ x\ }{\texttt{\ \^{}\ }}\texttt{\ 2\ }{\texttt{\ \$\ }}\texttt{\ }
). Inline math can be produced by omitting the whitespace (e.g.
\texttt{\ }{\texttt{\ \$\ }}\texttt{\ x\ }{\texttt{\ \^{}\ }}\texttt{\ 2\ }{\texttt{\ \$\ }}\texttt{\ }
). An overview over the syntax specific to math mode follows:

\begin{longtable}[]{@{}lll@{}}
\toprule\noalign{}
Name & Example & See \\
\midrule\noalign{}
\endhead
\bottomrule\noalign{}
\endlastfoot
Inline math &
\texttt{\ }{\texttt{\ \$\ }}\texttt{\ x\ }{\texttt{\ \^{}\ }}\texttt{\ 2\ }{\texttt{\ \$\ }}\texttt{\ }
& \href{/docs/reference/math/}{Math} \\
Block-level math &
\texttt{\ }{\texttt{\ \$\ }}\texttt{\ x\ }{\texttt{\ \^{}\ }}\texttt{\ 2\ }{\texttt{\ \$\ }}\texttt{\ }
& \href{/docs/reference/math/}{Math} \\
Bottom attachment &
\texttt{\ }{\texttt{\ \$\ }}\texttt{\ x\ }{\texttt{\ \_\ }}\texttt{\ 1\ }{\texttt{\ \$\ }}\texttt{\ }
& \href{/docs/reference/math/attach/}{\texttt{\ attach\ }} \\
Top attachment &
\texttt{\ }{\texttt{\ \$\ }}\texttt{\ x\ }{\texttt{\ \^{}\ }}\texttt{\ 2\ }{\texttt{\ \$\ }}\texttt{\ }
& \href{/docs/reference/math/attach/}{\texttt{\ attach\ }} \\
Fraction &
\texttt{\ }{\texttt{\ \$\ }}\texttt{\ 1\ +\ }{\texttt{\ (\ }}\texttt{\ a+b\ }{\texttt{\ )\ }}\texttt{\ }{\texttt{\ /\ }}\texttt{\ 5\ }{\texttt{\ \$\ }}\texttt{\ }
& \href{/docs/reference/math/frac/}{\texttt{\ frac\ }} \\
Line break &
\texttt{\ }{\texttt{\ \$\ }}\texttt{\ x\ }{\texttt{\ \textbackslash{}\ }}\texttt{\ y\ }{\texttt{\ \$\ }}\texttt{\ }
& \href{/docs/reference/text/linebreak/}{\texttt{\ linebreak\ }} \\
Alignment point &
\texttt{\ }{\texttt{\ \$\ }}\texttt{\ x\ }{\texttt{\ \&\ }}\texttt{\ =\ 2\ }{\texttt{\ \textbackslash{}\ }}\texttt{\ }{\texttt{\ \&\ }}\texttt{\ =\ 3\ }{\texttt{\ \$\ }}\texttt{\ }
& \href{/docs/reference/math/}{Math} \\
Variable access &
\texttt{\ }{\texttt{\ \$\ }}\texttt{\ }{\texttt{\ \#\ }}\texttt{\ }{\texttt{\ x\ }}\texttt{\ }{\texttt{\ \$\ }}\texttt{\ ,\ }{\texttt{\ \$\ }}\texttt{\ }{\texttt{\ pi\ }}\texttt{\ }{\texttt{\ \$\ }}\texttt{\ }
& \href{/docs/reference/math/}{Math} \\
Field access &
\texttt{\ }{\texttt{\ \$\ }}\texttt{\ }{\texttt{\ arrow\ }}\texttt{\ }{\texttt{\ .\ }}\texttt{\ }{\texttt{\ r\ }}\texttt{\ }{\texttt{\ .\ }}\texttt{\ }{\texttt{\ long\ }}\texttt{\ }{\texttt{\ \$\ }}\texttt{\ }
& \href{/docs/reference/scripting/\#fields}{Scripting} \\
Implied multiplication &
\texttt{\ }{\texttt{\ \$\ }}\texttt{\ x\ y\ }{\texttt{\ \$\ }}\texttt{\ }
& \href{/docs/reference/math/}{Math} \\
Symbol shorthand &
\texttt{\ }{\texttt{\ \$\ }}\texttt{\ }{\texttt{\ -\textgreater{}\ }}\texttt{\ }{\texttt{\ \$\ }}\texttt{\ }
,
\texttt{\ }{\texttt{\ \$\ }}\texttt{\ }{\texttt{\ !=\ }}\texttt{\ }{\texttt{\ \$\ }}\texttt{\ }
& \href{/docs/reference/symbols/sym/}{Symbols} \\
Text/string in math &
\texttt{\ }{\texttt{\ \$\ }}\texttt{\ a\ }{\texttt{\ "is\ natural"\ }}\texttt{\ }{\texttt{\ \$\ }}\texttt{\ }
& \href{/docs/reference/math/}{Math} \\
Math function call &
\texttt{\ }{\texttt{\ \$\ }}\texttt{\ }{\texttt{\ floor\ }}\texttt{\ }{\texttt{\ (\ }}\texttt{\ x\ }{\texttt{\ )\ }}\texttt{\ }{\texttt{\ \$\ }}\texttt{\ }
& \href{/docs/reference/math/}{Math} \\
Code expression &
\texttt{\ }{\texttt{\ \$\ }}\texttt{\ }{\texttt{\ \#\ }}\texttt{\ }{\texttt{\ rect\ }}\texttt{\ }{\texttt{\ (\ }}\texttt{\ width\ }{\texttt{\ :\ }}\texttt{\ }{\texttt{\ 1cm\ }}\texttt{\ }{\texttt{\ )\ }}\texttt{\ }{\texttt{\ \$\ }}\texttt{\ }
& \href{/docs/reference/scripting/\#expressions}{Scripting} \\
Character escape &
\texttt{\ }{\texttt{\ \$\ }}\texttt{\ x\ }{\texttt{\ \textbackslash{}\^{}\ }}\texttt{\ 2\ }{\texttt{\ \$\ }}\texttt{\ }
& \hyperref[escapes]{Below} \\
Comment &
\texttt{\ }{\texttt{\ \$\ }}\texttt{\ }{\texttt{\ /*\ comment\ */\ }}\texttt{\ }{\texttt{\ \$\ }}\texttt{\ }
& \hyperref[comments]{Below} \\
\end{longtable}

\subsection{Code mode}\label{code}

Within code blocks and expressions, new expressions can start without a
leading \texttt{\ \#\ } character. Many syntactic elements are specific
to expressions. Below is a table listing all syntax that is available in
code mode:

\begin{longtable}[]{@{}lll@{}}
\toprule\noalign{}
Name & Example & See \\
\midrule\noalign{}
\endhead
\bottomrule\noalign{}
\endlastfoot
None & \texttt{\ }{\texttt{\ none\ }}\texttt{\ } &
\href{/docs/reference/foundations/none/}{\texttt{\ none\ }} \\
Auto & \texttt{\ }{\texttt{\ auto\ }}\texttt{\ } &
\href{/docs/reference/foundations/auto/}{\texttt{\ auto\ }} \\
Boolean & \texttt{\ }{\texttt{\ false\ }}\texttt{\ } ,
\texttt{\ }{\texttt{\ true\ }}\texttt{\ } &
\href{/docs/reference/foundations/bool/}{\texttt{\ bool\ }} \\
Integer & \texttt{\ }{\texttt{\ 10\ }}\texttt{\ } ,
\texttt{\ }{\texttt{\ 0xff\ }}\texttt{\ } &
\href{/docs/reference/foundations/int/}{\texttt{\ int\ }} \\
Floating-point number & \texttt{\ }{\texttt{\ 3.14\ }}\texttt{\ } ,
\texttt{\ }{\texttt{\ 1e5\ }}\texttt{\ } &
\href{/docs/reference/foundations/float/}{\texttt{\ float\ }} \\
Length & \texttt{\ }{\texttt{\ 2pt\ }}\texttt{\ } ,
\texttt{\ }{\texttt{\ 3mm\ }}\texttt{\ } ,
\texttt{\ }{\texttt{\ 1em\ }}\texttt{\ } , .. &
\href{/docs/reference/layout/length/}{\texttt{\ length\ }} \\
Angle & \texttt{\ }{\texttt{\ 90deg\ }}\texttt{\ } ,
\texttt{\ }{\texttt{\ 1rad\ }}\texttt{\ } &
\href{/docs/reference/layout/angle/}{\texttt{\ angle\ }} \\
Fraction & \texttt{\ }{\texttt{\ 2fr\ }}\texttt{\ } &
\href{/docs/reference/layout/fraction/}{\texttt{\ fraction\ }} \\
Ratio & \texttt{\ }{\texttt{\ 50\%\ }}\texttt{\ } &
\href{/docs/reference/layout/ratio/}{\texttt{\ ratio\ }} \\
String & \texttt{\ }{\texttt{\ "hello"\ }}\texttt{\ } &
\href{/docs/reference/foundations/str/}{\texttt{\ str\ }} \\
Label &
\texttt{\ }{\texttt{\ \textless{}intro\textgreater{}\ }}\texttt{\ } &
\href{/docs/reference/foundations/label/}{\texttt{\ label\ }} \\
Math &
\texttt{\ }{\texttt{\ \$\ }}\texttt{\ x\ }{\texttt{\ \^{}\ }}\texttt{\ 2\ }{\texttt{\ \$\ }}\texttt{\ }
& \href{/docs/reference/math/}{Math} \\
Raw text &
\texttt{\ }{\texttt{\ \textasciigrave{}print(1)\textasciigrave{}\ }}\texttt{\ }
& \href{/docs/reference/text/raw/}{\texttt{\ raw\ }} \\
Variable access & \texttt{\ x\ } &
\href{/docs/reference/scripting/\#blocks}{Scripting} \\
Code block &
\texttt{\ }{\texttt{\ \{\ }}\texttt{\ }{\texttt{\ let\ }}\texttt{\ x\ }{\texttt{\ =\ }}\texttt{\ }{\texttt{\ 1\ }}\texttt{\ }{\texttt{\ ;\ }}\texttt{\ x\ }{\texttt{\ +\ }}\texttt{\ }{\texttt{\ 2\ }}\texttt{\ }{\texttt{\ \}\ }}\texttt{\ }
& \href{/docs/reference/scripting/\#blocks}{Scripting} \\
Content block &
\texttt{\ }{\texttt{\ {[}\ }}\texttt{\ }{\texttt{\ *Hello*\ }}\texttt{\ }{\texttt{\ {]}\ }}\texttt{\ }
& \href{/docs/reference/scripting/\#blocks}{Scripting} \\
Parenthesized expression &
\texttt{\ }{\texttt{\ (\ }}\texttt{\ }{\texttt{\ 1\ }}\texttt{\ }{\texttt{\ +\ }}\texttt{\ }{\texttt{\ 2\ }}\texttt{\ }{\texttt{\ )\ }}\texttt{\ }
& \href{/docs/reference/scripting/\#blocks}{Scripting} \\
Array &
\texttt{\ }{\texttt{\ (\ }}\texttt{\ }{\texttt{\ 1\ }}\texttt{\ }{\texttt{\ ,\ }}\texttt{\ }{\texttt{\ 2\ }}\texttt{\ }{\texttt{\ ,\ }}\texttt{\ }{\texttt{\ 3\ }}\texttt{\ }{\texttt{\ )\ }}\texttt{\ }
& \href{/docs/reference/foundations/array/}{Array} \\
Dictionary &
\texttt{\ }{\texttt{\ (\ }}\texttt{\ a\ }{\texttt{\ :\ }}\texttt{\ }{\texttt{\ "hi"\ }}\texttt{\ }{\texttt{\ ,\ }}\texttt{\ b\ }{\texttt{\ :\ }}\texttt{\ }{\texttt{\ 2\ }}\texttt{\ }{\texttt{\ )\ }}\texttt{\ }
& \href{/docs/reference/foundations/dictionary/}{Dictionary} \\
Unary operator & \texttt{\ }{\texttt{\ -\ }}\texttt{\ x\ } &
\href{/docs/reference/scripting/\#operators}{Scripting} \\
Binary operator & \texttt{\ x\ }{\texttt{\ +\ }}\texttt{\ y\ } &
\href{/docs/reference/scripting/\#operators}{Scripting} \\
Assignment &
\texttt{\ x\ }{\texttt{\ =\ }}\texttt{\ }{\texttt{\ 1\ }}\texttt{\ } &
\href{/docs/reference/scripting/\#operators}{Scripting} \\
Field access & \texttt{\ x\ }{\texttt{\ .\ }}\texttt{\ y\ } &
\href{/docs/reference/scripting/\#fields}{Scripting} \\
Method call &
\texttt{\ x\ }{\texttt{\ .\ }}\texttt{\ }{\texttt{\ flatten\ }}\texttt{\ }{\texttt{\ (\ }}\texttt{\ }{\texttt{\ )\ }}\texttt{\ }
& \href{/docs/reference/scripting/\#methods}{Scripting} \\
Function call &
\texttt{\ }{\texttt{\ min\ }}\texttt{\ }{\texttt{\ (\ }}\texttt{\ x\ }{\texttt{\ ,\ }}\texttt{\ y\ }{\texttt{\ )\ }}\texttt{\ }
& \href{/docs/reference/foundations/function/}{Function} \\
Argument spreading &
\texttt{\ }{\texttt{\ min\ }}\texttt{\ }{\texttt{\ (\ }}\texttt{\ }{\texttt{\ ..\ }}\texttt{\ nums\ }{\texttt{\ )\ }}\texttt{\ }
& \href{/docs/reference/foundations/arguments/}{Arguments} \\
Unnamed function &
\texttt{\ }{\texttt{\ (\ }}\texttt{\ x\ }{\texttt{\ ,\ }}\texttt{\ y\ }{\texttt{\ )\ }}\texttt{\ }{\texttt{\ =\textgreater{}\ }}\texttt{\ x\ }{\texttt{\ +\ }}\texttt{\ y\ }
& \href{/docs/reference/foundations/function/}{Function} \\
Let binding &
\texttt{\ }{\texttt{\ let\ }}\texttt{\ x\ }{\texttt{\ =\ }}\texttt{\ }{\texttt{\ 1\ }}\texttt{\ }
& \href{/docs/reference/scripting/\#bindings}{Scripting} \\
Named function &
\texttt{\ }{\texttt{\ let\ }}\texttt{\ }{\texttt{\ f\ }}\texttt{\ }{\texttt{\ (\ }}\texttt{\ x\ }{\texttt{\ )\ }}\texttt{\ }{\texttt{\ =\ }}\texttt{\ }{\texttt{\ 2\ }}\texttt{\ }{\texttt{\ *\ }}\texttt{\ x\ }
& \href{/docs/reference/foundations/function/}{Function} \\
Set rule &
\texttt{\ }{\texttt{\ set\ }}\texttt{\ }{\texttt{\ text\ }}\texttt{\ }{\texttt{\ (\ }}\texttt{\ }{\texttt{\ 14pt\ }}\texttt{\ }{\texttt{\ )\ }}\texttt{\ }
& \href{/docs/reference/styling/\#set-rules}{Styling} \\
Set-if rule &
\texttt{\ }{\texttt{\ set\ }}\texttt{\ }{\texttt{\ text\ }}\texttt{\ }{\texttt{\ (\ }}\texttt{\ }{\texttt{\ ..\ }}\texttt{\ }{\texttt{\ )\ }}\texttt{\ }{\texttt{\ if\ }}\texttt{\ ..\ }
& \href{/docs/reference/styling/\#set-rules}{Styling} \\
Show-set rule &
\texttt{\ }{\texttt{\ show\ }}\texttt{\ }{\texttt{\ heading\ }}\texttt{\ }{\texttt{\ :\ }}\texttt{\ }{\texttt{\ set\ }}\texttt{\ }{\texttt{\ block\ }}\texttt{\ }{\texttt{\ (\ }}\texttt{\ }{\texttt{\ ..\ }}\texttt{\ }{\texttt{\ )\ }}\texttt{\ }
& \href{/docs/reference/styling/\#show-rules}{Styling} \\
Show rule with function &
\texttt{\ }{\texttt{\ show\ }}\texttt{\ }{\texttt{\ raw\ }}\texttt{\ }{\texttt{\ :\ }}\texttt{\ it\ }{\texttt{\ =\textgreater{}\ }}\texttt{\ }{\texttt{\ \{\ }}\texttt{\ ..\ }{\texttt{\ \}\ }}\texttt{\ }
& \href{/docs/reference/styling/\#show-rules}{Styling} \\
Show-everything rule &
\texttt{\ }{\texttt{\ show\ }}\texttt{\ }{\texttt{\ :\ }}\texttt{\ }{\texttt{\ template\ }}\texttt{\ }
& \href{/docs/reference/styling/\#show-rules}{Styling} \\
Context expression &
\texttt{\ }{\texttt{\ context\ }}\texttt{\ text\ }{\texttt{\ .\ }}\texttt{\ lang\ }
& \href{/docs/reference/context/}{Context} \\
Conditional &
\texttt{\ }{\texttt{\ if\ }}\texttt{\ x\ }{\texttt{\ ==\ }}\texttt{\ }{\texttt{\ 1\ }}\texttt{\ }{\texttt{\ \{\ }}\texttt{\ ..\ }{\texttt{\ \}\ }}\texttt{\ }{\texttt{\ else\ }}\texttt{\ }{\texttt{\ \{\ }}\texttt{\ ..\ }{\texttt{\ \}\ }}\texttt{\ }
& \href{/docs/reference/scripting/\#conditionals}{Scripting} \\
For loop &
\texttt{\ }{\texttt{\ for\ }}\texttt{\ x\ }{\texttt{\ in\ }}\texttt{\ }{\texttt{\ (\ }}\texttt{\ }{\texttt{\ 1\ }}\texttt{\ }{\texttt{\ ,\ }}\texttt{\ }{\texttt{\ 2\ }}\texttt{\ }{\texttt{\ ,\ }}\texttt{\ }{\texttt{\ 3\ }}\texttt{\ }{\texttt{\ )\ }}\texttt{\ }{\texttt{\ \{\ }}\texttt{\ ..\ }{\texttt{\ \}\ }}\texttt{\ }
& \href{/docs/reference/scripting/\#loops}{Scripting} \\
While loop &
\texttt{\ }{\texttt{\ while\ }}\texttt{\ x\ }{\texttt{\ \textless{}\ }}\texttt{\ }{\texttt{\ 10\ }}\texttt{\ }{\texttt{\ \{\ }}\texttt{\ ..\ }{\texttt{\ \}\ }}\texttt{\ }
& \href{/docs/reference/scripting/\#loops}{Scripting} \\
Loop control flow &
\texttt{\ }{\texttt{\ break\ }}\texttt{\ ,\ }{\texttt{\ continue\ }}\texttt{\ }
& \href{/docs/reference/scripting/\#loops}{Scripting} \\
Return from function & \texttt{\ }{\texttt{\ return\ }}\texttt{\ x\ } &
\href{/docs/reference/foundations/function/}{Function} \\
Include module &
\texttt{\ }{\texttt{\ include\ }}\texttt{\ }{\texttt{\ "bar.typ"\ }}\texttt{\ }
& \href{/docs/reference/scripting/\#modules}{Scripting} \\
Import module &
\texttt{\ }{\texttt{\ import\ }}\texttt{\ }{\texttt{\ "bar.typ"\ }}\texttt{\ }
& \href{/docs/reference/scripting/\#modules}{Scripting} \\
Import items from module &
\texttt{\ }{\texttt{\ import\ }}\texttt{\ }{\texttt{\ "bar.typ"\ }}\texttt{\ }{\texttt{\ :\ }}\texttt{\ a\ }{\texttt{\ ,\ }}\texttt{\ b\ }{\texttt{\ ,\ }}\texttt{\ c\ }
& \href{/docs/reference/scripting/\#modules}{Scripting} \\
Comment & \texttt{\ }{\texttt{\ /*\ block\ */\ }}\texttt{\ } ,
\texttt{\ }{\texttt{\ //\ line\ }}\texttt{\ } &
\hyperref[comments]{Below} \\
\end{longtable}

\subsection{Comments}\label{comments}

Comments are ignored by Typst and will not be included in the output.
This is useful to exclude old versions or to add annotations. To comment
out a single line, start it with \texttt{\ //\ } :

\begin{verbatim}
// our data barely supports
// this claim

We show with $p < 0.05$
that the difference is
significant.
\end{verbatim}

\includegraphics[width=5in,height=\textheight,keepaspectratio]{/assets/docs/qmPJyf2DgB8m9bpdDccxUQAAAAAAAAAA.png}

Comments can also be wrapped between \texttt{\ /*\ } and \texttt{\ */\ }
. In this case, the comment can span over multiple lines:

\begin{verbatim}
Our study design is as follows:
/* Somebody write this up:
   - 1000 participants.
   - 2x2 data design. */
\end{verbatim}

\includegraphics[width=5in,height=\textheight,keepaspectratio]{/assets/docs/0bd3Pt_MGVIAagJ8npuMMAAAAAAAAAAA.png}

\subsection{Escape sequences}\label{escapes}

Escape sequences are used to insert special characters that are hard to
type or otherwise have special meaning in Typst. To escape a character,
precede it with a backslash. To insert any Unicode codepoint, you can
write a hexadecimal escape sequence:
\texttt{\ }{\texttt{\ \textbackslash{}u\{1f600\}\ }}\texttt{\ } . The
same kind of escape sequences also work in
\href{/docs/reference/foundations/str/}{strings} .

\begin{verbatim}
I got an ice cream for
\$1.50! \u{1f600}
\end{verbatim}

\includegraphics[width=5in,height=\textheight,keepaspectratio]{/assets/docs/2Hq1wVq0JUPd4EarGtBZUQAAAAAAAAAA.png}

\subsection{Paths}\label{paths}

Typst has various features that require a file path to reference
external resources such as images, Typst files, or data files. Paths are
represented as \href{/docs/reference/foundations/str/}{strings} . There
are two kinds of paths: Relative and absolute.

\begin{itemize}
\item
  A \textbf{relative path} searches from the location of the Typst file
  where the feature is invoked. It is the default:

\begin{verbatim}
#image("images/logo.png")
\end{verbatim}
\item
  An \textbf{absolute path} searches from the \emph{root} of the
  project. It starts with a leading \texttt{\ /\ } :

\begin{verbatim}
#image("/assets/logo.png")
\end{verbatim}
\end{itemize}

\subsubsection{Project root}\label{project-root}

By default, the project root is the parent directory of the main Typst
file. For security reasons, you cannot read any files outside of the
root directory.

If you want to set a specific folder as the root of your project, you
can use the CLI\textquotesingle s \texttt{\ -\/-root\ } flag. Make sure
that the main file is contained in the folder\textquotesingle s subtree!

\begin{verbatim}
typst compile --root .. file.typ
\end{verbatim}

In the web app, the project itself is the root directory. You can always
read all files within it, no matter which one is previewed (via the eye
toggle next to each Typst file in the file panel).

\subsubsection{Paths and packages}\label{paths-and-packages}

A package can only load files from its own directory. Within it,
absolute paths point to the package root, rather than the project root.
For this reason, it cannot directly load files from the project
directory. If a package needs resources from the project (such as a logo
image), you must pass the already loaded image, e.g. as a named
parameter
\texttt{\ logo:\ }{\texttt{\ image\ }}\texttt{\ }{\texttt{\ (\ }}\texttt{\ }{\texttt{\ "mylogo.svg"\ }}\texttt{\ }{\texttt{\ )\ }}\texttt{\ }
. Note that you can then still customize the image\textquotesingle s
appearance with a set rule within the package.

In the future, paths might become a
\href{https://github.com/typst/typst/issues/971}{distinct type from
strings} , so that they can retain knowledge of where they were
constructed. This way, resources could be loaded from a different root.

\href{/docs/reference/}{\pandocbounded{\includesvg[keepaspectratio]{/assets/icons/16-arrow-right.svg}}}

{ Reference } { Previous page }

\href{/docs/reference/styling/}{\pandocbounded{\includesvg[keepaspectratio]{/assets/icons/16-arrow-right.svg}}}

{ Styling } { Next page }
