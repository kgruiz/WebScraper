\title{typst.app/docs/reference/model/outline}

\begin{itemize}
\tightlist
\item
  \href{/docs}{\includesvg[width=0.16667in,height=0.16667in]{/assets/icons/16-docs-dark.svg}}
\item
  \includesvg[width=0.16667in,height=0.16667in]{/assets/icons/16-arrow-right.svg}
\item
  \href{/docs/reference/}{Reference}
\item
  \includesvg[width=0.16667in,height=0.16667in]{/assets/icons/16-arrow-right.svg}
\item
  \href{/docs/reference/model/}{Model}
\item
  \includesvg[width=0.16667in,height=0.16667in]{/assets/icons/16-arrow-right.svg}
\item
  \href{/docs/reference/model/outline/}{Outline}
\end{itemize}

\section{\texorpdfstring{\texttt{\ outline\ } {{ Element
}}}{ outline   Element }}\label{summary}

\phantomsection\label{element-tooltip}
Element functions can be customized with \texttt{\ set\ } and
\texttt{\ show\ } rules.

A table of contents, figures, or other elements.

This function generates a list of all occurrences of an element in the
document, up to a given depth. The element\textquotesingle s numbering
and page number will be displayed in the outline alongside its title or
caption. By default this generates a table of contents.

\subsection{Example}\label{example}

\begin{verbatim}
#outline()

= Introduction
#lorem(5)

= Prior work
#lorem(10)
\end{verbatim}

\includegraphics[width=5in,height=\textheight,keepaspectratio]{/assets/docs/pxzEoLgfS9GjzIb6I2LlEgAAAAAAAAAA.png}

\subsection{Alternative outlines}\label{alternative-outlines}

By setting the \texttt{\ target\ } parameter, the outline can be used to
generate a list of other kinds of elements than headings. In the example
below, we list all figures containing images by setting
\texttt{\ target\ } to
\texttt{\ figure\ }{\texttt{\ .\ }}\texttt{\ }{\texttt{\ where\ }}\texttt{\ }{\texttt{\ (\ }}\texttt{\ kind\ }{\texttt{\ :\ }}\texttt{\ image\ }{\texttt{\ )\ }}\texttt{\ }
. We could have also set it to just \texttt{\ figure\ } , but then the
list would also include figures containing tables or other material. For
more details on the \texttt{\ where\ } selector,
\href{/docs/reference/foundations/function/\#definitions-where}{see
here} .

\begin{verbatim}
#outline(
  title: [List of Figures],
  target: figure.where(kind: image),
)

#figure(
  image("tiger.jpg"),
  caption: [A nice figure!],
)
\end{verbatim}

\includegraphics[width=5in,height=\textheight,keepaspectratio]{/assets/docs/K0Fgir_M6IbOnlxFTpRoyAAAAAAAAAAA.png}

\subsection{Styling the outline}\label{styling-the-outline}

The outline element has several options for customization, such as its
\texttt{\ title\ } and \texttt{\ indent\ } parameters. If desired,
however, it is possible to have more control over the
outline\textquotesingle s look and style through the
\href{/docs/reference/model/outline/\#definitions-entry}{\texttt{\ outline.entry\ }}
element.

\subsection{\texorpdfstring{{ Parameters
}}{ Parameters }}\label{parameters}

\phantomsection\label{parameters-tooltip}
Parameters are the inputs to a function. They are specified in
parentheses after the function name.

{ outline } (

{ \hyperref[parameters-title]{title :}
\href{/docs/reference/foundations/none/}{none}
\href{/docs/reference/foundations/auto/}{auto}
\href{/docs/reference/foundations/content/}{content} , } {
\hyperref[parameters-target]{target :}
\href{/docs/reference/foundations/label/}{label}
\href{/docs/reference/foundations/selector/}{selector}
\href{/docs/reference/introspection/location/}{location}
\href{/docs/reference/foundations/function/}{function} , } {
\hyperref[parameters-depth]{depth :}
\href{/docs/reference/foundations/none/}{none}
\href{/docs/reference/foundations/int/}{int} , } {
\hyperref[parameters-indent]{indent :}
\href{/docs/reference/foundations/none/}{none}
\href{/docs/reference/foundations/auto/}{auto}
\href{/docs/reference/foundations/bool/}{bool}
\href{/docs/reference/layout/relative/}{relative}
\href{/docs/reference/foundations/function/}{function} , } {
\hyperref[parameters-fill]{fill :}
\href{/docs/reference/foundations/none/}{none}
\href{/docs/reference/foundations/content/}{content} , }

) -\textgreater{} \href{/docs/reference/foundations/content/}{content}

\subsubsection{\texorpdfstring{\texttt{\ title\ }}{ title }}\label{parameters-title}

\href{/docs/reference/foundations/none/}{none} {or}
\href{/docs/reference/foundations/auto/}{auto} {or}
\href{/docs/reference/foundations/content/}{content}

{{ Settable }}

\phantomsection\label{parameters-title-settable-tooltip}
Settable parameters can be customized for all following uses of the
function with a \texttt{\ set\ } rule.

The title of the outline.

\begin{itemize}
\tightlist
\item
  When set to \texttt{\ }{\texttt{\ auto\ }}\texttt{\ } , an appropriate
  title for the \href{/docs/reference/text/text/\#parameters-lang}{text
  language} will be used. This is the default.
\item
  When set to \texttt{\ }{\texttt{\ none\ }}\texttt{\ } , the outline
  will not have a title.
\item
  A custom title can be set by passing content.
\end{itemize}

The outline\textquotesingle s heading will not be numbered by default,
but you can force it to be with a show-set rule:
\texttt{\ }{\texttt{\ show\ }}\texttt{\ }{\texttt{\ outline\ }}\texttt{\ }{\texttt{\ :\ }}\texttt{\ }{\texttt{\ set\ }}\texttt{\ }{\texttt{\ heading\ }}\texttt{\ }{\texttt{\ (\ }}\texttt{\ numbering\ }{\texttt{\ :\ }}\texttt{\ }{\texttt{\ "1."\ }}\texttt{\ }{\texttt{\ )\ }}\texttt{\ }

Default: \texttt{\ }{\texttt{\ auto\ }}\texttt{\ }

\subsubsection{\texorpdfstring{\texttt{\ target\ }}{ target }}\label{parameters-target}

\href{/docs/reference/foundations/label/}{label} {or}
\href{/docs/reference/foundations/selector/}{selector} {or}
\href{/docs/reference/introspection/location/}{location} {or}
\href{/docs/reference/foundations/function/}{function}

{{ Settable }}

\phantomsection\label{parameters-target-settable-tooltip}
Settable parameters can be customized for all following uses of the
function with a \texttt{\ set\ } rule.

The type of element to include in the outline.

To list figures containing a specific kind of element, like a table, you
can write
\texttt{\ figure\ }{\texttt{\ .\ }}\texttt{\ }{\texttt{\ where\ }}\texttt{\ }{\texttt{\ (\ }}\texttt{\ kind\ }{\texttt{\ :\ }}\texttt{\ table\ }{\texttt{\ )\ }}\texttt{\ }
.

Default:
\texttt{\ heading\ }{\texttt{\ .\ }}\texttt{\ }{\texttt{\ where\ }}\texttt{\ }{\texttt{\ (\ }}\texttt{\ outlined\ }{\texttt{\ :\ }}\texttt{\ }{\texttt{\ true\ }}\texttt{\ }{\texttt{\ )\ }}\texttt{\ }

\includesvg[width=0.16667in,height=0.16667in]{/assets/icons/16-arrow-right.svg}
View example

\begin{verbatim}
#outline(
  title: [List of Tables],
  target: figure.where(kind: table),
)

#figure(
  table(
    columns: 4,
    [t], [1], [2], [3],
    [y], [0.3], [0.7], [0.5],
  ),
  caption: [Experiment results],
)
\end{verbatim}

\includegraphics[width=5in,height=\textheight,keepaspectratio]{/assets/docs/9oD_YO_3aaN85cAixeBP2gAAAAAAAAAA.png}

\subsubsection{\texorpdfstring{\texttt{\ depth\ }}{ depth }}\label{parameters-depth}

\href{/docs/reference/foundations/none/}{none} {or}
\href{/docs/reference/foundations/int/}{int}

{{ Settable }}

\phantomsection\label{parameters-depth-settable-tooltip}
Settable parameters can be customized for all following uses of the
function with a \texttt{\ set\ } rule.

The maximum level up to which elements are included in the outline. When
this argument is \texttt{\ }{\texttt{\ none\ }}\texttt{\ } , all
elements are included.

Default: \texttt{\ }{\texttt{\ none\ }}\texttt{\ }

\includesvg[width=0.16667in,height=0.16667in]{/assets/icons/16-arrow-right.svg}
View example

\begin{verbatim}
#set heading(numbering: "1.")
#outline(depth: 2)

= Yes
Top-level section.

== Still
Subsection.

=== Nope
Not included.
\end{verbatim}

\includegraphics[width=5in,height=\textheight,keepaspectratio]{/assets/docs/fYEfgTUmkbH0skbcMKeSFwAAAAAAAAAA.png}

\subsubsection{\texorpdfstring{\texttt{\ indent\ }}{ indent }}\label{parameters-indent}

\href{/docs/reference/foundations/none/}{none} {or}
\href{/docs/reference/foundations/auto/}{auto} {or}
\href{/docs/reference/foundations/bool/}{bool} {or}
\href{/docs/reference/layout/relative/}{relative} {or}
\href{/docs/reference/foundations/function/}{function}

{{ Settable }}

\phantomsection\label{parameters-indent-settable-tooltip}
Settable parameters can be customized for all following uses of the
function with a \texttt{\ set\ } rule.

How to indent the outline\textquotesingle s entries.

\begin{itemize}
\tightlist
\item
  \texttt{\ }{\texttt{\ none\ }}\texttt{\ } : No indent
\item
  \texttt{\ }{\texttt{\ auto\ }}\texttt{\ } : Indents the numbering of
  the nested entry with the title of its parent entry. This only has an
  effect if the entries are numbered (e.g., via
  \href{/docs/reference/model/heading/\#parameters-numbering}{heading
  numbering} ).
\item
  \href{/docs/reference/layout/relative/}{Relative length} : Indents the
  item by this length multiplied by its nesting level. Specifying
  \texttt{\ }{\texttt{\ 2em\ }}\texttt{\ } , for instance, would indent
  top-level headings (not nested) by
  \texttt{\ }{\texttt{\ 0em\ }}\texttt{\ } , second level headings by
  \texttt{\ }{\texttt{\ 2em\ }}\texttt{\ } (nested once), third-level
  headings by \texttt{\ }{\texttt{\ 4em\ }}\texttt{\ } (nested twice)
  and so on.
\item
  \href{/docs/reference/foundations/function/}{Function} : You can
  completely customize this setting with a function. That function
  receives the nesting level as a parameter (starting at 0 for top-level
  headings/elements) and can return a relative length or content making
  up the indent. For example,
  \texttt{\ n\ }{\texttt{\ =\textgreater{}\ }}\texttt{\ n\ }{\texttt{\ *\ }}\texttt{\ }{\texttt{\ 2em\ }}\texttt{\ }
  would be equivalent to just specifying
  \texttt{\ }{\texttt{\ 2em\ }}\texttt{\ } , while
  \texttt{\ n\ }{\texttt{\ =\textgreater{}\ }}\texttt{\ }{\texttt{\ {[}\ }}\texttt{\ →\ }{\texttt{\ {]}\ }}\texttt{\ }{\texttt{\ *\ }}\texttt{\ n\ }
  would indent with one arrow per nesting level.
\end{itemize}

\emph{Migration hints:} Specifying
\texttt{\ }{\texttt{\ true\ }}\texttt{\ } (equivalent to
\texttt{\ }{\texttt{\ auto\ }}\texttt{\ } ) or
\texttt{\ }{\texttt{\ false\ }}\texttt{\ } (equivalent to
\texttt{\ }{\texttt{\ none\ }}\texttt{\ } ) for this option is
deprecated and will be removed in a future release.

Default: \texttt{\ }{\texttt{\ none\ }}\texttt{\ }

\includesvg[width=0.16667in,height=0.16667in]{/assets/icons/16-arrow-right.svg}
View example

\begin{verbatim}
#set heading(numbering: "1.a.")

#outline(
  title: [Contents (Automatic)],
  indent: auto,
)

#outline(
  title: [Contents (Length)],
  indent: 2em,
)

#outline(
  title: [Contents (Function)],
  indent: n => [→ ] * n,
)

= About ACME Corp.
== History
=== Origins
#lorem(10)

== Products
#lorem(10)
\end{verbatim}

\includegraphics[width=5in,height=\textheight,keepaspectratio]{/assets/docs/VxzAmxCU1uGgVW2hebfhtwAAAAAAAAAA.png}

\subsubsection{\texorpdfstring{\texttt{\ fill\ }}{ fill }}\label{parameters-fill}

\href{/docs/reference/foundations/none/}{none} {or}
\href{/docs/reference/foundations/content/}{content}

{{ Settable }}

\phantomsection\label{parameters-fill-settable-tooltip}
Settable parameters can be customized for all following uses of the
function with a \texttt{\ set\ } rule.

Content to fill the space between the title and the page number. Can be
set to \texttt{\ }{\texttt{\ none\ }}\texttt{\ } to disable filling.

Default:
\texttt{\ }{\texttt{\ repeat\ }}\texttt{\ }{\texttt{\ (\ }}\texttt{\ body\ }{\texttt{\ :\ }}\texttt{\ }{\texttt{\ {[}\ }}\texttt{\ .\ }{\texttt{\ {]}\ }}\texttt{\ }{\texttt{\ )\ }}\texttt{\ }

\includesvg[width=0.16667in,height=0.16667in]{/assets/icons/16-arrow-right.svg}
View example

\begin{verbatim}
#outline(fill: line(length: 100%))

= A New Beginning
\end{verbatim}

\includegraphics[width=5in,height=\textheight,keepaspectratio]{/assets/docs/KQmhOQJ1ylUUEeut6OI0rQAAAAAAAAAA.png}

\subsection{\texorpdfstring{{ Definitions
}}{ Definitions }}\label{definitions}

\phantomsection\label{definitions-tooltip}
Functions and types and can have associated definitions. These are
accessed by specifying the function or type, followed by a period, and
then the definition\textquotesingle s name.

\subsubsection{\texorpdfstring{\texttt{\ entry\ } {{ Element
}}}{ entry   Element }}\label{definitions-entry}

\phantomsection\label{definitions-entry-element-tooltip}
Element functions can be customized with \texttt{\ set\ } and
\texttt{\ show\ } rules.

Represents each entry line in an outline, including the reference to the
outlined element, its page number, and the filler content between both.

This element is intended for use with show rules to control the
appearance of outlines. To customize an entry\textquotesingle s line,
you can build it from scratch by accessing the \texttt{\ level\ } ,
\texttt{\ element\ } , \texttt{\ body\ } , \texttt{\ fill\ } and
\texttt{\ page\ } fields on the entry.

outline { . } { entry } (

{ \href{/docs/reference/foundations/int/}{int} , } {
\href{/docs/reference/foundations/content/}{content} , } {
\href{/docs/reference/foundations/content/}{content} , } {
\href{/docs/reference/foundations/none/}{none}
\href{/docs/reference/foundations/content/}{content} , } {
\href{/docs/reference/foundations/content/}{content} , }

) -\textgreater{} \href{/docs/reference/foundations/content/}{content}

\begin{verbatim}
#set heading(numbering: "1.")

#show outline.entry.where(
  level: 1
): it => {
  v(12pt, weak: true)
  strong(it)
}

#outline(indent: auto)

= Introduction
= Background
== History
== State of the Art
= Analysis
== Setup
\end{verbatim}

\includegraphics[width=5in,height=\textheight,keepaspectratio]{/assets/docs/z5yX2QHZa1YP1epncxVx1wAAAAAAAAAA.png}

\paragraph{\texorpdfstring{\texttt{\ level\ }}{ level }}\label{definitions-entry-level}

\href{/docs/reference/foundations/int/}{int}

{Required} {{ Positional }}

\phantomsection\label{definitions-entry-level-positional-tooltip}
Positional parameters are specified in order, without names.

The nesting level of this outline entry. Starts at
\texttt{\ }{\texttt{\ 1\ }}\texttt{\ } for top-level entries.

\paragraph{\texorpdfstring{\texttt{\ element\ }}{ element }}\label{definitions-entry-element}

\href{/docs/reference/foundations/content/}{content}

{Required} {{ Positional }}

\phantomsection\label{definitions-entry-element-positional-tooltip}
Positional parameters are specified in order, without names.

The element this entry refers to. Its location will be available through
the
\href{/docs/reference/foundations/content/\#definitions-location}{\texttt{\ location\ }}
method on content and can be \href{/docs/reference/model/link/}{linked}
to.

\paragraph{\texorpdfstring{\texttt{\ body\ }}{ body }}\label{definitions-entry-body}

\href{/docs/reference/foundations/content/}{content}

{Required} {{ Positional }}

\phantomsection\label{definitions-entry-body-positional-tooltip}
Positional parameters are specified in order, without names.

The content which is displayed in place of the referred element at its
entry in the outline. For a heading, this would be its number followed
by the heading\textquotesingle s title, for example.

\paragraph{\texorpdfstring{\texttt{\ fill\ }}{ fill }}\label{definitions-entry-fill}

\href{/docs/reference/foundations/none/}{none} {or}
\href{/docs/reference/foundations/content/}{content}

{Required} {{ Positional }}

\phantomsection\label{definitions-entry-fill-positional-tooltip}
Positional parameters are specified in order, without names.

The content used to fill the space between the element\textquotesingle s
outline and its page number, as defined by the outline element this
entry is located in. When \texttt{\ }{\texttt{\ none\ }}\texttt{\ } ,
empty space is inserted in that gap instead.

Note that, when using show rules to override outline entries, it is
recommended to wrap the filling content in a
\href{/docs/reference/layout/box/}{\texttt{\ box\ }} with fractional
width. For example,
\texttt{\ }{\texttt{\ box\ }}\texttt{\ }{\texttt{\ (\ }}\texttt{\ width\ }{\texttt{\ :\ }}\texttt{\ }{\texttt{\ 1fr\ }}\texttt{\ }{\texttt{\ ,\ }}\texttt{\ }{\texttt{\ repeat\ }}\texttt{\ }{\texttt{\ {[}\ }}\texttt{\ -\ }{\texttt{\ {]}\ }}\texttt{\ }{\texttt{\ )\ }}\texttt{\ }
would show precisely as many \texttt{\ -\ } characters as necessary to
fill a particular gap.

\paragraph{\texorpdfstring{\texttt{\ page\ }}{ page }}\label{definitions-entry-page}

\href{/docs/reference/foundations/content/}{content}

{Required} {{ Positional }}

\phantomsection\label{definitions-entry-page-positional-tooltip}
Positional parameters are specified in order, without names.

The page number of the element this entry links to, formatted with the
numbering set for the referenced page.

\href{/docs/reference/model/numbering/}{\pandocbounded{\includesvg[keepaspectratio]{/assets/icons/16-arrow-right.svg}}}

{ Numbering } { Previous page }

\href{/docs/reference/model/par/}{\pandocbounded{\includesvg[keepaspectratio]{/assets/icons/16-arrow-right.svg}}}

{ Paragraph } { Next page }
