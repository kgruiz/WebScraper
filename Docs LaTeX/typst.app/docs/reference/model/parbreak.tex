\title{typst.app/docs/reference/model/parbreak}

\begin{itemize}
\tightlist
\item
  \href{/docs}{\includesvg[width=0.16667in,height=0.16667in]{/assets/icons/16-docs-dark.svg}}
\item
  \includesvg[width=0.16667in,height=0.16667in]{/assets/icons/16-arrow-right.svg}
\item
  \href{/docs/reference/}{Reference}
\item
  \includesvg[width=0.16667in,height=0.16667in]{/assets/icons/16-arrow-right.svg}
\item
  \href{/docs/reference/model/}{Model}
\item
  \includesvg[width=0.16667in,height=0.16667in]{/assets/icons/16-arrow-right.svg}
\item
  \href{/docs/reference/model/parbreak/}{Paragraph Break}
\end{itemize}

\section{\texorpdfstring{\texttt{\ parbreak\ } {{ Element
}}}{ parbreak   Element }}\label{summary}

\phantomsection\label{element-tooltip}
Element functions can be customized with \texttt{\ set\ } and
\texttt{\ show\ } rules.

A paragraph break.

This starts a new paragraph. Especially useful when used within code
like \href{/docs/reference/scripting/\#loops}{for loops} . Multiple
consecutive paragraph breaks collapse into a single one.

\subsection{Example}\label{example}

\begin{verbatim}
#for i in range(3) {
  [Blind text #i: ]
  lorem(5)
  parbreak()
}
\end{verbatim}

\includegraphics[width=5in,height=\textheight,keepaspectratio]{/assets/docs/Ugn0Cpe50EHdh4tXrmb4YAAAAAAAAAAA.png}

\subsection{Syntax}\label{syntax}

Instead of calling this function, you can insert a blank line into your
markup to create a paragraph break.

\subsection{\texorpdfstring{{ Parameters
}}{ Parameters }}\label{parameters}

\phantomsection\label{parameters-tooltip}
Parameters are the inputs to a function. They are specified in
parentheses after the function name.

{ parbreak } (

) -\textgreater{} \href{/docs/reference/foundations/content/}{content}

\href{/docs/reference/model/par/}{\pandocbounded{\includesvg[keepaspectratio]{/assets/icons/16-arrow-right.svg}}}

{ Paragraph } { Previous page }

\href{/docs/reference/model/quote/}{\pandocbounded{\includesvg[keepaspectratio]{/assets/icons/16-arrow-right.svg}}}

{ Quote } { Next page }
