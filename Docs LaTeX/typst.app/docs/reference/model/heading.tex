\title{typst.app/docs/reference/model/heading}

\begin{itemize}
\tightlist
\item
  \href{/docs}{\includesvg[width=0.16667in,height=0.16667in]{/assets/icons/16-docs-dark.svg}}
\item
  \includesvg[width=0.16667in,height=0.16667in]{/assets/icons/16-arrow-right.svg}
\item
  \href{/docs/reference/}{Reference}
\item
  \includesvg[width=0.16667in,height=0.16667in]{/assets/icons/16-arrow-right.svg}
\item
  \href{/docs/reference/model/}{Model}
\item
  \includesvg[width=0.16667in,height=0.16667in]{/assets/icons/16-arrow-right.svg}
\item
  \href{/docs/reference/model/heading/}{Heading}
\end{itemize}

\section{\texorpdfstring{\texttt{\ heading\ } {{ Element
}}}{ heading   Element }}\label{summary}

\phantomsection\label{element-tooltip}
Element functions can be customized with \texttt{\ set\ } and
\texttt{\ show\ } rules.

A section heading.

With headings, you can structure your document into sections. Each
heading has a \emph{level,} which starts at one and is unbounded
upwards. This level indicates the logical role of the following content
(section, subsection, etc.) A top-level heading indicates a top-level
section of the document (not the document\textquotesingle s title).

Typst can automatically number your headings for you. To enable
numbering, specify how you want your headings to be numbered with a
\href{/docs/reference/model/numbering/}{numbering pattern or function} .

Independently of the numbering, Typst can also automatically generate an
\href{/docs/reference/model/outline/}{outline} of all headings for you.
To exclude one or more headings from this outline, you can set the
\texttt{\ outlined\ } parameter to
\texttt{\ }{\texttt{\ false\ }}\texttt{\ } .

\subsection{Example}\label{example}

\begin{verbatim}
#set heading(numbering: "1.a)")

= Introduction
In recent years, ...

== Preliminaries
To start, ...
\end{verbatim}

\includegraphics[width=5in,height=\textheight,keepaspectratio]{/assets/docs/PajtbDMMN2eDYZCkAh9ZJwAAAAAAAAAA.png}

\subsection{Syntax}\label{syntax}

Headings have dedicated syntax: They can be created by starting a line
with one or multiple equals signs, followed by a space. The number of
equals signs determines the heading\textquotesingle s logical nesting
depth. The \texttt{\ offset\ } field can be set to configure the
starting depth.

\subsection{\texorpdfstring{{ Parameters
}}{ Parameters }}\label{parameters}

\phantomsection\label{parameters-tooltip}
Parameters are the inputs to a function. They are specified in
parentheses after the function name.

{ heading } (

{ \hyperref[parameters-level]{level :}
\href{/docs/reference/foundations/auto/}{auto}
\href{/docs/reference/foundations/int/}{int} , } {
\hyperref[parameters-depth]{depth :}
\href{/docs/reference/foundations/int/}{int} , } {
\hyperref[parameters-offset]{offset :}
\href{/docs/reference/foundations/int/}{int} , } {
\hyperref[parameters-numbering]{numbering :}
\href{/docs/reference/foundations/none/}{none}
\href{/docs/reference/foundations/str/}{str}
\href{/docs/reference/foundations/function/}{function} , } {
\hyperref[parameters-supplement]{supplement :}
\href{/docs/reference/foundations/none/}{none}
\href{/docs/reference/foundations/auto/}{auto}
\href{/docs/reference/foundations/content/}{content}
\href{/docs/reference/foundations/function/}{function} , } {
\hyperref[parameters-outlined]{outlined :}
\href{/docs/reference/foundations/bool/}{bool} , } {
\hyperref[parameters-bookmarked]{bookmarked :}
\href{/docs/reference/foundations/auto/}{auto}
\href{/docs/reference/foundations/bool/}{bool} , } {
\hyperref[parameters-hanging-indent]{hanging-indent :}
\href{/docs/reference/foundations/auto/}{auto}
\href{/docs/reference/layout/length/}{length} , } {
\href{/docs/reference/foundations/content/}{content} , }

) -\textgreater{} \href{/docs/reference/foundations/content/}{content}

\subsubsection{\texorpdfstring{\texttt{\ level\ }}{ level }}\label{parameters-level}

\href{/docs/reference/foundations/auto/}{auto} {or}
\href{/docs/reference/foundations/int/}{int}

{{ Settable }}

\phantomsection\label{parameters-level-settable-tooltip}
Settable parameters can be customized for all following uses of the
function with a \texttt{\ set\ } rule.

The absolute nesting depth of the heading, starting from one. If set to
\texttt{\ }{\texttt{\ auto\ }}\texttt{\ } , it is computed from
\texttt{\ offset\ }{\texttt{\ +\ }}\texttt{\ depth\ } .

This is primarily useful for usage in
\href{/docs/reference/styling/\#show-rules}{show rules} (either with
\href{/docs/reference/foundations/function/\#definitions-where}{\texttt{\ where\ }}
selectors or by accessing the level directly on a shown heading).

Default: \texttt{\ }{\texttt{\ auto\ }}\texttt{\ }

\includesvg[width=0.16667in,height=0.16667in]{/assets/icons/16-arrow-right.svg}
View example

\begin{verbatim}
#show heading.where(level: 2): set text(red)

= Level 1
== Level 2

#set heading(offset: 1)
= Also level 2
== Level 3
\end{verbatim}

\includegraphics[width=5in,height=\textheight,keepaspectratio]{/assets/docs/_pDm-P05bg_jGbl9uvGjlAAAAAAAAAAA.png}

\subsubsection{\texorpdfstring{\texttt{\ depth\ }}{ depth }}\label{parameters-depth}

\href{/docs/reference/foundations/int/}{int}

{{ Settable }}

\phantomsection\label{parameters-depth-settable-tooltip}
Settable parameters can be customized for all following uses of the
function with a \texttt{\ set\ } rule.

The relative nesting depth of the heading, starting from one. This is
combined with \texttt{\ offset\ } to compute the actual
\texttt{\ level\ } .

This is set by the heading syntax, such that
\texttt{\ }{\texttt{\ ==\ Heading\ }}\texttt{\ } creates a heading with
logical depth of 2, but actual level
\texttt{\ offset\ }{\texttt{\ +\ }}\texttt{\ }{\texttt{\ 2\ }}\texttt{\ }
. If you construct a heading manually, you should typically prefer this
over setting the absolute level.

Default: \texttt{\ }{\texttt{\ 1\ }}\texttt{\ }

\subsubsection{\texorpdfstring{\texttt{\ offset\ }}{ offset }}\label{parameters-offset}

\href{/docs/reference/foundations/int/}{int}

{{ Settable }}

\phantomsection\label{parameters-offset-settable-tooltip}
Settable parameters can be customized for all following uses of the
function with a \texttt{\ set\ } rule.

The starting offset of each heading\textquotesingle s \texttt{\ level\ }
, used to turn its relative \texttt{\ depth\ } into its absolute
\texttt{\ level\ } .

Default: \texttt{\ }{\texttt{\ 0\ }}\texttt{\ }

\includesvg[width=0.16667in,height=0.16667in]{/assets/icons/16-arrow-right.svg}
View example

\begin{verbatim}
= Level 1

#set heading(offset: 1, numbering: "1.1")
= Level 2

#heading(offset: 2, depth: 2)[
  I'm level 4
]
\end{verbatim}

\includegraphics[width=5in,height=\textheight,keepaspectratio]{/assets/docs/hKtWik5-HwMMejqOwDVKLAAAAAAAAAAA.png}

\subsubsection{\texorpdfstring{\texttt{\ numbering\ }}{ numbering }}\label{parameters-numbering}

\href{/docs/reference/foundations/none/}{none} {or}
\href{/docs/reference/foundations/str/}{str} {or}
\href{/docs/reference/foundations/function/}{function}

{{ Settable }}

\phantomsection\label{parameters-numbering-settable-tooltip}
Settable parameters can be customized for all following uses of the
function with a \texttt{\ set\ } rule.

How to number the heading. Accepts a
\href{/docs/reference/model/numbering/}{numbering pattern or function} .

Default: \texttt{\ }{\texttt{\ none\ }}\texttt{\ }

\includesvg[width=0.16667in,height=0.16667in]{/assets/icons/16-arrow-right.svg}
View example

\begin{verbatim}
#set heading(numbering: "1.a.")

= A section
== A subsection
=== A sub-subsection
\end{verbatim}

\includegraphics[width=5in,height=\textheight,keepaspectratio]{/assets/docs/dtIXlP8zFF4SfNqscPeLbAAAAAAAAAAA.png}

\subsubsection{\texorpdfstring{\texttt{\ supplement\ }}{ supplement }}\label{parameters-supplement}

\href{/docs/reference/foundations/none/}{none} {or}
\href{/docs/reference/foundations/auto/}{auto} {or}
\href{/docs/reference/foundations/content/}{content} {or}
\href{/docs/reference/foundations/function/}{function}

{{ Settable }}

\phantomsection\label{parameters-supplement-settable-tooltip}
Settable parameters can be customized for all following uses of the
function with a \texttt{\ set\ } rule.

A supplement for the heading.

For references to headings, this is added before the referenced number.

If a function is specified, it is passed the referenced heading and
should return content.

Default: \texttt{\ }{\texttt{\ auto\ }}\texttt{\ }

\includesvg[width=0.16667in,height=0.16667in]{/assets/icons/16-arrow-right.svg}
View example

\begin{verbatim}
#set heading(numbering: "1.", supplement: [Chapter])

= Introduction <intro>
In @intro, we see how to turn
Sections into Chapters. And
in @intro[Part], it is done
manually.
\end{verbatim}

\includegraphics[width=5in,height=\textheight,keepaspectratio]{/assets/docs/OZMUTnmWZCt9L0XUTDaRmQAAAAAAAAAA.png}

\subsubsection{\texorpdfstring{\texttt{\ outlined\ }}{ outlined }}\label{parameters-outlined}

\href{/docs/reference/foundations/bool/}{bool}

{{ Settable }}

\phantomsection\label{parameters-outlined-settable-tooltip}
Settable parameters can be customized for all following uses of the
function with a \texttt{\ set\ } rule.

Whether the heading should appear in the
\href{/docs/reference/model/outline/}{outline} .

Note that this property, if set to
\texttt{\ }{\texttt{\ true\ }}\texttt{\ } , ensures the heading is also
shown as a bookmark in the exported PDF\textquotesingle s outline (when
exporting to PDF). To change that behavior, use the
\texttt{\ bookmarked\ } property.

Default: \texttt{\ }{\texttt{\ true\ }}\texttt{\ }

\includesvg[width=0.16667in,height=0.16667in]{/assets/icons/16-arrow-right.svg}
View example

\begin{verbatim}
#outline()

#heading[Normal]
This is a normal heading.

#heading(outlined: false)[Hidden]
This heading does not appear
in the outline.
\end{verbatim}

\includegraphics[width=5in,height=\textheight,keepaspectratio]{/assets/docs/q3R6803Mv9D8hpPx5wD4TgAAAAAAAAAA.png}

\subsubsection{\texorpdfstring{\texttt{\ bookmarked\ }}{ bookmarked }}\label{parameters-bookmarked}

\href{/docs/reference/foundations/auto/}{auto} {or}
\href{/docs/reference/foundations/bool/}{bool}

{{ Settable }}

\phantomsection\label{parameters-bookmarked-settable-tooltip}
Settable parameters can be customized for all following uses of the
function with a \texttt{\ set\ } rule.

Whether the heading should appear as a bookmark in the exported
PDF\textquotesingle s outline. Doesn\textquotesingle t affect other
export formats, such as PNG.

The default value of \texttt{\ }{\texttt{\ auto\ }}\texttt{\ } indicates
that the heading will only appear in the exported PDF\textquotesingle s
outline if its \texttt{\ outlined\ } property is set to
\texttt{\ }{\texttt{\ true\ }}\texttt{\ } , that is, if it would also be
listed in Typst\textquotesingle s
\href{/docs/reference/model/outline/}{outline} . Setting this property
to either \texttt{\ }{\texttt{\ true\ }}\texttt{\ } (bookmark) or
\texttt{\ }{\texttt{\ false\ }}\texttt{\ } (don\textquotesingle t
bookmark) bypasses that behavior.

Default: \texttt{\ }{\texttt{\ auto\ }}\texttt{\ }

\includesvg[width=0.16667in,height=0.16667in]{/assets/icons/16-arrow-right.svg}
View example

\begin{verbatim}
#heading[Normal heading]
This heading will be shown in
the PDF's bookmark outline.

#heading(bookmarked: false)[Not bookmarked]
This heading won't be
bookmarked in the resulting
PDF.
\end{verbatim}

\includegraphics[width=5in,height=\textheight,keepaspectratio]{/assets/docs/_UvMUDZOtTdH4i83Hac2iwAAAAAAAAAA.png}

\subsubsection{\texorpdfstring{\texttt{\ hanging-indent\ }}{ hanging-indent }}\label{parameters-hanging-indent}

\href{/docs/reference/foundations/auto/}{auto} {or}
\href{/docs/reference/layout/length/}{length}

{{ Settable }}

\phantomsection\label{parameters-hanging-indent-settable-tooltip}
Settable parameters can be customized for all following uses of the
function with a \texttt{\ set\ } rule.

The indent all but the first line of a heading should have.

The default value of \texttt{\ }{\texttt{\ auto\ }}\texttt{\ } indicates
that the subsequent heading lines will be indented based on the width of
the numbering.

Default: \texttt{\ }{\texttt{\ auto\ }}\texttt{\ }

\includesvg[width=0.16667in,height=0.16667in]{/assets/icons/16-arrow-right.svg}
View example

\begin{verbatim}
#set heading(numbering: "1.")
#heading[A very, very, very, very, very, very long heading]
\end{verbatim}

\includegraphics[width=5in,height=\textheight,keepaspectratio]{/assets/docs/35Dg34kG-7rg1-RFp8FaIgAAAAAAAAAA.png}

\subsubsection{\texorpdfstring{\texttt{\ body\ }}{ body }}\label{parameters-body}

\href{/docs/reference/foundations/content/}{content}

{Required} {{ Positional }}

\phantomsection\label{parameters-body-positional-tooltip}
Positional parameters are specified in order, without names.

The heading\textquotesingle s title.

\href{/docs/reference/model/footnote/}{\pandocbounded{\includesvg[keepaspectratio]{/assets/icons/16-arrow-right.svg}}}

{ Footnote } { Previous page }

\href{/docs/reference/model/link/}{\pandocbounded{\includesvg[keepaspectratio]{/assets/icons/16-arrow-right.svg}}}

{ Link } { Next page }
