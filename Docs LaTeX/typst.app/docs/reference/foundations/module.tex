\title{typst.app/docs/reference/foundations/module}

\begin{itemize}
\tightlist
\item
  \href{/docs}{\includesvg[width=0.16667in,height=0.16667in]{/assets/icons/16-docs-dark.svg}}
\item
  \includesvg[width=0.16667in,height=0.16667in]{/assets/icons/16-arrow-right.svg}
\item
  \href{/docs/reference/}{Reference}
\item
  \includesvg[width=0.16667in,height=0.16667in]{/assets/icons/16-arrow-right.svg}
\item
  \href{/docs/reference/foundations/}{Foundations}
\item
  \includesvg[width=0.16667in,height=0.16667in]{/assets/icons/16-arrow-right.svg}
\item
  \href{/docs/reference/foundations/module/}{Module}
\end{itemize}

\section{\texorpdfstring{{ module }}{ module }}\label{summary}

An evaluated module, either built-in or resulting from a file.

You can access definitions from the module using
\href{/docs/reference/scripting/\#fields}{field access notation} and
interact with it using the
\href{/docs/reference/scripting/\#modules}{import and include syntaxes}
. Alternatively, it is possible to convert a module to a dictionary, and
therefore access its contents dynamically, using the
\href{/docs/reference/foundations/dictionary/\#constructor}{dictionary
constructor} .

\subsection{Example}\label{example}

\begin{verbatim}
#import "utils.typ"
#utils.add(2, 5)

#import utils: sub
#sub(1, 4)
\end{verbatim}

\includegraphics[width=5in,height=\textheight,keepaspectratio]{/assets/docs/itOPaialNOb62A81RHFv_wAAAAAAAAAA.png}

\href{/docs/reference/foundations/label/}{\pandocbounded{\includesvg[keepaspectratio]{/assets/icons/16-arrow-right.svg}}}

{ Label } { Previous page }

\href{/docs/reference/foundations/none/}{\pandocbounded{\includesvg[keepaspectratio]{/assets/icons/16-arrow-right.svg}}}

{ None } { Next page }
