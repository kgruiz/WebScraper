\title{typst.app/docs/reference/foundations/arguments}

\begin{itemize}
\tightlist
\item
  \href{/docs}{\includesvg[width=0.16667in,height=0.16667in]{/assets/icons/16-docs-dark.svg}}
\item
  \includesvg[width=0.16667in,height=0.16667in]{/assets/icons/16-arrow-right.svg}
\item
  \href{/docs/reference/}{Reference}
\item
  \includesvg[width=0.16667in,height=0.16667in]{/assets/icons/16-arrow-right.svg}
\item
  \href{/docs/reference/foundations/}{Foundations}
\item
  \includesvg[width=0.16667in,height=0.16667in]{/assets/icons/16-arrow-right.svg}
\item
  \href{/docs/reference/foundations/arguments/}{Arguments}
\end{itemize}

\section{\texorpdfstring{{ arguments }}{ arguments }}\label{summary}

Captured arguments to a function.

\subsection{Argument Sinks}\label{argument-sinks}

Like built-in functions, custom functions can also take a variable
number of arguments. You can specify an \emph{argument sink} which
collects all excess arguments as \texttt{\ ..sink\ } . The resulting
\texttt{\ sink\ } value is of the \texttt{\ arguments\ } type. It
exposes methods to access the positional and named arguments.

\begin{verbatim}
#let format(title, ..authors) = {
  let by = authors
    .pos()
    .join(", ", last: " and ")

  [*#title* \ _Written by #by;_]
}

#format("ArtosFlow", "Jane", "Joe")
\end{verbatim}

\includegraphics[width=5in,height=\textheight,keepaspectratio]{/assets/docs/DWzn69gGuCd1q_LVZvjEEgAAAAAAAAAA.png}

\subsection{Spreading}\label{spreading}

Inversely to an argument sink, you can \emph{spread} arguments, arrays
and dictionaries into a function call with the \texttt{\ ..spread\ }
operator:

\begin{verbatim}
#let array = (2, 3, 5)
#calc.min(..array)
#let dict = (fill: blue)
#text(..dict)[Hello]
\end{verbatim}

\includegraphics[width=5in,height=\textheight,keepaspectratio]{/assets/docs/kcmqtH9qxq6Bg8ZwwKnMCQAAAAAAAAAA.png}

\subsection{\texorpdfstring{Constructor
{}}{Constructor }}\label{constructor}

\phantomsection\label{constructor-constructor-tooltip}
If a type has a constructor, you can call it like a function to create a
new value of the type.

Construct spreadable arguments in place.

This function behaves like
\texttt{\ }{\texttt{\ let\ }}\texttt{\ }{\texttt{\ args\ }}\texttt{\ }{\texttt{\ (\ }}\texttt{\ }{\texttt{\ ..\ }}\texttt{\ sink\ }{\texttt{\ )\ }}\texttt{\ }{\texttt{\ =\ }}\texttt{\ sink\ }
.

{ arguments } (

{ \hyperref[constructor-parameters-arguments]{..} { any } }

) -\textgreater{}
\href{/docs/reference/foundations/arguments/}{arguments}

\begin{verbatim}
#let args = arguments(stroke: red, inset: 1em, [Body])
#box(..args)
\end{verbatim}

\includegraphics[width=5in,height=\textheight,keepaspectratio]{/assets/docs/JbzK099-rqq0pkW-oHCQsgAAAAAAAAAA.png}

\paragraph{\texorpdfstring{\texttt{\ arguments\ }}{ arguments }}\label{constructor-arguments}

{ any }

{Required} {{ Positional }}

\phantomsection\label{constructor-arguments-positional-tooltip}
Positional parameters are specified in order, without names.

{{ Variadic }}

\phantomsection\label{constructor-arguments-variadic-tooltip}
Variadic parameters can be specified multiple times.

The arguments to construct.

\subsection{\texorpdfstring{{ Definitions
}}{ Definitions }}\label{definitions}

\phantomsection\label{definitions-tooltip}
Functions and types and can have associated definitions. These are
accessed by specifying the function or type, followed by a period, and
then the definition\textquotesingle s name.

\subsubsection{\texorpdfstring{\texttt{\ at\ }}{ at }}\label{definitions-at}

Returns the positional argument at the specified index, or the named
argument with the specified name.

If the key is an \href{/docs/reference/foundations/int/}{integer} , this
is equivalent to first calling
\href{/docs/reference/foundations/arguments/\#definitions-pos}{\texttt{\ pos\ }}
and then
\href{/docs/reference/foundations/array/\#definitions-at}{\texttt{\ array.at\ }}
. If it is a \href{/docs/reference/foundations/str/}{string} , this is
equivalent to first calling
\href{/docs/reference/foundations/arguments/\#definitions-named}{\texttt{\ named\ }}
and then
\href{/docs/reference/foundations/dictionary/\#definitions-at}{\texttt{\ dictionary.at\ }}
.

self { . } { at } (

{ \href{/docs/reference/foundations/int/}{int}
\href{/docs/reference/foundations/str/}{str} , } {
\hyperref[definitions-at-parameters-default]{default :} { any } , }

) -\textgreater{} { any }

\paragraph{\texorpdfstring{\texttt{\ key\ }}{ key }}\label{definitions-at-key}

\href{/docs/reference/foundations/int/}{int} {or}
\href{/docs/reference/foundations/str/}{str}

{Required} {{ Positional }}

\phantomsection\label{definitions-at-key-positional-tooltip}
Positional parameters are specified in order, without names.

The index or name of the argument to get.

\paragraph{\texorpdfstring{\texttt{\ default\ }}{ default }}\label{definitions-at-default}

{ any }

A default value to return if the key is invalid.

\subsubsection{\texorpdfstring{\texttt{\ pos\ }}{ pos }}\label{definitions-pos}

Returns the captured positional arguments as an array.

self { . } { pos } (

) -\textgreater{} \href{/docs/reference/foundations/array/}{array}

\subsubsection{\texorpdfstring{\texttt{\ named\ }}{ named }}\label{definitions-named}

Returns the captured named arguments as a dictionary.

self { . } { named } (

) -\textgreater{}
\href{/docs/reference/foundations/dictionary/}{dictionary}

\href{/docs/reference/foundations/}{\pandocbounded{\includesvg[keepaspectratio]{/assets/icons/16-arrow-right.svg}}}

{ Foundations } { Previous page }

\href{/docs/reference/foundations/array/}{\pandocbounded{\includesvg[keepaspectratio]{/assets/icons/16-arrow-right.svg}}}

{ Array } { Next page }
