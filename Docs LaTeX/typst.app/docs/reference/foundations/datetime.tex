\title{typst.app/docs/reference/foundations/datetime}

\begin{itemize}
\tightlist
\item
  \href{/docs}{\includesvg[width=0.16667in,height=0.16667in]{/assets/icons/16-docs-dark.svg}}
\item
  \includesvg[width=0.16667in,height=0.16667in]{/assets/icons/16-arrow-right.svg}
\item
  \href{/docs/reference/}{Reference}
\item
  \includesvg[width=0.16667in,height=0.16667in]{/assets/icons/16-arrow-right.svg}
\item
  \href{/docs/reference/foundations/}{Foundations}
\item
  \includesvg[width=0.16667in,height=0.16667in]{/assets/icons/16-arrow-right.svg}
\item
  \href{/docs/reference/foundations/datetime/}{Datetime}
\end{itemize}

\section{\texorpdfstring{{ datetime }}{ datetime }}\label{summary}

Represents a date, a time, or a combination of both.

Can be created by either specifying a custom datetime using this
type\textquotesingle s constructor function or getting the current date
with
\href{/docs/reference/foundations/datetime/\#definitions-today}{\texttt{\ datetime.today\ }}
.

\subsection{Example}\label{example}

\begin{verbatim}
#let date = datetime(
  year: 2020,
  month: 10,
  day: 4,
)

#date.display() \
#date.display(
  "y:[year repr:last_two]"
)

#let time = datetime(
  hour: 18,
  minute: 2,
  second: 23,
)

#time.display() \
#time.display(
  "h:[hour repr:12][period]"
)
\end{verbatim}

\includegraphics[width=5in,height=\textheight,keepaspectratio]{/assets/docs/aJRkqg11vpsxBq0NzqAo0gAAAAAAAAAA.png}

\subsection{Datetime and Duration}\label{datetime-and-duration}

You can get a \href{/docs/reference/foundations/duration/}{duration} by
subtracting two datetime:

\begin{verbatim}
#let first-of-march = datetime(day: 1, month: 3, year: 2024)
#let first-of-jan = datetime(day: 1, month: 1, year: 2024)
#let distance = first-of-march - first-of-jan
#distance.hours()
\end{verbatim}

\includegraphics[width=5in,height=\textheight,keepaspectratio]{/assets/docs/xJIPnvV5Iiw8osdkiAUb_AAAAAAAAAAA.png}

You can also add/subtract a datetime and a duration to retrieve a new,
offset datetime:

\begin{verbatim}
#let date = datetime(day: 1, month: 3, year: 2024)
#let two-days = duration(days: 2)
#let two-days-earlier = date - two-days
#let two-days-later = date + two-days

#date.display() \
#two-days-earlier.display() \
#two-days-later.display()
\end{verbatim}

\includegraphics[width=5in,height=\textheight,keepaspectratio]{/assets/docs/R-BPj6xQMFasAxM1n3h_iwAAAAAAAAAA.png}

\subsection{Format}\label{format}

You can specify a customized formatting using the
\href{/docs/reference/foundations/datetime/\#definitions-display}{\texttt{\ display\ }}
method. The format of a datetime is specified by providing
\emph{components} with a specified number of \emph{modifiers} . A
component represents a certain part of the datetime that you want to
display, and with the help of modifiers you can define how you want to
display that component. In order to display a component, you wrap the
name of the component in square brackets (e.g. \texttt{\ {[}year{]}\ }
will display the year). In order to add modifiers, you add a space after
the component name followed by the name of the modifier, a colon and the
value of the modifier (e.g. \texttt{\ {[}month\ repr:short{]}\ } will
display the short representation of the month).

The possible combination of components and their respective modifiers is
as follows:

\begin{itemize}
\tightlist
\item
  \texttt{\ year\ } : Displays the year of the datetime.

  \begin{itemize}
  \tightlist
  \item
    \texttt{\ padding\ } : Can be either \texttt{\ zero\ } ,
    \texttt{\ space\ } or \texttt{\ none\ } . Specifies how the year is
    padded.
  \item
    \texttt{\ repr\ } Can be either \texttt{\ full\ } in which case the
    full year is displayed or \texttt{\ last\_two\ } in which case only
    the last two digits are displayed.
  \item
    \texttt{\ sign\ } : Can be either \texttt{\ automatic\ } or
    \texttt{\ mandatory\ } . Specifies when the sign should be
    displayed.
  \end{itemize}
\item
  \texttt{\ month\ } : Displays the month of the datetime.

  \begin{itemize}
  \tightlist
  \item
    \texttt{\ padding\ } : Can be either \texttt{\ zero\ } ,
    \texttt{\ space\ } or \texttt{\ none\ } . Specifies how the month is
    padded.
  \item
    \texttt{\ repr\ } : Can be either \texttt{\ numerical\ } ,
    \texttt{\ long\ } or \texttt{\ short\ } . Specifies if the month
    should be displayed as a number or a word. Unfortunately, when
    choosing the word representation, it can currently only display the
    English version. In the future, it is planned to support
    localization.
  \end{itemize}
\item
  \texttt{\ day\ } : Displays the day of the datetime.

  \begin{itemize}
  \tightlist
  \item
    \texttt{\ padding\ } : Can be either \texttt{\ zero\ } ,
    \texttt{\ space\ } or \texttt{\ none\ } . Specifies how the day is
    padded.
  \end{itemize}
\item
  \texttt{\ week\_number\ } : Displays the week number of the datetime.

  \begin{itemize}
  \tightlist
  \item
    \texttt{\ padding\ } : Can be either \texttt{\ zero\ } ,
    \texttt{\ space\ } or \texttt{\ none\ } . Specifies how the week
    number is padded.
  \item
    \texttt{\ repr\ } : Can be either \texttt{\ ISO\ } ,
    \texttt{\ sunday\ } or \texttt{\ monday\ } . In the case of
    \texttt{\ ISO\ } , week numbers are between 1 and 53, while the
    other ones are between 0 and 53.
  \end{itemize}
\item
  \texttt{\ weekday\ } : Displays the weekday of the date.

  \begin{itemize}
  \tightlist
  \item
    \texttt{\ repr\ } Can be either \texttt{\ long\ } ,
    \texttt{\ short\ } , \texttt{\ sunday\ } or \texttt{\ monday\ } . In
    the case of \texttt{\ long\ } and \texttt{\ short\ } , the
    corresponding English name will be displayed (same as for the month,
    other languages are currently not supported). In the case of
    \texttt{\ sunday\ } and \texttt{\ monday\ } , the numerical value
    will be displayed (assuming Sunday and Monday as the first day of
    the week, respectively).
  \item
    \texttt{\ one\_indexed\ } : Can be either \texttt{\ true\ } or
    \texttt{\ false\ } . Defines whether the numerical representation of
    the week starts with 0 or 1.
  \end{itemize}
\item
  \texttt{\ hour\ } : Displays the hour of the date.

  \begin{itemize}
  \tightlist
  \item
    \texttt{\ padding\ } : Can be either \texttt{\ zero\ } ,
    \texttt{\ space\ } or \texttt{\ none\ } . Specifies how the hour is
    padded.
  \item
    \texttt{\ repr\ } : Can be either \texttt{\ 24\ } or \texttt{\ 12\ }
    . Changes whether the hour is displayed in the 24-hour or 12-hour
    format.
  \end{itemize}
\item
  \texttt{\ period\ } : The AM/PM part of the hour

  \begin{itemize}
  \tightlist
  \item
    \texttt{\ case\ } : Can be \texttt{\ lower\ } to display it in lower
    case and \texttt{\ upper\ } to display it in upper case.
  \end{itemize}
\item
  \texttt{\ minute\ } : Displays the minute of the date.

  \begin{itemize}
  \tightlist
  \item
    \texttt{\ padding\ } : Can be either \texttt{\ zero\ } ,
    \texttt{\ space\ } or \texttt{\ none\ } . Specifies how the minute
    is padded.
  \end{itemize}
\item
  \texttt{\ second\ } : Displays the second of the date.

  \begin{itemize}
  \tightlist
  \item
    \texttt{\ padding\ } : Can be either \texttt{\ zero\ } ,
    \texttt{\ space\ } or \texttt{\ none\ } . Specifies how the second
    is padded.
  \end{itemize}
\end{itemize}

Keep in mind that not always all components can be used. For example, if
you create a new datetime with
\texttt{\ }{\texttt{\ datetime\ }}\texttt{\ }{\texttt{\ (\ }}\texttt{\ year\ }{\texttt{\ :\ }}\texttt{\ }{\texttt{\ 2023\ }}\texttt{\ }{\texttt{\ ,\ }}\texttt{\ month\ }{\texttt{\ :\ }}\texttt{\ }{\texttt{\ 10\ }}\texttt{\ }{\texttt{\ ,\ }}\texttt{\ day\ }{\texttt{\ :\ }}\texttt{\ }{\texttt{\ 13\ }}\texttt{\ }{\texttt{\ )\ }}\texttt{\ }
, it will be stored as a plain date internally, meaning that you cannot
use components such as \texttt{\ hour\ } or \texttt{\ minute\ } , which
would only work on datetimes that have a specified time.

\subsection{\texorpdfstring{Constructor
{}}{Constructor }}\label{constructor}

\phantomsection\label{constructor-constructor-tooltip}
If a type has a constructor, you can call it like a function to create a
new value of the type.

Creates a new datetime.

You can specify the
\href{/docs/reference/foundations/datetime/}{datetime} using a year,
month, day, hour, minute, and second.

\emph{Note} : Depending on which components of the datetime you specify,
Typst will store it in one of the following three ways:

\begin{itemize}
\tightlist
\item
  If you specify year, month and day, Typst will store just a date.
\item
  If you specify hour, minute and second, Typst will store just a time.
\item
  If you specify all of year, month, day, hour, minute and second, Typst
  will store a full datetime.
\end{itemize}

Depending on how it is stored, the
\href{/docs/reference/foundations/datetime/\#definitions-display}{\texttt{\ display\ }}
method will choose a different formatting by default.

{ datetime } (

{ \hyperref[constructor-parameters-year]{year :}
\href{/docs/reference/foundations/int/}{int} , } {
\hyperref[constructor-parameters-month]{month :}
\href{/docs/reference/foundations/int/}{int} , } {
\hyperref[constructor-parameters-day]{day :}
\href{/docs/reference/foundations/int/}{int} , } {
\hyperref[constructor-parameters-hour]{hour :}
\href{/docs/reference/foundations/int/}{int} , } {
\hyperref[constructor-parameters-minute]{minute :}
\href{/docs/reference/foundations/int/}{int} , } {
\hyperref[constructor-parameters-second]{second :}
\href{/docs/reference/foundations/int/}{int} , }

) -\textgreater{} \href{/docs/reference/foundations/datetime/}{datetime}

\begin{verbatim}
#datetime(
  year: 2012,
  month: 8,
  day: 3,
).display()
\end{verbatim}

\includegraphics[width=5in,height=\textheight,keepaspectratio]{/assets/docs/6mpnNRypNysjXvXstSouiwAAAAAAAAAA.png}

\paragraph{\texorpdfstring{\texttt{\ year\ }}{ year }}\label{constructor-year}

\href{/docs/reference/foundations/int/}{int}

The year of the datetime.

\paragraph{\texorpdfstring{\texttt{\ month\ }}{ month }}\label{constructor-month}

\href{/docs/reference/foundations/int/}{int}

The month of the datetime.

\paragraph{\texorpdfstring{\texttt{\ day\ }}{ day }}\label{constructor-day}

\href{/docs/reference/foundations/int/}{int}

The day of the datetime.

\paragraph{\texorpdfstring{\texttt{\ hour\ }}{ hour }}\label{constructor-hour}

\href{/docs/reference/foundations/int/}{int}

The hour of the datetime.

\paragraph{\texorpdfstring{\texttt{\ minute\ }}{ minute }}\label{constructor-minute}

\href{/docs/reference/foundations/int/}{int}

The minute of the datetime.

\paragraph{\texorpdfstring{\texttt{\ second\ }}{ second }}\label{constructor-second}

\href{/docs/reference/foundations/int/}{int}

The second of the datetime.

\subsection{\texorpdfstring{{ Definitions
}}{ Definitions }}\label{definitions}

\phantomsection\label{definitions-tooltip}
Functions and types and can have associated definitions. These are
accessed by specifying the function or type, followed by a period, and
then the definition\textquotesingle s name.

\subsubsection{\texorpdfstring{\texttt{\ today\ }}{ today }}\label{definitions-today}

Returns the current date.

datetime { . } { today } (

{ \hyperref[definitions-today-parameters-offset]{offset :}
\href{/docs/reference/foundations/auto/}{auto}
\href{/docs/reference/foundations/int/}{int} }

) -\textgreater{} \href{/docs/reference/foundations/datetime/}{datetime}

\begin{verbatim}
Today's date is
#datetime.today().display().
\end{verbatim}

\includegraphics[width=5in,height=\textheight,keepaspectratio]{/assets/docs/SOSDKByfy_YbHbk7NejgOQAAAAAAAAAA.png}

\paragraph{\texorpdfstring{\texttt{\ offset\ }}{ offset }}\label{definitions-today-offset}

\href{/docs/reference/foundations/auto/}{auto} {or}
\href{/docs/reference/foundations/int/}{int}

An offset to apply to the current UTC date. If set to
\texttt{\ }{\texttt{\ auto\ }}\texttt{\ } , the offset will be the local
offset.

Default: \texttt{\ }{\texttt{\ auto\ }}\texttt{\ }

\subsubsection{\texorpdfstring{\texttt{\ display\ }}{ display }}\label{definitions-display}

Displays the datetime in a specified format.

Depending on whether you have defined just a date, a time or both, the
default format will be different. If you specified a date, it will be
\texttt{\ {[}year{]}-{[}month{]}-{[}day{]}\ } . If you specified a time,
it will be \texttt{\ {[}hour{]}:{[}minute{]}:{[}second{]}\ } . In the
case of a datetime, it will be
\texttt{\ {[}year{]}-{[}month{]}-{[}day{]}\ {[}hour{]}:{[}minute{]}:{[}second{]}\ }
.

See the \href{/docs/reference/foundations/datetime/\#format}{format
syntax} for more information.

self { . } { display } (

{ \href{/docs/reference/foundations/auto/}{auto}
\href{/docs/reference/foundations/str/}{str} }

) -\textgreater{} \href{/docs/reference/foundations/str/}{str}

\paragraph{\texorpdfstring{\texttt{\ pattern\ }}{ pattern }}\label{definitions-display-pattern}

\href{/docs/reference/foundations/auto/}{auto} {or}
\href{/docs/reference/foundations/str/}{str}

{{ Positional }}

\phantomsection\label{definitions-display-pattern-positional-tooltip}
Positional parameters are specified in order, without names.

The format used to display the datetime.

Default: \texttt{\ }{\texttt{\ auto\ }}\texttt{\ }

\subsubsection{\texorpdfstring{\texttt{\ year\ }}{ year }}\label{definitions-year}

The year if it was specified, or
\texttt{\ }{\texttt{\ none\ }}\texttt{\ } for times without a date.

self { . } { year } (

) -\textgreater{} \href{/docs/reference/foundations/none/}{none}
\href{/docs/reference/foundations/int/}{int}

\subsubsection{\texorpdfstring{\texttt{\ month\ }}{ month }}\label{definitions-month}

The month if it was specified, or
\texttt{\ }{\texttt{\ none\ }}\texttt{\ } for times without a date.

self { . } { month } (

) -\textgreater{} \href{/docs/reference/foundations/none/}{none}
\href{/docs/reference/foundations/int/}{int}

\subsubsection{\texorpdfstring{\texttt{\ weekday\ }}{ weekday }}\label{definitions-weekday}

The weekday (counting Monday as 1) or
\texttt{\ }{\texttt{\ none\ }}\texttt{\ } for times without a date.

self { . } { weekday } (

) -\textgreater{} \href{/docs/reference/foundations/none/}{none}
\href{/docs/reference/foundations/int/}{int}

\subsubsection{\texorpdfstring{\texttt{\ day\ }}{ day }}\label{definitions-day}

The day if it was specified, or
\texttt{\ }{\texttt{\ none\ }}\texttt{\ } for times without a date.

self { . } { day } (

) -\textgreater{} \href{/docs/reference/foundations/none/}{none}
\href{/docs/reference/foundations/int/}{int}

\subsubsection{\texorpdfstring{\texttt{\ hour\ }}{ hour }}\label{definitions-hour}

The hour if it was specified, or
\texttt{\ }{\texttt{\ none\ }}\texttt{\ } for dates without a time.

self { . } { hour } (

) -\textgreater{} \href{/docs/reference/foundations/none/}{none}
\href{/docs/reference/foundations/int/}{int}

\subsubsection{\texorpdfstring{\texttt{\ minute\ }}{ minute }}\label{definitions-minute}

The minute if it was specified, or
\texttt{\ }{\texttt{\ none\ }}\texttt{\ } for dates without a time.

self { . } { minute } (

) -\textgreater{} \href{/docs/reference/foundations/none/}{none}
\href{/docs/reference/foundations/int/}{int}

\subsubsection{\texorpdfstring{\texttt{\ second\ }}{ second }}\label{definitions-second}

The second if it was specified, or
\texttt{\ }{\texttt{\ none\ }}\texttt{\ } for dates without a time.

self { . } { second } (

) -\textgreater{} \href{/docs/reference/foundations/none/}{none}
\href{/docs/reference/foundations/int/}{int}

\subsubsection{\texorpdfstring{\texttt{\ ordinal\ }}{ ordinal }}\label{definitions-ordinal}

The ordinal (day of the year), or
\texttt{\ }{\texttt{\ none\ }}\texttt{\ } for times without a date.

self { . } { ordinal } (

) -\textgreater{} \href{/docs/reference/foundations/none/}{none}
\href{/docs/reference/foundations/int/}{int}

\href{/docs/reference/foundations/content/}{\pandocbounded{\includesvg[keepaspectratio]{/assets/icons/16-arrow-right.svg}}}

{ Content } { Previous page }

\href{/docs/reference/foundations/decimal/}{\pandocbounded{\includesvg[keepaspectratio]{/assets/icons/16-arrow-right.svg}}}

{ Decimal } { Next page }
