\title{typst.app/docs/reference/introspection/query}

\begin{itemize}
\tightlist
\item
  \href{/docs}{\includesvg[width=0.16667in,height=0.16667in]{/assets/icons/16-docs-dark.svg}}
\item
  \includesvg[width=0.16667in,height=0.16667in]{/assets/icons/16-arrow-right.svg}
\item
  \href{/docs/reference/}{Reference}
\item
  \includesvg[width=0.16667in,height=0.16667in]{/assets/icons/16-arrow-right.svg}
\item
  \href{/docs/reference/introspection/}{Introspection}
\item
  \includesvg[width=0.16667in,height=0.16667in]{/assets/icons/16-arrow-right.svg}
\item
  \href{/docs/reference/introspection/query/}{Query}
\end{itemize}

\section{\texorpdfstring{\texttt{\ query\ } {{ Contextual
}}}{ query   Contextual }}\label{summary}

\phantomsection\label{contextual-tooltip}
Contextual functions can only be used when the context is known

Finds elements in the document.

The \texttt{\ query\ } functions lets you search your document for
elements of a particular type or with a particular label. To use it, you
first need to ensure that \href{/docs/reference/context/}{context} is
available.

\subsection{Finding elements}\label{finding-elements}

In the example below, we manually create a table of contents instead of
using the \href{/docs/reference/model/outline/}{\texttt{\ outline\ }}
function.

To do this, we first query for all headings in the document at level 1
and where \texttt{\ outlined\ } is true. Querying only for headings at
level 1 ensures that, for the purpose of this example, sub-headings are
not included in the table of contents. The \texttt{\ outlined\ } field
is used to exclude the "Table of Contents" heading itself.

Note that we open a \texttt{\ context\ } to be able to use the
\texttt{\ query\ } function.

\begin{verbatim}
#set page(numbering: "1")

#heading(outlined: false)[
  Table of Contents
]
#context {
  let chapters = query(
    heading.where(
      level: 1,
      outlined: true,
    )
  )
  for chapter in chapters {
    let loc = chapter.location()
    let nr = numbering(
      loc.page-numbering(),
      ..counter(page).at(loc),
    )
    [#chapter.body #h(1fr) #nr \ ]
  }
}

= Introduction
#lorem(10)
#pagebreak()

== Sub-Heading
#lorem(8)

= Discussion
#lorem(18)
\end{verbatim}

\includegraphics[width=5in,height=\textheight,keepaspectratio]{/assets/docs/jo-em7a3jFROfNLdVe33CwAAAAAAAAAA.png}
\includegraphics[width=5in,height=\textheight,keepaspectratio]{/assets/docs/jo-em7a3jFROfNLdVe33CwAAAAAAAAAB.png}

To get the page numbers, we first get the location of the elements
returned by \texttt{\ query\ } with
\href{/docs/reference/foundations/content/\#definitions-location}{\texttt{\ location\ }}
. We then also retrieve the
\href{/docs/reference/introspection/location/\#definitions-page-numbering}{page
numbering} and
\href{/docs/reference/introspection/counter/\#page-counter}{page
counter} at that location and apply the numbering to the counter.

\subsection{A word of caution}\label{caution}

To resolve all your queries, Typst evaluates and layouts parts of the
document multiple times. However, there is no guarantee that your
queries can actually be completely resolved. If you
aren\textquotesingle t careful a query can affect itselfâ€''leading to a
result that never stabilizes.

In the example below, we query for all headings in the document. We then
generate as many headings. In the beginning, there\textquotesingle s
just one heading, titled \texttt{\ Real\ } . Thus, \texttt{\ count\ } is
\texttt{\ 1\ } and one \texttt{\ Fake\ } heading is generated. Typst
sees that the query\textquotesingle s result has changed and processes
it again. This time, \texttt{\ count\ } is \texttt{\ 2\ } and two
\texttt{\ Fake\ } headings are generated. This goes on and on. As we can
see, the output has a finite amount of headings. This is because Typst
simply gives up after a few attempts.

In general, you should try not to write queries that affect themselves.
The same words of caution also apply to other introspection features
like \href{/docs/reference/introspection/counter/}{counters} and
\href{/docs/reference/introspection/state/}{state} .

\begin{verbatim}
= Real
#context {
  let elems = query(heading)
  let count = elems.len()
  count * [= Fake]
}
\end{verbatim}

\includegraphics[width=5in,height=\textheight,keepaspectratio]{/assets/docs/C2bjyzuukR06BSWIMgC89wAAAAAAAAAA.png}

\subsection{Command line queries}\label{command-line-queries}

You can also perform queries from the command line with the
\texttt{\ typst\ query\ } command. This command executes an arbitrary
query on the document and returns the resulting elements in serialized
form. Consider the following \texttt{\ example.typ\ } file which
contains some invisible
\href{/docs/reference/introspection/metadata/}{metadata} :

\begin{verbatim}
#metadata("This is a note") <note>
\end{verbatim}

You can execute a query on it as follows using Typst\textquotesingle s
CLI:

\begin{verbatim}
$ typst query example.typ "<note>"
[
  {
    "func": "metadata",
    "value": "This is a note",
    "label": "<note>"
  }
]
\end{verbatim}

Frequently, you\textquotesingle re interested in only one specific field
of the resulting elements. In the case of the \texttt{\ metadata\ }
element, the \texttt{\ value\ } field is the interesting one. You can
extract just this field with the \texttt{\ -\/-field\ } argument.

\begin{verbatim}
$ typst query example.typ "<note>" --field value
["This is a note"]
\end{verbatim}

If you are interested in just a single element, you can use the
\texttt{\ -\/-one\ } flag to extract just it.

\begin{verbatim}
$ typst query example.typ "<note>" --field value --one
"This is a note"
\end{verbatim}

\subsection{\texorpdfstring{{ Parameters
}}{ Parameters }}\label{parameters}

\phantomsection\label{parameters-tooltip}
Parameters are the inputs to a function. They are specified in
parentheses after the function name.

{ query } (

{ \href{/docs/reference/foundations/label/}{label}
\href{/docs/reference/foundations/selector/}{selector}
\href{/docs/reference/introspection/location/}{location}
\href{/docs/reference/foundations/function/}{function} , } {
\href{/docs/reference/foundations/none/}{none}
\href{/docs/reference/introspection/location/}{location} , }

) -\textgreater{} \href{/docs/reference/foundations/array/}{array}

\subsubsection{\texorpdfstring{\texttt{\ target\ }}{ target }}\label{parameters-target}

\href{/docs/reference/foundations/label/}{label} {or}
\href{/docs/reference/foundations/selector/}{selector} {or}
\href{/docs/reference/introspection/location/}{location} {or}
\href{/docs/reference/foundations/function/}{function}

{Required} {{ Positional }}

\phantomsection\label{parameters-target-positional-tooltip}
Positional parameters are specified in order, without names.

Can be

\begin{itemize}
\tightlist
\item
  an element function like a \texttt{\ heading\ } or \texttt{\ figure\ }
  ,
\item
  a \texttt{\ }{\texttt{\ \textless{}label\textgreater{}\ }}\texttt{\ }
  ,
\item
  a more complex selector like
  \texttt{\ heading\ }{\texttt{\ .\ }}\texttt{\ }{\texttt{\ where\ }}\texttt{\ }{\texttt{\ (\ }}\texttt{\ level\ }{\texttt{\ :\ }}\texttt{\ }{\texttt{\ 1\ }}\texttt{\ }{\texttt{\ )\ }}\texttt{\ }
  ,
\item
  or
  \texttt{\ }{\texttt{\ selector\ }}\texttt{\ }{\texttt{\ (\ }}\texttt{\ heading\ }{\texttt{\ )\ }}\texttt{\ }{\texttt{\ .\ }}\texttt{\ }{\texttt{\ before\ }}\texttt{\ }{\texttt{\ (\ }}\texttt{\ }{\texttt{\ here\ }}\texttt{\ }{\texttt{\ (\ }}\texttt{\ }{\texttt{\ )\ }}\texttt{\ }{\texttt{\ )\ }}\texttt{\ }
  .
\end{itemize}

Only
\href{/docs/reference/introspection/location/\#locatable}{locatable}
element functions are supported.

\subsubsection{\texorpdfstring{\texttt{\ location\ }}{ location }}\label{parameters-location}

\href{/docs/reference/foundations/none/}{none} {or}
\href{/docs/reference/introspection/location/}{location}

{{ Positional }}

\phantomsection\label{parameters-location-positional-tooltip}
Positional parameters are specified in order, without names.

\emph{Compatibility:} This argument is deprecated. It only exists for
compatibility with Typst 0.10 and lower and shouldn\textquotesingle t be
used anymore.

Default: \texttt{\ }{\texttt{\ none\ }}\texttt{\ }

\href{/docs/reference/introspection/metadata/}{\pandocbounded{\includesvg[keepaspectratio]{/assets/icons/16-arrow-right.svg}}}

{ Metadata } { Previous page }

\href{/docs/reference/introspection/state/}{\pandocbounded{\includesvg[keepaspectratio]{/assets/icons/16-arrow-right.svg}}}

{ State } { Next page }
