\title{typst.app/docs/reference/layout/ratio}

\begin{itemize}
\tightlist
\item
  \href{/docs}{\includesvg[width=0.16667in,height=0.16667in]{/assets/icons/16-docs-dark.svg}}
\item
  \includesvg[width=0.16667in,height=0.16667in]{/assets/icons/16-arrow-right.svg}
\item
  \href{/docs/reference/}{Reference}
\item
  \includesvg[width=0.16667in,height=0.16667in]{/assets/icons/16-arrow-right.svg}
\item
  \href{/docs/reference/layout/}{Layout}
\item
  \includesvg[width=0.16667in,height=0.16667in]{/assets/icons/16-arrow-right.svg}
\item
  \href{/docs/reference/layout/ratio/}{Ratio}
\end{itemize}

\section{\texorpdfstring{{ ratio }}{ ratio }}\label{summary}

A ratio of a whole.

Written as a number, followed by a percent sign.

\subsection{Example}\label{example}

\begin{verbatim}
#set align(center)
#scale(x: 150%)[
  Scaled apart.
]
\end{verbatim}

\includegraphics[width=5in,height=\textheight,keepaspectratio]{/assets/docs/xEgSJZQe3kQz-XQhwaSthwAAAAAAAAAA.png}

\href{/docs/reference/layout/place/}{\pandocbounded{\includesvg[keepaspectratio]{/assets/icons/16-arrow-right.svg}}}

{ Place } { Previous page }

\href{/docs/reference/layout/relative/}{\pandocbounded{\includesvg[keepaspectratio]{/assets/icons/16-arrow-right.svg}}}

{ Relative Length } { Next page }
