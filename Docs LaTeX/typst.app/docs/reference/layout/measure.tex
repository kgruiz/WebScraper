\title{typst.app/docs/reference/layout/measure}

\begin{itemize}
\tightlist
\item
  \href{/docs}{\includesvg[width=0.16667in,height=0.16667in]{/assets/icons/16-docs-dark.svg}}
\item
  \includesvg[width=0.16667in,height=0.16667in]{/assets/icons/16-arrow-right.svg}
\item
  \href{/docs/reference/}{Reference}
\item
  \includesvg[width=0.16667in,height=0.16667in]{/assets/icons/16-arrow-right.svg}
\item
  \href{/docs/reference/layout/}{Layout}
\item
  \includesvg[width=0.16667in,height=0.16667in]{/assets/icons/16-arrow-right.svg}
\item
  \href{/docs/reference/layout/measure/}{Measure}
\end{itemize}

\section{\texorpdfstring{\texttt{\ measure\ } {{ Contextual
}}}{ measure   Contextual }}\label{summary}

\phantomsection\label{contextual-tooltip}
Contextual functions can only be used when the context is known

Measures the layouted size of content.

The \texttt{\ measure\ } function lets you determine the layouted size
of content. By default an infinite space is assumed, so the measured
dimensions may not necessarily match the final dimensions of the
content. If you want to measure in the current layout dimensions, you
can combine \texttt{\ measure\ } and
\href{/docs/reference/layout/layout/}{\texttt{\ layout\ }} .

\subsection{Example}\label{example}

The same content can have a different size depending on the
\href{/docs/reference/context/}{context} that it is placed into. In the
example below, the
\texttt{\ }{\texttt{\ \#\ }}\texttt{\ }{\texttt{\ content\ }}\texttt{\ }
is of course bigger when we increase the font size.

\begin{verbatim}
#let content = [Hello!]
#content
#set text(14pt)
#content
\end{verbatim}

\includegraphics[width=5in,height=\textheight,keepaspectratio]{/assets/docs/AhP31noWwrcSQXbwnmO-hwAAAAAAAAAA.png}

For this reason, you can only measure when context is available.

\begin{verbatim}
#let thing(body) = context {
  let size = measure(body)
  [Width of "#body" is #size.width]
}

#thing[Hey] \
#thing[Welcome]
\end{verbatim}

\includegraphics[width=5in,height=\textheight,keepaspectratio]{/assets/docs/-y6AuN3J3rl7Gz1x_VRjjwAAAAAAAAAA.png}

The measure function returns a dictionary with the entries
\texttt{\ width\ } and \texttt{\ height\ } , both of type
\href{/docs/reference/layout/length/}{\texttt{\ length\ }} .

\subsection{\texorpdfstring{{ Parameters
}}{ Parameters }}\label{parameters}

\phantomsection\label{parameters-tooltip}
Parameters are the inputs to a function. They are specified in
parentheses after the function name.

{ measure } (

{ \hyperref[parameters-width]{width :}
\href{/docs/reference/foundations/auto/}{auto}
\href{/docs/reference/layout/length/}{length} , } {
\hyperref[parameters-height]{height :}
\href{/docs/reference/foundations/auto/}{auto}
\href{/docs/reference/layout/length/}{length} , } {
\href{/docs/reference/foundations/content/}{content} , } {
\href{/docs/reference/foundations/none/}{none} { styles } , }

) -\textgreater{}
\href{/docs/reference/foundations/dictionary/}{dictionary}

\subsubsection{\texorpdfstring{\texttt{\ width\ }}{ width }}\label{parameters-width}

\href{/docs/reference/foundations/auto/}{auto} {or}
\href{/docs/reference/layout/length/}{length}

The width available to layout the content.

Setting this to \texttt{\ }{\texttt{\ auto\ }}\texttt{\ } indicates
infinite available width.

Note that using the \texttt{\ width\ } and \texttt{\ height\ }
parameters of this function is different from measuring a sized
\href{/docs/reference/layout/block/}{\texttt{\ block\ }} containing the
content. In the following example, the former will get the dimensions of
the inner content instead of the dimensions of the block.

Default: \texttt{\ }{\texttt{\ auto\ }}\texttt{\ }

\includesvg[width=0.16667in,height=0.16667in]{/assets/icons/16-arrow-right.svg}
View example

\begin{verbatim}
#context measure(lorem(100), width: 400pt)

#context measure(block(lorem(100), width: 400pt))
\end{verbatim}

\includegraphics[width=5in,height=\textheight,keepaspectratio]{/assets/docs/kGPOcZfxzWEfqWzKQCJaFgAAAAAAAAAA.png}

\subsubsection{\texorpdfstring{\texttt{\ height\ }}{ height }}\label{parameters-height}

\href{/docs/reference/foundations/auto/}{auto} {or}
\href{/docs/reference/layout/length/}{length}

The height available to layout the content.

Setting this to \texttt{\ }{\texttt{\ auto\ }}\texttt{\ } indicates
infinite available height.

Default: \texttt{\ }{\texttt{\ auto\ }}\texttt{\ }

\subsubsection{\texorpdfstring{\texttt{\ content\ }}{ content }}\label{parameters-content}

\href{/docs/reference/foundations/content/}{content}

{Required} {{ Positional }}

\phantomsection\label{parameters-content-positional-tooltip}
Positional parameters are specified in order, without names.

The content whose size to measure.

\subsubsection{\texorpdfstring{\texttt{\ styles\ }}{ styles }}\label{parameters-styles}

\href{/docs/reference/foundations/none/}{none} {or} { styles }

{{ Positional }}

\phantomsection\label{parameters-styles-positional-tooltip}
Positional parameters are specified in order, without names.

\emph{Compatibility:} This argument is deprecated. It only exists for
compatibility with Typst 0.10 and lower and shouldn\textquotesingle t be
used anymore.

Default: \texttt{\ }{\texttt{\ none\ }}\texttt{\ }

\href{/docs/reference/layout/length/}{\pandocbounded{\includesvg[keepaspectratio]{/assets/icons/16-arrow-right.svg}}}

{ Length } { Previous page }

\href{/docs/reference/layout/move/}{\pandocbounded{\includesvg[keepaspectratio]{/assets/icons/16-arrow-right.svg}}}

{ Move } { Next page }
