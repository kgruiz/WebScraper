\title{typst.app/docs/reference/layout/direction}

\begin{itemize}
\tightlist
\item
  \href{/docs}{\includesvg[width=0.16667in,height=0.16667in]{/assets/icons/16-docs-dark.svg}}
\item
  \includesvg[width=0.16667in,height=0.16667in]{/assets/icons/16-arrow-right.svg}
\item
  \href{/docs/reference/}{Reference}
\item
  \includesvg[width=0.16667in,height=0.16667in]{/assets/icons/16-arrow-right.svg}
\item
  \href{/docs/reference/layout/}{Layout}
\item
  \includesvg[width=0.16667in,height=0.16667in]{/assets/icons/16-arrow-right.svg}
\item
  \href{/docs/reference/layout/direction/}{Direction}
\end{itemize}

\section{\texorpdfstring{{ direction }}{ direction }}\label{summary}

The four directions into which content can be laid out.

Possible values are:

\begin{itemize}
\tightlist
\item
  \texttt{\ ltr\ } : Left to right.
\item
  \texttt{\ rtl\ } : Right to left.
\item
  \texttt{\ ttb\ } : Top to bottom.
\item
  \texttt{\ btt\ } : Bottom to top.
\end{itemize}

These values are available globally and also in the direction
type\textquotesingle s scope, so you can write either of the following
two:

\begin{verbatim}
#stack(dir: rtl)[A][B][C]
#stack(dir: direction.rtl)[A][B][C]
\end{verbatim}

\includegraphics[width=5in,height=\textheight,keepaspectratio]{/assets/docs/43rZPR36KLZcf8RLRLjX0wAAAAAAAAAA.png}

\subsection{\texorpdfstring{{ Definitions
}}{ Definitions }}\label{definitions}

\phantomsection\label{definitions-tooltip}
Functions and types and can have associated definitions. These are
accessed by specifying the function or type, followed by a period, and
then the definition\textquotesingle s name.

\subsubsection{\texorpdfstring{\texttt{\ axis\ }}{ axis }}\label{definitions-axis}

The axis this direction belongs to, either
\texttt{\ }{\texttt{\ "horizontal"\ }}\texttt{\ } or
\texttt{\ }{\texttt{\ "vertical"\ }}\texttt{\ } .

self { . } { axis } (

)

\begin{verbatim}
#ltr.axis() \
#ttb.axis()
\end{verbatim}

\includegraphics[width=5in,height=\textheight,keepaspectratio]{/assets/docs/JrNsSPuIGz5d-HyvpKlmRAAAAAAAAAAA.png}

\subsubsection{\texorpdfstring{\texttt{\ start\ }}{ start }}\label{definitions-start}

The start point of this direction, as an alignment.

self { . } { start } (

) -\textgreater{} \href{/docs/reference/layout/alignment/}{alignment}

\begin{verbatim}
#ltr.start() \
#rtl.start() \
#ttb.start() \
#btt.start()
\end{verbatim}

\includegraphics[width=5in,height=\textheight,keepaspectratio]{/assets/docs/N9RQCkuykNN4FsJgRg06GgAAAAAAAAAA.png}

\subsubsection{\texorpdfstring{\texttt{\ end\ }}{ end }}\label{definitions-end}

The end point of this direction, as an alignment.

self { . } { end } (

) -\textgreater{} \href{/docs/reference/layout/alignment/}{alignment}

\begin{verbatim}
#ltr.end() \
#rtl.end() \
#ttb.end() \
#btt.end()
\end{verbatim}

\includegraphics[width=5in,height=\textheight,keepaspectratio]{/assets/docs/NDjcpeKFmKqoCGermlx1dAAAAAAAAAAA.png}

\subsubsection{\texorpdfstring{\texttt{\ inv\ }}{ inv }}\label{definitions-inv}

The inverse direction.

self { . } { inv } (

) -\textgreater{} \href{/docs/reference/layout/direction/}{direction}

\begin{verbatim}
#ltr.inv() \
#rtl.inv() \
#ttb.inv() \
#btt.inv()
\end{verbatim}

\includegraphics[width=5in,height=\textheight,keepaspectratio]{/assets/docs/kBDvCk2AJ9dPd5ZUJjxcOgAAAAAAAAAA.png}

\href{/docs/reference/layout/columns/}{\pandocbounded{\includesvg[keepaspectratio]{/assets/icons/16-arrow-right.svg}}}

{ Columns } { Previous page }

\href{/docs/reference/layout/fraction/}{\pandocbounded{\includesvg[keepaspectratio]{/assets/icons/16-arrow-right.svg}}}

{ Fraction } { Next page }
