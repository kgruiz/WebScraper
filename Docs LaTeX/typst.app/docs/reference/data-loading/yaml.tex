\title{typst.app/docs/reference/data-loading/yaml}

\begin{itemize}
\tightlist
\item
  \href{/docs}{\includesvg[width=0.16667in,height=0.16667in]{/assets/icons/16-docs-dark.svg}}
\item
  \includesvg[width=0.16667in,height=0.16667in]{/assets/icons/16-arrow-right.svg}
\item
  \href{/docs/reference/}{Reference}
\item
  \includesvg[width=0.16667in,height=0.16667in]{/assets/icons/16-arrow-right.svg}
\item
  \href{/docs/reference/data-loading/}{Data Loading}
\item
  \includesvg[width=0.16667in,height=0.16667in]{/assets/icons/16-arrow-right.svg}
\item
  \href{/docs/reference/data-loading/yaml/}{YAML}
\end{itemize}

\section{\texorpdfstring{\texttt{\ yaml\ }}{ yaml }}\label{summary}

Reads structured data from a YAML file.

The file must contain a valid YAML object or array. YAML mappings will
be converted into Typst dictionaries, and YAML sequences will be
converted into Typst arrays. Strings and booleans will be converted into
the Typst equivalents, null-values ( \texttt{\ null\ } ,
\texttt{\ \textasciitilde{}\ } or empty ``) will be converted into
\texttt{\ }{\texttt{\ none\ }}\texttt{\ } , and numbers will be
converted to floats or integers depending on whether they are whole
numbers. Custom YAML tags are ignored, though the loaded value will
still be present.

Be aware that integers larger than 2 \textsuperscript{63} -1 will be
converted to floating point numbers, which may give an approximative
value.

The YAML files in the example contain objects with authors as keys, each
with a sequence of their own submapping with the keys "title" and
"published"

\subsection{Example}\label{example}

\begin{verbatim}
#let bookshelf(contents) = {
  for (author, works) in contents {
    author
    for work in works [
      - #work.title (#work.published)
    ]
  }
}

#bookshelf(
  yaml("scifi-authors.yaml")
)
\end{verbatim}

\includegraphics[width=5in,height=\textheight,keepaspectratio]{/assets/docs/zhzvOjbNeHnb4ZYJg032GwAAAAAAAAAA.png}

\subsection{\texorpdfstring{{ Parameters
}}{ Parameters }}\label{parameters}

\phantomsection\label{parameters-tooltip}
Parameters are the inputs to a function. They are specified in
parentheses after the function name.

{ yaml } (

{ \href{/docs/reference/foundations/str/}{str} }

) -\textgreater{} { any }

\subsubsection{\texorpdfstring{\texttt{\ path\ }}{ path }}\label{parameters-path}

\href{/docs/reference/foundations/str/}{str}

{Required} {{ Positional }}

\phantomsection\label{parameters-path-positional-tooltip}
Positional parameters are specified in order, without names.

Path to a YAML file.

For more details, see the \href{/docs/reference/syntax/\#paths}{Paths
section} .

\subsection{\texorpdfstring{{ Definitions
}}{ Definitions }}\label{definitions}

\phantomsection\label{definitions-tooltip}
Functions and types and can have associated definitions. These are
accessed by specifying the function or type, followed by a period, and
then the definition\textquotesingle s name.

\subsubsection{\texorpdfstring{\texttt{\ decode\ }}{ decode }}\label{definitions-decode}

Reads structured data from a YAML string/bytes.

yaml { . } { decode } (

{ \href{/docs/reference/foundations/str/}{str}
\href{/docs/reference/foundations/bytes/}{bytes} }

) -\textgreater{} { any }

\paragraph{\texorpdfstring{\texttt{\ data\ }}{ data }}\label{definitions-decode-data}

\href{/docs/reference/foundations/str/}{str} {or}
\href{/docs/reference/foundations/bytes/}{bytes}

{Required} {{ Positional }}

\phantomsection\label{definitions-decode-data-positional-tooltip}
Positional parameters are specified in order, without names.

YAML data.

\subsubsection{\texorpdfstring{\texttt{\ encode\ }}{ encode }}\label{definitions-encode}

Encode structured data into a YAML string.

yaml { . } { encode } (

{ { any } }

) -\textgreater{} \href{/docs/reference/foundations/str/}{str}

\paragraph{\texorpdfstring{\texttt{\ value\ }}{ value }}\label{definitions-encode-value}

{ any }

{Required} {{ Positional }}

\phantomsection\label{definitions-encode-value-positional-tooltip}
Positional parameters are specified in order, without names.

Value to be encoded.

\href{/docs/reference/data-loading/xml/}{\pandocbounded{\includesvg[keepaspectratio]{/assets/icons/16-arrow-right.svg}}}

{ XML } { Previous page }

\href{/docs/guides/}{\pandocbounded{\includesvg[keepaspectratio]{/assets/icons/16-arrow-right.svg}}}

{ Guides } { Next page }
