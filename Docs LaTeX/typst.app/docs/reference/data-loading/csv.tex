\title{typst.app/docs/reference/data-loading/csv}

\begin{itemize}
\tightlist
\item
  \href{/docs}{\includesvg[width=0.16667in,height=0.16667in]{/assets/icons/16-docs-dark.svg}}
\item
  \includesvg[width=0.16667in,height=0.16667in]{/assets/icons/16-arrow-right.svg}
\item
  \href{/docs/reference/}{Reference}
\item
  \includesvg[width=0.16667in,height=0.16667in]{/assets/icons/16-arrow-right.svg}
\item
  \href{/docs/reference/data-loading/}{Data Loading}
\item
  \includesvg[width=0.16667in,height=0.16667in]{/assets/icons/16-arrow-right.svg}
\item
  \href{/docs/reference/data-loading/csv/}{CSV}
\end{itemize}

\section{\texorpdfstring{\texttt{\ csv\ }}{ csv }}\label{summary}

Reads structured data from a CSV file.

The CSV file will be read and parsed into a 2-dimensional array of
strings: Each row in the CSV file will be represented as an array of
strings, and all rows will be collected into a single array. Header rows
will not be stripped.

\subsection{Example}\label{example}

\begin{verbatim}
#let results = csv("example.csv")

#table(
  columns: 2,
  [*Condition*], [*Result*],
  ..results.flatten(),
)
\end{verbatim}

\includegraphics[width=5in,height=\textheight,keepaspectratio]{/assets/docs/wZK4j33X4RoMvhQZsQnpmQAAAAAAAAAA.png}

\subsection{\texorpdfstring{{ Parameters
}}{ Parameters }}\label{parameters}

\phantomsection\label{parameters-tooltip}
Parameters are the inputs to a function. They are specified in
parentheses after the function name.

{ csv } (

{ \href{/docs/reference/foundations/str/}{str} , } {
\hyperref[parameters-delimiter]{delimiter :}
\href{/docs/reference/foundations/str/}{str} , } {
\hyperref[parameters-row-type]{row-type :}
\href{/docs/reference/foundations/type/}{type} , }

) -\textgreater{} \href{/docs/reference/foundations/array/}{array}

\subsubsection{\texorpdfstring{\texttt{\ path\ }}{ path }}\label{parameters-path}

\href{/docs/reference/foundations/str/}{str}

{Required} {{ Positional }}

\phantomsection\label{parameters-path-positional-tooltip}
Positional parameters are specified in order, without names.

Path to a CSV file.

For more details, see the \href{/docs/reference/syntax/\#paths}{Paths
section} .

\subsubsection{\texorpdfstring{\texttt{\ delimiter\ }}{ delimiter }}\label{parameters-delimiter}

\href{/docs/reference/foundations/str/}{str}

The delimiter that separates columns in the CSV file. Must be a single
ASCII character.

Default: \texttt{\ }{\texttt{\ ","\ }}\texttt{\ }

\subsubsection{\texorpdfstring{\texttt{\ row-type\ }}{ row-type }}\label{parameters-row-type}

\href{/docs/reference/foundations/type/}{type}

How to represent the file\textquotesingle s rows.

\begin{itemize}
\tightlist
\item
  If set to \texttt{\ array\ } , each row is represented as a plain
  array of strings.
\item
  If set to \texttt{\ dictionary\ } , each row is represented as a
  dictionary mapping from header keys to strings. This option only makes
  sense when a header row is present in the CSV file.
\end{itemize}

Default: \texttt{\ array\ }

\subsection{\texorpdfstring{{ Definitions
}}{ Definitions }}\label{definitions}

\phantomsection\label{definitions-tooltip}
Functions and types and can have associated definitions. These are
accessed by specifying the function or type, followed by a period, and
then the definition\textquotesingle s name.

\subsubsection{\texorpdfstring{\texttt{\ decode\ }}{ decode }}\label{definitions-decode}

Reads structured data from a CSV string/bytes.

csv { . } { decode } (

{ \href{/docs/reference/foundations/str/}{str}
\href{/docs/reference/foundations/bytes/}{bytes} , } {
\hyperref[definitions-decode-parameters-delimiter]{delimiter :}
\href{/docs/reference/foundations/str/}{str} , } {
\hyperref[definitions-decode-parameters-row-type]{row-type :}
\href{/docs/reference/foundations/type/}{type} , }

) -\textgreater{} \href{/docs/reference/foundations/array/}{array}

\paragraph{\texorpdfstring{\texttt{\ data\ }}{ data }}\label{definitions-decode-data}

\href{/docs/reference/foundations/str/}{str} {or}
\href{/docs/reference/foundations/bytes/}{bytes}

{Required} {{ Positional }}

\phantomsection\label{definitions-decode-data-positional-tooltip}
Positional parameters are specified in order, without names.

CSV data.

\paragraph{\texorpdfstring{\texttt{\ delimiter\ }}{ delimiter }}\label{definitions-decode-delimiter}

\href{/docs/reference/foundations/str/}{str}

The delimiter that separates columns in the CSV file. Must be a single
ASCII character.

Default: \texttt{\ }{\texttt{\ ","\ }}\texttt{\ }

\paragraph{\texorpdfstring{\texttt{\ row-type\ }}{ row-type }}\label{definitions-decode-row-type}

\href{/docs/reference/foundations/type/}{type}

How to represent the file\textquotesingle s rows.

\begin{itemize}
\tightlist
\item
  If set to \texttt{\ array\ } , each row is represented as a plain
  array of strings.
\item
  If set to \texttt{\ dictionary\ } , each row is represented as a
  dictionary mapping from header keys to strings. This option only makes
  sense when a header row is present in the CSV file.
\end{itemize}

Default: \texttt{\ array\ }

\href{/docs/reference/data-loading/cbor/}{\pandocbounded{\includesvg[keepaspectratio]{/assets/icons/16-arrow-right.svg}}}

{ CBOR } { Previous page }

\href{/docs/reference/data-loading/json/}{\pandocbounded{\includesvg[keepaspectratio]{/assets/icons/16-arrow-right.svg}}}

{ JSON } { Next page }
