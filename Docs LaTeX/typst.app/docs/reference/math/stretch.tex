\title{typst.app/docs/reference/math/stretch}

\begin{itemize}
\tightlist
\item
  \href{/docs}{\includesvg[width=0.16667in,height=0.16667in]{/assets/icons/16-docs-dark.svg}}
\item
  \includesvg[width=0.16667in,height=0.16667in]{/assets/icons/16-arrow-right.svg}
\item
  \href{/docs/reference/}{Reference}
\item
  \includesvg[width=0.16667in,height=0.16667in]{/assets/icons/16-arrow-right.svg}
\item
  \href{/docs/reference/math/}{Math}
\item
  \includesvg[width=0.16667in,height=0.16667in]{/assets/icons/16-arrow-right.svg}
\item
  \href{/docs/reference/math/stretch/}{Stretch}
\end{itemize}

\section{\texorpdfstring{\texttt{\ stretch\ } {{ Element
}}}{ stretch   Element }}\label{summary}

\phantomsection\label{element-tooltip}
Element functions can be customized with \texttt{\ set\ } and
\texttt{\ show\ } rules.

Stretches a glyph.

This function can also be used to automatically stretch the base of an
attachment, so that it fits the top and bottom attachments.

Note that only some glyphs can be stretched, and which ones can depend
on the math font being used. However, most math fonts are the same in
this regard.

\begin{verbatim}
$ H stretch(=)^"define" U + p V $
$ f : X stretch(->>, size: #150%)_"surjective" Y $
$ x stretch(harpoons.ltrb, size: #3em) y
    stretch(\[, size: #150%) z $
\end{verbatim}

\includegraphics[width=5in,height=\textheight,keepaspectratio]{/assets/docs/s6743QhH3-etZ1y_QW-bLAAAAAAAAAAA.png}

\subsection{\texorpdfstring{{ Parameters
}}{ Parameters }}\label{parameters}

\phantomsection\label{parameters-tooltip}
Parameters are the inputs to a function. They are specified in
parentheses after the function name.

math { . } { stretch } (

{ \href{/docs/reference/foundations/content/}{content} , } {
\hyperref[parameters-size]{size :}
\href{/docs/reference/foundations/auto/}{auto}
\href{/docs/reference/layout/relative/}{relative} , }

) -\textgreater{} \href{/docs/reference/foundations/content/}{content}

\subsubsection{\texorpdfstring{\texttt{\ body\ }}{ body }}\label{parameters-body}

\href{/docs/reference/foundations/content/}{content}

{Required} {{ Positional }}

\phantomsection\label{parameters-body-positional-tooltip}
Positional parameters are specified in order, without names.

The glyph to stretch.

\subsubsection{\texorpdfstring{\texttt{\ size\ }}{ size }}\label{parameters-size}

\href{/docs/reference/foundations/auto/}{auto} {or}
\href{/docs/reference/layout/relative/}{relative}

{{ Settable }}

\phantomsection\label{parameters-size-settable-tooltip}
Settable parameters can be customized for all following uses of the
function with a \texttt{\ set\ } rule.

The size to stretch to, relative to the maximum size of the glyph and
its attachments.

Default: \texttt{\ }{\texttt{\ auto\ }}\texttt{\ }

\href{/docs/reference/math/sizes/}{\pandocbounded{\includesvg[keepaspectratio]{/assets/icons/16-arrow-right.svg}}}

{ Sizes } { Previous page }

\href{/docs/reference/math/styles/}{\pandocbounded{\includesvg[keepaspectratio]{/assets/icons/16-arrow-right.svg}}}

{ Styles } { Next page }
