\title{typst.app/docs/tutorial/writing-in-typst}

\begin{itemize}
\tightlist
\item
  \href{/docs}{\includesvg[width=0.16667in,height=0.16667in]{/assets/icons/16-docs-dark.svg}}
\item
  \includesvg[width=0.16667in,height=0.16667in]{/assets/icons/16-arrow-right.svg}
\item
  \href{/docs/tutorial/}{Tutorial}
\item
  \includesvg[width=0.16667in,height=0.16667in]{/assets/icons/16-arrow-right.svg}
\item
  \href{/docs/tutorial/writing-in-typst/}{Writing in Typst}
\end{itemize}

\section{Writing in Typst}\label{writing-in-typst}

Let\textquotesingle s get started! Suppose you got assigned to write a
technical report for university. It will contain prose, maths, headings,
and figures. To get started, you create a new project on the Typst app.
You\textquotesingle ll be taken to the editor where you see two panels:
A source panel where you compose your document and a preview panel where
you see the rendered document.

\pandocbounded{\includegraphics[keepaspectratio]{/assets/docs/1-writing-app.png}}

You already have a good angle for your report in mind. So
let\textquotesingle s start by writing the introduction. Enter some text
in the editor panel. You\textquotesingle ll notice that the text
immediately appears on the previewed page.

\begin{verbatim}
In this report, we will explore the
various factors that influence fluid
dynamics in glaciers and how they
contribute to the formation and
behaviour of these natural structures.
\end{verbatim}

\includegraphics[width=5in,height=\textheight,keepaspectratio]{/assets/docs/ePl1U-2a7w8qkmb3CLl_oAAAAAAAAAAA.png}

\emph{Throughout this tutorial, we\textquotesingle ll show code examples
like this one. Just like in the app, the first panel contains markup and
the second panel shows a preview. We shrunk the page to fit the examples
so you can see what\textquotesingle s going on.}

The next step is to add a heading and emphasize some text. Typst uses
simple markup for the most common formatting tasks. To add a heading,
enter the \texttt{\ =\ } character and to emphasize some text with
italics, enclose it in
\texttt{\ }{\texttt{\ \_underscores\_\ }}\texttt{\ } .

\begin{verbatim}
= Introduction
In this report, we will explore the
various factors that influence _fluid
dynamics_ in glaciers and how they
contribute to the formation and
behaviour of these natural structures.
\end{verbatim}

\includegraphics[width=5in,height=\textheight,keepaspectratio]{/assets/docs/p75v-z7QqVChplB2N8HZfwAAAAAAAAAA.png}

That was easy! To add a new paragraph, just add a blank line in between
two lines of text. If that paragraph needs a subheading, produce it by
typing \texttt{\ ==\ } instead of \texttt{\ =\ } . The number of
\texttt{\ =\ } characters determines the nesting level of the heading.

Now we want to list a few of the circumstances that influence glacier
dynamics. To do that, we use a numbered list. For each item of the list,
we type a \texttt{\ +\ } character at the beginning of the line. Typst
will automatically number the items.

\begin{verbatim}
+ The climate
+ The topography
+ The geology
\end{verbatim}

\includegraphics[width=5in,height=\textheight,keepaspectratio]{/assets/docs/U3IHQbhSNQ8ndkXIv_gPrgAAAAAAAAAA.png}

If we wanted to add a bulleted list, we would use the \texttt{\ -\ }
character instead of the \texttt{\ +\ } character. We can also nest
lists: For example, we can add a sub-list to the first item of the list
above by indenting it.

\begin{verbatim}
+ The climate
  - Temperature
  - Precipitation
+ The topography
+ The geology
\end{verbatim}

\includegraphics[width=5in,height=\textheight,keepaspectratio]{/assets/docs/xmS-BPiM_gDHkWk9_uhE_gAAAAAAAAAA.png}

\subsection{Adding a figure}\label{figure}

You think that your report would benefit from a figure.
Let\textquotesingle s add one. Typst supports images in the formats PNG,
JPEG, GIF, and SVG. To add an image file to your project, first open the
\emph{file panel} by clicking the box icon in the left sidebar. Here,
you can see a list of all files in your project. Currently, there is
only one: The main Typst file you are writing in. To upload another
file, click the button with the arrow in the top-right corner. This
opens the upload dialog, in which you can pick files to upload from your
computer. Select an image file for your report.

\pandocbounded{\includegraphics[keepaspectratio]{/assets/docs/1-writing-upload.png}}

We have seen before that specific symbols (called \emph{markup} ) have
specific meaning in Typst. We can use \texttt{\ =\ } , \texttt{\ -\ } ,
\texttt{\ +\ } , and \texttt{\ \_\ } to create headings, lists and
emphasized text, respectively. However, having a special symbol for
everything we want to insert into our document would soon become cryptic
and unwieldy. For this reason, Typst reserves markup symbols only for
the most common things. Everything else is inserted with
\emph{functions.} For our image to show up on the page, we use
Typst\textquotesingle s
\href{/docs/reference/visualize/image/}{\texttt{\ image\ }} function.

\begin{verbatim}
#image("glacier.jpg")
\end{verbatim}

\includegraphics[width=5in,height=\textheight,keepaspectratio]{/assets/docs/KwKlYCVb2uFZqZ8abt3-ggAAAAAAAAAA.png}

In general, a function produces some output for a set of
\emph{arguments} . When you \emph{call} a function within markup, you
provide the arguments and Typst inserts the result (the
function\textquotesingle s \emph{return value} ) into the document. In
our case, the \texttt{\ image\ } function takes one argument: The path
to the image file. To call a function in markup, we first need to type
the \texttt{\ \#\ } character, immediately followed by the name of the
function. Then, we enclose the arguments in parentheses. Typst
recognizes many different data types within argument lists. Our file
path is a short \href{/docs/reference/foundations/str/}{string of text}
, so we need to enclose it in double quotes.

The inserted image uses the whole width of the page. To change that,
pass the \texttt{\ width\ } argument to the \texttt{\ image\ } function.
This is a \emph{named} argument and therefore specified as a
\texttt{\ name:\ value\ } pair. If there are multiple arguments, they
are separated by commas, so we first need to put a comma behind the
path.

\begin{verbatim}
#image("glacier.jpg", width: 70%)
\end{verbatim}

\includegraphics[width=5in,height=\textheight,keepaspectratio]{/assets/docs/lpadKIOzcEsf_MGoSeZghAAAAAAAAAAA.png}

The \texttt{\ width\ } argument is a
\href{/docs/reference/layout/relative/}{relative length} . In our case,
we specified a percentage, determining that the image shall take up
\texttt{\ }{\texttt{\ 70\%\ }}\texttt{\ } of the page\textquotesingle s
width. We also could have specified an absolute value like
\texttt{\ }{\texttt{\ 1cm\ }}\texttt{\ } or
\texttt{\ }{\texttt{\ 0.7in\ }}\texttt{\ } .

Just like text, the image is now aligned at the left side of the page by
default. It\textquotesingle s also lacking a caption.
Let\textquotesingle s fix that by using the
\href{/docs/reference/model/figure/}{figure} function. This function
takes the figure\textquotesingle s contents as a positional argument and
an optional caption as a named argument.

Within the argument list of the \texttt{\ figure\ } function, Typst is
already in code mode. This means, you now have to remove the hash before
the image function call. The hash is only needed directly in markup (to
disambiguate text from function calls).

The caption consists of arbitrary markup. To give markup to a function,
we enclose it in square brackets. This construct is called a
\emph{content block.}

\begin{verbatim}
#figure(
  image("glacier.jpg", width: 70%),
  caption: [
    _Glaciers_ form an important part
    of the earth's climate system.
  ],
)
\end{verbatim}

\includegraphics[width=5in,height=\textheight,keepaspectratio]{/assets/docs/v5OnReUO8fD5Rfj2aJZVyQAAAAAAAAAA.png}

You continue to write your report and now want to reference the figure.
To do that, first attach a label to figure. A label uniquely identifies
an element in your document. Add one after the figure by enclosing some
name in angle brackets. You can then reference the figure in your text
by writing an \texttt{\ }{\texttt{\ @\ }}\texttt{\ } symbol followed by
that name. Headings and equations can also be labelled to make them
referenceable.

\begin{verbatim}
Glaciers as the one shown in
@glaciers will cease to exist if
we don't take action soon!

#figure(
  image("glacier.jpg", width: 70%),
  caption: [
    _Glaciers_ form an important part
    of the earth's climate system.
  ],
) <glaciers>
\end{verbatim}

\includegraphics[width=5in,height=\textheight,keepaspectratio]{/assets/docs/cwZ12iQ39B4L-_wQwhO2TAAAAAAAAAAA.png}

So far, we\textquotesingle ve passed content blocks (markup in square
brackets) and strings (text in double quotes) to our functions. Both
seem to contain text. What\textquotesingle s the difference?

A content block can contain text, but also any other kind of markup,
function calls, and more, whereas a string is really just a
\emph{sequence of characters} and nothing else.

For example, the image function expects a path to an image file. It
would not make sense to pass, e.g., a paragraph of text or another image
as the image\textquotesingle s path parameter. That\textquotesingle s
why only strings are allowed here. On the contrary, strings work
wherever content is expected because text is a valid kind of content.

\subsection{Adding a bibliography}\label{bibliography}

As you write up your report, you need to back up some of your claims.
You can add a bibliography to your document with the
\href{/docs/reference/model/bibliography/}{\texttt{\ bibliography\ }}
function. This function expects a path to a bibliography file.

Typst\textquotesingle s native bibliography format is
\href{https://github.com/typst/hayagriva/blob/main/docs/file-format.md}{Hayagriva}
, but for compatibility you can also use BibLaTeX files. As your
classmate has already done a literature survey and sent you a
\texttt{\ .bib\ } file, you\textquotesingle ll use that one. Upload the
file through the file panel to access it in Typst.

Once the document contains a bibliography, you can start citing from it.
Citations use the same syntax as references to a label. As soon as you
cite a source for the first time, it will appear in the bibliography
section of your document. Typst supports different citation and
bibliography styles. Consult the
\href{/docs/reference/model/bibliography/\#parameters-style}{reference}
for more details.

\begin{verbatim}
= Methods
We follow the glacier melting models
established in @glacier-melt.

#bibliography("works.bib")
\end{verbatim}

\includegraphics[width=5in,height=\textheight,keepaspectratio]{/assets/docs/QGPHT14ksdea0r_8vy01WAAAAAAAAAAA.png}

\subsection{Maths}\label{maths}

After fleshing out the methods section, you move on to the meat of the
document: Your equations. Typst has built-in mathematical typesetting
and uses its own math notation. Let\textquotesingle s start with a
simple equation. We wrap it in \texttt{\ \$\ } signs to let Typst know
it should expect a mathematical expression:

\begin{verbatim}
The equation $Q = rho A v + C$
defines the glacial flow rate.
\end{verbatim}

\includegraphics[width=5in,height=\textheight,keepaspectratio]{/assets/docs/_u5BjLoMFBZU2zg1OWULdgAAAAAAAAAA.png}

The equation is typeset inline, on the same line as the surrounding
text. If you want to have it on its own line instead, you should insert
a single space at its start and end:

\begin{verbatim}
The flow rate of a glacier is
defined by the following equation:

$ Q = rho A v + C $
\end{verbatim}

\includegraphics[width=5in,height=\textheight,keepaspectratio]{/assets/docs/GXI0mvGOqqSC165iRTK-QwAAAAAAAAAA.png}

We can see that Typst displayed the single letters \texttt{\ Q\ } ,
\texttt{\ A\ } , \texttt{\ v\ } , and \texttt{\ C\ } as-is, while it
translated \texttt{\ rho\ } into a Greek letter. Math mode will always
show single letters verbatim. Multiple letters, however, are interpreted
as symbols, variables, or function names. To imply a multiplication
between single letters, put spaces between them.

If you want to have a variable that consists of multiple letters, you
can enclose it in quotes:

\begin{verbatim}
The flow rate of a glacier is given
by the following equation:

$ Q = rho A v + "time offset" $
\end{verbatim}

\includegraphics[width=5in,height=\textheight,keepaspectratio]{/assets/docs/JSaojGBiKH-FLbIYqeWSgAAAAAAAAAAA.png}

You\textquotesingle ll also need a sum formula in your paper. We can use
the \texttt{\ sum\ } symbol and then specify the range of the summation
in sub- and superscripts:

\begin{verbatim}
Total displaced soil by glacial flow:

$ 7.32 beta +
  sum_(i=0)^nabla Q_i / 2 $
\end{verbatim}

\includegraphics[width=5in,height=\textheight,keepaspectratio]{/assets/docs/rTDyTGxJlXKPRHJub3ALRgAAAAAAAAAA.png}

To add a subscript to a symbol or variable, type a \texttt{\ \_\ }
character and then the subscript. Similarly, use the \texttt{\ \^{}\ }
character for a superscript. If your sub- or superscript consists of
multiple things, you must enclose them in round parentheses.

The above example also showed us how to insert fractions: Simply put a
\texttt{\ /\ } character between the numerator and the denominator and
Typst will automatically turn it into a fraction. Parentheses are
smartly resolved, so you can enter your expression as you would into a
calculator and Typst will replace parenthesized sub-expressions with the
appropriate notation.

\begin{verbatim}
Total displaced soil by glacial flow:

$ 7.32 beta +
  sum_(i=0)^nabla
    (Q_i (a_i - epsilon)) / 2 $
\end{verbatim}

\includegraphics[width=5in,height=\textheight,keepaspectratio]{/assets/docs/HgeB2Bx5Lh3a5NPfF6WEdwAAAAAAAAAA.png}

Not all math constructs have special syntax. Instead, we use functions,
just like the \texttt{\ image\ } function we have seen before. For
example, to insert a column vector, we can use the
\href{/docs/reference/math/vec/}{\texttt{\ vec\ }} function. Within math
mode, function calls don\textquotesingle t need to start with the
\texttt{\ \#\ } character.

\begin{verbatim}
$ v := vec(x_1, x_2, x_3) $
\end{verbatim}

\includegraphics[width=5in,height=\textheight,keepaspectratio]{/assets/docs/nj0pMnkuoX2t5FZ3_x6YKwAAAAAAAAAA.png}

Some functions are only available within math mode. For example, the
\href{/docs/reference/math/variants/\#functions-cal}{\texttt{\ cal\ }}
function is used to typeset calligraphic letters commonly used for sets.
The \href{/docs/reference/math/}{math section of the reference} provides
a complete list of all functions that math mode makes available.

One more thing: Many symbols, such as the arrow, have a lot of variants.
You can select among these variants by appending a dot and a modifier
name to a symbol\textquotesingle s name:

\begin{verbatim}
$ a arrow.squiggly b $
\end{verbatim}

\includegraphics[width=5in,height=\textheight,keepaspectratio]{/assets/docs/0GgQitNz41j-75F3FS6iAwAAAAAAAAAA.png}

This notation is also available in markup mode, but the symbol name must
be preceded with \texttt{\ \#sym.\ } there. See the
\href{/docs/reference/symbols/sym/}{symbols section} for a list of all
available symbols.

\subsection{Review}\label{review}

You have now seen how to write a basic document in Typst. You learned
how to emphasize text, write lists, insert images, align content, and
typeset mathematical expressions. You also learned about
Typst\textquotesingle s functions. There are many more kinds of content
that Typst lets you insert into your document, such as
\href{/docs/reference/model/table/}{tables} ,
\href{/docs/reference/visualize/}{shapes} , and
\href{/docs/reference/text/raw/}{code blocks} . You can peruse the
\href{/docs/reference/}{reference} to learn more about these and other
features.

For the moment, you have completed writing your report. You have already
saved a PDF by clicking on the download button in the top right corner.
However, you think the report could look a bit less plain. In the next
section, we\textquotesingle ll learn how to customize the look of our
document.

\href{/docs/tutorial/}{\pandocbounded{\includesvg[keepaspectratio]{/assets/icons/16-arrow-right.svg}}}

{ Tutorial } { Previous page }

\href{/docs/tutorial/formatting/}{\pandocbounded{\includesvg[keepaspectratio]{/assets/icons/16-arrow-right.svg}}}

{ Formatting } { Next page }
