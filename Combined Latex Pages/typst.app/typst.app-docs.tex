\title{typst.app/docs/reference}

\begin{itemize}
\tightlist
\item
  \href{/docs}{\includesvg[width=0.16667in,height=0.16667in]{/assets/icons/16-docs-dark.svg}}
\item
  \includesvg[width=0.16667in,height=0.16667in]{/assets/icons/16-arrow-right.svg}
\item
  \href{/docs/reference/}{Reference}
\end{itemize}

\section{Reference}\label{reference}

This reference documentation is a comprehensive guide to all of
Typst\textquotesingle s syntax, concepts, types, and functions. If you
are completely new to Typst, we recommend starting with the
\href{/docs/tutorial/}{tutorial} and then coming back to the reference
to learn more about Typst\textquotesingle s features as you need them.

\subsection{Language}\label{language}

The reference starts with a language part that gives an overview over
\href{/docs/reference/syntax/}{Typst\textquotesingle s syntax} and
contains information about concepts involved in
\href{/docs/reference/styling/}{styling documents,} using
\href{/docs/reference/scripting/}{Typst\textquotesingle s scripting
capabilities.}

\subsection{Functions}\label{functions}

The second part includes chapters on all functions used to insert,
style, transform, and layout content in Typst documents. Each function
is documented with a description of its purpose, a list of its
parameters, and examples of how to use it.

The final part of the reference explains all functions that are used
within Typst\textquotesingle s code mode to manipulate and transform
data. Just as in the previous part, each function is documented with a
description of its purpose, a list of its parameters, and examples of
how to use it.

\href{/docs/tutorial/making-a-template/}{\pandocbounded{\includesvg[keepaspectratio]{/assets/icons/16-arrow-right.svg}}}

{ Making a Template } { Previous page }

\href{/docs/reference/syntax/}{\pandocbounded{\includesvg[keepaspectratio]{/assets/icons/16-arrow-right.svg}}}

{ Syntax } { Next page }


\title{typst.app/docs/tutorial}

\begin{itemize}
\tightlist
\item
  \href{/docs}{\includesvg[width=0.16667in,height=0.16667in]{/assets/icons/16-docs-dark.svg}}
\item
  \includesvg[width=0.16667in,height=0.16667in]{/assets/icons/16-arrow-right.svg}
\item
  \href{/docs/tutorial/}{Tutorial}
\end{itemize}

\section{Tutorial}\label{tutorial}

Welcome to Typst\textquotesingle s tutorial! In this tutorial, you will
learn how to write and format documents in Typst. We will start with
everyday tasks and gradually introduce more advanced features. This
tutorial does not assume prior knowledge of Typst, other markup
languages, or programming. We do assume that you know how to edit a text
file.

The best way to start is to sign up to the Typst app for free and follow
along with the steps below. The app gives you instant preview, syntax
highlighting and helpful autocompletions. Alternatively, you can follow
along in your local text editor with the
\href{https://github.com/typst/typst}{open-source CLI} .

\subsection{When to use Typst}\label{when-typst}

Before we get started, let\textquotesingle s check what Typst is and
when to use it. Typst is a markup language for typesetting documents. It
is designed to be easy to learn, fast, and versatile. Typst takes text
files with markup in them and outputs PDFs.

Typst is a good choice for writing any long form text such as essays,
articles, scientific papers, books, reports, and homework assignments.
Moreover, Typst is a great fit for any documents containing mathematical
notation, such as papers in the math, physics, and engineering fields.
Finally, due to its strong styling and automation features, it is an
excellent choice for any set of documents that share a common style,
such as a book series.

\subsection{What you will learn}\label{learnings}

This tutorial has four chapters. Each chapter builds on the previous
one. Here is what you will learn in each of them:

\begin{enumerate}
\tightlist
\item
  \href{/docs/tutorial/writing-in-typst/}{Writing in Typst:} Learn how
  to write text and insert images, equations, and other elements.
\item
  \href{/docs/tutorial/formatting/}{Formatting:} Learn how to adjust the
  formatting of your document, including font size, heading styles, and
  more.
\item
  \href{/docs/tutorial/advanced-styling/}{Advanced Styling:} Create a
  complex page layout for a scientific paper with typographic features
  such as an author list and run-in headings.
\item
  \href{/docs/tutorial/making-a-template/}{Making a Template:} Build a
  reusable template from the paper you created in the previous chapter.
\end{enumerate}

We hope you\textquotesingle ll enjoy Typst!

\href{/docs/}{\pandocbounded{\includesvg[keepaspectratio]{/assets/icons/16-arrow-right.svg}}}

{ Overview } { Previous page }

\href{/docs/tutorial/writing-in-typst/}{\pandocbounded{\includesvg[keepaspectratio]{/assets/icons/16-arrow-right.svg}}}

{ Writing in Typst } { Next page }


\title{typst.app/docs/community}

\begin{itemize}
\tightlist
\item
  \href{/docs}{\includesvg[width=0.16667in,height=0.16667in]{/assets/icons/16-docs-dark.svg}}
\item
  \includesvg[width=0.16667in,height=0.16667in]{/assets/icons/16-arrow-right.svg}
\item
  \href{/docs/community/}{Community}
\end{itemize}

\section{Community}\label{community}

Hey and welcome to the Community page! We\textquotesingle re so glad
you\textquotesingle re here. Typst is developed by an early-stage
startup and it is still early days, but it would be pointless without
people like you who are interested in it.

We would love to not only hear from you but to also provide spaces where
you can discuss any topic around Typst, typesetting, writing, the
sciences, and typography with other likeminded people.

\textbf{Our \href{https://forum.typst.app/}{Forum} is the best place to
get answers for questions on Typst and to show off your creations.} If
you would like to chat with the community and shape the future
development of Typst, we would like to also invite you to our
\href{https://discord.gg/2uDybryKPe}{Discord server} . We coordinate our
Open-Source work there, but you can also iterate on Typst projects and
discuss off-topic things with the community members. Both the Forum and
the Discord server are open for everyone. Of course, you are also very
welcome to connect with us on social media (
\href{https://mastodon.social/@typst}{Mastodon} ,
\href{https://bsky.app/profile/typst.app}{Bluesky} ,
\href{https://instagram.com/typstapp/}{Instagram} ,
\href{https://linkedin.com/company/typst}{LinkedIn} , and
\href{https://github.com/typst}{GitHub} ).

\subsection{What to share?}\label{what-to-share}

For our community, we want to foster versatility and inclusivity. You
are welcome to post about any topic that you think would interest other
community members, but if you need a little inspiration, here are a few
ideas:

\begin{itemize}
\tightlist
\item
  Share and discuss your thoughts and ideas for new features or
  improvements you\textquotesingle d like to see in Typst
\item
  Showcase documents you\textquotesingle ve created with Typst, or share
  any unique or creative ways you\textquotesingle ve used the platform
\item
  Share importable files or templates that you use to style your
  documents
\item
  Alert us of bugs you encounter while using Typst
\end{itemize}

\subsection{Following the development}\label{following-the-development}

Typst is still under very active development and breaking changes can
occur at any point. The compiler is developed in the open on
\href{https://github.com/typst/typst}{GitHub} .

We will update the members of our Discord server and our social media
followers when new features become available. We\textquotesingle ll also
update you on the development progress of large features.

\subsection{How to support Typst}\label{support-typst}

If you want to support Typst, there are multiple ways to do that! You
can \href{https://github.com/typst/typst}{contribute to the code} or
\href{https://github.com/search?q=repo\%3Atypst\%2Ftypst+impl+LocalName+for&type=code}{translate
the strings in Typst} to your native language if it\textquotesingle s
not supported yet. You can also help us by
\href{https://typst.app/pricing}{subscribing to the paid tier of our web
app} or \href{https://github.com/sponsors/typst}{sponsoring our Open
Source efforts!} Multiple recurring sponsorship tiers are available and
all of them come with a set of goodies. No matter how you contribute,
thank you for your support!

\subsection{Community Rules}\label{rules}

We want to make our community a safe and inclusive space for everyone.
Therefore, we will not tolerate any sexual harassment, sexism, political
attacks, derogatory language or personal insults, racism, doxing, and
other inappropriate behaviour. We pledge to remove members that are in
violation of these rules. \href{https://typst.app/contact/}{Contact us}
if you think another community member acted inappropriately towards you.
All complaints will be reviewed and investigated promptly and fairly.

In addition, our \href{https://typst.app/privacy/}{privacy policy}
applies on all community spaces operated by us, such as the Discord
server. Please also note that the terms of service and privacy policies
of the respective services apply.

\subsection{See you soon!}\label{see-you}

Thanks again for learning more about Typst. We would be delighted to
meet you on our \href{https://discord.gg/2uDybryKPe}{Discord server} !

\href{/docs/roadmap/}{\pandocbounded{\includesvg[keepaspectratio]{/assets/icons/16-arrow-right.svg}}}

{ Roadmap } { Previous page }


\title{typst.app/docs/roadmap}

\begin{itemize}
\tightlist
\item
  \href{/docs}{\includesvg[width=0.16667in,height=0.16667in]{/assets/icons/16-docs-dark.svg}}
\item
  \includesvg[width=0.16667in,height=0.16667in]{/assets/icons/16-arrow-right.svg}
\item
  \href{/docs/roadmap/}{Roadmap}
\end{itemize}

\section{Roadmap}\label{roadmap}

This page lists planned features for the Typst language, compiler,
library and web app. Since priorities and development realities change,
this roadmap is not set in stone. Features that are listed here will not
necessarily be implemented and features that will be implemented might
be missing here. Moreover, this roadmap only lists larger, more
fundamental features and bugs.

Are you missing something on the roadmap? Typst relies on your feedback
as a user to plan for and prioritize new features. Get started by filing
a new issue on \href{https://github.com/typst/typst/issues}{GitHub} or
discuss your feature request with the
\href{https://typst.app/docs/community}{community} .

\subsection{Language and Compiler}\label{language-and-compiler}

\begin{itemize}
\item
  \textbf{Styling}

  \begin{itemize}
  \tightlist
  \item
    Support for revoking style rules
  \item
    Ancestry selectors (e.g., within)
  \item
    \st{Fix show rule recursion crashes}
  \item
    \st{Fix show-set issues}
  \end{itemize}
\item
  \textbf{Scripting}

  \begin{itemize}
  \tightlist
  \item
    Function for debug logging
  \item
    Fix issues with paths being strings
  \item
    Custom types (that work with set and show rules)
  \item
    Type hints
  \item
    Function hoisting (if feasible)
  \item
    \st{Data loading functions}
  \item
    \st{Support for compiler warnings}
  \item
    \st{Types as first-class values}
  \item
    \st{More fields and methods on primitives}
  \item
    \st{WebAssembly plugins}
  \item
    \st{Get values of set rules}
  \item
    \st{Replace locate, etc. with unified context system}
  \item
    \st{Allow expressions as dictionary keys}
  \item
    \st{Package management}
  \end{itemize}
\item
  \textbf{Model}

  \begin{itemize}
  \tightlist
  \item
    Fix issues with numbering patterns
  \item
    Support a path or bytes in places that currently only support paths,
    superseding \texttt{\ .decode\ } scoped functions
  \item
    Better support for custom referenceable things
  \item
    Richer built-in outline customization
  \item
    Enum continuation
  \item
    \st{Bibliography and citation customization via CSL (Citation Style
    Language)}
  \item
    \st{Relative counters, e.g. for figure numbering per section}
  \end{itemize}
\item
  \textbf{Text}

  \begin{itemize}
  \tightlist
  \item
    Font fallback warnings
  \item
    Bold, italic, and smallcaps synthesis
  \item
    Variable fonts support
  \item
    Ruby and Warichu
  \item
    Kashida justification
  \item
    \st{Support for basic CJK text layout rules}
  \item
    \st{Fix SVG font fallback}
  \item
    \st{Themes for raw text and more/custom syntaxes}
  \end{itemize}
\item
  \textbf{Math}

  \begin{itemize}
  \tightlist
  \item
    Fix syntactic quirks
  \item
    Fix single letter strings
  \item
    Fix font handling
  \item
    Fix attachment parsing priorities
  \item
    Provide more primitives
  \item
    Improve equation numbering
  \item
    Big fractions
  \end{itemize}
\item
  \textbf{Layout}

  \begin{itemize}
  \tightlist
  \item
    Fix footnote issues
  \item
    Fix issues with list (in particular baselines \& alignment)
  \item
    Support for "sticky" blocks that stay with the next one
  \item
    Improve widow \& orphan prevention
  \item
    Expand floating layout (e.g. over two columns)
  \item
    Support for side-floats and other "collision" layouts
  \item
    Better support for more canvas-like layouts
  \item
    Unified layout primitives across normal content and math
  \item
    Page adjustment from within flow
  \item
    Chained layout regions
  \item
    Grid-based typesetting
  \item
    Balanced columns
  \item
    Drop caps
  \item
    End notes, maybe margin notes
  \item
    \st{Footnotes}
  \item
    \st{Basic floating layout}
  \item
    \st{Row span and column span in table}
  \item
    \st{Per-cell table stroke customization}
  \end{itemize}
\item
  \textbf{Visualize}

  \begin{itemize}
  \tightlist
  \item
    Arrows
  \item
    Better path drawing, possibly path operations
  \item
    Color management
  \item
    \st{More configurable strokes}
  \item
    \st{Gradients}
  \item
    \st{Patterns}
  \end{itemize}
\item
  \textbf{Introspection}

  \begin{itemize}
  \tightlist
  \item
    Support for freezing content, so that e.g. numbers in it remain the
    same if it appears multiple times
  \end{itemize}
\item
  \textbf{Export}

  \begin{itemize}
  \tightlist
  \item
    PDF/A support
  \item
    HTML export
  \item
    Tagged PDF for Accessibility
  \item
    PDF/X support
  \item
    EPUB export
  \item
    \st{PNG export}
  \item
    \st{SVG export}
  \item
    \st{Support for transparency in PDF}
  \item
    \st{Fix issues with SVGs in PDF}
  \item
    \st{Fix emoji export in PDF} (not yet released)
  \item
    \st{Selectable text in SVGs in PDF} (not yet released)
  \item
    \st{Better font subsetting for smaller PDFs} (not yet released)
  \end{itemize}
\item
  \textbf{CLI}

  \begin{itemize}
  \tightlist
  \item
    Support for downloading fonts on-demand automatically
  \item
    \st{\mbox{\texttt{\ typst\ query\ }} for querying document elements}
  \item
    \st{\mbox{\texttt{\ typst\ init\ }} for creating a project from a
    template}
  \item
    \st{\mbox{\texttt{\ typst\ update\ }} for self-updating the CLI}
  \end{itemize}
\item
  \textbf{Tooling}

  \begin{itemize}
  \tightlist
  \item
    Documentation generator and doc comments
  \item
    Autoformatter
  \item
    Linter
  \end{itemize}
\item
  \textbf{Performance}

  \begin{itemize}
  \tightlist
  \item
    Reduce memory usage
  \item
    \st{Optimize runtime of optimal paragraph layout} (not yet released)
  \item
    \st{Parallelize layout engine} (not yet released)
  \end{itemize}
\item
  \textbf{Development}

  \begin{itemize}
  \tightlist
  \item
    Benchmarking
  \item
    Better contributor documentation
  \end{itemize}
\end{itemize}

\subsection{Web App}\label{web-app}

\begin{itemize}
\item
  \textbf{Editing}

  \begin{itemize}
  \tightlist
  \item
    Smarter \& more action buttons
  \item
    Inline documentation
  \item
    Preview autocomplete entry
  \item
    Color Picker
  \item
    Symbol picker
  \item
    Basic, built-in image editor (cropping, etc.)
  \item
    GUI inspector for editing function calls
  \item
    Cursor in preview
  \item
    \st{Hover tooltips for debugging}
  \item
    \st{Scroll to cursor position in preview}
  \item
    \st{Folders in projects}
  \item
    \st{Outline panel}
  \item
    \st{More export options}
  \item
    \st{Preview in a separate window}
  \item
    \st{Sync literature with Zotero and Mendely}
  \item
    \st{Paste modal}
  \item
    \st{Improve panel}
  \end{itemize}
\item
  \textbf{Writing}

  \begin{itemize}
  \tightlist
  \item
    Word count
  \item
    Structure view
  \item
    Text completion by LLM
  \item
    \st{Spell check}
  \end{itemize}
\item
  \textbf{Collaboration}

  \begin{itemize}
  \tightlist
  \item
    Change tracking
  \item
    Version history
  \item
    \st{Chat-like comments}
  \item
    \st{Git integration}
  \end{itemize}
\item
  \textbf{Project management}

  \begin{itemize}
  \tightlist
  \item
    Drag-and-drop for projects
  \item
    Template generation by LLM
  \item
    \st{LaTeX, Word, Markdown import}
  \item
    \st{Thumbnails for projects}
  \end{itemize}
\item
  \textbf{Settings}

  \begin{itemize}
  \tightlist
  \item
    Keyboard shortcuts configuration
  \item
    Better project settings
  \item
    Avatar Cropping
  \item
    \st{System Theme setting}
  \end{itemize}
\item
  \textbf{Other}

  \begin{itemize}
  \tightlist
  \item
    Offline PWA
  \item
    Mobile improvements
  \item
    Two-Factor Authentication
  \item
    Advanced search in projects
  \item
    Private packages in teams
  \item
    \st{LDAP Single sign-on}
  \item
    \st{Compiler version picker}
  \item
    \st{Presentation mode}
  \item
    \st{Support for On-Premises deployment}
  \item
    \st{Typst Universe}
  \end{itemize}
\end{itemize}

\href{/docs/changelog/earlier/}{\pandocbounded{\includesvg[keepaspectratio]{/assets/icons/16-arrow-right.svg}}}

{ Earlier } { Previous page }

\href{/docs/community/}{\pandocbounded{\includesvg[keepaspectratio]{/assets/icons/16-arrow-right.svg}}}

{ Community } { Next page }


\title{typst.app/docs/guides}

\begin{itemize}
\tightlist
\item
  \href{/docs}{\includesvg[width=0.16667in,height=0.16667in]{/assets/icons/16-docs-dark.svg}}
\item
  \includesvg[width=0.16667in,height=0.16667in]{/assets/icons/16-arrow-right.svg}
\item
  \href{/docs/guides/}{Guides}
\end{itemize}

\section{Guides}\label{guides}

Welcome to the Guides section! Here, you\textquotesingle ll find helpful
material for specific user groups or use cases. Currently, two guides
are available: An introduction to Typst for LaTeX users, and a detailed
look at page setup. Feel free to propose other topics for guides!

\subsection{List of Guides}\label{list-of-guides}

\begin{itemize}
\tightlist
\item
  \href{/docs/guides/guide-for-latex-users/}{Guide for LaTeX users}
\item
  \href{/docs/guides/page-setup-guide/}{Page setup guide}
\item
  \href{/docs/guides/table-guide/}{Table guide}
\end{itemize}

\href{/docs/reference/data-loading/yaml/}{\pandocbounded{\includesvg[keepaspectratio]{/assets/icons/16-arrow-right.svg}}}

{ YAML } { Previous page }

\href{/docs/guides/guide-for-latex-users/}{\pandocbounded{\includesvg[keepaspectratio]{/assets/icons/16-arrow-right.svg}}}

{ Guide for LaTeX users } { Next page }


