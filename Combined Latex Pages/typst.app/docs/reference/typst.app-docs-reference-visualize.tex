\title{typst.app/docs/reference/visualize/circle}

\begin{itemize}
\tightlist
\item
  \href{/docs}{\includesvg[width=0.16667in,height=0.16667in]{/assets/icons/16-docs-dark.svg}}
\item
  \includesvg[width=0.16667in,height=0.16667in]{/assets/icons/16-arrow-right.svg}
\item
  \href{/docs/reference/}{Reference}
\item
  \includesvg[width=0.16667in,height=0.16667in]{/assets/icons/16-arrow-right.svg}
\item
  \href{/docs/reference/visualize/}{Visualize}
\item
  \includesvg[width=0.16667in,height=0.16667in]{/assets/icons/16-arrow-right.svg}
\item
  \href{/docs/reference/visualize/circle/}{Circle}
\end{itemize}

\section{\texorpdfstring{\texttt{\ circle\ } {{ Element
}}}{ circle   Element }}\label{summary}

\phantomsection\label{element-tooltip}
Element functions can be customized with \texttt{\ set\ } and
\texttt{\ show\ } rules.

A circle with optional content.

\subsection{Example}\label{example}

\begin{verbatim}
// Without content.
#circle(radius: 25pt)

// With content.
#circle[
  #set align(center + horizon)
  Automatically \
  sized to fit.
]
\end{verbatim}

\includegraphics[width=5in,height=\textheight,keepaspectratio]{/assets/docs/H1niwFeoKUTVgzuqcmZ_VgAAAAAAAAAA.png}

\subsection{\texorpdfstring{{ Parameters
}}{ Parameters }}\label{parameters}

\phantomsection\label{parameters-tooltip}
Parameters are the inputs to a function. They are specified in
parentheses after the function name.

{ circle } (

{ \hyperref[parameters-radius]{radius :}
\href{/docs/reference/layout/length/}{length} , } {
\hyperref[parameters-width]{width :}
\href{/docs/reference/foundations/auto/}{auto}
\href{/docs/reference/layout/relative/}{relative} , } {
\hyperref[parameters-height]{height :}
\href{/docs/reference/foundations/auto/}{auto}
\href{/docs/reference/layout/relative/}{relative}
\href{/docs/reference/layout/fraction/}{fraction} , } {
\hyperref[parameters-fill]{fill :}
\href{/docs/reference/foundations/none/}{none}
\href{/docs/reference/visualize/color/}{color}
\href{/docs/reference/visualize/gradient/}{gradient}
\href{/docs/reference/visualize/pattern/}{pattern} , } {
\hyperref[parameters-stroke]{stroke :}
\href{/docs/reference/foundations/none/}{none}
\href{/docs/reference/foundations/auto/}{auto}
\href{/docs/reference/layout/length/}{length}
\href{/docs/reference/visualize/color/}{color}
\href{/docs/reference/visualize/gradient/}{gradient}
\href{/docs/reference/visualize/stroke/}{stroke}
\href{/docs/reference/visualize/pattern/}{pattern}
\href{/docs/reference/foundations/dictionary/}{dictionary} , } {
\hyperref[parameters-inset]{inset :}
\href{/docs/reference/layout/relative/}{relative}
\href{/docs/reference/foundations/dictionary/}{dictionary} , } {
\hyperref[parameters-outset]{outset :}
\href{/docs/reference/layout/relative/}{relative}
\href{/docs/reference/foundations/dictionary/}{dictionary} , } {
\hyperref[parameters-body]{}
\href{/docs/reference/foundations/none/}{none}
\href{/docs/reference/foundations/content/}{content} , }

) -\textgreater{} \href{/docs/reference/foundations/content/}{content}

\subsubsection{\texorpdfstring{\texttt{\ radius\ }}{ radius }}\label{parameters-radius}

\href{/docs/reference/layout/length/}{length}

{{ Settable }}

\phantomsection\label{parameters-radius-settable-tooltip}
Settable parameters can be customized for all following uses of the
function with a \texttt{\ set\ } rule.

The circle\textquotesingle s radius. This is mutually exclusive with
\texttt{\ width\ } and \texttt{\ height\ } .

Default: \texttt{\ }{\texttt{\ 0pt\ }}\texttt{\ }

\subsubsection{\texorpdfstring{\texttt{\ width\ }}{ width }}\label{parameters-width}

\href{/docs/reference/foundations/auto/}{auto} {or}
\href{/docs/reference/layout/relative/}{relative}

{{ Settable }}

\phantomsection\label{parameters-width-settable-tooltip}
Settable parameters can be customized for all following uses of the
function with a \texttt{\ set\ } rule.

The circle\textquotesingle s width. This is mutually exclusive with
\texttt{\ radius\ } and \texttt{\ height\ } .

In contrast to \texttt{\ radius\ } , this can be relative to the parent
container\textquotesingle s width.

Default: \texttt{\ }{\texttt{\ auto\ }}\texttt{\ }

\subsubsection{\texorpdfstring{\texttt{\ height\ }}{ height }}\label{parameters-height}

\href{/docs/reference/foundations/auto/}{auto} {or}
\href{/docs/reference/layout/relative/}{relative} {or}
\href{/docs/reference/layout/fraction/}{fraction}

{{ Settable }}

\phantomsection\label{parameters-height-settable-tooltip}
Settable parameters can be customized for all following uses of the
function with a \texttt{\ set\ } rule.

The circle\textquotesingle s height. This is mutually exclusive with
\texttt{\ radius\ } and \texttt{\ width\ } .

In contrast to \texttt{\ radius\ } , this can be relative to the parent
container\textquotesingle s height.

Default: \texttt{\ }{\texttt{\ auto\ }}\texttt{\ }

\subsubsection{\texorpdfstring{\texttt{\ fill\ }}{ fill }}\label{parameters-fill}

\href{/docs/reference/foundations/none/}{none} {or}
\href{/docs/reference/visualize/color/}{color} {or}
\href{/docs/reference/visualize/gradient/}{gradient} {or}
\href{/docs/reference/visualize/pattern/}{pattern}

{{ Settable }}

\phantomsection\label{parameters-fill-settable-tooltip}
Settable parameters can be customized for all following uses of the
function with a \texttt{\ set\ } rule.

How to fill the circle. See the
\href{/docs/reference/visualize/rect/\#parameters-fill}{rectangle\textquotesingle s
documentation} for more details.

Default: \texttt{\ }{\texttt{\ none\ }}\texttt{\ }

\subsubsection{\texorpdfstring{\texttt{\ stroke\ }}{ stroke }}\label{parameters-stroke}

\href{/docs/reference/foundations/none/}{none} {or}
\href{/docs/reference/foundations/auto/}{auto} {or}
\href{/docs/reference/layout/length/}{length} {or}
\href{/docs/reference/visualize/color/}{color} {or}
\href{/docs/reference/visualize/gradient/}{gradient} {or}
\href{/docs/reference/visualize/stroke/}{stroke} {or}
\href{/docs/reference/visualize/pattern/}{pattern} {or}
\href{/docs/reference/foundations/dictionary/}{dictionary}

{{ Settable }}

\phantomsection\label{parameters-stroke-settable-tooltip}
Settable parameters can be customized for all following uses of the
function with a \texttt{\ set\ } rule.

How to stroke the circle. See the
\href{/docs/reference/visualize/rect/\#parameters-stroke}{rectangle\textquotesingle s
documentation} for more details.

Default: \texttt{\ }{\texttt{\ auto\ }}\texttt{\ }

\subsubsection{\texorpdfstring{\texttt{\ inset\ }}{ inset }}\label{parameters-inset}

\href{/docs/reference/layout/relative/}{relative} {or}
\href{/docs/reference/foundations/dictionary/}{dictionary}

{{ Settable }}

\phantomsection\label{parameters-inset-settable-tooltip}
Settable parameters can be customized for all following uses of the
function with a \texttt{\ set\ } rule.

How much to pad the circle\textquotesingle s content. See the
\href{/docs/reference/layout/box/\#parameters-inset}{box\textquotesingle s
documentation} for more details.

Default:
\texttt{\ }{\texttt{\ 0\%\ }}\texttt{\ }{\texttt{\ +\ }}\texttt{\ }{\texttt{\ 5pt\ }}\texttt{\ }

\subsubsection{\texorpdfstring{\texttt{\ outset\ }}{ outset }}\label{parameters-outset}

\href{/docs/reference/layout/relative/}{relative} {or}
\href{/docs/reference/foundations/dictionary/}{dictionary}

{{ Settable }}

\phantomsection\label{parameters-outset-settable-tooltip}
Settable parameters can be customized for all following uses of the
function with a \texttt{\ set\ } rule.

How much to expand the circle\textquotesingle s size without affecting
the layout. See the
\href{/docs/reference/layout/box/\#parameters-outset}{box\textquotesingle s
documentation} for more details.

Default:
\texttt{\ }{\texttt{\ (\ }}\texttt{\ }{\texttt{\ :\ }}\texttt{\ }{\texttt{\ )\ }}\texttt{\ }

\subsubsection{\texorpdfstring{\texttt{\ body\ }}{ body }}\label{parameters-body}

\href{/docs/reference/foundations/none/}{none} {or}
\href{/docs/reference/foundations/content/}{content}

{{ Positional }}

\phantomsection\label{parameters-body-positional-tooltip}
Positional parameters are specified in order, without names.

{{ Settable }}

\phantomsection\label{parameters-body-settable-tooltip}
Settable parameters can be customized for all following uses of the
function with a \texttt{\ set\ } rule.

The content to place into the circle. The circle expands to fit this
content, keeping the 1-1 aspect ratio.

Default: \texttt{\ }{\texttt{\ none\ }}\texttt{\ }

\href{/docs/reference/visualize/}{\pandocbounded{\includesvg[keepaspectratio]{/assets/icons/16-arrow-right.svg}}}

{ Visualize } { Previous page }

\href{/docs/reference/visualize/color/}{\pandocbounded{\includesvg[keepaspectratio]{/assets/icons/16-arrow-right.svg}}}

{ Color } { Next page }


\title{typst.app/docs/reference/visualize/polygon}

\begin{itemize}
\tightlist
\item
  \href{/docs}{\includesvg[width=0.16667in,height=0.16667in]{/assets/icons/16-docs-dark.svg}}
\item
  \includesvg[width=0.16667in,height=0.16667in]{/assets/icons/16-arrow-right.svg}
\item
  \href{/docs/reference/}{Reference}
\item
  \includesvg[width=0.16667in,height=0.16667in]{/assets/icons/16-arrow-right.svg}
\item
  \href{/docs/reference/visualize/}{Visualize}
\item
  \includesvg[width=0.16667in,height=0.16667in]{/assets/icons/16-arrow-right.svg}
\item
  \href{/docs/reference/visualize/polygon/}{Polygon}
\end{itemize}

\section{\texorpdfstring{\texttt{\ polygon\ } {{ Element
}}}{ polygon   Element }}\label{summary}

\phantomsection\label{element-tooltip}
Element functions can be customized with \texttt{\ set\ } and
\texttt{\ show\ } rules.

A closed polygon.

The polygon is defined by its corner points and is closed automatically.

\subsection{Example}\label{example}

\begin{verbatim}
#polygon(
  fill: blue.lighten(80%),
  stroke: blue,
  (20%, 0pt),
  (60%, 0pt),
  (80%, 2cm),
  (0%,  2cm),
)
\end{verbatim}

\includegraphics[width=5in,height=\textheight,keepaspectratio]{/assets/docs/TuzATomarVg-0NmUVu3QFAAAAAAAAAAA.png}

\subsection{\texorpdfstring{{ Parameters
}}{ Parameters }}\label{parameters}

\phantomsection\label{parameters-tooltip}
Parameters are the inputs to a function. They are specified in
parentheses after the function name.

{ polygon } (

{ \hyperref[parameters-fill]{fill :}
\href{/docs/reference/foundations/none/}{none}
\href{/docs/reference/visualize/color/}{color}
\href{/docs/reference/visualize/gradient/}{gradient}
\href{/docs/reference/visualize/pattern/}{pattern} , } {
\hyperref[parameters-fill-rule]{fill-rule :}
\href{/docs/reference/foundations/str/}{str} , } {
\hyperref[parameters-stroke]{stroke :}
\href{/docs/reference/foundations/none/}{none}
\href{/docs/reference/foundations/auto/}{auto}
\href{/docs/reference/layout/length/}{length}
\href{/docs/reference/visualize/color/}{color}
\href{/docs/reference/visualize/gradient/}{gradient}
\href{/docs/reference/visualize/stroke/}{stroke}
\href{/docs/reference/visualize/pattern/}{pattern}
\href{/docs/reference/foundations/dictionary/}{dictionary} , } {
\hyperref[parameters-vertices]{..}
\href{/docs/reference/foundations/array/}{array} , }

) -\textgreater{} \href{/docs/reference/foundations/content/}{content}

\subsubsection{\texorpdfstring{\texttt{\ fill\ }}{ fill }}\label{parameters-fill}

\href{/docs/reference/foundations/none/}{none} {or}
\href{/docs/reference/visualize/color/}{color} {or}
\href{/docs/reference/visualize/gradient/}{gradient} {or}
\href{/docs/reference/visualize/pattern/}{pattern}

{{ Settable }}

\phantomsection\label{parameters-fill-settable-tooltip}
Settable parameters can be customized for all following uses of the
function with a \texttt{\ set\ } rule.

How to fill the polygon.

When setting a fill, the default stroke disappears. To create a
rectangle with both fill and stroke, you have to configure both.

Default: \texttt{\ }{\texttt{\ none\ }}\texttt{\ }

\subsubsection{\texorpdfstring{\texttt{\ fill-rule\ }}{ fill-rule }}\label{parameters-fill-rule}

\href{/docs/reference/foundations/str/}{str}

{{ Settable }}

\phantomsection\label{parameters-fill-rule-settable-tooltip}
Settable parameters can be customized for all following uses of the
function with a \texttt{\ set\ } rule.

The drawing rule used to fill the polygon.

See the
\href{/docs/reference/visualize/path/\#parameters-fill-rule}{path
documentation} for an example.

\begin{longtable}[]{@{}ll@{}}
\toprule\noalign{}
Variant & Details \\
\midrule\noalign{}
\endhead
\bottomrule\noalign{}
\endlastfoot
\texttt{\ "\ non-zero\ "\ } & Specifies that "inside" is computed by a
non-zero sum of signed edge crossings. \\
\texttt{\ "\ even-odd\ "\ } & Specifies that "inside" is computed by an
odd number of edge crossings. \\
\end{longtable}

Default: \texttt{\ }{\texttt{\ "non-zero"\ }}\texttt{\ }

\subsubsection{\texorpdfstring{\texttt{\ stroke\ }}{ stroke }}\label{parameters-stroke}

\href{/docs/reference/foundations/none/}{none} {or}
\href{/docs/reference/foundations/auto/}{auto} {or}
\href{/docs/reference/layout/length/}{length} {or}
\href{/docs/reference/visualize/color/}{color} {or}
\href{/docs/reference/visualize/gradient/}{gradient} {or}
\href{/docs/reference/visualize/stroke/}{stroke} {or}
\href{/docs/reference/visualize/pattern/}{pattern} {or}
\href{/docs/reference/foundations/dictionary/}{dictionary}

{{ Settable }}

\phantomsection\label{parameters-stroke-settable-tooltip}
Settable parameters can be customized for all following uses of the
function with a \texttt{\ set\ } rule.

How to \href{/docs/reference/visualize/stroke/}{stroke} the polygon.
This can be:

Can be set to \texttt{\ }{\texttt{\ none\ }}\texttt{\ } to disable the
stroke or to \texttt{\ }{\texttt{\ auto\ }}\texttt{\ } for a stroke of
\texttt{\ }{\texttt{\ 1pt\ }}\texttt{\ } black if and if only if no fill
is given.

Default: \texttt{\ }{\texttt{\ auto\ }}\texttt{\ }

\subsubsection{\texorpdfstring{\texttt{\ vertices\ }}{ vertices }}\label{parameters-vertices}

\href{/docs/reference/foundations/array/}{array}

{Required} {{ Positional }}

\phantomsection\label{parameters-vertices-positional-tooltip}
Positional parameters are specified in order, without names.

{{ Variadic }}

\phantomsection\label{parameters-vertices-variadic-tooltip}
Variadic parameters can be specified multiple times.

The vertices of the polygon. Each point is specified as an array of two
\href{/docs/reference/layout/relative/}{relative lengths} .

\subsection{\texorpdfstring{{ Definitions
}}{ Definitions }}\label{definitions}

\phantomsection\label{definitions-tooltip}
Functions and types and can have associated definitions. These are
accessed by specifying the function or type, followed by a period, and
then the definition\textquotesingle s name.

\subsubsection{\texorpdfstring{\texttt{\ regular\ }}{ regular }}\label{definitions-regular}

A regular polygon, defined by its size and number of vertices.

polygon { . } { regular } (

{ \hyperref[definitions-regular-parameters-fill]{fill :}
\href{/docs/reference/foundations/none/}{none}
\href{/docs/reference/visualize/color/}{color}
\href{/docs/reference/visualize/gradient/}{gradient}
\href{/docs/reference/visualize/pattern/}{pattern} , } {
\hyperref[definitions-regular-parameters-stroke]{stroke :}
\href{/docs/reference/foundations/none/}{none}
\href{/docs/reference/foundations/auto/}{auto}
\href{/docs/reference/layout/length/}{length}
\href{/docs/reference/visualize/color/}{color}
\href{/docs/reference/visualize/gradient/}{gradient}
\href{/docs/reference/visualize/stroke/}{stroke}
\href{/docs/reference/visualize/pattern/}{pattern}
\href{/docs/reference/foundations/dictionary/}{dictionary} , } {
\hyperref[definitions-regular-parameters-size]{size :}
\href{/docs/reference/layout/length/}{length} , } {
\hyperref[definitions-regular-parameters-vertices]{vertices :}
\href{/docs/reference/foundations/int/}{int} , }

) -\textgreater{} \href{/docs/reference/foundations/content/}{content}

\begin{verbatim}
#polygon.regular(
  fill: blue.lighten(80%),
  stroke: blue,
  size: 30pt,
  vertices: 3,
)
\end{verbatim}

\includegraphics[width=5in,height=\textheight,keepaspectratio]{/assets/docs/nSKAw-cASGAIxDorv3UyHgAAAAAAAAAA.png}

\paragraph{\texorpdfstring{\texttt{\ fill\ }}{ fill }}\label{definitions-regular-fill}

\href{/docs/reference/foundations/none/}{none} {or}
\href{/docs/reference/visualize/color/}{color} {or}
\href{/docs/reference/visualize/gradient/}{gradient} {or}
\href{/docs/reference/visualize/pattern/}{pattern}

How to fill the polygon. See the general
\href{/docs/reference/visualize/polygon/\#parameters-fill}{polygon\textquotesingle s
documentation} for more details.

\paragraph{\texorpdfstring{\texttt{\ stroke\ }}{ stroke }}\label{definitions-regular-stroke}

\href{/docs/reference/foundations/none/}{none} {or}
\href{/docs/reference/foundations/auto/}{auto} {or}
\href{/docs/reference/layout/length/}{length} {or}
\href{/docs/reference/visualize/color/}{color} {or}
\href{/docs/reference/visualize/gradient/}{gradient} {or}
\href{/docs/reference/visualize/stroke/}{stroke} {or}
\href{/docs/reference/visualize/pattern/}{pattern} {or}
\href{/docs/reference/foundations/dictionary/}{dictionary}

How to stroke the polygon. See the general
\href{/docs/reference/visualize/polygon/\#parameters-stroke}{polygon\textquotesingle s
documentation} for more details.

\paragraph{\texorpdfstring{\texttt{\ size\ }}{ size }}\label{definitions-regular-size}

\href{/docs/reference/layout/length/}{length}

The diameter of the
\href{https://en.wikipedia.org/wiki/Circumcircle}{circumcircle} of the
regular polygon.

Default: \texttt{\ }{\texttt{\ 1em\ }}\texttt{\ }

\paragraph{\texorpdfstring{\texttt{\ vertices\ }}{ vertices }}\label{definitions-regular-vertices}

\href{/docs/reference/foundations/int/}{int}

The number of vertices in the polygon.

Default: \texttt{\ }{\texttt{\ 3\ }}\texttt{\ }

\href{/docs/reference/visualize/pattern/}{\pandocbounded{\includesvg[keepaspectratio]{/assets/icons/16-arrow-right.svg}}}

{ Pattern } { Previous page }

\href{/docs/reference/visualize/rect/}{\pandocbounded{\includesvg[keepaspectratio]{/assets/icons/16-arrow-right.svg}}}

{ Rectangle } { Next page }


\title{typst.app/docs/reference/visualize/path}

\begin{itemize}
\tightlist
\item
  \href{/docs}{\includesvg[width=0.16667in,height=0.16667in]{/assets/icons/16-docs-dark.svg}}
\item
  \includesvg[width=0.16667in,height=0.16667in]{/assets/icons/16-arrow-right.svg}
\item
  \href{/docs/reference/}{Reference}
\item
  \includesvg[width=0.16667in,height=0.16667in]{/assets/icons/16-arrow-right.svg}
\item
  \href{/docs/reference/visualize/}{Visualize}
\item
  \includesvg[width=0.16667in,height=0.16667in]{/assets/icons/16-arrow-right.svg}
\item
  \href{/docs/reference/visualize/path/}{Path}
\end{itemize}

\section{\texorpdfstring{\texttt{\ path\ } {{ Element
}}}{ path   Element }}\label{summary}

\phantomsection\label{element-tooltip}
Element functions can be customized with \texttt{\ set\ } and
\texttt{\ show\ } rules.

A path through a list of points, connected by Bezier curves.

\subsection{Example}\label{example}

\begin{verbatim}
#path(
  fill: blue.lighten(80%),
  stroke: blue,
  closed: true,
  (0pt, 50pt),
  (100%, 50pt),
  ((50%, 0pt), (40pt, 0pt)),
)
\end{verbatim}

\includegraphics[width=5in,height=\textheight,keepaspectratio]{/assets/docs/fHH_90d6MEksjFQh_gCkDwAAAAAAAAAA.png}

\subsection{\texorpdfstring{{ Parameters
}}{ Parameters }}\label{parameters}

\phantomsection\label{parameters-tooltip}
Parameters are the inputs to a function. They are specified in
parentheses after the function name.

{ path } (

{ \hyperref[parameters-fill]{fill :}
\href{/docs/reference/foundations/none/}{none}
\href{/docs/reference/visualize/color/}{color}
\href{/docs/reference/visualize/gradient/}{gradient}
\href{/docs/reference/visualize/pattern/}{pattern} , } {
\hyperref[parameters-fill-rule]{fill-rule :}
\href{/docs/reference/foundations/str/}{str} , } {
\hyperref[parameters-stroke]{stroke :}
\href{/docs/reference/foundations/none/}{none}
\href{/docs/reference/foundations/auto/}{auto}
\href{/docs/reference/layout/length/}{length}
\href{/docs/reference/visualize/color/}{color}
\href{/docs/reference/visualize/gradient/}{gradient}
\href{/docs/reference/visualize/stroke/}{stroke}
\href{/docs/reference/visualize/pattern/}{pattern}
\href{/docs/reference/foundations/dictionary/}{dictionary} , } {
\hyperref[parameters-closed]{closed :}
\href{/docs/reference/foundations/bool/}{bool} , } {
\hyperref[parameters-vertices]{..}
\href{/docs/reference/foundations/array/}{array} , }

) -\textgreater{} \href{/docs/reference/foundations/content/}{content}

\subsubsection{\texorpdfstring{\texttt{\ fill\ }}{ fill }}\label{parameters-fill}

\href{/docs/reference/foundations/none/}{none} {or}
\href{/docs/reference/visualize/color/}{color} {or}
\href{/docs/reference/visualize/gradient/}{gradient} {or}
\href{/docs/reference/visualize/pattern/}{pattern}

{{ Settable }}

\phantomsection\label{parameters-fill-settable-tooltip}
Settable parameters can be customized for all following uses of the
function with a \texttt{\ set\ } rule.

How to fill the path.

When setting a fill, the default stroke disappears. To create a
rectangle with both fill and stroke, you have to configure both.

Default: \texttt{\ }{\texttt{\ none\ }}\texttt{\ }

\subsubsection{\texorpdfstring{\texttt{\ fill-rule\ }}{ fill-rule }}\label{parameters-fill-rule}

\href{/docs/reference/foundations/str/}{str}

{{ Settable }}

\phantomsection\label{parameters-fill-rule-settable-tooltip}
Settable parameters can be customized for all following uses of the
function with a \texttt{\ set\ } rule.

The drawing rule used to fill the path.

\begin{longtable}[]{@{}ll@{}}
\toprule\noalign{}
Variant & Details \\
\midrule\noalign{}
\endhead
\bottomrule\noalign{}
\endlastfoot
\texttt{\ "\ non-zero\ "\ } & Specifies that "inside" is computed by a
non-zero sum of signed edge crossings. \\
\texttt{\ "\ even-odd\ "\ } & Specifies that "inside" is computed by an
odd number of edge crossings. \\
\end{longtable}

Default: \texttt{\ }{\texttt{\ "non-zero"\ }}\texttt{\ }

\includesvg[width=0.16667in,height=0.16667in]{/assets/icons/16-arrow-right.svg}
View example

\begin{verbatim}
// We use `.with` to get a new
// function that has the common
// arguments pre-applied.
#let star = path.with(
  fill: red,
  closed: true,
  (25pt, 0pt),
  (10pt, 50pt),
  (50pt, 20pt),
  (0pt, 20pt),
  (40pt, 50pt),
)

#star(fill-rule: "non-zero")
#star(fill-rule: "even-odd")
\end{verbatim}

\includegraphics[width=5in,height=\textheight,keepaspectratio]{/assets/docs/MJEOUf62l7aK0PG-Hl3HKgAAAAAAAAAA.png}

\subsubsection{\texorpdfstring{\texttt{\ stroke\ }}{ stroke }}\label{parameters-stroke}

\href{/docs/reference/foundations/none/}{none} {or}
\href{/docs/reference/foundations/auto/}{auto} {or}
\href{/docs/reference/layout/length/}{length} {or}
\href{/docs/reference/visualize/color/}{color} {or}
\href{/docs/reference/visualize/gradient/}{gradient} {or}
\href{/docs/reference/visualize/stroke/}{stroke} {or}
\href{/docs/reference/visualize/pattern/}{pattern} {or}
\href{/docs/reference/foundations/dictionary/}{dictionary}

{{ Settable }}

\phantomsection\label{parameters-stroke-settable-tooltip}
Settable parameters can be customized for all following uses of the
function with a \texttt{\ set\ } rule.

How to \href{/docs/reference/visualize/stroke/}{stroke} the path. This
can be:

Can be set to \texttt{\ }{\texttt{\ none\ }}\texttt{\ } to disable the
stroke or to \texttt{\ }{\texttt{\ auto\ }}\texttt{\ } for a stroke of
\texttt{\ }{\texttt{\ 1pt\ }}\texttt{\ } black if and if only if no fill
is given.

Default: \texttt{\ }{\texttt{\ auto\ }}\texttt{\ }

\subsubsection{\texorpdfstring{\texttt{\ closed\ }}{ closed }}\label{parameters-closed}

\href{/docs/reference/foundations/bool/}{bool}

{{ Settable }}

\phantomsection\label{parameters-closed-settable-tooltip}
Settable parameters can be customized for all following uses of the
function with a \texttt{\ set\ } rule.

Whether to close this path with one last bezier curve. This curve will
takes into account the adjacent control points. If you want to close
with a straight line, simply add one last point that\textquotesingle s
the same as the start point.

Default: \texttt{\ }{\texttt{\ false\ }}\texttt{\ }

\subsubsection{\texorpdfstring{\texttt{\ vertices\ }}{ vertices }}\label{parameters-vertices}

\href{/docs/reference/foundations/array/}{array}

{Required} {{ Positional }}

\phantomsection\label{parameters-vertices-positional-tooltip}
Positional parameters are specified in order, without names.

{{ Variadic }}

\phantomsection\label{parameters-vertices-variadic-tooltip}
Variadic parameters can be specified multiple times.

The vertices of the path.

Each vertex can be defined in 3 ways:

\begin{itemize}
\tightlist
\item
  A regular point, as given to the
  \href{/docs/reference/visualize/line/}{\texttt{\ line\ }} or
  \href{/docs/reference/visualize/polygon/}{\texttt{\ polygon\ }}
  function.
\item
  An array of two points, the first being the vertex and the second
  being the control point. The control point is expressed relative to
  the vertex and is mirrored to get the second control point. The given
  control point is the one that affects the curve coming \emph{into}
  this vertex (even for the first point). The mirrored control point
  affects the curve going out of this vertex.
\item
  An array of three points, the first being the vertex and the next
  being the control points (control point for curves coming in and out,
  respectively).
\end{itemize}

\href{/docs/reference/visualize/line/}{\pandocbounded{\includesvg[keepaspectratio]{/assets/icons/16-arrow-right.svg}}}

{ Line } { Previous page }

\href{/docs/reference/visualize/pattern/}{\pandocbounded{\includesvg[keepaspectratio]{/assets/icons/16-arrow-right.svg}}}

{ Pattern } { Next page }


\title{typst.app/docs/reference/visualize/pattern}

\begin{itemize}
\tightlist
\item
  \href{/docs}{\includesvg[width=0.16667in,height=0.16667in]{/assets/icons/16-docs-dark.svg}}
\item
  \includesvg[width=0.16667in,height=0.16667in]{/assets/icons/16-arrow-right.svg}
\item
  \href{/docs/reference/}{Reference}
\item
  \includesvg[width=0.16667in,height=0.16667in]{/assets/icons/16-arrow-right.svg}
\item
  \href{/docs/reference/visualize/}{Visualize}
\item
  \includesvg[width=0.16667in,height=0.16667in]{/assets/icons/16-arrow-right.svg}
\item
  \href{/docs/reference/visualize/pattern/}{Pattern}
\end{itemize}

\section{\texorpdfstring{{ pattern }}{ pattern }}\label{summary}

A repeating pattern fill.

Typst supports the most common pattern type of tiled patterns, where a
pattern is repeated in a grid-like fashion, covering the entire area of
an element that is filled or stroked. The pattern is defined by a tile
size and a body defining the content of each cell. You can also add
horizontal or vertical spacing between the cells of the pattern.

\subsection{Examples}\label{examples}

\begin{verbatim}
#let pat = pattern(size: (30pt, 30pt))[
  #place(line(start: (0%, 0%), end: (100%, 100%)))
  #place(line(start: (0%, 100%), end: (100%, 0%)))
]

#rect(fill: pat, width: 100%, height: 60pt, stroke: 1pt)
\end{verbatim}

\includegraphics[width=5in,height=\textheight,keepaspectratio]{/assets/docs/coeD6IerbqenB1CPjs7dfAAAAAAAAAAA.png}

Patterns are also supported on text, but only when setting the
\href{/docs/reference/visualize/pattern/\#parameters-relative}{relativeness}
to either \texttt{\ }{\texttt{\ auto\ }}\texttt{\ } (the default value)
or \texttt{\ }{\texttt{\ "parent"\ }}\texttt{\ } . To create
word-by-word or glyph-by-glyph patterns, you can wrap the words or
characters of your text in \href{/docs/reference/layout/box/}{boxes}
manually or through a \href{/docs/reference/styling/\#show-rules}{show
rule} .

\begin{verbatim}
#let pat = pattern(
  size: (30pt, 30pt),
  relative: "parent",
  square(
    size: 30pt,
    fill: gradient
      .conic(..color.map.rainbow),
  )
)

#set text(fill: pat)
#lorem(10)
\end{verbatim}

\includegraphics[width=5in,height=\textheight,keepaspectratio]{/assets/docs/Vk9hYVErruhpSxeZVudFjQAAAAAAAAAA.png}

You can also space the elements further or closer apart using the
\href{/docs/reference/visualize/pattern/\#parameters-spacing}{\texttt{\ spacing\ }}
feature of the pattern. If the spacing is lower than the size of the
pattern, the pattern will overlap. If it is higher, the pattern will
have gaps of the same color as the background of the pattern.

\begin{verbatim}
#let pat = pattern(
  size: (30pt, 30pt),
  spacing: (10pt, 10pt),
  relative: "parent",
  square(
    size: 30pt,
    fill: gradient
     .conic(..color.map.rainbow),
  ),
)

#rect(
  width: 100%,
  height: 60pt,
  fill: pat,
)
\end{verbatim}

\includegraphics[width=5in,height=\textheight,keepaspectratio]{/assets/docs/yPTj9FTOvqrbv-4eK83U7gAAAAAAAAAA.png}

\subsection{Relativeness}\label{relativeness}

The location of the starting point of the pattern is dependent on the
dimensions of a container. This container can either be the shape that
it is being painted on, or the closest surrounding container. This is
controlled by the \texttt{\ relative\ } argument of a pattern
constructor. By default, patterns are relative to the shape they are
being painted on, unless the pattern is applied on text, in which case
they are relative to the closest ancestor container.

Typst determines the ancestor container as follows:

\begin{itemize}
\tightlist
\item
  For shapes that are placed at the root/top level of the document, the
  closest ancestor is the page itself.
\item
  For other shapes, the ancestor is the innermost
  \href{/docs/reference/layout/block/}{\texttt{\ block\ }} or
  \href{/docs/reference/layout/box/}{\texttt{\ box\ }} that contains the
  shape. This includes the boxes and blocks that are implicitly created
  by show rules and elements. For example, a
  \href{/docs/reference/layout/rotate/}{\texttt{\ rotate\ }} will not
  affect the parent of a gradient, but a
  \href{/docs/reference/layout/grid/}{\texttt{\ grid\ }} will.
\end{itemize}

\subsection{\texorpdfstring{Constructor
{}}{Constructor }}\label{constructor}

\phantomsection\label{constructor-constructor-tooltip}
If a type has a constructor, you can call it like a function to create a
new value of the type.

Construct a new pattern.

{ pattern } (

{ \hyperref[constructor-parameters-size]{size :}
\href{/docs/reference/foundations/auto/}{auto}
\href{/docs/reference/foundations/array/}{array} , } {
\hyperref[constructor-parameters-spacing]{spacing :}
\href{/docs/reference/foundations/array/}{array} , } {
\hyperref[constructor-parameters-relative]{relative :}
\href{/docs/reference/foundations/auto/}{auto}
\href{/docs/reference/foundations/str/}{str} , } {
\href{/docs/reference/foundations/content/}{content} , }

) -\textgreater{} \href{/docs/reference/visualize/pattern/}{pattern}

\begin{verbatim}
#let pat = pattern(
  size: (20pt, 20pt),
  relative: "parent",
  place(
    dx: 5pt,
    dy: 5pt,
    rotate(45deg, square(
      size: 5pt,
      fill: black,
    )),
  ),
)

#rect(width: 100%, height: 60pt, fill: pat)
\end{verbatim}

\includegraphics[width=5in,height=\textheight,keepaspectratio]{/assets/docs/s7EOLk1zJeZ_4afTw83qRwAAAAAAAAAA.png}

\paragraph{\texorpdfstring{\texttt{\ size\ }}{ size }}\label{constructor-size}

\href{/docs/reference/foundations/auto/}{auto} {or}
\href{/docs/reference/foundations/array/}{array}

The bounding box of each cell of the pattern.

Default: \texttt{\ }{\texttt{\ auto\ }}\texttt{\ }

\paragraph{\texorpdfstring{\texttt{\ spacing\ }}{ spacing }}\label{constructor-spacing}

\href{/docs/reference/foundations/array/}{array}

The spacing between cells of the pattern.

Default:
\texttt{\ }{\texttt{\ (\ }}\texttt{\ }{\texttt{\ 0pt\ }}\texttt{\ }{\texttt{\ ,\ }}\texttt{\ }{\texttt{\ 0pt\ }}\texttt{\ }{\texttt{\ )\ }}\texttt{\ }

\paragraph{\texorpdfstring{\texttt{\ relative\ }}{ relative }}\label{constructor-relative}

\href{/docs/reference/foundations/auto/}{auto} {or}
\href{/docs/reference/foundations/str/}{str}

The \hyperref[relativeness]{relative placement} of the pattern.

For an element placed at the root/top level of the document, the parent
is the page itself. For other elements, the parent is the innermost
block, box, column, grid, or stack that contains the element.

\begin{longtable}[]{@{}ll@{}}
\toprule\noalign{}
Variant & Details \\
\midrule\noalign{}
\endhead
\bottomrule\noalign{}
\endlastfoot
\texttt{\ "\ self\ "\ } & The gradient is relative to itself (its own
bounding box). \\
\texttt{\ "\ parent\ "\ } & The gradient is relative to its parent (the
parent\textquotesingle s bounding box). \\
\end{longtable}

Default: \texttt{\ }{\texttt{\ auto\ }}\texttt{\ }

\paragraph{\texorpdfstring{\texttt{\ body\ }}{ body }}\label{constructor-body}

\href{/docs/reference/foundations/content/}{content}

{Required} {{ Positional }}

\phantomsection\label{constructor-body-positional-tooltip}
Positional parameters are specified in order, without names.

The content of each cell of the pattern.

\href{/docs/reference/visualize/path/}{\pandocbounded{\includesvg[keepaspectratio]{/assets/icons/16-arrow-right.svg}}}

{ Path } { Previous page }

\href{/docs/reference/visualize/polygon/}{\pandocbounded{\includesvg[keepaspectratio]{/assets/icons/16-arrow-right.svg}}}

{ Polygon } { Next page }


\title{typst.app/docs/reference/visualize/rect}

\begin{itemize}
\tightlist
\item
  \href{/docs}{\includesvg[width=0.16667in,height=0.16667in]{/assets/icons/16-docs-dark.svg}}
\item
  \includesvg[width=0.16667in,height=0.16667in]{/assets/icons/16-arrow-right.svg}
\item
  \href{/docs/reference/}{Reference}
\item
  \includesvg[width=0.16667in,height=0.16667in]{/assets/icons/16-arrow-right.svg}
\item
  \href{/docs/reference/visualize/}{Visualize}
\item
  \includesvg[width=0.16667in,height=0.16667in]{/assets/icons/16-arrow-right.svg}
\item
  \href{/docs/reference/visualize/rect/}{Rectangle}
\end{itemize}

\section{\texorpdfstring{\texttt{\ rect\ } {{ Element
}}}{ rect   Element }}\label{summary}

\phantomsection\label{element-tooltip}
Element functions can be customized with \texttt{\ set\ } and
\texttt{\ show\ } rules.

A rectangle with optional content.

\subsection{Example}\label{example}

\begin{verbatim}
// Without content.
#rect(width: 35%, height: 30pt)

// With content.
#rect[
  Automatically sized \
  to fit the content.
]
\end{verbatim}

\includegraphics[width=5in,height=\textheight,keepaspectratio]{/assets/docs/uMLkrKs8AmOe9L-qU4CYKgAAAAAAAAAA.png}

\subsection{\texorpdfstring{{ Parameters
}}{ Parameters }}\label{parameters}

\phantomsection\label{parameters-tooltip}
Parameters are the inputs to a function. They are specified in
parentheses after the function name.

{ rect } (

{ \hyperref[parameters-width]{width :}
\href{/docs/reference/foundations/auto/}{auto}
\href{/docs/reference/layout/relative/}{relative} , } {
\hyperref[parameters-height]{height :}
\href{/docs/reference/foundations/auto/}{auto}
\href{/docs/reference/layout/relative/}{relative}
\href{/docs/reference/layout/fraction/}{fraction} , } {
\hyperref[parameters-fill]{fill :}
\href{/docs/reference/foundations/none/}{none}
\href{/docs/reference/visualize/color/}{color}
\href{/docs/reference/visualize/gradient/}{gradient}
\href{/docs/reference/visualize/pattern/}{pattern} , } {
\hyperref[parameters-stroke]{stroke :}
\href{/docs/reference/foundations/none/}{none}
\href{/docs/reference/foundations/auto/}{auto}
\href{/docs/reference/layout/length/}{length}
\href{/docs/reference/visualize/color/}{color}
\href{/docs/reference/visualize/gradient/}{gradient}
\href{/docs/reference/visualize/stroke/}{stroke}
\href{/docs/reference/visualize/pattern/}{pattern}
\href{/docs/reference/foundations/dictionary/}{dictionary} , } {
\hyperref[parameters-radius]{radius :}
\href{/docs/reference/layout/relative/}{relative}
\href{/docs/reference/foundations/dictionary/}{dictionary} , } {
\hyperref[parameters-inset]{inset :}
\href{/docs/reference/layout/relative/}{relative}
\href{/docs/reference/foundations/dictionary/}{dictionary} , } {
\hyperref[parameters-outset]{outset :}
\href{/docs/reference/layout/relative/}{relative}
\href{/docs/reference/foundations/dictionary/}{dictionary} , } {
\hyperref[parameters-body]{}
\href{/docs/reference/foundations/none/}{none}
\href{/docs/reference/foundations/content/}{content} , }

) -\textgreater{} \href{/docs/reference/foundations/content/}{content}

\subsubsection{\texorpdfstring{\texttt{\ width\ }}{ width }}\label{parameters-width}

\href{/docs/reference/foundations/auto/}{auto} {or}
\href{/docs/reference/layout/relative/}{relative}

{{ Settable }}

\phantomsection\label{parameters-width-settable-tooltip}
Settable parameters can be customized for all following uses of the
function with a \texttt{\ set\ } rule.

The rectangle\textquotesingle s width, relative to its parent container.

Default: \texttt{\ }{\texttt{\ auto\ }}\texttt{\ }

\subsubsection{\texorpdfstring{\texttt{\ height\ }}{ height }}\label{parameters-height}

\href{/docs/reference/foundations/auto/}{auto} {or}
\href{/docs/reference/layout/relative/}{relative} {or}
\href{/docs/reference/layout/fraction/}{fraction}

{{ Settable }}

\phantomsection\label{parameters-height-settable-tooltip}
Settable parameters can be customized for all following uses of the
function with a \texttt{\ set\ } rule.

The rectangle\textquotesingle s height, relative to its parent
container.

Default: \texttt{\ }{\texttt{\ auto\ }}\texttt{\ }

\subsubsection{\texorpdfstring{\texttt{\ fill\ }}{ fill }}\label{parameters-fill}

\href{/docs/reference/foundations/none/}{none} {or}
\href{/docs/reference/visualize/color/}{color} {or}
\href{/docs/reference/visualize/gradient/}{gradient} {or}
\href{/docs/reference/visualize/pattern/}{pattern}

{{ Settable }}

\phantomsection\label{parameters-fill-settable-tooltip}
Settable parameters can be customized for all following uses of the
function with a \texttt{\ set\ } rule.

How to fill the rectangle.

When setting a fill, the default stroke disappears. To create a
rectangle with both fill and stroke, you have to configure both.

Default: \texttt{\ }{\texttt{\ none\ }}\texttt{\ }

\includesvg[width=0.16667in,height=0.16667in]{/assets/icons/16-arrow-right.svg}
View example

\begin{verbatim}
#rect(fill: blue)
\end{verbatim}

\includegraphics[width=5in,height=\textheight,keepaspectratio]{/assets/docs/Xp0gewyTPs1ard61igAjJAAAAAAAAAAA.png}

\subsubsection{\texorpdfstring{\texttt{\ stroke\ }}{ stroke }}\label{parameters-stroke}

\href{/docs/reference/foundations/none/}{none} {or}
\href{/docs/reference/foundations/auto/}{auto} {or}
\href{/docs/reference/layout/length/}{length} {or}
\href{/docs/reference/visualize/color/}{color} {or}
\href{/docs/reference/visualize/gradient/}{gradient} {or}
\href{/docs/reference/visualize/stroke/}{stroke} {or}
\href{/docs/reference/visualize/pattern/}{pattern} {or}
\href{/docs/reference/foundations/dictionary/}{dictionary}

{{ Settable }}

\phantomsection\label{parameters-stroke-settable-tooltip}
Settable parameters can be customized for all following uses of the
function with a \texttt{\ set\ } rule.

How to stroke the rectangle. This can be:

\begin{itemize}
\tightlist
\item
  \texttt{\ }{\texttt{\ none\ }}\texttt{\ } to disable stroking
\item
  \texttt{\ }{\texttt{\ auto\ }}\texttt{\ } for a stroke of
  \texttt{\ }{\texttt{\ 1pt\ }}\texttt{\ }{\texttt{\ +\ }}\texttt{\ black\ }
  if and if only if no fill is given.
\item
  Any kind of \href{/docs/reference/visualize/stroke/}{stroke}
\item
  A dictionary describing the stroke for each side individually. The
  dictionary can contain the following keys in order of precedence:

  \begin{itemize}
  \tightlist
  \item
    \texttt{\ top\ } : The top stroke.
  \item
    \texttt{\ right\ } : The right stroke.
  \item
    \texttt{\ bottom\ } : The bottom stroke.
  \item
    \texttt{\ left\ } : The left stroke.
  \item
    \texttt{\ x\ } : The horizontal stroke.
  \item
    \texttt{\ y\ } : The vertical stroke.
  \item
    \texttt{\ rest\ } : The stroke on all sides except those for which
    the dictionary explicitly sets a size.
  \end{itemize}
\end{itemize}

Default: \texttt{\ }{\texttt{\ auto\ }}\texttt{\ }

\includesvg[width=0.16667in,height=0.16667in]{/assets/icons/16-arrow-right.svg}
View example

\begin{verbatim}
#stack(
  dir: ltr,
  spacing: 1fr,
  rect(stroke: red),
  rect(stroke: 2pt),
  rect(stroke: 2pt + red),
)
\end{verbatim}

\includegraphics[width=5in,height=\textheight,keepaspectratio]{/assets/docs/RNPJxaHVa6js_P-8fJFExAAAAAAAAAAA.png}

\subsubsection{\texorpdfstring{\texttt{\ radius\ }}{ radius }}\label{parameters-radius}

\href{/docs/reference/layout/relative/}{relative} {or}
\href{/docs/reference/foundations/dictionary/}{dictionary}

{{ Settable }}

\phantomsection\label{parameters-radius-settable-tooltip}
Settable parameters can be customized for all following uses of the
function with a \texttt{\ set\ } rule.

How much to round the rectangle\textquotesingle s corners, relative to
the minimum of the width and height divided by two. This can be:

\begin{itemize}
\tightlist
\item
  A relative length for a uniform corner radius.
\item
  A dictionary: With a dictionary, the stroke for each side can be set
  individually. The dictionary can contain the following keys in order
  of precedence:

  \begin{itemize}
  \tightlist
  \item
    \texttt{\ top-left\ } : The top-left corner radius.
  \item
    \texttt{\ top-right\ } : The top-right corner radius.
  \item
    \texttt{\ bottom-right\ } : The bottom-right corner radius.
  \item
    \texttt{\ bottom-left\ } : The bottom-left corner radius.
  \item
    \texttt{\ left\ } : The top-left and bottom-left corner radii.
  \item
    \texttt{\ top\ } : The top-left and top-right corner radii.
  \item
    \texttt{\ right\ } : The top-right and bottom-right corner radii.
  \item
    \texttt{\ bottom\ } : The bottom-left and bottom-right corner radii.
  \item
    \texttt{\ rest\ } : The radii for all corners except those for which
    the dictionary explicitly sets a size.
  \end{itemize}
\end{itemize}

Default:
\texttt{\ }{\texttt{\ (\ }}\texttt{\ }{\texttt{\ :\ }}\texttt{\ }{\texttt{\ )\ }}\texttt{\ }

\includesvg[width=0.16667in,height=0.16667in]{/assets/icons/16-arrow-right.svg}
View example

\begin{verbatim}
#set rect(stroke: 4pt)
#rect(
  radius: (
    left: 5pt,
    top-right: 20pt,
    bottom-right: 10pt,
  ),
  stroke: (
    left: red,
    top: yellow,
    right: green,
    bottom: blue,
  ),
)
\end{verbatim}

\includegraphics[width=5in,height=\textheight,keepaspectratio]{/assets/docs/P93tDNSSrvmdfXv2L7MmYQAAAAAAAAAA.png}

\subsubsection{\texorpdfstring{\texttt{\ inset\ }}{ inset }}\label{parameters-inset}

\href{/docs/reference/layout/relative/}{relative} {or}
\href{/docs/reference/foundations/dictionary/}{dictionary}

{{ Settable }}

\phantomsection\label{parameters-inset-settable-tooltip}
Settable parameters can be customized for all following uses of the
function with a \texttt{\ set\ } rule.

How much to pad the rectangle\textquotesingle s content. See the
\href{/docs/reference/layout/box/\#parameters-outset}{box\textquotesingle s
documentation} for more details.

Default:
\texttt{\ }{\texttt{\ 0\%\ }}\texttt{\ }{\texttt{\ +\ }}\texttt{\ }{\texttt{\ 5pt\ }}\texttt{\ }

\subsubsection{\texorpdfstring{\texttt{\ outset\ }}{ outset }}\label{parameters-outset}

\href{/docs/reference/layout/relative/}{relative} {or}
\href{/docs/reference/foundations/dictionary/}{dictionary}

{{ Settable }}

\phantomsection\label{parameters-outset-settable-tooltip}
Settable parameters can be customized for all following uses of the
function with a \texttt{\ set\ } rule.

How much to expand the rectangle\textquotesingle s size without
affecting the layout. See the
\href{/docs/reference/layout/box/\#parameters-outset}{box\textquotesingle s
documentation} for more details.

Default:
\texttt{\ }{\texttt{\ (\ }}\texttt{\ }{\texttt{\ :\ }}\texttt{\ }{\texttt{\ )\ }}\texttt{\ }

\subsubsection{\texorpdfstring{\texttt{\ body\ }}{ body }}\label{parameters-body}

\href{/docs/reference/foundations/none/}{none} {or}
\href{/docs/reference/foundations/content/}{content}

{{ Positional }}

\phantomsection\label{parameters-body-positional-tooltip}
Positional parameters are specified in order, without names.

{{ Settable }}

\phantomsection\label{parameters-body-settable-tooltip}
Settable parameters can be customized for all following uses of the
function with a \texttt{\ set\ } rule.

The content to place into the rectangle.

When this is omitted, the rectangle takes on a default size of at most
\texttt{\ }{\texttt{\ 45pt\ }}\texttt{\ } by
\texttt{\ }{\texttt{\ 30pt\ }}\texttt{\ } .

Default: \texttt{\ }{\texttt{\ none\ }}\texttt{\ }

\href{/docs/reference/visualize/polygon/}{\pandocbounded{\includesvg[keepaspectratio]{/assets/icons/16-arrow-right.svg}}}

{ Polygon } { Previous page }

\href{/docs/reference/visualize/square/}{\pandocbounded{\includesvg[keepaspectratio]{/assets/icons/16-arrow-right.svg}}}

{ Square } { Next page }


\title{typst.app/docs/reference/visualize/ellipse}

\begin{itemize}
\tightlist
\item
  \href{/docs}{\includesvg[width=0.16667in,height=0.16667in]{/assets/icons/16-docs-dark.svg}}
\item
  \includesvg[width=0.16667in,height=0.16667in]{/assets/icons/16-arrow-right.svg}
\item
  \href{/docs/reference/}{Reference}
\item
  \includesvg[width=0.16667in,height=0.16667in]{/assets/icons/16-arrow-right.svg}
\item
  \href{/docs/reference/visualize/}{Visualize}
\item
  \includesvg[width=0.16667in,height=0.16667in]{/assets/icons/16-arrow-right.svg}
\item
  \href{/docs/reference/visualize/ellipse/}{Ellipse}
\end{itemize}

\section{\texorpdfstring{\texttt{\ ellipse\ } {{ Element
}}}{ ellipse   Element }}\label{summary}

\phantomsection\label{element-tooltip}
Element functions can be customized with \texttt{\ set\ } and
\texttt{\ show\ } rules.

An ellipse with optional content.

\subsection{Example}\label{example}

\begin{verbatim}
// Without content.
#ellipse(width: 35%, height: 30pt)

// With content.
#ellipse[
  #set align(center)
  Automatically sized \
  to fit the content.
]
\end{verbatim}

\includegraphics[width=5in,height=\textheight,keepaspectratio]{/assets/docs/u35LFJMn0LDLxUBqOdjmvgAAAAAAAAAA.png}

\subsection{\texorpdfstring{{ Parameters
}}{ Parameters }}\label{parameters}

\phantomsection\label{parameters-tooltip}
Parameters are the inputs to a function. They are specified in
parentheses after the function name.

{ ellipse } (

{ \hyperref[parameters-width]{width :}
\href{/docs/reference/foundations/auto/}{auto}
\href{/docs/reference/layout/relative/}{relative} , } {
\hyperref[parameters-height]{height :}
\href{/docs/reference/foundations/auto/}{auto}
\href{/docs/reference/layout/relative/}{relative}
\href{/docs/reference/layout/fraction/}{fraction} , } {
\hyperref[parameters-fill]{fill :}
\href{/docs/reference/foundations/none/}{none}
\href{/docs/reference/visualize/color/}{color}
\href{/docs/reference/visualize/gradient/}{gradient}
\href{/docs/reference/visualize/pattern/}{pattern} , } {
\hyperref[parameters-stroke]{stroke :}
\href{/docs/reference/foundations/none/}{none}
\href{/docs/reference/foundations/auto/}{auto}
\href{/docs/reference/layout/length/}{length}
\href{/docs/reference/visualize/color/}{color}
\href{/docs/reference/visualize/gradient/}{gradient}
\href{/docs/reference/visualize/stroke/}{stroke}
\href{/docs/reference/visualize/pattern/}{pattern}
\href{/docs/reference/foundations/dictionary/}{dictionary} , } {
\hyperref[parameters-inset]{inset :}
\href{/docs/reference/layout/relative/}{relative}
\href{/docs/reference/foundations/dictionary/}{dictionary} , } {
\hyperref[parameters-outset]{outset :}
\href{/docs/reference/layout/relative/}{relative}
\href{/docs/reference/foundations/dictionary/}{dictionary} , } {
\hyperref[parameters-body]{}
\href{/docs/reference/foundations/none/}{none}
\href{/docs/reference/foundations/content/}{content} , }

) -\textgreater{} \href{/docs/reference/foundations/content/}{content}

\subsubsection{\texorpdfstring{\texttt{\ width\ }}{ width }}\label{parameters-width}

\href{/docs/reference/foundations/auto/}{auto} {or}
\href{/docs/reference/layout/relative/}{relative}

{{ Settable }}

\phantomsection\label{parameters-width-settable-tooltip}
Settable parameters can be customized for all following uses of the
function with a \texttt{\ set\ } rule.

The ellipse\textquotesingle s width, relative to its parent container.

Default: \texttt{\ }{\texttt{\ auto\ }}\texttt{\ }

\subsubsection{\texorpdfstring{\texttt{\ height\ }}{ height }}\label{parameters-height}

\href{/docs/reference/foundations/auto/}{auto} {or}
\href{/docs/reference/layout/relative/}{relative} {or}
\href{/docs/reference/layout/fraction/}{fraction}

{{ Settable }}

\phantomsection\label{parameters-height-settable-tooltip}
Settable parameters can be customized for all following uses of the
function with a \texttt{\ set\ } rule.

The ellipse\textquotesingle s height, relative to its parent container.

Default: \texttt{\ }{\texttt{\ auto\ }}\texttt{\ }

\subsubsection{\texorpdfstring{\texttt{\ fill\ }}{ fill }}\label{parameters-fill}

\href{/docs/reference/foundations/none/}{none} {or}
\href{/docs/reference/visualize/color/}{color} {or}
\href{/docs/reference/visualize/gradient/}{gradient} {or}
\href{/docs/reference/visualize/pattern/}{pattern}

{{ Settable }}

\phantomsection\label{parameters-fill-settable-tooltip}
Settable parameters can be customized for all following uses of the
function with a \texttt{\ set\ } rule.

How to fill the ellipse. See the
\href{/docs/reference/visualize/rect/\#parameters-fill}{rectangle\textquotesingle s
documentation} for more details.

Default: \texttt{\ }{\texttt{\ none\ }}\texttt{\ }

\subsubsection{\texorpdfstring{\texttt{\ stroke\ }}{ stroke }}\label{parameters-stroke}

\href{/docs/reference/foundations/none/}{none} {or}
\href{/docs/reference/foundations/auto/}{auto} {or}
\href{/docs/reference/layout/length/}{length} {or}
\href{/docs/reference/visualize/color/}{color} {or}
\href{/docs/reference/visualize/gradient/}{gradient} {or}
\href{/docs/reference/visualize/stroke/}{stroke} {or}
\href{/docs/reference/visualize/pattern/}{pattern} {or}
\href{/docs/reference/foundations/dictionary/}{dictionary}

{{ Settable }}

\phantomsection\label{parameters-stroke-settable-tooltip}
Settable parameters can be customized for all following uses of the
function with a \texttt{\ set\ } rule.

How to stroke the ellipse. See the
\href{/docs/reference/visualize/rect/\#parameters-stroke}{rectangle\textquotesingle s
documentation} for more details.

Default: \texttt{\ }{\texttt{\ auto\ }}\texttt{\ }

\subsubsection{\texorpdfstring{\texttt{\ inset\ }}{ inset }}\label{parameters-inset}

\href{/docs/reference/layout/relative/}{relative} {or}
\href{/docs/reference/foundations/dictionary/}{dictionary}

{{ Settable }}

\phantomsection\label{parameters-inset-settable-tooltip}
Settable parameters can be customized for all following uses of the
function with a \texttt{\ set\ } rule.

How much to pad the ellipse\textquotesingle s content. See the
\href{/docs/reference/layout/box/\#parameters-inset}{box\textquotesingle s
documentation} for more details.

Default:
\texttt{\ }{\texttt{\ 0\%\ }}\texttt{\ }{\texttt{\ +\ }}\texttt{\ }{\texttt{\ 5pt\ }}\texttt{\ }

\subsubsection{\texorpdfstring{\texttt{\ outset\ }}{ outset }}\label{parameters-outset}

\href{/docs/reference/layout/relative/}{relative} {or}
\href{/docs/reference/foundations/dictionary/}{dictionary}

{{ Settable }}

\phantomsection\label{parameters-outset-settable-tooltip}
Settable parameters can be customized for all following uses of the
function with a \texttt{\ set\ } rule.

How much to expand the ellipse\textquotesingle s size without affecting
the layout. See the
\href{/docs/reference/layout/box/\#parameters-outset}{box\textquotesingle s
documentation} for more details.

Default:
\texttt{\ }{\texttt{\ (\ }}\texttt{\ }{\texttt{\ :\ }}\texttt{\ }{\texttt{\ )\ }}\texttt{\ }

\subsubsection{\texorpdfstring{\texttt{\ body\ }}{ body }}\label{parameters-body}

\href{/docs/reference/foundations/none/}{none} {or}
\href{/docs/reference/foundations/content/}{content}

{{ Positional }}

\phantomsection\label{parameters-body-positional-tooltip}
Positional parameters are specified in order, without names.

{{ Settable }}

\phantomsection\label{parameters-body-settable-tooltip}
Settable parameters can be customized for all following uses of the
function with a \texttt{\ set\ } rule.

The content to place into the ellipse.

When this is omitted, the ellipse takes on a default size of at most
\texttt{\ }{\texttt{\ 45pt\ }}\texttt{\ } by
\texttt{\ }{\texttt{\ 30pt\ }}\texttt{\ } .

Default: \texttt{\ }{\texttt{\ none\ }}\texttt{\ }

\href{/docs/reference/visualize/color/}{\pandocbounded{\includesvg[keepaspectratio]{/assets/icons/16-arrow-right.svg}}}

{ Color } { Previous page }

\href{/docs/reference/visualize/gradient/}{\pandocbounded{\includesvg[keepaspectratio]{/assets/icons/16-arrow-right.svg}}}

{ Gradient } { Next page }


\title{typst.app/docs/reference/visualize/stroke}

\begin{itemize}
\tightlist
\item
  \href{/docs}{\includesvg[width=0.16667in,height=0.16667in]{/assets/icons/16-docs-dark.svg}}
\item
  \includesvg[width=0.16667in,height=0.16667in]{/assets/icons/16-arrow-right.svg}
\item
  \href{/docs/reference/}{Reference}
\item
  \includesvg[width=0.16667in,height=0.16667in]{/assets/icons/16-arrow-right.svg}
\item
  \href{/docs/reference/visualize/}{Visualize}
\item
  \includesvg[width=0.16667in,height=0.16667in]{/assets/icons/16-arrow-right.svg}
\item
  \href{/docs/reference/visualize/stroke/}{Stroke}
\end{itemize}

\section{\texorpdfstring{{ stroke }}{ stroke }}\label{summary}

Defines how to draw a line.

A stroke has a \emph{paint} (a solid color or gradient), a
\emph{thickness,} a line \emph{cap,} a line \emph{join,} a \emph{miter
limit,} and a \emph{dash} pattern. All of these values are optional and
have sensible defaults.

\subsection{Example}\label{example}

\begin{verbatim}
#set line(length: 100%)
#stack(
  spacing: 1em,
  line(stroke: 2pt + red),
  line(stroke: (paint: blue, thickness: 4pt, cap: "round")),
  line(stroke: (paint: blue, thickness: 1pt, dash: "dashed")),
  line(stroke: 2pt + gradient.linear(..color.map.rainbow)),
)
\end{verbatim}

\includegraphics[width=5in,height=\textheight,keepaspectratio]{/assets/docs/3NofubbwIllodsFawlNd8wAAAAAAAAAA.png}

\subsection{Simple strokes}\label{simple-strokes}

You can create a simple solid stroke from a color, a thickness, or a
combination of the two. Specifically, wherever a stroke is expected you
can pass any of the following values:

\begin{itemize}
\tightlist
\item
  A length specifying the stroke\textquotesingle s thickness. The color
  is inherited, defaulting to black.
\item
  A color to use for the stroke. The thickness is inherited, defaulting
  to \texttt{\ }{\texttt{\ 1pt\ }}\texttt{\ } .
\item
  A stroke combined from color and thickness using the \texttt{\ +\ }
  operator as in
  \texttt{\ }{\texttt{\ 2pt\ }}\texttt{\ }{\texttt{\ +\ }}\texttt{\ red\ }
  .
\end{itemize}

For full control, you can also provide a
\href{/docs/reference/foundations/dictionary/}{dictionary} or a
\texttt{\ stroke\ } object to any function that expects a stroke. The
dictionary\textquotesingle s keys may include any of the parameters for
the constructor function, shown below.

\subsection{Fields}\label{fields}

On a stroke object, you can access any of the fields listed in the
constructor function. For example,
\texttt{\ }{\texttt{\ (\ }}\texttt{\ }{\texttt{\ 2pt\ }}\texttt{\ }{\texttt{\ +\ }}\texttt{\ blue\ }{\texttt{\ )\ }}\texttt{\ }{\texttt{\ .\ }}\texttt{\ thickness\ }
is \texttt{\ }{\texttt{\ 2pt\ }}\texttt{\ } . Meanwhile,
\texttt{\ }{\texttt{\ stroke\ }}\texttt{\ }{\texttt{\ (\ }}\texttt{\ red\ }{\texttt{\ )\ }}\texttt{\ }{\texttt{\ .\ }}\texttt{\ cap\ }
is \texttt{\ }{\texttt{\ auto\ }}\texttt{\ } because
it\textquotesingle s unspecified. Fields set to
\texttt{\ }{\texttt{\ auto\ }}\texttt{\ } are inherited.

\subsection{\texorpdfstring{Constructor
{}}{Constructor }}\label{constructor}

\phantomsection\label{constructor-constructor-tooltip}
If a type has a constructor, you can call it like a function to create a
new value of the type.

Converts a value to a stroke or constructs a stroke with the given
parameters.

Note that in most cases you do not need to convert values to strokes in
order to use them, as they will be converted automatically. However,
this constructor can be useful to ensure a value has all the fields of a
stroke.

{ stroke } (

{ \href{/docs/reference/foundations/auto/}{auto}
\href{/docs/reference/visualize/color/}{color}
\href{/docs/reference/visualize/gradient/}{gradient}
\href{/docs/reference/visualize/pattern/}{pattern} , } {
\href{/docs/reference/foundations/auto/}{auto}
\href{/docs/reference/layout/length/}{length} , } {
\href{/docs/reference/foundations/auto/}{auto}
\href{/docs/reference/foundations/str/}{str} , } {
\href{/docs/reference/foundations/auto/}{auto}
\href{/docs/reference/foundations/str/}{str} , } {
\href{/docs/reference/foundations/none/}{none}
\href{/docs/reference/foundations/auto/}{auto}
\href{/docs/reference/foundations/str/}{str}
\href{/docs/reference/foundations/array/}{array}
\href{/docs/reference/foundations/dictionary/}{dictionary} , } {
\href{/docs/reference/foundations/auto/}{auto}
\href{/docs/reference/foundations/float/}{float} , }

) -\textgreater{} \href{/docs/reference/visualize/stroke/}{stroke}

\begin{verbatim}
#let my-func(x) = {
    x = stroke(x) // Convert to a stroke
    [Stroke has thickness #x.thickness.]
}
#my-func(3pt) \
#my-func(red) \
#my-func(stroke(cap: "round", thickness: 1pt))
\end{verbatim}

\includegraphics[width=5in,height=\textheight,keepaspectratio]{/assets/docs/oulcXDNcpunCSxVvCPXMJQAAAAAAAAAA.png}

\paragraph{\texorpdfstring{\texttt{\ paint\ }}{ paint }}\label{constructor-paint}

\href{/docs/reference/foundations/auto/}{auto} {or}
\href{/docs/reference/visualize/color/}{color} {or}
\href{/docs/reference/visualize/gradient/}{gradient} {or}
\href{/docs/reference/visualize/pattern/}{pattern}

{Required} {{ Positional }}

\phantomsection\label{constructor-paint-positional-tooltip}
Positional parameters are specified in order, without names.

The color or gradient to use for the stroke.

If set to \texttt{\ }{\texttt{\ auto\ }}\texttt{\ } , the value is
inherited, defaulting to \texttt{\ black\ } .

\paragraph{\texorpdfstring{\texttt{\ thickness\ }}{ thickness }}\label{constructor-thickness}

\href{/docs/reference/foundations/auto/}{auto} {or}
\href{/docs/reference/layout/length/}{length}

{Required} {{ Positional }}

\phantomsection\label{constructor-thickness-positional-tooltip}
Positional parameters are specified in order, without names.

The stroke\textquotesingle s thickness.

If set to \texttt{\ }{\texttt{\ auto\ }}\texttt{\ } , the value is
inherited, defaulting to \texttt{\ }{\texttt{\ 1pt\ }}\texttt{\ } .

\paragraph{\texorpdfstring{\texttt{\ cap\ }}{ cap }}\label{constructor-cap}

\href{/docs/reference/foundations/auto/}{auto} {or}
\href{/docs/reference/foundations/str/}{str}

{Required} {{ Positional }}

\phantomsection\label{constructor-cap-positional-tooltip}
Positional parameters are specified in order, without names.

How the ends of the stroke are rendered.

If set to \texttt{\ }{\texttt{\ auto\ }}\texttt{\ } , the value is
inherited, defaulting to \texttt{\ }{\texttt{\ "butt"\ }}\texttt{\ } .

\begin{longtable}[]{@{}ll@{}}
\toprule\noalign{}
Variant & Details \\
\midrule\noalign{}
\endhead
\bottomrule\noalign{}
\endlastfoot
\texttt{\ "\ butt\ "\ } & Square stroke cap with the edge at the
stroke\textquotesingle s end point. \\
\texttt{\ "\ round\ "\ } & Circular stroke cap centered at the
stroke\textquotesingle s end point. \\
\texttt{\ "\ square\ "\ } & Square stroke cap centered at the
stroke\textquotesingle s end point. \\
\end{longtable}

\paragraph{\texorpdfstring{\texttt{\ join\ }}{ join }}\label{constructor-join}

\href{/docs/reference/foundations/auto/}{auto} {or}
\href{/docs/reference/foundations/str/}{str}

{Required} {{ Positional }}

\phantomsection\label{constructor-join-positional-tooltip}
Positional parameters are specified in order, without names.

How sharp turns are rendered.

If set to \texttt{\ }{\texttt{\ auto\ }}\texttt{\ } , the value is
inherited, defaulting to \texttt{\ }{\texttt{\ "miter"\ }}\texttt{\ } .

\begin{longtable}[]{@{}ll@{}}
\toprule\noalign{}
Variant & Details \\
\midrule\noalign{}
\endhead
\bottomrule\noalign{}
\endlastfoot
\texttt{\ "\ miter\ "\ } & Segments are joined with sharp edges. Sharp
bends exceeding the miter limit are bevelled instead. \\
\texttt{\ "\ round\ "\ } & Segments are joined with circular corners. \\
\texttt{\ "\ bevel\ "\ } & Segments are joined with a bevel (a straight
edge connecting the butts of the joined segments). \\
\end{longtable}

\paragraph{\texorpdfstring{\texttt{\ dash\ }}{ dash }}\label{constructor-dash}

\href{/docs/reference/foundations/none/}{none} {or}
\href{/docs/reference/foundations/auto/}{auto} {or}
\href{/docs/reference/foundations/str/}{str} {or}
\href{/docs/reference/foundations/array/}{array} {or}
\href{/docs/reference/foundations/dictionary/}{dictionary}

{Required} {{ Positional }}

\phantomsection\label{constructor-dash-positional-tooltip}
Positional parameters are specified in order, without names.

The dash pattern to use. This can be:

\begin{itemize}
\tightlist
\item
  One of the predefined patterns:

  \begin{itemize}
  \tightlist
  \item
    \texttt{\ }{\texttt{\ "solid"\ }}\texttt{\ } or
    \texttt{\ }{\texttt{\ none\ }}\texttt{\ }
  \item
    \texttt{\ }{\texttt{\ "dotted"\ }}\texttt{\ }
  \item
    \texttt{\ }{\texttt{\ "densely-dotted"\ }}\texttt{\ }
  \item
    \texttt{\ }{\texttt{\ "loosely-dotted"\ }}\texttt{\ }
  \item
    \texttt{\ }{\texttt{\ "dashed"\ }}\texttt{\ }
  \item
    \texttt{\ }{\texttt{\ "densely-dashed"\ }}\texttt{\ }
  \item
    \texttt{\ }{\texttt{\ "loosely-dashed"\ }}\texttt{\ }
  \item
    \texttt{\ }{\texttt{\ "dash-dotted"\ }}\texttt{\ }
  \item
    \texttt{\ }{\texttt{\ "densely-dash-dotted"\ }}\texttt{\ }
  \item
    \texttt{\ }{\texttt{\ "loosely-dash-dotted"\ }}\texttt{\ }
  \end{itemize}
\item
  An \href{/docs/reference/foundations/array/}{array} with alternating
  lengths for dashes and gaps. You can also use the string
  \texttt{\ }{\texttt{\ "dot"\ }}\texttt{\ } for a length equal to the
  line thickness.
\item
  A \href{/docs/reference/foundations/dictionary/}{dictionary} with the
  keys \texttt{\ array\ } (same as the array above), and
  \texttt{\ phase\ } (of type
  \href{/docs/reference/layout/length/}{length} ), which defines where
  in the pattern to start drawing.
\end{itemize}

If set to \texttt{\ }{\texttt{\ auto\ }}\texttt{\ } , the value is
inherited, defaulting to \texttt{\ }{\texttt{\ none\ }}\texttt{\ } .

\includesvg[width=0.16667in,height=0.16667in]{/assets/icons/16-arrow-right.svg}
View options

\begin{longtable}[]{@{}ll@{}}
\toprule\noalign{}
Variant & Details \\
\midrule\noalign{}
\endhead
\bottomrule\noalign{}
\endlastfoot
\texttt{\ "\ solid\ "\ } & \\
\texttt{\ "\ dotted\ "\ } & \\
\texttt{\ "\ densely-dotted\ "\ } & \\
\texttt{\ "\ loosely-dotted\ "\ } & \\
\texttt{\ "\ dashed\ "\ } & \\
\texttt{\ "\ densely-dashed\ "\ } & \\
\texttt{\ "\ loosely-dashed\ "\ } & \\
\texttt{\ "\ dash-dotted\ "\ } & \\
\texttt{\ "\ densely-dash-dotted\ "\ } & \\
\texttt{\ "\ loosely-dash-dotted\ "\ } & \\
\end{longtable}

\includesvg[width=0.16667in,height=0.16667in]{/assets/icons/16-arrow-right.svg}
View example

\begin{verbatim}
#set line(length: 100%, stroke: 2pt)
#stack(
  spacing: 1em,
  line(stroke: (dash: "dashed")),
  line(stroke: (dash: (10pt, 5pt, "dot", 5pt))),
  line(stroke: (dash: (array: (10pt, 5pt, "dot", 5pt), phase: 10pt))),
)
\end{verbatim}

\includegraphics[width=5in,height=\textheight,keepaspectratio]{/assets/docs/P38gFluKZcw64WdZR85nHgAAAAAAAAAA.png}

\paragraph{\texorpdfstring{\texttt{\ miter-limit\ }}{ miter-limit }}\label{constructor-miter-limit}

\href{/docs/reference/foundations/auto/}{auto} {or}
\href{/docs/reference/foundations/float/}{float}

{Required} {{ Positional }}

\phantomsection\label{constructor-miter-limit-positional-tooltip}
Positional parameters are specified in order, without names.

Number at which protruding sharp bends are rendered with a bevel instead
or a miter join. The higher the number, the sharper an angle can be
before it is bevelled. Only applicable if \texttt{\ join\ } is
\texttt{\ }{\texttt{\ "miter"\ }}\texttt{\ } .

Specifically, the miter limit is the maximum ratio between the
corner\textquotesingle s protrusion length and the
stroke\textquotesingle s thickness.

If set to \texttt{\ }{\texttt{\ auto\ }}\texttt{\ } , the value is
inherited, defaulting to \texttt{\ }{\texttt{\ 4.0\ }}\texttt{\ } .

\includesvg[width=0.16667in,height=0.16667in]{/assets/icons/16-arrow-right.svg}
View example

\begin{verbatim}
#let points = ((15pt, 0pt), (0pt, 30pt), (30pt, 30pt), (10pt, 20pt))
#set path(stroke: 6pt + blue)
#stack(
    dir: ltr,
    spacing: 1cm,
    path(stroke: (miter-limit: 1), ..points),
    path(stroke: (miter-limit: 4), ..points),
    path(stroke: (miter-limit: 5), ..points),
)
\end{verbatim}

\includegraphics[width=5in,height=\textheight,keepaspectratio]{/assets/docs/3zeU1BuQq8_VfdTfAQbv5QAAAAAAAAAA.png}

\href{/docs/reference/visualize/square/}{\pandocbounded{\includesvg[keepaspectratio]{/assets/icons/16-arrow-right.svg}}}

{ Square } { Previous page }

\href{/docs/reference/introspection/}{\pandocbounded{\includesvg[keepaspectratio]{/assets/icons/16-arrow-right.svg}}}

{ Introspection } { Next page }


\title{typst.app/docs/reference/visualize/gradient}

\begin{itemize}
\tightlist
\item
  \href{/docs}{\includesvg[width=0.16667in,height=0.16667in]{/assets/icons/16-docs-dark.svg}}
\item
  \includesvg[width=0.16667in,height=0.16667in]{/assets/icons/16-arrow-right.svg}
\item
  \href{/docs/reference/}{Reference}
\item
  \includesvg[width=0.16667in,height=0.16667in]{/assets/icons/16-arrow-right.svg}
\item
  \href{/docs/reference/visualize/}{Visualize}
\item
  \includesvg[width=0.16667in,height=0.16667in]{/assets/icons/16-arrow-right.svg}
\item
  \href{/docs/reference/visualize/gradient/}{Gradient}
\end{itemize}

\section{\texorpdfstring{{ gradient }}{ gradient }}\label{summary}

A color gradient.

Typst supports linear gradients through the
\href{/docs/reference/visualize/gradient/\#definitions-linear}{\texttt{\ gradient.linear\ }
function} , radial gradients through the
\href{/docs/reference/visualize/gradient/\#definitions-radial}{\texttt{\ gradient.radial\ }
function} , and conic gradients through the
\href{/docs/reference/visualize/gradient/\#definitions-conic}{\texttt{\ gradient.conic\ }
function} .

A gradient can be used for the following purposes:

\begin{itemize}
\tightlist
\item
  As a fill to paint the interior of a shape:
  \texttt{\ }{\texttt{\ rect\ }}\texttt{\ }{\texttt{\ (\ }}\texttt{\ fill\ }{\texttt{\ :\ }}\texttt{\ gradient\ }{\texttt{\ .\ }}\texttt{\ }{\texttt{\ linear\ }}\texttt{\ }{\texttt{\ (\ }}\texttt{\ }{\texttt{\ ..\ }}\texttt{\ }{\texttt{\ )\ }}\texttt{\ }{\texttt{\ )\ }}\texttt{\ }
\item
  As a stroke to paint the outline of a shape:
  \texttt{\ }{\texttt{\ rect\ }}\texttt{\ }{\texttt{\ (\ }}\texttt{\ stroke\ }{\texttt{\ :\ }}\texttt{\ }{\texttt{\ 1pt\ }}\texttt{\ }{\texttt{\ +\ }}\texttt{\ gradient\ }{\texttt{\ .\ }}\texttt{\ }{\texttt{\ linear\ }}\texttt{\ }{\texttt{\ (\ }}\texttt{\ }{\texttt{\ ..\ }}\texttt{\ }{\texttt{\ )\ }}\texttt{\ }{\texttt{\ )\ }}\texttt{\ }
\item
  As the fill of text:
  \texttt{\ }{\texttt{\ set\ }}\texttt{\ }{\texttt{\ text\ }}\texttt{\ }{\texttt{\ (\ }}\texttt{\ fill\ }{\texttt{\ :\ }}\texttt{\ gradient\ }{\texttt{\ .\ }}\texttt{\ }{\texttt{\ linear\ }}\texttt{\ }{\texttt{\ (\ }}\texttt{\ }{\texttt{\ ..\ }}\texttt{\ }{\texttt{\ )\ }}\texttt{\ }{\texttt{\ )\ }}\texttt{\ }
\item
  As a color map you can
  \href{/docs/reference/visualize/gradient/\#definitions-sample}{sample}
  from:
  \texttt{\ gradient\ }{\texttt{\ .\ }}\texttt{\ }{\texttt{\ linear\ }}\texttt{\ }{\texttt{\ (\ }}\texttt{\ }{\texttt{\ ..\ }}\texttt{\ }{\texttt{\ )\ }}\texttt{\ }{\texttt{\ .\ }}\texttt{\ }{\texttt{\ sample\ }}\texttt{\ }{\texttt{\ (\ }}\texttt{\ }{\texttt{\ 50\%\ }}\texttt{\ }{\texttt{\ )\ }}\texttt{\ }
\end{itemize}

\subsection{Examples}\label{examples}

\begin{verbatim}
#stack(
  dir: ltr,
  spacing: 1fr,
  square(fill: gradient.linear(..color.map.rainbow)),
  square(fill: gradient.radial(..color.map.rainbow)),
  square(fill: gradient.conic(..color.map.rainbow)),
)
\end{verbatim}

\includegraphics[width=5in,height=\textheight,keepaspectratio]{/assets/docs/_ynuy5GKkV7ADtX87C9EiAAAAAAAAAAA.png}

Gradients are also supported on text, but only when setting the
\href{/docs/reference/visualize/gradient/\#definitions-relative}{relativeness}
to either \texttt{\ }{\texttt{\ auto\ }}\texttt{\ } (the default value)
or \texttt{\ }{\texttt{\ "parent"\ }}\texttt{\ } . To create
word-by-word or glyph-by-glyph gradients, you can wrap the words or
characters of your text in \href{/docs/reference/layout/box/}{boxes}
manually or through a \href{/docs/reference/styling/\#show-rules}{show
rule} .

\begin{verbatim}
#set text(fill: gradient.linear(red, blue))
#let rainbow(content) = {
  set text(fill: gradient.linear(..color.map.rainbow))
  box(content)
}

This is a gradient on text, but with a #rainbow[twist]!
\end{verbatim}

\includegraphics[width=4.85417in,height=\textheight,keepaspectratio]{/assets/docs/ch0LALUCwuQoVDnxrE_UZwAAAAAAAAAA.png}

\subsection{Stops}\label{stops}

A gradient is composed of a series of stops. Each of these stops has a
color and an offset. The offset is a
\href{/docs/reference/layout/ratio/}{ratio} between
\texttt{\ }{\texttt{\ 0\%\ }}\texttt{\ } and
\texttt{\ }{\texttt{\ 100\%\ }}\texttt{\ } or an angle between
\texttt{\ }{\texttt{\ 0deg\ }}\texttt{\ } and
\texttt{\ }{\texttt{\ 360deg\ }}\texttt{\ } . The offset is a relative
position that determines how far along the gradient the stop is located.
The stop\textquotesingle s color is the color of the gradient at that
position. You can choose to omit the offsets when defining a gradient.
In this case, Typst will space all stops evenly.

\subsection{Relativeness}\label{relativeness}

The location of the \texttt{\ }{\texttt{\ 0\%\ }}\texttt{\ } and
\texttt{\ }{\texttt{\ 100\%\ }}\texttt{\ } stops depends on the
dimensions of a container. This container can either be the shape that
it is being painted on, or the closest surrounding container. This is
controlled by the \texttt{\ relative\ } argument of a gradient
constructor. By default, gradients are relative to the shape they are
being painted on, unless the gradient is applied on text, in which case
they are relative to the closest ancestor container.

Typst determines the ancestor container as follows:

\begin{itemize}
\tightlist
\item
  For shapes that are placed at the root/top level of the document, the
  closest ancestor is the page itself.
\item
  For other shapes, the ancestor is the innermost
  \href{/docs/reference/layout/block/}{\texttt{\ block\ }} or
  \href{/docs/reference/layout/box/}{\texttt{\ box\ }} that contains the
  shape. This includes the boxes and blocks that are implicitly created
  by show rules and elements. For example, a
  \href{/docs/reference/layout/rotate/}{\texttt{\ rotate\ }} will not
  affect the parent of a gradient, but a
  \href{/docs/reference/layout/grid/}{\texttt{\ grid\ }} will.
\end{itemize}

\subsection{Color spaces and
interpolation}\label{color-spaces-and-interpolation}

Gradients can be interpolated in any color space. By default, gradients
are interpolated in the
\href{/docs/reference/visualize/color/\#definitions-oklab}{Oklab} color
space, which is a
\href{https://programmingdesignsystems.com/color/perceptually-uniform-color-spaces/index.html}{perceptually
uniform} color space. This means that the gradient will be perceived as
having a smooth progression of colors. This is particularly useful for
data visualization.

However, you can choose to interpolate the gradient in any supported
color space you want, but beware that some color spaces are not suitable
for perceptually interpolating between colors. Consult the table below
when choosing an interpolation space.

\begin{longtable}[]{@{}ll@{}}
\toprule\noalign{}
Color space & Perceptually uniform? \\
\midrule\noalign{}
\endhead
\bottomrule\noalign{}
\endlastfoot
\href{/docs/reference/visualize/color/\#definitions-oklab}{Oklab} &
\emph{Yes} \\
\href{/docs/reference/visualize/color/\#definitions-oklch}{Oklch} &
\emph{Yes} \\
\href{/docs/reference/visualize/color/\#definitions-rgb}{sRGB} &
\emph{No} \\
\href{/docs/reference/visualize/color/\#definitions-linear-rgb}{linear-RGB}
& \emph{Yes} \\
\href{/docs/reference/visualize/color/\#definitions-cmyk}{CMYK} &
\emph{No} \\
\href{/docs/reference/visualize/color/\#definitions-luma}{Grayscale} &
\emph{Yes} \\
\href{/docs/reference/visualize/color/\#definitions-hsl}{HSL} &
\emph{No} \\
\href{/docs/reference/visualize/color/\#definitions-hsv}{HSV} &
\emph{No} \\
\end{longtable}

\includegraphics[width=5in,height=\textheight,keepaspectratio]{/assets/docs/hDyhl3_sixunf7X8Ctx-hAAAAAAAAAAA.png}

\subsection{Direction}\label{direction}

Some gradients are sensitive to direction. For example, a linear
gradient has an angle that determines its direction. Typst uses a
clockwise angle, with 0° being from left to right, 90° from top to
bottom, 180° from right to left, and 270° from bottom to top.

\begin{verbatim}
#stack(
  dir: ltr,
  spacing: 1fr,
  square(fill: gradient.linear(red, blue, angle: 0deg)),
  square(fill: gradient.linear(red, blue, angle: 90deg)),
  square(fill: gradient.linear(red, blue, angle: 180deg)),
  square(fill: gradient.linear(red, blue, angle: 270deg)),
)
\end{verbatim}

\includegraphics[width=5in,height=\textheight,keepaspectratio]{/assets/docs/cXgxeaTP2ci7NL16a3rB_gAAAAAAAAAA.png}

\subsection{Presets}\label{presets}

Typst predefines color maps that you can use with your gradients. See
the
\href{/docs/reference/visualize/color/\#predefined-color-maps}{\texttt{\ color\ }}
documentation for more details.

\subsection{Note on file sizes}\label{note-on-file-sizes}

Gradients can be quite large, especially if they have many stops. This
is because gradients are stored as a list of colors and offsets, which
can take up a lot of space. If you are concerned about file sizes, you
should consider the following:

\begin{itemize}
\tightlist
\item
  SVG gradients are currently inefficiently encoded. This will be
  improved in the future.
\item
  PDF gradients in the
  \href{/docs/reference/visualize/color/\#definitions-oklab}{\texttt{\ color.oklab\ }}
  ,
  \href{/docs/reference/visualize/color/\#definitions-hsv}{\texttt{\ color.hsv\ }}
  ,
  \href{/docs/reference/visualize/color/\#definitions-hsl}{\texttt{\ color.hsl\ }}
  , and
  \href{/docs/reference/visualize/color/\#definitions-oklch}{\texttt{\ color.oklch\ }}
  color spaces are stored as a list of
  \href{/docs/reference/visualize/color/\#definitions-rgb}{\texttt{\ color.rgb\ }}
  colors with extra stops in between. This avoids needing to encode
  these color spaces in your PDF file, but it does add extra stops to
  your gradient, which can increase the file size.
\end{itemize}

\subsection{\texorpdfstring{{ Definitions
}}{ Definitions }}\label{definitions}

\phantomsection\label{definitions-tooltip}
Functions and types and can have associated definitions. These are
accessed by specifying the function or type, followed by a period, and
then the definition\textquotesingle s name.

\subsubsection{\texorpdfstring{\texttt{\ linear\ }}{ linear }}\label{definitions-linear}

Creates a new linear gradient, in which colors transition along a
straight line.

gradient { . } { linear } (

{ \hyperref[definitions-linear-parameters-stops]{..}
\href{/docs/reference/visualize/color/}{color}
\href{/docs/reference/foundations/array/}{array} , } {
\hyperref[definitions-linear-parameters-space]{space :} { any } , } {
\hyperref[definitions-linear-parameters-relative]{relative :}
\href{/docs/reference/foundations/auto/}{auto}
\href{/docs/reference/foundations/str/}{str} , } {
\href{/docs/reference/layout/direction/}{direction} , } {
\href{/docs/reference/layout/angle/}{angle} , }

) -\textgreater{} \href{/docs/reference/visualize/gradient/}{gradient}

\begin{verbatim}
#rect(
  width: 100%,
  height: 20pt,
  fill: gradient.linear(
    ..color.map.viridis,
  ),
)
\end{verbatim}

\includegraphics[width=5in,height=\textheight,keepaspectratio]{/assets/docs/3vCVaADmPcUqYOLma4-wcgAAAAAAAAAA.png}

\paragraph{\texorpdfstring{\texttt{\ stops\ }}{ stops }}\label{definitions-linear-stops}

\href{/docs/reference/visualize/color/}{color} {or}
\href{/docs/reference/foundations/array/}{array}

{Required} {{ Positional }}

\phantomsection\label{definitions-linear-stops-positional-tooltip}
Positional parameters are specified in order, without names.

{{ Variadic }}

\phantomsection\label{definitions-linear-stops-variadic-tooltip}
Variadic parameters can be specified multiple times.

The color \hyperref[stops]{stops} of the gradient.

\paragraph{\texorpdfstring{\texttt{\ space\ }}{ space }}\label{definitions-linear-space}

{ any }

The color space in which to interpolate the gradient.

Defaults to a perceptually uniform color space called
\href{/docs/reference/visualize/color/\#definitions-oklab}{Oklab} .

Default: \texttt{\ oklab\ }

\paragraph{\texorpdfstring{\texttt{\ relative\ }}{ relative }}\label{definitions-linear-relative}

\href{/docs/reference/foundations/auto/}{auto} {or}
\href{/docs/reference/foundations/str/}{str}

The \hyperref[relativeness]{relative placement} of the gradient.

For an element placed at the root/top level of the document, the parent
is the page itself. For other elements, the parent is the innermost
block, box, column, grid, or stack that contains the element.

\begin{longtable}[]{@{}ll@{}}
\toprule\noalign{}
Variant & Details \\
\midrule\noalign{}
\endhead
\bottomrule\noalign{}
\endlastfoot
\texttt{\ "\ self\ "\ } & The gradient is relative to itself (its own
bounding box). \\
\texttt{\ "\ parent\ "\ } & The gradient is relative to its parent (the
parent\textquotesingle s bounding box). \\
\end{longtable}

Default: \texttt{\ }{\texttt{\ auto\ }}\texttt{\ }

\paragraph{\texorpdfstring{\texttt{\ dir\ }}{ dir }}\label{definitions-linear-dir}

\href{/docs/reference/layout/direction/}{direction}

{{ Positional }}

\phantomsection\label{definitions-linear-dir-positional-tooltip}
Positional parameters are specified in order, without names.

The direction of the gradient.

Default: \texttt{\ ltr\ }

\paragraph{\texorpdfstring{\texttt{\ angle\ }}{ angle }}\label{definitions-linear-angle}

\href{/docs/reference/layout/angle/}{angle}

{Required} {{ Positional }}

\phantomsection\label{definitions-linear-angle-positional-tooltip}
Positional parameters are specified in order, without names.

The angle of the gradient.

\subsubsection{\texorpdfstring{\texttt{\ radial\ }}{ radial }}\label{definitions-radial}

Creates a new radial gradient, in which colors radiate away from an
origin.

The gradient is defined by two circles: the focal circle and the end
circle. The focal circle is a circle with center
\texttt{\ focal-center\ } and radius \texttt{\ focal-radius\ } , that
defines the points at which the gradient starts and has the color of the
first stop. The end circle is a circle with center \texttt{\ center\ }
and radius \texttt{\ radius\ } , that defines the points at which the
gradient ends and has the color of the last stop. The gradient is then
interpolated between these two circles.

Using these four values, also called the focal point for the starting
circle and the center and radius for the end circle, we can define a
gradient with more interesting properties than a basic radial gradient.

gradient { . } { radial } (

{ \hyperref[definitions-radial-parameters-stops]{..}
\href{/docs/reference/visualize/color/}{color}
\href{/docs/reference/foundations/array/}{array} , } {
\hyperref[definitions-radial-parameters-space]{space :} { any } , } {
\hyperref[definitions-radial-parameters-relative]{relative :}
\href{/docs/reference/foundations/auto/}{auto}
\href{/docs/reference/foundations/str/}{str} , } {
\hyperref[definitions-radial-parameters-center]{center :}
\href{/docs/reference/foundations/array/}{array} , } {
\hyperref[definitions-radial-parameters-radius]{radius :}
\href{/docs/reference/layout/ratio/}{ratio} , } {
\hyperref[definitions-radial-parameters-focal-center]{focal-center :}
\href{/docs/reference/foundations/auto/}{auto}
\href{/docs/reference/foundations/array/}{array} , } {
\hyperref[definitions-radial-parameters-focal-radius]{focal-radius :}
\href{/docs/reference/layout/ratio/}{ratio} , }

) -\textgreater{} \href{/docs/reference/visualize/gradient/}{gradient}

\begin{verbatim}
#stack(
  dir: ltr,
  spacing: 1fr,
  circle(fill: gradient.radial(
    ..color.map.viridis,
  )),
  circle(fill: gradient.radial(
    ..color.map.viridis,
    focal-center: (10%, 40%),
    focal-radius: 5%,
  )),
)
\end{verbatim}

\includegraphics[width=5in,height=\textheight,keepaspectratio]{/assets/docs/IfkE7bIcLhrH24l0nk0sIQAAAAAAAAAA.png}

\paragraph{\texorpdfstring{\texttt{\ stops\ }}{ stops }}\label{definitions-radial-stops}

\href{/docs/reference/visualize/color/}{color} {or}
\href{/docs/reference/foundations/array/}{array}

{Required} {{ Positional }}

\phantomsection\label{definitions-radial-stops-positional-tooltip}
Positional parameters are specified in order, without names.

{{ Variadic }}

\phantomsection\label{definitions-radial-stops-variadic-tooltip}
Variadic parameters can be specified multiple times.

The color \hyperref[stops]{stops} of the gradient.

\paragraph{\texorpdfstring{\texttt{\ space\ }}{ space }}\label{definitions-radial-space}

{ any }

The color space in which to interpolate the gradient.

Defaults to a perceptually uniform color space called
\href{/docs/reference/visualize/color/\#definitions-oklab}{Oklab} .

Default: \texttt{\ oklab\ }

\paragraph{\texorpdfstring{\texttt{\ relative\ }}{ relative }}\label{definitions-radial-relative}

\href{/docs/reference/foundations/auto/}{auto} {or}
\href{/docs/reference/foundations/str/}{str}

The \hyperref[relativeness]{relative placement} of the gradient.

For an element placed at the root/top level of the document, the parent
is the page itself. For other elements, the parent is the innermost
block, box, column, grid, or stack that contains the element.

\begin{longtable}[]{@{}ll@{}}
\toprule\noalign{}
Variant & Details \\
\midrule\noalign{}
\endhead
\bottomrule\noalign{}
\endlastfoot
\texttt{\ "\ self\ "\ } & The gradient is relative to itself (its own
bounding box). \\
\texttt{\ "\ parent\ "\ } & The gradient is relative to its parent (the
parent\textquotesingle s bounding box). \\
\end{longtable}

Default: \texttt{\ }{\texttt{\ auto\ }}\texttt{\ }

\paragraph{\texorpdfstring{\texttt{\ center\ }}{ center }}\label{definitions-radial-center}

\href{/docs/reference/foundations/array/}{array}

The center of the end circle of the gradient.

A value of
\texttt{\ }{\texttt{\ (\ }}\texttt{\ }{\texttt{\ 50\%\ }}\texttt{\ }{\texttt{\ ,\ }}\texttt{\ }{\texttt{\ 50\%\ }}\texttt{\ }{\texttt{\ )\ }}\texttt{\ }
means that the end circle is centered inside of its container.

Default:
\texttt{\ }{\texttt{\ (\ }}\texttt{\ }{\texttt{\ 50\%\ }}\texttt{\ }{\texttt{\ ,\ }}\texttt{\ }{\texttt{\ 50\%\ }}\texttt{\ }{\texttt{\ )\ }}\texttt{\ }

\paragraph{\texorpdfstring{\texttt{\ radius\ }}{ radius }}\label{definitions-radial-radius}

\href{/docs/reference/layout/ratio/}{ratio}

The radius of the end circle of the gradient.

By default, it is set to \texttt{\ }{\texttt{\ 50\%\ }}\texttt{\ } . The
ending radius must be bigger than the focal radius.

Default: \texttt{\ }{\texttt{\ 50\%\ }}\texttt{\ }

\paragraph{\texorpdfstring{\texttt{\ focal-center\ }}{ focal-center }}\label{definitions-radial-focal-center}

\href{/docs/reference/foundations/auto/}{auto} {or}
\href{/docs/reference/foundations/array/}{array}

The center of the focal circle of the gradient.

The focal center must be inside of the end circle.

A value of
\texttt{\ }{\texttt{\ (\ }}\texttt{\ }{\texttt{\ 50\%\ }}\texttt{\ }{\texttt{\ ,\ }}\texttt{\ }{\texttt{\ 50\%\ }}\texttt{\ }{\texttt{\ )\ }}\texttt{\ }
means that the focal circle is centered inside of its container.

By default it is set to the same as the center of the last circle.

Default: \texttt{\ }{\texttt{\ auto\ }}\texttt{\ }

\paragraph{\texorpdfstring{\texttt{\ focal-radius\ }}{ focal-radius }}\label{definitions-radial-focal-radius}

\href{/docs/reference/layout/ratio/}{ratio}

The radius of the focal circle of the gradient.

The focal center must be inside of the end circle.

By default, it is set to \texttt{\ }{\texttt{\ 0\%\ }}\texttt{\ } . The
focal radius must be smaller than the ending radius`.

Default: \texttt{\ }{\texttt{\ 0\%\ }}\texttt{\ }

\subsubsection{\texorpdfstring{\texttt{\ conic\ }}{ conic }}\label{definitions-conic}

Creates a new conic gradient, in which colors change radially around a
center point.

You can control the center point of the gradient by using the
\texttt{\ center\ } argument. By default, the center point is the center
of the shape.

gradient { . } { conic } (

{ \hyperref[definitions-conic-parameters-stops]{..}
\href{/docs/reference/visualize/color/}{color}
\href{/docs/reference/foundations/array/}{array} , } {
\hyperref[definitions-conic-parameters-angle]{angle :}
\href{/docs/reference/layout/angle/}{angle} , } {
\hyperref[definitions-conic-parameters-space]{space :} { any } , } {
\hyperref[definitions-conic-parameters-relative]{relative :}
\href{/docs/reference/foundations/auto/}{auto}
\href{/docs/reference/foundations/str/}{str} , } {
\hyperref[definitions-conic-parameters-center]{center :}
\href{/docs/reference/foundations/array/}{array} , }

) -\textgreater{} \href{/docs/reference/visualize/gradient/}{gradient}

\begin{verbatim}
#stack(
  dir: ltr,
  spacing: 1fr,
  circle(fill: gradient.conic(
    ..color.map.viridis,
  )),
  circle(fill: gradient.conic(
    ..color.map.viridis,
    center: (20%, 30%),
  )),
)
\end{verbatim}

\includegraphics[width=5in,height=\textheight,keepaspectratio]{/assets/docs/Mqmcewscuekk2Rsln7oKygAAAAAAAAAA.png}

\paragraph{\texorpdfstring{\texttt{\ stops\ }}{ stops }}\label{definitions-conic-stops}

\href{/docs/reference/visualize/color/}{color} {or}
\href{/docs/reference/foundations/array/}{array}

{Required} {{ Positional }}

\phantomsection\label{definitions-conic-stops-positional-tooltip}
Positional parameters are specified in order, without names.

{{ Variadic }}

\phantomsection\label{definitions-conic-stops-variadic-tooltip}
Variadic parameters can be specified multiple times.

The color \hyperref[stops]{stops} of the gradient.

\paragraph{\texorpdfstring{\texttt{\ angle\ }}{ angle }}\label{definitions-conic-angle}

\href{/docs/reference/layout/angle/}{angle}

The angle of the gradient.

Default: \texttt{\ }{\texttt{\ 0deg\ }}\texttt{\ }

\paragraph{\texorpdfstring{\texttt{\ space\ }}{ space }}\label{definitions-conic-space}

{ any }

The color space in which to interpolate the gradient.

Defaults to a perceptually uniform color space called
\href{/docs/reference/visualize/color/\#definitions-oklab}{Oklab} .

Default: \texttt{\ oklab\ }

\paragraph{\texorpdfstring{\texttt{\ relative\ }}{ relative }}\label{definitions-conic-relative}

\href{/docs/reference/foundations/auto/}{auto} {or}
\href{/docs/reference/foundations/str/}{str}

The \hyperref[relativeness]{relative placement} of the gradient.

For an element placed at the root/top level of the document, the parent
is the page itself. For other elements, the parent is the innermost
block, box, column, grid, or stack that contains the element.

\begin{longtable}[]{@{}ll@{}}
\toprule\noalign{}
Variant & Details \\
\midrule\noalign{}
\endhead
\bottomrule\noalign{}
\endlastfoot
\texttt{\ "\ self\ "\ } & The gradient is relative to itself (its own
bounding box). \\
\texttt{\ "\ parent\ "\ } & The gradient is relative to its parent (the
parent\textquotesingle s bounding box). \\
\end{longtable}

Default: \texttt{\ }{\texttt{\ auto\ }}\texttt{\ }

\paragraph{\texorpdfstring{\texttt{\ center\ }}{ center }}\label{definitions-conic-center}

\href{/docs/reference/foundations/array/}{array}

The center of the last circle of the gradient.

A value of
\texttt{\ }{\texttt{\ (\ }}\texttt{\ }{\texttt{\ 50\%\ }}\texttt{\ }{\texttt{\ ,\ }}\texttt{\ }{\texttt{\ 50\%\ }}\texttt{\ }{\texttt{\ )\ }}\texttt{\ }
means that the end circle is centered inside of its container.

Default:
\texttt{\ }{\texttt{\ (\ }}\texttt{\ }{\texttt{\ 50\%\ }}\texttt{\ }{\texttt{\ ,\ }}\texttt{\ }{\texttt{\ 50\%\ }}\texttt{\ }{\texttt{\ )\ }}\texttt{\ }

\subsubsection{\texorpdfstring{\texttt{\ sharp\ }}{ sharp }}\label{definitions-sharp}

Creates a sharp version of this gradient.

Sharp gradients have discrete jumps between colors, instead of a smooth
transition. They are particularly useful for creating color lists for a
preset gradient.

self { . } { sharp } (

{ \href{/docs/reference/foundations/int/}{int} , } {
\hyperref[definitions-sharp-parameters-smoothness]{smoothness :}
\href{/docs/reference/layout/ratio/}{ratio} , }

) -\textgreater{} \href{/docs/reference/visualize/gradient/}{gradient}

\begin{verbatim}
#set rect(width: 100%, height: 20pt)
#let grad = gradient.linear(..color.map.rainbow)
#rect(fill: grad)
#rect(fill: grad.sharp(5))
#rect(fill: grad.sharp(5, smoothness: 20%))
\end{verbatim}

\includegraphics[width=5in,height=\textheight,keepaspectratio]{/assets/docs/k1IrJvVHW9DTjXfwHdfh_QAAAAAAAAAA.png}

\paragraph{\texorpdfstring{\texttt{\ steps\ }}{ steps }}\label{definitions-sharp-steps}

\href{/docs/reference/foundations/int/}{int}

{Required} {{ Positional }}

\phantomsection\label{definitions-sharp-steps-positional-tooltip}
Positional parameters are specified in order, without names.

The number of stops in the gradient.

\paragraph{\texorpdfstring{\texttt{\ smoothness\ }}{ smoothness }}\label{definitions-sharp-smoothness}

\href{/docs/reference/layout/ratio/}{ratio}

How much to smooth the gradient.

Default: \texttt{\ }{\texttt{\ 0\%\ }}\texttt{\ }

\subsubsection{\texorpdfstring{\texttt{\ repeat\ }}{ repeat }}\label{definitions-repeat}

Repeats this gradient a given number of times, optionally mirroring it
at each repetition.

self { . } { repeat } (

{ \href{/docs/reference/foundations/int/}{int} , } {
\hyperref[definitions-repeat-parameters-mirror]{mirror :}
\href{/docs/reference/foundations/bool/}{bool} , }

) -\textgreater{} \href{/docs/reference/visualize/gradient/}{gradient}

\begin{verbatim}
#circle(
  radius: 40pt,
  fill: gradient
    .radial(aqua, white)
    .repeat(4),
)
\end{verbatim}

\includegraphics[width=5in,height=\textheight,keepaspectratio]{/assets/docs/ydbGAMwgwvGMCJpfMs1wAAAAAAAAAAAA.png}

\paragraph{\texorpdfstring{\texttt{\ repetitions\ }}{ repetitions }}\label{definitions-repeat-repetitions}

\href{/docs/reference/foundations/int/}{int}

{Required} {{ Positional }}

\phantomsection\label{definitions-repeat-repetitions-positional-tooltip}
Positional parameters are specified in order, without names.

The number of times to repeat the gradient.

\paragraph{\texorpdfstring{\texttt{\ mirror\ }}{ mirror }}\label{definitions-repeat-mirror}

\href{/docs/reference/foundations/bool/}{bool}

Whether to mirror the gradient at each repetition.

Default: \texttt{\ }{\texttt{\ false\ }}\texttt{\ }

\subsubsection{\texorpdfstring{\texttt{\ kind\ }}{ kind }}\label{definitions-kind}

Returns the kind of this gradient.

self { . } { kind } (

) -\textgreater{} \href{/docs/reference/foundations/function/}{function}

\subsubsection{\texorpdfstring{\texttt{\ stops\ }}{ stops }}\label{definitions-stops}

Returns the stops of this gradient.

self { . } { stops } (

) -\textgreater{} \href{/docs/reference/foundations/array/}{array}

\subsubsection{\texorpdfstring{\texttt{\ space\ }}{ space }}\label{definitions-space}

Returns the mixing space of this gradient.

self { . } { space } (

) -\textgreater{} { any }

\subsubsection{\texorpdfstring{\texttt{\ relative\ }}{ relative }}\label{definitions-relative}

Returns the relative placement of this gradient.

self { . } { relative } (

) -\textgreater{} \href{/docs/reference/foundations/auto/}{auto}

\subsubsection{\texorpdfstring{\texttt{\ angle\ }}{ angle }}\label{definitions-angle}

Returns the angle of this gradient.

self { . } { angle } (

) -\textgreater{} \href{/docs/reference/foundations/none/}{none}
\href{/docs/reference/layout/angle/}{angle}

\subsubsection{\texorpdfstring{\texttt{\ sample\ }}{ sample }}\label{definitions-sample}

Sample the gradient at a given position.

The position is either a position along the gradient (a
\href{/docs/reference/layout/ratio/}{ratio} between
\texttt{\ }{\texttt{\ 0\%\ }}\texttt{\ } and
\texttt{\ }{\texttt{\ 100\%\ }}\texttt{\ } ) or an
\href{/docs/reference/layout/angle/}{angle} . Any value outside of this
range will be clamped.

self { . } { sample } (

{ \href{/docs/reference/layout/angle/}{angle}
\href{/docs/reference/layout/ratio/}{ratio} }

) -\textgreater{} \href{/docs/reference/visualize/color/}{color}

\paragraph{\texorpdfstring{\texttt{\ t\ }}{ t }}\label{definitions-sample-t}

\href{/docs/reference/layout/angle/}{angle} {or}
\href{/docs/reference/layout/ratio/}{ratio}

{Required} {{ Positional }}

\phantomsection\label{definitions-sample-t-positional-tooltip}
Positional parameters are specified in order, without names.

The position at which to sample the gradient.

\subsubsection{\texorpdfstring{\texttt{\ samples\ }}{ samples }}\label{definitions-samples}

Samples the gradient at multiple positions at once and returns the
results as an array.

self { . } { samples } (

{ \hyperref[definitions-samples-parameters-ts]{..}
\href{/docs/reference/layout/angle/}{angle}
\href{/docs/reference/layout/ratio/}{ratio} }

) -\textgreater{} \href{/docs/reference/foundations/array/}{array}

\paragraph{\texorpdfstring{\texttt{\ ts\ }}{ ts }}\label{definitions-samples-ts}

\href{/docs/reference/layout/angle/}{angle} {or}
\href{/docs/reference/layout/ratio/}{ratio}

{Required} {{ Positional }}

\phantomsection\label{definitions-samples-ts-positional-tooltip}
Positional parameters are specified in order, without names.

{{ Variadic }}

\phantomsection\label{definitions-samples-ts-variadic-tooltip}
Variadic parameters can be specified multiple times.

The positions at which to sample the gradient.

\href{/docs/reference/visualize/ellipse/}{\pandocbounded{\includesvg[keepaspectratio]{/assets/icons/16-arrow-right.svg}}}

{ Ellipse } { Previous page }

\href{/docs/reference/visualize/image/}{\pandocbounded{\includesvg[keepaspectratio]{/assets/icons/16-arrow-right.svg}}}

{ Image } { Next page }


\title{typst.app/docs/reference/visualize/square}

\begin{itemize}
\tightlist
\item
  \href{/docs}{\includesvg[width=0.16667in,height=0.16667in]{/assets/icons/16-docs-dark.svg}}
\item
  \includesvg[width=0.16667in,height=0.16667in]{/assets/icons/16-arrow-right.svg}
\item
  \href{/docs/reference/}{Reference}
\item
  \includesvg[width=0.16667in,height=0.16667in]{/assets/icons/16-arrow-right.svg}
\item
  \href{/docs/reference/visualize/}{Visualize}
\item
  \includesvg[width=0.16667in,height=0.16667in]{/assets/icons/16-arrow-right.svg}
\item
  \href{/docs/reference/visualize/square/}{Square}
\end{itemize}

\section{\texorpdfstring{\texttt{\ square\ } {{ Element
}}}{ square   Element }}\label{summary}

\phantomsection\label{element-tooltip}
Element functions can be customized with \texttt{\ set\ } and
\texttt{\ show\ } rules.

A square with optional content.

\subsection{Example}\label{example}

\begin{verbatim}
// Without content.
#square(size: 40pt)

// With content.
#square[
  Automatically \
  sized to fit.
]
\end{verbatim}

\includegraphics[width=5in,height=\textheight,keepaspectratio]{/assets/docs/DjWoCmaGrn_miIIjOqjv7gAAAAAAAAAA.png}

\subsection{\texorpdfstring{{ Parameters
}}{ Parameters }}\label{parameters}

\phantomsection\label{parameters-tooltip}
Parameters are the inputs to a function. They are specified in
parentheses after the function name.

{ square } (

{ \hyperref[parameters-size]{size :}
\href{/docs/reference/foundations/auto/}{auto}
\href{/docs/reference/layout/length/}{length} , } {
\hyperref[parameters-width]{width :}
\href{/docs/reference/foundations/auto/}{auto}
\href{/docs/reference/layout/relative/}{relative} , } {
\hyperref[parameters-height]{height :}
\href{/docs/reference/foundations/auto/}{auto}
\href{/docs/reference/layout/relative/}{relative}
\href{/docs/reference/layout/fraction/}{fraction} , } {
\hyperref[parameters-fill]{fill :}
\href{/docs/reference/foundations/none/}{none}
\href{/docs/reference/visualize/color/}{color}
\href{/docs/reference/visualize/gradient/}{gradient}
\href{/docs/reference/visualize/pattern/}{pattern} , } {
\hyperref[parameters-stroke]{stroke :}
\href{/docs/reference/foundations/none/}{none}
\href{/docs/reference/foundations/auto/}{auto}
\href{/docs/reference/layout/length/}{length}
\href{/docs/reference/visualize/color/}{color}
\href{/docs/reference/visualize/gradient/}{gradient}
\href{/docs/reference/visualize/stroke/}{stroke}
\href{/docs/reference/visualize/pattern/}{pattern}
\href{/docs/reference/foundations/dictionary/}{dictionary} , } {
\hyperref[parameters-radius]{radius :}
\href{/docs/reference/layout/relative/}{relative}
\href{/docs/reference/foundations/dictionary/}{dictionary} , } {
\hyperref[parameters-inset]{inset :}
\href{/docs/reference/layout/relative/}{relative}
\href{/docs/reference/foundations/dictionary/}{dictionary} , } {
\hyperref[parameters-outset]{outset :}
\href{/docs/reference/layout/relative/}{relative}
\href{/docs/reference/foundations/dictionary/}{dictionary} , } {
\hyperref[parameters-body]{}
\href{/docs/reference/foundations/none/}{none}
\href{/docs/reference/foundations/content/}{content} , }

) -\textgreater{} \href{/docs/reference/foundations/content/}{content}

\subsubsection{\texorpdfstring{\texttt{\ size\ }}{ size }}\label{parameters-size}

\href{/docs/reference/foundations/auto/}{auto} {or}
\href{/docs/reference/layout/length/}{length}

{{ Settable }}

\phantomsection\label{parameters-size-settable-tooltip}
Settable parameters can be customized for all following uses of the
function with a \texttt{\ set\ } rule.

The square\textquotesingle s side length. This is mutually exclusive
with \texttt{\ width\ } and \texttt{\ height\ } .

Default: \texttt{\ }{\texttt{\ auto\ }}\texttt{\ }

\subsubsection{\texorpdfstring{\texttt{\ width\ }}{ width }}\label{parameters-width}

\href{/docs/reference/foundations/auto/}{auto} {or}
\href{/docs/reference/layout/relative/}{relative}

{{ Settable }}

\phantomsection\label{parameters-width-settable-tooltip}
Settable parameters can be customized for all following uses of the
function with a \texttt{\ set\ } rule.

The square\textquotesingle s width. This is mutually exclusive with
\texttt{\ size\ } and \texttt{\ height\ } .

In contrast to \texttt{\ size\ } , this can be relative to the parent
container\textquotesingle s width.

Default: \texttt{\ }{\texttt{\ auto\ }}\texttt{\ }

\subsubsection{\texorpdfstring{\texttt{\ height\ }}{ height }}\label{parameters-height}

\href{/docs/reference/foundations/auto/}{auto} {or}
\href{/docs/reference/layout/relative/}{relative} {or}
\href{/docs/reference/layout/fraction/}{fraction}

{{ Settable }}

\phantomsection\label{parameters-height-settable-tooltip}
Settable parameters can be customized for all following uses of the
function with a \texttt{\ set\ } rule.

The square\textquotesingle s height. This is mutually exclusive with
\texttt{\ size\ } and \texttt{\ width\ } .

In contrast to \texttt{\ size\ } , this can be relative to the parent
container\textquotesingle s height.

Default: \texttt{\ }{\texttt{\ auto\ }}\texttt{\ }

\subsubsection{\texorpdfstring{\texttt{\ fill\ }}{ fill }}\label{parameters-fill}

\href{/docs/reference/foundations/none/}{none} {or}
\href{/docs/reference/visualize/color/}{color} {or}
\href{/docs/reference/visualize/gradient/}{gradient} {or}
\href{/docs/reference/visualize/pattern/}{pattern}

{{ Settable }}

\phantomsection\label{parameters-fill-settable-tooltip}
Settable parameters can be customized for all following uses of the
function with a \texttt{\ set\ } rule.

How to fill the square. See the
\href{/docs/reference/visualize/rect/\#parameters-fill}{rectangle\textquotesingle s
documentation} for more details.

Default: \texttt{\ }{\texttt{\ none\ }}\texttt{\ }

\subsubsection{\texorpdfstring{\texttt{\ stroke\ }}{ stroke }}\label{parameters-stroke}

\href{/docs/reference/foundations/none/}{none} {or}
\href{/docs/reference/foundations/auto/}{auto} {or}
\href{/docs/reference/layout/length/}{length} {or}
\href{/docs/reference/visualize/color/}{color} {or}
\href{/docs/reference/visualize/gradient/}{gradient} {or}
\href{/docs/reference/visualize/stroke/}{stroke} {or}
\href{/docs/reference/visualize/pattern/}{pattern} {or}
\href{/docs/reference/foundations/dictionary/}{dictionary}

{{ Settable }}

\phantomsection\label{parameters-stroke-settable-tooltip}
Settable parameters can be customized for all following uses of the
function with a \texttt{\ set\ } rule.

How to stroke the square. See the
\href{/docs/reference/visualize/rect/\#parameters-stroke}{rectangle\textquotesingle s
documentation} for more details.

Default: \texttt{\ }{\texttt{\ auto\ }}\texttt{\ }

\subsubsection{\texorpdfstring{\texttt{\ radius\ }}{ radius }}\label{parameters-radius}

\href{/docs/reference/layout/relative/}{relative} {or}
\href{/docs/reference/foundations/dictionary/}{dictionary}

{{ Settable }}

\phantomsection\label{parameters-radius-settable-tooltip}
Settable parameters can be customized for all following uses of the
function with a \texttt{\ set\ } rule.

How much to round the square\textquotesingle s corners. See the
\href{/docs/reference/visualize/rect/\#parameters-radius}{rectangle\textquotesingle s
documentation} for more details.

Default:
\texttt{\ }{\texttt{\ (\ }}\texttt{\ }{\texttt{\ :\ }}\texttt{\ }{\texttt{\ )\ }}\texttt{\ }

\subsubsection{\texorpdfstring{\texttt{\ inset\ }}{ inset }}\label{parameters-inset}

\href{/docs/reference/layout/relative/}{relative} {or}
\href{/docs/reference/foundations/dictionary/}{dictionary}

{{ Settable }}

\phantomsection\label{parameters-inset-settable-tooltip}
Settable parameters can be customized for all following uses of the
function with a \texttt{\ set\ } rule.

How much to pad the square\textquotesingle s content. See the
\href{/docs/reference/layout/box/\#parameters-inset}{box\textquotesingle s
documentation} for more details.

Default:
\texttt{\ }{\texttt{\ 0\%\ }}\texttt{\ }{\texttt{\ +\ }}\texttt{\ }{\texttt{\ 5pt\ }}\texttt{\ }

\subsubsection{\texorpdfstring{\texttt{\ outset\ }}{ outset }}\label{parameters-outset}

\href{/docs/reference/layout/relative/}{relative} {or}
\href{/docs/reference/foundations/dictionary/}{dictionary}

{{ Settable }}

\phantomsection\label{parameters-outset-settable-tooltip}
Settable parameters can be customized for all following uses of the
function with a \texttt{\ set\ } rule.

How much to expand the square\textquotesingle s size without affecting
the layout. See the
\href{/docs/reference/layout/box/\#parameters-outset}{box\textquotesingle s
documentation} for more details.

Default:
\texttt{\ }{\texttt{\ (\ }}\texttt{\ }{\texttt{\ :\ }}\texttt{\ }{\texttt{\ )\ }}\texttt{\ }

\subsubsection{\texorpdfstring{\texttt{\ body\ }}{ body }}\label{parameters-body}

\href{/docs/reference/foundations/none/}{none} {or}
\href{/docs/reference/foundations/content/}{content}

{{ Positional }}

\phantomsection\label{parameters-body-positional-tooltip}
Positional parameters are specified in order, without names.

{{ Settable }}

\phantomsection\label{parameters-body-settable-tooltip}
Settable parameters can be customized for all following uses of the
function with a \texttt{\ set\ } rule.

The content to place into the square. The square expands to fit this
content, keeping the 1-1 aspect ratio.

When this is omitted, the square takes on a default size of at most
\texttt{\ }{\texttt{\ 30pt\ }}\texttt{\ } .

Default: \texttt{\ }{\texttt{\ none\ }}\texttt{\ }

\href{/docs/reference/visualize/rect/}{\pandocbounded{\includesvg[keepaspectratio]{/assets/icons/16-arrow-right.svg}}}

{ Rectangle } { Previous page }

\href{/docs/reference/visualize/stroke/}{\pandocbounded{\includesvg[keepaspectratio]{/assets/icons/16-arrow-right.svg}}}

{ Stroke } { Next page }


\title{typst.app/docs/reference/visualize/image}

\begin{itemize}
\tightlist
\item
  \href{/docs}{\includesvg[width=0.16667in,height=0.16667in]{/assets/icons/16-docs-dark.svg}}
\item
  \includesvg[width=0.16667in,height=0.16667in]{/assets/icons/16-arrow-right.svg}
\item
  \href{/docs/reference/}{Reference}
\item
  \includesvg[width=0.16667in,height=0.16667in]{/assets/icons/16-arrow-right.svg}
\item
  \href{/docs/reference/visualize/}{Visualize}
\item
  \includesvg[width=0.16667in,height=0.16667in]{/assets/icons/16-arrow-right.svg}
\item
  \href{/docs/reference/visualize/image/}{Image}
\end{itemize}

\section{\texorpdfstring{\texttt{\ image\ } {{ Element
}}}{ image   Element }}\label{summary}

\phantomsection\label{element-tooltip}
Element functions can be customized with \texttt{\ set\ } and
\texttt{\ show\ } rules.

A raster or vector graphic.

You can wrap the image in a
\href{/docs/reference/model/figure/}{\texttt{\ figure\ }} to give it a
number and caption.

Like most elements, images are \emph{block-level} by default and thus do
not integrate themselves into adjacent paragraphs. To force an image to
become inline, put it into a
\href{/docs/reference/layout/box/}{\texttt{\ box\ }} .

\subsection{Example}\label{example}

\begin{verbatim}
#figure(
  image("molecular.jpg", width: 80%),
  caption: [
    A step in the molecular testing
    pipeline of our lab.
  ],
)
\end{verbatim}

\includegraphics[width=5in,height=\textheight,keepaspectratio]{/assets/docs/znWnPh4HT5GrpkEcbnfOxAAAAAAAAAAA.png}

\subsection{\texorpdfstring{{ Parameters
}}{ Parameters }}\label{parameters}

\phantomsection\label{parameters-tooltip}
Parameters are the inputs to a function. They are specified in
parentheses after the function name.

{ image } (

{ \href{/docs/reference/foundations/str/}{str} , } {
\hyperref[parameters-format]{format :}
\href{/docs/reference/foundations/auto/}{auto}
\href{/docs/reference/foundations/str/}{str} , } {
\hyperref[parameters-width]{width :}
\href{/docs/reference/foundations/auto/}{auto}
\href{/docs/reference/layout/relative/}{relative} , } {
\hyperref[parameters-height]{height :}
\href{/docs/reference/foundations/auto/}{auto}
\href{/docs/reference/layout/relative/}{relative}
\href{/docs/reference/layout/fraction/}{fraction} , } {
\hyperref[parameters-alt]{alt :}
\href{/docs/reference/foundations/none/}{none}
\href{/docs/reference/foundations/str/}{str} , } {
\hyperref[parameters-fit]{fit :}
\href{/docs/reference/foundations/str/}{str} , }

) -\textgreater{} \href{/docs/reference/foundations/content/}{content}

\subsubsection{\texorpdfstring{\texttt{\ path\ }}{ path }}\label{parameters-path}

\href{/docs/reference/foundations/str/}{str}

{Required} {{ Positional }}

\phantomsection\label{parameters-path-positional-tooltip}
Positional parameters are specified in order, without names.

Path to an image file

For more details, see the \href{/docs/reference/syntax/\#paths}{Paths
section} .

\subsubsection{\texorpdfstring{\texttt{\ format\ }}{ format }}\label{parameters-format}

\href{/docs/reference/foundations/auto/}{auto} {or}
\href{/docs/reference/foundations/str/}{str}

{{ Settable }}

\phantomsection\label{parameters-format-settable-tooltip}
Settable parameters can be customized for all following uses of the
function with a \texttt{\ set\ } rule.

The image\textquotesingle s format. Detected automatically by default.

Supported formats are PNG, JPEG, GIF, and SVG. Using a PDF as an image
is \href{https://github.com/typst/typst/issues/145}{not currently
supported} .

\begin{longtable}[]{@{}ll@{}}
\toprule\noalign{}
Variant & Details \\
\midrule\noalign{}
\endhead
\bottomrule\noalign{}
\endlastfoot
\texttt{\ "\ png\ "\ } & Raster format for illustrations and transparent
graphics. \\
\texttt{\ "\ jpg\ "\ } & Lossy raster format suitable for photos. \\
\texttt{\ "\ gif\ "\ } & Raster format that is typically used for short
animated clips. \\
\texttt{\ "\ svg\ "\ } & The vector graphics format of the web. \\
\end{longtable}

Default: \texttt{\ }{\texttt{\ auto\ }}\texttt{\ }

\subsubsection{\texorpdfstring{\texttt{\ width\ }}{ width }}\label{parameters-width}

\href{/docs/reference/foundations/auto/}{auto} {or}
\href{/docs/reference/layout/relative/}{relative}

{{ Settable }}

\phantomsection\label{parameters-width-settable-tooltip}
Settable parameters can be customized for all following uses of the
function with a \texttt{\ set\ } rule.

The width of the image.

Default: \texttt{\ }{\texttt{\ auto\ }}\texttt{\ }

\subsubsection{\texorpdfstring{\texttt{\ height\ }}{ height }}\label{parameters-height}

\href{/docs/reference/foundations/auto/}{auto} {or}
\href{/docs/reference/layout/relative/}{relative} {or}
\href{/docs/reference/layout/fraction/}{fraction}

{{ Settable }}

\phantomsection\label{parameters-height-settable-tooltip}
Settable parameters can be customized for all following uses of the
function with a \texttt{\ set\ } rule.

The height of the image.

Default: \texttt{\ }{\texttt{\ auto\ }}\texttt{\ }

\subsubsection{\texorpdfstring{\texttt{\ alt\ }}{ alt }}\label{parameters-alt}

\href{/docs/reference/foundations/none/}{none} {or}
\href{/docs/reference/foundations/str/}{str}

{{ Settable }}

\phantomsection\label{parameters-alt-settable-tooltip}
Settable parameters can be customized for all following uses of the
function with a \texttt{\ set\ } rule.

A text describing the image.

Default: \texttt{\ }{\texttt{\ none\ }}\texttt{\ }

\subsubsection{\texorpdfstring{\texttt{\ fit\ }}{ fit }}\label{parameters-fit}

\href{/docs/reference/foundations/str/}{str}

{{ Settable }}

\phantomsection\label{parameters-fit-settable-tooltip}
Settable parameters can be customized for all following uses of the
function with a \texttt{\ set\ } rule.

How the image should adjust itself to a given area (the area is defined
by the \texttt{\ width\ } and \texttt{\ height\ } fields). Note that
\texttt{\ fit\ } doesn\textquotesingle t visually change anything if the
area\textquotesingle s aspect ratio is the same as the
image\textquotesingle s one.

\begin{longtable}[]{@{}ll@{}}
\toprule\noalign{}
Variant & Details \\
\midrule\noalign{}
\endhead
\bottomrule\noalign{}
\endlastfoot
\texttt{\ "\ cover\ "\ } & The image should completely cover the area
(preserves aspect ratio by cropping the image only horizontally or
vertically). This is the default. \\
\texttt{\ "\ contain\ "\ } & The image should be fully contained in the
area (preserves aspect ratio; doesn\textquotesingle t crop the image;
one dimension can be narrower than specified). \\
\texttt{\ "\ stretch\ "\ } & The image should be stretched so that it
exactly fills the area, even if this means that the image will be
distorted (doesn\textquotesingle t preserve aspect ratio and
doesn\textquotesingle t crop the image). \\
\end{longtable}

Default: \texttt{\ }{\texttt{\ "cover"\ }}\texttt{\ }

\includesvg[width=0.16667in,height=0.16667in]{/assets/icons/16-arrow-right.svg}
View example

\begin{verbatim}
#set page(width: 300pt, height: 50pt, margin: 10pt)
#image("tiger.jpg", width: 100%, fit: "cover")
#image("tiger.jpg", width: 100%, fit: "contain")
#image("tiger.jpg", width: 100%, fit: "stretch")
\end{verbatim}

\includegraphics[width=6.25in,height=\textheight,keepaspectratio]{/assets/docs/oZRwamqZZ0p_tV8oioYxxgAAAAAAAAAA.png}
\includegraphics[width=6.25in,height=\textheight,keepaspectratio]{/assets/docs/oZRwamqZZ0p_tV8oioYxxgAAAAAAAAAB.png}
\includegraphics[width=6.25in,height=\textheight,keepaspectratio]{/assets/docs/oZRwamqZZ0p_tV8oioYxxgAAAAAAAAAC.png}

\subsection{\texorpdfstring{{ Definitions
}}{ Definitions }}\label{definitions}

\phantomsection\label{definitions-tooltip}
Functions and types and can have associated definitions. These are
accessed by specifying the function or type, followed by a period, and
then the definition\textquotesingle s name.

\subsubsection{\texorpdfstring{\texttt{\ decode\ }}{ decode }}\label{definitions-decode}

Decode a raster or vector graphic from bytes or a string.

image { . } { decode } (

{ \href{/docs/reference/foundations/str/}{str}
\href{/docs/reference/foundations/bytes/}{bytes} , } {
\hyperref[definitions-decode-parameters-format]{format :}
\href{/docs/reference/foundations/auto/}{auto}
\href{/docs/reference/foundations/str/}{str} , } {
\hyperref[definitions-decode-parameters-width]{width :}
\href{/docs/reference/foundations/auto/}{auto}
\href{/docs/reference/layout/relative/}{relative} , } {
\hyperref[definitions-decode-parameters-height]{height :}
\href{/docs/reference/foundations/auto/}{auto}
\href{/docs/reference/layout/relative/}{relative}
\href{/docs/reference/layout/fraction/}{fraction} , } {
\hyperref[definitions-decode-parameters-alt]{alt :}
\href{/docs/reference/foundations/none/}{none}
\href{/docs/reference/foundations/str/}{str} , } {
\hyperref[definitions-decode-parameters-fit]{fit :}
\href{/docs/reference/foundations/str/}{str} , }

) -\textgreater{} \href{/docs/reference/foundations/content/}{content}

\begin{verbatim}
#let original = read("diagram.svg")
#let changed = original.replace(
  "#2B80FF", // blue
  green.to-hex(),
)

#image.decode(original)
#image.decode(changed)
\end{verbatim}

\includegraphics[width=5in,height=\textheight,keepaspectratio]{/assets/docs/yVFFVjYQ7xibSWu-658yNwAAAAAAAAAA.png}

\paragraph{\texorpdfstring{\texttt{\ data\ }}{ data }}\label{definitions-decode-data}

\href{/docs/reference/foundations/str/}{str} {or}
\href{/docs/reference/foundations/bytes/}{bytes}

{Required} {{ Positional }}

\phantomsection\label{definitions-decode-data-positional-tooltip}
Positional parameters are specified in order, without names.

The data to decode as an image. Can be a string for SVGs.

\paragraph{\texorpdfstring{\texttt{\ format\ }}{ format }}\label{definitions-decode-format}

\href{/docs/reference/foundations/auto/}{auto} {or}
\href{/docs/reference/foundations/str/}{str}

The image\textquotesingle s format. Detected automatically by default.

\begin{longtable}[]{@{}ll@{}}
\toprule\noalign{}
Variant & Details \\
\midrule\noalign{}
\endhead
\bottomrule\noalign{}
\endlastfoot
\texttt{\ "\ png\ "\ } & Raster format for illustrations and transparent
graphics. \\
\texttt{\ "\ jpg\ "\ } & Lossy raster format suitable for photos. \\
\texttt{\ "\ gif\ "\ } & Raster format that is typically used for short
animated clips. \\
\texttt{\ "\ svg\ "\ } & The vector graphics format of the web. \\
\end{longtable}

\paragraph{\texorpdfstring{\texttt{\ width\ }}{ width }}\label{definitions-decode-width}

\href{/docs/reference/foundations/auto/}{auto} {or}
\href{/docs/reference/layout/relative/}{relative}

The width of the image.

\paragraph{\texorpdfstring{\texttt{\ height\ }}{ height }}\label{definitions-decode-height}

\href{/docs/reference/foundations/auto/}{auto} {or}
\href{/docs/reference/layout/relative/}{relative} {or}
\href{/docs/reference/layout/fraction/}{fraction}

The height of the image.

\paragraph{\texorpdfstring{\texttt{\ alt\ }}{ alt }}\label{definitions-decode-alt}

\href{/docs/reference/foundations/none/}{none} {or}
\href{/docs/reference/foundations/str/}{str}

A text describing the image.

\paragraph{\texorpdfstring{\texttt{\ fit\ }}{ fit }}\label{definitions-decode-fit}

\href{/docs/reference/foundations/str/}{str}

How the image should adjust itself to a given area.

\begin{longtable}[]{@{}ll@{}}
\toprule\noalign{}
Variant & Details \\
\midrule\noalign{}
\endhead
\bottomrule\noalign{}
\endlastfoot
\texttt{\ "\ cover\ "\ } & The image should completely cover the area
(preserves aspect ratio by cropping the image only horizontally or
vertically). This is the default. \\
\texttt{\ "\ contain\ "\ } & The image should be fully contained in the
area (preserves aspect ratio; doesn\textquotesingle t crop the image;
one dimension can be narrower than specified). \\
\texttt{\ "\ stretch\ "\ } & The image should be stretched so that it
exactly fills the area, even if this means that the image will be
distorted (doesn\textquotesingle t preserve aspect ratio and
doesn\textquotesingle t crop the image). \\
\end{longtable}

\href{/docs/reference/visualize/gradient/}{\pandocbounded{\includesvg[keepaspectratio]{/assets/icons/16-arrow-right.svg}}}

{ Gradient } { Previous page }

\href{/docs/reference/visualize/line/}{\pandocbounded{\includesvg[keepaspectratio]{/assets/icons/16-arrow-right.svg}}}

{ Line } { Next page }


\title{typst.app/docs/reference/visualize/line}

\begin{itemize}
\tightlist
\item
  \href{/docs}{\includesvg[width=0.16667in,height=0.16667in]{/assets/icons/16-docs-dark.svg}}
\item
  \includesvg[width=0.16667in,height=0.16667in]{/assets/icons/16-arrow-right.svg}
\item
  \href{/docs/reference/}{Reference}
\item
  \includesvg[width=0.16667in,height=0.16667in]{/assets/icons/16-arrow-right.svg}
\item
  \href{/docs/reference/visualize/}{Visualize}
\item
  \includesvg[width=0.16667in,height=0.16667in]{/assets/icons/16-arrow-right.svg}
\item
  \href{/docs/reference/visualize/line/}{Line}
\end{itemize}

\section{\texorpdfstring{\texttt{\ line\ } {{ Element
}}}{ line   Element }}\label{summary}

\phantomsection\label{element-tooltip}
Element functions can be customized with \texttt{\ set\ } and
\texttt{\ show\ } rules.

A line from one point to another.

\subsection{Example}\label{example}

\begin{verbatim}
#set page(height: 100pt)

#line(length: 100%)
#line(end: (50%, 50%))
#line(
  length: 4cm,
  stroke: 2pt + maroon,
)
\end{verbatim}

\includegraphics[width=5in,height=\textheight,keepaspectratio]{/assets/docs/IBdLCKW0h9kNWs6W_8DKAwAAAAAAAAAA.png}

\subsection{\texorpdfstring{{ Parameters
}}{ Parameters }}\label{parameters}

\phantomsection\label{parameters-tooltip}
Parameters are the inputs to a function. They are specified in
parentheses after the function name.

{ line } (

{ \hyperref[parameters-start]{start :}
\href{/docs/reference/foundations/array/}{array} , } {
\hyperref[parameters-end]{end :}
\href{/docs/reference/foundations/none/}{none}
\href{/docs/reference/foundations/array/}{array} , } {
\hyperref[parameters-length]{length :}
\href{/docs/reference/layout/relative/}{relative} , } {
\hyperref[parameters-angle]{angle :}
\href{/docs/reference/layout/angle/}{angle} , } {
\hyperref[parameters-stroke]{stroke :}
\href{/docs/reference/layout/length/}{length}
\href{/docs/reference/visualize/color/}{color}
\href{/docs/reference/visualize/gradient/}{gradient}
\href{/docs/reference/visualize/stroke/}{stroke}
\href{/docs/reference/visualize/pattern/}{pattern}
\href{/docs/reference/foundations/dictionary/}{dictionary} , }

) -\textgreater{} \href{/docs/reference/foundations/content/}{content}

\subsubsection{\texorpdfstring{\texttt{\ start\ }}{ start }}\label{parameters-start}

\href{/docs/reference/foundations/array/}{array}

{{ Settable }}

\phantomsection\label{parameters-start-settable-tooltip}
Settable parameters can be customized for all following uses of the
function with a \texttt{\ set\ } rule.

The start point of the line.

Must be an array of exactly two relative lengths.

Default:
\texttt{\ }{\texttt{\ (\ }}\texttt{\ }{\texttt{\ 0\%\ }}\texttt{\ }{\texttt{\ +\ }}\texttt{\ }{\texttt{\ 0pt\ }}\texttt{\ }{\texttt{\ ,\ }}\texttt{\ }{\texttt{\ 0\%\ }}\texttt{\ }{\texttt{\ +\ }}\texttt{\ }{\texttt{\ 0pt\ }}\texttt{\ }{\texttt{\ )\ }}\texttt{\ }

\subsubsection{\texorpdfstring{\texttt{\ end\ }}{ end }}\label{parameters-end}

\href{/docs/reference/foundations/none/}{none} {or}
\href{/docs/reference/foundations/array/}{array}

{{ Settable }}

\phantomsection\label{parameters-end-settable-tooltip}
Settable parameters can be customized for all following uses of the
function with a \texttt{\ set\ } rule.

The offset from \texttt{\ start\ } where the line ends.

Default: \texttt{\ }{\texttt{\ none\ }}\texttt{\ }

\subsubsection{\texorpdfstring{\texttt{\ length\ }}{ length }}\label{parameters-length}

\href{/docs/reference/layout/relative/}{relative}

{{ Settable }}

\phantomsection\label{parameters-length-settable-tooltip}
Settable parameters can be customized for all following uses of the
function with a \texttt{\ set\ } rule.

The line\textquotesingle s length. This is only respected if
\texttt{\ end\ } is \texttt{\ }{\texttt{\ none\ }}\texttt{\ } .

Default:
\texttt{\ }{\texttt{\ 0\%\ }}\texttt{\ }{\texttt{\ +\ }}\texttt{\ }{\texttt{\ 30pt\ }}\texttt{\ }

\subsubsection{\texorpdfstring{\texttt{\ angle\ }}{ angle }}\label{parameters-angle}

\href{/docs/reference/layout/angle/}{angle}

{{ Settable }}

\phantomsection\label{parameters-angle-settable-tooltip}
Settable parameters can be customized for all following uses of the
function with a \texttt{\ set\ } rule.

The angle at which the line points away from the origin. This is only
respected if \texttt{\ end\ } is
\texttt{\ }{\texttt{\ none\ }}\texttt{\ } .

Default: \texttt{\ }{\texttt{\ 0deg\ }}\texttt{\ }

\subsubsection{\texorpdfstring{\texttt{\ stroke\ }}{ stroke }}\label{parameters-stroke}

\href{/docs/reference/layout/length/}{length} {or}
\href{/docs/reference/visualize/color/}{color} {or}
\href{/docs/reference/visualize/gradient/}{gradient} {or}
\href{/docs/reference/visualize/stroke/}{stroke} {or}
\href{/docs/reference/visualize/pattern/}{pattern} {or}
\href{/docs/reference/foundations/dictionary/}{dictionary}

{{ Settable }}

\phantomsection\label{parameters-stroke-settable-tooltip}
Settable parameters can be customized for all following uses of the
function with a \texttt{\ set\ } rule.

How to \href{/docs/reference/visualize/stroke/}{stroke} the line.

Default:
\texttt{\ }{\texttt{\ 1pt\ }}\texttt{\ }{\texttt{\ +\ }}\texttt{\ black\ }

\includesvg[width=0.16667in,height=0.16667in]{/assets/icons/16-arrow-right.svg}
View example

\begin{verbatim}
#set line(length: 100%)
#stack(
  spacing: 1em,
  line(stroke: 2pt + red),
  line(stroke: (paint: blue, thickness: 4pt, cap: "round")),
  line(stroke: (paint: blue, thickness: 1pt, dash: "dashed")),
  line(stroke: (paint: blue, thickness: 1pt, dash: ("dot", 2pt, 4pt, 2pt))),
)
\end{verbatim}

\includegraphics[width=5in,height=\textheight,keepaspectratio]{/assets/docs/Shwqpl9XrWkg6A1XzBok6AAAAAAAAAAA.png}

\href{/docs/reference/visualize/image/}{\pandocbounded{\includesvg[keepaspectratio]{/assets/icons/16-arrow-right.svg}}}

{ Image } { Previous page }

\href{/docs/reference/visualize/path/}{\pandocbounded{\includesvg[keepaspectratio]{/assets/icons/16-arrow-right.svg}}}

{ Path } { Next page }


\title{typst.app/docs/reference/visualize/color}

\begin{itemize}
\tightlist
\item
  \href{/docs}{\includesvg[width=0.16667in,height=0.16667in]{/assets/icons/16-docs-dark.svg}}
\item
  \includesvg[width=0.16667in,height=0.16667in]{/assets/icons/16-arrow-right.svg}
\item
  \href{/docs/reference/}{Reference}
\item
  \includesvg[width=0.16667in,height=0.16667in]{/assets/icons/16-arrow-right.svg}
\item
  \href{/docs/reference/visualize/}{Visualize}
\item
  \includesvg[width=0.16667in,height=0.16667in]{/assets/icons/16-arrow-right.svg}
\item
  \href{/docs/reference/visualize/color/}{Color}
\end{itemize}

\section{\texorpdfstring{{ color }}{ color }}\label{summary}

A color in a specific color space.

Typst supports:

\begin{itemize}
\tightlist
\item
  sRGB through the
  \href{/docs/reference/visualize/color/\#definitions-rgb}{\texttt{\ rgb\ }
  function}
\item
  Device CMYK through
  \href{/docs/reference/visualize/color/\#definitions-cmyk}{\texttt{\ cmyk\ }
  function}
\item
  D65 Gray through the
  \href{/docs/reference/visualize/color/\#definitions-luma}{\texttt{\ luma\ }
  function}
\item
  Oklab through the
  \href{/docs/reference/visualize/color/\#definitions-oklab}{\texttt{\ oklab\ }
  function}
\item
  Oklch through the
  \href{/docs/reference/visualize/color/\#definitions-oklch}{\texttt{\ oklch\ }
  function}
\item
  Linear RGB through the
  \href{/docs/reference/visualize/color/\#definitions-linear-rgb}{\texttt{\ color.linear-rgb\ }
  function}
\item
  HSL through the
  \href{/docs/reference/visualize/color/\#definitions-hsl}{\texttt{\ color.hsl\ }
  function}
\item
  HSV through the
  \href{/docs/reference/visualize/color/\#definitions-hsv}{\texttt{\ color.hsv\ }
  function}
\end{itemize}

\subsection{Example}\label{example}

\begin{verbatim}
#rect(fill: aqua)
\end{verbatim}

\includegraphics[width=5in,height=\textheight,keepaspectratio]{/assets/docs/k-6wh2l9TTXmPhzZxpahjQAAAAAAAAAA.png}

\subsection{Predefined colors}\label{predefined-colors}

Typst defines the following built-in colors:

\begin{longtable}[]{@{}ll@{}}
\toprule\noalign{}
Color & Definition \\
\midrule\noalign{}
\endhead
\bottomrule\noalign{}
\endlastfoot
\texttt{\ black\ } &
\texttt{\ }{\texttt{\ luma\ }}\texttt{\ }{\texttt{\ (\ }}\texttt{\ }{\texttt{\ 0\ }}\texttt{\ }{\texttt{\ )\ }}\texttt{\ } \\
\texttt{\ gray\ } &
\texttt{\ }{\texttt{\ luma\ }}\texttt{\ }{\texttt{\ (\ }}\texttt{\ }{\texttt{\ 170\ }}\texttt{\ }{\texttt{\ )\ }}\texttt{\ } \\
\texttt{\ silver\ } &
\texttt{\ }{\texttt{\ luma\ }}\texttt{\ }{\texttt{\ (\ }}\texttt{\ }{\texttt{\ 221\ }}\texttt{\ }{\texttt{\ )\ }}\texttt{\ } \\
\texttt{\ white\ } &
\texttt{\ }{\texttt{\ luma\ }}\texttt{\ }{\texttt{\ (\ }}\texttt{\ }{\texttt{\ 255\ }}\texttt{\ }{\texttt{\ )\ }}\texttt{\ } \\
\texttt{\ navy\ } &
\texttt{\ }{\texttt{\ rgb\ }}\texttt{\ }{\texttt{\ (\ }}\texttt{\ }{\texttt{\ "\#001f3f"\ }}\texttt{\ }{\texttt{\ )\ }}\texttt{\ } \\
\texttt{\ blue\ } &
\texttt{\ }{\texttt{\ rgb\ }}\texttt{\ }{\texttt{\ (\ }}\texttt{\ }{\texttt{\ "\#0074d9"\ }}\texttt{\ }{\texttt{\ )\ }}\texttt{\ } \\
\texttt{\ aqua\ } &
\texttt{\ }{\texttt{\ rgb\ }}\texttt{\ }{\texttt{\ (\ }}\texttt{\ }{\texttt{\ "\#7fdbff"\ }}\texttt{\ }{\texttt{\ )\ }}\texttt{\ } \\
\texttt{\ teal\ } &
\texttt{\ }{\texttt{\ rgb\ }}\texttt{\ }{\texttt{\ (\ }}\texttt{\ }{\texttt{\ "\#39cccc"\ }}\texttt{\ }{\texttt{\ )\ }}\texttt{\ } \\
\texttt{\ eastern\ } &
\texttt{\ }{\texttt{\ rgb\ }}\texttt{\ }{\texttt{\ (\ }}\texttt{\ }{\texttt{\ "\#239dad"\ }}\texttt{\ }{\texttt{\ )\ }}\texttt{\ } \\
\texttt{\ purple\ } &
\texttt{\ }{\texttt{\ rgb\ }}\texttt{\ }{\texttt{\ (\ }}\texttt{\ }{\texttt{\ "\#b10dc9"\ }}\texttt{\ }{\texttt{\ )\ }}\texttt{\ } \\
\texttt{\ fuchsia\ } &
\texttt{\ }{\texttt{\ rgb\ }}\texttt{\ }{\texttt{\ (\ }}\texttt{\ }{\texttt{\ "\#f012be"\ }}\texttt{\ }{\texttt{\ )\ }}\texttt{\ } \\
\texttt{\ maroon\ } &
\texttt{\ }{\texttt{\ rgb\ }}\texttt{\ }{\texttt{\ (\ }}\texttt{\ }{\texttt{\ "\#85144b"\ }}\texttt{\ }{\texttt{\ )\ }}\texttt{\ } \\
\texttt{\ red\ } &
\texttt{\ }{\texttt{\ rgb\ }}\texttt{\ }{\texttt{\ (\ }}\texttt{\ }{\texttt{\ "\#ff4136"\ }}\texttt{\ }{\texttt{\ )\ }}\texttt{\ } \\
\texttt{\ orange\ } &
\texttt{\ }{\texttt{\ rgb\ }}\texttt{\ }{\texttt{\ (\ }}\texttt{\ }{\texttt{\ "\#ff851b"\ }}\texttt{\ }{\texttt{\ )\ }}\texttt{\ } \\
\texttt{\ yellow\ } &
\texttt{\ }{\texttt{\ rgb\ }}\texttt{\ }{\texttt{\ (\ }}\texttt{\ }{\texttt{\ "\#ffdc00"\ }}\texttt{\ }{\texttt{\ )\ }}\texttt{\ } \\
\texttt{\ olive\ } &
\texttt{\ }{\texttt{\ rgb\ }}\texttt{\ }{\texttt{\ (\ }}\texttt{\ }{\texttt{\ "\#3d9970"\ }}\texttt{\ }{\texttt{\ )\ }}\texttt{\ } \\
\texttt{\ green\ } &
\texttt{\ }{\texttt{\ rgb\ }}\texttt{\ }{\texttt{\ (\ }}\texttt{\ }{\texttt{\ "\#2ecc40"\ }}\texttt{\ }{\texttt{\ )\ }}\texttt{\ } \\
\texttt{\ lime\ } &
\texttt{\ }{\texttt{\ rgb\ }}\texttt{\ }{\texttt{\ (\ }}\texttt{\ }{\texttt{\ "\#01ff70"\ }}\texttt{\ }{\texttt{\ )\ }}\texttt{\ } \\
\end{longtable}

The predefined colors and the most important color constructors are
available globally and also in the color type\textquotesingle s scope,
so you can write either \texttt{\ color.red\ } or just \texttt{\ red\ }
.

\includegraphics[width=11.66667in,height=\textheight,keepaspectratio]{/assets/docs/IWvUAQq21Ue1zu9gwjch-gAAAAAAAAAA.png}

\subsection{Predefined color maps}\label{predefined-color-maps}

Typst also includes a number of preset color maps that can be used for
\href{/docs/reference/visualize/gradient/\#definitions-linear}{gradients}
. These are simply arrays of colors defined in the module
\texttt{\ color.map\ } .

\begin{verbatim}
#circle(fill: gradient.linear(..color.map.crest))
\end{verbatim}

\includegraphics[width=5in,height=\textheight,keepaspectratio]{/assets/docs/uG6iVgmQwH_6_-1N42yKHwAAAAAAAAAA.png}

\begin{longtable}[]{@{}ll@{}}
\toprule\noalign{}
Map & Details \\
\midrule\noalign{}
\endhead
\bottomrule\noalign{}
\endlastfoot
\texttt{\ turbo\ } & A perceptually uniform rainbow-like color map. Read
\href{https://ai.googleblog.com/2019/08/turbo-improved-rainbow-colormap-for.html}{this
blog post} for more details. \\
\texttt{\ cividis\ } & A blue to gray to yellow color map. See
\href{https://bids.github.io/colormap/}{this blog post} for more
details. \\
\texttt{\ rainbow\ } & Cycles through the full color spectrum. This
color map is best used by setting the interpolation color space to
\href{/docs/reference/visualize/color/\#definitions-hsl}{HSL} . The
rainbow gradient is \textbf{not suitable} for data visualization because
it is not perceptually uniform, so the differences between values become
unclear to your readers. It should only be used for decorative
purposes. \\
\texttt{\ spectral\ } & Red to yellow to blue color map. \\
\texttt{\ viridis\ } & A purple to teal to yellow color map. \\
\texttt{\ inferno\ } & A black to red to yellow color map. \\
\texttt{\ magma\ } & A black to purple to yellow color map. \\
\texttt{\ plasma\ } & A purple to pink to yellow color map. \\
\texttt{\ rocket\ } & A black to red to white color map. \\
\texttt{\ mako\ } & A black to teal to yellow color map. \\
\texttt{\ vlag\ } & A light blue to white to red color map. \\
\texttt{\ icefire\ } & A light teal to black to yellow color map. \\
\texttt{\ flare\ } & A orange to purple color map that is perceptually
uniform. \\
\texttt{\ crest\ } & A blue to white to red color map. \\
\end{longtable}

Some popular presets are not included because they are not available
under a free licence. Others, like
\href{https://jakevdp.github.io/blog/2014/10/16/how-bad-is-your-colormap/}{Jet}
, are not included because they are not color blind friendly. Feel free
to use or create a package with other presets that are useful to you!

\includegraphics[width=5.8125in,height=\textheight,keepaspectratio]{/assets/docs/S2ExoTDRK30Xf9wXJbWIZgAAAAAAAAAA.png}

\subsection{\texorpdfstring{{ Definitions
}}{ Definitions }}\label{definitions}

\phantomsection\label{definitions-tooltip}
Functions and types and can have associated definitions. These are
accessed by specifying the function or type, followed by a period, and
then the definition\textquotesingle s name.

\subsubsection{\texorpdfstring{\texttt{\ luma\ }}{ luma }}\label{definitions-luma}

Create a grayscale color.

A grayscale color is represented internally by a single
\texttt{\ lightness\ } component.

These components are also available using the
\href{/docs/reference/visualize/color/\#definitions-components}{\texttt{\ components\ }}
method.

color { . } { luma } (

{ \href{/docs/reference/foundations/int/}{int}
\href{/docs/reference/layout/ratio/}{ratio} , } {
\href{/docs/reference/layout/ratio/}{ratio} , } {
\href{/docs/reference/visualize/color/}{color} , }

) -\textgreater{} \href{/docs/reference/visualize/color/}{color}

\begin{verbatim}
#for x in range(250, step: 50) {
  box(square(fill: luma(x)))
}
\end{verbatim}

\includegraphics[width=5in,height=\textheight,keepaspectratio]{/assets/docs/bCTOWkOtpDPjuD2iPgTajQAAAAAAAAAA.png}

\paragraph{\texorpdfstring{\texttt{\ lightness\ }}{ lightness }}\label{definitions-luma-lightness}

\href{/docs/reference/foundations/int/}{int} {or}
\href{/docs/reference/layout/ratio/}{ratio}

{Required} {{ Positional }}

\phantomsection\label{definitions-luma-lightness-positional-tooltip}
Positional parameters are specified in order, without names.

The lightness component.

\paragraph{\texorpdfstring{\texttt{\ alpha\ }}{ alpha }}\label{definitions-luma-alpha}

\href{/docs/reference/layout/ratio/}{ratio}

{Required} {{ Positional }}

\phantomsection\label{definitions-luma-alpha-positional-tooltip}
Positional parameters are specified in order, without names.

The alpha component.

\paragraph{\texorpdfstring{\texttt{\ color\ }}{ color }}\label{definitions-luma-color}

\href{/docs/reference/visualize/color/}{color}

{Required} {{ Positional }}

\phantomsection\label{definitions-luma-color-positional-tooltip}
Positional parameters are specified in order, without names.

Alternatively: The color to convert to grayscale.

If this is given, the \texttt{\ lightness\ } should not be given.

\subsubsection{\texorpdfstring{\texttt{\ oklab\ }}{ oklab }}\label{definitions-oklab}

Create an \href{https://bottosson.github.io/posts/oklab/}{Oklab} color.

This color space is well suited for the following use cases:

\begin{itemize}
\tightlist
\item
  Color manipulation such as saturating while keeping perceived hue
\item
  Creating grayscale images with uniform perceived lightness
\item
  Creating smooth and uniform color transition and gradients
\end{itemize}

A linear Oklab color is represented internally by an array of four
components:

\begin{itemize}
\tightlist
\item
  lightness ( \href{/docs/reference/layout/ratio/}{\texttt{\ ratio\ }} )
\item
  a ( \href{/docs/reference/foundations/float/}{\texttt{\ float\ }} or
  \href{/docs/reference/layout/ratio/}{\texttt{\ ratio\ }} . Ratios are
  relative to \texttt{\ }{\texttt{\ 0.4\ }}\texttt{\ } ; meaning
  \texttt{\ }{\texttt{\ 50\%\ }}\texttt{\ } is equal to
  \texttt{\ }{\texttt{\ 0.2\ }}\texttt{\ } )
\item
  b ( \href{/docs/reference/foundations/float/}{\texttt{\ float\ }} or
  \href{/docs/reference/layout/ratio/}{\texttt{\ ratio\ }} . Ratios are
  relative to \texttt{\ }{\texttt{\ 0.4\ }}\texttt{\ } ; meaning
  \texttt{\ }{\texttt{\ 50\%\ }}\texttt{\ } is equal to
  \texttt{\ }{\texttt{\ 0.2\ }}\texttt{\ } )
\item
  alpha ( \href{/docs/reference/layout/ratio/}{\texttt{\ ratio\ }} )
\end{itemize}

These components are also available using the
\href{/docs/reference/visualize/color/\#definitions-components}{\texttt{\ components\ }}
method.

color { . } { oklab } (

{ \href{/docs/reference/layout/ratio/}{ratio} , } {
\href{/docs/reference/foundations/float/}{float}
\href{/docs/reference/layout/ratio/}{ratio} , } {
\href{/docs/reference/foundations/float/}{float}
\href{/docs/reference/layout/ratio/}{ratio} , } {
\href{/docs/reference/layout/ratio/}{ratio} , } {
\href{/docs/reference/visualize/color/}{color} , }

) -\textgreater{} \href{/docs/reference/visualize/color/}{color}

\begin{verbatim}
#square(
  fill: oklab(27%, 20%, -3%, 50%)
)
\end{verbatim}

\includegraphics[width=5in,height=\textheight,keepaspectratio]{/assets/docs/1dGzDbwdYzYb5NzJEzQzFAAAAAAAAAAA.png}

\paragraph{\texorpdfstring{\texttt{\ lightness\ }}{ lightness }}\label{definitions-oklab-lightness}

\href{/docs/reference/layout/ratio/}{ratio}

{Required} {{ Positional }}

\phantomsection\label{definitions-oklab-lightness-positional-tooltip}
Positional parameters are specified in order, without names.

The lightness component.

\paragraph{\texorpdfstring{\texttt{\ a\ }}{ a }}\label{definitions-oklab-a}

\href{/docs/reference/foundations/float/}{float} {or}
\href{/docs/reference/layout/ratio/}{ratio}

{Required} {{ Positional }}

\phantomsection\label{definitions-oklab-a-positional-tooltip}
Positional parameters are specified in order, without names.

The a ("green/red") component.

\paragraph{\texorpdfstring{\texttt{\ b\ }}{ b }}\label{definitions-oklab-b}

\href{/docs/reference/foundations/float/}{float} {or}
\href{/docs/reference/layout/ratio/}{ratio}

{Required} {{ Positional }}

\phantomsection\label{definitions-oklab-b-positional-tooltip}
Positional parameters are specified in order, without names.

The b ("blue/yellow") component.

\paragraph{\texorpdfstring{\texttt{\ alpha\ }}{ alpha }}\label{definitions-oklab-alpha}

\href{/docs/reference/layout/ratio/}{ratio}

{Required} {{ Positional }}

\phantomsection\label{definitions-oklab-alpha-positional-tooltip}
Positional parameters are specified in order, without names.

The alpha component.

\paragraph{\texorpdfstring{\texttt{\ color\ }}{ color }}\label{definitions-oklab-color}

\href{/docs/reference/visualize/color/}{color}

{Required} {{ Positional }}

\phantomsection\label{definitions-oklab-color-positional-tooltip}
Positional parameters are specified in order, without names.

Alternatively: The color to convert to Oklab.

If this is given, the individual components should not be given.

\subsubsection{\texorpdfstring{\texttt{\ oklch\ }}{ oklch }}\label{definitions-oklch}

Create an \href{https://bottosson.github.io/posts/oklab/}{Oklch} color.

This color space is well suited for the following use cases:

\begin{itemize}
\tightlist
\item
  Color manipulation involving lightness, chroma, and hue
\item
  Creating grayscale images with uniform perceived lightness
\item
  Creating smooth and uniform color transition and gradients
\end{itemize}

A linear Oklch color is represented internally by an array of four
components:

\begin{itemize}
\tightlist
\item
  lightness ( \href{/docs/reference/layout/ratio/}{\texttt{\ ratio\ }} )
\item
  chroma ( \href{/docs/reference/foundations/float/}{\texttt{\ float\ }}
  or \href{/docs/reference/layout/ratio/}{\texttt{\ ratio\ }} . Ratios
  are relative to \texttt{\ }{\texttt{\ 0.4\ }}\texttt{\ } ; meaning
  \texttt{\ }{\texttt{\ 50\%\ }}\texttt{\ } is equal to
  \texttt{\ }{\texttt{\ 0.2\ }}\texttt{\ } )
\item
  hue ( \href{/docs/reference/layout/angle/}{\texttt{\ angle\ }} )
\item
  alpha ( \href{/docs/reference/layout/ratio/}{\texttt{\ ratio\ }} )
\end{itemize}

These components are also available using the
\href{/docs/reference/visualize/color/\#definitions-components}{\texttt{\ components\ }}
method.

color { . } { oklch } (

{ \href{/docs/reference/layout/ratio/}{ratio} , } {
\href{/docs/reference/foundations/float/}{float}
\href{/docs/reference/layout/ratio/}{ratio} , } {
\href{/docs/reference/layout/angle/}{angle} , } {
\href{/docs/reference/layout/ratio/}{ratio} , } {
\href{/docs/reference/visualize/color/}{color} , }

) -\textgreater{} \href{/docs/reference/visualize/color/}{color}

\begin{verbatim}
#square(
  fill: oklch(40%, 0.2, 160deg, 50%)
)
\end{verbatim}

\includegraphics[width=5in,height=\textheight,keepaspectratio]{/assets/docs/gEJt1PBpGTajcUm46S-JNgAAAAAAAAAA.png}

\paragraph{\texorpdfstring{\texttt{\ lightness\ }}{ lightness }}\label{definitions-oklch-lightness}

\href{/docs/reference/layout/ratio/}{ratio}

{Required} {{ Positional }}

\phantomsection\label{definitions-oklch-lightness-positional-tooltip}
Positional parameters are specified in order, without names.

The lightness component.

\paragraph{\texorpdfstring{\texttt{\ chroma\ }}{ chroma }}\label{definitions-oklch-chroma}

\href{/docs/reference/foundations/float/}{float} {or}
\href{/docs/reference/layout/ratio/}{ratio}

{Required} {{ Positional }}

\phantomsection\label{definitions-oklch-chroma-positional-tooltip}
Positional parameters are specified in order, without names.

The chroma component.

\paragraph{\texorpdfstring{\texttt{\ hue\ }}{ hue }}\label{definitions-oklch-hue}

\href{/docs/reference/layout/angle/}{angle}

{Required} {{ Positional }}

\phantomsection\label{definitions-oklch-hue-positional-tooltip}
Positional parameters are specified in order, without names.

The hue component.

\paragraph{\texorpdfstring{\texttt{\ alpha\ }}{ alpha }}\label{definitions-oklch-alpha}

\href{/docs/reference/layout/ratio/}{ratio}

{Required} {{ Positional }}

\phantomsection\label{definitions-oklch-alpha-positional-tooltip}
Positional parameters are specified in order, without names.

The alpha component.

\paragraph{\texorpdfstring{\texttt{\ color\ }}{ color }}\label{definitions-oklch-color}

\href{/docs/reference/visualize/color/}{color}

{Required} {{ Positional }}

\phantomsection\label{definitions-oklch-color-positional-tooltip}
Positional parameters are specified in order, without names.

Alternatively: The color to convert to Oklch.

If this is given, the individual components should not be given.

\subsubsection{\texorpdfstring{\texttt{\ linear-rgb\ }}{ linear-rgb }}\label{definitions-linear-rgb}

Create an RGB(A) color with linear luma.

This color space is similar to sRGB, but with the distinction that the
color component are not gamma corrected. This makes it easier to perform
color operations such as blending and interpolation. Although, you
should prefer to use the
\href{/docs/reference/visualize/color/\#definitions-oklab}{\texttt{\ oklab\ }
function} for these.

A linear RGB(A) color is represented internally by an array of four
components:

\begin{itemize}
\tightlist
\item
  red ( \href{/docs/reference/layout/ratio/}{\texttt{\ ratio\ }} )
\item
  green ( \href{/docs/reference/layout/ratio/}{\texttt{\ ratio\ }} )
\item
  blue ( \href{/docs/reference/layout/ratio/}{\texttt{\ ratio\ }} )
\item
  alpha ( \href{/docs/reference/layout/ratio/}{\texttt{\ ratio\ }} )
\end{itemize}

These components are also available using the
\href{/docs/reference/visualize/color/\#definitions-components}{\texttt{\ components\ }}
method.

color { . } { linear-rgb } (

{ \href{/docs/reference/foundations/int/}{int}
\href{/docs/reference/layout/ratio/}{ratio} , } {
\href{/docs/reference/foundations/int/}{int}
\href{/docs/reference/layout/ratio/}{ratio} , } {
\href{/docs/reference/foundations/int/}{int}
\href{/docs/reference/layout/ratio/}{ratio} , } {
\href{/docs/reference/foundations/int/}{int}
\href{/docs/reference/layout/ratio/}{ratio} , } {
\href{/docs/reference/visualize/color/}{color} , }

) -\textgreater{} \href{/docs/reference/visualize/color/}{color}

\begin{verbatim}
#square(fill: color.linear-rgb(
  30%, 50%, 10%,
))
\end{verbatim}

\includegraphics[width=5in,height=\textheight,keepaspectratio]{/assets/docs/C39dYHKq1AmgEkOU8XX2kQAAAAAAAAAA.png}

\paragraph{\texorpdfstring{\texttt{\ red\ }}{ red }}\label{definitions-linear-rgb-red}

\href{/docs/reference/foundations/int/}{int} {or}
\href{/docs/reference/layout/ratio/}{ratio}

{Required} {{ Positional }}

\phantomsection\label{definitions-linear-rgb-red-positional-tooltip}
Positional parameters are specified in order, without names.

The red component.

\paragraph{\texorpdfstring{\texttt{\ green\ }}{ green }}\label{definitions-linear-rgb-green}

\href{/docs/reference/foundations/int/}{int} {or}
\href{/docs/reference/layout/ratio/}{ratio}

{Required} {{ Positional }}

\phantomsection\label{definitions-linear-rgb-green-positional-tooltip}
Positional parameters are specified in order, without names.

The green component.

\paragraph{\texorpdfstring{\texttt{\ blue\ }}{ blue }}\label{definitions-linear-rgb-blue}

\href{/docs/reference/foundations/int/}{int} {or}
\href{/docs/reference/layout/ratio/}{ratio}

{Required} {{ Positional }}

\phantomsection\label{definitions-linear-rgb-blue-positional-tooltip}
Positional parameters are specified in order, without names.

The blue component.

\paragraph{\texorpdfstring{\texttt{\ alpha\ }}{ alpha }}\label{definitions-linear-rgb-alpha}

\href{/docs/reference/foundations/int/}{int} {or}
\href{/docs/reference/layout/ratio/}{ratio}

{Required} {{ Positional }}

\phantomsection\label{definitions-linear-rgb-alpha-positional-tooltip}
Positional parameters are specified in order, without names.

The alpha component.

\paragraph{\texorpdfstring{\texttt{\ color\ }}{ color }}\label{definitions-linear-rgb-color}

\href{/docs/reference/visualize/color/}{color}

{Required} {{ Positional }}

\phantomsection\label{definitions-linear-rgb-color-positional-tooltip}
Positional parameters are specified in order, without names.

Alternatively: The color to convert to linear RGB(A).

If this is given, the individual components should not be given.

\subsubsection{\texorpdfstring{\texttt{\ rgb\ }}{ rgb }}\label{definitions-rgb}

Create an RGB(A) color.

The color is specified in the sRGB color space.

An RGB(A) color is represented internally by an array of four
components:

\begin{itemize}
\tightlist
\item
  red ( \href{/docs/reference/layout/ratio/}{\texttt{\ ratio\ }} )
\item
  green ( \href{/docs/reference/layout/ratio/}{\texttt{\ ratio\ }} )
\item
  blue ( \href{/docs/reference/layout/ratio/}{\texttt{\ ratio\ }} )
\item
  alpha ( \href{/docs/reference/layout/ratio/}{\texttt{\ ratio\ }} )
\end{itemize}

These components are also available using the
\href{/docs/reference/visualize/color/\#definitions-components}{\texttt{\ components\ }}
method.

color { . } { rgb } (

{ \href{/docs/reference/foundations/int/}{int}
\href{/docs/reference/layout/ratio/}{ratio} , } {
\href{/docs/reference/foundations/int/}{int}
\href{/docs/reference/layout/ratio/}{ratio} , } {
\href{/docs/reference/foundations/int/}{int}
\href{/docs/reference/layout/ratio/}{ratio} , } {
\href{/docs/reference/foundations/int/}{int}
\href{/docs/reference/layout/ratio/}{ratio} , } {
\href{/docs/reference/foundations/str/}{str} , } {
\href{/docs/reference/visualize/color/}{color} , }

) -\textgreater{} \href{/docs/reference/visualize/color/}{color}

\begin{verbatim}
#square(fill: rgb("#b1f2eb"))
#square(fill: rgb(87, 127, 230))
#square(fill: rgb(25%, 13%, 65%))
\end{verbatim}

\includegraphics[width=5in,height=\textheight,keepaspectratio]{/assets/docs/eWivZbkq7oFotM06OeK92AAAAAAAAAAA.png}

\paragraph{\texorpdfstring{\texttt{\ red\ }}{ red }}\label{definitions-rgb-red}

\href{/docs/reference/foundations/int/}{int} {or}
\href{/docs/reference/layout/ratio/}{ratio}

{Required} {{ Positional }}

\phantomsection\label{definitions-rgb-red-positional-tooltip}
Positional parameters are specified in order, without names.

The red component.

\paragraph{\texorpdfstring{\texttt{\ green\ }}{ green }}\label{definitions-rgb-green}

\href{/docs/reference/foundations/int/}{int} {or}
\href{/docs/reference/layout/ratio/}{ratio}

{Required} {{ Positional }}

\phantomsection\label{definitions-rgb-green-positional-tooltip}
Positional parameters are specified in order, without names.

The green component.

\paragraph{\texorpdfstring{\texttt{\ blue\ }}{ blue }}\label{definitions-rgb-blue}

\href{/docs/reference/foundations/int/}{int} {or}
\href{/docs/reference/layout/ratio/}{ratio}

{Required} {{ Positional }}

\phantomsection\label{definitions-rgb-blue-positional-tooltip}
Positional parameters are specified in order, without names.

The blue component.

\paragraph{\texorpdfstring{\texttt{\ alpha\ }}{ alpha }}\label{definitions-rgb-alpha}

\href{/docs/reference/foundations/int/}{int} {or}
\href{/docs/reference/layout/ratio/}{ratio}

{Required} {{ Positional }}

\phantomsection\label{definitions-rgb-alpha-positional-tooltip}
Positional parameters are specified in order, without names.

The alpha component.

\paragraph{\texorpdfstring{\texttt{\ hex\ }}{ hex }}\label{definitions-rgb-hex}

\href{/docs/reference/foundations/str/}{str}

{Required} {{ Positional }}

\phantomsection\label{definitions-rgb-hex-positional-tooltip}
Positional parameters are specified in order, without names.

Alternatively: The color in hexadecimal notation.

Accepts three, four, six or eight hexadecimal digits and optionally a
leading hash.

If this is given, the individual components should not be given.

\includesvg[width=0.16667in,height=0.16667in]{/assets/icons/16-arrow-right.svg}
View example

\begin{verbatim}
#text(16pt, rgb("#239dad"))[
  *Typst*
]
\end{verbatim}

\includegraphics[width=5in,height=\textheight,keepaspectratio]{/assets/docs/rKfIt6nqSzoBRXt7k7BMOwAAAAAAAAAA.png}

\paragraph{\texorpdfstring{\texttt{\ color\ }}{ color }}\label{definitions-rgb-color}

\href{/docs/reference/visualize/color/}{color}

{Required} {{ Positional }}

\phantomsection\label{definitions-rgb-color-positional-tooltip}
Positional parameters are specified in order, without names.

Alternatively: The color to convert to RGB(a).

If this is given, the individual components should not be given.

\subsubsection{\texorpdfstring{\texttt{\ cmyk\ }}{ cmyk }}\label{definitions-cmyk}

Create a CMYK color.

This is useful if you want to target a specific printer. The conversion
to RGB for display preview might differ from how your printer reproduces
the color.

A CMYK color is represented internally by an array of four components:

\begin{itemize}
\tightlist
\item
  cyan ( \href{/docs/reference/layout/ratio/}{\texttt{\ ratio\ }} )
\item
  magenta ( \href{/docs/reference/layout/ratio/}{\texttt{\ ratio\ }} )
\item
  yellow ( \href{/docs/reference/layout/ratio/}{\texttt{\ ratio\ }} )
\item
  key ( \href{/docs/reference/layout/ratio/}{\texttt{\ ratio\ }} )
\end{itemize}

These components are also available using the
\href{/docs/reference/visualize/color/\#definitions-components}{\texttt{\ components\ }}
method.

Note that CMYK colors are not currently supported when PDF/A output is
enabled.

color { . } { cmyk } (

{ \href{/docs/reference/layout/ratio/}{ratio} , } {
\href{/docs/reference/layout/ratio/}{ratio} , } {
\href{/docs/reference/layout/ratio/}{ratio} , } {
\href{/docs/reference/layout/ratio/}{ratio} , } {
\href{/docs/reference/visualize/color/}{color} , }

) -\textgreater{} \href{/docs/reference/visualize/color/}{color}

\begin{verbatim}
#square(
  fill: cmyk(27%, 0%, 3%, 5%)
)
\end{verbatim}

\includegraphics[width=5in,height=\textheight,keepaspectratio]{/assets/docs/1LHigtpFCZVjSNs83fP0eAAAAAAAAAAA.png}

\paragraph{\texorpdfstring{\texttt{\ cyan\ }}{ cyan }}\label{definitions-cmyk-cyan}

\href{/docs/reference/layout/ratio/}{ratio}

{Required} {{ Positional }}

\phantomsection\label{definitions-cmyk-cyan-positional-tooltip}
Positional parameters are specified in order, without names.

The cyan component.

\paragraph{\texorpdfstring{\texttt{\ magenta\ }}{ magenta }}\label{definitions-cmyk-magenta}

\href{/docs/reference/layout/ratio/}{ratio}

{Required} {{ Positional }}

\phantomsection\label{definitions-cmyk-magenta-positional-tooltip}
Positional parameters are specified in order, without names.

The magenta component.

\paragraph{\texorpdfstring{\texttt{\ yellow\ }}{ yellow }}\label{definitions-cmyk-yellow}

\href{/docs/reference/layout/ratio/}{ratio}

{Required} {{ Positional }}

\phantomsection\label{definitions-cmyk-yellow-positional-tooltip}
Positional parameters are specified in order, without names.

The yellow component.

\paragraph{\texorpdfstring{\texttt{\ key\ }}{ key }}\label{definitions-cmyk-key}

\href{/docs/reference/layout/ratio/}{ratio}

{Required} {{ Positional }}

\phantomsection\label{definitions-cmyk-key-positional-tooltip}
Positional parameters are specified in order, without names.

The key component.

\paragraph{\texorpdfstring{\texttt{\ color\ }}{ color }}\label{definitions-cmyk-color}

\href{/docs/reference/visualize/color/}{color}

{Required} {{ Positional }}

\phantomsection\label{definitions-cmyk-color-positional-tooltip}
Positional parameters are specified in order, without names.

Alternatively: The color to convert to CMYK.

If this is given, the individual components should not be given.

\subsubsection{\texorpdfstring{\texttt{\ hsl\ }}{ hsl }}\label{definitions-hsl}

Create an HSL color.

This color space is useful for specifying colors by hue, saturation and
lightness. It is also useful for color manipulation, such as saturating
while keeping perceived hue.

An HSL color is represented internally by an array of four components:

\begin{itemize}
\tightlist
\item
  hue ( \href{/docs/reference/layout/angle/}{\texttt{\ angle\ }} )
\item
  saturation ( \href{/docs/reference/layout/ratio/}{\texttt{\ ratio\ }}
  )
\item
  lightness ( \href{/docs/reference/layout/ratio/}{\texttt{\ ratio\ }} )
\item
  alpha ( \href{/docs/reference/layout/ratio/}{\texttt{\ ratio\ }} )
\end{itemize}

These components are also available using the
\href{/docs/reference/visualize/color/\#definitions-components}{\texttt{\ components\ }}
method.

color { . } { hsl } (

{ \href{/docs/reference/layout/angle/}{angle} , } {
\href{/docs/reference/foundations/int/}{int}
\href{/docs/reference/layout/ratio/}{ratio} , } {
\href{/docs/reference/foundations/int/}{int}
\href{/docs/reference/layout/ratio/}{ratio} , } {
\href{/docs/reference/foundations/int/}{int}
\href{/docs/reference/layout/ratio/}{ratio} , } {
\href{/docs/reference/visualize/color/}{color} , }

) -\textgreater{} \href{/docs/reference/visualize/color/}{color}

\begin{verbatim}
#square(
  fill: color.hsl(30deg, 50%, 60%)
)
\end{verbatim}

\includegraphics[width=5in,height=\textheight,keepaspectratio]{/assets/docs/MqR1NhT-m_ImBDX23hY7xgAAAAAAAAAA.png}

\paragraph{\texorpdfstring{\texttt{\ hue\ }}{ hue }}\label{definitions-hsl-hue}

\href{/docs/reference/layout/angle/}{angle}

{Required} {{ Positional }}

\phantomsection\label{definitions-hsl-hue-positional-tooltip}
Positional parameters are specified in order, without names.

The hue angle.

\paragraph{\texorpdfstring{\texttt{\ saturation\ }}{ saturation }}\label{definitions-hsl-saturation}

\href{/docs/reference/foundations/int/}{int} {or}
\href{/docs/reference/layout/ratio/}{ratio}

{Required} {{ Positional }}

\phantomsection\label{definitions-hsl-saturation-positional-tooltip}
Positional parameters are specified in order, without names.

The saturation component.

\paragraph{\texorpdfstring{\texttt{\ lightness\ }}{ lightness }}\label{definitions-hsl-lightness}

\href{/docs/reference/foundations/int/}{int} {or}
\href{/docs/reference/layout/ratio/}{ratio}

{Required} {{ Positional }}

\phantomsection\label{definitions-hsl-lightness-positional-tooltip}
Positional parameters are specified in order, without names.

The lightness component.

\paragraph{\texorpdfstring{\texttt{\ alpha\ }}{ alpha }}\label{definitions-hsl-alpha}

\href{/docs/reference/foundations/int/}{int} {or}
\href{/docs/reference/layout/ratio/}{ratio}

{Required} {{ Positional }}

\phantomsection\label{definitions-hsl-alpha-positional-tooltip}
Positional parameters are specified in order, without names.

The alpha component.

\paragraph{\texorpdfstring{\texttt{\ color\ }}{ color }}\label{definitions-hsl-color}

\href{/docs/reference/visualize/color/}{color}

{Required} {{ Positional }}

\phantomsection\label{definitions-hsl-color-positional-tooltip}
Positional parameters are specified in order, without names.

Alternatively: The color to convert to HSL.

If this is given, the individual components should not be given.

\subsubsection{\texorpdfstring{\texttt{\ hsv\ }}{ hsv }}\label{definitions-hsv}

Create an HSV color.

This color space is useful for specifying colors by hue, saturation and
value. It is also useful for color manipulation, such as saturating
while keeping perceived hue.

An HSV color is represented internally by an array of four components:

\begin{itemize}
\tightlist
\item
  hue ( \href{/docs/reference/layout/angle/}{\texttt{\ angle\ }} )
\item
  saturation ( \href{/docs/reference/layout/ratio/}{\texttt{\ ratio\ }}
  )
\item
  value ( \href{/docs/reference/layout/ratio/}{\texttt{\ ratio\ }} )
\item
  alpha ( \href{/docs/reference/layout/ratio/}{\texttt{\ ratio\ }} )
\end{itemize}

These components are also available using the
\href{/docs/reference/visualize/color/\#definitions-components}{\texttt{\ components\ }}
method.

color { . } { hsv } (

{ \href{/docs/reference/layout/angle/}{angle} , } {
\href{/docs/reference/foundations/int/}{int}
\href{/docs/reference/layout/ratio/}{ratio} , } {
\href{/docs/reference/foundations/int/}{int}
\href{/docs/reference/layout/ratio/}{ratio} , } {
\href{/docs/reference/foundations/int/}{int}
\href{/docs/reference/layout/ratio/}{ratio} , } {
\href{/docs/reference/visualize/color/}{color} , }

) -\textgreater{} \href{/docs/reference/visualize/color/}{color}

\begin{verbatim}
#square(
  fill: color.hsv(30deg, 50%, 60%)
)
\end{verbatim}

\includegraphics[width=5in,height=\textheight,keepaspectratio]{/assets/docs/dEOjXMxlVX8xgAuMFF-gkQAAAAAAAAAA.png}

\paragraph{\texorpdfstring{\texttt{\ hue\ }}{ hue }}\label{definitions-hsv-hue}

\href{/docs/reference/layout/angle/}{angle}

{Required} {{ Positional }}

\phantomsection\label{definitions-hsv-hue-positional-tooltip}
Positional parameters are specified in order, without names.

The hue angle.

\paragraph{\texorpdfstring{\texttt{\ saturation\ }}{ saturation }}\label{definitions-hsv-saturation}

\href{/docs/reference/foundations/int/}{int} {or}
\href{/docs/reference/layout/ratio/}{ratio}

{Required} {{ Positional }}

\phantomsection\label{definitions-hsv-saturation-positional-tooltip}
Positional parameters are specified in order, without names.

The saturation component.

\paragraph{\texorpdfstring{\texttt{\ value\ }}{ value }}\label{definitions-hsv-value}

\href{/docs/reference/foundations/int/}{int} {or}
\href{/docs/reference/layout/ratio/}{ratio}

{Required} {{ Positional }}

\phantomsection\label{definitions-hsv-value-positional-tooltip}
Positional parameters are specified in order, without names.

The value component.

\paragraph{\texorpdfstring{\texttt{\ alpha\ }}{ alpha }}\label{definitions-hsv-alpha}

\href{/docs/reference/foundations/int/}{int} {or}
\href{/docs/reference/layout/ratio/}{ratio}

{Required} {{ Positional }}

\phantomsection\label{definitions-hsv-alpha-positional-tooltip}
Positional parameters are specified in order, without names.

The alpha component.

\paragraph{\texorpdfstring{\texttt{\ color\ }}{ color }}\label{definitions-hsv-color}

\href{/docs/reference/visualize/color/}{color}

{Required} {{ Positional }}

\phantomsection\label{definitions-hsv-color-positional-tooltip}
Positional parameters are specified in order, without names.

Alternatively: The color to convert to HSL.

If this is given, the individual components should not be given.

\subsubsection{\texorpdfstring{\texttt{\ components\ }}{ components }}\label{definitions-components}

Extracts the components of this color.

The size and values of this array depends on the color space. You can
obtain the color space using
\href{/docs/reference/visualize/color/\#definitions-space}{\texttt{\ space\ }}
. Below is a table of the color spaces and their components:

\begin{longtable}[]{@{}lllll@{}}
\toprule\noalign{}
Color space & C1 & C2 & C3 & C4 \\
\midrule\noalign{}
\endhead
\bottomrule\noalign{}
\endlastfoot
\href{/docs/reference/visualize/color/\#definitions-luma}{\texttt{\ luma\ }}
& Lightness & & & \\
\href{/docs/reference/visualize/color/\#definitions-oklab}{\texttt{\ oklab\ }}
& Lightness & \texttt{\ a\ } & \texttt{\ b\ } & Alpha \\
\href{/docs/reference/visualize/color/\#definitions-oklch}{\texttt{\ oklch\ }}
& Lightness & Chroma & Hue & Alpha \\
\href{/docs/reference/visualize/color/\#definitions-linear-rgb}{\texttt{\ linear-rgb\ }}
& Red & Green & Blue & Alpha \\
\href{/docs/reference/visualize/color/\#definitions-rgb}{\texttt{\ rgb\ }}
& Red & Green & Blue & Alpha \\
\href{/docs/reference/visualize/color/\#definitions-cmyk}{\texttt{\ cmyk\ }}
& Cyan & Magenta & Yellow & Key \\
\href{/docs/reference/visualize/color/\#definitions-hsl}{\texttt{\ hsl\ }}
& Hue & Saturation & Lightness & Alpha \\
\href{/docs/reference/visualize/color/\#definitions-hsv}{\texttt{\ hsv\ }}
& Hue & Saturation & Value & Alpha \\
\end{longtable}

For the meaning and type of each individual value, see the documentation
of the corresponding color space. The alpha component is optional and
only included if the \texttt{\ alpha\ } argument is \texttt{\ true\ } .
The length of the returned array depends on the number of components and
whether the alpha component is included.

self { . } { components } (

{ \hyperref[definitions-components-parameters-alpha]{alpha :}
\href{/docs/reference/foundations/bool/}{bool} }

) -\textgreater{} \href{/docs/reference/foundations/array/}{array}

\begin{verbatim}
// note that the alpha component is included by default
#rgb(40%, 60%, 80%).components()
\end{verbatim}

\includegraphics[width=5in,height=\textheight,keepaspectratio]{/assets/docs/dzB_dzQf4SM_Ou0eAcFH9AAAAAAAAAAA.png}

\paragraph{\texorpdfstring{\texttt{\ alpha\ }}{ alpha }}\label{definitions-components-alpha}

\href{/docs/reference/foundations/bool/}{bool}

Whether to include the alpha component.

Default: \texttt{\ }{\texttt{\ true\ }}\texttt{\ }

\subsubsection{\texorpdfstring{\texttt{\ space\ }}{ space }}\label{definitions-space}

Returns the constructor function for this color\textquotesingle s space:

\begin{itemize}
\tightlist
\item
  \href{/docs/reference/visualize/color/\#definitions-luma}{\texttt{\ luma\ }}
\item
  \href{/docs/reference/visualize/color/\#definitions-oklab}{\texttt{\ oklab\ }}
\item
  \href{/docs/reference/visualize/color/\#definitions-oklch}{\texttt{\ oklch\ }}
\item
  \href{/docs/reference/visualize/color/\#definitions-linear-rgb}{\texttt{\ linear-rgb\ }}
\item
  \href{/docs/reference/visualize/color/\#definitions-rgb}{\texttt{\ rgb\ }}
\item
  \href{/docs/reference/visualize/color/\#definitions-cmyk}{\texttt{\ cmyk\ }}
\item
  \href{/docs/reference/visualize/color/\#definitions-hsl}{\texttt{\ hsl\ }}
\item
  \href{/docs/reference/visualize/color/\#definitions-hsv}{\texttt{\ hsv\ }}
\end{itemize}

self { . } { space } (

) -\textgreater{} { any }

\begin{verbatim}
#let color = cmyk(1%, 2%, 3%, 4%)
#(color.space() == cmyk)
\end{verbatim}

\includegraphics[width=5in,height=\textheight,keepaspectratio]{/assets/docs/tfic_6Fu9JDbk4Tz2rYgKAAAAAAAAAAA.png}

\subsubsection{\texorpdfstring{\texttt{\ to-hex\ }}{ to-hex }}\label{definitions-to-hex}

Returns the color\textquotesingle s RGB(A) hex representation (such as
\texttt{\ \#ffaa32\ } or \texttt{\ \#020304fe\ } ). The alpha component
(last two digits in \texttt{\ \#020304fe\ } ) is omitted if it is equal
to \texttt{\ ff\ } (255 / 100\%).

self { . } { to-hex } (

) -\textgreater{} \href{/docs/reference/foundations/str/}{str}

\subsubsection{\texorpdfstring{\texttt{\ lighten\ }}{ lighten }}\label{definitions-lighten}

Lightens a color by a given factor.

self { . } { lighten } (

{ \href{/docs/reference/layout/ratio/}{ratio} }

) -\textgreater{} \href{/docs/reference/visualize/color/}{color}

\paragraph{\texorpdfstring{\texttt{\ factor\ }}{ factor }}\label{definitions-lighten-factor}

\href{/docs/reference/layout/ratio/}{ratio}

{Required} {{ Positional }}

\phantomsection\label{definitions-lighten-factor-positional-tooltip}
Positional parameters are specified in order, without names.

The factor to lighten the color by.

\subsubsection{\texorpdfstring{\texttt{\ darken\ }}{ darken }}\label{definitions-darken}

Darkens a color by a given factor.

self { . } { darken } (

{ \href{/docs/reference/layout/ratio/}{ratio} }

) -\textgreater{} \href{/docs/reference/visualize/color/}{color}

\paragraph{\texorpdfstring{\texttt{\ factor\ }}{ factor }}\label{definitions-darken-factor}

\href{/docs/reference/layout/ratio/}{ratio}

{Required} {{ Positional }}

\phantomsection\label{definitions-darken-factor-positional-tooltip}
Positional parameters are specified in order, without names.

The factor to darken the color by.

\subsubsection{\texorpdfstring{\texttt{\ saturate\ }}{ saturate }}\label{definitions-saturate}

Increases the saturation of a color by a given factor.

self { . } { saturate } (

{ \href{/docs/reference/layout/ratio/}{ratio} }

) -\textgreater{} \href{/docs/reference/visualize/color/}{color}

\paragraph{\texorpdfstring{\texttt{\ factor\ }}{ factor }}\label{definitions-saturate-factor}

\href{/docs/reference/layout/ratio/}{ratio}

{Required} {{ Positional }}

\phantomsection\label{definitions-saturate-factor-positional-tooltip}
Positional parameters are specified in order, without names.

The factor to saturate the color by.

\subsubsection{\texorpdfstring{\texttt{\ desaturate\ }}{ desaturate }}\label{definitions-desaturate}

Decreases the saturation of a color by a given factor.

self { . } { desaturate } (

{ \href{/docs/reference/layout/ratio/}{ratio} }

) -\textgreater{} \href{/docs/reference/visualize/color/}{color}

\paragraph{\texorpdfstring{\texttt{\ factor\ }}{ factor }}\label{definitions-desaturate-factor}

\href{/docs/reference/layout/ratio/}{ratio}

{Required} {{ Positional }}

\phantomsection\label{definitions-desaturate-factor-positional-tooltip}
Positional parameters are specified in order, without names.

The factor to desaturate the color by.

\subsubsection{\texorpdfstring{\texttt{\ negate\ }}{ negate }}\label{definitions-negate}

Produces the complementary color using a provided color space. You can
think of it as the opposite side on a color wheel.

self { . } { negate } (

{ \hyperref[definitions-negate-parameters-space]{space :} { any } }

) -\textgreater{} \href{/docs/reference/visualize/color/}{color}

\begin{verbatim}
#square(fill: yellow)
#square(fill: yellow.negate())
#square(fill: yellow.negate(space: rgb))
\end{verbatim}

\includegraphics[width=5in,height=\textheight,keepaspectratio]{/assets/docs/oBWZW_i_eZ8A9K_46wXLaQAAAAAAAAAA.png}

\paragraph{\texorpdfstring{\texttt{\ space\ }}{ space }}\label{definitions-negate-space}

{ any }

The color space used for the transformation. By default, a perceptual
color space is used.

Default: \texttt{\ oklab\ }

\subsubsection{\texorpdfstring{\texttt{\ rotate\ }}{ rotate }}\label{definitions-rotate}

Rotates the hue of the color by a given angle.

self { . } { rotate } (

{ \href{/docs/reference/layout/angle/}{angle} , } {
\hyperref[definitions-rotate-parameters-space]{space :} { any } , }

) -\textgreater{} \href{/docs/reference/visualize/color/}{color}

\paragraph{\texorpdfstring{\texttt{\ angle\ }}{ angle }}\label{definitions-rotate-angle}

\href{/docs/reference/layout/angle/}{angle}

{Required} {{ Positional }}

\phantomsection\label{definitions-rotate-angle-positional-tooltip}
Positional parameters are specified in order, without names.

The angle to rotate the hue by.

\paragraph{\texorpdfstring{\texttt{\ space\ }}{ space }}\label{definitions-rotate-space}

{ any }

The color space used to rotate. By default, this happens in a perceptual
color space (
\href{/docs/reference/visualize/color/\#definitions-oklch}{\texttt{\ oklch\ }}
).

Default: \texttt{\ oklch\ }

\subsubsection{\texorpdfstring{\texttt{\ mix\ }}{ mix }}\label{definitions-mix}

Create a color by mixing two or more colors.

In color spaces with a hue component (hsl, hsv, oklch), only two colors
can be mixed at once. Mixing more than two colors in such a space will
result in an error!

color { . } { mix } (

{ \hyperref[definitions-mix-parameters-colors]{..}
\href{/docs/reference/visualize/color/}{color}
\href{/docs/reference/foundations/array/}{array} , } {
\hyperref[definitions-mix-parameters-space]{space :} { any } , }

) -\textgreater{} \href{/docs/reference/visualize/color/}{color}

\begin{verbatim}
#set block(height: 20pt, width: 100%)
#block(fill: red.mix(blue))
#block(fill: red.mix(blue, space: rgb))
#block(fill: color.mix(red, blue, white))
#block(fill: color.mix((red, 70%), (blue, 30%)))
\end{verbatim}

\includegraphics[width=5in,height=\textheight,keepaspectratio]{/assets/docs/0jAT6gZPo0X02CVXUm7YpAAAAAAAAAAA.png}

\paragraph{\texorpdfstring{\texttt{\ colors\ }}{ colors }}\label{definitions-mix-colors}

\href{/docs/reference/visualize/color/}{color} {or}
\href{/docs/reference/foundations/array/}{array}

{Required} {{ Positional }}

\phantomsection\label{definitions-mix-colors-positional-tooltip}
Positional parameters are specified in order, without names.

{{ Variadic }}

\phantomsection\label{definitions-mix-colors-variadic-tooltip}
Variadic parameters can be specified multiple times.

The colors, optionally with weights, specified as a pair (array of
length two) of color and weight (float or ratio).

The weights do not need to add to
\texttt{\ }{\texttt{\ 100\%\ }}\texttt{\ } , they are relative to the
sum of all weights.

\paragraph{\texorpdfstring{\texttt{\ space\ }}{ space }}\label{definitions-mix-space}

{ any }

The color space to mix in. By default, this happens in a perceptual
color space (
\href{/docs/reference/visualize/color/\#definitions-oklab}{\texttt{\ oklab\ }}
).

Default: \texttt{\ oklab\ }

\subsubsection{\texorpdfstring{\texttt{\ transparentize\ }}{ transparentize }}\label{definitions-transparentize}

Makes a color more transparent by a given factor.

This method is relative to the existing alpha value. If the scale is
positive, calculates \texttt{\ alpha\ -\ alpha\ *\ scale\ } . Negative
scales behave like \texttt{\ color.opacify(-scale)\ } .

self { . } { transparentize } (

{ \href{/docs/reference/layout/ratio/}{ratio} }

) -\textgreater{} \href{/docs/reference/visualize/color/}{color}

\begin{verbatim}
#block(fill: red)[opaque]
#block(fill: red.transparentize(50%))[half red]
#block(fill: red.transparentize(75%))[quarter red]
\end{verbatim}

\includegraphics[width=5in,height=\textheight,keepaspectratio]{/assets/docs/bnNXhQKfjc4AYVaZ1T3e3wAAAAAAAAAA.png}

\paragraph{\texorpdfstring{\texttt{\ scale\ }}{ scale }}\label{definitions-transparentize-scale}

\href{/docs/reference/layout/ratio/}{ratio}

{Required} {{ Positional }}

\phantomsection\label{definitions-transparentize-scale-positional-tooltip}
Positional parameters are specified in order, without names.

The factor to change the alpha value by.

\subsubsection{\texorpdfstring{\texttt{\ opacify\ }}{ opacify }}\label{definitions-opacify}

Makes a color more opaque by a given scale.

This method is relative to the existing alpha value. If the scale is
positive, calculates \texttt{\ alpha\ +\ scale\ -\ alpha\ *\ scale\ } .
Negative scales behave like \texttt{\ color.transparentize(-scale)\ } .

self { . } { opacify } (

{ \href{/docs/reference/layout/ratio/}{ratio} }

) -\textgreater{} \href{/docs/reference/visualize/color/}{color}

\begin{verbatim}
#let half-red = red.transparentize(50%)
#block(fill: half-red.opacify(100%))[opaque]
#block(fill: half-red.opacify(50%))[three quarters red]
#block(fill: half-red.opacify(-50%))[one quarter red]
\end{verbatim}

\includegraphics[width=5in,height=\textheight,keepaspectratio]{/assets/docs/1fq--2OrISH1g8_dvUBroAAAAAAAAAAA.png}

\paragraph{\texorpdfstring{\texttt{\ scale\ }}{ scale }}\label{definitions-opacify-scale}

\href{/docs/reference/layout/ratio/}{ratio}

{Required} {{ Positional }}

\phantomsection\label{definitions-opacify-scale-positional-tooltip}
Positional parameters are specified in order, without names.

The scale to change the alpha value by.

\href{/docs/reference/visualize/circle/}{\pandocbounded{\includesvg[keepaspectratio]{/assets/icons/16-arrow-right.svg}}}

{ Circle } { Previous page }

\href{/docs/reference/visualize/ellipse/}{\pandocbounded{\includesvg[keepaspectratio]{/assets/icons/16-arrow-right.svg}}}

{ Ellipse } { Next page }


