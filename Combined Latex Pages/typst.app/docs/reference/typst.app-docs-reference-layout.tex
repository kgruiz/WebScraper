\title{typst.app/docs/reference/layout/repeat}

\begin{itemize}
\tightlist
\item
  \href{/docs}{\includesvg[width=0.16667in,height=0.16667in]{/assets/icons/16-docs-dark.svg}}
\item
  \includesvg[width=0.16667in,height=0.16667in]{/assets/icons/16-arrow-right.svg}
\item
  \href{/docs/reference/}{Reference}
\item
  \includesvg[width=0.16667in,height=0.16667in]{/assets/icons/16-arrow-right.svg}
\item
  \href{/docs/reference/layout/}{Layout}
\item
  \includesvg[width=0.16667in,height=0.16667in]{/assets/icons/16-arrow-right.svg}
\item
  \href{/docs/reference/layout/repeat/}{Repeat}
\end{itemize}

\section{\texorpdfstring{\texttt{\ repeat\ } {{ Element
}}}{ repeat   Element }}\label{summary}

\phantomsection\label{element-tooltip}
Element functions can be customized with \texttt{\ set\ } and
\texttt{\ show\ } rules.

Repeats content to the available space.

This can be useful when implementing a custom index, reference, or
outline.

Space may be inserted between the instances of the body parameter, so be
sure to adjust the
\href{/docs/reference/layout/repeat/\#parameters-justify}{\texttt{\ justify\ }}
parameter accordingly.

Errors if there no bounds on the available space, as it would create
infinite content.

\subsection{Example}\label{example}

\begin{verbatim}
Sign on the dotted line:
#box(width: 1fr, repeat[.])

#set text(10pt)
#v(8pt, weak: true)
#align(right)[
  Berlin, the 22nd of December, 2022
]
\end{verbatim}

\includegraphics[width=5in,height=\textheight,keepaspectratio]{/assets/docs/LGILa4453RB6xoEobzmQcAAAAAAAAAAA.png}

\subsection{\texorpdfstring{{ Parameters
}}{ Parameters }}\label{parameters}

\phantomsection\label{parameters-tooltip}
Parameters are the inputs to a function. They are specified in
parentheses after the function name.

{ repeat } (

{ \href{/docs/reference/foundations/content/}{content} , } {
\hyperref[parameters-gap]{gap :}
\href{/docs/reference/layout/length/}{length} , } {
\hyperref[parameters-justify]{justify :}
\href{/docs/reference/foundations/bool/}{bool} , }

) -\textgreater{} \href{/docs/reference/foundations/content/}{content}

\subsubsection{\texorpdfstring{\texttt{\ body\ }}{ body }}\label{parameters-body}

\href{/docs/reference/foundations/content/}{content}

{Required} {{ Positional }}

\phantomsection\label{parameters-body-positional-tooltip}
Positional parameters are specified in order, without names.

The content to repeat.

\subsubsection{\texorpdfstring{\texttt{\ gap\ }}{ gap }}\label{parameters-gap}

\href{/docs/reference/layout/length/}{length}

{{ Settable }}

\phantomsection\label{parameters-gap-settable-tooltip}
Settable parameters can be customized for all following uses of the
function with a \texttt{\ set\ } rule.

The gap between each instance of the body.

Default: \texttt{\ }{\texttt{\ 0pt\ }}\texttt{\ }

\subsubsection{\texorpdfstring{\texttt{\ justify\ }}{ justify }}\label{parameters-justify}

\href{/docs/reference/foundations/bool/}{bool}

{{ Settable }}

\phantomsection\label{parameters-justify-settable-tooltip}
Settable parameters can be customized for all following uses of the
function with a \texttt{\ set\ } rule.

Whether to increase the gap between instances to completely fill the
available space.

Default: \texttt{\ }{\texttt{\ true\ }}\texttt{\ }

\href{/docs/reference/layout/relative/}{\pandocbounded{\includesvg[keepaspectratio]{/assets/icons/16-arrow-right.svg}}}

{ Relative Length } { Previous page }

\href{/docs/reference/layout/rotate/}{\pandocbounded{\includesvg[keepaspectratio]{/assets/icons/16-arrow-right.svg}}}

{ Rotate } { Next page }


\title{typst.app/docs/reference/layout/pad}

\begin{itemize}
\tightlist
\item
  \href{/docs}{\includesvg[width=0.16667in,height=0.16667in]{/assets/icons/16-docs-dark.svg}}
\item
  \includesvg[width=0.16667in,height=0.16667in]{/assets/icons/16-arrow-right.svg}
\item
  \href{/docs/reference/}{Reference}
\item
  \includesvg[width=0.16667in,height=0.16667in]{/assets/icons/16-arrow-right.svg}
\item
  \href{/docs/reference/layout/}{Layout}
\item
  \includesvg[width=0.16667in,height=0.16667in]{/assets/icons/16-arrow-right.svg}
\item
  \href{/docs/reference/layout/pad/}{Padding}
\end{itemize}

\section{\texorpdfstring{\texttt{\ pad\ } {{ Element
}}}{ pad   Element }}\label{summary}

\phantomsection\label{element-tooltip}
Element functions can be customized with \texttt{\ set\ } and
\texttt{\ show\ } rules.

Adds spacing around content.

The spacing can be specified for each side individually, or for all
sides at once by specifying a positional argument.

\subsection{Example}\label{example}

\begin{verbatim}
#set align(center)

#pad(x: 16pt, image("typing.jpg"))
_Typing speeds can be
 measured in words per minute._
\end{verbatim}

\includegraphics[width=5in,height=\textheight,keepaspectratio]{/assets/docs/YnvzY3ls2HrcPgokDMxVqwAAAAAAAAAA.png}

\subsection{\texorpdfstring{{ Parameters
}}{ Parameters }}\label{parameters}

\phantomsection\label{parameters-tooltip}
Parameters are the inputs to a function. They are specified in
parentheses after the function name.

{ pad } (

{ \hyperref[parameters-left]{left :}
\href{/docs/reference/layout/relative/}{relative} , } {
\hyperref[parameters-top]{top :}
\href{/docs/reference/layout/relative/}{relative} , } {
\hyperref[parameters-right]{right :}
\href{/docs/reference/layout/relative/}{relative} , } {
\hyperref[parameters-bottom]{bottom :}
\href{/docs/reference/layout/relative/}{relative} , } {
\hyperref[parameters-x]{x :}
\href{/docs/reference/layout/relative/}{relative} , } {
\hyperref[parameters-y]{y :}
\href{/docs/reference/layout/relative/}{relative} , } {
\hyperref[parameters-rest]{rest :}
\href{/docs/reference/layout/relative/}{relative} , } {
\href{/docs/reference/foundations/content/}{content} , }

) -\textgreater{} \href{/docs/reference/foundations/content/}{content}

\subsubsection{\texorpdfstring{\texttt{\ left\ }}{ left }}\label{parameters-left}

\href{/docs/reference/layout/relative/}{relative}

{{ Settable }}

\phantomsection\label{parameters-left-settable-tooltip}
Settable parameters can be customized for all following uses of the
function with a \texttt{\ set\ } rule.

The padding at the left side.

Default:
\texttt{\ }{\texttt{\ 0\%\ }}\texttt{\ }{\texttt{\ +\ }}\texttt{\ }{\texttt{\ 0pt\ }}\texttt{\ }

\subsubsection{\texorpdfstring{\texttt{\ top\ }}{ top }}\label{parameters-top}

\href{/docs/reference/layout/relative/}{relative}

{{ Settable }}

\phantomsection\label{parameters-top-settable-tooltip}
Settable parameters can be customized for all following uses of the
function with a \texttt{\ set\ } rule.

The padding at the top side.

Default:
\texttt{\ }{\texttt{\ 0\%\ }}\texttt{\ }{\texttt{\ +\ }}\texttt{\ }{\texttt{\ 0pt\ }}\texttt{\ }

\subsubsection{\texorpdfstring{\texttt{\ right\ }}{ right }}\label{parameters-right}

\href{/docs/reference/layout/relative/}{relative}

{{ Settable }}

\phantomsection\label{parameters-right-settable-tooltip}
Settable parameters can be customized for all following uses of the
function with a \texttt{\ set\ } rule.

The padding at the right side.

Default:
\texttt{\ }{\texttt{\ 0\%\ }}\texttt{\ }{\texttt{\ +\ }}\texttt{\ }{\texttt{\ 0pt\ }}\texttt{\ }

\subsubsection{\texorpdfstring{\texttt{\ bottom\ }}{ bottom }}\label{parameters-bottom}

\href{/docs/reference/layout/relative/}{relative}

{{ Settable }}

\phantomsection\label{parameters-bottom-settable-tooltip}
Settable parameters can be customized for all following uses of the
function with a \texttt{\ set\ } rule.

The padding at the bottom side.

Default:
\texttt{\ }{\texttt{\ 0\%\ }}\texttt{\ }{\texttt{\ +\ }}\texttt{\ }{\texttt{\ 0pt\ }}\texttt{\ }

\subsubsection{\texorpdfstring{\texttt{\ x\ }}{ x }}\label{parameters-x}

\href{/docs/reference/layout/relative/}{relative}

{{ Settable }}

\phantomsection\label{parameters-x-settable-tooltip}
Settable parameters can be customized for all following uses of the
function with a \texttt{\ set\ } rule.

A shorthand to set \texttt{\ left\ } and \texttt{\ right\ } to the same
value.

Default:
\texttt{\ }{\texttt{\ 0\%\ }}\texttt{\ }{\texttt{\ +\ }}\texttt{\ }{\texttt{\ 0pt\ }}\texttt{\ }

\subsubsection{\texorpdfstring{\texttt{\ y\ }}{ y }}\label{parameters-y}

\href{/docs/reference/layout/relative/}{relative}

{{ Settable }}

\phantomsection\label{parameters-y-settable-tooltip}
Settable parameters can be customized for all following uses of the
function with a \texttt{\ set\ } rule.

A shorthand to set \texttt{\ top\ } and \texttt{\ bottom\ } to the same
value.

Default:
\texttt{\ }{\texttt{\ 0\%\ }}\texttt{\ }{\texttt{\ +\ }}\texttt{\ }{\texttt{\ 0pt\ }}\texttt{\ }

\subsubsection{\texorpdfstring{\texttt{\ rest\ }}{ rest }}\label{parameters-rest}

\href{/docs/reference/layout/relative/}{relative}

{{ Settable }}

\phantomsection\label{parameters-rest-settable-tooltip}
Settable parameters can be customized for all following uses of the
function with a \texttt{\ set\ } rule.

A shorthand to set all four sides to the same value.

Default:
\texttt{\ }{\texttt{\ 0\%\ }}\texttt{\ }{\texttt{\ +\ }}\texttt{\ }{\texttt{\ 0pt\ }}\texttt{\ }

\subsubsection{\texorpdfstring{\texttt{\ body\ }}{ body }}\label{parameters-body}

\href{/docs/reference/foundations/content/}{content}

{Required} {{ Positional }}

\phantomsection\label{parameters-body-positional-tooltip}
Positional parameters are specified in order, without names.

The content to pad at the sides.

\href{/docs/reference/layout/move/}{\pandocbounded{\includesvg[keepaspectratio]{/assets/icons/16-arrow-right.svg}}}

{ Move } { Previous page }

\href{/docs/reference/layout/page/}{\pandocbounded{\includesvg[keepaspectratio]{/assets/icons/16-arrow-right.svg}}}

{ Page } { Next page }


\title{typst.app/docs/reference/layout/grid}

\begin{itemize}
\tightlist
\item
  \href{/docs}{\includesvg[width=0.16667in,height=0.16667in]{/assets/icons/16-docs-dark.svg}}
\item
  \includesvg[width=0.16667in,height=0.16667in]{/assets/icons/16-arrow-right.svg}
\item
  \href{/docs/reference/}{Reference}
\item
  \includesvg[width=0.16667in,height=0.16667in]{/assets/icons/16-arrow-right.svg}
\item
  \href{/docs/reference/layout/}{Layout}
\item
  \includesvg[width=0.16667in,height=0.16667in]{/assets/icons/16-arrow-right.svg}
\item
  \href{/docs/reference/layout/grid/}{Grid}
\end{itemize}

\section{\texorpdfstring{\texttt{\ grid\ } {{ Element
}}}{ grid   Element }}\label{summary}

\phantomsection\label{element-tooltip}
Element functions can be customized with \texttt{\ set\ } and
\texttt{\ show\ } rules.

Arranges content in a grid.

The grid element allows you to arrange content in a grid. You can define
the number of rows and columns, as well as the size of the gutters
between them. There are multiple sizing modes for columns and rows that
can be used to create complex layouts.

While the grid and table elements work very similarly, they are intended
for different use cases and carry different semantics. The grid element
is intended for presentational and layout purposes, while the
\href{/docs/reference/model/table/}{\texttt{\ table\ }} element is
intended for, in broad terms, presenting multiple related data points.
In the future, Typst will annotate its output such that screenreaders
will announce content in \texttt{\ table\ } as tabular while a
grid\textquotesingle s content will be announced no different than
multiple content blocks in the document flow. Set and show rules on one
of these elements do not affect the other.

A grid\textquotesingle s sizing is determined by the track sizes
specified in the arguments. Because each of the sizing parameters
accepts the same values, we will explain them just once, here. Each
sizing argument accepts an array of individual track sizes. A track size
is either:

\begin{itemize}
\item
  \texttt{\ }{\texttt{\ auto\ }}\texttt{\ } : The track will be sized to
  fit its contents. It will be at most as large as the remaining space.
  If there is more than one \texttt{\ }{\texttt{\ auto\ }}\texttt{\ }
  track width, and together they claim more than the available space,
  the \texttt{\ }{\texttt{\ auto\ }}\texttt{\ } tracks will fairly
  distribute the available space among themselves.
\item
  A fixed or relative length (e.g.
  \texttt{\ }{\texttt{\ 10pt\ }}\texttt{\ } or
  \texttt{\ }{\texttt{\ 20\%\ }}\texttt{\ }{\texttt{\ -\ }}\texttt{\ }{\texttt{\ 1cm\ }}\texttt{\ }
  ): The track will be exactly of this size.
\item
  A fractional length (e.g. \texttt{\ }{\texttt{\ 1fr\ }}\texttt{\ } ):
  Once all other tracks have been sized, the remaining space will be
  divided among the fractional tracks according to their fractions. For
  example, if there are two fractional tracks, each with a fraction of
  \texttt{\ }{\texttt{\ 1fr\ }}\texttt{\ } , they will each take up half
  of the remaining space.
\end{itemize}

To specify a single track, the array can be omitted in favor of a single
value. To specify multiple \texttt{\ }{\texttt{\ auto\ }}\texttt{\ }
tracks, enter the number of tracks instead of an array. For example,
\texttt{\ columns:\ } \texttt{\ }{\texttt{\ 3\ }}\texttt{\ } is
equivalent to \texttt{\ columns:\ }
\texttt{\ }{\texttt{\ (\ }}\texttt{\ }{\texttt{\ auto\ }}\texttt{\ }{\texttt{\ ,\ }}\texttt{\ }{\texttt{\ auto\ }}\texttt{\ }{\texttt{\ ,\ }}\texttt{\ }{\texttt{\ auto\ }}\texttt{\ }{\texttt{\ )\ }}\texttt{\ }
.

\subsection{Examples}\label{examples}

The example below demonstrates the different track sizing options. It
also shows how you can use
\href{/docs/reference/layout/grid/\#definitions-cell}{\texttt{\ grid.cell\ }}
to make an individual cell span two grid tracks.

\begin{verbatim}
// We use `rect` to emphasize the
// area of cells.
#set rect(
  inset: 8pt,
  fill: rgb("e4e5ea"),
  width: 100%,
)

#grid(
  columns: (60pt, 1fr, 2fr),
  rows: (auto, 60pt),
  gutter: 3pt,
  rect[Fixed width, auto height],
  rect[1/3 of the remains],
  rect[2/3 of the remains],
  rect(height: 100%)[Fixed height],
  grid.cell(
    colspan: 2,
    image("tiger.jpg", width: 100%),
  ),
)
\end{verbatim}

\includegraphics[width=5in,height=\textheight,keepaspectratio]{/assets/docs/nU6HFHUP8AJwyw_E8LwJrgAAAAAAAAAA.png}

You can also
\href{/docs/reference/foundations/arguments/\#spreading}{spread} an
array of strings or content into a grid to populate its cells.

\begin{verbatim}
#grid(
  columns: 5,
  gutter: 5pt,
  ..range(25).map(str)
)
\end{verbatim}

\includegraphics[width=5in,height=\textheight,keepaspectratio]{/assets/docs/qtEXI9WWslJNDT0wWvWAggAAAAAAAAAA.png}

\subsection{Styling the grid}\label{styling-the-grid}

The grid\textquotesingle s appearance can be customized through
different parameters. These are the most important ones:

\begin{itemize}
\tightlist
\item
  \href{/docs/reference/layout/grid/\#parameters-fill}{\texttt{\ fill\ }}
  to give all cells a background
\item
  \href{/docs/reference/layout/grid/\#parameters-align}{\texttt{\ align\ }}
  to change how cells are aligned
\item
  \href{/docs/reference/layout/grid/\#parameters-inset}{\texttt{\ inset\ }}
  to optionally add internal padding to each cell
\item
  \href{/docs/reference/layout/grid/\#parameters-stroke}{\texttt{\ stroke\ }}
  to optionally enable grid lines with a certain stroke
\end{itemize}

If you need to override one of the above options for a single cell, you
can use the
\href{/docs/reference/layout/grid/\#definitions-cell}{\texttt{\ grid.cell\ }}
element. Likewise, you can override individual grid lines with the
\href{/docs/reference/layout/grid/\#definitions-hline}{\texttt{\ grid.hline\ }}
and
\href{/docs/reference/layout/grid/\#definitions-vline}{\texttt{\ grid.vline\ }}
elements.

Alternatively, if you need the appearance options to depend on a
cell\textquotesingle s position (column and row), you may specify a
function to \texttt{\ fill\ } or \texttt{\ align\ } of the form
\texttt{\ (column,\ row)\ =\textgreater{}\ value\ } . You may also use a
show rule on
\href{/docs/reference/layout/grid/\#definitions-cell}{\texttt{\ grid.cell\ }}
- see that element\textquotesingle s examples or the examples below for
more information.

Locating most of your styling in set and show rules is recommended, as
it keeps the grid\textquotesingle s or table\textquotesingle s actual
usages clean and easy to read. It also allows you to easily change the
grid\textquotesingle s appearance in one place.

\subsubsection{Stroke styling
precedence}\label{stroke-styling-precedence}

There are three ways to set the stroke of a grid cell: through
\href{/docs/reference/layout/grid/\#definitions-cell-stroke}{\texttt{\ grid\ }{\texttt{\ .\ }}\texttt{\ cell\ }
\textquotesingle s \texttt{\ stroke\ } field} , by using
\href{/docs/reference/layout/grid/\#definitions-hline}{\texttt{\ grid\ }{\texttt{\ .\ }}\texttt{\ hline\ }}
and
\href{/docs/reference/layout/grid/\#definitions-vline}{\texttt{\ grid\ }{\texttt{\ .\ }}\texttt{\ vline\ }}
, or by setting the
\href{/docs/reference/layout/grid/\#parameters-stroke}{\texttt{\ grid\ }
\textquotesingle s \texttt{\ stroke\ } field} . When multiple of these
settings are present and conflict, the \texttt{\ hline\ } and
\texttt{\ vline\ } settings take the highest precedence, followed by the
\texttt{\ cell\ } settings, and finally the \texttt{\ grid\ } settings.

Furthermore, strokes of a repeated grid header or footer will take
precedence over regular cell strokes.

\subsection{\texorpdfstring{{ Parameters
}}{ Parameters }}\label{parameters}

\phantomsection\label{parameters-tooltip}
Parameters are the inputs to a function. They are specified in
parentheses after the function name.

{ grid } (

{ \hyperref[parameters-columns]{columns :}
\href{/docs/reference/foundations/auto/}{auto}
\href{/docs/reference/foundations/int/}{int}
\href{/docs/reference/layout/relative/}{relative}
\href{/docs/reference/layout/fraction/}{fraction}
\href{/docs/reference/foundations/array/}{array} , } {
\hyperref[parameters-rows]{rows :}
\href{/docs/reference/foundations/auto/}{auto}
\href{/docs/reference/foundations/int/}{int}
\href{/docs/reference/layout/relative/}{relative}
\href{/docs/reference/layout/fraction/}{fraction}
\href{/docs/reference/foundations/array/}{array} , } {
\hyperref[parameters-gutter]{gutter :}
\href{/docs/reference/foundations/auto/}{auto}
\href{/docs/reference/foundations/int/}{int}
\href{/docs/reference/layout/relative/}{relative}
\href{/docs/reference/layout/fraction/}{fraction}
\href{/docs/reference/foundations/array/}{array} , } {
\hyperref[parameters-column-gutter]{column-gutter :}
\href{/docs/reference/foundations/auto/}{auto}
\href{/docs/reference/foundations/int/}{int}
\href{/docs/reference/layout/relative/}{relative}
\href{/docs/reference/layout/fraction/}{fraction}
\href{/docs/reference/foundations/array/}{array} , } {
\hyperref[parameters-row-gutter]{row-gutter :}
\href{/docs/reference/foundations/auto/}{auto}
\href{/docs/reference/foundations/int/}{int}
\href{/docs/reference/layout/relative/}{relative}
\href{/docs/reference/layout/fraction/}{fraction}
\href{/docs/reference/foundations/array/}{array} , } {
\hyperref[parameters-fill]{fill :}
\href{/docs/reference/foundations/none/}{none}
\href{/docs/reference/visualize/color/}{color}
\href{/docs/reference/visualize/gradient/}{gradient}
\href{/docs/reference/foundations/array/}{array}
\href{/docs/reference/visualize/pattern/}{pattern}
\href{/docs/reference/foundations/function/}{function} , } {
\hyperref[parameters-align]{align :}
\href{/docs/reference/foundations/auto/}{auto}
\href{/docs/reference/foundations/array/}{array}
\href{/docs/reference/layout/alignment/}{alignment}
\href{/docs/reference/foundations/function/}{function} , } {
\hyperref[parameters-stroke]{stroke :}
\href{/docs/reference/foundations/none/}{none}
\href{/docs/reference/layout/length/}{length}
\href{/docs/reference/visualize/color/}{color}
\href{/docs/reference/visualize/gradient/}{gradient}
\href{/docs/reference/foundations/array/}{array}
\href{/docs/reference/visualize/stroke/}{stroke}
\href{/docs/reference/visualize/pattern/}{pattern}
\href{/docs/reference/foundations/dictionary/}{dictionary}
\href{/docs/reference/foundations/function/}{function} , } {
\hyperref[parameters-inset]{inset :}
\href{/docs/reference/layout/relative/}{relative}
\href{/docs/reference/foundations/array/}{array}
\href{/docs/reference/foundations/dictionary/}{dictionary}
\href{/docs/reference/foundations/function/}{function} , } {
\hyperref[parameters-children]{..}
\href{/docs/reference/foundations/content/}{content} , }

) -\textgreater{} \href{/docs/reference/foundations/content/}{content}

\subsubsection{\texorpdfstring{\texttt{\ columns\ }}{ columns }}\label{parameters-columns}

\href{/docs/reference/foundations/auto/}{auto} {or}
\href{/docs/reference/foundations/int/}{int} {or}
\href{/docs/reference/layout/relative/}{relative} {or}
\href{/docs/reference/layout/fraction/}{fraction} {or}
\href{/docs/reference/foundations/array/}{array}

{{ Settable }}

\phantomsection\label{parameters-columns-settable-tooltip}
Settable parameters can be customized for all following uses of the
function with a \texttt{\ set\ } rule.

The column sizes.

Either specify a track size array or provide an integer to create a grid
with that many \texttt{\ }{\texttt{\ auto\ }}\texttt{\ } -sized columns.
Note that opposed to rows and gutters, providing a single track size
will only ever create a single column.

Default:
\texttt{\ }{\texttt{\ (\ }}\texttt{\ }{\texttt{\ )\ }}\texttt{\ }

\subsubsection{\texorpdfstring{\texttt{\ rows\ }}{ rows }}\label{parameters-rows}

\href{/docs/reference/foundations/auto/}{auto} {or}
\href{/docs/reference/foundations/int/}{int} {or}
\href{/docs/reference/layout/relative/}{relative} {or}
\href{/docs/reference/layout/fraction/}{fraction} {or}
\href{/docs/reference/foundations/array/}{array}

{{ Settable }}

\phantomsection\label{parameters-rows-settable-tooltip}
Settable parameters can be customized for all following uses of the
function with a \texttt{\ set\ } rule.

The row sizes.

If there are more cells than fit the defined rows, the last row is
repeated until there are no more cells.

Default:
\texttt{\ }{\texttt{\ (\ }}\texttt{\ }{\texttt{\ )\ }}\texttt{\ }

\subsubsection{\texorpdfstring{\texttt{\ gutter\ }}{ gutter }}\label{parameters-gutter}

\href{/docs/reference/foundations/auto/}{auto} {or}
\href{/docs/reference/foundations/int/}{int} {or}
\href{/docs/reference/layout/relative/}{relative} {or}
\href{/docs/reference/layout/fraction/}{fraction} {or}
\href{/docs/reference/foundations/array/}{array}

{{ Settable }}

\phantomsection\label{parameters-gutter-settable-tooltip}
Settable parameters can be customized for all following uses of the
function with a \texttt{\ set\ } rule.

The gaps between rows and columns.

If there are more gutters than defined sizes, the last gutter is
repeated.

This is a shorthand to set \texttt{\ column-gutter\ } and
\texttt{\ row-gutter\ } to the same value.

Default:
\texttt{\ }{\texttt{\ (\ }}\texttt{\ }{\texttt{\ )\ }}\texttt{\ }

\subsubsection{\texorpdfstring{\texttt{\ column-gutter\ }}{ column-gutter }}\label{parameters-column-gutter}

\href{/docs/reference/foundations/auto/}{auto} {or}
\href{/docs/reference/foundations/int/}{int} {or}
\href{/docs/reference/layout/relative/}{relative} {or}
\href{/docs/reference/layout/fraction/}{fraction} {or}
\href{/docs/reference/foundations/array/}{array}

{{ Settable }}

\phantomsection\label{parameters-column-gutter-settable-tooltip}
Settable parameters can be customized for all following uses of the
function with a \texttt{\ set\ } rule.

The gaps between columns.

Default:
\texttt{\ }{\texttt{\ (\ }}\texttt{\ }{\texttt{\ )\ }}\texttt{\ }

\subsubsection{\texorpdfstring{\texttt{\ row-gutter\ }}{ row-gutter }}\label{parameters-row-gutter}

\href{/docs/reference/foundations/auto/}{auto} {or}
\href{/docs/reference/foundations/int/}{int} {or}
\href{/docs/reference/layout/relative/}{relative} {or}
\href{/docs/reference/layout/fraction/}{fraction} {or}
\href{/docs/reference/foundations/array/}{array}

{{ Settable }}

\phantomsection\label{parameters-row-gutter-settable-tooltip}
Settable parameters can be customized for all following uses of the
function with a \texttt{\ set\ } rule.

The gaps between rows.

Default:
\texttt{\ }{\texttt{\ (\ }}\texttt{\ }{\texttt{\ )\ }}\texttt{\ }

\subsubsection{\texorpdfstring{\texttt{\ fill\ }}{ fill }}\label{parameters-fill}

\href{/docs/reference/foundations/none/}{none} {or}
\href{/docs/reference/visualize/color/}{color} {or}
\href{/docs/reference/visualize/gradient/}{gradient} {or}
\href{/docs/reference/foundations/array/}{array} {or}
\href{/docs/reference/visualize/pattern/}{pattern} {or}
\href{/docs/reference/foundations/function/}{function}

{{ Settable }}

\phantomsection\label{parameters-fill-settable-tooltip}
Settable parameters can be customized for all following uses of the
function with a \texttt{\ set\ } rule.

How to fill the cells.

This can be a color or a function that returns a color. The function
receives the cells\textquotesingle{} column and row indices, starting
from zero. This can be used to implement striped grids.

Default: \texttt{\ }{\texttt{\ none\ }}\texttt{\ }

\includesvg[width=0.16667in,height=0.16667in]{/assets/icons/16-arrow-right.svg}
View example

\begin{verbatim}
#grid(
  fill: (x, y) =>
    if calc.even(x + y) { luma(230) }
    else { white },
  align: center + horizon,
  columns: 4,
  inset: 2pt,
  [X], [O], [X], [O],
  [O], [X], [O], [X],
  [X], [O], [X], [O],
  [O], [X], [O], [X],
)
\end{verbatim}

\includegraphics[width=5in,height=\textheight,keepaspectratio]{/assets/docs/YWpStHlSHlCZTmUmBJs9XQAAAAAAAAAA.png}

\subsubsection{\texorpdfstring{\texttt{\ align\ }}{ align }}\label{parameters-align}

\href{/docs/reference/foundations/auto/}{auto} {or}
\href{/docs/reference/foundations/array/}{array} {or}
\href{/docs/reference/layout/alignment/}{alignment} {or}
\href{/docs/reference/foundations/function/}{function}

{{ Settable }}

\phantomsection\label{parameters-align-settable-tooltip}
Settable parameters can be customized for all following uses of the
function with a \texttt{\ set\ } rule.

How to align the cells\textquotesingle{} content.

This can either be a single alignment, an array of alignments
(corresponding to each column) or a function that returns an alignment.
The function receives the cells\textquotesingle{} column and row
indices, starting from zero. If set to
\texttt{\ }{\texttt{\ auto\ }}\texttt{\ } , the outer alignment is used.

You can find an example for this argument at the
\href{/docs/reference/model/table/\#parameters-align}{\texttt{\ table.align\ }}
parameter.

Default: \texttt{\ }{\texttt{\ auto\ }}\texttt{\ }

\subsubsection{\texorpdfstring{\texttt{\ stroke\ }}{ stroke }}\label{parameters-stroke}

\href{/docs/reference/foundations/none/}{none} {or}
\href{/docs/reference/layout/length/}{length} {or}
\href{/docs/reference/visualize/color/}{color} {or}
\href{/docs/reference/visualize/gradient/}{gradient} {or}
\href{/docs/reference/foundations/array/}{array} {or}
\href{/docs/reference/visualize/stroke/}{stroke} {or}
\href{/docs/reference/visualize/pattern/}{pattern} {or}
\href{/docs/reference/foundations/dictionary/}{dictionary} {or}
\href{/docs/reference/foundations/function/}{function}

{{ Settable }}

\phantomsection\label{parameters-stroke-settable-tooltip}
Settable parameters can be customized for all following uses of the
function with a \texttt{\ set\ } rule.

How to \href{/docs/reference/visualize/stroke/}{stroke} the cells.

Grids have no strokes by default, which can be changed by setting this
option to the desired stroke.

If it is necessary to place lines which can cross spacing between cells
produced by the \texttt{\ gutter\ } option, or to override the stroke
between multiple specific cells, consider specifying one or more of
\href{/docs/reference/layout/grid/\#definitions-hline}{\texttt{\ grid.hline\ }}
and
\href{/docs/reference/layout/grid/\#definitions-vline}{\texttt{\ grid.vline\ }}
alongside your grid cells.

Default:
\texttt{\ }{\texttt{\ (\ }}\texttt{\ }{\texttt{\ :\ }}\texttt{\ }{\texttt{\ )\ }}\texttt{\ }

\includesvg[width=0.16667in,height=0.16667in]{/assets/icons/16-arrow-right.svg}
View example

\begin{verbatim}
#set page(height: 13em, width: 26em)

#let cv(..jobs) = grid(
    columns: 2,
    inset: 5pt,
    stroke: (x, y) => if x == 0 and y > 0 {
      (right: (
        paint: luma(180),
        thickness: 1.5pt,
        dash: "dotted"
      ))
    },
    grid.header(grid.cell(colspan: 2)[
      *Professional Experience*
      #box(width: 1fr, line(length: 100%, stroke: luma(180)))
    ]),
    ..{
      let last = none
      for job in jobs.pos() {
        (
          if job.year != last [*#job.year*],
          [
            *#job.company* - #job.role _(#job.timeframe)_ \
            #job.details
          ]
        )
        last = job.year
      }
    }
  )

  #cv(
    (
      year: 2012,
      company: [Pear Seed & Co.],
      role: [Lead Engineer],
      timeframe: [Jul - Dec],
      details: [
        - Raised engineers from 3x to 10x
        - Did a great job
      ],
    ),
    (
      year: 2012,
      company: [Mega Corp.],
      role: [VP of Sales],
      timeframe: [Mar - Jun],
      details: [- Closed tons of customers],
    ),
    (
      year: 2013,
      company: [Tiny Co.],
      role: [CEO],
      timeframe: [Jan - Dec],
      details: [- Delivered 4x more shareholder value],
    ),
    (
      year: 2014,
      company: [Glorbocorp Ltd],
      role: [CTO],
      timeframe: [Jan - Mar],
      details: [- Drove containerization forward],
    ),
  )
\end{verbatim}

\includegraphics[width=5.95833in,height=\textheight,keepaspectratio]{/assets/docs/5kfvlcbAPUFkWJtXr3FdMgAAAAAAAAAA.png}
\includegraphics[width=5.95833in,height=\textheight,keepaspectratio]{/assets/docs/5kfvlcbAPUFkWJtXr3FdMgAAAAAAAAAB.png}

\subsubsection{\texorpdfstring{\texttt{\ inset\ }}{ inset }}\label{parameters-inset}

\href{/docs/reference/layout/relative/}{relative} {or}
\href{/docs/reference/foundations/array/}{array} {or}
\href{/docs/reference/foundations/dictionary/}{dictionary} {or}
\href{/docs/reference/foundations/function/}{function}

{{ Settable }}

\phantomsection\label{parameters-inset-settable-tooltip}
Settable parameters can be customized for all following uses of the
function with a \texttt{\ set\ } rule.

How much to pad the cells\textquotesingle{} content.

You can find an example for this argument at the
\href{/docs/reference/model/table/\#parameters-inset}{\texttt{\ table.inset\ }}
parameter.

Default:
\texttt{\ }{\texttt{\ (\ }}\texttt{\ }{\texttt{\ :\ }}\texttt{\ }{\texttt{\ )\ }}\texttt{\ }

\subsubsection{\texorpdfstring{\texttt{\ children\ }}{ children }}\label{parameters-children}

\href{/docs/reference/foundations/content/}{content}

{Required} {{ Positional }}

\phantomsection\label{parameters-children-positional-tooltip}
Positional parameters are specified in order, without names.

{{ Variadic }}

\phantomsection\label{parameters-children-variadic-tooltip}
Variadic parameters can be specified multiple times.

The contents of the grid cells, plus any extra grid lines specified with
the
\href{/docs/reference/layout/grid/\#definitions-hline}{\texttt{\ grid.hline\ }}
and
\href{/docs/reference/layout/grid/\#definitions-vline}{\texttt{\ grid.vline\ }}
elements.

The cells are populated in row-major order.

\subsection{\texorpdfstring{{ Definitions
}}{ Definitions }}\label{definitions}

\phantomsection\label{definitions-tooltip}
Functions and types and can have associated definitions. These are
accessed by specifying the function or type, followed by a period, and
then the definition\textquotesingle s name.

\subsubsection{\texorpdfstring{\texttt{\ cell\ } {{ Element
}}}{ cell   Element }}\label{definitions-cell}

\phantomsection\label{definitions-cell-element-tooltip}
Element functions can be customized with \texttt{\ set\ } and
\texttt{\ show\ } rules.

A cell in the grid. You can use this function in the argument list of a
grid to override grid style properties for an individual cell or
manually positioning it within the grid. You can also use this function
in show rules to apply certain styles to multiple cells at once.

For example, you can override the position and stroke for a single cell:

grid { . } { cell } (

{ \href{/docs/reference/foundations/content/}{content} , } {
\hyperref[definitions-cell-parameters-x]{x :}
\href{/docs/reference/foundations/auto/}{auto}
\href{/docs/reference/foundations/int/}{int} , } {
\hyperref[definitions-cell-parameters-y]{y :}
\href{/docs/reference/foundations/auto/}{auto}
\href{/docs/reference/foundations/int/}{int} , } {
\hyperref[definitions-cell-parameters-colspan]{colspan :}
\href{/docs/reference/foundations/int/}{int} , } {
\hyperref[definitions-cell-parameters-rowspan]{rowspan :}
\href{/docs/reference/foundations/int/}{int} , } {
\hyperref[definitions-cell-parameters-fill]{fill :}
\href{/docs/reference/foundations/none/}{none}
\href{/docs/reference/foundations/auto/}{auto}
\href{/docs/reference/visualize/color/}{color}
\href{/docs/reference/visualize/gradient/}{gradient}
\href{/docs/reference/visualize/pattern/}{pattern} , } {
\hyperref[definitions-cell-parameters-align]{align :}
\href{/docs/reference/foundations/auto/}{auto}
\href{/docs/reference/layout/alignment/}{alignment} , } {
\hyperref[definitions-cell-parameters-inset]{inset :}
\href{/docs/reference/foundations/auto/}{auto}
\href{/docs/reference/layout/relative/}{relative}
\href{/docs/reference/foundations/dictionary/}{dictionary} , } {
\hyperref[definitions-cell-parameters-stroke]{stroke :}
\href{/docs/reference/foundations/none/}{none}
\href{/docs/reference/layout/length/}{length}
\href{/docs/reference/visualize/color/}{color}
\href{/docs/reference/visualize/gradient/}{gradient}
\href{/docs/reference/visualize/stroke/}{stroke}
\href{/docs/reference/visualize/pattern/}{pattern}
\href{/docs/reference/foundations/dictionary/}{dictionary} , } {
\hyperref[definitions-cell-parameters-breakable]{breakable :}
\href{/docs/reference/foundations/auto/}{auto}
\href{/docs/reference/foundations/bool/}{bool} , }

) -\textgreater{} \href{/docs/reference/foundations/content/}{content}

\begin{verbatim}
#set text(15pt, font: "Noto Sans Symbols 2")
#show regex("[♚-♟︎]"): set text(fill: rgb("21212A"))
#show regex("[♔-♙]"): set text(fill: rgb("111015"))

#grid(
  fill: (x, y) => rgb(
    if calc.odd(x + y) { "7F8396" }
    else { "EFF0F3" }
  ),
  columns: (1em,) * 8,
  rows: 1em,
  align: center + horizon,

  [♖], [♘], [♗], [♕], [♔], [♗], [♘], [♖],
  [♙], [♙], [♙], [♙], [],  [♙], [♙], [♙],
  grid.cell(
    x: 4, y: 3,
    stroke: blue.transparentize(60%)
  )[♙],

  ..(grid.cell(y: 6)[♟],) * 8,
  ..([♜], [♞], [♝], [♛], [♚], [♝], [♞], [♜])
    .map(grid.cell.with(y: 7)),
)
\end{verbatim}

\includegraphics[width=3.125in,height=\textheight,keepaspectratio]{/assets/docs/hagMogxzgYo1z-9CqYbmiQAAAAAAAAAA.png}

You may also apply a show rule on \texttt{\ grid.cell\ } to style all
cells at once, which allows you, for example, to apply styles based on a
cell\textquotesingle s position. Refer to the examples of the
\href{/docs/reference/model/table/\#definitions-cell}{\texttt{\ table.cell\ }}
element to learn more about this.

\paragraph{\texorpdfstring{\texttt{\ body\ }}{ body }}\label{definitions-cell-body}

\href{/docs/reference/foundations/content/}{content}

{Required} {{ Positional }}

\phantomsection\label{definitions-cell-body-positional-tooltip}
Positional parameters are specified in order, without names.

The cell\textquotesingle s body.

\paragraph{\texorpdfstring{\texttt{\ x\ }}{ x }}\label{definitions-cell-x}

\href{/docs/reference/foundations/auto/}{auto} {or}
\href{/docs/reference/foundations/int/}{int}

{{ Settable }}

\phantomsection\label{definitions-cell-x-settable-tooltip}
Settable parameters can be customized for all following uses of the
function with a \texttt{\ set\ } rule.

The cell\textquotesingle s column (zero-indexed). This field may be used
in show rules to style a cell depending on its column.

You may override this field to pick in which column the cell must be
placed. If no row ( \texttt{\ y\ } ) is chosen, the cell will be placed
in the first row (starting at row 0) with that column available (or a
new row if none). If both \texttt{\ x\ } and \texttt{\ y\ } are chosen,
however, the cell will be placed in that exact position. An error is
raised if that position is not available (thus, it is usually wise to
specify cells with a custom position before cells with automatic
positions).

Default: \texttt{\ }{\texttt{\ auto\ }}\texttt{\ }

\includesvg[width=0.16667in,height=0.16667in]{/assets/icons/16-arrow-right.svg}
View example

\begin{verbatim}
#let circ(c) = circle(
    fill: c, width: 5mm
)

#grid(
  columns: 4,
  rows: 7mm,
  stroke: .5pt + blue,
  align: center + horizon,
  inset: 1mm,

  grid.cell(x: 2, y: 2, circ(aqua)),
  circ(yellow),
  grid.cell(x: 3, circ(green)),
  circ(black),
)
\end{verbatim}

\includegraphics[width=5in,height=\textheight,keepaspectratio]{/assets/docs/1ClWJM7tWFhsIyNZJlD1owAAAAAAAAAA.png}

\paragraph{\texorpdfstring{\texttt{\ y\ }}{ y }}\label{definitions-cell-y}

\href{/docs/reference/foundations/auto/}{auto} {or}
\href{/docs/reference/foundations/int/}{int}

{{ Settable }}

\phantomsection\label{definitions-cell-y-settable-tooltip}
Settable parameters can be customized for all following uses of the
function with a \texttt{\ set\ } rule.

The cell\textquotesingle s row (zero-indexed). This field may be used in
show rules to style a cell depending on its row.

You may override this field to pick in which row the cell must be
placed. If no column ( \texttt{\ x\ } ) is chosen, the cell will be
placed in the first column (starting at column 0) available in the
chosen row. If all columns in the chosen row are already occupied, an
error is raised.

Default: \texttt{\ }{\texttt{\ auto\ }}\texttt{\ }

\includesvg[width=0.16667in,height=0.16667in]{/assets/icons/16-arrow-right.svg}
View example

\begin{verbatim}
#let tri(c) = polygon.regular(
  fill: c,
  size: 5mm,
  vertices: 3,
)

#grid(
  columns: 2,
  stroke: blue,
  inset: 1mm,

  tri(black),
  grid.cell(y: 1, tri(teal)),
  grid.cell(y: 1, tri(red)),
  grid.cell(y: 2, tri(orange))
)
\end{verbatim}

\includegraphics[width=5in,height=\textheight,keepaspectratio]{/assets/docs/KqESjHcjVY-CskMVImXGSAAAAAAAAAAA.png}

\paragraph{\texorpdfstring{\texttt{\ colspan\ }}{ colspan }}\label{definitions-cell-colspan}

\href{/docs/reference/foundations/int/}{int}

{{ Settable }}

\phantomsection\label{definitions-cell-colspan-settable-tooltip}
Settable parameters can be customized for all following uses of the
function with a \texttt{\ set\ } rule.

The amount of columns spanned by this cell.

Default: \texttt{\ }{\texttt{\ 1\ }}\texttt{\ }

\paragraph{\texorpdfstring{\texttt{\ rowspan\ }}{ rowspan }}\label{definitions-cell-rowspan}

\href{/docs/reference/foundations/int/}{int}

{{ Settable }}

\phantomsection\label{definitions-cell-rowspan-settable-tooltip}
Settable parameters can be customized for all following uses of the
function with a \texttt{\ set\ } rule.

The amount of rows spanned by this cell.

Default: \texttt{\ }{\texttt{\ 1\ }}\texttt{\ }

\paragraph{\texorpdfstring{\texttt{\ fill\ }}{ fill }}\label{definitions-cell-fill}

\href{/docs/reference/foundations/none/}{none} {or}
\href{/docs/reference/foundations/auto/}{auto} {or}
\href{/docs/reference/visualize/color/}{color} {or}
\href{/docs/reference/visualize/gradient/}{gradient} {or}
\href{/docs/reference/visualize/pattern/}{pattern}

{{ Settable }}

\phantomsection\label{definitions-cell-fill-settable-tooltip}
Settable parameters can be customized for all following uses of the
function with a \texttt{\ set\ } rule.

The cell\textquotesingle s
\href{/docs/reference/layout/grid/\#parameters-fill}{fill} override.

Default: \texttt{\ }{\texttt{\ auto\ }}\texttt{\ }

\paragraph{\texorpdfstring{\texttt{\ align\ }}{ align }}\label{definitions-cell-align}

\href{/docs/reference/foundations/auto/}{auto} {or}
\href{/docs/reference/layout/alignment/}{alignment}

{{ Settable }}

\phantomsection\label{definitions-cell-align-settable-tooltip}
Settable parameters can be customized for all following uses of the
function with a \texttt{\ set\ } rule.

The cell\textquotesingle s
\href{/docs/reference/layout/grid/\#parameters-align}{alignment}
override.

Default: \texttt{\ }{\texttt{\ auto\ }}\texttt{\ }

\paragraph{\texorpdfstring{\texttt{\ inset\ }}{ inset }}\label{definitions-cell-inset}

\href{/docs/reference/foundations/auto/}{auto} {or}
\href{/docs/reference/layout/relative/}{relative} {or}
\href{/docs/reference/foundations/dictionary/}{dictionary}

{{ Settable }}

\phantomsection\label{definitions-cell-inset-settable-tooltip}
Settable parameters can be customized for all following uses of the
function with a \texttt{\ set\ } rule.

The cell\textquotesingle s
\href{/docs/reference/layout/grid/\#parameters-inset}{inset} override.

Default: \texttt{\ }{\texttt{\ auto\ }}\texttt{\ }

\paragraph{\texorpdfstring{\texttt{\ stroke\ }}{ stroke }}\label{definitions-cell-stroke}

\href{/docs/reference/foundations/none/}{none} {or}
\href{/docs/reference/layout/length/}{length} {or}
\href{/docs/reference/visualize/color/}{color} {or}
\href{/docs/reference/visualize/gradient/}{gradient} {or}
\href{/docs/reference/visualize/stroke/}{stroke} {or}
\href{/docs/reference/visualize/pattern/}{pattern} {or}
\href{/docs/reference/foundations/dictionary/}{dictionary}

{{ Settable }}

\phantomsection\label{definitions-cell-stroke-settable-tooltip}
Settable parameters can be customized for all following uses of the
function with a \texttt{\ set\ } rule.

The cell\textquotesingle s
\href{/docs/reference/layout/grid/\#parameters-stroke}{stroke} override.

Default:
\texttt{\ }{\texttt{\ (\ }}\texttt{\ }{\texttt{\ :\ }}\texttt{\ }{\texttt{\ )\ }}\texttt{\ }

\paragraph{\texorpdfstring{\texttt{\ breakable\ }}{ breakable }}\label{definitions-cell-breakable}

\href{/docs/reference/foundations/auto/}{auto} {or}
\href{/docs/reference/foundations/bool/}{bool}

{{ Settable }}

\phantomsection\label{definitions-cell-breakable-settable-tooltip}
Settable parameters can be customized for all following uses of the
function with a \texttt{\ set\ } rule.

Whether rows spanned by this cell can be placed in different pages. When
equal to \texttt{\ }{\texttt{\ auto\ }}\texttt{\ } , a cell spanning
only fixed-size rows is unbreakable, while a cell spanning at least one
\texttt{\ }{\texttt{\ auto\ }}\texttt{\ } -sized row is breakable.

Default: \texttt{\ }{\texttt{\ auto\ }}\texttt{\ }

\subsubsection{\texorpdfstring{\texttt{\ hline\ } {{ Element
}}}{ hline   Element }}\label{definitions-hline}

\phantomsection\label{definitions-hline-element-tooltip}
Element functions can be customized with \texttt{\ set\ } and
\texttt{\ show\ } rules.

A horizontal line in the grid.

Overrides any per-cell stroke, including stroke specified through the
grid\textquotesingle s \texttt{\ stroke\ } field. Can cross spacing
between cells created through the grid\textquotesingle s
\texttt{\ column-gutter\ } option.

An example for this function can be found at the
\href{/docs/reference/model/table/\#definitions-hline}{\texttt{\ table.hline\ }}
element.

grid { . } { hline } (

{ \hyperref[definitions-hline-parameters-y]{y :}
\href{/docs/reference/foundations/auto/}{auto}
\href{/docs/reference/foundations/int/}{int} , } {
\hyperref[definitions-hline-parameters-start]{start :}
\href{/docs/reference/foundations/int/}{int} , } {
\hyperref[definitions-hline-parameters-end]{end :}
\href{/docs/reference/foundations/none/}{none}
\href{/docs/reference/foundations/int/}{int} , } {
\hyperref[definitions-hline-parameters-stroke]{stroke :}
\href{/docs/reference/foundations/none/}{none}
\href{/docs/reference/layout/length/}{length}
\href{/docs/reference/visualize/color/}{color}
\href{/docs/reference/visualize/gradient/}{gradient}
\href{/docs/reference/visualize/stroke/}{stroke}
\href{/docs/reference/visualize/pattern/}{pattern}
\href{/docs/reference/foundations/dictionary/}{dictionary} , } {
\hyperref[definitions-hline-parameters-position]{position :}
\href{/docs/reference/layout/alignment/}{alignment} , }

) -\textgreater{} \href{/docs/reference/foundations/content/}{content}

\paragraph{\texorpdfstring{\texttt{\ y\ }}{ y }}\label{definitions-hline-y}

\href{/docs/reference/foundations/auto/}{auto} {or}
\href{/docs/reference/foundations/int/}{int}

{{ Settable }}

\phantomsection\label{definitions-hline-y-settable-tooltip}
Settable parameters can be customized for all following uses of the
function with a \texttt{\ set\ } rule.

The row above which the horizontal line is placed (zero-indexed). If the
\texttt{\ position\ } field is set to \texttt{\ bottom\ } , the line is
placed below the row with the given index instead (see that
field\textquotesingle s docs for details).

Specifying \texttt{\ }{\texttt{\ auto\ }}\texttt{\ } causes the line to
be placed at the row below the last automatically positioned cell (that
is, cell without coordinate overrides) before the line among the
grid\textquotesingle s children. If there is no such cell before the
line, it is placed at the top of the grid (row 0). Note that specifying
for this option exactly the total amount of rows in the grid causes this
horizontal line to override the bottom border of the grid, while a value
of 0 overrides the top border.

Default: \texttt{\ }{\texttt{\ auto\ }}\texttt{\ }

\paragraph{\texorpdfstring{\texttt{\ start\ }}{ start }}\label{definitions-hline-start}

\href{/docs/reference/foundations/int/}{int}

{{ Settable }}

\phantomsection\label{definitions-hline-start-settable-tooltip}
Settable parameters can be customized for all following uses of the
function with a \texttt{\ set\ } rule.

The column at which the horizontal line starts (zero-indexed,
inclusive).

Default: \texttt{\ }{\texttt{\ 0\ }}\texttt{\ }

\paragraph{\texorpdfstring{\texttt{\ end\ }}{ end }}\label{definitions-hline-end}

\href{/docs/reference/foundations/none/}{none} {or}
\href{/docs/reference/foundations/int/}{int}

{{ Settable }}

\phantomsection\label{definitions-hline-end-settable-tooltip}
Settable parameters can be customized for all following uses of the
function with a \texttt{\ set\ } rule.

The column before which the horizontal line ends (zero-indexed,
exclusive). Therefore, the horizontal line will be drawn up to and
across column \texttt{\ end\ -\ 1\ } .

A value equal to \texttt{\ }{\texttt{\ none\ }}\texttt{\ } or to the
amount of columns causes it to extend all the way towards the end of the
grid.

Default: \texttt{\ }{\texttt{\ none\ }}\texttt{\ }

\paragraph{\texorpdfstring{\texttt{\ stroke\ }}{ stroke }}\label{definitions-hline-stroke}

\href{/docs/reference/foundations/none/}{none} {or}
\href{/docs/reference/layout/length/}{length} {or}
\href{/docs/reference/visualize/color/}{color} {or}
\href{/docs/reference/visualize/gradient/}{gradient} {or}
\href{/docs/reference/visualize/stroke/}{stroke} {or}
\href{/docs/reference/visualize/pattern/}{pattern} {or}
\href{/docs/reference/foundations/dictionary/}{dictionary}

{{ Settable }}

\phantomsection\label{definitions-hline-stroke-settable-tooltip}
Settable parameters can be customized for all following uses of the
function with a \texttt{\ set\ } rule.

The line\textquotesingle s stroke.

Specifying \texttt{\ }{\texttt{\ none\ }}\texttt{\ } removes any lines
previously placed across this line\textquotesingle s range, including
hlines or per-cell stroke below it.

Default:
\texttt{\ }{\texttt{\ 1pt\ }}\texttt{\ }{\texttt{\ +\ }}\texttt{\ black\ }

\paragraph{\texorpdfstring{\texttt{\ position\ }}{ position }}\label{definitions-hline-position}

\href{/docs/reference/layout/alignment/}{alignment}

{{ Settable }}

\phantomsection\label{definitions-hline-position-settable-tooltip}
Settable parameters can be customized for all following uses of the
function with a \texttt{\ set\ } rule.

The position at which the line is placed, given its row ( \texttt{\ y\ }
) - either \texttt{\ top\ } to draw above it or \texttt{\ bottom\ } to
draw below it.

This setting is only relevant when row gutter is enabled (and
shouldn\textquotesingle t be used otherwise - prefer just increasing the
\texttt{\ y\ } field by one instead), since then the position below a
row becomes different from the position above the next row due to the
spacing between both.

Default: \texttt{\ top\ }

\subsubsection{\texorpdfstring{\texttt{\ vline\ } {{ Element
}}}{ vline   Element }}\label{definitions-vline}

\phantomsection\label{definitions-vline-element-tooltip}
Element functions can be customized with \texttt{\ set\ } and
\texttt{\ show\ } rules.

A vertical line in the grid.

Overrides any per-cell stroke, including stroke specified through the
grid\textquotesingle s \texttt{\ stroke\ } field. Can cross spacing
between cells created through the grid\textquotesingle s
\texttt{\ row-gutter\ } option.

grid { . } { vline } (

{ \hyperref[definitions-vline-parameters-x]{x :}
\href{/docs/reference/foundations/auto/}{auto}
\href{/docs/reference/foundations/int/}{int} , } {
\hyperref[definitions-vline-parameters-start]{start :}
\href{/docs/reference/foundations/int/}{int} , } {
\hyperref[definitions-vline-parameters-end]{end :}
\href{/docs/reference/foundations/none/}{none}
\href{/docs/reference/foundations/int/}{int} , } {
\hyperref[definitions-vline-parameters-stroke]{stroke :}
\href{/docs/reference/foundations/none/}{none}
\href{/docs/reference/layout/length/}{length}
\href{/docs/reference/visualize/color/}{color}
\href{/docs/reference/visualize/gradient/}{gradient}
\href{/docs/reference/visualize/stroke/}{stroke}
\href{/docs/reference/visualize/pattern/}{pattern}
\href{/docs/reference/foundations/dictionary/}{dictionary} , } {
\hyperref[definitions-vline-parameters-position]{position :}
\href{/docs/reference/layout/alignment/}{alignment} , }

) -\textgreater{} \href{/docs/reference/foundations/content/}{content}

\paragraph{\texorpdfstring{\texttt{\ x\ }}{ x }}\label{definitions-vline-x}

\href{/docs/reference/foundations/auto/}{auto} {or}
\href{/docs/reference/foundations/int/}{int}

{{ Settable }}

\phantomsection\label{definitions-vline-x-settable-tooltip}
Settable parameters can be customized for all following uses of the
function with a \texttt{\ set\ } rule.

The column before which the horizontal line is placed (zero-indexed). If
the \texttt{\ position\ } field is set to \texttt{\ end\ } , the line is
placed after the column with the given index instead (see that
field\textquotesingle s docs for details).

Specifying \texttt{\ }{\texttt{\ auto\ }}\texttt{\ } causes the line to
be placed at the column after the last automatically positioned cell
(that is, cell without coordinate overrides) before the line among the
grid\textquotesingle s children. If there is no such cell before the
line, it is placed before the grid\textquotesingle s first column
(column 0). Note that specifying for this option exactly the total
amount of columns in the grid causes this vertical line to override the
end border of the grid (right in LTR, left in RTL), while a value of 0
overrides the start border (left in LTR, right in RTL).

Default: \texttt{\ }{\texttt{\ auto\ }}\texttt{\ }

\paragraph{\texorpdfstring{\texttt{\ start\ }}{ start }}\label{definitions-vline-start}

\href{/docs/reference/foundations/int/}{int}

{{ Settable }}

\phantomsection\label{definitions-vline-start-settable-tooltip}
Settable parameters can be customized for all following uses of the
function with a \texttt{\ set\ } rule.

The row at which the vertical line starts (zero-indexed, inclusive).

Default: \texttt{\ }{\texttt{\ 0\ }}\texttt{\ }

\paragraph{\texorpdfstring{\texttt{\ end\ }}{ end }}\label{definitions-vline-end}

\href{/docs/reference/foundations/none/}{none} {or}
\href{/docs/reference/foundations/int/}{int}

{{ Settable }}

\phantomsection\label{definitions-vline-end-settable-tooltip}
Settable parameters can be customized for all following uses of the
function with a \texttt{\ set\ } rule.

The row on top of which the vertical line ends (zero-indexed,
exclusive). Therefore, the vertical line will be drawn up to and across
row \texttt{\ end\ -\ 1\ } .

A value equal to \texttt{\ }{\texttt{\ none\ }}\texttt{\ } or to the
amount of rows causes it to extend all the way towards the bottom of the
grid.

Default: \texttt{\ }{\texttt{\ none\ }}\texttt{\ }

\paragraph{\texorpdfstring{\texttt{\ stroke\ }}{ stroke }}\label{definitions-vline-stroke}

\href{/docs/reference/foundations/none/}{none} {or}
\href{/docs/reference/layout/length/}{length} {or}
\href{/docs/reference/visualize/color/}{color} {or}
\href{/docs/reference/visualize/gradient/}{gradient} {or}
\href{/docs/reference/visualize/stroke/}{stroke} {or}
\href{/docs/reference/visualize/pattern/}{pattern} {or}
\href{/docs/reference/foundations/dictionary/}{dictionary}

{{ Settable }}

\phantomsection\label{definitions-vline-stroke-settable-tooltip}
Settable parameters can be customized for all following uses of the
function with a \texttt{\ set\ } rule.

The line\textquotesingle s stroke.

Specifying \texttt{\ }{\texttt{\ none\ }}\texttt{\ } removes any lines
previously placed across this line\textquotesingle s range, including
vlines or per-cell stroke below it.

Default:
\texttt{\ }{\texttt{\ 1pt\ }}\texttt{\ }{\texttt{\ +\ }}\texttt{\ black\ }

\paragraph{\texorpdfstring{\texttt{\ position\ }}{ position }}\label{definitions-vline-position}

\href{/docs/reference/layout/alignment/}{alignment}

{{ Settable }}

\phantomsection\label{definitions-vline-position-settable-tooltip}
Settable parameters can be customized for all following uses of the
function with a \texttt{\ set\ } rule.

The position at which the line is placed, given its column (
\texttt{\ x\ } ) - either \texttt{\ start\ } to draw before it or
\texttt{\ end\ } to draw after it.

The values \texttt{\ left\ } and \texttt{\ right\ } are also accepted,
but discouraged as they cause your grid to be inconsistent between
left-to-right and right-to-left documents.

This setting is only relevant when column gutter is enabled (and
shouldn\textquotesingle t be used otherwise - prefer just increasing the
\texttt{\ x\ } field by one instead), since then the position after a
column becomes different from the position before the next column due to
the spacing between both.

Default: \texttt{\ start\ }

\subsubsection{\texorpdfstring{\texttt{\ header\ } {{ Element
}}}{ header   Element }}\label{definitions-header}

\phantomsection\label{definitions-header-element-tooltip}
Element functions can be customized with \texttt{\ set\ } and
\texttt{\ show\ } rules.

A repeatable grid header.

If \texttt{\ repeat\ } is set to \texttt{\ true\ } , the header will be
repeated across pages. For an example, refer to the
\href{/docs/reference/model/table/\#definitions-header}{\texttt{\ table.header\ }}
element and the
\href{/docs/reference/layout/grid/\#parameters-stroke}{\texttt{\ grid.stroke\ }}
parameter.

grid { . } { header } (

{ \hyperref[definitions-header-parameters-repeat]{repeat :}
\href{/docs/reference/foundations/bool/}{bool} , } {
\hyperref[definitions-header-parameters-children]{..}
\href{/docs/reference/foundations/content/}{content} , }

) -\textgreater{} \href{/docs/reference/foundations/content/}{content}

\paragraph{\texorpdfstring{\texttt{\ repeat\ }}{ repeat }}\label{definitions-header-repeat}

\href{/docs/reference/foundations/bool/}{bool}

{{ Settable }}

\phantomsection\label{definitions-header-repeat-settable-tooltip}
Settable parameters can be customized for all following uses of the
function with a \texttt{\ set\ } rule.

Whether this header should be repeated across pages.

Default: \texttt{\ }{\texttt{\ true\ }}\texttt{\ }

\paragraph{\texorpdfstring{\texttt{\ children\ }}{ children }}\label{definitions-header-children}

\href{/docs/reference/foundations/content/}{content}

{Required} {{ Positional }}

\phantomsection\label{definitions-header-children-positional-tooltip}
Positional parameters are specified in order, without names.

{{ Variadic }}

\phantomsection\label{definitions-header-children-variadic-tooltip}
Variadic parameters can be specified multiple times.

The cells and lines within the header.

\subsubsection{\texorpdfstring{\texttt{\ footer\ } {{ Element
}}}{ footer   Element }}\label{definitions-footer}

\phantomsection\label{definitions-footer-element-tooltip}
Element functions can be customized with \texttt{\ set\ } and
\texttt{\ show\ } rules.

A repeatable grid footer.

Just like the
\href{/docs/reference/layout/grid/\#definitions-header}{\texttt{\ grid.header\ }}
element, the footer can repeat itself on every page of the table.

No other grid cells may be placed after the footer.

grid { . } { footer } (

{ \hyperref[definitions-footer-parameters-repeat]{repeat :}
\href{/docs/reference/foundations/bool/}{bool} , } {
\hyperref[definitions-footer-parameters-children]{..}
\href{/docs/reference/foundations/content/}{content} , }

) -\textgreater{} \href{/docs/reference/foundations/content/}{content}

\paragraph{\texorpdfstring{\texttt{\ repeat\ }}{ repeat }}\label{definitions-footer-repeat}

\href{/docs/reference/foundations/bool/}{bool}

{{ Settable }}

\phantomsection\label{definitions-footer-repeat-settable-tooltip}
Settable parameters can be customized for all following uses of the
function with a \texttt{\ set\ } rule.

Whether this footer should be repeated across pages.

Default: \texttt{\ }{\texttt{\ true\ }}\texttt{\ }

\paragraph{\texorpdfstring{\texttt{\ children\ }}{ children }}\label{definitions-footer-children}

\href{/docs/reference/foundations/content/}{content}

{Required} {{ Positional }}

\phantomsection\label{definitions-footer-children-positional-tooltip}
Positional parameters are specified in order, without names.

{{ Variadic }}

\phantomsection\label{definitions-footer-children-variadic-tooltip}
Variadic parameters can be specified multiple times.

The cells and lines within the footer.

\href{/docs/reference/layout/fraction/}{\pandocbounded{\includesvg[keepaspectratio]{/assets/icons/16-arrow-right.svg}}}

{ Fraction } { Previous page }

\href{/docs/reference/layout/hide/}{\pandocbounded{\includesvg[keepaspectratio]{/assets/icons/16-arrow-right.svg}}}

{ Hide } { Next page }


\title{typst.app/docs/reference/layout/columns}

\begin{itemize}
\tightlist
\item
  \href{/docs}{\includesvg[width=0.16667in,height=0.16667in]{/assets/icons/16-docs-dark.svg}}
\item
  \includesvg[width=0.16667in,height=0.16667in]{/assets/icons/16-arrow-right.svg}
\item
  \href{/docs/reference/}{Reference}
\item
  \includesvg[width=0.16667in,height=0.16667in]{/assets/icons/16-arrow-right.svg}
\item
  \href{/docs/reference/layout/}{Layout}
\item
  \includesvg[width=0.16667in,height=0.16667in]{/assets/icons/16-arrow-right.svg}
\item
  \href{/docs/reference/layout/columns/}{Columns}
\end{itemize}

\section{\texorpdfstring{\texttt{\ columns\ } {{ Element
}}}{ columns   Element }}\label{summary}

\phantomsection\label{element-tooltip}
Element functions can be customized with \texttt{\ set\ } and
\texttt{\ show\ } rules.

Separates a region into multiple equally sized columns.

The \texttt{\ column\ } function lets you separate the interior of any
container into multiple columns. It will currently not balance the
height of the columns. Instead, the columns will take up the height of
their container or the remaining height on the page. Support for
balanced columns is planned for the future.

\subsection{Page-level columns}\label{page-level}

If you need to insert columns across your whole document, use the
\texttt{\ page\ } function\textquotesingle s
\href{/docs/reference/layout/page/\#parameters-columns}{\texttt{\ columns\ }
parameter} instead. This will create the columns directly at the
page-level rather than wrapping all of your content in a layout
container. As a result, things like
\href{/docs/reference/layout/pagebreak/}{pagebreaks} ,
\href{/docs/reference/model/footnote/}{footnotes} , and
\href{/docs/reference/model/par/\#definitions-line}{line numbers} will
continue to work as expected. For more information, also read the
\href{/docs/guides/page-setup-guide/\#columns}{relevant part of the page
setup guide} .

\subsection{Breaking out of columns}\label{breaking-out}

To temporarily break out of columns (e.g. for a paper\textquotesingle s
title), use parent-scoped floating placement:

\begin{verbatim}
#set page(columns: 2, height: 150pt)

#place(
  top + center,
  scope: "parent",
  float: true,
  text(1.4em, weight: "bold")[
    My document
  ],
)

#lorem(40)
\end{verbatim}

\includegraphics[width=5in,height=\textheight,keepaspectratio]{/assets/docs/qNRmHdtNgs8qpE-RUR-XyQAAAAAAAAAA.png}

\subsection{\texorpdfstring{{ Parameters
}}{ Parameters }}\label{parameters}

\phantomsection\label{parameters-tooltip}
Parameters are the inputs to a function. They are specified in
parentheses after the function name.

{ columns } (

{ \hyperref[parameters-count]{}
\href{/docs/reference/foundations/int/}{int} , } {
\hyperref[parameters-gutter]{gutter :}
\href{/docs/reference/layout/relative/}{relative} , } {
\href{/docs/reference/foundations/content/}{content} , }

) -\textgreater{} \href{/docs/reference/foundations/content/}{content}

\subsubsection{\texorpdfstring{\texttt{\ count\ }}{ count }}\label{parameters-count}

\href{/docs/reference/foundations/int/}{int}

{{ Positional }}

\phantomsection\label{parameters-count-positional-tooltip}
Positional parameters are specified in order, without names.

{{ Settable }}

\phantomsection\label{parameters-count-settable-tooltip}
Settable parameters can be customized for all following uses of the
function with a \texttt{\ set\ } rule.

The number of columns.

Default: \texttt{\ }{\texttt{\ 2\ }}\texttt{\ }

\subsubsection{\texorpdfstring{\texttt{\ gutter\ }}{ gutter }}\label{parameters-gutter}

\href{/docs/reference/layout/relative/}{relative}

{{ Settable }}

\phantomsection\label{parameters-gutter-settable-tooltip}
Settable parameters can be customized for all following uses of the
function with a \texttt{\ set\ } rule.

The size of the gutter space between each column.

Default:
\texttt{\ }{\texttt{\ 4\%\ }}\texttt{\ }{\texttt{\ +\ }}\texttt{\ }{\texttt{\ 0pt\ }}\texttt{\ }

\subsubsection{\texorpdfstring{\texttt{\ body\ }}{ body }}\label{parameters-body}

\href{/docs/reference/foundations/content/}{content}

{Required} {{ Positional }}

\phantomsection\label{parameters-body-positional-tooltip}
Positional parameters are specified in order, without names.

The content that should be layouted into the columns.

\href{/docs/reference/layout/colbreak/}{\pandocbounded{\includesvg[keepaspectratio]{/assets/icons/16-arrow-right.svg}}}

{ Column Break } { Previous page }

\href{/docs/reference/layout/direction/}{\pandocbounded{\includesvg[keepaspectratio]{/assets/icons/16-arrow-right.svg}}}

{ Direction } { Next page }


\title{typst.app/docs/reference/layout/move}

\begin{itemize}
\tightlist
\item
  \href{/docs}{\includesvg[width=0.16667in,height=0.16667in]{/assets/icons/16-docs-dark.svg}}
\item
  \includesvg[width=0.16667in,height=0.16667in]{/assets/icons/16-arrow-right.svg}
\item
  \href{/docs/reference/}{Reference}
\item
  \includesvg[width=0.16667in,height=0.16667in]{/assets/icons/16-arrow-right.svg}
\item
  \href{/docs/reference/layout/}{Layout}
\item
  \includesvg[width=0.16667in,height=0.16667in]{/assets/icons/16-arrow-right.svg}
\item
  \href{/docs/reference/layout/move/}{Move}
\end{itemize}

\section{\texorpdfstring{\texttt{\ move\ } {{ Element
}}}{ move   Element }}\label{summary}

\phantomsection\label{element-tooltip}
Element functions can be customized with \texttt{\ set\ } and
\texttt{\ show\ } rules.

Moves content without affecting layout.

The \texttt{\ move\ } function allows you to move content while the
layout still \textquotesingle sees\textquotesingle{} it at the original
positions. Containers will still be sized as if the content was not
moved.

\subsection{Example}\label{example}

\begin{verbatim}
#rect(inset: 0pt, move(
  dx: 6pt, dy: 6pt,
  rect(
    inset: 8pt,
    fill: white,
    stroke: black,
    [Abra cadabra]
  )
))
\end{verbatim}

\includegraphics[width=5in,height=\textheight,keepaspectratio]{/assets/docs/1MdBh-uXG6kGRG6DYdlcJAAAAAAAAAAA.png}

\subsection{\texorpdfstring{{ Parameters
}}{ Parameters }}\label{parameters}

\phantomsection\label{parameters-tooltip}
Parameters are the inputs to a function. They are specified in
parentheses after the function name.

{ move } (

{ \hyperref[parameters-dx]{dx :}
\href{/docs/reference/layout/relative/}{relative} , } {
\hyperref[parameters-dy]{dy :}
\href{/docs/reference/layout/relative/}{relative} , } {
\href{/docs/reference/foundations/content/}{content} , }

) -\textgreater{} \href{/docs/reference/foundations/content/}{content}

\subsubsection{\texorpdfstring{\texttt{\ dx\ }}{ dx }}\label{parameters-dx}

\href{/docs/reference/layout/relative/}{relative}

{{ Settable }}

\phantomsection\label{parameters-dx-settable-tooltip}
Settable parameters can be customized for all following uses of the
function with a \texttt{\ set\ } rule.

The horizontal displacement of the content.

Default:
\texttt{\ }{\texttt{\ 0\%\ }}\texttt{\ }{\texttt{\ +\ }}\texttt{\ }{\texttt{\ 0pt\ }}\texttt{\ }

\subsubsection{\texorpdfstring{\texttt{\ dy\ }}{ dy }}\label{parameters-dy}

\href{/docs/reference/layout/relative/}{relative}

{{ Settable }}

\phantomsection\label{parameters-dy-settable-tooltip}
Settable parameters can be customized for all following uses of the
function with a \texttt{\ set\ } rule.

The vertical displacement of the content.

Default:
\texttt{\ }{\texttt{\ 0\%\ }}\texttt{\ }{\texttt{\ +\ }}\texttt{\ }{\texttt{\ 0pt\ }}\texttt{\ }

\subsubsection{\texorpdfstring{\texttt{\ body\ }}{ body }}\label{parameters-body}

\href{/docs/reference/foundations/content/}{content}

{Required} {{ Positional }}

\phantomsection\label{parameters-body-positional-tooltip}
Positional parameters are specified in order, without names.

The content to move.

\href{/docs/reference/layout/measure/}{\pandocbounded{\includesvg[keepaspectratio]{/assets/icons/16-arrow-right.svg}}}

{ Measure } { Previous page }

\href{/docs/reference/layout/pad/}{\pandocbounded{\includesvg[keepaspectratio]{/assets/icons/16-arrow-right.svg}}}

{ Padding } { Next page }


\title{typst.app/docs/reference/layout/skew}

\begin{itemize}
\tightlist
\item
  \href{/docs}{\includesvg[width=0.16667in,height=0.16667in]{/assets/icons/16-docs-dark.svg}}
\item
  \includesvg[width=0.16667in,height=0.16667in]{/assets/icons/16-arrow-right.svg}
\item
  \href{/docs/reference/}{Reference}
\item
  \includesvg[width=0.16667in,height=0.16667in]{/assets/icons/16-arrow-right.svg}
\item
  \href{/docs/reference/layout/}{Layout}
\item
  \includesvg[width=0.16667in,height=0.16667in]{/assets/icons/16-arrow-right.svg}
\item
  \href{/docs/reference/layout/skew/}{Skew}
\end{itemize}

\section{\texorpdfstring{\texttt{\ skew\ } {{ Element
}}}{ skew   Element }}\label{summary}

\phantomsection\label{element-tooltip}
Element functions can be customized with \texttt{\ set\ } and
\texttt{\ show\ } rules.

Skews content.

Skews an element in horizontal and/or vertical direction. The layout
will act as if the element was not skewed unless you specify
\texttt{\ reflow:\ }{\texttt{\ true\ }}\texttt{\ } .

\subsection{Example}\label{example}

\begin{verbatim}
#skew(ax: -12deg)[
  This is some fake italic text.
]
\end{verbatim}

\includegraphics[width=5in,height=\textheight,keepaspectratio]{/assets/docs/FUtSyVs-Ma5rvUP8B0w5fQAAAAAAAAAA.png}

\subsection{\texorpdfstring{{ Parameters
}}{ Parameters }}\label{parameters}

\phantomsection\label{parameters-tooltip}
Parameters are the inputs to a function. They are specified in
parentheses after the function name.

{ skew } (

{ \hyperref[parameters-ax]{ax :}
\href{/docs/reference/layout/angle/}{angle} , } {
\hyperref[parameters-ay]{ay :}
\href{/docs/reference/layout/angle/}{angle} , } {
\hyperref[parameters-origin]{origin :}
\href{/docs/reference/layout/alignment/}{alignment} , } {
\hyperref[parameters-reflow]{reflow :}
\href{/docs/reference/foundations/bool/}{bool} , } {
\href{/docs/reference/foundations/content/}{content} , }

) -\textgreater{} \href{/docs/reference/foundations/content/}{content}

\subsubsection{\texorpdfstring{\texttt{\ ax\ }}{ ax }}\label{parameters-ax}

\href{/docs/reference/layout/angle/}{angle}

{{ Settable }}

\phantomsection\label{parameters-ax-settable-tooltip}
Settable parameters can be customized for all following uses of the
function with a \texttt{\ set\ } rule.

The horizontal skewing angle.

Default: \texttt{\ }{\texttt{\ 0deg\ }}\texttt{\ }

\includesvg[width=0.16667in,height=0.16667in]{/assets/icons/16-arrow-right.svg}
View example

\begin{verbatim}
#skew(ax: 30deg)[Skewed]
\end{verbatim}

\includegraphics[width=5in,height=\textheight,keepaspectratio]{/assets/docs/H9k2hlR_HYwp5MND40z3rgAAAAAAAAAA.png}

\subsubsection{\texorpdfstring{\texttt{\ ay\ }}{ ay }}\label{parameters-ay}

\href{/docs/reference/layout/angle/}{angle}

{{ Settable }}

\phantomsection\label{parameters-ay-settable-tooltip}
Settable parameters can be customized for all following uses of the
function with a \texttt{\ set\ } rule.

The vertical skewing angle.

Default: \texttt{\ }{\texttt{\ 0deg\ }}\texttt{\ }

\includesvg[width=0.16667in,height=0.16667in]{/assets/icons/16-arrow-right.svg}
View example

\begin{verbatim}
#skew(ay: 30deg)[Skewed]
\end{verbatim}

\includegraphics[width=5in,height=\textheight,keepaspectratio]{/assets/docs/DIs5kgGdkepXxpgHWt0vxAAAAAAAAAAA.png}

\subsubsection{\texorpdfstring{\texttt{\ origin\ }}{ origin }}\label{parameters-origin}

\href{/docs/reference/layout/alignment/}{alignment}

{{ Settable }}

\phantomsection\label{parameters-origin-settable-tooltip}
Settable parameters can be customized for all following uses of the
function with a \texttt{\ set\ } rule.

The origin of the skew transformation.

The origin will stay fixed during the operation.

Default: \texttt{\ center\ }{\texttt{\ +\ }}\texttt{\ horizon\ }

\includesvg[width=0.16667in,height=0.16667in]{/assets/icons/16-arrow-right.svg}
View example

\begin{verbatim}
X #box(skew(ax: -30deg, origin: center + horizon)[X]) X \
X #box(skew(ax: -30deg, origin: bottom + left)[X]) X \
X #box(skew(ax: -30deg, origin: top + right)[X]) X
\end{verbatim}

\includegraphics[width=5in,height=\textheight,keepaspectratio]{/assets/docs/2Hq4GFYS1tSqCnluz3jbcQAAAAAAAAAA.png}

\subsubsection{\texorpdfstring{\texttt{\ reflow\ }}{ reflow }}\label{parameters-reflow}

\href{/docs/reference/foundations/bool/}{bool}

{{ Settable }}

\phantomsection\label{parameters-reflow-settable-tooltip}
Settable parameters can be customized for all following uses of the
function with a \texttt{\ set\ } rule.

Whether the skew transformation impacts the layout.

If set to \texttt{\ }{\texttt{\ false\ }}\texttt{\ } , the skewed
content will retain the bounding box of the original content. If set to
\texttt{\ }{\texttt{\ true\ }}\texttt{\ } , the bounding box will take
the transformation of the content into account and adjust the layout
accordingly.

Default: \texttt{\ }{\texttt{\ false\ }}\texttt{\ }

\includesvg[width=0.16667in,height=0.16667in]{/assets/icons/16-arrow-right.svg}
View example

\begin{verbatim}
Hello #skew(ay: 30deg, reflow: true, "World")!
\end{verbatim}

\includegraphics[width=5in,height=\textheight,keepaspectratio]{/assets/docs/-k-PUuRezD-q6j7vk-xQWAAAAAAAAAAA.png}

\subsubsection{\texorpdfstring{\texttt{\ body\ }}{ body }}\label{parameters-body}

\href{/docs/reference/foundations/content/}{content}

{Required} {{ Positional }}

\phantomsection\label{parameters-body-positional-tooltip}
Positional parameters are specified in order, without names.

The content to skew.

\href{/docs/reference/layout/scale/}{\pandocbounded{\includesvg[keepaspectratio]{/assets/icons/16-arrow-right.svg}}}

{ Scale } { Previous page }

\href{/docs/reference/layout/h/}{\pandocbounded{\includesvg[keepaspectratio]{/assets/icons/16-arrow-right.svg}}}

{ Spacing (H) } { Next page }


\title{typst.app/docs/reference/layout/align}

\begin{itemize}
\tightlist
\item
  \href{/docs}{\includesvg[width=0.16667in,height=0.16667in]{/assets/icons/16-docs-dark.svg}}
\item
  \includesvg[width=0.16667in,height=0.16667in]{/assets/icons/16-arrow-right.svg}
\item
  \href{/docs/reference/}{Reference}
\item
  \includesvg[width=0.16667in,height=0.16667in]{/assets/icons/16-arrow-right.svg}
\item
  \href{/docs/reference/layout/}{Layout}
\item
  \includesvg[width=0.16667in,height=0.16667in]{/assets/icons/16-arrow-right.svg}
\item
  \href{/docs/reference/layout/align/}{Align}
\end{itemize}

\section{\texorpdfstring{\texttt{\ align\ } {{ Element
}}}{ align   Element }}\label{summary}

\phantomsection\label{element-tooltip}
Element functions can be customized with \texttt{\ set\ } and
\texttt{\ show\ } rules.

Aligns content horizontally and vertically.

\subsection{Example}\label{example}

Let\textquotesingle s start with centering our content horizontally:

\begin{verbatim}
#set page(height: 120pt)
#set align(center)

Centered text, a sight to see \
In perfect balance, visually \
Not left nor right, it stands alone \
A work of art, a visual throne
\end{verbatim}

\includegraphics[width=5in,height=\textheight,keepaspectratio]{/assets/docs/kcNIG-bYA8T9BUDnjCUJGgAAAAAAAAAA.png}

To center something vertically, use \emph{horizon} alignment:

\begin{verbatim}
#set page(height: 120pt)
#set align(horizon)

Vertically centered, \
the stage had entered, \
a new paragraph.
\end{verbatim}

\includegraphics[width=5in,height=\textheight,keepaspectratio]{/assets/docs/y9OO-MSDQIHWsGPc_6pNnAAAAAAAAAAA.png}

\subsection{Combining alignments}\label{combining-alignments}

You can combine two alignments with the \texttt{\ +\ } operator.
Let\textquotesingle s also only apply this to one piece of content by
using the function form instead of a set rule:

\begin{verbatim}
#set page(height: 120pt)
Though left in the beginning ...

#align(right + bottom)[
  ... they were right in the end, \
  and with addition had gotten, \
  the paragraph to the bottom!
]
\end{verbatim}

\includegraphics[width=5in,height=\textheight,keepaspectratio]{/assets/docs/gXaqAMYC8Licj_UCK0JSFgAAAAAAAAAA.png}

\subsection{Nested alignment}\label{nested-alignment}

You can use varying alignments for layout containers and the elements
within them. This way, you can create intricate layouts:

\begin{verbatim}
#align(center, block[
  #set align(left)
  Though centered together \
  alone \
  we \
  are \
  left.
])
\end{verbatim}

\includegraphics[width=5in,height=\textheight,keepaspectratio]{/assets/docs/B6Y-WWFtiUjCHNJ9B8R8vQAAAAAAAAAA.png}

\subsection{Alignment within the same
line}\label{alignment-within-the-same-line}

The \texttt{\ align\ } function performs block-level alignment and thus
always interrupts the current paragraph. To have different alignment for
parts of the same line, you should use
\href{/docs/reference/layout/h/}{fractional spacing} instead:

\begin{verbatim}
Start #h(1fr) End
\end{verbatim}

\includegraphics[width=5in,height=\textheight,keepaspectratio]{/assets/docs/jlafwbE2ZuISwJPQNRzA3gAAAAAAAAAA.png}

\subsection{\texorpdfstring{{ Parameters
}}{ Parameters }}\label{parameters}

\phantomsection\label{parameters-tooltip}
Parameters are the inputs to a function. They are specified in
parentheses after the function name.

{ align } (

{ \hyperref[parameters-alignment]{}
\href{/docs/reference/layout/alignment/}{alignment} , } {
\href{/docs/reference/foundations/content/}{content} , }

) -\textgreater{} \href{/docs/reference/foundations/content/}{content}

\subsubsection{\texorpdfstring{\texttt{\ alignment\ }}{ alignment }}\label{parameters-alignment}

\href{/docs/reference/layout/alignment/}{alignment}

{{ Positional }}

\phantomsection\label{parameters-alignment-positional-tooltip}
Positional parameters are specified in order, without names.

{{ Settable }}

\phantomsection\label{parameters-alignment-settable-tooltip}
Settable parameters can be customized for all following uses of the
function with a \texttt{\ set\ } rule.

The \href{/docs/reference/layout/alignment/}{alignment} along both axes.

Default: \texttt{\ start\ }{\texttt{\ +\ }}\texttt{\ top\ }

\includesvg[width=0.16667in,height=0.16667in]{/assets/icons/16-arrow-right.svg}
View example

\begin{verbatim}
#set page(height: 6cm)
#set text(lang: "ar")

مثال
#align(
  end + horizon,
  rect(inset: 12pt)[ركن]
)
\end{verbatim}

\includegraphics[width=5in,height=\textheight,keepaspectratio]{/assets/docs/3176vm6IE_BNfZrVpc9_xAAAAAAAAAAA.png}

\subsubsection{\texorpdfstring{\texttt{\ body\ }}{ body }}\label{parameters-body}

\href{/docs/reference/foundations/content/}{content}

{Required} {{ Positional }}

\phantomsection\label{parameters-body-positional-tooltip}
Positional parameters are specified in order, without names.

The content to align.

\href{/docs/reference/layout/}{\pandocbounded{\includesvg[keepaspectratio]{/assets/icons/16-arrow-right.svg}}}

{ Layout } { Previous page }

\href{/docs/reference/layout/alignment/}{\pandocbounded{\includesvg[keepaspectratio]{/assets/icons/16-arrow-right.svg}}}

{ Alignment } { Next page }


\title{typst.app/docs/reference/layout/rotate}

\begin{itemize}
\tightlist
\item
  \href{/docs}{\includesvg[width=0.16667in,height=0.16667in]{/assets/icons/16-docs-dark.svg}}
\item
  \includesvg[width=0.16667in,height=0.16667in]{/assets/icons/16-arrow-right.svg}
\item
  \href{/docs/reference/}{Reference}
\item
  \includesvg[width=0.16667in,height=0.16667in]{/assets/icons/16-arrow-right.svg}
\item
  \href{/docs/reference/layout/}{Layout}
\item
  \includesvg[width=0.16667in,height=0.16667in]{/assets/icons/16-arrow-right.svg}
\item
  \href{/docs/reference/layout/rotate/}{Rotate}
\end{itemize}

\section{\texorpdfstring{\texttt{\ rotate\ } {{ Element
}}}{ rotate   Element }}\label{summary}

\phantomsection\label{element-tooltip}
Element functions can be customized with \texttt{\ set\ } and
\texttt{\ show\ } rules.

Rotates content without affecting layout.

Rotates an element by a given angle. The layout will act as if the
element was not rotated unless you specify
\texttt{\ reflow:\ }{\texttt{\ true\ }}\texttt{\ } .

\subsection{Example}\label{example}

\begin{verbatim}
#stack(
  dir: ltr,
  spacing: 1fr,
  ..range(16)
    .map(i => rotate(24deg * i)[X]),
)
\end{verbatim}

\includegraphics[width=5in,height=\textheight,keepaspectratio]{/assets/docs/KRNlJxFzPXxwMsKBe0vSFQAAAAAAAAAA.png}

\subsection{\texorpdfstring{{ Parameters
}}{ Parameters }}\label{parameters}

\phantomsection\label{parameters-tooltip}
Parameters are the inputs to a function. They are specified in
parentheses after the function name.

{ rotate } (

{ \hyperref[parameters-angle]{}
\href{/docs/reference/layout/angle/}{angle} , } {
\hyperref[parameters-origin]{origin :}
\href{/docs/reference/layout/alignment/}{alignment} , } {
\hyperref[parameters-reflow]{reflow :}
\href{/docs/reference/foundations/bool/}{bool} , } {
\href{/docs/reference/foundations/content/}{content} , }

) -\textgreater{} \href{/docs/reference/foundations/content/}{content}

\subsubsection{\texorpdfstring{\texttt{\ angle\ }}{ angle }}\label{parameters-angle}

\href{/docs/reference/layout/angle/}{angle}

{{ Positional }}

\phantomsection\label{parameters-angle-positional-tooltip}
Positional parameters are specified in order, without names.

{{ Settable }}

\phantomsection\label{parameters-angle-settable-tooltip}
Settable parameters can be customized for all following uses of the
function with a \texttt{\ set\ } rule.

The amount of rotation.

Default: \texttt{\ }{\texttt{\ 0deg\ }}\texttt{\ }

\includesvg[width=0.16667in,height=0.16667in]{/assets/icons/16-arrow-right.svg}
View example

\begin{verbatim}
#rotate(-1.571rad)[Space!]
\end{verbatim}

\includegraphics[width=5in,height=\textheight,keepaspectratio]{/assets/docs/_kx75fW11u8TY_Zj6luytwAAAAAAAAAA.png}

\subsubsection{\texorpdfstring{\texttt{\ origin\ }}{ origin }}\label{parameters-origin}

\href{/docs/reference/layout/alignment/}{alignment}

{{ Settable }}

\phantomsection\label{parameters-origin-settable-tooltip}
Settable parameters can be customized for all following uses of the
function with a \texttt{\ set\ } rule.

The origin of the rotation.

If, for instance, you wanted the bottom left corner of the rotated
element to stay aligned with the baseline, you would set it to
\texttt{\ bottom\ +\ left\ } instead.

Default: \texttt{\ center\ }{\texttt{\ +\ }}\texttt{\ horizon\ }

\includesvg[width=0.16667in,height=0.16667in]{/assets/icons/16-arrow-right.svg}
View example

\begin{verbatim}
#set text(spacing: 8pt)
#let square = square.with(width: 8pt)

#box(square())
#box(rotate(30deg, origin: center, square()))
#box(rotate(30deg, origin: top + left, square()))
#box(rotate(30deg, origin: bottom + right, square()))
\end{verbatim}

\includegraphics[width=5in,height=\textheight,keepaspectratio]{/assets/docs/ZzBCk0ymiIeT5xo4XXc-8QAAAAAAAAAA.png}

\subsubsection{\texorpdfstring{\texttt{\ reflow\ }}{ reflow }}\label{parameters-reflow}

\href{/docs/reference/foundations/bool/}{bool}

{{ Settable }}

\phantomsection\label{parameters-reflow-settable-tooltip}
Settable parameters can be customized for all following uses of the
function with a \texttt{\ set\ } rule.

Whether the rotation impacts the layout.

If set to \texttt{\ }{\texttt{\ false\ }}\texttt{\ } , the rotated
content will retain the bounding box of the original content. If set to
\texttt{\ }{\texttt{\ true\ }}\texttt{\ } , the bounding box will take
the rotation of the content into account and adjust the layout
accordingly.

Default: \texttt{\ }{\texttt{\ false\ }}\texttt{\ }

\includesvg[width=0.16667in,height=0.16667in]{/assets/icons/16-arrow-right.svg}
View example

\begin{verbatim}
Hello #rotate(90deg, reflow: true)[World]!
\end{verbatim}

\includegraphics[width=5in,height=\textheight,keepaspectratio]{/assets/docs/i8AMp2vxmKn3Nn0wwA1Z0wAAAAAAAAAA.png}

\subsubsection{\texorpdfstring{\texttt{\ body\ }}{ body }}\label{parameters-body}

\href{/docs/reference/foundations/content/}{content}

{Required} {{ Positional }}

\phantomsection\label{parameters-body-positional-tooltip}
Positional parameters are specified in order, without names.

The content to rotate.

\href{/docs/reference/layout/repeat/}{\pandocbounded{\includesvg[keepaspectratio]{/assets/icons/16-arrow-right.svg}}}

{ Repeat } { Previous page }

\href{/docs/reference/layout/scale/}{\pandocbounded{\includesvg[keepaspectratio]{/assets/icons/16-arrow-right.svg}}}

{ Scale } { Next page }


\title{typst.app/docs/reference/layout/angle}

\begin{itemize}
\tightlist
\item
  \href{/docs}{\includesvg[width=0.16667in,height=0.16667in]{/assets/icons/16-docs-dark.svg}}
\item
  \includesvg[width=0.16667in,height=0.16667in]{/assets/icons/16-arrow-right.svg}
\item
  \href{/docs/reference/}{Reference}
\item
  \includesvg[width=0.16667in,height=0.16667in]{/assets/icons/16-arrow-right.svg}
\item
  \href{/docs/reference/layout/}{Layout}
\item
  \includesvg[width=0.16667in,height=0.16667in]{/assets/icons/16-arrow-right.svg}
\item
  \href{/docs/reference/layout/angle/}{Angle}
\end{itemize}

\section{\texorpdfstring{{ angle }}{ angle }}\label{summary}

An angle describing a rotation.

Typst supports the following angular units:

\begin{itemize}
\tightlist
\item
  Degrees: \texttt{\ }{\texttt{\ 180deg\ }}\texttt{\ }
\item
  Radians: \texttt{\ }{\texttt{\ 3.14rad\ }}\texttt{\ }
\end{itemize}

\subsection{Example}\label{example}

\begin{verbatim}
#rotate(10deg)[Hello there!]
\end{verbatim}

\includegraphics[width=5in,height=\textheight,keepaspectratio]{/assets/docs/bDyrcLTzr2eRmGWeZRN2_QAAAAAAAAAA.png}

\subsection{\texorpdfstring{{ Definitions
}}{ Definitions }}\label{definitions}

\phantomsection\label{definitions-tooltip}
Functions and types and can have associated definitions. These are
accessed by specifying the function or type, followed by a period, and
then the definition\textquotesingle s name.

\subsubsection{\texorpdfstring{\texttt{\ rad\ }}{ rad }}\label{definitions-rad}

Converts this angle to radians.

self { . } { rad } (

) -\textgreater{} \href{/docs/reference/foundations/float/}{float}

\subsubsection{\texorpdfstring{\texttt{\ deg\ }}{ deg }}\label{definitions-deg}

Converts this angle to degrees.

self { . } { deg } (

) -\textgreater{} \href{/docs/reference/foundations/float/}{float}

\href{/docs/reference/layout/alignment/}{\pandocbounded{\includesvg[keepaspectratio]{/assets/icons/16-arrow-right.svg}}}

{ Alignment } { Previous page }

\href{/docs/reference/layout/block/}{\pandocbounded{\includesvg[keepaspectratio]{/assets/icons/16-arrow-right.svg}}}

{ Block } { Next page }


\title{typst.app/docs/reference/layout/colbreak}

\begin{itemize}
\tightlist
\item
  \href{/docs}{\includesvg[width=0.16667in,height=0.16667in]{/assets/icons/16-docs-dark.svg}}
\item
  \includesvg[width=0.16667in,height=0.16667in]{/assets/icons/16-arrow-right.svg}
\item
  \href{/docs/reference/}{Reference}
\item
  \includesvg[width=0.16667in,height=0.16667in]{/assets/icons/16-arrow-right.svg}
\item
  \href{/docs/reference/layout/}{Layout}
\item
  \includesvg[width=0.16667in,height=0.16667in]{/assets/icons/16-arrow-right.svg}
\item
  \href{/docs/reference/layout/colbreak/}{Column Break}
\end{itemize}

\section{\texorpdfstring{\texttt{\ colbreak\ } {{ Element
}}}{ colbreak   Element }}\label{summary}

\phantomsection\label{element-tooltip}
Element functions can be customized with \texttt{\ set\ } and
\texttt{\ show\ } rules.

Forces a column break.

The function will behave like a
\href{/docs/reference/layout/pagebreak/}{page break} when used in a
single column layout or the last column on a page. Otherwise, content
after the column break will be placed in the next column.

\subsection{Example}\label{example}

\begin{verbatim}
#set page(columns: 2)
Preliminary findings from our
ongoing research project have
revealed a hitherto unknown
phenomenon of extraordinary
significance.

#colbreak()
Through rigorous experimentation
and analysis, we have discovered
a hitherto uncharacterized process
that defies our current
understanding of the fundamental
laws of nature.
\end{verbatim}

\includegraphics[width=5in,height=\textheight,keepaspectratio]{/assets/docs/MXyldqpQM7MpLi9gC6sPGAAAAAAAAAAA.png}

\subsection{\texorpdfstring{{ Parameters
}}{ Parameters }}\label{parameters}

\phantomsection\label{parameters-tooltip}
Parameters are the inputs to a function. They are specified in
parentheses after the function name.

{ colbreak } (

{ \hyperref[parameters-weak]{weak :}
\href{/docs/reference/foundations/bool/}{bool} }

) -\textgreater{} \href{/docs/reference/foundations/content/}{content}

\subsubsection{\texorpdfstring{\texttt{\ weak\ }}{ weak }}\label{parameters-weak}

\href{/docs/reference/foundations/bool/}{bool}

{{ Settable }}

\phantomsection\label{parameters-weak-settable-tooltip}
Settable parameters can be customized for all following uses of the
function with a \texttt{\ set\ } rule.

If \texttt{\ }{\texttt{\ true\ }}\texttt{\ } , the column break is
skipped if the current column is already empty.

Default: \texttt{\ }{\texttt{\ false\ }}\texttt{\ }

\href{/docs/reference/layout/box/}{\pandocbounded{\includesvg[keepaspectratio]{/assets/icons/16-arrow-right.svg}}}

{ Box } { Previous page }

\href{/docs/reference/layout/columns/}{\pandocbounded{\includesvg[keepaspectratio]{/assets/icons/16-arrow-right.svg}}}

{ Columns } { Next page }


\title{typst.app/docs/reference/layout/scale}

\begin{itemize}
\tightlist
\item
  \href{/docs}{\includesvg[width=0.16667in,height=0.16667in]{/assets/icons/16-docs-dark.svg}}
\item
  \includesvg[width=0.16667in,height=0.16667in]{/assets/icons/16-arrow-right.svg}
\item
  \href{/docs/reference/}{Reference}
\item
  \includesvg[width=0.16667in,height=0.16667in]{/assets/icons/16-arrow-right.svg}
\item
  \href{/docs/reference/layout/}{Layout}
\item
  \includesvg[width=0.16667in,height=0.16667in]{/assets/icons/16-arrow-right.svg}
\item
  \href{/docs/reference/layout/scale/}{Scale}
\end{itemize}

\section{\texorpdfstring{\texttt{\ scale\ } {{ Element
}}}{ scale   Element }}\label{summary}

\phantomsection\label{element-tooltip}
Element functions can be customized with \texttt{\ set\ } and
\texttt{\ show\ } rules.

Scales content without affecting layout.

Lets you mirror content by specifying a negative scale on a single axis.

\subsection{Example}\label{example}

\begin{verbatim}
#set align(center)
#scale(x: -100%)[This is mirrored.]
#scale(x: -100%, reflow: true)[This is mirrored.]
\end{verbatim}

\includegraphics[width=5in,height=\textheight,keepaspectratio]{/assets/docs/ShH8NomqhuEYrrdUbApjaAAAAAAAAAAA.png}

\subsection{\texorpdfstring{{ Parameters
}}{ Parameters }}\label{parameters}

\phantomsection\label{parameters-tooltip}
Parameters are the inputs to a function. They are specified in
parentheses after the function name.

{ scale } (

{ \hyperref[parameters-factor]{}
\href{/docs/reference/foundations/auto/}{auto}
\href{/docs/reference/layout/length/}{length}
\href{/docs/reference/layout/ratio/}{ratio} , } {
\hyperref[parameters-x]{x :}
\href{/docs/reference/foundations/auto/}{auto}
\href{/docs/reference/layout/length/}{length}
\href{/docs/reference/layout/ratio/}{ratio} , } {
\hyperref[parameters-y]{y :}
\href{/docs/reference/foundations/auto/}{auto}
\href{/docs/reference/layout/length/}{length}
\href{/docs/reference/layout/ratio/}{ratio} , } {
\hyperref[parameters-origin]{origin :}
\href{/docs/reference/layout/alignment/}{alignment} , } {
\hyperref[parameters-reflow]{reflow :}
\href{/docs/reference/foundations/bool/}{bool} , } {
\href{/docs/reference/foundations/content/}{content} , }

) -\textgreater{} \href{/docs/reference/foundations/content/}{content}

\subsubsection{\texorpdfstring{\texttt{\ factor\ }}{ factor }}\label{parameters-factor}

\href{/docs/reference/foundations/auto/}{auto} {or}
\href{/docs/reference/layout/length/}{length} {or}
\href{/docs/reference/layout/ratio/}{ratio}

{{ Positional }}

\phantomsection\label{parameters-factor-positional-tooltip}
Positional parameters are specified in order, without names.

{{ Settable }}

\phantomsection\label{parameters-factor-settable-tooltip}
Settable parameters can be customized for all following uses of the
function with a \texttt{\ set\ } rule.

The scaling factor for both axes, as a positional argument. This is just
an optional shorthand notation for setting \texttt{\ x\ } and
\texttt{\ y\ } to the same value.

Default: \texttt{\ }{\texttt{\ 100\%\ }}\texttt{\ }

\subsubsection{\texorpdfstring{\texttt{\ x\ }}{ x }}\label{parameters-x}

\href{/docs/reference/foundations/auto/}{auto} {or}
\href{/docs/reference/layout/length/}{length} {or}
\href{/docs/reference/layout/ratio/}{ratio}

{{ Settable }}

\phantomsection\label{parameters-x-settable-tooltip}
Settable parameters can be customized for all following uses of the
function with a \texttt{\ set\ } rule.

The horizontal scaling factor.

The body will be mirrored horizontally if the parameter is negative.

Default: \texttt{\ }{\texttt{\ 100\%\ }}\texttt{\ }

\subsubsection{\texorpdfstring{\texttt{\ y\ }}{ y }}\label{parameters-y}

\href{/docs/reference/foundations/auto/}{auto} {or}
\href{/docs/reference/layout/length/}{length} {or}
\href{/docs/reference/layout/ratio/}{ratio}

{{ Settable }}

\phantomsection\label{parameters-y-settable-tooltip}
Settable parameters can be customized for all following uses of the
function with a \texttt{\ set\ } rule.

The vertical scaling factor.

The body will be mirrored vertically if the parameter is negative.

Default: \texttt{\ }{\texttt{\ 100\%\ }}\texttt{\ }

\subsubsection{\texorpdfstring{\texttt{\ origin\ }}{ origin }}\label{parameters-origin}

\href{/docs/reference/layout/alignment/}{alignment}

{{ Settable }}

\phantomsection\label{parameters-origin-settable-tooltip}
Settable parameters can be customized for all following uses of the
function with a \texttt{\ set\ } rule.

The origin of the transformation.

Default: \texttt{\ center\ }{\texttt{\ +\ }}\texttt{\ horizon\ }

\includesvg[width=0.16667in,height=0.16667in]{/assets/icons/16-arrow-right.svg}
View example

\begin{verbatim}
A#box(scale(75%)[A])A \
B#box(scale(75%, origin: bottom + left)[B])B
\end{verbatim}

\includegraphics[width=5in,height=\textheight,keepaspectratio]{/assets/docs/dT49GhvKfj-Kj_N_KdtBqQAAAAAAAAAA.png}

\subsubsection{\texorpdfstring{\texttt{\ reflow\ }}{ reflow }}\label{parameters-reflow}

\href{/docs/reference/foundations/bool/}{bool}

{{ Settable }}

\phantomsection\label{parameters-reflow-settable-tooltip}
Settable parameters can be customized for all following uses of the
function with a \texttt{\ set\ } rule.

Whether the scaling impacts the layout.

If set to \texttt{\ }{\texttt{\ false\ }}\texttt{\ } , the scaled
content will be allowed to overlap other content. If set to
\texttt{\ }{\texttt{\ true\ }}\texttt{\ } , it will compute the new size
of the scaled content and adjust the layout accordingly.

Default: \texttt{\ }{\texttt{\ false\ }}\texttt{\ }

\includesvg[width=0.16667in,height=0.16667in]{/assets/icons/16-arrow-right.svg}
View example

\begin{verbatim}
Hello #scale(x: 20%, y: 40%, reflow: true)[World]!
\end{verbatim}

\includegraphics[width=5in,height=\textheight,keepaspectratio]{/assets/docs/8qEVgn4pU_8oLmlhe4cX2QAAAAAAAAAA.png}

\subsubsection{\texorpdfstring{\texttt{\ body\ }}{ body }}\label{parameters-body}

\href{/docs/reference/foundations/content/}{content}

{Required} {{ Positional }}

\phantomsection\label{parameters-body-positional-tooltip}
Positional parameters are specified in order, without names.

The content to scale.

\href{/docs/reference/layout/rotate/}{\pandocbounded{\includesvg[keepaspectratio]{/assets/icons/16-arrow-right.svg}}}

{ Rotate } { Previous page }

\href{/docs/reference/layout/skew/}{\pandocbounded{\includesvg[keepaspectratio]{/assets/icons/16-arrow-right.svg}}}

{ Skew } { Next page }


\title{typst.app/docs/reference/layout/direction}

\begin{itemize}
\tightlist
\item
  \href{/docs}{\includesvg[width=0.16667in,height=0.16667in]{/assets/icons/16-docs-dark.svg}}
\item
  \includesvg[width=0.16667in,height=0.16667in]{/assets/icons/16-arrow-right.svg}
\item
  \href{/docs/reference/}{Reference}
\item
  \includesvg[width=0.16667in,height=0.16667in]{/assets/icons/16-arrow-right.svg}
\item
  \href{/docs/reference/layout/}{Layout}
\item
  \includesvg[width=0.16667in,height=0.16667in]{/assets/icons/16-arrow-right.svg}
\item
  \href{/docs/reference/layout/direction/}{Direction}
\end{itemize}

\section{\texorpdfstring{{ direction }}{ direction }}\label{summary}

The four directions into which content can be laid out.

Possible values are:

\begin{itemize}
\tightlist
\item
  \texttt{\ ltr\ } : Left to right.
\item
  \texttt{\ rtl\ } : Right to left.
\item
  \texttt{\ ttb\ } : Top to bottom.
\item
  \texttt{\ btt\ } : Bottom to top.
\end{itemize}

These values are available globally and also in the direction
type\textquotesingle s scope, so you can write either of the following
two:

\begin{verbatim}
#stack(dir: rtl)[A][B][C]
#stack(dir: direction.rtl)[A][B][C]
\end{verbatim}

\includegraphics[width=5in,height=\textheight,keepaspectratio]{/assets/docs/43rZPR36KLZcf8RLRLjX0wAAAAAAAAAA.png}

\subsection{\texorpdfstring{{ Definitions
}}{ Definitions }}\label{definitions}

\phantomsection\label{definitions-tooltip}
Functions and types and can have associated definitions. These are
accessed by specifying the function or type, followed by a period, and
then the definition\textquotesingle s name.

\subsubsection{\texorpdfstring{\texttt{\ axis\ }}{ axis }}\label{definitions-axis}

The axis this direction belongs to, either
\texttt{\ }{\texttt{\ "horizontal"\ }}\texttt{\ } or
\texttt{\ }{\texttt{\ "vertical"\ }}\texttt{\ } .

self { . } { axis } (

)

\begin{verbatim}
#ltr.axis() \
#ttb.axis()
\end{verbatim}

\includegraphics[width=5in,height=\textheight,keepaspectratio]{/assets/docs/JrNsSPuIGz5d-HyvpKlmRAAAAAAAAAAA.png}

\subsubsection{\texorpdfstring{\texttt{\ start\ }}{ start }}\label{definitions-start}

The start point of this direction, as an alignment.

self { . } { start } (

) -\textgreater{} \href{/docs/reference/layout/alignment/}{alignment}

\begin{verbatim}
#ltr.start() \
#rtl.start() \
#ttb.start() \
#btt.start()
\end{verbatim}

\includegraphics[width=5in,height=\textheight,keepaspectratio]{/assets/docs/N9RQCkuykNN4FsJgRg06GgAAAAAAAAAA.png}

\subsubsection{\texorpdfstring{\texttt{\ end\ }}{ end }}\label{definitions-end}

The end point of this direction, as an alignment.

self { . } { end } (

) -\textgreater{} \href{/docs/reference/layout/alignment/}{alignment}

\begin{verbatim}
#ltr.end() \
#rtl.end() \
#ttb.end() \
#btt.end()
\end{verbatim}

\includegraphics[width=5in,height=\textheight,keepaspectratio]{/assets/docs/NDjcpeKFmKqoCGermlx1dAAAAAAAAAAA.png}

\subsubsection{\texorpdfstring{\texttt{\ inv\ }}{ inv }}\label{definitions-inv}

The inverse direction.

self { . } { inv } (

) -\textgreater{} \href{/docs/reference/layout/direction/}{direction}

\begin{verbatim}
#ltr.inv() \
#rtl.inv() \
#ttb.inv() \
#btt.inv()
\end{verbatim}

\includegraphics[width=5in,height=\textheight,keepaspectratio]{/assets/docs/kBDvCk2AJ9dPd5ZUJjxcOgAAAAAAAAAA.png}

\href{/docs/reference/layout/columns/}{\pandocbounded{\includesvg[keepaspectratio]{/assets/icons/16-arrow-right.svg}}}

{ Columns } { Previous page }

\href{/docs/reference/layout/fraction/}{\pandocbounded{\includesvg[keepaspectratio]{/assets/icons/16-arrow-right.svg}}}

{ Fraction } { Next page }


\title{typst.app/docs/reference/layout/h}

\begin{itemize}
\tightlist
\item
  \href{/docs}{\includesvg[width=0.16667in,height=0.16667in]{/assets/icons/16-docs-dark.svg}}
\item
  \includesvg[width=0.16667in,height=0.16667in]{/assets/icons/16-arrow-right.svg}
\item
  \href{/docs/reference/}{Reference}
\item
  \includesvg[width=0.16667in,height=0.16667in]{/assets/icons/16-arrow-right.svg}
\item
  \href{/docs/reference/layout/}{Layout}
\item
  \includesvg[width=0.16667in,height=0.16667in]{/assets/icons/16-arrow-right.svg}
\item
  \href{/docs/reference/layout/h/}{Spacing (H)}
\end{itemize}

\section{\texorpdfstring{\texttt{\ h\ } {{ Element
}}}{ h   Element }}\label{summary}

\phantomsection\label{element-tooltip}
Element functions can be customized with \texttt{\ set\ } and
\texttt{\ show\ } rules.

Inserts horizontal spacing into a paragraph.

The spacing can be absolute, relative, or fractional. In the last case,
the remaining space on the line is distributed among all fractional
spacings according to their relative fractions.

\subsection{Example}\label{example}

\begin{verbatim}
First #h(1cm) Second \
First #h(30%) Second
\end{verbatim}

\includegraphics[width=5in,height=\textheight,keepaspectratio]{/assets/docs/8wL-xYLR6Y7MLlpoIuX_vAAAAAAAAAAA.png}

\subsection{Fractional spacing}\label{fractional-spacing}

With fractional spacing, you can align things within a line without
forcing a paragraph break (like
\href{/docs/reference/layout/align/}{\texttt{\ align\ }} would). Each
fractionally sized element gets space based on the ratio of its fraction
to the sum of all fractions.

\begin{verbatim}
First #h(1fr) Second \
First #h(1fr) Second #h(1fr) Third \
First #h(2fr) Second #h(1fr) Third
\end{verbatim}

\includegraphics[width=5in,height=\textheight,keepaspectratio]{/assets/docs/pBCqhY9Aheurjnzy2VgPBgAAAAAAAAAA.png}

\subsection{Mathematical Spacing}\label{math-spacing}

In \href{/docs/reference/math/}{mathematical formulas} , you can
additionally use these constants to add spacing between elements:
\texttt{\ thin\ } (1/6Â~em), \texttt{\ med\ } (2/9Â~em),
\texttt{\ thick\ } (5/18Â~em), \texttt{\ quad\ } (1Â~em),
\texttt{\ wide\ } (2Â~em).

\subsection{\texorpdfstring{{ Parameters
}}{ Parameters }}\label{parameters}

\phantomsection\label{parameters-tooltip}
Parameters are the inputs to a function. They are specified in
parentheses after the function name.

{ h } (

{ \href{/docs/reference/layout/relative/}{relative}
\href{/docs/reference/layout/fraction/}{fraction} , } {
\hyperref[parameters-weak]{weak :}
\href{/docs/reference/foundations/bool/}{bool} , }

) -\textgreater{} \href{/docs/reference/foundations/content/}{content}

\subsubsection{\texorpdfstring{\texttt{\ amount\ }}{ amount }}\label{parameters-amount}

\href{/docs/reference/layout/relative/}{relative} {or}
\href{/docs/reference/layout/fraction/}{fraction}

{Required} {{ Positional }}

\phantomsection\label{parameters-amount-positional-tooltip}
Positional parameters are specified in order, without names.

How much spacing to insert.

\subsubsection{\texorpdfstring{\texttt{\ weak\ }}{ weak }}\label{parameters-weak}

\href{/docs/reference/foundations/bool/}{bool}

{{ Settable }}

\phantomsection\label{parameters-weak-settable-tooltip}
Settable parameters can be customized for all following uses of the
function with a \texttt{\ set\ } rule.

If \texttt{\ }{\texttt{\ true\ }}\texttt{\ } , the spacing collapses at
the start or end of a paragraph. Moreover, from multiple adjacent weak
spacings all but the largest one collapse.

Weak spacing in markup also causes all adjacent markup spaces to be
removed, regardless of the amount of spacing inserted. To force a space
next to weak spacing, you can explicitly write
\texttt{\ }{\texttt{\ \#\ }}\texttt{\ }{\texttt{\ "\ "\ }}\texttt{\ }
(for a normal space) or
\texttt{\ }{\texttt{\ \textasciitilde{}\ }}\texttt{\ } (for a
non-breaking space). The latter can be useful to create a construct that
always attaches to the preceding word with one non-breaking space,
independently of whether a markup space existed in front or not.

Default: \texttt{\ }{\texttt{\ false\ }}\texttt{\ }

\includesvg[width=0.16667in,height=0.16667in]{/assets/icons/16-arrow-right.svg}
View example

\begin{verbatim}
#h(1cm, weak: true)
We identified a group of _weak_
specimens that fail to manifest
in most cases. However, when
#h(8pt, weak: true) supported
#h(8pt, weak: true) on both sides,
they do show up.

Further #h(0pt, weak: true) more,
even the smallest of them swallow
adjacent markup spaces.
\end{verbatim}

\includegraphics[width=5in,height=\textheight,keepaspectratio]{/assets/docs/c_7b_9WV6STCF2ERdGhpfQAAAAAAAAAA.png}

\href{/docs/reference/layout/skew/}{\pandocbounded{\includesvg[keepaspectratio]{/assets/icons/16-arrow-right.svg}}}

{ Skew } { Previous page }

\href{/docs/reference/layout/v/}{\pandocbounded{\includesvg[keepaspectratio]{/assets/icons/16-arrow-right.svg}}}

{ Spacing (V) } { Next page }


\title{typst.app/docs/reference/layout/place}

\begin{itemize}
\tightlist
\item
  \href{/docs}{\includesvg[width=0.16667in,height=0.16667in]{/assets/icons/16-docs-dark.svg}}
\item
  \includesvg[width=0.16667in,height=0.16667in]{/assets/icons/16-arrow-right.svg}
\item
  \href{/docs/reference/}{Reference}
\item
  \includesvg[width=0.16667in,height=0.16667in]{/assets/icons/16-arrow-right.svg}
\item
  \href{/docs/reference/layout/}{Layout}
\item
  \includesvg[width=0.16667in,height=0.16667in]{/assets/icons/16-arrow-right.svg}
\item
  \href{/docs/reference/layout/place/}{Place}
\end{itemize}

\section{\texorpdfstring{\texttt{\ place\ } {{ Element
}}}{ place   Element }}\label{summary}

\phantomsection\label{element-tooltip}
Element functions can be customized with \texttt{\ set\ } and
\texttt{\ show\ } rules.

Places content relatively to its parent container.

Placed content can be either overlaid (the default) or floating.
Overlaid content is aligned with the parent container according to the
given
\href{/docs/reference/layout/place/\#parameters-alignment}{\texttt{\ alignment\ }}
, and shown over any other content added so far in the container.
Floating content is placed at the top or bottom of the container,
displacing other content down or up respectively. In both cases, the
content position can be adjusted with
\href{/docs/reference/layout/place/\#parameters-dx}{\texttt{\ dx\ }} and
\href{/docs/reference/layout/place/\#parameters-dy}{\texttt{\ dy\ }}
offsets without affecting the layout.

The parent can be any container such as a
\href{/docs/reference/layout/block/}{\texttt{\ block\ }} ,
\href{/docs/reference/layout/box/}{\texttt{\ box\ }} ,
\href{/docs/reference/visualize/rect/}{\texttt{\ rect\ }} , etc. A top
level \texttt{\ place\ } call will place content directly in the text
area of the current page. This can be used for absolute positioning on
the page: with a \texttt{\ top\ +\ left\ }
\href{/docs/reference/layout/place/\#parameters-alignment}{\texttt{\ alignment\ }}
, the offsets \texttt{\ dx\ } and \texttt{\ dy\ } will set the position
of the element\textquotesingle s top left corner relatively to the top
left corner of the text area. For absolute positioning on the full page
including margins, you can use \texttt{\ place\ } in
\href{/docs/reference/layout/page/\#parameters-foreground}{\texttt{\ page.foreground\ }}
or
\href{/docs/reference/layout/page/\#parameters-background}{\texttt{\ page.background\ }}
.

\subsection{Examples}\label{examples}

\begin{verbatim}
#set page(height: 120pt)
Hello, world!

#rect(
  width: 100%,
  height: 2cm,
  place(horizon + right, square()),
)

#place(
  top + left,
  dx: -5pt,
  square(size: 5pt, fill: red),
)
\end{verbatim}

\includegraphics[width=5in,height=\textheight,keepaspectratio]{/assets/docs/b3Ue37sNl2HDpslyo5trfgAAAAAAAAAA.png}

\subsection{Effect on the position of other
elements}\label{effect-on-other-elements}

Overlaid elements don\textquotesingle t take space in the flow of
content, but a \texttt{\ place\ } call inserts an invisible block-level
element in the flow. This can affect the layout by breaking the current
paragraph. To avoid this, you can wrap the \texttt{\ place\ } call in a
\href{/docs/reference/layout/box/}{\texttt{\ box\ }} when the call is
made in the middle of a paragraph. The alignment and offsets will then
be relative to this zero-size box. To make sure it
doesn\textquotesingle t interfere with spacing, the box should be
attached to a word using a word joiner.

For example, the following defines a function for attaching an
annotation to the following word:

\begin{verbatim}
#let annotate(..args) = {
  box(place(..args))
  sym.wj
  h(0pt, weak: true)
}

A placed #annotate(square(), dy: 2pt)
square in my text.
\end{verbatim}

\includegraphics[width=5in,height=\textheight,keepaspectratio]{/assets/docs/QIJqPsAAp5jqe-EB4bZF1gAAAAAAAAAA.png}

The zero-width weak spacing serves to discard spaces between the
function call and the next word.

\subsection{\texorpdfstring{{ Parameters
}}{ Parameters }}\label{parameters}

\phantomsection\label{parameters-tooltip}
Parameters are the inputs to a function. They are specified in
parentheses after the function name.

{ place } (

{ \hyperref[parameters-alignment]{}
\href{/docs/reference/foundations/auto/}{auto}
\href{/docs/reference/layout/alignment/}{alignment} , } {
\hyperref[parameters-scope]{scope :}
\href{/docs/reference/foundations/str/}{str} , } {
\hyperref[parameters-float]{float :}
\href{/docs/reference/foundations/bool/}{bool} , } {
\hyperref[parameters-clearance]{clearance :}
\href{/docs/reference/layout/length/}{length} , } {
\hyperref[parameters-dx]{dx :}
\href{/docs/reference/layout/relative/}{relative} , } {
\hyperref[parameters-dy]{dy :}
\href{/docs/reference/layout/relative/}{relative} , } {
\href{/docs/reference/foundations/content/}{content} , }

) -\textgreater{} \href{/docs/reference/foundations/content/}{content}

\subsubsection{\texorpdfstring{\texttt{\ alignment\ }}{ alignment }}\label{parameters-alignment}

\href{/docs/reference/foundations/auto/}{auto} {or}
\href{/docs/reference/layout/alignment/}{alignment}

{{ Positional }}

\phantomsection\label{parameters-alignment-positional-tooltip}
Positional parameters are specified in order, without names.

{{ Settable }}

\phantomsection\label{parameters-alignment-settable-tooltip}
Settable parameters can be customized for all following uses of the
function with a \texttt{\ set\ } rule.

Relative to which position in the parent container to place the content.

\begin{itemize}
\tightlist
\item
  If \texttt{\ float\ } is \texttt{\ }{\texttt{\ false\ }}\texttt{\ } ,
  then this can be any alignment other than
  \texttt{\ }{\texttt{\ auto\ }}\texttt{\ } .
\item
  If \texttt{\ float\ } is \texttt{\ }{\texttt{\ true\ }}\texttt{\ } ,
  then this must be \texttt{\ }{\texttt{\ auto\ }}\texttt{\ } ,
  \texttt{\ top\ } , or \texttt{\ bottom\ } .
\end{itemize}

When \texttt{\ float\ } is \texttt{\ }{\texttt{\ false\ }}\texttt{\ }
and no vertical alignment is specified, the content is placed at the
current position on the vertical axis.

Default: \texttt{\ start\ }

\subsubsection{\texorpdfstring{\texttt{\ scope\ }}{ scope }}\label{parameters-scope}

\href{/docs/reference/foundations/str/}{str}

{{ Settable }}

\phantomsection\label{parameters-scope-settable-tooltip}
Settable parameters can be customized for all following uses of the
function with a \texttt{\ set\ } rule.

Relative to which containing scope something is placed.

The parent scope is primarily used with figures and, for this reason,
the figure function has a mirrored
\href{/docs/reference/model/figure/\#parameters-scope}{\texttt{\ scope\ }
parameter} . Nonetheless, it can also be more generally useful to break
out of the columns. A typical example would be to
\href{/docs/guides/page-setup-guide/\#columns}{create a single-column
title section} in a two-column document.

Note that parent-scoped placement is currently only supported if
\texttt{\ float\ } is \texttt{\ }{\texttt{\ true\ }}\texttt{\ } . This
may change in the future.

\begin{longtable}[]{@{}ll@{}}
\toprule\noalign{}
Variant & Details \\
\midrule\noalign{}
\endhead
\bottomrule\noalign{}
\endlastfoot
\texttt{\ "\ column\ "\ } & Place into the current column. \\
\texttt{\ "\ parent\ "\ } & Place relative to the parent, letting the
content span over all columns. \\
\end{longtable}

Default: \texttt{\ }{\texttt{\ "column"\ }}\texttt{\ }

\includesvg[width=0.16667in,height=0.16667in]{/assets/icons/16-arrow-right.svg}
View example

\begin{verbatim}
#set page(height: 150pt, columns: 2)
#place(
  top + center,
  scope: "parent",
  float: true,
  rect(width: 80%, fill: aqua),
)

#lorem(25)
\end{verbatim}

\includegraphics[width=5in,height=\textheight,keepaspectratio]{/assets/docs/9xhEXBaN2g3N9Vju7GUzFwAAAAAAAAAA.png}

\subsubsection{\texorpdfstring{\texttt{\ float\ }}{ float }}\label{parameters-float}

\href{/docs/reference/foundations/bool/}{bool}

{{ Settable }}

\phantomsection\label{parameters-float-settable-tooltip}
Settable parameters can be customized for all following uses of the
function with a \texttt{\ set\ } rule.

Whether the placed element has floating layout.

Floating elements are positioned at the top or bottom of the parent
container, displacing in-flow content. They are always placed in the
in-flow order relative to each other, as well as before any content
following a later
\href{/docs/reference/layout/place/\#definitions-flush}{\texttt{\ place.flush\ }}
element.

Default: \texttt{\ }{\texttt{\ false\ }}\texttt{\ }

\includesvg[width=0.16667in,height=0.16667in]{/assets/icons/16-arrow-right.svg}
View example

\begin{verbatim}
#set page(height: 150pt)
#let note(where, body) = place(
  center + where,
  float: true,
  clearance: 6pt,
  rect(body),
)

#lorem(10)
#note(bottom)[Bottom 1]
#note(bottom)[Bottom 2]
#lorem(40)
#note(top)[Top]
#lorem(10)
\end{verbatim}

\includegraphics[width=5in,height=\textheight,keepaspectratio]{/assets/docs/t5SJ49ulSlCH5SgTOH20JAAAAAAAAAAA.png}
\includegraphics[width=5in,height=\textheight,keepaspectratio]{/assets/docs/t5SJ49ulSlCH5SgTOH20JAAAAAAAAAAB.png}

\subsubsection{\texorpdfstring{\texttt{\ clearance\ }}{ clearance }}\label{parameters-clearance}

\href{/docs/reference/layout/length/}{length}

{{ Settable }}

\phantomsection\label{parameters-clearance-settable-tooltip}
Settable parameters can be customized for all following uses of the
function with a \texttt{\ set\ } rule.

The spacing between the placed element and other elements in a floating
layout.

Has no effect if \texttt{\ float\ } is
\texttt{\ }{\texttt{\ false\ }}\texttt{\ } .

Default: \texttt{\ }{\texttt{\ 1.5em\ }}\texttt{\ }

\subsubsection{\texorpdfstring{\texttt{\ dx\ }}{ dx }}\label{parameters-dx}

\href{/docs/reference/layout/relative/}{relative}

{{ Settable }}

\phantomsection\label{parameters-dx-settable-tooltip}
Settable parameters can be customized for all following uses of the
function with a \texttt{\ set\ } rule.

The horizontal displacement of the placed content.

Default:
\texttt{\ }{\texttt{\ 0\%\ }}\texttt{\ }{\texttt{\ +\ }}\texttt{\ }{\texttt{\ 0pt\ }}\texttt{\ }

\includesvg[width=0.16667in,height=0.16667in]{/assets/icons/16-arrow-right.svg}
View example

\begin{verbatim}
#set page(height: 100pt)
#for i in range(16) {
  let amount = i * 4pt
  place(center, dx: amount - 32pt, dy: amount)[A]
}
\end{verbatim}

\includegraphics[width=5in,height=\textheight,keepaspectratio]{/assets/docs/kAqGzNrSyPcytdYDwTZgaQAAAAAAAAAA.png}

This does not affect the layout of in-flow content. In other words, the
placed content is treated as if it were wrapped in a
\href{/docs/reference/layout/move/}{\texttt{\ move\ }} element.

\subsubsection{\texorpdfstring{\texttt{\ dy\ }}{ dy }}\label{parameters-dy}

\href{/docs/reference/layout/relative/}{relative}

{{ Settable }}

\phantomsection\label{parameters-dy-settable-tooltip}
Settable parameters can be customized for all following uses of the
function with a \texttt{\ set\ } rule.

The vertical displacement of the placed content.

This does not affect the layout of in-flow content. In other words, the
placed content is treated as if it were wrapped in a
\href{/docs/reference/layout/move/}{\texttt{\ move\ }} element.

Default:
\texttt{\ }{\texttt{\ 0\%\ }}\texttt{\ }{\texttt{\ +\ }}\texttt{\ }{\texttt{\ 0pt\ }}\texttt{\ }

\subsubsection{\texorpdfstring{\texttt{\ body\ }}{ body }}\label{parameters-body}

\href{/docs/reference/foundations/content/}{content}

{Required} {{ Positional }}

\phantomsection\label{parameters-body-positional-tooltip}
Positional parameters are specified in order, without names.

The content to place.

\subsection{\texorpdfstring{{ Definitions
}}{ Definitions }}\label{definitions}

\phantomsection\label{definitions-tooltip}
Functions and types and can have associated definitions. These are
accessed by specifying the function or type, followed by a period, and
then the definition\textquotesingle s name.

\subsubsection{\texorpdfstring{\texttt{\ flush\ } {{ Element
}}}{ flush   Element }}\label{definitions-flush}

\phantomsection\label{definitions-flush-element-tooltip}
Element functions can be customized with \texttt{\ set\ } and
\texttt{\ show\ } rules.

Asks the layout algorithm to place pending floating elements before
continuing with the content.

This is useful for preventing floating figures from spilling into the
next section.

place { . } { flush } (

) -\textgreater{} \href{/docs/reference/foundations/content/}{content}

\begin{verbatim}
#lorem(15)

#figure(
  rect(width: 100%, height: 50pt),
  placement: auto,
  caption: [A rectangle],
)

#place.flush()

This text appears after the figure.
\end{verbatim}

\includegraphics[width=3.125in,height=\textheight,keepaspectratio]{/assets/docs/8qp5vfUImMtnXndzjQCsNQAAAAAAAAAA.png}
\includegraphics[width=3.125in,height=\textheight,keepaspectratio]{/assets/docs/8qp5vfUImMtnXndzjQCsNQAAAAAAAAAB.png}

\href{/docs/reference/layout/pagebreak/}{\pandocbounded{\includesvg[keepaspectratio]{/assets/icons/16-arrow-right.svg}}}

{ Page Break } { Previous page }

\href{/docs/reference/layout/ratio/}{\pandocbounded{\includesvg[keepaspectratio]{/assets/icons/16-arrow-right.svg}}}

{ Ratio } { Next page }


\title{typst.app/docs/reference/layout/alignment}

\begin{itemize}
\tightlist
\item
  \href{/docs}{\includesvg[width=0.16667in,height=0.16667in]{/assets/icons/16-docs-dark.svg}}
\item
  \includesvg[width=0.16667in,height=0.16667in]{/assets/icons/16-arrow-right.svg}
\item
  \href{/docs/reference/}{Reference}
\item
  \includesvg[width=0.16667in,height=0.16667in]{/assets/icons/16-arrow-right.svg}
\item
  \href{/docs/reference/layout/}{Layout}
\item
  \includesvg[width=0.16667in,height=0.16667in]{/assets/icons/16-arrow-right.svg}
\item
  \href{/docs/reference/layout/alignment/}{Alignment}
\end{itemize}

\section{\texorpdfstring{{ alignment }}{ alignment }}\label{summary}

Where to \href{/docs/reference/layout/align/}{align} something along an
axis.

Possible values are:

\begin{itemize}
\tightlist
\item
  \texttt{\ start\ } : Aligns at the
  \href{/docs/reference/layout/direction/\#definitions-start}{start} of
  the \href{/docs/reference/text/text/\#parameters-dir}{text direction}
  .
\item
  \texttt{\ end\ } : Aligns at the
  \href{/docs/reference/layout/direction/\#definitions-end}{end} of the
  \href{/docs/reference/text/text/\#parameters-dir}{text direction} .
\item
  \texttt{\ left\ } : Align at the left.
\item
  \texttt{\ center\ } : Aligns in the middle, horizontally.
\item
  \texttt{\ right\ } : Aligns at the right.
\item
  \texttt{\ top\ } : Aligns at the top.
\item
  \texttt{\ horizon\ } : Aligns in the middle, vertically.
\item
  \texttt{\ bottom\ } : Align at the bottom.
\end{itemize}

These values are available globally and also in the alignment
type\textquotesingle s scope, so you can write either of the following
two:

\begin{verbatim}
#align(center)[Hi]
#align(alignment.center)[Hi]
\end{verbatim}

\includegraphics[width=5in,height=\textheight,keepaspectratio]{/assets/docs/ZprGjLBPSUJ5f2a4fil8IAAAAAAAAAAA.png}

\subsection{2D alignments}\label{2d-alignments}

To align along both axes at the same time, add the two alignments using
the \texttt{\ +\ } operator. For example, \texttt{\ top\ +\ right\ }
aligns the content to the top right corner.

\begin{verbatim}
#set page(height: 3cm)
#align(center + bottom)[Hi]
\end{verbatim}

\includegraphics[width=5in,height=\textheight,keepaspectratio]{/assets/docs/X3ZrV0nn1RgePWtIVMB4XgAAAAAAAAAA.png}

\subsection{Fields}\label{fields}

The \texttt{\ x\ } and \texttt{\ y\ } fields hold the
alignment\textquotesingle s horizontal and vertical components,
respectively (as yet another \texttt{\ alignment\ } ). They may be
\texttt{\ }{\texttt{\ none\ }}\texttt{\ } .

\begin{verbatim}
#(top + right).x \
#left.x \
#left.y (none)
\end{verbatim}

\includegraphics[width=5in,height=\textheight,keepaspectratio]{/assets/docs/ecr8JX7jRnRHSrOlOhwwRwAAAAAAAAAA.png}

\subsection{\texorpdfstring{{ Definitions
}}{ Definitions }}\label{definitions}

\phantomsection\label{definitions-tooltip}
Functions and types and can have associated definitions. These are
accessed by specifying the function or type, followed by a period, and
then the definition\textquotesingle s name.

\subsubsection{\texorpdfstring{\texttt{\ axis\ }}{ axis }}\label{definitions-axis}

The axis this alignment belongs to.

\begin{itemize}
\tightlist
\item
  \texttt{\ }{\texttt{\ "horizontal"\ }}\texttt{\ } for
  \texttt{\ start\ } , \texttt{\ left\ } , \texttt{\ center\ } ,
  \texttt{\ right\ } , and \texttt{\ end\ }
\item
  \texttt{\ }{\texttt{\ "vertical"\ }}\texttt{\ } for \texttt{\ top\ } ,
  \texttt{\ horizon\ } , and \texttt{\ bottom\ }
\item
  \texttt{\ }{\texttt{\ none\ }}\texttt{\ } for 2-dimensional alignments
\end{itemize}

self { . } { axis } (

)

\begin{verbatim}
#left.axis() \
#bottom.axis()
\end{verbatim}

\includegraphics[width=5in,height=\textheight,keepaspectratio]{/assets/docs/OHqui2ES_RRnmxyOZdFWIgAAAAAAAAAA.png}

\subsubsection{\texorpdfstring{\texttt{\ inv\ }}{ inv }}\label{definitions-inv}

The inverse alignment.

self { . } { inv } (

) -\textgreater{} \href{/docs/reference/layout/alignment/}{alignment}

\begin{verbatim}
#top.inv() \
#left.inv() \
#center.inv() \
#(left + bottom).inv()
\end{verbatim}

\includegraphics[width=5in,height=\textheight,keepaspectratio]{/assets/docs/tBAxSGcUdyogNGn2l8Pm_QAAAAAAAAAA.png}

\href{/docs/reference/layout/align/}{\pandocbounded{\includesvg[keepaspectratio]{/assets/icons/16-arrow-right.svg}}}

{ Align } { Previous page }

\href{/docs/reference/layout/angle/}{\pandocbounded{\includesvg[keepaspectratio]{/assets/icons/16-arrow-right.svg}}}

{ Angle } { Next page }


\title{typst.app/docs/reference/layout/box}

\begin{itemize}
\tightlist
\item
  \href{/docs}{\includesvg[width=0.16667in,height=0.16667in]{/assets/icons/16-docs-dark.svg}}
\item
  \includesvg[width=0.16667in,height=0.16667in]{/assets/icons/16-arrow-right.svg}
\item
  \href{/docs/reference/}{Reference}
\item
  \includesvg[width=0.16667in,height=0.16667in]{/assets/icons/16-arrow-right.svg}
\item
  \href{/docs/reference/layout/}{Layout}
\item
  \includesvg[width=0.16667in,height=0.16667in]{/assets/icons/16-arrow-right.svg}
\item
  \href{/docs/reference/layout/box/}{Box}
\end{itemize}

\section{\texorpdfstring{\texttt{\ box\ } {{ Element
}}}{ box   Element }}\label{summary}

\phantomsection\label{element-tooltip}
Element functions can be customized with \texttt{\ set\ } and
\texttt{\ show\ } rules.

An inline-level container that sizes content.

All elements except inline math, text, and boxes are block-level and
cannot occur inside of a paragraph. The box function can be used to
integrate such elements into a paragraph. Boxes take the size of their
contents by default but can also be sized explicitly.

\subsection{Example}\label{example}

\begin{verbatim}
Refer to the docs
#box(
  height: 9pt,
  image("docs.svg")
)
for more information.
\end{verbatim}

\includegraphics[width=5in,height=\textheight,keepaspectratio]{/assets/docs/eB9NAzu2xk-O1miffozwKQAAAAAAAAAA.png}

\subsection{\texorpdfstring{{ Parameters
}}{ Parameters }}\label{parameters}

\phantomsection\label{parameters-tooltip}
Parameters are the inputs to a function. They are specified in
parentheses after the function name.

{ box } (

{ \hyperref[parameters-width]{width :}
\href{/docs/reference/foundations/auto/}{auto}
\href{/docs/reference/layout/relative/}{relative}
\href{/docs/reference/layout/fraction/}{fraction} , } {
\hyperref[parameters-height]{height :}
\href{/docs/reference/foundations/auto/}{auto}
\href{/docs/reference/layout/relative/}{relative} , } {
\hyperref[parameters-baseline]{baseline :}
\href{/docs/reference/layout/relative/}{relative} , } {
\hyperref[parameters-fill]{fill :}
\href{/docs/reference/foundations/none/}{none}
\href{/docs/reference/visualize/color/}{color}
\href{/docs/reference/visualize/gradient/}{gradient}
\href{/docs/reference/visualize/pattern/}{pattern} , } {
\hyperref[parameters-stroke]{stroke :}
\href{/docs/reference/foundations/none/}{none}
\href{/docs/reference/layout/length/}{length}
\href{/docs/reference/visualize/color/}{color}
\href{/docs/reference/visualize/gradient/}{gradient}
\href{/docs/reference/visualize/stroke/}{stroke}
\href{/docs/reference/visualize/pattern/}{pattern}
\href{/docs/reference/foundations/dictionary/}{dictionary} , } {
\hyperref[parameters-radius]{radius :}
\href{/docs/reference/layout/relative/}{relative}
\href{/docs/reference/foundations/dictionary/}{dictionary} , } {
\hyperref[parameters-inset]{inset :}
\href{/docs/reference/layout/relative/}{relative}
\href{/docs/reference/foundations/dictionary/}{dictionary} , } {
\hyperref[parameters-outset]{outset :}
\href{/docs/reference/layout/relative/}{relative}
\href{/docs/reference/foundations/dictionary/}{dictionary} , } {
\hyperref[parameters-clip]{clip :}
\href{/docs/reference/foundations/bool/}{bool} , } {
\hyperref[parameters-body]{}
\href{/docs/reference/foundations/none/}{none}
\href{/docs/reference/foundations/content/}{content} , }

) -\textgreater{} \href{/docs/reference/foundations/content/}{content}

\subsubsection{\texorpdfstring{\texttt{\ width\ }}{ width }}\label{parameters-width}

\href{/docs/reference/foundations/auto/}{auto} {or}
\href{/docs/reference/layout/relative/}{relative} {or}
\href{/docs/reference/layout/fraction/}{fraction}

{{ Settable }}

\phantomsection\label{parameters-width-settable-tooltip}
Settable parameters can be customized for all following uses of the
function with a \texttt{\ set\ } rule.

The width of the box.

Boxes can have \href{/docs/reference/layout/fraction/}{fractional}
widths, as the example below demonstrates.

\emph{Note:} Currently, only boxes and only their widths might be
fractionally sized within paragraphs. Support for fractionally sized
images, shapes, and more might be added in the future.

Default: \texttt{\ }{\texttt{\ auto\ }}\texttt{\ }

\includesvg[width=0.16667in,height=0.16667in]{/assets/icons/16-arrow-right.svg}
View example

\begin{verbatim}
Line in #box(width: 1fr, line(length: 100%)) between.
\end{verbatim}

\includegraphics[width=5in,height=\textheight,keepaspectratio]{/assets/docs/dzJroqkPcQ8j1yD6nZSE0AAAAAAAAAAA.png}

\subsubsection{\texorpdfstring{\texttt{\ height\ }}{ height }}\label{parameters-height}

\href{/docs/reference/foundations/auto/}{auto} {or}
\href{/docs/reference/layout/relative/}{relative}

{{ Settable }}

\phantomsection\label{parameters-height-settable-tooltip}
Settable parameters can be customized for all following uses of the
function with a \texttt{\ set\ } rule.

The height of the box.

Default: \texttt{\ }{\texttt{\ auto\ }}\texttt{\ }

\subsubsection{\texorpdfstring{\texttt{\ baseline\ }}{ baseline }}\label{parameters-baseline}

\href{/docs/reference/layout/relative/}{relative}

{{ Settable }}

\phantomsection\label{parameters-baseline-settable-tooltip}
Settable parameters can be customized for all following uses of the
function with a \texttt{\ set\ } rule.

An amount to shift the box\textquotesingle s baseline by.

Default:
\texttt{\ }{\texttt{\ 0\%\ }}\texttt{\ }{\texttt{\ +\ }}\texttt{\ }{\texttt{\ 0pt\ }}\texttt{\ }

\includesvg[width=0.16667in,height=0.16667in]{/assets/icons/16-arrow-right.svg}
View example

\begin{verbatim}
Image: #box(baseline: 40%, image("tiger.jpg", width: 2cm)).
\end{verbatim}

\includegraphics[width=5in,height=\textheight,keepaspectratio]{/assets/docs/jNZmXcLZQWKojb5Yhz3uEQAAAAAAAAAA.png}

\subsubsection{\texorpdfstring{\texttt{\ fill\ }}{ fill }}\label{parameters-fill}

\href{/docs/reference/foundations/none/}{none} {or}
\href{/docs/reference/visualize/color/}{color} {or}
\href{/docs/reference/visualize/gradient/}{gradient} {or}
\href{/docs/reference/visualize/pattern/}{pattern}

{{ Settable }}

\phantomsection\label{parameters-fill-settable-tooltip}
Settable parameters can be customized for all following uses of the
function with a \texttt{\ set\ } rule.

The box\textquotesingle s background color. See the
\href{/docs/reference/visualize/rect/\#parameters-fill}{rectangle\textquotesingle s
documentation} for more details.

Default: \texttt{\ }{\texttt{\ none\ }}\texttt{\ }

\subsubsection{\texorpdfstring{\texttt{\ stroke\ }}{ stroke }}\label{parameters-stroke}

\href{/docs/reference/foundations/none/}{none} {or}
\href{/docs/reference/layout/length/}{length} {or}
\href{/docs/reference/visualize/color/}{color} {or}
\href{/docs/reference/visualize/gradient/}{gradient} {or}
\href{/docs/reference/visualize/stroke/}{stroke} {or}
\href{/docs/reference/visualize/pattern/}{pattern} {or}
\href{/docs/reference/foundations/dictionary/}{dictionary}

{{ Settable }}

\phantomsection\label{parameters-stroke-settable-tooltip}
Settable parameters can be customized for all following uses of the
function with a \texttt{\ set\ } rule.

The box\textquotesingle s border color. See the
\href{/docs/reference/visualize/rect/\#parameters-stroke}{rectangle\textquotesingle s
documentation} for more details.

Default:
\texttt{\ }{\texttt{\ (\ }}\texttt{\ }{\texttt{\ :\ }}\texttt{\ }{\texttt{\ )\ }}\texttt{\ }

\subsubsection{\texorpdfstring{\texttt{\ radius\ }}{ radius }}\label{parameters-radius}

\href{/docs/reference/layout/relative/}{relative} {or}
\href{/docs/reference/foundations/dictionary/}{dictionary}

{{ Settable }}

\phantomsection\label{parameters-radius-settable-tooltip}
Settable parameters can be customized for all following uses of the
function with a \texttt{\ set\ } rule.

How much to round the box\textquotesingle s corners. See the
\href{/docs/reference/visualize/rect/\#parameters-radius}{rectangle\textquotesingle s
documentation} for more details.

Default:
\texttt{\ }{\texttt{\ (\ }}\texttt{\ }{\texttt{\ :\ }}\texttt{\ }{\texttt{\ )\ }}\texttt{\ }

\subsubsection{\texorpdfstring{\texttt{\ inset\ }}{ inset }}\label{parameters-inset}

\href{/docs/reference/layout/relative/}{relative} {or}
\href{/docs/reference/foundations/dictionary/}{dictionary}

{{ Settable }}

\phantomsection\label{parameters-inset-settable-tooltip}
Settable parameters can be customized for all following uses of the
function with a \texttt{\ set\ } rule.

How much to pad the box\textquotesingle s content.

\emph{Note:} When the box contains text, its exact size depends on the
current \href{/docs/reference/text/text/\#parameters-top-edge}{text
edges} .

Default:
\texttt{\ }{\texttt{\ (\ }}\texttt{\ }{\texttt{\ :\ }}\texttt{\ }{\texttt{\ )\ }}\texttt{\ }

\includesvg[width=0.16667in,height=0.16667in]{/assets/icons/16-arrow-right.svg}
View example

\begin{verbatim}
#rect(inset: 0pt)[Tight]
\end{verbatim}

\includegraphics[width=5in,height=\textheight,keepaspectratio]{/assets/docs/GVDpvIL_te6KlSASD3i2EQAAAAAAAAAA.png}

\subsubsection{\texorpdfstring{\texttt{\ outset\ }}{ outset }}\label{parameters-outset}

\href{/docs/reference/layout/relative/}{relative} {or}
\href{/docs/reference/foundations/dictionary/}{dictionary}

{{ Settable }}

\phantomsection\label{parameters-outset-settable-tooltip}
Settable parameters can be customized for all following uses of the
function with a \texttt{\ set\ } rule.

How much to expand the box\textquotesingle s size without affecting the
layout.

This is useful to prevent padding from affecting line layout. For a
generalized version of the example below, see the documentation for the
\href{/docs/reference/text/raw/\#parameters-block}{raw
text\textquotesingle s block parameter} .

Default:
\texttt{\ }{\texttt{\ (\ }}\texttt{\ }{\texttt{\ :\ }}\texttt{\ }{\texttt{\ )\ }}\texttt{\ }

\includesvg[width=0.16667in,height=0.16667in]{/assets/icons/16-arrow-right.svg}
View example

\begin{verbatim}
An inline
#box(
  fill: luma(235),
  inset: (x: 3pt, y: 0pt),
  outset: (y: 3pt),
  radius: 2pt,
)[rectangle].
\end{verbatim}

\includegraphics[width=5in,height=\textheight,keepaspectratio]{/assets/docs/68KQkm_HskMy1aDAbQWYdwAAAAAAAAAA.png}

\subsubsection{\texorpdfstring{\texttt{\ clip\ }}{ clip }}\label{parameters-clip}

\href{/docs/reference/foundations/bool/}{bool}

{{ Settable }}

\phantomsection\label{parameters-clip-settable-tooltip}
Settable parameters can be customized for all following uses of the
function with a \texttt{\ set\ } rule.

Whether to clip the content inside the box.

Clipping is useful when the box\textquotesingle s content is larger than
the box itself, as any content that exceeds the box\textquotesingle s
bounds will be hidden.

Default: \texttt{\ }{\texttt{\ false\ }}\texttt{\ }

\includesvg[width=0.16667in,height=0.16667in]{/assets/icons/16-arrow-right.svg}
View example

\begin{verbatim}
#box(
  width: 50pt,
  height: 50pt,
  clip: true,
  image("tiger.jpg", width: 100pt, height: 100pt)
)
\end{verbatim}

\includegraphics[width=5in,height=\textheight,keepaspectratio]{/assets/docs/RAY1IirASCSdH0pM4209bwAAAAAAAAAA.png}

\subsubsection{\texorpdfstring{\texttt{\ body\ }}{ body }}\label{parameters-body}

\href{/docs/reference/foundations/none/}{none} {or}
\href{/docs/reference/foundations/content/}{content}

{{ Positional }}

\phantomsection\label{parameters-body-positional-tooltip}
Positional parameters are specified in order, without names.

{{ Settable }}

\phantomsection\label{parameters-body-settable-tooltip}
Settable parameters can be customized for all following uses of the
function with a \texttt{\ set\ } rule.

The contents of the box.

Default: \texttt{\ }{\texttt{\ none\ }}\texttt{\ }

\href{/docs/reference/layout/block/}{\pandocbounded{\includesvg[keepaspectratio]{/assets/icons/16-arrow-right.svg}}}

{ Block } { Previous page }

\href{/docs/reference/layout/colbreak/}{\pandocbounded{\includesvg[keepaspectratio]{/assets/icons/16-arrow-right.svg}}}

{ Column Break } { Next page }


\title{typst.app/docs/reference/layout/fraction}

\begin{itemize}
\tightlist
\item
  \href{/docs}{\includesvg[width=0.16667in,height=0.16667in]{/assets/icons/16-docs-dark.svg}}
\item
  \includesvg[width=0.16667in,height=0.16667in]{/assets/icons/16-arrow-right.svg}
\item
  \href{/docs/reference/}{Reference}
\item
  \includesvg[width=0.16667in,height=0.16667in]{/assets/icons/16-arrow-right.svg}
\item
  \href{/docs/reference/layout/}{Layout}
\item
  \includesvg[width=0.16667in,height=0.16667in]{/assets/icons/16-arrow-right.svg}
\item
  \href{/docs/reference/layout/fraction/}{Fraction}
\end{itemize}

\section{\texorpdfstring{{ fraction }}{ fraction }}\label{summary}

Defines how the remaining space in a layout is distributed.

Each fractionally sized element gets space based on the ratio of its
fraction to the sum of all fractions.

For more details, also see the \href{/docs/reference/layout/h/}{h} and
\href{/docs/reference/layout/v/}{v} functions and the
\href{/docs/reference/layout/grid/}{grid function} .

\subsection{Example}\label{example}

\begin{verbatim}
Left #h(1fr) Left-ish #h(2fr) Right
\end{verbatim}

\includegraphics[width=5in,height=\textheight,keepaspectratio]{/assets/docs/Mh5sjFkAJFlbM1vm_65COgAAAAAAAAAA.png}

\href{/docs/reference/layout/direction/}{\pandocbounded{\includesvg[keepaspectratio]{/assets/icons/16-arrow-right.svg}}}

{ Direction } { Previous page }

\href{/docs/reference/layout/grid/}{\pandocbounded{\includesvg[keepaspectratio]{/assets/icons/16-arrow-right.svg}}}

{ Grid } { Next page }


\title{typst.app/docs/reference/layout/hide}

\begin{itemize}
\tightlist
\item
  \href{/docs}{\includesvg[width=0.16667in,height=0.16667in]{/assets/icons/16-docs-dark.svg}}
\item
  \includesvg[width=0.16667in,height=0.16667in]{/assets/icons/16-arrow-right.svg}
\item
  \href{/docs/reference/}{Reference}
\item
  \includesvg[width=0.16667in,height=0.16667in]{/assets/icons/16-arrow-right.svg}
\item
  \href{/docs/reference/layout/}{Layout}
\item
  \includesvg[width=0.16667in,height=0.16667in]{/assets/icons/16-arrow-right.svg}
\item
  \href{/docs/reference/layout/hide/}{Hide}
\end{itemize}

\section{\texorpdfstring{\texttt{\ hide\ } {{ Element
}}}{ hide   Element }}\label{summary}

\phantomsection\label{element-tooltip}
Element functions can be customized with \texttt{\ set\ } and
\texttt{\ show\ } rules.

Hides content without affecting layout.

The \texttt{\ hide\ } function allows you to hide content while the
layout still \textquotesingle sees\textquotesingle{} it. This is useful
to create whitespace that is exactly as large as some content. It may
also be useful to redact content because its arguments are not included
in the output.

\subsection{Example}\label{example}

\begin{verbatim}
Hello Jane \
#hide[Hello] Joe
\end{verbatim}

\includegraphics[width=5in,height=\textheight,keepaspectratio]{/assets/docs/w0ioP6Ne87hOMXgpgPJirgAAAAAAAAAA.png}

\subsection{\texorpdfstring{{ Parameters
}}{ Parameters }}\label{parameters}

\phantomsection\label{parameters-tooltip}
Parameters are the inputs to a function. They are specified in
parentheses after the function name.

{ hide } (

{ \href{/docs/reference/foundations/content/}{content} }

) -\textgreater{} \href{/docs/reference/foundations/content/}{content}

\subsubsection{\texorpdfstring{\texttt{\ body\ }}{ body }}\label{parameters-body}

\href{/docs/reference/foundations/content/}{content}

{Required} {{ Positional }}

\phantomsection\label{parameters-body-positional-tooltip}
Positional parameters are specified in order, without names.

The content to hide.

\href{/docs/reference/layout/grid/}{\pandocbounded{\includesvg[keepaspectratio]{/assets/icons/16-arrow-right.svg}}}

{ Grid } { Previous page }

\href{/docs/reference/layout/layout/}{\pandocbounded{\includesvg[keepaspectratio]{/assets/icons/16-arrow-right.svg}}}

{ Layout } { Next page }


\title{typst.app/docs/reference/layout/relative}

\begin{itemize}
\tightlist
\item
  \href{/docs}{\includesvg[width=0.16667in,height=0.16667in]{/assets/icons/16-docs-dark.svg}}
\item
  \includesvg[width=0.16667in,height=0.16667in]{/assets/icons/16-arrow-right.svg}
\item
  \href{/docs/reference/}{Reference}
\item
  \includesvg[width=0.16667in,height=0.16667in]{/assets/icons/16-arrow-right.svg}
\item
  \href{/docs/reference/layout/}{Layout}
\item
  \includesvg[width=0.16667in,height=0.16667in]{/assets/icons/16-arrow-right.svg}
\item
  \href{/docs/reference/layout/relative/}{Relative Length}
\end{itemize}

\section{\texorpdfstring{{ relative }}{ relative }}\label{summary}

A length in relation to some known length.

This type is a combination of a
\href{/docs/reference/layout/length/}{length} with a
\href{/docs/reference/layout/ratio/}{ratio} . It results from addition
and subtraction of a length and a ratio. Wherever a relative length is
expected, you can also use a bare length or ratio.

\subsection{Example}\label{example}

\begin{verbatim}
#rect(width: 100% - 50pt)

#(100% - 50pt).length \
#(100% - 50pt).ratio
\end{verbatim}

\includegraphics[width=5in,height=\textheight,keepaspectratio]{/assets/docs/eMTS_wIJ-8rLzP6A-A6wPAAAAAAAAAAA.png}

A relative length has the following fields:

\begin{itemize}
\tightlist
\item
  \texttt{\ length\ } : Its length component.
\item
  \texttt{\ ratio\ } : Its ratio component.
\end{itemize}

\href{/docs/reference/layout/ratio/}{\pandocbounded{\includesvg[keepaspectratio]{/assets/icons/16-arrow-right.svg}}}

{ Ratio } { Previous page }

\href{/docs/reference/layout/repeat/}{\pandocbounded{\includesvg[keepaspectratio]{/assets/icons/16-arrow-right.svg}}}

{ Repeat } { Next page }


\title{typst.app/docs/reference/layout/measure}

\begin{itemize}
\tightlist
\item
  \href{/docs}{\includesvg[width=0.16667in,height=0.16667in]{/assets/icons/16-docs-dark.svg}}
\item
  \includesvg[width=0.16667in,height=0.16667in]{/assets/icons/16-arrow-right.svg}
\item
  \href{/docs/reference/}{Reference}
\item
  \includesvg[width=0.16667in,height=0.16667in]{/assets/icons/16-arrow-right.svg}
\item
  \href{/docs/reference/layout/}{Layout}
\item
  \includesvg[width=0.16667in,height=0.16667in]{/assets/icons/16-arrow-right.svg}
\item
  \href{/docs/reference/layout/measure/}{Measure}
\end{itemize}

\section{\texorpdfstring{\texttt{\ measure\ } {{ Contextual
}}}{ measure   Contextual }}\label{summary}

\phantomsection\label{contextual-tooltip}
Contextual functions can only be used when the context is known

Measures the layouted size of content.

The \texttt{\ measure\ } function lets you determine the layouted size
of content. By default an infinite space is assumed, so the measured
dimensions may not necessarily match the final dimensions of the
content. If you want to measure in the current layout dimensions, you
can combine \texttt{\ measure\ } and
\href{/docs/reference/layout/layout/}{\texttt{\ layout\ }} .

\subsection{Example}\label{example}

The same content can have a different size depending on the
\href{/docs/reference/context/}{context} that it is placed into. In the
example below, the
\texttt{\ }{\texttt{\ \#\ }}\texttt{\ }{\texttt{\ content\ }}\texttt{\ }
is of course bigger when we increase the font size.

\begin{verbatim}
#let content = [Hello!]
#content
#set text(14pt)
#content
\end{verbatim}

\includegraphics[width=5in,height=\textheight,keepaspectratio]{/assets/docs/AhP31noWwrcSQXbwnmO-hwAAAAAAAAAA.png}

For this reason, you can only measure when context is available.

\begin{verbatim}
#let thing(body) = context {
  let size = measure(body)
  [Width of "#body" is #size.width]
}

#thing[Hey] \
#thing[Welcome]
\end{verbatim}

\includegraphics[width=5in,height=\textheight,keepaspectratio]{/assets/docs/-y6AuN3J3rl7Gz1x_VRjjwAAAAAAAAAA.png}

The measure function returns a dictionary with the entries
\texttt{\ width\ } and \texttt{\ height\ } , both of type
\href{/docs/reference/layout/length/}{\texttt{\ length\ }} .

\subsection{\texorpdfstring{{ Parameters
}}{ Parameters }}\label{parameters}

\phantomsection\label{parameters-tooltip}
Parameters are the inputs to a function. They are specified in
parentheses after the function name.

{ measure } (

{ \hyperref[parameters-width]{width :}
\href{/docs/reference/foundations/auto/}{auto}
\href{/docs/reference/layout/length/}{length} , } {
\hyperref[parameters-height]{height :}
\href{/docs/reference/foundations/auto/}{auto}
\href{/docs/reference/layout/length/}{length} , } {
\href{/docs/reference/foundations/content/}{content} , } {
\href{/docs/reference/foundations/none/}{none} { styles } , }

) -\textgreater{}
\href{/docs/reference/foundations/dictionary/}{dictionary}

\subsubsection{\texorpdfstring{\texttt{\ width\ }}{ width }}\label{parameters-width}

\href{/docs/reference/foundations/auto/}{auto} {or}
\href{/docs/reference/layout/length/}{length}

The width available to layout the content.

Setting this to \texttt{\ }{\texttt{\ auto\ }}\texttt{\ } indicates
infinite available width.

Note that using the \texttt{\ width\ } and \texttt{\ height\ }
parameters of this function is different from measuring a sized
\href{/docs/reference/layout/block/}{\texttt{\ block\ }} containing the
content. In the following example, the former will get the dimensions of
the inner content instead of the dimensions of the block.

Default: \texttt{\ }{\texttt{\ auto\ }}\texttt{\ }

\includesvg[width=0.16667in,height=0.16667in]{/assets/icons/16-arrow-right.svg}
View example

\begin{verbatim}
#context measure(lorem(100), width: 400pt)

#context measure(block(lorem(100), width: 400pt))
\end{verbatim}

\includegraphics[width=5in,height=\textheight,keepaspectratio]{/assets/docs/kGPOcZfxzWEfqWzKQCJaFgAAAAAAAAAA.png}

\subsubsection{\texorpdfstring{\texttt{\ height\ }}{ height }}\label{parameters-height}

\href{/docs/reference/foundations/auto/}{auto} {or}
\href{/docs/reference/layout/length/}{length}

The height available to layout the content.

Setting this to \texttt{\ }{\texttt{\ auto\ }}\texttt{\ } indicates
infinite available height.

Default: \texttt{\ }{\texttt{\ auto\ }}\texttt{\ }

\subsubsection{\texorpdfstring{\texttt{\ content\ }}{ content }}\label{parameters-content}

\href{/docs/reference/foundations/content/}{content}

{Required} {{ Positional }}

\phantomsection\label{parameters-content-positional-tooltip}
Positional parameters are specified in order, without names.

The content whose size to measure.

\subsubsection{\texorpdfstring{\texttt{\ styles\ }}{ styles }}\label{parameters-styles}

\href{/docs/reference/foundations/none/}{none} {or} { styles }

{{ Positional }}

\phantomsection\label{parameters-styles-positional-tooltip}
Positional parameters are specified in order, without names.

\emph{Compatibility:} This argument is deprecated. It only exists for
compatibility with Typst 0.10 and lower and shouldn\textquotesingle t be
used anymore.

Default: \texttt{\ }{\texttt{\ none\ }}\texttt{\ }

\href{/docs/reference/layout/length/}{\pandocbounded{\includesvg[keepaspectratio]{/assets/icons/16-arrow-right.svg}}}

{ Length } { Previous page }

\href{/docs/reference/layout/move/}{\pandocbounded{\includesvg[keepaspectratio]{/assets/icons/16-arrow-right.svg}}}

{ Move } { Next page }


\title{typst.app/docs/reference/layout/page}

\begin{itemize}
\tightlist
\item
  \href{/docs}{\includesvg[width=0.16667in,height=0.16667in]{/assets/icons/16-docs-dark.svg}}
\item
  \includesvg[width=0.16667in,height=0.16667in]{/assets/icons/16-arrow-right.svg}
\item
  \href{/docs/reference/}{Reference}
\item
  \includesvg[width=0.16667in,height=0.16667in]{/assets/icons/16-arrow-right.svg}
\item
  \href{/docs/reference/layout/}{Layout}
\item
  \includesvg[width=0.16667in,height=0.16667in]{/assets/icons/16-arrow-right.svg}
\item
  \href{/docs/reference/layout/page/}{Page}
\end{itemize}

\section{\texorpdfstring{\texttt{\ page\ } {{ Element
}}}{ page   Element }}\label{summary}

\phantomsection\label{element-tooltip}
Element functions can be customized with \texttt{\ set\ } and
\texttt{\ show\ } rules.

Layouts its child onto one or multiple pages.

Although this function is primarily used in set rules to affect page
properties, it can also be used to explicitly render its argument onto a
set of pages of its own.

Pages can be set to use \texttt{\ }{\texttt{\ auto\ }}\texttt{\ } as
their width or height. In this case, the pages will grow to fit their
content on the respective axis.

The \href{/docs/guides/page-setup-guide/}{Guide for Page Setup} explains
how to use this and related functions to set up a document with many
examples.

\subsection{Example}\label{example}

\begin{verbatim}
#set page("us-letter")

There you go, US friends!
\end{verbatim}

\includegraphics[width=12.75in,height=\textheight,keepaspectratio]{/assets/docs/Gsn3vxGfYJJE0DFa5w6toQAAAAAAAAAA.png}

\subsection{\texorpdfstring{{ Parameters
}}{ Parameters }}\label{parameters}

\phantomsection\label{parameters-tooltip}
Parameters are the inputs to a function. They are specified in
parentheses after the function name.

{ page } (

{ \hyperref[parameters-paper]{paper :}
\href{/docs/reference/foundations/str/}{str} , } {
\hyperref[parameters-width]{width :}
\href{/docs/reference/foundations/auto/}{auto}
\href{/docs/reference/layout/length/}{length} , } {
\hyperref[parameters-height]{height :}
\href{/docs/reference/foundations/auto/}{auto}
\href{/docs/reference/layout/length/}{length} , } {
\hyperref[parameters-flipped]{flipped :}
\href{/docs/reference/foundations/bool/}{bool} , } {
\hyperref[parameters-margin]{margin :}
\href{/docs/reference/foundations/auto/}{auto}
\href{/docs/reference/layout/relative/}{relative}
\href{/docs/reference/foundations/dictionary/}{dictionary} , } {
\hyperref[parameters-binding]{binding :}
\href{/docs/reference/foundations/auto/}{auto}
\href{/docs/reference/layout/alignment/}{alignment} , } {
\hyperref[parameters-columns]{columns :}
\href{/docs/reference/foundations/int/}{int} , } {
\hyperref[parameters-fill]{fill :}
\href{/docs/reference/foundations/none/}{none}
\href{/docs/reference/foundations/auto/}{auto}
\href{/docs/reference/visualize/color/}{color}
\href{/docs/reference/visualize/gradient/}{gradient}
\href{/docs/reference/visualize/pattern/}{pattern} , } {
\hyperref[parameters-numbering]{numbering :}
\href{/docs/reference/foundations/none/}{none}
\href{/docs/reference/foundations/str/}{str}
\href{/docs/reference/foundations/function/}{function} , } {
\hyperref[parameters-number-align]{number-align :}
\href{/docs/reference/layout/alignment/}{alignment} , } {
\hyperref[parameters-header]{header :}
\href{/docs/reference/foundations/none/}{none}
\href{/docs/reference/foundations/auto/}{auto}
\href{/docs/reference/foundations/content/}{content} , } {
\hyperref[parameters-header-ascent]{header-ascent :}
\href{/docs/reference/layout/relative/}{relative} , } {
\hyperref[parameters-footer]{footer :}
\href{/docs/reference/foundations/none/}{none}
\href{/docs/reference/foundations/auto/}{auto}
\href{/docs/reference/foundations/content/}{content} , } {
\hyperref[parameters-footer-descent]{footer-descent :}
\href{/docs/reference/layout/relative/}{relative} , } {
\hyperref[parameters-background]{background :}
\href{/docs/reference/foundations/none/}{none}
\href{/docs/reference/foundations/content/}{content} , } {
\hyperref[parameters-foreground]{foreground :}
\href{/docs/reference/foundations/none/}{none}
\href{/docs/reference/foundations/content/}{content} , } {
\href{/docs/reference/foundations/content/}{content} , }

) -\textgreater{} \href{/docs/reference/foundations/content/}{content}

\subsubsection{\texorpdfstring{\texttt{\ paper\ }}{ paper }}\label{parameters-paper}

\href{/docs/reference/foundations/str/}{str}

{{ Settable }}

\phantomsection\label{parameters-paper-settable-tooltip}
Settable parameters can be customized for all following uses of the
function with a \texttt{\ set\ } rule.

A standard paper size to set width and height.

This is just a shorthand for setting \texttt{\ width\ } and
\texttt{\ height\ } and, as such, cannot be retrieved in a context
expression.

\includesvg[width=0.16667in,height=0.16667in]{/assets/icons/16-arrow-right.svg}
View options

Default: \texttt{\ }{\texttt{\ "a4"\ }}\texttt{\ }

\includesvg[width=0.16667in,height=0.16667in]{/assets/icons/16-arrow-right.svg}
View paper sizes

\texttt{\ "\ a0\ "\ } , \texttt{\ "\ a1\ "\ } , \texttt{\ "\ a2\ "\ } ,
\texttt{\ "\ a3\ "\ } , \texttt{\ "\ a4\ "\ } , \texttt{\ "\ a5\ "\ } ,
\texttt{\ "\ a6\ "\ } , \texttt{\ "\ a7\ "\ } , \texttt{\ "\ a8\ "\ } ,
\texttt{\ "\ a9\ "\ } , \texttt{\ "\ a10\ "\ } , \texttt{\ "\ a11\ "\ }
, \texttt{\ "\ iso-b1\ "\ } , \texttt{\ "\ iso-b2\ "\ } ,
\texttt{\ "\ iso-b3\ "\ } , \texttt{\ "\ iso-b4\ "\ } ,
\texttt{\ "\ iso-b5\ "\ } , \texttt{\ "\ iso-b6\ "\ } ,
\texttt{\ "\ iso-b7\ "\ } , \texttt{\ "\ iso-b8\ "\ } ,
\texttt{\ "\ iso-c3\ "\ } , \texttt{\ "\ iso-c4\ "\ } ,
\texttt{\ "\ iso-c5\ "\ } , \texttt{\ "\ iso-c6\ "\ } ,
\texttt{\ "\ iso-c7\ "\ } , \texttt{\ "\ iso-c8\ "\ } ,
\texttt{\ "\ din-d3\ "\ } , \texttt{\ "\ din-d4\ "\ } ,
\texttt{\ "\ din-d5\ "\ } , \texttt{\ "\ din-d6\ "\ } ,
\texttt{\ "\ din-d7\ "\ } , \texttt{\ "\ din-d8\ "\ } ,
\texttt{\ "\ sis-g5\ "\ } , \texttt{\ "\ sis-e5\ "\ } ,
\texttt{\ "\ ansi-a\ "\ } , \texttt{\ "\ ansi-b\ "\ } ,
\texttt{\ "\ ansi-c\ "\ } , \texttt{\ "\ ansi-d\ "\ } ,
\texttt{\ "\ ansi-e\ "\ } , \texttt{\ "\ arch-a\ "\ } ,
\texttt{\ "\ arch-b\ "\ } , \texttt{\ "\ arch-c\ "\ } ,
\texttt{\ "\ arch-d\ "\ } , \texttt{\ "\ arch-e1\ "\ } ,
\texttt{\ "\ arch-e\ "\ } , \texttt{\ "\ jis-b0\ "\ } ,
\texttt{\ "\ jis-b1\ "\ } , \texttt{\ "\ jis-b2\ "\ } ,
\texttt{\ "\ jis-b3\ "\ } , \texttt{\ "\ jis-b4\ "\ } ,
\texttt{\ "\ jis-b5\ "\ } , \texttt{\ "\ jis-b6\ "\ } ,
\texttt{\ "\ jis-b7\ "\ } , \texttt{\ "\ jis-b8\ "\ } ,
\texttt{\ "\ jis-b9\ "\ } , \texttt{\ "\ jis-b10\ "\ } ,
\texttt{\ "\ jis-b11\ "\ } , \texttt{\ "\ sac-d0\ "\ } ,
\texttt{\ "\ sac-d1\ "\ } , \texttt{\ "\ sac-d2\ "\ } ,
\texttt{\ "\ sac-d3\ "\ } , \texttt{\ "\ sac-d4\ "\ } ,
\texttt{\ "\ sac-d5\ "\ } , \texttt{\ "\ sac-d6\ "\ } ,
\texttt{\ "\ iso-id-1\ "\ } , \texttt{\ "\ iso-id-2\ "\ } ,
\texttt{\ "\ iso-id-3\ "\ } , \texttt{\ "\ asia-f4\ "\ } ,
\texttt{\ "\ jp-shiroku-ban-4\ "\ } ,
\texttt{\ "\ jp-shiroku-ban-5\ "\ } ,
\texttt{\ "\ jp-shiroku-ban-6\ "\ } , \texttt{\ "\ jp-kiku-4\ "\ } ,
\texttt{\ "\ jp-kiku-5\ "\ } , \texttt{\ "\ jp-business-card\ "\ } ,
\texttt{\ "\ cn-business-card\ "\ } ,
\texttt{\ "\ eu-business-card\ "\ } , \texttt{\ "\ fr-tellière\ "\ } ,
\texttt{\ "\ fr-couronne-écriture\ "\ } ,
\texttt{\ "\ fr-couronne-édition\ "\ } , \texttt{\ "\ fr-raisin\ "\ } ,
\texttt{\ "\ fr-carré\ "\ } , \texttt{\ "\ fr-jésus\ "\ } ,
\texttt{\ "\ uk-brief\ "\ } , \texttt{\ "\ uk-draft\ "\ } ,
\texttt{\ "\ uk-foolscap\ "\ } , \texttt{\ "\ uk-quarto\ "\ } ,
\texttt{\ "\ uk-crown\ "\ } , \texttt{\ "\ uk-book-a\ "\ } ,
\texttt{\ "\ uk-book-b\ "\ } , \texttt{\ "\ us-letter\ "\ } ,
\texttt{\ "\ us-legal\ "\ } , \texttt{\ "\ us-tabloid\ "\ } ,
\texttt{\ "\ us-executive\ "\ } , \texttt{\ "\ us-foolscap-folio\ "\ } ,
\texttt{\ "\ us-statement\ "\ } , \texttt{\ "\ us-ledger\ "\ } ,
\texttt{\ "\ us-oficio\ "\ } , \texttt{\ "\ us-gov-letter\ "\ } ,
\texttt{\ "\ us-gov-legal\ "\ } , \texttt{\ "\ us-business-card\ "\ } ,
\texttt{\ "\ us-digest\ "\ } , \texttt{\ "\ us-trade\ "\ } ,
\texttt{\ "\ newspaper-compact\ "\ } ,
\texttt{\ "\ newspaper-berliner\ "\ } ,
\texttt{\ "\ newspaper-broadsheet\ "\ } ,
\texttt{\ "\ presentation-16-9\ "\ } ,
\texttt{\ "\ presentation-4-3\ "\ }

\subsubsection{\texorpdfstring{\texttt{\ width\ }}{ width }}\label{parameters-width}

\href{/docs/reference/foundations/auto/}{auto} {or}
\href{/docs/reference/layout/length/}{length}

{{ Settable }}

\phantomsection\label{parameters-width-settable-tooltip}
Settable parameters can be customized for all following uses of the
function with a \texttt{\ set\ } rule.

The width of the page.

Default: \texttt{\ }{\texttt{\ 595.28pt\ }}\texttt{\ }

\includesvg[width=0.16667in,height=0.16667in]{/assets/icons/16-arrow-right.svg}
View example

\begin{verbatim}
#set page(
  width: 3cm,
  margin: (x: 0cm),
)

#for i in range(3) {
  box(square(width: 1cm))
}
\end{verbatim}

\includegraphics[width=1.77083in,height=\textheight,keepaspectratio]{/assets/docs/xcDLR5uuky5aEnJroP3JfQAAAAAAAAAA.png}

\subsubsection{\texorpdfstring{\texttt{\ height\ }}{ height }}\label{parameters-height}

\href{/docs/reference/foundations/auto/}{auto} {or}
\href{/docs/reference/layout/length/}{length}

{{ Settable }}

\phantomsection\label{parameters-height-settable-tooltip}
Settable parameters can be customized for all following uses of the
function with a \texttt{\ set\ } rule.

The height of the page.

If this is set to \texttt{\ }{\texttt{\ auto\ }}\texttt{\ } , page
breaks can only be triggered manually by inserting a
\href{/docs/reference/layout/pagebreak/}{page break} . Most examples
throughout this documentation use
\texttt{\ }{\texttt{\ auto\ }}\texttt{\ } for the height of the page to
dynamically grow and shrink to fit their content.

Default: \texttt{\ }{\texttt{\ 841.89pt\ }}\texttt{\ }

\subsubsection{\texorpdfstring{\texttt{\ flipped\ }}{ flipped }}\label{parameters-flipped}

\href{/docs/reference/foundations/bool/}{bool}

{{ Settable }}

\phantomsection\label{parameters-flipped-settable-tooltip}
Settable parameters can be customized for all following uses of the
function with a \texttt{\ set\ } rule.

Whether the page is flipped into landscape orientation.

Default: \texttt{\ }{\texttt{\ false\ }}\texttt{\ }

\includesvg[width=0.16667in,height=0.16667in]{/assets/icons/16-arrow-right.svg}
View example

\begin{verbatim}
#set page(
  "us-business-card",
  flipped: true,
  fill: rgb("f2e5dd"),
)

#set align(bottom + end)
#text(14pt)[*Sam H. Richards*] \
_Procurement Manager_

#set text(10pt)
17 Main Street \
New York, NY 10001 \
+1 555 555 5555
\end{verbatim}

\includegraphics[width=3in,height=\textheight,keepaspectratio]{/assets/docs/NEPMLGLUMuCwXHZB6iu3CAAAAAAAAAAA.png}

\subsubsection{\texorpdfstring{\texttt{\ margin\ }}{ margin }}\label{parameters-margin}

\href{/docs/reference/foundations/auto/}{auto} {or}
\href{/docs/reference/layout/relative/}{relative} {or}
\href{/docs/reference/foundations/dictionary/}{dictionary}

{{ Settable }}

\phantomsection\label{parameters-margin-settable-tooltip}
Settable parameters can be customized for all following uses of the
function with a \texttt{\ set\ } rule.

The page\textquotesingle s margins.

\begin{itemize}
\tightlist
\item
  \texttt{\ }{\texttt{\ auto\ }}\texttt{\ } : The margins are set
  automatically to 2.5/21 times the smaller dimension of the page. This
  results in 2.5cm margins for an A4 page.
\item
  A single length: The same margin on all sides.
\item
  A dictionary: With a dictionary, the margins can be set individually.
  The dictionary can contain the following keys in order of precedence:

  \begin{itemize}
  \tightlist
  \item
    \texttt{\ top\ } : The top margin.
  \item
    \texttt{\ right\ } : The right margin.
  \item
    \texttt{\ bottom\ } : The bottom margin.
  \item
    \texttt{\ left\ } : The left margin.
  \item
    \texttt{\ inside\ } : The margin at the inner side of the page
    (where the
    \href{/docs/reference/layout/page/\#parameters-binding}{binding}
    is).
  \item
    \texttt{\ outside\ } : The margin at the outer side of the page
    (opposite to the
    \href{/docs/reference/layout/page/\#parameters-binding}{binding} ).
  \item
    \texttt{\ x\ } : The horizontal margins.
  \item
    \texttt{\ y\ } : The vertical margins.
  \item
    \texttt{\ rest\ } : The margins on all sides except those for which
    the dictionary explicitly sets a size.
  \end{itemize}
\end{itemize}

The values for \texttt{\ left\ } and \texttt{\ right\ } are mutually
exclusive with the values for \texttt{\ inside\ } and
\texttt{\ outside\ } .

Default: \texttt{\ }{\texttt{\ auto\ }}\texttt{\ }

\includesvg[width=0.16667in,height=0.16667in]{/assets/icons/16-arrow-right.svg}
View example

\begin{verbatim}
#set page(
 width: 3cm,
 height: 4cm,
 margin: (x: 8pt, y: 4pt),
)

#rect(
  width: 100%,
  height: 100%,
  fill: aqua,
)
\end{verbatim}

\includegraphics[width=1.77083in,height=\textheight,keepaspectratio]{/assets/docs/OMqyTIx7yDwyNT0DUFniFAAAAAAAAAAA.png}

\subsubsection{\texorpdfstring{\texttt{\ binding\ }}{ binding }}\label{parameters-binding}

\href{/docs/reference/foundations/auto/}{auto} {or}
\href{/docs/reference/layout/alignment/}{alignment}

{{ Settable }}

\phantomsection\label{parameters-binding-settable-tooltip}
Settable parameters can be customized for all following uses of the
function with a \texttt{\ set\ } rule.

On which side the pages will be bound.

\begin{itemize}
\tightlist
\item
  \texttt{\ }{\texttt{\ auto\ }}\texttt{\ } : Equivalent to
  \texttt{\ left\ } if the
  \href{/docs/reference/text/text/\#parameters-dir}{text direction} is
  left-to-right and \texttt{\ right\ } if it is right-to-left.
\item
  \texttt{\ left\ } : Bound on the left side.
\item
  \texttt{\ right\ } : Bound on the right side.
\end{itemize}

This affects the meaning of the \texttt{\ inside\ } and
\texttt{\ outside\ } options for margins.

Default: \texttt{\ }{\texttt{\ auto\ }}\texttt{\ }

\subsubsection{\texorpdfstring{\texttt{\ columns\ }}{ columns }}\label{parameters-columns}

\href{/docs/reference/foundations/int/}{int}

{{ Settable }}

\phantomsection\label{parameters-columns-settable-tooltip}
Settable parameters can be customized for all following uses of the
function with a \texttt{\ set\ } rule.

How many columns the page has.

If you need to insert columns into a page or other container, you can
also use the \href{/docs/reference/layout/columns/}{\texttt{\ columns\ }
function} .

Default: \texttt{\ }{\texttt{\ 1\ }}\texttt{\ }

\includesvg[width=0.16667in,height=0.16667in]{/assets/icons/16-arrow-right.svg}
View example

\begin{verbatim}
#set page(columns: 2, height: 4.8cm)
Climate change is one of the most
pressing issues of our time, with
the potential to devastate
communities, ecosystems, and
economies around the world. It's
clear that we need to take urgent
action to reduce our carbon
emissions and mitigate the impacts
of a rapidly changing climate.
\end{verbatim}

\includegraphics[width=5in,height=\textheight,keepaspectratio]{/assets/docs/Qem_NgF0Oyp_LY8JRVFBWQAAAAAAAAAA.png}

\subsubsection{\texorpdfstring{\texttt{\ fill\ }}{ fill }}\label{parameters-fill}

\href{/docs/reference/foundations/none/}{none} {or}
\href{/docs/reference/foundations/auto/}{auto} {or}
\href{/docs/reference/visualize/color/}{color} {or}
\href{/docs/reference/visualize/gradient/}{gradient} {or}
\href{/docs/reference/visualize/pattern/}{pattern}

{{ Settable }}

\phantomsection\label{parameters-fill-settable-tooltip}
Settable parameters can be customized for all following uses of the
function with a \texttt{\ set\ } rule.

The page\textquotesingle s background fill.

Setting this to something non-transparent instructs the printer to color
the complete page. If you are considering larger production runs, it may
be more environmentally friendly and cost-effective to source pre-dyed
pages and not set this property.

When set to \texttt{\ }{\texttt{\ none\ }}\texttt{\ } , the background
becomes transparent. Note that PDF pages will still appear with a
(usually white) background in viewers, but they are actually
transparent. (If you print them, no color is used for the background.)

The default of \texttt{\ }{\texttt{\ auto\ }}\texttt{\ } results in
\texttt{\ }{\texttt{\ none\ }}\texttt{\ } for PDF output, and
\texttt{\ white\ } for PNG and SVG.

Default: \texttt{\ }{\texttt{\ auto\ }}\texttt{\ }

\includesvg[width=0.16667in,height=0.16667in]{/assets/icons/16-arrow-right.svg}
View example

\begin{verbatim}
#set page(fill: rgb("444352"))
#set text(fill: rgb("fdfdfd"))
*Dark mode enabled.*
\end{verbatim}

\includegraphics[width=5in,height=\textheight,keepaspectratio]{/assets/docs/PLEs9jVtSM3FxsoATa6SAAAAAAAAAAAA.png}

\subsubsection{\texorpdfstring{\texttt{\ numbering\ }}{ numbering }}\label{parameters-numbering}

\href{/docs/reference/foundations/none/}{none} {or}
\href{/docs/reference/foundations/str/}{str} {or}
\href{/docs/reference/foundations/function/}{function}

{{ Settable }}

\phantomsection\label{parameters-numbering-settable-tooltip}
Settable parameters can be customized for all following uses of the
function with a \texttt{\ set\ } rule.

How to \href{/docs/reference/model/numbering/}{number} the pages.

If an explicit \texttt{\ footer\ } (or \texttt{\ header\ } for
top-aligned numbering) is given, the numbering is ignored.

Default: \texttt{\ }{\texttt{\ none\ }}\texttt{\ }

\includesvg[width=0.16667in,height=0.16667in]{/assets/icons/16-arrow-right.svg}
View example

\begin{verbatim}
#set page(
  height: 100pt,
  margin: (top: 16pt, bottom: 24pt),
  numbering: "1 / 1",
)

#lorem(48)
\end{verbatim}

\includegraphics[width=5in,height=\textheight,keepaspectratio]{/assets/docs/RY8f9OM2hb3s_Q3tS3fVgwAAAAAAAAAA.png}
\includegraphics[width=5in,height=\textheight,keepaspectratio]{/assets/docs/RY8f9OM2hb3s_Q3tS3fVgwAAAAAAAAAB.png}

\subsubsection{\texorpdfstring{\texttt{\ number-align\ }}{ number-align }}\label{parameters-number-align}

\href{/docs/reference/layout/alignment/}{alignment}

{{ Settable }}

\phantomsection\label{parameters-number-align-settable-tooltip}
Settable parameters can be customized for all following uses of the
function with a \texttt{\ set\ } rule.

The alignment of the page numbering.

If the vertical component is \texttt{\ top\ } , the numbering is placed
into the header and if it is \texttt{\ bottom\ } , it is placed in the
footer. Horizon alignment is forbidden. If an explicit matching
\texttt{\ header\ } or \texttt{\ footer\ } is given, the numbering is
ignored.

Default: \texttt{\ center\ }{\texttt{\ +\ }}\texttt{\ bottom\ }

\includesvg[width=0.16667in,height=0.16667in]{/assets/icons/16-arrow-right.svg}
View example

\begin{verbatim}
#set page(
  margin: (top: 16pt, bottom: 24pt),
  numbering: "1",
  number-align: right,
)

#lorem(30)
\end{verbatim}

\includegraphics[width=5in,height=\textheight,keepaspectratio]{/assets/docs/ErvjjUjlAuxqdtzputqWQAAAAAAAAAAA.png}

\subsubsection{\texorpdfstring{\texttt{\ header\ }}{ header }}\label{parameters-header}

\href{/docs/reference/foundations/none/}{none} {or}
\href{/docs/reference/foundations/auto/}{auto} {or}
\href{/docs/reference/foundations/content/}{content}

{{ Settable }}

\phantomsection\label{parameters-header-settable-tooltip}
Settable parameters can be customized for all following uses of the
function with a \texttt{\ set\ } rule.

The page\textquotesingle s header. Fills the top margin of each page.

\begin{itemize}
\tightlist
\item
  Content: Shows the content as the header.
\item
  \texttt{\ }{\texttt{\ auto\ }}\texttt{\ } : Shows the page number if a
  \texttt{\ numbering\ } is set and \texttt{\ number-align\ } is
  \texttt{\ top\ } .
\item
  \texttt{\ }{\texttt{\ none\ }}\texttt{\ } : Suppresses the header.
\end{itemize}

Default: \texttt{\ }{\texttt{\ auto\ }}\texttt{\ }

\includesvg[width=0.16667in,height=0.16667in]{/assets/icons/16-arrow-right.svg}
View example

\begin{verbatim}
#set par(justify: true)
#set page(
  margin: (top: 32pt, bottom: 20pt),
  header: [
    #set text(8pt)
    #smallcaps[Typst Academcy]
    #h(1fr) _Exercise Sheet 3_
  ],
)

#lorem(19)
\end{verbatim}

\includegraphics[width=5in,height=\textheight,keepaspectratio]{/assets/docs/nNqGFtf4s-uyEOhXjup1zgAAAAAAAAAA.png}

\subsubsection{\texorpdfstring{\texttt{\ header-ascent\ }}{ header-ascent }}\label{parameters-header-ascent}

\href{/docs/reference/layout/relative/}{relative}

{{ Settable }}

\phantomsection\label{parameters-header-ascent-settable-tooltip}
Settable parameters can be customized for all following uses of the
function with a \texttt{\ set\ } rule.

The amount the header is raised into the top margin.

Default:
\texttt{\ }{\texttt{\ 30\%\ }}\texttt{\ }{\texttt{\ +\ }}\texttt{\ }{\texttt{\ 0pt\ }}\texttt{\ }

\subsubsection{\texorpdfstring{\texttt{\ footer\ }}{ footer }}\label{parameters-footer}

\href{/docs/reference/foundations/none/}{none} {or}
\href{/docs/reference/foundations/auto/}{auto} {or}
\href{/docs/reference/foundations/content/}{content}

{{ Settable }}

\phantomsection\label{parameters-footer-settable-tooltip}
Settable parameters can be customized for all following uses of the
function with a \texttt{\ set\ } rule.

The page\textquotesingle s footer. Fills the bottom margin of each page.

\begin{itemize}
\tightlist
\item
  Content: Shows the content as the footer.
\item
  \texttt{\ }{\texttt{\ auto\ }}\texttt{\ } : Shows the page number if a
  \texttt{\ numbering\ } is set and \texttt{\ number-align\ } is
  \texttt{\ bottom\ } .
\item
  \texttt{\ }{\texttt{\ none\ }}\texttt{\ } : Suppresses the footer.
\end{itemize}

For just a page number, the \texttt{\ numbering\ } property typically
suffices. If you want to create a custom footer but still display the
page number, you can directly access the
\href{/docs/reference/introspection/counter/}{page counter} .

Default: \texttt{\ }{\texttt{\ auto\ }}\texttt{\ }

\includesvg[width=0.16667in,height=0.16667in]{/assets/icons/16-arrow-right.svg}
View example

\begin{verbatim}
#set par(justify: true)
#set page(
  height: 100pt,
  margin: 20pt,
  footer: context [
    #set align(right)
    #set text(8pt)
    #counter(page).display(
      "1 of I",
      both: true,
    )
  ]
)

#lorem(48)
\end{verbatim}

\includegraphics[width=5in,height=\textheight,keepaspectratio]{/assets/docs/GK4h2efX4DepjGGiUjzejQAAAAAAAAAA.png}
\includegraphics[width=5in,height=\textheight,keepaspectratio]{/assets/docs/GK4h2efX4DepjGGiUjzejQAAAAAAAAAB.png}

\subsubsection{\texorpdfstring{\texttt{\ footer-descent\ }}{ footer-descent }}\label{parameters-footer-descent}

\href{/docs/reference/layout/relative/}{relative}

{{ Settable }}

\phantomsection\label{parameters-footer-descent-settable-tooltip}
Settable parameters can be customized for all following uses of the
function with a \texttt{\ set\ } rule.

The amount the footer is lowered into the bottom margin.

Default:
\texttt{\ }{\texttt{\ 30\%\ }}\texttt{\ }{\texttt{\ +\ }}\texttt{\ }{\texttt{\ 0pt\ }}\texttt{\ }

\subsubsection{\texorpdfstring{\texttt{\ background\ }}{ background }}\label{parameters-background}

\href{/docs/reference/foundations/none/}{none} {or}
\href{/docs/reference/foundations/content/}{content}

{{ Settable }}

\phantomsection\label{parameters-background-settable-tooltip}
Settable parameters can be customized for all following uses of the
function with a \texttt{\ set\ } rule.

Content in the page\textquotesingle s background.

This content will be placed behind the page\textquotesingle s body. It
can be used to place a background image or a watermark.

Default: \texttt{\ }{\texttt{\ none\ }}\texttt{\ }

\includesvg[width=0.16667in,height=0.16667in]{/assets/icons/16-arrow-right.svg}
View example

\begin{verbatim}
#set page(background: rotate(24deg,
  text(18pt, fill: rgb("FFCBC4"))[
    *CONFIDENTIAL*
  ]
))

= Typst's secret plans
In the year 2023, we plan to take
over the world (of typesetting).
\end{verbatim}

\includegraphics[width=5in,height=\textheight,keepaspectratio]{/assets/docs/edMhg75ws-GIgq5IJNJbrQAAAAAAAAAA.png}

\subsubsection{\texorpdfstring{\texttt{\ foreground\ }}{ foreground }}\label{parameters-foreground}

\href{/docs/reference/foundations/none/}{none} {or}
\href{/docs/reference/foundations/content/}{content}

{{ Settable }}

\phantomsection\label{parameters-foreground-settable-tooltip}
Settable parameters can be customized for all following uses of the
function with a \texttt{\ set\ } rule.

Content in the page\textquotesingle s foreground.

This content will overlay the page\textquotesingle s body.

Default: \texttt{\ }{\texttt{\ none\ }}\texttt{\ }

\includesvg[width=0.16667in,height=0.16667in]{/assets/icons/16-arrow-right.svg}
View example

\begin{verbatim}
#set page(foreground: text(24pt)[🥸])

Reviewer 2 has marked our paper
"Weak Reject" because they did
not understand our approach...
\end{verbatim}

\includegraphics[width=5in,height=\textheight,keepaspectratio]{/assets/docs/UxB2Tju0zg4nh85hMFZNOwAAAAAAAAAA.png}

\subsubsection{\texorpdfstring{\texttt{\ body\ }}{ body }}\label{parameters-body}

\href{/docs/reference/foundations/content/}{content}

{Required} {{ Positional }}

\phantomsection\label{parameters-body-positional-tooltip}
Positional parameters are specified in order, without names.

The contents of the page(s).

Multiple pages will be created if the content does not fit on a single
page. A new page with the page properties prior to the function
invocation will be created after the body has been typeset.

\href{/docs/reference/layout/pad/}{\pandocbounded{\includesvg[keepaspectratio]{/assets/icons/16-arrow-right.svg}}}

{ Padding } { Previous page }

\href{/docs/reference/layout/pagebreak/}{\pandocbounded{\includesvg[keepaspectratio]{/assets/icons/16-arrow-right.svg}}}

{ Page Break } { Next page }


\title{typst.app/docs/reference/layout/length}

\begin{itemize}
\tightlist
\item
  \href{/docs}{\includesvg[width=0.16667in,height=0.16667in]{/assets/icons/16-docs-dark.svg}}
\item
  \includesvg[width=0.16667in,height=0.16667in]{/assets/icons/16-arrow-right.svg}
\item
  \href{/docs/reference/}{Reference}
\item
  \includesvg[width=0.16667in,height=0.16667in]{/assets/icons/16-arrow-right.svg}
\item
  \href{/docs/reference/layout/}{Layout}
\item
  \includesvg[width=0.16667in,height=0.16667in]{/assets/icons/16-arrow-right.svg}
\item
  \href{/docs/reference/layout/length/}{Length}
\end{itemize}

\section{\texorpdfstring{{ length }}{ length }}\label{summary}

A size or distance, possibly expressed with contextual units.

Typst supports the following length units:

\begin{itemize}
\tightlist
\item
  Points: \texttt{\ }{\texttt{\ 72pt\ }}\texttt{\ }
\item
  Millimeters: \texttt{\ }{\texttt{\ 254mm\ }}\texttt{\ }
\item
  Centimeters: \texttt{\ }{\texttt{\ 2.54cm\ }}\texttt{\ }
\item
  Inches: \texttt{\ }{\texttt{\ 1in\ }}\texttt{\ }
\item
  Relative to font size: \texttt{\ }{\texttt{\ 2.5em\ }}\texttt{\ }
\end{itemize}

You can multiply lengths with and divide them by integers and floats.

\subsection{Example}\label{example}

\begin{verbatim}
#rect(width: 20pt)
#rect(width: 2em)
#rect(width: 1in)

#(3em + 5pt).em \
#(20pt).em \
#(40em + 2pt).abs \
#(5em).abs
\end{verbatim}

\includegraphics[width=5in,height=\textheight,keepaspectratio]{/assets/docs/gpwKHS7y2wIB7BIxGEXoMwAAAAAAAAAA.png}

\subsection{Fields}\label{fields}

\begin{itemize}
\tightlist
\item
  \texttt{\ abs\ } : A length with just the absolute component of the
  current length (that is, excluding the \texttt{\ em\ } component).
\item
  \texttt{\ em\ } : The amount of \texttt{\ em\ } units in this length,
  as a \href{/docs/reference/foundations/float/}{float} .
\end{itemize}

\subsection{\texorpdfstring{{ Definitions
}}{ Definitions }}\label{definitions}

\phantomsection\label{definitions-tooltip}
Functions and types and can have associated definitions. These are
accessed by specifying the function or type, followed by a period, and
then the definition\textquotesingle s name.

\subsubsection{\texorpdfstring{\texttt{\ pt\ }}{ pt }}\label{definitions-pt}

Converts this length to points.

Fails with an error if this length has non-zero \texttt{\ em\ } units
(such as \texttt{\ 5em\ +\ 2pt\ } instead of just \texttt{\ 2pt\ } ).
Use the \texttt{\ abs\ } field (such as in
\texttt{\ (5em\ +\ 2pt).abs.pt()\ } ) to ignore the \texttt{\ em\ }
component of the length (thus converting only its absolute component).

self { . } { pt } (

) -\textgreater{} \href{/docs/reference/foundations/float/}{float}

\subsubsection{\texorpdfstring{\texttt{\ mm\ }}{ mm }}\label{definitions-mm}

Converts this length to millimeters.

Fails with an error if this length has non-zero \texttt{\ em\ } units.
See the
\href{/docs/reference/layout/length/\#definitions-pt}{\texttt{\ pt\ }}
method for more details.

self { . } { mm } (

) -\textgreater{} \href{/docs/reference/foundations/float/}{float}

\subsubsection{\texorpdfstring{\texttt{\ cm\ }}{ cm }}\label{definitions-cm}

Converts this length to centimeters.

Fails with an error if this length has non-zero \texttt{\ em\ } units.
See the
\href{/docs/reference/layout/length/\#definitions-pt}{\texttt{\ pt\ }}
method for more details.

self { . } { cm } (

) -\textgreater{} \href{/docs/reference/foundations/float/}{float}

\subsubsection{\texorpdfstring{\texttt{\ inches\ }}{ inches }}\label{definitions-inches}

Converts this length to inches.

Fails with an error if this length has non-zero \texttt{\ em\ } units.
See the
\href{/docs/reference/layout/length/\#definitions-pt}{\texttt{\ pt\ }}
method for more details.

self { . } { inches } (

) -\textgreater{} \href{/docs/reference/foundations/float/}{float}

\subsubsection{\texorpdfstring{\texttt{\ to-absolute\ }}{ to-absolute }}\label{definitions-to-absolute}

Resolve this length to an absolute length.

self { . } { to-absolute } (

) -\textgreater{} \href{/docs/reference/layout/length/}{length}

\begin{verbatim}
#set text(size: 12pt)
#context [
  #(6pt).to-absolute() \
  #(6pt + 10em).to-absolute() \
  #(10em).to-absolute()
]

#set text(size: 6pt)
#context [
  #(6pt).to-absolute() \
  #(6pt + 10em).to-absolute() \
  #(10em).to-absolute()
]
\end{verbatim}

\includegraphics[width=5in,height=\textheight,keepaspectratio]{/assets/docs/O8f4mxTZz-ziS7eclGAyvgAAAAAAAAAA.png}

\href{/docs/reference/layout/layout/}{\pandocbounded{\includesvg[keepaspectratio]{/assets/icons/16-arrow-right.svg}}}

{ Layout } { Previous page }

\href{/docs/reference/layout/measure/}{\pandocbounded{\includesvg[keepaspectratio]{/assets/icons/16-arrow-right.svg}}}

{ Measure } { Next page }


\title{typst.app/docs/reference/layout/v}

\begin{itemize}
\tightlist
\item
  \href{/docs}{\includesvg[width=0.16667in,height=0.16667in]{/assets/icons/16-docs-dark.svg}}
\item
  \includesvg[width=0.16667in,height=0.16667in]{/assets/icons/16-arrow-right.svg}
\item
  \href{/docs/reference/}{Reference}
\item
  \includesvg[width=0.16667in,height=0.16667in]{/assets/icons/16-arrow-right.svg}
\item
  \href{/docs/reference/layout/}{Layout}
\item
  \includesvg[width=0.16667in,height=0.16667in]{/assets/icons/16-arrow-right.svg}
\item
  \href{/docs/reference/layout/v/}{Spacing (V)}
\end{itemize}

\section{\texorpdfstring{\texttt{\ v\ } {{ Element
}}}{ v   Element }}\label{summary}

\phantomsection\label{element-tooltip}
Element functions can be customized with \texttt{\ set\ } and
\texttt{\ show\ } rules.

Inserts vertical spacing into a flow of blocks.

The spacing can be absolute, relative, or fractional. In the last case,
the remaining space on the page is distributed among all fractional
spacings according to their relative fractions.

\subsection{Example}\label{example}

\begin{verbatim}
#grid(
  rows: 3cm,
  columns: 6,
  gutter: 1fr,
  [A #parbreak() B],
  [A #v(0pt) B],
  [A #v(10pt) B],
  [A #v(0pt, weak: true) B],
  [A #v(40%, weak: true) B],
  [A #v(1fr) B],
)
\end{verbatim}

\includegraphics[width=5in,height=\textheight,keepaspectratio]{/assets/docs/DNC2m_0X9s5xLmHMABxCvgAAAAAAAAAA.png}

\subsection{\texorpdfstring{{ Parameters
}}{ Parameters }}\label{parameters}

\phantomsection\label{parameters-tooltip}
Parameters are the inputs to a function. They are specified in
parentheses after the function name.

{ v } (

{ \href{/docs/reference/layout/relative/}{relative}
\href{/docs/reference/layout/fraction/}{fraction} , } {
\hyperref[parameters-weak]{weak :}
\href{/docs/reference/foundations/bool/}{bool} , }

) -\textgreater{} \href{/docs/reference/foundations/content/}{content}

\subsubsection{\texorpdfstring{\texttt{\ amount\ }}{ amount }}\label{parameters-amount}

\href{/docs/reference/layout/relative/}{relative} {or}
\href{/docs/reference/layout/fraction/}{fraction}

{Required} {{ Positional }}

\phantomsection\label{parameters-amount-positional-tooltip}
Positional parameters are specified in order, without names.

How much spacing to insert.

\subsubsection{\texorpdfstring{\texttt{\ weak\ }}{ weak }}\label{parameters-weak}

\href{/docs/reference/foundations/bool/}{bool}

{{ Settable }}

\phantomsection\label{parameters-weak-settable-tooltip}
Settable parameters can be customized for all following uses of the
function with a \texttt{\ set\ } rule.

If \texttt{\ }{\texttt{\ true\ }}\texttt{\ } , the spacing collapses at
the start or end of a flow. Moreover, from multiple adjacent weak
spacings all but the largest one collapse. Weak spacings will always
collapse adjacent paragraph spacing, even if the paragraph spacing is
larger.

Default: \texttt{\ }{\texttt{\ false\ }}\texttt{\ }

\includesvg[width=0.16667in,height=0.16667in]{/assets/icons/16-arrow-right.svg}
View example

\begin{verbatim}
The following theorem is
foundational to the field:
#v(4pt, weak: true)
$ x^2 + y^2 = r^2 $
#v(4pt, weak: true)
The proof is simple:
\end{verbatim}

\includegraphics[width=5in,height=\textheight,keepaspectratio]{/assets/docs/7Xa6Zl_-zWfWaA6gosM_0QAAAAAAAAAA.png}

\href{/docs/reference/layout/h/}{\pandocbounded{\includesvg[keepaspectratio]{/assets/icons/16-arrow-right.svg}}}

{ Spacing (H) } { Previous page }

\href{/docs/reference/layout/stack/}{\pandocbounded{\includesvg[keepaspectratio]{/assets/icons/16-arrow-right.svg}}}

{ Stack } { Next page }


\title{typst.app/docs/reference/layout/stack}

\begin{itemize}
\tightlist
\item
  \href{/docs}{\includesvg[width=0.16667in,height=0.16667in]{/assets/icons/16-docs-dark.svg}}
\item
  \includesvg[width=0.16667in,height=0.16667in]{/assets/icons/16-arrow-right.svg}
\item
  \href{/docs/reference/}{Reference}
\item
  \includesvg[width=0.16667in,height=0.16667in]{/assets/icons/16-arrow-right.svg}
\item
  \href{/docs/reference/layout/}{Layout}
\item
  \includesvg[width=0.16667in,height=0.16667in]{/assets/icons/16-arrow-right.svg}
\item
  \href{/docs/reference/layout/stack/}{Stack}
\end{itemize}

\section{\texorpdfstring{\texttt{\ stack\ } {{ Element
}}}{ stack   Element }}\label{summary}

\phantomsection\label{element-tooltip}
Element functions can be customized with \texttt{\ set\ } and
\texttt{\ show\ } rules.

Arranges content and spacing horizontally or vertically.

The stack places a list of items along an axis, with optional spacing
between each item.

\subsection{Example}\label{example}

\begin{verbatim}
#stack(
  dir: ttb,
  rect(width: 40pt),
  rect(width: 120pt),
  rect(width: 90pt),
)
\end{verbatim}

\includegraphics[width=5in,height=\textheight,keepaspectratio]{/assets/docs/rblc_gO4o5qSEPJtXD1qPgAAAAAAAAAA.png}

\subsection{\texorpdfstring{{ Parameters
}}{ Parameters }}\label{parameters}

\phantomsection\label{parameters-tooltip}
Parameters are the inputs to a function. They are specified in
parentheses after the function name.

{ stack } (

{ \hyperref[parameters-dir]{dir :}
\href{/docs/reference/layout/direction/}{direction} , } {
\hyperref[parameters-spacing]{spacing :}
\href{/docs/reference/foundations/none/}{none}
\href{/docs/reference/layout/relative/}{relative}
\href{/docs/reference/layout/fraction/}{fraction} , } {
\hyperref[parameters-children]{..}
\href{/docs/reference/layout/relative/}{relative}
\href{/docs/reference/layout/fraction/}{fraction}
\href{/docs/reference/foundations/content/}{content} , }

) -\textgreater{} \href{/docs/reference/foundations/content/}{content}

\subsubsection{\texorpdfstring{\texttt{\ dir\ }}{ dir }}\label{parameters-dir}

\href{/docs/reference/layout/direction/}{direction}

{{ Settable }}

\phantomsection\label{parameters-dir-settable-tooltip}
Settable parameters can be customized for all following uses of the
function with a \texttt{\ set\ } rule.

The direction along which the items are stacked. Possible values are:

\begin{itemize}
\tightlist
\item
  \texttt{\ ltr\ } : Left to right.
\item
  \texttt{\ rtl\ } : Right to left.
\item
  \texttt{\ ttb\ } : Top to bottom.
\item
  \texttt{\ btt\ } : Bottom to top.
\end{itemize}

You can use the \texttt{\ start\ } and \texttt{\ end\ } methods to
obtain the initial and final points (respectively) of a direction, as
\texttt{\ alignment\ } . You can also use the \texttt{\ axis\ } method
to determine whether a direction is
\texttt{\ }{\texttt{\ "horizontal"\ }}\texttt{\ } or
\texttt{\ }{\texttt{\ "vertical"\ }}\texttt{\ } . The \texttt{\ inv\ }
method returns a direction\textquotesingle s inverse direction.

For example,
\texttt{\ ttb\ }{\texttt{\ .\ }}\texttt{\ }{\texttt{\ start\ }}\texttt{\ }{\texttt{\ (\ }}\texttt{\ }{\texttt{\ )\ }}\texttt{\ }
is \texttt{\ top\ } ,
\texttt{\ ttb\ }{\texttt{\ .\ }}\texttt{\ }{\texttt{\ end\ }}\texttt{\ }{\texttt{\ (\ }}\texttt{\ }{\texttt{\ )\ }}\texttt{\ }
is \texttt{\ bottom\ } ,
\texttt{\ ttb\ }{\texttt{\ .\ }}\texttt{\ }{\texttt{\ axis\ }}\texttt{\ }{\texttt{\ (\ }}\texttt{\ }{\texttt{\ )\ }}\texttt{\ }
is \texttt{\ }{\texttt{\ "vertical"\ }}\texttt{\ } and
\texttt{\ ttb\ }{\texttt{\ .\ }}\texttt{\ }{\texttt{\ inv\ }}\texttt{\ }{\texttt{\ (\ }}\texttt{\ }{\texttt{\ )\ }}\texttt{\ }
is equal to \texttt{\ btt\ } .

Default: \texttt{\ ttb\ }

\subsubsection{\texorpdfstring{\texttt{\ spacing\ }}{ spacing }}\label{parameters-spacing}

\href{/docs/reference/foundations/none/}{none} {or}
\href{/docs/reference/layout/relative/}{relative} {or}
\href{/docs/reference/layout/fraction/}{fraction}

{{ Settable }}

\phantomsection\label{parameters-spacing-settable-tooltip}
Settable parameters can be customized for all following uses of the
function with a \texttt{\ set\ } rule.

Spacing to insert between items where no explicit spacing was provided.

Default: \texttt{\ }{\texttt{\ none\ }}\texttt{\ }

\subsubsection{\texorpdfstring{\texttt{\ children\ }}{ children }}\label{parameters-children}

\href{/docs/reference/layout/relative/}{relative} {or}
\href{/docs/reference/layout/fraction/}{fraction} {or}
\href{/docs/reference/foundations/content/}{content}

{Required} {{ Positional }}

\phantomsection\label{parameters-children-positional-tooltip}
Positional parameters are specified in order, without names.

{{ Variadic }}

\phantomsection\label{parameters-children-variadic-tooltip}
Variadic parameters can be specified multiple times.

The children to stack along the axis.

\href{/docs/reference/layout/v/}{\pandocbounded{\includesvg[keepaspectratio]{/assets/icons/16-arrow-right.svg}}}

{ Spacing (V) } { Previous page }

\href{/docs/reference/visualize/}{\pandocbounded{\includesvg[keepaspectratio]{/assets/icons/16-arrow-right.svg}}}

{ Visualize } { Next page }


\title{typst.app/docs/reference/layout/pagebreak}

\begin{itemize}
\tightlist
\item
  \href{/docs}{\includesvg[width=0.16667in,height=0.16667in]{/assets/icons/16-docs-dark.svg}}
\item
  \includesvg[width=0.16667in,height=0.16667in]{/assets/icons/16-arrow-right.svg}
\item
  \href{/docs/reference/}{Reference}
\item
  \includesvg[width=0.16667in,height=0.16667in]{/assets/icons/16-arrow-right.svg}
\item
  \href{/docs/reference/layout/}{Layout}
\item
  \includesvg[width=0.16667in,height=0.16667in]{/assets/icons/16-arrow-right.svg}
\item
  \href{/docs/reference/layout/pagebreak/}{Page Break}
\end{itemize}

\section{\texorpdfstring{\texttt{\ pagebreak\ } {{ Element
}}}{ pagebreak   Element }}\label{summary}

\phantomsection\label{element-tooltip}
Element functions can be customized with \texttt{\ set\ } and
\texttt{\ show\ } rules.

A manual page break.

Must not be used inside any containers.

\subsection{Example}\label{example}

\begin{verbatim}
The next page contains
more details on compound theory.
#pagebreak()

== Compound Theory
In 1984, the first ...
\end{verbatim}

\includegraphics[width=5in,height=\textheight,keepaspectratio]{/assets/docs/MJju6am_GVBgtJWStEY3AwAAAAAAAAAA.png}
\includegraphics[width=5in,height=\textheight,keepaspectratio]{/assets/docs/MJju6am_GVBgtJWStEY3AwAAAAAAAAAB.png}

\subsection{\texorpdfstring{{ Parameters
}}{ Parameters }}\label{parameters}

\phantomsection\label{parameters-tooltip}
Parameters are the inputs to a function. They are specified in
parentheses after the function name.

{ pagebreak } (

{ \hyperref[parameters-weak]{weak :}
\href{/docs/reference/foundations/bool/}{bool} , } {
\hyperref[parameters-to]{to :}
\href{/docs/reference/foundations/none/}{none}
\href{/docs/reference/foundations/str/}{str} , }

) -\textgreater{} \href{/docs/reference/foundations/content/}{content}

\subsubsection{\texorpdfstring{\texttt{\ weak\ }}{ weak }}\label{parameters-weak}

\href{/docs/reference/foundations/bool/}{bool}

{{ Settable }}

\phantomsection\label{parameters-weak-settable-tooltip}
Settable parameters can be customized for all following uses of the
function with a \texttt{\ set\ } rule.

If \texttt{\ }{\texttt{\ true\ }}\texttt{\ } , the page break is skipped
if the current page is already empty.

Default: \texttt{\ }{\texttt{\ false\ }}\texttt{\ }

\subsubsection{\texorpdfstring{\texttt{\ to\ }}{ to }}\label{parameters-to}

\href{/docs/reference/foundations/none/}{none} {or}
\href{/docs/reference/foundations/str/}{str}

{{ Settable }}

\phantomsection\label{parameters-to-settable-tooltip}
Settable parameters can be customized for all following uses of the
function with a \texttt{\ set\ } rule.

If given, ensures that the next page will be an even/odd page, with an
empty page in between if necessary.

\begin{longtable}[]{@{}ll@{}}
\toprule\noalign{}
Variant & Details \\
\midrule\noalign{}
\endhead
\bottomrule\noalign{}
\endlastfoot
\texttt{\ "\ even\ "\ } & Next page will be an even page. \\
\texttt{\ "\ odd\ "\ } & Next page will be an odd page. \\
\end{longtable}

Default: \texttt{\ }{\texttt{\ none\ }}\texttt{\ }

\includesvg[width=0.16667in,height=0.16667in]{/assets/icons/16-arrow-right.svg}
View example

\begin{verbatim}
#set page(height: 30pt)

First.
#pagebreak(to: "odd")
Third.
\end{verbatim}

\includegraphics[width=5in,height=\textheight,keepaspectratio]{/assets/docs/_4CDe0eaU4eyZtVUd1ArigAAAAAAAAAA.png}
\includegraphics[width=5in,height=\textheight,keepaspectratio]{/assets/docs/_4CDe0eaU4eyZtVUd1ArigAAAAAAAAAB.png}
\includegraphics[width=5in,height=\textheight,keepaspectratio]{/assets/docs/_4CDe0eaU4eyZtVUd1ArigAAAAAAAAAC.png}

\href{/docs/reference/layout/page/}{\pandocbounded{\includesvg[keepaspectratio]{/assets/icons/16-arrow-right.svg}}}

{ Page } { Previous page }

\href{/docs/reference/layout/place/}{\pandocbounded{\includesvg[keepaspectratio]{/assets/icons/16-arrow-right.svg}}}

{ Place } { Next page }


\title{typst.app/docs/reference/layout/layout}

\begin{itemize}
\tightlist
\item
  \href{/docs}{\includesvg[width=0.16667in,height=0.16667in]{/assets/icons/16-docs-dark.svg}}
\item
  \includesvg[width=0.16667in,height=0.16667in]{/assets/icons/16-arrow-right.svg}
\item
  \href{/docs/reference/}{Reference}
\item
  \includesvg[width=0.16667in,height=0.16667in]{/assets/icons/16-arrow-right.svg}
\item
  \href{/docs/reference/layout/}{Layout}
\item
  \includesvg[width=0.16667in,height=0.16667in]{/assets/icons/16-arrow-right.svg}
\item
  \href{/docs/reference/layout/layout/}{Layout}
\end{itemize}

\section{\texorpdfstring{\texttt{\ layout\ }}{ layout }}\label{summary}

Provides access to the current outer container\textquotesingle s (or
page\textquotesingle s, if none) dimensions (width and height).

Accepts a function that receives a single parameter, which is a
dictionary with keys \texttt{\ width\ } and \texttt{\ height\ } , both
of type \href{/docs/reference/layout/length/}{\texttt{\ length\ }} . The
function is provided \href{/docs/reference/context/}{context} , meaning
you don\textquotesingle t need to use it in combination with the
\texttt{\ context\ } keyword. This is why
\href{/docs/reference/layout/measure/}{\texttt{\ measure\ }} can be
called in the example below.

\begin{verbatim}
#let text = lorem(30)
#layout(size => [
  #let (height,) = measure(
    block(width: size.width, text),
  )
  This text is #height high with
  the current page width: \
  #text
])
\end{verbatim}

\includegraphics[width=5in,height=\textheight,keepaspectratio]{/assets/docs/SI9ZxtAftdvELQJYlwu_CgAAAAAAAAAA.png}

Note that the \texttt{\ layout\ } function forces its contents into a
\href{/docs/reference/layout/block/}{block} -level container, so
placement relative to the page or pagebreaks are not possible within it.

If the \texttt{\ layout\ } call is placed inside a box with a width of
\texttt{\ }{\texttt{\ 800pt\ }}\texttt{\ } and a height of
\texttt{\ }{\texttt{\ 400pt\ }}\texttt{\ } , then the specified function
will be given the argument
\texttt{\ }{\texttt{\ (\ }}\texttt{\ width\ }{\texttt{\ :\ }}\texttt{\ }{\texttt{\ 800pt\ }}\texttt{\ }{\texttt{\ ,\ }}\texttt{\ height\ }{\texttt{\ :\ }}\texttt{\ }{\texttt{\ 400pt\ }}\texttt{\ }{\texttt{\ )\ }}\texttt{\ }
. If it is placed directly into the page, it receives the
page\textquotesingle s dimensions minus its margins. This is mostly
useful in combination with
\href{/docs/reference/layout/measure/}{measurement} .

You can also use this function to resolve
\href{/docs/reference/layout/ratio/}{\texttt{\ ratio\ }} to fixed
lengths. This might come in handy if you\textquotesingle re building
your own layout abstractions.

\begin{verbatim}
#layout(size => {
  let half = 50% * size.width
  [Half a page is #half wide.]
})
\end{verbatim}

\includegraphics[width=5in,height=\textheight,keepaspectratio]{/assets/docs/1AoOPrEARH2i9ZcdcamicAAAAAAAAAAA.png}

Note that the width or height provided by \texttt{\ layout\ } will be
infinite if the corresponding page dimension is set to
\texttt{\ }{\texttt{\ auto\ }}\texttt{\ } .

\subsection{\texorpdfstring{{ Parameters
}}{ Parameters }}\label{parameters}

\phantomsection\label{parameters-tooltip}
Parameters are the inputs to a function. They are specified in
parentheses after the function name.

{ layout } (

{ \href{/docs/reference/foundations/function/}{function} }

) -\textgreater{} \href{/docs/reference/foundations/content/}{content}

\subsubsection{\texorpdfstring{\texttt{\ func\ }}{ func }}\label{parameters-func}

\href{/docs/reference/foundations/function/}{function}

{Required} {{ Positional }}

\phantomsection\label{parameters-func-positional-tooltip}
Positional parameters are specified in order, without names.

A function to call with the outer container\textquotesingle s size. Its
return value is displayed in the document.

The container\textquotesingle s size is given as a
\href{/docs/reference/foundations/dictionary/}{dictionary} with the keys
\texttt{\ width\ } and \texttt{\ height\ } .

This function is called once for each time the content returned by
\texttt{\ layout\ } appears in the document. This makes it possible to
generate content that depends on the dimensions of its container.

\href{/docs/reference/layout/hide/}{\pandocbounded{\includesvg[keepaspectratio]{/assets/icons/16-arrow-right.svg}}}

{ Hide } { Previous page }

\href{/docs/reference/layout/length/}{\pandocbounded{\includesvg[keepaspectratio]{/assets/icons/16-arrow-right.svg}}}

{ Length } { Next page }


\title{typst.app/docs/reference/layout/block}

\begin{itemize}
\tightlist
\item
  \href{/docs}{\includesvg[width=0.16667in,height=0.16667in]{/assets/icons/16-docs-dark.svg}}
\item
  \includesvg[width=0.16667in,height=0.16667in]{/assets/icons/16-arrow-right.svg}
\item
  \href{/docs/reference/}{Reference}
\item
  \includesvg[width=0.16667in,height=0.16667in]{/assets/icons/16-arrow-right.svg}
\item
  \href{/docs/reference/layout/}{Layout}
\item
  \includesvg[width=0.16667in,height=0.16667in]{/assets/icons/16-arrow-right.svg}
\item
  \href{/docs/reference/layout/block/}{Block}
\end{itemize}

\section{\texorpdfstring{\texttt{\ block\ } {{ Element
}}}{ block   Element }}\label{summary}

\phantomsection\label{element-tooltip}
Element functions can be customized with \texttt{\ set\ } and
\texttt{\ show\ } rules.

A block-level container.

Such a container can be used to separate content, size it, and give it a
background or border.

\subsection{Examples}\label{examples}

With a block, you can give a background to content while still allowing
it to break across multiple pages.

\begin{verbatim}
#set page(height: 100pt)
#block(
  fill: luma(230),
  inset: 8pt,
  radius: 4pt,
  lorem(30),
)
\end{verbatim}

\includegraphics[width=5in,height=\textheight,keepaspectratio]{/assets/docs/ANNbdXVxvjEeHE66qUzAcwAAAAAAAAAA.png}
\includegraphics[width=5in,height=\textheight,keepaspectratio]{/assets/docs/ANNbdXVxvjEeHE66qUzAcwAAAAAAAAAB.png}

Blocks are also useful to force elements that would otherwise be inline
to become block-level, especially when writing show rules.

\begin{verbatim}
#show heading: it => it.body
= Blockless
More text.

#show heading: it => block(it.body)
= Blocky
More text.
\end{verbatim}

\includegraphics[width=5in,height=\textheight,keepaspectratio]{/assets/docs/oxrD9vHAqcb-9gLEkFF_PQAAAAAAAAAA.png}

\subsection{\texorpdfstring{{ Parameters
}}{ Parameters }}\label{parameters}

\phantomsection\label{parameters-tooltip}
Parameters are the inputs to a function. They are specified in
parentheses after the function name.

{ block } (

{ \hyperref[parameters-width]{width :}
\href{/docs/reference/foundations/auto/}{auto}
\href{/docs/reference/layout/relative/}{relative} , } {
\hyperref[parameters-height]{height :}
\href{/docs/reference/foundations/auto/}{auto}
\href{/docs/reference/layout/relative/}{relative}
\href{/docs/reference/layout/fraction/}{fraction} , } {
\hyperref[parameters-breakable]{breakable :}
\href{/docs/reference/foundations/bool/}{bool} , } {
\hyperref[parameters-fill]{fill :}
\href{/docs/reference/foundations/none/}{none}
\href{/docs/reference/visualize/color/}{color}
\href{/docs/reference/visualize/gradient/}{gradient}
\href{/docs/reference/visualize/pattern/}{pattern} , } {
\hyperref[parameters-stroke]{stroke :}
\href{/docs/reference/foundations/none/}{none}
\href{/docs/reference/layout/length/}{length}
\href{/docs/reference/visualize/color/}{color}
\href{/docs/reference/visualize/gradient/}{gradient}
\href{/docs/reference/visualize/stroke/}{stroke}
\href{/docs/reference/visualize/pattern/}{pattern}
\href{/docs/reference/foundations/dictionary/}{dictionary} , } {
\hyperref[parameters-radius]{radius :}
\href{/docs/reference/layout/relative/}{relative}
\href{/docs/reference/foundations/dictionary/}{dictionary} , } {
\hyperref[parameters-inset]{inset :}
\href{/docs/reference/layout/relative/}{relative}
\href{/docs/reference/foundations/dictionary/}{dictionary} , } {
\hyperref[parameters-outset]{outset :}
\href{/docs/reference/layout/relative/}{relative}
\href{/docs/reference/foundations/dictionary/}{dictionary} , } {
\hyperref[parameters-spacing]{spacing :}
\href{/docs/reference/layout/relative/}{relative}
\href{/docs/reference/layout/fraction/}{fraction} , } {
\hyperref[parameters-above]{above :}
\href{/docs/reference/foundations/auto/}{auto}
\href{/docs/reference/layout/relative/}{relative}
\href{/docs/reference/layout/fraction/}{fraction} , } {
\hyperref[parameters-below]{below :}
\href{/docs/reference/foundations/auto/}{auto}
\href{/docs/reference/layout/relative/}{relative}
\href{/docs/reference/layout/fraction/}{fraction} , } {
\hyperref[parameters-clip]{clip :}
\href{/docs/reference/foundations/bool/}{bool} , } {
\hyperref[parameters-sticky]{sticky :}
\href{/docs/reference/foundations/bool/}{bool} , } {
\hyperref[parameters-body]{}
\href{/docs/reference/foundations/none/}{none}
\href{/docs/reference/foundations/content/}{content} , }

) -\textgreater{} \href{/docs/reference/foundations/content/}{content}

\subsubsection{\texorpdfstring{\texttt{\ width\ }}{ width }}\label{parameters-width}

\href{/docs/reference/foundations/auto/}{auto} {or}
\href{/docs/reference/layout/relative/}{relative}

{{ Settable }}

\phantomsection\label{parameters-width-settable-tooltip}
Settable parameters can be customized for all following uses of the
function with a \texttt{\ set\ } rule.

The block\textquotesingle s width.

Default: \texttt{\ }{\texttt{\ auto\ }}\texttt{\ }

\includesvg[width=0.16667in,height=0.16667in]{/assets/icons/16-arrow-right.svg}
View example

\begin{verbatim}
#set align(center)
#block(
  width: 60%,
  inset: 8pt,
  fill: silver,
  lorem(10),
)
\end{verbatim}

\includegraphics[width=5in,height=\textheight,keepaspectratio]{/assets/docs/rmTSlZT-FzVZcPQGVLOIiwAAAAAAAAAA.png}

\subsubsection{\texorpdfstring{\texttt{\ height\ }}{ height }}\label{parameters-height}

\href{/docs/reference/foundations/auto/}{auto} {or}
\href{/docs/reference/layout/relative/}{relative} {or}
\href{/docs/reference/layout/fraction/}{fraction}

{{ Settable }}

\phantomsection\label{parameters-height-settable-tooltip}
Settable parameters can be customized for all following uses of the
function with a \texttt{\ set\ } rule.

The block\textquotesingle s height. When the height is larger than the
remaining space on a page and
\href{/docs/reference/layout/block/\#parameters-breakable}{\texttt{\ breakable\ }}
is \texttt{\ }{\texttt{\ true\ }}\texttt{\ } , the block will continue
on the next page with the remaining height.

Default: \texttt{\ }{\texttt{\ auto\ }}\texttt{\ }

\includesvg[width=0.16667in,height=0.16667in]{/assets/icons/16-arrow-right.svg}
View example

\begin{verbatim}
#set page(height: 80pt)
#set align(center)
#block(
  width: 80%,
  height: 150%,
  fill: aqua,
)
\end{verbatim}

\includegraphics[width=5in,height=\textheight,keepaspectratio]{/assets/docs/lezx_tGBIjN0y72kerj7yQAAAAAAAAAA.png}
\includegraphics[width=5in,height=\textheight,keepaspectratio]{/assets/docs/lezx_tGBIjN0y72kerj7yQAAAAAAAAAB.png}

\subsubsection{\texorpdfstring{\texttt{\ breakable\ }}{ breakable }}\label{parameters-breakable}

\href{/docs/reference/foundations/bool/}{bool}

{{ Settable }}

\phantomsection\label{parameters-breakable-settable-tooltip}
Settable parameters can be customized for all following uses of the
function with a \texttt{\ set\ } rule.

Whether the block can be broken and continue on the next page.

Default: \texttt{\ }{\texttt{\ true\ }}\texttt{\ }

\includesvg[width=0.16667in,height=0.16667in]{/assets/icons/16-arrow-right.svg}
View example

\begin{verbatim}
#set page(height: 80pt)
The following block will
jump to its own page.
#block(
  breakable: false,
  lorem(15),
)
\end{verbatim}

\includegraphics[width=5in,height=\textheight,keepaspectratio]{/assets/docs/I4HMzOAjAUbW-RK0a_YVHAAAAAAAAAAA.png}
\includegraphics[width=5in,height=\textheight,keepaspectratio]{/assets/docs/I4HMzOAjAUbW-RK0a_YVHAAAAAAAAAAB.png}

\subsubsection{\texorpdfstring{\texttt{\ fill\ }}{ fill }}\label{parameters-fill}

\href{/docs/reference/foundations/none/}{none} {or}
\href{/docs/reference/visualize/color/}{color} {or}
\href{/docs/reference/visualize/gradient/}{gradient} {or}
\href{/docs/reference/visualize/pattern/}{pattern}

{{ Settable }}

\phantomsection\label{parameters-fill-settable-tooltip}
Settable parameters can be customized for all following uses of the
function with a \texttt{\ set\ } rule.

The block\textquotesingle s background color. See the
\href{/docs/reference/visualize/rect/\#parameters-fill}{rectangle\textquotesingle s
documentation} for more details.

Default: \texttt{\ }{\texttt{\ none\ }}\texttt{\ }

\subsubsection{\texorpdfstring{\texttt{\ stroke\ }}{ stroke }}\label{parameters-stroke}

\href{/docs/reference/foundations/none/}{none} {or}
\href{/docs/reference/layout/length/}{length} {or}
\href{/docs/reference/visualize/color/}{color} {or}
\href{/docs/reference/visualize/gradient/}{gradient} {or}
\href{/docs/reference/visualize/stroke/}{stroke} {or}
\href{/docs/reference/visualize/pattern/}{pattern} {or}
\href{/docs/reference/foundations/dictionary/}{dictionary}

{{ Settable }}

\phantomsection\label{parameters-stroke-settable-tooltip}
Settable parameters can be customized for all following uses of the
function with a \texttt{\ set\ } rule.

The block\textquotesingle s border color. See the
\href{/docs/reference/visualize/rect/\#parameters-stroke}{rectangle\textquotesingle s
documentation} for more details.

Default:
\texttt{\ }{\texttt{\ (\ }}\texttt{\ }{\texttt{\ :\ }}\texttt{\ }{\texttt{\ )\ }}\texttt{\ }

\subsubsection{\texorpdfstring{\texttt{\ radius\ }}{ radius }}\label{parameters-radius}

\href{/docs/reference/layout/relative/}{relative} {or}
\href{/docs/reference/foundations/dictionary/}{dictionary}

{{ Settable }}

\phantomsection\label{parameters-radius-settable-tooltip}
Settable parameters can be customized for all following uses of the
function with a \texttt{\ set\ } rule.

How much to round the block\textquotesingle s corners. See the
\href{/docs/reference/visualize/rect/\#parameters-radius}{rectangle\textquotesingle s
documentation} for more details.

Default:
\texttt{\ }{\texttt{\ (\ }}\texttt{\ }{\texttt{\ :\ }}\texttt{\ }{\texttt{\ )\ }}\texttt{\ }

\subsubsection{\texorpdfstring{\texttt{\ inset\ }}{ inset }}\label{parameters-inset}

\href{/docs/reference/layout/relative/}{relative} {or}
\href{/docs/reference/foundations/dictionary/}{dictionary}

{{ Settable }}

\phantomsection\label{parameters-inset-settable-tooltip}
Settable parameters can be customized for all following uses of the
function with a \texttt{\ set\ } rule.

How much to pad the block\textquotesingle s content. See the
\href{/docs/reference/layout/box/\#parameters-inset}{box\textquotesingle s
documentation} for more details.

Default:
\texttt{\ }{\texttt{\ (\ }}\texttt{\ }{\texttt{\ :\ }}\texttt{\ }{\texttt{\ )\ }}\texttt{\ }

\subsubsection{\texorpdfstring{\texttt{\ outset\ }}{ outset }}\label{parameters-outset}

\href{/docs/reference/layout/relative/}{relative} {or}
\href{/docs/reference/foundations/dictionary/}{dictionary}

{{ Settable }}

\phantomsection\label{parameters-outset-settable-tooltip}
Settable parameters can be customized for all following uses of the
function with a \texttt{\ set\ } rule.

How much to expand the block\textquotesingle s size without affecting
the layout. See the
\href{/docs/reference/layout/box/\#parameters-outset}{box\textquotesingle s
documentation} for more details.

Default:
\texttt{\ }{\texttt{\ (\ }}\texttt{\ }{\texttt{\ :\ }}\texttt{\ }{\texttt{\ )\ }}\texttt{\ }

\subsubsection{\texorpdfstring{\texttt{\ spacing\ }}{ spacing }}\label{parameters-spacing}

\href{/docs/reference/layout/relative/}{relative} {or}
\href{/docs/reference/layout/fraction/}{fraction}

{{ Settable }}

\phantomsection\label{parameters-spacing-settable-tooltip}
Settable parameters can be customized for all following uses of the
function with a \texttt{\ set\ } rule.

The spacing around the block. When
\texttt{\ }{\texttt{\ auto\ }}\texttt{\ } , inherits the paragraph
\href{/docs/reference/model/par/\#parameters-spacing}{\texttt{\ spacing\ }}
.

For two adjacent blocks, the larger of the first block\textquotesingle s
\texttt{\ above\ } and the second block\textquotesingle s
\texttt{\ below\ } spacing wins. Moreover, block spacing takes
precedence over paragraph
\href{/docs/reference/model/par/\#parameters-spacing}{\texttt{\ spacing\ }}
.

Note that this is only a shorthand to set \texttt{\ above\ } and
\texttt{\ below\ } to the same value. Since the values for
\texttt{\ above\ } and \texttt{\ below\ } might differ, a
\href{/docs/reference/context/}{context} block only provides access to
\texttt{\ block\ }{\texttt{\ .\ }}\texttt{\ above\ } and
\texttt{\ block\ }{\texttt{\ .\ }}\texttt{\ below\ } , not to
\texttt{\ block\ }{\texttt{\ .\ }}\texttt{\ spacing\ } directly.

This property can be used in combination with a show rule to adjust the
spacing around arbitrary block-level elements.

Default: \texttt{\ }{\texttt{\ 1.2em\ }}\texttt{\ }

\includesvg[width=0.16667in,height=0.16667in]{/assets/icons/16-arrow-right.svg}
View example

\begin{verbatim}
#set align(center)
#show math.equation: set block(above: 8pt, below: 16pt)

This sum of $x$ and $y$:
$ x + y = z $
A second paragraph.
\end{verbatim}

\includegraphics[width=5in,height=\textheight,keepaspectratio]{/assets/docs/-Z0A6wte5TbEZ6mEwTPvngAAAAAAAAAA.png}

\subsubsection{\texorpdfstring{\texttt{\ above\ }}{ above }}\label{parameters-above}

\href{/docs/reference/foundations/auto/}{auto} {or}
\href{/docs/reference/layout/relative/}{relative} {or}
\href{/docs/reference/layout/fraction/}{fraction}

{{ Settable }}

\phantomsection\label{parameters-above-settable-tooltip}
Settable parameters can be customized for all following uses of the
function with a \texttt{\ set\ } rule.

The spacing between this block and its predecessor.

Default: \texttt{\ }{\texttt{\ auto\ }}\texttt{\ }

\subsubsection{\texorpdfstring{\texttt{\ below\ }}{ below }}\label{parameters-below}

\href{/docs/reference/foundations/auto/}{auto} {or}
\href{/docs/reference/layout/relative/}{relative} {or}
\href{/docs/reference/layout/fraction/}{fraction}

{{ Settable }}

\phantomsection\label{parameters-below-settable-tooltip}
Settable parameters can be customized for all following uses of the
function with a \texttt{\ set\ } rule.

The spacing between this block and its successor.

Default: \texttt{\ }{\texttt{\ auto\ }}\texttt{\ }

\subsubsection{\texorpdfstring{\texttt{\ clip\ }}{ clip }}\label{parameters-clip}

\href{/docs/reference/foundations/bool/}{bool}

{{ Settable }}

\phantomsection\label{parameters-clip-settable-tooltip}
Settable parameters can be customized for all following uses of the
function with a \texttt{\ set\ } rule.

Whether to clip the content inside the block.

Clipping is useful when the block\textquotesingle s content is larger
than the block itself, as any content that exceeds the
block\textquotesingle s bounds will be hidden.

Default: \texttt{\ }{\texttt{\ false\ }}\texttt{\ }

\includesvg[width=0.16667in,height=0.16667in]{/assets/icons/16-arrow-right.svg}
View example

\begin{verbatim}
#block(
  width: 50pt,
  height: 50pt,
  clip: true,
  image("tiger.jpg", width: 100pt, height: 100pt)
)
\end{verbatim}

\includegraphics[width=5in,height=\textheight,keepaspectratio]{/assets/docs/VV4XHW5eLH_lso6MwHK6pQAAAAAAAAAA.png}

\subsubsection{\texorpdfstring{\texttt{\ sticky\ }}{ sticky }}\label{parameters-sticky}

\href{/docs/reference/foundations/bool/}{bool}

{{ Settable }}

\phantomsection\label{parameters-sticky-settable-tooltip}
Settable parameters can be customized for all following uses of the
function with a \texttt{\ set\ } rule.

Whether this block must stick to the following one, with no break in
between.

This is, by default, set on heading blocks to prevent orphaned headings
at the bottom of the page.

Default: \texttt{\ }{\texttt{\ false\ }}\texttt{\ }

\includesvg[width=0.16667in,height=0.16667in]{/assets/icons/16-arrow-right.svg}
View example

\begin{verbatim}
// Disable stickiness of headings.
#show heading: set block(sticky: false)
#lorem(20)

= Chapter
#lorem(10)
\end{verbatim}

\includegraphics[width=5in,height=\textheight,keepaspectratio]{/assets/docs/9rTrIlbIWN6fRV2-gOoijQAAAAAAAAAA.png}
\includegraphics[width=5in,height=\textheight,keepaspectratio]{/assets/docs/9rTrIlbIWN6fRV2-gOoijQAAAAAAAAAB.png}

\subsubsection{\texorpdfstring{\texttt{\ body\ }}{ body }}\label{parameters-body}

\href{/docs/reference/foundations/none/}{none} {or}
\href{/docs/reference/foundations/content/}{content}

{{ Positional }}

\phantomsection\label{parameters-body-positional-tooltip}
Positional parameters are specified in order, without names.

{{ Settable }}

\phantomsection\label{parameters-body-settable-tooltip}
Settable parameters can be customized for all following uses of the
function with a \texttt{\ set\ } rule.

The contents of the block.

Default: \texttt{\ }{\texttt{\ none\ }}\texttt{\ }

\href{/docs/reference/layout/angle/}{\pandocbounded{\includesvg[keepaspectratio]{/assets/icons/16-arrow-right.svg}}}

{ Angle } { Previous page }

\href{/docs/reference/layout/box/}{\pandocbounded{\includesvg[keepaspectratio]{/assets/icons/16-arrow-right.svg}}}

{ Box } { Next page }


\title{typst.app/docs/reference/layout/ratio}

\begin{itemize}
\tightlist
\item
  \href{/docs}{\includesvg[width=0.16667in,height=0.16667in]{/assets/icons/16-docs-dark.svg}}
\item
  \includesvg[width=0.16667in,height=0.16667in]{/assets/icons/16-arrow-right.svg}
\item
  \href{/docs/reference/}{Reference}
\item
  \includesvg[width=0.16667in,height=0.16667in]{/assets/icons/16-arrow-right.svg}
\item
  \href{/docs/reference/layout/}{Layout}
\item
  \includesvg[width=0.16667in,height=0.16667in]{/assets/icons/16-arrow-right.svg}
\item
  \href{/docs/reference/layout/ratio/}{Ratio}
\end{itemize}

\section{\texorpdfstring{{ ratio }}{ ratio }}\label{summary}

A ratio of a whole.

Written as a number, followed by a percent sign.

\subsection{Example}\label{example}

\begin{verbatim}
#set align(center)
#scale(x: 150%)[
  Scaled apart.
]
\end{verbatim}

\includegraphics[width=5in,height=\textheight,keepaspectratio]{/assets/docs/xEgSJZQe3kQz-XQhwaSthwAAAAAAAAAA.png}

\href{/docs/reference/layout/place/}{\pandocbounded{\includesvg[keepaspectratio]{/assets/icons/16-arrow-right.svg}}}

{ Place } { Previous page }

\href{/docs/reference/layout/relative/}{\pandocbounded{\includesvg[keepaspectratio]{/assets/icons/16-arrow-right.svg}}}

{ Relative Length } { Next page }


