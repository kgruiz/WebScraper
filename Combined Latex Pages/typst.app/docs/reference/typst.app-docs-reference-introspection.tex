\title{typst.app/docs/reference/introspection/metadata}

\begin{itemize}
\tightlist
\item
  \href{/docs}{\includesvg[width=0.16667in,height=0.16667in]{/assets/icons/16-docs-dark.svg}}
\item
  \includesvg[width=0.16667in,height=0.16667in]{/assets/icons/16-arrow-right.svg}
\item
  \href{/docs/reference/}{Reference}
\item
  \includesvg[width=0.16667in,height=0.16667in]{/assets/icons/16-arrow-right.svg}
\item
  \href{/docs/reference/introspection/}{Introspection}
\item
  \includesvg[width=0.16667in,height=0.16667in]{/assets/icons/16-arrow-right.svg}
\item
  \href{/docs/reference/introspection/metadata/}{Metadata}
\end{itemize}

\section{\texorpdfstring{\texttt{\ metadata\ } {{ Element
}}}{ metadata   Element }}\label{summary}

\phantomsection\label{element-tooltip}
Element functions can be customized with \texttt{\ set\ } and
\texttt{\ show\ } rules.

Exposes a value to the query system without producing visible content.

This element can be retrieved with the
\href{/docs/reference/introspection/query/}{\texttt{\ query\ }} function
and from the command line with
\href{/docs/reference/introspection/query/\#command-line-queries}{\texttt{\ typst\ query\ }}
. Its purpose is to expose an arbitrary value to the introspection
system. To identify a metadata value among others, you can attach a
\href{/docs/reference/foundations/label/}{\texttt{\ label\ }} to it and
query for that label.

The \texttt{\ metadata\ } element is especially useful for command line
queries because it allows you to expose arbitrary values to the outside
world.

\begin{verbatim}
// Put metadata somewhere.
#metadata("This is a note") <note>

// And find it from anywhere else.
#context {
  query(<note>).first().value
}
\end{verbatim}

\includegraphics[width=5in,height=\textheight,keepaspectratio]{/assets/docs/sbF_Ac863-gI1m3qoL9avwAAAAAAAAAA.png}

\subsection{\texorpdfstring{{ Parameters
}}{ Parameters }}\label{parameters}

\phantomsection\label{parameters-tooltip}
Parameters are the inputs to a function. They are specified in
parentheses after the function name.

{ metadata } (

{ { any } }

) -\textgreater{} \href{/docs/reference/foundations/content/}{content}

\subsubsection{\texorpdfstring{\texttt{\ value\ }}{ value }}\label{parameters-value}

{ any }

{Required} {{ Positional }}

\phantomsection\label{parameters-value-positional-tooltip}
Positional parameters are specified in order, without names.

The value to embed into the document.

\href{/docs/reference/introspection/location/}{\pandocbounded{\includesvg[keepaspectratio]{/assets/icons/16-arrow-right.svg}}}

{ Location } { Previous page }

\href{/docs/reference/introspection/query/}{\pandocbounded{\includesvg[keepaspectratio]{/assets/icons/16-arrow-right.svg}}}

{ Query } { Next page }


\title{typst.app/docs/reference/introspection/counter}

\begin{itemize}
\tightlist
\item
  \href{/docs}{\includesvg[width=0.16667in,height=0.16667in]{/assets/icons/16-docs-dark.svg}}
\item
  \includesvg[width=0.16667in,height=0.16667in]{/assets/icons/16-arrow-right.svg}
\item
  \href{/docs/reference/}{Reference}
\item
  \includesvg[width=0.16667in,height=0.16667in]{/assets/icons/16-arrow-right.svg}
\item
  \href{/docs/reference/introspection/}{Introspection}
\item
  \includesvg[width=0.16667in,height=0.16667in]{/assets/icons/16-arrow-right.svg}
\item
  \href{/docs/reference/introspection/counter/}{Counter}
\end{itemize}

\section{\texorpdfstring{{ counter }}{ counter }}\label{summary}

Counts through pages, elements, and more.

With the counter function, you can access and modify counters for pages,
headings, figures, and more. Moreover, you can define custom counters
for other things you want to count.

Since counters change throughout the course of the document, their
current value is \emph{contextual.} It is recommended to read the
chapter on \href{/docs/reference/context/}{context} before continuing
here.

\subsection{Accessing a counter}\label{accessing}

To access the raw value of a counter, we can use the
\href{/docs/reference/introspection/counter/\#definitions-get}{\texttt{\ get\ }}
function. This function returns an
\href{/docs/reference/foundations/array/}{array} : Counters can have
multiple levels (in the case of headings for sections, subsections, and
so on), and each item in the array corresponds to one level.

\begin{verbatim}
#set heading(numbering: "1.")

= Introduction
Raw value of heading counter is
#context counter(heading).get()
\end{verbatim}

\includegraphics[width=5in,height=\textheight,keepaspectratio]{/assets/docs/jqVSznl_yGBcNN9ecF8OVAAAAAAAAAAA.png}

\subsection{Displaying a counter}\label{displaying}

Often, we want to display the value of a counter in a more
human-readable way. To do that, we can call the
\href{/docs/reference/introspection/counter/\#definitions-display}{\texttt{\ display\ }}
function on the counter. This function retrieves the current counter
value and formats it either with a provided or with an automatically
inferred \href{/docs/reference/model/numbering/}{numbering} .

\begin{verbatim}
#set heading(numbering: "1.")

= Introduction
Some text here.

= Background
The current value is: #context {
  counter(heading).display()
}

Or in roman numerals: #context {
  counter(heading).display("I")
}
\end{verbatim}

\includegraphics[width=5in,height=\textheight,keepaspectratio]{/assets/docs/7EUi61p1PXmzQyka_2NqiAAAAAAAAAAA.png}

\subsection{Modifying a counter}\label{modifying}

To modify a counter, you can use the \texttt{\ step\ } and
\texttt{\ update\ } methods:

\begin{itemize}
\item
  The \texttt{\ step\ } method increases the value of the counter by
  one. Because counters can have multiple levels , it optionally takes a
  \texttt{\ level\ } argument. If given, the counter steps at the given
  depth.
\item
  The \texttt{\ update\ } method allows you to arbitrarily modify the
  counter. In its basic form, you give it an integer (or an array for
  multiple levels). For more flexibility, you can instead also give it a
  function that receives the current value and returns a new value.
\end{itemize}

The heading counter is stepped before the heading is displayed, so
\texttt{\ Analysis\ } gets the number seven even though the counter is
at six after the second update.

\begin{verbatim}
#set heading(numbering: "1.")

= Introduction
#counter(heading).step()

= Background
#counter(heading).update(3)
#counter(heading).update(n => n * 2)

= Analysis
Let's skip 7.1.
#counter(heading).step(level: 2)

== Analysis
Still at #context {
  counter(heading).display()
}
\end{verbatim}

\includegraphics[width=5in,height=\textheight,keepaspectratio]{/assets/docs/EOYqv5YWVpmiQyBJoYpqQAAAAAAAAAAA.png}

\subsection{Page counter}\label{page-counter}

The page counter is special. It is automatically stepped at each
pagebreak. But like other counters, you can also step it manually. For
example, you could have Roman page numbers for your preface, then switch
to Arabic page numbers for your main content and reset the page counter
to one.

\begin{verbatim}
#set page(numbering: "(i)")

= Preface
The preface is numbered with
roman numerals.

#set page(numbering: "1 / 1")
#counter(page).update(1)

= Main text
Here, the counter is reset to one.
We also display both the current
page and total number of pages in
Arabic numbers.
\end{verbatim}

\includegraphics[width=5in,height=\textheight,keepaspectratio]{/assets/docs/PDCorO6nPZEoa3HjHUVgRwAAAAAAAAAA.png}
\includegraphics[width=5in,height=\textheight,keepaspectratio]{/assets/docs/PDCorO6nPZEoa3HjHUVgRwAAAAAAAAAB.png}

\subsection{Custom counters}\label{custom-counters}

To define your own counter, call the \texttt{\ counter\ } function with
a string as a key. This key identifies the counter globally.

\begin{verbatim}
#let mine = counter("mycounter")
#context mine.display() \
#mine.step()
#context mine.display() \
#mine.update(c => c * 3)
#context mine.display()
\end{verbatim}

\includegraphics[width=5in,height=\textheight,keepaspectratio]{/assets/docs/CxXLMyCvJp2FnmacPN3WUgAAAAAAAAAA.png}

\subsection{How to step}\label{how-to-step}

When you define and use a custom counter, in general, you should first
step the counter and then display it. This way, the stepping behaviour
of a counter can depend on the element it is stepped for. If you were
writing a counter for, let\textquotesingle s say, theorems, your
theorem\textquotesingle s definition would thus first include the
counter step and only then display the counter and the
theorem\textquotesingle s contents.

\begin{verbatim}
#let c = counter("theorem")
#let theorem(it) = block[
  #c.step()
  *Theorem #context c.display():*
  #it
]

#theorem[$1 = 1$]
#theorem[$2 < 3$]
\end{verbatim}

\includegraphics[width=5in,height=\textheight,keepaspectratio]{/assets/docs/af6Y7nOR_IldvYHIWDmkIQAAAAAAAAAA.png}

The rationale behind this is best explained on the example of the
heading counter: An update to the heading counter depends on the
heading\textquotesingle s level. By stepping directly before the
heading, we can correctly step from \texttt{\ 1\ } to \texttt{\ 1.1\ }
when encountering a level 2 heading. If we were to step after the
heading, we wouldn\textquotesingle t know what to step to.

Because counters should always be stepped before the elements they
count, they always start at zero. This way, they are at one for the
first display (which happens after the first step).

\subsection{Time travel}\label{time-travel}

Counters can travel through time! You can find out the final value of
the counter before it is reached and even determine what the value was
at any particular location in the document.

\begin{verbatim}
#let mine = counter("mycounter")

= Values
#context [
  Value here: #mine.get() \
  At intro: #mine.at(<intro>) \
  Final value: #mine.final()
]

#mine.update(n => n + 3)

= Introduction <intro>
#lorem(10)

#mine.step()
#mine.step()
\end{verbatim}

\includegraphics[width=5in,height=\textheight,keepaspectratio]{/assets/docs/wodRGpSsJgDZtfsMk_GNgwAAAAAAAAAA.png}

\subsection{Other kinds of state}\label{other-state}

The \texttt{\ counter\ } type is closely related to
\href{/docs/reference/introspection/state/}{state} type. Read its
documentation for more details on state management in Typst and why it
doesn\textquotesingle t just use normal variables for counters.

\subsection{\texorpdfstring{Constructor
{}}{Constructor }}\label{constructor}

\phantomsection\label{constructor-constructor-tooltip}
If a type has a constructor, you can call it like a function to create a
new value of the type.

Create a new counter identified by a key.

{ counter } (

{ \href{/docs/reference/foundations/str/}{str}
\href{/docs/reference/foundations/label/}{label}
\href{/docs/reference/foundations/selector/}{selector}
\href{/docs/reference/introspection/location/}{location}
\href{/docs/reference/foundations/function/}{function} }

) -\textgreater{} \href{/docs/reference/introspection/counter/}{counter}

\paragraph{\texorpdfstring{\texttt{\ key\ }}{ key }}\label{constructor-key}

\href{/docs/reference/foundations/str/}{str} {or}
\href{/docs/reference/foundations/label/}{label} {or}
\href{/docs/reference/foundations/selector/}{selector} {or}
\href{/docs/reference/introspection/location/}{location} {or}
\href{/docs/reference/foundations/function/}{function}

{Required} {{ Positional }}

\phantomsection\label{constructor-key-positional-tooltip}
Positional parameters are specified in order, without names.

The key that identifies this counter.

\begin{itemize}
\tightlist
\item
  If it is a string, creates a custom counter that is only affected by
  manual updates,
\item
  If it is the \href{/docs/reference/layout/page/}{\texttt{\ page\ }}
  function, counts through pages,
\item
  If it is a \href{/docs/reference/foundations/selector/}{selector} ,
  counts through elements that matches with the selector. For example,

  \begin{itemize}
  \tightlist
  \item
    provide an element function: counts elements of that type,
  \item
    provide a
    \href{/docs/reference/foundations/label/}{\texttt{\ }{\texttt{\ \textless{}label\textgreater{}\ }}\texttt{\ }}
    : counts elements with that label.
  \end{itemize}
\end{itemize}

\subsection{\texorpdfstring{{ Definitions
}}{ Definitions }}\label{definitions}

\phantomsection\label{definitions-tooltip}
Functions and types and can have associated definitions. These are
accessed by specifying the function or type, followed by a period, and
then the definition\textquotesingle s name.

\subsubsection{\texorpdfstring{\texttt{\ get\ } {{ Contextual
}}}{ get   Contextual }}\label{definitions-get}

\phantomsection\label{definitions-get-contextual-tooltip}
Contextual functions can only be used when the context is known

Retrieves the value of the counter at the current location. Always
returns an array of integers, even if the counter has just one number.

This is equivalent to
\texttt{\ counter\ }{\texttt{\ .\ }}\texttt{\ }{\texttt{\ at\ }}\texttt{\ }{\texttt{\ (\ }}\texttt{\ }{\texttt{\ here\ }}\texttt{\ }{\texttt{\ (\ }}\texttt{\ }{\texttt{\ )\ }}\texttt{\ }{\texttt{\ )\ }}\texttt{\ }
.

self { . } { get } (

) -\textgreater{} \href{/docs/reference/foundations/int/}{int}
\href{/docs/reference/foundations/array/}{array}

\subsubsection{\texorpdfstring{\texttt{\ display\ } {{ Contextual
}}}{ display   Contextual }}\label{definitions-display}

\phantomsection\label{definitions-display-contextual-tooltip}
Contextual functions can only be used when the context is known

Displays the current value of the counter with a numbering and returns
the formatted output.

\emph{Compatibility:} For compatibility with Typst 0.10 and lower, this
function also works without an established context. Then, it will create
opaque contextual content rather than directly returning the output of
the numbering. This behaviour will be removed in a future release.

self { . } { display } (

{ \href{/docs/reference/foundations/auto/}{auto}
\href{/docs/reference/foundations/str/}{str}
\href{/docs/reference/foundations/function/}{function} , } {
\hyperref[definitions-display-parameters-both]{both :}
\href{/docs/reference/foundations/bool/}{bool} , }

) -\textgreater{} { any }

\paragraph{\texorpdfstring{\texttt{\ numbering\ }}{ numbering }}\label{definitions-display-numbering}

\href{/docs/reference/foundations/auto/}{auto} {or}
\href{/docs/reference/foundations/str/}{str} {or}
\href{/docs/reference/foundations/function/}{function}

{{ Positional }}

\phantomsection\label{definitions-display-numbering-positional-tooltip}
Positional parameters are specified in order, without names.

A \href{/docs/reference/model/numbering/}{numbering pattern or a
function} , which specifies how to display the counter. If given a
function, that function receives each number of the counter as a
separate argument. If the amount of numbers varies, e.g. for the heading
argument, you can use an
\href{/docs/reference/foundations/arguments/}{argument sink} .

If this is omitted or set to \texttt{\ }{\texttt{\ auto\ }}\texttt{\ } ,
displays the counter with the numbering style for the counted element or
with the pattern \texttt{\ }{\texttt{\ "1.1"\ }}\texttt{\ } if no such
style exists.

Default: \texttt{\ }{\texttt{\ auto\ }}\texttt{\ }

\paragraph{\texorpdfstring{\texttt{\ both\ }}{ both }}\label{definitions-display-both}

\href{/docs/reference/foundations/bool/}{bool}

If enabled, displays the current and final top-level count together.
Both can be styled through a single numbering pattern. This is used by
the page numbering property to display the current and total number of
pages when a pattern like \texttt{\ }{\texttt{\ "1\ /\ 1"\ }}\texttt{\ }
is given.

Default: \texttt{\ }{\texttt{\ false\ }}\texttt{\ }

\subsubsection{\texorpdfstring{\texttt{\ at\ } {{ Contextual
}}}{ at   Contextual }}\label{definitions-at}

\phantomsection\label{definitions-at-contextual-tooltip}
Contextual functions can only be used when the context is known

Retrieves the value of the counter at the given location. Always returns
an array of integers, even if the counter has just one number.

The \texttt{\ selector\ } must match exactly one element in the
document. The most useful kinds of selectors for this are
\href{/docs/reference/foundations/label/}{labels} and
\href{/docs/reference/introspection/location/}{locations} .

\emph{Compatibility:} For compatibility with Typst 0.10 and lower, this
function also works without a known context if the \texttt{\ selector\ }
is a location. This behaviour will be removed in a future release.

self { . } { at } (

{ \href{/docs/reference/foundations/label/}{label}
\href{/docs/reference/foundations/selector/}{selector}
\href{/docs/reference/introspection/location/}{location}
\href{/docs/reference/foundations/function/}{function} }

) -\textgreater{} \href{/docs/reference/foundations/int/}{int}
\href{/docs/reference/foundations/array/}{array}

\paragraph{\texorpdfstring{\texttt{\ selector\ }}{ selector }}\label{definitions-at-selector}

\href{/docs/reference/foundations/label/}{label} {or}
\href{/docs/reference/foundations/selector/}{selector} {or}
\href{/docs/reference/introspection/location/}{location} {or}
\href{/docs/reference/foundations/function/}{function}

{Required} {{ Positional }}

\phantomsection\label{definitions-at-selector-positional-tooltip}
Positional parameters are specified in order, without names.

The place at which the counter\textquotesingle s value should be
retrieved.

\subsubsection{\texorpdfstring{\texttt{\ final\ } {{ Contextual
}}}{ final   Contextual }}\label{definitions-final}

\phantomsection\label{definitions-final-contextual-tooltip}
Contextual functions can only be used when the context is known

Retrieves the value of the counter at the end of the document. Always
returns an array of integers, even if the counter has just one number.

self { . } { final } (

{ \href{/docs/reference/foundations/none/}{none}
\href{/docs/reference/introspection/location/}{location} }

) -\textgreater{} \href{/docs/reference/foundations/int/}{int}
\href{/docs/reference/foundations/array/}{array}

\paragraph{\texorpdfstring{\texttt{\ location\ }}{ location }}\label{definitions-final-location}

\href{/docs/reference/foundations/none/}{none} {or}
\href{/docs/reference/introspection/location/}{location}

{{ Positional }}

\phantomsection\label{definitions-final-location-positional-tooltip}
Positional parameters are specified in order, without names.

\emph{Compatibility:} This argument is deprecated. It only exists for
compatibility with Typst 0.10 and lower and shouldn\textquotesingle t be
used anymore.

Default: \texttt{\ }{\texttt{\ none\ }}\texttt{\ }

\subsubsection{\texorpdfstring{\texttt{\ step\ }}{ step }}\label{definitions-step}

Increases the value of the counter by one.

The update will be in effect at the position where the returned content
is inserted into the document. If you don\textquotesingle t put the
output into the document, nothing happens! This would be the case, for
example, if you write
\texttt{\ }{\texttt{\ let\ }}\texttt{\ \_\ }{\texttt{\ =\ }}\texttt{\ }{\texttt{\ counter\ }}\texttt{\ }{\texttt{\ (\ }}\texttt{\ page\ }{\texttt{\ )\ }}\texttt{\ }{\texttt{\ .\ }}\texttt{\ }{\texttt{\ step\ }}\texttt{\ }{\texttt{\ (\ }}\texttt{\ }{\texttt{\ )\ }}\texttt{\ }
. Counter updates are always applied in layout order and in that case,
Typst wouldn\textquotesingle t know when to step the counter.

self { . } { step } (

{ \hyperref[definitions-step-parameters-level]{level :}
\href{/docs/reference/foundations/int/}{int} }

) -\textgreater{} \href{/docs/reference/foundations/content/}{content}

\paragraph{\texorpdfstring{\texttt{\ level\ }}{ level }}\label{definitions-step-level}

\href{/docs/reference/foundations/int/}{int}

The depth at which to step the counter. Defaults to
\texttt{\ }{\texttt{\ 1\ }}\texttt{\ } .

Default: \texttt{\ }{\texttt{\ 1\ }}\texttt{\ }

\subsubsection{\texorpdfstring{\texttt{\ update\ }}{ update }}\label{definitions-update}

Updates the value of the counter.

Just like with \texttt{\ step\ } , the update only occurs if you put the
resulting content into the document.

self { . } { update } (

{ \href{/docs/reference/foundations/int/}{int}
\href{/docs/reference/foundations/array/}{array}
\href{/docs/reference/foundations/function/}{function} }

) -\textgreater{} \href{/docs/reference/foundations/content/}{content}

\paragraph{\texorpdfstring{\texttt{\ update\ }}{ update }}\label{definitions-update-update}

\href{/docs/reference/foundations/int/}{int} {or}
\href{/docs/reference/foundations/array/}{array} {or}
\href{/docs/reference/foundations/function/}{function}

{Required} {{ Positional }}

\phantomsection\label{definitions-update-update-positional-tooltip}
Positional parameters are specified in order, without names.

If given an integer or array of integers, sets the counter to that
value. If given a function, that function receives the previous counter
value (with each number as a separate argument) and has to return the
new value (integer or array).

\href{/docs/reference/introspection/}{\pandocbounded{\includesvg[keepaspectratio]{/assets/icons/16-arrow-right.svg}}}

{ Introspection } { Previous page }

\href{/docs/reference/introspection/here/}{\pandocbounded{\includesvg[keepaspectratio]{/assets/icons/16-arrow-right.svg}}}

{ Here } { Next page }


\title{typst.app/docs/reference/introspection/here}

\begin{itemize}
\tightlist
\item
  \href{/docs}{\includesvg[width=0.16667in,height=0.16667in]{/assets/icons/16-docs-dark.svg}}
\item
  \includesvg[width=0.16667in,height=0.16667in]{/assets/icons/16-arrow-right.svg}
\item
  \href{/docs/reference/}{Reference}
\item
  \includesvg[width=0.16667in,height=0.16667in]{/assets/icons/16-arrow-right.svg}
\item
  \href{/docs/reference/introspection/}{Introspection}
\item
  \includesvg[width=0.16667in,height=0.16667in]{/assets/icons/16-arrow-right.svg}
\item
  \href{/docs/reference/introspection/here/}{Here}
\end{itemize}

\section{\texorpdfstring{\texttt{\ here\ } {{ Contextual
}}}{ here   Contextual }}\label{summary}

\phantomsection\label{contextual-tooltip}
Contextual functions can only be used when the context is known

Provides the current location in the document.

You can think of \texttt{\ here\ } as a low-level building block that
directly extracts the current location from the active
\href{/docs/reference/context/}{context} . Some other functions use it
internally: For instance,
\texttt{\ counter\ }{\texttt{\ .\ }}\texttt{\ }{\texttt{\ get\ }}\texttt{\ }{\texttt{\ (\ }}\texttt{\ }{\texttt{\ )\ }}\texttt{\ }
is equivalent to
\texttt{\ counter\ }{\texttt{\ .\ }}\texttt{\ }{\texttt{\ at\ }}\texttt{\ }{\texttt{\ (\ }}\texttt{\ }{\texttt{\ here\ }}\texttt{\ }{\texttt{\ (\ }}\texttt{\ }{\texttt{\ )\ }}\texttt{\ }{\texttt{\ )\ }}\texttt{\ }
.

Within show rules on
\href{/docs/reference/introspection/location/\#locatable}{locatable}
elements,
\texttt{\ }{\texttt{\ here\ }}\texttt{\ }{\texttt{\ (\ }}\texttt{\ }{\texttt{\ )\ }}\texttt{\ }
will match the location of the shown element.

If you want to display the current page number, refer to the
documentation of the
\href{/docs/reference/introspection/counter/}{\texttt{\ counter\ }}
type. While \texttt{\ here\ } can be used to determine the physical page
number, typically you want the logical page number that may, for
instance, have been reset after a preface.

\subsection{Examples}\label{examples}

Determining the current position in the document in combination with the
\href{/docs/reference/introspection/location/\#definitions-position}{\texttt{\ position\ }}
method:

\begin{verbatim}
#context [
  I am located at
  #here().position()
]
\end{verbatim}

\includegraphics[width=5in,height=\textheight,keepaspectratio]{/assets/docs/5PrDc8FIHOrLs_qUjTj6iwAAAAAAAAAA.png}

Running a \href{/docs/reference/introspection/query/}{query} for
elements before the current position:

\begin{verbatim}
= Introduction
= Background

There are
#context query(
  selector(heading).before(here())
).len()
headings before me.

= Conclusion
\end{verbatim}

\includegraphics[width=5in,height=\textheight,keepaspectratio]{/assets/docs/5DWH6TcZBrEjuGuSwKqf8AAAAAAAAAAA.png}

Refer to the
\href{/docs/reference/foundations/selector/}{\texttt{\ selector\ }} type
for more details on before/after selectors.

\subsection{\texorpdfstring{{ Parameters
}}{ Parameters }}\label{parameters}

\phantomsection\label{parameters-tooltip}
Parameters are the inputs to a function. They are specified in
parentheses after the function name.

{ here } (

) -\textgreater{}
\href{/docs/reference/introspection/location/}{location}

\href{/docs/reference/introspection/counter/}{\pandocbounded{\includesvg[keepaspectratio]{/assets/icons/16-arrow-right.svg}}}

{ Counter } { Previous page }

\href{/docs/reference/introspection/locate/}{\pandocbounded{\includesvg[keepaspectratio]{/assets/icons/16-arrow-right.svg}}}

{ Locate } { Next page }


\title{typst.app/docs/reference/introspection/locate}

\begin{itemize}
\tightlist
\item
  \href{/docs}{\includesvg[width=0.16667in,height=0.16667in]{/assets/icons/16-docs-dark.svg}}
\item
  \includesvg[width=0.16667in,height=0.16667in]{/assets/icons/16-arrow-right.svg}
\item
  \href{/docs/reference/}{Reference}
\item
  \includesvg[width=0.16667in,height=0.16667in]{/assets/icons/16-arrow-right.svg}
\item
  \href{/docs/reference/introspection/}{Introspection}
\item
  \includesvg[width=0.16667in,height=0.16667in]{/assets/icons/16-arrow-right.svg}
\item
  \href{/docs/reference/introspection/locate/}{Locate}
\end{itemize}

\section{\texorpdfstring{\texttt{\ locate\ } {{ Contextual
}}}{ locate   Contextual }}\label{summary}

\phantomsection\label{contextual-tooltip}
Contextual functions can only be used when the context is known

Determines the location of an element in the document.

Takes a selector that must match exactly one element and returns that
element\textquotesingle s
\href{/docs/reference/introspection/location/}{\texttt{\ location\ }} .
This location can, in particular, be used to retrieve the physical
\href{/docs/reference/introspection/location/\#definitions-page}{\texttt{\ page\ }}
number and
\href{/docs/reference/introspection/location/\#definitions-position}{\texttt{\ position\ }}
(page, x, y) for that element.

\subsection{Examples}\label{examples}

Locating a specific element:

\begin{verbatim}
#context [
  Introduction is at: \
  #locate(<intro>).position()
]

= Introduction <intro>
\end{verbatim}

\includegraphics[width=5in,height=\textheight,keepaspectratio]{/assets/docs/fizxN7L7L7E8uWpTd8_mMgAAAAAAAAAA.png}

\subsection{Compatibility}\label{compatibility}

In Typst 0.10 and lower, the \texttt{\ locate\ } function took a closure
that made the current location in the document available (like
\href{/docs/reference/introspection/here/}{\texttt{\ here\ }} does now).
This usage pattern is deprecated. Compatibility with the old way will
remain for a while to give package authors time to upgrade. To that
effect, \texttt{\ locate\ } detects whether it received a selector or a
user-defined function and adjusts its semantics accordingly. This
behaviour will be removed in the future.

\subsection{\texorpdfstring{{ Parameters
}}{ Parameters }}\label{parameters}

\phantomsection\label{parameters-tooltip}
Parameters are the inputs to a function. They are specified in
parentheses after the function name.

{ locate } (

{ \href{/docs/reference/foundations/label/}{label}
\href{/docs/reference/foundations/selector/}{selector}
\href{/docs/reference/introspection/location/}{location}
\href{/docs/reference/foundations/function/}{function} }

) -\textgreater{} \href{/docs/reference/foundations/content/}{content}
\href{/docs/reference/introspection/location/}{location}

\subsubsection{\texorpdfstring{\texttt{\ selector\ }}{ selector }}\label{parameters-selector}

\href{/docs/reference/foundations/label/}{label} {or}
\href{/docs/reference/foundations/selector/}{selector} {or}
\href{/docs/reference/introspection/location/}{location} {or}
\href{/docs/reference/foundations/function/}{function}

{Required} {{ Positional }}

\phantomsection\label{parameters-selector-positional-tooltip}
Positional parameters are specified in order, without names.

A selector that should match exactly one element. This element will be
located.

Especially useful in combination with

\begin{itemize}
\tightlist
\item
  \href{/docs/reference/introspection/here/}{\texttt{\ here\ }} to
  locate the current context,
\item
  a
  \href{/docs/reference/introspection/location/}{\texttt{\ location\ }}
  retrieved from some queried element via the
  \href{/docs/reference/foundations/content/\#definitions-location}{\texttt{\ location()\ }}
  method on content.
\end{itemize}

\href{/docs/reference/introspection/here/}{\pandocbounded{\includesvg[keepaspectratio]{/assets/icons/16-arrow-right.svg}}}

{ Here } { Previous page }

\href{/docs/reference/introspection/location/}{\pandocbounded{\includesvg[keepaspectratio]{/assets/icons/16-arrow-right.svg}}}

{ Location } { Next page }


\title{typst.app/docs/reference/introspection/query}

\begin{itemize}
\tightlist
\item
  \href{/docs}{\includesvg[width=0.16667in,height=0.16667in]{/assets/icons/16-docs-dark.svg}}
\item
  \includesvg[width=0.16667in,height=0.16667in]{/assets/icons/16-arrow-right.svg}
\item
  \href{/docs/reference/}{Reference}
\item
  \includesvg[width=0.16667in,height=0.16667in]{/assets/icons/16-arrow-right.svg}
\item
  \href{/docs/reference/introspection/}{Introspection}
\item
  \includesvg[width=0.16667in,height=0.16667in]{/assets/icons/16-arrow-right.svg}
\item
  \href{/docs/reference/introspection/query/}{Query}
\end{itemize}

\section{\texorpdfstring{\texttt{\ query\ } {{ Contextual
}}}{ query   Contextual }}\label{summary}

\phantomsection\label{contextual-tooltip}
Contextual functions can only be used when the context is known

Finds elements in the document.

The \texttt{\ query\ } functions lets you search your document for
elements of a particular type or with a particular label. To use it, you
first need to ensure that \href{/docs/reference/context/}{context} is
available.

\subsection{Finding elements}\label{finding-elements}

In the example below, we manually create a table of contents instead of
using the \href{/docs/reference/model/outline/}{\texttt{\ outline\ }}
function.

To do this, we first query for all headings in the document at level 1
and where \texttt{\ outlined\ } is true. Querying only for headings at
level 1 ensures that, for the purpose of this example, sub-headings are
not included in the table of contents. The \texttt{\ outlined\ } field
is used to exclude the "Table of Contents" heading itself.

Note that we open a \texttt{\ context\ } to be able to use the
\texttt{\ query\ } function.

\begin{verbatim}
#set page(numbering: "1")

#heading(outlined: false)[
  Table of Contents
]
#context {
  let chapters = query(
    heading.where(
      level: 1,
      outlined: true,
    )
  )
  for chapter in chapters {
    let loc = chapter.location()
    let nr = numbering(
      loc.page-numbering(),
      ..counter(page).at(loc),
    )
    [#chapter.body #h(1fr) #nr \ ]
  }
}

= Introduction
#lorem(10)
#pagebreak()

== Sub-Heading
#lorem(8)

= Discussion
#lorem(18)
\end{verbatim}

\includegraphics[width=5in,height=\textheight,keepaspectratio]{/assets/docs/jo-em7a3jFROfNLdVe33CwAAAAAAAAAA.png}
\includegraphics[width=5in,height=\textheight,keepaspectratio]{/assets/docs/jo-em7a3jFROfNLdVe33CwAAAAAAAAAB.png}

To get the page numbers, we first get the location of the elements
returned by \texttt{\ query\ } with
\href{/docs/reference/foundations/content/\#definitions-location}{\texttt{\ location\ }}
. We then also retrieve the
\href{/docs/reference/introspection/location/\#definitions-page-numbering}{page
numbering} and
\href{/docs/reference/introspection/counter/\#page-counter}{page
counter} at that location and apply the numbering to the counter.

\subsection{A word of caution}\label{caution}

To resolve all your queries, Typst evaluates and layouts parts of the
document multiple times. However, there is no guarantee that your
queries can actually be completely resolved. If you
aren\textquotesingle t careful a query can affect itselfâ€''leading to a
result that never stabilizes.

In the example below, we query for all headings in the document. We then
generate as many headings. In the beginning, there\textquotesingle s
just one heading, titled \texttt{\ Real\ } . Thus, \texttt{\ count\ } is
\texttt{\ 1\ } and one \texttt{\ Fake\ } heading is generated. Typst
sees that the query\textquotesingle s result has changed and processes
it again. This time, \texttt{\ count\ } is \texttt{\ 2\ } and two
\texttt{\ Fake\ } headings are generated. This goes on and on. As we can
see, the output has a finite amount of headings. This is because Typst
simply gives up after a few attempts.

In general, you should try not to write queries that affect themselves.
The same words of caution also apply to other introspection features
like \href{/docs/reference/introspection/counter/}{counters} and
\href{/docs/reference/introspection/state/}{state} .

\begin{verbatim}
= Real
#context {
  let elems = query(heading)
  let count = elems.len()
  count * [= Fake]
}
\end{verbatim}

\includegraphics[width=5in,height=\textheight,keepaspectratio]{/assets/docs/C2bjyzuukR06BSWIMgC89wAAAAAAAAAA.png}

\subsection{Command line queries}\label{command-line-queries}

You can also perform queries from the command line with the
\texttt{\ typst\ query\ } command. This command executes an arbitrary
query on the document and returns the resulting elements in serialized
form. Consider the following \texttt{\ example.typ\ } file which
contains some invisible
\href{/docs/reference/introspection/metadata/}{metadata} :

\begin{verbatim}
#metadata("This is a note") <note>
\end{verbatim}

You can execute a query on it as follows using Typst\textquotesingle s
CLI:

\begin{verbatim}
$ typst query example.typ "<note>"
[
  {
    "func": "metadata",
    "value": "This is a note",
    "label": "<note>"
  }
]
\end{verbatim}

Frequently, you\textquotesingle re interested in only one specific field
of the resulting elements. In the case of the \texttt{\ metadata\ }
element, the \texttt{\ value\ } field is the interesting one. You can
extract just this field with the \texttt{\ -\/-field\ } argument.

\begin{verbatim}
$ typst query example.typ "<note>" --field value
["This is a note"]
\end{verbatim}

If you are interested in just a single element, you can use the
\texttt{\ -\/-one\ } flag to extract just it.

\begin{verbatim}
$ typst query example.typ "<note>" --field value --one
"This is a note"
\end{verbatim}

\subsection{\texorpdfstring{{ Parameters
}}{ Parameters }}\label{parameters}

\phantomsection\label{parameters-tooltip}
Parameters are the inputs to a function. They are specified in
parentheses after the function name.

{ query } (

{ \href{/docs/reference/foundations/label/}{label}
\href{/docs/reference/foundations/selector/}{selector}
\href{/docs/reference/introspection/location/}{location}
\href{/docs/reference/foundations/function/}{function} , } {
\href{/docs/reference/foundations/none/}{none}
\href{/docs/reference/introspection/location/}{location} , }

) -\textgreater{} \href{/docs/reference/foundations/array/}{array}

\subsubsection{\texorpdfstring{\texttt{\ target\ }}{ target }}\label{parameters-target}

\href{/docs/reference/foundations/label/}{label} {or}
\href{/docs/reference/foundations/selector/}{selector} {or}
\href{/docs/reference/introspection/location/}{location} {or}
\href{/docs/reference/foundations/function/}{function}

{Required} {{ Positional }}

\phantomsection\label{parameters-target-positional-tooltip}
Positional parameters are specified in order, without names.

Can be

\begin{itemize}
\tightlist
\item
  an element function like a \texttt{\ heading\ } or \texttt{\ figure\ }
  ,
\item
  a \texttt{\ }{\texttt{\ \textless{}label\textgreater{}\ }}\texttt{\ }
  ,
\item
  a more complex selector like
  \texttt{\ heading\ }{\texttt{\ .\ }}\texttt{\ }{\texttt{\ where\ }}\texttt{\ }{\texttt{\ (\ }}\texttt{\ level\ }{\texttt{\ :\ }}\texttt{\ }{\texttt{\ 1\ }}\texttt{\ }{\texttt{\ )\ }}\texttt{\ }
  ,
\item
  or
  \texttt{\ }{\texttt{\ selector\ }}\texttt{\ }{\texttt{\ (\ }}\texttt{\ heading\ }{\texttt{\ )\ }}\texttt{\ }{\texttt{\ .\ }}\texttt{\ }{\texttt{\ before\ }}\texttt{\ }{\texttt{\ (\ }}\texttt{\ }{\texttt{\ here\ }}\texttt{\ }{\texttt{\ (\ }}\texttt{\ }{\texttt{\ )\ }}\texttt{\ }{\texttt{\ )\ }}\texttt{\ }
  .
\end{itemize}

Only
\href{/docs/reference/introspection/location/\#locatable}{locatable}
element functions are supported.

\subsubsection{\texorpdfstring{\texttt{\ location\ }}{ location }}\label{parameters-location}

\href{/docs/reference/foundations/none/}{none} {or}
\href{/docs/reference/introspection/location/}{location}

{{ Positional }}

\phantomsection\label{parameters-location-positional-tooltip}
Positional parameters are specified in order, without names.

\emph{Compatibility:} This argument is deprecated. It only exists for
compatibility with Typst 0.10 and lower and shouldn\textquotesingle t be
used anymore.

Default: \texttt{\ }{\texttt{\ none\ }}\texttt{\ }

\href{/docs/reference/introspection/metadata/}{\pandocbounded{\includesvg[keepaspectratio]{/assets/icons/16-arrow-right.svg}}}

{ Metadata } { Previous page }

\href{/docs/reference/introspection/state/}{\pandocbounded{\includesvg[keepaspectratio]{/assets/icons/16-arrow-right.svg}}}

{ State } { Next page }


\title{typst.app/docs/reference/introspection/location}

\begin{itemize}
\tightlist
\item
  \href{/docs}{\includesvg[width=0.16667in,height=0.16667in]{/assets/icons/16-docs-dark.svg}}
\item
  \includesvg[width=0.16667in,height=0.16667in]{/assets/icons/16-arrow-right.svg}
\item
  \href{/docs/reference/}{Reference}
\item
  \includesvg[width=0.16667in,height=0.16667in]{/assets/icons/16-arrow-right.svg}
\item
  \href{/docs/reference/introspection/}{Introspection}
\item
  \includesvg[width=0.16667in,height=0.16667in]{/assets/icons/16-arrow-right.svg}
\item
  \href{/docs/reference/introspection/location/}{Location}
\end{itemize}

\section{\texorpdfstring{{ location }}{ location }}\label{summary}

Identifies an element in the document.

A location uniquely identifies an element in the document and lets you
access its absolute position on the pages. You can retrieve the current
location with the
\href{/docs/reference/introspection/here/}{\texttt{\ here\ }} function
and the location of a queried or shown element with the
\href{/docs/reference/foundations/content/\#definitions-location}{\texttt{\ location()\ }}
method on content.

\subsection{Locatable elements}\label{locatable}

Currently, only a subset of element functions is locatable. Aside from
headings and figures, this includes equations, references, quotes and
all elements with an explicit label. As a result, you \emph{can} query
for e.g. \href{/docs/reference/model/strong/}{\texttt{\ strong\ }}
elements, but you will find only those that have an explicit label
attached to them. This limitation will be resolved in the future.

\subsection{\texorpdfstring{{ Definitions
}}{ Definitions }}\label{definitions}

\phantomsection\label{definitions-tooltip}
Functions and types and can have associated definitions. These are
accessed by specifying the function or type, followed by a period, and
then the definition\textquotesingle s name.

\subsubsection{\texorpdfstring{\texttt{\ page\ }}{ page }}\label{definitions-page}

Returns the page number for this location.

Note that this does not return the value of the
\href{/docs/reference/introspection/counter/}{page counter} at this
location, but the true page number (starting from one).

If you want to know the value of the page counter, use
\texttt{\ }{\texttt{\ counter\ }}\texttt{\ }{\texttt{\ (\ }}\texttt{\ page\ }{\texttt{\ )\ }}\texttt{\ }{\texttt{\ .\ }}\texttt{\ }{\texttt{\ at\ }}\texttt{\ }{\texttt{\ (\ }}\texttt{\ loc\ }{\texttt{\ )\ }}\texttt{\ }
instead.

Can be used with
\href{/docs/reference/introspection/here/}{\texttt{\ here\ }} to
retrieve the physical page position of the current context:

self { . } { page } (

) -\textgreater{} \href{/docs/reference/foundations/int/}{int}

\begin{verbatim}
#context [
  I am located on
  page #here().page()
]
\end{verbatim}

\includegraphics[width=5in,height=\textheight,keepaspectratio]{/assets/docs/0ToVSLLUesTLkEw_YsnJkwAAAAAAAAAA.png}

\subsubsection{\texorpdfstring{\texttt{\ position\ }}{ position }}\label{definitions-position}

Returns a dictionary with the page number and the x, y position for this
location. The page number starts at one and the coordinates are measured
from the top-left of the page.

If you only need the page number, use \texttt{\ page()\ } instead as it
allows Typst to skip unnecessary work.

self { . } { position } (

) -\textgreater{}
\href{/docs/reference/foundations/dictionary/}{dictionary}

\subsubsection{\texorpdfstring{\texttt{\ page-numbering\ }}{ page-numbering }}\label{definitions-page-numbering}

Returns the page numbering pattern of the page at this location. This
can be used when displaying the page counter in order to obtain the
local numbering. This is useful if you are building custom indices or
outlines.

If the page numbering is set to
\texttt{\ }{\texttt{\ none\ }}\texttt{\ } at that location, this
function returns \texttt{\ }{\texttt{\ none\ }}\texttt{\ } .

self { . } { page-numbering } (

) -\textgreater{} \href{/docs/reference/foundations/none/}{none}
\href{/docs/reference/foundations/str/}{str}
\href{/docs/reference/foundations/function/}{function}

\href{/docs/reference/introspection/locate/}{\pandocbounded{\includesvg[keepaspectratio]{/assets/icons/16-arrow-right.svg}}}

{ Locate } { Previous page }

\href{/docs/reference/introspection/metadata/}{\pandocbounded{\includesvg[keepaspectratio]{/assets/icons/16-arrow-right.svg}}}

{ Metadata } { Next page }


\title{typst.app/docs/reference/introspection/state}

\begin{itemize}
\tightlist
\item
  \href{/docs}{\includesvg[width=0.16667in,height=0.16667in]{/assets/icons/16-docs-dark.svg}}
\item
  \includesvg[width=0.16667in,height=0.16667in]{/assets/icons/16-arrow-right.svg}
\item
  \href{/docs/reference/}{Reference}
\item
  \includesvg[width=0.16667in,height=0.16667in]{/assets/icons/16-arrow-right.svg}
\item
  \href{/docs/reference/introspection/}{Introspection}
\item
  \includesvg[width=0.16667in,height=0.16667in]{/assets/icons/16-arrow-right.svg}
\item
  \href{/docs/reference/introspection/state/}{State}
\end{itemize}

\section{\texorpdfstring{{ state }}{ state }}\label{summary}

Manages stateful parts of your document.

Let\textquotesingle s say you have some computations in your document
and want to remember the result of your last computation to use it in
the next one. You might try something similar to the code below and
expect it to output 10, 13, 26, and 21. However this \textbf{does not
work} in Typst. If you test this code, you will see that Typst complains
with the following error message: \emph{Variables from outside the
function are read-only and cannot be modified.}

\begin{verbatim}
// This doesn't work!
#let x = 0
#let compute(expr) = {
  x = eval(
    expr.replace("x", str(x))
  )
  [New value is #x. ]
}

#compute("10") \
#compute("x + 3") \
#compute("x * 2") \
#compute("x - 5")
\end{verbatim}

\subsection{State and document markup}\label{state-and-markup}

Why does it do that? Because, in general, this kind of computation with
side effects is problematic in document markup and Typst is upfront
about that. For the results to make sense, the computation must proceed
in the same order in which the results will be laid out in the document.
In our simple example, that\textquotesingle s the case, but in general
it might not be.

Let\textquotesingle s look at a slightly different, but similar kind of
state: The heading numbering. We want to increase the heading counter at
each heading. Easy enough, right? Just add one. Well,
it\textquotesingle s not that simple. Consider the following example:

\begin{verbatim}
#set heading(numbering: "1.")
#let template(body) = [
  = Outline
  ...
  #body
]

#show: template

= Introduction
...
\end{verbatim}

\includegraphics[width=5in,height=\textheight,keepaspectratio]{/assets/docs/OC8Yphz4-mFQhH6Mm9lwwAAAAAAAAAAA.png}

Here, Typst first processes the body of the document after the show
rule, sees the \texttt{\ Introduction\ } heading, then passes the
resulting content to the \texttt{\ template\ } function and only then
sees the \texttt{\ Outline\ } . Just counting up would number the
\texttt{\ Introduction\ } with \texttt{\ 1\ } and the
\texttt{\ Outline\ } with \texttt{\ 2\ } .

\subsection{Managing state in Typst}\label{state-in-typst}

So what do we do instead? We use Typst\textquotesingle s state
management system. Calling the \texttt{\ state\ } function with an
identifying string key and an optional initial value gives you a state
value which exposes a few functions. The two most important ones are
\texttt{\ get\ } and \texttt{\ update\ } :

\begin{itemize}
\item
  The
  \href{/docs/reference/introspection/state/\#definitions-get}{\texttt{\ get\ }}
  function retrieves the current value of the state. Because the value
  can vary over the course of the document, it is a \emph{contextual}
  function that can only be used when
  \href{/docs/reference/context/}{context} is available.
\item
  The
  \href{/docs/reference/introspection/state/\#definitions-update}{\texttt{\ update\ }}
  function modifies the state. You can give it any value. If given a
  non-function value, it sets the state to that value. If given a
  function, that function receives the previous state and has to return
  the new state.
\end{itemize}

Our initial example would now look like this:

\begin{verbatim}
#let s = state("x", 0)
#let compute(expr) = [
  #s.update(x =>
    eval(expr.replace("x", str(x)))
  )
  New value is #context s.get().
]

#compute("10") \
#compute("x + 3") \
#compute("x * 2") \
#compute("x - 5")
\end{verbatim}

\includegraphics[width=5in,height=\textheight,keepaspectratio]{/assets/docs/TvB3cSxy6XwQVp0EXZ9-ewAAAAAAAAAA.png}

State managed by Typst is always updated in layout order, not in
evaluation order. The \texttt{\ update\ } method returns content and its
effect occurs at the position where the returned content is inserted
into the document.

As a result, we can now also store some of the computations in
variables, but they still show the correct results:

\begin{verbatim}
...

#let more = [
  #compute("x * 2") \
  #compute("x - 5")
]

#compute("10") \
#compute("x + 3") \
#more
\end{verbatim}

\includegraphics[width=5in,height=\textheight,keepaspectratio]{/assets/docs/leSHwxlkl8fBohZKt4lM4AAAAAAAAAAA.png}

This example is of course a bit silly, but in practice this is often
exactly what you want! A good example are heading counters, which is why
Typst\textquotesingle s
\href{/docs/reference/introspection/counter/}{counting system} is very
similar to its state system.

\subsection{Time Travel}\label{time-travel}

By using Typst\textquotesingle s state management system you also get
time travel capabilities! We can find out what the value of the state
will be at any position in the document from anywhere else. In
particular, the \texttt{\ at\ } method gives us the value of the state
at any particular location and the \texttt{\ final\ } methods gives us
the value of the state at the end of the document.

\begin{verbatim}
...

Value at `<here>` is
#context s.at(<here>)

#compute("10") \
#compute("x + 3") \
*Here.* <here> \
#compute("x * 2") \
#compute("x - 5")
\end{verbatim}

\includegraphics[width=5in,height=\textheight,keepaspectratio]{/assets/docs/FSbY2IZskPNKeQtPqbjroAAAAAAAAAAA.png}

\subsection{A word of caution}\label{caution}

To resolve the values of all states, Typst evaluates parts of your code
multiple times. However, there is no guarantee that your state
manipulation can actually be completely resolved.

For instance, if you generate state updates depending on the final value
of a state, the results might never converge. The example below
illustrates this. We initialize our state with \texttt{\ 1\ } and then
update it to its own final value plus 1. So it should be \texttt{\ 2\ }
, but then its final value is \texttt{\ 2\ } , so it should be
\texttt{\ 3\ } , and so on. This example displays a finite value because
Typst simply gives up after a few attempts.

\begin{verbatim}
// This is bad!
#let s = state("x", 1)
#context s.update(s.final() + 1)
#context s.get()
\end{verbatim}

\includegraphics[width=5in,height=\textheight,keepaspectratio]{/assets/docs/4ABrNAaHVbvzCF9JEmUebAAAAAAAAAAA.png}

In general, you should try not to generate state updates from within
context expressions. If possible, try to express your updates as
non-contextual values or functions that compute the new value from the
previous value. Sometimes, it cannot be helped, but in those cases it is
up to you to ensure that the result converges.

\subsection{\texorpdfstring{Constructor
{}}{Constructor }}\label{constructor}

\phantomsection\label{constructor-constructor-tooltip}
If a type has a constructor, you can call it like a function to create a
new value of the type.

Create a new state identified by a key.

{ state } (

{ \href{/docs/reference/foundations/str/}{str} , } { { any } , }

) -\textgreater{} \href{/docs/reference/introspection/state/}{state}

\paragraph{\texorpdfstring{\texttt{\ key\ }}{ key }}\label{constructor-key}

\href{/docs/reference/foundations/str/}{str}

{Required} {{ Positional }}

\phantomsection\label{constructor-key-positional-tooltip}
Positional parameters are specified in order, without names.

The key that identifies this state.

\paragraph{\texorpdfstring{\texttt{\ init\ }}{ init }}\label{constructor-init}

{ any }

{{ Positional }}

\phantomsection\label{constructor-init-positional-tooltip}
Positional parameters are specified in order, without names.

The initial value of the state.

Default: \texttt{\ }{\texttt{\ none\ }}\texttt{\ }

\subsection{\texorpdfstring{{ Definitions
}}{ Definitions }}\label{definitions}

\phantomsection\label{definitions-tooltip}
Functions and types and can have associated definitions. These are
accessed by specifying the function or type, followed by a period, and
then the definition\textquotesingle s name.

\subsubsection{\texorpdfstring{\texttt{\ get\ } {{ Contextual
}}}{ get   Contextual }}\label{definitions-get}

\phantomsection\label{definitions-get-contextual-tooltip}
Contextual functions can only be used when the context is known

Retrieves the value of the state at the current location.

This is equivalent to
\texttt{\ state\ }{\texttt{\ .\ }}\texttt{\ }{\texttt{\ at\ }}\texttt{\ }{\texttt{\ (\ }}\texttt{\ }{\texttt{\ here\ }}\texttt{\ }{\texttt{\ (\ }}\texttt{\ }{\texttt{\ )\ }}\texttt{\ }{\texttt{\ )\ }}\texttt{\ }
.

self { . } { get } (

) -\textgreater{} { any }

\subsubsection{\texorpdfstring{\texttt{\ at\ } {{ Contextual
}}}{ at   Contextual }}\label{definitions-at}

\phantomsection\label{definitions-at-contextual-tooltip}
Contextual functions can only be used when the context is known

Retrieves the value of the state at the given selector\textquotesingle s
unique match.

The \texttt{\ selector\ } must match exactly one element in the
document. The most useful kinds of selectors for this are
\href{/docs/reference/foundations/label/}{labels} and
\href{/docs/reference/introspection/location/}{locations} .

\emph{Compatibility:} For compatibility with Typst 0.10 and lower, this
function also works without a known context if the \texttt{\ selector\ }
is a location. This behaviour will be removed in a future release.

self { . } { at } (

{ \href{/docs/reference/foundations/label/}{label}
\href{/docs/reference/foundations/selector/}{selector}
\href{/docs/reference/introspection/location/}{location}
\href{/docs/reference/foundations/function/}{function} }

) -\textgreater{} { any }

\paragraph{\texorpdfstring{\texttt{\ selector\ }}{ selector }}\label{definitions-at-selector}

\href{/docs/reference/foundations/label/}{label} {or}
\href{/docs/reference/foundations/selector/}{selector} {or}
\href{/docs/reference/introspection/location/}{location} {or}
\href{/docs/reference/foundations/function/}{function}

{Required} {{ Positional }}

\phantomsection\label{definitions-at-selector-positional-tooltip}
Positional parameters are specified in order, without names.

The place at which the state\textquotesingle s value should be
retrieved.

\subsubsection{\texorpdfstring{\texttt{\ final\ } {{ Contextual
}}}{ final   Contextual }}\label{definitions-final}

\phantomsection\label{definitions-final-contextual-tooltip}
Contextual functions can only be used when the context is known

Retrieves the value of the state at the end of the document.

self { . } { final } (

{ \href{/docs/reference/foundations/none/}{none}
\href{/docs/reference/introspection/location/}{location} }

) -\textgreater{} { any }

\paragraph{\texorpdfstring{\texttt{\ location\ }}{ location }}\label{definitions-final-location}

\href{/docs/reference/foundations/none/}{none} {or}
\href{/docs/reference/introspection/location/}{location}

{{ Positional }}

\phantomsection\label{definitions-final-location-positional-tooltip}
Positional parameters are specified in order, without names.

\emph{Compatibility:} This argument is deprecated. It only exists for
compatibility with Typst 0.10 and lower and shouldn\textquotesingle t be
used anymore.

Default: \texttt{\ }{\texttt{\ none\ }}\texttt{\ }

\subsubsection{\texorpdfstring{\texttt{\ update\ }}{ update }}\label{definitions-update}

Update the value of the state.

The update will be in effect at the position where the returned content
is inserted into the document. If you don\textquotesingle t put the
output into the document, nothing happens! This would be the case, for
example, if you write
\texttt{\ }{\texttt{\ let\ }}\texttt{\ \_\ }{\texttt{\ =\ }}\texttt{\ }{\texttt{\ state\ }}\texttt{\ }{\texttt{\ (\ }}\texttt{\ }{\texttt{\ "key"\ }}\texttt{\ }{\texttt{\ )\ }}\texttt{\ }{\texttt{\ .\ }}\texttt{\ }{\texttt{\ update\ }}\texttt{\ }{\texttt{\ (\ }}\texttt{\ }{\texttt{\ 7\ }}\texttt{\ }{\texttt{\ )\ }}\texttt{\ }
. State updates are always applied in layout order and in that case,
Typst wouldn\textquotesingle t know when to update the state.

self { . } { update } (

{ { any } \href{/docs/reference/foundations/function/}{function} }

) -\textgreater{} \href{/docs/reference/foundations/content/}{content}

\paragraph{\texorpdfstring{\texttt{\ update\ }}{ update }}\label{definitions-update-update}

{ any } {or} \href{/docs/reference/foundations/function/}{function}

{Required} {{ Positional }}

\phantomsection\label{definitions-update-update-positional-tooltip}
Positional parameters are specified in order, without names.

If given a non function-value, sets the state to that value. If given a
function, that function receives the previous state and has to return
the new state.

\subsubsection{\texorpdfstring{\texttt{\ display\ }}{ display }}\label{definitions-display}

Displays the current value of the state.

\textbf{Deprecation planned:} Use
\href{/docs/reference/introspection/state/\#definitions-get}{\texttt{\ get\ }}
instead.

self { . } { display } (

{ \href{/docs/reference/foundations/none/}{none}
\href{/docs/reference/foundations/function/}{function} }

) -\textgreater{} \href{/docs/reference/foundations/content/}{content}

\paragraph{\texorpdfstring{\texttt{\ func\ }}{ func }}\label{definitions-display-func}

\href{/docs/reference/foundations/none/}{none} {or}
\href{/docs/reference/foundations/function/}{function}

{{ Positional }}

\phantomsection\label{definitions-display-func-positional-tooltip}
Positional parameters are specified in order, without names.

A function which receives the value of the state and can return
arbitrary content which is then displayed. If this is omitted, the value
is directly displayed.

Default: \texttt{\ }{\texttt{\ none\ }}\texttt{\ }

\href{/docs/reference/introspection/query/}{\pandocbounded{\includesvg[keepaspectratio]{/assets/icons/16-arrow-right.svg}}}

{ Query } { Previous page }

\href{/docs/reference/data-loading/}{\pandocbounded{\includesvg[keepaspectratio]{/assets/icons/16-arrow-right.svg}}}

{ Data Loading } { Next page }


