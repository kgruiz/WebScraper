\title{typst.app/docs/reference/math/styles}

\begin{itemize}
\tightlist
\item
  \href{/docs}{\includesvg[width=0.16667in,height=0.16667in]{/assets/icons/16-docs-dark.svg}}
\item
  \includesvg[width=0.16667in,height=0.16667in]{/assets/icons/16-arrow-right.svg}
\item
  \href{/docs/reference/}{Reference}
\item
  \includesvg[width=0.16667in,height=0.16667in]{/assets/icons/16-arrow-right.svg}
\item
  \href{/docs/reference/math/}{Math}
\item
  \includesvg[width=0.16667in,height=0.16667in]{/assets/icons/16-arrow-right.svg}
\item
  \href{/docs/reference/math/styles}{Styles}
\end{itemize}

\section{Styles}\label{summary}

Alternate letterforms within formulas.

These functions are distinct from the
\href{/docs/reference/text/text/}{\texttt{\ text\ }} function because
math fonts contain multiple variants of each letter.

\subsection{Functions}\label{functions}

\subsubsection{\texorpdfstring{\texttt{\ upright\ }}{ upright }}\label{functions-upright}

Upright (non-italic) font style in math.

math { . } { upright } (

{ \href{/docs/reference/foundations/content/}{content} }

) -\textgreater{} \href{/docs/reference/foundations/content/}{content}

\begin{verbatim}
$ upright(A) != A $
\end{verbatim}

\includegraphics[width=5in,height=\textheight,keepaspectratio]{/assets/docs/I3XzlEtlEFD5Cw96srS1ngAAAAAAAAAA.png}

\paragraph{\texorpdfstring{\texttt{\ body\ }}{ body }}\label{functions-upright-body}

\href{/docs/reference/foundations/content/}{content}

{Required} {{ Positional }}

\phantomsection\label{functions-upright-body-positional-tooltip}
Positional parameters are specified in order, without names.

The content to style.

\subsubsection{\texorpdfstring{\texttt{\ italic\ }}{ italic }}\label{functions-italic}

Italic font style in math.

For roman letters and greek lowercase letters, this is already the
default.

math { . } { italic } (

{ \href{/docs/reference/foundations/content/}{content} }

) -\textgreater{} \href{/docs/reference/foundations/content/}{content}

\paragraph{\texorpdfstring{\texttt{\ body\ }}{ body }}\label{functions-italic-body}

\href{/docs/reference/foundations/content/}{content}

{Required} {{ Positional }}

\phantomsection\label{functions-italic-body-positional-tooltip}
Positional parameters are specified in order, without names.

The content to style.

\subsubsection{\texorpdfstring{\texttt{\ bold\ }}{ bold }}\label{functions-bold}

Bold font style in math.

math { . } { bold } (

{ \href{/docs/reference/foundations/content/}{content} }

) -\textgreater{} \href{/docs/reference/foundations/content/}{content}

\begin{verbatim}
$ bold(A) := B^+ $
\end{verbatim}

\includegraphics[width=5in,height=\textheight,keepaspectratio]{/assets/docs/8-9k5ChF2PO13_x1ipPkAAAAAAAAAAAA.png}

\paragraph{\texorpdfstring{\texttt{\ body\ }}{ body }}\label{functions-bold-body}

\href{/docs/reference/foundations/content/}{content}

{Required} {{ Positional }}

\phantomsection\label{functions-bold-body-positional-tooltip}
Positional parameters are specified in order, without names.

The content to style.

\href{/docs/reference/math/stretch/}{\pandocbounded{\includesvg[keepaspectratio]{/assets/icons/16-arrow-right.svg}}}

{ Stretch } { Previous page }

\href{/docs/reference/math/op/}{\pandocbounded{\includesvg[keepaspectratio]{/assets/icons/16-arrow-right.svg}}}

{ Text Operator } { Next page }


\title{typst.app/docs/reference/math/mat}

\begin{itemize}
\tightlist
\item
  \href{/docs}{\includesvg[width=0.16667in,height=0.16667in]{/assets/icons/16-docs-dark.svg}}
\item
  \includesvg[width=0.16667in,height=0.16667in]{/assets/icons/16-arrow-right.svg}
\item
  \href{/docs/reference/}{Reference}
\item
  \includesvg[width=0.16667in,height=0.16667in]{/assets/icons/16-arrow-right.svg}
\item
  \href{/docs/reference/math/}{Math}
\item
  \includesvg[width=0.16667in,height=0.16667in]{/assets/icons/16-arrow-right.svg}
\item
  \href{/docs/reference/math/mat/}{Matrix}
\end{itemize}

\section{\texorpdfstring{\texttt{\ mat\ } {{ Element
}}}{ mat   Element }}\label{summary}

\phantomsection\label{element-tooltip}
Element functions can be customized with \texttt{\ set\ } and
\texttt{\ show\ } rules.

A matrix.

The elements of a row should be separated by commas, while the rows
themselves should be separated by semicolons. The semicolon syntax
merges preceding arguments separated by commas into an array. You can
also use this special syntax of math function calls to define custom
functions that take 2D data.

Content in cells can be aligned with the
\href{/docs/reference/math/mat/\#parameters-align}{\texttt{\ align\ }}
parameter, or content in cells that are in the same row can be aligned
with the \texttt{\ \&\ } symbol.

\subsection{Example}\label{example}

\begin{verbatim}
$ mat(
  1, 2, ..., 10;
  2, 2, ..., 10;
  dots.v, dots.v, dots.down, dots.v;
  10, 10, ..., 10;
) $
\end{verbatim}

\includegraphics[width=5in,height=\textheight,keepaspectratio]{/assets/docs/yiSilYGQ1wRBpIK3ON349AAAAAAAAAAA.png}

\subsection{\texorpdfstring{{ Parameters
}}{ Parameters }}\label{parameters}

\phantomsection\label{parameters-tooltip}
Parameters are the inputs to a function. They are specified in
parentheses after the function name.

math { . } { mat } (

{ \hyperref[parameters-delim]{delim :}
\href{/docs/reference/foundations/none/}{none}
\href{/docs/reference/foundations/str/}{str}
\href{/docs/reference/foundations/array/}{array}
\href{/docs/reference/symbols/symbol/}{symbol} , } {
\hyperref[parameters-align]{align :}
\href{/docs/reference/layout/alignment/}{alignment} , } {
\hyperref[parameters-augment]{augment :}
\href{/docs/reference/foundations/none/}{none}
\href{/docs/reference/foundations/int/}{int}
\href{/docs/reference/foundations/dictionary/}{dictionary} , } {
\hyperref[parameters-gap]{gap :}
\href{/docs/reference/layout/relative/}{relative} , } {
\hyperref[parameters-row-gap]{row-gap :}
\href{/docs/reference/layout/relative/}{relative} , } {
\hyperref[parameters-column-gap]{column-gap :}
\href{/docs/reference/layout/relative/}{relative} , } {
\hyperref[parameters-rows]{..}
\href{/docs/reference/foundations/array/}{array} , }

) -\textgreater{} \href{/docs/reference/foundations/content/}{content}

\subsubsection{\texorpdfstring{\texttt{\ delim\ }}{ delim }}\label{parameters-delim}

\href{/docs/reference/foundations/none/}{none} {or}
\href{/docs/reference/foundations/str/}{str} {or}
\href{/docs/reference/foundations/array/}{array} {or}
\href{/docs/reference/symbols/symbol/}{symbol}

{{ Settable }}

\phantomsection\label{parameters-delim-settable-tooltip}
Settable parameters can be customized for all following uses of the
function with a \texttt{\ set\ } rule.

The delimiter to use.

Can be a single character specifying the left delimiter, in which case
the right delimiter is inferred. Otherwise, can be an array containing a
left and a right delimiter.

Default:
\texttt{\ }{\texttt{\ (\ }}\texttt{\ }{\texttt{\ "("\ }}\texttt{\ }{\texttt{\ ,\ }}\texttt{\ }{\texttt{\ ")"\ }}\texttt{\ }{\texttt{\ )\ }}\texttt{\ }

\includesvg[width=0.16667in,height=0.16667in]{/assets/icons/16-arrow-right.svg}
View example

\begin{verbatim}
#set math.mat(delim: "[")
$ mat(1, 2; 3, 4) $
\end{verbatim}

\includegraphics[width=5in,height=\textheight,keepaspectratio]{/assets/docs/CpCAX34oIjWq-jvec_NKoQAAAAAAAAAA.png}

\subsubsection{\texorpdfstring{\texttt{\ align\ }}{ align }}\label{parameters-align}

\href{/docs/reference/layout/alignment/}{alignment}

{{ Settable }}

\phantomsection\label{parameters-align-settable-tooltip}
Settable parameters can be customized for all following uses of the
function with a \texttt{\ set\ } rule.

The horizontal alignment that each cell should have.

Default: \texttt{\ center\ }

\includesvg[width=0.16667in,height=0.16667in]{/assets/icons/16-arrow-right.svg}
View example

\begin{verbatim}
#set math.mat(align: right)
$ mat(-1, 1, 1; 1, -1, 1; 1, 1, -1) $
\end{verbatim}

\includegraphics[width=5in,height=\textheight,keepaspectratio]{/assets/docs/X3QXNtgXqEVUQfvJRQOPRwAAAAAAAAAA.png}

\subsubsection{\texorpdfstring{\texttt{\ augment\ }}{ augment }}\label{parameters-augment}

\href{/docs/reference/foundations/none/}{none} {or}
\href{/docs/reference/foundations/int/}{int} {or}
\href{/docs/reference/foundations/dictionary/}{dictionary}

{{ Settable }}

\phantomsection\label{parameters-augment-settable-tooltip}
Settable parameters can be customized for all following uses of the
function with a \texttt{\ set\ } rule.

Draws augmentation lines in a matrix.

\begin{itemize}
\tightlist
\item
  \texttt{\ }{\texttt{\ none\ }}\texttt{\ } : No lines are drawn.
\item
  A single number: A vertical augmentation line is drawn after the
  specified column number. Negative numbers start from the end.
\item
  A dictionary: With a dictionary, multiple augmentation lines can be
  drawn both horizontally and vertically. Additionally, the style of the
  lines can be set. The dictionary can contain the following keys:

  \begin{itemize}
  \tightlist
  \item
    \texttt{\ hline\ } : The offsets at which horizontal lines should be
    drawn. For example, an offset of \texttt{\ 2\ } would result in a
    horizontal line being drawn after the second row of the matrix.
    Accepts either an integer for a single line, or an array of integers
    for multiple lines. Like for a single number, negative numbers start
    from the end.
  \item
    \texttt{\ vline\ } : The offsets at which vertical lines should be
    drawn. For example, an offset of \texttt{\ 2\ } would result in a
    vertical line being drawn after the second column of the matrix.
    Accepts either an integer for a single line, or an array of integers
    for multiple lines. Like for a single number, negative numbers start
    from the end.
  \item
    \texttt{\ stroke\ } : How to
    \href{/docs/reference/visualize/stroke/}{stroke} the line. If set to
    \texttt{\ }{\texttt{\ auto\ }}\texttt{\ } , takes on a thickness of
    0.05em and square line caps.
  \end{itemize}
\end{itemize}

Default: \texttt{\ }{\texttt{\ none\ }}\texttt{\ }

\includesvg[width=0.16667in,height=0.16667in]{/assets/icons/16-arrow-right.svg}
View example

\begin{verbatim}
$ mat(1, 0, 1; 0, 1, 2; augment: #2) $
// Equivalent to:
$ mat(1, 0, 1; 0, 1, 2; augment: #(-1)) $
\end{verbatim}

\includegraphics[width=5in,height=\textheight,keepaspectratio]{/assets/docs/4iip0Z9ppDA0SxnJHJihkQAAAAAAAAAA.png}

\begin{verbatim}
$ mat(0, 0, 0; 1, 1, 1; augment: #(hline: 1, stroke: 2pt + green)) $
\end{verbatim}

\includegraphics[width=5in,height=\textheight,keepaspectratio]{/assets/docs/3PHAJpsviSZ-Rqtb3sBd4AAAAAAAAAAA.png}

\subsubsection{\texorpdfstring{\texttt{\ gap\ }}{ gap }}\label{parameters-gap}

\href{/docs/reference/layout/relative/}{relative}

{{ Settable }}

\phantomsection\label{parameters-gap-settable-tooltip}
Settable parameters can be customized for all following uses of the
function with a \texttt{\ set\ } rule.

The gap between rows and columns.

This is a shorthand to set \texttt{\ row-gap\ } and
\texttt{\ column-gap\ } to the same value.

Default:
\texttt{\ }{\texttt{\ 0\%\ }}\texttt{\ }{\texttt{\ +\ }}\texttt{\ }{\texttt{\ 0pt\ }}\texttt{\ }

\includesvg[width=0.16667in,height=0.16667in]{/assets/icons/16-arrow-right.svg}
View example

\begin{verbatim}
#set math.mat(gap: 1em)
$ mat(1, 2; 3, 4) $
\end{verbatim}

\includegraphics[width=5in,height=\textheight,keepaspectratio]{/assets/docs/kaypJSdE1P1lOWZ-cMMpyAAAAAAAAAAA.png}

\subsubsection{\texorpdfstring{\texttt{\ row-gap\ }}{ row-gap }}\label{parameters-row-gap}

\href{/docs/reference/layout/relative/}{relative}

{{ Settable }}

\phantomsection\label{parameters-row-gap-settable-tooltip}
Settable parameters can be customized for all following uses of the
function with a \texttt{\ set\ } rule.

The gap between rows.

Default:
\texttt{\ }{\texttt{\ 0\%\ }}\texttt{\ }{\texttt{\ +\ }}\texttt{\ }{\texttt{\ 0.2em\ }}\texttt{\ }

\includesvg[width=0.16667in,height=0.16667in]{/assets/icons/16-arrow-right.svg}
View example

\begin{verbatim}
#set math.mat(row-gap: 1em)
$ mat(1, 2; 3, 4) $
\end{verbatim}

\includegraphics[width=5in,height=\textheight,keepaspectratio]{/assets/docs/YNVJ8uCnPvrs8e0YkWIQFgAAAAAAAAAA.png}

\subsubsection{\texorpdfstring{\texttt{\ column-gap\ }}{ column-gap }}\label{parameters-column-gap}

\href{/docs/reference/layout/relative/}{relative}

{{ Settable }}

\phantomsection\label{parameters-column-gap-settable-tooltip}
Settable parameters can be customized for all following uses of the
function with a \texttt{\ set\ } rule.

The gap between columns.

Default:
\texttt{\ }{\texttt{\ 0\%\ }}\texttt{\ }{\texttt{\ +\ }}\texttt{\ }{\texttt{\ 0.5em\ }}\texttt{\ }

\includesvg[width=0.16667in,height=0.16667in]{/assets/icons/16-arrow-right.svg}
View example

\begin{verbatim}
#set math.mat(column-gap: 1em)
$ mat(1, 2; 3, 4) $
\end{verbatim}

\includegraphics[width=5in,height=\textheight,keepaspectratio]{/assets/docs/tKmrTRxYwVIL8x7N4tnRyQAAAAAAAAAA.png}

\subsubsection{\texorpdfstring{\texttt{\ rows\ }}{ rows }}\label{parameters-rows}

\href{/docs/reference/foundations/array/}{array}

{Required} {{ Positional }}

\phantomsection\label{parameters-rows-positional-tooltip}
Positional parameters are specified in order, without names.

{{ Variadic }}

\phantomsection\label{parameters-rows-variadic-tooltip}
Variadic parameters can be specified multiple times.

An array of arrays with the rows of the matrix.

\includesvg[width=0.16667in,height=0.16667in]{/assets/icons/16-arrow-right.svg}
View example

\begin{verbatim}
#let data = ((1, 2, 3), (4, 5, 6))
#let matrix = math.mat(..data)
$ v := matrix $
\end{verbatim}

\includegraphics[width=5in,height=\textheight,keepaspectratio]{/assets/docs/N-7caJ4FsPlOdlVrUrNk9gAAAAAAAAAA.png}

\href{/docs/reference/math/lr/}{\pandocbounded{\includesvg[keepaspectratio]{/assets/icons/16-arrow-right.svg}}}

{ Left/Right } { Previous page }

\href{/docs/reference/math/primes/}{\pandocbounded{\includesvg[keepaspectratio]{/assets/icons/16-arrow-right.svg}}}

{ Primes } { Next page }


\title{typst.app/docs/reference/math/frac}

\begin{itemize}
\tightlist
\item
  \href{/docs}{\includesvg[width=0.16667in,height=0.16667in]{/assets/icons/16-docs-dark.svg}}
\item
  \includesvg[width=0.16667in,height=0.16667in]{/assets/icons/16-arrow-right.svg}
\item
  \href{/docs/reference/}{Reference}
\item
  \includesvg[width=0.16667in,height=0.16667in]{/assets/icons/16-arrow-right.svg}
\item
  \href{/docs/reference/math/}{Math}
\item
  \includesvg[width=0.16667in,height=0.16667in]{/assets/icons/16-arrow-right.svg}
\item
  \href{/docs/reference/math/frac/}{Fraction}
\end{itemize}

\section{\texorpdfstring{\texttt{\ frac\ } {{ Element
}}}{ frac   Element }}\label{summary}

\phantomsection\label{element-tooltip}
Element functions can be customized with \texttt{\ set\ } and
\texttt{\ show\ } rules.

A mathematical fraction.

\subsection{Example}\label{example}

\begin{verbatim}
$ 1/2 < (x+1)/2 $
$ ((x+1)) / 2 = frac(a, b) $
\end{verbatim}

\includegraphics[width=5in,height=\textheight,keepaspectratio]{/assets/docs/9RFsr-VSObielPb4Nrr-zQAAAAAAAAAA.png}

\subsection{Syntax}\label{syntax}

This function also has dedicated syntax: Use a slash to turn
neighbouring expressions into a fraction. Multiple atoms can be grouped
into a single expression using round grouping parenthesis. Such
parentheses are removed from the output, but you can nest multiple to
force them.

\subsection{\texorpdfstring{{ Parameters
}}{ Parameters }}\label{parameters}

\phantomsection\label{parameters-tooltip}
Parameters are the inputs to a function. They are specified in
parentheses after the function name.

math { . } { frac } (

{ \href{/docs/reference/foundations/content/}{content} , } {
\href{/docs/reference/foundations/content/}{content} , }

) -\textgreater{} \href{/docs/reference/foundations/content/}{content}

\subsubsection{\texorpdfstring{\texttt{\ num\ }}{ num }}\label{parameters-num}

\href{/docs/reference/foundations/content/}{content}

{Required} {{ Positional }}

\phantomsection\label{parameters-num-positional-tooltip}
Positional parameters are specified in order, without names.

The fraction\textquotesingle s numerator.

\subsubsection{\texorpdfstring{\texttt{\ denom\ }}{ denom }}\label{parameters-denom}

\href{/docs/reference/foundations/content/}{content}

{Required} {{ Positional }}

\phantomsection\label{parameters-denom-positional-tooltip}
Positional parameters are specified in order, without names.

The fraction\textquotesingle s denominator.

\href{/docs/reference/math/equation/}{\pandocbounded{\includesvg[keepaspectratio]{/assets/icons/16-arrow-right.svg}}}

{ Equation } { Previous page }

\href{/docs/reference/math/lr/}{\pandocbounded{\includesvg[keepaspectratio]{/assets/icons/16-arrow-right.svg}}}

{ Left/Right } { Next page }


\title{typst.app/docs/reference/math/cancel}

\begin{itemize}
\tightlist
\item
  \href{/docs}{\includesvg[width=0.16667in,height=0.16667in]{/assets/icons/16-docs-dark.svg}}
\item
  \includesvg[width=0.16667in,height=0.16667in]{/assets/icons/16-arrow-right.svg}
\item
  \href{/docs/reference/}{Reference}
\item
  \includesvg[width=0.16667in,height=0.16667in]{/assets/icons/16-arrow-right.svg}
\item
  \href{/docs/reference/math/}{Math}
\item
  \includesvg[width=0.16667in,height=0.16667in]{/assets/icons/16-arrow-right.svg}
\item
  \href{/docs/reference/math/cancel/}{Cancel}
\end{itemize}

\section{\texorpdfstring{\texttt{\ cancel\ } {{ Element
}}}{ cancel   Element }}\label{summary}

\phantomsection\label{element-tooltip}
Element functions can be customized with \texttt{\ set\ } and
\texttt{\ show\ } rules.

Displays a diagonal line over a part of an equation.

This is commonly used to show the elimination of a term.

\subsection{Example}\label{example}

\begin{verbatim}
Here, we can simplify:
$ (a dot b dot cancel(x)) /
    cancel(x) $
\end{verbatim}

\includegraphics[width=2.91667in,height=\textheight,keepaspectratio]{/assets/docs/fVEZvXjKTk2s3WO88t3K8AAAAAAAAAAA.png}

\subsection{\texorpdfstring{{ Parameters
}}{ Parameters }}\label{parameters}

\phantomsection\label{parameters-tooltip}
Parameters are the inputs to a function. They are specified in
parentheses after the function name.

math { . } { cancel } (

{ \href{/docs/reference/foundations/content/}{content} , } {
\hyperref[parameters-length]{length :}
\href{/docs/reference/layout/relative/}{relative} , } {
\hyperref[parameters-inverted]{inverted :}
\href{/docs/reference/foundations/bool/}{bool} , } {
\hyperref[parameters-cross]{cross :}
\href{/docs/reference/foundations/bool/}{bool} , } {
\hyperref[parameters-angle]{angle :}
\href{/docs/reference/foundations/auto/}{auto}
\href{/docs/reference/layout/angle/}{angle}
\href{/docs/reference/foundations/function/}{function} , } {
\hyperref[parameters-stroke]{stroke :}
\href{/docs/reference/layout/length/}{length}
\href{/docs/reference/visualize/color/}{color}
\href{/docs/reference/visualize/gradient/}{gradient}
\href{/docs/reference/visualize/stroke/}{stroke}
\href{/docs/reference/visualize/pattern/}{pattern}
\href{/docs/reference/foundations/dictionary/}{dictionary} , }

) -\textgreater{} \href{/docs/reference/foundations/content/}{content}

\subsubsection{\texorpdfstring{\texttt{\ body\ }}{ body }}\label{parameters-body}

\href{/docs/reference/foundations/content/}{content}

{Required} {{ Positional }}

\phantomsection\label{parameters-body-positional-tooltip}
Positional parameters are specified in order, without names.

The content over which the line should be placed.

\subsubsection{\texorpdfstring{\texttt{\ length\ }}{ length }}\label{parameters-length}

\href{/docs/reference/layout/relative/}{relative}

{{ Settable }}

\phantomsection\label{parameters-length-settable-tooltip}
Settable parameters can be customized for all following uses of the
function with a \texttt{\ set\ } rule.

The length of the line, relative to the length of the diagonal spanning
the whole element being "cancelled". A value of
\texttt{\ }{\texttt{\ 100\%\ }}\texttt{\ } would then have the line span
precisely the element\textquotesingle s diagonal.

Default:
\texttt{\ }{\texttt{\ 100\%\ }}\texttt{\ }{\texttt{\ +\ }}\texttt{\ }{\texttt{\ 3pt\ }}\texttt{\ }

\includesvg[width=0.16667in,height=0.16667in]{/assets/icons/16-arrow-right.svg}
View example

\begin{verbatim}
$ a + cancel(x, length: #200%)
    - cancel(x, length: #200%) $
\end{verbatim}

\includegraphics[width=2.91667in,height=\textheight,keepaspectratio]{/assets/docs/_RSKVrNDnF5_pAJyRMmcrAAAAAAAAAAA.png}

\subsubsection{\texorpdfstring{\texttt{\ inverted\ }}{ inverted }}\label{parameters-inverted}

\href{/docs/reference/foundations/bool/}{bool}

{{ Settable }}

\phantomsection\label{parameters-inverted-settable-tooltip}
Settable parameters can be customized for all following uses of the
function with a \texttt{\ set\ } rule.

Whether the cancel line should be inverted (flipped along the y-axis).
For the default angle setting, inverted means the cancel line points to
the top left instead of top right.

Default: \texttt{\ }{\texttt{\ false\ }}\texttt{\ }

\includesvg[width=0.16667in,height=0.16667in]{/assets/icons/16-arrow-right.svg}
View example

\begin{verbatim}
$ (a cancel((b + c), inverted: #true)) /
    cancel(b + c, inverted: #true) $
\end{verbatim}

\includegraphics[width=2.91667in,height=\textheight,keepaspectratio]{/assets/docs/GWluRapeZy8kHQiZ5c3XbQAAAAAAAAAA.png}

\subsubsection{\texorpdfstring{\texttt{\ cross\ }}{ cross }}\label{parameters-cross}

\href{/docs/reference/foundations/bool/}{bool}

{{ Settable }}

\phantomsection\label{parameters-cross-settable-tooltip}
Settable parameters can be customized for all following uses of the
function with a \texttt{\ set\ } rule.

Whether two opposing cancel lines should be drawn, forming a cross over
the element. Overrides \texttt{\ inverted\ } .

Default: \texttt{\ }{\texttt{\ false\ }}\texttt{\ }

\includesvg[width=0.16667in,height=0.16667in]{/assets/icons/16-arrow-right.svg}
View example

\begin{verbatim}
$ cancel(Pi, cross: #true) $
\end{verbatim}

\includegraphics[width=2.91667in,height=\textheight,keepaspectratio]{/assets/docs/biIi09LikcDnwaA0WaNwJQAAAAAAAAAA.png}

\subsubsection{\texorpdfstring{\texttt{\ angle\ }}{ angle }}\label{parameters-angle}

\href{/docs/reference/foundations/auto/}{auto} {or}
\href{/docs/reference/layout/angle/}{angle} {or}
\href{/docs/reference/foundations/function/}{function}

{{ Settable }}

\phantomsection\label{parameters-angle-settable-tooltip}
Settable parameters can be customized for all following uses of the
function with a \texttt{\ set\ } rule.

How much to rotate the cancel line.

\begin{itemize}
\tightlist
\item
  If given an angle, the line is rotated by that angle clockwise with
  respect to the y-axis.
\item
  If \texttt{\ }{\texttt{\ auto\ }}\texttt{\ } , the line assumes the
  default angle; that is, along the rising diagonal of the content box.
\item
  If given a function \texttt{\ angle\ =\textgreater{}\ angle\ } , the
  line is rotated, with respect to the y-axis, by the angle returned by
  that function. The function receives the default angle as its input.
\end{itemize}

Default: \texttt{\ }{\texttt{\ auto\ }}\texttt{\ }

\includesvg[width=0.16667in,height=0.16667in]{/assets/icons/16-arrow-right.svg}
View example

\begin{verbatim}
$ cancel(Pi)
  cancel(Pi, angle: #0deg)
  cancel(Pi, angle: #45deg)
  cancel(Pi, angle: #90deg)
  cancel(1/(1+x), angle: #(a => a + 45deg))
  cancel(1/(1+x), angle: #(a => a + 90deg)) $
\end{verbatim}

\includegraphics[width=2.91667in,height=\textheight,keepaspectratio]{/assets/docs/OCEmML9KQSY4Sru0zk3XGwAAAAAAAAAA.png}

\subsubsection{\texorpdfstring{\texttt{\ stroke\ }}{ stroke }}\label{parameters-stroke}

\href{/docs/reference/layout/length/}{length} {or}
\href{/docs/reference/visualize/color/}{color} {or}
\href{/docs/reference/visualize/gradient/}{gradient} {or}
\href{/docs/reference/visualize/stroke/}{stroke} {or}
\href{/docs/reference/visualize/pattern/}{pattern} {or}
\href{/docs/reference/foundations/dictionary/}{dictionary}

{{ Settable }}

\phantomsection\label{parameters-stroke-settable-tooltip}
Settable parameters can be customized for all following uses of the
function with a \texttt{\ set\ } rule.

How to \href{/docs/reference/visualize/stroke/}{stroke} the cancel line.

Default: \texttt{\ }{\texttt{\ 0.5pt\ }}\texttt{\ }

\includesvg[width=0.16667in,height=0.16667in]{/assets/icons/16-arrow-right.svg}
View example

\begin{verbatim}
$ cancel(
  sum x,
  stroke: #(
    paint: red,
    thickness: 1.5pt,
    dash: "dashed",
  ),
) $
\end{verbatim}

\includegraphics[width=2.91667in,height=\textheight,keepaspectratio]{/assets/docs/KCV7eimRh0Q3LxZudj8IDAAAAAAAAAAA.png}

\href{/docs/reference/math/binom/}{\pandocbounded{\includesvg[keepaspectratio]{/assets/icons/16-arrow-right.svg}}}

{ Binomial } { Previous page }

\href{/docs/reference/math/cases/}{\pandocbounded{\includesvg[keepaspectratio]{/assets/icons/16-arrow-right.svg}}}

{ Cases } { Next page }


\title{typst.app/docs/reference/math/primes}

\begin{itemize}
\tightlist
\item
  \href{/docs}{\includesvg[width=0.16667in,height=0.16667in]{/assets/icons/16-docs-dark.svg}}
\item
  \includesvg[width=0.16667in,height=0.16667in]{/assets/icons/16-arrow-right.svg}
\item
  \href{/docs/reference/}{Reference}
\item
  \includesvg[width=0.16667in,height=0.16667in]{/assets/icons/16-arrow-right.svg}
\item
  \href{/docs/reference/math/}{Math}
\item
  \includesvg[width=0.16667in,height=0.16667in]{/assets/icons/16-arrow-right.svg}
\item
  \href{/docs/reference/math/primes/}{Primes}
\end{itemize}

\section{\texorpdfstring{\texttt{\ primes\ } {{ Element
}}}{ primes   Element }}\label{summary}

\phantomsection\label{element-tooltip}
Element functions can be customized with \texttt{\ set\ } and
\texttt{\ show\ } rules.

Grouped primes.

\begin{verbatim}
$ a'''_b = a^'''_b $
\end{verbatim}

\includegraphics[width=5in,height=\textheight,keepaspectratio]{/assets/docs/uHgNvego3SyqChIc3iZ9sQAAAAAAAAAA.png}

\subsection{Syntax}\label{syntax}

This function has dedicated syntax: use apostrophes instead of primes.
They will automatically attach to the previous element, moving
superscripts to the next level.

\subsection{\texorpdfstring{{ Parameters
}}{ Parameters }}\label{parameters}

\phantomsection\label{parameters-tooltip}
Parameters are the inputs to a function. They are specified in
parentheses after the function name.

math { . } { primes } (

{ \href{/docs/reference/foundations/int/}{int} }

) -\textgreater{} \href{/docs/reference/foundations/content/}{content}

\subsubsection{\texorpdfstring{\texttt{\ count\ }}{ count }}\label{parameters-count}

\href{/docs/reference/foundations/int/}{int}

{Required} {{ Positional }}

\phantomsection\label{parameters-count-positional-tooltip}
Positional parameters are specified in order, without names.

The number of grouped primes.

\href{/docs/reference/math/mat/}{\pandocbounded{\includesvg[keepaspectratio]{/assets/icons/16-arrow-right.svg}}}

{ Matrix } { Previous page }

\href{/docs/reference/math/roots/}{\pandocbounded{\includesvg[keepaspectratio]{/assets/icons/16-arrow-right.svg}}}

{ Roots } { Next page }


\title{typst.app/docs/reference/math/lr}

\begin{itemize}
\tightlist
\item
  \href{/docs}{\includesvg[width=0.16667in,height=0.16667in]{/assets/icons/16-docs-dark.svg}}
\item
  \includesvg[width=0.16667in,height=0.16667in]{/assets/icons/16-arrow-right.svg}
\item
  \href{/docs/reference/}{Reference}
\item
  \includesvg[width=0.16667in,height=0.16667in]{/assets/icons/16-arrow-right.svg}
\item
  \href{/docs/reference/math/}{Math}
\item
  \includesvg[width=0.16667in,height=0.16667in]{/assets/icons/16-arrow-right.svg}
\item
  \href{/docs/reference/math/lr}{Left/Right}
\end{itemize}

\section{Left/Right}\label{summary}

Delimiter matching.

The \texttt{\ lr\ } function allows you to match two delimiters and
scale them with the content they contain. While this also happens
automatically for delimiters that match syntactically, \texttt{\ lr\ }
allows you to match two arbitrary delimiters and control their size
exactly. Apart from the \texttt{\ lr\ } function, Typst provides a few
more functions that create delimiter pairings for absolute, ceiled, and
floored values as well as norms.

\subsection{Example}\label{example}

\begin{verbatim}
$ [a, b/2] $
$ lr(]sum_(x=1)^n], size: #50%) x $
$ abs((x + y) / 2) $
\end{verbatim}

\includegraphics[width=5in,height=\textheight,keepaspectratio]{/assets/docs/ftGuzhHsliOe05r2qFQMwQAAAAAAAAAA.png}

\subsection{Functions}\label{functions}

\subsubsection{\texorpdfstring{\texttt{\ lr\ } {{ Element
}}}{ lr   Element }}\label{functions-lr}

\phantomsection\label{functions-lr-element-tooltip}
Element functions can be customized with \texttt{\ set\ } and
\texttt{\ show\ } rules.

Scales delimiters.

While matched delimiters scale by default, this can be used to scale
unmatched delimiters and to control the delimiter scaling more
precisely.

math { . } { lr } (

{ \hyperref[functions-lr-parameters-size]{size :}
\href{/docs/reference/foundations/auto/}{auto}
\href{/docs/reference/layout/relative/}{relative} , } {
\href{/docs/reference/foundations/content/}{content} , }

) -\textgreater{} \href{/docs/reference/foundations/content/}{content}

\paragraph{\texorpdfstring{\texttt{\ size\ }}{ size }}\label{functions-lr-size}

\href{/docs/reference/foundations/auto/}{auto} {or}
\href{/docs/reference/layout/relative/}{relative}

{{ Settable }}

\phantomsection\label{functions-lr-size-settable-tooltip}
Settable parameters can be customized for all following uses of the
function with a \texttt{\ set\ } rule.

The size of the brackets, relative to the height of the wrapped content.

Default: \texttt{\ }{\texttt{\ auto\ }}\texttt{\ }

\paragraph{\texorpdfstring{\texttt{\ body\ }}{ body }}\label{functions-lr-body}

\href{/docs/reference/foundations/content/}{content}

{Required} {{ Positional }}

\phantomsection\label{functions-lr-body-positional-tooltip}
Positional parameters are specified in order, without names.

The delimited content, including the delimiters.

\subsubsection{\texorpdfstring{\texttt{\ mid\ } {{ Element
}}}{ mid   Element }}\label{functions-mid}

\phantomsection\label{functions-mid-element-tooltip}
Element functions can be customized with \texttt{\ set\ } and
\texttt{\ show\ } rules.

Scales delimiters vertically to the nearest surrounding
\texttt{\ }{\texttt{\ lr\ }}\texttt{\ }{\texttt{\ (\ }}\texttt{\ }{\texttt{\ )\ }}\texttt{\ }
group.

math { . } { mid } (

{ \href{/docs/reference/foundations/content/}{content} }

) -\textgreater{} \href{/docs/reference/foundations/content/}{content}

\begin{verbatim}
$ { x mid(|) sum_(i=1)^n w_i|f_i (x)| < 1 } $
\end{verbatim}

\includegraphics[width=5in,height=\textheight,keepaspectratio]{/assets/docs/op-SkIh83R9BuQA_mC41YAAAAAAAAAAA.png}

\paragraph{\texorpdfstring{\texttt{\ body\ }}{ body }}\label{functions-mid-body}

\href{/docs/reference/foundations/content/}{content}

{Required} {{ Positional }}

\phantomsection\label{functions-mid-body-positional-tooltip}
Positional parameters are specified in order, without names.

The content to be scaled.

\subsubsection{\texorpdfstring{\texttt{\ abs\ }}{ abs }}\label{functions-abs}

Takes the absolute value of an expression.

math { . } { abs } (

{ \hyperref[functions-abs-parameters-size]{size :}
\href{/docs/reference/foundations/auto/}{auto}
\href{/docs/reference/layout/relative/}{relative} , } {
\href{/docs/reference/foundations/content/}{content} , }

) -\textgreater{} \href{/docs/reference/foundations/content/}{content}

\begin{verbatim}
$ abs(x/2) $
\end{verbatim}

\includegraphics[width=5in,height=\textheight,keepaspectratio]{/assets/docs/WJLuRK0YgTAAKX7q_RtueAAAAAAAAAAA.png}

\paragraph{\texorpdfstring{\texttt{\ size\ }}{ size }}\label{functions-abs-size}

\href{/docs/reference/foundations/auto/}{auto} {or}
\href{/docs/reference/layout/relative/}{relative}

The size of the brackets, relative to the height of the wrapped content.

\paragraph{\texorpdfstring{\texttt{\ body\ }}{ body }}\label{functions-abs-body}

\href{/docs/reference/foundations/content/}{content}

{Required} {{ Positional }}

\phantomsection\label{functions-abs-body-positional-tooltip}
Positional parameters are specified in order, without names.

The expression to take the absolute value of.

\subsubsection{\texorpdfstring{\texttt{\ norm\ }}{ norm }}\label{functions-norm}

Takes the norm of an expression.

math { . } { norm } (

{ \hyperref[functions-norm-parameters-size]{size :}
\href{/docs/reference/foundations/auto/}{auto}
\href{/docs/reference/layout/relative/}{relative} , } {
\href{/docs/reference/foundations/content/}{content} , }

) -\textgreater{} \href{/docs/reference/foundations/content/}{content}

\begin{verbatim}
$ norm(x/2) $
\end{verbatim}

\includegraphics[width=5in,height=\textheight,keepaspectratio]{/assets/docs/YC6RjZ5CBxOUd9-0Ud9TzQAAAAAAAAAA.png}

\paragraph{\texorpdfstring{\texttt{\ size\ }}{ size }}\label{functions-norm-size}

\href{/docs/reference/foundations/auto/}{auto} {or}
\href{/docs/reference/layout/relative/}{relative}

The size of the brackets, relative to the height of the wrapped content.

\paragraph{\texorpdfstring{\texttt{\ body\ }}{ body }}\label{functions-norm-body}

\href{/docs/reference/foundations/content/}{content}

{Required} {{ Positional }}

\phantomsection\label{functions-norm-body-positional-tooltip}
Positional parameters are specified in order, without names.

The expression to take the norm of.

\subsubsection{\texorpdfstring{\texttt{\ floor\ }}{ floor }}\label{functions-floor}

Floors an expression.

math { . } { floor } (

{ \hyperref[functions-floor-parameters-size]{size :}
\href{/docs/reference/foundations/auto/}{auto}
\href{/docs/reference/layout/relative/}{relative} , } {
\href{/docs/reference/foundations/content/}{content} , }

) -\textgreater{} \href{/docs/reference/foundations/content/}{content}

\begin{verbatim}
$ floor(x/2) $
\end{verbatim}

\includegraphics[width=5in,height=\textheight,keepaspectratio]{/assets/docs/PDEHlUdVGIVhIYs9pZubiAAAAAAAAAAA.png}

\paragraph{\texorpdfstring{\texttt{\ size\ }}{ size }}\label{functions-floor-size}

\href{/docs/reference/foundations/auto/}{auto} {or}
\href{/docs/reference/layout/relative/}{relative}

The size of the brackets, relative to the height of the wrapped content.

\paragraph{\texorpdfstring{\texttt{\ body\ }}{ body }}\label{functions-floor-body}

\href{/docs/reference/foundations/content/}{content}

{Required} {{ Positional }}

\phantomsection\label{functions-floor-body-positional-tooltip}
Positional parameters are specified in order, without names.

The expression to floor.

\subsubsection{\texorpdfstring{\texttt{\ ceil\ }}{ ceil }}\label{functions-ceil}

Ceils an expression.

math { . } { ceil } (

{ \hyperref[functions-ceil-parameters-size]{size :}
\href{/docs/reference/foundations/auto/}{auto}
\href{/docs/reference/layout/relative/}{relative} , } {
\href{/docs/reference/foundations/content/}{content} , }

) -\textgreater{} \href{/docs/reference/foundations/content/}{content}

\begin{verbatim}
$ ceil(x/2) $
\end{verbatim}

\includegraphics[width=5in,height=\textheight,keepaspectratio]{/assets/docs/8M0cDo0mVWiDmMeZvIBqOAAAAAAAAAAA.png}

\paragraph{\texorpdfstring{\texttt{\ size\ }}{ size }}\label{functions-ceil-size}

\href{/docs/reference/foundations/auto/}{auto} {or}
\href{/docs/reference/layout/relative/}{relative}

The size of the brackets, relative to the height of the wrapped content.

\paragraph{\texorpdfstring{\texttt{\ body\ }}{ body }}\label{functions-ceil-body}

\href{/docs/reference/foundations/content/}{content}

{Required} {{ Positional }}

\phantomsection\label{functions-ceil-body-positional-tooltip}
Positional parameters are specified in order, without names.

The expression to ceil.

\subsubsection{\texorpdfstring{\texttt{\ round\ }}{ round }}\label{functions-round}

Rounds an expression.

math { . } { round } (

{ \hyperref[functions-round-parameters-size]{size :}
\href{/docs/reference/foundations/auto/}{auto}
\href{/docs/reference/layout/relative/}{relative} , } {
\href{/docs/reference/foundations/content/}{content} , }

) -\textgreater{} \href{/docs/reference/foundations/content/}{content}

\begin{verbatim}
$ round(x/2) $
\end{verbatim}

\includegraphics[width=5in,height=\textheight,keepaspectratio]{/assets/docs/tF8zASmAKWpzYdWTOE8zPAAAAAAAAAAA.png}

\paragraph{\texorpdfstring{\texttt{\ size\ }}{ size }}\label{functions-round-size}

\href{/docs/reference/foundations/auto/}{auto} {or}
\href{/docs/reference/layout/relative/}{relative}

The size of the brackets, relative to the height of the wrapped content.

\paragraph{\texorpdfstring{\texttt{\ body\ }}{ body }}\label{functions-round-body}

\href{/docs/reference/foundations/content/}{content}

{Required} {{ Positional }}

\phantomsection\label{functions-round-body-positional-tooltip}
Positional parameters are specified in order, without names.

The expression to round.

\href{/docs/reference/math/frac/}{\pandocbounded{\includesvg[keepaspectratio]{/assets/icons/16-arrow-right.svg}}}

{ Fraction } { Previous page }

\href{/docs/reference/math/mat/}{\pandocbounded{\includesvg[keepaspectratio]{/assets/icons/16-arrow-right.svg}}}

{ Matrix } { Next page }


\title{typst.app/docs/reference/math/class}

\begin{itemize}
\tightlist
\item
  \href{/docs}{\includesvg[width=0.16667in,height=0.16667in]{/assets/icons/16-docs-dark.svg}}
\item
  \includesvg[width=0.16667in,height=0.16667in]{/assets/icons/16-arrow-right.svg}
\item
  \href{/docs/reference/}{Reference}
\item
  \includesvg[width=0.16667in,height=0.16667in]{/assets/icons/16-arrow-right.svg}
\item
  \href{/docs/reference/math/}{Math}
\item
  \includesvg[width=0.16667in,height=0.16667in]{/assets/icons/16-arrow-right.svg}
\item
  \href{/docs/reference/math/class/}{Class}
\end{itemize}

\section{\texorpdfstring{\texttt{\ class\ } {{ Element
}}}{ class   Element }}\label{summary}

\phantomsection\label{element-tooltip}
Element functions can be customized with \texttt{\ set\ } and
\texttt{\ show\ } rules.

Forced use of a certain math class.

This is useful to treat certain symbols as if they were of a different
class, e.g. to make a symbol behave like a relation. The class of a
symbol defines the way it is laid out, including spacing around it, and
how its scripts are attached by default. Note that the latter can always
be overridden using
\href{/docs/reference/math/attach/\#functions-limits}{\texttt{\ limits\ }}
and
\href{/docs/reference/math/attach/\#functions-scripts}{\texttt{\ scripts\ }}
.

\subsection{Example}\label{example}

\begin{verbatim}
#let loves = math.class(
  "relation",
  sym.suit.heart,
)

$x loves y and y loves 5$
\end{verbatim}

\includegraphics[width=5in,height=\textheight,keepaspectratio]{/assets/docs/4-1urHqzMZfIf7fLTw_1MAAAAAAAAAAA.png}

\subsection{\texorpdfstring{{ Parameters
}}{ Parameters }}\label{parameters}

\phantomsection\label{parameters-tooltip}
Parameters are the inputs to a function. They are specified in
parentheses after the function name.

math { . } { class } (

{ \href{/docs/reference/foundations/str/}{str} , } {
\href{/docs/reference/foundations/content/}{content} , }

) -\textgreater{} \href{/docs/reference/foundations/content/}{content}

\subsubsection{\texorpdfstring{\texttt{\ class\ }}{ class }}\label{parameters-class}

\href{/docs/reference/foundations/str/}{str}

{Required} {{ Positional }}

\phantomsection\label{parameters-class-positional-tooltip}
Positional parameters are specified in order, without names.

The class to apply to the content.

\includesvg[width=0.16667in,height=0.16667in]{/assets/icons/16-arrow-right.svg}
View options

\begin{longtable}[]{@{}ll@{}}
\toprule\noalign{}
Variant & Details \\
\midrule\noalign{}
\endhead
\bottomrule\noalign{}
\endlastfoot
\texttt{\ "\ normal\ "\ } & The default class for non-special things. \\
\texttt{\ "\ punctuation\ "\ } & Punctuation, e.g. a comma. \\
\texttt{\ "\ opening\ "\ } & An opening delimiter, e.g. \texttt{\ (\ }
. \\
\texttt{\ "\ closing\ "\ } & A closing delimiter, e.g. \texttt{\ )\ }
. \\
\texttt{\ "\ fence\ "\ } & A delimiter that is the same on both sides,
e.g. \texttt{\ \textbar{}\ } . \\
\texttt{\ "\ large\ "\ } & A large operator like \texttt{\ sum\ } . \\
\texttt{\ "\ relation\ "\ } & A relation like \texttt{\ =\ } or
\texttt{\ prec\ } . \\
\texttt{\ "\ unary\ "\ } & A unary operator like \texttt{\ not\ } . \\
\texttt{\ "\ binary\ "\ } & A binary operator like \texttt{\ times\ }
. \\
\texttt{\ "\ vary\ "\ } & An operator that can be both unary or binary
like \texttt{\ +\ } . \\
\end{longtable}

\subsubsection{\texorpdfstring{\texttt{\ body\ }}{ body }}\label{parameters-body}

\href{/docs/reference/foundations/content/}{content}

{Required} {{ Positional }}

\phantomsection\label{parameters-body-positional-tooltip}
Positional parameters are specified in order, without names.

The content to which the class is applied.

\href{/docs/reference/math/cases/}{\pandocbounded{\includesvg[keepaspectratio]{/assets/icons/16-arrow-right.svg}}}

{ Cases } { Previous page }

\href{/docs/reference/math/equation/}{\pandocbounded{\includesvg[keepaspectratio]{/assets/icons/16-arrow-right.svg}}}

{ Equation } { Next page }


\title{typst.app/docs/reference/math/binom}

\begin{itemize}
\tightlist
\item
  \href{/docs}{\includesvg[width=0.16667in,height=0.16667in]{/assets/icons/16-docs-dark.svg}}
\item
  \includesvg[width=0.16667in,height=0.16667in]{/assets/icons/16-arrow-right.svg}
\item
  \href{/docs/reference/}{Reference}
\item
  \includesvg[width=0.16667in,height=0.16667in]{/assets/icons/16-arrow-right.svg}
\item
  \href{/docs/reference/math/}{Math}
\item
  \includesvg[width=0.16667in,height=0.16667in]{/assets/icons/16-arrow-right.svg}
\item
  \href{/docs/reference/math/binom/}{Binomial}
\end{itemize}

\section{\texorpdfstring{\texttt{\ binom\ } {{ Element
}}}{ binom   Element }}\label{summary}

\phantomsection\label{element-tooltip}
Element functions can be customized with \texttt{\ set\ } and
\texttt{\ show\ } rules.

A binomial expression.

\subsection{Example}\label{example}

\begin{verbatim}
$ binom(n, k) $
$ binom(n, k_1, k_2, k_3, ..., k_m) $
\end{verbatim}

\includegraphics[width=5in,height=\textheight,keepaspectratio]{/assets/docs/x7e1yoGny67cX0IzBxp69AAAAAAAAAAA.png}

\subsection{\texorpdfstring{{ Parameters
}}{ Parameters }}\label{parameters}

\phantomsection\label{parameters-tooltip}
Parameters are the inputs to a function. They are specified in
parentheses after the function name.

math { . } { binom } (

{ \href{/docs/reference/foundations/content/}{content} , } {
\hyperref[parameters-lower]{..}
\href{/docs/reference/foundations/content/}{content} , }

) -\textgreater{} \href{/docs/reference/foundations/content/}{content}

\subsubsection{\texorpdfstring{\texttt{\ upper\ }}{ upper }}\label{parameters-upper}

\href{/docs/reference/foundations/content/}{content}

{Required} {{ Positional }}

\phantomsection\label{parameters-upper-positional-tooltip}
Positional parameters are specified in order, without names.

The binomial\textquotesingle s upper index.

\subsubsection{\texorpdfstring{\texttt{\ lower\ }}{ lower }}\label{parameters-lower}

\href{/docs/reference/foundations/content/}{content}

{Required} {{ Positional }}

\phantomsection\label{parameters-lower-positional-tooltip}
Positional parameters are specified in order, without names.

{{ Variadic }}

\phantomsection\label{parameters-lower-variadic-tooltip}
Variadic parameters can be specified multiple times.

The binomial\textquotesingle s lower index.

\href{/docs/reference/math/attach/}{\pandocbounded{\includesvg[keepaspectratio]{/assets/icons/16-arrow-right.svg}}}

{ Attach } { Previous page }

\href{/docs/reference/math/cancel/}{\pandocbounded{\includesvg[keepaspectratio]{/assets/icons/16-arrow-right.svg}}}

{ Cancel } { Next page }


\title{typst.app/docs/reference/math/cases}

\begin{itemize}
\tightlist
\item
  \href{/docs}{\includesvg[width=0.16667in,height=0.16667in]{/assets/icons/16-docs-dark.svg}}
\item
  \includesvg[width=0.16667in,height=0.16667in]{/assets/icons/16-arrow-right.svg}
\item
  \href{/docs/reference/}{Reference}
\item
  \includesvg[width=0.16667in,height=0.16667in]{/assets/icons/16-arrow-right.svg}
\item
  \href{/docs/reference/math/}{Math}
\item
  \includesvg[width=0.16667in,height=0.16667in]{/assets/icons/16-arrow-right.svg}
\item
  \href{/docs/reference/math/cases/}{Cases}
\end{itemize}

\section{\texorpdfstring{\texttt{\ cases\ } {{ Element
}}}{ cases   Element }}\label{summary}

\phantomsection\label{element-tooltip}
Element functions can be customized with \texttt{\ set\ } and
\texttt{\ show\ } rules.

A case distinction.

Content across different branches can be aligned with the
\texttt{\ \&\ } symbol.

\subsection{Example}\label{example}

\begin{verbatim}
$ f(x, y) := cases(
  1 "if" (x dot y)/2 <= 0,
  2 "if" x "is even",
  3 "if" x in NN,
  4 "else",
) $
\end{verbatim}

\includegraphics[width=5in,height=\textheight,keepaspectratio]{/assets/docs/0X1AFPDieBd9jiawKpc0-AAAAAAAAAAA.png}

\subsection{\texorpdfstring{{ Parameters
}}{ Parameters }}\label{parameters}

\phantomsection\label{parameters-tooltip}
Parameters are the inputs to a function. They are specified in
parentheses after the function name.

math { . } { cases } (

{ \hyperref[parameters-delim]{delim :}
\href{/docs/reference/foundations/none/}{none}
\href{/docs/reference/foundations/str/}{str}
\href{/docs/reference/foundations/array/}{array}
\href{/docs/reference/symbols/symbol/}{symbol} , } {
\hyperref[parameters-reverse]{reverse :}
\href{/docs/reference/foundations/bool/}{bool} , } {
\hyperref[parameters-gap]{gap :}
\href{/docs/reference/layout/relative/}{relative} , } {
\hyperref[parameters-children]{..}
\href{/docs/reference/foundations/content/}{content} , }

) -\textgreater{} \href{/docs/reference/foundations/content/}{content}

\subsubsection{\texorpdfstring{\texttt{\ delim\ }}{ delim }}\label{parameters-delim}

\href{/docs/reference/foundations/none/}{none} {or}
\href{/docs/reference/foundations/str/}{str} {or}
\href{/docs/reference/foundations/array/}{array} {or}
\href{/docs/reference/symbols/symbol/}{symbol}

{{ Settable }}

\phantomsection\label{parameters-delim-settable-tooltip}
Settable parameters can be customized for all following uses of the
function with a \texttt{\ set\ } rule.

The delimiter to use.

Can be a single character specifying the left delimiter, in which case
the right delimiter is inferred. Otherwise, can be an array containing a
left and a right delimiter.

Default:
\texttt{\ }{\texttt{\ (\ }}\texttt{\ }{\texttt{\ "\{"\ }}\texttt{\ }{\texttt{\ ,\ }}\texttt{\ }{\texttt{\ "\}"\ }}\texttt{\ }{\texttt{\ )\ }}\texttt{\ }

\includesvg[width=0.16667in,height=0.16667in]{/assets/icons/16-arrow-right.svg}
View example

\begin{verbatim}
#set math.cases(delim: "[")
$ x = cases(1, 2) $
\end{verbatim}

\includegraphics[width=5in,height=\textheight,keepaspectratio]{/assets/docs/bErdOHWWOQLSKtsxtJeY5QAAAAAAAAAA.png}

\subsubsection{\texorpdfstring{\texttt{\ reverse\ }}{ reverse }}\label{parameters-reverse}

\href{/docs/reference/foundations/bool/}{bool}

{{ Settable }}

\phantomsection\label{parameters-reverse-settable-tooltip}
Settable parameters can be customized for all following uses of the
function with a \texttt{\ set\ } rule.

Whether the direction of cases should be reversed.

Default: \texttt{\ }{\texttt{\ false\ }}\texttt{\ }

\includesvg[width=0.16667in,height=0.16667in]{/assets/icons/16-arrow-right.svg}
View example

\begin{verbatim}
#set math.cases(reverse: true)
$ cases(1, 2) = x $
\end{verbatim}

\includegraphics[width=5in,height=\textheight,keepaspectratio]{/assets/docs/z6AQZKJsH9nM95e6Aw0hGgAAAAAAAAAA.png}

\subsubsection{\texorpdfstring{\texttt{\ gap\ }}{ gap }}\label{parameters-gap}

\href{/docs/reference/layout/relative/}{relative}

{{ Settable }}

\phantomsection\label{parameters-gap-settable-tooltip}
Settable parameters can be customized for all following uses of the
function with a \texttt{\ set\ } rule.

The gap between branches.

Default:
\texttt{\ }{\texttt{\ 0\%\ }}\texttt{\ }{\texttt{\ +\ }}\texttt{\ }{\texttt{\ 0.2em\ }}\texttt{\ }

\includesvg[width=0.16667in,height=0.16667in]{/assets/icons/16-arrow-right.svg}
View example

\begin{verbatim}
#set math.cases(gap: 1em)
$ x = cases(1, 2) $
\end{verbatim}

\includegraphics[width=5in,height=\textheight,keepaspectratio]{/assets/docs/-xscfzRH4Dw6Yi5TCvpkVwAAAAAAAAAA.png}

\subsubsection{\texorpdfstring{\texttt{\ children\ }}{ children }}\label{parameters-children}

\href{/docs/reference/foundations/content/}{content}

{Required} {{ Positional }}

\phantomsection\label{parameters-children-positional-tooltip}
Positional parameters are specified in order, without names.

{{ Variadic }}

\phantomsection\label{parameters-children-variadic-tooltip}
Variadic parameters can be specified multiple times.

The branches of the case distinction.

\href{/docs/reference/math/cancel/}{\pandocbounded{\includesvg[keepaspectratio]{/assets/icons/16-arrow-right.svg}}}

{ Cancel } { Previous page }

\href{/docs/reference/math/class/}{\pandocbounded{\includesvg[keepaspectratio]{/assets/icons/16-arrow-right.svg}}}

{ Class } { Next page }


\title{typst.app/docs/reference/math/sizes}

\begin{itemize}
\tightlist
\item
  \href{/docs}{\includesvg[width=0.16667in,height=0.16667in]{/assets/icons/16-docs-dark.svg}}
\item
  \includesvg[width=0.16667in,height=0.16667in]{/assets/icons/16-arrow-right.svg}
\item
  \href{/docs/reference/}{Reference}
\item
  \includesvg[width=0.16667in,height=0.16667in]{/assets/icons/16-arrow-right.svg}
\item
  \href{/docs/reference/math/}{Math}
\item
  \includesvg[width=0.16667in,height=0.16667in]{/assets/icons/16-arrow-right.svg}
\item
  \href{/docs/reference/math/sizes}{Sizes}
\end{itemize}

\section{Sizes}\label{summary}

Forced size styles for expressions within formulas.

These functions allow manual configuration of the size of equation
elements to make them look as in a display/inline equation or as if used
in a root or sub/superscripts.

\subsection{Functions}\label{functions}

\subsubsection{\texorpdfstring{\texttt{\ display\ }}{ display }}\label{functions-display}

Forced display style in math.

This is the normal size for block equations.

math { . } { display } (

{ \href{/docs/reference/foundations/content/}{content} , } {
\hyperref[functions-display-parameters-cramped]{cramped :}
\href{/docs/reference/foundations/bool/}{bool} , }

) -\textgreater{} \href{/docs/reference/foundations/content/}{content}

\begin{verbatim}
$sum_i x_i/2 = display(sum_i x_i/2)$
\end{verbatim}

\includegraphics[width=5in,height=\textheight,keepaspectratio]{/assets/docs/Kw_xKFEpG79sGcim5bh7SgAAAAAAAAAA.png}

\paragraph{\texorpdfstring{\texttt{\ body\ }}{ body }}\label{functions-display-body}

\href{/docs/reference/foundations/content/}{content}

{Required} {{ Positional }}

\phantomsection\label{functions-display-body-positional-tooltip}
Positional parameters are specified in order, without names.

The content to size.

\paragraph{\texorpdfstring{\texttt{\ cramped\ }}{ cramped }}\label{functions-display-cramped}

\href{/docs/reference/foundations/bool/}{bool}

Whether to impose a height restriction for exponents, like regular sub-
and superscripts do.

Default: \texttt{\ }{\texttt{\ false\ }}\texttt{\ }

\subsubsection{\texorpdfstring{\texttt{\ inline\ }}{ inline }}\label{functions-inline}

Forced inline (text) style in math.

This is the normal size for inline equations.

math { . } { inline } (

{ \href{/docs/reference/foundations/content/}{content} , } {
\hyperref[functions-inline-parameters-cramped]{cramped :}
\href{/docs/reference/foundations/bool/}{bool} , }

) -\textgreater{} \href{/docs/reference/foundations/content/}{content}

\begin{verbatim}
$ sum_i x_i/2
    = inline(sum_i x_i/2) $
\end{verbatim}

\includegraphics[width=5in,height=\textheight,keepaspectratio]{/assets/docs/yhhyiAgPa8_SZLz7nNtNqAAAAAAAAAAA.png}

\paragraph{\texorpdfstring{\texttt{\ body\ }}{ body }}\label{functions-inline-body}

\href{/docs/reference/foundations/content/}{content}

{Required} {{ Positional }}

\phantomsection\label{functions-inline-body-positional-tooltip}
Positional parameters are specified in order, without names.

The content to size.

\paragraph{\texorpdfstring{\texttt{\ cramped\ }}{ cramped }}\label{functions-inline-cramped}

\href{/docs/reference/foundations/bool/}{bool}

Whether to impose a height restriction for exponents, like regular sub-
and superscripts do.

Default: \texttt{\ }{\texttt{\ false\ }}\texttt{\ }

\subsubsection{\texorpdfstring{\texttt{\ script\ }}{ script }}\label{functions-script}

Forced script style in math.

This is the smaller size used in powers or sub- or superscripts.

math { . } { script } (

{ \href{/docs/reference/foundations/content/}{content} , } {
\hyperref[functions-script-parameters-cramped]{cramped :}
\href{/docs/reference/foundations/bool/}{bool} , }

) -\textgreater{} \href{/docs/reference/foundations/content/}{content}

\begin{verbatim}
$sum_i x_i/2 = script(sum_i x_i/2)$
\end{verbatim}

\includegraphics[width=5in,height=\textheight,keepaspectratio]{/assets/docs/UAO0CCEy42RrRJk6xg_ljgAAAAAAAAAA.png}

\paragraph{\texorpdfstring{\texttt{\ body\ }}{ body }}\label{functions-script-body}

\href{/docs/reference/foundations/content/}{content}

{Required} {{ Positional }}

\phantomsection\label{functions-script-body-positional-tooltip}
Positional parameters are specified in order, without names.

The content to size.

\paragraph{\texorpdfstring{\texttt{\ cramped\ }}{ cramped }}\label{functions-script-cramped}

\href{/docs/reference/foundations/bool/}{bool}

Whether to impose a height restriction for exponents, like regular sub-
and superscripts do.

Default: \texttt{\ }{\texttt{\ true\ }}\texttt{\ }

\subsubsection{\texorpdfstring{\texttt{\ sscript\ }}{ sscript }}\label{functions-sscript}

Forced second script style in math.

This is the smallest size, used in second-level sub- and superscripts
(script of the script).

math { . } { sscript } (

{ \href{/docs/reference/foundations/content/}{content} , } {
\hyperref[functions-sscript-parameters-cramped]{cramped :}
\href{/docs/reference/foundations/bool/}{bool} , }

) -\textgreater{} \href{/docs/reference/foundations/content/}{content}

\begin{verbatim}
$sum_i x_i/2 = sscript(sum_i x_i/2)$
\end{verbatim}

\includegraphics[width=5in,height=\textheight,keepaspectratio]{/assets/docs/EpmDoJiJrfbN7kA0Km7ujwAAAAAAAAAA.png}

\paragraph{\texorpdfstring{\texttt{\ body\ }}{ body }}\label{functions-sscript-body}

\href{/docs/reference/foundations/content/}{content}

{Required} {{ Positional }}

\phantomsection\label{functions-sscript-body-positional-tooltip}
Positional parameters are specified in order, without names.

The content to size.

\paragraph{\texorpdfstring{\texttt{\ cramped\ }}{ cramped }}\label{functions-sscript-cramped}

\href{/docs/reference/foundations/bool/}{bool}

Whether to impose a height restriction for exponents, like regular sub-
and superscripts do.

Default: \texttt{\ }{\texttt{\ true\ }}\texttt{\ }

\href{/docs/reference/math/roots/}{\pandocbounded{\includesvg[keepaspectratio]{/assets/icons/16-arrow-right.svg}}}

{ Roots } { Previous page }

\href{/docs/reference/math/stretch/}{\pandocbounded{\includesvg[keepaspectratio]{/assets/icons/16-arrow-right.svg}}}

{ Stretch } { Next page }


\title{typst.app/docs/reference/math/vec}

\begin{itemize}
\tightlist
\item
  \href{/docs}{\includesvg[width=0.16667in,height=0.16667in]{/assets/icons/16-docs-dark.svg}}
\item
  \includesvg[width=0.16667in,height=0.16667in]{/assets/icons/16-arrow-right.svg}
\item
  \href{/docs/reference/}{Reference}
\item
  \includesvg[width=0.16667in,height=0.16667in]{/assets/icons/16-arrow-right.svg}
\item
  \href{/docs/reference/math/}{Math}
\item
  \includesvg[width=0.16667in,height=0.16667in]{/assets/icons/16-arrow-right.svg}
\item
  \href{/docs/reference/math/vec/}{Vector}
\end{itemize}

\section{\texorpdfstring{\texttt{\ vec\ } {{ Element
}}}{ vec   Element }}\label{summary}

\phantomsection\label{element-tooltip}
Element functions can be customized with \texttt{\ set\ } and
\texttt{\ show\ } rules.

A column vector.

Content in the vector\textquotesingle s elements can be aligned with the
\href{/docs/reference/math/vec/\#parameters-align}{\texttt{\ align\ }}
parameter, or the \texttt{\ \&\ } symbol.

\subsection{Example}\label{example}

\begin{verbatim}
$ vec(a, b, c) dot vec(1, 2, 3)
    = a + 2b + 3c $
\end{verbatim}

\includegraphics[width=5in,height=\textheight,keepaspectratio]{/assets/docs/LnRm06lLMggD8fCQZdA66QAAAAAAAAAA.png}

\subsection{\texorpdfstring{{ Parameters
}}{ Parameters }}\label{parameters}

\phantomsection\label{parameters-tooltip}
Parameters are the inputs to a function. They are specified in
parentheses after the function name.

math { . } { vec } (

{ \hyperref[parameters-delim]{delim :}
\href{/docs/reference/foundations/none/}{none}
\href{/docs/reference/foundations/str/}{str}
\href{/docs/reference/foundations/array/}{array}
\href{/docs/reference/symbols/symbol/}{symbol} , } {
\hyperref[parameters-align]{align :}
\href{/docs/reference/layout/alignment/}{alignment} , } {
\hyperref[parameters-gap]{gap :}
\href{/docs/reference/layout/relative/}{relative} , } {
\hyperref[parameters-children]{..}
\href{/docs/reference/foundations/content/}{content} , }

) -\textgreater{} \href{/docs/reference/foundations/content/}{content}

\subsubsection{\texorpdfstring{\texttt{\ delim\ }}{ delim }}\label{parameters-delim}

\href{/docs/reference/foundations/none/}{none} {or}
\href{/docs/reference/foundations/str/}{str} {or}
\href{/docs/reference/foundations/array/}{array} {or}
\href{/docs/reference/symbols/symbol/}{symbol}

{{ Settable }}

\phantomsection\label{parameters-delim-settable-tooltip}
Settable parameters can be customized for all following uses of the
function with a \texttt{\ set\ } rule.

The delimiter to use.

Can be a single character specifying the left delimiter, in which case
the right delimiter is inferred. Otherwise, can be an array containing a
left and a right delimiter.

Default:
\texttt{\ }{\texttt{\ (\ }}\texttt{\ }{\texttt{\ "("\ }}\texttt{\ }{\texttt{\ ,\ }}\texttt{\ }{\texttt{\ ")"\ }}\texttt{\ }{\texttt{\ )\ }}\texttt{\ }

\includesvg[width=0.16667in,height=0.16667in]{/assets/icons/16-arrow-right.svg}
View example

\begin{verbatim}
#set math.vec(delim: "[")
$ vec(1, 2) $
\end{verbatim}

\includegraphics[width=5in,height=\textheight,keepaspectratio]{/assets/docs/5LFZJ9d25bljXFp6kARHcgAAAAAAAAAA.png}

\subsubsection{\texorpdfstring{\texttt{\ align\ }}{ align }}\label{parameters-align}

\href{/docs/reference/layout/alignment/}{alignment}

{{ Settable }}

\phantomsection\label{parameters-align-settable-tooltip}
Settable parameters can be customized for all following uses of the
function with a \texttt{\ set\ } rule.

The horizontal alignment that each element should have.

Default: \texttt{\ center\ }

\includesvg[width=0.16667in,height=0.16667in]{/assets/icons/16-arrow-right.svg}
View example

\begin{verbatim}
#set math.vec(align: right)
$ vec(-1, 1, -1) $
\end{verbatim}

\includegraphics[width=5in,height=\textheight,keepaspectratio]{/assets/docs/ZtHlp9Y4zEtz53Ydf5unLAAAAAAAAAAA.png}

\subsubsection{\texorpdfstring{\texttt{\ gap\ }}{ gap }}\label{parameters-gap}

\href{/docs/reference/layout/relative/}{relative}

{{ Settable }}

\phantomsection\label{parameters-gap-settable-tooltip}
Settable parameters can be customized for all following uses of the
function with a \texttt{\ set\ } rule.

The gap between elements.

Default:
\texttt{\ }{\texttt{\ 0\%\ }}\texttt{\ }{\texttt{\ +\ }}\texttt{\ }{\texttt{\ 0.2em\ }}\texttt{\ }

\includesvg[width=0.16667in,height=0.16667in]{/assets/icons/16-arrow-right.svg}
View example

\begin{verbatim}
#set math.vec(gap: 1em)
$ vec(1, 2) $
\end{verbatim}

\includegraphics[width=5in,height=\textheight,keepaspectratio]{/assets/docs/uiK2bQUKjIzcO3IGp7RZPwAAAAAAAAAA.png}

\subsubsection{\texorpdfstring{\texttt{\ children\ }}{ children }}\label{parameters-children}

\href{/docs/reference/foundations/content/}{content}

{Required} {{ Positional }}

\phantomsection\label{parameters-children-positional-tooltip}
Positional parameters are specified in order, without names.

{{ Variadic }}

\phantomsection\label{parameters-children-variadic-tooltip}
Variadic parameters can be specified multiple times.

The elements of the vector.

\href{/docs/reference/math/variants/}{\pandocbounded{\includesvg[keepaspectratio]{/assets/icons/16-arrow-right.svg}}}

{ Variants } { Previous page }

\href{/docs/reference/symbols/}{\pandocbounded{\includesvg[keepaspectratio]{/assets/icons/16-arrow-right.svg}}}

{ Symbols } { Next page }


\title{typst.app/docs/reference/math/roots}

\begin{itemize}
\tightlist
\item
  \href{/docs}{\includesvg[width=0.16667in,height=0.16667in]{/assets/icons/16-docs-dark.svg}}
\item
  \includesvg[width=0.16667in,height=0.16667in]{/assets/icons/16-arrow-right.svg}
\item
  \href{/docs/reference/}{Reference}
\item
  \includesvg[width=0.16667in,height=0.16667in]{/assets/icons/16-arrow-right.svg}
\item
  \href{/docs/reference/math/}{Math}
\item
  \includesvg[width=0.16667in,height=0.16667in]{/assets/icons/16-arrow-right.svg}
\item
  \href{/docs/reference/math/roots}{Roots}
\end{itemize}

\section{Roots}\label{summary}

Square and non-square roots.

\subsection{Example}\label{example}

\begin{verbatim}
$ sqrt(3 - 2 sqrt(2)) = sqrt(2) - 1 $
$ root(3, x) $
\end{verbatim}

\includegraphics[width=5in,height=\textheight,keepaspectratio]{/assets/docs/YJMQ-3S5QEsCnosYijnvKwAAAAAAAAAA.png}

\subsection{Functions}\label{functions}

\subsubsection{\texorpdfstring{\texttt{\ root\ } {{ Element
}}}{ root   Element }}\label{functions-root}

\phantomsection\label{functions-root-element-tooltip}
Element functions can be customized with \texttt{\ set\ } and
\texttt{\ show\ } rules.

A general root.

math { . } { root } (

{ \hyperref[functions-root-parameters-index]{}
\href{/docs/reference/foundations/none/}{none}
\href{/docs/reference/foundations/content/}{content} , } {
\href{/docs/reference/foundations/content/}{content} , }

) -\textgreater{} \href{/docs/reference/foundations/content/}{content}

\begin{verbatim}
$ root(3, x) $
\end{verbatim}

\includegraphics[width=5in,height=\textheight,keepaspectratio]{/assets/docs/5dcBKGUow3rGrUB1Eg_gjwAAAAAAAAAA.png}

\paragraph{\texorpdfstring{\texttt{\ index\ }}{ index }}\label{functions-root-index}

\href{/docs/reference/foundations/none/}{none} {or}
\href{/docs/reference/foundations/content/}{content}

{{ Positional }}

\phantomsection\label{functions-root-index-positional-tooltip}
Positional parameters are specified in order, without names.

{{ Settable }}

\phantomsection\label{functions-root-index-settable-tooltip}
Settable parameters can be customized for all following uses of the
function with a \texttt{\ set\ } rule.

Which root of the radicand to take.

Default: \texttt{\ }{\texttt{\ none\ }}\texttt{\ }

\paragraph{\texorpdfstring{\texttt{\ radicand\ }}{ radicand }}\label{functions-root-radicand}

\href{/docs/reference/foundations/content/}{content}

{Required} {{ Positional }}

\phantomsection\label{functions-root-radicand-positional-tooltip}
Positional parameters are specified in order, without names.

The expression to take the root of.

\subsubsection{\texorpdfstring{\texttt{\ sqrt\ }}{ sqrt }}\label{functions-sqrt}

A square root.

math { . } { sqrt } (

{ \href{/docs/reference/foundations/content/}{content} }

) -\textgreater{} \href{/docs/reference/foundations/content/}{content}

\begin{verbatim}
$ sqrt(3 - 2 sqrt(2)) = sqrt(2) - 1 $
\end{verbatim}

\includegraphics[width=5in,height=\textheight,keepaspectratio]{/assets/docs/5thyKdLM1Lrfm53ILJWqaQAAAAAAAAAA.png}

\paragraph{\texorpdfstring{\texttt{\ radicand\ }}{ radicand }}\label{functions-sqrt-radicand}

\href{/docs/reference/foundations/content/}{content}

{Required} {{ Positional }}

\phantomsection\label{functions-sqrt-radicand-positional-tooltip}
Positional parameters are specified in order, without names.

The expression to take the square root of.

\href{/docs/reference/math/primes/}{\pandocbounded{\includesvg[keepaspectratio]{/assets/icons/16-arrow-right.svg}}}

{ Primes } { Previous page }

\href{/docs/reference/math/sizes/}{\pandocbounded{\includesvg[keepaspectratio]{/assets/icons/16-arrow-right.svg}}}

{ Sizes } { Next page }


\title{typst.app/docs/reference/math/variants}

\begin{itemize}
\tightlist
\item
  \href{/docs}{\includesvg[width=0.16667in,height=0.16667in]{/assets/icons/16-docs-dark.svg}}
\item
  \includesvg[width=0.16667in,height=0.16667in]{/assets/icons/16-arrow-right.svg}
\item
  \href{/docs/reference/}{Reference}
\item
  \includesvg[width=0.16667in,height=0.16667in]{/assets/icons/16-arrow-right.svg}
\item
  \href{/docs/reference/math/}{Math}
\item
  \includesvg[width=0.16667in,height=0.16667in]{/assets/icons/16-arrow-right.svg}
\item
  \href{/docs/reference/math/variants}{Variants}
\end{itemize}

\section{Variants}\label{summary}

Alternate typefaces within formulas.

These functions are distinct from the
\href{/docs/reference/text/text/}{\texttt{\ text\ }} function because
math fonts contain multiple variants of each letter.

\subsection{Functions}\label{functions}

\subsubsection{\texorpdfstring{\texttt{\ serif\ }}{ serif }}\label{functions-serif}

Serif (roman) font style in math.

This is already the default.

math { . } { serif } (

{ \href{/docs/reference/foundations/content/}{content} }

) -\textgreater{} \href{/docs/reference/foundations/content/}{content}

\paragraph{\texorpdfstring{\texttt{\ body\ }}{ body }}\label{functions-serif-body}

\href{/docs/reference/foundations/content/}{content}

{Required} {{ Positional }}

\phantomsection\label{functions-serif-body-positional-tooltip}
Positional parameters are specified in order, without names.

The content to style.

\subsubsection{\texorpdfstring{\texttt{\ sans\ }}{ sans }}\label{functions-sans}

Sans-serif font style in math.

math { . } { sans } (

{ \href{/docs/reference/foundations/content/}{content} }

) -\textgreater{} \href{/docs/reference/foundations/content/}{content}

\begin{verbatim}
$ sans(A B C) $
\end{verbatim}

\includegraphics[width=5in,height=\textheight,keepaspectratio]{/assets/docs/QH7JeXflCs-wCjP8nkBWrQAAAAAAAAAA.png}

\paragraph{\texorpdfstring{\texttt{\ body\ }}{ body }}\label{functions-sans-body}

\href{/docs/reference/foundations/content/}{content}

{Required} {{ Positional }}

\phantomsection\label{functions-sans-body-positional-tooltip}
Positional parameters are specified in order, without names.

The content to style.

\subsubsection{\texorpdfstring{\texttt{\ frak\ }}{ frak }}\label{functions-frak}

Fraktur font style in math.

math { . } { frak } (

{ \href{/docs/reference/foundations/content/}{content} }

) -\textgreater{} \href{/docs/reference/foundations/content/}{content}

\begin{verbatim}
$ frak(P) $
\end{verbatim}

\includegraphics[width=5in,height=\textheight,keepaspectratio]{/assets/docs/e8XkJAdgWXZDqbWs94GeeQAAAAAAAAAA.png}

\paragraph{\texorpdfstring{\texttt{\ body\ }}{ body }}\label{functions-frak-body}

\href{/docs/reference/foundations/content/}{content}

{Required} {{ Positional }}

\phantomsection\label{functions-frak-body-positional-tooltip}
Positional parameters are specified in order, without names.

The content to style.

\subsubsection{\texorpdfstring{\texttt{\ mono\ }}{ mono }}\label{functions-mono}

Monospace font style in math.

math { . } { mono } (

{ \href{/docs/reference/foundations/content/}{content} }

) -\textgreater{} \href{/docs/reference/foundations/content/}{content}

\begin{verbatim}
$ mono(x + y = z) $
\end{verbatim}

\includegraphics[width=5in,height=\textheight,keepaspectratio]{/assets/docs/VdkE7dQPvzJTe7BxDZHcnwAAAAAAAAAA.png}

\paragraph{\texorpdfstring{\texttt{\ body\ }}{ body }}\label{functions-mono-body}

\href{/docs/reference/foundations/content/}{content}

{Required} {{ Positional }}

\phantomsection\label{functions-mono-body-positional-tooltip}
Positional parameters are specified in order, without names.

The content to style.

\subsubsection{\texorpdfstring{\texttt{\ bb\ }}{ bb }}\label{functions-bb}

Blackboard bold (double-struck) font style in math.

For uppercase latin letters, blackboard bold is additionally available
through \href{/docs/reference/symbols/sym/}{symbols} of the form
\texttt{\ NN\ } and \texttt{\ RR\ } .

math { . } { bb } (

{ \href{/docs/reference/foundations/content/}{content} }

) -\textgreater{} \href{/docs/reference/foundations/content/}{content}

\begin{verbatim}
$ bb(b) $
$ bb(N) = NN $
$ f: NN -> RR $
\end{verbatim}

\includegraphics[width=5in,height=\textheight,keepaspectratio]{/assets/docs/7qs4sC1Ha0vO_Ei_dnjHuQAAAAAAAAAA.png}

\paragraph{\texorpdfstring{\texttt{\ body\ }}{ body }}\label{functions-bb-body}

\href{/docs/reference/foundations/content/}{content}

{Required} {{ Positional }}

\phantomsection\label{functions-bb-body-positional-tooltip}
Positional parameters are specified in order, without names.

The content to style.

\subsubsection{\texorpdfstring{\texttt{\ cal\ }}{ cal }}\label{functions-cal}

Calligraphic font style in math.

math { . } { cal } (

{ \href{/docs/reference/foundations/content/}{content} }

) -\textgreater{} \href{/docs/reference/foundations/content/}{content}

\begin{verbatim}
Let $cal(P)$ be the set of ...
\end{verbatim}

\includegraphics[width=5in,height=\textheight,keepaspectratio]{/assets/docs/kqxr3_NhGcBq3QZhkIfIjwAAAAAAAAAA.png}

This corresponds both to LaTeX\textquotesingle s
\texttt{\ \textbackslash{}mathcal\ } and
\texttt{\ \textbackslash{}mathscr\ } as both of these styles share the
same Unicode codepoints. Switching between the styles is thus only
possible if supported by the font via
\href{/docs/reference/text/text/\#parameters-features}{font features} .

For the default math font, the roundhand style is available through the
\texttt{\ ss01\ } feature. Therefore, you could define your own version
of \texttt{\ \textbackslash{}mathscr\ } like this:

\begin{verbatim}
#let scr(it) = text(
  features: ("ss01",),
  box($cal(it)$),
)

We establish $cal(P) != scr(P)$.
\end{verbatim}

\includegraphics[width=5in,height=\textheight,keepaspectratio]{/assets/docs/PLEOQqYY9qiWLwCVv8j_HAAAAAAAAAAA.png}

(The box is not conceptually necessary, but unfortunately currently
needed due to limitations in Typst\textquotesingle s text style handling
in math.)

\paragraph{\texorpdfstring{\texttt{\ body\ }}{ body }}\label{functions-cal-body}

\href{/docs/reference/foundations/content/}{content}

{Required} {{ Positional }}

\phantomsection\label{functions-cal-body-positional-tooltip}
Positional parameters are specified in order, without names.

The content to style.

\href{/docs/reference/math/underover/}{\pandocbounded{\includesvg[keepaspectratio]{/assets/icons/16-arrow-right.svg}}}

{ Under/Over } { Previous page }

\href{/docs/reference/math/vec/}{\pandocbounded{\includesvg[keepaspectratio]{/assets/icons/16-arrow-right.svg}}}

{ Vector } { Next page }


\title{typst.app/docs/reference/math/op}

\begin{itemize}
\tightlist
\item
  \href{/docs}{\includesvg[width=0.16667in,height=0.16667in]{/assets/icons/16-docs-dark.svg}}
\item
  \includesvg[width=0.16667in,height=0.16667in]{/assets/icons/16-arrow-right.svg}
\item
  \href{/docs/reference/}{Reference}
\item
  \includesvg[width=0.16667in,height=0.16667in]{/assets/icons/16-arrow-right.svg}
\item
  \href{/docs/reference/math/}{Math}
\item
  \includesvg[width=0.16667in,height=0.16667in]{/assets/icons/16-arrow-right.svg}
\item
  \href{/docs/reference/math/op/}{Text Operator}
\end{itemize}

\section{\texorpdfstring{\texttt{\ op\ } {{ Element
}}}{ op   Element }}\label{summary}

\phantomsection\label{element-tooltip}
Element functions can be customized with \texttt{\ set\ } and
\texttt{\ show\ } rules.

A text operator in an equation.

\subsection{Example}\label{example}

\begin{verbatim}
$ tan x = (sin x)/(cos x) $
$ op("custom",
     limits: #true)_(n->oo) n $
\end{verbatim}

\includegraphics[width=5in,height=\textheight,keepaspectratio]{/assets/docs/n9yefElmfwTi92ejfLzhZwAAAAAAAAAA.png}

\subsection{Predefined Operators}\label{predefined}

Typst predefines the operators \texttt{\ arccos\ } , \texttt{\ arcsin\ }
, \texttt{\ arctan\ } , \texttt{\ arg\ } , \texttt{\ cos\ } ,
\texttt{\ cosh\ } , \texttt{\ cot\ } , \texttt{\ coth\ } ,
\texttt{\ csc\ } , \texttt{\ csch\ } , \texttt{\ ctg\ } ,
\texttt{\ deg\ } , \texttt{\ det\ } , \texttt{\ dim\ } ,
\texttt{\ exp\ } , \texttt{\ gcd\ } , \texttt{\ hom\ } , \texttt{\ id\ }
, \texttt{\ im\ } , \texttt{\ inf\ } , \texttt{\ ker\ } ,
\texttt{\ lg\ } , \texttt{\ lim\ } , \texttt{\ liminf\ } ,
\texttt{\ limsup\ } , \texttt{\ ln\ } , \texttt{\ log\ } ,
\texttt{\ max\ } , \texttt{\ min\ } , \texttt{\ mod\ } , \texttt{\ Pr\ }
, \texttt{\ sec\ } , \texttt{\ sech\ } , \texttt{\ sin\ } ,
\texttt{\ sinc\ } , \texttt{\ sinh\ } , \texttt{\ sup\ } ,
\texttt{\ tan\ } , \texttt{\ tanh\ } , \texttt{\ tg\ } and
\texttt{\ tr\ } .

\subsection{\texorpdfstring{{ Parameters
}}{ Parameters }}\label{parameters}

\phantomsection\label{parameters-tooltip}
Parameters are the inputs to a function. They are specified in
parentheses after the function name.

math { . } { op } (

{ \href{/docs/reference/foundations/content/}{content} , } {
\hyperref[parameters-limits]{limits :}
\href{/docs/reference/foundations/bool/}{bool} , }

) -\textgreater{} \href{/docs/reference/foundations/content/}{content}

\subsubsection{\texorpdfstring{\texttt{\ text\ }}{ text }}\label{parameters-text}

\href{/docs/reference/foundations/content/}{content}

{Required} {{ Positional }}

\phantomsection\label{parameters-text-positional-tooltip}
Positional parameters are specified in order, without names.

The operator\textquotesingle s text.

\subsubsection{\texorpdfstring{\texttt{\ limits\ }}{ limits }}\label{parameters-limits}

\href{/docs/reference/foundations/bool/}{bool}

{{ Settable }}

\phantomsection\label{parameters-limits-settable-tooltip}
Settable parameters can be customized for all following uses of the
function with a \texttt{\ set\ } rule.

Whether the operator should show attachments as limits in display mode.

Default: \texttt{\ }{\texttt{\ false\ }}\texttt{\ }

\href{/docs/reference/math/styles/}{\pandocbounded{\includesvg[keepaspectratio]{/assets/icons/16-arrow-right.svg}}}

{ Styles } { Previous page }

\href{/docs/reference/math/underover/}{\pandocbounded{\includesvg[keepaspectratio]{/assets/icons/16-arrow-right.svg}}}

{ Under/Over } { Next page }


\title{typst.app/docs/reference/math/accent}

\begin{itemize}
\tightlist
\item
  \href{/docs}{\includesvg[width=0.16667in,height=0.16667in]{/assets/icons/16-docs-dark.svg}}
\item
  \includesvg[width=0.16667in,height=0.16667in]{/assets/icons/16-arrow-right.svg}
\item
  \href{/docs/reference/}{Reference}
\item
  \includesvg[width=0.16667in,height=0.16667in]{/assets/icons/16-arrow-right.svg}
\item
  \href{/docs/reference/math/}{Math}
\item
  \includesvg[width=0.16667in,height=0.16667in]{/assets/icons/16-arrow-right.svg}
\item
  \href{/docs/reference/math/accent/}{Accent}
\end{itemize}

\section{\texorpdfstring{\texttt{\ accent\ } {{ Element
}}}{ accent   Element }}\label{summary}

\phantomsection\label{element-tooltip}
Element functions can be customized with \texttt{\ set\ } and
\texttt{\ show\ } rules.

Attaches an accent to a base.

\subsection{Example}\label{example}

\begin{verbatim}
$grave(a) = accent(a, `)$ \
$arrow(a) = accent(a, arrow)$ \
$tilde(a) = accent(a, \u{0303})$
\end{verbatim}

\includegraphics[width=5in,height=\textheight,keepaspectratio]{/assets/docs/wdLZED2cvtXKAU75vKtAKwAAAAAAAAAA.png}

\subsection{\texorpdfstring{{ Parameters
}}{ Parameters }}\label{parameters}

\phantomsection\label{parameters-tooltip}
Parameters are the inputs to a function. They are specified in
parentheses after the function name.

math { . } { accent } (

{ \href{/docs/reference/foundations/content/}{content} , } {
\href{/docs/reference/foundations/str/}{str}
\href{/docs/reference/foundations/content/}{content} , } {
\hyperref[parameters-size]{size :}
\href{/docs/reference/foundations/auto/}{auto}
\href{/docs/reference/layout/relative/}{relative} , }

) -\textgreater{} \href{/docs/reference/foundations/content/}{content}

\subsubsection{\texorpdfstring{\texttt{\ base\ }}{ base }}\label{parameters-base}

\href{/docs/reference/foundations/content/}{content}

{Required} {{ Positional }}

\phantomsection\label{parameters-base-positional-tooltip}
Positional parameters are specified in order, without names.

The base to which the accent is applied. May consist of multiple
letters.

\includesvg[width=0.16667in,height=0.16667in]{/assets/icons/16-arrow-right.svg}
View example

\begin{verbatim}
$arrow(A B C)$
\end{verbatim}

\includegraphics[width=5in,height=\textheight,keepaspectratio]{/assets/docs/aVpZuZcTglBCvF8kbjxN7AAAAAAAAAAA.png}

\subsubsection{\texorpdfstring{\texttt{\ accent\ }}{ accent }}\label{parameters-accent}

\href{/docs/reference/foundations/str/}{str} {or}
\href{/docs/reference/foundations/content/}{content}

{Required} {{ Positional }}

\phantomsection\label{parameters-accent-positional-tooltip}
Positional parameters are specified in order, without names.

The accent to apply to the base.

Supported accents include:

\begin{longtable}[]{@{}lll@{}}
\toprule\noalign{}
Accent & Name & Codepoint \\
\midrule\noalign{}
\endhead
\bottomrule\noalign{}
\endlastfoot
Grave & \texttt{\ grave\ } & \texttt{\ \textasciigrave{}\ } \\
Acute & \texttt{\ acute\ } & \texttt{\ ´\ } \\
Circumflex & \texttt{\ hat\ } & \texttt{\ \^{}\ } \\
Tilde & \texttt{\ tilde\ } & \texttt{\ \textasciitilde{}\ } \\
Macron & \texttt{\ macron\ } & \texttt{\ ¯\ } \\
Dash & \texttt{\ dash\ } & \texttt{\ ‾\ } \\
Breve & \texttt{\ breve\ } & \texttt{\ ˘\ } \\
Dot & \texttt{\ dot\ } & \texttt{\ .\ } \\
Double dot, Diaeresis & \texttt{\ dot.double\ } , \texttt{\ diaer\ } &
\texttt{\ ¨\ } \\
Triple dot & \texttt{\ dot.triple\ } & \texttt{\ ⃛\ } \\
Quadruple dot & \texttt{\ dot.quad\ } & \texttt{\ ⃜\ } \\
Circle & \texttt{\ circle\ } & \texttt{\ ∘\ } \\
Double acute & \texttt{\ acute.double\ } & \texttt{\ Ë?\ } \\
Caron & \texttt{\ caron\ } & \texttt{\ ˇ\ } \\
Right arrow & \texttt{\ arrow\ } , \texttt{\ -\textgreater{}\ } &
\texttt{\ →\ } \\
Left arrow & \texttt{\ arrow.l\ } , \texttt{\ \textless{}-\ } &
\texttt{\ �\ } \\
Left/Right arrow & \texttt{\ arrow.l.r\ } & \texttt{\ ↔\ } \\
Right harpoon & \texttt{\ harpoon\ } & \texttt{\ ⇀\ } \\
Left harpoon & \texttt{\ harpoon.lt\ } & \texttt{\ ↼\ } \\
\end{longtable}

\subsubsection{\texorpdfstring{\texttt{\ size\ }}{ size }}\label{parameters-size}

\href{/docs/reference/foundations/auto/}{auto} {or}
\href{/docs/reference/layout/relative/}{relative}

{{ Settable }}

\phantomsection\label{parameters-size-settable-tooltip}
Settable parameters can be customized for all following uses of the
function with a \texttt{\ set\ } rule.

The size of the accent, relative to the width of the base.

Default: \texttt{\ }{\texttt{\ auto\ }}\texttt{\ }

\href{/docs/reference/math/}{\pandocbounded{\includesvg[keepaspectratio]{/assets/icons/16-arrow-right.svg}}}

{ Math } { Previous page }

\href{/docs/reference/math/attach/}{\pandocbounded{\includesvg[keepaspectratio]{/assets/icons/16-arrow-right.svg}}}

{ Attach } { Next page }


\title{typst.app/docs/reference/math/stretch}

\begin{itemize}
\tightlist
\item
  \href{/docs}{\includesvg[width=0.16667in,height=0.16667in]{/assets/icons/16-docs-dark.svg}}
\item
  \includesvg[width=0.16667in,height=0.16667in]{/assets/icons/16-arrow-right.svg}
\item
  \href{/docs/reference/}{Reference}
\item
  \includesvg[width=0.16667in,height=0.16667in]{/assets/icons/16-arrow-right.svg}
\item
  \href{/docs/reference/math/}{Math}
\item
  \includesvg[width=0.16667in,height=0.16667in]{/assets/icons/16-arrow-right.svg}
\item
  \href{/docs/reference/math/stretch/}{Stretch}
\end{itemize}

\section{\texorpdfstring{\texttt{\ stretch\ } {{ Element
}}}{ stretch   Element }}\label{summary}

\phantomsection\label{element-tooltip}
Element functions can be customized with \texttt{\ set\ } and
\texttt{\ show\ } rules.

Stretches a glyph.

This function can also be used to automatically stretch the base of an
attachment, so that it fits the top and bottom attachments.

Note that only some glyphs can be stretched, and which ones can depend
on the math font being used. However, most math fonts are the same in
this regard.

\begin{verbatim}
$ H stretch(=)^"define" U + p V $
$ f : X stretch(->>, size: #150%)_"surjective" Y $
$ x stretch(harpoons.ltrb, size: #3em) y
    stretch(\[, size: #150%) z $
\end{verbatim}

\includegraphics[width=5in,height=\textheight,keepaspectratio]{/assets/docs/s6743QhH3-etZ1y_QW-bLAAAAAAAAAAA.png}

\subsection{\texorpdfstring{{ Parameters
}}{ Parameters }}\label{parameters}

\phantomsection\label{parameters-tooltip}
Parameters are the inputs to a function. They are specified in
parentheses after the function name.

math { . } { stretch } (

{ \href{/docs/reference/foundations/content/}{content} , } {
\hyperref[parameters-size]{size :}
\href{/docs/reference/foundations/auto/}{auto}
\href{/docs/reference/layout/relative/}{relative} , }

) -\textgreater{} \href{/docs/reference/foundations/content/}{content}

\subsubsection{\texorpdfstring{\texttt{\ body\ }}{ body }}\label{parameters-body}

\href{/docs/reference/foundations/content/}{content}

{Required} {{ Positional }}

\phantomsection\label{parameters-body-positional-tooltip}
Positional parameters are specified in order, without names.

The glyph to stretch.

\subsubsection{\texorpdfstring{\texttt{\ size\ }}{ size }}\label{parameters-size}

\href{/docs/reference/foundations/auto/}{auto} {or}
\href{/docs/reference/layout/relative/}{relative}

{{ Settable }}

\phantomsection\label{parameters-size-settable-tooltip}
Settable parameters can be customized for all following uses of the
function with a \texttt{\ set\ } rule.

The size to stretch to, relative to the maximum size of the glyph and
its attachments.

Default: \texttt{\ }{\texttt{\ auto\ }}\texttt{\ }

\href{/docs/reference/math/sizes/}{\pandocbounded{\includesvg[keepaspectratio]{/assets/icons/16-arrow-right.svg}}}

{ Sizes } { Previous page }

\href{/docs/reference/math/styles/}{\pandocbounded{\includesvg[keepaspectratio]{/assets/icons/16-arrow-right.svg}}}

{ Styles } { Next page }


\title{typst.app/docs/reference/math/attach}

\begin{itemize}
\tightlist
\item
  \href{/docs}{\includesvg[width=0.16667in,height=0.16667in]{/assets/icons/16-docs-dark.svg}}
\item
  \includesvg[width=0.16667in,height=0.16667in]{/assets/icons/16-arrow-right.svg}
\item
  \href{/docs/reference/}{Reference}
\item
  \includesvg[width=0.16667in,height=0.16667in]{/assets/icons/16-arrow-right.svg}
\item
  \href{/docs/reference/math/}{Math}
\item
  \includesvg[width=0.16667in,height=0.16667in]{/assets/icons/16-arrow-right.svg}
\item
  \href{/docs/reference/math/attach}{Attach}
\end{itemize}

\section{Attach}\label{summary}

Subscript, superscripts, and limits.

Attachments can be displayed either as sub/superscripts, or limits.
Typst automatically decides which is more suitable depending on the
base, but you can also control this manually with the
\texttt{\ scripts\ } and \texttt{\ limits\ } functions.

If you want the base to stretch to fit long top and bottom attachments
(for example, an arrow with text above it), use the
\href{/docs/reference/math/stretch/}{\texttt{\ stretch\ }} function.

\subsection{Example}\label{example}

\begin{verbatim}
$ sum_(i=0)^n a_i = 2^(1+i) $
\end{verbatim}

\includegraphics[width=5in,height=\textheight,keepaspectratio]{/assets/docs/QRQ31w2n3rdGvD8KZ-ysUQAAAAAAAAAA.png}

\subsection{Syntax}\label{syntax}

This function also has dedicated syntax for attachments after the base:
Use the underscore ( \texttt{\ \_\ } ) to indicate a subscript i.e.
bottom attachment and the hat ( \texttt{\ \^{}\ } ) to indicate a
superscript i.e. top attachment.

\subsection{Functions}\label{functions}

\subsubsection{\texorpdfstring{\texttt{\ attach\ } {{ Element
}}}{ attach   Element }}\label{functions-attach}

\phantomsection\label{functions-attach-element-tooltip}
Element functions can be customized with \texttt{\ set\ } and
\texttt{\ show\ } rules.

A base with optional attachments.

math { . } { attach } (

{ \href{/docs/reference/foundations/content/}{content} , } {
\hyperref[functions-attach-parameters-t]{t :}
\href{/docs/reference/foundations/none/}{none}
\href{/docs/reference/foundations/content/}{content} , } {
\hyperref[functions-attach-parameters-b]{b :}
\href{/docs/reference/foundations/none/}{none}
\href{/docs/reference/foundations/content/}{content} , } {
\hyperref[functions-attach-parameters-tl]{tl :}
\href{/docs/reference/foundations/none/}{none}
\href{/docs/reference/foundations/content/}{content} , } {
\hyperref[functions-attach-parameters-bl]{bl :}
\href{/docs/reference/foundations/none/}{none}
\href{/docs/reference/foundations/content/}{content} , } {
\hyperref[functions-attach-parameters-tr]{tr :}
\href{/docs/reference/foundations/none/}{none}
\href{/docs/reference/foundations/content/}{content} , } {
\hyperref[functions-attach-parameters-br]{br :}
\href{/docs/reference/foundations/none/}{none}
\href{/docs/reference/foundations/content/}{content} , }

) -\textgreater{} \href{/docs/reference/foundations/content/}{content}

\begin{verbatim}
$ attach(
  Pi, t: alpha, b: beta,
  tl: 1, tr: 2+3, bl: 4+5, br: 6,
) $
\end{verbatim}

\includegraphics[width=5in,height=\textheight,keepaspectratio]{/assets/docs/hP1S-FGMSbXQwhVNN2LoxQAAAAAAAAAA.png}

\paragraph{\texorpdfstring{\texttt{\ base\ }}{ base }}\label{functions-attach-base}

\href{/docs/reference/foundations/content/}{content}

{Required} {{ Positional }}

\phantomsection\label{functions-attach-base-positional-tooltip}
Positional parameters are specified in order, without names.

The base to which things are attached.

\paragraph{\texorpdfstring{\texttt{\ t\ }}{ t }}\label{functions-attach-t}

\href{/docs/reference/foundations/none/}{none} {or}
\href{/docs/reference/foundations/content/}{content}

{{ Settable }}

\phantomsection\label{functions-attach-t-settable-tooltip}
Settable parameters can be customized for all following uses of the
function with a \texttt{\ set\ } rule.

The top attachment, smartly positioned at top-right or above the base.

You can wrap the base in
\texttt{\ }{\texttt{\ limits\ }}\texttt{\ }{\texttt{\ (\ }}\texttt{\ }{\texttt{\ )\ }}\texttt{\ }
or
\texttt{\ }{\texttt{\ scripts\ }}\texttt{\ }{\texttt{\ (\ }}\texttt{\ }{\texttt{\ )\ }}\texttt{\ }
to override the smart positioning.

Default: \texttt{\ }{\texttt{\ none\ }}\texttt{\ }

\paragraph{\texorpdfstring{\texttt{\ b\ }}{ b }}\label{functions-attach-b}

\href{/docs/reference/foundations/none/}{none} {or}
\href{/docs/reference/foundations/content/}{content}

{{ Settable }}

\phantomsection\label{functions-attach-b-settable-tooltip}
Settable parameters can be customized for all following uses of the
function with a \texttt{\ set\ } rule.

The bottom attachment, smartly positioned at the bottom-right or below
the base.

You can wrap the base in
\texttt{\ }{\texttt{\ limits\ }}\texttt{\ }{\texttt{\ (\ }}\texttt{\ }{\texttt{\ )\ }}\texttt{\ }
or
\texttt{\ }{\texttt{\ scripts\ }}\texttt{\ }{\texttt{\ (\ }}\texttt{\ }{\texttt{\ )\ }}\texttt{\ }
to override the smart positioning.

Default: \texttt{\ }{\texttt{\ none\ }}\texttt{\ }

\paragraph{\texorpdfstring{\texttt{\ tl\ }}{ tl }}\label{functions-attach-tl}

\href{/docs/reference/foundations/none/}{none} {or}
\href{/docs/reference/foundations/content/}{content}

{{ Settable }}

\phantomsection\label{functions-attach-tl-settable-tooltip}
Settable parameters can be customized for all following uses of the
function with a \texttt{\ set\ } rule.

The top-left attachment (before the base).

Default: \texttt{\ }{\texttt{\ none\ }}\texttt{\ }

\paragraph{\texorpdfstring{\texttt{\ bl\ }}{ bl }}\label{functions-attach-bl}

\href{/docs/reference/foundations/none/}{none} {or}
\href{/docs/reference/foundations/content/}{content}

{{ Settable }}

\phantomsection\label{functions-attach-bl-settable-tooltip}
Settable parameters can be customized for all following uses of the
function with a \texttt{\ set\ } rule.

The bottom-left attachment (before base).

Default: \texttt{\ }{\texttt{\ none\ }}\texttt{\ }

\paragraph{\texorpdfstring{\texttt{\ tr\ }}{ tr }}\label{functions-attach-tr}

\href{/docs/reference/foundations/none/}{none} {or}
\href{/docs/reference/foundations/content/}{content}

{{ Settable }}

\phantomsection\label{functions-attach-tr-settable-tooltip}
Settable parameters can be customized for all following uses of the
function with a \texttt{\ set\ } rule.

The top-right attachment (after the base).

Default: \texttt{\ }{\texttt{\ none\ }}\texttt{\ }

\paragraph{\texorpdfstring{\texttt{\ br\ }}{ br }}\label{functions-attach-br}

\href{/docs/reference/foundations/none/}{none} {or}
\href{/docs/reference/foundations/content/}{content}

{{ Settable }}

\phantomsection\label{functions-attach-br-settable-tooltip}
Settable parameters can be customized for all following uses of the
function with a \texttt{\ set\ } rule.

The bottom-right attachment (after the base).

Default: \texttt{\ }{\texttt{\ none\ }}\texttt{\ }

\subsubsection{\texorpdfstring{\texttt{\ scripts\ } {{ Element
}}}{ scripts   Element }}\label{functions-scripts}

\phantomsection\label{functions-scripts-element-tooltip}
Element functions can be customized with \texttt{\ set\ } and
\texttt{\ show\ } rules.

Forces a base to display attachments as scripts.

math { . } { scripts } (

{ \href{/docs/reference/foundations/content/}{content} }

) -\textgreater{} \href{/docs/reference/foundations/content/}{content}

\begin{verbatim}
$ scripts(sum)_1^2 != sum_1^2 $
\end{verbatim}

\includegraphics[width=5in,height=\textheight,keepaspectratio]{/assets/docs/yVmcJ82GwTKFuNMU4shSjAAAAAAAAAAA.png}

\paragraph{\texorpdfstring{\texttt{\ body\ }}{ body }}\label{functions-scripts-body}

\href{/docs/reference/foundations/content/}{content}

{Required} {{ Positional }}

\phantomsection\label{functions-scripts-body-positional-tooltip}
Positional parameters are specified in order, without names.

The base to attach the scripts to.

\subsubsection{\texorpdfstring{\texttt{\ limits\ } {{ Element
}}}{ limits   Element }}\label{functions-limits}

\phantomsection\label{functions-limits-element-tooltip}
Element functions can be customized with \texttt{\ set\ } and
\texttt{\ show\ } rules.

Forces a base to display attachments as limits.

math { . } { limits } (

{ \href{/docs/reference/foundations/content/}{content} , } {
\hyperref[functions-limits-parameters-inline]{inline :}
\href{/docs/reference/foundations/bool/}{bool} , }

) -\textgreater{} \href{/docs/reference/foundations/content/}{content}

\begin{verbatim}
$ limits(A)_1^2 != A_1^2 $
\end{verbatim}

\includegraphics[width=5in,height=\textheight,keepaspectratio]{/assets/docs/_7kc3fTt948a-U1_9wdyzgAAAAAAAAAA.png}

\paragraph{\texorpdfstring{\texttt{\ body\ }}{ body }}\label{functions-limits-body}

\href{/docs/reference/foundations/content/}{content}

{Required} {{ Positional }}

\phantomsection\label{functions-limits-body-positional-tooltip}
Positional parameters are specified in order, without names.

The base to attach the limits to.

\paragraph{\texorpdfstring{\texttt{\ inline\ }}{ inline }}\label{functions-limits-inline}

\href{/docs/reference/foundations/bool/}{bool}

{{ Settable }}

\phantomsection\label{functions-limits-inline-settable-tooltip}
Settable parameters can be customized for all following uses of the
function with a \texttt{\ set\ } rule.

Whether to also force limits in inline equations.

When applying limits globally (e.g., through a show rule), it is
typically a good idea to disable this.

Default: \texttt{\ }{\texttt{\ true\ }}\texttt{\ }

\href{/docs/reference/math/accent/}{\pandocbounded{\includesvg[keepaspectratio]{/assets/icons/16-arrow-right.svg}}}

{ Accent } { Previous page }

\href{/docs/reference/math/binom/}{\pandocbounded{\includesvg[keepaspectratio]{/assets/icons/16-arrow-right.svg}}}

{ Binomial } { Next page }


\title{typst.app/docs/reference/math/underover}

\begin{itemize}
\tightlist
\item
  \href{/docs}{\includesvg[width=0.16667in,height=0.16667in]{/assets/icons/16-docs-dark.svg}}
\item
  \includesvg[width=0.16667in,height=0.16667in]{/assets/icons/16-arrow-right.svg}
\item
  \href{/docs/reference/}{Reference}
\item
  \includesvg[width=0.16667in,height=0.16667in]{/assets/icons/16-arrow-right.svg}
\item
  \href{/docs/reference/math/}{Math}
\item
  \includesvg[width=0.16667in,height=0.16667in]{/assets/icons/16-arrow-right.svg}
\item
  \href{/docs/reference/math/underover}{Under/Over}
\end{itemize}

\section{Under/Over}\label{summary}

Delimiters above or below parts of an equation.

The braces and brackets further allow you to add an optional annotation
below or above themselves.

\subsection{Functions}\label{functions}

\subsubsection{\texorpdfstring{\texttt{\ underline\ } {{ Element
}}}{ underline   Element }}\label{functions-underline}

\phantomsection\label{functions-underline-element-tooltip}
Element functions can be customized with \texttt{\ set\ } and
\texttt{\ show\ } rules.

A horizontal line under content.

math { . } { underline } (

{ \href{/docs/reference/foundations/content/}{content} }

) -\textgreater{} \href{/docs/reference/foundations/content/}{content}

\begin{verbatim}
$ underline(1 + 2 + ... + 5) $
\end{verbatim}

\includegraphics[width=5in,height=\textheight,keepaspectratio]{/assets/docs/kPv2rkuOYqE5xrS9gynyqwAAAAAAAAAA.png}

\paragraph{\texorpdfstring{\texttt{\ body\ }}{ body }}\label{functions-underline-body}

\href{/docs/reference/foundations/content/}{content}

{Required} {{ Positional }}

\phantomsection\label{functions-underline-body-positional-tooltip}
Positional parameters are specified in order, without names.

The content above the line.

\subsubsection{\texorpdfstring{\texttt{\ overline\ } {{ Element
}}}{ overline   Element }}\label{functions-overline}

\phantomsection\label{functions-overline-element-tooltip}
Element functions can be customized with \texttt{\ set\ } and
\texttt{\ show\ } rules.

A horizontal line over content.

math { . } { overline } (

{ \href{/docs/reference/foundations/content/}{content} }

) -\textgreater{} \href{/docs/reference/foundations/content/}{content}

\begin{verbatim}
$ overline(1 + 2 + ... + 5) $
\end{verbatim}

\includegraphics[width=5in,height=\textheight,keepaspectratio]{/assets/docs/brbtze6pYcbdDHZXqYtX4QAAAAAAAAAA.png}

\paragraph{\texorpdfstring{\texttt{\ body\ }}{ body }}\label{functions-overline-body}

\href{/docs/reference/foundations/content/}{content}

{Required} {{ Positional }}

\phantomsection\label{functions-overline-body-positional-tooltip}
Positional parameters are specified in order, without names.

The content below the line.

\subsubsection{\texorpdfstring{\texttt{\ underbrace\ } {{ Element
}}}{ underbrace   Element }}\label{functions-underbrace}

\phantomsection\label{functions-underbrace-element-tooltip}
Element functions can be customized with \texttt{\ set\ } and
\texttt{\ show\ } rules.

A horizontal brace under content, with an optional annotation below.

math { . } { underbrace } (

{ \href{/docs/reference/foundations/content/}{content} , } {
\hyperref[functions-underbrace-parameters-annotation]{}
\href{/docs/reference/foundations/none/}{none}
\href{/docs/reference/foundations/content/}{content} , }

) -\textgreater{} \href{/docs/reference/foundations/content/}{content}

\begin{verbatim}
$ underbrace(1 + 2 + ... + 5, "numbers") $
\end{verbatim}

\includegraphics[width=5in,height=\textheight,keepaspectratio]{/assets/docs/CQPrguDXpL2KqqF50rooNAAAAAAAAAAA.png}

\paragraph{\texorpdfstring{\texttt{\ body\ }}{ body }}\label{functions-underbrace-body}

\href{/docs/reference/foundations/content/}{content}

{Required} {{ Positional }}

\phantomsection\label{functions-underbrace-body-positional-tooltip}
Positional parameters are specified in order, without names.

The content above the brace.

\paragraph{\texorpdfstring{\texttt{\ annotation\ }}{ annotation }}\label{functions-underbrace-annotation}

\href{/docs/reference/foundations/none/}{none} {or}
\href{/docs/reference/foundations/content/}{content}

{{ Positional }}

\phantomsection\label{functions-underbrace-annotation-positional-tooltip}
Positional parameters are specified in order, without names.

{{ Settable }}

\phantomsection\label{functions-underbrace-annotation-settable-tooltip}
Settable parameters can be customized for all following uses of the
function with a \texttt{\ set\ } rule.

The optional content below the brace.

Default: \texttt{\ }{\texttt{\ none\ }}\texttt{\ }

\subsubsection{\texorpdfstring{\texttt{\ overbrace\ } {{ Element
}}}{ overbrace   Element }}\label{functions-overbrace}

\phantomsection\label{functions-overbrace-element-tooltip}
Element functions can be customized with \texttt{\ set\ } and
\texttt{\ show\ } rules.

A horizontal brace over content, with an optional annotation above.

math { . } { overbrace } (

{ \href{/docs/reference/foundations/content/}{content} , } {
\hyperref[functions-overbrace-parameters-annotation]{}
\href{/docs/reference/foundations/none/}{none}
\href{/docs/reference/foundations/content/}{content} , }

) -\textgreater{} \href{/docs/reference/foundations/content/}{content}

\begin{verbatim}
$ overbrace(1 + 2 + ... + 5, "numbers") $
\end{verbatim}

\includegraphics[width=5in,height=\textheight,keepaspectratio]{/assets/docs/kkBGSVxyTk5_L1k_EG8I3gAAAAAAAAAA.png}

\paragraph{\texorpdfstring{\texttt{\ body\ }}{ body }}\label{functions-overbrace-body}

\href{/docs/reference/foundations/content/}{content}

{Required} {{ Positional }}

\phantomsection\label{functions-overbrace-body-positional-tooltip}
Positional parameters are specified in order, without names.

The content below the brace.

\paragraph{\texorpdfstring{\texttt{\ annotation\ }}{ annotation }}\label{functions-overbrace-annotation}

\href{/docs/reference/foundations/none/}{none} {or}
\href{/docs/reference/foundations/content/}{content}

{{ Positional }}

\phantomsection\label{functions-overbrace-annotation-positional-tooltip}
Positional parameters are specified in order, without names.

{{ Settable }}

\phantomsection\label{functions-overbrace-annotation-settable-tooltip}
Settable parameters can be customized for all following uses of the
function with a \texttt{\ set\ } rule.

The optional content above the brace.

Default: \texttt{\ }{\texttt{\ none\ }}\texttt{\ }

\subsubsection{\texorpdfstring{\texttt{\ underbracket\ } {{ Element
}}}{ underbracket   Element }}\label{functions-underbracket}

\phantomsection\label{functions-underbracket-element-tooltip}
Element functions can be customized with \texttt{\ set\ } and
\texttt{\ show\ } rules.

A horizontal bracket under content, with an optional annotation below.

math { . } { underbracket } (

{ \href{/docs/reference/foundations/content/}{content} , } {
\hyperref[functions-underbracket-parameters-annotation]{}
\href{/docs/reference/foundations/none/}{none}
\href{/docs/reference/foundations/content/}{content} , }

) -\textgreater{} \href{/docs/reference/foundations/content/}{content}

\begin{verbatim}
$ underbracket(1 + 2 + ... + 5, "numbers") $
\end{verbatim}

\includegraphics[width=5in,height=\textheight,keepaspectratio]{/assets/docs/gOJp15FKm4cOEOHbW4H-OwAAAAAAAAAA.png}

\paragraph{\texorpdfstring{\texttt{\ body\ }}{ body }}\label{functions-underbracket-body}

\href{/docs/reference/foundations/content/}{content}

{Required} {{ Positional }}

\phantomsection\label{functions-underbracket-body-positional-tooltip}
Positional parameters are specified in order, without names.

The content above the bracket.

\paragraph{\texorpdfstring{\texttt{\ annotation\ }}{ annotation }}\label{functions-underbracket-annotation}

\href{/docs/reference/foundations/none/}{none} {or}
\href{/docs/reference/foundations/content/}{content}

{{ Positional }}

\phantomsection\label{functions-underbracket-annotation-positional-tooltip}
Positional parameters are specified in order, without names.

{{ Settable }}

\phantomsection\label{functions-underbracket-annotation-settable-tooltip}
Settable parameters can be customized for all following uses of the
function with a \texttt{\ set\ } rule.

The optional content below the bracket.

Default: \texttt{\ }{\texttt{\ none\ }}\texttt{\ }

\subsubsection{\texorpdfstring{\texttt{\ overbracket\ } {{ Element
}}}{ overbracket   Element }}\label{functions-overbracket}

\phantomsection\label{functions-overbracket-element-tooltip}
Element functions can be customized with \texttt{\ set\ } and
\texttt{\ show\ } rules.

A horizontal bracket over content, with an optional annotation above.

math { . } { overbracket } (

{ \href{/docs/reference/foundations/content/}{content} , } {
\hyperref[functions-overbracket-parameters-annotation]{}
\href{/docs/reference/foundations/none/}{none}
\href{/docs/reference/foundations/content/}{content} , }

) -\textgreater{} \href{/docs/reference/foundations/content/}{content}

\begin{verbatim}
$ overbracket(1 + 2 + ... + 5, "numbers") $
\end{verbatim}

\includegraphics[width=5in,height=\textheight,keepaspectratio]{/assets/docs/1FDacJmC0p-s0HOdRor7WgAAAAAAAAAA.png}

\paragraph{\texorpdfstring{\texttt{\ body\ }}{ body }}\label{functions-overbracket-body}

\href{/docs/reference/foundations/content/}{content}

{Required} {{ Positional }}

\phantomsection\label{functions-overbracket-body-positional-tooltip}
Positional parameters are specified in order, without names.

The content below the bracket.

\paragraph{\texorpdfstring{\texttt{\ annotation\ }}{ annotation }}\label{functions-overbracket-annotation}

\href{/docs/reference/foundations/none/}{none} {or}
\href{/docs/reference/foundations/content/}{content}

{{ Positional }}

\phantomsection\label{functions-overbracket-annotation-positional-tooltip}
Positional parameters are specified in order, without names.

{{ Settable }}

\phantomsection\label{functions-overbracket-annotation-settable-tooltip}
Settable parameters can be customized for all following uses of the
function with a \texttt{\ set\ } rule.

The optional content above the bracket.

Default: \texttt{\ }{\texttt{\ none\ }}\texttt{\ }

\subsubsection{\texorpdfstring{\texttt{\ underparen\ } {{ Element
}}}{ underparen   Element }}\label{functions-underparen}

\phantomsection\label{functions-underparen-element-tooltip}
Element functions can be customized with \texttt{\ set\ } and
\texttt{\ show\ } rules.

A horizontal parenthesis under content, with an optional annotation
below.

math { . } { underparen } (

{ \href{/docs/reference/foundations/content/}{content} , } {
\hyperref[functions-underparen-parameters-annotation]{}
\href{/docs/reference/foundations/none/}{none}
\href{/docs/reference/foundations/content/}{content} , }

) -\textgreater{} \href{/docs/reference/foundations/content/}{content}

\begin{verbatim}
$ underparen(1 + 2 + ... + 5, "numbers") $
\end{verbatim}

\includegraphics[width=5in,height=\textheight,keepaspectratio]{/assets/docs/L9b8yvtULgB5qqliYqLlbAAAAAAAAAAA.png}

\paragraph{\texorpdfstring{\texttt{\ body\ }}{ body }}\label{functions-underparen-body}

\href{/docs/reference/foundations/content/}{content}

{Required} {{ Positional }}

\phantomsection\label{functions-underparen-body-positional-tooltip}
Positional parameters are specified in order, without names.

The content above the parenthesis.

\paragraph{\texorpdfstring{\texttt{\ annotation\ }}{ annotation }}\label{functions-underparen-annotation}

\href{/docs/reference/foundations/none/}{none} {or}
\href{/docs/reference/foundations/content/}{content}

{{ Positional }}

\phantomsection\label{functions-underparen-annotation-positional-tooltip}
Positional parameters are specified in order, without names.

{{ Settable }}

\phantomsection\label{functions-underparen-annotation-settable-tooltip}
Settable parameters can be customized for all following uses of the
function with a \texttt{\ set\ } rule.

The optional content below the parenthesis.

Default: \texttt{\ }{\texttt{\ none\ }}\texttt{\ }

\subsubsection{\texorpdfstring{\texttt{\ overparen\ } {{ Element
}}}{ overparen   Element }}\label{functions-overparen}

\phantomsection\label{functions-overparen-element-tooltip}
Element functions can be customized with \texttt{\ set\ } and
\texttt{\ show\ } rules.

A horizontal parenthesis over content, with an optional annotation
above.

math { . } { overparen } (

{ \href{/docs/reference/foundations/content/}{content} , } {
\hyperref[functions-overparen-parameters-annotation]{}
\href{/docs/reference/foundations/none/}{none}
\href{/docs/reference/foundations/content/}{content} , }

) -\textgreater{} \href{/docs/reference/foundations/content/}{content}

\begin{verbatim}
$ overparen(1 + 2 + ... + 5, "numbers") $
\end{verbatim}

\includegraphics[width=5in,height=\textheight,keepaspectratio]{/assets/docs/0O_PdeP9aD4IdbiAFPLHcwAAAAAAAAAA.png}

\paragraph{\texorpdfstring{\texttt{\ body\ }}{ body }}\label{functions-overparen-body}

\href{/docs/reference/foundations/content/}{content}

{Required} {{ Positional }}

\phantomsection\label{functions-overparen-body-positional-tooltip}
Positional parameters are specified in order, without names.

The content below the parenthesis.

\paragraph{\texorpdfstring{\texttt{\ annotation\ }}{ annotation }}\label{functions-overparen-annotation}

\href{/docs/reference/foundations/none/}{none} {or}
\href{/docs/reference/foundations/content/}{content}

{{ Positional }}

\phantomsection\label{functions-overparen-annotation-positional-tooltip}
Positional parameters are specified in order, without names.

{{ Settable }}

\phantomsection\label{functions-overparen-annotation-settable-tooltip}
Settable parameters can be customized for all following uses of the
function with a \texttt{\ set\ } rule.

The optional content above the parenthesis.

Default: \texttt{\ }{\texttt{\ none\ }}\texttt{\ }

\subsubsection{\texorpdfstring{\texttt{\ undershell\ } {{ Element
}}}{ undershell   Element }}\label{functions-undershell}

\phantomsection\label{functions-undershell-element-tooltip}
Element functions can be customized with \texttt{\ set\ } and
\texttt{\ show\ } rules.

A horizontal tortoise shell bracket under content, with an optional
annotation below.

math { . } { undershell } (

{ \href{/docs/reference/foundations/content/}{content} , } {
\hyperref[functions-undershell-parameters-annotation]{}
\href{/docs/reference/foundations/none/}{none}
\href{/docs/reference/foundations/content/}{content} , }

) -\textgreater{} \href{/docs/reference/foundations/content/}{content}

\begin{verbatim}
$ undershell(1 + 2 + ... + 5, "numbers") $
\end{verbatim}

\includegraphics[width=5in,height=\textheight,keepaspectratio]{/assets/docs/qJR4zaGYtEbCSgwj0kBzVgAAAAAAAAAA.png}

\paragraph{\texorpdfstring{\texttt{\ body\ }}{ body }}\label{functions-undershell-body}

\href{/docs/reference/foundations/content/}{content}

{Required} {{ Positional }}

\phantomsection\label{functions-undershell-body-positional-tooltip}
Positional parameters are specified in order, without names.

The content above the tortoise shell bracket.

\paragraph{\texorpdfstring{\texttt{\ annotation\ }}{ annotation }}\label{functions-undershell-annotation}

\href{/docs/reference/foundations/none/}{none} {or}
\href{/docs/reference/foundations/content/}{content}

{{ Positional }}

\phantomsection\label{functions-undershell-annotation-positional-tooltip}
Positional parameters are specified in order, without names.

{{ Settable }}

\phantomsection\label{functions-undershell-annotation-settable-tooltip}
Settable parameters can be customized for all following uses of the
function with a \texttt{\ set\ } rule.

The optional content below the tortoise shell bracket.

Default: \texttt{\ }{\texttt{\ none\ }}\texttt{\ }

\subsubsection{\texorpdfstring{\texttt{\ overshell\ } {{ Element
}}}{ overshell   Element }}\label{functions-overshell}

\phantomsection\label{functions-overshell-element-tooltip}
Element functions can be customized with \texttt{\ set\ } and
\texttt{\ show\ } rules.

A horizontal tortoise shell bracket over content, with an optional
annotation above.

math { . } { overshell } (

{ \href{/docs/reference/foundations/content/}{content} , } {
\hyperref[functions-overshell-parameters-annotation]{}
\href{/docs/reference/foundations/none/}{none}
\href{/docs/reference/foundations/content/}{content} , }

) -\textgreater{} \href{/docs/reference/foundations/content/}{content}

\begin{verbatim}
$ overshell(1 + 2 + ... + 5, "numbers") $
\end{verbatim}

\includegraphics[width=5in,height=\textheight,keepaspectratio]{/assets/docs/vPA0v0E_JXwsC1BpaClcEgAAAAAAAAAA.png}

\paragraph{\texorpdfstring{\texttt{\ body\ }}{ body }}\label{functions-overshell-body}

\href{/docs/reference/foundations/content/}{content}

{Required} {{ Positional }}

\phantomsection\label{functions-overshell-body-positional-tooltip}
Positional parameters are specified in order, without names.

The content below the tortoise shell bracket.

\paragraph{\texorpdfstring{\texttt{\ annotation\ }}{ annotation }}\label{functions-overshell-annotation}

\href{/docs/reference/foundations/none/}{none} {or}
\href{/docs/reference/foundations/content/}{content}

{{ Positional }}

\phantomsection\label{functions-overshell-annotation-positional-tooltip}
Positional parameters are specified in order, without names.

{{ Settable }}

\phantomsection\label{functions-overshell-annotation-settable-tooltip}
Settable parameters can be customized for all following uses of the
function with a \texttt{\ set\ } rule.

The optional content above the tortoise shell bracket.

Default: \texttt{\ }{\texttt{\ none\ }}\texttt{\ }

\href{/docs/reference/math/op/}{\pandocbounded{\includesvg[keepaspectratio]{/assets/icons/16-arrow-right.svg}}}

{ Text Operator } { Previous page }

\href{/docs/reference/math/variants/}{\pandocbounded{\includesvg[keepaspectratio]{/assets/icons/16-arrow-right.svg}}}

{ Variants } { Next page }


\title{typst.app/docs/reference/math/equation}

\begin{itemize}
\tightlist
\item
  \href{/docs}{\includesvg[width=0.16667in,height=0.16667in]{/assets/icons/16-docs-dark.svg}}
\item
  \includesvg[width=0.16667in,height=0.16667in]{/assets/icons/16-arrow-right.svg}
\item
  \href{/docs/reference/}{Reference}
\item
  \includesvg[width=0.16667in,height=0.16667in]{/assets/icons/16-arrow-right.svg}
\item
  \href{/docs/reference/math/}{Math}
\item
  \includesvg[width=0.16667in,height=0.16667in]{/assets/icons/16-arrow-right.svg}
\item
  \href{/docs/reference/math/equation/}{Equation}
\end{itemize}

\section{\texorpdfstring{\texttt{\ equation\ } {{ Element
}}}{ equation   Element }}\label{summary}

\phantomsection\label{element-tooltip}
Element functions can be customized with \texttt{\ set\ } and
\texttt{\ show\ } rules.

A mathematical equation.

Can be displayed inline with text or as a separate block.

\subsection{Example}\label{example}

\begin{verbatim}
#set text(font: "New Computer Modern")

Let $a$, $b$, and $c$ be the side
lengths of right-angled triangle.
Then, we know that:
$ a^2 + b^2 = c^2 $

Prove by induction:
$ sum_(k=1)^n k = (n(n+1)) / 2 $
\end{verbatim}

\includegraphics[width=5in,height=\textheight,keepaspectratio]{/assets/docs/JtxOgQArvspfmmStl8-3_gAAAAAAAAAA.png}

By default, block-level equations will not break across pages. This can
be changed through
\texttt{\ }{\texttt{\ show\ }}\texttt{\ math\ }{\texttt{\ .\ }}\texttt{\ }{\texttt{\ equation\ }}\texttt{\ }{\texttt{\ :\ }}\texttt{\ }{\texttt{\ set\ }}\texttt{\ }{\texttt{\ block\ }}\texttt{\ }{\texttt{\ (\ }}\texttt{\ breakable\ }{\texttt{\ :\ }}\texttt{\ }{\texttt{\ true\ }}\texttt{\ }{\texttt{\ )\ }}\texttt{\ }
.

\subsection{Syntax}\label{syntax}

This function also has dedicated syntax: Write mathematical markup
within dollar signs to create an equation. Starting and ending the
equation with at least one space lifts it into a separate block that is
centered horizontally. For more details about math syntax, see the
\href{/docs/reference/math/}{main math page} .

\subsection{\texorpdfstring{{ Parameters
}}{ Parameters }}\label{parameters}

\phantomsection\label{parameters-tooltip}
Parameters are the inputs to a function. They are specified in
parentheses after the function name.

math { . } { equation } (

{ \hyperref[parameters-block]{block :}
\href{/docs/reference/foundations/bool/}{bool} , } {
\hyperref[parameters-numbering]{numbering :}
\href{/docs/reference/foundations/none/}{none}
\href{/docs/reference/foundations/str/}{str}
\href{/docs/reference/foundations/function/}{function} , } {
\hyperref[parameters-number-align]{number-align :}
\href{/docs/reference/layout/alignment/}{alignment} , } {
\hyperref[parameters-supplement]{supplement :}
\href{/docs/reference/foundations/none/}{none}
\href{/docs/reference/foundations/auto/}{auto}
\href{/docs/reference/foundations/content/}{content}
\href{/docs/reference/foundations/function/}{function} , } {
\href{/docs/reference/foundations/content/}{content} , }

) -\textgreater{} \href{/docs/reference/foundations/content/}{content}

\subsubsection{\texorpdfstring{\texttt{\ block\ }}{ block }}\label{parameters-block}

\href{/docs/reference/foundations/bool/}{bool}

{{ Settable }}

\phantomsection\label{parameters-block-settable-tooltip}
Settable parameters can be customized for all following uses of the
function with a \texttt{\ set\ } rule.

Whether the equation is displayed as a separate block.

Default: \texttt{\ }{\texttt{\ false\ }}\texttt{\ }

\subsubsection{\texorpdfstring{\texttt{\ numbering\ }}{ numbering }}\label{parameters-numbering}

\href{/docs/reference/foundations/none/}{none} {or}
\href{/docs/reference/foundations/str/}{str} {or}
\href{/docs/reference/foundations/function/}{function}

{{ Settable }}

\phantomsection\label{parameters-numbering-settable-tooltip}
Settable parameters can be customized for all following uses of the
function with a \texttt{\ set\ } rule.

How to \href{/docs/reference/model/numbering/}{number} block-level
equations.

Default: \texttt{\ }{\texttt{\ none\ }}\texttt{\ }

\includesvg[width=0.16667in,height=0.16667in]{/assets/icons/16-arrow-right.svg}
View example

\begin{verbatim}
#set math.equation(numbering: "(1)")

We define:
$ phi.alt := (1 + sqrt(5)) / 2 $ <ratio>

With @ratio, we get:
$ F_n = floor(1 / sqrt(5) phi.alt^n) $
\end{verbatim}

\includegraphics[width=5in,height=\textheight,keepaspectratio]{/assets/docs/ICkRN4qFA2wn3VV_dGJcKAAAAAAAAAAA.png}

\subsubsection{\texorpdfstring{\texttt{\ number-align\ }}{ number-align }}\label{parameters-number-align}

\href{/docs/reference/layout/alignment/}{alignment}

{{ Settable }}

\phantomsection\label{parameters-number-align-settable-tooltip}
Settable parameters can be customized for all following uses of the
function with a \texttt{\ set\ } rule.

The alignment of the equation numbering.

By default, the alignment is
\texttt{\ end\ }{\texttt{\ +\ }}\texttt{\ horizon\ } . For the
horizontal component, you can use \texttt{\ right\ } , \texttt{\ left\ }
, or \texttt{\ start\ } and \texttt{\ end\ } of the text direction; for
the vertical component, you can use \texttt{\ top\ } ,
\texttt{\ horizon\ } , or \texttt{\ bottom\ } .

Default: \texttt{\ end\ }{\texttt{\ +\ }}\texttt{\ horizon\ }

\includesvg[width=0.16667in,height=0.16667in]{/assets/icons/16-arrow-right.svg}
View example

\begin{verbatim}
#set math.equation(numbering: "(1)", number-align: bottom)

We can calculate:
$ E &= sqrt(m_0^2 + p^2) \
    &approx 125 "GeV" $
\end{verbatim}

\includegraphics[width=5in,height=\textheight,keepaspectratio]{/assets/docs/EjQKswH-OBAc5Rwhl-7WNQAAAAAAAAAA.png}

\subsubsection{\texorpdfstring{\texttt{\ supplement\ }}{ supplement }}\label{parameters-supplement}

\href{/docs/reference/foundations/none/}{none} {or}
\href{/docs/reference/foundations/auto/}{auto} {or}
\href{/docs/reference/foundations/content/}{content} {or}
\href{/docs/reference/foundations/function/}{function}

{{ Settable }}

\phantomsection\label{parameters-supplement-settable-tooltip}
Settable parameters can be customized for all following uses of the
function with a \texttt{\ set\ } rule.

A supplement for the equation.

For references to equations, this is added before the referenced number.

If a function is specified, it is passed the referenced equation and
should return content.

Default: \texttt{\ }{\texttt{\ auto\ }}\texttt{\ }

\includesvg[width=0.16667in,height=0.16667in]{/assets/icons/16-arrow-right.svg}
View example

\begin{verbatim}
#set math.equation(numbering: "(1)", supplement: [Eq.])

We define:
$ phi.alt := (1 + sqrt(5)) / 2 $ <ratio>

With @ratio, we get:
$ F_n = floor(1 / sqrt(5) phi.alt^n) $
\end{verbatim}

\includegraphics[width=5in,height=\textheight,keepaspectratio]{/assets/docs/LsvSGn7Nchg2dddv3zDBtAAAAAAAAAAA.png}

\subsubsection{\texorpdfstring{\texttt{\ body\ }}{ body }}\label{parameters-body}

\href{/docs/reference/foundations/content/}{content}

{Required} {{ Positional }}

\phantomsection\label{parameters-body-positional-tooltip}
Positional parameters are specified in order, without names.

The contents of the equation.

\href{/docs/reference/math/class/}{\pandocbounded{\includesvg[keepaspectratio]{/assets/icons/16-arrow-right.svg}}}

{ Class } { Previous page }

\href{/docs/reference/math/frac/}{\pandocbounded{\includesvg[keepaspectratio]{/assets/icons/16-arrow-right.svg}}}

{ Fraction } { Next page }


