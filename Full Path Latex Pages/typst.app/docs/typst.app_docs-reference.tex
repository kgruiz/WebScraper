\title{typst.app/docs/reference}

\begin{itemize}
\tightlist
\item
  \href{/docs}{\includesvg[width=0.16667in,height=0.16667in]{/assets/icons/16-docs-dark.svg}}
\item
  \includesvg[width=0.16667in,height=0.16667in]{/assets/icons/16-arrow-right.svg}
\item
  \href{/docs/reference/}{Reference}
\end{itemize}

\section{Reference}\label{reference}

This reference documentation is a comprehensive guide to all of
Typst\textquotesingle s syntax, concepts, types, and functions. If you
are completely new to Typst, we recommend starting with the
\href{/docs/tutorial/}{tutorial} and then coming back to the reference
to learn more about Typst\textquotesingle s features as you need them.

\subsection{Language}\label{language}

The reference starts with a language part that gives an overview over
\href{/docs/reference/syntax/}{Typst\textquotesingle s syntax} and
contains information about concepts involved in
\href{/docs/reference/styling/}{styling documents,} using
\href{/docs/reference/scripting/}{Typst\textquotesingle s scripting
capabilities.}

\subsection{Functions}\label{functions}

The second part includes chapters on all functions used to insert,
style, transform, and layout content in Typst documents. Each function
is documented with a description of its purpose, a list of its
parameters, and examples of how to use it.

The final part of the reference explains all functions that are used
within Typst\textquotesingle s code mode to manipulate and transform
data. Just as in the previous part, each function is documented with a
description of its purpose, a list of its parameters, and examples of
how to use it.

\href{/docs/tutorial/making-a-template/}{\pandocbounded{\includesvg[keepaspectratio]{/assets/icons/16-arrow-right.svg}}}

{ Making a Template } { Previous page }

\href{/docs/reference/syntax/}{\pandocbounded{\includesvg[keepaspectratio]{/assets/icons/16-arrow-right.svg}}}

{ Syntax } { Next page }
