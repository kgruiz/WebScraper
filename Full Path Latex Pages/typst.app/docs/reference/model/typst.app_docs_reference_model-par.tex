\title{typst.app/docs/reference/model/par}

\begin{itemize}
\tightlist
\item
  \href{/docs}{\includesvg[width=0.16667in,height=0.16667in]{/assets/icons/16-docs-dark.svg}}
\item
  \includesvg[width=0.16667in,height=0.16667in]{/assets/icons/16-arrow-right.svg}
\item
  \href{/docs/reference/}{Reference}
\item
  \includesvg[width=0.16667in,height=0.16667in]{/assets/icons/16-arrow-right.svg}
\item
  \href{/docs/reference/model/}{Model}
\item
  \includesvg[width=0.16667in,height=0.16667in]{/assets/icons/16-arrow-right.svg}
\item
  \href{/docs/reference/model/par/}{Paragraph}
\end{itemize}

\section{\texorpdfstring{\texttt{\ par\ } {{ Element
}}}{ par   Element }}\label{summary}

\phantomsection\label{element-tooltip}
Element functions can be customized with \texttt{\ set\ } and
\texttt{\ show\ } rules.

Arranges text, spacing and inline-level elements into a paragraph.

Although this function is primarily used in set rules to affect
paragraph properties, it can also be used to explicitly render its
argument onto a paragraph of its own.

\subsection{Example}\label{example}

\begin{verbatim}
#set par(
  first-line-indent: 1em,
  spacing: 0.65em,
  justify: true,
)

We proceed by contradiction.
Suppose that there exists a set
of positive integers $a$, $b$, and
$c$ that satisfies the equation
$a^n + b^n = c^n$ for some
integer value of $n > 2$.

Without loss of generality,
let $a$ be the smallest of the
three integers. Then, we ...
\end{verbatim}

\includegraphics[width=5in,height=\textheight,keepaspectratio]{/assets/docs/yrIipb0QYzuDEgQNZF57rwAAAAAAAAAA.png}

\subsection{\texorpdfstring{{ Parameters
}}{ Parameters }}\label{parameters}

\phantomsection\label{parameters-tooltip}
Parameters are the inputs to a function. They are specified in
parentheses after the function name.

{ par } (

{ \hyperref[parameters-leading]{leading :}
\href{/docs/reference/layout/length/}{length} , } {
\hyperref[parameters-spacing]{spacing :}
\href{/docs/reference/layout/length/}{length} , } {
\hyperref[parameters-justify]{justify :}
\href{/docs/reference/foundations/bool/}{bool} , } {
\hyperref[parameters-linebreaks]{linebreaks :}
\href{/docs/reference/foundations/auto/}{auto}
\href{/docs/reference/foundations/str/}{str} , } {
\hyperref[parameters-first-line-indent]{first-line-indent :}
\href{/docs/reference/layout/length/}{length} , } {
\hyperref[parameters-hanging-indent]{hanging-indent :}
\href{/docs/reference/layout/length/}{length} , } {
\href{/docs/reference/foundations/content/}{content} , }

) -\textgreater{} \href{/docs/reference/foundations/content/}{content}

\subsubsection{\texorpdfstring{\texttt{\ leading\ }}{ leading }}\label{parameters-leading}

\href{/docs/reference/layout/length/}{length}

{{ Settable }}

\phantomsection\label{parameters-leading-settable-tooltip}
Settable parameters can be customized for all following uses of the
function with a \texttt{\ set\ } rule.

The spacing between lines.

Leading defines the spacing between the
\href{/docs/reference/text/text/\#parameters-bottom-edge}{bottom edge}
of one line and the
\href{/docs/reference/text/text/\#parameters-top-edge}{top edge} of the
following line. By default, these two properties are up to the font, but
they can also be configured manually with a text set rule.

By setting top edge, bottom edge, and leading, you can also configure a
consistent baseline-to-baseline distance. You could, for instance, set
the leading to \texttt{\ }{\texttt{\ 1em\ }}\texttt{\ } , the top-edge
to \texttt{\ }{\texttt{\ 0.8em\ }}\texttt{\ } , and the bottom-edge to
\texttt{\ }{\texttt{\ -\ }}\texttt{\ }{\texttt{\ 0.2em\ }}\texttt{\ } to
get a baseline gap of exactly \texttt{\ }{\texttt{\ 2em\ }}\texttt{\ } .
The exact distribution of the top- and bottom-edge values affects the
bounds of the first and last line.

Default: \texttt{\ }{\texttt{\ 0.65em\ }}\texttt{\ }

\subsubsection{\texorpdfstring{\texttt{\ spacing\ }}{ spacing }}\label{parameters-spacing}

\href{/docs/reference/layout/length/}{length}

{{ Settable }}

\phantomsection\label{parameters-spacing-settable-tooltip}
Settable parameters can be customized for all following uses of the
function with a \texttt{\ set\ } rule.

The spacing between paragraphs.

Just like leading, this defines the spacing between the bottom edge of a
paragraph\textquotesingle s last line and the top edge of the next
paragraph\textquotesingle s first line.

When a paragraph is adjacent to a
\href{/docs/reference/layout/block/}{\texttt{\ block\ }} that is not a
paragraph, that block\textquotesingle s
\href{/docs/reference/layout/block/\#parameters-above}{\texttt{\ above\ }}
or
\href{/docs/reference/layout/block/\#parameters-below}{\texttt{\ below\ }}
property takes precedence over the paragraph spacing. Headings, for
instance, reduce the spacing below them by default for a better look.

Default: \texttt{\ }{\texttt{\ 1.2em\ }}\texttt{\ }

\subsubsection{\texorpdfstring{\texttt{\ justify\ }}{ justify }}\label{parameters-justify}

\href{/docs/reference/foundations/bool/}{bool}

{{ Settable }}

\phantomsection\label{parameters-justify-settable-tooltip}
Settable parameters can be customized for all following uses of the
function with a \texttt{\ set\ } rule.

Whether to justify text in its line.

Hyphenation will be enabled for justified paragraphs if the
\href{/docs/reference/text/text/\#parameters-hyphenate}{text
function\textquotesingle s \texttt{\ hyphenate\ } property} is set to
\texttt{\ }{\texttt{\ auto\ }}\texttt{\ } and the current language is
known.

Note that the current
\href{/docs/reference/layout/align/\#parameters-alignment}{alignment}
still has an effect on the placement of the last line except if it ends
with a
\href{/docs/reference/text/linebreak/\#parameters-justify}{justified
line break} .

Default: \texttt{\ }{\texttt{\ false\ }}\texttt{\ }

\subsubsection{\texorpdfstring{\texttt{\ linebreaks\ }}{ linebreaks }}\label{parameters-linebreaks}

\href{/docs/reference/foundations/auto/}{auto} {or}
\href{/docs/reference/foundations/str/}{str}

{{ Settable }}

\phantomsection\label{parameters-linebreaks-settable-tooltip}
Settable parameters can be customized for all following uses of the
function with a \texttt{\ set\ } rule.

How to determine line breaks.

When this property is set to \texttt{\ }{\texttt{\ auto\ }}\texttt{\ } ,
its default value, optimized line breaks will be used for justified
paragraphs. Enabling optimized line breaks for ragged paragraphs may
also be worthwhile to improve the appearance of the text.

\begin{longtable}[]{@{}
  >{\raggedright\arraybackslash}p{(\linewidth - 2\tabcolsep) * \real{0.5000}}
  >{\raggedright\arraybackslash}p{(\linewidth - 2\tabcolsep) * \real{0.5000}}@{}}
\toprule\noalign{}
\begin{minipage}[b]{\linewidth}\raggedright
Variant
\end{minipage} & \begin{minipage}[b]{\linewidth}\raggedright
Details
\end{minipage} \\
\midrule\noalign{}
\endhead
\bottomrule\noalign{}
\endlastfoot
\texttt{\ "\ simple\ "\ } & Determine the line breaks in a simple
first-fit style. \\
\texttt{\ "\ optimized\ "\ } & Optimize the line breaks for the whole
paragraph.

Typst will try to produce more evenly filled lines of text by
considering the whole paragraph when calculating line breaks. \\
\end{longtable}

Default: \texttt{\ }{\texttt{\ auto\ }}\texttt{\ }

\includesvg[width=0.16667in,height=0.16667in]{/assets/icons/16-arrow-right.svg}
View example

\begin{verbatim}
#set page(width: 207pt)
#set par(linebreaks: "simple")
Some texts feature many longer
words. Those are often exceedingly
challenging to break in a visually
pleasing way.

#set par(linebreaks: "optimized")
Some texts feature many longer
words. Those are often exceedingly
challenging to break in a visually
pleasing way.
\end{verbatim}

\includegraphics[width=4.3125in,height=\textheight,keepaspectratio]{/assets/docs/r-fawkmmJ6Sniwi8--ib5gAAAAAAAAAA.png}

\subsubsection{\texorpdfstring{\texttt{\ first-line-indent\ }}{ first-line-indent }}\label{parameters-first-line-indent}

\href{/docs/reference/layout/length/}{length}

{{ Settable }}

\phantomsection\label{parameters-first-line-indent-settable-tooltip}
Settable parameters can be customized for all following uses of the
function with a \texttt{\ set\ } rule.

The indent the first line of a paragraph should have.

Only the first line of a consecutive paragraph will be indented (not the
first one in a block or on the page).

By typographic convention, paragraph breaks are indicated either by some
space between paragraphs or by indented first lines. Consider reducing
the \href{/docs/reference/layout/block/\#parameters-spacing}{paragraph
spacing} to the
\href{/docs/reference/model/par/\#parameters-leading}{\texttt{\ leading\ }}
when using this property (e.g. using
\texttt{\ }{\texttt{\ \#\ }}\texttt{\ }{\texttt{\ set\ }}\texttt{\ }{\texttt{\ par\ }}\texttt{\ }{\texttt{\ (\ }}\texttt{\ spacing\ }{\texttt{\ :\ }}\texttt{\ }{\texttt{\ 0.65em\ }}\texttt{\ }{\texttt{\ )\ }}\texttt{\ }
).

Default: \texttt{\ }{\texttt{\ 0pt\ }}\texttt{\ }

\subsubsection{\texorpdfstring{\texttt{\ hanging-indent\ }}{ hanging-indent }}\label{parameters-hanging-indent}

\href{/docs/reference/layout/length/}{length}

{{ Settable }}

\phantomsection\label{parameters-hanging-indent-settable-tooltip}
Settable parameters can be customized for all following uses of the
function with a \texttt{\ set\ } rule.

The indent all but the first line of a paragraph should have.

Default: \texttt{\ }{\texttt{\ 0pt\ }}\texttt{\ }

\subsubsection{\texorpdfstring{\texttt{\ body\ }}{ body }}\label{parameters-body}

\href{/docs/reference/foundations/content/}{content}

{Required} {{ Positional }}

\phantomsection\label{parameters-body-positional-tooltip}
Positional parameters are specified in order, without names.

The contents of the paragraph.

\subsection{\texorpdfstring{{ Definitions
}}{ Definitions }}\label{definitions}

\phantomsection\label{definitions-tooltip}
Functions and types and can have associated definitions. These are
accessed by specifying the function or type, followed by a period, and
then the definition\textquotesingle s name.

\subsubsection{\texorpdfstring{\texttt{\ line\ } {{ Element
}}}{ line   Element }}\label{definitions-line}

\phantomsection\label{definitions-line-element-tooltip}
Element functions can be customized with \texttt{\ set\ } and
\texttt{\ show\ } rules.

A paragraph line.

This element is exclusively used for line number configuration through
set rules and cannot be placed.

The
\href{/docs/reference/model/par/\#definitions-line-numbering}{\texttt{\ numbering\ }}
option is used to enable line numbers by specifying a numbering format.

par { . } { line } (

{ \hyperref[definitions-line-parameters-numbering]{numbering :}
\href{/docs/reference/foundations/none/}{none}
\href{/docs/reference/foundations/str/}{str}
\href{/docs/reference/foundations/function/}{function} , } {
\hyperref[definitions-line-parameters-number-align]{number-align :}
\href{/docs/reference/foundations/auto/}{auto}
\href{/docs/reference/layout/alignment/}{alignment} , } {
\hyperref[definitions-line-parameters-number-margin]{number-margin :}
\href{/docs/reference/layout/alignment/}{alignment} , } {
\hyperref[definitions-line-parameters-number-clearance]{number-clearance
:} \href{/docs/reference/foundations/auto/}{auto}
\href{/docs/reference/layout/length/}{length} , } {
\hyperref[definitions-line-parameters-numbering-scope]{numbering-scope
:} \href{/docs/reference/foundations/str/}{str} , }

) -\textgreater{} \href{/docs/reference/foundations/content/}{content}

\begin{verbatim}
#set par.line(numbering: "1")

Roses are red. \
Violets are blue. \
Typst is there for you.
\end{verbatim}

\includegraphics[width=5in,height=\textheight,keepaspectratio]{/assets/docs/b257YLHUEagbFWlPeD4gEwAAAAAAAAAA.png}

The \texttt{\ numbering\ } option takes either a predefined
\href{/docs/reference/model/numbering/}{numbering pattern} or a function
returning styled content. You can disable line numbers for text inside
certain elements by setting the numbering to
\texttt{\ }{\texttt{\ none\ }}\texttt{\ } using show-set rules.

\begin{verbatim}
// Styled red line numbers.
#set par.line(
  numbering: n => text(red)[#n]
)

// Disable numbers inside figures.
#show figure: set par.line(
  numbering: none
)

Roses are red. \
Violets are blue.

#figure(
  caption: [Without line numbers.]
)[
  Lorem ipsum \
  dolor sit amet
]

The text above is a sample \
originating from distant times.
\end{verbatim}

\includegraphics[width=5in,height=\textheight,keepaspectratio]{/assets/docs/WJNwFvR3ObvODT-MbWqflAAAAAAAAAAA.png}

This element exposes further options which may be used to control other
aspects of line numbering, such as its
\href{/docs/reference/model/par/\#definitions-line-number-align}{alignment}
or
\href{/docs/reference/model/par/\#definitions-line-number-margin}{margin}
. In addition, you can control whether the numbering is reset on each
page through the
\href{/docs/reference/model/par/\#definitions-line-numbering-scope}{\texttt{\ numbering-scope\ }}
option.

\paragraph{\texorpdfstring{\texttt{\ numbering\ }}{ numbering }}\label{definitions-line-numbering}

\href{/docs/reference/foundations/none/}{none} {or}
\href{/docs/reference/foundations/str/}{str} {or}
\href{/docs/reference/foundations/function/}{function}

{{ Settable }}

\phantomsection\label{definitions-line-numbering-settable-tooltip}
Settable parameters can be customized for all following uses of the
function with a \texttt{\ set\ } rule.

How to number each line. Accepts a
\href{/docs/reference/model/numbering/}{numbering pattern or function} .

Default: \texttt{\ }{\texttt{\ none\ }}\texttt{\ }

\includesvg[width=0.16667in,height=0.16667in]{/assets/icons/16-arrow-right.svg}
View example

\begin{verbatim}
#set par.line(numbering: "I")

Roses are red. \
Violets are blue. \
Typst is there for you.
\end{verbatim}

\includegraphics[width=5in,height=\textheight,keepaspectratio]{/assets/docs/O-oJqYc-OwEoxappxK4AZAAAAAAAAAAA.png}

\paragraph{\texorpdfstring{\texttt{\ number-align\ }}{ number-align }}\label{definitions-line-number-align}

\href{/docs/reference/foundations/auto/}{auto} {or}
\href{/docs/reference/layout/alignment/}{alignment}

{{ Settable }}

\phantomsection\label{definitions-line-number-align-settable-tooltip}
Settable parameters can be customized for all following uses of the
function with a \texttt{\ set\ } rule.

The alignment of line numbers associated with each line.

The default of \texttt{\ }{\texttt{\ auto\ }}\texttt{\ } indicates a
smart default where numbers grow horizontally away from the text,
considering the margin they\textquotesingle re in and the current text
direction.

Default: \texttt{\ }{\texttt{\ auto\ }}\texttt{\ }

\includesvg[width=0.16667in,height=0.16667in]{/assets/icons/16-arrow-right.svg}
View example

\begin{verbatim}
#set par.line(
  numbering: "I",
  number-align: left,
)

Hello world! \
Today is a beautiful day \
For exploring the world.
\end{verbatim}

\includegraphics[width=5in,height=\textheight,keepaspectratio]{/assets/docs/XfwBMgYjt2fGeRgFr_kj4AAAAAAAAAAA.png}

\paragraph{\texorpdfstring{\texttt{\ number-margin\ }}{ number-margin }}\label{definitions-line-number-margin}

\href{/docs/reference/layout/alignment/}{alignment}

{{ Settable }}

\phantomsection\label{definitions-line-number-margin-settable-tooltip}
Settable parameters can be customized for all following uses of the
function with a \texttt{\ set\ } rule.

The margin at which line numbers appear.

\emph{Note:} In a multi-column document, the line numbers for paragraphs
inside the last column will always appear on the \texttt{\ end\ } margin
(right margin for left-to-right text and left margin for right-to-left),
regardless of this configuration. That behavior cannot be changed at
this moment.

Default: \texttt{\ start\ }

\includesvg[width=0.16667in,height=0.16667in]{/assets/icons/16-arrow-right.svg}
View example

\begin{verbatim}
#set par.line(
  numbering: "1",
  number-margin: right,
)

= Report
- Brightness: Dark, yet darker
- Readings: Negative
\end{verbatim}

\includegraphics[width=5in,height=\textheight,keepaspectratio]{/assets/docs/vf0ZBrlygVUABySMskTaKQAAAAAAAAAA.png}

\paragraph{\texorpdfstring{\texttt{\ number-clearance\ }}{ number-clearance }}\label{definitions-line-number-clearance}

\href{/docs/reference/foundations/auto/}{auto} {or}
\href{/docs/reference/layout/length/}{length}

{{ Settable }}

\phantomsection\label{definitions-line-number-clearance-settable-tooltip}
Settable parameters can be customized for all following uses of the
function with a \texttt{\ set\ } rule.

The distance between line numbers and text.

The default value of \texttt{\ }{\texttt{\ auto\ }}\texttt{\ } results
in a clearance that is adaptive to the page width and yields reasonable
results in most cases.

Default: \texttt{\ }{\texttt{\ auto\ }}\texttt{\ }

\includesvg[width=0.16667in,height=0.16667in]{/assets/icons/16-arrow-right.svg}
View example

\begin{verbatim}
#set par.line(
  numbering: "1",
  number-clearance: 4pt,
)

Typesetting \
Styling \
Layout
\end{verbatim}

\includegraphics[width=5in,height=\textheight,keepaspectratio]{/assets/docs/MgiUB3LoxE0JROWoHJPslgAAAAAAAAAA.png}

\paragraph{\texorpdfstring{\texttt{\ numbering-scope\ }}{ numbering-scope }}\label{definitions-line-numbering-scope}

\href{/docs/reference/foundations/str/}{str}

{{ Settable }}

\phantomsection\label{definitions-line-numbering-scope-settable-tooltip}
Settable parameters can be customized for all following uses of the
function with a \texttt{\ set\ } rule.

Controls when to reset line numbering.

\emph{Note:} The line numbering scope must be uniform across each page
run (a page run is a sequence of pages without an explicit pagebreak in
between). For this reason, set rules for it should be defined before any
page content, typically at the very start of the document.

\begin{longtable}[]{@{}ll@{}}
\toprule\noalign{}
Variant & Details \\
\midrule\noalign{}
\endhead
\bottomrule\noalign{}
\endlastfoot
\texttt{\ "\ document\ "\ } & Indicates the line number counter spans
the whole document, that is, is never automatically reset. \\
\texttt{\ "\ page\ "\ } & Indicates the line number counter should be
reset at the start of every new page. \\
\end{longtable}

Default: \texttt{\ }{\texttt{\ "document"\ }}\texttt{\ }

\includesvg[width=0.16667in,height=0.16667in]{/assets/icons/16-arrow-right.svg}
View example

\begin{verbatim}
#set par.line(
  numbering: "1",
  numbering-scope: "page",
)

First line \
Second line
#pagebreak()
First line again \
Second line again
\end{verbatim}

\includegraphics[width=5in,height=\textheight,keepaspectratio]{/assets/docs/MmmIOu-UB2sC4GlOg3oj9AAAAAAAAAAA.png}
\includegraphics[width=5in,height=\textheight,keepaspectratio]{/assets/docs/MmmIOu-UB2sC4GlOg3oj9AAAAAAAAAAB.png}

\href{/docs/reference/model/outline/}{\pandocbounded{\includesvg[keepaspectratio]{/assets/icons/16-arrow-right.svg}}}

{ Outline } { Previous page }

\href{/docs/reference/model/parbreak/}{\pandocbounded{\includesvg[keepaspectratio]{/assets/icons/16-arrow-right.svg}}}

{ Paragraph Break } { Next page }
