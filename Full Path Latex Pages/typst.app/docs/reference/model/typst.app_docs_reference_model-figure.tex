\title{typst.app/docs/reference/model/figure}

\begin{itemize}
\tightlist
\item
  \href{/docs}{\includesvg[width=0.16667in,height=0.16667in]{/assets/icons/16-docs-dark.svg}}
\item
  \includesvg[width=0.16667in,height=0.16667in]{/assets/icons/16-arrow-right.svg}
\item
  \href{/docs/reference/}{Reference}
\item
  \includesvg[width=0.16667in,height=0.16667in]{/assets/icons/16-arrow-right.svg}
\item
  \href{/docs/reference/model/}{Model}
\item
  \includesvg[width=0.16667in,height=0.16667in]{/assets/icons/16-arrow-right.svg}
\item
  \href{/docs/reference/model/figure/}{Figure}
\end{itemize}

\section{\texorpdfstring{\texttt{\ figure\ } {{ Element
}}}{ figure   Element }}\label{summary}

\phantomsection\label{element-tooltip}
Element functions can be customized with \texttt{\ set\ } and
\texttt{\ show\ } rules.

A figure with an optional caption.

Automatically detects its kind to select the correct counting track. For
example, figures containing images will be numbered separately from
figures containing tables.

\subsection{Examples}\label{examples}

The example below shows a basic figure with an image:

\begin{verbatim}
@glacier shows a glacier. Glaciers
are complex systems.

#figure(
  image("glacier.jpg", width: 80%),
  caption: [A curious figure.],
) <glacier>
\end{verbatim}

\includegraphics[width=5in,height=\textheight,keepaspectratio]{/assets/docs/udw8J1zW8CDfoYB1XTzdLgAAAAAAAAAA.png}

You can also insert \href{/docs/reference/model/table/}{tables} into
figures to give them a caption. The figure will detect this and
automatically use a separate counter.

\begin{verbatim}
#figure(
  table(
    columns: 4,
    [t], [1], [2], [3],
    [y], [0.3s], [0.4s], [0.8s],
  ),
  caption: [Timing results],
)
\end{verbatim}

\includegraphics[width=5in,height=\textheight,keepaspectratio]{/assets/docs/_RaOJik9O5UoQO8Eq6OM9gAAAAAAAAAA.png}

This behaviour can be overridden by explicitly specifying the
figure\textquotesingle s \texttt{\ kind\ } . All figures of the same
kind share a common counter.

\subsection{Figure behaviour}\label{figure-behaviour}

By default, figures are placed within the flow of content. To make them
float to the top or bottom of the page, you can use the
\href{/docs/reference/model/figure/\#parameters-placement}{\texttt{\ placement\ }}
argument.

If your figure is too large and its contents are breakable across pages
(e.g. if it contains a large table), then you can make the figure itself
breakable across pages as well with this show rule:

\begin{verbatim}
#show figure: set block(breakable: true)
\end{verbatim}

See the
\href{/docs/reference/layout/block/\#parameters-breakable}{block}
documentation for more information about breakable and non-breakable
blocks.

\subsection{Caption customization}\label{caption-customization}

You can modify the appearance of the figure\textquotesingle s caption
with its associated
\href{/docs/reference/model/figure/\#definitions-caption}{\texttt{\ caption\ }}
function. In the example below, we emphasize all captions:

\begin{verbatim}
#show figure.caption: emph

#figure(
  rect[Hello],
  caption: [I am emphasized!],
)
\end{verbatim}

\includegraphics[width=5in,height=\textheight,keepaspectratio]{/assets/docs/_XYhSBTt1dmjYR9A4n_aCgAAAAAAAAAA.png}

By using a
\href{/docs/reference/foundations/function/\#definitions-where}{\texttt{\ where\ }}
selector, we can scope such rules to specific kinds of figures. For
example, to position the caption above tables, but keep it below for all
other kinds of figures, we could write the following show-set rule:

\begin{verbatim}
#show figure.where(
  kind: table
): set figure.caption(position: top)

#figure(
  table(columns: 2)[A][B][C][D],
  caption: [I'm up here],
)
\end{verbatim}

\includegraphics[width=5in,height=\textheight,keepaspectratio]{/assets/docs/5FXY-vQKID4Q1FYsR4Ix9AAAAAAAAAAA.png}

\subsection{\texorpdfstring{{ Parameters
}}{ Parameters }}\label{parameters}

\phantomsection\label{parameters-tooltip}
Parameters are the inputs to a function. They are specified in
parentheses after the function name.

{ figure } (

{ \href{/docs/reference/foundations/content/}{content} , } {
\hyperref[parameters-placement]{placement :}
\href{/docs/reference/foundations/none/}{none}
\href{/docs/reference/foundations/auto/}{auto}
\href{/docs/reference/layout/alignment/}{alignment} , } {
\hyperref[parameters-scope]{scope :}
\href{/docs/reference/foundations/str/}{str} , } {
\hyperref[parameters-caption]{caption :}
\href{/docs/reference/foundations/none/}{none}
\href{/docs/reference/foundations/content/}{content} , } {
\hyperref[parameters-kind]{kind :}
\href{/docs/reference/foundations/auto/}{auto}
\href{/docs/reference/foundations/str/}{str}
\href{/docs/reference/foundations/function/}{function} , } {
\hyperref[parameters-supplement]{supplement :}
\href{/docs/reference/foundations/none/}{none}
\href{/docs/reference/foundations/auto/}{auto}
\href{/docs/reference/foundations/content/}{content}
\href{/docs/reference/foundations/function/}{function} , } {
\hyperref[parameters-numbering]{numbering :}
\href{/docs/reference/foundations/none/}{none}
\href{/docs/reference/foundations/str/}{str}
\href{/docs/reference/foundations/function/}{function} , } {
\hyperref[parameters-gap]{gap :}
\href{/docs/reference/layout/length/}{length} , } {
\hyperref[parameters-outlined]{outlined :}
\href{/docs/reference/foundations/bool/}{bool} , }

) -\textgreater{} \href{/docs/reference/foundations/content/}{content}

\subsubsection{\texorpdfstring{\texttt{\ body\ }}{ body }}\label{parameters-body}

\href{/docs/reference/foundations/content/}{content}

{Required} {{ Positional }}

\phantomsection\label{parameters-body-positional-tooltip}
Positional parameters are specified in order, without names.

The content of the figure. Often, an
\href{/docs/reference/visualize/image/}{image} .

\subsubsection{\texorpdfstring{\texttt{\ placement\ }}{ placement }}\label{parameters-placement}

\href{/docs/reference/foundations/none/}{none} {or}
\href{/docs/reference/foundations/auto/}{auto} {or}
\href{/docs/reference/layout/alignment/}{alignment}

{{ Settable }}

\phantomsection\label{parameters-placement-settable-tooltip}
Settable parameters can be customized for all following uses of the
function with a \texttt{\ set\ } rule.

The figure\textquotesingle s placement on the page.

\begin{itemize}
\tightlist
\item
  \texttt{\ }{\texttt{\ none\ }}\texttt{\ } : The figure stays in-flow
  exactly where it was specified like other content.
\item
  \texttt{\ }{\texttt{\ auto\ }}\texttt{\ } : The figure picks
  \texttt{\ top\ } or \texttt{\ bottom\ } depending on which is closer.
\item
  \texttt{\ top\ } : The figure floats to the top of the page.
\item
  \texttt{\ bottom\ } : The figure floats to the bottom of the page.
\end{itemize}

The gap between the main flow content and the floating figure is
controlled by the
\href{/docs/reference/layout/place/\#parameters-clearance}{\texttt{\ clearance\ }}
argument on the \texttt{\ place\ } function.

Default: \texttt{\ }{\texttt{\ none\ }}\texttt{\ }

\includesvg[width=0.16667in,height=0.16667in]{/assets/icons/16-arrow-right.svg}
View example

\begin{verbatim}
#set page(height: 200pt)

= Introduction
#figure(
  placement: bottom,
  caption: [A glacier],
  image("glacier.jpg", width: 60%),
)
#lorem(60)
\end{verbatim}

\includegraphics[width=5in,height=\textheight,keepaspectratio]{/assets/docs/AvTTV4CvkxyZB8XrzNUT3wAAAAAAAAAA.png}
\includegraphics[width=5in,height=\textheight,keepaspectratio]{/assets/docs/AvTTV4CvkxyZB8XrzNUT3wAAAAAAAAAB.png}

\subsubsection{\texorpdfstring{\texttt{\ scope\ }}{ scope }}\label{parameters-scope}

\href{/docs/reference/foundations/str/}{str}

{{ Settable }}

\phantomsection\label{parameters-scope-settable-tooltip}
Settable parameters can be customized for all following uses of the
function with a \texttt{\ set\ } rule.

Relative to which containing scope the figure is placed.

Set this to \texttt{\ }{\texttt{\ "parent"\ }}\texttt{\ } to create a
full-width figure in a two-column document.

Has no effect if \texttt{\ placement\ } is
\texttt{\ }{\texttt{\ none\ }}\texttt{\ } .

\begin{longtable}[]{@{}ll@{}}
\toprule\noalign{}
Variant & Details \\
\midrule\noalign{}
\endhead
\bottomrule\noalign{}
\endlastfoot
\texttt{\ "\ column\ "\ } & Place into the current column. \\
\texttt{\ "\ parent\ "\ } & Place relative to the parent, letting the
content span over all columns. \\
\end{longtable}

Default: \texttt{\ }{\texttt{\ "column"\ }}\texttt{\ }

\includesvg[width=0.16667in,height=0.16667in]{/assets/icons/16-arrow-right.svg}
View example

\begin{verbatim}
#set page(height: 250pt, columns: 2)

= Introduction
#figure(
  placement: bottom,
  scope: "parent",
  caption: [A glacier],
  image("glacier.jpg", width: 60%),
)
#lorem(60)
\end{verbatim}

\includegraphics[width=5in,height=\textheight,keepaspectratio]{/assets/docs/_zX5K9NHfd2mYYCeJmag7wAAAAAAAAAA.png}
\includegraphics[width=5in,height=\textheight,keepaspectratio]{/assets/docs/_zX5K9NHfd2mYYCeJmag7wAAAAAAAAAB.png}

\subsubsection{\texorpdfstring{\texttt{\ caption\ }}{ caption }}\label{parameters-caption}

\href{/docs/reference/foundations/none/}{none} {or}
\href{/docs/reference/foundations/content/}{content}

{{ Settable }}

\phantomsection\label{parameters-caption-settable-tooltip}
Settable parameters can be customized for all following uses of the
function with a \texttt{\ set\ } rule.

The figure\textquotesingle s caption.

Default: \texttt{\ }{\texttt{\ none\ }}\texttt{\ }

\subsubsection{\texorpdfstring{\texttt{\ kind\ }}{ kind }}\label{parameters-kind}

\href{/docs/reference/foundations/auto/}{auto} {or}
\href{/docs/reference/foundations/str/}{str} {or}
\href{/docs/reference/foundations/function/}{function}

{{ Settable }}

\phantomsection\label{parameters-kind-settable-tooltip}
Settable parameters can be customized for all following uses of the
function with a \texttt{\ set\ } rule.

The kind of figure this is.

All figures of the same kind share a common counter.

If set to \texttt{\ }{\texttt{\ auto\ }}\texttt{\ } , the figure will
try to automatically determine its kind based on the type of its body.
Automatically detected kinds are
\href{/docs/reference/model/table/}{tables} and
\href{/docs/reference/text/raw/}{code} . In other cases, the inferred
kind is that of an \href{/docs/reference/visualize/image/}{image} .

Setting this to something other than
\texttt{\ }{\texttt{\ auto\ }}\texttt{\ } will override the automatic
detection. This can be useful if

\begin{itemize}
\tightlist
\item
  you wish to create a custom figure type that is not an
  \href{/docs/reference/visualize/image/}{image} , a
  \href{/docs/reference/model/table/}{table} or
  \href{/docs/reference/text/raw/}{code} ,
\item
  you want to force the figure to use a specific counter regardless of
  its content.
\end{itemize}

You can set the kind to be an element function or a string. If you set
it to an element function other than
\href{/docs/reference/model/table/}{\texttt{\ table\ }} ,
\href{/docs/reference/text/raw/}{\texttt{\ raw\ }} or
\href{/docs/reference/visualize/image/}{\texttt{\ image\ }} , you will
need to manually specify the figure\textquotesingle s supplement.

Default: \texttt{\ }{\texttt{\ auto\ }}\texttt{\ }

\includesvg[width=0.16667in,height=0.16667in]{/assets/icons/16-arrow-right.svg}
View example

\begin{verbatim}
#figure(
  circle(radius: 10pt),
  caption: [A curious atom.],
  kind: "atom",
  supplement: [Atom],
)
\end{verbatim}

\includegraphics[width=5in,height=\textheight,keepaspectratio]{/assets/docs/gnEhUtPlQLC9DmHftY4vzQAAAAAAAAAA.png}

\subsubsection{\texorpdfstring{\texttt{\ supplement\ }}{ supplement }}\label{parameters-supplement}

\href{/docs/reference/foundations/none/}{none} {or}
\href{/docs/reference/foundations/auto/}{auto} {or}
\href{/docs/reference/foundations/content/}{content} {or}
\href{/docs/reference/foundations/function/}{function}

{{ Settable }}

\phantomsection\label{parameters-supplement-settable-tooltip}
Settable parameters can be customized for all following uses of the
function with a \texttt{\ set\ } rule.

The figure\textquotesingle s supplement.

If set to \texttt{\ }{\texttt{\ auto\ }}\texttt{\ } , the figure will
try to automatically determine the correct supplement based on the
\texttt{\ kind\ } and the active
\href{/docs/reference/text/text/\#parameters-lang}{text language} . If
you are using a custom figure type, you will need to manually specify
the supplement.

If a function is specified, it is passed the first descendant of the
specified \texttt{\ kind\ } (typically, the figure\textquotesingle s
body) and should return content.

Default: \texttt{\ }{\texttt{\ auto\ }}\texttt{\ }

\includesvg[width=0.16667in,height=0.16667in]{/assets/icons/16-arrow-right.svg}
View example

\begin{verbatim}
#figure(
  [The contents of my figure!],
  caption: [My custom figure],
  supplement: [Bar],
  kind: "foo",
)
\end{verbatim}

\includegraphics[width=5in,height=\textheight,keepaspectratio]{/assets/docs/_ow3s-d4xSBN6VX-nVHVzQAAAAAAAAAA.png}

\subsubsection{\texorpdfstring{\texttt{\ numbering\ }}{ numbering }}\label{parameters-numbering}

\href{/docs/reference/foundations/none/}{none} {or}
\href{/docs/reference/foundations/str/}{str} {or}
\href{/docs/reference/foundations/function/}{function}

{{ Settable }}

\phantomsection\label{parameters-numbering-settable-tooltip}
Settable parameters can be customized for all following uses of the
function with a \texttt{\ set\ } rule.

How to number the figure. Accepts a
\href{/docs/reference/model/numbering/}{numbering pattern or function} .

Default: \texttt{\ }{\texttt{\ "1"\ }}\texttt{\ }

\subsubsection{\texorpdfstring{\texttt{\ gap\ }}{ gap }}\label{parameters-gap}

\href{/docs/reference/layout/length/}{length}

{{ Settable }}

\phantomsection\label{parameters-gap-settable-tooltip}
Settable parameters can be customized for all following uses of the
function with a \texttt{\ set\ } rule.

The vertical gap between the body and caption.

Default: \texttt{\ }{\texttt{\ 0.65em\ }}\texttt{\ }

\subsubsection{\texorpdfstring{\texttt{\ outlined\ }}{ outlined }}\label{parameters-outlined}

\href{/docs/reference/foundations/bool/}{bool}

{{ Settable }}

\phantomsection\label{parameters-outlined-settable-tooltip}
Settable parameters can be customized for all following uses of the
function with a \texttt{\ set\ } rule.

Whether the figure should appear in an
\href{/docs/reference/model/outline/}{\texttt{\ outline\ }} of figures.

Default: \texttt{\ }{\texttt{\ true\ }}\texttt{\ }

\subsection{\texorpdfstring{{ Definitions
}}{ Definitions }}\label{definitions}

\phantomsection\label{definitions-tooltip}
Functions and types and can have associated definitions. These are
accessed by specifying the function or type, followed by a period, and
then the definition\textquotesingle s name.

\subsubsection{\texorpdfstring{\texttt{\ caption\ } {{ Element
}}}{ caption   Element }}\label{definitions-caption}

\phantomsection\label{definitions-caption-element-tooltip}
Element functions can be customized with \texttt{\ set\ } and
\texttt{\ show\ } rules.

The caption of a figure. This element can be used in set and show rules
to customize the appearance of captions for all figures or figures of a
specific kind.

In addition to its \texttt{\ pos\ } and \texttt{\ body\ } , the
\texttt{\ caption\ } also provides the figure\textquotesingle s
\texttt{\ kind\ } , \texttt{\ supplement\ } , \texttt{\ counter\ } , and
\texttt{\ numbering\ } as fields. These parts can be used in
\href{/docs/reference/foundations/function/\#definitions-where}{\texttt{\ where\ }}
selectors and show rules to build a completely custom caption.

figure { . } { caption } (

{ \hyperref[definitions-caption-parameters-position]{position :}
\href{/docs/reference/layout/alignment/}{alignment} , } {
\hyperref[definitions-caption-parameters-separator]{separator :}
\href{/docs/reference/foundations/auto/}{auto}
\href{/docs/reference/foundations/content/}{content} , } {
\href{/docs/reference/foundations/content/}{content} , }

) -\textgreater{} \href{/docs/reference/foundations/content/}{content}

\begin{verbatim}
#show figure.caption: emph

#figure(
  rect[Hello],
  caption: [A rectangle],
)
\end{verbatim}

\includegraphics[width=5in,height=\textheight,keepaspectratio]{/assets/docs/_9Rae3k-14dcb00bWW4ciAAAAAAAAAAA.png}

\paragraph{\texorpdfstring{\texttt{\ position\ }}{ position }}\label{definitions-caption-position}

\href{/docs/reference/layout/alignment/}{alignment}

{{ Settable }}

\phantomsection\label{definitions-caption-position-settable-tooltip}
Settable parameters can be customized for all following uses of the
function with a \texttt{\ set\ } rule.

The caption\textquotesingle s position in the figure. Either
\texttt{\ top\ } or \texttt{\ bottom\ } .

Default: \texttt{\ bottom\ }

\includesvg[width=0.16667in,height=0.16667in]{/assets/icons/16-arrow-right.svg}
View example

\begin{verbatim}
#show figure.where(
  kind: table
): set figure.caption(position: top)

#figure(
  table(columns: 2)[A][B],
  caption: [I'm up here],
)

#figure(
  rect[Hi],
  caption: [I'm down here],
)

#figure(
  table(columns: 2)[A][B],
  caption: figure.caption(
    position: bottom,
    [I'm down here too!]
  )
)
\end{verbatim}

\includegraphics[width=5in,height=\textheight,keepaspectratio]{/assets/docs/IdFKmiavSqMTEqn8wUXuUgAAAAAAAAAA.png}

\paragraph{\texorpdfstring{\texttt{\ separator\ }}{ separator }}\label{definitions-caption-separator}

\href{/docs/reference/foundations/auto/}{auto} {or}
\href{/docs/reference/foundations/content/}{content}

{{ Settable }}

\phantomsection\label{definitions-caption-separator-settable-tooltip}
Settable parameters can be customized for all following uses of the
function with a \texttt{\ set\ } rule.

The separator which will appear between the number and body.

If set to \texttt{\ }{\texttt{\ auto\ }}\texttt{\ } , the separator will
be adapted to the current
\href{/docs/reference/text/text/\#parameters-lang}{language} and
\href{/docs/reference/text/text/\#parameters-region}{region} .

Default: \texttt{\ }{\texttt{\ auto\ }}\texttt{\ }

\includesvg[width=0.16667in,height=0.16667in]{/assets/icons/16-arrow-right.svg}
View example

\begin{verbatim}
#set figure.caption(separator: [ --- ])

#figure(
  rect[Hello],
  caption: [A rectangle],
)
\end{verbatim}

\includegraphics[width=5in,height=\textheight,keepaspectratio]{/assets/docs/F47AgUphmXiVn12oCb_ECAAAAAAAAAAA.png}

\paragraph{\texorpdfstring{\texttt{\ body\ }}{ body }}\label{definitions-caption-body}

\href{/docs/reference/foundations/content/}{content}

{Required} {{ Positional }}

\phantomsection\label{definitions-caption-body-positional-tooltip}
Positional parameters are specified in order, without names.

The caption\textquotesingle s body.

Can be used alongside \texttt{\ kind\ } , \texttt{\ supplement\ } ,
\texttt{\ counter\ } , \texttt{\ numbering\ } , and
\texttt{\ location\ } to completely customize the caption.

\includesvg[width=0.16667in,height=0.16667in]{/assets/icons/16-arrow-right.svg}
View example

\begin{verbatim}
#show figure.caption: it => [
  #underline(it.body) |
  #it.supplement
  #context it.counter.display(it.numbering)
]

#figure(
  rect[Hello],
  caption: [A rectangle],
)
\end{verbatim}

\includegraphics[width=5in,height=\textheight,keepaspectratio]{/assets/docs/JxID--FAnIhAECKLMVFiVwAAAAAAAAAA.png}

\href{/docs/reference/model/emph/}{\pandocbounded{\includesvg[keepaspectratio]{/assets/icons/16-arrow-right.svg}}}

{ Emphasis } { Previous page }

\href{/docs/reference/model/footnote/}{\pandocbounded{\includesvg[keepaspectratio]{/assets/icons/16-arrow-right.svg}}}

{ Footnote } { Next page }
