\title{typst.app/docs/reference/foundations/content}

\begin{itemize}
\tightlist
\item
  \href{/docs}{\includesvg[width=0.16667in,height=0.16667in]{/assets/icons/16-docs-dark.svg}}
\item
  \includesvg[width=0.16667in,height=0.16667in]{/assets/icons/16-arrow-right.svg}
\item
  \href{/docs/reference/}{Reference}
\item
  \includesvg[width=0.16667in,height=0.16667in]{/assets/icons/16-arrow-right.svg}
\item
  \href{/docs/reference/foundations/}{Foundations}
\item
  \includesvg[width=0.16667in,height=0.16667in]{/assets/icons/16-arrow-right.svg}
\item
  \href{/docs/reference/foundations/content/}{Content}
\end{itemize}

\section{\texorpdfstring{{ content }}{ content }}\label{summary}

A piece of document content.

This type is at the heart of Typst. All markup you write and most
\href{/docs/reference/foundations/function/}{functions} you call produce
content values. You can create a content value by enclosing markup in
square brackets. This is also how you pass content to functions.

\subsection{Example}\label{example}

\begin{verbatim}
Type of *Hello!* is
#type([*Hello!*])
\end{verbatim}

\includegraphics[width=5in,height=\textheight,keepaspectratio]{/assets/docs/X4qekl24YgH3SaXf1J0tagAAAAAAAAAA.png}

Content can be added with the \texttt{\ +\ } operator,
\href{/docs/reference/scripting/\#blocks}{joined together} and
multiplied with integers. Wherever content is expected, you can also
pass a \href{/docs/reference/foundations/str/}{string} or
\texttt{\ }{\texttt{\ none\ }}\texttt{\ } .

\subsection{Representation}\label{representation}

Content consists of elements with fields. When constructing an element
with its \emph{element function,} you provide these fields as arguments
and when you have a content value, you can access its fields with
\href{/docs/reference/scripting/\#field-access}{field access syntax} .

Some fields are required: These must be provided when constructing an
element and as a consequence, they are always available through field
access on content of that type. Required fields are marked as such in
the documentation.

Most fields are optional: Like required fields, they can be passed to
the element function to configure them for a single element. However,
these can also be configured with
\href{/docs/reference/styling/\#set-rules}{set rules} to apply them to
all elements within a scope. Optional fields are only available with
field access syntax when they were explicitly passed to the element
function, not when they result from a set rule.

Each element has a default appearance. However, you can also completely
customize its appearance with a
\href{/docs/reference/styling/\#show-rules}{show rule} . The show rule
is passed the element. It can access the element\textquotesingle s field
and produce arbitrary content from it.

In the web app, you can hover over a content variable to see exactly
which elements the content is composed of and what fields they have.
Alternatively, you can inspect the output of the
\href{/docs/reference/foundations/repr/}{\texttt{\ repr\ }} function.

\subsection{\texorpdfstring{{ Definitions
}}{ Definitions }}\label{definitions}

\phantomsection\label{definitions-tooltip}
Functions and types and can have associated definitions. These are
accessed by specifying the function or type, followed by a period, and
then the definition\textquotesingle s name.

\subsubsection{\texorpdfstring{\texttt{\ func\ }}{ func }}\label{definitions-func}

The content\textquotesingle s element function. This function can be
used to create the element contained in this content. It can be used in
set and show rules for the element. Can be compared with global
functions to check whether you have a specific kind of element.

self { . } { func } (

) -\textgreater{} \href{/docs/reference/foundations/function/}{function}

\subsubsection{\texorpdfstring{\texttt{\ has\ }}{ has }}\label{definitions-has}

Whether the content has the specified field.

self { . } { has } (

{ \href{/docs/reference/foundations/str/}{str} }

) -\textgreater{} \href{/docs/reference/foundations/bool/}{bool}

\paragraph{\texorpdfstring{\texttt{\ field\ }}{ field }}\label{definitions-has-field}

\href{/docs/reference/foundations/str/}{str}

{Required} {{ Positional }}

\phantomsection\label{definitions-has-field-positional-tooltip}
Positional parameters are specified in order, without names.

The field to look for.

\subsubsection{\texorpdfstring{\texttt{\ at\ }}{ at }}\label{definitions-at}

Access the specified field on the content. Returns the default value if
the field does not exist or fails with an error if no default value was
specified.

self { . } { at } (

{ \href{/docs/reference/foundations/str/}{str} , } {
\hyperref[definitions-at-parameters-default]{default :} { any } , }

) -\textgreater{} { any }

\paragraph{\texorpdfstring{\texttt{\ field\ }}{ field }}\label{definitions-at-field}

\href{/docs/reference/foundations/str/}{str}

{Required} {{ Positional }}

\phantomsection\label{definitions-at-field-positional-tooltip}
Positional parameters are specified in order, without names.

The field to access.

\paragraph{\texorpdfstring{\texttt{\ default\ }}{ default }}\label{definitions-at-default}

{ any }

A default value to return if the field does not exist.

\subsubsection{\texorpdfstring{\texttt{\ fields\ }}{ fields }}\label{definitions-fields}

Returns the fields of this content.

self { . } { fields } (

) -\textgreater{}
\href{/docs/reference/foundations/dictionary/}{dictionary}

\begin{verbatim}
#rect(
  width: 10cm,
  height: 10cm,
).fields()
\end{verbatim}

\includegraphics[width=5in,height=\textheight,keepaspectratio]{/assets/docs/zNlYUwJ_V8GS40gGav-GlwAAAAAAAAAA.png}

\subsubsection{\texorpdfstring{\texttt{\ location\ }}{ location }}\label{definitions-location}

The location of the content. This is only available on content returned
by \href{/docs/reference/introspection/query/}{query} or provided by a
\href{/docs/reference/styling/\#show-rules}{show rule} , for other
content it will be \texttt{\ }{\texttt{\ none\ }}\texttt{\ } . The
resulting location can be used with
\href{/docs/reference/introspection/counter/}{counters} ,
\href{/docs/reference/introspection/state/}{state} and
\href{/docs/reference/introspection/query/}{queries} .

self { . } { location } (

) -\textgreater{} \href{/docs/reference/foundations/none/}{none}
\href{/docs/reference/introspection/location/}{location}

\href{/docs/reference/foundations/calc/}{\pandocbounded{\includesvg[keepaspectratio]{/assets/icons/16-arrow-right.svg}}}

{ Calculation } { Previous page }

\href{/docs/reference/foundations/datetime/}{\pandocbounded{\includesvg[keepaspectratio]{/assets/icons/16-arrow-right.svg}}}

{ Datetime } { Next page }
