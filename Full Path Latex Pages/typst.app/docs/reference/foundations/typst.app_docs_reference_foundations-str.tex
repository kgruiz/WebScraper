\title{typst.app/docs/reference/foundations/str}

\begin{itemize}
\tightlist
\item
  \href{/docs}{\includesvg[width=0.16667in,height=0.16667in]{/assets/icons/16-docs-dark.svg}}
\item
  \includesvg[width=0.16667in,height=0.16667in]{/assets/icons/16-arrow-right.svg}
\item
  \href{/docs/reference/}{Reference}
\item
  \includesvg[width=0.16667in,height=0.16667in]{/assets/icons/16-arrow-right.svg}
\item
  \href{/docs/reference/foundations/}{Foundations}
\item
  \includesvg[width=0.16667in,height=0.16667in]{/assets/icons/16-arrow-right.svg}
\item
  \href{/docs/reference/foundations/str/}{String}
\end{itemize}

\section{\texorpdfstring{{ str }}{ str }}\label{summary}

A sequence of Unicode codepoints.

You can iterate over the grapheme clusters of the string using a
\href{/docs/reference/scripting/\#loops}{for loop} . Grapheme clusters
are basically characters but keep together things that belong together,
e.g. multiple codepoints that together form a flag emoji. Strings can be
added with the \texttt{\ +\ } operator,
\href{/docs/reference/scripting/\#blocks}{joined together} and
multiplied with integers.

Typst provides utility methods for string manipulation. Many of these
methods (e.g., \texttt{\ split\ } , \texttt{\ trim\ } and
\texttt{\ replace\ } ) operate on \emph{patterns:} A pattern can be
either a string or a \href{/docs/reference/foundations/regex/}{regular
expression} . This makes the methods quite versatile.

All lengths and indices are expressed in terms of UTF-8 bytes. Indices
are zero-based and negative indices wrap around to the end of the
string.

You can convert a value to a string with this type\textquotesingle s
constructor.

\subsection{Example}\label{example}

\begin{verbatim}
#"hello world!" \
#"\"hello\n  world\"!" \
#"1 2 3".split() \
#"1,2;3".split(regex("[,;]")) \
#(regex("\d+") in "ten euros") \
#(regex("\d+") in "10 euros")
\end{verbatim}

\includegraphics[width=5in,height=\textheight,keepaspectratio]{/assets/docs/gK89AnI9k7dy82m9R3F1jgAAAAAAAAAA.png}

\subsection{Escape sequences}\label{escapes}

Just like in markup, you can escape a few symbols in strings:

\begin{itemize}
\tightlist
\item
  \texttt{\ }{\texttt{\ \textbackslash{}\textbackslash{}\ }}\texttt{\ }
  for a backslash
\item
  \texttt{\ }{\texttt{\ \textbackslash{}"\ }}\texttt{\ } for a quote
\item
  \texttt{\ }{\texttt{\ \textbackslash{}n\ }}\texttt{\ } for a newline
\item
  \texttt{\ }{\texttt{\ \textbackslash{}r\ }}\texttt{\ } for a carriage
  return
\item
  \texttt{\ }{\texttt{\ \textbackslash{}t\ }}\texttt{\ } for a tab
\item
  \texttt{\ }{\texttt{\ \textbackslash{}u\{1f600\}\ }}\texttt{\ } for a
  hexadecimal Unicode escape sequence
\end{itemize}

\subsection{\texorpdfstring{Constructor
{}}{Constructor }}\label{constructor}

\phantomsection\label{constructor-constructor-tooltip}
If a type has a constructor, you can call it like a function to create a
new value of the type.

Converts a value to a string.

\begin{itemize}
\tightlist
\item
  Integers are formatted in base 10. This can be overridden with the
  optional \texttt{\ base\ } parameter.
\item
  Floats are formatted in base 10 and never in exponential notation.
\item
  From labels the name is extracted.
\item
  Bytes are decoded as UTF-8.
\end{itemize}

If you wish to convert from and to Unicode code points, see the
\href{/docs/reference/foundations/str/\#definitions-to-unicode}{\texttt{\ to-unicode\ }}
and
\href{/docs/reference/foundations/str/\#definitions-from-unicode}{\texttt{\ from-unicode\ }}
functions.

{ str } (

{ \href{/docs/reference/foundations/int/}{int}
\href{/docs/reference/foundations/float/}{float}
\href{/docs/reference/foundations/str/}{str}
\href{/docs/reference/foundations/bytes/}{bytes}
\href{/docs/reference/foundations/label/}{label}
\href{/docs/reference/foundations/decimal/}{decimal}
\href{/docs/reference/foundations/version/}{version}
\href{/docs/reference/foundations/type/}{type} , } {
\hyperref[constructor-parameters-base]{base :}
\href{/docs/reference/foundations/int/}{int} , }

) -\textgreater{} \href{/docs/reference/foundations/str/}{str}

\begin{verbatim}
#str(10) \
#str(4000, base: 16) \
#str(2.7) \
#str(1e8) \
#str(<intro>)
\end{verbatim}

\includegraphics[width=5in,height=\textheight,keepaspectratio]{/assets/docs/06jR9z-fP-M4eu8XB2MFnAAAAAAAAAAA.png}

\paragraph{\texorpdfstring{\texttt{\ value\ }}{ value }}\label{constructor-value}

\href{/docs/reference/foundations/int/}{int} {or}
\href{/docs/reference/foundations/float/}{float} {or}
\href{/docs/reference/foundations/str/}{str} {or}
\href{/docs/reference/foundations/bytes/}{bytes} {or}
\href{/docs/reference/foundations/label/}{label} {or}
\href{/docs/reference/foundations/decimal/}{decimal} {or}
\href{/docs/reference/foundations/version/}{version} {or}
\href{/docs/reference/foundations/type/}{type}

{Required} {{ Positional }}

\phantomsection\label{constructor-value-positional-tooltip}
Positional parameters are specified in order, without names.

The value that should be converted to a string.

\paragraph{\texorpdfstring{\texttt{\ base\ }}{ base }}\label{constructor-base}

\href{/docs/reference/foundations/int/}{int}

The base (radix) to display integers in, between 2 and 36.

Default: \texttt{\ }{\texttt{\ 10\ }}\texttt{\ }

\subsection{\texorpdfstring{{ Definitions
}}{ Definitions }}\label{definitions}

\phantomsection\label{definitions-tooltip}
Functions and types and can have associated definitions. These are
accessed by specifying the function or type, followed by a period, and
then the definition\textquotesingle s name.

\subsubsection{\texorpdfstring{\texttt{\ len\ }}{ len }}\label{definitions-len}

The length of the string in UTF-8 encoded bytes.

self { . } { len } (

) -\textgreater{} \href{/docs/reference/foundations/int/}{int}

\subsubsection{\texorpdfstring{\texttt{\ first\ }}{ first }}\label{definitions-first}

Extracts the first grapheme cluster of the string. Fails with an error
if the string is empty.

self { . } { first } (

) -\textgreater{} \href{/docs/reference/foundations/str/}{str}

\subsubsection{\texorpdfstring{\texttt{\ last\ }}{ last }}\label{definitions-last}

Extracts the last grapheme cluster of the string. Fails with an error if
the string is empty.

self { . } { last } (

) -\textgreater{} \href{/docs/reference/foundations/str/}{str}

\subsubsection{\texorpdfstring{\texttt{\ at\ }}{ at }}\label{definitions-at}

Extracts the first grapheme cluster after the specified index. Returns
the default value if the index is out of bounds or fails with an error
if no default value was specified.

self { . } { at } (

{ \href{/docs/reference/foundations/int/}{int} , } {
\hyperref[definitions-at-parameters-default]{default :} { any } , }

) -\textgreater{} { any }

\paragraph{\texorpdfstring{\texttt{\ index\ }}{ index }}\label{definitions-at-index}

\href{/docs/reference/foundations/int/}{int}

{Required} {{ Positional }}

\phantomsection\label{definitions-at-index-positional-tooltip}
Positional parameters are specified in order, without names.

The byte index. If negative, indexes from the back.

\paragraph{\texorpdfstring{\texttt{\ default\ }}{ default }}\label{definitions-at-default}

{ any }

A default value to return if the index is out of bounds.

\subsubsection{\texorpdfstring{\texttt{\ slice\ }}{ slice }}\label{definitions-slice}

Extracts a substring of the string. Fails with an error if the start or
end index is out of bounds.

self { . } { slice } (

{ \href{/docs/reference/foundations/int/}{int} , } {
\href{/docs/reference/foundations/none/}{none}
\href{/docs/reference/foundations/int/}{int} , } {
\hyperref[definitions-slice-parameters-count]{count :}
\href{/docs/reference/foundations/int/}{int} , }

) -\textgreater{} \href{/docs/reference/foundations/str/}{str}

\paragraph{\texorpdfstring{\texttt{\ start\ }}{ start }}\label{definitions-slice-start}

\href{/docs/reference/foundations/int/}{int}

{Required} {{ Positional }}

\phantomsection\label{definitions-slice-start-positional-tooltip}
Positional parameters are specified in order, without names.

The start byte index (inclusive). If negative, indexes from the back.

\paragraph{\texorpdfstring{\texttt{\ end\ }}{ end }}\label{definitions-slice-end}

\href{/docs/reference/foundations/none/}{none} {or}
\href{/docs/reference/foundations/int/}{int}

{{ Positional }}

\phantomsection\label{definitions-slice-end-positional-tooltip}
Positional parameters are specified in order, without names.

The end byte index (exclusive). If omitted, the whole slice until the
end of the string is extracted. If negative, indexes from the back.

Default: \texttt{\ }{\texttt{\ none\ }}\texttt{\ }

\paragraph{\texorpdfstring{\texttt{\ count\ }}{ count }}\label{definitions-slice-count}

\href{/docs/reference/foundations/int/}{int}

The number of bytes to extract. This is equivalent to passing
\texttt{\ start\ +\ count\ } as the \texttt{\ end\ } position. Mutually
exclusive with \texttt{\ end\ } .

\subsubsection{\texorpdfstring{\texttt{\ clusters\ }}{ clusters }}\label{definitions-clusters}

Returns the grapheme clusters of the string as an array of substrings.

self { . } { clusters } (

) -\textgreater{} \href{/docs/reference/foundations/array/}{array}

\subsubsection{\texorpdfstring{\texttt{\ codepoints\ }}{ codepoints }}\label{definitions-codepoints}

Returns the Unicode codepoints of the string as an array of substrings.

self { . } { codepoints } (

) -\textgreater{} \href{/docs/reference/foundations/array/}{array}

\subsubsection{\texorpdfstring{\texttt{\ to-unicode\ }}{ to-unicode }}\label{definitions-to-unicode}

Converts a character into its corresponding code point.

str { . } { to-unicode } (

{ \href{/docs/reference/foundations/str/}{str} }

) -\textgreater{} \href{/docs/reference/foundations/int/}{int}

\begin{verbatim}
#"a".to-unicode() \
#("a\u{0300}"
   .codepoints()
   .map(str.to-unicode))
\end{verbatim}

\includegraphics[width=5in,height=\textheight,keepaspectratio]{/assets/docs/q50tz6WAJPnwtBCYWbHrIwAAAAAAAAAA.png}

\paragraph{\texorpdfstring{\texttt{\ character\ }}{ character }}\label{definitions-to-unicode-character}

\href{/docs/reference/foundations/str/}{str}

{Required} {{ Positional }}

\phantomsection\label{definitions-to-unicode-character-positional-tooltip}
Positional parameters are specified in order, without names.

The character that should be converted.

\subsubsection{\texorpdfstring{\texttt{\ from-unicode\ }}{ from-unicode }}\label{definitions-from-unicode}

Converts a unicode code point into its corresponding string.

str { . } { from-unicode } (

{ \href{/docs/reference/foundations/int/}{int} }

) -\textgreater{} \href{/docs/reference/foundations/str/}{str}

\begin{verbatim}
#str.from-unicode(97)
\end{verbatim}

\includegraphics[width=5in,height=\textheight,keepaspectratio]{/assets/docs/vNzcsGO4Zd_u-P4qNnxrDQAAAAAAAAAA.png}

\paragraph{\texorpdfstring{\texttt{\ value\ }}{ value }}\label{definitions-from-unicode-value}

\href{/docs/reference/foundations/int/}{int}

{Required} {{ Positional }}

\phantomsection\label{definitions-from-unicode-value-positional-tooltip}
Positional parameters are specified in order, without names.

The code point that should be converted.

\subsubsection{\texorpdfstring{\texttt{\ contains\ }}{ contains }}\label{definitions-contains}

Whether the string contains the specified pattern.

This method also has dedicated syntax: You can write
\texttt{\ }{\texttt{\ "bc"\ }}\texttt{\ }{\texttt{\ in\ }}\texttt{\ }{\texttt{\ "abcd"\ }}\texttt{\ }
instead of
\texttt{\ }{\texttt{\ "abcd"\ }}\texttt{\ }{\texttt{\ .\ }}\texttt{\ }{\texttt{\ contains\ }}\texttt{\ }{\texttt{\ (\ }}\texttt{\ }{\texttt{\ "bc"\ }}\texttt{\ }{\texttt{\ )\ }}\texttt{\ }
.

self { . } { contains } (

{ \href{/docs/reference/foundations/str/}{str}
\href{/docs/reference/foundations/regex/}{regex} }

) -\textgreater{} \href{/docs/reference/foundations/bool/}{bool}

\paragraph{\texorpdfstring{\texttt{\ pattern\ }}{ pattern }}\label{definitions-contains-pattern}

\href{/docs/reference/foundations/str/}{str} {or}
\href{/docs/reference/foundations/regex/}{regex}

{Required} {{ Positional }}

\phantomsection\label{definitions-contains-pattern-positional-tooltip}
Positional parameters are specified in order, without names.

The pattern to search for.

\subsubsection{\texorpdfstring{\texttt{\ starts-with\ }}{ starts-with }}\label{definitions-starts-with}

Whether the string starts with the specified pattern.

self { . } { starts-with } (

{ \href{/docs/reference/foundations/str/}{str}
\href{/docs/reference/foundations/regex/}{regex} }

) -\textgreater{} \href{/docs/reference/foundations/bool/}{bool}

\paragraph{\texorpdfstring{\texttt{\ pattern\ }}{ pattern }}\label{definitions-starts-with-pattern}

\href{/docs/reference/foundations/str/}{str} {or}
\href{/docs/reference/foundations/regex/}{regex}

{Required} {{ Positional }}

\phantomsection\label{definitions-starts-with-pattern-positional-tooltip}
Positional parameters are specified in order, without names.

The pattern the string might start with.

\subsubsection{\texorpdfstring{\texttt{\ ends-with\ }}{ ends-with }}\label{definitions-ends-with}

Whether the string ends with the specified pattern.

self { . } { ends-with } (

{ \href{/docs/reference/foundations/str/}{str}
\href{/docs/reference/foundations/regex/}{regex} }

) -\textgreater{} \href{/docs/reference/foundations/bool/}{bool}

\paragraph{\texorpdfstring{\texttt{\ pattern\ }}{ pattern }}\label{definitions-ends-with-pattern}

\href{/docs/reference/foundations/str/}{str} {or}
\href{/docs/reference/foundations/regex/}{regex}

{Required} {{ Positional }}

\phantomsection\label{definitions-ends-with-pattern-positional-tooltip}
Positional parameters are specified in order, without names.

The pattern the string might end with.

\subsubsection{\texorpdfstring{\texttt{\ find\ }}{ find }}\label{definitions-find}

Searches for the specified pattern in the string and returns the first
match as a string or \texttt{\ }{\texttt{\ none\ }}\texttt{\ } if there
is no match.

self { . } { find } (

{ \href{/docs/reference/foundations/str/}{str}
\href{/docs/reference/foundations/regex/}{regex} }

) -\textgreater{} \href{/docs/reference/foundations/none/}{none}
\href{/docs/reference/foundations/str/}{str}

\paragraph{\texorpdfstring{\texttt{\ pattern\ }}{ pattern }}\label{definitions-find-pattern}

\href{/docs/reference/foundations/str/}{str} {or}
\href{/docs/reference/foundations/regex/}{regex}

{Required} {{ Positional }}

\phantomsection\label{definitions-find-pattern-positional-tooltip}
Positional parameters are specified in order, without names.

The pattern to search for.

\subsubsection{\texorpdfstring{\texttt{\ position\ }}{ position }}\label{definitions-position}

Searches for the specified pattern in the string and returns the index
of the first match as an integer or
\texttt{\ }{\texttt{\ none\ }}\texttt{\ } if there is no match.

self { . } { position } (

{ \href{/docs/reference/foundations/str/}{str}
\href{/docs/reference/foundations/regex/}{regex} }

) -\textgreater{} \href{/docs/reference/foundations/none/}{none}
\href{/docs/reference/foundations/int/}{int}

\paragraph{\texorpdfstring{\texttt{\ pattern\ }}{ pattern }}\label{definitions-position-pattern}

\href{/docs/reference/foundations/str/}{str} {or}
\href{/docs/reference/foundations/regex/}{regex}

{Required} {{ Positional }}

\phantomsection\label{definitions-position-pattern-positional-tooltip}
Positional parameters are specified in order, without names.

The pattern to search for.

\subsubsection{\texorpdfstring{\texttt{\ match\ }}{ match }}\label{definitions-match}

Searches for the specified pattern in the string and returns a
dictionary with details about the first match or
\texttt{\ }{\texttt{\ none\ }}\texttt{\ } if there is no match.

The returned dictionary has the following keys:

\begin{itemize}
\tightlist
\item
  \texttt{\ start\ } : The start offset of the match
\item
  \texttt{\ end\ } : The end offset of the match
\item
  \texttt{\ text\ } : The text that matched.
\item
  \texttt{\ captures\ } : An array containing a string for each matched
  capturing group. The first item of the array contains the first
  matched capturing, not the whole match! This is empty unless the
  \texttt{\ pattern\ } was a regex with capturing groups.
\end{itemize}

self { . } { match } (

{ \href{/docs/reference/foundations/str/}{str}
\href{/docs/reference/foundations/regex/}{regex} }

) -\textgreater{} \href{/docs/reference/foundations/none/}{none}
\href{/docs/reference/foundations/dictionary/}{dictionary}

\paragraph{\texorpdfstring{\texttt{\ pattern\ }}{ pattern }}\label{definitions-match-pattern}

\href{/docs/reference/foundations/str/}{str} {or}
\href{/docs/reference/foundations/regex/}{regex}

{Required} {{ Positional }}

\phantomsection\label{definitions-match-pattern-positional-tooltip}
Positional parameters are specified in order, without names.

The pattern to search for.

\subsubsection{\texorpdfstring{\texttt{\ matches\ }}{ matches }}\label{definitions-matches}

Searches for the specified pattern in the string and returns an array of
dictionaries with details about all matches. For details about the
returned dictionaries, see above.

self { . } { matches } (

{ \href{/docs/reference/foundations/str/}{str}
\href{/docs/reference/foundations/regex/}{regex} }

) -\textgreater{} \href{/docs/reference/foundations/array/}{array}

\paragraph{\texorpdfstring{\texttt{\ pattern\ }}{ pattern }}\label{definitions-matches-pattern}

\href{/docs/reference/foundations/str/}{str} {or}
\href{/docs/reference/foundations/regex/}{regex}

{Required} {{ Positional }}

\phantomsection\label{definitions-matches-pattern-positional-tooltip}
Positional parameters are specified in order, without names.

The pattern to search for.

\subsubsection{\texorpdfstring{\texttt{\ replace\ }}{ replace }}\label{definitions-replace}

Replace at most \texttt{\ count\ } occurrences of the given pattern with
a replacement string or function (beginning from the start). If no count
is given, all occurrences are replaced.

self { . } { replace } (

{ \href{/docs/reference/foundations/str/}{str}
\href{/docs/reference/foundations/regex/}{regex} , } {
\href{/docs/reference/foundations/str/}{str}
\href{/docs/reference/foundations/function/}{function} , } {
\hyperref[definitions-replace-parameters-count]{count :}
\href{/docs/reference/foundations/int/}{int} , }

) -\textgreater{} \href{/docs/reference/foundations/str/}{str}

\paragraph{\texorpdfstring{\texttt{\ pattern\ }}{ pattern }}\label{definitions-replace-pattern}

\href{/docs/reference/foundations/str/}{str} {or}
\href{/docs/reference/foundations/regex/}{regex}

{Required} {{ Positional }}

\phantomsection\label{definitions-replace-pattern-positional-tooltip}
Positional parameters are specified in order, without names.

The pattern to search for.

\paragraph{\texorpdfstring{\texttt{\ replacement\ }}{ replacement }}\label{definitions-replace-replacement}

\href{/docs/reference/foundations/str/}{str} {or}
\href{/docs/reference/foundations/function/}{function}

{Required} {{ Positional }}

\phantomsection\label{definitions-replace-replacement-positional-tooltip}
Positional parameters are specified in order, without names.

The string to replace the matches with or a function that gets a
dictionary for each match and can return individual replacement strings.

\paragraph{\texorpdfstring{\texttt{\ count\ }}{ count }}\label{definitions-replace-count}

\href{/docs/reference/foundations/int/}{int}

If given, only the first \texttt{\ count\ } matches of the pattern are
placed.

\subsubsection{\texorpdfstring{\texttt{\ trim\ }}{ trim }}\label{definitions-trim}

Removes matches of a pattern from one or both sides of the string, once
or repeatedly and returns the resulting string.

self { . } { trim } (

{ \href{/docs/reference/foundations/none/}{none}
\href{/docs/reference/foundations/str/}{str}
\href{/docs/reference/foundations/regex/}{regex} , } {
\hyperref[definitions-trim-parameters-at]{at :}
\href{/docs/reference/layout/alignment/}{alignment} , } {
\hyperref[definitions-trim-parameters-repeat]{repeat :}
\href{/docs/reference/foundations/bool/}{bool} , }

) -\textgreater{} \href{/docs/reference/foundations/str/}{str}

\paragraph{\texorpdfstring{\texttt{\ pattern\ }}{ pattern }}\label{definitions-trim-pattern}

\href{/docs/reference/foundations/none/}{none} {or}
\href{/docs/reference/foundations/str/}{str} {or}
\href{/docs/reference/foundations/regex/}{regex}

{{ Positional }}

\phantomsection\label{definitions-trim-pattern-positional-tooltip}
Positional parameters are specified in order, without names.

The pattern to search for. If \texttt{\ }{\texttt{\ none\ }}\texttt{\ }
, trims white spaces.

Default: \texttt{\ }{\texttt{\ none\ }}\texttt{\ }

\paragraph{\texorpdfstring{\texttt{\ at\ }}{ at }}\label{definitions-trim-at}

\href{/docs/reference/layout/alignment/}{alignment}

Can be \texttt{\ start\ } or \texttt{\ end\ } to only trim the start or
end of the string. If omitted, both sides are trimmed.

\paragraph{\texorpdfstring{\texttt{\ repeat\ }}{ repeat }}\label{definitions-trim-repeat}

\href{/docs/reference/foundations/bool/}{bool}

Whether to repeatedly removes matches of the pattern or just once.
Defaults to \texttt{\ }{\texttt{\ true\ }}\texttt{\ } .

Default: \texttt{\ }{\texttt{\ true\ }}\texttt{\ }

\subsubsection{\texorpdfstring{\texttt{\ split\ }}{ split }}\label{definitions-split}

Splits a string at matches of a specified pattern and returns an array
of the resulting parts.

self { . } { split } (

{ \href{/docs/reference/foundations/none/}{none}
\href{/docs/reference/foundations/str/}{str}
\href{/docs/reference/foundations/regex/}{regex} }

) -\textgreater{} \href{/docs/reference/foundations/array/}{array}

\paragraph{\texorpdfstring{\texttt{\ pattern\ }}{ pattern }}\label{definitions-split-pattern}

\href{/docs/reference/foundations/none/}{none} {or}
\href{/docs/reference/foundations/str/}{str} {or}
\href{/docs/reference/foundations/regex/}{regex}

{{ Positional }}

\phantomsection\label{definitions-split-pattern-positional-tooltip}
Positional parameters are specified in order, without names.

The pattern to split at. Defaults to whitespace.

Default: \texttt{\ }{\texttt{\ none\ }}\texttt{\ }

\subsubsection{\texorpdfstring{\texttt{\ rev\ }}{ rev }}\label{definitions-rev}

Reverse the string.

self { . } { rev } (

) -\textgreater{} \href{/docs/reference/foundations/str/}{str}

\href{/docs/reference/foundations/selector/}{\pandocbounded{\includesvg[keepaspectratio]{/assets/icons/16-arrow-right.svg}}}

{ Selector } { Previous page }

\href{/docs/reference/foundations/style/}{\pandocbounded{\includesvg[keepaspectratio]{/assets/icons/16-arrow-right.svg}}}

{ Style } { Next page }
