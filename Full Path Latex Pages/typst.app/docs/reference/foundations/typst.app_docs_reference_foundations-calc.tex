\title{typst.app/docs/reference/foundations/calc}

\begin{itemize}
\tightlist
\item
  \href{/docs}{\includesvg[width=0.16667in,height=0.16667in]{/assets/icons/16-docs-dark.svg}}
\item
  \includesvg[width=0.16667in,height=0.16667in]{/assets/icons/16-arrow-right.svg}
\item
  \href{/docs/reference/}{Reference}
\item
  \includesvg[width=0.16667in,height=0.16667in]{/assets/icons/16-arrow-right.svg}
\item
  \href{/docs/reference/foundations/}{Foundations}
\item
  \includesvg[width=0.16667in,height=0.16667in]{/assets/icons/16-arrow-right.svg}
\item
  \href{/docs/reference/foundations/calc}{Calculation}
\end{itemize}

\section{Calculation}\label{summary}

Module for calculations and processing of numeric values.

These definitions are part of the \texttt{\ calc\ } module and not
imported by default. In addition to the functions listed below, the
\texttt{\ calc\ } module also defines the constants \texttt{\ pi\ } ,
\texttt{\ tau\ } , \texttt{\ e\ } , and \texttt{\ inf\ } .

\subsection{Functions}\label{functions}

\subsubsection{\texorpdfstring{\texttt{\ abs\ }}{ abs }}\label{functions-abs}

Calculates the absolute value of a numeric value.

calc { . } { abs } (

{ \href{/docs/reference/foundations/int/}{int}
\href{/docs/reference/foundations/float/}{float}
\href{/docs/reference/layout/length/}{length}
\href{/docs/reference/layout/angle/}{angle}
\href{/docs/reference/layout/ratio/}{ratio}
\href{/docs/reference/layout/fraction/}{fraction}
\href{/docs/reference/foundations/decimal/}{decimal} }

) -\textgreater{} { any }

\begin{verbatim}
#calc.abs(-5) \
#calc.abs(5pt - 2cm) \
#calc.abs(2fr) \
#calc.abs(decimal("-342.440"))
\end{verbatim}

\includegraphics[width=5in,height=\textheight,keepaspectratio]{/assets/docs/1nPNk-RAyXUEHrAszyCnUgAAAAAAAAAA.png}

\paragraph{\texorpdfstring{\texttt{\ value\ }}{ value }}\label{functions-abs-value}

\href{/docs/reference/foundations/int/}{int} {or}
\href{/docs/reference/foundations/float/}{float} {or}
\href{/docs/reference/layout/length/}{length} {or}
\href{/docs/reference/layout/angle/}{angle} {or}
\href{/docs/reference/layout/ratio/}{ratio} {or}
\href{/docs/reference/layout/fraction/}{fraction} {or}
\href{/docs/reference/foundations/decimal/}{decimal}

{Required} {{ Positional }}

\phantomsection\label{functions-abs-value-positional-tooltip}
Positional parameters are specified in order, without names.

The value whose absolute value to calculate.

\subsubsection{\texorpdfstring{\texttt{\ pow\ }}{ pow }}\label{functions-pow}

Raises a value to some exponent.

calc { . } { pow } (

{ \href{/docs/reference/foundations/int/}{int}
\href{/docs/reference/foundations/float/}{float}
\href{/docs/reference/foundations/decimal/}{decimal} , } {
\href{/docs/reference/foundations/int/}{int}
\href{/docs/reference/foundations/float/}{float} , }

) -\textgreater{} \href{/docs/reference/foundations/int/}{int}
\href{/docs/reference/foundations/float/}{float}
\href{/docs/reference/foundations/decimal/}{decimal}

\begin{verbatim}
#calc.pow(2, 3) \
#calc.pow(decimal("2.5"), 2)
\end{verbatim}

\includegraphics[width=5in,height=\textheight,keepaspectratio]{/assets/docs/YQoOsFNxPEgW0b-n9B_VrAAAAAAAAAAA.png}

\paragraph{\texorpdfstring{\texttt{\ base\ }}{ base }}\label{functions-pow-base}

\href{/docs/reference/foundations/int/}{int} {or}
\href{/docs/reference/foundations/float/}{float} {or}
\href{/docs/reference/foundations/decimal/}{decimal}

{Required} {{ Positional }}

\phantomsection\label{functions-pow-base-positional-tooltip}
Positional parameters are specified in order, without names.

The base of the power.

If this is a
\href{/docs/reference/foundations/decimal/}{\texttt{\ decimal\ }} , the
exponent can only be an \href{/docs/reference/foundations/int/}{integer}
.

\paragraph{\texorpdfstring{\texttt{\ exponent\ }}{ exponent }}\label{functions-pow-exponent}

\href{/docs/reference/foundations/int/}{int} {or}
\href{/docs/reference/foundations/float/}{float}

{Required} {{ Positional }}

\phantomsection\label{functions-pow-exponent-positional-tooltip}
Positional parameters are specified in order, without names.

The exponent of the power.

\subsubsection{\texorpdfstring{\texttt{\ exp\ }}{ exp }}\label{functions-exp}

Raises a value to some exponent of e.

calc { . } { exp } (

{ \href{/docs/reference/foundations/int/}{int}
\href{/docs/reference/foundations/float/}{float} }

) -\textgreater{} \href{/docs/reference/foundations/float/}{float}

\begin{verbatim}
#calc.exp(1)
\end{verbatim}

\includegraphics[width=5in,height=\textheight,keepaspectratio]{/assets/docs/D3jiA5mgoQIx6MVn4Oy4zwAAAAAAAAAA.png}

\paragraph{\texorpdfstring{\texttt{\ exponent\ }}{ exponent }}\label{functions-exp-exponent}

\href{/docs/reference/foundations/int/}{int} {or}
\href{/docs/reference/foundations/float/}{float}

{Required} {{ Positional }}

\phantomsection\label{functions-exp-exponent-positional-tooltip}
Positional parameters are specified in order, without names.

The exponent of the power.

\subsubsection{\texorpdfstring{\texttt{\ sqrt\ }}{ sqrt }}\label{functions-sqrt}

Calculates the square root of a number.

calc { . } { sqrt } (

{ \href{/docs/reference/foundations/int/}{int}
\href{/docs/reference/foundations/float/}{float} }

) -\textgreater{} \href{/docs/reference/foundations/float/}{float}

\begin{verbatim}
#calc.sqrt(16) \
#calc.sqrt(2.5)
\end{verbatim}

\includegraphics[width=5in,height=\textheight,keepaspectratio]{/assets/docs/rSjz1bWkkKYqxmWjxezFTwAAAAAAAAAA.png}

\paragraph{\texorpdfstring{\texttt{\ value\ }}{ value }}\label{functions-sqrt-value}

\href{/docs/reference/foundations/int/}{int} {or}
\href{/docs/reference/foundations/float/}{float}

{Required} {{ Positional }}

\phantomsection\label{functions-sqrt-value-positional-tooltip}
Positional parameters are specified in order, without names.

The number whose square root to calculate. Must be non-negative.

\subsubsection{\texorpdfstring{\texttt{\ root\ }}{ root }}\label{functions-root}

Calculates the real nth root of a number.

If the number is negative, then n must be odd.

calc { . } { root } (

{ \href{/docs/reference/foundations/float/}{float} , } {
\href{/docs/reference/foundations/int/}{int} , }

) -\textgreater{} \href{/docs/reference/foundations/float/}{float}

\begin{verbatim}
#calc.root(16.0, 4) \
#calc.root(27.0, 3)
\end{verbatim}

\includegraphics[width=5in,height=\textheight,keepaspectratio]{/assets/docs/g3rxlqoTGgoCjLtiE7bcKAAAAAAAAAAA.png}

\paragraph{\texorpdfstring{\texttt{\ radicand\ }}{ radicand }}\label{functions-root-radicand}

\href{/docs/reference/foundations/float/}{float}

{Required} {{ Positional }}

\phantomsection\label{functions-root-radicand-positional-tooltip}
Positional parameters are specified in order, without names.

The expression to take the root of

\paragraph{\texorpdfstring{\texttt{\ index\ }}{ index }}\label{functions-root-index}

\href{/docs/reference/foundations/int/}{int}

{Required} {{ Positional }}

\phantomsection\label{functions-root-index-positional-tooltip}
Positional parameters are specified in order, without names.

Which root of the radicand to take

\subsubsection{\texorpdfstring{\texttt{\ sin\ }}{ sin }}\label{functions-sin}

Calculates the sine of an angle.

When called with an integer or a float, they will be interpreted as
radians.

calc { . } { sin } (

{ \href{/docs/reference/foundations/int/}{int}
\href{/docs/reference/foundations/float/}{float}
\href{/docs/reference/layout/angle/}{angle} }

) -\textgreater{} \href{/docs/reference/foundations/float/}{float}

\begin{verbatim}
#calc.sin(1.5) \
#calc.sin(90deg)
\end{verbatim}

\includegraphics[width=5in,height=\textheight,keepaspectratio]{/assets/docs/DRz-f64JvhHssm4WgSEL2QAAAAAAAAAA.png}

\paragraph{\texorpdfstring{\texttt{\ angle\ }}{ angle }}\label{functions-sin-angle}

\href{/docs/reference/foundations/int/}{int} {or}
\href{/docs/reference/foundations/float/}{float} {or}
\href{/docs/reference/layout/angle/}{angle}

{Required} {{ Positional }}

\phantomsection\label{functions-sin-angle-positional-tooltip}
Positional parameters are specified in order, without names.

The angle whose sine to calculate.

\subsubsection{\texorpdfstring{\texttt{\ cos\ }}{ cos }}\label{functions-cos}

Calculates the cosine of an angle.

When called with an integer or a float, they will be interpreted as
radians.

calc { . } { cos } (

{ \href{/docs/reference/foundations/int/}{int}
\href{/docs/reference/foundations/float/}{float}
\href{/docs/reference/layout/angle/}{angle} }

) -\textgreater{} \href{/docs/reference/foundations/float/}{float}

\begin{verbatim}
#calc.cos(1.5) \
#calc.cos(90deg)
\end{verbatim}

\includegraphics[width=5in,height=\textheight,keepaspectratio]{/assets/docs/RQec6gdJF5QvRwZkvI-50gAAAAAAAAAA.png}

\paragraph{\texorpdfstring{\texttt{\ angle\ }}{ angle }}\label{functions-cos-angle}

\href{/docs/reference/foundations/int/}{int} {or}
\href{/docs/reference/foundations/float/}{float} {or}
\href{/docs/reference/layout/angle/}{angle}

{Required} {{ Positional }}

\phantomsection\label{functions-cos-angle-positional-tooltip}
Positional parameters are specified in order, without names.

The angle whose cosine to calculate.

\subsubsection{\texorpdfstring{\texttt{\ tan\ }}{ tan }}\label{functions-tan}

Calculates the tangent of an angle.

When called with an integer or a float, they will be interpreted as
radians.

calc { . } { tan } (

{ \href{/docs/reference/foundations/int/}{int}
\href{/docs/reference/foundations/float/}{float}
\href{/docs/reference/layout/angle/}{angle} }

) -\textgreater{} \href{/docs/reference/foundations/float/}{float}

\begin{verbatim}
#calc.tan(1.5) \
#calc.tan(90deg)
\end{verbatim}

\includegraphics[width=5in,height=\textheight,keepaspectratio]{/assets/docs/Mu6UfN_4464KJhy78wvp_wAAAAAAAAAA.png}

\paragraph{\texorpdfstring{\texttt{\ angle\ }}{ angle }}\label{functions-tan-angle}

\href{/docs/reference/foundations/int/}{int} {or}
\href{/docs/reference/foundations/float/}{float} {or}
\href{/docs/reference/layout/angle/}{angle}

{Required} {{ Positional }}

\phantomsection\label{functions-tan-angle-positional-tooltip}
Positional parameters are specified in order, without names.

The angle whose tangent to calculate.

\subsubsection{\texorpdfstring{\texttt{\ asin\ }}{ asin }}\label{functions-asin}

Calculates the arcsine of a number.

calc { . } { asin } (

{ \href{/docs/reference/foundations/int/}{int}
\href{/docs/reference/foundations/float/}{float} }

) -\textgreater{} \href{/docs/reference/layout/angle/}{angle}

\begin{verbatim}
#calc.asin(0) \
#calc.asin(1)
\end{verbatim}

\includegraphics[width=5in,height=\textheight,keepaspectratio]{/assets/docs/R-fP2bsKqek6CrHRxsmlvQAAAAAAAAAA.png}

\paragraph{\texorpdfstring{\texttt{\ value\ }}{ value }}\label{functions-asin-value}

\href{/docs/reference/foundations/int/}{int} {or}
\href{/docs/reference/foundations/float/}{float}

{Required} {{ Positional }}

\phantomsection\label{functions-asin-value-positional-tooltip}
Positional parameters are specified in order, without names.

The number whose arcsine to calculate. Must be between -1 and 1.

\subsubsection{\texorpdfstring{\texttt{\ acos\ }}{ acos }}\label{functions-acos}

Calculates the arccosine of a number.

calc { . } { acos } (

{ \href{/docs/reference/foundations/int/}{int}
\href{/docs/reference/foundations/float/}{float} }

) -\textgreater{} \href{/docs/reference/layout/angle/}{angle}

\begin{verbatim}
#calc.acos(0) \
#calc.acos(1)
\end{verbatim}

\includegraphics[width=5in,height=\textheight,keepaspectratio]{/assets/docs/34tvtgPRx9Zb0oFQtdkEngAAAAAAAAAA.png}

\paragraph{\texorpdfstring{\texttt{\ value\ }}{ value }}\label{functions-acos-value}

\href{/docs/reference/foundations/int/}{int} {or}
\href{/docs/reference/foundations/float/}{float}

{Required} {{ Positional }}

\phantomsection\label{functions-acos-value-positional-tooltip}
Positional parameters are specified in order, without names.

The number whose arcsine to calculate. Must be between -1 and 1.

\subsubsection{\texorpdfstring{\texttt{\ atan\ }}{ atan }}\label{functions-atan}

Calculates the arctangent of a number.

calc { . } { atan } (

{ \href{/docs/reference/foundations/int/}{int}
\href{/docs/reference/foundations/float/}{float} }

) -\textgreater{} \href{/docs/reference/layout/angle/}{angle}

\begin{verbatim}
#calc.atan(0) \
#calc.atan(1)
\end{verbatim}

\includegraphics[width=5in,height=\textheight,keepaspectratio]{/assets/docs/Ks5iB4MwWXNAVXeSMihDJAAAAAAAAAAA.png}

\paragraph{\texorpdfstring{\texttt{\ value\ }}{ value }}\label{functions-atan-value}

\href{/docs/reference/foundations/int/}{int} {or}
\href{/docs/reference/foundations/float/}{float}

{Required} {{ Positional }}

\phantomsection\label{functions-atan-value-positional-tooltip}
Positional parameters are specified in order, without names.

The number whose arctangent to calculate.

\subsubsection{\texorpdfstring{\texttt{\ atan2\ }}{ atan2 }}\label{functions-atan2}

Calculates the four-quadrant arctangent of a coordinate.

The arguments are \texttt{\ (x,\ y)\ } , not \texttt{\ (y,\ x)\ } .

calc { . } { atan2 } (

{ \href{/docs/reference/foundations/int/}{int}
\href{/docs/reference/foundations/float/}{float} , } {
\href{/docs/reference/foundations/int/}{int}
\href{/docs/reference/foundations/float/}{float} , }

) -\textgreater{} \href{/docs/reference/layout/angle/}{angle}

\begin{verbatim}
#calc.atan2(1, 1) \
#calc.atan2(-2, -3)
\end{verbatim}

\includegraphics[width=5in,height=\textheight,keepaspectratio]{/assets/docs/R3PgftYITRsSLYBeQqKe3wAAAAAAAAAA.png}

\paragraph{\texorpdfstring{\texttt{\ x\ }}{ x }}\label{functions-atan2-x}

\href{/docs/reference/foundations/int/}{int} {or}
\href{/docs/reference/foundations/float/}{float}

{Required} {{ Positional }}

\phantomsection\label{functions-atan2-x-positional-tooltip}
Positional parameters are specified in order, without names.

The X coordinate.

\paragraph{\texorpdfstring{\texttt{\ y\ }}{ y }}\label{functions-atan2-y}

\href{/docs/reference/foundations/int/}{int} {or}
\href{/docs/reference/foundations/float/}{float}

{Required} {{ Positional }}

\phantomsection\label{functions-atan2-y-positional-tooltip}
Positional parameters are specified in order, without names.

The Y coordinate.

\subsubsection{\texorpdfstring{\texttt{\ sinh\ }}{ sinh }}\label{functions-sinh}

Calculates the hyperbolic sine of a hyperbolic angle.

calc { . } { sinh } (

{ \href{/docs/reference/foundations/float/}{float} }

) -\textgreater{} \href{/docs/reference/foundations/float/}{float}

\begin{verbatim}
#calc.sinh(0) \
#calc.sinh(1.5)
\end{verbatim}

\includegraphics[width=5in,height=\textheight,keepaspectratio]{/assets/docs/Si7LVr220y-yjr6frD5mYQAAAAAAAAAA.png}

\paragraph{\texorpdfstring{\texttt{\ value\ }}{ value }}\label{functions-sinh-value}

\href{/docs/reference/foundations/float/}{float}

{Required} {{ Positional }}

\phantomsection\label{functions-sinh-value-positional-tooltip}
Positional parameters are specified in order, without names.

The hyperbolic angle whose hyperbolic sine to calculate.

\subsubsection{\texorpdfstring{\texttt{\ cosh\ }}{ cosh }}\label{functions-cosh}

Calculates the hyperbolic cosine of a hyperbolic angle.

calc { . } { cosh } (

{ \href{/docs/reference/foundations/float/}{float} }

) -\textgreater{} \href{/docs/reference/foundations/float/}{float}

\begin{verbatim}
#calc.cosh(0) \
#calc.cosh(1.5)
\end{verbatim}

\includegraphics[width=5in,height=\textheight,keepaspectratio]{/assets/docs/Vut_ujHW8enJAdOI95v6bgAAAAAAAAAA.png}

\paragraph{\texorpdfstring{\texttt{\ value\ }}{ value }}\label{functions-cosh-value}

\href{/docs/reference/foundations/float/}{float}

{Required} {{ Positional }}

\phantomsection\label{functions-cosh-value-positional-tooltip}
Positional parameters are specified in order, without names.

The hyperbolic angle whose hyperbolic cosine to calculate.

\subsubsection{\texorpdfstring{\texttt{\ tanh\ }}{ tanh }}\label{functions-tanh}

Calculates the hyperbolic tangent of an hyperbolic angle.

calc { . } { tanh } (

{ \href{/docs/reference/foundations/float/}{float} }

) -\textgreater{} \href{/docs/reference/foundations/float/}{float}

\begin{verbatim}
#calc.tanh(0) \
#calc.tanh(1.5)
\end{verbatim}

\includegraphics[width=5in,height=\textheight,keepaspectratio]{/assets/docs/8omHKWMEXh9ltcsWpm4RDQAAAAAAAAAA.png}

\paragraph{\texorpdfstring{\texttt{\ value\ }}{ value }}\label{functions-tanh-value}

\href{/docs/reference/foundations/float/}{float}

{Required} {{ Positional }}

\phantomsection\label{functions-tanh-value-positional-tooltip}
Positional parameters are specified in order, without names.

The hyperbolic angle whose hyperbolic tangent to calculate.

\subsubsection{\texorpdfstring{\texttt{\ log\ }}{ log }}\label{functions-log}

Calculates the logarithm of a number.

If the base is not specified, the logarithm is calculated in base 10.

calc { . } { log } (

{ \href{/docs/reference/foundations/int/}{int}
\href{/docs/reference/foundations/float/}{float} , } {
\hyperref[functions-log-parameters-base]{base :}
\href{/docs/reference/foundations/float/}{float} , }

) -\textgreater{} \href{/docs/reference/foundations/float/}{float}

\begin{verbatim}
#calc.log(100)
\end{verbatim}

\includegraphics[width=5in,height=\textheight,keepaspectratio]{/assets/docs/4te-fP3EFYf9CFfXTNeLbgAAAAAAAAAA.png}

\paragraph{\texorpdfstring{\texttt{\ value\ }}{ value }}\label{functions-log-value}

\href{/docs/reference/foundations/int/}{int} {or}
\href{/docs/reference/foundations/float/}{float}

{Required} {{ Positional }}

\phantomsection\label{functions-log-value-positional-tooltip}
Positional parameters are specified in order, without names.

The number whose logarithm to calculate. Must be strictly positive.

\paragraph{\texorpdfstring{\texttt{\ base\ }}{ base }}\label{functions-log-base}

\href{/docs/reference/foundations/float/}{float}

The base of the logarithm. May not be zero.

Default: \texttt{\ }{\texttt{\ 10.0\ }}\texttt{\ }

\subsubsection{\texorpdfstring{\texttt{\ ln\ }}{ ln }}\label{functions-ln}

Calculates the natural logarithm of a number.

calc { . } { ln } (

{ \href{/docs/reference/foundations/int/}{int}
\href{/docs/reference/foundations/float/}{float} }

) -\textgreater{} \href{/docs/reference/foundations/float/}{float}

\begin{verbatim}
#calc.ln(calc.e)
\end{verbatim}

\includegraphics[width=5in,height=\textheight,keepaspectratio]{/assets/docs/ahMgc30uVaXMdJx4f9b76gAAAAAAAAAA.png}

\paragraph{\texorpdfstring{\texttt{\ value\ }}{ value }}\label{functions-ln-value}

\href{/docs/reference/foundations/int/}{int} {or}
\href{/docs/reference/foundations/float/}{float}

{Required} {{ Positional }}

\phantomsection\label{functions-ln-value-positional-tooltip}
Positional parameters are specified in order, without names.

The number whose logarithm to calculate. Must be strictly positive.

\subsubsection{\texorpdfstring{\texttt{\ fact\ }}{ fact }}\label{functions-fact}

Calculates the factorial of a number.

calc { . } { fact } (

{ \href{/docs/reference/foundations/int/}{int} }

) -\textgreater{} \href{/docs/reference/foundations/int/}{int}

\begin{verbatim}
#calc.fact(5)
\end{verbatim}

\includegraphics[width=5in,height=\textheight,keepaspectratio]{/assets/docs/Hx0vydXttNRUJbbdDSvGlwAAAAAAAAAA.png}

\paragraph{\texorpdfstring{\texttt{\ number\ }}{ number }}\label{functions-fact-number}

\href{/docs/reference/foundations/int/}{int}

{Required} {{ Positional }}

\phantomsection\label{functions-fact-number-positional-tooltip}
Positional parameters are specified in order, without names.

The number whose factorial to calculate. Must be non-negative.

\subsubsection{\texorpdfstring{\texttt{\ perm\ }}{ perm }}\label{functions-perm}

Calculates a permutation.

Returns the \texttt{\ k\ } -permutation of \texttt{\ n\ } , or the
number of ways to choose \texttt{\ k\ } items from a set of
\texttt{\ n\ } with regard to order.

calc { . } { perm } (

{ \href{/docs/reference/foundations/int/}{int} , } {
\href{/docs/reference/foundations/int/}{int} , }

) -\textgreater{} \href{/docs/reference/foundations/int/}{int}

\begin{verbatim}
$ "perm"(n, k) &= n!/((n - k)!) \
  "perm"(5, 3) &= #calc.perm(5, 3) $
\end{verbatim}

\includegraphics[width=5in,height=\textheight,keepaspectratio]{/assets/docs/7mAf4sPmhe6rKKzamBE-iAAAAAAAAAAA.png}

\paragraph{\texorpdfstring{\texttt{\ base\ }}{ base }}\label{functions-perm-base}

\href{/docs/reference/foundations/int/}{int}

{Required} {{ Positional }}

\phantomsection\label{functions-perm-base-positional-tooltip}
Positional parameters are specified in order, without names.

The base number. Must be non-negative.

\paragraph{\texorpdfstring{\texttt{\ numbers\ }}{ numbers }}\label{functions-perm-numbers}

\href{/docs/reference/foundations/int/}{int}

{Required} {{ Positional }}

\phantomsection\label{functions-perm-numbers-positional-tooltip}
Positional parameters are specified in order, without names.

The number of permutations. Must be non-negative.

\subsubsection{\texorpdfstring{\texttt{\ binom\ }}{ binom }}\label{functions-binom}

Calculates a binomial coefficient.

Returns the \texttt{\ k\ } -combination of \texttt{\ n\ } , or the
number of ways to choose \texttt{\ k\ } items from a set of
\texttt{\ n\ } without regard to order.

calc { . } { binom } (

{ \href{/docs/reference/foundations/int/}{int} , } {
\href{/docs/reference/foundations/int/}{int} , }

) -\textgreater{} \href{/docs/reference/foundations/int/}{int}

\begin{verbatim}
#calc.binom(10, 5)
\end{verbatim}

\includegraphics[width=5in,height=\textheight,keepaspectratio]{/assets/docs/3evQc1ME4eQqbzXhrmJ5lAAAAAAAAAAA.png}

\paragraph{\texorpdfstring{\texttt{\ n\ }}{ n }}\label{functions-binom-n}

\href{/docs/reference/foundations/int/}{int}

{Required} {{ Positional }}

\phantomsection\label{functions-binom-n-positional-tooltip}
Positional parameters are specified in order, without names.

The upper coefficient. Must be non-negative.

\paragraph{\texorpdfstring{\texttt{\ k\ }}{ k }}\label{functions-binom-k}

\href{/docs/reference/foundations/int/}{int}

{Required} {{ Positional }}

\phantomsection\label{functions-binom-k-positional-tooltip}
Positional parameters are specified in order, without names.

The lower coefficient. Must be non-negative.

\subsubsection{\texorpdfstring{\texttt{\ gcd\ }}{ gcd }}\label{functions-gcd}

Calculates the greatest common divisor of two integers.

calc { . } { gcd } (

{ \href{/docs/reference/foundations/int/}{int} , } {
\href{/docs/reference/foundations/int/}{int} , }

) -\textgreater{} \href{/docs/reference/foundations/int/}{int}

\begin{verbatim}
#calc.gcd(7, 42)
\end{verbatim}

\includegraphics[width=5in,height=\textheight,keepaspectratio]{/assets/docs/qOIwQyXpCnrSkONAOnIDgAAAAAAAAAAA.png}

\paragraph{\texorpdfstring{\texttt{\ a\ }}{ a }}\label{functions-gcd-a}

\href{/docs/reference/foundations/int/}{int}

{Required} {{ Positional }}

\phantomsection\label{functions-gcd-a-positional-tooltip}
Positional parameters are specified in order, without names.

The first integer.

\paragraph{\texorpdfstring{\texttt{\ b\ }}{ b }}\label{functions-gcd-b}

\href{/docs/reference/foundations/int/}{int}

{Required} {{ Positional }}

\phantomsection\label{functions-gcd-b-positional-tooltip}
Positional parameters are specified in order, without names.

The second integer.

\subsubsection{\texorpdfstring{\texttt{\ lcm\ }}{ lcm }}\label{functions-lcm}

Calculates the least common multiple of two integers.

calc { . } { lcm } (

{ \href{/docs/reference/foundations/int/}{int} , } {
\href{/docs/reference/foundations/int/}{int} , }

) -\textgreater{} \href{/docs/reference/foundations/int/}{int}

\begin{verbatim}
#calc.lcm(96, 13)
\end{verbatim}

\includegraphics[width=5in,height=\textheight,keepaspectratio]{/assets/docs/BsSZQG52_995RG9zgRumuAAAAAAAAAAA.png}

\paragraph{\texorpdfstring{\texttt{\ a\ }}{ a }}\label{functions-lcm-a}

\href{/docs/reference/foundations/int/}{int}

{Required} {{ Positional }}

\phantomsection\label{functions-lcm-a-positional-tooltip}
Positional parameters are specified in order, without names.

The first integer.

\paragraph{\texorpdfstring{\texttt{\ b\ }}{ b }}\label{functions-lcm-b}

\href{/docs/reference/foundations/int/}{int}

{Required} {{ Positional }}

\phantomsection\label{functions-lcm-b-positional-tooltip}
Positional parameters are specified in order, without names.

The second integer.

\subsubsection{\texorpdfstring{\texttt{\ floor\ }}{ floor }}\label{functions-floor}

Rounds a number down to the nearest integer.

If the number is already an integer, it is returned unchanged.

Note that this function will always return an
\href{/docs/reference/foundations/int/}{integer} , and will error if the
resulting \href{/docs/reference/foundations/float/}{\texttt{\ float\ }}
or \href{/docs/reference/foundations/decimal/}{\texttt{\ decimal\ }} is
larger than the maximum 64-bit signed integer or smaller than the
minimum for that type.

calc { . } { floor } (

{ \href{/docs/reference/foundations/int/}{int}
\href{/docs/reference/foundations/float/}{float}
\href{/docs/reference/foundations/decimal/}{decimal} }

) -\textgreater{} \href{/docs/reference/foundations/int/}{int}

\begin{verbatim}
#calc.floor(500.1)
#assert(calc.floor(3) == 3)
#assert(calc.floor(3.14) == 3)
#assert(calc.floor(decimal("-3.14")) == -4)
\end{verbatim}

\includegraphics[width=5in,height=\textheight,keepaspectratio]{/assets/docs/3pMWbIkij09wRgebD43VQgAAAAAAAAAA.png}

\paragraph{\texorpdfstring{\texttt{\ value\ }}{ value }}\label{functions-floor-value}

\href{/docs/reference/foundations/int/}{int} {or}
\href{/docs/reference/foundations/float/}{float} {or}
\href{/docs/reference/foundations/decimal/}{decimal}

{Required} {{ Positional }}

\phantomsection\label{functions-floor-value-positional-tooltip}
Positional parameters are specified in order, without names.

The number to round down.

\subsubsection{\texorpdfstring{\texttt{\ ceil\ }}{ ceil }}\label{functions-ceil}

Rounds a number up to the nearest integer.

If the number is already an integer, it is returned unchanged.

Note that this function will always return an
\href{/docs/reference/foundations/int/}{integer} , and will error if the
resulting \href{/docs/reference/foundations/float/}{\texttt{\ float\ }}
or \href{/docs/reference/foundations/decimal/}{\texttt{\ decimal\ }} is
larger than the maximum 64-bit signed integer or smaller than the
minimum for that type.

calc { . } { ceil } (

{ \href{/docs/reference/foundations/int/}{int}
\href{/docs/reference/foundations/float/}{float}
\href{/docs/reference/foundations/decimal/}{decimal} }

) -\textgreater{} \href{/docs/reference/foundations/int/}{int}

\begin{verbatim}
#calc.ceil(500.1)
#assert(calc.ceil(3) == 3)
#assert(calc.ceil(3.14) == 4)
#assert(calc.ceil(decimal("-3.14")) == -3)
\end{verbatim}

\includegraphics[width=5in,height=\textheight,keepaspectratio]{/assets/docs/XVF6AbxDnXwmraGN-Eh1MgAAAAAAAAAA.png}

\paragraph{\texorpdfstring{\texttt{\ value\ }}{ value }}\label{functions-ceil-value}

\href{/docs/reference/foundations/int/}{int} {or}
\href{/docs/reference/foundations/float/}{float} {or}
\href{/docs/reference/foundations/decimal/}{decimal}

{Required} {{ Positional }}

\phantomsection\label{functions-ceil-value-positional-tooltip}
Positional parameters are specified in order, without names.

The number to round up.

\subsubsection{\texorpdfstring{\texttt{\ trunc\ }}{ trunc }}\label{functions-trunc}

Returns the integer part of a number.

If the number is already an integer, it is returned unchanged.

Note that this function will always return an
\href{/docs/reference/foundations/int/}{integer} , and will error if the
resulting \href{/docs/reference/foundations/float/}{\texttt{\ float\ }}
or \href{/docs/reference/foundations/decimal/}{\texttt{\ decimal\ }} is
larger than the maximum 64-bit signed integer or smaller than the
minimum for that type.

calc { . } { trunc } (

{ \href{/docs/reference/foundations/int/}{int}
\href{/docs/reference/foundations/float/}{float}
\href{/docs/reference/foundations/decimal/}{decimal} }

) -\textgreater{} \href{/docs/reference/foundations/int/}{int}

\begin{verbatim}
#calc.trunc(15.9)
#assert(calc.trunc(3) == 3)
#assert(calc.trunc(-3.7) == -3)
#assert(calc.trunc(decimal("8493.12949582390")) == 8493)
\end{verbatim}

\includegraphics[width=5in,height=\textheight,keepaspectratio]{/assets/docs/0ASdokWmhACxp3cbdBzSiwAAAAAAAAAA.png}

\paragraph{\texorpdfstring{\texttt{\ value\ }}{ value }}\label{functions-trunc-value}

\href{/docs/reference/foundations/int/}{int} {or}
\href{/docs/reference/foundations/float/}{float} {or}
\href{/docs/reference/foundations/decimal/}{decimal}

{Required} {{ Positional }}

\phantomsection\label{functions-trunc-value-positional-tooltip}
Positional parameters are specified in order, without names.

The number to truncate.

\subsubsection{\texorpdfstring{\texttt{\ fract\ }}{ fract }}\label{functions-fract}

Returns the fractional part of a number.

If the number is an integer, returns \texttt{\ 0\ } .

calc { . } { fract } (

{ \href{/docs/reference/foundations/int/}{int}
\href{/docs/reference/foundations/float/}{float}
\href{/docs/reference/foundations/decimal/}{decimal} }

) -\textgreater{} \href{/docs/reference/foundations/int/}{int}
\href{/docs/reference/foundations/float/}{float}
\href{/docs/reference/foundations/decimal/}{decimal}

\begin{verbatim}
#calc.fract(-3.1)
#assert(calc.fract(3) == 0)
#assert(calc.fract(decimal("234.23949211")) == decimal("0.23949211"))
\end{verbatim}

\includegraphics[width=5in,height=\textheight,keepaspectratio]{/assets/docs/3TGIWh2MEGFIDAB8C1nEQAAAAAAAAAAA.png}

\paragraph{\texorpdfstring{\texttt{\ value\ }}{ value }}\label{functions-fract-value}

\href{/docs/reference/foundations/int/}{int} {or}
\href{/docs/reference/foundations/float/}{float} {or}
\href{/docs/reference/foundations/decimal/}{decimal}

{Required} {{ Positional }}

\phantomsection\label{functions-fract-value-positional-tooltip}
Positional parameters are specified in order, without names.

The number to truncate.

\subsubsection{\texorpdfstring{\texttt{\ round\ }}{ round }}\label{functions-round}

Rounds a number to the nearest integer away from zero.

Optionally, a number of decimal places can be specified.

If the number of digits is negative, its absolute value will indicate
the amount of significant integer digits to remove before the decimal
point.

Note that this function will return the same type as the operand. That
is, applying \texttt{\ round\ } to a
\href{/docs/reference/foundations/float/}{\texttt{\ float\ }} will
return a \texttt{\ float\ } , and to a
\href{/docs/reference/foundations/decimal/}{\texttt{\ decimal\ }} ,
another \texttt{\ decimal\ } . You may explicitly convert the output of
this function to an integer with
\href{/docs/reference/foundations/int/}{\texttt{\ int\ }} , but note
that such a conversion will error if the \texttt{\ float\ } or
\texttt{\ decimal\ } is larger than the maximum 64-bit signed integer or
smaller than the minimum integer.

In addition, this function can error if there is an attempt to round
beyond the maximum or minimum integer or \texttt{\ decimal\ } . If the
number is a \texttt{\ float\ } , such an attempt will cause
\texttt{\ float\ }{\texttt{\ .\ }}\texttt{\ inf\ } or
\texttt{\ }{\texttt{\ -\ }}\texttt{\ float\ }{\texttt{\ .\ }}\texttt{\ inf\ }
to be returned for maximum and minimum respectively.

calc { . } { round } (

{ \href{/docs/reference/foundations/int/}{int}
\href{/docs/reference/foundations/float/}{float}
\href{/docs/reference/foundations/decimal/}{decimal} , } {
\hyperref[functions-round-parameters-digits]{digits :}
\href{/docs/reference/foundations/int/}{int} , }

) -\textgreater{} \href{/docs/reference/foundations/int/}{int}
\href{/docs/reference/foundations/float/}{float}
\href{/docs/reference/foundations/decimal/}{decimal}

\begin{verbatim}
#calc.round(3.1415, digits: 2)
#assert(calc.round(3) == 3)
#assert(calc.round(3.14) == 3)
#assert(calc.round(3.5) == 4.0)
#assert(calc.round(3333.45, digits: -2) == 3300.0)
#assert(calc.round(-48953.45, digits: -3) == -49000.0)
#assert(calc.round(3333, digits: -2) == 3300)
#assert(calc.round(-48953, digits: -3) == -49000)
#assert(calc.round(decimal("-6.5")) == decimal("-7"))
#assert(calc.round(decimal("7.123456789"), digits: 6) == decimal("7.123457"))
#assert(calc.round(decimal("3333.45"), digits: -2) == decimal("3300"))
#assert(calc.round(decimal("-48953.45"), digits: -3) == decimal("-49000"))
\end{verbatim}

\includegraphics[width=5in,height=\textheight,keepaspectratio]{/assets/docs/S2fXMNcPylTq6uwl7ZRpoAAAAAAAAAAA.png}

\paragraph{\texorpdfstring{\texttt{\ value\ }}{ value }}\label{functions-round-value}

\href{/docs/reference/foundations/int/}{int} {or}
\href{/docs/reference/foundations/float/}{float} {or}
\href{/docs/reference/foundations/decimal/}{decimal}

{Required} {{ Positional }}

\phantomsection\label{functions-round-value-positional-tooltip}
Positional parameters are specified in order, without names.

The number to round.

\paragraph{\texorpdfstring{\texttt{\ digits\ }}{ digits }}\label{functions-round-digits}

\href{/docs/reference/foundations/int/}{int}

If positive, the number of decimal places.

If negative, the number of significant integer digits that should be
removed before the decimal point.

Default: \texttt{\ }{\texttt{\ 0\ }}\texttt{\ }

\subsubsection{\texorpdfstring{\texttt{\ clamp\ }}{ clamp }}\label{functions-clamp}

Clamps a number between a minimum and maximum value.

calc { . } { clamp } (

{ \href{/docs/reference/foundations/int/}{int}
\href{/docs/reference/foundations/float/}{float}
\href{/docs/reference/foundations/decimal/}{decimal} , } {
\href{/docs/reference/foundations/int/}{int}
\href{/docs/reference/foundations/float/}{float}
\href{/docs/reference/foundations/decimal/}{decimal} , } {
\href{/docs/reference/foundations/int/}{int}
\href{/docs/reference/foundations/float/}{float}
\href{/docs/reference/foundations/decimal/}{decimal} , }

) -\textgreater{} \href{/docs/reference/foundations/int/}{int}
\href{/docs/reference/foundations/float/}{float}
\href{/docs/reference/foundations/decimal/}{decimal}

\begin{verbatim}
#calc.clamp(5, 0, 4)
#assert(calc.clamp(5, 0, 10) == 5)
#assert(calc.clamp(5, 6, 10) == 6)
#assert(calc.clamp(decimal("5.45"), 2, decimal("45.9")) == decimal("5.45"))
#assert(calc.clamp(decimal("5.45"), decimal("6.75"), 12) == decimal("6.75"))
\end{verbatim}

\includegraphics[width=5in,height=\textheight,keepaspectratio]{/assets/docs/IT7doIU2fH1UJf0E_SPc6QAAAAAAAAAA.png}

\paragraph{\texorpdfstring{\texttt{\ value\ }}{ value }}\label{functions-clamp-value}

\href{/docs/reference/foundations/int/}{int} {or}
\href{/docs/reference/foundations/float/}{float} {or}
\href{/docs/reference/foundations/decimal/}{decimal}

{Required} {{ Positional }}

\phantomsection\label{functions-clamp-value-positional-tooltip}
Positional parameters are specified in order, without names.

The number to clamp.

\paragraph{\texorpdfstring{\texttt{\ min\ }}{ min }}\label{functions-clamp-min}

\href{/docs/reference/foundations/int/}{int} {or}
\href{/docs/reference/foundations/float/}{float} {or}
\href{/docs/reference/foundations/decimal/}{decimal}

{Required} {{ Positional }}

\phantomsection\label{functions-clamp-min-positional-tooltip}
Positional parameters are specified in order, without names.

The inclusive minimum value.

\paragraph{\texorpdfstring{\texttt{\ max\ }}{ max }}\label{functions-clamp-max}

\href{/docs/reference/foundations/int/}{int} {or}
\href{/docs/reference/foundations/float/}{float} {or}
\href{/docs/reference/foundations/decimal/}{decimal}

{Required} {{ Positional }}

\phantomsection\label{functions-clamp-max-positional-tooltip}
Positional parameters are specified in order, without names.

The inclusive maximum value.

\subsubsection{\texorpdfstring{\texttt{\ min\ }}{ min }}\label{functions-min}

Determines the minimum of a sequence of values.

calc { . } { min } (

{ \hyperref[functions-min-parameters-values]{..} { any } }

) -\textgreater{} { any }

\begin{verbatim}
#calc.min(1, -3, -5, 20, 3, 6) \
#calc.min("typst", "is", "cool")
\end{verbatim}

\includegraphics[width=5in,height=\textheight,keepaspectratio]{/assets/docs/afOSrjdOAc_1RzzU2hxUIgAAAAAAAAAA.png}

\paragraph{\texorpdfstring{\texttt{\ values\ }}{ values }}\label{functions-min-values}

{ any }

{Required} {{ Positional }}

\phantomsection\label{functions-min-values-positional-tooltip}
Positional parameters are specified in order, without names.

{{ Variadic }}

\phantomsection\label{functions-min-values-variadic-tooltip}
Variadic parameters can be specified multiple times.

The sequence of values from which to extract the minimum. Must not be
empty.

\subsubsection{\texorpdfstring{\texttt{\ max\ }}{ max }}\label{functions-max}

Determines the maximum of a sequence of values.

calc { . } { max } (

{ \hyperref[functions-max-parameters-values]{..} { any } }

) -\textgreater{} { any }

\begin{verbatim}
#calc.max(1, -3, -5, 20, 3, 6) \
#calc.max("typst", "is", "cool")
\end{verbatim}

\includegraphics[width=5in,height=\textheight,keepaspectratio]{/assets/docs/B8vbsVaOK7Ilt-aRhfDiFwAAAAAAAAAA.png}

\paragraph{\texorpdfstring{\texttt{\ values\ }}{ values }}\label{functions-max-values}

{ any }

{Required} {{ Positional }}

\phantomsection\label{functions-max-values-positional-tooltip}
Positional parameters are specified in order, without names.

{{ Variadic }}

\phantomsection\label{functions-max-values-variadic-tooltip}
Variadic parameters can be specified multiple times.

The sequence of values from which to extract the maximum. Must not be
empty.

\subsubsection{\texorpdfstring{\texttt{\ even\ }}{ even }}\label{functions-even}

Determines whether an integer is even.

calc { . } { even } (

{ \href{/docs/reference/foundations/int/}{int} }

) -\textgreater{} \href{/docs/reference/foundations/bool/}{bool}

\begin{verbatim}
#calc.even(4) \
#calc.even(5) \
#range(10).filter(calc.even)
\end{verbatim}

\includegraphics[width=5in,height=\textheight,keepaspectratio]{/assets/docs/YVF-q96_WeIoAbwweursAAAAAAAAAAAA.png}

\paragraph{\texorpdfstring{\texttt{\ value\ }}{ value }}\label{functions-even-value}

\href{/docs/reference/foundations/int/}{int}

{Required} {{ Positional }}

\phantomsection\label{functions-even-value-positional-tooltip}
Positional parameters are specified in order, without names.

The number to check for evenness.

\subsubsection{\texorpdfstring{\texttt{\ odd\ }}{ odd }}\label{functions-odd}

Determines whether an integer is odd.

calc { . } { odd } (

{ \href{/docs/reference/foundations/int/}{int} }

) -\textgreater{} \href{/docs/reference/foundations/bool/}{bool}

\begin{verbatim}
#calc.odd(4) \
#calc.odd(5) \
#range(10).filter(calc.odd)
\end{verbatim}

\includegraphics[width=5in,height=\textheight,keepaspectratio]{/assets/docs/54xiVFQnQ9FIdgInF0A_jAAAAAAAAAAA.png}

\paragraph{\texorpdfstring{\texttt{\ value\ }}{ value }}\label{functions-odd-value}

\href{/docs/reference/foundations/int/}{int}

{Required} {{ Positional }}

\phantomsection\label{functions-odd-value-positional-tooltip}
Positional parameters are specified in order, without names.

The number to check for oddness.

\subsubsection{\texorpdfstring{\texttt{\ rem\ }}{ rem }}\label{functions-rem}

Calculates the remainder of two numbers.

The value \texttt{\ calc.rem(x,\ y)\ } always has the same sign as
\texttt{\ x\ } , and is smaller in magnitude than \texttt{\ y\ } .

This can error if given a
\href{/docs/reference/foundations/decimal/}{\texttt{\ decimal\ }} input
and the dividend is too small in magnitude compared to the divisor.

calc { . } { rem } (

{ \href{/docs/reference/foundations/int/}{int}
\href{/docs/reference/foundations/float/}{float}
\href{/docs/reference/foundations/decimal/}{decimal} , } {
\href{/docs/reference/foundations/int/}{int}
\href{/docs/reference/foundations/float/}{float}
\href{/docs/reference/foundations/decimal/}{decimal} , }

) -\textgreater{} \href{/docs/reference/foundations/int/}{int}
\href{/docs/reference/foundations/float/}{float}
\href{/docs/reference/foundations/decimal/}{decimal}

\begin{verbatim}
#calc.rem(7, 3) \
#calc.rem(7, -3) \
#calc.rem(-7, 3) \
#calc.rem(-7, -3) \
#calc.rem(1.75, 0.5)
\end{verbatim}

\includegraphics[width=5in,height=\textheight,keepaspectratio]{/assets/docs/h9kAd8BZ_4qaZUm7WWIpgQAAAAAAAAAA.png}

\paragraph{\texorpdfstring{\texttt{\ dividend\ }}{ dividend }}\label{functions-rem-dividend}

\href{/docs/reference/foundations/int/}{int} {or}
\href{/docs/reference/foundations/float/}{float} {or}
\href{/docs/reference/foundations/decimal/}{decimal}

{Required} {{ Positional }}

\phantomsection\label{functions-rem-dividend-positional-tooltip}
Positional parameters are specified in order, without names.

The dividend of the remainder.

\paragraph{\texorpdfstring{\texttt{\ divisor\ }}{ divisor }}\label{functions-rem-divisor}

\href{/docs/reference/foundations/int/}{int} {or}
\href{/docs/reference/foundations/float/}{float} {or}
\href{/docs/reference/foundations/decimal/}{decimal}

{Required} {{ Positional }}

\phantomsection\label{functions-rem-divisor-positional-tooltip}
Positional parameters are specified in order, without names.

The divisor of the remainder.

\subsubsection{\texorpdfstring{\texttt{\ div-euclid\ }}{ div-euclid }}\label{functions-div-euclid}

Performs euclidean division of two numbers.

The result of this computation is that of a division rounded to the
integer \texttt{\ n\ } such that the dividend is greater than or equal
to \texttt{\ n\ } times the divisor.

calc { . } { div-euclid } (

{ \href{/docs/reference/foundations/int/}{int}
\href{/docs/reference/foundations/float/}{float}
\href{/docs/reference/foundations/decimal/}{decimal} , } {
\href{/docs/reference/foundations/int/}{int}
\href{/docs/reference/foundations/float/}{float}
\href{/docs/reference/foundations/decimal/}{decimal} , }

) -\textgreater{} \href{/docs/reference/foundations/int/}{int}
\href{/docs/reference/foundations/float/}{float}
\href{/docs/reference/foundations/decimal/}{decimal}

\begin{verbatim}
#calc.div-euclid(7, 3) \
#calc.div-euclid(7, -3) \
#calc.div-euclid(-7, 3) \
#calc.div-euclid(-7, -3) \
#calc.div-euclid(1.75, 0.5) \
#calc.div-euclid(decimal("1.75"), decimal("0.5"))
\end{verbatim}

\includegraphics[width=5in,height=\textheight,keepaspectratio]{/assets/docs/496IGVvoarlmERajiTFs_gAAAAAAAAAA.png}

\paragraph{\texorpdfstring{\texttt{\ dividend\ }}{ dividend }}\label{functions-div-euclid-dividend}

\href{/docs/reference/foundations/int/}{int} {or}
\href{/docs/reference/foundations/float/}{float} {or}
\href{/docs/reference/foundations/decimal/}{decimal}

{Required} {{ Positional }}

\phantomsection\label{functions-div-euclid-dividend-positional-tooltip}
Positional parameters are specified in order, without names.

The dividend of the division.

\paragraph{\texorpdfstring{\texttt{\ divisor\ }}{ divisor }}\label{functions-div-euclid-divisor}

\href{/docs/reference/foundations/int/}{int} {or}
\href{/docs/reference/foundations/float/}{float} {or}
\href{/docs/reference/foundations/decimal/}{decimal}

{Required} {{ Positional }}

\phantomsection\label{functions-div-euclid-divisor-positional-tooltip}
Positional parameters are specified in order, without names.

The divisor of the division.

\subsubsection{\texorpdfstring{\texttt{\ rem-euclid\ }}{ rem-euclid }}\label{functions-rem-euclid}

This calculates the least nonnegative remainder of a division.

Warning: Due to a floating point round-off error, the remainder may
equal the absolute value of the divisor if the dividend is much smaller
in magnitude than the divisor and the dividend is negative. This only
applies for floating point inputs.

In addition, this can error if given a
\href{/docs/reference/foundations/decimal/}{\texttt{\ decimal\ }} input
and the dividend is too small in magnitude compared to the divisor.

calc { . } { rem-euclid } (

{ \href{/docs/reference/foundations/int/}{int}
\href{/docs/reference/foundations/float/}{float}
\href{/docs/reference/foundations/decimal/}{decimal} , } {
\href{/docs/reference/foundations/int/}{int}
\href{/docs/reference/foundations/float/}{float}
\href{/docs/reference/foundations/decimal/}{decimal} , }

) -\textgreater{} \href{/docs/reference/foundations/int/}{int}
\href{/docs/reference/foundations/float/}{float}
\href{/docs/reference/foundations/decimal/}{decimal}

\begin{verbatim}
#calc.rem-euclid(7, 3) \
#calc.rem-euclid(7, -3) \
#calc.rem-euclid(-7, 3) \
#calc.rem-euclid(-7, -3) \
#calc.rem-euclid(1.75, 0.5) \
#calc.rem-euclid(decimal("1.75"), decimal("0.5"))
\end{verbatim}

\includegraphics[width=5in,height=\textheight,keepaspectratio]{/assets/docs/ysX2HLC-rfWACinwigYcWgAAAAAAAAAA.png}

\paragraph{\texorpdfstring{\texttt{\ dividend\ }}{ dividend }}\label{functions-rem-euclid-dividend}

\href{/docs/reference/foundations/int/}{int} {or}
\href{/docs/reference/foundations/float/}{float} {or}
\href{/docs/reference/foundations/decimal/}{decimal}

{Required} {{ Positional }}

\phantomsection\label{functions-rem-euclid-dividend-positional-tooltip}
Positional parameters are specified in order, without names.

The dividend of the remainder.

\paragraph{\texorpdfstring{\texttt{\ divisor\ }}{ divisor }}\label{functions-rem-euclid-divisor}

\href{/docs/reference/foundations/int/}{int} {or}
\href{/docs/reference/foundations/float/}{float} {or}
\href{/docs/reference/foundations/decimal/}{decimal}

{Required} {{ Positional }}

\phantomsection\label{functions-rem-euclid-divisor-positional-tooltip}
Positional parameters are specified in order, without names.

The divisor of the remainder.

\subsubsection{\texorpdfstring{\texttt{\ quo\ }}{ quo }}\label{functions-quo}

Calculates the quotient (floored division) of two numbers.

Note that this function will always return an
\href{/docs/reference/foundations/int/}{integer} , and will error if the
resulting \href{/docs/reference/foundations/float/}{\texttt{\ float\ }}
or \href{/docs/reference/foundations/decimal/}{\texttt{\ decimal\ }} is
larger than the maximum 64-bit signed integer or smaller than the
minimum for that type.

calc { . } { quo } (

{ \href{/docs/reference/foundations/int/}{int}
\href{/docs/reference/foundations/float/}{float}
\href{/docs/reference/foundations/decimal/}{decimal} , } {
\href{/docs/reference/foundations/int/}{int}
\href{/docs/reference/foundations/float/}{float}
\href{/docs/reference/foundations/decimal/}{decimal} , }

) -\textgreater{} \href{/docs/reference/foundations/int/}{int}

\begin{verbatim}
$ "quo"(a, b) &= floor(a/b) \
  "quo"(14, 5) &= #calc.quo(14, 5) \
  "quo"(3.46, 0.5) &= #calc.quo(3.46, 0.5) $
\end{verbatim}

\includegraphics[width=5in,height=\textheight,keepaspectratio]{/assets/docs/AEhIvOjgCcBZo0GMCLQ9tQAAAAAAAAAA.png}

\paragraph{\texorpdfstring{\texttt{\ dividend\ }}{ dividend }}\label{functions-quo-dividend}

\href{/docs/reference/foundations/int/}{int} {or}
\href{/docs/reference/foundations/float/}{float} {or}
\href{/docs/reference/foundations/decimal/}{decimal}

{Required} {{ Positional }}

\phantomsection\label{functions-quo-dividend-positional-tooltip}
Positional parameters are specified in order, without names.

The dividend of the quotient.

\paragraph{\texorpdfstring{\texttt{\ divisor\ }}{ divisor }}\label{functions-quo-divisor}

\href{/docs/reference/foundations/int/}{int} {or}
\href{/docs/reference/foundations/float/}{float} {or}
\href{/docs/reference/foundations/decimal/}{decimal}

{Required} {{ Positional }}

\phantomsection\label{functions-quo-divisor-positional-tooltip}
Positional parameters are specified in order, without names.

The divisor of the quotient.

\href{/docs/reference/foundations/bytes/}{\pandocbounded{\includesvg[keepaspectratio]{/assets/icons/16-arrow-right.svg}}}

{ Bytes } { Previous page }

\href{/docs/reference/foundations/content/}{\pandocbounded{\includesvg[keepaspectratio]{/assets/icons/16-arrow-right.svg}}}

{ Content } { Next page }

\textbf{On this page}

\begin{itemize}
\tightlist
\item
  \hyperref[summary]{Summary}
\item
  \hyperref[functions]{Functions}

  \begin{itemize}
  \tightlist
  \item
    \hyperref[functions-abs]{Absolute}

    \begin{itemize}
    \tightlist
    \item
      \hyperref[functions-abs-value]{value}
    \end{itemize}
  \item
    \hyperref[functions-pow]{Power}

    \begin{itemize}
    \tightlist
    \item
      \hyperref[functions-pow-base]{base}
    \item
      \hyperref[functions-pow-exponent]{exponent}
    \end{itemize}
  \item
    \hyperref[functions-exp]{Exponential}

    \begin{itemize}
    \tightlist
    \item
      \hyperref[functions-exp-exponent]{exponent}
    \end{itemize}
  \item
    \hyperref[functions-sqrt]{Square Root}

    \begin{itemize}
    \tightlist
    \item
      \hyperref[functions-sqrt-value]{value}
    \end{itemize}
  \item
    \hyperref[functions-root]{Root}

    \begin{itemize}
    \tightlist
    \item
      \hyperref[functions-root-radicand]{radicand}
    \item
      \hyperref[functions-root-index]{index}
    \end{itemize}
  \item
    \hyperref[functions-sin]{Sine}

    \begin{itemize}
    \tightlist
    \item
      \hyperref[functions-sin-angle]{angle}
    \end{itemize}
  \item
    \hyperref[functions-cos]{Cosine}

    \begin{itemize}
    \tightlist
    \item
      \hyperref[functions-cos-angle]{angle}
    \end{itemize}
  \item
    \hyperref[functions-tan]{Tangent}

    \begin{itemize}
    \tightlist
    \item
      \hyperref[functions-tan-angle]{angle}
    \end{itemize}
  \item
    \hyperref[functions-asin]{Arcsine}

    \begin{itemize}
    \tightlist
    \item
      \hyperref[functions-asin-value]{value}
    \end{itemize}
  \item
    \hyperref[functions-acos]{Arccosine}

    \begin{itemize}
    \tightlist
    \item
      \hyperref[functions-acos-value]{value}
    \end{itemize}
  \item
    \hyperref[functions-atan]{Arctangent}

    \begin{itemize}
    \tightlist
    \item
      \hyperref[functions-atan-value]{value}
    \end{itemize}
  \item
    \hyperref[functions-atan2]{Four-quadrant Arctangent}

    \begin{itemize}
    \tightlist
    \item
      \hyperref[functions-atan2-x]{x}
    \item
      \hyperref[functions-atan2-y]{y}
    \end{itemize}
  \item
    \hyperref[functions-sinh]{Hyperbolic Sine}

    \begin{itemize}
    \tightlist
    \item
      \hyperref[functions-sinh-value]{value}
    \end{itemize}
  \item
    \hyperref[functions-cosh]{Hyperbolic Cosine}

    \begin{itemize}
    \tightlist
    \item
      \hyperref[functions-cosh-value]{value}
    \end{itemize}
  \item
    \hyperref[functions-tanh]{Hyperbolic Tangent}

    \begin{itemize}
    \tightlist
    \item
      \hyperref[functions-tanh-value]{value}
    \end{itemize}
  \item
    \hyperref[functions-log]{Logarithm}

    \begin{itemize}
    \tightlist
    \item
      \hyperref[functions-log-value]{value}
    \item
      \hyperref[functions-log-base]{base}
    \end{itemize}
  \item
    \hyperref[functions-ln]{Natural Logarithm}

    \begin{itemize}
    \tightlist
    \item
      \hyperref[functions-ln-value]{value}
    \end{itemize}
  \item
    \hyperref[functions-fact]{Factorial}

    \begin{itemize}
    \tightlist
    \item
      \hyperref[functions-fact-number]{number}
    \end{itemize}
  \item
    \hyperref[functions-perm]{Permutation}

    \begin{itemize}
    \tightlist
    \item
      \hyperref[functions-perm-base]{base}
    \item
      \hyperref[functions-perm-numbers]{numbers}
    \end{itemize}
  \item
    \hyperref[functions-binom]{Binomial}

    \begin{itemize}
    \tightlist
    \item
      \hyperref[functions-binom-n]{n}
    \item
      \hyperref[functions-binom-k]{k}
    \end{itemize}
  \item
    \hyperref[functions-gcd]{Greatest Common Divisor}

    \begin{itemize}
    \tightlist
    \item
      \hyperref[functions-gcd-a]{a}
    \item
      \hyperref[functions-gcd-b]{b}
    \end{itemize}
  \item
    \hyperref[functions-lcm]{Least Common Multiple}

    \begin{itemize}
    \tightlist
    \item
      \hyperref[functions-lcm-a]{a}
    \item
      \hyperref[functions-lcm-b]{b}
    \end{itemize}
  \item
    \hyperref[functions-floor]{Floor}

    \begin{itemize}
    \tightlist
    \item
      \hyperref[functions-floor-value]{value}
    \end{itemize}
  \item
    \hyperref[functions-ceil]{Ceil}

    \begin{itemize}
    \tightlist
    \item
      \hyperref[functions-ceil-value]{value}
    \end{itemize}
  \item
    \hyperref[functions-trunc]{Truncate}

    \begin{itemize}
    \tightlist
    \item
      \hyperref[functions-trunc-value]{value}
    \end{itemize}
  \item
    \hyperref[functions-fract]{Fractional}

    \begin{itemize}
    \tightlist
    \item
      \hyperref[functions-fract-value]{value}
    \end{itemize}
  \item
    \hyperref[functions-round]{Round}

    \begin{itemize}
    \tightlist
    \item
      \hyperref[functions-round-value]{value}
    \item
      \hyperref[functions-round-digits]{digits}
    \end{itemize}
  \item
    \hyperref[functions-clamp]{Clamp}

    \begin{itemize}
    \tightlist
    \item
      \hyperref[functions-clamp-value]{value}
    \item
      \hyperref[functions-clamp-min]{min}
    \item
      \hyperref[functions-clamp-max]{max}
    \end{itemize}
  \item
    \hyperref[functions-min]{Minimum}

    \begin{itemize}
    \tightlist
    \item
      \hyperref[functions-min-values]{values}
    \end{itemize}
  \item
    \hyperref[functions-max]{Maximum}

    \begin{itemize}
    \tightlist
    \item
      \hyperref[functions-max-values]{values}
    \end{itemize}
  \item
    \hyperref[functions-even]{Even}

    \begin{itemize}
    \tightlist
    \item
      \hyperref[functions-even-value]{value}
    \end{itemize}
  \item
    \hyperref[functions-odd]{Odd}

    \begin{itemize}
    \tightlist
    \item
      \hyperref[functions-odd-value]{value}
    \end{itemize}
  \item
    \hyperref[functions-rem]{Remainder}

    \begin{itemize}
    \tightlist
    \item
      \hyperref[functions-rem-dividend]{dividend}
    \item
      \hyperref[functions-rem-divisor]{divisor}
    \end{itemize}
  \item
    \hyperref[functions-div-euclid]{Euclidean Division}

    \begin{itemize}
    \tightlist
    \item
      \hyperref[functions-div-euclid-dividend]{dividend}
    \item
      \hyperref[functions-div-euclid-divisor]{divisor}
    \end{itemize}
  \item
    \hyperref[functions-rem-euclid]{Euclidean Remainder}

    \begin{itemize}
    \tightlist
    \item
      \hyperref[functions-rem-euclid-dividend]{dividend}
    \item
      \hyperref[functions-rem-euclid-divisor]{divisor}
    \end{itemize}
  \item
    \hyperref[functions-quo]{Quotient}

    \begin{itemize}
    \tightlist
    \item
      \hyperref[functions-quo-dividend]{dividend}
    \item
      \hyperref[functions-quo-divisor]{divisor}
    \end{itemize}
  \end{itemize}
\end{itemize}

\begin{itemize}
\tightlist
\item
  \href{/}{Home}
\item
  \href{/pricing/}{Pricing}
\item
  \href{/docs/}{Documentation}
\item
  \href{/universe/}{Universe}
\item
  \href{/about/}{About Us}
\item
  \href{/contact/}{Contact Us}
\item
  \href{/privacy/}{Privacy}
\item
  \href{https://typst.app/terms}{Terms and Conditions}
\item
  \href{/legal/}{Legal (Impressum)}
\end{itemize}

\begin{itemize}
\tightlist
\item
  \href{https://forum.typst.app}{Forum}
\item
  \href{/tools/}{Tools}
\item
  \href{/blog/}{Blog}
\item
  \href{https://github.com/typst/}{GitHub}
\item
  \href{https://discord.gg/2uDybryKPe}{Discord}
\item
  \href{https://mastodon.social/@typst}{Mastodon}
\item
  \href{https://bsky.app/profile/typst.app}{Bluesky}
\item
  \href{https://www.linkedin.com/company/typst/}{LinkedIn}
\item
  \href{https://instagram.com/typstapp/}{Instagram}
\end{itemize}

Made in Berlin
