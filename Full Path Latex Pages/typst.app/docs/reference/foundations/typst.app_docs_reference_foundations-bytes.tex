\title{typst.app/docs/reference/foundations/bytes}

\begin{itemize}
\tightlist
\item
  \href{/docs}{\includesvg[width=0.16667in,height=0.16667in]{/assets/icons/16-docs-dark.svg}}
\item
  \includesvg[width=0.16667in,height=0.16667in]{/assets/icons/16-arrow-right.svg}
\item
  \href{/docs/reference/}{Reference}
\item
  \includesvg[width=0.16667in,height=0.16667in]{/assets/icons/16-arrow-right.svg}
\item
  \href{/docs/reference/foundations/}{Foundations}
\item
  \includesvg[width=0.16667in,height=0.16667in]{/assets/icons/16-arrow-right.svg}
\item
  \href{/docs/reference/foundations/bytes/}{Bytes}
\end{itemize}

\section{\texorpdfstring{{ bytes }}{ bytes }}\label{summary}

A sequence of bytes.

This is conceptually similar to an array of
\href{/docs/reference/foundations/int/}{integers} between
\texttt{\ }{\texttt{\ 0\ }}\texttt{\ } and
\texttt{\ }{\texttt{\ 255\ }}\texttt{\ } , but represented much more
efficiently. You can iterate over it using a
\href{/docs/reference/scripting/\#loops}{for loop} .

You can convert

\begin{itemize}
\tightlist
\item
  a \href{/docs/reference/foundations/str/}{string} or an
  \href{/docs/reference/foundations/array/}{array} of integers to bytes
  with the \href{/docs/reference/foundations/bytes/}{\texttt{\ bytes\ }}
  constructor
\item
  bytes to a string with the
  \href{/docs/reference/foundations/str/}{\texttt{\ str\ }} constructor,
  with UTF-8 encoding
\item
  bytes to an array of integers with the
  \href{/docs/reference/foundations/array/}{\texttt{\ array\ }}
  constructor
\end{itemize}

When \href{/docs/reference/data-loading/read/}{reading} data from a
file, you can decide whether to load it as a string or as raw bytes.

\begin{verbatim}
#bytes((123, 160, 22, 0)) \
#bytes("Hello 😃")

#let data = read(
  "rhino.png",
  encoding: none,
)

// Magic bytes.
#array(data.slice(0, 4)) \
#str(data.slice(1, 4))
\end{verbatim}

\includegraphics[width=5in,height=\textheight,keepaspectratio]{/assets/docs/sJtYFgVyQkDZELEHje5ywwAAAAAAAAAA.png}

\subsection{\texorpdfstring{Constructor
{}}{Constructor }}\label{constructor}

\phantomsection\label{constructor-constructor-tooltip}
If a type has a constructor, you can call it like a function to create a
new value of the type.

Converts a value to bytes.

\begin{itemize}
\tightlist
\item
  Strings are encoded in UTF-8.
\item
  Arrays of integers between \texttt{\ }{\texttt{\ 0\ }}\texttt{\ } and
  \texttt{\ }{\texttt{\ 255\ }}\texttt{\ } are converted directly. The
  dedicated byte representation is much more efficient than the array
  representation and thus typically used for large byte buffers (e.g.
  image data).
\end{itemize}

{ bytes } (

{ \href{/docs/reference/foundations/str/}{str}
\href{/docs/reference/foundations/bytes/}{bytes}
\href{/docs/reference/foundations/array/}{array} }

) -\textgreater{} \href{/docs/reference/foundations/bytes/}{bytes}

\begin{verbatim}
#bytes("Hello 😃") \
#bytes((123, 160, 22, 0))
\end{verbatim}

\includegraphics[width=5in,height=\textheight,keepaspectratio]{/assets/docs/PlfVajGmfDLMY6p8X4S3BwAAAAAAAAAA.png}

\paragraph{\texorpdfstring{\texttt{\ value\ }}{ value }}\label{constructor-value}

\href{/docs/reference/foundations/str/}{str} {or}
\href{/docs/reference/foundations/bytes/}{bytes} {or}
\href{/docs/reference/foundations/array/}{array}

{Required} {{ Positional }}

\phantomsection\label{constructor-value-positional-tooltip}
Positional parameters are specified in order, without names.

The value that should be converted to bytes.

\subsection{\texorpdfstring{{ Definitions
}}{ Definitions }}\label{definitions}

\phantomsection\label{definitions-tooltip}
Functions and types and can have associated definitions. These are
accessed by specifying the function or type, followed by a period, and
then the definition\textquotesingle s name.

\subsubsection{\texorpdfstring{\texttt{\ len\ }}{ len }}\label{definitions-len}

The length in bytes.

self { . } { len } (

) -\textgreater{} \href{/docs/reference/foundations/int/}{int}

\subsubsection{\texorpdfstring{\texttt{\ at\ }}{ at }}\label{definitions-at}

Returns the byte at the specified index. Returns the default value if
the index is out of bounds or fails with an error if no default value
was specified.

self { . } { at } (

{ \href{/docs/reference/foundations/int/}{int} , } {
\hyperref[definitions-at-parameters-default]{default :} { any } , }

) -\textgreater{} { any }

\paragraph{\texorpdfstring{\texttt{\ index\ }}{ index }}\label{definitions-at-index}

\href{/docs/reference/foundations/int/}{int}

{Required} {{ Positional }}

\phantomsection\label{definitions-at-index-positional-tooltip}
Positional parameters are specified in order, without names.

The index at which to retrieve the byte.

\paragraph{\texorpdfstring{\texttt{\ default\ }}{ default }}\label{definitions-at-default}

{ any }

A default value to return if the index is out of bounds.

\subsubsection{\texorpdfstring{\texttt{\ slice\ }}{ slice }}\label{definitions-slice}

Extracts a subslice of the bytes. Fails with an error if the start or
end index is out of bounds.

self { . } { slice } (

{ \href{/docs/reference/foundations/int/}{int} , } {
\href{/docs/reference/foundations/none/}{none}
\href{/docs/reference/foundations/int/}{int} , } {
\hyperref[definitions-slice-parameters-count]{count :}
\href{/docs/reference/foundations/int/}{int} , }

) -\textgreater{} \href{/docs/reference/foundations/bytes/}{bytes}

\paragraph{\texorpdfstring{\texttt{\ start\ }}{ start }}\label{definitions-slice-start}

\href{/docs/reference/foundations/int/}{int}

{Required} {{ Positional }}

\phantomsection\label{definitions-slice-start-positional-tooltip}
Positional parameters are specified in order, without names.

The start index (inclusive).

\paragraph{\texorpdfstring{\texttt{\ end\ }}{ end }}\label{definitions-slice-end}

\href{/docs/reference/foundations/none/}{none} {or}
\href{/docs/reference/foundations/int/}{int}

{{ Positional }}

\phantomsection\label{definitions-slice-end-positional-tooltip}
Positional parameters are specified in order, without names.

The end index (exclusive). If omitted, the whole slice until the end is
extracted.

Default: \texttt{\ }{\texttt{\ none\ }}\texttt{\ }

\paragraph{\texorpdfstring{\texttt{\ count\ }}{ count }}\label{definitions-slice-count}

\href{/docs/reference/foundations/int/}{int}

The number of items to extract. This is equivalent to passing
\texttt{\ start\ +\ count\ } as the \texttt{\ end\ } position. Mutually
exclusive with \texttt{\ end\ } .

\href{/docs/reference/foundations/bool/}{\pandocbounded{\includesvg[keepaspectratio]{/assets/icons/16-arrow-right.svg}}}

{ Boolean } { Previous page }

\href{/docs/reference/foundations/calc/}{\pandocbounded{\includesvg[keepaspectratio]{/assets/icons/16-arrow-right.svg}}}

{ Calculation } { Next page }

\textbf{On this page}

\begin{itemize}
\tightlist
\item
  \hyperref[summary]{Summary}
\item
  \hyperref[constructor]{Constructor}

  \begin{itemize}
  \tightlist
  \item
    \hyperref[constructor-value]{value}
  \end{itemize}
\item
  \hyperref[definitions]{Definitions}

  \begin{itemize}
  \tightlist
  \item
    \hyperref[definitions-len]{Length}
  \item
    \hyperref[definitions-at]{At}

    \begin{itemize}
    \tightlist
    \item
      \hyperref[definitions-at-index]{index}
    \item
      \hyperref[definitions-at-default]{default}
    \end{itemize}
  \item
    \hyperref[definitions-slice]{Slice}

    \begin{itemize}
    \tightlist
    \item
      \hyperref[definitions-slice-start]{start}
    \item
      \hyperref[definitions-slice-end]{end}
    \item
      \hyperref[definitions-slice-count]{count}
    \end{itemize}
  \end{itemize}
\end{itemize}

\begin{itemize}
\tightlist
\item
  \href{/}{Home}
\item
  \href{/pricing/}{Pricing}
\item
  \href{/docs/}{Documentation}
\item
  \href{/universe/}{Universe}
\item
  \href{/about/}{About Us}
\item
  \href{/contact/}{Contact Us}
\item
  \href{/privacy/}{Privacy}
\item
  \href{https://typst.app/terms}{Terms and Conditions}
\item
  \href{/legal/}{Legal (Impressum)}
\end{itemize}

\begin{itemize}
\tightlist
\item
  \href{https://forum.typst.app}{Forum}
\item
  \href{/tools/}{Tools}
\item
  \href{/blog/}{Blog}
\item
  \href{https://github.com/typst/}{GitHub}
\item
  \href{https://discord.gg/2uDybryKPe}{Discord}
\item
  \href{https://mastodon.social/@typst}{Mastodon}
\item
  \href{https://bsky.app/profile/typst.app}{Bluesky}
\item
  \href{https://www.linkedin.com/company/typst/}{LinkedIn}
\item
  \href{https://instagram.com/typstapp/}{Instagram}
\end{itemize}

Made in Berlin
