\title{typst.app/docs/reference/foundations/dictionary}

\begin{itemize}
\tightlist
\item
  \href{/docs}{\includesvg[width=0.16667in,height=0.16667in]{/assets/icons/16-docs-dark.svg}}
\item
  \includesvg[width=0.16667in,height=0.16667in]{/assets/icons/16-arrow-right.svg}
\item
  \href{/docs/reference/}{Reference}
\item
  \includesvg[width=0.16667in,height=0.16667in]{/assets/icons/16-arrow-right.svg}
\item
  \href{/docs/reference/foundations/}{Foundations}
\item
  \includesvg[width=0.16667in,height=0.16667in]{/assets/icons/16-arrow-right.svg}
\item
  \href{/docs/reference/foundations/dictionary/}{Dictionary}
\end{itemize}

\section{\texorpdfstring{{ dictionary }}{ dictionary }}\label{summary}

A map from string keys to values.

You can construct a dictionary by enclosing comma-separated
\texttt{\ key:\ value\ } pairs in parentheses. The values do not have to
be of the same type. Since empty parentheses already yield an empty
array, you have to use the special \texttt{\ (:)\ } syntax to create an
empty dictionary.

A dictionary is conceptually similar to an array, but it is indexed by
strings instead of integers. You can access and create dictionary
entries with the \texttt{\ .at()\ } method. If you know the key
statically, you can alternatively use
\href{/docs/reference/scripting/\#fields}{field access notation} (
\texttt{\ .key\ } ) to access the value. Dictionaries can be added with
the \texttt{\ +\ } operator and
\href{/docs/reference/scripting/\#blocks}{joined together} . To check
whether a key is present in the dictionary, use the \texttt{\ in\ }
keyword.

You can iterate over the pairs in a dictionary using a
\href{/docs/reference/scripting/\#loops}{for loop} . This will iterate
in the order the pairs were inserted / declared.

\subsection{Example}\label{example}

\begin{verbatim}
#let dict = (
  name: "Typst",
  born: 2019,
)

#dict.name \
#(dict.launch = 20)
#dict.len() \
#dict.keys() \
#dict.values() \
#dict.at("born") \
#dict.insert("city", "Berlin ")
#("name" in dict)
\end{verbatim}

\includegraphics[width=5in,height=\textheight,keepaspectratio]{/assets/docs/1ByIQqDPZ4VVxPmFNoQXgwAAAAAAAAAA.png}

\subsection{\texorpdfstring{Constructor
{}}{Constructor }}\label{constructor}

\phantomsection\label{constructor-constructor-tooltip}
If a type has a constructor, you can call it like a function to create a
new value of the type.

Converts a value into a dictionary.

Note that this function is only intended for conversion of a
dictionary-like value to a dictionary, not for creation of a dictionary
from individual pairs. Use the dictionary syntax
\texttt{\ (key:\ value)\ } instead.

{ dictionary } (

{ \href{/docs/reference/foundations/module/}{module} }

) -\textgreater{}
\href{/docs/reference/foundations/dictionary/}{dictionary}

\begin{verbatim}
#dictionary(sys).at("version")
\end{verbatim}

\includegraphics[width=5in,height=\textheight,keepaspectratio]{/assets/docs/vrwNZ5Jfl6kz7gYnEOsM0AAAAAAAAAAA.png}

\paragraph{\texorpdfstring{\texttt{\ value\ }}{ value }}\label{constructor-value}

\href{/docs/reference/foundations/module/}{module}

{Required} {{ Positional }}

\phantomsection\label{constructor-value-positional-tooltip}
Positional parameters are specified in order, without names.

The value that should be converted to a dictionary.

\subsection{\texorpdfstring{{ Definitions
}}{ Definitions }}\label{definitions}

\phantomsection\label{definitions-tooltip}
Functions and types and can have associated definitions. These are
accessed by specifying the function or type, followed by a period, and
then the definition\textquotesingle s name.

\subsubsection{\texorpdfstring{\texttt{\ len\ }}{ len }}\label{definitions-len}

The number of pairs in the dictionary.

self { . } { len } (

) -\textgreater{} \href{/docs/reference/foundations/int/}{int}

\subsubsection{\texorpdfstring{\texttt{\ at\ }}{ at }}\label{definitions-at}

Returns the value associated with the specified key in the dictionary.
May be used on the left-hand side of an assignment if the key is already
present in the dictionary. Returns the default value if the key is not
part of the dictionary or fails with an error if no default value was
specified.

self { . } { at } (

{ \href{/docs/reference/foundations/str/}{str} , } {
\hyperref[definitions-at-parameters-default]{default :} { any } , }

) -\textgreater{} { any }

\paragraph{\texorpdfstring{\texttt{\ key\ }}{ key }}\label{definitions-at-key}

\href{/docs/reference/foundations/str/}{str}

{Required} {{ Positional }}

\phantomsection\label{definitions-at-key-positional-tooltip}
Positional parameters are specified in order, without names.

The key at which to retrieve the item.

\paragraph{\texorpdfstring{\texttt{\ default\ }}{ default }}\label{definitions-at-default}

{ any }

A default value to return if the key is not part of the dictionary.

\subsubsection{\texorpdfstring{\texttt{\ insert\ }}{ insert }}\label{definitions-insert}

Inserts a new pair into the dictionary. If the dictionary already
contains this key, the value is updated.

self { . } { insert } (

{ \href{/docs/reference/foundations/str/}{str} , } { { any } , }

)

\paragraph{\texorpdfstring{\texttt{\ key\ }}{ key }}\label{definitions-insert-key}

\href{/docs/reference/foundations/str/}{str}

{Required} {{ Positional }}

\phantomsection\label{definitions-insert-key-positional-tooltip}
Positional parameters are specified in order, without names.

The key of the pair that should be inserted.

\paragraph{\texorpdfstring{\texttt{\ value\ }}{ value }}\label{definitions-insert-value}

{ any }

{Required} {{ Positional }}

\phantomsection\label{definitions-insert-value-positional-tooltip}
Positional parameters are specified in order, without names.

The value of the pair that should be inserted.

\subsubsection{\texorpdfstring{\texttt{\ remove\ }}{ remove }}\label{definitions-remove}

Removes a pair from the dictionary by key and return the value.

self { . } { remove } (

{ \href{/docs/reference/foundations/str/}{str} , } {
\hyperref[definitions-remove-parameters-default]{default :} { any } , }

) -\textgreater{} { any }

\paragraph{\texorpdfstring{\texttt{\ key\ }}{ key }}\label{definitions-remove-key}

\href{/docs/reference/foundations/str/}{str}

{Required} {{ Positional }}

\phantomsection\label{definitions-remove-key-positional-tooltip}
Positional parameters are specified in order, without names.

The key of the pair to remove.

\paragraph{\texorpdfstring{\texttt{\ default\ }}{ default }}\label{definitions-remove-default}

{ any }

A default value to return if the key does not exist.

\subsubsection{\texorpdfstring{\texttt{\ keys\ }}{ keys }}\label{definitions-keys}

Returns the keys of the dictionary as an array in insertion order.

self { . } { keys } (

) -\textgreater{} \href{/docs/reference/foundations/array/}{array}

\subsubsection{\texorpdfstring{\texttt{\ values\ }}{ values }}\label{definitions-values}

Returns the values of the dictionary as an array in insertion order.

self { . } { values } (

) -\textgreater{} \href{/docs/reference/foundations/array/}{array}

\subsubsection{\texorpdfstring{\texttt{\ pairs\ }}{ pairs }}\label{definitions-pairs}

Returns the keys and values of the dictionary as an array of pairs. Each
pair is represented as an array of length two.

self { . } { pairs } (

) -\textgreater{} \href{/docs/reference/foundations/array/}{array}

\href{/docs/reference/foundations/decimal/}{\pandocbounded{\includesvg[keepaspectratio]{/assets/icons/16-arrow-right.svg}}}

{ Decimal } { Previous page }

\href{/docs/reference/foundations/duration/}{\pandocbounded{\includesvg[keepaspectratio]{/assets/icons/16-arrow-right.svg}}}

{ Duration } { Next page }
