\title{typst.app/docs/reference/foundations/function}

\begin{itemize}
\tightlist
\item
  \href{/docs}{\includesvg[width=0.16667in,height=0.16667in]{/assets/icons/16-docs-dark.svg}}
\item
  \includesvg[width=0.16667in,height=0.16667in]{/assets/icons/16-arrow-right.svg}
\item
  \href{/docs/reference/}{Reference}
\item
  \includesvg[width=0.16667in,height=0.16667in]{/assets/icons/16-arrow-right.svg}
\item
  \href{/docs/reference/foundations/}{Foundations}
\item
  \includesvg[width=0.16667in,height=0.16667in]{/assets/icons/16-arrow-right.svg}
\item
  \href{/docs/reference/foundations/function/}{Function}
\end{itemize}

\section{\texorpdfstring{{ function }}{ function }}\label{summary}

A mapping from argument values to a return value.

You can call a function by writing a comma-separated list of function
\emph{arguments} enclosed in parentheses directly after the function
name. Additionally, you can pass any number of trailing content blocks
arguments to a function \emph{after} the normal argument list. If the
normal argument list would become empty, it can be omitted. Typst
supports positional and named arguments. The former are identified by
position and type, while the latter are written as
\texttt{\ name:\ value\ } .

Within math mode, function calls have special behaviour. See the
\href{/docs/reference/math/}{math documentation} for more details.

\subsection{Example}\label{example}

\begin{verbatim}
// Call a function.
#list([A], [B])

// Named arguments and trailing
// content blocks.
#enum(start: 2)[A][B]

// Version without parentheses.
#list[A][B]
\end{verbatim}

\includegraphics[width=5in,height=\textheight,keepaspectratio]{/assets/docs/h8ulslRsTYE05Pu4qy5C6AAAAAAAAAAA.png}

Functions are a fundamental building block of Typst. Typst provides
functions for a variety of typesetting tasks. Moreover, the markup you
write is backed by functions and all styling happens through functions.
This reference lists all available functions and how you can use them.
Please also refer to the documentation about
\href{/docs/reference/styling/\#set-rules}{set} and
\href{/docs/reference/styling/\#show-rules}{show} rules to learn about
additional ways you can work with functions in Typst.

\subsection{Element functions}\label{element-functions}

Some functions are associated with \emph{elements} like
\href{/docs/reference/model/heading/}{headings} or
\href{/docs/reference/model/table/}{tables} . When called, these create
an element of their respective kind. In contrast to normal functions,
they can further be used in
\href{/docs/reference/styling/\#set-rules}{set rules} ,
\href{/docs/reference/styling/\#show-rules}{show rules} , and
\href{/docs/reference/foundations/selector/}{selectors} .

\subsection{Function scopes}\label{function-scopes}

Functions can hold related definitions in their own scope, similar to a
\href{/docs/reference/scripting/\#modules}{module} . Examples of this
are
\href{/docs/reference/foundations/assert/\#definitions-eq}{\texttt{\ assert.eq\ }}
or
\href{/docs/reference/model/list/\#definitions-item}{\texttt{\ list.item\ }}
. However, this feature is currently only available for built-in
functions.

\subsection{Defining functions}\label{defining-functions}

You can define your own function with a
\href{/docs/reference/scripting/\#bindings}{let binding} that has a
parameter list after the binding\textquotesingle s name. The parameter
list can contain mandatory positional parameters, named parameters with
default values and
\href{/docs/reference/foundations/arguments/}{argument sinks} .

The right-hand side of a function binding is the function body, which
can be a block or any other expression. It defines the
function\textquotesingle s return value and can depend on the
parameters. If the function body is a
\href{/docs/reference/scripting/\#blocks}{code block} , the return value
is the result of joining the values of each expression in the block.

Within a function body, the \texttt{\ return\ } keyword can be used to
exit early and optionally specify a return value. If no explicit return
value is given, the body evaluates to the result of joining all
expressions preceding the \texttt{\ return\ } .

Functions that don\textquotesingle t return any meaningful value return
\href{/docs/reference/foundations/none/}{\texttt{\ none\ }} instead. The
return type of such functions is not explicitly specified in the
documentation. (An example of this is
\href{/docs/reference/foundations/array/\#definitions-push}{\texttt{\ array.push\ }}
).

\begin{verbatim}
#let alert(body, fill: red) = {
  set text(white)
  set align(center)
  rect(
    fill: fill,
    inset: 8pt,
    radius: 4pt,
    [*Warning:\ #body*],
  )
}

#alert[
  Danger is imminent!
]

#alert(fill: blue)[
  KEEP OFF TRACKS
]
\end{verbatim}

\includegraphics[width=5in,height=\textheight,keepaspectratio]{/assets/docs/56wK4TQtzRt7_B3OQOxb7QAAAAAAAAAA.png}

\subsection{Importing functions}\label{importing-functions}

Functions can be imported from one file (
\href{/docs/reference/scripting/\#modules}{\texttt{\ module\ }} ) into
another using \texttt{\ }{\texttt{\ import\ }}\texttt{\ } . For example,
assume that we have defined the \texttt{\ alert\ } function from the
previous example in a file called \texttt{\ foo.typ\ } . We can import
it into another file by writing
\texttt{\ }{\texttt{\ import\ }}\texttt{\ }{\texttt{\ "foo.typ"\ }}\texttt{\ }{\texttt{\ :\ }}\texttt{\ alert\ }
.

\subsection{Unnamed functions}\label{unnamed}

You can also created an unnamed function without creating a binding by
specifying a parameter list followed by \texttt{\ =\textgreater{}\ } and
the function body. If your function has just one parameter, the
parentheses around the parameter list are optional. Unnamed functions
are mainly useful for show rules, but also for settable properties that
take functions like the page function\textquotesingle s
\href{/docs/reference/layout/page/\#parameters-footer}{\texttt{\ footer\ }}
property.

\begin{verbatim}
#show "once?": it => [#it #it]
once?
\end{verbatim}

\includegraphics[width=5in,height=\textheight,keepaspectratio]{/assets/docs/yXee-w_McX_Uho7Ghovc-QAAAAAAAAAA.png}

\subsection{Note on function purity}\label{note-on-function-purity}

In Typst, all functions are \emph{pure.} This means that for the same
arguments, they always return the same result. They cannot "remember"
things to produce another value when they are called a second time.

The only exception are built-in methods like
\href{/docs/reference/foundations/array/\#definitions-push}{\texttt{\ array.push(value)\ }}
. These can modify the values they are called on.

\subsection{\texorpdfstring{{ Definitions
}}{ Definitions }}\label{definitions}

\phantomsection\label{definitions-tooltip}
Functions and types and can have associated definitions. These are
accessed by specifying the function or type, followed by a period, and
then the definition\textquotesingle s name.

\subsubsection{\texorpdfstring{\texttt{\ with\ }}{ with }}\label{definitions-with}

Returns a new function that has the given arguments pre-applied.

self { . } { with } (

{ \hyperref[definitions-with-parameters-arguments]{..} { any } }

) -\textgreater{} \href{/docs/reference/foundations/function/}{function}

\paragraph{\texorpdfstring{\texttt{\ arguments\ }}{ arguments }}\label{definitions-with-arguments}

{ any }

{Required} {{ Positional }}

\phantomsection\label{definitions-with-arguments-positional-tooltip}
Positional parameters are specified in order, without names.

{{ Variadic }}

\phantomsection\label{definitions-with-arguments-variadic-tooltip}
Variadic parameters can be specified multiple times.

The arguments to apply to the function.

\subsubsection{\texorpdfstring{\texttt{\ where\ }}{ where }}\label{definitions-where}

Returns a selector that filters for elements belonging to this function
whose fields have the values of the given arguments.

self { . } { where } (

{ \hyperref[definitions-where-parameters-fields]{..} { any } }

) -\textgreater{} \href{/docs/reference/foundations/selector/}{selector}

\begin{verbatim}
#show heading.where(level: 2): set text(blue)
= Section
== Subsection
=== Sub-subsection
\end{verbatim}

\includegraphics[width=5in,height=\textheight,keepaspectratio]{/assets/docs/VOR4DpWIitR8ukDkDB2RigAAAAAAAAAA.png}

\paragraph{\texorpdfstring{\texttt{\ fields\ }}{ fields }}\label{definitions-where-fields}

{ any }

{Required} {{ Positional }}

\phantomsection\label{definitions-where-fields-positional-tooltip}
Positional parameters are specified in order, without names.

{{ Variadic }}

\phantomsection\label{definitions-where-fields-variadic-tooltip}
Variadic parameters can be specified multiple times.

The fields to filter for.

\href{/docs/reference/foundations/float/}{\pandocbounded{\includesvg[keepaspectratio]{/assets/icons/16-arrow-right.svg}}}

{ Float } { Previous page }

\href{/docs/reference/foundations/int/}{\pandocbounded{\includesvg[keepaspectratio]{/assets/icons/16-arrow-right.svg}}}

{ Integer } { Next page }

\textbf{On this page}

\begin{itemize}
\tightlist
\item
  \hyperref[summary]{Summary}
\item
  \hyperref[example]{Example}
\item
  \hyperref[element-functions]{Element Functions}
\item
  \hyperref[function-scopes]{Function Scopes}
\item
  \hyperref[defining-functions]{Defining Functions}
\item
  \hyperref[importing-functions]{Importing Functions}
\item
  \hyperref[unnamed]{Unnamed}
\item
  \hyperref[note-on-function-purity]{Note On Function Purity}
\item
  \hyperref[definitions]{Definitions}

  \begin{itemize}
  \tightlist
  \item
    \hyperref[definitions-with]{With}

    \begin{itemize}
    \tightlist
    \item
      \hyperref[definitions-with-arguments]{arguments}
    \end{itemize}
  \item
    \hyperref[definitions-where]{Where}

    \begin{itemize}
    \tightlist
    \item
      \hyperref[definitions-where-fields]{fields}
    \end{itemize}
  \end{itemize}
\end{itemize}

\begin{itemize}
\tightlist
\item
  \href{/}{Home}
\item
  \href{/pricing/}{Pricing}
\item
  \href{/docs/}{Documentation}
\item
  \href{/universe/}{Universe}
\item
  \href{/about/}{About Us}
\item
  \href{/contact/}{Contact Us}
\item
  \href{/privacy/}{Privacy}
\item
  \href{https://typst.app/terms}{Terms and Conditions}
\item
  \href{/legal/}{Legal (Impressum)}
\end{itemize}

\begin{itemize}
\tightlist
\item
  \href{https://forum.typst.app}{Forum}
\item
  \href{/tools/}{Tools}
\item
  \href{/blog/}{Blog}
\item
  \href{https://github.com/typst/}{GitHub}
\item
  \href{https://discord.gg/2uDybryKPe}{Discord}
\item
  \href{https://mastodon.social/@typst}{Mastodon}
\item
  \href{https://bsky.app/profile/typst.app}{Bluesky}
\item
  \href{https://www.linkedin.com/company/typst/}{LinkedIn}
\item
  \href{https://instagram.com/typstapp/}{Instagram}
\end{itemize}

Made in Berlin
