\title{typst.app/docs/reference/foundations/decimal}

\begin{itemize}
\tightlist
\item
  \href{/docs}{\includesvg[width=0.16667in,height=0.16667in]{/assets/icons/16-docs-dark.svg}}
\item
  \includesvg[width=0.16667in,height=0.16667in]{/assets/icons/16-arrow-right.svg}
\item
  \href{/docs/reference/}{Reference}
\item
  \includesvg[width=0.16667in,height=0.16667in]{/assets/icons/16-arrow-right.svg}
\item
  \href{/docs/reference/foundations/}{Foundations}
\item
  \includesvg[width=0.16667in,height=0.16667in]{/assets/icons/16-arrow-right.svg}
\item
  \href{/docs/reference/foundations/decimal/}{Decimal}
\end{itemize}

\section{\texorpdfstring{{ decimal }}{ decimal }}\label{summary}

A fixed-point decimal number type.

This type should be used for precise arithmetic operations on numbers
represented in base 10. A typical use case is representing currency.

\subsection{Example}\label{example}

\begin{verbatim}
Decimal: #(decimal("0.1") + decimal("0.2")) \
Float: #(0.1 + 0.2)
\end{verbatim}

\includegraphics[width=5in,height=\textheight,keepaspectratio]{/assets/docs/W31Kvh6BvfIgTgIeq2uIEQAAAAAAAAAA.png}

\subsection{Construction and casts}\label{construction-and-casts}

To create a decimal number, use the
\texttt{\ }{\texttt{\ decimal\ }}\texttt{\ }{\texttt{\ (\ }}\texttt{\ string\ }{\texttt{\ )\ }}\texttt{\ }
constructor, such as in
\texttt{\ }{\texttt{\ decimal\ }}\texttt{\ }{\texttt{\ (\ }}\texttt{\ }{\texttt{\ "3.141592653"\ }}\texttt{\ }{\texttt{\ )\ }}\texttt{\ }
\textbf{(note the double quotes!)} . This constructor preserves all
given fractional digits, provided they are representable as per the
limits specified below (otherwise, an error is raised).

You can also convert any
\href{/docs/reference/foundations/int/}{integer} to a decimal with the
\texttt{\ }{\texttt{\ decimal\ }}\texttt{\ }{\texttt{\ (\ }}\texttt{\ int\ }{\texttt{\ )\ }}\texttt{\ }
constructor, e.g.
\texttt{\ }{\texttt{\ decimal\ }}\texttt{\ }{\texttt{\ (\ }}\texttt{\ }{\texttt{\ 59\ }}\texttt{\ }{\texttt{\ )\ }}\texttt{\ }
. However, note that constructing a decimal from a
\href{/docs/reference/foundations/float/}{floating-point number} , while
supported, \textbf{is an imprecise conversion and therefore
discouraged.} A warning will be raised if Typst detects that there was
an accidental \texttt{\ float\ } to \texttt{\ decimal\ } cast through
its constructor, e.g. if writing
\texttt{\ }{\texttt{\ decimal\ }}\texttt{\ }{\texttt{\ (\ }}\texttt{\ }{\texttt{\ 3.14\ }}\texttt{\ }{\texttt{\ )\ }}\texttt{\ }
(note the lack of double quotes, indicating this is an accidental
\texttt{\ float\ } cast and therefore imprecise). It is recommended to
use strings for constant decimal values instead (e.g.
\texttt{\ }{\texttt{\ decimal\ }}\texttt{\ }{\texttt{\ (\ }}\texttt{\ }{\texttt{\ "3.14"\ }}\texttt{\ }{\texttt{\ )\ }}\texttt{\ }
).

The precision of a \texttt{\ float\ } to \texttt{\ decimal\ } cast can
be slightly improved by rounding the result to 15 digits with
\href{/docs/reference/foundations/calc/\#functions-round}{\texttt{\ calc.round\ }}
, but there are still no precision guarantees for that kind of
conversion.

\subsection{Operations}\label{operations}

Basic arithmetic operations are supported on two decimals and on pairs
of decimals and integers.

Built-in operations between \texttt{\ float\ } and \texttt{\ decimal\ }
are not supported in order to guard against accidental loss of
precision. They will raise an error instead.

Certain \texttt{\ calc\ } functions, such as trigonometric functions and
power between two real numbers, are also only supported for
\texttt{\ float\ } (although raising \texttt{\ decimal\ } to integer
exponents is supported). You can opt into potentially imprecise
operations with the
\texttt{\ }{\texttt{\ float\ }}\texttt{\ }{\texttt{\ (\ }}\texttt{\ decimal\ }{\texttt{\ )\ }}\texttt{\ }
constructor, which casts the \texttt{\ decimal\ } number into a
\texttt{\ float\ } , allowing for operations without precision
guarantees.

\subsection{Displaying decimals}\label{displaying-decimals}

To display a decimal, simply insert the value into the document. To only
display a certain number of digits,
\href{/docs/reference/foundations/calc/\#functions-round}{round} the
decimal first. Localized formatting of decimals and other numbers is not
yet supported, but planned for the future.

You can convert decimals to strings using the
\href{/docs/reference/foundations/str/}{\texttt{\ str\ }} constructor.
This way, you can post-process the displayed representation, e.g. to
replace the period with a comma (as a stand-in for proper built-in
localization to languages that use the comma).

\subsection{Precision and limits}\label{precision-and-limits}

A \texttt{\ decimal\ } number has a limit of 28 to 29 significant
base-10 digits. This includes the sum of digits before and after the
decimal point. As such, numbers with more fractional digits have a
smaller range. The maximum and minimum \texttt{\ decimal\ } numbers have
a value of
\texttt{\ }{\texttt{\ 79228162514264337593543950335\ }}\texttt{\ } and
\texttt{\ }{\texttt{\ -\ }}\texttt{\ }{\texttt{\ 79228162514264337593543950335\ }}\texttt{\ }
respectively. In contrast with
\href{/docs/reference/foundations/float/}{\texttt{\ float\ }} , this
type does not support infinity or NaN, so overflowing or underflowing
operations will raise an error.

Typical operations between \texttt{\ decimal\ } numbers, such as
addition, multiplication, and
\href{/docs/reference/foundations/calc/\#functions-pow}{power} to an
integer, will be highly precise due to their fixed-point representation.
Note, however, that multiplication and division may not preserve all
digits in some edge cases: while they are considered precise, digits
past the limits specified above are rounded off and lost, so some loss
of precision beyond the maximum representable digits is possible. Note
that this behavior can be observed not only when dividing, but also when
multiplying by numbers between 0 and 1, as both operations can push a
number\textquotesingle s fractional digits beyond the limits described
above, leading to rounding. When those two operations do not surpass the
digit limits, they are fully precise.

\subsection{\texorpdfstring{Constructor
{}}{Constructor }}\label{constructor}

\phantomsection\label{constructor-constructor-tooltip}
If a type has a constructor, you can call it like a function to create a
new value of the type.

Converts a value to a \texttt{\ decimal\ } .

It is recommended to use a string to construct the decimal number, or an
\href{/docs/reference/foundations/int/}{integer} (if desired). The
string must contain a number in the format
\texttt{\ }{\texttt{\ "3.14159"\ }}\texttt{\ } (or
\texttt{\ }{\texttt{\ "-3.141519"\ }}\texttt{\ } for negative numbers).
The fractional digits are fully preserved; if that\textquotesingle s not
possible due to the limit of significant digits (around 28 to 29) having
been reached, an error is raised as the given decimal number
wouldn\textquotesingle t be representable.

While this constructor can be used with
\href{/docs/reference/foundations/float/}{floating-point numbers} to
cast them to \texttt{\ decimal\ } , doing so is \textbf{discouraged} as
\textbf{this cast is inherently imprecise.} It is easy to accidentally
perform this cast by writing
\texttt{\ }{\texttt{\ decimal\ }}\texttt{\ }{\texttt{\ (\ }}\texttt{\ }{\texttt{\ 1.234\ }}\texttt{\ }{\texttt{\ )\ }}\texttt{\ }
(note the lack of double quotes), which is why Typst will emit a warning
in that case. Please write
\texttt{\ }{\texttt{\ decimal\ }}\texttt{\ }{\texttt{\ (\ }}\texttt{\ }{\texttt{\ "1.234"\ }}\texttt{\ }{\texttt{\ )\ }}\texttt{\ }
instead for that particular case (initialization of a constant decimal).
Also note that floats that are NaN or infinite cannot be cast to
decimals and will raise an error.

{ decimal } (

{ \href{/docs/reference/foundations/int/}{int}
\href{/docs/reference/foundations/float/}{float}
\href{/docs/reference/foundations/str/}{str} }

) -\textgreater{} \href{/docs/reference/foundations/decimal/}{decimal}

\begin{verbatim}
#decimal("1.222222222222222")
\end{verbatim}

\includegraphics[width=5in,height=\textheight,keepaspectratio]{/assets/docs/RfqlB85Q5lIVeebJq7RlmgAAAAAAAAAA.png}

\paragraph{\texorpdfstring{\texttt{\ value\ }}{ value }}\label{constructor-value}

\href{/docs/reference/foundations/int/}{int} {or}
\href{/docs/reference/foundations/float/}{float} {or}
\href{/docs/reference/foundations/str/}{str}

{Required} {{ Positional }}

\phantomsection\label{constructor-value-positional-tooltip}
Positional parameters are specified in order, without names.

The value that should be converted to a decimal.

\href{/docs/reference/foundations/datetime/}{\pandocbounded{\includesvg[keepaspectratio]{/assets/icons/16-arrow-right.svg}}}

{ Datetime } { Previous page }

\href{/docs/reference/foundations/dictionary/}{\pandocbounded{\includesvg[keepaspectratio]{/assets/icons/16-arrow-right.svg}}}

{ Dictionary } { Next page }

\textbf{On this page}

\begin{itemize}
\tightlist
\item
  \hyperref[summary]{Summary}
\item
  \hyperref[example]{Example}
\item
  \hyperref[construction-and-casts]{Construction And Casts}
\item
  \hyperref[operations]{Operations}
\item
  \hyperref[displaying-decimals]{Displaying Decimals}
\item
  \hyperref[precision-and-limits]{Precision And Limits}
\item
  \hyperref[constructor]{Constructor}

  \begin{itemize}
  \tightlist
  \item
    \hyperref[constructor-value]{value}
  \end{itemize}
\end{itemize}

\begin{itemize}
\tightlist
\item
  \href{/}{Home}
\item
  \href{/pricing/}{Pricing}
\item
  \href{/docs/}{Documentation}
\item
  \href{/universe/}{Universe}
\item
  \href{/about/}{About Us}
\item
  \href{/contact/}{Contact Us}
\item
  \href{/privacy/}{Privacy}
\item
  \href{https://typst.app/terms}{Terms and Conditions}
\item
  \href{/legal/}{Legal (Impressum)}
\end{itemize}

\begin{itemize}
\tightlist
\item
  \href{https://forum.typst.app}{Forum}
\item
  \href{/tools/}{Tools}
\item
  \href{/blog/}{Blog}
\item
  \href{https://github.com/typst/}{GitHub}
\item
  \href{https://discord.gg/2uDybryKPe}{Discord}
\item
  \href{https://mastodon.social/@typst}{Mastodon}
\item
  \href{https://bsky.app/profile/typst.app}{Bluesky}
\item
  \href{https://www.linkedin.com/company/typst/}{LinkedIn}
\item
  \href{https://instagram.com/typstapp/}{Instagram}
\end{itemize}

Made in Berlin
