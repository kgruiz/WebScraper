\title{typst.app/docs/reference/visualize/polygon}

\begin{itemize}
\tightlist
\item
  \href{/docs}{\includesvg[width=0.16667in,height=0.16667in]{/assets/icons/16-docs-dark.svg}}
\item
  \includesvg[width=0.16667in,height=0.16667in]{/assets/icons/16-arrow-right.svg}
\item
  \href{/docs/reference/}{Reference}
\item
  \includesvg[width=0.16667in,height=0.16667in]{/assets/icons/16-arrow-right.svg}
\item
  \href{/docs/reference/visualize/}{Visualize}
\item
  \includesvg[width=0.16667in,height=0.16667in]{/assets/icons/16-arrow-right.svg}
\item
  \href{/docs/reference/visualize/polygon/}{Polygon}
\end{itemize}

\section{\texorpdfstring{\texttt{\ polygon\ } {{ Element
}}}{ polygon   Element }}\label{summary}

\phantomsection\label{element-tooltip}
Element functions can be customized with \texttt{\ set\ } and
\texttt{\ show\ } rules.

A closed polygon.

The polygon is defined by its corner points and is closed automatically.

\subsection{Example}\label{example}

\begin{verbatim}
#polygon(
  fill: blue.lighten(80%),
  stroke: blue,
  (20%, 0pt),
  (60%, 0pt),
  (80%, 2cm),
  (0%,  2cm),
)
\end{verbatim}

\includegraphics[width=5in,height=\textheight,keepaspectratio]{/assets/docs/TuzATomarVg-0NmUVu3QFAAAAAAAAAAA.png}

\subsection{\texorpdfstring{{ Parameters
}}{ Parameters }}\label{parameters}

\phantomsection\label{parameters-tooltip}
Parameters are the inputs to a function. They are specified in
parentheses after the function name.

{ polygon } (

{ \hyperref[parameters-fill]{fill :}
\href{/docs/reference/foundations/none/}{none}
\href{/docs/reference/visualize/color/}{color}
\href{/docs/reference/visualize/gradient/}{gradient}
\href{/docs/reference/visualize/pattern/}{pattern} , } {
\hyperref[parameters-fill-rule]{fill-rule :}
\href{/docs/reference/foundations/str/}{str} , } {
\hyperref[parameters-stroke]{stroke :}
\href{/docs/reference/foundations/none/}{none}
\href{/docs/reference/foundations/auto/}{auto}
\href{/docs/reference/layout/length/}{length}
\href{/docs/reference/visualize/color/}{color}
\href{/docs/reference/visualize/gradient/}{gradient}
\href{/docs/reference/visualize/stroke/}{stroke}
\href{/docs/reference/visualize/pattern/}{pattern}
\href{/docs/reference/foundations/dictionary/}{dictionary} , } {
\hyperref[parameters-vertices]{..}
\href{/docs/reference/foundations/array/}{array} , }

) -\textgreater{} \href{/docs/reference/foundations/content/}{content}

\subsubsection{\texorpdfstring{\texttt{\ fill\ }}{ fill }}\label{parameters-fill}

\href{/docs/reference/foundations/none/}{none} {or}
\href{/docs/reference/visualize/color/}{color} {or}
\href{/docs/reference/visualize/gradient/}{gradient} {or}
\href{/docs/reference/visualize/pattern/}{pattern}

{{ Settable }}

\phantomsection\label{parameters-fill-settable-tooltip}
Settable parameters can be customized for all following uses of the
function with a \texttt{\ set\ } rule.

How to fill the polygon.

When setting a fill, the default stroke disappears. To create a
rectangle with both fill and stroke, you have to configure both.

Default: \texttt{\ }{\texttt{\ none\ }}\texttt{\ }

\subsubsection{\texorpdfstring{\texttt{\ fill-rule\ }}{ fill-rule }}\label{parameters-fill-rule}

\href{/docs/reference/foundations/str/}{str}

{{ Settable }}

\phantomsection\label{parameters-fill-rule-settable-tooltip}
Settable parameters can be customized for all following uses of the
function with a \texttt{\ set\ } rule.

The drawing rule used to fill the polygon.

See the
\href{/docs/reference/visualize/path/\#parameters-fill-rule}{path
documentation} for an example.

\begin{longtable}[]{@{}ll@{}}
\toprule\noalign{}
Variant & Details \\
\midrule\noalign{}
\endhead
\bottomrule\noalign{}
\endlastfoot
\texttt{\ "\ non-zero\ "\ } & Specifies that "inside" is computed by a
non-zero sum of signed edge crossings. \\
\texttt{\ "\ even-odd\ "\ } & Specifies that "inside" is computed by an
odd number of edge crossings. \\
\end{longtable}

Default: \texttt{\ }{\texttt{\ "non-zero"\ }}\texttt{\ }

\subsubsection{\texorpdfstring{\texttt{\ stroke\ }}{ stroke }}\label{parameters-stroke}

\href{/docs/reference/foundations/none/}{none} {or}
\href{/docs/reference/foundations/auto/}{auto} {or}
\href{/docs/reference/layout/length/}{length} {or}
\href{/docs/reference/visualize/color/}{color} {or}
\href{/docs/reference/visualize/gradient/}{gradient} {or}
\href{/docs/reference/visualize/stroke/}{stroke} {or}
\href{/docs/reference/visualize/pattern/}{pattern} {or}
\href{/docs/reference/foundations/dictionary/}{dictionary}

{{ Settable }}

\phantomsection\label{parameters-stroke-settable-tooltip}
Settable parameters can be customized for all following uses of the
function with a \texttt{\ set\ } rule.

How to \href{/docs/reference/visualize/stroke/}{stroke} the polygon.
This can be:

Can be set to \texttt{\ }{\texttt{\ none\ }}\texttt{\ } to disable the
stroke or to \texttt{\ }{\texttt{\ auto\ }}\texttt{\ } for a stroke of
\texttt{\ }{\texttt{\ 1pt\ }}\texttt{\ } black if and if only if no fill
is given.

Default: \texttt{\ }{\texttt{\ auto\ }}\texttt{\ }

\subsubsection{\texorpdfstring{\texttt{\ vertices\ }}{ vertices }}\label{parameters-vertices}

\href{/docs/reference/foundations/array/}{array}

{Required} {{ Positional }}

\phantomsection\label{parameters-vertices-positional-tooltip}
Positional parameters are specified in order, without names.

{{ Variadic }}

\phantomsection\label{parameters-vertices-variadic-tooltip}
Variadic parameters can be specified multiple times.

The vertices of the polygon. Each point is specified as an array of two
\href{/docs/reference/layout/relative/}{relative lengths} .

\subsection{\texorpdfstring{{ Definitions
}}{ Definitions }}\label{definitions}

\phantomsection\label{definitions-tooltip}
Functions and types and can have associated definitions. These are
accessed by specifying the function or type, followed by a period, and
then the definition\textquotesingle s name.

\subsubsection{\texorpdfstring{\texttt{\ regular\ }}{ regular }}\label{definitions-regular}

A regular polygon, defined by its size and number of vertices.

polygon { . } { regular } (

{ \hyperref[definitions-regular-parameters-fill]{fill :}
\href{/docs/reference/foundations/none/}{none}
\href{/docs/reference/visualize/color/}{color}
\href{/docs/reference/visualize/gradient/}{gradient}
\href{/docs/reference/visualize/pattern/}{pattern} , } {
\hyperref[definitions-regular-parameters-stroke]{stroke :}
\href{/docs/reference/foundations/none/}{none}
\href{/docs/reference/foundations/auto/}{auto}
\href{/docs/reference/layout/length/}{length}
\href{/docs/reference/visualize/color/}{color}
\href{/docs/reference/visualize/gradient/}{gradient}
\href{/docs/reference/visualize/stroke/}{stroke}
\href{/docs/reference/visualize/pattern/}{pattern}
\href{/docs/reference/foundations/dictionary/}{dictionary} , } {
\hyperref[definitions-regular-parameters-size]{size :}
\href{/docs/reference/layout/length/}{length} , } {
\hyperref[definitions-regular-parameters-vertices]{vertices :}
\href{/docs/reference/foundations/int/}{int} , }

) -\textgreater{} \href{/docs/reference/foundations/content/}{content}

\begin{verbatim}
#polygon.regular(
  fill: blue.lighten(80%),
  stroke: blue,
  size: 30pt,
  vertices: 3,
)
\end{verbatim}

\includegraphics[width=5in,height=\textheight,keepaspectratio]{/assets/docs/nSKAw-cASGAIxDorv3UyHgAAAAAAAAAA.png}

\paragraph{\texorpdfstring{\texttt{\ fill\ }}{ fill }}\label{definitions-regular-fill}

\href{/docs/reference/foundations/none/}{none} {or}
\href{/docs/reference/visualize/color/}{color} {or}
\href{/docs/reference/visualize/gradient/}{gradient} {or}
\href{/docs/reference/visualize/pattern/}{pattern}

How to fill the polygon. See the general
\href{/docs/reference/visualize/polygon/\#parameters-fill}{polygon\textquotesingle s
documentation} for more details.

\paragraph{\texorpdfstring{\texttt{\ stroke\ }}{ stroke }}\label{definitions-regular-stroke}

\href{/docs/reference/foundations/none/}{none} {or}
\href{/docs/reference/foundations/auto/}{auto} {or}
\href{/docs/reference/layout/length/}{length} {or}
\href{/docs/reference/visualize/color/}{color} {or}
\href{/docs/reference/visualize/gradient/}{gradient} {or}
\href{/docs/reference/visualize/stroke/}{stroke} {or}
\href{/docs/reference/visualize/pattern/}{pattern} {or}
\href{/docs/reference/foundations/dictionary/}{dictionary}

How to stroke the polygon. See the general
\href{/docs/reference/visualize/polygon/\#parameters-stroke}{polygon\textquotesingle s
documentation} for more details.

\paragraph{\texorpdfstring{\texttt{\ size\ }}{ size }}\label{definitions-regular-size}

\href{/docs/reference/layout/length/}{length}

The diameter of the
\href{https://en.wikipedia.org/wiki/Circumcircle}{circumcircle} of the
regular polygon.

Default: \texttt{\ }{\texttt{\ 1em\ }}\texttt{\ }

\paragraph{\texorpdfstring{\texttt{\ vertices\ }}{ vertices }}\label{definitions-regular-vertices}

\href{/docs/reference/foundations/int/}{int}

The number of vertices in the polygon.

Default: \texttt{\ }{\texttt{\ 3\ }}\texttt{\ }

\href{/docs/reference/visualize/pattern/}{\pandocbounded{\includesvg[keepaspectratio]{/assets/icons/16-arrow-right.svg}}}

{ Pattern } { Previous page }

\href{/docs/reference/visualize/rect/}{\pandocbounded{\includesvg[keepaspectratio]{/assets/icons/16-arrow-right.svg}}}

{ Rectangle } { Next page }

\textbf{On this page}

\begin{itemize}
\tightlist
\item
  \hyperref[summary]{Summary}
\item
  \hyperref[example]{Example}
\item
  \hyperref[parameters]{Parameters}

  \begin{itemize}
  \tightlist
  \item
    \hyperref[parameters-fill]{fill}
  \item
    \hyperref[parameters-fill-rule]{fill-rule}
  \item
    \hyperref[parameters-stroke]{stroke}
  \item
    \hyperref[parameters-vertices]{vertices}
  \end{itemize}
\item
  \hyperref[definitions]{Definitions}

  \begin{itemize}
  \tightlist
  \item
    \hyperref[definitions-regular]{Regular Polygon}

    \begin{itemize}
    \tightlist
    \item
      \hyperref[definitions-regular-fill]{fill}
    \item
      \hyperref[definitions-regular-stroke]{stroke}
    \item
      \hyperref[definitions-regular-size]{size}
    \item
      \hyperref[definitions-regular-vertices]{vertices}
    \end{itemize}
  \end{itemize}
\end{itemize}

\begin{itemize}
\tightlist
\item
  \href{/}{Home}
\item
  \href{/pricing/}{Pricing}
\item
  \href{/docs/}{Documentation}
\item
  \href{/universe/}{Universe}
\item
  \href{/about/}{About Us}
\item
  \href{/contact/}{Contact Us}
\item
  \href{/privacy/}{Privacy}
\item
  \href{https://typst.app/terms}{Terms and Conditions}
\item
  \href{/legal/}{Legal (Impressum)}
\end{itemize}

\begin{itemize}
\tightlist
\item
  \href{https://forum.typst.app}{Forum}
\item
  \href{/tools/}{Tools}
\item
  \href{/blog/}{Blog}
\item
  \href{https://github.com/typst/}{GitHub}
\item
  \href{https://discord.gg/2uDybryKPe}{Discord}
\item
  \href{https://mastodon.social/@typst}{Mastodon}
\item
  \href{https://bsky.app/profile/typst.app}{Bluesky}
\item
  \href{https://www.linkedin.com/company/typst/}{LinkedIn}
\item
  \href{https://instagram.com/typstapp/}{Instagram}
\end{itemize}

Made in Berlin
