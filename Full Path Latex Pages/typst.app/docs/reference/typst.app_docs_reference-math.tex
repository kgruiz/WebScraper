\title{typst.app/docs/reference/math}

\begin{itemize}
\tightlist
\item
  \href{/docs}{\includesvg[width=0.16667in,height=0.16667in]{/assets/icons/16-docs-dark.svg}}
\item
  \includesvg[width=0.16667in,height=0.16667in]{/assets/icons/16-arrow-right.svg}
\item
  \href{/docs/reference/}{Reference}
\item
  \includesvg[width=0.16667in,height=0.16667in]{/assets/icons/16-arrow-right.svg}
\item
  \href{/docs/reference/math/}{Math}
\end{itemize}

\section{Math}\label{summary}

Typst has special \href{/docs/reference/syntax/\#math}{syntax} and
library functions to typeset mathematical formulas. Math formulas can be
displayed inline with text or as separate blocks. They will be typeset
into their own block if they start and end with at least one space (e.g.
\texttt{\ }{\texttt{\ \$\ }}\texttt{\ x\ }{\texttt{\ \^{}\ }}\texttt{\ 2\ }{\texttt{\ \$\ }}\texttt{\ }
).

\subsection{Variables}\label{variables}

In math, single letters are always displayed as is. Multiple letters,
however, are interpreted as variables and functions. To display multiple
letters verbatim, you can place them into quotes and to access single
letter variables, you can use the
\href{/docs/reference/scripting/\#expressions}{hash syntax} .

\begin{verbatim}
$ A = pi r^2 $
$ "area" = pi dot "radius"^2 $
$ cal(A) :=
    { x in RR | x "is natural" } $
#let x = 5
$ #x < 17 $
\end{verbatim}

\includegraphics[width=5in,height=\textheight,keepaspectratio]{/assets/docs/hSTnanxnhN2cMLti2SpIlwAAAAAAAAAA.png}

\subsection{Symbols}\label{symbols}

Math mode makes a wide selection of
\href{/docs/reference/symbols/sym/}{symbols} like \texttt{\ pi\ } ,
\texttt{\ dot\ } , or \texttt{\ RR\ } available. Many mathematical
symbols are available in different variants. You can select between
different variants by applying
\href{/docs/reference/symbols/symbol/}{modifiers} to the symbol. Typst
further recognizes a number of shorthand sequences like
\texttt{\ =\textgreater{}\ } that approximate a symbol. When such a
shorthand exists, the symbol\textquotesingle s documentation lists it.

\begin{verbatim}
$ x < y => x gt.eq.not y $
\end{verbatim}

\includegraphics[width=5in,height=\textheight,keepaspectratio]{/assets/docs/3QjDlBq8e4sckxD76_cbbgAAAAAAAAAA.png}

\subsection{Line Breaks}\label{line-breaks}

Formulas can also contain line breaks. Each line can contain one or
multiple \emph{alignment points} ( \texttt{\ \&\ } ) which are then
aligned.

\begin{verbatim}
$ sum_(k=0)^n k
    &= 1 + ... + n \
    &= (n(n+1)) / 2 $
\end{verbatim}

\includegraphics[width=5in,height=\textheight,keepaspectratio]{/assets/docs/4Y4RfouYZm3Jgju-7W3SZAAAAAAAAAAA.png}

\subsection{Function calls}\label{function-calls}

Math mode supports special function calls without the hash prefix. In
these "math calls", the argument list works a little differently than in
code:

\begin{itemize}
\tightlist
\item
  Within them, Typst is still in "math mode". Thus, you can write math
  directly into them, but need to use hash syntax to pass code
  expressions (except for strings, which are available in the math
  syntax).
\item
  They support positional and named arguments, but don\textquotesingle t
  support trailing content blocks and argument spreading.
\item
  They provide additional syntax for 2-dimensional argument lists. The
  semicolon ( \texttt{\ ;\ } ) merges preceding arguments separated by
  commas into an array argument.
\end{itemize}

\begin{verbatim}
$ frac(a^2, 2) $
$ vec(1, 2, delim: "[") $
$ mat(1, 2; 3, 4) $
$ lim_x =
    op("lim", limits: #true)_x $
\end{verbatim}

\includegraphics[width=5in,height=\textheight,keepaspectratio]{/assets/docs/gWTBh8i7ZWskmajIpEpUWQAAAAAAAAAA.png}

To write a verbatim comma or semicolon in a math call, escape it with a
backslash. The colon on the other hand is only recognized in a special
way if directly preceded by an identifier, so to display it verbatim in
those cases, you can just insert a space before it.

Functions calls preceded by a hash are normal code function calls and
not affected by these rules.

\subsection{Alignment}\label{alignment}

When equations include multiple \emph{alignment points} (
\texttt{\ \&\ } ), this creates blocks of alternatingly right- and
left-aligned columns. In the example below, the expression
\texttt{\ (3x\ +\ y)\ /\ 7\ } is right-aligned and \texttt{\ =\ 9\ } is
left-aligned. The word "given" is also left-aligned because
\texttt{\ \&\&\ } creates two alignment points in a row, alternating the
alignment twice. \texttt{\ \&\ \&\ } and \texttt{\ \&\&\ } behave
exactly the same way. Meanwhile, "multiply by 7" is right-aligned
because just one \texttt{\ \&\ } precedes it. Each alignment point
simply alternates between right-aligned/left-aligned.

\begin{verbatim}
$ (3x + y) / 7 &= 9 && "given" \
  3x + y &= 63 & "multiply by 7" \
  3x &= 63 - y && "subtract y" \
  x &= 21 - y/3 & "divide by 3" $
\end{verbatim}

\includegraphics[width=5in,height=\textheight,keepaspectratio]{/assets/docs/8SM9qVyRZ_Elks_C9882dAAAAAAAAAAA.png}

\subsection{Math fonts}\label{math-fonts}

You can set the math font by with a
\href{/docs/reference/styling/\#show-rules}{show-set rule} as
demonstrated below. Note that only special OpenType math fonts are
suitable for typesetting maths.

\begin{verbatim}
#show math.equation: set text(font: "Fira Math")
$ sum_(i in NN) 1 + i $
\end{verbatim}

\includegraphics[width=5in,height=\textheight,keepaspectratio]{/assets/docs/qG9Xcf2X5Ju0E76URIxfZgAAAAAAAAAA.png}

\subsection{Math module}\label{math-module}

All math functions are part of the \texttt{\ math\ }
\href{/docs/reference/scripting/\#modules}{module} , which is available
by default in equations. Outside of equations, they can be accessed with
the \texttt{\ math.\ } prefix.

\subsection{Definitions}\label{definitions}

\begin{itemize}
\tightlist
\item
  \href{/docs/reference/math/accent/}{\texttt{\ accent\ }} { Attaches an
  accent to a base. }
\item
  \href{/docs/reference/math/attach}{attach} { Subscript, superscripts,
  and limits. }
\item
  \href{/docs/reference/math/binom/}{\texttt{\ binom\ }} { A binomial
  expression. }
\item
  \href{/docs/reference/math/cancel/}{\texttt{\ cancel\ }} { Displays a
  diagonal line over a part of an equation. }
\item
  \href{/docs/reference/math/cases/}{\texttt{\ cases\ }} { A case
  distinction. }
\item
  \href{/docs/reference/math/class/}{\texttt{\ class\ }} { Forced use of
  a certain math class. }
\item
  \href{/docs/reference/math/equation/}{\texttt{\ equation\ }} { A
  mathematical equation. }
\item
  \href{/docs/reference/math/frac/}{\texttt{\ frac\ }} { A mathematical
  fraction. }
\item
  \href{/docs/reference/math/lr}{lr} { Delimiter matching. }
\item
  \href{/docs/reference/math/mat/}{\texttt{\ mat\ }} { A matrix. }
\item
  \href{/docs/reference/math/op/}{\texttt{\ op\ }} { A text operator in
  an equation. }
\item
  \href{/docs/reference/math/primes/}{\texttt{\ primes\ }} { Grouped
  primes. }
\item
  \href{/docs/reference/math/roots}{roots} { Square and non-square
  roots. }
\item
  \href{/docs/reference/math/sizes}{sizes} { Forced size styles for
  expressions within formulas. }
\item
  \href{/docs/reference/math/stretch/}{\texttt{\ stretch\ }} { Stretches
  a glyph. }
\item
  \href{/docs/reference/math/styles}{styles} { Alternate letterforms
  within formulas. }
\item
  \href{/docs/reference/math/underover}{underover} { Delimiters above or
  below parts of an equation. }
\item
  \href{/docs/reference/math/variants}{variants} { Alternate typefaces
  within formulas. }
\item
  \href{/docs/reference/math/vec/}{\texttt{\ vec\ }} { A column vector.
  }
\end{itemize}

\href{/docs/reference/text/upper/}{\pandocbounded{\includesvg[keepaspectratio]{/assets/icons/16-arrow-right.svg}}}

{ Uppercase } { Previous page }

\href{/docs/reference/math/accent/}{\pandocbounded{\includesvg[keepaspectratio]{/assets/icons/16-arrow-right.svg}}}

{ Accent } { Next page }

\textbf{On this page}

\begin{itemize}
\tightlist
\item
  \hyperref[summary]{Summary}
\item
  \hyperref[variables]{Variables}
\item
  \hyperref[symbols]{Symbols}
\item
  \hyperref[line-breaks]{Line Breaks}
\item
  \hyperref[function-calls]{Function Calls}
\item
  \hyperref[alignment]{Alignment}
\item
  \hyperref[math-fonts]{Math Fonts}
\item
  \hyperref[math-module]{Math Module}
\item
  \hyperref[definitions]{Definitions}
\end{itemize}

\begin{itemize}
\tightlist
\item
  \href{/}{Home}
\item
  \href{/pricing/}{Pricing}
\item
  \href{/docs/}{Documentation}
\item
  \href{/universe/}{Universe}
\item
  \href{/about/}{About Us}
\item
  \href{/contact/}{Contact Us}
\item
  \href{/privacy/}{Privacy}
\item
  \href{https://typst.app/terms}{Terms and Conditions}
\item
  \href{/legal/}{Legal (Impressum)}
\end{itemize}

\begin{itemize}
\tightlist
\item
  \href{https://forum.typst.app}{Forum}
\item
  \href{/tools/}{Tools}
\item
  \href{/blog/}{Blog}
\item
  \href{https://github.com/typst/}{GitHub}
\item
  \href{https://discord.gg/2uDybryKPe}{Discord}
\item
  \href{https://mastodon.social/@typst}{Mastodon}
\item
  \href{https://bsky.app/profile/typst.app}{Bluesky}
\item
  \href{https://www.linkedin.com/company/typst/}{LinkedIn}
\item
  \href{https://instagram.com/typstapp/}{Instagram}
\end{itemize}

Made in Berlin
