\title{typst.app/docs/reference/introspection/metadata}

\begin{itemize}
\tightlist
\item
  \href{/docs}{\includesvg[width=0.16667in,height=0.16667in]{/assets/icons/16-docs-dark.svg}}
\item
  \includesvg[width=0.16667in,height=0.16667in]{/assets/icons/16-arrow-right.svg}
\item
  \href{/docs/reference/}{Reference}
\item
  \includesvg[width=0.16667in,height=0.16667in]{/assets/icons/16-arrow-right.svg}
\item
  \href{/docs/reference/introspection/}{Introspection}
\item
  \includesvg[width=0.16667in,height=0.16667in]{/assets/icons/16-arrow-right.svg}
\item
  \href{/docs/reference/introspection/metadata/}{Metadata}
\end{itemize}

\section{\texorpdfstring{\texttt{\ metadata\ } {{ Element
}}}{ metadata   Element }}\label{summary}

\phantomsection\label{element-tooltip}
Element functions can be customized with \texttt{\ set\ } and
\texttt{\ show\ } rules.

Exposes a value to the query system without producing visible content.

This element can be retrieved with the
\href{/docs/reference/introspection/query/}{\texttt{\ query\ }} function
and from the command line with
\href{/docs/reference/introspection/query/\#command-line-queries}{\texttt{\ typst\ query\ }}
. Its purpose is to expose an arbitrary value to the introspection
system. To identify a metadata value among others, you can attach a
\href{/docs/reference/foundations/label/}{\texttt{\ label\ }} to it and
query for that label.

The \texttt{\ metadata\ } element is especially useful for command line
queries because it allows you to expose arbitrary values to the outside
world.

\begin{verbatim}
// Put metadata somewhere.
#metadata("This is a note") <note>

// And find it from anywhere else.
#context {
  query(<note>).first().value
}
\end{verbatim}

\includegraphics[width=5in,height=\textheight,keepaspectratio]{/assets/docs/sbF_Ac863-gI1m3qoL9avwAAAAAAAAAA.png}

\subsection{\texorpdfstring{{ Parameters
}}{ Parameters }}\label{parameters}

\phantomsection\label{parameters-tooltip}
Parameters are the inputs to a function. They are specified in
parentheses after the function name.

{ metadata } (

{ { any } }

) -\textgreater{} \href{/docs/reference/foundations/content/}{content}

\subsubsection{\texorpdfstring{\texttt{\ value\ }}{ value }}\label{parameters-value}

{ any }

{Required} {{ Positional }}

\phantomsection\label{parameters-value-positional-tooltip}
Positional parameters are specified in order, without names.

The value to embed into the document.

\href{/docs/reference/introspection/location/}{\pandocbounded{\includesvg[keepaspectratio]{/assets/icons/16-arrow-right.svg}}}

{ Location } { Previous page }

\href{/docs/reference/introspection/query/}{\pandocbounded{\includesvg[keepaspectratio]{/assets/icons/16-arrow-right.svg}}}

{ Query } { Next page }
