\title{typst.app/docs/reference/introspection/here}

\begin{itemize}
\tightlist
\item
  \href{/docs}{\includesvg[width=0.16667in,height=0.16667in]{/assets/icons/16-docs-dark.svg}}
\item
  \includesvg[width=0.16667in,height=0.16667in]{/assets/icons/16-arrow-right.svg}
\item
  \href{/docs/reference/}{Reference}
\item
  \includesvg[width=0.16667in,height=0.16667in]{/assets/icons/16-arrow-right.svg}
\item
  \href{/docs/reference/introspection/}{Introspection}
\item
  \includesvg[width=0.16667in,height=0.16667in]{/assets/icons/16-arrow-right.svg}
\item
  \href{/docs/reference/introspection/here/}{Here}
\end{itemize}

\section{\texorpdfstring{\texttt{\ here\ } {{ Contextual
}}}{ here   Contextual }}\label{summary}

\phantomsection\label{contextual-tooltip}
Contextual functions can only be used when the context is known

Provides the current location in the document.

You can think of \texttt{\ here\ } as a low-level building block that
directly extracts the current location from the active
\href{/docs/reference/context/}{context} . Some other functions use it
internally: For instance,
\texttt{\ counter\ }{\texttt{\ .\ }}\texttt{\ }{\texttt{\ get\ }}\texttt{\ }{\texttt{\ (\ }}\texttt{\ }{\texttt{\ )\ }}\texttt{\ }
is equivalent to
\texttt{\ counter\ }{\texttt{\ .\ }}\texttt{\ }{\texttt{\ at\ }}\texttt{\ }{\texttt{\ (\ }}\texttt{\ }{\texttt{\ here\ }}\texttt{\ }{\texttt{\ (\ }}\texttt{\ }{\texttt{\ )\ }}\texttt{\ }{\texttt{\ )\ }}\texttt{\ }
.

Within show rules on
\href{/docs/reference/introspection/location/\#locatable}{locatable}
elements,
\texttt{\ }{\texttt{\ here\ }}\texttt{\ }{\texttt{\ (\ }}\texttt{\ }{\texttt{\ )\ }}\texttt{\ }
will match the location of the shown element.

If you want to display the current page number, refer to the
documentation of the
\href{/docs/reference/introspection/counter/}{\texttt{\ counter\ }}
type. While \texttt{\ here\ } can be used to determine the physical page
number, typically you want the logical page number that may, for
instance, have been reset after a preface.

\subsection{Examples}\label{examples}

Determining the current position in the document in combination with the
\href{/docs/reference/introspection/location/\#definitions-position}{\texttt{\ position\ }}
method:

\begin{verbatim}
#context [
  I am located at
  #here().position()
]
\end{verbatim}

\includegraphics[width=5in,height=\textheight,keepaspectratio]{/assets/docs/5PrDc8FIHOrLs_qUjTj6iwAAAAAAAAAA.png}

Running a \href{/docs/reference/introspection/query/}{query} for
elements before the current position:

\begin{verbatim}
= Introduction
= Background

There are
#context query(
  selector(heading).before(here())
).len()
headings before me.

= Conclusion
\end{verbatim}

\includegraphics[width=5in,height=\textheight,keepaspectratio]{/assets/docs/5DWH6TcZBrEjuGuSwKqf8AAAAAAAAAAA.png}

Refer to the
\href{/docs/reference/foundations/selector/}{\texttt{\ selector\ }} type
for more details on before/after selectors.

\subsection{\texorpdfstring{{ Parameters
}}{ Parameters }}\label{parameters}

\phantomsection\label{parameters-tooltip}
Parameters are the inputs to a function. They are specified in
parentheses after the function name.

{ here } (

) -\textgreater{}
\href{/docs/reference/introspection/location/}{location}

\href{/docs/reference/introspection/counter/}{\pandocbounded{\includesvg[keepaspectratio]{/assets/icons/16-arrow-right.svg}}}

{ Counter } { Previous page }

\href{/docs/reference/introspection/locate/}{\pandocbounded{\includesvg[keepaspectratio]{/assets/icons/16-arrow-right.svg}}}

{ Locate } { Next page }
