\title{typst.app/docs/reference/layout/fraction}

\begin{itemize}
\tightlist
\item
  \href{/docs}{\includesvg[width=0.16667in,height=0.16667in]{/assets/icons/16-docs-dark.svg}}
\item
  \includesvg[width=0.16667in,height=0.16667in]{/assets/icons/16-arrow-right.svg}
\item
  \href{/docs/reference/}{Reference}
\item
  \includesvg[width=0.16667in,height=0.16667in]{/assets/icons/16-arrow-right.svg}
\item
  \href{/docs/reference/layout/}{Layout}
\item
  \includesvg[width=0.16667in,height=0.16667in]{/assets/icons/16-arrow-right.svg}
\item
  \href{/docs/reference/layout/fraction/}{Fraction}
\end{itemize}

\section{\texorpdfstring{{ fraction }}{ fraction }}\label{summary}

Defines how the remaining space in a layout is distributed.

Each fractionally sized element gets space based on the ratio of its
fraction to the sum of all fractions.

For more details, also see the \href{/docs/reference/layout/h/}{h} and
\href{/docs/reference/layout/v/}{v} functions and the
\href{/docs/reference/layout/grid/}{grid function} .

\subsection{Example}\label{example}

\begin{verbatim}
Left #h(1fr) Left-ish #h(2fr) Right
\end{verbatim}

\includegraphics[width=5in,height=\textheight,keepaspectratio]{/assets/docs/Mh5sjFkAJFlbM1vm_65COgAAAAAAAAAA.png}

\href{/docs/reference/layout/direction/}{\pandocbounded{\includesvg[keepaspectratio]{/assets/icons/16-arrow-right.svg}}}

{ Direction } { Previous page }

\href{/docs/reference/layout/grid/}{\pandocbounded{\includesvg[keepaspectratio]{/assets/icons/16-arrow-right.svg}}}

{ Grid } { Next page }
