\title{typst.app/docs/reference/layout/length}

\begin{itemize}
\tightlist
\item
  \href{/docs}{\includesvg[width=0.16667in,height=0.16667in]{/assets/icons/16-docs-dark.svg}}
\item
  \includesvg[width=0.16667in,height=0.16667in]{/assets/icons/16-arrow-right.svg}
\item
  \href{/docs/reference/}{Reference}
\item
  \includesvg[width=0.16667in,height=0.16667in]{/assets/icons/16-arrow-right.svg}
\item
  \href{/docs/reference/layout/}{Layout}
\item
  \includesvg[width=0.16667in,height=0.16667in]{/assets/icons/16-arrow-right.svg}
\item
  \href{/docs/reference/layout/length/}{Length}
\end{itemize}

\section{\texorpdfstring{{ length }}{ length }}\label{summary}

A size or distance, possibly expressed with contextual units.

Typst supports the following length units:

\begin{itemize}
\tightlist
\item
  Points: \texttt{\ }{\texttt{\ 72pt\ }}\texttt{\ }
\item
  Millimeters: \texttt{\ }{\texttt{\ 254mm\ }}\texttt{\ }
\item
  Centimeters: \texttt{\ }{\texttt{\ 2.54cm\ }}\texttt{\ }
\item
  Inches: \texttt{\ }{\texttt{\ 1in\ }}\texttt{\ }
\item
  Relative to font size: \texttt{\ }{\texttt{\ 2.5em\ }}\texttt{\ }
\end{itemize}

You can multiply lengths with and divide them by integers and floats.

\subsection{Example}\label{example}

\begin{verbatim}
#rect(width: 20pt)
#rect(width: 2em)
#rect(width: 1in)

#(3em + 5pt).em \
#(20pt).em \
#(40em + 2pt).abs \
#(5em).abs
\end{verbatim}

\includegraphics[width=5in,height=\textheight,keepaspectratio]{/assets/docs/gpwKHS7y2wIB7BIxGEXoMwAAAAAAAAAA.png}

\subsection{Fields}\label{fields}

\begin{itemize}
\tightlist
\item
  \texttt{\ abs\ } : A length with just the absolute component of the
  current length (that is, excluding the \texttt{\ em\ } component).
\item
  \texttt{\ em\ } : The amount of \texttt{\ em\ } units in this length,
  as a \href{/docs/reference/foundations/float/}{float} .
\end{itemize}

\subsection{\texorpdfstring{{ Definitions
}}{ Definitions }}\label{definitions}

\phantomsection\label{definitions-tooltip}
Functions and types and can have associated definitions. These are
accessed by specifying the function or type, followed by a period, and
then the definition\textquotesingle s name.

\subsubsection{\texorpdfstring{\texttt{\ pt\ }}{ pt }}\label{definitions-pt}

Converts this length to points.

Fails with an error if this length has non-zero \texttt{\ em\ } units
(such as \texttt{\ 5em\ +\ 2pt\ } instead of just \texttt{\ 2pt\ } ).
Use the \texttt{\ abs\ } field (such as in
\texttt{\ (5em\ +\ 2pt).abs.pt()\ } ) to ignore the \texttt{\ em\ }
component of the length (thus converting only its absolute component).

self { . } { pt } (

) -\textgreater{} \href{/docs/reference/foundations/float/}{float}

\subsubsection{\texorpdfstring{\texttt{\ mm\ }}{ mm }}\label{definitions-mm}

Converts this length to millimeters.

Fails with an error if this length has non-zero \texttt{\ em\ } units.
See the
\href{/docs/reference/layout/length/\#definitions-pt}{\texttt{\ pt\ }}
method for more details.

self { . } { mm } (

) -\textgreater{} \href{/docs/reference/foundations/float/}{float}

\subsubsection{\texorpdfstring{\texttt{\ cm\ }}{ cm }}\label{definitions-cm}

Converts this length to centimeters.

Fails with an error if this length has non-zero \texttt{\ em\ } units.
See the
\href{/docs/reference/layout/length/\#definitions-pt}{\texttt{\ pt\ }}
method for more details.

self { . } { cm } (

) -\textgreater{} \href{/docs/reference/foundations/float/}{float}

\subsubsection{\texorpdfstring{\texttt{\ inches\ }}{ inches }}\label{definitions-inches}

Converts this length to inches.

Fails with an error if this length has non-zero \texttt{\ em\ } units.
See the
\href{/docs/reference/layout/length/\#definitions-pt}{\texttt{\ pt\ }}
method for more details.

self { . } { inches } (

) -\textgreater{} \href{/docs/reference/foundations/float/}{float}

\subsubsection{\texorpdfstring{\texttt{\ to-absolute\ }}{ to-absolute }}\label{definitions-to-absolute}

Resolve this length to an absolute length.

self { . } { to-absolute } (

) -\textgreater{} \href{/docs/reference/layout/length/}{length}

\begin{verbatim}
#set text(size: 12pt)
#context [
  #(6pt).to-absolute() \
  #(6pt + 10em).to-absolute() \
  #(10em).to-absolute()
]

#set text(size: 6pt)
#context [
  #(6pt).to-absolute() \
  #(6pt + 10em).to-absolute() \
  #(10em).to-absolute()
]
\end{verbatim}

\includegraphics[width=5in,height=\textheight,keepaspectratio]{/assets/docs/O8f4mxTZz-ziS7eclGAyvgAAAAAAAAAA.png}

\href{/docs/reference/layout/layout/}{\pandocbounded{\includesvg[keepaspectratio]{/assets/icons/16-arrow-right.svg}}}

{ Layout } { Previous page }

\href{/docs/reference/layout/measure/}{\pandocbounded{\includesvg[keepaspectratio]{/assets/icons/16-arrow-right.svg}}}

{ Measure } { Next page }
