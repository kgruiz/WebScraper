\title{typst.app/docs/reference/styling}

\begin{itemize}
\tightlist
\item
  \href{/docs}{\includesvg[width=0.16667in,height=0.16667in]{/assets/icons/16-docs-dark.svg}}
\item
  \includesvg[width=0.16667in,height=0.16667in]{/assets/icons/16-arrow-right.svg}
\item
  \href{/docs/reference/}{Reference}
\item
  \includesvg[width=0.16667in,height=0.16667in]{/assets/icons/16-arrow-right.svg}
\item
  \href{/docs/reference/styling/}{Styling}
\end{itemize}

\section{Styling}\label{styling}

Typst includes a flexible styling system that automatically applies
styling of your choice to your document. With \emph{set rules,} you can
configure basic properties of elements. This way, you create most common
styles. However, there might not be a built-in property for everything
you wish to do. For this reason, Typst further supports \emph{show
rules} that can completely redefine the appearance of elements.

\subsection{Set rules}\label{set-rules}

With set rules, you can customize the appearance of elements. They are
written as a \href{/docs/reference/foundations/function/}{function call}
to an
\href{/docs/reference/foundations/function/\#element-functions}{element
function} preceded by the \texttt{\ }{\texttt{\ set\ }}\texttt{\ }
keyword (or
\texttt{\ }{\texttt{\ \#\ }}\texttt{\ }{\texttt{\ set\ }}\texttt{\ } in
markup). Only optional parameters of that function can be provided to
the set rule. Refer to each function\textquotesingle s documentation to
see which parameters are optional. In the example below, we use two set
rules to change the
\href{/docs/reference/text/text/\#parameters-font}{font family} and
\href{/docs/reference/model/heading/\#parameters-numbering}{heading
numbering} .

\begin{verbatim}
#set heading(numbering: "I.")
#set text(
  font: "New Computer Modern"
)

= Introduction
With set rules, you can style
your document.
\end{verbatim}

\includegraphics[width=5in,height=\textheight,keepaspectratio]{/assets/docs/nW0VZeyhJmtpweEOjJR_fgAAAAAAAAAA.png}

A top level set rule stays in effect until the end of the file. When
nested inside of a block, it is only in effect until the end of that
block. With a block, you can thus restrict the effect of a rule to a
particular segment of your document. Below, we use a content block to
scope the list styling to one particular list.

\begin{verbatim}
This list is affected: #[
  #set list(marker: [--])
  - Dash
]

This one is not:
- Bullet
\end{verbatim}

\includegraphics[width=5in,height=\textheight,keepaspectratio]{/assets/docs/6ckQbXFff1zDBcdWezXxpgAAAAAAAAAA.png}

Sometimes, you\textquotesingle ll want to apply a set rule
conditionally. For this, you can use a \emph{set-if} rule.

\begin{verbatim}
#let task(body, critical: false) = {
  set text(red) if critical
  [- #body]
}

#task(critical: true)[Food today?]
#task(critical: false)[Work deadline]
\end{verbatim}

\includegraphics[width=5in,height=\textheight,keepaspectratio]{/assets/docs/_UlmqEOdrmM6d-OQ9TsAXwAAAAAAAAAA.png}

\subsection{Show rules}\label{show-rules}

With show rules, you can deeply customize the look of a type of element.
The most basic form of show rule is a \emph{show-set rule.} Such a rule
is written as the \texttt{\ }{\texttt{\ show\ }}\texttt{\ } keyword
followed by a \href{/docs/reference/foundations/selector/}{selector} , a
colon and then a set rule. The most basic form of selector is an
\href{/docs/reference/foundations/function/\#element-functions}{element
function} . This lets the set rule only apply to the selected element.
In the example below, headings become dark blue while all other text
stays black.

\begin{verbatim}
#show heading: set text(navy)

= This is navy-blue
But this stays black.
\end{verbatim}

\includegraphics[width=5in,height=\textheight,keepaspectratio]{/assets/docs/DS2Pe3XVhhMMVWT9eUfjSQAAAAAAAAAA.png}

With show-set rules you can mix and match properties from different
functions to achieve many different effects. But they still limit you to
what is predefined in Typst. For maximum flexibility, you can instead
write a show rule that defines how to format an element from scratch. To
write such a show rule, replace the set rule after the colon with an
arbitrary \href{/docs/reference/foundations/function/}{function} . This
function receives the element in question and can return arbitrary
content. The available \href{/docs/reference/scripting/\#fields}{fields}
on the element passed to the function again match the parameters of the
respective element function. Below, we define a show rule that formats
headings for a fantasy encyclopedia.

\begin{verbatim}
#set heading(numbering: "(I)")
#show heading: it => [
  #set align(center)
  #set text(font: "Inria Serif")
  \~ #emph(it.body)
     #counter(heading).display(
       it.numbering
     ) \~
]

= Dragon
With a base health of 15, the
dragon is the most powerful
creature.

= Manticore
While less powerful than the
dragon, the manticore gets
extra style points.
\end{verbatim}

\includegraphics[width=5in,height=\textheight,keepaspectratio]{/assets/docs/YrvkqpSwoILjuqAerzw9CAAAAAAAAAAA.png}

Like set rules, show rules are in effect until the end of the current
block or file.

Instead of a function, the right-hand side of a show rule can also take
a literal string or content block that should be directly substituted
for the element. And apart from a function, the left-hand side of a show
rule can also take a number of other \emph{selectors} that define what
to apply the transformation to:

\begin{itemize}
\item
  \textbf{Everything:}
  \texttt{\ }{\texttt{\ show\ }}\texttt{\ }{\texttt{\ :\ }}\texttt{\ rest\ }{\texttt{\ =\textgreater{}\ }}\texttt{\ ..\ }\\
  Transform everything after the show rule. This is useful to apply a
  more complex layout to your whole document without wrapping everything
  in a giant function call.
\item
  \textbf{Text:}
  \texttt{\ }{\texttt{\ show\ }}\texttt{\ }{\texttt{\ "Text"\ }}\texttt{\ }{\texttt{\ :\ }}\texttt{\ ..\ }\\
  Style, transform or replace text.
\item
  \textbf{Regex:}
  \texttt{\ }{\texttt{\ show\ }}\texttt{\ }{\texttt{\ regex\ }}\texttt{\ }{\texttt{\ (\ }}\texttt{\ }{\texttt{\ "\textbackslash{}w+"\ }}\texttt{\ }{\texttt{\ )\ }}\texttt{\ }{\texttt{\ :\ }}\texttt{\ ..\ }\\
  Select and transform text with a regular expression for even more
  flexibility. See the documentation of the
  \href{/docs/reference/foundations/regex/}{\texttt{\ regex\ } type} for
  details.
\item
  \textbf{Function with fields:}
  \texttt{\ }{\texttt{\ show\ }}\texttt{\ heading\ }{\texttt{\ .\ }}\texttt{\ }{\texttt{\ where\ }}\texttt{\ }{\texttt{\ (\ }}\texttt{\ level\ }{\texttt{\ :\ }}\texttt{\ }{\texttt{\ 1\ }}\texttt{\ }{\texttt{\ )\ }}\texttt{\ }{\texttt{\ :\ }}\texttt{\ ..\ }\\
  Transform only elements that have the specified fields. For example,
  you might want to only change the style of level-1 headings.
\item
  \textbf{Label:}
  \texttt{\ }{\texttt{\ show\ }}\texttt{\ }{\texttt{\ \textless{}intro\textgreater{}\ }}\texttt{\ }{\texttt{\ :\ }}\texttt{\ ..\ }\\
  Select and transform elements that have the specified label. See the
  documentation of the
  \href{/docs/reference/foundations/label/}{\texttt{\ label\ } type} for
  more details.
\end{itemize}

\begin{verbatim}
#show "Project": smallcaps
#show "badly": "great"

We started Project in 2019
and are still working on it.
Project is progressing badly.
\end{verbatim}

\includegraphics[width=5in,height=\textheight,keepaspectratio]{/assets/docs/NBzIViTspdnPhsbo3WGDLAAAAAAAAAAA.png}

\href{/docs/reference/syntax/}{\pandocbounded{\includesvg[keepaspectratio]{/assets/icons/16-arrow-right.svg}}}

{ Syntax } { Previous page }

\href{/docs/reference/scripting/}{\pandocbounded{\includesvg[keepaspectratio]{/assets/icons/16-arrow-right.svg}}}

{ Scripting } { Next page }
