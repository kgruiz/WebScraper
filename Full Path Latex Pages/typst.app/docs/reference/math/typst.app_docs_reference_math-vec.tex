\title{typst.app/docs/reference/math/vec}

\begin{itemize}
\tightlist
\item
  \href{/docs}{\includesvg[width=0.16667in,height=0.16667in]{/assets/icons/16-docs-dark.svg}}
\item
  \includesvg[width=0.16667in,height=0.16667in]{/assets/icons/16-arrow-right.svg}
\item
  \href{/docs/reference/}{Reference}
\item
  \includesvg[width=0.16667in,height=0.16667in]{/assets/icons/16-arrow-right.svg}
\item
  \href{/docs/reference/math/}{Math}
\item
  \includesvg[width=0.16667in,height=0.16667in]{/assets/icons/16-arrow-right.svg}
\item
  \href{/docs/reference/math/vec/}{Vector}
\end{itemize}

\section{\texorpdfstring{\texttt{\ vec\ } {{ Element
}}}{ vec   Element }}\label{summary}

\phantomsection\label{element-tooltip}
Element functions can be customized with \texttt{\ set\ } and
\texttt{\ show\ } rules.

A column vector.

Content in the vector\textquotesingle s elements can be aligned with the
\href{/docs/reference/math/vec/\#parameters-align}{\texttt{\ align\ }}
parameter, or the \texttt{\ \&\ } symbol.

\subsection{Example}\label{example}

\begin{verbatim}
$ vec(a, b, c) dot vec(1, 2, 3)
    = a + 2b + 3c $
\end{verbatim}

\includegraphics[width=5in,height=\textheight,keepaspectratio]{/assets/docs/LnRm06lLMggD8fCQZdA66QAAAAAAAAAA.png}

\subsection{\texorpdfstring{{ Parameters
}}{ Parameters }}\label{parameters}

\phantomsection\label{parameters-tooltip}
Parameters are the inputs to a function. They are specified in
parentheses after the function name.

math { . } { vec } (

{ \hyperref[parameters-delim]{delim :}
\href{/docs/reference/foundations/none/}{none}
\href{/docs/reference/foundations/str/}{str}
\href{/docs/reference/foundations/array/}{array}
\href{/docs/reference/symbols/symbol/}{symbol} , } {
\hyperref[parameters-align]{align :}
\href{/docs/reference/layout/alignment/}{alignment} , } {
\hyperref[parameters-gap]{gap :}
\href{/docs/reference/layout/relative/}{relative} , } {
\hyperref[parameters-children]{..}
\href{/docs/reference/foundations/content/}{content} , }

) -\textgreater{} \href{/docs/reference/foundations/content/}{content}

\subsubsection{\texorpdfstring{\texttt{\ delim\ }}{ delim }}\label{parameters-delim}

\href{/docs/reference/foundations/none/}{none} {or}
\href{/docs/reference/foundations/str/}{str} {or}
\href{/docs/reference/foundations/array/}{array} {or}
\href{/docs/reference/symbols/symbol/}{symbol}

{{ Settable }}

\phantomsection\label{parameters-delim-settable-tooltip}
Settable parameters can be customized for all following uses of the
function with a \texttt{\ set\ } rule.

The delimiter to use.

Can be a single character specifying the left delimiter, in which case
the right delimiter is inferred. Otherwise, can be an array containing a
left and a right delimiter.

Default:
\texttt{\ }{\texttt{\ (\ }}\texttt{\ }{\texttt{\ "("\ }}\texttt{\ }{\texttt{\ ,\ }}\texttt{\ }{\texttt{\ ")"\ }}\texttt{\ }{\texttt{\ )\ }}\texttt{\ }

\includesvg[width=0.16667in,height=0.16667in]{/assets/icons/16-arrow-right.svg}
View example

\begin{verbatim}
#set math.vec(delim: "[")
$ vec(1, 2) $
\end{verbatim}

\includegraphics[width=5in,height=\textheight,keepaspectratio]{/assets/docs/5LFZJ9d25bljXFp6kARHcgAAAAAAAAAA.png}

\subsubsection{\texorpdfstring{\texttt{\ align\ }}{ align }}\label{parameters-align}

\href{/docs/reference/layout/alignment/}{alignment}

{{ Settable }}

\phantomsection\label{parameters-align-settable-tooltip}
Settable parameters can be customized for all following uses of the
function with a \texttt{\ set\ } rule.

The horizontal alignment that each element should have.

Default: \texttt{\ center\ }

\includesvg[width=0.16667in,height=0.16667in]{/assets/icons/16-arrow-right.svg}
View example

\begin{verbatim}
#set math.vec(align: right)
$ vec(-1, 1, -1) $
\end{verbatim}

\includegraphics[width=5in,height=\textheight,keepaspectratio]{/assets/docs/ZtHlp9Y4zEtz53Ydf5unLAAAAAAAAAAA.png}

\subsubsection{\texorpdfstring{\texttt{\ gap\ }}{ gap }}\label{parameters-gap}

\href{/docs/reference/layout/relative/}{relative}

{{ Settable }}

\phantomsection\label{parameters-gap-settable-tooltip}
Settable parameters can be customized for all following uses of the
function with a \texttt{\ set\ } rule.

The gap between elements.

Default:
\texttt{\ }{\texttt{\ 0\%\ }}\texttt{\ }{\texttt{\ +\ }}\texttt{\ }{\texttt{\ 0.2em\ }}\texttt{\ }

\includesvg[width=0.16667in,height=0.16667in]{/assets/icons/16-arrow-right.svg}
View example

\begin{verbatim}
#set math.vec(gap: 1em)
$ vec(1, 2) $
\end{verbatim}

\includegraphics[width=5in,height=\textheight,keepaspectratio]{/assets/docs/uiK2bQUKjIzcO3IGp7RZPwAAAAAAAAAA.png}

\subsubsection{\texorpdfstring{\texttt{\ children\ }}{ children }}\label{parameters-children}

\href{/docs/reference/foundations/content/}{content}

{Required} {{ Positional }}

\phantomsection\label{parameters-children-positional-tooltip}
Positional parameters are specified in order, without names.

{{ Variadic }}

\phantomsection\label{parameters-children-variadic-tooltip}
Variadic parameters can be specified multiple times.

The elements of the vector.

\href{/docs/reference/math/variants/}{\pandocbounded{\includesvg[keepaspectratio]{/assets/icons/16-arrow-right.svg}}}

{ Variants } { Previous page }

\href{/docs/reference/symbols/}{\pandocbounded{\includesvg[keepaspectratio]{/assets/icons/16-arrow-right.svg}}}

{ Symbols } { Next page }
