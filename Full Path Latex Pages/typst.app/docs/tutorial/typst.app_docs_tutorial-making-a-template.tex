\title{typst.app/docs/tutorial/making-a-template}

\begin{itemize}
\tightlist
\item
  \href{/docs}{\includesvg[width=0.16667in,height=0.16667in]{/assets/icons/16-docs-dark.svg}}
\item
  \includesvg[width=0.16667in,height=0.16667in]{/assets/icons/16-arrow-right.svg}
\item
  \href{/docs/tutorial/}{Tutorial}
\item
  \includesvg[width=0.16667in,height=0.16667in]{/assets/icons/16-arrow-right.svg}
\item
  \href{/docs/tutorial/making-a-template/}{Making a Template}
\end{itemize}

\section{Making a Template}\label{making-a-template}

In the previous three chapters of this tutorial, you have learned how to
write a document in Typst, apply basic styles, and customize its
appearance in-depth to comply with a publisher\textquotesingle s style
guide. Because the paper you wrote in the previous chapter was a
tremendous success, you have been asked to write a follow-up article for
the same conference. This time, you want to take the style you created
in the previous chapter and turn it into a reusable template. In this
chapter you will learn how to create a template that you and your team
can use with just one show rule. Let\textquotesingle s get started!

\subsection{A toy template}\label{toy-template}

In Typst, templates are functions in which you can wrap your whole
document. To learn how to do that, let\textquotesingle s first review
how to write your very own functions. They can do anything you want them
to, so why not go a bit crazy?

\begin{verbatim}
#let amazed(term) = box[✨ #term ✨]

You are #amazed[beautiful]!
\end{verbatim}

\includegraphics[width=5in,height=\textheight,keepaspectratio]{/assets/docs/hf-0MuyTNtENvqMuqT5IlgAAAAAAAAAA.png}

This function takes a single argument, \texttt{\ term\ } , and returns a
content block with the \texttt{\ term\ } surrounded by sparkles. We also
put the whole thing in a box so that the term we are amazed by cannot be
separated from its sparkles by a line break.

Many functions that come with Typst have optional named parameters. Our
functions can also have them. Let\textquotesingle s add a parameter to
our function that lets us choose the color of the text. We need to
provide a default color in case the parameter isn\textquotesingle t
given.

\begin{verbatim}
#let amazed(term, color: blue) = {
  text(color, box[✨ #term ✨])
}

You are #amazed[beautiful]!
I am #amazed(color: purple)[amazed]!
\end{verbatim}

\includegraphics[width=5in,height=\textheight,keepaspectratio]{/assets/docs/DeOx9bmyxPapZywkKVbTFwAAAAAAAAAA.png}

Templates now work by wrapping our whole document in a custom function
like \texttt{\ amazed\ } . But wrapping a whole document in a giant
function call would be cumbersome! Instead, we can use an "everything"
show rule to achieve the same with cleaner code. To write such a show
rule, put a colon directly behind the show keyword and then provide a
function. This function is given the rest of the document as a
parameter. The function can then do anything with this content. Since
the \texttt{\ amazed\ } function can be called with a single content
argument, we can just pass it by name to the show rule.
Let\textquotesingle s try it:

\begin{verbatim}
#show: amazed
I choose to focus on the good
in my life and let go of any
negative thoughts or beliefs.
In fact, I am amazing!
\end{verbatim}

\includegraphics[width=5in,height=\textheight,keepaspectratio]{/assets/docs/gIv_i_LbdQ0VPwrL8LD78QAAAAAAAAAA.png}

Our whole document will now be passed to the \texttt{\ amazed\ }
function, as if we wrapped it around it. Of course, this is not
especially useful with this particular function, but when combined with
set rules and named arguments, it can be very powerful.

\subsection{Embedding set and show rules}\label{set-and-show-rules}

To apply some set and show rules to our template, we can use
\texttt{\ set\ } and \texttt{\ show\ } within a content block in our
function and then insert the document into that content block.

\begin{verbatim}
#let template(doc) = [
  #set text(font: "Inria Serif")
  #show "something cool": [Typst]
  #doc
]

#show: template
I am learning something cool today.
It's going great so far!
\end{verbatim}

\includegraphics[width=5in,height=\textheight,keepaspectratio]{/assets/docs/A-HDnb3ZV5ZLdSR0m_DP1QAAAAAAAAAA.png}

Just like we already discovered in the previous chapter, set rules will
apply to everything within their content block. Since the everything
show rule passes our whole document to the \texttt{\ template\ }
function, the text set rule and string show rule in our template will
apply to the whole document. Let\textquotesingle s use this knowledge to
create a template that reproduces the body style of the paper we wrote
in the previous chapter.

\begin{verbatim}
#let conf(title, doc) = {
  set page(
    paper: "us-letter",
    header: align(
      right + horizon,
      title
    ),
    columns: 2,
    ...
  )
  set par(justify: true)
  set text(
    font: "Libertinus Serif",
    size: 11pt,
  )

  // Heading show rules.
  ...

  doc
}

#show: doc => conf(
  [Paper title],
  doc,
)

= Introduction
#lorem(90)

...
\end{verbatim}

\includegraphics[width=12.75in,height=\textheight,keepaspectratio]{/assets/docs/Zq1nZR6oWo-01oCtP5oKDAAAAAAAAAAA.png}

We copy-pasted most of that code from the previous chapter. The two
differences are this:

\begin{enumerate}
\item
  We wrapped everything in the function \texttt{\ conf\ } using an
  everything show rule. The function applies a few set and show rules
  and echoes the content it has been passed at the end.
\item
  Moreover, we used a curly-braced code block instead of a content
  block. This way, we don\textquotesingle t need to prefix all set rules
  and function calls with a \texttt{\ \#\ } . In exchange, we cannot
  write markup directly in the code block anymore.
\end{enumerate}

Also note where the title comes from: We previously had it inside of a
variable. Now, we are receiving it as the first parameter of the
template function. To do so, we passed a closure (that\textquotesingle s
a function without a name that is used right away) to the everything
show rule. We did that because the \texttt{\ conf\ } function expects
two positional arguments, the title and the body, but the show rule will
only pass the body. Therefore, we add a new function definition that
allows us to set a paper title and use the single parameter from the
show rule.

\subsection{Templates with named arguments}\label{named-arguments}

Our paper in the previous chapter had a title and an author list.
Let\textquotesingle s add these things to our template. In addition to
the title, we want our template to accept a list of authors with their
affiliations and the paper\textquotesingle s abstract. To keep things
readable, we\textquotesingle ll add those as named arguments. In the
end, we want it to work like this:

\begin{verbatim}
#show: doc => conf(
  title: [Towards Improved Modelling],
  authors: (
    (
      name: "Theresa Tungsten",
      affiliation: "Artos Institute",
      email: "tung@artos.edu",
    ),
    (
      name: "Eugene Deklan",
      affiliation: "Honduras State",
      email: "e.deklan@hstate.hn",
    ),
  ),
  abstract: lorem(80),
  doc,
)

...
\end{verbatim}

Let\textquotesingle s build this new template function. First, we add a
default value to the \texttt{\ title\ } argument. This way, we can call
the template without specifying a title. We also add the named
\texttt{\ authors\ } and \texttt{\ abstract\ } parameters with empty
defaults. Next, we copy the code that generates title, abstract and
authors from the previous chapter into the template, replacing the fixed
details with the parameters.

The new \texttt{\ authors\ } parameter expects an
\href{/docs/reference/foundations/array/}{array} of
\href{/docs/reference/foundations/dictionary/}{dictionaries} with the
keys \texttt{\ name\ } , \texttt{\ affiliation\ } and \texttt{\ email\ }
. Because we can have an arbitrary number of authors, we dynamically
determine if we need one, two or three columns for the author list.
First, we determine the number of authors using the
\href{/docs/reference/foundations/array/\#definitions-len}{\texttt{\ .len()\ }}
method on the \texttt{\ authors\ } array. Then, we set the number of
columns as the minimum of this count and three, so that we never create
more than three columns. If there are more than three authors, a new row
will be inserted instead. For this purpose, we have also added a
\texttt{\ row-gutter\ } parameter to the \texttt{\ grid\ } function.
Otherwise, the rows would be too close together. To extract the details
about the authors from the dictionary, we use the
\href{/docs/reference/scripting/\#fields}{field access syntax} .

We still have to provide an argument to the grid for each author: Here
is where the array\textquotesingle s
\href{/docs/reference/foundations/array/\#definitions-map}{\texttt{\ map\ }
method} comes in handy. It takes a function as an argument that gets
called with each item of the array. We pass it a function that formats
the details for each author and returns a new array containing content
values. We\textquotesingle ve now got one array of values that
we\textquotesingle d like to use as multiple arguments for the grid. We
can do that by using the
\href{/docs/reference/foundations/arguments/}{\texttt{\ spread\ }
operator} . It takes an array and applies each of its items as a
separate argument to the function.

The resulting template function looks like this:

\begin{verbatim}
#let conf(
  title: none,
  authors: (),
  abstract: [],
  doc,
) = {
  // Set and show rules from before.
  ...

  set align(center)
  text(17pt, title)

  let count = authors.len()
  let ncols = calc.min(count, 3)
  grid(
    columns: (1fr,) * ncols,
    row-gutter: 24pt,
    ..authors.map(author => [
      #author.name \
      #author.affiliation \
      #link("mailto:" + author.email)
    ]),
  )

  par(justify: false)[
    *Abstract* \
    #abstract
  ]

  set align(left)
  doc
}
\end{verbatim}

\subsection{A separate file}\label{separate-file}

Most of the time, a template is specified in a different file and then
imported into the document. This way, the main file you write in is kept
clutter free and your template is easily reused. Create a new text file
in the file panel by clicking the plus button and name it
\texttt{\ conf.typ\ } . Move the \texttt{\ conf\ } function definition
inside of that new file. Now you can access it from your main file by
adding an import before the show rule. Specify the path of the file
between the \texttt{\ }{\texttt{\ import\ }}\texttt{\ } keyword and a
colon, then name the function that you want to import.

Another thing that you can do to make applying templates just a bit more
elegant is to use the
\href{/docs/reference/foundations/function/\#definitions-with}{\texttt{\ .with\ }}
method on functions to pre-populate all the named arguments. This way,
you can avoid spelling out a closure and appending the content argument
at the bottom of your template list. Templates on
\href{https://typst.app/universe/}{Typst Universe} are designed to work
with this style of function call.

\begin{verbatim}
#import "conf.typ": conf
#show: conf.with(
  title: [
    Towards Improved Modelling
  ],
  authors: (
    (
      name: "Theresa Tungsten",
      affiliation: "Artos Institute",
      email: "tung@artos.edu",
    ),
    (
      name: "Eugene Deklan",
      affiliation: "Honduras State",
      email: "e.deklan@hstate.hn",
    ),
  ),
  abstract: lorem(80),
)

= Introduction
#lorem(90)

== Motivation
#lorem(140)

== Problem Statement
#lorem(50)

= Related Work
#lorem(200)
\end{verbatim}

\includegraphics[width=12.75in,height=\textheight,keepaspectratio]{/assets/docs/BxllQV4yc0ikxppO7QP73AAAAAAAAAAA.png}

We have now converted the conference paper into a reusable template for
that conference! Why not share it in the
\href{https://forum.typst.app/}{Forum} or on
\href{https://discord.gg/2uDybryKPe}{Typst\textquotesingle s Discord
server} so that others can use it too?

\subsection{Review}\label{review}

Congratulations, you have completed Typst\textquotesingle s Tutorial! In
this section, you have learned how to define your own functions and how
to create and apply templates that define reusable document styles.
You\textquotesingle ve made it far and learned a lot. You can now use
Typst to write your own documents and share them with others.

We are still a super young project and are looking for feedback. If you
have any questions, suggestions or you found a bug, please let us know
in the \href{https://forum.typst.app/}{Forum} , on our
\href{https://discord.gg/2uDybryKPe}{Discord server} , on
\href{https://github.com/typst/typst/}{GitHub} , or via the web
app\textquotesingle s feedback form (always available in the Help menu).

So what are you waiting for? \href{https://typst.app}{Sign up} and write
something!

\href{/docs/tutorial/advanced-styling/}{\pandocbounded{\includesvg[keepaspectratio]{/assets/icons/16-arrow-right.svg}}}

{ Advanced Styling } { Previous page }

\href{/docs/reference/}{\pandocbounded{\includesvg[keepaspectratio]{/assets/icons/16-arrow-right.svg}}}

{ Reference } { Next page }

\textbf{On this page}

\begin{itemize}
\tightlist
\item
  \hyperref[toy-template]{Toy Template}
\item
  \hyperref[set-and-show-rules]{Set And Show Rules}
\item
  \hyperref[named-arguments]{Named Arguments}
\item
  \hyperref[separate-file]{Separate File}
\item
  \hyperref[review]{Review}
\end{itemize}

\begin{itemize}
\tightlist
\item
  \href{/}{Home}
\item
  \href{/pricing/}{Pricing}
\item
  \href{/docs/}{Documentation}
\item
  \href{/universe/}{Universe}
\item
  \href{/about/}{About Us}
\item
  \href{/contact/}{Contact Us}
\item
  \href{/privacy/}{Privacy}
\item
  \href{https://typst.app/terms}{Terms and Conditions}
\item
  \href{/legal/}{Legal (Impressum)}
\end{itemize}

\begin{itemize}
\tightlist
\item
  \href{https://forum.typst.app}{Forum}
\item
  \href{/tools/}{Tools}
\item
  \href{/blog/}{Blog}
\item
  \href{https://github.com/typst/}{GitHub}
\item
  \href{https://discord.gg/2uDybryKPe}{Discord}
\item
  \href{https://mastodon.social/@typst}{Mastodon}
\item
  \href{https://bsky.app/profile/typst.app}{Bluesky}
\item
  \href{https://www.linkedin.com/company/typst/}{LinkedIn}
\item
  \href{https://instagram.com/typstapp/}{Instagram}
\end{itemize}

Made in Berlin
