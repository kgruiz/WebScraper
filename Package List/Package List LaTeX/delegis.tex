\title{typst.app/universe/package/delegis}

\phantomsection\label{banner}
\phantomsection\label{template-thumbnail}
\pandocbounded{\includegraphics[keepaspectratio]{https://packages.typst.org/preview/thumbnails/delegis-0.3.0-small.webp}}

\section{delegis}\label{delegis}

{ 0.3.0 }

A package and template for drafting legislative content in a
German-style structuring, such as for bylaws, etc.

\href{/app?template=delegis&version=0.3.0}{Create project in app}

\phantomsection\label{readme}
\begin{longtable}[]{@{}lll@{}}
\toprule\noalign{}
\endhead
\bottomrule\noalign{}
\endlastfoot
\pandocbounded{\includegraphics[keepaspectratio]{https://github.com/typst/packages/raw/main/packages/preview/delegis/0.3.0/demo-1.png}}
&
\pandocbounded{\includegraphics[keepaspectratio]{https://github.com/typst/packages/raw/main/packages/preview/delegis/0.3.0/demo-2.png}}
&
\pandocbounded{\includegraphics[keepaspectratio]{https://github.com/typst/packages/raw/main/packages/preview/delegis/0.3.0/demo-3.png}} \\
\end{longtable}

A package and template for drafting legislative content in a
German-style structuring, such as for bylaws, etc.

While the template is designed to be used in German documents, all
strings are customizable. You can have a look at the
\texttt{\ delegis.typ\ } to see all available parameters.

\subsection{General Usage}\label{general-usage}

While this \texttt{\ README.md\ } gives you a brief overview of the
package’s usage, we recommend that you use the template (in the
\texttt{\ template\ } folder) as a starting point instead.

\subsubsection{Importing the Package}\label{importing-the-package}

\begin{Shaded}
\begin{Highlighting}[]
\NormalTok{\#import "@preview/delegis:0.3.0": *}
\end{Highlighting}
\end{Shaded}

\subsubsection{Initializing the
template}\label{initializing-the-template}

\begin{Shaded}
\begin{Highlighting}[]
\NormalTok{\#show: delegis.with(}
\NormalTok{  // Metadata}
\NormalTok{  title: "Vereinsordnung zu ABCDEF", // title of the law/bylaw/...}
\NormalTok{  abbreviation: "ABCDEFVO", // abbreviation of the law/bylaw/...}
\NormalTok{  resolution: "3. Beschluss des Vorstands vom 24.01.2024", // resolution number and date}
\NormalTok{  in{-}effect: "24.01.2024", // date when it comes into effect}
\NormalTok{  draft: false, // whether this is a draft}
\NormalTok{  // Template}
\NormalTok{  logo: image("wuespace.jpg", alt: "WüSpace e. V."), // logo of the organization, shown on the first page}
\NormalTok{)}
\end{Highlighting}
\end{Shaded}

\subsubsection{Sections}\label{sections}

Sections are auto-detected as long as they follow the pattern
\texttt{\ §\ 1\ ...\ } or \texttt{\ §\ 1a\ ...\ } in its own
paragraph:

\begin{Shaded}
\begin{Highlighting}[]
\NormalTok{§ 1 Geltungsbereich}

\NormalTok{(1) }
\NormalTok{Diese Ordnung gilt für alle Mitglieder des Vereins.}

\NormalTok{(2) }
\NormalTok{Sie regelt die Mitgliedschaft im Verein.}

\NormalTok{§ 2 Mitgliedschaft}

\NormalTok{(1) }
\NormalTok{Die Mitgliedschaft im Verein ist freiwillig.}

\NormalTok{(2) }
\NormalTok{Sie kann jederzeit gekündigt werden.}

\NormalTok{§ 2a Ehrenmitgliedschaft}

\NormalTok{(1) }
\NormalTok{Die Ehrenmitgliedschaft wird durch den Vorstand verliehen.}
\end{Highlighting}
\end{Shaded}

Alternatively (or if you want to use special characters otherwise not
supported, such as \texttt{\ *\ } ), you can also use the
\texttt{\ \#section{[}number{]}{[}title{]}\ } function:

\begin{Shaded}
\begin{Highlighting}[]
\NormalTok{\#section[§ 3][Administrator*innen]}
\end{Highlighting}
\end{Shaded}

\subsubsection{Hierarchical Divisions}\label{hierarchical-divisions}

If you want to add more structure to your sections, you can use normal
Typst headings. Note that only the level 6 headings are reserved for the
section numbers:

\begin{Shaded}
\begin{Highlighting}[]
\NormalTok{= Allgemeine Bestimmungen}

\NormalTok{§ 1 ABC}

\NormalTok{§ 2 DEF}

\NormalTok{= Besondere Bestimmungen}

\NormalTok{§ 3 GHI}

\NormalTok{§ 4 JKL}
\end{Highlighting}
\end{Shaded}

Delegis will automatically use a numbering scheme for the divisions that
is in line with the “Handbuch der Rechtsförmlichkeit�, Rn. 379 f.
If you want to customize the division titles, you can do so by setting
the \texttt{\ division-prefixes\ } parameter in the \texttt{\ delegis\ }
function:

\begin{Shaded}
\begin{Highlighting}[]
\NormalTok{\#show: delegis.with(}
\NormalTok{  division{-}prefixes: ("Teil", "Kapitel", "Abschnitt", "Unterabschnitt")}
\NormalTok{)}
\end{Highlighting}
\end{Shaded}

\subsubsection{Sentence Numbering}\label{sentence-numbering}

If a paragraph contains multiple sentences, you can number them by
adding a \texttt{\ \#s\textasciitilde{}\ } at the beginning of the
sentences:

\begin{Shaded}
\begin{Highlighting}[]
\NormalTok{§ 3 Mitgliedsbeiträge}

\NormalTok{\#s\textasciitilde{}Die Mitgliedsbeiträge sind monatlich zu entrichten.}
\NormalTok{\#s\textasciitilde{}Sie sind bis zum 5. des Folgemonats zu zahlen.}
\end{Highlighting}
\end{Shaded}

This automatically adds corresponding sentence numbers in superscript.

\subsubsection{Referencing other
Sections}\label{referencing-other-sections}

Referencing works manually by specifying the section number. While
automations would be feasible, we have found that in practice, they’re
not as useful as they might seem for legislative documents.

In some cases, referencing sections using \texttt{\ §\ X\ } could be
mis-interpreted as a new section. To avoid this, use the non-breaking
space character \texttt{\ \textasciitilde{}\ } between the
\texttt{\ §\ } and the number:

\begin{Shaded}
\begin{Highlighting}[]
\NormalTok{§ 5 Inkrafttreten}

\NormalTok{Diese Ordnung tritt am 24.01.2024 in Kraft. §\textasciitilde{}4 bleibt unberührt.}
\end{Highlighting}
\end{Shaded}

\subsection{Changelog}\label{changelog}

\subsubsection{v0.3.0}\label{v0.3.0}

\paragraph{Features}\label{features}

\begin{itemize}
\tightlist
\item
  Adjust numbered list / enumeration numbering to be in line with
  “Handbuch der Rechtsförmlichkeit�, Rn. 374
\item
  Make division titles (e.g., “Part�, “Chapter�, “Division�)
  customizable and conform to the “Handbuch der
  Rechtsförmlichkeit�, Rn. 379 f.
\end{itemize}

\subsubsection{v0.2.0}\label{v0.2.0}

\paragraph{Features}\label{features-1}

\begin{itemize}
\tightlist
\item
  Add \texttt{\ \#metadata\ } fields for usage with
  \texttt{\ typst\ query\ } . You can now use
  \texttt{\ typst\ query\ file.typ\ "\textless{}field\textgreater{}"\ -\/-field\ value\ -\/-one\ }
  with \texttt{\ \textless{}field\textgreater{}\ } being one of the
  following to query metadata fields in the command line:

  \begin{itemize}
  \tightlist
  \item
    \texttt{\ \textless{}title\textgreater{}\ }
  \item
    \texttt{\ \textless{}abbreviation\textgreater{}\ }
  \item
    \texttt{\ \textless{}resolution\textgreater{}\ }
  \item
    \texttt{\ \textless{}in-effect\textgreater{}\ }
  \end{itemize}
\item
  Add \texttt{\ \#section{[}§\ 1{]}{[}ABC{]}\ } function to enable
  previously unsupported special chars (such as \texttt{\ *\ } ) in
  section headings. Note that this was previously possible using
  \texttt{\ \#unnumbered{[}§\ 1\textbackslash{}\ ABC{]}\ } , but the
  new function adds a semantically better-fitting alternative to this
  fix.
\item
  Improve heading style rules. This also fixes an incompatibility with
  \texttt{\ pandoc\ } , meaning it’s now possible to use
  \texttt{\ pandoc\ } to convert delegis documents to HTML, etc.
\item
  Set the footnote numbering to \texttt{\ {[}1{]}\ } to not collide with
  sentence numbers.
\end{itemize}

\paragraph{Bug Fixes}\label{bug-fixes}

\begin{itemize}
\tightlist
\item
  Fix a typo in the \texttt{\ str-draft\ } variable name that lead to
  draft documents resulting in a syntax error.
\item
  Fix hyphenation issues with the abbreviation on the title page
  (hyphenation between the parentheses and the abbreviation itself)
\end{itemize}

\subsubsection{v0.1.0}\label{v0.1.0}

Initial Release

\href{/app?template=delegis&version=0.3.0}{Create project in app}

\subsubsection{How to use}\label{how-to-use}

Click the button above to create a new project using this template in
the Typst app.

You can also use the Typst CLI to start a new project on your computer
using this command:

\begin{verbatim}
typst init @preview/delegis:0.3.0
\end{verbatim}

\includesvg[width=0.16667in,height=0.16667in]{/assets/icons/16-copy.svg}

\subsubsection{About}\label{about}

\begin{description}
\tightlist
\item[Author :]
\href{https://github.com/wuespace}{WüSpace e. V.}
\item[License:]
MIT
\item[Current version:]
0.3.0
\item[Last updated:]
May 22, 2024
\item[First released:]
March 16, 2024
\item[Archive size:]
13.4 kB
\href{https://packages.typst.org/preview/delegis-0.3.0.tar.gz}{\pandocbounded{\includesvg[keepaspectratio]{/assets/icons/16-download.svg}}}
\item[Repository:]
\href{https://github.com/wuespace/delegis}{GitHub}
\item[Discipline :]
\begin{itemize}
\tightlist
\item[]
\item
  \href{https://typst.app/universe/search/?discipline=law}{Law}
\end{itemize}
\item[Categor y :]
\begin{itemize}
\tightlist
\item[]
\item
  \pandocbounded{\includesvg[keepaspectratio]{/assets/icons/16-envelope.svg}}
  \href{https://typst.app/universe/search/?category=office}{Office}
\end{itemize}
\end{description}

\subsubsection{Where to report issues?}\label{where-to-report-issues}

This template is a project of WüSpace e. V. . Report issues on
\href{https://github.com/wuespace/delegis}{their repository} . You can
also try to ask for help with this template on the
\href{https://forum.typst.app}{Forum} .

Please report this template to the Typst team using the
\href{https://typst.app/contact}{contact form} if you believe it is a
safety hazard or infringes upon your rights.

\phantomsection\label{versions}
\subsubsection{Version history}\label{version-history}

\begin{longtable}[]{@{}ll@{}}
\toprule\noalign{}
Version & Release Date \\
\midrule\noalign{}
\endhead
\bottomrule\noalign{}
\endlastfoot
0.3.0 & May 22, 2024 \\
\href{https://typst.app/universe/package/delegis/0.2.0/}{0.2.0} & May
17, 2024 \\
\href{https://typst.app/universe/package/delegis/0.1.0/}{0.1.0} & March
16, 2024 \\
\end{longtable}

Typst GmbH did not create this template and cannot guarantee correct
functionality of this template or compatibility with any version of the
Typst compiler or app.
