\title{typst.app/universe/package/marge}

\phantomsection\label{banner}
\section{marge}\label{marge}

{ 0.1.0 }

Easy-to-use but powerful and smart margin notes.

\phantomsection\label{readme}
A package for easy-to-use but powerful and smart margin notes.

\subsection{Usage}\label{usage}

The main function provided by this package is \texttt{\ sidenote\ } ,
which allows you to create margin notes. The function takes a single
positional argument (the text of the note) and several optional keyword
arguments for customization:

\begin{longtable}[]{@{}lll@{}}
\toprule\noalign{}
Parameter & Description & Default \\
\midrule\noalign{}
\endhead
\bottomrule\noalign{}
\endlastfoot
\texttt{\ side\ } & The margin where the note should be placed. &
\texttt{\ auto\ } \\
\texttt{\ dy\ } & The custom offset by which the note should be moved
along the y-axis. & \texttt{\ 0pt\ } \\
\texttt{\ padding\ } & The space between the note and the page or
content border. & \texttt{\ 2em\ } \\
\texttt{\ gap\ } & The minimum gap between this and neighboring notes. &
\texttt{\ 0.4em\ } \\
\texttt{\ numbering\ } & How the note should be numbered. &
\texttt{\ none\ } \\
\texttt{\ counter\ } & The counter to use for numbering. &
\texttt{\ counter("sidenote")\ } \\
\texttt{\ format\ } & The “show rule� for the note. &
\texttt{\ it\ =\textgreater{}\ it.default\ } \\
\end{longtable}

The parameters allow maximum flexibility and often allow values of
different types:

\begin{itemize}
\tightlist
\item
  The \texttt{\ side\ } parameter can be set to \texttt{\ auto\ } ,
  \texttt{\ "inside"\ } , \texttt{\ "outside"\ } or any horizontal
  \texttt{\ alignment\ } value. If set to \texttt{\ auto\ } , the note
  is placed on the larger of the two margins. If they are equally large,
  it is placed on the \texttt{\ "outside"\ } margin.
\item
  If the \texttt{\ dy\ } parameter has a relative part, it is resolved
  relative to the height of the note.
\item
  The \texttt{\ padding\ } parameter can be set either to a single
  length value or a dictionary. If a dictionary is used, the keys can be
  any horizontal alignment value, as well as \texttt{\ inside\ } and
  \texttt{\ outside\ } .
\item
  With the \texttt{\ counter\ } parameter, you can for example combine
  the numbering of footnotes and sidenotes.
\end{itemize}

An especially useful feature is the \texttt{\ format\ } parameter, as it
emulates the behavior of a show rule via a function. That function is
called with the context of the note and receives a dictionary with the
following keys:

\begin{longtable}[]{@{}lll@{}}
\toprule\noalign{}
Key & Description & Value or type \\
\midrule\noalign{}
\endhead
\bottomrule\noalign{}
\endlastfoot
\texttt{\ side\ } & The side of the page the note is placed on. &
\texttt{\ left\ } or \texttt{\ right\ } \\
\texttt{\ numbering\ } & The numbering of the note. & \texttt{\ str\ } ,
\texttt{\ function\ } or \texttt{\ none\ } \\
\texttt{\ counter\ } & The counter used for numbering the note. &
\texttt{\ counter\ } \\
\texttt{\ padding\ } & The padding of the note, resolved to
\texttt{\ left\ } and \texttt{\ right\ } . & \texttt{\ dictionary\ } \\
\texttt{\ margin\ } & The size of the margin, which the note is placed
on. & \texttt{\ length\ } \\
\texttt{\ source\ } & The location in the document where the note is
defined. & \texttt{\ location\ } \\
\texttt{\ body\ } & The content of the note. & \texttt{\ str\ } \\
\texttt{\ default\ } & The default look of the note. &
\texttt{\ content\ } \\
\end{longtable}

As the dictionary itself is not an element, you cannot directly use it
within the \texttt{\ format\ } function as you would be able to in a
normal show rule. To still be able to build upon the default look of the
note without having to reconstruct it, the \texttt{\ default\ } key is
provided. The default style sets the font size to \texttt{\ 0.85em\ }
and the paragraph’s leading to \texttt{\ 0.5em\ } , matching the
default style of footnotes. This can of course be overridden.

Aside from the customizability, the package also provides automatic
overlap and overflow protection. If a note would overlap with another
note, it is moved further down the page, so that the \texttt{\ gap\ }
parameters of both notes are respected. If a note would overflow the
page, it is moved upwards, so that the bottom of the note is aligned
with the bottom of the page content. Any previous notes, which would
then overlap with the moved note, are also moved accordingly.

\subsubsection{Note about pages with automatic
width}\label{note-about-pages-with-automatic-width}

If a note is placed in the right margin of a page with width set to
\texttt{\ auto\ } , additional configuration is necessary. As the final
width of the page is not known when the note is placed, the note’s
position cannot be calculated. To place notes on the right margin of
such pages, the package provides a \texttt{\ container\ } , which is
supposed to be included in the page’s \texttt{\ background\ } or
\texttt{\ foreground\ } :

\begin{Shaded}
\begin{Highlighting}[]
\NormalTok{\#import "@preview/marge:0.1.0": sidenote, container}

\NormalTok{\#set page(width: auto, background: container)}

\NormalTok{...}
\end{Highlighting}
\end{Shaded}

The use of the \texttt{\ container\ } variable is detected automatically
by the package, so that an error can be raised when it is required but
not set.

\subsubsection{Note about layout convergence and
performance}\label{note-about-layout-convergence-and-performance}

This package makes heavy use of states and contextual blocks, causing
Typst to require multiple layout passes to fully resolve the final
layout. Usually, the limit imposed by Typst is sufficient, but I cannot
guarantee that this will remain true for large documents with a lot of
notes. If you happen to run into this limit, you can try using the
\texttt{\ container\ } variable as mentioned above, as it can reduce the
number of layout passes required.

As each layout iteration adds to the total compile time, the use of the
\texttt{\ container\ } can also be beneficial for performance reasons.
Another performance tip is to keep the size of paragraphs containing
margin notes small, as the line breaking algorithm cannot be memoized
when the paragraph contains a note.

\subsubsection{Note about how lengths are
resolved}\label{note-about-how-lengths-are-resolved}

When a length is given in a context-dependent way (i.e. in
\texttt{\ em\ } units), it is resolved relative to the font size of the
\emph{content} , not the font size of the note (which is smaller by
default). This has the unfortunate side effect that a gap set to
\texttt{\ 0pt\ } will still have some space due to the content
paragraph’s leading (which is also larger than default leading of the
note). Similarly, if notes are defined in a context with a larger font
size, the padding and gap values may unexpectedly be larger than of
neighboring notes.

\subsection{Example}\label{example}

\begin{Shaded}
\begin{Highlighting}[]
\NormalTok{\#import "@preview/marge:0.1.0": sidenote}

\NormalTok{\#set page(margin: (right: 5cm))}
\NormalTok{\#set par(justify: true)}

\NormalTok{\#let sidenote = sidenote.with(numbering: "1", padding: 1em)}

\NormalTok{The Simpsons is an iconic animated series that began in 1989}
\NormalTok{\#sidenote[The show holds the record for the most episodes of any}
\NormalTok{American sitcom.]. The show features the Simpson family: Homer,}
\NormalTok{Marge, Bart, Lisa, and Maggie. }

\NormalTok{Bart is the rebellious son who often gets into trouble, and Lisa}
\NormalTok{is the intelligent and talented daughter \#sidenote[Lisa is known}
\NormalTok{for her saxophone playing and academic achievements.]. Baby}
\NormalTok{Maggie, though silent, has had moments of surprising brilliance}
\NormalTok{\#sidenote[Maggie once shot Mr. Burns in a dramatic plot twist.].}
\end{Highlighting}
\end{Shaded}

\pandocbounded{\includesvg[keepaspectratio]{https://github.com/typst/packages/raw/main/packages/preview/marge/0.1.0/assets/example.svg}}

\subsubsection{How to add}\label{how-to-add}

Copy this into your project and use the import as \texttt{\ marge\ }

\begin{verbatim}
#import "@preview/marge:0.1.0"
\end{verbatim}

\includesvg[width=0.16667in,height=0.16667in]{/assets/icons/16-copy.svg}

Check the docs for
\href{https://typst.app/docs/reference/scripting/\#packages}{more
information on how to import packages} .

\subsubsection{About}\label{about}

\begin{description}
\tightlist
\item[Author :]
Eric Biedert
\item[License:]
MIT
\item[Current version:]
0.1.0
\item[Last updated:]
November 19, 2024
\item[First released:]
November 19, 2024
\item[Minimum Typst version:]
0.11.0
\item[Archive size:]
7.98 kB
\href{https://packages.typst.org/preview/marge-0.1.0.tar.gz}{\pandocbounded{\includesvg[keepaspectratio]{/assets/icons/16-download.svg}}}
\item[Repository:]
\href{https://github.com/EpicEricEE/typst-marge}{GitHub}
\item[Categor ies :]
\begin{itemize}
\tightlist
\item[]
\item
  \pandocbounded{\includesvg[keepaspectratio]{/assets/icons/16-package.svg}}
  \href{https://typst.app/universe/search/?category=components}{Components}
\item
  \pandocbounded{\includesvg[keepaspectratio]{/assets/icons/16-list-unordered.svg}}
  \href{https://typst.app/universe/search/?category=model}{Model}
\item
  \pandocbounded{\includesvg[keepaspectratio]{/assets/icons/16-layout.svg}}
  \href{https://typst.app/universe/search/?category=layout}{Layout}
\end{itemize}
\end{description}

\subsubsection{Where to report issues?}\label{where-to-report-issues}

This package is a project of Eric Biedert . Report issues on
\href{https://github.com/EpicEricEE/typst-marge}{their repository} . You
can also try to ask for help with this package on the
\href{https://forum.typst.app}{Forum} .

Please report this package to the Typst team using the
\href{https://typst.app/contact}{contact form} if you believe it is a
safety hazard or infringes upon your rights.

\phantomsection\label{versions}
\subsubsection{Version history}\label{version-history}

\begin{longtable}[]{@{}ll@{}}
\toprule\noalign{}
Version & Release Date \\
\midrule\noalign{}
\endhead
\bottomrule\noalign{}
\endlastfoot
0.1.0 & November 19, 2024 \\
\end{longtable}

Typst GmbH did not create this package and cannot guarantee correct
functionality of this package or compatibility with any version of the
Typst compiler or app.
