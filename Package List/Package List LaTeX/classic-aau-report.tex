\title{typst.app/universe/package/classic-aau-report}

\phantomsection\label{banner}
\phantomsection\label{template-thumbnail}
\pandocbounded{\includegraphics[keepaspectratio]{https://packages.typst.org/preview/thumbnails/classic-aau-report-0.1.0-small.webp}}

\section{classic-aau-report}\label{classic-aau-report}

{ 0.1.0 }

An example package.

\href{/app?template=classic-aau-report&version=0.1.0}{Create project in
app}

\phantomsection\label{readme}
Unofficial Typst template for project reports at Aalborg University
(AAU). This is based on the LaTeX template
\url{https://github.com/jkjaer/aauLatexTemplates} .

The template is generic to any field of study, but defaults to Computer
Science.

\subsection{Usage}\label{usage}

Click “Create project in app�.

Or via the CLI

\begin{Shaded}
\begin{Highlighting}[]
\ExtensionTok{typst}\NormalTok{ init @preview/classic{-}aau{-}report}
\end{Highlighting}
\end{Shaded}

\textbf{NOTE:} The template tries to use the
\texttt{\ Palatino\ Linotype\ } font, which is \emph{not} available in
Typst. It is available
\href{https://github.com/Tinggaard/classic-aau-report/tree/main/fonts}{here}

To use it in the \emph{web-app} , put the \texttt{\ .ttf\ } files
anywhere in the project tree.

To use it \emph{locally} specify the \texttt{\ -\/-font-path\ } flag (or
see the
\href{https://typst.app/docs/reference/text/text/\#parameters-font}{docs}
).

\subsection{Confugiration}\label{confugiration}

The \texttt{\ project\ } function takes the following (optional)
arguments:

\begin{itemize}
\item
  \texttt{\ meta\ } : Metadata about the project

  \begin{itemize}
  \tightlist
  \item
    \texttt{\ project-group\ } : The project group name
  \item
    \texttt{\ participants\ } : A list of participants
  \item
    \texttt{\ supervisors\ } : A list of supervisors
  \item
    \texttt{\ field-of-study\ } : The field of study
  \item
    \texttt{\ project-type\ } : The type of project
  \end{itemize}
\item
  \texttt{\ en\ } : English project info

  \begin{itemize}
  \tightlist
  \item
    \texttt{\ title\ } : The title of the project
  \item
    \texttt{\ theme\ } : The theme of the project
  \item
    \texttt{\ abstract\ } : The English abstract of the project
  \item
    \texttt{\ department\ } : The department name
  \item
    \texttt{\ department-url\ } : The department URL
  \end{itemize}
\item
  \texttt{\ dk\ } : Danish project info

  \begin{itemize}
  \tightlist
  \item
    \texttt{\ title\ } : The Danish title of the project
  \item
    \texttt{\ theme\ } : The theme of the project in Danish
  \item
    \texttt{\ abstract\ } : The Danish abstract of the project
  \item
    \texttt{\ department\ } : The department name in Danish
  \item
    \texttt{\ department-url\ } : The Danish department URL
  \end{itemize}
\end{itemize}

The defaults are as follows:

\begin{Shaded}
\begin{Highlighting}[]
\NormalTok{\#let defaults = (}
\NormalTok{  meta: (}
\NormalTok{    project{-}group: "No group name provided",}
\NormalTok{    participants: (),}
\NormalTok{    supervisors: (),}
\NormalTok{    field{-}of{-}study: "Computer Science",}
\NormalTok{    project{-}type: "Semester Project"}
\NormalTok{  ),}
\NormalTok{  en: (}
\NormalTok{    title: "Untitled",}
\NormalTok{    theme: "",}
\NormalTok{    abstract: [],}
\NormalTok{    department: "Department of Computer Science",}
\NormalTok{    department{-}url: "https://www.cs.aau.dk",}
\NormalTok{  ),}
\NormalTok{  dk: (}
\NormalTok{    title: "Uden titel",}
\NormalTok{    theme: "",}
\NormalTok{    abstract: [],}
\NormalTok{    department: "Institut for Datalogi",}
\NormalTok{    department{-}url: "https://www.dat.aau.dk",}
\NormalTok{  ),}
\NormalTok{)}
\end{Highlighting}
\end{Shaded}

Furthermore, the template exports the shawrules

\begin{itemize}
\tightlist
\item
  \texttt{\ frontmatter\ } : Sets the page numbering to arabic and
  chapter numbering to none
\item
  \texttt{\ mainmatter\ } : Sets the chapter numbering
  \texttt{\ Chapter\ } followed by a number.
\item
  \texttt{\ backmatter\ } : Sets the chapter numbering back to none
\item
  \texttt{\ appendix\ } : Sets the chapter numbering to
  \texttt{\ Appeendix\ } followed by a letter.
\end{itemize}

To use it in an existing project, add the following show rule to the top
of your file.

\begin{Shaded}
\begin{Highlighting}[]
\NormalTok{\#include "@preview/classic{-}aau{-}report:0.1.0": project, frontmatter, mainmatter, backmatter, appendix}

\NormalTok{// Any of the below can be omitted, the defaults are either empty values or CS specific}
\NormalTok{\#show: project.with(}
\NormalTok{  meta: (}
\NormalTok{    project{-}group: "CS{-}xx{-}DAT{-}y{-}zz",}
\NormalTok{    participants: (}
\NormalTok{      "Alice",}
\NormalTok{      "Bob",}
\NormalTok{      "Chad",}
\NormalTok{    ),}
\NormalTok{    supervisors: "John McClane"}
\NormalTok{  ),}
\NormalTok{  en: (}
\NormalTok{    title: "An awesome project",}
\NormalTok{    theme: "Writing a project in Typst",}
\NormalTok{    abstract: [],}
\NormalTok{  ),}
\NormalTok{  dk: (}
\NormalTok{    title: "Et fantastisk projekt",}
\NormalTok{    theme: "Et projekt i Typst",}
\NormalTok{    abstract: [],}
\NormalTok{  ),}
\NormalTok{)}

\NormalTok{// \#show{-}todos()}

\NormalTok{\#show: frontmatter}
\NormalTok{\#include "chapters/introduction.typ"}

\NormalTok{\#show: mainmatter}
\NormalTok{\#include "chapters/problem{-}analysis.typ"}
\NormalTok{\#include "chapters/conclusion.typ"}

\NormalTok{\#show: backmatter}
\NormalTok{\#bibliography("references.bib", title: "References")}

\NormalTok{\#show: appendix}
\NormalTok{\#include "appendices/code{-}snippets.typ"}
\end{Highlighting}
\end{Shaded}

\href{/app?template=classic-aau-report&version=0.1.0}{Create project in
app}

\subsubsection{How to use}\label{how-to-use}

Click the button above to create a new project using this template in
the Typst app.

You can also use the Typst CLI to start a new project on your computer
using this command:

\begin{verbatim}
typst init @preview/classic-aau-report:0.1.0
\end{verbatim}

\includesvg[width=0.16667in,height=0.16667in]{/assets/icons/16-copy.svg}

\subsubsection{About}\label{about}

\begin{description}
\tightlist
\item[Author :]
\href{https://github.com/Tinggaard}{Jens Tinggaard}
\item[License:]
MIT
\item[Current version:]
0.1.0
\item[Last updated:]
November 22, 2024
\item[First released:]
November 22, 2024
\item[Minimum Typst version:]
0.12.0
\item[Archive size:]
149 kB
\href{https://packages.typst.org/preview/classic-aau-report-0.1.0.tar.gz}{\pandocbounded{\includesvg[keepaspectratio]{/assets/icons/16-download.svg}}}
\item[Repository:]
\href{https://github.com/Tinggaard/classic-aau-report}{GitHub}
\item[Categor ies :]
\begin{itemize}
\tightlist
\item[]
\item
  \pandocbounded{\includesvg[keepaspectratio]{/assets/icons/16-speak.svg}}
  \href{https://typst.app/universe/search/?category=report}{Report}
\item
  \pandocbounded{\includesvg[keepaspectratio]{/assets/icons/16-mortarboard.svg}}
  \href{https://typst.app/universe/search/?category=thesis}{Thesis}
\end{itemize}
\end{description}

\subsubsection{Where to report issues?}\label{where-to-report-issues}

This template is a project of Jens Tinggaard . Report issues on
\href{https://github.com/Tinggaard/classic-aau-report}{their repository}
. You can also try to ask for help with this template on the
\href{https://forum.typst.app}{Forum} .

Please report this template to the Typst team using the
\href{https://typst.app/contact}{contact form} if you believe it is a
safety hazard or infringes upon your rights.

\phantomsection\label{versions}
\subsubsection{Version history}\label{version-history}

\begin{longtable}[]{@{}ll@{}}
\toprule\noalign{}
Version & Release Date \\
\midrule\noalign{}
\endhead
\bottomrule\noalign{}
\endlastfoot
0.1.0 & November 22, 2024 \\
\end{longtable}

Typst GmbH did not create this template and cannot guarantee correct
functionality of this template or compatibility with any version of the
Typst compiler or app.
