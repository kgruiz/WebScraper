\title{typst.app/universe/package/note-me}

\phantomsection\label{banner}
\section{note-me}\label{note-me}

{ 0.3.0 }

Adds GitHub-style Admonitions (Alerts) to Typst.

\phantomsection\label{readme}
\begin{quote}
{[}!NOTE{]} Add GitHub style admonitions (also known as alerts) to
Typst.
\end{quote}

\subsection{Usage}\label{usage}

Import this package, and do

\begin{Shaded}
\begin{Highlighting}[]
\NormalTok{// Import from @preview namespace is suggested}
\NormalTok{// \#import "@preview/note{-}me:0.3.0": *}

\NormalTok{// Import from @local namespace is only for debugging purpose}
\NormalTok{// \#import "@local/note{-}me:0.3.0": *}

\NormalTok{// Import relatively is for development purpose}
\NormalTok{\#import "lib.typ": *}

\NormalTok{= Basic Examples}

\NormalTok{\#note[}
\NormalTok{  Highlights information that users should take into account, even when skimming.}
\NormalTok{]}

\NormalTok{\#tip[}
\NormalTok{  Optional information to help a user be more successful.}
\NormalTok{]}

\NormalTok{\#important[}
\NormalTok{  Crucial information necessary for users to succeed.}
\NormalTok{]}

\NormalTok{\#warning[}
\NormalTok{  Critical content demanding immediate user attention due to potential risks.}
\NormalTok{]}

\NormalTok{\#caution[}
\NormalTok{  Negative potential consequences of an action.}
\NormalTok{]}

\NormalTok{\#admonition(}
\NormalTok{  icon{-}path: "icons/stop.svg",}
\NormalTok{  color: color.fuchsia,}
\NormalTok{  title: "Customize",}
\NormalTok{  foreground{-}color: color.white,}
\NormalTok{  background{-}color: color.black,}
\NormalTok{)[}
\NormalTok{  The icon, (theme) color, title, foreground and background color are customizable.}
\NormalTok{]}

\NormalTok{\#admonition(}
\NormalTok{  icon{-}string: read("icons/light{-}bulb.svg"),}
\NormalTok{  color: color.fuchsia,}
\NormalTok{  title: "Customize",}
\NormalTok{)[}
\NormalTok{  The icon can be specified as a string of SVG. This is useful if the user want to use an SVG icon that is not available in this package.}
\NormalTok{]}

\NormalTok{\#admonition(}
\NormalTok{  icon: [🙈],}
\NormalTok{  color: color.fuchsia,}
\NormalTok{  title: "Customize",}
\NormalTok{)[}
\NormalTok{  Or, pass a content directly as the icon...}
\NormalTok{]}

\NormalTok{= More Examples}

\NormalTok{\#todo[}
\NormalTok{  Fix \textasciigrave{}note{-}me\textasciigrave{} package.}
\NormalTok{]}


\NormalTok{= Prevent Page Breaks from Breaking Admonitions}

\NormalTok{\#box(}
\NormalTok{  width: 1fr,}
\NormalTok{  height: 50pt,}
\NormalTok{  fill: gray,}
\NormalTok{)}

\NormalTok{\#note[}
\NormalTok{  \#lorem(100)}
\NormalTok{]}
\end{Highlighting}
\end{Shaded}

\pandocbounded{\includesvg[keepaspectratio]{https://github.com/typst/packages/raw/main/packages/preview/note-me/0.3.0/example.svg}}

Further Reading:

\begin{itemize}
\tightlist
\item
  \url{https://github.com/orgs/community/discussions/16925}
\item
  \url{https://docs.asciidoctor.org/asciidoc/latest/blocks/admonitions/}
\end{itemize}

\subsection{Style}\label{style}

It borrows the style of GitHub’s admonition.

\begin{quote}
{[}!NOTE{]}\\
Highlights information that users should take into account, even when
skimming.
\end{quote}

\begin{quote}
{[}!TIP{]} Optional information to help a user be more successful.
\end{quote}

\begin{quote}
{[}!IMPORTANT{]}\\
Crucial information necessary for users to succeed.
\end{quote}

\begin{quote}
{[}!WARNING{]}\\
Critical content demanding immediate user attention due to potential
risks.
\end{quote}

\begin{quote}
{[}!CAUTION{]} Negative potential consequences of an action.
\end{quote}

\subsection{Credits}\label{credits}

The admonition icons are from
\href{https://github.com/primer/octicons}{Octicons} .

\subsubsection{How to add}\label{how-to-add}

Copy this into your project and use the import as \texttt{\ note-me\ }

\begin{verbatim}
#import "@preview/note-me:0.3.0"
\end{verbatim}

\includesvg[width=0.16667in,height=0.16667in]{/assets/icons/16-copy.svg}

Check the docs for
\href{https://typst.app/docs/reference/scripting/\#packages}{more
information on how to import packages} .

\subsubsection{About}\label{about}

\begin{description}
\tightlist
\item[Author :]
Flandia Yingman
\item[License:]
MIT
\item[Current version:]
0.3.0
\item[Last updated:]
September 30, 2024
\item[First released:]
February 11, 2024
\item[Archive size:]
5.02 kB
\href{https://packages.typst.org/preview/note-me-0.3.0.tar.gz}{\pandocbounded{\includesvg[keepaspectratio]{/assets/icons/16-download.svg}}}
\item[Repository:]
\href{https://github.com/FlandiaYingman/note-me}{GitHub}
\end{description}

\subsubsection{Where to report issues?}\label{where-to-report-issues}

This package is a project of Flandia Yingman . Report issues on
\href{https://github.com/FlandiaYingman/note-me}{their repository} . You
can also try to ask for help with this package on the
\href{https://forum.typst.app}{Forum} .

Please report this package to the Typst team using the
\href{https://typst.app/contact}{contact form} if you believe it is a
safety hazard or infringes upon your rights.

\phantomsection\label{versions}
\subsubsection{Version history}\label{version-history}

\begin{longtable}[]{@{}ll@{}}
\toprule\noalign{}
Version & Release Date \\
\midrule\noalign{}
\endhead
\bottomrule\noalign{}
\endlastfoot
0.3.0 & September 30, 2024 \\
\href{https://typst.app/universe/package/note-me/0.2.1/}{0.2.1} & March
8, 2024 \\
\href{https://typst.app/universe/package/note-me/0.1.1/}{0.1.1} &
February 25, 2024 \\
\href{https://typst.app/universe/package/note-me/0.1.0/}{0.1.0} &
February 11, 2024 \\
\end{longtable}

Typst GmbH did not create this package and cannot guarantee correct
functionality of this package or compatibility with any version of the
Typst compiler or app.
