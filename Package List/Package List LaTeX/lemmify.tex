\title{typst.app/universe/package/lemmify}

\phantomsection\label{banner}
\section{lemmify}\label{lemmify}

{ 0.1.6 }

Theorem typesetting library.

\phantomsection\label{readme}
Lemmify is a library for typesetting mathematical theorems in typst. It
aims to be easy to use while trying to be as flexible and idiomatic as
possible. This means that the interface might change with updates to
typst (for example if user-defined element functions are introduced).
But no functionality should be lost.

\subsection{Basic Usage}\label{basic-usage}

To get started with Lemmify, follow these steps:

\begin{enumerate}
\tightlist
\item
  Import the Lemmify library:
\end{enumerate}

\begin{Shaded}
\begin{Highlighting}[]
\NormalTok{\#import "@preview/lemmify:0.1.6": *}
\end{Highlighting}
\end{Shaded}

\begin{enumerate}
\setcounter{enumi}{1}
\tightlist
\item
  Define the default styling for a few default theorem types:
\end{enumerate}

\begin{Shaded}
\begin{Highlighting}[]
\NormalTok{\#let (}
\NormalTok{  theorem, lemma, corollary,}
\NormalTok{  remark, proposition, example,}
\NormalTok{  proof, rules: thm{-}rules}
\NormalTok{) = default{-}theorems("thm{-}group", lang: "en")}
\end{Highlighting}
\end{Shaded}

\begin{enumerate}
\setcounter{enumi}{2}
\tightlist
\item
  Apply the generated styling:
\end{enumerate}

\begin{Shaded}
\begin{Highlighting}[]
\NormalTok{\#show: thm{-}rules}
\end{Highlighting}
\end{Shaded}

\begin{enumerate}
\setcounter{enumi}{3}
\tightlist
\item
  Create theorems, lemmas, and proofs using the defined styling:
\end{enumerate}

\begin{Shaded}
\begin{Highlighting}[]
\NormalTok{\#theorem(name: "Some theorem")[}
\NormalTok{  Theorem content goes here.}
\NormalTok{]\textless{}thm\textgreater{}}

\NormalTok{\#proof[}
\NormalTok{  Complicated proof.}
\NormalTok{]\textless{}proof\textgreater{}}

\NormalTok{@proof and @thm[theorem]}
\end{Highlighting}
\end{Shaded}

\begin{enumerate}
\setcounter{enumi}{4}
\tightlist
\item
  Customize the styling further using show rules. For example, to add a
  red box around proofs:
\end{enumerate}

\begin{Shaded}
\begin{Highlighting}[]
\NormalTok{\#show thm{-}selector("thm{-}group", subgroup: "proof"): it =\textgreater{} box(}
\NormalTok{  it,}
\NormalTok{  stroke: red + 1pt,}
\NormalTok{  inset: 1em}
\NormalTok{)}
\end{Highlighting}
\end{Shaded}

The result should now look something like this:

\pandocbounded{\includegraphics[keepaspectratio]{https://github.com/Marmare314/lemmify/assets/49279081/ba5e7ed4-336d-4b1a-8470-99fa23bf5119}}

\subsection{Useful examples}\label{useful-examples}

If you do not want to reset the theorem counter on headings you can use
the \texttt{\ max-reset-level\ } parameter:

\begin{Shaded}
\begin{Highlighting}[]
\NormalTok{default{-}theorems("thm{-}group", max{-}reset{-}level: 0)}
\end{Highlighting}
\end{Shaded}

It specifies the highest level at which the counter is reset. To
manually reset the counter you can use the
\texttt{\ thm-reset-counter\ } function.

\begin{center}\rule{0.5\linewidth}{0.5pt}\end{center}

By specifying \texttt{\ numbering:\ none\ } you can create unnumbered
theorems.

\begin{Shaded}
\begin{Highlighting}[]
\NormalTok{\#example(numbering: none)[}
\NormalTok{  Some example.}
\NormalTok{]}
\end{Highlighting}
\end{Shaded}

To make all examples unnumbered you could use the following code:

\begin{Shaded}
\begin{Highlighting}[]
\NormalTok{\#let example = example.with(numbering: none)}
\end{Highlighting}
\end{Shaded}

\begin{center}\rule{0.5\linewidth}{0.5pt}\end{center}

To create other types (or subgroups) of theorems you can use the
\texttt{\ new-theorems\ } function.

\begin{Shaded}
\begin{Highlighting}[]
\NormalTok{\#let (note, rules) = new{-}theorems("thm{-}group", ("note": text(red)[Note]))}
\NormalTok{\#show: rules}
\end{Highlighting}
\end{Shaded}

If you have already defined custom styling you will notice that the
newly created theorem does not use it. You can create a dictionary to
make applying it again easier.

\begin{Shaded}
\begin{Highlighting}[]
\NormalTok{\#let my{-}styling = (}
\NormalTok{  thm{-}styling: thm{-}styling{-}simple,}
\NormalTok{  thm{-}numbering: ..,}
\NormalTok{  ref{-}styling: ..}
\NormalTok{)}

\NormalTok{\#let (note, rules) = new{-}theorems("thm{-}group", ("note": "Note), ..my{-}styling)}
\end{Highlighting}
\end{Shaded}

\begin{center}\rule{0.5\linewidth}{0.5pt}\end{center}

By varying the \texttt{\ group\ } parameter you can create independently
numbered theorems:

\begin{Shaded}
\begin{Highlighting}[]
\NormalTok{\#let (}
\NormalTok{  theorem, proof,}
\NormalTok{  rules: thm{-}rules{-}a}
\NormalTok{) = default{-}theorems("thm{-}group{-}a")}
\NormalTok{\#let (}
\NormalTok{  definition,}
\NormalTok{  rules: thm{-}rules{-}b}
\NormalTok{) = default{-}theorems("thm{-}group{-}b")}

\NormalTok{\#show: thm{-}rules{-}a}
\NormalTok{\#show: thm{-}rules{-}b}
\end{Highlighting}
\end{Shaded}

\begin{center}\rule{0.5\linewidth}{0.5pt}\end{center}

To specify parameters of the
\href{https://github.com/typst/packages/raw/main/packages/preview/lemmify/0.1.6/\#styling-parameters}{styling}
functions the \texttt{\ .with\ } function is used.

\begin{Shaded}
\begin{Highlighting}[]
\NormalTok{\#let (}
\NormalTok{  theorem,}
\NormalTok{  rules: thm{-}rules}
\NormalTok{) = default{-}theorems(}
\NormalTok{  "thm{-}group",}
\NormalTok{  thm{-}numbering: thm{-}numbering{-}heading.with(max{-}heading{-}level: 2)}
\NormalTok{)}
\end{Highlighting}
\end{Shaded}

\subsection{Example}\label{example}

\begin{Shaded}
\begin{Highlighting}[]
\NormalTok{\#import "@preview/lemmify:0.1.6": *}

\NormalTok{\#let my{-}thm{-}style(}
\NormalTok{  thm{-}type, name, number, body}
\NormalTok{) = grid(}
\NormalTok{  columns: (1fr, 3fr),}
\NormalTok{  column{-}gutter: 1em,}
\NormalTok{  stack(spacing: .5em, strong(thm{-}type), number, emph(name)),}
\NormalTok{  body}
\NormalTok{)}

\NormalTok{\#let my{-}styling = (}
\NormalTok{  thm{-}styling: my{-}thm{-}style}
\NormalTok{)}

\NormalTok{\#let (}
\NormalTok{  theorem, rules}
\NormalTok{) = default{-}theorems("thm{-}group", lang: "en", ..my{-}styling)}
\NormalTok{\#show: rules}
\NormalTok{\#show thm{-}selector("thm{-}group"): box.with(inset: 1em)}

\NormalTok{\#lorem(20)}
\NormalTok{\#theorem[}
\NormalTok{  \#lorem(40)}
\NormalTok{]}
\NormalTok{\#lorem(20)}
\NormalTok{\#theorem(name: "Some theorem")[}
\NormalTok{  \#lorem(30)}
\NormalTok{]}
\end{Highlighting}
\end{Shaded}

\pandocbounded{\includegraphics[keepaspectratio]{https://github.com/Marmare314/lemmify/assets/49279081/b3c72b3e-7e21-4acd-82bb-3d63f87ec84b}}

\subsection{Documentation}\label{documentation}

The two most important functions are:

\texttt{\ default-theorems\ } : Create a default set of theorems based
on the given language and styling.

\begin{itemize}
\tightlist
\item
  \texttt{\ group\ } : The group id.
\item
  \texttt{\ lang\ } : The language to which the theorems are adapted.
\item
  \texttt{\ thm-styling\ } , \texttt{\ thm-numbering\ } ,
  \texttt{\ ref-styling\ } : Styling parameters are explained in further
  detail in the
  \href{https://github.com/typst/packages/raw/main/packages/preview/lemmify/0.1.6/\#styling-parameters}{Styling}
  section.
\item
  \texttt{\ proof-styling\ } : Styling which is only applied to proofs.
\item
  \texttt{\ max-reset-level\ } : The highest heading level on which
  theorems are still reset.
\end{itemize}

\texttt{\ new-theorems\ } : Create custom sets of theorems with the
given styling.

\begin{itemize}
\tightlist
\item
  \texttt{\ group\ } : The group id.
\item
  \texttt{\ subgroup-map\ } : Mapping from group id to some argument.
  The simple styles use \texttt{\ thm-type\ } as the argument (ie
  “Beispiel� or “Example� for group id “example�)
\item
  \texttt{\ thm-styling\ } , \texttt{\ thm-numbering\ } ,
  \texttt{\ ref-styling\ } , \texttt{\ ref-numbering\ } : Styling which
  to apply to all subgroups.
\end{itemize}

\begin{center}\rule{0.5\linewidth}{0.5pt}\end{center}

\texttt{\ use-proof-numbering\ } : Decreases the numbering of a theorem
function by one. See
\href{https://github.com/typst/packages/raw/main/packages/preview/lemmify/0.1.6/\#styling}{Styling}
for more information.

\begin{center}\rule{0.5\linewidth}{0.5pt}\end{center}

\texttt{\ thm-selector\ } : Returns a selector for all theorems of the
specified group. If subgroup is specified, only the theorems belonging
to it will be selected.

\begin{center}\rule{0.5\linewidth}{0.5pt}\end{center}

There are also a few functions to help with resetting counters.

\texttt{\ thm-reset-counter\ } : Reset theorem group counter manually.
Returned content needs to added to the document.

\texttt{\ thm-reset-counter-heading-at\ } : Reset theorem group counter
at headings of the specified level. Returns a rule that needs to be
shown.

\texttt{\ thm-reset-counter-heading\ } : Reset theorem group counter at
headings of at most the specified level. Returns a rule that needs to be
shown.

\subsubsection{Styling parameters}\label{styling-parameters}

If possible the best way to adapt the look of theorems is to use show
rules as shown
\href{https://github.com/typst/packages/raw/main/packages/preview/lemmify/0.1.6/\#basic-usage}{above}
, but this is not always possible. For example if we wanted theorems to
start with \texttt{\ 1.1\ Theorem\ } instead of
\texttt{\ Theorem\ 1.1\ } . You can provide the following functions to
adapt the look of the theorems.

\begin{center}\rule{0.5\linewidth}{0.5pt}\end{center}

\texttt{\ thm-styling\ } : A function:
\texttt{\ (arg,\ name,\ number,\ body)\ -\textgreater{}\ content\ } ,
that allows you to define the styling for different types of theorems.
Below only the \texttt{\ arg\ } will be specified.

Pre-defined functions

\begin{itemize}
\tightlist
\item
  \texttt{\ thm-style-simple(thm-type)\ } : \textbf{thm-type num}
  \emph{(name)} body
\item
  \texttt{\ thm-style-proof(thm-type)\ } : \textbf{thm-type num}
  \emph{(name)} body â--¡
\item
  \texttt{\ thm-style-reversed(thm-type)\ } : \textbf{num thm-type}
  \emph{(name)} body
\end{itemize}

\begin{center}\rule{0.5\linewidth}{0.5pt}\end{center}

\texttt{\ thm-numbering\ } : A function:
\texttt{\ figure\ -\textgreater{}\ content\ } , that determines how
theorems are numbered.

Pre-defined functions: (Assume heading is 1.1 and theorem count is 2)

\begin{itemize}
\tightlist
\item
  \texttt{\ thm-numbering-heading\ } : 1.1.2

  \begin{itemize}
  \tightlist
  \item
    \texttt{\ max-heading-level\ } : only use the a limited number of
    subheadings. In this case if it is set to \texttt{\ 1\ } the result
    would be \texttt{\ 1.2\ } instead.
  \end{itemize}
\item
  \texttt{\ thm-numbering-linear\ } : 2
\item
  \texttt{\ thm-numbering-proof\ } : No visible content is returned, but
  the counter is reduced by 1 (so that the proof keeps the same count as
  the theorem). Useful in combination with
  \texttt{\ use-proof-numbering\ } to create theorems that reference the
  previous theorem (like proofs).
\end{itemize}

\begin{center}\rule{0.5\linewidth}{0.5pt}\end{center}

\texttt{\ ref-styling\ } : A function:
\texttt{\ (arg,\ thm-numbering,\ ref)\ -\textgreater{}\ content\ } , to
style theorem references.

Pre-defined functions:

\begin{itemize}
\tightlist
\item
  \texttt{\ thm-ref-style-simple(thm-type)\ }

  \begin{itemize}
  \tightlist
  \item
    \texttt{\ @thm\ -\textgreater{}\ thm-type\ 1.1\ }
  \item
    \texttt{\ @thm{[}custom{]}\ -\textgreater{}\ custom\ 1.1\ }
  \end{itemize}
\end{itemize}

\begin{center}\rule{0.5\linewidth}{0.5pt}\end{center}

\texttt{\ ref-numbering\ } : Same as \texttt{\ thm-numbering\ } but only
applies to the references.

\subsection{Roadmap}\label{roadmap}

\begin{itemize}
\tightlist
\item
  More pre-defined styles.

  \begin{itemize}
  \tightlist
  \item
    Referencing theorems by name.
  \end{itemize}
\item
  Support more languages.
\item
  Better documentation.
\item
  Outlining theorems.
\end{itemize}

If you are encountering any bugs, have questions or are missing
features, feel free to open an issue on
\href{https://github.com/Marmare314/lemmify}{Github} .

\subsection{Changelog}\label{changelog}

\begin{itemize}
\item
  Version 0.1.6

  \begin{itemize}
  \tightlist
  \item
    Add Portuguese translation (
    \href{https://github.com/PgBiel}{@PgBiel} )
  \item
    Add Catalan translation (
    \href{https://github.com/Eloitor}{@Eloitor} )
  \item
    Add Spanish translation (
    \href{https://github.com/mismorgano}{@mismorgano} )
  \item
    Remove extra space before empty supplements (
    \href{https://github.com/PgBiel}{@PgBiel} )
  \item
    Use ref-styling parameter of default-theorems
  \end{itemize}
\item
  Version 0.1.5

  \begin{itemize}
  \tightlist
  \item
    Add Russian translation (
    \href{https://github.com/WeetHet}{@WeetHet} )
  \end{itemize}
\item
  Version 0.1.4

  \begin{itemize}
  \tightlist
  \item
    Fix error on unnamed theorems
  \end{itemize}
\item
  Version 0.1.3

  \begin{itemize}
  \tightlist
  \item
    Allow “1.1.� numbering style by default
  \item
    Ignore unnumbered subheadings
  \item
    Add max-heading-level parameter to thm-numbering-heading
  \item
    Adapt lemmify to typst version 0.8.0
  \end{itemize}
\item
  Version 0.1.2

  \begin{itemize}
  \tightlist
  \item
    Better error message on unnumbered headings
  \item
    Add Italian translations (
    \href{https://github.com/porcaror}{@porcaror} )
  \end{itemize}
\item
  Version 0.1.1

  \begin{itemize}
  \tightlist
  \item
    Add Dutch translations (
    \href{https://github.com/BroodjeKroepoek}{@BroodjeKroepoek} )
  \item
    Add French translations ( \href{https://github.com/MDLC01}{@MDLC01}
    )
  \item
    Fix size of default styles and make them breakable
  \end{itemize}
\end{itemize}

\subsubsection{How to add}\label{how-to-add}

Copy this into your project and use the import as \texttt{\ lemmify\ }

\begin{verbatim}
#import "@preview/lemmify:0.1.6"
\end{verbatim}

\includesvg[width=0.16667in,height=0.16667in]{/assets/icons/16-copy.svg}

Check the docs for
\href{https://typst.app/docs/reference/scripting/\#packages}{more
information on how to import packages} .

\subsubsection{About}\label{about}

\begin{description}
\tightlist
\item[Author :]
Marmare314
\item[License:]
GPL-3.0-only
\item[Current version:]
0.1.6
\item[Last updated:]
August 29, 2024
\item[First released:]
July 2, 2023
\item[Archive size:]
18.2 kB
\href{https://packages.typst.org/preview/lemmify-0.1.6.tar.gz}{\pandocbounded{\includesvg[keepaspectratio]{/assets/icons/16-download.svg}}}
\item[Repository:]
\href{https://github.com/Marmare314/lemmify}{GitHub}
\end{description}

\subsubsection{Where to report issues?}\label{where-to-report-issues}

This package is a project of Marmare314 . Report issues on
\href{https://github.com/Marmare314/lemmify}{their repository} . You can
also try to ask for help with this package on the
\href{https://forum.typst.app}{Forum} .

Please report this package to the Typst team using the
\href{https://typst.app/contact}{contact form} if you believe it is a
safety hazard or infringes upon your rights.

\phantomsection\label{versions}
\subsubsection{Version history}\label{version-history}

\begin{longtable}[]{@{}ll@{}}
\toprule\noalign{}
Version & Release Date \\
\midrule\noalign{}
\endhead
\bottomrule\noalign{}
\endlastfoot
0.1.6 & August 29, 2024 \\
\href{https://typst.app/universe/package/lemmify/0.1.5/}{0.1.5} &
December 3, 2023 \\
\href{https://typst.app/universe/package/lemmify/0.1.4/}{0.1.4} &
September 26, 2023 \\
\href{https://typst.app/universe/package/lemmify/0.1.3/}{0.1.3} &
September 22, 2023 \\
\href{https://typst.app/universe/package/lemmify/0.1.2/}{0.1.2} & July
24, 2023 \\
\href{https://typst.app/universe/package/lemmify/0.1.1/}{0.1.1} & July
8, 2023 \\
\href{https://typst.app/universe/package/lemmify/0.1.0/}{0.1.0} & July
2, 2023 \\
\end{longtable}

Typst GmbH did not create this package and cannot guarantee correct
functionality of this package or compatibility with any version of the
Typst compiler or app.
