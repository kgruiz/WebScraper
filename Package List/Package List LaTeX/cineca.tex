\title{typst.app/universe/package/cineca}

\phantomsection\label{banner}
\section{cineca}\label{cineca}

{ 0.4.0 }

A package to create calendar with events.

\phantomsection\label{readme}
CINECA Is Not an Electric Calendar App, but a Typst package to create
calendars with events.

\subsection{Usage}\label{usage}

The package now support creating events from ICS files (thanks
@Geronymos). To do so, read an ICS file and parse with
\texttt{\ ics-parser()\ } .

\begin{Shaded}
\begin{Highlighting}[]
\NormalTok{\#let events2 = ics{-}parser(read("sample.ics")).map(event =\textgreater{} (}
\NormalTok{  event.dtstart, }
\NormalTok{  event.dtstart,}
\NormalTok{  event.dtend,}
\NormalTok{  event.summary}
\NormalTok{))}

\NormalTok{\#calendar(events2, hour{-}range: (10, 14))}
\end{Highlighting}
\end{Shaded}

\subsubsection{Day view}\label{day-view}

\texttt{\ calendar(events,\ hour-range,\ minute-height,\ template,\ stroke)\ }

Parameters:

\begin{itemize}
\tightlist
\item
  \texttt{\ events\ } : An array of events. Each item is a 4-element
  array:

  \begin{itemize}
  \tightlist
  \item
    Date. Can be \texttt{\ datetime()\ } or valid args of
    \texttt{\ day()\ } .
  \item
    Start time. Can be valid args of \texttt{\ time()\ } .
  \item
    End time. Can be valid args of \texttt{\ time()\ } .
  \item
    Event body. Can be anything. Passed to the template.body to show
    more details.
  \end{itemize}
\item
  \texttt{\ hour-range\ } : Then range of hours, affacting the range of
  the calendar. Default: \texttt{\ (8,\ 20)\ } .
\item
  \texttt{\ minute-height\ } : Height of per minute. Each minute occupys
  a row. This number is to control the height of each row. Default:
  \texttt{\ 0.8pt\ } .
\item
  \texttt{\ template\ } : Templates for headers, times, or events. It
  takes a dictionary of the following entries: \texttt{\ header\ } ,
  \texttt{\ time\ } , and \texttt{\ event\ } . Default: \texttt{\ (:)\ }
  .
\item
  \texttt{\ stroke\ } : A stroke style to control the style of the
  default stroke, or a function taking two parameters
  \texttt{\ (x,\ y)\ } to control the stroke. The first row is the
  dates, and the first column is the times. Default: \texttt{\ none\ } .
\end{itemize}

\begin{quote}
{[}!NOTE{]} See below for more details about the format of start time
and end time.
\end{quote}

Example:

\pandocbounded{\includegraphics[keepaspectratio]{https://github.com/typst/packages/raw/main/packages/preview/cineca/0.4.0/test/day-view.png}}

\subsubsection{Month view}\label{month-view}

\texttt{\ calendar-month(events,\ template,\ sunday-first,\ ..args)\ }

\begin{itemize}
\tightlist
\item
  \texttt{\ events\ } : Event list. Each element is a two-element array.

  \begin{itemize}
  \tightlist
  \item
    Day. A datetime object.
  \item
    Additional information for showing a day. It actually depends on the
    template \texttt{\ day-body\ } . For the deafult template, it
    requires a content.
  \end{itemize}
\item
  \texttt{\ template\ } : Templates for headers, times, or events. It
  takes a dictionary of the following entries: \texttt{\ day-body\ } ,
  \texttt{\ day-head\ } , \texttt{\ month-head\ } , and
  \texttt{\ layout\ } .
\item
  \texttt{\ sunday-first\ } : Whether to put sunday as the first day of
  a week.
\item
  \texttt{\ ..args\ } : Additional arguments for the calendar’s grid.
\end{itemize}

Example:

\begin{Shaded}
\begin{Highlighting}[]
\NormalTok{\#let events = (}
\NormalTok{  (daytime("2024{-}2{-}1", "9:0:0"), [Lecture]),}
\NormalTok{  (daytime("2024{-}2{-}1", "10:0:0"), [Tutorial]),}
\NormalTok{  (daytime("2024{-}2{-}2", "10:0:0"), [Meeting]),}
\NormalTok{  (daytime("2024{-}2{-}10", "12:0:0"), [Lunch]),}
\NormalTok{  (daytime("2024{-}2{-}25", "8:0:0"), [Train]),}
\NormalTok{)}

\NormalTok{\#calendar{-}month(}
\NormalTok{  events,}
\NormalTok{  sunday{-}first: false,}
\NormalTok{  template: (}
\NormalTok{    month{-}head: (content) =\textgreater{} strong(content)}
\NormalTok{  )}
\NormalTok{)}
\end{Highlighting}
\end{Shaded}

\begin{Shaded}
\begin{Highlighting}[]
\NormalTok{\#let events2 = (}
\NormalTok{  (datetime(year: 2024, month: 5, day: 1, hour: 9, minute: 0, second: 0), ([Lecture], blue)),}
\NormalTok{  (datetime(year: 2024, month: 5, day: 1, hour: 10, minute: 0, second: 0), ([Tutorial], red)),}
\NormalTok{  (datetime(year: 2024, month: 5, day: 1, hour: 11, minute: 0, second: 0), [Lab]),}
\NormalTok{)}

\NormalTok{\#calendar{-}month(}
\NormalTok{  events2,}
\NormalTok{  sunday{-}first: true,}
\NormalTok{  rows: (2em,) * 2 + (8em,),}
\NormalTok{  template: (}
\NormalTok{    day{-}body: (day, events) =\textgreater{} \{}
\NormalTok{      show: block.with(width: 100\%, height: 100\%, inset: 2pt)}
\NormalTok{      set align(left + top)}
\NormalTok{      stack(}
\NormalTok{        spacing: 2pt,}
\NormalTok{        pad(bottom: 4pt, text(weight: "bold", day.display("[day]"))),}
\NormalTok{        ..events.map(((d, e)) =\textgreater{} \{}
\NormalTok{          let col = if type(e) == array and e.len() \textgreater{} 1 \{ e.at(1) \} else \{ yellow \}}
\NormalTok{          show: block.with(}
\NormalTok{            fill: col.lighten(40\%),}
\NormalTok{            stroke: col,}
\NormalTok{            width: 100\%,}
\NormalTok{            inset: 2pt,}
\NormalTok{            radius: 2pt}
\NormalTok{          )}
\NormalTok{          d.display("[hour]")}
\NormalTok{          h(4pt)}
\NormalTok{          if type(e) == array \{ e.at(0) \} else \{ e \}}
\NormalTok{        \})}
\NormalTok{      )}
\NormalTok{    \}}
\NormalTok{  )}
\NormalTok{)}
\end{Highlighting}
\end{Shaded}

\pandocbounded{\includegraphics[keepaspectratio]{https://github.com/typst/packages/raw/main/packages/preview/cineca/0.4.0/test/month-view.png}}

\subsubsection{Month-summary view}\label{month-summary-view}

\texttt{\ calendar-month-summary(events,\ template,\ sunday-first,\ ..args)\ }

\begin{itemize}
\tightlist
\item
  \texttt{\ events\ } : Event list. Each element is a two-element array.

  \begin{itemize}
  \tightlist
  \item
    Day. A datetime object.
  \item
    Additional information for showing a day. It actually depends on the
    template \texttt{\ day-body\ } . For the deafult template, it
    requires an array of two elements.

    \begin{itemize}
    \tightlist
    \item
      Shape. A function specify how to darw the shape, such as
      \texttt{\ circle\ } .
    \item
      Arguments. Further arguments for render a shape.
    \end{itemize}
  \end{itemize}
\item
  \texttt{\ template\ } : Templates for headers, times, or events. It
  takes a dictionary of the following entries: \texttt{\ day-body\ } ,
  \texttt{\ day-head\ } , \texttt{\ month-head\ } , and
  \texttt{\ layout\ } .
\item
  \texttt{\ sunday-first\ } : Whether to put sunday as the first day of
  a week.
\item
  \texttt{\ ..args\ } : Additional arguments for the calendar’s grid.
\end{itemize}

Example:

\begin{Shaded}
\begin{Highlighting}[]
\NormalTok{\#let events = (}
\NormalTok{  (day("2024{-}02{-}21"), (circle, (stroke: color.green, inset: 2pt))),}
\NormalTok{  (day("2024{-}02{-}22"), (circle, (stroke: color.green, inset: 2pt))),}
\NormalTok{  (day("2024{-}05{-}27"), (circle, (stroke: color.green, inset: 2pt))),}
\NormalTok{  (day("2024{-}05{-}28"), (circle, (stroke: color.blue, inset: 2pt))),}
\NormalTok{  (day("2024{-}05{-}29"), (circle, (stroke: color.blue, inset: 2pt))),}
\NormalTok{  (day("2024{-}06{-}03"), (circle, (stroke: color.blue, inset: 2pt))),}
\NormalTok{  (day("2024{-}06{-}04"), (circle, (stroke: color.yellow, inset: 2pt))),}
\NormalTok{  (day("2024{-}06{-}05"), (circle, (stroke: color.yellow, inset: 2pt))),}
\NormalTok{  (day("2024{-}06{-}10"), (circle, (stroke: color.red, inset: 2pt))),}
\NormalTok{)}

\NormalTok{\#calendar{-}month{-}summary(}
\NormalTok{  events: events}
\NormalTok{)}

\NormalTok{\#calendar{-}month{-}summary(}
\NormalTok{  events: events,}
\NormalTok{  sunday{-}first: true}
\NormalTok{)}

\NormalTok{// An empty calendar}
\NormalTok{\#calendar{-}month{-}summary(}
\NormalTok{  events: (}
\NormalTok{    (datetime(year: 2024, month: 05, day: 21), (none,)),}
\NormalTok{  ),}
\NormalTok{  stroke: 1pt,}
\NormalTok{)}
\end{Highlighting}
\end{Shaded}

\pandocbounded{\includegraphics[keepaspectratio]{https://github.com/typst/packages/raw/main/packages/preview/cineca/0.4.0/test/month-summary.png}}

\subsection{Day/Time/Daytime Format}\label{daytimedaytime-format}

In addition to using \texttt{\ datetime()\ } to set up time, the package
provided some other ways, supported by functions \texttt{\ day()\ } ,
\texttt{\ time()\ } , and \texttt{\ daytime()\ } .

\begin{Shaded}
\begin{Highlighting}[]
\NormalTok{{-} \#time(8)}
\NormalTok{{-} \#time(8, 10)}
\NormalTok{{-} \#time(8, 10, 30)}
\NormalTok{{-} \#time("8.30")}
\NormalTok{{-} \#time(decimal("12.10"))}
\NormalTok{{-} \#time(14.10)            // 24{-}hour format}
\NormalTok{{-} \#time("8:10:08")}

\NormalTok{{-} \#day(2024)}
\NormalTok{{-} \#day(2024, 2)}
\NormalTok{{-} \#day(2024, 2, 5)    // year, month, day}
\NormalTok{{-} \#day("2024{-}3{-}7")    // ISO format (year{-}month{-}day)}
\NormalTok{{-} \#day("26/12/2024")  // British format (day/month/year)}

\NormalTok{{-} \#daytime(2024)}
\NormalTok{{-} \#daytime(2024, 2)}
\NormalTok{{-} \#daytime(2024, 2, 5)}
\NormalTok{{-} \#daytime(2024, 2, 5, 8)}
\NormalTok{{-} \#daytime(2024, 2, 5, 8, 10)}
\NormalTok{{-} \#daytime("2024{-}6{-}1", 8)}
\NormalTok{{-} \#daytime("2024{-}6{-}1", 8, 10)}
\NormalTok{{-} \#daytime("2024{-}6{-}1", 8, 10, 30)}
\NormalTok{{-} \#daytime(2024, "12:00")}
\NormalTok{{-} \#daytime(2024, 2, "12:00")}
\NormalTok{{-} \#daytime(2024, 2, 5, "12:00")}
\NormalTok{{-} \#daytime("2024{-}3{-}7", "11:30:45")}
\NormalTok{{-} \#daytime("2024{-}12{-}26 8:30")}
\end{Highlighting}
\end{Shaded}

\subsection{Limitations}\label{limitations}

\begin{itemize}
\tightlist
\item
  Page breaking may be incorrect.
\item
  Items will overlap when they happens at the same time.
\end{itemize}

\subsubsection{How to add}\label{how-to-add}

Copy this into your project and use the import as \texttt{\ cineca\ }

\begin{verbatim}
#import "@preview/cineca:0.4.0"
\end{verbatim}

\includesvg[width=0.16667in,height=0.16667in]{/assets/icons/16-copy.svg}

Check the docs for
\href{https://typst.app/docs/reference/scripting/\#packages}{more
information on how to import packages} .

\subsubsection{About}\label{about}

\begin{description}
\tightlist
\item[Author :]
HPDell
\item[License:]
MIT
\item[Current version:]
0.4.0
\item[Last updated:]
November 25, 2024
\item[First released:]
April 1, 2024
\item[Archive size:]
7.04 kB
\href{https://packages.typst.org/preview/cineca-0.4.0.tar.gz}{\pandocbounded{\includesvg[keepaspectratio]{/assets/icons/16-download.svg}}}
\item[Repository:]
\href{https://github.com/HPdell/typst-cineca}{GitHub}
\item[Categor ies :]
\begin{itemize}
\tightlist
\item[]
\item
  \pandocbounded{\includesvg[keepaspectratio]{/assets/icons/16-layout.svg}}
  \href{https://typst.app/universe/search/?category=layout}{Layout}
\item
  \pandocbounded{\includesvg[keepaspectratio]{/assets/icons/16-chart.svg}}
  \href{https://typst.app/universe/search/?category=visualization}{Visualization}
\end{itemize}
\end{description}

\subsubsection{Where to report issues?}\label{where-to-report-issues}

This package is a project of HPDell . Report issues on
\href{https://github.com/HPdell/typst-cineca}{their repository} . You
can also try to ask for help with this package on the
\href{https://forum.typst.app}{Forum} .

Please report this package to the Typst team using the
\href{https://typst.app/contact}{contact form} if you believe it is a
safety hazard or infringes upon your rights.

\phantomsection\label{versions}
\subsubsection{Version history}\label{version-history}

\begin{longtable}[]{@{}ll@{}}
\toprule\noalign{}
Version & Release Date \\
\midrule\noalign{}
\endhead
\bottomrule\noalign{}
\endlastfoot
0.4.0 & November 25, 2024 \\
\href{https://typst.app/universe/package/cineca/0.3.0/}{0.3.0} &
November 18, 2024 \\
\href{https://typst.app/universe/package/cineca/0.2.1/}{0.2.1} & July 1,
2024 \\
\href{https://typst.app/universe/package/cineca/0.2.0/}{0.2.0} & May 22,
2024 \\
\href{https://typst.app/universe/package/cineca/0.1.0/}{0.1.0} & April
1, 2024 \\
\end{longtable}

Typst GmbH did not create this package and cannot guarantee correct
functionality of this package or compatibility with any version of the
Typst compiler or app.
