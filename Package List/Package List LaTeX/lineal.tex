\title{typst.app/universe/package/lineal}

\phantomsection\label{banner}
\section{lineal}\label{lineal}

{ 0.1.0 }

Build elegent slide decks with Typst

\phantomsection\label{readme}
IPA: /ˈlɪniəl/

Made up of, or having the characteristic of, lines.

Lineal is a Typst template for generating beautifully clean and
configurably awesome slides.

\pandocbounded{\includegraphics[keepaspectratio]{https://github.com/typst/packages/raw/main/packages/preview/lineal/0.1.0/assets/img/demo.gif}}

\subsection{Philosophy}\label{philosophy}

As a long time user of TeX, I have developed and embedded countless
LaTeX applications into personal and corporate environments, e.g.
automating documentation, conference materials, posters, resumes, etc.

However, LaTeX is showing its age. Compiling a some 30-slide
presentation, for instance, takes perhaps a second, and may requires
multiple renders to sync TikZ positioning and cross-document
referencing. Typst remediates these issues in real-time and with better
control and confidence in its data modelling.

Hence, Lineal was born. A professional set of slides produced near
instantly, readily equipped with configurable theming and a suite of
flexible components.

\subsection{Usage}\label{usage}

Lineal is available through Typst Universe. Ensure you have installed
Typst locally or are familiar with the Typst
\href{https://typst.app/}{web app} or the
\href{https://marketplace.visualstudio.com/items?itemName=myriad-dreamin.tinymist}{Tinymist
LSP} extensions for VS Code.

Get started by importing the package and populating your own
\texttt{\ /content/\textless{}slug\textgreater{}.typ\ } files:

\begin{Shaded}
\begin{Highlighting}[]
\NormalTok{\#import "@preview/lineal:0.1.0": lineal{-}theme}

\NormalTok{\#show: lineal{-}theme.with(}
\NormalTok{  aspect{-}ratio: "16{-}9",}
\NormalTok{  config{-}info(}
\NormalTok{    title: [$bb("L")"ineal"$],}
\NormalTok{    subtitle: [A Typst slide template],}
\NormalTok{    author: [Author],}
\NormalTok{    date: datetime.today(),}
\NormalTok{    institution: [Institution],}
\NormalTok{    logo: [Logo],}
\NormalTok{  ),}
\NormalTok{)}

\NormalTok{\#title{-}slide()}

\NormalTok{\#include "content/index.typ"}
\end{Highlighting}
\end{Shaded}

Marking up content is as you would with any other Typst document, where
the section ( \texttt{\ =\ \textless{}section-title\textgreater{}\ } )
and subsection ( \texttt{\ ==\ \textless{}slide-title\textgreater{}\ } )
shorthands generate the new section slides with dynamic outline and new
tracked slides respectively.

\subsection{Contributing}\label{contributing}

PRs are very welcome. If you think Lineal could be improved in any way
or is missing a feature, please raise a request 😎

\subsection{Acknowledgements}\label{acknowledgements}

A heartfelt thank you to the team behind
\href{https://github.com/typst/typst}{Typst} , developing a product that
not only preserves the beauty of LaTeX’s typesetting, but improves on
its developer experience in every way, in line with ongoing community
feedback.

The creators of the
\href{https://github.com/touying-typ/touying}{\texttt{\ Touying\ }} and
\href{https://github.com/andreasKroepelin/polylux}{\texttt{\ Polylux\ }}
Typst packages, on which Lineal is built.

\subsubsection{How to add}\label{how-to-add}

Copy this into your project and use the import as \texttt{\ lineal\ }

\begin{verbatim}
#import "@preview/lineal:0.1.0"
\end{verbatim}

\includesvg[width=0.16667in,height=0.16667in]{/assets/icons/16-copy.svg}

Check the docs for
\href{https://typst.app/docs/reference/scripting/\#packages}{more
information on how to import packages} .

\subsubsection{About}\label{about}

\begin{description}
\tightlist
\item[Author :]
\href{https://github.com/ellsphillips}{ellsphillips}
\item[License:]
MIT
\item[Current version:]
0.1.0
\item[Last updated:]
November 28, 2024
\item[First released:]
November 28, 2024
\item[Archive size:]
7.40 kB
\href{https://packages.typst.org/preview/lineal-0.1.0.tar.gz}{\pandocbounded{\includesvg[keepaspectratio]{/assets/icons/16-download.svg}}}
\item[Repository:]
\href{https://github.com/ellsphillips/lineal}{GitHub}
\item[Categor y :]
\begin{itemize}
\tightlist
\item[]
\item
  \pandocbounded{\includesvg[keepaspectratio]{/assets/icons/16-presentation.svg}}
  \href{https://typst.app/universe/search/?category=presentation}{Presentation}
\end{itemize}
\end{description}

\subsubsection{Where to report issues?}\label{where-to-report-issues}

This package is a project of ellsphillips . Report issues on
\href{https://github.com/ellsphillips/lineal}{their repository} . You
can also try to ask for help with this package on the
\href{https://forum.typst.app}{Forum} .

Please report this package to the Typst team using the
\href{https://typst.app/contact}{contact form} if you believe it is a
safety hazard or infringes upon your rights.

\phantomsection\label{versions}
\subsubsection{Version history}\label{version-history}

\begin{longtable}[]{@{}ll@{}}
\toprule\noalign{}
Version & Release Date \\
\midrule\noalign{}
\endhead
\bottomrule\noalign{}
\endlastfoot
0.1.0 & November 28, 2024 \\
\end{longtable}

Typst GmbH did not create this package and cannot guarantee correct
functionality of this package or compatibility with any version of the
Typst compiler or app.
