\title{typst.app/universe/package/wrap-it}

\phantomsection\label{banner}
\section{wrap-it}\label{wrap-it}

{ 0.1.1 }

Wrap text around figures and content

{ } Featured Package

\phantomsection\label{readme}
Until \ul{\ul{\url{https://github.com/typst/typst/issues/553}}} is
resolved, \texttt{\ typst\ } doesn’t natively support wrapping text
around figures or other content. However, you can use
\texttt{\ wrap-it\ } to mimic much of this functionality:

\begin{itemize}
\item
  Wrapping images left or right of their text
\item
  Specifying margins
\item
  And more
\end{itemize}

Detailed descriptions of each parameter are available in the
\ul{\ul{\href{https://github.com/ntjess/wrap-it/blob/main/docs/manual.pdf}{wrap-it
documentation}}} .

The easiest method is to import \texttt{\ wrap-it:\ wrap-content\ } from
the \texttt{\ @preview\ } package:

\begin{Shaded}
\begin{Highlighting}[]
\NormalTok{\#import "@preview/wrap{-}it:0.1.0": wrap{-}content}
\end{Highlighting}
\end{Shaded}

\subsection{Vanilla}\label{vanilla}

\begin{Shaded}
\begin{Highlighting}[]
\NormalTok{\#let fig = figure(}
\NormalTok{rect(fill: teal, radius: 0.5em, width: 8em),}
\NormalTok{caption: [A figure],}
\NormalTok{)}
\NormalTok{\#let body = lorem(30)}
\NormalTok{\#wrap{-}content(fig, body)}
\end{Highlighting}
\end{Shaded}

\pandocbounded{\includegraphics[keepaspectratio]{https://www.github.com/ntjess/wrap-it/raw/v0.1.1/assets/example-1.png}}

\subsection{Changing alignment and
margin}\label{changing-alignment-and-margin}

\begin{Shaded}
\begin{Highlighting}[]
\NormalTok{\#wrap{-}content(}
\NormalTok{fig,}
\NormalTok{body,}
\NormalTok{align: bottom + right,}
\NormalTok{column{-}gutter: 2em}
\NormalTok{)}
\end{Highlighting}
\end{Shaded}

\pandocbounded{\includegraphics[keepaspectratio]{https://www.github.com/ntjess/wrap-it/raw/v0.1.1/assets/example-2.png}}

\subsection{Uniform margin around the
image}\label{uniform-margin-around-the-image}

The easiest way to get a uniform, highly-customizable margin is through
boxing your image:

\begin{Shaded}
\begin{Highlighting}[]
\NormalTok{\#let boxed = box(fig, inset: 0.25em)}
\NormalTok{\#wrap{-}content(boxed)[}
\NormalTok{\#lorem(30)}
\NormalTok{]}
\end{Highlighting}
\end{Shaded}

\pandocbounded{\includegraphics[keepaspectratio]{https://www.github.com/ntjess/wrap-it/raw/v0.1.1/assets/example-3.png}}

\subsection{Wrapping two images in the same
paragraph}\label{wrapping-two-images-in-the-same-paragraph}

Note that for longer captions (as is the case in the bottom figure
below), providing an explicit \texttt{\ columns\ } parameter is
necessary to inform caption text of where to wrap.

\begin{Shaded}
\begin{Highlighting}[]
\NormalTok{\#let fig2 = figure(}
\NormalTok{rect(fill: lime, radius: 0.5em),}
\NormalTok{caption: [\#lorem(10)],}
\NormalTok{)}
\NormalTok{\#wrap{-}top{-}bottom(}
\NormalTok{bottom{-}kwargs: (columns: (1fr, 2fr)),}
\NormalTok{box(fig, inset: 0.25em),}
\NormalTok{fig2,}
\NormalTok{lorem(50),}
\NormalTok{)}
\end{Highlighting}
\end{Shaded}

\pandocbounded{\includegraphics[keepaspectratio]{https://www.github.com/ntjess/wrap-it/raw/v0.1.1/assets/example-4.png}}

\subsection{Adding a label to a wrapped
figure}\label{adding-a-label-to-a-wrapped-figure}

Typst can only append labels to figures in content mode. So, when
wrapping text around a figure that needs a label, you must first place
your figure in a content block with its label, then wrap it:

\begin{Shaded}
\begin{Highlighting}[]
\NormalTok{\#show ref: it =\textgreater{} underline(text(blue, it))}
\NormalTok{\#let fig = [}
\NormalTok{  \#figure(}
\NormalTok{    rect(fill: red, radius: 0.5em, width: 8em),}
\NormalTok{    caption:[Labeled]}
\NormalTok{  )\textless{}fig:lbl\textgreater{}}
\NormalTok{]}
\NormalTok{\#wrap{-}content(fig, [Fortunately, @fig:lbl\textquotesingle{}s label can be referenced within the wrapped text. \#lorem(15)])}
\end{Highlighting}
\end{Shaded}

\pandocbounded{\includegraphics[keepaspectratio]{https://www.github.com/ntjess/wrap-it/raw/v0.1.1/assets/example-5.png}}

\subsubsection{How to add}\label{how-to-add}

Copy this into your project and use the import as \texttt{\ wrap-it\ }

\begin{verbatim}
#import "@preview/wrap-it:0.1.1"
\end{verbatim}

\includesvg[width=0.16667in,height=0.16667in]{/assets/icons/16-copy.svg}

Check the docs for
\href{https://typst.app/docs/reference/scripting/\#packages}{more
information on how to import packages} .

\subsubsection{About}\label{about}

\begin{description}
\tightlist
\item[Author :]
Nathan Jessurun
\item[License:]
Unlicense
\item[Current version:]
0.1.1
\item[Last updated:]
November 28, 2024
\item[First released:]
January 26, 2024
\item[Archive size:]
5.30 kB
\href{https://packages.typst.org/preview/wrap-it-0.1.1.tar.gz}{\pandocbounded{\includesvg[keepaspectratio]{/assets/icons/16-download.svg}}}
\item[Repository:]
\href{https://github.com/ntjess/wrap-it}{GitHub}
\end{description}

\subsubsection{Where to report issues?}\label{where-to-report-issues}

This package is a project of Nathan Jessurun . Report issues on
\href{https://github.com/ntjess/wrap-it}{their repository} . You can
also try to ask for help with this package on the
\href{https://forum.typst.app}{Forum} .

Please report this package to the Typst team using the
\href{https://typst.app/contact}{contact form} if you believe it is a
safety hazard or infringes upon your rights.

\phantomsection\label{versions}
\subsubsection{Version history}\label{version-history}

\begin{longtable}[]{@{}ll@{}}
\toprule\noalign{}
Version & Release Date \\
\midrule\noalign{}
\endhead
\bottomrule\noalign{}
\endlastfoot
0.1.1 & November 28, 2024 \\
\href{https://typst.app/universe/package/wrap-it/0.1.0/}{0.1.0} &
January 26, 2024 \\
\end{longtable}

Typst GmbH did not create this package and cannot guarantee correct
functionality of this package or compatibility with any version of the
Typst compiler or app.
