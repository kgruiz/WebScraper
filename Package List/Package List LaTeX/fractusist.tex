\title{typst.app/universe/package/fractusist}

\phantomsection\label{banner}
\section{fractusist}\label{fractusist}

{ 0.1.0 }

Create a variety of wonderful fractals in Typst.

\phantomsection\label{readme}
Create a variety of wonderful fractals in Typst.

\subsection{Examples}\label{examples}

The example below creates a dragon curve of the 12th iteration with the
\texttt{\ dragon-curve\ } function.

\pandocbounded{\includegraphics[keepaspectratio]{https://github.com/typst/packages/raw/main/packages/preview/fractusist/0.1.0/examples/dragon-curve-n12.png}}

Show code

\begin{Shaded}
\begin{Highlighting}[]
\NormalTok{\#set page(width: auto, height: auto, margin: 0pt)}

\NormalTok{\#dragon{-}curve(}
\NormalTok{  12,}
\NormalTok{  step{-}size: 6,}
\NormalTok{  stroke{-}style: stroke(}
\NormalTok{    paint: gradient.linear(..color.map.crest, angle: 45deg),}
\NormalTok{    thickness: 3pt,}
\NormalTok{    cap: "square"}
\NormalTok{  )}
\NormalTok{)}
\end{Highlighting}
\end{Shaded}

\subsection{Features}\label{features}

\begin{itemize}
\tightlist
\item
  Use SVG backend for image rendering.
\item
  Generate fractals using
  \href{https://en.wikipedia.org/wiki/L-system}{L-system} .
\item
  The number of iterations, step size, fill and stroke styles, etc. of
  generated fractals could be customized.
\end{itemize}

\subsection{Usage}\label{usage}

Import the latest version of this package with:

\begin{Shaded}
\begin{Highlighting}[]
\NormalTok{\#import "@preview/fractusist:0.1.0": *}
\end{Highlighting}
\end{Shaded}

Each function generates a specific fractal. The input and output
arguments of all functions have a similar style. Typical input arguments
are as follows:

\begin{itemize}
\tightlist
\item
  \texttt{\ n\ } : the number of iterations ( \textbf{the valid range of
  values depends on the specific function} ).
\item
  \emph{\texttt{\ step-size\ }} : step size (in pt).
\item
  \emph{\texttt{\ fill-style\ }} : fill style, can be \texttt{\ none\ }
  or color or gradient ( \textbf{exists only when the curve is closed}
  ).
\item
  \emph{\texttt{\ stroke-style\ }} : stroke style, can be
  \texttt{\ none\ } or color or gradient or stroke object.
\item
  \emph{\texttt{\ width\ }} : the width of the image.
\item
  \emph{\texttt{\ height\ }} : the height of the image.
\item
  \emph{\texttt{\ fit\ }} : how the image should adjust itself to a
  given area, “cover� / “contain� / “stretch�.
\end{itemize}

The content returned is the \texttt{\ image\ } element.

For more codes with these functions see
\href{https://github.com/typst/packages/raw/main/packages/preview/fractusist/0.1.0/tests}{tests}
.

\subsection{Reference}\label{reference}

\subsubsection{Dragon}\label{dragon}

\begin{itemize}
\tightlist
\item
  \texttt{\ dragon-curve\ } : Generate dragon curve (n: range
  \textbf{{[}0, 16{]}} ).
\end{itemize}

\begin{Shaded}
\begin{Highlighting}[]
\NormalTok{\#let dragon{-}curve(n, step{-}size: 10, stroke{-}style: black + 1pt, width: auto, height: auto, fit: "cover") = \{...\}}
\end{Highlighting}
\end{Shaded}

\subsubsection{Hilbert}\label{hilbert}

\begin{itemize}
\tightlist
\item
  \texttt{\ hilbert-curve\ } : Generate 2D Hilbert curve. (n: range
  \textbf{{[}1, 8{]}} ).
\end{itemize}

\begin{Shaded}
\begin{Highlighting}[]
\NormalTok{\#let hilbert{-}curve(n, step{-}size: 10, stroke{-}style: black + 1pt, width: auto, height: auto, fit: "cover") = \{...\}}
\end{Highlighting}
\end{Shaded}

\begin{itemize}
\tightlist
\item
  \texttt{\ peano-curve\ } : Generate 2D Peano curve (n: range
  \textbf{{[}1, 5{]}} ).
\end{itemize}

\begin{Shaded}
\begin{Highlighting}[]
\NormalTok{\#let peano{-}curve(n, step{-}size: 10, stroke{-}style: black + 1pt, width: auto, height: auto, fit: "cover") = \{...\}}
\end{Highlighting}
\end{Shaded}

\subsubsection{Koch}\label{koch}

\begin{itemize}
\tightlist
\item
  \texttt{\ koch-curve\ } : Generate Koch curve (n: range \textbf{{[}0,
  6{]}} ).
\end{itemize}

\begin{Shaded}
\begin{Highlighting}[]
\NormalTok{\#let koch{-}curve(n, step{-}size: 10, stroke{-}style: black + 1pt, width: auto, height: auto, fit: "cover") = \{...\}}
\end{Highlighting}
\end{Shaded}

\begin{itemize}
\tightlist
\item
  \texttt{\ koch-snowflake\ } : Generate Koch snowflake (n: range
  \textbf{{[}0, 6{]}} ).
\end{itemize}

\begin{Shaded}
\begin{Highlighting}[]
\NormalTok{\#let koch{-}snowflake(n, step{-}size: 10, fill{-}style: none, stroke{-}style: black + 1pt, width: auto, height: auto, fit: "cover") = \{...\}}
\end{Highlighting}
\end{Shaded}

\subsubsection{Sierpiński}\label{sierpiuxe5ski}

\begin{itemize}
\tightlist
\item
  \texttt{\ sierpinski-curve\ } : Generate classic Sierpiński curve (n:
  range \textbf{{[}0, 7{]}} ).
\end{itemize}

\begin{Shaded}
\begin{Highlighting}[]
\NormalTok{\#let sierpinski{-}curve(n, step{-}size: 10, fill{-}style: none, stroke{-}style: black + 1pt, width: auto, height: auto, fit: "cover") = \{...\}}
\end{Highlighting}
\end{Shaded}

\begin{itemize}
\tightlist
\item
  \texttt{\ sierpinski-square-curve\ } : Generate Sierpiński square
  curve (n: range \textbf{{[}0, 7{]}} ).
\end{itemize}

\begin{Shaded}
\begin{Highlighting}[]
\NormalTok{\#let sierpinski{-}square{-}curve(n, step{-}size: 10, fill{-}style: none, stroke{-}style: black + 1pt, width: auto, height: auto, fit: "cover") = \{...\}}
\end{Highlighting}
\end{Shaded}

\begin{itemize}
\tightlist
\item
  \texttt{\ sierpinski-arrowhead-curve\ } : Generate Sierpiński
  arrowhead curve (n: range \textbf{{[}0, 8{]}} ).
\end{itemize}

\begin{Shaded}
\begin{Highlighting}[]
\NormalTok{\#let sierpinski{-}arrowhead{-}curve(n, step{-}size: 10, stroke{-}style: black + 1pt, width: auto, height: auto, fit: "cover") = \{...\}}
\end{Highlighting}
\end{Shaded}

\begin{itemize}
\tightlist
\item
  \texttt{\ sierpinski-triangle\ } : Generate 2D Sierpiński triangle
  (n: range \textbf{{[}0, 6{]}} ).
\end{itemize}

\begin{Shaded}
\begin{Highlighting}[]
\NormalTok{\#let sierpinski{-}triangle(n, step{-}size: 10, fill{-}style: none, stroke{-}style: black + 1pt, width: auto, height: auto, fit: "cover") = \{...\}}
\end{Highlighting}
\end{Shaded}

\subsubsection{How to add}\label{how-to-add}

Copy this into your project and use the import as
\texttt{\ fractusist\ }

\begin{verbatim}
#import "@preview/fractusist:0.1.0"
\end{verbatim}

\includesvg[width=0.16667in,height=0.16667in]{/assets/icons/16-copy.svg}

Check the docs for
\href{https://typst.app/docs/reference/scripting/\#packages}{more
information on how to import packages} .

\subsubsection{About}\label{about}

\begin{description}
\tightlist
\item[Author :]
\href{https://github.com/liuguangxi}{Guangxi Liu}
\item[License:]
MIT
\item[Current version:]
0.1.0
\item[Last updated:]
May 6, 2024
\item[First released:]
May 6, 2024
\item[Minimum Typst version:]
0.11.0
\item[Archive size:]
5.75 kB
\href{https://packages.typst.org/preview/fractusist-0.1.0.tar.gz}{\pandocbounded{\includesvg[keepaspectratio]{/assets/icons/16-download.svg}}}
\item[Repository:]
\href{https://github.com/liuguangxi/fractusist}{GitHub}
\item[Discipline s :]
\begin{itemize}
\tightlist
\item[]
\item
  \href{https://typst.app/universe/search/?discipline=computer-science}{Computer
  Science}
\item
  \href{https://typst.app/universe/search/?discipline=mathematics}{Mathematics}
\end{itemize}
\item[Categor ies :]
\begin{itemize}
\tightlist
\item[]
\item
  \pandocbounded{\includesvg[keepaspectratio]{/assets/icons/16-package.svg}}
  \href{https://typst.app/universe/search/?category=components}{Components}
\item
  \pandocbounded{\includesvg[keepaspectratio]{/assets/icons/16-chart.svg}}
  \href{https://typst.app/universe/search/?category=visualization}{Visualization}
\end{itemize}
\end{description}

\subsubsection{Where to report issues?}\label{where-to-report-issues}

This package is a project of Guangxi Liu . Report issues on
\href{https://github.com/liuguangxi/fractusist}{their repository} . You
can also try to ask for help with this package on the
\href{https://forum.typst.app}{Forum} .

Please report this package to the Typst team using the
\href{https://typst.app/contact}{contact form} if you believe it is a
safety hazard or infringes upon your rights.

\phantomsection\label{versions}
\subsubsection{Version history}\label{version-history}

\begin{longtable}[]{@{}ll@{}}
\toprule\noalign{}
Version & Release Date \\
\midrule\noalign{}
\endhead
\bottomrule\noalign{}
\endlastfoot
0.1.0 & May 6, 2024 \\
\end{longtable}

Typst GmbH did not create this package and cannot guarantee correct
functionality of this package or compatibility with any version of the
Typst compiler or app.
