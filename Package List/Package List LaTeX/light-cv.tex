\title{typst.app/universe/package/light-cv}

\phantomsection\label{banner}
\phantomsection\label{template-thumbnail}
\pandocbounded{\includegraphics[keepaspectratio]{https://packages.typst.org/preview/thumbnails/light-cv-0.1.1-small.webp}}

\section{light-cv}\label{light-cv}

{ 0.1.1 }

Minimalistic CV template for your own CV. Please install the font
awesome fonts on your system before using the template.

\href{/app?template=light-cv&version=0.1.1}{Create project in app}

\phantomsection\label{readme}
This is my CV template written in Typst. You can find a two example CVs
in this repository as PDFs:

\begin{itemize}
\tightlist
\item
  \href{https://github.com/AnsgarLichter/light-cv/blob/main/cv-de.pdf}{German
  CV}
\item
  \href{https://github.com/AnsgarLichter/light-cv/blob/main/cv-en.pdf}{English
  CV}
\end{itemize}

\subsection{Setup}\label{setup}

To use the CV you have to install the font awesome fonts for the icons
to work. Please refer to the intstructons of the font awesome package
itself. You can find these on: -
\href{https://typst.app/universe/package/fontawesome}{Typst Universe} -
\href{https://github.com/duskmoon314/typst-fontawesome}{GitHub} .

\subsection{Functions}\label{functions}

\begin{enumerate}
\item
  \texttt{\ header\ } : Creates a page haeder including your name,
  current job title or any other sub title, socials and profile picture

  \begin{itemize}
  \tightlist
  \item
    \texttt{\ full-name\ } : your full name
  \item
    \texttt{\ job-title\ } : your current job title rendered below your
    name
  \item
    \texttt{\ socials\ } : array containing all socials. Every social
    must have the following properties: \texttt{\ icon\ } ,
    \texttt{\ link\ } and \texttt{\ text\ }
  \item
    \texttt{\ profile-picture\ } : path to your profile picture
  \end{itemize}
\item
  \texttt{\ section\ } : Creates a new section, e. g.
  \texttt{\ Professional\ Experience\ } or \texttt{\ Education\ }

  \begin{itemize}
  \tightlist
  \item
    \texttt{\ title\ } : section’s title
  \end{itemize}
\item
  \texttt{\ entry\ } : Adds an entry / item to the current section

  \begin{itemize}
  \tightlist
  \item
    \texttt{\ title\ } : the entry’s title, e. g. your job title
  \item
    \texttt{\ company-or-university\ } : the name of the institution
    which you were at, e. g. company or university
  \item
    \texttt{\ date\ } : start and end date of this entry, e. g. 2020 -
    2022
  \item
    \texttt{\ location\ } : describes where you worked, e. g. London, UK
  \item
    \texttt{\ logo\ } : path to the logo of this entry
  \item
    ``description`: description what you have done - normally supplied
    as a list
  \end{itemize}
\end{enumerate}

\subsection{Customization}\label{customization}

In the folder \texttt{\ settings\ } you will a file
\texttt{\ styles.typ\ } which includes all used styles. You can
customize them as you want to.

\subsection{Multi Language Support}\label{multi-language-support}

If you want to add a new language, copy the \texttt{\ cv-en.typ\ } and
rename it. Afterwards you can adapt the text correspondingly. Maybe I
will introduce i18n in the future.

\subsection{Insipration}\label{insipration}

A big thanks to
\href{https://github.com/mintyfrankie/brilliant-CV}{brilliant-CV} as
this project was an inspiraton for my CV and for the scripting.

\subsection{Questions \& Issues}\label{questions-issues}

If you have questions, plase create a
\href{https://github.com/AnsgarLichter/light-cv/discussions}{discussion}
. If you have an issue, please create an
\href{https://github.com/AnsgarLichter/light-cv/issues}{issue} .

\href{/app?template=light-cv&version=0.1.1}{Create project in app}

\subsubsection{How to use}\label{how-to-use}

Click the button above to create a new project using this template in
the Typst app.

You can also use the Typst CLI to start a new project on your computer
using this command:

\begin{verbatim}
typst init @preview/light-cv:0.1.1
\end{verbatim}

\includesvg[width=0.16667in,height=0.16667in]{/assets/icons/16-copy.svg}

\subsubsection{About}\label{about}

\begin{description}
\tightlist
\item[Author :]
Ansgar Lichter
\item[License:]
MIT
\item[Current version:]
0.1.1
\item[Last updated:]
May 6, 2024
\item[First released:]
April 17, 2024
\item[Archive size:]
414 kB
\href{https://packages.typst.org/preview/light-cv-0.1.1.tar.gz}{\pandocbounded{\includesvg[keepaspectratio]{/assets/icons/16-download.svg}}}
\item[Repository:]
\href{https://github.com/AnsgarLichter/cv-typst-template}{GitHub}
\item[Categor y :]
\begin{itemize}
\tightlist
\item[]
\item
  \pandocbounded{\includesvg[keepaspectratio]{/assets/icons/16-user.svg}}
  \href{https://typst.app/universe/search/?category=cv}{CV}
\end{itemize}
\end{description}

\subsubsection{Where to report issues?}\label{where-to-report-issues}

This template is a project of Ansgar Lichter . Report issues on
\href{https://github.com/AnsgarLichter/cv-typst-template}{their
repository} . You can also try to ask for help with this template on the
\href{https://forum.typst.app}{Forum} .

Please report this template to the Typst team using the
\href{https://typst.app/contact}{contact form} if you believe it is a
safety hazard or infringes upon your rights.

\phantomsection\label{versions}
\subsubsection{Version history}\label{version-history}

\begin{longtable}[]{@{}ll@{}}
\toprule\noalign{}
Version & Release Date \\
\midrule\noalign{}
\endhead
\bottomrule\noalign{}
\endlastfoot
0.1.1 & May 6, 2024 \\
\href{https://typst.app/universe/package/light-cv/0.1.0/}{0.1.0} & April
17, 2024 \\
\end{longtable}

Typst GmbH did not create this template and cannot guarantee correct
functionality of this template or compatibility with any version of the
Typst compiler or app.
