\title{typst.app/universe/package/postercise}

\phantomsection\label{banner}
\section{postercise}\label{postercise}

{ 0.1.0 }

Postercise allows users to easily create academic research posters with
different themes using Typst.

\phantomsection\label{readme}
\emph{Postercise} allows users to easily create academic research
posters with different themes using \href{https://typst.app/}{Typst} .

\pandocbounded{\includegraphics[keepaspectratio]{https://img.shields.io/github/license/dangh3014/postercise}}
\pandocbounded{\includegraphics[keepaspectratio]{https://img.shields.io/github/v/release/dangh3014/postercise}}
\pandocbounded{\includegraphics[keepaspectratio]{https://img.shields.io/github/stars/dangh3014/postercise}}

\subsection{Getting started}\label{getting-started}

You can get \textbf{Postercise} from the official package repository by
entering the following.

\begin{Shaded}
\begin{Highlighting}[]
\NormalTok{\#import "@preview/postercise:0.1.0": *}
\end{Highlighting}
\end{Shaded}

Another option is to use \textbf{Postercise} as a local module by
downloading the package files into your project folder.

Next you will need to import a theme, set up the page and font, and call
the \texttt{\ show\ } command.

\begin{Shaded}
\begin{Highlighting}[]
\NormalTok{\#import themes.basic: *}

\NormalTok{\#set page(width: 24in, height: 18in)}
\NormalTok{\#set text(size: 24pt)}

\NormalTok{\#show: theme}
\end{Highlighting}
\end{Shaded}

To add content to the poster, use the \texttt{\ poster-content\ }
command.

\begin{Shaded}
\begin{Highlighting}[]
\NormalTok{\#poster{-}content()[}
\NormalTok{  // Content goes here}
\NormalTok{]}
\end{Highlighting}
\end{Shaded}

There are a few options for types of content that should be added inside
the \texttt{\ poster-content\ } . The body of the poster can be typed as
normal, or two box styles are provided to headings and/or highlight
content in special ways.

\begin{Shaded}
\begin{Highlighting}[]
\NormalTok{\#normal{-}box[]}
\NormalTok{\#focus{-}box[]}
\end{Highlighting}
\end{Shaded}

Basic information like title and authors is placed as options using the
\texttt{\ poster-header\ } script.

\begin{Shaded}
\begin{Highlighting}[]
\NormalTok{\#poster{-}header(}
\NormalTok{  title: [Title],}
\NormalTok{  authors: [Author],}
\NormalTok{)}
\end{Highlighting}
\end{Shaded}

Finally, additional content can be added to the footer with the
\texttt{\ poster-footer\ } script.

\begin{Shaded}
\begin{Highlighting}[]
\NormalTok{\#poster{-}footer[]}
\end{Highlighting}
\end{Shaded}

Again, as a reminder, all of these scripts should be called from inside
of the \texttt{\ poster-content\ } block.

Using these commands, it is easy to produce posters like the following:
\pandocbounded{\includegraphics[keepaspectratio]{https://raw.githubusercontent.com/dangh3014/postercise/main/examples/postercise-examples.png}}

\subsection{More details}\label{more-details}

\subsubsection{\texorpdfstring{\texttt{\ themes\ }}{ themes }}\label{themes}

Currently, 3 themes are available. Use one of these \texttt{\ import\ }
commands to load that theme.

\begin{Shaded}
\begin{Highlighting}[]
\NormalTok{\#import themes.basic: *}
\NormalTok{\#import themes.better: *}
\NormalTok{\#import themes.boxes: *}
\end{Highlighting}
\end{Shaded}

\subsubsection{\texorpdfstring{\texttt{\ show:\ theme.with()\ }}{ show: theme.with() }}\label{show-theme.with}

Theme options allow you to adjust the color scheme, as well as the color
and size of the content in the header. The defaults are shown below.
(The ‘better.typ’ theme defaults to different titletext color and
size.)

\begin{Shaded}
\begin{Highlighting}[]
\NormalTok{\#show: theme.with(}
\NormalTok{  primary{-}color: rgb(28,55,103), // Dark blue}
\NormalTok{  background{-}color: white,}
\NormalTok{  accent{-}color: rgb(243,163,30), // Yellow}
\NormalTok{  titletext{-}color: white,}
\NormalTok{  titletext{-}size: 2em,}
\NormalTok{)}
\end{Highlighting}
\end{Shaded}

\subsubsection{\texorpdfstring{\texttt{\ poster-content(){[}{]}\ }}{ poster-content(){[}{]} }}\label{poster-content}

The only option for the main content is the number of columns. This
defaults to 3 for most themes. For the “better.typ� theme, there is
1 column and content is placed in the leftmost column below
\texttt{\ poster-header\ } .

\begin{Shaded}
\begin{Highlighting}[]
\NormalTok{\#poster{-}content(col: 3)[}
\NormalTok{  // Content goes here}
\NormalTok{]}
\end{Highlighting}
\end{Shaded}

\subsubsection{\texorpdfstring{\texttt{\ normal-box(){[}{]}\ } and
\texttt{\ focus-box(){[}{]}\ }}{ normal-box(){[}{]}  and  focus-box(){[}{]} }}\label{normal-box-and-focus-box}

By default, these boxes use the no fill and the accent-color fill,
respectively. However, they do accept color as an option, and will add a
primary-color stroke around the box if a color is given. For the
“better.typ� theme, use \texttt{\ focus-box\ } to place content in
the center column.

\begin{Shaded}
\begin{Highlighting}[]
\NormalTok{\#normal{-}box(color: none)[}
\NormalTok{  // Content}
\NormalTok{]}

\NormalTok{\#focus{-}box(color: none)[}
\NormalTok{  // Content}
\NormalTok{]}
\end{Highlighting}
\end{Shaded}

\subsubsection{\texorpdfstring{\texttt{\ poster-header()\ }}{ poster-header() }}\label{poster-header}

Available options for the poster header for most themes are shown below.
Note that logos should be explicitly labeled as images. Logos are not
currently displayed in the header in the “better.typ� theme.

\begin{Shaded}
\begin{Highlighting}[]
\NormalTok{\#poster{-}header(}
\NormalTok{  title: [Title],}
\NormalTok{  subtitle: [Subtitle],}
\NormalTok{  author: [Author],}
\NormalTok{  affiliation: [Affiliation],}
\NormalTok{  logo{-}1: image("placeholder.png")}
\NormalTok{  logo{-}2: image("placeholder.png") }
\NormalTok{)}
\end{Highlighting}
\end{Shaded}

\subsubsection{\texorpdfstring{\texttt{\ poster-footer{[}{]}\ }}{ poster-footer{[}{]} }}\label{poster-footer}

This command does not currently have any extra options. The content is
typically placed at the bottom of the poster, but it is placed in the
rightmost column for the “better.typ� theme.

\begin{Shaded}
\begin{Highlighting}[]
\NormalTok{\#poster{-}footer[}
\NormalTok{  // Content}
\NormalTok{]}
\end{Highlighting}
\end{Shaded}

\subsection{Known Issues}\label{known-issues}

\begin{itemize}
\tightlist
\item
  The bibliography does not work properly and must be done manually
\item
  Figure captions do not number correctly and must be done manually
\end{itemize}

\subsection{Planned Features/Themes}\label{planned-featuresthemes}

\begin{itemize}
\tightlist
\item
  Themes that use color gradients and background images
\item
  Add QR code generation
\end{itemize}

\subsubsection{How to add}\label{how-to-add}

Copy this into your project and use the import as
\texttt{\ postercise\ }

\begin{verbatim}
#import "@preview/postercise:0.1.0"
\end{verbatim}

\includesvg[width=0.16667in,height=0.16667in]{/assets/icons/16-copy.svg}

Check the docs for
\href{https://typst.app/docs/reference/scripting/\#packages}{more
information on how to import packages} .

\subsubsection{About}\label{about}

\begin{description}
\tightlist
\item[Author :]
Daniel King
\item[License:]
MIT
\item[Current version:]
0.1.0
\item[Last updated:]
May 27, 2024
\item[First released:]
May 27, 2024
\item[Archive size:]
5.11 kB
\href{https://packages.typst.org/preview/postercise-0.1.0.tar.gz}{\pandocbounded{\includesvg[keepaspectratio]{/assets/icons/16-download.svg}}}
\item[Repository:]
\href{https://github.com/dangh3014/postercise/}{GitHub}
\item[Categor y :]
\begin{itemize}
\tightlist
\item[]
\item
  \pandocbounded{\includesvg[keepaspectratio]{/assets/icons/16-pin.svg}}
  \href{https://typst.app/universe/search/?category=poster}{Poster}
\end{itemize}
\end{description}

\subsubsection{Where to report issues?}\label{where-to-report-issues}

This package is a project of Daniel King . Report issues on
\href{https://github.com/dangh3014/postercise/}{their repository} . You
can also try to ask for help with this package on the
\href{https://forum.typst.app}{Forum} .

Please report this package to the Typst team using the
\href{https://typst.app/contact}{contact form} if you believe it is a
safety hazard or infringes upon your rights.

\phantomsection\label{versions}
\subsubsection{Version history}\label{version-history}

\begin{longtable}[]{@{}ll@{}}
\toprule\noalign{}
Version & Release Date \\
\midrule\noalign{}
\endhead
\bottomrule\noalign{}
\endlastfoot
0.1.0 & May 27, 2024 \\
\end{longtable}

Typst GmbH did not create this package and cannot guarantee correct
functionality of this package or compatibility with any version of the
Typst compiler or app.
