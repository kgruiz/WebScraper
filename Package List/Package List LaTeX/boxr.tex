\title{typst.app/universe/package/boxr}

\phantomsection\label{banner}
\section{boxr}\label{boxr}

{ 0.1.0 }

A modular, and easy to use, package for creating cardboard cutouts in
Typst.

\phantomsection\label{readme}
Boxr is a modular, and easy to use, package for creating cardboard
cutouts in Typst.

\subsection{Usage}\label{usage}

Create a boxr structure in your project with the following code:

\begin{verbatim}
#import "@preview/boxr:0.1.0": *

#render-structure(
  "box",
  width: 100pt,
  height: 100pt,
  depth: 100pt,
  tab-size: 20pt,
  [
    Hello from boxr!
  ]
)
\end{verbatim}

The \texttt{\ render-structure\ } function is the main function for
boxr. It either takes a path to one of the default structures provided
by boxr (e.g.: \texttt{\ "box"\ } ) or an unpacked json file with your
own custom structure (e.g.: \texttt{\ json(my-structure.json)\ } ).
These describe the structure of the cutout.\\
The other named arguments depend on the structure you are rendering. All
unnamed arguments are passed to the structure as content and will be
rendered on each box face (not triangles or tabs).

\subsection{Creating your own
structures}\label{creating-your-own-structures}

Structures are defined in \texttt{\ .json\ } files. An example structure
that just shows a box with a tab on one face is shown below:

\begin{verbatim}
{
  "variables": ["height", "width", "tab-size"],
  "width": "width",
  "height": "height + tab-size",
  "offset-x": "",
  "offset-y": "tab-size",
  "root": {
    "type": "box",
    "id": 0,
    "width": "width",
    "height": "height",
    "children": {
      "top": "tab(tab-size, tab-size)"
    }
  }
}
\end{verbatim}

The \texttt{\ variables\ } key is a list of variable names that can be
passed to the structure. These will be required to be passed to the
\texttt{\ render-structure\ } function.\\
The \texttt{\ width\ } and \texttt{\ height\ } keys are evaluated to
calculate the width and height of the structure.\\
The \texttt{\ offset-x\ } and \texttt{\ offset-y\ } keys are evaluated
to place the structure in the middle of its bounds. It is relative to
the root node. In this case for example, the top tab adds a
\texttt{\ tab-size\ } on top of the box as opposed to the bottom, where
there is no tab. So this \texttt{\ tab-size\ } is added to the
\texttt{\ offset-y\ } .\\
\texttt{\ root\ } denotes the first node in the structure.\\
A node can be of the following types:

\begin{itemize}
\tightlist
\item
  \texttt{\ box\ } :

  \begin{itemize}
  \tightlist
  \item
    The root node has a \texttt{\ width\ } and a \texttt{\ height\ } .
    All following nodes have a \texttt{\ size\ } . Child nodes use
    \texttt{\ size\ } and the parent node’s \texttt{\ width\ } and
    \texttt{\ height\ } to calculate their own width and height.
  \item
    Can have \texttt{\ children\ } nodes.
  \item
    Can have an \texttt{\ id\ } key that is used to place content on the
    face of the box. The id-th unnamed argument is placed on the face.
    Multiple faces can have the same id.
  \item
    Can have a \texttt{\ no-fold\ } key. If this exists, no fold stroke
    will be drawn between this box and its parent.
  \end{itemize}
\item
  \texttt{\ triangle-\textless{}left\textbar{}right\textgreater{}\ } :

  \begin{itemize}
  \tightlist
  \item
    Has a \texttt{\ width\ } and \texttt{\ height\ } .
  \item
    \texttt{\ left\ } and \texttt{\ right\ } denote the direction the
    other right angled line is facing relative to the base.
  \item
    Can have \texttt{\ children\ } nodes.
  \item
    Can have a \texttt{\ no-fold\ } key. If this exists, no fold stroke
    will be drawn between this triangle and its parent.
  \end{itemize}
\item
  \texttt{\ tab\ } :

  \begin{itemize}
  \tightlist
  \item
    Is not a json object, but a string that denotes a tab. The tab is
    placed on the parent node.
  \item
    Has a tab-size and a cutin-size inside the \texttt{\ ()\ } separted
    by a \texttt{\ ,\ } .
  \end{itemize}
\item
  \texttt{\ none\ } :

  \begin{itemize}
  \tightlist
  \item
    Is not a json object, but a string that denotes no node. This is
    useful for deleting a cut-stroke between two nodes.
  \end{itemize}
\end{itemize}

Every string value in the json file (
\texttt{\ width:\ "\_\_",\ height:\ "\_\_",\ ...\ offset-x/y:\ "\_\_"\ }
and the values between the \texttt{\ \textbar{}\ } for tabs) is
evaluated as regular typst code. This means that you can use all named
variables passed to the structure. All inputs are converted to points
and the result of the evaluation will be converted back to a length.

\subsubsection{How to add}\label{how-to-add}

Copy this into your project and use the import as \texttt{\ boxr\ }

\begin{verbatim}
#import "@preview/boxr:0.1.0"
\end{verbatim}

\includesvg[width=0.16667in,height=0.16667in]{/assets/icons/16-copy.svg}

Check the docs for
\href{https://typst.app/docs/reference/scripting/\#packages}{more
information on how to import packages} .

\subsubsection{About}\label{about}

\begin{description}
\tightlist
\item[Author :]
\href{https://github.com/Lypsilonx}{Lypsilonx}
\item[License:]
MIT
\item[Current version:]
0.1.0
\item[Last updated:]
May 23, 2024
\item[First released:]
May 23, 2024
\item[Archive size:]
6.23 kB
\href{https://packages.typst.org/preview/boxr-0.1.0.tar.gz}{\pandocbounded{\includesvg[keepaspectratio]{/assets/icons/16-download.svg}}}
\item[Repository:]
\href{https://github.com/Lypsilonx/boxr}{GitHub}
\item[Discipline :]
\begin{itemize}
\tightlist
\item[]
\item
  \href{https://typst.app/universe/search/?discipline=design}{Design}
\end{itemize}
\item[Categor ies :]
\begin{itemize}
\tightlist
\item[]
\item
  \pandocbounded{\includesvg[keepaspectratio]{/assets/icons/16-chart.svg}}
  \href{https://typst.app/universe/search/?category=visualization}{Visualization}
\item
  \pandocbounded{\includesvg[keepaspectratio]{/assets/icons/16-layout.svg}}
  \href{https://typst.app/universe/search/?category=layout}{Layout}
\end{itemize}
\end{description}

\subsubsection{Where to report issues?}\label{where-to-report-issues}

This package is a project of Lypsilonx . Report issues on
\href{https://github.com/Lypsilonx/boxr}{their repository} . You can
also try to ask for help with this package on the
\href{https://forum.typst.app}{Forum} .

Please report this package to the Typst team using the
\href{https://typst.app/contact}{contact form} if you believe it is a
safety hazard or infringes upon your rights.

\phantomsection\label{versions}
\subsubsection{Version history}\label{version-history}

\begin{longtable}[]{@{}ll@{}}
\toprule\noalign{}
Version & Release Date \\
\midrule\noalign{}
\endhead
\bottomrule\noalign{}
\endlastfoot
0.1.0 & May 23, 2024 \\
\end{longtable}

Typst GmbH did not create this package and cannot guarantee correct
functionality of this package or compatibility with any version of the
Typst compiler or app.
