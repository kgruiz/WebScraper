\title{typst.app/universe/package/great-theorems}

\phantomsection\label{banner}
\section{great-theorems}\label{great-theorems}

{ 0.1.1 }

Straightforward and functional theorem/proof environments.

\phantomsection\label{readme}
This package allows you to make \textbf{theorem/proof/remark/…}
blocks.

Features:

\begin{itemize}
\tightlist
\item
  supports advanced counters through both
  \href{https://typst.app/universe/package/headcount/}{headcount} and
  \href{https://typst.app/universe/package/rich-counters/}{rich-counters}
\item
  easy adjustment of style:

  \begin{itemize}
  \tightlist
  \item
    change prefix
  \item
    change how title is displayed
  \item
    change formatting of body
  \item
    change suffix
  \item
    change numbering style
  \item
    configure \emph{all} parameters of the
    \href{https://typst.app/docs/reference/layout/block/}{\texttt{\ block\ }}
    , including background color, stroke color, rounded corners, inset,
    …
  \end{itemize}
\item
  can adjust style also on individual basis (e.g. to highlight main
  theorem)
\item
  works with labels/references
\item
  sane and smart defaults
\end{itemize}

\subsection{Showcase}\label{showcase}

In the following example we use
\href{https://typst.app/universe/package/rich-counters/}{rich-counters}
to configure section-based counters. You can also use
\href{https://typst.app/universe/package/headcount/}{headcount} .

\begin{Shaded}
\begin{Highlighting}[]
\NormalTok{\#import "@preview/great{-}theorems:0.1.1": *}
\NormalTok{\#import "@preview/rich{-}counters:0.2.1": *}

\NormalTok{\#set heading(numbering: "1.1")}
\NormalTok{\#show: great{-}theorems{-}init}

\NormalTok{\#show link: text.with(fill: blue)}

\NormalTok{\#let mathcounter = rich{-}counter(}
\NormalTok{  identifier: "mathblocks",}
\NormalTok{  inherited\_levels: 1}
\NormalTok{)}

\NormalTok{\#let theorem = mathblock(}
\NormalTok{  blocktitle: "Theorem",}
\NormalTok{  counter: mathcounter,}
\NormalTok{)}

\NormalTok{\#let lemma = mathblock(}
\NormalTok{  blocktitle: "Lemma",}
\NormalTok{  counter: mathcounter,}
\NormalTok{)}

\NormalTok{\#let remark = mathblock(}
\NormalTok{  blocktitle: "Remark",}
\NormalTok{  prefix: [\_Remark.\_],}
\NormalTok{  inset: 5pt,}
\NormalTok{  fill: lime,}
\NormalTok{  radius: 5pt,}
\NormalTok{)}

\NormalTok{\#let proof = proofblock()}

\NormalTok{= Some Heading}

\NormalTok{\#theorem[}
\NormalTok{  This is some theorem.}
\NormalTok{] \textless{}mythm\textgreater{}}

\NormalTok{\#lemma[}
\NormalTok{  This is a lemma. Maybe it\textquotesingle{}s used to prove @mythm.}
\NormalTok{]}

\NormalTok{\#proof[}
\NormalTok{  This is a proof.}
\NormalTok{]}

\NormalTok{= Another Heading}

\NormalTok{\#theorem(title: "some title")[}
\NormalTok{  This is a theorem with a title.}
\NormalTok{] \textless{}thm2\textgreater{}}

\NormalTok{\#proof(of: \textless{}thm2\textgreater{})[}
\NormalTok{  This is a proof of the theorem which has a title.}
\NormalTok{]}

\NormalTok{\#remark[}
\NormalTok{  This is a remark.}
\NormalTok{  The remark box has some custom styling applied.}
\NormalTok{]}
\end{Highlighting}
\end{Shaded}

\pandocbounded{\includegraphics[keepaspectratio]{https://github.com/typst/packages/raw/main/packages/preview/great-theorems/0.1.1/example.png}}

\subsection{Usage}\label{usage}

\subsubsection{\texorpdfstring{\texttt{\ great-theorems-init\ }}{ great-theorems-init }}\label{great-theorems-init}

First, make sure to apply the following inital \texttt{\ show\ } rule to
your document:

\begin{Shaded}
\begin{Highlighting}[]
\NormalTok{\#show: great{-}theorems{-}init}
\end{Highlighting}
\end{Shaded}

This is important to make the blocks have the correct alignment and to
display references correctly.

\subsubsection{\texorpdfstring{\texttt{\ mathblock\ }}{ mathblock }}\label{mathblock}

The main constructor you will use is \texttt{\ mathblock\ } , which
allows you to construct a theorem/proof/remark/… environment in
exactly the way you like it.

Please see the showcase above for on example on how to use it. We now
list and explain all possible arguments.

\begin{itemize}
\item
  \texttt{\ blocktitle\ } (required)

  Usually something like \texttt{\ "Theorem"\ } or \texttt{\ "Lemma"\ }
  . Determines how references are displayed, and also determines the
  default \texttt{\ prefix\ } .
\item
  \texttt{\ counter\ } (default: \texttt{\ none\ } )

  If you want your \texttt{\ mathblock\ } to be counted, pass the
  counter here. Accepts either a Typst-native
  \href{https://typst.app/docs/reference/introspection/counter/}{\texttt{\ counter\ }}
  (which can be made to depend on the section with the
  \href{https://typst.app/universe/package/headcount/}{headcount}
  package) or a \texttt{\ rich-counter\ } from the
  \href{https://typst.app/universe/package/rich-counters/}{rich-counters}
  package. If you want multiple \texttt{\ mathblock\ } environments to
  share the same counter, just pass the same counter to all of them.
\item
  \texttt{\ numbering\ } (default: \texttt{\ "1.1"\ } )

  The numbering style that should be used to display the counters.

  \textbf{Note:} If you use the
  \href{https://typst.app/universe/package/headcount/}{headcount}
  package for your counters, you have to pass the
  \texttt{\ dependent-numbering\ } here.
\item
  \texttt{\ prefix\ } (default: contructed from \texttt{\ blocktitle\ }
  , bold style)

  What should be displayed before the body. If you didn’t pass a
  counter, it should just be a piece of content like
  \texttt{\ {[}*Theorem.*{]}\ } . \emph{If you passed a counter} , it
  should a function/closure, which takes the current counter value as an
  argument and returns the corresponding prefix; for example
  \texttt{\ (count)\ =\textgreater{}\ {[}*Theorem\ \#count.*{]}\ }
\item
  \texttt{\ titlix\ } (default:
  \texttt{\ title\ =\textgreater{}\ {[}(\#title){]}\ } )

  How a title should be displayed. Will be placed after the prefix if a
  title is present. Must be function which takes the title and returns
  the corresponding content that should be displayed.
\item
  \texttt{\ suffix\ } (default: \texttt{\ none\ } )

  A suffix that will be displayed after the body.
\item
  \texttt{\ bodyfmt\ } (default:
  \texttt{\ body\ =\textgreater{}\ body\ } i.e. no special formatting)

  A function that will style/transform the body. For example, if you
  want your theorem contents to be displayed in oblique style, you could
  pass \texttt{\ text.with(style:\ "oblique")\ } .
\item
  arguments for the surrounding
  \href{https://typst.app/docs/reference/layout/block/}{\texttt{\ block\ }}

  The \texttt{\ mathblock\ } , as the name suggests, is surrounded by a
  \href{https://typst.app/docs/reference/layout/block/}{\texttt{\ block\ }}
  , which can be styled to have a background color, stroke color,
  rounded corners, etc. . You can just pass all arguments that you could
  pass to a \texttt{\ block\ } also to \texttt{\ mathblock\ } , and it
  will be “passed through� the surrounding \texttt{\ block\ } . For
  example, you could write
  \texttt{\ \#let\ theorem\ =\ mathblock(...,\ fill:\ yellow,\ inset:\ 5pt)\ }
  .
\end{itemize}

So far we have discussed how you \emph{setup} your environment with
\texttt{\ \#let\ theorem\ =\ mathblock(...)\ } . Now let’s discuss how
to use the resulting \texttt{\ theorem\ } command. Again, please see the
showcase above for some examples on how to use it. We now list and
explain all possible arguments (apart from the body).

\begin{itemize}
\item
  \texttt{\ title\ } (default: \texttt{\ none\ } )

  This allows you to set a title for your theorem/lemma/…, which will
  be displayed according to \texttt{\ titlix\ } .
\item
  all the arguments from \texttt{\ mathblock\ } , except
  \texttt{\ blocktitle\ } and \texttt{\ counter\ }

  You can change all the parameters of your \texttt{\ mathblock\ } also
  on an individual basis, i.e. for each occurrence separately, by just
  passing the respective arguments, including \texttt{\ numbering\ } ,
  \texttt{\ prefix\ } , \texttt{\ titlix\ } , \texttt{\ suffix\ } ,
  \texttt{\ bodyfmt\ } , and arguments for \texttt{\ block\ } . These
  will take precedence over the global configuration.
\end{itemize}

\subsubsection{\texorpdfstring{\texttt{\ proofblock\ }}{ proofblock }}\label{proofblock}

Also a proof environment can be constructed with \texttt{\ mathblock\ }
, for example:

\begin{verbatim}
#let proof = mathblock(
  blocktitle: "Proof",
  prefix: [_Proof._],
  suffix: [#h(1fr) $square$],
)
\end{verbatim}

However, for convenience, we have made another \texttt{\ proofblock\ }
constructor. It works exactly the same as \texttt{\ mathblock\ } , the
only differences being:

\begin{itemize}
\tightlist
\item
  it has different default values for \texttt{\ blocktitle\ } ,
  \texttt{\ prefix\ } , and \texttt{\ suffix\ }
\item
  it has no \texttt{\ counter\ } and \texttt{\ numbering\ } argument
\item
  the \texttt{\ titlix\ } argument is replaced with a
  \texttt{\ prefix\_with\_of\ } argument (also consisting of a
  function), which will be used as a prefix when the constructed
  environment is used with \texttt{\ of\ } parameter
\end{itemize}

The constructed environment will have the following changes compared to
an environment constructed with \texttt{\ mathblock\ }

\begin{itemize}
\item
  the \texttt{\ title\ } argument is replaced with an \texttt{\ of\ }
  argument, which is used to denote to which theorem/lemma/… the proof
  belongs

  This can be either just content, or a label, in which case a reference
  to the label is displayed.
\end{itemize}

\subsection{FAQ}\label{faq}

\begin{itemize}
\item
  \emph{What is the difference to the ctheorems package?}

  You can achieve pretty much the same results with both packages. One
  goal of \texttt{\ great-theorems\ } was to have a cleaner
  implementation, for example by separating the counter functionality
  from the theorem block functionality. \texttt{\ ctheorems\ } also uses
  deprecated Typst functionality that will soon be removed. In the end,
  however, in comes down to personal preference, and
  \texttt{\ ctheorems\ } was certainly a big inspiration for this
  package!
\item
  \emph{How to set up the counters the way I want?}

  Please consult the documentation of
  \href{https://typst.app/universe/package/headcount/}{headcount} and
  \href{https://typst.app/universe/package/rich-counters/}{rich-counters}
  respectively, we support both packages as well as native
  \href{https://typst.app/docs/reference/introspection/counter/}{\texttt{\ counter\ }}
  s.
\item
  \emph{My theorems are all center aligned?!}

  You forgot to put the initial show rule at the start of your document:

\begin{Shaded}
\begin{Highlighting}[]
\NormalTok{\#show: great{-}theorems{-}init}
\end{Highlighting}
\end{Shaded}
\item
  \emph{My theorems break across pages, how do I stop that behavior?}

  You can pass \texttt{\ breakable:\ false\ } to \texttt{\ mathblock\ }
  to construct a non-breakable environment.
\item
  \emph{I have a default style for all my theorems/lemmas/remarks/…,
  and I’m writing boilerplate when I construct theorem/lemma/remark
  environments.}

  You can essentially define your own defaults like this:

\begin{Shaded}
\begin{Highlighting}[]
\NormalTok{\#let my\_mathblock = mathblock.with(fill: yellow, radius: 5pt, inset: 5pt)}

\NormalTok{\#let theorem = my\_mathblock(...)}
\NormalTok{\#let lemma = my\_mathblock(...)}
\NormalTok{\#let remark = my\_mathblock(...)}
\NormalTok{...}
\end{Highlighting}
\end{Shaded}
\item
  \emph{The documentation is too short or unclear… how do I do X?}

  Please just open an
  \href{https://github.com/jbirnick/typst-great-theorems/issues}{issue
  on GitHub} , and I will happily answer your question and extend the
  documentation!
\end{itemize}

\subsubsection{How to add}\label{how-to-add}

Copy this into your project and use the import as
\texttt{\ great-theorems\ }

\begin{verbatim}
#import "@preview/great-theorems:0.1.1"
\end{verbatim}

\includesvg[width=0.16667in,height=0.16667in]{/assets/icons/16-copy.svg}

Check the docs for
\href{https://typst.app/docs/reference/scripting/\#packages}{more
information on how to import packages} .

\subsubsection{About}\label{about}

\begin{description}
\tightlist
\item[Author :]
\href{https://jbirnick.net}{Johann Birnick}
\item[License:]
MIT
\item[Current version:]
0.1.1
\item[Last updated:]
October 22, 2024
\item[First released:]
October 16, 2024
\item[Archive size:]
5.32 kB
\href{https://packages.typst.org/preview/great-theorems-0.1.1.tar.gz}{\pandocbounded{\includesvg[keepaspectratio]{/assets/icons/16-download.svg}}}
\item[Repository:]
\href{https://github.com/jbirnick/typst-great-theorems}{GitHub}
\item[Discipline s :]
\begin{itemize}
\tightlist
\item[]
\item
  \href{https://typst.app/universe/search/?discipline=mathematics}{Mathematics}
\item
  \href{https://typst.app/universe/search/?discipline=computer-science}{Computer
  Science}
\item
  \href{https://typst.app/universe/search/?discipline=physics}{Physics}
\item
  \href{https://typst.app/universe/search/?discipline=engineering}{Engineering}
\item
  \href{https://typst.app/universe/search/?discipline=philosophy}{Philosophy}
\item
  \href{https://typst.app/universe/search/?discipline=education}{Education}
\end{itemize}
\item[Categor ies :]
\begin{itemize}
\tightlist
\item[]
\item
  \pandocbounded{\includesvg[keepaspectratio]{/assets/icons/16-package.svg}}
  \href{https://typst.app/universe/search/?category=components}{Components}
\item
  \pandocbounded{\includesvg[keepaspectratio]{/assets/icons/16-list-unordered.svg}}
  \href{https://typst.app/universe/search/?category=model}{Model}
\end{itemize}
\end{description}

\subsubsection{Where to report issues?}\label{where-to-report-issues}

This package is a project of Johann Birnick . Report issues on
\href{https://github.com/jbirnick/typst-great-theorems}{their
repository} . You can also try to ask for help with this package on the
\href{https://forum.typst.app}{Forum} .

Please report this package to the Typst team using the
\href{https://typst.app/contact}{contact form} if you believe it is a
safety hazard or infringes upon your rights.

\phantomsection\label{versions}
\subsubsection{Version history}\label{version-history}

\begin{longtable}[]{@{}ll@{}}
\toprule\noalign{}
Version & Release Date \\
\midrule\noalign{}
\endhead
\bottomrule\noalign{}
\endlastfoot
0.1.1 & October 22, 2024 \\
\href{https://typst.app/universe/package/great-theorems/0.1.0/}{0.1.0} &
October 16, 2024 \\
\end{longtable}

Typst GmbH did not create this package and cannot guarantee correct
functionality of this package or compatibility with any version of the
Typst compiler or app.
