\title{typst.app/universe/package/ssrn-scribe}

\phantomsection\label{banner}
\phantomsection\label{template-thumbnail}
\pandocbounded{\includegraphics[keepaspectratio]{https://packages.typst.org/preview/thumbnails/ssrn-scribe-0.6.0-small.webp}}

\section{ssrn-scribe}\label{ssrn-scribe}

{ 0.6.0 }

Personal working paper template for general doc and SSRN paper.

\href{/app?template=ssrn-scribe&version=0.6.0}{Create project in app}

\phantomsection\label{readme}
Following the official tutorial, I create a single-column paper template
for general use. You can use it for papers published on SSRN etc.

\subsection{How to use}\label{how-to-use}

\subsubsection{Use as a template
package}\label{use-as-a-template-package}

Typst integrated the template with their official package manager. You
can use it as the other third-party packages.

You only need to enter the following command in the terminal to
initialize the template.

\begin{Shaded}
\begin{Highlighting}[]
\ExtensionTok{typst}\NormalTok{ init @preview/ssrn{-}scribe}
\end{Highlighting}
\end{Shaded}

If will generate a subfolder \texttt{\ ssrn-scribe\ } including the
\texttt{\ main.typ\ } file in the current directory with the latest
version of the template.

\subsubsection{Mannully use}\label{mannully-use}

\begin{enumerate}
\tightlist
\item
  Download the template or clone the repository.
\item
  generate your bibliography file using \texttt{\ .biblatex\ } and store
  the file in the same directory of the template.
\item
  modify the \texttt{\ main.typ\ } file in the subfolder
  \texttt{\ /template\ } and compile it. \textbf{\emph{Note:} You should
  have \texttt{\ paper\_template.typ\ } and \texttt{\ main.typ\ } in the
  same directory.}
\end{enumerate}

In the template, you can modify the following parameters:

\texttt{\ maketitle\ } is a boolean ( \textbf{compulsory} ). If
\texttt{\ maketitle=true\ } , the template will generate a new page for
the title. Otherwise, the title will be shown on the first page.

\begin{itemize}
\tightlist
\item
  \texttt{\ maketitle=true\ } :
\end{itemize}

\begin{longtable}[]{@{}llll@{}}
\toprule\noalign{}
Parameter & Default & Optional & Description \\
\midrule\noalign{}
\endhead
\bottomrule\noalign{}
\endlastfoot
\texttt{\ font\ } & “PT Serif� & Yes & The font of the paper. You
can choose “Times New Roman� or “Palatino� \\
\texttt{\ fontsize\ } & 11pt & Yes & The font size of the paper. You can
choose 10pt or 12pt \\
\texttt{\ title\ } & “Title� & No & The title of the paper \\
\texttt{\ subtitle\ } & none & Yes & The subtitle of the paper, use
“� or {[}{]} \\
\texttt{\ authors\ } & none & No & The authors of the paper \\
\texttt{\ date\ } & none & Yes & The date of the paper \\
\texttt{\ abstract\ } & none & Yes & The abstract of the paper \\
\texttt{\ keywords\ } & none & Yes & The keywords of the paper \\
\texttt{\ JEL\ } & none & Yes & The JEL codes of the paper \\
\texttt{\ acknowledgments\ } & none & Yes & The acknowledgment of the
paper \\
\texttt{\ bibliography\ } & none & Yes & The bibliography of the paper
\texttt{\ bibliography:\ bibliography("bib.bib",\ title:\ "References",\ style:\ "apa")\ } \\
\end{longtable}

\begin{itemize}
\tightlist
\item
  \texttt{\ maketitle=false\ } :
\end{itemize}

\begin{longtable}[]{@{}llll@{}}
\toprule\noalign{}
Parameter & Default & Optional & Description \\
\midrule\noalign{}
\endhead
\bottomrule\noalign{}
\endlastfoot
\texttt{\ font\ } & “PT Serif� & Yes & The font of the paper. You
can choose “Times New Roman� or “Palatino� \\
\texttt{\ fontsize\ } & 11pt & Yes & The font size of the paper. You can
choose 10pt or 12pt \\
\texttt{\ title\ } & “Title� & No & The title of the paper \\
\texttt{\ subtitle\ } & none & Yes & The subtitle of the paper, use
“� or {[}{]} \\
\texttt{\ authors\ } & none & No & The authors of the paper \\
\texttt{\ date\ } & none & Yes & The date of the paper \\
\texttt{\ bibliography\ } & none & Yes & The bibliography of the paper
\texttt{\ bibliography:\ bibliography("bib.bib",\ title:\ "References",\ style:\ "apa")\ } \\
\end{longtable}

\textbf{Note: You need to keep the comma at the end of the first bracket
of the author’s list, even if you have only one author.}

\begin{Shaded}
\begin{Highlighting}[]
\NormalTok{    (}
\NormalTok{    name: "",}
\NormalTok{    affiliation: "", // optional}
\NormalTok{    email: "", // optional}
\NormalTok{    note: "", // optional}
\NormalTok{    ),}
\end{Highlighting}
\end{Shaded}

\begin{Shaded}
\begin{Highlighting}[]
\NormalTok{\#import "@preview/ssrn{-}scribe:0.6.0": *}

\NormalTok{\#show: paper.with(}
\NormalTok{  font: "PT Serif", // "Times New Roman"}
\NormalTok{  fontsize: 12pt, // 12pt}
\NormalTok{  maketitle: true, // whether to add new page for title}
\NormalTok{  title: [\#lorem(5)], // title }
\NormalTok{  subtitle: "A work in progress", // subtitle}
\NormalTok{  authors: (}
\NormalTok{    (}
\NormalTok{      name: "Theresa Tungsten",}
\NormalTok{      affiliation: "Artos Institute",}
\NormalTok{      email: "tung@artos.edu",}
\NormalTok{      note: "123",}
\NormalTok{    ),}
\NormalTok{  ),}
\NormalTok{  date: "July 2023",}
\NormalTok{  abstract: lorem(80), // replace lorem(80) with [ Your abstract here. ]}
\NormalTok{  keywords: [}
\NormalTok{    Imputation,}
\NormalTok{    Multiple Imputation,}
\NormalTok{    Bayesian,],}
\NormalTok{  JEL: [G11, G12],}
\NormalTok{  acknowledgments: "This paper is a work in progress. Please do not cite without permission.", }
\NormalTok{  // bibliography: bibliography("bib.bib", title: "References", style: "apa"),}
\NormalTok{)}
\NormalTok{= Introduction}
\NormalTok{\#lorem(50)}
\end{Highlighting}
\end{Shaded}

\subsection{Preview}\label{preview}

\subsubsection{Example}\label{example}

Here is a screenshot of the template:
\pandocbounded{\includegraphics[keepaspectratio]{https://minioapi.pjx.ac.cn/img1/2024/03/63ce084e2a43bc2e7e31bd79315a0fb5.png}}

\subsubsection{\texorpdfstring{Example-brief with
\texttt{\ maketitle=true\ }}{Example-brief with  maketitle=true }}\label{example-brief-with-maketitletrue}

\pandocbounded{\includegraphics[keepaspectratio]{https://minioapi.pjx.ac.cn/img1/2024/06/8d203bd7f2fbf20b39b33334f0ee4a36.png}}

\subsubsection{\texorpdfstring{Example-brief with
\texttt{\ maketitle=false\ }}{Example-brief with  maketitle=false }}\label{example-brief-with-maketitlefalse}

\pandocbounded{\includegraphics[keepaspectratio]{https://minioapi.pjx.ac.cn/img1/2024/06/83dd5821409031ce0a2c2a15e014cc60.png}}

\href{/app?template=ssrn-scribe&version=0.6.0}{Create project in app}

\subsubsection{How to use}\label{how-to-use-1}

Click the button above to create a new project using this template in
the Typst app.

You can also use the Typst CLI to start a new project on your computer
using this command:

\begin{verbatim}
typst init @preview/ssrn-scribe:0.6.0
\end{verbatim}

\includesvg[width=0.16667in,height=0.16667in]{/assets/icons/16-copy.svg}

\subsubsection{About}\label{about}

\begin{description}
\tightlist
\item[Author :]
jxpeng98
\item[License:]
MIT
\item[Current version:]
0.6.0
\item[Last updated:]
June 11, 2024
\item[First released:]
March 20, 2024
\item[Archive size:]
4.07 kB
\href{https://packages.typst.org/preview/ssrn-scribe-0.6.0.tar.gz}{\pandocbounded{\includesvg[keepaspectratio]{/assets/icons/16-download.svg}}}
\item[Repository:]
\href{https://github.com/jxpeng98/Typst-Paper-Template}{GitHub}
\item[Categor y :]
\begin{itemize}
\tightlist
\item[]
\item
  \pandocbounded{\includesvg[keepaspectratio]{/assets/icons/16-atom.svg}}
  \href{https://typst.app/universe/search/?category=paper}{Paper}
\end{itemize}
\end{description}

\subsubsection{Where to report issues?}\label{where-to-report-issues}

This template is a project of jxpeng98 . Report issues on
\href{https://github.com/jxpeng98/Typst-Paper-Template}{their
repository} . You can also try to ask for help with this template on the
\href{https://forum.typst.app}{Forum} .

Please report this template to the Typst team using the
\href{https://typst.app/contact}{contact form} if you believe it is a
safety hazard or infringes upon your rights.

\phantomsection\label{versions}
\subsubsection{Version history}\label{version-history}

\begin{longtable}[]{@{}ll@{}}
\toprule\noalign{}
Version & Release Date \\
\midrule\noalign{}
\endhead
\bottomrule\noalign{}
\endlastfoot
0.6.0 & June 11, 2024 \\
\href{https://typst.app/universe/package/ssrn-scribe/0.5.0/}{0.5.0} &
April 5, 2024 \\
\href{https://typst.app/universe/package/ssrn-scribe/0.4.9/}{0.4.9} &
March 20, 2024 \\
\end{longtable}

Typst GmbH did not create this template and cannot guarantee correct
functionality of this template or compatibility with any version of the
Typst compiler or app.
