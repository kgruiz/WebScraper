\title{typst.app/universe/package/unofficial-fhict-document-template}

\phantomsection\label{banner}
\phantomsection\label{template-thumbnail}
\pandocbounded{\includegraphics[keepaspectratio]{https://packages.typst.org/preview/thumbnails/unofficial-fhict-document-template-1.1.1-small.webp}}

\section{unofficial-fhict-document-template}\label{unofficial-fhict-document-template}

{ 1.1.1 }

This is a document template for creating professional-looking documents
with Typst, tailored for FHICT (Fontys Hogeschool ICT).

\href{/app?template=unofficial-fhict-document-template&version=1.1.1}{Create
project in app}

\phantomsection\label{readme}
\pandocbounded{\includegraphics[keepaspectratio]{https://img.shields.io/github/stars/TomVer99/FHICT-typst-template?style=flat-square}}
\pandocbounded{\includegraphics[keepaspectratio]{https://img.shields.io/github/v/release/TomVer99/FHICT-typst-template?style=flat-square}}

\pandocbounded{\includegraphics[keepaspectratio]{https://img.shields.io/maintenance/Yes/2024?style=flat-square}}
\pandocbounded{\includegraphics[keepaspectratio]{https://img.shields.io/github/issues-raw/TomVer99/FHICT-typst-template?label=Issues&style=flat-square}}
\pandocbounded{\includegraphics[keepaspectratio]{https://img.shields.io/github/commits-since/TomVer99/FHICT-typst-template/latest?style=flat-square}}

This is a document template for creating professional-looking documents
with Typst, tailored for FHICT (Fontys Hogeschool ICT).

\subsection{Introduction}\label{introduction}

Creating well-structured and visually appealing documents is crucial in
academic and professional settings. This template is designed to help
FHICT students and faculty produce professional looking documents.

\includegraphics[width=0.49\linewidth,height=\textheight,keepaspectratio]{https://github.com/typst/packages/raw/main/packages/preview/unofficial-fhict-document-template/1.1.1/thumbnail.png}
\includegraphics[width=0.49\linewidth,height=\textheight,keepaspectratio]{https://github.com/typst/packages/raw/main/packages/preview/unofficial-fhict-document-template/1.1.1/showcase-r.png}

\subsection{Features}\label{features}

\begin{itemize}
\tightlist
\item
  Consistent formatting for titles, headings, subheadings, paragraphs
  and other elements.
\item
  Clean and professional document layout.
\item
  FHICT Style.
\item
  Configurable document options.
\item
  Helper functions.
\item
  Multiple languages support (nl, en, de, fr, es).
\end{itemize}

\subsection{Requirements}\label{requirements}

\begin{itemize}
\tightlist
\item
  Roboto font installed on your system.
\item
  Typst builder installed on your system (Explained in
  \texttt{\ Getting\ Started\ } ).
\end{itemize}

\subsection{Getting Started}\label{getting-started}

To get started with this Typst document template, follow these steps:

\begin{enumerate}
\tightlist
\item
  \textbf{Check for the roboto font} : Check if you have the roboto font
  installed on your system. If you don’t, you can download it from
  \href{https://fonts.google.com/specimen/Roboto}{Google Fonts} .
\item
  \textbf{Install Typst} : I recommend to use VSCode with
  \href{https://marketplace.visualstudio.com/items?itemName=myriad-dreamin.tinymist}{Tinymist
  Typst Extension} . You will also need a PDF viewer in VSCode if you
  want to view the document live.
\item
  \textbf{Import the template} : Import the template into your own typst
  document.
  \texttt{\ \#import\ "@preview/unofficial-fhict-document-template:1.1.1":\ *\ }
\item
  \textbf{Set the available options} : Set the available options in the
  template file to your liking.
\item
  \textbf{Start writing} : Start writing your document.
\end{enumerate}

\subsection{Helpful Links / Resources}\label{helpful-links-resources}

\begin{itemize}
\tightlist
\item
  The manual contains a list of all available options and helper
  functions. It can be found
  \href{https://github.com/TomVer99/FHICT-typst-template/blob/main/documentation/manual.pdf}{here}
  or attached to the latest release.
\item
  The \href{https://typst.app/docs/}{Typst Documentation} is a great
  resource for learning how to use Typst.
\item
  The bibliography file is written in
  \href{http://www.bibtex.org/Format/}{BibTeX} . You can use
  \href{https://truben.no/latex/bibtex/}{BibTeX Editor} to easily create
  and edit your bibliography.
\item
  You can use sub files to split your document into multiple files. This
  is especially useful for large documents.
\end{itemize}

\subsection{Contributing}\label{contributing}

I welcome contributions to improve and expand this document template. If
you have ideas, suggestions, or encounter issues, please consider
contributing by creating a pull request or issue.

\subsubsection{Adding a new language}\label{adding-a-new-language}

Currently, the template supports the following languages:
\texttt{\ Dutch\ } \texttt{\ (nl)\ } , \texttt{\ English\ }
\texttt{\ (en)\ } , \texttt{\ German\ } \texttt{\ (de)\ } ,
\texttt{\ French\ } \texttt{\ (fr)\ } , and \texttt{\ Spanish\ }
\texttt{\ (es)\ } . If you want to add a new language, you can do so by
following these steps:

\begin{enumerate}
\tightlist
\item
  Add the language to the \texttt{\ language.yml\ } file in the
  \texttt{\ assets\ } folder. Copy the \texttt{\ en\ } section and
  replace the values with the new language.
\item
  Add a flag \texttt{\ XX-flag.svg\ } to the \texttt{\ assets\ } folder.
\item
  Update the README with the new language.
\item
  Create a pull request with the changes.
\end{enumerate}

\subsection{Disclaimer}\label{disclaimer}

This template / repository is not endorsed by, directly affiliated with,
maintained, authorized or sponsored by Fontys Hogeschool ICT. It is
provided as-is, without any warranty or guarantee of any kind. Use at
your own risk.

The author was/is a student at Fontys Hogeschool ICT and created this
template for personal use. It is shared publicly in the hope that it
will be useful to others.

\href{/app?template=unofficial-fhict-document-template&version=1.1.1}{Create
project in app}

\subsubsection{How to use}\label{how-to-use}

Click the button above to create a new project using this template in
the Typst app.

You can also use the Typst CLI to start a new project on your computer
using this command:

\begin{verbatim}
typst init @preview/unofficial-fhict-document-template:1.1.1
\end{verbatim}

\includesvg[width=0.16667in,height=0.16667in]{/assets/icons/16-copy.svg}

\subsubsection{About}\label{about}

\begin{description}
\tightlist
\item[Author :]
TomVer99
\item[License:]
MIT
\item[Current version:]
1.1.1
\item[Last updated:]
November 12, 2024
\item[First released:]
June 3, 2024
\item[Minimum Typst version:]
0.12.0
\item[Archive size:]
227 kB
\href{https://packages.typst.org/preview/unofficial-fhict-document-template-1.1.1.tar.gz}{\pandocbounded{\includesvg[keepaspectratio]{/assets/icons/16-download.svg}}}
\item[Repository:]
\href{https://github.com/TomVer99/FHICT-typst-template}{GitHub}
\item[Categor ies :]
\begin{itemize}
\tightlist
\item[]
\item
  \pandocbounded{\includesvg[keepaspectratio]{/assets/icons/16-speak.svg}}
  \href{https://typst.app/universe/search/?category=report}{Report}
\item
  \pandocbounded{\includesvg[keepaspectratio]{/assets/icons/16-layout.svg}}
  \href{https://typst.app/universe/search/?category=layout}{Layout}
\item
  \pandocbounded{\includesvg[keepaspectratio]{/assets/icons/16-mortarboard.svg}}
  \href{https://typst.app/universe/search/?category=thesis}{Thesis}
\end{itemize}
\end{description}

\subsubsection{Where to report issues?}\label{where-to-report-issues}

This template is a project of TomVer99 . Report issues on
\href{https://github.com/TomVer99/FHICT-typst-template}{their
repository} . You can also try to ask for help with this template on the
\href{https://forum.typst.app}{Forum} .

Please report this template to the Typst team using the
\href{https://typst.app/contact}{contact form} if you believe it is a
safety hazard or infringes upon your rights.

\phantomsection\label{versions}
\subsubsection{Version history}\label{version-history}

\begin{longtable}[]{@{}ll@{}}
\toprule\noalign{}
Version & Release Date \\
\midrule\noalign{}
\endhead
\bottomrule\noalign{}
\endlastfoot
1.1.1 & November 12, 2024 \\
\href{https://typst.app/universe/package/unofficial-fhict-document-template/1.1.0/}{1.1.0}
& November 6, 2024 \\
\href{https://typst.app/universe/package/unofficial-fhict-document-template/1.0.2/}{1.0.2}
& September 17, 2024 \\
\href{https://typst.app/universe/package/unofficial-fhict-document-template/1.0.1/}{1.0.1}
& September 11, 2024 \\
\href{https://typst.app/universe/package/unofficial-fhict-document-template/1.0.0/}{1.0.0}
& August 19, 2024 \\
\href{https://typst.app/universe/package/unofficial-fhict-document-template/0.11.0/}{0.11.0}
& July 22, 2024 \\
\href{https://typst.app/universe/package/unofficial-fhict-document-template/0.10.1/}{0.10.1}
& June 12, 2024 \\
\href{https://typst.app/universe/package/unofficial-fhict-document-template/0.10.0/}{0.10.0}
& June 3, 2024 \\
\end{longtable}

Typst GmbH did not create this template and cannot guarantee correct
functionality of this template or compatibility with any version of the
Typst compiler or app.
