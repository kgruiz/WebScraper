\title{typst.app/universe/package/stv-vub-huisstijl}

\phantomsection\label{banner}
\phantomsection\label{template-thumbnail}
\pandocbounded{\includegraphics[keepaspectratio]{https://packages.typst.org/preview/thumbnails/stv-vub-huisstijl-0.1.0-small.webp}}

\section{stv-vub-huisstijl}\label{stv-vub-huisstijl}

{ 0.1.0 }

An unofficial template to get the look of the Vrije Universiteit Brussel
(VUB) huisstijl in Typst.

\href{/app?template=stv-vub-huisstijl&version=0.1.0}{Create project in
app}

\phantomsection\label{readme}
An unofficial template to get the look of the
\href{https://www.vub.be/}{Vrije Universiteit Brussel (VUB)} huisstijl
in Typst based on \href{https://gitlab.com/rubdos/texlive-vub}{this
LaTeX template}

\subsection{Getting Started}\label{getting-started}

You can choose “Start from template� in the web app, and search for
\texttt{\ vub-huisstijl\ } .

If you are running Typst locally, you can use the following command to
initialize the template:

\begin{Shaded}
\begin{Highlighting}[]
\NormalTok{typst init @preview/stv{-}vub{-}huisstijl:0.1.0}
\end{Highlighting}
\end{Shaded}

\subsubsection{Fonts}\label{fonts}

The package makes use of the “TeX Gyre Adventor� font, with
“Roboto� as a fallback. These should be installed for the title page
to look right. They are available for free, and also come bundled with
texlive.

\subsection{Note}\label{note}

This only provides a template for a thesis title page, not for slides.
That can be added in the future.

\subsection{About the name}\label{about-the-name}

St V ( \href{https://en.wikipedia.org/wiki/Saint_Verhaegen}{Saint
Verhaegen} ) is an important part of the folklore of the VUB and the
ULB.

\href{/app?template=stv-vub-huisstijl&version=0.1.0}{Create project in
app}

\subsubsection{How to use}\label{how-to-use}

Click the button above to create a new project using this template in
the Typst app.

You can also use the Typst CLI to start a new project on your computer
using this command:

\begin{verbatim}
typst init @preview/stv-vub-huisstijl:0.1.0
\end{verbatim}

\includesvg[width=0.16667in,height=0.16667in]{/assets/icons/16-copy.svg}

\subsubsection{About}\label{about}

\begin{description}
\tightlist
\item[Author :]
\href{https://github.com/WannesMalfait}{Wannes Malfait}
\item[License:]
MIT
\item[Current version:]
0.1.0
\item[Last updated:]
October 21, 2024
\item[First released:]
October 21, 2024
\item[Archive size:]
8.02 kB
\href{https://packages.typst.org/preview/stv-vub-huisstijl-0.1.0.tar.gz}{\pandocbounded{\includesvg[keepaspectratio]{/assets/icons/16-download.svg}}}
\item[Repository:]
\href{https://github.com/WannesMalfait/vub-huisstijl-typst/}{GitHub}
\item[Categor y :]
\begin{itemize}
\tightlist
\item[]
\item
  \pandocbounded{\includesvg[keepaspectratio]{/assets/icons/16-mortarboard.svg}}
  \href{https://typst.app/universe/search/?category=thesis}{Thesis}
\end{itemize}
\end{description}

\subsubsection{Where to report issues?}\label{where-to-report-issues}

This template is a project of Wannes Malfait . Report issues on
\href{https://github.com/WannesMalfait/vub-huisstijl-typst/}{their
repository} . You can also try to ask for help with this template on the
\href{https://forum.typst.app}{Forum} .

Please report this template to the Typst team using the
\href{https://typst.app/contact}{contact form} if you believe it is a
safety hazard or infringes upon your rights.

\phantomsection\label{versions}
\subsubsection{Version history}\label{version-history}

\begin{longtable}[]{@{}ll@{}}
\toprule\noalign{}
Version & Release Date \\
\midrule\noalign{}
\endhead
\bottomrule\noalign{}
\endlastfoot
0.1.0 & October 21, 2024 \\
\end{longtable}

Typst GmbH did not create this template and cannot guarantee correct
functionality of this template or compatibility with any version of the
Typst compiler or app.
