\title{typst.app/universe/package/wenyuan-campaign}

\phantomsection\label{banner}
\phantomsection\label{template-thumbnail}
\pandocbounded{\includegraphics[keepaspectratio]{https://packages.typst.org/preview/thumbnails/wenyuan-campaign-0.1.0-small.webp}}

\section{wenyuan-campaign}\label{wenyuan-campaign}

{ 0.1.0 }

Easily write DnD 5e style campaign documents.

\href{/app?template=wenyuan-campaign&version=0.1.0}{Create project in
app}

\phantomsection\label{readme}
A template for writing RPG campaigns imitating the 5e theme. This was
made as a typst version of the \$\textbackslash LaTeX\$ package
\href{https://github.com/rpgtex/DND-5e-LaTeX-Template}{DnD 5e LaTeX
Template} , though it is not functionally nor entirely visually similar.

Packages:

\begin{itemize}
\tightlist
\item
  \texttt{\ droplet:0.3.1\ }
\end{itemize}

Fonts:

\begin{itemize}
\tightlist
\item
  TeX Gyre Bonum
\item
  Scaly Sans
\item
  Scaly Sans Caps
\item
  Royal Initalen
\item
  京è?¯è€?宋ä½`` KingHwa OldSong
\end{itemize}

\emph{\textbf{Please note: in an effort to reduce the file size of the
template, fonts are included in MY repository only, not in the typst
official one.}} You may find the fonts in my
\href{https://github.com/yanwenywan/typst-packages/tree/master/wenyuan-campaign/0.1.0/template/fonts}{github
repository in the fonts folder} , or download them yourself, or heck
provide your own fonts to your liking.

\begin{verbatim}
typst init @preview/wenyuan-campaign:0.1.0
\end{verbatim}

This will copy over all required fonts and comes prefilled with the
standard template so you can see how it works. To use this you need to
either install all the fonts locally or pass the folder into
-\/-font-path when compiling.

To initialise the style, do:

\begin{Shaded}
\begin{Highlighting}[]
\NormalTok{\#import "@preview/wenyuan{-}campaign:0.1.0": *}

\NormalTok{\#show: conf.with() }
\end{Highlighting}
\end{Shaded}

Very easy.

Optionally, you may set all the theme fonts from the configure function
(the defaults are shown):

\begin{Shaded}
\begin{Highlighting}[]
\NormalTok{\#import "@preview/wenyuan{-}campaign:0.1.0": *}

\NormalTok{\#show: conf.with(}
\NormalTok{    fontsize: 10pt,}
\NormalTok{    mainFont: ("TeX Gyre Bonum", "KingHwa\_OldSong"),}
\NormalTok{    titleFont: ("TeX Gyre Bonum", "KingHwa\_OldSong"),}
\NormalTok{    sansFont: ("Scaly Sans Remake", "KingHwa\_OldSong"),}
\NormalTok{    sansSmallcapsFont: ("Scaly Sans Caps", "KingHwa\_OldSong"),}
\NormalTok{    dropcapFont: "Royal Initialen"}
\NormalTok{) }
\end{Highlighting}
\end{Shaded}

You are encouraged to copy the template files and modify them if they
are not up to your liking.

\textbf{set-theme-colour} \texttt{\ (colour:\ color)\ } .\\
Sets a theme colour from the colours package of this module or any other
colour you wantâ€''up to you if it looks bad :)\\
The colours recommended are:

\begin{quote}
phbgreen, phbcyan, phbmauve, phbtan, dmglavender, dmgcoral, dmgslategrey
(-ay), dmglilac
\end{quote}

\begin{center}\rule{0.5\linewidth}{0.5pt}\end{center}

\textbf{make-title}
\texttt{\ (title:\ content,\ subtitle:\ content\ =\ {[}{]},\ author:\ content\ =\ {[}{]},\ date:\ content\ =\ {[}{]},\ anything-before:\ content\ =\ {[}{]},\ anything-after:\ content\ =\ {[}{]})\ }
.\\
Makes a simple title page.

Parameters:

\begin{itemize}
\tightlist
\item
  title: main book title
\item
  subtitle: (optional) subtitle
\item
  author: (optional)
\item
  date: (optional) â€`` just acts as a separate line, can be used for
  anything else
\item
  anything-before: (optional) this is put before the title
\item
  anything-after: (optional) this is put after the date
\end{itemize}

\begin{center}\rule{0.5\linewidth}{0.5pt}\end{center}

\textbf{drop-paragraph}
\texttt{\ (small-caps:\ string\ =\ "",\ body:\ content)\ } .\\
Makes a paragraph with a drop capital.

Parameters:

\begin{itemize}
\tightlist
\item
  small-caps: (optional) any text you wish to be rendered in small caps,
  like how DnD does it
\item
  body: anything else
\end{itemize}

\begin{center}\rule{0.5\linewidth}{0.5pt}\end{center}

\textbf{bump} \texttt{\ ()\ } .\\
Manually does a 1em paragraph space

\begin{center}\rule{0.5\linewidth}{0.5pt}\end{center}

\textbf{namedpar} \texttt{\ (title:\ content,\ content:\ content)\ } .\\
A paragraph with a bold italic name at the start.

Parameters:

\begin{itemize}
\tightlist
\item
  title: the bold italic name; a full stop/period is added immediately
  after for you
\item
  content: everything else
\end{itemize}

\begin{center}\rule{0.5\linewidth}{0.5pt}\end{center}

\textbf{namedpar-block}
\texttt{\ (title:\ content,\ content:\ content)\ } .\\
See \textbf{namedpar} , but this one is in a block environment.

\begin{center}\rule{0.5\linewidth}{0.5pt}\end{center}

\textbf{readaloud} \texttt{\ {[}{]}\ } .\\
A tan-coloured read-aloud box with some decorations.

\begin{center}\rule{0.5\linewidth}{0.5pt}\end{center}

\textbf{comment-box}
\texttt{\ (title:\ content\ =\ {[}{]},\ content:\ content)\ } .\\
A theme-coloured plain comment box.

Parameters:

\begin{itemize}
\tightlist
\item
  title: (optional) a title in bold small caps
\item
  content
\end{itemize}

\begin{center}\rule{0.5\linewidth}{0.5pt}\end{center}

\textbf{fancy-comment-box}
\texttt{\ (title:\ content\ =\ {[}{]},\ content:\ content)\ } .\\
A theme-coloured fancy comment box with decorations.

Parameters:

\begin{itemize}
\tightlist
\item
  title: (optional) a title in bold small caps
\item
  content
\end{itemize}

\begin{center}\rule{0.5\linewidth}{0.5pt}\end{center}

\textbf{dndtable} \texttt{\ (...)\ } . A theme-coloured dnd-style table.
Parameters are identical to table except stroke, fill, and inset are not
included.

\begin{center}\rule{0.5\linewidth}{0.5pt}\end{center}

\textbf{sctitle} \texttt{\ {[}{]}\ } .\\
Makes a small caps header block.

\begin{center}\rule{0.5\linewidth}{0.5pt}\end{center}

\textbf{begin-stat} \texttt{\ {[}{]}\ } .\\
Begins the monster statblock environment.

\begin{center}\rule{0.5\linewidth}{0.5pt}\end{center}

\textbf{begin-item} \texttt{\ {[}{]}\ } .\\
Begins the item environment.

\emph{\textbf{Important.}} Statblocks are provided under the
\texttt{\ stat\ } namespace, and will only work as intended in a
\texttt{\ beginStat\ } block. All statblock functions must be prepended
with \texttt{\ stat\ } .

\subsection{stat functions}\label{stat-functions}

\textbf{dice} \texttt{\ (value:\ str)\ }\\
Parses a dice string (e.g., \texttt{\ 3d6\ } , \texttt{\ 3d6+2\ } , or
\texttt{\ 3d6-1\ } ) and returns a formatted dice value (e.g., “10
(3d6)�). Specifically, the types of strings it accepts are:

\begin{quote}
\texttt{\ \textbackslash{}d+d\textbackslash{}d+({[}+-{]}\textbackslash{}d+)?\ }
(number \texttt{\ d\ } number \texttt{\ +/-\ } number)
\end{quote}

(You need to ensure the string is correct.)

\begin{center}\rule{0.5\linewidth}{0.5pt}\end{center}

\textbf{dice-raw}
\texttt{\ (num-dice:\ int,\ dice-face:\ int,\ modifier:\ int)\ }\\
A helper function for the above, optionally used. It takes all values as
integers and prints the correct formatting.

\begin{center}\rule{0.5\linewidth}{0.5pt}\end{center}

\textbf{statheading} \texttt{\ (title,\ desc\ =\ {[}{]})\ }\\
Takes a title and description, formatting it into a top-level monster
name heading. \texttt{\ desc\ } is the description of the monster, e.g.,
\emph{Medium humanoid, neutral evil} , but it can be anything.

\begin{center}\rule{0.5\linewidth}{0.5pt}\end{center}

\textbf{stroke} \texttt{\ ()\ } .\\
Draws a red stroke with a fading right edge.

\begin{center}\rule{0.5\linewidth}{0.5pt}\end{center}

\textbf{main-stats}
\texttt{\ (ac\ =\ "",\ hp-dice\ =\ "",\ speed\ =\ "30ft",\ hp-etc\ =\ "")\ }\\
Produces \textbf{Armor Class} , \textbf{Hit Points} , and \textbf{Speed}
in one go. All fields are optional. \texttt{\ hp-dice\ } accepts a
\emph{valid dice string only} â€''if you do not want to use dice, leave
it blank and use \texttt{\ hp-etc\ } . No restrictions on other fields.

\begin{center}\rule{0.5\linewidth}{0.5pt}\end{center}

\textbf{ability} \texttt{\ (str,\ dex,\ con,\ int,\ wis,\ cha)\ }\\
Takes the six ability scores (base values) as integers and formats them
into a table with appropriate modifiers.

\begin{center}\rule{0.5\linewidth}{0.5pt}\end{center}

\textbf{challenge} \texttt{\ (cr:\ str)\ }\\
Takes a numeric challenge rating (as a string) and formats it along with
the XP (if the challenge rating is valid). All CRs between 0â€``30 are
valid, including the fractional \texttt{\ 1/8\ } , \texttt{\ 1/4\ } ,
\texttt{\ 1/2\ } (which can also be written in decimal form, e.g.,
\texttt{\ 0.125\ } ).

\begin{center}\rule{0.5\linewidth}{0.5pt}\end{center}

\textbf{skill} \texttt{\ (title,\ contents)\ }\\
Takes a title and description, creating a single skills entry. For
example, \texttt{\ \#skill("Challenge",\ challenge(1))\ } will produce
(in red):

\begin{quote}
\textbf{Challenge} 1 (200 XP)
\end{quote}

(This uses \texttt{\ challenge\ } from above.)

\textbf{Section headers} such as \emph{Actions} or \emph{Reactions} are
done using the second-level header \texttt{\ ==\ }

\textbf{Action names} â€`` the names that go in front of actions /
abilities are done using the third level header \texttt{\ ===\ } (do not
leave a blank line between the header and its body text)

\emph{\textbf{Important.}} Basic item capability is provided under the
\texttt{\ item\ } namespace, and will only work as intended in a
\texttt{\ beginItem\ } block. All item functions must be prepended with
\texttt{\ item\ } .

\subsection{item functions}\label{item-functions}

\textbf{Item Name} is done with the top-level header \texttt{\ =\ }

\textbf{Section headers} are the second level header \texttt{\ ==\ }

\textbf{Abilities and named paragraphs} are the third level header
\texttt{\ ===\ }

\begin{center}\rule{0.5\linewidth}{0.5pt}\end{center}

\textbf{smalltext} \texttt{\ {[}{]}\ } . Half-size text for item
subheadings

\textbf{flavourtext} \texttt{\ {[}{]}\ } . Indented italic flavour text

\pandocbounded{\includegraphics[keepaspectratio]{https://github.com/typst/packages/raw/main/packages/preview/wenyuan-campaign/0.1.0/sample.png}}

\begin{itemize}
\tightlist
\item
  The overall style is based on the
  \href{https://github.com/rpgtex/DND-5e-LaTeX-Template}{Dnd 5e LaTeX
  Template} , which in turn replicate the base DnD aesthetic.
\item
  TeX Gyre Bonum by GUST e-Foundry is used for the body text
\item
  Scaly Sans and Scaly Sans Caps are part of
  \href{https://github.com/jonathonf/solbera-dnd-fonts}{Solbera’s CC
  Alternatives to DnD Fonts} and are used for main body text.
  \emph{\textbf{Note that these fonts are CC-BY-SA i.e. Share-Alike, so
  keep that in mind. This shouldn’t affect homebrew created using
  these fonts (just like how a painting made with a CC-BY-SA art program
  isn’t itself CC-BY-SA) but what do I know I’m not a lawyer.}}
\item
  \href{https://zhuanlan.zhihu.com/p/637491623}{KingHwa\_OldSong}
  (京è?¯è€?宋ä½``) is a traditional Chinese print font used for all
  CJK text (if present, mostly because I need it)
\end{itemize}

\href{/app?template=wenyuan-campaign&version=0.1.0}{Create project in
app}

\subsubsection{How to use}\label{how-to-use}

Click the button above to create a new project using this template in
the Typst app.

You can also use the Typst CLI to start a new project on your computer
using this command:

\begin{verbatim}
typst init @preview/wenyuan-campaign:0.1.0
\end{verbatim}

\includesvg[width=0.16667in,height=0.16667in]{/assets/icons/16-copy.svg}

\subsubsection{About}\label{about}

\begin{description}
\tightlist
\item[Author :]
\href{https://github.com/yanwenywan}{Yan Xin}
\item[License:]
Apache-2.0
\item[Current version:]
0.1.0
\item[Last updated:]
November 28, 2024
\item[First released:]
November 28, 2024
\item[Archive size:]
403 kB
\href{https://packages.typst.org/preview/wenyuan-campaign-0.1.0.tar.gz}{\pandocbounded{\includesvg[keepaspectratio]{/assets/icons/16-download.svg}}}
\item[Repository:]
\href{https://github.com/yanwenywan/typst-packages/tree/master/wenyuan-campaign}{GitHub}
\item[Categor ies :]
\begin{itemize}
\tightlist
\item[]
\item
  \pandocbounded{\includesvg[keepaspectratio]{/assets/icons/16-layout.svg}}
  \href{https://typst.app/universe/search/?category=layout}{Layout}
\item
  \pandocbounded{\includesvg[keepaspectratio]{/assets/icons/16-text.svg}}
  \href{https://typst.app/universe/search/?category=text}{Text}
\item
  \pandocbounded{\includesvg[keepaspectratio]{/assets/icons/16-docs.svg}}
  \href{https://typst.app/universe/search/?category=book}{Book}
\end{itemize}
\end{description}

\subsubsection{Where to report issues?}\label{where-to-report-issues}

This template is a project of Yan Xin . Report issues on
\href{https://github.com/yanwenywan/typst-packages/tree/master/wenyuan-campaign}{their
repository} . You can also try to ask for help with this template on the
\href{https://forum.typst.app}{Forum} .

Please report this template to the Typst team using the
\href{https://typst.app/contact}{contact form} if you believe it is a
safety hazard or infringes upon your rights.

\phantomsection\label{versions}
\subsubsection{Version history}\label{version-history}

\begin{longtable}[]{@{}ll@{}}
\toprule\noalign{}
Version & Release Date \\
\midrule\noalign{}
\endhead
\bottomrule\noalign{}
\endlastfoot
0.1.0 & November 28, 2024 \\
\end{longtable}

Typst GmbH did not create this template and cannot guarantee correct
functionality of this template or compatibility with any version of the
Typst compiler or app.
