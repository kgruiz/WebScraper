\title{typst.app/universe/package/tgm-hit-protocol}

\phantomsection\label{banner}
\phantomsection\label{template-thumbnail}
\pandocbounded{\includegraphics[keepaspectratio]{https://packages.typst.org/preview/thumbnails/tgm-hit-protocol-0.1.0-small.webp}}

\section{tgm-hit-protocol}\label{tgm-hit-protocol}

{ 0.1.0 }

Protocol template for students of the HIT department at TGM Wien

\href{/app?template=tgm-hit-protocol&version=0.1.0}{Create project in
app}

\phantomsection\label{readme}
This is a port of the
\href{https://github.com/TGM-HIT/latex-protocol/}{LaTeX protocol
template} available for students of the information technology
department at the TGM technical secondary school in Vienna.

\subsection{Getting Started}\label{getting-started}

Using the Typst web app, you can create a project by e.g. using this
link: \url{https://typst.app/?template=tgm-hit-protocol&version=latest}
.

To work locally, use the following command:

\begin{Shaded}
\begin{Highlighting}[]
\ExtensionTok{typst}\NormalTok{ init @preview/tgm{-}hit{-}protocol}
\end{Highlighting}
\end{Shaded}

\subsection{Usage}\label{usage}

The template (
\href{https://github.com/typst/packages/raw/main/packages/preview/tgm-hit-protocol/0.1.0/main.pdf}{rendered
PDF} ) contains thesis writing advice (in German) as example content. If
you are looking for the details of this template package’s function,
take a look at the
\href{https://github.com/typst/packages/raw/main/packages/preview/tgm-hit-protocol/0.1.0/docs/manual.pdf}{manual}
.

\href{/app?template=tgm-hit-protocol&version=0.1.0}{Create project in
app}

\subsubsection{How to use}\label{how-to-use}

Click the button above to create a new project using this template in
the Typst app.

You can also use the Typst CLI to start a new project on your computer
using this command:

\begin{verbatim}
typst init @preview/tgm-hit-protocol:0.1.0
\end{verbatim}

\includesvg[width=0.16667in,height=0.16667in]{/assets/icons/16-copy.svg}

\subsubsection{About}\label{about}

\begin{description}
\tightlist
\item[Author s :]
\href{https://github.com/k1W1M4ng0}{Simon Gao} \&
\href{https://github.com/SillyFreak/}{Clemens Koza}
\item[License:]
MIT
\item[Current version:]
0.1.0
\item[Last updated:]
October 10, 2024
\item[First released:]
October 10, 2024
\item[Minimum Typst version:]
0.11.0
\item[Archive size:]
80.4 kB
\href{https://packages.typst.org/preview/tgm-hit-protocol-0.1.0.tar.gz}{\pandocbounded{\includesvg[keepaspectratio]{/assets/icons/16-download.svg}}}
\item[Repository:]
\href{https://github.com/TGM-HIT/typst-protocol}{GitHub}
\item[Discipline :]
\begin{itemize}
\tightlist
\item[]
\item
  \href{https://typst.app/universe/search/?discipline=computer-science}{Computer
  Science}
\end{itemize}
\item[Categor y :]
\begin{itemize}
\tightlist
\item[]
\item
  \pandocbounded{\includesvg[keepaspectratio]{/assets/icons/16-speak.svg}}
  \href{https://typst.app/universe/search/?category=report}{Report}
\end{itemize}
\end{description}

\subsubsection{Where to report issues?}\label{where-to-report-issues}

This template is a project of Simon Gao and Clemens Koza . Report issues
on \href{https://github.com/TGM-HIT/typst-protocol}{their repository} .
You can also try to ask for help with this template on the
\href{https://forum.typst.app}{Forum} .

Please report this template to the Typst team using the
\href{https://typst.app/contact}{contact form} if you believe it is a
safety hazard or infringes upon your rights.

\phantomsection\label{versions}
\subsubsection{Version history}\label{version-history}

\begin{longtable}[]{@{}ll@{}}
\toprule\noalign{}
Version & Release Date \\
\midrule\noalign{}
\endhead
\bottomrule\noalign{}
\endlastfoot
0.1.0 & October 10, 2024 \\
\end{longtable}

Typst GmbH did not create this template and cannot guarantee correct
functionality of this template or compatibility with any version of the
Typst compiler or app.
