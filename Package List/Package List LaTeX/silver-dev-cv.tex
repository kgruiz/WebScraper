\title{typst.app/universe/package/silver-dev-cv}

\phantomsection\label{banner}
\phantomsection\label{template-thumbnail}
\pandocbounded{\includegraphics[keepaspectratio]{https://packages.typst.org/preview/thumbnails/silver-dev-cv-1.0.1-small.webp}}

\section{silver-dev-cv}\label{silver-dev-cv}

{ 1.0.1 }

A CV template by an engineer-recruiter, used by https://silver.dev

\href{/app?template=silver-dev-cv&version=1.0.1}{Create project in app}

\phantomsection\label{readme}
This Typst CV template is a streamlined version of the the Latex
template \href{https://github.com/jxpeng98/Typst-CV-Resume}{Modernpro} .

\subsection{How to start}\label{how-to-start}

\subsubsection{Use Typst CLI}\label{use-typst-cli}

If you use Typst CLI, you can use the following command to create a new
project:

\begin{Shaded}
\begin{Highlighting}[]
\ExtensionTok{typst}\NormalTok{ init silver{-}dev{-}cv}
\end{Highlighting}
\end{Shaded}

It will create a folder named \texttt{\ silver-dev-cv\ } with the
following structure:

\begin{Shaded}
\begin{Highlighting}[]
\NormalTok{silver{-}dev{-}cv}
\NormalTok{└── cv.typ}
\end{Highlighting}
\end{Shaded}

\subsubsection{Typst website}\label{typst-website}

If you want to use the template via \href{https://typst.app/}{Typst} ,
You can \texttt{\ start\ from\ template\ } and search for
\texttt{\ silver-dev-cv\ } .

\subsection{How to use the template}\label{how-to-use-the-template}

\subsubsection{The arguments}\label{the-arguments}

The template has the following arguments:

\begin{longtable}[]{@{}lll@{}}
\toprule\noalign{}
Argument & Description & Default \\
\midrule\noalign{}
\endhead
\bottomrule\noalign{}
\endlastfoot
\texttt{\ font-type\ } & The font type. You can choose any supported
font in your system. & \texttt{\ Times\ New\ Roman\ } \\
\texttt{\ continue-header\ } & Whether to continue the header on the
follwing pages. & \texttt{\ false\ } \\
\texttt{\ name\ } & Your name. & \texttt{\ ""\ } \\
\texttt{\ address\ } & Your address. & \texttt{\ ""\ } \\
\texttt{\ lastupdated\ } & Whether to show the last updated date. &
\texttt{\ true\ } \\
\texttt{\ pagecount\ } & Whether to show the page count. &
\texttt{\ true\ } \\
\texttt{\ date\ } & The date of the CV. & \texttt{\ today\ } \\
\texttt{\ contacts\ } & contact details, e.g phone number, email, etc. &
\texttt{\ (text:\ "",\ link:\ "")\ } \\
\end{longtable}

\subsubsection{Starting the CV}\label{starting-the-cv}

\begin{Shaded}
\begin{Highlighting}[]
\NormalTok{\#import "@preview/silver{-}dev{-}cv:1.0.0": *}

\NormalTok{\#show: cv.with(}
\NormalTok{  font{-}type: "PT Serif",}
\NormalTok{  continue{-}header: "false",}
\NormalTok{  name: "",}
\NormalTok{  address: "",}
\NormalTok{  lastupdated: "true",}
\NormalTok{  pagecount: "true",}
\NormalTok{  date: "2024{-}07{-}03",}
\NormalTok{  contacts: (}
\NormalTok{    (text: "08856", link: ""),}
\NormalTok{    (text: "example.com", link: "https://www.example.com"),}
\NormalTok{    (text: "github.com", link: "https://www.github.com"),}
\NormalTok{    (text: "123@example.com", link: "mailto:123@example.com"),}
\NormalTok{  )}
\NormalTok{)}
\end{Highlighting}
\end{Shaded}

\subsubsection{Content}\label{content}

Once you set up the arguments, you can start to add details to your CV /
Resume.

I preset the following functions for you to create different parts:

\begin{longtable}[]{@{}ll@{}}
\toprule\noalign{}
Function & Description \\
\midrule\noalign{}
\endhead
\bottomrule\noalign{}
\endlastfoot
\texttt{\ \#section("Section\ Name")\ } & Start a new section \\
\texttt{\ \#sectionsep\ } & End the section \\
\texttt{\ \#oneline-title-item(title:\ "",\ content:\ "")\ } & Add a
one-line item ( \textbf{Title:} content) \\
\texttt{\ \#oneline-two(entry1:\ "",\ entry2:\ "")\ } & Add a one-line
item with two entries, aligned left and right \\
\texttt{\ \#descript("descriptions")\ } & Add a description for
self-introduction \\
\texttt{\ \#award(award:\ "",\ date:\ "",\ institution:\ "")\ } & Add an
award ( \textbf{award} , \emph{institution} \emph{date} ) \\
\texttt{\ \#education(institution:\ "",\ major:\ "",\ date:\ "",\ institution:\ "",\ core-modules:\ "")\ }
& Add an education experience \\
\texttt{\ \#job(position:\ "",\ institution:\ "",\ location:\ "",\ date:\ "",\ description:\ {[}{]})\ }
& Add a job experience (description is optional) \\
\texttt{\ \#twoline-item(entry1:\ "",\ entry2:\ "",\ entry3:\ "",\ entry4:\ "")\ }
& Two line items, similar to education and job experiences \\
\end{longtable}

\subsection{License}\label{license}

The template is released under the MIT License. For more information,
please refer to the
\href{https://github.com/jxpeng98/Typst-CV-Resume/blob/main/LICENSE}{LICENSE}
file.

\href{/app?template=silver-dev-cv&version=1.0.1}{Create project in app}

\subsubsection{How to use}\label{how-to-use}

Click the button above to create a new project using this template in
the Typst app.

You can also use the Typst CLI to start a new project on your computer
using this command:

\begin{verbatim}
typst init @preview/silver-dev-cv:1.0.1
\end{verbatim}

\includesvg[width=0.16667in,height=0.16667in]{/assets/icons/16-copy.svg}

\subsubsection{About}\label{about}

\begin{description}
\tightlist
\item[Author s :]
Gabriel Benmergui \& Santiago Barraza
\item[License:]
MIT
\item[Current version:]
1.0.1
\item[Last updated:]
November 26, 2024
\item[First released:]
October 31, 2024
\item[Archive size:]
4.13 kB
\href{https://packages.typst.org/preview/silver-dev-cv-1.0.1.tar.gz}{\pandocbounded{\includesvg[keepaspectratio]{/assets/icons/16-download.svg}}}
\item[Categor y :]
\begin{itemize}
\tightlist
\item[]
\item
  \pandocbounded{\includesvg[keepaspectratio]{/assets/icons/16-user.svg}}
  \href{https://typst.app/universe/search/?category=cv}{CV}
\end{itemize}
\end{description}

\subsubsection{Where to report issues?}\label{where-to-report-issues}

This template is a project of Gabriel Benmergui and Santiago Barraza .
You can also try to ask for help with this template on the
\href{https://forum.typst.app}{Forum} .

Please report this template to the Typst team using the
\href{https://typst.app/contact}{contact form} if you believe it is a
safety hazard or infringes upon your rights.

\phantomsection\label{versions}
\subsubsection{Version history}\label{version-history}

\begin{longtable}[]{@{}ll@{}}
\toprule\noalign{}
Version & Release Date \\
\midrule\noalign{}
\endhead
\bottomrule\noalign{}
\endlastfoot
1.0.1 & November 26, 2024 \\
\href{https://typst.app/universe/package/silver-dev-cv/1.0.0/}{1.0.0} &
October 31, 2024 \\
\end{longtable}

Typst GmbH did not create this template and cannot guarantee correct
functionality of this template or compatibility with any version of the
Typst compiler or app.
