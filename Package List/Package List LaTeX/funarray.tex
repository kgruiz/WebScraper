\title{typst.app/universe/package/funarray}

\phantomsection\label{banner}
\section{funarray}\label{funarray}

{ 0.4.0 }

Package providing convenient functional functions to use on arrays.

\phantomsection\label{readme}
This package provides some convinient functional functions for
\href{https://typst.app/}{typst} to use on arrays.

\subsection{Usage}\label{usage}

To use this package simply
\texttt{\ \#import\ "@preview/funarray:0.3.0"\ } . To import all
functions use \texttt{\ :\ *\ } and for specific ones, use either the
module or as described in the
\href{https://typst.app/docs/reference/scripting\#modules}{typst docs} .

\subsection{Important note}\label{important-note}

Almost all functions are one-liners, which could, instead of being
loaded via a package import, also be just copied directly into your
source files.

\subsection{Dokumentation}\label{dokumentation}

A prettier und easier to read version of the documentation exists in the
example folder, which is done in typst and exported to pdf. Otherwise,
bellow is the markdown version.

\subsection{Functions}\label{functions}

Let us define
\texttt{\ a\ =\ (1,\ "not\ prime",\ 2,\ "prime",\ 3,\ "prime",\ 4,\ "not\ prime",\ 5,\ "prime")\ }

\subsubsection{chunks}\label{chunks}

The chunks function translates the array to an array of array. It groups
the elements to chunks of a given size and collects them in an bigger
array.

\texttt{\ chunks(a,\ 2)\ =\ (\ (1,\ "not\ prime"),\ (2,\ "prime"),\ (3,\ "prime"),\ (4,\ "not\ prime"),\ (5,\ "prime")\ )\ }

\subsubsection{unzip}\label{unzip}

The unzip function is the inverse of the zip method, it transforms an
array of pairs to a pair of vectors. You can also give input an array of
\texttt{\ n\ } -tuples resulting in in \texttt{\ n\ } arrays.

\texttt{\ unzip(b)\ =\ (\ (1,\ 2,\ 3,\ 4,\ 5),\ (\ "not\ prime",\ "prime",\ "prime",\ "not\ prime",\ "prime"\ )\ )\ }

\subsubsection{cycle}\label{cycle}

The cycle function concatenates the array to itself until it has a given
size.

\begin{Shaded}
\begin{Highlighting}[]
\NormalTok{let c = cycle(range(5), 8)}
\NormalTok{c = (0, 1, 2, 3, 4, 0, 1, 2)}
\end{Highlighting}
\end{Shaded}

Note that there is also the functionality to concatenate with
\texttt{\ +\ } and \texttt{\ *\ } in typst.

\subsubsection{windows and
circular-windows}\label{windows-and-circular-windows}

This function provides a running window

\texttt{\ windows(c,\ 5)\ =\ (\ (0,\ 1,\ 2,\ 3,\ 4),\ (1,\ 2,\ 3,\ 4,\ 0),\ (2,\ 3,\ 4,\ 0,\ 1),\ (3,\ 4,\ 0,\ 1,\ 2)\ )\ }

whereas the circular version wraps over.

\texttt{\ circular-windows(c,\ 5)\ =\ (\ (0,\ 1,\ 2,\ 3,\ 4),\ (1,\ 2,\ 3,\ 4,\ 0),\ (2,\ 3,\ 4,\ 0,\ 1),\ (3,\ 4,\ 0,\ 1,\ 2),\ (4,\ 0,\ 1,\ 2,\ 4),\ (0,\ 1,\ 2,\ 4,\ 0),\ (1,\ 2,\ 4,\ 0,\ 1),\ (2,\ 4,\ 0,\ 1,\ 2)\ )\ }

\subsubsection{partition and
partition-map}\label{partition-and-partition-map}

The partition function seperates the array in two according to a
predicate function. The result is an array with all elements, where the
predicate returned true followed by an array with all elements, where
the predicate returned false.

\begin{Shaded}
\begin{Highlighting}[]
\NormalTok{let (primesp, nonprimesp) = partition(b, x =\textgreater{} x.at(1) == "prime")}
\NormalTok{primesp = ((2, "prime"), (3, "prime"), (5, "prime"))}
\NormalTok{nonprimesp = ((1, "not prime"), (4, "not prime"))}
\end{Highlighting}
\end{Shaded}

There is also a partition-map function, which after partition also
applies a second function on both collections.

\begin{Shaded}
\begin{Highlighting}[]
\NormalTok{let (primes, nonprimes) = partition{-}map(b, x =\textgreater{} x.at(1) == "prime", x =\textgreater{} x.at(0))}
\NormalTok{primes = (2, 3, 5)}
\NormalTok{nonprimes = (1, 4)}
\end{Highlighting}
\end{Shaded}

\subsubsection{group-by}\label{group-by}

This functions groups according to a predicate into maximally sized
chunks, where all elements have the same predicate value.

\begin{Shaded}
\begin{Highlighting}[]
\NormalTok{let f = (0,0,1,1,1,0,0,1)}
\NormalTok{let g = group{-}by(f, x =\textgreater{} x == 0)}
\NormalTok{g = ((0, 0), (1, 1, 1), (0, 0), (1,))}
\end{Highlighting}
\end{Shaded}

\subsubsection{flatten}\label{flatten}

Typst has a \texttt{\ flatten\ } method for arrays, however that method
acts recursively. For instance

\texttt{\ (((1,2,3),\ (2,3)),\ ((1,2,3),\ (1,2))).flatten()\ =\ (1,\ 2,\ 3,\ 2,\ 3,\ 1,\ 2,\ 3,\ 1,\ 2)\ }

Normally, one would only have flattened one level. To do this, we can
use the typst array concatenation method +, or by folding, the sum
method for arrays:

\texttt{\ (((1,2,3),\ (2,3)),\ ((1,2,3),\ (1,2))).sum()\ =\ ((1,\ 2,\ 3),\ (2,\ 3),\ (1,\ 2,\ 3),\ (1,\ 2))\ }

To handle further depth, one can use flatten again, so that in our
example:

\texttt{\ (((1,2,3),\ (2,3)),\ ((1,2,3),\ (1,2))).sum().sum()\ =\ (((1,2,3),\ (2,3)),\ ((1,2,3),\ (1,2))).flatten()\ }

\subsubsection{intersperse}\label{intersperse}

This function has been removed in version 0.3, as typst 0.8 provides
such functionality by default.

\subsubsection{take-while and
skip-while}\label{take-while-and-skip-while}

These functions do exactly as they say.

\begin{Shaded}
\begin{Highlighting}[]
\NormalTok{take{-}while(h, x =\textgreater{} x \textless{} 1) = (0, 0, 0.25, 0.5, 0.75)}
\NormalTok{skip{-}while(h, x =\textgreater{} x \textless{} 1) = (1, 1, 1, 0.25, 0.5, 0.75, 0, 0, 0.25, 0.5, 0.75, 1)}
\end{Highlighting}
\end{Shaded}

\subsection{Unsafe Functions}\label{unsafe-functions}

The core functions are defined in \texttt{\ funarray-unsafe.typ\ } .
However, assertions (error checking) are not there and it is generally
not being advised to use these directly. Still, if being cautious, one
can use the imported \texttt{\ funarray-unsafe\ } module in
\texttt{\ funarray(.typ)\ } . All function names are the same.

To do this from the package, do as follows:

\begin{verbatim}
#import @preview/funarray:0.3.0

#funarray.funarray-unsafe.chunks(range(10), 3)
\end{verbatim}

\subsubsection{How to add}\label{how-to-add}

Copy this into your project and use the import as \texttt{\ funarray\ }

\begin{verbatim}
#import "@preview/funarray:0.4.0"
\end{verbatim}

\includesvg[width=0.16667in,height=0.16667in]{/assets/icons/16-copy.svg}

Check the docs for
\href{https://typst.app/docs/reference/scripting/\#packages}{more
information on how to import packages} .

\subsubsection{About}\label{about}

\begin{description}
\tightlist
\item[Author :]
Ludwig Austermann
\item[License:]
MIT
\item[Current version:]
0.4.0
\item[Last updated:]
October 24, 2023
\item[First released:]
August 1, 2023
\item[Minimum Typst version:]
0.8.0
\item[Archive size:]
4.19 kB
\href{https://packages.typst.org/preview/funarray-0.4.0.tar.gz}{\pandocbounded{\includesvg[keepaspectratio]{/assets/icons/16-download.svg}}}
\item[Repository:]
\href{https://github.com/ludwig-austermann/typst-funarray}{GitHub}
\end{description}

\subsubsection{Where to report issues?}\label{where-to-report-issues}

This package is a project of Ludwig Austermann . Report issues on
\href{https://github.com/ludwig-austermann/typst-funarray}{their
repository} . You can also try to ask for help with this package on the
\href{https://forum.typst.app}{Forum} .

Please report this package to the Typst team using the
\href{https://typst.app/contact}{contact form} if you believe it is a
safety hazard or infringes upon your rights.

\phantomsection\label{versions}
\subsubsection{Version history}\label{version-history}

\begin{longtable}[]{@{}ll@{}}
\toprule\noalign{}
Version & Release Date \\
\midrule\noalign{}
\endhead
\bottomrule\noalign{}
\endlastfoot
0.4.0 & October 24, 2023 \\
\href{https://typst.app/universe/package/funarray/0.3.0/}{0.3.0} &
September 25, 2023 \\
\href{https://typst.app/universe/package/funarray/0.2.0/}{0.2.0} &
August 1, 2023 \\
\end{longtable}

Typst GmbH did not create this package and cannot guarantee correct
functionality of this package or compatibility with any version of the
Typst compiler or app.
