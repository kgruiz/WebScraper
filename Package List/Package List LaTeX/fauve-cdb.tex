\title{typst.app/universe/package/fauve-cdb}

\phantomsection\label{banner}
\phantomsection\label{template-thumbnail}
\pandocbounded{\includegraphics[keepaspectratio]{https://packages.typst.org/preview/thumbnails/fauve-cdb-0.1.0-small.webp}}

\section{fauve-cdb}\label{fauve-cdb}

{ 0.1.0 }

The unofficial implementation of the Collège Doctoral de Bretagne
thesis manuscript template.

\href{/app?template=fauve-cdb&version=0.1.0}{Create project in app}

\phantomsection\label{readme}
Typst template for doctoral dissertations of the French
\href{https://www.doctorat-bretagne.fr/}{Collège doctoral de Bretagne
(CdB)} . The original LaTeX template can be found
\href{https://gitlab.com/ed-matisse/latex-template}{here} .

You can use this template in the Typst web app by clicking “Start from
template� on the dashboard and searching for \texttt{\ fauve-cdb\ } .

Alternatively, you can use the CLI to kick this project off using the
command

\begin{verbatim}
typst init @preview/fauve-cdb
\end{verbatim}

Typst will create a new directory with all the files needed to get you
started.

The original LaTeX template allows selecting different themes
corresponding to different schools of the CdB. For now, we only
implemented the \href{https://ed-matisse.doctorat-bretagne.fr/}{MATISSE}
theme.

\begin{quote}
Fauve is an artistic movement of which French painter
\href{https://en.wikipedia.org/wiki/Henri_Matisse}{Henri Matisse} was a
leader.
\end{quote}

\href{/app?template=fauve-cdb&version=0.1.0}{Create project in app}

\subsubsection{How to use}\label{how-to-use}

Click the button above to create a new project using this template in
the Typst app.

You can also use the Typst CLI to start a new project on your computer
using this command:

\begin{verbatim}
typst init @preview/fauve-cdb:0.1.0
\end{verbatim}

\includesvg[width=0.16667in,height=0.16667in]{/assets/icons/16-copy.svg}

\subsubsection{About}\label{about}

\begin{description}
\tightlist
\item[Author s :]
\href{mailto:timothe.albouy@gmail.com}{Timothé Albouy} \&
\href{https://grodino.github.io}{Augustin Godinot}
\item[License:]
MIT-0
\item[Current version:]
0.1.0
\item[Last updated:]
September 25, 2024
\item[First released:]
September 25, 2024
\item[Archive size:]
122 kB
\href{https://packages.typst.org/preview/fauve-cdb-0.1.0.tar.gz}{\pandocbounded{\includesvg[keepaspectratio]{/assets/icons/16-download.svg}}}
\item[Discipline s :]
\begin{itemize}
\tightlist
\item[]
\item
  \href{https://typst.app/universe/search/?discipline=computer-science}{Computer
  Science}
\item
  \href{https://typst.app/universe/search/?discipline=mathematics}{Mathematics}
\end{itemize}
\item[Categor y :]
\begin{itemize}
\tightlist
\item[]
\item
  \pandocbounded{\includesvg[keepaspectratio]{/assets/icons/16-mortarboard.svg}}
  \href{https://typst.app/universe/search/?category=thesis}{Thesis}
\end{itemize}
\end{description}

\subsubsection{Where to report issues?}\label{where-to-report-issues}

This template is a project of Timothé Albouy and Augustin Godinot . You
can also try to ask for help with this template on the
\href{https://forum.typst.app}{Forum} .

Please report this template to the Typst team using the
\href{https://typst.app/contact}{contact form} if you believe it is a
safety hazard or infringes upon your rights.

\phantomsection\label{versions}
\subsubsection{Version history}\label{version-history}

\begin{longtable}[]{@{}ll@{}}
\toprule\noalign{}
Version & Release Date \\
\midrule\noalign{}
\endhead
\bottomrule\noalign{}
\endlastfoot
0.1.0 & September 25, 2024 \\
\end{longtable}

Typst GmbH did not create this template and cannot guarantee correct
functionality of this template or compatibility with any version of the
Typst compiler or app.
