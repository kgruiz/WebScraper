\title{typst.app/universe/package/athena-tu-darmstadt-exercise}

\phantomsection\label{banner}
\phantomsection\label{template-thumbnail}
\pandocbounded{\includegraphics[keepaspectratio]{https://packages.typst.org/preview/thumbnails/athena-tu-darmstadt-exercise-0.1.0-small.webp}}

\section{athena-tu-darmstadt-exercise}\label{athena-tu-darmstadt-exercise}

{ 0.1.0 }

Exercise template for TU Darmstadt (Technische Universität Darmstadt).

\href{/app?template=athena-tu-darmstadt-exercise&version=0.1.0}{Create
project in app}

\phantomsection\label{readme}
These \textbf{unofficial} templates can be used to write in
\href{https://github.com/typst/typst}{Typst} with the corporate design
of \href{https://www.tu-darmstadt.de/}{TU Darmstadt} .

\paragraph{Disclaimer}\label{disclaimer}

Please ask your supervisor if you are allowed to use typst and this
template for your thesis or other documents. Note that this template is
not checked by TU Darmstadt for correctness. Thus, this template does
not guarantee completeness or correctness. Also, note that submission in
TUbama requires PDF/A which typst currently can’t export to (
\url{https://github.com/typst/typst/issues/2942} ). You can use a
converter to convert from the typst output to PDF/A, but check that
there are no losses during the conversion. CMYK color space support may
be required for printing which is also currently not supported by typst
( \url{https://github.com/typst/typst/issues/2942} ), but this is not
relevant when you just submit online.

\subsection{Implemented Templates}\label{implemented-templates}

The templates imitate the style of the corresponding latex templates in
\href{https://github.com/tudace/tuda_latex_templates}{tuda\_latex\_templates}
. Note that there can be visual differences between the original latex
template and the typst template (you may open an issue when you find
one).

For missing features, ideas or other problems you can just open an issue
:wink:. Contributions are also welcome.

\begin{longtable}[]{@{}
  >{\raggedright\arraybackslash}p{(\linewidth - 6\tabcolsep) * \real{0.2500}}
  >{\raggedright\arraybackslash}p{(\linewidth - 6\tabcolsep) * \real{0.2500}}
  >{\raggedright\arraybackslash}p{(\linewidth - 6\tabcolsep) * \real{0.2500}}
  >{\raggedright\arraybackslash}p{(\linewidth - 6\tabcolsep) * \real{0.2500}}@{}}
\toprule\noalign{}
\begin{minipage}[b]{\linewidth}\raggedright
Template
\end{minipage} & \begin{minipage}[b]{\linewidth}\raggedright
Preview
\end{minipage} & \begin{minipage}[b]{\linewidth}\raggedright
Example
\end{minipage} & \begin{minipage}[b]{\linewidth}\raggedright
Scope
\end{minipage} \\
\midrule\noalign{}
\endhead
\bottomrule\noalign{}
\endlastfoot
\href{https://github.com/JeyRunner/tuda-typst-templates/blob/main/templates/tudapub/template/tudapub.typ}{tudapub}
&
\includegraphics[width=\linewidth,height=3.125in,keepaspectratio]{https://raw.githubusercontent.com/JeyRunner/tuda-typst-templates/refs/heads/main/templates/tudapub/preview/tudapub_prev-01.png}
& \begin{minipage}[t]{\linewidth}\raggedright
\href{https://github.com/JeyRunner/tuda-typst-templates/blob/main/example_tudapub.pdf}{example\_tudapub.pdf}\\
\href{https://github.com/JeyRunner/tuda-typst-templates/blob/main/example_tudapub.typ}{example\_tudapub.typ}\strut
\end{minipage} & Master and Bachelor thesis \\
\href{https://github.com/JeyRunner/tuda-typst-templates/blob/main/templates/tudaexercise/template/tudaexercise.typ}{tudaexercise}
&
\includegraphics[width=\linewidth,height=3.125in,keepaspectratio]{https://raw.githubusercontent.com/JeyRunner/tuda-typst-templates/refs/heads/main/templates/tudaexercise/preview/tudaexercise_prev-1.png}
&
\href{https://github.com/JeyRunner/tuda-typst-templates/blob/main/templates_examples/tudaexercise/main.typ}{Example
File} & Exercises \\
\end{longtable}

\subsection{Usage}\label{usage}

Create a new typst project based on this template locally.

\begin{Shaded}
\begin{Highlighting}[]
\CommentTok{\# for tudapub}
\ExtensionTok{typst}\NormalTok{ init @preview/athena{-}tu{-}darmstadt{-}thesis}
\BuiltInTok{cd}\NormalTok{ athena{-}tu{-}darmstadt{-}thesis}

\CommentTok{\# for tudaexercise}
\ExtensionTok{typst}\NormalTok{ init @preview/athena{-}tu{-}darmstadt{-}exercise}
\BuiltInTok{cd}\NormalTok{ athena{-}tu{-}darmstadt{-}exercise}
\end{Highlighting}
\end{Shaded}

Or create a project on the typst web app based on this template.

Or do a manual installation of this template.

For a manual setup create a folder for your writing project and download
this template into the `templates` folder:

\begin{Shaded}
\begin{Highlighting}[]
\FunctionTok{mkdir}\NormalTok{ my\_exercise }\KeywordTok{\&\&} \BuiltInTok{cd}\NormalTok{ my\_exercise}
\FunctionTok{git}\NormalTok{ clone https://github.com/JeyRunner/tuda{-}typst{-}templates}
\end{Highlighting}
\end{Shaded}

\subsubsection{Logo and Font Setup}\label{logo-and-font-setup}

Download the tud logo from
\href{https://download.hrz.tu-darmstadt.de/protected/ULB/tuda_logo.pdf}{download.hrz.tu-darmstadt.de/protected/ULB/tuda\_logo.pdf}
and put it into the \texttt{\ asssets/logos\ } folder. Now execute the
following script in the \texttt{\ asssets/logos\ } folder to convert it
into an svg:

\begin{Shaded}
\begin{Highlighting}[]
\BuiltInTok{cd}\NormalTok{ asssets/logos}
\ExtensionTok{./convert\_logo.sh}
\end{Highlighting}
\end{Shaded}

Note: The here used \texttt{\ pdf2svg\ } command might not be available.
In this case we recommend a online converter like
\href{https://tools.pdf24.org/en/pdf-to-svg}{PDF24 Tools} . There also
is a \href{https://github.com/FussballAndy/typst-img-to-local}{tool} to
install images as local typst packages.

Also download the required fonts \texttt{\ Roboto\ } and
\texttt{\ XCharter\ } :

\begin{Shaded}
\begin{Highlighting}[]
\BuiltInTok{cd}\NormalTok{ asssets/fonts}
\ExtensionTok{./download\_fonts.sh}
\end{Highlighting}
\end{Shaded}

Optionally you can install all fonts in the folders in
\texttt{\ fonts\ } on your system. But you can also use Typst’s
\texttt{\ -\/-font-path\ } option. Or install them in a folder and add
the folder to \texttt{\ TYPST\_FONT\_PATHS\ } for a single font folder.

Note: wget might not be available. In this case either download it or
replace the command with something like
\texttt{\ curl\ \textless{}url\textgreater{}\ -o\ \textless{}filename\textgreater{}\ -L\ }

Create a main.typ file for the manual template installation.

Create a simple `main.typ` in the root folder (`my\_exercise`) of your
new project:

\begin{Shaded}
\begin{Highlighting}[]
\NormalTok{\#import "tuda{-}typst{-}templates/templates/tudaexercise/template/lib.typ": *}

\NormalTok{\#show: tudaexercise.with(}
\NormalTok{  info: (}
\NormalTok{    title: "My Exercise",}
\NormalTok{    auhtor: "Your name",}
\NormalTok{    sheetnumber: 1    }
\NormalTok{  ),}
\NormalTok{  logo: image("tuda{-}typst{-}templates/assets/logos/tuda\_logo.svg")}
\NormalTok{)}

\NormalTok{= My First Task}
\NormalTok{Some Text}
\end{Highlighting}
\end{Shaded}

\subsubsection{Compile you typst file}\label{compile-you-typst-file}

\begin{Shaded}
\begin{Highlighting}[]
\ExtensionTok{typst} \AttributeTok{{-}{-}watch}\NormalTok{ main.typ }\AttributeTok{{-}{-}font{-}path}\NormalTok{ asssets/fonts/}
\end{Highlighting}
\end{Shaded}

This will watch your file and recompile it to a pdf when the file is
saved. For writing, you can use
\href{https://code.visualstudio.com/}{Vscode} with these extensions:
\href{https://marketplace.visualstudio.com/items?itemName=myriad-dreamin.tinymist}{Tinymist
Typst} . Or use the \href{https://typst.app/}{typst web app} (here you
need to upload the logo and the fonts).

Note that we add \texttt{\ -\/-font-path\ } to ensure that the correct
fonts are used. Due to a bug (typst/typst\#2917 typst/typst\#2098) typst
sometimes uses the font \texttt{\ Roboto\ condensed\ } instead of
\texttt{\ Roboto\ } . To be on the safe side, double-check the embedded
fonts in the pdf (there should be no \texttt{\ Roboto\ condensed\ } ).
What also works is to uninstall/deactivate all
\texttt{\ Roboto\ condensed\ } fonts from your system.

\subsection{Todos}\label{todos}

\begin{itemize}
\tightlist
\item
  \href{https://github.com/JeyRunner/tuda-typst-templates/blob/main/templates/tudapub/TODO.md}{todos
  of thesis template}
\end{itemize}

\href{/app?template=athena-tu-darmstadt-exercise&version=0.1.0}{Create
project in app}

\subsubsection{How to use}\label{how-to-use}

Click the button above to create a new project using this template in
the Typst app.

You can also use the Typst CLI to start a new project on your computer
using this command:

\begin{verbatim}
typst init @preview/athena-tu-darmstadt-exercise:0.1.0
\end{verbatim}

\includesvg[width=0.16667in,height=0.16667in]{/assets/icons/16-copy.svg}

\subsubsection{About}\label{about}

\begin{description}
\tightlist
\item[Author s :]
\href{https://github.com/JeyRunner}{JeyRunner} \&
\href{https://github.com/FussballAndy}{FussballAndy}
\item[License:]
MIT
\item[Current version:]
0.1.0
\item[Last updated:]
November 25, 2024
\item[First released:]
November 25, 2024
\item[Minimum Typst version:]
0.12.0
\item[Archive size:]
10.9 kB
\href{https://packages.typst.org/preview/athena-tu-darmstadt-exercise-0.1.0.tar.gz}{\pandocbounded{\includesvg[keepaspectratio]{/assets/icons/16-download.svg}}}
\item[Repository:]
\href{https://github.com/JeyRunner/tuda-typst-templates}{GitHub}
\item[Categor y :]
\begin{itemize}
\tightlist
\item[]
\item
  \pandocbounded{\includesvg[keepaspectratio]{/assets/icons/16-layout.svg}}
  \href{https://typst.app/universe/search/?category=layout}{Layout}
\end{itemize}
\end{description}

\subsubsection{Where to report issues?}\label{where-to-report-issues}

This template is a project of JeyRunner and FussballAndy . Report issues
on \href{https://github.com/JeyRunner/tuda-typst-templates}{their
repository} . You can also try to ask for help with this template on the
\href{https://forum.typst.app}{Forum} .

Please report this template to the Typst team using the
\href{https://typst.app/contact}{contact form} if you believe it is a
safety hazard or infringes upon your rights.

\phantomsection\label{versions}
\subsubsection{Version history}\label{version-history}

\begin{longtable}[]{@{}ll@{}}
\toprule\noalign{}
Version & Release Date \\
\midrule\noalign{}
\endhead
\bottomrule\noalign{}
\endlastfoot
0.1.0 & November 25, 2024 \\
\end{longtable}

Typst GmbH did not create this template and cannot guarantee correct
functionality of this template or compatibility with any version of the
Typst compiler or app.
