\title{typst.app/universe/package/aloecius-aip}

\phantomsection\label{banner}
\phantomsection\label{template-thumbnail}
\pandocbounded{\includegraphics[keepaspectratio]{https://packages.typst.org/preview/thumbnails/aloecius-aip-0.0.1-small.webp}}

\section{aloecius-aip}\label{aloecius-aip}

{ 0.0.1 }

Typst template for reproducing AIP - Journal of Chemical Physics paper
(draft)

\href{/app?template=aloecius-aip&version=0.0.1}{Create project in app}

\phantomsection\label{readme}
This is a typst template for reproducing papers of American Institute of
Physics (AIP) publishing house, mainly draft version of Journal of
Chemical Physics. This is inspired by the overleaf
\$\textbackslash LaTeX\$ template of AIP journals.

\subsection{Usage}\label{usage}

You can use this template with typst web app by simply clicking on
“Start from template� on the dashboard and searching for
\texttt{\ aloecius-aip\ } .

For local usage, you can use the typst CLI by invoking the following
command

\begin{verbatim}
typst init @preview/aloecius-aip
\end{verbatim}

typst will automatically create a new directory with all the necessary
files needed to compile the project.

\subsection{Configuration}\label{configuration}

The preamble or the header of the document should be written in the
following way with your own necessary input variables to recreate the
same formatting as seen in the
\href{https://github.com/typst/packages/raw/main/packages/preview/aloecius-aip/0.0.1/sample.pdf}{\texttt{\ sample.pdf\ }}

\begin{verbatim}
#import "@preview/aloecius-aip:0.0.1": *

#show: article.with(
  title: "Typst Template for Journal of Chemical Physics (Draft)",
  authors: (
    "Author 1": author-meta(
      "GU",
      email: "user1@domain.com",
    ),
    "Author 2": author-meta(
      "GU",
      cofirst: false
    ),
    "Author 3": author-meta(
      "UG"
    )
  ),
  affiliations: (
    "UG": "University of Global Sciences",
    "GU": "Institute for Chemistry, Global University of Sciences"
  ),
  abstract: [
  Here goes the abstract. 
  ],
  bib: bibliography("./reference.bib")
)
\end{verbatim}

\subsection{Important Variables}\label{important-variables}

\begin{itemize}
\tightlist
\item
  \texttt{\ title\ } : Title of the paper
\item
  \texttt{\ authors\ } : A dictionary connecting the key as name of the
  author(s) and the value to be the affiliation of them including
  university, email, mail address, authorship and an alias, an example
  usage is shown below
\end{itemize}

\begin{verbatim}
Example:
(
  "Author Name": (
    "affiliation": "affiliation-label",
    "email": "author.name@example.com", // Optional
    "address": "Mail address",  // Optional
    "name": "Alias Name", // Optional
    "cofirst": false // Optional, identify whether this author is the co-first author
    )
)
\end{verbatim}

\begin{itemize}
\tightlist
\item
  \texttt{\ affiliations\ } : Dictionary of affiliations where keys are
  affiliations labels and values are affiliations addresses, and example
  usage is as follows
\end{itemize}

\begin{verbatim}
Example:
 (
    "affiliation-label": "Institution Name, University Name, Road, Post Code, Country"
 )
\end{verbatim}

\begin{itemize}
\tightlist
\item
  \texttt{\ abstract\ } : Abstract of the paper
\item
  \texttt{\ bib\ } : passing the bibliography file wrapped into the
  typst \texttt{\ bibliography()\ } function, both
  \texttt{\ Hayagriva\ } and \texttt{\ .bib\ } format is supported.
\end{itemize}

\href{/app?template=aloecius-aip&version=0.0.1}{Create project in app}

\subsubsection{How to use}\label{how-to-use}

Click the button above to create a new project using this template in
the Typst app.

You can also use the Typst CLI to start a new project on your computer
using this command:

\begin{verbatim}
typst init @preview/aloecius-aip:0.0.1
\end{verbatim}

\includesvg[width=0.16667in,height=0.16667in]{/assets/icons/16-copy.svg}

\subsubsection{About}\label{about}

\begin{description}
\tightlist
\item[Author :]
Raunak Farhaz
\item[License:]
MIT
\item[Current version:]
0.0.1
\item[Last updated:]
July 3, 2024
\item[First released:]
July 3, 2024
\item[Minimum Typst version:]
0.11.1
\item[Archive size:]
13.7 kB
\href{https://packages.typst.org/preview/aloecius-aip-0.0.1.tar.gz}{\pandocbounded{\includesvg[keepaspectratio]{/assets/icons/16-download.svg}}}
\item[Repository:]
\href{https://github.com/Raunak12775/aloecius-aip}{GitHub}
\item[Discipline s :]
\begin{itemize}
\tightlist
\item[]
\item
  \href{https://typst.app/universe/search/?discipline=chemistry}{Chemistry}
\item
  \href{https://typst.app/universe/search/?discipline=physics}{Physics}
\item
  \href{https://typst.app/universe/search/?discipline=mathematics}{Mathematics}
\end{itemize}
\item[Categor y :]
\begin{itemize}
\tightlist
\item[]
\item
  \pandocbounded{\includesvg[keepaspectratio]{/assets/icons/16-atom.svg}}
  \href{https://typst.app/universe/search/?category=paper}{Paper}
\end{itemize}
\end{description}

\subsubsection{Where to report issues?}\label{where-to-report-issues}

This template is a project of Raunak Farhaz . Report issues on
\href{https://github.com/Raunak12775/aloecius-aip}{their repository} .
You can also try to ask for help with this template on the
\href{https://forum.typst.app}{Forum} .

Please report this template to the Typst team using the
\href{https://typst.app/contact}{contact form} if you believe it is a
safety hazard or infringes upon your rights.

\phantomsection\label{versions}
\subsubsection{Version history}\label{version-history}

\begin{longtable}[]{@{}ll@{}}
\toprule\noalign{}
Version & Release Date \\
\midrule\noalign{}
\endhead
\bottomrule\noalign{}
\endlastfoot
0.0.1 & July 3, 2024 \\
\end{longtable}

Typst GmbH did not create this template and cannot guarantee correct
functionality of this template or compatibility with any version of the
Typst compiler or app.
