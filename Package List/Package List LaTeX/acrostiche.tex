\title{typst.app/universe/package/acrostiche}

\phantomsection\label{banner}
\section{acrostiche}\label{acrostiche}

{ 0.4.1 }

Manage acronyms and their definitions in Typst.

\phantomsection\label{readme}
Manages acronyms so you don’t have to.

\subsection{Quick Start}\label{quick-start}

\begin{verbatim}
#import "@preview/acrostiche:0.4.0": *

#init-acronyms((
  "WTP": ("Wonderful Typst Package","Wonderful Typst Packages"),
))

Acrostiche is a #acr("WTP")! This #acr("WTP") enables easy acronyms manipulation.

Its main features are auto-expansion of the first occurence, global or selective expansion reset #reset-all-acronyms(), implicit or manual plural form support (there may be multiple #acrpl("WTP")), and customizable index printing. Have Fun!
\end{verbatim}

\subsection{Usage}\label{usage}

The main goal of Acrostiche is to keep track of which acronyms to
define.

\subsubsection{Define acronyms}\label{define-acronyms}

All acronyms used with Acrostiche must be defined in a dictionary passed
to the \texttt{\ init-acronyms\ } function. There are two possible forms
for the definition, depending on the required features.

\paragraph{Simple Definitions}\label{simple-definitions}

For a quick and easy definion, you can use the acronym to display as the
key and an array of one or more strings as the singular and plural
versions of the expanded meaning of the acronym.

\begin{verbatim}
#init-acronyms((
  "SDA": ("Simply Defined Acronym","Simply Defined Acronyms"),
  "ASDA": ("Another Simply Defined Acronym","Another Simply Defined Acronyms"),
))
\end{verbatim}

If there is only a singular version of the definition, the array
contains only one value. If there are both singular and plural versions,
define the definition as an array where the first item is the singular
definition and the second item is the plural.

\paragraph{Advanced Definitions}\label{advanced-definitions}

If you find yourself needing more flexibility when defining the
acronyms, you can pass a dictionary for each acronym. The expected keys
are: \texttt{\ short\ } the singular short form to display,
\texttt{\ short-pl\ } the plural short form, \texttt{\ long\ } singular
long (expanded) form to display, and \texttt{\ long-pl\ } the plural
long form. The main benefit of this definition is to use a separate key
for calling the acronym, useful when acronyms are long and tedious to
write.

\begin{verbatim}
#init-acronyms((
  "la": (
short: "LATATW",
long: "Long And Tedious Acronym To Write",
short-pl: "LATAsTW",
long-pl: "Long And Tedious Acronyms To Write"),
))
\end{verbatim}

Any other keys than these will be discarded.

\subsubsection{Call Acrostiche
functions}\label{call-acrostiche-functions}

Once the acronyms are defined, you can use them in the text with the
\texttt{\ \#acr(...)\ } function. The argument is the acronym as a
string (for example, “BIOS�). On the first call of the function, it
prints the acronym with its definition (for example, “Basic
Input/Output System (BIOS)�). On the next calls, it prints only the
acronym.

To get the plural version of the acronym, you can use the
\texttt{\ \#acrpl(...)\ } function that adds an ‘s’ after the
acronym. If a plural version of the definition is provided, it will be
used if the first use of the acronym is plural. Otherwise, the singular
version is used, and a trailing ‘s’ is added.

To intentionally print the full version of the acronym (definition +
acronym, as for the first instance), without affecting the state, you
can use the \texttt{\ \#acrfull(...)\ } function. For the plural
version, use the \texttt{\ \#acrfullpl(...)\ } function. Both functions
have shortcuts with \texttt{\ \#acrf(...)\ } and
\texttt{\ \#acrfpl(...)\ } .

At any point in the document, you can reset acronyms with the functions
\texttt{\ \#reset-acronym(...)\ } (for a single acronym) or
\texttt{\ reset-all-acronyms()\ } (to reset all acronyms). After a
reset, the next use of the acronym is expanded. Both functions have
shortcuts with \texttt{\ \#racr(...)\ } and \texttt{\ \#raacr(...)\ } .

You can also print an index of all acronyms used in the document with
the \texttt{\ \#print-index()\ } function. The index is printed as a
section for which you can choose the heading level, the numbering, and
the outline parameters (with respectively the \texttt{\ level:\ int\ } ,
\texttt{\ numbering:\ none\ \textbar{}\ string\ \textbar{}\ function\ }
, and \texttt{\ outlined:\ bool\ } parameters). You can also choose
their order with the \texttt{\ sorted:\ string\ } parameter that accepts
either an empty string (print in the order they are defined), “up�
(print in ascending alphabetical order), or “down� (print in
descending alphabetical order). By default, the index contains all the
acronyms you defined. You can choose to only display acronyms that are
actually used in the document by passing \texttt{\ used-only:\ true\ }
to the function. Warning, the detection of used acronym uses the states
at the end of the document. Thus, if you reset an acronym and do not use
it again until the end, it will not appear in the index. You can use the
\texttt{\ title:\ string\ } parameter to change the name of the heading
for the index section. The default value is “Acronyms Index�.
Passing an empty string for \texttt{\ title\ } results in the index
having no heading (i.e., no section for the index). You can customize
the string displayed after the acronym in the list with the
\texttt{\ delimiter:\ ":"\ } parameter. To adjust the spacing between
the acronyms adjust the
\texttt{\ row-gutter:\ auto\ \textbar{}\ int\ \textbar{}\ relative\ \textbar{}\ fraction\ \textbar{}\ array\ }
parameter, the default is \texttt{\ 2pt\ } .

Finally, you can call the \texttt{\ \#display-def(...)\ } function to
display the definition of an acronym. Set the \texttt{\ plural\ }
parameter to true to get the plural version.

\subsubsection{Functions Summary:}\label{functions-summary}

\begin{longtable}[]{@{}ll@{}}
\toprule\noalign{}
\textbf{Function} & \textbf{Description} \\
\midrule\noalign{}
\endhead
\bottomrule\noalign{}
\endlastfoot
\textbf{\#init-acronyms(…)} & Initializes the acronyms by defining
them in a dictionary where the keys are acronyms and the values are
definitions. \\
\textbf{\#acr(…)} & Prints the acronym with its definition on the
first call, then just the acronym in subsequent calls. \\
\textbf{\#acrpl(…)} & Prints the plural version of the acronym. Uses
plural definition if available, otherwise adds an ‘s’ to the
acronym. \\
\textbf{\#acrfull(…)} & Displays the full (long) version of the
acronym without affecting the state or tracking its usage. \\
\textbf{\#acrfullpl(…)} & Displays the full plural version of the
acronym without affecting the state or tracking its usage. \\
\textbf{\#reset-acronym(…)} & Resets a single acronym so the next
usage will include its definition again. \\
\textbf{\#reset-all-acronyms()} & Resets all acronyms so the next usage
will include their definitions again. \\
\textbf{\#print-index(…)} & Prints an index of all acronyms used, with
customizable heading level, order, and display parameters. \\
\textbf{\#display-def(…)} & Displays the definition of an acronym. Use
\texttt{\ plural:\ true\ } to display the plural version of the
definition. \\
\textbf{racr, raacr, acrf, acrfpl} & Shortcuts names for respectively
\texttt{\ reset-acronym\ } , \texttt{\ reset-all-acronyms\ } ,
\texttt{\ acrfull\ } , and \texttt{\ acrfullpl\ } . \\
\end{longtable}

\subsection{Advanced Definitions}\label{advanced-definitions-1}

This is a bit of a hacky feature coming from pure serendipity. There is
no enforcement of the type of the definitions. Most users would
naturally use strings as definitions, but any other content is
acceptable. For example, you set your definition to a content block with
rainbow-fille text, or even an image. The rainbow text is kinda cool
because the gradient depend on the position in the page so depending on
the position of first use the acronym will have a pseudo-random color.

If you use anything else than string for the definition, do not forget
the trailing comma to force the definition to be an array (an array of a
single element is not an array in Typst at the time of writing this). I
cannot guarantee that arbitrary content will remain available in future
versions but I will do my best to keep it as it is kinda cool. If you
find cool uses, please reach out to show me!

\begin{quote}
Yes it is posible to build nest/recursive acronyms definitions. If you
point to another acronym, it all works fine. If you point to the same
acronym, you obviously create a recursive situation, and it fails. It
will not converge, and the compiler will warn you and will panic. Be
nice to the compiler, don\textquotesingle t throw recursive traps.
\end{quote}

Here is a minimal working example of funky acronyms:

\begin{verbatim}
#import "@preview/acrostiche:0.4.0": *                                                           
#init-acronyms((
  "RFA": ([#text(fill: gradient.linear(..color.map.rainbow))[Rainbow Filled Acronym]],),                                                             
  "NA": ([Nested #acr("RFA") Acronym],)
))
#acr("NA")
\end{verbatim}

\subsection{Possible Errors:}\label{possible-errors}

\begin{itemize}
\tightlist
\item
  If an acronym is not defined, an error will tell you which one is
  causing the error. Simply add it to the dictionary or check the
  spelling.
\item
  \texttt{\ display-def\ } leverages the state \texttt{\ display\ }
  function and only works if the return value is actually printed in the
  document. For more information on states, see the Typst documentation
  on states.
\item
  Acrostiche uses a state named \texttt{\ acronyms\ } to keep track of
  the definitions and usage. If you redefined this state or use it
  manually in your document, unexpacted behaviour might happen.
\end{itemize}

Thank you to the contributors: \textbf{caemor} , \textbf{AurelWeinhold}
, \textbf{daniel-eder} , \textbf{iostapyshyn} .

If you notice any bug or want to contribute a new feature, please open
an issue or a merge request on the fork
\href{https://github.com/Grisely/packages}{Grisely/packages}

\subsubsection{How to add}\label{how-to-add}

Copy this into your project and use the import as
\texttt{\ acrostiche\ }

\begin{verbatim}
#import "@preview/acrostiche:0.4.1"
\end{verbatim}

\includesvg[width=0.16667in,height=0.16667in]{/assets/icons/16-copy.svg}

Check the docs for
\href{https://typst.app/docs/reference/scripting/\#packages}{more
information on how to import packages} .

\subsubsection{About}\label{about}

\begin{description}
\tightlist
\item[Author :]
Grizzly
\item[License:]
MIT
\item[Current version:]
0.4.1
\item[Last updated:]
November 21, 2024
\item[First released:]
July 6, 2023
\item[Archive size:]
6.52 kB
\href{https://packages.typst.org/preview/acrostiche-0.4.1.tar.gz}{\pandocbounded{\includesvg[keepaspectratio]{/assets/icons/16-download.svg}}}
\item[Repository:]
\href{https://github.com/Grisely/packages}{GitHub}
\item[Categor ies :]
\begin{itemize}
\tightlist
\item[]
\item
  \pandocbounded{\includesvg[keepaspectratio]{/assets/icons/16-hammer.svg}}
  \href{https://typst.app/universe/search/?category=utility}{Utility}
\item
  \pandocbounded{\includesvg[keepaspectratio]{/assets/icons/16-list-unordered.svg}}
  \href{https://typst.app/universe/search/?category=model}{Model}
\end{itemize}
\end{description}

\subsubsection{Where to report issues?}\label{where-to-report-issues}

This package is a project of Grizzly . Report issues on
\href{https://github.com/Grisely/packages}{their repository} . You can
also try to ask for help with this package on the
\href{https://forum.typst.app}{Forum} .

Please report this package to the Typst team using the
\href{https://typst.app/contact}{contact form} if you believe it is a
safety hazard or infringes upon your rights.

\phantomsection\label{versions}
\subsubsection{Version history}\label{version-history}

\begin{longtable}[]{@{}ll@{}}
\toprule\noalign{}
Version & Release Date \\
\midrule\noalign{}
\endhead
\bottomrule\noalign{}
\endlastfoot
0.4.1 & November 21, 2024 \\
\href{https://typst.app/universe/package/acrostiche/0.4.0/}{0.4.0} &
October 31, 2024 \\
\href{https://typst.app/universe/package/acrostiche/0.3.5/}{0.3.5} &
October 28, 2024 \\
\href{https://typst.app/universe/package/acrostiche/0.3.4/}{0.3.4} &
October 22, 2024 \\
\href{https://typst.app/universe/package/acrostiche/0.3.3/}{0.3.3} &
September 24, 2024 \\
\href{https://typst.app/universe/package/acrostiche/0.3.2/}{0.3.2} &
July 10, 2024 \\
\href{https://typst.app/universe/package/acrostiche/0.3.1/}{0.3.1} &
January 6, 2024 \\
\href{https://typst.app/universe/package/acrostiche/0.3.0/}{0.3.0} &
August 22, 2023 \\
\href{https://typst.app/universe/package/acrostiche/0.2.0/}{0.2.0} &
July 8, 2023 \\
\href{https://typst.app/universe/package/acrostiche/0.1.0/}{0.1.0} &
July 6, 2023 \\
\end{longtable}

Typst GmbH did not create this package and cannot guarantee correct
functionality of this package or compatibility with any version of the
Typst compiler or app.
