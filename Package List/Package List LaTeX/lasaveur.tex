\title{typst.app/universe/package/lasaveur}

\phantomsection\label{banner}
\section{lasaveur}\label{lasaveur}

{ 0.1.3 }

Porting vim-latex\textquotesingle s math shorthands to Typst. An
accommendating vim syntax file is provided in the repo.

\phantomsection\label{readme}
This is a Typst package for speedy mathematical input, inspired by
\href{https://github.com/vim-latex/vim-latex}{vim-latex} . This project
is named after my Vim plugin
\href{https://github.com/yangwenbo99/vimtex-lasaveur}{vimtex-lasaveur} ,
which ports the operations in vim-latex to
\href{https://github.com/lervag/vimtex}{vimtex} .

\subsection{Usages in Typst}\label{usages-in-typst}

Either use the file released in “Releases� or import using the
following command:

\begin{Shaded}
\begin{Highlighting}[]
\NormalTok{\#import "@preview/lasaveur:0.1.3": *}
\end{Highlighting}
\end{Shaded}

This script generates a Typst library that defines shorthand commands
for various mathematical symbols and functions. Here’s an overview of
what it provides and how a user can use it:

\begin{enumerate}
\tightlist
\item
  Mathematical Functions:

  \begin{itemize}
  \tightlist
  \item
    Usage: \texttt{\ f\textless{}key\textgreater{}(argument)\ }
  \item
    Examples: \texttt{\ fh(x)\ } for hat, \texttt{\ ft(x)\ } for tilde,
    \texttt{\ f2(x)\ } for square root
  \end{itemize}
\item
  Font Styles:

  \begin{itemize}
  \tightlist
  \item
    Usage: \texttt{\ f\textless{}key\textgreater{}(argument)\ }
  \item
    Examples: \texttt{\ fb(x)\ } for bold, \texttt{\ fbb(x)\ } for
    blackboard bold, \texttt{\ fca(x)\ } for calligraphic
  \end{itemize}
\item
  Greek Letters:

  \begin{itemize}
  \tightlist
  \item
    Usage: \texttt{\ k\textless{}key\textgreater{}\ }
  \item
    Examples: \texttt{\ ka\ } for α (alpha), \texttt{\ kb\ } for β
    (beta), \texttt{\ kG\ } for Î`` (capital Gamma)
  \end{itemize}
\item
  Common Mathematical Symbols:

  \begin{itemize}
  \tightlist
  \item
    Usage: \texttt{\ g\textless{}key\textgreater{}\ }
  \item
    Examples: \texttt{\ g8\ } for ∞ (infinity), \texttt{\ gU\ } for
    ∪ (union), \texttt{\ gI\ } for ∩ (intersection)
  \end{itemize}
\item
  LaTeX-compatible Symbols:

  \begin{itemize}
  \tightlist
  \item
    Usage: Direct LaTeX command names
  \item
    Examples: \texttt{\ partial\ } for ∂, \texttt{\ infty\ } for ∞,
    \texttt{\ cdot\ } for â‹
  \end{itemize}
\item
  Arrows:

  \begin{itemize}
  \tightlist
  \item
    Usage: \texttt{\ ar.\textless{}key\textgreater{}\ }
  \item
    Examples: \texttt{\ ar.l\ } for â†?, \texttt{\ ar.r\ } for â†',
    \texttt{\ ar.lr\ } for â†''
  \end{itemize}
\end{enumerate}

Users can employ these shorthands in their Typst documents to quickly
input mathematical symbols and functions. The exact prefix for each
category (like \texttt{\ f\ } for functions or \texttt{\ k\ } for Greek
letters) can be customized using command-line arguments when running the
script.

For instance, in a Typst document, after importing the generated
library, a user could write:

\begin{Shaded}
\begin{Highlighting}[]
\NormalTok{$fh(x) + ka + g8 + ar.r$}
\end{Highlighting}
\end{Shaded}

This would produce: xÌ‚ + α + ∞ + â†'

The script provides a wide range of symbols covering most common
mathematical notations, making it easier and faster to type complex
mathematical expressions in Typst â€`` especially for users migrating
from vim-latex.

\subsection{Accompanying Vim Syntax
File}\label{accompanying-vim-syntax-file}

The syntax file provides more advanced and correct concealing for both
Typst’s built-in math syntax and the lasaveur shorthands. Download the
syntax file from the “Releases� section and place it in your
\texttt{\ \textasciitilde{}/.vim/after/syntax/\ } directory. The
\texttt{\ syntax.vim\ } file in the repo is supposed to be used by the
generation script and it \emph{will not work} if directly sourced in
Vim.

\subsubsection{How to add}\label{how-to-add}

Copy this into your project and use the import as \texttt{\ lasaveur\ }

\begin{verbatim}
#import "@preview/lasaveur:0.1.3"
\end{verbatim}

\includesvg[width=0.16667in,height=0.16667in]{/assets/icons/16-copy.svg}

Check the docs for
\href{https://typst.app/docs/reference/scripting/\#packages}{more
information on how to import packages} .

\subsubsection{About}\label{about}

\begin{description}
\tightlist
\item[Author :]
\href{https://github.com/yangwenbo99}{Paul Yang}
\item[License:]
MIT
\item[Current version:]
0.1.3
\item[Last updated:]
August 22, 2024
\item[First released:]
August 22, 2024
\item[Archive size:]
2.25 kB
\href{https://packages.typst.org/preview/lasaveur-0.1.3.tar.gz}{\pandocbounded{\includesvg[keepaspectratio]{/assets/icons/16-download.svg}}}
\item[Repository:]
\href{https://github.com/yangwenbo99/typst-lasaveur}{GitHub}
\item[Categor y :]
\begin{itemize}
\tightlist
\item[]
\item
  \pandocbounded{\includesvg[keepaspectratio]{/assets/icons/16-hammer.svg}}
  \href{https://typst.app/universe/search/?category=utility}{Utility}
\end{itemize}
\end{description}

\subsubsection{Where to report issues?}\label{where-to-report-issues}

This package is a project of Paul Yang . Report issues on
\href{https://github.com/yangwenbo99/typst-lasaveur}{their repository} .
You can also try to ask for help with this package on the
\href{https://forum.typst.app}{Forum} .

Please report this package to the Typst team using the
\href{https://typst.app/contact}{contact form} if you believe it is a
safety hazard or infringes upon your rights.

\phantomsection\label{versions}
\subsubsection{Version history}\label{version-history}

\begin{longtable}[]{@{}ll@{}}
\toprule\noalign{}
Version & Release Date \\
\midrule\noalign{}
\endhead
\bottomrule\noalign{}
\endlastfoot
0.1.3 & August 22, 2024 \\
\end{longtable}

Typst GmbH did not create this package and cannot guarantee correct
functionality of this package or compatibility with any version of the
Typst compiler or app.
