\title{typst.app/universe/package/nifty-ntnu-thesis}

\phantomsection\label{banner}
\phantomsection\label{template-thumbnail}
\pandocbounded{\includegraphics[keepaspectratio]{https://packages.typst.org/preview/thumbnails/nifty-ntnu-thesis-0.1.1-small.webp}}

\section{nifty-ntnu-thesis}\label{nifty-ntnu-thesis}

{ 0.1.1 }

An NTNU thesis template

\href{/app?template=nifty-ntnu-thesis&version=0.1.1}{Create project in
app}

\phantomsection\label{readme}
Port of \href{https://github.com/COPCSE-NTNU/thesis-NTNU}{thesis-NTNU}
template to Typst.
\href{https://github.com/saimnaveediqbal/thesis-NTNU-typst/blob/main/template/main.typ}{main.pdf}
contains a full usage example, see
\href{https://github.com/saimnaveediqbal/thesis-NTNU-typst/blob/main/example.pdf}{example.pdf}
for a rendered pdf.

To use this template you need to import it at the beginning of your
document:

\begin{Shaded}
\begin{Highlighting}[]
\NormalTok{\#import "@preview/nifty{-}ntnu{-}thesis:0.1.0": *}
\end{Highlighting}
\end{Shaded}

The template has many arguments you can specify:

\begin{longtable}[]{@{}llll@{}}
\toprule\noalign{}
Argument & Default Value & Type & Description \\
\midrule\noalign{}
\endhead
\bottomrule\noalign{}
\endlastfoot
\texttt{\ title\ } & \texttt{\ Title\ } & {[}content{]} & The title of
the thesis. \\
\texttt{\ short-title\ } & \texttt{\ Title\ } & {[}content{]} & Short
form of the title. If specified, will show up in the header \\
\texttt{\ author\ } & \texttt{\ Author\ } & {[}array{]} & An array of
authors \\
\texttt{\ short-author\ } & `` & {[}string{]} & Short form version of
the authors. If specified, will show up in header \\
\texttt{\ font\ } & \texttt{\ Charter\ } & {[}string{]} & Main font of
template \\
\texttt{\ raw-font\ } & \texttt{\ DejaVu\ Sans\ Mono\ } & {[}string{]} &
Font used for code listings \\
\texttt{\ paper-size\ } & \texttt{\ a4\ } & {[}string{]} & Specify a
{[}paper size string{]} to change the page size. \\
\texttt{\ date\ } & \texttt{\ datetime.today()\ } & {[}datetime{]} & The
date that will be displayed on the cover page. \\
\texttt{\ date-format\ } &
\texttt{\ {[}day\ padding:zero{]}/{[}month\ repr:numerical{]}/{[}year\ repr:full{]}\ }
& {[}string{]} & The format for the date that will be displayed on the
cover page. By default, the date will be displayed as
\texttt{\ DD/MM/YYYY\ } . \\
\texttt{\ abstract-en\ } & \texttt{\ none\ } & {[}content{]} & English
abstract shown before main content. \\
\texttt{\ abstract-no\ } & \texttt{\ none\ } & {[}content{]} & Norwegian
abstract shown before main content. \\
\texttt{\ preface\ } & \texttt{\ none\ } & {[}content{]} & The preface
for your work. The preface content is shown on its own separate page
after the abstracts. \\
\texttt{\ table-of-contents\ } & \texttt{\ outline()\ } & {[}content{]}
& The table of contents. Setting this to \texttt{\ none\ } will disable
the table of contents. \\
\texttt{\ titlepage\ } & \texttt{\ false\ } & {[}bool{]} & Whether to
display the titlepage or not. \\
\texttt{\ bibliography\ } & \texttt{\ none\ } & {[}content{]} & The
bibliography function or none. Specifying this will configure numeric,
IEEE-style citations. \\
\texttt{\ chapter-pagebreak\ } & \texttt{\ true\ } & {[}bool{]} &
Setting this to \texttt{\ false\ } will prevent chapters from starting
on a new page. \\
\texttt{\ chapters-on-odd\ } & \texttt{\ false\ } & {[}bool{]} & Setting
this to \texttt{\ false\ } will prevent chapters from only starting on
an odd page. \\
\texttt{\ figure-index\ } &
\texttt{\ (enabled:\ true,\ title:\ "Figures")\ } & {[}dictionary{]} &
Setting this to \texttt{\ true\ } will display a index of image figures
at the end of the document. \\
\texttt{\ table-index\ } &
\texttt{\ (enabled:\ true,\ title:\ "Tables")\ } & {[}dictionary{]} &
Setting this to \texttt{\ true\ } will display a index of table figures
at the end of the document. \\
\texttt{\ listing-index\ } &
\texttt{\ (enabled:\ true,\ title:\ "Listings")\ } & {[}dictionary{]} &
Setting this to \texttt{\ true\ } will display a index of listing (code
block) figures at the end of the document. \\
\end{longtable}

\begin{itemize}
\tightlist
\item
  {[} {]} Styling for:

  \begin{itemize}
  \tightlist
  \item
    {[}x{]} Code blocks
  \item
    {[}x{]} Tables
  \item
    {[}x{]} Header and footer
  \item
    {[}x{]} Lists
  \end{itemize}
\item
  {[}x{]} Subfigures
\item
  {[}x{]} Abstract
\item
  {[}x{]} Preface
\item
  {[}x{]} Code block captions
\item
  {[}x{]} Bibliography
\item
  {[} {]} Norwegian language support
\item
  {[}x{]} Proper figure numbering
\item
  {[}x{]} Short title in header
\item
  {[}x{]} Multiple authors
\item
  {[}x{]} Start chapters on only odd pages
\end{itemize}

Thanks to:

\begin{itemize}
\tightlist
\item
  The creator of the
  \href{https://github.com/talal/ilm/blob/main/lib.typ}{ILM template}
  which I used as the basis for this.
\item
  The creators of the original
  \href{https://github.com/COPCSE-NTNU/thesis-NTNU}{NTNU thesis
  template}
\item
  The creators of the
  \href{https://github.com/maucejo/elsearticle}{elsearticle template}
  for their implementation of subfigures and appendix environment
\end{itemize}

\href{/app?template=nifty-ntnu-thesis&version=0.1.1}{Create project in
app}

\subsubsection{How to use}\label{how-to-use}

Click the button above to create a new project using this template in
the Typst app.

You can also use the Typst CLI to start a new project on your computer
using this command:

\begin{verbatim}
typst init @preview/nifty-ntnu-thesis:0.1.1
\end{verbatim}

\includesvg[width=0.16667in,height=0.16667in]{/assets/icons/16-copy.svg}

\subsubsection{About}\label{about}

\begin{description}
\tightlist
\item[Author :]
Saim Iqbal
\item[License:]
MIT
\item[Current version:]
0.1.1
\item[Last updated:]
November 6, 2024
\item[First released:]
August 29, 2024
\item[Archive size:]
856 kB
\href{https://packages.typst.org/preview/nifty-ntnu-thesis-0.1.1.tar.gz}{\pandocbounded{\includesvg[keepaspectratio]{/assets/icons/16-download.svg}}}
\item[Repository:]
\href{https://github.com/saimnaveediqbal/thesis-NTNU-typst}{GitHub}
\item[Categor y :]
\begin{itemize}
\tightlist
\item[]
\item
  \pandocbounded{\includesvg[keepaspectratio]{/assets/icons/16-mortarboard.svg}}
  \href{https://typst.app/universe/search/?category=thesis}{Thesis}
\end{itemize}
\end{description}

\subsubsection{Where to report issues?}\label{where-to-report-issues}

This template is a project of Saim Iqbal . Report issues on
\href{https://github.com/saimnaveediqbal/thesis-NTNU-typst}{their
repository} . You can also try to ask for help with this template on the
\href{https://forum.typst.app}{Forum} .

Please report this template to the Typst team using the
\href{https://typst.app/contact}{contact form} if you believe it is a
safety hazard or infringes upon your rights.

\phantomsection\label{versions}
\subsubsection{Version history}\label{version-history}

\begin{longtable}[]{@{}ll@{}}
\toprule\noalign{}
Version & Release Date \\
\midrule\noalign{}
\endhead
\bottomrule\noalign{}
\endlastfoot
0.1.1 & November 6, 2024 \\
\href{https://typst.app/universe/package/nifty-ntnu-thesis/0.1.0/}{0.1.0}
& August 29, 2024 \\
\end{longtable}

Typst GmbH did not create this template and cannot guarantee correct
functionality of this template or compatibility with any version of the
Typst compiler or app.
