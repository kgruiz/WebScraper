\title{typst.app/universe/package/diatypst}

\phantomsection\label{banner}
\phantomsection\label{template-thumbnail}
\pandocbounded{\includegraphics[keepaspectratio]{https://packages.typst.org/preview/thumbnails/diatypst-0.3.0-small.webp}}

\section{diatypst}\label{diatypst}

{ 0.3.0 }

Easy slides with Typst â€`` sensible defaults, easy syntax, well-styled

{ } Featured Template

\href{/app?template=diatypst&version=0.3.0}{Create project in app}

\phantomsection\label{readme}
\emph{easy slides in typst}

Features:

\begin{itemize}
\tightlist
\item
  easy delimiter for slides and sections (just use headings)
\item
  sensible styling
\item
  dot counter in upper right corner (like LaTeX beamer)
\item
  adjustable color-theme
\item
  default show rules for terms, code, lists, … that match color-theme
\end{itemize}

Example Presentation

\begin{longtable}[]{@{}llll@{}}
\toprule\noalign{}
Title Slide & Section & Content & Outline \\
\midrule\noalign{}
\endhead
\bottomrule\noalign{}
\endlastfoot
\pandocbounded{\includegraphics[keepaspectratio]{https://github.com/typst/packages/raw/main/packages/preview/diatypst/0.3.0/screenshots/Example-Title.jpg}}
&
\pandocbounded{\includegraphics[keepaspectratio]{https://github.com/typst/packages/raw/main/packages/preview/diatypst/0.3.0/screenshots/Example-Section.jpg}}
&
\pandocbounded{\includegraphics[keepaspectratio]{https://github.com/typst/packages/raw/main/packages/preview/diatypst/0.3.0/screenshots/Example-Slide.jpg}}
&
\pandocbounded{\includegraphics[keepaspectratio]{https://github.com/typst/packages/raw/main/packages/preview/diatypst/0.3.0/screenshots/Example-TOC.jpg}} \\
\end{longtable}

These example slides and a usage guide are available in the
\texttt{\ example\ } Folder on GitHub as a
\href{https://github.com/skriptum/diatypst/blob/main/example/example.typ}{.typ
file} and a
\href{https://github.com/skriptum/diatypst/blob/main/example/example.pdf}{compiled
PDF}

\subsection{Usage}\label{usage}

To start a presentation, initialize it in your typst document:

\begin{Shaded}
\begin{Highlighting}[]
\NormalTok{\#import "@preview/diatypst:0.2.0": *}
\NormalTok{\#show: slides.with(}
\NormalTok{  title: "Diatypst", // Required}
\NormalTok{  subtitle: "easy slides in typst",}
\NormalTok{  date: "01.07.2024",}
\NormalTok{  authors: ("John Doe"),}
\NormalTok{)}
\NormalTok{...}
\end{Highlighting}
\end{Shaded}

Then, insert your content.

\begin{itemize}
\tightlist
\item
  Level-one headings corresponds to new sections.
\item
  Level-two headings corresponds to new slides.
\item
  or manually create a new slide with a \texttt{\ \#pagebreak()\ }
\end{itemize}

\begin{Shaded}
\begin{Highlighting}[]
\NormalTok{...}

\NormalTok{= First Section}

\NormalTok{== First Slide}

\NormalTok{\#lorem(20)}
\end{Highlighting}
\end{Shaded}

\emph{diatypst} is also available on the
\href{https://typst.app/universe/package/diatypst}{Typst Universe} for
easy importing into a project on typst.app

\subsection{Options}\label{options}

all available Options to initialize the template with

\begin{longtable}[]{@{}lll@{}}
\toprule\noalign{}
Keyword & Description & Default \\
\midrule\noalign{}
\endhead
\bottomrule\noalign{}
\endlastfoot
\emph{title} & Title of your Presentation, visible also in footer &
\texttt{\ none\ } but required! \\
\emph{subtitle} & Subtitle, also visible in footer &
\texttt{\ none\ } \\
\emph{date} & a normal string presenting your date &
\texttt{\ none\ } \\
\emph{authors} & either string or array of strings &
\texttt{\ none\ } \\
\emph{layout} & one of “small�, “medium�, “large�, adjusts
sizing of the elements on the slides & \texttt{\ "medium"\ } \\
\emph{ratio} & aspect ratio of the slides, e.g 16/9 &
\texttt{\ 4/3\ } \\
\emph{title-color} & Color to base the Elements of the Presentation on &
\texttt{\ blue.darken(50\%)\ } \\
\emph{count} & whether to display the dots for pages in upper right
corner & \texttt{\ true\ } \\
\emph{footer} & whether to display the footer at the bottom &
\texttt{\ true\ } \\
\emph{toc} & whether to display the table of contents &
\texttt{\ true\ } \\
\emph{footer-title} & custom text in the footer title (left) & same as
\emph{title} \\
\emph{footer-subtitle} & custom text in the footer subtitle (right) &
same as \emph{subtitle} \\
\end{longtable}

\subsection{Quarto}\label{quarto}

This template is also available as a \href{https://quarto.org/}{Quarto}
extension. To use it, add it to your project with the following command:

\begin{Shaded}
\begin{Highlighting}[]
\ExtensionTok{quarto}\NormalTok{ add skriptum/diatypst/diaquarto}
\end{Highlighting}
\end{Shaded}

Then, create a qmd file with the following YAML frontmatter:

\begin{Shaded}
\begin{Highlighting}[]
\FunctionTok{title}\KeywordTok{:}\AttributeTok{ }\StringTok{"Untitled"}
\CommentTok{...}
\CommentTok{format:}
\CommentTok{  diaquarto{-}typst: }
\CommentTok{    layout: medium \# small, medium, large}
\CommentTok{    ratio: 16/9 \# any ratio possible }
\CommentTok{    title{-}color: "013220" \# Any Hex code for the title color (without \#)}
\end{Highlighting}
\end{Shaded}

\subsection{Inspiration}\label{inspiration}

this template is inspired by
\href{https://github.com/glambrechts/slydst}{slydst} , and takes part of
the code from it. If you want simpler slides, look here!

The word \emph{Diatypst} is inspired by the ease of use of a
\href{https://de.wikipedia.org/wiki/Diaprojektor}{\textbf{Dia}
-projektor} (German for Slide Projector) and the
\href{https://en.wikipedia.org/wiki/Diatype_(machine)}{Diatype}

\href{/app?template=diatypst&version=0.3.0}{Create project in app}

\subsubsection{How to use}\label{how-to-use}

Click the button above to create a new project using this template in
the Typst app.

You can also use the Typst CLI to start a new project on your computer
using this command:

\begin{verbatim}
typst init @preview/diatypst:0.3.0
\end{verbatim}

\includesvg[width=0.16667in,height=0.16667in]{/assets/icons/16-copy.svg}

\subsubsection{About}\label{about}

\begin{description}
\tightlist
\item[Author :]
skriptum (https://github.com/skriptum)
\item[License:]
MIT-0
\item[Current version:]
0.3.0
\item[Last updated:]
November 18, 2024
\item[First released:]
July 22, 2024
\item[Minimum Typst version:]
0.12.0
\item[Archive size:]
4.89 kB
\href{https://packages.typst.org/preview/diatypst-0.3.0.tar.gz}{\pandocbounded{\includesvg[keepaspectratio]{/assets/icons/16-download.svg}}}
\item[Repository:]
\href{https://github.com/skriptum/Diatypst}{GitHub}
\item[Categor y :]
\begin{itemize}
\tightlist
\item[]
\item
  \pandocbounded{\includesvg[keepaspectratio]{/assets/icons/16-presentation.svg}}
  \href{https://typst.app/universe/search/?category=presentation}{Presentation}
\end{itemize}
\end{description}

\subsubsection{Where to report issues?}\label{where-to-report-issues}

This template is a project of skriptum (https://github.com/skriptum) .
Report issues on \href{https://github.com/skriptum/Diatypst}{their
repository} . You can also try to ask for help with this template on the
\href{https://forum.typst.app}{Forum} .

Please report this template to the Typst team using the
\href{https://typst.app/contact}{contact form} if you believe it is a
safety hazard or infringes upon your rights.

\phantomsection\label{versions}
\subsubsection{Version history}\label{version-history}

\begin{longtable}[]{@{}ll@{}}
\toprule\noalign{}
Version & Release Date \\
\midrule\noalign{}
\endhead
\bottomrule\noalign{}
\endlastfoot
0.3.0 & November 18, 2024 \\
\href{https://typst.app/universe/package/diatypst/0.2.0/}{0.2.0} &
November 6, 2024 \\
\href{https://typst.app/universe/package/diatypst/0.1.0/}{0.1.0} & July
22, 2024 \\
\end{longtable}

Typst GmbH did not create this template and cannot guarantee correct
functionality of this template or compatibility with any version of the
Typst compiler or app.
