\title{typst.app/universe/package/vartable}

\phantomsection\label{banner}
\section{vartable}\label{vartable}

{ 0.1.2 }

A simple package to make variation table

\phantomsection\label{readme}
An easy way to render variation table on typst, built on
\href{https://github.com/Jollywatt/typst-fletcher}{fletcher}\\
The
\href{https://github.com/Le-foucheur/Typst-VarTable/blob/main/documentation.pdf}{documention}

\begin{Shaded}
\begin{Highlighting}[]
\NormalTok{\#import "@preview/Tabvar:0.1.0": tabvar}
\end{Highlighting}
\end{Shaded}

\subsubsection{Trigonometric functions}\label{trigonometric-functions}

Turn this :

\begin{Shaded}
\begin{Highlighting}[]
\NormalTok{\#import }\StringTok{"@preview/Tabvar:0.1.0"}\OperatorTok{:}\NormalTok{ tabvar}

\NormalTok{\#}\FunctionTok{tabvar}\NormalTok{(}
\NormalTok{  init}\OperatorTok{:}\NormalTok{ (}
\NormalTok{    variable}\OperatorTok{:}\NormalTok{ $x$}\OperatorTok{,}
\NormalTok{    label}\OperatorTok{:}\NormalTok{ (}
\NormalTok{      ([sign }\KeywordTok{of}\NormalTok{ cos]}\OperatorTok{,} \StringTok{"Sign"}\NormalTok{)}\OperatorTok{,}
\NormalTok{      ([variation }\KeywordTok{of}\NormalTok{ cos]}\OperatorTok{,} \StringTok{"Variation"}\NormalTok{)}\OperatorTok{,}
\NormalTok{      ([sign }\KeywordTok{of}\NormalTok{ sin]}\OperatorTok{,} \StringTok{"Sign"}\NormalTok{)}\OperatorTok{,}
\NormalTok{      ([variation }\KeywordTok{of}\NormalTok{ sin]}\OperatorTok{,} \StringTok{"Variation"}\NormalTok{)}\OperatorTok{,}
\NormalTok{    )}\OperatorTok{,}
\NormalTok{  )}\OperatorTok{,}
\NormalTok{  domain}\OperatorTok{:}\NormalTok{ ($0$}\OperatorTok{,}\NormalTok{ $ pi }\OperatorTok{/} \DecValTok{2}\NormalTok{ $}\OperatorTok{,}\NormalTok{ $ pi $}\OperatorTok{,} \FunctionTok{$}\NormalTok{ (}\DecValTok{2}\ErrorTok{pi}\NormalTok{) }\OperatorTok{/} \DecValTok{3}\NormalTok{ $}\OperatorTok{,}\NormalTok{ $ }\DecValTok{2}\NormalTok{ pi $)}\OperatorTok{,}
\NormalTok{  content}\OperatorTok{:}\NormalTok{ (}
\NormalTok{    ($}\OperatorTok{{-}}\NormalTok{$}\OperatorTok{,}\NormalTok{ ()}\OperatorTok{,}\NormalTok{ $}\OperatorTok{+}\NormalTok{$}\OperatorTok{,}\NormalTok{ ())}\OperatorTok{,}
\NormalTok{    (}
\NormalTok{      (top}\OperatorTok{,}\NormalTok{ $1$)}\OperatorTok{,}
\NormalTok{      ()}\OperatorTok{,}
\NormalTok{      (bottom}\OperatorTok{,}\NormalTok{ $}\OperatorTok{{-}}\DecValTok{1}\ErrorTok{$}\NormalTok{)}\OperatorTok{,}
\NormalTok{      ()}\OperatorTok{,}
\NormalTok{      (top}\OperatorTok{,}\NormalTok{ $1$)}\OperatorTok{,}
\NormalTok{    )}\OperatorTok{,}
\NormalTok{    ($}\OperatorTok{+}\NormalTok{$}\OperatorTok{,}\NormalTok{ $}\OperatorTok{{-}}\NormalTok{$}\OperatorTok{,}\NormalTok{ ()}\OperatorTok{,}\NormalTok{ $}\OperatorTok{+}\NormalTok{$)}\OperatorTok{,}
\NormalTok{    (}
\NormalTok{      (center}\OperatorTok{,}\NormalTok{ $0$)}\OperatorTok{,}
\NormalTok{      (top}\OperatorTok{,}\NormalTok{ $1$)}\OperatorTok{,}
\NormalTok{      ()}\OperatorTok{,}
\NormalTok{      (bottom}\OperatorTok{,}\NormalTok{ $}\OperatorTok{{-}}\DecValTok{1}\ErrorTok{$}\NormalTok{)}\OperatorTok{,}
\NormalTok{      (top}\OperatorTok{,}\NormalTok{ $1$)}\OperatorTok{,}
\NormalTok{    )}\OperatorTok{,}
\NormalTok{  )}\OperatorTok{,}
\NormalTok{)}
\end{Highlighting}
\end{Shaded}

Into this

\pandocbounded{\includegraphics[keepaspectratio]{https://github.com/typst/packages/raw/main/packages/preview/vartable/0.1.2/examples/trigonometricFunction.png}}

\subsubsection{hyperbolic function \$f(x) = 1/x
\$}\label{hyperbolic-function-fx-1x}

\begin{Shaded}
\begin{Highlighting}[]
\NormalTok{\#import }\StringTok{"@preview/Tabvar:0.1.0"}\OperatorTok{:}\NormalTok{ tabvar}

\NormalTok{\#}\FunctionTok{tabvar}\NormalTok{(}
\NormalTok{    init}\OperatorTok{:}\NormalTok{ (}
\NormalTok{        variable}\OperatorTok{:}\NormalTok{ $x$}\OperatorTok{,}
\NormalTok{    label}\OperatorTok{:}\NormalTok{ (}
\NormalTok{        ([sign }\KeywordTok{of}\NormalTok{ $f$]}\OperatorTok{,} \StringTok{"Sign"}\NormalTok{)}\OperatorTok{,}
\NormalTok{      ([variation }\KeywordTok{of}\NormalTok{ $f$]}\OperatorTok{,} \StringTok{"Variation"}\NormalTok{)}\OperatorTok{,}
\NormalTok{    )}\OperatorTok{,}
\NormalTok{  )}\OperatorTok{,}
\NormalTok{  domain}\OperatorTok{:}\NormalTok{ ($ }\OperatorTok{{-}}\NormalTok{oo $}\OperatorTok{,}\NormalTok{ $ }\DecValTok{0}\NormalTok{ $}\OperatorTok{,}\NormalTok{ $ }\OperatorTok{+}\NormalTok{oo $)}\OperatorTok{,}
\NormalTok{  content}\OperatorTok{:}\NormalTok{ (}
\NormalTok{      ($}\OperatorTok{+}\NormalTok{$}\OperatorTok{,}\NormalTok{ (}\StringTok{"||"}\OperatorTok{,}\NormalTok{ $}\OperatorTok{+}\NormalTok{$))}\OperatorTok{,}
\NormalTok{    (}
\NormalTok{        (center}\OperatorTok{,}\NormalTok{ $0$)}\OperatorTok{,}
\NormalTok{      (bottom}\OperatorTok{,}\NormalTok{ top}\OperatorTok{,} \StringTok{"||"}\OperatorTok{,}\NormalTok{ $}\OperatorTok{{-}}\NormalTok{oo$}\OperatorTok{,}\NormalTok{ $}\OperatorTok{+}\NormalTok{oo$)}\OperatorTok{,}
\NormalTok{      (center}\OperatorTok{,}\NormalTok{ $0$)}\OperatorTok{,}
\NormalTok{    )}\OperatorTok{,}
\NormalTok{  )}\OperatorTok{,}
\NormalTok{)}
\end{Highlighting}
\end{Shaded}

\pandocbounded{\includegraphics[keepaspectratio]{https://github.com/typst/packages/raw/main/packages/preview/vartable/0.1.2/examples/hyperbolicFuntion.png}}

\begin{itemize}
\tightlist
\item
  if you put too wide an element for the last value of a variation
  table, this can create a space between the edge of the table and the
  lines separating the lines of the table
\end{itemize}

\pandocbounded{\includegraphics[keepaspectratio]{https://github.com/typst/packages/raw/main/packages/preview/vartable/0.1.2/examples/bug1.png}}

\subsection{·change log·}\label{uxe2change-loguxe2}

\paragraph{0.1.2 :}\label{uxe2}

\begin{itemize}
\tightlist
\item
  Support \texttt{\ fletcher\ 0.5.2\ }
\end{itemize}

\paragraph{0.1.1 :}\label{uxe2-1}

\begin{itemize}
\tightlist
\item
  added customisation of separator bars between signs
\end{itemize}

\subparagraph{0.1.0 :}\label{uxe2-2}

\begin{itemize}
\tightlist
\item
  publishing the package
\end{itemize}

\subsubsection{How to add}\label{how-to-add}

Copy this into your project and use the import as \texttt{\ vartable\ }

\begin{verbatim}
#import "@preview/vartable:0.1.2"
\end{verbatim}

\includesvg[width=0.16667in,height=0.16667in]{/assets/icons/16-copy.svg}

Check the docs for
\href{https://typst.app/docs/reference/scripting/\#packages}{more
information on how to import packages} .

\subsubsection{About}\label{about}

\begin{description}
\tightlist
\item[Author :]
Le\_Foucheur
\item[License:]
MIT
\item[Current version:]
0.1.2
\item[Last updated:]
October 29, 2024
\item[First released:]
July 2, 2024
\item[Archive size:]
114 kB
\href{https://packages.typst.org/preview/vartable-0.1.2.tar.gz}{\pandocbounded{\includesvg[keepaspectratio]{/assets/icons/16-download.svg}}}
\item[Repository:]
\href{https://github.com/Le-foucheur/Typst-VarTable}{GitHub}
\item[Categor y :]
\begin{itemize}
\tightlist
\item[]
\item
  \pandocbounded{\includesvg[keepaspectratio]{/assets/icons/16-chart.svg}}
  \href{https://typst.app/universe/search/?category=visualization}{Visualization}
\end{itemize}
\end{description}

\subsubsection{Where to report issues?}\label{where-to-report-issues}

This package is a project of Le\_Foucheur . Report issues on
\href{https://github.com/Le-foucheur/Typst-VarTable}{their repository} .
You can also try to ask for help with this package on the
\href{https://forum.typst.app}{Forum} .

Please report this package to the Typst team using the
\href{https://typst.app/contact}{contact form} if you believe it is a
safety hazard or infringes upon your rights.

\phantomsection\label{versions}
\subsubsection{Version history}\label{version-history}

\begin{longtable}[]{@{}ll@{}}
\toprule\noalign{}
Version & Release Date \\
\midrule\noalign{}
\endhead
\bottomrule\noalign{}
\endlastfoot
0.1.2 & October 29, 2024 \\
\href{https://typst.app/universe/package/vartable/0.1.1/}{0.1.1} &
October 14, 2024 \\
\href{https://typst.app/universe/package/vartable/0.1.0/}{0.1.0} & July
2, 2024 \\
\end{longtable}

Typst GmbH did not create this package and cannot guarantee correct
functionality of this package or compatibility with any version of the
Typst compiler or app.
