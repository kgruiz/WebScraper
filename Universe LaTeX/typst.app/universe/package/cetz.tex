\title{typst.app/universe/package/cetz}

\phantomsection\label{banner}
\section{cetz}\label{cetz}

{ 0.3.1 }

Drawing with Typst made easy, providing an API inspired by TikZ and
Processing. Includes modules for plotting, charts and tree layout.

{ } Featured Package

\phantomsection\label{readme}
CeTZ (CeTZ, ein Typst Zeichenpaket) is a library for drawing with
\href{https://typst.app/}{Typst} with an API inspired by TikZ and
\href{https://processing.org/}{Processing} .

\subsection{Examples}\label{examples}

\begin{longtable}[]{@{}lll@{}}
\toprule\noalign{}
\endhead
\bottomrule\noalign{}
\endlastfoot
\href{https://github.com/typst/packages/raw/main/packages/preview/cetz/0.3.1/gallery/karls-picture.typ}{\includegraphics[width=2.60417in,height=\textheight,keepaspectratio]{https://github.com/typst/packages/raw/main/packages/preview/cetz/0.3.1/gallery/karls-picture.png}}
&
\href{https://github.com/typst/packages/raw/main/packages/preview/cetz/0.3.1/gallery/tree.typ}{\includegraphics[width=2.60417in,height=\textheight,keepaspectratio]{https://github.com/typst/packages/raw/main/packages/preview/cetz/0.3.1/gallery/tree.png}}
&
\href{https://github.com/typst/packages/raw/main/packages/preview/cetz/0.3.1/gallery/waves.typ}{\includegraphics[width=2.60417in,height=\textheight,keepaspectratio]{https://github.com/typst/packages/raw/main/packages/preview/cetz/0.3.1/gallery/waves.png}} \\
Karl\textquotesingle s Picture & Tree Layout & Waves \\
\end{longtable}

\emph{Click on the example image to jump to the code.}

\subsection{Usage}\label{usage}

For information, see the
\href{https://cetz-package.github.io/docs}{online manual} .

To use this package, simply add the following code to your document:

\begin{verbatim}
#import "@preview/cetz:0.3.1"

#cetz.canvas({
  import cetz.draw: *
  // Your drawing code goes here
})
\end{verbatim}

\subsection{CeTZ Libraries}\label{cetz-libraries}

\begin{itemize}
\tightlist
\item
  \href{https://github.com/cetz-package/cetz-plot}{cetz-plot - Plotting
  and Charts Library}
\item
  \href{https://github.com/cetz-package/cetz-venn}{cetz-venn - Simple
  two- or three-set Venn diagrams}
\end{itemize}

\subsection{Installing}\label{installing}

To install the CeTZ package under
\href{https://github.com/typst/packages?tab=readme-ov-file\#local-packages}{your
local typst package dir} you can use the \texttt{\ install\ } script
from the repository.

\begin{Shaded}
\begin{Highlighting}[]
\ExtensionTok{just}\NormalTok{ install}
\end{Highlighting}
\end{Shaded}

The installed version can be imported by prefixing the package name with
\texttt{\ @local\ } .

\begin{Shaded}
\begin{Highlighting}[]
\NormalTok{\#import "@local/cetz:0.3.1"}

\NormalTok{\#cetz.canvas(\{}
\NormalTok{  import cetz.draw: *}
\NormalTok{  // Your drawing code goes here}
\NormalTok{\})}
\end{Highlighting}
\end{Shaded}

\subsubsection{Just}\label{just}

This project uses \href{https://github.com/casey/just}{just} , a handy
command runner.

You can run all commands without having \texttt{\ just\ } installed,
just have a look into the \texttt{\ justfile\ } . To install
\texttt{\ just\ } on your system, use your systems package manager. On
Windows, \href{https://doc.rust-lang.org/cargo/}{Cargo} (
\texttt{\ cargo\ install\ just\ } ),
\href{https://chocolatey.org/}{Chocolatey} (
\texttt{\ choco\ install\ just\ } ) and
\href{https://just.systems/man/en/chapter_4.html}{some other sources}
can be used. You need to run it from a \texttt{\ sh\ } compatible shell
on Windows (e.g git-bash).

\subsection{Testing}\label{testing}

This package comes with some unit tests under the \texttt{\ tests\ }
directory. To run all tests you can run the \texttt{\ just\ test\ }
target. You need to have
\href{https://github.com/tingerrr/typst-test/}{\texttt{\ typst-test\ }}
in your \texttt{\ PATH\ } :
\texttt{\ cargo\ install\ typst-test\ -\/-git\ https://github.com/tingerrr/typst-test\ }
.

\subsection{Projects using CeTZ}\label{projects-using-cetz}

\begin{itemize}
\tightlist
\item
  \href{https://github.com/fenjalien/cirCeTZ}{cirCeTZ} A port of
  \href{https://github.com/circuitikz/circuitikz}{circuitikz} to Typst.
\item
  \href{https://github.com/sitandr/conchord}{conchord} Package for
  writing lyrics with chords that generates fretboard diagrams using
  CeTZ.
\item
  \href{https://github.com/jneug/typst-finite}{finite} Finite is a Typst
  package for rendering finite automata.
\item
  \href{https://github.com/Jollywatt/typst-fletcher}{fletcher} Package
  for drawing commutative diagrams and figures with arrows.
\item
  \href{https://github.com/ThatOneCalculator/riesketcher}{riesketcher}
  Package for drawing Riemann sums.
\end{itemize}

\subsubsection{How to add}\label{how-to-add}

Copy this into your project and use the import as \texttt{\ cetz\ }

\begin{verbatim}
#import "@preview/cetz:0.3.1"
\end{verbatim}

\includesvg[width=0.16667in,height=0.16667in]{/assets/icons/16-copy.svg}

Check the docs for
\href{https://typst.app/docs/reference/scripting/\#packages}{more
information on how to import packages} .

\subsubsection{About}\label{about}

\begin{description}
\tightlist
\item[Author s :]
\href{https://github.com/johannes-wolf}{Johannes Wolf} \&
\href{https://github.com/fenjalien}{fenjalien}
\item[License:]
LGPL-3.0-or-later
\item[Current version:]
0.3.1
\item[Last updated:]
October 21, 2024
\item[First released:]
July 8, 2023
\item[Minimum Typst version:]
0.12.0
\item[Archive size:]
74.3 kB
\href{https://packages.typst.org/preview/cetz-0.3.1.tar.gz}{\pandocbounded{\includesvg[keepaspectratio]{/assets/icons/16-download.svg}}}
\item[Repository:]
\href{https://github.com/cetz-package/cetz}{GitHub}
\item[Categor y :]
\begin{itemize}
\tightlist
\item[]
\item
  \pandocbounded{\includesvg[keepaspectratio]{/assets/icons/16-chart.svg}}
  \href{https://typst.app/universe/search/?category=visualization}{Visualization}
\end{itemize}
\end{description}

\subsubsection{Where to report issues?}\label{where-to-report-issues}

This package is a project of Johannes Wolf and fenjalien . Report issues
on \href{https://github.com/cetz-package/cetz}{their repository} . You
can also try to ask for help with this package on the
\href{https://forum.typst.app}{Forum} .

Please report this package to the Typst team using the
\href{https://typst.app/contact}{contact form} if you believe it is a
safety hazard or infringes upon your rights.

\phantomsection\label{versions}
\subsubsection{Version history}\label{version-history}

\begin{longtable}[]{@{}ll@{}}
\toprule\noalign{}
Version & Release Date \\
\midrule\noalign{}
\endhead
\bottomrule\noalign{}
\endlastfoot
0.3.1 & October 21, 2024 \\
\href{https://typst.app/universe/package/cetz/0.3.0/}{0.3.0} & October
15, 2024 \\
\href{https://typst.app/universe/package/cetz/0.2.2/}{0.2.2} & March 18,
2024 \\
\href{https://typst.app/universe/package/cetz/0.2.1/}{0.2.1} & February
23, 2024 \\
\href{https://typst.app/universe/package/cetz/0.2.0/}{0.2.0} & January
16, 2024 \\
\href{https://typst.app/universe/package/cetz/0.1.2/}{0.1.2} & October
1, 2023 \\
\href{https://typst.app/universe/package/cetz/0.1.1/}{0.1.1} & September
11, 2023 \\
\href{https://typst.app/universe/package/cetz/0.1.0/}{0.1.0} & August
19, 2023 \\
\href{https://typst.app/universe/package/cetz/0.0.2/}{0.0.2} & July 31,
2023 \\
\href{https://typst.app/universe/package/cetz/0.0.1/}{0.0.1} & July 8,
2023 \\
\end{longtable}

Typst GmbH did not create this package and cannot guarantee correct
functionality of this package or compatibility with any version of the
Typst compiler or app.
