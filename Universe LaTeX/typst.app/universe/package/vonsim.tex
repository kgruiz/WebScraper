\title{typst.app/universe/package/vonsim}

\phantomsection\label{banner}
\section{vonsim}\label{vonsim}

{ 0.1.0 }

Syntax highlighting support for VonSim.

\phantomsection\label{readme}
This package adds the ability to syntax highlighting VonSim source code
in Typst.

\subsection{How to use}\label{how-to-use}

To add global support for VonSim, just add these lines and use a raw
block with \texttt{\ vonsim\ } as its language.

\begin{Shaded}
\begin{Highlighting}[]
\NormalTok{\#import "@preview/vonsim:0.1.0": init{-}vonsim}

\NormalTok{// Adds global support for VonSim}
\NormalTok{\#show: init{-}vonsim}

\NormalTok{// Highlight VonSim code}
\NormalTok{\textasciigrave{}\textasciigrave{}\textasciigrave{}vonsim}
\NormalTok{; Welcome to VonSim!}
\NormalTok{; This is an example program that calculates the first}
\NormalTok{; n numbers of the Fibonacci sequence, and stores them}
\NormalTok{; starting at memory position 1000h.}

\NormalTok{     n  equ 10    ; Calculate the first 10 numbers}

\NormalTok{        org 1000h}
\NormalTok{start   db 1}

\NormalTok{        org 2000h}
\NormalTok{        mov bx, offset start + 1}
\NormalTok{        mov al, 0}
\NormalTok{        mov ah, start}

\NormalTok{loop:   cmp bx, offset start + n}
\NormalTok{        jns finish}
\NormalTok{        mov cl, ah}
\NormalTok{        add cl, al}
\NormalTok{        mov al, ah}
\NormalTok{        mov ah, cl}
\NormalTok{        mov [bx], cl}
\NormalTok{        inc bx}
\NormalTok{        jmp loop}
\NormalTok{finish: hlt}
\NormalTok{        end}
\NormalTok{\textasciigrave{}\textasciigrave{}\textasciigrave{}}
\end{Highlighting}
\end{Shaded}

Alternatively, use \texttt{\ init-vonsim-full\ } to also use the VonSim
theme.

\subsubsection{How to add}\label{how-to-add}

Copy this into your project and use the import as \texttt{\ vonsim\ }

\begin{verbatim}
#import "@preview/vonsim:0.1.0"
\end{verbatim}

\includesvg[width=0.16667in,height=0.16667in]{/assets/icons/16-copy.svg}

Check the docs for
\href{https://typst.app/docs/reference/scripting/\#packages}{more
information on how to import packages} .

\subsubsection{About}\label{about}

\begin{description}
\tightlist
\item[Author :]
\href{https://github.com/JuanM04}{Juan Martín Seery}
\item[License:]
AGPL-3.0-only
\item[Current version:]
0.1.0
\item[Last updated:]
June 10, 2024
\item[First released:]
June 10, 2024
\item[Minimum Typst version:]
0.11.0
\item[Archive size:]
13.8 kB
\href{https://packages.typst.org/preview/vonsim-0.1.0.tar.gz}{\pandocbounded{\includesvg[keepaspectratio]{/assets/icons/16-download.svg}}}
\item[Repository:]
\href{https://github.com/vonsim/typst-package}{GitHub}
\item[Categor y :]
\begin{itemize}
\tightlist
\item[]
\item
  \pandocbounded{\includesvg[keepaspectratio]{/assets/icons/16-text.svg}}
  \href{https://typst.app/universe/search/?category=text}{Text}
\end{itemize}
\end{description}

\subsubsection{Where to report issues?}\label{where-to-report-issues}

This package is a project of Juan Martín Seery . Report issues on
\href{https://github.com/vonsim/typst-package}{their repository} . You
can also try to ask for help with this package on the
\href{https://forum.typst.app}{Forum} .

Please report this package to the Typst team using the
\href{https://typst.app/contact}{contact form} if you believe it is a
safety hazard or infringes upon your rights.

\phantomsection\label{versions}
\subsubsection{Version history}\label{version-history}

\begin{longtable}[]{@{}ll@{}}
\toprule\noalign{}
Version & Release Date \\
\midrule\noalign{}
\endhead
\bottomrule\noalign{}
\endlastfoot
0.1.0 & June 10, 2024 \\
\end{longtable}

Typst GmbH did not create this package and cannot guarantee correct
functionality of this package or compatibility with any version of the
Typst compiler or app.
