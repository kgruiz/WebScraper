\title{typst.app/universe/package/scrutinize}

\phantomsection\label{banner}
\section{scrutinize}\label{scrutinize}

{ 0.3.0 }

A library for building exams, tests, etc. with Typst

\phantomsection\label{readme}
Scrutinize is a library for building exams, tests, etc. with Typst. It
has three general areas of focus:

\begin{itemize}
\tightlist
\item
  It helps with grading information: record the points that can be
  reached for each question and make them available for creating grading
  keys.
\item
  It provides a selection of question writing utilities, such as
  multiple choice or true/false questions.
\item
  It supports the creation of sample solutions by allowing to switch
  between the normal and “pre-filled� exam.
\end{itemize}

Right now, providing a styled template is not part of this package’s
scope. Also, visual customization of the provided question templates is
currently nonexistent.

See the
\href{https://github.com/typst/packages/raw/main/packages/preview/scrutinize/0.3.0/docs/manual.pdf}{manual}
for details.

\subsection{Example}\label{example}

\begin{longtable}[]{@{}ll@{}}
\toprule\noalign{}
\endhead
\bottomrule\noalign{}
\endlastfoot
\href{https://github.com/typst/packages/raw/main/packages/preview/scrutinize/0.3.0/gallery/gk-ek-austria.typ}{\pandocbounded{\includegraphics[keepaspectratio]{https://github.com/typst/packages/raw/main/packages/preview/scrutinize/0.3.0/thumbnail.png}}}
&
\href{https://github.com/typst/packages/raw/main/packages/preview/scrutinize/0.3.0/gallery/gk-ek-austria.typ}{\pandocbounded{\includegraphics[keepaspectratio]{https://github.com/typst/packages/raw/main/packages/preview/scrutinize/0.3.0/thumbnail-solved.png}}} \\
\end{longtable}

This example can be found in the
\href{https://github.com/typst/packages/raw/main/packages/preview/scrutinize/0.3.0/gallery/}{gallery}
. Here are some excerpts from it:

\begin{Shaded}
\begin{Highlighting}[]
\NormalTok{\#import "@preview/scrutinize:0.3.0" as scrutinize: grading, task, solution, task{-}kinds}
\NormalTok{\#import task{-}kinds: free{-}form, gap, choice}
\NormalTok{\#import task: t}

\NormalTok{// ... document setup ...}

\NormalTok{\#context \{}
\NormalTok{  let ts = task.all(level: 2)}
\NormalTok{  let total = grading.total{-}points(ts)}

\NormalTok{  let grades = grading.grades(}
\NormalTok{    [F],}
\NormalTok{    0.6 * total,}
\NormalTok{    [D],}
\NormalTok{    0.7 * total,}
\NormalTok{    [C],}
\NormalTok{    0.8 * total,}
\NormalTok{    [B],}
\NormalTok{    0.9 * total,}
\NormalTok{    [A],}
\NormalTok{  )}

\NormalTok{  // ... show the grading key ...}
\NormalTok{\}}

\NormalTok{// ...}

\NormalTok{= Basic competencies {-}{-} theoretical part B}

\NormalTok{\#lorem(40)}

\NormalTok{== Writing}
\NormalTok{\#t(category: "b", points: 4)}
\NormalTok{\#lorem(30)}

\NormalTok{\#free{-}form.lines(stretch: 180\%, lorem(20))}

\NormalTok{== Multiple Choice}
\NormalTok{\#t(category: "b", points: 2)}
\NormalTok{\#lorem(30)}

\NormalTok{\#\{}
\NormalTok{  set align(center)}
\NormalTok{  choice.multiple((}
\NormalTok{    (lorem(3), true),}
\NormalTok{    (lorem(5), true),}
\NormalTok{    (lorem(4), false),}
\NormalTok{  ))}
\NormalTok{\}}
\end{Highlighting}
\end{Shaded}

\subsubsection{How to add}\label{how-to-add}

Copy this into your project and use the import as
\texttt{\ scrutinize\ }

\begin{verbatim}
#import "@preview/scrutinize:0.3.0"
\end{verbatim}

\includesvg[width=0.16667in,height=0.16667in]{/assets/icons/16-copy.svg}

Check the docs for
\href{https://typst.app/docs/reference/scripting/\#packages}{more
information on how to import packages} .

\subsubsection{About}\label{about}

\begin{description}
\tightlist
\item[Author :]
\href{https://github.com/SillyFreak/}{Clemens Koza}
\item[License:]
MIT
\item[Current version:]
0.3.0
\item[Last updated:]
October 14, 2024
\item[First released:]
January 8, 2024
\item[Minimum Typst version:]
0.11.0
\item[Archive size:]
11.2 kB
\href{https://packages.typst.org/preview/scrutinize-0.3.0.tar.gz}{\pandocbounded{\includesvg[keepaspectratio]{/assets/icons/16-download.svg}}}
\item[Repository:]
\href{https://github.com/SillyFreak/typst-scrutinize}{GitHub}
\item[Discipline :]
\begin{itemize}
\tightlist
\item[]
\item
  \href{https://typst.app/universe/search/?discipline=education}{Education}
\end{itemize}
\item[Categor ies :]
\begin{itemize}
\tightlist
\item[]
\item
  \pandocbounded{\includesvg[keepaspectratio]{/assets/icons/16-list-unordered.svg}}
  \href{https://typst.app/universe/search/?category=model}{Model}
\item
  \pandocbounded{\includesvg[keepaspectratio]{/assets/icons/16-code.svg}}
  \href{https://typst.app/universe/search/?category=scripting}{Scripting}
\item
  \pandocbounded{\includesvg[keepaspectratio]{/assets/icons/16-envelope.svg}}
  \href{https://typst.app/universe/search/?category=office}{Office}
\end{itemize}
\end{description}

\subsubsection{Where to report issues?}\label{where-to-report-issues}

This package is a project of Clemens Koza . Report issues on
\href{https://github.com/SillyFreak/typst-scrutinize}{their repository}
. You can also try to ask for help with this package on the
\href{https://forum.typst.app}{Forum} .

Please report this package to the Typst team using the
\href{https://typst.app/contact}{contact form} if you believe it is a
safety hazard or infringes upon your rights.

\phantomsection\label{versions}
\subsubsection{Version history}\label{version-history}

\begin{longtable}[]{@{}ll@{}}
\toprule\noalign{}
Version & Release Date \\
\midrule\noalign{}
\endhead
\bottomrule\noalign{}
\endlastfoot
0.3.0 & October 14, 2024 \\
\href{https://typst.app/universe/package/scrutinize/0.2.0/}{0.2.0} &
July 15, 2024 \\
\href{https://typst.app/universe/package/scrutinize/0.1.0/}{0.1.0} &
January 8, 2024 \\
\end{longtable}

Typst GmbH did not create this package and cannot guarantee correct
functionality of this package or compatibility with any version of the
Typst compiler or app.
