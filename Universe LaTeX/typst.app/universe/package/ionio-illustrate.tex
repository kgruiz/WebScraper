\title{typst.app/universe/package/ionio-illustrate}

\phantomsection\label{banner}
\section{ionio-illustrate}\label{ionio-illustrate}

{ 0.2.0 }

Mass spectra with annotations for typst.

\phantomsection\label{readme}
\phantomsection\label{readme-top}{}

\href{https://github.com/jamesxx/ionio-illustrate/blob/master/LICENSE}{\pandocbounded{\includegraphics[keepaspectratio]{https://img.shields.io/github/license/jamesxx/ionio-illustrate}}}
\href{https://github.com/typst/packages/tree/main/packages/preview/ionio-illustrate}{\pandocbounded{\includegraphics[keepaspectratio]{https://img.shields.io/badge/typst-package-239dad}}}
\href{https://github.com/JamesxX/ionio-illustrate/tags}{\pandocbounded{\includegraphics[keepaspectratio]{https://img.shields.io/github/v/tag/jamesxx/ionio-illustrate}}}

This package implements a Cetz chart-like object for displying mass
spectrometric data in Typst documents. It allows for individually styled
mass peaks, callouts, titles, and mass callipers.\\

\href{https://github.com/jamesxx/ionio-illustrate/blob/main/manual.pdf}{\textbf{Explore
the docs »}}\\
\strut \\
\href{https://github.com/jamesxx/ionio-illustrate/issues}{Report Bug} ·
\href{https://github.com/jamesxx/ionio-illustrate/issues}{Request
Feature}

\subsection{Getting Started}\label{getting-started}

To make use of the \texttt{\ ionio-illustrate\ } package, you’ll need
to add it to your project like shown below. Make sure you are importing
a version that supports your end goal.

\begin{Shaded}
\begin{Highlighting}[]
\NormalTok{\#import "@preview/ionio{-}illustrate:0.2.0": *}
\end{Highlighting}
\end{Shaded}

Then, load in your mass spectrum data and pass it through to the package
like so. Data should be 2D array, and by default the mass-charge ratio
is in the first column, and the relative intensities are in the second
column.

\begin{Shaded}
\begin{Highlighting}[]
\NormalTok{\#let data = csv("isobutelene\_epoxide.csv")}

\NormalTok{\#let ms = mass{-}spectrum(massspec, args: (}
\NormalTok{  size: (12,6),}
\NormalTok{  range: (0,100),}
\NormalTok{)) }

\NormalTok{\#figure((ms.display)())}
\end{Highlighting}
\end{Shaded}

\pandocbounded{\includegraphics[keepaspectratio]{https://github.com/typst/packages/raw/main/packages/preview/ionio-illustrate/0.2.0/gallery/isobulelene_epoxide.typ.png}}

There are many ways to further enhance your spectrum, please check out
the manual to find out how.

(
\href{https://github.com/typst/packages/raw/main/packages/preview/ionio-illustrate/0.2.0/\#readme-top}{back
to top} )

\subsection{Roadmap}\label{roadmap}

\begin{itemize}
\tightlist
\item
  {[}x{]} Pass style options through to the plot (tracker: \#1)
\item
  {[} {]} Better placement of text depending on plot size
\item
  {[} {]} Improve default step on axes
\item
  {[} {]} Add support for callouts that are not immediately above their
  assigned peak

  \begin{itemize}
  \tightlist
  \item
    {[} {]} Automatically detect when two annotations are too close, and
    display accordingly
  \end{itemize}
\item
  {[} {]} Move to new Typst type system (waiting on upstream)
\item
  {[} {]} Add in function for displaying skeletal structure of chemical
\item
  {[} {]} Optional second axis for absolute intensity
\item
  {[} {]} Add additional display functions

  \begin{itemize}
  \tightlist
  \item
    {[} {]} Figure out function signature for multiple data sets
  \item
    {[} {]} Overlayed and shifted
  \item
    {[} {]} Horizontal reflection

    \begin{itemize}
    \tightlist
    \item
      {[} {]} How to update existing extras?
    \end{itemize}
  \end{itemize}
\end{itemize}

See the \href{https://github.com/jamesxx/ionio-illustrate/issues}{open
issues} for a full list of proposed features (and known issues).

(
\href{https://github.com/typst/packages/raw/main/packages/preview/ionio-illustrate/0.2.0/\#readme-top}{back
to top} )

\subsection{Contributing}\label{contributing}

Contributions are what make the open source community such an amazing
place to learn, inspire, and create. Any contributions you make are
\textbf{greatly appreciated} .

If you have a suggestion that would make this better, please fork the
repo and create a pull request. You can also simply open an issue with
the tag “enhancement�. Don’t forget to give the project a star!
Thanks again!

\begin{enumerate}
\tightlist
\item
  Fork the Project
\item
  Create your Feature Branch (
  \texttt{\ git\ checkout\ -b\ feature/AmazingFeature\ } )
\item
  Commit your Changes (
  \texttt{\ git\ commit\ -m\ \textquotesingle{}Add\ some\ AmazingFeature\textquotesingle{}\ }
  )
\item
  Push to the Branch (
  \texttt{\ git\ push\ origin\ feature/AmazingFeature\ } )
\item
  Open a Pull Request
\end{enumerate}

(
\href{https://github.com/typst/packages/raw/main/packages/preview/ionio-illustrate/0.2.0/\#readme-top}{back
to top} )

\subsection{License}\label{license}

Distributed under the MIT License. See
\href{https://github.com/jamesxx/ionio-illustrate/blob/master/LICENSE}{\texttt{\ LICENSE\ }}
for more information.

(
\href{https://github.com/typst/packages/raw/main/packages/preview/ionio-illustrate/0.2.0/\#readme-top}{back
to top} )

\subsection{Gallery}\label{gallery}

\pandocbounded{\includegraphics[keepaspectratio]{https://github.com/typst/packages/raw/main/packages/preview/ionio-illustrate/0.2.0/gallery/linalool.typ.png}}

(
\href{https://github.com/typst/packages/raw/main/packages/preview/ionio-illustrate/0.2.0/\#readme-top}{back
to top} )

\subsubsection{How to add}\label{how-to-add}

Copy this into your project and use the import as
\texttt{\ ionio-illustrate\ }

\begin{verbatim}
#import "@preview/ionio-illustrate:0.2.0"
\end{verbatim}

\includesvg[width=0.16667in,height=0.16667in]{/assets/icons/16-copy.svg}

Check the docs for
\href{https://typst.app/docs/reference/scripting/\#packages}{more
information on how to import packages} .

\subsubsection{About}\label{about}

\begin{description}
\tightlist
\item[Author :]
James (Fuzzy) Swift
\item[License:]
MIT
\item[Current version:]
0.2.0
\item[Last updated:]
October 22, 2023
\item[First released:]
October 21, 2023
\item[Archive size:]
5.76 kB
\href{https://packages.typst.org/preview/ionio-illustrate-0.2.0.tar.gz}{\pandocbounded{\includesvg[keepaspectratio]{/assets/icons/16-download.svg}}}
\item[Repository:]
\href{https://github.com/JamesxX/ionio-illustrate}{GitHub}
\end{description}

\subsubsection{Where to report issues?}\label{where-to-report-issues}

This package is a project of James (Fuzzy) Swift . Report issues on
\href{https://github.com/JamesxX/ionio-illustrate}{their repository} .
You can also try to ask for help with this package on the
\href{https://forum.typst.app}{Forum} .

Please report this package to the Typst team using the
\href{https://typst.app/contact}{contact form} if you believe it is a
safety hazard or infringes upon your rights.

\phantomsection\label{versions}
\subsubsection{Version history}\label{version-history}

\begin{longtable}[]{@{}ll@{}}
\toprule\noalign{}
Version & Release Date \\
\midrule\noalign{}
\endhead
\bottomrule\noalign{}
\endlastfoot
0.2.0 & October 22, 2023 \\
\href{https://typst.app/universe/package/ionio-illustrate/0.1.0/}{0.1.0}
& October 21, 2023 \\
\end{longtable}

Typst GmbH did not create this package and cannot guarantee correct
functionality of this package or compatibility with any version of the
Typst compiler or app.
