\title{typst.app/universe/package/smooth-tmlr}

\phantomsection\label{banner}
\phantomsection\label{template-thumbnail}
\pandocbounded{\includegraphics[keepaspectratio]{https://packages.typst.org/preview/thumbnails/smooth-tmlr-0.4.0-small.webp}}

\section{smooth-tmlr}\label{smooth-tmlr}

{ 0.4.0 }

Paper template for submission to Transaction on Machine Learning
Research (TMLR)

\href{/app?template=smooth-tmlr&version=0.4.0}{Create project in app}

\phantomsection\label{readme}
\subsection{Usage}\label{usage}

You can use this template in the Typst web app by clicking \emph{Start
from template} on the dashboard and searching for
\texttt{\ smooth-tmlr\ } .

Alternatively, you can use the CLI to kick this project off using the
command

\begin{Shaded}
\begin{Highlighting}[]
\NormalTok{typst init @preview/smooth{-}tmlr}
\end{Highlighting}
\end{Shaded}

Typst will create a new directory with all the files needed to get you
started.

\subsection{Example Papers}\label{example-papers}

Here are an example paper in
\href{https://github.com/daskol/typst-templates}{LaTeX} and in
\href{https://github.com/daskol/typst-templates/\#colored-annotations}{Typst}
.

\subsection{Configuration}\label{configuration}

This template exports the \texttt{\ tmlr\ } function with the following
named arguments.

\begin{itemize}
\tightlist
\item
  \texttt{\ title\ } : The paper’s title as content.
\item
  \texttt{\ authors\ } : An array of author dictionaries. Each of the
  author dictionaries must have a name key and can have the keys
  department, organization, location, and email.
\item
  \texttt{\ keywords\ } : Publication keywords (used in PDF metadata).
\item
  \texttt{\ date\ } : Creation date (used in PDF metadata).
\item
  \texttt{\ abstract\ } : The content of a brief summary of the paper or
  none. Appears at the top under the title.
\item
  \texttt{\ bibliography\ } : The result of a call to the bibliography
  function or none. The function also accepts a single, positional
  argument for the body of the paper.
\item
  \texttt{\ appendix\ } : Content to append after bibliography section.
\item
  \texttt{\ accepted\ } : If this is set to \texttt{\ false\ } then
  anonymized ready for submission document is produced;
  \texttt{\ accepted:\ true\ } produces camera-redy version. If the
  argument is set to \texttt{\ none\ } then preprint version is produced
  (can be uploaded to arXiv).
\item
  \texttt{\ review\ } : Hypertext link to review on OpenReview.
\item
  \texttt{\ pubdate\ } : Date of publication (used only month and date).
\end{itemize}

The template will initialize your package with a sample call to the
\texttt{\ tmlr\ } function in a show rule. If you want to change an
existing project to use this template, you can add a show rule at the
top of your file.

\subsection{Issues}\label{issues}

This template is developed at
\href{https://github.com/daskol/typst-templates}{daskol/typst-templates}
repo. Please report all issues there.

\begin{itemize}
\item
  While author instruction says the all text should be in sans serif
  font Computer Modern Bright, only headers and titles are in sans font
  Computer Modern Sans and the rest of text is causal Computer Modern
  Serif (or Roman). To be precice, the original template uses Latin
  Modern, a descendant of Computer Modern. In this tempalte we stick to
  CMU (Computer Modern Unicode) font family.
\item
  In the original template paper, the word \textbf{Abstract} is of large
  font size (12pt) and bold. This affects slightly line spacing. We
  don’t know how to adjust Typst to reproduce this feature of the
  reference template but this issue does not impact a lot on visual
  appearence and layouting.
\item
  In the original template special level-3 sections like “Author
  Contributions� or “Acknowledgements� are not added to outline.
  We add them to outline as level-1 header but still render them as
  level-3 headers.
\item
  ICML-like bibliography style. In this case, the bibliography slightly
  differs from the one in the original example paper. The main
  difference is that we prefer to use author’s lastname at first place
  to search an entry faster.
\item
  In the original template a lot of vertical space is inserted before
  and after graphics and table figures. It is unclear why so much space
  are inserted. We belive that the reason is how Typst justify content
  verticaly. Nevertheless, we append a page break after “Default
  Notation� section in order to show that spacing does not differ
  visually.
\item
  Another issue is related to Typst’s inablity to produce colored
  annotation. In order to mitigte the issue, we add a script which
  modifies annotations and make them colored.

\begin{Shaded}
\begin{Highlighting}[]
\NormalTok{../colorize{-}annotations.py \textbackslash{}}
\NormalTok{    example{-}paper.typst.pdf example{-}paper{-}colored.typst.pdf}
\end{Highlighting}
\end{Shaded}

  See
  \href{https://github.com/daskol/typst-templates/\#colored-annotations}{README.md}
  for details.
\end{itemize}

\href{/app?template=smooth-tmlr&version=0.4.0}{Create project in app}

\subsubsection{How to use}\label{how-to-use}

Click the button above to create a new project using this template in
the Typst app.

You can also use the Typst CLI to start a new project on your computer
using this command:

\begin{verbatim}
typst init @preview/smooth-tmlr:0.4.0
\end{verbatim}

\includesvg[width=0.16667in,height=0.16667in]{/assets/icons/16-copy.svg}

\subsubsection{About}\label{about}

\begin{description}
\tightlist
\item[Author :]
\href{mailto:d.bershatsky2@skoltech.ru}{Daniel Bershatsky}
\item[License:]
MIT
\item[Current version:]
0.4.0
\item[Last updated:]
April 29, 2024
\item[First released:]
March 28, 2024
\item[Minimum Typst version:]
0.10.0
\item[Archive size:]
21.3 kB
\href{https://packages.typst.org/preview/smooth-tmlr-0.4.0.tar.gz}{\pandocbounded{\includesvg[keepaspectratio]{/assets/icons/16-download.svg}}}
\item[Repository:]
\href{https://github.com/daskol/typst-templates}{GitHub}
\item[Discipline s :]
\begin{itemize}
\tightlist
\item[]
\item
  \href{https://typst.app/universe/search/?discipline=computer-science}{Computer
  Science}
\item
  \href{https://typst.app/universe/search/?discipline=mathematics}{Mathematics}
\end{itemize}
\item[Categor y :]
\begin{itemize}
\tightlist
\item[]
\item
  \pandocbounded{\includesvg[keepaspectratio]{/assets/icons/16-atom.svg}}
  \href{https://typst.app/universe/search/?category=paper}{Paper}
\end{itemize}
\end{description}

\subsubsection{Where to report issues?}\label{where-to-report-issues}

This template is a project of Daniel Bershatsky . Report issues on
\href{https://github.com/daskol/typst-templates}{their repository} . You
can also try to ask for help with this template on the
\href{https://forum.typst.app}{Forum} .

Please report this template to the Typst team using the
\href{https://typst.app/contact}{contact form} if you believe it is a
safety hazard or infringes upon your rights.

\phantomsection\label{versions}
\subsubsection{Version history}\label{version-history}

\begin{longtable}[]{@{}ll@{}}
\toprule\noalign{}
Version & Release Date \\
\midrule\noalign{}
\endhead
\bottomrule\noalign{}
\endlastfoot
0.4.0 & April 29, 2024 \\
\href{https://typst.app/universe/package/smooth-tmlr/0.3.0/}{0.3.0} &
March 28, 2024 \\
\end{longtable}

Typst GmbH did not create this template and cannot guarantee correct
functionality of this template or compatibility with any version of the
Typst compiler or app.
