\title{typst.app/universe/package/titleize}

\phantomsection\label{banner}
\section{titleize}\label{titleize}

{ 0.1.1 }

Turn strings into title case

\phantomsection\label{readme}
Small wrapper around the
\href{https://crates.io/crates/titlecase}{titlecase} library to convert
text to title case. It follows the
\href{https://daringfireball.net/2008/05/title_case}{rules defined by
John Gruber} . For more details, refer to the library.

\texttt{\ titlecase\ } applies a show rule, that by default transforms
every string of at least four characters. This limit can be changed with
the \texttt{\ limit\ } parameter. Especially with equations, the results
can be a bit unpredictable, so proceed with care.

\begin{Shaded}
\begin{Highlighting}[]
\NormalTok{\#import "@preview/titleize:0.1.1": titlecase}

\NormalTok{\#for s in (}
\NormalTok{  "Being productive on linux",}
\NormalTok{  "Finding an alternative to Mac OS X — part 2",}
\NormalTok{  [an example with small words and sub{-}phrases: "the example"],}
\NormalTok{) [}
\NormalTok{  \#s =\textgreater{} \#titlecase(s) \textbackslash{}}
\NormalTok{]}

\NormalTok{\#let debug{-}print(x) = \{}
\NormalTok{  if type(x) == content \{}
\NormalTok{    let fields = x.fields()}
\NormalTok{    let func = x.func()}
\NormalTok{    [}
\NormalTok{      \#repr(func)}
\NormalTok{      \#for (k, v) in fields [}
\NormalTok{        {-} \#k: \#debug{-}print(v)}
\NormalTok{      ]}
\NormalTok{    ]}
\NormalTok{  \} else \{}
\NormalTok{    if type(x) == array [}
\NormalTok{      array}
\NormalTok{      \#for y in x [}
\NormalTok{        {-} \#debug{-}print(y)}
\NormalTok{      ]}
\NormalTok{    ] else [}
\NormalTok{      \#repr(type(x)) (\#repr(x))}
\NormalTok{    ]}
\NormalTok{  \}}
\NormalTok{\}}

\NormalTok{\#show: titlecase}

\NormalTok{= This is a test, even with math $a = b + c$}

\NormalTok{In math, text can appear in various places:}

\NormalTok{$}
\NormalTok{  a\_"for example in a subscript" \&= "or in a longer text" \textbackslash{}}
\NormalTok{  f(x) \&= sin(x)}
\NormalTok{$}
\end{Highlighting}
\end{Shaded}

\pandocbounded{\includegraphics[keepaspectratio]{https://github.com/typst/packages/raw/main/packages/preview/titleize/0.1.1/example.png}}

\subsubsection{How to add}\label{how-to-add}

Copy this into your project and use the import as \texttt{\ titleize\ }

\begin{verbatim}
#import "@preview/titleize:0.1.1"
\end{verbatim}

\includesvg[width=0.16667in,height=0.16667in]{/assets/icons/16-copy.svg}

Check the docs for
\href{https://typst.app/docs/reference/scripting/\#packages}{more
information on how to import packages} .

\subsubsection{About}\label{about}

\begin{description}
\tightlist
\item[Author :]
\href{mailto:mail@solidtux.de}{Daniel Hauck}
\item[License:]
MIT
\item[Current version:]
0.1.1
\item[Last updated:]
October 15, 2024
\item[First released:]
October 7, 2024
\item[Archive size:]
253 kB
\href{https://packages.typst.org/preview/titleize-0.1.1.tar.gz}{\pandocbounded{\includesvg[keepaspectratio]{/assets/icons/16-download.svg}}}
\item[Repository:]
\href{https://gitlab.com/SolidTux/titleize}{GitLab}
\item[Categor ies :]
\begin{itemize}
\tightlist
\item[]
\item
  \pandocbounded{\includesvg[keepaspectratio]{/assets/icons/16-text.svg}}
  \href{https://typst.app/universe/search/?category=text}{Text}
\item
  \pandocbounded{\includesvg[keepaspectratio]{/assets/icons/16-code.svg}}
  \href{https://typst.app/universe/search/?category=scripting}{Scripting}
\item
  \pandocbounded{\includesvg[keepaspectratio]{/assets/icons/16-hammer.svg}}
  \href{https://typst.app/universe/search/?category=utility}{Utility}
\end{itemize}
\end{description}

\subsubsection{Where to report issues?}\label{where-to-report-issues}

This package is a project of Daniel Hauck . Report issues on
\href{https://gitlab.com/SolidTux/titleize}{their repository} . You can
also try to ask for help with this package on the
\href{https://forum.typst.app}{Forum} .

Please report this package to the Typst team using the
\href{https://typst.app/contact}{contact form} if you believe it is a
safety hazard or infringes upon your rights.

\phantomsection\label{versions}
\subsubsection{Version history}\label{version-history}

\begin{longtable}[]{@{}ll@{}}
\toprule\noalign{}
Version & Release Date \\
\midrule\noalign{}
\endhead
\bottomrule\noalign{}
\endlastfoot
0.1.1 & October 15, 2024 \\
\href{https://typst.app/universe/package/titleize/0.1.0/}{0.1.0} &
October 7, 2024 \\
\end{longtable}

Typst GmbH did not create this package and cannot guarantee correct
functionality of this package or compatibility with any version of the
Typst compiler or app.
