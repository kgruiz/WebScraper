\title{typst.app/universe/package/badformer}

\phantomsection\label{banner}
\phantomsection\label{template-thumbnail}
\pandocbounded{\includegraphics[keepaspectratio]{https://packages.typst.org/preview/thumbnails/badformer-0.1.0-small.webp}}

\section{badformer}\label{badformer}

{ 0.1.0 }

Retro-gaming in Typst. Reach the goal and complete the mission.

\href{/app?template=badformer&version=0.1.0}{Create project in app}

\phantomsection\label{readme}
Reach the goal in this retro-inspired wireframing platformer. Play in 3
dimensions and compete for the lowest number of steps to win!

This small game is playable in the Typst editor and best enjoyed with
the web app or \texttt{\ typst\ watch\ } . It was first released for the
24 Days to Christmas campaign in winter of 2023.

\subsection{Usage}\label{usage}

You can use this template in the Typst web app by clicking “Start from
template� on the dashboard and searching for \texttt{\ badformer\ } .

Alternatively, you can use the CLI to kick this project off using the
command

\begin{verbatim}
typst init @preview/badformer
\end{verbatim}

Typst will create a new directory with all the files needed to get you
started.

Move with WASD and jump with space. You can also display a minimap by
pressing E.

\subsection{Configuration}\label{configuration}

This template exports the \texttt{\ game\ } function, which accepts a
positional argument for the game input.

The template will initialize your package with a sample call to the
\texttt{\ game\ } function in a show rule. If you want to change an
existing project to use this template, you can add a show rule like this
at the top of your file:

\begin{Shaded}
\begin{Highlighting}[]
\NormalTok{\#import "@preview/badformer:0.1.0": game}
\NormalTok{\#show: game(read("main.typ"))}

\NormalTok{// Move with WASD and jump with space.}
\end{Highlighting}
\end{Shaded}

\href{/app?template=badformer&version=0.1.0}{Create project in app}

\subsubsection{How to use}\label{how-to-use}

Click the button above to create a new project using this template in
the Typst app.

You can also use the Typst CLI to start a new project on your computer
using this command:

\begin{verbatim}
typst init @preview/badformer:0.1.0
\end{verbatim}

\includesvg[width=0.16667in,height=0.16667in]{/assets/icons/16-copy.svg}

\subsubsection{About}\label{about}

\begin{description}
\tightlist
\item[Author :]
\href{https://typst.app}{Typst GmbH}
\item[License:]
MIT-0
\item[Current version:]
0.1.0
\item[Last updated:]
March 6, 2024
\item[First released:]
March 6, 2024
\item[Minimum Typst version:]
0.10.0
\item[Archive size:]
5.43 kB
\href{https://packages.typst.org/preview/badformer-0.1.0.tar.gz}{\pandocbounded{\includesvg[keepaspectratio]{/assets/icons/16-download.svg}}}
\item[Repository:]
\href{https://github.com/typst/templates}{GitHub}
\item[Categor y :]
\begin{itemize}
\tightlist
\item[]
\item
  \pandocbounded{\includesvg[keepaspectratio]{/assets/icons/16-smile.svg}}
  \href{https://typst.app/universe/search/?category=fun}{Fun}
\end{itemize}
\end{description}

\subsubsection{Where to report issues?}\label{where-to-report-issues}

This template is a project of Typst GmbH . Report issues on
\href{https://github.com/typst/templates}{their repository} . You can
also try to ask for help with this template on the
\href{https://forum.typst.app}{Forum} .

\phantomsection\label{versions}
\subsubsection{Version history}\label{version-history}

\begin{longtable}[]{@{}ll@{}}
\toprule\noalign{}
Version & Release Date \\
\midrule\noalign{}
\endhead
\bottomrule\noalign{}
\endlastfoot
0.1.0 & March 6, 2024 \\
\end{longtable}
