\title{typst.app/universe/package/unichar}

\phantomsection\label{banner}
\section{unichar}\label{unichar}

{ 0.3.0 }

A partial port of the Unicode Character Database.

\phantomsection\label{readme}
This package ports part of the
\href{https://www.unicode.org/reports/tr44/}{Unicode Character Database}
to Typst. Notably, it includes information from
\href{https://unicode.org/reports/tr44/\#UnicodeData.txt}{UnicodeData.txt}
and \href{https://unicode.org/reports/tr44/\#Blocks.txt}{Blocks.txt} .

\subsection{Usage}\label{usage}

This package defines a single function: \texttt{\ codepoint\ } . It lets
you get the information related to a specific codepoint. The codepoint
can be specified as a string containing a single character, or with its
value.

\begin{Shaded}
\begin{Highlighting}[]
\NormalTok{\#codepoint("√").name \textbackslash{}}
\NormalTok{\#codepoint(sym.times).block.name \textbackslash{}}
\NormalTok{\#codepoint(0x00C9).general{-}category \textbackslash{}}
\NormalTok{\#codepoint(sym.eq).math{-}class}
\end{Highlighting}
\end{Shaded}

\pandocbounded{\includesvg[keepaspectratio]{https://github.com/typst/packages/raw/main/packages/preview/unichar/0.3.0/examples/example-1.svg}}

You can display a codepoint in the style of
\href{https://en.wikipedia.org/wiki/Template:Unichar}{Template:Unichar}
using the \texttt{\ show\ } entry:

\begin{Shaded}
\begin{Highlighting}[]
\NormalTok{\#codepoint("¤").show \textbackslash{}}
\NormalTok{\#codepoint(sym.copyright).show \textbackslash{}}
\NormalTok{\#codepoint(0x1249).show \textbackslash{}}
\NormalTok{\#codepoint(0x100000).show}
\end{Highlighting}
\end{Shaded}

\pandocbounded{\includesvg[keepaspectratio]{https://github.com/typst/packages/raw/main/packages/preview/unichar/0.3.0/examples/example-2.svg}}

\subsection{Changelog}\label{changelog}

\subsubsection{Version 0.3.0}\label{version-0.3.0}

\begin{itemize}
\item
  Add \texttt{\ math-class\ } attribute to codepoints.

  \begin{itemize}
  \tightlist
  \item
    Some codepoints have their math class overridden by Typst. This is
    the Unicode math class, not the one used by Typst.
  \end{itemize}
\item
  The \texttt{\ id\ } of codepoints now returns a string without the
  \texttt{\ "U+"\ } prefix.
\end{itemize}

\subsubsection{Version 0.2.0}\label{version-0.2.0}

\begin{itemize}
\item
  Codepoints now have an \texttt{\ id\ } attribute which is its
  corresponding “U+xxxx� string.
\item
  The \texttt{\ block\ } attribute of a codepoint now contains a
  \texttt{\ name\ } , a \texttt{\ start\ } , and a \texttt{\ size\ } .
\item
  Fix an issue that made some codepoints cause a panic.
\item
  Include data from NameAlias.txt.
\end{itemize}

\subsubsection{Version 0.1.0}\label{version-0.1.0}

\begin{itemize}
\tightlist
\item
  Add the \texttt{\ codepoint\ } function.
\end{itemize}

\subsubsection{How to add}\label{how-to-add}

Copy this into your project and use the import as \texttt{\ unichar\ }

\begin{verbatim}
#import "@preview/unichar:0.3.0"
\end{verbatim}

\includesvg[width=0.16667in,height=0.16667in]{/assets/icons/16-copy.svg}

Check the docs for
\href{https://typst.app/docs/reference/scripting/\#packages}{more
information on how to import packages} .

\subsubsection{About}\label{about}

\begin{description}
\tightlist
\item[Author :]
\href{https://github.com/MDLC01}{Malo}
\item[License:]
MIT AND Unicode-3.0
\item[Current version:]
0.3.0
\item[Last updated:]
September 19, 2024
\item[First released:]
September 14, 2024
\item[Minimum Typst version:]
0.11.0
\item[Archive size:]
202 kB
\href{https://packages.typst.org/preview/unichar-0.3.0.tar.gz}{\pandocbounded{\includesvg[keepaspectratio]{/assets/icons/16-download.svg}}}
\item[Repository:]
\href{https://github.com/MDLC01/unichar}{GitHub}
\item[Categor ies :]
\begin{itemize}
\tightlist
\item[]
\item
  \pandocbounded{\includesvg[keepaspectratio]{/assets/icons/16-code.svg}}
  \href{https://typst.app/universe/search/?category=scripting}{Scripting}
\item
  \pandocbounded{\includesvg[keepaspectratio]{/assets/icons/16-integration.svg}}
  \href{https://typst.app/universe/search/?category=integration}{Integration}
\end{itemize}
\end{description}

\subsubsection{Where to report issues?}\label{where-to-report-issues}

This package is a project of Malo . Report issues on
\href{https://github.com/MDLC01/unichar}{their repository} . You can
also try to ask for help with this package on the
\href{https://forum.typst.app}{Forum} .

Please report this package to the Typst team using the
\href{https://typst.app/contact}{contact form} if you believe it is a
safety hazard or infringes upon your rights.

\phantomsection\label{versions}
\subsubsection{Version history}\label{version-history}

\begin{longtable}[]{@{}ll@{}}
\toprule\noalign{}
Version & Release Date \\
\midrule\noalign{}
\endhead
\bottomrule\noalign{}
\endlastfoot
0.3.0 & September 19, 2024 \\
\href{https://typst.app/universe/package/unichar/0.2.0/}{0.2.0} &
September 15, 2024 \\
\href{https://typst.app/universe/package/unichar/0.1.0/}{0.1.0} &
September 14, 2024 \\
\end{longtable}

Typst GmbH did not create this package and cannot guarantee correct
functionality of this package or compatibility with any version of the
Typst compiler or app.
