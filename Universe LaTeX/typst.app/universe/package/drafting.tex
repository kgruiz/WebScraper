\title{typst.app/universe/package/drafting}

\phantomsection\label{banner}
\section{drafting}\label{drafting}

{ 0.2.1 }

Helpful functions for content positioning and margin comments/notes

{ } Featured Package

\phantomsection\label{readme}
\subsection{Setup}\label{setup}

\texttt{\ drafting\ } exists in the official typst package repository,
so the recommended approach is to import it from the
\texttt{\ preview\ } namespace:

\begin{Shaded}
\begin{Highlighting}[]
\NormalTok{\#import "@preview/drafting:0.2.1"}
\end{Highlighting}
\end{Shaded}

Margin notes cannot lay themselves out correctly until they know your
page size and margins. By default, they occupy nearly the entirety of
the left or right margin, but you can provide explicit left/right bounds
if desired:

\begin{Shaded}
\begin{Highlighting}[]
\NormalTok{// Example:}
\NormalTok{// Default margin in typst is 2.5cm, but we want to use 2cm}
\NormalTok{// On the left}
\NormalTok{\#set{-}page{-}properties(margin{-}left: 2cm)}
\end{Highlighting}
\end{Shaded}

\subsection{The basics}\label{the-basics}

\begin{Shaded}
\begin{Highlighting}[]
\NormalTok{\#lorem(20)}
\NormalTok{\#margin{-}note(side: left)[Hello, world!]}
\NormalTok{\#lorem(10)}
\NormalTok{\#margin{-}note[Hello from the other side]}
\NormalTok{\#margin{-}note[When notes are about to overlap, they\textquotesingle{}re automatically shifted]}
\NormalTok{\#margin{-}note(stroke: aqua + 3pt)[To avoid collision]}
\NormalTok{\#lorem(25)}
\NormalTok{\#margin{-}note(stroke: green, side: left)[You can provide two positional arguments if you want to highlight a phrase associated with your note.][The first is text which should be inline{-}noted, and the second is the standard margin note.]}

\NormalTok{\#let caution{-}rect = rect.with(inset: 1em, radius: 0.5em, fill: orange.lighten(80\%))}
\NormalTok{\#inline{-}note(rect: caution{-}rect)[}
\NormalTok{  Be aware that \textasciigrave{}typst\textasciigrave{} will complain when 4 notes overlap, and stop automatically avoiding collisions when 5 or more notes}
\NormalTok{  overlap. This is because the compiler stops attempting to reposition notes after a few attempts}
\NormalTok{  (initial layout + adjustment for each note).}

\NormalTok{  You can manually adjust the position of notes with \textasciigrave{}dy\textasciigrave{} to silence the warning.}
\NormalTok{]}
\end{Highlighting}
\end{Shaded}

\pandocbounded{\includegraphics[keepaspectratio]{https://www.github.com/ntjess/typst-drafting/raw/v0.2.1/assets/example-1.png}}

\subsection{Adjusting the default
style}\label{adjusting-the-default-style}

All function defaults are customizable through updating the module
state:

\begin{Shaded}
\begin{Highlighting}[]
\NormalTok{\#lorem(14) \#margin{-}note[Default style]}
\NormalTok{\#lorem(10)}
\NormalTok{\#set{-}margin{-}note{-}defaults(stroke: orange, side: left)}
\NormalTok{\#margin{-}note[Updated style]}
\NormalTok{\#lorem(10)}
\end{Highlighting}
\end{Shaded}

\pandocbounded{\includegraphics[keepaspectratio]{https://www.github.com/ntjess/typst-drafting/raw/v0.2.1/assets/example-2.png}}

Even deeper customization is possible by overriding the default
\texttt{\ rect\ } :

\begin{Shaded}
\begin{Highlighting}[]
\NormalTok{\#import "@preview/colorful{-}boxes:1.1.0": stickybox}

\NormalTok{\#let default{-}rect(stroke: none, fill: none, width: 0pt, content) = \{}
\NormalTok{  set text(0.9em)}
\NormalTok{  stickybox(rotation: 30deg, width: width/1.5, content)}
\NormalTok{\}}
\NormalTok{\#set{-}margin{-}note{-}defaults(rect: default{-}rect, stroke: none, side: right)}

\NormalTok{\#lorem(20)}
\NormalTok{\#margin{-}note(dy: {-}5em)[Why not use sticky notes in the margin?]}

\NormalTok{// Undo changes from this example}
\NormalTok{\#set{-}margin{-}note{-}defaults(rect: rect, stroke: red)}
\end{Highlighting}
\end{Shaded}

\pandocbounded{\includegraphics[keepaspectratio]{https://www.github.com/ntjess/typst-drafting/raw/v0.2.1/assets/example-3.png}}

\subsection{Multiple document
reviewers}\label{multiple-document-reviewers}

\begin{Shaded}
\begin{Highlighting}[]
\NormalTok{\#let reviewer{-}a = margin{-}note.with(stroke: blue)}
\NormalTok{\#let reviewer{-}b = margin{-}note.with(stroke: purple)}
\NormalTok{\#lorem(10)}
\NormalTok{\#reviewer{-}a[Comment from reviewer A]}
\NormalTok{\#lorem(5)}
\NormalTok{\#reviewer{-}b(side: left)[Reviewer B comment]}
\NormalTok{\#lorem(10)}
\end{Highlighting}
\end{Shaded}

\pandocbounded{\includegraphics[keepaspectratio]{https://www.github.com/ntjess/typst-drafting/raw/v0.2.1/assets/example-4.png}}

\subsection{Inline Notes}\label{inline-notes}

\begin{Shaded}
\begin{Highlighting}[]
\NormalTok{\#lorem(10)}
\NormalTok{\#inline{-}note[The default inline note will split the paragraph at its location]}
\NormalTok{\#lorem(10)}
\NormalTok{\#inline{-}note(par{-}break: false, stroke: (paint: orange, dash: "dashed"))[}
\NormalTok{  But you can specify \textasciigrave{}par{-}break: false\textasciigrave{} to prevent this}
\NormalTok{]}
\NormalTok{\#lorem(10)}
\end{Highlighting}
\end{Shaded}

\pandocbounded{\includegraphics[keepaspectratio]{https://www.github.com/ntjess/typst-drafting/raw/v0.2.1/assets/example-5.png}}

\subsection{Hiding notes for print
preview}\label{hiding-notes-for-print-preview}

\begin{Shaded}
\begin{Highlighting}[]
\NormalTok{\#set{-}margin{-}note{-}defaults(hidden: true)}

\NormalTok{\#lorem(20)}
\NormalTok{\#margin{-}note[This will respect the global "hidden" state]}
\NormalTok{\#margin{-}note(hidden: false, dy: {-}2.5em)[This note will never be hidden]}
\NormalTok{// Undo these changes}
\NormalTok{\#set{-}margin{-}note{-}defaults(hidden: false)}
\end{Highlighting}
\end{Shaded}

\pandocbounded{\includegraphics[keepaspectratio]{https://www.github.com/ntjess/typst-drafting/raw/v0.2.1/assets/example-6.png}}

\subsection{Precise placement: rule
grid}\label{precise-placement-rule-grid}

Need to measure space for fine-tuned positioning? You can use
\texttt{\ rule-grid\ } to cross-hatch the page with rule lines:

\begin{Shaded}
\begin{Highlighting}[]
\NormalTok{\#rule{-}grid(width: 10cm, height: 3cm, spacing: 20pt)}
\NormalTok{\#place(}
\NormalTok{  dx: 180pt,}
\NormalTok{  dy: 40pt,}
\NormalTok{  rect(fill: white, stroke: red, width: 1in, "This will originate at (180pt, 40pt)")}
\NormalTok{)}

\NormalTok{// Optionally specify divisions of the smallest dimension to automatically calculate}
\NormalTok{// spacing}
\NormalTok{\#rule{-}grid(dx: 10cm + 3em, width: 3cm, height: 1.2cm, divisions: 5, square: true,  stroke: green)}

\NormalTok{// The rule grid doesn\textquotesingle{}t take up space, so add it explicitly}
\NormalTok{\#v(3cm + 1em)}
\end{Highlighting}
\end{Shaded}

\pandocbounded{\includegraphics[keepaspectratio]{https://www.github.com/ntjess/typst-drafting/raw/v0.2.1/assets/example-7.png}}

\subsection{Absolute positioning}\label{absolute-positioning}

What about absolutely positioning something regardless of margin and
relative location? \texttt{\ absolute-place\ } is your friend. You can
put content anywhere:

\begin{Shaded}
\begin{Highlighting}[]
\NormalTok{\#context \{}
\NormalTok{  let (dx, dy) = (here().position().x, here().position().y)}
\NormalTok{  let content{-}str = (}
\NormalTok{    "This absolutely{-}placed box will originate at (" + repr(dx) + ", " + repr(dy) + ")"}
\NormalTok{    + " in page coordinates"}
\NormalTok{  )}
\NormalTok{  absolute{-}place(}
\NormalTok{    dx: dx, dy: dy,}
\NormalTok{    rect(}
\NormalTok{      fill: green.lighten(60\%),}
\NormalTok{      radius: 0.5em,}
\NormalTok{      width: 2.5in,}
\NormalTok{      height: 0.5in,}
\NormalTok{      [\#align(center + horizon, content{-}str)]}
\NormalTok{    )}
\NormalTok{  )}
\NormalTok{\}}
\NormalTok{\#v(0.5in)}
\end{Highlighting}
\end{Shaded}

\pandocbounded{\includegraphics[keepaspectratio]{https://www.github.com/ntjess/typst-drafting/raw/v0.2.1/assets/example-8.png}}

The “rule-grid� also supports absolute placement at the top-left of
the page by passing \texttt{\ relative:\ false\ } . This is helpful for
“rule“-ing the whole page.

\subsubsection{How to add}\label{how-to-add}

Copy this into your project and use the import as \texttt{\ drafting\ }

\begin{verbatim}
#import "@preview/drafting:0.2.1"
\end{verbatim}

\includesvg[width=0.16667in,height=0.16667in]{/assets/icons/16-copy.svg}

Check the docs for
\href{https://typst.app/docs/reference/scripting/\#packages}{more
information on how to import packages} .

\subsubsection{About}\label{about}

\begin{description}
\tightlist
\item[Author :]
Nathan Jessurun
\item[License:]
Unlicense
\item[Current version:]
0.2.1
\item[Last updated:]
November 25, 2024
\item[First released:]
September 3, 2023
\item[Minimum Typst version:]
0.12.0
\item[Archive size:]
7.98 kB
\href{https://packages.typst.org/preview/drafting-0.2.1.tar.gz}{\pandocbounded{\includesvg[keepaspectratio]{/assets/icons/16-download.svg}}}
\item[Repository:]
\href{https://github.com/ntjess/typst-drafting}{GitHub}
\item[Categor ies :]
\begin{itemize}
\tightlist
\item[]
\item
  \pandocbounded{\includesvg[keepaspectratio]{/assets/icons/16-layout.svg}}
  \href{https://typst.app/universe/search/?category=layout}{Layout}
\item
  \pandocbounded{\includesvg[keepaspectratio]{/assets/icons/16-hammer.svg}}
  \href{https://typst.app/universe/search/?category=utility}{Utility}
\end{itemize}
\end{description}

\subsubsection{Where to report issues?}\label{where-to-report-issues}

This package is a project of Nathan Jessurun . Report issues on
\href{https://github.com/ntjess/typst-drafting}{their repository} . You
can also try to ask for help with this package on the
\href{https://forum.typst.app}{Forum} .

Please report this package to the Typst team using the
\href{https://typst.app/contact}{contact form} if you believe it is a
safety hazard or infringes upon your rights.

\phantomsection\label{versions}
\subsubsection{Version history}\label{version-history}

\begin{longtable}[]{@{}ll@{}}
\toprule\noalign{}
Version & Release Date \\
\midrule\noalign{}
\endhead
\bottomrule\noalign{}
\endlastfoot
0.2.1 & November 25, 2024 \\
\href{https://typst.app/universe/package/drafting/0.2.0/}{0.2.0} & March
16, 2024 \\
\href{https://typst.app/universe/package/drafting/0.1.2/}{0.1.2} &
December 11, 2023 \\
\href{https://typst.app/universe/package/drafting/0.1.1/}{0.1.1} &
September 11, 2023 \\
\href{https://typst.app/universe/package/drafting/0.1.0/}{0.1.0} &
September 3, 2023 \\
\end{longtable}

Typst GmbH did not create this package and cannot guarantee correct
functionality of this package or compatibility with any version of the
Typst compiler or app.
