\title{typst.app/universe/package/tidy}

\phantomsection\label{banner}
\section{tidy}\label{tidy}

{ 0.3.0 }

Documentation generator for Typst code in Typst.

{ } Featured Package

\phantomsection\label{readme}
\emph{Keep it tidy.}

\href{https://typst.app/universe/package/tidy}{\pandocbounded{\includegraphics[keepaspectratio]{https://img.shields.io/badge/dynamic/toml?url=https\%3A\%2F\%2Fraw.githubusercontent.com\%2FMc-Zen\%2Ftidy\%2Fmain\%2Ftypst.toml&query=\%24.package.version&prefix=v&logo=typst&label=package&color=239DAD}}}
\href{https://github.com/Mc-Zen/tidy/blob/main/LICENSE}{\pandocbounded{\includegraphics[keepaspectratio]{https://img.shields.io/badge/license-MIT-blue}}}
\href{https://github.com/Mc-Zen/tidy/releases/download/v0.3.0/tidy-guide.pdf}{\pandocbounded{\includegraphics[keepaspectratio]{https://img.shields.io/badge/manual-.pdf-purple}}}

\textbf{tidy} is a package that generates documentation directly in
\href{https://typst.app/}{Typst} for your Typst modules. It parses
docstring comments similar to javadoc and co. and can be used to easily
build a beautiful reference section for the parsed module. Within the
docstring you may use (almost) any Typst syntax âˆ' so markup, equations
and even figures are no problem!

Features:

\begin{itemize}
\tightlist
\item
  \textbf{Customizable} output styles.
\item
  Automatically
  \href{https://github.com/typst/packages/raw/main/packages/preview/tidy/0.3.0/\#example}{\textbf{render
  code examples}} .
\item
  \textbf{Annotate types} of parameters and return values.
\item
  Automatically read off default values for named parameters.
\item
  \href{https://github.com/typst/packages/raw/main/packages/preview/tidy/0.3.0/\#generate-a-help-command-for-you-package}{\textbf{Help}
  feature} for your package.
\item
  \href{https://github.com/typst/packages/raw/main/packages/preview/tidy/0.3.0/\#docstring-tests}{Docstring
  tests} .
\end{itemize}

The
\href{https://github.com/Mc-Zen/tidy/releases/download/v0.3.0/tidy-guide.pdf}{guide}
fully describes the usage of this module and defines the format for the
docstrings.

\subsection{Usage}\label{usage}

Using \texttt{\ tidy\ } is as simple as writing some docstrings and
calling:

\begin{Shaded}
\begin{Highlighting}[]
\NormalTok{\#import "@preview/tidy:0.3.0"}

\NormalTok{\#let docs = tidy.parse{-}module(read("my{-}module.typ"))}
\NormalTok{\#tidy.show{-}module(docs, style: tidy.styles.default)}
\end{Highlighting}
\end{Shaded}

The available predefined styles are currenty
\texttt{\ tidy.styles.default\ } and \texttt{\ tidy.styles.minimal\ } .
Custom styles can be added by hand (take a look at the
\href{https://github.com/Mc-Zen/tidy/releases/download/v0.3.0/tidy-guide.pdf}{guide}
).

\subsection{Example}\label{example}

A full example on how to use this module for your own package (maybe
even consisting of multiple files) can be found at
\href{https://github.com/Mc-Zen/tidy/tree/main/examples}{examples} .

\begin{Shaded}
\begin{Highlighting}[]
\NormalTok{/// This function computes the cardinal sine, $sinc(x)=sin(x)/x$. }
\NormalTok{///}
\NormalTok{/// \#example(\textasciigrave{}\#sinc(0)\textasciigrave{}, mode: "markup")}
\NormalTok{///}
\NormalTok{/// {-} x (int, float): The argument for the cardinal sine function. }
\NormalTok{/// {-}\textgreater{} float}
\NormalTok{\#let sinc(x) = if x == 0 \{1\} else \{calc.sin(x) / x\}}
\end{Highlighting}
\end{Shaded}

\textbf{tidy} turns this into:

\subsubsection{\texorpdfstring{\protect\pandocbounded{\includesvg[keepaspectratio]{https://github.com/typst/packages/raw/main/packages/preview/tidy/0.3.0/docs/images/sincx-docs.svg}}}{Tidy example output}}\label{tidy-example-output}

\subsection{Access user-defined functions and
images}\label{access-user-defined-functions-and-images}

The code in the docstrings is evaluated via \texttt{\ eval()\ } . In
order to access user-defined functions and images, you can make use of
the \texttt{\ scope\ } argument of \texttt{\ tidy.parse-module()\ } :

\begin{Shaded}
\begin{Highlighting}[]
\NormalTok{\#\{}
\NormalTok{    import "my{-}module.typ"}
\NormalTok{    let module = tidy.parse{-}module(read("my{-}module.typ"))}
\NormalTok{    let an{-}image = image("img.png")}
\NormalTok{    tidy.show{-}module(}
\NormalTok{        module,}
\NormalTok{        style: tidy.styles.default,}
\NormalTok{        scope: (my{-}module: my{-}module, img: an{-}image)}
\NormalTok{    )}
\NormalTok{\}}
\end{Highlighting}
\end{Shaded}

The docstrings in \texttt{\ my-module.typ\ } may now access the image
with \texttt{\ \#img\ } and can call any function or variable from
\texttt{\ my-module\ } in the style of
\texttt{\ \#my-module.my-function()\ } . This makes rendering examples
right in the docstrings as easy as a breeze!

\subsection{Generate a help command for you
package}\label{generate-a-help-command-for-you-package}

With \textbf{tidy} , you can add a help command to you package that
allows users to obtain the documentation of a specific definition or
parameter right in the document. This is similar to CLI-style help
commands. If you have already written docstrings for your package, it is
quite low-effort to add this feature. Once set up, the end-user can use
it like this:

\begin{Shaded}
\begin{Highlighting}[]
\NormalTok{// happily coding, but how do I use this one complex function again?}

\NormalTok{\#mypackage.help("func")}
\NormalTok{\#mypackage.help("func(param1)") // print only parameter description of param1}
\end{Highlighting}
\end{Shaded}

This will print the documentation of \texttt{\ func\ } directly into the
document â€'' no need to look it up in a manual. Read up in the
\href{https://github.com/Mc-Zen/tidy/releases/download/v0.3.0/tidy-guide.pdf}{guide}
for setup instructions.

\subsection{Docstring tests}\label{docstring-tests}

It is possible to add simple docstring tests â€'' assertions that will
be run when the documentation is generated. This is useful if you want
to keep small tests and documentation in one place.

\begin{Shaded}
\begin{Highlighting}[]
\NormalTok{/// \#test(}
\NormalTok{///   \textasciigrave{}num.my{-}square(2) == 4\textasciigrave{},}
\NormalTok{///   \textasciigrave{}num.my{-}square(4) == 16\textasciigrave{},}
\NormalTok{/// )}
\NormalTok{\#let my{-}square(n) = n * n}
\end{Highlighting}
\end{Shaded}

With the short-hand syntax, a unfulfilled assertion will even print the
line number of the failed test:

\begin{Shaded}
\begin{Highlighting}[]
\NormalTok{/// \textgreater{}\textgreater{}\textgreater{} my{-}square(2) == 4}
\NormalTok{/// \textgreater{}\textgreater{}\textgreater{} my{-}square(4) == 16}
\NormalTok{\#let my{-}square(n) = n * n}
\end{Highlighting}
\end{Shaded}

A few test assertion functions are available to improve readability,
simplicity, and error messages. Currently, these are
\texttt{\ eq(a,\ b)\ } for equality tests, \texttt{\ ne(a,\ b)\ } for
inequality tests and \texttt{\ approx(a,\ b,\ eps:\ 1e-10)\ } for
floating point comparisons. These assertion helper functions are always
available within docstring tests (with both \texttt{\ test()\ } and
\texttt{\ \textgreater{}\textgreater{}\textgreater{}\ } syntax).

\subsection{Changelog}\label{changelog}

\subsubsection{v0.3.0}\label{v0.3.0}

\begin{itemize}
\tightlist
\item
  New features:

  \begin{itemize}
  \tightlist
  \item
    Help feature.
  \item
    \texttt{\ preamble\ } option for examples (e.g., to add
    \texttt{\ import\ } statements).
  \item
    more options for \texttt{\ show-module\ } :
    \texttt{\ omit-private-definitions\ } ,
    \texttt{\ omit-private-parameters\ } ,
    \texttt{\ enable-cross-references\ } , \texttt{\ local-names\ } (for
    configuring language-specific strings).
  \end{itemize}
\item
  Improvements:

  \begin{itemize}
  \tightlist
  \item
    Allow using \texttt{\ show-example()\ } as standalone.
  \item
    Updated type names that changed with Typst 0.8.0, e.g., integer
    -\textgreater{} int.
  \end{itemize}
\item
  Fixes:

  \begin{itemize}
  \tightlist
  \item
    allow examples with ratio widths if \texttt{\ scale-preview\ } is
    not \texttt{\ auto\ } .
  \item
    \texttt{\ show-outline\ }
  \item
    explicitly use \texttt{\ raw(lang:\ none)\ } for types and function
    names.
  \end{itemize}
\end{itemize}

\subsubsection{v0.2.0}\label{v0.2.0}

\begin{itemize}
\tightlist
\item
  New features:

  \begin{itemize}
  \tightlist
  \item
    Add executable examples to docstrings.
  \item
    Documentation for variables (as well as functions).
  \item
    Docstring tests.
  \item
    Rainbow-colored types \texttt{\ color\ } and \texttt{\ gradient\ } .
  \end{itemize}
\item
  Improvements:

  \begin{itemize}
  \tightlist
  \item
    Allow customization of cross-references through
    \texttt{\ show-reference()\ } .
  \item
    Allow customization of spacing between functions through styles.
  \item
    Allow color customization (especially for the \texttt{\ default\ }
    theme).
  \end{itemize}
\item
  Fixes:

  \begin{itemize}
  \tightlist
  \item
    Empty parameter descriptions are omitted (if the corresponding
    option is set).
  \item
    Trim newline characters from parameter descriptions.
  \end{itemize}
\item
  âš~ï¸? Breaking changes:

  \begin{itemize}
  \tightlist
  \item
    Before, cross-references for functions using the \texttt{\ @@\ }
    syntax could omit the function parentheses. Now this is not possible
    anymore, since such references refer to variables now.
  \item
    (only concerning custom styles) The style functions
    \texttt{\ show-outline()\ } , \texttt{\ show-parameter-list\ } , and
    \texttt{\ show-type()\ } now take \texttt{\ style-args\ } arguments
    as well.
  \end{itemize}
\end{itemize}

\subsubsection{v0.1.0}\label{v0.1.0}

Initial Release.

\subsubsection{How to add}\label{how-to-add}

Copy this into your project and use the import as \texttt{\ tidy\ }

\begin{verbatim}
#import "@preview/tidy:0.3.0"
\end{verbatim}

\includesvg[width=0.16667in,height=0.16667in]{/assets/icons/16-copy.svg}

Check the docs for
\href{https://typst.app/docs/reference/scripting/\#packages}{more
information on how to import packages} .

\subsubsection{About}\label{about}

\begin{description}
\tightlist
\item[Author :]
\href{https://github.com/Mc-Zen}{Mc-Zen}
\item[License:]
MIT
\item[Current version:]
0.3.0
\item[Last updated:]
May 14, 2024
\item[First released:]
August 8, 2023
\item[Minimum Typst version:]
0.6.0
\item[Archive size:]
17.6 kB
\href{https://packages.typst.org/preview/tidy-0.3.0.tar.gz}{\pandocbounded{\includesvg[keepaspectratio]{/assets/icons/16-download.svg}}}
\item[Repository:]
\href{https://github.com/Mc-Zen/tidy}{GitHub}
\item[Categor ies :]
\begin{itemize}
\tightlist
\item[]
\item
  \pandocbounded{\includesvg[keepaspectratio]{/assets/icons/16-hammer.svg}}
  \href{https://typst.app/universe/search/?category=utility}{Utility}
\item
  \pandocbounded{\includesvg[keepaspectratio]{/assets/icons/16-code.svg}}
  \href{https://typst.app/universe/search/?category=scripting}{Scripting}
\item
  \pandocbounded{\includesvg[keepaspectratio]{/assets/icons/16-list-unordered.svg}}
  \href{https://typst.app/universe/search/?category=model}{Model}
\end{itemize}
\end{description}

\subsubsection{Where to report issues?}\label{where-to-report-issues}

This package is a project of Mc-Zen . Report issues on
\href{https://github.com/Mc-Zen/tidy}{their repository} . You can also
try to ask for help with this package on the
\href{https://forum.typst.app}{Forum} .

Please report this package to the Typst team using the
\href{https://typst.app/contact}{contact form} if you believe it is a
safety hazard or infringes upon your rights.

\phantomsection\label{versions}
\subsubsection{Version history}\label{version-history}

\begin{longtable}[]{@{}ll@{}}
\toprule\noalign{}
Version & Release Date \\
\midrule\noalign{}
\endhead
\bottomrule\noalign{}
\endlastfoot
0.3.0 & May 14, 2024 \\
\href{https://typst.app/universe/package/tidy/0.2.0/}{0.2.0} & January
3, 2024 \\
\href{https://typst.app/universe/package/tidy/0.1.0/}{0.1.0} & August 8,
2023 \\
\end{longtable}

Typst GmbH did not create this package and cannot guarantee correct
functionality of this package or compatibility with any version of the
Typst compiler or app.
