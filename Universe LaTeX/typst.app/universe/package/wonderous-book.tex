\title{typst.app/universe/package/wonderous-book}

\phantomsection\label{banner}
\phantomsection\label{template-thumbnail}
\pandocbounded{\includegraphics[keepaspectratio]{https://packages.typst.org/preview/thumbnails/wonderous-book-0.1.1-small.webp}}

\section{wonderous-book}\label{wonderous-book}

{ 0.1.1 }

A fiction book template with running headers and serif typography

\href{/app?template=wonderous-book&version=0.1.1}{Create project in app}

\phantomsection\label{readme}
A book template for fiction. The template contains a title page, a table
of contents, and a chapter template.

Dynamic running headers contain the title of the chapter and the book.

\subsection{Usage}\label{usage}

You can use this template in the Typst web app by clicking “Start from
template� on the dashboard and searching for
\texttt{\ wonderous-book\ } .

Alternatively, you can use the CLI to kick this project off using the
command

\begin{verbatim}
typst init @preview/wonderous-book
\end{verbatim}

Typst will create a new directory with all the files needed to get you
started.

\subsection{Configuration}\label{configuration}

This template exports the \texttt{\ book\ } function with the following
named arguments:

\begin{itemize}
\tightlist
\item
  \texttt{\ title\ } : The book’s title as content.
\item
  \texttt{\ author\ } : Content or an array of content to specify the
  author.
\item
  \texttt{\ paper-size\ } : Defaults to \texttt{\ iso-b5\ } . Specify a
  \href{https://typst.app/docs/reference/layout/page/\#parameters-paper}{paper
  size string} to change the page format.
\item
  \texttt{\ dedication\ } : Who or what this book is dedicated to as
  content or \texttt{\ none\ } . Will appear on its own page.
\item
  \texttt{\ publishing-info\ } : Details for the front matter of this
  book as content or \texttt{\ none\ } .
\end{itemize}

The function also accepts a single, positional argument for the body of
the book.

The template will initialize your package with a sample call to the
\texttt{\ book\ } function in a show rule. If you, however, want to
change an existing project to use this template, you can add a show rule
like this at the top of your file:

\begin{Shaded}
\begin{Highlighting}[]
\NormalTok{\#import "@preview/wonderous{-}book:0.1.1": book}

\NormalTok{\#show: book.with(}
\NormalTok{  title: [Liam\textquotesingle{}s Playlist],}
\NormalTok{  author: "Janet Doe",}
\NormalTok{  dedication: [for Rachel],}
\NormalTok{  publishing{-}info: [}
\NormalTok{    UK Publishing, Inc. \textbackslash{}}
\NormalTok{    6 Abbey Road \textbackslash{}}
\NormalTok{    Vaughnham, 1PX 8A3}

\NormalTok{    \#link("https://example.co.uk/")}

\NormalTok{    971{-}1{-}XXXXXX{-}XX{-}X}
\NormalTok{  ],}
\NormalTok{)}

\NormalTok{// Your content goes below.}
\end{Highlighting}
\end{Shaded}

\href{/app?template=wonderous-book&version=0.1.1}{Create project in app}

\subsubsection{How to use}\label{how-to-use}

Click the button above to create a new project using this template in
the Typst app.

You can also use the Typst CLI to start a new project on your computer
using this command:

\begin{verbatim}
typst init @preview/wonderous-book:0.1.1
\end{verbatim}

\includesvg[width=0.16667in,height=0.16667in]{/assets/icons/16-copy.svg}

\subsubsection{About}\label{about}

\begin{description}
\tightlist
\item[Author :]
\href{https://typst.app}{Typst GmbH}
\item[License:]
MIT-0
\item[Current version:]
0.1.1
\item[Last updated:]
October 29, 2024
\item[First released:]
March 6, 2024
\item[Minimum Typst version:]
0.12.0
\item[Archive size:]
4.06 kB
\href{https://packages.typst.org/preview/wonderous-book-0.1.1.tar.gz}{\pandocbounded{\includesvg[keepaspectratio]{/assets/icons/16-download.svg}}}
\item[Repository:]
\href{https://github.com/typst/templates}{GitHub}
\item[Categor y :]
\begin{itemize}
\tightlist
\item[]
\item
  \pandocbounded{\includesvg[keepaspectratio]{/assets/icons/16-docs.svg}}
  \href{https://typst.app/universe/search/?category=book}{Book}
\end{itemize}
\end{description}

\subsubsection{Where to report issues?}\label{where-to-report-issues}

This template is a project of Typst GmbH . Report issues on
\href{https://github.com/typst/templates}{their repository} . You can
also try to ask for help with this template on the
\href{https://forum.typst.app}{Forum} .

\phantomsection\label{versions}
\subsubsection{Version history}\label{version-history}

\begin{longtable}[]{@{}ll@{}}
\toprule\noalign{}
Version & Release Date \\
\midrule\noalign{}
\endhead
\bottomrule\noalign{}
\endlastfoot
0.1.1 & October 29, 2024 \\
\href{https://typst.app/universe/package/wonderous-book/0.1.0/}{0.1.0} &
March 6, 2024 \\
\end{longtable}
