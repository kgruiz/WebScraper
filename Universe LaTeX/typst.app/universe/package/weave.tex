\title{typst.app/universe/package/weave}

\phantomsection\label{banner}
\section{weave}\label{weave}

{ 0.2.0 }

A helper library for chaining lambda abstractions

\phantomsection\label{readme}
A helper library for chaining lambda abstractions, imitating the
\texttt{\ \textbar{}\textgreater{}\ } or \texttt{\ .\ } operator in some
functional languages.

The function \texttt{\ compose\ } is the \texttt{\ pipe\ } function in
the mathematical order. Functions suffixed with underscore have their
arguments flipped.

\subsection{Changelog}\label{changelog}

\begin{itemize}
\tightlist
\item
  0.2.0 Redesigned interface to work with typst’s \texttt{\ with\ }
  keyword.
\item
  0.1.0 Initial release
\end{itemize}

\subsection{Basic usage}\label{basic-usage}

It can help improve readability with nested applications to a content
value, or make the diff cleaner.

\begin{Shaded}
\begin{Highlighting}[]
\NormalTok{\#compose\_((}
\NormalTok{  text.with(blue),}
\NormalTok{  emph,}
\NormalTok{  strong,}
\NormalTok{  underline,}
\NormalTok{  strike,}
\NormalTok{))[This is a very long content with a lot of words]}
\NormalTok{// Is equivalent to}
\NormalTok{\#text(}
\NormalTok{  blue,}
\NormalTok{  emph(}
\NormalTok{    strong(}
\NormalTok{      underline(}
\NormalTok{        strike[This is a very long content with a lot of words]}
\NormalTok{      )}
\NormalTok{    )}
\NormalTok{  )}
\NormalTok{)}
\end{Highlighting}
\end{Shaded}

You can use it for show rules just like the example above.

\begin{Shaded}
\begin{Highlighting}[]
\NormalTok{\#show link: compose\_.with((}
\NormalTok{  text.with(fill: blue),}
\NormalTok{  emph,}
\NormalTok{  underline,}
\NormalTok{))}
\NormalTok{// These two are equivalent}
\NormalTok{\#show link: text.with(fill: blue)}
\NormalTok{\#show link: emph}
\NormalTok{\#show link: underline}
\end{Highlighting}
\end{Shaded}

This can also be useful when you need to destructure lists, as it allows
creating binds that are scoped by each lambda expression.

\begin{Shaded}
\begin{Highlighting}[]
\NormalTok{\#let two\_and\_one = pipe(}
\NormalTok{  (1, 2),}
\NormalTok{  (}
\NormalTok{    ((a, b)) =\textgreater{} (a, b, {-}1), // becomes a list of length three}
\NormalTok{    ((a, b, \_)) =\textgreater{} (b, a), // discard the third element and swap}
\NormalTok{  ),}
\NormalTok{)}
\end{Highlighting}
\end{Shaded}

\subsubsection{How to add}\label{how-to-add}

Copy this into your project and use the import as \texttt{\ weave\ }

\begin{verbatim}
#import "@preview/weave:0.2.0"
\end{verbatim}

\includesvg[width=0.16667in,height=0.16667in]{/assets/icons/16-copy.svg}

Check the docs for
\href{https://typst.app/docs/reference/scripting/\#packages}{more
information on how to import packages} .

\subsubsection{About}\label{about}

\begin{description}
\tightlist
\item[Author :]
\href{https://github.com/leana8959}{Léana 江}
\item[License:]
MIT
\item[Current version:]
0.2.0
\item[Last updated:]
October 21, 2024
\item[First released:]
October 21, 2024
\item[Archive size:]
1.92 kB
\href{https://packages.typst.org/preview/weave-0.2.0.tar.gz}{\pandocbounded{\includesvg[keepaspectratio]{/assets/icons/16-download.svg}}}
\item[Categor y :]
\begin{itemize}
\tightlist
\item[]
\item
  \pandocbounded{\includesvg[keepaspectratio]{/assets/icons/16-code.svg}}
  \href{https://typst.app/universe/search/?category=scripting}{Scripting}
\end{itemize}
\end{description}

\subsubsection{Where to report issues?}\label{where-to-report-issues}

This package is a project of Léana 江 . You can also try to ask for
help with this package on the \href{https://forum.typst.app}{Forum} .

Please report this package to the Typst team using the
\href{https://typst.app/contact}{contact form} if you believe it is a
safety hazard or infringes upon your rights.

\phantomsection\label{versions}
\subsubsection{Version history}\label{version-history}

\begin{longtable}[]{@{}ll@{}}
\toprule\noalign{}
Version & Release Date \\
\midrule\noalign{}
\endhead
\bottomrule\noalign{}
\endlastfoot
0.2.0 & October 21, 2024 \\
\href{https://typst.app/universe/package/weave/0.1.0/}{0.1.0} & October
21, 2024 \\
\end{longtable}

Typst GmbH did not create this package and cannot guarantee correct
functionality of this package or compatibility with any version of the
Typst compiler or app.
