\title{typst.app/universe/package/modern-acad-cv}

\phantomsection\label{banner}
\phantomsection\label{template-thumbnail}
\pandocbounded{\includegraphics[keepaspectratio]{https://packages.typst.org/preview/thumbnails/modern-acad-cv-0.1.1-small.webp}}

\section{modern-acad-cv}\label{modern-acad-cv}

{ 0.1.1 }

A CV template for academics based on moderncv LaTeX package.

\href{/app?template=modern-acad-cv&version=0.1.1}{Create project in app}

\phantomsection\label{readme}
This template for an academic CV serves the peculiarities of academic
CVs. If you are not an academic, this template is not useful. Most of
the times in academics, applicants need to show everything they have
done. This makes it a bit cumbersome doing it by single entries. In
addition, academics might apply to institutions around the globe, making
it necessary to send translated CVs or at least translations of some
parts (i.e., title of papers in different languages).

This template serves these special needs in introducting automated
sections based on indicated \texttt{\ yaml\ } -files. Furthermore, it
has a simplified multilingual support by setting different headers,
title etc. for different languages (by the user in the \texttt{\ yaml\ }
-fields). With this template, it might be more handy to keep your CV
easier on track, especially when you need in different languages, since
managing a \texttt{\ yaml\ } -file is easier than checking typesetting
files against each other.

This template is influenced by LaTeX’s
\href{https://github.com/moderncv/moderncv}{moderncv} and its typst
translation
\href{https://github.com/DeveloperPaul123/modern-cv}{moderner-cv} .

\subsection{Fonts}\label{fonts}

In this template, the use of FontAwesome icons via the
\href{https://typst.app/universe/package/fontawesome}{fontawesome typst
package} is possible, as well as the icons from Academicons
\href{https://typst.app/universe/package/use-academicons}{use-academicons
typst package} . To use these icons properly, you need to install each
fonts on your system. You can download
\href{https://fontawesome.com/download}{fontawesome here} and
\href{https://jpswalsh.github.io/academicons/}{academicons here} . Both
typst packages will be load by the template itself.

Furthermore, I included my favorite font
\href{https://fonts.google.com/specimen/Fira+Sans}{Fira Sans} . You can
download it here
\href{https://fonts.google.com/specimen/Fira+Sans}{here} , or just
change the font argument in \texttt{\ modern-acad-cv()\ } .

\subsection{Usage}\label{usage}

The main function to load the construct of the academic CV is
\texttt{\ modern-acad-cv()\ } . After importing the template, you can
call it right away. If you don’t have
\href{https://fonts.google.com/specimen/Fira+Sans}{Fira Sans} installed,
choose a different font. Examples are given below.

\begin{Shaded}
\begin{Highlighting}[]
\NormalTok{\#import "@preview/modern{-}acad{-}cv:0.1.1": *}

\NormalTok{\#show: modern{-}acad{-}cv.with(}
\NormalTok{  metadata,}
\NormalTok{  multilingual,}
\NormalTok{  lang: "en",}
\NormalTok{  font: ("Fira Sans", "Andale Mono", "Roboto"),}
\NormalTok{  show{-}date: true,}
\NormalTok{  body}
\NormalTok{)    }

\NormalTok{// ...}
\end{Highlighting}
\end{Shaded}

In the remainder, I show basic settings and how to use the automated
functions with the corresponding \texttt{\ yaml\ } -file.

\subsubsection{\texorpdfstring{Setting up the main file and the access
to the \texttt{\ yaml\ }
-files}{Setting up the main file and the access to the  yaml  -files}}\label{setting-up-the-main-file-and-the-access-to-the-yaml--files}

A first step in your document is to invoke the template. Second, since
this template works with \texttt{\ yaml\ } -files in the background you
need to specify paths to each \texttt{\ yaml\ } -file you want to use
throughout the document.

The template comes along with the \texttt{\ metadata.yaml\ } . In the
beginning of this yaml-file you set colors. Feel free to change it to
your preferred color scheme.

\begin{Shaded}
\begin{Highlighting}[]
\FunctionTok{colors}\KeywordTok{:}
\AttributeTok{  }\FunctionTok{main\_color}\KeywordTok{:}\AttributeTok{ }\StringTok{"\#579D90"}
\AttributeTok{  }\FunctionTok{lightgray\_color}\KeywordTok{:}\AttributeTok{ }\StringTok{"\#d5d5d5"}
\AttributeTok{  }\FunctionTok{gray\_color}\KeywordTok{:}\AttributeTok{ }\StringTok{"\#737373"}
\AttributeTok{  ...}
\end{Highlighting}
\end{Shaded}

At the beginning of your document, you just set then set the
metadata-object:

\begin{Shaded}
\begin{Highlighting}[]
\NormalTok{\#import "@preview/modern{-}acad{-}cv:0.1.0": *}

\NormalTok{\#let metadata = yaml("metadata.yaml")}
\end{Highlighting}
\end{Shaded}

Initially, the \texttt{\ metadata.yaml\ } is located on the same level
as the \texttt{\ example.typ\ } . All other \texttt{\ yaml\ } -files are
saved in the folder \texttt{\ dbs\ } . Since \texttt{\ typ\ } -documents
search for paths from the root of the document in that the function is
called, you have to give the databases for the entry along the
\texttt{\ metadata.yaml\ } within each function call.

\subsubsection{socials}\label{socials}

Contact details are important. In this CV template, you have the
possibility to use fontawesome icons and academicons. To use socials,
you just need to specify in \texttt{\ metadata.yaml\ } , the wanted
entries.

As you can see below, you set a category, i.e. email or lattes and then
you have to define four arguments: \texttt{\ username\ } ,
\texttt{\ prefix\ } , \texttt{\ icon\ } , and \texttt{\ set\ } . The
\texttt{\ username\ } will be used for constructing the link and will be
shown next to the logo. The \texttt{\ prefix\ } is needed to build the
valid link. The \texttt{\ icon\ } is the name of the icon in the
respective set, which is chosen in \texttt{\ set\ } .

\begin{Shaded}
\begin{Highlighting}[]
\FunctionTok{personal}\KeywordTok{:}
\AttributeTok{  }\FunctionTok{name}\KeywordTok{:}\AttributeTok{ }\KeywordTok{[}\StringTok{"Mustermensch, Momo"}\KeywordTok{]}
\AttributeTok{  }\FunctionTok{socials}\KeywordTok{:}
\AttributeTok{    }\FunctionTok{email}\KeywordTok{:}
\AttributeTok{      }\FunctionTok{username}\KeywordTok{:}\AttributeTok{ momo@mustermensch.com}
\AttributeTok{      }\FunctionTok{prefix}\KeywordTok{:}\AttributeTok{ }\StringTok{"mailto:"}
\AttributeTok{      }\FunctionTok{icon}\KeywordTok{:}\AttributeTok{ paper{-}plane}
\AttributeTok{      }\FunctionTok{set}\KeywordTok{:}\AttributeTok{ fa}
\AttributeTok{    }\FunctionTok{homepage}\KeywordTok{:}
\AttributeTok{      }\FunctionTok{username}\KeywordTok{:}\AttributeTok{ momo.github.io}
\AttributeTok{      }\FunctionTok{prefix}\KeywordTok{:}\AttributeTok{ https://}
\AttributeTok{      }\FunctionTok{icon}\KeywordTok{:}\AttributeTok{ globe}
\AttributeTok{      }\FunctionTok{set}\KeywordTok{:}\AttributeTok{ fa}
\AttributeTok{    }\FunctionTok{orcid}\KeywordTok{:}
\AttributeTok{      }\FunctionTok{username}\KeywordTok{:}\AttributeTok{ 0000{-}0000{-}0000{-}0000}
\AttributeTok{      }\FunctionTok{prefix}\KeywordTok{:}\AttributeTok{ https://orcid.org}
\AttributeTok{      }\FunctionTok{icon}\KeywordTok{:}\AttributeTok{ orcid}
\AttributeTok{      }\FunctionTok{set}\KeywordTok{:}\AttributeTok{ ai}
\AttributeTok{    }\FunctionTok{lattes}\KeywordTok{:}
\AttributeTok{      }\FunctionTok{username}\KeywordTok{:}\AttributeTok{ }\StringTok{"1234567891234567"}
\AttributeTok{      }\FunctionTok{prefix}\KeywordTok{:}\AttributeTok{ http://lattes.cnpq.br/}
\AttributeTok{      }\FunctionTok{icon}\KeywordTok{:}\AttributeTok{ lattes}
\AttributeTok{      }\FunctionTok{set}\KeywordTok{:}\AttributeTok{ ai}
\AttributeTok{    ...}
\end{Highlighting}
\end{Shaded}

\subsubsection{Language setting \&
headers}\label{language-setting-headers}

In order to support changing headers, you need to specify the language
and the different content for each header in each language in the
\texttt{\ i18n.yaml\ } in the folder \texttt{\ dbs\ } .

The structure of the yaml is simple:

\begin{Shaded}
\begin{Highlighting}[]
\FunctionTok{lang}\KeywordTok{:}
\AttributeTok{  }\FunctionTok{de}\KeywordTok{:}
\AttributeTok{    }\FunctionTok{subtitle}\KeywordTok{:}\AttributeTok{ Short CV}
\AttributeTok{    }\FunctionTok{education}\KeywordTok{:}\AttributeTok{ Hochschulbildung}
\AttributeTok{    }\FunctionTok{work}\KeywordTok{:}\AttributeTok{ Akademische Berufserfahrung (Auswahl)}
\AttributeTok{    }\FunctionTok{grants}\KeywordTok{:}\AttributeTok{ Fördermittel, Stipendien \& Preise}
\AttributeTok{    ...}
\AttributeTok{  }\FunctionTok{en}\KeywordTok{:}
\AttributeTok{    }\FunctionTok{subtitle}\KeywordTok{:}\AttributeTok{ Short CV}
\AttributeTok{    }\FunctionTok{education}\KeywordTok{:}\AttributeTok{ Higher education}
\AttributeTok{    }\FunctionTok{work}\KeywordTok{:}\AttributeTok{ Academic work experience (selection)}
\AttributeTok{    }\FunctionTok{grants}\KeywordTok{:}\AttributeTok{ Scholarships \& awards}
\AttributeTok{    ...}
\AttributeTok{  }\FunctionTok{pt}\KeywordTok{:}
\AttributeTok{    }\FunctionTok{subtitle}\KeywordTok{:}\AttributeTok{ Currículo}
\AttributeTok{    }\FunctionTok{education}\KeywordTok{:}\AttributeTok{ Formação acadêmica}
\AttributeTok{    }\FunctionTok{work}\KeywordTok{:}\AttributeTok{ Atuação profissional (seleção)}
\AttributeTok{    }\FunctionTok{grants}\KeywordTok{:}\AttributeTok{ Bolsas de estudo e prémios}
\AttributeTok{    ...}
\end{Highlighting}
\end{Shaded}

For each language, you want to use later, you have to define all the
entries. Reminder, don’t change the entry names, since the functions
won’t find it under different names without changing the functions.

First you have to set up a variable that inherits the ISO-language code,
save the database into an object (here \texttt{\ multilingual\ } ) and
then give the object \texttt{\ multilingual\ } and \texttt{\ language\ }
to the function \texttt{\ create-headers\ } :

\begin{Shaded}
\begin{Highlighting}[]
\NormalTok{// set the language of the document}
\NormalTok{\#let language = "pt"      }

\NormalTok{// loading multilingual database}
\NormalTok{\#let multilingual = yaml("dbs/i18n.yaml")}

\NormalTok{// defining variables}
\NormalTok{\#let headerLabs = create{-}headers(multilingual, lang: language)}
\end{Highlighting}
\end{Shaded}

You create an object \texttt{\ headerLabs\ } that uses the function
\texttt{\ create-headers()\ } which will define the headers as you
provided in the \texttt{\ yaml\ } . Then by switching the language
object, all headers (if used accordingly to the naming in the
\texttt{\ yaml\ } ) will change directly.

Throughout the document you then reference the created
\texttt{\ headerLabs\ } object. If you change language, and values are
provided, these automatically change.

\begin{Shaded}
\begin{Highlighting}[]
\NormalTok{= \#headerLabs.at("work")}

\NormalTok{...}

\NormalTok{= \#headerLabs.at("education")}

\NormalTok{...}

\NormalTok{= \#headerLabs.at("grants")}
\end{Highlighting}
\end{Shaded}

\subsubsection{Automated functions}\label{automated-functions}

All of the following functions share common arguments: \texttt{\ what\ }
, \texttt{\ multilingual\ } , and \texttt{\ lang\ } . In
\texttt{\ what\ } , you always declare the database you want to use with
the function.

For example, to get work entries, you choose \texttt{\ work\ } , which
you defined beforehand as input from \texttt{\ work.yaml\ } . In the
\texttt{\ multilingual\ } argument, you just pass the
\texttt{\ multilingual\ } object. In \texttt{\ lang\ } you pass your
\texttt{\ language\ } object.

\begin{Shaded}
\begin{Highlighting}[]
\NormalTok{\#let multilingual = yaml("dbs/multilingual.yaml")}
\NormalTok{\#let work = yaml("dbs/work.yaml")}
\NormalTok{\#let language = "pt"}

\NormalTok{// Function call with objects}
\NormalTok{\#cv{-}auto{-}stc(work, multilingual, lang: language)}
\end{Highlighting}
\end{Shaded}

\subsubsection{Sorting publications and referencing your own name or
correpsonding}\label{sorting-publications-and-referencing-your-own-name-or-correpsonding}

Since \texttt{\ typst\ } so far does not support multiple bibliographies
or subsetting these, this function let you choose specific entries via
the \texttt{\ entries\ } argument or group of entries by the
\texttt{\ tag\ } argument. Furthermore, you can indicate a string in
\texttt{\ me\ } that can be highlighted in every output entry (i.e.,
your formatted name). So far, this function leads to another function
that create APA-style format, if you want to use any other citation
style, you need to download the template on
\href{https://github.com/bpkleer/modern-acad-cv}{github} , introduce
your own styling and then add it in the \texttt{\ cv-refs()\ } function.

\begin{Shaded}
\begin{Highlighting}[]
\NormalTok{\#let multilingual = yaml("dbs/multilingual.yaml")}
\NormalTok{\#let refs = yaml("dbs/refs/yaml")}

\NormalTok{// function call of group of peer{-}reviewed with tag \textasciigrave{}peer\textasciigrave{}}
\NormalTok{\#cv{-}refs(refs, multilingual, tag: "peer", me: [Mustermensch, M.], lang: language)}
\end{Highlighting}
\end{Shaded}

You see in the example pictures that I used this function to built five
different subheaders, i.e. for peer reviewed articles (
\texttt{\ tag:\ "peer"\ } ) and chapters in edited books (
\texttt{\ tag:\ "edited"\ } ). You can define the tags how you want,
however, they need to put them into
\texttt{\ tag:\ \textless{}str\textgreater{}\ } .

Sometimes, it is not only necessary to highlight your own name, you
might also want to indicate yourself as corresponding author. This can
be done through the \texttt{\ refs.yaml\ } which adhere to
\href{https://github.com/typst/hayagriva}{Hayagriva} . By adding an
argument \texttt{\ corresponding\ } in the yaml and setting the value to
\texttt{\ true\ } , a small \texttt{\ C\ } will appear next to your
name.

\begin{Shaded}
\begin{Highlighting}[]
\FunctionTok{Mustermensch2023}\KeywordTok{:}
\AttributeTok{  }\FunctionTok{type}\KeywordTok{:}\AttributeTok{ }\StringTok{"article"}
\AttributeTok{  }\FunctionTok{date}\KeywordTok{:}\AttributeTok{ }\DecValTok{2023}
\AttributeTok{  }\FunctionTok{page{-}range}\KeywordTok{:}\AttributeTok{ 55{-}78}
\AttributeTok{  }\FunctionTok{title}\KeywordTok{:}\AttributeTok{ }\StringTok{"Populism and Social Media: A Comparative Study of Political Mobilization"}
\AttributeTok{  }\FunctionTok{tags}\KeywordTok{:}\AttributeTok{ }\StringTok{"peer"}
\AttributeTok{  }\FunctionTok{author}\KeywordTok{:}\AttributeTok{ }\KeywordTok{[}\AttributeTok{ }\StringTok{"Mustermensch, Momo"}\KeywordTok{,}\AttributeTok{ }\StringTok{"Rivera, Casey"}\AttributeTok{ }\KeywordTok{]}
\AttributeTok{  }\FunctionTok{corresponding}\KeywordTok{:}\AttributeTok{ }\CharTok{true}
\AttributeTok{  }\FunctionTok{parent}\KeywordTok{:}
\AttributeTok{    }\FunctionTok{title}\KeywordTok{:}\AttributeTok{ }\StringTok{"Journal of Political Communication"}
\AttributeTok{    }\FunctionTok{volume}\KeywordTok{:}\AttributeTok{ }\DecValTok{41}
\AttributeTok{    }\FunctionTok{issue}\KeywordTok{:}\AttributeTok{ }\DecValTok{3}
\AttributeTok{  }\FunctionTok{serial{-}number}\KeywordTok{:}
\AttributeTok{    }\FunctionTok{doi}\KeywordTok{:}\AttributeTok{ }\StringTok{"10.1016/j.jpolcom.2023.102865"}
\end{Highlighting}
\end{Shaded}

For applications abroad, it might be worth to translate at least title
of the publications so that other persons easily can see what the paper
is about. In every \texttt{\ title\ } argument, you can therefore
provide a dictionary with the language codes and the titles. Keep the
original title in \texttt{\ main\ } and the translations with the
corresponding language shortcut (i.e., \texttt{\ "en"\ } or
\texttt{\ "pt"\ } ). The function prints the main and translated title,
depending on the provided translation in the \texttt{\ refs.yaml\ } . Be
aware, here you find not \texttt{\ de\ } in the dictionary, instead you
find \texttt{\ main\ } . The original title needs to be wrapped in
\texttt{\ main\ } .

\begin{Shaded}
\begin{Highlighting}[]
\FunctionTok{Mustermensch2023}\KeywordTok{:}
\AttributeTok{  }\FunctionTok{type}\KeywordTok{:}\AttributeTok{ }\StringTok{"article"}
\AttributeTok{  }\FunctionTok{date}\KeywordTok{:}\AttributeTok{ }\DecValTok{2023}
\AttributeTok{  }\FunctionTok{page{-}range}\KeywordTok{:}\AttributeTok{ 55{-}78}
\AttributeTok{  }\FunctionTok{title}\KeywordTok{:}\AttributeTok{ }
\AttributeTok{    }\FunctionTok{main}\KeywordTok{:}\AttributeTok{ }\StringTok{"Populismus und soziale Medien: Eine vergleichende Studie zur politischen Mobilisierung"}
\AttributeTok{    }\FunctionTok{en}\KeywordTok{:}\AttributeTok{ }\StringTok{"Populism and Social Media: A Comparative Study of Political Mobilization"}
\AttributeTok{    }\FunctionTok{pt}\KeywordTok{:}\AttributeTok{ }\StringTok{"Populismo e redes sociais: Um Estudo Comparativo de Mobilização Política"}
\AttributeTok{  }\FunctionTok{tags}\KeywordTok{:}\AttributeTok{ }\StringTok{"peer"}
\AttributeTok{  }\FunctionTok{author}\KeywordTok{:}\AttributeTok{ }\KeywordTok{[}\AttributeTok{ }\StringTok{"Mustermensch, Momo"}\KeywordTok{,}\AttributeTok{ }\StringTok{"Rivera, Casey"}\AttributeTok{ }\KeywordTok{]}
\AttributeTok{  }\FunctionTok{corresponding}\KeywordTok{:}\AttributeTok{ }\CharTok{true}
\AttributeTok{  }\FunctionTok{parent}\KeywordTok{:}
\AttributeTok{    }\FunctionTok{title}\KeywordTok{:}\AttributeTok{ }\StringTok{"Journal of Political Communication"}
\AttributeTok{    }\FunctionTok{volume}\KeywordTok{:}\AttributeTok{ }\DecValTok{41}
\AttributeTok{    }\FunctionTok{issue}\KeywordTok{:}\AttributeTok{ }\DecValTok{3}
\AttributeTok{  }\FunctionTok{serial{-}number}\KeywordTok{:}
\AttributeTok{    }\FunctionTok{doi}\KeywordTok{:}\AttributeTok{ }\StringTok{"10.1016/j.jpolcom.2023.102865"}
\end{Highlighting}
\end{Shaded}

\subsubsection{cv-auto-skills()}\label{cv-auto-skills}

Instead of just enumerating your skills or your knowledge of specific
software, you can build a skill-matrix with this function. In this
skill-matrix, you can have sections, i.e. \emph{Computer Languages} ,
\emph{Programs} and \emph{Languages} . These sections are the highest
level in the corresponding \texttt{\ skills.yaml\ } :

\begin{Shaded}
\begin{Highlighting}[]
\FunctionTok{computer}\KeywordTok{:}
\AttributeTok{  ...}
\FunctionTok{programs}\KeywordTok{:}
\AttributeTok{  ...}
\FunctionTok{languages}\KeywordTok{:}
\AttributeTok{  ...}
\end{Highlighting}
\end{Shaded}

You can then define in each categories specific skills, i.e. German and
Portuguese in \texttt{\ languages\ } :

\begin{Shaded}
\begin{Highlighting}[]
\FunctionTok{computer}\KeywordTok{:}
\AttributeTok{  ...}
\FunctionTok{programs}\KeywordTok{:}
\AttributeTok{  ...}
\FunctionTok{languages}\KeywordTok{:}
\AttributeTok{  }\FunctionTok{german}\KeywordTok{:}
\AttributeTok{    ...}
\AttributeTok{  }\FunctionTok{portugues}\KeywordTok{:}
\AttributeTok{   ...}
\end{Highlighting}
\end{Shaded}

For each entry, you have to define \texttt{\ name\ } ,
\texttt{\ level\ } and \texttt{\ description\ } .

\begin{Shaded}
\begin{Highlighting}[]
\FunctionTok{languages}\KeywordTok{:}\AttributeTok{ }
\AttributeTok{  ...}
\AttributeTok{  }\FunctionTok{pt}\KeywordTok{:}\AttributeTok{ }
\AttributeTok{    }\FunctionTok{name}\KeywordTok{:}
\AttributeTok{      }\FunctionTok{de}\KeywordTok{:}\AttributeTok{ Portugiesisch}
\AttributeTok{      }\FunctionTok{en}\KeywordTok{:}\AttributeTok{ Portuguese}
\AttributeTok{      }\FunctionTok{pt}\KeywordTok{:}\AttributeTok{ Português}
\AttributeTok{    }\FunctionTok{level}\KeywordTok{:}\AttributeTok{ }\DecValTok{3}
\AttributeTok{    }\FunctionTok{description}\KeywordTok{:}
\AttributeTok{      }\FunctionTok{de}\KeywordTok{:}\AttributeTok{ fortgeschritten}
\AttributeTok{      }\FunctionTok{en}\KeywordTok{:}\AttributeTok{ advanced}
\AttributeTok{      }\FunctionTok{pt}\KeywordTok{:}\AttributeTok{ avançado}
\end{Highlighting}
\end{Shaded}

As you can see, you can again define language-dependent names in
\texttt{\ name\ } and descriptions in \texttt{\ description\ } .
\texttt{\ level\ } is a numeric value and indicates how many of the four
boxes are filled to indicate you level of proficiency. If you don’t
have the need for a CV of different languages, you can directly define
\texttt{\ name\ } or \texttt{\ description\ } .

You have to call the function with three objects \texttt{\ skills\ } ,
\texttt{\ multilingual\ } , and \texttt{\ metadata\ } and the
corresponding \texttt{\ language\ } of the document:

\begin{Shaded}
\begin{Highlighting}[]
\NormalTok{\#let skills = yaml("dbs/skills.yaml")}
\NormalTok{\#let multilingual = yaml("dbs/multilingual.yaml")}
\NormalTok{\#let metadata = yaml("dbs/metadata.yaml")}
\NormalTok{\#let language = "pt"}

\NormalTok{\#cv{-}auto{-}skills(skills, multilingual, metadata, lang: language)}
\end{Highlighting}
\end{Shaded}

\subsubsection{Print your info without any
formatting}\label{print-your-info-without-any-formatting}

The function \texttt{\ cv-auto\ } is the base function for printing the
provided infos in the specified \texttt{\ yaml\ } file with no further
formatting. The functions \texttt{\ cv-auto-stc\ } and
\texttt{\ cv-auto-stp\ } do only differ in the point that
\texttt{\ cv-auto-stc\ } both give the title in bold,
\texttt{\ cv-auto-stp\ } puts the subtitle in parentheses and
\texttt{\ cv-auto-stc\ } puts the subtitle after a comma.

The structure of the corresponding \texttt{\ yaml\ } files is simple: in
each entry you can have the following entries: \texttt{\ title\ } ,
\texttt{\ subtitle\ } , \texttt{\ location\ } , \texttt{\ description\ }
and \texttt{\ left\ } . \texttt{\ title\ } is mandatory,
\texttt{\ subtitle\ } , \texttt{\ location\ } , and
\texttt{\ description\ } are voluntary. In all functions you need to
specify \texttt{\ left\ } , which indicates period of time, or year. For
\texttt{\ title\ } , \texttt{\ subtitle\ } , \texttt{\ location\ } , and
\texttt{\ description\ } , you can provide a dictionary for different
languages (see below).

\begin{Shaded}
\begin{Highlighting}[]
\FunctionTok{master}\KeywordTok{:}
\AttributeTok{  }\FunctionTok{title}\KeywordTok{:}
\AttributeTok{    }\FunctionTok{de}\KeywordTok{:}\AttributeTok{ Master of Arts}
\AttributeTok{    }\FunctionTok{en}\KeywordTok{:}\AttributeTok{ Master of Arts}
\AttributeTok{    }\FunctionTok{pt}\KeywordTok{:}\AttributeTok{ Pós{-}Graduação}
\AttributeTok{  }\FunctionTok{subtitle}\KeywordTok{:}
\AttributeTok{    }\FunctionTok{de}\KeywordTok{:}\AttributeTok{ Sozialwissenschaften}
\AttributeTok{    }\FunctionTok{en}\KeywordTok{:}\AttributeTok{ Social Sciences}
\AttributeTok{    }\FunctionTok{pt}\KeywordTok{:}\AttributeTok{ Ciências Sociais}
\AttributeTok{  }\FunctionTok{location}\KeywordTok{:}
\AttributeTok{    }\FunctionTok{de}\KeywordTok{:}\AttributeTok{ Exzellenz{-}Universität}
\AttributeTok{    }\FunctionTok{en}\KeywordTok{:}\AttributeTok{ University of Excellence}
\AttributeTok{    }\FunctionTok{pt}\KeywordTok{:}\AttributeTok{ Universidade de Excelência}
\AttributeTok{  }\FunctionTok{description}\KeywordTok{:}
\AttributeTok{    }\FunctionTok{de}\KeywordTok{:}\AttributeTok{ mit Auszeichnung}
\AttributeTok{    }\FunctionTok{en}\KeywordTok{:}\AttributeTok{ with distinction}
\AttributeTok{    }\FunctionTok{pt}\KeywordTok{:}\AttributeTok{ com distinção}
\AttributeTok{  }\FunctionTok{left}\KeywordTok{:}\AttributeTok{ }\StringTok{"2014"}
\end{Highlighting}
\end{Shaded}

In your main document, you then easily call the function and transfer
the standard arguments \texttt{\ what\ } , \texttt{\ metadata\ } , and
\texttt{\ lang\ } .

\begin{Shaded}
\begin{Highlighting}[]
\NormalTok{// section of education }
\NormalTok{\#let education = yaml("dbs/education.yaml")}
\NormalTok{\#let multilingual = yaml("dbs/multilingual.yaml")}
\NormalTok{\#let language = "pt"}

\NormalTok{\#cv{-}auto{-}stp(education, multilingual, lang: language) }

\NormalTok{// section of work positions}
\NormalTok{\#let work = yaml("dbs/work.yaml")}
\NormalTok{\#let multilingual = yaml("dbs/multilingual.yaml")}
\NormalTok{\#let language = "pt"}

\NormalTok{\#cv{-}auto{-}stc(work, multilingual, lang: language)}

\NormalTok{// section of given talks}
\NormalTok{\#let talks = yaml("dbs/talks.yaml")}
\NormalTok{\#let multilingual = yaml("dbs/multilingual.yaml")}
\NormalTok{\#let language = "pt"}

\NormalTok{\#cv{-}auto(talks, multilingual, lang: language)}
\end{Highlighting}
\end{Shaded}

\subsubsection{Creating a list instead of single
entrie}\label{creating-a-list-instead-of-single-entrie}

Sometimes, instead of giving every entry, you want to group by year.
Another example for this case could be that you want to summarize your
memberships or reviewer duties.

The function \texttt{\ cv-auto-list\ } uses just the standard input:

\begin{Shaded}
\begin{Highlighting}[]
\NormalTok{\#let conferences = yaml("dbs/conferences.yaml")}
\NormalTok{\#let multilingual = yaml("dbs/multilingual.yaml")}
\NormalTok{\#let language = "pt"}

\NormalTok{\#cv{-}auto{-}list(conferences, multilingual, lang: language)}
\end{Highlighting}
\end{Shaded}

The corresponding \texttt{\ yaml\ } file is differently organized: The
entry point in the file is the corresponding year. In every year, you
organize your entries (i.e. conference participations). In each entry in
a year, you have the \texttt{\ name\ } and \texttt{\ action\ } entry.
You can provide a dictionary for the \texttt{\ name\ } . For
\texttt{\ action\ } , I used \texttt{\ P\ } and \texttt{\ C\ } , for
\emph{paper/presentation} and \emph{chair} . You can then manually
define this upfront the function call for the reader, or you use the
\texttt{\ i18n.yaml\ } , indicate the explanations for each language in
\texttt{\ exp-confs\ } and then it automatically changes with the
specific language code.

\begin{Shaded}
\begin{Highlighting}[]
\FunctionTok{"2024"}\KeywordTok{:}
\AttributeTok{  }\FunctionTok{conference2}\KeywordTok{:}
\AttributeTok{    }\FunctionTok{name}\KeywordTok{:}\AttributeTok{ European Conference on Gender and Politics}
\AttributeTok{    }\FunctionTok{action}\KeywordTok{:}\AttributeTok{ P}
\AttributeTok{  }\FunctionTok{conference1}\KeywordTok{:}
\AttributeTok{    }\FunctionTok{name}\KeywordTok{:}\AttributeTok{ ECPR General Conference}
\AttributeTok{    }\FunctionTok{action}\KeywordTok{:}\AttributeTok{ P, C}
\end{Highlighting}
\end{Shaded}

The action will be added after each conference name in superscripts.

\subsubsection{Creating a table}\label{creating-a-table}

This case is mostly used for listing your prior teaching experience. The
corresponding \texttt{\ teaching.yaml\ } for this description, is
organized as followed:

\begin{Shaded}
\begin{Highlighting}[]
\FunctionTok{"2024"}\KeywordTok{:}
\AttributeTok{  }\FunctionTok{course1}\KeywordTok{:}
\AttributeTok{    }\FunctionTok{summer}\KeywordTok{:}\AttributeTok{ T}
\AttributeTok{    }\FunctionTok{name}\KeywordTok{:}
\AttributeTok{      }\FunctionTok{de}\KeywordTok{:}\AttributeTok{ }\StringTok{"Statistik+: Einstieg in R leicht gemacht"}
\AttributeTok{      }\FunctionTok{en}\KeywordTok{:}\AttributeTok{ }\StringTok{"Statistics+: Starting with R (de)"}
\AttributeTok{      }\FunctionTok{pt}\KeywordTok{:}\AttributeTok{ }\StringTok{"Estatística+: Começando com R (de)"}
\AttributeTok{    }\FunctionTok{study}\KeywordTok{:}
\AttributeTok{      }\FunctionTok{de}\KeywordTok{:}\AttributeTok{ Bachelor}
\AttributeTok{      }\FunctionTok{en}\KeywordTok{:}\AttributeTok{ Bachelor}
\AttributeTok{      }\FunctionTok{pt}\KeywordTok{:}\AttributeTok{ Graduação}
\AttributeTok{  ...}
\end{Highlighting}
\end{Shaded}

First you indicate the year \texttt{\ "2024"\ } and then you organize
all courses you gave within that year (i.e. here \texttt{\ course1\ } ).
Mandatory are \texttt{\ name\ } and \texttt{\ study\ } . For both you
can indicate a single value or a dictionary corresponding to your chosen
languages. You can provide \texttt{\ summer\ } if you want to indicate
differences for terms. This is \texttt{\ boolean\ } , the specific word
is then given in the \texttt{\ i18n.yaml\ } under
\texttt{\ table-winter\ } resp. \texttt{\ table-summer\ } .

The function then uses again just the standard arguments and plots a
table with the indicated year, name, and study area.

\begin{Shaded}
\begin{Highlighting}[]
\NormalTok{\#let teaching = yaml("dbs/teaching.yaml")}
\NormalTok{\#let multilingual = yaml("dbs/multilingual.yaml")}
\NormalTok{\#let language = "pt"}

\NormalTok{\#cv{-}table{-}teaching(teaching, multilingual, lang: language)}
\end{Highlighting}
\end{Shaded}

\subsubsection{cv-auto-cats()}\label{cv-auto-cats}

In case you want to directly print entries from categories that belong
to one \texttt{\ yaml\ } -file, you can use \texttt{\ cv-auto-cats\ } .
This will print the header for each subcategory and then the belonging
entries.

An example is given in \texttt{\ training.yaml\ } . In this file,
further training is given by categories (i.e., methods and didactics).
Within the categories you have here courses and then \texttt{\ title\ }
, \texttt{\ location\ } , and \texttt{\ left\ } . \texttt{\ location\ }
and \texttt{\ title\ } can be dictionaries if you want to translate
between different languages.

\begin{Shaded}
\begin{Highlighting}[]
\FunctionTok{methods}\KeywordTok{:}
\AttributeTok{  }\FunctionTok{course2}\KeywordTok{:}
\AttributeTok{    }\FunctionTok{title}\KeywordTok{:}\AttributeTok{ Bayesian modelling in the Social Sciences}
\AttributeTok{    }\FunctionTok{location}\KeywordTok{:}\AttributeTok{ An expensive Spring Seminar}
\AttributeTok{    }\FunctionTok{left}\KeywordTok{:}\AttributeTok{ }\StringTok{"2024"}
\AttributeTok{  ...}
\FunctionTok{didactics}\KeywordTok{:}
\AttributeTok{  }\FunctionTok{course2}\KeywordTok{:}
\AttributeTok{    }\FunctionTok{title}\KeywordTok{:}
\AttributeTok{      }\FunctionTok{de}\KeywordTok{:}\AttributeTok{ Konfliktkompetenz I + II}
\AttributeTok{      }\FunctionTok{en}\KeywordTok{:}\AttributeTok{ Conflict competence I + II}
\AttributeTok{      }\FunctionTok{pt}\KeywordTok{:}\AttributeTok{ Competência de conflitos I + II}
\AttributeTok{    }\FunctionTok{location}\KeywordTok{:}\AttributeTok{ }
\AttributeTok{      }\FunctionTok{de}\KeywordTok{:}\AttributeTok{ Universitätsallianz}
\AttributeTok{      }\FunctionTok{en}\KeywordTok{:}\AttributeTok{ University Alliance}
\AttributeTok{      }\FunctionTok{pt}\KeywordTok{:}\AttributeTok{ Aliança Universitária}
\AttributeTok{    }\FunctionTok{left}\KeywordTok{:}\AttributeTok{ }\StringTok{"2019"}
\end{Highlighting}
\end{Shaded}

Call the function as usal:

\begin{Shaded}
\begin{Highlighting}[]
\NormalTok{\#let training = yaml("dbs/training.yaml")}
\NormalTok{\#let multilingual = yaml("dbs/multilingual.yaml")}
\NormalTok{\#let language = "pt"}
\NormalTok{\#let headerLabs = create{-}headers(multilingual, lang: language)}

\NormalTok{\#cv{-}auto{-}cats(training, multilingual, headerLabs, lang: language)}
\end{Highlighting}
\end{Shaded}

\subsubsection{Special cases: long
names}\label{special-cases-long-names}

If you have a long name that crosses the social media side, just set the
argument \texttt{\ split\ } to \texttt{\ true\ } within
\texttt{\ metadata.yaml\ } :

\begin{Shaded}
\begin{Highlighting}[]
\CommentTok{...}
\CommentTok{  personal:}
\CommentTok{    name: ["Mustermensch, Momo"]}
\CommentTok{    split: true}
\CommentTok{  ...    }
\end{Highlighting}
\end{Shaded}

\subsection{Examples}\label{examples}

\pandocbounded{\includegraphics[keepaspectratio]{https://github.com/typst/packages/raw/main/packages/preview/modern-acad-cv/0.1.1/assets/example1.png}}
\pandocbounded{\includegraphics[keepaspectratio]{https://github.com/typst/packages/raw/main/packages/preview/modern-acad-cv/0.1.1/assets/example2.png}}
\pandocbounded{\includegraphics[keepaspectratio]{https://github.com/typst/packages/raw/main/packages/preview/modern-acad-cv/0.1.1/assets/example3.png}}

\href{/app?template=modern-acad-cv&version=0.1.1}{Create project in app}

\subsubsection{How to use}\label{how-to-use}

Click the button above to create a new project using this template in
the Typst app.

You can also use the Typst CLI to start a new project on your computer
using this command:

\begin{verbatim}
typst init @preview/modern-acad-cv:0.1.1
\end{verbatim}

\includesvg[width=0.16667in,height=0.16667in]{/assets/icons/16-copy.svg}

\subsubsection{About}\label{about}

\begin{description}
\tightlist
\item[Author :]
\href{mailto:philipp.kleer@posteo.com}{bpkleer (Philipp Kleer)}
\item[License:]
MIT
\item[Current version:]
0.1.1
\item[Last updated:]
November 28, 2024
\item[First released:]
August 23, 2024
\item[Minimum Typst version:]
0.12.0
\item[Archive size:]
25.1 kB
\href{https://packages.typst.org/preview/modern-acad-cv-0.1.1.tar.gz}{\pandocbounded{\includesvg[keepaspectratio]{/assets/icons/16-download.svg}}}
\item[Repository:]
\href{https://github.com/bpkleer/typst-modern-acad-cv}{GitHub}
\item[Categor y :]
\begin{itemize}
\tightlist
\item[]
\item
  \pandocbounded{\includesvg[keepaspectratio]{/assets/icons/16-user.svg}}
  \href{https://typst.app/universe/search/?category=cv}{CV}
\end{itemize}
\end{description}

\subsubsection{Where to report issues?}\label{where-to-report-issues}

This template is a project of bpkleer (Philipp Kleer) . Report issues on
\href{https://github.com/bpkleer/typst-modern-acad-cv}{their repository}
. You can also try to ask for help with this template on the
\href{https://forum.typst.app}{Forum} .

Please report this template to the Typst team using the
\href{https://typst.app/contact}{contact form} if you believe it is a
safety hazard or infringes upon your rights.

\phantomsection\label{versions}
\subsubsection{Version history}\label{version-history}

\begin{longtable}[]{@{}ll@{}}
\toprule\noalign{}
Version & Release Date \\
\midrule\noalign{}
\endhead
\bottomrule\noalign{}
\endlastfoot
0.1.1 & November 28, 2024 \\
\href{https://typst.app/universe/package/modern-acad-cv/0.1.0/}{0.1.0} &
August 23, 2024 \\
\end{longtable}

Typst GmbH did not create this template and cannot guarantee correct
functionality of this template or compatibility with any version of the
Typst compiler or app.
