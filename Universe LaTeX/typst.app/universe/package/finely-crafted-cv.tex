\title{typst.app/universe/package/finely-crafted-cv}

\phantomsection\label{banner}
\phantomsection\label{template-thumbnail}
\pandocbounded{\includegraphics[keepaspectratio]{https://packages.typst.org/preview/thumbnails/finely-crafted-cv-0.1.0-small.webp}}

\section{finely-crafted-cv}\label{finely-crafted-cv}

{ 0.1.0 }

A modern résumé/curriculum vitæ template with high attention to
detail.

\href{/app?template=finely-crafted-cv&version=0.1.0}{Create project in
app}

\phantomsection\label{readme}
This Typst template provides a clean and professional format for
creating a curriculum vitae (CV) or résumé. It comes with functions
and styles to help you easily generate a well-structured document,
complete with sections for education, experience, skills, and more.

\subsection{Features}\label{features}

\begin{itemize}
\tightlist
\item
  \textbf{Modern Design:} Aesthetic and professional layout designed for
  readability.
\item
  \textbf{Responsive Header \& Footer:} Includes contact information
  dynamically.
\end{itemize}

\subsection{Usage}\label{usage}

To use this template, import it with the version number and utilize the
\texttt{\ resume\ } or \texttt{\ cv\ } function:

\begin{Shaded}
\begin{Highlighting}[]
\NormalTok{\#import "@preview/finely{-}crafted{-}cv:0.1.0": *}

\NormalTok{\#show: resume.with(}
\NormalTok{  name: "Amira Patel",}
\NormalTok{  tagline: "Innovative marine biologist with 15+ years of experience in ocean conservation and research.",}
\NormalTok{  keywords: "marine biology, conservation, research, education, patents",}
\NormalTok{  email: "amira.patel@oceandreams.org",}
\NormalTok{  phone: "+1{-}305{-}555{-}7890",}
\NormalTok{  linkedin{-}username: "amirapatel",}
\NormalTok{  thumbnail: image("assets/my{-}qr{-}code.svg"),}
\NormalTok{)}

\NormalTok{= Introduction}

\NormalTok{\#lorem(100)}

\NormalTok{= Experience}

\NormalTok{\#company{-}heading("Some Company", start: "March 2018", end: "Present", icon: image("icons/earth.svg"))[}
\NormalTok{  \#job{-}heading("Some Job", location: "Some Location")[}
\NormalTok{    {-} Here is an achievement}
\NormalTok{    {-} Here\textquotesingle{}s another one.}
\NormalTok{  ]}
\NormalTok{  // companies can have multiple jobs}
\NormalTok{  \#job{-}heading("First Job", location: "Some Location")[}
\NormalTok{    {-} Here is an achievement}
\NormalTok{    {-} Here\textquotesingle{}s another one.}
\NormalTok{  ]}
\NormalTok{]}

\NormalTok{// for companies which have less detail, you can use the \textasciigrave{}comment\textasciigrave{} instead of a}
\NormalTok{// body of tasks, as follows:}
\NormalTok{\#company{-}heading("Another Company", start: "July 2005", end: "August 2009", icon: image("icons/microscope.svg"))[}
\NormalTok{  \#job{-}heading("Another Job", location: "Another Location",}
\NormalTok{    comment: [Contributed to 7 published studies. \#footnote[Visit https://amirapatel.org/publications for full list of publications.]]}
\NormalTok{  )[]}
\NormalTok{]}

\NormalTok{= Education}

\NormalTok{// school{-}heading is an alias for company{-}heading, accepts the same parameters as company{-}heading}
\NormalTok{\#school{-}heading("University of California, San Diego", start: "Fall 2001", end: "Spring 2005", icon: image("icons/graduation{-}cap.svg"))[}
\NormalTok{  // degree{-}heading is an alias for job{-}heading, accepts the same parameters as job{-}heading}
\NormalTok{  \#degree{-}heading("Ph.D. in Marine Biology")[]}
\NormalTok{]}
\end{Highlighting}
\end{Shaded}

\subsection{Functions and Parameters}\label{functions-and-parameters}

\subsubsection{\texorpdfstring{\texttt{\ resume\ } or
\texttt{\ cv\ }}{ resume  or  cv }}\label{resume-or-cv}

This is the main function to create a CV document.

\begin{itemize}
\tightlist
\item
  \textbf{Parameters:}

  \begin{itemize}
  \tightlist
  \item
    \texttt{\ name\ } : (String) Your full name. Default is “YOUR NAME
    HERE�.
  \item
    \texttt{\ tagline\ } : (String) A brief description of your
    professional identity or mission.
  \item
    \texttt{\ paper\ } : (String) The paper size, default is
    “us-letter�.
  \item
    \texttt{\ heading-font\ } : (Font) Font for headings, customizable.
  \item
    \texttt{\ body-font\ } : (Font) Font for body text, customizable.
  \item
    \texttt{\ body-size\ } : (Size) Font size for body text.
  \item
    \texttt{\ email\ } : (String) Your email address.
  \item
    \texttt{\ phone\ } : (String) Your phone number.
  \item
    \texttt{\ linkedin-username\ } : (String) Your LinkedIn username.
  \item
    \texttt{\ keywords\ } : (String) Keywords for searchability.
  \item
    \texttt{\ thumbnail\ } : (Image) Thumbnail or QR code image,
    optional.
  \item
    \texttt{\ body\ } : (Block) The main content of your CV.
  \end{itemize}
\end{itemize}

\subsubsection{\texorpdfstring{\texttt{\ company-heading\ }}{ company-heading }}\label{company-heading}

Used to create a heading for a company or organization.

\begin{itemize}
\tightlist
\item
  \textbf{Parameters:}

  \begin{itemize}
  \tightlist
  \item
    \texttt{\ name\ } : (String) Name of the company.
  \item
    \texttt{\ start\ } : (String) Start date.
  \item
    \texttt{\ end\ } : (String) End date, optional.
  \item
    \texttt{\ icon\ } : (Image) Icon image associated with the company,
    optional.
  \item
    \texttt{\ body\ } : (Block) Content related to the company role or
    tasks.
  \end{itemize}
\end{itemize}

\subsubsection{\texorpdfstring{\texttt{\ job-heading\ }}{ job-heading }}\label{job-heading}

Defines a job title within a company heading.

\begin{itemize}
\tightlist
\item
  \textbf{Parameters:}

  \begin{itemize}
  \tightlist
  \item
    \texttt{\ title\ } : (String) Job title.
  \item
    \texttt{\ location\ } : (String) Location of the job, optional.
  \item
    \texttt{\ start\ } : (String) Start date, optional.
  \item
    \texttt{\ end\ } : (String) End date, optional.
  \item
    \texttt{\ comment\ } : (String) Additional comments or notes,
    optional.
  \item
    \texttt{\ body\ } : (Block) Tasks or responsibilities.
  \end{itemize}
\end{itemize}

\subsubsection{\texorpdfstring{\texttt{\ school-heading\ }}{ school-heading }}\label{school-heading}

Alias for \texttt{\ company-heading\ } , used for educational
institutions.

\subsubsection{\texorpdfstring{\texttt{\ degree-heading\ }}{ degree-heading }}\label{degree-heading}

Alias for \texttt{\ job-heading\ } , used for academic degrees or
certifications.

\subsection{License}\label{license}

This template is released under the MIT License.

\href{/app?template=finely-crafted-cv&version=0.1.0}{Create project in
app}

\subsubsection{How to use}\label{how-to-use}

Click the button above to create a new project using this template in
the Typst app.

You can also use the Typst CLI to start a new project on your computer
using this command:

\begin{verbatim}
typst init @preview/finely-crafted-cv:0.1.0
\end{verbatim}

\includesvg[width=0.16667in,height=0.16667in]{/assets/icons/16-copy.svg}

\subsubsection{About}\label{about}

\begin{description}
\tightlist
\item[Author :]
\href{mailto:steve@waits.net}{Stephen Waits}
\item[License:]
MIT
\item[Current version:]
0.1.0
\item[Last updated:]
October 22, 2024
\item[First released:]
October 22, 2024
\item[Archive size:]
28.5 kB
\href{https://packages.typst.org/preview/finely-crafted-cv-0.1.0.tar.gz}{\pandocbounded{\includesvg[keepaspectratio]{/assets/icons/16-download.svg}}}
\item[Repository:]
\href{https://github.com/swaits/typst-collection}{GitHub}
\item[Discipline s :]
\begin{itemize}
\tightlist
\item[]
\item
  \href{https://typst.app/universe/search/?discipline=business}{Business}
\item
  \href{https://typst.app/universe/search/?discipline=communication}{Communication}
\end{itemize}
\item[Categor y :]
\begin{itemize}
\tightlist
\item[]
\item
  \pandocbounded{\includesvg[keepaspectratio]{/assets/icons/16-user.svg}}
  \href{https://typst.app/universe/search/?category=cv}{CV}
\end{itemize}
\end{description}

\subsubsection{Where to report issues?}\label{where-to-report-issues}

This template is a project of Stephen Waits . Report issues on
\href{https://github.com/swaits/typst-collection}{their repository} .
You can also try to ask for help with this template on the
\href{https://forum.typst.app}{Forum} .

Please report this template to the Typst team using the
\href{https://typst.app/contact}{contact form} if you believe it is a
safety hazard or infringes upon your rights.

\phantomsection\label{versions}
\subsubsection{Version history}\label{version-history}

\begin{longtable}[]{@{}ll@{}}
\toprule\noalign{}
Version & Release Date \\
\midrule\noalign{}
\endhead
\bottomrule\noalign{}
\endlastfoot
0.1.0 & October 22, 2024 \\
\end{longtable}

Typst GmbH did not create this template and cannot guarantee correct
functionality of this template or compatibility with any version of the
Typst compiler or app.
