\title{typst.app/universe/package/stack-pointer}

\phantomsection\label{banner}
\section{stack-pointer}\label{stack-pointer}

{ 0.1.0 }

A library for visualizing the execution of (imperative) computer
programs

{ } Featured Package

\phantomsection\label{readme}
Stack Pointer is a library for visualizing the execution of (imperative)
computer programs, particularly in terms of effects on the call stack:
stack frames and local variables therein.

Stack Pointer lets you represent an example program (e.g. a C or Java
program) using typst code with minimal hassle, and get the execution
state of that program at different points in time. For example, the
following C program

\begin{Shaded}
\begin{Highlighting}[]
\DataTypeTok{int}\NormalTok{ main}\OperatorTok{()} \OperatorTok{\{}
  \DataTypeTok{int}\NormalTok{ x }\OperatorTok{=}\NormalTok{ foo}\OperatorTok{();}
  \ControlFlowTok{return} \DecValTok{0}\OperatorTok{;}
\OperatorTok{\}}

\DataTypeTok{int}\NormalTok{ foo}\OperatorTok{()} \OperatorTok{\{}
  \ControlFlowTok{return} \DecValTok{0}\OperatorTok{;}
\OperatorTok{\}}
\end{Highlighting}
\end{Shaded}

would be represented by the following Typst code (see the
\href{https://github.com/typst/packages/raw/main/packages/preview/stack-pointer/0.1.0/docs/manual.pdf}{manual}
for a detailled explanation):

\begin{Shaded}
\begin{Highlighting}[]
\NormalTok{\#let steps = execute(\{}
\NormalTok{  let foo() = func("foo", 6, l =\textgreater{} \{}
\NormalTok{    l(0)}
\NormalTok{    l(1); retval(0)}
\NormalTok{  \})}
\NormalTok{  let main() = func("main", 1, l =\textgreater{} \{}
\NormalTok{    l(0)}
\NormalTok{    l(1)}
\NormalTok{    let (x, ..rest) = foo(); rest}
\NormalTok{    l(1, push("x", x))}
\NormalTok{    l(2)}
\NormalTok{  \})}
\NormalTok{  main(); l(none)}
\NormalTok{\})}
\end{Highlighting}
\end{Shaded}

The \texttt{\ steps\ } variable now contains an array, where each
element corresponds to one of the mentioned lines of code.

Take a look at
\href{https://github.com/typst/packages/raw/main/packages/preview/stack-pointer/0.1.0/gallery/sum.pdf}{this
complete example} of using Stack Pointer together with
\href{https://polylux.dev/book/}{Polylux} .

\subsubsection{How to add}\label{how-to-add}

Copy this into your project and use the import as
\texttt{\ stack-pointer\ }

\begin{verbatim}
#import "@preview/stack-pointer:0.1.0"
\end{verbatim}

\includesvg[width=0.16667in,height=0.16667in]{/assets/icons/16-copy.svg}

Check the docs for
\href{https://typst.app/docs/reference/scripting/\#packages}{more
information on how to import packages} .

\subsubsection{About}\label{about}

\begin{description}
\tightlist
\item[Author :]
\href{https://github.com/SillyFreak/}{Clemens Koza}
\item[License:]
MIT
\item[Current version:]
0.1.0
\item[Last updated:]
July 15, 2024
\item[First released:]
July 15, 2024
\item[Archive size:]
4.29 kB
\href{https://packages.typst.org/preview/stack-pointer-0.1.0.tar.gz}{\pandocbounded{\includesvg[keepaspectratio]{/assets/icons/16-download.svg}}}
\item[Repository:]
\href{https://github.com/SillyFreak/typst-stack-pointer}{GitHub}
\item[Discipline :]
\begin{itemize}
\tightlist
\item[]
\item
  \href{https://typst.app/universe/search/?discipline=computer-science}{Computer
  Science}
\end{itemize}
\item[Categor ies :]
\begin{itemize}
\tightlist
\item[]
\item
  \pandocbounded{\includesvg[keepaspectratio]{/assets/icons/16-code.svg}}
  \href{https://typst.app/universe/search/?category=scripting}{Scripting}
\item
  \pandocbounded{\includesvg[keepaspectratio]{/assets/icons/16-presentation.svg}}
  \href{https://typst.app/universe/search/?category=presentation}{Presentation}
\end{itemize}
\end{description}

\subsubsection{Where to report issues?}\label{where-to-report-issues}

This package is a project of Clemens Koza . Report issues on
\href{https://github.com/SillyFreak/typst-stack-pointer}{their
repository} . You can also try to ask for help with this package on the
\href{https://forum.typst.app}{Forum} .

Please report this package to the Typst team using the
\href{https://typst.app/contact}{contact form} if you believe it is a
safety hazard or infringes upon your rights.

\phantomsection\label{versions}
\subsubsection{Version history}\label{version-history}

\begin{longtable}[]{@{}ll@{}}
\toprule\noalign{}
Version & Release Date \\
\midrule\noalign{}
\endhead
\bottomrule\noalign{}
\endlastfoot
0.1.0 & July 15, 2024 \\
\end{longtable}

Typst GmbH did not create this package and cannot guarantee correct
functionality of this package or compatibility with any version of the
Typst compiler or app.
