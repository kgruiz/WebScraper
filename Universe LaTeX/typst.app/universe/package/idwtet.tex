\title{typst.app/universe/package/idwtet}

\phantomsection\label{banner}
\section{idwtet}\label{idwtet}

{ 0.3.0 }

Package for uniform, correct and simplified typst code demonstration.

\phantomsection\label{readme}
The name \texttt{\ idwtet\ } stands for “I Don’t Wanna Type
Everything Twice�. It provides a \texttt{\ typst-ex\ } and a
\texttt{\ typst-ex-code\ } codeblock, which \emph{shows \textbf{and}
executes} typst code.

It is meant for code demonstration, e.g. when publishing a package, and
provides some niceties:

\begin{itemize}
\tightlist
\item
  the code should always be correct in the examples: As the example code
  is used for the code block, but also for evaluation, there is no need
  to write it twice
\item
  easy configuration with simple, uniform and good look
\end{itemize}

However, there are some limitations:

\begin{itemize}
\tightlist
\item
  Every code block has its own local scope and the default behaviour is
  that variables are not reachable from outside. A similar restriction
  lies on import statements. That is why, there is the
  \texttt{\ eval-scope\ } argument, which captures variables and
  simulates global variables. Additionally, a \texttt{\ typst\ }
  codeblock is provided for a consistent look.
\item
  Locality can be displayed to the users by automatically wrapping code
  in \texttt{\ typst-ex-code\ } , but \texttt{\ typst-ex\ } does not
  provide such functionality. It might thus be difficult for users to
  understand code examples this way.
\item
  The page width has to be defined in absolute terms. It is quite nice,
  for a showcase, to take the least possible space, but tracking the
  widths of all boxes and then setting the page width accordingly is not
  (yet) possible.
\end{itemize}

\subsection{Usage}\label{usage}

Only one function is defined,
\texttt{\ init(body,\ bcolor:\ luma(210),\ inset:\ 5pt,\ border:\ 2pt,\ radius:\ 2pt,\ content-font:\ "linux\ libertine",\ code-font-size:\ 9pt,\ content-font-size:\ 11pt,\ code-return-box:\ true,\ wrap-code:\ false,\ eval-scope:\ (:),\ escape-bracket:\ "\%")\ }
, which is supposed to be used with a show rule.

Then raw codeblocks (with \texttt{\ block=true\ } ) of the languages
\texttt{\ typst\ } , \texttt{\ typst-ex\ } , \texttt{\ typst-code\ } and
\texttt{\ typst-ex-code\ } are modified. The main feature of this
package are \texttt{\ typst-ex\ } and \texttt{\ typst-ex-code\ } . The
\texttt{\ typst\ } and \texttt{\ typst-code\ } blocks do not evaluate
anything, but their design fits that of the others.

The parameters of \texttt{\ init\ } are:

\begin{itemize}
\tightlist
\item
  \texttt{\ body\ } : for usage with show rule, hence the whole
  document.
\item
  \texttt{\ bcolor\ } : the background- (and border-) color of the
  blocks
\item
  \texttt{\ inset\ } : inset param of code and content blocks, should be
  ≥ 2pt
\item
  \texttt{\ border\ } : border thickness
\item
  \texttt{\ radius\ } : block radius
\item
  \texttt{\ content-font\ } : The font used in the previewed content /
  result.
\item
  \texttt{\ code-font-size\ } : The fontsize used in the code blocks.
\item
  \texttt{\ content-font-size\ } : The fontsize used in the preview
  content / result.
\item
  \texttt{\ code-return-box\ } : If to show the code return type on
  \texttt{\ typst-ex-code\ } blocks.
\item
  \texttt{\ wrap-code\ } : If to wrap the code in \texttt{\ \#\{\ } and
  \texttt{\ \}\ } , to symbolize local scope.
\item
  \texttt{\ eval-scope\ } : A dictionary with the keys as the variable
  names and the values as another dictionary with keys
  \texttt{\ value\ } and \texttt{\ code\ } , both of these are optional.
  The former has the defined value, the latter the code recreate the
  variable, for usage in the code blocks.
\item
  \texttt{\ escape-bracket\ } : The text to wrap a variable with, to
  access the \texttt{\ code\ } part of a \texttt{\ eval-scope\ }
  variable.
\end{itemize}

\subsection{Hiding code and
replacements}\label{hiding-code-and-replacements}

There are currently two methods to change the code:

\begin{itemize}
\tightlist
\item
  use the \texttt{\ eval-scope\ } argument from the \texttt{\ init\ }
  function. There is a \texttt{\ code\ } field in the dictionaries,
  which enables the usage of the key escaped in
  \texttt{\ escape-bracket\ } to be replaced in the codeblock code half
  and to be removed in the codeblock result half, as the value is given
  via scope. Take a look at the example below, where
  \texttt{\ \%ouset\%\ } is used this way.
\item
  use the \texttt{\ ENDHIDDEN\ } feature. When escaped in
  \texttt{\ escape-bracket\ } , everything above the statement is
  removed from the codeblock code half BUT everything (without the
  \texttt{\ ENDHIDDEN\ } statement) is evaluated. Take a look at the
  example in the examples folder.
\end{itemize}

\subsection{Example}\label{example}

\begin{Shaded}
\begin{Highlighting}[]
\NormalTok{\#set page(margin: 0.5cm, width: 14cm, height: auto)}
\NormalTok{\#import "@preview/idwtet:0.1.0"}
\NormalTok{\#show: idwtet.init.with(eval{-}scope: (}
\NormalTok{  ouset: (}
\NormalTok{    value: \{import "@preview/ouset:0.1.1": ouset; ouset\},}
\NormalTok{    code: "\#import \textbackslash{}"@preview/ouset:0.1.1\textbackslash{}": ouset"}
\NormalTok{  )}
\NormalTok{))}

\NormalTok{== ouset package \#text(gray)[(v0.1.1)]}
\NormalTok{\textasciigrave{}\textasciigrave{}\textasciigrave{}typst{-}ex}
\NormalTok{\%ouset\%}
\NormalTok{$}
\NormalTok{"Expression 1" ouset(\&, \textless{}==\textgreater{}, "Theorem 1") "Expression 2"\textbackslash{}}
\NormalTok{               ouset(\&, ==\textgreater{},, "Theorem 7") "Expression 3"}
\NormalTok{$}
\NormalTok{\textasciigrave{}\textasciigrave{}\textasciigrave{}}
\NormalTok{Or something like}
\NormalTok{\textasciigrave{}\textasciigrave{}\textasciigrave{}typst{-}ex{-}code}
\NormalTok{let a = range(10)}
\NormalTok{a}
\NormalTok{\textasciigrave{}\textasciigrave{}\textasciigrave{}}
\end{Highlighting}
\end{Shaded}

Further examples are given in the repo example folder.

\subsubsection{How to add}\label{how-to-add}

Copy this into your project and use the import as \texttt{\ idwtet\ }

\begin{verbatim}
#import "@preview/idwtet:0.3.0"
\end{verbatim}

\includesvg[width=0.16667in,height=0.16667in]{/assets/icons/16-copy.svg}

Check the docs for
\href{https://typst.app/docs/reference/scripting/\#packages}{more
information on how to import packages} .

\subsubsection{About}\label{about}

\begin{description}
\tightlist
\item[Author :]
Ludwig Austermann
\item[License:]
MIT
\item[Current version:]
0.3.0
\item[Last updated:]
September 25, 2023
\item[First released:]
August 19, 2023
\item[Minimum Typst version:]
0.8.0
\item[Archive size:]
3.84 kB
\href{https://packages.typst.org/preview/idwtet-0.3.0.tar.gz}{\pandocbounded{\includesvg[keepaspectratio]{/assets/icons/16-download.svg}}}
\item[Repository:]
\href{https://github.com/ludwig-austermann/typst-idwtet}{GitHub}
\end{description}

\subsubsection{Where to report issues?}\label{where-to-report-issues}

This package is a project of Ludwig Austermann . Report issues on
\href{https://github.com/ludwig-austermann/typst-idwtet}{their
repository} . You can also try to ask for help with this package on the
\href{https://forum.typst.app}{Forum} .

Please report this package to the Typst team using the
\href{https://typst.app/contact}{contact form} if you believe it is a
safety hazard or infringes upon your rights.

\phantomsection\label{versions}
\subsubsection{Version history}\label{version-history}

\begin{longtable}[]{@{}ll@{}}
\toprule\noalign{}
Version & Release Date \\
\midrule\noalign{}
\endhead
\bottomrule\noalign{}
\endlastfoot
0.3.0 & September 25, 2023 \\
\href{https://typst.app/universe/package/idwtet/0.2.0/}{0.2.0} & August
19, 2023 \\
\end{longtable}

Typst GmbH did not create this package and cannot guarantee correct
functionality of this package or compatibility with any version of the
Typst compiler or app.
