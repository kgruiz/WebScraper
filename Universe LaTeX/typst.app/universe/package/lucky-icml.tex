\title{typst.app/universe/package/lucky-icml}

\phantomsection\label{banner}
\phantomsection\label{template-thumbnail}
\pandocbounded{\includegraphics[keepaspectratio]{https://packages.typst.org/preview/thumbnails/lucky-icml-0.2.1-small.webp}}

\section{lucky-icml}\label{lucky-icml}

{ 0.2.1 }

ICML-style paper template to publish at conferences for International
Conference on Machine Learning

\href{/app?template=lucky-icml&version=0.2.1}{Create project in app}

\phantomsection\label{readme}
\subsection{Usage}\label{usage}

You can use this template in the Typst web app by clicking \emph{Start
from template} on the dashboard and searching for
\texttt{\ lucky-icml\ } .

Alternatively, you can use the CLI to kick this project off using the
command

\begin{Shaded}
\begin{Highlighting}[]
\NormalTok{typst init @preview/lucky{-}icml}
\end{Highlighting}
\end{Shaded}

Typst will create a new directory with all the files needed to get you
started.

\subsection{Configuration}\label{configuration}

This template exports the \texttt{\ icml2024\ } function with the
following named arguments.

\begin{itemize}
\tightlist
\item
  \texttt{\ title\ } : The paper’s title as content.
\item
  \texttt{\ authors\ } : An array of author dictionaries. Each of the
  author dictionaries must have a name key and can have the keys
  department, organization, location, and email.
\item
  \texttt{\ abstract\ } : The content of a brief summary of the paper or
  none. Appears at the top under the title.
\item
  \texttt{\ bibliography\ } : The result of a call to the bibliography
  function or none. The function also accepts a single, positional
  argument for the body of the paper.
\item
  \texttt{\ accepted\ } : If this is set to \texttt{\ false\ } then
  anonymized ready for submission document is produced;
  \texttt{\ accepted:\ true\ } produces camera-redy version. If the
  argument is set to \texttt{\ none\ } then preprint version is produced
  (can be uploaded to arXiv).
\end{itemize}

The template will initialize your package with a sample call to the
\texttt{\ icml2024\ } function in a show rule. If you want to change an
existing project to use this template, you can add a show rule at the
top of your file.

\subsection{Issues}\label{issues}

This template is developed at
\href{https://github.com/daskol/typst-templates}{daskol/typst-templates}
repo. Please report all issues there.

\subsubsection{Running Title}\label{running-title}

\begin{enumerate}
\tightlist
\item
  Runing title should be 10pt above the main text. With top margin 1in
  it gives that a solid line should be located at 62pt. Actual, position
  is 57pt in the original template.
\item
  Default value between header ruler and header text baseline is 4pt in
  \texttt{\ fancyhdr\ } . But actual value is 3pt due to thickness of a
  ruler in 1pt.
\end{enumerate}

\subsubsection{Page Numbering}\label{page-numbering}

\begin{enumerate}
\tightlist
\item
  Basis line of page number should be located 25pt below of main text.
  There is a discrepancy in about \textasciitilde1pt.
\end{enumerate}

\subsubsection{Heading}\label{heading}

\begin{enumerate}
\tightlist
\item
  Required space after level 3 headers is 0.1in or 7.2pt. Actual space
  size is large (e.g. distance between Section 2.3.1 header and text
  after it about 12pt).
\end{enumerate}

\subsubsection{Figures and Tables}\label{figures-and-tables}

\begin{enumerate}
\tightlist
\item
  At the moment Typst has limited support for multi-column documents. It
  allows define multi-column blocks and documents but there is no
  ability to typeset complex layout (e.g. page width figures or tables
  in two-column documents).
\end{enumerate}

\subsubsection{Citations and References}\label{citations-and-references}

\begin{enumerate}
\item
  There is a small bug in CSL processor which fails to recognize
  bibliography entries with \texttt{\ chapter\ } field. It is already
  report and will be fixed in the future.
\item
  There is no suitable bibliography style so we use default APA while
  ICML requires APA-like style but not exact APA. The difference is that
  ICML APA-like style places entry year at the end of reference entry.
  In order to fix the issue, we need to describe ICML bibliography style
  in CSL-format.
\item
  In the original template links are colored with dark blue. We can
  reproduce appearance with some additional effort. First of all,
  \texttt{\ icml2024.csl\ } shoule be updated as follows.

\begin{verbatim}
diff --git a/icml/icml2024.csl b/icml/icml2024.csl
index 3b9e9a2..3fe9f74 100644
--- a/icml/icml2024.csl
+++ b/icml/icml2024.csl
@@ -1648,7 +1648,8 @@
       
       
     
-    
+    
+    
       
         
         
\end{verbatim}

  Then instead of convenient citation shortcut
  \texttt{\ @cite-key1\ @cite-key2\ } , one should use special procedure
  \texttt{\ refer\ } with variadic arguments.

\begin{Shaded}
\begin{Highlighting}[]
\NormalTok{\#refer(\textless{}cite{-}key1\textgreater{}, \textless{}cite{-}key2\textgreater{})}
\end{Highlighting}
\end{Shaded}
\end{enumerate}

\href{/app?template=lucky-icml&version=0.2.1}{Create project in app}

\subsubsection{How to use}\label{how-to-use}

Click the button above to create a new project using this template in
the Typst app.

You can also use the Typst CLI to start a new project on your computer
using this command:

\begin{verbatim}
typst init @preview/lucky-icml:0.2.1
\end{verbatim}

\includesvg[width=0.16667in,height=0.16667in]{/assets/icons/16-copy.svg}

\subsubsection{About}\label{about}

\begin{description}
\tightlist
\item[Author :]
\href{mailto:d.bershatsky2@skoltech.ru}{Daniel Bershatsky}
\item[License:]
MIT
\item[Current version:]
0.2.1
\item[Last updated:]
March 19, 2024
\item[First released:]
March 19, 2024
\item[Minimum Typst version:]
0.10.0
\item[Archive size:]
51.2 kB
\href{https://packages.typst.org/preview/lucky-icml-0.2.1.tar.gz}{\pandocbounded{\includesvg[keepaspectratio]{/assets/icons/16-download.svg}}}
\item[Repository:]
\href{https://github.com/daskol/typst-templates}{GitHub}
\item[Discipline s :]
\begin{itemize}
\tightlist
\item[]
\item
  \href{https://typst.app/universe/search/?discipline=computer-science}{Computer
  Science}
\item
  \href{https://typst.app/universe/search/?discipline=mathematics}{Mathematics}
\end{itemize}
\item[Categor y :]
\begin{itemize}
\tightlist
\item[]
\item
  \pandocbounded{\includesvg[keepaspectratio]{/assets/icons/16-atom.svg}}
  \href{https://typst.app/universe/search/?category=paper}{Paper}
\end{itemize}
\end{description}

\subsubsection{Where to report issues?}\label{where-to-report-issues}

This template is a project of Daniel Bershatsky . Report issues on
\href{https://github.com/daskol/typst-templates}{their repository} . You
can also try to ask for help with this template on the
\href{https://forum.typst.app}{Forum} .

Please report this template to the Typst team using the
\href{https://typst.app/contact}{contact form} if you believe it is a
safety hazard or infringes upon your rights.

\phantomsection\label{versions}
\subsubsection{Version history}\label{version-history}

\begin{longtable}[]{@{}ll@{}}
\toprule\noalign{}
Version & Release Date \\
\midrule\noalign{}
\endhead
\bottomrule\noalign{}
\endlastfoot
0.2.1 & March 19, 2024 \\
\end{longtable}

Typst GmbH did not create this template and cannot guarantee correct
functionality of this template or compatibility with any version of the
Typst compiler or app.
