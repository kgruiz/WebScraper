\title{typst.app/universe/package/zero}

\phantomsection\label{banner}
\section{zero}\label{zero}

{ 0.3.0 }

Advanced scientific number formatting.

{ } Featured Package

\phantomsection\label{readme}
\emph{Advanced scientific number formatting.}

\href{https://typst.app/universe/package/zero}{\pandocbounded{\includegraphics[keepaspectratio]{https://img.shields.io/badge/dynamic/toml?url=https\%3A\%2F\%2Fraw.githubusercontent.com\%2FMc-Zen\%2Fzero\%2Fv0.3.0\%2Ftypst.toml&query=\%24.package.version&prefix=v&logo=typst&label=package&color=239DAD}}}
\href{https://github.com/Mc-Zen/zero/actions/workflows/run_tests.yml}{\pandocbounded{\includesvg[keepaspectratio]{https://github.com/Mc-Zen/zero/actions/workflows/run_tests.yml/badge.svg}}}
\href{https://github.com/Mc-Zen/zero/blob/main/LICENSE}{\pandocbounded{\includegraphics[keepaspectratio]{https://img.shields.io/badge/license-MIT-blue}}}

\begin{itemize}
\tightlist
\item
  \href{https://github.com/typst/packages/raw/main/packages/preview/zero/0.3.0/\#introduction}{Introduction}
\item
  \href{https://github.com/typst/packages/raw/main/packages/preview/zero/0.3.0/\#quick-demo}{Quick
  Demo}
\item
  \href{https://github.com/typst/packages/raw/main/packages/preview/zero/0.3.0/\#documentation}{Documentation}
\item
  \href{https://github.com/typst/packages/raw/main/packages/preview/zero/0.3.0/\#table-alignment}{Table
  alignment}
\item
  \href{https://github.com/typst/packages/raw/main/packages/preview/zero/0.3.0/\#zero-for-third-party-packages}{Zero
  for third-party packages}
\end{itemize}

\subsection{Introduction}\label{introduction}

Proper number formatting requires some love for detail to guarantee a
readable and clear output. This package provides tools to ensure
consistent formatting and to simplify the process of following
established publication practices. Key features are

\begin{itemize}
\tightlist
\item
  \textbf{standardized} formatting,
\item
  digit
  \href{https://github.com/typst/packages/raw/main/packages/preview/zero/0.3.0/\#grouping}{\textbf{grouping}}
  , e.g., 299 792 458 instead of 299792458,
\item
  \textbf{plug-and-play} number
  \href{https://github.com/typst/packages/raw/main/packages/preview/zero/0.3.0/\#table-alignment}{\textbf{alignment
  in tables}} ,
\item
  quick scientific notation, e.g., \texttt{\ "2e4"\ } becomes
  2Ã---10â?´,
\item
  symmetric and asymmetric
  \href{https://github.com/typst/packages/raw/main/packages/preview/zero/0.3.0/\#specifying-uncertainties}{\textbf{uncertainties}}
  ,
\item
  \href{https://github.com/typst/packages/raw/main/packages/preview/zero/0.3.0/\#rounding}{\textbf{rounding}}
  in various modes,
\item
  and some specials for package authors.
\end{itemize}

A number in scientific notation consists of three parts of which the
latter two are optional. The first part is the \emph{mantissa} that may
consist of an \emph{integer} and a \emph{fractional} part. In many
fields of science, values are not known exactly and the corresponding
\emph{uncertainty} is then given along with the mantissa. Lastly, to
facilitate reading very large or small numbers, the mantissa may be
multiplied with a \emph{power} of 10 (or another base).

The anatomy of a formatted number is shown in the following figure.

\pandocbounded{\includegraphics[keepaspectratio]{https://github.com/user-attachments/assets/7ca9fa48-b732-4f4e-911f-b719e83305be}}

\subsection{Quick Demo}\label{quick-demo}

\begin{longtable}[]{@{}llll@{}}
\toprule\noalign{}
Code & Output & Code & Output \\
\midrule\noalign{}
\endhead
\bottomrule\noalign{}
\endlastfoot
\texttt{\ num("1.2e4")\ } & 1.2Ã---10â?´ & \texttt{\ num{[}1.2e4{]}\ } &
1.2Ã---10â?´ \\
\texttt{\ num("-5e-4")\ } & âˆ'5Ã---10â?»â?´ &
\texttt{\ num(fixed:\ -2){[}0.02{]}\ } & 2Ã---10â?»Â² \\
\texttt{\ num("9.81+-.01")\ } & 9.81±0.01 &
\texttt{\ num("9.81+0.02-.01")\ } & 9.81�²₋� \\
\texttt{\ num("9.81+-.01e2")\ } & (9.81±0.01)Ã---10² &
\texttt{\ num(base:\ 2){[}3e4{]}\ } & 3Ã---2â?´ \\
\end{longtable}

\subsection{Documentation}\label{documentation}

\begin{itemize}
\tightlist
\item
  \href{https://github.com/typst/packages/raw/main/packages/preview/zero/0.3.0/\#num}{Function
  \texttt{\ num\ }}
\item
  \href{https://github.com/typst/packages/raw/main/packages/preview/zero/0.3.0/\#grouping}{Grouping}
\item
  \href{https://github.com/typst/packages/raw/main/packages/preview/zero/0.3.0/\#rounding}{Rounding}
\item
  \href{https://github.com/typst/packages/raw/main/packages/preview/zero/0.3.0/\#specifying-uncertainties}{Uncertainties}
\item
  \href{https://github.com/typst/packages/raw/main/packages/preview/zero/0.3.0/\#table-alignment}{Table
  alignment}
\end{itemize}

\subsubsection{\texorpdfstring{\texttt{\ num\ }}{ num }}\label{num}

The function \texttt{\ num()\ } is the heart of \emph{Zero} . It
provides a wide range of number formatting utilities and its default
values are configurable via \texttt{\ set-num()\ } which takes the same
named arguments as \texttt{\ num()\ } .

\begin{Shaded}
\begin{Highlighting}[]
\NormalTok{\#let num(}
\NormalTok{  number:                 str | content | int | float | dictionary | array,}
\NormalTok{  digits:                 auto | int = auto,}
\NormalTok{  fixed:                  none | int = none,}

\NormalTok{  decimal{-}separator:      str = ".",}
\NormalTok{  product:                content = sym.times,}
\NormalTok{  tight:                  boolean = false,}
\NormalTok{  math:                   boolean = true,}
\NormalTok{  omit{-}unity{-}mantissa:    boolean = true,}
\NormalTok{  positive{-}sign:          boolean = false,}
\NormalTok{  positive{-}sign{-}exponent: boolean = false,}
\NormalTok{  base:                   int | content = 10,}
\NormalTok{  uncertainty{-}mode:       str = "separate",}
\NormalTok{  round:                  dictionary,}
\NormalTok{  group:                  dictionary,}
\NormalTok{)}
\end{Highlighting}
\end{Shaded}

\begin{itemize}
\tightlist
\item
  \texttt{\ number:\ str\ \textbar{} content\ \textbar{}\ int\ \textbar{}\ float\ \textbar{}\ array\ }
  : Number input; \texttt{\ str\ } is preferred. If the input is
  \texttt{\ content\ } , it may only contain text nodes. Numeric types
  \texttt{\ int\ } and \texttt{\ float\ } are supported but not
  encouraged because of information loss (e.g., the number of trailing
  “0� digits or the exponent). The remaining types
  \texttt{\ dictionary\ } and \texttt{\ array\ } are intended for
  advanced use, see
  \href{https://github.com/typst/packages/raw/main/packages/preview/zero/0.3.0/\#zero-for-third-party-packages}{below}
  .
\item
  \texttt{\ digits:\ auto\ \textbar{} int\ =\ auto\ } : Truncates the
  number at a given (positive) number of decimal places or pads the
  number with zeros if necessary. This is independent of
  \href{https://github.com/typst/packages/raw/main/packages/preview/zero/0.3.0/\#rounding}{rounding}
  .
\item
  \texttt{\ fixed:\ none\ \textbar{}\ int\ =\ none\ } : If not
  \texttt{\ none\ } , forces a fixed exponent. Additional exponents
  given in the number input are taken into account.
\item
  \texttt{\ decimal-separator:\ str\ =\ "."\ } : Specifies the marker
  that is used for separating integer and decimal part.
\item
  \texttt{\ product:\ content\ =\ sym.times\ } : Specifies the
  multiplication symbol used for scientific notation.
\item
  \texttt{\ tight:\ boolean\ =\ false\ } : If true, tight spacing is
  applied between operands (applies to Ã--- and ±).
\item
  \texttt{\ math:\ boolean\ =\ true\ } : If set to \texttt{\ false\ } ,
  the parts of the number won’t be wrapped in a
  \texttt{\ math.equation\ } wherever feasible. This makes it possible
  to use \texttt{\ num()\ } with non-math fonts to some extent. Powers
  are always rendered in math mode.
\item
  \texttt{\ omit-unity-mantissa:\ boolean\ =\ false\ } : Determines
  whether a mantissa of 1 is omitted in scientific notation, e.g., 10â?´
  instead of 1·10�.
\item
  \texttt{\ positive-sign:\ boolean\ =\ false\ } : If set to
  \texttt{\ true\ } , positive coefficients are shown with a + sign.
\item
  \texttt{\ positive-sign-exponent:\ boolean\ =\ false\ } : If set to
  \texttt{\ true\ } , positive exponents are shown with a + sign.
\item
  \texttt{\ base:\ int\ \textbar{}\ content\ =\ 10\ } : The base used
  for scientific power notation.
\item
  \texttt{\ uncertainty-mode:\ str\ =\ "separate"\ } : Selects one of
  the modes \texttt{\ "separate"\ } , \texttt{\ "compact"\ } , or
  \texttt{\ "compact-separator"\ } for displaying uncertainties. The
  different behaviors are shown below:
\end{itemize}

\begin{longtable}[]{@{}lll@{}}
\toprule\noalign{}
\texttt{\ "separate"\ } & \texttt{\ "compact"\ } &
\texttt{\ "compact-separator"\ } \\
\midrule\noalign{}
\endhead
\bottomrule\noalign{}
\endlastfoot
1.7±0.2 & 1.7(2) & 1.7(2) \\
6.2±2.1 & 6.2(21) & 6.2(2.1) \\
1.7��˙̇²₋₀.₠& 1.7�²₋₠& 1.7�²₋₠\\
1.7â?ºÂ²Ë™Ì‡â?°â‚‹â‚\ldots.â‚€ & 1.7â?ºÂ²â?°â‚‹â‚\ldots â‚€ &
1.7â?ºÂ²Ë™Ì‡â?°â‚‹â‚\ldots.â‚€ \\
\end{longtable}

\begin{itemize}
\tightlist
\item
  \texttt{\ round:\ dictionary\ } : You can provide one or more rounding
  options in a dictionary. Also see
  \href{https://github.com/typst/packages/raw/main/packages/preview/zero/0.3.0/\#rounding}{rounding}
  .
\item
  \texttt{\ group:\ dictionary\ } : You can provide one or more grouping
  options in a dictionary. Also see
  \href{https://github.com/typst/packages/raw/main/packages/preview/zero/0.3.0/\#grouping}{grouping}
  .
\end{itemize}

Configuration example:

\begin{Shaded}
\begin{Highlighting}[]
\NormalTok{\#set{-}num(product: math.dot, tight: true)}
\end{Highlighting}
\end{Shaded}

\subsubsection{Grouping}\label{grouping}

Digit grouping is important for keeping large figures readable. It is
customary to separate thousands with a thin space, a period, comma, or
an apostrophe (however, we discourage using a period or a comma to avoid
confusion since both are used for decimal separators in various
countries).

\pandocbounded{\includegraphics[keepaspectratio]{https://github.com/user-attachments/assets/1f53ae33-3e99-483d-ac6a-6e3cbed5484b}}

Digit grouping can be configured with the \texttt{\ set-group()\ }
function.

\begin{Shaded}
\begin{Highlighting}[]
\NormalTok{\#let set{-}group(}
\NormalTok{  size:       int = 3, }
\NormalTok{  separator:  content = sym.space.thin,}
\NormalTok{  threshold:  int = 5}
\NormalTok{)}
\end{Highlighting}
\end{Shaded}

\begin{itemize}
\tightlist
\item
  \texttt{\ size:\ int\ =\ 3\ } : Determines the size of the groups.
\item
  \texttt{\ separator:\ content\ =\ sym.space.thin\ } : Separator
  between groups.
\item
  \texttt{\ threshold:\ int\ =\ 5\ } : Necessary number of digits needed
  for digit grouping to kick in. Four-digit numbers for example are
  usually not grouped at all since they can still be read easily.
\end{itemize}

Configuration example:

\begin{Shaded}
\begin{Highlighting}[]
\NormalTok{\#set{-}group(separator: "\textquotesingle{}", threshold: 4)}
\end{Highlighting}
\end{Shaded}

Grouping can be turned off altogether by setting the
\texttt{\ threshold\ } to \texttt{\ calc.inf\ } .

\subsubsection{Rounding}\label{rounding}

Rounding can be configured with the \texttt{\ set-round()\ } function.

\begin{Shaded}
\begin{Highlighting}[]
\NormalTok{\#let set{-}round(}
\NormalTok{  mode:       none | str = none,}
\NormalTok{  precision:  int = 2,}
\NormalTok{  pad:        boolean = true,}
\NormalTok{  direction:  str = "nearest",}
\NormalTok{)}
\end{Highlighting}
\end{Shaded}

\begin{itemize}
\tightlist
\item
  \texttt{\ mode:\ none\ \textbar{} str\ =\ none\ } : Sets the
  rounding mode. The possible options are

  \begin{itemize}
  \tightlist
  \item
    \texttt{\ none\ } : Rounding is turned off.
  \item
    \texttt{\ "places"\ } : The number is rounded to the number of
    decimal places given by the \texttt{\ precision\ } parameter.
  \item
    \texttt{\ "figures"\ } : The number is rounded to a number of
    significant figures given by the \texttt{\ precision\ } parameter.
  \item
    \texttt{\ "uncertainty"\ } : Requires giving an uncertainty value.
    The uncertainty is rounded to significant figures according to the
    \texttt{\ precision\ } argument and then the number is rounded to
    the same number of decimal places as the uncertainty.
  \end{itemize}
\item
  \texttt{\ precision:\ int\ =\ 2\ } : The precision to round to. Also
  see parameter \texttt{\ mode\ } .
\item
  \texttt{\ pad:\ boolean\ =\ true\ } : Whether to pad the number with
  zeros if the number has fewer digits than the rounding precision.
\item
  \texttt{\ direction:\ str\ =\ "nearest"\ } : Sets the rounding
  direction.

  \begin{itemize}
  \tightlist
  \item
    \texttt{\ "nearest"\ } : Rounding takes place in the usual fashion,
    rounding to the nearer number, e.g., 2.34 â†' 2.3 and 2.36 â†' 2.4.
  \item
    \texttt{\ "down"\ } : Always rounds down, e.g., 2.38 â†' 2.3 and
    2.30 â†' 2.3.
  \item
    \texttt{\ "up"\ } : Always rounds up, e.g., 2.32 â†' 2.4 and 2.30
    â†' 2.3.
  \end{itemize}
\end{itemize}

\subsubsection{Specifying uncertainties}\label{specifying-uncertainties}

There are two ways of specifying uncertainties:

\begin{itemize}
\tightlist
\item
  Applying an uncertainty to the least significant digits using
  parentheses, e.g., \texttt{\ 2.3(4)\ } ,
\item
  Denoting an absolute uncertainty, e.g., \texttt{\ 2.3+-0.4\ } becomes
  2.3±0.4.
\end{itemize}

Zero supports both and can convert between these two, so that you can
pick the displayed style (configured via \texttt{\ uncertainty-mode\ } ,
see above) independently of the input style.

How do uncertainties interplay with exponents? The uncertainty needs to
come first, and the exponent applies to both the mantissa and the
uncertainty, e.g., \texttt{\ num("1.23+-.04e2")\ } becomes

(1.23 ± 0.03)Ã---10²

Note that the mantissa is now put in parentheses to disambiguate the
application of the power.

In some cases, the uncertainty is asymmetric which can be expressed via
\texttt{\ num("1.23+0.02-0.01")\ }

1.23��˙̇�²₋₀.₀�

\subsubsection{Table alignment}\label{table-alignment}

In scientific publication, presenting many numbers in a readable fashion
can be a difficult discipline. A good starting point is to align numbers
in a table at the decimal separator. With \emph{Zero} , this can be
accomplished by using \texttt{\ ztable\ } , a wrapper for the built-in
\texttt{\ table\ } function. It features an additional parameter
\texttt{\ format\ } which takes an array of \texttt{\ none\ } ,
\texttt{\ auto\ } , or \texttt{\ dictionary\ } values to turn on number
alignment for specific columns.

\begin{Shaded}
\begin{Highlighting}[]
\NormalTok{\#ztable(}
\NormalTok{  columns: 3,}
\NormalTok{  align: center,}
\NormalTok{  format: (none, auto, auto),}
\NormalTok{  $n$, $α$, $β$,}
\NormalTok{  [1], [3.45], [{-}11.1],}
\NormalTok{  ..}
\NormalTok{)}
\end{Highlighting}
\end{Shaded}

Non-number entries (e.g., in the header) are automatically recognized in
some cases and will not be aligned. In ambiguous cases, adding a leading
or trailing space tells \emph{Zero} not to apply alignment to this cell,
e.g., \texttt{\ {[}Angle\ {]}\ } instead of \texttt{\ {[}Angle{]}\ } .

\pandocbounded{\includegraphics[keepaspectratio]{https://github.com/user-attachments/assets/2effb7f0-0d9b-401a-92e1-20461d0c1fcb}}

In addition, you can prefix or suffix a numeral with content wrapped by
the function \texttt{\ nonum{[}{]}\ } to mark it as \emph{not belonging
to the number} . The remaining content may still be recognized as a
number and formatted/aligned accordingly.

\begin{Shaded}
\begin{Highlighting}[]
\NormalTok{\#ztable(}
\NormalTok{  format: (auto,),}
\NormalTok{  [\#nonum[€]123.0\#nonum(footnote[A special number])],}
\NormalTok{  [12.111],}
\NormalTok{)}
\end{Highlighting}
\end{Shaded}

\pandocbounded{\includegraphics[keepaspectratio]{https://github.com/user-attachments/assets/270ae789-2a8c-44a3-b3a9-0ca588bfbad3}}

Zero not only aligns numbers at the decimal point but also at the
uncertainty and exponent part. Moreover, by passing a
\texttt{\ dictionary\ } instead of \texttt{\ auto\ } , a set of
\texttt{\ num()\ } arguments to apply to all numbers in a column can be
specified.

\begin{Shaded}
\begin{Highlighting}[]
\NormalTok{\#ztable(}
\NormalTok{  columns: 4,}
\NormalTok{  align: center,}
\NormalTok{  format: (none, auto, auto, (digits: 1)),}
\NormalTok{  $n$, $α$, $β$, $γ$,}
\NormalTok{  [1], [3.45e2], [{-}11.1+{-}3], [0],}
\NormalTok{  ..}
\NormalTok{)}
\end{Highlighting}
\end{Shaded}

\pandocbounded{\includegraphics[keepaspectratio]{https://github.com/user-attachments/assets/c96941bc-f002-4b93-b2cd-705c8104682f}}

\subsection{Zero for third-party
packages}\label{zero-for-third-party-packages}

This package provides some useful extras for third-party packages that
generate formatted numbers (for example graphics libraries).

Instead of passing a \texttt{\ str\ } to \texttt{\ num()\ } , it is also
possible to pass a dictionary of the form

\begin{Shaded}
\begin{Highlighting}[]
\NormalTok{(}
\NormalTok{  mantissa:  str | int | float,}
\NormalTok{  e:         none | str,}
\NormalTok{  pm:        none | array}
\NormalTok{)}
\end{Highlighting}
\end{Shaded}

This way, parsing the number can be avoided which makes especially sense
for packages that generate numbers (e.g., tick labels for a diagram
axis) with independent mantissa and exponent.

Furthermore, \texttt{\ num()\ } also allows \texttt{\ array\ } arguments
for \texttt{\ number\ } which allows for more efficient batch-processing
of numbers with the same setup. In this case, the caller of the function
needs to provide \texttt{\ context\ } .

\subsection{Changelog}\label{changelog}

\subsubsection{Version 0.3.0}\label{version-0.3.0}

\begin{itemize}
\tightlist
\item
  Adds \texttt{\ nonum{[}{]}\ } function that can be used to mark
  content in cells as \emph{not belonging to the number} . The remaining
  content may still be recognized as a number and formatted/aligned
  accordingly. The content wrapped by \texttt{\ nonum{[}{]}\ } is
  preserved.
\item
  Fixes number alignment tables with new version Typst 0.12.
\end{itemize}

\subsubsection{Version 0.2.0}\label{version-0.2.0}

\begin{itemize}
\tightlist
\item
  Adds support for using non-math fonts for \texttt{\ num\ } via the
  option \texttt{\ math\ } . This can be activated by calling
  \texttt{\ \#set-num(math:\ false)\ } .
\item
  Performance improvements for both \texttt{\ num()\ } and
  \texttt{\ ztable(9)\ }
\end{itemize}

\subsubsection{Version 0.1.0}\label{version-0.1.0}

\subsubsection{How to add}\label{how-to-add}

Copy this into your project and use the import as \texttt{\ zero\ }

\begin{verbatim}
#import "@preview/zero:0.3.0"
\end{verbatim}

\includesvg[width=0.16667in,height=0.16667in]{/assets/icons/16-copy.svg}

Check the docs for
\href{https://typst.app/docs/reference/scripting/\#packages}{more
information on how to import packages} .

\subsubsection{About}\label{about}

\begin{description}
\tightlist
\item[Author :]
\href{https://github.com/Mc-Zen}{Mc-Zen}
\item[License:]
MIT
\item[Current version:]
0.3.0
\item[Last updated:]
October 28, 2024
\item[First released:]
September 16, 2024
\item[Minimum Typst version:]
0.11.0
\item[Archive size:]
15.7 kB
\href{https://packages.typst.org/preview/zero-0.3.0.tar.gz}{\pandocbounded{\includesvg[keepaspectratio]{/assets/icons/16-download.svg}}}
\item[Repository:]
\href{https://github.com/Mc-Zen/zero}{GitHub}
\item[Categor ies :]
\begin{itemize}
\tightlist
\item[]
\item
  \pandocbounded{\includesvg[keepaspectratio]{/assets/icons/16-chart.svg}}
  \href{https://typst.app/universe/search/?category=visualization}{Visualization}
\item
  \pandocbounded{\includesvg[keepaspectratio]{/assets/icons/16-layout.svg}}
  \href{https://typst.app/universe/search/?category=layout}{Layout}
\end{itemize}
\end{description}

\subsubsection{Where to report issues?}\label{where-to-report-issues}

This package is a project of Mc-Zen . Report issues on
\href{https://github.com/Mc-Zen/zero}{their repository} . You can also
try to ask for help with this package on the
\href{https://forum.typst.app}{Forum} .

Please report this package to the Typst team using the
\href{https://typst.app/contact}{contact form} if you believe it is a
safety hazard or infringes upon your rights.

\phantomsection\label{versions}
\subsubsection{Version history}\label{version-history}

\begin{longtable}[]{@{}ll@{}}
\toprule\noalign{}
Version & Release Date \\
\midrule\noalign{}
\endhead
\bottomrule\noalign{}
\endlastfoot
0.3.0 & October 28, 2024 \\
\href{https://typst.app/universe/package/zero/0.2.0/}{0.2.0} & October
4, 2024 \\
\href{https://typst.app/universe/package/zero/0.1.0/}{0.1.0} & September
16, 2024 \\
\end{longtable}

Typst GmbH did not create this package and cannot guarantee correct
functionality of this package or compatibility with any version of the
Typst compiler or app.
