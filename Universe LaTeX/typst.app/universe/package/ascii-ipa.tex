\title{typst.app/universe/package/ascii-ipa}

\phantomsection\label{banner}
\section{ascii-ipa}\label{ascii-ipa}

{ 2.0.0 }

Converter for ASCII representations of the International Phonetic
Alphabet (IPA)

\phantomsection\label{readme}
ðŸ''„ ASCII / IPA conversion for Typst

This package allows you to easily convert different ASCII
representations of the International Phonetic Alphabet (IPA) to and from
the IPA. It also offers some minor utilities to make phonetic
transcriptions easier to use overall. The package is being maintained
\href{https://github.com/imatpot/typst-ascii-ipa}{here} .

Note: This is an extended port of the
\href{https://github.com/tirimid/ipa-translate}{\texttt{\ ipa-translate\ }}
Rust crate by \href{https://github.com/tirimid}{tirimid} ’s conversion
features into native Typst. Most conversions are implemented according
to
\href{https://en.wikipedia.org/wiki/Comparison_of_ASCII_encodings_of_the_International_Phonetic_Alphabet}{this
Wikipedia article} which in turn relies of the official specifications
of the respective ASCII representations.

\subsection{Conversion}\label{conversion}

The package supports multiple ASCII representations for the IPA with one
function each:

\begin{longtable}[]{@{}ll@{}}
\toprule\noalign{}
Notation & Function name \\
\midrule\noalign{}
\endhead
\bottomrule\noalign{}
\endlastfoot
Branner & \texttt{\ \#branner(...)\ } \\
Praat & \texttt{\ \#praat(...)\ } \\
SIL & \texttt{\ \#sil(...)\ } \\
X-SAMPA & \texttt{\ \#xsampa(...)\ } \\
\end{longtable}

They all return the converted value as a
\href{https://typst.app/docs/reference/foundations/str/}{\texttt{\ string\ }}
and accept the set of same parameters:

\begin{longtable}[]{@{}lllll@{}}
\toprule\noalign{}
Parameter & Type & Positional / Named & Default & Description \\
\midrule\noalign{}
\endhead
\bottomrule\noalign{}
\endlastfoot
\texttt{\ value\ } &
\href{https://typst.app/docs/reference/foundations/str/}{\texttt{\ string\ }}
& positional & & Main input to the function. Usually the transcription
in the corresponsing ASCII-based notation. \\
\texttt{\ reverse\ } &
\href{https://typst.app/docs/reference/foundations/bool/}{\texttt{\ bool\ }}
& named & \texttt{\ false\ } & Reverses the conversion. Pass Unicode IPA
into \texttt{\ value\ } to get the corresponsing ASCII-based notation
back. \\
\end{longtable}

\subsubsection{Examples}\label{examples}

All examples use the Swiss German word
\href{https://als.wikipedia.org/wiki/Chuchich\%C3\%A4schtli}{⟨Chuchichäschtli⟩
{[}ˈχʊχË?iˌχæʃË?tlɪ{]}} for the conversion.

\begin{Shaded}
\begin{Highlighting}[]
\NormalTok{\#import "@preview/ascii{-}ipa:2.0.0": *}

\NormalTok{// returns "ˈχʊχːiˌχæʃːtlɪ"}
\NormalTok{\#branner("\textquotesingle{}XUX:i,Xae)S:tlI")}

\NormalTok{// returns "\textquotesingle{}XUX:i,Xae)S:tlI"}
\NormalTok{\#branner("ˈχʊχːiˌχæʃːtlɪ", reverse: true)}

\NormalTok{// returns "ˈχʊχːiˌχæʃːtlɪ"}
\NormalTok{\#praat("\textbackslash{}\textbackslash{}\textquotesingle{}1\textbackslash{}\textbackslash{}cf\textbackslash{}\textbackslash{}hs\textbackslash{}\textbackslash{}cf\textbackslash{}\textbackslash{}:f\textbackslash{}\textbackslash{}\textquotesingle{}2\textbackslash{}\textbackslash{}ae\textbackslash{}\textbackslash{}sh\textbackslash{}\textbackslash{}:ftl\textbackslash{}\textbackslash{}ic")}

\NormalTok{// returns "\textbackslash{}\textbackslash{}\textquotesingle{}1\textbackslash{}\textbackslash{}cf\textbackslash{}\textbackslash{}hs\textbackslash{}\textbackslash{}cf\textbackslash{}\textbackslash{}:f\textbackslash{}\textbackslash{}\textquotesingle{}2\textbackslash{}\textbackslash{}ae\textbackslash{}\textbackslash{}sh\textbackslash{}\textbackslash{}:ftl\textbackslash{}\textbackslash{}ic"}
\NormalTok{\#praat("ˈχʊχːiˌχæʃːtlɪ", reverse: true)}

\NormalTok{// returns "ˈχʊχːiˌχæʃːtlɪ"}
\NormalTok{\#sil("\}x=u\textless{}x=:i\}\}x=a\textless{}s=:tli=")}

\NormalTok{// returns "\}x=u\textless{}x=:i\}\}x=a\textless{}s=:tli="}
\NormalTok{\#sil("ˈχʊχːiˌχæʃːtlɪ", reverse: true)}

\NormalTok{// returns "ˈχʊχːiˌχæʃːtlɪ"}
\NormalTok{\#xsampa("\textbackslash{}"XUX:i\%X\{S:tlI")}

\NormalTok{// returns "\textbackslash{}"XUX:i\%X\{S:tlI"}
\NormalTok{\#xsampa("ˈχʊχːiˌχæʃːtlɪ", reverse: true)}
\end{Highlighting}
\end{Shaded}

\subsubsection{\texorpdfstring{With
\texttt{\ raw\ }}{With  raw }}\label{with-raw}

You can also use
\href{https://typst.app/docs/reference/text/raw/}{\texttt{\ raw\ }} for
the conversion. This is useful if the notation uses a lot of
backslashes.

\begin{Shaded}
\begin{Highlighting}[]
\NormalTok{\#import "@preview/ascii{-}ipa:2.0.0": praat}

\NormalTok{// regular string}
\NormalTok{\#praat("\textbackslash{}\textbackslash{}\textquotesingle{}1\textbackslash{}\textbackslash{}cf\textbackslash{}\textbackslash{}hs\textbackslash{}\textbackslash{}cf\textbackslash{}\textbackslash{}:f\textbackslash{}\textbackslash{}\textquotesingle{}2\textbackslash{}\textbackslash{}ae\textbackslash{}\textbackslash{}sh\textbackslash{}\textbackslash{}:ftl\textbackslash{}\textbackslash{}ic")}

\NormalTok{// raw}
\NormalTok{\#praat(\textasciigrave{}\textbackslash{}\textquotesingle{}1\textbackslash{}cf\textbackslash{}hs\textbackslash{}cf\textbackslash{}:f\textbackslash{}\textquotesingle{}2\textbackslash{}ae\textbackslash{}sh\textbackslash{}:ftl\textbackslash{}ic\textasciigrave{})}
\end{Highlighting}
\end{Shaded}

Note: \texttt{\ raw\ } will not play nicely with notations that use
\texttt{\ \textasciigrave{}\ } a lot.

\subsection{Brackets \& Braces}\label{brackets-braces}

You can easily mark your notation text as different types of brackets or
braces.

\begin{Shaded}
\begin{Highlighting}[]
\NormalTok{\#import "@preview/ascii{-}ipa:2.0.0": *}

\NormalTok{\#phonetic("prʲɪˈvʲet") // [prʲɪˈvʲet]}
\NormalTok{\#phnt("prʲɪˈvʲet")     // [prʲɪˈvʲet]}

\NormalTok{\#precise("prʲɪˈvʲet") // ⟦prʲɪˈvʲet⟧}
\NormalTok{\#prec("prʲɪˈvʲet")    // ⟦prʲɪˈvʲet⟧}

\NormalTok{\#phonemic("prɪvet") // /prɪvet/}
\NormalTok{\#phnm("prɪvet")     // /prɪvet/}

\NormalTok{\#morphophonemic("prɪvet") // ⫽prɪvet⫽}
\NormalTok{\#mphnm("prɪvet")          // ⫽prɪvet⫽}

\NormalTok{\#indistinguishable("prʲɪˈvʲet") // (prʲɪˈvʲet)}
\NormalTok{\#idst("prʲɪˈvʲet")              // (prʲɪˈvʲet)}

\NormalTok{\#obscured("prʲɪˈvʲet") // ⸨prʲɪˈvʲet⸩}
\NormalTok{\#obsc("prʲɪˈvʲet")     // ⸨prʲɪˈvʲet⸩}

\NormalTok{\#orthographic("привет") // ⟨привет⟩}
\NormalTok{\#orth("привет")         // ⟨привет⟩}

\NormalTok{\#transliterated("privyet") // ⟪privyet⟫}
\NormalTok{\#trlt("privyet")           // ⟪privyet⟫}

\NormalTok{\#prosodic("prʲɪˈvʲet") // \{prʲɪˈvʲet\}}
\NormalTok{\#prsd("prʲɪˈvʲet")     // \{prʲɪˈvʲet\}}
\end{Highlighting}
\end{Shaded}

\subsubsection{How to add}\label{how-to-add}

Copy this into your project and use the import as \texttt{\ ascii-ipa\ }

\begin{verbatim}
#import "@preview/ascii-ipa:2.0.0"
\end{verbatim}

\includesvg[width=0.16667in,height=0.16667in]{/assets/icons/16-copy.svg}

Check the docs for
\href{https://typst.app/docs/reference/scripting/\#packages}{more
information on how to import packages} .

\subsubsection{About}\label{about}

\begin{description}
\tightlist
\item[Author :]
imatpot
\item[License:]
MIT
\item[Current version:]
2.0.0
\item[Last updated:]
May 14, 2024
\item[First released:]
March 26, 2024
\item[Minimum Typst version:]
0.10.0
\item[Archive size:]
9.84 kB
\href{https://packages.typst.org/preview/ascii-ipa-2.0.0.tar.gz}{\pandocbounded{\includesvg[keepaspectratio]{/assets/icons/16-download.svg}}}
\item[Repository:]
\href{https://github.com/imatpot/typst-ascii-ipa}{GitHub}
\item[Discipline :]
\begin{itemize}
\tightlist
\item[]
\item
  \href{https://typst.app/universe/search/?discipline=linguistics}{Linguistics}
\end{itemize}
\item[Categor y :]
\begin{itemize}
\tightlist
\item[]
\item
  \pandocbounded{\includesvg[keepaspectratio]{/assets/icons/16-text.svg}}
  \href{https://typst.app/universe/search/?category=text}{Text}
\end{itemize}
\end{description}

\subsubsection{Where to report issues?}\label{where-to-report-issues}

This package is a project of imatpot . Report issues on
\href{https://github.com/imatpot/typst-ascii-ipa}{their repository} .
You can also try to ask for help with this package on the
\href{https://forum.typst.app}{Forum} .

Please report this package to the Typst team using the
\href{https://typst.app/contact}{contact form} if you believe it is a
safety hazard or infringes upon your rights.

\phantomsection\label{versions}
\subsubsection{Version history}\label{version-history}

\begin{longtable}[]{@{}ll@{}}
\toprule\noalign{}
Version & Release Date \\
\midrule\noalign{}
\endhead
\bottomrule\noalign{}
\endlastfoot
2.0.0 & May 14, 2024 \\
\href{https://typst.app/universe/package/ascii-ipa/1.1.1/}{1.1.1} &
March 26, 2024 \\
\href{https://typst.app/universe/package/ascii-ipa/1.1.0/}{1.1.0} &
March 26, 2024 \\
\href{https://typst.app/universe/package/ascii-ipa/1.0.0/}{1.0.0} &
March 26, 2024 \\
\end{longtable}

Typst GmbH did not create this package and cannot guarantee correct
functionality of this package or compatibility with any version of the
Typst compiler or app.
