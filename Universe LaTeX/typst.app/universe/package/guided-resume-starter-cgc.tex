\title{typst.app/universe/package/guided-resume-starter-cgc}

\phantomsection\label{banner}
\phantomsection\label{template-thumbnail}
\pandocbounded{\includegraphics[keepaspectratio]{https://packages.typst.org/preview/thumbnails/guided-resume-starter-cgc-2.0.0-small.webp}}

\section{guided-resume-starter-cgc}\label{guided-resume-starter-cgc}

{ 2.0.0 }

A guided starter resume for people looking to focus on highlighting
their experience -\/- without having to worry about the hassle of
formatting.

\href{/app?template=guided-resume-starter-cgc&version=2.0.0}{Create
project in app}

\phantomsection\label{readme}
This template is a starter resume for people looking to focus on the
content of their resume, without having to worry about the hassle of
formatting.

\subsection{Get Started!}\label{get-started}

\subsubsection{Quickstart: Typst
Universe}\label{quickstart-typst-universe}

\begin{enumerate}
\tightlist
\item
  If you haven’t already, \href{https://typst.app/}{create a (free!)
  Typst account} .
\item
  Once you have an account, go to the template on
  \href{https://typst.app/universe/package/resume-starter-cgc}{Typst
  Universe}
\item
  Click on “Create Project in App�, give your project a title, and
  press “Create�.
\item
  Start editing! This copy is your own personal copy to edit however you
  want!
\end{enumerate}

There are two files included in this project:

\begin{itemize}
\tightlist
\item
  \texttt{\ starter.typ\ } contains the full template, along with a
  written guide to help you put your best (single-paged) foot forward!
\item
  \texttt{\ resume.typ\ } is the same template, but without the full
  guide included.
\item
  \texttt{\ templates/resume.template.typ\ } contains the formatting and
  style for the underlying pieces.
\end{itemize}

\textbf{I would highly recommend reading \texttt{\ starter.typ\ } or
skimming through the
\href{https://blog.chaoticgood.computer/writing/notes/typst-resume-template}{online
guide} to understand best practices when using the template.}

\subsubsection{Alternative: Typst CLI}\label{alternative-typst-cli}

If you’d prefer to simply download \& modify the template, you can use
the \href{https://github.com/typst/typst}{Typst CLI} to download it
instead:

\begin{Shaded}
\begin{Highlighting}[]
\ExtensionTok{typst}\NormalTok{ init @preview/resume{-}starter{-}cgc}
\end{Highlighting}
\end{Shaded}

\subsection{Layout}\label{layout}

\subsubsection{Header}\label{header}

The resume can be created with a header with the following attributes:

\begin{itemize}
\tightlist
\item
  \texttt{\ author\ } : Your name
\item
  \texttt{\ location\ } : The city, state/province, and country you
  reside in.
\item
  \texttt{\ contacts\ } : A list of contact information and additional
  information
\end{itemize}

\paragraph{Header Example}\label{header-example}

\begin{Shaded}
\begin{Highlighting}[]
\NormalTok{\#show: resume.with(}
\NormalTok{  author: "Dr. Emmit \textbackslash{}"Doc\textbackslash{}" Brown",}
\NormalTok{  location: "Hill Valley, CA",}
\NormalTok{  contacts: (}
\NormalTok{    [\#link("mailto:your\_email@yourmail.com")[Email]],}
\NormalTok{    [\#link("https://your{-}cool{-}site.com")[Website]],}
\NormalTok{    [\#link("https://github.com/your{-}linkedin")[GitHub]],}
\NormalTok{    [\#link("https://linkedin.com/in/your{-}linkedin")[LinkedIn]],}
\NormalTok{  )}
\NormalTok{)}
\end{Highlighting}
\end{Shaded}

\subsubsection{Education}\label{education}

The Education ( \texttt{\ \#edu\ } ) section can be used to highlight
for formal education and certifications.

\begin{itemize}
\tightlist
\item
  \texttt{\ institution\ } : Name of the institution where you study, or
  have graduated from.
\item
  \texttt{\ date\ } : Your graduation date, or expected graduation date.
\item
  \texttt{\ degrees\ } : The degrees you received at the institution

  \begin{itemize}
  \tightlist
  \item
    Each entry is two sections: the \textbf{title} of the degree, and
    the \textbf{subject} that you studied.
  \end{itemize}
\item
  \texttt{\ gpa\ } (optional): Your GPA, or other additional
  information.
\end{itemize}

\paragraph{Education Example}\label{education-example}

\begin{Shaded}
\begin{Highlighting}[]
\NormalTok{\#edu(}
\NormalTok{  institution: "University of Colombia",}
\NormalTok{  date: "Aug 1948",}
\NormalTok{  gpa: "3.9 of 4.0, Summa Cum Laude",}
\NormalTok{  degrees: (}
\NormalTok{    ("Bachelor\textquotesingle{}s of Science", "Nuclear Engineering"),}
\NormalTok{    ("Minors", "Automobile Design, Arabic"),}
\NormalTok{    ("Focus", "Childcare, Education")}
\NormalTok{  ),}
\NormalTok{)}
\end{Highlighting}
\end{Shaded}

\subsubsection{Skills}\label{skills}

An additional Skills ( \texttt{\ \#skills\ } ) section to list skills
relevant to the job you’re applying for.

The input is a list of \texttt{\ Label:\ Skills{[}{]}\ } , in order to
easily toggle comments on skills that you may want to leave in but not
render for a particular application.

\paragraph{Skills Example}\label{skills-example}

\begin{Shaded}
\begin{Highlighting}[]
\NormalTok{\#skills((}
\NormalTok{  ("Expertise", (}
\NormalTok{    [Theoretical Physics],}
\NormalTok{    [Time Travel],}
\NormalTok{    [Nuclear Material Management],}
\NormalTok{    [Student Mentoring],}
\NormalTok{  )),}
\NormalTok{  ("Software", (}
\NormalTok{    [AutoDesk CAD],}
\NormalTok{    [Delorean OS],}
\NormalTok{    [Windows 1],}
\NormalTok{  )),}
\NormalTok{  ("Languages", (}
\NormalTok{    [C++],}
\NormalTok{    [C Language],}
\NormalTok{    [MatLab],}
\NormalTok{    [Punch Cards],}
\NormalTok{  )),}
\NormalTok{))}
\end{Highlighting}
\end{Shaded}

\subsubsection{Experience}\label{experience}

The bulk of your resume, the Experience ( \texttt{\ \#exp\ } ) sections
provide a compact \& concise formatting for bulleted details of your
previous and current work experience.

This section is meant to be flexible, and can also be used to talk about
projects and other experiences that may fall outside of the traditional
definition of “work.�

\begin{itemize}
\tightlist
\item
  \texttt{\ role\ } : The title of your position/role in this
  experience.
\item
  \texttt{\ project\ } : The company you worked at, or the name of the
  project you worked on.
\item
  \texttt{\ date\ } : The start and end dates of this experience.
\item
  \texttt{\ details\ } : A description of the work you did in this
  position

  \begin{itemize}
  \tightlist
  \item
    It is \textbf{highly encouraged} to use bullet points in this
    section.
  \end{itemize}
\item
  \texttt{\ location\ } (optional): The location of the experience
\item
  \texttt{\ summary\ } (optional): A brief summary of the company’s
  mission or project goal.
\end{itemize}

\paragraph{Experience Example}\label{experience-example}

\begin{Shaded}
\begin{Highlighting}[]
\NormalTok{\#exp(}
\NormalTok{  role: "Theoretical Physics Consultant",}
\NormalTok{  project: "Doc Brown\textquotesingle{}s Garage",}
\NormalTok{  date: "June 1953 {-} Oct 2015",}
\NormalTok{  location: "Hill Valley, CA",}
\NormalTok{  summary: "Specializing in development of time travel devices and student tutoring",}
\NormalTok{  details: [}
\NormalTok{    {-} Lead development of time travel devices, resulting in the ability to travel back and forth through time}
\NormalTok{    {-} Managed and executed a budget of \textbackslash{}$14 million dollars gained from an unexplained family fortune}
\NormalTok{    {-} Oversaw QA testing for time travel devices, minimizing risk of maternal time{-}travel related incidents}
\NormalTok{  ]}
\NormalTok{)}
\end{Highlighting}
\end{Shaded}

\subsection{Questions \& Suggestions}\label{questions-suggestions}

Have any questions, comments, or suggestions about the template? Please
feel free to reach out at
\href{mailto:mentoring@chaoticgood.computer}{\texttt{\ mentoring@chaoticgood.computer\ }}
!

\href{/app?template=guided-resume-starter-cgc&version=2.0.0}{Create
project in app}

\subsubsection{How to use}\label{how-to-use}

Click the button above to create a new project using this template in
the Typst app.

You can also use the Typst CLI to start a new project on your computer
using this command:

\begin{verbatim}
typst init @preview/guided-resume-starter-cgc:2.0.0
\end{verbatim}

\includesvg[width=0.16667in,height=0.16667in]{/assets/icons/16-copy.svg}

\subsubsection{About}\label{about}

\begin{description}
\tightlist
\item[Author s :]
\href{https://chaoticgood.computer}{Spencer Elkington} \&
\href{mailto:spencer@chaoticgood.computer}{Spencer Elkington}
\item[License:]
Unlicense
\item[Current version:]
2.0.0
\item[Last updated:]
May 6, 2024
\item[First released:]
May 6, 2024
\item[Archive size:]
22.6 kB
\href{https://packages.typst.org/preview/guided-resume-starter-cgc-2.0.0.tar.gz}{\pandocbounded{\includesvg[keepaspectratio]{/assets/icons/16-download.svg}}}
\item[Repository:]
\href{https://github.com/chaoticgoodcomputing/typst-resume-starter}{GitHub}
\item[Categor y :]
\begin{itemize}
\tightlist
\item[]
\item
  \pandocbounded{\includesvg[keepaspectratio]{/assets/icons/16-user.svg}}
  \href{https://typst.app/universe/search/?category=cv}{CV}
\end{itemize}
\end{description}

\subsubsection{Where to report issues?}\label{where-to-report-issues}

This template is a project of Spencer Elkington and Spencer Elkington .
Report issues on
\href{https://github.com/chaoticgoodcomputing/typst-resume-starter}{their
repository} . You can also try to ask for help with this template on the
\href{https://forum.typst.app}{Forum} .

Please report this template to the Typst team using the
\href{https://typst.app/contact}{contact form} if you believe it is a
safety hazard or infringes upon your rights.

\phantomsection\label{versions}
\subsubsection{Version history}\label{version-history}

\begin{longtable}[]{@{}ll@{}}
\toprule\noalign{}
Version & Release Date \\
\midrule\noalign{}
\endhead
\bottomrule\noalign{}
\endlastfoot
2.0.0 & May 6, 2024 \\
\end{longtable}

Typst GmbH did not create this template and cannot guarantee correct
functionality of this template or compatibility with any version of the
Typst compiler or app.
