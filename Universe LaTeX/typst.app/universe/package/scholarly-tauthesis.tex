\title{typst.app/universe/package/scholarly-tauthesis}

\phantomsection\label{banner}
\phantomsection\label{template-thumbnail}
\pandocbounded{\includegraphics[keepaspectratio]{https://packages.typst.org/preview/thumbnails/scholarly-tauthesis-0.9.0-small.webp}}

\section{scholarly-tauthesis}\label{scholarly-tauthesis}

{ 0.9.0 }

A template for writing Tampere University theses.

\href{/app?template=scholarly-tauthesis&version=0.9.0}{Create project in
app}

\phantomsection\label{readme}
This is a TAU thesis template written in the
\href{https://github.com/typst/typst}{\texttt{\ typst\ }} typesetting
language, a potential successor to LaTeΧ. The version of typst used to
test this template is
\href{https://github.com/typst/typst/releases/tag/v0.12.0}{\texttt{\ 0.12.0\ }}
.

\subsection{Using the template on
typst.app}\label{using-the-template-on-typst.app}

This template is also available on
\href{https://typst.app/universe}{Typst Universe} as
\href{https://typst.app/universe/package/scholarly-tauthesis}{\texttt{\ scholarly-tauthesis\ }}
. Simply create an account on \url{https://typst.app/} and start a new
\texttt{\ scholarly-tauthesis\ } project by clicking on \textbf{Start
from template} and searching for \textbf{scholarly-tauthesis} .

If you have initialized your project with an older stable version of
this template and wish to upgrade to a newer release, the simplest way
to do it is to change the value of \texttt{\ \$VERSION\ } ≥
\texttt{\ 0.9.0\ } in the import statements

\begin{Shaded}
\begin{Highlighting}[]
\NormalTok{\#import "@preview/scholarly{-}tauthesis:$VERSION" as tauthesis}
\end{Highlighting}
\end{Shaded}

to correspond to a newer released version. Alternatively, you could
download the \texttt{\ tauthesis.typ\ } file from the
\href{https://gitlab.com/tuni-official/thesis-templates/tau-typst-thesis-template}{thesis
template repository} , and upload it into you project on
\url{https://typst.app/} . Then use

\begin{Shaded}
\begin{Highlighting}[]
\NormalTok{\#import "path/to/tauthesis.typ" as tauthesis}
\end{Highlighting}
\end{Shaded}

instead of

\begin{Shaded}
\begin{Highlighting}[]
\NormalTok{\#import "@preview/scholarly{-}tauthesis:$VERSION" as tauthesis}
\end{Highlighting}
\end{Shaded}

to import the \texttt{\ tauthesis\ } module.

\subsubsection{Note}\label{note}

Versions of this template before 0.9.0 do not actually work with
typst.app due to a packaging issue.

\subsection{Local installation}\label{local-installation}

If \href{https://typst.app/universe}{Typst Universe} is online, this
template will be downloaded automatically to

\begin{verbatim}
$CACHEDIR/typst/packages/preview/scholarly-tauthesis/$VERSION/
\end{verbatim}

when one runs the command

\begin{verbatim}
typst init @preview/scholarly-tauthesis:$VERSION mythesis
\end{verbatim}

Here \texttt{\ \$VERSION\ } should be ≥ 0.9.0. The value
\texttt{\ \$CACHEDIR\ } for your OS can be discovered from
\url{https://docs.rs/dirs/latest/dirs/fn.cache_dir.html} .

For a manual installation, download the contents of this repository via
Git or as a ZIP file from the template
\href{https://gitlab.com/tuni-official/thesis-templates/tau-typst-thesis-template/-/tags}{tags}
page. Then, make a symbolic link

\begin{verbatim}
$DATADIR/typst/packages/preview/scholarly-tauthesis/$VERSION/ → /path/to/root/of/tauthesis/
\end{verbatim}

so that a local installation of \texttt{\ typst\ } can discover the
\texttt{\ tauthesis.typ\ } file no matter where you are running it from.
To find out the value \texttt{\ \$DATADIR\ } for your operating system,
see \url{https://docs.rs/dirs/latest/dirs/fn.data_dir.html} . The value
\texttt{\ \$VERSION\ } is the version \texttt{\ A.B.C\ } ≥
\texttt{\ 0.9.0\ } of this template you wish to use.

Once the package has been installed, the command

\begin{verbatim}
typst init @preview/scholarly-tauthesis:$VERSION mythesis
\end{verbatim}

creates a folder \texttt{\ mythesis\ } with the template files in place.
Simply make the \texttt{\ mythesis\ } folder you current working
directory and run

\begin{Shaded}
\begin{Highlighting}[]
\ExtensionTok{typst}\NormalTok{ compile main.typ}
\end{Highlighting}
\end{Shaded}

in the shell of your choice to compile the document from scratch.
Alternatively, type

\begin{Shaded}
\begin{Highlighting}[]
\ExtensionTok{typst}\NormalTok{ watch main.typ }\OperatorTok{\&\textgreater{}}\NormalTok{ typst.log }\KeywordTok{\&}
\end{Highlighting}
\end{Shaded}

to have a \href{https://github.com/typst/typst}{\texttt{\ typst\ }}
process watch the file for changes and compile it when a file is
changed. Possible error messages can then be viewed by checking the
contents of the mentioned file \texttt{\ typst.log\ } .

This template can also be uploaded to the typst online editor at
\url{https://typst.app/} . However, the file paths related to the
\texttt{\ tauthesis\ } file will need to be changed if this is done
manually. See the tutorial at \url{https://typst.app/docs/tutorial/} to
learn the basics of the language. Some examples are also given in the
template itself.

\subsection{Archiving the final version of your
work}\label{archiving-the-final-version-of-your-work}

Before submitting your thesis to the university archives, it needs to be
converted to PDF/A format. Typst versions ≥ 0.12.0 should support the
creation of PDF/A-2b files, when run with the command

\begin{Shaded}
\begin{Highlighting}[]
\ExtensionTok{typst}\NormalTok{ compile }\AttributeTok{{-}{-}pdf{-}standard}\NormalTok{ a{-}2b template/main.typ}
\end{Highlighting}
\end{Shaded}

If a verification program such as
\href{https://docs.verapdf.org/install/}{veraPDF} still complains that
the file \texttt{\ template/main.pdf\ } does not conform to the
standard, the Muuntaja-service of Tampere University should be used to
do the final conversion. See the related instructions (
\href{https://libguides.tuni.fi/opinnaytteet/pdfa}{link} ) for how to do
it. Basically it boils down to feeding your compiled PDF document to the
converter at \href{https://muuntaja.tuni.fi/}{https://muuntaja.tuni.fi}
. \textbf{Remember to check that the output of the converter is not
corrupted, before submitting your thesis to the archives.}

\subsection{Usage}\label{usage}

You can either write your entire \emph{main matter} in the
\href{https://github.com/typst/packages/raw/main/packages/preview/scholarly-tauthesis/0.9.0/template/main.typ}{\texttt{\ main.typ\ }}
file, or more preferrably, split it into multiple chapter-specific files
and place those in the
\href{https://github.com/typst/packages/raw/main/packages/preview/scholarly-tauthesis/0.9.0/template/content}{\texttt{\ contents/\ }}
folder, which this template tries to demonstrate. If you choose to write
your own commands (functions) in the
\href{https://github.com/typst/packages/raw/main/packages/preview/scholarly-tauthesis/0.9.0/template/preamble.typ}{\texttt{\ preamble.typ\ }}
file, this needs to be imported at the start of each chapter you plan to
use the commands in. Sections that come before the main matter, like the
Finnish and English abstracts (
\href{https://github.com/typst/packages/raw/main/packages/preview/scholarly-tauthesis/0.9.0/template/content/tiivistelm\%C3\%A4.typ}{\texttt{\ tiivistelmä.typ\ }}
\textbar{}
\href{https://github.com/typst/packages/raw/main/packages/preview/scholarly-tauthesis/0.9.0/template/content/abstract.typ}{\texttt{\ abstract.typ\ }}
) and
\href{https://github.com/typst/packages/raw/main/packages/preview/scholarly-tauthesis/0.9.0/template/content/preface.typ}{\texttt{\ preface.typ\ }}
must \emph{not} be removed from the
\href{https://github.com/typst/packages/raw/main/packages/preview/scholarly-tauthesis/0.9.0/template/content}{\texttt{\ contents\ }}
folder, as the automation supposes that they are located there.

You should probably \emph{not} modify the file
\href{https://github.com/typst/packages/raw/main/packages/preview/scholarly-tauthesis/0.9.0/tauthesis.typ}{\texttt{\ tauthesis.typ\ }}
, unless there is a bug that needs fixing right now, and not when the
maintainer of this project manages to find the time to do it.

\subsection{Contributing}\label{contributing}

Issues may be created in the issue tracker on the
\href{https://gitlab.com/tuni-official/thesis-templates/tau-typst-thesis-template}{template
GitLab repository} , if one has a GitLab account. Merge requests may
also be performed after GitLab account creation, and forking the
project. See GitLab’s documentation on this to find out how to do it
\href{https://docs.gitlab.com/ee/user/project/repository/forking_workflow.html}{link}
.

\subsection{License}\label{license}

This project itself uses the MIT license. See the
\href{https://github.com/typst/packages/raw/main/packages/preview/scholarly-tauthesis/0.9.0/LICENSE}{LICENSE}
file for details.

\href{/app?template=scholarly-tauthesis&version=0.9.0}{Create project in
app}

\subsubsection{How to use}\label{how-to-use}

Click the button above to create a new project using this template in
the Typst app.

You can also use the Typst CLI to start a new project on your computer
using this command:

\begin{verbatim}
typst init @preview/scholarly-tauthesis:0.9.0
\end{verbatim}

\includesvg[width=0.16667in,height=0.16667in]{/assets/icons/16-copy.svg}

\subsubsection{About}\label{about}

\begin{description}
\tightlist
\item[Author :]
\href{mailto:santtu.soderholm@tuni.fi}{Santtu Söderholm}
\item[License:]
MIT
\item[Current version:]
0.9.0
\item[Last updated:]
November 12, 2024
\item[First released:]
April 9, 2024
\item[Minimum Typst version:]
0.12.0
\item[Archive size:]
36.4 kB
\href{https://packages.typst.org/preview/scholarly-tauthesis-0.9.0.tar.gz}{\pandocbounded{\includesvg[keepaspectratio]{/assets/icons/16-download.svg}}}
\item[Repository:]
\href{https://gitlab.com/tuni-official/thesis-templates/tau-typst-thesis-template}{GitLab}
\item[Discipline :]
\begin{itemize}
\tightlist
\item[]
\item
  \href{https://typst.app/universe/search/?discipline=education}{Education}
\end{itemize}
\item[Categor y :]
\begin{itemize}
\tightlist
\item[]
\item
  \pandocbounded{\includesvg[keepaspectratio]{/assets/icons/16-mortarboard.svg}}
  \href{https://typst.app/universe/search/?category=thesis}{Thesis}
\end{itemize}
\end{description}

\subsubsection{Where to report issues?}\label{where-to-report-issues}

This template is a project of Santtu Söderholm . Report issues on
\href{https://gitlab.com/tuni-official/thesis-templates/tau-typst-thesis-template}{their
repository} . You can also try to ask for help with this template on the
\href{https://forum.typst.app}{Forum} .

Please report this template to the Typst team using the
\href{https://typst.app/contact}{contact form} if you believe it is a
safety hazard or infringes upon your rights.

\phantomsection\label{versions}
\subsubsection{Version history}\label{version-history}

\begin{longtable}[]{@{}ll@{}}
\toprule\noalign{}
Version & Release Date \\
\midrule\noalign{}
\endhead
\bottomrule\noalign{}
\endlastfoot
0.9.0 & November 12, 2024 \\
\href{https://typst.app/universe/package/scholarly-tauthesis/0.8.0/}{0.8.0}
& October 21, 2024 \\
\href{https://typst.app/universe/package/scholarly-tauthesis/0.7.0/}{0.7.0}
& September 17, 2024 \\
\href{https://typst.app/universe/package/scholarly-tauthesis/0.6.2/}{0.6.2}
& April 29, 2024 \\
\href{https://typst.app/universe/package/scholarly-tauthesis/0.5.0/}{0.5.0}
& April 15, 2024 \\
\href{https://typst.app/universe/package/scholarly-tauthesis/0.4.1/}{0.4.1}
& April 13, 2024 \\
\href{https://typst.app/universe/package/scholarly-tauthesis/0.4.0/}{0.4.0}
& April 9, 2024 \\
\end{longtable}

Typst GmbH did not create this template and cannot guarantee correct
functionality of this template or compatibility with any version of the
Typst compiler or app.
