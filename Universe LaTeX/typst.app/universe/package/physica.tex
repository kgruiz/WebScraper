\title{typst.app/universe/package/physica}

\phantomsection\label{banner}
\section{physica}\label{physica}

{ 0.9.3 }

Math constructs for science and engineering: derivative, differential,
vector field, matrix, tensor, Dirac braket, hbar, transpose, conjugate,
many operators, and more.

{ } Featured Package

\phantomsection\label{readme}
:green\_book: The
\href{https://github.com/Leedehai/typst-physics/blob/v0.9.3/physica-manual.pdf}{manual}
.

\includegraphics[width=5.67708in,height=\textheight,keepaspectratio]{https://github.com/Leedehai/typst-physics/assets/18319900/ed86198a-8ddb-4473-aed3-8111d5ecde60}

\href{https://github.com/Leedehai/typst-physics/actions/workflows/ci.yml}{\pandocbounded{\includesvg[keepaspectratio]{https://github.com/Leedehai/typst-physics/actions/workflows/ci.yml/badge.svg}}}
\href{https://github.com/Leedehai/typst-physics/releases/latest}{\pandocbounded{\includegraphics[keepaspectratio]{https://img.shields.io/github/v/release/Leedehai/typst-physics.svg?color=gold}}}

Available in the collection of
\href{https://typst.app/docs/packages/}{Typst packages} :
\texttt{\ \#import\ "@preview/physica:0.9.3":\ *\ }

\begin{quote}
physica \emph{noun} .

\begin{itemize}
\tightlist
\item
  Latin, study of nature
\end{itemize}
\end{quote}

This \href{https://typst.app/}{Typst} package provides handy typesetting
utilities for natural sciences, including:

\begin{itemize}
\tightlist
\item
  Braces,
\item
  Vectors and vector fields,
\item
  Matrices, including Jacobian and Hessian,
\item
  Smartly render \texttt{\ ..\^{}T\ } as transpose and
  \texttt{\ ..\^{}+\ } as dagger (conjugate transpose),
\item
  Dirac braket notations,
\item
  Common math functions,
\item
  Differentials and derivatives, including partial derivatives of mixed
  orders with automatic order summation,
\item
  Familiar “h-bar�, tensor abstract index notations, isotopes,
  Taylor series term,
\item
  Signal sequences i.e. digital timing diagrams.
\end{itemize}

\subsection{A quick look}\label{a-quick-look}

See the
\href{https://github.com/Leedehai/typst-physics/blob/v0.9.3/physica-manual.pdf}{manual}
for more details and examples.

\pandocbounded{\includegraphics[keepaspectratio]{https://github.com/Leedehai/typst-physics/assets/18319900/4a9f40df-f753-4324-8114-c682d270e9c7}}

A larger
\href{https://github.com/Leedehai/typst-physics/blob/master/demo.typ}{demo.typ}
:

\pandocbounded{\includegraphics[keepaspectratio]{https://github.com/Leedehai/typst-physics/assets/18319900/75b94ef8-cc98-434f-be5f-bfac1ef6aef9}}

\subsection{Using physica in your Typst
document}\label{using-physica-in-your-typst-document}

\subsubsection{\texorpdfstring{With \texttt{\ typst\ } package
management
(recommended)}{With  typst  package management (recommended)}}\label{with-typst-package-management-recommended}

See \url{https://github.com/typst/packages} . If you are using the
Typst’s web app, packages listed there are readily available; if you
are using the Typst compiler locally, it downloads packages on-demand
and caches them on-disk, see
\href{https://github.com/typst/packages\#downloads}{here} for details.

\includegraphics[width=1.80208in,height=\textheight,keepaspectratio]{https://github.com/Leedehai/typst-physics/assets/18319900/f2a3a2bd-3ef7-4383-ab92-9a71affb4e12}

\begin{Shaded}
\begin{Highlighting}[]
\NormalTok{// Style 1}
\NormalTok{\#import "@preview/physica:0.9.3": *}

\NormalTok{$ curl (grad f), tensor(T, {-}mu, +nu), pdv(f,x,y,[1,2]) $}
\end{Highlighting}
\end{Shaded}

\begin{Shaded}
\begin{Highlighting}[]
\NormalTok{// Style 2}
\NormalTok{\#import "@preview/physica:0.9.3": curl, grad, tensor, pdv}

\NormalTok{$ curl (grad f), tensor(T, {-}mu, +nu), pdv(f,x,y,[1,2]) $}
\end{Highlighting}
\end{Shaded}

\begin{Shaded}
\begin{Highlighting}[]
\NormalTok{// Style 3}
\NormalTok{\#import "@preview/physica:0.9.3"}

\NormalTok{$ physica.curl (physica.grad f), physica.tensor(T, {-}mu, +nu), physica.pdv(f,x,y,[1,2]) $}
\end{Highlighting}
\end{Shaded}

\subsubsection{\texorpdfstring{Without \texttt{\ typst\ } package
management}{Without  typst  package management}}\label{without-typst-package-management}

Similar to examples above, but import with the undecorated file path
like \texttt{\ "physica.typ"\ } .

\subsection{Typst version}\label{typst-version}

The version requirement for the compiler is in
\href{https://github.com/typst/packages/raw/main/packages/preview/physica/0.9.3/typst.toml}{typst.toml}
’s \texttt{\ compiler\ } field. If you are using an unsupported Typst
version, the compiler will throw an error. You may want to update your
compiler with \texttt{\ typst\ update\ } , or choose an earlier version
of the \texttt{\ physica\ } package.

Developed with compiler version:

\begin{Shaded}
\begin{Highlighting}[]
\ExtensionTok{$}\NormalTok{ typst }\AttributeTok{{-}{-}version}
\ExtensionTok{typst}\NormalTok{ 0.10.0 }\ErrorTok{(}\ExtensionTok{70ca0d25}\KeywordTok{)}
\end{Highlighting}
\end{Shaded}

\subsection{Manual}\label{manual}

See the
\href{https://github.com/Leedehai/typst-physics/blob/v0.9.3/physica-manual.pdf}{manual}
for a more comprehensive coverage, a PDF file generated directly with
the \href{https://typst.app/}{Typst} binary.

To regenerate the manual, use command

\begin{Shaded}
\begin{Highlighting}[]
\ExtensionTok{typst}\NormalTok{ watch physica{-}manual.typ}
\end{Highlighting}
\end{Shaded}

\subsection{Contribution}\label{contribution}

\begin{itemize}
\item
  Bug fixes are welcome!
\item
  New features: welcome as well. If it is small, feel free to create a
  pull request. If it is large, the best first step is creating an issue
  and let us explore the design together. Some features might warrant a
  package on its own.
\item
  Testing: currently testing is done by closely inspecting the generated
  \href{https://github.com/Leedehai/typst-physics/blob/v0.9.3/physica-manual.pdf}{manual}
  . This does not scale well. I plan to add programmatic testing by
  comparing rendered pictures with golden images.
\end{itemize}

\subsection{Change log}\label{change-log}

\href{https://github.com/Leedehai/typst-physics/blob/v0.9.3/changelog.md}{changelog.md}
.

\subsection{License}\label{license}

\begin{itemize}
\tightlist
\item
  Code: the
  \href{https://github.com/typst/packages/raw/main/packages/preview/physica/0.9.3/LICENSE.txt}{MIT
  License} .
\item
  Docs: the
  \href{https://creativecommons.org/licenses/by-nd/4.0/}{Creative
  Commons BY-ND 4.0 license} .
\end{itemize}

\subsubsection{How to add}\label{how-to-add}

Copy this into your project and use the import as \texttt{\ physica\ }

\begin{verbatim}
#import "@preview/physica:0.9.3"
\end{verbatim}

\includesvg[width=0.16667in,height=0.16667in]{/assets/icons/16-copy.svg}

Check the docs for
\href{https://typst.app/docs/reference/scripting/\#packages}{more
information on how to import packages} .

\subsubsection{About}\label{about}

\begin{description}
\tightlist
\item[Author :]
Leedehai
\item[License:]
MIT
\item[Current version:]
0.9.3
\item[Last updated:]
April 2, 2024
\item[First released:]
September 8, 2023
\item[Minimum Typst version:]
0.10.0
\item[Archive size:]
11.1 kB
\href{https://packages.typst.org/preview/physica-0.9.3.tar.gz}{\pandocbounded{\includesvg[keepaspectratio]{/assets/icons/16-download.svg}}}
\item[Repository:]
\href{https://github.com/Leedehai/typst-physics}{GitHub}
\item[Discipline s :]
\begin{itemize}
\tightlist
\item[]
\item
  \href{https://typst.app/universe/search/?discipline=chemistry}{Chemistry}
\item
  \href{https://typst.app/universe/search/?discipline=communication}{Communication}
\item
  \href{https://typst.app/universe/search/?discipline=economics}{Economics}
\item
  \href{https://typst.app/universe/search/?discipline=education}{Education}
\item
  \href{https://typst.app/universe/search/?discipline=engineering}{Engineering}
\item
  \href{https://typst.app/universe/search/?discipline=geology}{Geology}
\item
  \href{https://typst.app/universe/search/?discipline=mathematics}{Mathematics}
\item
  \href{https://typst.app/universe/search/?discipline=physics}{Physics}
\end{itemize}
\item[Categor ies :]
\begin{itemize}
\tightlist
\item[]
\item
  \pandocbounded{\includesvg[keepaspectratio]{/assets/icons/16-package.svg}}
  \href{https://typst.app/universe/search/?category=components}{Components}
\item
  \pandocbounded{\includesvg[keepaspectratio]{/assets/icons/16-hammer.svg}}
  \href{https://typst.app/universe/search/?category=utility}{Utility}
\end{itemize}
\end{description}

\subsubsection{Where to report issues?}\label{where-to-report-issues}

This package is a project of Leedehai . Report issues on
\href{https://github.com/Leedehai/typst-physics}{their repository} . You
can also try to ask for help with this package on the
\href{https://forum.typst.app}{Forum} .

Please report this package to the Typst team using the
\href{https://typst.app/contact}{contact form} if you believe it is a
safety hazard or infringes upon your rights.

\phantomsection\label{versions}
\subsubsection{Version history}\label{version-history}

\begin{longtable}[]{@{}ll@{}}
\toprule\noalign{}
Version & Release Date \\
\midrule\noalign{}
\endhead
\bottomrule\noalign{}
\endlastfoot
0.9.3 & April 2, 2024 \\
\href{https://typst.app/universe/package/physica/0.9.2/}{0.9.2} &
January 15, 2024 \\
\href{https://typst.app/universe/package/physica/0.9.1/}{0.9.1} &
December 23, 2023 \\
\href{https://typst.app/universe/package/physica/0.9.0/}{0.9.0} &
December 7, 2023 \\
\href{https://typst.app/universe/package/physica/0.8.1/}{0.8.1} &
November 1, 2023 \\
\href{https://typst.app/universe/package/physica/0.8.0/}{0.8.0} &
September 13, 2023 \\
\href{https://typst.app/universe/package/physica/0.7.5/}{0.7.5} &
September 8, 2023 \\
\end{longtable}

Typst GmbH did not create this package and cannot guarantee correct
functionality of this package or compatibility with any version of the
Typst compiler or app.
