\title{typst.app/universe/package/abiding-ifacconf}

\phantomsection\label{banner}
\phantomsection\label{template-thumbnail}
\pandocbounded{\includegraphics[keepaspectratio]{https://packages.typst.org/preview/thumbnails/abiding-ifacconf-0.1.0-small.webp}}

\section{abiding-ifacconf}\label{abiding-ifacconf}

{ 0.1.0 }

An IFAC-style paper template to publish at conferences for International
Federation of Automatic Control

\href{/app?template=abiding-ifacconf&version=0.1.0}{Create project in
app}

\phantomsection\label{readme}
\subsection{(unofficial) IFAC Conference Template for
Typst}\label{unofficial-ifac-conference-template-for-typst}

IFAC stands for \href{https://ifac-control.org/}{International
Federation of Automatic Control} . This repository is meant to be a port
of the existing author tools for conference papers (e.g., for LaTeX, see
\href{https://www.ifac-control.org/conferences/author-guide/copy_of_ifacconf_latex.zip/view}{ifacconf\_latex.zip}
) for Typst.

\subsection{Usage}\label{usage}

Running the following command will create a new directory with all the
files that are needed:

\begin{verbatim}
typst init @preview/abiding-ifacconf
\end{verbatim}

\subsection{Configuration}\label{configuration}

This template exports the \texttt{\ ifacconf\ } function with the
following named arguments:

\begin{itemize}
\tightlist
\item
  \texttt{\ authors\ } : (default: ()) array of authors. For each author
  you can specify a name, email (optional), and affiliation. The
  affiliation must be an integer corresponding to an entry in the
  1-indexed affiliations list (or 0 for no affiliation).
\item
  \texttt{\ affiliations\ } : (default: ()) array of affiliations. For
  each affiliation you can specify a department, organization, and
  address. Everything is optional (i.e., an affiliation can be an empty
  array).
\item
  \texttt{\ abstract\ } : (default: none) the paper’s abstract. Can be
  omitted if you don’t have one.
\item
  \texttt{\ keywords\ } : (default: ()) array of keywords to display
  after the abstract
\item
  \texttt{\ sponsor\ } : (default: none) acknowledgment of sponsor or
  financial support (appears as a footnote on the first page)
\end{itemize}

\subsection{Minimal Working Example}\label{minimal-working-example}

\begin{Shaded}
\begin{Highlighting}[]
\NormalTok{\#import "@preview:abiding{-}ifacconf:0.1.0": *}
\NormalTok{\#show: ifacconf{-}rules}
\NormalTok{\#show: ifacconf.with(}
\NormalTok{  title: "Minimal Working Example",}
\NormalTok{  authors: (}
\NormalTok{    (}
\NormalTok{      name: "First A. Author",}
\NormalTok{      email: "author@boulder.nist.gov",}
\NormalTok{      affiliation: 1,}
\NormalTok{    ),}
\NormalTok{  ),}
\NormalTok{  affiliations: (}
\NormalTok{    (}
\NormalTok{      department: "Engineering",}
\NormalTok{      organization: "National Institute of Standards and Technology",}
\NormalTok{      address: "Boulder, CO 80305 USA",}
\NormalTok{    ),}
\NormalTok{  ),}
\NormalTok{  abstract: [}
\NormalTok{    Abstract should be 50{-}100 words.}
\NormalTok{  ],}
\NormalTok{  keywords: ("keyword1", "keyword2"),}
\NormalTok{  sponsor: [}
\NormalTok{    Sponsor information.}
\NormalTok{  ],}
\NormalTok{)}

\NormalTok{= Introduction}

\NormalTok{A minimum working example (with bibliography) @Abl56.}

\NormalTok{\#lorem(80)}

\NormalTok{\#lorem(80)}

\NormalTok{\#bibliography("refs.bib")}
\end{Highlighting}
\end{Shaded}

\subsection{Full(er) Example}\label{fuller-example}

See
\href{https://github.com/avonmoll/ifacconf-typst/blob/main/template/main.typ}{\texttt{\ main.typ\ }}
.

\subsection{Dependencies}\label{dependencies}

\begin{itemize}
\tightlist
\item
  typst 0.11.0
\item
  ctheorems 1.1.0 (a Typst package for handling theorem-like
  environments)
\end{itemize}

\subsection{Notes, features, etc.}\label{notes-features-etc.}

\begin{itemize}
\tightlist
\item
  the call to \texttt{\ \#show:\ ifacconf-rules\ } is necessary for some
  show rules defined in \texttt{\ template.typ\ } to get activated
\item
  \texttt{\ ifac-conference.csl\ } is a lightly modified version of
  \texttt{\ apa.csl\ } and is included in order to change the citation
  format from, e.g., \texttt{\ (Able\ 1956)\ } to
  \texttt{\ Able\ (1956)\ } in order to match
  \texttt{\ ifacconf\_latex\ }
\item
  Tables have formatting rules that get activated inside calls to
  \texttt{\ figure\ } with \texttt{\ kind:\ "table"\ } ; a convenience
  function \texttt{\ tablefig\ } is provided which sets this
  automatically
\item
  all theorem-like environments that were available in
  \texttt{\ ifacconf\_latex\ } are defined in \texttt{\ template.typ\ }
  ; simply call, for example,
  \texttt{\ \#theorem{[}Content...{]}\ ...\ \#proof{[}Proof...{]}\ }
\item
  the LaTeX version does not include a QED symbol at the end of proofs,
  however one is included here (this is easy to change)
\item
  Typst did not seem to like BibTeX citation keys containing colons
  (which was how they came from \texttt{\ ifacconf\_latex\ } )
\item
  alignment for linebreaks in long equations is somewhat manual (e.g.,
  for equation (2) in \texttt{\ ifacconf.typ\ } ) but probably there is
  a better way to handle this now or in the future
\item
  the files \texttt{\ refs.bib\ } (essentially) and
  \texttt{\ bifurcation.jpg\ } come from \texttt{\ ifacconf\_latex\ }
\item
  the file \texttt{\ ifacconf.typ\ } is modeled directly after
  \texttt{\ ifacconf.tex\ } by Juan a. de la Puente
\item
  the \texttt{\ citep\ } function renders citations like
  \texttt{\ (Keohane,\ 1958)\ } instead of the default style of
  \texttt{\ Keohane\ (1958)\ }
\end{itemize}

\subsection{License}\label{license}

This template is licensed according to the MIT No Attribution license
(see \texttt{\ LICENSE.MD\ } ).

The files in the \texttt{\ CSL\ } folder are licensed according to
\texttt{\ CSL/LICENSE.md\ } (CC BY/SA 4.0) because it is a lightly
modified version of \texttt{\ apa.csl\ } by Brenton M. Wiernik which is
also licensed by a CC BY/SA license.

\href{/app?template=abiding-ifacconf&version=0.1.0}{Create project in
app}

\subsubsection{How to use}\label{how-to-use}

Click the button above to create a new project using this template in
the Typst app.

You can also use the Typst CLI to start a new project on your computer
using this command:

\begin{verbatim}
typst init @preview/abiding-ifacconf:0.1.0
\end{verbatim}

\includesvg[width=0.16667in,height=0.16667in]{/assets/icons/16-copy.svg}

\subsubsection{About}\label{about}

\begin{description}
\tightlist
\item[Author :]
\href{https://avonmoll.github.io}{Alexander Von Moll}
\item[License:]
MIT-0
\item[Current version:]
0.1.0
\item[Last updated:]
March 21, 2024
\item[First released:]
March 21, 2024
\item[Minimum Typst version:]
0.11.0
\item[Archive size:]
27.8 kB
\href{https://packages.typst.org/preview/abiding-ifacconf-0.1.0.tar.gz}{\pandocbounded{\includesvg[keepaspectratio]{/assets/icons/16-download.svg}}}
\item[Repository:]
\href{https://github.com/avonmoll/ifacconf-typst}{GitHub}
\item[Discipline s :]
\begin{itemize}
\tightlist
\item[]
\item
  \href{https://typst.app/universe/search/?discipline=computer-science}{Computer
  Science}
\item
  \href{https://typst.app/universe/search/?discipline=engineering}{Engineering}
\end{itemize}
\item[Categor y :]
\begin{itemize}
\tightlist
\item[]
\item
  \pandocbounded{\includesvg[keepaspectratio]{/assets/icons/16-atom.svg}}
  \href{https://typst.app/universe/search/?category=paper}{Paper}
\end{itemize}
\end{description}

\subsubsection{Where to report issues?}\label{where-to-report-issues}

This template is a project of Alexander Von Moll . Report issues on
\href{https://github.com/avonmoll/ifacconf-typst}{their repository} .
You can also try to ask for help with this template on the
\href{https://forum.typst.app}{Forum} .

Please report this template to the Typst team using the
\href{https://typst.app/contact}{contact form} if you believe it is a
safety hazard or infringes upon your rights.

\phantomsection\label{versions}
\subsubsection{Version history}\label{version-history}

\begin{longtable}[]{@{}ll@{}}
\toprule\noalign{}
Version & Release Date \\
\midrule\noalign{}
\endhead
\bottomrule\noalign{}
\endlastfoot
0.1.0 & March 21, 2024 \\
\end{longtable}

Typst GmbH did not create this template and cannot guarantee correct
functionality of this template or compatibility with any version of the
Typst compiler or app.
