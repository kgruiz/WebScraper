\title{typst.app/universe/package/elsearticle}

\phantomsection\label{banner}
\phantomsection\label{template-thumbnail}
\pandocbounded{\includegraphics[keepaspectratio]{https://packages.typst.org/preview/thumbnails/elsearticle-0.4.0-small.webp}}

\section{elsearticle}\label{elsearticle}

{ 0.4.0 }

Conversion of the LaTeX elsearticle.cls

{ } Featured Template

\href{/app?template=elsearticle&version=0.4.0}{Create project in app}

\phantomsection\label{readme}
\href{https://github.com/typst/packages/raw/main/packages/preview/elsearticle/0.4.0/}{\pandocbounded{\includesvg[keepaspectratio]{https://img.shields.io/badge/Version-0.4.0-cornflowerblue.svg}}}
\href{https://github.com/maucejo/elsearticle/blob/main/LICENSE}{\pandocbounded{\includegraphics[keepaspectratio]{https://img.shields.io/badge/License-MIT-forestgreen}}}
\href{https://github.com/maucejo/elsearticle/blob/main/docs/manual.pdf}{\pandocbounded{\includegraphics[keepaspectratio]{https://img.shields.io/badge/doc-.pdf-mediumpurple}}}

\texttt{\ elsearticle\ } is a Typst template that aims to mimic the
Elsevier article LaTeX class, a.k.a. \texttt{\ elsearticle.cls\ } ,
provided by Elsevier to format manuscript properly for submission to
their journals.

\subsection{Basic usage}\label{basic-usage}

This section provides the minimal amount of information to get started
with the template. For more detailed information, see the
\href{https://github.com/maucejo/elsearticle/blob/main/docs/manual.pdf}{manual}
.

To use the \texttt{\ elsearticle\ } template, you need to include the
following line at the beginning of your \texttt{\ typ\ } file:

\begin{Shaded}
\begin{Highlighting}[]
\NormalTok{\#import "@preview/elsearticle:0.4.0": *}
\end{Highlighting}
\end{Shaded}

\subsubsection{Initializing the
template}\label{initializing-the-template}

After importing \texttt{\ elsearticle\ } , you have to initialize the
template by a show rule with the \texttt{\ \#elsearticle()\ } command.
This function takes an optional argument to specify the title of the
document.

\begin{itemize}
\tightlist
\item
  \texttt{\ title\ } : Title of the paper
\item
  \texttt{\ author\ } : List of the authors of the paper
\item
  \texttt{\ abstract\ } : Abstract of the paper
\item
  \texttt{\ journal\ } : Name of the journal
\item
  \texttt{\ keywords\ } : List of keywords of the paper
\item
  \texttt{\ format\ } : Format of the paper. Possible values are
  \texttt{\ preprint\ } , \texttt{\ review\ } , \texttt{\ 1p\ } ,
  \texttt{\ 3p\ } , \texttt{\ 5p\ }
\item
  \texttt{\ numcol\ } : Number of columns of the paper. Possible values
  are 1 and 2
\item
  \texttt{\ line-numbering\ } : Enable line numbering. Possible values
  are \texttt{\ true\ } and \texttt{\ false\ }
\end{itemize}

\subsection{Additional features}\label{additional-features}

The \texttt{\ elsearticle\ } template provides additional features to
help you format your document properly.

\subsubsection{Appendix}\label{appendix}

To activate the appendix environment, all you have to do is to place the
following command in your document:

\begin{Shaded}
\begin{Highlighting}[]
\NormalTok{\#show: appendix}

\NormalTok{// Appendix content here}
\end{Highlighting}
\end{Shaded}

\subsubsection{Subfigures}\label{subfigures}

Subfigures are not built-in features of Typst, but the
\texttt{\ elsearticle\ } template provides a way to handle them. It is
based on the \texttt{\ subpar\ } package that allows you to create
subfigures and properly reference them.

\begin{Shaded}
\begin{Highlighting}[]
\NormalTok{  \#subfigure(}
\NormalTok{    figure(image("image1.png"), caption: []), \textless{}figa\textgreater{},}
\NormalTok{    figure(image("image2.png"), caption: []), \textless{}figb\textgreater{},}
\NormalTok{    columns: (1fr, 1fr),}
\NormalTok{    caption: [(a) Left image and (b) Right image],}
\NormalTok{    label: \textless{}fig\textgreater{}}
\NormalTok{  )}
\end{Highlighting}
\end{Shaded}

\subsubsection{Equations}\label{equations}

The \texttt{\ elsearticle\ } template provides the
\texttt{\ \#nonumeq()\ } function to create unnmbered equations. The
latter function can be used as follows:

\begin{Shaded}
\begin{Highlighting}[]
\NormalTok{\#nonumeq[$}
\NormalTok{  y = f(x)}
\NormalTok{  $}
\NormalTok{]}
\end{Highlighting}
\end{Shaded}

\subsection{Roadmap}\label{roadmap}

\emph{Article format}

\begin{itemize}
\tightlist
\item
  {[}x{]} Preprint
\item
  {[}x{]} Review
\item
  {[}x{]} 1p
\item
  {[}x{]} 3p
\item
  {[}x{]} 5p
\end{itemize}

\emph{Environment}

\begin{itemize}
\tightlist
\item
  {[}x{]} Implementation of the \texttt{\ appendix\ } environment
\end{itemize}

\emph{Figures and tables}

\begin{itemize}
\tightlist
\item
  {[}x{]} Implementation of the \texttt{\ subfigure\ } environment
\end{itemize}

\emph{Equations}

\begin{itemize}
\tightlist
\item
  {[}x{]} Proper referencing of equations w.r.t. the context
\item
  {[}x{]} Use of the \texttt{\ equate\ } package to number each equation
  of a system as “(1a)�
\end{itemize}

\emph{Other features}

\begin{itemize}
\tightlist
\item
  {[}x{]} Line numbering - Line numbering - Use the built-in
  \texttt{\ par.line\ } function available from Typst v0.12
\end{itemize}

\subsection{License}\label{license}

MIT licensed

Copyright © 2024 Mathieu AUCEJO (maucejo)

\href{/app?template=elsearticle&version=0.4.0}{Create project in app}

\subsubsection{How to use}\label{how-to-use}

Click the button above to create a new project using this template in
the Typst app.

You can also use the Typst CLI to start a new project on your computer
using this command:

\begin{verbatim}
typst init @preview/elsearticle:0.4.0
\end{verbatim}

\includesvg[width=0.16667in,height=0.16667in]{/assets/icons/16-copy.svg}

\subsubsection{About}\label{about}

\begin{description}
\tightlist
\item[Author :]
Mathieu Aucejo
\item[License:]
MIT
\item[Current version:]
0.4.0
\item[Last updated:]
November 18, 2024
\item[First released:]
July 22, 2024
\item[Archive size:]
95.5 kB
\href{https://packages.typst.org/preview/elsearticle-0.4.0.tar.gz}{\pandocbounded{\includesvg[keepaspectratio]{/assets/icons/16-download.svg}}}
\item[Repository:]
\href{https://github.com/maucejo/elsearticle}{GitHub}
\item[Categor y :]
\begin{itemize}
\tightlist
\item[]
\item
  \pandocbounded{\includesvg[keepaspectratio]{/assets/icons/16-speak.svg}}
  \href{https://typst.app/universe/search/?category=report}{Report}
\end{itemize}
\end{description}

\subsubsection{Where to report issues?}\label{where-to-report-issues}

This template is a project of Mathieu Aucejo . Report issues on
\href{https://github.com/maucejo/elsearticle}{their repository} . You
can also try to ask for help with this template on the
\href{https://forum.typst.app}{Forum} .

Please report this template to the Typst team using the
\href{https://typst.app/contact}{contact form} if you believe it is a
safety hazard or infringes upon your rights.

\phantomsection\label{versions}
\subsubsection{Version history}\label{version-history}

\begin{longtable}[]{@{}ll@{}}
\toprule\noalign{}
Version & Release Date \\
\midrule\noalign{}
\endhead
\bottomrule\noalign{}
\endlastfoot
0.4.0 & November 18, 2024 \\
\href{https://typst.app/universe/package/elsearticle/0.3.0/}{0.3.0} &
October 21, 2024 \\
\href{https://typst.app/universe/package/elsearticle/0.2.1/}{0.2.1} &
September 27, 2024 \\
\href{https://typst.app/universe/package/elsearticle/0.2.0/}{0.2.0} &
August 1, 2024 \\
\href{https://typst.app/universe/package/elsearticle/0.1.0/}{0.1.0} &
July 22, 2024 \\
\end{longtable}

Typst GmbH did not create this template and cannot guarantee correct
functionality of this template or compatibility with any version of the
Typst compiler or app.
