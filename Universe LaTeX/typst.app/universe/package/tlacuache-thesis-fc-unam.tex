\title{typst.app/universe/package/tlacuache-thesis-fc-unam}

\phantomsection\label{banner}
\phantomsection\label{template-thumbnail}
\pandocbounded{\includegraphics[keepaspectratio]{https://packages.typst.org/preview/thumbnails/tlacuache-thesis-fc-unam-0.1.1-small.webp}}

\section{tlacuache-thesis-fc-unam}\label{tlacuache-thesis-fc-unam}

{ 0.1.1 }

Template para escribir una tesis para la facultad de ciencias.

\href{/app?template=tlacuache-thesis-fc-unam&version=0.1.1}{Create
project in app}

\phantomsection\label{readme}
Este es un template para tesis de la facultad de ciencias, en la
Universidad Nacional Autónoma de México (UNAM).

This is a thesis template for the Science Faculty at Universidad
Nacional Autónoma de México (UNAM) based on my thesis.

\subsection{Uso/Usage}\label{usousage}

En la aplicación web de Typst da click en “Start from template� y
busca \texttt{\ tlacuache-thesis-fc-unam\ } .

In the Typst web app simply click “Start from template� on the
dashboard and search for \texttt{\ tlacuache-thesis-fc-unam\ } .

Si estas usando la versión de teminal usa el comando: From the CLI you
can initialize the project with the command:

\begin{Shaded}
\begin{Highlighting}[]
\ExtensionTok{typst}\NormalTok{ init @preview/tlacuache{-}thesis{-}fc{-}unam:0.1.1}
\end{Highlighting}
\end{Shaded}

\subsection{Configuración/Configuration}\label{configuraciuxe3uxb3nconfiguration}

Para configurar tu tesis puedes hacerlo con estas lineas al inicio de tu
archivo principal.

To set the thesis template, you can use the following lines in your main
file.

\begin{Shaded}
\begin{Highlighting}[]
\NormalTok{\#import "@preview/tlacuache{-}thesis{-}fc{-}unam:0.1.1": thesis}

\NormalTok{\#show: thesis.with(}
\NormalTok{  ttitulo: [Titulo],}
\NormalTok{  grado: [Licenciatura],}
\NormalTok{  autor: [Autor],}
\NormalTok{  asesor: [Asesor],}
\NormalTok{  lugar: [Ciudad de México, México],}
\NormalTok{  agno: [\#datetime.today().year()],}
\NormalTok{  bibliography: bibliography("references.bib"),}
\NormalTok{)}

\NormalTok{// Tu tesis va aquí}
\end{Highlighting}
\end{Shaded}

Tambien puedes utilizar estas lineas para crear capítulos con
bibliografía, si deseas crear un pdf solomente para el capítulo.

You could also create a pdf for just a chapter with bibliography, by
using the following lines.

\begin{Shaded}
\begin{Highlighting}[]
\NormalTok{\#import "@preview/tlacuache{-}thesis{-}fc{-}unam:0.1.1": chapter}

\NormalTok{// completamente opcional cargar la bibliografía, compilar el capítulo}
\NormalTok{\#show: chapter.with(bibliography: bibliography("references.bib"))}

\NormalTok{// Tu capítulo va aquí}
\end{Highlighting}
\end{Shaded}

Si quieres crear pdf aun mas cortos, puedes utilizar estas lineas para
crear un pdf solo para el sección de tu capítulo.

You could even create a pdf for just a section of a chapter.

\begin{Shaded}
\begin{Highlighting}[]
\NormalTok{\#import "@preview/tlacuache{-}thesis{-}fc{-}unam:0.1.1": section}

\NormalTok{// completamente opcional cargar la bibliografía, compilar el sección}
\NormalTok{\#show: section.with(bibliography: bibliography("references.bib"))}

\NormalTok{// Tu sección va aquí}
\end{Highlighting}
\end{Shaded}

\href{/app?template=tlacuache-thesis-fc-unam&version=0.1.1}{Create
project in app}

\subsubsection{How to use}\label{how-to-use}

Click the button above to create a new project using this template in
the Typst app.

You can also use the Typst CLI to start a new project on your computer
using this command:

\begin{verbatim}
typst init @preview/tlacuache-thesis-fc-unam:0.1.1
\end{verbatim}

\includesvg[width=0.16667in,height=0.16667in]{/assets/icons/16-copy.svg}

\subsubsection{About}\label{about}

\begin{description}
\tightlist
\item[Author :]
David Valencia, davidalencia@ciencias.unam.mx
\item[License:]
MIT
\item[Current version:]
0.1.1
\item[Last updated:]
April 9, 2024
\item[First released:]
April 9, 2024
\item[Archive size:]
3.14 MB
\href{https://packages.typst.org/preview/tlacuache-thesis-fc-unam-0.1.1.tar.gz}{\pandocbounded{\includesvg[keepaspectratio]{/assets/icons/16-download.svg}}}
\item[Categor y :]
\begin{itemize}
\tightlist
\item[]
\item
  \pandocbounded{\includesvg[keepaspectratio]{/assets/icons/16-mortarboard.svg}}
  \href{https://typst.app/universe/search/?category=thesis}{Thesis}
\end{itemize}
\end{description}

\subsubsection{Where to report issues?}\label{where-to-report-issues}

This template is a project of David Valencia,
davidalencia@ciencias.unam.mx . You can also try to ask for help with
this template on the \href{https://forum.typst.app}{Forum} .

Please report this template to the Typst team using the
\href{https://typst.app/contact}{contact form} if you believe it is a
safety hazard or infringes upon your rights.

\phantomsection\label{versions}
\subsubsection{Version history}\label{version-history}

\begin{longtable}[]{@{}ll@{}}
\toprule\noalign{}
Version & Release Date \\
\midrule\noalign{}
\endhead
\bottomrule\noalign{}
\endlastfoot
0.1.1 & April 9, 2024 \\
\end{longtable}

Typst GmbH did not create this template and cannot guarantee correct
functionality of this template or compatibility with any version of the
Typst compiler or app.
