\title{typst.app/universe/package/quetta}

\phantomsection\label{banner}
\section{quetta}\label{quetta}

{ 0.2.0 }

Write Tengwar easily with Typst.

\phantomsection\label{readme}
A simple module to write
\href{https://en.wikipedia.org/wiki/Tengwar}{tengwar} in
\href{https://typst.app/}{Typst} .

\subsection{Requirements}\label{requirements}

\begin{itemize}
\tightlist
\item
  \href{https://github.com/typst/typst}{Typst} version 0.11.0 or 0.11.1
\item
  The
  \href{https://www.fontspace.com/tengwar-annatar-font-f2244}{Tengwar
  Annatar} fonts version 1.20
\end{itemize}

To use this module with the \href{https://typst.app/}{Typst web app} ,
you need to upload the font files to your project.

\subsection{Usage}\label{usage}

The main functionality of this module is provided by functions taking
content and converting all text in Tenwar:

\begin{itemize}
\tightlist
\item
  \texttt{\ quenya\ } converts text using the mode of Quenya,
\item
  \texttt{\ gondor\ } converts text using the Sindarin mode of Gondor.
\end{itemize}

The original text is used as a phonetic transcription. (This module does
not translate English into Quenya or Sindarin.) See the
\href{https://github.com/FlorentCLMichel/quetta/blob/main/manual.pdf}{manual}
for more information.

The following line may be used to convert the whole document below to
Tengwar in Quenya mode (other \texttt{\ show\ } rules might interfere
with it):

\begin{verbatim}
#show: quetta.quenya
\end{verbatim}

\textbf{Example:}

\begin{verbatim}
#import "@preview/quetta:0.2.0"

// Use the function `quenya` to write a small amount of text in Tengwar (Quenya mode)
#text(size: 16pt, 
      fill: gradient.linear(blue, green)
     )[#box(quetta.quenya[_tengwar_])]

#v(1em)

// A `show` rule may be more convenient for larger contents; beware that it may interfere with other ones, though
#show: quetta.quenya

Namárië!

#h(1em) _Namárië!_

#h(2em) *Namárië!*
\end{verbatim}

\subsection{Roadmap}\label{roadmap}

\begin{itemize}
\tightlist
\item
  Number conversion: done
\item
  Support for the Quenya mode: done
\item
  Support for the mode of Gondor: done
\item
  Support for the mode of Beleriand: backlog
\item
  Support for the Black Speech: backlog
\end{itemize}

\subsection{Changelog}\label{changelog}

\subsubsection{v0.2.0}\label{v0.2.0}

\begin{itemize}
\tightlist
\item
  Add support for Sindarinâ€''Mode of Gondor
\item
  \textbf{Breaking change:} The symbol used to prevent combination was
  changed from \texttt{\ :\ } to \texttt{\ \textbar{}\ } .
\item
  Small changes to the kerning between several tengwar and to tehtar
  positions.
\end{itemize}

\subsubsection{v0.1.0}\label{v0.1.0}

Initial release with Quenya support.

\subsection{How can I contribute?}\label{how-can-i-contribute}

I (the original author) am definitely not en expert in either Typst nor
Tengwar. I could thus use some help in all areas. I would especially
welcome contributions or suggestions on the following:

\begin{itemize}
\tightlist
\item
  Identify and resolve inefficiencies in the Typst code.
\item
  Identify cases where the result differs from the expected one. (In
  particular, there are probably rules for writing in Tengwar that I
  either am not aware of or have not properly understood. Any advice on
  that is warmly welcome!)
\item
  References on Tengar, Quenya, and Sindarin.
\item
  Support for other Tengwar fonts.
\end{itemize}

\subsubsection{How to add}\label{how-to-add}

Copy this into your project and use the import as \texttt{\ quetta\ }

\begin{verbatim}
#import "@preview/quetta:0.2.0"
\end{verbatim}

\includesvg[width=0.16667in,height=0.16667in]{/assets/icons/16-copy.svg}

Check the docs for
\href{https://typst.app/docs/reference/scripting/\#packages}{more
information on how to import packages} .

\subsubsection{About}\label{about}

\begin{description}
\tightlist
\item[Author :]
\href{https://github.com/FlorentCLMichel}{Florent Michel}
\item[License:]
MIT
\item[Current version:]
0.2.0
\item[Last updated:]
September 24, 2024
\item[First released:]
July 31, 2024
\item[Minimum Typst version:]
0.11.0
\item[Archive size:]
8.96 kB
\href{https://packages.typst.org/preview/quetta-0.2.0.tar.gz}{\pandocbounded{\includesvg[keepaspectratio]{/assets/icons/16-download.svg}}}
\item[Repository:]
\href{https://github.com/FlorentCLMichel/quetta}{GitHub}
\item[Discipline s :]
\begin{itemize}
\tightlist
\item[]
\item
  \href{https://typst.app/universe/search/?discipline=linguistics}{Linguistics}
\item
  \href{https://typst.app/universe/search/?discipline=literature}{Literature}
\end{itemize}
\item[Categor ies :]
\begin{itemize}
\tightlist
\item[]
\item
  \pandocbounded{\includesvg[keepaspectratio]{/assets/icons/16-text.svg}}
  \href{https://typst.app/universe/search/?category=text}{Text}
\item
  \pandocbounded{\includesvg[keepaspectratio]{/assets/icons/16-world.svg}}
  \href{https://typst.app/universe/search/?category=languages}{Languages}
\item
  \pandocbounded{\includesvg[keepaspectratio]{/assets/icons/16-smile.svg}}
  \href{https://typst.app/universe/search/?category=fun}{Fun}
\end{itemize}
\end{description}

\subsubsection{Where to report issues?}\label{where-to-report-issues}

This package is a project of Florent Michel . Report issues on
\href{https://github.com/FlorentCLMichel/quetta}{their repository} . You
can also try to ask for help with this package on the
\href{https://forum.typst.app}{Forum} .

Please report this package to the Typst team using the
\href{https://typst.app/contact}{contact form} if you believe it is a
safety hazard or infringes upon your rights.

\phantomsection\label{versions}
\subsubsection{Version history}\label{version-history}

\begin{longtable}[]{@{}ll@{}}
\toprule\noalign{}
Version & Release Date \\
\midrule\noalign{}
\endhead
\bottomrule\noalign{}
\endlastfoot
0.2.0 & September 24, 2024 \\
\href{https://typst.app/universe/package/quetta/0.1.0/}{0.1.0} & July
31, 2024 \\
\end{longtable}

Typst GmbH did not create this package and cannot guarantee correct
functionality of this package or compatibility with any version of the
Typst compiler or app.
