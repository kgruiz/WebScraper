\title{typst.app/universe/package/wrap-indent}

\phantomsection\label{banner}
\section{wrap-indent}\label{wrap-indent}

{ 0.1.0 }

Wrap content in custom functions with just indentation

\phantomsection\label{readme}
\texttt{\ wrap-indent\ } is a package for wrapping content in custom
functions with just indentation. This lets you avoid using trailing
square brackets to wrap content, instead you just indent it!

This system works by re-purposing Typst’s existing
\href{https://typst.app/docs/reference/model/terms/}{term-list} syntax
via a custom show rule on \texttt{\ terms.item\ } . We give it our
custom function within
\href{https://typst.app/docs/reference/introspection/state/}{state} via
a new \texttt{\ wrap-in()\ } function.

\subsection{Here’s a minimal
example!}\label{hereuxe2s-a-minimal-example}

\pandocbounded{\includegraphics[keepaspectratio]{https://github.com/typst/packages/raw/main/packages/preview/wrap-indent/0.1.0/example_page1.png}}

\begin{Shaded}
\begin{Highlighting}[]
\NormalTok{\#set page(height: auto, width: 3.5in, margin: 0.25in)}

\NormalTok{\#import "@preview/wrap{-}indent:0.1.0": wrap{-}in, allow{-}wrapping}

\NormalTok{\#show terms.item: allow{-}wrapping}

\NormalTok{/ First {-}{-}:}
\NormalTok{  A normal term list}

\NormalTok{  with multiple paragraphs}

\NormalTok{But this text is separated}


\NormalTok{\#line(length: 100\%)}


\NormalTok{\#let custom{-}block(content) = rect(content,}
\NormalTok{  fill: orange.lighten(90\%),}
\NormalTok{  stroke: 1.5pt + gradient.linear(..color.map.flare)}
\NormalTok{)}

\NormalTok{/ \#wrap{-}in(custom{-}block):}
\NormalTok{  A *custom block* using the \textasciigrave{}wrap{-}in\textasciigrave{} function}

\NormalTok{  with indented text \textbackslash{}}
\NormalTok{  over multiple lines}

\NormalTok{And this text is \_still\_ separated!}
\end{Highlighting}
\end{Shaded}

And in its own code block, here’s the required initialization:

\begin{Shaded}
\begin{Highlighting}[]
\NormalTok{\#import "@preview/wrap{-}indent:0.1.0": wrap{-}in, allow{-}wrapping}

\NormalTok{\#show terms.item: allow{-}wrapping}
\end{Highlighting}
\end{Shaded}

\subsection{And here’s a more complicated
example!}\label{and-hereuxe2s-a-more-complicated-example}

\pandocbounded{\includegraphics[keepaspectratio]{https://github.com/typst/packages/raw/main/packages/preview/wrap-indent/0.1.0/example_page2.png}}

\begin{Shaded}
\begin{Highlighting}[]
\NormalTok{\#set page(height: auto, width: 4.1in, margin: 0.25in)}

\NormalTok{\#show heading: set text(size: 0.75em)}
\NormalTok{\#show heading: set block(below: 1em)}
\NormalTok{\#set heading(numbering: "1) ")}

\NormalTok{= Normal function call:}

\NormalTok{// A function for wrapping some text:}
\NormalTok{\#let custom{-}quote(body) = rect(}
\NormalTok{  body,}
\NormalTok{  width: 100\%,}
\NormalTok{  fill: luma(95\%),}
\NormalTok{  stroke: (left: 2pt + luma(30\%))}
\NormalTok{)}

\NormalTok{\#custom{-}quote[}
\NormalTok{  Some text in a \_custom quote\_ spread over multiple lines}
\NormalTok{  so it actually looks like it was typed in a document.}
\NormalTok{]}
\NormalTok{This text is outside the quote box}


\NormalTok{= Wrappped function call!}

\NormalTok{/ \#wrap{-}in(custom{-}quote):}
\NormalTok{  Some text in a \_custom quote\_ spread over multiple lines}
\NormalTok{  so it actually looks like it was typed in a document.}

\NormalTok{This text is \_still\_ outside the quote box!}


\NormalTok{= Arbitrary functions should \_just work\#emoji.tm;\_}

\NormalTok{/ \#wrap{-}in(x =\textgreater{} ellipse(align(center, x),}
\NormalTok{    stroke: 3pt + gradient.conic(..color.map.rainbow)}
\NormalTok{  )):}
\NormalTok{  Some text in a \_rainbow ellipse\_ spread}
\NormalTok{  over multiple lines so it actually looks}
\NormalTok{  like it was typed in a document.}


\NormalTok{= One{-}liners look great!}

\NormalTok{/ \#wrap{-}in(underline): Here\textquotesingle{}s one line underlined}


\NormalTok{= Let\textquotesingle{}s do some math:}

\NormalTok{\#let named{-}thm(name) = (content) =\textgreater{} \{}
\NormalTok{  pad(left: 2em, par(hanging{-}indent: {-}2em)[}
\NormalTok{    *Theorem* (\#name) \#emph(content)}
\NormalTok{  ])}
\NormalTok{\}}

\NormalTok{/ \#wrap{-}in(named{-}thm("Operational Soundness")):}
\NormalTok{  If $med tack e : tau$ and $e$ reduces to $e\textquotesingle{}$}
\NormalTok{  by zero or more steps and $"Irred"(e\textquotesingle{})$,}
\NormalTok{  then $e\textquotesingle{} in "Val"$ and $med tack e\textquotesingle{} : tau$.}


\NormalTok{= In{-}line styling doesn\textquotesingle{}t create blocks:}

\NormalTok{/ \#wrap{-}in(highlight):}
\NormalTok{  This text is highlighted.}
\NormalTok{This text isn\textquotesingle{}t.}

\NormalTok{Notice how there was *no* paragraph break between the}
\NormalTok{two sentences? This is a useful result that makes}
\NormalTok{\textasciigrave{}wrap{-}indent\textasciigrave{} really flexible!}

\NormalTok{(if you want separate blocks, use \textasciigrave{}block\textasciigrave{} in your function)}


\NormalTok{= Does it work with nesting?}

\NormalTok{/ \#wrap{-}in(custom{-}quote):}
\NormalTok{  Testing...}
\NormalTok{  / \#wrap{-}in(align.with(center)):}
\NormalTok{    / \#wrap{-}in(rect):}
\NormalTok{      / \#wrap{-}in(emph):}
\NormalTok{        Signs point to yes!}


\NormalTok{= Final thoughts}

\NormalTok{/ Note {-}{-}:}
\NormalTok{  Regular term lists still work!}

\NormalTok{/ Disclaimer {-}{-}:}
\NormalTok{  You may run into issues with other term list}
\NormalTok{  show rules conflicting with this rule. \textbackslash{}}
\NormalTok{  (although set rules should be unaffected)}

\NormalTok{  If you run into issues, \_let me know!\_ I\textquotesingle{}d love to hear}
\NormalTok{  about it to make this package as robust as possible.}


\NormalTok{= And}

\NormalTok{\#let big{-}statement(content) = \{}
\NormalTok{  align(center, text(}
\NormalTok{    underline(stroke: 1.5pt, content),}
\NormalTok{    size: 32pt,}
\NormalTok{    weight: "bold",}
\NormalTok{    style: "italic",}
\NormalTok{    fill: eastern,}
\NormalTok{  ))}
\NormalTok{\}}

\NormalTok{/ \#wrap{-}in(big{-}statement):}
\NormalTok{  That\textquotesingle{}s a wrap!}
\end{Highlighting}
\end{Shaded}

\subsection{References}\label{references}

You can find my original writeup here for more context:\\
\url{https://typst.app/project/r5ogFas7lj7E48iHw_M4yh}

And also see the GitHub issue that prompted me to make this:\\
\url{https://github.com/typst/typst/issues/1921}

\subsubsection{How to add}\label{how-to-add}

Copy this into your project and use the import as
\texttt{\ wrap-indent\ }

\begin{verbatim}
#import "@preview/wrap-indent:0.1.0"
\end{verbatim}

\includesvg[width=0.16667in,height=0.16667in]{/assets/icons/16-copy.svg}

Check the docs for
\href{https://typst.app/docs/reference/scripting/\#packages}{more
information on how to import packages} .

\subsubsection{About}\label{about}

\begin{description}
\tightlist
\item[Author :]
Ian Wrzesinski (LectronPusher)
\item[License:]
MIT
\item[Current version:]
0.1.0
\item[Last updated:]
May 3, 2024
\item[First released:]
May 3, 2024
\item[Archive size:]
3.69 kB
\href{https://packages.typst.org/preview/wrap-indent-0.1.0.tar.gz}{\pandocbounded{\includesvg[keepaspectratio]{/assets/icons/16-download.svg}}}
\item[Categor ies :]
\begin{itemize}
\tightlist
\item[]
\item
  \pandocbounded{\includesvg[keepaspectratio]{/assets/icons/16-code.svg}}
  \href{https://typst.app/universe/search/?category=scripting}{Scripting}
\item
  \pandocbounded{\includesvg[keepaspectratio]{/assets/icons/16-hammer.svg}}
  \href{https://typst.app/universe/search/?category=utility}{Utility}
\end{itemize}
\end{description}

\subsubsection{Where to report issues?}\label{where-to-report-issues}

This package is a project of Ian Wrzesinski (LectronPusher) . You can
also try to ask for help with this package on the
\href{https://forum.typst.app}{Forum} .

Please report this package to the Typst team using the
\href{https://typst.app/contact}{contact form} if you believe it is a
safety hazard or infringes upon your rights.

\phantomsection\label{versions}
\subsubsection{Version history}\label{version-history}

\begin{longtable}[]{@{}ll@{}}
\toprule\noalign{}
Version & Release Date \\
\midrule\noalign{}
\endhead
\bottomrule\noalign{}
\endlastfoot
0.1.0 & May 3, 2024 \\
\end{longtable}

Typst GmbH did not create this package and cannot guarantee correct
functionality of this package or compatibility with any version of the
Typst compiler or app.
