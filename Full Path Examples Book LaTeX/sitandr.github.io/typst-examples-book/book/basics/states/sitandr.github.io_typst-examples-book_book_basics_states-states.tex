\title{sitandr.github.io/typst-examples-book/book/basics/states/states}

\section{\texorpdfstring{\hyperref[states]{States}}{States}}\label{states}

This section is outdated. It may be still useful, but it is strongly
recommended to study new context system (using the reference).

Before we start something practical, it is important to understand
states in general.

Here is a good explanation of why do we \emph{need} them:
\href{https://typst.app/docs/reference/meta/state/}{Official Reference
about states} . It is highly recommended to read it first.

So instead of

\begin{verbatim}
#let x = 0
#let compute(expr) = {
  // eval evaluates string as Typst code
  // to calculate new x value
  x = eval(
    expr.replace("x", str(x))
  )
  [New value is #x.]
}

#compute("10") \
#compute("x + 3") \
#compute("x * 2") \
#compute("x - 5")
\end{verbatim}

\textbf{THIS DOES NOT COMPILE:} Variables from outside the function are
read-only and cannot be modified

Instead, you should write

\begin{verbatim}
#let s = state("x", 0)
#let compute(expr) = [
  // updates x current state with this function
  #s.update(x =>
    eval(expr.replace("x", str(x)))
  )
  // and displays it
  New value is #context s.get().
]

#compute("10") \
#compute("x + 3") \
#compute("x * 2") \
#compute("x - 5")

The computations will be made _in order_ they are _located_ in the document. So if you create computations first, but put them in the document later... See yourself:

#let more = [
  #compute("x * 2") \
  #compute("x - 5")
]

#compute("10") \
#compute("x + 3") \
#more
\end{verbatim}

\pandocbounded{\includesvg[keepaspectratio]{typst-img/9a88397d1a9b5a44b1a3a218894595121bd4c5ec875df2b960638f2925060334-1.svg}}

\subsection{\texorpdfstring{\hyperref[context-magic]{Context
magic}}{Context magic}}\label{context-magic}

So what does this magic
\texttt{\ }{\texttt{\ context\ s.get()\ }}\texttt{\ } mean?

\begin{quote}
\href{https://typst.app/docs/reference/context/}{Context in Reference}
\end{quote}

In short, it specifies what part of your code (or markup) can
\emph{depend on states outside} . This context-expression is packed then
as one object, and it is evaluated on layout stage.

That means it is impossible to look from "normal" code at whatever is
inside the \texttt{\ }{\texttt{\ context\ }}\texttt{\ } . This is a
black box that would be known \emph{only after putting it into the
document} .

We will discuss \texttt{\ }{\texttt{\ context\ }}\texttt{\ } features
later.

\subsection{\texorpdfstring{\hyperref[operations-with-states]{Operations
with states}}{Operations with states}}\label{operations-with-states}

\subsubsection{\texorpdfstring{\hyperref[creating-new-state]{Creating
new state}}{Creating new state}}\label{creating-new-state}

\begin{verbatim}
#let x = state("state-id")
#let y = state("state-id", 2)

#x, #y

State is #context x.get() \ // the same as
#context [State is #y.get()] \ // the same as
#context {"State is" + str(y.get())}
\end{verbatim}

\pandocbounded{\includesvg[keepaspectratio]{typst-img/4a52375bdeea2b7ca31dc51740563d01b3678f817dd6bc8c349d0714c2ac503f-1.svg}}

\subsubsection{\texorpdfstring{\hyperref[update]{Update}}{Update}}\label{update}

Updating is \emph{a content} that is an instruction. That instruction
tells compiler that in this place of document the state \emph{should be
updated} .

\begin{verbatim}
#let x = state("x", 0)
#context x.get() \
#let _ = x.update(3)
// nothing happens, we don't put `update` into the document flow
#context x.get()

#repr(x.update(3)) // this is how that content looks \

#context x.update(3)
#context x.get() // Finally!
\end{verbatim}

\pandocbounded{\includesvg[keepaspectratio]{typst-img/3732a9c7bca8c4faedf9b024e09e647a65222c8244e9f3235a6057dfebc0a511-1.svg}}

Here we can see one of \emph{important
\texttt{\ }{\texttt{\ context\ }}\texttt{\ } traits} : it "sees" states
from outside, but can\textquotesingle t see how they change inside it:

\begin{verbatim}
#let x = state("x", 0)

#context {
  x.update(3)
  str(x.get())
}
\end{verbatim}

\pandocbounded{\includesvg[keepaspectratio]{typst-img/78e500b80cb85e086a81302e2ce3dad88cb4304d4685b88e3f59111bc71f6748-1.svg}}

\subsubsection{\texorpdfstring{\hyperref[id-collision]{ID
collision}}{ID collision}}\label{id-collision}

\emph{TLDR; \textbf{Never allow colliding states.}}

States are described by their id-s, if they are the same, the code will
break.

So, if you write functions or loops that are used several times,
\emph{be careful} !

\begin{verbatim}
#let f(x) = {
  // return new state…
  // …but their id-s are the same!
  // so it will always be the same state!
  let y = state("x", 0)
  y.update(y => y + x)
  context y.get()
}

#let a = f(2)
#let b = f(3)

#a, #b \
#raw(repr(a) + "\n" + repr(b))
\end{verbatim}

\pandocbounded{\includesvg[keepaspectratio]{typst-img/31a3e88747ed09ae6078bd3caf986f0e6ba744e055d0889d92bfa23941e7e451-1.svg}}

However, this \emph{may seem} okay:

\begin{verbatim}
// locations in code are different!
#let x = state("state-id")
#let y = state("state-id", 2)

#x, #y
\end{verbatim}

\pandocbounded{\includesvg[keepaspectratio]{typst-img/1901e1449942d821c66f53bd6bc5fda10d63591aa45346fdf88bcbc3f2ab3425-1.svg}}

But in fact, it \emph{isn\textquotesingle t} :

\begin{verbatim}
#let x = state("state-id")
#let y = state("state-id", 2)

#context [#x.get(); #y.get()]

#x.update(3)

#context [#x.get(); #y.get()]
\end{verbatim}

\pandocbounded{\includesvg[keepaspectratio]{typst-img/9185a298f9bcf8c519fa85481b9272e6ef3a00c117a0904d0509920a6abef8b2-1.svg}}
