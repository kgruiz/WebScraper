\title{sitandr.github.io/typst-examples-book/book/basics/states/counters}

\section{\texorpdfstring{\hyperref[counters]{Counters}}{Counters}}\label{counters}

This section is outdated. It may be still useful, but it is strongly
recommended to study new context system (using the reference).

Counters are special states that \emph{count} elements of some type. As
with states, you can create your own with identifier strings.

\emph{Important:} to initiate counters of elements, you need to
\emph{set numbering for them} .

\subsection{\texorpdfstring{\hyperref[states-methods]{States
methods}}{States methods}}\label{states-methods}

Counters are states, so they can do all things states can do.

\begin{verbatim}
#set heading(numbering: "1.")

= Background
#counter(heading).update(3)
#counter(heading).update(n => n * 2)

== Analysis
Current heading number: #counter(heading).display().
\end{verbatim}

\pandocbounded{\includesvg[keepaspectratio]{typst-img/c57c9907a5f238f0b5eee74f8c23c57a5e2d5b0c9cbf7ebd1befdfcbd33289df-1.svg}}

\begin{verbatim}
#let mine = counter("mycounter")
#mine.display()

#mine.step()
#mine.display()

#mine.update(c => c * 3)
#mine.display()
\end{verbatim}

\pandocbounded{\includesvg[keepaspectratio]{typst-img/876103777c9564f0bb524f83a988a6d444c4e889baed31ee960548d90f3233e2-1.svg}}

\subsection{\texorpdfstring{\hyperref[displaying-counters]{Displaying
counters}}{Displaying counters}}\label{displaying-counters}

\begin{verbatim}
#set heading(numbering: "1.")

= Introduction
Some text here.

= Background
The current value is:
#counter(heading).display()

Or in roman numerals:
#counter(heading).display("I")
\end{verbatim}

\pandocbounded{\includesvg[keepaspectratio]{typst-img/1ac65f4be42131b3cca1d7c56c6c60c3932a703e5e499c1c5cb874458028abea-1.svg}}

Counters also support displaying \emph{both current and final values}
out-of-box:

\begin{verbatim}
#set heading(numbering: "1.")

= Introduction
Some text here.

#counter(heading).display(both: true) \
#counter(heading).display("1 of 1", both: true) \
#counter(heading).display(
  (num, max) => [#num of #max],
   both: true
)

= Background
The current value is: #counter(heading).display()
\end{verbatim}

\pandocbounded{\includesvg[keepaspectratio]{typst-img/af9d0da905bbb2215461b07b39653ef3890ff11a364afe018dae4ce4216f4961-1.svg}}

\subsection{\texorpdfstring{\hyperref[step]{Step}}{Step}}\label{step}

That\textquotesingle s quite easy, for counters you can increment value
using \texttt{\ }{\texttt{\ step\ }}\texttt{\ } . It works the same way
as \texttt{\ }{\texttt{\ update\ }}\texttt{\ } .

\begin{verbatim}
#set heading(numbering: "1.")

= Introduction
#counter(heading).step()

= Analysis
Let's skip 3.1.
#counter(heading).step(level: 2)

== Analysis
At #counter(heading).display().
\end{verbatim}

\pandocbounded{\includesvg[keepaspectratio]{typst-img/12446a2258e9862d8df8b6b250ff14efbb9c35da165a2a04e8c4aa12c9b68cdf-1.svg}}

\subsection{\texorpdfstring{\hyperref[you-can-use-counters-in-your-functions]{You
can use counters in your
functions:}}{You can use counters in your functions:}}\label{you-can-use-counters-in-your-functions}

\begin{verbatim}
#let c = counter("theorem")
#let theorem(it) = block[
  #c.step()
  *Theorem #c.display():*
  #it
]

#theorem[$1 = 1$]
#theorem[$2 < 3$]
\end{verbatim}

\pandocbounded{\includesvg[keepaspectratio]{typst-img/0f178f614e49a7400d646926705364a92ca3d4d888423b2693f071f83ce09e7d-1.svg}}
