\title{sitandr.github.io/typst-examples-book/book/basics/scripting/conditions}

\section{\texorpdfstring{\hyperref[conditions--loops]{Conditions \&
loops}}{Conditions \& loops}}\label{conditions--loops}

\subsection{\texorpdfstring{\hyperref[conditions]{Conditions}}{Conditions}}\label{conditions}

\begin{quote}
See
\href{https://typst.app/docs/reference/scripting/\#conditionals}{official
documentation} .
\end{quote}

In Typst, you can use \texttt{\ }{\texttt{\ if-else\ }}\texttt{\ }
statements. This is especially useful inside function bodies to vary
behavior depending on arguments types or many other things.

\begin{verbatim}
#if 1 < 2 [
  This is shown
] else [
  This is not.
]
\end{verbatim}

\pandocbounded{\includesvg[keepaspectratio]{typst-img/2e914defa3353d6fd42ed58c37a97aedcc2237cfe20228f0cc0d223dfff4619a-1.svg}}

Of course, \texttt{\ }{\texttt{\ else\ }}\texttt{\ } is unnecessary:

\begin{verbatim}
#let a = 3

#if a < 4 {
  a = 5
}

#a
\end{verbatim}

\pandocbounded{\includesvg[keepaspectratio]{typst-img/a7264774be154606a44d829d31edae18bf686262ccea66de9ed97fa20c720bd8-1.svg}}

You can also use \texttt{\ }{\texttt{\ else\ if\ }}\texttt{\ } statement
(known as \texttt{\ }{\texttt{\ elif\ }}\texttt{\ } in Python):

\begin{verbatim}
#let a = 5

#if a < 4 {
  a = 5
} else if a < 6 {
  a = -3
}

#a
\end{verbatim}

\pandocbounded{\includesvg[keepaspectratio]{typst-img/9f65678fc26af2d197d979e1b0a5295ed64037ee00c30fa28c9c417a2c7dc308-1.svg}}

\subsubsection{\texorpdfstring{\hyperref[booleans]{Booleans}}{Booleans}}\label{booleans}

\texttt{\ }{\texttt{\ if,\ else\ if,\ else\ }}\texttt{\ } accept
\emph{only boolean} values as a switch. You can combine booleans as
described in \href{./types.html\#boolean-bool}{types section} :

\begin{verbatim}
#let a = 5

#if (a > 1 and a <= 4) or a == 5 [
    `a` matches the condition
]
\end{verbatim}

\pandocbounded{\includesvg[keepaspectratio]{typst-img/21d3a48404d4e0c59bc0fccb114fdeac7384189db0020247796f44b0e9a7c362-1.svg}}

\subsection{\texorpdfstring{\hyperref[loops]{Loops}}{Loops}}\label{loops}

\begin{quote}
See \href{https://typst.app/docs/reference/scripting/\#loops}{official
documentation} .
\end{quote}

There are two kinds of loops: \texttt{\ }{\texttt{\ while\ }}\texttt{\ }
and \texttt{\ }{\texttt{\ for\ }}\texttt{\ } . While repeats body while
the condition is met:

\begin{verbatim}
#let a = 3

#while a < 100 {
    a *= 2
    str(a)
    " "
}
\end{verbatim}

\pandocbounded{\includesvg[keepaspectratio]{typst-img/ece06c012663616cac05b0f365bd02ea5607dcddfaa0249963088ceff797c100-1.svg}}

\texttt{\ }{\texttt{\ for\ }}\texttt{\ } iterates over all elements of
sequence. The sequence may be an
\texttt{\ }{\texttt{\ array\ }}\texttt{\ } ,
\texttt{\ }{\texttt{\ string\ }}\texttt{\ } or
\texttt{\ }{\texttt{\ dictionary\ }}\texttt{\ } (
\texttt{\ }{\texttt{\ for\ }}\texttt{\ } iterates over its
\emph{key-value pairs} ).

\begin{verbatim}
#for c in "ABC" [
  #c is a letter.
]
\end{verbatim}

\pandocbounded{\includesvg[keepaspectratio]{typst-img/9e70091e4c1f276d548f8200329298bf6b98946c331ca4630fec8313d5a91eff-1.svg}}

To iterate to all numbers from \texttt{\ }{\texttt{\ a\ }}\texttt{\ } to
\texttt{\ }{\texttt{\ b\ }}\texttt{\ } , use
\texttt{\ }{\texttt{\ range(a,\ b+1)\ }}\texttt{\ } :

\begin{verbatim}
#let s = 0

#for i in range(3, 6) {
    s += i
    [Number #i is added to sum. Now sum is #s.]
}
\end{verbatim}

\pandocbounded{\includesvg[keepaspectratio]{typst-img/1e3d95ee79d7bc6989e40ff1e27c0ef6e3b152a1e5f8a0df5b2819621e0e299f-1.svg}}

Because range is end-exclusive this is equal to

\begin{verbatim}
#let s = 0

#for i in (3, 4, 5) {
    s += i
    [Number #i is added to sum. Now sum is #s.]
}
\end{verbatim}

\pandocbounded{\includesvg[keepaspectratio]{typst-img/6158d29261339f8f285d592deff8992ca129ce32264abcdcf6734ac44cf558a4-1.svg}}

\begin{verbatim}
#let people = (Alice: 3, Bob: 5)

#for (name, value) in people [
    #name has #value apples.
]
\end{verbatim}

\pandocbounded{\includesvg[keepaspectratio]{typst-img/50ff0963afe8c9ec5a0562d518431b63d5dd3810525f55f084f812452b11eb21-1.svg}}

\subsubsection{\texorpdfstring{\hyperref[break-and-continue]{Break and
continue}}{Break and continue}}\label{break-and-continue}

Inside loops can be used \texttt{\ }{\texttt{\ break\ }}\texttt{\ } and
\texttt{\ }{\texttt{\ continue\ }}\texttt{\ } commands.
\texttt{\ }{\texttt{\ break\ }}\texttt{\ } breaks loop, jumping outside.
\texttt{\ }{\texttt{\ continue\ }}\texttt{\ } jumps to next loop
iteration.

See the difference on these examples:

\begin{verbatim}
#for letter in "abc nope" {
  if letter == " " {
    // stop when there is space
    break
  }

  letter
}
\end{verbatim}

\pandocbounded{\includesvg[keepaspectratio]{typst-img/a744551cab635d3ab70d9bf4258bb5fc26fe384f8e9f487ad0b8eee986ffe581-1.svg}}

\begin{verbatim}
#for letter in "abc nope" {
  if letter == " " {
    // skip the space
    continue
  }

  letter
}
\end{verbatim}

\pandocbounded{\includesvg[keepaspectratio]{typst-img/bbb719820f986e52fbf64306536766ecbfd7264d29429a5c62d1bd648a4754c5-1.svg}}
