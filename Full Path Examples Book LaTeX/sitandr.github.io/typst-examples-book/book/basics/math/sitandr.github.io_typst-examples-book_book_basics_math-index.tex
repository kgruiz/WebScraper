\title{sitandr.github.io/typst-examples-book/book/basics/math/index}

\section{\texorpdfstring{\hyperref[math]{Math}}{Math}}\label{math}

Math is a special environment that has special features related to...
math.

\subsection{\texorpdfstring{\hyperref[syntax]{Syntax}}{Syntax}}\label{syntax}

To start math environment, \texttt{\ }{\texttt{\ \$\ }}\texttt{\ } . The
spacing around \texttt{\ }{\texttt{\ \$\ }}\texttt{\ } will make it
either \emph{inline} math (smaller, used in text) or \emph{display} math
(used on math equations on their own).

\begin{verbatim}
// This is inline math
Let $a$, $b$, and $c$ be the side
lengths of right-angled triangle.
Then, we know that:

// This is display math
$ a^2 + b^2 = c^2 $

Prove by induction:

// You can use new lines as spacing too!
$
sum_(k=1)^n k = (n(n+1)) / 2
$
\end{verbatim}

\pandocbounded{\includesvg[keepaspectratio]{typst-img/068db3a521a38c3acede771ebb6342807cca4fd98baf5b2b508184a6854ea8ff-1.svg}}

\subsection{\texorpdfstring{\hyperref[mathequation]{Math.equation}}{Math.equation}}\label{mathequation}

The element that math is displayed in is called
\texttt{\ }{\texttt{\ math.equation\ }}\texttt{\ } . You can use it for
set/show rules:

\begin{verbatim}
#show math.equation: set text(red)

$
integral_0^oo (f(t) + g(t))/2
$
\end{verbatim}

\pandocbounded{\includesvg[keepaspectratio]{typst-img/94e0532dd7224d08e966cb82834283efd8889d7f117b04116e721a788bfcc16c-1.svg}}

Any symbol/command that is available in math, \emph{is also available}
in code mode using \texttt{\ }{\texttt{\ math.command\ }}\texttt{\ } :

\begin{verbatim}
#math.integral, #math.underbrace([a + b], [c])
\end{verbatim}

\pandocbounded{\includesvg[keepaspectratio]{typst-img/b4ca12d7f34ed342f3cb3fba2ee1f5b58faa8fceecb74671baacd9166fcbb5aa-1.svg}}

\subsection{\texorpdfstring{\hyperref[letters-and-commands]{Letters and
commands}}{Letters and commands}}\label{letters-and-commands}

Typst aims to have as simple and effective syntax for math as possible.
That means no special symbols, just using commands.

To make it short, Typst uses several simple rules:

\begin{itemize}
\item
  All single-letter words \emph{turn into variables} . That includes any
  \emph{unicode symbols} too!
\item
  All multi-letter words \emph{turn into commands} . They may be
  built-in commands (available with math.something outside of math
  environment). Or they \textbf{may be user-defined variables/functions}
  . If the command \textbf{isn\textquotesingle t defined} , there will
  be \textbf{compilation error} .

  If you use kebab-case or snake\_case for variables you want to use in
  math, you will have to refer to them as \#snake-case-variable.
\item
  To write simple text, use quotes:

\begin{verbatim}
$a "equals to" 2$
\end{verbatim}

  \pandocbounded{\includesvg[keepaspectratio]{typst-img/811f30ede68d08bec254f184c1be319958c3e11f9f9d58c40b2f460bba037e3d-1.svg}}

  Spacing matters there!

\begin{verbatim}
$a "is" 2$, $a"is"2$
\end{verbatim}

  \pandocbounded{\includesvg[keepaspectratio]{typst-img/9cc2d263c76646c623e1e6b73756e1fe1e2c56d7fe0324ee945652107e6456ba-1.svg}}
\item
  You can turn it into multi-letter variables using
  \texttt{\ }{\texttt{\ italic\ }}\texttt{\ } :

\begin{verbatim}
$(italic("mass") v^2)/2$
\end{verbatim}

  \pandocbounded{\includesvg[keepaspectratio]{typst-img/141d8a3b9beb3559387411170f7378078c80cb2ff80d8d5f5345c3231f55df9c-1.svg}}
\end{itemize}

Commands see
\href{https://typst.app/docs/reference/math/\#definitions}{there} (go to
the links to see the commands).

All symbols see
\href{https://typst.app/docs/reference/symbols/sym/}{there} .

\subsection{\texorpdfstring{\hyperref[multiline-equations]{Multiline
equations}}{Multiline equations}}\label{multiline-equations}

To create multiline \emph{display equation} , use the same symbol as in
markup mode: \texttt{\ }{\texttt{\ \textbackslash{}\ }}\texttt{\ } :

\begin{verbatim}
$
a = b\
a = c
$
\end{verbatim}

\pandocbounded{\includesvg[keepaspectratio]{typst-img/2f16d9e64e38ff22ca27a09b0d8eaef1b020e4eccd7d2ce1380e10a0efcea163-1.svg}}

\subsection{\texorpdfstring{\hyperref[escaping]{Escaping}}{Escaping}}\label{escaping}

Any symbol that is used may be escaped with
\texttt{\ }{\texttt{\ \textbackslash{}\ }}\texttt{\ } , like in markup
mode. For example, you can disable fraction:

\begin{verbatim}
$
a  / b \
a \/ b
$
\end{verbatim}

\pandocbounded{\includesvg[keepaspectratio]{typst-img/e7931e55a2772ee992446af8506d8d25b96167e3bb71d5c63ed8ca156530f2d9-1.svg}}

The same way it works with any other syntax.

\subsection{\texorpdfstring{\hyperref[wrapping-inline-math]{Wrapping
inline math}}{Wrapping inline math}}\label{wrapping-inline-math}

Sometimes, when you write large math, it may be too close to text
(especially for some long letter tails).

\begin{verbatim}
#lorem(17) $display(1)/display(1+x^n)$ #lorem(20)
\end{verbatim}

\pandocbounded{\includesvg[keepaspectratio]{typst-img/a9cce2b851a01939a0abfc02e8cd994d20c465d2800cf64c5c6051ead5bc4e9a-1.svg}}

You may easily increase the distance it by wrapping into box:

\begin{verbatim}
#lorem(17) #box($display(1)/display(1+x^n)$, inset: 0.2em) #lorem(20)
\end{verbatim}

\pandocbounded{\includesvg[keepaspectratio]{typst-img/ee9fc5a3ec529a9f3e811a70724c1585c294d82454c22ee9343235556f572792-1.svg}}
