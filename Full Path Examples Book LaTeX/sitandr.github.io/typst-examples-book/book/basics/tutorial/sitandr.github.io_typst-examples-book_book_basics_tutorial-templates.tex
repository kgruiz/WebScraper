\title{sitandr.github.io/typst-examples-book/book/basics/tutorial/templates}

\section{\texorpdfstring{\hyperref[templates]{Templates}}{Templates}}\label{templates}

\subsection{\texorpdfstring{\hyperref[templates-1]{Templates}}{Templates}}\label{templates-1}

If you want to reuse styling in other files, you can use the
\emph{template} idiom. Because \texttt{\ }{\texttt{\ set\ }}\texttt{\ }
and \texttt{\ }{\texttt{\ show\ }}\texttt{\ } rules are only active in
their current scope, they will not affect content in a file you imported
your file into. But functions can circumvent this in a predictable way:

\begin{verbatim}
// define a function that:
// - takes content
// - applies styling to it
// - returns the styled content
#let apply-template(body) = [
  #show heading.where(level: 1): emph
  #set heading(numbering: "1.1")
  // ...
  #body
]
\end{verbatim}

This is equivalent to:

\begin{verbatim}
// we can reduce the number of hashes needed here by using scripting mode
// same as above but we exchanged `[...]` for `{...}` to switch from markup
// into scripting mode
#let apply-template(body) = {
  show heading.where(level: 1): emph
  set heading(numbering: "1.1")
  // ...
  body
}
\end{verbatim}

Then in your main file:

\begin{verbatim}
#import "template.typ": apply-template
#show: apply-template
\end{verbatim}

\emph{This will apply a "template" function to the rest of your
document!}

\subsubsection{\texorpdfstring{\hyperref[passing-arguments]{Passing
arguments}}{Passing arguments}}\label{passing-arguments}

\begin{verbatim}
// add optional named arguments
#let apply-template(body, name: "My document") = {
  show heading.where(level: 1): emph
  set heading(numbering: "1.1")

  align(center, text(name, size: 2em))

  body
}
\end{verbatim}

Then, in template file:

\begin{verbatim}
#import "template.typ": apply-template

// `func.with(..)` applies the arguments to the function and returns the new
// function with those defaults applied
#show: apply-template.with(name: "Report")

// it is functionally the same as this
#let new-template(..args) = apply-template(name: "Report", ..args)
#show: new-template
\end{verbatim}

Writing templates is fairly easy if you understand
\href{../scripting/index.html}{scripting} .

See more information about writing templates in
\href{https://typst.app/docs/tutorial/making-a-template/}{Official
Tutorial} .

There is no official repository for templates yet, but there are a
plenty community ones in
\href{https://github.com/qjcg/awesome-typst?ysclid=lj8pur1am7431908794\#general}{awesome-typst}
.
