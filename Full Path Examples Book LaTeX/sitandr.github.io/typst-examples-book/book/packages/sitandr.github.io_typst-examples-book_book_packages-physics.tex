\title{sitandr.github.io/typst-examples-book/book/packages/physics}

\section{\texorpdfstring{\hyperref[physics]{Physics}}{Physics}}\label{physics}

\subsection{\texorpdfstring{\hyperref[physica]{\texttt{\ }{\texttt{\ physica\ }}\texttt{\ }}}{  physica  }}\label{physica}

\begin{quote}
Physica (Latin for \emph{natural sciences} ) provides utilities that
simplify otherwise complex and repetitive mathematical expressions in
natural sciences.
\end{quote}

\begin{quote}
Its
\href{https://github.com/Leedehai/typst-physics/blob/master/physica-manual.pdf}{manual}
provides a full set of demonstrations of how the package could be
helpful.
\end{quote}

\subsubsection{\texorpdfstring{\hyperref[mathematical-physics]{Mathematical
physics}}{Mathematical physics}}\label{mathematical-physics}

The \href{./math.html\#common-notations}{packages/math.md} page has more
examples on its math capabilities. Below is a preview that may be of
particular interest in the domain of physics:

\begin{itemize}
\tightlist
\item
  Calculus: differential, ordinary and partial derivatives

  \begin{itemize}
  \tightlist
  \item
    Optional function name,
  \item
    Optional order number or array of order numbers,
  \item
    Customizable "d" symbol and product joiner (say, exterior product),
  \item
    Overridable total order calculation,
  \end{itemize}
\item
  Vectors and vector fields: div, grad, curl,
\item
  Taylor expansion,
\item
  Dirac braket notations,
\item
  Tensors with abstract index notations,
\item
  Matrix transpose and dagger (conjugate transpose).
\item
  Special matrices: determinant, (anti-)diagonal, identity, zero,
  Jacobian, Hessian, etc.
\end{itemize}

A partial glimpse:

\begin{verbatim}
#import "@preview/physica:0.9.1": *
#show: super-T-as-transpose // put in a #[...] to limit its scope...
#show: super-plus-as-dagger // ...or use scripts() to manually override

$ dd(x,y,2) quad dv(f,x,d:Delta)      quad pdv(,x,y,[2i+1,2+i]) quad
  vb(a) va(a) vu(a_i)  quad mat(laplacian, div; grad, curl)     quad
  tensor(T,+a,-b,+c)   quad ket(phi)  quad A^+ e^scripts(+) A^T integral^T $
\end{verbatim}

\pandocbounded{\includesvg[keepaspectratio]{typst-img/fa8a12d2904a08958d4f83d69dda6bb38308b431055a25790d286a250e364c6c-1.svg}}

\subsubsection{\texorpdfstring{\hyperref[isotopes]{Isotopes}}{Isotopes}}\label{isotopes}

\begin{verbatim}
#import "@preview/physica:0.9.1": isotope

// a: mass number A
// z: the atomic number Z
$
isotope(I, a:127), quad isotope("Fe", z:26), quad
isotope("Tl",a:207,z:81) --> isotope("Pb",a:207,z:82) + isotope(e,a:0,z:-1)
$
\end{verbatim}

\pandocbounded{\includesvg[keepaspectratio]{typst-img/b290d801c6760a41e50520401d9e72cb63a8691aa136308cbad87349e7e436f0-1.svg}}

\subsubsection{\texorpdfstring{\hyperref[reduced-planck-constant-hbar]{Reduced
Planck constant
(hbar)}}{Reduced Planck constant (hbar)}}\label{reduced-planck-constant-hbar}

In the default font, the Typst built-in symbol
\texttt{\ }{\texttt{\ planck.reduce\ }}\texttt{\ } looks a bit off: on
letter "h" there is a slash instead of a horizontal bar, contrary to the
symbol\textquotesingle s colloquial name "h-bar". This package offers
\texttt{\ }{\texttt{\ hbar\ }}\texttt{\ } to render the symbol in the
familiar form⁠. Contrast:

\begin{verbatim}
#import "@preview/physica:0.9.1": hbar

$ E = planck.reduce omega => E = hbar omega, wide
  frac(planck.reduce^2, 2m) => frac(hbar^2, 2m), wide
  (pi G^2) / (planck.reduce c^4) => (pi G^2) / (hbar c^4), wide
  e^(frac(i(p x - E t), planck.reduce)) => e^(frac(i(p x - E t), hbar)) $
\end{verbatim}

\pandocbounded{\includesvg[keepaspectratio]{typst-img/efab3b0486d1cddc3388248c4100e1cc919088cdb93f3e072001547c40005f01-1.svg}}

\subsection{\texorpdfstring{\hyperref[quill-quantum-diagrams]{\texttt{\ }{\texttt{\ quill\ }}\texttt{\ }
: quantum
diagrams}}{  quill   : quantum diagrams}}\label{quill-quantum-diagrams}

\begin{quote}
See \href{https://github.com/Mc-Zen/quill/tree/main}{documentation} .
\end{quote}

\begin{verbatim}
#import "@preview/quill:0.2.0": *
#quantum-circuit(
  lstick($|0〉$), gate($H$), ctrl(1), rstick($(|00〉+|11〉)/√2$, n: 2), [\ ],
  lstick($|0〉$), 1, targ(), 1
)
\end{verbatim}

\pandocbounded{\includesvg[keepaspectratio]{typst-img/bd14c65cd60e1efc4d15ae7234e364c6d5740a168e2cb275743ed1fbcc9483eb-1.svg}}

\begin{verbatim}
#import "@preview/quill:0.2.0": *

#let ancillas = (setwire(0), 5, lstick($|0〉$), setwire(1), targ(), 2, [\ ],
setwire(0), 5, lstick($|0〉$), setwire(1), 1, targ(), 1)

#quantum-circuit(
  scale-factor: 80%,
  lstick($|ψ〉$), 1, 10pt, ctrl(3), ctrl(6), $H$, 1, 15pt, 
    ctrl(1), ctrl(2), 1, [\ ],
  ..ancillas, [\ ],
  lstick($|0〉$), 1, targ(), 1, $H$, 1, ctrl(1), ctrl(2), 
    1, [\ ],
  ..ancillas, [\ ],
  lstick($|0〉$), 2, targ(),  $H$, 1, ctrl(1), ctrl(2), 
    1, [\ ],
  ..ancillas
)
\end{verbatim}

\pandocbounded{\includesvg[keepaspectratio]{typst-img/597640923e31369199c6e7158de9094a2c94f2c5dae6ced72c6b83b1067fa8e4-1.svg}}

\begin{verbatim}
#import "@preview/quill:0.2.0": *

#quantum-circuit(
  lstick($|psi〉$),  ctrl(1), gate($H$), 1, ctrl(2), meter(), [\ ],
  lstick($|beta_00〉$, n: 2), targ(), 1, ctrl(1), 1, meter(), [\ ],
  3, gate($X$), gate($Z$),  midstick($|psi〉$)
)
\end{verbatim}

\pandocbounded{\includesvg[keepaspectratio]{typst-img/cc71bc052c7a80c702289f780ee42a168c1491076dd5934408373895ca95c35e-1.svg}}
