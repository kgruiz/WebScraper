\title{sitandr.github.io/typst-examples-book/book/packages/math}

\section{\texorpdfstring{\hyperref[math]{Math}}{Math}}\label{math}

\subsection{\texorpdfstring{\hyperref[general]{General}}{General}}\label{general}

\subsubsection{\texorpdfstring{\hyperref[physica]{\texttt{\ }{\texttt{\ physica\ }}\texttt{\ }}}{  physica  }}\label{physica}

\begin{quote}
Physica (Latin for \emph{natural sciences} ) provides utilities that
simplify otherwise complex and repetitive mathematical expressions in
natural sciences.
\end{quote}

\begin{quote}
Its
\href{https://github.com/Leedehai/typst-physics/blob/master/physica-manual.pdf}{manual}
provides a full set of demonstrations of how the package could be
helpful.
\end{quote}

\paragraph{\texorpdfstring{\hyperref[common-notations]{Common
notations}}{Common notations}}\label{common-notations}

\begin{itemize}
\tightlist
\item
  Calculus: differential, ordinary and partial derivatives

  \begin{itemize}
  \tightlist
  \item
    Optional function name,
  \item
    Optional order number or an array of thereof,
  \item
    Customizable "d" symbol and product joiner (say, exterior product),
  \item
    Overridable total order calculation,
  \end{itemize}
\item
  Vectors and vector fields: div, grad, curl,
\item
  Taylor expansion,
\item
  Dirac braket notations,
\item
  Tensors with abstract index notations,
\item
  Matrix transpose and dagger (conjugate transpose).
\item
  Special matrices: determinant, (anti-)diagonal, identity, zero,
  Jacobian, Hessian, etc.
\end{itemize}

Below is a preview of those notations.

\begin{verbatim}
#import "@preview/physica:0.9.1": * // Symbol names annotated below

#table(
  columns: 4, align: horizon, stroke: none, gutter: 1em,

  // vectors: bold, unit, arrow
  [$ vb(a), vb(e_i), vu(a), vu(e_i), va(a), va(e_i) $],
  // dprod (dot product), cprod (cross product), iprod (innerproduct)
  [$ a dprod b, a cprod b, iprod(a, b) $],
  // laplacian (different from built-in laplace)
  [$ dot.double(u) = laplacian u =: laplace u $],
  // grad, div, curl (vector fields)
  [$ grad phi, div vb(E), \ curl vb(B) $],
)
\end{verbatim}

\pandocbounded{\includesvg[keepaspectratio]{typst-img/3be7ef86d6c5f7044a42c69fcf93afccd936eb0fcbe987122702c7dda467480f-1.svg}}

\begin{verbatim}
#import "@preview/physica:0.9.1": * // Symbol names annotated below

#table(
  columns: 4, align: horizon, stroke: none, gutter: 1em,

  // Row 1.
  // dd (differential), var (variation), difference
  [$ dd(f), var(f), difference(f) $],
  // dd, with an order number or an array thereof
  [$ dd(x,y), dd(x,y,2), \ dd(x,y,[1,n]), dd(vb(x),t,[3,]) $],
  // dd, with custom "d" symbol and joiner
  [$ dd(x,y,p:and), dd(x,y,d:Delta), \ dd(x,y,z,[1,1,n+1],d:d,p:dot) $],
  // dv (ordinary derivative), with custom "d" symbol and joiner
  [$ dv(phi,t,d:Delta), dv(phi,t,d:upright(D)), dv(phi,t,d:delta) $],

  // Row 2.
  // dv, with optional function name and order
  [$ dv(,t) (dv(x,t)) = dv(x,t,2) $],
  // pdv (partial derivative)
  [$ pdv(f,x,y,2), pdv(,x,y,[k,]) $],
  // pdv, with auto-added overridable total
  [$ pdv(,x,y,[i+2,i+1]), pdv(,y,x,[i+1,i+2],total:3+2i) $],
  // In a flat form
  [$ dv(u,x,s:slash), \ pdv(u,x,y,2,s:slash) $],
)
\end{verbatim}

\pandocbounded{\includesvg[keepaspectratio]{typst-img/0835a840454f88ed2e3b98ddfe37d6f8026729812372a6298d86129611f348c3-1.svg}}

\begin{verbatim}
#import "@preview/physica:0.9.1": * // Symbol names annotated below

#table(
  columns: 3, align: horizon, stroke: none, gutter: 1em,

  // tensor
  [$ tensor(T,+a,-b,-c) != tensor(T,-b,-c,+a) != tensor(T,+a',-b,-c) $],
  // Set builder notation
  [$ Set(p, {q^*, p} = 1) $],
  // taylorterm (Taylor series term)
  [$ taylorterm(f,x,x_0,1) \ taylorterm(f,x,x_0,(n+1)) $],
)
\end{verbatim}

\pandocbounded{\includesvg[keepaspectratio]{typst-img/5c08b65761578f38762229692a33e2b05f096aa8fb7859238b1018240f054d10-1.svg}}

\begin{verbatim}
#import "@preview/physica:0.9.1": * // Symbol names annotated below

#table(
  columns: 3, align: horizon, stroke: none, gutter: 1em,

  // expval (mean/expectation value), eval (evaluation boundary)
  [$ expval(X) = eval(f(x)/g(x))^oo_1 $],
  // Dirac braket notations
  [$
    bra(u), braket(u), braket(u, v), \
    ket(u), ketbra(u), ketbra(u, v), \
    mel(phi, hat(p), psi) $],
  // Superscript show rules that need to be enabled explicitly.
  // If put in a content block, they only control that block's scope.
  [
    #show: super-T-as-transpose // "..^T" just like handwriting
    #show: super-plus-as-dagger // "..^+" just like handwriting
    $ op("conj")A^T =^"def" A^+ \
      e^scripts(T), e^scripts(+) $ ], // Override with scripts()
)
\end{verbatim}

\pandocbounded{\includesvg[keepaspectratio]{typst-img/8965812a9c349892988d61872ff06418581098d15169b775f67b30e3460dd854-1.svg}}

\paragraph{\texorpdfstring{\hyperref[matrices]{Matrices}}{Matrices}}\label{matrices}

In addition to Typst\textquotesingle s built-in
\texttt{\ }{\texttt{\ mat()\ }}\texttt{\ } to write a matrix, physica
provides a number of handy tools to make it even easier.

\begin{verbatim}
#import "@preview/physica:0.9.1": TT, mdet

$
// Matrix transpose with "TT", though it is recommended to
// use super-T-as-transpose so that "A^T" also works (more on that later).
A^TT,
// Determinant with "mdet(...)".
det mat(a, b; c, d) := mdet(a, b; c, d)
$
\end{verbatim}

\pandocbounded{\includesvg[keepaspectratio]{typst-img/7eccaa3a0cf838bca4daf9ebf573452506d3ea724086fcda0c9eb4264e66b5d9-1.svg}}

Diagonal matrix
\texttt{\ }{\texttt{\ dmat(\ }}\texttt{\ }{\texttt{\ ...\ }}\texttt{\ }{\texttt{\ )\ }}\texttt{\ }
, antidiagonal matrix
\texttt{\ }{\texttt{\ admat(\ }}\texttt{\ }{\texttt{\ ...\ }}\texttt{\ }{\texttt{\ )\ }}\texttt{\ }
, identity matrix \texttt{\ }{\texttt{\ imat(n)\ }}\texttt{\ } , and
zero matrix \texttt{\ }{\texttt{\ zmat(n)\ }}\texttt{\ } .

\begin{verbatim}
#import "@preview/physica:0.9.1": dmat, admat, imat, zmat

$ dmat(1, 2)  dmat(1, a_1, xi, fill:0)               quad
  admat(1, 2) admat(1, a_1, xi, fill:dot, delim:"[") quad
  imat(2)     imat(3, delim:"{",fill:*) quad
  zmat(2)     zmat(3, delim:"|") $
\end{verbatim}

\pandocbounded{\includesvg[keepaspectratio]{typst-img/66bbf5294be293cc58d98de6ca078eb17c58539169325e6b59a6aa78e7a49f62-1.svg}}

Jacobian matrix with
\texttt{\ }{\texttt{\ jmat(func;\ }}\texttt{\ }{\texttt{\ ...\ }}\texttt{\ }{\texttt{\ )\ }}\texttt{\ }
or the longer name \texttt{\ }{\texttt{\ jacobianmatrix\ }}\texttt{\ } ,
Hessian matrix with
\texttt{\ }{\texttt{\ hmat(func;\ }}\texttt{\ }{\texttt{\ ...\ }}\texttt{\ }{\texttt{\ )\ }}\texttt{\ }
or the longer name \texttt{\ }{\texttt{\ hessianmatrix\ }}\texttt{\ } ,
and finally \texttt{\ }{\texttt{\ xmat(row,\ col,\ func)\ }}\texttt{\ }
to build a matrix.

\begin{verbatim}
#import "@preview/physica:0.9.1": jmat, hmat, xmat

$
jmat(f_1,f_2; x,y) jmat(f,g; x,y,z; delim:"[") quad
hmat(f; x,y)       hmat(; x,y; big:#true)      quad

#let elem-ij = (i,j) => $g^(#(i - 1)#(j - 1)) = #calc.pow(i,j)$
xmat(2, 2, #elem-ij)
$
\end{verbatim}

\pandocbounded{\includesvg[keepaspectratio]{typst-img/7cd30cef52b187d17459d7806a94d5ae56118d0f969760bbbaabeb83007e6869-1.svg}}

\subsubsection{\texorpdfstring{\hyperref[mitex]{\texttt{\ }{\texttt{\ mitex\ }}\texttt{\ }}}{  mitex  }}\label{mitex}

\begin{quote}
MiTeX provides LaTeX support powered by WASM in Typst, including
real-time rendering of LaTeX math equations. You can also use LaTeX
syntax to write
\texttt{\ }{\texttt{\ \textbackslash{}r\ }}\texttt{\ }{\texttt{\ ef\ }}\texttt{\ }
and
\texttt{\ }{\texttt{\ \textbackslash{}l\ }}\texttt{\ }{\texttt{\ abel\ }}\texttt{\ }
.
\end{quote}

\begin{quote}
Please refer to the \href{https://github.com/mitex-rs/mitex}{manual} for
more details.
\end{quote}

\begin{verbatim}
#import "@preview/mitex:0.2.4": *

Write inline equations like #mi("x") or #mi[y].

Also block equations:

#mitex(`
  \newcommand{\f}[2]{#1f(#2)}
  \f\relax{x} = \int_{-\infty}^\infty
    \f\hat\xi\,e^{2 \pi i \xi x}
    \,d\xi
`)

Text mode:

#mitext(`
  \iftypst
    #set math.equation(numbering: "(1)", supplement: "equation")
  \fi

  An inline equation $x + y$ and block \eqref{eq:pythagoras}.

  \begin{equation}
    a^2 + b^2 = c^2 \label{eq:pythagoras}
  \end{equation}
`)
\end{verbatim}

\pandocbounded{\includesvg[keepaspectratio]{typst-img/a3ff500a39b6d93d85b223af0aa162a5bfbe93fad3436dba80ee022638ed727a-1.svg}}

\subsubsection{\texorpdfstring{\hyperref[i-figured]{\texttt{\ }{\texttt{\ i-figured\ }}\texttt{\ }}}{  i-figured  }}\label{i-figured}

Configurable equation numbering per section in Typst. There is also
figure numbering per section, see more examples in its
\href{https://github.com/RubixDev/typst-i-figured}{manual} .

\begin{verbatim}
#import "@preview/i-figured:0.2.3"

// make sure you have some heading numbering set
#set heading(numbering: "1.1")

// apply the show rules (these can be customized)
#show heading: i-figured.reset-counters
#show math.equation: i-figured.show-equation.with(
  level: 1,
  zero-fill: true,
  leading-zero: true,
  numbering: "(1.1)",
  prefix: "eqt:",
  only-labeled: false,  // numbering all block equations implicitly
  unnumbered-label: "-",
)


= Introduction

You can write inline equations such as $x + y$, and numbered block equations like:

$ phi.alt := (1 + sqrt(5)) / 2 $ <ratio>

To reference a math equation, please use the `eqt:` prefix. For example, with @eqt:ratio, we have:

$ F_n = floor(1 / sqrt(5) phi.alt^n) $


= Appdendix

Additionally, you can use the <-> tag to indicate that a block formula should not be numbered:

$ y = integral_1^2 x^2 dif x $ <->

Subsequent math equations will continue to be numbered as usual:

$ F_n = floor(1 / sqrt(5) phi.alt^n) $
\end{verbatim}

\pandocbounded{\includesvg[keepaspectratio]{typst-img/b338b679a09371841be9322ac7cee901b6a1415582c3495677833602e344cae0-1.svg}}

\subsection{\texorpdfstring{\hyperref[theorems]{Theorems}}{Theorems}}\label{theorems}

\subsubsection{\texorpdfstring{\hyperref[ctheorem]{\texttt{\ }{\texttt{\ ctheorem\ }}\texttt{\ }}}{  ctheorem  }}\label{ctheorem}

A numbered theorem environment in Typst. See more examples in its
\href{https://github.com/sahasatvik/typst-theorems/blob/main/manual.pdf}{manual}
.

\begin{verbatim}
#import "@preview/ctheorems:1.1.0": *
#show: thmrules

#set page(width: 16cm, height: auto, margin: 1.5cm)
#set heading(numbering: "1.1")

#let theorem = thmbox("theorem", "Theorem", fill: rgb("#eeffee"))
#let corollary = thmplain("corollary", "Corollary", base: "theorem", titlefmt: strong)
#let definition = thmbox("definition", "Definition", inset: (x: 1.2em, top: 1em))

#let example = thmplain("example", "Example").with(numbering: none)
#let proof = thmplain(
  "proof", "Proof", base: "theorem",
  bodyfmt: body => [#body #h(1fr) $square$]
).with(numbering: none)

= Prime Numbers
#lorem(7)
#definition[ A natural number is called a #highlight[_prime number_] if ... ]
#example[
  The numbers $2$, $3$, and $17$ are prime. See @cor_largest_prime shows that
  this list is not exhaustive!
]
#theorem("Euclid")[There are infinitely many primes.]
#proof[
  Suppose to the contrary that $p_1, p_2, dots, p_n$ is a finite enumeration
  of all primes. ... a contradiction.
]
#corollary[
  There is no largest prime number.
] <cor_largest_prime>
#corollary[There are infinitely many composite numbers.]
\end{verbatim}

\pandocbounded{\includesvg[keepaspectratio]{typst-img/54d7817ddc4a8481da09052aa51dc6e4dde19bd85f40173b92750c402b07ff73-1.svg}}

\subsubsection{\texorpdfstring{\hyperref[lemmify]{\texttt{\ }{\texttt{\ lemmify\ }}\texttt{\ }}}{  lemmify  }}\label{lemmify}

Lemmify is another theorem evironment generator with many selector and
numbering capabilities. See documentations in its
\href{https://github.com/Marmare314/lemmify}{readme} .

\begin{verbatim}
#import "@preview/lemmify:0.1.5": *

#let my-thm-style(
  thm-type, name, number, body
) = grid(
  columns: (1fr, 3fr),
  column-gutter: 1em,
  stack(spacing: .5em, [#strong(thm-type) #number], emph(name)),
  body
)
#let my-styling = ( thm-styling: my-thm-style )
#let (
  definition, theorem, proof, lemma, rules
) = default-theorems("thm-group", lang: "en", ..my-styling)
#show: rules
#show thm-selector("thm-group"): box.with(inset: 0.8em)
#show thm-selector("thm-group", subgroup: "theorem"): it => box(
  it, fill: rgb("#eeffee"))

#set heading(numbering: "1.1")

= Prime numbers
#lorem(7) @proof and @thm[theorem]
#definition[ A natural number is called a #highlight[_prime number_] if ... ]
#theorem(name: "Theorem name")[There are infinitely many primes.]<thm>
#proof[
  Suppose to the contrary that $p_1, p_2, dots, p_n$ is a finite enumeration
  of all primes. ... #highlight[_a contradiction_].]<proof>
#lemma[There are infinitely many composite numbers.]
\end{verbatim}

\pandocbounded{\includesvg[keepaspectratio]{typst-img/46b0a27243980ee99b20133dbba1f00d4d819adff6e645ca0749820f5caf3589-1.svg}}
